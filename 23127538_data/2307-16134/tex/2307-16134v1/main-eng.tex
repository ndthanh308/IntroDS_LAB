% paper on triangle tilings
% version of 10/12/2019
\documentclass[11pt]{article}


\usepackage{amsmath,amssymb,amsthm,amstext}

%\usepackage{amsmath,amssymb,amstext}(
\usepackage{graphicx,epsfig}
%\usepackage{showlabels}
\usepackage{fullpage}

\newcommand{\halpha}{\}\alpha}
\renewcommand{\le}{\leqslant}
\renewcommand{\ge}{\geqslant}

%\newtheorem{corollary}{Corollary}
%\newf{theorem{theorem}{Theorem}
%\newtheorem{corollary}[theorem]{Corollary}
%\newtheorem{lemma}{Lemma}
%\theoremstyle{remark}
%\newtheorem{definition}{Defintion}
%\newtheorem{remark}{Remark}

%\usepackage{amscd}
\newcommand{\Z}{\mathbb Z}
\newcommand{\R}{\mathbb R}
\newcommand{\st}{\text{S}}
\newcommand{\eps}{\varepsilon}
%\newtheorem{corollary}{Corollary}
\newtheorem{theorem}{Theorem}
\newtheorem{corollary}[theorem]{Corollary}
\newtheorem{lemma}{Lemma}
\theoremstyle{remark}
\newtheorem{definition}{Definition}
\newtheorem{remark}{Remark}
\renewcommand{\labelenumi}{(\theenumi)}
\date{}

\title{A new simple family of non-periodic tilings with square tiles}
\author{Nikolay Vereshchagin\\
Moscow State University, HSE University, Yandex%
\thanks{This paper was prepared within the framework of the HSE University Basic Research Program.}}
\begin{document}

\maketitle
\begin{abstract}
We define a new family of non-periodic tilings with square tiles that is mutually locally derivable with some family of tilings with isosceles right triangles. Both families are defined by simple local rules, and the proof of their non-periodicity is as simple as that of the non-periodicity of Robinson's tilings.
\end{abstract}  

\textbf{Keywords:}
non-periodic tilings, Domino problem, aperiodic tile sets, substitution tilings  
  
  
%  % Figure environment removed

\section{Introduction}

The study of non-periodic tilings of the plane by square tiles is motivated by the Domino problem: 
to construct an algorithm that, given local rules to attach  tiles, 
finds out whether it is possible to tile the entire plane according to these rules. 
The non-existence of such an algorithm was proved by Berger in~\cite{berger}. 
The main step in the proof is the construction of local rules such that any tiling according to these rules 
is non-periodic. The construction of such a family was subsequently 
simplified by Robinson~\cite{rob}. An even simpler construction using the 
same idea is given in~\cite{dls}.

In this paper, we propose a new such family. 
The local rules for these tilings and the proof of non-periodicity are approximately as easy as 
those in~\cite{rob} and~\cite{dls}. The novel thing is that our tilings are based  
on tilings by isosceles right triangles. 
In fact, we first construct a family of non-periodic tilings with 
triangles for which the local rules ensure that the triangles 
can be grouped  into square tiles. 
Then we rewrite the local rules for triangle tilings as local rules to attach resulting square tiles. 
Similarly, tilings by isosceles triangles with angles divisible by $36^\circ$ yield Penrose tilings P3 by rhombuses, which is called the Robinson decomposition of Penrose tilings. To use conventional terminology, 
our square tilings are mutually locally derivable to triangular tilings.

In the next section, we define our family of tilings with square tiles, and in 
Section~\ref{s3} we define tilings with triangles and provide proofs. 
The proof of the non-periodicity of tilings by triangles (Theorem~\ref{th3}), 
and hence of the non-periodicity of the family of tilings by square tiles (Theorem~\ref{th1}), 
uses a well-known technique based on self-similarity. 
Namely, we consider the following composition scheme: 
two triangles with a common leg are combined into one triangle. 
Applying this scheme to any tiling satisfying local rules we obtain a new tiling that again satisfies the same rules. 
It is well known that any family of tilings with such a property can only consist of non-periodic tilings. 
Moreover, we prove that all tilings in our family are substitution tilings (Theorem~\ref{th4}). 
This means that any  its finite fragment occurs in some supertile. 
The latter are defined as tilings that can be obtained from single tiles by several decompositions.

\section{Tilings with square tiles} \label{s2}

Our tiles are shown in Fig.~\ref{pic46}. 
% Figure environment removed
The sides drawn in black have one of two orientations and one of two colors, 
green or red. Thus, each of the four pictures depicts 
$4^4=256$ different tiles. In addition, these tiles can be rotated by angles that are multiples of $90^\circ$.%
\footnote{It may seem that the second tile is obtained from the first by $180^\circ$
rotation, and the fourth from the third. However, this is not the case --- 
with such a rotation, the slanted arrow coming from the upper right corner would change its color.} 
Tiles cannot be flipped. 
Thus, the total number of tiles is $4^6=4096$. A specific tile is shown in Fig.~\ref{pic45}.
% Figure environment removed

The local rules:
\begin{enumerate}
\item Tiles are attached only side-to-side, and the shared sides must have matching colors and orientations. 
\item Adjacent triangles 
% (from the pattern on the tiles) 
must have different colors. 
\item We will call  the \emph{crossing} centered at a given vertex the set of arrows entering 
or leaving this vertex.  The rule restricts possible crossings: only the crossings shown
in Fig.~\ref{pic44} are legal, as well as crossings obtained from them by rotations by angles that are multiples of $45^\circ$. 
% (both the arrows from the pattern on the tiles and the arrows on the sides of the tiles are taken into account). 
% Figure environment removed
%These are exactly the same figures that occur in the centers of the tiles, but 
%rotated by $45^\circ$. 
Legal crossings can be described as follows: 
a pair of arrows passes through the center of the crossing without changing color or direction. 
This pair of arrows is called the \emph{axis of the crossing}. 
There are also two arrows of different colors perpendicular to the axis, 
entering the center. In addition to these four arrows, there are four more, coming out of the center at an angle of 
$45^\circ$ to the axis. If you go to the center of the crossing perpendicular to the axis, 
then the green arrow goes to the right, and the red arrow to the left. 
\end{enumerate}
\begin{definition}
A \emph{tiling} is a set of tiles whose interiors are pair wise disjoint.
A  \emph{correct tiling by square tiles} is
a tiling of the whole plane with the above tiles that satisfies these rules. 
\end{definition}

\begin{theorem}\label{th1}
There is a correct tiling with square tiles, and any such tiling is non-periodic. 
\end{theorem}

The proof of the second part of this theorem uses the \emph{unique composition property}.
In any tiling of the plane, tiles can be grouped in a unique way into the so-called macro-tiles. 
Each macro-tile corresponds to some tile of the original set, 
and when macro-tiles are replaced by the corresponding tiles, the local rules are preserved.

This method will not be applied to the original tilings, 
but to tilings with isosceles right triangle obtained from the original tilings as follows. 
Let us cut each tile into 4 right-angled triangles along the diagonal arrows. 
We obtain a tiling with triangles. 
The original local rules become local rules for triangular tiles. 
To these rules, we add yet another rule that guarantees that triangles join into square tiles. 
In any tiling, according to the rules obtained, the tiles are divided into pairs 
of tiles with a common leg. Each such pair will correspond to one triangular tile, 
and when replacing it with the corresponding tile, the local rules will be preserved. 
This operation is called \emph{the composition}. Since the local rules are preserved, 
the triangles in the resulting tiling
should again join into square tiles. 
However, new square tiles can be composed of triangles from different square tiles in the original tiling. 
Therefore, composition can be defined only for tilings with triangles, but not for tilings with square tiles.

A similar technique is also used in the proof of the non-periodicity of Penrose tilings P3 by rhombuses of two shapes. 
The  rhombuses are also partitioned into triangles, which is called the Robinson decomposition.
Robinson rectangles can be grouped into larger rectangles,  which again  can be grouped into 
rhombuses.

So, we move on to tilings with triangular tiles.

\section{Tilings with triangles}\label{s3}

\subsection{Tiles and local rules}
The tiles are isosceles right triangles colored red or green. 
Each side of a tile has an orientation and a color, 
either red or green. Thus each tile is specified by seven bits. 
In addition, tiles can be rotated by 90-degree angles, 
so the number of tiles is $2^9=512$ up to shift. 
Two examples of tiles are shown in Fig.~\ref{pic17}. 
% Figure environment removed
%\begin{quote}
Crossings of the form $C_8$ are exactly legal crossings for tilings with square tiles. 
Crossings of the form $C_4$ appear at the centers of square tiles when they are cut into triangles.
\item
If we go to the center of the $C_4$ crossing perpendicular to the axis, then there should be a red triangle on the left, 
and a green triangle on the right. The axis of the $C_4$ crossing is defined similarly to the axis of $C_8$, 
as the pair of uni-color unidirectional arrows passing through the center.
\end{enumerate}

We will call tilings that satisfy these local rules \emph{correct triangle tilings}. 
For tilings of \emph{parts} of the plane, we require that the third and fourth conditions 
hold only for internal vertices.

\begin{lemma}
(a) If a correct tiling $T$ with square tiles is cut into triangles, 
then a correct tiling $T'$ with triangles is obtained. 
(b) Conversely, any correct triangular tiling $T'$of the plane can be obtained from some correct tiling $T$ with square 
tiles by cutting it into triangles. This tiling $T$ is unique.
\end{lemma}
\begin{proof}
(a) Conditions (1) and (2) for $T'$ follow from conditions 1 and 2 for $T$. 

Condition (3): every crossing in $T'$ either coincides with the corresponding crossing in $T$ and hence is of the form $C_8$,
or is the crossing in the center of a tile   in $T'$ and hence is of the form $C_4$.

Condition (4) is verified by hand.

(b) Since in a correct tiling of the plane by triangles there can only be a crossing $C_4$ at the vertex of any right angle, 
its tiles can be grouped into quadruples, each of which  forms a square tile from our set. Obviously, this partition is unique.

Since the local rules for tilings by squares essentially coincide with the local rules for tilings by triangles, the tiling $T$ is correct.
\end{proof}

 Thus, to prove Theorem 1 we have to prove the existence and non-periodicity of correct tilings by triangles.

\subsection{Substitution}

To prove existence and non-periodicity of correct tilings by triangles, 
consider the following substitution: each triangular tile is cut by its height into two tiles similar 
to the original one with the ratio $\sqrt2$. 
The height is oriented from the vertex of the right angle to the hypotenuse and has the same color as the original tile. 
The triangle left from the height, when viewed in the direction of the arrow, is  called  \emph{left}, it is colored red, 
and the right triangle is called \emph{right}, it is colored green (Fig.~\ref{pic5}). 
All sides (including the two halves of the hypotenuse) of the original triangle retain their color and orientation.
% Figure environment removed
For example, applying the substitution to the left tile in Fig.~\ref{pic17}, we will get a tiling in Fig.~\ref{pic18}.
% Figure environment removed

Using this substitution, we define the operations of decomposition and composition of the given tiling.
To decompose a tiling, we apply the substitution to all its tiles.
The resulting tiling tiles the same part of the plane with smaller tiles. 
This tiling is then stretched by a factor of $\sqrt 2$,
and the resulting tiling is called the \emph{decomposition} the original tiling. 
It is easy to see that the decomposition operation is injective.
Indeed, we call two tiles \emph{siblings} if they are obtained by substitution from the same tile.
Then for each tile $F$ there exists a unique sibling $G$, provided we ignore 
the color and orientation of edges. Indeed, if $F$ is a red tile, then the only its sibling is the green tile
obtained from $F$ by $90^\circ$ rotation \emph{counterclockwise}, with respect to the vertex of the right angle of $F$.
If $F$ is a green tile, then the rotation is \emph{clockwise}.

Thus, for each tiling, there is at most one tiling of which it is the decomposition. If such a tiling exists, then it is called \emph{the composition}
of the original tiling. The decomposition of the tiling $T$ will be denoted by $\sigma T$, and the composition by $\sigma^{-1} T$.

\subsection{Supertiles}
\emph{A level $n$} supertile is the $n$-fold decomposition of a single tile. 
Fig.~\ref{pic4} shows a level 6 supertile obtained by $6$-fold decomposition from the left tile in Fig.~\ref{pic18}.
% Figure environment removed
Supertiles of level $n$ will be denoted by $S_n$.

\begin{remark}
It can be proven by induction that only three (not 8) different side orientations of inner red tiles and
three different side orientations of inner green tiles occur in supertiles (Fig.~\ref{pic6}). 
Therefore, we can assume that the total number of triangular tiles is $(3+3)\cdot 8\cdot 4=192$ (and not 512).
% Figure environment removed
\end{remark}

\subsection{Existence of tilings satisfying local rules}

First we prove the existence of correct tilings by triangles of the entire plane. 
To do this, it suffices to prove that all supertiles satisfy the local rules. 
Why is this enough? For example, we can argue as follows: we can verify that the $S_6$ supertile in 
Fig.~\ref{pic4} contains strictly inside the tile in Fig.~\ref{pic18} on the left, 
from which this supertile was obtained by sixfold decomposition. 
Therefore, if we apply the sixfold decomposition to the supertile in Fig.~\ref{pic4},
we will get a supertile $S_{12}$ including the tiling $S_6$.
Iterating this operation, we get a tower of supertiles 
$$
S_0\subset S_6\subset S_{12}\subset S_{18}\subset\dots, 
$$
whose union is   a correct tiling of the entire plane. 
Thus, it suffices to prove the following
\begin{lemma}
Any supertile is a correct tiling.
\end{lemma}
\begin{proof}
This is proved by induction. The induction base is trivial, since any tile constitutes a correct tiling.

For the inductive step, we have to show that the decomposition of a correct tiling $T$ is again correct. 
But there is a problem here. This is true for tilings of the plane, but not for tilings of its parts. 
For example, a tiling consisting of two triangles with a common leg in Fig.~\ref{pic27},% Figure environment removed
is correct (the third condition is satisfied because there are no internal vertices), but its decomposition is not. 
To make the inductive step, it is necessary to impose a restriction on the crossings at the boundary points. 
To prove the lemma, it will suffice to require that the crossings with the center at any vertex of the boundary, 
except for the extreme ones, are halves of the legal crossings, that is, they have one of the two forms in Fig.~\ref{pic26}.
% Figure environment removed

Now we can make the inductive step. 
Assume that $T$ is a supertile and is correct. We have to show that so is its decomposition $\sigma T$.
It is obvious that the first condition 
(tiles border side by side and colors and orientations of shared sides match) %for  $\sigma T$
is inherited during decomposition.

Let us verify the second condition (adjacent triangles have different colors) for  $\sigma T$. 
Let $F$ be a tile in $\sigma T$ 
and let $G$ be its sibling. W.l.o.g. assume that  $F$ is a right, and hence green, tile (Fig.~\ref{pic28}).
% Figure environment removed
We need to prove that all neighbors of $F$ are red.

The tile $G$ is red by definition of the substitution.
Besides $G$, the tile $F$ can have other neighbors.
Namely, if the hypotenuse of $I$ does not lie on the edge of $T$, then the tiling $T$ contains the tile $E$ bordering $I$ along the hypotenuse.
As a result of cutting $E$, two tiles will appear, of which it is the left, and therefore red, tile that borders $F$ along the leg.

Moreover, if the leg of the tile $I$ does not lie on the boundary of the supertile, then the tiling $T$ contains the tile $H$,
which is the reflection of the tile $I$ with respect to this leg.
%The tile $H$ is oriented exactly as it is drawn,
%that is, its right angle adjoins the vertex $A$.
Indeed, otherwise the crossing centered at the vertex $A$ would be illegal
(right and acute angles cannot converge at a legal crossing).
When the tile $H$ is decomposed, it is the left, and therefore red, tile that is adjacent to $F$.

Let us check the third condition for the tiling $\sigma T$ (that all crossings are legal).
To do this, we need to check that during decomposition, a legal crossing is converted into a legal one,
and the same holds for half crossings in Fig.~\ref{pic26}.
This is done by a direct verification.
In addition, we need to check that all the crossings at the new vertices are legal as well.

New vertices are the centers of  hypotenuses of tiles from $T$.
The hypotenuse itself becomes the axis of the new crossing, and the new crossing is $C_4$ by the definition of the substitution.

 
Finally, we check the fourth condition. The crossings of the form $C_4$ are only the crossings at the centers of the hypotenuses of the tiles in $T$,
and by the definition of substitution, all the left triangles in them are red and the right ones are green.

The third and fourth conditions for crossings centered on the boundary are verified in a similar way.
\end{proof}

\subsection{Non-periodicity of tilings satisfying local rules}

The non-periodicity of correct tilings follows from the fact that any
tiling of the plane has a structure in the following sense:
for any $n$ it can be partitioned uniquely into level $n$ supertiles.
This is a simple consequence of the following 

\begin{theorem}\label{th2}
  Any correct tiling is composable and its composition is correct.
\end{theorem}
\begin{proof}
  Consider any tile of a given correct tiling  $T$ and the vertex of its right angle.
  The crossing at this vertex is $C_4$.
  At this crossing, each tile has a sibling.
  Let us combine them into one, erasing the common leg, and leave all the colors and orientations as they were.
  The hypotenuse of the resulting triangle is formed by connecting two arrows on the axis of the crossing.
  Since they are equally colored and oriented, the hypotenuse will be correctly oriented and correctly colored.
  The resulting tiling is the composition of the original one.
  Let us prove that it is correct.
  
  We are given that $T$ is correct and need to prove that $\sigma^{-1} T$ is correct.
  First we verify condition (1), that is, prove that $\sigma^{-1} T$ is side-to-side and shared sides have matching colors and orientations.
  This is obvious for the legs of the tiles from $\sigma^{-1} T$, since they are the hypotenuses of the tiles from $T$,
  and by the condition (1) for $T$ the hypotenuses are adjacent to the hypotenuses of the same color and orientation.


  Now let us prove the same for the hypotenuses of tiles from $\sigma^{-1} T$.
  Consider a tile $F$ from $\sigma^{-1} T$ (Fig.~\ref{pic13}(a)). W.l.o.g. assume that it is red.
  We need to show that the tile $G$ shown in Fig.~\ref{pic13}(a) belongs to $\sigma^{-1} T$. To this end,
  consider the midpoint of the hypotenuse, denoted by the letter $A$. The crossing centered on $A$ in the tiling $T$ can only be $C_4$,
  so there are two tiles in $T$ under the hypotenuse $A$,
  as shown in Fig.~\ref{pic13}(b). These two tiles, when composed, give the sought  tile $G$.
% Figure environment removed 

As a bi-product of this argument, we see that the colors of the tiles $F$ and $G$ are different because the colors of the vertical arrows at the crossing $C_4$ are different.
Therefore, condition (2) is satisfied for triangles $\sigma^{-1} T$ with a common hypotenuse. Let us verify condition (2) for tiles
that share a leg. Let the tiles $F$ and $G$ have a common leg (Fig.~\ref{pic13}(c)).
Then their colors are equal to the colors of two arrows in the tiling $T$ (Fig.~\ref{pic13}(d)),
connecting the point $A$, the common vertex of right angles of $F$ and $G$, with the centers of their hypotenuses.
These two arrows form a right angle and at their common origin (point $A$) there can only be the crossing $C_8$ in the tiling $T$.
By a routine check, we can see that any two orthogonal arrows with a common origin at the center of the crossing $C_8$ have  different colors.

Let us verify condition (3). In the course of composition, the crossings $C_4$ disappear.
We need to show that crossings of the form $C_8$ are transformed into themselves or into $C_4$.
In all crossings $C_8$ the sides of the tiles adjacent to the center of the crossing alternate --- hypotenuse, leg, hypotenuse, and so on.
And the colors of the tiles also alternate. Therefore, 4 options arise, depending on whether the alternation of colors begins with red or green,
and the alternation of sides --- from the leg or hypotenuse.
These four options are shown in Fig.~\ref{pic16}.
 % Figure environment removed

 Let us first prove that the first option is impossible. Indeed, the crossings at the lower and upper blue points cannot be legal.
 Indeed, the axes of these crossings must be directed vertically, since otherwise an arrow would go from 
 its center in the direction orthogonal to the axis.
 For the upper crossing, the axis should be directed downwards, and for the lower one, upwards.
 But then the colors of both triangles with a vertex at the blue points should be opposite to the existing colors.

 In the next two crossings, the siblings to all tiles are not adjacent to the crossing. For this reason, when composed, the crossings do not change.
 Finally, in the fourth case, the sibling for each tile is also adjacent to the center. When decomposed, the eight tiles produce four tiles,
 and we get the crossing $C_4$. In this case,
 condition (4) is satisfied for the resulting crossing $C_4$. Indeed, in any $C_8$ crossing,
the red arrow goes to the left of the arrow entering its center, and the green arrow goes to the right.
Hence a red triangle is to the left of the arrow entering the center of the obtained $C_4$ crossing,
and a green one to the right.
\end{proof}

\begin{theorem}\label{th3}
Any correct tiling of the plane by triangles is non-periodic. Consequently, any correct tiling by square tiles is also non-periodic.
\end{theorem}
\begin{proof}
  Let $T$ be a correct tiling of the plane by triangles and $\sigma^{-1}T$ be its composition.
Assume that $a\ne 0$ is its period, that is $T=T+a$.
Then the vector $a/\sqrt2$ is the period of $\sigma^{-1}T$. Indeed,
$$
\sigma^{-1}T=
\sigma^{-1}(T+a)= (\sigma^{-1}T)+a/\sqrt2.
$$
It is proved similarly that $a/2$ is the period of the double composition of $T$.
Repeating this argument many times, we get a tiling whose period is much less than the size of the tiles, which is impossible.
\end{proof}

\subsection{Substitution tilings}

\begin{definition}
  A tiling is called a \emph{substitution tiling} (for the given substitution) if every its finite fragment
  occurs in a supertile.
\end{definition}

Since supertiles are correct tilings, any substitution tiling is correct. It turns out that the opposite is also true.

\begin{theorem}\label{th4}
Any correct tiling of the plane by triangles is a substitution tiling.
\end{theorem}
\begin{proof}
Consider any fragment  $P$
of a correct tiling $T$ of the  plane. We have to show that $P$ occurs in a supertile. To this end add in $P$ a finite number of tiles 
from  $T$ so that $P$ becomes an inner part of the resulting fragment $Q$ of $T$.

By Theorem~\ref{th2} for any $k$ we can compose the given tiling  $k$ times
and the resulting tiling $\sigma^{-k}T$ is correct.  
Call a \emph{crown in $\sigma^{-k}T$ centered at a vertex $A$} of a tile in $\sigma^{-k}T$
the set of all tiles from $\sigma^{-k}T$ that include $A$.  Since  $\sigma^{-k}T$
is a correct tiling, all its crowns have a form shown on Fig.~\ref{pic31} (see the proof of Th.~\ref{th2}).
 % Figure environment removed

Consider the sets of the form $\sigma^k C$ where $C$ is a crown within the tiling
$\sigma^{-k}T$. 
As $k$ increases, these sets increase as well. If $k$ is large enough, then the set $Q$ 
is covered by a  single such set, say by $\sigma^k C$,
that is,  $Q\subset \sigma^k C$. As $\sigma^{-k}T$  
is a correct tiling, it can have only crowns shown on   Fig.~\ref{pic31}.

Let us show that all crowns from 
 Fig.~\ref{pic31} appear in  supertiles provided we ignore colors and orientations of its outer 
 sides (shown in black color on Fig.~\ref{pic31}).  For crowns of the form (a) 
it can be verified  by hand: all the four such crowns occur within  supertiles $S_5$ shown on Fig.~\ref{pic34}.
 % Figure environment removed
Again it is easy to verify  by hand that the substitution transforms the crowns as follows:
(a)$\to$(b)$\to$(c)$\to$(d)$\to$(c). Hence all  the four crowns (b) occur within supertiles $S_6$,
the crowns (c)  within  supertiles $S_7$,
and the crowns (d)  within supertile $S_8$.

It follows that the crown $C$, whose $k$-fold decomposition covers $Q$, 
appears in a supertile, say in $S_n$. Therefore
the tiling $\sigma^{k}C$ appears in the supertile $S_{n+k}$. Hence the patch $Q$  
appears in that supertile provided we ignore colors and orientations of its
outer sides. Since $P\subset Q$ and no side of the  patch $P$ is an outer side of $Q$,
we are done. 
\end{proof}

\section{Acknowledgments. }
The author is sincerely grateful to the participants of the Kolmogorov seminar and of the 
International academic conference “Graphs, Games and Models'' (October 12-15, 2022, Maikop, Adygea)
for helpful discussions.

\begin{thebibliography}9

\bibitem{berger}
Robert Berger, The undecidability of the domino problem, Memoirs of the American Mathematical Society, 66 (1966) 1--2.

\bibitem{dls}
 Bruno Durand, Leonid Levin, Alexander Shen.
  Local rules and global order, or aperiodic tilings.
The Mathematical Intelligencer, 27:1 (2005) 64--68


%\bibitem{dsv}
%Durand B., Shen A., Vereshchagin N. On the Structure of Ammann A2 Tilings
%Discrete and Computational Geometry. 2019 (см также https://arxiv.org/abs/1112.2896).

%\bibitem{G}
 % Chaim Goodman-Strauss, Matching Rules and Substitution Tilings,
 % Annals of Mathematics 147 (1998) 181-223.

%\bibitem{GS}
%Branko Gr\"unbaum, Geoffrey C. Shephard, Tilings and Patterns. Freeman,
%New York 1986

%\bibitem{ver}
%Nikolay Vereshchagin. Aperiodic Tilings by Right Triangles. In: Descriptional Complexity of Formal Systems - 16th International Workshop, DCFS 2014, Turku, Finland, August 5-8, 2014. Proceedings. Lecture notes in computer sciences, Vol. 8614. Berlin : Springer Verlag, 2014. P. 29-41.

%\bibitem{konig}
 % D. K\"onig. Theorie der Endlichen und Unendlichen Graphen: Kombinatorische Topologie der %Streckenkomplexe (in German), Leipzig: Akad. Verlag, 1936.

\bibitem{penrose}
 Roger  Penrose, Pentaplexity, Eureka, 39 (1978) 16--22. 
  
%  \bibitem{te}
 % D. Frettlöh, E. Harriss, F. Gähler, Tilings encyclopedia, https://tilings.math.uni-bielefeld.de/


\bibitem{rob} Raphael
 Robinson, Undecidability and nonperiodicity of tilings in the plane, Invent. Math., 12 (1971) 177-209


%\bibitem{solomyak}  
%  B. Solomyak, Nonperiodicity implies unique composition for self-similar translationally finite tilings, Discrete and Computational Geometry 20 (1998) 265-279.

%\bibitem{lo}  
%Danzer, Ludwig and van Ophuysen, Gerrit, A species of planar triangular tilings with inflation factor $\sqrt{-\tau}$ Res. Bull. Panjab Univ. Sci. 2000, 50, 1-4, pp. 137--175 (2001)

\end{thebibliography}
\end{document}
\appendix
\section{Доказательства}

Будем называть конечное замощение легальным, если оно является фрагментом некоторой суперплитки.
Пусть дано подстановочное замощение $T$. Сначала докажем, что для каждого фрагмента $P$ в $T$
существует расширяющий его фрагмент $Q\supset P$, имеющий легальное укрупнение.
Для этого добавим в $P$ столько плиток из $T$, чтобы расстояние между любой плиткой из $P$ 
и границей полученного замощения было больше диаметра всех правых частей подстановки.
Полученный фрагмент обозначим через $P'$. Поскольку $T$ подстановочно, $P'$
легальный фрагмент, $P'\subset S_n$. Суперплитка  $S_n$ является измельчением некоторой суперплитки $S_{n-1}$.
Рассмотрим фрагмент $Q\subset S_{n-1}$, состоящий из тех плиток, измельчение которых целиком
лежит в $P$. Поскольку все плитки из $P$ далеко лежат от границы $P'$, все они принадлежат
измельчению $Q$.

Пользуясь доказанным утверждением, мы можем построить цепь
укрупняемых фрагментов $P_1\subset P_2\subset \dots$ замощения $T$ таких,
что $P_i$ имеет легальное укрупнение $Q_i$ и продолжает $P_{i-1}$  во все стороны. Если бы
образцы $Q_i$ также образовывали цепь, то объединение этой цепи было бы искомым укрупнением $T$.
Но это вовсе не обязательно так, поскольку укрупнение в $P_i$ и $P_{i+1}$ может происходить
по-разному. 
%Очевидно, что нам достаточно выделить
%подпоследовательность в последовательности $Q_1,Q_2,\dots$,
%образующую цепь по включению

Справиться с этой проблемой можно с помощью леммы Кёнига.
Выделим в $T$ некоторую плитку $A$ и рассмотрим орграф, вершинами которого являются 
легальные образцы $V$, измельчение которых, во-первых, содержит плитку $A$,
а во-вторых включено в $T$.  

Из $V$ в $W$ проведем дугу,
если $V$ --- подмножество  $W$, причем
границы областей, замощаемых $V$ и $W$ не пресекаются (то есть, $W$ расширяет $V$ во все стороны)
и каждая плитка из  $W\setminus V$ касается границы области, замощаемой $V$ (то есть, никакую плитку из $W\setminus V$ нельзя выкинуть,
не разрушив предыдущее свойство). 
В этом графе  каждая вершина имеет лишь конечное число детей.
Еще добавим в этот граф дополнительную вершину (корень), из которой идут дуги во все вершины степени 0 (таковыми
являются замощения, состоящие из единственной плитки, измельчение которой содержат плитку $A$).

По построению, каждая вершина этого графа имеет лишь конечное число соседей (вершин, в которые из нее ведет дуга).
В самом у корня лишь конечное число соседей, а именно столько, сколько раз плитка $A$ входит
в правые части подстановки.
У любой другой вершины тоже конечное число соседей: 
для каждой вершины на границе  области, замощаемой $V$ нужно выбрать
хотя бы одну добавляемую плитку из $T\setminus\sigma V$. Это можно сделать конечным числом 
способов. Затем для каждой из этих добавленных  плиток нужно
выбрать как именно она будет укрупняться (с какими плитками из $T$). Это тоже можно сделать конечным 
числом способов.  

Наконец, докажем, что в этом графе есть сколь угодно длинные пути из корня. 
По доказанному ранее, можно найти сколь угодно большой фрагмент $P\subset T$, имеющий легальное укрупнение $V$,
содержащий $A$  причем так, что $A$ расположено достаточно далеко от границы области, замащиваемой $V$.
Сгруппируем плитка фрагмента $P$ так, как это сделано для получения $V$.
Затем построим путь в графе $V_0,V_1,V_2,\dots$ из укрупнений растущих фрагментов $T$, получаемых указанной группировкой
пока получившийся фрагмент все еще умещается внутри $V$.
Чем больше $P$, тем более длинный путь мы получим.

 




\end{document}
