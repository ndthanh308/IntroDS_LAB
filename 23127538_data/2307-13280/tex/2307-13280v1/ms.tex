\documentclass[nofootinbib,reprint,amsmath,amssymb,showkeys, showpacs,aps]{revtex4-1}
\usepackage{graphicx}
\usepackage[T1]{fontenc}
\usepackage{dcolumn}
\usepackage{xcolor,colortbl}
\usepackage{multirow}
\usepackage{bm}
\newcommand{\qm}[1]{``#1''}
\usepackage{hyperref}
\hypersetup{colorlinks, linkcolor={red},citecolor={blue},urlcolor={blue}}  
\newcommand\ChangeRT[1]{\noalign{\hrule height #1}}
\newcommand{\vdf}[1]{\textcolor{red}{#1}}
\usepackage{accents}
\usepackage{scalerel}
\usepackage{tabularx,siunitx}
\usepackage{wasysym}
\newcommand{\udt}[3]{#1^{#2}_{\phantom{#2}#3}}
\newcommand{\ududt}[5]{#1^{#2\phantom{#3}#4}_{\phantom{#2}#3\phantom{#4}#5}}
\newcommand{\dut}[3]{#1_{#2}^{\phantom{#2}#3}}
\newcommand{\dudt}[4]{#1_{#2\phantom{#3}#4}^{\phantom{#2}#3}}
\newcommand{\udut}[4]{#1^{#2\phantom{#3}#4}_{\phantom{#2}#3}}
\newcommand{\lc}[1]{\accentset{\circ}{#1}}%Levi-Civita connection
\newcommand{\tg}[1]{\accentset{\land}{#1}}%Teleparallel connection
\newcommand{\stg}[1]{\accentset{\scaleto{\diamond}{3.5pt}}{#1}}%Symmetric Teleparallel connection
\newcommand{\dd}{{\rm d}}
\newcommand{\pa}{\partial}
\newcommand{\del}[1]{\delta_g #1}
\newcommand{\pg}[1]{\frac{\pa #1}{\pa g^{\mu\nu}}}

\begin{document}

\title[ The role of the boundary term in  $f(Q,B)$ symmetric teleparallel gravity]{The role of the boundary term in  $f(Q,B)$ symmetric teleparallel gravity}


\author{Salvatore Capozziello$^{1,2,3}$}\email{capozziello@unina.it}
\author{Vittorio De Falco$^{2,3}$}\email{v.defalco@ssmeridionale.it}
\author{Carmen Ferrara$^{2,3}$} \email{carmen.ferrara-ssm@unina.it}


\affiliation{$^1$ Universit\`{a} degli studi di Napoli \qm{Federico II}, Dipartimento di Fisica \qm{Ettore Pancini}, Complesso Universitario di Monte S. Angelo, Via Cintia Edificio 6, 80126 Napoli, Italy\\
$^2$Scuola Superiore Meridionale,  Largo San Marcellino 10, 80138 Napoli, Italy,\\
$^3$ Istituto Nazionale di Fisica Nucleare, Sezione di Napoli, Complesso Universitario di Monte S. Angelo, Via Cintia Edificio 6, 80126 Napoli, Italy}

\date{\today}

\begin{abstract}
In the framework of metric-affine gravity, we consider  the role of the boundary term in   Symmetric Teleparallel Gravity assuming $f(Q,B)$ models  where $f$ is a smooth function of the non-metricity scalar $Q$ and the  related boundary term $B$. Starting from  a variational approach, we derive the  field equations and compare them with respect to those of  $f(Q)$ gravity in the limit of $B\to0$. It is possible to show that    $f(Q,B)=f(Q-B)$ models  are dynamically equivalent to $f(R)$ gravity  as in the case of   teleparallel  $f(\tilde{B}-T)$ gravity  (where $B\neq \tilde{B}$).  Furtherrmore, conservation laws are derived.  In this perspective, considering boundary terms in $ f(Q)$ gravity represents the last ingredient towards  the Extended Geometric Trinity of Gravity where $f(R)$, $f(T,\tilde{B})$ and $f(Q,B)$ can be dealt under the same standard.
\end{abstract}

\maketitle
\section{Introduction}
\label{sec:intro}
General Relativity (GR) is the best and well-tested theory of gravity so far available. However, it cannot be the final theory of gravitational interaction because of some  fundamental reasons related to its IR and UV behaviors. At astrophysical and  cosmological scales,  we need  mechanisms capable of   explaining the clustering of structures and the accelerated expansion\footnote{Difficulties in detecting new fundamental particles  to explain the dark sector point out  that alternative approaches have to be pursued in order to match phenomenology.} \cite{Carroll2001,Rocco2019,Planckcollaboration2020}. At microscopic scales, GR is not  renormalizable.  The fact that  a coherent quantum gravity theory  does not yet exist \cite{Hooft1974,Gleiser2005} means that some extensions or modifications of GR are needed. To  address the aforementioned issues, the current most-followed strategy consists in formulating  gravitational theories admitting  GR in some limits and including, in general,  further degrees of freedom  \cite{Clifton2006, Faraoni2010,Odintsov2011,Capozziello2011R,Oikonomou,Celia, Celia2,Lavinia}.  All these theories come from  motivations related to the fundamental structure and principles of gravitational field (see Ref. \cite{CANTATA} for a comprehensive review).

In this  debate,  the validity of Equivalence Principle at quantum scales  as well as   related features like  causal and geodesic structure play a  main role. In particular, in view of dealing with gravity as a gauge theory, metricity is of central importance. Relaxing the hypothesis that metricity principle holds in any case opens the way to the possibility to achieve a fundamental theory overcoming the shortcomings of GR. Among metric-affine theories, Symmetric Teleparallel Gravity and its extensions are assuming  a prominent role in the discussion to achieve a final theory of gravity. In this approach,  dynamics is described by the non-metricity scalar $Q$ derived from the non-metricity tensor 
$Q_{\alpha\mu\nu}=\nabla_\alpha g_{\mu\nu}$ (see e.g.  \cite{Beltran2018,DAmbrosio2022}). Recently, the  extension  $f(Q)$, where $f$ is a smooth function of $Q$,   has been  exploited to describe  bouncing cosmology \cite{Bajardi2020}, to describe late time accelerated expansion  \cite{Rocco2022,Solanki2022}, and early time inflationary behavior \cite{Shokri}, to  constrain gravitational wave observations \cite{Soudi2019}, as well as in high-energy astrophysics for black hole  \cite{DAmbrosio2022, Mustafa} and wormhole  \cite{Parsaei2022, Sharma} solutions.
Noether symmetries of $f(Q)$ gravity are discussed in \cite{Kostas} while minisuperspace quantum cosmology is developed in \cite{Quantum}.

In particular, for $f(Q)=a\ Q+b$, with $a,b$ real constants, it is possible to  retrieve the Symmetric Teleparallel Equivalent of GR (STEGR), where the Lagrangian of GR and STEGR differ only for a boundary term $B$. This implies that the two mathematical frameworks produce exactly the same field equations, albeit they seem to be \emph{a-priori}  completely unrelated theories. Beside them, there is also the Teleparallel Equivalent of GR (TEGR), based on the torsion scalar $T$ and the related boundary term $\tilde{B}$ (with $B\neq\tilde{B}$), which is another equivalent formulation of GR \cite{Pereira,Cai}. 

The three theories  whose dynamics is embedded in $R$, the Ricci curvature scalar. $T$, the torsion scalar, and $Q$, the non-metricity scalar,  constitute  the so-called \emph{Geometric Trinity of Gravity}  \cite{Tomi}. See also \cite{ Capozziello2022} for a discussion.

The goal of this article aims at presenting the  $f(Q,B)$ gravity, configuring as an extension of $f(Q)$ gravity and, more in general, of STEGR. We analyse some features of this model and we discover that $f(Q,B)=f(Q-B)$ represents a dynamically equivalent formulation of $f(R)$ gravity (being an extension of GR), where $f$ is a smooth function of the Ricci scalar curvature $R$. The paper is organized as follows: in Sec. \ref{sec:f(Q)_gravity} we briefly recall the $f(Q)$ gravity and set out our notations.  Sec. \ref{sec:f(Q,B)_gravity} is devoted to  modified $f(Q,B)$ gravity model verifying the consistency of the obtained equations and introducing the concept of Extended Geometric Trinity of Gravity. In Sec. \ref{sec:end} we  draw the conclusions.

\emph{Notations.}   The spacetime metric is  $g_{\mu\nu}=\eta_{AB} e^A_{\mu}e^B_{\nu}$ where $e^A_{\mu}$ are the tetrad fields on the tangent space with Minkowskian metric $\eta_{AB}$.  The determinant of the metric $g_{\mu \nu}$ is denoted by $g$ and $e=\sqrt{-g}$. Round (square) brackets around a pair of indices stands for the  symmetrization (antisymmetrization) procedure, i.e., $A_{(ij)}=A_{ij}+A_{ji}$ (respectively, $A_{[ij]}=A_{ij}-A_{ji}$).  All quantities with an over-circle denote objects framed in GR, like $\lc{\Gamma}^\lambda{}_{\mu\nu},\lc{\nabla}_\alpha$; whereas quantities without any marked symbols are framed in the Symmetric Teleparallel Gravity, like $\Gamma^\lambda{}_{\mu\nu},\nabla_\alpha$. The coupling constant in the metric field equations is $\chi=\frac{8\pi G}{c^4}$. We indicate the partial derivatives of $f$ with $f_X(X,Y)=\frac{\pa f}{\pa X}$, and $f_{XX}(X,Y)=\frac{\pa{}^2f}{\pa X^2}$. The same   with respect to $Y$ or mixed derivatives with respect to $X$ and $Y$.

 \section{$f(Q)$ symmetric teleparallel gravity}
\label{sec:f(Q)_gravity}
A subclass of metric-affine geometries is represented by the Symmetric Teleparallel Gravity theories, characterized by vanishing curvature and torsion \cite{Capozziello2022}. The only surviving quantity is the non-metricity tensor
\begin{subequations}
\begin{align}
Q_{\alpha\mu\nu}&=\nabla_\alpha g_{\mu\nu}=\partial_\alpha g_{\mu\nu}-\Gamma^\lambda_{\alpha\mu}g_{\nu\lambda}-\Gamma^\lambda_{\alpha\nu}g_{\mu\lambda},\label{eq:nonm_ten}
\end{align}
\end{subequations}
expressing the failure of the metric compatibility when different from zero. In this framework, metric and affine connection are two independent geometrical objects. The former is deputed to define the \emph{casual structure}, whereas the latter  describes the \emph{geodesic structure}. 

Specifically, it is possible to take into account   the $f(Q)$ gravity, expressed in terms of the following action
\begin{equation} \label{eq:f(Q)_action}
S_Q=\int \dd^4x\ e \biggr{[}\frac{1}{2\chi}f(Q)+\mathcal{L}_m\biggr{]},
\end{equation}
where $Q$ is the \emph{non-metricity scalar} defined as 
\begin{equation}
Q=-\frac{Q_{\alpha\mu\nu}Q^{\alpha\mu\nu}}{4}+\frac{Q_{\alpha\mu\nu}Q^{\alpha\nu\mu}}{2}+\frac{Q_\alpha Q^\alpha}{4} -\frac{Q_\alpha \bar{Q}^\alpha}{2},
\end{equation}
where $Q_\alpha=Q_{\alpha\mu}{}^\mu$ and $\bar{Q}_\alpha=Q^\mu_{\ \mu\alpha}$. The non-metricity scalar can be also written as follows
\begin{equation}
\lc{R}=Q-B,\qquad B=\lc{\nabla}_\lambda\tilde{Q}^\lambda=\frac{1}{e}\partial_\lambda(e \tilde{Q}^\lambda),
\end{equation}
where $\tilde{Q}^\lambda=Q^\lambda-\bar{Q}^\lambda$, $\lc{R}$ is  the Ricci scalar curvature of GR, $\lc{\nabla}$ is  the GR covariant derivative, and $B$  the boundary term. 

If the action $S_Q$ is derived with respect to the metric (i.e., $\delta_g S_Q=0$), we obtain the \emph{metric field equations of second-order in $g_{\mu\nu}$}, namely $\mathcal{M}_{\mu\nu}=\chi \Theta_{\mu\nu}$ (see Appendix \ref{sec:f(Q)_gravity_VF}, for derivation and  details), with \cite{DAmbrosio2022}
\begin{subequations} 
\begin{align} 
\mathcal{M}_{\mu\nu}&=\frac{2}{e}\nabla_\alpha(eP^\alpha_{\ \mu\nu}f_Q)+\frac{1}{e}q_{\mu\nu}f_Q-\frac{1}{2}g_{\mu\nu}f, \label{eq:field-equations}\\
\Theta_{\mu\nu}&=-\frac{2}{e}\frac{\partial\mathcal{L}_m}{\partial g^{\mu\nu}}, \label{eq:stress-energy-tensor}
\end{align} 
\end{subequations} 
where 
\begin{subequations} 
\begin{align} 
P^\alpha_{\ \mu\nu}&=\frac{1}{2e}\frac{\partial(e Q)}{\partial Q_\alpha{}^{\mu\nu}}=\notag\\
&=-\frac{Q^\alpha_{\ \mu\nu}}{4}+\frac{Q_{(\mu}{}^\alpha_{\ \nu)}}{4}+\frac{g_{\mu\nu}\tilde{Q}^\alpha}{4}-\frac{\delta^\alpha_{\ (\mu}Q_{\nu)}}{8},\label{eq:P}\\
\frac{1}{e}q_{\mu\nu}&=\frac{1}{e}\frac{\partial(e Q)}{\partial g^{\mu\nu}}+\frac{1}{2}g_{\mu\nu}Q\notag\\
&=P_{\nu\rho\sigma}Q_\mu{}^{\rho\sigma}-2P_{\rho\sigma\mu}Q^{\rho\sigma}{}_\nu-\frac{1}{2}g_{\mu\nu}Q.\label{eq:qmunu}
\end{align} 
\end{subequations} 
$\mathcal{M}_{\mu\nu}$ can be recast in the GR-like form as follows\cite{DAmbrosio2022}
\begin{align} 
\mathcal{M}_{\mu\nu}&=f_Q\lc{G}_{\mu\nu}-\frac{1}{2}g_{\mu\nu}\biggr{[}f-f_QQ\biggr{]}+2P^\alpha_{\ \mu\nu}\pa_\alpha f_{Q},\label{eq:FE}
\end{align}
where $\lc{G}_{\mu\nu}$ is the GR Einstein tensor, and the last term can be written as $2P^\alpha_{\ \mu\nu}\pa_\alpha f_{Q}=2P^\alpha_{\ \mu\nu}f_{QQ}\partial_\alpha Q$.
Here, we use the following condition 
\begin{equation} \label{eq:COND_EQ}
\lc{G}_{\mu\nu}=\frac{2}{e}\nabla_\alpha(2eP^\alpha{}_{\mu\nu})+\frac{1}{e}q_{\mu\nu}-\frac{1}{2} g_{\mu\nu}Q.
\end{equation}
See Eq. (145) in Ref. \cite{Capozziello2022}.
 

If we derive the action with respect to the affine connection (i.e., $\delta_\Gamma S_Q=0$), we obtain the \emph{connection field equations} $\mathcal{C_\alpha}=0$, where \cite{DAmbrosio2022}
\begin{align} 
\mathcal{C_\alpha}&=\nabla_\mu\nabla_\nu\left[P^{\mu\nu}{}_\alpha e f_{Q}(Q)\right].\label{eq:CE}
\end{align}
The above equation can be found by introducing the Lagrange multipliers subjected to the constraints of vanishing torsion and curvature. Then the hypermomentum can be defined as follows  \cite{Beltran2018}
\begin{equation} 
\mathfrak{H}^\lambda{}_{\mu\nu}=-\frac{1}{2}\frac{\partial\mathcal{L}_m}{\partial \Gamma^\alpha{}_{\mu\nu}}.
\end{equation} 
Requiring the  hypermomentum conservation (i.e.,  $\nabla_\mu \nabla_\nu \mathfrak{H}_\alpha{}^{\mu\nu}=0$) and using the symmetry properties of the aforementioned Lagrange multipliers, we come to Eq. \eqref{eq:CE} (see Sec. IV--A in Ref. \cite{Beltran2018}, for more details). 

It is worth noticing that the conservation laws of the energy-momentum tensor with respect to the GR covariant divergence (i.e., $\lc{\nabla}^\mu \Theta_{\mu\nu}=0$) implies
\begin{align} \label{eq:cl_f(Q)}
\lc{\nabla}^\mu \mathcal{M}_{\mu\nu}&=\partial^\mu f_Q \left(\lc{G}_{\mu\nu}+\frac{1}{2}g_{\mu\nu}Q+2\lc{\nabla}^\lambda P_{\mu\lambda\nu}\right)\notag\\
&+2P_{\mu\lambda\nu}\lc{\nabla}^\lambda (\partial^\mu f_Q)=0,
\end{align} 
which is not identically satisfied, but it represents an additional constraint to be considered. Of course, Eq. \eqref{eq:cl_f(Q)}  holds in STEGR, as soon as  $f_Q=1$. 
 
\section{Improving the theory with a boundary term} 
\label{sec:f(Q,B)_gravity}
We propose an extension of $f(Q)$ gravity, by considering a generic smooth function of the non-metricity scalar and of the boundary term $B$, namely the $f(Q,B)$ gravity.
Let's  start from the  action
\begin{equation} \label{eq:f(Q,B)_action}
S_{QB}=\int   \dd^4x \  e  \biggr{[}\frac{1}{2\chi}f(Q,B)+\mathcal{L}_m\biggr{]}. 
\end{equation}
Varying $S_{QB}$ with respect to the metric (i.e., $\delta_g S_{QB}=0$), we obtain the following metric field equations (see Appendix \ref{sec:f(Q,B)_gravity_VF}, for their derivations and details)
\begin{align} \label{eq:EFE1}
&\frac{2}{e}\nabla_\alpha(P^\alpha_{\ \mu\nu}ef_Q)+q_{\mu\nu}f_Q-\frac{1}{2}g_{\mu\nu}f+\frac{1}{2}g_{\mu\nu}f_B B\notag\\
&+(\partial_\lambda f_B)\biggr{[}\frac{1}{2}g_{\mu\nu}\tilde{Q}^\lambda-U^\lambda_{\mu\nu}\biggr{]}=\chi\Theta_{\mu\nu},
\end{align}
where (cf. Eq. \ref{eq:Ulun})
\begin{align}
(\partial_\lambda f_B)U^\lambda_{\ \mu\nu}&=\left[\Gamma^\lambda_{\ \mu\nu}-\frac{1}{2}\left(\delta^\lambda_\mu \Gamma^\rho_{\ \nu\rho}+\delta^\lambda_\nu \Gamma^\rho_{\ \mu\rho}\right)\right.\notag\\
&-\lc{\Gamma}^\lambda_{\ \mu\nu}\biggr{]}(\pa_\lambda f_B)+\frac{1}{2}\left(\pa_\nu f_B\lc{\Gamma}^\rho_{\ \mu\rho}+\pa_\mu f_B\lc{\Gamma}^\rho_{\ \nu\rho}\right)\notag\\
&-\lc{\nabla}_\mu\lc{\nabla}_\nu f_B +g_{\mu\nu}\lc{\Box}f_B.
\end{align}
Using Eq. \eqref{eq:COND_EQ}, the field equations \eqref{eq:EFE1} become
\begin{align} \label{eq:EFE2}
&\lc{G}_{\mu\nu} f_Q-\frac{1}{2}g_{\mu\nu}\biggr{(}f-f_QQ-f_B B\biggr{)}+(\partial_\lambda f_Q)2P^\lambda_{\ \mu\nu}\notag\\
&-(\partial_\lambda f_B)\biggr{[}-\frac{1}{2}g_{\mu\nu}\tilde{Q}^\lambda+U^\lambda_{\mu\nu}\biggr{]}=\chi\Theta_{\mu\nu}.
\end{align}
The above equation can be also written as 
\begin{align} \label{eq:EFE2_bis}
&\lc{G}_{\mu\nu} f_Q-\frac{1}{2}g_{\mu\nu}\biggr{(}f-f_QQ-f_B B\biggr{)}+\partial_\lambda(f_Q+f_B)2P^\lambda_{\ \mu\nu}\notag\\
&-(\partial_\lambda f_B)\biggr{[}2P^\lambda_{\ \mu\nu}-\frac{1}{2}g_{\mu\nu}\tilde{Q}^\lambda+U^\lambda_{\mu\nu}\biggr{]}=\chi\Theta_{\mu\nu}.
\end{align}
It is possible to prove that (see Appendix \ref{sec:f(Q,B)_gravity_VF})
\begin{equation} \label{eq:boundary_term_mean}
2P^\lambda_{\ \mu\nu}-\frac{1}{2}g_{\mu\nu}\tilde{Q}^\lambda+U^\lambda_{\mu\nu}=g_{\mu\nu}\lc{\Box}f_B-\lc{\nabla}_\mu\lc{\nabla}_\nu f_B.
\end{equation}
Therefore, Eq. \eqref{eq:EFE2} becomes
\begin{subequations}  \label{eq:EFE-final}
\begin{align} 
\mathcal{H}_{\mu\nu}&=\chi\Theta_{\mu\nu},\label{eq:EFE3}\\
\mathcal{H}_{\mu\nu}&=\lc{G}_{\mu\nu} f_Q-\frac{1}{2}g_{\mu\nu}\biggr{(}f-f_QQ-f_B B\biggr{)}\notag\\
&+\partial_\lambda(f_Q+f_B)2P^\lambda_{\ \mu\nu}-g_{\mu\nu}\lc{\Box}f_B+\lc{\nabla}_\mu\lc{\nabla}_\nu f_B.\label{eq:EFE3b}
\end{align}
\end{subequations}
It is important to note that the addition of a boundary term $B$ fulfills an important role, because it allows the $f(Q)$ gravity to ascend from second to fourth order field equations (cf. Eq. \eqref{eq:boundary_term_mean}). The reader can find a discussion on $f(T,B)$ gravity in Refs. \cite{Bahamonde2015, Sebastian, Capriolo}.

Adopting the same strategy employed in Sec. \ref{sec:f(Q)_gravity} to derive Eq. \eqref{eq:CE}, we finally obtain the connection field equation in $f(Q,B)$ gravity, namely
\begin{equation} \label{eq:ECE}
\nabla_\mu\nabla_\nu\left[P^{\mu\nu}{}_\alpha e(f_Q+f_B)\right]=0.
\end{equation}

\subsection{Consistency check}
\label{sec:consistency}
The consistency of the ensued results can be checked by Eqs. \eqref{eq:EFE-final} and \eqref{eq:ECE}. 

\begin{enumerate}
%[1]
\item For $B=0$, we have $f_B=0$ and the field equations \eqref{eq:EFE-final} reduce to  $f(Q)$ gravity  \eqref{eq:FE}, as well as the connection equations  \eqref{eq:ECE} reduce to Eq. \eqref{eq:CE}.
%[2]
\item For $R=Q-B$, we have $f(\lc{R})=f(Q-B)$ and $F(\lc{R})=f_R(\lc{R})=f_Q=-f_B$. From this preliminary analysis, we immediately note that Eq. \eqref{eq:EFE-final} is equivalent to the $f(R)$ gravity, namely \cite{Faraoni2010}
\begin{align}
&\lc{G}_{\mu\nu} F-\frac{1}{2}g_{\mu\nu}\biggr{(}f-F\lc{R}\biggr{)}\notag\\
&+g_{\mu\nu}\lc{\Box}F-\lc{\nabla}_\mu\lc{\nabla}_\nu F=\chi\Theta_{\mu\nu}.
\end{align}
Clearly, Eq. \eqref{eq:ECE} is trivial in this framework.  
%[3]
\item Finally, we require that the stress-energy tensor is conserved under the action of the GR covariant divergence, namely $\lc{\nabla}^\mu \Theta_{\mu\nu}=0$, which implies 
\begin{align} \label{eq:cl_f(Q,B)}
\lc{\nabla}^\mu \mathcal{H}_{\mu\nu}&=\pa^\mu(f_Q+f_B)\biggr{[}\lc{R}_{\mu\nu}+\frac{1}{2}g_{\mu\nu}B+2\lc{\nabla}^\lambda P_{\lambda\mu\nu}\biggr{]}\notag\\
&+\lc{\nabla}^\lambda [\partial^\mu(f_Q+f_B)]2P_{\lambda\mu\nu}=0,
\end{align}
being not identically satisfied. This imposes thus a further constraint, as it already occurs in the $f(Q)$ gravity (see Eq. \eqref{eq:cl_f(Q)} and discussion below). However, it is important to note that Eq. \eqref{eq:cl_f(Q,B)} holds in the $f(Q-B)$ gravity, because $f_Q=-f_B$.
\end{enumerate}

\subsection{ The extension of Geometric Trinity of Gravity }
\label{sec:EGTG}
As said above,  Geometric Trinity of Gravity gives  three dynamically equivalent formulations of GR. They are based on  Lagrangians  containing $R,T,Q$ representing  the Ricci curvature, the torsion, and the non-metricity scalars, respectively. It is possible to demonstrate that these Lagrangians are equivalent up to a boundary term  which is different in GR, TEGR,  and STEGR. This equivalence practically means that TEGR and STEGR have the same field equations of GR.

However, the main issue arises, when we pass to the related extended theories, represented by $f(R),f(T),f(Q)$ gravities, which are not dynamically equivalent, because  $f(R)$ gravity, in metric formalism,  is a  fourth-order theory, whereas  $f(T)$ and $f(Q)$ are second-order theories. Nevertheless,  it is possible to restore the equivalence among  extended theories via the addition of an appropriate boundary term. Indeed, in the general frameworks $f(T,B_1)$ and $f(Q,B_2)$ (where usually $B_1\neq B_2$), we have that $f(B_1-T),f(Q-B_2)$ are dynamically equivalent to $f(R)$ gravity. These three theories constitute what we may dub \emph{Extended Geometric Trinity of Gravity} (see Fig. \ref{fig:Fig1}). While $R,B_1-T,Q-B_2$ is a geometric trinity of gravity of second-order, $f(R), f(B_1-T),f(Q-B_2)$ configures to be as a geometric trinity of gravity of fourth-order. In summary, adding  boundary terms means to improve the number of degrees of freedom because they act as effective scalar fields.  Clearly, this procedure can be extended to higher-order metric-affine theories formulated in metric, teleparallel and symmetric teleparallel formalisms.
% Figure environment removed

\section{Discussion and conclusions}
\label{sec:end}
In this paper, we discussed $f(Q,B)$ gravity, which is an extension  of Symmetric Teleparallel Gravity endowed with some interesting properties. Thanks to the introduction of an appropriate boundary term $B$, it is possible to lift up the $f(Q)$ gravity from  second-order to   fourth-order in the field equations.   In particular,  the  model $f(Q,B)=f(Q-B)$ is dynamically equivalent to $f(R)$ gravity. Furthermore, the theory is consistent with $f(Q)$ gravity in the limit of $B\to0$.  We can say that   $f(R),f(B_1-T),f(Q-B_2)$ (with $B_1\neq B_2$) give rise to an  extension of trinity gravity so that we can deal with  the {\it Extended Geometric Trinity of Gravity} being the three formulations of fourth-order in the field equations. 

It is worth noticing that these results   provide a sort of route to extend equivalent representations of gravity to any higher-order dynamics (see also \cite{Capriolo2}). This fact can be extremely interesting in cosmological  and astrophysical applications. For  example, as mentioned above, being $f(R)$ gravity of fourth-order in metric formalism, it seems that it cannot be directly compared with $f(T)$ and $f(Q)$ which are of second order  like TEGR and STEGR.  If one considers cosmology, for example, it seems that dynamics coming from    Starobinsky gravity \cite{Starobinsky}, i.e. $f(R)=R+\alpha R^2$, cannot be compared with $f(T)=T+\alpha_1 T^2$ or $f(Q)=Q+\alpha_2 Q^2$ \cite{Shokri} due to the different order of related field equations. The introduction of boundary terms  restores the same differential order into the  equations and then information coming from  different representations of gravity appears on the same ground. Similar considerations can be developed for black hole solutions or, in general, for any self-gravitating compact object.

In a  forthcoming paper, we will study  how boundary terms impact   phenomenology.   In particular, it is straightforward to show that  the matching of cosmological  models with     observations results improved thanks to the presence of boundary terms.




\section*{Acknowledgements}
This paper is based upon work from COST Action CA21136 Addressing observational tensions in cosmology with systematics and fundamental physics (CosmoVerse) supported by COST (European Cooperation in Science and Technology). Authors acknowledge the Istituto Nazionale di
Fisica Nucleare (iniziative specifiche QGSKY, TEONGRAV and MOONLIGHT2).
S.C. and V.D.F. thank the  Gruppo Nazionale di Fisica Matematica of Istituto Nazionale di Alta Matematica for the support.  We are  grateful to  Francesco Bajardi for  useful discussions and suggestions.

\appendix
\section{Variational principle for $f(Q)$ gravity and field equations}
\label{sec:f(Q)_gravity_VF}
Starting from Eq. \eqref{eq:f(Q)_action} and considering the variation of the action with respect to the metric, we have
\begin{align} \label{eq:starting_eq}
\delta_g(ef)&=\biggr{[}\del{e} f+ e f_Q \del{Q}\biggr{]}\delta_g g^{\mu\nu},
\end{align}
where\footnote{The variation $\pg{e}=-\frac{1}{2}g_{\mu\nu}e$ can be obtained via the \emph{Jacobi formula}, expressing the derivative of the determinant of a matrix $A$ in terms of the adjugate of $A$ and the derivative of $A$ \cite{Magnus1999}.}
\begin{subequations}
\begin{align}
f\del{e}&=-\frac{1}{2}g_{\mu\nu}ef \del g^{\mu\nu},\\
ef_Q \del{Q}&=ef_Q\left[\pg{Q}\del g^{\mu\nu}+\frac{\partial Q}{\partial Q_\alpha^{\ \mu\nu}}\del Q_\alpha^{\ \mu\nu}\right]\notag\\
&-\pg{e}f_Q Q\del g^{\mu\nu}-e\pg{f_Q}Q\del g^{\mu\nu}\notag\\
&+ef_Q(2P^\alpha_{\ \mu\nu})(-\nabla_\alpha\del g^{\mu\nu})\notag\\
&=\Biggr{[}\frac{1}{2}g_{\mu\nu}f_QQ+f_Q\frac{1}{e}\frac{\pa (eQ)}{\pa g^{\mu\nu}}\notag\\
&\qquad+\frac{1}{e}\nabla(2ef_QP^\alpha_{\ \mu\nu})\Biggr{]}e\del g^{\mu\nu}\notag\\
&=\Biggr{[}\frac{1}{e}q_{\mu\nu}f_Q+\frac{1}{e}\nabla(2ef_QP^\alpha_{\ \mu\nu})\Biggr{]}e\del g^{\mu\nu}.
\end{align}
\end{subequations}
In the above implications we have used the definition of $P^\alpha_{\ \mu\nu}$ (cf. Eq. \eqref{eq:P})\footnote{In Eq. \eqref{eq:equ2}, we have used the following relation
\begin{align}
Q_\alpha^{\ \mu\nu}&=g^{\mu\beta}g^{\nu\gamma}Q_{\alpha\beta\gamma}=g^{\mu\beta}g^{\nu\gamma}(\pa_\alpha g_{\beta\gamma}-\Gamma^\lambda_{\ \alpha\beta}g_{\lambda\gamma}-\Gamma^\lambda_{\ \alpha\gamma}g_{\lambda\beta})\notag\\
&=\pa_\alpha(g^{\mu\beta}g^{\nu\gamma} g_{\beta\gamma})- g^{\nu\gamma} g_{\beta\gamma}\pa_\alpha g^{\mu\beta}- g^{\mu\beta} g_{\beta\gamma}\pa_\alpha g^{\nu\gamma}\notag\\
&-g^{\mu\beta}g^{\nu\gamma}g_{\lambda\gamma}\Gamma^\lambda_{\ \alpha\beta}-g^{\mu\beta}g^{\nu\gamma}g_{\lambda\beta}\Gamma^\lambda_{\ \alpha\gamma}\notag\\
&=-(\pa_\alpha g^{\mu\nu}+\Gamma^\nu_{\alpha\beta}g^{\mu\beta}+\Gamma^\mu_{\alpha\beta}g^{\nu\beta})=-\nabla_\alpha g^{\mu\nu}. \label{eq:equ2}
\end{align}}, and $q_{\mu\nu}$ (see Eq. \eqref{eq:qmunu}). For an alternative calculation of $\del Q$, we suggest the reader to see Appendices A and B in Ref. \cite{Xu2019}.

Inserting the above findings in Eq. \eqref{eq:starting_eq}, we obtain 
\begin{align}
\frac{2}{e}\nabla_\alpha(eP^\alpha_{\ \mu\nu}f_Q)+\frac{1}{e}q_{\mu\nu}f_Q-\frac{1}{2}g_{\mu\nu}f=\chi \Theta_{\mu\nu}.
\end{align}

\section{Variational principle for  $f(Q,B)$ gravity and field equations}
\label{sec:f(Q,B)_gravity_VF}
Starting from the modified action \eqref{eq:f(Q,B)_action} and varying it with respect to the metric, we obtain
\begin{subequations} \label{eq:start}
\begin{align} 
\delta_g(e\mathcal{L}_m)&=-e\chi\Theta_{\mu\nu}\del g^{\mu\nu}, \label{eq:start_a}\\
\delta_g(ef)&=\del e f+e f_Q\del{Q}+ ef_B \del{B}, \label{eq:start_b}
\end{align} 
\end{subequations} 
where
\begin{subequations} \label{eq:second}
\begin{align}
&\del e f+e f_Q\del{Q}=e\mathcal{M}_{\mu\nu},\\
&ef_B \del{B}=ef_B\del\left[\frac{1}{e}\partial_\lambda (e\tilde{Q}^\lambda)\right]\notag\\
&=\frac{1}{2}g_{\mu\nu}e f_B B\del g^{\mu\nu}+(\pa_\lambda f_B)\left(\frac{1}{2}g_{\mu\nu}\tilde{Q}^\lambda\right)e\del g^{\mu\nu}\notag\\
&\quad-(\pa_\lambda f_B)(e\del \tilde{Q}^\lambda)\notag\\
&=\biggr{[}\frac{g_{\mu\nu}}{2}f_B B +(\partial_\lambda f_B)\biggr{(}\frac{g_{\mu\nu}}{2}\tilde{Q}^\lambda-U^\lambda_{\mu\nu}\biggr{)}\biggr{]}e\del g^{\mu\nu},
\end{align}
\end{subequations} 
with $e(\pa_\lambda f_B)\del \tilde{Q}^\lambda=eU^\lambda_{\ \mu\nu} \del g^{\mu\nu}$. Let us explicitly calculate this last term, where we obtain
\begin{align} \label{eq:first_term}
e(\pa_\lambda f_B)\del \tilde{Q}^\lambda&=\del{}\biggr{[}g^{\lambda\alpha}g^{\sigma\rho}(\partial_\alpha g_{\sigma\rho}-\partial_\rho g_{\sigma\alpha})\notag\\
&-g^{\lambda\alpha}\Gamma^\rho_{\ \alpha\rho}+g^{\sigma\rho}\Gamma^\lambda_{\ \sigma\rho}\biggr{]}e(\partial_\lambda f_B)\notag\\
&=\underbrace{e(\partial_\lambda f_B)\left[\del{(g^{\lambda\alpha}g^{\sigma\rho})}\right](\partial_\alpha g_{\sigma\rho}-\partial_\rho g_{\sigma\alpha})}_{(A_1)}\notag\\
&+\underbrace{e(\partial_\lambda f_B)g^{\lambda\alpha}g^{\sigma\rho}\left[\del{}(\partial_\alpha g_{\sigma\rho}-\partial_\rho g_{\sigma\alpha})\right]}_{(A_2)}\notag\\
&+\underbrace{e(\partial_\lambda f_B)\del{}(g^{\sigma\rho}\Gamma^\lambda_{\ \sigma\rho}-g^{\lambda\alpha}\Gamma^\rho_{\ \alpha\rho})}_{(A_3)}.
\end{align}
Calculating all the above terms, we eventually have\footnote{We remind that the Levi-Civita affine connection $\lc{\Gamma}^\lambda_{\ \mu\nu}$ is
\begin{equation}
\lc{\Gamma}^\lambda_{\ \mu\nu}=\frac{1}{2}g^{\lambda\alpha}\left(\partial_\mu g_{\alpha\nu}+\partial_\nu g_{\alpha\mu}-\partial_\alpha g_{\mu \nu}\right).
\end{equation}}
\begin{align}
A_1&=\left\{-\lc{\Gamma}^\lambda_{\ \mu\nu}+\frac{1}{2}\partial_\alpha g_{\mu\nu} g^{\lambda\alpha}+\frac{1}{2}g^{\sigma\rho}(\delta^\lambda_\mu \partial_\nu g_{\sigma\rho}+\delta^\lambda_\nu \partial_\mu g_{\sigma\rho})\right.\notag\\
&\left.-\frac{1}{2}g^{\sigma\rho}(\delta^\lambda_\mu \partial_\rho g_{\sigma\nu}+\delta^\lambda_\nu \partial_\rho g_{\sigma\mu})\right\}e(\pa_\lambda f_B)\del g^\mu\nu,\\
A_2&=\left\{\Biggr{[}-2g^{\lambda\alpha}\partial_\alpha g_{\mu\nu}+\frac{1}{2}g^{\sigma\rho}\left(\delta^\lambda_\nu \partial_\rho g_{\mu\sigma}+\delta^\lambda_\mu\partial_\rho g_{\nu\sigma}\right)\right.\notag\\
&+\frac{1}{2}g^{\lambda\alpha}(\partial_\mu g_{\nu\alpha}+\partial_\nu g_{\mu\alpha})+\pa_\alpha(g^{\lambda\alpha}g_{\mu\nu})\notag\\
& +\lc{\Gamma}^\rho_{\ \alpha\rho}g^{\lambda\alpha}g_{\mu\nu}\Biggr{]}(\pa_\lambda f_B)+g^{\alpha\beta}g_{\mu\nu}\pa_\alpha\pa_\beta f_B -\pa_\mu \pa_\nu f_B\notag\\
&\left.-\frac{1}{2}\left(\pa_\nu f_B \lc{\Gamma}^\rho_{\ \mu\rho}+\pa_\mu f_B \lc{\Gamma}^\rho_{\ \nu\rho}\right)\right\}e\del g^{\mu\nu},\\
A_3&=\left\{\Gamma^\lambda_{\ \mu\nu}-\frac{1}{2}\left(\delta^\lambda_\mu \Gamma^\rho_{\ \nu\rho}+\delta^\lambda_\nu \Gamma^\rho_{\ \mu\rho}\right)\right\}e(\pa_\lambda f_B)\del g^{\mu\nu},
\end{align}
where in $A_1$ and $A_3$ we have employed 
\begin{align}
\pg{g^{\rho\sigma}}=\frac{1}{2}\left(\delta^{\mu\rho}\delta^{\nu\sigma}+\delta^{\nu\rho}\delta^{\mu\sigma}\right);
\end{align}
instead in $A_2$ we have exchanged $\partial_\alpha$ and $\delta_g$ and used
\begin{subequations}
\begin{align}
\del g_{\rho\sigma}&=-\frac{1}{2}\left(g_{\rho\mu}g_{\sigma\nu}+g_{\sigma\mu}g_{\rho\nu}\right)\del g^{\mu\nu},\\
\del e&=e \lc{\Gamma}^\rho_{\ \alpha\rho}.
\end{align}
\end{subequations}

Therefore, from Eqs. \eqref{eq:first_term} we have (without $e\del g^{\mu\nu}$)
\begin{align} \label{eq:Ulun}
(\partial_\lambda f_B)U^\lambda_{\ \mu\nu}&=A_1+A_2+A_3\notag\\
&=\left[\Gamma^\lambda_{\ \mu\nu}-\frac{1}{2}\left(\delta^\lambda_\mu \Gamma^\rho_{\ \nu\rho}+\delta^\lambda_\nu \Gamma^\rho_{\ \mu\rho}\right)\right.\notag\\
&-\lc{\Gamma}^\lambda_{\ \mu\nu}\biggr{]}(\pa_\lambda f_B)+\frac{1}{2}\left(\pa_\nu f_B\lc{\Gamma}^\rho_{\ \mu\rho}+\pa_\mu f_B\lc{\Gamma}^\rho_{\ \nu\rho}\right)\notag\\
&-\lc{\nabla}_\mu\lc{\nabla}_\nu f_B +g_{\mu\nu}\lc{\Box}f_B,
\end{align}
where we have used the following identities
\begin{subequations}
\begin{align}
&\partial_\alpha g_{\beta\gamma}=\lc{\Gamma}^\rho_{\ \alpha\beta}g_{\rho\gamma}+\lc{\Gamma}^\rho_{\ \alpha\gamma}g_{\rho\beta},\\
&\partial_\alpha g^{\beta\gamma}=-g^{\beta\gamma}g^{\gamma\sigma}\partial_\alpha g_{\rho\sigma},\\
&\frac{1}{2}g^{\sigma\rho}(\partial_\lambda f_B)(\delta^\lambda_\mu\pa_\nu g_{\sigma\rho}+\delta^\lambda_\nu\pa_\mu g_{\sigma\rho})\notag\\
&=\frac{1}{2}\left[\pa_\mu f_B\lc{\Gamma}^\rho_{\ \nu\rho}+\pa_\nu f_B\lc{\Gamma}^\rho_{\ \mu\rho}\right].
\end{align}
\end{subequations}

Starting from Eqs. \eqref{eq:start} and considering Eqs. \eqref{eq:second} and \eqref{eq:Ulun}, we obtain the following metric field equations \eqref{eq:EFE1}. Now, we calculate the following terms 
\begin{align} \label{eq:P+Q}
2P^\lambda_{\ \mu\nu}-\frac{1}{2}g_{\mu\nu}\tilde{Q}^\lambda&=-\frac{1}{2}Q^\lambda_{\ \mu\nu}+\frac{1}{2}\left(Q_\mu{}^\lambda_{\ \nu}-Q_\nu{}^\lambda_{\ \mu}\right)\notag\\
&-\frac{1}{4}\left(\delta^\lambda_\mu Q_\nu+\delta^\lambda_\nu Q_\mu\right)\notag\\
&=\lc{\Gamma}^\lambda_{\ \mu\nu}-\frac{1}{4}\left(\delta^\lambda_\mu\partial_\nu g_{\sigma\rho}+\delta^\lambda_\nu\partial_\mu g_{\sigma\rho}\right)g^{\sigma\rho}\notag\\
&-\Gamma^\lambda_{\ \mu\nu}+\frac{1}{2}\left(\delta^\lambda_\mu \Gamma^\alpha_{\ \nu\alpha}+\delta^\lambda_\nu \Gamma^\alpha_{\ \mu\alpha}\right)\notag\\
&=\lc{\Gamma}^\lambda_{\ \mu\nu}-\frac{1}{2}\left(\pa_\nu f_B\lc{\Gamma}^\rho_{\ \mu\rho}+\pa_\mu f_B\lc{\Gamma}^\rho_{\ \nu\rho}\right)\notag\\
&-\Gamma^\lambda_{\ \mu\nu}+\frac{1}{2}\left(\delta^\lambda_\mu \Gamma^\alpha_{\ \nu\alpha}+\delta^\lambda_\nu \Gamma^\alpha_{\ \mu\alpha}\right).
\end{align}

After simple algebra, we obtain Eq. \eqref{eq:EFE2_bis}. Gathering Eqs. \eqref{eq:Ulun} and \eqref{eq:P+Q}, we prove Eq. \eqref{eq:boundary_term_mean}. Defined $\Psi_{\mu\nu}=(\partial_\lambda f_B)(2P^\lambda_{\ \mu\nu}-\frac{1}{2}g_{\mu\nu}\tilde{Q}^\lambda+U^\lambda_{\ \mu\nu})$, we have
\begin{align} \label{eq:Psi_1}
\Psi_{\mu\nu}&=g_{\mu\nu}\lc{\Box}f_B-\lc{\nabla}_\mu\lc{\nabla}_\nu f_B.
\end{align}
The field equations of $f(Q,B)$ gravity are (cf. Eq. \eqref{eq:EFE-final})
\begin{align} 
&\lc{G}_{\mu\nu} f_Q-\frac{1}{2}g_{\mu\nu}\biggr{(}f-f_QQ-f_B B\biggr{)}+\partial_\lambda(f_Q\notag\\
&+f_B)2P^\lambda_{\ \mu\nu}-g_{\mu\nu}\lc{\Box}f_B+\lc{\nabla}_\mu\lc{\nabla}_\nu f_B=\chi\Theta_{\mu\nu}.
\end{align}

\bibliography{references}

\end{document}
