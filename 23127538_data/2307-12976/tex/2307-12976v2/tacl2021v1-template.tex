% File tacl2021v1.tex
% Dec. 15, 2021

% The English content of this file was modified from various *ACL instructions
% by Lillian Lee and Kristina Toutanova
%
% LaTeXery is mostly all adapted from acl2018.sty.

\documentclass[11pt,a4paper]{article}
\usepackage{times,latexsym}
\usepackage{url}
\usepackage[T1]{fontenc}

%% Package options:
%% Short version: "hyperref" and "submission" are the defaults.
%% More verbose version:
%% Most compact command to produce a submission version with hyperref enabled
%%    \usepackage[]{tacl2021v1}
%% Most compact command to produce a "camera-ready" version
%%    \usepackage[acceptedWithA]{tacl2021v1}
%% Most compact command to produce a double-spaced copy-editor's version
%%    \usepackage[acceptedWithA,copyedit]{tacl2021v1}
%
%% If you need to disable hyperref in any of the above settings (see Section
%% "LaTeX files") in the TACL instructions), add ",nohyperref" in the square
%% brackets. (The comma is a delimiter in case there are multiple options specified.)

\usepackage[acceptedWithA]{tacl2021v1}
% \setlength\titlebox{10cm} % <- for Option 2 below

%%%% Material in this block is specific to generating TACL instructions
\usepackage{xspace,mfirstuc,tabulary}
\newcommand{\dateOfLastUpdate}{Dec. 15, 2021}
\newcommand{\styleFileVersion}{tacl2021v1}

\newcommand{\ex}[1]{{\sf #1}}

\newif\iftaclinstructions
\taclinstructionsfalse % AUTHORS: do NOT set this to true
\iftaclinstructions
\renewcommand{\confidential}{}
\renewcommand{\anonsubtext}{(No author info supplied here, for consistency with
TACL-submission anonymization requirements)}
\newcommand{\instr}
\fi

%
\iftaclpubformat % this "if" is set by the choice of options
\newcommand{\taclpaper}{final version\xspace}
\newcommand{\taclpapers}{final versions\xspace}
\newcommand{\Taclpaper}{Final version\xspace}
\newcommand{\Taclpapers}{Final versions\xspace}
\newcommand{\TaclPapers}{Final Versions\xspace}
\else
\newcommand{\taclpaper}{submission\xspace}
\newcommand{\taclpapers}{{\taclpaper}s\xspace}
\newcommand{\Taclpaper}{Submission\xspace}
\newcommand{\Taclpapers}{{\Taclpaper}s\xspace}
\newcommand{\TaclPapers}{Submissions\xspace}
\fi

%%%% End TACL-instructions-specific macro block
%%%%


\usepackage{times}
\usepackage{latexsym}
\usepackage{booktabs}
\usepackage[T1]{fontenc}
\usepackage[utf8]{inputenc}
\usepackage{microtype}
\usepackage{inconsolata}
\usepackage{graphicx}
\usepackage{multirow}
\usepackage{cleveref}

\usepackage[shortlabels]{enumitem}

\interfootnotelinepenalty=10000

\newcommand{\ripple}{\textsc{RippleEdits}}
\newcommand{\recentlyemerged}{\textsc{Recent}}
\newcommand{\fakefacts}{\textsc{Random}}
\newcommand{\topviews}{\textsc{Popular}}

\newcommand{\wikidata}{\textsc{WikiData}}   
\newcommand{\wikipedia}{\textsc{Wikipedia}}

\newcommand{\gpt}{\textsc{GPT-2}}
\newcommand{\gptm}{\textsc{GPT2-M}}
\newcommand{\gptl}{\textsc{GPT2-L}}
% \newcommand{\gptxl}{\textsc{GPT2-XL}}
\newcommand{\gptxl}{\textsc{GPT-2}}
\newcommand{\gptt}{\textsc{GPT-3}}
\newcommand{\gptf}{\textsc{GPT-4}}
\newcommand{\gptj}{\textsc{GPT-J}}
\newcommand{\gptneo}{\textsc{GPT-Neo}}
\newcommand{\llama}{\textsc{LLaMA}}

\newcommand{\mend}{\textsc{MEND}}
\newcommand{\rome}{\textsc{ROME}}
\newcommand{\memit}{\textsc{MEMIT}}
\newcommand{\incontext}{\textsc{IN-CONTEXT Editing}}

% \newcommand{\axes}{\textsc{Evaluation Criteria}}
\newcommand{\numtests}{\textsc{6}}
\newcommand{\logicalgeneralization}{\emph{Logical Generalization}}
\newcommand{\compositionality}{\emph{Compositionality I}}
\newcommand{\forwardcompositionality}{\emph{Compositionality II}}
\newcommand{\relationspecificity}{\emph{Relation Specificity}}
\newcommand{\forgetfulness}{\emph{Preservation}}
\newcommand{\aliasing}{\emph{Subject Aliasing}}
\newcommand{\avg}{\emph{AVG}}


\definecolor{myblue}{RGB}{61,133,198}
\definecolor{mygreen}{RGB}{147,196,125}
\definecolor{myorange}{RGB}{246,178,107}


\newcommand{\relationset}{\textsc{Relation Set}}

\newif\ifcomments
\commentstrue
\ifcomments
\newcommand{\mg}[1]{\textcolor{purple}{MG: #1}}
\newcommand\oy[1]{\textcolor{magenta}{[OY: #1]}}
\newcommand\ag[1]{\textcolor{red}{[AG: #1]}}
\newcommand\rc[1]{\textcolor{blue}{[RC: #1]}}
\else
\newcommand{\mg}[1]{}
\newcommand\oy[1]{}
\newcommand\ag[1]{}
\fi

\title{Evaluating the Ripple Effects of Knowledge Editing in Language Models}

\author{Roi Cohen$^1$~~~~Eden Biran$^1$~~~~Ori Yoran$^1$~~~~Amir Globerson$^{1,2}$~~~~Mor Geva$^{1,2,\thanks{~~Work done at Google DeepMind.}}$ \vspace{5pt}\\
$^1$Blavatnik School of Computer Science, Tel Aviv University\vspace{5pt}~~~$^2$Google Research\\
\small{\texttt{\{roi1, edenbiran, oriy\}@mail.tau.ac.il, \{gamir, morgeva\}@tauex.tau.ac.il}}\\
}

% Author information does not appear in the pdf unless the "acceptedWithA" option is given

% The author block may be formatted in one of two ways:

% Option 1. Author’s address is underneath each name, centered.

% \author{
%   Template Author1\Thanks{The {\em actual} contributors to this instruction
%     document and corresponding template file are given in Section
%     \ref{sec:contributors}.} 
%   \\
%   Template Affiliation1/Address Line 1
%   \\
%   Template Affiliation1/Address Line 2
%   \\
%   Template Affiliation1/Address Line 2
%   \\
%   \texttt{template.email1example.com}
%   \And
%   Template Author2 
%   \\
%   Template Affiliation2/Address Line 1
%   \\
%   Template Affiliation2/Address Line 2
%   \\
%   Template Affiliation2/Address Line 2
%   \\
%   \texttt{template.email2@example.com}
% }

% % Option 2.  Author’s address is linked with superscript
% % characters to its name, author names are grouped, centered.

% \author{
%   Template Author1\Thanks{The {\em actual} contributors to this instruction
%     document and corresponding template file are given in Section
%     \ref{sec:contributors}.}$^\diamond$ 
%   \and
%   Template Author2$^\dagger$
%   \\
%   \ \\
%   $^\diamond$Template Affiliation1/Address Line 1
%   \\
%   Template Affiliation1/Address Line 2
%   \\
%   Template Affiliation1/Address Line 2
%   \\
%   \texttt{template.email1example.com}
%   \\
%   \ \\
%   \\
%   $^\dagger$Template Affiliation2/Address Line 1
%   \\
%   Template Affiliation2/Address Line 2
%   \\
%   Template Affiliation2/Address Line 2
%   \\
%   \texttt{template.email2@example.com}
% }

\date{}

\usepackage{setspace}
% \doublespacing
\begin{document}
\maketitle

\begin{abstract}
Graph Neural Networks (GNNs) have proven to be effective in processing and learning from graph-structured data.
However, previous works mainly focused on understanding single graph inputs while many real-world applications require pair-wise analysis for graph-structured data (e.g., scene graph matching, code searching, and drug-drug interaction prediction).
To this end, recent works have shifted their focus to learning the interaction between pairs of graphs.
Despite their improved performance, these works were still limited in that the interactions were considered at the node-level, resulting in high computational costs and suboptimal performance.
To address this issue, we propose a novel and efficient graph-level approach for extracting interaction representations using co-attention in graph pooling. 
Our method, Co-Attention Graph Pooling (CAGPool), exhibits competitive performance relative to existing methods in both classification and regression tasks using real-world datasets, while maintaining lower computational complexity.

\end{abstract}

\section{Introduction}
Deep learning models have been widely used in many applications.
For example, BERT~\citep{devlin_bert_2019}, GPT-3~\citep{brown_language_2020}, and T5~\citep{raffel_exploring_2020} achieved state-of-the-art~(SOTA) results on different natural language processing~(NLP) tasks. 
For computer vision~(CV), Transformer-like models such as ViT~\citep{dosovitskiy_image_2021} and Swin Transformer~\citep{liu_swin_2021} deliver excellent accuracy performance upon multiple tasks. 


At the same time, training deep learning models has been a critical problem troubling the community due to the long training time, especially for those large models with billions of parameters~\citep{brown_language_2020}. 
In order to enhance the training efficiency, researchers propose some manually designed parallel training strategies~\citep{narayanan_efficient_2021,shazeer_mesh-tensorflow_2018,xu_gspmd_2021}. 
However, selecting, tuning, and combining these strategies require extensive domain knowledge in deep learning models and hardware environments. With the increasing diversity of modern hardware architectures~\cite{flynn_very_1966,flynn_computer_1972} and the rapid development of deep learning models, these manually designed approaches are bringing heavier burdens to developers. 
Hence, \emph{automatic parallelism} is introduced to automate the parallel strategy searching for training models.


There are two main categories of parallelism in deep learning models: inter-layer parallelism~\citep{huang_gpipe_2019,narayanan_pipedream_2019,narayanan_memory-efficient_2021,fan_dapple_2021,li_chimera_2021,lepikhin_gshard_2021,du_glam_2022,fedus_switch_2022} and intra-layer parallelism~\citep{li_pytorch_2020,narayanan_efficient_2021,rasley_deepspeed_2020,fairscale_authors_fairscale_2021}. 
Inter-layer parallelism partitions the model into disjoint sets on different devices without slicing tensors. 
Alternatively, intra-layer parallelism partitions tensors in a layer along one or more axes and distributes them across different devices.


Current automatic parallelism techniques focus on optimizing strategies within these two categories. However, they treat these two categories separately. 
Some methods~\citep{zhao_vpipe_2022,jia_exploring_2018,cai_tensoropt_2022,wang_supporting_2019,jia_beyond_2019,schaarschmidt_automap_2021,liu_colossal-auto_2023} overlook potential opportunities for inter- or intra-layer parallelism, the others optimize inter- and intra-layer parallelism hierarchically and sequentially~\citep{narayanan_pipedream_2019,fan_dapple_2021,he_pipetransformer_2021,tarnawski_efficient_2020,tarnawski_piper_2021,zheng_alpa_2022}. 
As a result, current automatic parallelism techniques often fail to achieve the global optima and instead become trapped in local optima. 
Therefore, a unified inter- and intra-layer approach is needed to enhance the effectiveness of automatic parallelism.


This paper aims to find the optimal parallelism strategy while simultaneously considering inter- and intra-layer parallelism. 
It enables us to search in a more extensive strategy space where the globally optimal solution lurk. 
However, unifying inter- and intra-layer parallelism in automatic parallelism brings us two challenges. 
Firstly, to adopt a unified perspective on the inter- and intra-layer automatic parallelism, we should not formalize them with separate formulations as prior works. Therefore, how can we express these parallelism strategies in a unified formulation? 
Secondly, previous methods take a long time to obtain the solution with a limited strategy space. Therefore, how can we ensure that the best solution can be obtained in a reasonable time while expanding the strategy space?


To solve the above challenges, we propose UniAP. For the first challenge, UniAP adopts the mixed integer quadratic programming~(MIQP)~\citep{lazimy_mixed_1982} to search for the globally optimal parallel strategy automatically. 
It unifies the inter- and intra-layer automatic parallelism in a single MIQP formulation. 
For the second challenge, our complexity analysis and experimental results show that UniAP can obtain the globally optimal solution in a significantly shorter time.


The contributions of this paper are summarized as follows: 
\begin{itemize}
    \item We propose UniAP, the first framework to unify inter- and intra-layer automatic parallelism in model training.
    \item The optimal parallel strategies discovered by UniAP exhibit scalability on training throughput and strategy searching time.
    \item The experimental results show that UniAP speeds up model training on four Transformer-like models by up to 1.70$\times$ and reduces the strategy searching time by up to 16$\times$, compared with the SOTA method.
\end{itemize}

% \section{Preliminaries}
% \paragraph{Input feature attribution methods.}
% Consider a linear model $f(x) = w_1 x_1 + w_2 x_2$. To explain which feature is more important for predicting the value of f(x), we can compare their coefficients. If $w_1 = 1000$ and $w_2 = 0.01$, we can say that $x_1$ would be weighed more than $x_2$. This type of explanation assumes that the values of $x_1$ and $x_2$ are of the same order. This is true in the case of most inputs to the neural network models, for example image pixels. 
% Gradients are the general way of discussing the coefficient with respect to a particular feature to discuss its importance.
% \textbf{Element-wise product of gradient into input} \textsc{grad $\odot$ input} \cite{Shrikumar2016NotJA}, provides global importance about the input feature in the model's output. 
% \cite{} have used it show the feature importance in attention models. 
% , as compared to just the gradient.
% details of computing the attribution with math 
% Assume $\mathbf{x}$ is a real-valued input feature vector (for any modality). For discrete inputs, real-valued vector obtained after passing the feature through a look-up embedding.

% , but there is no clear superior attribution technique over another. 

% Instead of considering attributions over pixels, \textbf{XRAI} \cite{Kapishnikov2019XRAIBA} computes the effective attributions of integrated gradients over overly segmented image. The image is segmented based on similarity such as color, which makes the segment boundaries align with the edges. The segmentation is done at multiple scales to obtain a set of overlapping image segments.
% Assume that attribution mask over an image $I$ of size ${H\times W}$ is $A$ of the same size. 
% Using graph-based segmentations over multiple scale parameters, we obtain a set of segments $\mathcal{S}$. 
% Let a pixel be indexed by $i$ in the original image. For a segment $s$, the gain can be calculated by $g_s = \sum_{i \in s\backslash M} \frac{A_i}{area(s\backslash M)}$. 
% The segment with maximum gain is selected as  attribution to update the XRAI saliency set $\mathcal{M}$.
% The process is repeated with the remaining segments until the area of the mask set is equal to that of the image. 
% While this method seems to produce slightly better visual attributions over other variants of IG, it is sensitive to the size of segmentation scales and dilation factor. We consider  $XRAI(\cdot)$ to denote this attribution method for visual attribution analysis in \S \ref{subsec:visual_attr}.   
% which create grainy regions. 
% However, this method depends on the size of segmentation scales selected for computation. Further, dilation added to the final attribution masks to include edges may depict an inflated version of model's actual feature importance. 
% In this work, $XRAI(\cdot)$ denotes that this attribution method is applied.
%  \vspace{-0.5em}
\section{Related Work}
%  \vspace{-0.3em}
\label{sec:related_work}
\paragraph{Interpretability and explainability } Recent work in multimodal explainability in autonomous vehicles \cite{gilpin-2021-multimodal} uses symbolic explanations to debug and process outputs out of sub-components.
In contrast, we address the challenge of post-hoc multimodal interpretability for any existing end-to-end trained differentiable policies. \textsc{grad $\odot$ input}~\cite{Shrikumar2016NotJA},  a simple and modality-agnostic attribution that works on par with recent methods~\cite{Ancona2017AUV}. We use this method to compute multimodal attribution at inputs to the fusion layer to weigh how each modality contributes to the decision-making. 
% as it has been shown to work at par compared to the recent gradient-based attribution techniques~\cite{Ancona2017AUV}.
While \textsc{grad $\odot$ input} is a modality-agnostic starting point for attributions, 
it is not easy to understand, especially for images. Among recent works to improve visual attribution  \cite{Smilkov2017SmoothGradRN, Simonyan2014DeepIC, ig, sturmfels2020visualizing, Xu_2020_CVPR, Kapishnikov2021GuidedIG, Kapishnikov2019XRAIBA},  we use XRAI~\cite{Kapishnikov2019XRAIBA} for vision-specific analysis as it produces visually intuitive attributions by relying on regions, not individual pixels. 
% \cite{Smilkov2017SmoothGradRN} proposed ways to visually sharpen these vanilla gradient-based attributions. ~\cite{Simonyan2014DeepIC}  applying Gaussian noise perturbations over averaged over a sufficient number of samples.
% describe IG
% IG \cite{ig} and path methods have been studied as a cost-sharing method called Aumann-Shapley. 
% Attribution based on IG preserves axiomatic properties like \textit{sensitivity} and \textit{implementation invariance}.
% While IG aggregate the gradients on sampling inputs on a straight line between the baseline and the input, there are several paths possible in higher dimensional spaces and corresponding different attribution.
% Recent works build on IG to obtain more visually intuitive attributions, like SHAP Deep Explainer~\cite{sturmfels2020visualizing}, Blur IG ~\cite{Xu_2020_CVPR}, Guided IG~\cite{Kapishnikov2021GuidedIG} and XRAI~\cite{Kapishnikov2019XRAIBA}. Qualitatively, XRAI showed visually intuitive attributions by relying on regions and not individual pixels.  
% Interpretability using gradient-based attribution techniques is quite similar to adversarial attacks \cite{Goodfellow2015ExplainingAH} and adversarial training for robustness \cite{Bai2021RecentAI}, as both fundamentally rely on gradient of the input feature with respect to the output. 
% Do we need a figure to show the difference in attributions with just gradient vs gradxinput? 
\vspace{-0.8em}
\paragraph{Language-driven task benchmarks}

There are many benchmarks to study an agent’s ability to follow natural language instructions \cite{ALFRED20, padmakumar2022teach, gu2022vision,  mahmoudieh2022zero}. 
% While most existing settings apply only to either navigation \cite{} or manipulation \cite{}, 
% we conside one of the benchmarks which handles both, that is,
% navigation (Anderson et al., 2018; Chen
% et al., 2019), object manipulation (Misra et al.,
% 2017; Zhu et al., 2017) and embodied reasoning
% (Das et al., 2018a; Gordon et al., 2018). 
ALFRED \cite{ALFRED20} serves as a suitable testbed for this analysis as these tasks require both high reasoning for navigation and manipulation. ALFRED dataset provides visual demonstrations collected through PDDL planning in 3D Unity household environments and natural language description of the high-level goal and low-level instructions annotated by MTurkers. 
The benchmarks provide evaluation metrics for the overall task goal completion success rate (SR) and those weighted by the expert's path length (PLWSR)
% over seen and unseen tasks
and have reported a huge gap in the performance of learning algorithms and humans at these tasks. 
% ALFRED  is a benchmarking environment that provides natural language instructions annotated by MTurkers on egocentric visual sequences of actions taken for everyday household tasks. As ALFRED is a simulated environment on Unity3D game engine, the visual demonstrations are collected based on PDDL planning. 

\vspace{-0.8em}
\paragraph{End-to-end Learned Policies} We investigate the end-to-end learned policies for the task, such that, the gradient can be attributed at a task level. While we do not discuss modular yet differentiable policies like \cite{min2021film} \cite{DBLP:journals/corr/ZhouC15}, tying the gradient across multiple modular learned components is a direction for future work.
% as 
% tying the gradient among modular learned components in future work. 
In our work, we consider the checkpoints of policies trained on the ALFRED dataset. Broadly, these policies are of two types: (a) sequence-to-sequence models, that are, the one proposed with ALFRED dataset (Baseline) \cite{ALFRED20} and Modular Object-Centric Approach (MOCA) \cite{Singh2021FactorizingPA}, (b) transformer-based models, that are Episodic Transformers (ET) \cite{pashevich2021episodic}, and Hierarchical Tasks via Unified Transformers (HiTUT) \cite{Zhang2021HierarchicalTL}. Refer Table~\ref{tab:policiesarch} to compare architectural details \footnote{Previous action is modeled with learned embedding look-up in all these policies.}.
% \textbf{Seq2Seq(Baseline)} \cite{ALFRED20} is a single-stream Seq-to-Seq model with progress monitoring, processing the visual frames through  a frozen ResNet-18 encoder, language through bi-LSTM and soft attention and fusion of the latent visual, language and previous action encodings through an LSTM layer.
%%%% The visual frames are encoded by a frozen ResNet-18 encoder. The language instruction tokens are processed with a bi-LSTM and soft attention. The latent encodings for visual, language and previous action are passed through an LSTM.
% \textbf{MOCA} \cite{Singh2021FactorizingPA} presents a factorized model into two, i.e. interactive perception and action policy. The inputs to the action policy model are language encoding from bi-LSTM, visual embedding from a pretrained ResNet-18, and previous action embedding; all concatenated as input to an LSTM with residual connection.
% \textbf{Episodic Transformers} \cite{pashevich2021episodic} proposes a transformer architecture that encodes the language instructions and the sequence of visual observations and actions to predict subsequent actions per visual frame. Visual observations are processed through pretrained ResNet-50, language tokens passed through a transformer encoder pre-trained with synthetic language targets,  and action are encoded by embedding look-up. 

% Please add the following required packages to your document preamble:
% \usepackage{booktabs}
% Please add the following required packages to your document preamble:
% \usepackage{booktabs}
% Please add the following required packages to your document preamble:
% \usepackage{booktabs}
\begin{table}[t]
\centering
%  \vspace{-1em}
\caption{Policies trained on ALFRED Dataset and their architectures for each modality}
\label{tab:policiesarch}
\begin{tabular}{@{}llll@{}}
\toprule
Policies & Visual                                                                       & Language                       & Fusion                                                                   \\ \midrule
Baseline \cite{ALFRED20} & Frozen ResNet-18                                                             & Learned Embedding, Bi-LSTM     & LSTM                                                                     \\
MOCA \cite{Singh2021FactorizingPA}    & \begin{tabular}[c]{@{}l@{}}Frozen ResNet-18\\ + Dynamic Filters\end{tabular} & Learned Embedding, Bi-LSTM     & \begin{tabular}[c]{@{}l@{}}LSTM with \\ residual connection\end{tabular} \\
ET \cite{pashevich2021episodic}      & Frozen ResNet-50                                                             & Learned Embedding, Transformer & Transformer Encoder                                                      \\
HiTUT \cite{Zhang2021HierarchicalTL}   & Frozen MaskRCNN                                                              & Learned Embedding, FC, LN      & Transformer Encoder                                                  \\ \bottomrule
\end{tabular}
\vspace{-0.2em}
\end{table}
% EmBERT




 
%  provide spurious 
%  explanations and 
%  may not 
%  In cases where the attribution may 
%  this method depends on the underlying attribution methods such as IG. 

% Figure environment removed


\section{Ripple Effects of Factual Edits}
\label{sec:rethinking}

We focus on evaluating the downstream effect of a given edit, i.e., given an edit $(e,r,o) \rightarrow (e,r,o')$, we expect certain facts related to the edit to change as well. Consider, for example, the edit shown in Fig.~\ref{figure:intro}. Changing the team for which Messi plays might also affect the league he plays in and his country of residence.
Formally, for a given model, assume a knowledge-graph ${\mathcal{K} := \{(e_i, r_i, o_i)\}_{i=1}^{N}}$ of $N$ factual triplets, representing the model's knowledge, and let $\delta: (e,r,o) \rightarrow (e,r,o')$ be an edit request for $\mathcal{K}$.
We define the \textit{ripple effect} of $\delta$ on $\mathcal{K}$, as the set of triplets $\mathcal{R}(\delta)$ that the model implicitly needs to inject, modify, or delete from $\mathcal{K}$ to reflect the world state after the edit. 

Notably, different edits can cause ripple effects of varying magnitudes. For example, changing the country of Rome from Italy to France, will entail many follow-up changes, such as the country in which the Colosseum is located, the language spoken in Rome, and so forth. 
In contrast, updating the siblings of Prince (Fig.~\ref{figure:examples}) is both more realistic and should result in a more local effect. 
We refer to the number of facts affected by a single edit $\delta$ (i.e. $|\mathcal{R}(\delta)|$) as its \textit{severity}.
In general, editing popular entities that appeared frequently during training is likely to introduce more changes, and thus, editing their properties has a higher severity.

\subsection{Evaluation Criteria}
\label{subsec:criteria}

We wish to evaluate how well models capture the ripple effects of factual edits. However, given that ripple effects can potentially span a large number of implied edits, we focus on evaluating modified facts that are within a 2-hop distance from the subject or object of the edit.
Concretely, for an edit $\delta: (e, r, o) \rightarrow (e, r, o^*)$, we evaluate the ripple effect $\mathcal{R}(\delta)$, via the following evaluation criteria (examples are shown in Fig.~\ref{figure:examples}): 

\begin{enumerate}
[leftmargin=*,topsep=2pt,parsep=1pt]
    \item \textbf{\logicalgeneralization{} (LG)}:
    Relations in a knowledge graph satisfy certain logical constraints. For example, the relation \texttt{Sibling} is symmetric and therefore if $(e, \texttt{Sibling} ,o)$ is true then $(o,\texttt{Sibling},e)$ is also true, and vice versa (Fig.~\ref{figure:examples}A). Likewise, the relation \texttt{Location} is transitive so $(e,\texttt{Location},o) \wedge (o,\texttt{Location},z) \Rightarrow (e,\texttt{Location},z)$. We wish to check that such logical implications about the subject $e$, the original object $o$, and the new object $o^*$, hold after editing. We focus and elaborate on specific constraints in \S\ref{sec:rippledit}.
    
    \item \textbf{\compositionality{} (CI)}: As $\delta$ alters one edge in a knowledge graph, we can check the composition of this edge with other edges. Namely, we test if the model can compose the edited fact with other facts about the target object.
    Let $(o, r', z)$ and $(o^*, r', z^*)$ be two facts of the same relation about $o$ and $o^*$,
    respectively. Also, denote by $r''=r\circ r'$ the complex relation expressing the composition of $r$ and $r'$ (e.g., $r''=\texttt{Profession of sibling}$ for $r=\texttt{Sibling}$ and $r'=\texttt{Profession}$).  
    Then, after the edit $\delta$, we expect the following change  $(e, r'', z) \rightarrow (e, r'', z^*)$.
    For example (Fig.~\ref{figure:examples}B), the professions of the siblings of \texttt{Prince} can be modified once a new sibling is injected.
    
    \item \textbf{\forwardcompositionality{} (CII)}:
    We test if the model can compose the edited fact with facts about a different subject $e' \neq e$.
    Formally, let $(e', r', e)$ be a fact about $e'$ with $e$ as its object, and denote by $r''=r'\circ r$ the complex relation expressing the composition of $r'$ and $r$ (see an example in criterion 2). After the edit $\delta$, the following change is expected for the subject $e'$:
    $(e', r'', o) \rightarrow (e', r'', o^*)$. 
    For instance (Fig.~\ref{figure:examples}C), changing the siblings of \texttt{Prince} also modifies the siblings of the founder of \texttt{Paisley Park Records} (i.e., $r''$ is a complex relation expressing ``siblings of the founder'').
    
    \item \textbf{\aliasing{} (SA)}: We test that editing a fact about $e$ induces the same edit to other entities $e'$ that are aliases of $e$, namely, $(e', r, o) \rightarrow (e', r, o^*)$. 
    For instance (Fig.~\ref{figure:examples}D), modifying the siblings of \texttt{Prince}, should also modify the sibling of his alias, \texttt{Prince Roger Nelson}.
    
    \item \textbf{\forgetfulness{} (PV)}: If $r$ is a one-to-many relation, then adding a new object should not affect the other objects encoded about $e$. 
    Hence, in such cases, we expect that any existing triplet $(e, r, o')$ for an object $o' \neq o^*$ would remain following the edit. 
    For example (Fig.~\ref{figure:examples}E), after inserting the sibling \texttt{Nicholas Carminowe} for \texttt{Prince}, the fact that \texttt{Tyka Nelson} is also his sibling should be retained.
    
    \item \textbf{\relationspecificity{} (RS)}: We test that facts about $e$, with relations whose objects are not influenced by $o$, are indeed not affected by the edit.
    For example (Fig.~\ref{figure:examples}F), modifying the sibling of \texttt{Prince} should not change his \texttt{Mother}. Note that these facts complement those evaluated by \logicalgeneralization{}.
\end{enumerate}

\noindent In \S\ref{sec:datacollection}, we describe how we generate factual editing evaluations, based on the above criteria.

\subsection{Related Work}
\label{subsec:currentstate}

\paragraph{Knowledge Editing Methods}
Several methods have been proposed to edit the factual knowledge encoded in a model. 
\citet{de-cao-etal-2021-editing} and \citet{mitchell2022fast} suggested to use hyper-networks to update the model weights. 
In addition, \citet{meng2022locating, meng2022mass} proposed to modify encoded facts by updating the weights of MLP layers, following recent observations that these layers can be cast as key-value memories \cite{geva-etal-2021-transformer} that store factual knowledge \cite{dai-etal-2022-knowledge}.
Other methods learn encodings that update the hidden representations created during model inference \cite{hernandez2023inspecting}, or augment the input context with edits \cite{zhong2023mquake, zheng2023edit}.
In \S\ref{subsec:experimental_settings}, we discuss state-of-the-art KE methods used in this work in greater detail.

Separately from factual KE, recent works have also studied how to inject new facts into a model. Previous methods suggested unsupervised pre-training \cite{roberts2020much, ijcai2021p552}, semi-parametric methods methods where external information is added from a knowledge-base \cite{zhang-etal-2019-ernie, peters-etal-2019-knowledge, lewis2021retrievalaugmented, zhang2022greaselm}, using adapters to store knowledge \cite{wang2020kadapter}, or extending the MLP layers \cite{yao2022kformer}.

\paragraph{Knowledge Editing Evaluation}
Recently, there has been a growing interest in KE evaluation \cite{yao2023editing}.
The prominent benchmarks for evaluating factual KE are the Zero-Shot Relation Extraction
(zsRE) \cite{levy-etal-2017-zero, de-cao-etal-2021-editing} and CounterFact \cite{meng2022locating}. zsRE is a question-answering dataset for relation-specific queries, which includes human generated paraphrases that are used to measure robustness to semantically equivalent inputs. For example, for the triplet  (\texttt{x}, \texttt{Country}, \texttt{y}), zsRE contains queries such as ``\emph{In which country is x?}''. CounterFact offers a more challenging setting, where edits are counterfactuals of a low probability, such as changing the \texttt{City} of \texttt{The Louvre} from \texttt{Paris} to \texttt{Rome}.

Evaluation in zsRE and CounterFact focuses on three primary aspects of (a) \textit{efficacy}: checking that the model generates the target object post-editing, (b) \textit{paraphrasing}: testing robustness in generating the target for paraphrases of the input, and (c) \textit{specificity}: verifying that facts not related to the edit are unaffected.
In addition, CounterFact evaluates the generation quality of the edited model when prompted with the edit's subject, measuring: \textit{consistency}, i.e., similarity with subjects that share the same property as the edited object, and \textit{fluency} in terms of repetitiveness of the generated text.
More broadly, previous work evaluated to which extent LMs have beliefs \cite{sep-formal-belief, kassner-etal-2021-beliefbank, hase-etal-2023-methods}, and \citet{hase-etal-2023-methods} examined if updating beliefs propagates to entailed facts, extending the Wikidata5m dataset \cite{wang-etal-2021-kepler} to test editing specificity.

Recently, \citet{onoe2023lms} introduce the task of \emph{entity knowledge propagation}, aiming to examine the extent to which models are able to reason about emergent entities that did not appear in pre-training. In addition, \citet{hoelscherobermaier2023detecting} show that existing KE methods can have unwanted side effects and suffer from low specificity.
A concurrent work by \citet{zhong2023mquake} introduces MQUAKE, a benchmark that tests the ability of models to perform multi-hop reasoning after edits. 
While each of these benchmarks focuses on a single consequence of editing, \ripple{} provides a general framework for evaluating various types of edit ripple effects.
Last, \citet{gupta2023editing} focus on editing commonsense knowledge and introduce MEMIT-CSKPROBE, a dataset for semantic generalization of commonsense edits. \ripple{} is different from MEMIT-CSKPROBE as it evaluates editing of factual knowledge rather than commonsense knowledge.
\section{The \ripple{} Benchmark}
\label{sec:rippledit}

In this section, we describe a data generation pipeline (\S\ref{sec:datacollection}) for factual edit requests and queries for evaluating their ripple effects.
Then, we apply our pipeline to create the \ripple{} benchmark for comprehensive KE evaluation (\S\ref{sec:datastats}), and validate the quality of the data (\S\ref{sec:data_quality}).


% Figure environment removed

\subsection{Data Generation Pipeline}
\label{sec:datacollection}

We describe our data generation process (illustrated in Fig.~\ref{figure:pipeline}), that creates KE evaluation examples, each consisting of a factual edit request and a set of test queries that follow our criteria. Since the pipeline involves manual writing of templates and logical rules per relation, we restrict the edits and test queries to a fixed set of $N_{rel}$ basic relations.\footnote{The full list of relations is available in our codebase, example relations are shown in Fig.~\ref{figure:top_relation_stats}.}

\paragraph{Step 1: Factual triplets collection}
The first step of the pipeline (Fig.~\ref{figure:pipeline}A) is to collect facts, from which we will later create edit requests. 
To this end, we use \wikidata{}, a relational knowledge base consisting of facts that are expressed as triplets $(e, r, o)$, where $e$ is a subject entity, $r$ is a relation,
and $o$ is an object. We collect triplets of three types:

\begin{itemize}
[leftmargin=*,topsep=2pt,parsep=1pt]
\item \textbf{\recentlyemerged{}}: To create ``real'' plausible edit requests, we collect triplets that were inserted to \wikidata{} only recently, and represent relatively new facts. Therefore, they can be used to create injection edit requests for models that were trained before these facts were introduced, to simulate cases of an out-of-date model that requires factual updates.
We collect such facts by randomly sampling triplets that have been modified during a range of 250 days after July 2022.

\item \textbf{\fakefacts{}}:
We collect triplets corresponding to random facts, for which we will later generate modification edits (similarly to \citet{meng2022locating}). These edits simulate factual edits that are meant to fix incorrect model predictions (e.g., predicting that the capital of Germany is Frankfurt).
To this end, we divide the entities in \wikidata{} into 10 uniform buckets, based on the number of triplets associated with them. Intuitively, this can be viewed as a popularity measure. Then, we sample $N_{ent}$ entities from each group and randomly choose one triplet for each entity.

\item \textbf{\topviews{}}:
The two previous triplet types are randomly sampled from the entire knowledge base, and most of them are likely to represent facts about tail entities (except perhaps for a small subset in the top bucket).
Such entities are often not captured by models \cite{mallen-etal-2023-trust}, and therefore not suitable for testing modification edits.
To address this, we sample triplets from \wikidata{} with a subject that is a \emph{popular entity}, namely it appears in one of the top-viewed pages in Wikipedia.\footnote{
We extracted the entities whose corresponding Wikipedia page was included in the top-1000 most viewed pages in at least one month during 2020-2022.}
Importantly, these types of triplets allow controlling for the ripple effect severity (\S\ref{sec:rethinking}), i.e., how models handle the ripple effects of popular entities versus tail entities. 

\end{itemize} 

\paragraph{Step 2: Edits generation}
Once we obtain factual triplets, we turn to generate edit requests for them (Fig.~\ref{figure:pipeline}B).
For \recentlyemerged{}, triplets represent new facts that are meant to be injected to the model, assuming that the latter was trained before these facts were introduced to the world. 
Hence, for \recentlyemerged{}, the target triplet for injection is the triplet itself. 

For \fakefacts{} and \topviews{} triplets, we create an edit by generating a target triplet as follows. First, for every relation $r$, we create a set of candidate object entities $O_r$ by sampling $N_{cand}$ triplets $(e_1, r ,o_1),..., (e_{N_{cand}}, r ,o_{N_{cand}})$ with the relation $r$,
and extracting their objects $O_r = \{o_1,..., o_{N_{cand}}\}$.
Then, for every triplet $(e,r,o)$ in \fakefacts{} and \topviews{}, we sample a target object $o' \neq o$ from $O_r$. 
Sampling the target object from triplets with the same relation makes the edit request technically consistent with the original triplet -- the target object is of the same ``type'' as the original object (for example, a triplet with the relation \texttt{Capital} will get a new object of type \texttt{City}). The new triplet $(e,r,o')$ will thus result in a ``fake'' fact, since it attaches a wrong object $o'$ to the pair $(e,r)$. For example if \fakefacts{} contains the triplet (\texttt{France}, \texttt{Capital}, \texttt{Paris}), its edit could be (\texttt{France}, \texttt{Capital}, \texttt{London}).

\paragraph{Step 3: Evaluation tests generation}
The next step in the pipeline is to create ripple effect evaluations for the factual edits we collected (Fig.~\ref{figure:pipeline}C).
To this end, we implement the evaluation criteria introduced in \S\ref{subsec:criteria}, and generate test queries for each criterion. 
Each test query corresponds to a triplet of subject and object entities and a possibly complex relation, that is expected to be true post-editing.
In what follows, we provide details on our implementation, using objects from \wikidata{}.

For an entity $e$, we denote by $\mathcal{S}(e)$ the set of triplets in \wikidata{} in which $e$ is the subject, and by $\mathcal{T}(e)$ the set of triplets in which $e$ is the object.
Moreover, for every relation $r$, we manually define a set $D_r$ of relations that semantically depend on it.
Namely, for a given subject, changing $r$'s target object is expected to change the target objects for the relations $D_r$. For instance, the set $D_r$ for the relation $r =$ \texttt{Mother}, includes the relations \texttt{Sibling}, \texttt{Sister}, \texttt{Brother}, \texttt{Aunt}, and \texttt{Uncle}, among others.
Then, for every relation $r' \in D_r$, we craft a logical rule for obtaining the new target for that relation post-editing. For instance, for the relation $r=$ \texttt{Sibling}, we set a logical rule for $r'=$ \texttt{Mother} such that if $(e,r,e')$ and $(e',r',z')$ are true for entities $e, e', z'$, then $(e,r',z')$ should also be true. 

Given an edit $(e, r ,o) \rightarrow (e, r, o^*)$, we use $D_r$ to generate test queries for \logicalgeneralization{} and \relationspecificity{}. 
For \logicalgeneralization{}, we apply the rule corresponding to each relation $r' \in D_r$ to obtain a set of test queries $(x, r', z')$ about $x\in\{e,o,o^*\}$, where $z'$ is the target obtained from the logical rule.
For \relationspecificity{}, we create a test query for every triplet in $\mathcal{S}(e)$ with a relation that is \textit{not} in $D_r$ (but is in our set of $N_{rel}$ relations).

To generate text queries for \compositionality{}, we iterate through $\mathcal{S}(o^*)$ and for each triplet $(o^*,r', z) \in \mathcal{S}(o^*)$, we construct a two-hop query $(e,r\circ r',z)$ about $e$, with $z$ as the answer. Similarly, 
for \forwardcompositionality{}, we iterate through $\mathcal{T}(e)$ and for each triplet $(z ,r', e) \in \mathcal{T}(e)$, we construct a two-hop query $(z,r'\circ r,o^*)$ about $z$ with $o^*$ as the answer.
For \aliasing{}, we use information maintained by \wikidata{} to create a test query $(e', r, o^*)$ for every alias $e'$ of $e$. 
Last, for \forgetfulness{} we create test triplets $(e, r, o_1), ..., (e, r, o_n)$ that check if the model retained the original objects $\{o_1, ..., o_n\}$ in addition to the new edited object $o^*$.




\begin{table}[t]
\setlength\tabcolsep{3.3pt}
\setlength{\belowcaptionskip}{-8pt}
\footnotesize
\centering
        \begin{tabular}{llrr} \\  
        % \toprule
        & \recentlyemerged{} & \fakefacts{} & \topviews{} \\ \midrule
        \# of factual edits & 2,000 & 1,000 & 1,000 \\ 
        \# of queries per edit & $26.8$ & $18.8$ & $25.6$ \\
        % \# of test queries per edit & $5.24$ & $3.1$ & $4.2$ \\ \midrule
        \# of queries per criterion & $5.24$ & $3.1$ & $4.2$ \\ \midrule
        \# of LG queries &$2.5$ &$3.6$ &$2.6$ \\
        \# of CI queries &$11.7$  &$4.7$ &$6.1$ \\ 
        \# of CII queries &$5.1$ &$5.1$ &$3.9$ \\
        \# of SA queries &$1.8$ &$1.3$ &$4.7$ \\
        \# of PV queries &$0.6$ &$0.4$ &$0.5$ \\ 
        \# of RS queries &$5.1$ &$3.7$ &$7.8$ \\ \midrule
        Subject triplets count & $31.7$ & $13.3$ & $115.2$ \\
        Subject page back-links & $278.1$ & $121.6$ & $3934.5$ \\
        Subject page views & $189.6$ & $67.91$ & $7376.5$ \\ \midrule
        Object triplets count & $192.4$ & $46.4$ & $39.5$ \\
        Object page back-links & $18634.2$ & $3065.0$ & $2136.0$ \\
        Object page views & $2852.4$ & $1379.7$ & $1176.7$ \\ 
        \bottomrule
        \end{tabular} 
\caption{Statistics per subset of \ripple{}, showing the average of different metrics. Breakdown by evaluation criteria shows the number of queries of each criterion per edit. For a given subject/object entity, triplets count is the number of \wikidata{} facts it is associated with, page back-links is the number of Wikipedia pages with a link to the entity's page, and page views is the recent average daily view count of the entity's page.}
% over the week of June 18th, 2023.
\label{tab:datset_stats} 
\end{table}

% Figure environment removed

\paragraph{Step 4: Phrasing in natural language}
\label{paragraph:phrasing_in_nl}
At this point (Fig.~\ref{figure:pipeline}D), we have factual edit requests and their corresponding test queries. To use them as inputs to LMs, we convert them from triplet-form to natural language (NL). 
To this end, we manually craft a template NL phrase per relation (this is feasible since we use a fixed set of relations), and use it to convert all the triplets with this relation.
For instance, the template \texttt{``The date of birth of <$e$> is''} converts triplets with the relation $r=$ \texttt{Date of Birth} and a subject entity $e$.

For the \forgetfulness{} triplets generated for an edit $(e, r, \{o_1, ..., o_n\}) \rightarrow (e, r, \{o_1, ..., o_n, o^*\})$, where $o^*$ is a new object added to a set of possibly multiple ($n\geq0$) objects, we form a single NL query about other objects than the edited one, e.g., 
\texttt{``The award received by <$e$> which is not <$o^*$> is''}.


\subsection{Data Statistics}
\label{sec:datastats}

We used our data generation pipeline to collect edits for 2,000 \recentlyemerged{} facts, 1,000 \fakefacts{} facts, and 1,000 \topviews{} facts, focusing on $N_{rel}=54$ basic relations for which we manually crafted NL templates and logical rules.\footnote{We release the templates and rules in our codebase.} To obtain the \fakefacts{} subset, we set $N_{ent}=200$ to sample 200 facts from each entity group in \wikidata{}. For edits generation of \fakefacts{} and \topviews{}, we set $N_{cand}=100,000$.
We call our diagnostic benchmark \ripple{}, and publicly release it to the research community.
Notably, \ripple{} focuses on ripple edits and is meant to complement existing benchmarks, and so it does not include previous evaluations, such as subject specificity and model consistency. 


Statistics on \ripple{} are presented in Table~\ref{tab:datset_stats}, showing that our generation process resulted in 18-26 test queries per edit and over $3$ queries per evaluation test, on average. Moreover, \topviews{} edits contain more popular subjects (as intended), while \recentlyemerged{} edits have more popular objects.
Fig.~\ref{figure:top_relation_stats} shows the top relations and their frequency in each subset of \ripple{}, demonstrating the diversity of the generated facts.


\subsection{Data Quality}
\label{sec:data_quality}

We conducted a manual analysis to validate that our generation pipeline produces valid test queries. Concretely, we sampled 200 random test queries from \ripple{} and checked the following two requirements: (a) \emph{soundness}: the triplet that represents a given test query should be semantically correct, namely, the entity type of the object should match the relation type and the relation type should match the entity type of the subject. For example, queries such as \emph{``The capital of Hilary Clinton is''} or \emph{``The sibling of Lebron James is Los Angeles''} would have been disqualified. (b) \emph{grammatically correct}: we check that the phrasing of the test query in natural language is grammatical. 

We found that 100\% of the queries were sound (i.e., semantically clear and correct), showing that the data curating process was designed properly.
Furthermore, 98.5\% of the queries were grammatically correct, while the ones which were not contain entity representations in a non-English language. This shows that our templates are general enough to properly fit various entity names.


\section{Experiments}
\label{sec:experiments}

We use \ripple{} to evaluate recent KE methods on multiple LMs, and show that despite  substantial progress on existing benchmarks, current KE methods struggle to introduce consistent changes to the model's knowledge after an edit. Moreover, a simple in-context editing baseline where generation is conditioned on the edited facts is more consistent, leaves ample room for improvement for future methods.

% Figure environment removed


\subsection{Evaluation Setting}
\label{subsec:experimental_settings}

\paragraph{Data}
\ripple{} is meant to be used as a diagnostic dataset to evaluate the ripple effects resulting from an editing operation. Therefore, to evaluate the performance of an editing method on a given model, the data first needs to be adjusted such that (a) only cases of successful edits are evaluated, and (b) only test queries that the model answered correctly pre-editing are used for evaluation. 

Concretely, for a given editing method $\mathcal{F}$, a model $\mathcal{M}$, an edit request $x: (e,r,o) \rightarrow (e,r,o')$, is included in the evaluation if the following conditions are met when applying $\mathcal{F}$ to $\mathcal{M}$ and $x$: (a) $\mathcal{M}$ successfully generates $o'$ when queried about $e$ and $r$, namely, the edits has successfully been applied, and (b) $\mathcal{M}$ successfully generates the correct objects for queries corresponding to the tests before applying the edit. 
For example, we check that the model can predict the children of $o'$ before asking about $e$'s new \texttt{siblings}, and that it predicts the mother of $o'$ before asking about the new \texttt{maternal grandmother} of $e$.

\paragraph{Editing methods} 
We evaluate three KE methods:
MEND \cite{mitchell2022fast}, ROME \cite{meng2022locating}, and MEMIT \cite{meng2022mass}. MEND trains a network that modifies gradients to produce local edits when presented with a desirable input-output pair. ROME rank-one updates to the weights of the Transformer's MLP layers to update specific factual associations. MEMIT is an extension of ROME that is also capable of editing many facts at once.

\begin{table}[t]
\setlength\tabcolsep{4pt}
    \centering
    \footnotesize
    \begin{tabular}{lllccrr}
    & \multicolumn{2}{c}{\recentlyemerged{}} & \multicolumn{2}{c}{\fakefacts{}} & \multicolumn{2}{c}{\topviews{}} \\ 
    & Edits & Tests & Edits & Tests & Edits & Tests \\ \midrule
        \gptxl{}
         &$853$ &$29\%$ &$689$ &$33\%$ &$722$ &$71\%$  \\ 
         \gptj{}
         &$801$ &$33\%$ &$717$ &$34\%$ &$760$ &$76\%$  \\ 
         \gptneo{} 
         &$989$ &$45\%$ &$801$ &$46\%$ &$828$ &$86\%$ \\ 
         \llama{} 
         &$847$ &$44\%$ &$796$ &$49\%$ &$784$ &$87\%$   \\ 
         \gptt{} 
         &$822$ &$55\%$ &$760$ &$74\%$ &$665$ &$94\%$  \\
         \bottomrule 
    \end{tabular}
    \caption{(a) Number of edits we considered in our evaluation (that is, edits that have successfully occurred), from each of the data types, averaged over \rome{}, \memit{} and \mend{}, for the models: \gptxl{}, \gptj{}, \gptneo{} and \llama{}, and using the ICE baseline for \gptt{}. (b) Portion of queries, on average, that have been considered during our evaluation, namely that their conditions have been met.}
\label{table:filtered_tests_portion}
\end{table}

\paragraph{Baseline} Motivated by the recent success of LMs to learn in-context and follow instructions \cite{NEURIPS2020_1457c0d6, ouyang2022training, liu2023pre}, we propose an in-context editing (ICE) baseline for factual editing. Unlike the above methods, it does not introduce changes to the model parameters, but rather generation is conditioned on the new fact. 
Concretely, given an edit $(e, r, o) \rightarrow (e, r, o^*)$ and a test query $q$, we use the following prompt to obtain an answer from the model: \texttt{``Imagine that <$o^*$> would have been <$P_r$>''}, where $P_r$ is a particular manual-phrased proposition corresponding to $r$, such as \emph{``The mother of <$e$>''} when $r=$ \texttt{Mother} and $e$ is the subject.
% (similar process as in \S\ref{paragraph:phrasing_in_nl}).
An example is illustrated in Fig.~\ref{figure:ice_demonstration}.



\begin{table}[t]
\setlength\tabcolsep{3pt}
    \centering
    \footnotesize
    \resizebox{0.999\linewidth}{!}{
    \begin{tabular}{llcccccc|r}
    & & LG & CI & CII & SA & PV & RS & Avg. \\ \midrule
        \multirow{3}{*}{\gptxl{}} 
         & ROME &$20.2$ &$35.6$ &$46.8$ &$86.8$ &$100$ &$55.4$ &$57.5$  \\
         & MEMIT &$21.8$ &$30.3$ &$46.2$ &$92.9$ &$100$ &$56.8$ &$58.0$  \\
         & MEND &$28.9$ &$23.7$ &$20.7$ &$87.1$ &$100$ &$51.9$ &$52.1$  \\ \midrule
         \multirow{2}{*}{\gptj{}} 
         & ROME &$15.2$ &$29.5$ &$50.5$ &$90.3$ &$99.4$ &$60.0$ &$57.5$  \\
         & MEMIT &$18.0$ &$35.0$ &$48.1$ &$88.4$ &$98.6$ &$42.2$ &$55.0$  \\ \midrule
         \multirow{2}{*}{\gptneo{}} 
         & ROME &$27.2$ &$54.3$ &$69.4$ &$98.9$ &$98.4$ &$80.3$ &$71.4$    \\
         & ICE &$48.3$ &$29.0$ &$62.2$ &$100$ &$99.4$ &$80.7$ &$69.9$  \\ \midrule
         \multirow{2}{*}{\llama{}} 
         & ROME &$16.7$ &$47.8$ &$50.0$ &$93.6$ &$97.6$ &$59.3$ &$60.8$  \\
         & ICE &$59.6$ &$74.8$ &$85.0$ &$100$ &$99.5$ &$77.9$ &$82.8$   \\ \midrule
         \gptt{} 
         & ICE &$33.3$ &$100$ &$91.3$ &$100$ &$100$ &$73.1$ &$82.8$  \\
         \bottomrule 
    \end{tabular}
    }
    \caption{Accuracy on the \recentlyemerged{} subset, by \mend{}, \rome{}, \memit{}, and the ICE baseline, on \gptxl{}, \gptj{}, \gptneo{}, \llama{}, and \gptt{}. 
    % Evaluation on Forgetfulness is not applicable for \recentlyemerged{} edits (see \S\ref{sec:datacollection}).
    }
\label{table:recently_emerged_results}
\end{table}



% \begin{table*}[t]
% \setlength{\belowcaptionskip}{-10pt}
% \setlength\tabcolsep{2.5pt}
% \footnotesize
% \begin{center}
% \begin{tabular}{l  cc | cc | ccc | cc | c | c }
%  & \multicolumn{2}{c}{\gptm{}} & \multicolumn{2}{c}{\gptl{}} & \multicolumn{3}{c}{\gptxl{}} &  \multicolumn{2}{c}{\gptj{}} & \multicolumn{1}{c}{\llama{}} & \multicolumn{1}{c}{\gptneo{}} \\ [0.1cm]
% \multicolumn{1}{c}{} & \textbf{\mend}  & \textbf{\rome} & \textbf{\mend}  & \textbf{\rome}  & \textbf{\mend}  & \textbf{\rome} & \textbf{\memit}  & \textbf{\rome} & \textbf{\memit}  & \textbf{\rome} & \textbf{\rome}  \\
% \toprule
% \relationspecificity{}          &$49.5$    &$52.6$   
%                                 &$49.0$   &$52.7$    
%                                 &$51.9$   &$55.4$   &$56.8$ 
%                                         &$60.0$   &$42.2$  
%                                         &$59.3$        
%                                          &$80.3$         \\
% \logicalgeneralization{}        &$20.1$    &$15.5$   
%                                 &$23.3$   &$15.4$    
%                                 &$28.9$   &$20.2$   &$21.8$ 
%                                      &$15.2$   &$18.0$  
%                                      &$16.7$        
%                                 &$27.2$         \\
% \compositionality{}            &$14.4$    &$26.7$   
%                                 &$19.0$   &$35.9$    
%                                  &$23.7$   &$35.6$   &$30.3$ 
%                                 &$29.5$   &$35.0$  
%                                 &$47.8$        
%                                 &$54.3$         \\
% \forwardcompositionality{}      &$24.2$    &$40.6$   
%                                 &$17.7$   &$42.0$    
%                                  &$20.7$   &$46.8$   &$46.2$ 
%                                 &$50.5$   &$48.1$  
%                                 &$50.0$        
%                                 &$69.4$         \\
% \aliasing{}                     &$85.6$    &$88.0$   
%                                 &$94.9$   &$92.3$    
%                                  &$87.1$   &$86.8$   &$92.9$ 
%                                 &$90.3$   &$88.4$  
%                                 &$93.6$        
%                                 &$98.9$         \\
% \forgetfulness{}                     & -    & - & -  & -  & -   & - & - & - & - & -& -         \\
% \midrule
% AVG                             &$38.8$    &$44.7$   
%                                 &$40.8$   &$47.7$    
%                                  &$42.5$   &$49.0$   &$49.6$ 
%                                 &$49.1$   &$46.3$  
%                                 &$53.5$        
%                                 &$66.0$         \\ 
% \bottomrule
% \end{tabular}
% \end{center}
% \caption{Accuracy scores of \mend{}, \rome{} and \memit{} with \gptl{}, \gptxl{}, \gptj{} and \gptneo{}, testing on \recentlyemerged{}{}}
% \label{table:recently_emerged_results}
% \end{table*}

\begin{table}[t]
\setlength{\belowcaptionskip}{-10pt}
\setlength\tabcolsep{3pt}
    \centering
    \footnotesize
    \resizebox{0.999\linewidth}{!}{
    \begin{tabular}{llllccrr|r}
    & & LG & CI & CII & SA & PV & RS & Avg. \\ \midrule
        \multirow{3}{*}{\gptxl{}} 
         & ROME &$53.6$ &$31.6$ &$44.4$ &$94.9$ &$9.9$ &$38.9$ &$45.5$  \\
         & MEMIT &$58.4$ &$30.5$ &$49.8$ &$100$ &$20.0$ &$36.2$ &$49.1$   \\
         & MEND &$62.5$ &$16.7$ &$14.6$ &$91.3$ &$17.7$ &$30.1$ &$38.8$   \\ \midrule
         \multirow{2}{*}{\gptj{}} 
         & ROME &$53.8$ &$40.8$ &$49.9$ &$93.8$ &$15.2$ &$39.4$ &$48.8$   \\
         & MEMIT &$53.0$ &$35.7$ &$48.2$ &$95.6$ &$18.2$ &$39.9$ &$48.4$   \\ \midrule
         \multirow{2}{*}{\gptneo{}} 
         & ROME &$61.6$ &$49.4$ &$57.1$ &$100$ &$30.8$ &$50.7$ &$58.3$  \\
         & ICE &$78.6$ &$90.0$ &$55.6$ &$100$ &$100$ &$61.9$ &$81.0$    \\ \midrule
         \multirow{2}{*}{\llama{}}
         & ROME &$54.3$ &$35.5$ &$49.5$ &$96.0$ &$17.8$ &$38.9$ &$48.7$ \\
         & ICE &$71.1$ &$73.8$ &$80.3$ &$100$ &$100$ &$69.6$ &$82.5$\\  \midrule
         \gptt{} 
         & ICE &$69.0$ &$83.3$ &$89.7$ &$100$ &$100$ &$100$ &$90.3$  \\
         \bottomrule
    \end{tabular}
    }
    \caption{Accuracy on the \fakefacts{} subset, by \mend{}, \rome{}, \memit{}, and the ICE baseline, on \gptxl{}, \gptj{}, \gptneo{}, \llama{}, and \gptt{}.}
\label{table:fake_facts_results}
\end{table}


% \begin{table*}[t]
% \setlength{\belowcaptionskip}{-10pt}
% \setlength\tabcolsep{2.5pt}
% \footnotesize
% \begin{center}
% \begin{tabular}{l  cc | cc | ccc | cc | c | c }
%  & \multicolumn{2}{c}{\gptm{}} & \multicolumn{2}{c}{\gptl{}} & \multicolumn{3}{c}{\gptxl{}} &  \multicolumn{2}{c}{\gptj{}} & \multicolumn{1}{c}{\llama{}} & \multicolumn{1}{c}{\gptneo{}} \\ [0.1cm]
% \multicolumn{1}{c}{} & \textbf{\mend}  & \textbf{\rome} & \textbf{\mend}  & \textbf{\rome}  & \textbf{\mend}  & \textbf{\rome} & \textbf{\memit}  & \textbf{\rome} & \textbf{\memit}  & \textbf{\rome} & \textbf{\rome}  \\
% \toprule
% \relationspecificity{}          & $33.1$    &$33.3$   
%                     & $33.9$    &$32.1$    
%                      & $30.1$    &$38.9$   &$36.2$ 
%                                 &$39.4$   &$39.9$  
%                                 &$38.9$         
%                                 &$50.7$         \\
% \logicalgeneralization{}    & $59.7$    &$63.1$   
%                     & $62.0$    &$61.4$    
%                      & $62.5$    &$53.6$   &$58.4$ 
%                                 &$53.8$   &$53.0$  
%                                 &$54.3$         
%                                 &$61.6$         \\
% \compositionality{}            & $9.8$    &$25.5$   
%                     & $16.4$    &$38.3$    
%                      & $16.7$    &$31.6$   &$30.5$ 
%                                 &$40.8$   &$35.7$  
%                                 &$35.5$         
%                                 &$49.4$         \\
% \forwardcompositionality{}      & $14.1$    &$35.8$   
%                     & $11.9$    &$38.1$    
%                      & $14.6$    &$44.4$   &$49.8$ 
%                                 &$49.9$   &$48.2$  
%                                 &$49.5$         
%                                 &$57.1$         \\
% \aliasing{}    & $88.9$    &$91.1$   
%                     & $89.9$    &$91.3$    
%                      & $91.3$    &$94.9$   &$100$ 
%                                 &$93.8$   &$95.6$  
%                                 &$96.0$         
%                                 &$100$         \\
% \forgetfulness{}    & $15.4$    &$10.1$   
%                     & $16.2$    &$13.7$    
%                      & $17.7$    &$9.9$   &$20.0$ 
%                                 &$15.2$   &$18.2$  
%                                 &$17.8$         
%                                 &$30.8$         \\
% \midrule
% AVG                    & $36.3$    &$43.1$   
%                     & $38.4$    &$45.8$    
%                      & $38.8$    &$45.5$   &$49.1$ 
%                                 &$48.8$   &$48.4$  
%                                 &$48.7$         
%                                 &$58.3$         \\
% \bottomrule
% \end{tabular}
% \end{center}
% \caption{Accuracy scores of \mend{}, \rome{} and \memit{} with \gptl{}, \gptxl{}, \gptj{} and \gptneo{}, testing on \fakefacts{}}
% \label{table:fake_facts_results}
% \end{table*}

\begin{table}[t]
\setlength\tabcolsep{3pt}
% \setlength{\belowcaptionskip}{-8pt}
    \centering
    \footnotesize
    \resizebox{0.999\linewidth}{!}{
    \begin{tabular}{llllccrr|r}
    & & LG & CI & CII & SA & PV & RS & Avg. \\ \midrule
        \multirow{3}{*}{\gptxl{}} 
         & ROME &$5.7$ &$46.4$ &$21.8$ &$100$ &$100$ &$18.5$ &$48.7$  \\
         & MEMIT &$6.7$ &$45.2$ &$21.2$ &$100$ &$100$ &$24.3$ &$49.6$   \\
         & MEND &$25.9$ &$10.7$ &$5.4$ &$100$ &$100$ &$21.2$ &$43.9$   \\ \midrule
         \multirow{2}{*}{\gptj{}} 
         & ROME &$5.5$ &$44.1$ &$21.0$ &$98.6$ &$99.0$ &$22.3$ &$48.4$   \\
         & MEMIT &$7.0$ &$45.9$ &$23.7$ &$100$ &$100$ &$24.8$ &$50.2$   \\ \midrule
         \multirow{2}{*}{\gptneo{}} 
         & ROME &$36.4$ &$29.4$ &$41.6$ &$100$ &$100$ &$50.8$ &$59.7$  \\
         & ICE &$37.5$ &$92.4$ &$40.1$ &$100$ &$100$ &$74.4$ &$74.1$    \\ \midrule
         \multirow{2}{*}{\llama{}} 
         & ROME &$22.0$ &$37.4$ &$16.2$ &$100$ &$100$ &$20.6$ &$49.4$ \\
         & ICE &$57.2$ &$85.1$ &$67.6$ &$100$ &$100$ &$78.0$ &$81.3$   \\ \midrule
         \gptt{} 
         & ICE &$31.0$ &$86.1$ &$65.6$ &$100$ &$100$ &$83.8$ &$77.7$  \\
         \bottomrule
    \end{tabular}
    }
    \caption{Accuracy on the \topviews{} subset, by \mend{}, \rome{}, \memit{}, and the ICE baseline, on \gptxl{}, \gptj{}, \gptneo{}, \llama{}, and \gptt{}.}
\label{table:top_views_results}
\end{table}


% \begin{table*}[t]
% \setlength{\belowcaptionskip}{-10pt}
% \setlength\tabcolsep{2.5pt}
% \footnotesize
% \begin{center}
% \begin{tabular}{l  cc | cc | ccc | cc | c | c }
%  & \multicolumn{2}{c}{\gptm{}} & \multicolumn{2}{c}{\gptl{}} & \multicolumn{3}{c}{\gptxl{}} &  \multicolumn{2}{c}{\gptj{}} & \multicolumn{1}{c}{\llama{}} & \multicolumn{1}{c}{\gptneo{}} \\ [0.1cm]
% \multicolumn{1}{c}{} & \textbf{\mend}  & \textbf{\rome} & \textbf{\mend}  & \textbf{\rome}  & \textbf{\mend}  & \textbf{\rome} & \textbf{\memit}  & \textbf{\rome} & \textbf{\memit}  & \textbf{\rome} & \textbf{\rome}  \\
% \toprule
% \relationspecificity{}          & $22.2$    &$22.1$   
%                     & $23.5$    &$17.4$    
%                      & $21.2$    &$18.5$   &$24.3$ 
%                                 &$22.3$   &$24.8$  
%                                 &$20.6$         
%                                 &$50.8$         \\
% \logicalgeneralization{}    & $19.8$    &$5.4$      
%                         & $20.0$    &$5.7$    
%                         & $25.9$    &$5.7$   &$6.7$ 
%                                     &$5.5$   &$7.0$  
%                                    &$22.0$     
%                                    &$36.4$     \\
% \compositionality{}            & $6.7$    &$27.2$   
%                     & $8.7$    &$26.6$    
%                      & $10.7$    &$46.4$   &$45.2$ 
%                                 &$44.1$   &$45.9$  
%                                 &$37.4$      
%                                 &$29.4$     \\ 
% \forwardcompositionality{}      & $5.3$    &$21.2$   
%                     & $6.0$    &$25.8$     
%                      & $5.4$    &$21.8$   &$21.2$ 
%                                &$21.0$   &$23.7$  
%                                &$16.2$     
%                                &$41.6$      \\
% \aliasing{}    & $100$    &$100$         
%                         & $99.2$    &$100$    
%                          & $100$    &$100$   &$100$ 
%                                     &$98.6$   &$100$  
%                                     &$100$     
%                                     &$100$      \\
% \forgetfulness    & $89.9$    &$99.5$    
%                         & $95.3$    &$99.1$   
%                          & $100$    &$100$   &$100$ 
%                                     &$99.0$   &$100$  
%                                    &$100$      
%                                    &$100$       \\
% \midrule
% AVG                    & $40.6$    & $45.9$   
%                         & $42.1$   & $45.8$ 
%                          & $43.9$   & $48.7$  &$49.6$
%                                     &$48.4$  & $50.2$ 
%                                    &$49.4$      
%                                   & $59.7$       \\
% \bottomrule
% \end{tabular}
% \end{center}
% \caption{Accuracy scores of \mend{}, \rome{} and \memit{} with \gptl{}, \gptxl{}, \gptj{} and \gptneo{}, testing on \topviews{}.}
% \label{table:top_views_results}
% \end{table*}

\paragraph{Models} 
We use 4 recent auto-regressive decoder-only LMs of different sizes: GPT-2 XL \cite{radford2019language} with 
% 345M (GPT2-M), 762M (GPT2-L), and 1,542M
1.5B parameters,
GPT-J \cite{chen2021evaluating} with 6B parameters, LLaMA with 7B parameters, \cite{touvron2023llama}, and GPT-NeoX with 20B parameters \cite{black2022gpt}.
In addition, as our baseline does not require access to the model parameters, we also evaluate it on the closed-source models GPT-3 \texttt{text-davinci-003} with 175B parameters \cite{brown2020language}. 

For all model-method combinations, except for \rome{} with \llama{}, we use the official implementation and hyperparameters from \citet{meng2022locating}. 
% We use the implementation and hyperparameters by \citet{meng2022locating} to edit \gptneo{} with ROME, \gptj{} with ROME and MEMIT, and \gptxl{} with all the 3 methods. 
Also, we adjust ROME to \llama{} by following the method and utilizing the authors' codebase.
Tab.~\ref{table:filtered_tests_portion} shows the number of edits and test queries left, for every model, after filtering out non-successful edits and inapplicable test queries (as described above).

\paragraph{Evaluation} Each model-method pair is evaluated separately, on every subset of \ripple{}. For each evaluation criteria, we first compute the average accuracy over the test queries per example, and then average over all the examples. For every test query, we let the model generate a maximum of 20 token. We consider a generation as successful if one of the target object's aliases appears in the text. In cases of multiple gold target objects (as in \forgetfulness{} test queries), we evaluate each target object separately and consider the generation as correct if the generation was correct with respect to at least one object.

\subsection{Results}
Tab.~\ref{table:recently_emerged_results},~\ref{table:fake_facts_results},~\ref{table:top_views_results} show the evaluation results on the \recentlyemerged{}, \fakefacts{}, and \topviews{} subsets, respectively.
Considering the average scores across all subsets, we observe that existing editing methods struggle to handle the ripple effect induced by editing facts, with low average accuracy of $38-66$ across all models.
This suggests that, while KE methods demonstrate high capability in making local updates to the model's knowledge, these changes are mostly applied at a surface-level without propagating to other related facts.

Moreover, comparing results across test criteria shows that some \ripple{} criteria are handled better than others. For example, while results for the \aliasing{} criteria that measures generalization to paraphrases are high (86.8 or higher across all settings), results for the other criteria are lower and vary between models, methods, and splits (e.g for the \logicalgeneralization{} criteria, results are at a very low $5.5$ on the \topviews{} split with \gptj{} and \rome{}, but are much higher at $71.1$ on the \fakefacts{} split with \llama{} and in-context editing).
Next, we analyze our results across the different dimensions, to reveal fine-grained insights on when current KE methods succeed and fail.

Importantly, we observe that our in-context editing baseline obtains the best overall results. Specifically, ICE outperforms \rome{} by more than 10 points for \gptneo{} and 30 points for \llama{}, on average. Although \gptt{} with ICE performs best on average, the 7B \llama{} is highly competitive, performing better or similarly on the \recentlyemerged{} and \topviews{} splits.

% Figure environment removed

\paragraph{Results across model size}
We analyze how editing performance on \ripple{} is influenced by the model size. To this end, we further evaluate \rome{} on smaller versions of \gpt{} -- with 345M (GPT2-M) and 762M (GPT2-L) parameters, and plot the average accuracy over the three subsets of \ripple{} as a function of model size.
Fig.~\ref{figure:acc_as_a_function_of_size} presents the results, showing that editing performance increases in model size, with \rome{} obtaining substantially higher accuracy when applied to larger models. 
Nevertheless, our results (Tab.~\ref{table:recently_emerged_results},~\ref{table:fake_facts_results},~\ref{table:top_views_results}) show that when using ICE, the 7B \llama{} is competitive with the much larger \gptt{}, suggesting that simply scaling the model size may not be sufficient to fix the drawbacks of current editing methods. 

\paragraph{Results across editing methods}

\begin{table}[t]
\setlength{\belowcaptionskip}{-10pt}
\footnotesize
\begin{center}
\begin{tabular}{lccc}
 & \mend{} & \rome{} & \memit{} \\ [0.1cm]
\toprule
\relationspecificity{}          & $34.4$    &$37.6$   &$39.1$  \\
\logicalgeneralization{}    & $39.1$    &$26.5$   &$29.0$  \\
\compositionality{}            & $17.0$    &$37.9$   &$35.3$   \\ 
\forwardcompositionality{}    & $13.6$    &$37.7$   &$39.1$    \\
\bottomrule
\end{tabular}
\end{center}
\caption{Accuracy of \mend{}, \rome{} and \memit{}, using \gptxl{}, averaged over the three \ripple{} splits - \recentlyemerged{}, \fakefacts{} and \topviews{}.}
\label{table:res_across_methods}
\end{table}

Tab.~\ref{table:res_across_methods} shows the results of \mend{}, \rome{} and \memit{}, on \gptxl{} across the \ripple{} evaluation criteria, while averaging over the three data subsets. Interestingly, \mend{} outperforms \rome{} and \memit{} in \logicalgeneralization{}, but is worse in \compositionality{} and \forwardcompositionality{}, suggesting that different methods might better capture different types of ripple effects.

% Figure environment removed

\paragraph{Results across data splits}
Fig.~\ref{figure:test_scores_per_split} displays results across evaluation splits and criteria. 
Splits differ in whether edited facts are counterfactual or real and in the popularity of the edited entities.
When comparing the \recentlyemerged{} split that examines injection of real recent facts to the counterfactual \fakefacts{} and \topviews{} splits, we observe that for \relationspecificity{}, performance is best on the \recentlyemerged{} split.
Comparing the \fakefacts{} and \topviews{} splits, that differ in the popularity of the edited entities, we see that while \logicalgeneralization{} is higher for \fakefacts{}, \forgetfulness{} is higher for \topviews{}. These results suggest that, although retaining correct knowledge is easier for popular entities, updating other facts that logically follow from an edit is harder for popular entities.


\section{Conclusion}

In this work, we present \texttt{vox2vec} --- a self-supervised framework for voxel-wise representation learning in medical imaging. Our method expands the contrastive learning setup to the feature pyramid architecture allowing to pre-train effective representations in full resolution. By pre-training a FPN backbone to extract informative representations from unlabeled data, our method scales to large datasets across multiple task domains. We pre-train a FPN architecture on more than 6500 CT images and test it on various segmentation tasks, including different organs and tumors segmentation in three setups: linear probing, non-linear probing, and fine-tuning. Our model outperformed existing methods in all regimes. Moreover, \texttt{vox2vec} establishes a new state-of-the-art result on the linear and non-linear probing scenarios. 

Still, this work has a few limitations to consider. We plan to investigate further how the performance of \texttt{vox2vec} scales with the increasing size of the pre-training dataset and the pre-trained architecture size. Another interesting research direction is exploring the effectiveness of \texttt{vox2vec} in the domain adaptation and few-shot learning scenarios.


\iftaclpubformat

% \section{Including acknowledgments}
% Acknowledgments appear immediately before the references.  Do not number this
% section.\footnote{In \LaTeX, one can use {\tt {\textbackslash}section*} instead
% of {\tt {\textbackslash}section}.} If you found the reviewers' or Action
% Editor's comments helpful, consider acknowledging them.
\else
\fi

\section*{Acknowledgments}
We thank Maor Ivgi and Gal Elidan for valuable feedback and constructive suggestions.
This work is supported in part by the Israeli Science Foundation.


\bibliography{tacl2021}
\bibliographystyle{acl_natbib}

\iftaclpubformat

\onecolumn

\appendix

% \section{Author/Affiliation Options as set forth by MIT Press}
% \label{sec:authorformatting}

% Option 1. Author’s address is underneath each name, centered.

% \begin{quote}\centering
%   \begin{tabular}{c}
%     \textbf{First Author} \\
%     First Affiliation \\
%     First Address 1 \\
%     First Address 2 \\
%     \texttt{first.email@example.com}
%   \end{tabular}
%   \ 
%   \begin{tabular}{c}
%     \textbf{Second Author} \\
%     Second Affiliation \\
%     Second Address 1 \\
%     Second Address 2 \\
%     \texttt{second.email@example.com}
%   \end{tabular}

%   \begin{tabular}{c}
%     \textbf{Third Author} \\
%     Third Affiliation \\
%     Third Address 1 \\
%     Third Address 2 \\
%     \texttt{third.email@example.com}
%   \end{tabular}
% \end{quote}
  

% Option 2. Author’s address is linked with superscript characters to its name,
% author names are grouped, centered.

% \begin{quote}\centering
%     \textbf{First Author$^\diamond$} \quad \textbf{Second Author$^\dagger$} \quad
%     \textbf{Third Author$^\ddagger$}
%     \\ \ \\
%     $^\diamond$First Affiliation \\
%     First Address 1 \\
%     First Address 2 \\
%     \texttt{first.email@example.com}
%      \\ \ \\
%      $^\dagger$Second Affiliation \\
%     Second Address 1 \\
%     Second Address 2 \\
%     \texttt{second.email@example.com}
%      \\ \ \\
%     $^\ddagger$Third Affiliation \\
%     Third Address 1 \\
%     Third Address 2 \\
%     \texttt{third.email@example.com}
% \end{quote}
  
\fi

\end{document}


