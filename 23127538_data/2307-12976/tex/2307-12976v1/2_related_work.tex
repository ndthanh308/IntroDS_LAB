
\section{Problem Setting}
\label{sec:setting}

We consider editing of \textit{factual knowledge}, where facts are expressed as triplets $(e, r, o)$ of a subject entity $e$ (e.g. \texttt{Eiffel Tower}), a relation $r$ (e.g. \texttt{City}), and an object $o$ (e.g. \texttt{Paris}).
In a standard KE setting, an edit request $(e, r, o) \rightarrow (e, r, o')$ is made to modify a fact encoded by the model -- setting a new target object $o \rightarrow o'$ of a given subject-relation pair.

We follow this setting, while distinguishing between three cases, based on the knowledge encoded in the model before the edit and the relation of the edit: (a) modification of facts that are already encoded in the model $(e, r, o) \rightarrow (e, r, o')$, that is, updating the object $o \rightarrow o'$ for a given subject entity $e$ and relation $r$, (b) injection of new facts $(e, r, o')$ that are not captured by the model. For one-to-one relations like \texttt{Date of birth}, where there is a single object for a given subject, an injection edit can be viewed as populating an empty object $(e, r, \emptyset) \rightarrow (e, r, o')$. For one-to-many relations, such as \texttt{Sibling} and \texttt{Occupation}, an injection edit augments the set of objects $(e, r, \{o_1,..,o_n\}) \rightarrow (e, r, \{o_1,..,o_n, o'\})$.
Whether an edit is viewed as a modification or injection, depends on whether that information was captured in the model before the edit. 

Evaluating whether a specific fact (before or after an edit) is encoded by a model is typically done by testing if the model predicts the object for various input queries that represent the subject and relation (see more details in \S\ref{subsec:currentstate}). 


