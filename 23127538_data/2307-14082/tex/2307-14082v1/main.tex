\RequirePackage[2018-12-01]{latexrelease}
\documentclass[showpacs,showkeys,11pt,
preprint,preprintnumbers,nofootinbib,
groupedaddress,superscriptaddress,amsmath,amssymb]{revtex4}
\usepackage{amsfonts}
\usepackage{amsmath}
\usepackage{graphicx,subfigure}
\usepackage{caption}
%\usepackage{subcaption}
\usepackage[export]{adjustbox}% http://ctan.org/pkg/adjustbox
%\usepackage{morefloats}
%\usepackage[demo]{graphicx}
\usepackage{epsfig}
\usepackage{url}
\usepackage{multirow}
\usepackage{feynmp}
%\excludecomment{figure}
%\usepackage{mymacros}
%\usepackage{breqn}
\newcommand\emc{E=mc^{2}}
\newcommand {\be}{\begin{equation}}
\newcommand {\ee}{\end{equation}}
\newcommand {\ea}{\end{eqnarray}}
\newcommand {\invfb}{$fb^{-1}$}
\newcommand {\tanb}{$\tan\beta~$}
\newcommand {\ra}{\rightarrow}
\newcommand {\sinb}{s_{\beta}}
\newcommand {\cosb}{c_{\beta}}
\newcommand {\sina}{s_{\alpha}}
\newcommand {\cosa}{c_{\alpha}}
\newcommand {\sbma}{s_{\beta-\alpha}}
\newcommand {\cbma}{c_{\beta-\alpha}}
\newcommand {\sapb}{s_{\alpha+\beta}}
\newcommand {\capb}{c_{\alpha+\beta}}
\newcommand {\sw}{s_{W}}
\newcommand {\tb}{t_{\beta}}
\newcommand {\ctb}{t_{\beta}^{-1}}

\begin{document}
\title{Observability of Parameter Space for Charged Higgs Boson in its bosonic decays in Two Higgs Doublet Model Type-1}

\pacs{12.60.Fr, %  extensions of Higgs sector
      14.80.Fd  %  charged Higgs
}
\keywords{Charged Higgs, MSSM, LHC}
%%%%%%%%%%%%%%%%%%%%%%%%%%%%%%%%%%%%%%%%%%%%%%%%%%%%%%%%%%%%%%%%%%%%%%%%%%%%%%%%
\author{Ijaz Ahmed}
\email{Ijaz.ahmed@fuuast.edu.pk}
\affiliation{Federal Urdu University of Arts, Science and Technology, Islamabad Pakistan}
\author{Waqas Ahmad}
\email{ahmadscientist77@gmail.com}
\affiliation{Riphah International University, Sector I-14, Hajj Complex, Islamabad Pakistan}
\author{ M. S. Amjad}
\email{sohailamjad@nutech.edu.pk}
\affiliation{National University of Technology, Islamabad Pakistan}

\begin{abstract}
This study explores the possibility of discovering $H^{\pm}$ through its bosonic decays, i.e. $H^{\pm}\rightarrow W^\pm\phi$ (where $\phi$ = h or A), within the Type-I Two Higgs Doublet Model (2HDM). The main objective is to demonstrate the available parameter space after applying the recent experimental and theoretical exclusion limits. We suggest that for $m_{H^\pm}$ = 150 GeV is the most probable mass for the $H^\pm\rightarrow W^\pm\phi$ decay channel in $pp$ collisions at $\sqrt{s}$ = 8, 13 and 14 TeV. Therefore we propose that this channel may be used as an alternative to $H^\pm\rightarrow \tau^\pm\nu$. 
%This study explores the possibility of discovering $H^{+}$ through its bosonic decays, i.e. $H^{+}\rightarrow W\phi$ (where $\phi$ = h or A) within Type-I Two Higgs Doublet Model (2HDM). The main objective is to demonstrate the available parameter space after applying the recent experimental and theoretical exclusion limits. We suggest that for the $H^{+}\rightarrow W^{+}\phi$ for light $H^{+}$ in Type-I 2HDM, $m_{H^+}$ = 150 GeV is the most probable in proton-proton collisions at $\sqrt{s}$ = 8, 13 and 14 TeV center of mass energies and thus $H^{+}\rightarrow W^{+}\phi$ might be utilized as one possible channel alternative to $H^{+}\rightarrow \tau^{+} \nu_{\tau}$ decay channel.
 
\end{abstract}

\maketitle

\section{Introduction}
In 2HDM, $H^{\pm}$ is allowed to decay freely in fermions and gauge bosons. In Type-I, $ H^{\pm} \longrightarrow AW^{\pm}$ (decay to a neutral Higgs ``A" and ``W-boson") is a dominant channel.
The decay mode into $\tau^{+}\nu_\tau$ reaches branching ratios of more than 90$\%$ below the $t\bar{b}$ threshold and the muonic one ranges at a few $10^{-4}$ \cite{decaysupression1}. All other leptonic decay channels of the charged Higgs bosons are not important to be considered. As we know that there is a fermiophobic charged Higgs decay for large $tan\beta$ values, and it decays like $H^{\pm}\rightarrow W^{\pm}\phi$  ($ \phi=h, H, A$) if kinematically allowed and it would be dominant decay even for virtual $W^{\pm}$. Thus, it is the most encouraging mode of decay, for larger $tan\beta$ values. In Type-I 2HDM, the bosonic decays of light $H^{\pm}$ were recently studied at the LHC \cite{2017}. For decay processes $H^{\pm}\rightarrow W^{\pm}A$ and $H^{\pm}\rightarrow W^{\pm}h$, Branching Ratios are calculated in \cite{2017}. The $H^{\pm}\rightarrow W^{\pm}h$ reaches a BR of 10\% below the top-bottom threshold, at $tan\beta$ values from 2-3 and $m_{H^{\pm}}$ = 160 GeV.

%This arises from $H^{\pm}$ production modes.  via one of the main three production modes at LHC and successive decay of $H^{\pm}$ into W gauge boson and neutral Higgs boson $\phi$, where ($\phi = h, A, H$).
For the production process, $pp \rightarrow tb H^{\pm}$, which is generally the most dominant mode for $H^{+}$, SM inclusive processes with top-quark pairs are inevitably an important background regardless of how $\phi$  decays \cite{alves2017charged}. Hence for more conventional 2HDM scenarios, the signal process from $\phi \rightarrow \tau\tau$ and $\phi \rightarrow bb$ provides promising avenue near the alignment limit \cite{coleppa2014charged,kling2015light}. Moreover in 2HDM, BR($\phi \rightarrow \tau\tau$) and BR($\phi \rightarrow bb$) with the relative size predicted under additional model assumptions, the search results from the two signatures may be combined to improve the sensitivity coverage of the 2HDM parameter space.
The {\tt 2HDMC-1.7.0} \cite{2hdmc} is used to put theoretical constraints and experimental bounds are applied. For that purpose, HiggsBounds\cite{higgsbounds} and HiggsSignals\cite{higgssignals} libraries are interfaced with 2HDMC, and also ScannerS\cite{scanners} is used to put the most recent experimental bound on the selected parameters and compare whether it is allowed or not experimentally.

%In THDM, \begin{math} H^{\pm} \end{math} is allowed to decay freely in fermions and gauge-bosons. 2HDM Type-I's viable parameter space phenomenologically feature \begin{math} H^{\pm} \end{math} that decays in neutral Higgs "A" and "W-boson" dominantly. In 2HDM Type-I, the decay \begin{math} \tau^+ \nu_\tau \end{math} is very less significant with branching ratio usually the order of \begin{math} m^2_\tau /m^2_t \approx 10^{-4} \end{math}. So this leads to the identification, impossible if fermionic decay becomes dominant at large hadron collider. As we know that there is fermiophobic charged Higgs boson for large \begin{math} \tan\beta \end{math} values and it decays like \begin{math} H^{\pm}\rightarrow W^{\pm}\phi \end{math} where (\begin{math} \phi=h\hspace{1mm}or\hspace{1mm} H \hspace{1mm} or \hspace{1mm} A \end{math}) if kinematically allowed and it would be dominant decay even for virtual \begin{math} W^{\pm} \end{math}. Thus, the most encouraging modes of decay for larger \begin{math} \tan\beta \end{math} values are into W-boson with one neutral Higgs boson. In 2HDM Type-I, bosonic decays of light \begin{math} H^{\pm} \end{math} are recently studied at LHC \cite{2017}. For decay processes \begin{math} H^{\pm}\rightarrow W^{\pm}A^o \end{math} and \begin{math} H^{\pm}\rightarrow W^{\pm}h \end{math}, BR's are calculated in \cite{2017}. %Results clearly tells us that for channel \begin{math} H^{\pm}\rightarrow W^{\pm}h \end{math}, the BR upto 0.1 enhanced before top-bottom threshold at \begin{math} \tan\beta \end{math} varies from 2-3 and fixed \begin{math} m_{H^{\pm}} \end{math} = 160 Gev. For all other channels, the decay \begin{math} H^{\pm} \rightarrow tb \end{math} becomes prominant above top-bottom threshold. The decay \begin{math} H^{\pm}\rightarrow W^{\pm}A^o \end{math} is a leading one channel with light CP-odd Higgs \begin{math} m_{A^o} \leq 100 Gev \end{math} and \begin{math} 100\leq m_{H^{\pm}}\leq200 \end{math} while the decay \begin{math} H^{\pm}\rightarrow tb \end{math} is the second leading channel. These arises from \begin{math} H^{\pm} \end{math} production modes via one of main three production modes at LHC and successive decay of \begin{math} H^{\pm} \end{math} into W gauge boson and neutral Higgs boson \begin{math} \phi \end{math} where (\begin{math} \phi = h \hspace{1mm}or\hspace{1mm}A\hspace{1mm}or\hspace{1mm} H \end{math}). For production process \begin{math} pp \rightarrow tb H^{\pm} \end{math}, which is generally the most predominant mode of production of $H^{+}$, SM inclusive processes with top-quark pairs are inevitaly an important background quasi-regardless of how \begin{math} \phi \end{math} (neutral Higgs bosons) of signal process decays\cite{alves2017charged}. Hence expected for more conventional 2HDM scenerios, the signal process from \begin{math} \phi \rightarrow \tau\tau \end{math} and \begin{math} \phi \rightarrow bb \end{math} provides more promising avenue near the alignment limit\cite{coleppa2014charged,kling2015light}. Moreover in 2HDM, BR(\begin{math} \phi \rightarrow \tau\tau  \end{math}) and BR(\begin{math} \phi \rightarrow bb  \end{math}) with relative size predicted under additional model assumption, from 2-signatures search results may combined to increase the spatial sensitiveness of the 2HDM parameters.

\section{Review of 2HDM}
The scalar potential of 2HDM \cite{2HDMref} has 14 parameters, including the charge violation, and CP violation. The general term for scalar potential is as follows,
\newline
\begin{multline}
V=m^2_{11} \Phi^{\dagger}_1 \Phi_1 + m^2_{22} \Phi^{\dagger}_2 \Phi_2 - m^2_{12} \biggl\{ \Phi^{\dagger}_1 \Phi_2 + h.c. \biggr\} + \frac{\lambda_1}{2} (\Phi^{\dagger}_1 \Phi_1)^2 + \\
 \frac{\lambda_2}{2} (\Phi^{\dagger}_2 \Phi_2)^2 + \lambda_3 (\Phi^{\dagger}_1 \Phi_1)(\Phi^{\dagger}_2 \Phi_2) + \lambda_4 (\Phi^{\dagger}_1 \Phi_2) (\Phi^{\dagger}_2 \Phi_1) + \\
  \biggl\{ \frac{\lambda_5}{2} (\Phi^{\dagger}_1 \Phi_2)^2 + \left[ \lambda_6 (\Phi^{\dagger}_1 \Phi_1) + \lambda_7 (\Phi^{\dagger}_2 \Phi_2) \right] (\Phi^{\dagger}_1 \Phi_2) + h.c. \biggl\}
\end{multline}
where $( \lambda_{i}, i = 1,2,3,...., 7)$  are dimensionless coupling parameters, $ m^2_{11} , m^2_{22}$ and $m^2_{12}$ are squares of masses. To treat the 2HDM potential as charge and parity conserving potential, all the parameters should be real. The vacuum expectation value VEV is acquired by each scalar-doublet when electroweak symmetry breaks. The two doublets are,
%From above expression, \begin{math} \hspace{1mm} ( \lambda_{i} \hspace{1mm} where \hspace{1mm} i = 1,2,3,...., 7) \end{math} \hspace{1mm} are dimension-free coupling parameters, \begin{math} m^2_{11} , m^2_{22} \hspace{1mm} and \hspace{1mm} m^2_{12} \end{math} are squares of parameters of mass. From these parameters, \begin{math} m^2_{12} \end{math} and \begin{math} \lambda_{i} \hspace{1mm} (where i = 5,6,7) \end{math} are complex parameters while others are real. It is assumed that all the parameters that are used are real, in order to treat potential as Charge-Parity conserving because charge parity of complex-parameters explicitly violated by potential with presence of non-zero imaginary parts. VEV (vacuum expectation value) is acquired by each scalar-doublet when electroweak symmetry breaks i.e. given as

\begin{equation}
<\Phi_1>=\binom{0}{\frac{v_1}{\sqrt{2}}} , <\Phi_2>=\binom{0}{\frac{v_2}{\sqrt{2}}}
\end{equation}
These two doublets lead to eight fields among which three correspond to massive $W^{\pm}$ and $Z^0$ vector bosons, and the remaining five fields lead to five physical Higgs bosons.
%From the minimization of above potential we get eight fields for a specific zone, three fields are absorbed by $W^{\pm}$ and $Z^0$ vector gauge bosons and we are left with only five physical Higgs fields.
\begin{equation}
\Phi_i=\binom{\Phi^+_i}{\frac{(v_i+\rho_i+\iota\eta_i)}{\sqrt{2}}}
\end{equation}
where   i=1,2
with \begin{math} V_1=Vcos\beta, V_2=Vsin\beta \hspace{1mm} and \hspace{1mm} V_1,V_2 \geq 0 \end{math}
\newline
It satisfy the condition $V_{SM} = \sqrt{v^2_1+v^2_2}$. The experimentally obtained value of $V_{SM}$ is  246.22 GeV. The obtained fields are given as,
%It satisfy the condition \begin{math} V_{SM} = \sqrt{v^2_1+v^2_2} \end{math}. Experimentally \begin{math} V_{SM} = 246.22 GeV \end{math}. The obtained fields are given as,
\begin{equation}
\binom{\rho_a}{\rho_b}= \begin{pmatrix}
cos\alpha & -sin\alpha\\sin\alpha & cos\alpha
\end{pmatrix} \binom{H}{h} \hspace{1mm},\hspace{1mm} \binom{\eta_a}{\eta_b} = \begin{pmatrix}
cos\beta & -sin\beta\\sin\beta & cos\beta
\end{pmatrix} \binom{G^0}{A}
\end{equation}
and
\begin{equation}
\binom{\Phi^{\dagger}_a}{\Phi^{\dagger}_b}= \begin{pmatrix}
cos\beta & -sin\beta\\sin\beta & cos\beta
\end{pmatrix} \binom{G^{\dagger}}{H^{\dagger}}
\end{equation}
Mass-matrix of charged Higgs states are diagonalized by rotational angle and it is defined as $tan\beta = \frac{V_{2}}{V_{1}}$. Similarly mass-matrix of scalar Higgs states are diagonalized by the rotational angle \begin{math} \alpha \end{math} and satisfies following relation,

\begin{equation}
\tan(2\alpha)=\frac{2(-m^2_{12}+\lambda_{345}V_1V_2)}{m^2_{12}(V_2/V_1 - V_1/V_2)+\lambda_1V^2_1-\lambda_2V^2_2}
\end{equation}
where,\begin{math} \hspace{1cm} \lambda_{345}=\lambda_3+\lambda_4+\lambda_5 \end{math}


%%ATHER................ Continue from here

\subsection{Theoretical and Experimental Bounds}
The general potential of 2HDM is too complex as compared to the one in the standard model SM. The 2HDM imposes theoretical constraints on the potential to guarantee the stability of potential.\newline
The Higgs potential should be positive throughout the field space. This ensures that a stable vacuum configuration for asymptotically large field values is maintained. The quartic terms are the leading terms at large field space values. The following substitutions are helpful,

%As compared SM, General potential of 2HDM is too complex. Theoretically 2HDM imposes constraints on general potential to guarantee the stability of potential.\newline
%To maintain a stable vacuum configuration for asymptotically large field values, the positivity of Higgs potential should be mandatory throughout the field space. In general potential, the quartic terms are the leading terms at large field space values. In order to reach the mandatory conditions for the positivity of potential, following substitutions help us,
\begin{equation}
\mid \Phi_1 \mid=r cos \phi \hspace{1mm},\hspace{1mm} \mid\Phi_2\mid=rsin\phi \hspace{4mm}and\hspace{4mm} \frac{\Phi_1\Phi^{\dagger}_2}{\mid\Phi_1\mid\mid\Phi_2\mid} =\rho\exp(\iota\theta)
\end{equation}
\newline
where, \hspace{1cm} \begin{math} \rho=\mid0-1\mid \hspace{1mm},\hspace{1mm} \theta=\mid0-2\pi\mid \hspace{3mm}and\hspace{3mm} \phi=\mid0-\frac{\pi}{2}\mid \end{math}

After making these substitutions, and omitting the common factor $r^4$, the quartic terms of the potential can be expressed as, %and ignoring the common factor (\begin{math} r^4 \end{math}),
\begin{multline}
V_{(4)}=\frac{1}{2}\lambda_1 cos^4\phi + \frac{1}{2}\lambda_2 sin^4\phi + \lambda_3cos^2\phi sin^2\phi+\lambda_4\rho^2cos^2\phi sin^2\phi \\
+ \lambda_5\rho^2 cos^2\phi sin^2\phi cos2\theta + [\lambda_6 cos^2\phi + \lambda_7 sin^2\phi]2\rho cos\phi sin\phi cos\theta
\end{multline}
From above equation we can see that the potential will be positive if,
\begin{equation}
\lambda_1>0 \hspace{4mm},\hspace{4mm}\lambda_2>0\hspace{4mm}and\hspace{4mm}\lambda_3>-\sqrt{\lambda_1\lambda_2}
\end{equation}
If \begin{math} \lambda_{6} = 0 = \lambda_{7} \end{math}, a case for natural conservation of flavor, then an extra condition must be satisfied,
\begin{equation}
\lambda_3+\lambda_4-\mid\lambda_5\mid>-\sqrt{\lambda_1\lambda_2}
\end{equation}
If either \begin{math} \lambda_{6} \not= 0 \hspace{1mm} or \hspace{1mm} \lambda_{7} \not= 0 \end{math} then the above condition changes to,
\begin{equation}
\lambda_3+\lambda_4-\mid\lambda_5\mid>-\sqrt{\lambda_1\lambda_2}
\end{equation}
\newline
Scattering matrices are unitary in order to conserve probability. In the theory of weak couplings, the contribution of higher order terms decreases gradually. While in the theory of strong couplings, individual contributions increase arbitrarily. The eigenvalues (\begin{math} L_i \end{math} ) of S-matrices must satisfy the condition \begin{math} L_i \leq 16\pi \end{math} in order to achieve the tree-level unitarity that means the saturation of S-matrices up to tree-level unitarity.\newline
Perturbation constraints requires that the quartic Higgs couplings must satisfy the condition \begin{math} \mid C_{H_{i},H_{j},H_{k},H_{l}} \mid \leq 4\pi \end{math}. One can imagine that some interaction channels are non-perturbative while others are perturbative. \begin{math} \mid \lambda_i \mid = 4\pi\xi \end{math} , is another way to explain this constraint, where \begin{math} \xi = 0.8 \end{math} \footnote{The value is arbitrarily chosen as an upper bound}. This gives $\mid \lambda_i \mid \leq 10$ for $\lambda_i$, as the upper bound. % The value of soft parameter \begin{math} \mu \end{math} is compatible with $M_{H^\pm}$ and $M_2$. % when large values of masses are bound to constraint.

Alongside theoretical constraints, there are also experimental constraints coming from B-Physics and various experiments on different colliders from recent Higgs searches. Here we discuss some important constraints. Also, using SuperIso V.3.2\cite{mahmoudi2008superiso}, we list the SM prediction values provided in this category. The Standard Model BR for $(B_\mu \rightarrow \tau\nu)_{SM}$ reported in \cite{mahmoudi2008superiso} is: %in the large \begin{math} tan\beta \end{math} region, for spinless mediator \begin{math} H^{\pm} \end{math} no such suppression, however both contributions can be comparable \cite{PhysRevD.48.2342}. Share of 2HDM factorizes in ratio \begin{math} R^{exp}_{SM} \end{math} as compared to standard model.


%As we know that scattering matrix of any order must be unitary matrix. In theory of weak couplings, share of higher orders goes on decreasing generally. While in theory of strongly couplings, individual contributions goes on increasing arbitrarily. The eigen values (\begin{math} L_i \end{math} ) of S-matrices should must satisfy the condition \begin{math} L_i \leq 16\pi \end{math} in order to achieve the tree-level unitarity that means the saturation of S-matrices upto tree-level unitarity.\newline
%Perturbativity is the constraint that the theory must remain inside of perturbative region, in order to achieve the quartic Higgs couplings must satisfy the condition \begin{math} \mid C_{H_{i},H_{j},H_{k},H_{l}} \mid \leq 4\pi \end{math}. One can imagine that some interaction channels are strong while others are perturbative, in a mixed situation. \begin{math} \mid \lambda_i \mid = 4\pi\xi \end{math} , is another way to explain this constraint, where \begin{math} \xi = 0.8 \end{math}. This gives \begin{math} \mid \lambda_i \mid \leq 10 \hspace{1mm} for \hspace{1mm} \lambda_i \end{math}, is upper bound. The value of soft parameter \begin{math} \mu \end{math} is compatible with \begin{math} M_{H^\pm} and M_2 \end{math} when large values of masses bound to constraint.\newline
%Along side by side with theoretical contraints, there are also experimental constraints coming from B-Physics constraints and various experiments on different colliders from recent Higgs searches. Here we discuss some important constraints out of various. And by using SuperIso V.3.2, we enlist the SM prediction values provided in this category\cite{mahmoudi2008superiso}.\newline
%In case of ($(B_\mu \rightarrow \tau\nu)_{SM}$), in the large \begin{math} tan\beta \end{math} region, for spin less mediator \begin{math} H^{\pm} \end{math} no such suppression, however both contributions can be comparable\cite{PhysRevD.48.2342}. Share of 2HDM factorizes in ratio \begin{math} R^{exp}_{SM} \end{math} as compared to standard model. Calculated SM BR numerical value is\cite{mahmoudi2008superiso}.

\begin{equation}
BR(B_\mu\rightarrow\tau\nu)_{SM} \hspace{4mm}=\hspace{4mm}(1.01\hspace{2mm} \pm \hspace{2mm} 0.29)\hspace{2mm} \times \hspace{2mm} 10^{-4}
\end{equation}
The standard model estimation may, in fact, be contrasted to the most recent heavy flavour averaging (HFAG) result.\cite{asner2010averages}. 
\begin{equation}
BR(B_\mu\rightarrow\tau\nu)_{exp} \hspace{4mm}=\hspace{4mm}(1.64\hspace{2mm} \pm \hspace{2mm} 0.34)\hspace{2mm} \times \hspace{2mm} 10^{-4}
\end{equation}
so the ratio will become,
\begin{equation}
R^{exp}_{SM}\hspace{4mm}=\hspace{4mm}\frac{BR(B_\mu\rightarrow\tau\nu)_{exp}}{BR(B_\mu\rightarrow\tau\nu)_{SM}} \hspace{4mm}=\hspace{4mm}1.62\hspace{2mm}\pm\hspace{2mm}0.54
\end{equation}
This causes the exclusion of two sectors of \begin{math} (tan\beta)/m_{H^\pm} \end{math} ratio in 2HDM \cite{mahmoudi2010flavor}. This implies that for \begin{math} tan\beta \geq 1 \end{math}, the mass of charged Higgs must be greater than 800 GeV for Type-II 2HDM\cite{800gev2hdm}. As we know that the ( \begin{math} B_\mu \rightarrow {\tau}{\nu}_{\tau} \end{math} ) decay depends on \begin{math} \mid V_{ub} \mid \end{math} so the  ( \begin{math} B_\mu \rightarrow Dl \nu \end{math} ) (semi-leptonic) decay also depends on \begin{math} \mid V_{ub} \mid \end{math} , i.e. more precisely known as compared to \begin{math} \mid V_{ub} \mid \end{math}  and the branching ratio of ( \begin{math} B_\mu \rightarrow {\tau}{\nu}_{\tau} \end{math} ) is fifty times greater than the branching ratio of (\begin{math} B_\mu \rightarrow {\tau}{\nu} \end{math}) in standard model but it is still difficult to detect because two neutrinos exist in its final state. The 2HDM deals only with the numerator of the ratio.

\begin{equation}
{\xi}_{Dl\nu_\tau} \hspace{4mm}=\hspace{4mm} \frac{BR(B\rightarrow D{\tau}{\nu}_{\tau})}{BR(B\rightarrow Dl{\nu}_{\tau})}
\end{equation}
and allow us to reduce to some extent the theoretical uncertainties. The experimental outcomes by BaBar collaborations and SM predictions \cite{mahmoudi2010flavor} are as follows.

%and allow us to reduce to some extant the theoretical uncertainties. The experimental outcomes by BaBar collaborations\cite{babar2012evidence} and SM predictions\cite{mahmoudi2010flavor} are as follows.
\begin{equation}
{\xi}^{SM}_{Dl{\nu}_{\tau}} \hspace{4mm}=\hspace{4mm} (29.7 \hspace{2mm} \pm \hspace{2mm} 3)\hspace{2mm} \times \hspace{2mm} 10^{-2}
\end{equation}
\begin{equation}
{\xi}^{exp}_{Dl{\nu}_{\tau}} \hspace{4mm}=\hspace{4mm} (44.0 \hspace{2mm} \pm \hspace{2mm} 5.8 \hspace{2mm} \pm \hspace{2mm} 4.2)\hspace{2mm} \times \hspace{2mm} 10^{-2}
\end{equation}
\newline
For ($B \rightarrow X_s\gamma$), this special transition is mediated by \begin{math} H^{\pm} \end{math} and it includes Flavor-changing neutral current (FCNC) and \begin{math} W^{\pm} \end{math} contributions.
%contributes in this transition when it can also be comparable to SM. 
As to the respective BR, the contribution of charged Higgs is always positive so to probe Type-II 2HDM, this can be used efficiently. For this transition, the NNLO-SM predicted $(3.34\pm0.22)\times10^{-4}$ for $BR(B \rightarrow X_s\gamma)_{SM}$ \cite{mehmodi-2008}.
So, for $BR(B \rightarrow X_s\gamma)_{exp}$ recently experimentally calculated value is $(3.32\pm0.15)\times10^{-4}$,
For Type-II Yukawa interactions, this constraint excludes the light-charged Higgs. Higher order analysis \cite{Borzumati:1998nx} estimated lower limit of \begin{math} M_{H^{\pm}} \end{math} is 380 GeV at 95\% C.L. However, it is important to mention that the bound in \cite{Borzumati:1998nx} does not include novel experimental and theoretical predictions and hence the numerical results maybe outdated.\newline
%The uncertainty comes from decay constant \begin{math} f_{Ds} \end{math} value. 

For ($D_S \rightarrow \tau\nu$), the SM  prediction is $(3.32\pm0.15)\times10^{-4}$ \cite{mahmoudi2008superiso}, for \begin{math} f_{Ds} = 0.248 \pm 2.5 GeV \end{math} \cite{PhysRevD.82.114504} and the updated experimental calculation for $BR(D_s \rightarrow \tau\nu)_{exp}$ is $(5.51\pm0.24)\times10^{-2}$ \cite{theheavyflavoraveraginggroup2011averages}, for ($ B_{d/s} \rightarrow \mu^+\mu^- $). At large $tan \beta$ values, the lower limit on charged Higgs mass $m_{H^{\pm}}$ is given in \cite{logan2000bs}.
%The lower limit established by the non-observability of these decays at large \begin{math} \tan\beta \end{math} values on \begin{math} H^{\pm} \end{math} \hspace{1mm} mass are given in \cite{logan2000bs}. 
For decays $BR(B_s \rightarrow {\mu}^+{\mu}^+)_{SM}$ and $BR(B_d \rightarrow {\mu}^+{\mu}^+)_{SM}$, SM predictions are $(3.54\pm0.27)\times10^{-9}$ and $(1.1\pm0.1)\times10^{-9}$ respectively \cite{mahmoudi2008superiso}. Experimental results for the limits of these decays at 95\% C.L. are $BR(B_s \rightarrow {\mu}^+{\mu}^+)_{exp} < 4.5 \times 10^{-9}$ and $BR(B_d \rightarrow {\mu}^+{\mu}^+)_{exp} < 1.0 \times 10^{-9}$ presented by LHCb collaboration \cite{aaij2012strong}. If the results from ATLAS and CMS \cite{lhcbatlas} in above limits are also added then more strict limits are $BR(B_s \rightarrow {\mu}^+{\mu}^+)_{exp} < 4.2 \times 10^{-9}$ and $BR(B_d \rightarrow {\mu}^+{\mu}^+)_{exp} < 8.1 \times 10^{-10}$.

%For ($B \rightarrow X_s\gamma$), this special transition is mediated by \begin{math} H^{\pm} \end{math} and it includes FCNC and  where \begin{math} W^{\pm} \end{math} contributes in this transition when it can also comparable to SM. As to the respective BR, the contribution of charged Higgs is always positive so to probe 2HDM type-II, this can be used efficiently. For this transition, the NNLO-SM predicted $(3.34\pm0.22)\times10^{-4}$ for $BR(B \rightarrow X_s\gamma)_{SM}$ \cite{Mahmoudi_2008}.
%So, for $BR(B \rightarrow X_s\gamma)_{exp}$ recently experimentally calculated value is $(3.32\pm0.15)\times10^{-4}$,
%For type-II yukawa interactions, this constraint excludes the light charged Higgs. Higher order analysis\cite{Borzumati:1998nx} estimated lower limit of \begin{math} M_{H^{\pm}} \end{math} is 380 GeV at 95\% C.L.\newline
%For ($D_S \rightarrow \tau\nu$), on light charged Higgs, this decay also put constraints. The uncertainty comes from decay constant \begin{math} f_{Ds} \end{math} value, in this decay. For this decay, the SM  prediction is $(3.32\pm0.15)\times10^{-4}$ \cite{mahmoudi2008superiso}, for \begin{math} f_{Ds} = 0.248 \pm 2.5 GeV \end{math} \cite{PhysRevD.82.114504} and the updated experimental calculation for $BR(D_s \rightarrow \tau\nu)_{exp}$ is $(5.51\pm0.24)\times10^{-2}$ \cite{theheavyflavoraveraginggroup2011averages}, for ($ B_{d/s} \rightarrow \mu^+\mu^- $), lower limit established by the non-observability of these decays at large \begin{math} \tan\beta \end{math} \cite{logan2000bs} values on \begin{math} H^{\pm} \end{math} \hspace{1mm} mass. For decays $BR(B_s \rightarrow {\mu}^+{\mu}^+)_{SM}$ and $BR(B_d \rightarrow {\mu}^+{\mu}^+)_{SM}$, SM predictions are $(3.54\pm0.27)\times10^{-9}$ and $(1.1\pm0.1)\times10^{-9}$ respectively \cite{mahmoudi2008superiso}. Experimental results for the limits of these decays at 95\% C.L. are $BR(B_s \rightarrow {\mu}^+{\mu}^+)_{exp} < 4.5 \times 10^{-9}$ and $BR(B_d \rightarrow {\mu}^+{\mu}^+)_{exp} < 1.0 \times 10^{-9}$ presented by LHCb collaboration\cite{aaij2012strong}. If the results from ATLAS and CMS\cite{lhcbatlas} in above limits are also added then more strict limits are $BR(B_s \rightarrow {\mu}^+{\mu}^+)_{exp} < 4.2 \times 10^{-9}$ and $BR(B_d \rightarrow {\mu}^+{\mu}^+)_{exp} < 8.1 \times 10^{-10}$.

\section{Discussion}
%%mentioning BR procedure to calculate by Ijaz
The branching ratios are calculated using HDECAY \cite{hdecay} through the anyHdecay interface of ref \cite{scanners}, and predictions for gluon-fusion and bb-associated Higgs production at hadron colliders are obtained using tabulated results from SUSHI \cite{sushi}. The V H-associated (sub)channel cross section predictions are made using the HiggsBounds parametrizations, and the charged Higgs production in association with a top-quark is tested using the HiggsBounds  and HiggsSignals.
These above constraints and calculations are implemented in ScannerS.
%%%ATHER.................START HERE
Figure \ref{BRHtoWPhi}, shows BR\begin{math}(H^{+}\rightarrow W^{+}h) \end{math} with respect to $\tan \beta$ values for different masses of $m_{H^+}$. One can see that for $m_{H^{+}}=800$ and $1000$ \hspace{1mm} GeV, the BR\begin{math}(H^{+}\rightarrow W^{+}h) \end{math} remains between $ 70\% \to 80\%$. For $m_{H^+}=600 \, \textrm{GeV}$, the maximum BR reaches more than $80 \%$ for large $\tan \beta$ values. For $\tan \beta > 3$, the maximum BR is achieved for $m_H^+ = 400$ GeV. For $m_{H^+}=200 \, \textrm{GeV}$, the BR remains less than $20\%$ across the $\tan \beta$ range. 

%that the maximum BR\begin{math}(H^{+}\rightarrow W^{+}h) \end{math} is obtained for \begin{math} m_{H^{+}}=400 \end{math}\hspace{1mm} GeV at  9$\leq$ tan$\beta \leq$ 10. Furthermore, one can see that for \begin{math} m_{H^{+}}=(800,1000) \end{math}\hspace{1mm} GeV, the BR\begin{math}(H^{+}\rightarrow W^{+}h) \end{math} is very close to $ 70\% \leq $ BR$(H^{+}\rightarrow W^{+}h) \leq 80\% $. For \begin{math} m_{H^{+}}=600 \end{math}\hspace{1mm} GeV,  For \begin{math} m_{H^{+}}=600 \end{math}\hspace{1mm} GeV, the BR\begin{math}(H^{+}\rightarrow W^{+}h) \end{math} are higher than the BR of higher masses as expected.\\
%
%. 1, shows that max[BR\begin{math}(H^{+}\rightarrow Wh) \end{math}] obtained for \begin{math} m_{H^{+}}=400 \end{math}\hspace{1mm} GeV at  9$\leq$ tan$\beta \leq$ 10 i.e. approximately closed to 100\%. For \begin{math} m_{H^{+}}=200  \end{math}\hspace{1mm} GeV, min[BR\begin{math}(H^{+}\rightarrow Wh) \end{math}] i.e. hardly $ 0\% \leq $ BR$(H^{+}\rightarrow Wh) \leq 20\% $. In further analysis of FIG. 1, it is observed that for \begin{math} m_{H^{+}}=(800,1000) \end{math}\hspace{1mm} GeV, the BR\begin{math}(H^{+}\rightarrow Wh) \end{math} lies very much closed to each mass i.e. approximately $ 70\% \leq $ BR$(H^{+}\rightarrow Wh) \leq 80\% $. For \begin{math} m_{H^{+}}=600 \end{math}\hspace{1mm} GeV, 2{nd} maxmium BR obtained at tan\begin{math} \beta\approx7 \end{math} i.e. BR\begin{math}(H^{+}\rightarrow Wh) \approx 90\% \end{math}. For \begin{math} m_{H^{+}}=600 \end{math}\hspace{1mm} GeV, the BR's\begin{math}(H^{+}\rightarrow Wh) \end{math} are relatively close to each but higher than the BR of higher masses.\\\\\\\\\\\\\\

% Figure environment removed

% Figure environment removed
Figure \ref{BRHtoTB} shows that, for \begin{math} m_{H^{+}}=200 \end{math}\hspace{1mm} GeV at \begin{math} 2\leq tan\beta \leq 3 \end{math}, the BR\begin{math}(H^{+}\rightarrow tb) \end{math} is the most dominant i.e. BR\begin{math}(H^{+}\rightarrow tb) \approx 100\% \end{math}. One can see from Figure \ref{BRHtoWPhi} and \ref{BRHtoTB} that for smaller $tan \beta$ values, the $H^+ \to tb$ decay mode is most dominant while for higher values, the $H^+ \to Wh$ becomes most probable. However, for $m_{H^+} = 200$ GeV, the scenario is different, as BR\begin{math}(H^{+}\rightarrow tb) \end{math}  is dominant while  BR\begin{math}(H^{+}\rightarrow Wh) \end{math} is less probable.



%Figure. 2 shows that, for \begin{math} m_{H^{+}}=200 \end{math}\hspace{1mm} GeV at \begin{math} 2\leq tan\beta \leq 3 \end{math}, the BR\begin{math}(H^{+}\rightarrow tb) \end{math} becomes more dominant i.e. BR\begin{math}(H^{+}\rightarrow tb) \approx 100\% \end{math}. And it will also be observed in FIG. 3(right). FIG. 1(right) and FIG. 2(right) are nearly reversed to each others, e.g. for \begin{math} m_{H^{+}}=200 \end{math}\hspace{1mm} GeV, FIG. 1(right) shows min[BR] while FIG. 2(right) shows max[BR]. In FIG. 2(rigth), the BR\begin{math}(H^{+}\rightarrow tb) \end{math} is decreasing with increasing \begin{math} m_{H^{+}}\end{math} step by step. And for \begin{math} m_{H^{+}}=1000 \end{math}\hspace{1mm} GeV, FIG 2(right) shows min[BR\begin{math}(H^{+}\rightarrow tb) \end{math}]. Decay of \begin{math} H^{+} \end{math} to top-bottom quark takes over if \begin{math} m_{H^{+}}> m_{t}+m_{b} \end{math}.


% Figure environment removed

% Figure environment removed

Figure \ref{BRWhandtbConstraints} (left) is the selection of points in ($tan\beta$,BR($H^{+}\rightarrow W^{+}h)$  parameter space when both theoretical and experimental constraints are applied at different charged Higgs masses while the Figure \ref{BRWhandtbConstraints} (right) is the  BR($H^{+}\rightarrow tb$)  parameter space. Effects of varying the parameters fixed here, will be shown and discussed later.


Figure \ref{tanBvsmass3D} shows that for \begin{math} tan{\beta} \in [2,3] \end{math} and \begin{math} m_{H^+} \in [150,210] \end{math}, the BR\begin{math}(H^{+} \rightarrow W^{+}h) \end{math} shows up in the region from [0.06,0.07] as shown in the vertical palette. It is also observed that for \begin{math} tan{\beta} \in [2,3] \end{math} and \begin{math} m_{H^+} \in [150,210] \end{math}, the BR\begin{math}(H^{+} \rightarrow tb) \end{math} shows up in the region from [0.65,0.7] as shown in the vertical palette. Under a same range of \begin{math} m_{H^{+}} \end{math} and $tan\beta$, BR\begin{math}(H^{+}\rightarrow tb) \end{math} becomes boosted towards the point that it overrides all the other decay modes. When going over light-charged Higgs to heavy-charged Higgs, we see there are most observable states that exist in the defined region.

% Figure environment removed

% Figure environment removed

From Figure \ref{BRvsmAWh} and Figure \ref{BRvsmAtb}, it is observed that the experimental bounds do not pose any further constraints  for $H^{+}\rightarrow W^{+}A$ and $H^{+}\rightarrow tb$.

%it is not much different from the impact of theoretical and experimental bounds for $H^{+}\rightarrow W^{+}A$ and $H^{+}\rightarrow$tb

% Figure environment removed

Figure \ref{mAvsmHplus3D}, shows that except some short regions, the BR\begin{math}(H^{+}\rightarrow WA) \end{math} is dominant and becomes 100\% in the mass range \begin{math} m_{H^+} = [100,200] \end{math}. The tb-decay mode is the least dominant because of the kinematic constraints and it only becomes prominent  for $m_{H^+}$ above 180 GeV. It is clear that BR\begin{math}(H^{+}\rightarrow WA) \end{math} could be the leading decay channel, i.e, for \begin{math} m_{A} \leq 100 \end{math}\hspace{1mm} GeV and any mass of $m_{H^+}$.

% Figure environment removed

Figure \ref{tanBvsSinBmABR} shows a scan over arbitrarily chosen and fixed $m_{H^{+}}$ between 160 GeV and 180 GeV i.e. 170 GeV. It shows the size of BR($H^{\pm}\rightarrow W^{\pm{*}}h + W^{\pm{*}}A$) over the $sin(\beta-\alpha)$ vs $tan\beta$ (left) and ${m^{2}}_{12}$ vs $tan\beta$ (right). The right panel shows the effect of the soft $Z_2$ breaking term $m_{12}$. It is clear that for some special choices of $tan\beta$ and ${m^{2}}_{12}$, the BR($H^{\pm}\rightarrow W^{\pm{*}}h + W^{\pm{*}}A$) could reach above 50\% for $sin(\beta - \alpha)$ between 0.6 and 0.7. The right figure is drawn by fixing $sin(\beta - \alpha)$ at 0.65, and variation is shown as a function of $m_{12}^2$. The favorable green zones are located in regions with $m_{12}^2 > 12,000$ GeV.


%While in the right panel, the LHC data favor $sin(\beta-\alpha)$ to be rather $\approx$ 1, i.e. decoupling limit and it implies $cos(\beta-\alpha)\approx 0$. The coupling $W^{\pm}H^{\mp}h$ proportional to $cos(\beta-\alpha)$ and is suppressed for $sin(\beta-\alpha)\approx 1$ while the coupling $W^{\pm}H^{\mp}A$ has no factor of suppression. And that fact will make BR($H^{\pm}\rightarrow W^{\pm{*}}h$) smaller than BR($H^{\pm}\rightarrow W^{\pm{*}}A$) for $m_{h}$=$m_{A}$. In the scenario where $m_{h}$=$m_{A}$, when BR($H^{\pm}\rightarrow t^{*}b$) is maximal BR($H^{\pm}\rightarrow W^{\pm{*}}h + W^{\pm{*}}A$) is suppressed and vice versa i.e. both are anti-correlated. \\

% Figure environment removed

Figure \ref{sigmaBRvsmHp}, shows $\sigma \times$ BR($t\rightarrow H^{+}b$) $\times$ BR($H^{+}\rightarrow W\phi$) with $\phi=h$ (left) and $\phi=A$ (right) over light charged Higgs mass range 120-180 GeV in collisions of proton and proton at three distinct centre-of-mass energies i.e. $\sqrt{s}$ = 8 TeV, $\sqrt{s}$ = 13 TeV and $\sqrt{s}$ = 14 TeV with three different distinguishable colors.  

% Figure environment removed

Figure \ref{sigmaBRtanB} shows the variation of $\sigma \times$ BR($t\rightarrow H^{+}b$) $\times$ BR($H^{+}\rightarrow W\phi$) as a function of $tan \beta$, ranging between 2-20, while keeping the charged higgs mass $m_H^{\pm}$ fixed at 170 GeV. Three distinct curves represent different centre of mass energies, as described earlier.

%obtained by fixing the charged Higgs mass at 150 GeV and varies a tan$\beta$ parameter from 2 to 20 except fixing at specific value wth same distinguishable color coding for $\sqrt{s}$=8 TeV, $\sqrt{s}$=13 TeV and $\sqrt{s}$=14 TeV.

\section{Conclusion} 
In this study, Type-I 2HDM is used as the theoretical foundation, the scenario selected is similar to the standard model, with the lighter scalar Higgs (h) acting as the SM Higgs boson, in search for a potential discovery channel for the light-$H^{+}$ via bosonic decay channels. Bosonic decays of $H^{+}$, are investigated explicitly, as this mode has not been explored in-depth till now. All $m_{H^{+}}$ values are taken such that they are allowed kinematically.  As it is observed when $m_A$=$m_{H^{+}}$ and $sin(\beta-\alpha)$ fixed at 0.85 then for $m_{H^{+}}$=400 GeV at 9$\leq tan\beta \leq$ 10, BR($H^{+}\rightarrow W^{+}h$) approximately 100\% while at $m_{H^{+}}$=160 GeV maximum of 20\%-30\% of branching ratio observed. 
  
For higher masses of charged Higgs boson, $H^{+}\rightarrow W^{+}A$ decay channel becomes less dominant but it is observed at 160 GeV and 180 GeV of charged Higgs mass becomes most dominant when $m_{A}$ varies between 20 and 120 GeV and $sin(\beta-\alpha)$ varies in the range [-1,1]. When $m_{H^{+}}$ fixed arbitrarily at 170 GeV then it can clearly be observed that BR($H^{+}\rightarrow W^{+}h$)+BR($H^{+}\rightarrow W^{+}h$) becomes up to 50\% observable $m_A$=$m_{h}$=125 GeV and fixing $sin(\beta-\alpha)$ at 0.65. The highest value of $\sigma$ is observed at $m_{H^{+}}$=150 GeV at 8 TeV, 13 TeV and 14 TeV of $\sqrt{s}$ in $pp$-collisions. So by fixing $m_{H^{+}}$=150 GeV, the highest value of $\sigma$ is observed at $tan\beta$=2 and lowest at $tan\beta$=20 for all three different values of $\sqrt{s}$. It is also observed that with an increase of $tan\beta$, the value of cross-section decreases for $H^{+}\rightarrow W\phi$ (where $\phi$=h or A). Several scans are performed for observing bosonic decay modes of $H^{\pm}$ from $[200-1000]~GeV$ mass range and then confining the range 100-200 GeV and in the next step fixing it at 170 GeV for both decay modes ($W^{+}h$, $W^{+}A$) by applying theoretical bounds and experimental bounds from most recent Higgs searches. It can be seen that the possibility exists for light $H^{+}$ to decay via $H^{+}\rightarrow W^{+}h$ channel or $H^{+}\rightarrow W^{+}A$ channel (bosonic decay modes). These scans are for observing and calculating branching ratios and cross-sections for different bosonic decay modes and making a comparison with top-bottom decay modes. This study shows that light charged Higgs boson is possibly observable via bosonic decay channels and it leads the experimentalists to an alternative possible discovery channel for light charged Higgs boson. This study provides a specifically good way to examine beyond standard model(BSM) Higgs bosons as well as validate the sustainability and feasibility of the 2HDM in the selected parameter space.\\
\textbf{Availability of data and materials}\\
Data sharing not applicable to this article as no datasets were generated or analysed during the current study.\\

 \bibliographystyle{ieeetr}
 \begin{thebibliography}{99}

\bibitem{decaysupression1}
M. Spira. Higgs Boson Production and Decay at Hadron Colliders. Progress in Particle and Nuclear Physics Vol 95 (2017) pg 98 - 159. https://doi.org/10.1016%2Fj.ppnp.2017.04.001
 
\bibitem{decaysupression}
The CMS collaboration., Khachatryan, V., Sirunyan, A.M. et al. Search for a charged Higgs boson in pp collisions at $\sqrt{s}=$8 TeV. J. High Energ. Phys. 2015, 18 (2015). https://doi.org/10.1007/JHEP11(2015)018

\bibitem{2017}
A.~Arhrib, R.~Benbrik, and S.~Moretti.
\newblock Bosonic decays of charged Higgs bosons in a 2HDM Type-I.
\newblock {The European Physical Journal C}, 77\penalty0 (9), Sep 2017.
\newblock ISSN 1434-6052.
\newblock doi{10.1140/epjc/s10052-017-5197-7}.
\newblock URL \url{http://dx.doi.org/10.1140/epjc/s10052-017-5197-7}.


\bibitem{alves2017charged}
Daniele~SM Alves, Sonia El~Hedri, Anna~Maria Taki, and Neal Weiner.
\newblock Charged Higgs signals in $t \bar t H$ searches.
\newblock {Physical Review D}, 96\penalty0 (7):\penalty0 075032, 2017.


\bibitem{coleppa2014charged}
Baradhwaj Coleppa, Felix Kling, and Shufang Su.
\newblock Charged Higgs search via aw$\pm$/hw$\pm$channel.
\newblock {Journal of High Energy Physics}, 2014\penalty0 (12):\penalty0
  148, 2014.

\bibitem{kling2015light}
Felix Kling, Adarsh Pyarelal, and Shufang Su.
\newblock Light charged Higgs bosons to aw/hw via top decay.
\newblock {Journal of High Energy Physics}, 2015\penalty0 (11):\penalty0
  1--21, 2015.

\bibitem{2hdmc}
David Eriksson, Johan Rathsman, Oscar Stål. 2HDMC - Two-Higgs-Doublet Model Calculator. Comput.Phys.Commun.181:189-205,2010. https://doi.org/10.48550/arXiv.0902.0851


\bibitem{higgsbounds}
P. Bechtle, O. Brein, S. Heinemeyer, G. Weiglein, K. E. Williams.
HiggsBounds: Confronting Arbitrary Higgs Sectors with Exclusion Bounds from LEP and the Tevatron. Comput.Phys.Commun.181:138-167,2010. https://doi.org/10.48550/arXiv.0811.4169
\bibitem{higgssignals}
P. Bechtle, S. Heinemeyer, O. Stål, T. Stefaniak, G. Weiglein.
HiggsSignals: Confronting arbitrary Higgs sectors with measurements at the Tevatron and the LHC. Eur.Phys.J. C74 (2014) 2711. https://doi.org/10.48550/arXiv.1305.1933
\bibitem{scanners}
M. Mühlleitner, M. O. P. Sampaio, R. Santos, Jonas Wittbrodt.
ScannerS: parameter scans in extended scalar sectors. Eur. Phys. J. C (2022) 82:198. https://doi.org/10.1140/epjc/s10052-022-10139-w

\bibitem{2HDMref}
G. C. Branco, P. M. Ferreira, L. Lavoura, M. N. Rebelo, Marc Sher, and Joao P. Silva. Theory and phenomenology of two-Higgs-doublet models. Phys. Rept., 516:1-102, 2012, 1106.0034. https://doi.org/10.48550/arXiv.1106.0034
%Rikard Enberg,a William Klemm,ab Stefano Morettic and Shoaib Munir, Signatures of the Type-I 2HDM at the LHCarXiv:1812.08623v2 [hep-ph] 30 Dec 2018

\bibitem{mahmoudi2008superiso}
F.~Mahmoudi.
\newblock SuperIso v3.0: A program for calculating flavor physics observables in 2HDM and supersymmetry. 
\newblock {Computer Physics Communications}, 180 :1579-1613,2009
\newblock doi{10.1016/j.cpc.2009.02.017}.
\newblock URL \url{https://doi.org/10.1016/j.cpc.2009.02.017}.

\bibitem{asner2010averages}
D~Asner, Sw~Banerjee, R~Bernhard, S~Blyth, A~Bozek, C~Bozzi, DG~Cassel,
  G~Cavoto, G~Cibinetto, J~Coleman, et~al.
\newblock Averages of b-hadron, c-hadron, and tau-lepton properties.
\newblock {arXiv preprint arXiv:1010.1589}, 2010.

\bibitem{mahmoudi2010flavor}
Farvah Mahmoudi and Oscar St{\aa}l.
\newblock Flavor constraints on two-Higgs-doublet models with general diagonal yukawa couplings.
\newblock {Physical Review D}, 81\penalty0 (3):\penalty0 035016, 2010.

\bibitem{800gev2hdm}
 M. Misiak, Abdur Rehman, and Matthias Steinhauser.
\newblock Towards $\bar B \longrightarrow X_s \gamma$ at the NNLO in QCD without interpolation in $m_c$. J. High Energ. Phys. 2020, 175 (2020), 2002.01548.
% On charm-mass dependent NNLO corrections to $\bar B \longrightarrow X_s \gamma$. 	arXiv:2002.03021
%%%%%%%%%%%%%%%%
\bibitem{mehmodi-2008}
David Eriksson and  Farvah Mahmoudi and  Oscar Stål,
Charged Higgs bosons in minimal supersymmetry: updated constraints and experimental prospects. Journal of High Energy Physics, 11 (2008) 035. https://dx.doi.org/10.1088/1126-6708/2008/11/035


\bibitem{Borzumati:1998nx}
Francesca Borzumati and Christoph Greub.
\newblock {Two Higgs doublet model predictions for anti-B ---\ensuremath{>}
  X(s) gamma in NLO QCD: Addendum}.
\newblock {Phys. Rev. D}, 59:\penalty0 057501, 1999.
\newblock doi{10.1103/PhysRevD.59.057501}.


\bibitem{PhysRevD.82.114504}
C.~T.~H. Davies, C.~McNeile, E.~Follana, G.~P. Lepage, H.~Na, and
  J.~Shigemitsu.
\newblock Update: Precision ${D}_{s}$ decay constant from full lattice qcd using very fine lattices.
\newblock {Phys. Rev. D}, 82:\penalty0 114504, Dec 2010.
\newblock doi{10.1103/PhysRevD.82.114504}.
\newblock URL \url{https://link.aps.org/doi/10.1103/PhysRevD.82.114504}.

\bibitem{theheavyflavoraveraginggroup2011averages}
The Heavy Flavor~Averaging Group, D.~Asner, Sw. Banerjee, R.~Bernhard,
  S.~Blyth, A.~Bozek, C.~Bozzi, D.~G. Cassel, G.~Cavoto, G.~Cibinetto,
  J.~Coleman, W.~Dungel, T.~J. Gershon, L.~Gibbons, B.~Golob, R.~Harr,
  K.~Hayasaka, H.~Hayashii, C.~J. Lin, D.~Lopes Pegna, R.~Louvot, A.~Lusiani,
  V.~Luth, B.~Meadows, S.~Nishida, D.~Pedrini, M.~Purohit, M.~Rama, M.~Roney,
  O.~Schneider, C.~Schwanda, A.~J. Schwartz, B.~Shwartz, J.~G. Smith,
  R.~Tesarek, D.~Tonelli, K.~Trabelsi, P.~Urquijo, and R.~Van Kooten.
\newblock Averages of b-hadron, c-hadron, and tau-lepton properties, 2011.

\bibitem{logan2000bs}
Heather~E Logan and Ulrich Nierste.
\newblock $B_{s}, d \rightarrow l^{+} l^{-}$ in a two$-$Higgs-doublet model.
\newblock {Nuclear Physics B}, 586\penalty0 (1-2):\penalty0 39--55, 2000.


\bibitem{aaij2012strong} R~Aaij, C~Abellan Beteta, A~Adametz, B~Adeva, M~Adinolfi, C~Adrover,   A~Affolder, Ziad Ajaltouni, J~Albrecht, F~Alessio, et~al. \newblock Strong constraints on the rare decays $B_S^0 \longrightarrow \mu^+ \mu^-$ and $B^0 \longrightarrow \mu^+ \mu^-$. \newblock {Physical review letters}, 108\penalty0 (23):\penalty0 231801,   2012.

\bibitem{lhcbatlas}
CMS LHCb.
\newblock Atlas collaborations, lhcb-conf-2012-017.

\bibitem{hdecay}
A. Djouadi, J. Kalinowski, M. Spira. HDECAY: A Program for Higgs boson decays in the standard model and its supersymmetric extension. Comput.Phys.Commun.108:56-74,1998.

\bibitem{sushi}
R. V. Harlander, S. Liebler, H. Mantler. SusHi: A program for the calculation of Higgs production in gluon fusion and bottom-quark annihilation in the Standard Model and the MSSM. Comput.Phys.Commun.184:1605-1617, 2013. https://doi.org/10.1016/j.cpc.2013.02.006.




%\bibitem{PhysRevD.48.2342}
%Wei-Shu Hou.
%\newblock Enhanced charged Higgs boson effects in
 % ${B}^{\ensuremath{-}}\ensuremath{\rightarrow}\ensuremath{\tau}\overline{\ensuremath{\nu}},
 % \ensuremath{\mu}\overline{\ensuremath{\nu}}$ and
 % $b\ensuremath{\rightarrow}\ensuremath{\tau}\overline{\ensuremath{\nu}}+x$.
%\newblock {Phys. Rev. D}, 48:\penalty0 2342--2344, Sep 1993.
%\newblock \doi{10.1103/PhysRevD.48.2342}.
%\newblock URL \url{https://link.aps.org/doi/10.1103/PhysRevD.48.2342}.





\end{thebibliography}
\end{document}