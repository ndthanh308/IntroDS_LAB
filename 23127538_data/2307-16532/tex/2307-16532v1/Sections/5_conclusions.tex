\section{Conclusion and Future Work}
In this work, we have proposed a novel method for radar raw data fusion with other sensors in a BEV space. Our proposed EchoFusion is concise and effective, which outperforms previous work by a significant margin. We are the first to demonstrate the potential of radar as a low-cost alternative for LiDAR in autonomous driving systems through thorough analyses and experiments. 

This work is only a starting point to study how raw radar data can be exploited. However, many attempts are limited by the available datasets. We urge a large-scale and high-quality dataset. However, the acquisition of high-quality multi-modality data with accurate annotation needs great effort and deliberate design for clock synchronization and high storage demand. We will try to build the dataset to facilitate further research.

\textbf{Societal Impacts} Our method can be deployed in the autonomous driving system. Performance loss caused by improper usage may increase security risks.
%In conclusion, our proposed method of incorporating raw radar data with image data offers a promising solution for improving the accuracy and efficiency of autonomous driving systems. By skipping the traditional radar signal processing pipeline and using specifically designed attention mechanisms for sensors with different properties, we are able to release the power of radar data and achieve promising performance that is only slightly weaker than pure LiDAR. In addition to the benefits of our proposed method, it is worth noting that the limited size and inaccurate annotation of existing datasets are still in the way of exploration on the radar. We hope that our work will inspire further research on the use of radar data, as well as efforts to improve the quality and quantity of available datasets. 