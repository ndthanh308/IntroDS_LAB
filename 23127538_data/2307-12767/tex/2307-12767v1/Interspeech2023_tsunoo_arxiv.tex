\documentclass{INTERSPEECH2023}

% 2023-01-06 modified by Simon King (Simon.King@ed.ac.uk)  

% **************************************
% *    DOUBLE-BLIND REVIEW SETTINGS    *
% **************************************
% Comment out \interspeechcameraready when submitting the 
% paper for review.
% If your paper is accepted, uncomment this to produce the
%  'camera ready' version to submit for publication.
\interspeechcameraready 


% **************************************
% *                                    *
% *      STOP !   DO NOT DELETE !      *
% *          READ THIS FIRST           *
% *                                    *
% * This template also includes        *
% * important INSTRUCTIONS that you    *
% * must follow when preparing your    *
% * paper. Read it BEFORE replacing    *
% * the content with your own work.    *
% **************************************

\def\z{{\mathbf z}}
\def\h{{\mathbf h}}
\def\Z{\mathcal{Z}}
\def\Y{\mathcal{Y}}
\def\H{\mathcal{H}}
\def\FSM{\mathrm{FSM}}
\def\FSync{\mathrm{FSync}}
\def\LSync{\mathrm{LSync}}
\def\LSM{\mathrm{LSM}}
\def\AED{\mathrm{AED}}
\def\LM{\mathrm{LM}}
\def\CTC{\mathrm{CTC}}
\def\Enc{\mathrm{Enc}}
\def\Dec{\mathrm{Dec}}
\def\Prefix{\mathrm{Pfx}}
\def\Score{\mathrm{Score}}
\def\ScoreMod{\mathrm{ScoreMod}}
\def\AncestorPruning{\mathrm{AncestorPruning}}
\def\aed{\mathrm{aed}}
\def\lm{\mathrm{lm}}
\def\fs{\mathrm{F}}
\def\ls{\mathrm{L}}
\def\fl{\mathrm{FL}}
\def\len{\mathrm{len}}
\def\ctc{\mathrm{ctc}}
\def\att{\mathrm{att}}
\def\best{\mathrm{best}}
\def\top{\mathrm{top}}
\def\b{\mathrm{(b)}}
\def\n{\mathrm{(n)}}
\def\T{\mathrm{T}}
\def\S{\mathrm{S}}
\def\L{\mathcal{L}}
\def\D{\mathcal{D}}
\def\F{\mathcal{F}}
\def\V{\mathcal{V}}

\title{Integration of Frame- and Label-synchronous Beam Search for \\Streaming Encoder--decoder Speech Recognition}
\name{Emiru Tsunoo$^1$, Hayato Futami$^1$, Yosuke Kashiwagi$^1$, Siddhant Arora$^2$, Shinji Watanabe$^2$}
%The maximum number of authors in the author list is 20. If the number of contributing authors is more than this, they should be listed in a footnote or the acknowledgement section.
\address{
  $^1$Sony Group Corporation, Japan\\
  $^2$Carnegie Mellon University, U.S.A.}
\email{emiru.tsunoo@sony.com}

\usepackage[
backend=biber,
style=ieee,
% more than 5 authors will be "et al."
citestyle=numeric-comp,
%maxbibnames=3,
maxbibnames=10,
maxcitenames=3,
% omit non-useful information
doi=false,isbn=false,url=false,eprint=false
]{biblatex}
\usepackage{comment}
\usepackage{algorithm}
% \usepackage{algorithmic}
\usepackage{algpseudocode}
\renewcommand{\algorithmicrequire}{\textbf{Input:}}
\renewcommand{\algorithmicensure}{\textbf{Output:}}
\algnewcommand{\algorithmicand}{\textbf{ and }}
\algnewcommand{\algorithmicor}{\textbf{ or }}
\algnewcommand{\OR}{\algorithmicor}
\algnewcommand{\AND}{\algorithmicand}
\newcommand{\argmax}{\mathop{\rm arg~max}\limits}


\addbibresource{mybib.bib}
\defbibheading{bibliography}[\refname]{}
\renewcommand*{\bibfont}{\footnotesize}
\DeclareSourcemap{
	\maps[datatype=bibtex, overwrite=true]{
		\map{
		    % regex rules for the consistent conference titles
			\step[fieldsource=booktitle,
			match=\regexp{.*Interspeech.*},
			replace={Proc. Interspeech}]
			\step[fieldsource=journal,
			match=\regexp{.*INTERSPEECH.*},
			replace={Proc. Interspeech}]
			\step[fieldsource=booktitle,
			match=\regexp{.*ICASSP.*},
			replace={Proc. ICASSP}]
			\step[fieldsource=booktitle,
			match=\regexp{.*icassp_inpress.*},
			replace={Proc. ICASSP (in press)}]
			\step[fieldsource=booktitle,
			match=\regexp{.*Acoustics,.*Speech.*and.*Signal.*Processing.*},
			replace={Proc. ICASSP}]
			\step[fieldsource=booktitle,
			match=\regexp{.*International.*Conference.*on.*Learning.*Representations.*},
			replace={Proc. ICLR}]
			\step[fieldsource=booktitle,
			match=\regexp{.*International.*Conference.*on.*Computational.*Linguistics.*},
			replace={Proc. COLING}]
			\step[fieldsource=booktitle,
			match=\regexp{.*SIGdial.*Meeting.*on.*Discourse.*and.*Dialogue.*},
			replace={Proc. SIGDIAL}]
			\step[fieldsource=booktitle,
			match=\regexp{.*International.*Conference.*on.*Machine.*Learning.*},
			replace={Proc. ICML}]
			\step[fieldsource=booktitle,
			match=\regexp{.*North.*American.*Chapter.*of.*the.*Association.*for.*Computational.*Linguistics:.*Human.*Language.*Technologies.*},
			replace={Proc. NAACL}]
			\step[fieldsource=booktitle,
			match=\regexp{.*Empirical.*Methods.*in.*Natural.*Language.*Processing.*},
			replace={Proc. EMNLP}]
			\step[fieldsource=booktitle,
			match=\regexp{.*Association.*for.*Computational.*Linguistics.*},
			replace={Proc. ACL}]
			\step[fieldsource=booktitle,
			match=\regexp{.*Automatic.*Speech.*Recognition.*and.*Understanding.*},
			replace={Proc. ASRU}]
			\step[fieldsource=booktitle,
			match=\regexp{.*Spoken.*Language.*Technology.*},
			replace={Proc. SLT}]
			\step[fieldsource=booktitle,
			match=\regexp{.*Speech.*Synthesis.*Workshop.*},
			replace={Proc. SSW}]
			\step[fieldsource=booktitle,
			match=\regexp{.*workshop.*on.*speech.*synthesis.*},
			replace={Proc. SSW}]
			\step[fieldsource=booktitle,
			match=\regexp{.*Advances.*in.*neural.*information.*processing.*},
			replace={Proc. NeurIPS}]
			\step[fieldsource=booktitle,
			match=\regexp{.*Advances.*in.*Neural.*Information.*Processing.*},
			replace={Proc. NeurIPS}]
			\step[fieldsource=booktitle,
			match=\regexp{.*Workshop.*on.* Applications.* of.* Signal.*Processing.*to.*Audio.*and.*Acoustics.*},
			replace={Proc. WASPAA}]
			% omit non-useful information not supported in the usepackage options
			\step[fieldsource=publisher,
			match=\regexp{.+},
			replace={{}}]
			\step[fieldsource=month,
			match=\regexp{.+},
			replace={{}}]
			\step[fieldsource=location,
			match=\regexp{.+},
			replace={{}}]
			\step[fieldsource=address,
			match=\regexp{.+},
			replace={{}}]
			\step[fieldsource=organization,
			match=\regexp{.+},
			replace={{}}]
		}
	}
}
\begin{document}

\maketitle
 
\begin{abstract}
% 1000 characters. ASCII characters only. No citations.
Although frame-based models, such as CTC and transducers, have an affinity for streaming automatic speech recognition, their decoding uses no future knowledge, which could lead to incorrect pruning. Conversely, label-based attention encoder--decoder mitigates this issue using soft attention to the input, while it tends to overestimate labels biased towards its training domain, unlike CTC. We exploit these complementary attributes and propose to integrate the frame- and label-synchronous (F-/L-Sync) decoding alternately performed within a single beam-search scheme. F-Sync decoding leads the decoding for block-wise processing, while L-Sync decoding provides the prioritized hypotheses using look-ahead future frames within a block. We maintain the hypotheses from both decoding methods to perform effective pruning. Experiments demonstrate that the proposed search algorithm achieves lower error rates compared to the other search methods, while being robust against out-of-domain situations.
%Although frame-based models, such as CTC and transducers, have an affinity for streaming automatic speech recognition, their decoding uses no future knowledge, which could lead to incorrect pruning. Conversely, label-based attention encoder--decoder mitigates this issue using soft attention to the input, while it has a linguistic bias towards its training domain, unlike CTC. We exploit these complementary attributes and propose to integrate the frame- and label-synchronous (F-/L-Sync) decoding alternately performed within a single beam-search scheme. F-Sync decoding leads the decoding for block-wise streaming processing, while L-Sync decoding provides the prioritized hypotheses using look-ahead future frames within a block. We maintain the hypotheses from both decoding methods to perform effective pruning. Experiments demonstrate that the proposed search algorithm achieves lower error rates compared to the other search methods, while being robust against out-of-domain situations.

%Automatic speech recognition bridges two modalities, from speech, which is modeled frame-by-frame, to sequential labels. Although frame-based models, such as CTC and transducers, have an affinity for streaming processing, their decoding compares the scores without future knowledge, which could lead an incorrect pruning. Conversely, label-based models using soft attention to all the input, such as attention-based decoder, mitigate the above issue.  However, they typically have linguistic bias towards its training domain unlike CTC, and label bias problem where incorrectly estimated labels with strong bias cannot be easily recovered. In this study, we exploit these complementary attributes and propose to integrate both frame- and label-synchronous (F-/L-Sync) decoding within a single beam-search scheme. For streaming processing, F-Sync decoding expands the hypotheses during the beam search, while L-Sync decoding provide the prioritized hypotheses based on the previous hypothesis set. The L-Sync hypotheses have priority for survival in pruning, until all the scores of expanded successors exceed the score of them. Experiments demonstrate that the proposed search algorithm performs effective pruning and achieve lower error rates compared to the other search methods, while being robust against out-of-domain situations.
%: 1) the probabilistic scores are compared with partial input, and 2) label probabilities of various lengths are fused. Although label-based models, such as attention-based decoder, mitigate the above issues, they typically have linguistic bias towards its training domain unlike CTC. In this study, we exploit these complementary attributes and propose to integrate both hypotheses from frame- and label-synchronous decoding within a single beam-search scheme. (The hypotheses from the latter uses information of the entire input; thus they perform as they push forward the score for better pruning.) Experiments demonstrate that the proposed search algorithm has an advantage in accuracy compared to the other search methods, while being robust against out-of-domain situations.
%(Specific results)


%Attention-based encoder--decoder (AED) speech recognition models are powerful architectures with decoding in auto-regressive manner.
%However, the AED architecture has strong dependency on the training domain.
%On the other hand, Connectionist Temporal Classification (CTC) is context-independent module, which is a natural choice for streaming style system, but has difficulty to fuse with a language model (LM).
%While those two modules are combined in the beam search of a joint model, We propose yet another joint search algorithm to mitigate this domain dependency problem.
%The new beam search is based on CTC decoding, which is called frame-synchronous search.
%We first investigate better approaches to incorporate with LM.
%Subsequently, on top of that, the AED provides a guide from the aspect of label-synchronous search.
%Because those search methods are complementary each other, the combined beam search is suitable for streaming system and robust to the domain mismatch.
%In experiments, we confirm that the proposed beam search performed better especially in out-of-domain scenarios.
%(Specific results)
\end{abstract}
\noindent\textbf{Index Terms}: speech recognition, beam search, attention-based encoder--decoder, CTC




\section{Introduction}
Streaming style automatic speech recognition (ASR) is essential for better user experiences.
Among end-to-end ASR models, connectionist temporal classification (CTC) \cite{graves06, graves14, 
miao15, amodei16} and transducers \cite{graves13rnnt,gulati2020,zhang2020transformer} successfully model the temporal phenomenon of speech in frame-by-frame computation.
These models are referred to as frame-synchronous (F-Sync) models and they have an affinity for streaming processing \cite{dong19, yu2021fastemit, shi2021emformer}.
Conversely, the output of ASR is a label-base, which is suitable for modeling with label-synchronous (L-Sync) models, such as attention-based decoders (AttDecs) \cite{chorowski15, chan16} and language models (LMs) \cite{mikolov10, irie2019language}.
L-Sync models estimate labels in an autoregressive manner with the given context of the previously estimated output.
Generally, L-Sync models have a strong ability to model label sequences, and they are used to improve the ASR performance through fusion or rescoring \cite{chorowski17_interspeech, kannan2018analysis, sainath2019two, zhou2021phoneme}. 

During decoding, F-Sync models expand the hypotheses at each time frame.
F-Sync decoding can be efficiently computed in a beam search by maintaining scores of hypotheses ending with a blank token and non-blank tokens, respectively \cite{graves06}.
% During beam search, an LM or any L-Sync model can be incorporated into the score computation for the hypotheses ending with non-blank tokens.
However, the hypotheses are pruned at each time frame by using only partial information from the input speech.
% Although \cite{saon2020alignment} attempt to mitigate the problem, this limitation sometimes causes unreliable pruning.
This limitation sometimes causes unreliable pruning.
%In addition, the hypotheses in the beam have different lengths, for which the L-Sync model multiplies different number of probabilities; thus, they are unfair for comparison during pruning because the ranges of score are different.
Conversely, during L-Sync decoding, the hypotheses are extended token-by-token by making soft attention to all the input and the previous output sequence.
Thus, this approach has an advantage in pruning over F-Sync decoding without future information.
% , which only uses partial information.
%However, it is well-known that the decoder network is linguistically biased toward the training domain, which has recently been mitigated by estimating internal LMs \cite{meng2021ilme,,zeineldeen2021investigating}\footnote{This issue also happens in transducers.}.
%Additionally, 
However, it has a label bias problem \cite{lafferty2001conditional, bengio2015, murray2018correcting}, where once the model overestimates a label, it is difficult to downgrade it by the suffix distribution.
% Furthermore, likelihoods for stop-token degenerate for unexpectedly long outputs, which ends up with falsely generating long sequence \cite{murray2018correcting}.
Furthermore, it is difficult to use the L-Sync models in streaming ASR, where block processing is used \cite{tsunoo19, shi2021emformer} and their decoding is generally complicated \cite{moritz19, li2021head, tsunoo2022run}.
%For streaming processing, the decoder must detect the end-point in each processing blocks, which complicates the system \cite{moritz19, li2021head, tsunoo2022run}.
%Joint models have  aspect of both L-Sync and F-Sync because they are trained using the multi-task learning of the AttDec and CTC \cite{watanabe17,karita19}.
%However, the decoding is led by the AttDec, which is a L-Sync model, and CTC only predicts the scores for the top-$N$ hypotheses from the AttDec.

Both F-Sync and L-Sync decoding are complementary.
Several studies have attempted to combine F-Sync and L-Sync decoding.
Watanabe {\it et al.} proposed the joint model, where AttDec and CTC are jointly trained, and CTC rescores the hypotheses generated by AttDec during inference.
Sainath {\it et al.} proposed two-pass decoding \cite{sainath2019two}, in which the AttDec rescores $N$-best list of the transducer's hypotheses.
Li {\it et al.} combined separate F-Sync and L-Sync models where the former produces a lattice and the latter rescores it \cite{li2021combining}.
Yan {\it et al.} introduce AttDec score to F-Sync decoding of CTC in machine translation and speech translation tasks \cite{yan2022ctc}.
Additionally, \cite{dong2020comparison, zhou2021equivalence} experimentally compared F-Sync and L-Sync decoding.
However, in those studies, one is merely used for rescoring the hypotheses generated by the other beforehand, and none simultaneously considers the hypotheses from the individual decoding methods.

This study exploits complementary attributes of the F-Sync and L-Sync decoding, and proposes integrating both in a single beam search scheme.
%to mitigate the problems.
To take advantage of F-Sync decoding, which is easy to incorporate with block-wise streaming ASR, the proposed beam search primarily runs in an F-Sync manner.
%To overcome the unreliable partial F-Sync score comparison in pruning, L-Sync decoding provides the hypotheses with future knowledge that are also maintained in the beam.
%The L-Sync hypotheses have priority for survival in pruning, until all the scores of their expanded successors exceed the score of them. 
To mitigate the issue of unreliable partial F-Sync score comparison, we employ the L-Sync beam search based on the shortest token-length prefixes of the hypotheses. 
The selected hypotheses by L-Sync are then preserved in the subsequent F-Sync pruning steps and adjusted as the hypotheses expand, ensuring that the most promising ones are maintained in the proposed beam search.
Experiments demonstrate that the proposed search algorithm performs effectively pruning in the frame--label grid search, and achieve lower error rates compared to the other search methods in English and Japanese datasets.
Evaluation using various domains in each language indicates that the proposed method is robust against out-of-domain situations.

%In this study, we propose a new beam search algorithm that integrates both F-Sync and L-Sync decoding to mitigate the problems in both methods.
%To take advantage of F-Sync decoding, which is easy to incorporate with block-wise streaming ASR, the proposed beam search runs in an F-Sync manner.
%To overcome the unreliable partial F-Sync score comparison in pruning, we maintain not only hypotheses from F-Sync decoding but also the ones from L-Sync decoding, which make soft attention to all the time frames in the processed blocks.
%As the scores of hypotheses from L-Sync decoding considers information beyond the current time step in F-Sync decoding, they work similarly as a look-ahead LM \cite{ortmanns1996language}. 
%(Because L-Sync decoding generally align output sequence lengths, we apply L-Sync scores for the equal length of prefixes of all the hypotheses to fairly compare scores in pruning, which also mitigate the length difference problem in F-Sync decoding.)a
%Experiments demonstrate that the proposed search algorithm has an advantage in performance comparing to the F-Sync search, whereas it is more robust against out-of-domain situations than the L-Sync search.

\begin{comment}
End-to-end (E2E) automatic speech recognition (ASR) has been progressed to be state-of-the-art with a lot of various architectures, including connectionist temporal classification (CTC) \cite{graves06, %graves14, 
miao15, amodei16}, attention-based encoder--decoder (AED) models \cite{chorowski15, chan16}, and transducers \cite{
graves13rnnt,gulati2020,zhang2020transformer}.
In particular, joint models of CTC and AED achieve high performance \cite{watanabe17,karita19}.

AEDs and transducers infer in autoregressive manner, which depends on previous output sequence.
Thus, these models are also strongly depended on the training domains.
These models do not result in sufficient performance in the mismatched domain, as reported in 
\cite{delrio21_interspeech}.
This dependency is recently regarded as an internal language model (LM), a linguistic bias of E2E ASR.
Although some studies mitigate the problem by subtracting it from the ASR posterior and improve performance in both cross-domain and intra-domain scenarios \cite{meng2021ilme,,zeineldeen2021investigating},
it is still disputable in the extreme case such as command utterances and spontaneous chat where grammatical structures are broken.

It is more desirable for ASR systems to process in streaming because the smaller latency the more friendly to users.
CTC is context-independent module, which is a natural choice for streaming style system \cite{dong19}.
Besides, because of the context-independence characteristic, CTC is relatively robust to the domain mismatch.
The decoding process runs frame-by-frame, which is 
known as frame-synchronous beam search, same as transducers\cite{hannun2014first, hwang2016character}. 
However, it is difficult to fuse with LM without a length regulator term as in \cite{hwang2016character}.

On the other hand, decoding in AED is label-synchronous, which extend tokens one after another, same as LMs.
In a joint model \cite{watanabe17}, both AED and CTC scores are taken into account, but AED takes the lead of the process; thus the decoding is still dependent on the training domains.
For streaming processing, the decoder has to detect the end-point in each processing blocks, which complicates the system \cite{moritz19, li2021head, tsunoo2022run}.

Besides joint decoding in \cite{watanabe17}, some studies combine label-synchronous and frame-synchronous decoding.
Sainath {\it et al.} propose two-pass decoding \cite{sainath2019two}, in which AED model rescores $N$-best list of RNN-T hypotheses.
Li {\it et al.} combines separate frame-synchronous and labe-synchronous models, as the former produces lattice and the latter rescores it \cite{li2021combining}.
Both decoding methods are compared in the context of ASR \cite{dong2020comparison} as well as machine translation \cite{yan2022ctc}.


In this paper, we propose yet another joint search algorithm to mitigate this domain dependency problem.
The new beam search is based on CTC decoding, which is called frame-synchronous search.
We first investigate better approaches to incorporate with LM.
Subsequently, on top of that, the AED provides a guide from the aspect of label-synchronous search.
Because those search methods are complementary each other, the combined beam search is suitable for streaming system and robust to the domain mismatch.
In experiments, we confirm that the proposed beam search performed better especially in out-of-domain scenarios.
(Specific results)
\end{comment}

\section{F-Sync Decoding and L-Sync Decoding for Streaming ASR}
\subsection{F-Sync decoding}
\label{ssec:fsync}
%In addition to the classical hybrid neural network-hidden Markov model, 
F-Sync models include CTC \cite{graves06, graves14, 
miao15, amodei16} and transducer variants \cite{graves13rnnt,gulati2020,zhang2020transformer}, which are suitable for streaming ASR.
In streaming ASR, blockwise processing is widely used \cite{tsunoo19, shi2021emformer}.
Let $T_b$ be the last frame of $b$-th block.
An acoustic representation sequence $\H^{T_b}=\{\h_t|1\leq t \leq T_b\}$ is extracted by an encoder from speech input.
CTC and transducers introduce a blank token, $\phi$, to align $\H^{T_b}$ with a different $L$-length of label sequence $\Y^{L}=\{y_l|1\leq l \leq L\}$.
The F-Sync model estimates $\phi$-augmented tokens $z_t \in \V \cup \{\phi\}$ at time frame $t$, where $\V$ is the vocabulary.
The estimated sequence $\Z^{T'}=\{z_t|1\leq t \leq T'\}$ can be mapped to a label sequence using mapping function $\F:\Z^{T'}\rightarrow \Y^{L}$.
In the case of CTC, $T'=T_b$, and $T'=T_b+L$ for the transducers, which can emit tokens without consuming frames.
%Typically, input speech is encoded into an intermediate acoustic representation using encoder $\Enc$.
%\begin{align}
%   \h^{T} &= \Enc(\X^{T}) 
%\end{align}
Based on $\h_{t}$ and previous output label $y_{l-1}$, the F-sync model, $\FSM(\cdot)$, calculates a posterior of $z_t$.
\vspace{-0.1cm}
\begin{align}
    q_{\fs}(z_t|\h_{t},y_{l-1}) = \FSM(\h_{t}, y_{l-1})
\end{align}
For CTC, $q_{\fs}$ does not depend on the previous output.
Probability computation of $\Y^{l}$ at $t$ is efficiently performed as 
\vspace{-0.1cm}
\begin{align}
    p_{\fs}(\Y^{l}|\H^{t}) &= \gamma(\Y_\b^{l},t) + \gamma(\Y_\n^{l},t) \label{eq:fsprob} \\
    \gamma(\Y_\b^{l},t) &= \sum_{\F_{\fs}(\Z^{t})=\Y^{l}:z_t=\phi}p_{\fs}(\Y^{l}|\H^{t-1})q(z_t|\h_t,y_{l}) \nonumber \\
    \gamma(\Y_\n^{l},t) &= \sum_{\F_{\fs}(\Z^{t})=\Y^{l}:z_t=y_l}p_{\fs}(\Y^{l-1}|\H^{t-1})q(z_t|\h_t,y_{l-1}), \nonumber 
\end{align}
where $\Y_\b$ is the label sequence ending with $\phi$, and $\Y_\n$ is the other \cite{graves06}.
%by preserving the probabilities of the sequence that end with $\phi$, $\Y_\b$, and the other, $Y_\n$ \cite{graves06}.

F-Sync decoding is performed frame-by-frame.
%It is natural choice for the F-Sync models, and literature shows that the F-Sync decoding performs better than the L-Sync decoding for CTC and transducers \cite{zhou2021equivalence, yan2022ctc}.
%Although CTC and transducer have an option to use either the F-Sync decoding or the L-Sync decoding, the F-Sync decoding is a natural choice, and literature shows that the F-Sync decoding performs better than the L-Sync decoding for CTC and transducers \cite{zhou2021equivalence}.
At each time $t$, probability \eqref{eq:fsprob} is used on a logarithmic scale as F-Sync score $\alpha_{\fs}$.
\vspace{-0.2cm}
\begin{align}
    \alpha_{\fs}(\Y^{l},t) = \log p_{\fs}(\Y^{l}|\H^{t}) \label{eq:fsscore}
\end{align}
The hypotheses are copied by blank transition or expanded at each time step with $\FSync: \Y^{l} \rightarrow \Y^{l}, \Y^{l+1}$.
In the beam search with a fixed beam size $B$, the $\top{B}(\cdot)$ function returns a set of the top $B$ hypotheses at step $t$, denoted as $\Omega_{\fs,t}$:
\vspace{-0.1cm}
\begin{align}
%    \Omega_{\fs,t} = \FSync(B,\alpha(\Y,t)) = \{ \Y_{\fs}\in \top{B}(\alpha(\Y,t)) \} \label{eq:fset}
    \Omega_{\fs,t} = \top{B}(\alpha_{\fs}(\Y,t) | \Y \in \FSync(\Y))  \label{eq:fset}
\end{align}
Since the F-Sync decoding has no knowledge of future inputs even when the block processing has certain look-ahead frames \cite{tsunoo2022run, shi2021emformer}, i.e., the score is conditioned only on $\H^{t}$ at step $t$, it %more likely to 
sometimes incorrectly prune the correct hypothesis.
% equation

%The F-Sync decoding, however, is suitable for streaming ASR.
%In streaming ASR, blockwise processing is widely used \cite{tsunoo19, shi2021emformer}.
%The encoder runs in blockwise; thus the intermediate representation for $b$ blocks can be written as $\h^{T_b}$, where $T_b$ is the last frame of $b$-th block.
%Since the F-Sync requires only partial input $\H^t$ at step $t\leq T_{b}$, it can be applied directly to the streaming models without modification.

\subsection{L-Sync models with blockwise streaming processing}
\label{ssec:lsync}
We categorize AttDecs \cite{chorowski15, chan16} and LMs \cite{mikolov10, irie2019language} as L-Sync models.
At each step $i$, the L-Sync models predict the probability distribution of the next label $y_{i}$ using previous output $\Y^{i-1}$ and input $\H^{T_b}$.
L-Sync score $\beta_{\ls}$ is defined as
\vspace{-0.2cm}
\begin{align}
    \beta_{\ls}(\Y^{i}) =& \sum_{j=1}^{i} \log p_{\ls}(y_{j}|\Y^{j-1}, \H^{T_b}).  \label{eq:lsscore}
\end{align}
For LM, $p_{\ls}(y_{j})$ does not depend on input $\H^{T_b}$.
All the hypotheses are simultaneously extended with $\LSync: \Y^{i-1} \rightarrow \Y^{i}$.
Beam search at step $i$ is performed to maintain top $B$ hypotheses as in F-Sync decoding.
\vspace{-0.1cm}
\begin{align}
    %\Omega_{\ls,i} = \LSync(B,\beta(\Y^{i})) = \{ \Y_{\ls}\in \top{B}(\beta(\Y^{i})) \} \label{eq:lset}
    \Omega_{\ls,i} = \top{B}(\beta_{\ls}(\Y^{i}) | \Y^{i} \in \LSync(\Y^{i-1}))  \label{eq:lset}
\end{align}
This contextual dependency enables the L-Sync models to provide a richer representation of label sequences.
However, L-Sync models have the following two drawbacks:
\begin{itemize}
    %\item \underline{Linguistic bias problem}\footnote{Transducers have the same issue.}: L-Sync models tend to be linguistically biased toward the training domain. 
    %They do not result in sufficient performance in the mismatched domain as \cite{delrio21_interspeech} reported. 
    \item \underline{Label bias problem}: Once the model overestimates a label biased toward its training domain, the following expansion cannot easily recover it \cite{lafferty2001conditional, bengio2015, murray2018correcting}.
    \item \underline{Endpoint problem}: For streaming systems, the AttDec needs to detect an endpoint, which is the point to stop decoding with limited $\H^{T_b}$.
    This requires additional mechanisms to detect \cite{moritz19, li2021head, tsunoo2022run}.
    Once the decoding step $i$ passes the endpoint, the AttDec tends to emit unreliable tokens; thus it is better for the AttDec to proceed decoding conservatively in the streaming systems \cite{tsunoo2021slt}.
\end{itemize}


\subsection{F-Sync and L-Sync score fusion}
\label{ssec:fusion}
Because F-Sync and L-Sync models are complementary, the aforementioned problems in F-Sync and L-Sync decoding can be alleviated by fusing both scores in pruning.
\subsubsection{Score fusion in F-Sync decoding}
Typically, LMs are fused in the F-Sync decoding.
As in \cite{hannun2014deep, hwang2016character}, the scores (e.g., Eqs.~\eqref{eq:fsscore} and \eqref{eq:lsscore}) are simply combined with weights $\lambda$.
%More generally, the combined score is defined as
In the case of CTC, the combined score is defined as 
\vspace{-0.3cm}
\begin{align}
    s_{\fs}(\Y^{l}, t) = &\lambda_{\ctc} \alpha_{\ctc}(\Y^{l}, t) + \lambda_{\lm} \beta_{\lm}(\Y^{l}) \nonumber \\ &+ \lambda_{\att} \beta_{\att}(\Y^{l}) + \lambda_{\len}|\Y^{l}| \label{eq:fscore} 
%s_{\fs}(\Y^{l}, t) = &\lambda_{1} \alpha_{\fs}(\Y^{l}, t) + \lambda_{2} \beta_{\lm}(\Y^{l}) + \lambda_{3} \beta_{\att}(\Y^{l}) + \lambda_{4}|\Y^{l}| \label{eq:fscore} 
\end{align}
The last term in \eqref{eq:fscore} is a label reword, which uses the length of hypothesis $\Y^{l}$.
This mitigates unfair score comparison of different length in the beam, which, however, requires careful tuning.
%In the F-Sync beam search, two points make the decoding unstable.
%First, the first term in \eqref{eq:score} is based only on limited information of $\Z^{t}$.
%First, the first term in \eqref{eq:score} depends only on partial feature $h_t$.
%(equation)
%Second, the top $B$ hypotheses are often of different lengths.
%Thus, the L-Sync scores \eqref{eq:lsscore} are not within the same range (different $i$'s), which is not reliable for pruning.
%Although the insertion bonus mitigate the second problem, it requires careful tuning.
%We overcome these issues by proposing a new integrated beam search in Sec.~\ref{sec:integ}.

%\subsection{Streaming joint ASR model (rethink structure)}
\subsubsection{Score fusion in L-Sync decoding}
\label{ssec:joint}
%In this study, we follow the streaming encoder--decoder architecture \cite{tsunoo2022run}, which is based on the joint ASR model \cite{watanabe17}.
The joint model \cite{watanabe17} has both CTC and AttDec modules, which are trained in a multi-task learning framework.
Its decoding is based on L-Sync search led by the AttDec.
During the decoding, the score is a combination of the CTC, LM, and AttDec, as follows:
\vspace{-0.3cm}
\begin{align}
    s_{\ls}(\Y^{i},i) =& \lambda_{\ctc} \alpha_{\ctc}(\Y^{i}\dots,T_b) + \lambda_{\lm} \beta_{\lm}(\Y^{i}) \nonumber \\ 
    & + \lambda_{\att}\beta_{\att}(\Y^{i}) + \lambda_{\len}|\Y^{i}| \label{eq:jointscore} \\
    \alpha_{\ctc}(\Y^{i}\dots,T_b) =& \sum_{y_{i+1} \in \V} \alpha_{\ctc}(\Y^{i+1},T_b) \label{eq:prefix}
\end{align}
%where $\lambda_{\lm}$ and $\lambda_{\ctc}$ are tunable weight parameters.
\eqref{eq:prefix} is a CTC prefix score %\footnote{The prefix score can also use transducer scores when the transducer is jointly trained.} 
that accumulates probabilities of all the suffixes of $\Y^{i}$. % using \eqref{eq:fsscore} ($\fs=\ctc$).
The prefix scores are calculated over only the hypotheses generated by the AttDec; thus, it is sometimes difficult to recover incorrect estimation of the AttDec by itself.
Note that when $t=T_b$ and $i=l=L$, F-Sync score \eqref{eq:fsscore} and the prefix score \eqref{eq:prefix} become equivalent; thus F-Sync decoding and L-Sync decoding eventually evaluate the same score, $s_{\fs}(Y^L,T_b) = s_{\ls}(Y^L,L)$.



\begin{comment}
\section{Streaming Encoder--decoder ASR}
\subsection{Joint CTC/Attention Architecture}
A joint architecture of CTC and AED was first introduced by \cite{watanabe17}.
Both modules share the same encoder and the CTC layer estimates independent probability of tokens while the decoder layers estimate probability based on the previous output sequence.
Let the audio features be $\X=\{\x_t|1\leq t \leq T\}$, and the probabilities are calculated by using intermediate representation $\h$, as follows.
\begin{align}
    \h &= \Enc(\X) \\
    p_{\ctc}(z_t|\X) &= \CTC(\h) \\
    p_{\aed}(y_i|\X, y_{0:i-1}) &= \Dec(\h, y_{0:i-1})
\end{align}
Conformer \cite{gulati2020} is a popular choice for $\Enc$, or block processing modules for streaming ASR \cite{tsunoo19, shi2021emformer}.
$\Dec$ is typically Transformer or RNNs that make attention to $\h$ with given context $y_{0:i-1}$.
Both CTC and AED are trained with multi-task learning framework.


\subsection{Label-synchronous beam search of the joint model}

The joint decoding is carried out by combining logarithmic probability scores of both modules.
In the beam search, accumulated forward probability $alpha$s are calculated for CTC and AED, and summed up with weights along with language model probability $p_{\lm}$.
\begin{align}
    \alpha_{\ctc}(\Y_{1:i}, \Z_{1:T}) &= \log\sum_{\Z_{1:T} \in \Y_{1:i}} \prod_{z_{t} \in \Z_{1:T}} p_{\ctc}(z_t|\X) \label{eq:prefix} \\
    \alpha_{\aed}(\Y_{1:i}) &= \sum_{j=0}^{i} \log p_{\aed}(y_j|\X, y_{0:j-1}) \\
    \alpha_{\lm}(\Y_{1:i}) &= \sum_{j=0}^{i} \log p_{\lm}(y_j|y_{0:j-1}) \\
    \Score_{\mathrm{lb}}(\Y_{1:i}) &= \lambda_{\ctc}\alpha_{\ctc}(,\Z_{1:T}) + \lambda_{\aed}\alpha_{\aed} + \lambda_{\lm}\alpha_{\lm}, \label{eq:lbscore}
\end{align}
where Eq~\ref{eq:lbscore} uses abbreviated form.
The decoding process is led by AED.
Eq.~(\ref{eq:prefix}) is called prefix score, which is calculated with only prefixes given by top $N$ hypotheses of AED ($\Y_{1:i}$).
Therefore, fundamentally, CTC cannot recover hypotheses that AED drops. 

\subsection{Frame-synchronous beam search for CTC}
Frame-synchronous decoding accumulate the probability frame-by-frame, and further fusion methods with word-level and character-level LMs are described in \cite{hannun2014first} and \cite{hwang2016character}.
It is carried out by expanding hypotheses frame by frame based on the following score.
\begin{align}
    \Score_{\mathrm{fr}}(t) &= \lambda_{\ctc}\alpha_{\ctc}(,\Z_{1:t}) + \lambda_{\aed}\alpha_{\aed} + \lambda_{\lm}\alpha_{\lm}, \label{eq:frscore}
\end{align}
The difference between label-synchronous is that the CTC score is based on the frames up to current time step $t$.
\end{comment}

\section{Integrated Beam Search of the F-Sync and L-Sync Decoding}
\label{sec:integ}

%\subsection{Hypotheses combination of the F-/L-Sync decoding}
As discussed in Sec.~\ref{ssec:fsync}, the F-Sync beam search is likely to incorrectly prune the hypothesis without considering future look-ahead input, which can be prevented by the AttDec that uses all $T_b$ frames.
Conversely, the label bias problems in the AttDec (Sec.~\ref{ssec:lsync}) can be mitigated by using the F-Sync models.
Therefore, we propose maintaining both the F-Sync and L-Sync hypotheses in extended beam $B'>=B$.
Those hypotheses are generated individually from each decoding, and the step increment of $t$ of the F-Sync decoding and $i$ of the L-Sync decoding are performed alternately in an integrated single search algorithm, namely, FL-Sync beam search.


%\subsubsection{Prefix score fusion}
\subsection{Prefix score fusion}
\label{sssec:prefixfusion}
In the FL-Sync beam search, We choose F-Sync to lead decoding because it is suitable for streaming ASR.
Thus, at each step $t$, the hypotheses are copied and expanded based on F-Sync decoding as in Sec.~\ref{ssec:fsync}.
The score for hypothesis $\Y^{l}$ is extended from \eqref{eq:fscore} and \eqref{eq:jointscore} to include both $i$ and $t$ as 
\vspace{-0.2cm}
\begin{align}
 s_{\fl}(\Y^{l}, i, t) = &\lambda_{\ctc} \alpha_{\ctc}(\Y^{l}, t) + \lambda_{\lm} \beta_{\lm}(\Y^{i}) \nonumber \\&+ \lambda_{\att} \beta_{\att}(\Y^{i}) + \lambda_{\len}|\Y^{i}|. \label{eq:score} 
\end{align}
%F-Sync scores $\alpha_{\fs}$ is proved by CTC, while L-Sync scores, $\beta_{\ls}$, are provided by the AttDec and LM.
While the fist term $\alpha_{\ctc}$ defined in \eqref{eq:fsscore} is computed for $\Y^{l}$, $\beta_{\lm}$ and $\beta_{\att}$ are computed only for its prefix $\Y^{i}$, where $i \leq l$ is the current L-Sync decoding step.
The reason for using $i$ instead of $l$ is that the token prediction of the AttDec becomes unreliable when it exceeds the endpoint in streaming processing, as discussed in Sec.~\ref{ssec:lsync}.
%, the L-Sync decoding must run conservatively.
Therefore, we avoid aggressively computing the scores for the entire $l$ labels for the score fusion and conservatively use $i$ instead.
Thus, the integrated beam search aims to perform efficient pruning on this 2D grid space of $(i,t)$.


Conversely, the L-Sync decoding expand all the $i$-length prefix of the hypotheses in hypothesis set $\Omega_{\fl,t}$, i.e., $\Y^{i} \in \Prefix_{i}(\Omega_{\fl,t})$, where $\Prefix_{i}(\cdot)$ is an $i-$length prefix extraction function.
Therefore, to perform L-Sync decoding, the minimum length of the hypotheses needs to be $\min|\Y^{l} \in \Omega_{\fl,t}| \geq i$.
The L-Sync decoding advances its step $i \rightarrow i+1$ once the minimum length becomes larger than $i$.
%The L-Sync decoding advances its step $i$ once $(i+1)$-length prefix score fusion can be performed to all the hypotheses in $\Omega_{\fl,t}$, i.e., $\min|\Y^{l} \in \Omega_{\fl,t}| = i+1$.
%Since the beam size of the L-Sync decoding is $B$, it expands $B$ prefixes of the top-$B'$ hypotheses in $\Omega_{\fl,t}$.
A graphical explanation of this is provided in Fig.~\ref{fig:online}.
The red circles indicate that the F-Sync beam search expands the hypotheses in each time step.
The L-Sync models add partial scores for the common length to these hypotheses (blue bars).
In the last step of F-Sync decoding, $T_b$, the remaining steps of L-Sync for each hypothesis are consumed, and $\beta_{\ls}(\Y^{l})$ is applied to each hypothesis to determine the best candidate.



% Figure environment removed

% Figure environment removed



\begin{algorithm}[t]       
\hspace{-3cm}
\caption{FL-Sync beam search algorithm.}   
\label{alg:decode}                  
\footnotesize
\begin{algorithmic}[1]     
\Require last frame number of current block $T_b$, acoustic features $\H^{T_b}$, beam width for AttDec $B$, total beam width $B'$
\Ensure $\Omega_{\fl,T_b}$: hypotheses for current block $b$
\State \textbf{Initialize:} $y_0\gets\langle \mathrm{sos}\rangle$, $\Omega_{\fl,0} \gets \{y_{0}\}$, $t\gets 1$, $i\gets1$ %$\hat{\Omega} \gets \{\}$, 
\While{$t < T_b$}
%\State $\Omega_{\fl,t}\gets \{\}$ %, $\Tilde{\Omega}\gets \{\}$
\State $\hat{\Omega} \gets \{\}$
\If{$i < \min(|Y \in \Omega_{\fl,t-1}|)$}
\State $i \gets \min(|Y \in \Omega_{\fl,t-1}|)$
%\If{$m > i$}
%\State $i\gets m$
%\State $\Omega_{\fl, t} \gets \LSync(B,s_{\ls}(\Y^{i},i))$
\State $\hat{\Omega} \gets \LSync(\Prefix_{i-1}(\Omega_{\fl,t-1}))$ \Comment{Prefix L-Sync search}
\State $\hat{\Omega} \gets \top{B}(s_{\ls}(\Y^{i},i)| \Y^{i} \in \hat{\Omega})$ \Comment{Keep top-$B$ hyps}
\EndIf
%\State $\Omega_{\fl, t} \gets \{\Omega_{\fl, t}, \FSync(B'-B,s_{\fl}(\Y^{l},i,t))\}$ 
\State $\hat{\Omega} \gets \hat{\Omega} \cup \FSync(\Omega_{\fl,t-1})$  \Comment{FSync search}
%\EndIf
%\State $\Tilde{\Omega}\gets \LSync(B,s_{\ls}(\Y^{i},i))$ 
\State $\hat{\Omega} \gets \AncestorPruning(\hat{\Omega})$  \Comment{Sec. \ref{sssec:depth}}
\State $\Omega_{\fl, t} \gets$ OnlyPriority$(\hat{\Omega})$ \Comment{Sec. \ref{sssec:prefixfusion}}
\State $\Omega_{\fl, t} \gets \Omega_{\fl, t} \cup \top{(B'-|\Omega_{\fl, t}|)}(s_{\fl}(\Y,i,t)|\Y \in \hat{\Omega})$
%\State $\Theta_{\ctc} \gets$ top-$k(q_{\fs}(z_t|\H^{T_b}))$
%\For{$\Y\in\Omega_{t-1}$}
%\For{$z\in\Theta_{\ctc}$}
%\If{$z=\phi$ \OR $z=y_{|\Y|-1}$}
%\State $\hat{\Y}\gets\Y$ \Comment{copy}
%\Else
%\State $\hat{\Y}\gets\Y \circ z$ \Comment{expand}
%\EndIf
%\State $\Tilde{\Omega} \gets \{\Tilde{\Omega}, \hat{\Y}\}$
%\EndFor
%\EndFor
%\State $\Tilde{\Omega} \gets \mathrm{Merge}(\Tilde{\Omega})$
%\State $\Omega_{t} \gets$ OnlyPriority$(\Tilde{\Omega})$
%\State $\Omega_{t} \gets$ RemainingTop-$2B(\Score(\Y \in \Tilde{\Omega},t, i))$
\State $t \gets t+1$
\EndWhile
%\While {$i < |\Y_{\best}\in\Omega_{T_b}|$}
%\Comment{To determin the best hypothesis}
%\State $i \gets i+1$
%\State $\Omega_{T_b} \gets$ ReSort$(\Score(\Y \in %\Omega_{T_b},T_b,i))$
%\EndWhile
\State \Return $\Omega_{\fl,T_b}$
\end{algorithmic}
\end{algorithm}


%\subsubsection{Prioritized hypotheses from L-Sync decoding}
\subsection{Prioritized hypotheses from L-Sync decoding}
\label{sssec:priority}
The L-Sync decoding generates hypotheses $\Y_{\ls}^{i}$ based on \eqref{eq:jointscore} using all the input, including future look-ahead, which are useful to complement the weakness of F-Sync decoding.
However, the hypotheses are pruned based on \eqref{eq:score}, whose first term, $\alpha_{\ctc}(Y^{l},t)$, uses only partial $t$-frame input unlike \eqref{eq:jointscore}, and the hypotheses can still be incorrectly pruned.
To prevent from dropping L-Sync hypotheses $\Y_{\ls}^{i}$ in the early stage, we prioritize them for survival.
At step $t$, we first preserve those prioritized hypotheses of L-Sync decoding, then F-Sync decoding is performed to fill the remaining hypotheses in the beam $B'$, as
\vspace{-0.1cm}
\begin{align}
    %\Omega_{\fl, t} = &\{\Y_{\ls}^{i}\in {\top}B\left[s_{\ls}(\Y^{i},T_b)\right]\} \nonumber \\ &\cap \{\Y_{\fs}^{l}\in {\top}(B'-B)\left[s_{\fl}(\Y^{1},i,t)\right]\},
    %\Omega_{\fl, t} = &\LSync(B,s_{\ls}(\Y^{i},i)) \cup \FSync(B'-B,s_{\fl}(\Y^{l},i,t)). \label{eq:merge}
    \Omega_{\fl, t} = &\top{B}(s_{\ls}(\Y^{i},i) | \Y^{i} \in \LSync(\Prefix_{i-1}(\Omega_{\fl,t-1}))) \nonumber \\
    & \cup \top{(B'-B)}(s_{\fl}(\Y^{l},i,t) | \Y^{l} \in \FSync(\Omega_{\fl,t-1})) \label{eq:merge} 
\end{align}

\begin{comment}
In \eqref{eq:score}, we selectively use $s_{\fl}(\Y^{i}, i, T_b)$ for L-sync hypothesis $Y_{\ls}^{i}$, where prefix score \eqref{eq:prefix} is used as the F-Sync score $\alpha_{\ctc}$, as follows:
\begin{align}
    \alpha'_{\ctc}(\Y^{l},t) = \left\{ \begin{array}{ll}
    \alpha_{\ctc}(\Y^{l},t) & (\Y^{l}=\Y_{\fs}^{l}) \\
    \alpha_{\ctc}(\Y^{i} \dots ,T_b) & (\Y^{l}=\Y_{\ls}^{i}) 
    \end{array}\right.
\end{align}
The prefix score accumulates probabilities for all possible suffixes; thus $\alpha_{\ctc}(\Y^{i} \dots ,T_b) \geq \max_{t} \alpha_{\ctc}(\Y^{i},t)$ is satisfied.
By using this modified score in the pruning, $Y_{\ls}^{i}$ become most likely to survive.    
\end{comment}

\vspace{-0.3cm}
%\subsubsection{Depth-pruning for the prioritized hypothesis}
\subsection{Ancestor-pruning for the prioritized hypothesis}
\label{sssec:depth}
The priority of the hypothesis $\Y_{\ls}^{i}$ is no longer valid once the L-Sync decoding increase its step $i\rightarrow i+1$, as it generate new set of hypotheses $\Y_{\ls}^{i+1}$.
Further, to prune unnecessary hypotheses of L-Sync decoding, we propose to perform ancestor-pruning, which is similar to the depth-pruning in \cite{hwang2016character}.
As the F-Sync decoding proceed, it expands the hypothesis $\Y^{i} \rightarrow \Y^{i+1}$, as shown in Fig.~\ref{fig:decode}.
The prioritized root hypothesis $\Y^{i}$ is indicated by the red box and the expanded successors $\Y^{i+*}$ are indicated by the blue circles.
We compare the scores of all the successors and the root hypothesis, denoted as $s_{\fl}$ in Fig.~\ref{fig:decode}.
When all the scores of successors become greater than the score of the root, $s_{\fl}(\Y_{\ls}^{i},i,t) < \min(s_{\fl}(\Y^{i+*},i,t))$, we prune away prioritized root hypothesis $\Y_{\ls}^{i}$, because any new successor from the root $\Y_{\ls}^{i}$ can no longer achieve higher score than the current successors due to the $0 \leq q(z_t|\h_t,y_i) \leq 1$ restriction.






\begin{comment}
    
\subsection{Hypotheses combination of the F-Sync and L-Sync decoding with prioritization}
\label{ssec:integ}
As discussed in Sec.~\ref{ssec:fusion}, the F-Sync beam search must prune hypotheses even with the scores calculated with only limited time frames.
Conversely, the L-Sync models, specifically the AttDec, can use the given input information $\H^{T_b}$ as in \eqref{eq:attscore}.
Therefore, we propose both the F-Sync and L-Sync hypotheses in the extended beam: $B$ hypotheses from F-Sync and $B$ from L-Sync decoding.
For the streaming joint ASR model (Sec.~\ref{ssec:joint}), CTC beam search provides $B$ hypotheses and AttDec provide the other $B$.
From L-Sync point of view, this hybrid approach could overcome the domain robustness problem discussed in Sec.~\ref{ssec:fusion}, because F-Sync models are expected to depend less on the training domains.

Fundamentally, the integrated beam search can be led by either F-Sync or L-Sync. 
We choose F-Sync to lead decoding because it is suitable for streaming ASR; thus the hypotheses from both models are expanded at each time step.
The hypotheses from L-Sync decoding are based on \eqref{eq:jointscore}, which considers entire input $\H^{T_b}$.
In pruning, however, all the hypotheses are sorted based on score \eqref{eq:score} for fair comparison, which uses only $t$-frame information.
As the result, the scores for the hypotheses from L-Sync decoding decreases and are dropped off from the beam.
% Therefore, in pruning, score \eqref{eq:modscore} is compared to maintain the top $2B$ hypotheses.
% However, the hypotheses from the L-Sync decoding are based on \eqref{eq:jointscore}, which considers the entire input $\X^{T_b}$.
% When the score \eqref{eq:modscore} of the L-Sync hypothesis is recomputed for score sorting in pruning , the score decrease at current time step $t$, which could be dropped off from the beam.
To exploit the advantage of using the L-Sync hypotheses, we prioritize them for survival regardless of the score in the pruning procedure.
%The prioritization is also reasonable when a strong pretrained model
This prioritization is valid until the next step of L-Sync decoding.


\subsection{Online rescoring for F-Sync beam search}
\label{ssec:rescore}
The second problem is that, there are various length of hypotheses in the F-Sync beam search, which could lead to an unfair score comparison for pruning.
This is also problematic when we combine them with L-Sync decoding, because the L-Sync score is generally proportional to the sequence length of hypotheses.
Instead of carefully tuning the parameters, we propose applying the L-Sync score only for the same partial length of the hypotheses. 
Let $m$ be the minimum length of hypotheses in the beam.
L-Sync scores are applied only to $m$-length prefixes.
The score \eqref{eq:score} can be rewritten as follows:
\begin{align}
    \Score(\Y^{i}, t, m) = &\alpha_{\fs}(\Y^{i}, t) + \lambda_{\ls} \beta_{\ls}(\Y^{m}) + \lambda_{\len}|\Y^{m}| \label{eq:modscore} 
\end{align}
(In particular, transformer models \cite{vaswani17} require quadratic computational cost, and this modification contributes to avoiding unnecessary computation for every restriction-free expansion of hypotheses by F-Sync models.)

In pruning, the score comparison is mostly based on F-Sync score $\alpha_{\fs}(\Y^{i}, t)$, with the aid of partial rescoring $\beta_{\ls}(\Y^{m})$.
In the last step, $T$, the remaining L-sync scores $\beta_{\ls}$ are applied to determine the best hypothesis.
A graphical explanation of this is provided in Fig.~\ref{fig:online}.
The red circles indicate that the F-Sync beam search expands the hypotheses in each time step.
The L-Sync models add partial scores for the common length to these hypotheses (blue bars).
L-Sync decoding proceeds in one-step, once the minimum length of the hypotheses becomes greater than the current step.
We refer to this approach as online rescoring.

\end{comment}

\begin{table*}[t]
  \caption{Comparison of searching algorithms of streaming ASR in English datasets in WER.  All the models were trained with Librispeech.  ID refers to in-domain. In all the decoding methods, the LM of the target domain is fused.}
  \label{tab:eng}
  %\vspace{1mm}
  \vspace{-0.3cm}
  \centering
%  \begin{tabular}{l|cc|cc}
  %\hspace{-0.5cm}
  \scalebox{0.9}{
  \begin{tabular}{l||c|c||cc|cc|c||cc}
    \hline
    & total  & L-Sync & \multicolumn{2}{c|}{LS$\rightarrow$TEDLIUM3} &  \multicolumn{2}{c|}{LS$\rightarrow$Switchboard} & LS$\rightarrow$STOP&\multicolumn{2}{c}{Librispeech (ID)} 
    \\
      & beam size& beam size & dev & test & SW & CH & & test-clean  & test-other\\
    \hline
    % No LM & & & &27.1 & 1.54\\
    Streaming L-Sync decoding \cite{tsunoo2022run}  &&5& 13.7 & 13.1 & 30.2 & 35.7 & 14.5& 3.0 & 8.1 \\
    CTC F-Sync decoding \cite{hwang2016character}  &10&& 15.0 & 14.4 & 31.0 & 37.2 & 19.2 & 3.3 & 9.0\\
    CTC+AttDec F-Sync decoding   &10&& 14.6 & 14.0& 31.4 & 36.6 &19.4 & 3.1& 8.6\\
    Integ. FL-Sync decoding (proposed)  &10&5& {\bf 13.4} & {\bf 12.7} & {\bf 29.9} & {\bf 35.0} & {\bf 14.1} & {\bf 2.9} & {\bf 7.9}\\
    
    
    \hline
  \end{tabular}
  }
  %\vspace{-0.4cm}
\end{table*}


\begin{table*}[t]
  \caption{Comparison of searching algorithms of streaming ASR in Japanese datasets in CER.  All the models were trained with CSJ+LaboroTV (LTV). ID refers to in-domain.  In all the decoding methods, the general LM is fused.}
  \label{tab:jpn}
  %\vspace{1mm}
  \vspace{-0.3cm}
  \centering
%  \begin{tabular}{l|cc|cc}
  %\hspace{-0.5cm}
  \scalebox{0.9}{
  \begin{tabular}{l||c|c||c|c|c|c||ccc}
    \hline
    & total & L-Sync & CSJ+LTV & CSJ+LTV & CSJ+LTV & CSJ+LTV &\multicolumn{3}{c}{CSJ (ID)} \\
      &beam size & beam size & $\rightarrow$TEDxJP & $\rightarrow$News& $\rightarrow$CMD1& $\rightarrow$CMD2 & eval1 & eval2 & eval3\\
    \hline
    % No LM & & & &27.1 & 1.54\\
    Streaming L-Sync decoding \cite{tsunoo2022run}  &&5& 12.1 & 3.5 & 4.1 & 12.8 & 8.0 & 7.5 & {\bf 7.0}\\
    CTC F-Sync decoding \cite{hwang2016character} &10&& 12.4 & {4.9} & 5.1& { 10.1} & 8.7&7.8&7.3\\
    CTC+AttDec F-Sync decoding  &10&& 12.1 & 3.5& 3.5& 10.7& 8.0&7.4&7.4\\
    Integ. FL-Sync decoding (proposed)  &10&5& {\bf 11.9} & {\bf 3.0} & {\bf 3.2} & {\bf 9.6} & {\bf 7.9} & {\bf 7.3} & 7.1 \\
    
    \hline
  \end{tabular}
  }
  %\vspace{-0.4cm}
\end{table*}
\subsection{FL-Sync beam search algorithm}
The proposed beam search is summarized in Algorithm~\ref{alg:decode}. %, and it is graphically shown in Fig.~\ref{fig:decode}.
As the FL-Sync decoding is performed based on the F-Sync frame-wise pruning, $t$ is increased at each step in the loop.
First, L-Sync decoding is performed if step $i<\min(|Y \in \Omega_{\fl,t-1}|)$ as descried in Sec.~\ref{sssec:prefixfusion}.
The L-Sync decoding expands the prefixes of the hypotheses (Line~6) and provides prioritized hypotheses $\Y^i_{\ls}$, which are maintained in temporary hypothesis set $\hat{\Omega}$ (Line~7).
Then the F-Sync decoding is performed based on \eqref{eq:score}, and merged with the prioritized hypotheses (Line~9). %, which corresponds to \eqref{eq:merge}.
Subsequently, in Line~10, the ancestor-pruning described in Sec.~\ref{sssec:depth} is performed, which removes unnecessary hypotheses.
The hypotheses from L-Sync decoding have priority for survival (Line~11), then lastly the hypothesis set is filled from the temporary set up to total beam size, $B'$ (Line~12).
Note that, even the total beam size increases from $B$ to $B'$, computational impact is marginal from $B$-beam L-Sync decoding, because the F-Sync model, e.g., CTC, generally requires much less computation.

%The L-Sync model provide the prioritized hypotheses, as indicated by the red boxes in Fig~\ref{fig:decode}.
%Subsequently, the ordinary copy and expansion process of F-Sync decoding follows, as shown by the blue circles in Fig.~\ref{fig:decode}.
%After merging the hypotheses, all prioritized hypotheses are retained for the next decoding step (Line 20).
%Then remaining hypotheses are kept from the highest score until the number of hypotheses reaches the total beam size $2B$ (Line 21).
%(In the last time step, to determine the best hypothesis, L-Sync decoding is performed if it does not cover the entire best sequence $\Y_{\best}$.)
%Note that, even the total beam size doubles from $B$ to $2B$, computational impact is marginal from $B$-beam L-Sync decoding, because the F-Sync model, e.g., CTC, generally requires less computation.





\begin{comment}
\section{Integrated Beam Search of Label- and Frame-synchronous Decoding}
\subsection{Online rescoring of LM for frame-synchronous beam search}
\label{ssec:rescore}
The frame-synchronous decoding has a problem when it is fused with LM, because in each time step, it has hypotheses that are different sequence length; thus score comparison is actually unfair.
For instance, at frame $t$, in the case there are two hypotheses with a length of 1 and 2, the former has a score of $\log p_{\lm}(y_1|y_0)$ while the latter has $\log p_{\lm}(y_2|y_0:1) + \log p_{\lm}(y_1|y_0)$ for LM scores.
Although the length regularizer term $|\Y_1:i|$ mitigate this issue, it requires careful tuning.

Instead, we propose online rescoring by label-synchronous modules.
%While expanding hypotheses, the label-synchronous score $\alpha_{\lbl}$ is not applied, and only frame-synchronous scores are accumulated.
%After the minimum sequence length in the beam reaches the token length, $\alpha_{\lbl}$ is fairly applied to each hypothesis.
In comparison to the rescoring scheme as in \cite{sainath2019two}, it performs rescoring to the limited part of sequences in the beam, and based on the score, the beam search continues carrying out.
Therefore, we refer this approach as online rescoring by the label-synchronous modules.
The procedure depicted in Fig.~\ref{fig:online}.
We will demonstrate its effectiveness in Sec.~\ref{ssec:online}.


\subsection{Frame-synchronous beam search with guides from label-synchronous hypotheses}
\label{ssec:integ}
In the online rescoring, until the label-synchronous modules apply scores, it simply follow time-synchornous decoding procedure; thus it cannot gain any support from the LSMs.
To mitigate this issue, we propose a new integrated beam search of label- and frame-synchronous decoding, in which the LSMs provide priority labels for hypotheses from LSMs.
The prioritized hypotheses are forced to survive in the beam search process, until they finally get rescored by LSMs.
Thus, in the beam, there are always prioritized hypotheses from LSMs and ordinary hypotheses expanded from FSMs.
The proposed beam search is depicted in Fig.~\ref{fig:decode}.
% Figure environment removed
\end{comment}




\section{Experiments}
To evaluate the effectiveness of the proposed FL-Sync beam search, we evaluated cross-domain and intra-domain scenarios in English and Japanese datasets.

%\subsection{Search algorithm comparison in English datasets}
%\label{ssec:english}
%\subsubsection{Experimental setup}
\subsection{Experimental setup}
\label{ssec:english_setup}
For the English evaluation, we trained a streaming encoder--decoder ASR model following \cite{tsunoo2022run} with the Librispeech dataset \cite{panayotov15}, a read speech corpus.
It was applied to three various domains: The TED-LIUM 3 \cite{hernandez2018ted} dev/test set (a spontaneous lecture style), Hub5’00 (telephony-style conversation) having Switchboard (SWB) and CallHome (CHM) subsets, and voice-command-style STOP dataset \cite{tomasello2023stop}.

For Japanese, we trained a streaming ASR model with a merged set of the lecture style CSJ \cite{csj} and LaboroTV corpus of TV program \cite{ando2021construction}.
We used three evaluation sets of the CSJ dataset for intra-domain setup, and TEDxJP \cite{ando2021construction} for cross-domain setup.
To cover various range of domains, We also used in-house evaluation data; 720 read news utterances (News) spoken by four males and four females, 2,620 voice commands (CMD1) and far-field 1,768 commands (CMD2), each spoken by 10 hired speakers.

The input acoustic features were 80-dimensional filter bank features.
The input features were applied mean normalization with a sliding window.
The encoder consisted of 12 blocks of conformer \cite{gulati2020}.
The AttDec had six decoder blocks.
Each block consisted of four-head 256-unit attention layers and 2048-unit feed-forward layers.
Contextual block encoding \cite{tsunoo19} was applied to the encoder with a block size of 40, a shift size of 16, and a look-ahead size of 16.
The models were trained using multitask learning with CTC loss \cite{watanabe17}, with a weight of 0.3.
%We used the Adam optimizer and Noam learning rate decay, and applied SpecAugment \cite{park19}.

For English evaluation, external LMs were trained using each target dataset for adaptation.
LMs were four-layer unidirectional LSTM with 2048 units, using the byte-pair encoding (BPE) subword tokenization with 5000 token classes.
%The LMs were fused with a weight of 0.6/0.4 for the cross-/intra-domain evaluation in English ($\lambda_{\lm}$ in \eqref{eq:score}).
For Japanese, we collected over 27 million sentences to train a general LM with 10,000 BPE tokens, applied to all the tasks.

We compared our proposed FL-Sync decoding with streaming L-Sync decoding \cite{tsunoo2022run}, and CTC F-Sync decoding \cite{hwang2016character}.
We also evaluated AttDec score fusion for CTC F-Sync decoding for comparison.
%For LM fusion in CTC decoding, we set the LM weight ($\lambda_{\lm}$ in \eqref{eq:fscore}) as 0.4, and did not apply the AttDec score ($\lambda_{\att}=0$).
%For the CTC F-Sync decoding methods, $(\lambda_{\lm}, \lambda_{\att})=(0.4,0)$ or $(0.4,0.4)$ were used for English datasets, and $\lambda_{\lm}=0.1$ was used for Japanese, instead.
For the CTC F-Sync decoding methods, $\lambda_{\lm}=0.4$ was used for English datasets, and $\lambda_{\lm}=0.1$ for Japanese, instead.
For the baseline L-Sync decoding and the proposed FL-Sync decoding, we adopted $(\lambda_{\lm}, \lambda_{\ctc})=(0.4,0.4)$ for English intra-domain, $(0.6,0.4)$ for English cross-domain, and $(0.3,0.5)$ for Japanese experiments.
$\lambda_{\att}$ was set as $(1-\lambda_{\ctc})$.
We adopted $\lambda_{\len}=1$ for all the evaluation.
We used $B'=10$ and $B=5$ as a beam size for total FL-Sync decoding and L-Sync decoding in it, respectively.
For fair comparison, we used the same beam sizes for the baseline approaches.
%In the baseline L-Sync, the hypotheses were expanded to $B\times 3B$ to calculate CTC prefix score $\alpha_{\ctc}$ in \eqref{eq:attscore} based on the top $3B$ hypotheses of the AttDec.
%For CTC F-Sync decoding, the hypotheses were expanded to $B \times B$.
%In the proposed FL-Sync, it followed F-Sync expansion with a total beam size of $2B$; thus the hypotheses were expanded to $2B \times B$, which was smaller than the baseline L-Sync space.


\subsection{Results of English evaluation}
The word error rate (WER) results are summarized in Table~\ref{tab:eng}.
%Streaming L-Sync decoding was generally better than CTC F-Sync decoding, which indicated that AttDec had more ability to model sequence, and were effective to be fused with LMs.
Streaming L-Sync decoding was generally better than CTC F-Sync decoding, particularly in the voice-command STOP dataset.
This indicated that it is more effective in pruning to use input information including future look-ahead frames.
Although the AttDec score fusion improved accuracy on CTC F-Sync decoding, the proposed FL-Sync decoding showed large improvement from those F-Sync decoding methods because it also preserved hypotheses from the L-Sync decoding.
When we compare FL-Sync decoding with the L-Sync baseline decoding, our proposed method performed robustly in the cross-domain scenarios; for instance, WER was improved from 13.1\% to 12.7\% in TEDLIUM3 test set.
Furthermore, we confirmed that it did not degrade the intra-domain performance, or even observed slight improvement from 8.1\% to 7.9\% in test-other, for instance.
Since the scores, \eqref{eq:fscore}, \eqref{eq:jointscore}, and \eqref{eq:score}, eventually become the same in all the search methods, the results show that our method is effective in pruning because of maintaining hypotheses from both L-Sync and F-Sync decoding.
%Thus, FL-Sync ended with better results than the baseline L-Sync decoding.

\subsection{Results of Japanese evaluation}
We evaluated character error rates (CERs), which are listed in Table~\ref{tab:jpn}.
The results followed similar tendency to the English evaluation.
As we used the general LM, it did not exactly match to some of the target domains.
%As the result, we observed that CTC F-Sync decoding had difficulty in fusing LM.
As a result, we observed that even in the source domain CSJ eval sets, CERs were relatively higher than the other literature \cite{tsunoo2022run, karita2021comparative}. 
Furthermore, the baseline L-Sync decoding had difficulty in adapting to the target domains, in particular in CMD2 data, which tend not to be grammatically structured.
Since our proposed FL-Sync decoding maintained both L-Sync and F-Sync decoding, it recovered the domain bias and was robust to the cross-domain situation.
%In particular in the voice commands, which tend not to be grammatically structured, the L-Sync had disadvantage on such a target domain.
% Our proposed methods, on the other hand, gained improvement over the L-Sync decoding, because of the CTC hypothesis inclusion.

\begin{comment}
\subsection{Search algorithm comparison in Japanese datasets.}
\subsubsection{Experimental setup}
For Japanese, we trained a streaming ASR model using the same configuration as Sec.~\ref{ssec:english_setup}, with a merged set of the lecture style CSJ \cite{csj} and LaboroTV corpus of TV program \cite{ando2021construction}.
We used three evaluation sets of CSJ corpus for intra-domain setup, and TEDxJP \cite{ando2021construction} for cross-domain setup.
We also used in-house evaluation data; 720 read news utterances (News) spoken by four males and four females, 2,620 and 1,768 voice commands (CMD1, CMD2) each spoken by 10 hired speakers, to cover various range of domains.
We collected over 27 million sentences to train a general LM.
We used 10,000 BPE tokens for Japanese.
%The fusion parameters for L-Sync and FL-Sync decoding were $\lambda_{\ctc}=0.4$ and $\lambda_{\lm}=0.4$ for all tasks.
%The best LM weight for CTC F-Sync was found to be 0.1, using CSJ data.
We used $(\lambda_{\lm}, \lambda_{\att})=(0.4,0.4)$ for all the search methods except CTC F-Sync using $\lambda_{\lm}=0.1$.

\subsubsection{Results}
We evaluated character error rates (CERs), which are listed in Table~\ref{tab:jpn}.
As we used general LM, it did not exactly match to some of the target domains. 
%As the result, we observed that CTC F-Sync decoding had difficulty in fusing LM.
As a result, we observed that the baseline L-Sync decoding had difficulty in adapting to the target domains.
Since our proposed FL-Sync can take advantage of F-Sync decoding, it was robust to the cross-domain situation, and overall achieved lower CERs than the L-Sync baseline decoding.
Further, because the voice commands tend not to be grammatically structured, the L-Sync decoding did not perform well in those tasks.
Our proposed methods, on the other hand, gained improvement over the L-Sync decoding, because of CTC hypothesis inclusion.
\end{comment}

\begin{comment}
    
\subsection{LM fusion comparison for F-Sync beam search}
\label{ssec:online}

\begin{table}[t]
  \caption{Comparison in frame-synchronous beam search using Librispeech (WER).}
  \label{tab:fs-libri}
  %\vspace{1mm}
  \vspace{-0.3cm}
  \centering
%  \begin{tabular}{l|cc|cc}
  %\hspace{-0.5cm}
  \scalebox{0.9}{
  \begin{tabular}{l|ccc|cc}
    \hline
   Frame-synchronous & test-clean & test-other \\
    \hline\hline
    % No LM & & & &27.1 & 1.54\\
    CTC frame-sync. \cite{hwang2016character} &3.4 & 9.3 \\
    + LM online rescoring (Sec.~\ref{ssec:rescore}) & 4.1 & 10.2 \\
   % ++ Att online rescoring(Sec.~\ref{ssec:rescore}) & & (5.2)  \\
    Integ. Label/Frame-sync. (Sec.~\ref{ssec:integ}) & {\bf 2.9} & {\bf 7.9} \\
    \hline
  \end{tabular}
  }
  %\vspace{-0.4cm}
\end{table}
\end{comment}


\section{Conclusion}
We have proposed a new FL-Sync beam search, which integrates complementary F-Sync and Sync decoding to mitigate the problems in both the decoding.
The proposed beam search primarily runs in an F-Sync manner to incorporate it with block-wise streaming ASR.
To overcome the unreliable partial F-Sync score comparison in pruning, L-Sync decoding provide the prioritized hypotheses with future look-ahead input frames, which are also maintained in the beam to perform effective pruning.
%The L-Sync hypotheses have priority for survival in pruning, until all the scores of expanded successors exceed the score of them. 
%Experiments showed that the proposed search algorithm had an advantage in performance comparing to the baseline F-Sync and L-Sync search.
%We also observed that FL-Sync also had more robustness against out-of-domain situations than L-Sync decoding.


%\bibliographystyle{IEEEtran}
%\bibliography{mybib}
\section{References}
{
\printbibliography
}
\end{document}
