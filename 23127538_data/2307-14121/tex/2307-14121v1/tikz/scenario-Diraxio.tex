\documentclass{standalone}
\begin{document}
\tikzset{
  blackline/.style={thin, draw=black, postaction={decorate},
    decoration={markings, mark=at position 0.6 with {\arrow[black]{triangle 45}}}}
}

\begin{tikzpicture}[font=\small,thick]
 
% Start block
 \node[draw,
    minimum width=1cm,
    minimum height=1cm,
] (block1) {$\dot{\theta}$ couples to $\nu_R$};
 

\node[draw,
    minimum width=1cm,
    minimum height=1cm,
    right=0.7cm of block1,
    fill=white
] (block2) {\begin{tabular}{c} $H S \rightarrow \overline{L} \nu_R$ \\ $L S\rightarrow H^\dagger \nu_R$ \end{tabular}};



 \node[draw,
   circle,
   minimum size = 0.3 cm,
   right= 0.7cm of block2,
   fill=gray
] (block11) { $\mu_\text{L}  = -\frac{3}{16} c_{\nu_R} \dot{\theta}$};

 \node[draw,
   circle,
   minimum size = 0.3 cm,
   above  = 0.7 cm of block11,
   fill=gray
] (block12) {$\mu_{\nu_R} = \frac{79}{112} c_{\nu_R} \dot{\theta}$};


 \node[draw,
   circle,
   minimum size = 0.3 cm,
   below  = 0.7 cm of block11,
   fill=gray
] (block13) {$\mu_H = -\frac{3}{28} c_{\nu_R} \dot{\theta}$};


 \node[draw,
   circle,
   minimum size = 0.3 cm,
   right  = 0.7 cm of block11,
   fill=green
] (block15) {$\mu_\text{B-L}^\text{SM} =\frac{79}{112} c_{\nu_R} \dot{\theta}$};



 \node[draw,
   circle,
   minimum size = 0.3 cm,
   right = 0.7 cm of block15,
    fill=red
] (block14) { $\mu_\text{B} =  \frac{28}{79} \mu_\text{B-L}^\text{SM} $};





\draw[-latex ] (block1) -- (block2)  node[right,below,midway]{};
\draw[-latex ] (block2) -- (block11) node[right,above,midway]{};
\draw[-latex ] (block2) -- (block12) node[right,above,midway]{};
\draw[-latex ] (block2) -- (block13) node[right,above,midway]{};
\draw[-latex ] (block11) -- (block15) node[right,above,midway]{SM};
\draw[-latex ] (block13) -- (block15) node[left,below,midway]{$\quad$SM};

\draw[red,-latex,double] (block15) -- (block14) node[above,midway]{sph.};
\draw[violet,latex-latex ] (block11) -- (block12)  node[right,midway]{washout  (slow)};

\end{tikzpicture}
\end{document}