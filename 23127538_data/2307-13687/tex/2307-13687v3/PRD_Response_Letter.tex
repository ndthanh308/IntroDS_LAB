\documentclass[11pt]{article}
\usepackage[utf8]{inputenc}	% Para caracteres en español
\usepackage{amsmath,amsthm,amsfonts,amssymb,amscd}
\usepackage{multirow,booktabs}
\usepackage[table]{xcolor}
\usepackage{fullpage}
\usepackage{lastpage}
\usepackage{enumitem}
\usepackage{fancyhdr}
\usepackage{mathrsfs}
\usepackage{wrapfig}
\usepackage{setspace}
\usepackage{calc}
\usepackage{multicol}
\usepackage{cancel}
\usepackage[retainorgcmds]{IEEEtrantools}
\usepackage[margin=3cm]{geometry}
\usepackage{amsmath}
\usepackage{graphicx} % Required for inserting images
\usepackage[normalem]{ulem} % strikethrough 
\usepackage{empheq}
\usepackage{framed}
\usepackage[most]{tcolorbox}
\usepackage{xcolor}
\usepackage{hyperref}

\newlength{\tabcont}
\setlength{\parindent}{0.0in}
\setlength{\parskip}{0.05in}
\colorlet{shadecolor}{gray!10}
\parindent 0in
\parskip 12pt
\geometry{margin=1in, headsep=0.25in}
\theoremstyle{definition}
\newtheorem{defn}{Definition}
\newtheorem{reg}{Rule}
\newtheorem{exer}{Exercise}
\newtheorem{note}{Note}

\newcommand{\ra}{ $\rightarrow$ }

\begin{document}

\setcounter{section}{1}

\thispagestyle{empty}

\begin{center}
    \section*{Response to Reviewer Comments}
\end{center}

Thank you for your comments on our PRD submission of \emph{Score-based Diffusion Models for Generating Liquid Argon Time Projection Chamber Images}. 
We have modified our original text to incorporate all your feedback. 
Below is an outline of our reasoning behind our changes, specifically in response to your three primary concerns. 
We are also attaching a version of the paper with tracked changes for your convenience. 


1.
We agree that including epoch 10 in the analyses is repetitive and redundant. As such, we have removed epoch 10 from the  SSNet plots (Fig. 7) and Physics Analyses (Figs 9, 10, 12, and Table I). Similarly, for the VQ-VAE plots in Appendix E. We have also removed all references to epoch 10 in the text. The fact that the model performs poorly initially and rapidly improves can be inferred from the loss curve.  


2.
The speed of image generation is a valid concern. 
However, we seek to postpone the discussion of speed to later work with a more realistic dataset (512x512 image size). 
We have made this more clear in the discussion section (Section VIII). 


3.
You are correct that there will be several challenges to overcome before we can apply this approach to physics analyses and/or simulation. 
We have expanded the discussion section (Section VIII) to outline a number of these challenges and to note some possible directions to address these challenges.
We also modified the introduction to make the point that broadly there is the challenge of producing images that accurately reproduce the patterns in LArTPCs and the challenge of efficiency, which requires larger images and better throughput. 
We do this to make it more clear that we only address the first challenge in this work. 
Making this much more clear right away is important, and so we thank you again for this comment. 

We have implemented all your suggested text corrections and a few additional ones of our own. 
These are purely grammatical and typo corrections that do not affect the content of the work. 

We also received some external requests to cite related work in HEP utilizing GANs and Normalizing Flows. 
To accommodate these requests we have added text to the introduction that provides additional context to the ongoing work in these areas. 

We greatly appreciate your review and the chance to improve our work.
Hopefully, we have adequately addressed your concerns. 
Please let us know if you have any further questions or suggestions. 

Sincerely, \\
Zeviel Imani, Shuchin Aeron, Taritree Wongjirad


% \section*{Epoch 10 Analysis}
% \emph{I became a bit frustrated when reading through the manuscript because of the discussion of the 10 epoch
% results in each of different results sections. It is clear from the loss function that epoch 10 will give worse
% results, and it is certainly clear after the results from the SSNet track vs shower results in section VII-A.
% At this point the authors should make it clear that 10 epochs is clearly a very poor choice of model and
% say that it will not be discussed further. The continued discussion of the 10 epoch results adds nothing
% to the manuscript and actually detracts from the much more impressive results that are presented.
% }

% \section*{Speed Concerns}
% \emph{It is my understanding that diffusion processes can be significantly slower than other approaches such as
% GANs. With 50,000 (64 x 64) images produced in 40 hours on two GPUs, this is likely much slower than
% a simple LArTPC simulation using Geant4, though it is likely to be faster than a full detector simulation
% that deals with cosmic rays as well as neutrinos and more detailed simulations of electronics responses
% etc. Since the introduction makes a strong statement about the slow speed of simulations, I think this
% discussion needs to be revisited in the conclusion because it is a major motivation why deep learning
% approaches are being employed for simulation. In particular, do the authors have an idea on how much
% slower image generation would be with a more realistic image size? Moving from (64 x 64) images to (512
% x 512), assuming a linear scaling in time with respect to the number of pixels, would reduce the image
% production rate from 1250 per hour to approximately 20.}

% \section*{Expanded Applications}
% \emph{I would like to see a bit more discussion in the conclusion of what is needed to make these images more
% useful for physics analysis. Having accurately generated images such as those produced by this model is
% clearly a significant milestone and proof-of-principle, but there are a number of steps that will be required
% before having a model to entirely replace the simulation. Such a model would need to be conditional on
% physics quantities such as the neutrino type and neutrino energy as a minimum.}

\end{document}
