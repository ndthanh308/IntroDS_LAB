\documentclass[%
 reprint,
%superscriptaddress,
%groupedaddress,
%unsortedaddress,
%runinaddress,
%frontmatterverbose, 
%preprint,
%preprintnumbers,
%nofootinbib,
%nobibnotes,
%bibnotes,
 amsmath,amssymb,
 aps,
%pra,
%prb,
%rmp,
%prstab,
%prstper,
%floatfix,
]{revtex4-2}

%\usepackage{iclr2021_conference,times}

% Optional math commands from https://github.com/goodfeli/dlbook_notation.
%%%%% NEW MATH DEFINITIONS %%%%%
\newtheorem{property}{Property}
\newtheorem{definition}{Definition}
\newtheorem{theorem}{Theorem}
\newtheorem{lemma}{Lemma}
\newtheorem{corollary}{Corollary}
\DeclarePairedDelimiter\abs{\lvert}{\rvert}
\DeclarePairedDelimiter\norm{\lVert}{\rVert}
\makeatletter
\let\oldabs\abs
\def\abs{\@ifstar{\oldabs}{\oldabs*}}
\let\oldnorm\norm
\def\norm{\@ifstar{\oldnorm}{\oldnorm*}}
\makeatother

% Mark sections of captions for referring to divisions of figures
\newcommand{\figleft}{{\em (Left) }}
\newcommand{\figcenter}{{\em (Center) }}
\newcommand{\figright}{{\em (Right)}}
\newcommand{\figtop}{{\em (Top) }}
\newcommand{\figbottom}{{\em (Bottom) }}
\newcommand{\captiona}{{\em (a) }}
\newcommand{\captionb}{{\em (b) }}
\newcommand{\captionc}{{\em (c) }}
\newcommand{\captiond}{{\em (d) }}

% Highlight a newly defined term
\newcommand{\newterm}[1]{{\bf #1}}


\def\figref#1{figure~\ref{#1}}
\def\Figref#1{Figure~\ref{#1}}
\def\twofigref#1#2{figures \ref{#1} and \ref{#2}}
\def\quadfigref#1#2#3#4{figures \ref{#1}, \ref{#2}, \ref{#3} and \ref{#4}}
\def\secref#1{section~\ref{#1}}
\def\Secref#1{Section~\ref{#1}}
\def\twosecrefs#1#2{sections \ref{#1} and \ref{#2}}
\def\secrefs#1#2#3{sections \ref{#1}, \ref{#2} and \ref{#3}}
\def\eqref#1{equation~\ref{#1}}
\def\Eqref#1{Equation~\ref{#1}}
% A raw reference to an equation---avoid using if possible
\def\plaineqref#1{\ref{#1}}
% Reference to a chapter, lower-case.
\def\chapref#1{chapter~\ref{#1}}
% Reference to an equation, upper case.
\def\Chapref#1{Chapter~\ref{#1}}
% Reference to a range of chapters
\def\rangechapref#1#2{chapters\ref{#1}--\ref{#2}}
% Reference to an algorithm, lower-case.
\def\algref#1{algorithm~\ref{#1}}
% Reference to an algorithm, upper case.
\def\Algref#1{Algorithm~\ref{#1}}
\def\twoalgref#1#2{algorithms \ref{#1} and \ref{#2}}
\def\Twoalgref#1#2{Algorithms \ref{#1} and \ref{#2}}
% Reference to a part, lower case
\def\partref#1{part~\ref{#1}}
% Reference to a part, upper case
\def\Partref#1{Part~\ref{#1}}
\def\twopartref#1#2{parts \ref{#1} and \ref{#2}}

% Random variables
\def\reta{{\textnormal{$\eta$}}}
\def\ra{{\textnormal{a}}}

% Random vectors
\def\rvepsilon{{\mathbf{\epsilon}}}
\def\rvtheta{{\mathbf{\theta}}}
\def\rva{{\mathbf{a}}}

% Elements of random vectors
\def\erva{{\textnormal{a}}}
\def\ervb{{\textnormal{b}}}

% Random matrices
\def\rmA{{\mathbf{A}}}
\def\rmB{{\mathbf{B}}}

% Elements of random matrices
\def\ermA{{\textnormal{A}}}
\def\ermB{{\textnormal{B}}}

\def\fvec{{\mathbf{f}}}
\def\bff{{\mathbf{f}}}
\def\bfg{{\mathbf{g}}}
% Vectors
\def\vzero{{\bm{0}}}
\def\vone{{\bm{1}}}
\def\vmu{{\bm{\mu}}}
\def\vtheta{{\bm{\theta}}}
\def\va{{\bm{a}}}
\def\vb{{\bm{b}}}
\def\vc{{\bm{c}}}
\def\vd{{\bm{d}}}
\def\ve{{\bm{e}}}
\def\vf{{\bm{f}}}
\def\vg{{\bm{g}}}
\def\vh{{\bm{h}}}
\def\vi{{\bm{i}}}
\def\vj{{\bm{j}}}
\def\vk{{\bm{k}}}
\def\vl{{\bm{l}}}
\def\vm{{\bm{m}}}
\def\vn{{\bm{n}}}
\def\vo{{\bm{o}}}
\def\vp{{\bm{p}}}
\def\vq{{\bm{q}}}
\def\vr{{\bm{r}}}
\def\vs{{\bm{s}}}
\def\vt{{\bm{t}}}
\def\vu{{\bm{u}}}
\def\vv{{\bm{v}}}
\def\vw{{\bm{w}}}
\def\vx{{\bm{x}}}
\def\vy{{\bm{y}}}
\def\vz{{\bm{z}}}

% Matrix
\def\mA{{\bm{A}}}

% Tensor
\DeclareMathAlphabet{\mathsfit}{\encodingdefault}{\sfdefault}{m}{sl}
\SetMathAlphabet{\mathsfit}{bold}{\encodingdefault}{\sfdefault}{bx}{n}
\newcommand{\tens}[1]{\bm{\mathsfit{#1}}}
\def\tA{{\tens{A}}}
\def\tB{{\tens{B}}}
\def\tC{{\tens{C}}}
\def\tD{{\tens{D}}}
\def\tE{{\tens{E}}}
\def\tF{{\tens{F}}}
\def\tG{{\tens{G}}}
\def\tH{{\tens{H}}}
\def\tI{{\tens{I}}}
\def\tJ{{\tens{J}}}
\def\tK{{\tens{K}}}
\def\tL{{\tens{L}}}
\def\tM{{\tens{M}}}
\def\tN{{\tens{N}}}
\def\tO{{\tens{O}}}
\def\tP{{\tens{P}}}
\def\tQ{{\tens{Q}}}
\def\tR{{\tens{R}}}
\def\tS{{\tens{S}}}
\def\tT{{\tens{T}}}
\def\tU{{\tens{U}}}
\def\tV{{\tens{V}}}
\def\tW{{\tens{W}}}
\def\tX{{\tens{X}}}
\def\tY{{\tens{Y}}}
\def\tZ{{\tens{Z}}}


% Graph
\def\gA{{\mathcal{A}}}
\def\gB{{\mathcal{B}}}
\def\gC{{\mathcal{C}}}
\def\dataset{{\mathcal{D}}}
\def\gE{{\mathcal{E}}}
\def\gF{{\mathcal{F}}}
\def\fourier{{\mathcal{F}}}
\def\gG{{\mathcal{G}}}
\def\gH{{\mathcal{H}}}
\def\gI{{\mathcal{I}}}
\def\gJ{{\mathcal{J}}}
\def\gK{{\mathcal{K}}}
\def\gL{{\mathcal{L}}}
\def\loss{{\mathcal{L}}}
\def\gM{{\mathcal{M}}}
\def\gN{{\mathcal{N}}}
\def\normal{{\mathcal{N}}}
\def\gaussian{{\mathcal{N}}}
\def\gO{{\mathcal{O}}}
\def\gP{{\mathcal{P}}}
\def\gQ{{\mathcal{Q}}}
\def\gR{{\mathcal{R}}}
\def\gS{{\mathcal{S}}}
\def\gT{{\mathcal{T}}}
\def\gU{{\mathcal{U}}}
\def\uniform{{\mathcal{U}}}
\def\gV{{\mathcal{V}}}
\def\gW{{\mathcal{W}}}
\def\gX{{\mathcal{X}}}
\def\gY{{\mathcal{Y}}}
\def\gZ{{\mathcal{Z}}}

\def\algebra{{\mathscr{A}}}
\def\borel{{\mathscr{B}}}
\def\manifold{{\mathscr{M}}}

% Sets
\def\sA{{\mathbb{A}}}
\def\sB{{\mathbb{B}}}
\def\complex{{\mathbb{C}}}
\def\sD{{\mathbb{D}}}
\def\expectation{{\mathbb{E}}}
\newcommand{\E}{\mathbb{E}}
\def\sF{{\mathbb{F}}}
\def\sG{{\mathbb{G}}}
\def\sH{{\mathbb{H}}}
\def\sI{{\mathbb{I}}}
\def\sJ{{\mathbb{J}}}
\def\sK{{\mathbb{K}}}
\def\sL{{\mathbb{L}}}
\def\sM{{\mathbb{M}}}
\def\natural{{\mathbb{N}}}
\def\sO{{\mathbb{O}}}
\def\sP{{\mathbb{P}}}
\def\rational{{\mathbb{Q}}}
\def\real{{\mathbb{R}}}
\newcommand{\R}{\mathbb{R}}
\def\sS{{\mathbb{S}}}
\def\sphere{{\mathbb{S}}}
\def\sT{{\mathbb{T}}}
\def\sU{{\mathbb{U}}}
\def\sV{{\mathbb{V}}}
\def\sW{{\mathbb{W}}}
\def\sX{{\mathbb{X}}}
\def\sY{{\mathbb{Y}}}
\def\integer{{\mathbb{Z}}}
\def\indicator{{\mathbbm{1}}}

% Entries of a matrix
\def\emLambda{{\Lambda}}
\def\emA{{A}}
\def\emB{{B}}
\def\emC{{C}}
\def\emD{{D}}
\def\emE{{E}}
\def\emF{{F}}
\def\emG{{G}}
\def\emH{{H}}
\def\emI{{I}}
\def\emJ{{J}}
\def\emK{{K}}
\def\emL{{L}}
\def\emM{{M}}
\def\emN{{N}}
\def\emO{{O}}
\def\emP{{P}}
\def\emQ{{Q}}
\def\emR{{R}}
\def\emS{{S}}
\def\emT{{T}}
\def\emU{{U}}
\def\emV{{V}}
\def\emW{{W}}
\def\emX{{X}}
\def\emY{{Y}}
\def\emZ{{Z}}
\def\emSigma{{\Sigma}}

% entries of a tensor
% Same font as tensor, without \bm wrapper
\newcommand{\etens}[1]{\mathsfit{#1}}
\def\etLambda{{\etens{\Lambda}}}
\def\etA{{\etens{A}}}
\def\etB{{\etens{B}}}
\def\etC{{\etens{C}}}
\def\etD{{\etens{D}}}
\def\etE{{\etens{E}}}
\def\etF{{\etens{F}}}
\def\etG{{\etens{G}}}
\def\etH{{\etens{H}}}
\def\etI{{\etens{I}}}
\def\etJ{{\etens{J}}}
\def\etK{{\etens{K}}}
\def\etL{{\etens{L}}}
\def\etM{{\etens{M}}}
\def\etN{{\etens{N}}}
\def\etO{{\etens{O}}}
\def\etP{{\etens{P}}}
\def\etQ{{\etens{Q}}}
\def\etR{{\etens{R}}}
\def\etS{{\etens{S}}}
\def\etT{{\etens{T}}}
\def\etU{{\etens{U}}}
\def\etV{{\etens{V}}}
\def\etW{{\etens{W}}}
\def\etX{{\etens{X}}}
\def\etY{{\etens{Y}}}
\def\etZ{{\etens{Z}}}

\def\ceil#1{\lceil #1 \rceil}
\def\floor#1{\lfloor #1 \rfloor}
\def\eps{{\epsilon}}

\newcommand{\pder}[1]{\frac{\partial}{\partial #1}}

\newcommand{\half}{\frac{1}{2}}
\newcommand{\limNinf}{\lim_{N \to \infty}}
\newcommand{\limTzero}{\lim_{\tau \to 0}}


\newcommand{\cmark}{\ding{51}}
\newcommand{\xmark}{\ding{55}}

\newcommand{\layer}{\mathcal{H}}
\newcommand{\defeq}{\triangleq}
%\newcommand{\defeq}{vcentcolon=}
\newcommand{\domain}{\Omega}
\newcommand{\grad}{\nabla}

\newcommand{\cin}{c_{\rm{in}}}
\newcommand{\cout}{c_{\rm{out}}}
\newcommand{\intdomain}{\int_{\domain}}
\newcommand{\network}{\gT}
\newcommand{\subnet}{\gK}
\newcommand{\map}{\gR} %\gR

\newcommand{\innerproduct}[2]{\langle #1, #2 \rangle}
\newcommand{\mcsum}[1][j]{\frac{1}{N}\sum_{#1=1}^N}

\newcommand{\inrspace}[1][c]{\gF_{#1}}

\DeclareMathOperator*{\argmax}{arg\,max}
\DeclareMathOperator*{\argmin}{arg\,min}

\let\ab\allowbreak


\usepackage{hyperref}
\usepackage{url}
\usepackage{graphicx}% Include figure files
\usepackage{dcolumn, xcolor}% Align table columns on decimal point
\usepackage{bm}% bold math
\usepackage{subcaption}
\usepackage{float}
\usepackage{makecell}
\usepackage{afterpage}
\usepackage{tabularx}


\newcommand{\fix}{\marginpar{FIX}}
\newcommand{\new}{\marginpar{NEW}}


\begin{document}

\title{Score-based Diffusion Models for Generating Liquid Argon Time Projection Chamber Images}% Force line breaks with \\
%\thanks{A footnote to the article title}%

\author{Zeviel Imani}
  \email{zeviel.imani@tufts.edu}
\author{Taritree Wongjirad}
  \email{taritree.wonjiradg@tufts.edu}
\affiliation{Department of Physics and Astronomy, Tufts University, Medford, Massachusetts\\
The NSF AI Institute for Artificial Intelligence and Fundamental Interactions\\
}

\author{Shuchin Aeron}
\email{shuchin.aeron@tufts.edu}
\affiliation{
Department of Electrical and Computer Engineering, Tufts University, Medford, Massachusetts\\
The NSF AI Institute for Artificial Intelligence and Fundamental Interactions\\
}
%
% The \author macro works with any number of authors. There are two commands
% used to separate the names and addresses of multiple authors: \And and \AND.
%
% Using \And between authors leaves it to \LaTeX{} to determine where to break
% the lines. Using \AND forces a linebreak at that point. So, if \LaTeX{}
% puts 3 of 4 authors names on the first line, and the last on the second
% line, try using \AND instead of \And before the third author name.

\begin{abstract}
In this paper, we show that a \emph{hybrid} approach to generative modeling via combining
the decoder from an autoencoder together with an explicit generative model for the latent
space is a promising method for producing images of 
particle trajectories in a liquid argon time projection chamber (LArTPC).
LArTPCs are a type of particle physics detector used by several
current and future experiments focused on studies of the neutrino.
We implement a Vector-Quantized Variational Autoencoder (VQ-VAE) 
and PixelCNN which produces images with LArTPC-like features and introduce
a method to evaluate the quality of the images using
a semantic segmentation that identifies important physics-based features.
\end{abstract}

\maketitle

\section{Introduction}

%% WHY DID WE DO THIS
Liquid argon time projection chambers (LArTPC)~(\cite{rubbia1977liquid,chen1976p496})
are a class of detector playing a prominent role in current and 
future experiments studying the neutrino, one of the fundamental particles ~(\cite{amerio2004design,anderson2012argoneut,acciarri2017design,acciarri2017design,abi2018dune}).
Through precision measurements of the behavior of the neutrino, 
the experiments aim to further our understanding of the physical laws 
that govern our universe.
LArTPCs are an appealing choice of detector technology because they
can be built to large sizes -- kiloton-scale detectors with 
O(10 m) dimensions -- while also economically instrumented to capture the
features of charged particle trajectories at the O(mm) scale.
This combination enables LArTPC experiments to record
a large number of high resolution observations of neutrino interactions.

The data produced by LArTPCs can be naturally arranged into an image-like
format which capture projections of particle trajectories.
Charged particles traversing the detector
create clouds of ionization electrons along their path.
The amount of ionization created is proportional to 
the amount of energy lost by the particle.
The pattern of ionization and the amount of energy lost
in the detector depends on the particle type and momentum.
This makes it possible to analyze the images and infer the
sets of particles and their energies.
Figure~\ref{fig:gen_examples}(b) shows some examples of images produced.
% by various particle types.

The need for efficient image analysis has motivated the the use of deep convolutional neural networks~(\cite{lecun1998gradient,krizhevsky2017imagenet}) to
identify key features or objects within LArTPC images for physics analyses~(\cite{acciarri2017convolutional,collaboration2019deep,abratenko2020convolutional,abratenko2020semantic,drielsma2020data,abi2020neutrino}). 
Recent efforts have focused primarily towards
mapping an image or portions within it to quantities 
such as different classes of particle trajectories~(\cite{abratenko2020semantic}), 
categories of neutrino interactions~(\cite{abi2018dune}), 
or identify individual particles~(\cite{abratenko2020convolutional}).

Less explored, however, are generative models for LArTPCs.
The ultimate goal for these models would be to receive a list of particles,
their position, and momentum and 
produce an image containing their trajectories through the detector, thereby providing a faster alternative to the
detailed physics-based simulation of the detector.
There have been some efforts in producing particle physics data via generative networks.
In~\cite{alonso2020image} images with tracks, similar to what might
be found in a LArTPC, are generated through the
help of an explicit physics model. 
Nevertheless, this preliminary approach is not suitable towards capturing the rich shower-like patterns, varying track-like, and mixed patterns that are present in LArTPC images. To this end we revisit the recent developments in generative modeling and argue for a particular type of model that exhibits promising behavior. 
% We begin by a short overview of this vast and still rapidly evolving domain 
% and then discuss how generative modeling can help advance experiments using LArTPCs.

% Broadly, generative models, given a set of data instances $X$ and labels $Y$,
% capture the joint probability $P(X,Y)$ or just $P(X)$ if there are no labels.
% Generative networks capture this probability distribution implicitly.
% The networks, in theory, do so by produce examples of the data with the right frequencies.

% For LArTPCs, generative networks would produce example images ($X$) of trajectories
% with the right distribution of underlying physical content ($Y$), 
% e.g. momentum distribution, particle species frequencies.
% Generative networks for LArTPCs enable new approaches to reconstruction
% and analysis.

There are two ways to specify a probability distribution, viz., explicit vs implicit. 
In explicit models, an explicit form of distribution is specified, say $\mathsf{P}_{\bm{X}}(\bm{x}; \Theta)$, where $\Theta$ is set of parameters. 
In implicit models, the main idea is that if a random variable $\bm{X}$ has distribution $\mathsf{P}$, this distribution is implicitly specified via a transformation. 
That is, $\bm{X} = G(\bm{Z})$ where $G$ is a map and $\bm{Z} \sim \mathsf{P}_{\bm{Z}}$.
Given $G, \mathsf{P}_{\bm{Z}}$ it is possible to \textit{compute} an explicit form, $\mathsf{P}_{\bm{X}}(\bm{x})$, but it is computationally hard when $G$ is complex, say a deep neural network, and especially when $\mathsf{P}_{\bm{Z}}$ is assumed to be \textit{simple} and lower-dimensional compared to $\bm{X}$. On the other hand, this allows one to model complex distributions and generate IID \textit{samples} from $\mathsf{P}_{\bm{X}}$ via IID samples from $P_{\bm{Z}}$. 
Hence the name generative modeling. 
Table \ref{tab:gans} provides a rough and apologetically incomplete (for lack of space) literature survey in this context.
The main point we want to highlight here is that hybrid models may behave better towards modeling images from LArTPC experiments compared to fully implicit or fully explicit models. 
We single out the Vector Qunatized(VQ)-VAE \cite{van2017neural} and Probabilistic Autoencoder (PAE) \cite{bohm2020probabilistic} as two recent models that are combine implicit modeling with an explicit model towards an overall generative network. 
Between VQ-VAE and PAE, the main difference is in the way the latent space is regularized. 
The latent space in VQ-VAE consists of a \textit{finite} set of quantization points, while in PAE it is a continuous subset of $\mathbf{R}^d$. 
Of these two, in this paper we work with VQ-VAE since the pixel CNN approach that explicitly models the quantized latent space, 
in spirit, also models the time evolution a particle trajectory through the detector. 
A full comparison between the two and various tradeoffs is on-going and will be reported in a future manuscript.

% To simplify exposition, as it is standard in related literature, the distribution associated with $G(\bm{Z})$ when $\bm{Z} \sim \mathsf{P}_{\bm{Z}}$ is denoted $G_{\#}\mathsf{P}_{\bm{Z}}$ - and is referred to as the \textit{pushforward} of $\mathsf{P}_{\bm{Z}}$ under $G$. Table \ref{tab:gans} summarizes some of popular models used in the literature. 


\onecolumngrid
  


\begin{table}[]
\begin{tabular}{|l|l|}
\hline
Popular Models/Methods             & Type                                        \\ \hline
\makecell{Normalizing Flows \\\cite{NF_Review, papamakarios2019normalizing}} & Explicit                                   \\ \hline
\makecell{Pixel-CNN \\ \cite{van2016conditional, salimans2017pixelcnn++}}       &  Explicit                                    \\ \hline
\makecell{ Variational Auto-Encoders (VAEs) \\\cite{kingma2019introduction, bousquet2017optimal} }          & Implicit                                    \\ \hline
\makecell{Generative Adversarial Networks \\\cite{goodfellow2014generative,arjovsky2017wasserstein} \\\cite{ gulrajani2017improved, li2017mmd, an2019ae}}             & Implicit                                    \\ \hline
\makecell{Vector Quantized-VAE, Probabilistic AE (PAE)  \\ \cite{van2017neural,bohm2020probabilistic}}           & \makecell{Explicit Latent + Implicit Decoder} \\ \hline
% \makecell{PAE \\ \cite{bohm2020probabilistic}}           & \makecell{\textbf{Hybrid} \\Implicit Latent + Explicit Decoder} \\ \hline
\end{tabular}
\caption{Table summarizing popular approaches for generative modeling. To emphasize the pertinent differences, we categorize into explicit, implicit, and hybrid models. Hybrid models utilize explicit model to generate a low-dimensional latent with an implicit model that comes from the decoder of an autoencoder.}
\vspace{-4mm}
\label{tab:gans}
\end{table}

\twocolumngrid

% \textbf{The landscape of the models that learn to generate data either implicitly or explicitly (in low-dimensional situations), combined with the dimensionality reduction and low-dimensional representation methods, gives one immense flexibility in approaching the generative modeling of data. In this paper we essentially present the results one such design choice, leaving the more detailed exploration of this design space to ongoing and future work.}\\

\textbf{Why generative models for LArTPCs?} - Generative networks enable computationally efficient means to generate trajectory examples, thereby bypassing and/or compliment the traditional simulation chain
consisting of particle transport and detector signal modeling.
This would make it easier to meet the demand
for example data required by physics analyses.
% This is especially true given the typical strategies employed for
% systematic uncertainty quantification.
% The expectation for some set of observables,
% e.g. the energy spectrum of reconstruction neutrino interactions,
% is estimated using simulated data produced by models of neutrino-nucleus interactions,
% the production of the neutrino flux from the collision of protons with nuclei,
% and the physics of the detector.
% The uncertainties in this models are often propagated via
% the generation and analysis of additional data sets with different parameter values of the model. 
% Generative models can help speed up the production of such data sets.

Generative models also open the path towards a complimentary approach to event reconstruction.
With models that can produce trajectory examples, conditional on parameters such as
momentum, one can extract quantities like
momenta or particle ID by comparing generated images produced by different 
physical parameters and choosing the best match by a likelihood function or possibly
a learned loss function implemented via a neural network.
One would iterate until new hypotheses fail to improve the loss. 
Concretely, given a data image $\bm{d}$ and a set of generative models $G_p, p = 1,..,P$, one approach inspired by recent use of deep networks in inverse problems \cite{bora2017compressed, jalal2020robust}
% results in inverse problems in imaging where a generative model is treated as a \textit{deep image prior}, see e.g. \cite{daniels2020reducing, Asim18}, 
would be to solve for, 
\begin{align}
\label{eq:1}
    \hat{p} = \arg \min_{p} \left\{ \min_{\bm{z}} \| G_p(\bm{z}) - \bm{d}\|^2 + \lambda \log \mathsf{P}_{\bm{Z}}(\bm{z}) \right\}
\end{align}
Another motivation for studying generative networks is understanding
ways to represent the data that enable different applications.
For example, developing good representations of the data
can lead to a compression scheme with tolerable losses.
A tolerable level of mistakes in a compression algorithm
might be defined to be the same level of changes to the raw wire signals coming from 
the range of kernel parameters choices for deconvolving wire signals.
If a compression scheme can be achieved, this alleviates IO bottle necks
in executing physics analyses on the large data sets produced by neutrino experiments.

% Representing LArTPC images is also interesting as a test bed for generative models.
% The features in these images can be difficult to learn.

% The optimization problem in side the brackets in Equation~\eqref{eq:1} can be efficiently solved using simple gradient descent.

% At the very high level, a generative model is learnt by first picking a \textit{measure of discrepancy} between $G_{\#}P_{\bm{Z}}$ and empirical measure $\hat{\mathsf{P}}_{\bm{X}}$\footnote{Evidently, $\mathsf{P}_{\bm{X}}$ is only available via samples so the best one can do is to approximate this via the empirical measure... }. The choice of this discrepancy is indeed very important and depending on this choice Wasserstein distance based GANs were proposed and analyzed in \cite{xx}, f-GANs were proposed and analyzed in \cite{xx},... These methods directly learn to generate data from base distribution. Clearly, one can also use an AE, 

% \subsection{Autoregressive models and PixelCNN}



% \textcolor{blue}{We are using an autoregressive model for the latent code - Is this the reason why we see that the images that are generated have cropped or shorter trajectories? - may be the autoregressive model doesn't have a long term memory. Can we use a OT approach here? - may be this is future work.}

% \subsection{Why LArTPCs for generative models?} 



% Background on PixelCNN.
\section{Methods}

The training of a the generative model proceeds in two phases
and follows the work in~\cite{van2017neural}.
The first phase is to train a VQ-VAE network to 
properly reconstruct images.
Through this process, the VQ-VAE network
learns a map from detector images to
a latent "code" image where each pixel 
is assigned the index of one of k-vectors in a d-dimensional
feature embedding space.
In the next phase, a set of training images are 
 mapped into a code image.
A PixelCNN network \cite{salimans2017pixelcnn++} is then trained to learn the prior 
over the latent code indices of these images.
Once trained, the PixelCNN  can be used to generate a novel
code image.
This code image is then passed into the decoder of the VQ-VAE
in order to generate a detector image.
For the training data, we used publicly available examples 
of LArTPC images produced by the DeepLearnPhysics collaboration. 
The images contain trajectories from one of five possible particle species:
$e^-$, $\gamma$, $\mu^-$, $\pi^+$, or proton ($p$).
% In this study, the size of the images are reduced from 256x256 
% to 64x64 by cropping in regions near the image centroid.
% Only crops containing a minimum pixel sum are kept.
% A training set of 50k images and a test set of 10k images were made. 
% The training set is used to train both the VQ-VAE and the PixelCNN.
% Likewise, the test set is used to validate both.
%Please see Section~\ref{sec:vqvae_details} in the supplement 
%for details of the implementation.
Please see the supplement for details of the implementation.

In order to provide a measure of image quality, we studied 
the output of a semantic segmentation network (SSNet)
trained to classify individual pixels as examples
from one of two categories: track or shower.
These categories come from physics of how particles travel through matter.
Heavier charged particles, which include the pion, proton, and muon, travel
primarily along a linear path often referred to as a ``track."
Electrons, which have a much smaller mass, are more easily deflected
by electromagnetic (EM) interactions with atoms.
Furthermore, EM interactions can induce the creation of photons
which, being electrically neutral, do not produce a
visible trajectory for some distance before possibly interacting 
and producing a new electron or an electron-positron pair. 
This can repeat and result in a cascade of trajectories 
referred to as an EM shower, or ``shower".
Track and shower labels are useful to LArTPC analyses and already some form of this
network is currently in use by experiments. 
Therefore, we use the similarity in the SSNet output between
real and generated images as a proxy for how well 
the generative model can reproduce LArTPC image features and thus a proxy of the ``visual quality" of generated images.
We implement a network based on the work in~\cite{abratenko2020semantic}.
%Details of this implementation can be found in Section~\ref{sec:ssnet_details} of the supplement.
Details of this implementation can be found in the supplement.

\section{Results}

Images generated by the PixelCNN+VQ-VAE decoder contain patterns of charge that
resemble both track- and shower-type trajectories in LArTPCs.
Figure~\ref{fig:gen_examples} provide examples of both generated and training images.
There are more samples in the supplement.
Visual inspection leads to the following observations.
Local patches of the generated images are beginning to resemble those from LArTPCs.
In particular, "v"-shape or branching structures characteristic of 
shower trajectories are visible.
Extended lines without branching, characteristic of ``tracks", are also observed.
Isolated shower-like trajectories we believe could fool an expert.
However, taken as a whole, the generated trajectories still have clear flaws. 
Overall, there seems to be a bias towards producing shower-like features.
The network, furthermore, has trouble producing extended structure for 
longer tracks and larger showers. 
Tracks seem to be shorter in the generated than test images.
Larger shower-like regions often do not exhibit the structure one might expect.
% Improving the coherence of larger trajectories is something that
% can be addressed by techniques such as those in~\cite{razavi2019generating}
% where latent code images are produced at different image scales.
% At times, there are connections between track and shower trajectories
% that look nonphysical, e.g. a shower might lead into a track trajectory
% which usually does not occur in the detector.

We compliment visual inspection with a study of a track-shower pixel labeling network. 
Figure~\ref{fig:ssnetstudy} shows the fraction of pixels with a given label score
for both generated and test images.
Ideally, if the generated images were indistinguishable from the test images, 
there would be little different in the histograms.
We see, however, an increased number of pixels labeled as shower-like in
the generated images and a relative deficit of pixels categorized as track-like.
This correlates with what was visually observed.
We also compared two configurations where the number of quantized vectors 
was halved to 256.
We found the visual quality to be reduced.
This was corroborated in the SSNet scores through: 1) less track pixels per image, 
2) a further excess in shower pixels, and 3) a larger population of lower confidence shower pixels.

%\afterpage{
% Figure environment removed

% Figure environment removed
%\clearpage
%}

\section{Conclusions}

We present work towards a generative model which can produce convincing LArTPC images.
As far as we know, this is the first demonstration of a generative network
that produces shower trajectories and the first that produces track trajectories 
without an underlying physics-based model.
The model produces patterns that do resemble track and shower trajectories,
but there is clear need for improvement.
Primarily, features at larger scales are difficult for the network.
There are avenues, such as the use of 
hierarchical code maps at different image scales in~\cite{razavi2019generating}, 
that could improve these.
Furthermore, the way that the PixelCNN calculates the probabilities can better 
capture the time-evolution of particle trajectories using tools such as those in~\cite{jain2020locally}, 
e.g. convolutions proceed from the center out rather in the current raster-scan order.
We also plan to move towards conditional generation where the particle species and momenta can be specified. 
This gets us closer towards applying these models to event reconstruction.
Finally, the types of features found in LArTPC images have a different nature than the common data sets that the machine learning community develops on.
This, we believe, makes LArTPC images an interesting data set for studying different approaches in generative modeling.

\subsubsection*{Acknowledgments}

% Use unnumbered third level headings for the acknowledgments. All
% acknowledgments, including those to funding agencies, go at the end of the paper.

This material is based upon work supported 
by the U.S. Department of Energy (DOE)
and the National Science Foundation (NSF).
T.W. was supported by the U.S. DOE, 
Office of High Energy Physics under Grant No. DE-SC0007866. 
S.A. was funded by the NSF under CAREER Award No. CCF:1553075.
This work was also supported by the National Science Foundation under Cooperative Agreement PHY-2019786 (The NSF AI Institute for Artificial Intelligence and Fundamental Interactions, http://iaifi.org/).
\clearpage

\bibliography{mybib}
\bibliographystyle{iclr2021_conference}

\clearpage

\appendix

{\LARGE  SUPPLEMENTARY to ``Towards Designing and Exploiting Generative Networks for Neutrino Physics Experiments using Liquid Argon Time Projection Chambers."}

\beginsupplement
\begin{refsection}

\section{Pseudocode Example of Cumulative Disruption Algorithm} \label{sec:psuedocode}

For readers seeking a succinct code-like description of our cumulative disruption curve algorithm, we have included \cref{lst:psuedocode}.

\begin{lstlisting}[label=lst:psuedocode, language=Python, caption=Pseudocode for disruption algorithm]
disruption = []
for c in communities:
    remaining = 0
    original = 0
    removeCommunity(c)
    for user in users:
        if degree(user) > 0:
            remaining += degree(user)
            original += originalDegree(user)
    disruption += [1 - (remaining / original)]
\end{lstlisting}

Note that when calculating disruption on large networks, it is much more efficient to cache the size of the smallest community that each user participates in. We can then sort all users by the order in which they will be removed, and avoid computationally expensive references to a graph or adjacency matrix for each removal-step in the algorithm.

\section{Applications to Unipartite Networks} \label{sec:unipartite}

Our influence metric is intended for settings with clearly defined communities. For example, participation in subreddits, membership on a Mastodon server, or committing to a software code repository, all discretely identify users as members of those explicitly-bounded groups. However, network data is often presented in a unipartite configuration such as users following other users. If it is still desirable to delineate communities and measure their influence in these settings, then they can be converted into compatible bipartite networks using the following procedure:

\begin{enumerate}
    \item Apply a context-appropriate community detection algorithm to label each user as belonging to one community

    \item Create a vertex for each community

    \item Replace all user-user edges with user-to-community edges, where the edge weight is equal to the number of unipartite edges each user had to other nodes in that community

    \item Apply our influence metric to the resulting bipartite graph
\end{enumerate}

An example of this procedure is illustrated in \cref{fig:unipartite}, using a unipartite Watts-Strogatz small-world network (100 nodes, 5 neighbors, rewiring probability of 5\%), and label-propagation for community detection. The unipartite graph is shown in the top-left with community labels visualized with color. It is converted to a bipartite representation shown in the upper-right, and the effect of removing each community is illustrated in the bottom frame.

% Figure environment removed






\section{Calculating the Area Under the Disruption Curve} \label{sec:auc_explanation}

For \cref{fig:real_networks_auc,fig:toy_networks_auc,fig:assortativity_auc} we use the area under the disruption curve as a single-variable summary of how centralized a network is around its largest communities. To calculate the AUC, we use a trapezoidal approximation in logarithmic space.

We chose a trapezoidal approximation to calculate the area even with limited sample points from real-world networks. Integration is possible for purely analytic disruption curve simulations as in \cref{sec:analytic_simulations}, but this is not feasible for our non-Erd\H{o}s-R\'{e}nyi networks, so we use a trapezoidal approximation for all synthetic networks for consistency.

We measure the AUC in logarithmic space, because measuring in linear space would heavily weight the influence of the smallest communities that are removed last, and our primary interest is in examining the influence of the largest communities on the broader population. 

\section{Synthetic Network Topology Details} \label{sec:toy_examples}

We measure centralization on a variety of synthetic networks introduced in \cref{sec:disruption_toy}. In this section, we include further description and visualization of the synthetic networks used.

Bipartite Near-Star networks are analogous to a unipartite star network with duplicate edges, but in a bipartite setting. Starting with a unipartite star, replace each edge from the hub to a leaf with a two-path from the hub community to a new ``user" vertex, to the leaf community. Duplicate edges from the unipartite hub to leaves are converted into multiple users that share a community, and serve to break ties when pruning communities for disruption curves. This is illustrated in \cref{fig:star}.

% Figure environment removed

For our ``Powerlaw" networks we follow a bipartite configuration model. We first create vertices representing the desired number of communities and users. We then draw from a powerlaw distribution with an assigned $\gamma$ exponent, and assign the drawn degree to each community. Then, we create a corresponding number of edges, wiring each community to users drawn uniformly at random without replacement. This yields networks where communities follow a powerlaw degree distribution, while users follow a normal degree distribution.

Bipartite community-user networks can be visualized in a flat plane, as in \cref{fig:centralization-pl}, or as a multi-layer graph, as in \cref{fig:pl-toy}. A multi-layer representation may be beneficial for representing inter-community relationships that are not explained by shared users, such as Mastodon federation agreements, or shared moderator staff in two subverses. However, these multiplex relationships were deemed out-of-scope for our current work.

% Figure environment removed




\begin{comment}
  #data structure for the dispersion metric
  D = np.zeros(nm)

  #calculate dispersion
  cumu_sum=0
  for n in np.arange(0,nm):
    cumu_sum += n*pn[n]
    #calculate U_n
    if pn[n]==0:
      continue
    Pnpm = Pnm[n,:]/np.sum(Pnm[n,:])
    U=0
    for m in np.arange(0,mm):
      if(np.sum(Pnm[:,m])>0):
        Pnmp = Pnm[:,m]/np.sum(Pnm[:,m])
        prob = np.sum(Pnmp[n:-1])
        U+=Pnpm[m]*prob**(m-1)
    D[n] = n*pn[n]*(1-U)/(cumu_sum-n*pn[n]*U)
\end{comment}

\section{Mathematical Analysis of Disruption in Random Networks} \label{sec:analytic_simulations}

We here calculate the disruption curves for random bipartite networks parameterized by their joint-degree distribution. This approach therefore fixes the distribution $\lbrace g_m \rbrace$ of communities $m$ per user, the distribution $\lbrace p_n \rbrace$ of community size $n$, and the joint-distribution $P_{n,m}$ for the degree of the node and community involved in a random bipartite link. Beyond these constraints, the networks are fully random but allow us to explore the role of heterogeneous connectivity at the user and community level as well as the impact of correlations between both levels.

We wish to calculate the disruption $D(n)$ involved when removing communities of size $n'<n$ in these random networks. By definition of the bipartite network, we know that $np_n$ edges are removed when removing communities of size $n$. Once again, we define disruption as the fraction of \textit{remaining} edges disrupted by communities of size $n$ during the pruning process. It is thus given by the number of edges that belong to communities of size $n$ minus the fraction $u_n$ of those that are the sole edge of the corresponding users (since these users are removed in the pruning) divided by the number of edges belonging to communities of size equal or smaller than $n$ minus the $u_nnp_n$ users removed. We write:

\vspace{2em}
\begin{equation}
    D(n) = \frac{
            \eqnmarkbox[NavyBlue]{bigedges}{np_n}
            -
            \eqnmarkbox[OliveGreen]{prunededges}{u_nnp_n}
        }{
            \eqnmarkbox[WildStrawberry]{remainingedges}{\sum_{n'\leq n}n'p_{n'}}
            -
            \eqnmarkbox[OliveGreen]{prunededges2}{u_nnp_n}
        } \; .
    % Here's Laurent's original expression
    %D(n) = \frac{np_n-u_nnp_n}{-u_nnp_n + \sum_{n'\leq n}n'p_{n'}} \; .
\end{equation}
\annotate[yshift=1em]{above,left}{bigedges}{Edges to comms. of size n}
\annotate[yshift=1em]{above,right}{prunededges}{Edges to removed users}
%\annotate[yshift=-1em]{below,right}{prunededges2}{Edges for removed users}
\annotate[yshift=-0.5em]{below}{remainingedges}{Edges to comms. n or smaller}
\vspace{2em}

The quantity $u_n$ can also be defined as the probability that a random user of a community of size $n$ has no community smaller than $n$. It can therefore be calculated like so:

\vspace{1em}
\begin{equation}
    u_n = \mathlarger{\sum}_m 
        \eqnmarkbox[NavyBlue]{users_in_n_with_m}{\frac{P_{n,m}}{\sum_{m'}P_{n,m'}}}
        \left(
            \eqnmarkbox[OliveGreen]{users_with_m_larger_than_n}{\frac{\sum_{n'\geq n} P_{n',m}}{\sum_{n'}P_{n',m}}}
        \right)^{m-1} \; .
    %u_n = \mathlarger{\sum}_m \frac{P_{n,m}}{\sum_{m'}P_{n,m}} \left(\frac{\sum_{n'\geq n} P_{n',m}}{\sum_{n'}P_{n',m}}\right)^{m-1} \; .
    \label{eq:un}
\end{equation}
\annotate[yshift=1em]{above,right}{users_in_n_with_m}{Fraction of users in comm. \\ \sffamily \footnotesize size n that have m edges}
\annotate[yshift=-0.5em]{below,left}{users_with_m_larger_than_n}{Fraction of users with m edges\\ \sffamily \footnotesize in comms. larger than size n}
\vspace{2.5em}

In the previous equation, we sum over every possible type of node in a community of size $n$, which will have a number of \textit{other} communities $m-1$ proportional to $P_{n,m}$, and ask for all of these communities to be larger or equal to $n$, which will be proportional to the sum of $P_{n',m}$ over all $n'$ larger or equal to $n$. Normalizing the probabilities appropriately yields Eq. (\ref{eq:un}) as written.

Note that these equations assume that edges are unweighted, and that there are no duplicate edges, which is what we expect from an infinite random simple graph. In our real-world data sets there are often duplicate edges (for example, one user following several different users on a Mastodon instance), which we compress to weighted edges for convenience.

Despite this difference between the analytical expression and real socio-technical networks, the analysis of random infinite graphs can be useful to test how disruption is impacted by simple network statistics such as degree distributions or correlations in the joint community-user degree matrix $P_{n,m}$. 

In a simple experiment, we create a random Erd\H{o}s-R\'{e}nyi-like bipartite network and correlated equivalent networks with the same degree distributions and variable community-user degree matrices $P_{n,m}$. The random network has a simple $P^{\textrm{rand}}_{n,m} \propto np_n mg_m$ (normalized) which we can modify manually. To do so, we calculate the maximally correlated $P^{\textrm{max}}_{n,m}$ by assigning users with highest degrees $m_{\textrm{max}}$ to the largest communities available before doing the same to users with the next higher degree and so on all the way down. We can do the same to calculate $P^{\textrm{min}}_{n,m}$ by assigning users with the lowest degree to the largest communities and working our way up in the user degree distribution. We can then create arbitrary community-user degree matrix $P_{n,m}$ by interpolating between linearly with $(1-\rho) P^{\textrm{rand}}_{n,m} + \rho P^{\textrm{max}}_{n,m}$ or $(1-\rho) P^{\textrm{rand}}_{n,m} + \rho P^{\textrm{min}}_{n,m}$.

Our results are shown in \cref{fig:assortivity_random_networks}. We find that positive user-community degree correlations increase disruption and therefore \textit{centralizes} the resulting socio-technical network. Conversely, negative correlations decreases correlations and \textit{decentralizes} the network. That being said, the relative effect of correlations is relatively small as the networks are still otherwise completely random.

% Figure environment removed

\section{Further Analysis of Assortativity} \label{sec:supplemental_assortativity}

There are multiple interpretations of degree assortativity in a bipartite setting. The linear correlation between user degrees and community degrees measures whether high-degree users are likely to be connected to high-degree communities. In our network definitions edges represent activity, like follow relationships or participation in conversations, so this measures whether active users are likely to be connected to communities with lots of activity. However, a second metric of interest is whether large communities are likely to be connected to other large communities, or in other words, the  assortativity of a unipartite-projected community-community graph. This can also be broken into two sub-cases: assortativity of community size (do communities with many users share users with other high-population communities), and assortativity of degree (do communities with lots of activity share users with other high-activity communities).

These three notions of assortativity are not independent; we might expect that users with lots of activity are active in communities with high populations, and may act as bridges between multiple communities with high activity and high population. However, the three metrics are not guaranteed to correlate and should be measured separately.

While rewiring to promote user-community degree assortativity, we also plotted the changes in community-community degree assortativity, shown in \cref{fig:assortivity_user_vs_community}. Strikingly, the community assortativity \textit{decreases} as we rewire to promote user assortativity. This is because as we rewire edges to focus user connections on the largest communities we implicitly decrease the number of edges between communities. This also matches the changes in disruption in \cref{fig:assortativity_auc}: increasing assortativity may reconnect large and insular communities with the rest of the network, briefly increasing their influence, but continued assortativity rewiring also cuts bridges to and between smaller communities, yielding a sparse network that is far less centralized.

% Figure environment removed

To further explore the relationship between these types of assortativity, we also rewired networks in the reverse direction: for randomly selected pairs of edges, we rewired those edges to \textit{decrease} user to community activity assortativity. We have plotted the change in disruption curves (\cref{fig:disassortative_auc}) and correlation between assortativity metrics (\cref{fig:disassortivity_user_vs_community}). In most networks, decreasing activity assortativity lowers centralization, although the effect diminishes as the network topology more closely approximates a random network. The one exception is the Penumbra; this network has such sparse inter-community connections that any perturbation of edges increases the cross-community links and therefore \textit{increases} centralization.

% Figure environment removed

% Figure environment removed

\section{Cumulative Impact on Giant Component Size} \label{sec:giant_components}

Some readers may be interested in how removing large communities influences the giant component size on each network. This is closely related to the cumulative population size in the top sub-plots of \cref{fig:real_networks_size_comparison} and \cref{fig:toy_networks_size_comparison}. Intuition suggests that the size of the giant component will be inversely proportional to the number of cumulative communities removed; as more large communities are pruned, the giant component should shrink. This relationship holds so long as the remaining communities are interlinked, but falters once a ``bridge" community is removed and the giant component splinters. Therefore, sparsely connected networks where bridges are more prominent will have a chaotic giant component size, while more densely connected networks will present a smooth curve until most communities are pruned. This relationship is illustrated in \cref{fig:real_giant_component}. Most curves are smooth until the tail of the distribution, with two notable exceptions: Voat's giant component changes once the largest insular communities are removed (see \cref{fig:voat_render}), and the Penumbra's curve is much ``spikier" as a result of its highly sparse structure.

% Figure environment removed

Measuring the change in giant component size captures some of the same features as our disruption metric. In particular, removing large insular communities may not change the giant component size if the community is completely isolated from the giant component, so this captures some aspect of both the size and topological role of a community. However, the impact of a community is boolean: if it touches the giant component, then removing the community will shrink the giant component by the size of that community. There is no distinction between a minimally integrated and tightly integrated community. Measuring the impact of a community in terms of fraction of edges severed, rather than component vertex size, offers finer insight into the interplay between size distribution and network structure.



\section{Comparison to Network Bottlenecking} \label{sec:cheeger}

The Cheeger number \cite{cheeger} is a single-valued metric representing how large of a ``bottleneck" inhibits conductance across a graph. It is typically written as:

\vspace{2em}
\begin{equation}
    h(G) = \min \left\{
        \frac{
                \eqnmarkbox[NavyBlue]{cheeger_crossedges}{|\partial A|}
            }{
                \eqnmarkbox[OliveGreen]{cheeger_alledges}{|A|}
            }
        : \eqnmarkbox[WildStrawberry]{cheeger_subset}{A \subseteq V(G)}, 
        \eqnmarkbox[Plum]{cheeger_bounds}{0 < |A| \leq \frac{1}{2} |V(G)|}
    \right\} 
\end{equation}
\annotate[yshift=1.2em]{above}{cheeger_crossedges}{Edges crossing the boundary of A}
\annotate[yshift=-0.2em]{below}{cheeger_alledges}{All edges in+across A}
\annotate[yshift=0.8em]{above}{cheeger_subset}{A is a subset of vertices of G}
\annotate[yshift=-2em]{below,left}{cheeger_bounds}{A contains at most half of all vertices}
\vspace{2em}

Our measurement of how much a community influences a larger population, and the Cheeger measurement of whether a community is a ``bottleneck" bear some conceptual similarities. Therefore, we compare our metric to the Cheeger number in two ways. First, we create a ``local Cheeger number," following an identical equation $\frac{|\partial A|}{|A|}$, but where $A$ is defined as the set of communities we are pruning, rather than via a global search. Second, we estimate bounds on the global Cheeger value of the graph. Since evaluating the graph conductance of all possible subsets of vertices is an NP-hard problem \cite{kaibel2004expansion}, it is impractical to directly measure the Cheeger constant on most large graphs. Fortunately, the Cheeger inequality offers upper and lower bounds on the Cheeger number based on the second eigenvalue of the normalized Laplacian of the adjacency matrix of G as follows:

$$\lambda_2/2 \leq h(G) \leq \sqrt{2\lambda_2}$$

Since they are sparse, these bounds can be calculated even on large real-world datasets. 
Unfortunately, in our tests the bounds are quite wide (see \cref{fig:cheeger}), limiting the utility of this approximation. We have plotted a comparison of the ``local" Cheeger number, bounds of the global Cheeger number, and our disruption metric, for a variety of simulated networks.

% Figure environment removed

\printbibliography[heading=subbibliography]
\end{refsection}


%%%% PLEASE MODIFY supplementary.text %%%%%%%%%

%\section{Formalism for physicists}

% {\LARGE  SUPPLEMENTARY to ``Towards Designing and Exploiting Generative Networks for Neutrino Physics Experiments using Liquid Argon Time Projection Chambers."}

% % \title{Towards Designing and Exploiting Generative Networks for Neutrino Physics Experiments using Liquid Argon Time Projection Chambers}

% % \maketitle


% \section{Additional details on Methods}

% \subsection{VQ-VAE Implementation Details}
% \label{sec:vqvae_details}

% \quad The VQVAE's structure is best explained as a two-part network. 
% The first network is much like a classical autoencoder, the caveat being that vector quantization is applied to the latent codes, 
% and that extra losses are computed to optimize the quantization. 
% The second network is an autoregressive generative model that generates latent vectors one component at a time.\par
% For the encoder and decoder, we use convolutional and deconvolutional neural networks respectively. 
% The CNN's are arranged into blocks. Each block contains a convolutional layer with ReLU activation, a batch-normalization layer, and finally another ReLU activation. Our experience has been that ReLU-BN-ReLU blocks produce sets of quantization vectors that lead to more realistic generation (compared to just BN-ReLU). This should be investigated in future work.
% In the convolutional encoder, all blocks except the last downsample the input image by a factor of two. 
% In the deconvolutional decoder, all blocks except the first upsample the image by a factor of two.
% Encoder outputs are fed through a vector quantization layer before being passed to the decoder. \par
% For the autoregressive model used to model the distribution of quantization vectors over the latent vector,
% we employ a masked and gated PixelCNN model. 


% % Figure environment removed

% \subsection{PixelCNN Implementation Details}

% Our PixelCNN network is composed of six gated, masked, convolutional blocks. 
% Each block is composed of a horizontal and vertical stack. 
% Each stack is composed of masked convolution, a gating layer, and a residual layer. 
% Information from the vertical stack is passed to the horizontal stack. 
% For the vertical stack, gating occurs after the residual layer, 
% while in the horizontal stack gating occurs first. \par
% The convolutional blocks are followed by two convolutional layers with an amount of output filters equal to the number of quantization vectors. 
% The activations are passed into a Softmax function to approximate the likelihood of each quantization vector at that component of the latent code.\par Our VQVAE implementation is based on the works of Ken Leidal and Amelie Royer, 
% which can be found at \url{https://github.com/kkleidal/GatedPixelCNNPyTorch} 
% and \url{https://github.com/ameroyer/ameroyer.github.io} respectively.

% % Figure environment removed


% % % Figure environment removed


% \subsection{Training Data Details}
% \label{subsec:genmodel_training_data}

% We used publicly available examples of LArTPC images produced by
% the DeepLearnPhysics collaboration~\cite{dlpdata}. 
% Among the available data from this source,
% we used the 50k single-particle image data set to train the VQ-VAE.
% We use the separate 40K single-particle image 
% data set to test the reconstructions of the VQ-VAE.
% Images in both the train and test set consist of a 256x256 image~\cite{dlpsingle}.
% The particle generated for each image is chosen among five species: $e^-$, $\gamma$, $\mu^-$, $\pi^+$, and protons ($p$).
% The momentum of each particle is chosen from a uniform distribution between the following ranges 
% ($e^-$) 35.5 to 800 MeV/c, 
% ($\gamma$) 35 to 800 MeV/c, 
% ($\mu^-$) 90 to 800 MeV/c,
% ($\pi^+$) 105 to 800 MeV/c, and
% ($p$) 105 to 800 MeV/c.
% The particle is simulated inside a large volume of argon.
% It is propagated via Geant4. Afterwards,  a 2.56 m$^3$ box is chosen in 3D that
% maximizes the particle's trajectory within the volume and recorded in the file.
% The 2D images are created as 2D projections ($xy$, $yz$, $zx$) of the 3D charge depositions.
% All images included in the data set are guaranteed to have 2D projection images with at least 10 non-zero pixels.
% We use the $zx$ projections.

% \subsection{Generator Model Training Procedure Details}

% % \textcolor{red}{Needs meta parameter details.}


% \begin{table}[h!]
% \centering
%     \begin{tabular}{c c c c c}
%     Num. Quantization Vectors  & Quant. Vector Dim.  & Enc. Filters & Dec. Filters.  & Batch Size\\
%     \hline
%     512 & 8 & [16, 32] & [32, 16] & 512
% \end{tabular}
% \caption{VQVAE Meta Parameters}
% \label{tab:VQVAE Metaparameters}
% \end{table}
% \begin{table}[h!]
% \centering
%     \begin{tabular}{c c c}
%     Num. PixelCNN Blocks & Num. Filters per Block &  Batch Size\\
%     \hline
%     6 & 128 & 512
% \end{tabular}
% \caption{PixelCNN Meta Parameters}
% \end{table}

% We train the network in two parts. 
% First, the autoencoder is trained on a set of fifty thousand, 64x64, single channel LArTPC images. 
% Training is conducted with a  batch size of 512 images and with the Adam optimizer initialized with a learning rate of 3e-4.\par
% For a single training loop, three losses are computed. 
% First is the mean squared error between the reconstructed image and the true image. 
% This loss is propagated through the decoder and encoder. 
% Because the vector quantization process contains an argmin operation, 
% which gradient can not pass through, 
% gradients are copied from the beginning of the first layer of the decoder to the last layer of the encoder.\par
% Bypassing the quantization layer means that the codebook must be learned independently of the reconstruction error. 
% We use an L2 loss between the encoder's unquantized outputs and quantized outputs (with a stop-gradient operation applied to the unquantized outputs) to learn the quantization vectors. 
% To ensure that the encoder commits to a specific set of quantization vectors,
% a second L2 loss is introduced between the unquantized outputs and the quantized outputs, 
% with the stop-gradient being applied this time to the quantized outputs.\par
% The PixelCNN network is trained using the cross entropy loss between the quantization vector predictions and the true quantization vector. We use the Adam optimizer and initialize it with a learning rate of 1e-3.




% % \section{Validation of PixelCNN training}

% % \textcolor{red}{If we are going to have this section, this would contain Paul's  likelihood distribution comparisons.}

% % % Figure environment removed

% \subsection{Training of the Track/Shower labeling model}
% \label{sec:ssnet_details}

% The quality of the generated images is quantified using a convolutional
% semantic segmentation network (SSNet) modeled after the network described in
% ~\cite{abratenko2020semantic}.
% The architecture of the track/shower semantic segmentation network
% is the same in~\cite{abratenko2020semantic} with the one exception being that
% dense convolutions are used instead of sparse submanifold convolutions.
% The network is structured as a U-Net with ResNet layers.
% Four pairs of down-sampling and up-sampling layers are combined
% before a convolution layer outputs three classes: background, shower, and track. 

% The images used to train the track/shower network
% is related to the data used to train the generative network.
% The two data sets were created using the same traditional simulation chain.
% The track/shower data set is different
% because it contains the pixel-wise truth labels.
% The training sample consisted of 15k 256x256 examples.
% A test set with 10k was used to monitor the training of the network for over-fitting.
% The network was trained using Adam with a batch size of 16, momentum of 0.9, and a weight-decay of 1.0e-4.
% The network was trained for 20 epochs with a starting learning rate of 1.0e-3.
% The learning rate was cut in half every 4 epochs.
% No significant difference in train versus test set loss and accuracy was seen, 
% so the last saved checkpoint is used for the studies in this work.

% Data augmentation techniques were applied to the training set. 
% This included flipping the image along the horizontal and vertical axis, 
% transposing the image, and scaling the pixel values across an individual 
% images by a random value between 0.90 and 1.1 drawn uniformly.
% The pixel-wise classification accuracy after training was 98.5\% per pixel.


% \section{Additional SSNet Study}

% % \subsection{SSNet output versus PixelCNN training epoch}

% As described in the main text, the output of SSNet is
% used as a metric to  evaluate and compare image quality.
% One validation study we did for this metric was to compare the SSNet output
% on generated samples produced by the same model after several epochs of training.
% Our expectation is that the comparison to the output on test images should
% improve with increasing epoch.
% Figure~\ref{fig:ssnet_vs_epoch} shows two distributions. 
% The first is a distribution of the number of pixels labeled as shower or track per image.
% The second is a distribution of the class score for pixels above threshold.
% The first tells us how often pixels within patches with shower-like or track-like features are produced.
% The second tells us how confident the network is that the pixel is part
% of a region that matches track or shower features.
% Both plots compare distributions computed for generated images for three successive
% PixelCNN training checkpoints to the distribution 
% computed over test images (i.e. not generated). 
% We find that the generated distributions better match the test distribution
% as the model is trained longer.
% In order to provide a score to see improvement, we calculate the KL divergence between the generated distributions and the test distribution.

% % Figure environment removed

% \begin{table}[]
% %% Updated with final_k512_p1-p4
% \centering
%     \begin{tabular}{c|c c c c}
%     Checkpoint  & Num. Shower Pix.  & Shower Score & Num. Track Pix.  & Track Score\\
%     \hline
%     \#1: 100 epochs  & 0.36 & 0.22 & 0.41 & 0.21 \\
%     \#2: 200 epochs  & 0.42 & 0.19 & 0.22 & 0.19\\
%     \#3: 300 epochs  & 0.37 & 0.17 & 0.22 & 0.17\\
%     \#4: 400 epochs  & 0.24 & 0.13 & 0.18 & 0.13
% \end{tabular}
% \caption{KL-divergence between SSNet output distributions calculated 
% from generated images versus test images. 
% Smaller values are better.
% This table shows the metric for successive PixelCNN training checkpoints.}
% \label{tab:ssnet_score_vs_epoch}
% \end{table}


% % \subsection{SSNet output versus number of VQ-VAE code-vectors, $k$}

% % We compared the image quality for those generated using a VQ-VAE with
% % two different values for the number of quantized vectors, $k$: 256 and 512.
% % KL-divergence of the $k=512$ models were lower for the different SSNet distributions we checked. 
% % For the number of track pixel distribution, the KL-divergence was
% % 0.18 and 0.21, for the $k=512$ model and $k=256$ model, respectively. 
% % For the number of shower pixel distribution, 0.24 versus 0.55;
% % for the track score  distribution, 0.13 versus 0.22; 
% % and for the shower score distribution, 0.13 versus 0.23.


% % % Figure environment removed

% % We performed an experiment to 

% \section{Publicly available code, models, and generated image sets}
% \label{sec:publiccode}

% The code implementing all the models discussed here can be found on github. 
% Model weights for the VQ-VAE, PixelCNN,
% and track/shower semantic segmentation network
% can be found on Zenodo. 
% A sample of generated images are also provided on Zenodo.
% URLs are not given here in order to respect the double-blind review process,
% but will be made available after the review.
% % The VQ-VAE encoder and decoder weights are at X.
% % The PixelCNN model weights are at X.
% % Finally, we provide a 10k-image sample of generated images.
% % The generated sample can be downloaded at X.

% % The URL is not given here in order to preserve blindness. at \texttt{github.com/NuTufts/LARTPC-VQVAE}.

% \section{Additional Example Images}
% \label{sec:additional_examples}

% In this section, we provide additional image samples to view.

% % \afterpage{%
% % % Figure environment removed
% % \clearpage
% % }

% % \afterpage{%
% % % Figure environment removed
% % \clearpage
% % }

% \afterpage{%
% % Figure environment removed
% \clearpage
% }

% \afterpage{%
% % Figure environment removed
% \clearpage
% }

% %\DeclareCaptionLabelSeparator{}{}

% % The \nocite command causes all entries in a bibliography to be printed out
% % whether or not they are actually referenced in the text. This is appropriate
% % for the sample file to show the different styles of references, but authors
% % most likely will not want to use it.
% %\nocite{*}





\end{document}
%
% ****** End of file apssamp.tex ******
