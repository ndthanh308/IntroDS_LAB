\documentclass[conference, 10pt]{IEEEtran}
\IEEEoverridecommandlockouts

% The preceding line is only needed to identify funding in the first footnote. If that is unneeded, please comment it out.
\usepackage{url}
\usepackage{cite}
\usepackage{amsmath,amssymb,amsfonts}
\usepackage{algorithmic}
\usepackage{graphicx}
\usepackage{textcomp}
\usepackage{xcolor}
\def\BibTeX{{\rm B\kern-.05em{\sc i\kern-.025em b}\kern-.08em
    T\kern-.1667em\lower.7ex\hbox{E}\kern-.125emX}}

\usepackage{enumitem}

\usepackage{romannum}
\usepackage{textcomp}
\usepackage{tabularx}
\usepackage{diagbox}
\usepackage{multirow}
\usepackage{hhline}

\usepackage{bbding}
\usepackage{pifont}
\usepackage{wasysym}

\usepackage{cleveref}
\crefformat{section}{\S#2#1#3}
\crefformat{subsection}{\S#2#1#3}
\crefformat{subsubsection}{\S#2#1#3}

% for source codes
\usepackage{listings}
\usepackage{color}
\definecolor{dkgreen}{rgb}{0,0.6,0}
\definecolor{gray}{rgb}{0.5,0.5,0.5}
\definecolor{mauve}{rgb}{0.58,0,0.82}
\definecolor{backcolour}{rgb}{0.95,0.95,0.92}
\lstset{
  %frame=tb,
  language=C++,
  backgroundcolor=\color{backcolour},
  commentstyle=\color{dkgreen},
  keywordstyle=\color{blue},
  numberstyle=\tiny\color{gray},
  stringstyle=\color{mauve},
  basicstyle={\small\ttfamily},
  breaklines=true,
  breakatwhitespace=false,
  showstringspaces=false,
  aboveskip=2mm,
  belowskip=2mm,
  columns=flexible,
  numbers=left,
  numbersep=5pt,
  captionpos=b,
  tabsize=2,
  escapeinside={(*@}{@*)}
}

% for figures & subfigures
\usepackage{graphics}
\usepackage{caption}
\usepackage{subcaption}
\graphicspath{{./figures/}}

\usepackage{amsmath}
\usepackage{mathrsfs}

\begin{document}

\title{FLiCR: A Fast and Lightweight LiDAR Point Cloud Compression Based on Lossy RI}
\author{}

\author{\IEEEauthorblockN{Jin Heo}
\IEEEauthorblockA{\textit{Georgia Institute of Technology} \\
Atlanta, Georgia, USA \\
jheo33@gatech.edu}
\and
\IEEEauthorblockN{Christopher Phillips}
\IEEEauthorblockA{\textit{Adeia} \\
Hartwell, Georgia, USA \\
chris.phillips@adeia.com}
\and
\IEEEauthorblockN{Ada Gavrilovska}
\IEEEauthorblockA{\textit{Georgia Institute of Technology} \\
Atlanta, Georgia, USA \\
ada@cc.gatech.edu}
}


%\renewcommand\footnotetextcopyrightpermission[1]{}

% Additional packages

\pagenumbering{arabic}
%\thispagestyle{plain}

\newcommand{\jin}[1]{\textcolor{blue}{JH: #1}}
\newcommand{\ada}[1]{\textcolor{red}{AG: #1}}
\newcommand{\blue}[1]{\textcolor{blue}{#1}}
\newcommand{\red}[1]{\textcolor{red}{#1}}
\newcommand{\green}[1]{\textcolor{green}{#1}}
\newcommand{\mul}{$\times$}

\newenvironment{tightitemize}%
 {\begin{list}{$\bullet$}{%
 		\setlength{\leftmargin}{10pt}
        \setlength{\itemsep}{0pt}%
        \setlength{\parsep}{0pt}%
        \setlength{\topsep}{0pt}%
        \setlength{\parskip}{0pt}%
        }%
 }%
{\end{list}}

\newcommand{\specialcell}[2][c]{%
  \begin{tabular}[#1]{@{}c@{}}#2\end{tabular}}


\maketitle
\thispagestyle{plain}
\pagestyle{plain}

\begin{abstract}
Light detection and ranging (LiDAR) sensors are becoming available on modern mobile devices and provide a 3D sensing capability.
This new capability is beneficial for perceptions in various use cases, but it is challenging for resource-constrained mobile devices to use the perceptions in real-time because of their high computational complexity.
In this context, edge computing can be used to enable LiDAR online perceptions, but offloading the perceptions on the edge server requires a  low-latency, lightweight, and efficient compression due to the large volume of LiDAR point clouds data.

This paper presents FLiCR, a fast and lightweight LiDAR point cloud compression method for enabling edge-assisted online perceptions.
FLiCR is based on range images (RI) as an intermediate representation (IR), and dictionary coding for compressing RIs.
FLiCR achieves its benefits by leveraging lossy RIs, and we show the efficiency of bytestream compression is largely improved with quantization and subsampling.
In addition, we identify the limitation of current quality metrics for presenting the entropy of a point cloud, and introduce a new metric that reflects both point-wise and entropy-wise qualities for lossy IRs.
The evaluation results show FLiCR is more suitable for edge-assisted real-time perceptions than the existing LiDAR compressions, and we demonstrate the effectiveness of our compression and metric with the evaluations on 3D object detection and LiDAR SLAM.
\end{abstract}

{\let\thefootnote\relax\footnote{{© 2022 IEEE.  Personal use of this material is permitted.  Permission from IEEE must be obtained for all other uses, in any current or future media, including reprinting/republishing this material for advertising or promotional purposes, creating new collective works, for resale or redistribution to servers or lists, or reuse of any copyrighted component of this work in other works.}}}

\begin{IEEEkeywords}
lidar, lidar point cloud, lidar point cloud compression, 3D point cloud compression, remote lidar perceptions, real-time perception service, range image compression, edge computing
\end{IEEEkeywords}

\section{Introduction}
Current quantum hardware is unable to carry out universal quantum computations due to the buildup of errors that occur during the computation. 
The magnitude of the individual error is currently above the value that the Threshold Theorem requires in order to kick-start quantum error correction and fault-tolerant quantum computation~\cite[Section 10.6]{nielsen_chuang_2010}. 
Although the experimentally achieved fidelity rates are promising and the error bounds are inching closer to the required threshold, we will have to work for the foreseeable future with quantum hardware with errors that build-up during the computation.  This implies that we can only do a limited number of steps before the output of the computation has become completely uncorrelated with the intended one.

For fault-tolerant quantum computing, we repeat four steps: 
1) We apply a number of single and two-qubit quantum gates, in parallel whenever possible; 
2) We perform a syndrome measurement on a subset of the qubits; 
3) We perform fast classical computations to determine which errors have occurred and how to correct them; 
and, 4) We apply correction terms based on the classical computations.
We then repeat these four steps with a next sequence of gates. 
These four steps are essential to fault-tolerant quantum computing. 


The starting point of this work is to use the four steps outlined above, not to carry out error correction and fault-tolerant computation, but to enhance short, constant-depth, {\em uncorrected} quantum circuits that perform single qubit gates and {\em nearest-neighbor} two qubit gates. 
Since in the long run we will have to implement error-correction and fault-tolerant computation anyhow, and this is done by such a four-step process, why not make other use of this architecture? Moreover, on some of the quantum hardware platforms, these operations are already in place.
Embracing this idea we naturally arrive at the question: what is the computational power of \textit{low-depth} quantum-classical circuits organized as in the four steps outlined above? 
We thus investigate circuits that execute a small, ideally constant, number of stages, where at each stage we may apply, in parallel, single qubit gates and {\em nearest-neighbor} two qubit gates, followed by measurements, followed by low-depth classical computations of which the outcome can control quantum gates in later stages. 
It is not clear, at first, whether such circuits, especially with constant depth, can do anything remotely useful. 
But we will see that this is indeed the case: many quantum computations can be done by such circuits in constant depth. 
By parallelizing quantum computations in this way, we improve the overall computational capabilities of these circuits, as we do not incur errors on qubits that are idle, simply because qubits are not idle for a very long time. 
Furthermore, reducing the depth of quantum circuits, at the cost of increasing width, allows the circuit to be run faster even if errors occur.

The first usage of such a four-step layout, not to do error correction, but to perform computations, can be found in the paradigm of measurement-based quantum computing~\cite{gottesman1999demonstrating,raussendorf2001one,jozsa2006introduction,clark2007generalised}: 
A universal form of quantum computing where a quantum state is prepared and operations are performed by measuring qubits in different bases, depending on previous measurements and intermediate measurements.

\citeauthor{PhamSvore2013} were the first to formalize the four-step protocol for performing computations~\cite{PhamSvore2013}. They included specific hardware topologies by considering two-dimensional graphs for imposing constraints on qubit interactions. In their model, they develop circuits for particularly useful multi-qubit gates, including specifying costs in the width, number of qubits, depth, number of concurrent time steps, size, and total number of non-Identity operations.
As a result, they find an algorithm that factors integers in polylogarithmic depth.
\citeauthor{Browne:2011} showed that the main tool in the work by \citeauthor{PhamSvore2013}, the fan-out gate, can also be replaced by additional log-depth classical computations in the measurement-based quantum computing setting~\cite{Browne:2011}.

More recently, \citeauthor{Cirac:2021} introduced a scheme to implement unitary operations involving quantum circuits combined with Local Operations and Classical Communication ($\mathsf{LOCC}$) channels: $\mathsf{LOCC}$-assisted quantum circuits~\cite{Cirac:2021}. Similarly to the four-step scheme we just described, they allow for a short depth circuit to be run on the qubits, followed by one round of $\mathsf{LOCC}$, in which ancilla qubits are measured and local unitaries are applied based on the measurement outcomes. They show that in this model any 1D transitionally invariant matrix-product state (MPS) with fixed bond dimension is in the same phase of matter as the trivial state. Similar ideas can be found in~\cite{TVV_NonAbelianTopologicalOrder_2022, tantivasadakarn2021long}.

In this work, we introduce a new model, called \textit{Local Alternating Quantum-Classical Computations} ($\LAQCC$). In this model we alternate between running quantum circuits (constrained by locality), ending in the measurement of a subset of qubits, and fast classical computations based on the measurement results. The outcome of the classical computations are then used to control future quantum circuits. We allow for flexibility in this model, by giving different constraints to the power of both the quantum circuits and the classical circuits as well as the number of alternations between them. 
Most attention will be given to $\LAQCC$ containing quantum circuits of constant depth, classical circuits of logarithmic depth and at most a constant number of alternations between them. 
Any circuit constructed in this model is considered to be of constant depth. 
We restrict ourselves to logarithmic depth classical computations, as this is the first natural and non-trivial extension beyond constant-depth classical computations. 
Constant-depth classical computations do however also have an equivalent constant-depth quantum implementation.

The definition of $\LAQCC$ sharpens the original definition of \citeauthor{PhamSvore2013} by adding constraints to the intermediate classical computations. This allows us to bound the power of $\LAQCC$ from above. 

The main result of \citeauthor{Cirac:2021}, that 1D translational invariant MPS with fixed bond dimension can be prepared by $\mathsf{LOCC}$-assisted circuits, relies on local symmetries of the MPS. These symmetries allow them to prepare local states (on a constant number of qubits) and glue them together by doing one round of the appropriate entangling measurement and corrections, after which they run a round of local unitaries to get the desired result. This general scheme for preparing states that exhibit an MPS description with the appropriate local symmetries requires only geometrically local unitaries and one round of measurement and corrections an therefore is accessible in $\LAQCC$. Studying different local symmetries, known as Symmetry Protected Topological (SPT) phases of matter, to find measurement-based constant depth circuits for states is a broad ongoing field of research~\cite{TVV_NonAbelianTopologicalOrder_2022, tantivasadakarn2021long, smith2023deterministic}. 
All these schemes have a $\LAQCC$ implementation.

%$\LAQCC$-circuits also exist for general schemes of preparing local states, based on the local tensors, and gluing them together using one round of entangled measurement and corrections, based on the local symmetry. 
%The main result of \citeauthor{Cirac:2021}, that 1D translational invariant MPS with fixed bond dimension can be prepared by $\mathsf{LOCC}$-assisted circuits, relies heavily on local symmetries of the MPS and as a result also has an equivalent $\LAQCC$ implementation. 
%The corrections applied after the measurement round are local unitaries depending on the local symmetries of the MPS. 

 

%This general scheme of preparing local states, based on the local tensors, and gluing it together by doing one round of entangled measurement and corrections, based on the local symmetry, is accessible in $\LAQCC$.
Note however that \citeauthor{Cirac:2021} also suggest a circuit for the $W$-state.
This circuit uses sequentially and dependent measurement-based corrections of the ancilla qubits. 
These dependent measurements translate to sequential alternations between the quantum and classical circuits and therefore increase the total depth to linear depth, exceeding the constant-depth constraints imposed by $\LAQCC$-circuits. 

We study the power of the $\LAQCC$ model with respect to state preparation, showing that even with only constant quantum-depth and logarithmic classical depth it remains possible to prepare states with long-range entanglement.
Another surprising result is that it is unlikely that $\LAQCC$ circuits are classically simulatable. We show that any instantaneous quantum polynomial-time (IQP) circuit~\cite{Bremner2010,Shepherd2009} has an $\LAQCC$ implementation.
Classical simulation of IQP circuits implies the collapse of the polynomial hierarchy to the third level, which is not believed to be true~\cite{Bremner2017}. Therefore, we expect that $\LAQCC$ circuits are unlikely to be classically simulatable. We bound the power of $\LAQCC$ by showing that it is contained in $\QNC^1$, the class of polynomial-size, log-depth circuits.

Next, we also study the power that intermediate classical calculations can add to quantum computations, by considering a new model that alternates between polynomially many polynomial-depth quantum circuits and unbounded classical computations
We study this model by doing a complexity theoretical analysis, where we draw inspiration from the notions of complexity given by \citeauthor{RosenthalYuen:2022}, \citeauthor{MetgerYuen:2023}, and \citeauthor{Aaronson:2004}.
All three complexity notions are based on the notion of state preparation, instead of more traditional definition of complexity such as the decidability of a computational problem. 
The first two consider classes based on sequences of quantum states preparable by a polynomial-sized quantum circuit, where the circuits are uniformly generated by a computational class, for instance, the class $\mathsf{PSPACE}$, which results in the complexity class $\mathsf{StatePSPACE}$~\cite{RosenthalYuen:2022,MetgerYuen:2023}.
The third notion considers a relative complexity, where the complexity is measured between two given states, and is measured by the number of gates, from a given gate-set, required to transform one state in another state~\cite{Aaronson:2004}. 
For our definition of state preparation complexity, we drop the uniformity constraint from~\cite{RosenthalYuen:2022,MetgerYuen:2023} and define a class as $\mathsf{StateX}$, which refers to states preparable by circuits of type $\mathsf{X}$. 
As an example, if $\mathsf{X} = \QNC^0$, this results in the class $\mathsf{StateQNC^0}$, which is the set of states preparable from the $\ket{0}^n$ state by poly-size constant-depth circuits. 
This notion is similar to the relative complexity from~\cite{Aaronson:2004}, where one state is the  $\ket{0}^n$ state and instead of counting the number of gates we consider the set of states preparable by a fixed number of gates. Using this notion of complexity we show that any state preparable by an $\LAQCC^*$ circuit is also preparable by a $\mathsf{PostQPoly}$ circuit, the class of circuits of polynomial depth with an additional post-selection gate. 

All Clifford circuits have a constant-depth $\LAQCC$ implementation, implying that any stabilizer state can be implemented by a constant-depth $\LAQCC$ circuit, see Section~\ref{sec:clifford_circuits} for a proof of this statement. 
Efficient circuits for stabilizer states have been known already through measurement-based quantum computing. Therefore this paper focuses on the preparation of non-stabilizer states, and as a surprising result we find novel constant-depth protocols for four very natural classes of non-stabilizer states.
Despite the extensive research into these four classes of non-stabilizer states and the many applications of them, no efficient constant- or low-depth state preparation protocols are known yet. We specifically consider these four classes as they are all often used as initial states in other algorithms.

The first state is a uniform superposition over an arbitrary number of states. 
This state finds applications in many quantum algorithms, as they often start with a uniform superposition over multiple states. 
This superposition is often achieved by applying Hadamard gates to every qubit due to its simplicity to prepare. 
Yet, the analysis of many algorithms, such as Shor's algorithm~\cite{Shor:1997}, would benefit from a different initial superposition. 
The circuit to prepare the uniform superposition over an arbitrary number of states uses an exact version of Grover search as a subroutine, that turns a probabilistic circuit, with a known constant probability of success, into a deterministic circuit. 
We use the circuit for preparing a uniform superposition over an arbitrary number of states as a subroutine in the next two quantum state preparation protocols. 

The second state is the $W$-state, the uniform superposition over all computational basis states of Hamming-weight~$1$, a natural long-ranged entangled state that displays a fundamentally nonequivalent type of entanglement from the Greenberger–Horne–Zeilinger state~\cite{WState:2000}, for which $\LAQCC$-type constant-depth circuits were previously known~\cite{PhamSvore2013, Cirac:2021}. 
The $W$-state is often used as benchmark for new quantum hardware~\cite{Haffner2005,Neeley2010,GarciaPerez:2021}. 
A novel way to prepare the $W$-state therefore gives a new way to benchmark different quantum devices with each other. 
A circuit for preparing the $W$-state was given in~\cite{Cirac:2021}, but this implementation requires sequentially alternating measurements followed by local unitaries, which in the $\LAQCC$ model is not considered to be of constant depth. 
We improve this protocol by giving an $\LAQCC$ implementation of the $W$-state, based on a compress-uncompress method that links the one-hot and binary encoding of integers.

The third state considered is the Dicke state, a generalization of the $W$-state, a superposition over all computational basis states with Hamming-weight $k$~\cite{Dicke:1954}. 
Dicke states have relevance in various practical settings.
For instance, for quantum game theory~\cite{zdemir2007}, quantum storage~\cite{Bacon_Compress:2006,Plesch:2010}, quantum error correction~\cite{ouyang2014permutation}, quantum metrology~\cite{toth2012multipartite}, and quantum networking~\cite{prevedel2009experimental}. 
Dicke states have been used as a starting state for variational optimization algorithms, most notably Quantum Alternating Operator Ansatz (QAOA)~\cite{Hadfield2019}, to find solutions to problems such as Maximum k-vertex Cover~\cite{Brandhofer2022,cook2020quantum}.
The ground states of physical Hamiltonians describing one-dimensional chains tend to show a resemblance to Dicke states such as states resulting from the Bethe ansatz, making them an ideal starting state when investigating the ground state behavior of these Hamiltonians~\cite{TDL_BetheAnsatzDerivation:2010,B_ExcitedStateQuantumPhaseTransitions:2013,DickeTransitions:2021}. 
For instance, the algorithm by \citeauthor{van2021preparing}, who give an algorithm to prepare the Bethe ansatz eigenstates of the spin-1/2 XXZ spin chain, starts by first preparing a Dicke state~\cite{van2021preparing}. 
A Dicke-state preparation protocol based on the compress-uncompress methodology used in the $W$-state furthermore finds applications in entanglement distillation, where the entanglement of a large state is concentrated on only a few qubits. 
Efficient deterministic circuits for preparing Dicke states have been proposed by \citeauthor{bartschi2019deterministic}~\cite{bartschi2019deterministic, bartschi2022deterministic_short_depth}. 
They provide a quantum circuit of depth $\mathO(k \log(\frac{n}{k}))$, allowing arbitrary connectivity, to prepare a Dicke state, which they conjecture to be optimal when $k$ is constant. 
In this work, we provide a constant-depth $\LAQCC$ circuit below their conjectured bound already for constant $k$. 
However, this does not directly disprove their conjecture, as we allow for intermediate measurements and classical computations. 
More significantly, we even construct constant-depth $\LAQCC$ circuits for $k = \mathO(\sqrt{n})$ greatly improving their bound.
This construction extends the compress-uncompress method for the $W$-state combined with additional subroutines. 

We continue with a log-depth state preparation protocol for the Dicke-state for arbitrary $k$. 
This protocol implements an efficient transformation between the factoradic number representation and the combinatorial number representation of a positive integer. 
The combinatorial number representation relates directly to the Dicke state. 
The provided efficient transformation between number representation systems might be of independent interest. 

We conclude by modifying our protocol for preparing a Dicke-state to a protocol that prepares quantum many-body scar states in constant-depth. 
These states have low entanglement and longer coherence times than states with similar energy density.
These characteristics make many-body scar states interesting to analyze and relevant within physics.
Many-body scar states appear for instance in the AKLT model~\cite{AKLT:1987,MRBAR:2018,MRB:2018} and different spin models~\cite{SI:2019,MOBFR:2020}.
Known methods for preparing these states have polynomial-depth~\cite{Gustafson:2023}, whereas our circuit has constant depth. 

% We conclude by studying the power that intermediate classical calculations can add to quantum computations. 
% In this study, we define a new model that relaxes constant-depth quantum circuits to polynomial depth quantum circuits, log-depth classical calculations to unbounded classical computations and a constant number of alternations to a polynomial number of alternations. 
% We call this model $\LAQCC^*$. 
% We study this model by doing a complexity theoretical analysis, where we draw inspiration from the notions of complexity given by \citeauthor{RosenthalYuen:2022}, \citeauthor{MetgerYuen:2023}, and \citeauthor{Aaronson:2004}.
% All three complexity notions are based on the notion of state preparation, instead of more traditional definition of complexity such as the decidability of a computational problem. 
% The first two consider classes based on sequences of quantum states preparable by a polynomial-sized quantum circuit, where the circuits are uniformly generated by a computational class, for instance, the class $\mathsf{PSPACE}$, which results in the complexity class $\mathsf{StatePSPACE}$~\cite{RosenthalYuen:2022,MetgerYuen:2023}.
% The third notion considers a relative complexity, where the complexity is measured between two given states, and is measured by the number of gates, from a given gate-set, required to transform one state in another state~\cite{Aaronson:2004}. 
% For our definition of state preparation complexity, we drop the uniformity constraint from~\cite{RosenthalYuen:2022,MetgerYuen:2023} and define a class as $\mathsf{StateX}$, which refers to states preparable by circuits of type $\mathsf{X}$. 
% As an example, if $\mathsf{X} = \QNC^0$, this results in the class $\mathsf{StateQNC^0}$, which is the set of states preparable from the $\ket{0}^n$ state by poly-size constant-depth circuits. 
% This notion is similar to the relative complexity from~\cite{Aaronson:2004}, where one state is the  $\ket{0}^n$ state and instead of counting the number of gates we consider the set of states preparable by a fixed number of gates. Using this notion of complexity we show that any state preparable by an $\LAQCC^*$ circuit is also preparable by a $\mathsf{PostQPoly}$ circuit, the class of circuits of polynomial depth with an additional post-selection gate. 

\paragraph{Summary of results}
\begin{itemize}
    \item We give a new definition of a computational model that captures the power of the four step process: applying a constant number of layers of one- and two-qubit gates; performing a syndrome measurement; perform a fast classical computation determining corrections; apply corrections. We call this model \emph{Local Alternating Quantum Classical Computations}, or $\LAQCC$ for short. In this model we bound the allowed quantum operations, intermediate classical calculations, and number of rounds separately. In Section~\ref{sec:LAQCC_model} we define this model and give a list of operations based on results from literature contained in this computational model. In some of these operations we explicitly use that we allow for multiple, but at most constant, rounds  of corrections.
    \item  We show show that there exist $\LAQCC$ circuits that can not be weakly simulated in Section~\ref{sec:IQP_in_LAQCC}. We further show that for every $\LAQCC$ circuit there exists a $\QNC^1$ circuit simulating it perfectly, in Section~\ref{sec:LAQCC_in_QNC1}.
    \item We introduce a new type computational complexity for preparing states and show that the extension of $\LAQCC$ where we allow a polynomial number of rounds and unbounded classical computation, is contained in $\mathsf{PostQPoly}$, the class of polynomial circuits with post-selection, in Section~\ref{sec:Complexity results}.
    \item We show a protocol to prepare the uniform superposition state of size $q$ in $\LAQCC$ using $\mathO(\ceil{\log_2(q)}^2)$ qubits in Section~\ref{sec:superposition_modulo_q}. 
    \item We show a protocol to prepare the $W_n$ state in $\LAQCC$ using $\mathO(n\log(n))$ qubits in Section~\ref{sec:W_state_in_LAQCC}.
    \item We show two ways of preparing the Dicke-$(n,k)$ state. The first method is in $\LAQCC$, works up to $k = \mathO(\sqrt{n})$, uses $\mathO(n^2\log(n))$ qubits, and is found in Section~\ref{sec:dicke:small_k}. The second method is in $\LAQCC\text{-}\mathsf{LOG}$ (an extension of $\LAQCC$ allowing for logarithmic number of alterations instead of constant), works for any $k$, uses $\mathO(\text{poly}(n))$ qubits, and is found in Section~\ref{sec:Dicke_in_LAQCC_LOG}. 
    \item We extend on our $\LAQCC$ method of generating Dicke-$(n,k)$ states for $k = \mathO(\sqrt{n})$ and show a protocol to generate many-body scar states for a particular Hamiltonian in $\LAQCC$ (Section~\ref{sec:many_body_scar}). 
\end{itemize}
Summarized in a table, we provide the following state generation protocols:
\begin{table}[htb]
\centering
\begin{tabular}{l|l|l|l}
\textbf{State description} & \textbf{Width} & \textbf{Depth} & \textbf{Implementation}\\
\hline 
Uniform superposition mod $q$: $\frac{1}{\sqrt{q}} \sum_{i = 0}^{q-1}\ket{i}$ & $\mathO(\ceil{\log^2 q})$ & $\mathO(1)$ & Section~\ref{sec:superposition_modulo_q}\\

$W$-state: $\frac{1}{\sqrt{n}}\sum_{i = 0}^{n-1}\ket{e_i}$ & $\mathO(n \log n)$ & $\mathO(1)$ & Section~\ref{sec:W_state_in_LAQCC}\\

Dicke-$(n,k)$, $k = \mathO(\sqrt{n})$: $\binom{n}{k}^{-1/2}\sum_{x \in \{0,1\}^n: |x| = k} \ket{x}$ &  $\mathO(n^2\log n)$ & $\mathO(1)$ 
&Section~\ref{sec:dicke:small_k}\\

Dicke-$(n,k)$: $\binom{n}{k}^{-1/2}\sum_{x \in \{0,1\}^n: |x| = k} \ket{x}$ & $\mathO(\text{poly}(n))$ & $\mathO(\log n)$ &Section~\ref{sec:Dicke_in_LAQCC_LOG}\\

QMBS: $\ket{S_k} = \frac{1}{k! \sqrt{\mathcal N(n,k)}}(Q^\dagger)^k \ket{\Omega}$ &  $\mathO(n^2\log n)$ & $\mathO(1)$  &  Section~\ref{sec:many_body_scar}
\end{tabular}
\caption{Summary of state preparation protocols given in this paper.}
\label{tab:sate_prep}
\end{table}
In the entry for the quantum many-body scar state $Q$ denotes the raising operator and $\mathcal N(n,k)=\binom{n-k-1}{k}$. 
Section~\ref{sec:many_body_scar} will provide more details on the variables and the implementation. 

\paragraph{Organization of the paper}
\noindent We first introduce relevant preliminaries in Section~\ref{sec:preliminaries}. 
In Section~\ref{sec:LAQCC_model} we formally define the class of Local Alternating Quantum-Classical Computations ($\LAQCC$). We also show that any Clifford circuit can be implemented in constant depth $\LAQCC$ (a result based on a result from measurement-based quantum computing~\cite{jozsa2006introduction}). 
This result allows us to give many useful multi-qubit gates and routines in Section~\ref{sec:gates_created_in_LAQCC}. 
Beyond that we show that constant depth $\LAQCC$ circuits are contained in $\QNC^1$ and that any $\mathsf{IQP}$ circuit has an $\LAQCC$ implementation.
We conclude this section with an analysis of a more powerful instantiation of $\LAQCC$ and show an inclusion with respect to the class $\mathsf{PostQPoly}$, which is the class of circuits of polynomial depth with one additional post-selection gate. 
In Section~\ref{sec:state_prep_in_LAQCC} we give $\LAQCC$ circuit implementations for preparing the uniform superposition over an arbitrary number of states, the $W$-state and the Dicke state up to $k = \mathO(\sqrt{n})$. We furthermore give a log-depth circuit implementation for preparing the Dicke state for any $k$. We conclude by showing a $\LAQCC$ circuit for generating many body scar states of a particular type of Hamiltonian.


\vspacebeforesection
\section{Background}
\label{sec:background}

In this section, we provide the necessary background information to ensure a comprehensive understanding of the attack described in this paper. We start with a description of the Distributed Hash Table (DHT) used by IPFS, followed by its content resolution mechanisms. We also detail techniques for network size estimation, necessary for our attack detection and mitigation mechanisms.

\vspacebeforesection
\subsection{IPFS DHT}
\label{sec:kad_dht}

We review the features of the Kademlia DHT~\cite{maymounkov2002kademlia} and its \texttt{libp2p} implementation~\cite{libp2p_github} that are the most relevant to our attack.
To participate in the DHT, each peer generates a public/private key pair and derives an identity $\peerid \in \{0,1\}^{256}$ as the hash of its public key.
Ideally, each peer generates a random key pair and, therefore, peer IDs are distributed uniformly and independently over the space $\{0,1\}^{256}$.
While honest nodes follow this rule, malicious nodes may generate and choose from an arbitrary number of key pairs.
Each peer maintains a routing table consisting of $m=256$ buckets.
The $i$-th bucket contains the addresses of up to $k=20$ peers whose peer IDs share a common prefix of exactly $i$ bits with the peer's own peer ID. 

%
A new participant node joins the IPFS network by contacting one of the hardcoded bootstrap nodes. This bootstrap node provides the new node with some initial peers allowing it to join the DHT. The new node uses this information to perform a walk through the DHT towards its own peer ID.
The walk allows to: \textit{(i)}~make sure that there is no other node in the network with the same ID; \textit{(ii)}~discover new peers and fill the newcomer's DHT routing table. At the same time, the newcomer establishes \bitswap~\cite{de2021accelerating} connections to a subset of encountered peers (usually around 300 of them). The core role of the \bitswap protocol is to enable bilateral content transfer and to play the role of a cache for recently-accessed content.

The main DHT operation $\Call{GetClosestPeers}{\key}$ returns the $k=20$ closest peers to $\key$. 
%
In Kademlia, the distance between two keys $x$ and $y$ in the key space is given by $x \oplus y \in \{0,...,2^{256}-1\}$, where $\oplus$ denotes the bitwise XOR operation on the keys; the resulting binary string is interpreted as an integer.
%
When a client wants to find the peers with IDs closest to $\key$, it sends a request to the $\alpha=3$ peers in its routing table whose peer IDs are closest to $\key$. Each of these peers returns the $k$ closest peers to $\key$ in its own routing table and the addresses of these peers. 
%
The client again sends a request to the $\alpha$ peers closest to $\key$, among peers in its routing table and those whose addresses it just received. This process repeats until the client does not find any more peers closer to $\key$.
Due to network churn and imperfect routing tables, we observed in our experiments that successive calls to $\Call{GetClosestPeers}{\key}$ do not always return the same set of $k=20$ peers (we provide more details in \Cref{sec:evaluation}, \Cref{fig:20closest}). This is an important limitation affecting our attack.

\vspacebeforesection
\subsection{Content Resolution in IPFS}
\label{sec:ipfs}

IPFS is a content-centric network.
It allows its participant to request files without specifying their location. 
%
Content is indexed by content IDs $\cid \in \{0,1\}^{256}$ that are derived from a hash of that content.
Both peer IDs and CIDs are used as keys in the DHT.
Each node can play the role of a \provider, \downloader, or \resolver. 
The process of content advertisement and resolution is illustrated in \Cref{fig:add_get_provider}.

%
When a \provider wishes to publish content with a given $\cid$ on IPFS, it creates a \emph{provider record} that contains $cid$ and the \provider's address.
During a $\Call{Provide}{\cid}$ operation, the \provider first uses $\Call{GetClosestPeers}{\cid}$ to locate the $k=20$ peers with their peer IDs closest to $\cid$, 
%
and then sends them a $\mathsf{PutProvider}$ message including the provider record (\Cref{fig:add_get_provider}(a)).
We call the peers that hold provider records for $\cid$ the \emph{resolvers} for $\cid$.

Each CID can have several \providers. In fact, by default, each IPFS client becomes a provider for each piece of content it downloads for a fixed amount of time (12h, 24h, or 48h depending on the client version or custom configuration). As a result, the system provides an auto-scaling feature with supply automatically rising with demand.

%
When a \downloader wishes to fetch a piece of content, it first sends a request to all its \bitswap peers. If none of them has the content, the \downloader uses the DHT-based resolution system. We stress that the \bitswap protocol plays the supporting role of a cache in the dissemination of popular files. However, the mechanism does not provide reliable content resolution, in particular for new or less popular content. %

When \bitswap unstructured search fails, the \downloader resolves $\cid$ using $\Call{FindProviders}{\cid}$. This operation uses a DHT walk identical to that of $\Call{GetClosestPeers}{\cid}$ to find $k$ \resolvers but also queries encountered nodes for a provider record for $\cid$ (\Cref{fig:add_get_provider}(b)). The process terminates when either 20 \providers have been found, or all \resolvers have been asked. Querying all encountered nodes (\ie, not only the designated \resolvers) is useful because some of the encountered nodes may have a provider record in their cache.
%

Upon receiving a provider record, the client connects to the address specified in the provider record to retrieve the actual content (\Cref{fig:add_get_provider}(c)).
Provider records are not authenticated, and therefore malicious \providers may respond with incorrect provider records (or may not respond at all). However, the integrity of the content is preserved because the hash of the retrieved content can be verified against its $\cid$.
%


%

\input{img/add_get_provider.tex}

\vspacebeforesection
\subsection{Network Size Estimator}
\label{sec:netsize}

The number of nodes in a decentralized system is generally unknown due to the avoidance of centralized membership management.
This number is nonetheless useful for optimizations, deciding on individual node configurations, or security mechanisms.
Various methods were proposed for the decentralized estimation of unstructured and structured networks~\cite{eli-sohl-dht-size-estimation,kostoulas2005decentralized, manku2003symphony}.
We use in this work a mechanism developed initially by Protocol Labs as part of a mechanism for decreasing the latency of publishing content in IPFS~\cite{network-size-estimation-notion,network-size-estimation-github-pr}.

%
%
%
%
%
%
%
%
%
%

Each node in the DHT refreshes its routing table periodically (every $10$ minutes in \texttt{libp2p}). 
For this, the node samples $m$ random keys (one for each bucket of its routing table)
%
and queries the DHT to obtain the $k=20$ closest peer IDs to each key.
Using these, the node then computes the average distance between each one of these keys $\key_j$ for $j=1,\dots,m$ and their $i$-th closest peer ID for $i=1,...,k$ (with $m=256$ and $k=20$).
\begin{equation}
    \label{equ:avg-dist}
    \overline{D}_i = \frac{1}{m} \sum_{j=1}^m \operatorname{dist}(\key_j, \peerid_{j}^{(i)})
\end{equation}
where $\peerid_{j}^{(i)}$ is the $i$-th closest peer ID to $\key_j$.
With $N$ peers in the DHT and peer IDs uniformly distributed in the hash space, the expected distance between a $\key$ and its $i$-th closest peer ID is $\frac{2^{256}i}{N+1}$. The node then runs a least square regression to compute the value of $N$ for which the expected distances best fit the empirical average distances, \ie,
\begin{equation}
    \label{equ:netsize-least-squares}
    \hat{N} = \arg\min_{N} \sum_{i=1}^k \left(\overline{D}_i - \frac{2^{256}i}{N+1}\right)^2.
\end{equation}
The resulting estimate $\hat{N}$ can be computed in closed form.
%

When a node starts running, it must perform DHT queries for a few random keys to initialize its network size estimate. 
Since a larger number of queries will result in higher accuracy, making more queries than what is needed to initialize one's routing table is recommended.
Thereafter, keeping the estimate up-to-date does not require any excess DHT queries beyond what is already used for refreshing the routing table as this is done frequently (every 10 minutes).

While the network size estimate has a stochastic variance resulting from the probability distribution of the honest peer IDs, it is hard for an attacker to bias the estimate significantly. Since the estimator uses the density of peer IDs around keys chosen uniformly at random, the adversary would require numerous Sybil nodes (on the order of the whole network size) to significantly affect the peer ID density around those keys.

\section{Motivation}
\label{sec:motivation}

IGNORE THIS FILE, WILL DO IN INTRO

%------------------------------------------------------------------------------
\section{Intermediate Representations for FL\lowercase{i}CR}
\label{sec:ir}
%------------------------------------------------------------------------------

For meeting the aforesaid requirements, it is important to select a proper IR because the compression is dependent on each IR.
In this section, we microbenchmark the IR conversions and point out the benefit of range images (RIs) over the others in the context of enabling remote online perceptions.

\begin{table}[htbp]
  \caption{The latencies (ms) of each IR construction.}
  \begin{center}
  \begin{tabular}{ |c|c|c|c|c|c| }
    \hline
             & RI    & Parallel RI   & Octree & K-d tree & Mesh \\ \hline
     Desktop & 11.78 & \textbf{6.72} & 30.67  & 13.21    & 1872 \\ \hline
     Jetson  & 16.34 & \textbf{9.26} & 32.11  & 32.44    & 2755 \\ \hline
  \end{tabular}
  \label{tab:riconv}
  \end{center}
\end{table}

Table~\ref{tab:riconv} shows the conversion latencies of the IRs with the LiDAR point clouds from the KITTI dataset~\cite{geiger2013vision}.
We use PCL~\cite{rusu20113d} implementations for octree and k-d tree,
and mesh conversion is based on the algorithm of Marton \emph{et al.}~\cite{marton2009fast}.
The RI conversion is our implementation, and the parallelized version is with OpenMP~\cite{dagum1998openmp}.
The RIs are generated by converting the raw points in the 3D Cartesian coordinates to the spherical coordinates.
Equation~\ref{eq:1} shows the conversion and $r$, $\theta$, and $\phi$ are the radial distance, polar angle, and azimuthal angle each.
When $\theta$ and $\phi$ are calculated, they are mapped to the frame pixel by the sensor's angular precisions.
For example, Velodyne HDL-64E used in the KITTI dataset has 0.08\textdegree~and 0.35\textdegree~for horizontal and azimuthal precisions with 360\textdegree~of the horizontal field of view (FoV) and 64 vertical lasers~\cite{hdl64}.
Thus, each scan's point cloud would be mapped to a 2D frame of 4500$\times$64.
By adjusting the parameters of precisions and FoV, RI can work on diverse LiDAR sensors.

\begin{equation}\label{eq:1}
\begin{aligned}
  r = \sqrt{x^2+y^2+z^2} \\
  \theta = arccos\left(\frac{z}{r}\right) \\
  \phi = arctan\left(\frac{y}{x}\right)
\end{aligned}
\end{equation}

In our results, the parallelized RI conversion shows the lowest latency on both desktop and Jetson as RI has advantages over other IRs in terms of simplicity and parallelism.
For the octree and k-d tree, there have been many efforts for their parallelized constructions~\cite{shevtsov2007highly, wehr2018parallel, lauterbach2009fast, karras2012maximizing, wu2011sah}.
However, as pointed out in the previous works, their hierarchical structures inherently make the construction processes sequential, and it is challenging to fully parallelize their constructions.
In contrast, the RI conversion can be easily parallelized since each point conversion of RI is completely independent from the others.
For the mesh, its generation from point clouds requires triangulation algorithms and calculating the surface normal for each mesh.
These processes require iterating each point and finding nearest
neighbors to generate a mesh, and the mesh conversion has high
computational complexity and is not suitable for real-time due to its large magnitude of execution time~\cite{marton2009fast, salman2010feature, guan2020voxel}.

Furthermore, these IRs have different theoretical complexities for the conversion.
The time complexities of the IR constructions are $O(n)$ for the RI and $O(n\log{}n)$ for the trees and mesh; the trees require  binary search for each point insertion, and the mesh construction needs nearest neighbor searches for the normal estimation and triangulation.
In addition, there is a side benefit that various image-processing techniques can be used for RIs.
Based on these observations, we adapt RI as the
target IR.


%------------------------------------------------------------------------------
\section{FL\lowercase{i}CR: Range Image Compression}
\label{sec:ricomp}
%------------------------------------------------------------------------------
Following selecting RI as the appropriate IR, the compression also needs to be efficient, low-latency, and lightweight.
In this section, we describe how to achieve the objectives of the compression method.
First, we identify the distortion issue of the current image codecs to LiDAR RIs.
Second, we explore the opportunity of lossy RIs for the downstream compression steps via RI quantization and subsampling.
We argue the lossless bytestream compressions can be hugely enhanced in terms of the compression efficiency and low latency through the lossy representation.
However, it compromises the point cloud quality, and we present the possible issues of FLiCR with lossy RIs.


\begin{table}[htbp]
  \caption{The compression ratios and qualities of H.264 with different QPs for 100 RIs of 4500$\times$64 and 8 bpp.}
  \begin{center}
  \begin{tabular}{ |c|c|c|c|c| }
    \hline
                            & QP 0    & QP 10 & QP 20  & QP 30 \\ \hline
    Compression Ratio       & 12.85   & 13.33 & 16.41  & 35.01 \\ \hline
    PSNR (dB)               & 63.18   & 48.21 & 37.61  & 38.12 \\ \hline
    CD (cm)                 & 2.23    & 16.25 & 126.06 & 107.16\\ \hline
  \end{tabular}
  \end{center}
  \label{tab:qpt}
\end{table}

% Figure environment removed



\subsection{Issues with Current Image Compressions}
\label{sec:issuecodecs}
By representing LiDAR point clouds as images, it becomes possible to leverage the existing image-processing infrastructures and techniques.
With the popularity of video streaming, modern processor platforms and GPUs are equipped with dedicated hardware modules for the standard codecs such as H.264 and HEVC~\cite{intelquicksink, snapd, nvcodec}.
These codecs efficiently encode and decode continuous images with spatial and temporal optimizations~\cite{richardson2004h, sullivan2012overview}, and the pervasive accelerators enable the codecs in a low-latency and efficient way even with commodity mobile devices.

In this context, it seems appropriate to rely on the existing codecs with hardware accelerators at first glance.
However, we argue the existing codecs specialized for human vision are hardly applicable to RI compression.
The lossy image compression algorithms for human vision fully utilize
the characteristics of the human eyes to remove the data with minimal impacts to visual quality as much as possible; one example is to convert an image into the frequency domain via  discrete cosine transform (DCT) or fast Fourier transform (FFT) and the high frequency has more loss than the low-frequency data~\cite{richardson2004h, sullivan2012overview, marcellin2000overview}.
While the techniques that leverage the nature of human vision work well for normal images, the point cloud details are effectively lost as a result of the frequency-domain loss in LiDAR RIs.


Figure~\ref{fig:qp} shows the reconstructed point clouds from the RIs encoded and decoded via H.264 with different quantization parameters (QP).
QP regulates how much spatial detail is retained and is set from 0 for lossless to 51 for the most lossy compression.
As QP increases, the spatial detail is aggregated so that the encoded bit rate drops at the expense of data loss, resulting in lower quality~\cite{richardson2004h}.
The reconstructed point clouds become vague and noisy with high QPs.
Table~\ref{tab:qpt} shows the averaged compression ratio and quality metrics of the reconstructed point cloud with different QPs.
With the visual results of the reconstructed point clouds, the PSNR and CD results become worse drastically while the compression ratio increases moderately.
Considering that the quantization parameter such as QP or CRF of video streaming is usually set around 20 and 30 as a rule of thumb (FFmpeg's H.264 default CRF is 23~\cite{ffmpegh264}), these results show the current human-vision codecs are unsuitable for the RI compression.
For preserving the quality of point clouds, the codec quantization parameter should be set for lossless (QP 0), but it is at the cost of the lower compression efficiency, as Google Draco achieves $\sim$33\% higher compression ratio in Table~\ref{tab:expcc}.



\begin{table}[htbp]
  \caption{The existing quality metrics with sampling error (SE) for the subsampled RIs of 8 bpp.}
  \begin{center}
  \begin{tabular}{ |c|c|c|c|c| }
    \hline
               & 2048$\times$64    & 1024$\times$64 & 512$\times$64  & 256$\times$64 \\ \hline
     PSNR (dB) & 62.4              & 61.41          & 58.61          & 53.71         \\ \hline
     CD (cm)   & 5.37              & 9.23           & 15.22          & 36.17         \\ \hline
     SE        & 21.03\%           & 58.77\%        & 78.95\%        & 89.22\%       \\ \hline
  \end{tabular}
  \label{tab:oldmetrics}
  \end{center}
\end{table}

% Figure environment removed



\subsection{RI Quantization and Subsampling}
RI has been used for losslessly mapping LiDAR point clouds to 2D
frames, and previous work only applies quantization of bit-per-point (bpp)~\cite{tu2019real, tu2019point, tu2016compressing, ahn2014large, feng2020real, houshiar20153d}.
In these prior works, the main objective is to maximize the compression efficiency while maintaining the point cloud quality as high as possible.
However, we argue that there are more optimization opportunities with lossy RI to decrease not only the data size but the downstream compression tasks' complexities.
Specifically, the RI resolution is determined by the sensor's precisions as mentioned in Section~\ref{sec:ir}, and the subsampling of point clouds can be done by adjusting the precision parameters; the 3D points are coarsely mapped to a 2D frame.
Figure~\ref{fig:subsample} shows the visualizations of reconstructed point clouds from the subsampled RIs.
From the raw point cloud of Figure~\ref{fig:rawpcrep}, we reduced the precision parameters to map it to the RIs of four different lower resolutions.
Even with the lowest subsampled RI of 16 KB with 8 bpp, the shapes of scanned objects are recognizable.


While the subsampled and quantized RI has advantages for data reduction and compression with lower latencies, it would affect the performance of the perception tasks.
So, a quality metric for the point clouds from lossy RIs needs to reflect both the quantization and subsampling errors.
The currently used metrics, PSNR and CD, reflect the quantization error well, but the sampling error of lossy RIs is not represented effectively as these metrics are defined with the point-to-point distances between point clouds (see Section~\ref{sec:metric} for more details).

Table~\ref{tab:oldmetrics} shows PSNR, CD, and sampling error (SE) of the point clouds from four RI resolutions.
In the results, the changes of PSNR and CD exhibit different trends from SE because SE is about the number of lost points from the original point cloud (the entropy-wise quality) while PSNR and CD are with the distances of the closest point pairs between two point clouds (the point-wise quality).

The current metrics’ issue is they only count the point-to-point distances, and each point distance is calculated by finding the nearest point in the comparing point cloud. So, when the point clouds have different numbers of points, they are limited to represent this difference in the total number of points in the point clouds.
To address the limitations of the existing metrics, we propose a unified metric for both the point-wise quality and the information amount to measure quantitatively the impacts on the downstream perceptions from lossy RIs in Section~\ref{sec:metric}.

% Figure environment removed



\subsection{Lossless Compression with Lossy RIs}

% Figure environment removed

As shown in Section~\ref{sec:issuecodecs}, the application of lossy
video codecs to RIs results in lower compression efficiency or can distort the point clouds in the 3D space.
The previous RI compression methods apply the image compression algorithm at lower efficiencies or propose effective lossless RI compression algorithms via spatial and temporal optimizations~\cite{ahn2014large, feng2020real, houshiar20153d, tu2016compressing, tu2019point, tu2019real}.
%However, they only leverage the quantization of bit precisions with lossless RI mapping and their complex algorithms have downsides for low-latency and lightweight while showing high compression ratios.
However, they partially leverage the opportunities of lossy RIs only
with  bit quantization, and their complex algorithms have downsides
% low-latency and lightweight
in terms of latency and overheads, while showing high compression ratios.
For satisfying the low-latency, lightweight, and efficiency requirements, we use the existing bytestream compression algorithm, dictionary coding, and enhance its efficiency by fully leveraging the RI quantization and subsampling.

Dictionary coding is a lossless compression algorithm for bytestreams
and deflates the bytestream by replacing the repeating patterns with shorter references.
Dictionary coding algorithms have been extensively studied with corpus
text data, and they are with simple bit/byte operations and lower
computation complexities in terms of the space-time tradeoff~\cite{shanmugasundaram2011comparative}.
So, they provide the benefits of being lightweight and low-latency, with simple operations and do not distort point clouds unexpectedly.
Even with such advantages, the direct application of bytestream compressions to the raw point cloud and unquantized RIs of floating values is inefficient in terms of the compression ratio (RLE and Dict Coding in Table~\ref{tab:expcc} and Figure~\ref{fig:losslesscr}).

To improve the efficiency, we fully utilize both quantization and subsampling.
The underlying assumption of our approach is dictionary coding uses the repeating features in a bytestream and there is a higher probability of recurring patterns when limiting the representation space of quantized RIs.
The compression pipeline is shown in Figure~\ref{fig:rioverview}.
FLiCR with dictionary coding has similarity to previous RI-based works in terms of leveraging local spatial features, but it is more advantageous in meeting the requirements with its simplicity.
Explicitly, compared to the recent RI-based compression (RT-ST~\cite{feng2020real} in Table~\ref{tab:expcc}), FLiCR shows lower encoding and decoding latencies and energy usage, with the higher compression ratio, as shown in Table~\ref{tab:oursvsdraco}.

Among the dictionary coding algorithms, we use LZ77~\cite{ziv1977universal} and compare it with RLE.
We measure the efficiency improvement and quality reduction by the quantization and subsampling with LZ77 and RLE, and Figure~\ref{fig:losslessours} shows the results of different resolutions of RIs quantized by 8 bpp.
While the end-to-end latencies of the whole compression pipeline can
decrease only with subsampling (see Figure~\ref{fig:losslesslat}),
both quantization and subsampling are required to improve the
compression ratios effectively, as shown in Figure~\ref{fig:losslesscr}.
Then, dictionary coding shows  larger growth than RLE, and these results support our assumption about the performance improvement of dictionary coding with lossy RIs.

Although FLiCR achieves compression efficiency and reduced latency, it
is at the cost of degradation of the point cloud quality by the quantization and subsampling errors, as shown in Figures~\ref{fig:losslesspsnr} and~\ref{fig:losslesscd}.
Since the reduced point cloud quality can have an impact on the downstream perceptions, we evaluate FLiCR with the state-of-the-art LiDAR perceptions and analyze the errors' impacts in Section~\ref{sec:eval}.

% Figure environment removed

Figure~\ref{fig:latbreak} shows the latency breakdowns of FLiCR on our testbed.
In the case where Jetson is a mobile client and the desktop is a server, the end-to-end latency is $\sim$39 ms ($\sim$60\% of Google Draco~\cite{draco}) even with the highest RI resolution; it takes 27 ms for client encoding and 12 ms for server decoding.
With the 256$\times$64 resolution, the end-to-end latency is $\sim$10 ms which is $\sim$16\% of Draco.
Since the large portion of the end-to-end latency is the conversion time between the point cloud and RI, the end-to-end latencies can be largely reduced if a device has a dedicated hardware logic for the RI conversion.
As the RI resolution gets lower, the quantization and compression latencies decrease.
These results show FLiCR fully leverages the synergistic effect by quantization and subsampling for the bytestream compressions.

%------------------------------------------------------------------------------
\section{\lowercase{e}PSNR: Quality Metric for L\lowercase{i}DAR Point Clouds}
\label{sec:metric}
%------------------------------------------------------------------------------

The errors in the RI conversion process can affect the performance of LiDAR perceptions.
PSNR and CD (or RMSE) have been broadly used as the quality metrics of
3D point clouds~\cite{biswas2020muscle, huang2020octsqueeze, tu2016compressing, que2021voxelcontext, feng2020real, tu2019point, tu2019real}, but by definition they are point-wise quality metrics and do not reflect the point loss effectively.
In the context of our approach with lossy RIs, we argue a metric for both point-wise quality and overall information amount is essential.


\begin{equation}\label{eq:2}
Dist(p, C) = \min_{{p_c}}\left( (p_c-p)^2 \right)
\end{equation}

\begin{equation}\label{eq:2-2}
MSE(C_1, C_2) = \frac{1}{\left\| C_2 \right\|}\sum_{i=0}^{\left\| C_2 \right\|-1} \left\{ Dist(p_{c_2}, C_1) \right\}
\end{equation}


Both PSNR and CD use the mean squared error (MSE) of the point-wise distances between two point clouds.
When $C_1$ is the original point cloud and $C_2$ is the reconstructed point cloud, the distance between a point in $C_2$ and the corresponding point in $C_1$ is calculated by Equation~\ref{eq:2}.
So, the corresponding point in $C_1$ is of the shortest distance to the point in $C_2$.
Then, MSE between two point clouds is defined by Equation~\ref{eq:2-2}.

\begin{equation}\label{eq:3}
  \scalebox{.9}{$CD(C_{orig}, C_{comp}) = MSE(C_{orig}, C_{comp}) + MSE(C_{comp}, C_{orig})$}
\end{equation}

\begin{equation}\label{eq:4}
  \scalebox{.9}{$PSNR(C_{orig}, C_{comp}) = 10\ log\left( \frac{Max^2}{MSE(C_{orig}, C_{comp})} \right)$}
\end{equation}

Then, PSNR and CD are defined as Equation~\ref{eq:3} and~\ref{eq:4}.
CD is the sum of reciprocal MSEs between two point clouds, and PSNR is the ratio of the peak LiDAR sensor range to MSE.
As indicated by their definitions, these metrics are based on MSE and are determined by the point-to-point distance.
They natively represent the quality loss by the quantization error, but the subsampling error is not effectively represented even if the impact of SE is shown mildly as some nearest points can be lost in the reconstructed point cloud.
Specifically, in the current metrics, it is possible to get a high-quality result even with a few points of  small distances to a point cloud of  large number of points.
It is caused by the unstructured nature of the LiDAR point cloud; there is no point-to-point correspondence between LiDAR point clouds having different numbers of points.
In the case of the normal images, the total number of pixels is fixed without SE and the point-wise metrics work well.

\begin{equation}\label{eq:5}
SE = \frac{\left\| C_{orig} - C_{comp} \right\|}{\left\| C_{orig} \right\|},\ 0 \le SE\le 1
\end{equation}

Based on our observation, we argue it is inappropriate to use the metrics representing only the point-wise quality for LiDAR point clouds.
To address the limitation of the current metrics, we propose a new
single-number metric, {\bf entropy-reflecting PSNR (ePSNR)}, by extending PSNR.
ePSNR is designed to indicate both the point-wise and entropy-wise quality of a point cloud.

SE is related to the total information (entropy) loss in a point cloud
because it is the percent of the lost points, as in Equation~\ref{eq:5}.
One naive way of making PSNR reflect the entropy is to multiply $1-SE$ to PSNR with the assumption that the entropy is $1-SE$ and SE is exactly the same with the actual entropy loss, $\mathscr{L}_{SE}$.
However, instead of the naive way, we extend PSNR by estimating $\mathscr{L}_{SE}$.
Our underlying assumption is $\mathscr{L}_{SE}$ is not exactly the same with SE and follows the exponential distribution as Equation~\ref{eq:6}.
The intuition for this assumption is that SE can have minimal impacts on the downstream perceptions as far as the total amount of necessary information is preserved for the perception algorithms.
It means there would be a knee of the curve in the graph of the entropy function.

\begin{equation}\label{eq:6}
  Assumption:\quad\mathscr{L}_{SE} \sim \mathcal{\text{exp}(\beta)}
\end{equation}

When $\mathscr{L}_{SE}$ follows the exponential distribution, the entropy function $\mathcal{F}(SE)$ can be defined with the cummulative distribution function (CDF) of the exponential distribution as Equation~\ref{eq:7}.
This entropy function is a probability function estimating the actual entropy of the remaining points in a point cloud with the given SE.

\begin{equation}\label{eq:7}
\mathcal{F}(SE) = \mathrm{P}(\mathrm{E}>x) = e^{-\frac{x}{\beta}}\quad where\ x = 1-SE \\
\end{equation}

With our entropy function, ePSNR is defined as Equation~\ref{eq:8}.
Since it is based on PSNR, the point-wise quality with the quantization error is represented while reflecting the entropy with the given SE.
When SE is small, ePSNR would be almost same with the original PSNR,
but would start to decrease exponentially when SE gets larger, by its definition.
ePSNR has two parameters: $\alpha$ as a derivative adjusting factor to prevent too steep or shallow distribution and $\beta$ of the exponential distribution.

\begin{equation}\label{eq:8}
\begin{gathered}
\scalebox{.9}{$ePSNR(C_{orig}, C_{comp}) =  PSNR\times \left\{ 1 - (SE \times (\mathcal{F}(SE)+\alpha)) \right\},$}
  \\ \scalebox{.9}{$0\le \mathcal{F}(SE)+ \alpha \le 1$}
\end{gathered}
\end{equation}


\section{Evaluation} \label{sec:evaluation}

\begin{table*}[tbp]
\centering
\small
\begin{tabular}{cccccccccc}
\toprule
& \multicolumn{3}{c}{\msr} & \multicolumn{3}{c}{\negc} & \multicolumn{3}{c}{\wsj} \\
& Acc. & F1 & wF1 & Acc. & F1 & wF1 & Acc. & F1 & wF1 \\ \cmidrule(lr){2-4} \cmidrule(lr){5-7} \cmidrule(lr){8-10} 
\udel & 66.86 & 56.76 & 64.3 & \textbf{80.80} & 55.45 & 77.9 & 63.74 & 64.23 & 63.2 \\
\icsi & \underline{71.19} & 64.73 & 70.4 & 80.36 & 64.53 & \underline{78.6} & 64.62 & 64.15 & 63.4 \\
\cnts & 68.59 & 61.39 & 67.2 & 78.68 & 61.62 & 76.8 & 64.31 & 64.59 & 64.4 \\
\osu & 68.02 & 60.28 & 66.6 & 79.24 & 57.04 & 76.5 & 69.20 & 69.63 & 68.9 \\
\isg & 67.05 & 58.83 & 65.3 & 77.34 & 59.52 & 75.6 & 69.15 & 69.35 & 69.2 \\ \midrule
\bert & \textbf{71.68} & \underline{66.70} & \textbf{71.4} & 77.79 & \underline{72.87} & 77.7 & \underline{80.95} & \underline{80.93} & \underline{80.9} \\
\roberta & 70.91 & \textbf{67.53} & \underline{70.7} & \textbf{80.80} & \textbf{77.29} & \textbf{80.7} & \textbf{82.61} & \textbf{82.70} & \textbf{82.6} \\ \midrule
Average & 69.19 & 62.32 & 67.99 & 79.29 & 64.05 & 77.69 & 70.65 & 70.80 & 70.37 \\
\bottomrule
\end{tabular}
\caption{\label{tab:performance} Overall accuracy (Acc.), macro-averaged F1 (F1), and weighted-macro F1 (wF1) scores of the algorithms depicted in Section~\ref{sec:algorithm}. For instance, \msr-\udel refers to a C5.0 classifier trained on the \msr~corpus, using the feature set mentioned in \citet{greenbacker-mccoy-2009-udel}.}
%Its Acc., F1 and wF1 of this model are 66.86, 56.76, and 64.3, respectively.}
\end{table*}


In this section, we introduce the evaluation protocol and report the performance of the models.

\subsection{Implementation Details} \label{sec:implementation}

For \bert and \roberta, we used \textit{bert-base-cased} and \textit{roberta-base}, both from Hugging Face. For fine-tuning, we set the batch size to 16, the learning rate to 1e-3, the dropout rate to 0.5, and the size of the output layer to 256. We ran each model for 20 epochs and used the one that achieved the highest F1 score on the development set. The implementation details of the classic ML-based models can be found in Appendix~\ref{sec:appendixML}.

\subsection{Evaluation Protocol} \label{sec:protocol}

The main evaluation metric in the GREC-MSR shared tasks was accuracy. 
In addition to accuracy, we also report macro-F1 and weighted-macro F1. We argue that different metrics evaluate algorithms from different perspectives and provide us with different meaningful insights. 
For pragmatic tasks like REG, it makes sense to ask how well an algorithm performs on naturally distributed data which is often imbalanced. For these cases, reporting accuracy and weighted F1 are logical. 
Furthermore, analogous to other classification tasks, minority categories should not be overlooked. Take as an example the class \emph{description} in the \negc corpus, which occurs only 4\%. If a model fails to produce this class, the produced document might sound unnatural. Therefore, it is important to ensure that an algorithm is not over- or under-generating certain classes. Looking into accuracy and macro-F1 together provides insights into such cases.

\subsection{Performance of the Models}\label{subsec:overallacc}

The overall accuracy of the models, their macro F1, and their weighted-macro F1 are presented in Table \ref{tab:performance}. 
We also present the ranking of the models based on these scores in Appendix~\ref{sec:app_rank}. 


\paragraph{PLM-based Models.} The best-performing models across all corpora and metrics are PLM-based models.  In six out of nine rankings, \bert and \roberta are ranked as the top two models. The sole exception is \negc, where \bert is the second worst model. The benefit of using PLMs is the largest on the \wsj corpus. For example, \roberta improves the macro F1 score from 69.63 (i.e., the performance of the best ML-based model) to 82.70.


\paragraph{ML-based Models.} In contrast to the robust performance of the PLM models, the performance of the classic ML models is more corpus-dependent. In the case of \msr and \negc, \icsi is the best-performing model, while in the case of \wsj, it is at the bottom section of the rankings. Another interesting observation is the performance of the \udel models. In terms of accuracy, \udel has the highest performance in \negc, while it has the lowest performance in both \msr and \wsj. In terms of macro-F1 rankings, the \negc \udel model dropped from first to last place, whereas \bert improved from penultimate place to second place. In general, our ML models yielded lower scores than the original models used in the GREC study \citep{belz2009generating}. This could be attributed to a variety of factors, including differences in feature engineering and model parameters.

\paragraph{Comparing Different Metrics.} 

Upon comparing average scores across the three metrics, we observe that for \msr and \negc, PLMs are clear winners only when macro-F1 is the metric in question. However, for \wsj, PLMs are winners on all three metrics. This may be because the distribution of categories in \wsj is much more balanced than in the other two corpora.
\section{Related Work}
\label{appsec: related work}
Bayesian causal discovery literature has primarily focused on inference in linear models with closed-form posteriors or marginalized parameters. Early works considered sampling directed acyclic graphs (DAGs) for discrete~\cite{cooper1992bayesian, madigan1995bayesian, heckerman2006bayesian} and Gaussian random variables~\cite{friedman2003being, tong2001active} using Markov chain Monte Carlo (MCMC) in the DAG space. However, these approaches exhibit slow mixing and convergence~\cite{eaton2012bayesian,grzegorczyk2008improving}, often requiring restrictions on number of parents~\cite{kuipers2017partition}. %Alternative exact dynamic programming methods are limited to small settings~\cite{koivisto2012advances}. 

Recent advances in variational inference~\cite{zhang2018advances} have facilitated graph inference in DAG space, with gradient-based methods employing the NOTEARS DAG penalty \cite{zheng2018dags}.\cite{annadani2021variational} samples DAGs from autoregressive adjacency matrix distributions, while \cite{lorch2021dibs} utilizes Stein variational approach \cite{liu2016stein} for DAGs and causal model parameters. \cite{cundy2021bcd} proposed a variational inference framework on node orderings using the gumbel-sinkhorn gradient estimator \cite{mena2018learning}. \cite{deleu2022bayesian,nishikawa2022bayesian} employ the GFlowNet framework \cite{bengio2021gflownet} for inferring the DAG posterior. Most methods, except\cite{lorch2021dibs} are restricted to linear models, while \cite{lorch2021dibs} has high computational costs and lacks DAG generation guarantees compared to our method.
% at least quadratic scaling complexity, both with respect to the number of nodes (due to the DAG penalty) as well as number of posterior samples. Our proposed approach instead has linear complexity with respect to number of posterior samples and does not require any additional DAG penalty.     

In contrast, \emph{quasi-Bayesian} methods, such as DAG bootstrap \cite{friedman2013data}, demonstrate competitive performance. DAG bootstrap resamples data and estimates a single DAG using PC \cite{spirtes2000causation}, GES \cite{chickering2002optimal}, or similar algorithms, weighting the obtained DAGs by their unnormalized posterior probabilities. Recent neural network-based works employ variational inference to learn DAG distributions and point estimates for nonlinear model parameters \cite{charpentier2022differentiable,geffner2022deep}.
%------------------------------------------------------------------------------
\section{Limitations and Future Work}
\label{sec:limitfuture}
%------------------------------------------------------------------------------

Although we show the effectiveness of FLiCR and ePSNR, there are still some remaining  limitations.
Firstly, as we observed with the end-to-end experiments, perception models pre-trained with the original data lose their predictive performance when used with point clouds reconstructed from lossy RIs.
To alleviate this issue, there is an opportunity to make the perception models robust to point clouds from different RI resolutions.
Another opportunity is to develop  dedicated hardware logic for the processing steps in Figure~\ref{fig:rioverview}.
As shown in Figure~\ref{fig:latbreak}, the RI conversion takes a large portion of the end-to-end latency.
Accelerating the conversion process would further improve the latency benefits of FLiCR. 
%
In addition, ePSNR has a limitation.
While ePSNR as a single-number metric effectively represents the point-wise and entropy-wise point cloud qualities, it requires two parameters: $\alpha$ and $\beta$.
We manually set these parameters for our experiments, but it is not scalable.
Therefore, there is a need to further develop a tuning methodology for these parameters, or to further refine the quality metric for LiDAR point clouds.
\section{Conclusion and Future Work}
In this work, I design corruption-robust algorithms for the Lipschitz contextual search problem. I present the \emph{agnostic checking} technique and demonstrate its effectiveness in designing corruption-robust algorithms. There are several open problems for future research. First, in the algorithm I propose for pricing loss, the schedule for agnostic checks is fixed upfront. Can the learner design an adaptive checking schedule for the pricing loss? Second, this work assumes the learner has knowledge of the Lipschitz constant $L$. Can the learner design efficient no-regret algorithms without knowledge of $L$? 

\section*{Acknowledgment}

G.M. and F.T. would like to acknowledge the \gls*{ERC} since part of this work has received funding from the \gls*{ERC} under the European Union’s Horizon 2020 research and innovation programme (Grant Agreement No.\,864697).



%\bibliographystyle{ACM-Reference-Format}
\bibliographystyle{IEEEtran}
\bibliography{sample-base}

\end{document}
\endinput
%%
%% End of file `sample-authordraft.tex'.
