%------------------------------------------------------------------------------
\section{\lowercase{e}PSNR: Quality Metric for L\lowercase{i}DAR Point Clouds}
\label{sec:metric}
%------------------------------------------------------------------------------

The errors in the RI conversion process can affect the performance of LiDAR perceptions.
PSNR and CD (or RMSE) have been broadly used as the quality metrics of
3D point clouds~\cite{biswas2020muscle, huang2020octsqueeze, tu2016compressing, que2021voxelcontext, feng2020real, tu2019point, tu2019real}, but by definition they are point-wise quality metrics and do not reflect the point loss effectively.
In the context of our approach with lossy RIs, we argue a metric for both point-wise quality and overall information amount is essential.


\begin{equation}\label{eq:2}
Dist(p, C) = \min_{{p_c}}\left( (p_c-p)^2 \right)
\end{equation}

\begin{equation}\label{eq:2-2}
MSE(C_1, C_2) = \frac{1}{\left\| C_2 \right\|}\sum_{i=0}^{\left\| C_2 \right\|-1} \left\{ Dist(p_{c_2}, C_1) \right\}
\end{equation}


Both PSNR and CD use the mean squared error (MSE) of the point-wise distances between two point clouds.
When $C_1$ is the original point cloud and $C_2$ is the reconstructed point cloud, the distance between a point in $C_2$ and the corresponding point in $C_1$ is calculated by Equation~\ref{eq:2}.
So, the corresponding point in $C_1$ is of the shortest distance to the point in $C_2$.
Then, MSE between two point clouds is defined by Equation~\ref{eq:2-2}.

\begin{equation}\label{eq:3}
  \scalebox{.9}{$CD(C_{orig}, C_{comp}) = MSE(C_{orig}, C_{comp}) + MSE(C_{comp}, C_{orig})$}
\end{equation}

\begin{equation}\label{eq:4}
  \scalebox{.9}{$PSNR(C_{orig}, C_{comp}) = 10\ log\left( \frac{Max^2}{MSE(C_{orig}, C_{comp})} \right)$}
\end{equation}

Then, PSNR and CD are defined as Equation~\ref{eq:3} and~\ref{eq:4}.
CD is the sum of reciprocal MSEs between two point clouds, and PSNR is the ratio of the peak LiDAR sensor range to MSE.
As indicated by their definitions, these metrics are based on MSE and are determined by the point-to-point distance.
They natively represent the quality loss by the quantization error, but the subsampling error is not effectively represented even if the impact of SE is shown mildly as some nearest points can be lost in the reconstructed point cloud.
Specifically, in the current metrics, it is possible to get a high-quality result even with a few points of  small distances to a point cloud of  large number of points.
It is caused by the unstructured nature of the LiDAR point cloud; there is no point-to-point correspondence between LiDAR point clouds having different numbers of points.
In the case of the normal images, the total number of pixels is fixed without SE and the point-wise metrics work well.

\begin{equation}\label{eq:5}
SE = \frac{\left\| C_{orig} - C_{comp} \right\|}{\left\| C_{orig} \right\|},\ 0 \le SE\le 1
\end{equation}

Based on our observation, we argue it is inappropriate to use the metrics representing only the point-wise quality for LiDAR point clouds.
To address the limitation of the current metrics, we propose a new
single-number metric, {\bf entropy-reflecting PSNR (ePSNR)}, by extending PSNR.
ePSNR is designed to indicate both the point-wise and entropy-wise quality of a point cloud.

SE is related to the total information (entropy) loss in a point cloud
because it is the percent of the lost points, as in Equation~\ref{eq:5}.
One naive way of making PSNR reflect the entropy is to multiply $1-SE$ to PSNR with the assumption that the entropy is $1-SE$ and SE is exactly the same with the actual entropy loss, $\mathscr{L}_{SE}$.
However, instead of the naive way, we extend PSNR by estimating $\mathscr{L}_{SE}$.
Our underlying assumption is $\mathscr{L}_{SE}$ is not exactly the same with SE and follows the exponential distribution as Equation~\ref{eq:6}.
The intuition for this assumption is that SE can have minimal impacts on the downstream perceptions as far as the total amount of necessary information is preserved for the perception algorithms.
It means there would be a knee of the curve in the graph of the entropy function.

\begin{equation}\label{eq:6}
  Assumption:\quad\mathscr{L}_{SE} \sim \mathcal{\text{exp}(\beta)}
\end{equation}

When $\mathscr{L}_{SE}$ follows the exponential distribution, the entropy function $\mathcal{F}(SE)$ can be defined with the cummulative distribution function (CDF) of the exponential distribution as Equation~\ref{eq:7}.
This entropy function is a probability function estimating the actual entropy of the remaining points in a point cloud with the given SE.

\begin{equation}\label{eq:7}
\mathcal{F}(SE) = \mathrm{P}(\mathrm{E}>x) = e^{-\frac{x}{\beta}}\quad where\ x = 1-SE \\
\end{equation}

With our entropy function, ePSNR is defined as Equation~\ref{eq:8}.
Since it is based on PSNR, the point-wise quality with the quantization error is represented while reflecting the entropy with the given SE.
When SE is small, ePSNR would be almost same with the original PSNR,
but would start to decrease exponentially when SE gets larger, by its definition.
ePSNR has two parameters: $\alpha$ as a derivative adjusting factor to prevent too steep or shallow distribution and $\beta$ of the exponential distribution.

\begin{equation}\label{eq:8}
\begin{gathered}
\scalebox{.9}{$ePSNR(C_{orig}, C_{comp}) =  PSNR\times \left\{ 1 - (SE \times (\mathcal{F}(SE)+\alpha)) \right\},$}
  \\ \scalebox{.9}{$0\le \mathcal{F}(SE)+ \alpha \le 1$}
\end{gathered}
\end{equation}

