%------------------------------------------------------------------------------
\section{Limitations and Future Work}
\label{sec:limitfuture}
%------------------------------------------------------------------------------

Although we show the effectiveness of FLiCR and ePSNR, there are still some remaining  limitations.
Firstly, as we observed with the end-to-end experiments, perception models pre-trained with the original data lose their predictive performance when used with point clouds reconstructed from lossy RIs.
To alleviate this issue, there is an opportunity to make the perception models robust to point clouds from different RI resolutions.
Another opportunity is to develop  dedicated hardware logic for the processing steps in Figure~\ref{fig:rioverview}.
As shown in Figure~\ref{fig:latbreak}, the RI conversion takes a large portion of the end-to-end latency.
Accelerating the conversion process would further improve the latency benefits of FLiCR. 
%
In addition, ePSNR has a limitation.
While ePSNR as a single-number metric effectively represents the point-wise and entropy-wise point cloud qualities, it requires two parameters: $\alpha$ and $\beta$.
We manually set these parameters for our experiments, but it is not scalable.
Therefore, there is a need to further develop a tuning methodology for these parameters, or to further refine the quality metric for LiDAR point clouds.