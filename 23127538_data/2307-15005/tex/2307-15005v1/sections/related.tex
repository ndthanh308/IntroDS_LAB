%------------------------------------------------------------------------------
\section{Related Work}
\label{sec:related}
%------------------------------------------------------------------------------

Given the popularity of 3D point clouds, there are many point cloud compression methods.
Firstly, there are MPEG standard specifications: video-based point cloud compression (V-PCC) and geometry-based PCC (G-PCC)~\cite{graziosi2020overview}.
V-PCC converts 3D point clouds into 2D frames and compresses the frames with MPEG video codecs.
G-PCC directly leverages the octree structure as the intermediate representation (IR), and compresses the octree of point clouds.
Other than G-PCC, Google Draco~\cite{draco} and Point Cloud Library (PCL) compressors~\cite{rusu20113d} utilize tree structures including k-d tree and octree.
After generating the tree structure from a point cloud, the occupancy information with the leaf nodes is coded, and entropy or arithmetic coding is applied to compress the coded information~\cite{schnabel2006octree, devillers2000geometric}.
For range image compression, Tu \emph{et al.} present  direct mapping of sensor data to 2D frames by each laser ID with precision and compress these raw RIs using image compression methods~\cite{tu2016compressing}.
Other RI-based compression methods convert the raw sensor data from Cartesian coordinates into spherical coordinates by using the LiDAR sensor design~\cite{ahn2014large, feng2020real, houshiar20153d}.
Feng \emph{et al.} propose spatial encoding in the plane granularity and temporal optimization with scene alignment and prediction by using IMU fusion.
Even though these existing compression methods show decent compression performance, it is hard to apply them to our target use case of online remote perceptions because of their high latency magnitudes, as described in Section~\ref{sec:motivation}.

Recently, there has been research to utilize machine learning (ML) for LiDAR point cloud compression.
One popular approach is with the octree because the high compression ratio can be achieved by coding the tree into a more compact bytestream with  well-predicted occupancy information of a given tree~\cite{schnabel2006octree}.
By fully utilizing the relationship of neighboring nodes in the octree, the state-of-the-art works train the ML models to predict the distribution of the octree nodes~\cite{que2021voxelcontext, nguyen2021multiscale, biswas2020muscle, huang2020octsqueeze}.
With the predicted distribution, the occupancy information and nodes are effectively coded by assigning proper bits to each node of non-empty child nodes.
For RI-based ML approaches, the spatial optimization is done by using the encoder and decoder networks trained with RIs of point clouds~\cite{tu2019point, tu2019real}.
Some of these ML algorithms achieve sufficiently low latency to run in real-time~\cite{que2021voxelcontext, huang2020octsqueeze}. However, they are not practical for mobile users, because they rely on high-end processors and GPUs, which are usually unavailable for mobile devices.
Even if a mobile device has such computing resources, there is another issue with its limited battery.

