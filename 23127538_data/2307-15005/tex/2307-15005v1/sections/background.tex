%------------------------------------------------------------------------------
\section{Background}
\label{sec:background}
%------------------------------------------------------------------------------

\noindent\textbf{3D Point Clouds.\quad} A 3D point cloud is a set of points in the 3D space.
Point clouds can be categorized into two categories by their characteristics: structured and unstructured.
The unstructured (raw) point cloud is a sequence of the coordinate values of 3D points (usually \emph{x, y, z} in a Cartesian coordinate system), optionally with other attributes such as reflection intensities.
The structured point cloud is a point set organized with geometric or hierarchical structure contexts including meshes, octrees, etc.
A LiDAR point cloud is an unstructured point cloud directly captured from LiDAR sensors.


% Figure environment removed


\noindent\textbf{Unstructured Point Cloud Compression.\quad} There are
diverse existing compression methods, but a common thread across them
is to convert raw point clouds into structured intermediate
representations (IRs) and apply compression algorithms to the IRs, as
shown in Figure~\ref{fig:pcprocess}.
The compression process is tied to each IR, and the commonly used IRs are k-d tree, octree, mesh, and range image.
Figure~\ref{fig:reps} shows different IR visualizations from a raw point cloud.
Compression methods are categorized into geometry-based or image-based
compression, based on the used IRs.
Geometry-based compression uses the tree structures or mesh~\cite{rusu20113d, devillers2000geometric, mammou2019g, que2021voxelcontext, huang2020octsqueeze, biswas2020muscle, nguyen2021multiscale, draco}, and the image-based compression maps the point clouds into 2D frames~\cite{feng2020real, tu2016compressing, tu2019point, tu2019real, sun2019novel}.
The geometry-based compressions code their IRs and compress the coded IRs, and the image-based approaches utilize the existing codecs or present their own techniques for compressing the mapped images.
More details of the existing methods appear in Section~\ref{sec:related}.

% Figure environment removed

