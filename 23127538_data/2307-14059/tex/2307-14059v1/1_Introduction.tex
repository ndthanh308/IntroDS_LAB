\section{Introduction \& Motivation}
Recommender system applications on the web often operate in a ranking fashion, showing ordered lists of items to users in an attempt to optimise some metric(s) of interest.
Such metrics typically reflect user satisfaction, business goals or fairness concerns.
With the ranking paradigm comes an important caveat: items that are shown at higher positions are more likely to be \emph{exposed} to the user, and this discrepancy should be taken into account when considering data from logged user interactions~\cite{Joachims2007}.
Indeed, it has implications for evaluation~\cite{Hofmann2014, Castells2022}, learning~\cite{Joachims2017}, and fairness of exposure~\cite{Diaz2020, Oosterhuis2021, Jeunen2021B}.
The problem of \emph{position bias} and its relevance to recommendation systems has been well-studied in recent years~\cite{Vardasbi2020, Chen2022, Ruffini2022, Oosterhuis2023}.
\looseness=-1

Most of these existing works focus on the classical Information Retrieval (IR) task of web search, where documents are ranked as search results to be surfaced for a given query.
Effective methods for de-biasing in web search are often transferable to recommendation domains, when we replace \emph{queries} with \emph{users} and \emph{documents} with \emph{items}.
As a result, evaluation metrics such as Normalised Discounted Cumulative Gain (nDCG) are a common choice when assessing top-$n$ recommendation quality~\cite{Valcarce2020}.
An often overlooked point is that the \emph{discount} is directly related to \emph{position bias}, and that well-chosen discount functions are necessary to consider nDCG an unbiased offline estimator of online reward~\cite{Jeunen2021thesis}.
This is a desirable feat, as discrepancies between off- and on-line evaluation results have plagued the recommender systems field for years~\cite{Beel2013, Garcin2014, Rossetti2016, Gilotte2018, Jeunen2018, Jeunen2019DS}.
Models of user behaviour, such as those underlying the rank-biased precision (RBP)~\cite{Moffat2008} or expected reciprocal rank (ERR)~\cite{Chapelle2009} metrics, can be used to construct discount functions for nDCG-like metrics that emulate the empirical position bias in a given system well.
Existing work in this area has largely focused on web search~\cite{chuklin2015click}, with extensions to general recommendation use-cases in e-commerce~\cite{Mei2022}.

\emph{Short-video feeds} on social media platforms, however, imply very different interaction paradigms than those prevalent in web search.
Indeed, users are unlikely to abandon the feed after, for example, \emph{liking} a post.
There is no \emph{information}, but rather an \emph{entertainment} need for users scrolling the feed.
As such, user models that are prevalent in other application areas are not directly applicable to our use-case~\cite{Zhang2020}.
Aside from more general work by \citeauthor{Wu2021}~\cite{Wu2021}, this topic has received relatively little research attention.
We specifically focus on a 1\textsuperscript{st}-level feed in a hierarchical structure, where users can either keep scrolling the current feed, or enter a ``\emph{more-like-this}'' 2\textsuperscript{nd}-level feed via any 1\textsuperscript{st}-level item.
Indeed, such user interfaces have gained popularity recently, and can be found on Reddit, Instagram and ShareChat, among others.
Our hypothesis is that users come to the platform with a ``\emph{scrolling budget}'', reflecting how far they are willing to scroll before abandoning the feed.
This \emph{budget} is personalised, context-dependent, and drawn from a discrete power-law distribution such as the Yule-Simon distribution with shape parameter $\rho$~\cite{Yule1925, Simon1955}.
Figure~\ref{fig:yulesimon}(a) visualises how this family of distributions can represent a wide variety of stochastic budgets and, hence, scrolling behaviours.

We show how the survival function of this distribution can be used to obtain closed-form estimates for personalised exposure probabilities that have a sound theoretical basis, and show how they pave the way for improved unbiased evaluation and learning-to-rank in feed recommendation settings.

The main contributions we present in this paper are the following:
\begin{enumerate}
\item We propose a novel Contextual, Personalised, Probabilistic POsition bias model for feed recommendations: \texttt{C-3PO}.
\item We empirically validate using real-world data that \texttt{C-3PO} is better able to capture exposure probabilities than existing methods, whilst having a stronger theoretical basis.
\item We show how \texttt{C-3PO} can be used for improved unbiased evaluation and learning in feed ranking scenarios.
\end{enumerate}