%\documentclass[sigconf, screen, dvipsnames, anonymous]{acmart}
\documentclass[manuscript, screen, dvipsnames]{acmart}
% Remove DOI Information for proposal template
% \makeatletter
% \renewcommand\@formatdoi[1]{\ignorespaces}
% \makeatother
% Package definitions
% \usepackage[noend]{algorithmic}
% \usepackage{algorithm}
\usepackage{multirow}
% \usepackage{flushend}
\usepackage{amsmath}
% \usepackage{amssymb}
% \usepackage{mathtools}
% \usepackage[acronym]{glossaries}
% \usepackage{units}
% \usepackage{booktabs}
\usepackage[inline]{enumitem} % Inline lists
\usepackage{subcaption}
% \usepackage{bm} % Bold greek
\usepackage{xcolor}

 \usepackage{lipsum}
% Dashed lines in tables
\usepackage{arydshln}
\makeatletter
\def\adl@drawiv#1#2#3{
        \hskip.5\tabcolsep
        \xleaders#3{#2.5\@tempdimb #1{1}#2.5\@tempdimb}%
                #2\z@ plus1fil minus1fil\relax
        \hskip.5\tabcolsep}
\newcommand{\cdashlinelr}[1]{%
  \noalign{\vskip\aboverulesep
          \global\let\@dashdrawstore\adl@draw
          \global\let\adl@draw\adl@drawiv}
  \cdashline{#1}
  \noalign{\global\let\adl@draw\@dashdrawstore
          \vskip\belowrulesep}}
\makeatother

% PGM in Tikz
\usepackage{tikz}
\usetikzlibrary{automata, arrows, bayesnet, bending}

\usepackage{tikzscale}

% Reduce space after floats (such as tables and figures)
\setlength{\textfloatsep}{1pt plus 0pt minus .5pt}
\setlength{\intextsep}{1pt plus 0pt minus 1.0pt}
\setlength{\abovecaptionskip}{2pt plus 0pt minus 1pt}
\setlength{\belowcaptionskip}{2pt plus 0pt minus 1pt}
% Reduce space around equations
\setlength{\abovedisplayskip}{2pt plus 1pt minus 1pt}
\setlength{\belowdisplayskip}{2pt plus 1pt minus 1pt}

% Argmax as mathematical operator
\DeclareMathOperator*{\argmax}{arg\,max}
\DeclareMathOperator*{\argmin}{arg\,min}
\DeclareMathOperator{\vect}{vec}

%Conference
\copyrightyear{2023}
\acmYear{2023}
\setcopyright{acmlicensed}
\acmConference[RecSys '23]{Seventeenth ACM Conference on Recommender Systems}{September 18--22, 2023}{Singapore, Singapore}
\acmBooktitle{Seventeenth ACM Conference on Recommender Systems (RecSys '23), September 18--22, 2023, Singapore, Singapore}
\acmPrice{15.00}
\acmDOI{10.1145/3604915.3608777}
\acmISBN{979-8-4007-0241-9/23/09}

\pdfoutput=1

\begin{document}
\title{A Probabilistic Position Bias Model for Short-Video Recommendation Feeds}

\author{Olivier Jeunen}
\affiliation{
  \institution{ShareChat}
  \city{Edinburgh}
  \country{United Kingdom}
} 

\begin{abstract}
Modern web-based platforms often show ranked lists of recommendations to users, in an attempt to maximise user satisfaction or business metrics.
Typically, the goal of such systems boils down to maximising the exposure probability ---conversely, minimising the rank--- for items that are deemed ``\emph{reward-maximising}'' according to some metric of interest.
This general framing comprises music or movie streaming applications, as well as e-commerce, restaurant or job recommendations, and even web search.
\emph{Position bias} or \emph{user} models can be used to estimate exposure probabilities for each use-case, specifically tailored to \emph{how} users interact with the presented rankings.
A unifying factor in these diverse problem settings is that typically only one or several items will be engaged with (clicked, streamed, purchased, et cetera) before a user leaves the ranked list.

\emph{Short-video feeds} on social media platforms diverge from this general framing in several ways, most notably that users do not tend to leave the feed after, for example, \emph{liking} a post.
Indeed, seemingly infinite feeds invite users to scroll further down the ranked list.
For this reason, existing position bias or user models tend to fall short in such settings, as they do not accurately capture users' interaction modalities.
In this work, we propose a novel and probabilistically sound personalised position bias model for feed recommendations.
We focus on a 1\textsuperscript{st}-level feed in a hierarchical structure, where users may enter a 2\textsuperscript{nd}-level feed via any given 1\textsuperscript{st}-level item.
We posit that users come to the platform with a given \emph{scrolling budget} that is drawn according to a discrete power-law distribution, and show how the \emph{survival function} of said distribution can be used to obtain closed-form estimates for personalised exposure probabilities.
Empirical insights gained through data from a large-scale social media platform show how our probabilistic position bias model more accurately captures empirical exposure than existing models, and paves the way for improved \emph{unbiased} evaluation and learning-to-rank.
\end{abstract}

%
% The code below should be generated by the tool at
% http://dl.acm.org/ccs.cfm
% Please copy and paste the code instead of the example below.
\begin{CCSXML}
<ccs2012>
   <concept>
       <concept_id>10002951.10003317.10003347.10003350</concept_id>
       <concept_desc>Information systems~Recommender systems</concept_desc>
       <concept_significance>500</concept_significance>
       </concept>
   <concept>
       <concept_id>10002951.10003317.10003359</concept_id>
       <concept_desc>Information systems~Evaluation of retrieval results</concept_desc>
       <concept_significance>500</concept_significance>
       </concept>
   <concept>
       <concept_id>10002951.10003317.10003371</concept_id>
       <concept_desc>Information systems~Specialized information retrieval</concept_desc>
       <concept_significance>300</concept_significance>
       </concept>
   <concept>
       <concept_id>10010147.10010257.10010293.10010300</concept_id>
       <concept_desc>Computing methodologies~Learning in probabilistic graphical models</concept_desc>
       <concept_significance>100</concept_significance>
       </concept>
 </ccs2012>
\end{CCSXML}

\ccsdesc[300]{Information systems~Specialized information retrieval}
\ccsdesc[500]{Information systems~Recommender systems}
\ccsdesc[500]{Information systems~Evaluation of retrieval results}
\ccsdesc[100]{Computing methodologies~Learning in probabilistic graphical models}

\keywords{Probabilistic Modelling; Position Bias; Mean Reciprocal Rank}

\maketitle

%%%%%%%%%%%%%%%%%%%%%%%%%%%%%%%%%%%%%%%%%%%%%%%%%%%%%%%%%%%%%%%%%%%%%%%%%%%%%%%%
\section{Introduction}

Autonomous driving (AD) %with deep learning networks 
has shown promising achievements and is considered an important technological breakthrough that could revolutionize the future of transportation. Currently, ensuring the safety of autonomous driving systems has become a topic of extensive development.
% There has been much discussion on how to verify the safety of autonomous driving systems.
One traditional solution for safety tests is to exhaustively enumerate real scenarios for validation. Nevertheless, this process is not only labor-intensive and costly but also dangerous. Simulation has emerged as a robust, safe, and efficient alternative for training and evaluating AD software and algorithms~\cite{li2019aads, amini2020learning, amini2022vista}.

% Figure environment removed

Recently, neural radiance field (NeRF)~\cite{mildenhall2020nerf} has gained significant attention in AD simulation~\cite{drivesim}. This approach leverages multi-view images to construct a 3D scene and enable novel view synthesis for both indoor and outdoor applications. When it comes to constructing NeRF models in AD simulation, there are two options available: 1) collecting a large amount of data to cover as many viewpoints as possible, and constructing a fine-grained scene offline; 2) directly using log data from road tests to quickly create an environment and dynamically simulate driving scenarios. The first choice can deliver high-quality simulation~\cite{tancik2022block} by transforming the problem of view extrapolation into view interpolation through the use of large amounts of data. However, it is time- and cost-intensive, which makes it challenging to generalize. As for the second choice, the collected images from log data are usually similar to each other along the running trajectory, which may result in unsatisfactory outcomes, particularly when the camera pose is placed out-of-trajectory (see \figref{figSupportComp} as an example), semantic consistency cannot be guaranteed when synthesizing images from deviated views. We observe this problem under this data condition in all neural radiance approaches, and to the best of our knowledge, none of the existing work has solved this issue.
In our opinion, semantic consistency is crucial for AD simulation, and synthesizing on deviated views is unavoidable for scalability.

AD simulation usually involves map data for planning and control, which can be obtained from a prebuilt High-Definition Map (HD Map) or an online mapping module. While the map data may not be pixel-perfect, it can provide semantic-level information that is useful for enhancing the semantic consistency of the trained neural radiance field.
In this paper, we propose incorporating map priors into neural radiance fields to enhance the semantic consistency and rendering quality of deviated driving view synthesis. Firstly, we employ ground information from maps to supervise the density field of NeRF, providing a more reliable road base for semantic entities. Next, we propose sampling rays to simulate unseen views. Unlike most NeRF augmentation methods~\cite{zhang2022ray, chen2022geoaug}, we utilize ground and lane information in sampling computations to guide the radiance field. More importantly, we model the above two supervision methods as weak supervision by using an uncertainty parameter and propose an uncertainty tempering scheme to increase the uncertainty. This ensures that map priors only guide the training process rather than enforce it towards their absolute values. As a result, our proposed method not only improves the rendering quality of interpolated novel view synthesis quantitatively but also enhances the semantic consistency of deviated novel view synthesis. 
Our contributions can be summarized as follows:
% We summarize the contributions of this paper as follows.



% To overcome the limitations of the collected data, this paper proposes a novel approach that leverages map information to enhance the semantic consistency of the synthesized driving views. 

% Autonomous driving (AD) vehicles are being trained with the help of deep learning networks and have shown promising achievements. This technology is considered to be a breakthrough that could change the way of transportation in the near future. However, there are many discussions on how to verify or judge the safety of autonomous driving systems.
% A straightforward solution towards the safety tests is to exhaustively enumerate real scenarios for validation as many as possible. However, the process of implementing different real scenarios is not only labor-intensive and costly, but also dangerous. Simulation has been proved to be an alternative, which is robust, safe, efficient in training, and evaluating AD software and algorithms.
% Now, the emerging technology of neural radiance field (NeRF)~\cite{} leverages multi-view images to construct a 3D scene and enable novel view synthesis for many indoor and outdoor applications. For AD simulation, there are two choices for constructing NeRF models: 1) collect a large amount of data, such as LiDAR and camera data, similar to mapping, to construct a fine-grained scene offline; or 2) directly use the log file (typically in the format of ROS bag) to rapidly create an environment and then dynamically simulate the driving scenarios.
% The first choice can achieve high-quality simulation, but it is time-consuming and expensive, making it difficult to generalize to very large scales. On the other hand, the second option is fast but can lead to low-quality simulation due to the data being sparse and similar to each other in log data. This paper tackles the problem raised by choosing the latter option and attempts to improve the quality of out-of-trajectory driving view synthesis by incorporating map information. This approach is practical for many autonomous driving tests.
% In conclusion, the use of NeRF technology for AD simulation is a promising avenue for training and evaluating AD software and algorithms. While both options for constructing NeRF models have their pros and cons, this paper addresses the challenges of the second option and proposes a potential solution to improve the quality of simulation.

%There exist a few attempts to facilitate training a NeRF model for synthesizing out-of-trajectory (or called as extrapo trajectory) views.


\begin{itemize}
    \item We propose a novel method to incorporate commonly used map priors in AD scenes into neural radiance fields to improve the out-of-trajectory driving view synthesis.
    \item We explicitly model the uncertainty in map priors as a parameter and propose an uncertainty tempering scheme to guide the training process of the neural radiance field.
    \item Experiments demonstrated that the proposed method can improve the semantic consistency of out-of-trajectory views and the rendering quality of novel view trajectory interpolation.
\end{itemize}

Our proposed method is easy to implement, can be easily plugged into existing NeRF algorithms, and has the capability of extending to other formats of priors.
\section{Methods} \label{sec:methods}
\subsection{Data} \label{ssec:data}

EEG is a non-invasive technique to record the brain's electrical activity. EEG data in this paper refers to these measurements, used often in research and healthcare to identify neurological conditions. In this work, we use five publicly accessible datasets, namely TUH EEG Corpus \cite{tuheeg}, TUH EEG Artifact (TUAR) Corpus, TUH EEG Events (TUEV) Corpus, TUH EEG Seizure (TUSZ) Corpus \cite{tuhz} and the EEG Motor Movement/Imagery (MMIDB) Dataset \cite{mmidb}.

\quad The TUH EEG Corpus contains 69,652 clinical and unlabeled EEG recordings obtained from Temple University Hospital (TUH). The TUH EEG Artifact Corpus, a labeled subset of the TUH EEG Corpus, includes annotations for five distinct artifacts including eye movement artifact (\emph{eyem}). The TUEV is a subset of the TUH EEG Corpus and includes annotations of event-based EEG segments. There are numerous categories, but we primarily focus on five key classes: (1) technical artifacts (\textit{artf}), (2) background (\textit{bckg}), (3) generalized periodic epileptiform discharge (\textit{gped}), (4) periodic lateralized epileptiform discharge (\textit{pled}), and (5) spike and slow wave (\textit{spsw}). The TUSZ contains EEG signals with manually annotated data for seizure events.

\quad The MMIDB EEG dataset consists of data from 109 participants who are performing or imagining specific motor tasks; our main interest is the moments when subjects either close or imagine closing their left or right fist following a visual cue. We are excluding participants S088, S090, S092, and S100 due to missing data, resulting in 105 participants.

%\subsubsection{Resting-State Dataset} \label{sssec:resting}
\quad In the construction of brain anatomy concepts, it is imperative to obtain an extensive collection of resting-state EEG data. Due to the limited availability of public datasets with the requisite size and reliability, we utilized The TUH EEG Corpus and source localization to develop a dedicated anatomically labeled resting-state dataset. A set of predefined criteria were employed, including the number of EEG channels, minimum duration, minimum sampling frequency, scaling, and the exclusion of extreme values, which led to the elimination of approximately 90\% of the initial EEG recordings. Following this, a manual examination of a part of the remaining data was performed, ultimately yielding 200 human-verified resting-state EEG recordings, corresponding to an aggregate of about 70 hours of EEG data.

%\subsection{Preprocessing of data}
\quad In the process of downstream fine-tuning and concept formation, we employ 19 EEG channels, namely $\textit{Fp1}$, $\textit{Fp2}$, $\textit{F7}$, $\textit{F3}$, $\textit{Fz}$, $\textit{F4}$, $\textit{F8}$, $\textit{T7}$, $\textit{C3}$, $\textit{Cz}$, $\textit{C4}$, $\textit{T8}$, $\textit{T5}$, $\textit{P3}$, $\textit{Pz}$, $\textit{P4}$, $\textit{T6}$, $\textit{O1}$, and $\textit{O2}$ (see the MNE documentation \cite{doi:10.1098/rsta.2011.0081} for more information). These channels originate from the initial pre-training of BENDR using The TUH EEG Corpus. In instances where the datasets lack these channels, we establish the following mapping: $T3 \mapsto T7$, $T4 \mapsto T8$, $P7 \mapsto T5$, and $P8 \mapsto T6$. We also resample the corresponding EEG data to a 256 Hz sampling frequency and apply a high-pass FIRWIN filter with a 0.1 Hz cutoff, a low-pass FIRWIN filter with a 100.0 Hz cutoff, and a 60 Hz FIRWIN notch filter to eliminate powerline noise. In situations where preprocessing cannot be performed, the EEG is excluded. Finally, we scale each trial to the range $[-1, 1]$ and append a relative amplitude channel, see \cite{BENDR}, resulting in a total of 20 channels.

%\subsection{Training of BENDR}

\subsection{Training}
Pre-training of BENDR is based on the large set of unlabelled EEG data from The TUH EEG Corpus. The pre-training procedure is largely based on \verb|wav2vec 2.0| and involves two main stages: The convolutional stage and the transformer stage. The convolutional stage generates a sequence of representations (BENDRs) that summarize the original input. This sequence is then fed into the transformer stage, which adjusts its output to be most similar to the encoded representation at each position. The layers affected during pre-training are the feature encoder and the transformer. Kostas et al.\ \cite{BENDR} kindly made the pre-trained weights of the encoder and contextualizer publicly available, and this is the model that we have employed here.
%\footnote{Pre-trained model weights:\\ \url{https://github.com/SPOClab-ca/BENDR/releases/tag/v0.1-alpha}}.

%\subsubsection{Downstream fine-tuning}
\quad The LHB model architecture described in Figure \ref{fig:linear_head_bendr} is used for downstream fine-tuning. We aim to optimize the model for two distinct binary classification objectives. First, the model is fine-tuned for the differentiation between \emph{seizure} and \emph{non-seizure} events, using the TUSZ Corpus with 60-second window segments. The hyperparameters are determined using Bayesian optimization to maximize the validation $F_1$-score. The fine-tuning employs a batch size of 80, a learning rate of $1 \times 10^{-4}$, and $30$ epochs. This results in a model with a balanced accuracy of $0.73 \pm 0.07$. 

\quad In our second fine-tuning example, the model is adapted for the differentiation between \emph{Left Fist Movement} versus \emph{Right Fist Movement}, using the MMIDB EEG Dataset with 4-second window segments. We are using both the imaginary and performed task data from the 105 participants. We train the model for $7$ epochs with a batch size of $4$ and a learning rate of $1 \times 10^{-5}$. The hyperparameters were chosen based on the best validation balanced accuracy from leave-one-subject-out cross-validation where the model was trained for 50 epochs and the best model was retained. The specific hyperparameter configuration aligns with the optimal hyperparameters found by the original authors \cite{BENDR} and we find a similar balanced accuracy of $0.83 \pm 0.02$.


\subsection{Constructing Concepts}
\label{subsec:explanatory_concepts}

To construct human-aligned explanatory EEG concepts, a number of initial investigations were conducted. The data processing involved follows the methodology previously mentioned. In this section, we provide a general pipeline overview and discuss several choices made throughout the process.

\vspace{0.5em}
{\bf Concepts from Labeled EEG Data}: Using the labeled EEG data from the TUAR and TUEV Corpus and the MMIDB EEG Dataset, we create concepts representing activities within specific time windows. Each annotated segment of the EEG data is divided into windows of predetermined length and assigned the corresponding label.

\quad In the TUEV Corpus, we define concepts for the spike/short wave (\textit{spsw}), periodic lateralized epileptic discharge (\textit{pled}), general period epileptic discharge (\textit{gped}), technical artifact (\textit{artf}), and background (bckg) with 60-second windows. This approach aligns with the length of the \textit{seizure} classifier.

\quad Lastly, we examine the eye movement (\textit{eyem}) from the TUAR Corpus and \textit{Left Fist Movement} and \textit{Right Fist Movement} from the MMIDB EEG Dataset, both using 4-second windows. These different-sized windows then constitute examples of concepts defined based on their labels.


\vspace{0.5em}
{\bf Anatomical Concepts from Unlabeled EEG Data}:
The objective is to identify concepts representing specific frequency bands within distinct areas of the cortex, e.g. \emph{alpha activity in pre-motor cortex} or \emph{gamma activity in early visual cortex}. To obtain a non-task-specific representation of each cortical area, we utilize resting-state EEG data, as it spontaneously generates activity throughout the cortex. For this purpose, we use a subset of The TUH EEG Corpus, as described above.

\quad To define anatomical concepts, EEG data is segmented into 4-second windows, with the first and last 5 seconds of each sequence  excluded to minimize artifact contamination. The data is then divided into five frequency bands with a FIRWIN bandpass filter: \emph{delta} (0.5-4Hz), \emph{theta} (4-8Hz), \emph{alpha} (8-12Hz), \emph{beta} (12-30Hz), and \emph{gamma} (30-70Hz). The inverse operator for the forward model is computed using eLORETA \cite{doi:10.1098/rsta.2011.0081} via the MNE Python library. Since the spatial resolution is not critical, minimal regularization of $1 \times 10^{-4}$ is applied.

\quad Using the combined version of the multi-modal parcellation of the human cerebral cortex, HCPMMP1 \cite{f8095709e11547daa07262682e1545f2} and the inverse operator, the average power of electrical activity in 23 cortical areas for each hemisphere is determined.

\quad Our interest lies in cortical areas exhibiting the greatest deviation from typical activity within a specific frequency band. However, cortical areas are not equidistant from the scalp or consistent in baseline activity across bands. To normalize for these differences in the distribution of cortical activity, we compute the mean and standard deviation of the power in each cortical area for each frequency band on an EEG session level, which will be employed in various ways. We call these the baseline mean and the baseline standard deviation.

\quad We explore possible approaches to how the baseline means and standard deviation for each EEG session could be used to normalize the power of 4-second windows within that session. The options include dividing by the baseline standard deviation to account for scalp source variation, subtracting or dividing by the baseline mean to identify the cortical area with the greatest deviation, taking the absolute difference or not, and selecting a single cortical area across all frequency bands or only within a specific band.

\quad Identifying a single frequency and cortical area for each 4-second window of EEG data is a challenging task without prior work to guide the process, and each method presents its own limitations. We specifically look for \emph{alpha} desynchronization in the cerebral cortex during imagined or actual movement and closed or open eyes in the MMIDB EEG dataset, i.e., that \emph{alpha} activity in cortical areas decreases when activated.
Using a paired t-test to examine the presence of lateralization in cortical activities for different methods, we found that the preferred approach is to choose the area which maximizes the absolute difference between the given time window's power and the baseline mean, divided by the baseline standard deviation, only within specific frequency bands.

\vspace{0.5em}
{\bf Random Concepts:} Construction of CAVs calls for data examples that are considered random with respect to the concept of interest. In all experiments, random concepts consisting of 4-second or 60-second windows were randomly sampled from resting-state data obtained from the subset of the TUH EEG Corpus and unannotated sections of the TUAR dataset. 

 \subsection{Experiments} \label{subsec:experiments}
We investigate two approaches for defining explanatory concepts in EEG data. The TCAV method is then employed to evaluate whether the LHB model uses specifically defined human-aligned concepts of EEG data. For all concepts, the resulting activation vectors for all five bottlenecks in the LHB model architecture are examined to determine if they significantly align with the latent representations of class data in the model. We conduct the following experiments:
% Figure environment removed
%
\begin{enumerate}
\item \textbf{Sanity Checks:} We verify the TCAV method and construction of concepts function as intended through a series of sanity checks when classifying \textit{Left Fist Movement}.
\item \textbf{Event-based Concepts:} We assess whether the LHB model leverages specific EEG events in the classification of \textit{seizure}.
\item \textbf{Anatomy/Frequency-based Concepts:} We investigate if the LHB model employs lateralization in cortical activity in the \emph{alpha} band for classifying \textit{Left Fist Movement}. The chosen cortical areas are based on their relevance to the classification task.
\end{enumerate}
%
In the experiments, we use the TCAV method with a regularized linear model and stochastic gradient descent (SGD) learning, setting the regularization parameter $\alpha = 0.1$ to learn the decision boundary between explanatory and random concepts. We employ 50 random concepts and a maximum of 40 examples per concept. These parameters were chosen to increase statistical power. The mean TCAV scores for the target concept examples and the random examples are compared using the non-parametric Mann-Whitney U Rank test, as opposed to the t-test used in the original TCAV method, as we observed a clear violation of the normality assumption for the TCAV scores. To mitigate Type I errors, the p-values are corrected for each experiment employing the conservative Bonferroni method, after which we claim significance if the corrected p-value is below $0.05$.


\section{Experimental Results \& Discussion}\label{sec:experiments}
We wish to answer the following research questions experimentally:
\begin{description}
\item[\textbf{RQ1}] \textit{Is our proposed probabilistic method able to model exposure probabilities more accurately than existing methods?}
\item[\textbf{RQ2}] \textit{Can the model leverage contextual signals effectively?}
\item[\textbf{RQ3}] \textit{Are the obtained position biases useful for downstream tasks, such as unbiased offline evaluation?}
\end{description}

Naturally, position biases are heavily influenced by specific use-cases, platforms and interface choices.
The methods we propose in this work are motivated by a short-video feed recommendation use-case, and even though our proposed framework is generally applicable, we expect the Yule-Simon instantiation to only hold merit in similar use-cases.

In order to empirically validate the performance of both our and earlier proposed methods, we require \emph{interventional} data with logged views $V$, ranks $R$, and contexts $X$.
To the best of our knowledge and at the time of writing, we are unaware of any such datasets being publicly available.
Existing Learning-to-Rank (LTR) datasets do not contain rank interventions and deal with web search use-cases, which imply very different modalities to ours.
For this reason, we need to resort to proprietary datasets, but additionally release an open-source Jupyter notebook to reproduce the position bias curves visualised in Figure~\ref{fig:yulesimon} at \href{https://github.com/olivierjeunen/C-3PO-recsys-2023}{github.com/olivierjeunen/C-3PO-recsys-2023}.

\subsection{Estimating Exposure Probabilities (RQ1--2)}
We obtain a sample of 1 million sessions of feed view events on a social media platform, where rank interventions occurred following Fig.~\ref{fig:PGM}, collected over five days in February 2023.
We perform an 80-20\% train-test split, aiming to predict whether recommendations were viewed based on their rank and contextual information.

We compare several non-contextual variants: the standard DCG discount function as well as the logarithmic and exponential forms in Eq.~\ref{eq:dcg}, and the probabilistic method based on the Yule-Simon distribution introduced in Eq.~\ref{eq:prob}.
The latter three methods include a single parameter ($\alpha, \gamma, \rho$ respectively), which we learn to minimise NLL@$K$ on the training set, following the procedure laid out in \S\ref{sec:learning}.
We implement this in Python 3.9 with the SciPy library~\cite{Virtanen2020}.

As an additional baseline, we include a non-parametric method that predicts the empirical average from the training data.
This approach should be expected to outperform the aforementioned methods, but it requires a hard-coded probability at every rank instead of the single parameter that the logarithmic, exponential, or probabilistic forms require.
Additionally, this approach cannot easily be extended to incorporate contextual information $X$.

For the contextual case, we adopt a single continuous user-based feature describing users' past average scroll depth, as well as a single continuous context-based feature, describing average scroll depth at the time of day. 
We adopt a simple linear model to estimate the distribution parameter from this input $X$: $\rho_{\theta}(x) = \theta^{\intercal}x$.
The functional forms for the parameters $\alpha$ and $\gamma$ are analogous.
As such, the contextual and personalised methods consist of only \emph{three} parameters each (assuming $x$ includes a constant 1-feature, emulating a bias term in $\theta$).
Even in this simplistic scenario, the contextual and personalised methods significantly outperform those that do not consider this information, as shown in Table~\ref{tab:results}.
Our contextual, personalised, probabilistic position bias model \texttt{C-3PO} achieves the lowest NLL@$K$ for a wide range of $K$, whilst requiring a minimum of learnable parameters or computing resources.
This yields a desirable trade-off between parsimony and model expressiveness when compared to complex model classes like neural networks (which would typically require orders of magnitude more parameters). 
We observe that this additionally allows us to be sample-efficient, as our method already performs well with only $\mathcal{O}(10^{4})$ samples.
Indeed, instead of modelling the entire curve at every possible value of $r$, our proposed method outputs a single scalar which can be used to obtain position bias estimates for all natural numbers.
The inductive bias we enjoy from well-motivated mathematical models greatly improves the methods' real-world usability, when compared to neural-network based alternatives.

\begin{table}[t]
    \centering
    \begin{tabular}{lcccccc}
    \toprule
    \multirow{2}{*}{\textbf{Model}} & \multicolumn{5}{c}{\textbf{Negative Log-Likelihood (NLL)}}\\
     ~& \textbf{@5} & \textbf{@10} & \textbf{@25} & \textbf{@50} & \textbf{@100} \\
    \cline{2-6}

    $\widehat{\mathsf{P}}_{{\rm dcg}}(V|R)$ & 0.5453 & 0.6320 & 0.5998 & 0.4973 & 0.3763\\
    $\widehat{\mathsf{P}}_{\log}(V|R)$ & \underline{0.5159} & \underline{0.6001} & 0.5900 & 0.5036 & 0.3833\\
    $\widehat{\mathsf{P}}_{\exp}(V|R)$ & 0.5202 & 0.6158 & 0.6089 & 0.5101 & 0.3673\\
    $\widehat{\mathsf{P}}_{\rm prob}(V|R)$ & \underline{0.5162} & \underline{0.6002} &\underline{0.5873} & \underline{0.4891} & \underline{0.3495}\vspace{1ex}\\
    \cdashline{1-6}
    \vspace{-2ex}~\\
    $\widehat{\mathsf{P}}_{{\rm empirical}}(V|R)$ & 0.5157 & 0.5999 & 0.5843 & 0.4813 & 0.3369\vspace{1ex}\\
    \cdashline{1-6}
    \vspace{-2ex}~\\
    $\widehat{\mathsf{P}}_{\log}(V|R,X)$ & \textbf{0.4852} & \textbf{0.5620} & 0.5577 & 0.4806 & 0.3555\\ 
    $\widehat{\mathsf{P}}_{\exp}(V|R,X)$ & 0.4883 & 0.5761 & 0.5778 & 0.4959 & 0.3652\\ 
    $\widehat{\mathsf{P}}_{\rm prob}(V|R,X)$ & \textbf{0.4850} & \textbf{0.5620} &\textbf{0.5551} & \textbf{0.4651} & \textbf{0.3325}\\ 
\bottomrule
    \end{tabular}
    \caption{NLL for position bias models on observed data, lower is better.
    The top-group are independent of contextual information, the middle baseline is a non-parametric method that predicts a sample average, the bottom-group include three parameters that were optimised via linear regression. 
    Marked fields indicate stat. sig. improvements over other methods in the same group at a 99\% level.}
    \label{tab:results}
\end{table}

\subsection{Unbiased Offline Evaluation (RQ3)}
The main task position bias models need to perform, is to deliver offline estimates of online performance.
Given a dataset of logged impressions $\mathcal{D}\coloneqq \{(x_{i}, a_{i}, r_{i}, c_{i})_{i=1}^{N}\}$ (contexts, actions, ranks, rewards) , we wish to estimate the expected reward we would have obtained under some different ranking policy $\pi$.
This policy maps a context $X$ and set of \emph{candidate} items $\mathcal{A}_{x}$ to a ranked list.
We will denote with the shorthand notation $\pi(a|x)$ the rank that item $a$ will be placed at when $\pi$ is presented with context $x$ (assuming $\mathcal{A}_{x}$ given).
Note that this framing is easily extended to more general stochastic ranking policies~\cite{Oosterhuis2021}.
Then, a dataset $\mathcal{D}$ and position bias model $\widehat{\mathsf{P}}$ can be used to to obtain an unbiased estimate of the reward we would obtain under $\pi$:
\vspace{-1ex}
\begin{equation}\label{eq:unbiased_dcg}
    \mathop{\mathbb{E}}\limits_{r \sim \pi}[C] \stackrel{1}{\approx} {\rm DCG}_{\widehat{\mathsf{P}}}(\mathcal{D}, \pi) \stackrel{2}{\approx}
    \frac{1}{N}\sum_{i=1}^{N}
    c_{i} \cdot \frac{\widehat{\mathsf{P}}(V=1|R=\pi(a_{i}|x_{i}), X=x_{i})}{\widehat{\mathsf{P}}(V=1|R=r_{i}, X=x_{i})} .
\end{equation}
Here, the first approximation $\stackrel{1}{\approx}$ is due to the inherent assumptions of the DCG metric (compared to, e.g., cascade-based alternatives), whereas the second only exists because we resort to an empirical average over the observed data $\mathcal{D}$ and estimated position biases via $\widehat{\mathsf{P}}$.
Assuming unbiasedness of $\widehat{\mathsf{P}}$, the unbiasedness of the metric in Eq.~\ref{eq:unbiased_dcg} is easily recognised, as it is an application of importance sampling or IPS~\cite{Owen2013}.
As is typical for IPS-based methods, techniques like capping or introducing control variates can improve their finite-sample performance by reducing variance~\cite{Gilotte2018, Swaminathan2015snips}.
We do not consider such extensions in this short article, but remark that they are likely to further improve performance.

To validate the utility of these offline estimates, we perform an online experiment on a social media platform that operates a short-video recommendation feed.
Thus, we obtain samples from the reward distribution by sampling $\mathbb{E}_{r \sim \pi}[C]$ directly and taking an empirical average per day, for five days.
Then, for varying context-independent position bias models (optimised $@100$), we obtain offline estimates of online reward via Eq.~\ref{eq:unbiased_dcg}, and evaluate the offline estimates by Pearson's correlation coefficient between the ground truth and the offline estimate, over 5 days.

Table~\ref{tab:results2} shows relative improvements in correlation over the classical DCG formulation.
We observe that our probabilistically motivated position bias model is able to significantly improve the offline-online correlation compared to existing methods, and conjecture that the context-dependent variant can lead to further improvements.
This highlights the importance of a well-motivated position bias model, and is a strong argument in favour of our proposed methods.

\begin{table}[t]
    \vspace{-3ex}
    \centering
    \begin{tabular}{lllccccccc}
    \toprule
    \textbf{Position Bias Model} &~&~& $\widehat{\mathsf{P}}_{{\rm dcg}}(V|R)$ &~& $\widehat{\mathsf{P}}_{{\rm log}}(V|R)$ &~& $\widehat{\mathsf{P}}_{{\rm exp}}(V|R)$ &~& $\widehat{\mathsf{P}}_{{\rm prob}}(V|R)$ \\
         \cline{4-10}
\textbf{Relative Improvement} &~&~& 100\% &~& -3\% &~& -20\% &~& \textbf{+16\%} \\
    
\bottomrule
    \end{tabular}
    \caption{Relative correlation improvement over $\widehat{\mathsf{P}}_{{\rm dcg}}$ between DCG estimates and online metrics, higher is better.}
    \label{tab:results2}
\end{table}
\section{Discussion}\label{sec:Ccl}
We introduced in this paper a generative covariance model with repeated eigenvalues called SPCA, which generalises PPCA~\citep{tipping_probabilistic_1999} and IPPCA~\citep{bouveyron_intrinsic_2011} under a unique geometric framework relying on flag manifolds.
We noticed that the parsimony of PPCA comes from the low-rank model and the emergence of a multidimensional isotropic noise eigenspace. This raised the natural question of extending the isotropy constraint to the signal space.
The SPCA model showed that assuming distinct eigenvalues in the signal space---as PPCA does---is not justified in practice. Hence, SPCA could circumvent this issue by equalising the adjacent eigenvalues with small gaps and gathering the associated eigenvectors into multidimensional eigenspaces.
We confirmed our expectations on synthetic and real datasets, showing how SPCA models achieve a better complexity/goodness-of-fit tradeoff than PPCA.
The code is available on GitHub\footnote{\url{https://github.com/tomszwagier/stratified-pca}}.


SPCA is at an early stage of research and its development has been requiring several limiting choices that could be relaxed and improved in future works.
A first limit is the choice of the BIC for model selection. Indeed, the BIC is known to favor under-parameterised models and not work very well in the small-data regime. However, this does not prevent it from being widely used due to its simplicity. Therefore, it provides an elementary way to highlight the interest of SPCA, similarly as~\cite{tipping_probabilistic_1999} used a simple model selection criterion when introducing PPCA. One could later investigate extensions of~\cite{minka_automatic_2000} (which is relying on a geometric interpretation of PPCA with Stiefel manifolds) and~\cite{drton_bayesian_2017} to SPCA models.
A second limit is the linear-Gaussian nature of SPCA which is not suited to real data. Some nonlinear and non-Gaussian extensions could therefore be considered in the future. The probable lack of analytic solution would involve optimisation on flag manifolds~\citep{ye_optimization_2021}. Due to the cost of inference for each model, we might need to replace discrete model selection with a global optimisation scheme on the space of all SPCA models. The latter being stratified by eigenvalue multiplicity, we could benefit from recent works on stratified optimisation~\citep{leygonie_gradient_2023, olikier_first-order_2023}.



SPCA also comes with several interesting perspectives.
First, it unleashes a whole new family of parsimonious linear-Gaussian models interpolating between the isotropic model and the full covariance one. Hence when a PPCA model overfits and the associated IPPCA model underfits, a better model might lie in the SPCA family.
Second, the multidimensional eigenspaces obtained by gathering eigenvectors associated with distinct sample eigenvalues could provide robust, invariant and interpretable feature subspaces~\citep{hyvarinen_emergence_2000}. Indeed, just like the first eigenvectors can be interpreted as modes of variation~\citep{castro_principal_1986}, the eigenspaces inferred from SPCA could be interpreted as multidimensional attributes, and the norms of projection onto them as their level of expressiveness.
Third, SPCA brings a statistical framework to the flag-based multiscale modeling of datasets. Indeed, several works use flags to represent datasets, be it in an independent component analysis~\citep{nishimori_riemannian_2006} or principal component analysis~\citep{ma_flag_2021} context, enriching the already well developed literature on Grassmannians and Stiefel manifolds for dimension reduction~\citep{edelman_geometry_1998}.
In this paper, by introducing a generative model whose maximum likelihood estimate coincides with the minimiser of the \emph{accumulated unexplained variance} criterion~\citep{pennec_barycentric_2018}, we enrich the previous works and enable for instance to perform flag-type selection.
Fourth, beyond statistical modelling, SPCA provides a low-dimensional approximation of any symmetric matrix. Applications could therefore be investigated in spectral clustering~\citep{ng_spectral_2001} and shape analysis~\citep{lefevre_perturbation_2023}, where repeated eigenvalues in the graph Laplacian are prone to occur, as well as in variational Bayesian methods, where parsimonious Gaussian models can be used to approximate posterior distributions of parameters~\citep{lambert_limited-memory_2023}.


% \begin{acks}
% \end{acks}

\bibliographystyle{ACM-Reference-Format}
\bibliography{bibliography}


\end{document}