\documentclass[aps,twocolumn,showpacs,preprintnumbers,amsmath,amssymb]{revtex4-1}
\usepackage{latexsym}
\usepackage{graphicx}% Include figure files
\usepackage{dcolumn}% Align table columns on decimal point
\usepackage{bm}% bold math
%\usepackage[notref,notcite]{showkeys}% Only for work in progress, displays labels
\usepackage{xcolor}
\usepackage{empheq}
\usepackage{float}
\usepackage{enumitem}
\usepackage{amssymb,amsmath}
\usepackage{slashed}
\usepackage{hyperref}

%\usepackage[encapsulated]{CJK}%
\usepackage{CJK} %For KOREAN FONT



%\bibliographystyle{naturemag}
%\bibliographystyle{ieeetr}





\newcommand{\half}{{{\textstyle\frac{1}{2}}}}
\newcommand{\quarter}{{{\textstyle\frac{1}{4}}}}
\newcommand{\be}{\begin{equation}}
\newcommand{\ee}{\end{equation} }
\newcommand{\beqa}{\begin{eqnarray} }
\newcommand{\eeqa}{\end{eqnarray} }
\newcommand{\ba}{\begin{array}}
\newcommand{\ea}{\end{array}}


\newcommand{\su}{\mathbf{su}}
\newcommand{\so}{\mathbf{so}}
\newcommand{\SO}{\mathbf{SO}}
\newcommand{\Spin}{\mathbf{Spin}}
\newcommand{\GL}{\mathbf{GL}}
\newcommand{\osp}{\mathbf{osp}}
\newcommand{\Det}{{{\rm{Det\,}}}}


\newcommand{\rma}{{\rm a}}
\newcommand{\rmb}{{\rm b}}
\newcommand{\rmc}{{\rm c}}
\newcommand{\rmd}{{\rm d}}
\newcommand{\rme}{{\rm e}}
\newcommand{\rmf}{{\rm f}}


\newcommand{\mba}{{\mathbf{a}}}
\newcommand{\mbb}{{\mathbf{b}}}
\newcommand{\mbc}{{\mathbf{c}}}
\newcommand{\mbd}{{\mathbf{d}}}


\newcommand{\odd}{\mathbf{o}(D,D)}
\newcommand{\ODD}{\mathbf{O}(D,D)}
\newcommand{\soDD}{\mathbf{so}(D,D)}
\newcommand{\ssoDD}{{\scriptscriptstyle{\mathbf{so}(D,D)}}}
\newcommand{\etaodd}{{\cJ}}
\newcommand{\SOD}{{\SO(1,D{-1})}}
\newcommand{\oSOD}{{{\SO}(D{-1},1)}}
\newcommand{\SpinD}{{\Spin(1,D{-1})}}
\newcommand{\oSpinD}{{{\Spin}(D{-1},1)}}
\newcommand{\Spint}{{\Spin(1,9)}}

\newcommand{\Ott}{\mathbf{O}(10,10)}
\newcommand{\sott}{\mathbf{so}(10,10)}
\newcommand{\ssott}{{\scriptscriptstyle{\mathbf{so}(10,10)}}}
\newcommand{\SOt}{{\SO(1,9)}}
\newcommand{\oSOt}{{{\SO}(9,1)}}
\newcommand{\sot}{{\so(1,9)}}
\newcommand{\osot}{{{\so}(9,1)}}


\newcommand{\mbg}{\mathbf{g}}

\newcommand{\YM}{\rm{YM}}

\newcommand\alphap{{\alpha^{\prime}}}

\newcommand\tr{{\rm tr}}
\newcommand\Tr{{\rm Tr}}
\newcommand\const{{\nu}}
\newcommand\rd{{\rm d}}
\newcommand\rx{{\rm x}}
\newcommand\bfx{{\bf x}}


\newcommand\SDFT{{\rm SDFT}}

\newcommand\cA{{\cal A}}
\newcommand\cB{{\cal B}}
\newcommand\cC{{\cal C}}
\newcommand\cD{{\cal D}}
\newcommand\cE{{\cal E}}
\newcommand\cF{{\cal F}}
\newcommand\cG{{\cal G}}
\newcommand\cH{{\cal H}}
\newcommand\cI{{\cal I}}
\newcommand\cJ{{\cal J}}
\newcommand\cK{{\cal K}}
\newcommand\cL{{\cal L}}
\newcommand\cM{{\cal M}}
\newcommand\cN{{\cal N}}
\newcommand\cO{{\cal O}}
\newcommand\cP{{\cal P}}
\newcommand\cQ{{\cal Q}}
\newcommand\cR{{\cal R}}
\newcommand\cS{{\cal S}}
\newcommand\cT{{\cal T}}
\newcommand\cV{{\cal V}}



\newcommand\bcP{{\bar{\cP}}}

\newcommand\hcF{{\hat{\cal F}}}
\newcommand\hcL{{\hat{\cal L}}}
\newcommand\hGamma{{\hat{\Gamma}}}

\newcommand\wT{{\widetilde{T}}}
\newcommand\nD{{D^{\prime}\!}}
\newcommand\nDGamma{{\Gamma^{\prime}\!}}


\newcommand\rbC{{\scriptscriptstyle{\mathbf{C}}}}

\newcommand\vl{{\vec{l}}}
\newcommand\vm{{\vec{m}}}
\newcommand\vn{{\vec{n}}}

\newcommand\halpha{{\hat{\alpha}}}
\newcommand\hbeta{{\hat{\beta}}}
\newcommand\hgamma{{\hat{\gamma}}}
\newcommand\hdelta{{\hat{\delta}}}
\newcommand\hd{p}
\newcommand\hD{\hat{D}}
\newcommand\hR{{\hat{R}}}
\newcommand\gammap{{\gamma^{\prime}{}}}
\newcommand\Dp{D^{\prime}}
\newcommand\Dpp{D^{\prime\prime}}
\newcommand\cDp{\cD^{\prime}}
\newcommand\cDpp{\cD^{\prime\prime}}
\newcommand\Vp{{V^{\prime}}{}}
\newcommand\brVp{{\brV^{\prime}}{}}
\newcommand\rhop{{\rho^{\prime}}{}}
\newcommand\psip{{\psi^{\prime}}{}}
\newcommand\brrhop{{\brrho^{\prime}}{}}
\newcommand\brpsip{{\brpsi^{\prime}}{}}
\newcommand\Gammap{{\Gamma^{\prime}}{}}
\newcommand\Gammapp{{\Gamma^{\prime\prime}}{}}
\newcommand\Phippp{{\Phi^{\prime\prime\prime}}{}}
\newcommand\brPhip{{\brPhi^{\prime}}{}}
\def\brgammap{{\brgamma^{\prime}}{}}
\newcommand\Deltap{{\Delta^{\prime}}{}}
\newcommand\Deltapp{{\Delta^{\prime\prime}}{}}
\newcommand\Deltappp{{\Delta^{\prime\prime\prime}}{}}



%
\newcommand\cDs{\cD^{\star}}
%
%%\newcommand\cDh{\cD^{\rho}}
%%\newcommand\cDhh{\cD^{\psi}}
%
\newcommand\cDh{\cD^{\sharp}}
\newcommand\cDhh{\cD^{\flat}}
\newcommand\Gammah{\Gamma^{\sharp}}
\newcommand\Gammahh{\Gamma^{\flat}}
\newcommand\cDss{\cD^{\star\star}}
\newcommand\cDsss{\cD^{\star\star\star}}
\def\Gammas{\Gamma^{\star}}
\def\Gammass{\Gamma^{\star\star}}
\def\Gammasss{\Gamma^{\star\star\star}}
\def\Deltas{\Delta^{\star}}
\def\Deltass{\Delta^{\star\star}}
\def\brDeltas{\brDelta^{\star}}
\def\brDeltass{\brDelta^{\star\star}}
\newcommand\Phis{\Phi^{\star}}
\newcommand\Phiss{\Phi^{\star\star}}
\def\Deltavare{\Delta^{\varepsilon}}



\newcommand\cDvare{\cD^{\varepsilon}}
\newcommand\cDpsi{\cD^{\psi}}
\newcommand\cDrho{\cD^{\rho}}
\def\Deltapsi{\Delta^{\psi}}
\def\Deltarho{\Delta^{\rho}}
\newcommand\Phipsi{\Phi^{\psi}}
\newcommand\Phirho{\Phi^{\rho}}
\newcommand\Phivare{\Phi^{\varepsilon}}
\def\Gammao{\Gamma^{\scriptscriptstyle{0}}}
\def\Phio{\Phi^{\scriptscriptstyle{0}}}
\def\DOo{\DO^{\scriptscriptstyle{0}}}
\def\brPhio{\brPhi^{\scriptscriptstyle{0}}}
\def\szero{{\scriptscriptstyle{0}}}
\def\cDo{\cD^{\scriptscriptstyle{0}}}
\def\hcD{\hat{\cD}}
\def\hPhi{\hat{\Phi}}
\def\hDelta{\hat{\Delta}}

\newcommand\dis{\displaystyle}

\newcommand\ai{\scriptscriptstyle{A}}
\newcommand\bi{\scriptscriptstyle{B}}
\newcommand\ci{\scriptscriptstyle{C}}
\newcommand\di{\scriptscriptstyle{D}}
\newcommand\ei{\scriptscriptstyle{E}}
\newcommand\ffi{\scriptscriptstyle{F}}
\newcommand\gi{\scriptscriptstyle{G}}
\newcommand\hi{\scriptscriptstyle{H}}
\newcommand\ii{\scriptscriptstyle{I}}

\def\tA{\tilde{A}}
\def\tB{\tilde{B}}
\def\tC{\tilde{C}}
\def\tD{\widetilde{D}}
\def\tE{\tilde{E}}
\def\tF{\tilde{F}}
\def\tK{\tilde{K}}
\def\tR{{\tilde{R}}}
\def\tx{\tilde{x}}


\def\pbre{{\bar{e}^{\prime}{}}}
\def\pre{{e^{\prime}{}}}
\def\bra{\bar{a}}
\def\brb{\bar{b}}
\def\bre{\bar{e}}
\def\brvare{\bar{\varepsilon}}
\def\breta{\bar{\eta}}
\def\bralpha{\bar{\alpha}}
\def\brbeta{\bar{\beta}}
\def\brgamma{\bar{\gamma}}
\def\brdelta{\bar{\delta}}
\def\brrho{\bar{\rho}}
\def\brpsi{\bar{\psi}}
\def\brk{{\bar{k}}}
\def\brl{{\bar{l}}}
\def\brm{{\bar{m}}}
\def\brn{{\bar{n}}}
\def\brp{{\bar{p}}}
\def\brq{{\bar{q}}}
\def\brr{{\bar{r}}}
\def\brs{{\bar{s}}}
\def\bromega{{\bar{\omega}}}
\def\brOmega{{\bar{\Omega}}}
\def\brPhi{{{\bar{\Phi}}}}
\def\brDelta{{{\bar{\Delta}}}}
\def\brA{\bar{A}} 
\def\brB{\bar{B}}
\def\brC{\bar{C}}
\def\brF{\bar{F}}
\def\brH{\bar{H}}
\def\brL{\bar{L}}
\def\brR{\bar{R}}
\def\brS{\bar{S}}
\def\brU{{\bar{U}}}
\def\brV{{\bar{V}}}
\def\brP{{\bar{P}}}
\def\brT{{\bar{T}}}
\def\brLambda{{\bar{\Lambda}}}


\def\brFp{{\brF^{\prime}{}}}
\def\Fp{{F^{\prime}{}}}
%%%
%%\def\Tw{{T_{\omega}}}
%%%
\def\Tw{{T}}

\def\tPhi{{\widetilde{\Phi}}}
\def\tbrPhi{{\widetilde{\brPhi}}}

\def\hPhi{{\hat{\Phi}}}
\def\hbrPhi{{\hat{\brPhi}}}

\newcommand{\G}{\mathbf{G}}
\newcommand{\DO}{\mathbf{\nabla}}
\newcommand{\doD}{\hat{D}}
\newcommand{\na}{{\nabla}}
\newcommand{\trd}{{\bigtriangledown}}

\def\i{\mathrm{i}}

\def\a{\alpha}

%\newcommand\Rs{{\mathfrak{R}_{\rm{s}}}}
%\newcommand\gs{{\mathfrak{g}_{\rm{s}}}}

\def\rpartial{{\stackrel{\leftarrow}{\partial}}}
\def\lpartial{{\stackrel{\rightarrow}{\partial}}}



%%%%%%%  Greek letters %%%%%%%%%%%%%%%%%%

\def\s{\sigma}
\def\g{\gamma}
\def\t{\tau}
\def\a{\alpha}
\def\b{\beta}
\def\d{\delta}
\def\k{\kappa}
\def\eps{\epsilon}
\def\la{{\lambda}}
\def\l{{\lambda}}
\def\o{{\omega}}
\def\w{{\omega}}
\def\O{{\Omega}}
\def\G{\Gamma}


%%%%%%%%%%%%Notation for this paper%
\def\R{{L}}
\def\sc{{\phi}}
\def\l{\lambda}
\def\ve{\varepsilon}
\newcommand{\re}{{\rm e}}
\def\bar{\overline}


\newcommand\Imtak[1]{\textcolor{purple}{(IJ: #1)}}
\newcommand\alfredo[1]{\textcolor{blue}{(AGL: #1)}}
\newcommand\augniva[1]{\textcolor{red}{(AR: #1)}}
\newcommand\augnivaalt[1]{\textcolor{violet}{(AR: #1)}}
\newcommand{\nn}{\nonumber}
\renewcommand{\=}{\;  = \;}
\def\ads2{AdS$_2$}


% Some colors
\newcommand{\red}[1]{{\color{red} #1 \color{black}}}
\newcommand{\green}[1]{{\color{green} #1 \color{black}}}
\newcommand{\blue}[1]{{\color{blue} #1 \color{black}}}

% For comments use
\newcommand{\KL}[1]{\green{\fbox{\bf Kanghoon: #1}}} % for comments please use \KL{comment}
\newcommand{\JHP}[1]{\blue{\fbox{\bf JH: #1}}} % for comments please use \JHP{comment}


%%%%%%%%%%%%%%%%%%%%%%     Line Spacing   %%%%%%%%%%%%%%%%%%%%%%%
%%%%%
%%%\renewcommand{\baselinestretch}{1.24}   % 1.5 spacing btwn text lines

\setlength{\jot}{5pt}                 % spacing btwn the rows of an eqnarray
\renewcommand{\arraystretch}{1.6}       % spacing btwn the rows of a non-eqn array
%%%%%
%%$\!{}^{{\scriptscriptstyle{\!\textunderscore}}}$
%%%%
\begin{document}
\begin{CJK}{UTF8}{mj}
%%%
%%\begin{CJK*}{KS}{mj}
%%\CJKfamily{mj}
%%%
%\preprint{CERN-PH-TH/2011-278}
\title{String amplitudes and mutual information in confining backgrounds: \\ the partonic behavior}
\author{Mahdis Ghodrati} 
\email{mahdis.ghodrati@apctp.org}
\affiliation{%{~}\\
Asia Pacific Center for Theoretical Physics, Pohang University of Science and Technology, Pohang 37673, Korea }





%\date{\today}
\begin{abstract}
We present the connections between the behaviors of string scattering amplitudes and mutual information, in several holographic confining backgrounds, demonstrating further the ability of these quantum information measures in capturing the QCD phenomena. We show analogies between the logarithmic branch cut behavior of the string scattering amplitude in $4d$, $ \mathcal{A}_4 $,  at low and medium Mandelstam variable $s$ observed in \cite{Bianchi:2021sug}, which is due to the dependence of the string tension on the holographic coordinate, and the branch cut behavior observed in mutual information and critical distance $D_c$ observed at low-cut-off variable $u_{KK}$ observed in our previous work \cite{Ghodrati:2021ozc}. It can also be seen that in both cases, as $s$ or $u_{KK}$ increases, the peaks in the branch cuts fade away in the form of $\text{Re}\lbrack \mathcal{A}_4 \rbrack \propto s^{-1}$.  This observation can further establish the ER$=$EPR conjecture and the general interdependence between the scattering amplitudes and entanglement entropy.

\end{abstract}
\maketitle
\end{CJK}


\section{Introduction}


In the SLAC laboratory, it has been observed that at high energy and fixed angle, the hadronic scattering amplitudes fall off based on power low behavior in the Mandelstam variable $s$ (the center of mass energy) and not based on exponential behavior in $s$. This observation is incompatible with the results coming from string theory.  On the other hand, using holography, an interesting connection between entanglement entropy (EE) and the string scattering amplitude has been noticed in \cite{Seki:2014pca, Hubeny:2014zna, Semenoff:2011ng,Hubeny:2014zna,Hubeny:2014kma}. According to these connections, \cite{Seki:2014pca, Hubeny:2014zna,Semenoff:2011ng,Hubeny:2014zna,Hubeny:2014kma,Polchinski:2002jw,Andreev:2004sy,Hatta:2007he,Pire:2008zf,BallonBayona:2007qr,Brower:2000rp,Brunner:2015oqa}, we can now show that the richer mixed correlation measures such as negativity and mutual information (MI) can further establish connections between the behaviors of scattering amplitudes and the pattern of quantum entanglement.  Specifically, we noticed that the shapes of the singularities in the behavior of scattering amplitudes, observed in \cite{Bianchi:2021sug}, that form logarithmic branch cuts (rather than poles), and are more pronounced in small $s$, and fade away in the larger $s$, are indeed correct and physical, and can be explained by the connections with entanglement entropy and mutual information, as we have observed numerically the same structure in our previous works on MI in confining geometries \cite{Ghodrati:2021ozc, Ghodrati:2022kuk}. 


In \cite{Bianchi:2021sug}, using holographic confining backgrounds, the partonic behavior of scattering processes at large and medium ranges of Mandelstam variable $s$ have been explored, where the branch cut singularities at smaller values of $s$ have been observed, whereas the peaks decline at large values $s$. This work was based on the previous works of Polchinski and Strassler \cite{Polchinski:2001tt}, whose main result was writing the $4d$ amplitude in terms of the $10d$ scattering amplitude as
\begin{gather}\label{eq:A4rel1}
\mathcal{A}_4 (s,t,u) \propto \int dr \sqrt{-g} \times \mathcal{A}_{10} \lbrack \tilde{s}(r), \tilde{t}(r), \tilde{u}(r) \rbrack,
\end{gather}
where here $\mathcal{A}_{10}$ is the $10d$ string amplitude, $r$ is the holographic radial coordinate, and $\Psi (r)$ is the wave function of the scattered state. In the holographic QCD models with a non-trivial dilaton field such as Maldacena-Nunez \cite{Maldacena:2000mw} or even Witten-QCD \cite{Witten:1997ep,Seiberg:1994aj}, where the dilaton field depends on the radial coordinate, the above relation should instead be written as
\begin{gather}\label{eq:A4rel2}
\mathcal{A}_4 (s,t,u) = \int_{u_{KK} }^\infty  du \sqrt{-g} e^{-2 \phi}  \Big( \prod_{i=1}^{4}  \psi_i(u)  \Big)  \mathcal{A}_{10} \lbrack \tilde{s}(r), \tilde{t}(r), \tilde{u}(r) \rbrack.
\end{gather}


Then, there, the amplitudes has been expanded around its poles and it has been shown that the singularities, instead of being the actual poles, would be in the form of branch points or branch cuts with finite imaginary parts (which is the sign of bound state creation) and therefore, the authors of \cite{Bianchi:2021sug}, suggested that the prescription of Polchinski and Strassler \cite{Polchinski:2001tt} would fail in this limit. Then, in these scenarios, one might think that other tools such as new holographic mixed quantum information and correlation measures, such as mutual information, negativity or entanglement of purification could be new tools to probe these scenarios and give new information.

In order to reinforce such implementations, we propose a strong connection between the behavior of entanglement entropy and quantum correlation measures among two mixed subsystems at various energy and scales of the setup, and the behaviors of string scattering amplitudes in the holographic QCD backgrounds, where the quantized behavior and the creation of bound states could be observed in both cases. This connection is specifically found using these three works \cite{Ghodrati:2021ozc, Bianchi:2021sug, Hubeny:2014zna}, and based on the numerical studies performed in confining geometries.


The main point is that the spectrum of QCD, unlike theories such as $\mathcal{N}=4$ SYM, is discrete, which is the result of the IR hard wall and the boundary condition that it imposes.  This discrete spectrum then can be caught by the quantum information correlation measures. Also, measures similar to the case of amplitude, as found in \cite{Bianchi:2021sug}, could catch the transition from soft to hard scattering, or the ``bending" trajectories in confining models.  Moreover, entanglement entropy and mutual information can detect the effects of asymptotic freedom where the gauge couplings and the binding energies become smaller. At lower energies they could also detect the ``quantized" behavior of such binding energies.

It is worths to notice that most of the closed strings reside near the wall which causes the wave function to be peaked there at the wall, instead of vanishing there. This fact then could show an effect on the entanglement patterns and mixed correlations saddles, specially, this would be more pronounced at lower energies. 

Therefore, due to the connections between the partonic nature of hadron scattering and holography \cite{Bianchi:2021sug, Kang:2004jd, Pire:2008zf}, one would expect that the entanglement patterns and these scattering structures would have strong connections among them as well, which we can now show here.

\section{Critical distance in confining backgrounds}

First, turning to the EE side, in \cite{Kol:2014nqa,Brandhuber:1998er}, it has been shown that for a general confining background with the geometry
\begin{gather}
ds^2= \alpha(u) \lbrack \beta(u) du^2 + dx^\mu dx_\mu \rbrack + g_{ij} d \theta^i d\theta^j, \nonumber\\
 u_{KK} < u < \infty, \ \ x^\mu ( \mu =0,1, . . . ,d), \ \  \theta^i ( i=d+2, . . . 9),
\end{gather}
the length of a strip and its entanglement entropy, in terms of the minimum of the holographic radial coordinate, $u_0$, can be written as
\begin{gather}
L(u_0)= 2 \int_{u_0}^\infty du \sqrt{\frac{\beta(u) }{ \frac{H(u) }{H(u_0) } -1} }, \nonumber\\
S(u_0) = \frac{V_0}{ 2 G_N} \int_{u_0}^\infty du \sqrt{ \frac{\beta(u) H(u) }{1- \frac{H(u_0) }{H(u)} } }, 
\end{gather}
where $H(u)= e^{-4 \phi} V^2_{\text{int}} \alpha(u)^d$ and $V_{\text{int} } = \int d \vec{\theta} \sqrt{\text{det} \lbrack g_{ij}\rbrack}$.
Then, for two strips of width $L$ and distance $D$ between them, the mutual information would be
\begin{gather}
I(D,L)= 2 S(L)-S(D) - S(2L+D),
\end{gather}
and the critical distance $D_c$ is where $I(D_c,L)=0$.  This quantity, $D_c$, has been implemented in our previous works \cite{Ghodrati:2022hbb, Ghodrati:2022kuk,Ghodrati:2020vzm, Ghodrati:2021ozc,Ghodrati:2019hnn} to probe the quantum correlation properties of various geometries.


In \cite{Ghodrati:2021ozc}, we noticed that, in the background of confining geometries, the critical distance between two strips, which is related to mutual information and in fact is a measure of mixed correlations, shows singularities at smaller values of $u_{KK}$ and statistically smooths out in the bigger values of $u_{KK}$. This behavior is especially pronounced in the Witten-QCD geometry, figure 19 of \cite{Ghodrati:2021ozc}, which is shown again in figure \ref{fig:wedgewittenQCD} here. The same singularity jumps can also be seen in the background of Klebanov-Tseytlin geometry \cite{Klebanov:2000nc, Klebanov:2000hb}, shown in figure \ref{fig:wedgeKT} here.

 % Figure environment removed


Similar to the results of \cite{Bianchi:2021sug}, it could be seen that the peaks of $D_c$ which correspond to the peaks of mutual information and mixed correlation measures are stronger and more noticeable in the medium (or rather smaller) regions of $u_{KK}$ which corresponds to small values of Mandelstam variable $s$. For bigger $s$ (and bigger $t$ where the angle is fixed), corresponding to bigger values of $u_{KK}$, the peaks would fade away. This is due to the effects of asymptotic freedom where the gauge couplings and the binding energies of the bound states are small. However, from figures \ref{fig:wedgewittenQCD} and \ref{fig:wedgeKT}, it could be seen again that there is no peak at larger $u_{KK}$, which is due to the fact that the binding energy, length and mass of the strings become smaller in that regimes too.



 % Figure environment removed


 % Figure environment removed

The connection between critical distance $D_c$ and $u_{KK}$ in the background of Sakai-Sugimoto \cite{Sakai:2004cn} is shown in figure \ref{fig:SakaiDc}. As this figure has been created from the real part of entanglement entropy $S(u_0)$ versus the real part of the size of the strips $L(u_0)$, one would expect that, similarly  in the ``real" part of the amplitude, a power low decay behavior should be observed at bigger values of $s$ or $u$, as indeed one can see such behavior from figure \ref{fig:SakaiDc}. Note that, the behavior of the amplitude is in fact $\text{Re}\lbrack \mathcal{A}_4 \rbrack \sim s^{-1}$ and $\text{Im} \lbrack \mathcal{A}_4 \rbrack \sim s^{2- \Delta/2}$, where $\Delta$ is the sum of the scaling dimensions of the operators that contribute as $\prod_{i=1} \psi_i(r) \sim r^{- \Delta} $, ($\Delta$ can be replaced by the twist in QCD models).

\section{String scattering amplitude in confining geometries}

Now turning to the string amplitude side, the $10d$ amplitude in \cite{Bianchi:2021sug} has been found to be
\begin{gather}
\mathcal{A}_{10} = 4^4 \sum_{n=0}^\infty (-1)^n \frac{(n+1)^2}{(n!)^2} \frac{(1+c_+^4 + c_-^4)}{\alpha' s/4 - (n+1)} \times \nonumber\\
\frac{\Gamma \left( (n+1) c_+\right)  \Gamma \left( (n+1) c_- \right) }{\Gamma \left (1-(n+1) c_+\right) \Gamma \left(1-(n+1)c_-\right)  } \equiv \sum_{n=0}^\infty \frac{\mathcal{R} (\theta)}{\alpha' s/4-(n+1)}.
\end{gather}


Higher $\alpha' s$ corresponds to higher energies and bigger $u_{KK}$.

In terms of $s$ and the angle $\theta$, this amplitude can be written as
\begin{gather}
\mathcal{A}_{10}=\frac{1}{32} \alpha ^4 s^4 (\cos (2 \theta )+7)^2 \frac{\Gamma \left(-\frac{1}{4} (s \alpha )\right)}{\Gamma \left(\frac{s \alpha }{4}+1\right)}\times \\
\frac{\Gamma \left(-\frac{1}{8} s \alpha  (\cos (\theta )-1)\right)}{\Gamma \left(\frac{1}{8} (-s \alpha +s \text{$\alpha $p} \cos (\theta )+8)\right)} \times \frac{\Gamma \left(\frac{1}{8} s \alpha  (\cos (\theta )+1)\right)}{\Gamma \left(1-\frac{1}{8} s \alpha  (\cos (\theta )+1)\right)}.
\end{gather}


Its behavior versus the Mandelstam variable $s$ in confining holographic backgrounds such as Witten-QCD has been shown in \cite{Bianchi:2021sug}. The specific confining backgrounds considered there include the hard wall model, soft wall model and Witten's QCD model, showing the same singularity patterns as the EE and MI as shown in figure \ref{fig:Aalpha22} here. 
Then, using the relations \ref{eq:A4rel1} and \ref{eq:A4rel2}, $\mathcal{A}_{10}$ can be integrated and the scattering amplitude in $4d$ can be derived.  The results for different values of $\theta$ are shown in figure \ref{fig:SakaiDc2}, as for our main figure, which should be compared with those of $D_c$ for our various confining models.


The reason for the peculiar behavior in generating singularities in the form of branch cuts at low energies, is the dependence of the string tension on the holographic radial coordinate.  In their work, one could see that in $10d$ or $11d$ geometries, the amplitude drops faster compared to lower $d$ geometries, which is similar to the smoothing out of the singularities in the mixed-correlation measures in higher dimensions.


 % Figure environment removed




 % Figure environment removed




The connections between $s$ and $\theta$ have also been further investigated in figure \ref{fig:stheta}. Note that, there, $0 <\theta < 9\pi$, where 10 branch cuts can be observed. In general, with the upper bound of $n \pi$, $n+1$ branch cuts would be present.  Also, it can be seen that for bigger $s$, the branch cuts become dispersed, which is also the case for mixed correlation quantum information measures such as mutual information and $D_c$.

 % Figure environment removed



One could see that in the simple case of fixed $t$, $\Delta S_E \propto (\log s)^2 $, as shown in figure \ref{fig:pureEE}. For the mixed correlation measures and in the presence of a hard wall which creates the ``four" main saddles \cite{Dong:2021oad,Ghodrati:2022hbb},  the behaviors becomes more interesting and rich as shown in figures \ref{fig:wedgewittenQCD} and \ref{fig:wedgeKT}.   So as the parameter $D_c$ which comes from the mutual information is related to the sum of several entanglement entropies, therefore, figure \ref{fig:pureEE} is comparable to the case of the confining and mixed setup of figures \ref{fig:wedgewittenQCD} and \ref{fig:wedgeKT}. 

 % Figure environment removed





It is worthwhile to mention here that in \cite{Bianchi:2021sug}, the fixed angle limit approximation where $s \to \infty$ while $s/t$ is fixed, has also been studied, where the behavior of the real part of the $A_{10}$ is shown in figure \ref{fig:fixedtheta}. Note that, this limit is not physical as there is in fact imaginary contributions and therefore this case is not related to the quantum mixed correlation measure which we have seen in the full physical $A_{10}$ case without this approximation.


The case of the Regge limit, where $s \to \infty$ while $t$ is fixed, gives figures \ref{fig:ReggeLimit}, where again the full phase diagram cannot be observed in this setup and taking the full scenario is needed. This case then can only detect two main saddles.


 % Figure environment removed


 % Figure environment removed



One important point worths to emphasize again here is that the scattering process will always be dominated by the interior regions which is close to the IR wall at $u_{KK}$, or $u_{\Lambda}$, and therefore this parameter should indeed be the main factor in determining the saddle. The mixed quantum correlations would also be more dominant close to that wall leading to the result that it can probe the phase transitions rather effectively.



Another point is that in \cite{Hubeny:2014zna},  the disc amplitude of the scattering of two heavy quark-anti-quark pairs which are accelerating in the opposite direction, has been studied by employing the perturbation theory,  and then using that the entanglement entropy of the quark and anti-quark has been found approximately as $S= \sqrt{\lambda}$, where $\lambda$ is the 't Hooft coupling, indicating again the connections between these quantities.


Also, in reference \cite{Seki:2014pca}, it has been shown that as any interaction between particles would change the entanglement entropy (EE), therefore the change of EE should be related to the scattering amplitude. This is also the case for all other mixed correlation measures such as mutual information and negativity. So the scattering amplitude should also be able to track the four saddles of mixed correlations \cite{Dong:2021oad,Ghodrati:2022hbb}, as we have noticed. 
 The entanglement entropy can also be written in terms of a Wilson loop $\langle W \rangle$ as $S_E= (1- c \lambda \partial_\lambda ) \log \langle W \rangle$, and thus at large 'Hooft couplings $\lambda$, the change of EE at leading order in $\lambda$ is related to the gluon scattering amplitude as
 \begin{gather}
 \Delta S_E \sim \frac{(1-\frac{1}{2} c ) \sqrt{\lambda} }{8\pi} \left ( \log \frac{s}{t} \right )^2.
 \end{gather}
 Accordingly, the same relations should be present between the mixed correlation measures such as MI, negativity, entanglement and complexity of purification, and the string scattering amplitudes, as we have traced out here.


Another interesting observation is to note the links between the change of EE, or other correlation measures, and the Bremsstralung of radiative corrections as $\lambda \partial_\lambda \log \langle W \rangle $ in the case of accelerating quark-antiquarks pair, which is proportional to the Bremsstralung function. So the spikes that we noticed during the phase transitions in the plot of $D_c$ could be considered as the K-lines in Bremsstralung radiation.



Bear in mind that our observation for the connections is only for closed strings (glueballs). In the case of open strings, which describe mesons or baryons, there would be two kinds of entanglement entropy, which makes finding the connections between mutual information and scattering amplitude more difficult. The EE there includes shares from the entanglement of strings endpoints in the gluons (or flavor branes) and the entanglement between the gluons themselves, making the framework more complicated.\\




\section{Conclusion}

We showed connections between the logarithmic branch cut singularities, at low Mandelstam variable $s$, of the open string scattering amplitude and the singular behaviors observed in the mutual information and critical distance $D_c$, in holographic confining backgrounds. These singularities actually come from the dependence of the string tension on the holographic radial coordinate and are a sign of the creation of bound states at low energies.

The power low decay of both quantities matches also at larger energies, $s$ or $u_{KK}$, where the real part of both follows a $s^{-1}$ curve. This observation can further establish the interconnections announced previously between the patterns of entanglement entropy and the scattering amplitude, and also can further strengthen the ER$=$EPR conjecture. In addition, it proposes that the string scattering amplitude can serve in detecting the phase transitions.


Hence, one could expect that in confining backgrounds, there would be other connections between the phase diagrams of mixed quantum correlation measures such as mutual information or critical distance $D_c$, negativity, entanglement and complexity of purification, and the open string scattering amplitude $\mathcal{A}$, where specifying the details for each case is the aim of our future works. 



\section*{acknowledgments}

We would like to thank Dorin Weissman for sharing his codes and useful discussions. I also thank Stephen Angus and Jesus Cruz Rojas for reading the manuscrip and their comments. This work has been supported by an appointment to the JRG Program at the APCTP through the Science and Technology Promotion Fund and Lottery Fund of the Korean Government. It has also been supported by the Korean Local Governments - Gyeongsangbuk-do Province and Pohang City - and by the National Research Foundation of Korea (NRF) funded by the Korean government (MSIT) (grant numbers 2021R1F1A1048531 and 2021R1A2C1010834).



 \medskip

\bibliography{amplSon.bib}
\end{document}
