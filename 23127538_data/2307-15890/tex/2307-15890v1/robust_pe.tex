\documentclass[9pt]{article} % For LaTeX2e
\usepackage{hyperref}
%\documentclass[journal, onecolumn, draftcls]{IEEEtran}
\usepackage{url}
\usepackage{smile}
\usepackage{graphicx} % more modern
\usepackage{algorithm}
\usepackage{algorithmic}
\usepackage{epstopdf}
\usepackage[margin=1.0in]{geometry}
\usepackage[export]{adjustbox}% http://ctan.org/pkg/adjustbox
\usepackage{mathtools, cuted}
%\usepackage{natbib}
\usepackage{bbm}
\usepackage{wrapfig}
\usepackage{subcaption}
\usepackage{caption}
\usepackage{enumitem}
\usepackage{tabularx}
\usepackage{smile}
\usepackage{tcolorbox}
\usepackage{enumitem}
\usepackage{mathtools, cuted}
\usepackage{caption}
\usepackage{subcaption}

% equation number according to section
\numberwithin{equation}{section}


%\usepackage{mathpazo}
%\usepackage{times}
%\usepackage{kpfonts}

%no indent
%\setlength\parindent{0pt}

%%% line number
%\usepackage{lineno}
%\linenumbers


% sort citation 
\usepackage[nocompress]{cite}


\hypersetup{
    colorlinks=true,
    linkcolor=blue,
    anchorcolor=blue, 
    citecolor=blue,
}

% allow linebreak in display 
\allowdisplaybreaks


% for revision 
\newcommand{\yan}[1]{{\color{blue}[#1]}}


% redine frac into tfrac 
\let\oldfrac\frac
\renewcommand{\frac}[2]{\tfrac{#1}{#2}}



\title{
First-order Policy Optimization for  Robust Policy Evaluation
    }
\author{
    Yan Li   \thanks{H. Milton Stewart School of Industrial and Systems Engineering, Georgia Institute of Technology, Atlanta, GA, 30332. (E-mail: \url{yli939@gatech.edu}).}
        \and 
    Guanghui Lan \thanks{H. Milton Stewart School of Industrial and Systems Engineering, Georgia Institute of Technology, Atlanta, GA, 30332. (E-mail: \url{george.lan@isye.gatech.edu}).}
}
%\date{\today}
\date{\vspace{-6ex}}

\begin{document}
{
\makeatletter
\addtocounter{footnote}{1} % to get dagger instead of star
\renewcommand\thefootnote{\@fnsymbol\c@footnote}%
\makeatother
\maketitle
}

\maketitle



\begin{abstract}
We adopt a policy optimization viewpoint towards policy evaluation for robust Markov decision process with $\mathrm{s}$-rectangular ambiguity sets. The developed method, named first-order policy evaluation (FRPE), provides the first unified framework for robust policy evaluation in both deterministic (offline) and stochastic (online) settings, with either tabular representation or generic function approximation. In particular, we establish linear convergence in the deterministic setting, and $\tilde{\cO}(1/\epsilon^2)$ sample complexity in the stochastic setting. FRPE also extends naturally to evaluating the robust state-action value function with $(\mathrm{s}, \mathrm{a})$-rectangular ambiguity sets. We discuss the application of the developed results for stochastic policy optimization of large-scale robust MDPs. 
\end{abstract}

% Figure environment removed

\section{Introduction}
Automatic 3D reconstruction of clothed humans using image inputs has gained increasing significance due to its potential applications in a wide array of AR/VR scenarios. High-fidelity reconstructions typically depend on sophisticated capture systems, which are developed with dense camera arrays~\cite{collet2015high,joo2015panoptic,joo2018total}, programmable light-stages~\cite{Vlasic2009, guo2019relightables}, and depth sensors~\cite{newcombe2011kinectfusion,DoubleFusion,BodyFusion,dou2016fusion4d,newcombe2015dynamicfusion}. However, stringent capture environments equipped with complex hardware pose significant challenges for consumer-level applications.


In this context, considerable research effort has been dedicated to developing methods that allow for more flexible capture configurations, such as utilizing a few RGB inputs. Among these works, learning implicit functions \cite{iccv2020PIFu, saito2020pifuhd, hong2021stereopifu} has proven effective in achieving highly detailed reconstructions by integrating the advancements of deep neural networks. These methods employ large multi-layer perceptrons (MLPs) to predict the occupancy probability or truncated signed distance function (TSDF) value of every queried 3D point based on its associated local feature, which is extracted from images. They can recover a continuous surface at arbitrary resolutions without topology restrictions.


However, in typical MLP-based implicit networks, the occupancy or TSDF value at each location is solved independently with planar image features, rendering them less capable of addressing challenging cases such as occlusions. Consequently, these methods suffer from generalization and robustness issues, particularly when tackling strong occlusions caused by large motion or multiple interacting humans. 
Some follow-up studies  \cite{zheng2021deepmulticap,zheng2021pamir,huang2020arch} utilize an extra geometric model, SMPL~\cite{Loper2015}, to improve robustness by introducing strong shape priors. 
Their success typically relies on the assumption of geometrical similarity \cite{huang2020arch} between the shape prior and target reconstruction, making them intractable for handling complex cases with loose clothes and sensitive to errors in SMPL model fitting.



%\ping{this paragraph sounds like `TSDF is better than MLP/SMPL, and we use TSDF to solve the problem'. But in Sec 3, we are telling a different story, saying `MLP needs a 3D convolutional encoder'. We need to make these two sections consistent.}\sicong{I think in this paragraph we claim that the TSDF}


%We opt for Trucated Signed Distance Funtion (TSDF) volumetric representations as they are naturally suitable for convolution operations, which have shown remarkable performance for learning hierarchical features on 2D visual perception tasks \cite{SunXLW19}. 
%Meanwhile, TSDF also describes the gradual geometry change around shape surface, which is not reflected by occupancy volume. 

We instead revisit the 3D volumetric representation and resort to 3D convolutional neural networks (CNNs) for feature learning, due to their impressive performance in feature learning and the ability to incorporate spatial context. However, volumetric methods and 3D convolution involve discretization, which might raise concerns regarding whether a discretized volume can preserve subtle geometric details as continuous representations learned in implicit functions. We investigate the relationship between volume resolution and quantization error on synthetic data by converting target mesh objects to TSDF volumes, as shown in Figure~\ref{fig:quantization_error}. We observe that the quantization errors are significantly reduced by increasing volume resolution and become nearly negligible when reaching a relatively high resolution (e.g., 512 or higher). In other words, achieving fine-detailed reconstruction is not supposed to be restricted by the use of volume representations as long as a proper volume resolution is utilized. Therefore, we present a method with high-resolution feature volumes, e.g., 256 and 512, while traditional volumetric methods \cite{varol18_bodynet,gilbert2018volumetric} are often limited to much lower resolutions, such as 32 or 128.



On the other hand, an increase in volume resolution may lead to a cubic growth of memory overhead \cite{8100085}. Reducing memory costs while guaranteeing the granularity of volumetric representations is necessary for pursuing high-quality reconstruction. Thus, we adopt a coarse-to-fine approach and cull away irrelevant voxels to build a sparse high-resolution feature volume. At the coarse level, the network computes an initial TSDF by applying a U-Net with sparse 3D CNN \cite{3DSemanticSegmentationWithSubmanifoldSparseConvNet} on the sparse feature volume, which is carved by a visual hull. Through our experiments, it turns out that more than 95\% of the volume grids are discarded by the visual hull culling, making the sparse 3D CNN efficient. At the fine level, the network focuses on a narrow band near the zero-level set of the initial TSDF and discretizes the narrow band with smaller voxels. By employing this narrow-band culling, we further shrink the sampling space, resulting in a relatively small range of grid numbers (usually 300K--500K in our experiments) even with a high volume resolution of 512. The remaining voxels in the narrow band are associated with features that fuse high-frequency information from the computed normal maps upon the low-frequency shape from the coarse level to compute the TSDF at high resolution. The final mesh is then extracted from the TSDF using the Marching-Cube algorithm ~\cite{Lorensen87marchingcubes}.
% Different from the u-net sturcture to preserve global topology context, we then apply a shallow 3dcnn to compute the final TSDF $D_{final}$ which contain more local geometry detail.




% \ping{this paragraph can be expanded. It is an important contribution and often ignored by other works. stress on the novel idea of regressing blending weights instead of colors}

In addition to geometry, high-quality mesh texture is also a crucial factor contributing to visual appearance. Directly computing a color field in 3D space, as in \cite{iccv2020PIFu}, struggles to capture high-frequency texture details, while the neural radiance field (NeRF) \cite{yu2020pixelnerf} or the DoubleField~\cite{shao2022doublefield} require expensive per-instance optimization and are often unstable for sparse input images. In contrast, we adopt an image-based rendering approach to compute a texture atlas map, which is efficient and widely supported in existing computer graphics tools. 
Specifically, we compute a blending weight at each 3D point on the mesh surface to determine its color as a weighted average of the colors at its image projections. The blending weights can be computed at a relatively coarse resolution, e.g., 512 volume resolution in our case, and leave texture details to the high-resolution images, such as 1K or 2K. Unlike previous methods that generate blurry texturing results under sparse input, our method generalizes well on both synthetic and real data with just a few input views. 
Figure~\ref{fig:teaser} shows two examples reconstructed by our method. Despite the challenging garment, pose, and occlusion, our method recovers faithful shape, normal, and texture on the right.

%with a wide variety of poses and clothing styles, and it is also adaptive to handle input image with arbitrary resolutions.
%\sicong{For this concern we claim that when the resolution of dicretized volume meets certain threshold (which is 256 in our experiment), the quantization error can be neglected.} 



In summary, the main contributions of this paper are as follows:
\begin{itemize}
\vspace{-0.1in}
  \item 
  We revisit the 3D volumetric representation and demonstrate that it can support clothed human reconstruction with equal or even better performance compared to implicit representation. 
  \item 
  We develop a memory and computation-efficient method for high-resolution volumetric reconstruction using sophisticated sparse 3D CNN, coarse-to-fine estimation, and voxel culling by visual hull and narrow bands. 
  \item 
  We introduce a novel method to compute a texture atlas map, which captures rich appearance details from high-resolution input images.
  \item 
  We achieve impressive results on standard benchmark datasets Twindom and MultiHuman, significantly reducing the point-2-surface (P2S) precision to approximately 0.2cm from just six input views, with more than $50\%$ error reduction compared to the state-of-the-art methods, including DoubleField~\cite{shao2022doublefield} and PIFuHD~\cite{saito2020pifuhd}.
\end{itemize}
%!TEX root = ./robust_pe.tex
%\newpage

\section{Robust Policy Evaluation as Policy Optimization}\label{sec_pe_as_po}
This section adopts a policy optimization viewpoint towards policy evaluation, by first formulating the robust policy evaluation problem as a Markov decision problem of the nature.
We then identify a few key structures of the formulated MDP that will prove useful for our subsequent development. 

Consider a MDP of nature, denoted by $\mathfrak{M}$,  defined as follows.
The state space is given by $\cS$, and
the set of possible actions  at any state $s \in \cS$ is given by $ \cD_s \subseteq \Delta_{\cS}^{\abs{\cA}}$ (cf. \eqref{def_ambiguity_set_structure}).
%We will write $\DD = [\DD_{a_1}, \ldots, \DD_{a_{\abs{\cA}}}]$ for any $\DD \in \Delta_{\cS}^{\abs{\cA}}$.
%Equivalently, any possible action $\DD \in \cD_s$ specifies $\abs{\cA}$ elements in $\Delta_{\cS}$,  each denoted as $\DD_{a}$ for every $a \in \cA$.
At state $s \in \cS$, upon making an action $\DD \in \cD_s$, the conditional distribution of the next state $s' \in \cS$ is given by 
\begin{align}\label{transit_kernel_of_nature_mdp}
\mathfrak{P}(s' | s, \DD) \coloneqq  \tsum_{a \in \cA} \sbr{(1-\zeta) \overline{\PP}_{s,a}(s')  + \zeta \DD_{a}(s')} \vartheta(a|s).
\end{align}
%Clearly, the above transition kernel $\mathfrak{P}$ is affine with respect to the action of the nature  $\DD$. 
Finally, the cost function associated with $(s, \DD)$ for any $\DD \in \cD_s$ is given by 
\begin{align*}
\mathfrak{C}(s) \coloneqq - \tsum_{a \in \cA} \vartheta(a|s) c(s,a),
\end{align*}
 and the discount factor is set as $\gamma$.

A non-randomized policy of the nature is denoted as $\pi: \cS \to \cD_s$. 
%It is clear that $\pi$ uniquely determines $\DD^{\pi} \in \cD$ defined in \eqref{eq_cD_set}.
%Let us denote $\DD^{\pi(s)} = \pi(s)$, and 
%Let us define $\DD^{\pi(s)}  = \pi(s)$ for any policy $\pi$.
For any policy $\pi$ and $s \in \cS$, let us define $\DD^{\pi(s)} \equiv \pi(s) \in \cD_s$.
For notational clarity we will sometimes use these two quantities interchangeably. 
We then define
\begin{align}\label{kernel_defined_by_nature_policy}
\PP^{\pi}_{s,a} \coloneqq (1-\zeta) \overline{\PP}_{s,a}  + \zeta \DD^{\pi(s)}_{a}, ~ (s,a) \in \cS \times \cA.
\end{align}
%as the transition kernel of the original planner when the nature's policy is $\pi$.
Consequently, from \eqref{transit_kernel_of_nature_mdp} it holds that 
\begin{align}\label{state_transit_of_nature_given_policy}
\mathfrak{P}(s'|s,  \pi(s)) =  \tsum_{a \in \cA} \sbr{(1-\zeta) \overline{\PP}_{s,a}(s')  + \zeta \DD^{\pi(s)}_{a}(s')} \vartheta(a|s)
= \tsum_{a \in \cA} \PP^{\pi}_{s,a}(s') \vartheta(a|s).
\end{align}
We define the value function of policy $\pi$ as 
\begin{align*}
%\label{eq_def_value_func_nature}
\cV^{\pi} (s) \coloneqq 
\EE \sbr{\tsum_{t=0}^\infty \gamma^t \mathfrak{C}(s_t) \big| s_0 = s, s_{t+1} \sim \mathfrak{P}(\cdot| s_t, \pi(s_t) ) , t \geq 0}, ~~ \forall s \in \cS,
\end{align*}
and the goal of the nature is to find the optimal policy of 
\begin{align}\label{eq_def_optmal_value_nature}
\textstyle
\min_{\pi \in \Pi} \cV^{\pi} (s),
\end{align}
where $\Pi: s \mapsto \cD_s$ is the set of non-randomized stationary policies of the nature.

\begin{lemma}\label{lemma_value_correspondence}
For any $\pi \in \Pi$, we have 
\begin{align}\label{eq_nature_value_as_player_value}
\cV^{\pi}(s) = - V^{\vartheta}_{\PP^{\pi}}(s), ~ \forall s \in \cS,
\end{align}
with $V^{\vartheta}_{\PP^{\pi}}$ defined in \eqref{eq_standard_value_function}.
In addition, let $\cV^*$ denote the optimal value function of \eqref{eq_def_optmal_value_nature}.
Then 
\begin{align}\label{nature_opt_as_robust_value}
\cV^*(s) = - V^{\vartheta}_r(s), ~ \forall s \in \cS,
\end{align}
where $V^{\vartheta}_r$ is defined as in \eqref{eq_def_robust_value}.
\end{lemma}

\begin{proof}
It is clear that the $\cV^{\pi}$ satisfies the following dynamic programming equation 
\begin{align*}
\cV^{\pi} (s) & = \mathfrak{C}(s) + \gamma \tsum_{s' \in \cS} \mathfrak{P}(s'|s,  \pi(s)) \cV^{\pi}(s') \\
 & = - \tsum_{a \in \cA} \vartheta(a|s) c(s,a)
 + \gamma  \tsum_{s' \in \cS} \tsum_{a \in \cA} \PP^{\pi}_{s,a}(s') \vartheta(a|s) \cV^{\pi}(s'), ~ \forall s \in \cS.
\end{align*}
where the last equality follows from \eqref{state_transit_of_nature_given_policy}.
The above relation implies that $- \cV^{\pi}$ is the fixed point of operator 
\begin{align*}
(\cT^{\pi} V)(s) = 
\tsum_{a \in \cA} \vartheta(a|s) c(s,a)
 + \gamma  \tsum_{s' \in \cS} \tsum_{a \in \cA} \PP^{\pi}_{s,a}(s') \vartheta(a|s) V(s'), ~ \forall s \in \cS.
\end{align*}
On the other hand, it is known that $V^{\vartheta}_{\PP^{\pi}}$ is the unique fixed point of $\cT^{\pi}$, from which
we obtain \eqref{eq_nature_value_as_player_value}.
% That is, the value function of the nature's policy $\cV^{\pi}$ corresponds to the negative value function of the policy $\pi$ within $\cM_{\PP^{\pi}}$.
In addition,  Bellman optimality condition of MDP \eqref{eq_def_optmal_value_nature} yields  
\begin{align*}
\cV^*(s) & = \min_{\DD \in \cD_s} \mathfrak{C}(s)  + \gamma \tsum_{s' \in \cS} \mathfrak{P}(s' |s, \DD) \cV^*(s') \\ 
& = \min_{\DD \in \cD_s} -  \tsum_{a \in \cA} \vartheta(a|s) c(s,a) 
+ \gamma \tsum_{s' \in \cS} \tsum_{a \in \cA}\sbr{(1-\zeta) \overline{\PP}_{s,a}(s')  + \zeta \DD_{a}(s')}\vartheta(a|s) \cV^*(s') \\
& = \min_{\PP \in \cP_s} -  \tsum_{a \in \cA} \vartheta(a|s) c(s,a) 
+ \gamma \tsum_{s' \in \cS} \tsum_{a \in \cA} \PP_{a}(s') \vartheta(a|s) \cV^*(s') , ~ \forall s \in \cS.
%\\
%& =  - \tsum_{a \in \cA} \vartheta(a|s) c(s,a) 
%+ \gamma  \tsum_{a \in \cA}   \vartheta(a|s)  \tsum_{s' \in \cS} \min_{\PP_{s,a} \in \cP_{s,a}} \PP_{s,a}(s') \cV^*(s'), ~ \forall s \in \cS.
\end{align*}
Clearly, $-\cV^*$ is the fixed point of operator 
\begin{align}\label{def_robust_ballmen_op}
(\cT V)(s) = \max_{\PP \in \cP_s} \tsum_{a \in \cA} \vartheta(a|s) c(s,a) 
+ \gamma \tsum_{s' \in \cS} \tsum_{a \in \cA} \PP_{a}(s') \vartheta(a|s) V(s'), ~ \forall s \in \cS. 
\end{align}
On the other hand, it is well known that $V^{\vartheta}_{r}$ is the unique fixed point of $\cT V$ \cite{wiesemann2013robust}. 
Consequently we obtain \eqref{nature_opt_as_robust_value}.
%.
%That is, the optimal value function of the nature \eqref{eq_def_value_func_nature} corresponds to the negative robust value function $V^{\pi}_r$ of the policy,
%as both are the (unique) solution of the above dynamic programming equation.
\end{proof}


In view of the above observations, the robust policy evaluation problem \eqref{eq_def_robust_value} can be equivalently solved by solving a Markov decision process \eqref{eq_def_optmal_value_nature}  of nature with finite state space and continuous action space. 
To this end, let us define the following problem:
\begin{align}\label{policy_opt_obj_nature}
\textstyle
\min_{\pi \in \Pi} \cbr{f(\pi) \coloneqq \EE_{s \sim \rho} \sbr{\cV^{\pi}(s)}},
\end{align}
where $\rho$ is any distribution with full support over $\cS$.
Our end goal is to develop efficient methods that can be applied to solve \eqref{policy_opt_obj_nature}.


The state-action value function of the nature, also know as the Q-function, is defined by
\begin{align}\label{def_q_func_nature}
\cQ^{\pi}(s, \DD) & \coloneqq 
\EE \sbr{\tsum_{t=0}^\infty \gamma^t \mathfrak{C}(s_t) \big| s_0 = s, s_1 \sim   \mathfrak{P}(\cdot| s, \DD ), s_{t+1} \sim \mathfrak{P}(\cdot| s_t, \pi(s_t) ), t \geq 1 } . 
%\\
%& =  \mathfrak{C}(s) + \gamma \EE_{s' \sim   \mathfrak{P}(\cdot| s, \DD )} \sbr{\cV^{\pi}(s')} \\
%& = \mathfrak{C}(s) +  
% \gamma \tsum_{s' \in \cS} \tsum_{a \in \cA}\sbr{(1-\zeta) \overline{\PP}_{s,a}(s')  + \zeta \DD_{s,a}(s')}\vartheta(a|s) \cV^{\pi}(s')  \\
%& =  \mathfrak{C}(s) +  
% \gamma  \tsum_{a \in \cA} \vartheta(a|s) 
% \inner{(1-\zeta) \overline{\PP}_{s,a} + \zeta \DD_{s,a}}{\cV^{\pi}} \\
% & = 
% \mathfrak{C}(s) + \gamma \inner{(1-\zeta) \overline{\PP}_s + \zeta \DD}{\cV^{\pi}_{\vartheta, s}}, 
\end{align}
Clearly one also has 
\begin{align}\label{relation_q_and_v}
\cQ^\pi(s,\DD) = \mathfrak{C}(s) + \gamma \EE_{s' \sim \mathfrak{P}(\cdot | s,\DD)} \sbr{\cV^{\pi}(s')}.
\end{align}
%for any $ \DD \in \cD_s$, 
%where in the last equality we define $\cV^{\pi}_{\vartheta, s} = \vartheta(\cdot|s) \otimes \cV^{\pi} \in \RR^{\abs{\cA} \abs{\cS}}$.
We next show that $Q^{\pi}(s, \cdot)$ is indeed an affine function over $\cD_s$, an immediate yet important property that we will exploit in the ensuing development. 

\begin{proposition}\label{prop_q_structure}
For any $\DD \in \cD_s$, we have 
\begin{align*}
\cQ^{\pi}(s, \DD)
=  \mathfrak{C}(s) + \gamma \inner{(1-\zeta) \overline{\PP}_s + \zeta \DD}{\cV^{\pi}_{\vartheta, s}}, 
\end{align*}
where $\cV^{\pi}_{\vartheta, s} \coloneqq \vartheta(\cdot|s) \otimes \cV^{\pi} \in \RR^{ \abs{\cS} \abs{\cA}}$.
%and $\overline{\PP}_s \coloneqq [\overline{\PP}_{s, a_1}, \ldots, \overline{\PP}_{s, a_{\abs{\cA}}}] \in \Delta_{\cS}^{\abs{\cA}}$.
\end{proposition}

\begin{proof}
We have 
\begin{align*}
\cQ^{\pi}(s, \DD)
& =  \mathfrak{C}(s) + \gamma \EE_{s' \sim   \mathfrak{P}(\cdot| s, \DD )} \sbr{\cV^{\pi}(s')} \\
& = \mathfrak{C}(s) +  
 \gamma \tsum_{s' \in \cS} \tsum_{a \in \cA}\sbr{(1-\zeta) \overline{\PP}_{s,a}(s')  + \zeta \DD_{a}(s')}\vartheta(a|s) \cV^{\pi}(s')  \\
& =  \mathfrak{C}(s) +  
 \gamma  \tsum_{a \in \cA} \vartheta(a|s) 
 \inner{(1-\zeta) \overline{\PP}_{s,a} + \zeta \DD_{a}}{\cV^{\pi}} \\
 & = 
 \mathfrak{C}(s) + \gamma \inner{(1-\zeta) \overline{\PP}_s + \zeta \DD}{\cV^{\pi}_{\vartheta, s}},
\end{align*}
which completes the proof.
\end{proof}

%\yan{need to define $d_{\rho}^{\pi}$ within the perf diff lemma}
Our ensuing discussions repeatedly make use of the discounted visitation measure, defined as $d_{\rho}^{\pi}(s) = (1-\gamma) \tsum_{s' \in \cS} \rho(s') \tsum_{t=0}^\infty \gamma^t \mathtt{P}^{\pi}(s_t = s| s_0=s')$ for every $s \in \cS$, where $\mathtt{P}^{\pi}(s_t = s| s_0=s')$ denotes the probability of reaching state $s$, if running $\vartheta$ starting from $s'$ within MDP $\cM_{\PP^\pi}$.
In particular, we write $d_{s}^{\pi}$ when $\rho$ has support $\cbr{s}$. 
We next establish the difference of values for two policies of nature. 

\begin{lemma}\label{lemma_perf_diff}
For a pair of policies $\pi, \pi'$, and any $s\in \cS$,  we have
\begin{align}\label{eq_perf_diff}
\cV^{\pi'}(s) - \cV^{\pi}(s) = \frac{1}{1-\gamma}
\EE_{s' \sim d_{s}^{\pi'}} \sbr{
\cQ^{\pi}(s', \pi'(s')) - 
\cQ^{\pi}(s', \pi(s'))
}
\end{align}
Equivalently, by defining $\phi^{\pi}( s, \pi'(s)) 
%\coloneqq \cQ^{\pi}(s, \pi'(s)) - 
%\cQ^{\pi}(s, \pi(s)) 
\coloneqq \gamma \zeta \inner{\pi'(s) - \pi(s)}{\cV^{\pi}_{\vartheta, s}}$, then 
\begin{align}\label{eq_perf_diff_linearized}
\cV^{\pi'}(s) - \cV^{\pi}(s) = \frac{1}{1-\gamma}
\EE_{s' \sim d_{s}^{\pi'}} \sbr{\phi^{\pi}(s', \pi'(s'))}
\end{align}
\end{lemma}


\begin{proof}
%The proof of \eqref{eq_perf_diff} follows standard steps of performance difference lemma for finite MDPs \cite{lan2021policy, kakade2002approximately} and hence is omitted here.
%\yan{need to expand on this one}
%Let $\xi_(s)$ denote the 
Let $\xi'(s) = \cbr{s_0 = s, \pi'(s_0), s_1, \pi'(s_1), \ldots} $ denote the trajectory generated by $\pi'$ within $\mathfrak{M}$. 
That is 
\begin{align*}
s_{t+1} \sim \mathfrak{P}(\cdot|s_t, \pi'(s_t)),
\end{align*}
or equivalently, in view of \eqref{state_transit_of_nature_given_policy}, that
\begin{align}\label{state_transition_distribution_equivalence}
s_{t+1} \sim \tsum_{a \in \cA} \vartheta(a|s_t) \PP^{\pi'}_{s_t,a} (\cdot) .
\end{align}
We then obtain 
\begin{align*}
\cV^{\pi'}(s) - \cV^{\pi}(s)
& \overset{(a)}{=} \EE_{\xi'(s)} \sbr{\tsum_{t=0}^\infty \gamma^t \rbr{ \mathfrak{C}(s_t) + \gamma \cV^{\pi}(s_{t+1}) - \cV^{\pi}(s_t)}  + \cV^\pi(s_0) }   - \cV^\pi(s)  \\
& \overset{(b)}{=} \EE_{\xi'(s)} \sbr{\tsum_{t=0}^\infty \gamma^t \rbr{ \cQ^{\pi}(s_t, \pi'(s_t)) - \cV^{\pi}(s_t)}  } \\
& \overset{(c)}{=} \frac{1}{1-\gamma} \EE_{s' \sim d_s^{\pi'}} \sbr{\cQ^{\pi}(s', \pi'(s')) - \cV^\pi(s')},
\end{align*}
where $(a)$ follows from the definition of $\cV^{\pi'}(s)$, 
 $(b)$  follows from $s_0 = s$ and \eqref{relation_q_and_v},
and $(c)$ follows from \eqref{state_transition_distribution_equivalence} and the definition of $d_s^{\pi'}$.
Then \eqref{eq_perf_diff} follows  by noting that $\cV^{\pi}(s) = \cQ^\pi(s, \pi(s))$. 
Finally, \eqref{eq_perf_diff_linearized} follows from \eqref{eq_perf_diff} and Proposition \ref{prop_q_structure}.
\end{proof}

Interested readers might find the formulated MDP of nature challenging upon initial examination. 
In particular, as nature has a continuous action space, even evaluating the state-action value function \eqref{def_q_func_nature} seems to be challenging, a crucial quantity for policy improvement. 
It is also unclear whether one should and how to parameterize the policy of nature. 
In the next section, we proceed to develop the first-order robust policy evaluation (FRPE) method that exploits the structural properties established in this section and overcomes the aforementioned difficulties.























%!TEX root = ./robust_pe.tex
%\newpage

\section{Deterministic Robust Policy Evaluation}\label{sec_deterministic}
The deterministic setting we consider in this section assumes that $\overline{\PP}$ is known. 
%As a consequence,  $\PP^\pi$ defined in \eqref{kernel_defined_by_nature_policy} is available to the planner, for any fixed policy $\pi$ of nature.
We separate our discussions into two parts. 
The first part deals with the tabular setting, where the state space is relatively small, and exact computation is affordable.
We then discuss the extension to handling large state space in the second part, which involves inexact computation. 


\subsection{Tabular Setting}


The proposed method, first-order robust policy evaluation (FRPE),  assumes the access to an oracle with the following capability:
for any $\pi_k$ of the nature, it returns the value function $V^{\vartheta}_{\PP^{\pi_k}}$ of $\vartheta$ within MDP $\cM_{\PP^{\pi_k}}$. 
Equipped with this oracle, FRPE then updates the policy of the nature according to 
\begin{align*}
	\pi_{k+1}(s) = \gamma \zeta \argmin_{\DD \in \cD_s}  \tsum_{t=0}^{k}  \beta_t \inner{ \DD}{{\cV}^{\pi_t}_{\vartheta, s}} + \lambda_k w(\DD), ~ \forall s \in \cS,
\end{align*}
where $w(\cdot)$ is a strictly convex function over $\Delta_{\cS}^{\abs{\cA}}$.
%where ${\cV}^{\pi_t}_{\vartheta, s} = \vartheta(\cdot|s) \otimes {\cV}^{\pi_t}$.
For tabular setting this evaluation oracle can be easily implemented. 
Given that $\overline{\PP}$ and $\pi_k$ are both known,  one can directly compute $\PP^{\pi_k}$ defined in \eqref{kernel_defined_by_nature_policy}. 
Then evaluating $V^{\vartheta}_{\PP^{\pi_k}}$ reduces to a standard policy evaluation problem with known transition kernel $\PP^{\pi_k}$, which can be computed by either solving a linear system or fixed point iteration. 


\begin{algorithm}[t!]
  \caption{First-order Robust Policy Evaluation (FRPE)}
  \begin{algorithmic}
%    \REQUIRE Input
%    \ENSURE Output
    \STATE {\bf Input:} $\cbr{(\beta_k, \lambda_k)}$.
    \STATE {\bf Initialize:} arbitrary initial policy $\pi_0 \in \Pi$.
 \FOR{$k = 0, 1, \ldots$}
 	\STATE Evaluate ${\cV}^{\pi_k} = - {V}^{\vartheta}_{\PP^{\pi_k}}$.
	\STATE Update:
	\begin{align}\label{raw_update_deterministic_rpe}
	\pi_{k+1}(s) = \gamma \zeta \argmin_{\DD \in \cD_s}  \tsum_{t=0}^{k}  \beta_t \inner{ \DD}{{\cV}^{\pi_t}_{\vartheta, s}} + \lambda_k w(\DD), ~ \forall s \in \cS.
	\end{align}
	\STATE	 where 
	$
		{\cV}^{\pi_t}_{\vartheta, s} \coloneqq \vartheta(\cdot|s) \otimes {\cV}^{\pi_t} .
	$
    \ENDFOR
%    \RETURN $\cV^{\pi_k}$
     \end{algorithmic}
\end{algorithm}

It should be noted that to perform \eqref{raw_update_deterministic_rpe} one does not necessarily need to store the historical $\{\cV^{\pi_t}\}_{t=0}^k$.
Instead one can maintain a proper running average of these historical values.
From the definition of $\phi^{\pi}$ in Lemma \ref{lemma_perf_diff}, it is also clear that \eqref{raw_update_deterministic_rpe} is equivalent to the following update: 
\begin{align}
\pi_{k+1}(s) & = \argmin_{\DD \in \cD_s} \tsum_{t= 0}^k \beta_t \phi_t(s, \DD)  + \lambda_k w(\DD) \nonumber \\
& = \argmin_{\DD \in \cD_s}  \Phi_k(s, \DD)  + \lambda_k w(\DD) , \label{pda_nabular_deterministic_update}
\end{align}
where we have defined 
\begin{align*}
\phi_t \coloneqq \phi^{\pi_t}, ~ 
\Phi_k \coloneqq \tsum_{t=0}^k \beta_t \phi_t.
\end{align*} 
Going forward we will often make use of the  Bregman divergence associated with $w(\cdot)$, defined as 
\begin{align*}
\cB(\DD, \DD') \coloneqq w(\DD') - w(\DD) - \inner{\nabla  w(\DD)}{\DD' - \DD}.
\end{align*}
%We assume $w(\cdot)$ is a strictly convex function over $\Delta_{\cS}^{\abs{\cA}}$ and its associated Bregman divergence is defined as 
%\begin{align*}
%\cB(\DD, \DD') \coloneqq w(\DD) - w(\DD') - \inner{\nabla  w(\DD')}{\DD - \DD'}.
%\end{align*}
%Many of our ensuing discussions consider the case where the distance-generating function is defined as 
%\begin{align}\label{dgf_negative_entropy}
%w(\DD) =  \tsum_{a \in \cA} \tsum_{s' \in \cS} \DD_{a}(s') \log \DD_{a}(s').
%\end{align}
%In this case, it can be readily verify that  
%\begin{align}\label{bregman_divergence_negative_entropy}
%\cB(\DD, \DD') & =  \tsum_{a \in \cA} \tsum_{s' \in \cS} \DD_a(s') \log \rbr{ 
%\frac{\DD_a(s')}{\DD'_a(s)}
%} 
%% \nonumber \\ 
%% & = \tsum_{a \in \cA} \mathrm{KL}(\DD_{a} \Vert \DD'_{a})   \nonumber \\
%%& \geq \frac{1}{2\abs{\cA}} \norm{\DD - \DD'}_1^2, 
% \geq \frac{1}{2} \tsum_{a \in \cA} \norm{\DD_a - \DD'_a}_1^2,
%%~ \forall \DD, \DD' \in \Delta_{\cS}^{\abs{\cA}}.
%\end{align}
%where the last inequality follows from the Pinsker's inequality.
As \eqref{raw_update_deterministic_rpe} is invariant when shifting $w$ by a constant, without loss of generality we assume $\inf_{\DD \in \Delta_{\cS}^{\abs{\cA}}} w(\DD) \geq 0$.
We also require  $\overline{w} = \sup_{\DD \in \Delta_{\cS}^{\abs{\cA}}} w(\DD) < \infty$.




We start by  providing a simple characterization on each step of FRPE.


\begin{lemma}\label{lemma_pda_determinsitic_step_characterization}
Let $\Phi_{-1} \equiv 0$, $\lambda_{-1} = 0$, and $\lambda_k \geq \lambda_{k-1}$ for every $k \geq 1$.
Then for any $k \geq 0$, we have 
\begin{align}
& \Phi_k(s, \pi_{k+1}(s)) + \lambda_k \cB( \pi_{k+1}(s), \DD) \leq \Phi_k(s, \DD), ~ \forall \DD \in \cD_s, \label{pda_nhree_point} \\
& \beta_k \phi_k(s, \pi_{k+1}(s)) 
 \leq \Phi_k(s, \pi_{k+1}(s))  - \Phi_{k-1}(s, \pi_k(s)) - \lambda_{k-1} \cB(\pi_k(s), \pi_{k+1}(s))  \label{raw_progress_ineq_pda} .
\end{align}
%where $\cB(\DD, \DD') \coloneqq w(\DD) - w(\DD') - \inner{\nabla  w(\DD')}{\DD - \DD'}$ denotes the Bregman divergence between actions $\DD$ and $\DD'$.
\end{lemma}

\begin{proof}
Clearly, from the optimality condition of \eqref{pda_nabular_deterministic_update}, we obtain \eqref{pda_nhree_point}.
Given the definition of $\Phi_k$, we have 
\begin{align*}
\beta_k \phi_k(s, \DD) & = \Phi_k(s, \DD) - \Phi_{k-1}(s, \DD) - (\lambda_k - \lambda_{k-1}) w(\DD) \\
& \leq \Phi_k(s, \DD) - \Phi_{k-1}(s, \DD) \\
& \leq \Phi_k(s, \DD) - \Phi_{k-1}(s, \pi_k(s)) - \lambda_{k-1} \cB(\pi_k(s), \DD), ~ \forall k \geq 0.
\end{align*}
Further taking $\DD = \pi_{k+1}(s)$ in the above inequality yields \eqref{raw_progress_ineq_pda}.
\end{proof}

With Lemma \ref{lemma_pda_determinsitic_step_characterization} in place, we  proceed to establish the generic convergence properties of FRPE.

\begin{lemma}\label{lemma_pda_deterministic_generic}
Suppose $\lambda_k \geq \lambda_{k-1}$ for every $k \geq 1$, then 
\begin{align*}
\tsum_{t=0}^k 
\beta_t \rbr{f(\pi_{t+1}) - f(\pi^*)}
 \leq 
\tsum_{t=0}^k 
\beta_t \rbr{
1  - \frac{1-\gamma}{\norm{{d_{\rho}^{\pi^*}}/{\rho}}_\infty}
}
\rbr{f(\pi_t) - f(\pi^*)} 
+  \frac{2 \lambda_k\overline{w}}{1-\gamma}.
\end{align*}
%where $\overline{w} = \sup_{\DD \in \Delta_{\cS}^{\abs{\cA}}} w(\DD) $. 
\end{lemma}

\begin{proof}
%Clearly, from the optimality condition of \eqref{pda_nabular_deterministic_update}, we obtain
%\begin{align}
%\label{pda_nhree_point}
%\Phi_k(s, \pi_{k+1}(s)) + \lambda_k \cB(\DD, \pi_{k+1}(s)) \leq \Phi_k(s, \DD), ~ \forall \DD \in \cD_s,
%\end{align} 
%where $\cB(\DD, \DD') = w(\DD) - w(\DD') - \inner{\nabla  w(\DD')}{\DD - \DD'}$ denotes the Bregman divergence between actions $\DD$ and $\DD'$.
%Given the definition of $\Phi_k$, we have 
%\begin{align*}
%\beta_k \phi_k(s, \DD) & = \Phi_k(s, \DD) - \Phi_{k-1}(s, \DD) - (\lambda_k - \lambda_{k-1}) w(\DD) \\
%& \leq \Phi_k(s, \DD) - \Phi_{k-1}(s, \DD) \\
%& \leq \Phi_k(s, \DD) - \Phi_{k-1}(s, \pi_k(s)) - \lambda_{k-1} \cB(\DD, \pi_k(s)), ~ \forall k \geq 0,
%\end{align*}
%where we have defined $\Phi_{-1} \equiv 0$ and $\lambda_{-1} = 0$.
%Further taking $\DD = \pi_{k+1}(s)$ in the above inequality yields 
From \eqref{raw_progress_ineq_pda} in Lemma \ref{lemma_pda_determinsitic_step_characterization} we have
\begin{align}
\beta_k \phi_k(s, \pi_{k+1}(s)) 
& \leq \Phi_k(s, \pi_{k+1}(s))  - \Phi_{k-1}(s, \pi_k(s)) - \lambda_{k-1} \cB( \pi_k(s), \pi_{k+1}(s)) \nonumber \\
& \overset{(a)}{\leq} \Phi_k(s, \pi_{k}(s)) - \Phi_{k-1}(s, \pi_k(s)) - \lambda_{k-1} \cB(\pi_k(s), \pi_{k+1}(s))  - \lambda_k \cB( \pi_{k+1}(s), \pi_k(s))  \nonumber  \\
& \overset{(b)}\leq \beta_k \phi_k(s, \pi_k(s)) + (\lambda_k - \lambda_{k-1}) w(\pi_k(s)) \nonumber \\
&\overset{(c)} = (\lambda_k - \lambda_{k-1}) w(\pi_k(s)), \nonumber \\
& \leq (\lambda_k - \lambda_{k-1}) \overline{w} \label{subgrad_inner_ub}, 
\end{align}
where $(a)$ follows from applying \eqref{pda_nhree_point} again;
$(b)$ follows from the definition of $\Phi_k$;
 $(c)$ follows from the trivial identity $\phi_k(s, \pi_k(s)) = 0$ given the definition of $\phi_k$.
 Combing \eqref{subgrad_inner_ub} with Lemma \ref{lemma_perf_diff}, we have
\begin{align}
\cV^{\pi_{k+1}}(s) - \cV^{\pi_k}(s) 
& = \frac{1}{1-\gamma} \EE_{d_s^{\pi_{k+1}}} \sbr{
\phi_k(s', \pi_{k+1}(s') )- \frac{(\lambda_k - \lambda_{k-1}) \overline{w}}{\beta_k} 
} 
+  \frac{(\lambda_k - \lambda_{k-1} ) \overline{w}}{ (1-\gamma) \beta_k } 
 \nonumber \\
& \leq 
\frac{1}{1-\gamma} d_s^{\pi_{k+1}}(s) \sbr{
\phi_k(s', \pi_{k+1}(s') )- \frac{(\lambda_k - \lambda_{k-1}) \overline{w}}{\beta_k} 
}
+ \frac{(\lambda_k - \lambda_{k-1} ) \overline{w}}{ (1-\gamma) \beta_k } \nonumber  \\
& = 
\phi_k(s, \pi_{k+1}(s) ) +  \frac{ \gamma(\lambda_k - \lambda_{k-1}) \overline{w}}{(1-\gamma) \beta_k} . \label{eq_pda_approx_progress}
\end{align}
It is also clear from the first equality above and \eqref{subgrad_inner_ub} that 
\begin{align}
\cV^{\pi_{k+1}}(s) - \cV^{\pi_k}(s) \leq  \frac{ (\lambda_k - \lambda_{k-1}) \overline{w}}{(1-\gamma) \beta_k} . \label{eq_pda_approx_monotone}
\end{align}
Now taking the telescopic sum of \eqref{raw_progress_ineq_pda}, we obtain 
\begin{align*}
\tsum_{t=0}^k \beta_t \phi_t(s, \pi_{t+1}(s))
& \leq \Phi_k(s, \pi_{k+1}(s)) - \Phi_{-1}(s, \pi_0(s)) \\
%- \lambda_{-1} \cB(\pi_0(s), \pi_1(s)) \\
& \overset{(d)}{=} \Phi_k(s, \pi_{k+1}(s)) \\
& \overset{(e)}{\leq} \Phi_k(s, \DD) = \tsum_{t=0}^k \beta_t \phi_t(s, \DD) + \lambda_k w(\DD) .
\end{align*}
where $(d)$ applies the definition of $\Phi_{-1} \equiv 0$,
and $(e)$ applies \eqref{pda_nhree_point}.
Combining the above inequality with \eqref{eq_pda_approx_progress}, 
we have 
\begin{align*}
\tsum_{t=0}^k \beta_t \rbr{ \cV^{\pi_{t+1}}(s) - \cV^{\pi_t}(s) -  \frac{(\lambda_t - \lambda_{t-1} ) \overline{w}}{\beta_t (1-\gamma)}}
\leq \tsum_{t=0}^k \beta_t \phi_t(s, \DD)  +  \lambda_k \overline{w}.
\end{align*}
Substituting  $\DD = \pi^*(s)$ into the above for every $s\in \cS$,  and aggregating the resulting inequalities by weight $d_{s }^{\pi^*}$, we obtain 
\begin{align*}
& \tsum_{t=0}^k \EE_{s' \sim d_{s}^{\pi^*}} \big[   \beta_t ( \cV^{\pi_{t+1}}(s) - \cV^{\pi_t}(s) -  \frac{(\lambda_t - \lambda_{t-1} ) \overline{w}}{\beta_t (1-\gamma)}) \big] \\
 \leq &  \tsum_{t=0}^k \beta_t \EE_{s' \sim d_s^{\pi^*}}\sbr{ \phi_t(s, \pi^*(s))} +  \lambda_k \overline{w} \\
 \overset{(f)}{=} & 
(1-\gamma) \tsum_{t=0}^k \beta_t \rbr{\cV^{\pi^*}(s) - \cV^{\pi_t}(s)} + \lambda_k \overline{w},
\end{align*} 
where $(f)$ applies Lemma \ref{lemma_perf_diff}.
It remains to take expectation with respect to $s \sim \rho$ of the above inequality and obtain
%Combining the above inequality and  \eqref{eq_pda_approx_monotone}, we can further obtain 
\begin{align*}
&  \tsum_{t=0}^k  \big\lVert\frac{d_{\rho}^{\pi^*}}{\rho}\big\rVert_\infty \EE_{s \sim \rho}
\sbr{ \beta_t ( \cV^{\pi_{t+1}}(s) - \cV^{\pi_t}(s) -  \frac{(\lambda_t - \lambda_{t-1} ) \overline{w}}{\beta_t (1-\gamma)})} \\
\overset{(h)}{\leq} & 
\tsum_{t=0}^k \EE_{s \sim d_{\rho}^{\pi^*}} \sbr{  \beta_t ( \cV^{\pi_{t+1}}(s) - \cV^{\pi_t}(s) -  \frac{(\lambda_t - \lambda_{t-1} ) \overline{w}}{\beta_t (1-\gamma)})} \\
 \leq & 
 (1-\gamma) \tsum_{t=0}^k \beta_t \EE_{s \sim \rho} \sbr{\cV^{\pi^*}(s) - \cV^{\pi_t}(s)} + \lambda_k \overline{w},
\end{align*} 
where $(h)$ utilizes \eqref{eq_pda_approx_monotone}, from which the term inside expectation is non-positive.
Simple arrangement of the above relation yields 
\begin{align*}
\tsum_{t=0}^k 
\beta_t \rbr{f(\pi_{t+1}) - f(\pi^*)}
& \leq 
\tsum_{t=0}^k 
\beta_t \rbr{
1  - \frac{1-\gamma}{\norm{{d_{\rho}^{\pi^*}}/{\rho}}_\infty}
}
\rbr{f(\pi_t) - f(\pi^*)} 
+  \rbr{ \frac{1}{1-\gamma} + {\norm{{d_{\rho}^{\pi^*}}/{\rho}}_\infty}^{-1} } \lambda_k\overline{w}  \\
& \leq 
\tsum_{t=0}^k 
\beta_t \rbr{
1  - \frac{1-\gamma}{\norm{{d_{\rho}^{\pi^*}}/{\rho}}_\infty}
}
\rbr{f(\pi_t) - f(\pi^*)} 
+ \frac{ 2 \lambda_k \overline{w}}{1-\gamma}  .
\end{align*}
The proof is then completed.
\end{proof}


We are ready to establish the linear convergence of FRPE with proper specification of $\cbr{(\beta_k, \lambda_k)}$.


\begin{theorem}\label{thrm_linear_convergence_frpe}
Take $\beta_k = \rbr{
1  - \frac{1-\gamma}{\norm{{d_{\rho}^{\pi^*}}/{\rho}}_\infty}
}^{-k}$ and $\lambda_k = \lambda \zeta > 0 $,  then 
\begin{align*}
f(\pi_{k+1}) - f(\pi^*) \leq 
\rbr{
1  - \frac{1-\gamma}{\norm{{d_{\rho}^{\pi^*}}/{\rho}}_\infty}
}^{k} \sbr{ f(\pi_0) - f(\pi^*) + \frac{2 \lambda \zeta \overline{w}}{1-\gamma} }.
\end{align*}
\end{theorem}

\begin{proof}
The  claim follows from a direct application of Lemma \ref{lemma_pda_deterministic_generic}, combined with the choice of $\cbr{(\beta_k, \lambda_k)}$.
\end{proof}

%\yan{remark on comparison to standard approach using fixed point iteration, mention the failure of the latter approach to function approximation, and then motivate the discussion of linear function approximation}

In view of Theorem \ref{thrm_linear_convergence_frpe}, FRPE in the deterministic setting attains linear convergence.
In particular, the performance of applying FRPE to the offline robust MDP problem described in Example \ref{example_offline_rmdp}  matches the classical robust policy evaluation methods based on fixed point iteration \cite{nilim2005robust, iyengar2005robust}. 
Notably both methods requires computing the robust value for every state. 
In the following section we proceed to discuss the setting where approximate computation is required, and demonstrate the clear advantage of FRPE. 


%Now suppose 
%\begin{align*}
%\beta_t \rbr{
%1  - \frac{1-\gamma}{\norm{{d_{\rho}^{\pi^*}}/{\rho}}_\infty}
%}
%\leq \beta_{n-1}, ~ \beta_0 = 1,
%\end{align*}
%we then obtain 
%\begin{align*}
%\beta_k \rbr{f(\pi_{k+1}) - f(\pi^*)}
%\leq   f(\pi_0) - f(\pi^*) + 2 \lambda_k \overline{w},
%\end{align*}
%which implies 
%\begin{align*}
%f(\pi_{k+1}) - f(\pi^*) \leq 
%\rbr{
%1  - \frac{1-\gamma}{\norm{{d_{\rho}^{\pi^*}}/{\rho}}_\infty}
%}^{k} \sbr{ f(\pi_0) - f(\pi^*) + 2 \lambda \overline{w} }
%\end{align*}



%
%\yan{add the constant stepsize version, show its iteration complexity scales with $\zeta$, hence more favorable when the ambiguity set's size $\zeta$ becomes small
%
%Wait, it seems that the iteration complexity above already implicitly scales with $\zeta$, to see this, simply take $\lambda = \zeta$, and the right hand side both scales as $\cO(\zeta)$.
%}








\subsection{FRPE with Function Approximation}

We now discuss the extension of FRPE to handle large state space.
In such a setting both parameterizing the policy $\pi$ of the nature and its value function $\cV^{\pi}$ can be costly.
We start by observing the that FRPE does not require explicit parameterization  of $\pi$.
Namely, the latest policy $\pi_{t+1}(s)$ can be generated incrementally according to \eqref{raw_update_deterministic_rpe} whenever its value is needed. 
On the other hand, parameterizing $\cV^{\pi}$ can be efficiently done by employing function approximation with relatively few parameters. 
We now discuss an example of such approach, and establish the convergence of FRPE in the presence of potential approximation error.

\begin{example}
Suppose a feature map $\psi: \cS \to \RR^d$ is given, we consider the problem of approximating $\cV^{\pi}(\cdot)$ by $\psi(\cdot)^\top \theta^\pi$ so the difference of these two functions can be properly controlled. 
In view of Lemma \ref{lemma_value_correspondence}, this is equivalent to approximating $V^{\vartheta}_{\PP^\pi}(s)$ by $-\psi(\cdot)^\top \theta^\pi$, a long studied problem with numerous off-the-shelf methods.
%which by itself is simply approximating the value function of $\vartheta$ with respect to MDP $\cM_{\PP^\pi}$.
%Such a problem has been long studied in the reinforcement learning and dynamic programming literature. 
In particular, 
as $\PP^\pi$ is known,
one can consider solving the so-called projected Bellman equation
\begin{align}\label{projected_bellman_equation}
\Psi \theta^\pi = \Pi_{\Psi, \nu^{\vartheta}_{\PP^\pi}} \cT (\Psi \theta^\pi),
\end{align}
where $\Psi \in \RR^{\abs{\cS} \times d}$ denotes the feature matrix constructed by applying $\psi(\cdot)$ to every state,
$\Pi_{\Psi, \nu^{\vartheta}_{\PP^\pi}}$ denotes the projection onto $\mathrm{span}(\Psi)$ in $\norm{\cdot}_{\nu^{\vartheta}_{\PP^\pi}}$ norm,
$ \nu^{\vartheta}_{\PP^\pi}$ denotes the stationary state distribution induced by $\vartheta$ within MDP $\cM_{\PP^\pi}$,
and $\cT$ denotes the Bellman operator of $\vartheta$ within MDP $\cM_{\PP^\pi}$.
A feature to note here is that solution $\theta^\pi$ exists for \eqref{projected_bellman_equation} \cite{580874}, and the latter  being a linear system of $\theta^\pi$ implies efficient methods for solving it \cite{karczmarz1937angenaherte, li2023accelerated}.
The approximation error is captured by $\norm{\Psi \theta^\pi - \cV^\pi}_\infty$, which depends on the expressiveness of $\psi$. 
An alternative to solving \eqref{projected_bellman_equation} is to minimize the mean-squared Bellman error, which also yields a linear system and does not require information of $\nu^{\vartheta}_{\PP^{\pi}}$.
In Section \ref{sec_stoch_linear_approx} we study this approach in the stochastic~setting.
\end{example}

%At this point it is worth mentioning a few prior studies that take a direct approach towards robust policy evaluation, by extending  the projected Bellman equation \eqref{projected_bellman_equation} to the robust setting with linear function approximation \cite{badrinath2021robust, tamar2013scaling}. 
%In this case, the objective is to solve 
%\begin{align}\label{robust_projected_bellman_equation}
%\Psi \theta^* = \Pi_{\Psi,  \nu} \cT_{\mathrm{r}} (\Psi \theta^*)
%\end{align}
% to obtain estimate $\Psi \theta^* \approx V^{\vartheta}_r$. 
%Compared to \eqref{projected_bellman_equation}, the operator $\cT_{\mathrm{r}}$ corresponds to the so-called robust Bellman operator defined in \eqref{def_robust_ballmen_op}, and $\nu$ is the stationary distribution for certain exploration policy.
%The most critical limitation of such approach is that \eqref{robust_projected_bellman_equation} does not necessarily admit a solution, as the operator $\Pi_{\Psi,  \nu} \cT_{\mathrm{r}}$ is no longer a contraction. 
%Indeed it is shown in \cite{tamar2013scaling} that a fairly restrictive yet necessary condition is required to certify the existence of $\theta^*$ satisfying \eqref{robust_projected_bellman_equation}.
%This assumption also seems difficult to verify even if the model is known to the planner.
%On the other hand, if one seeks the approximate solution of \eqref{robust_projected_bellman_equation} by seeking to minimize its mean-squared error $\norm{\Psi \theta^* - \Pi_{\Psi,  \nu} \cT_{\mathrm{r}} (\Psi \theta^*)}^2_2$, the resulting objective can be easily non-convex due to the non-linearity of $\cT_{\mathrm{r}}$. 
 

In the remainder of this section we proceed to establish the convergence of FRPE with arbitrary approximation scheme, as long as the error can be controlled. 
%We now turn our attention to the convergence of FRPE where potential function approximation error exists in the evaluated $\cV^\pi$. 
Namely, instead of using exact $\cV^{\pi_t}$ in the update \eqref{raw_update_deterministic_rpe} of FRPE, one only has an approximation  $\tilde{\cV}^{\pi_t}$ such that 
\begin{align}\label{funx_approx_deterministic}
\norm{\cV^{\pi_t} - \tilde{\cV}^{\pi_t} }_\infty \leq \varepsilon_{\mathrm{approx}} 
\end{align}
for some $\varepsilon_{\mathrm{approx}}  > 0$.
%This substantially generalizes prior approaches mentioned above, by lifting any technical conditions and being applicable to broader approximation schemes.
The FRPE method in this setting takes the following update:
\begin{align}\label{raw_update_deterministic_rpe_func_approx}
\textstyle
	\pi_{k+1}(s) = \gamma \zeta \argmin_{\DD \in \cD_s}  \tsum_{t=0}^{k}  \beta_t \inner{ \DD}{\tilde{\cV}^{\pi_t}_{\vartheta, s}} + \lambda_k w(\DD), ~ \forall s \in \cS,
\end{align}
where $\tilde{\cV}^{\pi_t}_{\vartheta, s} \coloneqq \vartheta(\cdot|s) \otimes \tilde{\cV}^{\pi_t}$.
Clearly, \eqref{raw_update_deterministic_rpe_func_approx} is similar to \eqref{raw_update_deterministic_rpe} except that we replace the exact value function $\tilde{\cV}^{\pi_t}$ with its approximation $\tilde{\cV}^{\pi_t}$.
%We proceed to show that FRPE remains stable in the presence of approximation error.
It is clear that  \eqref{raw_update_deterministic_rpe_func_approx} is equivalent to:
\begin{align}
\pi_{k+1}(s) & = \argmin_{\DD \in \cD_s} \tsum_{t= 0}^k \beta_t \tilde{\phi}_t(s, \DD)  + \lambda_k w_s(\DD) \nonumber \\
& = \argmin_{\DD \in \cD_s}  \tilde{\Phi}_k(s, \DD)  + \lambda_k w_s(\DD) , \label{pda_nabular_func_approx_update}
\end{align}
where  
$
\tilde{\phi}_t(s, \DD)  = \gamma \zeta \inner{\DD - \pi_t(s)}{\tilde{\cV}^{\pi_t}_{\vartheta, s}}$ and  $\tilde{\Phi}_k  = \tsum_{t=0}^k \beta_t \tilde{\phi}_t .
$
%Let us also define 
%\begin{align}\label{def_noise_in_phi}
%\delta_n(s, \DD)  \coloneqq \tilde{\phi}_t(s, \DD)  - \phi_t (s, \DD) = \gamma \zeta \inner{\DD - \pi_t(s)}{\tilde{\cV}^{\pi}_{\vartheta, s} - {\cV}^{\pi}_{\vartheta, s}},
%\end{align}
%where the last equality follows from the definition of $\tilde{\phi}_t$ and $\phi_t$.




The following lemma follows from similar lines as in Lemma \ref{lemma_pda_determinsitic_step_characterization}.


\begin{lemma}\label{lemma_pda_determinsitic_step_characterization_func_approx}
Let $\Phi_{-1} \equiv 0$, $\lambda_{-1} = 0$, and $\lambda_k \geq \lambda_{k-1}$ for every $k \geq 1$.
Then for any $k \geq 0$, we have 
\begin{align}
& \tilde{\Phi}_k(s, \pi_{k+1}(s)) + \lambda_k \cB(\pi_{k+1}(s), \DD) \leq \tilde{\Phi}_k(s, \DD), ~ \forall \DD \in \cD_s, \label{pda_nhree_point_func_approx} \\
& \beta_k \tilde{\phi}_k(s, \pi_{k+1}(s)) 
 \leq \tilde{\Phi}_k(s, \pi_{k+1}(s))  - \tilde{\Phi}_{k-1}(s, \pi_k(s)) - \lambda_{k-1} \cB(\pi_k(s), \pi_{k+1}(s))  \label{raw_progress_ineq_pda_func_approx} .
\end{align}
%where $\cB(\DD, \DD') \coloneqq w(\DD) - w(\DD') - \inner{\nabla  w(\DD')}{\DD - \DD'}$ denotes the Bregman divergence between actions $\DD$ and $\DD'$.
\end{lemma}

We now proceed to establish the convergence properties of FRPE in the presence of approximation error. 


\begin{theorem}
Suppose \eqref{funx_approx_deterministic} holds, 
then taking $\beta_k = \rbr{
1  - \frac{1-\gamma}{\norm{{d_{\rho}^{\pi^*}}/{\rho}}_\infty}
}^{-k}$ and $\lambda_k = \lambda \zeta > 0 $ yields 
\begin{align*}
f(\pi_{k+1}) - f(\pi^*) \leq 
\rbr{
1  - \frac{1-\gamma}{\norm{{d_{\rho}^{\pi^*}}/{\rho}}_\infty}
}^{k} \sbr{ f(\pi_0) - f(\pi^*) + \frac{2 \lambda \zeta \overline{w}}{1-\gamma} }
+ \frac{2 (1+ \norm{{d_\rho^{\pi^*}}/{\rho}}_\infty) \gamma \zeta  \varepsilon_{\mathrm{approx}} }{(1-\gamma)^2}.
\end{align*}
\end{theorem}

\begin{proof}
We start by noting that following the same lines as we show \eqref{subgrad_inner_ub}, one has 
\begin{align}
\beta_k \tilde{\phi}_k(s, \pi_{k+1}(s)) 
 \leq (\lambda_k - \lambda_{k-1}) \overline{w} \label{subgrad_inner_ub_func_approx}.
\end{align}
Similar to \eqref{eq_pda_approx_progress}, we obtain 
\begin{align}
& \cV^{\pi_{k+1}} (s) - \cV^{\pi_k} (s) 
-  \frac{1}{1-\gamma} \EE_{s' \sim d_s^{\pi_{k+1}}} \sbr{
\phi_k(s', \pi_{k+1}(s')) - \tilde{\phi}_k(s', \pi_{k+1}(s')) 
} \nonumber \\
\overset{(a)}{ =} & \frac{1}{1-\gamma} \EE_{s' \sim d_s^{\pi_{k+1}}} \sbr{
\tilde{\phi}_k(s', \pi_{k+1}(s')) 
}  \nonumber \\
= & \frac{1}{1-\gamma} \EE_{s' \sim d_s^{\pi_{k+1}}} \sbr{
\tilde{\phi}_k(s', \pi_{k+1}(s')) - \frac{(\lambda_k - \lambda_{k-1})\overline{w}}{\beta_k}
}
+ \frac{(\lambda_k - \lambda_{k-1}) \overline{w}}{(1-\gamma)\beta_k}  \label{approx_progress_func_approx_raw_1} \\
\overset{(b)}{\leq} & 
\tilde{\phi}_k(s, \pi_{k+1}(s)) + \frac{(\lambda_k - \lambda_{k-1} )\overline{w}}{\beta_k (1-\gamma)}, \label{inner_product_lb_func_approx}
\end{align}
where $(a)$ follows from Lemma \ref{lemma_perf_diff}, and $(b)$ follows from \eqref{subgrad_inner_ub_func_approx}.
In addition, from the definition of $\phi_k$ and $\tilde{\phi}_k$, one has 
\begin{align}\label{func_approx_error_effect_on_q}
\abs{\phi_k(s, \DD) - \tilde{\phi}_k(s,\DD)}
& = \abs{
\tsum_{a \in \cA} \gamma \zeta \inner{\DD_a - \DD^{\pi_k(s)}_a}{ \cV^{\pi_k} - \hat{\cV}^{\pi_k} } \vartheta(a|s)
}  \nonumber \\
& \leq 2 \gamma \zeta \norm{\cV^{\pi_k} - \hat{\cV}^{\pi_k}}_\infty \leq 2  \gamma \zeta \varepsilon_{\mathrm{approx}}.
\end{align}
where the last inequality follows from H\"{o}lder's inequality.
Combining \eqref{subgrad_inner_ub_func_approx},  \eqref{approx_progress_func_approx_raw_1},  and \eqref{func_approx_error_effect_on_q} also yields 
\begin{align}\label{approx_progress_func_approx_raw_3}
 \cV^{\pi_{k+1}} (s) - \cV^{\pi_k} (s)  \leq \frac{(\lambda_k - \lambda_{k-1}) \overline{w}}{(1-\gamma) \beta_k} + \frac{2 \gamma \zeta \varepsilon_{\mathrm{approx}}}{1-\gamma}.
\end{align}
Repeating the same lines after \eqref{eq_pda_approx_monotone} in the proof of Lemma \ref{lemma_pda_deterministic_generic},
with \eqref{raw_progress_ineq_pda_func_approx}, and  \eqref{inner_product_lb_func_approx}, \eqref{approx_progress_func_approx_raw_3} replaced by \eqref{raw_progress_ineq_pda}, 
\eqref{eq_pda_approx_progress}, and \eqref{eq_pda_approx_monotone},
we obtain 
\begin{align*}
\tsum_{t=0}^k 
\beta_t \rbr{f(\pi_{t+1}) - f(\pi^*)}
& \leq 
\tsum_{t=0}^k 
\beta_t \rbr{
1  - \frac{1-\gamma}{\norm{{d_{\rho}^{\pi^*}}/{\rho}}_\infty}
}
\rbr{f(\pi_t) - f(\pi^*)} 
+  \frac{2 \lambda_k\overline{w}}{1-\gamma} \\
& ~~~~~~
+ \tsum_{t=0}^k \beta_t \frac{2 \gamma \zeta  \varepsilon_{\mathrm{approx}}}{1-\gamma} \rbr{1 + \norm{\frac{d_\rho^{\pi^*}}{\rho}}_\infty^{-1}}.
\end{align*}
Plugging the choice of $\cbr{(\beta_k, \lambda_k)}$ yields the desired result.
%\begin{align*}
%f(\pi_{k+1}) - f(\pi^*) \leq 
%\rbr{
%1  - \frac{1-\gamma}{\norm{{d_{\rho}^{\pi^*}}/{\rho}}_\infty}
%}^{k} \sbr{ f(\pi_0) - f(\pi^*) + \frac{2 \lambda \zeta \overline{w}}{1-\gamma} }
%+ \frac{2 (1+ \norm{{d_\rho^{\pi^*}}/{\rho}}_\infty) \gamma \zeta  \varepsilon_{\mathrm{approx}} }{(1-\gamma)^2}.
%\end{align*}
\end{proof}



%\yan{comparison to Mannor's prior work after technical discussion}
%An immediate observation is that FRPE does not necessarily require parameterizing the policy. 
%Instead, the updated policy $\pi_{t+1}(s)$ at any given state $s$ can be computed whenever it is need within the evaluation procedure. 




At this point it is worth mentioning a few prior studies that take a direct approach towards robust policy evaluation with linear function approximation, by extending the projected Bellman equation \eqref{projected_bellman_equation} to the robust setting \cite{roy2017reinforcement, tamar2014scaling}.
In this case, the objective is to solve 
\begin{align}\label{robust_projected_bellman_equation}
\Psi \theta^* = \Pi_{\Psi,  \nu} \cT_{\mathrm{r}} (\Psi \theta^*)
\end{align}
 to obtain estimate $\Psi \theta^* \approx V^{\vartheta}_r$. 
Compared to \eqref{projected_bellman_equation}, the operator $\cT_{\mathrm{r}}$ corresponds to the so-called robust Bellman operator defined in \eqref{def_robust_ballmen_op}, and $\nu$ is the stationary distribution for certain exploration policy.
An important limitation of such approach is that \eqref{robust_projected_bellman_equation} does not necessarily admit a solution, as the operator $\Pi_{\Psi,  \nu} \cT_{\mathrm{r}}$ is no longer a contraction. 
Indeed it is shown in \cite{tamar2014scaling} that an restrictive yet necessary condition is required to certify the existence of $\theta^*$ satisfying \eqref{robust_projected_bellman_equation}.
This assumption also seems difficult to verify even if the model is known to the planner, and only asymptotic convergence has been established depsite its restrictive nature.


On the other hand, if one seeks the approximate solution of \eqref{robust_projected_bellman_equation} by minimizing its mean-squared error $\norm{\Psi \theta^* - \Pi_{\Psi,  \nu} \cT_{\mathrm{r}} (\Psi \theta^*)}^2_2$, the resulting objective can be easily non-convex due to the non-linearity of $\cT_{\mathrm{r}}$. 
 It should be clear that FRPE substantially improves the aforementioned approaches, by removing unrealistic assumptions while being applicable to broader approximation schemes.


It should also be noted that for large-scale offline robust MDPs, both the model $\overline{\PP}$ and ambiguity set $\cD$ can be difficult to store. 
Consequently, in solving the linear system defined by \eqref{projected_bellman_equation} or the mean-squared Bellman error, one might proceed with an incremental manner.   
This can be done, for instance, by Kaczmarz method \cite{karczmarz1937angenaherte} and its randomized variants \cite{strohmer2009randomized, gower2015randomized}.
One can also utilize the stochastic variant of FRPE to be discussed in the next section.



%\yan{should mention that only **asymptotic results** are obtained with this approach!!!}
%
%\yan{should we mention that stochasticity is also allowed even when $\overline{\PP}$ is known }
















%!TEX root = ./robust_pe.tex

\section{Stochastic Robust Policy Evaluation}\label{sec_stochastic}

Compared to the deterministic setting, in the stochastic setting we do not have exact information on $\overline{\PP}$. Instead only transition samples with distribution governed by $\overline{\PP}$ are available.
As in Section \ref{sec_deterministic}, we initiate our discussion with tabular setting.
%, where the size of the state space is relatively small compared to available computational resource. 
Then we proceed to discuss the convergence of the proposed method with linear function approximation to handle large state spaces.\footnote{
The method to be developed can indeed be combined with general function approximation schemes, provided certain off-policy evaluation problem can be solved to a properly pre-specified accuracy. We defer related discussions to Section~\ref{sec_conclusion}. 
}
In both cases we present sample complexities that appears to be completely new for stochastic robust policy evaluation.








We start by introducing the framework of stochastic first-order robust policy evaluation (SFRPE).
Compared to Section \ref{sec_deterministic}, 
the distance generating function $w_s(\cdot)$ in the stochastic setting can potentially depend on $s$ through $\vartheta(\cdot|s)$. 
We assume in addition that the Bregman divergence induced by $w_s(\cdot)$ satisfies 
\begin{align}\label{sc_in_group_norm}
\cB_s(\DD, \DD') \coloneqq w_s(\DD') - w_s(\DD) - \inner{\nabla  w_s(\DD)}{\DD' - \DD}
 \geq  \frac{\mu_w}{2} \tsum_{a \in \cA} \vartheta(a|s) \norm{\DD_a - \DD'_a}_1^2
\end{align}
for some $\mu_w > 0$.
Many of our ensuing discussions consider the case where   
\begin{align}\label{dgf_negative_entropy}
w_s(\DD) =  \tsum_{a \in \cA} \vartheta(a|s) \tsum_{s' \in \cS} \DD_{a}(s') \log \DD_{a}(s') + \log \abs{\cS}.
\end{align}
In this case, it can be readily verified that $\cB_s(\DD, \DD')  =  \tsum_{a \in \cA}\vartheta(a|s) \tsum_{s' \in \cS}  \DD'_a(s') \log \rbr{ 
{\DD'_a(s')}/{\DD_a(s')}
}$,
and 
\begin{align}
0  \leq   w_s(\DD)  & \leq \log \abs{\cS}, \label{bregman_divergence_negative_entropy_lb_ub} \\
\cB_s(\DD, \DD')  
%=  \tsum_{a \in \cA}\vartheta(a|s) \tsum_{s' \in \cS}  \DD_a(s') \log \rbr{ 
%\frac{\DD_a(s')}{\DD'_a(s')}
%} 
&  \geq \frac{1}{2} \tsum_{a \in \cA}  \vartheta(a|s) \norm{\DD_a - \DD'_a}_1^2, \label{bregman_divergence_negative_entropy_sc}
%~ \forall \DD, \DD' \in \Delta_{\cS}^{\abs{\cA}}.
\end{align}
where \eqref{bregman_divergence_negative_entropy_sc} follows from the Pinsker's inequality.
Similar to the deterministic setting, we require $\overline{w} = \sup_{s \in \cS} \sup_{\DD \in \Delta_{\cS}^{\abs{\cA}}} w_s(\DD) < \infty$.
In view of \eqref{bregman_divergence_negative_entropy_lb_ub} and \eqref{bregman_divergence_negative_entropy_sc}, for $w_s(\cdot)$ defined in \eqref{dgf_negative_entropy} one can take $\overline{w} =\log \abs{\cA}$ and $\mu_w = 1$.




\begin{algorithm}[t]
  \caption{Stochastic First-order Robust Policy Evaluation (SFRPE)}
  \begin{algorithmic}
%    \REQUIRE Input
%    \ENSURE Output
    \STATE {\bf Input:} $\cbr{(\beta_k, \lambda_k)}$.
    \STATE {\bf Initialize:} arbitrary initial policy $\pi_0 \in \Pi$.
    \FOR{$k = 0, 1, \ldots$}
 	\STATE  Run stochastic evaluation operator to obtain $\hat{\cV}^{\pi_k} $.
%	form $\hat{\cV}^{\pi_n}_{\vartheta, s}$ for every $s \in \cS$, where 
%	\begin{align*}
%	\hat{\cV}^{\pi_n}_{\vartheta, s} \coloneqq \vartheta(\cdot|s) \otimes \hat{\cV}^{\pi_n} .
%	\end{align*}
	\STATE  Update:
	\begin{align}\label{raw_update_stoch_rpe}
	\textstyle
	\pi_{k+1}(s) = \gamma \zeta \argmin_{\DD \in \cD_s}  \tsum_{t=0}^{k}  \beta_t \inner{ \DD}{\hat{\cV}^{\pi_t}_{\vartheta, s}} + \lambda_k w_s(\DD), ~ \forall s \in \cS,
	\end{align}
	where 
	$
		\hat{\cV}^{\pi_t}_{\vartheta, s} \coloneqq \vartheta(\cdot|s) \otimes \hat{\cV}^{\pi_t} .
	$
    \ENDFOR
    \RETURN $\tsum_{t=1}^k \theta_t \hat{\cV}^{\pi_t}$, where 
    \begin{align}\label{def_theta}
	\theta_t = \beta_t / (\tsum_{t=1}^k \beta_t) .
	\end{align}
     \end{algorithmic}
\end{algorithm}


\begin{remark}
%It turns out that divergence \eqref{dgf_negative_entropy} enjoys at least two apparent advantages. 
%In purely online robust MDP problems (cf. Example \ref{online_robust_mdp}) where $\cD_s = \Delta_{\cS}^{\abs{\cA}}$, it can readily seen that update of SFRPE \eqref{raw_update_stoch_rpe} has closed-form solution
%$
%\DD^{\pi_{n+1}(s)}_a \propto \exp (
%- \gamma \zeta \tsum_{t=0}^n \hat{\cV}^{\pi_t} / \lambda_n
%)
%$
%for any $a \in \cA$.
The weighted construction of divergence \eqref{dgf_negative_entropy}  appears to be essential. In particular, setting equal weights in \eqref{dgf_negative_entropy} would result in a sample complexity that linearly depends on the action space $\abs{\cA}$ despite we are evaluating the robust value function.
%
%\yan{Need to mention on the motivation of state-dependent divergence -- remove the dependence on the action space}
%\yan{need to mention how the update can be solved efficiently for large state space, or has explicit solution}
%\yan{both are related to the choice of divergence}
\end{remark}

It is clear that update \eqref{raw_update_stoch_rpe} is equivalent to the following update
\begin{align}
\pi_{k+1}(s) & = \argmin_{\DD \in \cD_s} \tsum_{t= 0}^k \beta_t \hat{\phi}_t(s, \DD)  + \lambda_k w_s(\DD) \nonumber \\
& = \argmin_{\DD \in \cD_s}  \hat{\Phi}_k(s, \DD)  + \lambda_k w_s(\DD) , \label{pda_tabular_stoch_update}
\end{align}
where  
$\hat{\phi}_t(s, \DD)  = \gamma \zeta \inner{\DD - \pi_t(s)}{\hat{\cV}^{\pi_t}_{\vartheta, s}}$ and $ \hat{\Phi}_k  = \tsum_{t=0}^k \beta_t \hat{\phi}_t$.
Let us also define 
\begin{align}\label{def_noise_in_phi}
\delta_t(s, \DD)  \coloneqq \hat{\phi}_t(s, \DD)  - \phi_t (s, \DD) = \gamma \zeta \inner{\DD - \pi_t(s)}{\hat{\cV}^{\pi_t}_{\vartheta, s} - {\cV}^{\pi_t}_{\vartheta, s}},
\end{align}
where the last equality follows from the definition of $\hat{\phi}_t$ and $\phi_t$.

The following lemma follows the exact same argument as in Lemma \ref{lemma_pda_determinsitic_step_characterization}.

\begin{lemma}\label{lemma_pda_stochastic_step_characterization}
Define $\hat{\Phi}_{-1} \equiv 0$ and $\lambda_{-1} = 0$, and let $\lambda_k \geq \lambda_{k-1}$ for every $k \geq 1$.
Then for any $k \geq 0$, we have 
\begin{align}
& \hat{\Phi}_k(s, \pi_{k+1}(s)) + \lambda_k \cB_s(\pi_{k+1}(s), \DD) \leq \hat{\Phi}_k(s, \DD), ~ \forall \DD \in \cD_s, \label{pda_nhree_point_stoch} \\
& \beta_k \hat{\phi}_k(s, \pi_{k+1}(s)) 
 \leq \hat{\Phi}_k(s, \pi_{k+1}(s))  - \hat{\Phi}_{k-1}(s, \pi_k(s)) - \lambda_{k-1} \cB_s( \pi_k(s), \pi_{k+1}(s))  \label{raw_progress_ineq_pda_stoch} .
\end{align}
\end{lemma}

We next establish some generic convergence properties of SFRPE. 


\begin{lemma}\label{lemma_generic_prop_stoch}
Let  $\lambda_t \geq \lambda_{t-1}$ for every $n \geq 1$,  $\beta_0 = 0$.
%For any $k \geq 1$, define 
%\begin{align}\label{def_theta}
%\theta_t = \beta_t / (\tsum_{t=1}^k \beta_t) .
%\end{align}
Then
%Suppose $\EE_{|k} \sbr{\norm{\hat{\cV}^{\pi_k}}_\infty^2} \leq M$ for any $k \geq 0$,  then for any $s \in \cS$, 
\begin{align}
 \tsum_{t=1}^k \theta_t \rbr{\hat{\cV}^{\pi_t}(s) - \cV^{\pi_t}(s)}
&  \leq \tsum_{t=1}^k \theta_t 
\hat{\cV}^{\pi_t}(s) - \cV^*(s) \nonumber \\
& \leq \rbr{\tsum_{t=1}^k \beta_t}^{-1}  \tsum_{t=1}^k \frac{\beta_t^2 \gamma^2 \zeta^2  \norm{\hat{\cV}^{\pi_t}}_\infty^2}{2 \mu_w \lambda_{t-1} (1-\gamma)}
+  \rbr{\tsum_{t=1}^k \beta_t}^{-1} \frac{\lambda_k \overline{w}}{1-\gamma} \nonumber \\
& ~~~ + \tsum_{t=1}^k \frac{\theta_t}{1-\gamma}  \EE_{s' \sim d_{s}^{\pi^*}} \sbr{ \delta_t(s', \pi^*(s'))} 
+ \tsum_{t=1}^k \theta_t \rbr{\hat{\cV}^{\pi_t}(s) - \cV^{\pi_t}(s)}. \label{eq_generic_prop_stoch}
\end{align}
\end{lemma}

\begin{proof}
%Let us define $\DD^{\pi(s)}  = \pi(s)$ for any policy $\pi$.
%For notational clarity we will sometimes use these two quantities interchangeably. 
Taking the telescopic sum of \eqref{raw_progress_ineq_pda_stoch}  yields
% \yan{change the annotation to the following, and modify the ensuing results on dependence of $\mu_w$ accordingly}
\begin{align*}
\hat{\Phi}_0 (s, \pi_1(s)) & \leq 
\hat{\Phi}_k(s, \pi_{k+1}(s)) 
- \tsum_{t=1}^k \lambda_{t-1} \cB_s( \pi_t(s), \pi_{t+1}(s)) 
- \tsum_{t=1}^k \beta_t \hat{\phi}_t (s, \pi_{t+1}(s)) \\
& \overset{(a)}{\leq} 
 \hat{\Phi}_k(s, \pi_{k+1}(s)) 
-   \tsum_{t=1}^k \tsum_{a \in \cA} \frac{\lambda_{t-1} \mu_w \vartheta(a|s)}{2} \norm{\DD^{\pi_{t+1}(s)}_a - \DD^{\pi_t(s)}_a }_1^2  \\
& ~~~ - \tsum_{t=1}^k \beta_t  \gamma \zeta \inner{\pi_{t+1}(s) - \pi_t(s)}{\hat{\cV}^{t}_{\vartheta, s}} \\
& \overset{(b)}{=} 
 \hat{\Phi}_k(s, \pi_{k+1}(s)) 
-  \tsum_{t=1}^k \tsum_{a \in \cA}  \frac{\lambda_{t-1} \mu_w \vartheta(a|s)}{2} \norm{\DD^{\pi_{t+1}(s)}_a - \DD^{\pi_t(s)}_a }_1^2  \\
& ~~~  -  \tsum_{t=1}^k \beta_t  \tsum_{a \in \cA}  \gamma \zeta  \vartheta(a|s) \inner{\DD^{\pi_{t+1}(s)}_a - \DD^{\pi_t(s)}_a}{\hat{\cV}^{\pi_t}} \\
& \overset{(c)}{\leq} 
 \hat{\Phi}_k(s, \pi_{k+1}(s)) 
-  \tsum_{t=1}^k \tsum_{a \in \cA}  \frac{\lambda_{t-1} \mu_w \vartheta(a|s)}{2}  \norm{\DD^{\pi_{t+1}(s)}_a - \DD^{\pi_t(s)}_a }_1^2  \\
& ~~~  -   \tsum_{t=1}^k \tsum_{a \in \cA} \beta_t  \gamma \zeta  \vartheta(a|s)   \norm{\hat{\cV}^{\pi_t}}_\infty \norm{\DD^{\pi_{t+1}(s)}_a - \DD^{\pi_t(s)}_a}_1 \\
& \overset{(d)}{\leq} 
 \hat{\Phi}_k(s, \pi_{k+1}(s)) 
+   \tsum_{t=1}^k \tsum_{a \in \cA} \frac{\beta_t^2 \gamma^2 \zeta^2  \norm{\hat{\cV}^{\pi_t}}_\infty^2 \vartheta(a|s)}{2 \mu_w \lambda_{t-1}} \\
& =
 \hat{\Phi}_k(s, \pi_{k+1}(s)) 
+   \tsum_{t=1}^k \frac{\beta_t^2 \gamma^2 \zeta^2  \norm{\hat{\cV}^{\pi_t}}_\infty^2 }{2 \mu_w \lambda_{t-1}} \\
&\overset{(e)}{\leq} \hat{\Phi}_k(s, \DD)  + \tsum_{t=1}^k \frac{\beta_t^2 \gamma^2 \zeta^2  \norm{\hat{\cV}^{\pi_t}}_\infty^2}{2 \mu_w \lambda_{t-1}}, ~ \forall \DD \in \cD_s,
\end{align*}
where $(a)$ follows from the definition of $\hat{\phi}_t$ and \eqref{sc_in_group_norm};
$(b)$ follows from the definition of $\hat{\cV}^t_{\vartheta, s}$;
$(c)$ follows from H\"{o}lder's inequality and the definition of $M$;
$(d)$ follows from Young's inequality; 
and $(e)$ applies \eqref{pda_nhree_point_stoch}.
Since $\beta_0 = 0$ and $w_s(\cdot) \geq 0$, the above inequality implies
\begin{align*}
0 \leq \tsum_{t=1}^k \beta_t {\phi}_t(s, \DD) +  \tsum_{t=1}^k \frac{\beta_t^2 \gamma^2 \zeta^2  \norm{\hat{\cV}^{\pi_t}}_\infty^2}{2 \mu_w \lambda_{t-1}} + \lambda_k w_s(\DD)
+ \tsum_{t=1}^k \beta_t  \delta_t(s,\DD),
\end{align*}
%where $\delta_t \coloneqq \hat{\phi}_n - \phi_t$.
Now consider  aggregating the above inequalities by weights $d_{s}^{\pi^*}$ after taking $\DD=\pi^*(s)$ therein.
From Lemma \ref{lemma_perf_diff} we then obtain  
\begin{align*}
0 \leq \tsum_{t=1}^k \beta_t (1-\gamma) \rbr{\cV^*(s) - \cV^{\pi_t}(s)} 
+  \tsum_{t=1}^k \frac{\beta_t^2 \gamma^2 \zeta^2  \norm{\hat{\cV}^{\pi_t}}_\infty^2}{2 \mu_w \lambda_{t-1}} + \lambda_k \overline{w}
+ \tsum_{t=1}^k \beta_t \EE_{s' \sim d_{s}^{\pi^*}} \sbr{ \delta_t(s', \pi^*(s'))},
\end{align*}
Simple rearrangement of the above inequality yields the desired claim.
%\begin{align*}
%\tsum_{s \in \cS} \rho(s) \tsum_{t=1}^k \theta_t \sbr{
%\cV^{\pi_t}(s) - \cV^*(s)
% }
%& \leq \rbr{\tsum_{t=1}^k \beta_t}^{-1}  \tsum_{t=1}^k \frac{\beta_t^2 \gamma^2 \zeta^2 M^2}{2 \mu_w \lambda_{t-1}}
%+  \rbr{\tsum_{t=1}^k \beta_t}^{-1} \lambda_k \overline{w} \\
%& ~~~ +  \rbr{\tsum_{t=1}^k \beta_t}^{-1}  \rbr{ \tsum_{t=1}^k \beta_t \EE_{s \sim d_{\rho}^{\pi^*}} \sbr{ \delta_t(s, \pi^*(s))} }.
%\end{align*}
\end{proof}


We make the following terminology for accuracy certificate.

\begin{definition}[$\epsilon$-estimator]\label{def_acc_certificate}
For any $\epsilon \geq 0$, we say that a randomized quantity $\hat{\cV}$ is an $\epsilon$-estimator of the robust value function $\cV^*$, defined in \eqref{nature_opt_as_robust_value}, in expectation (resp. in high probability) if 
$-\epsilon \leq \EE \sbr{\hat{\cV}(s)} - \cV^*(s) \leq \epsilon$ (resp. $-\epsilon \leq  {\hat{\cV}}(s) - \cV^*(s) \leq \epsilon$ with high probability) for every $s \in \cS$.
\end{definition}

With Lemma \ref{lemma_generic_prop_stoch} in place, we proceed to establish the convergence of  SFRPE in expectation. 


\begin{proposition}\label{thrm_stoch_generic_convergence_expectation}
Fix $\lambda > 0$ and set  
\begin{align}\label{param_choice_stoch_general_expectation}
\beta_k = k^{1/2},   ~ \lambda_k = (k+1)\lambda, ~\forall k \geq 0.
\end{align}
Suppose 
\begin{align}\label{stoch_expecation_conv_bias_condition}
 \norm{\EE_{|k}\hat{\cV}^{\pi_k} - \cV^{\pi_k}}_\infty \leq \varepsilon, ~ \EE_{|k} \sbr{\norm{\hat{\cV}^{\pi_k}}_\infty^2} \leq M, ~ \forall k \geq 1.
\end{align}
Then for any $k \geq 1$, 
\begin{align}\label{ineq_expecatation_sfrpe_opt_gap}
- \varepsilon 
\leq 
\EE \sbr{ \tsum_{t=1}^k \theta_t 
\hat{\cV}^{\pi_t}(s)} - \cV^*(s)  
& \leq  \frac{\gamma^2 \zeta^2 M^2}{\mu_w \lambda \sqrt{k} (1-\gamma)}
+  \frac{4 \lambda  \overline{w}}{\sqrt{k} (1-\gamma)}  + (\frac{2 \gamma \zeta}{1-\gamma} + 1) \varepsilon, ~ \forall s \in \cS.
\end{align}
In particular, taking $\lambda = \frac{\gamma \zeta M}{2 \sqrt{\mu_w \overline{w}}}$ yields 
%\yan{discuss on mis-specify $M$ in choosing $\lambda$ affects the convergence -- looks like linear over-estimate of $M$ leads to linear factor of the upper bound, point out to section on linear function approximation}
\begin{align*}
- \varepsilon 
\leq 
\EE \sbr{ \tsum_{t=1}^k \theta_t 
\hat{\cV}^{\pi_t}(s)} - \cV^*(s)  
& \leq  \frac{4 \gamma \zeta M \sqrt{\overline{w}}}{(1-\gamma) \sqrt{\mu_w k}}  + \frac{3 \varepsilon}{1-\gamma}, ~ \forall s \in \cS.
\end{align*}
%\begin{align*}
%- \varepsilon 
%\leq 
%\EE \sbr{ \tsum_{n=1}^k \theta_n 
%\hat{\cV}^{\pi_n}(s)} - \cV^*(s)  
%& \leq \rbr{\tsum_{n=1}^k \beta_n}^{-1}  \tsum_{n=1}^k \frac{\beta_n^2 \gamma^2 \zeta^2 M^2}{2 \mu_w \lambda_{n-1}}
%+  \rbr{\tsum_{n=1}^k \beta_n}^{-1} \lambda_k \overline{w}  + (2 \gamma \zeta + 1) \varepsilon.
%\end{align*} 
\end{proposition}

\begin{remark}
%In view of the choice of $\lambda$ in Proposition \ref{thrm_stoch_generic_convergence_expectation}, 
When $M$ is unknown, one can instead use an estimate $\hat{M}$ and accordingly choose $\lambda = \frac{\gamma \zeta \hat{M}}{2 \sqrt{\mu_w \overline{w}}}$.
The price for using such an estimate is an $\max \cbr{\hat{M}/M, M/\hat{M}}$ factor increase for the right-hand side of \eqref{ineq_expecatation_sfrpe_opt_gap}.
This would be particularly helpful for our discussion in Section \ref{sec_stoch_linear_approx}, when exact information of $M$ can be difficult to obtain.
\end{remark}


\begin{proof}
Given the definition of $\delta_t(s, \DD)$ in \eqref{def_noise_in_phi}, we obtain 
\begin{align*}
\abs{\EE\sbr{\delta_t(s, \DD)}} 
=\gamma \zeta  \abs{\EE\sbr{ \inner{\DD - \pi_t(s)}{\EE_{|t}\hat{\cV}^{\pi_t}_{\vartheta, s} - {\cV}^{\pi_t}_{\vartheta, s}} }}
\overset{(a)}{\leq} 2 \gamma \zeta \norm{\EE_{|t}\hat{\cV}^{\pi_t} - \cV^{\pi_t}}_\infty
= 2 \gamma \zeta \varepsilon,  ~ \forall s \in \cS, \DD \in \cD_s,
\end{align*}
where $(a)$ follows from a direct application of H\"{o}lder's inequality combined with the definition of $\hat{V}^{\pi}_{\vartheta,s}$ and ${V}^{\pi}_{\vartheta,s}$.
Taking expectation of \eqref{eq_generic_prop_stoch} in Lemma \ref{lemma_generic_prop_stoch} and applying the above inequality yields
\begin{align*}
- \varepsilon 
\leq 
\EE \sbr{ \tsum_{t=1}^k \theta_t 
\hat{\cV}^{\pi_t}(s)} - \cV^*(s)  
& \leq \rbr{\tsum_{t=1}^k \beta_t}^{-1}  \tsum_{t=1}^k \frac{\beta_t^2 \gamma^2 \zeta^2 M^2}{2 \mu_w \lambda_{t-1}(1-\gamma)}
+  \rbr{\tsum_{t=1}^k \beta_t}^{-1} \frac{\lambda_k \overline{w}}{1-\gamma}  + (\frac{2 \gamma \zeta}{1-\gamma} + 1) \varepsilon.
\end{align*} 
The rest of the claims then follow from direct computations after plugging \eqref{param_choice_stoch_general_expectation} into the above inequality. 
%In particular, taking
%\begin{align*}
%\beta_0 = 0, ~ \beta_n = n^{1/2}, ~ \forall n \geq 1; ~ \lambda_n = (n+1)\lambda, ~\forall n \geq 0, 
%\end{align*}
%we obtain 
%\begin{align*}
%- \varepsilon 
%\leq 
%\EE \sbr{ \tsum_{n=1}^k \theta_n 
%\hat{\cV}^{\pi_n}(s)} - \cV^*(s)  
%& \leq  \frac{\gamma^2 \zeta^2 M^2}{\mu_w \lambda \sqrt{k}}
%+  \frac{4 \lambda  \overline{w}}{\sqrt{k}}  + (2 \gamma \zeta + 1) \varepsilon.
%\end{align*}
%Taking $\lambda = \frac{\gamma \zeta M}{2 \sqrt{\mu_w}}$, and using the fact that $\gamma, \zeta \in [0,1]$ yields 
%\begin{align*}
%- \varepsilon 
%\leq 
%\EE \sbr{ \tsum_{n=1}^k \theta_n 
%\hat{\cV}^{\pi_n}(s)} - \cV^*(s)  
%& \leq  \frac{4 \gamma \zeta M}{\sqrt{\mu_w k}}  + 3 \varepsilon.
%\end{align*}
\end{proof}


%Clearly, the above convergence guarantees hinge upon the existence of a policy evaluation operator that certifies condition \eqref{stoch_expecation_conv_bias_condition}.
%We next discuss two policy evaluation operators with this capability,
%and consequently determine the sample complexities of SFRPE when instantiated with these operators. 

Up to now, our discussion is based on the existence of stochastic evaluation operators that can certify noise condition \eqref{stoch_expecation_conv_bias_condition}.
For the remainder of our discussions we proceed to construct such evaluation operators for both tabular setting and with linear function approximation. 
Consequently one can invoke Proposition \ref{thrm_stoch_generic_convergence_expectation} to establish the output of SFRPE being an $\epsilon$-estimator of the robust value function in expectation with $\cO(\abs{\cS} /\epsilon^2)$ sample complexity.
It turns out that one can also obtain a much stronger result with essentially the same number of samples.
In particular, we show  later in this section that the output of SFRPE is an $\epsilon$-estimator of the robust value function with high probability.
This improvement seems to be important for applications of SFRPE in stochastic policy optimization of robust MDPs \cite{li2022first}.


\subsection{Tabular Setting} 

%We then proceed to introduce two policy evaluation operators that can certify condition \eqref{stoch_expecation_conv_bias_condition}.
%Consequently one can invoke Proposition \ref{thrm_stoch_generic_convergence_expectation} to obtain the sample complexity of SFRPE instantiated with these evaluation operators. 
%More importantly, we will also establish direct control over  $ { \tsum_{n=1}^k \theta_n 
%\hat{\cV}^{\pi_n}(s)} - \cV^*(s)  $ with high probability as opposed to the expectation bound in Proposition \ref{thrm_stoch_generic_convergence_expectation}.


%{\bf Simulator-based Evaluation (SE).}
The first evaluation operator presented in Algorithm \ref{alg_se}, named simulator-based evaluation (SE), 
 assumes the access to a so-called simulator of MDP $\cM_{\overline{\PP}}$, and performs an $l$-step process for estimating the value function $\cV^\pi$.
 At each step, SE generates a transition pair $(s,s')$ for each state, where $s'$ denotes the random next state upon committing to an action $a \sim \vartheta(\cdot|s)$ at the state $s$. 
 This transition pairs are then used  to construct auxiliary matrix estimates and update the estimator $\hat{\cV}^\pi$ in an incremental fashion.

\begin{algorithm}[t!]
  \caption{Simulator-based Evaluation (SE)}
  \begin{algorithmic}\label{alg_se}
    \STATE {\bf Initialize:} $R_{-1} = I$ and $\hat{\cV}^{\pi}_{-1} = 0$.
    \STATE Construct $\DD^{\pi, \vartheta} \in \RR^{\abs{\cS} \times \abs{\cS}}$ as
	$
	\DD^{\pi, \vartheta}(s, s') = \tsum_{a \in \cA} \vartheta(a|s) \DD^{\pi(s)}_a (s')    , ~ \forall (s, s') \in \cS \times \cS.
    	$
%	\yan{probably replace this also by sample}
        \FOR{ $i = 0, 1, ... l-1$}
    \STATE Set $\cD = \emptyset$. For each $s \in \cS$, commit action $a \sim \vartheta(\cdot|s)$, and collect $s'$. Save $(s, s')$ into  $\cD$.
    \STATE Construct $\hat{\PP}_i \in \RR^{\abs{\cS} \times \abs{\cS}}$ such that 
    \begin{align*}
     \hat{\PP}_i (s, s') =
    \begin{cases}
     1, ~ (s,s') \in \cD; \\
     0, ~ (s, s') \notin \cD.
     \end{cases}
     \end{align*}
    \STATE Update $R_i = R_{i-1} ((1-\zeta) \hat{\PP}_i + \zeta \DD^{\pi, \vartheta})$, and
    $
    \hat{\cV}^{\pi}_i= \hat{\cV}^{\pi}_{i-1} + \gamma^i R_i \mathfrak{C}.
   $
     \ENDFOR
%    \RETURN $\tsum_{t=1}^k \theta_t \hat{\cV}^{\pi_t}$, where $\theta_t = \beta_t / (\tsum_{t=1}^k \beta_t) $.
\RETURN $\hat{\cV}^{\pi} = \hat{\cV}^{\pi}_l$.
     \end{algorithmic}
\end{algorithm}



%The SE estimator is defined as 
%\begin{align*}
%\hat{\cV}^{\pi}(s) = - \hat{V}^{\vartheta}_{\PP^{\pi}}(s), ~ \text{where} ~ \hat{V}^{\vartheta}_{\PP^{\pi}}(s) = \tsum_{t=0}^{l} \gamma^t c(S_t, A_t).
%\end{align*}
%We make the following immediate observations.

\begin{proposition}\label{prop_se_properties}
For any fixed policy $\pi$, let $\xi$ denote the sample used by the SE operator, then  
\begin{align*}
\norm{\EE_{\xi} \hat{\cV}^{\pi} - \cV^{\pi}}_\infty \leq \frac{\gamma^l}{1-\gamma},~
\norm{\hat{\cV}^{\pi}}_\infty \leq \frac{1}{1-\gamma}.
\end{align*}
\end{proposition}

\begin{proof}
It should be clear that $\EE_{|i} \sbr{ (1-\zeta) \hat{\PP}_i + \zeta \DD^{\pi, \vartheta}} = \mathtt{P}^\pi$,
where $\mathtt{P}^{\pi}$ denotes the transition matrix of the state chain $\cbr{S_t}$ induced by $\vartheta$ within $\cM_{\PP^\pi}$, with $\PP^\pi$ defined as in \eqref{kernel_defined_by_nature_policy}.
Consequently 
from the definition of $R_i$, we obtain
\begin{align*}
\EE_\xi \sbr{R_i}  =  \rbr{ \mathtt{P}^{\pi}}^i  ~ ; ~ \norm{R_i}_\infty  \leq 1, ~ \forall 0 \leq i \leq l,
\end{align*} 
where $\norm{R_i}_\infty$ denotes the matrix $\norm{\cdot}_\infty$ norm.
In addition, one also has 
\begin{align*}
\cV^\pi = (I - \gamma \mathtt{P}^\pi)^{-1}  \mathfrak{C} =  \tsum_{i=0}^\infty \gamma^i \rbr{\mathtt{P}^{\pi}}^i \mathfrak{C}.
\end{align*}
The desired claim then follows immediately from the above two observations and $\norm{\mathfrak{C}}_\infty \leq 1$.
%Consequently, it holds that 
%\begin{align*}
%\norm{\EE_\xi \sbr{\hat{\cV}^{\pi}} - \cV^\pi}_\infty \leq \frac{\gamma^l}{1-\gamma},~
%\norm{\hat{\cV}^{\pi}} \leq \frac{1}{1-\gamma}. 
%\end{align*}
%The rest
%\begin{align*}
%\abs{\EE_\xi \hat{V}^{\vartheta}_{\PP^{\pi}}(s) - V^{\vartheta}_{\PP^{\pi}}(s)} \leq \frac{\gamma^l}{1-\gamma},
%~ \norm{\hat{V}^{\vartheta}_{\PP^{\pi}}}_\infty \leq \frac{1}{1-\gamma}.
%\end{align*}
%The rest of the claims then follows from the above relation.
\end{proof}

\begin{remark}
It should be noted that one can also avoid the construction of $\DD^{\pi, \vartheta}$ via sampling. 
Namely, in addition to the sampled $(s, a, s')$, one also samples $s'' \sim \DD^{\pi(s)}_a(\cdot)$. 
Then $\hat{\DD}^{\pi, \vartheta}_i$ can be constructed in the same way as $\PP_i$ using transition pair $(s, s'')$. 
Accordingly matrix $R_i$ is updated by $R_i = R_{i-1} ((1-\zeta) \hat{\PP}_i + \zeta \hat{\DD}_i^{\pi, \vartheta})$.
\end{remark}


%{\bf Independent Trajectories (IT).}
%IT assumes the access to a so-called simulator. 
%Specifically, for any to be evaluated policy $\pi$, IT generates a trajectory $\xi^{\pi}_{\vartheta}(s)$ of length $l$ as follows: 
%\begin{align*}
%\xi^{\pi}(s) = (S_0=s, A_0, ,  S_1, A_1, , \ldots, S_{l}, A_l ), ~ \text{where} ~ A_t \sim \vartheta(\cdot| S_t), S_{t+1} \sim \PP^{\pi}_{S_t, A_t},
%\end{align*} 
%where $\PP^{\pi}$ is defined as in \eqref{kernel_defined_by_nature_policy}.
%Let us denote $\xi = \cbr{\xi^{\pi}_{\vartheta}(s)}_{s \in \cS}$, then 
%the IT estimator is defined as 
%\begin{align*}
%\hat{\cV}^{\pi}(s) = - \hat{V}^{\vartheta}_{\PP^{\pi}}(s), ~ \text{where} ~ \hat{V}^{\vartheta}_{\PP^{\pi}}(s) = \tsum_{t=0}^{l} \gamma^t c(S_t, A_t).
%\end{align*}
%We make the following immediate observations.
%
%\begin{proposition}\label{prop_se_properties}
%For any fixed policy $\pi$, SE operator yields 
%\begin{align*}
%\norm{\EE_{\xi} \hat{\cV}^{\pi} - \cV^{\pi}}_\infty \leq \frac{\gamma^l}{1-\gamma},~
%\norm{\hat{\cV}^{\pi}}_\infty \leq \frac{1}{1-\gamma}.
%\end{align*}
%\end{proposition}
%
%\begin{proof}
%It is clear that from the definition of $\PP^{\pi}$ and $\xi$, we obtain 
%\begin{align*}
%\abs{\EE_\xi \hat{V}^{\vartheta}_{\PP^{\pi}}(s) - V^{\vartheta}_{\PP^{\pi}}(s)} \leq \frac{\gamma^l}{1-\gamma},
%~ \norm{\hat{V}^{\vartheta}_{\PP^{\pi}}}_\infty \leq \frac{1}{1-\gamma}.
%\end{align*}
%The rest of the claims follow trivially from \eqref{eq_nature_value_as_player_value}.
%\end{proof}


We are now ready to establish the sample complexity of SFRPE with the SE operator. 
As our first result, we establish an $\cO(\abs{\cS}/\epsilon^2)$ sample complexity for SFRPE to output an $\epsilon$-estimator (in expectation) of the robust value function.


\begin{theorem}\label{thrm_sample_se_expectation}
Suppose SFRPE is instantiated with the SE operator, and 
\begin{align*}
\beta_k = k^{1/2},   ~ \lambda_k =  \frac{ (k+1) \gamma \zeta }{2 (1-\gamma) \sqrt{\mu_w \overline{w}}}, ~\forall k \geq 0.
\end{align*}
For any $\epsilon > 0$, to find an approximate robust value such that 
\begin{align*}
-\epsilon \leq \EE \sbr{\tsum_{t=1}^k \theta_t \hat{\cV}^{\pi_t}}(s) - \cV^{*}(s) \leq \epsilon,  ~ s \in \cS,
\end{align*}
SFRPE needs at most $k = 1 + \frac{64 \gamma^2 \zeta^2  \overline{w}}{\mu_w (1-\gamma)^4 \epsilon^2}$ iterations.
The total number of samples can be bounded by 
\begin{align}\label{eq_num_samples_se_expectation}
{\cO} \rbr{ \frac{\gamma^2 \zeta^2 \abs{\cS} \overline{w} \log(1/\epsilon) }{\mu_w \epsilon^2 (1-\gamma)^5 } + \frac{\abs{\cS}}{1-\gamma} \log \rbr{\frac{1}{\epsilon}} }.
\end{align}
In particular, when the distance generating function $w_s(\cdot)$ is set as in \eqref{dgf_negative_entropy}, the number of samples required is bounded by 
\begin{align*}
{\cO} \rbr{ \frac{\gamma^2 \zeta^2 \abs{\cS}  \log \abs{\cS} \log(1/\epsilon) }{ \epsilon^2 (1-\gamma)^5 } + \frac{\abs{\cS}}{1-\gamma} \log \rbr{\frac{1}{\epsilon}} }.
\end{align*}
\end{theorem}


\begin{proof}
Given Lemma \ref{lemma_value_correspondence} and  Proposition \ref{thrm_stoch_generic_convergence_expectation}, it suffices to make sure 
\begin{align*}
\varepsilon \leq \frac{\epsilon(1-\gamma)}{6}, ~ \frac{4 \gamma \zeta M \sqrt{\overline{w}}}{\sqrt{\mu_w k}} \leq \frac{\epsilon(1-\gamma)}{2},
\end{align*}
which in view of Proposition \ref{prop_se_properties}, can be readily satisfied by 
\begin{align*}
l = \frac{1}{1-\gamma} \log \rbr{\frac{6(1-\gamma)^2}{\epsilon}}, ~ k = 1+ \frac{64 \gamma^2 \zeta^2 M^2 \overline{w}}{\mu_w (1-\gamma)^2 \epsilon^2} = 1 + \frac{64 \gamma^2 \zeta^2  \overline{w}}{\mu_w (1-\gamma)^4 \epsilon^2}.
\end{align*}
Consequently, the total number of samples required is bounded by 
\begin{align*}
\abs{\cS} \cdot l \cdot k = {\cO} \rbr{ \frac{\gamma^2 \zeta^2 \overline{w} \abs{\cS} \log(1/\epsilon)}{\mu_w \epsilon^2 (1-\gamma)^5 } + \frac{\abs{\cS}}{1-\gamma} \log \rbr{\frac{1}{\epsilon}} }.
\end{align*}
The rest of the claim follows from \eqref{bregman_divergence_negative_entropy_lb_ub} and \eqref{bregman_divergence_negative_entropy_sc}, from which we conclude the proof.
\end{proof}

%\yan{need some remark on how the sample complexity scales with $\zeta$, and how finite samples are needed despite have continuous action space}
In view of Theorem \ref{thrm_sample_se_expectation}, it is worth noting here that despite SFRPE being applied to solve the MDP $\mathfrak{M}$ of nature with continuous action space,
its sample complexity is independent of the action space.
 This is in sharp contrast when solving general MDPs, where linear dependence on the size of the action space is necessary.
The obtained sample complexity decomposes into two terms that clearly delineates the role of robustness in affecting the sample complexity.
In particular, the first term corresponds to the price we pay for robustness,
and the second term corresponds to the number of samples needed for estimating non-robust value function that is an $\epsilon$-estimator in expectation. 

We next establish the convergence of SFRPE instantiated with the SE operator in high probability. 

\begin{theorem}\label{thrm_stoch_se_high_prob}
Suppose SFRPE is instantiated with the SE operator, and 
\begin{align}\label{se_high_prob_param_choice}
\beta_k = k^{1/2},   ~ \lambda_k =  \frac{ (k+1) \gamma \zeta }{2 (1-\gamma) \sqrt{\mu_w \overline{w}}}, ~\forall k \geq 0.
\end{align}
Then for any $k \geq 0$ and any $\delta \in (0,1)$, with probability at least $1-\delta$ we have 
\begin{align}
& -\sbr{ \frac{\gamma^l}{1-\gamma} + \frac{4}{(1-\gamma)\sqrt{k}} \log \rbr{\frac{4 \abs{\cS} }{\delta}}}  \nonumber \\
 \leq &  \tsum_{t=1}^k \theta_t 
\hat{\cV}^{\pi_t}(s) - \cV^*(s) \nonumber \\
 \leq  & \frac{4 \gamma \zeta  \sqrt{\overline{w}}}{(1-\gamma)^2 \sqrt{\mu_w k}}
+  \frac{\gamma \zeta}{1-\gamma} 
\rbr{
\frac{2 \gamma^l}{1-\gamma} 
+ \frac{8}{(1-\gamma) \sqrt{k}} \sqrt{  \log(\frac{2 \abs{\cS} }{\delta})}
}
+ 
 \frac{\gamma^l}{1-\gamma} + \frac{4}{(1-\gamma)\sqrt{k}} \sqrt{ \log \rbr{\frac{4 \abs{\cS} }{\delta}}}. \label{high_prob_err_bound_se}
\end{align}
Moreover, the total number of samples required by SFRPE to output $-\epsilon \leq \tsum_{t=1}^k \theta_t 
\hat{\cV}^{\pi_t}(s) - \cV^*(s) \leq \epsilon$ with at least probability $1-\delta$  is bounded by 
\begin{align}\label{se_sample_high_prob}
{\cO} \rbr{
\frac{\gamma^2 \zeta^2 \overline{w} \abs{\cS} \log(1/\epsilon) }{(1-\gamma)^5 \mu_w \epsilon^2} \log \rbr{\frac{\abs{\cS}}{\delta}} 
+ \frac{\abs{\cS}  \log(1/\epsilon)}{(1-\gamma)^3 \epsilon^2} \log \rbr{\frac{\abs{\cS}}{\delta}} 
}.
\end{align}
In particular, when the distance generating function $w_s(\cdot)$ is set as in \eqref{dgf_negative_entropy},  the total number of samples required can be bounded by 
\begin{align*}
{\cO} \rbr{
\frac{\gamma^2 \zeta^2  \abs{\cS} \log \abs{\cS}  \log(1/\epsilon)}{(1-\gamma)^5  \epsilon^2} \log \rbr{\frac{\abs{\cS}}{\delta}} 
+ \frac{\abs{\cS}  \log(1/\epsilon)}{(1-\gamma)^3 \epsilon^2} \log \rbr{\frac{\abs{\cS}}{\delta}} 
}.
\end{align*}
\end{theorem}

\begin{proof}
%Let us first consider bounding 
%$\tsum_{t=1}^k \theta_t \rbr{\hat{\cV}^{\pi_t}(s) - \cV^{\pi_t}(s)}$.
%It can be directly verified that 
%\begin{align}\label{eq_opt_error_accumulation}
%\rbr{\tsum_{t=1}^k \beta_n}^{-1}  \tsum_{t=1}^k \frac{\beta_n^2 \gamma^2 \zeta^2  \norm{\hat{\cV}^{\pi_t}}_\infty^2}{2 \mu_w \lambda_{n-1} (1-\gamma)} \leq  \frac{4 \gamma \zeta M \sqrt{\overline{w}}}{(1-\gamma) \sqrt{\mu_w k}}.
%\end{align}
Fixing $s \in \cS$, for any $\delta > 0$, from Proposition \ref{prop_se_properties}, applying Azuma–Hoeffding inequality yields 
%\yan{expand this, skipped definition of $\theta_t$ in the computation, need to follow the second part}
\begin{align*}
\abs{ \tsum_{t=1}^k \theta_t \rbr{\hat{\cV}^{\pi_t}(s) - \cV^{\pi_t}(s)}  } 
 \leq 
\tsum_{t=1}^k  \frac{\theta_t \gamma^l}{1-\gamma} 
+ \frac{1}{1-\gamma} \sqrt{2 \tsum_{t=1}^k \theta_t^2 \log(\frac{4}{\delta})}
 \overset{(a)}{\leq} \frac{\gamma^l}{1-\gamma} + \frac{4}{(1-\gamma)\sqrt{k}} \sqrt{\log \rbr{\frac{4}{\delta}}}, 
\end{align*}
with probability $1 - \delta/ 2$,
where $(a)$ follows from the definition of $\theta_t$ in \eqref{def_theta} together with $\beta_n = n^{1/2}$.
Applying union bound  yields 
\begin{align}\label{se_high_prob_accum_noise_1}
\abs{ \tsum_{t=1}^k \theta_t \rbr{\hat{\cV}^{\pi_t}(s) - \cV^{\pi_t}(s)} }\leq \frac{\gamma^l}{1-\gamma} + \frac{4}{(1-\gamma)\sqrt{k}} \log \rbr{\frac{4 \abs{\cS} }{\delta}}, ~ \forall s \in \cS, 
\end{align}
with probability $1 - \delta/ 2$.
%We proceed to bound $\tsum_{t=1}^k \frac{\theta_t}{1-\gamma}  \EE_{s' \sim d_{s}^{\pi^*}} \sbr{ \delta_t(s', \pi^*(s'))} $.
Fixing $\DD \in \cD_s$, by definition, we have 
\begin{align*}
\delta_t(s,\DD) = \gamma \zeta \tsum_{a \in \cA}  \vartheta(a|s) \inner{\DD_a - \DD^{\pi_t(s)}_a}{\hat{\cV}^{\pi_t} - \cV^{\pi_t}}.
\end{align*}
From Proposition \ref{prop_se_properties}, it is immediate that 
\begin{align*}
\abs{\EE_{|t}  \tsum_{a \in \cA}  \vartheta(a|s)   \inner{\DD_a - \DD^{\pi_t(s)}_a}{\hat{\cV}^{\pi_t} - \cV^{\pi_t}}} \leq \frac{2 \gamma^l}{1-\gamma} ,~
 \abs{  \tsum_{a \in \cA}  \vartheta(a|s) \inner{\DD_a - \DD^{\pi_t(s)}_a}{\hat{\cV}^{\pi_t} - \cV^{\pi_t}}} \leq \frac{2}{1-\gamma}.
\end{align*}
Consequently, applying Azuma–Hoeffding inequality yields
\begin{align*}
\tsum_{t=1}^k   \theta_t   \tsum_{a \in \cA}  \vartheta(a|s) \inner{\DD_a - \DD^{\pi_t(s)}_a}{\hat{\cV}^{\pi_t} - \cV^{\pi_t}}
\leq \frac{2 \gamma^l}{1-\gamma} 
+ \frac{2}{1-\gamma} \sqrt{2 \tsum_{t=1}^k \theta_t^2 \log(\frac{2}{\delta})} 
\leq 
\frac{2 \gamma^l}{1-\gamma} 
+ \frac{8}{1-\gamma} \sqrt{\log(\frac{2}{\delta})} ,
\end{align*}
with probability $1- {\delta}/{2}$, for any $s \in \cS$.
Applying union bound again, we obtain 
\begin{align*}
\tsum_{t=1}^k  \theta_t   \tsum_{a \in \cA}  \vartheta(a|s) \inner{\DD_a - \DD^{\pi_t(s)}_a}{\hat{\cV}^{\pi_t} - \cV^{\pi_t}}
\leq \frac{2 \gamma^l}{1-\gamma} 
+  \frac{8}{1-\gamma} \sqrt{\log(\frac{2 \abs{\cS}}{\delta})} , ~\forall s \in \cS , 
%+ \frac{2}{1-\gamma} \sqrt{2 \tsum_{t=1}^k \theta_t^2 \log(\frac{2 \abs{\cS} \abs{\cA}}{\delta})}, ~ \forall (s,a) \in \cS \times\cA, 
\end{align*}
with probability $1- {\delta}/{2}$.
Setting $\DD = \pi^*(s)$ in the above inequality yields  
\begin{align}
\tsum_{t=1}^k \frac{\theta_t}{1-\gamma}  \EE_{s' \sim d_{s}^{\pi^*}} \sbr{ \delta_t(s', \pi^*(s'))}
%&\leq 
%\frac{\gamma \zeta}{1-\gamma} 
%\rbr{
%\frac{2 \gamma^l}{1-\gamma} 
%+ \frac{2}{1-\gamma} \sqrt{2 \tsum_{t=1}^k \theta_t^2 \log(\frac{2 \abs{\cS} \abs{\cA}}{\delta})}
%}
% \nonumber  \\
 & \leq 
 \frac{\gamma \zeta}{1-\gamma} 
\rbr{
\frac{2 \gamma^l}{1-\gamma} 
+ \frac{8}{(1-\gamma) \sqrt{k}} \sqrt{  \log(\frac{2 \abs{\cS} }{\delta})}
}. \label{se_high_prob_accum_noise_2}
\end{align}
%where the last inequality follows again from $\beta_n = n^{1/2}$ and the definition of $\theta_t$.
Substituting \eqref{se_high_prob_param_choice}, \eqref{se_high_prob_accum_noise_1}, and \eqref{se_high_prob_accum_noise_2}
into \eqref{eq_generic_prop_stoch}, and noting that $\norm{\hat{\cV}^{\pi_t}}_\infty \leq \frac{1}{1-\gamma}$ therein yields \eqref{high_prob_err_bound_se}.
Finally, \eqref{se_sample_high_prob} follows from a similar argument as in Theorem \ref{thrm_sample_se_expectation} and applying \eqref{high_prob_err_bound_se} and Proposition~\ref{prop_se_properties}.
%and using $\beta_n = n^{1/2}$, $\lambda_n = (n+1) \lambda$ with $\lambda = \frac{\gamma \zeta M}{2 \sqrt{\mu_w \overline{w}}}$ and $M = \frac{1}{1-\gamma}$,
%we obtain
%\begin{align*}
%& -\sbr{ \frac{\gamma^l}{1-\gamma} + \frac{4}{(1-\gamma)\sqrt{k}} \log \rbr{\frac{4 \abs{\cS} }{\delta}}} \\
% \leq &  \tsum_{t=1}^k \theta_t 
%\hat{\cV}^{\pi_t}(s) - \cV^*(s) \nonumber \\
% \leq  & \frac{4 \gamma \zeta  \sqrt{\overline{w}}}{(1-\gamma)^2 \sqrt{\mu_w k}}
%+  \frac{\gamma \zeta}{1-\gamma} 
%\rbr{
%\frac{2 \gamma^l}{1-\gamma} 
%+ \frac{8}{(1-\gamma) \sqrt{k}} \sqrt{  \log(\frac{2 \abs{\cS} \abs{\cA}}{\delta})}
%}
%+ 
% \frac{\gamma^l}{1-\gamma} + \frac{4}{(1-\gamma)\sqrt{k}} \log \rbr{\frac{4 \abs{\cS} }{\delta}}.
%\end{align*}
%The proof is then completed.
\end{proof}

A few remarks are in order for interpreting Theorem \ref{thrm_stoch_se_high_prob}.
First, it is clear that the sample complexity in \eqref{se_sample_high_prob} is comparable to that of \eqref{eq_num_samples_se_expectation}, while the accuracy certificate is now stated with high probability instead of expectation. 
Second, similar to Theorem \ref{thrm_sample_se_expectation}, the sample complexity in \eqref{se_sample_high_prob} consists of two terms of different roots.
The first term corresponds to the price of robustness,
and the second term corresponds to the number of samples required for estimating the standard value function up to $\epsilon$-accuracy with high probability. 
Clearly, when $\zeta =0$ the established sample complexities are tight for evaluating the standard value function in both expectation and in high probability.
It is also important to note that 
in view of \eqref{se_sample_high_prob},  when $\zeta = \cO (1-\gamma)$,  the robust value function can be evaluated with the same number of samples as for evaluating standard value function. 
Consequently there is no additional price of robustness for robust policy evaluation with small-sized ambiguity sets.


The development of SFRPE in this section appears to be new in several aspects. 
The convergence of SFRPE  is established in both expectation and high probability, while existing development of stochastic robust policy evaluation only establish convergence in expectation for mean-squared error \cite{li2022first}. 
Additionally, the impact of the ambiguity set size on the sample complexity has not been previously reported.
%In addition, the size of the ambiguity set in affecting the sample complexity has not been reported before.  
Though these results already hint upon potential benefits of the SFRPE  framework, in the next section we proceed to demonstrate its true advantage of scaling robust policy evaluation to large-scale problems, 
a scenario that appears yet to have an algorithmic solution.
%where no previous algorithmic solution exists.

%
%\yan{need some remark on how the sample complexity scales with $\zeta$,
%\begin{itemize}
%\item the first term of the sample complexity corresponds to the price we pay for robustness -- notably when $\zeta = \cO(1-\gamma)$ there is no price to pay for robustness!
%\item the second term corresponds to the one for standard evaluation 
%\end{itemize}
%}
%
%
%\yan{comparison to prior work: this is the first work where linear dependence on state space size can be obtained. previous approach, aside from making strong assumptions, is inherently subject to the exploration issue over state space}
%
%
%



































%!TEX root = ./robust_pe.tex


\subsection{SFRPE with Linear Function Approximation}\label{sec_stoch_linear_approx}





Unless stated otherwise, going forward we let $\norm{\cdot} = \norm{\cdot}_2$.
For robust MDP with large state space, exact policy evaluation becomes prohibitive. 
In this case one can instead seek to learn a linearly parameterized $\cV^{\pi}_{\theta}(\cdot) \coloneqq   \psi(\cdot)^\top \theta$ that approximate $\cV^{\pi}(\cdot)$ well, 
where $\psi: \cS \to \RR^d$ is the so-called feature map, and we assume without loss of generality that $\norm{\psi(\cdot)}\leq 1$.
In view of Lemma \ref{lemma_value_correspondence}, this is equivalent to approximation of $V^{\vartheta}_{\PP^\pi}$ by $- \psi(\cdot)^\top \theta$. 
As $V^{\vartheta}_{\PP^\pi}$ itself is the value function of $\vartheta$ within MDP $\cM_{\PP^\pi}$, we consider solving\footnote{
Interested readers might suggest directly learning the robust value function of $V^{\vartheta}_r$ -- the ultimate goal of this manuscript -- by simply formulating a least-squares objective that fits the mean-squared robust  Bellman error. 
Though intuitively appealing, this perspective can be computationally intractable. Namely, one can easily construct a $\mathrm{s}$-rectangular robust MDP instance where the resulting  least-squares objective of robust mean-squared Bellman error  is non-convex even in the tabular setting.
Accordingly one can only seek to find the stationary point of the least-square objective  \cite{roy2017reinforcement}.
}: 
\begin{align}\label{ls_objective}
\textstyle
\min_{\theta \in \RR^d} \cbr{g(\theta) \coloneqq  \norm{\Psi \theta - \gamma \mathtt{P}^\pi \Psi \theta - \mathfrak{C}}_{\nu}^2 },
\end{align}
which corresponds to the mean-squared Bellman error of $\vartheta$ within MDP $\cM_{\PP^\pi}$.
Here $\Psi \in \RR^{\abs{\cS} \times d}$ denotes the feature matrix induced by $\psi$ applied to every state, and $\mathtt{P}^{\pi}$ denotes the  transition matrix of the state chain $\cbr{S_t}$ induced by $\vartheta$ within $\cM_{\PP^\pi}$.
It is important to note that here we do not know $\mathtt{P}^\pi$.
Instead, we only assume the access to sample from $\cM_{\overline{\PP}}$.
We introduce the following standard assumption on the $\nu$ and $\psi$. 




\begin{algorithm}[t]
  \caption{Stochastic Least-squares Policy Evaluation (SLPE)}
  \begin{algorithmic}\label{alg_lspe}
    \STATE {\bf Input:} Stepsizes $\cbr{\eta_t}$.
    \STATE {\bf Initialize:} $\theta_0 \in \RR^d$.
        \FOR{ $t = 0, 1, ... T-1$}
    \STATE Sample $s_t \sim \nu$, commit action $a_t \sim \vartheta(\cdot |s_t)$. Sample independent $x_t, x_t' \sim \overline{\PP}_{s_t, a_t}$
    and $y_t, y_t' \sim \DD^{\pi(s_t)}_{a_t}$. 
    \STATE Update:
    \begin{align*}
    \theta_{t+1} = \theta_t - \eta_t \sbr{
    \psi(s_t)^\top \theta_t - \gamma \rbr{(1-\zeta) \psi(x_t') + \zeta \psi(y_t') }^\top \theta_t - \mathfrak{C}(s_t)
    }
    \sbr{
    \psi(s_t) - \gamma \rbr{ (1-\zeta) \psi(x_t) +  \zeta \psi(y_t) }
    }
    \end{align*}
     \ENDFOR
%    \RETURN $\tsum_{t=1}^k \theta_t \cV^{\pi_t}$, where $\theta_t = \beta_t / (\tsum_{t=1}^k \beta_t) $.
\RETURN $\theta_T$, and $\hat{\cV}^{\pi}(\cdot) = \cV^{\pi}_{\theta_T}(\cdot) \coloneqq  \psi(\cdot)^\top \theta_T$.
     \end{algorithmic}
\end{algorithm}



%\yan{should mention that directly using robust bellman operator to form least square objective leads to non-convex objective}



\begin{assumption}\label{assump_sampling_and_feature}
The distribution $\nu$ has full support and the feature matrix $\Psi \in \RR^{\abs{\cS} \times d}$ is non-singular.
That is, $\diag(\nu) \succ 0$ and $\sigma_{\min}(\Psi) > 0$.
\end{assumption}

Following Assumption \ref{assump_sampling_and_feature},
it holds that $\mu = \lambda_{\min} ( \Psi^\top (I - \gamma \mathtt{P}^{\pi})^\top \diag(\nu) (I - \gamma \mathtt{P}^{\pi}) \Psi) ) > 0$.
We also denote $L = \lambda_{\max} \rbr{ \Psi^\top (I - \gamma \mathtt{P}^{\pi})^\top \diag(\nu) (I - \gamma \mathtt{P}^{\pi}) \Psi}$.
Let us denote $\theta^\pi$ as the unique solution of \eqref{ls_objective}, and use 
\begin{align*}
\textstyle
\varepsilon_{\mathrm{approx}} \coloneqq \sup_{\pi \in \Pi} \norm{ \cV^{\pi}_{\theta^\pi} - \cV^{\pi} }_\infty
%= \sup_{\pi \in \Pi} \norm{ V^{\vartheta}_{\PP^\pi} - \Psi \theta^{\pi} }_\infty
\end{align*}
 to characterize the function approximation error of $\cV^{\pi}$. 
Clearly, when $\Psi = I_{\abs{\cS}}$, we have $\varepsilon_{\mathrm{approx}} = 0$.
Our ensuing discussions will often make use of the following quantities to simplify presentation:
\begin{align*}
\textstyle
%c_1 = 4 \norm{\theta^{\pi}} + 2, 
r_{\Theta} = \max_{\pi \in \Pi} \norm{\theta^\pi} , 
~
c_1 = 4 r_{\Theta} + 2 .
\end{align*}
With Assumption \ref{assump_sampling_and_feature} it holds that $r_{\Theta} < \infty$ as $\theta^{\pi}$ is a continuous mapping from $\Pi$ to $\RR^d$.


The stochastic least-squares policy evaluation (SLPE) method (Algorithm \ref{alg_lspe}) can be viewed as solving \eqref{ls_objective} by stochastic gradient descent. 
It can be seen that SLPE requires a simulator of $\cM_{\PP^\pi}$ to draw sample $s_t$, a condition that mainly serves to keep the technical discussion concise. With a slightly more complex analysis  one can also sample $s_t$ by following the trajectory of $\vartheta$ within $\cM_{\PP^\pi}$  \cite{kotsalis2020simple}. 


\begin{lemma}
Define operator $F: \RR^d \to \RR^d$ as 
\begin{align*}
F(\theta) = \rbr{(I - \gamma \mathtt{P}^{\pi}) \Psi}^\top \diag(\nu) (\Psi \theta - \gamma \mathtt{P}^{\pi} \Psi \theta - \mathfrak{C}).
\end{align*}
%Suppose $\diag(\nu) \succ 0$ and $\sigma_{\min}(\Psi) > 0$, then  
Then   $F(\theta^{\pi}) = 0$, and
\begin{align}
\inner{F(\theta)}{\theta - \theta^{\pi}}   \geq  \mu \norm{\theta - \theta^{\pi}}^2. \label{ctd_monotone}
\end{align}
%where  $F(\theta^{\pi}) = 0$.
%Given Assumption \ref{assump_ergodic} we also have $\mu > 0$, $M \succ \mathbf{0}$ and $\theta^{\pi}$ being unique.
\end{lemma}
\begin{proof}
The first part of the claim directly follows from the optimality condition of \eqref{ls_objective}.
In addition, 
\begin{align}
\inner{F(\theta)}{\theta - \theta^{\pi}}
& = \inner{F(\theta) - F(\theta^{\pi})}{\theta - \theta^{\pi}} \nonumber \\
& = \inner{\Psi^\top (I - \gamma \mathtt{P}^{\pi})^\top \diag(\nu)(I-\gamma \mathtt{P}^\pi) \Psi (\theta - \theta^{\pi})}{\theta - \theta^{\pi}} \nonumber \\
& \geq  \mu \norm{\theta - \theta^{\pi}}^2, \nonumber
\end{align}
where the last inequality follows from the definition of $\mu$.
\end{proof}


We proceed by characterizing each step of SLPE and establish the boundedness of iterates in expectation. 

\begin{lemma}\label{lemma_ctd_recursion}
%Define $c_1 = 4 \norm{\theta^{\pi}} + 2$, then
We have
%\begin{align*}
%L= \sigma_{\max}\rbr{\Psi^\top (I - \gamma \mathtt{P}^\pi) \Psi} , ~ & \kappa_2  = \sigma_{\max} \rbr{\Psi^\top (I-\gamma \mathtt{P}^\pi)},  
%\kappa_3 =  \lambda_{\max} \rbr{\Psi^\top \Psi} + 1, \\
%c_1 = 2 \norm{\theta^{\pi}} + 1, 
%~ & \varepsilon_{\mathrm{approx}} = \sup_{\pi \in \Pi} \norm{\Psi \theta^{\pi} - V^{\vartheta}_{\PP^{\pi}}},
%\end{align*}
%where $V^{\vartheta}_{{\PP}^\pi}$ denotes the value function of $\vartheta$ within $\cM_{\mathtt{P}^\pi}$.
\begin{align}
 \EE \sbr{\norm{\theta_{t+1} - \theta^{\pi}}^2 }\leq 
 \rbr{
1 - 2 \eta_t \mu + 32 \eta_t^2  
}
\EE \sbr{\norm{ \theta_t - \theta^{\pi}}^2} + 2 \eta_t^2 c_1^2 .
\label{convergence_mse}
\end{align}
In particular, setting $\eta_t = \eta \leq \frac{\mu}{32}$ yields
%\begin{align}\label{param_for_norm_bound}
%\eta_t = \eta \leq \frac{\mu}{32}.
%%~ (L + \kappa_2) m \rho^\tau \leq \frac{\mu}{4}.
%%T \geq  \frac{1}{\eta \mu} \log 2.
%\end{align}
%Then we have 
\begin{align}\label{mse_bound_each_epoch_iter}
\EE \sbr{\norm{{\theta}_t - \theta^{\pi}}^2} \leq \norm{\theta_0 - \theta^{\pi}}^2 + 
c_1^2.
%+ \frac{2m \kappa_2 \rho^\tau}{\mu} \varepsilon^2_{\mathrm{approx}}.
\end{align}
 
\end{lemma}

\begin{proof}
Clearly,  each update in SLPE (Algorithm \ref{alg_lspe}) is equivalent to the following
\begin{align}\label{ctd_update_equiv_form}
\textstyle
\theta_{t+1} = \argmin_{\theta \in \RR^d} \eta_t \inner{\hat{F}_t(\theta_t)}{\theta} - \frac{1}{2} \norm{\theta - \theta_t}^2,
\end{align}
where 
$
\hat{F}_t (\theta) =  \sbr{
    \psi(s_t)^\top \theta - \gamma \rbr{(1-\zeta) \psi(x_t') + \zeta \psi(y_t') }^\top \theta - \mathfrak{C}(s_t)
    }
    \sbr{
    \psi(s_t) - \gamma \rbr{ (1-\zeta) \psi(x_t) +  \zeta \psi(y_t) }
    }
$.
From the optimality condition of \eqref{ctd_update_equiv_form}, we obtain 
\begin{align}\label{ctd_three_point}
\eta_t \inner{\hat{F}_t(\theta_t)}{ \theta_t - \theta} + \eta_t \inner{\hat{F}_t (\theta_t)}{\theta_{t+1} - \theta_t}
+ \frac{1}{2} \norm{\theta_{t+1} - \theta_t}^2 \leq \frac{1}{2} \norm{\theta - \theta_t}^2 - \frac{1}{2} \norm{\theta - \theta_{t+1}}^2.
\end{align}
We now make the following two observations. First,  from the definition of $(s_t, x_t, x_t', y_t, y_t')$ it is clear that 
\begin{align*}
 \EE_{|t} \sbr{ \hat{F}_t(\theta) }
 = \EE_{|s_t, t} \big[ \sbr{\psi(s_t)^\top \theta - \gamma  \mathtt{P}^\pi_{s_t, \cdot} \Psi \theta - \mathfrak{C}(s_t)}
 \sbr{\psi(s_t) - \gamma \mathtt{P}^\pi_{s_t, \cdot} \Psi } \big]
 = F(\theta), ~ \forall \theta \in \RR^d,
\end{align*}
where $\EE_{|t} \sbr{\cdot}$ denotes the conditional expectation with respect to the $\sigma$-algebra up to (excluding) iteration $t$.
Consequently, by denoting $\delta_t = \hat{F}_t(\theta_t) - F(\theta_t)$, then
\begin{align}
\EE_{|t} \sbr{\delta_t} = 0.
 \label{bias_conditional_expectation}
\end{align}
%where $\EE_{|t} \sbr{\cdot}$ denotes the conditional expectation with respect to the $\sigma$-algebra up to (excluding) iteration $t$ of epoch $e$.
%where 
%\begin{align*}
%L= \sigma_{\max}\rbr{\Psi^\top (I - \gamma \mathtt{P}^\pi) \Psi} , ~ \kappa_2 = \sigma_{\max} \rbr{\Psi^\top (I-\gamma \mathtt{P}^\pi)},  ~ \varepsilon_{\mathrm{approx}} = \norm{\Psi \theta^{\pi} - V^{\vartheta}_{{\PP}^\pi}}.
%\end{align*}
Second, we have 
$ \hat{F}_t(\theta_t)  = \hat{F}_t(\theta_t) - \hat{F}_t(\theta^{\pi}) + \hat{F}_t(\theta^{\pi})  $,
and 
\begin{align}
\norm{ \hat{F}_t(\theta_t) - \hat{F}_t(\theta^{\pi})}
& = \norm{
\sbr{\psi(s_t) -  \gamma \rbr{(1-\zeta) \psi(x_t') + \zeta\psi(y_t')}}^\top  (\theta_t - \theta^\pi)  \sbr{
    \psi(s_t) - \gamma \rbr{ (1-\zeta) \psi(x_t) +  \zeta \psi(y_t) }
    }
} \nonumber  \\
& \leq 4 \norm{\theta_t - \theta^\pi},  \label{norm_bound_diff_stoch_op} \\
\norm{\hat{F}_t(\theta^{\pi})}
& = \norm{
\sbr{
    \psi(s_t)^\top \theta^\pi - \gamma \rbr{(1-\zeta) \psi(x_t') + \zeta \psi(y_t') }^\top \theta^\pi - \mathfrak{C}(s_t)
    }
    \sbr{
    \psi(s_t) - \gamma \rbr{ (1-\zeta) \psi(x_t) +  \zeta \psi(y_t) }
    }
} \nonumber \\
& \leq 4 \norm{\theta^\pi} + 2. \label{norm_bound_stoch_op}
\end{align}
Hence it follows that 
\begin{align*}
\inner{\hat{F}_t(\theta_t)}{\theta_{t+1} - \theta_t} 
& = \inner{  \hat{F}_t(\theta_t) - \hat{F}_t(\theta^{\pi}) + \hat{F}_t(\theta^{\pi})}{\theta_{t+1} - \theta_t} \nonumber \\
& \geq 
- 4 \norm{\theta_t - \theta^{\pi}} \norm{\theta_{t+1} - \theta_t} 
- \norm{\hat{F}_t(\theta^{\pi})} \norm{\theta_{t+1} - \theta_t} \nonumber  \\
& \geq 
-( 4 \norm{\theta_t - \theta^{\pi}} + c_1) \norm{\theta_{t+1} - \theta_t} . 
%\label{ctd_subgrad_inner}
\end{align*}
%where 
%\begin{align*}
%\kappa_3 =  \lambda_{\max} \rbr{\Psi^\top \Psi} + 1,~
%c_1 = 2 \norm{\theta^{\pi}} + 1.
%\end{align*}
Substituting  the above relation into \eqref{ctd_three_point} yields 
\begin{align*}
\eta_t \inner{\hat{F}_t(\theta_t)}{ \theta_t - \theta} - \eta_t ( 4 \norm{\theta_t - \theta^{\pi}} + c_1) \norm{\theta_{t+1} - \theta_t}
+  \frac{1}{2} \norm{\theta_{t+1} - \theta_t}^2 \leq \frac{1}{2} \norm{\theta - \theta_t}^2 - \frac{1}{2} \norm{\theta - \theta_{t+1}}^2,
\end{align*}
which after applying Cauchy-Schwarz inequality and the definition of $\delta_t$, gives 
\begin{align*}
\eta_t \inner{{F}(\theta_t)}{ \theta_t - \theta} - 16 \eta_t^2  \norm{\theta_t - \theta^{\pi}}^2
- \eta_t^2 c_1^2
 \leq \frac{1}{2} \norm{\theta - \theta_t}^2 - \frac{1}{2} \norm{\theta - \theta_{t+1}}^2
 - \eta_t \inner{\delta_t}{\theta_t - \theta}.
\end{align*}
 Setting $\theta = \theta^{\pi}$, and further plugging \eqref{ctd_monotone} into the above relation yields 
\begin{align}\label{norm_recursion_with_noise}
\frac{1}{2} \norm{\theta_{t+1} - \theta^{\pi}}^2 
\leq (\frac{1}{2} - \eta_t \mu + 16 \eta_t^2 ) \norm{\theta_t - \theta^{\pi}}^2
+ \eta_t^2 c_1^2 +  \eta_t \inner{\delta_t}{\theta_t - \theta^{\pi}}.
\end{align}
Taking expectation on both sides and applying  \eqref{bias_conditional_expectation}, we obtain \eqref{ctd_update_equiv_form}.
 Finally, recursive application of \eqref{convergence_mse}  with $\eta_t = \eta \leq \frac{\mu}{32}$ yields 
\begin{align*}
%\label{convergence_mse_clean}
\EE \sbr{\norm{\theta_t - \theta^{\pi}}^2 }
\leq (1-\eta \mu)^t  \EE \sbr{ \norm{\theta_{0} - \theta^{\pi}}^2 } + \frac{2 \eta c_1^2}{\mu} ,
%+ \frac{2m \kappa_2 \rho^\tau}{\mu} \varepsilon^2_{\mathrm{approx}},
\end{align*}
from which the desired claim follows.
%\begin{align*}
%\frac{1}{2} \EE \sbr{\norm{\theta_{t+1} - \theta^{\pi}}^2 }\leq 
% \rbr{
%\frac{1}{2} - \eta_t \mu + 16 \eta_t^2   
%}
%\EE \sbr{\norm{ \theta_t - \theta^{\pi}}^2} + \eta_t^2 c_1^2.
%%\label{convergence_mse}
%\end{align*}
%Further dividing both sides by $\Gamma_{t+1}$, where 
%\begin{align*}
%\Gamma_t = \begin{cases}
%1, ~ & t = 0; \\
%\Gamma_{t-1}  \rbr{1 - 2 \eta_t \mu + 2 \eta_t^2 \kappa_3^2 + 2 \eta_t m (L + \kappa_2) \rho^\tau }, & t \geq 1,
%\end{cases}
%\end{align*}
%and taking the telescopic sum of the resulting inequality, we obtain 
%\begin{align*}
%\frac{1}{2 \Gamma_k} \EE \sbr{ \norm{\theta_k - \theta^{\pi}}^2 } \leq \frac{1}{2} \norm{\theta_0 - \theta^{\pi}}^2 + \tsum_{t=0}^{k-1} \frac{\eta_t^2}{\Gamma_{t+1}} c_1^2
%+ \tsum_{t=0}^{k-1} \frac{\eta_t}{\Gamma_{t+1}} \kappa_2 \rho^\tau \varepsilon^2_{\mathrm{approx}}.
%\end{align*}
%Hence we conclude that 
%\begin{align*}
%\EE \sbr{ \norm{\theta_k - \theta^{\pi}}^2 }
% \leq \Gamma_k  \norm{\theta_0 - \theta^{\pi}}^2
% + 2 \Gamma_k c_1^2 \tsum_{t=0}^{k-1} \frac{\eta_t^2}{\Gamma_{t+1}} 
% + 2 \Gamma_k  \kappa_2 m \rho^\tau \varepsilon^2_{\mathrm{approx}} \tsum_{t=0}^{k-1} \frac{\eta_t}{\Gamma_{t+1}} .
%\end{align*}
\end{proof}

%The next lemma establishes the boundedness of $\norm{ \theta_t - \theta^{\pi}}^2$ in expectation.


%We proceed to show that $ \EE \sbr{\norm{\theta_t - \theta^{\pi}}^2}$ is indeed bounded with proper parameter specifications.
%
%\begin{lemma}\label{lemma_norm_bound_in_expectation}
%Set 
%\begin{align}\label{param_for_norm_bound}
%\eta_t = \eta \leq \frac{\mu}{32}.
%%~ (L + \kappa_2) m \rho^\tau \leq \frac{\mu}{4}.
%%T \geq  \frac{1}{\eta \mu} \log 2.
%\end{align}
%Then we have 
%\begin{align}\label{mse_bound_each_epoch_iter}
%\EE \sbr{\norm{{\theta}_t - \theta^{\pi}}^2} \leq \norm{\theta_0 - \theta^{\pi}}^2 + 
%\frac{2 \eta c_1^2}{\mu} .
%%+ \frac{2m \kappa_2 \rho^\tau}{\mu} \varepsilon^2_{\mathrm{approx}}.
%\end{align}
%\end{lemma}
%
%\begin{proof}
%%We proceed to show that $ \EE \sbr{\norm{\theta_t - \theta^{\pi}}^2}$ is indeed bounded with proper parameter specifications.
%%Let 
%%\begin{align}\label{param_for_norm_bound}
%%\eta_t = \eta \leq \frac{\mu}{4 \kappa_3^2}, ~ (L + \kappa_2) m \rho^\tau \leq \frac{\mu}{4},
%%\end{align}
% Recursive application of \eqref{convergence_mse} combined with \eqref{param_for_norm_bound} yields 
%\begin{align}\label{convergence_mse_clean}
%\EE \sbr{\norm{\theta_t - \theta^{\pi}}^2 }
%\leq (1-\eta \mu)^t  \EE \sbr{ \norm{\theta_{0} - \theta^{\pi}}^2 } + \frac{2 \eta c_1^2}{\mu} 
%%+ \frac{2m \kappa_2 \rho^\tau}{\mu} \varepsilon^2_{\mathrm{approx}},
%\end{align}
%from which the desired claim follows.
%%Consequently, with the choice of $T$ 
%%%\begin{align}\label{ctd_expectation_bd_choice_of_t}
%%%T \geq  \frac{1}{\eta \mu} \log 2,
%%%%T = \ceil{ \frac{1}{\eta \mu} \log 2},
%%%\end{align} 
%%and the definition of $\cbr{{\theta}}$, we obtain from \eqref{convergence_mse_clean} that 
%%\begin{align}\label{mse_bound_each_epoch}
%%\EE \sbr{\norm{{\theta} - \theta^{\pi}}^2} \leq \norm{\theta_0 - \theta^{\pi}}^2 + 
%%\frac{4 \eta c_1^2}{\mu} 
%%+ \frac{4m \kappa_2 \rho^\tau}{\mu} \varepsilon^2_{\mathrm{approx}}.
%%\end{align}
%%Applying \eqref{convergence_mse_clean} again with \eqref{mse_bound_each_epoch} then yields the desired claim. 
%%\begin{align}\label{mse_bound_each_epoch_iter}
%%\EE \sbr{\norm{{\theta}_t - \theta^{\pi}}^2} \leq \norm{\theta_0 - \theta^{\pi}}^2 + 
%%\frac{6 \eta c_1^2}{\mu} 
%%+ \frac{6m \kappa_2 \rho^\tau}{\mu} \varepsilon^2_{\mathrm{approx}}.
%%\end{align}
%\end{proof}


The next lemma establishes the fast bias reduction  of the estimated value function.

\begin{lemma}\label{lemma_bias_linear}
%Let $L = \lambda_{\max} \rbr{ \Psi^\top (I - \gamma \mathtt{P}^{\pi})^\top \diag(\nu) (I - \gamma \mathtt{P}^{\pi}) \Psi}$, and
Denote $\overline{\theta}_t = \EE \sbr{\theta_t}$, and let
\begin{align}\label{param_choice_bias_reduction}
\eta_t = \eta \leq \frac{\mu}{ L^2}.
%T \geq  \frac{1}{\eta \mu} \log 2.
\end{align}
Then we have 
\begin{align}\label{bound_on_parameter_bias}
\norm{\overline{\theta}_t - \theta^{\pi}}^2
\leq 
(1 - \eta \mu)^t \norm{\theta_0 - \theta^{\pi}}^2.
\end{align}
Consequently, 
%by denoting $\cV^{\pi}_{\theta} = - \Psi \theta$, 
we obtain 
\begin{align}\label{bound_on_value_bias}
\norm{\EE \sbr{ \cV^{\pi}_{\theta_t}} - \cV^{\pi}}_\infty \leq  (1 - \eta \mu)^{t/2} \norm{\theta_0 - \theta^{\pi}}
+  \varepsilon_{\mathrm{approx}} .
\end{align}
%\yan{need a statement on the err of the estimated value}
\end{lemma}

\begin{proof}
%Define $\overline{\theta}_t = \EE\sbr{\theta_t}$ and $\overline{\theta} = \EE\sbr{\theta}$. 
Since we have $\theta_{t+1} = \theta_t - \eta_t \hat{F}_t(\theta_t)$, taking expectation on both sides yields 
\begin{align*}
\overline{\theta}_{t+1} = \EE\sbr{\theta_{t+1}}
= \EE\sbr{\theta_t} - \eta_t \EE [\hat{F}_t(\theta_t)] 
& = \EE\sbr{\theta_t}
-  \eta_t \EE [ \EE_{|t} [\hat{F}_t(\theta_t)] ] \\
& = 
 \EE\sbr{\theta_t}
- \eta_t \EE [ F(\theta_t) + \EE_{|t} [\delta_t] ]  \\
& = 
\overline{\theta}_t 
- \eta_t  F(\overline{\theta}_t)
\end{align*}
where the last equality follows from the linearity of $F(\cdot)$ and \eqref{bias_conditional_expectation}.
Consequently, we have 
\begin{align}
\norm{\overline{\theta}_{t+1} - \theta^{\pi}} 
& = \norm{ \overline{\theta}_t 
-  \eta_t  F(\overline{\theta}_t)
- \theta^{\pi}
}  \nonumber \\
& \leq
\norm{\overline{\theta}_t - \theta^{\pi}}^2 - 2 \eta_t \inner{F(\overline{\theta}_t) }{\overline{\theta}_t - \theta^{\pi}}
+   \eta_t^2 \norm{F(\overline{\theta}_t)}^2 . 
\label{bias_progress_raw}
\end{align}
%We proceed to bound $\norm{\EE\sbr{\delta_t}}$ and $\norm{F(\overline{\theta}_t)}$ separately. 
%For $\norm{\EE\sbr{\delta_t}}$, we have 
%\begin{align}
%\norm{\EE[\delta_t]}
%= \norm{\EE[ \EE_{|t}[\delta_t] ]}
%&\overset{(a)}{ =} \lVert
%\EE \big[
% \Psi^\top (\EE_{|t}[\hat{M}] - M)(I -\gamma \mathtt{P}^\pi) \Psi (\theta_t  -  \theta^{\pi} )
% \nonumber \\
%& ~~~~~~ + \Psi^\top (\EE_{|t}[\hat{M}] - M) (I -\gamma \mathtt{P}^\pi)  [\Psi \theta^{\pi} - V^{\vartheta}_{{\PP}^\pi}] 
%\big]
%\rVert \nonumber \\
%& \leq 
%m \rho^\tau \sbr{ L \EE \sbr{\norm{\theta_t - \theta^{\pi}}}
%+  \kappa_2 \varepsilon_{\mathrm{approx}} }, \label{norm_bound_on_bias},
%\end{align}
%where $(a)$ follows from \eqref{bias_conditional_expectation}.
In addition,  it holds that 
\begin{align}
\norm{F(\overline{\theta}_t)}
& = \norm{F(\overline{\theta}_t) - F(\theta^\pi)}  = \norm{\Psi^\top (I - \gamma \mathtt{P}^{\pi})^\top \diag(\nu) (I - \gamma \mathtt{P}^{\pi}) \Psi (\overline{\theta}_t - \theta^{\pi}}  \leq L  \norm{\overline{\theta}_t - \theta^{\pi}}. 
\label{bound_on_op_norm}
\end{align}
Hence by plugging \eqref{bound_on_op_norm} into \eqref{bias_progress_raw}, and applying \eqref{ctd_monotone}, we obtain 
\begin{align}
\norm{\overline{\theta}_{t+1} - \theta^{\pi}}^2 
& \leq (1- 2\eta_t \mu + L^2 \eta_t^2) \norm{\overline{\theta}_t - \theta^{\pi}}^2 
\label{convergence_bias}
\end{align}
%
%We proceed to show that $ \EE \sbr{\norm{\theta_t - \theta^{\pi}}^2}$ is indeed bounded with proper parameter specifications.
%Let 
%\begin{align}\label{param_for_norm_bound}
%\eta_t = \eta \leq \frac{\mu}{4 \kappa_3^2}, ~ (L + \kappa_2) m \rho^\tau \leq \frac{\mu}{4},
%\end{align}
%then recursive application of \eqref{convergence_mse} yields 
%\begin{align}\label{convergence_mse_clean}
%\EE \sbr{\norm{\theta_t - \theta^{\pi}}^2 }
%\leq (1-\eta \mu)^t  \EE \sbr{ \norm{\theta_{0} - \theta^{\pi}}^2 } + \frac{2 \eta c_1^2}{\mu} 
%+ \frac{2m \kappa_2 \rho^\tau}{\mu} \varepsilon^2_{\mathrm{approx}}.
%\end{align}
%Consequently, with 
%\begin{align}\label{ctd_expectation_bd_choice_of_t}
%T \geq  \frac{1}{\eta \mu} \log 2,
%%T = \ceil{ \frac{1}{\eta \mu} \log 2},
%\end{align} 
%and the definition of $\cbr{{\theta}}$, we obtain from \eqref{convergence_mse_clean} that 
%\begin{align}\label{mse_bound_each_epoch}
%\EE \sbr{\norm{{\theta} - \theta^{\pi}}^2} \leq \norm{\theta_0 - \theta^{\pi}}^2 + 
%\frac{4 \eta c_1^2}{\mu} 
%+ \frac{4m \kappa_2 \rho^\tau}{\mu} \varepsilon^2_{\mathrm{approx}}.
%\end{align}
%Applying \eqref{convergence_mse_clean} again with \eqref{mse_bound_each_epoch} then shows 
%\begin{align}\label{mse_bound_each_epoch_iter}
%\EE \sbr{\norm{{\theta}_t - \theta^{\pi}}^2} \leq \norm{\theta_0 - \theta^{\pi}}^2 + 
%\frac{6 \eta c_1^2}{\mu} 
%+ \frac{6m \kappa_2 \rho^\tau}{\mu} \varepsilon^2_{\mathrm{approx}}.
%\end{align}
%Clearly, \eqref{param_for_norm_bound}  holds given the parameter choice specified in \eqref{param_choice_bias_reduction}, and consequently Lemma \ref{lemma_norm_bound_in_expectation} applies.
%On the other hand, suppose further that 
%\begin{align}\label{param_for_bias_bound_1}
%\eta \leq \frac{\mu}{8 L^2}, ~
%m \rho^{\tau} \max \cbr{ L, \kappa_2} \leq \frac{\mu}{4},
%~
%m \rho^\tau  \leq \eta \kappa_2, ~ \eta^2 m^2 \rho^{2 \tau} \leq \frac{1}{2},
%\end{align}
%then recursive application of \eqref{convergence_bias}, combined with  \eqref{mse_bound_each_epoch_iter} yields 
%\begin{align}\label{convergence_bias_cleaner}
%& \norm{\overline{\theta}_t - \theta^{\pi}}^2 \nonumber \\
%\leq &  (1-\eta \mu)^{t} \norm{\theta_0 - \theta^{\pi}}^2 
%+ \frac{8 \eta \kappa_2^2}{\mu}  \varepsilon^2_{\mathrm{approx}} 
%+ \frac{m \rho^\tau}{\mu} (L + 4 \eta m \rho^\tau L^2)
%(
%\norm{\theta_0 - \theta^{\pi}}^2 + \frac{2 \eta c_1^2 }{\mu} +  \frac{2m \kappa_2 \rho^\tau}{\mu} \varepsilon^2_{\mathrm{approx}}
%).
%\end{align}
%Further requiring that 
%\begin{align}\label{param_for_bias_bound_2}
%\eta \leq \frac{\mu }{16 \kappa_2^2},
%~
% \frac{m \rho^\tau}{\mu} (L + 4 \eta m \rho^\tau L^2)
%(
%\norm{\theta_0 - \theta^{\pi}}^2 + \frac{2 \eta c_1^2 }{\mu} +  \frac{2m \kappa_2 \rho^\tau}{\mu} \varepsilon^2_{\mathrm{approx}}
%) \leq \frac{ \varepsilon^2_{\mathrm{approx}}}{2},
%\end{align}
The above relation simplifies to \eqref{bound_on_parameter_bias} with choice of $\cbr{\eta_t}$ in \eqref{param_choice_bias_reduction}.
%\begin{align*}
%\norm{\overline{\theta}_t - \theta^{\pi}}^2
%\leq 
%(1 - \eta \mu)^t \norm{\theta_0 - \theta^{\pi}}^2
%+  \varepsilon^2_{\mathrm{approx}} .
%\end{align*}
%Consequently, with the definition of $\overline{\theta}$ and the choice of $T$, we obtain   
%\begin{align*}
%\norm{\overline{\theta}^{(e+1)} - \theta^{\pi}}^2 \leq \frac{1}{2} \norm{\overline{\theta} - \theta^{\pi}}^2 
%+ \varepsilon^2_{\mathrm{approx}} ,
%\end{align*}
%from which \eqref{bound_on_parameter_bias} follows.
%It remains to note that \eqref{param_for_bias_bound_1} and \eqref{param_for_bias_bound_2} can be satisfied by the parameter choice specified in \eqref{param_choice_bias_reduction}.
%, from which we obtain \eqref{bound_on_parameter_bias}.
Finally, \eqref{bound_on_value_bias} follows from the direct application of \eqref{bound_on_parameter_bias},
by noting that 
\begin{align*}
\abs{\EE \sbr{ \cV^{\pi}_{\theta_t}}(s) - \cV^{\pi}(s)}
& \leq 
 \abs{\EE \sbr{ \cV^{\pi}_{\theta_t}}(s) - \cV^\pi_{\theta^\pi}(s)} +  \abs{ \cV^\pi_{\theta^\pi}(s)- \cV^{\pi}(s)}
\\ & \leq 
 \abs{ \rbr{\overline{\theta}_t - \theta^\pi}^\top \psi(s)} +  \abs{ \cV^\pi_{\theta^\pi}(s)- \cV^{\pi}(s)} \\
 & \leq 
  (1 - \eta \mu)^{t/2} \norm{\theta_0 - \theta^{\pi}} +  \varepsilon_{\mathrm{approx}} .
\end{align*}
The proof is then completed.
\end{proof}






With Lemma \ref{lemma_ctd_recursion} and \ref{lemma_bias_linear} in place, we are ready to establish the sample complexity of SFRPE with the SLPE operator, 
in order to output an $\epsilon$-estimator of the robust value function in expectation.
%For notational simplicity, going forward we denote 
%$r_{\Theta} = \max_{\pi \in \Pi} \norm{\theta^\pi} < \infty$.





\begin{theorem}\label{thrm_lspe_expectation}
For any $\epsilon \geq \frac{8 \varepsilon_{\mathrm{approx}} }{(1-\gamma)}$,
let SFRPE be instantiated with SLPE operator with  evaluation parameters
\begin{align*}
\eta \leq \frac{\mu}{L^2 +  32} , ~ \theta_0 = 0, ~ T = \cO \rbr{ \frac{1}{\eta \mu} \log \rbr{\frac{r_\Theta }{ \epsilon}} }, 
\end{align*}
 and  optimization parameters
\begin{align*}
\beta_k = k^{1/2},   ~ \lambda_k = \frac{ (k+1) M \gamma \zeta }{2  \sqrt{\mu \overline{w}}}, ~\forall k \geq 0, 
\end{align*}
where $M^2 \geq 4 \rbr{r_\Theta^2 + c_1^2  + \varepsilon^2_{\mathrm{approx}}} + {2}/{(1-\gamma)^2}$.
To find an approximate robust value such that 
\begin{align*}
-\epsilon \leq \EE \sbr{\tsum_{t=1}^k \theta_t \cV^{\pi_t}}(s) - \cV^*(s) \leq \epsilon,  ~ s \in \cS,
\end{align*}
SFRPE needs at most $k = 1 + \frac{256 \gamma^2 \zeta^2 M^2 \overline{w}}{(1-\gamma)^2 \mu \epsilon^2}$ iterations.
%where $M^2 = 4 \rbr{r_\Theta^2 + 1  + \varepsilon^2_{\mathrm{approx}}} + {2}/{(1-\gamma)^2}$.
The total number of samples can be bounded by 
\begin{align}\label{eq_sample_lspe_expectation}
\cO \rbr{
\frac{1}{\eta \mu} 
\rbr{1 + \frac{\gamma^2 \zeta^2 M^2 \overline{w}}{(1-\gamma)^2 \mu_w \epsilon^2}}
\log \rbr{\frac{r_\Theta }{ \epsilon}}
} .
% \cO \rbr{
%\frac{ L^2 + \kappa_2^2 + \kappa_3^2 + c_1^2}{ \mu^2} 
%\rbr{1 + \frac{\gamma^2 \zeta^2  M^2 \overline{w}}{(1-\gamma)^2 \mu \epsilon^2}}
%\log \rbr{\frac{r_\Theta }{(1-\gamma) \epsilon}}
%} .
\end{align}
In particular, when the distance generating function $w(\cdot)$ is set as in \eqref{dgf_negative_entropy}, the number of samples required is bounded by 
\begin{align*}
\cO \rbr{
\frac{1}{\eta \mu} 
\rbr{1 + \frac{\gamma^2 \zeta^2 M^2 \log \abs{\cS} }{(1-\gamma)^2  \epsilon^2}}
\log \rbr{\frac{r_\Theta }{\epsilon}}
} .
% \cO \rbr{
%\rbr{ L^2 + \kappa_2^2 + \kappa_3^2 + c_1^2}
%\rbr{1 + \frac{\gamma^2 \zeta^2  M^2   \log \abs{\cS} }{(1-\gamma)^2  \epsilon^2}}
%\log \rbr{\frac{r_\Theta }{(1-\gamma) \epsilon}}
%} .
\end{align*}
\end{theorem}

\begin{proof}
With a slight overload of notation, let us denote $\theta_k$ as the parameters output by SLPE at the $k$-th iteration of SFRPE. 
Then given the choice of parameters, one can apply Lemma \ref{lemma_ctd_recursion} and obtain  
%$
%\EE \sbr{ \norm{\theta_k - \theta^{\pi_k}}^2} \leq r_\Theta^2 + c_1^2
%$,
%and hence 
%\yan{ambiguity in notation $\theta_k$ here}
\begin{align*}
%\label{ctd_estimate_expectation_bound_on_korm}
\EE \sbr{ \norm{\cV^{\pi_k}_{\theta_k} - \cV^{\pi_k}}_\infty^2} 
\leq 2 \rbr{ \EE \norm{\theta_k - \theta^{\pi_k}}^2   + \varepsilon^2_{\mathrm{approx}}} 
\leq 2 \rbr{r_\Theta^2 + c_1^2  + \varepsilon^2_{\mathrm{approx}}}.
\end{align*}
Consequently, 
\begin{align*}
\EE  \sbr{ \norm{\cV^{\pi_k}_{\theta_k}}_\infty^2} \leq 2 \rbr{\EE \sbr{ \norm{\cV^{\pi_k}_{\theta_k} - \cV^{\pi_k}}_\infty^2} + \norm{\cV^{\pi_k}}_\infty^2 } \leq  4 \rbr{r_\Theta^2 + c_1^2  + \varepsilon^2_{\mathrm{approx}}} + {2}/{(1-\gamma)^2} .
\end{align*}
In addition, from Lemma \ref{lemma_bias_linear}, 
for any $\epsilon \geq \frac{8 \varepsilon_{\mathrm{approx}}}{1-\gamma} $, 
taking $T = \cO\rbr{ \frac{1}{\eta \mu} \log \rbr{\frac{r_\Theta }{ \epsilon}}}$ further yields that 
\eqref{stoch_expecation_conv_bias_condition} is satisfied with 
\begin{align*}
\frac{3\varepsilon}{1-\gamma} = \frac{3 \epsilon}{4}, ~ M^2 = 4 \rbr{r_\Theta^2 + c_1^2  + \varepsilon^2_{\mathrm{approx}}} + {2}/{(1-\gamma)^2} .
\end{align*}
Combining the above relation and Proposition \ref{thrm_stoch_generic_convergence_expectation}, the total number of iterations required by SFRPE is bounded by 
$
k = 1 + \frac{256 \gamma^2 \zeta^2 M^2 \overline{w}}{(1-\gamma)^2 \mu_w \epsilon^2}.
$
The total number of samples can be bounded by 
\begin{align*}
T \cdot k  & = \cO \rbr{
\frac{1}{\eta \mu} 
\rbr{1 + \frac{\gamma^2 \zeta^2 M^2 \overline{w}}{(1-\gamma)^2 \mu_w \epsilon^2}}
\log \rbr{\frac{r_\Theta }{\epsilon}}
} .
% \\
%& = \cO \rbr{
%\frac{ L^2 + \kappa_2^2 + \kappa_3^2 + c_1^2}{ \mu^2} 
%\rbr{1 + \frac{\gamma^2 \zeta^2  M^2 \overline{w}}{(1-\gamma)^2 \mu_w \epsilon^2}}
%\log \rbr{\frac{r_\Theta }{(1-\gamma) \epsilon}}
%} .
\end{align*}
The proof is then completed.
%We conclude the proof by noting  taking $\eta = \frac{\mu}{16(L^2 + \kappa_2^2 + \kappa_3^2 + c_1^2)}$ suffices to satisfy the condition \eqref{xx} of Lemma \ref{lemma_bias_linear}
\end{proof}

In view of Theorem \ref{thrm_lspe_expectation}, SFRPE yields an $\tilde{\cO}({\zeta^2}/ \epsilon^2 + \log(1/\epsilon))$ sample complexity with linear function approximation.
Notably, this appears to be the first method of stochastic robust policy evaluation beyond tabular settings with non-asymptotic convergence, and does so without restrictive assumptions on the transition kernel and discount factor \cite{tamar2014scaling, roy2017reinforcement}. 
Similar to Theorem \ref{thrm_sample_se_expectation} and \ref{thrm_stoch_se_high_prob}, the obtained sample complexity in \eqref{eq_sample_lspe_expectation} admits a natural decomposition, with the first term corresponding to learning the standard value function up to $\epsilon$-accuracy in bias, and the second term corresponding to the price of robustness. 


With the same spirit as Theorem \ref{thrm_stoch_se_high_prob}, we proceed to show that with the same number of samples, SFRPE with SLPE operator can indeed output an $\epsilon$-estimator of the robust value function in high probability.
To this end, we first establish the following high probability bound on the iterate produced by SLPE operator. 
The challenge for such a statement primarily comes from that the noise in SLPE itself depends on the boundedness of the iterate, thus preventing a direct application of standard concentration argument. 
The following lemma constructs an approximate martingale sequence that upper bounds the noise of SLPE in high probability, with increment of the former sequence  bounded with proper choice of stepsize. 
Similar argument can also be found in \cite{li2023policy} for bounded increments but potentially non-zero conditional expectation.
%Such an argument can be viewed as a generalization of \cite{li2023policy} to unbounded sequence. 
%In view of this we proceed to show that the iterate of SLPE is indeed bounded in high probability. 
%In view of this observation, we adopt a similar technical idea that can be found in \cite{li2023policy}.
%, originally designed for establishing inherent exploration properties  of stochastic policy optimization.

\begin{lemma}\label{lemma_norm_bound_high_prob}
Fix total iterations $T > 0$.
For any $\delta \in (0,1)$, let $\eta_t = \eta \coloneqq \alpha / \sqrt{T}$ with 
\begin{align}
\alpha & \leq 
\min \big\{
%\frac{\mu \sqrt{T}}{\kappa_3^2}, ~
\frac{\mu }{32}, ~
\frac{1}{{192 \sbr{L^2 + c_1^2 + 16} }\log(2T / \delta)},~
\frac{1}{2 c_1}, ~
\frac{1}{4 G}
\big\},  \label{param_choice_bounded_norm_high_prob} \\
G & = 4 \sqrt{\log(2T /\delta)} \big[(L + 4) \norm{\theta_0 - \theta^{\pi}}^2 + c_1 \norm{\theta_0 - \theta^{\pi}}\big] + c_1. \label{G_bounded_norm_high_prob}
\end{align}
%where $G \geq 4 \sqrt{\log(2T /\delta)} \big[(L + \kappa_3) \norm{\theta_0 - \theta^{\pi}}^2 + (c_1 + \kappa_2  \varepsilon_{\mathrm{approx}}) \norm{\theta_0 - \theta^{\pi}}\big] + c_1$.
Then 
with probability at least $1-\delta$,
\begin{align*}
\norm{\theta_t - \theta^{\pi}}^2
\leq  \norm{\theta_0 - \theta^{\pi}}^2 + 1, ~ \forall t \leq T.
\end{align*}
\end{lemma}
%Going forward let us fix the epoch length $T > 0$ and the number of epochs $E > 0$.
%and assume $\eta_t = \eta \coloneqq \alpha / \sqrt{T}$ for some $\alpha > 0$ to be determined later. 
\begin{proof}
We begin by noting that  
\begin{align*}
\delta_t  = \hat{F}_t(\theta_t) - F(\theta_t)  =  \hat{F}_t(\theta_t)  - \hat{F}_t(\theta^\pi) -  F(\theta_t)  +  \hat{F}_t(\theta^\pi).
\end{align*}
Consequently, it is clear that  
\begin{align}\label{eq_norm_bound_with_noise_version}
\inner{\delta_t}{\theta_t - \theta^{\pi}}
& = 
 \inner{ \hat{F}_t(\theta_t)  - \hat{F}_t(\theta^\pi)}{\theta_t - \theta^\pi}
- \inner{F(\theta_t) }{\theta_t - \theta^\pi}
+ \inner{\hat{F}_t(\theta^\pi)}{\theta_t - \theta^\pi} \nonumber 
\\
& \leq 
(L + 4) \norm{\theta_t - \theta^\pi}^2  + 
 c_1 \norm{\theta_t - \theta^\pi},
\end{align}
where the last inequality applies \eqref{norm_bound_diff_stoch_op}, \eqref{norm_bound_stoch_op}, and \eqref{bound_on_op_norm}.
In addition, note that \eqref{norm_recursion_with_noise}  still holds.  Hence with 
$
 \eta \leq \frac{\mu}{32}, 
$
%or equivalently, $\alpha \leq \frac{\mu \sqrt{T}}{\kappa_3^2}$, 
or equivalently, $\alpha \leq \frac{\mu }{32}$, 
we obtain 
\begin{align}\label{recursion_on_norm_const_stepsize}
\norm{\theta_{t+1} - \theta^{\pi}}^2 
\leq  \norm{\theta_t - \theta^{\pi}}^2
+ 2\eta^2 c_1^2 +  2 \eta \inner{\delta_t}{\theta_t - \theta^{\pi}}.
\end{align}
Let us define random sequences $\cbr{X_t \coloneqq \inner{\delta_t}{\theta_t - \theta^{\pi}}}$, $\cbr{\tilde{X}_t \coloneqq \inner{\delta_t}{\theta_t - \theta^{\pi}} \mathbbm{1}_{\cG_t}}$,
where  $\cG_t = \cbr{Y_t \leq G \sqrt{t}}$, and
\begin{align*}
 Y_0 \equiv \tilde{Y}_0 \equiv 0,  ~
Y_t  = Y_{t-1} + X_{t-1}, ~
\tilde{Y}_t  = \tilde{Y}_{t-1} + \tilde{X}_{t-1}.
\end{align*}
%Now consider event $\cG_t = \cbr{Y_t \leq G \sqrt{te}}$ for some $G > 0$.
%Now consider event 
%$\cG_t = \cbr{\tsum_{i=0}^{t-1} \inner{\delta_i}{\theta_i - \theta^{\pi}} \leq G \sqrt{t}}$ for some $G > 0$.
Then over $ \cG_t$, 
taking the telescopic sum of \eqref{recursion_on_norm_const_stepsize}  yields   
\begin{align}
\norm{\theta_t - \theta^{\pi}}^2 & \leq \norm{\theta_0 - \theta^{\pi}}^2 + 2 t \eta^2 c_1^2 + 2 \eta G \sqrt{t} \nonumber \\
& \leq  \norm{\theta_0 - \theta^{\pi}}^2 + 2 \alpha^2 c_1^2  + 2 \alpha G  \coloneqq M_{(\alpha, G)}.
\label{norm_bound_on_good_event_generic}
\end{align}
We proceed to establish that \eqref{norm_bound_on_good_event_generic} indeed holds with probability at least $1-\delta$ given proper choice of $(\alpha, G)$.

%Let us define random sequences $\cbr{X_t \coloneqq \inner{\delta_t}{\theta_t - \theta^{\pi}}}$, $\cbr{\tilde{X}_t \coloneqq \inner{\delta_t}{\theta_t - \theta^{\pi}} \mathbbm{1}_{\cG_t}}$,
%together with 
%\begin{align*}
%Y_t = \tsum_{i=0}^{t-1} X_i, ~ \tilde{Y}_t = \tsum_{i=0}^{t-1} X_i', ~ Y_0 \equiv Y_0' \equiv 0.
%\end{align*}
%Then by definition we have $\cG_t = \cbr{Y_t \leq G \sqrt{t}}$.
First, it is  clear that 
\begin{align}\label{aux_sequence_norm_bound}
\abs{\tilde{X}_t} = \abs{\inner{\delta_t}{\theta_t - \theta^{\pi}} \mathbbm{1}_{\cG_t}} \leq (L + 4) M_{(\alpha, G)} + c_1 \sqrt{M_{(\alpha, G)}},
\end{align}
where the last inequality follows from \eqref{eq_norm_bound_with_noise_version} and \eqref{norm_bound_on_good_event_generic}.
Moreover, we also have 
\begin{align}
\abs{\EE_{|t}\sbr{\tilde{X}_t}}
& =  \abs{\EE_{|t}\sbr{\inner{\delta_t}{\theta_t - \theta^{\pi}}}  \mathbbm{1}_{\cG_t}}  = 0,\label{aux_sequence_conditional_expectation}
\end{align}
where the last equality follows from \eqref{bias_conditional_expectation}.
In view of \eqref{aux_sequence_norm_bound} and \eqref{aux_sequence_conditional_expectation}, we can now apply Azuma–Hoeffding inequality to $\cbr{\tilde{Y}_t}$ and obtain 
\begin{align}
\abs{
\tilde{Y}_t
} \leq 2 \sqrt{t}  \big[ (L + 4) M_{(\alpha, G)} + c_1 \sqrt{M_{(\alpha, G)}} \big]\sqrt{\log(2T/ \delta)}, 
~ \forall t \leq T,
\label{aux_sequence_accumulation_raw}
\end{align}
with probability $1-\delta$.
Through direct computation, one can verify that \eqref{aux_sequence_accumulation_raw} and the $(\alpha, G)$ specified in \eqref{param_choice_bounded_norm_high_prob}, \eqref{G_bounded_norm_high_prob}, together with the definition of $M_{(\alpha, G)}$, implies 
%\begin{align}
%& \sbr{ c_1 + \kappa_2  \varepsilon_{\mathrm{approx}} + L + \kappa_3} \sqrt{\alpha} \leq \frac{1}{8 \sqrt{\log(2T/\delta)}}, 
%~ \tau = \tilde{\cO} (t_{\mathrm{mix}}), \label{param_alpha_tau} \\
%& G \geq 4 \sqrt{\log(2T /\delta)} \big[(L + \kappa_3) \norm{\theta_0 - \theta^{\pi}}^2 + (c_1 + \kappa_2  \varepsilon_{\mathrm{approx}}) \norm{\theta_0 - \theta^{\pi}}\big] + c_1 , \label{param_choice_G}
%\end{align}
\begin{align}\label{aux_seq_accumulation_bounded}
\abs{\tilde{Y}_t} \leq G \sqrt{t }, ~ \forall t \leq T,
\end{align}
with probability $1-\delta$.
%and  $G = 4 \sqrt{\log(2k /\delta)} \sbr{(L + \kappa_3) \norm{\theta_0 - \theta^{\pi}}^2 + (c_1 + \kappa_2  \varepsilon_{\mathrm{approx}}) \norm{\theta_0 - \theta^{\pi}}} + c_1 + 1$.
Let us denote the event corresponding to \eqref{aux_seq_accumulation_bounded} by $\cG$.
%Clearly, from the definition of $\cbr{\cG_t}$ and \eqref{aux_seq_accumulation_bounded} we obtain $\cap_{t \leq T} \cG_t \subseteq \cG$.
Our next goal is to show that $Y_t = \tilde{Y}_t$ over $\cG$ for every $t \leq T$, and consequently 
\begin{align}\label{noise_accumulation_bound}
\abs{Y_t} \leq G \sqrt{t}, ~ \forall t \leq T, 
\end{align}
with probability $1-\delta$.
We proceed with an inductive argument. 
The claim trivially holds at $t = 0$.
Suppose the claim holds at iteration $t \geq 0$, then for any $\omega \in \cG$, 
\begin{align}
Y_{t+1}(\omega) & = Y_t(\omega) + X_t(\omega) \nonumber \\
& = Y_t(\omega) + X_t(\omega) \mathbbm{1}_{\cG}(\omega) \nonumber \\
& \overset{(a)}{=} Y_t(\omega) + X_t(\omega) \mathbbm{1}_{\cbr{\tilde{Y}_t \leq G \sqrt{t}}}(\omega) \nonumber \\
& \overset{(b)}{=} Y_t(\omega) + X_t(\omega) \mathbbm{1}_{\cbr{Y_t \leq G \sqrt{t}}}(\omega) \nonumber \\
& \overset{(c)}{=} Y_t(\omega) + \tilde{X}_t(\omega) \nonumber \\
& = \tilde{Y}_{t+1}(\omega), \label{equiv_induction_step_1}
\end{align}
where $(a)$ follows from the definition of $\cG$ and \eqref{aux_seq_accumulation_bounded}, 
 $(b)$ follows from the induction hypothesis that $Y_t (\omega) = \tilde{Y}_t(\omega)$ for $\omega \in \cG$,
and $(c)$ applies the definition of $\tilde{X}_t$.
Hence \eqref{equiv_induction_step_1} completes the induction step.
%On the other hand, suppose the claim holds for all $t \leq T$ of epoch $e$, then for any $\omega \in \cG$,
%\begin{align*}
%Y_{0}^{(e+1)}(\omega) & = Y_{T}(\omega) = \tilde{Y}_{T}(\omega)  = \tilde{Y}^{(e+1)}_{0}(\omega).
%\end{align*}
%\begin{align*}
%Y_{0}^{(e+1)}(\omega) & = Y_{T}(\omega) \\
%& = Y_{T-1}(\omega) + X_{T-1}(\omega) \mathbbm{1}_{\cG}(\omega) \\
%& = Y_{T-1}(\omega) + X_{T-1}(\omega) \mathbbm{1}_{\cbr{\tilde{Y}_{T-1} \leq G \sqrt{T-1 + e T}}}(\omega) \\
%& =Y_{T-1}(\omega) + X_{T-1}(\omega) \mathbbm{1}_{\cbr{Y_{T-1} \leq G \sqrt{T-1 + e T}}}(\omega) \\
%& = Y_{T-1}(\omega) + \tilde{X}_{T-1}(\omega) \\
%& = \tilde{Y}^{(e+1)}_{0}(\omega),
%\end{align*}
%Combining \eqref{equiv_induction_step_1} and the above relation completes the induction step.

In view of \eqref{norm_bound_on_good_event_generic} and \eqref{noise_accumulation_bound}, we conclude that
% for $(\alpha, \tau, G)$ specified  in  \eqref{param_alpha_tau} and \eqref{param_choice_G}, 
\begin{align*}
\norm{\theta_t - \theta^{\pi}}^2
\leq \norm{\theta_0 - \theta^{\pi}}^2 + 2 \alpha^2 c_1^2  + 2 \alpha G,
\end{align*}
with probability at least $1-\delta$.
The desired claim follows immediately by noting that $\alpha \leq \min\cbr{\frac{1}{2 c_1}, ~
\frac{1}{4 G} }$.
\end{proof}
%With \eqref{xx}, \eqref{xx} and \eqref{xx} in place, we are now ready to complete the proof.
%Clearly, with an inductive argument, one can readily show that with 
%\begin{align*}
%& \sbr{ c_1 + \kappa_2  \varepsilon_{\mathrm{approx}} + L + \kappa_3} \sqrt{\alpha} \leq \frac{1}{8 \sqrt{\log(2T/\delta)}}, ~
%\alpha c_1 \leq 1, 
%~ \tau = \cO(t_{\mathrm{mix}}),  \\
%& G \geq 4 \sqrt{\log(2k /\delta)} \big[(L + \kappa_3) (e + 1) \norm{\theta_0 - \theta^{\pi}}^2 + (c_1 + \kappa_2  \varepsilon_{\mathrm{approx}}) \sqrt{e + 1} \norm{\theta_0 - \theta^{\pi}}\big] + c_1 + 1, \\
%& 
% 2 \alpha^2 c_1^2  + 2 \alpha G \leq \norm{\theta_0 - \theta^{\pi}}^2.
%\end{align*}






By combining Lemma \ref{lemma_bias_linear} and \ref{lemma_norm_bound_high_prob}, we proceed to establish that SLPE with proper parameter specification yields fast bias reduction while controlling boundedness of the estimated value function $\hat{\cV}^\pi$. 


\begin{lemma}\label{lemma_high_prob_norm_and_bias}
Fix total iterations $T > 0$ a priori in SLPE.
For any $\delta \in (0,1)$ and any $\varepsilon \geq 2 \varepsilon_{\mathrm{approx}}$, let the parameters in SLPE be chosen as 
\begin{align*}
\eta_t = \alpha /\sqrt{T}, 
%~ T = \frac{2 \log^2 \rbr{4{\norm{\theta_0 -\theta^{\pi}}}/{\varepsilon}}}{\alpha^2 \mu^2}, 
\end{align*} 
with 
$
\alpha  \leq 
\min \big\{
\frac{\mu }{L^2 + 32}, 
%\frac{\mu \sqrt{T}}{\kappa_3^2}, ~
\frac{1}{192 \sbr{L^2 + c_1^2 + 16} \log(2T / \delta)},
\frac{1}{2 c_1}, 
\frac{1}{4 G}
\big\},
%\\
%G & \geq 4 \sqrt{\log(2T /\delta)} \big[(L + \kappa_3) \norm{\theta_0 - \theta^{\pi}}^2 + (c_1 + \kappa_2  \varepsilon_{\mathrm{approx}}) \norm{\theta_0 - \theta^{\pi}}\big] + c_1. 
$
and $G$ defined in \eqref{G_bounded_norm_high_prob}.
Then the number of iterations required by SLPE to output 
\begin{align}\label{ctd_bias_bd_with_norm_bd}
\norm{\EE \sbr{ \cV^{\pi}_{\theta_t}} - \cV^{\pi} }_\infty \leq  \varepsilon
\end{align}
  is bounded by
\begin{align}\label{ctd_num_iter_bias_and_norm}
T = \cO \rbr{
 \frac{\log^2 \rbr{{ \norm{\theta_0 -\theta^{\pi}}}/{\varepsilon}}}{\alpha^2 \mu^2} 
 }.
%\end{align}
%The number of samples can be bounded by 
%\begin{align}
%\label{ctd_num_samples_bias_and_norm}
% \cO \rbr{
%\frac{t_{\mathrm{mix}}}{\alpha^2 \mu^2} \log^2 \rbr{\frac{\norm{\theta_0 -\theta^{\pi}}}{ \varepsilon}}
%}.
\end{align}
In addition, we have 
\begin{align}\label{ctd_norm_bound_bias_and_norm}
\norm{\cV^{\pi}_{\theta_t} -  \cV^{\pi} }_\infty
\leq  \norm{\theta_0 - \theta^{\pi}} + 1 + \varepsilon_{\mathrm{approx}}, ~ \forall t \leq T ,
\end{align}
with probability at least $1-\delta$.
\end{lemma}

\begin{proof}
Clearly,  $\eta = \alpha/\sqrt{T}$ with the choice of specified $\alpha$ satisfy 
 \eqref{param_choice_bias_reduction}.
 Consequently, one can  apply \eqref{bound_on_value_bias} in Lemma \ref{lemma_bias_linear}, and obtain that for any $\varepsilon \geq 2 \varepsilon_{\mathrm{approx}}$, 
SLPE outputs 
$
\norm{\EE \sbr{ \cV^{\pi}_{\theta_t}} - \cV^{\pi}}_\infty \leq \varepsilon
$
in 
\begin{align*}
T =\cO \rbr{\frac{1}{\eta \mu} \log \rbr{\frac{\norm{\theta_0 - \theta^*}}{\varepsilon}}}
\end{align*} 
steps. Combining the above relation with the definition of $\eta = \alpha /\sqrt{T}$ implies \eqref{ctd_num_iter_bias_and_norm}.
%\eqref{ctd_num_samples_bias_and_norm} then follows from \eqref{ctd_num_iter_bias_and_norm} and the choice of $\tau = \tilde{\cO}(t_{\mathrm{mix}})$. 
%The total number of samples consumed by SLPE is bounded by 
%\begin{align*}
%T \cdot \tau 
%= \cO \rbr{
%\frac{t_{\mathrm{mix}}}{\alpha^2 \mu^2} \log^2 \rbr{\frac{\norm{\theta_0 -\theta^{\pi}}}{\varepsilon}}
%}.
%\end{align*}
Finally, \eqref{ctd_norm_bound_bias_and_norm} follows  from 
 \begin{align*}
  |\cV^{\pi}_{\theta_t} (s) - \cV^{\pi}  |_\infty \leq \norm{\theta_t - \theta^*} +  \varepsilon_{\mathrm{approx}} \leq  \norm{\theta_0 - \theta^{\pi}} + 1 + \varepsilon_{\mathrm{approx}}, ~ \forall s \in \cS,
 \end{align*}
 where the last inequality applies Lemma \ref{lemma_norm_bound_high_prob}.
 The proof is then completed.
\end{proof}

%
%\begin{remark}
%One can also readily employ Lemma \ref{lemma_ctd_recursion} and \ref{lemma_norm_bound_high_prob} to establish the convergence of $\EE \sbr{\norm{\overline{\theta}_t -\theta^{\pi}}^2} $ with high probability, for proper defined ergodic iterate $\overline{\theta}_t$.
%We omit its explicit discussion to keep the scope of the manuscript concise. 
%\yan{remark on extension to least square to improve the dependence on visitation measure, with simplified analysis. also this can exploits the generator}
%\end{remark}

With Lemma \ref{lemma_bias_linear} and \ref{lemma_norm_bound_high_prob} in place, we can now establish the sample complexity of SFRPE with the SLPE operator 
that outputs an $\epsilon$-estimator of the robust value function
with high probability.




\begin{theorem}\label{thrm_sample_slpe_high_prob}
Fix total iterations $k > 0$ in SFRPE and $\delta \in (0,1)$.
For any $\epsilon \geq \frac{8 \varepsilon_{\mathrm{approx}} }{ (1-\gamma)}$, 
 let SFRPE be instantiated with the SLPE operator with evaluation parameters 
\begin{align*}
\eta_t = \alpha /\sqrt{T},  ~
\theta_0 = 0, 
~ T = \cO \rbr{
 \frac{\log^2 \rbr{{ r_\Theta}/{\epsilon}}}{\alpha^2 \mu^2} 
 }, 
\end{align*} 
where 
\begin{align*}
\alpha  & \leq 
\min \big\{
\frac{\mu }{L^2 + 32}, ~
%\frac{\mu \sqrt{T}}{\kappa_3^2}, ~
\frac{1}{192 \sbr{L^2 + c_1^2 + 16} \log(12 Tk/ \delta)},~
\frac{1}{2 c_1}, ~
\frac{1}{4 G}
\big\}
\\
G&   = 4 \sqrt{\log(12 Tk/\delta)} \big[(L + 4) r_{\Theta}^2 + c_1   r_\Theta \big] + c_1,
\end{align*}
and optimization parameters specified as
\begin{align*}
\beta_t = t^{1/2},   ~ \lambda_t =  \frac{ (t+1) M \gamma \zeta }{2  \sqrt{\mu \overline{w}}}, ~\forall t \leq k,
\end{align*}
where $M \geq r_\Theta + 1 + \varepsilon_{\mathrm{approx}} + \frac{1}{1-\gamma}$.
Then with probability at least $1-\delta$ we have 
\begin{align}
& -\frac{\epsilon}{4} -  \frac{4M}{\sqrt{k}} \sqrt{\log (\frac{6 (k+1)\abs{\cS}}{\delta})} \\
  \leq &  \tsum_{t=1}^k \theta_t 
\cV^{\pi_t}(s) - \cV^*(s) \nonumber \\
 \leq & \frac{4 \gamma \zeta M \sqrt{\overline{w}}}{(1-\gamma) \sqrt{\mu k}}  + \frac{3 \epsilon }{4}
+ \frac{8 \gamma \zeta M}{(1-\gamma) \sqrt{k}} \sqrt{\log (\frac{6 (k+1)\abs{\cS}}{\delta})}
+ \frac{4M}{\sqrt{k}} \sqrt{\log (\frac{6(k+1) \abs{\cS}}{\delta})}, ~ \forall s \in \cS,
 \label{high_prob_err_bound_ctd}
\end{align}
The total number of samples required by SFRPE to output $-  \epsilon \leq \tsum_{t=1}^k \theta_t
\cV^{\pi_t}(s) - \cV^{\pi^*}(s) \leq  \epsilon$ for all $s \in \cS$ with at least probability $1-\delta$  is bounded by 
\begin{align}\label{ctd_sample_high_prob}
\tilde{\cO} \rbr{
\frac{1}{\alpha^2 \mu^2}
\rbr{
\frac{\gamma^2 \zeta^2 M^2 \overline{w}}{(1-\gamma)^2 \mu_w \epsilon^2}
+ \frac{M^2}{\epsilon^2}
}
 \log^2 \rbr{\frac{r_\Theta}{\epsilon}}
}.
\end{align}
In particular, when the distance generating function $w_s(\cdot)$ is set as in \eqref{dgf_negative_entropy}, the total number of samples required can be bounded by 
\begin{align*}
\tilde{\cO} \rbr{
\frac{1}{\alpha^2 \mu^2}
\rbr{
\frac{\gamma^2 \zeta^2 M^2 \log \abs{\cS}}{(1-\gamma)^2  \epsilon^2}
+ \frac{M^2}{\epsilon^2}
}
 \log^2 \rbr{\frac{r_\Theta}{\epsilon}}
}.
\end{align*}
\end{theorem}

\begin{proof}
The essential argument is similar to that of Theorem \ref{thrm_stoch_se_high_prob}, but we will use Lemma \ref{lemma_high_prob_norm_and_bias} instead of Proposition \ref{thrm_stoch_generic_convergence_expectation}.
%Denote $M = r_\Theta + 1 + \varepsilon_{\mathrm{approx}}$.
Clearly the choice of parameters satisfies conditions of Lemma \ref{lemma_high_prob_norm_and_bias}, and hence
\begin{align}\label{norm_bound_every_iter_whp}
\norm{\cV^{\pi_t}_{\theta_t}  }_\infty \leq M, ~
\norm{\EE \sbr{ \cV^{\pi_t}_{\theta_t}} - \cV^{\pi_t}}_\infty \leq   \frac{(1-\gamma) \epsilon}{4},  ~ \forall t \leq k, 
\end{align}
 with probability $1 -  \delta / 3$.
Combining the above relation with Proposition 34 of \cite{tao2015random} yields
\begin{align}\label{ctd_noise_accumulation_1}
\abs{ \tsum_{t=1}^k \theta_t \rbr{\cV^{\pi_t}(s) - \cV^{\pi_t}(s)}  } 
  \leq 
\frac{(1-\gamma) \epsilon}{4} 
+  M \sqrt{2 \tsum_{t=1}^k \theta_t^2 \log(\frac{2}{\delta})}  \leq \frac{(1-\gamma) \epsilon}{4} + \frac{4 M }{\sqrt{k}} \sqrt{\log(\frac{2 }{\delta})}, 
\end{align}
with probability at least $1 - \delta / 6$, for every $s\in \cS$. Further applying union bound yields 
\begin{align*}
\abs{ \tsum_{t=1}^k \theta_t \rbr{\cV^{\pi_t}(s) - \cV^{\pi_t}(s)}  } 
  \leq \frac{(1-\gamma) \epsilon}{4} + \frac{4 M }{\sqrt{k}} \sqrt{\log(\frac{2 \abs{\cS} }{\delta})}, ~ \forall s \in \cS, 
\end{align*}
with probability at least $1 -  \delta/3$.
Applying the same treatment, one can also show that 
\begin{align}\label{ctd_noise_accumulation_2}
\tsum_{t=1}^k \frac{\theta_t}{1-\gamma}  \EE_{s' \sim d_{s}^{\pi^*}} \sbr{ \delta_t(s', \pi^*(s'))}
 & \leq 
 \frac{\gamma \zeta}{1-\gamma} 
\rbr{
 \frac{(1-\gamma) \epsilon}{2}
+ \frac{8 M }{ \sqrt{k}} \sqrt{  \log(\frac{2 \abs{\cS} }{\delta})}
}, ~ \forall s \in \cS,
\end{align}
with probability at least $1 - \delta/3$. 
By plugging \eqref{norm_bound_every_iter_whp},  \eqref{ctd_noise_accumulation_1} and \eqref{ctd_noise_accumulation_2} into Lemma \ref{lemma_generic_prop_stoch}, we obtain
\begin{align*}
-\frac{\epsilon}{4} -  \frac{4M}{\sqrt{k}} \sqrt{\log (\frac{\abs{2 \cS}}{\delta})}
&  \leq \tsum_{t=1}^k \theta_t 
\cV^{\pi_t}(s) - \cV^{\pi^*}(s) \nonumber \\
& \leq \rbr{\tsum_{t=1}^k \beta_t}^{-1}  \tsum_{t=1}^k \frac{\beta_t^2 \gamma^2 \zeta^2  M^2}{2 \mu \lambda_{t-1} (1-\gamma)}
+  \rbr{\tsum_{t=1}^k \beta_t}^{-1} \frac{\lambda_k \overline{w}}{1-\gamma} \nonumber \\
& ~~~ +  \frac{3 \epsilon }{4}
+ \frac{8 \gamma \zeta M}{(1-\gamma) \sqrt{k}} \sqrt{\log (\frac{2 \abs{\cS}}{\delta})}
+ \frac{4M}{\sqrt{k}} \sqrt{\log (\frac{2 \abs{\cS}}{\delta})}, ~ \forall s \in \cS,
\end{align*}
with probability $1-  \delta $. 
Plugging the choice of $\cbr{(\beta_t, \lambda_t)}$ into the above relation yields \eqref{high_prob_err_bound_ctd}.
%\begin{align*}
%& -\epsilon -  \frac{4M}{\sqrt{k}} \sqrt{\log (\frac{6 (k+1)\abs{\cS}}{\delta})} \\
%  \leq &  \tsum_{t=1}^k \theta_t 
%\cV^{\pi_t}(s) - \cV^{\pi^*}(s) \nonumber \\
% \leq & \frac{4 \gamma \zeta M \sqrt{\overline{w}}}{(1-\gamma) \sqrt{\mu k}}  + \rbr{1 + \frac{2}{1-\gamma}} \epsilon 
%+ \frac{8 \gamma \zeta M}{(1-\gamma) \sqrt{k}} \sqrt{\log (\frac{6 (k+1)\abs{\cS}}{\delta})}
%+ \frac{4M}{\sqrt{k}} \sqrt{\log (\frac{6(k+1) \abs{\cS}}{\delta})}, 
%\end{align*}
%for any $s \in \cS$, with probability at least $1-\delta$.
Given \eqref{high_prob_err_bound_ctd}, the total number of iterations by SFRPE  to output $-  \epsilon \leq \tsum_{t=1}^k \theta_t 
\cV^{\pi_t}(s) - \cV^{\pi^*}(s) \leq  \epsilon$ can be bounded by 
\begin{align*}
k = \tilde{\cO} \rbr{
\frac{\gamma^2 \zeta^2 M^2 \overline{w}}{(1-\gamma)^2 \mu_w \epsilon^2}
+ \frac{M^2}{\epsilon^2}
}.
\end{align*}
We conclude the proof by noting that the total number of samples required is $k T$.
%The total number of samples is bounded by 
%\begin{align*}
%\cO \rbr{
%\frac{t_{\mathrm{mix}}}{\alpha^2 \mu^2}
%\rbr{
%\frac{\gamma^2 \zeta^2 M^2 \overline{w}}{(1-\gamma)^2 \mu_w \epsilon^2}
%+ \frac{M^2}{\epsilon^2}
%}
% \log^2 \rbr{\frac{r_\Theta}{\epsilon}}
%}
%\end{align*}
%Finally, \eqref{ctd_sample_high_prob} follows from the above relation and \eqref{ctd_num_samples_bias_and_norm}.
\end{proof}

In view of Theorem \ref{thrm_sample_slpe_high_prob}, SFRPE method instantiated with LSPE operator attains an $\tilde{\cO}(\zeta^2/\epsilon^2 + 1/\epsilon^2)$ sample complexity to output an $\epsilon$-estimator of the robust value function in high probability. 
Clearly the accuracy certificate of Theorem \ref{thrm_sample_slpe_high_prob} is stated in a stronger sense compared to the expectation statement in Theorem \ref{thrm_lspe_expectation}.
As before, the sample complexity possesses two terms that can be attributed to the price of robustness and standard value function estimation, respectively. 


\vspace{6pt}
%Before we conclude our discussion in this section.
%\begin{remark}[Applications to Stochastic Policy Optimization for Robust MDPs]
{\bf  Applications to Stochastic Policy Optimization for Robust MDPs.}
We conclude this section by briefly demonstrating the application of the developed results in the context of  stochastic policy optimization for robust MDPs. 
Consider solving large-scale robust MDPs with $(\mathrm{s}, \mathrm{a})$-rectangular sets using the stochastic robust policy mirror descent (SRPMD) method in \cite{li2022first}.
If log-linear policy class is employed, and one applies the same feature map for the policy class and the robust state-action value function, then applying SFRPE for learning the robust state-action value function immediately implies an $\tilde{\cO}(1/\epsilon^2)$ sample complexity of SRPMD for finding an $\epsilon$-optimal robust policy (cf. Proposition 5.1, \cite{li2022first}).
In particular, this simple application already yields the first sample complexity of policy gradient methods applied to robust MDPs beyond tabular~settings. 

% Theorem \ref{xx} can be particularly useful when combined with existing first-order stochastic robust policy optimization method \cite{xx}.
%Namely, when log-linear policy class is employed in the outer policy optimization method, Theorem \ref{xx} can be readily incorporated 

%
%
%\begin{theorem}
%For any $\epsilon \geq \frac{8 \varepsilon_{\mathrm{approx}} }{(1-\gamma)}$,
%let SFRPE be instantiated with SLPE operator with  evaluation parameters
%\begin{align*}
%\eta \leq \frac{\mu}{L^2 +  32} , ~ \theta_0 = 0, ~ T = \cO \rbr{ \frac{1}{\eta \mu} \log \rbr{\frac{r_\Theta }{ \epsilon}} }, 
%\end{align*}
% and  optimization parameters
%\begin{align*}
%\beta_n = n^{1/2},   ~ \lambda_n = \frac{ (n+1) M \gamma \zeta }{2  \sqrt{\mu \overline{w}}}, ~\forall n \geq 0, 
%\end{align*}
%where $M^2 = 4 \rbr{r_\Theta^2 + c_1^2  + \varepsilon^2_{\mathrm{approx}}} + {2}/{(1-\gamma)^2}$.
%To find an approximate robust value such that 
%\begin{align*}
%-\epsilon \leq \EE \sbr{\tsum_{n=1}^k \theta_n \cV^{\pi_n}}(s) + V^{\vartheta}_r(s) \leq \epsilon,  ~ s \in \cS,
%\end{align*}
%SFRPE needs at most $k = 1 + \frac{256 \gamma^2 \zeta^2 M^2 \overline{w}}{(1-\gamma)^2 \mu \epsilon^2}$ iterations.
%%where $M^2 = 4 \rbr{r_\Theta^2 + 1  + \varepsilon^2_{\mathrm{approx}}} + {2}/{(1-\gamma)^2}$.
%The total number of samples can be bounded by 
%\begin{align*}
%\cO \rbr{
%\frac{1}{\eta \mu} 
%\rbr{1 + \frac{\gamma^2 \zeta^2 M^2 \overline{w}}{(1-\gamma)^2 \mu_w \epsilon^2}}
%\log \rbr{\frac{r_\Theta }{ \epsilon}}
%} .
%% \cO \rbr{
%%\frac{ L^2 + \kappa_2^2 + \kappa_3^2 + c_1^2}{ \mu^2} 
%%\rbr{1 + \frac{\gamma^2 \zeta^2  M^2 \overline{w}}{(1-\gamma)^2 \mu \epsilon^2}}
%%\log \rbr{\frac{r_\Theta }{(1-\gamma) \epsilon}}
%%} .
%\end{align*}
%In particular, when the distance generating function $w(\cdot)$ is set as in \eqref{dgf_negative_entropy}, the number of samples required is bounded by 
%\begin{align*}
%\cO \rbr{
%\frac{1}{\eta \mu} 
%\rbr{1 + \frac{\gamma^2 \zeta^2 M^2 \log \abs{\cS} }{(1-\gamma)^2  \epsilon^2}}
%\log \rbr{\frac{r_\Theta }{\epsilon}}
%} .
%% \cO \rbr{
%%\rbr{ L^2 + \kappa_2^2 + \kappa_3^2 + c_1^2}
%%\rbr{1 + \frac{\gamma^2 \zeta^2  M^2   \log \abs{\cS} }{(1-\gamma)^2  \epsilon^2}}
%%\log \rbr{\frac{r_\Theta }{(1-\gamma) \epsilon}}
%%} .
%\end{align*}
%\end{theorem}
%
%\begin{proof}
%With the choice of parameters one can apply Lemma \ref{lemma_ctd_recursion} and obtain  
%$
%\EE \sbr{ \norm{\theta_n - \theta^\pi}^2} \leq \norm{\theta_0 - \theta^\pi}^2 + c_1^2 \leq r_\Theta^2 + c_1^2
%$,
%and hence 
%\begin{align*}
%%\label{ctd_estimate_expectation_bound_on_norm}
%\EE \sbr{ \norm{\cV^{\pi_n}_{\theta_n} - \cV^{\pi_n}}_\infty^2} 
%\leq 2 \rbr{ \EE \norm{\theta_n - \theta^*}^2   + \varepsilon^2_{\mathrm{approx}}} 
%\leq 2 \rbr{r_\Theta^2 + c_1^2  + \varepsilon^2_{\mathrm{approx}}}.
%\end{align*}
%Consequently, 
%\begin{align*}
%\EE  \sbr{ \norm{\cV^{\pi_n}_{\theta_n}}_\infty^2} \leq 2 \rbr{\EE \sbr{ \norm{\cV^{\pi_n}_{\theta_n} - \cV^{\pi_n}}_\infty^2} + \norm{\cV^\pi}_\infty^2 } \leq  4 \rbr{r_\Theta^2 + c_1^2  + \varepsilon^2_{\mathrm{approx}}} + {2}/{(1-\gamma)^2} .
%\end{align*}
%In addition, from Lemma \ref{lemma_bias_linear}, 
%for any $\epsilon \geq \frac{8 \varepsilon_{\mathrm{approx}}}{1-\gamma} $, 
%taking $T = \cO\rbr{ \frac{1}{\eta \mu} \log \rbr{\frac{r_\Theta }{ \epsilon}}}$ further yields that 
%\eqref{stoch_expecation_conv_bias_condition} is satisfied with 
%\begin{align*}
%\frac{3\varepsilon}{1-\gamma} = \frac{3 \epsilon}{4}, ~ M^2 = 4 \rbr{r_\Theta^2 + c_1^2  + \varepsilon^2_{\mathrm{approx}}} + {2}/{(1-\gamma)^2} .
%\end{align*}
%Combining the above relation and Proposition \ref{thrm_stoch_generic_convergence_expectation}, the total number of iterations required by SFRPE is bounded by 
%$
%k = 1 + \frac{256 \gamma^2 \zeta^2 M^2 \overline{w}}{(1-\gamma)^2 \mu_w \epsilon^2}.
%$
%The total number of samples can be bounded by 
%\begin{align*}
%T \cdot k  & = \cO \rbr{
%\frac{1}{\eta \mu} 
%\rbr{1 + \frac{\gamma^2 \zeta^2 M^2 \overline{w}}{(1-\gamma)^2 \mu_w \epsilon^2}}
%\log \rbr{\frac{r_\Theta }{\epsilon}}
%} .
%% \\
%%& = \cO \rbr{
%%\frac{ L^2 + \kappa_2^2 + \kappa_3^2 + c_1^2}{ \mu^2} 
%%\rbr{1 + \frac{\gamma^2 \zeta^2  M^2 \overline{w}}{(1-\gamma)^2 \mu_w \epsilon^2}}
%%\log \rbr{\frac{r_\Theta }{(1-\gamma) \epsilon}}
%%} .
%\end{align*}
%The proof is then completed.
%%We conclude the proof by noting  taking $\eta = \frac{\mu}{16(L^2 + \kappa_2^2 + \kappa_3^2 + c_1^2)}$ suffices to satisfy the condition \eqref{xx} of Lemma \ref{lemma_bias_linear}
%\end{proof}
%







%
%\newpage 
%
%
%\begin{lemma}
%sdfs
%\end{lemma}
%
%For any $0 \leq e \leq E$, we will denote $\cF$ the $\sigma$-algebra up to (excluding) epoch $e$.
%Going forward let us consider any fixed epoch $0 \leq e \leq E$.
%
%Going forward let us fix the epoch length $T > 0$,
%and assume $\eta_t = \eta \coloneqq \alpha / \sqrt{T}$ for some $\alpha > 0$ to be determined later. 
%We begin by noting that 
%\begin{align*}
%\delta_t &= \Psi^\top \hat{M} (\Psi\theta_t - \gamma \hat{\mathtt{P}}^\pi \theta_t - \mathfrak{C}) - 
%\Psi^\top {M} (\Psi\theta_t - \gamma {\mathtt{P}}^\pi \theta_t - \mathfrak{C}) \\
%& = \Psi^\top \hat{M} (I - \gamma \hat{\mathtt{P}}^\pi) \Psi (\theta_t - \theta^{\pi})
%- \Psi^\top {M} (I - \gamma {\mathtt{P}}^\pi) \Psi (\theta_t - \theta^{\pi}) \\
%& ~~~~~~ +  \Psi^\top \hat{M} (\Psi\theta^{\pi} - \gamma \hat{\mathtt{P}}^\pi \theta^{\pi} - \mathfrak{C}) 
%- \Psi^\top M (\Psi \theta^{\pi} - \gamma {\mathtt{P}}^\pi \Psi \theta^{\pi} - \mathfrak{C}) \\
%& = \Psi^\top \hat{M} (I - \gamma \hat{\mathtt{P}}^\pi) \Psi (\theta_t - \theta^{\pi})
%- \Psi^\top {M} (I - \gamma {\mathtt{P}}^\pi) \Psi (\theta_t - \theta^{\pi}) \\
%& ~~~~~~ +  \Psi^\top \hat{M} (\Psi\theta^{\pi} - \gamma \hat{\mathtt{P}}^\pi \theta^{\pi} - \mathfrak{C}) 
%- \Psi^\top M (I - \gamma {\mathtt{P}}^\pi) (\Psi \theta^{\pi} - \cV^{\pi}).
%\end{align*}
%Consequently, it is clear that  
%\begin{align}\label{eq_norm_bound_with_noise_version}
%\inner{\delta_t}{\theta_t - \theta^{\pi}}
%\leq (L + \kappa_3) \norm{\theta_t - \theta^{\pi}}^2 + (c_1 + \kappa_2 \varepsilon_{\mathrm{approx}}) \norm{\theta_t - \theta^{\pi}}.
%\end{align}
%In addition, note that \eqref{norm_recursion_with_noise}  still holds,  hence with 
%\begin{align*}
% \eta \leq \frac{\mu}{\kappa_3^2}, 
%\end{align*}
%we obtain 
%\begin{align}\label{recursion_on_norm_const_stepsize}
%\norm{\theta_{t+1} - \theta^{\pi}}^2 
%\leq  \norm{\theta_t - \theta^{\pi}}^2
%+ 2\eta^2 c_1^2 +  2 \eta \inner{\delta_t}{\theta_t - \theta^{\pi}}.
%\end{align}
%
%Let us define random sequences $\cbr{X_t \coloneqq \inner{\delta_t}{\theta_t - \theta^{\pi}}}$, $\cbr{X_t' \coloneqq \inner{\delta_t}{\theta_t - \theta^{\pi}} \mathbbm{1}_{\cG_t}}$,
%together with 
%\begin{align*}
%Y_t = \tsum_{i=0}^{t-1} X_i, ~ Y_t' = \tsum_{i=0}^{t-1} X_i', ~ Y_0 \equiv Y_0' \equiv 0.
%\end{align*}
%Now consider event $\cG_t = \cbr{Y_t \leq G \sqrt{t}}$ for some $G > 0$.
%%Now consider event 
%%$\cG_t = \cbr{\tsum_{i=0}^{t-1} \inner{\delta_i}{\theta_i - \theta^{\pi}} \leq G \sqrt{t}}$ for some $G > 0$.
%Then over $\cG_t$, 
%taking the telescopic sum of \eqref{recursion_on_norm_const_stepsize}  yields that for any $t \leq T$, 
%\begin{align}
%\norm{\theta_t - \theta^{\pi}}^2 & \leq \norm{\theta_0 - \theta^{\pi}}^2 + 2 t \eta^2 c_1^2 + 2 \eta G \sqrt{t} \nonumber \\
%& \leq  \norm{\theta_0 - \theta^{\pi}}^2 + 2 \alpha^2 c_1^2  + 2 \alpha G  \coloneqq M_{(\alpha, G)}.
%\label{norm_bound_on_good_event_generic}
%\end{align}
%We proceed to establish that \eqref{norm_bound_on_good_event_generic} indeed holds with high probability given proper choice of $(\alpha, \tau, G)$.
%
%%Let us define random sequences $\cbr{X_t \coloneqq \inner{\delta_t}{\theta_t - \theta^{\pi}}}$, $\cbr{X_t' \coloneqq \inner{\delta_t}{\theta_t - \theta^{\pi}} \mathbbm{1}_{\cG_t}}$,
%%together with 
%%\begin{align*}
%%Y_t = \tsum_{i=0}^{t-1} X_i, ~ Y_t' = \tsum_{i=0}^{t-1} X_i', ~ Y_0 \equiv Y_0' \equiv 0.
%%\end{align*}
%%Then by definition we have $\cG_t = \cbr{Y_t \leq G \sqrt{t}}$.
%First, it is  clear that 
%\begin{align}\label{aux_sequence_norm_bound}
%\abs{X_t'} = \abs{\inner{\delta_t}{\theta_t - \theta^{\pi}} \mathbbm{1}_{\cG_t}} \leq (L + \kappa_3) M_{(\alpha, G)} + (c_1 + \kappa_2 \varepsilon_{\mathrm{approx}}) \sqrt{M_{(\alpha, G)}},
%\end{align}
%where the last inequality follows from \eqref{eq_norm_bound_with_noise_version} and \eqref{norm_bound_on_good_event_generic}.
%Moreover, we also have 
%\begin{align}
%\abs{\EE_{|t}\sbr{\inner{\delta_t}{\theta_t - \theta^{\pi}} \mathbbm{1}_{\cG_t}}}
%& =  \abs{\EE_{|t}\sbr{\inner{\delta_t}{\theta_t - \theta^{\pi}}}  \mathbbm{1}_{\cG_t}}  \nonumber \\
%& \overset{(a)}{\leq} m \sbr{ (L + \kappa_2) \rho^\tau \norm{\theta_t - \theta^{\pi}}^2 + \kappa_2 \rho^\tau  \varepsilon^2_{\mathrm{approx}}} \mathbbm{1}_{\cG_t} \nonumber \\
%& \overset{(b)}{\leq} m \sbr{ (L + \kappa_2) \rho^\tau M_{(\alpha, G)} + \kappa_2 \rho^\tau  \varepsilon^2_{\mathrm{approx}}} \label{aux_sequence_conditional_expectation}
%\end{align}
%where $(a)$ follows from \eqref{ctd_bias_expectation}, and $(b)$ follows from \eqref{norm_bound_on_good_event_generic}.
%In view of \eqref{aux_sequence_norm_bound} and \eqref{aux_sequence_conditional_expectation}, we can now apply Azuma–Hoeffding inequality to $\cbr{Y_t'}$ and obtain that conditioned on $\cF$, 
%\begin{align}
%\abs{
%Y_t'
%}
%& \leq t m \sbr{ (L + \kappa_2) \rho^\tau M_{(\alpha, G)} + \kappa_2 \rho^\tau  \varepsilon^2_{\mathrm{approx}}} \nonumber \\
%& ~~~~~~ + \sqrt{t}  \big[ (L + \kappa_3) M_{(\alpha, G)} + (c_1 + \kappa_2 \varepsilon_{\mathrm{approx}}) \sqrt{M_{(\alpha, G)}} \big]\sqrt{\log(2T / \delta)}, 
%~ \forall t \leq T.
%\label{aux_sequence_accumulation_raw}
%\end{align}
%with probability $1-\delta$.
%Through direct computation, one can verify that with
%\begin{align}
%& \sbr{ c_1 + \kappa_2  \varepsilon_{\mathrm{approx}} + L + \kappa_3} \sqrt{\alpha} \leq \frac{1}{8 \sqrt{\log(2T/\delta)}}, ~
%\alpha c_1 \leq 1, 
%~ \tau = \cO(t_{\mathrm{mix}}), \label{param_alpha_tau} \\
%& G \geq 4 \sqrt{\log(2T /\delta)} \sbr{(L + \kappa_3) \norm{\theta_0 - \theta^{\pi}}^2 + (c_1 + \kappa_2  \varepsilon_{\mathrm{approx}}) \norm{\theta_0 - \theta^{\pi}}} + c_1 + 1, \label{param_choice_G}
%\end{align}
% \eqref{aux_sequence_accumulation_raw}, combined with the definition of $M_{(\alpha, G)}$ implies that conditioned on $\cF$,
%\begin{align}\label{aux_seq_accumulation_bounded}
%\abs{Y_t'} \leq G \sqrt{t}, ~ \forall t \leq T,
%\end{align}
%with probability $1-\delta$.
%%and  $G = 4 \sqrt{\log(2k /\delta)} \sbr{(L + \kappa_3) \norm{\theta_0 - \theta^{\pi}}^2 + (c_1 + \kappa_2  \varepsilon_{\mathrm{approx}}) \norm{\theta_0 - \theta^{\pi}}} + c_1 + 1$.
%Let us denote the event corresponding to \eqref{aux_seq_accumulation_bounded} by $\cG$.
%%Clearly, from the definition of $\cbr{\cG_t}$ and \eqref{aux_seq_accumulation_bounded} we obtain $\cap_{t \leq T} \cG_t \subseteq \cG$.
%Our next goal is to show that $Y_t = Y_t'$ over $\cG$ for every $t \leq T$, and consequently 
%\begin{align}\label{noise_accumulation_bound}
%\abs{Y_t} \leq G \sqrt{t}, ~ \forall t \leq T,
%\end{align}
%with probability $1-\delta$, conditioned on $\cF$.
%We proceed with an inductive argument. 
%The claim trivially holds at $t = 0$.
%Suppose the claim holds at $t$, then for any $\omega \in \cG$, 
%\begin{align*}
%Y_{t+1}(\omega) & = Y_t(\omega) + X_t(\omega) \\
%& = Y_t(\omega) + X_t(\omega) \mathbbm{1}_{\cG}(\omega) \\
%& \overset{(c)}{=} Y_t(\omega) + X_t(\omega) \mathbbm{1}_{\cbr{Y_t' \leq G \sqrt{t}}}(\omega) \\
%& \overset{(d)}{=} Y_t(\omega) + X_t(\omega) \mathbbm{1}_{\cbr{Y_t \leq G \sqrt{t}}}(\omega) \\
%& = Y_t(\omega) + X_t'(\omega) \\
%& = Y_{t+1}'(\omega),
%\end{align*}
%where $(c)$ follows from the definition of $\cG$ and \eqref{aux_seq_accumulation_bounded}, 
%and $(d)$ follows from the induction hypothesis that $Y_t (\omega) = Y_t'(\omega)$ for $\omega \in \cG$.
%In view of \eqref{norm_bound_on_good_event_generic}, \eqref{noise_accumulation_bound} and the definition of $\cbr{Y_t}$, we conclude that
%conditioned on $\cF$,
%% for $(\alpha, \tau, G)$ specified  in  \eqref{param_alpha_tau} and \eqref{param_choice_G}, 
%\begin{align*}
%\norm{\theta_t - \theta^{\pi}}^2
%\leq \norm{\theta_0 - \theta^{\pi}}^2 + 2 \alpha^2 c_1^2  + 2 \alpha G,
%\end{align*}
%with probability at least $1-\delta$.
%
%With \eqref{xx}, \eqref{xx} and \eqref{xx} in place, we are now ready to complete the proof.
%Clearly, with an inductive argument, one can readily show that with 
%\begin{align*}
%& \sbr{ c_1 + \kappa_2  \varepsilon_{\mathrm{approx}} + L + \kappa_3} \sqrt{\alpha} \leq \frac{1}{8 \sqrt{\log(2T/\delta)}}, ~
%\alpha c_1 \leq 1, 
%~ \tau = \cO(t_{\mathrm{mix}}),  \\
%& G \geq 4 \sqrt{\log(2k /\delta)} \big[(L + \kappa_3) (e + 1) \norm{\theta_0 - \theta^{\pi}}^2 + (c_1 + \kappa_2  \varepsilon_{\mathrm{approx}}) \sqrt{e + 1} \norm{\theta_0 - \theta^{\pi}}\big] + c_1 + 1, \\
%& 
% 2 \alpha^2 c_1^2  + 2 \alpha G \leq \norm{\theta_0 - \theta^{\pi}}^2.
%\end{align*}
%
%












\section{Conclusion and Future Work}
In this work, I design corruption-robust algorithms for the Lipschitz contextual search problem. I present the \emph{agnostic checking} technique and demonstrate its effectiveness in designing corruption-robust algorithms. There are several open problems for future research. First, in the algorithm I propose for pricing loss, the schedule for agnostic checks is fixed upfront. Can the learner design an adaptive checking schedule for the pricing loss? Second, this work assumes the learner has knowledge of the Lipschitz constant $L$. Can the learner design efficient no-regret algorithms without knowledge of $L$? 


%\newpage
\bibliographystyle{plain}
\bibliography{references}


\end{document}
