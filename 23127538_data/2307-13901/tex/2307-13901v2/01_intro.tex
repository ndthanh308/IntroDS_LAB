\section{Introduction}
\label{sec:intro}
% To insert a figure: \def\bC{{\beta_{_{C}}}}
\def\nL{{\mathcal{L}}}
\def\nW{{\mathcal{W}}}

%  LaTeX support: latex@mdpi.com 
%  For support, please attach all files needed for compiling as well as the log file, and specify your operating system, LaTeX version, and LaTeX editor.

%=================================================================
\documentclass[journal,article,submit,pdftex,moreauthors]{Definitions/mdpi} 
\renewcommand{\linenumbers}{}
%--------------------
% Class Options:
%--------------------
%----------
% journal
%----------
% Choose between the following MDPI journals:
% acoustics, actuators, addictions, admsci, adolescents, aerobiology, aerospace, agriculture, agriengineering, agrochemicals, agronomy, ai, air, algorithms, allergies, alloys, analytica, analytics, anatomia, animals, antibiotics, antibodies, antioxidants, applbiosci, appliedchem, appliedmath, applmech, applmicrobiol, applnano, applsci, aquacj, architecture, arm, arthropoda, arts, asc, asi, astronomy, atmosphere, atoms, audiolres, automation, axioms, bacteria, batteries, bdcc, behavsci, beverages, biochem, bioengineering, biologics, biology, biomass, biomechanics, biomed, biomedicines, biomedinformatics, biomimetics, biomolecules, biophysica, biosensors, biotech, birds, bloods, blsf, brainsci, breath, buildings, businesses, cancers, carbon, cardiogenetics, catalysts, cells, ceramics, challenges, chemengineering, chemistry, chemosensors, chemproc, children, chips, cimb, civileng, cleantechnol, climate, clinpract, clockssleep, cmd, coasts, coatings, colloids, colorants, commodities, compounds, computation, computers, condensedmatter, conservation, constrmater, cosmetics, covid, crops, cryptography, crystals, csmf, ctn, curroncol, cyber, dairy, data, ddc, dentistry, dermato, dermatopathology, designs, devices, diabetology, diagnostics, dietetics, digital, disabilities, diseases, diversity, dna, drones, dynamics, earth, ebj, ecologies, econometrics, economies, education, ejihpe, electricity, electrochem, electronicmat, electronics, encyclopedia, endocrines, energies, eng, engproc, entomology, entropy, environments, environsciproc, epidemiologia, epigenomes, est, fermentation, fibers, fintech, fire, fishes, fluids, foods, forecasting, forensicsci, forests, foundations, fractalfract, fuels, future, futureinternet, futurepharmacol, futurephys, futuretransp, galaxies, games, gases, gastroent, gastrointestdisord, gels, genealogy, genes, geographies, geohazards, geomatics, geosciences, geotechnics, geriatrics, grasses, gucdd, hazardousmatters, healthcare, hearts, hemato, hematolrep, heritage, higheredu, highthroughput, histories, horticulturae, hospitals, humanities, humans, hydrobiology, hydrogen, hydrology, hygiene, idr, ijerph, ijfs, ijgi, ijms, ijns, ijpb, ijtm, ijtpp, ime, immuno, informatics, information, infrastructures, inorganics, insects, instruments, inventions, iot, j, jal, jcdd, jcm, jcp, jcs, jcto, jdb, jeta, jfb, jfmk, jimaging, jintelligence, jlpea, jmmp, jmp, jmse, jne, jnt, jof, joitmc, jor, journalmedia, jox, jpm, jrfm, jsan, jtaer, jvd, jzbg, kidneydial, kinasesphosphatases, knowledge, land, languages, laws, life, liquids, literature, livers, logics, logistics, lubricants, lymphatics, machines, macromol, magnetism, magnetochemistry, make, marinedrugs, materials, materproc, mathematics, mca, measurements, medicina, medicines, medsci, membranes, merits, metabolites, metals, meteorology, methane, metrology, micro, microarrays, microbiolres, micromachines, microorganisms, microplastics, minerals, mining, modelling, molbank, molecules, mps, msf, mti, muscles, nanoenergyadv, nanomanufacturing,\gdef\@continuouspages{yes}} nanomaterials, ncrna, ndt, network, neuroglia, neurolint, neurosci, nitrogen, notspecified, %%nri, nursrep, nutraceuticals, nutrients, obesities, oceans, ohbm, onco, %oncopathology, optics, oral, organics, organoids, osteology, oxygen, parasites, parasitologia, particles, pathogens, pathophysiology, pediatrrep, pharmaceuticals, pharmaceutics, pharmacoepidemiology,\gdef\@ISSN{2813-0618}\gdef\@continuous pharmacy, philosophies, photochem, photonics, phycology, physchem, physics, physiologia, plants, plasma, platforms, pollutants, polymers, polysaccharides, poultry, powders, preprints, proceedings, processes, prosthesis, proteomes, psf, psych, psychiatryint, psychoactives, publications, quantumrep, quaternary, qubs, radiation, reactions, receptors, recycling, regeneration, religions, remotesensing, reports, reprodmed, resources, rheumato, risks, robotics, ruminants, safety, sci, scipharm, sclerosis, seeds, sensors, separations, sexes, signals, sinusitis, skins, smartcities, sna, societies, socsci, software, soilsystems, solar, solids, spectroscj, sports, standards, stats, std, stresses, surfaces, surgeries, suschem, sustainability, symmetry, synbio, systems, targets, taxonomy, technologies, telecom, test, textiles, thalassrep, thermo, tomography, tourismhosp, toxics, toxins, transplantology, transportation, traumacare, traumas, tropicalmed, universe, urbansci, uro, vaccines, vehicles, venereology, vetsci, vibration, virtualworlds, viruses, vision, waste, water, wem, wevj, wind, women, world, youth, zoonoticdis 
% For posting an early version of this manuscript as a preprint, you may use "preprints" as the journal. Changing "submit" to "accept" before posting will remove line numbers.

%---------
% article
%---------
% The default type of manuscript is "article", but can be replaced by: 
% abstract, addendum, article, book, bookreview, briefreport, casereport, comment, commentary, communication, conferenceproceedings, correction, conferencereport, entry, expressionofconcern, extendedabstract, datadescriptor, editorial, essay, erratum, hypothesis, interestingimage, obituary, opinion, projectreport, reply, retraction, review, perspective, protocol, shortnote, studyprotocol, systematicreview, supfile, technicalnote, viewpoint, guidelines, registeredreport, tutorial
% supfile = supplementary materials

%----------
% submit
%----------
% The class option "submit" will be changed to "accept" by the Editorial Office when the paper is accepted. This will only make changes to the frontpage (e.g., the logo of the journal will get visible), the headings, and the copyright information. Also, line numbering will be removed. Journal info and pagination for accepted papers will also be assigned by the Editorial Office.

%------------------
% moreauthors
%------------------
% If there is only one author the class option oneauthor should be used. Otherwise use the class option moreauthors.

%---------
% pdftex
%---------
% The option pdftex is for use with pdfLaTeX. Remove "pdftex" for (1) compiling with LaTeX & dvi2pdf (if eps figures are used) or for (2) compiling with XeLaTeX.

%=================================================================
% MDPI internal commands - do not modify
\firstpage{1} 
\makeatletter 
\setcounter{page}{\@firstpage} 
\makeatother
\pubvolume{1}
\issuenum{1}
\articlenumber{0}
\pubyear{2023}
\copyrightyear{2023}
%\externaleditor{Academic Editor: Firstname Lastname}
\datereceived{ } 
\daterevised{ } % Comment out if no revised date
\dateaccepted{ } 
\datepublished{ } 
%\datecorrected{} % For corrected papers: "Corrected: XXX" date in the original paper.
%\dateretracted{} % For corrected papers: "Retracted: XXX" date in the original paper.
\hreflink{https://doi.org/} % If needed use \linebreak
%\doinum{}
%\pdfoutput=1 % Uncommented for upload to arXiv.org

%=================================================================
% Add packages and commands here. The following packages are loaded in our class file: fontenc, inputenc, calc, indentfirst, fancyhdr, graphicx, epstopdf, lastpage, ifthen, float, amsmath, amssymb, lineno, setspace, enumitem, mathpazo, booktabs, titlesec, etoolbox, tabto, xcolor, colortbl, soul, multirow, microtype, tikz, totcount, changepage, attrib, upgreek, array, tabularx, pbox, ragged2e, tocloft, marginnote, marginfix, enotez, amsthm, natbib, hyperref, cleveref, scrextend, url, geometry, newfloat, caption, draftwatermark, seqsplit
% cleveref: load \crefname definitions after \begin{document}

%=================================================================
% Please use the following mathematics environments: Theorem, Lemma, Corollary, Proposition, Characterization, Property, Problem, Example, ExamplesandDefinitions, Hypothesis, Remark, Definition, Notation, Assumption
%% For proofs, please use the proof environment (the amsthm package is loaded by the MDPI class).

%=================================================================
% Full title of the paper (Capitalized)
\Title{Axion field influence on Josephson junction quasipotential}

% MDPI internal command: Title for citation in the left column
\TitleCitation{Axion field influence on Josephson junction quasipotential}

% Author Orchid ID: enter ID or remove command
\newcommand{\orcidauthorA}{0000-0003-2363-7699} 
\newcommand{\orcidauthorB}{0000-0001-5496-1518} 
\newcommand{\orcidauthorC}{0000-0002-6625-3989} 
\newcommand{\orcidauthorD}{0000-0002-4348-9956}
\newcommand{\orcidauthorE}{0000-0003-3546-8618} 
\newcommand{\orcidauthorF}{0000-0002-3683-2509} % Add \orcidA{} behind the author's name
%\newcommand{\orcidauthorB}{0000-0000-0000-000X} % Add \orcidB{} behind the author's name

% Authors, for the paper (add full first names)
\Author{Roberto Grimaudo $^{1}$\orcidA{}, Davide Valenti $^{1}$\orcidB{}, Bernardo Spagnolo $^{1,2}$\orcidC{}, Antonio Troisi $^{3}$\orcidD{}, Giovanni Filatrella $^{3,4}$\orcidE{} and Claudio Guarcello $^{5,4}$\orcidF{}}

%\longauthorlist{yes}

% MDPI internal command: Authors, for metadata in PDF
\AuthorNames{Roberto Grimaudo, Davide Valenti, Bernardo Spagnolo, Antonio Troisi, Giovanni Filatrella and Claudio Guarcello}

% MDPI internal command: Authors, for citation in the left column
\AuthorCitation{Grimaudo R.; Valenti D.; Spagnolo B.; Troisi A.; Filatrella G.; Guarcello C.}
% If this is a Chicago style journal: Lastname, Firstname, Firstname Lastname, and Firstname Lastname.

% Affiliations / Addresses (Add [1] after \address if there is only one affiliation.)
\address{%
$^{1}$ \quad Dipartimento di Fisica e Chimica ``E. Segr\`e'', Group of Theoretical Interdisciplinary Physics, Universit\`a degli Studi di Palermo, Viale delle Scienze, Ed. 18, I-90128 Palermo, Italy;\\
$^{2}$ \quad Lobachevskii University of Nizhnii Novgorod, 23 Gagarin Ave. Nizhnii Novgorod 603950 Russia;\\
$^{3}$ \quad Dep. of Sciences and Technologies, University of Sannio, Via De Sanctis, Benevento I-82100, Italy;\\
$^{4}$ \quad INFN, Sezione di Napoli Gruppo Collegato di Salerno, Complesso Universitario di Monte S. Angelo, I-80126 Napoli, Italy;\\
$^{5}$ \quad Dipartimento di Fisica ``E.R. Caianiello'', Universit\`a di Salerno, Via Giovanni Paolo II, 132, I-84084 Fisciano (SA), Italy;}

% Contact information of the corresponding author
\corres{Correspondence: cguarcello@unisa.it (C.G) giovanni.filatrella@unisannio.it (G.F.)}

% Current address and/or shared authorship
%\firstnote{Current address: Affiliation 3.} 
%\secondnote{These authors contributed equally to this work.}
% The commands \thirdnote{} till \eighthnote{} are available for further notes

%\simplesumm{} % Simple summary

%\conference{} % An extended version of a conference paper

% Abstract (Do not insert blank lines, i.e. \\) 
\abstract{The direct effect of an axion field on Josephson junctions is analyzed through the consequences on the effective potential barrier that prevents the junction from switching from the superconducting to the finite-voltage state. 
We describe a method to reliably compute the quasipotential with stochastic simulations, which allows to span the coupling parameter from weakly interacting axion to tight interactions.
As a result, we obtain that the axion field induces a change in the potential barrier, therefore determining a significant detectable effect for such a kind of elusive particle. }

% Keywords
\keyword{Josephson junction; axion; quasipotential; switching dynamics; noise; detection} 

% The fields PACS, MSC, and JEL may be left empty or commented out if not applicable
%\PACS{J0101}
%\MSC{}
%\JEL{}

%%%%%%%%%%%%%%%%%%%%%%%%%%%%%%%%%%%%%%%%%%
% Only for the journal Diversity
%\LSID{\url{http://}}

%%%%%%%%%%%%%%%%%%%%%%%%%%%%%%%%%%%%%%%%%%
% Only for the journal Applied Sciences
%\featuredapplication{Authors are encouraged to provide a concise description of the specific application or a potential application of the work. This section is not mandatory.}
%%%%%%%%%%%%%%%%%%%%%%%%%%%%%%%%%%%%%%%%%%

%%%%%%%%%%%%%%%%%%%%%%%%%%%%%%%%%%%%%%%%%%
% Only for the journal Data
%\dataset{DOI number or link to the deposited data set if the data set is published separately. If the data set shall be published as a supplement to this paper, this field will be filled by the journal editors. In this case, please submit the data set as a supplement.}
%\datasetlicense{License under which the data set is made available (CC0, CC-BY, CC-BY-SA, CC-BY-NC, etc.)}

%%%%%%%%%%%%%%%%%%%%%%%%%%%%%%%%%%%%%%%%%%
% Only for the journal Toxins
%\keycontribution{The breakthroughs or highlights of the manuscript. Authors can write one or two sentences to describe the most important part of the paper.}

%%%%%%%%%%%%%%%%%%%%%%%%%%%%%%%%%%%%%%%%%%
% Only for the journal Encyclopedia
%\encyclopediadef{For entry manuscripts only: please provide a brief overview of the entry title instead of an abstract.}

%%%%%%%%%%%%%%%%%%%%%%%%%%%%%%%%%%%%%%%%%%
% Only for the journal Advances in Respiratory Medicine
%\addhighlights{yes}
%\renewcommand{\addhighlights}{%

%\noindent This is an obligatory section in “Advances in Respiratory Medicine”, whose goal is to increase the discoverability and readability of the article via search engines and other scholars. Highlights should not be a copy of the abstract, but a simple text allowing the reader to quickly and simplified find out what the article is about and what can be cited from it. Each of these parts should be devoted up to 2~bullet points.\vspace{3pt}\\
%\textbf{What are the main findings?}
% \begin{itemize}[labelsep=2.5mm,topsep=-3pt]
% \item First bullet.
% \item Second bullet.
% \end{itemize}\vspace{3pt}
%\textbf{What is the implication of the main finding?}
% \begin{itemize}[labelsep=2.5mm,topsep=-3pt]
% \item First bullet.
% \item Second bullet.
% \end{itemize}
%}

%%%%%%%%%%%%%%%%%%%%%%%%%%%%%%%%%%%%%%%%%%
\begin{document}

%%%%%%%%%%%%%%%%%%%%%%%%%%%%%%%%%%%%%%%%%%
\section{Introduction}

Nowadays, in the search for cold dark matter candidates, among others, axion particles were theoretically predicted, but their detection remains elusive, for the very weak interaction that they are supposed to have with ordinary matter~\cite{Preskili83}.
Since Josephson junctions (JJs) proved to be very sensitive devices, close to detecting a single photon~\cite{Alesini20}, a natural idea was to exploit them to detect the electromagnetic field produced by axion decay~\cite{Rettaroli21}.
A different possibility, which is the framework for the present paper, is to exploit the direct interaction between JJ and axions~\cite{Bec13} in a detector~\cite{Grimaudo22,Gri23}. This scheme was also proposed as a key to understand unclear ``events'' in Josephson's response~\cite{Bec17,Hof04,Bae08,He11,Gol12,Bre13,Wang22} and would simplify the usual detection schemes through a reduced setting.
To date, the main idea behind this direct interaction is that the axion decays into the junction with a rather high probability, which would be achieved in a resonant cavity through the Primakoff effect only through a huge magnetic field (orders of magnitude above any realistic field): thus, a "Josephson cavity" is much more effective to detect the axion than a resonant cavity~\cite{Bec13} .
This scheme, however, misses a full theory of the Josephson-axion interaction and a detailed scheme of the changes induced in the JJ dynamics, which could possibly lead to detectable consequences. 
We here concentrate on the latter problem, assuming that the axion-JJ interaction exists, although the interaction parameter is unknown.
In the original proposal it was suggested to look at deviations from the locked dynamics of the JJ to an external radio frequency -- the so-called Shapiro steps -- that could be altered by the extra Cooper's pairs created by the axion decay. 
More recently, it has been proposed by some of the authors of the present paper to assume a different standpoint: to bias the JJ in the superconducting metastable state through a dc external drive and to observe the passages to the finite voltage state in the presence of the axion-JJ coupling, under the influence of thermal fluctuations~\cite{Grimaudo22}. 
The idea is that these switches are altered by the interaction of the JJ with the axions, and it is therefore possible to infer the existence of the axions if the switching is, in some statistical sense, changed; a more detailed analysis can also offer an estimate of the JJ-axion coupling from switching time measurements~\cite{Grimaudo22}. A successive approach assumed the JJ operating as a qubit, and in such a way the qubit-axion interaction being detected as axion-induced oscillations of the qubit state~\cite{Gri23}.
The general idea of exploiting a JJ to detect a weak signal, even embedded in a noisy background, is fairly well-established, a JJ being essentially a threshold device operating via a switching mechanism. The presence of a noise background is a condition typical of open systems, such as, for instance, biological and ecological systems~\cite{Val12,Lis15,Valenti16} and financial markets~\cite{Valenti18}, which has to be taken into account in view of better modeling their dynamics. Josephson junctions have been also proposed as noise detectors~\cite{Tob04,Pek04,Ank07,Suk07,Tim07,Hua07,Gra08,Fil10,Add12,Gua13,Gua19,Gua20,Gua21-2} and play a leading role in the search for possible protocols and schemes for the detection of single photons~\cite{Wal17,Kuz18,GuaBra19,GuaBraSol19,Rev20,Yab21,Pied21,Gua21,Pan22-1,Pan22-2}. 

In this paper we further investigate the consequences of a direct interaction between axions and JJ, to show that a certain quantity, namely, the \emph{quasipotential}~\cite{Graham85}, can be introduced for this non-equilibrium system and that it is possible to determine the quasipotential in the presence of the axions.
To investigate the quasipotential is an advantage, as it can be determined with numerical simulations at a relatively high noise intensity, i.e., high temperature. 
Indeed, the quasipotential (as the ordinary potential) is not affected by noise; if the quasipotential is known, it is possible to predict the average escape time at very low temperature with the Arrhenius law, with a considerable saving of simulation time~\cite{Kau88}. 
We do so with a twofold objective: in the first place, as already mentioned, to better understand the consequences that a direct interaction between JJ and axion would have, and therefore to pave the way towards a practical implementation of the device to detect axions; on the second hand, not less important, to compute the quasipotential of the axion-JJ system, a very convenient quantity in the analysis of low temperature devices (as already demonstrated for Shapiro steps~\cite{Kau96} and cavity-induced synchronization~\cite{Pou19}) to infer the properties at a very low noise and, consequently, very long escape times. In fact, the characteristic time scale of the system is of the order of $[1-10]\;\text{ps}$, being the inverse of Josephson characteristic frequency which, as we shall see later, generally falls in the range $[0.1-1]\;\text{THz}$; this means that numerical realizations, even close to real experimental times (which could take up to milliseconds, e.g., for experiments involving switching current distributions), require extremely long simulations. Moreover, these have to be repeated several times in order to obtain complete statistics. Indeed, stochastic analyses, such as the one we propose, require the repetition of the same experiment, i.e., of the same numerical simulation, for a reasonably large number of times under the same conditions, in order to allow for reliable statistical analyses. In conclusion, quasipotential analysis makes it possible to extract useful information at reasonably high temperatures, that means within reasonable simulation times, and then allows to extrapolate relevant information even at low temperatures, where numerical simulations time would become prohibitive. 

The paper is organized as follows: Sect.~\ref{Model} presents the model to describe: the JJ (Sect.~\ref{ModelJJ}), the axion field (Sect.~\ref{Modelaxions}), and the interacting axion-JJ system (Sect.~\ref{ModelJJaxions}). Sect.~\ref{Results} defines the quasipotential for this system and computes its behavior as a function of the interaction. 
Finally, in Sect.~\ref{Conclusions} conclusions are drawn.




\section{Model} 
\label{Model}

In this section we outline the models for the JJ, see Sect.~\ref{ModelJJ}, the axion, see Sect.~\ref{Modelaxions}, and their interaction, see Sect.~\ref{ModelJJaxions}. 
We show the potential of the JJ alone -- a cosinusoidal washboard potential, see Eq.~\eqref{Washboard App} below -- that exhibits an activation energy barrier, whose changes due to the interaction with the axions are the focus of the present work.
Also, some details for the numerical simulations of the stochastic equations are given in Sect.~\ref{ModelJJ}.


\subsection{RCSJ Model}
\label{ModelJJ}

Let us consider the usual model for a superconducting junction, schematically represented in Fig.~\ref{fig: Device}(a), given by the following equations~\cite{Bar82,Lik86}:
\begin{eqnarray}
\label{JJcurrent}
I_\varphi = I_c \sin{\varphi},\\
\label{JJvoltage}
V = \frac{\Phi_0}{2\pi}\frac{d\varphi}{dt},
\end{eqnarray}
where $\Phi_0=h/(2e)$ is the flux quantum, with $e$ and $h$ being the electron charge and the Planck constant, respectively, $I_c$ is the maximum Josephson current that can flow through the device, and $\varphi$ is the Josephson phase difference.

For a real device, one assumes for instance that the two superconductors have lateral dimensions $\nL$ and $\nW$ smaller than the Josephson penetration depth, $\lambda_{_{J}} = \sqrt{\Phi_0/(2\pi \mu_0 t_d J_c)}$ (here, $t_d=\lambda_{L,1}+\lambda_{L,2}+d$ is the effective magnetic thickness, with $\lambda_{L}$ and $d$ being the London penetration depths and the insulating layer thickness, respectively, $\mu_0$ is the vacuum permeability, and $J_c$ is the critical current area density).
The dynamics of the Josephson phase $\varphi$ for a dissipative, current-biased small JJ can thus be studied within the resistively and capacitively shunted junction (RCSJ) framework~\cite{Bar82,GuaVal15,Spa17,McC68,Gua19,Gua20}

%
%
\begin{equation}
\left ( \frac{\Phi_0}{2\pi} \right )^{\!\!2}\!\! C \frac{d^2 \varphi}{d t^2}+\left ( \frac{\Phi_0}{2\pi} \right )^{\!\!2}\!\!\frac{1}{R} \frac{d \varphi}{d t}+\frac{d }{d \varphi}U 
= \left ( \frac{\Phi_0}{2\pi} \right )( I_N+I_b),
\label{RCSJ App}
\end{equation}
%
with $R$ and $C$ the normal-state resistance and capacitance of the JJ, respectively, and $I_N$ and $I_b$ the thermal noise and the bias current, respectively.
The corresponding normalized dynamics can be reformulated (for sinusoidal potential of standard tunnel JJ, albeit other shapes are possible~\cite{Bee92}) through the equation:
%
%
\begin{equation}
\beta_c\frac{d^2 \varphi (\tau_c)}{d\tau_c^2}+ \frac{d \varphi (\tau_c)}{d\tau_c} +\frac{d }{d \varphi}\mathcal{U}(\varphi,i_b) = i_{n}(\tau_c) + i_b,
\label{RCSJnormOc}
\end{equation}
%
%
where time is normalized to the inverse of the characteristic frequency, that is $\tau_c = \omega_c~t$ with $\omega_c=\left ( 2\pi/\Phi_0 \right )I_cR$, $i_b= I_b / I_c$ and $i_n= I_n / I_c$ are the normalized external bias current and thermal noise current, and $\beta_c=\omega_c RC$ is the Stewart-McCumber parameter. 
We stress that the JJ response is usually quite fast, since the characteristic frequency of JJ falls within the range $ [0.1,1]\;\text{THz}$. 
Another way to obtain a dimensionless form of Eq.~\eqref{RCSJ App} consists in normalizing with respect to the plasma frequency $\omega_p=\sqrt{2eI_c/\hbar C}$.
In this case, time is normalized respect to the inverse plasma frequency, i.e., $\tau_p = \omega_p~t$, and the equation in normalized units contains a damping parameter $\alpha=\beta_{_{C}}^{-1/2}$, which multiplies the first time-derivative of the phase. 

The normalized potential, $\mathcal{U}$, is the so-called \emph{washboard potential}, which depends upon the normalized bias current, $i_b$, and the Josephson phase according to
%
% 
\begin{equation}
\mathcal{U}(\varphi,i_b)=\frac{U(\varphi,i_b)}{{E_{J_0}}}=\left [1- \cos(\varphi) -i_b\varphi\right ].
\label{Washboard App}
\end{equation}
%
%
The potential can be expressed in physical units defining the Josephson energy $E_{J_0}=\left ( \Phi_0/2\pi \right )I_c$. 
The resulting activation energy barrier, $\Delta U(i_b)$, confines the phase $\varphi$ in a metastable potential minimum and can be calculated as the difference between the maximum and minimum value of the normalized potential $U(\varphi,i_b)$, see Fig.~\ref{fig: Device}(b).
In units of $E_{J_0}$, it can be expressed as
%
% 
\begin{equation}
{\Delta \mathcal{U}(i_b)}=\frac{\Delta {U}(i_b)}{{E_{J_0}}}=2 \left [ \sqrt{1-i_b^2} -i_b\arccos(i_b)\right ].
\label{activationenergybarrier App}
\end{equation}
%
%
In the phase particle picture, the term $i_b$ represents the tilting of the potential profile; increasing $i_b$ the slope of the washboard increases and the height 
$\Delta \mathcal{U}(i_b)$ of the rightward potential barrier reduces, until this activation energy vanishes altogether for $i_b=1$, that is when the bias current reaches its critical value $I_c$. 
During the motion, different regimes are governed by the Stewart-McCumber parameter $\bC$.
 A highly damped (or overdamped) junction corresponds to $\bC\ll 1$, that is a small capacitance and/or a small resistance. 
 Instead, a junction with $\bC\gg 1$ has a large capacitance and/or a large resistance, and is weakly damped (or underdamped)~\footnote{With the alternative normalized mentioned before, the under- and overdamped regimes correspond to $\alpha \ll 1$ and $\alpha \gg 1$, respectively.}.
 For the purposes of this work, it is important to notice that in the underdamped regime once the phase has passed the barrier, a finite velocity, and hence, as per Eq.~\eqref{JJvoltage}, a finite voltage, appears.
It is therefore possible to detect the passage of the Josephson phase over the barrier through the appearance of a finite voltage, a key point to employ a JJ as a detector. In fact, the phase $\varphi$ itself is not directly accessible, while the passage over the barrier is signaled by a measurable voltage drop across the junction.
The procedure can be briefly schematized as follows. 
The JJ is prepared in some static configuration (at which corresponds a zero net voltage), exposed to some supposedly existing perturbation, and the junction is left to evolve. 
If the signal was not present, the passage only occurs under the effect of thermal noise, and it is given by the usual Kramers law~\cite{Kra40}. The presence of the signal is ascertained through deviations of the thermal escapes~\cite{Fil10,Pie21} -- as will be discussed in more details below.


In this work, the random current is modeled as a delta-correlated Gaussian white noise associated to the normal-state resistance of the junction, $R$, with the usual statistical properties:
\begin{eqnarray}
\label{averageD}
\langle i_n (\tau)\rangle\ &=& 0,\\
\label{sigmaD}
\langle i_n (\tau) i_n (\tau+\tilde{\tau})\rangle &=& 2D\,\delta (\tilde{\tau}). 
\end{eqnarray}
The amplitude of the normalized correlation is connected with the physical temperature $T$ through the relation~\cite{Bar82}
%
\begin{eqnarray}
\label{WNAmp}
D= \frac{k_BT}{R}\frac{\omega_c}{I^2_c},%=\frac{k_BT}{E_{J_0}},
\end{eqnarray}
%
here $k_B$ is the Boltzmann constant.
We note that, by normalizing time with respect to the characteristic frequency $\omega_c$ (as we do in our numerical simulations), the normalized noise intensity in Eq.~\eqref{WNAmp} can be recast as $D={k_BT}/{E_{J_0}}$, i.e, the ratio between the thermal energy and the Josephson coupling energy, $E_{J_0}$, without reference to the damping; instead, normalizing with respect to the plasma frequency, $\omega_p$, the normalized noise intensity becomes $D=\alpha{k_BT}/{E_{J_0}}$.
Thus, for Gaussian fluctuations of amplitude $D$, the stochastic independent increment employed in the numerical simulations reads
$\Delta i_N \simeq \sqrt{ 2 D \Delta t\; }\; N\left(0,1 \right)$.
Here, $N\left(0, 1 \right)$ indicates a Gaussianly distributed random function with zero mean and unit standard deviation. 




%Another way to obtain a dimensionless form of Eq.~\eqref{RCSJ App} consists in normalizing with respect to the plasma frequency $\omega_p=\sqrt{2eI_c/\hbar C}$.
%In the latter case, the normalized RCSJ equation \eqref{RCSJ App} reads
%
%
%\begin{equation}
%\frac{d^2 \varphi (\tau_p)}{d\tau_p^2}+ \alpha \frac{d \varphi (\tau_p)}{d\tau_p} + \sin \left [ \varphi\left ( \tau_p \right ) \right ] = i_{n}(\tau_p) + i_b,
%\label{RCSJnormOp App}
%\end{equation}
%
%
%where time is normalized respect to the inverse plasma frequency (that reads $\omega_p=\sqrt{{\Phi_0}/{2\pi C}}$ ), $\tau_p = \omega_p~t$, $\alpha=1/\sqrt(\omega_p~R~C)=1/\sqrt{\bC}$ is the damping parameter. 
%With this time normalization the under- and over-damped regimes correspond to $\alpha \ll 1$ and $\alpha \gg 1$, respectively.


\subsection{Axion}
\label{Modelaxions}

If one describes the axion field $a$ in the Friedman-Robertson-Walker metric, the equation of motion of the axion misalignment angle $\theta$ and the axion coupling constant $f_a$, namely $a=f_a\,\theta$~\cite{Sik83,Vis13}, reads
%
\begin{equation}
\frac{d^2 \theta (t)}{dt^2}+ H \frac{d \theta (t)}{dt} + \frac{m_a^2c^4}{\hbar^2} \sin \left [ \theta \left ( t \right ) \right ] = 0,
\label{AxionEq}
\end{equation}
%
complemented with spatial gradients that are here omitted. 
The above model includes the forcing term $\sin(\theta)$ due to quantum chromodynamics instanton effects. As one can observe, there is a formal similarity between the equation of motion governing the axion and the RCSJ systems, being the axion dynamics analogous to an unbiased RCSJ. Moreover, in normalized units, the parameters are of the same order of magnitude.
In Eq.~\eqref{AxionEq}, $H \approx 2 \times 10^{-18} ~ s^{-1}$ is the Hubble parameter and $m_a$ is the axion mass. 
The typical ranges of parameters that are allowed for dark matter axions are~\cite{sik09,Duf09}: 
$ 3 \times 10^9 ~ \text{GeV} \leq f_a \leq 10^{12} ~ \text{GeV}$ and $ 6 \times 10^{-6} ~ \text{eV} \leq m_a c^2 \leq 2 \times 10^{-3} ~ \text{eV}$. 
The prediction of the axion's mass, based on the average of the results from five independent condensed matter experiments, is $ m_a c^2 = (106 \pm 6) \mu eV$~\cite{,Hof04,Bae08,He11,Gol12,Bre13,Wang22}.
%

%
% Figure environment removed
%


\subsection{Axion-JJ System}
\label{ModelJJaxions}

According to the approach of Refs.~\cite{Grimaudo22,Yan20}, the interaction between axion and JJ can be formally written as 
%
%
\begin{subequations}\label{Orig Diff Eqs Syst}
\begin{align}
\ddot{\varphi} + a_1 \dot{\varphi} + b_1 \sin(\varphi) &= \gamma (\ddot{\theta} - \ddot{\varphi}) \label{Orig Diff Eqs Syst a},\\
\ddot{\theta} + a_2 \dot{\theta} + b_2 \sin(\theta) &= \gamma (\ddot{\varphi} - \ddot{\theta}),
\end{align}
\label{Orig Diff Eqs Syst}
\end{subequations}
%
%
%
\noindent where $(a_1, a_2)$ and $(b_1, b_2)$ are the dissipation and frequency parameters, respectively; $\gamma$ is the coupling constant between the two systems, whose values one wants to infer from the experiments. 
This model, which succeeds in explaining some experimental anomalies~\cite{Hof04,Bae08,He11,Gol12,Bre13,Wang22}, is based on the possibility to formally treat the axion as an effective JJ, and therefore to consider the system in Eqs.~\eqref{Orig Diff Eqs Syst} as equivalent to two capacitively coupled JJs~\cite{Blac09}.

To model the Josephson phase dynamics with a bias current and thermal fluctuations, Eqs.~\eqref{Orig Diff Eqs Syst} can be conveniently rewritten as 
%
\begin{subequations}
\label{Diff Eqs Syst omegac}
\begin{align}
%\label{DiffEqsJJ}
{\beta_c \over k_2}~\ddot{\varphi}+\dot{\varphi}+\sin(\varphi)+{k_1 \over k_2}~\varepsilon~\sin(\theta) &= i_b+i_n, \label{Diff Eqs Syst omegac phi} \\
%\label{DiffEqsaxion}
{\beta_c \over k_1}~\ddot{\theta}+\dot{\varphi}+\sin(\varphi)+{k_2 \over k_1}~\varepsilon~\sin(\theta) &= i_b+i_n, \label{Diff Eqs Syst omegac theta}
\end{align}
\end{subequations}
%
%
with 
%
%
\begin{equation}
\begin{aligned}
&k_1  =  {\gamma \over 1+2\gamma}, \qquad k_2  =  {1+\gamma \over 1+2\gamma}, \qquad \varepsilon  =  \left(\frac{m_ac^2}{\hbar\omega_p}\right)^2.
\label{epsilon}
\end{aligned}
\end{equation}
%
%&k_1 = {\gamma \over 1+2\gamma}, \qquad k_2 = {1+\gamma \over 1+2\gamma}, \\
%& \beta_c = \left( \frac{\omega_c}{\omega_p} \right)^2,\quad \varepsilon = \left({m_ac^2 \over \hbar\omega_p}\right)^2,
%
%
%
 The $\varepsilon$ parameter indicates the ratio between the axion energy and the 
Josephson plasma energy, $\hbar\omega_p$, and can be chosen -- within the JJ fabrication constraints -- to select the most convenient working point for the detection of an axion field interacting with the JJ. 
Indeed, the Josephson plasma frequency, and therefore the energy ratio $\varepsilon$, can be ``adjusted'' as needed, for $I_c$ can be lowered either by raising the temperature~\cite{Dub01} or by applying a magnetic field~\cite{Ber08} or a gate voltage~\cite{Du08}. 
The purpose is to determine the working point at which the system is most responsive to the axion perturbation. 
As the detection is performed through the analysis of the escape times, the response is measured in the precise sense that the distribution of the escape times for the axion-JJ coupled system deviates the most from the Josephson response in the absence of axions. In Ref.~\cite{Grimaudo22} it was in fact showed that at $\epsilon\lesssim1$ the average switching time approaches a minimum due to an axion-induced resonant activation phenomenon, for the occurrence of an effective frequency matching between axion and JJ, whereas the optimal experimental conditions for a JJ-based axion detection scheme should involve a Josephson plasma energy lower than the axion energy, i.e., $\epsilon>1$.


In this work, we trace the change in escape time (that makes the axion-JJ interaction detectable) back to the change in the effective potential barrier that confines the system to the static zero-voltage configuration. In the following Sect.~\ref{Results} we discuss how to compute the effective energy.
%, while we now discuss the numerical values of the axion parameters that enter Eq.(\ref{Diff Eqs Syst omegac}(b)). 

The integration of the stochastic Eqs.~\eqref{Diff Eqs Syst omegac} is performed with a finite-difference explicit method, using a time integration step $\Delta t=10^{-2}$, a maximum integration time $t_{max}=10^6$, initial conditions $\varphi(0)=\arcsin{(i_b)}$ 
 and $\theta(0)=\dot{\varphi}(0)=\dot{\theta}(0)=0$, and repeating each simulation $N=10^4$ times for each set of parameter values. Other parameters useful for the calculations are set as $\beta_c=100$ (i.e., an underdamped regime) and $\varepsilon=1$.


\section{Calculation of the Quasipotential}
\label{Results}

The non-equilibrium system in Eq.~\eqref{Orig Diff Eqs Syst} does not admit an ordinary potential. However, it is possible to define an effective, or quasi, potential that keeps the system in the static configuration. The axion-JJ coupled system eventually switches from the superconducting state ($V\propto d\varphi/dt=0$) to the resistive state ($V\propto d\varphi/dt\neq 0$), when the combined effect of noise and axion interaction allows the JJ to overcome the effective energy barrier. 
In this picture, one can think of the axion effect on the JJ as some perturbation that changes (more precisely, lowers, as we shall demonstrate below) the effective energy barrier. The advantage is that the change of this effective energy is independent of noise and therefore holds at any (sufficiently low) temperature.
The main difficulty is therefore to determine how the coupling between Eqs.~\eqref{Diff Eqs Syst omegac} amounts to a change in the quasipotential. 
To begin with, we show how to compute the quasipotential, that is the effective energy that must be overcome to induce a switch. 
The basic logic is as follows: suppose that the Arrhenius behavior~\cite{Graham85}
\begin{equation}
\label{Kramers}
\tau = \lim_{D\to0}\tau_0 \exp{\frac{\kappa}{D} }
\end{equation}
is valid. 
The hypothesis obviously holds for the "pure" JJ system, i.e, Eq.~\eqref{RCSJnormOc}, for which $\kappa\equiv\Delta \mathcal{U}$. 
One can make the further conjecture that the average escape time is exponential in the inverse of the noise intensity, with some coefficient $\kappa \neq \Delta \mathcal{U}$, such that, in the limit of small noise,
\begin{equation}
\label{exprelation}
 \log{\frac{\tau}{\tau_0}} = \kappa \frac{1}{D}.
\end{equation}
Under general assumptions~\footnote{For out-of-equilibrium systems that do not admit an ordinary potential, it is possible to define a non-equilibrium potential with properties analogous to those of an ordinary potential, as long as there is a single time-independent probability distribution that can be reached from any initial distribution as the weakly stochastic dynamical system approaches its steady state~\cite{Graham86}.}, it is fair to suppose that the exponential relation holds in the limit of vanishing noise, and one can thus interpret the coefficient $\kappa$ as an effective energy barrier:
\begin{equation}\label{quasipotential}
\Delta \mathcal{U}_{eff} \equiv \kappa = \lim_{D \rightarrow \infty}\frac{ \log{\tau/\tau_0}}{1/D}.
\end{equation}
In other words, the slope $\kappa$ of the relation \eqref{exprelation} can be interpreted as a {\it bona fide} potential barrier, in the limit of small noise. 
The advantage of this interpretation is twofold. 
On the one end, it gives a physically intuitive interpretation to the effect of the axion field; as we shall prove in the following subsection, the axion lowers the confining barrier, and the lowering is enhanced by the coupling $\gamma$. 
On the other hand the quasipotential offers a practical advantage, because it allows to extrapolate the results to very low values of noise, that is in the region where escape times are prohibitively long and extremely difficult to reach with simulations.
It is in fact enough to enter the regime of exponential decay to determine $\kappa$, and then to exploit such value for any lower value of the noise intensity $D$. 
The effective potential \eqref{quasipotential} can be numerically retrieved with several estimates of the escape time $\tau$ as a function of the noise amplitude $D$; in the plot $\log{\tau}$ vs $1/D$ the prefactor $\tau_0$ is the $y$-axis intercept and $\kappa$ the slope of the relationship. 
More precisely, the two quantities should be computed in the exponential regime, that is discarding the data for high $D$ to ensure that the asymptotic regime \eqref{Kramers} has been entered, as shown in Fig.~\ref{Kramers_fit}. 

% Figure environment removed


%\subsection{Behavior of the Quasi-Potential}

The axions are revealed through the difference between the electrical responses of an ideal JJ without external perturbation, and the same junction under the influence of an axion field. This difference is determined on average by the quasipotential. However, for any finite sampling the actual measured average is subject to fluctuations. Therefore, the better detection is obtained with the method to which pertains the best signal-to-noise ratio (SNR), where the signal is the measured average difference and the noise is due to the fluctuations around the average, i.e, the standard deviation of the sampled mean. The SNR thus computed is often measured through the Kumar-Carroll index~\cite{Kumar84}. However, for simplicity in this work we merely observe the effect of axion on the effective potential felt by the axion-JJ system.


For the uncoupled regime, $\gamma =0$, the numerical quasipotential should be equal to the ordinary potential; in fact, in Fig.~\ref{fig:QP} one only observes a modest discrepancy, $\eta<10\%$, due to the finite temperature (the quasipotential should be computed in the limit $D^{-1}\rightarrow \infty$, or more accurately $\Delta \mathcal{U}_{eff}/D \rightarrow \infty$) and to the finite number of realizations over which the average is calculated. 
This discrepancy is obtained as the percentage difference between $\Delta \mathcal{U}_{eff}$ estimated from the data of Fig.~\ref{fig:QP} at the lowest noise, i.e., $\gamma=0.001$, and the analytical washboard activation energy $\Delta \mathcal{U}$, see Eq.~\eqref{activationenergybarrier App}, that is $ \eta=\left ( \Delta \mathcal{U}-\Delta \mathcal{U}_{eff} \right )/\Delta \mathcal{U}=\{7.4\%, 8.0\%,\text{ and }6.1\%\}$ for $i_b=\{0.1, 0.5,\text{ and }0.8\}$, respectively.
It is indeed remarkable that, despite the relatively high noise ($\Delta \mathcal{U}_{eff}/ D \in [3-6]$ from data in Figs.~\ref{Kramers_fit} and~\ref{fig:QP}), the agreement is good.
This observation highlights the advantage of the quasipotential method, because one can use short escape times to effectively evaluate the effective quasipotential through Eq.~\eqref{quasipotential}. Shortly, we have demonstrated that the quasipotential for the JJ-axion system can be numerically evaluated with relative ease, while the result can be extrapolated to much lower values of noise, and hence to much longer escape times.

Finally, we want to exploit the estimation of the effect of the quasipotential for the detection of the axion. 
This is summarized in Fig.~\ref{fig:QP}, where the slope of the escape times versus $\gamma$ is identified with the quasipotential.
For each of the three different values of the bias current considered, $i_b=\{0.1, 0.5,\text{ and }0.8\}$, it is proven that the coupling $\gamma$ lowers the effective energy barrier. 
The effect seems more uniform for $i_b = 0.1$, while for shallow barriers [e.g., see $i_b=0.8$ in Fig.~\ref{fig:QP}c)] the change is more evident only for $\gamma\gtrsim0.1$. In fact, the greater $i_b$, the higher the $\gamma$ value above which the coupling with the axion produces an effect on the quasipotential. Interestingly, the bias current, which actually represents an easily controllable parameter, was also proven to have a significant impact on the emerging of resonances in the switching times discussed in Ref.~\cite{Grimaudo22}.
It is therefore evident from our simulations that a lower bias current is more convenient. Moreover, a different $\gamma$ gives quite different quasipotential values, such that the overall effect of the coupling between JJ and axion is to reduce the effective height of the potential barrier. 
%\newpage



% Figure environment removed

\section{Conclusions}
\label{Conclusions}

It has been shown that if the direct interaction between a solid state superconducting device, i.e., a Josephson junction, and the dark matter candidate named axion is assumed, a modification of the response to noise of the former arises \cite{Grimaudo22,Gri23}. This aspect offers an opportunity for the detection of this elusive particle.
Using JJs to detect axions has been shown to be beneficial for several reasons. First, JJs are superconducting devices that can operate at very low temperatures, and, hence, at very low noise. Second, they are very fast elements, with typical characteristic frequencies from GHz to THz, and therefore large amount of data can be collected in a brief time. Third, some parameters of the Josephson device can be adjusted to tune the coupling with the axion. The bias current is a further degree of freedom that can be exploited to tune the effective barrier, as shown in Fig.~\ref{fig:QP}.

In a nutshell, axion signature can be sketched as follows: the JJ-axion interaction facilitates, in the presence of noise, the escapes of the Josephson phase from the superconducting to the finite-voltage state. 
This change can be described through the quasipotential, which is an effective energy barrier that summarizes the response to noise in the limit of small fluctuations. 
The introduction of the quasipotential allows to extrapolate the behavior at very low noise values, at which numerical simulations become prohibitively long. 
It is thus possible to reconstruct the response at low temperature through simulations performed at relatively high values of fluctuations, an advantage that has been already exploited in several applications, as for instance the Josephson voltage standard for which even very rare escapes are relevant to maintain the high accuracy required by metrological standards~\cite{Kau96}. 
Analogously, for weak signal detection it is important to mimic occurrence of rare events in a quite noisy environment~\cite{Pie21}.
In this work we have extended the method to the interaction between an underdamped JJ and an axion field. 
Within this framework, it has been possible to demonstrate that the interaction is summarized by a quasipotential, and to determine the behavior of this effect through the quasipotential as a function of the JJ-axion interaction. 
In particular, it has been established that the quasipotential depends upon the strength of the interaction, in the precise meaning that the stronger the interaction, the lower the quasipotential effective barrier.
An ideal experiment comparing the escape times of a JJ subject only to noise with those of a junction subject to the same noise and an axion field could  reveal the presence of axions. This would be evidenced by a decrease in the mean escape time, and the magnitude of this decrease would provide a quantitative estimate of the JJ-axion interaction.
Furthermore, the decrease should persist at any noise value, even if very small, i.e. as small as necessary to achieve the desired SNR. Finally, numerical simulations demonstrated that the observed behavior holds for different bias points, thus providing an additional tunable parameter for experimental setup.



A word of caution: the change in potential energy is only one of the ingredients for an accurate calculation of SNR, which requires the estimation of fluctuations for finite sampling, as previously done for Josephson-based single photon detection schemes~\cite{Pied21,Gua21,Piedjou21}, and provides additional information on the sample size needed to determine exclusion graphs or the residual operator characteristic~\cite{Add12,Filatrella23}.

A further refinement of this approach could be achieved through the \emph{principle of minimum available noise energy}~\cite{Kautz88}. Shortly, the idea could be to follow the path of the unperturbed JJ to determine the critical point, i.e., the separatrix between the two basins of attraction belonging to different stable points. By doing this for the deterministic noise-free system, it is possible to calculate the energy in the minimum and the minimum work required to bring the system from the initial stable point to this critical point, i.e., exactly the quasipotential. Any other work involving a different trajectory between these two stable points is larger than the actual quasipotential.
If the method provides a close estimate of the quasipotential through stochastic lengthy numerical calculations, it may be exploited to search the vast parameter space with straightforward deterministic analysis. 

To conclude, our study aims to provide further insights into the interplay between noise, switching dynamic of the JJ and available signal statistical properties, to enhance the understanding of axion JJ-based detector, and at the same time its robustness and reliability. 
Through the characterization of the noise-induced effects and the understanding of their implications, we wish to contribute to the development of better detectors and of quantum technology devices with improved performances.








%\bibliography{bibliofile}

%%%%%%%%%%%%%%%%%%%%%%%%%%%%%%%%%%%%%%%%%%
\vspace{6pt} 

%%%%%%%%%%%%%%%%%%%%%%%%%%%%%%%%%%%%%%%%%%
%% optional
%\supplementary{The following supporting information can be downloaded at:  \linksupplementary{s1}, Figure S1: title; Table S1: title; Video S1: title.}

% Only for the journal Methods and Protocols:
% If you wish to submit a video article, please do so with any other supplementary material.
% \supplementary{The following supporting information can be downloaded at: \linksupplementary{s1}, Figure S1: title; Table S1: title; Video S1: title. A supporting video article is available at doi: link.}

%%%%%%%%%%%%%%%%%%%%%%%%%%%%%%%%%%%%%%%%%%
\authorcontributions{Conceptualization, R.G. and C.G.; methodology, R.G., D.V., and C.G.; software, R.G.; validation, R.G., D.V. and C.G.; formal analysis, R.G.; investigation, R.G. and C.G.; resources, D.V.; data curation, R.G. and C.G.; writing---original draft preparation, G.F. and C.G.; writing---review and editing, R.G., D.V., B.S., A.T., G.F. and C.G.; visualization, R.G., D.V., B.S., A.T., G.F. and C.G.; supervision, D.V., B.S., G.F. and C.G; funding acquisition, D.V. All authors have read and agreed to the published version of the manuscript.}

%\funding{Please add: ``This research received no external funding'' or ``This research was funded by NAME OF FUNDER grant number XXX.'' and  and ``The APC was funded by XXX''. Check carefully that the details given are accurate and use the standard spelling of funding agency names at \url{https://search.crossref.org/funding}, any errors may affect your future funding.}

%\institutionalreview{In this section, you should add the Institutional Review Board Statement and approval number, if relevant to your study. You might choose to exclude this statement if the study did not require ethical approval. Please note that the Editorial Office might ask you for further information. Please add “The study was conducted in accordance with the Declaration of Helsinki, and approved by the Institutional Review Board (or Ethics Committee) of NAME OF INSTITUTE (protocol code XXX and date of approval).” for studies involving humans. OR “The animal study protocol was approved by the Institutional Review Board (or Ethics Committee) of NAME OF INSTITUTE (protocol code XXX and date of approval).” for studies involving animals. OR “Ethical review and approval were waived for this study due to REASON (please provide a detailed justification).” OR “Not applicable” for studies not involving humans or animals.}

%\informedconsent{Any research article describing a study involving humans should contain this statement. Please add ``Informed consent was obtained from all subjects involved in the study.'' OR ``Patient consent was waived due to REASON (please provide a detailed justification).'' OR ``Not applicable'' for studies not involving humans. You might also choose to exclude this statement if the study did not involve humans.

%Written informed consent for publication must be obtained from participating patients who can be identified (including by the patients themselves). Please state ``Written informed consent has been obtained from the patient(s) to publish this paper'' if applicable.}

\dataavailability{The data presented in this study are available on reasonable request from the corresponding authors.} 

\acknowledgments{R.G. acknowledges financial support from the PRIN Project PRJ-0232 - Impact of Climate Change on the biogeochemistry of Contaminants in the Mediterranean sea (ICCC). All the authors acknowledge the support of the Ministry of University and Research of Italian Government.}

\conflictsofinterest{The authors declare no conflict of interest.} 

%%%%%%%%%%%%%%%%%%%%%%%%%%%%%%%%%%%%%%%%%%
%% Optional
%\sampleavailability{Samples of the compounds ... are available from the authors.}

%% Only for journal Encyclopedia
%\entrylink{The Link to this entry published on the encyclopedia platform.}

\abbreviations{Abbreviations}{
The following abbreviations are used in this manuscript:\\

\noindent 
\begin{tabular}{@{}ll}
JJ & Josephson junction\\
RCSJ & resistively and capacitively shunted junction
\end{tabular}
}

%%%%%%%%%%%%%%%%%%%%%%%%%%%%%%%%%%%%%%%%%%


%%%%%%%%%%%%%%%%%%%%%%%%%%%%%%%%%%%%%%%%%%
\begin{adjustwidth}{-\extralength}{0cm}
%\printendnotes[custom] % Un-comment to print a list of endnotes

%\bibliography{bibliofile}

\begin{thebibliography}{999}

\bibitem[Preskill \em{et~al.}(1983)Preskill, Wise, and Wilczek]{Preskili83}
Preskill, J.; Wise, M.B.; Wilczek, F.
\newblock Cosmology of the invisible axion.
\newblock {\em Physics Letters B} {\bf 1983}, {\em 120},~127--132.
\newblock {\url{https://doi.org/https://doi.org/10.1016/0370-2693(83)90637-8}}.

\bibitem[Alesini \em{et~al.}(2020)Alesini, Babusci, Barone, B., Beretta,
  Bianchini, Castellano, Chiarello, Di~Gioacchino, Falferi, Felici, Filatrella,
  Foggetta, Gallo, Gatti, Giazotto, Lamanna, Ligabue, Ligato, Ligi, Maccarrone,
  Margesin, Mattioli, Monticone, Oberto, Pagano, Paolucci, Rajteri, Rettaroli,
  Rolandi, Spagnolo, Toncelli, and Torrioli]{Alesini20}
Alesini, D.; Babusci, D.; Barone, C.; B., B.; Beretta, M.M.; Bianchini, L.;
  Castellano, G.; Chiarello, F.; Di~Gioacchino, D.; Falferi, P.;  et~al.
\newblock Status of the SIMP Project: Toward the Single Microwave Photon
  Detection.
\newblock {\em Journal of Low Temperature Physics} {\bf 2020}, {\em 199},~348--
  354.
\newblock {\url{https://doi.org/10.1007/s10909-020-02381-x}}.

\bibitem[Rettaroli \em{et~al.}(2021)Rettaroli, Alesini, Babusci, Barone,
  Buonomo, Beretta, Castellano, Chiarello, Di~Gioacchino, Felici, Filatrella,
  Foggetta, Gallo, Gatti, Ligi, Maccarrone, Mattioli, Pagano, Tocci, and
  Torrioli]{Rettaroli21}
Rettaroli, A.; Alesini, D.; Babusci, D.; Barone, C.; Buonomo, B.; Beretta,
  M.M.; Castellano, G.; Chiarello, F.; Di~Gioacchino, D.; Felici, G.;  et~al.
\newblock Josephson Junctions as Single Microwave Photon Counters: Simulation
  and Characterization.
\newblock {\em Instruments} {\bf 2021}, {\em 5}.
\newblock {\url{https://doi.org/10.3390/instruments5030025}}.

\bibitem[Beck(2013)]{Bec13}
Beck, C.
\newblock Possible Resonance Effect of Axionic Dark Matter in Josephson
  Junctions.
\newblock {\em Phys. Rev. Lett.} {\bf 2013}, {\em 111},~231801.

\bibitem[Grimaudo \em{et~al.}(2022)Grimaudo, Valenti, Spagnolo, Filatrella, and
  Guarcello]{Grimaudo22}
Grimaudo, R.; Valenti, D.; Spagnolo, B.; Filatrella, G.; Guarcello, C.
\newblock Josephson-junction-based axion detection through resonant activation.
\newblock {\em Phys. Rev. D} {\bf 2022}, {\em 105},~033007.
\newblock {\url{https://doi.org/10.1103/PhysRevD.105.033007}}.

\bibitem[Grimaudo \em{et~al.}(2023)Grimaudo, Valenti, Filatrella, Spagnolo, and
  Guarcello]{Gri23}
Grimaudo, R.; Valenti, D.; Filatrella, G.; Spagnolo, B.; Guarcello, C.
\newblock Coupled quantum pendula as a possible model for
  Josephson-junction-based axion detection.
\newblock {\em Chaos, Solitons \& Fractals} {\bf 2023}, {\em 173},~113745.
\newblock {\url{https://doi.org/https://doi.org/10.1016/j.chaos.2023.113745}}.

\bibitem[Beck(2017)]{Bec17}
Beck, C.
\newblock Possible resonance effect of dark matter axions in SNS Josephson
  junctions.
\newblock {\em PoS} {\bf 2017}, {\em EPS-HEP2017},~058.

\bibitem[Hoffmann \em{et~al.}(2004)Hoffmann, Lefloch, Sanquer, and
  Pannetier]{Hof04}
Hoffmann, C.; Lefloch, F.; Sanquer, M.; Pannetier, B.
\newblock Mesoscopic transition in the shot noise of diffusive
  superconductor--normal-metal--superconductor junctions.
\newblock {\em Phys. Rev. B} {\bf 2004}, {\em 70},~180503.

\bibitem[Bae \em{et~al.}(2008)Bae, Dinsmore~III, Sahu, Lee, and
  Bezryadin]{Bae08}
Bae, M.H.; Dinsmore~III, R.C.; Sahu, M.; Lee, H.J.; Bezryadin, A.
\newblock Zero-crossing Shapiro steps in high-${T}_{c}$ superconducting
  microstructures tailored by a focused ion beam.
\newblock {\em Phys. Rev. B} {\bf 2008}, {\em 77},~144501.

\bibitem[He \em{et~al.}(2011)He, Wang, and Chan]{He11}
He, L.; Wang, J.; Chan, M.H.
\newblock Shapiro Steps in the Absence of Microwave Radiation.
\newblock {\em arXiv preprint arXiv:1107.0061} {\bf 2011}.

\bibitem[Golikova \em{et~al.}(2012)Golikova, H\"ubler, Beckmann, Batov,
  Karminskaya, Kupriyanov, Golubov, and Ryazanov]{Gol12}
Golikova, T.E.; H\"ubler, F.; Beckmann, D.; Batov, I.E.; Karminskaya, T.Y.;
  Kupriyanov, M.Y.; Golubov, A.A.; Ryazanov, V.V.
\newblock Double proximity effect in hybrid planar superconductor-(normal
  metal/ferromagnet)-superconductor structures.
\newblock {\em Phys. Rev. B} {\bf 2012}, {\em 86},~064416.

\bibitem[Bretheau \em{et~al.}(2013)Bretheau, Girit, Pothier, Esteve, and
  Urbina]{Bre13}
Bretheau, L.; Girit, {\c{C}}.{\"O}.; Pothier, H.; Esteve, D.; Urbina, C.
\newblock Exciting Andreev pairs in a superconducting atomic contact.
\newblock {\em Nature} {\bf 2013}, {\em 499},~312--315.

\bibitem[Wang \em{et~al.}(2022)Wang, Wang, and Wang]{Wang22}
Wang, J.; Wang, Z.; Wang, P.
\newblock Magnetic field enhanced critical current and subharmonic structures
  in dissipative superconducting gold nanowires.
\newblock {\em Quantum Frontiers} {\bf 2022}, {\em 1},~21.
\newblock {\url{https://doi.org/10.1007/s44214-022-00021-x}}.

\bibitem[Valenti \em{et~al.}(2012)Valenti, Denaro, La~Cognata, Spagnolo,
  Bonanno, Basilone, Mazzola, Zgozi, and Aronica]{Val12}
Valenti, D.; Denaro, G.; La~Cognata, A.; Spagnolo, B.; Bonanno, A.; Basilone,
  G.; Mazzola, S.; Zgozi, S.; Aronica, S.
\newblock Picophytoplankton Dynamics in Noisy Marine Environment.
\newblock {\em Acta Phys. Pol. B} {\bf 2012}, {\em 43},~1227.
\newblock {\url{https://doi.org/https://doi.org/10.5506/APhysPolB.43.1227}}.

\bibitem[Lisowski \em{et~al.}(2015)Lisowski, Valenti, Spagnolo, Bier, and
  Gudowska-Nowak]{Lis15}
Lisowski, B.; Valenti, D.; Spagnolo, B.; Bier, M.; Gudowska-Nowak, E.
\newblock Stepping molecular motor amid L\'evy white noise.
\newblock {\em Phys. Rev. E} {\bf 2015}, {\em 91},~042713.
\newblock {\url{https://doi.org/10.1103/PhysRevE.91.042713}}.

\bibitem[Valenti \em{et~al.}(2016)Valenti, Denaro, Spagnolo, Mazzola, Basilone,
  Conversano, Brunet, and Bonanno]{Valenti16}
Valenti, D.; Denaro, G.; Spagnolo, B.; Mazzola, S.; Basilone, G.; Conversano,
  F.; Brunet, C.; Bonanno, A.
\newblock Stochastic models for phytoplankton dynamics in Mediterranean Sea.
\newblock {\em Ecological Complexity} {\bf 2016}, {\em 27},~84--103.
\newblock Mathematical Ecology and Epidemiology,
  {\url{https://doi.org/https://doi.org/10.1016/j.ecocom.2015.06.001}}.

\bibitem[Valenti \em{et~al.}(2018)Valenti, Fazio, and Spagnolo]{Valenti18}
Valenti, D.; Fazio, G.; Spagnolo, B.
\newblock Stabilizing effect of volatility in financial markets.
\newblock {\em Phys. Rev. E} {\bf 2018}, {\em 97},~062307.
\newblock {\url{https://doi.org/10.1103/PhysRevE.97.062307}}.

\bibitem[Tobiska and Nazarov(2004)]{Tob04}
Tobiska, J.; Nazarov, Y.V.
\newblock Josephson Junctions as Threshold Detectors for Full Counting
  Statistics.
\newblock {\em Phys. Rev. Lett.} {\bf 2004}, {\em 93},~106801.
\newblock {\url{https://doi.org/10.1103/PhysRevLett.93.106801}}.

\bibitem[Pekola(2004)]{Pek04}
Pekola, J.P.
\newblock Josephson Junction as a Detector of Poissonian Charge Injection.
\newblock {\em Phys. Rev. Lett.} {\bf 2004}, {\em 93},~206601.
\newblock {\url{https://doi.org/10.1103/PhysRevLett.93.206601}}.

\bibitem[Ankerhold(2007)]{Ank07}
Ankerhold, J.
\newblock Detecting Charge Noise with a Josephson Junction: A Problem of
  Thermal Escape in Presence of Non-Gaussian Fluctuations.
\newblock {\em Phys. Rev. Lett.} {\bf 2007}, {\em 98},~036601.
\newblock {\url{https://doi.org/10.1103/PhysRevLett.98.036601}}.

\bibitem[Sukhorukov and Jordan(2007)]{Suk07}
Sukhorukov, E.V.; Jordan, A.N.
\newblock Stochastic Dynamics of a Josephson Junction Threshold Detector.
\newblock {\em Phys. Rev. Lett.} {\bf 2007}, {\em 98},~136803.
\newblock {\url{https://doi.org/10.1103/PhysRevLett.98.136803}}.

\bibitem[Timofeev \em{et~al.}(2007)Timofeev, Meschke, Peltonen, Heikkil\"a, and
  Pekola]{Tim07}
Timofeev, A.V.; Meschke, M.; Peltonen, J.T.; Heikkil\"a, T.T.; Pekola, J.P.
\newblock Wideband Detection of the Third Moment of Shot Noise by a Hysteretic
  Josephson Junction.
\newblock {\em Phys. Rev. Lett.} {\bf 2007}, {\em 98},~207001.
\newblock {\url{https://doi.org/10.1103/PhysRevLett.98.207001}}.

\bibitem[Huard \em{et~al.}(2007)Huard, Pothier, Birge, Esteve, Waintal, and
  Ankerhold]{Hua07}
Huard, B.; Pothier, H.; Birge, N.O.; Esteve, D.; Waintal, X.; Ankerhold, J.
\newblock Josephson junctions as detectors for non-Gaussian noise.
\newblock {\em Annalen der Physik} {\bf 2007}, {\em 16},~736--750.

\bibitem[Grabert(2008)]{Gra08}
Grabert, H.
\newblock Theory of a Josephson junction detector of non-Gaussian noise.
\newblock {\em Phys. Rev. B} {\bf 2008}, {\em 77},~205315.
\newblock {\url{https://doi.org/10.1103/PhysRevB.77.205315}}.

\bibitem[Filatrella and Pierro(2010)]{Fil10}
Filatrella, G.; Pierro, V.
\newblock Detection of noise-corrupted sinusoidal signals with Josephson
  junctions.
\newblock {\em Phys. Rev. E} {\bf 2010}, {\em 82},~046712.
\newblock {\url{https://doi.org/10.1103/PhysRevE.82.046712}}.

\bibitem[Addesso \em{et~al.}(2012)Addesso, Filatrella, and Pierro]{Add12}
Addesso, P.; Filatrella, G.; Pierro, V.
\newblock Characterization of escape times of Josephson junctions for signal
  detection.
\newblock {\em Phys. Rev. E} {\bf 2012}, {\em 85},~016708.
\newblock {\url{https://doi.org/10.1103/PhysRevE.85.016708}}.

\bibitem[Guarcello \em{et~al.}(2013)Guarcello, Valenti, Augello, and
  Spagnolo]{Gua13}
Guarcello, C.; Valenti, D.; Augello, G.; Spagnolo, B.
\newblock The Role of Non-Gaussian Sources in the Transient Dynamics of Long
  Josephson Junctions.
\newblock {\em Acta Phys. Pol. B} {\bf 2013}, {\em 44},~997--1005.
\newblock {\url{https://doi.org/10.5506/APhysPolB.44.997}}.

\bibitem[Guarcello \em{et~al.}(2019)Guarcello, Valenti, Spagnolo, Pierro, and
  Filatrella]{Gua19}
Guarcello, C.; Valenti, D.; Spagnolo, B.; Pierro, V.; Filatrella, G.
\newblock Josephson-based Threshold Detector for L\'evy-Distributed Current
  Fluctuations.
\newblock {\em Physical Review Applied} {\bf 2019}, {\em 11},~044078.

\bibitem[Guarcello \em{et~al.}(2020)Guarcello, Filatrella, Spagnolo, Pierro,
  and Valenti]{Gua20}
Guarcello, C.; Filatrella, G.; Spagnolo, B.; Pierro, V.; Valenti, D.
\newblock Voltage drop across Josephson junctions for L\'evy noise detection.
\newblock {\em Physical Review Research} {\bf 2020}, {\em 2},~043332.

\bibitem[Guarcello(2021)]{Gua21-2}
Guarcello, C.
\newblock L\'evy noise effects on Josephson junctions.
\newblock {\em Chaos Solitons Fract} {\bf 2021}, {\em 153},~111531.
\newblock {\url{https://doi.org/https://doi.org/10.1016/j.chaos.2021.111531}}.

\bibitem[Walsh \em{et~al.}(2017)Walsh, Efetov, Lee, Heuck, Crossno, Ohki, Kim,
  Englund, and Fong]{Wal17}
Walsh, E.D.; Efetov, D.K.; Lee, G.H.; Heuck, M.; Crossno, J.; Ohki, T.A.; Kim,
  P.; Englund, D.; Fong, K.C.
\newblock Graphene-Based Josephson-Junction Single-Photon Detector.
\newblock {\em Phys. Rev. Applied} {\bf 2017}, {\em 8},~024022.
\newblock {\url{https://doi.org/10.1103/PhysRevApplied.8.024022}}.

\bibitem[{Kuzmin} \em{et~al.}(2018){Kuzmin}, {Sobolev}, {Gatti}, {Di
  Gioacchino}, {Crescini}, {Gordeeva}, and {Il'ichev}]{Kuz18}
{Kuzmin}, L.S.; {Sobolev}, A.S.; {Gatti}, C.; {Di Gioacchino}, D.; {Crescini},
  N.; {Gordeeva}, A.; {Il'ichev}, E.
\newblock Single Photon Counter Based on a Josephson Junction at 14 GHz for
  Searching Galactic Axions.
\newblock {\em IEEE Transactions on Applied Superconductivity} {\bf 2018}, {\em
  28},~1--5.

\bibitem[Guarcello \em{et~al.}(2019{\natexlab{a}})Guarcello, Braggio, Solinas,
  and Giazotto]{GuaBra19}
Guarcello, C.; Braggio, A.; Solinas, P.; Giazotto, F.
\newblock Nonlinear Critical-Current Thermal Response of an Asymmetric
  Josephson Tunnel Junction.
\newblock {\em Phys. Rev. Applied} {\bf 2019}, {\em 11},~024002.

\bibitem[Guarcello \em{et~al.}(2019{\natexlab{b}})Guarcello, Braggio, Solinas,
  Pepe, and Giazotto]{GuaBraSol19}
Guarcello, C.; Braggio, A.; Solinas, P.; Pepe, G.P.; Giazotto, F.
\newblock Josephson-Threshold Calorimeter.
\newblock {\em Phys. Rev. Applied} {\bf 2019}, {\em 11},~054074.

\bibitem[Revin \em{et~al.}(2020)Revin, Pankratov, Gordeeva, Yablokov, Rakut,
  Zbrozhek, and Kuzmin]{Rev20}
Revin, L.S.; Pankratov, A.L.; Gordeeva, A.V.; Yablokov, A.A.; Rakut, I.V.;
  Zbrozhek, V.O.; Kuzmin, L.S.
\newblock Microwave photon detection by an Al Josephson junction.
\newblock {\em Beilstein Journal of Nanotechnology} {\bf 2020}, {\em
  11},~960--965.

\bibitem[Yablokov \em{et~al.}(2021)Yablokov, Glushkov, Pankratov, Gordeeva,
  Kuzmin, and Il’ichev]{Yab21}
Yablokov, A.; Glushkov, E.; Pankratov, A.; Gordeeva, A.; Kuzmin, L.;
  Il’ichev, E.
\newblock Resonant response drives sensitivity of Josephson escape detector.
\newblock {\em Chaos Solitons Fract} {\bf 2021}, {\em 148},~111058.
\newblock {\url{https://doi.org/https://doi.org/10.1016/j.chaos.2021.111058}}.

\bibitem[{Piedjou Komnang} \em{et~al.}(2021){Piedjou Komnang}, Guarcello,
  Barone, Gatti, Pagano, Pierro, Rettaroli, and Filatrella]{Pied21}
{Piedjou Komnang}, A.; Guarcello, C.; Barone, C.; Gatti, C.; Pagano, S.;
  Pierro, V.; Rettaroli, A.; Filatrella, G.
\newblock Analysis of Josephson junctions switching time distributions for the
  detection of single microwave photons.
\newblock {\em Chaos, Solitons \& Fractals} {\bf 2021}, {\em 142},~110496.
\newblock {\url{https://doi.org/https://doi.org/10.1016/j.chaos.2020.110496}}.

\bibitem[Guarcello \em{et~al.}(2021)Guarcello, Piedjou~Komnang, Barone,
  Rettaroli, Gatti, Pagano, and Filatrella]{Gua21}
Guarcello, C.; Piedjou~Komnang, A.S.; Barone, C.; Rettaroli, A.; Gatti, C.;
  Pagano, S.; Filatrella, G.
\newblock Josephson-Based Scheme for the Detection of Microwave Photons.
\newblock {\em Phys. Rev. Applied} {\bf 2021}, {\em 16},~054015.

\bibitem[Pankratov \em{et~al.}(2022{\natexlab{a}})Pankratov, Revin, Gordeeva,
  Yablokov, Kuzmin, and Il'ichev]{Pan22-1}
Pankratov, A.L.; Revin, L.S.; Gordeeva, A.V.; Yablokov, A.A.; Kuzmin, L.S.;
  Il'ichev, E.
\newblock Towards a microwave single-photon counter for searching axions.
\newblock {\em npj Quantum Information} {\bf 2022}, {\em 8},~61.
\newblock {\url{https://doi.org/10.1038/s41534-022-00569-5}}.

\bibitem[Pankratov \em{et~al.}(2022{\natexlab{b}})Pankratov, Gordeeva, Revin,
  Ladeynov, Yablokov, and Kuzmin]{Pan22-2}
Pankratov, A.L.; Gordeeva, A.V.; Revin, L.S.; Ladeynov, D.A.; Yablokov, A.A.;
  Kuzmin, L.S.
\newblock Approaching microwave photon sensitivity with Al Josephson junctions.
\newblock {\em Beilstein Journal of Nanotechnology} {\bf 2022}, {\em
  13},~582--589.

\bibitem[Graham and T\'el(1985)]{Graham85}
Graham, R.; T\'el, T.
\newblock Weak-noise limit of Fokker-Planck models and non differentiable
  potentials for dissipative dynamical systems.
\newblock {\em Phys. Rev. A} {\bf 1985}, {\em 31},~1109.

\bibitem[Kautz(1988)]{Kau88}
Kautz, R.L.
\newblock Thermally induced escape: The principle of minimum available noise
  energy.
\newblock {\em Physical Review A} {\bf 1988}, {\em 38},~2066--2080.

\bibitem[Kautz(1996)]{Kau96}
Kautz, R.L.
\newblock Noise, chaos, and the Josephson voltage standard.
\newblock {\em Reports on Progress in Physics} {\bf 1996}, {\em 59},~935--992.

\bibitem[Pountougnigni \em{et~al.}(2019)Pountougnigni, Yamapi, Filatrella, and
  Tchawoua]{Pou19}
Pountougnigni, O.V.; Yamapi, R.; Filatrella, G.; Tchawoua, C.
\newblock Noise and disorder effects in a series of birhythmic Josephson
  junctions coupled to a resonator.
\newblock {\em Phys. Rev. E} {\bf 2019}, {\em 99},~032220.
\newblock {\url{https://doi.org/10.1103/PhysRevE.99.032220}}.

\bibitem[Barone and Paterno(1982)]{Bar82}
Barone, A.; Paterno, G.
\newblock {\em Physics and applications of the Josephson effect}; Wiley, New
  York,  1982.

\bibitem[Likharev(1986)]{Lik86}
Likharev, K.
\newblock {\em Dynamics of Josephson Junctions and Circuits} {\bf 1986}.

\bibitem[Guarcello \em{et~al.}(2015)Guarcello, Valenti, and Spagnolo]{GuaVal15}
Guarcello, C.; Valenti, D.; Spagnolo, B.
\newblock Phase dynamics in graphene-based Josephson junctions in the presence
  of thermal and correlated fluctuations.
\newblock {\em Phys. Rev. B} {\bf 2015}, {\em 92},~174519.

\bibitem[Spagnolo \em{et~al.}(2017)Spagnolo, Guarcello, Magazz\'u, Carollo,
  Persano~Adorno, and Valenti]{Spa17}
Spagnolo, B.; Guarcello, C.; Magazz\'u, L.; Carollo, A.; Persano~Adorno, D.;
  Valenti, D.
\newblock Nonlinear Relaxation Phenomena in Metastable Condensed Matter
  Systems.
\newblock {\em Entropy} {\bf 2017}, {\em 19}.

\bibitem[McCumber(1968)]{McC68}
McCumber, D.E.
\newblock Effect of ac Impedance on dc Voltage‐Current Characteristics of
  Superconductor Weak‐Link Junctions.
\newblock {\em Journal of Applied Physics} {\bf 1968}, {\em 39},~3113--3118.

\bibitem[Beenakker(1992)]{Bee92}
Beenakker, C.W.J.
\newblock Three ``Universal'' Mesoscopic Josephson Effects.
\newblock In Proceedings of the Low-Dimensional Electronic Systems; Bauer, G.;
  Kuchar, F.; Heinrich, H., Eds.; Springer Berlin Heidelberg: Berlin,
  Heidelberg,  1992; pp. 78--82.

\bibitem[Kramers(1940)]{Kra40}
Kramers, H.
\newblock Brownian motion in a field of force and the diffusion model of
  chemical reactions.
\newblock {\em Physica} {\bf 1940}, {\em 7},~284 -- 304.
\newblock
  {\url{https://doi.org/http://dx.doi.org/10.1016/S0031-8914(40)90098-2}}.

\bibitem[{Piedjou Komnang} \em{et~al.}(2021){Piedjou Komnang}, Guarcello,
  Barone, Gatti, Pagano, Pierro, Rettaroli, and Filatrella]{Pie21}
{Piedjou Komnang}, A.; Guarcello, C.; Barone, C.; Gatti, C.; Pagano, S.;
  Pierro, V.; Rettaroli, A.; Filatrella, G.
\newblock Analysis of Josephson junctions switching time distributions for the
  detection of single microwave photons.
\newblock {\em Chaos Solitons Fract} {\bf 2021}, {\em 142},~110496.
\newblock {\url{https://doi.org/https://doi.org/10.1016/j.chaos.2020.110496}}.

\bibitem[Sikivie(1983)]{Sik83}
Sikivie, P.
\newblock Experimental Tests of the "Invisible" Axion.
\newblock {\em Phys. Rev. Lett.} {\bf 1983}, {\em 51},~1415--1417.

\bibitem[Visinelli(2013)]{Vis13}
Visinelli, L.
\newblock Axion-electromagnetic waves.
\newblock {\em Modern Physics Letters A} {\bf 2013}, {\em 28},~1350162.

\bibitem[Sikivie and Yang(2009)]{sik09}
Sikivie, P.; Yang, Q.
\newblock Bose-Einstein condensation of dark matter axions.
\newblock {\em Physical Review Letters} {\bf 2009}, {\em 103},~111301.

\bibitem[Duffy and van Bibber(2009)]{Duf09}
Duffy, L.D.; van Bibber, K.
\newblock Axions as dark matter particles.
\newblock {\em New Journal of Physics} {\bf 2009}, {\em 11},~105008.

\bibitem[Yan and Beck(2020)]{Yan20}
Yan, J.; Beck, C.
\newblock Nonlinear dynamics of coupled axion-Josephson junction systems.
\newblock {\em Physica D: Nonlinear Phenomena} {\bf 2020}, {\em 403},~132294.

\bibitem[Blackburn \em{et~al.}(2009)Blackburn, Marchese, Cirillo, and
  Gr{\o}nbech-Jensen]{Blac09}
Blackburn, J.A.; Marchese, J.E.; Cirillo, M.; Gr{\o}nbech-Jensen, N.
\newblock Classical analysis of capacitively coupled superconducting qubits.
\newblock {\em Physical Review B} {\bf 2009}, {\em 79},~054516.

\bibitem[Dubos \em{et~al.}(2001)Dubos, Courtois, Pannetier, Wilhelm, Zaikin,
  and Sch\"on]{Dub01}
Dubos, P.; Courtois, H.; Pannetier, B.; Wilhelm, F.K.; Zaikin, A.D.; Sch\"on,
  G.
\newblock Josephson critical current in a long mesoscopic S-N-S junction.
\newblock {\em Phys. Rev. B} {\bf 2001}, {\em 63},~064502.

\bibitem[Bergeret and Cuevas(2008)]{Ber08}
Bergeret, F.S.; Cuevas, J.C.
\newblock The Vortex State and Josephson Critical Current of a Diffusive SNS
  Junction.
\newblock {\em Journal of Low Temperature Physics} {\bf 2008}, {\em
  153},~304--324.

\bibitem[Du \em{et~al.}(2008)Du, Skachko, and Andrei]{Du08}
Du, X.; Skachko, I.; Andrei, E.Y.
\newblock Josephson current and multiple Andreev reflections in graphene SNS
  junctions.
\newblock {\em Phys. Rev. B} {\bf 2008}, {\em 77},~184507.

\bibitem[Graham and T\'el(1986)]{Graham86}
Graham, R.; T\'el, T.
\newblock Nonequilibrium potential for coexisting attractors.
\newblock {\em Phys. Rev. A} {\bf 1986}, {\em 33},~1322--1337.
\newblock {\url{https://doi.org/10.1103/PhysRevA.33.1322}}.

\bibitem[Risken(1989)]{Risken89}
Risken, H.
\newblock {\em The Fokker-Planck Equation: Methods of solution and
  Applications}; Springer, Berlin,  1989.

\bibitem[Kumar and Carroll(1984)]{Kumar84}
Kumar, B.V.K.V.; Carroll, C.W.
\newblock {Performance Of Wigner Distribution Function Based Detection
  Methods}.
\newblock {\em Optical Engineering} {\bf 1984}, {\em 23},~732 -- 737.
\newblock {\url{https://doi.org/10.1117/12.7973372}}.

\bibitem[Piedjou~Komnang \em{et~al.}(2021)Piedjou~Komnang, Guarcello, Barone,
  Pagano, and Filatrella]{Piedjou21}
Piedjou~Komnang, A.S.; Guarcello, C.; Barone, C.; Pagano, S.; Filatrella, G.
\newblock Analysis of Josephson Junction Lifetimes for the Detection of Single
  Photons in a Thermal Noise Background.
\newblock In Proceedings of the 2021 IEEE 14th Workshop on Low Temperature
  Electronics (WOLTE),  2021, pp. 1--4.
\newblock {\url{https://doi.org/10.1109/WOLTE49037.2021.9555447}}.

\bibitem[Filatrella \em{et~al.}(2023)Filatrella, Barone, Carapella, Gatti,
  Granata, Guarcello, Mauro, Komnang, Pierro, Rettaroli, and
  Pagano]{Filatrella23}
Filatrella, G.; Barone, C.; Carapella, G.; Gatti, C.; Granata, V.; Guarcello,
  C.; Mauro, C.; Komnang, A.P.; Pierro, V.; Rettaroli, A.;  et~al.
\newblock Theoretical and Numerical Estimate of Signal-to-Noise Ratio in the
  Analysis of Josephson Junctions Lifetime for Photon Detection.
\newblock {\em IEEE Transactions on Applied Superconductivity} {\bf 2023}, {\em
  33},~1--5.
\newblock {\url{https://doi.org/10.1109/TASC.2022.3214500}}.

\bibitem[Kautz(1988)]{Kautz88}
Kautz, R.L.
\newblock Thermally induced escape: The principle of minimum available noise
  energy.
\newblock {\em Phys. Rev. A} {\bf 1988}, {\em 38},~2066--2080.
\newblock {\url{https://doi.org/10.1103/PhysRevA.38.2066}}.

\end{thebibliography}



\PublishersNote{}
\end{adjustwidth}
\end{document}


% Or table: \def\bC{{\beta_{_{C}}}}
\def\nL{{\mathcal{L}}}
\def\nW{{\mathcal{W}}}

%  LaTeX support: latex@mdpi.com 
%  For support, please attach all files needed for compiling as well as the log file, and specify your operating system, LaTeX version, and LaTeX editor.

%=================================================================
\documentclass[journal,article,submit,pdftex,moreauthors]{Definitions/mdpi} 
\renewcommand{\linenumbers}{}
%--------------------
% Class Options:
%--------------------
%----------
% journal
%----------
% Choose between the following MDPI journals:
% acoustics, actuators, addictions, admsci, adolescents, aerobiology, aerospace, agriculture, agriengineering, agrochemicals, agronomy, ai, air, algorithms, allergies, alloys, analytica, analytics, anatomia, animals, antibiotics, antibodies, antioxidants, applbiosci, appliedchem, appliedmath, applmech, applmicrobiol, applnano, applsci, aquacj, architecture, arm, arthropoda, arts, asc, asi, astronomy, atmosphere, atoms, audiolres, automation, axioms, bacteria, batteries, bdcc, behavsci, beverages, biochem, bioengineering, biologics, biology, biomass, biomechanics, biomed, biomedicines, biomedinformatics, biomimetics, biomolecules, biophysica, biosensors, biotech, birds, bloods, blsf, brainsci, breath, buildings, businesses, cancers, carbon, cardiogenetics, catalysts, cells, ceramics, challenges, chemengineering, chemistry, chemosensors, chemproc, children, chips, cimb, civileng, cleantechnol, climate, clinpract, clockssleep, cmd, coasts, coatings, colloids, colorants, commodities, compounds, computation, computers, condensedmatter, conservation, constrmater, cosmetics, covid, crops, cryptography, crystals, csmf, ctn, curroncol, cyber, dairy, data, ddc, dentistry, dermato, dermatopathology, designs, devices, diabetology, diagnostics, dietetics, digital, disabilities, diseases, diversity, dna, drones, dynamics, earth, ebj, ecologies, econometrics, economies, education, ejihpe, electricity, electrochem, electronicmat, electronics, encyclopedia, endocrines, energies, eng, engproc, entomology, entropy, environments, environsciproc, epidemiologia, epigenomes, est, fermentation, fibers, fintech, fire, fishes, fluids, foods, forecasting, forensicsci, forests, foundations, fractalfract, fuels, future, futureinternet, futurepharmacol, futurephys, futuretransp, galaxies, games, gases, gastroent, gastrointestdisord, gels, genealogy, genes, geographies, geohazards, geomatics, geosciences, geotechnics, geriatrics, grasses, gucdd, hazardousmatters, healthcare, hearts, hemato, hematolrep, heritage, higheredu, highthroughput, histories, horticulturae, hospitals, humanities, humans, hydrobiology, hydrogen, hydrology, hygiene, idr, ijerph, ijfs, ijgi, ijms, ijns, ijpb, ijtm, ijtpp, ime, immuno, informatics, information, infrastructures, inorganics, insects, instruments, inventions, iot, j, jal, jcdd, jcm, jcp, jcs, jcto, jdb, jeta, jfb, jfmk, jimaging, jintelligence, jlpea, jmmp, jmp, jmse, jne, jnt, jof, joitmc, jor, journalmedia, jox, jpm, jrfm, jsan, jtaer, jvd, jzbg, kidneydial, kinasesphosphatases, knowledge, land, languages, laws, life, liquids, literature, livers, logics, logistics, lubricants, lymphatics, machines, macromol, magnetism, magnetochemistry, make, marinedrugs, materials, materproc, mathematics, mca, measurements, medicina, medicines, medsci, membranes, merits, metabolites, metals, meteorology, methane, metrology, micro, microarrays, microbiolres, micromachines, microorganisms, microplastics, minerals, mining, modelling, molbank, molecules, mps, msf, mti, muscles, nanoenergyadv, nanomanufacturing,\gdef\@continuouspages{yes}} nanomaterials, ncrna, ndt, network, neuroglia, neurolint, neurosci, nitrogen, notspecified, %%nri, nursrep, nutraceuticals, nutrients, obesities, oceans, ohbm, onco, %oncopathology, optics, oral, organics, organoids, osteology, oxygen, parasites, parasitologia, particles, pathogens, pathophysiology, pediatrrep, pharmaceuticals, pharmaceutics, pharmacoepidemiology,\gdef\@ISSN{2813-0618}\gdef\@continuous pharmacy, philosophies, photochem, photonics, phycology, physchem, physics, physiologia, plants, plasma, platforms, pollutants, polymers, polysaccharides, poultry, powders, preprints, proceedings, processes, prosthesis, proteomes, psf, psych, psychiatryint, psychoactives, publications, quantumrep, quaternary, qubs, radiation, reactions, receptors, recycling, regeneration, religions, remotesensing, reports, reprodmed, resources, rheumato, risks, robotics, ruminants, safety, sci, scipharm, sclerosis, seeds, sensors, separations, sexes, signals, sinusitis, skins, smartcities, sna, societies, socsci, software, soilsystems, solar, solids, spectroscj, sports, standards, stats, std, stresses, surfaces, surgeries, suschem, sustainability, symmetry, synbio, systems, targets, taxonomy, technologies, telecom, test, textiles, thalassrep, thermo, tomography, tourismhosp, toxics, toxins, transplantology, transportation, traumacare, traumas, tropicalmed, universe, urbansci, uro, vaccines, vehicles, venereology, vetsci, vibration, virtualworlds, viruses, vision, waste, water, wem, wevj, wind, women, world, youth, zoonoticdis 
% For posting an early version of this manuscript as a preprint, you may use "preprints" as the journal. Changing "submit" to "accept" before posting will remove line numbers.

%---------
% article
%---------
% The default type of manuscript is "article", but can be replaced by: 
% abstract, addendum, article, book, bookreview, briefreport, casereport, comment, commentary, communication, conferenceproceedings, correction, conferencereport, entry, expressionofconcern, extendedabstract, datadescriptor, editorial, essay, erratum, hypothesis, interestingimage, obituary, opinion, projectreport, reply, retraction, review, perspective, protocol, shortnote, studyprotocol, systematicreview, supfile, technicalnote, viewpoint, guidelines, registeredreport, tutorial
% supfile = supplementary materials

%----------
% submit
%----------
% The class option "submit" will be changed to "accept" by the Editorial Office when the paper is accepted. This will only make changes to the frontpage (e.g., the logo of the journal will get visible), the headings, and the copyright information. Also, line numbering will be removed. Journal info and pagination for accepted papers will also be assigned by the Editorial Office.

%------------------
% moreauthors
%------------------
% If there is only one author the class option oneauthor should be used. Otherwise use the class option moreauthors.

%---------
% pdftex
%---------
% The option pdftex is for use with pdfLaTeX. Remove "pdftex" for (1) compiling with LaTeX & dvi2pdf (if eps figures are used) or for (2) compiling with XeLaTeX.

%=================================================================
% MDPI internal commands - do not modify
\firstpage{1} 
\makeatletter 
\setcounter{page}{\@firstpage} 
\makeatother
\pubvolume{1}
\issuenum{1}
\articlenumber{0}
\pubyear{2023}
\copyrightyear{2023}
%\externaleditor{Academic Editor: Firstname Lastname}
\datereceived{ } 
\daterevised{ } % Comment out if no revised date
\dateaccepted{ } 
\datepublished{ } 
%\datecorrected{} % For corrected papers: "Corrected: XXX" date in the original paper.
%\dateretracted{} % For corrected papers: "Retracted: XXX" date in the original paper.
\hreflink{https://doi.org/} % If needed use \linebreak
%\doinum{}
%\pdfoutput=1 % Uncommented for upload to arXiv.org

%=================================================================
% Add packages and commands here. The following packages are loaded in our class file: fontenc, inputenc, calc, indentfirst, fancyhdr, graphicx, epstopdf, lastpage, ifthen, float, amsmath, amssymb, lineno, setspace, enumitem, mathpazo, booktabs, titlesec, etoolbox, tabto, xcolor, colortbl, soul, multirow, microtype, tikz, totcount, changepage, attrib, upgreek, array, tabularx, pbox, ragged2e, tocloft, marginnote, marginfix, enotez, amsthm, natbib, hyperref, cleveref, scrextend, url, geometry, newfloat, caption, draftwatermark, seqsplit
% cleveref: load \crefname definitions after \begin{document}

%=================================================================
% Please use the following mathematics environments: Theorem, Lemma, Corollary, Proposition, Characterization, Property, Problem, Example, ExamplesandDefinitions, Hypothesis, Remark, Definition, Notation, Assumption
%% For proofs, please use the proof environment (the amsthm package is loaded by the MDPI class).

%=================================================================
% Full title of the paper (Capitalized)
\Title{Axion field influence on Josephson junction quasipotential}

% MDPI internal command: Title for citation in the left column
\TitleCitation{Axion field influence on Josephson junction quasipotential}

% Author Orchid ID: enter ID or remove command
\newcommand{\orcidauthorA}{0000-0003-2363-7699} 
\newcommand{\orcidauthorB}{0000-0001-5496-1518} 
\newcommand{\orcidauthorC}{0000-0002-6625-3989} 
\newcommand{\orcidauthorD}{0000-0002-4348-9956}
\newcommand{\orcidauthorE}{0000-0003-3546-8618} 
\newcommand{\orcidauthorF}{0000-0002-3683-2509} % Add \orcidA{} behind the author's name
%\newcommand{\orcidauthorB}{0000-0000-0000-000X} % Add \orcidB{} behind the author's name

% Authors, for the paper (add full first names)
\Author{Roberto Grimaudo $^{1}$\orcidA{}, Davide Valenti $^{1}$\orcidB{}, Bernardo Spagnolo $^{1,2}$\orcidC{}, Antonio Troisi $^{3}$\orcidD{}, Giovanni Filatrella $^{3,4}$\orcidE{} and Claudio Guarcello $^{5,4}$\orcidF{}}

%\longauthorlist{yes}

% MDPI internal command: Authors, for metadata in PDF
\AuthorNames{Roberto Grimaudo, Davide Valenti, Bernardo Spagnolo, Antonio Troisi, Giovanni Filatrella and Claudio Guarcello}

% MDPI internal command: Authors, for citation in the left column
\AuthorCitation{Grimaudo R.; Valenti D.; Spagnolo B.; Troisi A.; Filatrella G.; Guarcello C.}
% If this is a Chicago style journal: Lastname, Firstname, Firstname Lastname, and Firstname Lastname.

% Affiliations / Addresses (Add [1] after \address if there is only one affiliation.)
\address{%
$^{1}$ \quad Dipartimento di Fisica e Chimica ``E. Segr\`e'', Group of Theoretical Interdisciplinary Physics, Universit\`a degli Studi di Palermo, Viale delle Scienze, Ed. 18, I-90128 Palermo, Italy;\\
$^{2}$ \quad Lobachevskii University of Nizhnii Novgorod, 23 Gagarin Ave. Nizhnii Novgorod 603950 Russia;\\
$^{3}$ \quad Dep. of Sciences and Technologies, University of Sannio, Via De Sanctis, Benevento I-82100, Italy;\\
$^{4}$ \quad INFN, Sezione di Napoli Gruppo Collegato di Salerno, Complesso Universitario di Monte S. Angelo, I-80126 Napoli, Italy;\\
$^{5}$ \quad Dipartimento di Fisica ``E.R. Caianiello'', Universit\`a di Salerno, Via Giovanni Paolo II, 132, I-84084 Fisciano (SA), Italy;}

% Contact information of the corresponding author
\corres{Correspondence: cguarcello@unisa.it (C.G) giovanni.filatrella@unisannio.it (G.F.)}

% Current address and/or shared authorship
%\firstnote{Current address: Affiliation 3.} 
%\secondnote{These authors contributed equally to this work.}
% The commands \thirdnote{} till \eighthnote{} are available for further notes

%\simplesumm{} % Simple summary

%\conference{} % An extended version of a conference paper

% Abstract (Do not insert blank lines, i.e. \\) 
\abstract{The direct effect of an axion field on Josephson junctions is analyzed through the consequences on the effective potential barrier that prevents the junction from switching from the superconducting to the finite-voltage state. 
We describe a method to reliably compute the quasipotential with stochastic simulations, which allows to span the coupling parameter from weakly interacting axion to tight interactions.
As a result, we obtain that the axion field induces a change in the potential barrier, therefore determining a significant detectable effect for such a kind of elusive particle. }

% Keywords
\keyword{Josephson junction; axion; quasipotential; switching dynamics; noise; detection} 

% The fields PACS, MSC, and JEL may be left empty or commented out if not applicable
%\PACS{J0101}
%\MSC{}
%\JEL{}

%%%%%%%%%%%%%%%%%%%%%%%%%%%%%%%%%%%%%%%%%%
% Only for the journal Diversity
%\LSID{\url{http://}}

%%%%%%%%%%%%%%%%%%%%%%%%%%%%%%%%%%%%%%%%%%
% Only for the journal Applied Sciences
%\featuredapplication{Authors are encouraged to provide a concise description of the specific application or a potential application of the work. This section is not mandatory.}
%%%%%%%%%%%%%%%%%%%%%%%%%%%%%%%%%%%%%%%%%%

%%%%%%%%%%%%%%%%%%%%%%%%%%%%%%%%%%%%%%%%%%
% Only for the journal Data
%\dataset{DOI number or link to the deposited data set if the data set is published separately. If the data set shall be published as a supplement to this paper, this field will be filled by the journal editors. In this case, please submit the data set as a supplement.}
%\datasetlicense{License under which the data set is made available (CC0, CC-BY, CC-BY-SA, CC-BY-NC, etc.)}

%%%%%%%%%%%%%%%%%%%%%%%%%%%%%%%%%%%%%%%%%%
% Only for the journal Toxins
%\keycontribution{The breakthroughs or highlights of the manuscript. Authors can write one or two sentences to describe the most important part of the paper.}

%%%%%%%%%%%%%%%%%%%%%%%%%%%%%%%%%%%%%%%%%%
% Only for the journal Encyclopedia
%\encyclopediadef{For entry manuscripts only: please provide a brief overview of the entry title instead of an abstract.}

%%%%%%%%%%%%%%%%%%%%%%%%%%%%%%%%%%%%%%%%%%
% Only for the journal Advances in Respiratory Medicine
%\addhighlights{yes}
%\renewcommand{\addhighlights}{%

%\noindent This is an obligatory section in “Advances in Respiratory Medicine”, whose goal is to increase the discoverability and readability of the article via search engines and other scholars. Highlights should not be a copy of the abstract, but a simple text allowing the reader to quickly and simplified find out what the article is about and what can be cited from it. Each of these parts should be devoted up to 2~bullet points.\vspace{3pt}\\
%\textbf{What are the main findings?}
% \begin{itemize}[labelsep=2.5mm,topsep=-3pt]
% \item First bullet.
% \item Second bullet.
% \end{itemize}\vspace{3pt}
%\textbf{What is the implication of the main finding?}
% \begin{itemize}[labelsep=2.5mm,topsep=-3pt]
% \item First bullet.
% \item Second bullet.
% \end{itemize}
%}

%%%%%%%%%%%%%%%%%%%%%%%%%%%%%%%%%%%%%%%%%%
\begin{document}

%%%%%%%%%%%%%%%%%%%%%%%%%%%%%%%%%%%%%%%%%%
\section{Introduction}

Nowadays, in the search for cold dark matter candidates, among others, axion particles were theoretically predicted, but their detection remains elusive, for the very weak interaction that they are supposed to have with ordinary matter~\cite{Preskili83}.
Since Josephson junctions (JJs) proved to be very sensitive devices, close to detecting a single photon~\cite{Alesini20}, a natural idea was to exploit them to detect the electromagnetic field produced by axion decay~\cite{Rettaroli21}.
A different possibility, which is the framework for the present paper, is to exploit the direct interaction between JJ and axions~\cite{Bec13} in a detector~\cite{Grimaudo22,Gri23}. This scheme was also proposed as a key to understand unclear ``events'' in Josephson's response~\cite{Bec17,Hof04,Bae08,He11,Gol12,Bre13,Wang22} and would simplify the usual detection schemes through a reduced setting.
To date, the main idea behind this direct interaction is that the axion decays into the junction with a rather high probability, which would be achieved in a resonant cavity through the Primakoff effect only through a huge magnetic field (orders of magnitude above any realistic field): thus, a "Josephson cavity" is much more effective to detect the axion than a resonant cavity~\cite{Bec13} .
This scheme, however, misses a full theory of the Josephson-axion interaction and a detailed scheme of the changes induced in the JJ dynamics, which could possibly lead to detectable consequences. 
We here concentrate on the latter problem, assuming that the axion-JJ interaction exists, although the interaction parameter is unknown.
In the original proposal it was suggested to look at deviations from the locked dynamics of the JJ to an external radio frequency -- the so-called Shapiro steps -- that could be altered by the extra Cooper's pairs created by the axion decay. 
More recently, it has been proposed by some of the authors of the present paper to assume a different standpoint: to bias the JJ in the superconducting metastable state through a dc external drive and to observe the passages to the finite voltage state in the presence of the axion-JJ coupling, under the influence of thermal fluctuations~\cite{Grimaudo22}. 
The idea is that these switches are altered by the interaction of the JJ with the axions, and it is therefore possible to infer the existence of the axions if the switching is, in some statistical sense, changed; a more detailed analysis can also offer an estimate of the JJ-axion coupling from switching time measurements~\cite{Grimaudo22}. A successive approach assumed the JJ operating as a qubit, and in such a way the qubit-axion interaction being detected as axion-induced oscillations of the qubit state~\cite{Gri23}.
The general idea of exploiting a JJ to detect a weak signal, even embedded in a noisy background, is fairly well-established, a JJ being essentially a threshold device operating via a switching mechanism. The presence of a noise background is a condition typical of open systems, such as, for instance, biological and ecological systems~\cite{Val12,Lis15,Valenti16} and financial markets~\cite{Valenti18}, which has to be taken into account in view of better modeling their dynamics. Josephson junctions have been also proposed as noise detectors~\cite{Tob04,Pek04,Ank07,Suk07,Tim07,Hua07,Gra08,Fil10,Add12,Gua13,Gua19,Gua20,Gua21-2} and play a leading role in the search for possible protocols and schemes for the detection of single photons~\cite{Wal17,Kuz18,GuaBra19,GuaBraSol19,Rev20,Yab21,Pied21,Gua21,Pan22-1,Pan22-2}. 

In this paper we further investigate the consequences of a direct interaction between axions and JJ, to show that a certain quantity, namely, the \emph{quasipotential}~\cite{Graham85}, can be introduced for this non-equilibrium system and that it is possible to determine the quasipotential in the presence of the axions.
To investigate the quasipotential is an advantage, as it can be determined with numerical simulations at a relatively high noise intensity, i.e., high temperature. 
Indeed, the quasipotential (as the ordinary potential) is not affected by noise; if the quasipotential is known, it is possible to predict the average escape time at very low temperature with the Arrhenius law, with a considerable saving of simulation time~\cite{Kau88}. 
We do so with a twofold objective: in the first place, as already mentioned, to better understand the consequences that a direct interaction between JJ and axion would have, and therefore to pave the way towards a practical implementation of the device to detect axions; on the second hand, not less important, to compute the quasipotential of the axion-JJ system, a very convenient quantity in the analysis of low temperature devices (as already demonstrated for Shapiro steps~\cite{Kau96} and cavity-induced synchronization~\cite{Pou19}) to infer the properties at a very low noise and, consequently, very long escape times. In fact, the characteristic time scale of the system is of the order of $[1-10]\;\text{ps}$, being the inverse of Josephson characteristic frequency which, as we shall see later, generally falls in the range $[0.1-1]\;\text{THz}$; this means that numerical realizations, even close to real experimental times (which could take up to milliseconds, e.g., for experiments involving switching current distributions), require extremely long simulations. Moreover, these have to be repeated several times in order to obtain complete statistics. Indeed, stochastic analyses, such as the one we propose, require the repetition of the same experiment, i.e., of the same numerical simulation, for a reasonably large number of times under the same conditions, in order to allow for reliable statistical analyses. In conclusion, quasipotential analysis makes it possible to extract useful information at reasonably high temperatures, that means within reasonable simulation times, and then allows to extrapolate relevant information even at low temperatures, where numerical simulations time would become prohibitive. 

The paper is organized as follows: Sect.~\ref{Model} presents the model to describe: the JJ (Sect.~\ref{ModelJJ}), the axion field (Sect.~\ref{Modelaxions}), and the interacting axion-JJ system (Sect.~\ref{ModelJJaxions}). Sect.~\ref{Results} defines the quasipotential for this system and computes its behavior as a function of the interaction. 
Finally, in Sect.~\ref{Conclusions} conclusions are drawn.




\section{Model} 
\label{Model}

In this section we outline the models for the JJ, see Sect.~\ref{ModelJJ}, the axion, see Sect.~\ref{Modelaxions}, and their interaction, see Sect.~\ref{ModelJJaxions}. 
We show the potential of the JJ alone -- a cosinusoidal washboard potential, see Eq.~\eqref{Washboard App} below -- that exhibits an activation energy barrier, whose changes due to the interaction with the axions are the focus of the present work.
Also, some details for the numerical simulations of the stochastic equations are given in Sect.~\ref{ModelJJ}.


\subsection{RCSJ Model}
\label{ModelJJ}

Let us consider the usual model for a superconducting junction, schematically represented in Fig.~\ref{fig: Device}(a), given by the following equations~\cite{Bar82,Lik86}:
\begin{eqnarray}
\label{JJcurrent}
I_\varphi = I_c \sin{\varphi},\\
\label{JJvoltage}
V = \frac{\Phi_0}{2\pi}\frac{d\varphi}{dt},
\end{eqnarray}
where $\Phi_0=h/(2e)$ is the flux quantum, with $e$ and $h$ being the electron charge and the Planck constant, respectively, $I_c$ is the maximum Josephson current that can flow through the device, and $\varphi$ is the Josephson phase difference.

For a real device, one assumes for instance that the two superconductors have lateral dimensions $\nL$ and $\nW$ smaller than the Josephson penetration depth, $\lambda_{_{J}} = \sqrt{\Phi_0/(2\pi \mu_0 t_d J_c)}$ (here, $t_d=\lambda_{L,1}+\lambda_{L,2}+d$ is the effective magnetic thickness, with $\lambda_{L}$ and $d$ being the London penetration depths and the insulating layer thickness, respectively, $\mu_0$ is the vacuum permeability, and $J_c$ is the critical current area density).
The dynamics of the Josephson phase $\varphi$ for a dissipative, current-biased small JJ can thus be studied within the resistively and capacitively shunted junction (RCSJ) framework~\cite{Bar82,GuaVal15,Spa17,McC68,Gua19,Gua20}

%
%
\begin{equation}
\left ( \frac{\Phi_0}{2\pi} \right )^{\!\!2}\!\! C \frac{d^2 \varphi}{d t^2}+\left ( \frac{\Phi_0}{2\pi} \right )^{\!\!2}\!\!\frac{1}{R} \frac{d \varphi}{d t}+\frac{d }{d \varphi}U 
= \left ( \frac{\Phi_0}{2\pi} \right )( I_N+I_b),
\label{RCSJ App}
\end{equation}
%
with $R$ and $C$ the normal-state resistance and capacitance of the JJ, respectively, and $I_N$ and $I_b$ the thermal noise and the bias current, respectively.
The corresponding normalized dynamics can be reformulated (for sinusoidal potential of standard tunnel JJ, albeit other shapes are possible~\cite{Bee92}) through the equation:
%
%
\begin{equation}
\beta_c\frac{d^2 \varphi (\tau_c)}{d\tau_c^2}+ \frac{d \varphi (\tau_c)}{d\tau_c} +\frac{d }{d \varphi}\mathcal{U}(\varphi,i_b) = i_{n}(\tau_c) + i_b,
\label{RCSJnormOc}
\end{equation}
%
%
where time is normalized to the inverse of the characteristic frequency, that is $\tau_c = \omega_c~t$ with $\omega_c=\left ( 2\pi/\Phi_0 \right )I_cR$, $i_b= I_b / I_c$ and $i_n= I_n / I_c$ are the normalized external bias current and thermal noise current, and $\beta_c=\omega_c RC$ is the Stewart-McCumber parameter. 
We stress that the JJ response is usually quite fast, since the characteristic frequency of JJ falls within the range $ [0.1,1]\;\text{THz}$. 
Another way to obtain a dimensionless form of Eq.~\eqref{RCSJ App} consists in normalizing with respect to the plasma frequency $\omega_p=\sqrt{2eI_c/\hbar C}$.
In this case, time is normalized respect to the inverse plasma frequency, i.e., $\tau_p = \omega_p~t$, and the equation in normalized units contains a damping parameter $\alpha=\beta_{_{C}}^{-1/2}$, which multiplies the first time-derivative of the phase. 

The normalized potential, $\mathcal{U}$, is the so-called \emph{washboard potential}, which depends upon the normalized bias current, $i_b$, and the Josephson phase according to
%
% 
\begin{equation}
\mathcal{U}(\varphi,i_b)=\frac{U(\varphi,i_b)}{{E_{J_0}}}=\left [1- \cos(\varphi) -i_b\varphi\right ].
\label{Washboard App}
\end{equation}
%
%
The potential can be expressed in physical units defining the Josephson energy $E_{J_0}=\left ( \Phi_0/2\pi \right )I_c$. 
The resulting activation energy barrier, $\Delta U(i_b)$, confines the phase $\varphi$ in a metastable potential minimum and can be calculated as the difference between the maximum and minimum value of the normalized potential $U(\varphi,i_b)$, see Fig.~\ref{fig: Device}(b).
In units of $E_{J_0}$, it can be expressed as
%
% 
\begin{equation}
{\Delta \mathcal{U}(i_b)}=\frac{\Delta {U}(i_b)}{{E_{J_0}}}=2 \left [ \sqrt{1-i_b^2} -i_b\arccos(i_b)\right ].
\label{activationenergybarrier App}
\end{equation}
%
%
In the phase particle picture, the term $i_b$ represents the tilting of the potential profile; increasing $i_b$ the slope of the washboard increases and the height 
$\Delta \mathcal{U}(i_b)$ of the rightward potential barrier reduces, until this activation energy vanishes altogether for $i_b=1$, that is when the bias current reaches its critical value $I_c$. 
During the motion, different regimes are governed by the Stewart-McCumber parameter $\bC$.
 A highly damped (or overdamped) junction corresponds to $\bC\ll 1$, that is a small capacitance and/or a small resistance. 
 Instead, a junction with $\bC\gg 1$ has a large capacitance and/or a large resistance, and is weakly damped (or underdamped)~\footnote{With the alternative normalized mentioned before, the under- and overdamped regimes correspond to $\alpha \ll 1$ and $\alpha \gg 1$, respectively.}.
 For the purposes of this work, it is important to notice that in the underdamped regime once the phase has passed the barrier, a finite velocity, and hence, as per Eq.~\eqref{JJvoltage}, a finite voltage, appears.
It is therefore possible to detect the passage of the Josephson phase over the barrier through the appearance of a finite voltage, a key point to employ a JJ as a detector. In fact, the phase $\varphi$ itself is not directly accessible, while the passage over the barrier is signaled by a measurable voltage drop across the junction.
The procedure can be briefly schematized as follows. 
The JJ is prepared in some static configuration (at which corresponds a zero net voltage), exposed to some supposedly existing perturbation, and the junction is left to evolve. 
If the signal was not present, the passage only occurs under the effect of thermal noise, and it is given by the usual Kramers law~\cite{Kra40}. The presence of the signal is ascertained through deviations of the thermal escapes~\cite{Fil10,Pie21} -- as will be discussed in more details below.


In this work, the random current is modeled as a delta-correlated Gaussian white noise associated to the normal-state resistance of the junction, $R$, with the usual statistical properties:
\begin{eqnarray}
\label{averageD}
\langle i_n (\tau)\rangle\ &=& 0,\\
\label{sigmaD}
\langle i_n (\tau) i_n (\tau+\tilde{\tau})\rangle &=& 2D\,\delta (\tilde{\tau}). 
\end{eqnarray}
The amplitude of the normalized correlation is connected with the physical temperature $T$ through the relation~\cite{Bar82}
%
\begin{eqnarray}
\label{WNAmp}
D= \frac{k_BT}{R}\frac{\omega_c}{I^2_c},%=\frac{k_BT}{E_{J_0}},
\end{eqnarray}
%
here $k_B$ is the Boltzmann constant.
We note that, by normalizing time with respect to the characteristic frequency $\omega_c$ (as we do in our numerical simulations), the normalized noise intensity in Eq.~\eqref{WNAmp} can be recast as $D={k_BT}/{E_{J_0}}$, i.e, the ratio between the thermal energy and the Josephson coupling energy, $E_{J_0}$, without reference to the damping; instead, normalizing with respect to the plasma frequency, $\omega_p$, the normalized noise intensity becomes $D=\alpha{k_BT}/{E_{J_0}}$.
Thus, for Gaussian fluctuations of amplitude $D$, the stochastic independent increment employed in the numerical simulations reads
$\Delta i_N \simeq \sqrt{ 2 D \Delta t\; }\; N\left(0,1 \right)$.
Here, $N\left(0, 1 \right)$ indicates a Gaussianly distributed random function with zero mean and unit standard deviation. 




%Another way to obtain a dimensionless form of Eq.~\eqref{RCSJ App} consists in normalizing with respect to the plasma frequency $\omega_p=\sqrt{2eI_c/\hbar C}$.
%In the latter case, the normalized RCSJ equation \eqref{RCSJ App} reads
%
%
%\begin{equation}
%\frac{d^2 \varphi (\tau_p)}{d\tau_p^2}+ \alpha \frac{d \varphi (\tau_p)}{d\tau_p} + \sin \left [ \varphi\left ( \tau_p \right ) \right ] = i_{n}(\tau_p) + i_b,
%\label{RCSJnormOp App}
%\end{equation}
%
%
%where time is normalized respect to the inverse plasma frequency (that reads $\omega_p=\sqrt{{\Phi_0}/{2\pi C}}$ ), $\tau_p = \omega_p~t$, $\alpha=1/\sqrt(\omega_p~R~C)=1/\sqrt{\bC}$ is the damping parameter. 
%With this time normalization the under- and over-damped regimes correspond to $\alpha \ll 1$ and $\alpha \gg 1$, respectively.


\subsection{Axion}
\label{Modelaxions}

If one describes the axion field $a$ in the Friedman-Robertson-Walker metric, the equation of motion of the axion misalignment angle $\theta$ and the axion coupling constant $f_a$, namely $a=f_a\,\theta$~\cite{Sik83,Vis13}, reads
%
\begin{equation}
\frac{d^2 \theta (t)}{dt^2}+ H \frac{d \theta (t)}{dt} + \frac{m_a^2c^4}{\hbar^2} \sin \left [ \theta \left ( t \right ) \right ] = 0,
\label{AxionEq}
\end{equation}
%
complemented with spatial gradients that are here omitted. 
The above model includes the forcing term $\sin(\theta)$ due to quantum chromodynamics instanton effects. As one can observe, there is a formal similarity between the equation of motion governing the axion and the RCSJ systems, being the axion dynamics analogous to an unbiased RCSJ. Moreover, in normalized units, the parameters are of the same order of magnitude.
In Eq.~\eqref{AxionEq}, $H \approx 2 \times 10^{-18} ~ s^{-1}$ is the Hubble parameter and $m_a$ is the axion mass. 
The typical ranges of parameters that are allowed for dark matter axions are~\cite{sik09,Duf09}: 
$ 3 \times 10^9 ~ \text{GeV} \leq f_a \leq 10^{12} ~ \text{GeV}$ and $ 6 \times 10^{-6} ~ \text{eV} \leq m_a c^2 \leq 2 \times 10^{-3} ~ \text{eV}$. 
The prediction of the axion's mass, based on the average of the results from five independent condensed matter experiments, is $ m_a c^2 = (106 \pm 6) \mu eV$~\cite{,Hof04,Bae08,He11,Gol12,Bre13,Wang22}.
%

%
% Figure environment removed
%


\subsection{Axion-JJ System}
\label{ModelJJaxions}

According to the approach of Refs.~\cite{Grimaudo22,Yan20}, the interaction between axion and JJ can be formally written as 
%
%
\begin{subequations}\label{Orig Diff Eqs Syst}
\begin{align}
\ddot{\varphi} + a_1 \dot{\varphi} + b_1 \sin(\varphi) &= \gamma (\ddot{\theta} - \ddot{\varphi}) \label{Orig Diff Eqs Syst a},\\
\ddot{\theta} + a_2 \dot{\theta} + b_2 \sin(\theta) &= \gamma (\ddot{\varphi} - \ddot{\theta}),
\end{align}
\label{Orig Diff Eqs Syst}
\end{subequations}
%
%
%
\noindent where $(a_1, a_2)$ and $(b_1, b_2)$ are the dissipation and frequency parameters, respectively; $\gamma$ is the coupling constant between the two systems, whose values one wants to infer from the experiments. 
This model, which succeeds in explaining some experimental anomalies~\cite{Hof04,Bae08,He11,Gol12,Bre13,Wang22}, is based on the possibility to formally treat the axion as an effective JJ, and therefore to consider the system in Eqs.~\eqref{Orig Diff Eqs Syst} as equivalent to two capacitively coupled JJs~\cite{Blac09}.

To model the Josephson phase dynamics with a bias current and thermal fluctuations, Eqs.~\eqref{Orig Diff Eqs Syst} can be conveniently rewritten as 
%
\begin{subequations}
\label{Diff Eqs Syst omegac}
\begin{align}
%\label{DiffEqsJJ}
{\beta_c \over k_2}~\ddot{\varphi}+\dot{\varphi}+\sin(\varphi)+{k_1 \over k_2}~\varepsilon~\sin(\theta) &= i_b+i_n, \label{Diff Eqs Syst omegac phi} \\
%\label{DiffEqsaxion}
{\beta_c \over k_1}~\ddot{\theta}+\dot{\varphi}+\sin(\varphi)+{k_2 \over k_1}~\varepsilon~\sin(\theta) &= i_b+i_n, \label{Diff Eqs Syst omegac theta}
\end{align}
\end{subequations}
%
%
with 
%
%
\begin{equation}
\begin{aligned}
&k_1  =  {\gamma \over 1+2\gamma}, \qquad k_2  =  {1+\gamma \over 1+2\gamma}, \qquad \varepsilon  =  \left(\frac{m_ac^2}{\hbar\omega_p}\right)^2.
\label{epsilon}
\end{aligned}
\end{equation}
%
%&k_1 = {\gamma \over 1+2\gamma}, \qquad k_2 = {1+\gamma \over 1+2\gamma}, \\
%& \beta_c = \left( \frac{\omega_c}{\omega_p} \right)^2,\quad \varepsilon = \left({m_ac^2 \over \hbar\omega_p}\right)^2,
%
%
%
 The $\varepsilon$ parameter indicates the ratio between the axion energy and the 
Josephson plasma energy, $\hbar\omega_p$, and can be chosen -- within the JJ fabrication constraints -- to select the most convenient working point for the detection of an axion field interacting with the JJ. 
Indeed, the Josephson plasma frequency, and therefore the energy ratio $\varepsilon$, can be ``adjusted'' as needed, for $I_c$ can be lowered either by raising the temperature~\cite{Dub01} or by applying a magnetic field~\cite{Ber08} or a gate voltage~\cite{Du08}. 
The purpose is to determine the working point at which the system is most responsive to the axion perturbation. 
As the detection is performed through the analysis of the escape times, the response is measured in the precise sense that the distribution of the escape times for the axion-JJ coupled system deviates the most from the Josephson response in the absence of axions. In Ref.~\cite{Grimaudo22} it was in fact showed that at $\epsilon\lesssim1$ the average switching time approaches a minimum due to an axion-induced resonant activation phenomenon, for the occurrence of an effective frequency matching between axion and JJ, whereas the optimal experimental conditions for a JJ-based axion detection scheme should involve a Josephson plasma energy lower than the axion energy, i.e., $\epsilon>1$.


In this work, we trace the change in escape time (that makes the axion-JJ interaction detectable) back to the change in the effective potential barrier that confines the system to the static zero-voltage configuration. In the following Sect.~\ref{Results} we discuss how to compute the effective energy.
%, while we now discuss the numerical values of the axion parameters that enter Eq.(\ref{Diff Eqs Syst omegac}(b)). 

The integration of the stochastic Eqs.~\eqref{Diff Eqs Syst omegac} is performed with a finite-difference explicit method, using a time integration step $\Delta t=10^{-2}$, a maximum integration time $t_{max}=10^6$, initial conditions $\varphi(0)=\arcsin{(i_b)}$ 
 and $\theta(0)=\dot{\varphi}(0)=\dot{\theta}(0)=0$, and repeating each simulation $N=10^4$ times for each set of parameter values. Other parameters useful for the calculations are set as $\beta_c=100$ (i.e., an underdamped regime) and $\varepsilon=1$.


\section{Calculation of the Quasipotential}
\label{Results}

The non-equilibrium system in Eq.~\eqref{Orig Diff Eqs Syst} does not admit an ordinary potential. However, it is possible to define an effective, or quasi, potential that keeps the system in the static configuration. The axion-JJ coupled system eventually switches from the superconducting state ($V\propto d\varphi/dt=0$) to the resistive state ($V\propto d\varphi/dt\neq 0$), when the combined effect of noise and axion interaction allows the JJ to overcome the effective energy barrier. 
In this picture, one can think of the axion effect on the JJ as some perturbation that changes (more precisely, lowers, as we shall demonstrate below) the effective energy barrier. The advantage is that the change of this effective energy is independent of noise and therefore holds at any (sufficiently low) temperature.
The main difficulty is therefore to determine how the coupling between Eqs.~\eqref{Diff Eqs Syst omegac} amounts to a change in the quasipotential. 
To begin with, we show how to compute the quasipotential, that is the effective energy that must be overcome to induce a switch. 
The basic logic is as follows: suppose that the Arrhenius behavior~\cite{Graham85}
\begin{equation}
\label{Kramers}
\tau = \lim_{D\to0}\tau_0 \exp{\frac{\kappa}{D} }
\end{equation}
is valid. 
The hypothesis obviously holds for the "pure" JJ system, i.e, Eq.~\eqref{RCSJnormOc}, for which $\kappa\equiv\Delta \mathcal{U}$. 
One can make the further conjecture that the average escape time is exponential in the inverse of the noise intensity, with some coefficient $\kappa \neq \Delta \mathcal{U}$, such that, in the limit of small noise,
\begin{equation}
\label{exprelation}
 \log{\frac{\tau}{\tau_0}} = \kappa \frac{1}{D}.
\end{equation}
Under general assumptions~\footnote{For out-of-equilibrium systems that do not admit an ordinary potential, it is possible to define a non-equilibrium potential with properties analogous to those of an ordinary potential, as long as there is a single time-independent probability distribution that can be reached from any initial distribution as the weakly stochastic dynamical system approaches its steady state~\cite{Graham86}.}, it is fair to suppose that the exponential relation holds in the limit of vanishing noise, and one can thus interpret the coefficient $\kappa$ as an effective energy barrier:
\begin{equation}\label{quasipotential}
\Delta \mathcal{U}_{eff} \equiv \kappa = \lim_{D \rightarrow \infty}\frac{ \log{\tau/\tau_0}}{1/D}.
\end{equation}
In other words, the slope $\kappa$ of the relation \eqref{exprelation} can be interpreted as a {\it bona fide} potential barrier, in the limit of small noise. 
The advantage of this interpretation is twofold. 
On the one end, it gives a physically intuitive interpretation to the effect of the axion field; as we shall prove in the following subsection, the axion lowers the confining barrier, and the lowering is enhanced by the coupling $\gamma$. 
On the other hand the quasipotential offers a practical advantage, because it allows to extrapolate the results to very low values of noise, that is in the region where escape times are prohibitively long and extremely difficult to reach with simulations.
It is in fact enough to enter the regime of exponential decay to determine $\kappa$, and then to exploit such value for any lower value of the noise intensity $D$. 
The effective potential \eqref{quasipotential} can be numerically retrieved with several estimates of the escape time $\tau$ as a function of the noise amplitude $D$; in the plot $\log{\tau}$ vs $1/D$ the prefactor $\tau_0$ is the $y$-axis intercept and $\kappa$ the slope of the relationship. 
More precisely, the two quantities should be computed in the exponential regime, that is discarding the data for high $D$ to ensure that the asymptotic regime \eqref{Kramers} has been entered, as shown in Fig.~\ref{Kramers_fit}. 

% Figure environment removed


%\subsection{Behavior of the Quasi-Potential}

The axions are revealed through the difference between the electrical responses of an ideal JJ without external perturbation, and the same junction under the influence of an axion field. This difference is determined on average by the quasipotential. However, for any finite sampling the actual measured average is subject to fluctuations. Therefore, the better detection is obtained with the method to which pertains the best signal-to-noise ratio (SNR), where the signal is the measured average difference and the noise is due to the fluctuations around the average, i.e, the standard deviation of the sampled mean. The SNR thus computed is often measured through the Kumar-Carroll index~\cite{Kumar84}. However, for simplicity in this work we merely observe the effect of axion on the effective potential felt by the axion-JJ system.


For the uncoupled regime, $\gamma =0$, the numerical quasipotential should be equal to the ordinary potential; in fact, in Fig.~\ref{fig:QP} one only observes a modest discrepancy, $\eta<10\%$, due to the finite temperature (the quasipotential should be computed in the limit $D^{-1}\rightarrow \infty$, or more accurately $\Delta \mathcal{U}_{eff}/D \rightarrow \infty$) and to the finite number of realizations over which the average is calculated. 
This discrepancy is obtained as the percentage difference between $\Delta \mathcal{U}_{eff}$ estimated from the data of Fig.~\ref{fig:QP} at the lowest noise, i.e., $\gamma=0.001$, and the analytical washboard activation energy $\Delta \mathcal{U}$, see Eq.~\eqref{activationenergybarrier App}, that is $ \eta=\left ( \Delta \mathcal{U}-\Delta \mathcal{U}_{eff} \right )/\Delta \mathcal{U}=\{7.4\%, 8.0\%,\text{ and }6.1\%\}$ for $i_b=\{0.1, 0.5,\text{ and }0.8\}$, respectively.
It is indeed remarkable that, despite the relatively high noise ($\Delta \mathcal{U}_{eff}/ D \in [3-6]$ from data in Figs.~\ref{Kramers_fit} and~\ref{fig:QP}), the agreement is good.
This observation highlights the advantage of the quasipotential method, because one can use short escape times to effectively evaluate the effective quasipotential through Eq.~\eqref{quasipotential}. Shortly, we have demonstrated that the quasipotential for the JJ-axion system can be numerically evaluated with relative ease, while the result can be extrapolated to much lower values of noise, and hence to much longer escape times.

Finally, we want to exploit the estimation of the effect of the quasipotential for the detection of the axion. 
This is summarized in Fig.~\ref{fig:QP}, where the slope of the escape times versus $\gamma$ is identified with the quasipotential.
For each of the three different values of the bias current considered, $i_b=\{0.1, 0.5,\text{ and }0.8\}$, it is proven that the coupling $\gamma$ lowers the effective energy barrier. 
The effect seems more uniform for $i_b = 0.1$, while for shallow barriers [e.g., see $i_b=0.8$ in Fig.~\ref{fig:QP}c)] the change is more evident only for $\gamma\gtrsim0.1$. In fact, the greater $i_b$, the higher the $\gamma$ value above which the coupling with the axion produces an effect on the quasipotential. Interestingly, the bias current, which actually represents an easily controllable parameter, was also proven to have a significant impact on the emerging of resonances in the switching times discussed in Ref.~\cite{Grimaudo22}.
It is therefore evident from our simulations that a lower bias current is more convenient. Moreover, a different $\gamma$ gives quite different quasipotential values, such that the overall effect of the coupling between JJ and axion is to reduce the effective height of the potential barrier. 
%\newpage



% Figure environment removed

\section{Conclusions}
\label{Conclusions}

It has been shown that if the direct interaction between a solid state superconducting device, i.e., a Josephson junction, and the dark matter candidate named axion is assumed, a modification of the response to noise of the former arises \cite{Grimaudo22,Gri23}. This aspect offers an opportunity for the detection of this elusive particle.
Using JJs to detect axions has been shown to be beneficial for several reasons. First, JJs are superconducting devices that can operate at very low temperatures, and, hence, at very low noise. Second, they are very fast elements, with typical characteristic frequencies from GHz to THz, and therefore large amount of data can be collected in a brief time. Third, some parameters of the Josephson device can be adjusted to tune the coupling with the axion. The bias current is a further degree of freedom that can be exploited to tune the effective barrier, as shown in Fig.~\ref{fig:QP}.

In a nutshell, axion signature can be sketched as follows: the JJ-axion interaction facilitates, in the presence of noise, the escapes of the Josephson phase from the superconducting to the finite-voltage state. 
This change can be described through the quasipotential, which is an effective energy barrier that summarizes the response to noise in the limit of small fluctuations. 
The introduction of the quasipotential allows to extrapolate the behavior at very low noise values, at which numerical simulations become prohibitively long. 
It is thus possible to reconstruct the response at low temperature through simulations performed at relatively high values of fluctuations, an advantage that has been already exploited in several applications, as for instance the Josephson voltage standard for which even very rare escapes are relevant to maintain the high accuracy required by metrological standards~\cite{Kau96}. 
Analogously, for weak signal detection it is important to mimic occurrence of rare events in a quite noisy environment~\cite{Pie21}.
In this work we have extended the method to the interaction between an underdamped JJ and an axion field. 
Within this framework, it has been possible to demonstrate that the interaction is summarized by a quasipotential, and to determine the behavior of this effect through the quasipotential as a function of the JJ-axion interaction. 
In particular, it has been established that the quasipotential depends upon the strength of the interaction, in the precise meaning that the stronger the interaction, the lower the quasipotential effective barrier.
An ideal experiment comparing the escape times of a JJ subject only to noise with those of a junction subject to the same noise and an axion field could  reveal the presence of axions. This would be evidenced by a decrease in the mean escape time, and the magnitude of this decrease would provide a quantitative estimate of the JJ-axion interaction.
Furthermore, the decrease should persist at any noise value, even if very small, i.e. as small as necessary to achieve the desired SNR. Finally, numerical simulations demonstrated that the observed behavior holds for different bias points, thus providing an additional tunable parameter for experimental setup.



A word of caution: the change in potential energy is only one of the ingredients for an accurate calculation of SNR, which requires the estimation of fluctuations for finite sampling, as previously done for Josephson-based single photon detection schemes~\cite{Pied21,Gua21,Piedjou21}, and provides additional information on the sample size needed to determine exclusion graphs or the residual operator characteristic~\cite{Add12,Filatrella23}.

A further refinement of this approach could be achieved through the \emph{principle of minimum available noise energy}~\cite{Kautz88}. Shortly, the idea could be to follow the path of the unperturbed JJ to determine the critical point, i.e., the separatrix between the two basins of attraction belonging to different stable points. By doing this for the deterministic noise-free system, it is possible to calculate the energy in the minimum and the minimum work required to bring the system from the initial stable point to this critical point, i.e., exactly the quasipotential. Any other work involving a different trajectory between these two stable points is larger than the actual quasipotential.
If the method provides a close estimate of the quasipotential through stochastic lengthy numerical calculations, it may be exploited to search the vast parameter space with straightforward deterministic analysis. 

To conclude, our study aims to provide further insights into the interplay between noise, switching dynamic of the JJ and available signal statistical properties, to enhance the understanding of axion JJ-based detector, and at the same time its robustness and reliability. 
Through the characterization of the noise-induced effects and the understanding of their implications, we wish to contribute to the development of better detectors and of quantum technology devices with improved performances.








%\bibliography{bibliofile}

%%%%%%%%%%%%%%%%%%%%%%%%%%%%%%%%%%%%%%%%%%
\vspace{6pt} 

%%%%%%%%%%%%%%%%%%%%%%%%%%%%%%%%%%%%%%%%%%
%% optional
%\supplementary{The following supporting information can be downloaded at:  \linksupplementary{s1}, Figure S1: title; Table S1: title; Video S1: title.}

% Only for the journal Methods and Protocols:
% If you wish to submit a video article, please do so with any other supplementary material.
% \supplementary{The following supporting information can be downloaded at: \linksupplementary{s1}, Figure S1: title; Table S1: title; Video S1: title. A supporting video article is available at doi: link.}

%%%%%%%%%%%%%%%%%%%%%%%%%%%%%%%%%%%%%%%%%%
\authorcontributions{Conceptualization, R.G. and C.G.; methodology, R.G., D.V., and C.G.; software, R.G.; validation, R.G., D.V. and C.G.; formal analysis, R.G.; investigation, R.G. and C.G.; resources, D.V.; data curation, R.G. and C.G.; writing---original draft preparation, G.F. and C.G.; writing---review and editing, R.G., D.V., B.S., A.T., G.F. and C.G.; visualization, R.G., D.V., B.S., A.T., G.F. and C.G.; supervision, D.V., B.S., G.F. and C.G; funding acquisition, D.V. All authors have read and agreed to the published version of the manuscript.}

%\funding{Please add: ``This research received no external funding'' or ``This research was funded by NAME OF FUNDER grant number XXX.'' and  and ``The APC was funded by XXX''. Check carefully that the details given are accurate and use the standard spelling of funding agency names at \url{https://search.crossref.org/funding}, any errors may affect your future funding.}

%\institutionalreview{In this section, you should add the Institutional Review Board Statement and approval number, if relevant to your study. You might choose to exclude this statement if the study did not require ethical approval. Please note that the Editorial Office might ask you for further information. Please add “The study was conducted in accordance with the Declaration of Helsinki, and approved by the Institutional Review Board (or Ethics Committee) of NAME OF INSTITUTE (protocol code XXX and date of approval).” for studies involving humans. OR “The animal study protocol was approved by the Institutional Review Board (or Ethics Committee) of NAME OF INSTITUTE (protocol code XXX and date of approval).” for studies involving animals. OR “Ethical review and approval were waived for this study due to REASON (please provide a detailed justification).” OR “Not applicable” for studies not involving humans or animals.}

%\informedconsent{Any research article describing a study involving humans should contain this statement. Please add ``Informed consent was obtained from all subjects involved in the study.'' OR ``Patient consent was waived due to REASON (please provide a detailed justification).'' OR ``Not applicable'' for studies not involving humans. You might also choose to exclude this statement if the study did not involve humans.

%Written informed consent for publication must be obtained from participating patients who can be identified (including by the patients themselves). Please state ``Written informed consent has been obtained from the patient(s) to publish this paper'' if applicable.}

\dataavailability{The data presented in this study are available on reasonable request from the corresponding authors.} 

\acknowledgments{R.G. acknowledges financial support from the PRIN Project PRJ-0232 - Impact of Climate Change on the biogeochemistry of Contaminants in the Mediterranean sea (ICCC). All the authors acknowledge the support of the Ministry of University and Research of Italian Government.}

\conflictsofinterest{The authors declare no conflict of interest.} 

%%%%%%%%%%%%%%%%%%%%%%%%%%%%%%%%%%%%%%%%%%
%% Optional
%\sampleavailability{Samples of the compounds ... are available from the authors.}

%% Only for journal Encyclopedia
%\entrylink{The Link to this entry published on the encyclopedia platform.}

\abbreviations{Abbreviations}{
The following abbreviations are used in this manuscript:\\

\noindent 
\begin{tabular}{@{}ll}
JJ & Josephson junction\\
RCSJ & resistively and capacitively shunted junction
\end{tabular}
}

%%%%%%%%%%%%%%%%%%%%%%%%%%%%%%%%%%%%%%%%%%


%%%%%%%%%%%%%%%%%%%%%%%%%%%%%%%%%%%%%%%%%%
\begin{adjustwidth}{-\extralength}{0cm}
%\printendnotes[custom] % Un-comment to print a list of endnotes

%\bibliography{bibliofile}

\begin{thebibliography}{999}

\bibitem[Preskill \em{et~al.}(1983)Preskill, Wise, and Wilczek]{Preskili83}
Preskill, J.; Wise, M.B.; Wilczek, F.
\newblock Cosmology of the invisible axion.
\newblock {\em Physics Letters B} {\bf 1983}, {\em 120},~127--132.
\newblock {\url{https://doi.org/https://doi.org/10.1016/0370-2693(83)90637-8}}.

\bibitem[Alesini \em{et~al.}(2020)Alesini, Babusci, Barone, B., Beretta,
  Bianchini, Castellano, Chiarello, Di~Gioacchino, Falferi, Felici, Filatrella,
  Foggetta, Gallo, Gatti, Giazotto, Lamanna, Ligabue, Ligato, Ligi, Maccarrone,
  Margesin, Mattioli, Monticone, Oberto, Pagano, Paolucci, Rajteri, Rettaroli,
  Rolandi, Spagnolo, Toncelli, and Torrioli]{Alesini20}
Alesini, D.; Babusci, D.; Barone, C.; B., B.; Beretta, M.M.; Bianchini, L.;
  Castellano, G.; Chiarello, F.; Di~Gioacchino, D.; Falferi, P.;  et~al.
\newblock Status of the SIMP Project: Toward the Single Microwave Photon
  Detection.
\newblock {\em Journal of Low Temperature Physics} {\bf 2020}, {\em 199},~348--
  354.
\newblock {\url{https://doi.org/10.1007/s10909-020-02381-x}}.

\bibitem[Rettaroli \em{et~al.}(2021)Rettaroli, Alesini, Babusci, Barone,
  Buonomo, Beretta, Castellano, Chiarello, Di~Gioacchino, Felici, Filatrella,
  Foggetta, Gallo, Gatti, Ligi, Maccarrone, Mattioli, Pagano, Tocci, and
  Torrioli]{Rettaroli21}
Rettaroli, A.; Alesini, D.; Babusci, D.; Barone, C.; Buonomo, B.; Beretta,
  M.M.; Castellano, G.; Chiarello, F.; Di~Gioacchino, D.; Felici, G.;  et~al.
\newblock Josephson Junctions as Single Microwave Photon Counters: Simulation
  and Characterization.
\newblock {\em Instruments} {\bf 2021}, {\em 5}.
\newblock {\url{https://doi.org/10.3390/instruments5030025}}.

\bibitem[Beck(2013)]{Bec13}
Beck, C.
\newblock Possible Resonance Effect of Axionic Dark Matter in Josephson
  Junctions.
\newblock {\em Phys. Rev. Lett.} {\bf 2013}, {\em 111},~231801.

\bibitem[Grimaudo \em{et~al.}(2022)Grimaudo, Valenti, Spagnolo, Filatrella, and
  Guarcello]{Grimaudo22}
Grimaudo, R.; Valenti, D.; Spagnolo, B.; Filatrella, G.; Guarcello, C.
\newblock Josephson-junction-based axion detection through resonant activation.
\newblock {\em Phys. Rev. D} {\bf 2022}, {\em 105},~033007.
\newblock {\url{https://doi.org/10.1103/PhysRevD.105.033007}}.

\bibitem[Grimaudo \em{et~al.}(2023)Grimaudo, Valenti, Filatrella, Spagnolo, and
  Guarcello]{Gri23}
Grimaudo, R.; Valenti, D.; Filatrella, G.; Spagnolo, B.; Guarcello, C.
\newblock Coupled quantum pendula as a possible model for
  Josephson-junction-based axion detection.
\newblock {\em Chaos, Solitons \& Fractals} {\bf 2023}, {\em 173},~113745.
\newblock {\url{https://doi.org/https://doi.org/10.1016/j.chaos.2023.113745}}.

\bibitem[Beck(2017)]{Bec17}
Beck, C.
\newblock Possible resonance effect of dark matter axions in SNS Josephson
  junctions.
\newblock {\em PoS} {\bf 2017}, {\em EPS-HEP2017},~058.

\bibitem[Hoffmann \em{et~al.}(2004)Hoffmann, Lefloch, Sanquer, and
  Pannetier]{Hof04}
Hoffmann, C.; Lefloch, F.; Sanquer, M.; Pannetier, B.
\newblock Mesoscopic transition in the shot noise of diffusive
  superconductor--normal-metal--superconductor junctions.
\newblock {\em Phys. Rev. B} {\bf 2004}, {\em 70},~180503.

\bibitem[Bae \em{et~al.}(2008)Bae, Dinsmore~III, Sahu, Lee, and
  Bezryadin]{Bae08}
Bae, M.H.; Dinsmore~III, R.C.; Sahu, M.; Lee, H.J.; Bezryadin, A.
\newblock Zero-crossing Shapiro steps in high-${T}_{c}$ superconducting
  microstructures tailored by a focused ion beam.
\newblock {\em Phys. Rev. B} {\bf 2008}, {\em 77},~144501.

\bibitem[He \em{et~al.}(2011)He, Wang, and Chan]{He11}
He, L.; Wang, J.; Chan, M.H.
\newblock Shapiro Steps in the Absence of Microwave Radiation.
\newblock {\em arXiv preprint arXiv:1107.0061} {\bf 2011}.

\bibitem[Golikova \em{et~al.}(2012)Golikova, H\"ubler, Beckmann, Batov,
  Karminskaya, Kupriyanov, Golubov, and Ryazanov]{Gol12}
Golikova, T.E.; H\"ubler, F.; Beckmann, D.; Batov, I.E.; Karminskaya, T.Y.;
  Kupriyanov, M.Y.; Golubov, A.A.; Ryazanov, V.V.
\newblock Double proximity effect in hybrid planar superconductor-(normal
  metal/ferromagnet)-superconductor structures.
\newblock {\em Phys. Rev. B} {\bf 2012}, {\em 86},~064416.

\bibitem[Bretheau \em{et~al.}(2013)Bretheau, Girit, Pothier, Esteve, and
  Urbina]{Bre13}
Bretheau, L.; Girit, {\c{C}}.{\"O}.; Pothier, H.; Esteve, D.; Urbina, C.
\newblock Exciting Andreev pairs in a superconducting atomic contact.
\newblock {\em Nature} {\bf 2013}, {\em 499},~312--315.

\bibitem[Wang \em{et~al.}(2022)Wang, Wang, and Wang]{Wang22}
Wang, J.; Wang, Z.; Wang, P.
\newblock Magnetic field enhanced critical current and subharmonic structures
  in dissipative superconducting gold nanowires.
\newblock {\em Quantum Frontiers} {\bf 2022}, {\em 1},~21.
\newblock {\url{https://doi.org/10.1007/s44214-022-00021-x}}.

\bibitem[Valenti \em{et~al.}(2012)Valenti, Denaro, La~Cognata, Spagnolo,
  Bonanno, Basilone, Mazzola, Zgozi, and Aronica]{Val12}
Valenti, D.; Denaro, G.; La~Cognata, A.; Spagnolo, B.; Bonanno, A.; Basilone,
  G.; Mazzola, S.; Zgozi, S.; Aronica, S.
\newblock Picophytoplankton Dynamics in Noisy Marine Environment.
\newblock {\em Acta Phys. Pol. B} {\bf 2012}, {\em 43},~1227.
\newblock {\url{https://doi.org/https://doi.org/10.5506/APhysPolB.43.1227}}.

\bibitem[Lisowski \em{et~al.}(2015)Lisowski, Valenti, Spagnolo, Bier, and
  Gudowska-Nowak]{Lis15}
Lisowski, B.; Valenti, D.; Spagnolo, B.; Bier, M.; Gudowska-Nowak, E.
\newblock Stepping molecular motor amid L\'evy white noise.
\newblock {\em Phys. Rev. E} {\bf 2015}, {\em 91},~042713.
\newblock {\url{https://doi.org/10.1103/PhysRevE.91.042713}}.

\bibitem[Valenti \em{et~al.}(2016)Valenti, Denaro, Spagnolo, Mazzola, Basilone,
  Conversano, Brunet, and Bonanno]{Valenti16}
Valenti, D.; Denaro, G.; Spagnolo, B.; Mazzola, S.; Basilone, G.; Conversano,
  F.; Brunet, C.; Bonanno, A.
\newblock Stochastic models for phytoplankton dynamics in Mediterranean Sea.
\newblock {\em Ecological Complexity} {\bf 2016}, {\em 27},~84--103.
\newblock Mathematical Ecology and Epidemiology,
  {\url{https://doi.org/https://doi.org/10.1016/j.ecocom.2015.06.001}}.

\bibitem[Valenti \em{et~al.}(2018)Valenti, Fazio, and Spagnolo]{Valenti18}
Valenti, D.; Fazio, G.; Spagnolo, B.
\newblock Stabilizing effect of volatility in financial markets.
\newblock {\em Phys. Rev. E} {\bf 2018}, {\em 97},~062307.
\newblock {\url{https://doi.org/10.1103/PhysRevE.97.062307}}.

\bibitem[Tobiska and Nazarov(2004)]{Tob04}
Tobiska, J.; Nazarov, Y.V.
\newblock Josephson Junctions as Threshold Detectors for Full Counting
  Statistics.
\newblock {\em Phys. Rev. Lett.} {\bf 2004}, {\em 93},~106801.
\newblock {\url{https://doi.org/10.1103/PhysRevLett.93.106801}}.

\bibitem[Pekola(2004)]{Pek04}
Pekola, J.P.
\newblock Josephson Junction as a Detector of Poissonian Charge Injection.
\newblock {\em Phys. Rev. Lett.} {\bf 2004}, {\em 93},~206601.
\newblock {\url{https://doi.org/10.1103/PhysRevLett.93.206601}}.

\bibitem[Ankerhold(2007)]{Ank07}
Ankerhold, J.
\newblock Detecting Charge Noise with a Josephson Junction: A Problem of
  Thermal Escape in Presence of Non-Gaussian Fluctuations.
\newblock {\em Phys. Rev. Lett.} {\bf 2007}, {\em 98},~036601.
\newblock {\url{https://doi.org/10.1103/PhysRevLett.98.036601}}.

\bibitem[Sukhorukov and Jordan(2007)]{Suk07}
Sukhorukov, E.V.; Jordan, A.N.
\newblock Stochastic Dynamics of a Josephson Junction Threshold Detector.
\newblock {\em Phys. Rev. Lett.} {\bf 2007}, {\em 98},~136803.
\newblock {\url{https://doi.org/10.1103/PhysRevLett.98.136803}}.

\bibitem[Timofeev \em{et~al.}(2007)Timofeev, Meschke, Peltonen, Heikkil\"a, and
  Pekola]{Tim07}
Timofeev, A.V.; Meschke, M.; Peltonen, J.T.; Heikkil\"a, T.T.; Pekola, J.P.
\newblock Wideband Detection of the Third Moment of Shot Noise by a Hysteretic
  Josephson Junction.
\newblock {\em Phys. Rev. Lett.} {\bf 2007}, {\em 98},~207001.
\newblock {\url{https://doi.org/10.1103/PhysRevLett.98.207001}}.

\bibitem[Huard \em{et~al.}(2007)Huard, Pothier, Birge, Esteve, Waintal, and
  Ankerhold]{Hua07}
Huard, B.; Pothier, H.; Birge, N.O.; Esteve, D.; Waintal, X.; Ankerhold, J.
\newblock Josephson junctions as detectors for non-Gaussian noise.
\newblock {\em Annalen der Physik} {\bf 2007}, {\em 16},~736--750.

\bibitem[Grabert(2008)]{Gra08}
Grabert, H.
\newblock Theory of a Josephson junction detector of non-Gaussian noise.
\newblock {\em Phys. Rev. B} {\bf 2008}, {\em 77},~205315.
\newblock {\url{https://doi.org/10.1103/PhysRevB.77.205315}}.

\bibitem[Filatrella and Pierro(2010)]{Fil10}
Filatrella, G.; Pierro, V.
\newblock Detection of noise-corrupted sinusoidal signals with Josephson
  junctions.
\newblock {\em Phys. Rev. E} {\bf 2010}, {\em 82},~046712.
\newblock {\url{https://doi.org/10.1103/PhysRevE.82.046712}}.

\bibitem[Addesso \em{et~al.}(2012)Addesso, Filatrella, and Pierro]{Add12}
Addesso, P.; Filatrella, G.; Pierro, V.
\newblock Characterization of escape times of Josephson junctions for signal
  detection.
\newblock {\em Phys. Rev. E} {\bf 2012}, {\em 85},~016708.
\newblock {\url{https://doi.org/10.1103/PhysRevE.85.016708}}.

\bibitem[Guarcello \em{et~al.}(2013)Guarcello, Valenti, Augello, and
  Spagnolo]{Gua13}
Guarcello, C.; Valenti, D.; Augello, G.; Spagnolo, B.
\newblock The Role of Non-Gaussian Sources in the Transient Dynamics of Long
  Josephson Junctions.
\newblock {\em Acta Phys. Pol. B} {\bf 2013}, {\em 44},~997--1005.
\newblock {\url{https://doi.org/10.5506/APhysPolB.44.997}}.

\bibitem[Guarcello \em{et~al.}(2019)Guarcello, Valenti, Spagnolo, Pierro, and
  Filatrella]{Gua19}
Guarcello, C.; Valenti, D.; Spagnolo, B.; Pierro, V.; Filatrella, G.
\newblock Josephson-based Threshold Detector for L\'evy-Distributed Current
  Fluctuations.
\newblock {\em Physical Review Applied} {\bf 2019}, {\em 11},~044078.

\bibitem[Guarcello \em{et~al.}(2020)Guarcello, Filatrella, Spagnolo, Pierro,
  and Valenti]{Gua20}
Guarcello, C.; Filatrella, G.; Spagnolo, B.; Pierro, V.; Valenti, D.
\newblock Voltage drop across Josephson junctions for L\'evy noise detection.
\newblock {\em Physical Review Research} {\bf 2020}, {\em 2},~043332.

\bibitem[Guarcello(2021)]{Gua21-2}
Guarcello, C.
\newblock L\'evy noise effects on Josephson junctions.
\newblock {\em Chaos Solitons Fract} {\bf 2021}, {\em 153},~111531.
\newblock {\url{https://doi.org/https://doi.org/10.1016/j.chaos.2021.111531}}.

\bibitem[Walsh \em{et~al.}(2017)Walsh, Efetov, Lee, Heuck, Crossno, Ohki, Kim,
  Englund, and Fong]{Wal17}
Walsh, E.D.; Efetov, D.K.; Lee, G.H.; Heuck, M.; Crossno, J.; Ohki, T.A.; Kim,
  P.; Englund, D.; Fong, K.C.
\newblock Graphene-Based Josephson-Junction Single-Photon Detector.
\newblock {\em Phys. Rev. Applied} {\bf 2017}, {\em 8},~024022.
\newblock {\url{https://doi.org/10.1103/PhysRevApplied.8.024022}}.

\bibitem[{Kuzmin} \em{et~al.}(2018){Kuzmin}, {Sobolev}, {Gatti}, {Di
  Gioacchino}, {Crescini}, {Gordeeva}, and {Il'ichev}]{Kuz18}
{Kuzmin}, L.S.; {Sobolev}, A.S.; {Gatti}, C.; {Di Gioacchino}, D.; {Crescini},
  N.; {Gordeeva}, A.; {Il'ichev}, E.
\newblock Single Photon Counter Based on a Josephson Junction at 14 GHz for
  Searching Galactic Axions.
\newblock {\em IEEE Transactions on Applied Superconductivity} {\bf 2018}, {\em
  28},~1--5.

\bibitem[Guarcello \em{et~al.}(2019{\natexlab{a}})Guarcello, Braggio, Solinas,
  and Giazotto]{GuaBra19}
Guarcello, C.; Braggio, A.; Solinas, P.; Giazotto, F.
\newblock Nonlinear Critical-Current Thermal Response of an Asymmetric
  Josephson Tunnel Junction.
\newblock {\em Phys. Rev. Applied} {\bf 2019}, {\em 11},~024002.

\bibitem[Guarcello \em{et~al.}(2019{\natexlab{b}})Guarcello, Braggio, Solinas,
  Pepe, and Giazotto]{GuaBraSol19}
Guarcello, C.; Braggio, A.; Solinas, P.; Pepe, G.P.; Giazotto, F.
\newblock Josephson-Threshold Calorimeter.
\newblock {\em Phys. Rev. Applied} {\bf 2019}, {\em 11},~054074.

\bibitem[Revin \em{et~al.}(2020)Revin, Pankratov, Gordeeva, Yablokov, Rakut,
  Zbrozhek, and Kuzmin]{Rev20}
Revin, L.S.; Pankratov, A.L.; Gordeeva, A.V.; Yablokov, A.A.; Rakut, I.V.;
  Zbrozhek, V.O.; Kuzmin, L.S.
\newblock Microwave photon detection by an Al Josephson junction.
\newblock {\em Beilstein Journal of Nanotechnology} {\bf 2020}, {\em
  11},~960--965.

\bibitem[Yablokov \em{et~al.}(2021)Yablokov, Glushkov, Pankratov, Gordeeva,
  Kuzmin, and Il’ichev]{Yab21}
Yablokov, A.; Glushkov, E.; Pankratov, A.; Gordeeva, A.; Kuzmin, L.;
  Il’ichev, E.
\newblock Resonant response drives sensitivity of Josephson escape detector.
\newblock {\em Chaos Solitons Fract} {\bf 2021}, {\em 148},~111058.
\newblock {\url{https://doi.org/https://doi.org/10.1016/j.chaos.2021.111058}}.

\bibitem[{Piedjou Komnang} \em{et~al.}(2021){Piedjou Komnang}, Guarcello,
  Barone, Gatti, Pagano, Pierro, Rettaroli, and Filatrella]{Pied21}
{Piedjou Komnang}, A.; Guarcello, C.; Barone, C.; Gatti, C.; Pagano, S.;
  Pierro, V.; Rettaroli, A.; Filatrella, G.
\newblock Analysis of Josephson junctions switching time distributions for the
  detection of single microwave photons.
\newblock {\em Chaos, Solitons \& Fractals} {\bf 2021}, {\em 142},~110496.
\newblock {\url{https://doi.org/https://doi.org/10.1016/j.chaos.2020.110496}}.

\bibitem[Guarcello \em{et~al.}(2021)Guarcello, Piedjou~Komnang, Barone,
  Rettaroli, Gatti, Pagano, and Filatrella]{Gua21}
Guarcello, C.; Piedjou~Komnang, A.S.; Barone, C.; Rettaroli, A.; Gatti, C.;
  Pagano, S.; Filatrella, G.
\newblock Josephson-Based Scheme for the Detection of Microwave Photons.
\newblock {\em Phys. Rev. Applied} {\bf 2021}, {\em 16},~054015.

\bibitem[Pankratov \em{et~al.}(2022{\natexlab{a}})Pankratov, Revin, Gordeeva,
  Yablokov, Kuzmin, and Il'ichev]{Pan22-1}
Pankratov, A.L.; Revin, L.S.; Gordeeva, A.V.; Yablokov, A.A.; Kuzmin, L.S.;
  Il'ichev, E.
\newblock Towards a microwave single-photon counter for searching axions.
\newblock {\em npj Quantum Information} {\bf 2022}, {\em 8},~61.
\newblock {\url{https://doi.org/10.1038/s41534-022-00569-5}}.

\bibitem[Pankratov \em{et~al.}(2022{\natexlab{b}})Pankratov, Gordeeva, Revin,
  Ladeynov, Yablokov, and Kuzmin]{Pan22-2}
Pankratov, A.L.; Gordeeva, A.V.; Revin, L.S.; Ladeynov, D.A.; Yablokov, A.A.;
  Kuzmin, L.S.
\newblock Approaching microwave photon sensitivity with Al Josephson junctions.
\newblock {\em Beilstein Journal of Nanotechnology} {\bf 2022}, {\em
  13},~582--589.

\bibitem[Graham and T\'el(1985)]{Graham85}
Graham, R.; T\'el, T.
\newblock Weak-noise limit of Fokker-Planck models and non differentiable
  potentials for dissipative dynamical systems.
\newblock {\em Phys. Rev. A} {\bf 1985}, {\em 31},~1109.

\bibitem[Kautz(1988)]{Kau88}
Kautz, R.L.
\newblock Thermally induced escape: The principle of minimum available noise
  energy.
\newblock {\em Physical Review A} {\bf 1988}, {\em 38},~2066--2080.

\bibitem[Kautz(1996)]{Kau96}
Kautz, R.L.
\newblock Noise, chaos, and the Josephson voltage standard.
\newblock {\em Reports on Progress in Physics} {\bf 1996}, {\em 59},~935--992.

\bibitem[Pountougnigni \em{et~al.}(2019)Pountougnigni, Yamapi, Filatrella, and
  Tchawoua]{Pou19}
Pountougnigni, O.V.; Yamapi, R.; Filatrella, G.; Tchawoua, C.
\newblock Noise and disorder effects in a series of birhythmic Josephson
  junctions coupled to a resonator.
\newblock {\em Phys. Rev. E} {\bf 2019}, {\em 99},~032220.
\newblock {\url{https://doi.org/10.1103/PhysRevE.99.032220}}.

\bibitem[Barone and Paterno(1982)]{Bar82}
Barone, A.; Paterno, G.
\newblock {\em Physics and applications of the Josephson effect}; Wiley, New
  York,  1982.

\bibitem[Likharev(1986)]{Lik86}
Likharev, K.
\newblock {\em Dynamics of Josephson Junctions and Circuits} {\bf 1986}.

\bibitem[Guarcello \em{et~al.}(2015)Guarcello, Valenti, and Spagnolo]{GuaVal15}
Guarcello, C.; Valenti, D.; Spagnolo, B.
\newblock Phase dynamics in graphene-based Josephson junctions in the presence
  of thermal and correlated fluctuations.
\newblock {\em Phys. Rev. B} {\bf 2015}, {\em 92},~174519.

\bibitem[Spagnolo \em{et~al.}(2017)Spagnolo, Guarcello, Magazz\'u, Carollo,
  Persano~Adorno, and Valenti]{Spa17}
Spagnolo, B.; Guarcello, C.; Magazz\'u, L.; Carollo, A.; Persano~Adorno, D.;
  Valenti, D.
\newblock Nonlinear Relaxation Phenomena in Metastable Condensed Matter
  Systems.
\newblock {\em Entropy} {\bf 2017}, {\em 19}.

\bibitem[McCumber(1968)]{McC68}
McCumber, D.E.
\newblock Effect of ac Impedance on dc Voltage‐Current Characteristics of
  Superconductor Weak‐Link Junctions.
\newblock {\em Journal of Applied Physics} {\bf 1968}, {\em 39},~3113--3118.

\bibitem[Beenakker(1992)]{Bee92}
Beenakker, C.W.J.
\newblock Three ``Universal'' Mesoscopic Josephson Effects.
\newblock In Proceedings of the Low-Dimensional Electronic Systems; Bauer, G.;
  Kuchar, F.; Heinrich, H., Eds.; Springer Berlin Heidelberg: Berlin,
  Heidelberg,  1992; pp. 78--82.

\bibitem[Kramers(1940)]{Kra40}
Kramers, H.
\newblock Brownian motion in a field of force and the diffusion model of
  chemical reactions.
\newblock {\em Physica} {\bf 1940}, {\em 7},~284 -- 304.
\newblock
  {\url{https://doi.org/http://dx.doi.org/10.1016/S0031-8914(40)90098-2}}.

\bibitem[{Piedjou Komnang} \em{et~al.}(2021){Piedjou Komnang}, Guarcello,
  Barone, Gatti, Pagano, Pierro, Rettaroli, and Filatrella]{Pie21}
{Piedjou Komnang}, A.; Guarcello, C.; Barone, C.; Gatti, C.; Pagano, S.;
  Pierro, V.; Rettaroli, A.; Filatrella, G.
\newblock Analysis of Josephson junctions switching time distributions for the
  detection of single microwave photons.
\newblock {\em Chaos Solitons Fract} {\bf 2021}, {\em 142},~110496.
\newblock {\url{https://doi.org/https://doi.org/10.1016/j.chaos.2020.110496}}.

\bibitem[Sikivie(1983)]{Sik83}
Sikivie, P.
\newblock Experimental Tests of the "Invisible" Axion.
\newblock {\em Phys. Rev. Lett.} {\bf 1983}, {\em 51},~1415--1417.

\bibitem[Visinelli(2013)]{Vis13}
Visinelli, L.
\newblock Axion-electromagnetic waves.
\newblock {\em Modern Physics Letters A} {\bf 2013}, {\em 28},~1350162.

\bibitem[Sikivie and Yang(2009)]{sik09}
Sikivie, P.; Yang, Q.
\newblock Bose-Einstein condensation of dark matter axions.
\newblock {\em Physical Review Letters} {\bf 2009}, {\em 103},~111301.

\bibitem[Duffy and van Bibber(2009)]{Duf09}
Duffy, L.D.; van Bibber, K.
\newblock Axions as dark matter particles.
\newblock {\em New Journal of Physics} {\bf 2009}, {\em 11},~105008.

\bibitem[Yan and Beck(2020)]{Yan20}
Yan, J.; Beck, C.
\newblock Nonlinear dynamics of coupled axion-Josephson junction systems.
\newblock {\em Physica D: Nonlinear Phenomena} {\bf 2020}, {\em 403},~132294.

\bibitem[Blackburn \em{et~al.}(2009)Blackburn, Marchese, Cirillo, and
  Gr{\o}nbech-Jensen]{Blac09}
Blackburn, J.A.; Marchese, J.E.; Cirillo, M.; Gr{\o}nbech-Jensen, N.
\newblock Classical analysis of capacitively coupled superconducting qubits.
\newblock {\em Physical Review B} {\bf 2009}, {\em 79},~054516.

\bibitem[Dubos \em{et~al.}(2001)Dubos, Courtois, Pannetier, Wilhelm, Zaikin,
  and Sch\"on]{Dub01}
Dubos, P.; Courtois, H.; Pannetier, B.; Wilhelm, F.K.; Zaikin, A.D.; Sch\"on,
  G.
\newblock Josephson critical current in a long mesoscopic S-N-S junction.
\newblock {\em Phys. Rev. B} {\bf 2001}, {\em 63},~064502.

\bibitem[Bergeret and Cuevas(2008)]{Ber08}
Bergeret, F.S.; Cuevas, J.C.
\newblock The Vortex State and Josephson Critical Current of a Diffusive SNS
  Junction.
\newblock {\em Journal of Low Temperature Physics} {\bf 2008}, {\em
  153},~304--324.

\bibitem[Du \em{et~al.}(2008)Du, Skachko, and Andrei]{Du08}
Du, X.; Skachko, I.; Andrei, E.Y.
\newblock Josephson current and multiple Andreev reflections in graphene SNS
  junctions.
\newblock {\em Phys. Rev. B} {\bf 2008}, {\em 77},~184507.

\bibitem[Graham and T\'el(1986)]{Graham86}
Graham, R.; T\'el, T.
\newblock Nonequilibrium potential for coexisting attractors.
\newblock {\em Phys. Rev. A} {\bf 1986}, {\em 33},~1322--1337.
\newblock {\url{https://doi.org/10.1103/PhysRevA.33.1322}}.

\bibitem[Risken(1989)]{Risken89}
Risken, H.
\newblock {\em The Fokker-Planck Equation: Methods of solution and
  Applications}; Springer, Berlin,  1989.

\bibitem[Kumar and Carroll(1984)]{Kumar84}
Kumar, B.V.K.V.; Carroll, C.W.
\newblock {Performance Of Wigner Distribution Function Based Detection
  Methods}.
\newblock {\em Optical Engineering} {\bf 1984}, {\em 23},~732 -- 737.
\newblock {\url{https://doi.org/10.1117/12.7973372}}.

\bibitem[Piedjou~Komnang \em{et~al.}(2021)Piedjou~Komnang, Guarcello, Barone,
  Pagano, and Filatrella]{Piedjou21}
Piedjou~Komnang, A.S.; Guarcello, C.; Barone, C.; Pagano, S.; Filatrella, G.
\newblock Analysis of Josephson Junction Lifetimes for the Detection of Single
  Photons in a Thermal Noise Background.
\newblock In Proceedings of the 2021 IEEE 14th Workshop on Low Temperature
  Electronics (WOLTE),  2021, pp. 1--4.
\newblock {\url{https://doi.org/10.1109/WOLTE49037.2021.9555447}}.

\bibitem[Filatrella \em{et~al.}(2023)Filatrella, Barone, Carapella, Gatti,
  Granata, Guarcello, Mauro, Komnang, Pierro, Rettaroli, and
  Pagano]{Filatrella23}
Filatrella, G.; Barone, C.; Carapella, G.; Gatti, C.; Granata, V.; Guarcello,
  C.; Mauro, C.; Komnang, A.P.; Pierro, V.; Rettaroli, A.;  et~al.
\newblock Theoretical and Numerical Estimate of Signal-to-Noise Ratio in the
  Analysis of Josephson Junctions Lifetime for Photon Detection.
\newblock {\em IEEE Transactions on Applied Superconductivity} {\bf 2023}, {\em
  33},~1--5.
\newblock {\url{https://doi.org/10.1109/TASC.2022.3214500}}.

\bibitem[Kautz(1988)]{Kautz88}
Kautz, R.L.
\newblock Thermally induced escape: The principle of minimum available noise
  energy.
\newblock {\em Phys. Rev. A} {\bf 1988}, {\em 38},~2066--2080.
\newblock {\url{https://doi.org/10.1103/PhysRevA.38.2066}}.

\end{thebibliography}



\PublishersNote{}
\end{adjustwidth}
\end{document}



Object detection constitutes a pivotal task in the field of computer vision, entailing the critical process of identifying and localizing objects present within an image. Applications of object detection models include autonomous vehicles, surveillance, robotics, and augmented reality \cite{yolo_applications}. The central problem of deploying deep learning-based object detection solutions on embedded hardware platforms is the amount of computation, memory, and power required for their inference \cite{lane2017squeezing}. This necessitates the development of efficient object detection models specialized for low-footprint hardware devices to achieve an optimal trade-off of accuracy and latency.
% Figure environment removed

For years, the state-of-the-art (SOTA) deep learning approach to object detection has been the series of YOLO architectures \cite{yolo_survey}. In recent years, remarkable strides have been taken in advancing YOLO-like single-stage object detectors, prioritizing real-time operation while simultaneously striving for higher accuracy and deployability on low-power devices. These advancements have primarily focused on enhancing various components of the detection pipeline. Key areas of improvement include the design of accurate and efficient backbone and neck structures within the network \cite{wang2023yolov7}, exploration of different detection head designs (e.g. anchor-based \cite{wang2023yolov7} vs. anchor-free \cite{ge2021yolox}), utilization of diverse loss functions \cite{dfl}, and implementation of novel training procedures including innovative data augmentation techniques \cite{dataaug}. 
These collective efforts have continually refined and evolved YOLO-like architectures, enhancing object detection effectiveness and efficiency in real-time scenarios. The differences between consecutive YOLO versions, such as YOLOv5 \cite{Jocher_YOLOv5_by_Ultralytics_2020} and YOLOv6 \cite{li2023yolov6}, span various pipeline components, making it challenging to isolate their individual contributions. This paper aims to address these challenges by providing a fair comparison of recent YOLO versions under controlled conditions (e.g. same training loop for all models) to
demonstrate the impact of the backbone and neck structure of
YOLO-based models in embedded inference applications. 
We also use the collected accuracy and latency data for multiple YOLO-based detector variations to empirically evaluate training-free performance predictors commonly used in neural architecture search \cite{abdelfattah2021zero}. We summarize our contributions as follows: 

% Zero-cost-proxy \cite{abdelfattah2021zero} techniques used in neural architecture search (NAS) \cite{mellor2021neural} to minimize the time needed for search by using training free methods to estimate accuracies. There are many different zero cost proxies methods like Zico \cite{li2023zico}, fisher \cite{turner2019blockswap}, jacobCov \cite{mellor2021neural}, SNIP \cite{lee2018snip}, Synflow \cite{tanaka2020pruning} used in the literature.  \textit{YOLOBench} can also be looked as a NAS space. This reveals and interesting observation that, while most of the state-of-the-art zero-cost (training-free) proxies for model accuracy estimation are outperformed by simple baselines such as MAC count, the NWOT estimator \cite{mellor2021neural} can be effectively used to identify potential Pareto-optimal YOLO detectors in a training-free manner. 

\begin{itemize}
    \item We provide a latency-accuracy benchmark of $550$+ YOLO-based object detection models on $4$ different datasets, called \textit{YOLOBench}. All the models are validated on $4$ different embedded hardware platforms (Intel CPU, ARM CPU, Nvidia GPU, NPU),
    % \item We provide a latency-accuracy benchmark of 550+ YOLO-based object detection models on 5 different datasets and 4 different embedded hardware platforms (CPU, ARM, GPU, NPU) called \textit{YOLOBench}    

    \item We show that if modern detection heads and training techniques are implemented for the detector training pipeline, multiple backbone and neck variations, including those of older architectures (e.g. YOLOv3 and YOLOv4), can be used to achieve state-of-the-art latency-accuracy trade-off,
    
    % \item We show that although years of progress in YOLO architecture design have provided improvements, a fair benchmarking of YOLO variations in the same training setup reveals that older models (such as YOLOv3) are still optimal in certain latency regions.
    
    %\item We show that GPU-timing benchmarks do not necessarily translate to other hardware, highlighting the potential role of hardware-aware architecture design and effective model scaling techniques.% Matteo: This /item can be improved..

    \item Looking at \textit{YOLOBench} as a neural architecture search (NAS) space, we demonstrate that, while most of the state-of-the-art zero-cost (training-free) proxies for model accuracy estimation are outperformed by simple baselines such as MAC count, the NWOT estimator \cite{mellor2021neural} can be effectively used to identify potential Pareto-optimal YOLO detectors in a training-free manner,
    
    % \item Leverage YOLOBench to demonstrate to analyze the effectiveness of zero-cost-proxy methods for model accuracy estimation. 

    \item We showcase the effectiveness of the NWOT estimator for
    optimal detector prediction by using it to identify a YOLO-like model with FBNetV3 backbone that outperforms YOLOv8 on the Raspberry Pi 4 ARM CPU.
    
    % \item Looking at \textit{YOLOBench} as a neural architecture search (NAS) space, we demonstrate that state-of-the-art training-free proxies for architecture accuracy estimation are outperformed by simple baselines such as MAC count.
\end{itemize}

%The following sections summarizes the YOLOBench architecture space by summarizing different YOLO models. It follows by the explanation of the methodolgy used to collect the latency and accuracy numbers needed for this benchmark and concludes with the discussion of results using the Pareto optimal model graphs. At the end, YOLOBench is used to analyze the effectiveness of zero-cost proxy methods. 

