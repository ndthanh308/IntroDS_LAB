\section{Experimental Apparatus and Data Analysis}
\label{sec:analysis}

A description of the ALICE detector and its performance can be found in Refs.~\cite{ALICE:2014sbx,ALICE:2008ngc, ALICE:2022wpn}. The main detectors used for this analysis are the Inner Tracking System (ITS)~\cite{ALICE:2010tia} for track and vertex reconstruction, the Time Projection Chamber (TPC)~\cite{Alme:2010ke} for track reconstruction and particle identification (PID) via the measurement of the specific energy loss, and the Time-Of-Flight (TOF)~\cite{Akindinov:2013tea} detector for PID via the measurement of the particle flight time from the interaction point to the detector. These detectors are located inside a large solenoidal magnet providing a magnetic field of up to 0.5 T parallel to the LHC beam direction and cover the pseudorapidity interval $|\eta|<0.9$.  A minimum-bias interaction trigger was used, requiring coincident signals in the V0A and V0C detectors~\cite{Abbas:2013taa}, two scintillator arrays covering the full azimuth in the pseudorapidity intervals  $2.8 < \eta < 5.1$ (V0A) and $-3.7 < \eta < -1.7$ (V0C). An online selection based on the V0 signal amplitudes was also applied in order to enhance the sample of midcentral collisions as an additional trigger class. Background events from beam--gas interactions were rejected offline using the timing information provided by the V0 and the neutron Zero-Degree Calorimeter (ZDC)~\cite{Arnaldi:1999zz}. Events used in the analysis were required to have a primary vertex reconstructed within $\pm10~\rm{cm}$ from the nominal interaction point along the beam axis. Centrality intervals for events were defined in terms of percentiles of the hadronic Pb--Pb cross section based on the signal amplitude of the V0 detectors~\cite{ALICE:2015juo}. After the aforementioned selections, a sample of about $85\times10^6$ events in the 30--50\% centrality class was utilised for further analysis, corresponding to an integrated luminosity of \lumi $\simeq56~\mu\mathrm{b}^{-1}$~\cite{ALICE:2018tvk}. 

The \Dzero mesons and their charge conjugates were reconstructed via the hadronic decay channel $\mathrm{D^0 \to K^-}$ $\mathrm{\uppi^+}$ with branching ratio $\mathrm{BR} = (3.947 \pm 0.030) \%$~\cite{ParticleDataGroup:2022pth}. The \Dzero-meson candidates were selected combining pairs of tracks with opposite charge signs, each with $\pt > 0.3~\GeV/c$ and $|\eta|<0.8$. The selection criteria require at least 70 (out of 159) associated space points in the TPC, a minimum of two (out of six) measured clusters in the ITS, with at least one in either of the two innermost layers, and a fit quality $\chi^2/\mathrm{ndf}<1.25$ in the TPC. These track selection criteria reduce the \Dzero-meson acceptance in rapidity, which falls steeply to zero for $|y|>0.5$ at low \pt and for $|y|>0.8$ for $\pt>5~\GeV/c$. Thus, a fiducial acceptance selection $|y| < y_{\rm fid}(\pt)$ was applied to grant a uniform acceptance inside the rapidity range considered. The $y_{\mathrm{fid}}(\pt)$ value was defined as a second-order polynomial function, increasing from 0.5 to 0.8 in $0 < \pt < 5~\GeV/c$, and as a constant term, $y_{\mathrm{fid}}=0.8$, for $\pt>5~\GeV/c$.

A machine-learning approach with multi-class classification based on Boosted Decision Trees (BDT) was adopted to simultaneously suppress the large combinatorial background and separate the contributions of prompt and non-prompt \Dzero mesons. The implementation of the BDT algorithm provided by the XGBoost~\cite{Chen:2016XST} library was employed. Samples of prompt and non-prompt \Dzero mesons for the BDT training were obtained from Monte Carlo (MC) samples, which simulated the Pb--Pb events at \fNN with the HIJING v1.383 generator~\cite{Wang:1991hta}. Additional \ccbar or \bbbar quark pairs were injected in each simulated event using the PYTHIA 8.243 event generator~\cite{Sjostrand:2006za, Sjostrand:2014zea} (Monash 2013 tune~\cite{Skands:2014pea}) to enrich the MC sample of prompt and non-prompt \Dzero-meson signals. The generated particles were transported through the experimental apparatus using the GEANT3 transport package~\cite{Brun:1994aa}. 
Samples for the combinatorial background were obtained from candidates in the sideband region in the data, i.e.~$5\sigma < |\Delta M| < 9\sigma$ in the invariant mass distribution, where $\Delta M$ is the difference between the invariant mass and the mean of signal distribution, and $\sigma$ is the invariant-mass resolution. Before the training, loose selections on kinematic and topological variables were applied to the \Dzero-meson candidates to reduce the computation time. The training variables provided to the BDTs were mainly based on the displacement of the $\rm D^0$ decay vertex from the primary vertex of the collision. These included the impact parameter of the \Dzero-meson daughter tracks, the distance between the \Dzero-meson decay vertex and the primary vertex, and the cosine of the pointing angle between the \Dzero-meson candidate line of flight (the vector connecting the primary and secondary vertices) and its reconstructed momentum vector, as well as the PID information of the decay tracks. A detailed description of the training procedure is reported in Ref.~\cite{ALICE:2021mgk}. Independent BDTs were trained in the different \pt intervals of the analysis. Subsequently, the BDTs were applied to the experimental data sample to obtain the BDT scores related to the candidate probability to be a non-prompt \Dzero meson or to belong to the combinatorial background. Selections were applied on the scores to reduce the large combinatorial background and to obtain different fractions of non-prompt $\rm D^0$ candidates ($\fnonprompt$).
The \Dzero-meson $v_2$ coefficient was measured with the Scalar Product (SP) method~\cite{Voloshin:2008dg, Luzum:2012da, Adler:2002pu},

\begin{equation}
\label{eq:SP}
\vn\{\mathrm{SP}\}=\langle\langle \pmb{u}_{\mathrm{2}}\cdot\frac{\pmb{Q}_{\mathrm{2}}^{\rm V0C*}}{M^{\rm V0C}}\rangle\rangle\bigg{/}\sqrt{\frac{\langle\frac{\pmb{Q}_{\mathrm{2}}^{\rm V0C}}{M^{\rm V0C}}\cdot\frac{\pmb{Q}_{\mathrm{2}}^{\rm V0A*}}{M^{\rm V0A}}\rangle\langle\frac{\pmb{Q}_{\mathrm{2}}^{\rm V0C}}{M^{\rm V0C}}\cdot\frac{\pmb{Q}_{\mathrm{2}}^{\rm TPC*}}{M^{\rm TPC}}\rangle}{\langle\frac{\pmb{Q}_{\mathrm{2}}^{\rm V0A}}{M^{\rm V0A}}\cdot\frac{\pmb{Q}_{\mathrm{2}}^{\rm TPC*}}{M^{\rm TPC}}\rangle}} = \langle\langle \pmb{u}_{\mathrm{2}}\cdot\frac{\pmb{Q}_{\mathrm{2}}^{\rm V0C*}}{M^{\rm V0C}}\rangle\rangle \big{/} R_{\rm 2},
\end{equation} 

where \textbf{\textit{u}}$_{\mathrm{2}} = e^{i\mathrm{2}\varphi_\mathrm{D^{0}}}$ is the unit flow vector of the \Dzero-meson candidate with azimuthal angle $\varphi_{\mathrm{D^{0}}}$. $\pmb{Q}^k_{\mathrm{2}}$ and $M^k$ are the subevent $\mathrm{2}^\mathrm{nd}$ harmonic flow vector and multiplicity for the subevent $k$, respectively. The denominator, called the resolution ($R_{\rm 2}$), is calculated with the formula introduced in Ref.~\cite{Voloshin:2008dg}, where the three subevents are defined by the particles measured in the V0C, V0A, and TPC detectors, respectively. For the TPC detector, the azimuthal angles of charged tracks reconstructed with $|\eta|<0.8$ and the number of measured tracks were used to calculate the $\Qn$ vector and $M$. For the V0A and V0C detectors, the $\Qn$ vectors were calculated from the azimuthal distribution of the energy deposition in the detector scintillators and $M$ is the sum of the amplitudes measured in each channel~\cite{ALICE:2020iug}. 
The $\Qn$ vectors are recalibrated using a recentering procedure~\cite{Selyuzhenkov:2007zi} to correct for effects of non-uniform acceptance. The nonflow effects are supressed by the pseudorapidity gaps between the TPC, V0A, and V0C detectors~\cite{ALICE:2018rtz}. The single bracket $\langle\rangle$ in Eq.~\ref{eq:SP} refers to an average over all the events, while the double brackets $\langle\langle\rangle\rangle$ denote the average over all particles in the considered $\pt$ interval and all events. The $R_{\rm 2}$ is extracted as a function of the collision centrality.

The \Dzero-meson $v_2$ cannot be measured directly using Eq.~\ref{eq:SP} since \Dzero mesons cannot be identified on a particle-by-particle basis. Therefore, a simultaneous fit to the invariant-mass spectrum and the $v_2$ distribution as a function of the invariant mass ($M_\mathrm{K\pi}$) was performed for \Dzero candidates in each $\pt$ interval, in order to measure the raw yields and the $v_2$ coefficients. The measured total 
elliptic flow coefficient, $\vntot$, can be written as a weighted sum of the $v_2$ of the \Dzero-meson candidates ($\vnsig$), and that of background ($\vnbkg$)~\cite{Borghini:2004ra} as
\begin{equation}
\label{eq:vnTot}
	\vntot (M_\mathrm{K\pi}) = \vnsig \frac{\Nsig}{\Nsig + \Nbkg}(M_\mathrm{K\pi}) + \vnbkg (M_\mathrm{K\pi})\frac{\Nbkg}{\Nsig + \Nbkg}(M_\mathrm{K\pi}),
\end{equation}
where $\Nsig$ and $\Nbkg$ are the raw signal and background yields, respectively. The fit function for the $\Dzero$-candidate invariant-mass distribution was composed of a Gaussian term to describe the signal and an exponential distribution for the background. The contribution of signal candidates with the reflected K--$\pi$ mass assignment was taken into account with an additional term, which is small thanks to the good PID capability. It was parameterised by fitting the simulated invariant-mass distribution with a double Gaussian function.  To improve the stability of the fits, the widths of the signal peaks were fixed to the values extracted from the fits of the invariant-mass distributions in the prompt enhanced sample, given the naturally larger abundance of prompt compared to non-prompt candidates. In the simultaneous fit, the $\vn$ parameter for the candidates with wrong K--$\pi$ mass assignment was set to be equal to $\vnsig$, provided that the origin of these candidates are real $\Dzero$ mesons. 
The $\vnsig$ was measured from the fit to the $\vntot$ distribution with the function of Eq.~\ref{eq:vnTot}, where $\vnbkg$ is a linear as a function of $M_\mathrm{K\pi}$ for $\pt>3~\GeV/c$. For $\pt<3~\GeV/c$, a second-order polynomial function was used to parametrise $\vnbkg$($M_\mathrm{K\pi}$).  Figure~\ref{fig:Dmass_vs_v2} shows an example of the simultaneous fit to the invariant-mass spectrum and $v_2^\mathrm{tot}$ as a function of $M_\mathrm{K\pi}$  with low (left panel) and high (right panel) non-prompt \Dzero-meson candidate BDT score selections  in $3<\pt<4~\GeV/c$ in the 30--50$\%$ centrality class.
% Figure environment removed

The reconstructed \Dzero-meson signals are a mixture of prompt and non-prompt \Dzero mesons. The $\vnsig$ is therefore a linear combination of prompt ($\vnprompt$) and non-prompt ($\vnfd$) contributions, which can be expressed as
\begin{equation}
\label{eq:vnprompt}
\vnsig = (1 - \fnonprompt) \vnprompt +  \fnonprompt \vnfd,
\end{equation}

where $\fnonprompt$ is estimated as a function of $\pt$ with a data-driven method, which is based on the construction of data samples with different abundances of prompt and non-prompt candidates. A set of raw yields \rawY{i} (index $i$ refers to a given selection on the BDT scores) can be obtained by varying the selection on the BDT score, which is related to the candidate probability to be a non-prompt \Dzero meson. These raw yields are related to the corresponding acceptance times efficiency \AccEff of prompt and non-prompt \Dzero mesons according to the equation

\begin{equation}
   \effP{i}\, \Np +  \effNP{i}\, \Nnp - \rawY{i} = \delta_{i},
\label{eq:set123}
\end{equation}

where $\delta_{i}$ represents a residual that accounts for the equation not summing exactly to 0 due to the uncertainties on $\rawY{i}$, $\effNP{i}$, and $\effP{i}$. By applying at least two different BDT selections and extracting the yields, the corrected yields of prompt (\Np) and non-prompt (\Nnp) \Dzero mesons can be obtained from Eq.~\ref{eq:set123} via a $\chi^2$ minimisation. More details can be found in Ref.~\cite{ALICE:2021mgk}. The left panel of Fig.~\ref{fig:D0Cut_npv2_fit} shows an example of the raw-yield distributions as a function of the minimum non-prompt \Dzero-meson BDT score threshold used in such a $\chi^2$-minimisation procedure in $3 < \pt < 4~\GeV/c$ for the 30--50\% centrality class. The raw yield decreases with the increasing minimum threshold for the score to be a non-prompt \Dzero meson, corresponding to an increasing non-prompt \Dzero-meson fraction. 
The prompt and non-prompt components of the raw yields for each BDT-based selection obtained from the $\chi^2$-minimisation approach, $\effP{i} \times \Np$ and $\effNP{i} \times\Nnp$, are shown in the histograms with red and blue colour, respectively, and their sum is reported by the green line. The values of \Nnp and \Np can be used to estimate the non-prompt \Dzero-meson fraction in the raw yield for any set of selections $i$ using
\begin{equation}
\label{eq:fnpromptSystem}
  \fnonprompt^{ i} = \frac{\effNP{ i} \, \Nnp}{\effNP{ i} \, \Nnp+\effP{ i} \,\Np}.
\end{equation}

The $\vnsig$ was determined for three or four non-overlapping intervals of BDT score to be non-prompt \Dzero mesons, depending on the number of candidates in each $p_{\rm T}$ interval. The result was extrapolated to \fnonprompt = 0 and \fnonprompt = 1 using a linear fit according to Eq.~\ref{eq:vnprompt} in order to estimate the $v_2$ values for prompt and non-prompt \Dzero mesons, respectively. A similar approach was adopted in Ref.~\cite{CMS:2020qul}. The right panel of Fig.~\ref{fig:D0Cut_npv2_fit} shows the linear fit of $\vnsig$ as a function of \fnonprompt in $3 < \pt < 4~\GeV/c$. The blue band represents  the 1$\sigma$ confidence interval obtained from the linear fit, which is considered as the statistical uncertainty of the $\vnsig$. As a crosscheck about the correlation of the statistical uncertainties on $\vnsig$ between different values of  $\fnonprompt$, the statistical uncertainty was also calculated with the Jackknife method~\cite{Efron:1979bxm} and found to be consistent with the fit method.



% Figure environment removed

