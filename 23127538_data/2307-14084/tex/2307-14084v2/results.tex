\section{Results}
\label{sec:results}

% Figure environment removed

The measured non-prompt \Dzero-meson elliptic flow at midrapidity ($|y|<0.8$) in the 30--50\% centrality class is shown in Fig.~\ref{fig:npD0_v2_vs_promptD0} as a function of \pt. The weighted mean of the non-prompt \Dzero-meson $v_2$ in the measured \pt range ($2 < \pt < 12~\GeV/c$) is 2.7$\sigma$ above 0. No significant \pt dependence of the $v_2$ is observed. The results obtained are compatible within uncertainties with those submitted for publication by CMS~\cite{CMS:2022vfn}, which have smaller statistical uncertainty. In the left panel of Fig.~\ref{fig:npD0_v2_vs_promptD0}, the non-prompt \Dzero-meson $v_2$ is compared with the average $v_2$ of prompt $\rm D^0$, $\rm D^+$, and $\rm D^{*+}$ mesons~\cite{ALICE:2020iug}. The non-prompt $\rm D^0$-meson $v_2$ is lower than that of prompt non-strange D mesons with 3.2$\sigma$ significance in $2 < \pt < 8~\GeV/c$, indicating a different degree of participation to the collective motion of the medium between charm and beauty quarks. 

The measured $v_2$ of non-prompt \Dzero mesons is compared with several theoretical models implementing beauty-quark transport in a hydrodynamically expanding QGP phase~\cite{He:2019vgs, Cao:2016gvr, Cao:2017hhk, Li:2020umn, Li:2020kax, Li:2021xbd, Ke:2018jem, Ke:2018tsh} in the right panel of Fig.~\ref{fig:npD0_v2_vs_promptD0}.  All of the considered calculations include  collisional interactions between  beauty quarks and medium constituents. In addition, the LBT~\cite{Cao:2017hhk,Cao:2016gvr}, LIDO~\cite{Ke:2018jem, Ke:2018tsh}, LGR~\cite{Li:2020umn}, and Langevin~\cite{Li:2020kax, Li:2021xbd} models also include radiative processes. Beauty-quark hadronisation via coalescence is considered for all models in addition to the fragmentation mechanism. Although the models are implemented with different assumptions on the interactions in the QGP and hadronic phases, and on the medium expansion, all of them provide a reasonable description of the measurement 
within uncertainties. More precise measurements will further constrain model parameters, especially on the spatial diffusion coefficient of beauty quarks, which are implemented differently in the various models. 

Figure~\ref{fig:npD0_v2_vs_btoe_models} shows the comparison between the $v_2$ of electrons from beauty-hadron decays (\btoe)~\cite{ALICE:2020hdw} and the non-prompt \Dzero-meson $v_2$ measurements. They are compatible in the common \pt interval within uncertainties. The LIDO model provides reasonable descriptions for these measurements and is consistent with the \pt shape in the data. Note that, the \pt of beauty-decay hadrons is not the same \pt of B mesons due to the decay kinematics. The good agreement between the predictions for B-meson and non-prompt \Dzero-meson $v_2$ from LIDO indicates that the decay kinematics do not play a significant role in the beauty-hadron $v_2$ measurements.

% Figure environment removed






