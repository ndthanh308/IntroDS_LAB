\section{Systematic Uncertainties}
\label{sec:syst}
Four major sources of systematic uncertainties were considered for the measurement of the non-prompt \Dzero-meson $v_2$: (i) the signal extraction from the invariant-mass and $v_{2}^{\rm tot}$ distributions; (ii) the non-prompt fraction estimation; (iii) the D-meson \pt shape in the simulation; and (iv) the centrality dependence of the SP denominator ($R_{\rm 2}$). All sources of systematic uncertainties were treated as uncorrelated and added in quadrature to obtain the total systematic uncertainties. Table~\ref{tab:sysunc_table} summarises the estimated values of the systematic uncertainties for each \pt interval.


\begin{table}[tb]
\caption{Summary of the systematic uncertainties on the measurement of the non-prompt \Dzero-meson $v_2$. The ranges of the uncertainties are quoted as absolute uncertainties, except those on the $R_{\rm 2}$ as relative uncertainty.}
\centering
\renewcommand*{\arraystretch}{1.2}
\begin{tabular}[t]{l|>{\centering}p{0.05\linewidth}>{\centering}p{0.05\linewidth}|>{\centering}p{0.05\linewidth}>{\centering}p{0.05\linewidth}|>{\centering}p{0.05\linewidth}>{\centering}p{0.05\linewidth}|>{\centering}p{0.05\linewidth}>{\centering}p{0.05\linewidth}|>{\centering}p{0.05\linewidth}>{\centering}p{0.05\linewidth}>{\centering}p{0.05\linewidth}}
\toprule
$\pt~(\GeV/c)$
& $2\mbox{--}3$   & \multicolumn{1}{c|}{$3\mbox{--}4$} 
& $4\mbox{--}5$   & \multicolumn{1}{c|}{$5\mbox{--}6$} 
& $6\mbox{--}8$   & \multicolumn{1}{c}{$8\mbox{--}12$}  \\
\midrule

Signal extraction                             
& 0.011   & \multicolumn{1}{c|}{0.012}   
& 0.011   & \multicolumn{1}{c|}{0.011}   
& 0.012   & \multicolumn{1}{c}{0.013}   \\
Non-prompt fraction estimation                          
& 0.005  & \multicolumn{1}{c|}{0.002}   
& 0.002   & \multicolumn{1}{c|}{0.001}   
& 0.001  & \multicolumn{1}{c}{0.001}   \\
MC D-meson \pt distribution                           
& 0.004   & \multicolumn{1}{c|}{0.004}    
& 0.002   & \multicolumn{1}{c|}{0.001}  
& 0.001   & \multicolumn{1}{c}{0.001}   \\
\multicolumn{1}{l|}{$R_{\rm 2}$ determination}  
& 0.5\%  & \multicolumn{1}{c|}{0.5\%}   
& 0.5\%   & \multicolumn{1}{c|}{0.5\%}   
& 0.5\%   & \multicolumn{1}{c}{0.5\%}   \\ 

\bottomrule
\end{tabular}
\label{tab:sysunc_table}	
\end{table}

The systematic uncertainty of the signal extraction from the invariant-mass and $v_{2}^{\rm tot}$ distributions is due to a possible imperfect modelling of the signal and background distributions. It was evaluated by repeating the simultaneous fit with different configurations. In particular, the fit range, signal width within the statistical uncertainties obtained with prompt enhanced sample, and background fit functions used for the invariant-mass and $\vntot$ distributions were varied. The systematic uncertainty was defined as the RMS of the distribution of the resulting $\vnfd$ obtained from all these variations. The second source of systematic uncertainty arises from the uncertainty on the determination of the \fnonprompt of \Dzero mesons with the minimisation method described in Section~\ref{sec:analysis}. In this method, the raw yields and the efficiencies obtained with several sets of selections are used in order to extract the prompt and non-prompt components. It is therefore sensitive to possible imperfections of the data description in the MC simulations. They were therefore evaluated by using alternative sets of selections for the aforementioned $\chi^2$-minimisation approach~\cite{ALICE:2021mgk}; the RMS of the resulting $\vnfd$ distribution was considered as the systematic uncertainty. The systematic effects due to possible differences between the real and simulated \pt spectra  were estimated by applying different weights to the \pt distributions of prompt \Dzero mesons and of the parent beauty hadrons in the case of non-prompt \Dzero mesons. In the default analysis procedure, the weights were defined to match the shape given by FONLL in pp collisions~\cite{Cacciari:2001td, Cacciari:1998it} multiplied by the nuclear modification factor (\RAA) prediction from the TAMU model~\cite{He:2014cla}. The FONLL spectrum multiplied by the \RAA from the LIDO model~\cite{ Ke:2018tsh} was used as an alternative shape for the systematic evaluation. The effect due to flow-related modifications of the parent beauty-hadron $p_{\rm T}$ spectra was found to be negligible with respect to the assigned $p_{\rm T}$-shape systematic uncertainty. The contribution of the SP denominator $R_{\rm 2}$ to the systematic uncertainty is due to the centrality dependence. It was evaluated as the difference of the centrality-integrated $R_{\rm 2}$ values with those obtained from weighted average $R_{\rm 2}$ values in narrow centrality intervals using the \Dzero-meson yields as weights~\cite{ALICE:2020iug}.

