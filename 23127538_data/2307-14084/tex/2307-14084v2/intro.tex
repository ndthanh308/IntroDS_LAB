\section{Introduction}
\label{sec:intro}
A phase of matter made of deconfined quarks and gluons, called the quark–gluon plasma (QGP), is created in ultrarelativistic heavy-ion collisions, as supported by several measurements at the SPS, RHIC, and LHC particle accelerators~\cite{NA50:2000brc, WA97:1999uwz, BRAHMS:2004adc, PHENIX:2004vcz, PHOBOS:2004zne, STAR:2005gfr, Roland:2014jsa, Braun-Munzinger:2015hba,ALICE:2022wpn}. The QGP formed in such extreme conditions is considered to be a nearly perfect fluid~\cite{Heinz:2013th}. Heavy quarks (charm and beauty), mostly produced via hard partonic scattering processes on a timescale shorter than the QGP formation time~\cite{Liu:2012ax, Andronic:2015wma}, are effective probes of the properties and dynamics of the QGP.
They interact with the medium constituents, losing energy via radiative and collisional processes~\cite{Braaten:1991we}. The significant suppression of charm- and beauty-hadron  production yields at intermediate and high transverse momentum ($\pt >$ 6$~\GeV/c$) observed in heavy-ion collisions at both RHIC~\cite{Adare:2010de,Abelev:2006db,Adamczyk:2014uip,Adler:2005xv,Adare:2015hla} and LHC~\cite{Adam:2015sza,Abelev:2012qh,Adam:2016khe,Adam:2016wyz,Khachatryan:2016ypw,Sirunyan:2017xss,Acharya:2018hre,CMS:2018eso,CMS:2017uoy,ALICE:2021rxa,ALICE:2021kfc,ALICE:2021bib,ALICE:2022tji,ALICE:2022xrg,ALICE:2022iba}, compared to appropriately scaled yields from proton--proton (pp) collisions, indicates a substantial energy loss of heavy quarks in the QGP. 

The azimuthal anisotropy in momentum space of final-state hadrons acts as an additional observable to probe the properties of the QGP. In non-central nucleus--nucleus collisions, the spatial anisotropy in the initial matter distribution due to the asymmetry of the nuclear overlap region is transferred to the final-state particle momentum distribution via multiple collisions, a phenomenon referred to as anisotropic flow~\cite{Ollitrault:1992bk,Voloshin:1994mz}. The anisotropic flow is quantified by the harmonic coefficients $v_\mathrm{n}=\langle \cos [\mathrm{n}(\varphi-\Psi_\mathrm{n})]\rangle$ of the Fourier expansion of the particle azimuthal angle ($\varphi$) relative to the collision symmetry planes with angles $\Psi_\mathrm{n}$ for the $\rm n^{\rm th}$ harmonic. The second harmonic, $v_2$, also known as elliptic flow, is the largest coefficient in non-central heavy-ion collisions. At low \pt ($\pt < 6~\GeV/c$), the heavy-flavour $v_2$ can help to quantify the extent to which charm and beauty quarks participate in the collective expansion of the medium~\cite{Batsouli:2002qf} and the fraction of heavy quarks hadronising via recombination with light quarks in the QGP medium in the intermediate \pt region ($6 < \pt < 10~\GeV/c$)~\cite{Molnar:2004ph, Greco:2003vf}. In addition, at high \pt ($\pt > 10~\GeV/c$), the $v_2$ of heavy-flavour hadrons can constrain the path-length dependence of energy loss in the medium for heavy quarks~\cite{Gyulassy:2000gk,Shuryak:2001me}.

D mesons and charm-hadron decay leptons show a positive $v_2$ in nucleus–nucleus collisions at both RHIC~\cite{PHENIX:2006iih, Adare:2010de,Adamczyk:2014yew,Adamczyk:2017xur} and LHC~\cite{Abelev:2013lca,Abelev:2014ipa,Adam:2016ssk,Adam:2015pga,Acharya:2017qps,Sirunyan:2017plt,Acharya:2018bxo, CMS:2017vhp, ALICE:2020iug, CMS:2020bnz} energies. The comparison of experimental measurements with theoretical models indicates that charm quarks participate in the collective expansion of the medium, and both collisional processes and the hadronisation of charm quarks via coalescence with light quarks are important to describe the observed elliptic flow~\cite{Uphoff:2012gb,He:2014cla,Monteno:2011gq,Cao:2013ita,Song:2015ykw,Nahrgang:2013xaa,Uphoff:2014hza,Beraudo:2014boa,Cao:2017hhk,Beraudo:2021ont}. In particular, the D-meson $v_2$ has a  magnitude similar to the $v_2$ of charged pions for $3<\pt < 6~\GeV/c$, suggesting that low-\pt charm quarks have a relaxation time comparable to the QGP lifetime~\cite{Moore:2004tg}. Due to their higher mass, beauty quarks are unlikely to reach thermalisation in the medium, therefore their azimuthal anisotropy can give further insight into the interactions of heavy quarks with the medium~\cite{Francis:2015daa,Liu:2016ysz,Riek:2010fk,Capellino:2022nvf}. The experimental information is still poor for the beauty-hadron $v_2$ at low momentum. The elliptic flow of \Jpsi mesons originating from beauty-hadron decays (non-prompt) measured by the CMS and ATLAS Collaborations is consistent with zero within large uncertainties for $\pt > 3~\GeV/c$~\cite{CMS:2016mah, ATLAS:2018xms}. The $v_2$ of leptons from beauty-hadron decays measured by ALICE and ATLAS is found to be positive~\cite{ALICE:2020hdw,ATLAS:2020yxw}. However, due to the small lepton masses, correlations between the kinematic variables (\pt and direction) of the beauty hadrons and the decay leptons are broad. This is improved when choosing a decay into a heavier particle. A measurement of the non-prompt \Dzero-meson $v_2$ has been recently submitted for publication by CMS~\cite{CMS:2022vfn}.

In this letter, the measurement of the non-prompt \Dzero-meson $v_2$ at midrapidty ($|y|<0.8$) in Pb--Pb collisions at a centre-of-mass energy per nucleon pair $\sqrtsNN=5.02$~TeV with the ALICE detector is reported. The \Dzero-meson $v_2$ is measured with the Scalar Product (SP) method~\cite{Luzum:2012da,Voloshin:2008dg} in midcentral collisions (30--50\% centrality class). The non-prompt \Dzero-meson $v_2$ is extracted and compared with previous measurements of the prompt non-strange D-meson $v_2$ (average of $\rm D^0$, $\rm D^+$, and $\rm D^{*+}$) and the $v_2$ of electrons from beauty-hadron decays, as well as with theoretical models based on beauty-quark transport in the QGP. 