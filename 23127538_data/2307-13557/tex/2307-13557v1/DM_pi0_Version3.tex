\documentclass[11pt]{imsart}
\usepackage[utf8]{inputenc}
\usepackage[dvipsnames]{xcolor}
\usepackage{amsthm}
\usepackage{amsmath,bbm, amssymb, dsfont, amsfonts}
\usepackage{array}
\usepackage{url}		% Für urls
\usepackage{xspace}		% Korrekte Leerzeichen nach Makros
\usepackage{booktabs}
\usepackage{ulem}
\usepackage{rotating}	% For rotated graphics
\usepackage{pdflscape}  % For rotated graphics
\usepackage{fancyvrb}   % Für R-Code (5.4.2009)
\usepackage{epstopdf}
\usepackage{graphicx}
\usepackage{latexsym}
\usepackage{amssymb}
\usepackage{a4wide}
\usepackage{caption}
\usepackage{subcaption}
\usepackage{natbib}
\usepackage{multirow}
\usepackage{hhline}


\graphicspath{{img/}}

\usepackage[pdftex,                %
bookmarks         = true,%     % Signets
bookmarksnumbered = true,%     % Signets num�rot�s
pdfpagemode       = None,%     % Signets/vignettes ferm� à l'ouverture
pdfstartview      = FitH,%     % La page prend toute la largeur
pdfpagelayout     = SinglePage,% Vue par page
colorlinks        = true,%     % Liens en couleur
linkcolor		  = magenta,%  % Liens equations
citecolor		  = blue,%  %Liens biblio
urlcolor          = orange,%  % Couleur des liens externes
pdfborder         = {0 0 0}%   % Style de bordure : ici, pas de bordure
]{hyperref}%   % Utilisation de HyperTeX

% address after reference
\makeatletter
\newcommand{\addresseshere}{%
  \enddoc@text\let\enddoc@text\relax
}
\makeatother
\setlength{\parskip}{0mm}

%macros for usual proba/stat maths symbols
\renewcommand{\emph}[1]{{\it #1}}
\newcommand{\wh}{\widehat}
\newcommand{\mbf}{\mathbf}
%\newcommand{\E}{\mathbf{E}}
\newcommand{\bias}{\textnormal{Bias}}
\newcommand{\var}{\textnormal{Var}}
\newcommand{\E}{\mathbb{E}}
\newcommand{\R}{\mathbb{R}}
\renewcommand{\l}{\ell}
\renewcommand{\P}{\mathbf{P}}
\newcommand{\cH}{\mathcal{H}}
\newcommand{\Bmtilde}{\tilde{B}_m}
\newcommand{\PBin}{\mathbf{PBin}}
\newcommand{\Bin}{\mathbf{Bin}}
\newcommand{\ind}[1]{{\mbf{1}\{#1\}}}
\newcommand{\unifrv}{\mathcal{U}}
%\newcommand{\expov}{\textnormal{Exp}}
\newcommand{\expov}{\mathcal{E}}

%macros for comments
\newcommand{\iq}[1]{\textcolor{Purple}{#1}}
\newcommand{\et}[1]{\textcolor{magenta}{#1}}
\newcommand{\seb}[1]{\textcolor{cyan}{#1}}

%macros for MT error criteria
\newcommand{\FDR}{\textnormal{FDR}}
\newcommand{\mFDR}{\textnormal{mFDR}}
\newcommand{\FDP}{\textnormal{FDP}}
\newcommand{\FDX}{\textnormal{FDX}}
\newcommand{\FWER}{\textnormal{FWER}}

% macros for the MT procedures
\newcommand{\ol}{\overline} 
\newcommand{\wt}{\widetilde}
\newcommand{\BH}{[\textnormal{BH}]} 
\newcommand{\LR}{[\textnormal{LR}]} 
\newcommand{\GR}{[\textnormal{GR}]} 
\newcommand{\PB}{[\textnormal{PB}]}
\newcommand{\HLR}{[\textnormal{HLR}]} 
\newcommand{\HGR}{[\textnormal{HGR}]} 

%\newcommand{\WLR}{[\textnormal{wLR}]} 
%\newcommand{\WLRAM}{[\textnormal{wLR-AM}]} 
%\newcommand{\WLRGM}{[\textnormal{wLR-GM}]} 
%\newcommand{\WPBAM}{[\textnormal{wPB-AM}]}
%\newcommand{\WPBGM}{[\textnormal{wPB-GM}]} 
%\newcommand{\WGR}{[\textnormal{wGR}]} 
%\newcommand{\WGRAM}{[\textnormal{wGR-AM}]}
%\newcommand{\WGRGM}{[\textnormal{wGR-GM}]}
%
%\newcommand{\DLR}{[\textnormal{DLR}]} 
%\newcommand{\DGR}{[\textnormal{DGR}]} 
%\newcommand{\DPB}{[\textnormal{DPB}]}
%
%\newcommand{\FAM}{F^{\textnormal{AM}}}
%\newcommand{\FGM}{F^{\textnormal{GM}}}

%\newcommand{\Abonf}{\mathcal{A}^{\mbox{\tiny OB}}}
%\newcommand{\AAbonf}{\mathcal{A}^{\mbox{\tiny AOB}}}
%\newcommand{\alphabonf}{\alpha^{\mbox{\tiny OB}}}
%\newcommand{\alphaAbonf}{\alpha^{\mbox{\tiny AOB}}}
%\newcommand{\alphaLORD}{\alpha^{\mbox{\tiny LORD}}}
%\newcommand{\ALORD}{\mathcal{A}^{\mbox{\tiny LORD}}}
%\newcommand{\alphaALORD}{\alpha^{\mbox{\tiny ALORD}}}
%\newcommand{\AALORD}{\mathcal{A}^{\mbox{\tiny ALORD}}}
%
%\newcommand{\AOBSURE}{\mathcal{A}^{\mbox{\tiny $\rho$OB}}}
%\newcommand{\AAOBSURE}{\mathcal{A}^{\mbox{\tiny $\rho$AOB}}}
%\newcommand{\alphaOBSURE}{\alpha^{\mbox{\tiny $\rho$OB}}}
%\newcommand{\alphaAOBSURE}{\alpha^{\mbox{\tiny $\rho$AOB}}}
%\newcommand{\ALORDSURE}{\mathcal{A}^{\mbox{\tiny $\rho$LORD}}}
%\newcommand{\AALORDSURE}{\mathcal{A}^{\mbox{\tiny $\rho$ALORD}}}
%\newcommand{\alphaLORDSURE}{\alpha^{\mbox{\tiny $\rho$LORD}}}
%\newcommand{\alphaALORDSURE}{\alpha^{\mbox{\tiny $\rho$ALORD}}}


%macros for estimators
\newcommand{\mPCmidp}{\widehat{m}_0^{\text {mid-PC}}}
\newcommand{\mPC}{\widehat{m}_0^{\text {PC}}}
\newcommand{\mPCNew}{\widehat{m}_0^{\text {PC,new}}}
\newcommand{\mPoly}{\widehat{m}_0^{\text {Poly}}(r, \lambda)}
\newcommand{\mPol}{\widehat{m}_0^{\text {Poly}}}


\newcommand{\mPCOrig}{\widehat{m}_0^{\text {PC,2006}}}
\newcommand{\mresStorey}{\widehat{m}_0^{\text {resc-Storey}}}
\newcommand{\mresPC}{\widehat{m}_0^{\text {resc-PC}}}
\newcommand{\mrandom}{\widehat{m}_0^r}
\newcommand{\mStorey}{\widehat{m}_0^{\text {Storey}}}
\newcommand{\mdu}{\widehat{m}_0^{\text {du}}}
\newcommand{\mmidp}{\widehat{m}_0^{\text {mid}}}
\newcommand{\mrand}{\widehat{m}_0^{\text {rand}}}
%\newcommand{\mZZKA}{\widehat{m}_0^{\text {A}}}
%\newcommand{\mZZKB}{\widehat{m}_0^{\text {B}}}
\newcommand{\mZZKA}{\widehat{m}_0^{\text {Exp}}}
\newcommand{\mZZKB}{\widehat{m}_0^{\text {PC,ZZD}}}
\newcommand{\mZZKC}{\widehat{m}_0^{\text {C}}}

\newcommand{\nudu}{\nu^{\text {du}}}
\newcommand{\nuduStorey}{\nu^{\text {du-Storey}}}
\newcommand{\nuduPC}{\nu^{\text {du-PC}}}
\newcommand{\numid}{\nu^{\text {mid}}}
\newcommand{\numidStorey}{\nu^{\text {mid-Storey}}}
\newcommand{\numidPC}{\nu^{\text {mid-PC}}}

\newcommand{\Fdu}{F^{\text{sd}}}
\newcommand{\Fmid}{F^{\text{mid}}}


%macros for convex ordering
%\newcommand{\lecx}{\le_{\text{cx}}}

% other macros
%\newcommand{\loosum}{\sum_{i \in \mathcal{H}_{0} \backslash  \{ h \} } }
%\newcommand{\nullset}{\mathcal{H}_{0}}
\newcommand{\cxorder}{\leqslant_{\text{cx}}}
\newcommand{\stoorder}{\leqslant_{\text{st}}}
\newcommand{\gest}{\geqslant_{\text{st}}}

% other macros
\newcommand{\loosum}{\sum_{i \in \mathcal{H}_{0} \backslash  \{ h \} } }
\newcommand{\nullset}{\mathcal{H}_{0}}
\newcommand{\altset}{\mathcal{H}_{1}}



\newtheorem{theorem}{Theorem}[section]
\newtheorem{corollary}{Corollary}[section]
\newtheorem{proposition}[theorem]{Proposition}
\newtheorem{algorithm}{Algorithm}[section]
\newtheorem{definition}[theorem]{Definition}
\newtheorem{remark}{Remark}[section]
\newtheorem{example}{Example}[section]
\newtheorem{lemma}[theorem]{Lemma}


\defcitealias{ZZD2011}{[ZZD]}
\defcitealias{shaked2007stochastic}{[SS]}

%\oddsidemargin-4mm
%\topmargin-20mm
%\textwidth167mm
%\textheight240mm

%\renewcommand\arraystretch{1.3}
%\pagestyle{empty}


\begin{document}
%\frontmatter
\title{ A unified class of null proportion estimators with plug-in FDR control
%Some new results on plug-in estimators for FDR control
}
\author{Sebastian D\"ohler, Iqraa Meah}
%\affil[1]{Hochschule Darmstadt, Darmstadt}
%\affil[2]{LPSM, Sorbonne University and Hochschule Darmstadt, Darmstadt}
\runtitle{Unified class of $\pi_{0}$ estimators with plug-in control}
\date{\today}
%\subjclass[2000]{Primary 62P05; Secondary 91G40, 62J15}
%   % AMS keywords (used in AMS journals)
\begin{abstract}
%	The Benjamini-Hochberg (BH) procedure for controlling the false discovery rate (FDR) is an indispensable tool in modern high dimensional data analysis. 
	Since the work of \cite{Storey2004}, it is well-known that the performance of the Benjamini-Hochberg (BH) procedure can be improved by incorporating estimators of the number (or proportion) of null hypotheses, yielding an adaptive BH procedure which still controls FDR. 
	Several such plug-in estimators have been proposed since then, for some of these, like Storey's estimator, plug-in FDR control has been established, while for some others, e.g. the estimator of \cite{PC2006}, some gaps remain to be closed. 
	In this work we introduce a unified class of estimators, which encompasses existing and new estimators and unifies proofs of  plug-in FDR control using simple convex ordering arguments. 
%	We also show that \iq{estimators from the class can be combined with a convex combination without compromising plug-in FDR control.}
	{We also show that any convex combination of such  estimators once more yields  estimators with guaranteed plug-in FDR control.}
%	any such estimators, multiplied with fixed weights can be combined into a single estimator without compromising plug-in FDR control. 	
	Additionally, the flexibility of the new class of estimators also allows incorporating distributional informations on the $p$-values. 
	We illustrate this for the case of discrete tests, where the null distributions of the $p$-values are typically known. 
	In that setting, we describe two generic approaches for adapting any estimator from the general class to the discrete setting while guaranteeing plug-in FDR control. 
	While the focus of this paper is on presenting the generality and flexibility of the new class of estimators, we also include some analyses on simulated and real data.
	
%	Incorporating an estimator of the number of nulls $m_{0}$ in the BH procedure, termed as performing an adaptive plug-in procedure, should allow the practitioner to make more discoveries while keeping the FDR under control. 
%	Since the work of \cite{Storey2004}, a pool of different estimators has been proposed, however some of the estimators do not come with valid plug-in FDR control or require cumbersome parameter tuning.
%	Additionally, some of these estimators are introduced with different intuitions when in fact they could be regarded as particular cases of a general underlying estimator. 
%	In this work, we propose to introduce a class of estimators with a general formula, based on local estimates of $m_{0}$, that encompasses existing and new estimators.
%	We prove plug-in FDR control for any estimator of this general class, hence unifying plug-in FDR control proofs for existing and new estimators using simple convex ordering arguments.
%	This generalization also allows the incorporation of available distributional information of $p$-values when these are more conservative as is often the case in discrete settings.
%	At last, we analyze existing and new estimators belonging to the general class on empirical data and provide a heuristic on the choice of the estimator. 
	
	
%	The issue of designing performant estimators have been well addressed so a pool of different estimators exists. However, some of these estimators, althougth providing good performance in practice, do not come with a control guarantee for the plug-in FDR.
%	Additionally, some of the estimators are not the best suited for heteregeneous and more conservative p-values as one can encounter in a discrete setting. 
%	In this work we propose a class of estimators with a general shape unifying different types of existing and novel estimators with proven plug-in FDR control.
%	This general class seizes the idea of rescaling : the estimator sums transformed p-values and rescales this sum by the expectation of the transformation under the null. 
%	The general class allows to have hybrid estimators between different well-known types of estimators and also to make a convex combination between different estimators.
	
	
\end{abstract}
%\begin{keyword}[class=AMS]
%	\kwd[Primary ]{62H15}
%	\kwd[; secondary ]{62Q05}
%\end{keyword}
\begin{keyword}
	\kwd{False discovery rate}\kwd{Adaptive BH procedure}\kwd{Convex ordering of random variables}  \kwd{Discrete hypothesis testing} \kwd{Multiple hypothesis testing} %\sout{\kwd{Estimation of the number of null hypotheses} \kwd{Irwin-Hall distribution}}
\end{keyword}

\maketitle
\tableofcontents

%\iq{ 
%Status quo:
%\begin{itemize}
%	\item verify BR condition for slightly modified PC estimator under uniform setting : until now gap in lit 
%	\item going from estimation in the uniform setting to estimation in the discrete setting
%	\begin{enumerate}
%		\item PC with mid-p-values (gap in lit)
%		\item Storey rescaled
%		\item Randomization approach (check name in ref + check if there is formula). This is a generic approach for improving unif estimators to the discrete case (indep of the method).
%	\end{enumerate}
%\end{itemize}
%Further ideas:
%\begin{itemize}
%	\item perf for plug-in BH procedure 
%	\item general recipe to derive estimator verifying BR condition 
%	\item Storey with mid-p-values ? 
%	\item PC after selection Storey-like selection phase 
%	\item Rescaled PC ? 
%	\item properties for estimator of FDP
%	\item BR condition analog for discrete FDR procedures (check DDR)
%\end{itemize}
%}
%
%\newpage 


The problem of the presence or absence of phase transition is central in statistical mechanics. To prove the existence of phase transition, the standard idea is to define a notion of contour and use \textit{Peierls' argument} \cite{Peierls.1936}. In the usual Ising model \cite{Ising_25}, particles of the system interact only with their nearest-neighbors. On ferromagnetic long-range Ising models \cite{Anderson_Yuval_69}, there is interaction between each pair of spins in the lattice. The Hamiltonian of the model is given formally by
\begin{equation*}
    H(\sigma) = - \sum_{x,y\in \Z^d}J_{xy}\sigma_x\sigma_y,
\end{equation*}
where $J_{xy}=J|x-y|^{-\alpha}$, $J>0$, $\alpha > d$. It is well-known that the Peierls' argument in dimension 2 implies phase transition for Ising models with nearest-neighbors or long-range interactions when $d\geq 2$, using correlation inequalities. For the unidimensional lattice, it was known that short-range models do not present phase transition. In the long-range case, a different behavior was expected depending on the exponent $\alpha$ (see \cite{Kac_Thompson_69}), but the problem was challenging since contours were first created as multidimensional objects.

In dimension $d=1$, phase transition was proved first in 1969 by Dyson \cite{Dyson.69}, for $\alpha \in (1,2)$, by proving phase transition in an auxiliary model and then using correlation inequalities. In 1982, Fr{\"o}hlich and Spencer \cite{Frohlich.Spencer.82} introduced a notion of one-dimensional contours and then applied the Peierls' argument to show phase transition for the critical value $\alpha = 2$. These contours were inspired by the multiscale techniques previously introduced to study the Berezinskii-Kosterlitz-Thouless transition in two-dimensional continuous spin systems \cite{FS81}. Later, Cassandro, Ferrari, Merola and Presutti  \cite{Cassandro.05} extended the contour argument previously available for $\alpha=2$ to exponents $\alpha\in (3-\frac{\ln 3}{\ln 2}, 2)$, with the additional restriction that the nearest-neighbor interaction is strong, i.e.,  ${J(1)\gg 1}$; this restriction was removed for a subclass of interactions in \cite{Bissacot.Endo.18}. Further results were obtained using contour arguments, such as the decay of correlations, cluster expansions, phase transition with random interactions, etc; some references with these results are \cite{ Cassandro.Merola.Picco.17, Cassandro.Merola.Picco.Rozikov.14, Imbrie.82, Imbrie.Newman.88, Johansson.91}. 

In the multidimensional setting ($d\geq 2$), Ginibre, Grossmann, and Ruelle, in \cite{Ginibre.Grossmann.Ruelle.66}, proved the phase transition for $\alpha > d+1$, using an enhanced version of Peierls' argument and the usual contours. Park proposed a different notion of contour for long-range systems in \cite{Park.88.I, Park.88.II}, extending the Pirogov-Sinai theory available for short-range interactions assuming $\alpha > 3d+1$, although he can also consider Potts models with his methods. Some results in the literature suggest that truly long-range effects appear only when $d < \alpha \leq d+1$, see for instance, \cite{Biskup_Chayes_Kivelson_07}. Recently, Affonso, Bissacot, Endo and Handa \cite{Affonso.2021}, inspired by the ideas from Fr{\"o}hlich and Spencer in \cite{FS81, Frohlich.Spencer.82}, introduced a version of multiscale multidimensional contour and proved phase transition by a contour argument in the whole region $\alpha > d$. They can consider long-range Ising models with deterministic decaying fields, first introduced in the context of nearest-neighbor interactions in \cite{Bissacot_Cioletti_10}. For these models, the lack of analyticity of the free energy does not imply phase transition since these models have the same free energy as the models with zero field. It is expected that fields decaying slowly imply uniqueness. In this setting, a contour argument is useful for proofs of phase transitions as well for uniqueness, some papers with models with deterministic decaying fields are \cite{Aoun_Ott_Velenik_23, Bissacot_Cass_Cio_Pres_15, Bissacot.Endo.18, Cioletti_Vila_2016}.

The Random Field Ising model (RFIM) \cite{Imry.Ma.75} is the nearest-neighbor Ising model with an additional external field acting on each site $(h_x)_{x\in\Z^d}$ that is a family of i.i.d. Gaussian random variable with mean 0 and variance 1. Formally, the Hamiltonian of the model is given by
\begin{equation*}
    H(\sigma) = - \sum_{\substack{x,y\in \Z^d \\|x-y|=1}}J\sigma_x\sigma_y  - \varepsilon\sum_{x\in\Z^d}h_x\sigma_x,
\end{equation*}
where $J>0$, $\varepsilon>0$, $\alpha > d$ and $d \geq 1$. A detailed account of the history of the phase transition problem for this model, as well as detailed proofs, was given in \cite{Bovier.06}. Here we present a brief overview.

During the 1980s, the question of the specific dimension where phase transition for the RFIM should happen attracted much attention and was a topic of heated debate. Two convincing arguments were dividing the physics community. One of them, due to Imry and Ma \cite{Imry.Ma.75}, was a non-rigorous application of the Peierls' argument together with the use of the isoperimetric inequality. The key idea of Peierls' argument is to define a notion of contour and calculate the energy cost of "erasing" each contour, i.e., the energy cost of flipping all spins inside the contour. When there is no external field, that energy necessary to flip the spins in a region $A\subset \Z^d$ is of the order of the boundary $|\partial A|$. When we add an external field, we get an extra cost depending on this field. Imry and Ma argued that this cost should be approximately $\sqrt{|A|}$, which is smaller than $|\partial A|$ for all regions only when $d\geq 3$, so this should be the region where phase transition occurs. The other argument, due to Parisi and Sourlas \cite{Parisi.Sourlas.79}, based on dimensional reduction, predicted that the $d$-dimensional RFIM would behave like the $d-2$-dimensional nearest-neighbor Ising model, therefore presenting phase transition only when $d\geq 4$. 

The question was settled by two celebrated papers showing that Imry and Ma's prediction was correct. First, in 1988, Bricmont and Kupiainen \cite{Bricmont.Kupiainen.88} showed that there is phase transition almost surely in $d\geq3$, for low temperatures and variance $\varepsilon$ small enough. Their proof uses a rigorous renormalization group analysis for the short-range case and it is considered involved. Still, they claimed that the result works for any model with a suitable contour representation and centered sub-gaussian external field. Later on, Aizenman and Wehr \cite{Aizenman.Wehr.90} proved uniqueness for $d\leq 2$. For detailed proofs of these results, we refer the reader to \cite{Bovier.06} (see also \cite{Berretti.85, Camia.18, Frohlich.Imbre.84,  Klein.Masooman.97} for more uniqueness results). 

Recently, Ding and Zhuang, see \cite{Ding2021}, provided a simpler proof of the phase transition, not using RGM. And in  \cite{Ding.Liu.Xia.22}, Ding, Liu and Xia proved that if $\beta_c(d)$ is the critical inverse of the temperature of the Ising model with no field, for all $\beta>\beta_c(d)$ there exists a critical value $\varepsilon_0(d, \beta)$ such that the RFIM with $\varepsilon \leq \varepsilon_0$ presents phase transition. 

In the present paper, we are considering a long-range Ising model with a random field, whose Hamiltonian is given formally by
\begin{equation*}
    H(\sigma) = - \sum_{x,y\in \Z^d}J_{xy}\sigma_x\sigma_y - \varepsilon\sum_{x\in\Z^d}h_x\sigma_x,
\end{equation*}
where $J_{xy}=J|x-y|^{-\alpha}$, $J, \varepsilon>0$, $\alpha > d$ and $h_x\in\mathbb{R}$, $d\geq 3$.
Until now, the only known result in the long-range setting is for the one-dimensional long-range Ising model with a random field, by Cassandro, Orlandi, and Picco \cite{Cassandro.Picco.09}. They used the contours of \cite{Cassandro.05} to show the phase transition for the model when $\alpha\in (3-\frac{\ln 3}{\ln 2}, \frac{3}{2})$, under the assumption $J(1) \gg 1$. We stress that, as remarked by Aizenman, Greenblatt, and Lebowitz \cite{Aizenman_Greenblatt_Lebowitz_2012}, although their argument does not work for the whole region for the exponent $\alpha$, the phase transition holds for values close to the critical value $\alpha=3/2$, since by the Aizenman-Wehr theorem we know that there is uniqueness for $\alpha>3/2$.

The argument from Ding and Zhuang in \cite{Ding2021}, for $d\geq3$, involves controlling the probability of a bad event, which is closely related to controlling the quantity $$\sup_{\substack{0\in A\subset\Z^d \\ A \text{ connected }}}\frac{\sum_{x\in A}h_x}{|\partial A|},$$ known as the greedy animal lattice normalized by the boundary. The greedy animal lattice normalized by the size, instead of the boundary, was extensively studied for general distributions of $(h_x)_{x\in\Z^d}$, see \cite{Cox_Gandolfi_Griffin_Kesten_93, Gandolfi_Kesten_94, Hammond_06, Martin_02}. When we normalize by the boundary, an argument by Fisher, Fr\"{o}hlich and Spencer \cite{FFS84} shows that the expected value of the greedy animal lattice is constant. In dimension $d=2$, the expected value is not finite, see \cite{Ding.Wirth.20}. The supremum is taken over connected regions containing the origin since the interiors of the usual Peierls contours are of this form.


For the long-range model, the interior of contours is not necessarily connected. In fact, long-range contours may have considerably large diameters with respect to their size, so their interiors can be very sparse. To avoid this, we define contours, strongly inspired by the $(M,a,r)$-partition in \cite{Affonso.2021}, using a multiscaled procedure that assures that the contours have no cluster with small density.  With them, we generalize the arguments by Fisher-Fr\"{o}hlich-Spencer \cite{FFS84}, and prove that the expected value of the greedy animal lattice is constant, even considering regions not necessarily connected in the supremum. Then, we prove the phase transition for $d\geq 3$. The main result of this paper is the following.
\begin{theorem*}Given $d\geq 3$, $\alpha>d$, there exists $\beta_c\coloneqq\beta(d, \alpha)$ and $\varepsilon_c\coloneqq\varepsilon(d, \alpha)$ such that, for $\beta >\beta_c$ and $\varepsilon\leq \varepsilon_c$, the extremal Gibbs measures $\mu_{\beta, \varepsilon}^+$ and $\mu_{\beta, \varepsilon}^-$ are distinct, that is, $\mu_{\beta, \varepsilon}^+ \neq \mu_{\beta, \varepsilon}^-$ $\mathbb{P}$-almost surely. Therefore the long-range random field Ising model presents phase transition.
\end{theorem*}

This paper is divided as follows. In Section 2, we define the model and the contours, and suitable generalizations to the constructions in \cite{Affonso.2021} are introduced.  In Section 3, we define two bad events of the external field and prove that they occur with a small probability.  In Section 4, we present the proof of the phase transition.
\documentclass[12pt]{amsart}

\input{"Preamble"}

%\usepackage{standalone} 
%\def\inmain{1}

\title{Linear categories}

\author{G. Stefanich}

\externaldocument{Introduction}
\externaldocument{Semisimple}
\externaldocument{GRings}

\date{}

\begin{document}

%%%%%%%%%%%%%%%%%%%%%%%%%%%%%%%%%%%%%%%%%%%%%%%%%%%%%%%%%%%%%%%%%%%%%%%%
%%%%%%%%%%%%%%%%%%%%%%%%%%%%%%%%%%%%%%%%%%%%%%%%%%%%%%%%%%%%%%%%%%%%%%%%
%%%%%%%%%%%%%%%%%%%%%%%%%%%%%%%%%%%%%%%%%%%%%%%%%%%%%%%%%%%%%%%%%%%%%%%%
%%%%%%%%%%%%%%%%%%%%%%%%%%%%%%%%%%%%%%%%%%%%%%%%%%%%%%%%%%%%%%%%%%%%%%%%
%%%%%%%%%%%%%%%%%%%%%%%%%%%%%%%%%%%%%%%%%%%%%%%%%%%%%%%%%%%%%%%%%%%%%%%%
%%%%%%%%%%%%%%%%%%%%%%%%%%%%%%%%%%%%%%%%%%%%%%%%%%%%%%%%%%%%%%%%%%%%%%%%

\section{Linear categories}

This section contains preliminary material on the theory of linear categories that will be used throughout the paper. We begin in \ref{subsection linear cats} with a review of the notion of cocomplete category linear over a commutative algebra in $\catl$.  This recovers in particular the notion of cocomplete category linear over a (connective) commutative ring spectrum. We include here a proof of the fact that dualizable categories linear over a presentable base are automatically presentable, which forms the first step in the proof of the main theorems of this paper.

In \ref{subsection abelian} we study the theory of Grothendieck abelian categories linear over a base symmetric monoidal Grothendieck abelian category $\Acal$. We show that some basic aspects of the theory of Grothendieck abelian categories (tensor products, Gabriel-Popescu theorem) hold in this relative context as long as we require $\Acal$ to be generated by compact projective objects and rigid. We then discuss the notion of flatness for objects in an $\Acal$-linear Grothendieck abelian category, which will be needed in section \ref{section G rings}.

In \ref{subsection spectral categories} we review the notions of spectral and semisimple categories, and prove an $\acal$-linear version of a basic structure result from \cite{Spectral} that relates spectral categories to self-injective von Neumann regular algebras. This will be used in our classification of smooth categories over a rigid semisimple base in section \ref{section invertible semisimple}.

Finally, in \ref{subsection prestables} we review the theory of Grothendieck prestable categories from \cite{SAG}, and discuss a version relative to a base symmetric monoidal Grothendieck prestable category. For the most part, the material here is parallel to  that of \ref{subsection abelian}. We also include a general discussion of how linearity interacts with the passage to derived categories.

\subsection{General notions}\label{subsection linear cats}

We begin with some  background on the notion of linear category.

\begin{definition}
Let $\Mcal$ be a commutative algebra in $\catl$. An $\Mcal$-linear cocomplete category is an $\Mcal$-module in $\catl$. An $\Mcal$-linear colimit preserving functor is a morphism of $\Mcal$-modules in $\catl$.
\end{definition}

\begin{example}
Let $\Mcal$ be a commutative algebra in $\catl$. Then $\Mcal$ has a structure of   $\Mcal$-linear cocomplete category. For every   $\Mcal$-linear cocomplete category $\ccal$, evaluation at the unit induces an equivalence between  the category of   $\mcal$-linear colimit preserving functors $\Mcal \rightarrow \ccal$ and $\ccal$. The inverse to this  equivalence associates to each object $X$ in $\ccal$ an $\Mcal$-linear enhancement of the functor $- \otimes X: \Mcal \rightarrow \ccal$. We may summarize this by saying that $\Mcal$ is the free $\Mcal$-linear cocomplete category on one object.
\end{example}

\begin{example}\label{example LMod}
Let $\Mcal$ be a commutative algebra in $\catl$ and let $A$ be an algebra in $\Mcal$. Then the category $\LMod_A(\Mcal)$ of left $A$-modules in $\Mcal$ has a structure of $\Mcal$-linear cocomplete category. Thinking about $A$ as a left $A$-module we obtain an object of $\LMod_A(\Mcal)$, which itself admits a right $A$-module structure. In fact $A$ is the universal algebra in $\Mcal$ equipped with a right action on the left $A$-module $A$: in other words, $A$ is (the opposite of) the algebra of endomorphisms of the left $A$-module $A$.

Assume now given another   $\Mcal$-linear  cocomplete category $\Ccal$. Then for each  $\Mcal$-linear colimit preserving functor $f: \LMod_A(\Mcal) \rightarrow \Ccal$ we obtain a right $A$-module $f(A)$ in $\Ccal$. The assignment $f \mapsto f(A)$ turns out to induce an equivalence between the category of  $\Mcal$-linear colimit preserving functors $\LMod_A(\Mcal) \rightarrow \Ccal$ and the category of right $A$-modules in $\ccal$ (\cite{HA} theorem 4.8.4.1). The inverse to this equivalence maps a right $A$-module $M$ in $\Ccal$ to an $\Mcal$-linear enhancement of  the relative tensor product functor $M \otimes_A - : \LMod_A(\Mcal) \rightarrow \Ccal$. We may summarize this by saying that $\LMod_A(\Mcal)$ is the universal   $\Mcal$-linear cocomplete  category on a right $A$-module.
\end{example}

If $\Mcal$ is a commutative algebra in $\catl$ then $\Mod_{\Mcal}(\catl)$ inherits a closed symmetric monoidal structure from $\catl$. It makes sense in particular to consider dualizable and invertible objects in $\Mod_{\Mcal}(\catl)$. The following proposition shows that as long as $\Mcal$ is presentable, dualizability automatically implies presentability.

\begin{proposition}\label{proposition dualizable is presentable}
Let $\Mcal$ be a presentable symmetric monoidal category and let $\ccal$ be a dualizable object of $\Mod_\Mcal(\catl)$. Then $\ccal$ is presentable.
\end{proposition}
\begin{proof}
Let $\kappa$ be the smallest large cardinal. It is proven in \cite{Pres} section 5.1 (in particular, proposition 5.1.7 and corollary 5.1.15) that $\Mod_\Mcal(\catl)$ is a very large presentable category $\kappa$-compactly generated by those $\Mcal$-modules which belong to $\Pr^L$. Since the symmetric monoidal structure on $\Mod_\Mcal(\catl)$ is compatible with large colimits and the unit is $\kappa$-compact, we have that every dualizable object is $\kappa$-compact and therefore presentable.
\end{proof}

For each commutative algebra $\Mcal$ in $\catl$ we will denote by $- \otimes_{\Mcal}-$ the tensor product on $\Mod_\Mcal(\catl)$, and by $\Funct_{\Mcal}(-,-)$ the internal Hom. If $\Mcal$ is presentable, then these bifunctors restrict to $\Mod_{\Mcal}(\Pr^L)$.

\begin{example}\label{example tensor with amod}
Let $\Mcal$ be a commutative algebra in $\catl$. Let $A$ be an algebra in $\Mcal$ and let $\Ccal$ be an   $\Mcal$-linear cocomplete category. Then we have an $\Mcal$-bilinear functor 
\[
\LMod_A(\Mcal) \times \Ccal \rightarrow \LMod_A(\Ccal)
\]
that sends a pair $(M, X)$ to $M \otimes X$.  This induces an equivalence 
\[
\LMod_A(\Mcal) \otimes_\Mcal \Ccal = \LMod_A(\Ccal)
\]
 (see \cite{HA} theorem 4.8.4.6).  In particular, if $\Ccal$ is the category $\RMod_B(\Mcal)$ of right modules over some algebra $B$ in $\Mcal$ we obtain an equivalence 
\[
\LMod_A(\Mcal) \otimes_\Mcal \RMod_B(\Mcal) = \LMod_A(\RMod_B(\Mcal)) = {}_A\kr\BMod_B(\Mcal).
\]
\end{example}

\begin{example}
Let $\Mcal$ be a commutative algebra in $\catl$ and let $A$ be an algebra in $\Mcal$. Then by example \ref{example tensor with amod} we have an equivalence 
\[
\LMod_A(\Mcal) \otimes_\Mcal \RMod_A(\Mcal) = \LMod_A(\RMod_B(\Mcal)) = {}_A\kr\BMod_A(\Mcal).
\]
 The diagonal bimodule for $A$ defines an object in $\LMod_A(\Mcal) \otimes_\Mcal \RMod_A(\Mcal)$, which then extends uniquely to an $\Mcal$-linear colimit preserving functor 
 \[
 \eta: \Mcal \rightarrow \LMod_A(\Mcal) \otimes_\Mcal \RMod_A(\Mcal).
 \]
 As discussed in \cite{HA} remark 4.8.4.8, the map $\eta$ exhibits $\LMod_A(\Mcal)$ and $\RMod_A(\Mcal)$ as dual objects in $\Mod_{\Mcal}(\catl)$. 
\end{example}

\begin{remark}
Let $\Mcal$ be a commutative algebra in $\catl$. Then we have a symmetric monoidal colimit preserving functor $\Cat \rightarrow \Mod_\Mcal(\catl)$ obtained by composing the free cocompletion functor $\Cat \rightarrow \catl$ with the free module functor $\catl \rightarrow \Mod_\mcal(\catl)$. It follows from this that $\Mod_\Mcal(\catl)$ has a structure of symmetric monoidal $2$-category, with the Hom category between two objects $\ccal$ and $\dcal$ being given by the category underlying $\Funct_{\Mcal}(\ccal, \dcal)$.

It makes sense in particular to consider adjunctions in $\Mod_\Mcal(\catl)$. Given an $\Mcal$-linear colimit preserving functor $f: \ccal \rightarrow \dcal$, we have that $f$ admits a right (resp. left) adjoint in $\Mod_\mcal(\catl)$ if and only if it admits a colimit preserving right (resp. left) adjoint as a functor of categories, which commutes strictly with the action of $\Mcal$.
\end{remark}

Assume given a morphism $f: \Mcal \rightarrow \Mcal'$ of commutative algebras in $\catl$. Then we obtain a symmetric monoidal extension of scalars functor 
\[
- \otimes_\Mcal \Mcal' : \Mod_{\Mcal}(\catl) \rightarrow \Mod_{\Mcal'}(\catl),
\] and a restriction of scalars right adjoint to it. If $\Ccal$ is an $\Mcal$-linear cocomplete category then the unit of the adjunction provides an $\Mcal$-linear colimit preserving functor $\Ccal \rightarrow \Ccal \otimes_\Mcal \Mcal'$ which we call extension of scalars along $f$. A right adjoint to it (which automatically exists if we work with presentable categories) is called restriction of scalars along $f$.

\begin{example}
Let $f: \Mcal \rightarrow \Mcal'$ be a morphism of commutative algebras in $\catl$. Let $A$ be an algebra in $\Mcal$ and consider the algebra $f(A)$ in $\Mcal'$. We may regard $\LMod_{f(A)}(\Mcal')$ as a cocomplete $\Mcal$-linear category by restriction of scalars long $f$. Then $f(A)$ becomes a right $A$-module in $\LMod_{f(A)}(\Mcal')$, so it induces an $\Mcal$-linear colimit preserving functor $\LMod_A(\Mcal) \rightarrow \LMod_{f(A)}(\Mcal')$, as discussed in example \ref{example LMod}. It follows from the universal properties of $\LMod_A(\Mcal)$ and $\LMod_{f(A)}(\Mcal')$  that this functor induces an equivalence 
\[
\LMod_A(\Mcal) \otimes_{\Mcal} \Mcal' = \LMod_{f(A)}(\Mcal').
\]
\end{example}

\begin{example}\label{example stabilize linear}
 Let $\Mcal$ be a commutative algebra in $\catl$ and let $\Mcal' = \Mcal \otimes \Sp$ be the stabilization of $\Mcal$. Then for each  $\Mcal$-linear cocomplete category $\ccal$ the extension of scalars functor $\Ccal \rightarrow \Ccal \otimes_\Mcal \Mcal'$ presents $\Ccal \otimes_\Mcal \Mcal'$ as the stabilization of $\Ccal$. In other words, the stabilization of an $\Mcal$-module is automatically a module over the stabilization of $\Mcal$. It follows in particular that the functor of restriction of scalars $\Mod_{\Mcal'}(\catl) \rightarrow \Mod_{\Mcal}(\catl)$ is fully faithful, and its image consists of those $\Mcal$-modules which are stable.
\end{example}

\begin{example}\label{example truncate linear}
Let $\Mcal$ be  a commutative algebra in $\catl$. Fix an $n \geq 1$ and  let $\Mcal' = \Mcal \otimes \Spc_{\leq n-1}$. Then for each $\Mcal$-linear cocomplete category $\ccal$ the extension of scalars functor $\Ccal \rightarrow \Ccal \otimes_{\Mcal} \Mcal'$ induces an equivalence $\Ccal \otimes_{\Mcal} \Mcal' = \Ccal \otimes \Spc_{\leq n-1}$. It follows in particular that the functor of restriction of scalars $\Mod_{\Mcal'}(\catl) \rightarrow \Mod_{\Mcal}(\catl)$ is fully faithful, and its image consists of those $\Mcal$-modules which are $(n,1)$-categories.
\end{example}


We now specialize the above discussion to obtain a notion of cocomplete categories linear over a base connective commutative ring spectrum.

\begin{definition}\label{def R linear cocomplete}
Let $R$ be a connective commutative ring spectrum. An $R$-linear cocomplete category is a $\Mod_R^\cn$-linear cocomplete category. An $R$-linear colimit preserving functor is a morphism in $\Mod_{\Mod_R^\cn}(\catl)$.
\end{definition} 

\begin{remark}
Let $R$ be a connective commutative ring spectrum. Then every $R$-linear cocomplete category is automatically additive. 
\end{remark}

\begin{remark}
Let $R$ be a connective commutative ring spectrum and let $\Ccal$ be an $R$-linear cocomplete category. The action of $\Mod_R^\cn$ on $\Ccal$ provides a monoidal functor $\Mod_R^\cn \rightarrow \Funct(\Ccal, \Ccal)$, which after passing to endomorphisms of the identity yields an $E_2$-map from $R$ into the center of $\Ccal$. In particular, given an element $x$ in $R$ and an object $X$ in $\Ccal$ we have an endofunctor of $X$ given by
\[
X = R \otimes X \xrightarrow{x \otimes \id} R \otimes X = X
\]
which we usually denote by $x: X \rightarrow X$ and call the action of $x$ on $X$.
\end{remark}

\begin{remark}
If $R$ is a (non necessarily connective) commutative ring spectrum then one may consider $\Mod_R$-linear cocomplete categories. We call these $R$-linear cocomplete stable categories. In the case when $R$ is connective, it follows from example \ref{example stabilize linear} that an $R$-linear cocomplete stable category in this sense is the same as an $R$-linear cocomplete category in the sense of definition \ref{def R linear cocomplete} which is in addition stable.
\end{remark}

\begin{remark}
For every connective commutative ring spectrum $R$ one may consider $(\Mod^\cn_R)_{\leq n-1}$-linear categories. We call these $R$-linear cocomplete $(n,1)$-categories. It follows from example \ref{example truncate linear} that an $R$-linear cocomplete $(n,1)$-category in this sense is the same as an $R$-linear cocomplete category in the sense of definition \ref{def R linear cocomplete} which is in addition an $(n,1)$-category. This will frequently be used in the case when $n = 0$ and $R$ is a (classical) commutative ring: in this case one obtains a notion of classical $R$-linear cocomplete category, which is simply a $\Mod^\heartsuit_R$-linear category.
\end{remark}

\begin{example}
Let $R$ be a connective commutative ring spectrum and let $\ccal$ be an $R$-linear cocomplete category. Then specializing examples \ref{example stabilize linear} and \ref{example truncate linear} yields the following:
\begin{itemize}
\item The stabilization $\ccal \otimes \Sp$ has a structure of $R$-linear cocomplete stable category.
\item For each $n \geq 0$ the $(n,1)$-category $\ccal \otimes \Spc_{\leq n-1}$ as a structure of $\tau_{\leq n-1}(R)$-linear cocomplete $(n,1)$-category.
\end{itemize}
\end{example}

\begin{remark}
Let $f: R \rightarrow R'$ be a morphism of connective commutative ring spectra. Then we obtain a symmetric monoidal extension of scalars functor $f^*: \Mod^\cn_R \rightarrow \Mod^\cn_{R'}$. For each $R$-linear cocomplete category $\Ccal$ we will denote by $\Ccal \otimes_R R'$ its extension of scalars along $f^*$, and by $- \otimes_R R': \Ccal \rightarrow \Ccal \otimes_R R'$ the corresponding extension of scalars functor. The composition $\Ccal \rightarrow \Ccal \otimes_R R' \rightarrow \Ccal$ is given by tensoring with the $R$-module $R'$, while the unit of the adjunction is given by tensoring with the map of $R$-modules $R \rightarrow R'$. Similar considerations apply to the case when $R$ and $R'$ are non necessarily connective commutative ring spectra and we extend scalars along $f^*: \Mod_R  \rightarrow \Mod_{R'} $.
\end{remark}

%%%%%%%%%%%%%%%%%%%%%%%%%%%%%%%%%%%%%%%%%%%%%%%%%%%%%%%%%%%%%%%%%%%%%%%%
%%%%%%%%%%%%%%%%%%%%%%%%%%%%%%%%%%%%%%%%%%%%%%%%%%%%%%%%%%%%%%%%%%%%%%%%
%%%%%%%%%%%%%%%%%%%%%%%%%%%%%%%%%%%%%%%%%%%%%%%%%%%%%%%%%%%%%%%%%%%%%%%%
%%%%%%%%%%%%%%%%%%%%%%%%%%%%%%%%%%%%%%%%%%%%%%%%%%%%%%%%%%%%%%%%%%%%%%%%
%%%%%%%%%%%%%%%%%%%%%%%%%%%%%%%%%%%%%%%%%%%%%%%%%%%%%%%%%%%%%%%%%%%%%%%%
%%%%%%%%%%%%%%%%%%%%%%%%%%%%%%%%%%%%%%%%%%%%%%%%%%%%%%%%%%%%%%%%%%%%%%%%

\subsection{Grothendieck abelian categories}\label{subsection abelian}

We now proceed with some recollections on the theory of Grothendieck abelian categories.

\begin{definition}\label{def grothendieck abelian}
A Grothendieck abelian category is an abelian category $\Ccal$ which is presentable and such that filtered colimits in $\Ccal$ are exact. We denote by $\Groth_1$ the category of Grothendieck abelian categories and colimit preserving functors.
\end{definition}

Definition \ref{def grothendieck abelian} is equivalent to the (perhaps more common) definition where presentability of $\Ccal$ is replaced by the requirement that $\Ccal$ is locally small, admits small colimits, and admits a generator. The following result provides an ample source of examples.

\begin{proposition}[\cite{SAG} proposition 10.6.3.1]
Let $\Ccal, \Dcal$ be presentable categories and assume given a functor $G: \Dcal \rightarrow \Ccal$ which is conservative and preserves small limits\footnote{In fact only preservation of finite limits is necessary.} and colimits. If $\Ccal$ is a Grothendieck abelian category, then so is $\Dcal$.
\end{proposition}

\begin{corollary}
Let $\Acal$ be a Grothendieck abelian category equipped with a monoidal structure compatible with colimits. Let $A$ be an algebra in $\Acal$. Then the category $\LMod_A(\Acal)$ of left $A$-modules in $\Acal$ is a Grothendieck abelian category.
\end{corollary}

Limits of Grothendieck abelian categories exist along left exact colimit preserving functors:

\begin{proposition}[\cite{SAG} proposition C.5.4.21]\label{prop limits along left exact}
Let $F: \Ical \rightarrow \Groth_1$ be a diagram whose transition maps are left exact. Then $F$ admits a limit which is preserved by the inclusion of $\Groth_1$ inside $\Pr^L$.
\end{proposition}

\begin{corollary}
The category $\Groth_1$ admits small products, and these are preserved by the inclusions $\Groth_1 \rightarrow \Pr^L \rightarrow \cathat$.
\end{corollary}

We can also form colimits of diagrams of right adjointable diagrams:

\begin{proposition}\label{proposition colimits of right adjointable}
Let $F: \Ical \rightarrow \Groth_1$ be a diagram whose transition maps admit colimit preserving right adjoints. Then $F$ admits a colimit which is preserved by the inclusion of $\Groth_1$ inside $\Pr^L$.
\end{proposition}
\begin{proof}
It suffices to show that the colimit of $F$ in $\Pr^L$ is Grothendieck abelian. This colimit agrees with the limit of the diagram $F^R : \Ical^\op \rightarrow \cathat$ obtained by passing to right adjoints of the morphisms in $F$. This a diagram of Grothendieck abelian categories and left exact colimit preserving functors. The fact that the limit is Grothendieck abelian now follows from proposition \ref{prop limits along left exact}.
\end{proof}

\begin{corollary}
The category $\Groth_1$ admits small direct sums, and these are preserved by the inclusion $\Groth_1 \rightarrow \Pr^L$. In particular, small direct sums and small direct products agree in $\Groth_1$.
\end{corollary}

There is a good theory of tensor products of Grothendieck abelian categories:

\begin{theorem}[\cite{SAG} theorem C.5.4.16, \cite{Tensor} theorem 5.4]\label{teo tensor product abelian}
Let $\Ccal, \Dcal$ be Grothendieck abelian categories. Then their tensor product $\Ccal \otimes \Dcal$ (formed in $\Pr^L$) is Grothendieck abelian. In particular, the symmetric monoidal structure on the category $\Mod_{\Ab}(\Pr^L)$ of presentable additive $(1,1)$-categories and colimit preserving functors restricts to a symmetric monoidal structure on $\Groth_1$.
\end{theorem}

In particular, it makes sense to consider commutative algebras in $\Groth_1$. We call these symmetric monoidal Grothendieck abelian categories. Note that this terminology leaves implicit the fact that the tensor operation commutes with colimits in each variable.

\begin{definition}
Let $\acal$ be a symmetric monoidal Grothendieck abelian category. An $\acal$-linear Grothendieck abelian category is an object of $\Mod_\Acal(\Groth_1)$.
\end{definition}

In other words, an $\acal$-linear Grothendieck abelian category  is an $\acal$-linear presentable category (in the sense of section \ref{subsection linear cats}) which is in addition a Grothendieck abelian category. In the case when $\acal = \Mod_R^\heartsuit$ is the category of classical modules over a connective $E_\infty$-ring $R$, we call these $R$-linear Grothendieck abelian categories.

Since the class of colimits that we have available in $\Groth_1$ is relatively restricted, one has to be careful when forming relative tensor products. There is however a class of commutative algebras that admit a well behaved theory of relative tensor products.

\begin{definition}
Let $\acal$ be a commutative algebra in $\groth$. Assume that $\acal$ is generated by compact projective objects. We say that $\acal$ is rigid if compact projective and dualizable objects of $\acal$ coincide.
\end{definition}

\begin{example}
Let $R$ be a commutative ring. Then $\Mod_R(\Ab)$ is rigid.
\end{example}

We fix for the remainder of this section a base symmetric monoidal Grothendieck abelian category $\acal$, generated by compact projective objects and rigid.

\begin{remark}\label{remark tensor compact projectives}
 Let $\Ccal$ be an $\Acal$-linear Grothendieck abelian category. Then if $X$ is a compact projective object of $\Acal$ the functor $X \otimes - : \Ccal \rightarrow \Ccal$ admits both a left and a right adjoint, given by $X^\vee \otimes -$. In particular, $X \otimes -$ admits a colimit preserving right adjoint, and therefore it maps compact projective objects of $\Ccal$ to compact projective objects.
\end{remark}

\begin{remark}
Let $f: \Ccal \rightarrow \Dcal$ be a morphism in $\Mod_\Acal(\catl)$ and assume that $f$ admits a right adjoint $f^R$ (as a functor of categories, ignoring the $\Acal$-action). Then $f^R$ commutes laxly with the action of $\Acal$, and strictly with the action of the dualizable objects in $\Acal$. In particular, since $\Acal$ is generated under colimits by dualizable objects, we see that if $f^R$ is colimit preserving then it commutes strictly with the action of $\Acal$.
\end{remark}

\begin{proposition}\label{proposition action has colimit preserving adjoint}
Let $\ccal$ be an $\acal$-module in $\catl$. Then the action map $\acal \otimes \ccal \rightarrow \ccal$ admits a colimit preserving right adjoint.
\end{proposition}
\begin{proof}
We first prove the proposition in the case when $\Ccal = \Acal \otimes \Ccal'$ is a free $\Acal$-module. In this case the action map is obtained by tensoring the multiplication map $\mu: \Acal \otimes \Acal \rightarrow \Acal$ with the identity on $\Ccal'$. We may thus reduce to showing that $\mu$ admits a colimit preserving right adjoint. This is a consequence of the fact that $\Acal$ is generated by compact projective objects and that such objects are preserved by tensor products.

We now prove the general case. Consider the Bar resolution $\Acal^{\otimes \bullet + 1} \otimes \Ccal$ of $\Ccal$. Then the action map $\Acal \otimes \Ccal \rightarrow \Ccal$ is the colimit of the action maps $\Acal \otimes (\Acal^{\otimes \bullet + 1} \otimes \Ccal) \rightarrow \Acal^{\otimes \bullet + 1} \otimes \Ccal$. Since the Bar resolution is levelwise free we see that each of these maps admits a colimit preserving right adjoint. To prove the proposition it remains to show that for each face map $\sigma: [n] \rightarrow [n+1]$ in $\Delta$ the induced commutative square
\begin{equation}
\begin{tikzcd}
\Acal \otimes (\Acal^{\otimes n+2} \otimes \Ccal) \arrow{d}{} \arrow{r}{} & \Acal \otimes (\Acal^{\otimes n+1} \otimes \Ccal) \arrow{d}{} \\
\Acal^{\otimes n+2} \otimes \Ccal \arrow{r}{}  & \Acal^{\otimes n+1} \otimes \Ccal
\end{tikzcd}
\end{equation}
is vertically right adjointable. If $\sigma$ is not the $0$-th face then the above square is a tensor product of the square
\[
\begin{tikzcd}
\Acal \otimes \Acal \arrow{d}{\mu} \arrow{r}{\id} & \Acal \otimes \Acal \arrow{d}{\mu} \\
\Acal \arrow{r}{\id} & \Acal 
\end{tikzcd}
\] 
with
\[
\begin{tikzcd}
\Acal^{\otimes n+1} \otimes \Ccal \arrow{d}{\id} \arrow{r}{} & \Acal^{\otimes n} \otimes \Ccal \arrow{d}{\id} \\
\Acal^{\otimes n+1} \otimes \Ccal \arrow{r}{}  & \Acal^{\otimes n} \otimes \Ccal
\end{tikzcd}
\]
and our claim follows from the fact that these two are vertically right adjointable. It remains to analyze the case when $\sigma$ is the $0$-th face. In this case the square (1) is obtained by tensoring the square
\[
\begin{tikzcd}
\Acal \otimes \Acal \otimes \Acal \arrow{d}{\mu \otimes \id} \arrow{r}{\id \otimes \mu} & \Acal \otimes \Acal \arrow{d}{\mu} \\
\Acal \otimes \Acal \arrow{r}{\mu} & \Acal
\end{tikzcd}
\]
with $\Acal^{\otimes n} \otimes \Ccal$.  We may thus reduce to showing that the above is vertically right adjointable. This amounts to showing that the right adjoint to the multiplication map $\Acal \otimes \Acal \rightarrow \Acal$ is $\Acal$-linear. This follows from the fact that $\Acal$ is generated under colimits by dualizable objects.
\end{proof}

\begin{corollary}\label{coro tensor products over A}
The full subcategory of $\Mod_\acal(\Pr^L)$ on the $\acal$-linear Grothendieck abelian categories is closed under tensor products. In other words, $\Mod_\acal(\groth_1)$ admits a symmetric monoidal structure that makes the inclusion $\Mod_\acal(\groth_1) \rightarrow \Mod_\acal(\Pr^L)$ symmetric monoidal.
\end{corollary}
\begin{proof}
Let $\Ccal$ and $\Dcal$ be a pair of $\acal$-linear Grothendieck abelian categories. The relative tensor product $\ccal \otimes_\acal \Dcal$ in $\Pr^L$  is the geometric realization of the Bar construction $\ccal \otimes \acal^{\otimes \bullet} \otimes \Dcal$. By virtue of proposition \ref{proposition colimits of right adjointable}, to show that $\ccal \otimes_\acal \Dcal$ is Grothendieck abelian, it suffices to show that the face maps in the Bar construction admit colimit preserving right adjoints. This follows from proposition \ref{proposition action has colimit preserving adjoint}.
\end{proof}

\begin{corollary}
Let $\ccal$ and $\dcal$ be a pair of $\acal$-linear Grothendieck abelian categories. Then the functor $\ccal \otimes \dcal \rightarrow \ccal \otimes_\acal \dcal$ admits a colimit preserving right adjoint. In particular, for each pair of compact projective objects $X$ in $\ccal$ and $Y$ in $\dcal$, the object $X \otimes Y$ in $\ccal \otimes_\acal \dcal$ is compact projective.
\end{corollary}
\begin{proof}
Follows directly from the fact that the functor $\ccal \otimes \dcal \rightarrow \ccal \otimes_\acal \dcal$ arises from the geometric realization of a simplicial diagram whose face maps have colimit preserving right adjoints.
\end{proof}

We will frequently use the following relative variant of the notion of generator:

\begin{definition}\label{definition A generator}
Let $\ccal$ be an $\acal$-linear Grothendieck abelian category. An object $G$ in $\ccal$ is said to be an $\acal$-generator if $\ccal$ is generated by the family of objects $X \otimes G$ with {$X$ in $\acal$.} 
\end{definition}

\begin{remark}
Let $\Ccal$ be an $\acal$-linear Grothendieck abelian category. Then an object $G$ in $\ccal$ is an $\acal$-generator if and only if $\ccal$ is generated by the family of objects $X \otimes G$ with $X$ a compact projective object of $\acal$.
\end{remark}

The following is an $\acal$-linear version of the Gabriel-Popescu theorem:

\begin{proposition}\label{prop lex localization}
Let $\Ccal$ be an $\Acal$-linear Grothendieck abelian category and let $G$ be an $\acal$-generator for $\Ccal$. Let $A$ be the opposite of the algebra of endomorphisms of $G$ associated to the action of $\Acal$ on $\Ccal$. Then the functor
\[
G \otimes_A - : \LMod_A(\Acal) \rightarrow \Ccal 
\]
is an $\Acal$-linear left exact localization. Furthermore, it is an equivalence if and only if $G$ is compact projective.
\end{proposition}
\begin{proof}
Let $\Ccal_0$ be the full subcategory of $\ccal$ on the objects of the form $X \otimes G$ where $X$ is a compact projective object of $\Acal$, and let $\Dcal$ be the full subcategory of $\LMod_A(\Acal)$ on the objects of the form $X \otimes A$ where $X$ is a compact projective object of $\Acal$. If $X, Y$ are compact projective objects of $\Acal$ then we have
\begin{align*}
\Hom_{\LMod_A(\Acal)}(X \otimes A, Y \otimes A) &= \Hom_{\LMod_A(\Acal)}(X \otimes Y^\vee \otimes A, A) \\ &= \Hom_{\Ccal}(X \otimes Y^\vee \otimes G, G)  \\ &= \Hom_{\Ccal}(X \otimes G, Y \otimes G)
\end{align*}
and therefore $G \otimes_A -$ restricts to an equivalence $\Dcal \rightarrow \Ccal_0$. We now have a commutative square of  categories
\[
\begin{tikzcd}
\Pcal^1_\Sigma(\Dcal) \arrow{d}{} \arrow{r}{} & \Pcal^1_\Sigma(\Ccal_0) \arrow{d}{} \\
\LMod_A(\Acal) \arrow{r}{ G \otimes_A -} & \Ccal
\end{tikzcd}
\]
where the categories on the top row are the $(1,1)$-categories obtained from $\Dcal$ and $\Ccal_0$ by freely adjoining sifted colimits, and the vertical arrows are the unique sifted colimit preserving extensions of the inclusions $\Dcal \rightarrow \LMod_A(\Acal)$ and $\Ccal_0 \rightarrow \Ccal$. The upper horizontal arrow is an equivalence since $G \otimes_A -$ restricts to an equivalence $\Dcal \rightarrow \Ccal_0$. Furthermore, $\Dcal$ is a generating family of compact projective objects of $\LMod_A(\Acal)$, and therefore the left vertical arrow is an equivalence as well. 

The fact that $G \otimes_A -$ is a left exact localization now follows from the fact that the functor $\Pcal^1_\Sigma(\Ccal_0) \rightarrow \Ccal$ is a left exact localization, due to the many object version of the classical Gabriel-Popescu theorem (see \cite{Kuhn} theorem 2.1, or theorem C.2.2.1 of \cite{SAG}). It remains to show that $G \otimes_A -$ is an equivalence if and only if $G$ is compact projective. The only if direction follows from the fact that $A$ is a compact projective left $A$-module. To prove the if direction, we observe that if $G$ is compact projective then $\Ccal_0$ is a generating family of compact projective objects of $\Ccal$, so that the functor $\Pcal^1_\Sigma(\Ccal_0) \rightarrow \Ccal$ is an equivalence.
\end{proof}

We finish this section with a discussion of flatness in the context of $\Acal$-linear Grothendieck abelian categories.

\begin{definition}\label{definition flat abelian}
Let $\Ccal$ be an $\Acal$-linear Grothendieck abelian category. We say that an object $X$ in $\Ccal$ is flat over $\acal$ if the functor $- \otimes X : \Acal \rightarrow \Ccal$ is left exact. In cases when the base $\acal$ is clear from the context we simply  say that $X$ is flat.
\end{definition}

\begin{example}
Let $R$ be a commutative ring and let $\Ccal$ be an $R$-linear Grothendieck abelian category. Since every monomorphism of $R$-modules is a transfinite composition of pushouts of inclusions of ideals into $R$, we have that an object $X$ in $\Ccal$ is flat if and only if the morphism
\[
I \otimes X \xrightarrow{i \otimes \id} R \otimes X = X
\]
is a monomorphism for all inclusions of ideals $i: I \rightarrow R$.
\end{example}

As a particular case of definition \ref{definition flat abelian} we obtain a notion of flatness for objects of $\acal$. These are characterized by the following variant of Lazard's theorem:

\begin{proposition}\label{prop lazard classico}
Let $X$ be an object of $\Acal$. Then $X$ is flat if and only if it is a filtered colimit of compact projective objects.
\end{proposition}
\begin{proof}
We first show that if $X$ is a filtered colimit of compact projective objects then it is flat. Since filtered colimits in $\Acal$ are left exact it suffices to consider the case when $X$ is compact projective. This follows from the fact that the functor $- \otimes X : \Acal \rightarrow \Acal$ has a left adjoint.

Assume now that $X$ is flat. Let $\acal^\cp$ be the full subcategory of $\acal$ on the compact projective objects and consider the functor $F(-): (\acal^\cp)^\op \rightarrow \Spc$ represented by $X$. We wish to show that this functor defines an ind-object of $\acal^\cp$. Let $D: \acal^\cp \rightarrow (\acal^\cp)^\op$ be the dualization equivalence. We will prove that $F(D(-)): \acal^\cp \rightarrow \Spc$ defines a pro-object of $\acal^\cp$.

Let $p: \Ecal \rightarrow \acal$ be the left fibration associated to the functor $\Hom_{\acal}(1_\acal, - \otimes X)$. Then the base change of $p$ to $\acal^\cp$ is the left fibration classifying $F(D(-))$. We have to show that every finite diagram $G: \Ical \rightarrow \Ecal \times_\acal \acal^\cp$ admits a left cone. The fact that $X$ is flat implies that the functor $\Hom_{\acal}(1_\acal, - \otimes X)$ is left exact, and therefore $G$ extends to a left cone $G^\lhd: \Ical^\lhd \rightarrow \Ecal$. Let $\overline{Y} = (Y, \rho: 1_\acal \rightarrow Y \otimes X)$ be the value of $G^\lhd$ at the cone point. To show that $G$ extends to a left cone in $\Ecal \times_\acal \acal^\cp$ it is enough to prove that $\overline{Y}$ receives a map from an object in $\Ecal \times_\acal \acal^\cp$. This amounts to showing that there exists a map $Y' \rightarrow Y$ from a compact projective object with the property that $\rho$ factors through $Y' \otimes X$. This follows from the fact that $1_\acal$ is compact projective.
\end{proof}

We now study the behavior of flatness under tensor products.

\begin{proposition}\label{prop tensoring left exact}
Let $f: \Ccal \rightarrow \Ccal'$ and $g: \Dcal \rightarrow \Dcal'$   morphisms in $\Mod_{\Acal}(\Groth_1)$. If $f$ and $g$ are left exact then $f \otimes_{\Acal} g : \Ccal \otimes_\acal \Dcal \rightarrow \Ccal' \otimes_\Acal \Dcal'$ is left exact.
\end{proposition}
\begin{proof}
Since $f \otimes_\Acal g$ is the composition of $f \otimes_\Acal \id_{\Dcal}$ and $\id_{\Ccal'} \otimes_\Acal g$, it suffices to prove that these two functors are left exact. Changing the role of $f$ and $g$ we may reduce to showing that $f \otimes_\Acal \id_{\Dcal}$ is left exact. Pick an algebra $B$ in $\Acal$ and an $\Acal$-linear left exact localization $q: \LMod_B(\Acal) \rightarrow \Dcal$. We have a commutative square of categories
\[
\begin{tikzcd}
\Ccal\otimes_\Acal \LMod_B(\Acal) \arrow{r}{f \otimes  \id} \arrow{d}{\id \otimes  q} & \Ccal' \otimes_\Acal \LMod_B(\Acal) \arrow{d}{\id \otimes  q} \\
\Ccal \otimes_\Acal \Dcal \arrow{r}{f \otimes  \id} & \Ccal' \otimes_\Acal \Dcal.
\end{tikzcd}
\]
Here the upper horizontal arrow is equivalent to the functor $\LMod_B(\Ccal) \rightarrow \LMod_B(\Ccal')$ induced by $f$, and is therefore left exact. To prove the proposition it will suffice to show that the left vertical arrow is a left exact localization, and that the right vertical arrow is left exact. Changing the role of $\Ccal$ and $\Ccal'$ we see that it suffices to show that the left vertical arrow is a left exact localization.

Pick an algebra $A$ in $\Acal$ and an $\Acal$-linear left exact localization $p: \LMod_A(\Acal) \rightarrow \Ccal$. We now have a commutative square of categories
\[
\begin{tikzcd}
\LMod_A(\Acal) \otimes_\acal \LMod_B(\Acal) \arrow{r}{p \otimes  \id} \arrow{d}{\id \otimes  q} & \Ccal \otimes_\Acal \LMod_B(\Acal) \arrow{d}{\id \otimes   q} \\
\LMod_A(\Acal) \otimes_\acal  \Dcal \arrow{r}{p \otimes \id} & \Ccal \otimes_\Acal \Dcal.
\end{tikzcd}
\]
The upper horizontal arrow is equivalent to the functor $\LMod_B(\LMod_A(\Acal))  \rightarrow \LMod_B(\Ccal)$ induced by $p$, and is therefore a left exact localization. Similarly, the left vertical arrow is a left exact localization. To prove the proposition it will suffice to show that the diagonal map $p \otimes  q$ is a left exact localization as well.

We have that $p \otimes q$ is an epimorphism in $\Pr^L$, with the property that a map 
\[
f: \LMod_A(\Acal) \otimes_\Acal \LMod_B(\Acal)  \rightarrow \Ecal
\]
factors through $\Ccal \otimes_\Acal \Dcal$ if and only if its restriction to $\LMod_A(\Acal) \otimes \LMod_B(\Acal)$ factors through $\Ccal \otimes \Dcal$. Similarly, the upper horizontal arrow (resp. left vertical arrow) is an epimorphism with the property that $f$ factors through it if and only if the restriction of $f$ to $\LMod_A(\Acal) \otimes \LMod_B(\Acal)$  factors through $\Ccal \otimes \LMod_{B}(\Acal)$ (resp. $\LMod_A(\Acal) \otimes \Dcal$). It follows that $f$ factors through $p \otimes  q$ if and only if it factors through both $p \otimes  \id$ and $\id \otimes  q$, so that $p \otimes  q$ is a localization at the union of the class of arrows inverted by the latter two maps. The fact that $p \otimes  q$ is left exact now follows from \cite{SAG} lemma C.4.3.1.
\end{proof}

\begin{corollary}\label{coro tensor flats}
Let $\Ccal, \Dcal$ be $\Acal$-linear Grothendieck abelian categories, and let $X, Y$ be flat objects of $\Ccal$ and $\Dcal$ respectively. Then the object $X \otimes Y$ in $\Ccal \otimes_\Acal \Dcal$ is flat.
\end{corollary}
\begin{proof}
Let $F: \Acal \rightarrow \Ccal$ (resp. $G: \Acal \rightarrow \Dcal$) be the unique $\Acal$-linear colimit preserving functor sending the unit to $X$ (resp. $Y$). Then $X \otimes Y$ is the image of the unit under the composite functor
\[
\Acal = \Acal \otimes_\Acal \Acal \xrightarrow{F \otimes G} \Ccal \otimes_\Acal \Dcal.
\]
To show that $X \otimes Y$ is flat we must show that the above functor is left exact. This is a direct consequence of proposition \ref{prop tensoring left exact}.
\end{proof}

%%%%%%%%%%%%%%%%%%%%%%%%%%%%%%%%%%%%%%%%%%%%%%%%%%%%%%%%%%%%%%%%%%%%%%%%
%%%%%%%%%%%%%%%%%%%%%%%%%%%%%%%%%%%%%%%%%%%%%%%%%%%%%%%%%%%%%%%%%%%%%%%%
%%%%%%%%%%%%%%%%%%%%%%%%%%%%%%%%%%%%%%%%%%%%%%%%%%%%%%%%%%%%%%%%%%%%%%%%
%%%%%%%%%%%%%%%%%%%%%%%%%%%%%%%%%%%%%%%%%%%%%%%%%%%%%%%%%%%%%%%%%%%%%%%%
%%%%%%%%%%%%%%%%%%%%%%%%%%%%%%%%%%%%%%%%%%%%%%%%%%%%%%%%%%%%%%%%%%%%%%%%
%%%%%%%%%%%%%%%%%%%%%%%%%%%%%%%%%%%%%%%%%%%%%%%%%%%%%%%%%%%%%%%%%%%%%%%%

\subsection{Spectral categories}\label{subsection spectral categories}

We now review the notion of spectral categories from \cite{Spectral}.

\begin{definition}
Let $\Ccal$ be a Grothendieck abelian category. We say that $\Ccal$ is spectral if every exact sequence in $\Ccal$ splits.
\end{definition}

Various finiteness conditions become equivalent for objects in a spectral category:

\begin{proposition}\label{prop equiv finiteness}
Let $\Ccal$ be a spectral category and let $X$ be an object in $\Ccal$. The following are equivalent:
\begin{enumerate}[\normalfont(a)]
\item $X$ is a finite direct sum of simple objects.
\item $X$ is compact.
\item $X$ is finitely generated.\footnote{An object $X$ in a Grothendieck abelian category is said to be finitely generated if $X$ is compact as an object in its poset of subobjects.}
\end{enumerate}
\end{proposition}
\begin{proof}
Condition (b) clearly implies (c). Assume now that $X$ is finitely generated. Since every subobject of $X$ is a direct summand of $X$, we see that every subobject of $X$ is also finitely generated. Hence $X$ is Noetherian. Assume now given a decreasing sequence of subobjects $X_n$ of $X$. Using the fact that $\Ccal$ is spectral we may inductively construct a sequence of complements $X_n^c$ for $X_n$ with the property that $X_n^c \subseteq X_{n+1}^c$ for all $n$. Since $X$ is Noetherian we have that the sequence $X_n^c$ is eventually constant, and hence $X_n$ is eventually constant as well. We conclude that $X$ is also Artinian, and so it has finite length. The fact that $X$ is spectral now implies that $X$ is a finite direct sum of simple objects. Thus we see that (c) implies (a).

It remains to show that (a) implies (b). For this it suffices to show that if $S$ is a simple object in $\Ccal$, then $S$ is compact. We will do so by showing that  $\Hom^\enh_\Ccal(S, -) : \Ccal \rightarrow \Ab$ preserves colimits. The fact that it is right exact follows from the fact that $\Ccal$ is spectral. We may therefore reduce to showing that $\Hom^\enh_\Ccal(S, -)$ preserves infinite direct sums. 

Let $Y_\alpha$ be a family of objects of $\Ccal$ indexed by a set $\Lambda$. We need to show that the map
\[
\bigoplus_{\alpha \in \Lambda} \Hom^\enh_\Ccal(S, Y_\alpha) \rightarrow \Hom^\enh_\Ccal(S, \bigoplus_{\alpha \in \Lambda} Y_\alpha)
\]
is an isomorphism. The fact that the above is a monomorphism is a general fact about Grothendieck abelian categories (and does not use the simplicity of $S$). It remains to show that every morphism $S \rightarrow \bigoplus_{\alpha \in \Lambda} Y_\alpha$ factors through $\bigoplus_{\alpha \in \Lambda'} Y_\alpha$ for some finite subset $\Lambda' \subseteq \Lambda$. This follows from the fact that $S$ is finitely generated, since $\bigoplus_{\alpha \in \Lambda}Y_\alpha$ is the filtered union of the subobjects $\bigoplus_{\alpha \in \Lambda'} Y_\alpha$ over all finite $\Lambda'$.
\end{proof}

The most familiar spectral categories are the semisimple ones, which admit a number of equivalent characterizations:

\begin{proposition}\label{prop equivalences semisimple}
Let $\Ccal$ be a Grothendieck abelian category. The following are equivalent:
\begin{enumerate}[\normalfont (a)]
\item $\Ccal$ is locally finitely generated\footnote{A Grothendieck abelian category $\Ccal$ is said to be locally finitely generated if it is generated by its finitely generated objects.} and spectral.
\item $\Ccal$ is generated by compact projective objects and spectral.
\item Every object of $\Ccal$ is a direct sum of simple objects.
\end{enumerate}
\end{proposition}
\begin{proof}
The fact that (a) and (b) are equivalent, and the fact that (c) implies these, are both consequences of proposition \ref{prop equiv finiteness}. It remains to show that if (a) and (b) hold then every object of $\Ccal$ is a direct sum of simple objects. Let $X$ be an object of $\Ccal$. We construct a strictly increasing transfinite sequence of subobjects $X_\alpha$ of $X$ by induction as follows:
\begin{itemize}
\item Let $X_0 = 0$.
\item If $\alpha$ is a limit ordinal then we let $X_\alpha = \colim_{\beta < \alpha} X_\beta$.
\item Assume $\alpha = \beta + 1$ is a successor ordinal and $X_\alpha \neq X$. Choose a complement $Y$ for $X_\beta$ inside $X$. Since $\Ccal$ is assumed to be locally finitely generated we may pick a nonzero finitely generated subobject $Y'$ of $Y$. Since $\Ccal$ is spectral, an application of proposition \ref{prop equiv finiteness} shows that $Y'$ contains a simple subobject $S$. We let $X_\alpha = X_\beta \oplus S$.
\end{itemize}
The above construction ends whenever it reaches a small cardinal $\alpha$ with $X_\alpha = X$. The proposition now follows from the fact that $X_\alpha$ is a direct sum of simple objects for all $\alpha$.
\end{proof}

\begin{definition}
Let $\Ccal$ be a Grothendieck abelian category. We say that $\ccal$ is semisimple if it satisfies the equivalent conditions of proposition \ref{prop equivalences semisimple}.
\end{definition}

We will be interested in understanding spectral categories linear over a base. We fix for the remainder of this section a symmetric monoidal Grothendieck abelian category $\acal$, rigid and generated by compact projective objects.  We will need the notion of von Neumann regularity for algebras in $\acal$. Before introducing this notion we recall some basic concepts from ring theory in the relative context:

\begin{definition}
Let $A$ be an algebra in $\Acal$. 
\begin{itemize}
\item A left ideal of $A$ is a subobject of $A$ in $\LMod_A(\Acal)$.
\item A left $A$-module $M$ is said to be flat if the functor $- \otimes_A M : \RMod_A(\acal) \rightarrow \acal$ is left exact.
\item We say that $R$ is left self-injective if $R$ is an injective object of $\LMod_A(\acal)$.
\end{itemize}
We also define the right variants of the above notions in a similar way.
\end{definition}

\begin{lemma}\label{lemma equivalences dualizable}
Let $A$ be an algebra in $\Acal$ and let $M$ be a left $A$-module. The following are equivalent:
\begin{enumerate}[\normalfont(a)]
\item The left module $M$ admits a left dual.
\item The functor $- \otimes_A M : \RMod_A(\acal) \rightarrow \acal$   preserves limits.
\item $M$ is finitely generated projective.
\item $M$ is a retract of a free left $A$-module on a dualizable object of $\Acal$.
\item $M$ is finitely presented and flat.
\end{enumerate}
\end{lemma}
\begin{proof}
The existence of a left dual to $M$ is equivalent to the existence of an $\Acal$-linear left adjoint to $-\otimes_A M$. Since $\Acal$ is generated under colimits by its dualizable objects any left adjoint to $- \otimes_A M$ is automatically $\Acal$-linear. The equivalence of (a) and (b) now follows directly from the adjoint functor theorem.

The fact that (c) implies (d) is a direct consequence of the fact that $\LMod_A(\Acal)$ is generated by free modules on dualizable objects. The fact that (d) implies (c) follows from the fact that the property of being compact projective is stable under retracts and passage to free modules. 

We now show that (a) implies (d). If $M$ admits a left dual $M^\vee$ then  we have 
\[
\Hom_{\RMod_A(\Acal)}(M^\vee, -) = \Hom_\Acal(1_\acal, - \otimes_A M)
\] which preserves sifted colimits since the unit in $\Acal$ is compact projective. It follows that in this case $M^\vee$ is a compact projective right $A$-module. Since $\RMod_A(\Acal)$ is generated by objects of the form $A \otimes V$ with $V$ a dualizable object of $\Acal$ we conclude that $M^\vee$ is a retract of $V \otimes A$ for some dualizable $V$. The right $A$-module $V \otimes A$ admits a right dual given by $V^\vee \otimes A$. It follows that $M$ is a retract of $V^\vee \otimes A$, so that (d) holds.

Assume now that (d) holds, so that $M$ is retract of $A \otimes V$ for some dualizable $V$. Then $- \otimes_A M$ is a retract of the composite functor
\[
\RMod_A(\Acal) \xrightarrow{ - \otimes_A A}  \Acal \xrightarrow{ \otimes V} \Acal.
\]
Each of the two functors above preserve limits. It follows that $- \otimes_A M$ preserves limits, so that (b) holds.

It remains to show that properties (a) through (d) are equivalent to (e). Assume first that (a) through (d) hold. The flatness of $M$ is then a consequence of (b), while the fact that $M$ is finitely presented follows from (d).
 
We finish the proof by showing that (e) implies (b). Since $M$ is flat, it is enough to show that $- \otimes_A M$ preserves products. Pick an exact sequence $P' \rightarrow P \rightarrow M \rightarrow 0$ with $P, P'$ finitely generated projective left $A$-modules. Let $N_\alpha$ be a family of right $A$-modules. Then we have a commutative diagram
\[
\begin{tikzcd}
(\prod_\alpha N_\alpha) \otimes_A P' \arrow{d}{} \arrow{r}{} & (\prod_\alpha N_\alpha) \otimes_A P \arrow{r}{} \arrow{d}{} & (\prod_\alpha N_\alpha) \otimes_A M \arrow{r}{} \arrow{d}{} & 0  \\
\prod_\alpha N_\alpha \otimes_A P' \arrow{r}{} &\prod_\alpha N_\alpha \otimes_A P \arrow{r}{} & \prod_\alpha N_\alpha \otimes_A M \arrow{r}{}& 0.
\end{tikzcd}
\]
The upper row is evidently exact, and the bottom row is exact since $\Acal$ is generated by compact projective objects (and thus products are exact in $\acal$). We now finish by observing that the left and middle vertical arrows are isomorphisms, by applying the equivalence of (b) and (c) for the modules $P$ and $P'$.
\end{proof}

\begin{proposition}\label{prop equivalences VN}
Let $A$ be an algebra in $\Acal$. The following are equivalent:
\begin{enumerate}[\normalfont(a)]
\item Every finitely generated submodule of a finitely generated projective left $A$-module is a direct summand.
\item Every finitely generated submodule of a finitely generated projective right $A$-module is a direct summand.
\item Every finitely presented left $A$-module is projective.
\item Every finitely presented right $A$-module is projective.
\item Every left $A$-module is flat.
\item Every right $A$-module is flat.
\end{enumerate}
\end{proposition}
\begin{proof}
Let $N \subseteq M$ be a finitely generated submodule of a finitely generated projective left $A$-module. Then $M/N$ is a finitely presented left $R$-module. Conversely, every finitely presented left $A$-module may be written in such a way. It follows from this that (a) and (c) are equivalent.

Since every left $A$-module is a filtered colimit of finitely presented left $A$-modules, and filtered colimits preserve flatness, we see that (c) implies (e). The fact that (e) implies (c) is a direct consequence of lemma \ref{lemma equivalences dualizable}.

The same arguments applied to $A^\op$ prove the equivalence of (b), (d), and (f). To finish it suffices to show that the left versions imply the right versions. Assume that (a) holds. Let $i: M' \rightarrow M$ be an inclusion of left $A$-modules and let $N$ be a right $A$-module. We will show that $N \otimes_A i$ is a monomorphism.

Write $M$ as a filtered colimit of a family of finitely presented left $A$-modules $M_\alpha$. Then $i$ is a filtered colimit of the induced inclusions $i_\alpha: M' \times_{M} M_\alpha \rightarrow M_\alpha$. It suffices to show that $N \otimes_A i_\alpha$ is a monomorphism for all $\alpha$. In other words, we may reduce to the case when $M$ is finitely presented.

Write $M'$ as a filtered union of finitely generated subobjects $M'_\beta$. Then $i$ is the filtered colimit of the inclusions $i_\beta: M'_\beta \rightarrow M$, and it suffices to show that $N \otimes_A i_\beta$ is a monomorphism for all $\beta$. In other words, we may further reduce to the case when $M'$ is finitely generated. This now follows from the fact that $i$ is the inclusion of a summand.
\end{proof}

\begin{definition}
Let $A$ be an algebra in $\Acal$. We say that $A$ is von Neumann regular if it satisfies the equivalent conditions of proposition \ref{prop equivalences VN}.
\end{definition}

The following is an $\acal$-linear version of \cite{Spectral} theorem 2.1:

\begin{proposition}\label{prop classif spectral}
Let $\Ccal$ be an $\Acal$-linear spectral category. Then there exists a left self-injective von Neumann regular algebra $A$ in $\Acal$ and an $\Acal$-linear left exact localization 
\[
\LMod_A(\Acal) \rightarrow \Ccal.
\]
\end{proposition}
\begin{proof}
Let $G$ be an $\acal$-generator for $\Ccal$ and let $A$ be the opposite to the algebra of endomorphisms of $G$. We will show that $A$ is left self-injective von Neumann regular.

Denote by $q: \LMod_A(\Acal) \rightarrow \Ccal$ the functor of tensoring with $G$ and by $i$ its right adjoint. Since $q$ is left exact and every object of $\Ccal$ is injective, we see that every left $A$-module in the image of $i$ is injective. In particular this holds for $i(G) = A$, and so $A$ is left self-injective.

It remains to show that $A$ is von Neumann regular. We will do so by showing that condition (a) in proposition \ref{prop equivalences VN} holds. Let $M$ be a finitely generated projective left $A$-module and let $N$ be a finitely generated submodule of $M$. We may write $N$ as the image of a map $\alpha: M' \rightarrow M$ of finitely generated projective left $A$-modules. Each of $M$ and $M'$ is a direct summand of a left $A$-module of the form $A \otimes X$ with $X$ a dualizable object of $\Acal$. Since $A \otimes X = i(G \otimes X)$ and $\Ccal$ is idempotent complete, we see that $M$ and $M'$, and therefore also $\alpha$, belong to the image of $i$. Since $\Ccal$ is spectral, every morphism in $\Ccal$ may be written as the composition of a retraction followed by a section. Hence $\alpha$ is a composition of a retraction followed by a section. This is necessarily equivalent to the image factorization for $\alpha$, so we deduce that the inclusion $N \rightarrow M$ is a section, as desired.
\end{proof}

%%%%%%%%%%%%%%%%%%%%%%%%%%%%%%%%%%%%%%%%%%%%%%%%%%%%%%%%%%%%%%%%%%%%%%%%
%%%%%%%%%%%%%%%%%%%%%%%%%%%%%%%%%%%%%%%%%%%%%%%%%%%%%%%%%%%%%%%%%%%%%%%%
%%%%%%%%%%%%%%%%%%%%%%%%%%%%%%%%%%%%%%%%%%%%%%%%%%%%%%%%%%%%%%%%%%%%%%%%
%%%%%%%%%%%%%%%%%%%%%%%%%%%%%%%%%%%%%%%%%%%%%%%%%%%%%%%%%%%%%%%%%%%%%%%%
%%%%%%%%%%%%%%%%%%%%%%%%%%%%%%%%%%%%%%%%%%%%%%%%%%%%%%%%%%%%%%%%%%%%%%%%
%%%%%%%%%%%%%%%%%%%%%%%%%%%%%%%%%%%%%%%%%%%%%%%%%%%%%%%%%%%%%%%%%%%%%%%%

\subsection{Grothendieck prestable categories} \label{subsection prestables}

We now review the theory of Grothendieck prestable categories, as introduced in \cite{SAG} appendix C.

\begin{definition}\label{def prestable}
A Grothendieck prestable category is a presentable category $\Ccal$ satisfying the following properties:
\begin{enumerate}[\normalfont (a)]
\item The initial and final objects of $\Ccal$ agree (that is, $\Ccal$ is pointed).
\item Every cofiber sequence in $\Ccal$ is also a fiber sequence.
\item Every map in $\Ccal$ of the form $f: X \rightarrow \Sigma(Y)$ is the cofiber of its fiber.
\item Filtered colimits and finite limits commute in $\Ccal$.
\end{enumerate}

We denote by $\Groth_\infty$ the full subcategory of $\Pr^L$ on the Grothendieck prestable categories.
\end{definition}

\begin{remark}\label{remark groth prestable vs t structure}
Let $\Ccal$ be a Grothendieck prestable category. Then the functor $\Ccal \rightarrow \Ccal \otimes \Sp = \Sp(\Ccal)$ is fully faithful, and identifies $\Ccal$ with the connective half of t-structure on $\Sp(\Ccal)$. In particular, $\Ccal$ is additive, and moreover it makes sense to consider for each nonnegative integer $n$ the homology functor $H_n = \Omega^n: \Ccal \rightarrow \Ccal^\heartsuit = \Ccal_{\leq 0}$. Note that $\Ccal^\heartsuit$ is a Grothendieck abelian category.
\end{remark}

It turns out that the assignment $\Ccal \mapsto \Sp(\Ccal)$ provides a one to one correspondence between Grothendieck prestable categories and presentable stable categories equipped with a with right complete t-structure compatible with filtered colimits. One virtue of working with Grothendieck prestable categories instead of t-structures is that being Grothendieck prestable is a property of a category (as opposed to a t-structure on a presentable stable category which is a piece of structure).

\begin{example}
Let $\Ccal$ be a Grothendieck abelian category. Then the derived category $\der(\Ccal)$ is a presentable stable category with a right complete t-structure compatible with filtered colimits. By virtue of remark \ref{remark groth prestable vs t structure} the connective half of this t-structure is Grothendieck prestable. We denote this category by $\der(\Ccal)_{\geq 0}$.
\end{example}

The following result provides an ample source of Grothendieck prestable categories:

\begin{proposition}[\cite{SAG} proposition 10.4.3.1]
Let $\Ccal, \Dcal$ be presentable categories and assume given a functor $G: \Ccal \rightarrow \Dcal$ which is conservative and preserves small limits\footnote{Only preservation of finite limits is necessary.} and colimits. If $\Ccal$ is a Grothendieck prestable category then so is $\Dcal$.
\end{proposition}

\begin{corollary}
Let $\Acal$ be a Grothendieck prestable category equipped with a monoidal structure compatible with colimits. Let $A$ be an algebra in $\Acal$. Then the category $\LMod_A(\Acal)$ of left $A$-modules in $\Acal$ is a Grothendieck prestable category.
\end{corollary}

The notion of projective object in a Grothendieck abelian category admits a version in the setting of Grothendieck prestable categories:

\begin{definition}
Let $\Ccal$ be a Grothendieck prestable category. We say that an object $P$ in $\Ccal$ is projective if every map $X \rightarrow P$ in $\Ccal$ which is an epimorphism on $H_0$ admits a section.
\end{definition}

\begin{remark}
Let $\Ccal$ be a Grothendieck prestable category. Then an object $P$ in $\Ccal$ is projective if and only if $\Hom_\Ccal(P, - ): \Ccal \rightarrow \Spc$ preserves geometric realizations. In other words, if and only if $P$ is projective in the sense of \cite{HTT} section 5.5.8.
\end{remark}

We may think about Grothendieck prestable categories generated under colimits by compact projective objects as many object versions of connective ring spectra. In that setting there is a close relation between projective modules over a connective ring spectrum $R$ and projective modules over $\pi_0(R)$ (see \cite{HA} corollary 7.2.2.19). The following proposition is an extension of that relation:

\begin{proposition}\label{proposition projectives vs heart}
Let $\Ccal$ be a Grothendieck prestable category generated under colimits by compact projective objects. Then
\begin{enumerate}[\normalfont (1)]
\item The truncation functor $H_0 : \ccal \rightarrow \ccal^\heartsuit$ sends projective objects to projective objects and compact objects to compact objects.
\item The $0$-truncations of the compact projective objects of $\ccal$ provide a family of compact projective generators for $\ccal^\heartsuit$.
\item The functor $\Ho(H_0): \Ho(\ccal) \rightarrow \Ho(\ccal^\heartsuit) = \ccal^\heartsuit$ induced at the level of homotopy categories restricts to an equivalence between the full subcategories on the projective objects, which in turn restricts to an equivalence on the full subcategories on the compact projective objects.
\end{enumerate}
\end{proposition}
\begin{proof}
We first prove (1). The fact that $H_0$ sends compact objects to compact objects follows directly from the fact that the inclusion $\ccal^\heartsuit \rightarrow \ccal$ preserves filtered colimits. The fact that $H_0$ sends projective objects to projective objects follows from the fact that the inclusion $\ccal^\heartsuit \rightarrow \ccal$ maps epimorphisms to morphisms which induce epimorphisms on $H_0$. 

Item (2) follows directly from (1) together with the fact that $H_0$ is a localization. It  remains to establish (3). We first prove fully faithfulness. Let $X, Y$ be a pair of projective objects of $\ccal$. Then the map $\Hom_{\ccal}(X, Y) \rightarrow \Hom_{\ccal^\heartsuit}(H_0(X), H_0(Y))$ induced by $H_0$ is equivalent to the map $\eta_*: \Hom_{\ccal}(X, Y) \rightarrow \Hom_{\ccal}(X, H_0(Y))$ of composition with the unit $\eta: Y \rightarrow H_0(Y)$. The fact that $X$ is projective and $\eta$ induces an equivalence on $H_0$ implies that $\eta_*$ is an effective epimorphism. Its fiber is given by $\Hom_{\ccal}(X, \tau_{\geq 1}(Y))$ which is connected since $X$ is projective. We conclude that $\eta_*$ induces an equivalence on $\pi_0$, and therefore $\Ho(H_0)$ is fully faithful on the full subcategory on the projective objects.

It remains to prove surjectivity. In other words, we have to show that every (compact) projective object of $\ccal^\heartsuit$ is the image under $H_0$ of a (compact) projective object of $\ccal$. We establish the case of compact projective objects, the proof in the projective case being similar. Let $Y$ be a compact projective object of $\ccal^\heartsuit$.  Applying (2) we may find a compact projective object $X$ in $\ccal$ such that $Y$ is a retract of $H_0(X)$. Let $r: H_0(X) \rightarrow H_0(X)$ be the induced retraction. The fully faithfulness part of (3) allows us to lift $r$ to an idempotent endomorphism $\rho$ of the image of $X$ inside $\Ho(\ccal)$. Let $X'$ be a representative in $\ccal$ of the image of $\rho$. Then $X'$ is a direct summand of $X$ and therefore it is compact projective. The proof finishes by observing that $H_0(X') = \operatorname{Im}(r) = Y$. 
\end{proof}

There is a good theory of tensor products of Grothendieck prestable categories:

\begin{theorem}[\cite{SAG} theorem C.4.2.1]\label{teo tensor product prestable}
Let $\Ccal, \Dcal$ be Grothendieck prestable categories. Then their tensor product $\Ccal \otimes \Dcal$ (formed in $\Pr^L$) is Grothendieck prestable. In particular, the symmetric monoidal structure on the category $\Mod_{\Sp^\cn}(\Pr^L)$ of presentable additive categories and colimit preserving functors restricts to a symmetric monoidal structure on $\Groth_\infty$.
\end{theorem}

In particular, it makes sense to consider commutative algebras in $\Groth_\infty$. We call these symmetric monoidal Grothendieck prestable categories. Note that this terminology leaves implicit the fact that the tensor operation commutes with colimits in each variable.

\begin{definition}
Let $\Mcal$ be a commutative algebra in $\Groth_\infty$. An $\Mcal$-linear Grothendieck prestable category is an object of $\Mod_\Mcal(\Groth_\infty)$.
\end{definition}

In other words, an $\Mcal$-linear Grothendieck prestable category is an $\Mcal$-linear presentable category (in the sense of section \ref{subsection linear cats}) which is in addition a Grothendieck prestable category. In the case when $\Mcal = \Mod^\cn_R$ is the category of connective modules over a connective $E_\infty$-ring $R$ we call these $R$-linear Grothendieck prestable categories.

Since the class of colimits that we have available in $\Groth_\infty$ is relatively restricted, one has to be careful when forming relative tensor products. There is however a class of commutative algebras that admit a well behaved theory of relative tensor products.

\begin{definition}
Let $\Mcal$ be a commutative algebra in $\Groth_\infty$. Assume that $\Mcal$ is generated under colimits by compact projective objects. We say that $\Mcal$ is rigid if compact projective and dualizable objects of $\Mcal$ coincide.
\end{definition}

\begin{example}
Let $R$ be a connective commutative ring spectrum. Then $\Mod_R^\cn$ is rigid.
\end{example}

We fix for the remainder of this section a symmetric monoidal Grothendieck prestable category $\Mcal$ generated under colimits by compact projective objects and rigid. We begin with the observation that $\Mcal$-linear Grothendieck prestable categories are closed under tensor products:

\begin{proposition}
The full subcategory of $\Mod_\Mcal(\Pr^L)$ on the $\Mcal$-linear Grothendieck prestable categories is closed under tensor products. In other words, $\Mod_\Mcal(\Groth_\infty)$ admits a symmetric monoidal structure that makes the inclusion $\Mod_\Mcal(\Groth_\infty) \rightarrow \Mod_\Mcal(\Pr^L)$ symmetric monoidal.
\end{proposition}
\begin{proof}
Completely analogous to the proof of corollary \ref{coro tensor products over A}. One first shows that for any $\Mcal$-module $\Ccal$ in $\Pr^L$ the action map  $\Mcal \otimes \Ccal \rightarrow \Ccal$ admits a colimit preserving right adjoint, imitating the proof of proposition \ref{proposition action has colimit preserving adjoint}. The role of proposition \ref{proposition colimits of right adjointable} is then played by \cite{SAG} remark C.3.5.4.
\end{proof}

We now formulate a Grothendieck prestable version of proposition \ref{prop lex localization}.

\begin{definition}\label{definition generator in groth prestable}
Let $\ccal$ be an $\Mcal$-linear Grothendieck prestable category. We say that an object $G$ in $\ccal$ is an $\Mcal$-generator if for every object $Y$ in $\ccal$ there exists an object $X$ in $\Mcal$ and a morphism $X \otimes G \rightarrow Y$ inducing an epimorphism on $H_0$.
\end{definition}

In the context of Grothendieck prestable categories some attention needs to be paid to the distinction between generators and colimit generators. Unlike the situation with Grothendieck abelian categories, it is possible for an object $G$ in $\ccal$ to be an $\Mcal$-generator in the sense of definition \ref{definition generator in groth prestable} and the family of objects $X \otimes G$ not generate $\ccal$ under colimits (for instance, $0$ is always an $\Mcal$-generator whenever $\ccal$ is stable). As shown in \cite{SAG} theorem 2.1.6, the distinction disappears when $\ccal$ is assumed to be separated:

\begin{definition}
Let $\Ccal$ be a Grothendieck prestable category. We say that $\Ccal$ is separated if it contains no nonzero $\infty$-connective objects.\footnote{An object of $\Ccal$ is $\infty$-connective if all its homology objects vanish.}
\end{definition}

\begin{proposition}\label{prop lex localization prestable}
Let $\Ccal$ be a separated $\Mcal$-linear Grothendieck prestable category and let $G$ be an $\Mcal$-generator for $\Ccal$. Let $A$ be the opposite of the algebra of endomorphisms of $G$ associated to the action of $\Mcal$ on $\Ccal$. Then the functor
\[
G \otimes_A - : \LMod_A(\Mcal) \rightarrow \Ccal 
\]
is an $\Mcal$-linear left exact localization. Furthermore, it is an equivalence if and only if $G$ is compact projective.
\end{proposition}
\begin{proof}
Completely analogous to the proof of proposition \ref{prop lex localization}, where the role of the classical many object Gabriel-Popescu theorem  is played by \cite{SAG} theorem C.2.1.6.
\end{proof}

We now turn to a discussion of flatness in the context of $\Mcal$-linear Grothendieck prestable categories.

\begin{definition}
Let $\Ccal$ be an $\Mcal$-linear Grothendieck prestable category. We say that an object $X$ in $\Ccal$ is flat over $\mcal$ if the functor $- \otimes X : \Mcal \rightarrow \Ccal$ is left exact. In cases when the base $\mcal$ is clear from the context we simply say that $X$ is flat.
\end{definition}

\begin{example}
Let $R$ be a connective $E_\infty$-ring and let $\Ccal$ be an $R$-linear Grothendieck prestable category. It follows from \cite{SAG} proposition C.3.2.1 that an object $X$ in $\Ccal$ is flat if and only if $M \otimes X$ is $0$-truncated for all $0$-truncated $R$-modules $M$. Since every $0$-truncated $R$-module is a filtered colimit of cyclic $\pi_0(R)$-modules, we see that $X$ is flat if and only if $M \otimes X$ is $0$-truncated for every cyclic $\pi_0(R)$-module $M$.
\end{example}

\begin{proposition}\label{prop lazard prestable}
Let $X$ be an object of $\Mcal$. Then $X$ is flat if and only if it is a filtered colimit of compact projective objects.
\end{proposition}
\begin{proof}
Analogous to the proof of proposition \ref{prop lazard classico}.
\end{proof}

\begin{proposition}\label{prop tensoring left exact prestable}
Let $f: \Ccal \rightarrow \Ccal'$ and $g: \Dcal \rightarrow \Dcal'$ be  morphisms in $\Mod_{\Mcal}(\Groth_\infty)$. If $f$ and $g$ are left exact then $f \otimes_{\Mcal} g : \Ccal \otimes_\Mcal \Dcal \rightarrow \Ccal' \otimes_\Mcal \Dcal'$ is left exact.
\end{proposition}
\begin{proof}
Analogous to the proof of proposition \ref{prop tensoring left exact}.
\end{proof}
\begin{corollary}\label{coro tensor flats prestable}
Let $\Ccal, \Dcal$ be $\Mcal$-linear Grothendieck prestable categories, and let $X, Y$ be flat objects of $\Ccal$ and $\Dcal$ respectively. Then the object $X \otimes Y$ in $\Ccal \otimes_\Mcal \Dcal$ is flat.
\end{corollary}
\begin{proof}
Analogous to the proof of corollary \ref{coro tensor flats}.
\end{proof}

In the presence of flatness, projectivity of objects may be checked after passing to $H_0$:

\begin{proposition}\label{prop check compact projective on heart}
Let $\ccal$ be an $\Mcal$-linear Grothendieck prestable category. Assume that $\ccal$ is generated under colimits by compact projective objects and that compact projective objects in $\ccal$ are flat. Then an object $X$ in $\ccal$ is projective if and only if it is flat and $H_0(X)$ is projective in $\ccal^\heartsuit$.
\end{proposition}
\begin{proof}
Since every projective object is a retract of a direct sum of compact projective objects, and flat objects are closed under retracts and direct sums, we see that every projective object of $\ccal$ is flat. The fact that $H_0$ sends projective objects to projective objects was already observed in proposition \ref{proposition projectives vs heart}. This finishes the proof of the only if direction.

 Assume now that $X$ is flat and $H_0(X)$ is projective. Applying proposition \ref{proposition projectives vs heart} we may find a projective object $X'$ in $\ccal$ and an isomorphism $H_0(X') = H_0(X)$. The fact that $X'$ is projective allows us to lift this isomorphism to a morphism $f: X' \rightarrow X$.   We claim that $f$ is an isomorphism. To do so it suffices to prove that $f \otimes 1_{\mcal}$ is an isomorphism. Since both $X$ and $X'$ are flat and $\ccal$ is separated we may reduce to proving that $H_0(f \otimes H_n(1_{\mcal}))$ is an isomorphism for all $n \geq 0$. This agrees with $H_0(H_0(f) \otimes H_n(1_\mcal))$, which is an isomorphism by virtue of the fact that $H_0(f)$ is an isomorphism.
\end{proof}

We now discuss the operation of passage to derived categories for linear Grothendieck abelian categories. Fix for the remainder of this section a symmetric monoidal Grothendieck abelian category $\acal$, rigid and generated by compact projective objects.

\begin{construction}\label{construction smon structure on der 1}
Let $\Acal^\cp$ be the full subcategory of $\Acal$ on the compact projective objects and equip $\acal^\cp$ with the symmetric monoidal structure restricted from $\acal$. Note that   $\der(\Acal)_{\geq 0}$ is  obtained by freely adjoining colimits to $\Acal^\cp$. We equip $\der(\Acal)_{\geq 0}$  with the unique colimit preserving extension of the existing symmetric monoidal structure on $\acal^{cp}$. Since $\der(\acal)$ is the stabilization of $\der(\acal)$, we may further extend our symmetric monoidal structure uniquely to a symmetric monoidal structure compatible with colimits on $\der(\acal)$.
\end{construction}

\begin{remark}
Construction \ref{construction smon structure on der 1} makes $\der(\Acal)_{\geq 0}$ into a rigid commutative algebra in $\Groth_\infty$. It is in fact the unique way to equip  $\der(\Acal)_{\geq 0}$ with a rigid commutative algebra structure making the truncation functor $\der(\Acal)_{\geq 0} \rightarrow \acal$ symmetric monoidal.
\end{remark}

A variant of construction \ref{construction smon structure on der 1} allows us to give the connective derived category of an $\acal$-linear Grothendieck abelian category the structure of a  $\der(\acal)_{\geq 0}$-linear Grothendieck prestable category:

\begin{construction}\label{construction smon structure on der 2}
Let $\ccal$ be an $\acal$-linear Grothendieck abelian category. We view the $\acal$-linear structure as a monoidal finite coproduct preserving functor $f: \acal^\cp \rightarrow \Funct^L_{\lex}(\ccal, \Ccal)$ where the target is the category of left exact colimit preserving endofunctors of $\ccal$. Passing to derived functors provides a monoidal equivalence 
\[
\der(-): \Funct^L_{\lex}(\ccal, \Ccal) \rightarrow \Funct^L_{\lex}(\der(\Ccal)_{\geq 0}, \der(\Ccal)_{\geq 0}).
\]
 We equip $\der(\ccal)_{\geq 0}$ with the $\der(\acal)_{\geq 0}$-linear structure arising from the  monoidal finite coproduct preserving functor $\der(-) \circ f : \acal^\cp \rightarrow \Funct^L_{\lex}(\der(\Ccal)_{\geq 0}, \der(\Ccal)_{\geq 0})$. We equip $\der(\ccal) = \Sp(\der(\ccal)_{\geq 0})$ with the induced $\der(\acal) = \Sp(\der(\acal)_{\geq 0})$-linear structure.
\end{construction}

\begin{remark}
Let $\Ccal$ be an $\acal$-linear Grothendieck abelian category. Then the $\der(\acal)_{\geq 0}$-linear structure on $\der(\ccal)_{\geq 0}$ from construction \ref{construction smon structure on der 2} is the unique such structure making the truncation functor $\der(\ccal)_{\geq 0} \rightarrow \ccal$ into a $\der(\ccal)_{\geq 0}$-linear functor.
\end{remark}

\begin{notation}
The inclusion $\acal \rightarrow \der(\acal)$ is generally not symmetric monoidal. When we wish to emphasize the distinction between both symmetric monoidal structures we will write $\otimes^L$ for the tensor product in $\der(\acal) $, and $\otimes$ for the tensor product in $\acal$. If it is clear from the context which operation is being used, we will simply write $\otimes$ instead of $\otimes^L$. The same considerations apply to the case of the action of $\der(\acal)$ on the   derived category of an $\acal$-linear Grothendieck abelian category.
\end{notation}

The following proposition relates the notion of flatness  of objects in $\acal$-linear Grothendieck abelian categories from section \ref{subsection abelian} with the notion studied in this section.

\begin{proposition}\label{prop flat in C vs DC}
Let $\ccal$ be a separated $\der(\acal)_{\geq 0}$-linear Grothendieck prestable category and let $X$ be an object of $\ccal$. The following are equivalent:
\begin{enumerate}[\normalfont(1)]
\item  $X$ is flat over $\der(\acal)_{\geq 0}$.
\item $X$ is $0$-truncated and $H_1(Y \otimes X) = 0$ for all $Y$ in $\acal$.
\item  $X$ is $0$-truncated and flat over $\acal$ (as an object of $\ccal^\heartsuit$).
\end{enumerate}
\end{proposition}
\begin{proof}
We first show that (1) implies (2). Since $X$ is flat and the unit $1_\acal$ is $0$-truncated, we have that $X = 1_\acal \otimes X$ is $0$-truncated. Furthermore, appealing once again to the flatness of $X$ we have that for every $Y$ in $\acal$ the object $Y \otimes X$ is $0$-truncated, and hence $H_1(Y \otimes X) = 0$, as desired. To see that (2) implies (3) we must show that if  $i: Z \rightarrow Z'$ is a monomorphism in $\Acal$ then $H_0(i \otimes X)$ is a monomorphism in $\ccal$. Indeed, the kernel of $H_0(i \otimes X)$ receives an epimorphism from $H_1(\coker(i)\otimes X)$, which vanishes.

It remains to prove that (3) implies (1). Let $Y$ be a $0$-truncated object of $\der(\acal)_{\geq 0}$. Pick a resolution $Y_\bullet$ of $Y$ by compact projective objects. Then $Y \otimes X$ is the realization of $Y_\bullet \otimes X$. Since $Y_\bullet$ is levelwise compact projective we have that $Y_\bullet \otimes X$ is a diagram of $0$-truncated objects. The fact that $X$ is a flat object of $\ccal^\heartsuit$ now implies that $Y_\bullet \otimes X$ is a resolution of $H_0(Y \otimes X)$. Since $\ccal$ is separated we have that $Y \otimes X = H_0(Y \otimes X)$ is $0$-truncated. Since $Y$ was arbitrary we conclude that $X$ is flat, as desired.
\end{proof}

\begin{remark}
Let $\ccal$ be an $\acal$-linear Grothendieck abelian category. For each $n \geq 0$ we denote by $\Tor_n(-,-): \acal \times \ccal \rightarrow \ccal$ the composite functor
\[
\acal \times \ccal = \der(\acal)^\heartsuit \times \der(\ccal)^\heartsuit \hookrightarrow \der(\acal)_{\geq 0} \times \der(\ccal)_{\geq 0} \xrightarrow{-\otimes-} \der(\ccal)_{\geq 0} \xrightarrow{H_n(-)} \ccal.
\]
Specializing proposition \ref{prop flat in C vs DC} we see that an object $X$ in $\der(\ccal)_{\geq 0}$ is flat over $\der(\acal)_{\geq 0}$ if and only if it is $0$-truncated and $\Tor_1(Y, X) = 0$ for all $Y$ in $\acal$.
\end{remark}

%%%%%%%%%%%%%%%%%%%%%%%%%%%%%%%%%%%%%%%%%%%%%%%%%%%%%%%%%%%%%%%%%%%%%%%%
%%%%%%%%%%%%%%%%%%%%%%%%%%%%%%%%%%%%%%%%%%%%%%%%%%%%%%%%%%%%%%%%%%%%%%%%
%%%%%%%%%%%%%%%%%%%%%%%%%%%%%%%%%%%%%%%%%%%%%%%%%%%%%%%%%%%%%%%%%%%%%%%%
%%%%%%%%%%%%%%%%%%%%%%%%%%%%%%%%%%%%%%%%%%%%%%%%%%%%%%%%%%%%%%%%%%%%%%%%
%%%%%%%%%%%%%%%%%%%%%%%%%%%%%%%%%%%%%%%%%%%%%%%%%%%%%%%%%%%%%%%%%%%%%%%%
%%%%%%%%%%%%%%%%%%%%%%%%%%%%%%%%%%%%%%%%%%%%%%%%%%%%%%%%%%%%%%%%%%%%%%%%

\ifx\inmain\undefined
\bibliographystyle{myamsalpha2}
\bibliography{References}
\fi
\end{document}


\section{A unified class of plug-in estimators} \label{sec:uniform}

In this section, we introduce a new class of estimators for $m_0$ (or equivalently $\pi_0$) that mathematically guarantees plug-in FDR control. 
{It is based on sums of suitably transformed $p$-values, allowing us to   recover classical estimators, such as the Storey \eqref{def:m0:Storey} and the PC (slightly modified) \eqref{def:m0:PC:2006} estimators, and also to define new estimators. We first present a general result for single estimators and then show that plug-in FDR control is also preserved for convex combinations of such estimators. }
%\iq{This new class of estimator provides a generic formula to construct estimators based on sums of suitably transformed $p$-values, allowing to recover classical estimators, such as the Storey \eqref{def:m0:Storey} or the PC (slightly modified) \eqref{def:m0:PC:2006} estimator, and also to define new estimators.
%We first present the mathematical guarantees verified by the general class and then present a useful property allowing to combine estimators of the class.\\}
%
% and includes the classical Storey estimator \eqref{def:m0:Storey} as well as a very simple modification of the PC estimator \eqref{def:m0:PC:2006}, which we compare to the ZZD estimator \eqref{def:PCZZD}. 
%Using our general result, we also define and investigate some further estimators.
%\subsection{Definition and plug-in FDR control} \label{ssec:def:plugin:general}
To start, assume that the $p$-values are transformed by certain functions $g \in \mathcal{G}$, with %\footnote{TO DO seb : actually we work with $f(x)=g(x)/\nu(g)$, which is a density on $[0,1]$. Is the interpretation as a likelihood helpful for motivating the estimators below? \iq{see the remark below}} 
\begin{align} \label{eq:def:class:transformations}
	\mathcal{G}&= \{g:[0,1] \rightarrow [0,1] : \text{ $g$ is non-decreasing and $\E g(U) > 0 $,  where $U \sim \unifrv[0,1]$}\}.
\end{align} 
%In the classical setting (see e.g. \ref{label}) the same transformation (and rescaling) is applied to each $p$-value. 
Accordingly we define the class of estimators $\mathcal{F}_0$ as
\begin{align}
	\mathcal{F}_0&= \left\{ \widehat{m}_0 : [0,1]^m \rightarrow  [0,\infty) \vert\: \widehat{m}_0 (p_1, \ldots, p_m) = \frac{1}{{\nu(g)}} \left(1+ \sum_{i=1}^m g(p_i) \right), g \in \mathcal{G}
	\right\}, \label{eq:def:class:estimators:0}
\end{align} 
where $\nu(g)= \E g(U)$ for any $g \in \mathcal{G}$ with $U \sim \unifrv[0,1]$ (for brevity we sometimes omit the $g$ in $\nu$ when there is no ambiguity concerning the function $g$).
The class $\mathcal{F}_0$ contains the classical estimator $\mStorey$ \eqref{def:m0:Storey} by taking $g(u)=\ind{u>\lambda}$ and  $\nu=1-\lambda$. 
It also contains a slightly modified version $\mPCNew$  of the classical estimator $\mPCOrig$ \eqref{def:m0:PC:2006} obtained from taking $g(u)=u$ with $\nu=1/2$, i.e. 
\begin{align}
	\mPCNew &= 2 + 2 \sum_{i=1}^{m} p_i  \label{eq:def:PC:new}.
\end{align}
In Section~\ref{sec:example:homogeneous} we will introduce some additional estimators and discuss $\mPCNew$ in more detail. 
{The rationale behind the definitions of the classes $\mathcal{G}$ and $\mathcal{F}_0$ is two-fold. Requiring that $g$ is non-decreasing ensures that $\widehat{m}_0$ is coordinatewise non-decreasing, allowing  us to apply Theorem \ref{thm:IMC}. 
The quantity $g(p_i)/\nu$ can be interpreted as the (local) contribution of $p_i$ to the estimate of $m_0$. 
If we expect large $p$-values to provide evidence for null hypotheses, then it seems reasonable to require $g$ to be non-decreasing. Rescaling $g(p_i)$ by $\nu= \E g(U)$ is a simple way of ensuring that $\sum_{i=1}^m g(p_i)/\nu$ is conservatively biased in the sense that $\E(\sum_{i=1}^m g(p_i)/\nu)\ge m_0$ in any constellation of null and alternative hypotheses. 
This type of conservativeness may however not be strong enough for plug-in control. 
As our main result -- Proposition \ref{prop:plugin:control:general:g} below -- shows, simply adding $1/\nu$ as a 'safety margin' to the above estimate is enough for ensuring plug-in FDR control.}\\
%\footnote{TO DO seb :here or earlier: more motivation ... all our estimators are rescaled estimators .... under \eqref{superunif} we have conservativeness: $\E  \widehat{m}_0 \ge m_0$. This is not enough however, for establishing plug-in control. Proposition \ref{prop:plugin:control:general:g} shows that adding $\frac{1}{min(\nu_1, \ldots, \nu_m ) }$  is sufficent. ... In order to provide some more intuition for the rescaling approach in estimators $\widehat{m}_0$ in \eqref{def:m0:Bin:general}, it is easily seen that under \eqref{superunif} these estimators are always conservative. }\footnote{with a slight abuse of notation since we identify the mapping with the estimator...}
{In some situations $p$-values under the null may be heterogeneous, i.e. the $p$-values may  have different distributions under the null, so that using an individual transformation for each $p$-value may be helpful. 
%\footnote{TO DO iqraa : try it on weighted $p$-values so we have an example for continuous heterogeneous setting}. 
To this end, we introduce the following  richer and more flexible class of estimators. }
%In some situations, e.g. when multiple discrete tests are performed (see Section \ref{sec:discretestimators}), the distribution of the  $p$-values under the null may be heterogeneous. 
%In that case it may be helpful to use an individual transformation for each $p$-value, which leads to the richer and more flexible class of estimators

%\begin{equation}
%	\begin{split}
%		\mathcal{F}= \left\{ &\widehat{m}_0 : [0,1]^m \rightarrow  [0,\infty) \vert\: \widehat{m}_0 (p_1, \ldots, p_m) = \frac{1}{min(\nu_1, \ldots, \nu_m ) } + \sum_{i=1}^m \frac{g_i(p_i)}{\nu_i}, \\
%	 	& \left. \text{ $g_i \in \mathcal{G}$ and $\nu_{i} = \E[g_{i}(U)]$, with $U \sim \unifrv[0,1]$ for all $i$} \right\}.
%	\end{split}
%	\label{eq:def:class:estimators}
%\end{equation} 

\begin{equation}
	\begin{split}
		\mathcal{F} = \biggl\{ \widehat{m}_0 : [0,1]^m &\rightarrow [0,\infty) \ \bigg\vert \ \widehat{m}_0 (p_1, \ldots, p_m) = \frac{1}{\min(\nu_1, \ldots, \nu_m)} \\
		&+ \sum_{i=1}^m \frac{g_i(p_i)}{\nu_i}, \quad \text{with $g_i \in \mathcal{G}$ and $\nu_{i} = \E[g_{i}(U)]$, $U \sim \unifrv[0,1]$ for all $i$} \biggr\}.
	\end{split}
	\label{eq:def:class:estimators}
\end{equation}

We state our main result on plug-in FDR control for this {more general} class below.
Clearly,  $\mathcal{F}_0 \subset \mathcal{F}$, so that the results stated for $\mathcal{F}$ {also hold} for $\mathcal{F}_0$. 
%\iq{\seb{Thanks Iqraa, for this remark. I thought about it some more and then wrote the above explanation without referring to the likelihood aspect since the connection is not so clear to me  ... We can talk about how/what we want to deal with this next week.}
%\begin{remark}
%	$\frac{g_i(p_i)}{\nu_i}$ defines a density on $[0, 1]$ so that we can interpret the sum part of the estimator as a likelihood.
%	A good choice of transformation $g$ should allow to distinguish potential nulls from the signal, so we expect $\frac{g_i(p_i)}{\nu_i}$ to be closer to one when $p_{i}$ is truly null thus allowing to increase the likelihood.
%	In that sense, $\frac{g_i(p_i)}{\nu_i} $ can be interpreted as a local estimate of \seb{$m_0$}$\widehat{m}_0$  \footnote{ or of $\mathcal{H}_{0}$, (or $\widehat{m}_0/m$?)}.  
%\end{remark}
%}
%This class of estimators is simple but still flexible enough to deal with heterogenous null $p$-values, as we shall see in Section \ref{later}. In classical settings, the same transformation is applied to all $p$-values. We also introduce a more constrained but simpler class of estimators $\mathcal{F}_0 \subset \mathcal{F}$ where the same transformation is applied to all $p$-values:
%It says that any positive, inc.  reasing and bounded transformation of $p$-values can be used to obtain valid plug-in BH procedures for FDR control.
%Our main technical tool is the following upper bound in the convex ordering of a transformed uniform rv by Bernoulli rv's. 
We now present an upper bound in the convex order  for transformed uniform random variables in terms of  Bernoulli random variables, which is the main technical tool  we use for proving  plug-in FDR control for the class $\mathcal{F}$.
%This result provides convex ordering upper bound on the for Bernoulli transformed uniform random variables.
\begin{lemma} \label{lemma:Bernoulli:cx}
	For any $g \in \mathcal{G}$ we have $g(U) \cxorder \Bin (1,\nu)$ and $\nu = \E g(U)$, where $U \sim \unifrv[0,1]$, {and the notation $\cxorder$ denotes the convex ordering (see Definition~\ref{app:def:cxorder})}.
\end{lemma} 

\begin{proof}
	For $U \sim \unifrv[0,1]$ define $X=g(U)$, so that $\E(X)=\nu$.
	Let  $l_X = \inf_{x \in [0, 1]} g(x)$, $u_X= \sup_{x \in [0, 1]} g(x)$ be the lower and upper endpoints of the support of $X$, and define a two-point distribution $Y$ concentrated on $\{l_X,u_X\}$ by $P(Y=l_X)=(u_X-\nu)/(u_X-l_X)$ and $P(Y=u_X)=(\nu-l_X)/(u_X-l_X)$.
	By Lemma~\ref{app:lemma:SS:3-A-24} 
%	Theorem 3.A.24 in \citetalias{shaked2007stochastic} 
	we have $X \cxorder Y$. \\
	Now let $Z \sim \Bin \left(1, \nu \right)$ and denote the distribution function of $Y$ and $Z$ by $F$ and $G$. 
	Clearly, $\E Y = \nu = \E Z$.  
%	\footnote{TO DO iqraa: need to specify that the important fact is the sign change, not the ordering of positive and negative}
	Since $[l_X,u_X] \subset [0,1]$ and both $Y$ and $Z$ are two-point distributions, the function $G-F$ posesses one crossing point on $[0, 1)$.
%	 and the sign sequence of $G-F$ is $+,-$. 
	Indeed, for $t \in [0, l_{X}), \quad F(t) = 0$ while $G(t) = 1 - \nu$ so that $G-F$ is positive, and for $ t \in [u_{X}, 1), F(t) = 1$ while $G(t) = 1 - \nu$ so that $G-F$ is negative. 
	For $t \in [l_{X}, u_{X})$, $G-F$ can be positive or negative depending on $\nu$. 
	Overall, the sign sequence of $G-F$ is $+, -$ so that Lemma~\ref{app:lemma:SS:3-A-44} 
%	Theorem 3.A.44 in \citetalias{shaked2007stochastic} 
	implies that $Y \cxorder Z$ and the claim follows.
\end{proof}

The following proposition is our main result on plug-in FDR control.

\begin{proposition}\label{prop:plugin:control:general:g}
%	Let $g: [0,1] \rightarrow  [0,1]$ be non-decreasing with $g(0)=0$ and $g(1)=1$ and let $\nu=\int_{0}^{1} g(x) dx$. 	
	Assume that $p_1, \ldots, p_m$ are mutually independent and \eqref{superunif} holds. 
	Then \eqref{eq:IMC} holds true for any estimator $\widehat{m}_0 \in \mathcal{F}$, where $\mathcal{F}$ is defined by \eqref{eq:def:class:estimators}. 
	In particular, the BH plug-in procedure \eqref{eq:khat:BH} using $\widehat{m}_0$ controls FDR at level $\alpha$. 
\end{proposition} 

%Since $\mathcal{F}_0 \subset \mathcal{F}$ the proposition immediately implies FDR control for the BH procedure for estimators $\widehat{m}_0 \in \mathcal{F}_0$.

\begin{proof}
	Since $\widehat{m}_0$ is coordinatewise non-decreasing, it is sufficient to verify \eqref{eq:IMC}.  
	For {any} $h \in \nullset$, monotonicity and super-uniformity give us $\widehat{m}_0( p_{0, h})  \gest 1/\nu+S_0$, where $\nu= min_{l \in \nullset \setminus \{ h \}} \nu_{l} $, and $S_0 = \sum_{\ell \in \nullset \setminus \{h\}} g_\ell(U_\ell)/\nu_\ell $ with $(U_\ell)_{\ell \in \nullset}$ i.i.d random variables distributed according to $\unifrv[0,1]$. 
	By Lemma \ref{lemma:Bernoulli:cx} we have $g_\ell(U_\ell) \cxorder \Bin (1,\nu_\ell)$ and Lemma~\ref{app:lemma:SS:3-A-48} gives $\Bin(1,\nu_i)/\nu_i \cxorder \Bin(1,\nu)/\nu$. 
	Since the convex ordering is preserved under convolutions (see Lemma~\ref{app:lemma:SS:3-A-12}) we obtain $ \nu S_0 \cxorder \Bin (m_0-1, \nu)$. 
	Finally, the mapping $x \mapsto \nu/(1+x)$  is convex on $[0, \infty)$ and therefore from the Definition~\ref{app:def:cxorder} of $\cxorder$ we obtain that 
	\begin{align} \label{eq:proofprop3.1}
		\E \left( \frac{1}{\widehat{m}_0( p_{0, h})}\right) &\le \E \left( \frac{1}{\frac{1}{\nu}+S_0}\right) = \E \left( \frac{\nu}{1+\nu S_0}\right) \le \E \left( \frac{\nu}{1+\Bin (m_0-1, \nu)}\right) \le \frac{1}{m_0},
	\end{align}
	where the last bound is a well-known result for the inverse moment of Binomial distributions (see e.g. \ref{lemma:IM:exp:bin} in Appendix) so that \eqref{eq:IMC} is proved. 
	The statement on plug-in FDR control now follows from Theorem \ref{thm:IMC}.
\end{proof}
By taking $g(u)=\ind{u>\lambda}$ we have $\nu S_0 \sim \Bin (m_0-1, \nu)$ and therefore the second inequality from the right  in \eqref{eq:proofprop3.1} can be replaced by an equality. Thus, it may be tempting to conclude that $\mStorey$ is optimal. In the case of a Dirac-Uniform constellation of $p$-values (see \cite{BR2009})  this is indeed true, since $\nu \mStorey \sim 1 + \Bin (m_0-1, \nu)$ and therefore the left inequality in \eqref{eq:proofprop3.1} can also be replaced by an equality. In more general settings however,  other choices of $g$ may be better, as the results in Section~\ref{ssec:numerical:results} show. 



%\iq{The second to last inequality of \eqref{eq:proofprop3.1} could be replaced by an equality by taking $g(u)=\ind{u>\lambda}$ providing the sharpest possible bound. 
%Thus it may be tempting to conclude that Storey estimator must be optimal, however this would only holds true in the Dirac-Uniform setting. 
%The proof suggest that taking $g(u)=\ind{u>\lambda}$ would be the best possible choice to provide the sharpest upper bound on $m_{0}$, however this only holds true in the Dirac-Uniform case. 
%Indeed, as the results in Section~\ref{ssec:numerical:results} suggest, other choices of $g$ may be better in non Dirac-Uniform setting.}
We highlight that introducing a general class of estimators as \eqref{eq:def:class:estimators} with Proposition~\ref{prop:plugin:control:general:g} allows a unified  proof of plug-in FDR control for known estimators like $\mStorey$ and $\mPCNew$ and also for new estimators that we will define  in Section \ref{sec:example:homogeneous}.
Additionally, the classes $\mathcal{F}_0$ and $\mathcal{F}$ possess stability properties that make it possible to combine various  plug-in estimators while maintaining FDR control.
\begin{proposition}\label{prop:convexity}
	Let  $\widehat{m}_1, \widehat{m}_2 \in \mathcal{F}$, where $\mathcal{F}$ is defined by \eqref{eq:def:class:estimators} and let $\lambda \in [0,1]$.   
	Then the BH plug-in procedure \eqref{eq:khat:BH} using $\widehat{m}_0 = \lambda \widehat{m}_1 + (1- \lambda) \widehat{m}_2$ controls FDR at level $\alpha$.
\end{proposition}

\begin{proof} We show that $\widehat{m}_0$ satisfies \eqref{eq:IMC}. Let $\widehat{m}_1, \widehat{m}_2 \in \mathcal{F}$ have the representation	
\begin{align*}
\widehat{m}_1 &= \frac{1}{\nu }+ \sum_{i=1}^m \frac{g_i(p_i)}{\nu_i} \qquad \text{and} \qquad 
\widehat{m}_2 = \frac{1}{\mu }+ \sum_{i=1}^m \frac{h_i(p_i)}{\mu_i},
\intertext{where $\nu=min(\nu_1, \ldots, \nu_m )$ and $\mu=min(\mu_1, \ldots, \mu_m )$ so that}
\widehat{m}_0 &= \frac{\lambda}{\nu }+\frac{1-\lambda}{\mu } + \sum_{i=1}^m \frac{\lambda g_i(p_i)}{\nu_i} + \frac{(1-\lambda)h_i(p_i)}{\mu_i}
%\intertext{and define}
%\kappa_i&= \frac{\lambda \mu_i}{\lambda \mu_i+ (1-\lambda)\mu_i}\\
%f_i &= \kappa_i g_i + (1-\kappa_i)h_i.
\end{align*}	
and define weights $\kappa_i = \frac{\lambda \mu_i}{\lambda \mu_i+ (1-\lambda)\nu_i}$ and transformations $f_i = \kappa_i g_i + (1-\kappa_i)h_i$. Clearly, $\kappa_i \in [0,1]$ and  $f_i \in \mathcal{G}$ and we introduce $\epsilon_i=\E (f_i)=\kappa_i \nu_i + (1-\kappa_i)\mu_i$. From the above definitions we obtain with some straightforward algebra
\begin{align}
\lambda &= \frac{\kappa_i \nu_i}{\epsilon_i} \qquad \text{and} \qquad 1- \lambda = \frac{(1-\kappa_i) \mu_i}{\epsilon_i} \label{eq:kappa:lambda}
\intertext{which yields}
\frac{\lambda}{\nu_i}g_i& + \frac{(1-\lambda)}{\mu_i}h_i = \frac{\kappa_i }{\epsilon_i}g_i + \frac{(1-\kappa_i) }{\epsilon_i} h_i=\frac{f_i}{\epsilon_i}. \label{eq:convex:functions}
\intertext{From \eqref{eq:kappa:lambda} we have}
\frac{\lambda}{\nu }&= \max \left(  \frac{\lambda}{\nu_1}, \ldots, \frac{\lambda}{\nu_m} \right) =  \max \left(  \frac{\kappa_1}{\epsilon_1}, \ldots, \frac{\kappa_m}{\epsilon_m} \right) \qquad \text{and} \notag \\
\frac{1-\lambda}{\mu }&= \max \left(  \frac{1-\lambda}{\mu_1}, \ldots, \frac{1-\lambda}{\mu_m} \right) = \max \left(  \frac{1-\kappa_1}{\epsilon_1}, \ldots, \frac{1-\kappa_m}{\epsilon_m} \right) \notag
\intertext{so that the sub-additivity of the $\max$ function now yields the bound}
\frac{\lambda}{\nu }&+\frac{1-\lambda}{\mu } \ge \frac{1}{\epsilon}, \label{eq:convex:coefficients}
%\intertext{where $\epsilon= \min(\epsilon_1, \ldots, \epsilon_m)$. 
\end{align}
where $\epsilon= \min(\epsilon_1, \ldots, \epsilon_m)$. Combining \eqref{eq:convex:functions} and \eqref{eq:convex:coefficients} now gives us
\begin{align*}
\widehat{m}_0 &= \frac{\lambda}{\nu }+\frac{1-\lambda}{\mu } + \sum_{i=1}^m \frac{\lambda g_i(p_i)}{\nu_i} + \frac{(1-\lambda)h_i(p_i)}{\mu_i} \ge 
\frac{1}{\epsilon} + \sum_{i=1}^m \frac{f_i(p_i)}{\epsilon_i} := \widetilde{m}_0
\end{align*}
with $\widetilde{m}_0 \in  \mathcal{F}$ and from Proposition \ref{prop:plugin:control:general:g} we know that \eqref{eq:IMC} holds true for $\widetilde{m}_0$ and therefore also for $\widehat{m}_0$.
\end{proof}
The proof shows that $\mathcal{F}$ is ``almost'' convex in the sense that whenever equality holds in \eqref{eq:convex:coefficients}  we have $\widehat{m}_0= \widetilde{m}_0 \in \mathcal{F}$. If $\widehat{m}_1, \widehat{m}_2 \in \mathcal{F}_0$, i.e. each estimator uses only a single transformation function then it is easy to see that equality holds in \eqref{eq:convex:coefficients} which leads to the following result:

\begin{proposition}\label{prop:convexity:0}%\footnote{\iq{Shouldn't we state the result for a finite number of estimators $m_{1}, \dots, m_{l}$ since convexity allows for convex combination}}
	The class of estimators $ \mathcal{F}_0$ given by \eqref{eq:def:class:estimators:0} is convex. 
	In particular this implies that for any $\widehat{m}_1, \widehat{m}_2 \in \mathcal{F}_0$  and $\lambda \in [0,1]$ the BH plug-in procedure  \eqref{eq:khat:BH} controls FDR at level $\alpha$ for the estimator  $\widehat{m}_0 = \lambda \widehat{m}_1 + (1- \lambda) \widehat{m}_2$.
\end{proposition}

%\begin{proof}
%For $j \in \{1,2\}$ let $\widehat{m}_j \in \mathcal{F}$ with $g_j$ (and associated $\nu_j$) satisfying (T). For $0 \le \lambda \le  1$ define $\widehat{m}_0 = \lambda \widehat{m}_1 + (1- \lambda) \widehat{m}_2$, $\kappa = \frac{\lambda \nu_2}{\lambda \nu_2 + (1-\lambda) \nu_1}$ anf $g_0 = \kappa g_1 + (1 - \kappa) g_2$.
%
%Clearly, $\kappa \in [0,1]$ and it is easily seen that $\lambda$ can be expressed $\lambda = \frac{\kappa \nu_1}{\kappa \nu_1 + (1-\kappa) \nu_2}$ and $g_0$ satisfies (T) with $\nu_0=\E g_0(U)= \kappa \nu_1 + (1-\kappa) \nu_2$. Thus we have
%\begin{align*}
%\widehat{m}_0 &= \lambda \widehat{m}_1 + (1- \lambda) \widehat{m}_2= \frac{1}{\kappa \nu_1 + (1-\kappa) \nu_2} \cdot \left[ \kappa \nu_1 \widehat{m}_1 + (1-\kappa) \nu_2 \widehat{m}_2\right]\\
%&= \frac{1}{\kappa \nu_1 + (1-\kappa) \nu_2} \cdot \left[ \kappa \left(1+ \sum_{i=1}^{m} g_1(p_i)\right) + (1-\kappa) \left(1+ \sum_{i=1}^{m} g_2(p_i)\right) \right]\\
%&=\frac{1}{\nu_0} \cdot \left[ 1+ \sum_{i=1}^{m} g_0(p_i) \right] 
%\end{align*}
%and so $\widehat{m}_0 \in \mathcal{F}$.
%\end{proof}


%Proposition \ref{prop:convexity} combined with Proposition \ref{prop:plugin:control:general:g} immediately  implies that weighted estimators from $\mathcal{F}$ provide valid FDR control.
%
%\begin{corollary}\label{coro:weighted:estimators}
%If we take any estimators $\widehat{m}_1, \ldots, \widehat{m}_K \in \mathcal{F}$ and any weights $\lambda_1, \dots, \lambda_K \in [0,1]$ with $\sum_{k=1}^{K} \lambda_k=1$, the BH plug-in procedure \eqref{eq:khat:BH} using the weighted estimator $\widehat{m}_0 = \sum_{k=1}^{K} \lambda_k \widehat{m}_k $ controls FDR at level $\alpha$.	
%\end{corollary} 

%Discussion (to do):
%
%\begin{itemize}
%	\item Heesen + Janssen: Data dependent convex combinations, weighted estimators, inspection points, a practical guide for weighting Storey's estimators $g(u)=...$
%	\item Liang + Nettleton
%	\item Sarkar (2008)? 
%\end{itemize}


\section{Homogeneous estimators} \label{sec:example:homogeneous}
In this section we focus on the class  {of homogeneous estimators} $\mathcal{F}_0$  given by \eqref{eq:def:class:estimators:0}, i.e. on estimators of the form
\begin{align*}
	\widehat{m}_0  &= \widehat{m}_0  (p_1, \ldots, p_m)= \frac{1}{\nu} \left(1+ \sum_{i=1}^m g(p_i) \right),
\end{align*}
where $g \in \mathcal{G}$ and $\nu= \nu(g)= \E g(U)  > 0$,  with $U \sim \unifrv[0,1]$. 
As mentioned before, this class includes the classical estimator $\mStorey$ \eqref{def:m0:Storey} and the new estimator $\mPCNew$  \eqref{eq:def:PC:new}, and also gives the scientist freedom to define new estimators with proven plug-in FDR control thanks to  Proposition \ref{prop:plugin:control:general:g}
%\iq{A practical transformation $g$ could be a weighting function that weighs down small $p$-values and weighs up large $p$-values.
%{The transformation $g$ can be interpreted as a weighting function that weighs down small $p$-values and weighs up large $p$-values.}
}
%Proposition \ref{prop:plugin:control:general:g} gives the data analyst great freedom for defining new estimators that come with guarantees for plug-in FDR control. 
%The function $g$ can be interpreted as a weighting function that weighs down small $p$-values and weighs up large $p$-values. 
There are many conceivable ways in which this can be done. 
As an ad hoc example, we define a polynomial estimator of degree $r \ge 0$ and thresholding parameter $\lambda \in [0, 1)$ by taking $\widehat{m}_0$ as above in \eqref{eq:def:class:estimators:0} with $g(u)= g_{r, \lambda}(u) = u^{r} \cdot \ind{u>\lambda}$, so that $\nu= \frac{1 - \lambda^{r+1}}{r+1}$. 
This gives us 
	\begin{align} \label{eq:def:m0poly}
		\mPoly = \frac{r+1}{1 - \lambda^{r+1}}  + \frac{r+1}{1 - \lambda^{r+1}}  \sum_{i =1}^{ m} p_{i}^{r} \cdot \ind{p_{i}>\lambda}.
	\end{align}
It is easily seen that the classical estimators $\mStorey$ and $\mPCNew$ are particular instances of $\mPoly$ with $r=0$ for $\mStorey$ and $r=1$ and $\lambda = 0$ for $\mPCNew$.
Taking $r=1$ and $\lambda > 0$ yields a hybrid estimator which combines $\mStorey$ and $\mPCNew$ which has the potential to combine the strengths of both methods. For all estimators $\mPoly$ plug-in FDR control follows immediately from Proposition~\ref{prop:plugin:control:general:g}. 
%\footnote{should we present these estimators in equations to be able to refer to them later ? \seb{may be a good idea, i think it depens in wether we need it ... i propose to wait for now. Depeneding on figure 2 we should decide which ones to introduce formally}}
%\begin{enumerate}
%	\item[(iii)] $g(u)=u \cdot \ind{u>\lambda}$ with $\nu=(1-\lambda^2)/2$ yields a hybrid combination of $\mStorey$ and $\mPCNew$ which may combine the strengths of $\mStorey$ and $\mPCNew$. 
%	This idea can be extended to formally define a polynomial estimator 
%	Since we always set $\lambda=1/2$ throughout the paper, we only consider the degree $r$ as a parameter 
%	\item[(iv)] $g(u)=u^2$ with $\nu=1/3$ yields an estimator based on quadratic transformation of $p$-values. 
%	This example could be broadened into a class of polynomial estimators that we investigate in Section~\ref{} \footnote{TO DO later : define polynomial estimator to be more general. Dedicate a section to it ? \seb{Could make sense, depending on how much results we have on this.}}
%\end{enumerate}
%The rationale for estimators (iii) and (iv) is ad-hoc and heuristic. 
%\sout{Figure~\ref{fig:g_functions} illustrates the function $u \mapsto g_{r, \lambda}(u)$ for various choices of $r$.} 
For illustrational purposes we effectively only use $r$ as a parameter and set the thresholding parameter to  the classical value of $\lambda=1/2$ throughout the paper. 
These examples are primarily meant to illustrate the freedom and flexibility Proposition~\ref{prop:plugin:control:general:g} {allows for the class $\mathcal{F}_0$ and should not be interpreted as {recommendations for} optimal choices. These examples are investigated further in the following section.
	 
%\begin{center}
%	% Figure environment removed
%\end{center}


\subsection{Numerical results} \label{ssec:numerical:results}
%To guide the choice of the estimator, we provide in this section a comparison of the performance of the different estimators introduced above. 
%We first propose to 
%In this section we analyze numerically the performances of the estimators introduced above on simulated and real data. 
%In this section we investigate numerically the properties of in the setting of one-sided gaussian testing, i.e. we assume that $X_1, \ldots, X_m \sim N(\mu,1)$ ... \footnote{TO DO iqraa :more details needed}

%\subsubsection{Simulation results} \label{sssec:simu:bias:variance:mse}


Here we compare the performance of several estimators from the class $\mathcal{F}_0$  in a Gaussian one-sided testing setting. We assume that we observe $X_{1}, \dots, X_{m}$  independent random variables with $X_{i} \sim \mathcal{N}(0, 1)$  for $i \in \nullset$ and $X_{i} \sim \mathcal{N}(\mu, 1)$  for $i \in \altset$, with $\mu > 0$, and we test $H_{0,i}: \mu = 0$ vs. $H_{1,i}: \mu > 0$. For a given signal strength $\mu >0$ under the alternative, closed-form expressions for the expectation and variance of $\widehat{m}_0$ are available, see Appendix~\ref{appendix:sec:one:sided:gaussian:testing} for more details. Thus we can numerically compare the mean squared error (MSE) of estimators from  $\mathcal{F}_0$.
%To compare the performance of estimators from class $\mathcal{F}$, we compute the analytical mean squared error (MSE). 
%This is possible because the Gaussian setting is simple enough to provide a closed form formula for the expectation and variance of transformed $p$-values under the alternative, see Appendix~\ref{appendix:sec:one:sided:gaussian:testing} for more details.}


For this analysis, we fix $m=10 \: 000$ and first compare the MSE w.r.t. to the signal strength $\mu$ of the alternatives with a fixed proportion of true nulls $\pi_{0} = 0.6$. 
Then, we compare the MSE w.r.t. the proportion of true nulls $\pi_{0}$ with a fixed signal strength $\mu=1.5$. 
The considered estimators for this comparison are $\mStorey$ \eqref{def:m0:Storey}, $\mPCNew$ \eqref{eq:def:PC:new}, $\mPol(1, 1/2)$, and $\mPol(2, 1/2)$ (see \eqref{eq:def:m0poly} for both).
The MSE is evaluated in terms of $\pi_{0}$ for  better readability and displayed in Figure~\ref{fig:gaussian:mse}. 
The qualitative comparison between the estimators remains consistent across both panels of Figure~\ref{fig:gaussian:mse}: $\mPCNew$ has the poorest performance, characterized by the largest MSE, followed by $\mStorey$.
While the polynomial approach shows some benefits, the improvement is not particularly remarkable except for small to moderate values of $\pi_{0}$. 
For larger values of $\pi_{0}$ or $\mu$, there are no noticeable differences in performance.
%The qualitative comparison between the estimators is consistent between both panels of Figure~\ref{fig:gaussian:mse}:  $\mPCNew$ has the worst performance with the largest MSE, followed by $\mStorey$. 
%Although, the polynomial approach seems to be beneficial, the improvement is not striking besides for small to moderate values of $\pi_{0}$, and for large values of $\pi_{0}$ or $\mu$ there actually no more noticeable differences between the performance.
%Figure~\ref{fig:gaussian:mse} corroborates the estimation results presented in Figure~\ref{fig:gaussian:estim} : we can see that the hybrid estimator combining quadratic transformation of $p$-values and thresholding has the smallest MSE while $\mPCNew$ has the largest one. 


\begin{center}
	% Figure environment removed
\end{center}


%\begin{itemize}
%	\item consistency between both plots 
%	\item PC is the worst, followed by Storey
%	\item polynomial approach seems to be beneficial but no striking improvement 
%	\item for $\pi_{0}$ large or strong signal no more difference 
%\end{itemize}




%\begin{center}
%	% Figure environment removed
%\end{center}






%\footnote{TO DO iqraa : restructure the section : remove the subsection about mse just present simu, plots and then mse plot }

%\iq{We first analyze the estimator on simulated data in the one-sided Gaussian testing setting.
%In this setting we assume that we observe data $X_{1}, \dots X_{m}$ as i.i.d realizations of $\mathcal{N}(\mu, 1)$, and we test the null hypothesis $\mathcal{H}_{0, i} : \mu = 0$ vs the alternative $\mathcal{H}_{1, i} : \mu \geq 0$ simultaneously for all $i \in {1, \dots, m}$.
%Following this setting, we simulate $m=10 \: 000$ $p$-values and carry the estimations over $1 \: 000$ Monte-Carlo simulations. 
%The thresholding parameter of Storey type estimators is set to $\lambda = 0.5$, and the nominal control level is set to $\alpha = 0.05$.
%The point wise performance of the estimators and the power of the plug-in BH procedure, for a signal strength of $\mu= 1.5$ under the alternative, are presented in Figure~\ref{fig:gaussian:estim}\footnote{TO DO iqraa : add ticks in the x-axis, and uniformize colors of procedures between plots} (results for other lower and higher signal strength are displayed in the Appendix~ ). 
%For a better readability, we present the results in terms of estimation of $\pi_{0}$.}
%
%\iq{From the point estimation comparison (left panel of Figure~\ref{fig:gaussian:estim}) we can see that that both version of the \cite{PC2006} estimator $\mPCNew$ and $\mZZKB$ are the most over-conservative estimator throughout the whole range of true $\pi_{0} \in [0.1, 0.9]$.
%For small values of true $\pi_{0}$ (say $\pi_{0} < 0.5$), the hybrid estimator combining quadratic transformation of $p$-values and thresholding provides the best performance, followed by the hybrid combining $\mPCNew$  and $\mStorey$ or the raw $\mStorey$. 
%This indicates that the in the uniform setting, thresholding based estimator are better suited to estimate the proportion of true null, and combining this thresholding with a polynomial transformation of the $p$-values further improves the estimation. 
%In section ? we provide a more detailed investigation of the polynomial-thresholding based estimator that provide some insights for the choice of the degree of the polynomial.
%Each point estimation was plug-in the Bh procedure to compare the power of the adaptive BH procedure with these different estimators. 
%Along with the plug-in BH procedures we also run the raw BH and the oracle adaptive BH procedures to allow a better readability of the results: power of adaptive BH procedures close to the power of the raw BH procedure indicates that there is no improvement using adaptivity, while power of adaptive BH procedures close to the power of the oracle adaptive BH procedure indicates that the adaptivity has an impact on the procedure. \footnote{Should we crop the plot to a smaller range of true $\pi_{0}$ because after 0.5 everything is very close to 0 ...}}
%\begin{center}
%	% Figure environment removed
%\end{center}
%






%
%
%
%\begin{center}
%	% Figure environment removed
%\end{center}

%\begin{center}
%	% Figure environment removed
%\end{center}

%Here we present an investigation into the bias, variance and MSE of the estimators presented in Section \ref{ssec:def:plugin:general} in the setting of one-sided gaussian testing. If $X_0 \sim g(p_i)|H_0$ and $X_1 \sim g(p_i)|H_1 \sim f_1$ we obtain ...





%Using the formulas in Appendix \ref{appendix:ssec:one:sided:gaussian:testing} we calculate $\bias, \var ,MSE$ for   $g=...$.  Figure \ref{label} shows that... 

%\subsubsection{Simulation results.} \label{sssec:bias:variance:mse}
%To do:  describe simulation  results for $\widehat{m}_0$'s and power in the settings of \cite{ZZD2011}.



%\subsubsection{Analysis of empirical data} 
%\footnote {To DO iqraa : try the estimators on this real dataset and discuss if keep it here or in the appendix. Also make a plot with varying lambda }
%\footnote{TO DO iqraa : look for other data sets maybe the one used in PC}

%Maybe use the Hedenfalk data.

%\footnote{To Do later : maybe make a section about optimal polynomial estimator}

\subsection{More details on the Pounds and Cheng estimator} \label{ssec:ComparePCNew:PCZZD}


While FDR control for $\mStorey$ is a classical result following from Theorem \ref{thm:IMC} (\cite{BR2009,benjamini2006adaptive}), much less is known about the validity of  $\mPCOrig$ as a plug-in estimator. 
{Indeed, \cite{PC2006} introduced their estimator \eqref{def:m0:PC:2006} primarily to obtain  a robust estimate of FDR.} 
To the best of our knowlegde, the only  related result on plug-in FDR control was obtained by \cite{ZZD2011}, who defined the following  modified version of $\mPCOrig$:
\begin{align}
	\mZZKB& = C(m) \cdot \min \left[ m, \max \left(s(m), 2 \cdot \sum_{i =1}^m p_i \right) \right], \label{def:PCZZD}
\end{align}
where the correction factors $C(m)$ and $s(m)$ are chosen in such a way so that \eqref{eq:IMC} holds. 
{However, determining} these factors is non-trivial and requires extensive use of numerical integration and approximations methods (see Supplement B in \cite{ZZD2011} for further details) so that no simple representation of $C(m)$ and $s(m)$ is available (for selected values of $m$, Table S1 in \cite{ZZD2011} lists values for the correction factors). 

By contrast, our new modification \eqref{eq:def:PC:new}  is extremely simple and, as we show in Section \ref{sec:discretestimators}, can be adapted easily to e.g. discrete tests, thus confirming a conjecture in \cite{PC2006}.
%By contrast, the modified estimator $\mPCNew$ belonging to our class of estimators is extremely simple and can be adapted easily to e.g. discrete tests (confirming an unproven claim by \cite{PC2006}).
Its validity for plug-in FDR control follows directly from Proposition \ref{prop:plugin:control:general:g} and involves no sophisticated asymptotic or numerical approximations. 
Supplementary material in Appendix \ref{appendix:ssec:ComparePCNew:PCZZD} shows that the two versions of the PC estimator behave more or less identically. 
Nevertheless, we argue in favor of using $\mPCNew$ since it is both conceptually and computationally much simpler than $\mZZKB$.


%Suppose we are interested in how likely the new estimator $\mPCNew$ is more conservative than $\mZZKB$.  
%\begin{align*}
%	\mZZKB& = C(m) \cdot \min \left[ m, \max \left(s(m), 2 \cdot \sum_{i =1}^m p_i \right) \right]
%\end{align*}


% As a particular example, we follow \cite{ZZD2011} who chose $m=500$ in their simulation study.  From their Table S1 we can read off the  values $C(500)=1.011709$ and $s(500)=98$ so that here \eqref{def:PCZZD} becomes
%	\begin{align}
%		\mZZKB & = 1.011709 \cdot \min [ 500, \max (98, 2 \sum_{i =1}^m p_i)]   \label{eq:def:zeisel:estimtor:500}
%	\end{align}
%	and  we can conclude that  $\mPCNew> \mZZKB$ iff $2 \sum_{i =1}^{500} p_i \in (C(500)\cdot 98 -2, 2/(C(500)-1) \cup (500\cdot C(500)-2,1000)$. 
%	
%	
%	We now present a rough asymptotic approximation for the probability that $\mPCNew$ is more conservative than $\mZZKB$. In typical applications, most hypotheses are nulls. If we additionally assume that the signals are strong, we have $2 \sum_{i =1}^m p_i \approx 2 \sum_{i \in \nullset} p_i =:S$ 
%	and this distribution can be approximated  by the CLT so that we obtain for a given $m_0$ 
%	\begin{align*}
%		\P (\mPCNew> \mZZKB) &= \P(S>500\cdot C(500)-2) \approx \overline{\Phi} \left(\sqrt{\frac{3}{m_0}}\cdot (500\cdot C(500)-2-m_0)\right)
%	\end{align*}
%	Figure \ref{fig:comparisonzzd}  illustrates that in the range $m_0=450, \ldots, 500$ these probabilities are quite small and even in the worst case scenario $m_0=500$ the probability is less than $1/3$. Simulation findings (see Section \ref{ssec:one:sided:gaussian:testing}) show that the two versions of the PC estimator behave more or less identically, however we argue for using $\mPCNew$ since it is both conceptually and computationally much simpler than $\mZZKB$.

%\seb{remove this comment?}In our modification of the original PC estimator we have concentrated on deriving a simple and elegant result + easy implementation!. Truncating the estimator, as \cite{zeisel2011fdr} do, makes thing more complicated. 			However, it is worth noting that if we could bound the inverse moment of a truncated Erlang-rv we coiuld still use our approach, since the mapping $x \mapsto \frac{1}{\min (1+x,a)}$ is convex for any $a>0$... 

%\begin{lemma}\label{lemma:basic}
%	For $\lambda \in [0,1]$ let $F_\lambda \sim \unifrv \left[ 1 - \frac{1-\lambda}{1+\lambda}, 1 + \frac{1-\lambda}{1+\lambda}\right]$ (with the convention $F_1 \sim \delta_{\{1\}}$). Then we have
%	\begin{align}
%	F_\lambda & \cxorder F_0 \cxorder \expov(1),
%	\end{align}
%	where $\cxorder$ denotes the usual convex order (see \cite{shaked2007stochastic}).
%\end{lemma}

%\begin{proof}
%For the  bound on the l.h.s.  we show that the family $(F_\lambda)_\lambda$ is monotone w.r.t $\cxorder$ in the sense that $F_\lambda  \cxorder F_{\lambda'}$ for $\lambda' \le \lambda$. For $\lambda' \le \lambda$ the function $F_{\lambda'}-F_{\lambda}$  has exactly one sign change (at $x=1$) and the sign sequence is $+, -$\footnote{present graph?}. Since the two distributions have the same mean, \cite[Theorem 3.A.44]{shaked2007stochastic} implies that 		$F_\lambda  \cxorder F_{\lambda'}$. 
%
%For the bound on the r.h.s. we observe that $F_0 \sim \unifrv [0,2]$. Since $\expov(1)$ has decreasing density on $[0, \infty)$ with mean 1, the claim follows from Theorem 3.A.46 b) in \cite{shaked2007stochastic}.
%\end{proof}



%\newpage

%\section{Estimators for heterogeneous $p$-values / From uniform to discrete plug-in estimators / New (Adapting ?) estimators  for discrete $p$-values} \label{sec:discretestimators}
\section{Adjusted estimators for discrete $p$-values} \label{sec:discretestimators}

In this section we assume -- additionally to mutual independence of the $p$-values -- that the null distribution functions $F_1, \ldots, F_m$ are known. 
As a particular application we consider the setting of discrete $p$-values (see Section~\ref{ssec:background} for more detailed references).
%which has attracted some research interest in recent years \footnote{TO DO iqraa : add ref}. 
%\seb{\sout{For simplicity of exposition we assume throughout that  $\widehat{m}_0 \in \mathcal{F}_0$ even though the results in this section also hold for  $\widehat{m}_0 \in \mathcal{F}$.} I removed this, because we dont need $\widehat{m}_0 \in \mathcal{F}_0$ in Section 5.3}
The classical plug-in estimators, like the \cite{Storey2002} estimator defined in \eqref{def:m0:Storey}, were developed for uniformly distributed $p$-values under the nulls, 
and can thus suffer of an inflated bias when computed under \eqref{superunif} assumption. 

{To illustrate the problem, we compare the bias of an arbitrary estimator $\widehat{m}_0 \in \mathcal{F}_0$ under the uniform setting with the bias under the super-uniform setting. In the classical uniform case,} 
%As a motivating point, we formulate  the bias for any estimator $\widehat{m}_0 \in \mathcal{F}_0$ and compare it between the uniform and the super-uniform setting. 
%For any estimator $\widehat{m}_0 \in \mathcal{F}_0$, under the classical uniform setting, 
{ considering marginally independent $p$-values {$p_{i} \sim X_0$ for $i\in \nullset$, and $p_{i} \sim X_1$ for $i\in \altset$}, for some variables $X_{0}, X_{1}$ defined on $[0, 1]$}, the bias is seen to be
\begin{align}
 	\bias [\widehat{m}_0 ] &= \E[\widehat{m}_0] - m_{0} = \frac{1}{\nu} (1 + m_{1} \E [g(X_{1})]).  \label{eq:uniformbias}
	\intertext{In contrast, under the super-uniform setting, still considering {independent} $p$-values under the null and the alternative, the bias is }
	\bias [\widehat{m}_0 ] &= \frac{1}{\nu} (1 +  m_{1} \E [g(X_{1})]) +  \frac{1}{\nu}  m_{0} ( \E [g(X_{0})] - \nu).\label{eq:superuniformbias}
\end{align}

%\begin{equation} 
% 	 
%\end{equation}
 Recall that $\nu = \E [g(U)]$ with $ U \sim \unifrv [0, 1]$, thus under super-uniformity {$\E [g(X_{0})] \geq \nu$ (see the characterization of the usual stochastic order in Appendix \ref{appendix:auxres})}, which shows that an additional source of conservativeness is {present} in this case. In general, practitioners use classical estimators without worrying about $p$-values distributions, ingenuously expecting the estimator to perform according to the ``uniform''  bias \eqref{eq:uniformbias} when in fact it often performs according to the ``super-uniform'' bias \eqref{eq:superuniformbias}.
This motivates the need for a correction in the estimator that will aim at deflating \eqref{eq:superuniformbias}.\\
%\seb{remove the following?} to get closer to \eqref{eq:uniformbias}, which is the simplest way to fight the over conservativeness.\footnote{to do iqraa : reference/discuss previous/known approaches.}
Super-uniformity does not solely appear in the discrete setting, it also occurs e.g. when testing composite nulls,  
however in the discrete setting additional information on the $p$-values c.d.f (as defined by \eqref{equ:Fi}) may be available and leveraged to correct the over-conservativeness. 
In this section, we present two such approaches that incorporate the available knowledge of $F_{i}$ -- the $p$-value c.d.f under the null -- in the estimators.
The standard way of defining $p$-values for discrete tests leads to distribution functions that satisfy \eqref{discrete} and \eqref{superunif}. 
{As we later introduce transformed $p$-values with transformed distribution, for clearer distinguishability the c.d.f associated to these standard discrete $p$-values are denoted  by $\Fdu_1, \ldots, \Fdu_m$ (where the upper-script ``sd'' denotes ``standard discrete'')}.


%\iq{TO DO 
%\begin{itemize}
%	\item add ref to talk about dealing with discreteness 
%	\item
%	\item
%\end{itemize}
%}

\subsection{{Transformations of discrete $p$-values}} \label{ssec:SuperUniformity:Discrete}
%Correcting means to make it behave in the expected way by the user, indeed the practitioner uses classical estimator without worrying about the distribution of the $p$-values. 
%So under super-uniformity the practitioner might expect the estimator to perform according to the ``uniform''  bias \eqref{eq:uniformbias} while in fact it performs according to \eqref{eq:superuniformbias}.
%The correction aims at deflating \eqref{eq:superuniformbias} to be the closest to \eqref{eq:uniformbias} which is the simplest way to fight the over conservativeness.
%\begin{itemize}
%	\item make it feels like the only quality criterion is the bias 
%	\item 
%\end{itemize}
%\seb{T.b.d. I think we need no more detailed  definitions of discrete df's, therefore i have removed them.}
%\iq{hence we start by formally defining these latter.
%\begin{definition} \label{def:discrete:cdf}
%	A function $F$ is said to be to be a discrete c.d.f of random variable valued in $[0, 1]$ if it is right continuous, and piecewise constant with \iq{jumping ?} points given by a set $\mathcal{A}_{i}$ called the support. 
%	It means that $\mathcal{A}_{i}$ is defined by points $0 = x_{i, 0} < x_{i, 1} < \ldots < x_{i, K_{i}} < x_{i, K_{i} + 1} =1$ such that $ F_{i}(x_{i, j}) = x_{i, j} \mbox{ for all } 1 \le j \le K_{i} + 1 $.
%\end{definition}
%}
%
%\seb{\begin{definition} \label{def:discrete:cdf:a}
%		A distribution function $F$ is said to be the c.d.f of a \emph{discrete-uniform} random variable  if
%		\begin{itemize}
%			\item it is piecewise constant,
%			\item the discontinuity points of $F$ are given by a finite (or countable?) set $\{0,1\} \subset \mathcal{A} \subset [0,1]$ called the \emph{support} of $F$
%			\item $F(x)=x$ for all $x \in \mathcal{A}$.
%		\end{itemize}
%\end{definition}}\footnote{to do: make this consistent with (Discrete) in Section 2?}
%
In order to reduce the individual conservatism of $p$-values caused by super-uniformity, various transformation of discrete $p$-values have been proposed, see e.g. \cite{habiger2015multiple}. 
Perhaps the most popular transformation is the so-called \emph{mid-$p$-value}. 
For the realization $x$ of the random variable $X$, let $p(x)$ be the (realized) standard $p$-value. 
Now define the \emph{mid-$p$-value} \citep{RubindelanchyHeardLawson2017} $q(x)$ given the observation $x$ as
\begin{align} \label{eq:def:mip}
	q(x) &= p(x) - \frac{1}{2} P_0 (p(X)=p(x)),
\end{align}
where $P_0$ denotes the distribution of $X$ under the null (for simplicity, we assume that such a unique distribution exists).
%After this transformation, the distribution function of the mid-$p$-value $q$, is no longer super-uniform but ``shrunk'' towards zero (for an example see Figure \ref{label} \footnote{TO DO iqraa : do a figure comparing p and mi p values}). \footnote{To do Iqraa: more on expectation}
{Transforming the $p$-value through \eqref{eq:def:mip} helps to mimic the behavior of a uniform random variable in expectation. Indeed, we always have $\mathbb{E}[q(X)] = 1/2$, see \cite{berry1995mid} for more details. 
However, the distribution of the mid-$p$-value is no longer super-uniform but shrunk toward 0 as displayed in Figure~\ref{fig:midpvsp}.}
In what follows, we denote by $\Fmid_1, \ldots, \Fmid_m$ the distribution functions of the mid-$p$-values associated with the distribution functions $\Fdu_1, \ldots, \Fdu_m$ of the standard $p$-values.  
In Section \ref{ssec:adjusting:rescaling} we show how the distribution functions of standard discrete or mid-$p$-values can be used in $m_0$-estimators introduced in Section \ref{sec:uniform}, while preserving plug-in FDR control. 

Another transformation to reduce the conservativeness of discrete $p$-values uses so-called \emph{randomized $p$-values} which are defined in our context by
\begin{align}\label{eq:def:randomp}
	r(x,u) &= p(x) - u \cdot P_0 (p(X)=p(x)),
\end{align}
where $u$ is the realization of a uniform random variable $U \sim \unifrv[0,1]$, independent of $X$. 
Alternatively to the notation $r(x,u)$, we will also use (with a slight abuse) $r(p,u)$, where $p=p(x)$ is the standard $p$-value obtained from observation $x$.
Randomized $p$-values and mid-$p$ values are related via the conditional expectation on the observations $q(x)=\E_U [r(x,U) | X = x]$.
{Randomization leads to an (unconditional) uniform behavior, however at the cost of introducing an additional source of randomness which makes its use controversial for decisions on individual hypotheses, see e.g. \cite{habiger2011randomised} for a discussion. }
We show in Section \ref{ssec:randomization:approach} that for estimation purposes however, {randomized $p$-values can be beneficial for obtaining an efficient non-randomized estimator.}
\begin{center}
	% Figure environment removed
\end{center}



%when combined in a generic way with estimators from . 

%Throughout this whole section, we assume that the set of c.d.f $\mathcal{F} = \{F_{i}, 1\leq i \leq m\}$, associated to the $p$-values $p_{1}, \dots, p_{m}$, is \emph{known} and verifies \eqref{superunif} for each $1\leq i \leq m\ $. 
%Note that these are classical assumptions made in the discrete multiple testing setting \iq{ref ?}.
%In the discrete setting, each $F_{i} \in \mathcal{F}$ can be seen as a discretized version of a uniform random variable on $[0, 1]$, where the whole probability mass between two points $(x_\ell,x_{\ell+1}]$ is placed at $x_{\ell+1}$, see Figure~ for an example of  $F_{i}$ obtained from Fisher exact tests \footnote{present a graph?} 
%\iq{This is the standard way in which (conservative) $p$-values are constructed from discrete test statistics \cite{bibid}.\footnote{Finite support ... no accumulation points. Problem for poisson test? We only need this assumption in the proof of rescaled PC estimator.}}
%
%
%\begin{definition}[generalized inverse distribution function]
%	Let $F: \R \rightarrow [0,1]$ be a distribution function. Define the generalized inverse as
%	\begin{align*}
%	F^{-1} (y) &= \inf \{x \in \R: F(x) \ge y \} .
%	\end{align*}  
%\end{definition}

\subsection{Adjusting the rescaling constants}\label{ssec:adjusting:rescaling}

{The first approach for adjusting estimators to discrete $p$-values is tailored  to estimators from the class $\mathcal{F}_0$ and adjusts the rescaling constant $\nu$ in $\widehat{m}_0$. In fact, this approach is not limited to  the discrete setting, and can also be applied for  arbitrary $p$-value distributions.}
%\sout{We present a slightly more general result for arbitrary heterogeneous (not necessarily discrete) distributions.} 
%\iq{This approach is in fact general and can allow the use of any arbitrary super-uniform\footnote{I think you cannot go with non super-uniform distribution is you want plug-in FDR control, bc we need each $\nu_{i} > 0$ which can be guaranteed only with super-uniformity, see the proof.}, not necessarily discrete, heterogeneous distributions.}

\begin{proposition}\label{prop:plugin:discrete:control:general:g}
	%	Let $g: [0,1] \rightarrow  [0,1]$ be non-decreasing with $g(0)=0$ and $g(1)=1$ and let $\nu=\int_{0}^{1} g(x) dx$. 	
	Assume that $p_1, \ldots, p_m$ are mutually independent and the null distribution functions $F_1, \ldots, F_m$ are known. 
	For any $\widehat{m}_0 \in \mathcal{F}_0$ \eqref{eq:def:class:estimators:0} with 
	\begin{align}
		\widehat{m}_0 (p_1, \ldots, p_m)&= \frac{1}{\nu(g)} \left(1+ \sum_{i=1}^m g(p_i) \right) \label{eq:def:discrete:nonadjusted:estimator}
		\intertext{define the adjusted estimator}
		\widehat{m}^{\text{adj}}_0 (p_1, \ldots, p_m)&= \frac{1}{min(\nu^{\text{adj}}_1, \ldots, \nu^{\text{adj}}_m ) }+ \sum_{i=1}^m \frac{g(p_i)}{\nu^{\text{adj}}_i} \label{eq:def:discrete:adjusted:estimator}
	\end{align}
	where $\nu^{\text{adj}}_i = \E_{p_i \sim F_i} [g(p_i)]$, is the expectation {of the transformed p-value}  taken w.r.t. $F_i$. 
	Then the BH plug-in procedure \eqref{eq:khat:BH} using $\widehat{m}^{\text{adj}}_0$ controls  FDR at level $\alpha$.
\end{proposition} 

%\seb{
%Comments:
%\begin{itemize}
%	\item  The proposition is a general result, not limited to the discrete setting. Can this be useful/interesting?
%	\item The null distribution functions $F_1, \ldots, F_m$ need not be super-uniform! The price we have to pay for this may be smaller rescaling constants...
%\end{itemize}
%}


\begin{proof}

%We show that $\widehat{m}^{\text{a}}_0 \in \mathcal{F}$ and conclude using Proposition~\eqref{prop:plugin:control:general:g}.\\
Without loss of generality we assume that $\nu^{\text{adj}}_i > 0$, otherwise the sum and minimum in  \eqref{eq:def:discrete:adjusted:estimator} is to  be taken over the index set $\{i : \nu^{\text{adj}}_i > 0\}$.
	
For any $i \in \{1, \dots, m\}$ define $g_i : [0,1] \rightarrow [0,1]$ by $g_i(y) = g \circ F^{-1}_i (y)$ for $y \in (0, 1]$, where $F^{-1}_i (y) = \inf \{x \in \R: F_i(x) \ge y \}$ is the generalized inverse of $F_{i}$, and set $g_i(0)=g(0)$.
Since $g \in \mathcal{G}$ and $F^{-1}_i $ are both nondecreasing, so is $g_i$.
For $i \in \mathcal{H}_{0}$, with $U \sim \unifrv[0, 1]$, we have  $p_{i} \sim F^{-1}_i (U)$ {by Proposition 2 in \cite{EmbrechtsHofert2013}}, so that $g_{i}(U) \sim g(p_{i})$, which implies that $ \E [g_{i}(U)] = \E_{p_i \sim F_i} [g(p_i)] = \nu^{\text{adj}}_i $, so that \eqref{eq:def:discrete:adjusted:estimator} %restricted to $\nullset$
 belongs to the class $\mathcal{F}$.\\
Now let $z_{1}, \dots z_{m}$ be independent random variables with $z_{i} \sim \unifrv[0, 1]$ for $i \in \nullset$ and $z_{i} \sim \delta_{0}$ for $ i \in \altset$ (Dirac-Uniform configuration). 
Since $g_{i}(U) \sim g(p_{i})$ for $i \in \nullset$, we have 
\begin{align*}
	\widetilde{m}^{\text{adj}}_0 (z_1, \ldots, z_m)  &= \frac{1}{min(\nu^{\text{adj}}_1, \ldots, \nu^{\text{adj}}_m ) }+ \sum_{i \in \nullset} \frac{g_{i}(z_i)}{\nu^{\text{adj}}_i} \\
									  &\sim \frac{1}{min(\nu^{\text{adj}}_1, \ldots, \nu^{\text{adj}}_m ) }+ \sum_{i \in \nullset} \frac{g(p_i)}{\nu^{\text{adj}}_i}
									  \leq \widehat{m}^{\text{adj}}_0 (p_1, \ldots, p_m) \quad \mbox{(a.s.)}
\end{align*}
Since $\widetilde{m}^{\text{adj}}_0 (z_1, \ldots, z_m) \in \mathcal{F}$, by Proposition~\ref{prop:plugin:control:general:g} \eqref{eq:IMC} holds for $\widetilde{m}^{\text{adj}}_0 (z_1, \ldots, z_m)$, and since $\widetilde{m}^{\text{adj}}_0 (z_1, \ldots, z_m) \leq \widehat{m}^{\text{adj}}_0 (p_1, \ldots, p_m)$ (a.s.), \eqref{eq:IMC} also holds for $\widehat{m}^{\text{adj}}_0 (p_1, \ldots, p_m)$.
\end{proof}
%For any $y \in [0,1]$ define the generalized inverse of $F_i$ as $F^{-1}_i (y) = \inf \{x \in \R: F_i(x) \ge y \}$ (and $F^{-1}_i (0)= - \infty$). 
%Define the function $g_i : [0,1] \rightarrow [0,1]$ by  $g_i(y) = g \circ F^{-1}_i (y)$ for $y \in (0,1]$ and set $g_i(0)=g(0)$.
%Since $g \in \mathcal{G}$, for each $i \in [\![ 1, m]\!]$, $g_i(y)$ is non-decreasing, and for $U \sim \unifrv[0,1]$, $\E [g \circ F^{-1}_i (U)] = \E_{p_i \sim F_i} [g(p_i)]$ since $F^{-1}_i (U) \sim F_{i}$. 
%Finally,  $\E_{p_i \sim F_i} [g(p_i)] \geq  \E [g(U)]$ because $F_{i}$ is super-uniform, so $\nu^{\text{a}}_i = \E_{p_i \sim F_i} [g(p_i)] > 0$ so $g_i \in \mathcal{G}$ and $\widehat{m}^{\text{a}}_0 \in \mathcal{F}$.


% Then $g_i \in \mathcal{G}$.
	
%Let $\widetilde{p}_1, \ldots, \widetilde{p}_m$ be independent rv's taking values in $[0,1]$, so that $p_i \sim F^{-1}_i (\widetilde{p}_i)$. If $i \in \nullset$, Proposition 2 in \cite{EmbrechtsHofert2013} guarantees that taking  $\widetilde{p}_i \sim \unifrv[0,1]$ works. For $i \in \altset$ the distribution of $\widetilde{p}_i$ can be constructed by ...
%\seb{Iqraa: Could you complete the proof?}

%Thus we have 
%\begin{align*}
%\widehat{m}^{\text{a}}_0 (p_1, \ldots, p_m)&= \frac{1}{min(\nu^{\text{a}}_1, \ldots, \nu^{\text{a}}_m ) }+ \sum_{i=1}^m \frac{g(p_i)}{\nu^{\text{a}}_i} \sim 
%\frac{1}{min(\nu^{\text{a}}_1, \ldots, \nu^{\text{a}}_m ) }+ \sum_{i=1}^m \frac{g_i(\widetilde{p}_i)}{\nu^{\text{a}}_i} = \widetilde{m}_0 (\widetilde{p}_1, \ldots, \widetilde{p}_m)
%\end{align*}
%and since the rescaling constants are evaluated unter the nulls we have $\nu^{\text{a}}_i = \E_{p_i \sim F_i} g(p_i)= \E_{\widetilde{p}_i \sim \unifrv[0,1]} g_i(\widetilde{p}_i) = \nu (g_i)$ so that  $\widetilde{m}_0 \in \mathcal{F}$. Since the $(\widetilde{p}_1, \ldots, \widetilde{p}_m)$ are independent and super-uniform under the null the claim now follows from Proposition \ref{prop:plugin:control:general:g}.



%In the case of super-uniform $p$-values, the 'offset term' in $\widehat{m}^{\text{a}}_0$ can be replaced by the simpler but more conservative quantity $1/\nu$.
% In appendix~\eqref{}, we present for completeness, a version of $\widehat{m}^{\text{a}}_0$ where the 'offset term' is replaced by the simpler but more conservative quantity $1/\nu$.
%This version may be interesting from a mathematical viewpoint,  but it provides no practical benefit  since the $\nu^{\text{a}}_i$'s need to be computed in any case.
% While this result, which we have presented for completeness,  may be interesting from a mathematical viewpoint,  it provides no practical benefit,  since the  $\nu^{\text{a}}_i$'s need to be computed in any case. 
%\sout{For practical applications, we therefore prefer to work with $ \widehat{m}^{\text{a}}_0$, defined in \eqref{eq:def:discrete:adjusted:estimator}. }


Following Proposition \ref{prop:plugin:discrete:control:general:g}, we define the discrete-uniform estimator using \eqref{eq:def:discrete:adjusted:estimator} with {standard} discrete $p$-values  $p_1, \ldots, p_m$ and their distribution functions  $\Fdu_1, \ldots, \Fdu_m$
%\footnote{The upperscript \textit{du} indicates that we use the standard discrete distribution function $F_i$ of $p$-values under the null} 

%For the special case of standard  we follow Proposition  by defining for a given $\widehat{m_0} \in \mathcal{F}_0$ with $g \in \mathcal{G}$  the \emph{discrete-uniform}\footnote{to do iqraa : Can we find a better name?} estimator 
\begin{align}
	\mdu (p_1, \ldots, p_m) &= \frac{1}{min(\nudu_1, \ldots, \nudu_m ) }+ \sum_{i=1}^m \frac{g(p_i)}{\nudu_i}, \label{eq:def:m0du}
\end{align}
where $\nudu_i=\E_{p_i \sim \Fdu_{i}} [g(p_i)]$. \begin{corollary}
	Assume that $p_1, \ldots, p_m$ are mutually independent {and  \eqref{superunif} holds} with null distribution functions $\Fdu_1, \ldots, \Fdu_m$ that are known. 
	Then the BH plug-in procedure  \eqref{eq:khat:BH} using $\mdu$ as in  \eqref{eq:def:m0du}controls  FDR at level $\alpha$. 
	Moreover $\mdu \le \widehat{m}_0$ (a.s.), {where $\widehat{m}_0$ is the base non-adjusted estimator \eqref{eq:def:discrete:nonadjusted:estimator}}.
\end{corollary}	
{The last statement of the corollary shows that for standard discrete $p$-values the estimator $\mdu$ is guaranteed to perform better than $\widehat{m_0}$.}
{This follows from the fact that $\nudu_1, \ldots, \nudu_m \ge \nu$ because $g$ is non-decreasing and \eqref{superunif} holds (see Appendix~\ref{appendix:auxres}).}


%and the expectation is taken using the standard discrete-uniform distribution function $\Fdu_i$ under the null. 
For classical estimators, the adjusted rescaling constants {can be computed easily, using}
\begin{itemize}
	\item $\nuduStorey_i=1-\Fdu_i(\lambda)$;
	\item $\nuduPC_i = \sum_{x \in \mathcal{A}_i} x \cdot P (p_i=x)$, where $\mathcal{A}_i$ denotes the support of $\Fdu_i$.
\end{itemize}


Similarly {to \eqref{eq:def:m0du}}, we define a mid $p$-value estimator using \eqref{eq:def:discrete:adjusted:estimator} with mid-$p$-values $q_1, \ldots, q_m$ and their distribution functions $\Fmid_1, \ldots, \Fmid_m$ 
\begin{align}\label{eq:def:rescmidp}
	\mmidp (q_1, \ldots, q_m) &= \frac{1}{min(\numid_1, \ldots, \numid_m ) }+ \sum_{i=1}^m \frac{g(q_i)}{\numid_i}
\end{align}
where $\numid_i=\E_{q_i \sim \Fmid_i} [g(q_i)]$ is the expectation taken under the null using the  mid $p$-value distribution function $\Fmid_i$. 
For $\mStorey$ we have $\numidStorey_i = 1 - \Fmid_i(\lambda) \le 1-\Fdu_i(\lambda)=\nuduStorey_i$ and  $g(q_i)=\ind{q_i > \lambda} \le \ind{p_i > \lambda}=g(p_i)$ so that $\mmidp$ can be smaller or larger than $\mdu$, depending on the specific constellation. 
In the case of the PC estimator {we have $g(x)=x$ and since $\E q_i = 1/2$ for any mid-$p$-value (see \cite{berry1995mid}) we have $\numid_1=\ldots =\numid_m = 1/2$} so that in this case the mid-$p$ estimator has a particularly simple representation. 
{Combining this with the fact that $q_i \le p_i$ $(a.s.)$ gives us the following result.}

\begin{corollary}
	Assume that $p_1, \ldots, p_m$ are mutually independent {and super-uniform under the null (i.e. \eqref{superunif} holds),} and let $q_1, \ldots, q_m$ denote the corresponding mid-$p$-values. 
	Then the mid-$p$ estimator of $\mPCNew$ is given by
	\begin{align*}
		\mPCmidp (q_1, \ldots, q_m) &= 2+ 2\cdot  \sum_{i=1}^m q_i 
	\end{align*}
	and  the BH plug-in procedure  \eqref{eq:khat:BH} using $\mPCmidp (q_1, \ldots, q_m)$ 	controls  FDR at level $\alpha$. 
	Moreover, $\mPCmidp \le \widehat{m_0} $ $(a.s.)$, {where $\widehat{m}_0$ is the base non-adjusted estimator \eqref{eq:def:discrete:nonadjusted:estimator}}.
\end{corollary}

This result implies that for the PC estimator with discrete data we can simply use $2+ 2\cdot  \sum_{i=1}^m q_i$ instead of the more conservative  $2+ 2\cdot  \sum_{i=1}^m p_i$ estimator without losing plug-in FDR control.  
We point out that the mid-$p$-values are used exclusively for estimating $m_0$ in the plug-in procedure defined by \eqref{eq:khat:BH}  while the (ordered) standard discrete $p$-values $p_{(k)}$ are used in the final decision step. 

%\begin{proof} \seb{\sout{For the PC estimator we have $g(x)=x$ and since $\E q_i = 1/2$ for any mid-$p$-value (see \ref{label}) we have $\numid_1=\ldots =\numid_m = 1/2$ and the result follows.}} \end{proof}

\subsection{A randomization approach} \label{ssec:randomization:approach}
Here we describe an approach related to \cite{dickhaus2012analyze} who argue for using randomization methods in estimating $m_0$ on discrete data.
For any estimator $\widehat{m}_0$, not necessarily belonging to $\mathcal{F}_0$ define the associated \emph{expected randomized estimator} as 
\begin{align}
	\mrand (p_1, \ldots, p_m) &=  \left[ \E_{(U_1, \ldots, U_m)} \left(\frac{1}{\widehat{m}_0 (r(p_1,U_1), \ldots, r( p_m,U_m))}\right)\right]^{-1} \label{def:m0:rand}
\end{align}
where $U_1, \ldots, U_m \sim \unifrv[0,1]$ denote i.i.d uniform random variables independent of $ (p_1, \ldots, p_m)$. 
Thus, for fixed $(p_1, \ldots, p_m)$ this estimator is obtained by taking the expectation over the randomized $p$-values associated with $(p_1, \ldots, p_m)$. %\footnote{to do seb : Comments: reproducibility,No unpleasant side effects (additional randomness introduced) due to randomization due to taking the expectation. Not 'full' randomization but 'conditional in expectation' (?), describe similarities and ifferences to Dickhaus ...} 
In most cases \eqref{def:m0:rand} is analytically intractable, we therefore use Monte-Carlo approximation of $\mrand$ obtained by averaging a large number of simulations of 
$\widetilde{m}_0 (r(p_1,U_1), \ldots, r( p_m,U_m))$ (the vector $(U_1, \ldots, U_m)$ is simulated many times, while $(p_1, \ldots, p_m)$ is kept fixed). 
Again, this approach comes with guaranteed FDR plug-in control.

\begin{corollary} \label{coro:randomized:plugin:control}
Assume that $p_1, \ldots, p_m$ are mutually independent {and super-uniform under the null (i.e. \eqref{superunif} holds)} and let $\widehat{m}_0$   {satisfy the conditions of Theorem \ref*{thm:IMC}}. 
	Then the BH plug-in procedure  \eqref{eq:khat:BH} using $\mrand (p_1, \ldots, p_m) $ defined by \eqref{def:m0:rand} controls  FDR at level $\alpha$. {Moreover $\mrand \le \widehat{m}_0$ (a.s.).}
\end{corollary}

%\seb{I have  rephrased the corollary since rand works for any $\widehat{m}_0$ which satisfies Theorem \ref{thm:IMC}.}

\begin{proof}
The proof uses Theorem \ref{thm:IMC}. First, we show that $\mrand (p_1, \ldots, p_m)$ is coordinatewise non-decreasing. 
For fixed $(u_1, \ldots, u_m) \in [0,1]^m$ each (realized) randomized $p$-value $r_i=r(p_i,u_i)$ is non-decreasing in $p_i$. Since $\widehat{m}_0 \in \mathcal{F}$ is coordinatewise non-decreasing in $(p_1, \ldots, p_m)$, the function $1/\widehat{m}_0(r(\cdot,u_1), \ldots, r(\cdot,u_m))$ is coordinatewise decreasing for all $(u_1, \ldots, u_m) \in [0,1]^m$ and so is its expectation which implies that $\mrand$ is coordinatewise non-decreasing.
To establish \eqref{eq:IMC},   we denote for $h \in \nullset$ by $r_{0,h}$ the set of randomized $p$-values $(r_1, \ldots,r_m)$, where $r_h$ has been replaced by $0$. 
By the definition of $\mrand$ we have
\begin{align*}
	\E_{(p_1, \ldots, p_m)} \left[\frac{1}{\mrand(p_{0,h})}\right] &= \E_{(p_1, \ldots, p_m)} \left[ \E_{(U_1, \ldots, U_m)} \frac{1}{\widehat{m}_0(r_{0,h})} \right]=\E_{(r_1, \ldots, r_m)} \left[\frac{1}{\widehat{m}_0(r_{0,h})} \right].
\end{align*}
where the second equality follows from the fact that for super-uniform $p$-value $p_{h} = 0$,  the associated randomized $p$-value $r(p_{h}, u) = 0$ (a.s) by Definition~\eqref{eq:def:randomp}.
Since the $(r_1, \ldots, r_m)$ are mutually independent and uniform under the null and $\widehat{m}_0 \in \mathcal{F}$, the bound \eqref{eq:IMC} for $\widehat{m}_0(r_{0,h})$ now follows %from Proposition \ref{prop:plugin:control:general:g} so that 
since $\widehat{m}_0$ satisfies the conditions of Theorem \ref{thm:IMC}. Therefore, the r.h.s. of the last equation can be bounded by $1/m_0$ and {plug-in FDR control} for $\mrand$ now follows from Theorem \ref{thm:IMC}. 
{To see that the last statement of the corollary holds true, observe that since $\widehat{m}_0$ is coordinatewise non-decreasing and $r(p_i, U_i) \le r(p_i,0)=p_i$ we have $\widehat{m}_0 (r(p_1,U_1), \ldots, r( p_m,U_m)) \le \widehat{m}_0(p_1,\ldots,p_m)$ and therefore the r.h.s. of \eqref{def:m0:rand} is bounded by $\widehat{m}_0(p_1,\ldots,p_m)$ $(a.s.)$.}
\end{proof}

\cite{dickhaus2012analyze} argue for using randomized $p$-values in (essentially) Storey's estimator, i.e. applying $\mStorey$ to $(r_1, \ldots,r_m)$ instead of $(p_1, \ldots, p_m)$ which yields a random estimate that should provide  a better estimate for $m_{0}$. 
They show that plugging this estimator  into the Bonferroni procedure yields asymptotic control of the Familywise Error Rate (FWER) under certain assumptions. 
They also point out that if fully reproducible results are desired it may be more appropriate to work with the conditional expectation w.r.t. randomization, i.e. using $\E_{(U_1, \ldots, U_m)} \left(\mStorey (r_1, \ldots,r_m)\right)$. 
Corollary \ref{coro:randomized:plugin:control} shows that we can obtain similar guarantees w.r.t. to plug-in FDR control in a finite-sample setting for {any estimator  $\widehat{m}_0$ satisfying  the conditions of Theorem \ref*{thm:IMC} and in particular for  $\widehat{m}_0 \in \mathcal{F}_0$ by using conditional expectation w.r.t. randomization.}  
The slightly complicated form of  \eqref{def:m0:rand} is a natural consequence of Theorem \ref{thm:IMC}, but if the variance of $\widehat{m}_0 (r(p_1,U_1), \ldots, r( p_m,U_m))$ w.r.t. $U_1, \ldots, U_m$  is small we have the approximation $\mrand (p_1, \ldots, p_m) \approx \E_{(U_1, \ldots, U_m)} \widehat{m}_0 (r(p_1,U_1), \ldots, r( p_m,U_m))$.
%\begin{align*}
%	\mrand (p_1, \ldots, p_m) &\approx \E_{(U_1, \ldots, U_m)} \widehat{m}_0 (r(p_1,U_1), \ldots, r( p_m,U_m)).
%\end{align*}



\subsection{Simulation results}\label{ssec:simu:discrete}

In this section, we analyze how the discrete adjustments can improve the base estimators on simulated data. 
%More specifically, we simulate $m = 500$ experiments in which the goal is to detect differences between two groups by counting the number of successes/failures in each group.
More specifically, we follow \cite{DDR2018} by simulating a two-sample problem in which a vector of $m= 500$ independent binary responses is observed for $N=25$ subjects in both groups. 
The goal is to test the $m$ null hypotheses $H_{0,i}$: '$p_{1i} = p_{2i}$', $i = 1,...,m$ where $p_{1i}$ and $p_{2i}$ are the success probabilities for the $i^{th}$ binary response in group A and B respectively. 
Thus, for each hypothesis $i$, the data can be summarized by a $2 \times 2$ contingency table, and we use (two-sided) Fisher's exact tests (FETs) for testing $H_{0i}$. 
The  $m$ hypotheses are split in three groups of size 
$m_1$, $m_2$, and $m_3$ such that $m = m_1 + m_2 + m_3$.
Then, the binary responses are generated as i.i.d Bernoulli of probability 0.01 ($\Bin(1,0.01)$) at $m_1$ positions for both groups,
i.i.d $\Bin(1,0.10$) at $m_2$ positions for both groups, 
and i.i.d $\Bin(1,0.10)$ at $m_3$ positions for one group
and i.i.d $\Bin(1,p_3)$ at $m_3$ positions for the other group.
Thus, null hypotheses are true for $m_1 + m_2$ positions, while they are false for $m_3$ positions (set $\cH_1$). 
We interpret $p_3$ as the strength of the signal and set it to 0.4, while $\pi_{1} = \frac{m_3}{m}$, corresponds to the proportion of signal. 
Also, $m_1$ and $m_2$ are both taken equal to $\frac{m - m_3}{2}$.

We first compare  the base estimators  $\mStorey$ \eqref{def:m0:Storey}, $\mPCNew$ \eqref{eq:def:PC:new}, and $\mPol(2, 1/2)$ \eqref{eq:def:m0poly}  with their standard discrete rescaled versions.
 %calculated following $\mdu$ as in  \eqref{eq:def:m0du}. 
 Figure~\ref{fig:FETestim_base_vs_resc} displays the estimation results for a grid of true $\pi_{0} \in \{0.1, \dots, 0.9 \}$.
 We can see that incorporating discreteness leads to considerable improvements for all estimators over the entire range of $\pi_{0}$ values. This is particularly relevant for large values of $\pi_{0}$ where the base estimators may lead to a strong deterioration in the power of the plug-in BH procedure.
 {On another note, among the base estimators we can see that $\mPol(2, 1/2)$ performs poorly compared to the results of Section~\ref{ssec:numerical:results}. 
 This seems plausible since a large portion of $p$-values are equal to 1 in the discrete setting in contrast to the Gaussian setting. 
 For these $p$-values, the contribution in the $\mPol(2, 1/2)$ estimator is equal to the constant $\nu= \frac{3}{1 - 1/2^{3}} = \frac{24}{7} $ which is much larger than the corresponding contribution of $\nu=2$ in the Storey estimator.
% Since the rescaling constant $\nu$ of $\mPol(2, 1/2)$, $\nu= \frac{1 - 1/2^{3}}{3} = \frac{24}{7} $ is much larger than the rescaling constant  of 2 for the Storey estimator, it inflates 
% longer compensated by the sum in \eqref{eq:def:m0poly}
 }  
 %a significant improvement across the entire range of $\pi_{0}$ values, particularly for large values of $\pi_{0}$.
 %This further motivates the use of discrete adjustments as the base estimators tend to be excessively conservative and consistently exceed 1 for larger values of $\pi_{0}$, rendering them useless.
\begin{center}
	% Figure environment removed
\end{center}
In a second step, we compare the different discrete adjustments $\mdu$ as in  \eqref{eq:def:m0du}, $\mmidp$ as in \eqref{eq:def:rescmidp}, and $\mrand$ as in \eqref{def:m0:rand} in Figure~\ref{fig:FETestim_all_disc}, where we display the estimation results for three values of true $\pi_{0} \in \{0.2, 0.5, 0.7 \}$.
We can see that there are no relevant differences between the different adjustments. Therefore, there is no strong reason to advocate a specific type of adjustment since they yield similar outcomes.
\begin{center}
	% Figure environment removed
\end{center}

\subsection{Real data analysis}

Finally, we compare the performance of base and discrete estimators on three different datasets. 
The first dataset consists of data provided by the International Mice Phenotyping Consortium (IMPC)  \citep{karp2017prevalence}, which coordinates studies on the genotype influence on mouse phenotype. 
This dataset includes, for each of the $m=266952$ studied genes, the counts of normal and abnormal 
phenotypes thus providing multiple two by two contingency tables, which can be analysed using FETs. 
Then we analyze the methylation dataset for cytosines of Arabidopsis in Lister et al. (2008) which is part of the R-package \texttt{fdrDiscreteNull} of \cite{chen2015fdrdiscretenull}.
This dataset  contains $m=3525$ counts for a biological entity under two different biological conditions or treatments also analyzed using FETs.
Finally, the third dataset, provided by the Regulatory Agency in the United Kingdom, includes adverse drug reactions due to medicines and healthcare products. It contains the number of reported cases of amnesia as well as the total number of adverse events reported for each of the $m = 2446$ drugs in the database.  For more details we refer to \cite{heller2011false} and to the accompanying R-package \texttt{discreteMTP} of \cite{heller2012discretemtp}, which also contains the data. \cite{heller2011false} investigate the association between reports of amnesia and suspected drugs by performing for each drug (one-sided) FETs.


From the results in Table~\ref{tab:SummaryRandmized} we can see that taking discreteness into account is always beneficial, regardless of the adjustment used.
Depending on the type of discreteness and the amount of signal contained in the data, adjusting for discreteness  can provide a great  improvement in some cases. Indeed, as the example of the IMPC data shows, base estimators may not be able to recognize the presence of any alternatives. However, the discrete estimators clearly suggest that a considerable amount of alternatives is present. 



%Indeed, without discrete adjustments the estimators are unable to reveal the signal. 

%\begin{itemize}
%	\item describe the datasets (see, online paper, and paper with Guillermo)
%	\item taking discreteness into account is always beneficial regardless of the method used
%	\item depending on the type of discreteness and the amount of signal the adjustment can lead to big improvement (see diff between IMPC)
%	\item for IMPC signal get lost if no adjustment, continuous estimators obscure the amount of signal, discrete estimator reveals the signal
%
%\end{itemize}


\begin{table}[!ht]
		\caption{$\pi_{0}$-estimates for base estimators and adjusted discrete estimators on three different datasets containing discrete data.}\label{tab:SummaryRandmized}
	\centering
	\begin{tabular}{llccc}
		\toprule
		~ & ~ & \multicolumn{3}{c}{Dataset} \\ \cmidrule{3-5}
		Adjustment & Estimator & IMPC  & Arabidopsis & Pharmacovigilance \\ \midrule
		standard (none) & Storey                 & 1.26     &     0.67        &     1.79   \\
		& PC                      & 1.26    &     0.73      &       1.79 \\
		& Poly$(2, 1/2)$         & 2.16     &      0.75       &        2.97\\\midrule[0em]
		rescaled (du) & Storey                 & 0.63     &      0.59       &      1.05  \\ 
		& PC                      &  0.63    &      0.64       &       1.04 \\
		& Poly$(2, 1/2)$        &  0.63    &      0.57       &        1.10\\ \midrule[0em]
		rescaled (mid) & Storey                 & 0.63   &       0.63      &       1.03 \\
		& PC                      &  0.63    &      0.64       &        1.05\\
		& Poly$(2, 1/2)$         & 0.63    &       0.58      &        1.11\\ \midrule[0em]
		randomized & Storey                  &  0.63    &     0.58        &       1.08 \\
		& PC                       &   0.63   &        0.64     &        1.06\\
		& Poly$(2, 1/2)$         &    0.63  &       0.56      &       1.14 \\
		\bottomrule
	\end{tabular}
\end{table}



\newpage
%\input{Simulations}
%\input{EmpiricalData}
\section{Conclusion}\label{sec:conclusion}

This paper presents our empirical domain knowledge distillation framework using ChatGPT and discusses our observations from the framework application experiments in the autonomous driving domain. The key finding is that: 1) with proper design of prompt engineering and execution flow, fully automated domain knowledge (in the ontology format) distillation is possible. However, due to the randomness in the response and the butterfly effect, the quality of fully automated distillation results is not guaranteed. To address this, we develop a web-based assistant to enable manual supervision and early intervention at runtime. We hope our findings and tools inspire future research toward revolutionizing the engineering processes of knowledge-based systems across domains.
\appendix
\pagebreak
\section{Auxiliary definitions and results}\label{appendix:auxres}
%Here we present some definitions and results from stochastic ordering.
%Bounds for inverse moments of sume distributions.
%We follow the 
%assume univariate r.v
%Definitions and results from \citetalias{shaked2007stochastic}
In this appendix we recall some definitions and results of stochastic ordering following the presentation in  \citetalias{shaked2007stochastic}, to which we also refer the reader for further details.
We also recall a well-known bound on the inverse moment of the Binomial distribution.
\begin{definition}[Stochastic order]
	Let $X$ and $Y$ be two random variables such that
	\begin{align*}
		\P(X > x) \leq \P(Y > x) \quad \text{for all } x \in (-\infty, \infty),
	\end{align*}
	Then $X$ is said to be smaller than $Y$ in the usual stochastic order denoted by $X \stoorder Y$.
\end{definition}
	
An equivalent characterization of the stochastic order is that  $X \stoorder Y$ $\Leftrightarrow$ $\E[g(X)] \leq \E[g(Y)]$, for  all  non-decreasing functions $g : \R \rightarrow \R$ for which the expectations exist (see (1.A.7) in \citetalias{shaked2007stochastic})).	
%\begin{corollary}\label{app:cor:stoorder}
%	
%\end{corollary}

\begin{definition}[Convex order] \label{app:def:cxorder}
	Let $X$ and $Y$ be two random variables such that 
		\begin{align*}
		\E(\phi( X)) \leq \E(\phi( Y)) \quad \text{for all convex functions } \phi : \R \rightarrow \R,
	\end{align*}
	provided the expectations exist. Then $X$ is said to be smaller than $Y$ in the convex order denoted as $X \cxorder Y$.
\end{definition}

The next results follows from the definition of convex ordering, see Chapter~3 of \citetalias{shaked2007stochastic}.
%The convex ordering has a number of characterization
%We restate here the ones we use in the paper.

\begin{lemma}[Theorem 3.A.24 in \citetalias{shaked2007stochastic}] \label{app:lemma:SS:3-A-24}
	Let $X$ be a random variable with mean $\E X$. 
	Denote the left (right) endpoint of the support of $X$ by $l_X\left[u_X\right]$. 
	Let $Z$ be a random variable such that $\P\left\{Z=l_X\right\} = \left(u_X-\E X\right) /\left(u_X-l_X\right)$ and $\P\left\{Z=u_X\right\}=(\E X-$ $\left.l_X\right) /\left(u_X-l_X\right)$. 
	Then
	$$
		\E X \cxorder X \cxorder Z
	$$
	where $\E X$ denotes a random variable that takes on the value $\E X$ with probability 1 (the left handside just restates Jensen's inequality).	
\end{lemma}


\begin{lemma}[Theorem 3.A.44 in \citetalias{shaked2007stochastic}] \label{app:lemma:SS:3-A-44}
	Let $X$ and $Y$ be two random variables with equal means, density functions $f$ and $g$, distribution functions $F$ and $G$, and survival functions $\bar{F}$ and $\bar{G}$, respectively. 
	Denote by $S^{-}(a)$ the number of sign changes for function $a$. 
	Then $X \leq_{\mathrm{cx}} Y$ if any of the following conditions hold:
	\begin{align*}
		& S^{-}(g-f)=2 \mbox{ and the sign sequence is } +,-,+;\\ 
		& S^{-}(\bar{F}-\bar{G})=1 \mbox{ and the sign sequence is } +,- ;\\ 
		& S^{-}(G-F)=1 \mbox{ and the sign sequence is } +, -.
	\end{align*}
\end{lemma}

\begin{proposition}[Theorem 3.A.12 d) in \citetalias{shaked2007stochastic}] \label{app:lemma:SS:3-A-12}
	Let $X_1, X_2, \ldots, X_m$ be a set of independent random variables and let $Y_1, Y_2, \ldots, Y_m$ be another set of independent random variables. 
	If $X_i \cxorder Y_i$ for $i=1,2, \ldots, m$, then
	$$
		\sum_{j=1}^m X_j \cxorder \sum_{j=1}^m Y_j .
	$$
	That is, the convex order is closed under convolutions.
\end{proposition}


\begin{lemma} [Example 3.A.48 in \citetalias{shaked2007stochastic}] \label{app:lemma:SS:3-A-48}
	Let $X$ and $Y$ be Bernoulli random variables with parameters $p$ and $q$, respectively, with $0<p \leq q \leq 1$. 
	Then
	$$
		\frac{X}{p} \geq_{\mbox{cx}} \frac{Y}{q}.
	$$
%	Let $X_1, X_2, \ldots, X_n$ be nonnegative exchangeable random variables, and let $I_{p_1}, I_{p_2}, \ldots, I_{p_n}$ and $I_{q_1}, I_{q_2}, \ldots, I_{q_n}$ be independent Bernoulli 	random variables that are independent of $X_1, X_2, \ldots, X_n$. If $\boldsymbol{p} \prec \boldsymbol{q}$, then
%	$$
%	 \sum_{i=1}^n I_{q_i} X_i \cxorder \sum_{i=1}^n I_{p_i} X_i 
%	$$
\end{lemma}

\begin{lemma}[Inverse moment for the Binomial distribution]\label{lemma:IM:exp:bin}
Let $B_1, \ldots, B_k \sim \Bin(1,q)$. Then $\E [ 1/(1+\sum_{i=1}^k B_i) ]\le 1/((k+1)q)$.
\end{lemma}

\begin{proof} See e.g. \cite{benjamini2006adaptive}.
\end{proof}	

%\begin{corollary}
%	In the setting  of Proposition \ref{prop:plugin:discrete:control:general:g}  assume additionally that 
% 	$p_1, \ldots, p_m$ are  super-uniform under the null (i.e. \eqref{superunif} holds). Define the estimator
% 	\begin{align*}
%		 \widetilde{m}^{\text{a}}_0 (p_1, \ldots, p_m)&= \frac{1}{\nu }+ \sum_{i=1}^m \frac{g(p_i)}{\nu^{\text{a}}_i}.
%	 \end{align*}
%  	Then the BH plug-in procedure  \eqref{eq:khat:BH} using $\widetilde{m}^{\text{a}}_0$ controls FDR at level $\alpha$
%	
%\end{corollary}
%
%\begin{proof}
%	If the $F_i$ are super-uniform we have (under each null) $U \stoorder p_i$, where $U \sim \unifrv [0,1]$ and since $g \in \mathcal{G}$ is non-decreasing this yields $\nu^{\text{a}}_i = \E g(p_i) \ge \E g(U) = \nu $ for all $i$ and thus $\widetilde{m}^{\text{a}}_0 \ge \widehat{m}^{\text{a}}_0$.
%\end{proof}


\section{Complements to Section \ref{ssec:numerical:results}} \label{appendix:sec:one:sided:gaussian:testing} 

In the context of Gaussian one-sided testing described in Section~\ref{ssec:numerical:results}, let $\widehat{m}_0 (p_1, \ldots, p_m) = \frac{1}{\nu} \left(1+ \sum_{i=1}^m g(p_i) \right)  \in \mathcal{F}_0$. Define $X_0 \sim g(p_i)$ for $i \in \nullset$ and $X_1 \sim g(p_i)$ where $ i \in \altset$. Then we have
\begin{align*}
	\bias (\widehat{m}_0) &= \frac{1}{\nu} \E \left(1+ \sum_{i \in \altset}g(p_i)\right)=(1+(m-m_0)\cdot \E X_1)/\nu,\\
	\var (\widehat{m}_0) &= \frac{1}{\nu^2} \var \left( \sum_{i \in \nullset}g(p_i)+ \sum_{i \in \altset}g(p_i)\right)= (m_0 \cdot \var(X_0)+ (m-m_0)\cdot \var(X_1))/\nu^2, \qquad \text{with}\\
	\var(X_0) &= \int_{0}^1 g(u)^2 du - \left[\int_{0}^1 g(u) du\right]^2,\\
	\var(X_1) &= \int_{0}^1 g(u)^2 f_1(u) du - \left[\int_{0}^1 g(u) f_1(u) du\right]^2.
\end{align*}
where $f_1(t)= \exp \left(- \mu \cdot \Phi^{-1}(t)-\mu^2/2\right)$ denotes the density of the $p$-values under the alternative.



\section{Complements to Section \ref{ssec:ComparePCNew:PCZZD}} \label{appendix:ssec:ComparePCNew:PCZZD}

Here we present some numerical results, comparing the performance of $\mPCNew$ (see \eqref{eq:def:PC:new}) and $\mZZKB$ (see \eqref{def:PCZZD}) for $m=500$, where the correction factors $C(500)=1.011709$ and $s(500)=98$ are taken from Table S1 in \cite{ZZD2011}. 

We first analyze the two estimators on simulated data in a one-sided Gaussian testing setting where we observe realizations of independent rv's $X_{1}, \dots, X_{m_0} \sim N(0,1)$ and $X_{m_0 +1}, \dots, X_{500} \sim N(1.5,1)$ for $1000$ Monte-Carlo simulation runs and a varying range of $m_0 =50,100, \dots, 450$. 
We obtain $500$ $p$-values by testing the null hypotheses $H_{0, i} : \mu = 0$ vs. the alternatives $H_{1, i} : \mu  > 0$ simultaneously for all $i \in \{1, \dots, 500\}$ and calculate $\mPCNew$ and $\mZZKB$ as well as the number of rejections obtained from the plug-in BH procedure in \eqref{eq:khat:BH} with $\alpha=0.05$. 
\begin{center}
	% Figure environment removed
\end{center}
Figure~\ref{fig:PCzzd_vs_our} shows that over a wide range of true $m_0$ values, $\mPCNew$ and $\mZZKB$ yield comparable results both w.r.t. the point estimates and for the number of  rejections. 
In fact, $\mPCNew$ appears to be slightly more efficient than $\mZZKB$.
 
Another comparison can be obtained when we assume that the signal under the alternative is strong and that most hypotheses are nulls. 
In this case  we have $2 \sum_{i =1}^m p_i \approx 2 \sum_{i \in \nullset} p_i =:S$ so that we can use the Central Limit Theorem to quantify the probability that $\mPCNew$ is more conservative than $ \mZZKB$
\begin{align*}
	\P (\mPCNew > \mZZKB) & = \P(S > m \cdot C(m) - 2) \approx \overline{\Phi} \left(\sqrt{\frac{3}{m_0}}\cdot (m\cdot C(m)-(m_0+2))\right).
\end{align*}
Figure \ref{fig:comparisonzzd} shows that this probability, for various values of the true $m_{0}$, is quite small and even under the complete null ($m_{0} = 500$) it is bounded by $1/3$.

% Figure environment removed	
 

%An alternative proof of plug-in control for $\mPCNew$ using the exponential distribution, instead of the Bernoulli distribution, as a bounding device is also possible (see \ref{label}\footnote{to do seb : in the appendix: exponential bound does not require bounded $g$ + the exponential distribution can also be used as a bounding device to give a stremalined proof for ZZD's exponential estimator.}). 
 
 

%\subsection{Additional Figures for numerical calculations}

\section{Additional Figures for simulated data of Section~\ref{sec:example:homogeneous}}
We provide additional results on simulated data in the Gaussian one-sided testing setting described in Section~\ref{ssec:numerical:results}, with $m=10000$ and $\mu=1.5$.
Figure~\ref{fig:gaussian_simu} displays estimation results for  $\pi_{0}$ over $1000$ Monte-Carlo replications.  
They are in line with the analytical comparisons of the MSE provided in Figure~\ref{fig:gaussian:mse}. 
Alongside, we also provide results on power, defined as the ratio of the number of true discoveries to the number of alternatives, for the corresponding plug-in BH (abbreviated in ABH for adaptive BH) procedures using each of the estimators, the raw BH and oracle plug-in BH (using the true $m_{0}$). 
The procedures are run for a fixed level $\alpha = 0.05$.
%The power results show that the raw BH provides the worst performance while the oracle BH provides the best performance as expected.
{The power enhancement among the different plug-in estimators' is not striking except perhaps for very small values of $\pi_{0}$ were where we recover the same performance ranking as in Figure~\ref{fig:gaussian:mse}. 
For larger values of $\pi_{0}$, the differences in power is not perceptible anymore, every procedure behaves poorly as there is less and less signal.}
%\footnote{todo: BH worst, oracle best, plug-in in between, however for $pi>1/2$ everything is bad...}
\begin{center}
	% Figure environment removed
\end{center}

%\subsection{Additional Figures for simulated data os Section~\ref{ssec:simu:discrete}}

\section{Upper and lower bounds for the inverse moment of the uniform sum distribution }

The Pounds and Cheng estimator is closely related to the sum of independent uniform random variables. 	
This distribution plays a role in various contexts and is also known as the \emph{Irwin-Hall} distribution (for more details, see \cite{JohnsonKotz1970}). 
As an auxiliary result, we give lower and upper bounds for the inverse moment of this distribution.
\begin{lemma}[Inverse moments for Erlang distributions]\label{lemma:IM:Erlang}
	Let $E_1, \ldots, E_k \sim \expov(1)$ be independent exponentially distributed random variables. 
	Then
	 $\E [ 1/\sum_{i=1}^k E_i]\le 1/(k-1).$	
\end{lemma}

\begin{proof} Since $X=\sum_{i=1}^k E_i$ is Gamma-distributed with shape  $\alpha=k$ and  inverse scale parameter $\beta=1$  then $1/X$ is Inverse-gamma distributed with mean $\beta/(\alpha-1)$, see \cite{gelman2013bayesian}.
\end{proof}	

\begin{proposition}(Inverse moment for sums of uniforms)\label{prop:IMSumUnif}
	For $k \ge 2$ let $U_1, U_2, \ldots,U_k \sim \unifrv[0,1] $ iid. Then we have
	\begin{align}
	\frac{2}{k} &\le \E \left( \frac{1}{\sum_{i=1}^{k} U_i}\right) \le 	\frac{2}{k-1}
	\end{align}
\end{proposition}

\begin{proof} Let $E_1,E_2, \ldots,E_k \sim \expov(1)$ iid. From Theorems 3.A.24 and 3.A.46 in \citetalias{shaked2007stochastic} we have for $i=1, \ldots, ,k$
	\begin{align*}
	1 & \cxorder 2U_i \cxorder E_i
	\intertext{and since the  convex ordering is preserved under convolutions (see \citetalias[Theorem 3.A.12.]{shaked2007stochastic}) we obtain}
	k & \cxorder \sum_{i=1}^{k} 2U_i \cxorder \sum_{i=1}^{k} E_i. 
	\intertext{Together with the convexity of the mapping $x \mapsto 1/x$ on $(0,1)$ this yields}
	\frac{1}{k} &\le \E \left( \frac{1}{\sum_{i=1}^{k} 2U_i}\right) \le \E \left( \frac{1}{\sum_{i=1}^{k} E_i}\right) \le	\frac{1}{k-1},
		\end{align*}
		where the last inequality follows from  Lemma \ref{lemma:IM:Erlang}.
\end{proof}


%\input{AdditionalMaterial}








\pagebreak
\bibliographystyle{apalike}
\bibliography{biblio}
\end{document}