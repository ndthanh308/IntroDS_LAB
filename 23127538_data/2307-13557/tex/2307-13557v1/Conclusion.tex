\section{Discussion} \label{sec:conclusion}

In this paper we introduce a unified class of $m_0$-estimators with mathematical guarantees on plug-in FDR control in a classical setting. 
We also describe some general approaches for adjusting $m_0$-estimators constructed for continuous $p$-values to discrete $p$-values. 
While we show that these results are useful both from a methodological viewpoint and for practical purposes, there a numerous possibilities for further investigations, some of which we describe now.

While we  focus on FDR control in this paper, it is clear, that our new estimators can also be used for FDR estimation, see \cite{Storey2002}. 
For the discrete estimators from Section \ref{sec:discretestimators} that are uniformly better than their classical counterparts, this implies that the corresponding FDR estimators are uniformly better es well.

In Section \ref{sec:example:homogeneous}, we describe the performance of various polynomial estimators in a one-sided Gaussian testing framework with the aim of illustrating the flexibility of our Proposition~\ref{prop:plugin:control:general:g} on plug-in FDR control.
%in the choice of estimator which is made possible by our general result on plug-in FDR control. 
The numerical results in Section \ref{ssec:numerical:results} show that $\mPol(2,1/2)$ performs uniformly better in terms of MSE than the other estimators {in this specific framework}. 
In a second step, it might be interesting to study whether optimal  estimators can be derived, either within the whole class $\mathcal{F}_0$ or perhaps within some sub-class like polynomial estimators. This way, it may eventually be possible to obtain more efficient estimators in practice, or at least give the user some guidance for choosing estimators from the class $\mathcal{F}_0$. 

We would also like to point out some connections to the work of \cite{heesen2016dynamic}, who split the unit interval into an estimation region on which an estimator of $m_{0}$ is constructed and a rejection region on which the BH procedure is run. 
Thus these estimators do not use all available $p$-values, in contrast to our approach. 
\cite{heesen2016dynamic} derive a general sufficient criterion, similar  to Theorem \ref{thm:IMC}, for finite sample plug-in FDR control which they apply to Storey-type estimators %\sout{and to ``generalized Storey'' estimators that use (in our notation) the transformation function $g(u)= \ind{(\gamma, \lambda]}(u)$ which lead} and to histogram-like estimators (see \cite{Macdonald2019} for a definition). 
and histogram-type estimators  (see \cite{Macdonald2019}). 
In contrast to our approach, the transformation function $g$ applied to $p$-values need not be monotone, however it is unclear whether e.g. smooth functions fit into this framework.  
Their approach also accommodates ``dynamization'', which allows data-dependent tuning of parameters, see \cite{Macdonald2019}. 
{We may wonder if this approach can be used for the estimators of our class, but as this question exceeds the scope of this paper, we leave it for future research.}
%\sout{If this were the case, it might be possible to replace $\ind{(\gamma, \lambda]}(u)$ by some smooth e.g. kernel-based functions, which may also lead to favorable asymptotic performance of the estimators, see \cite{Neuvial2013}. As this question exceeds the scope of this paper, we leave it for future research.}

While we use polynomial estimators as simple examples that include both the classical Storey and Pounds and Cheng estimators, other choices are conceivable. 
Taking, for instance, certain kernel-type transformation functions would lead to estimators that are advantageous from an asymptotic viewpoint, see \cite{Neuvial2013}. 
We leave this topic for future research.

In Section \ref{sec:discretestimators} we  illustrate how information on the null distribution functions of discrete $p$-values can be used to obtain more efficient $m_0$-estimators. 
This information can be seen as a special case of auxiliary covariates, for which it is well-known  that their incorporation into multiple testing procedures e.g. by weighting, can be highly beneficial (see \cite{IHW,Guillermo}). 
We would like to mention that our methods for $m_0$-estimation  are not limited to the special case of discrete $p$-values but should be able to accomodate these types of heterogeneity as well. 
Thus, they might also be useful in the settings described above for obtaining more efficient $m_0$-estimators.

While we assume independence of the $p$-values throughout this paper, it is well-known that dependence may adversely affect the performance of $m_0$-estimators  or may require re-adjustments of tuning parameters like $\lambda$ in $\mStorey$ (see e.g. \cite{BR2009}). 
Thus, it might be interesting to investigate the behaviour of our new estimators under various types of dependency.

Constructing  multiple testing procedures for discrete data that provide finite sample plug-in FDR control is challenging.  
In this paper we make some progress by obtaining improved discrete estimators for $m_0$.
%that can be used in the 'estimation step' of the plug-in BH procedure \eqref{eq:khat:BH}. 
While using these discrete estimators in the plug-in BH procedure provide more power than using classical estimators (based on uniformly distributed $p$-values under the null), it is still not ideal because the discreteness is ignored in the rejection stage of the procedure.
%$\pi_{0}$ values, particularly for large values of $\pi_{0}$. 
\cite{DDR2018} propose discrete variants of the standard (i.e. non plug-in) BH procedure.  
They also sketch a possible plug-in method based on combining this procedure with estimators of $m_0$,  but caution that it comes without mathematical guarantees. 
Thus, as \cite{Macdonald2019} pointed out, it still remains an open problem to develop procedures that integrate discreteness of the data in both the estimation of $m_0$ and the rejection of $p$-values.

%\begin{itemize}
%	\item Optimal (oracle) choice of $g$? Admissibility e.g. in one-side gaussian testing? Quadratic and hybrid seem to yield uniform improvements w.r.t. MSE but also $R$(?) over Storey... Maybe focus on simpler mFDR.
%	\item Dependence?
%	\item Analytic calculations for (m)FDR in the spirit of Etienne and Fanny?
%	\item Structured signals and nulls
%	\item Data-dependent choice of $g$?
%	\item DiscreteFDR may perform better than uniform BH + discrete plug-in estimator. Open question: How to combine (disrcete) plug-in estimators with discrete FDR methods.
%	\item \seb{Is $\mrand$ the most natural approach?}\left( 
%\end{itemize}

\section*{Acknowledgements}
This work is part of project DO 2463/1-1, funded by the Deutsche Forschungsgemeinschaft. 
The authors thank Etienne Roquain for insightful discussions and helpful suggestions which significantly improved the manuscript.