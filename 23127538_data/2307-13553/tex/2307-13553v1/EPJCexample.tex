\documentclass[a4paper,superscriptaddress,amsmath,amssymb,11pt]{article}
\pdfoutput=1 % if your are submitting a pdflatex (i.e. if you have
             % images in pdf, png or jpg format)

\usepackage{EPJCpub} % for details on the use of the package, please
                     % see the JHEP-author-manual

\usepackage[T1]{fontenc} % if needed
\usepackage[utf8]{inputenc}
\usepackage{graphicx}% Include figure files
\usepackage{dcolumn}% Align table columns on decimal point
\usepackage{bm}% bold math
\usepackage{float}
\usepackage{multirow}
\usepackage{verbatim}
\usepackage{amsmath}
\usepackage{graphicx,subfigure,epsfig}
\usepackage{mathrsfs}
\usepackage{hyperref}
\hypersetup{
    colorlinks=true,
    citecolor=blue,
    linkcolor=blue,
    filecolor=magenta,
    urlcolor=black,}%cyan lianjieyanse
\usepackage{color}

\newcommand{\be}{\begin{equation}}
\newcommand{\ee}{\end{equation}}



\title{Modelling the black holes surrounded by a dark matter halo in the galactic center of M87}


%% %simple case: 2 authors, same institution
%% \author{A. Uthor}
%% \author{and A. Nother Author}
%% \affiliation{Institution,\\Address, Country}

% more complex case: 4 authors, 3 institutions, 2 footnotes
\author[a]{Dong Liu,}
\author[b]{Yi Yang,}
%\author[b]{Ali \"Ovg\"un,}
\author[a]{Zhaoyi Xu}
\author[a,1]{and Zheng-Wen Long\note{Corresponding author.}}


% The "\note" macro will give a warning: "Ignoring empty anchor..."
% you can safely ignore it.

\affiliation[a]{College of Physics, Guizhou University, Guiyang, 550025, China}
\affiliation[b]{School of Mathematics and Statistics, Guizhou University of Finance and Economics, Guiyang, 550025, China}
%\affiliation[b]{Physics Department, Eastern Mediterranean University, Famagusta, 99628 North Cyprus via Mersin 10, Turkey}

% e-mail addresses: one for each author, in the same order as the authors
\emailAdd{dongliuvv@yeah.net}%{gs.dongliu19@gzu.edu.cn}dongliuvv@yeah.net
\emailAdd{yangyigz@yeah.net}%{gs.yangyi17@gzu.edu.cn}
%\emailAdd{ali.ovgun@emu.edu.tr}
\emailAdd{zyxu@gzu.edu.cn}
\emailAdd{zwlong@gzu.edu.cn}



\abstract{In this paper, the structure of a dark matter halo can be well described by the mass model of M87 and the Einasto profile for the cold dark matter model, i.e., $\rho_{\text{halo}} (r)=\rho_e \exp ( -2 \alpha ^{-1} ((r/r_e)^\alpha -1 ) )$ \href{https://doi.org/10.1038/s41586-020-2642-9}{[Nature \textbf{585}, 39–42 (2020)]}. Under these conditions, we construct a solution of a static spherically symmetric black hole in a dark matter halo. Then, using the Newman-janis algorithm, we extend this static solution to the case of rotation, and obtain a solution for the Kerr-like black hole. We prove that this solution of the Kerr-like black hole is indeed a solution to the Einstein field equations. Finally, taking M87 as an example, we study and analyze some physical properties of this Kerr-like black hole, and then compare them with the Kerr black hole. These research results for the black hole in a dark matter halo may  indirectly provide an effective method for detecting the existence of dark matter.}

%These physical properties are as follows: 1) The horizons of the Kerr-like black hole; 2) The ergosphere of the Kerr-like black hole; 3) the motion equation of neutral particle near the Kerr-like black hole; 4) The geodesic equations of lightlike particle and photon regions; 5) The shadow of the Kerr-like black hole in a dark matter halo.
\keywords{Dark matter, Black hole, Modified gravity}

\begin{document}
\maketitle
\flushbottom

\section{Introduction}\label{s1}
Black hole is a fascinating and mysterious celestial body predicted by General Relativity. General Relativity predicts that a sufficiently compact mass can warp spacetime, forming the gravitational field. In other words, when the mass of the celestial body is large enough, the celestial body will make the gravitational field so strong that it can prevent light from escaping, and eventually a black hole may be emerged. The famous physicists K. Schwarzschild and R. Kerr respectively solved the Einstein field equations of the vacuum, and then  they obtained the well-known the exact solution of the Schwarzschild black hole \cite{Schwarzschild:1916uq} and Kerr black hole \cite{Kerr:1963ud}. In 2019, the Event Horizon Telescope (EHT) released the first image of a black hole, that is a supermassive black hole at the center of M87 in the Virgo elliptical galaxy \cite{EventHorizonTelescope:2019dse}. The photo of the black hole is composed of two parts, that is the outer circular bright area and the inner dark area. The bright region near the black hole represents the accretion disk, which appears distorted due to gravitational lensing. And the side of the detector away from the accretion disk will be relatively darker, forming a dark region. The images of the black hole shadow can help us to understand the spacetime geometry of the event horizon for the black hole and its speed of rotation. From these Refs. \cite{Jusufi:2019nrn,Okyay:2021nnh,Churilova:2021tgn,Ovgun:2018tua,Bambi:2019tjh,Konoplya:2021slg,Khan:2019gco,He:2020dfo,Kuang:2022xjp}, we can learn more about the study of the black hole shadow. In 2022, the EHT Collaboration released the first photo of Sgr A*, that is the supermassive black hole in the center of the Milky Way \cite{EventHorizonTelescope:2022apq}. Therefore, it is generally accepted for us that black hole exists in our universe.

On the other hand, more and more observational data indicate the existence of dark matter, such as the rotation curves of spiral galaxies, the cosmic microwave background radiation and gravitational lensing, etc. Based on these observational data, astronomers have proposed a series of dark matter models to study dark matter, such as the cold dark matter (CDM) model \cite{Navarro:1995iw,Navarro:1996gj}, the warm dark matter (WDM) model \cite{Dodelson:1993je}, the scalar field dark matter (SFDM) model \cite{Urena-Lopez:2002ptf} and the self-interacting dark matter (SIDM) model \cite{Spergel:1999mh}. In these dark matter models, the density profile of dark matter can always be well described by relatively few variable parameters, such as the density parameter $\rho$ and the characteristic radius $r$. So far, various density profiles of dark matter models have been proposed, and we list some of them as follows. F. Navarro et al. gave the NFW profile for studying the dark matter halo structure in the CDM model \cite{Navarro:1995iw}. Then, T. Matos et al. gave the spacetime geometry of the pure dark matter halo for this NFW profile \cite{Matos:2003nb}. L. Urena-Lopez et al. gave the density profile of the scalar field dark matter model by solving the Einstein K-G equation coupled with the scalar field \cite{Urena-Lopez:2002ptf}. A. Burkert gave the density of the URC profile from the dark matter halos of seven dwarf spiral galaxies \cite{Burkert:1995yz}. The accuracy of the density profile of dark matter may directly determine the difficulty of finding dark matter. Here, we main focus on the density profile proposed by J. Einasto for the CDM model \cite{1965On}. J. Wang et al. showed in their recent research that the NFW profile and the Einasto profile can well describe the structure of dark matter halo in the entire range of the mass \cite{Wang:2019ftp}. More density profiles of dark matter can be found from P. Salucci in Ref.\cite{Salucci:2018hqu}.

Based on these interesting physical backgrounds above, a large number of the individuals are beginning to study the interaction between black holes and dark matter. If there is dark matter near the black hole, due to the strong gravity of the black hole itself, the dark matter will be distributed in a certain structure near the black hole and disappear at a certain distance away from the black hole. Using Newtonian approximations, P. Gondolo et al. first obtained the density profile of dark matter near black hole at the center of the Milky Way \cite{Gondolo:1999ef}. L. Sadeghian et al. used accurate Schwarzschild geometry to perform relativistic calculations, and corrected the density profile of P. Gondolo's dark matter \cite{Sadeghian:2013laa}. Xu et al. combined the NFW profile in the dark matter halo structure and the spike structure, and then they obtained the spacetime geometry of the black hole immsered in dark matter \cite{Xu_2018,Xu:2021dkv,Xu:2020jpv}. K. Jusuf et al. obtained the spacetime geometry of the URC profile dark matter in the center of the M87 galaxy, and studied the impacts of dark matter parameters on the shadow of the black hole \cite{Jusufi:2019nrn}. S. Nampalliwar et al. tested the impacts of dark matter spike on the black hole at the center of the Milky Way \cite{Nampalliwar:2021tyz}. At the same time, a recent article published in Nature once again showed that the NFW profile and the Einasto profile of dark matter can well describe the astronomically predicted the structure of dark matter halo \cite{Wang:2019ftp}. If it is said that there is dark matter near the black hole, then this dark matter must be able to change the spacetime geometry of the black hole. Therefore, it is meaningful to study the impacts of dark matter on the spacetime geometry of the black hole. So, in this work, we will mainly study the impacts of the Einasto profile of dark matter on the black hole. In addition, over the years, the gravitational waves produced by the merger of binary black holes have also been gradually detected by LIGO/Virgo \cite{LIGO1,LIGO2,LIGO3,LIGO4}. The source of gravitational waves is usually due to perturbations in the black hole or additional sources such as dark matter. Different gravitational waves may be generated according to different black holes, and we can use this to judge the type of black hole or the background space-time. V et al. indicated that a black hole in the perturbed state can emit gravitational waves \cite{Vishveshwara:1970zz}. And this gravitational wave is usually dominated by the excited oscillation mode of complex frequency, that is quasinormal mode (QNM) \cite{Moderski:2005hf}. The real part of quasinormal mode represents the oscillation frequency of the black hole when a black hole is perturbed, and the imaginary part of that represents the decay rate, which is also named the damping \cite{Konoplya:2011qq}. Therefore, as a characteristic sound, the QNM from the black hole can provide us a new method to identify the black hole in our universe. The recent studies about quasinormal modes of the black hole in the dark matter can refer to these Refs. \cite{Zhang:2021bdr,Zhang:2022roh,Liu:2021xfb,Liu:2022ygf,Konoplya:2022hbl}. In the future, with the increase of LIGO/Virgo observation data and the photos of the black hole shadow, it is possible to detect such black holes in a dark matter halo, which also provides an effective method for indirect detection of the existence of dark matter.

This paper is organized as follows. In section \ref{s2}, we introduce a much-concerned Einasto profile that can well describe the cold dark matter model, and then we give the metric for this pure dark matter spacetime. In section \ref{s3}, starting from Einstein field equations, we introduce a spherically symmetric black hole into this pure dark matter spacetime, and then we give the metric of the Schwarzschild-like black hole in a dark matter halo. In section \ref{s4}, using the Newman-Janis algorithm, we extend the solution of this Schwarzschild-like black hole to the case of the Kerr-like black hole. Besides, in Appendix \ref{sa}, we prove that this solution of the Kerr-like black hole is a solution of Einstein field equations. In section \ref{s5}, we study and analyze some basic physical properties of this Kerr-like black hole in a dark matter halo, and then we compare these results with the Kerr black hole. Here, we list the following physical properties: 1) The horizons of the Kerr-like black hole; 2) The ergosphere of the Kerr-like black hole; 3) The motion equation of neutral particle near the Kerr-like black hole; 4) The geodesic equations of lightlike particle and photon regions; 5) The shadow of the Kerr-like black hole in a dark matter halo. Finally, section \ref{s6} is our conclusion and discussion. The relevant parameters of the galaxy M87 are $\rho_e=6.9 \times 10^6 M_\odot/kpc^3, r_e=91.2kpc$ and the mass of the black hole at the center of the M87 is $6.5 \times 10^9 M_\odot$ \cite{EventHorizonTelescope:2019dse}. In this paper, we mainly use the black hole units of $G=c=M_\text{BH}=1$, the radius of the black hole (BH) is given by $r_{\text{BH }}=2 GM_\text{BH}/c^2$.



\section{Density profile of dark matter and its spacetime geometry}\label{s2}
Recently, one density profile firstly proposed by J. Einasto has received widespread attention \cite{1965On}. J. Wang et al. use this density profile to fit the observed data of the halo masses (over 20 orders of magnitude), and they find that the observed data can be well described by this density profile of the simple two-parameter \cite{Wang:2019ftp}. This density profile is also known as the Einasto profile, and it is satisfied with the following from,
\begin{equation}
\rho_{\text{eina}} (r)=\rho_e \exp [ -2 \alpha ^{-1} ((\frac{r}{r_e})^\alpha -1 ) ],
\label{e21}
\end{equation}
where, $r_e$ is the characteristic radius and its logarithmic slope is $\text{d ln} \rho / \text{d ln r}=-2$, $\rho_e$ is the halo density and $\alpha$ is a shape parameter\footnote{The authors in Ref. \cite{Wang:2019ftp} fixed this shape parameter $\alpha$ at $\alpha=0.16$. In our work, we reserve $\alpha$ to participate in the calculation later.}. Based on the Eq. (\ref{e21}), the mass distribution of the dark matter halo can be given by the following form,
\begin{equation}
\begin{aligned}
M_{\text{eina}}=4\pi \int_{0}^{r}\rho_{\text{eina}} (r')r'^2dr' =\frac{4\pi e^{2/\alpha } \rho_e \left ( 8^{-1/\alpha } r_e^3 \alpha ^{3/\alpha }\Gamma [\frac{3}{\alpha }] -r^3 \text{Ei}[\frac{-3 + \alpha}{\alpha },\frac{2}{\alpha }(\frac{r}{r_e})^\alpha] \right ) }{\alpha }
\label{e22}
\end{aligned}
\end{equation}
where, $\text{Ei}[n,z]$ is the exponential integral function $\text{Ei}_n(z)$ and $\Gamma[z]$ is the gamma function $\Gamma(z)$. According to Newtonian theory, for a test particle in the equatorial plane of spherical symmetric spacetime, its tangential velocity can be determined by the mass distribution of dark matter. Therefore, the tangential velocity $V$ can be defined as
\begin{equation}
\begin{aligned}
V_{\text{eina}}=\sqrt{\frac{M_\text{eina}}{r}}= \sqrt{4\pi e^{2/\alpha } \rho_e \left ( 8^{-1/\alpha } r_e^3 \alpha ^{3/\alpha }\Gamma [\frac{3}{\alpha }]/r -r^2 \text{Ei}[\frac{-3 + \alpha}{\alpha },\frac{2}{\alpha }(\frac{r}{r_e})^\alpha] \right )/\alpha  }
\label{e23}
\end{aligned}
\end{equation}
On the other hand, for the pure dark matter spacetime, its metric can always be written as a static spherical symmetric form,
\begin{equation}
\begin{aligned}
ds^{2}=-f(r)dt^{2}+\frac{1}{g(r)}dr^{2}+r^{2}(d\theta ^{2}+\sin^{2}\theta d\phi ^{2}),
\label{e24}
\end{aligned}
\end{equation}
where, $f(r)$ is the redshift function and $g(r)$ is the shape function. The relationship between the redshift function $f(r)$ and the tangential velocity $V$ is as follows \cite{Matos:2003nb},
\begin{equation}
V^2=\frac{r}{\sqrt{f(r)}}\frac{d\sqrt{f(r)}}{dr}=r\frac{dln\sqrt{f(r)}}{dr}
\label{e25}
\end{equation}
Here, we mainly consider the case of spherical symmetry in Eq. (\ref{e24}), that is, $f(r)=g(r)$. According to Eqs. (\ref{e23}) and (\ref{e25}), we can get an analytical solution of the metric coefficient $f(r)$ in a dark matter halo, and it has the following form,
\begin{equation}
f_{\text{eina}}(r)=\exp\left (- \frac{4\pi 2^{1-3/\alpha } e^{2/\alpha } r^2 ((r/r_e)^{\alpha }/\alpha )^{-3/\alpha }\rho_e \Gamma [\frac{3}{\alpha },0,\frac{2(\frac{r}{r_e})^\alpha} {\alpha} ]}{\alpha}  \right )
\label{e26}
\end{equation}
where, $\Gamma[a,z_0,z_1]$ is the generalized incomplete gamma function $\Gamma(a,z_0)-\Gamma(a,z_1)$. As an example, we consider the galaxy of M87 and its relevant parameter are $\rho_e=6.9 \times 10^6 M_\odot/kpc^3, r_e=91.2kpc$. The mass of the black hole in the center of M87 is $M_{\text{BH}} = 6.5 \times 10^{9}M_{\bigodot}$.  In this case, the dark matter parameters can be converted in the black hole units (BHU) by the following formula: $r_\text{e}(\text{BHU})=r_e / (2 G M_{\text{BH}}/c^2) \times r_\text{BH}$ and $\rho_\text{e}(\text{BHU})=\rho_e/(M_\text{BH}/(4/3 \pi (2 G M_{\text{BH}}/c^2)^3))\times \rho_{\text{BH}}$. For the $f(r)$ in Eq. (\ref{e26}), it is not difficult to find the following form,
\begin{equation}
\lim_{\rho_e\rightarrow 0}f_{\text{eina}}(r)=1,\lim_{r\rightarrow \infty }f_{\text{eina}}(r)=1,
\label{e27}
\end{equation}
The Eqs. (\ref{e27}) show that when the density of dark matter is $0$ or the distance is infinity away from the black hole, the dark matter is absent. Therefore, the spacetime return to Minkowshi flat spacetime.


\section{Schwarzschild-like black hole in a dark matter halo}\label{s3}
In this section, we extend our results (Eq. \ref{e26}) to the case of black hole in a dark matter halo. We will strictly follow the method recorded in the Refs. \cite{Xu_2018,Xu:2021dkv}. First of all, we need to solve the Einstein field equation of pure dark matter spacetime, which has the following form,
\begin{equation}
R_{\mu \nu }-\frac{1}{2}g_{\mu \nu }R=\kappa ^2 T_{\mu \nu }(\text{DM-halo}),
\label{e31}
\end{equation}
where, $R_{\mu \nu }$ is Ricci tensor, $g_{\mu \nu }$ is the metric of pure dark matter spacetime and $R$ is Ricci scalar. The energy momentum tensor of this spacetime can be defined as ${T^\nu}_\mu=g^{\nu \alpha }T_{\mu \alpha }=diag[-\rho ,p_r,p,p]$, and $g^{\nu \alpha }$ is the inverse metric of the spacetime. Taking Eq. (\ref{e24}) into Eq. (\ref{e31}), we can obtain the following equation,
\begin{equation}
\begin{aligned}
&\kappa ^2 {T^{t}}_{t}(\text{DM-halo})=g(r)\left ( \frac{1}{r}\frac{g'(r)}{g(r)}+\frac{1}{r^2} \right ) - \frac{1}{r^2}, \\
&\kappa ^2 {T^{r}}_{r}(\text{DM-halo})=g(r)\left ( \frac{1}{r^2} + \frac{1}{r}\frac{f'(r)}{f(r)} \right )-\frac{1}{r^2},\\
&\kappa ^2 {T^{\theta }}_{\theta }(\text{DM-halo})=\kappa ^2 {T^{\phi }}_{\phi }(\text{DM-halo})\\
&=\frac{1}{2}g(r)\left ( \frac{f''(r)f(r)-f'(r)^2}{f(r)^2}+\frac{f'(r)^2}{2f(r)^2} +\frac{1}{r}(\frac{f'(r)}{f(r)}+\frac{g'(r)}{g(r)}) + \frac{f'(r)g'(r)}{2f(r)g(r)}\right ).
\label{e32}
\end{aligned}
\end{equation}
When the black hole is in a dark matter halo, then the energy momentum tensor can be written as ${T^\nu}_\mu={T^\nu}_\mu(\text{BH}) +{T^\nu}_\mu(\text{DM-halo})$. For Schwarzschild black hole, the energy momentum tensor is $0$, that is, ${T^\nu}_\mu(\text{BH})=0$. That is to say, we only need to consider the energy-momentum tensor ${T^\nu}_\mu(\text{DM-halo})$ of the dark matter halo. Therefore, based on the above analysis, we can assume the new metric of black hole in a dark matter halo to be in the following form,
\begin{equation}
\begin{aligned}
ds^{2}=-(f(r)+F1(r))dt^{2}+\frac{1}{g(r)+G1(r)}dr^{2}+r^{2}(d\theta ^{2}+\sin^{2}\theta d\phi ^{2}),
\label{e33}
\end{aligned}
\end{equation}
where, $f(r)$ and $g(r)$ are the coefficients of the pure dark matter halo. Then Eq. (\ref{e31}) can be written as
\begin{equation}
R_{\mu \nu }-\frac{1}{2}g_{\mu \nu }R=\kappa ^2 (T_{\mu \nu }(\text{BH})+T_{\mu \nu }(\text{DM-halo})).
\label{e34}
\end{equation}
\begin{comment}
\begin{equation}
\begin{aligned}
&\kappa ^2 {T^{t}}_{t}(\text{BH+DM})=4((g(r)+G1(r))\left ( \frac{4}{r^2}+\frac{1}{r}\frac{g'(r)+G1'(r)}{g(r)+G1(r)} \right ) - \frac{1}{r^2})=4(g(r)\left ( \frac{4}{r^2}+\frac{1}{r}\frac{g'(r)}{g(r)} \right ) - \frac{1}{r^2}) \\
&\kappa ^2 {T^{r}}_{r}(\text{BH+DM})=4((g(r)+G1(r))\left ( \frac{4}{r^2} + \frac{4}{r}\frac{f'(r)+F1'(r)}{f(r)+F1(r)} \right )-\frac{1}{r^2})=4(g(r)\left ( \frac{4}{r^2} + \frac{4}{r}\frac{f'(r)}{f(r)}\right )-\frac{1}{r^2})\\
\end{aligned}
\end{equation}
\end{comment}
Taking Eq. (\ref{e33}) into Eq. (\ref{e34}) and then comparing with Eq. (\ref{e32}), we can obtain the following forms,
\begin{equation}
\begin{aligned}
&(g(r)+G1(r))\left ( \frac{1}{r^2}+\frac{1}{r}\frac{g'(r)+G1'(r)}{g(r)+G1(r)} \right ) =g(r)\left ( \frac{1}{r^2}+\frac{1}{r}\frac{g'(r)}{g(r)} \right ),\\
&(g(r)+G1(r))\left ( \frac{1}{r^2} + \frac{1}{r}\frac{f'(r)+F1'(r)}{f(r)+F1(r)} \right )=g(r)\left ( \frac{1}{r^2} + \frac{1}{r}\frac{f'(r)}{f(r)}\right ).
\label{e35}
\end{aligned}
\end{equation}
The first equation of Eq. (\ref{e35}) is only related to function $G1$. The second equation is related to functions $G1,F1$ and it can be simplified to the following form
\begin{equation}
\begin{aligned}
\frac{f'(r)+F1'(r)}{f(r)+F1(r)} = \frac{g(r)}{g(r)+G1(r)}\left ( \frac{1}{r} + \frac{f'(r)}{f(r)}\right ) -\frac{1}{r},
\label{e36}
\end{aligned}
\end{equation}
Through the numerical method with the boundary condition of Schwarzschild black hole, we can get the analytical expressions of functions $G1, F1$, which are as follows,
\begin{equation}
\begin{aligned}
&F1(r)=\exp\left ( \int \frac{g(r)}{g(r)+G1(r)}\left ( \frac{1}{r}+\frac{f'(r)}{f(r)} \right )dr-\frac{1}{r}dr \right )-f(r),\\
&G1(r)=-\frac{2M_{\text{BH}}}{r},
\label{e37}
\end{aligned}
\end{equation}
where, $M_\text{BH}$ is the mass of the black hole. If considering $f(r)=g(r)$ in Eq. (\ref{e35}), it is not difficult to find that $F1(r)=G1(r)=-2M_\text{BH}/r$. Therefore, the new metric for the Schwarzschild-like black hole in a dark matter halo is given by the following form,
\begin{equation}
\begin{aligned}
ds^{2}=-F(r)dt^{2}+\frac{1}{G(r)}dr^{2}+H(r)(d\theta ^{2}+\sin^{2}\theta d\phi ^{2}),
\label{e38}
\end{aligned}
\end{equation}
where, $F(r)=f(r)-F1(r), G(r)=g(r)-G1(r), H(r)=r^{2}$ and $f(r)/g(r)$ represent the factor terms for considering pure dark matter halo in Eq. (\ref{e26}). Finally, the metric coefficients for the Schwarzschild-like black hole in a dark matter halo are as follows
\begin{equation}
F(r)=G(r)=\exp\left (- \frac{4\pi 2^{1-3/\alpha } e^{2/\alpha } r^2 ((r/r_e)^{\alpha }/\alpha )^{-3/\alpha }\rho_e \Gamma [\frac{3}{\alpha },0,\frac{2(\frac{r}{r_e})^\alpha} {\alpha} ]}{\alpha}  \right )-\frac{2M_{\text{BH}}}{r}
\label{e39}
\end{equation}
If the dark matter is absent, that is, $\rho_e=0$, and then the Eq. (\ref{e39}) degenerates into the Schwarzschild black hole.




\section{Kerr-like black hole in a dark matter halo}\label{s4}
In the previous section, we derived the metric of Schwarzschild-like black hole in a dark matter halo from the Einasto profile. Next, we will extend this Schwarzschild-like black hole metric to the Kerr-like metric. The method we used is Newman-Janis (N-J) algorithm, which is one of the commonly used and efficient methods \cite{Newman:1965tw,Azreg-Ainou:2014pra,Azreg-Ainou:2014nra}. Strictly following this method, we firstly transform Eq. (\ref{e38}) from Boyer-Lindquist coordinates $(t,r,\theta,\phi)$ to Eddington-Finkelstein coordinates ($u,r,\theta,\phi$),
\begin{equation}
du=dt -\frac{dr}{\sqrt{F(r)G(r)}},
\label{e41}
\end{equation}
and then we can obtain the following form,
\begin{equation}
ds^2=-F(r)du^2 -2\sqrt{\frac{F(r)}{G(r)}}dudr+H(r)(d\theta ^2 + \sin^2\theta d\phi ^2),
\label{e42}
\end{equation}
where, $F(r), G(r), H(r)$ are the metric functions. In the null tetrad, the inverse metric of Schwarzschild-like black hole can be composed of basis vectors $l^{\mu},n^{\mu},m^{\mu},\bar{m}^{\mu}$, and it has the following form,
\begin{equation}
g^{\mu\nu}=-l^{\mu}n^{\nu}-l^{\nu}n^{\mu}+m^{\mu}\bar{m}^{\nu}+m^{\nu}\bar{m}^{\mu},
\label{e43}
\end{equation}
where, these basis vectors $l^{\mu},n^{\mu},m^{\mu},\bar{m}^{\mu}$ are as follows,
\begin{equation}
\begin{aligned}
&l^{\mu}=\delta _r^\mu,\\
&n^\mu=\sqrt{\frac{G(r)}{F(r)}}\delta _u^\mu-\frac{1}{2}G(r)\delta _r^\mu,\\
&m^\mu=\frac{1}{\sqrt{2H(r)}}\left ( \delta _\theta ^\mu + \frac{i}{\sin{\theta }}\delta _\phi ^\mu \right ),\\
&\bar{m}^\mu=\frac{1}{\sqrt{2H(r)}}\left ( \delta _\theta ^\mu - \frac{i}{\sin{\theta }}\delta _\phi ^\mu \right ),
\label{e44}
\end{aligned}
\end{equation}
The relationship between these basis vectors satisfies the conditions of normalization, orthogonality and isotropy. According to the Newman-Janis algorithm, the spacetime coordinates between different observers satisfy the complex transformation. Therefore, we can write the new coordinates as follows,
\begin{equation}
\begin{aligned}
x'^\mu=x^\mu+ia(\delta _r^\mu - \delta^\mu_\mu )\cos\theta \rightarrow \left\{\begin{matrix}\begin{aligned}
&u'=u-ia\cos\theta \\
&r'=r+ia\cos\theta \\
&\theta '=\theta \\
&\phi '=\phi
\end{aligned}
\end{matrix}\right.
\label{e45}
\end{aligned}
\end{equation}
where, $a$ is rotation parameter and then these metric functions can be rewritten as $F(r)\rightarrow\mathcal{F}(r,a,\theta)$, $G(r)\rightarrow\mathcal{G}(r,a,\theta)$, $H(r)\rightarrow \Sigma(r,a,\theta)$. Therefore, Eq. (\ref{e43}) can be written in the null tetrad using basis vectors as follows,
\begin{equation}
\begin{aligned}
&l^{\mu}=\delta _r^\mu,\\
&n^\mu=\sqrt{\frac{\mathcal{G}}{\mathcal{F}}}\delta _u^\mu-\frac{1}{2}\mathcal{G}\delta _r^\mu,\\
&m^\mu=\frac{1}{\sqrt{2\Sigma }}\left ( \delta _\theta ^\mu +ia\sin\theta (\delta _u^\mu-\theta _r^\mu)+ \frac{i}{\sin{\theta }}\delta _\phi ^\mu \right ),\\
&\bar{m}^\mu=\frac{1}{\sqrt{2\Sigma }}\left ( \delta _\theta ^\mu -ia\sin\theta (\delta _u^\mu-\theta _r^\mu)- \frac{i}{\sin{\theta }}\delta _\phi ^\mu \right ).
\label{e46}
\end{aligned}
\end{equation}
With the newly defined quartile vectors, we can get the contravariant non-zero components of the new metric $g^{\mu\nu}$ from Eq. (\ref{e43}), which are as follows,
\begin{equation}
\begin{aligned}
&g^{uu}=\frac{a^2\sin\theta^2}{\Sigma },g^{rr}=\mathcal{G}+\frac{a^2\sin\theta^2}{\Sigma },g^{\theta \theta }=\frac{1}{\Sigma },g^{\phi \phi }=\frac{1}{\Sigma \sin\theta ^2},\\
&g^{ur}=g^{ru}=-\sqrt{\frac{\mathcal{G}}{\mathcal{F}}}-\frac{a^2\sin\theta^2}{\Sigma },g^{u\phi }=g^{\phi u}=\frac{a}{\Sigma },g^{r\phi }=g^{\phi r}=-\frac{a}{\Sigma},
\label{e47}
\end{aligned}
\end{equation}
where, $\Sigma=r^2+a^2\cos^2\theta$. Based on these inverse metric and then taking the inverse of Eq. (\ref{e47}), the deformed Kerr black hole in Eddington Finkelstein coordinates has the following form,
\begin{equation}
\begin{aligned}
ds^{2}=&-\mathcal{F}du^{2}-2\sqrt{\frac{\mathcal{F}}{\mathcal{G}}}dudr+2a\sin^2\theta \left (\mathcal{F}- \sqrt{\frac{\mathcal{F}}{\mathcal{G}}} \right )dud\phi +2a\sin^2\theta \sqrt{\frac{\mathcal{F}}{\mathcal{G}}}drd\phi \\
&+\Sigma d\theta ^2  + \sin^2\theta \left [ \Sigma +a^2\sin^2\theta \left ( 2\sqrt{\frac{\mathcal{F}}{\mathcal{G}}} - \mathcal{F} \right ) \right ]d\phi ^2.
\label{e48}
\end{aligned}
\end{equation}
Finally, we only need to convert this EF coordinates into the coordinates we are familiar with, that is, BLC coordinates. Firstly, we introduce the following transformations,
\begin{equation}
\begin{aligned}
du=dt-\frac{K(r)+a^2}{G(r)H(r)+a^2 }dr, d\phi =d\varphi -\frac{a}{G(r)H(r)+a^2 }dr,
\label{e49}
\end{aligned}
\end{equation}
where, $K(r)=H(r)\sqrt{G(r)/F(r)}$ and
\begin{equation}
\begin{aligned}
\mathcal{F}(r,a,\theta)=\frac{(G(r)H(r)+a^2\cos^2\theta)\Sigma }{(K(r)+a^2\cos^2\theta)^2},
\mathcal{G}(r,a,\theta)=\frac{G(r)H(r)+a^2\cos^2\theta }{\Sigma },
\label{e410}
\end{aligned}
\end{equation}
and then taking Eqs. (\ref{e49}), (\ref{e410}) into (\ref{e48}), we can obtain the final form of metric under BLC coordinates,
\begin{equation}
\begin{aligned}
ds^2= & -\frac{\left(G H+a^2 \cos ^2 \theta\right) \Sigma}{\left(K+a^2 \cos ^2 \theta\right)^2} dt^2+\frac{\Sigma}{G H+a^2} dr^2-2 a \sin ^2 \theta\left[\frac{K-G H}{\left(K+a^2 \cos ^2 \theta\right)^2}\right] \Sigma dt d \phi \\
&+ \Sigma d\theta^2  +\Sigma \sin ^2 \theta\left[1+a^2 \sin ^2 \theta \frac{2 K-G H+a^2 \cos ^2 \theta}{\left(K+a^2 \cos ^2 \theta\right)^2}\right] d\phi^2,
\label{e411}
\end{aligned}
\end{equation}
where, $G, H, K$ are the function of $r$ we introduced before but the specific form of $\Sigma$ is unknown. In trying to find a suitable $\Sigma$, we need to detect whether it is real physical solutions. Luckily, we have known the solution of a rotating regular black hole. Therefore, we introduce the following constraint, that is $\Sigma=r^2+a^2\cos^2\theta$. With the case $F(r)=G(r), H(r)=r^2$ we introduced before, we find that $K(r)=H(r)=r^2$. Finally, the Eq. (\ref{e411}) can be written as Kerr-like form we are familiar,
\begin{equation}
\begin{aligned}
ds^2= & -(1-\frac{r^2-g(r)r^2+2Mr}{\Sigma }) dt^2 +\frac{\Sigma}{\Delta } dr^2 -2 a \sin ^2 \theta(\frac{r^2-g(r)r^2+2Mr}{\Sigma }) dt d \phi \\
&+ \Sigma d\theta^2 +\left ( (a^2+r^2)\sin^2 \theta +\frac{a^2\sin^4\theta(r^2-g(r)r^2+2Mr)}{\Sigma }\right ) d\phi^2,%_{\text{BH}}
\label{e412}
\end{aligned}
\end{equation}
where,
\begin{equation}
\begin{aligned}
\Delta&=r^2G(r)+a^2=r^2(g(r)-2M/r)+a^2\\
&=\exp\left ( \frac{2\alpha\text{ln}r-4\pi 2^{1-3/\alpha } e^{2/\alpha } r^2 ((r/r_e)^{\alpha }/\alpha )^{-3/\alpha }\rho_e \Gamma [\frac{3}{\alpha },0,\frac{2(\frac{r}{r_e})^\alpha} {\alpha} ]}{\alpha}  \right )-2Mr+a^2.
\label{e413}
\end{aligned}
\end{equation}
It is easy to find that in Eq. (\ref{e412}), if the dark matter is absent, (i.e., $\rho_e=0$, and then $g(r)=1$)), the metric of black hole in a dark matter halo degenerates into the Kerr metric,
\begin{equation}
\begin{aligned}
ds^2=&  -(1-\frac{2Mr}{\Sigma }) dt^2 +\frac{\Sigma}{\Delta } dr^2 -\frac{4Mr a  \sin ^2 \theta }{\Sigma } dt d \phi + \Sigma d\theta^2 \\
&+\left ( (a^2+r^2)\sin^2 \theta +\frac{2Mr a^2\sin^4\theta }{\Sigma }\right ) d\phi^2.
\label{e414}
\end{aligned}
\end{equation}



\section{Some properties of black holes in a dark matter halo}\label{s5}
Through the introduction of the previous sections, we have strictly given the metrics of the Schwarzschild-like black hole and the Kerr-like black hole in a dark matter halo. The difference between them and Schwarzschild black hole and Kerr black hole is that there is an additional dark matter term, that is $f(r), g(r)$ in Eqs. (\ref{e39}) and (\ref{e412}). In this section, we will introduce the physical properties of this special black hole from several aspects such as horizons of black hole, shape of the ergosphere, motion equation of the neutral particle and so on. Finally, we compare these properties with the Kerr black hole, so that readers can understand the black hole in a dark matter halo more intuitively.
\subsection{Horizons of black hole in a dark matter halo}
The event horizon of a black hole is the outermost boundary of the black hole. In black hole physics, the event horizon of a black hole is a type of zero-surface with spacetime symmetry. General relativity shows that the Schwarzschild black hole has only one event horizon, which is determined entirely by the mass of the black hole. And the Kerr black hole has two event horizons, namely the event horizon and the Cauchy horizon, which are completely determined by the mass and angular momentum of the black hole.

For the Kerr-like black hole in a dark matter halo, its horizons can be found by solving $\Delta(r)=0$ in Eq. (\ref{e413}), that is,
\begin{equation}
\begin{aligned}
\exp\left ( \frac{2\alpha\text{ln}r-4\pi 2^{1-3/\alpha } e^{2/\alpha } r^2 ((r/r_e)^{\alpha }/\alpha )^{-3/\alpha }\rho_e \Gamma [\frac{3}{\alpha },0,\frac{2(\frac{r}{r_e})^\alpha} {\alpha} ]}{\alpha}  \right )-2Mr+a^2=0,
\label{e51}
\end{aligned}
\end{equation}
In Figure \ref{f1}, we present the  functional image of $\Delta(r)$ as a function of the independent variate $r$ both in dark matter spacetime (left panel) and Kerr spacetime (middle panel). The number of these intersections with the x-axis in the image represents the number of roots of the equation $\Delta(r)=0$, which in black hole physics represents the horizons of the black hole. Our results show that black hole has the same number of horizons as Kerr black hole in a dark matter halo. The outer horizon decreases with the increase of the rotation parameter $a$, while the inner horizon increases with the increase of $a$ until the inner and outer event horizons are equal, that is, the extreme black hole. In order to distinguish the difference between the black hole in a dark matter halo and the Kerr spacetime, we define their difference $\delta \Delta(r)=\Delta_{\text{DM}}(r)-\Delta_{\text{Kerr}}(r)$, and then their differences as a function of rotation parameter $a$ are given in Figure \ref{f1} (right panel) at a fixed $r$. Our results show that the $\Delta(r)$ of the dark matter halo is greater than that of the Kerr spacetime, that is, $\Delta_{\text{DM}} > \Delta_{\text{Kerr}}$. In the right panel of Figure \ref{f1}, the maximum difference between them is approximately $7 \times 10^{-11}$, and the difference between them will increase with the increasing of variable parameter $r$.
% Figure environment removed

\subsection{Shape of ergosphere of black hole in a dark matter halo}
Next, let us study the shape of the ergosphere of this Kerr-like black hole in a dark matter halo. Firstly, in Figure \ref{f2}, we give the cross-section diagram of Kerr-like black hole in the dark matter halo in the $xz$-plane. In the black hole physics, the ergosphere is composed of two parts, one is the area surrounded by the outer infinite redshift surface and the outer event horizon, and the other is the area surrounded by the inner infinite redshift surface and the inner event horizon. The central area in the middle is a singular ring. Then, the inner and outer horizons and the infinite redshift surface of the Kerr-like black hole can be obtained by solving the Eqs. $\Delta(r)=0$ and $g_{tt}=0$, respectively, that is Eq. (\ref{e51}) and
\begin{equation}
\begin{aligned}
\exp\left ( \frac{2\alpha\text{ln}r-4\pi 2^{1-3/\alpha } e^{2/\alpha } r^2 ((r/r_e)^{\alpha }/\alpha )^{-3/\alpha }\rho_e \Gamma [\frac{3}{\alpha },0,\frac{2(\frac{r}{r_e})^\alpha} {\alpha} ]}{\alpha}  \right )-2Mr+a^2\cos^2\theta=0,
\label{e52}
\end{aligned}
\end{equation}
From these figures in Figure \ref{f2}, for the Kerr-like black hole in a dark matter halo, we find that the outer horizon and outer infinite redshift surface decrease with the increase of rotation parameter $a$, while the inner horizon and inner infinite redshift surface increase with the increase of rotation parameter $a$. In addition, according to the calculation of the horizons in Figure \ref{f1}, dark matter may also increase the range of the ergosphere for the Kerr-like black hole.
% Figure environment removed



\subsection{Motion equations of neutral particle near black hole in a dark matter halo}
In this subsection, we are interested in the neutral particle motion near the Kerr-like black hole in a dark matter halo, which can be solved by geodesic equations. Firstly, we rewrite Eq. (\ref{e412}) in the Kerr-like form,
\begin{equation}
\begin{aligned}
ds^2= & -(1-\frac{2R(r)r}{\Sigma }) dt^2 +\frac{\Sigma}{\Delta } dr^2 -\frac{4 R(r) r a \sin ^2 \theta }{\Sigma } dt d \phi + \Sigma d\theta^2 \\
&+\left ( (a^2+r^2)\sin^2 \theta +\frac{2 R(r) r a^2\sin^4\theta }{\Sigma }\right ) d\phi^2,
\label{e53}
\end{aligned}
\end{equation}
where,
\begin{equation}
\begin{aligned}
2R(r)=2M + r-rg(r), \quad  \Delta =r^2-2R(r)r+a^2, \quad \Sigma=r^2+a^2\cos^2 \theta
\label{e54}
\end{aligned}
\end{equation}
The geodesic equation of the neutral particle in a dark matter halo satisfy the Lagrangian equation \cite{Khan:2019gco},
\begin{equation}
\mathscr{L}=\frac{1}{2}g_{{\mu \nu }}\dot{x}^\mu \dot{x}^\nu,
\label{e55}
\end{equation}
where, $g_{\mu\nu}$ is the metric of Eq. (\ref{e53}) and $\dot{x}^\mu$ denotes the first-order partial derivative to the affine parameter of the coordinate $x^\mu$. The 4-momentum $p_\mu$ of the neutral particle from Eq. (\ref{e53}) is $p_\mu =\partial \mathscr{L} / \partial \dot{x}^\mu =g_{\mu\nu}\dot{x}^\nu$, and then
\begin{equation}
\begin{aligned}
-p_t=g_{tt}\dot{t}+g_{t\phi }\dot{\phi }=E, \quad p_\phi=g_{t\phi }\dot{t}+g_{\phi\phi }\dot{\phi }=L, \quad p_r=g_{rr}\dot{r}, \quad p_\theta =g_{\theta \theta }\dot{\theta },
\label{e56}
\end{aligned}
\end{equation}
where, $E$ and $L$ are the energy and angular momentum of the neutral particle, respectively. In addition, $p_t$ and $p_\phi$ are a conserved quantity because the Lagrangian is not related to the coordinates $t$ and $\phi$. Therefore, the Kerr-like black hole in a dark matter halo has the properties of steady state and axisymmetric. Through the first two equations in Eq. (\ref{e56}), we can reverse solve $\dot{t}$ and $\dot{\phi}$, that is,
\begin{equation}
\begin{aligned}
\dot{t}&=\frac{1}{\Sigma \Delta }(-E(a^2+r^2)\Sigma +2ar(-L-aE\sin^2\theta)R(r)),\\
\dot{\phi}&=-\frac{1}{\Sigma \Delta \sin^2\theta }(-L \Sigma +2r(L+aE\sin^2\theta)R(r)).\\
\label{e57}
\end{aligned}
\end{equation}
Based on the momentum and Lagrangian above, the Hamiltonian of the neutral particle motion can be written as the following form,
\begin{equation}
\begin{aligned}
H=-p_t\dot{t}+p_r\dot{r}+p_\theta\dot{\theta}+p_\phi\dot{\phi}-\mathscr{L},
\label{e58}
\end{aligned}
\end{equation}
Taking Eqs. (\ref{e53}), (\ref{e55}) and (\ref{e56}) into account, Eq. (\ref{e58}) can be written as
\begin{equation}
\begin{aligned}
2H&=-(g_{tt}\dot{t}+g_{t\phi}\dot{\phi})\dot{t}+g_{rr}\dot{r}^2+g_{\theta\theta}\dot{\theta}^2+(g_{t\phi}\dot{t}+g_{\phi\phi}\dot{\phi})\dot{\phi}\\
&=E\dot{t}+L\dot{\phi}+\frac{\Sigma }{\Delta }\dot{r}^2+\Sigma \dot{\theta}^2=m=constant,
\label{e59}
\end{aligned}
\end{equation}
where, $m$ is the mass of the neutral particle, and $m=-1,0,1$ means the timelike, lightlike (or null) and spacelike geodesics, respectively. Eqs. (\ref{e57}), (\ref{e58}), (\ref{e59}) are very important, because they can be used to study the properties of particle near black hole. Eq. (\ref{e59}) is a second order partial differential equation. Usually, a separation constant can be introduced to reduce it to a radial equation and an angular equation. Where, the radial equation describes the motion of the neutral particle near the black hole. As an example, we study the neutral particle whose orbits lie in the equatorial plane (i.e., $\theta=\pi/2, m=0$), and its motion equation is given by the following form
\begin{equation}
\begin{aligned}
\dot{r}^2=E^2+\frac{1}{r^2}(a^2E^2-L^2)+\frac{2R(r)}{r^3}(aE+L)^2.
\label{e510}
\end{aligned}
\end{equation}
In particular, when the dark matter is absent ($\rho_e=0$), that is $g(r) = 1$ in $R(r)$, the radial equation will return to the case of the Kerr vacuum.


\subsection{The geodesic equations of lightlike particle and photon regions}
In the previous subsection, we calculated the motion equation for the neutral particle using the momentum and Lagrangian based on the Eq. (\ref{e53}). Here, we rewrite this Equation in the following form,
\begin{equation}
\begin{aligned}
ds^2= & -(1-\frac{2R(r)r}{\Sigma }) dt^2 +\frac{\Sigma}{\Delta } dr^2 -\frac{4 R(r) r a \sin ^2 \theta }{\Sigma } dt d \phi + \Sigma d\theta^2 \\
&+ \frac{\sin^2\theta}{\Sigma }((r^2+a^2)^2-a^2\Delta \sin^2\theta) d\phi^2,
\label{e511}
\end{aligned}
\end{equation}
where,
\begin{equation}
\begin{aligned}
2R(r)=2M + r-rg(r), \quad  \Delta =r^2-2R(r)r+a^2, \quad \Sigma=r^2+a^2\cos^2 \theta.
\label{e512}
\end{aligned}
\end{equation}
In this subsection, we will focus on the motion of light particles and their photon regions. Therefore, we need to know the geodesic equations describing the lightlike particles. The lightlike geodesic equation can be obtained by combining the momentum and the Hamilton-Jacobi equation,
\begin{equation}
-\frac{\partial S}{\partial \sigma }=\frac{1}{2}g^{\mu\nu}\frac{\partial S}{\partial x^\mu}\frac{\partial S}{\partial x^\nu},
\label{e513}
\end{equation}
where, $S$ is Hamiltonian-Jacobi action, $\sigma$ is an affine parameter and $g^{\mu\nu}$ is the inverse metric of Eq. (\ref{e511}). For a test particle moving along the geodesic of the Kerr-like black hole, the Hamiltonian-Jacobi action $S$ can always be written in the separated form,
\begin{equation}
S=\frac{1}{2}m^2\sigma -Et+L\phi +S_\theta (\theta )+S_r(r),
\label{e514}
\end{equation}
where, $m$ is the mass of the particle (the photon takes $m=0$). Taking Eqs. (\ref{e511}) and (\ref{e514}) into account, the Eq. (\ref{e513}) can be written as the following two equations,
that is
\begin{equation}
\Delta  S'(r)^2-\frac{((r^2+a^2)E-aL)^2}{\Delta}+(L-aE)^2=-K,
\label{e515}
\end{equation}
and
\begin{equation}
S'(\theta )^2-a^2 E^2 \cos^2\theta+L^2 \cot^2\theta=K,
\label{e516}
\end{equation}
where $K$ is a constant obtained by separating variables, which is called Carter constant \cite{Carter:1968rr}. Besides, we note that $\partial S /\partial x^\mu = p_\mu$, and then it is not difficult to obtain the geodetic equations of the lightlike particles,
\begin{equation}
\begin{aligned}
&\Sigma  \dot{r} = \sqrt{R},\\
&\Sigma  \dot{\theta } = \sqrt{\Theta},\\
&\Sigma \dot{t}=E\left ( \frac{(a^2+r^2)(a^2+r^2-a\lambda )}{\Delta }-a( a \sin^2\theta - \lambda) \right ),\\
&\Sigma \dot{\phi }=-E\left ( \frac{a(a^2+r^2+a \lambda \csc^2 \theta)}{\Delta }-(a+\lambda \csc^2\theta ) \right ),\\
\label{e517}
\end{aligned}
\end{equation}
where,
\begin{equation}
\begin{aligned}
\lambda &=L/E, \\
\eta &=K/E^2,\\
R&=E^2\left ( (a^2+r^2-a\lambda )^2- \Delta ((\lambda-a)^2+\eta ) \right ), \\
\Theta &=E^2\left [ \eta -\lambda ^2\cot^2\theta +a^2\cos^2\theta \right ],
\label{e518}
\end{aligned}
\end{equation}
the dot (·) in the Eq. (\ref{e517}) denotes the derivative to the proper time $\tau$. For a photon in a spherical orbit, it must satisfy the following conditions,
\begin{equation}
R(r) \mid _{r=r_c}=0,    \frac{\mathrm{d} R(r)}{\mathrm{d} r}\mid _{r=r_c}=0,
\label{e519}
\end{equation}
and then we can obtain the following form,
\begin{equation}
\begin{aligned}
\lambda &=\frac{-4r\Delta(r)+a^2\Delta '(r)+r^2\Delta '(r)}{a \Delta '(r)}\mid _{r=r_c}, \\
\eta &=\frac{r^2 (16a^2\Delta (r)-16\Delta (r)^2+8r\Delta (r)\Delta '(r)-r^2\Delta '(r)^2)}{a^2\Delta '(r)^2}\mid _{r=r_c},
\label{e520}
\end{aligned}
\end{equation}
where, $\Delta '(r)$ represents the derivative of $\Delta$ with respect to $r$, and $r_c$ is the orbital radius of photon sphere. Taking Eq. (\ref{e520}) into (\ref{e518}) can be obtained
\begin{equation}
R''(r)=8 E^2 \left(r^2+\frac{2 r\Delta (r) \left(\Delta '(r)-r \Delta ''(r)\right)}{\Delta'(r)^2}\right) \mid _{r=r_c},
\label{e521}
\end{equation}
In particular, if $R''(r_c)>0$, the radial equation corresponding to the lightlike geodesics is unstable, but $R''(r_c)<0$ is stable. The unstable orbit of the photon will determine the shape of shadow of the black hole. The range of the photon regions $r_c$ can be obtained by the equation $\Theta \geq 0$, that is
\begin{equation}
-16\Delta(r)^2-r^2\Delta'(r)^2+8\Delta(r)(2a^2+r\Delta'(r)) \mid _{r=r_c} \leq 0,
\label{e522}
\end{equation}
Combining Eqs. (\ref{e521}) and (\ref{e522}), the range of photon region $r_c$ may be $r_{cmin}\leq r_c \leq r_{cmax}$. Here, $r_{cmin}, r_{cmax}$ are the minimum and maximum positions of the photon regions, respectively. It can be seen that there are maximum and minimum values of the photon sphere.



\subsection{The shadow of the Kerr-like black hole in a dark matter halo}
In the previous subsection, we calculated the geodesic equations for the motion of lightlike particle and the photon regions in a dark matter halo. Next, we will study the black hole shadow of the Kerr-like black hole in a dark matter halo based on these geodesic equations, and compare them with that of the Kerr black hole. For an asymptotically flat spacetime, we can easily calculate the shadow of the black hole in celestial coordinates, and the formula of the corner radius of the shadow has the same form as that of the shadow corner radius of the black hole in Minkowshi spacetime \cite{Hioki:2009na}, that is
\begin{equation}
\begin{aligned}
\alpha &=\lim_{r_0\rightarrow \infty }\left ( -r_0^2 \sin(\theta )\frac{d\phi }{dr}\mid _{\theta =\theta _0} \right ),\\
\beta &=\lim_{r_0\rightarrow \infty }\left [ r_0^2\frac{d\theta }{dr}\mid _{\theta =\theta _0} \right ],
\label{e523}
\end{aligned}
\end{equation}
where, $(r_0,\theta_0)$ is the position of the observer, and the motion of the light particle is described by $d\phi /dr $ and $d\theta /dr$ from Eq. (\ref{e517}). Here, we choose to observe the shape of shadow of the Kerr-like black hole on the equator plane, that is, $\theta_0=\pi/2$. Observing the shadow of the Kerr-like black hole on the equator plane means that we observe the shadow of the black hole from the direction perpendicular to the equator plane of the black hole. This viewing angle can provide some interesting and important information, because from this angle, we can see the impacts of the rotation parameter $a$ of the black hole on the shadow of the black hole. Then the above equation can be written as
\begin{equation}
\begin{aligned}
\alpha &=-\frac{\lambda }{\sin(\theta )}\mid_{\theta =\frac{\pi}{2}}=-\lambda,  \\
\beta &=\pm \sqrt{\eta -a^2\cos(\theta )^2-\lambda ^2\cot(\theta )^2}\mid_{\theta =\frac{\pi}{2}}=\pm\sqrt{\eta }.
\label{e524}
\end{aligned}
\end{equation}
Taking Eqs. (\ref{e413}) and (\ref{e520}) into account, we can obtain the Celestial coordinates in a dark matter halo, that is
\begin{equation}
\begin{aligned}
\alpha =&-\frac{r^2 \left(a^2+r^2\right) g'(r)+2 r \left(a^2-r^2\right) g(r)-2 a^2 M-4 a^2 r+6 M r^2}{a \left(r^2 \left(-g'(r)\right)-2 r g(r)+2 M\right)},\\
\beta =&\pm \sqrt{\frac{r^5 \left(8 a^2 g'(r)+\left(2 g(r)-r g'(r)\right) \left(r^2 g'(r)-2 r g(r)+12
   M\right)\right)+4 M r^3 \left(4 a^2-9 M r\right)}{a^2 \left(r^2 g'(r)+2 r g(r)-2 M\right)^2}}
\label{e525}
\end{aligned}
\end{equation}
where, $g'(r)$ denotes $d g(r)/dr$ and $g(r)=f(r)$ can be obtained in Eq. (\ref{e26}) in this work. In the Figure \ref{f3}, we give the shadow of the Kerr-like black hole and the Kerr black hole with the different rotation parameter $a$ for observers located at equatorial plane. Our results show that with the increasing of the rotation parameter $a$, the degree of distortion of the black hole shadow gradually increases. Besides, in Figure \ref{f4}, we also study the impact of different dark matter parameters on the black hole shadow, and compare them with the Kerr black hole ($\rho_e=0$). We find that the shape of the black hole shadow in a dark matter halo is similar to the Kerr black hole, which also shows that the dark matter has little impact on the shape of the black hole shadow. %$\Delta '(r)$ denotes $d\Delta (r)/dr$ and $\Delta (r)= r^2G(r)+a^2=r^2(g(r)-2M/r)+a^2$.
% Figure environment removed
%%%%%%%%%%%%%%%%%%%%%
% Figure environment removed

%\newpage
\section{Conclusions and Discussions}\label{s6}
\indent In this paper, using the Einasto profile in a dark matter halo, we obtained the metric of the Schwarzschild-like black hole located in the galactic center of M87. And then using the Newman-Janis algorithm, we extended our solution to the case of the Kerr-like black hole, and this solution was satisfied with the Einstein field equations. Besides, we also studied and analyzed the basic physical properties of these black holes in the galactic center of M87 and then compared some results of them with the Kerr black hole. We found that dark matter may have a positive impact on the physical properties of black hole. Our conclusions are as follows:\\
\indent (1) When the black hole is in a dark matter halo, our results show that the dark matter can effectively increase the event horizons of the black hole and expand the range of the ergosphere of the Kerr-like black hole.\\
\indent (2) In a dark matter halo, for the area of ergosphere of the Kerr-like black hole, that is the outer horizon and outer infinite redshift surface decrease with the increase of rotation parameter $a$, while the inner horizon and inner infinite redshift surface increase with the increase of rotation parameter $a$.\\
\indent (3) In a dark matter halo, the motion equation for neutral particles near the black hole is given by Eq. (\ref{e59}). In particular, for photons in the equatorial plane, the motion equation of the photons is given by Eq. (\ref{e510}).\\
\indent (4) In a dark matter halo, the geodesic equation for light-like particles is given by Eq. (\ref{e517}). By calculating the circular orbit of the photon, we find that the photon region $r_c$ is theoretically between the maximum and minimum values.\\
\indent (5) In a dark matter halo, for the shadow of the Kerr-like black hole, our results show that with the increasing of the rotation parameter $a$, the degree of distortion of the black hole shadow gradually increases. Besides, we also study the impact of different dark matter parameters on the black hole shadow, and compare them with the Kerr black hole (i.e., $\rho_e=0$). We find that the shape of the black hole shadow in a dark matter halo is similar to the Kerr black hole, which also shows that the dark matter has little impact on the shape of the black hole shadow.\\
\indent (6) In the future, these results of the black hole in a dark matter halo may be detected, which indirectly provides an effective method for detecting the existence of dark matter.\\
\indent At last, it would be interesting to study the quasinormal modes of a black hole in a dark matter halo. The quasinormal mode is the main mode after the merger of binary black holes, and it will carry important information of the black hole. Meanwhile, the black hole shadows and photon rings can also be used to detect the existence of the black hole. Therefore, in the next step of work, we will study the quasinormal mode of the black hole and the photon ring phenomenon gradually, expecting to provide some directions for the existence of the black hole in a dark matter halo.

\appendix
\section{Einstein field equations}\label{sa}
In this appendix, based on Eq. (\ref{e412}), we will prove that this metric is a solution to Einstein field equations. Firstly, we rewrite Eq. (\ref{e412}) in Kerr-like form,
\begin{equation}
\begin{aligned}
ds^2= & -(1-\frac{2R(r)r}{\Sigma }) dt^2 +\frac{\Sigma}{\Delta } dr^2 -\frac{4 R(r) r a \sin ^2 \theta }{\Sigma } dt d \phi + \Sigma d\theta^2 \\
&+\left ( (a^2+r^2)\sin^2 \theta +\frac{2 R(r) r a^2\sin^4\theta }{\Sigma }\right ) d\phi^2,
\label{ea1}
\end{aligned}
\end{equation}
where,
\begin{equation}
\begin{aligned}
2R(r)=2M + r-rg(r), \Delta =r^2-2R(r)r+a^2, \Sigma=r^2+a^2\cos^2 \theta.
\label{ea2}
\end{aligned}
\end{equation}
Considering that the Kerr-like black hole in a dark matter halo, it should satisfy the Einstein field equations, that is
\begin{equation}
\begin{aligned}
G_{\mu\nu}=R_{\mu\nu}-\frac{1}{2}g_{\mu\nu}R=8\pi T_{\mu\nu},
\label{ea3}
\end{aligned}
\end{equation}
Taking Eq. (\ref{ea1}) into the Einstein field equations, we can obtain the following non-zero Einstein tensors,
\begin{equation}
\begin{aligned}
&G_{tt}=  \frac{2(a^2r^2+r^4-a^4\cos^2\theta \sin^2\theta -2r^3R(r) )R'(r)}{\Sigma ^3} -\frac{a^2r\sin^2\theta R''(r)}{\Sigma ^2},      \\
&G_{rr} =   -\frac{2r^2R'(r)}{\Sigma \Delta },     \\
&G_{t\phi}=   -\frac{2a\sin^2 \theta ((r^a+a^2)(a^2\cos^2\theta-r^2)+2r^3R(r))R'(r) }{\Sigma ^3}     +\frac{ar(a^2+r^2)\sin^2\theta R''(r)}{\Sigma ^2}, \\
&G_{\theta \theta }=  -\frac{2a\cos^2\theta R'(r)}{\Sigma } -rR''(r)      \\
&G_{\phi \phi }= -\frac{a^2\sin^2\theta((a^2+r^2)(a^2+(a^2+2r^2)\cos 2\theta)+4r^3\sin^2 \theta R(r))R'(r)}{\Sigma ^3} \\
&-\frac{r\sin^2\theta (a^2+r^2)^2R''(r)}{\Sigma ^2}.
\label{ea4}
\end{aligned}
\end{equation}
On the other hand, %energy momentum tensor $T_{\mu\nu}$ of Eq. 5.1 can also be composed of four orthonormal bases and Einstein tensors.
in Eq. (\ref{ea3}), $T_{\mu\nu}$ is the energy momentum tensor and $T^{\mu\nu} =e^\mu_a e^\nu_b T^{ab}$, and $T^{ab}=diag[1/(-\rho_\epsilon, p_r, p_\theta,p_\phi)]$. Then the inverse of the energy-momentum tensor can be represented in the orthonormal basis as follows
\begin{equation}
\begin{aligned}
T^{\mu \nu}=-1/\rho_\epsilon e_t^\mu e_t^\nu + 1/p_r e_r^\mu e_r^\nu + 1/p_\theta e_\theta^\mu e_\theta^\nu + 1/p_\phi e_\phi^\mu e_\phi^\nu,
\label{ea5}
\end{aligned}
\end{equation}
where, the forms of the orthonormal bases are as follows,
\begin{equation}
\begin{aligned}
&e^\mu_t=\frac{1}{\sqrt{\Sigma \Delta} }(r^2+a^2,0,0,a),\\
&e^\mu_r=\sqrt{\frac{\Delta }{\Sigma}}(0,1,0,0),\\
&e^\mu_\theta=\frac{1}{\sqrt{\Sigma }}(0,0,1,0),\\
&e^\mu_\phi =-\frac{1}{\sqrt{\Sigma \sin^2\theta }}(a \sin^2 \theta,0,0,1),\\
\label{ea6}
\end{aligned}
\end{equation}
Therefore, taking Eqs. (\ref{ea3}), (\ref{ea4}) and (\ref{ea6}) into account, we can obtain the following results,
\begin{equation}
\begin{aligned}
&8\pi \rho_\epsilon =-e^\mu_t e^\nu_t G_{\mu\nu},\\
&8\pi p_r =e^\mu_r e^\nu_r G_{\mu\nu}=g^{rr}G_{rr},\\
&8\pi p_\theta  =e^\mu_\theta  e^\nu_\theta  G_{\mu\nu}=g^{\theta \theta }G_{\theta \theta },\\
&8\pi p_\phi =-e^\mu_\phi e^\nu_\phi G_{\mu\nu},\\
\label{ea7}
\end{aligned}
\end{equation}
Finally, we obtain expressions for the energy momentum tensors of a rotating black hole,
\begin{equation}
\begin{aligned}
\rho_\epsilon =-p_r=\frac{2r^2R'(r)}{8\pi\Sigma ^2}, p_\theta=p_\phi=p_r-\frac{rR''(r)+2R'(r)}{8\pi\Sigma },
\label{ea8}
\end{aligned}
\end{equation}
where, $R'(r)$ denotes $dR(r)/dr$. These results indicate that Eq. (\ref{e412}) is indeed a solution to Einstein field equations.



\begin{comment}
\section{Exact solution of deformed black hole in a dark matter halo}
In CDM model, the density profile of dark matter is better described by the general NFW profile from Navarro, Frenk and White,
\begin{equation}
\rho_{\text{gnfw}} (r)=\rho_g (\frac{r}{r_g})^{-\gamma }(1+\frac{r}{r_g})^{\gamma -3},
\label{e1}
\end{equation}
where, $\rho_g$ is the normalization density of dark matter, and $r_g$ is the characteristic radius of dark matter halo.
For CDM model,
\begin{equation}
M_{\text{gnfw}}=4\pi \int_{0}^{r}\rho_{\text{gnfw}} (r')r'^2dr'=-4\pi (-r)^{\gamma } r^{-\gamma } r_g^3 \rho _g \text{B}[-\frac{r}{r_g},3-\gamma ,-2+\gamma ]
\label{e3}
\end{equation}
where, $\text{B}[a,b,c]$ is the Euler Beta function $\text{B}_a(b,c)$. In this work, we use the parameter $0<\gamma<1.5$ from [1].

\begin{equation}
V_{\text{gnfw}}=\sqrt{\frac{M_\text{DM}}{r}}= \sqrt{4\pi (-r)^{-1+\gamma } r^{-\gamma } r_g^3 \rho _g \text{B}[-\frac{r}{r_g},3-\gamma ,-2+\gamma ]}
\end{equation}

\begin{equation}
f_{\text{gnfw}}(r)=\exp\left (8 \pi r^{-1-\gamma } r_g^3 \rho_g (\frac{r^3 (r+r_g)^{-2+\gamma }}{2r_g - \gamma r_g }+(-r)^\gamma  \text{B}[-\frac{r}{r_g},3-\gamma, -2+\gamma  ]) \right )
\end{equation}

\end{comment}


\acknowledgments
This research was funded by the National Natural Science Foundation of China (Grant No.12265007) and the Natural Science Special Research Foundation of Guizhou University (Grant No.X2020068).


%\newpage

\bibliographystyle{EPJC}
\bibliography{EPJCexample}
\end{document}
