

The emergence of large language models~\cite{zhao2023survey}, which have demonstrated remarkable progress in understanding human expression, is profoundly impacting the AI community. These models, equipped with vast amounts of data and large-scale neural networks, exhibit impressive capabilities in comprehending human language and generating text that closely resembles our own. Among these abilities are reasoning~\cite{huang2022towards}, few-shot learning~\cite{brown2020language}, and the incorporation of extensive world knowledge within pre-trained models~\cite{zhao2023survey}. This marks a significant breakthrough in the field of artificial intelligence, leading to a revolution in our interactions with machines. Consequently, large language models have become indispensable across various applications, ranging from natural language processing and machine translation to creative content generation and chatbot development.
The introduction of ChatGPT, in particular, has gained significant attention from the human community, prompting reflections on the transformative power of large language models and their potential to push the boundaries of what AI can achieve. This disruptive technology holds the promise of transforming how we interact with and leverage AI in countless domains, opening up new possibilities and opportunities for innovation. As these language models continue to advance and evolve, they are likely to shape the future of artificial intelligence, empowering us to explore uncharted territories and unlock even greater potential in human-machine collaboration.


Personalization, the art of tailoring experiences to individual preferences, stands as an essential and dynamic connection that bridges the gap between humans and machines. In today's technologically-driven world, personalization plays a pivotal role in enhancing user interactions and engagements with a diverse array of digital platforms and services. By adapting to individual preferences, personalization systems empower machines to cater to each user's unique needs, leading to more efficient and enjoyable interactions. Moreover, personalization goes beyond mere content recommendations; it encompasses various facets of user experiences, encompassing user interfaces, communication styles, and more. As artificial intelligence continues to advance, personalization becomes increasingly sophisticated in handling large volumes of interactions and diverse user intents. This calls for the development of more advanced techniques to tackle complex scenarios and provide even more enjoyable and satisfying experiences. The pursuit of improved personalization is driven by the desire to better understand users and cater to their ever-evolving needs. As technology evolves, personalization systems will likely continue to evolve, ultimately creating a future where human-machine interactions are seamlessly integrated into every aspect of our lives, offering personalized and tailored experiences that enrich our daily routines.


Large language models, with their deep and broad capabilities, have the potential to revolutionize personalization systems, transforming the way humans interact and expanding the scope of personalization. the interaction between humans and machines can no longer be simply classified as active and passive, just like traditional search engines and recommendation systems. However, these large language models go beyond simple information filtering and they offer a diverse array of additional functionalities. Specifically, user intent will be actively and comprehensively explored, allowing for more direct and seamless communication between users and systems through natural language. Unlike traditional technologies that relied on abstract and less interpretable ID-based information representation, large language models enable a more profound understanding of users' accurate demands and interests. This deeper comprehension paves the way for higher-quality personalized services, meeting users' needs and preferences in a more refined and effective manner. Moreover, the integration of various tools is greatly enhanced by the capabilities of large language models, significantly broadening the possibilities and scenarios for personalized systems. By transforming user requirements into plans, including understanding, generating, and executing them, users can access a diverse range of information and services. Importantly, users remain unaware of the intricate and complex transformations happening behind the scenes, as they experience a seamless end-to-end model. From this point, the potential of large language models in personal is largely unexplored. 


This paper addresses the challenges in personalization and explores the potential solutions using large language models. In the existing related work, LaMP~\cite{salemi2023lamp} introduces a novel benchmark for training and evaluating language models in producing personalized outputs for information retrieval systems. On the other hand, other related surveys~\cite{wu2023survey,lin2023can,fan2023recommender} focus mainly on traditional personalization techniques, such as recommender systems. From the perspective of learning mechanisms, LLM4Rec~\cite{wu2023survey} delves into both Discriminative LLM for Recommendation and Generative LLM for Recommendation. Regarding the adaptation of LLM for recommender systems in terms of 'Where' and 'How', Li et al~\cite{lin2023can} concentrate on the overall pipeline in industrial recommender phases. Fan et al~\cite{fan2023recommender}, on the other hand, conduct a review with a focus on pre-training, fine-tuning, and prompting approaches. While these works discuss pre-trained language models like Bert and GPT for ease of analysis, they dedicate limited attention to the emergent capabilities of large language models. This paper aims to fill this gap by examining the unique and powerful abilities of large language models in the context of personalization, and further expand the scope of personalization with tools.


The remaining of this survey is organized as follows: we review the personalization and large language models in Section 2 to overview the development and challenges. Then we carefully discuss the potential actors of large language models for personalization from Section 3, following the simple utilization of emergent capabilities and the complex integration with other tools. We also discuss the potential challenges when large language models are adapted for personalization.