Personalization, a nuanced art that tailors experiences to the unique preferences and needs of individual users, has become a cornerstone of modern artificial intelligence. In this section, we explore the captivating world of personalized techniques and their profound impact on user interactions with AI systems. We will delve into three key aspects of personalization: recommender systems, personalized assistance, and personalized search. These techniques not only enhance user satisfaction but also exemplify the evolution of AI, where machines seamlessly integrate with our lives, understanding us on a profound level. By tailoring recommendations, providing customized assistance, and delivering personalized search results, AI systems have the potential to create a truly immersive and individualized user experience.

\subsubsection{Recommender Systems}
Recommender systems play a pivotal role in personalization, revolutionizing the way users discover and engage with content. These systems aim to predict and suggest items of interest to individual users, such as movies, products, or articles, based on their historical interactions and preferences.

Regarding the development of recommender systems, they have evolved significantly over the years, with collaborative filtering~\cite{resnick1994grouplens,pan2008one} being one of the earliest and most influential approaches. Collaborative filtering relies on user-item interaction data to identify patterns and make recommendations based on users with similar preferences. Traditional solutions, such as matrix factorization~\cite{koren2009matrix} and user/item-based approaches~\cite{wang2006unifying}, extract potentially interesting items based on the idea that users who have shown similar preferences in the past are likely to have similar preferences in the future. While effective, collaborative filtering has limitations, such as the "cold start" problem for new users and items.
To address these limitations, content-based filtering~\cite{pazzani2007content} emerged, which considers the content of items to make recommendations. It leverages the features and attributes of items to find similarities and make personalized suggestions. These features can be grouped into user-side information, such as user profiles, item-side information~\cite{wang2011collaborative,zhang2016collaborative}, such as item brands and item categories, and interaction-based information~\cite{liu2019nrpa}, such as reviews and comments. 
However, content-based filtering may struggle to capture complex user preferences and discover diverse recommendations restricted by the limited feature representations.

In recent years, deep learning has gained significant attention in the field of recommender systems due to its ability to model complex patterns and interactions in user-item data~\cite{wang2015collaborative}. Deep learning-based methods have shown promising results in capturing sequential, temporal, and contextual information, as well as extracting meaningful representations from large-scale data. With the introduction of deep networks, high-order interactions between features of users and items are well captured to extract user interest. Deep learning-based methods offer approaches to capture high-order interactions by employing techniques like attention mechanisms~\cite{zhou2018deep,zhou2019deep} and graph based networks~\cite{wang2019neural} to mining complex relationships between user and item. These methods have been shown to enhance recommendation performance by considering higher-order dependencies and inter-item relationships. Another area of deep learning-based recommender systems is sequential recommenders, specifically designed to handle sequential user-item interactions, such as user behavior sequences over time. Self-Attentions~\cite{kang2018self} and Gated Recurrent Units (GRUs)~\cite{hidasi2015session} are popular choices for modeling sequential data in recommender systems. These models excel in capturing temporal dependencies and context, making them well-suited for tasks like next-item recommendation and session-based recommendation. Sequential-based models can take into account the order in which items are interacted with and learn patterns of user behavior that evolve over time. Furthermore, the rise of language models like BERT has further advanced recommender systems by enabling a better understanding of both natural language features and user sequential behaviors~\cite{sun2019bert4rec}. These language models can capture deep semantic representations and world knowledge, enriching the recommendation process and facilitating more personalized and context-aware recommendations.
Overall, the application of deep learning techniques in recommender systems has opened new avenues for research and innovation, promising to revolutionize the field of personalized recommendations and enhance user experiences.


\subsubsection{Personalized Assistance}
Personalization Assistance refers to the use of artificial intelligence and machine learning techniques to tailor and customize experiences, products, or content based on individual preferences, behavior, and characteristics of users. By analyzing individual preferences, behaviors, and characteristics, it creates a personalized ecosystem that enhances user engagement and satisfaction.
In contrast to traditional recommender systems, which rely on predicting user interests passively, personalized assistance takes a more proactive approach. It ventures into the realm of predicting users' next intentions or actions by utilizing contextual information, such as historical instructions and speech signals. This deeper level of understanding enables the system to cater to users' needs in a more anticipatory and intuitive manner.
At the core of this capability lies the incorporation of cutting-edge technologies like natural language processing (NLP) and computer vision. These advanced tools empower the system to recognize and interpret user intentions, whether conveyed through spoken or written language, or even visual cues.
Moreover, the potential of personalized assistance extends beyond static recommendations to dynamic and context-aware interactions. As the system becomes more familiar with a user's preferences and patterns, it adapts and refines its recommendations in real-time, keeping pace with the ever-changing needs and preferences of the user.

Conversational Recommender Systems mark a remarkable stride forward in the realm of personalized assistance. By engaging users in interactive conversations, these systems delve deeper into their preferences and fine-tune their recommendations accordingly. Leveraging the power of natural language understanding, these conversational recommenders adeptly interpret user queries and responses, culminating in a seamless and engaging user experience.
Notable instances of personalized assistance products, such as Siri and Microsoft Cortana, have already proven their effectiveness on mobile devices. Additionally, the integration of large language models like ChatGPT further elevates the capabilities of conversational recommenders, promising even more enhanced user experiences.
As this technology continues to progress, we can anticipate its growing significance across diverse industries, including healthcare, education, finance, and entertainment. 
While the growth of conversational recommenders and personalized assistance promises immense benefits, it is imperative to develop these products responsibly. Upholding user privacy and ensuring transparent data handling practices are essential to maintain user trust and safeguard sensitive information.
