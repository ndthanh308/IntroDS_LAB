% Rewritten by Jin Chen
With the development of large language models, there is an observation that LLMs exhibit reasoning abilities~\cite{huang2022towards,wei2022emergent} when they are sufficiently large, which is fundamental for human intelligence for decision-making and problem-solving. By providing the models with the `chain of thoughts'~\cite{wei2022chain}, such as prompting with \textit{'let us think about it step by step'}, the large language models exhibit emergent abilities for reasoning and can arrive at conclusions or judgments according to the evidence or logics. Accordingly, for recommender systems, large language models are capable of reasoning to help user interest mining, thus improving performance. 
 
\subsection{Making Direct Recommendations}

\begin{table*}[]
% 表格换行使用
\newcommand{\tabincell}[2]{
\begin{tabular}{@{}#1@{}}#2\end{tabular}
}

    \caption{Zero/few-shot learners of LLMs for RS}
    \label{tab:zero_show_recs}
    \resizebox{1.\textwidth}{!}{
\begin{tabular}{c|c|c|c|c|c|c}
\toprule
Approach                                                                & LLM backbone                                                                & Task                           & Metric                    & Datasets    & ICL & COT                                                      \\ \midrule
\cite{liu2023chatgpt}    & gpt-3.5-turbo                                   & \tabincell{c}{rating prediction \\ sequential recommendation \\  direct recommendation \\   explanation generation \\ review summarization }         & \tabincell{c}{RMSE,MAE \\HR,NDCG \\ HR,NDCG \\     BLUE4,ROUGE,Human Eval\\BLUE4,ROUGE,Human Eval }             & Amazon Beauty   & $\checkmark$ &                                  \\ \midrule
\cite{dai2023uncovering} & \tabincell{c}{text-davinci-002\\text-davinci-003\\gpt-3.5-turbo}  & \tabincell{c}{point-wise\\pair-wise\\list-wise}                     & NDCG,MRR & \tabincell{c}{MovieLens-1M\\Amazon-Book\\Amazon-Music\\MIND-small} & $\checkmark$ & \\  \midrule
\cite{kang2023llms}            & \tabincell{c}{Flan-U-PALM\\gpt-3.5-turbo \\text-davinci-003} & \tabincell{c}{rating prediction\\ranking prediction}              & \tabincell{c}{RMSE,MAE \\ROC-AUC}                  & \tabincell{c}{MovieLens-1M\\Amazon-Books}                 & $\checkmark$ & \\ \midrule
\cite{wang2023zero}                              & text-davinci-003                                                     & reranking& NDCG,HR                   & MovieLens 100K  & $\checkmark$&$\checkmark$                                                  \\ \midrule
\cite{hou2023large}                              & gpt-3.5-turbo                                                               & reranking& NDCG                      & \tabincell{c}{MovieLens-1M\\Amazon-Games}   & $\checkmark$ &   \\ \midrule
~\cite{li2023preliminary} & gpt-3.5-turbo & reranking & Precision & MIND & $\checkmark$ & \\ \bottomrule                                   
\end{tabular}
}
\end{table*}



In-context learning~\cite{dong2022survey,dai2022can,min2022rethinking,levy2022diverse,xieexplanation,olsson2022context,akyurek2022learning} is one of the emergent abilities of LLMs that differentiate LLMs from previous pre-trained language models, where, given a natural language instruction and task demonstrations, LLMs would generate the output by completing the word sequence without training or tuning~\cite{brown2020language}. As for in-context learning, the prompt follows by the task instruction and/or the several input-output pairs to demonstrate the task and a test input is added to require the LLM to make predictions. The input-output pair is called a \textit{shot}. This emergent ability enables prediction on new cases without tuning unlike previous machine learning. 


% Figure environment removed

In the realm of recommender systems, numerous studies have explored the performance of zero-shot/few-shot learning using large language models, covering the common recommendation tasks such as rating prediction, and ranking prediction. These studies evaluate the ability of language models to provide recommendations without explicit tuning, as summarized in Table~\ref{tab:zero_show_recs}, where all methods adopt in-context learning for direct recommenders. The general process can be attached in Figure~\ref{fig:zero_few_shot}. Accordingly, we have the following findings:
\begin{itemize}
    \item The aforementioned studies primarily focused on evaluating zero-shot/few-shot recommenders using open-domain datasets, predominantly in domains such as movies and books. Large language models are trained on extensive open-domain datasets, enabling them to possess a significant amount of common-sense knowledge, including information about well-known movies. However, when it comes to private domain data, such as e-commerce products or specific locations, the ability of zero-shot recommenders lacks of validation, which is expected to be challenging.
    \item Current testing methods necessitate the integration of additional modules to validate the performance of zero-shot recommenders for specific tasks. In particular, for ranking tasks that involve providing a list of items in order of preference, a candidate generation module is employed to narrow down the pool of items~\cite{wang2023zero} and \cite{hou2023large}. Generative-based models like gpt-3.5-turbo generate results in a generative manner rather than relying on recall from existing memories, thus requiring additional modules to implement ID-based item recommendations.
    \item From the perspective of recommendation performance, zero-shot recommenders exhibit some capabilities and few-shot learners perform better than zero-shot recommenders. However, there still exists a substantial gap when compared to traditional recommendation models, particularly fine-tuned large language models designed specifically for recommenders, such as P5\cite{geng2022recommendation} and M6-Rec~\cite{cui2022m6}. This highlights that large language models do not possess a significant advantage in personalized modeling.
\end{itemize}

Another important emergent ability is the \textit{`step by step'} reasoning, where LLMs can solve complex tasks by utilizing prompts including previous intermediate reasoning steps, called the `chain of thoughts' strategy~\cite{wei2022chain}. Wang and Lim~\cite{wang2023zero} design a three-step prompt, namely NIR, to capture user preferences, extract the most representative movies and rerank the items after item filtering. Such a multi-step reasoning strategy significantly improves recommendation performance. 


\subsection{Reasoning for Automated Selection}
% Written by Xingmei
Automated Machine Learning (AutoML) is widely applied in recommender systems to eliminate the costly manual setup with trials and errors. The search space in recommender systems can be categorized in (1) Embedding size (2) Feature (3) Feature interaction (4) Model architecture. Embedding size search, such as~\cite{liu2021learnable,liu2020automated,deng2021deeplight,ginart2021mixed} seeks for appropriate embedding size for each feature to avoid resources overconsumption. Searching for features consisting of raw feature search\cite{wang2022autofield,lin2022adafs} and synthetic feature search\cite{tsang2020feature,yuanfei2019autocross}, which selects a subset from the set of original or cross features to maintain informative features to reduce both computation and space cost. Feature interaction search, such as~\cite{liu2020autofis,liu2020autogroup, chen2019bayesian,xie2021fives,su2021detecting}, automatically filters out feature interactions that are not helpful. Model architecture search, like~\cite{song2020autoctr,zhao2021ameir,wei2021autoias,cheng2022nasr}, expands the search space to the integral architectures. The search strategy shifts from the discrete reinforcement learning process, which iteratively samples architectures for training and is time-consuming, into the differentiable searching, which adaptively selects architectures within one-shot learning to circumvent the computational burden, for more efficient convergence. The evaluation for each sampled architecture then acts as the signal to adjust the selections. That is, there is a decision maker who memorizes the prior results of previous architecture choices and analyzes the prior results to give the next recommended choice. 



The emergent LLMs actually have excellent memorization and reasoning capability that would work for automated learning. Several works have attempted to validate the potential of automated machine learning with LLMs. Preliminarily, GPT-NAS~\cite{yu2023gptnas} takes advantage of generative capability of LLMs. The architecture of networks are formulated into sequential characters, and thus the generation of network architectures can be easily achieved through the generative pre-training models. NAS-Bench-101~\cite{ying2019bench} is utilized for pre-training and the state-of-the-art results are used for fine-tuning. The generative pre-training models produce reasonable architectures, which would reduce the search space for later genetic algorithms for searching optimal architectures. The relatively advanced reasoning ability is further evaluated in GENIUS~\cite{zheng2023cangpt}, where GPT-4 is employed as a black-box agent to generate potential better-performing architectures according to previous trials including tried architectures with their evaluation performance. According to the results, GPT-4 can generate good architecture networks, showing the potential for more complicated tasks. Yet it is too difficult for LLMs to directly make decisions on challenging technical problems only by prompting. To balance efficiency and interpretability, one approach is to integrate the LLMs into certain search strategies, where the genetic algorithm guides the search process and LLMs generate the candidate crossovers. LLMatic~\cite{nasir2023llmatic} and EvoPrompting~\cite{chen2023evoprompting} use code-LLMs as mutation and crossover operators for a genetic NAS algorithm. During evolution, each generation has a certain probability of deciding whether to perform crossover or mutation to produce new offspring. Crossover and mutation are generated by prompting LLMs. Such a solution integrates LLM into the genetic search algorithm, which would achieve better performances than direct reasoning.
% over the whole space. 



The research mentioned above brings valuable insights to the field of automated learning in recommender systems. However, there are several challenges that need to be addressed. Firstly, the search space in recommender systems is considerably more complex, encompassing diverse types of search space and facing significant volume issues. This complexity poses a challenge in effectively exploring and optimizing the search space.
Secondly, compared to the common architecture search in other domains, recommender systems lack a strong foundation of knowledge regarding the informative components within the search space, especially the effective high-order feature interactions. Unlike well-established network structures in other areas, recommender systems operate in various domains and scenarios, resulting in diverse and domain-specific components. 
Addressing these challenges and advancing the understanding of the search space and informative components in recommender systems will pave the way for significant improvements in automated learning approaches.



