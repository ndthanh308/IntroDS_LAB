%\documentclass[lettersize,journal,onecolumn]{IEEEtran}
\documentclass[lettersize,journal]{IEEEtran}
\usepackage{amsmath,amsfonts}
\usepackage{algorithmic}
\usepackage{algorithm}
\usepackage{array}
\usepackage[caption=false,font=footnotesize]{subfig}
\usepackage{textcomp}
\usepackage{stfloats}
\usepackage{url}
\usepackage{verbatim}
\usepackage{graphicx}
\usepackage{cite}
\usepackage{xcolor}
\usepackage{color}
\usepackage{multirow}

%\usepackage{setspace}
%\doublespacing

\newcolumntype{P}[1]{>{\centering\arraybackslash}p{#1}}
\definecolor{green(html/cssgreen)}{rgb}{0.0, 0.5, 0.0}

\newcommand{\rv}[1]{\textcolor{red}{#1}}
\newcommand{\ov}[1]{\textcolor{orange}{#1}}
\newcommand{\bv}[1]{\textcolor{blue}{#1}}
\newcommand{\gv}[1]{\textcolor{green(html/cssgreen)}{#1}}

\hyphenation{op-tical net-works semi-conduc-tor IEEE-Xplore}
% updated with editorial comments 8/9/2021

\begin{document}

%\title{Exploring Large-Scale Data in \\ Fully-Unsupervised Re-Identification Setups}

\title{Large-scale Fully-Unsupervised Re-Identification}

%\author{IEEE Publication Technology,~\IEEEmembership{Staff,~IEEE,}

\author{Gabriel~Bertocco,
        Fernanda~Andal\'{o},~\IEEEmembership{Member,~IEEE,}
        Terrance~E.~Boult,~\IEEEmembership{Fellow~Member,~IEEE,} and~Anderson~Rocha,~\IEEEmembership{Senior~Member,~IEEE}% <-this % stops a space
\thanks{Gabriel Bertocco is a Ph.D. student at the Artificial Intelligence Lab. (\textbf{Recod.ai}), Institute of Computing, University of Campinas, Brazil}% <-this % stops a space
\thanks{Fernanda~Andal\'{o} is a researcher associated with the Artificial Intelligence Lab. (\textbf{Recod.ai}), Institute of Computing, University of Campinas, Brazil.}% <-this % stops a space
\thanks{Terrance~Boult is a professor at University of Colorado Colorado Springs (UCCS) and Chair of the Vision And Security Technology Lab (UCCS VAST LAB).}% <-this %
\thanks{Anderson Rocha is a Full-Professor for Artificial Intelligence and Digital Forensics and Chair of the Artificial Intelligence Lab. (\textbf{Recod.ai}) at the Institute of Computing, University of Campinas, Brazil. 
}}% <-this % stops a space

% The paper headers
\markboth{Paper submitted for a possible publication in an IEEE Transactions}%
{Bertocco \MakeLowercase{\textit{et al.}}: Large-scale Fully-Unsupervised Re-Identification}

%\IEEEpubid{0000--0000/00\$00.00~\copyright~2021 IEEE}
% Remember, if you use this you must call \IEEEpubidadjcol in the second
% column for its text to clear the IEEEpubid mark.

\maketitle

\begin{abstract}
Fully-unsupervised Person and Vehicle Re-Identification have received increasing attention due to their broad applicability in areas such as surveillance, forensics, event understanding, and smart cities, without requiring any manual annotation. However, most of the prior art has been evaluated in datasets that have just a couple thousand samples. Such small-data setups often allow the use of costly techniques in terms of time and memory footprints, such as Re-Ranking, to improve clustering results. Moreover, some previous work even pre-selects the best clustering hyper-parameters for each dataset, which is unrealistic in a large-scale fully-unsupervised scenario. In this context, this work tackles a more realistic scenario and proposes two strategies to learn from large-scale unlabeled data. The first strategy performs a local neighborhood sampling to reduce the dataset size in each iteration without violating neighborhood relationships. A second strategy leverages a novel Re-Ranking technique, which has a lower time upper bound complexity and reduces the memory complexity from $\mathcal{O}(n^{2})$ to $\mathcal{O}(kn)$ with $k \ll n$. To avoid the need for pre-selection of specific hyper-parameter values for the clustering algorithm, we also present a novel scheduling algorithm that adjusts the density parameter during training, to leverage the diversity of samples and keep the learning robust to noisy labeling. Finally, due to the complementary knowledge learned by different models in an ensemble, we also introduce a co-training strategy that relies upon the permutation of predicted pseudo-labels, among the backbones, with no need for any hyper-parameters or weighting optimization. The proposed methodology outperforms the state-of-the-art methods in well-known benchmarks and in the challenging large-scale Veri-Wild dataset, with a faster and memory-efficient Re-Ranking strategy, and a large-scale, noisy-robust, and ensemble-based learning approach.    
\end{abstract}

\begin{IEEEkeywords}
Unsupervised Person and Vehicle Re-Identification, Re-Ranking, noisy-robust learning, co-training, ensemble-based learning  
\end{IEEEkeywords}

\section{Introduction}
\label{sec:introduction}

\IEEEPARstart{C}{lustering}-based self-supervised learning has attracted attention in the last years due to its capability of learning from fully-unlabeled datasets by performing clustering and finetuning with pseudo-labels. However, most works, like SWaV~\cite{caron2020unsupervised} and DeepCluster~\cite{caron2018deep}, employ a clustering-based strategy as a pretext task and further finetune the model in the labeled (or partially labeled) downstream task such as image classification or semantic segmentation. In this context, those models leverage the self-supervised solution as an intermediate state before performing labeled fine-tuning, instead of directly applying them to some task. Other works, such as ASPC-DA~\cite{guo2019adaptive}, employ the clustering-based solution directly to the target task without requiring a pretext task and, consequently, no labeled data. However, most of them have focused on simpler datasets with a lower number of classes, such as MNIST~\cite{lecun1998gradient} and FASHION~\cite{xiao2017fashion}. %, and USPS\footnote{http://www.cad.zju.edu.cn/home/dengcai/Data/MLData.html}. 

Conversely, we leverage clustering-based self-supervised learning as the final solution for deployment, but in more complex scenarios, with challenging intra-class variations and a greater number of classes in large-scale scenarios. Specifically, we focus on fully-unsupervised Person and Vehicle Re-Identification. 
Since it is a fully-unsupervised problem, we do not consider any task-related information or labeling to design our solution, opening the path for further applications beyond re-identification tasks.

%\IEEEPARstart{F}{ully}-unsupervised Person and Vehicle Re-
The fully-unsupervised Person and Vehicle Re-
Identification tasks have a myriad of applications, such as in surveillance, forensics, event understanding, intelligence monitoring, and smart cities, without requiring any label or side information for training. Moreover, with the growing amount of data recorded daily from a range of heterogeneous sources, from surveillance cameras to aerial and mobile devices (e.g., UAVs), the fully-unsupervised re-identification task can face applications with hundreds of thousands of unlabeled samples. The development of fully-unsupervised and scalable re-identification methods is then paramount to solving these more demanding and realistic scenarios.

One obstacle to the deployment of re-identification state-of-art methods is that most of them have reached top-tier performance in small datasets. The largest well-known Person Re-Identification benchmarks \texttt{Market1501}~\cite{zheng2015scalable}, \texttt{DukeMTMC-ReID}~\cite{ristani2016performance}\footnote{\texttt{DukeMTMC-ReID} has been discontinued, and it must not be used for evaluation and benchmarking anymore. For this reason and following the recent literature, we \textbf{do not} use it for evaluation. More details in \textit{https://www.dukechronicle.com/article/2019/06/duke-university-facial-
recognition-data-set-study-surveillance-video-students-china-uyghur}}, and \texttt{MSMT17}~\cite{wei2018person} have 12,936, 16,522, and 32,621 training images, respectively. For Vehicle Re-Identification, the \texttt{Veri}~\cite{liu2016large} dataset has 37,778 training images. Since the training sets are rather small, researchers could employ effective but costly techniques to achieve top-tier performance, such as Re-Ranking~\cite{zhong2017re} and Co-Training~\cite{ge2020mutual, zhai2020multiple, zhai2022population}. The Re-Ranking technique~\cite{zhong2017re} has memory complexity of $\mathcal{O}(n^2)$ (where $n$ is the training set size) and time complexity of $\mathcal{O}(n^3 + nk^3)$ (where $k$ is the number of reciprocal neighbors, see Section~\ref{subsec:local_reranking_clustering} for further details). And Co-Training usually involves cross-supervision where the confidence level of the samples from one model is used to weigh the loss functions from the other models in the ensemble, and all loss functions are optimized at once with many hyper-parameters to control the contribution for each term. Furthermore, these hyper-parameter values can be hard to tune, mainly in large-scale fully-unsupervised datasets without a validation set. 

Aiming to deploy scalable and affordable solutions for large-scale Unsupervised Person/Vehicle Re-Identification, we design a novel pipeline to alleviate the discussed challenges. We propose a novel Re-Ranking method that leverages the previously proposed k-Reciprocal Encoding~\cite{zhong2017re} but considers just a local neighborhood to calculate the Jaccard distance, without relying on their set expansion nor in local query expansion. This allows us to reduce the time complexity and keep the memory complexity in $\mathcal{O}(kn)$ where ($k \ll n$). 

Moreover, we also propose a sampling method based on the local neighborhood for a randomly chosen point. We have different local neighborhoods in each epoch, which effectively reduces the training set size without violating the neighborhood properties of the selected point. This preserves the hard-positive and hard-negative samples for effective training and, at the same time, reduces memory and time complexities. 

Since co-training has shown impressive performance in dealing with noisy data, we also propose a simple co-training strategy based on the co-training theory but not requiring any human supervision or hyper-parameter tuning. During training, we generate pseudo-labels for unlabeled training data using different convolutional neural network backbones. Each backbone generates its own feature space and then its own set of pseudo-labels. We propose switching the pseudo-labels among the backbones; thus, one backbone supervises the other through pseudo-label predictions. % To verify our performances, we evaluate our methods not just in the well-known benchmarks but also in the large-scale VehicleID~\cite{liu2016deep} and Veri-Wild~\cite{lou2019veri} vehicle datasets which have 113,346 and 277,797 training images which are $3\times$ and $7.35\times$ respectively bigger than the Veri training set (the biggest one among the well-known benchmarks). 

Beyond the discussed large-scale challenges, a fundamental aspect has been overlooked: the choice of the clustering hyper-parameter. Most methods employ DBSCAN~\cite{ester1996density}, which is controlled by the density parameter $\varepsilon$, and some works consider an optimal value for each dataset. For instance, AdaMG~\cite{peng2023adaptive} sets $\varepsilon = 0.5$ for \texttt{Market1501} and $\varepsilon = 0.7$ for \texttt{MSMT17}. When they fix the same value for both datasets, the performance drops. The recent ISE~\cite{zhang2022implicit} and RTMem~\cite{yin2023real} also set different hyper-parameter per dataset to achieve their best performance. However, since there is no validation set in a fully-unsupervised scenario, we argue it is unrealistic to select a specific value per dataset.

To alleviate the burden of selecting an optimal hyper-parameter per dataset, we analyze the clustering problem from the perspective of noisy-labeling-robust learning. During training, feature vectors are extracted, clustered, and pseudo-labels are assigned to them, which are then used for finetuning. In the first training epochs, the backbones have little knowledge about the dataset, so it is expected that the features are not too discriminative.
Therefore, there is more noisy pseudo labeling in the first iterations, which can require a tighter density parameter $\varepsilon$. As the model gets more robust during training, generating better features, it allows the loosening of the density parameter to include more hard-positive samples in the clusters. However, if we keep it loose until the end of the training, it might include too many non-matching samples, so we should decrease $\varepsilon$ after the feature space is reasonably tuned to allow final feature learning. Finally, if variations harden defining a stopping criterium, it might be interesting to hold the parameter constant at the end. This motivates us to design a new scheduling scheme for $\varepsilon$, which is used for all datasets without any hyperparameter tuning. We show that it reaches state-of-the-art performance in all evaluation scenarios, even outperforming the ones that select a dataset-specific hyper-parameter for optimal performance.
% where it starts at a low value, reaches a maximum in the middle of training, and decreases back until it reaches a plateau. We keep the same scheduling for all datasets without any hyperparameter tuning. We also show that it reaches state-of-the-art performance in all evaluation scenarios, even surpassing the ones that select a dataset-specific hyper-parameter for optimal performance.

To verify the performance of the proposed methods, we evaluate them not just in the well-known benchmarks but also in the large-scale \texttt{VehicleID}~\cite{liu2016deep} and \texttt{Veri-Wild}~\cite{lou2019veri} datasets, which are % have 113,346 and 277,797 training images which are 
$3\times$ and $7.35\times$ larger, respectively, than the Veri dataset (the largest among well-known benchmarks).

The key contributions of our work are:
\begin{itemize}
    \item A neighborhood-based sampling method to decrease the dataset size in each epoch. By preserving the neighborhood, it is able to keep hard-positive and hard-negative samples for model learning.  
    \item A Re-Ranking method that considers just the top-$k$ nearest neighbors and does not need set expansion nor local query expansion as previous methods. In this way, we effectively reduce the time and memory complexities. 
    \item A density parameter scheduling to deal with noisy data during training and, at the same time, bring diverse samples together. We keep the scheduling scheme for all datasets, avoiding hyper-parameter tuning.
    \item A co-training method where we switch the predicted labels among the involved backbones. This allows us to consider co-training without human intervention and parameter tuning. 
    \item We consider the large-scale \texttt{VehicleID} and \texttt{Veri-Wild datasets}, infrequently used in the prior art, to verify the model's performance in truly large-scale scenarios.
    
\end{itemize}

\section{Related Work}
\label{sec:related_work}

\subsection{Re-Ranking-based approaches}

Most Re-Ranking techniques are designed to address rank retrieval during the evaluation phase. Given a query image and the set of gallery images, the initial ranking list is obtained.
%which is the sorted list of the gallery images based on the similarity (or distance) to the query image. 
Usually, methods take this initial list and rank the samples again based on some strategy to enhance the retrieval. 

In~\cite{garcia2017discriminant}, the authors define the content and context sets. Given a probe image, they calculate its ranking list with the gallery images. The set with the closest images to the probe is called the content set, which is used with the original ranking list to create the context set and improve the ranking performance. 
%, and, based on the dissimilarity values, they divide the gallery images into three disjoint sets. The set with the closest images to the probe is called the content set, which is used along with the original probe ranking list to create the context set and improve the ranking performance.
In~\cite{bai2017scalable}, the authors use a graph-based on-the-fly affinity learning considering the labeled training, gallery, and probe sets to refine the ranking of the probe to the gallery set. In~\cite{sarfraz2018pose}, the authors employ the probe-to-neighbor distance and neighbor-to-neighbor distance to refine the ranking list by considering an expanded neighborhood from each gallery sample retrieved in the top matches to the probe. 
%They also employed the ranking-list distance based on~\cite{jarvis1973clustering} to avoid costly distance computation among different ranking lists. 
The authors in~\cite{luo2019spectral} propose the Spectral Feature Transformation (SFT), where they optimize the model to generate feature representation that optimizes the Min-Cut problem considering the labeled samples at batch level. During the evaluation, given a query sample, they perform SFT in the top-ranked gallery features to improve the retrieval performance.

The most well-known Re-Ranking method is the k-Reciprocal Encoding~\cite{zhong2017re} which has been extensively used for unsupervised re-identification. It improves the feature distances during training and generates clusters with high diversity and true-positive rate. The $k$-Reciprocal nearest neighbors are calculated for each training sample and respective expanded set. Then Jaccard distances are computed between training points considering each expanded set, and these distances are averaged in the Local Query Expansion step. %This process effectively increases clustering performance by including diverse examples in the clusters. 
However, it is time- and memory-consuming~\cite{jarvis1973clustering, luo2019spectral}. 

%They calculate the $k$-Reciprocal nearest neighbors $R(p,k)$ for each sample $p$ in the training set, and the set $R(q,k/2)$ where $q \in R(p,k)$. If $|R(p,k) \cap R(q,k/2)| \geq \frac{2}{3}|R(q,k/2)|$, they generate the expansion set $R*(p,k) = R(p,k) \cup R(q,k/2)$ to include the hard-positive samples. Then they calculate the Jaccard distances between training points considering each $R*(p,k)$ and average the distances in the Local Query Expansion step. This process effectively increases the clustering performance by including diverse examples on a cluster. However, it is time and memory-consuming~\cite{jarvis1973clustering, luo2019spectral}. 

%Since most of the Unsupervised Re-Identification works have employed their methods in relatively small datasets, such as MSMT17 and Veri, the time and memory costs have been affordable so far. However, as the number of training samples grows to more than 100K samples (e.g., VehicleID and Veri-Wild),
%the cost of employing Re-Ranking~\cite{zhong2017re} increases, making the training slower and with a high memory footprint.

In this context, we propose Local Re-Ranking, which redesigns the neighborhood-based distance calculation to decrease time complexity and keep state-of-the-art performance.
%We keep or even improve the state-of-art performance in the well-known Person and Vehicle benchmarks as well as in the large-scale VehicleID and Veri-Wild datasets. %with a less complex Re-Ranking technique. 

\subsection{Noise-robust Feature Learning}
\label{sec:noise_robust_learning}
% The problem of learning under noisy labels has received increasing attention since the publication of one of its seminal works~\cite {angluin1988learning}. 
Large-scale datasets are prone to annotation error due to their size or to the complexity in identifying the positive samples even with human supervision~\cite{frenay2013classification}. This introduces noise in the learning process, hindering generalization.

Several works have tackled model learning with noisy labels~\cite{frenay2013classification, han2018co, han2020survey, song2022learning}. They usually estimate the Transition Matrix, which reflects the probability of the samples from one class being misclassified as other classes present in the dataset. With the estimated Transition Matrix, some works~\cite{sukhbaatar2014training} propose to change the probability distribution on the final softmax layer or to reweight the final loss function~\cite{patrini2017making} to alleviate the influence of noisy samples. Based on the memorization effect in which Deep Neural Networks first learn from clean samples and then from noisy samples~\cite{han2018co, yu2019does, wang2019co}, some works have proposed robust regularization through implicit or explicit regularization~\cite{goodfellow2014explaining, xia2020robust}
and co-training methods~\cite{han2018co, wang2019co}. The fundamental idea of co-training in noisy-label scenarios is to have two (or more) backbones and select the small-loss samples from one peer backbone to train the other. Usually, the same batch of images is fed to both peers, and just the top-$r$\% small-loss samples are kept from each one. Then one peer optimizes its weights with the small-loss samples from another. Rate $r$ is decreased along the training to keep fewer samples to train the peers as they start learning from noisy samples due to the memorization effect. %For further details, we refer to the surveys~\cite{frenay2013classification, han2020survey, song2022learning}.

Based on the memorization effect, we propose to control the tightness of the generated clusters by changing the density criteria in a novel manner. But instead of selecting small-loss samples as done previously, we consider the presence of noisy labeling in the feature space.

\subsection{Co-training for Person Re-Identification}
\label{sec:co_training_reid}
Despite the progress in noise-robust feature learning, most methods assume data %(despite being noisy) 
is annotated. Instead, we consider unlabeled data and propose a clustering-based solution to generate pseudo-labels for model fine-tuning. Since the backbones have not been pre-trained in any other ReID-related dataset (just in ImageNet), the features are naturally noisy, which results in noisy labels and imperfect clustering. 

Prior Unsupervised Person ReID (UPReID) art also faces the same problem, and some works apply co-training to deal with noisy labeling. MMT~\cite{ge2020mutual}, MEB-Net~\cite{zhai2020multiple}, and PEG~\cite{zhai2022population} share the same principle of multiple models learning from each other's hard and soft labels at the batch level to encourage knowledge decoupling and robustness to noise. %They vary in the number of peer networks and in the strategy to train the models. MMT has two peer networks, while MEB has three and a regularization authority term to give more confidence to the peer that better groups the data in the clustering stage. PEG has a pool of backbones available, which are selected in turn to create the committee of peer networks. They also propose to change their hyper-parameters to imitate mutation to leverage an evolutionary-based approach. However, 
These methods employ complex co-training strategies with a lot of terms and hyper-parameters in their loss functions or in the selection of the peer networks, which might be challenging to tune and deploy in large-scale unlabeled scenarios. 

We propose a simpler co-training strategy, by permuting the predicted labels among the peers instead of performing soft/hard cross-supervision. We take advantage of co-training, but with a hyper-parameter-free strategy that does not require any manual or grid-searching-based parameter selection. 

\subsection{Unsupervised Re-Identification}

To tackle unsupervised re-identification, some methods rely on pre-training using a source ReID dataset~\cite{% zhong2020learning, zou2020joint, 
chen2020enhancing, 9521886, ge2020mutual, zhai2020multiple} to acquire prior knowledge. Other methods, like IICS~\cite{xuan2021intra}, CAP~\cite{wang2021camera}, CASTOR~\cite{xu2022pseudo}, and PPSL~\cite{wu2022pseudo},  rely on other information, such as camera labels. Since our method is fully unsupervised, i.e., it \textbf{does not} rely on camera labels, we focus on methods operating in the same setup. We present in the Supplementary Material a comparison table to methods considering camera labels as side information. 

%IICS~\cite{xuan2021intra} leverages intra-camera training by dividing samples into sets according to their camera labels and clustering each one. A backbone is trained in a multi-task manner (one task per camera), and clustering is run for the whole dataset, grouping samples of the same identity seen from different cameras.
%CAP~\cite{wang2021camera} performs clustering by assigning pseudo-labels for each sample of the dataset. They obtain camera proxies on each cluster for intra- and inter-camera training. 
ICE~\cite{Chen_2021_ICCV} has two versions: camera-aware and camera-agnostic. The first considers camera proxies for each cluster. The second considers only a cluster proxy, which is a feature average regardless of camera labels. They use a proxy-based loss, and hard- and soft-instance losses.
%CASTOR~\cite{xu2022pseudo} performs pre- and post-clustering processing to refine the distance matrix and pseudo-labels. In the pre-clustering, they leverage camera feature distances to regularize identity feature distances and avoid the cross-camera impact. In post-clustering, they proposed a $k$-reciprocal nearest neighbors-based search for outlier reassignment. Liu~\textit{et. al.}~\cite{liu2022unsupervised} also proposed to calculate the similarity between different camera domains and regularize the distances to consider cross-camera variance. They proposed a combination of instance- and cluster-based feature memories for model optimization. 
%PPSL~\cite{wu2022pseudo} performs patch-based clustering where surrogate vectors are generated for each class based on the extracted patches, and images from different cameras and the respective gradients are used to perform clustering. Moreover, it also considers intra-similarity learning to explore different parts of the same image, and inter-similarity learning to align features of same-cluster images across cameras.   
SpCL~\cite{ge2020self} uses a self-paced strategy that introduces some metrics to measure cluster reliability: cluster independence and compactness. If both are higher than predefined thresholds, the cluster is kept within the feature space. RLCC~\cite{zhang2021refining} refines clusters by a consensus among iterations. Pseudo-labels are created by considering the ones generated in previous iterations, keeping the training stable.
CACL~\cite{li2022cluster} proposes to suppress the dominant colors in images, providing a more robust feature description, and a novel pseudo-label refinement method.
CCons~\cite{dai2022cluster} uses contrastive learning, with which they select the hardest cluster sample in the batch to update the cluster centroid.
ISE~\cite{zhang2022implicit} synthesizes novel feature examples from real ones to refine sample distribution, aiming to generate clusters with a higher true positive rate, as well as avoiding subdivision of the samples from the same class in different clusters. 
PPLR~\cite{cho2022part} employs a part-based model that creates feature spaces from different parts of the feature map. In each one, they calculate the nearest neighbors of the samples and propose a cross-agreement metric to refine the proposed pseudo-labels.
GRACL~\cite{zhang2022global} keeps two memory banks where one holds the sample, from each cluster, most dissimilar from other clusters, and the other holds the sample most dissimilar from the positive samples. This approach encourages compactness and separability.
AdaMG~\cite{peng2023adaptive} performs clustering with different density parameters to generate multiple pseudo-labels. Then a teacher-student model is trained considering each set of pseudo-label, and the cluster feature memory bank is updated based on sample reliability, alleviating noisy labeling impact.   

For Unsupervised Vehicle Re-Identification (UVReID), MLPL~\cite{he2022multi} adopts a multi-level feature description by extracting a global feature and four local features for clustering and feature learning. They also propose converging and promoting stages to learn from global and local features separately and jointly, based on the consistency score of the global- and local-based clustering results. 
%VR-PROUD~\cite{bashir2019vr} trains a side network to predict vehicle colors in each cluster. Since the same vehicle is supposed to have the same color across all cameras, they get the dominant color in each cluster and discard the vehicles with different colors assuming those images are false-positive samples. 
RLCC~\cite{zhang2021refining} and PPLR~\cite{cho2022part} are also evaluated under the UVReID scenario.

%Compared to such methods, our formulation herein does not rely on any camera labeling, requiring only the bounding boxes of the objects of interest (e.g., people, vehicles); therefore, it best fits this last category of methods. Moreover, 
Our method differs from the others by proposing a regime to adapt a noise-robust density parameter over time, alleviating the burden of choosing optimal hyper-parameters, and a co-training strategy to avoid error amplification by the backbones. 


\section{Proposed Method}
\label{sec:proposed_method}

In this section, we provide the rationale for each part of the proposed pipeline (Figure~\ref{fig:full_pipeline}) to tackle large-scale unlabeled Re-Identification under noisy labeling. We assume we have $m = 3$ backbones, but it can be extended to any $m \geq 2$.

%When a clustering algorithm is employed (DBSCAN~\cite{ester1996density}, K-Means~\cite{lloyd1982kmeans}), one of the main questions is: which are the most suitable values for the hyper-parameters to obtain meaningful clusters? This question is especially important since most of the clustering-and-finetuning algorithms rely on the generated pseudo-labels to perform models finetuning. In the case of DBSCAN (the clustering algorithm chosen for this work) there are two hyper-parameters: $minPts$ and $\varepsilon$. The first defines the minimum cluster size, and the second defines the neighborhood size of each point to define a dense region. 
%Prior works~\cite{} have reported the DBSCAN is more sensible to the density parameter ($\varepsilon$). 
%However, due to the lack of enough discrimination ability mainly in the early iterations the model might not be able to create representative clusters with high true positive rates and diverse examples for model training. Formally, let $X=\{x_{i}\}_{i=1}^{N}$ be an unlabeled dataset, and $f(.|\theta)$ a Deep Convolutional Neural Network (DCNN) backbone with parameters $\theta$. When we perform the feature extraction we have a feature vector set $F$ where $F_{i} = f(x_{i}|\theta)$, which are clustered. The resulting pseudo-labeling will rely in three factors: the mapping function (in our case is the backbone $f(.|\theta)$) that maps the input image to the feature space, the dataset $D$ used to initialize the respective weights of the mapping function, and the $\varepsilon$ to define the neighborhood size of the data points. Mathematically the pseudo-label set for $X$ can be written as $Y(f,D,\varepsilon) = \{y_{i}\}_{i=1}^{N}$. Given $M$ different mapping functions (DCNN backbones) with their weights initialized over the same dataset (different $f$, but same $D$ and $\varepsilon$) we can have different feature representations. 
%For this reason, a lot of different architectures along the last years have been proposed and trained over ImageNet and each one reports different performances on validation and test sets. Then each one will create different feature space for $X$ even if initialized over the same dataset. 
%Consequently, different pseudo-labels will be found even if we keep the same $\varepsilon$ parameter for clustering. In the same way, if we keep the same mapping function but change the dataset (same $f$ and same $\varepsilon$, but different $D$) where its weights have been initialized we will end up again with different feature representations resulting in different clustering results. Finally, for a mapping function with its weights initialized over a dataset, when we get the feature representation from $X$ but we vary the $\varepsilon$ we will have different labels for each samples (same $f$, same $D$ but different $\varepsilon$). In this sense, we propose to address each of the three variables that controls the pseudo-labels generation. To provide complement knowledge we leverage three backbones (mapping functions) in our method in order to each one generate different feature spaces. To well initialize their weights we firstly self-supervised pre-train hem over $X$ to get an initial knowledge related to the domain problem. And finally we propose a novel strategy to assign different $\varepsilon$ parameter along each training iteration to address the diversity/noise bias that we are going to introduce and be able to learn from noisy data. 

% Figure environment removed

\subsection{Self-supervised Initialization}

Usually, previous methods adopt ResNet50~\cite{he2016deep} pre-trained on ImageNet~\cite{deng2009imagenet} for model initialization. Despite ImageNet being a general dataset for classification, the pre-trained backbone might not produce discriminant features for clustering in the UPReID and UVReID scenarios. More recently, self-supervised pretext tasks have been widely explored for initialization~\cite{caron2020unsupervised, he2020momentum, chen2020simple, caron2021emerging, zbontar2021barlow} and further application in downstream tasks.  In this context, we propose pre-training the model on the target unlabeled dataset before 
applying our pipeline. Since we assume a fully-unlabeled scenario, we 
leverage a self-supervised pre-training based on Barlow 
Twins~\cite{zbontar2021barlow}. %\rv{It does not rely on labels, and it is a negative-free mining approach since it relies only on the image and its respective augmentation without contrasting to the other images in the batch or dictionary.} 
We perform this initialization step for all backbones. More details are described in the Supplementary Material.

\subsection{Local Neighborhood Sampling (LNS)}
\label{sec:local_neighborhood_sampling}
It is a common requirement that the entire distance matrix, with all training feature vectors, be loaded into memory. However, as the number of considered samples grows, memory and time complexity increase which might lead to a costly deployment in a large-scale scenario. For instance, in~\cite{huang2019ultra}, the authors reported that a distance matrix of size $n \times n$, with $n = 10^{6}$, takes up $7450.58$ GB of memory. %, assuming each matrix entry is in double-precision format.
The largest datasets used in prior Unsupervised ReID works are \texttt{MSMT17} for Person and \texttt{Veri} for Vehicle Re-Identification. Since they have fewer than $40K$ training samples, previous methods might not have faced significant memory and time issues. However, there are several practical scenarios in which larger datasets are required~\cite{padilha2020forensic}. In these cases, a more efficient solution is required. 

To tackle this, given an unlabeled dataset $X = \{x_{i}^{n}\}$ with $n$ images, we propose a novel neighborhood-based approach for sampling a subset $X_{s}\subset X$ (Stage 1, Figure~\ref{fig:full_pipeline}). More specifically, we first select a backbone $f_{m}$ ($1 \leq m \leq 3$) randomly from the available backbones to perform feature extraction and obtain the feature vector set $F_{m}$ (blue points in Stage 1, Figure~\ref{fig:full_pipeline}). Then we randomly select a feature vector $v \in F_{m}$ and calculate its top-$p\%$ ($0 \leq p \leq 100$) nearest neighbors set in $F_{m}$, which defines $X_{s}\subset X$ (yellow points in Stage 1, Figure~\ref{fig:full_pipeline}). That is, $X_{s}$ comprises the elements of the whole dataset that are closer to $v$. The computational cost is related to computing the distances from $v$ to every other vector in $F_{m}$ and sorting the distance vector. Since cosine distance is used, the complexity is $\mathcal{O}(nd + nlgn)$, where $d$ is the feature dimension. In a practical scenario, we usually have $k \ll n$, and the distance computation is performed in parallel, so the actual complexity is $\mathcal{O}(nlgn)$. Once the set $X_{s}$ is defined, it is used by all backbones to perform feature extraction and Re-Ranking in Stages 2 and 3, respectively (blue points in Figure~\ref{fig:full_pipeline}). The set $X_{s}$ is redefined every three pipeline iterations by another randomly selected backbone, which brings diversity into training. In Section~\ref{sec:comparison_state_of_art}, we show the impact of $p$ in terms of performance and speedup. 
 %To validate our strategy we leverage the largest Re-Identification dataset Veri-Wild~\cite{} which has $277,797$ images  in the training set from $30,000$ vehicles captured from $172$ different cameras in a $200km^{2}$ area in a city. Since MSMT17 is the largest person re-identification dataset, we also evaluate our strategy over it despite being almost ten times smaller than Veri-Wild. 
 
%We do not consider other large-scale datasets, such as ImageNet~\cite{deng2009imagenet}, %or Places~\cite{wang2020online} 
%since our goal is to tackle challenging re-identification scenarios with significant variations in light, points of view, resolution, occlusions, and background. Therefore, we employ large-scale Re-Identification datasets only. 
 

\subsection{Local Re-Ranking}
\label{subsec:local_reranking_clustering}

Prior art usually employs the $k$-Reciprocal Encoding algorithm~\cite{zhong2017re} (Full Re-Ranking -- FRR) to account for the context (neighborhood) of each sample. This helps to compute a more robust distance measure and allows hard positive samples to be closer in the feature space. Despite its effectiveness, we argue it might not be efficient, in terms of memory and space, when considering large-scale scenarios. 

To improve efficiency when dealing with large-scale datasets while keeping the advantages of neighborhood-based distance refinement, we propose a new Local Re-Ranking (LRR) algorithm. At a given step, it only considers the local neighborhood of two samples. The idea is to consider samples in common in given neighborhoods, avoiding the full comparison between all samples in the training set. Without loss of generality, consider a feature vector set $F_{m}$ ($1 \leq m \leq 3$) created after feature extraction by one of the backbones in Stage 2. We first calculate the $k$-Nearest Neighbors set $N(x_{i},k)$ for each sample $x_{i} \in F_{m}$, and the local distance matrix $D_{loc} \in R^{n \times k}$, where the $i$-th line is the distance between $x_{i}$ and each element in $N(x_{i},k)$ after applying the exponential decay function. That is, $D_{loc}(i,j) = e^{-d(x_{i}, x_{j})}$, where $x_{j} \in N(x_{i},k)$ and $d(.,.)$ is the Euclidean distance. The lower (or greater) the distance, the closer to one (or zero) they are after the exponential decay transformation. After that, we employ our neighborhood calculation based on the Jaccard distance to refine the distance between each $x_{i}$ and its nearest neighbors. For each sample $x_{j} \in N(x_{i}, k)$ we calculate the following sets:
\begin{equation}
\label{eq:intersection_set_equation}
    I(x_{i}, x_{j}) = \{p | p \in N(x_{i}, k) \wedge p \in N(x_{j}, k)\},
\end{equation}
\begin{equation}
\label{eq:just_in_Aset}
   E(x_{i}, x_{j}) = N(x_{i},k) \backslash I(x_{i},x_{j}).
\end{equation}
$I(x_{i}, x_{j})$ is the Inclusion set, which contains the common neighbor elements, and $E(x_{i})$ is the Exclusion set, which contains elements in $N(x_{i}, k)$ but not in $N(x_{j},k)$. Following~\cite{zhong2017re}, we assume the greater the cardinality of $I(x_{i}, x_{j})$, the more likely $x_{i}$ and $x_{j}$ are samples from the same class. In light of this, we propose the following to refine the distance:
\begin{equation}
\label{eq:sum_min}
   s_{min} = \sum_{p \in I(x_{i}, x_{j})} min(D_{loc}(i,p), D_{loc}(j,p)),
\end{equation}
\begin{equation}
\label{eq:sum_max}
   s_{max} = \sum_{p \in I(x_{i}, x_{j})} max(D_{loc}(i,p), D_{loc}(j,p)),
\end{equation}
\begin{equation}
\label{eq:sum_xi}
   s(x_{i}, x_{j}) = s_{i,j} = \sum_{p \in E(x_{i}, x_{j})} D_{loc}(i,p),
\end{equation}
\begin{equation}
\label{eq:sum_xj}
   s(x_{j}, x_{i}) = s_{j,i} = \sum_{p \in E(x_{j}, x_{i})} D_{loc}(j,p),
\end{equation}
\begin{equation}
\label{eq:iou_distance}
   D_{IoU}(x_{i}, x_{j}) = \frac{s_{min}}{s_{max} + s_{i,j} + s_{j,i}},
\end{equation}
\noindent where $s_{min}$ is the sum of the minimum values when comparing the distances of $x_{i}$ and $x_{j}$ to a common neighbor $p$ (the same for $s_{max}$ but considering maximum values), and $s_{i, j}$ (or $s_{j, i}$) is the sum of the distances between sample $x_i$ (or $x_j$) and its neighbors that are not in the intersection $I(x_i, x_j)$. 

In extreme cases, when $x_{i}$ and $x_{j}$ have all their neighbors in common, we have $I(x_{i}, x_{j}) = N(x_{i}, k) = N(x_{j}, k)$, $E(x_{i}, x_{j}) = E(x_{j}, x_{i}) = \emptyset$, $s_{i,j}= 0$ and $s_{j,i} = 0$, and Equation~\ref{eq:iou_distance} becomes $0 \leq D_{IoU}(x_{i}, x_{j}) = s_{min}/s_{max} \leq 1$. Conversely, when they do not have any neighbors in common, we have $I(x_{i}, x_{j}) = \emptyset$, $E(x_{i}, x_{j}) = N(x_{i}, k)$, $E(x_{j}, x_{i}) = N(x_{j}, k)$, $s_{min} = 0$, $s_{max} = 0$, $s_{i, j} \neq 0$ and $s_{j, i} \neq 0$, then $D_{IoU}(x_{i}, x_{j}) = 0$. Therefore, we see that $ 0 \leq D_{IoU}(x_{i}, x_{j}) \leq 1$, and the closer it is to 1, the more likely it is for $x_{i}$ and $x_{j}$ to be from the same class. Finally, to convert $D_{IoU}$ into a distance measure, we define the refined distance matrix $R$ as
\begin{equation}
\label{eq:relation_iou_distance}
  R(x_{i}, x_{j}) =
    \begin{cases}
      1 - D_{IoU}(x_{i}, x_{j}), & \text{if $x_{i} \in N(x_{j}, k)$}\\
      1, & \text{otherwise},
    \end{cases}       
\end{equation}
\noindent for each $x_{j} \in N(x_{i}, k)$. Note that $R \in \mathbb{R}^{n \times k}$ is used to perform clustering in further steps.

In terms of complexity, we compare LRR and FRR theoretically. %without considering hardware optimization. %, such as parallelism. 
In FRR, the authors first compute the $k$-Nearest Neighbors set for each training sample in $\mathcal{O}(n^{2}\log{n})$. Then, they calculate the $k$-reciprocal nearest neighbor set in $\mathcal{O}(nk^{2})$, and the incremental set in $\mathcal{O}(nk^{3}/2)$. The full distance matrix is created based on the incremental set in $\mathcal{O}(n^2)$, and the final Jaccard distances are computed in $\mathcal{O}(n^3/2)$. Finally, they perform the local query expansion in $\mathcal{O}(n^2k)$. Therefore, FRR's complexity is $\mathcal{O}(n^3 + n^{2}\log{n} + n^2k + n^2 + nk^{3} + nk^{2}) = \mathcal{O}(n^3)$. For LRR, we also calculate the $k$-Nearest Neighbors in $\mathcal{O}(n^{2}\log{n})$ and distance matrix $D_{loc}$ in $\mathcal{O}(nk)$. The intersection set $I(x_{i}, x_{j})$ and the sets $E(x_{i})$ and $E(x_{j})$ are computed in $\mathcal{O}(3k^2)$. Equations~\ref{eq:sum_min} and~\ref{eq:sum_max} can be calculated in a single pass in $\mathcal{O}(k)$. Equations~\ref{eq:sum_xi} and~\ref{eq:sum_xj} are computed in 
$\mathcal{O}(k)$, so the final complexity is $\mathcal{O}(3k^2 + 3k)$ for a single pair $x_{i}$ and $x_{j}$. Considering all possible pairs in $D_{loc}$ and the whole training set, the complexity is $\mathcal{O}(3nk^3 + 3nk^2)$. Therefore, LRR's total complexity is $\mathcal{O}(n^{2}\log{n} + nk^3 + nk^2 + nk) = \mathcal{O}(n^{2}\log{n})$ with $k \ll n$. 
%where the training set size $n$ is much greater than the nearest neighbor set size ($k << n$). 
Our model is more efficient and applicable to large-scale datasets, which is also corroborated by the practical time analysis (Section~\ref{sec:ablation_study}). %We also provide a Cython-based implementation for practical speedup.   

\subsection{Noise-Robust Density Scheduling}
Previous clustering-based methods often assume strong assumptions. For instance, when using DBSCAN, it is common to use some side information to define $\varepsilon$, such as holding it fixed or calculating it using extra information.
Some works select a fixed $\varepsilon$ parameter during the unsupervised training in the clustering step. As the backbone's weights are constantly changing, creating different feature spaces, a fixed clustering hyper-parameter is likely suboptimal. %In early iterations, the backbones are strongly biased by the camera view, i.e., feature vectors from images from the same identity and same camera tend to be grouped together. So, if the $\varepsilon$ is too high, clusters might include false-positive samples. 
In early iterations, the backbones are strongly biased by the camera view, i.e., feature vectors from images from the same identity and the same camera tend to be grouped together. Conversely, images from the same person from different cameras are farther away than images from different identities but with the same camera. So, if $\varepsilon$ is too high, clusters might include false-positive samples from the same camera.
As the training progresses, the backbones start to learn camera-invariant representations, and more cross-view images from the same class are included in the clusters, but $\varepsilon$ might not be large enough to include diverse examples from different cameras. Hence, methods that consider a fixed $\varepsilon$ face a noise/diversity trade-off.  

Other unsupervised methods set an optimal $\varepsilon$ per dataset~\cite{zhang2022implicit, peng2023adaptive, yin2023real} because each dataset has a different complexity level. %For instance, Market1501~\cite{zheng2015scalable} has 12,936 training images from 751 identities recorded from six different outdoor cameras on a university campus, while MSMT17~\cite{wei2018person} has 32,621 images from 1,041 identities recorded from fifteen cameras (twelve outdoors and three indoors) in an entrance place along different days and months. 
For instance, previous research has reported a lower $\varepsilon$ for the \texttt{Market1501} dataset, as it has less variability in terms of cameras and identities, and a greater $\varepsilon$ for \texttt{MSMT17} due to a higher camera diversity. However, this selection is unrealistic since the main assumption is the full absence of labels or any other side information. It is not trivial to find an optimal value when assuming no prior knowledge about the dataset complexity~\cite{hou2016dsets, sharma2017knn}. Aiming to propose a more general approach, closer to a real deployment scenario, we introduce an $\varepsilon$ scheduling scheme to address the diversity/noise trade-off during training, without requiring per-dataset tuning.

We employ DBSCAN for clustering samples given their distances in matrix $R$ (Equation~\ref{eq:relation_iou_distance}). As distances are normalized between $0.0$ and $1.0$, the $\varepsilon$ value must be selected within this range. %A smaller value, as mentioned before, might create clusters with high true positive rates, but most of them will have images just from one camera since the background and camera view dominates the feature representation in early iterations. However, when the backbone is optimized to get samples from the same cluster closer and apart from the other clusters, it starts learning how to separate the samples from two (or more) different clusters that compress images from the same camera. Consequently, to some extent, the model alleviates the camera bias and starts to group more and more true positive cross-view images in the clusters. However, if $\varepsilon$ is fixed in a small value, it might not be enough to include in the same cluster most of the cross-view images since true positive cross-view images might be spread in the feature space. In counterpart, if the $\varepsilon$ has been large since the early iterations, it might include more cross-view images along the training, but it will also include more noisy samples, which can hinder the model's performance. 
We propose a novel $\varepsilon$ scheduling scheme, in which it starts from a low value and gradually increases, following a cosine scheduling, until half of the training epochs, as shown in Figure~\ref{fig:eps_scheduling}. We call this first phase the \textit{warmup}. As $\varepsilon$ is progressively increased, we allow the clustering algorithm to consider more cross-view images in a smoother manner. As previously reported by noise-robust feature learning methods~\cite{han2018co, yu2019does, song2022learning}, the method first learns from clean examples before starting to learn from the noisy ones. Since we start with high true-positive rate clusters in early iterations and smoothly increase the margin, the effect of noisy samples is alleviated even if the noise ratio starts to increase.

If we keep $\varepsilon$ in its highest value until the end of the training, the noisy samples are overemphasized, and the backbone overfits~\cite{han2018co, yu2019does, han2020survey, song2022learning}. To tackle this, we gradually decrease $\varepsilon$ to avoid adding too many noisy samples to the clusters. We call this the \textit{annealing} phase. After the model reaches $75\%$ of training epochs, $\varepsilon$ is kept in a plateau until the end of the training, in a \textit{steady state} phase. As the backbone learns to group the cross-view images in the first half of the training, the cross-view image representations tend to be closer at this stage. Thus, the backbone can keep learning from diverse samples as $\varepsilon$ is gradually decreased, alleviating the impact of the noisy samples. While our experiments use a fixed number of epochs, the scheduling would also allow a temporal stopping criterion. %, inspired by ~\cite{han2018co}. They propose supervised training where they select small-loss samples from one peer network to train the other. 
%They gradually decrease the small-loss threshold to avoid the influence of noisy samples and get more and more confident samples for training. 
%In counterpart, we redesigned the solution for the fully-unlabeled scenarios by not working in the loss value, but in the neighborhood density definition to obtain a model operating in fully-unlabeled complex scenarios, avoiding the hard and challenging task of selecting a proper $\varepsilon$ for each application. It is important to note that \textbf{our whole formulation does not consider the camera labeling}. It learns to group cross-view images in a fully-unsupervised and automatic way.  
% Figure environment removed

\subsection{Co-training}
\label{sec:co_training}
After defining $\varepsilon$ for the current epoch and performing clustering for each of the $M$ backbones, we have a sequence $Y$ of $M$ pseudo-label sets. It is defined as $Y = (Y_{m})_{m=1}^{M}$, where $Y_{m}$ is the set of pseudo-labels generated by clustering the features extracted by the $m$-th backbone. 

In previous Person Re-Identification works~\cite{ge2020mutual, zhai2020multiple, zhai2022population}, co-training is considered but with a mechanism involving hard and soft supervision among the backbones, often leading to a complex loss function and optimization process. In our case, we propose a parameter-free co-training strategy by permuting the generated pseudo-labels among the backbones. Formally, we generate a random permutation $Y^{'}$ of the sequence $Y$, giving that $Y^{'}_{m} \neq Y_{m}$, i.e., a backbone should never be supervised by its own pseudo-label set.

After that, each backbone $f_{m}$ is trained with a permuted pseudo-label set $Y^{'}_{m}$ (Stage 6, Figure~\ref{fig:full_pipeline}). As an illustrative example, let $Y_{1} = (-1,0,1)$ be the pseudo-labels generated by $f_1$ and $Y_{2} = (0, 0, 1)$ by $f_2$. If they are permuted, then $Y^{'}_{1}= (0, 0, 1)$ and $Y^{'}_{2}= (-1,0,1)$ are the sets carried on to the next stage for $f_1$ and $f_2$, respectively.
%is given in Table~\ref{tab:co_training_example}. 
This encourages complementary knowledge sharing among the backbones and alleviates error amplification since a backbone is supervised by one of other $M-1$ pseudo-labels sets. This solution outperforms all co-training techniques used in PReID methods, in the most complex scenarios. The permuted labeling is employed in the next stage to optimize the loss functions.

%\begin{table}[ht]
%\vspace{-55pt}
%\caption{\rv{An illustrative example of the proposed co-training method with five points. The initial labeling proposed by the clustering algorithm is in the first column. The second column indicates which labeling set is going to replace the current one. The third column indicate the permuted labeling set which will supervise each backbone. For instance, $Y^{'}_{1}$ will supervise the first backbone, $Y^{'}_{2}$ will supervise the second backbone and so on.}}
%\label{tab:co_training_example}
%\centering
%\begin{tabular}{P{2.7cm}|P{1.7cm}|P{2.7cm}}
%\hline
%Initial labeling & Labeling from & Permuted Labeling  \\ \hline
%\gv{$Y_{1} = (-1,0,1,0,2)$} & \ov{$Y{3}$} & \ov{$Y^{'}_{1}= (-1,0,0,1,2)$} \\
%\rule{0pt}{10pt}\bv{$Y_{2} = (0,0,1,0,2)$} & \gv{$Y{1}$} & \gv{$Y^{'}_{2}=(-1,0,1,0,2)$} \\
%\rule{0pt}{10pt}\ov{$Y_{3} = (-1,0,0,1,2)$} & \bv{$Y{2}$} & \bv{$Y^{'}_{3}=(0,0,1,0,2)$} \\


%\hline
%\end{tabular}
%\end{table}

%denoted by $Y^{'} = \{Y^{'}_{m}\}_{m=1}^{M}$ restricting that $Y_{m} \neq Y^{'}_{m}$. In other words, after the permutation, we make sure each pseudo-label set does not supervise its own backbone. After that we train each backbone $f_{m}$ with the respective pseudo-labels $Y^{'}_{m}$ (Stage 6, Figure~\ref{fig:full_pipeline}). This encourages complementary knowledge sharing among the backbones and alleviates error amplification since the backbone is always supervised by one of the $M-1$ pseudo-labels sets. With this simple but powerful solution, we outperform all the co-training Person Re-Identification methods in the most complex scenarios.

%we have $M$ pseudo-labels sets $Y = \{Y_{m}\}_{m=1}^{M}$ where $Y_{m}$ are the pseudo-labels generated by clustering the features extracted by the $m^{th}$ backbone. In prior Person Re-Identification works~\cite{ge2020mutual, zhai2020multiple, zhai2022population}, they perform co-training among the backbones but with a complex mechanism involving hard and soft supervision among the backbones leading to a complex loss function and optimization. In our case, we propose a parameter-free and simpler co-training strategy by just switching the pseudo-labels generated among the models. Formally, we generate a random permutation of $Y$ denoted by $Y^{'} = \{Y^{'}_{m}\}_{m=1}^{M}$ restricting that $Y_{m} \neq Y^{'}_{m}$. In other words, after the permutation, we make sure each pseudo-label set does not supervise its own backbone. After that we train each backbone $f_{m}$ with the respective pseudo-labels $Y^{'}_{m}$. This is shown in Stage 6 in~\ref{fig:full_pipeline}. This encourages complementary knowledge sharing among the backbones and alleviates error amplification since the backbone is always supervised by one of the $M-1$ pseudo-labels sets. With this simple but powerful solution, we outperform all the co-training Person Re-Identification methods in the most complex scenarios. 

\subsection{Optimization and self-ensembling}
The loss function (Stage 7, Figure~\ref{fig:full_pipeline}) comprises two terms: the proxy loss $L_{proxy}$ and the hard loss $L_{hard}$. For one backbone $f_{m}$, we first randomly select one sample per cluster in $Y^{'}_{m}$ to be the proxy of the cluster. As $Y^{'}_{m}$ contains samples from $c_{m}$ different clusters, we have the set $P_{m} = \{p_{m}^{1}, ..., p_{m}^{c_{m}} \}$ with $c_{m}$ proxies. Given a batch $B$ of norm-1 image feature vectors, we calculate the proxy loss as follows:
\begin{equation}
\label{eq:proxy_loss}
   L_{proxy} = -\frac{1}{|B|}\sum_{v \in B}log \frac{exp(v\cdot p_{m}^{+}/\tau)}{\sum_{q \in P_{m}} exp(v\cdot q/\tau)},
\end{equation}
\noindent where $p_{m}^{+}$ is the proxy of the cluster of feature vector $v$, $v\cdot q$ is the dot product between $v$ and $q$, and $\tau$ is a temperature hyper-parameter to control the shape of the distribution. %All vectors have norm 1.

Moreover, we consider the hard-positive and hard-negative samples in the batch~\cite{hermans2017defense} to increase feature representation:
\begin{equation}
\label{eq:batch_hard}
   L_{hard} = -\frac{1}{|B|}\sum_{v \in B}log \frac{exp(v\cdot v_{pos}/\tau)}{exp(v\cdot v_{pos}/\tau) + exp(v \cdot v_{neg}/\tau)},
\end{equation}
\noindent where $v_{pos}$ is the furthest sample in $v$'s cluster and in $B$, and $v_{neg}$ is the closest sample to $v$ in $B$ from a different cluster. 

$L_{proxy}$ is a more global loss because it considers all proxies in the calculation, while $L_{hard}$ enforces a more local view by considering just in-batch samples. The final loss function is
\begin{equation}
\label{eq:final_loss}
   L_{final} = L_{proxy} + \lambda L_{hard},
\end{equation}
\noindent where $\lambda$ controls the contributions of $L_{hard}$. In the Supplementary Material, we present the impact of $\tau$ and $\lambda$. 

To avoid noise amplification during training, we also perform the self-ensembling of each model's weights as usually done in previous methods~\cite{he2020momentum, caron2021emerging}. For each backbone $f_m$, we keep a self-ensembled model with parameters $\Theta_{m}^{t}$. They are updated as $\Theta_{m}^{t}:= \beta\Theta_{m}^{t-1} + (1-\beta)\theta_{m}^{t}$, where $\theta_{m}^{t}$ are the parameters of backbone $f_m$ optimized by Equation~\ref{eq:final_loss}, and $\beta$ is an inertia hyper-parameter set to $0.999$ as in~\cite{he2020momentum}. % in all models and experiments.

\subsection{Inference}

After the training pipeline, the inference is made by ranking all gallery samples based on the distance to a query sample. We extract feature vectors for all \textit{query} and \textit{gallery} sets using the self-ensembled models. These sets are denoted by $F_{q}^{m}$ and $F_{g}^{m}$, respectively, with $1 \leq m \leq M$. We calculate pairwise distances between samples of $F_{q}^{m}$ and $F_{g}^{m}$, resulting in a distance matrix $D_{q2g}^{m} \in \mathbb{R}^{|F_{q}^{m}| \times |F_{g}^{m}|}$. %where $|Q|$ and $|G|$ are the number of samples in the query and gallery sets. 

A final distance matrix $\overline{\rm D}_{q2g}$ is obtained by averaging all matrices element-wise:
\begin{equation}
\label{eq:distance_ensemble_evaluation}
    \overline{\rm D}_{q2g} = \frac{1}{M}\sum_{m=1}^{M}D_{q2g}^{m}.
\end{equation}
Each row of $\overline{\rm D}_{q2g}$ holds the distances from a query to the gallery samples. We sort these distances to get the closest class to the query sample. As done in previous works, we remove gallery images with the same class and camera of the query to assess performance in a true cross-camera scenario.

\section{Experiments and Results}
\label{sec:experiments}

In this section, we present the datasets, metrics, and results of our proposed pipeline compared to the prior art in the fully-unsupervised Person and Vehicle ReID problem.

\subsection{Datasets and Implementation Details}
\label{subsec:datasets}
We evaluate our method in two Person ReID, \texttt{Market1501}~\cite{zheng2015scalable} and \texttt{MSMT17}~\cite{wei2018person}, and three Vehicle ReID datasets, \texttt{Veri}~\cite{liu2016large}, \texttt{VehicleID}~\cite{liu2016deep}, and \texttt{Veri-Wild}~\cite{lou2019veri}, as summarized in Table~\ref{tab:dataset_information}. The last two are large-scale datasets with more than $100K$ images in the training set and with three evaluation scenarios. \texttt{Veri-Wild}~\cite{lou2019veri} is the most challenging one compressing 174 cameras while \texttt{Market1501}, \texttt{MSMT17}, and \texttt{Veri} compress six, fifiteen and twenty cameras respectively. The number of cameras in \texttt{VehicleID} is not informed.

\begin{table}[ht]
\caption{Information about the employed datasets.}
\centering
\begin{tabular}{P{1.0cm}|P{0.7cm}|P{0.8cm}|P{0.7cm}|P{0.8cm}|P{0.7cm}|P{0.8cm}}
&
\multicolumn{2}{|c|}{\textbf{train}} &
%\multicolumn{2}{|c}{\textbf{MSMT17}} &
\multicolumn{2}{|c}{\textbf{gallery}} &
\multicolumn{2}{|c}{\textbf{query}} \\ \hline
& \#IDs & \#Images & \#IDs & \#Images & \#IDs & \#Images \\ \hline
Market & 751 & 12,936 & 751 & 15,913 & 750 & 3,368\\
MSMT17 & 1,041 & 32,621 & 3,060 & 82,161 & 3,060 & 11,659\\
Veri & 576 & 37,778 & 200 & 11,579 & 200 & 1,678\\ \hline
\multirow{3}{*}{VehicleID} & \multirow{3}{*}{13,164} & \multirow{3}{*}{113,346} & 800 & 800 & 800 & 5,693 \\
& & & 1,600 & 1,600 & 1,600 & 11,777\\
& & & 2,400 & 2,400 & 2,400 & 17,377\\
\hline
\multirow{3}{*}{Veri-Wild} & \multirow{3}{*}{30,671} & \multirow{3}{*}{277,797} & 3,000 & 38,861 & 3,000 & 3,000 \\
& & & 5,000 & 64,389 & 5,000 & 5,000\\
& & & 10,000 & 128,517 & 10,000 & 10,000\\
\hline
\end{tabular}
\label{tab:dataset_information}
\end{table}


%\begin{itemize}
%    \item \texttt{Market1501}~\cite{zheng2015scalable}: 12,936 images of 751 identities in the training set. The test set is divided into 3,368 images for the query set and 15,913 images for the gallery. We removed ``junk'' images from the gallery set as done in previous work. It comprises six non-overlapping cameras; each identity is captured by at least two.
    
%    \item \texttt{MSMT17}~\cite{wei2018person}: 32,621 images of 1,401 identities in the training set. The test set is divided into 11,659 images for the query set and 82,161 images for the gallery. It comprises 15 cameras recording three periods (morning, afternoon, and night) on four days. It contains 15 cameras (12 outdoor, and 3 indoor). At least two cameras capture each identity.
    
%    \item \texttt{Veri}~\cite{liu2016large}: 37,778 images of 576 vehicles in the training set. The test set is divided into 1,678 images for the query set and 11,579 for the gallery. It comprises 20 cameras.

%    \item \texttt{VehicleID}~\cite{liu2016deep}: 113,346 images of 13164 vehicles in the training set. The test set has three evaluation scenarios, with 6,493, 13,377, and 19,777 images of 800, 1,600, and 2,400 vehicles, respectively. Following the original protocol, we select one image from each vehicle as the gallery, and the remaining images are the query set. We repeat this process 10 times and report the average result.
    
%    \item \texttt{Veri-Wild}~\cite{lou2019veri}: 277,797 images of 30,671 vehicles in the training set, recorded from 173 different cameras along different days. The test set has three evaluation scenarios with 3,000/38,861, 5,000/64,389, and 1,000/128,517 query/gallery images, respectively. It is the most challenging large-scale ReID dataset. 
    
%\end{itemize}

For evaluation, we calculate the Cumulative Matching Curve (CMC), from which we report Rank-1 (R1), Rank-5 (R5), Rank-10 (R10), and mean Average Precision (mAP).

%\subsection{Implementation Details}

We adopt ResNet50~\cite{he2016deep}, DenseNet121~\cite{huang2017densely}, and OSNet~\cite{zhou2019omni} pre-trained in ImageNet~\cite{deng2009imagenet} as our backbones. We use Pytorch~\cite{NEURIPS2019_9015}, Torchreid~\cite{torchreid}, and FAISS~\cite{johnson2019billion} as supporting libraries. %as commonly done by the prior art. 
We first pre-train all backbones in each Person/Vehicle ReID dataset using the self-supervised Barlow Twins strategy. We randomly select one backbone to extract features and obtain the top-$p\%$ ($p$ can be $25\%$, $50\%$, $75\%$ or $100\%$) samples in LNS, every three epochs. We linearly warm up the learning rate from $3.5e-5$ to $3.5e-4$ in the first 10 epochs and keep it fixed for the remaining training in a total of $40$ epochs. We use the Adam~\cite{kingma2014adam} optimizer with weight decay set to $5e-4$. The batches are created by randomly sampling $16$ pseudo-identities (clusters) and $12$ images from each. We sample batches until all pseudo-identities are covered, then we repeat this process five times before going to the next epoch. Before each epoch, we renew the class proxies. The parameter $\tau$ is set to $0.04$ and $\lambda$ to $0.5$ in all experiments. 

We trained the models in five TITAN RTX GPUs with 24GB of memory each. One GPU is left just for Local Re-Ranking and clustering, and the others are used for training. Further implementation details are in the Supplementary Material.

\subsection{Comparison to State-of-The-Art methods}
\label{sec:comparison_state_of_art}

% We compare our method with the prior art, and results are shown in Tables~\ref{tab:state_of_art_reid_fully_unsupervised},~\ref{tab:state_of_art_reid_ensembles},~\ref{tab:state_of_art_vehicle},~\ref{tab:vehicleID}, and~\ref{tab:veri_wild}. 

Table~\ref{tab:state_of_art_reid_fully_unsupervised} shows our method compared to prior fully-unsupervised Person ReID models that, \textbf{like ours, do not consider camera labels}. Results for other methods that use side information are in the Supplementary Material. We flag the methods that tune the clustering parameter per dataset because they cannot be used in a realistic fully-unsupervised scenario (column \textit{CPD}). All comparison tables show mAP and R1. R5 and R10 are shown in the Supplementary Material.  

AdaMG~\cite{peng2023adaptive} achieves good results in \texttt{Market} and \texttt{MSMT17} by setting $\varepsilon = 0.5$ and $\varepsilon = 0.7$, respectively, but with a fixed $\varepsilon$, they face a huge performance drop~\cite{peng2023adaptive}. With our $\varepsilon$ scheduling scheme, we outperformed AdaMG by $1.2$ and $0.1$ percentage points (p.p.) in mAP and R1, respectively, in \texttt{Market}, and by $5.2$ and $4.6$ p.p. in the challenging \texttt{MSMT17}.

% In Table~\ref{tab:state_of_art_reid_fully_unsupervised}, we compared our models to prior Person Re-Identification datasets.  We compare ours to fully-unsupervised models \textbf{without considering camera labeling}. Further models considering camera labeling are compared in the Supplementary Material. It is important to point out some of the methods in Table~\ref{tab:state_of_art_reid_fully_unsupervised} tune the $\varepsilon$ parameter of DBSCAN (or the clustering hyper-parameter involved in the pipeline) in each dataset to achieve the best performance, which is unrealistic in a fully-unsupervised scenario since it does not have any kind of validation data to properly pick the best clustering hyper-parameters. For instance, the best model in prior art AdaMG~\cite{peng2023adaptive} achieves the performance in Market and MSMT17 by setting $\varepsilon = 0.5$ and $\varepsilon = 0.7$. Besides, taking their own ablation study table (Table IX in~\cite{peng2023adaptive}) it is possible to see that if they keep $\varepsilon = 0.5$ for both datasets, they face a huge performance dropping in mAP ($38.0\%$ to $11.4\%$) and in R1 ($66.3\%$ to $32.8\%$) in MSMT17 which is the most challenging Person ReID dataset. Our model in counterpart does not rely on specific tuning in any dataset, and the same $\varepsilon$ scheduling is employed in all datasets. We outperformed AdaMG by $1.2$ and $0.1$ in mAP and R1, respectively, in Market, and by $5.2$ and $4.6$ in mAP and R1 in the challenging MSMT17. 

Due to the proposed Local Re-Ranking, our model has a lower Re-Ranking memory footprint ($\mathcal{O}(kN)$ with $k \ll N$) compared to the best methods HHCL, GRACL, and AdaMG ($\mathcal{O}(N^2)$) and still outperforms them in mAP and R1 in \texttt{Market} and in all metrics in \texttt{MSMT17}. Our method also outperforms all other methods~\cite{nikhal2021unsupervised, zheng2022clustering, ding2019dispersion, sun2021unsupervised, yin2021multi, pan2020unsupervised, wang2020unsupervised, nikhal2022multi, yu2022graph, tang2021fully, zeng2020hierarchical, tang2021unsupervised, tang2022unsupervised, zhao2022unsupervised, yin2022unsupervised, si2023diversity, sun2022hybrid, li2023multi, pang2023reliability, zhang2022unsupervised, jin2022meta, zhang2022graph, wang2023relation} by a large margin. 

% Our method also ranks second place in MSMT17 using just $75\%$ of the data in each iteration, while the remaining methods start with $100\%$, showing our method is able to mine useful patterns even with a reduced amount of data. Further analysis is given in Section~\ref{sec:result_speedup_local_sampling}. 

Overall, our method provides the best mAP and R1 without relying upon any dataset-specific hyper-parameter and with a more efficient Re-Ranking strategy. 

\begin{table}[ht]
%\vspace{-55pt}
\caption{Comparison with relevant fully-unsupervised Person ReID methods. The best result is highlighted in \bv{blue}, the second best in \gv{green}, and the third in \ov{orange}. RRMC means Re-Ranking Memory Complexity and CPD (Cluster Parameter per Dataset) indicates if the method relies on specific clustering parameters per dataset. (p\%) means that p\% of all data points are sampled in LNS at each epoch.}
\label{tab:state_of_art_reid_fully_unsupervised}
\centering
\begin{tabular}{p{1.6cm}| p{1.2cm}|P{0.75cm}|P{0.4cm}|p{0.4cm}|p{0.4cm}|p{0.4cm}|p{0.4cm}}
\hline
\multicolumn{1}{c|}{} &
\multicolumn{1}{|c|}{} &
\multicolumn{1}{|c|}{} &
\multicolumn{1}{|c|}{} &
\multicolumn{2}{|c|}{\textbf{Market}} &
\multicolumn{2}{|c}{\textbf{MSMT17}} \\ \hline
%\multicolumn{14}{|c|}{\textit{Fully Unsupervised}}  \\ \hline
Method & Reference & RRMC & CPD & mAP & R1
& mAP & R1 \\ \hline
%BUC~\cite{lin2019bottom} & AAAI'19 & 38.3 & 66.2 & 79.6 & 84.5 & 27.5 & 47.4 & 62.6 & 68.4 & - & - & - & - \\
%Nikhal et. al.~\cite{nikhal2021unsupervised} & WACV'21 & - & \gv{No} & 35.7 & 63.9 & 78.8 & 85.1 & - & - & - & - \\
%Zheng \textit{et. al.}~\cite{zheng2022clustering} & TOMM'21 & - & \gv{No} & 42.0 & 68.8 & - & - & - & - & - & - \\
%DBC~\cite{ding2019dispersion} & BMVC'19 & - & \gv{No} & 41.3 & 69.2 & 83.0 & 87.8 & - & - & - & -\\ 
%GPUFL~\cite{sun2021unsupervised} & ICIP'21 & - & \rv{Yes} & 42.3 & 69.6 & - & - & - & - & - & -  \\
%MV-ReID~\cite{yin2021multi} & SPL'21 & - & \gv{No} & 45.6 & 73.3 & 85.3 & 89.1 & - & - & - & - \\
%MFC-LCDM~\cite{pan2020unsupervised} & ICASSP'20 & - & \gv{No} & 53.6 & 77.5 & 87.9 & 91.7 & - & - & - & - \\
%MMCL~\cite{wang2020unsupervised} & CVPR'20 & - & \gv{No} & 45.5 & 80.3 & 89.4 & 92.3 & 11.2 & 35.4 & 44.8 & 49.8 \\ 
%MCGA~\cite{nikhal2022multi} & TBIOM'22 & - & \gv{No} & - & - & - & - & 14.0 & 36.1 & 49.5 & 56.2 \\
%GSMLP-BMLC~\cite{yu2022graph} & AI'22 & - & \gv{No} & 48.4 & 80.9 & 90.2 & 93.1 & 15.8 & 36.2 & 44.0 & 48.1 \\
%MLJT~\cite{tang2021fully} & Access'21 & - & \gv{No} &55.3 & 81.6 & 90.4 & 93.5 & - & - & - & -\\
%HCT~\cite{zeng2020hierarchical} & CVPR'20 & - & \gv{No} & 56.4 & 80.0 & 91.6 & 95.2 & - & - & - & - \\
%NNCT~\cite{tang2021unsupervised} & ICIP'21 & - & \gv{No} & 57.6 & 85.2 & 92.3 & 94.3 & - & - & - & - \\
%SNNet~\cite{tang2022unsupervised} & ICIT'22 & - & \gv{No} & 58.3 & 85.1 & 92.6 & 94.3 & - & - & - & - \\
%PNNM~\cite{zhao2022unsupervised} & DSP'22 & - & \gv{No} & 45.6 & 71.8 & 84.0 & 88.3 & 19.5 & 43.1 & 56.8 & 65.4 \\
ABMT~\cite{chen2020enhancing} & WACV'20 & $\mathcal{O}(N^{2})$ & \gv{No} & 65.1 & 82.6 & - & -\\
SpCL~\cite{ge2020self} & NeurIPS'20 & $\mathcal{O}(N^{2})$ & \gv{No} & 73.1 & 88.1 & 19.1 & 42.3\\
GCL+~\cite{chen2022learning} & TPAMI'22 & $\mathcal{O}(N^{2})$ & \gv{No} & 69.3 & 89.0 & 22.0 & 47.9 \\ 
GSam~\cite{han2022rethinking} & TIP'22 & $\mathcal{O}(N^{2})$ & \gv{No} & 79.2 & 92.3 & 24.6 & 56.2 \\
%SpCL-IBN~\cite{ge2020self} & NeurIPS'20 & 73.8 & 88.4 & 95.3 & 97.3 & - & - & - & - & 24.0 & 48.9 & 61.8 & 67.1 \\
RLCC~\cite{zhang2021refining} & CVPR'21 & $\mathcal{O}(N^{2})$ & \gv{No}&  77.7 & 90.8 & 27.9 & 56.5 \\
%MaskPre~\cite{yin2022unsupervised} & PR'22 & - & \gv{No} & 77.7 & 90.4 & 96.4 & 97.9 & 26.7 & 58.5 & 69.3 & 73.8 \\
HCLP~\cite{zheng2021online} & ICCV'21 & - & \gv{No} & 78.1 & 91.1 & 26.9 & 53.7\\
%DFC~\cite{si2023diversity} & IPM'23 & - & \gv{No} &  78.6 & 90.8 & 97.5 & - & - & - & - & - \\
ICE~\cite{Chen_2021_ICCV} & ICCV'21 & $\mathcal{O}(N^{2})$ & \gv{No} &79.5 & 92.0 & 29.8 & 59.0\\
%MGCE-HCL~\cite{sun2022hybrid} & ACPR'22 & - & \gv{No} & 79.6 & 92.1 & - & - & - & - & - & - \\
%MPC~\cite{li2023multi} & CVIU'23 &$\mathcal{O}(|C_{i}|^{2})$ & \gv{No} & 77.4 & 90.9 & 96.4 & 97.6 & - & - & - & - \\
CACL~\cite{li2022cluster} & TIP'22 & - & \gv{No} & 80.9 & 92.7 & 23.0 & 48.9\\
HDCRL~\cite{cheng2022hybrid} & TIP'22 & - & \gv{No} & 81.7 & 92.4 & 24.6 & 50.2\\
PPLR~\cite{cho2022part} & CVPR'22 & $\mathcal{O}(N^{2})$ & \gv{No} & 81.5 & 92.8 & 31.4 & 61.1\\
%RMCL~\cite{pang2023reliability} & KBS'23 & $\mathcal{O}(N^{2})$ & \gv{No} & 81.7 & 93.0 & 97.6 & 98.4 & 32.5 & 62.3 & 73.6 & 78.0 \\
%Zhang et. al.~\cite{zhang2022unsupervised} & AI'22 & $\mathcal{O}(N^{2})$ & \gv{No} & 83.0 & 93.2 & 97.3 & - & - & - & - & - \\
RTMem~\cite{yin2023real} & TIP'23 & - & \rv{Yes} & 83.0 & 92.8 & 32.8 & 57.1\\
CCons~\cite{dai2022cluster} & ACCV'22 & $\mathcal{O}(N^{2})$ & \gv{No} & 83.0 & 92.9 & 33.0 & 62.0 \\
%MCL~\cite{jin2022meta} & ICM'22 & $\mathcal{O}(N^{2})$ & \rv{Yes} & 83.3 & 93.0 & - & - & 33.4 & 62.9 & - & - \\
%GCM~\cite{zhang2022graph} & SIVP'22 & $\mathcal{O}(N^{2})$ & \gv{No} & 83.4 & 93.3 & - & - & - & - & - & - \\
ISE~\cite{zhang2022implicit} & CVPR'22 & - & \rv{Yes} & \gv{84.7} & \bv{94.0} & 35.0 & 64.7\\
%Zheng~\cite{zheng2022multi} & ArXiv'22 & $\mathcal{O}(N^{2})$ & \gv{No} & 83.1 & 92.8 & 97.1 & 98.0 & 35.6 & 63.8 & 75.3 & 79.5 \\
HHCL~\cite{hu2021hard} & NIDC'21 & $\mathcal{O}(N^{2})$ & \gv{No} & 84.2 & \ov{93.4} & - & -\\
GRACL~\cite{zhang2022global} & TCSVT'22 & $\mathcal{O}(N^{2})$ & \gv{No} &  83.7 & 93.2 & 34.6 & 64.0\\
%CGCL~\cite{miao2022confidence} & ArXiv'22 & - & \gv{No} & 85.3 & 94.2 & 97.6 & 98.5 & 34.6 & 63.4 & 74.6 & 79.3 \\
%NCPLR~\cite{cheng2022neighbour} & ArXiv'22 & - & \rv{Yes} & 86.3 & 94.3 & 98.0 & 98.7 & 35.7 & 66.3 & 76.9 & 80.6 \\
AdaMG~\cite{peng2023adaptive} & TCSVT'23 &  $\mathcal{O}(N^{2})$ & \rv{Yes} & \ov{84.6} & \gv{93.9} & \ov{38.0} & \ov{66.3}\\
%DCCT~\cite{chen2022dual} & ArXiv'22 &  $\mathcal{O}(N^{2})$ & \rv{Yes} & 86.3 & 94.4 & 97.7 & 98.5 & 41.8 & 68.7 & 79.0 & 82.6 \\ 
\hline
%\textbf{Ours} & & 84.4 & 93.2 & 96.7 & 97.6 & 45.3 & 71.9 & 81.0 & 83.9\\
%\textbf{Ours (best $k$)} & & 85.7 & 93.9 & 97.2 & 98.0 &  &  &  & \\
\textbf{Ours (25\%)} &  & $\mathcal{O}(kN)$& \gv{No} & - & - & 32.0 & 60.5\\
\textbf{Ours (50\%)} & & $\mathcal{O}(kN)$ & \gv{No} & - & - & 24.3 & 50.4\\
\textbf{Ours (75\%)} & & $\mathcal{O}(kN)$ & \gv{No} & 82.9 & 92.6 & \gv{39.3} & \gv{67.3}\\
\textbf{Ours (100\%)} &  & $\mathcal{O}(kN)$ & \gv{No} & \bv{85.8} & \bv{94.0} & \bv{43.2} & \bv{70.9}\\

\hline
\end{tabular}
\end{table}

\begin{table}[ht]
%\vspace{-55pt}
\caption{Comparison with relevant ensemble-based Person ReID methods. ``\# BB'' shows the number of backbones used in training or evaluation, ``Src'' means if a labeled dataset has been used to initialize the model before training: ``M'' for \texttt{Market1501} and ``D'' for \texttt{DukeMTMC-ReID}. The best result is highlighted in \bv{blue} and the second best in \gv{green}.}.
\label{tab:state_of_art_reid_ensembles}
\centering
\begin{tabular}{p{1.2cm}|P{0.95cm}|P{0.7cm}|P{0.25cm}|P{0.4cm}|P{0.4cm}|P{0.25cm}|P{0.4cm}|P{0.3cm}}
\hline
\multicolumn{3}{c|}{} &
\multicolumn{3}{|c|}{\textbf{Market}} &
%\multicolumn{1}{|c|}{} &
\multicolumn{3}{|c}{\textbf{MSMT17}} \\ \hline
Method & Reference & \# BB & Src & mAP & R1
& Src & mAP & R1 \\ \hline
ACT~\cite{yang2020asymmetric} & AAAI'20 & 2 & D & 60.6 & 80.5 & - & - & - \\
MMT~\cite{ge2020mutual} & ICLR'20 & 2 & D & 71.2 & 87.7 & M & 22.9 & 49.2\\ 
MEB~\cite{zhai2020multiple} & ECCV'20 & 3 & D & 76.0 & 89.9 & - & - & -\\
UST~\cite{9521886} & TIFS'21 & 3 & D & 78.4 & 92.9 & M & 33.2 & 62.3\\
ESSL~\cite{bertocco2022reasoning} & TIFS'23 & 3 & - & 83.4 & 92.9 & - & 42.6 & 68.2 \\ 
PEG~\cite{zhai2022population} & IJCV'22 & 8 & - & \bv{87.1} & \bv{94.6} & - & \gv{41.8} & \gv{69.1}\\
\hline
\textbf{Ours} & & 3 & - & \gv{85.8} & \gv{94.0} & - & \bv{43.2} & \bv{70.9}\\
\hline
\end{tabular}
\end{table}

Since our method is based on three different architectures, we also compare it to prior ensemble-based methods (Table~\ref{tab:state_of_art_reid_ensembles}). We obtain the second-best performance in \texttt{Market}. However, the best method, PEG~\cite{zhai2022population}, utilizes 8 backbones, which include ResNet50 and DenseNet121, like ours. Our third backbone, OSNet, has a lower memory footprint compared to the other backbones. They also employ a complex evolutionary-based strategy and co-training where the backbones are selected at different moments with different losses supervising each other. Our results were achieved with just three backbones and a simpler and parameter-free co-training. In \texttt{MSMT17}, which is more complex than \texttt{Market}, we outperform PEG by $1.4$ and $1.8$ p.p. in mAP and R1, respectively. Therefore, with a simpler strategy, we can still take advantage of the complementary knowledge from different backbones and outperform prior art in more complex scenarios.    

In the \texttt{Veri} dataset (Table~\ref{tab:state_of_art_vehicle}), we have the best R1 among all methods. Although we obtain the second-best result considering mAP, MSCL has a much lower R1 score, and they present a higher memory footprint for Re-Ranking. We are also a margin better than BUC~\cite{lin2019bottom}, MMCL~\cite{wang2020unsupervised}, and SSML~\cite{yu2021unsupervised}.

In the large-scale and challenging \texttt{VehicleID} (Table~\ref{tab:vehicleID}), we scored first in R1 in the most difficult scenario (TS = 2400) along with CCons~\cite{dai2022cluster}. However, CCons leverages a more memory-complex Re-Ranking. %, and lower by just by $0.2$ in mAP. 
In the other scenarios, we scored first or, at least, second place. %However, CCons leverages a more memory-complex Re-Ranking, and we overcome them in all other datasets by a large margin.
The results for \texttt{Test Size = 800} are presented in the Supplementary Material.

The \texttt{Veri-Wild} dataset has been less employed in the fully-unsupervised scenario. Usually, prior art utilizes camera labels, as shown in the Supplementary Material. Considering the few methods that, like ours, do not consider camera labels, we outperform them in the three evaluation scenarios in all metrics (Table~\ref{tab:veri_wild}). More specifically, in the most challenging setup (\texttt{VW-Large}), we outperform prior art by $4.5$ and $1.7$ p.p., in mAP and R1, respectively, with $75\%$ of the data. Furthermore, we provide the second- and third-best results with $100\%$ and $50\%$ of the data, respectively. The results for \texttt{Veri-Wild-Small} are presented in the Supplementary.

%Since the camera labeling has a strong capacity to help the models even without identity annotation, just a few models have reported performance in the fully-unsupervised scenario, as seen in Table~\ref{tab:veri_wild}, which is the focus of our work. Our method outperforms prior art in the three evaluation scenarios in all metrics. More specifically, in the most challenging setup (``Veri-Wild (Large)''), we outperform prior art by $4.2$ and $1.5$ in mAP and R1, respectively, with $100\%$ of data, and we can get even further improvement with $75\%$ of the data. 

\begin{table}[ht]
\caption{Comparison with relevant fully-unsupervised Vehicle ReID methods in \texttt{Veri776}. The best result is highlighted in \bv{blue}, the second best in \gv{green}, and the third in \ov{orange}. RRMC means Re-Ranking Memory Complexity, and CPD (Cluster Parameter per Dataset) indicates if the method relies on specific clustering parameters per dataset. (p\%) means that p\% of all data points are sampled in LNS at each epoch.}
\label{tab:state_of_art_vehicle}
\centering
\begin{tabular}{p{1.8cm}|p{1.2cm}|P{0.75cm}|P{0.4cm}|p{0.45cm}|p{0.40cm}|p{0.40cm}}
\hline
\multicolumn{1}{c|}{} &
\multicolumn{1}{|c|}{} &
\multicolumn{1}{|c|}{} &
\multicolumn{1}{|c|}{} &
\multicolumn{3}{|c}{Veri} \\
\hline
%\multicolumn{1}{|c|}{} &
%\multicolumn{1}{|c|}{} &
%\multicolumn{1}{|c|}{} &
%\multicolumn{1}{|c|}{} &
%\multicolumn{3}{|c|}{\textit{Part-based Models}} \\ \hline
%Method & Reference & RR-MC & CPPD? & mAP & R1 & R5  \\ \hline
%PPLR~\cite{cho2022part} & CVPR'22 & $\mathcal{O}(N^{2})$ & \gv{No} & 41.6 & 85.6 & 91.1\\ 
%MLPL~\cite{he2022multi} & TVT'22 & - & \gv{No} & 45.1 & 88.3 & 91.1\\ \hline
%\multicolumn{1}{|c|}{} &
%\multicolumn{1}{|c|}{} &
%\multicolumn{1}{|c|}{} &
%\multicolumn{1}{|c|}{} &
%\multicolumn{3}{|c|}{\textit{Attribute-based Models}} \\ \hline
%Method & Reference & RR-MC & CPPD? & mAP & R1 & R5\\ \hline
%VRPRD~\cite{bashir2019vr} & PR'19 & - & \gv{No} & 40.1 & 83.2 & 91.1 \\ \hline
%\multicolumn{1}{|c|}{} &
%\multicolumn{1}{|c|}{} &
%\multicolumn{1}{|c|}{} &
%\multicolumn{1}{|c|}{} &
%\multicolumn{3}{|c|}{\textit{Single-part-based Models}} \\ \hline
Method & Reference & RRMC & CPD & mAP & R1 & R5  \\ \hline
%BUC~\cite{lin2019bottom} & AAAI'19 & - & \gv{No} & 21.2 & 54.7 & 70.4  \\
%PAL~\cite{peng2020unsupervised} & ArXiv'20 & - & \gv{No} & 42.0 & 68.2 & 79.9 \\
%MMCL~\cite{wang2020unsupervised} & CVPR'20 & - & \gv{No} & 24.2 & 71.8 & 75.9 \\
%SSML~\cite{yu2021unsupervised} & IROS'21 & - & \gv{No} & 26.7 & 74.5 & 80.3\\
SpCL~\cite{ge2020self} & NeurIPS'20 & $\mathcal{O}(N^{2})$ & \gv{No} & 36.9 & 79.9 & 86.8 \\
GRACL~\cite{zhang2022global} & TCSVT'22 & $\mathcal{O}(N^{2})$ & \gv{No} & 39.4 & 82.9 & - \\
RLCC~\cite{zhang2021refining} & CVPR'21 & $\mathcal{O}(N^{2})$ & \gv{No} & 39.6 & \ov{83.4} & 88.8\\
CCons~\cite{dai2022cluster} & ACCV'22 & $\mathcal{O}(N^{2})$ & \gv{No} & 40.8 & \gv{86.2} & \bv{90.5}\\
AdaMG~\cite{peng2023adaptive} & TCSVT'23 & $\mathcal{O}(N^{2})$ & \rv{Yes} &  41.0 & \gv{86.2} & \gv{90.6} \\ 
RTMem~\cite{yin2023real} & TIP'23 & - & \rv{Yes} & \gv{41.8} & 81.6 & 87.0\\
%NCPLR~\cite{cheng2022neighbour} & ArXiv'22 &  - & \rv{Yes} & 42.0 & 85.8 & 90.5\\
MSCL~\cite{wang2022unsupervised} & SVIP'22 & $\mathcal{O}(N^{2})$ & \gv{No} &\bv{45.9} & 81.2 & - \\ 
\hline
\textbf{Ours (25\%)} & & $\mathcal{O}(kN)$ & \gv{No} & 28.0 & 66.8 & 72.2  \\
\textbf{Ours (50\%)} & & $\mathcal{O}(kN)$ & \gv{No} & 40.8 & 84.5 & 88.6  \\
\textbf{Ours (75\%)} & & $\mathcal{O}(kN)$ & \gv{No} & \ov{41.7} & \gv{86.2} & \ov{89.9}  \\
\textbf{Ours (100\%)} & & $\mathcal{O}(kN)$ & \gv{No} & 41.3 & \bv{86.3} & \ov{89.9} \\
\hline
\end{tabular}
\end{table}

\begin{table}[ht]
\caption{Comparison with relevant fully-unsupervised Vehicle ReID methods in \texttt{VehicleID}. The best result is highlighted in \bv{blue}, the second best in \gv{green}, and the third in \ov{orange}. RRMC means Re-Ranking Memory Complexity and CPD (Cluster Parameter per Dataset) indicates if the method relies on specific clustering parameters per dataset. Methods with * are reproduced from~\cite{he2022multi}. Speedup values are measured in comparison to the version with $100\%$ of the data. (p\%) means that p\% of all data points are sampled in LNS at each epoch. TS holds for Test Size.}
\label{tab:vehicleID}
\centering
\begin{tabular}{p{1.58cm}| P{1.0cm}|P{0.75cm}|P{0.40cm}|p{0.45cm}|p{0.45cm}|p{0.45cm}|p{0.45cm}}
\hline
\multicolumn{1}{c|}{} &
\multicolumn{1}{|c|}{} &
\multicolumn{1}{|c|}{} &
\multicolumn{1}{|c|}{} &
%\multicolumn{3}{|c|}{\textbf{Test size  = 800}} &
\multicolumn{2}{|c|}{\textbf{TS = 1600}} &
\multicolumn{2}{|c}{\textbf{TS = 2400}} \\
\hline
%\multicolumn{14}{|c|}{\textit{Camera-based}} \\ \hline
%\multicolumn{14}{|c|}{\textit{Segmentation-based}} \\ \hline
%\multicolumn{13}{|c|}{\textit{Part-based Models}} \\ \hline
%Method & Reference & RR-MC & CPD & mAP & R1 & R5 & mAP & R1 & %R5 & mAP & R1 & R5 \\ \hline
%MLPL~\cite{he2022multi}* & TVT'22 & - & \gv{No} & 65.3 & 61.1 & %69.8 & 62.7 & 57.3 & 68.4 & 59.6 & 52.4 & 66.2 \\ \hline
%\multicolumn{13}{|c|}{\textit{Fully Unsupervised}} \\ \hline
Method & Reference & RRMC & CPD & mAP & R1 & mAP & R1\\ \hline
BUC~\cite{lin2019bottom}* & AAAI'19 & - & \gv{No} & 46.2 & 45.9 & 42.4 & 39.7\\
%PAL~\cite{peng2020unsupervised}* & ArXiv'20 & - & \gv{No} & 53.5 & 50.3 & 64.9 & 48.1 & 44.3 & 60.9 & 45.1 & 41.1 & 59.1 \\ 
MAC~\cite{zhu2022manifold} & KBS'22 & - & \gv{No} & 51.9 & 47.5 & 47.4 & 44.4\\
%MSCL~\cite{wang2022unsupervised} & SVIP'22 & $\mathcal{O}(N^{2})$ & \gv{No} &  - & - & - & - & - & - & 52.9 & \bv{51.8} & - \\
SpCL~\cite{ge2020self}* & NIPS'20 & $\mathcal{O}(N^{2})$ & \gv{No} & 58.7 & 53.1 & 54.3 & \ov{48.9}\\
CCons~\cite{dai2022cluster}* & ACCV'22 & $\mathcal{O}(N^{2})$ & \gv{No} & \bv{60.3} & \bv{54.0} & \bv{57.1} & \bv{50.1} \\
%TCL~\cite{shen2023triplet} & ArXiV'23 & & & 66.3 & 60.4 & - & 63.7 & 56.2 & - & 61.1 & 52.9 & -  \\
\hline
& Speedup & \multicolumn{6}{c}{} \\ \hline
\textbf{Ours (25\%)} & $4.97\times$ & $\mathcal{O}(kN)$ & \gv{No} & 59.0 & 52.3 & 55.6 & 48.5\\
\textbf{Ours (50\%)} & $2.44\times$ &  $\mathcal{O}(kN)$ & \gv{No} & \ov{59.2} & 52.8 & 56.0 & \ov{48.9}\\
\textbf{Ours (75\%)} & $1.31\times$ &  $\mathcal{O}(kN)$ & \gv{No} & \gv{59.7} & \ov{53.3} & \ov{56.6} & \gv{49.6} \\
\textbf{Ours (100\%)} & $1.00\times$ &  $\mathcal{O}(kN)$ & \gv{No} & \gv{59.7} & \gv{53.4} & \gv{56.9} & \bv{50.1}\\
\hline
\end{tabular}
\end{table}


\begin{table}[ht]
\caption{Comparison with relevant fully-unsupervised Vehicle ReID methods in \texttt{Veri-Wild} in the medium and large setups. The best result is highlighted in \bv{blue}, the second best in \gv{green}, and the third in \ov{orange}. RRMC means Re-Ranking Memory Complexity and CPD (Cluster Parameter per Dataset) indicates if the method relies on specific clustering parameters per dataset. Speedup values are measured in comparison to the version with $100\%$ of the data. (p\%) means that p\% of all data points are sampled in LNS at each epoch.}
\label{tab:veri_wild}
\centering
\begin{tabular}{p{1.58cm}| P{1.0cm}|P{0.75cm}|P{0.4cm}|p{0.45cm}|p{0.4cm}|p{0.45cm}|p{0.4cm}}
\hline
\multicolumn{1}{c|}{} &
\multicolumn{1}{|c|}{} &
\multicolumn{1}{|c|}{} &
\multicolumn{1}{|c|}{} &
%\multicolumn{3}{|c|}{\textbf{Veri-Wild (Small)}} &
\multicolumn{2}{|c|}{\textbf{VW-Med}} &
\multicolumn{2}{|c}{\textbf{VW-Large}} \\
\hline
Method & Reference & RRMC & CPD & mAP & R1 & mAP & R1\\ \hline
BUC~\cite{lin2019bottom} & AAAI'19 &  - & \gv{No} & 14.8 & 33.8 & 9.2 & 25.2\\ 
MMCL~\cite{wang2020unsupervised} & CVPR'20 & - & \gv{No} & 19.2 & 39.1 & 14.1 & 33.1\\
SSML~\cite{yu2021unsupervised} & IROS'21 & - & \gv{No} & 20.4 & 43.9 & 15.8 & 34.7\\
%MSCL~\cite{wang2022unsupervised} & SVIP'22 & - & - & - & - & - & - & \bv{30.4} & \bv{57.2} & - \\
\hline
& Speedup & \multicolumn{6}{c}{} \\ \hline
\textbf{Ours (25\%)} & $7.60\times$ & $\mathcal{O}(kN)$ & \gv{No}  & 23.6 & 42.2 & 18.0 & 32.2\\ 
\textbf{Ours (50\%)} & $2.62\times$ & $\mathcal{O}(kN)$ & \gv{No} & \ov{25.6} & \ov{45.7} & \ov{19.8} & \ov{35.5}\\ 
\textbf{Ours (75\%)} & $1.48\times$ & $\mathcal{O}(kN)$ & \gv{No}  & \bv{26.0} & \bv{46.8} & \bv{20.3} & \bv{36.4}\\ 
\textbf{Ours (100\%)} & $1.00\times$ & $\mathcal{O}(kN)$ & \gv{No} & \gv{25.8} & \gv{46.4} & \gv{20.0} & \gv{36.2}\\ 
%\textbf{Ours (best $k$)}  & &  &  &  &  &  &  &  &  & &  & &  \\ 
\hline
\end{tabular}
\end{table}

\subsection{Results and Speedup with LNS}
\label{sec:result_speedup_local_sampling}

We evaluate our proposed LNS, for which we can take different percentages from the whole data at each iteration. %Based on the explanation of Section~\ref{sec:local_neighborhood_sampling}, we take different percentages from the whole data to feed current iterations, which means fewer data is used for feature extraction, Local Re-Ranking, Clustering, and so on. 
In Table~\ref{tab:state_of_art_reid_fully_unsupervised}, we show our results for fully-unsupervised Person ReID with different amounts of data.

With $75\%$ of the data, we obtain the second-best result for \texttt{MSMT17} against other methods that use the whole dataset. When we reduce it to $50\%$ we face a natural performance drop, but even with $25\%$ of the data, we are still on par or even better than top-tier methods. This shows our method can mine useful patterns even with a reduced amount of data.

%which increases back with $25\%$ of the data. We argue that this happens because of the random sampling process, which can sample some regions without discriminant information, which can hinder the performance. However, it is important to note that with $25\%$ of the data, we are still on par or even better than recent top-tier models that consider $100\%$ such as RTMem~\cite{yin2023real} ($60.5\%$ vs $57.1\%$ in R1), PPLR~\cite{cho2022part} ($32\%$ vs $31.4\%$ in mAP), and outperforming by a margin in all metrics CACL~\cite{li2022cluster} and HDCRL~\cite{cheng2022hybrid}. In Market, we see we have a little performance dropping for $75\%$ of the data, and since it is a relatively small dataset of $12,936$ images on training with less identities and variability (i.e., less number of cameras), the model generated degenerated clusters when we reduce it to $50\%$ or $25\%$. However, we argue it is expected since we designed our strategy to operate in large-scale training sets.

In \texttt{Veri} (Table~\ref{tab:state_of_art_vehicle}), the performance drops when considering fewer data, but the difference is small even with $50\%$ of the data. We still achieve competitive performance in comparison to previous methods using the whole dataset, and we use a more memory-efficient Re-Ranking.

In \texttt{VehicleID}, performance is stable across all percentages (Table~\ref{tab:vehicleID}). In the most challenging evaluation scenario (TS = 2400), the drop in mAP and R1 is $1.3$ and $1.6$ p.p., respectively, when going from $100\%$ to $25\%$ of the data. This is significantly lower than we observe in smaller datasets, such as \texttt{Veri} and \texttt{MSMT17}. We expect this to happen since our method was tailored to large-scale scenarios.

%our models drop performance from $100\%$ to $25\%$ of the data by $1.3$ and $1.6$ in mAP and R1 which is significantly lower than the dropping considering the smaller Veri and MSMT17. The dropping is even smaller in the other two evaluation scenarios. Moreover, our $25\%$-model is almost 5 times faster ($4.97\times$) than the $100\%$-model with affordable loss of accuracy (less than $2\%$) in the most complex evaluations scenarios. That is an important remark for large-scale fully-unlabeled datasets where a faster training and deployment is required. We also see speedups in other percentages with even less performance dropping. 

In \texttt{Veri-Wild} (Table~\ref{tab:veri_wild}), our method outperforms the prior art in most metrics with different amounts of data. With $50\%$, we perform better in all scenarios. With only $25\%$, we still outperform prior art by $2.2$ p.p. in mAP, in a harder scenario (Large), and in all metrics in the other scenarios. With this reduced amount of data, the speedup is $7.6\times$ in comparison to training with the whole dataset. With such a high speedup, %and a drop around $2.0$ p.p. in mAP, 
it can be a good trade-off for fast training and deployment. This shows our Local Neighborhood Sampling can effectively reduce the dataset size while keeping superior results in large-scale datasets. 

%The performance difference between the models trained with $100\%$ and $25\%$ are a little higher than for VehicleID, but considering a speedup of $7.60x$ and performance dropping around $2.0$ for mAP, it can be a good trade-off considering a fast training and deployment. This show our proposed Local Neighborhood Sampling can effectively reduce the training dataset size and keep high performance mainly to the large-scale datasets. 

\subsection{Visualization of Results}

Figures~\ref{fig:veri_visualizations} and~\ref{fig:market_visualizations} depict the activated regions for the top-5 retrieved images from the gallery given a query image in \texttt{Veri} and \texttt{Market} datasets, respectively. In the top rows (correct matches), we can see that our model is able to learn fine-grained and point-of-view invariant features. In all images for both Vehicle and Person Re-Identification, just a few regions of the image are strongly activated over the identity (redder regions of the activation maps), showing that our model focuses on specific discriminant parts and is invariant to the background (no activation in any background region). In the bottom rows, we see that our model retrieves images with high visual similarity to the query. It would be hard even for a human to tell they are from different identities. % Despite these errors, our model is still able to detect fine-grained parts under different point of views and resolutions without being influenced by the background. 

% Figure environment removed

% Figure environment removed


\section{Ablation Study}
\label{sec:ablation_study}

We evaluate the impact of each contribution. For more controlled experimentation, when we change one factor of the method, the others remain unchanged. Experiments for the Barlow Twins initialization and the sensitivity to hyper-parameters $\tau$ and $\lambda$ are in the Supplementary Material.

\begin{table}[ht]
\caption{Ablation study. We compare the proposed Local Re-Ranking (LRR) with the Full Re-Ranking (FRR) as well as the influence of the noise-robust $\varepsilon$ scheduling (NR-$\varepsilon$) and Co-Training (CT). The \rv{\underline{ underlined}} results assume an oracle where the best $\varepsilon$ for DBSCAN is selected per dataset.}
\centering
\begin{tabular}{P{0.45cm}|P{0.1cm}|P{0.7cm}P{0.25cm}|P{0.45cm}|P{0.4cm}|P{0.45cm}|P{0.4cm}|P{0.45cm}|P{0.4cm}}
\multicolumn{4}{c}{} &
\multicolumn{2}{|c|}{\textbf{Market}} &
%\multicolumn{2}{|c}{\textbf{MSMT17}} &
\multicolumn{2}{|c}{\textbf{MSMT17}} &
\multicolumn{2}{|c}{\textbf{Veri776}} \\ \hline
& & NR-$\varepsilon$ & CT & mAP & R1 & mAP & R1 & mAP & R1 \\ \hline
\multirow{3}{*}{FRR} & \#1  &  & & \rv{\underline{86.8}} & \rv{\underline{94.4}} & \rv{\underline{42.7}} & \rv{\underline{69.7}} & \rv{\underline{39.7}} & \rv{\underline{84.6}}\\
& \#2  &\checkmark &  & 87.2 & 94.4 & 34.4 & 64.9 & 36.5 & 81.1\\ 
& \#3 &\checkmark & \checkmark & 87.8 & 94.8 & 39.1 & 68.3 & 39.5 & 84.7\\ \hline
\multirow{4}{*}{LRR} & \#4  &  & & \rv{\underline{86.4}} & \rv{\underline{94.1}} & \rv{\underline{42.5}} & \rv{\underline{69.3}} & \rv{\underline{41.6}} & \rv{\underline{85.1}} \\
& \#5 &\checkmark & & 84.8 & 93.3 & 38.8 & 67.1 & 39.8 & 82.8\\
%& \#6 & & \checkmark & \checkmark & - & - & 38.5 & 66.7 & 41.3 &  86.8\\
& \#6 & \checkmark & \checkmark & 85.8 & 94.0 & 43.2 & 70.9 & 41.3 & 86.3\\
\hline
\end{tabular}
\label{tab:ablation_study}
\end{table}


\subsection{Comparison between Local and Full Re-Ranking}

We compare our proposed LRR to FRR in terms of accuracy and time when applied in our proposed pipeline. For fairness, we run both methods on the same machine.

Table~\ref{tab:ablation_study} shows the comparison between LRR and FRR when our contributions are considered: Noise-Robust Density Scheduling (NR-$\varepsilon$), and Co-Training (CT). It is clear our LRR is even more advantageous as the complexity of the dataset increases. When NR-$\varepsilon$ and CT are not used (Line \#1 vs. line \#4 in Table~\ref{tab:ablation_study}), we assume we know the best $\varepsilon$ per dataset. In this case, for \texttt{Market} and \texttt{MSMT17}, our LRR is below FRR just by a small margin and surpasses it in the complex \texttt{Veri776} dataset. Considering all contributions (line \#3 vs. line \#6), our LRR surpasses FRR in \texttt{MSMT17} and \texttt{Veri776}.

% compare both of them in our proposed pipeline and how the performances changes when different parts are considered along with one of them. In general, we see our LRR is below FRR just by a tiny margin or surpass it in a lof of scenarios. For fairness, we compare them in our pipeline when we remove the Noise-Robust Density Scheduling (NR $\varepsilon$) and the proposed Co-Training (CT). Those results are shown in line \#1 vs line \#4. For Market and MSMT17 our LRR is below FRR just by a low margin and surpass it in Veri776, which has more images and more number of cameras than the other two datasets. Considering the whole pipeline (line \#3 vs line \#6) our LRR surpass FRR now in both MSMT17 and Veri776. Similar conclusions can be drawn from the remaining comparison lines. 

We also compare re-ranking execution time when considering LRR vs. FRR (Figure~\ref{fig:FRR_vs_rest}). Corroborated by the theoretical time analysis presented in Section~\ref{subsec:local_reranking_clustering}, our LRR is faster in all datasets, which becomes more evident in the large-scale datasets \texttt{VehicleID} and \texttt{Veri-Wild}, with a Python implementation (LRR-P). LRR-P is approximately $3.14\times$ and $40\times$ faster in \texttt{VehicleID} and \texttt{Veri-Wild}, respectively. 

Targeting practical scenarios, we also developed a Cython implementation of LRR (LRR-C), which boosts, even more, the time efficiency, as shown in Figure~\ref{fig:LRR-P_vs_LRR-C}. This implementation is an improvement over our own Python implementation of LRR and is $25\times$ and $257\times$ faster than FRR in \texttt{VehicleID} and \texttt{Veri-Wild}, respectively. This analysis shows LRR is more time-efficient, theoretically and in practice, which is more suitable for large-scale scenarios.          

%Besides the accuracy, our method also show big time performances improvements as shown in Figure~\ref{fig:FRR_vs_rest}. Following the theoretical time analysis presented in Section~\ref{subsec:local_reranking_clustering}, our LRR (``LRR-P'') has lower time upper bound in all datasets and it becomes more evident in the large-scale datasets VehicleID and Veri-Wild. More specifically, our proposed LRR is approximately $3.14\times$ and $40\times$ faster in VehicleID and Veri-Wild, respectively. Considering practical scenarios, we also release our Cython implementation of LRR which boosts even more the time performance as shown in Figure~\ref{fig:LRR-P_vs_LRR-C}. With this implementation, we improve over our own Python implementation of LRR and we are $25\times$ and $257\times$ faster than the FRR in VehicleID and Veri-Wild respectively. Those results show our LRR has similar or even better performance than the well-know FRR but with a better theoretical and practical time (and memory as shown in Section~\ref{sec:comparison_state_of_art}), which is more suitable for large-scale training datasets.         

% Figure environment removed

Finally, in Figure~\ref{fig:LRR_k}, we show the impact of the nearest neighbors parameter $k$ in \texttt{Market} and \texttt{Veri}, which display contrasting behaviors when $k$ is changed. For \texttt{Market}, performance decreases for $k > 20$ but, in \texttt{Veri}, performance increases with $k$. Since our method is fully-unsupervised, we keep $k = 20$ for all other experiments and datasets.
%However, it is important to point out we have not chosen the best $k$ for each dataset, since our model operates in the fully-unsupervised scenario. So, we set $k = 20$ for all experiments.
% Figure environment removed

\subsection{Impact of the Noise-Robust Density Scheduling}

We verify the impact of the proposed Noise-Robust Density Scheduling. Figure~\ref{fig:eps_sensitivity} shows mAP and R1 in three datasets when using our scheme and with a fixed $\varepsilon$ during training.
% Figure environment removed

The performance of our proposed scheduling scheme is better than considering fixed $\varepsilon$ values for the majority of the tested values. For MSMT17 and Veri, $\varepsilon = 0.65$ or $\varepsilon = 0.7$ are optimal values; however, directly setting those values is unrealistic in fully-unsupervised scenarios. Indeed, when we compare our pipeline with and without the Noise-Robust Density Scheduling (line \#1 vs. line \#2, and line \#4 vs. line \#5 in Table~\ref{tab:ablation_study}), there is a small performance drop since our method does not have oracle knowledge about the optimal clustering parameter for each dataset. Even so, our proposed scheme does not require any grid-searching or manual selection of hyper-parameters.

%We see that our proposed scheduling has performance on pair to the ones considering a fixed $\varepsilon$ along the whole training.  For MSMT17 and Veri, $\varepsilon = 0.65$ or $\varepsilon = 0.7$ achieve the best performances, however directly set those values in our pipeline is unrealistic since our models operates in the fully-unsupervised scenario. Indeed, when we compare our pipeline with and without the Noise-Robust Density Scheduling (line \#1 vs line \#2, and line \#4 vs line \#5 in Table~\ref{tab:ablation_study}) we see a little performance dropping since our pipeline don't have any oracle knowledge about the optimal clustering parameter for each dataset. Even so, our propose scheduling is a little bit lower than those values and it does not require any grid-searching, human intervention or manual selection of the clustering threshold in the fully-unlabeled scenario. 

To compare our Noise-Robust Density Scheduling with other possible schemes, we test four alternatives: only the warm-up stage (Figure~\ref{fig:just_warmup}); only the annealing stage (Figure~\ref{fig:just_annealing}); and two adaptations of the cosine learning rate scheduling~\cite{loshchilov2016sgdr} but applied to the density parameter (Figures~\ref{fig:full_cycle} and~\ref{fig:full_and_half_cycle}). The results are reported in Table~\ref{tab:alternative_scheduling_results}.

%One can think if the proposed scheduling is indeed the best for the considered datasets. Since there are an unlimited number of possible schedulings, it is unfeasible to test all of them against the proposed one. For this reason, we come with four other schedulings. Two of them are based on each behavior presented in the proposed one: the first just has the ``warmup'' stage (Figure~\ref{fig:just_warmup}) and the second just the noise-robust adaptation stage (Figure~\ref{fig:just_noiserobust}). The other two are inspired in the well-known cosine learning rate scheduling proposed in~\cite{loshchilov2016sgdr}, however adapted to the density parameter scheduling in our work (Figures~\ref{fig:full_cycle} and~\ref{fig:full_and_half_cycle}). The results are depicted in Table~\ref{tab:alternative_scheduling_results}.  
% Figure environment removed

\begin{table}[ht]
\caption{Our method's performance with different $\varepsilon$ scheduling schemes. The best results are in \bv{blue}.}
\centering
\begin{tabular}{P{2.8cm}|P{0.45cm}|P{0.4cm}|P{0.45cm}|P{0.4cm}|P{0.45cm}|P{0.4cm}}
\multicolumn{1}{c}{} &
\multicolumn{2}{|c|}{\textbf{Market}} &
\multicolumn{2}{|c}{\textbf{MSMT17}} &
\multicolumn{2}{|c}{\textbf{Veri776}} \\ \hline
& mAP & R1 & mAP & R1 & mAP & R1 \\ \hline
%Scheduling-1 & 85.7 & 93.8 & 46.2 & 73.1 & 41.7 & 87.1\\
%Scheduling-2 & 83.6 & 92.8 & 45.3 & 71.9 & 42.6 & 87.2\\
Only warm up & 85.3 & 93.7 & 41.8 & 70.0 & \bv{42.0} & 86.0\\
Only annealing & 86.5 & 94.2 & 35.9 & 66.1 & 39.1 & 83.7\\
One cosine cycle & 86.0 & 94.1 & 40.0 & 69.3 & 40.2 & 85.5\\
One + half cosine cycle & \bv{86.8} & \bv{94.4} & 43.1 & 70.7 & \bv{42.0} & \bv{87.0}\\ \hline
Ours & 85.8 & 94.0 & \bv{43.2} & \bv{70.9} & 41.3 & 86.3 \\\hline
\end{tabular}
\label{tab:alternative_scheduling_results}
\end{table}

Despite the fourth alternative (One + half cosine cycle) being better in \texttt{Market} and \texttt{Veri}, it is possible to verify that the cosine scheduling is sensible to the number of cycles. Our proposed scheduling outperforms the one-cycle cosine scheduling in all metrics in \texttt{MSMT17} and \texttt{Veri776}, and it is marginally inferior in \texttt{Market}. The first two strategies (Figures~\ref{fig:just_warmup} and~\ref{fig:just_annealing}) can be interpreted as part of the cosine scheduling with just a single half cycle (only \textit{warmup} or only \textit{annealing}). Our strategy performs better in most metrics in comparison to them. Therefore, to achieve the best performance with cosine scheduling, tuning the number of cycles is necessary, which is hard to do in a large-scale fully-unsupervised scenario. Our scheduling, on the other hand, does not require the number of cycles to be set. Based on the noise-robust learning theory, we designed our scheduling to be directly employed in large-scale fully-unsupervised setups.

 



% As expected, each part has its own merits in the training and achieves competitive performance to ours. Despite the ``Full and half cycle'' being better in Market and Veri, it has a lower performance than ours in MSMT17, and we score the second best in Veri776 dataset being below just by $0.7$ in both metrics. In counterpart the ``Full cycle`` is worse than ours in MSMT17 and Veri776. Similar conclusions we can take from the other schedulings. In general, we see that the performance difference is at most $1.0$ in all metrics in all datasets showing our competitiveness to a well-stablished scheduling.  


\subsection{Impact of Co-Training}

The impact of the proposed Co-Training strategy is shown in Table~\ref{tab:ablation_study}. When we remove the Co-Training strategy (lines \#2 and \#5) all backbones are supervised by their own generated pseudo-labels, and there is no knowledge sharing among them. We clearly see a performance drop for all metrics in all datasets. The main reason is that a backbone does not have a chance to correct itself based on the knowledge of others. In this case, any noise is propagated and amplified during training without the possibility of recovering. When co-training is in place, this problem is mitigated as discussed in Sections~\ref{sec:co_training_reid} and~\ref{sec:co_training}. Therefore, our proposed Co-Training yields performance gains for both Re-Ranking strategies without any hyper-parameter tuning or human intervention.

%Each one is optimized using their own pseudo labels, then any noise present in propagate and amplified along the epochs without any chance of correction. When they share the pseudo-labels, the noise can be mitigated as discussed in Sections~\ref{sec:co_training_reid} and~\ref{sec:co_training}. 
%Therefore, our proposed Co-Training yields performance gains for both Re-Ranking strategies, without any hyper-parameter tuning or human intervention. 




\section{Conclusions and Future Work}

We presented a novel method for fully-unsupervised Person and Vehicle Re-identification in large-scale scenarios. Most prior works rely upon costly techniques or consider unrealistic assumptions, making them infeasible for real-world deployment. For instance, they might select dataset-specific hyper-parameters, use re-ranking techniques that scale cubically or adopt ensemble methods that hinder training.

We provide contributions that enable the deployment of re-identification models in large-scale real-world applications, without the necessity for label or dataset-specific information: Local Neighborhood Sampling, Local Re-Ranking, Noise-Robust Density Scheduling, and simple Co-Training.

LNS selects a neighborhood around a random point, reducing the dataset size at each iteration. Local Re-Ranking reduces the memory and time complexities of re-ranking, by decreasing the amount of data necessary at each step, while still producing superior results. Noise-Robust Density Scheduling provides parameter-free clustering, taking into consideration the evolution of the backbones during training, which positively impacts the ability of the clustering step to deal with noisy labeling. Finally, our Co-Training technique enables inexpensive knowledge sharing among the backbones.% by simply permuting pseudo-labels sets.

Our experiments consider datasets that are often not used by other works due to their high complexity. We provide an extensive ablation study showing that our method oftentimes provides the best results with fewer assumptions. It also offers a good trade-off between execution time, memory consumption, and accuracy, especially in large-scale datasets. 

We apply our method to two different domains---person and vehicle re-identification---which are distinct in terms of the target objects and prominent features. This indicates that our contributions can be applied to other domains with similar characteristics, broadening their application.

One aspect that was only superficially explored is the number of backbones and the considered architectures. Our model relies on ResNet50 and DenseNet121 which could be replaced by lighter models, bringing even more gains in terms of memory and execution time. Another interesting exploration relates to knowledge-distilling techniques, aiming to transfer the knowledge from the ensemble to a single backbone. 

%In this work we have found we can achieve state-of-the-art performance in large-scale dataset without relying a per-dataset clustering parameter selection nor in a costly Re-Ranking strategy. Our findings show that it is possible to take advantage of the Re-Ranking strategy considering just a local neighborhood analysis without relying in full point-to-point analysis which reduce the time and memory complexities. We also show that a scheduling for the density parameter on clustering is more appropriate since the feature space presents different degrees of reliability along training and is constantly changing. This avoids the burden of choosing a fixed density parameter for each dataset in a fully-unsupervised context. Finally, our proposed co-training strategy shows we can enjoy the merits of co-training without complex weighting and cross-supervision strategies with just in a label switching strategy, without relying on hyper-parameters or human intervention. 

%We also point out our model does not rely on any dataset-based information, in other words, our model operates for Person and Vehicle Re-Identification without informing it which one it is working with. Moreover, our model does not rely on any labeling or side information. Therefore our model is general enough to be employed in any fully-unsupervised scenario in any machine learning or AI field. We focused in the Person and Vehicle Re-Identification due to its challenges such as high intra-class variations and inter-class similarity. This allows us to validate our solution in truly large-scale scenarios where the labeling is challenging, time consuming, laborious due to the capturing conditions of the identities. Since further problems also face the same challenging our model has potential application in further tasks. 

As a more advanced future study, we intend to employ our solution in event understanding, leveraging its power also to produce contextual information. 
This can be done by correlating different types of targets, like people, vehicles, and places. As these targets are re-identified within an event, we can understand how they are correlated to each other and in time. This is important, for instance, in forensic investigations.   

%we will employ or solution in event understanding in the forensics scenario by aiming to find correlation between people, vehicles and places. In this way we can propose an understanding about the semantics and how the main pieces of the event are correlated to help further investigation.   


\section*{Acknowledgments}
We thank the financial support of the São Paulo Research Foundation (FAPESP), grants D\'ej\`aVu \#2017/12646-3, \#2019/15825-1, and \#2022/02299-2. We also thank the UCCS VAST Lab and Recod.ai Lab for hardware support enabling the experiments conducted in this research. 





%{\appendices
%\section*{Proof of the First Zonklar Equation}
%Appendix one text goes here.
% You can choose not to have a title for an appendix if you want by leaving the argument blank
%\section*{Proof of the Second Zonklar Equation}
%Appendix two text goes here.}

\bibliographystyle{IEEEtran}
\bibliography{IEEEabrv, refs}

%\section{Biography Section}
%If you have an EPS/PDF photo (graphicx package needed), extra braces are
%needed around the contents of the optional argument to biography to prevent
%the LaTeX parser from getting confused when it sees the complicated
% $\backslash${\tt{includegraphics}} command within an optional argument. (You can create
% your own custom macro containing the $\backslash${\tt{includegraphics}} command to make things
%simpler here.)
 
%\vspace{11pt}

%\bf{If you include a photo:}\vspace{-33pt}
%\begin{IEEEbiography}[{% Figure removed}]{Michael Shell}
%Use $\backslash${\tt{begin\{IEEEbiography\}}} and then for the 1st argument use $\backslash${\tt{includegraphics}} to declare and link the author photo.
%Use the author name as the 3rd argument followed by the biography text.
%\end{IEEEbiography}

%\vspace{11pt}

%\bf{If you will not include a photo:}\vspace{-33pt}
%\begin{IEEEbiographynophoto}{John Doe}
%Use $\backslash${\tt{begin\{IEEEbiographynophoto\}}} and the author name as the argument followed by the biography text.
%\end{IEEEbiographynophoto}




\vfill

\end{document}


