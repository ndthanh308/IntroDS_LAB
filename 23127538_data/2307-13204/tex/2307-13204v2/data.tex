\section{Problem Formulation}
We consider the problem of estimating the posterior distribution $P(G^*|O,I)$ given an object point cloud $O$ and a language instruction $I$ specifying the target object class and the target task (e.g., ``Use the \textit{knife} to \textit{cut}" and ``to \textit{drink}, grasp the \textit{mug}"), where $G^*$ is the space of all task-oriented grasps. Each grasp $g \in G^*$ is a parallel-jaw gripper pose $(R, T)$ given in $SE(3)$,  where $R \in SO(3)$ is the 3D orientation and $T \in \mathbb{R}^3$ is the 3D translation of the gripper. To ensure the generalization to novel concepts, we incorporate the semantic knowledge from an LLM $P(G^*|O,I,D)$, where $D$ is language descriptions of the concepts.

Following the convention in previous work \cite{murali2021same}, \textcolor{red}{we factorize the estimation process into two steps:} (1) task-agnostic grasp sampling $P(G|O)$, and (2) task-oriented grasp evaluation $P(S|O, I, D, g)$, where $S$ is the probability of success for $g$. Since the first step is well-studied by prior works\cite{ten2018using}\cite{mousavian20196}, we assume the access to a set of stable task-agnostic grasp candidates and focus on the second step.

\section{Data Generation}\label{data}

\subsection{Description Generation}
\subsection{Instruction Generation}







