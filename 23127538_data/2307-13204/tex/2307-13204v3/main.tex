%%%%%%%%%%%%%%%%%%%%%%%%%%%%%%%%%%%%%%%%%%%%%%%%%%%%%%%%%%%%%%%%%%%%%%%%%%%%%%%%
%2345678901234567890123456789012345678901234567890123456789012345678901234567890
%        1         2         3         4         5         6         7         8

\documentclass[letterpaper, 10 pt, conference]{ieeeconf}  % Comment this line out if you need a4paper

%\documentclass[a4paper, 10pt, conference]{ieeeconf}      % Use this line for a4 paper

\IEEEoverridecommandlockouts                              % This command is only needed if 
                                                          % you want to use the \thanks command

\overrideIEEEmargins                                      % Needed to meet printer requirements.

% add package for hyperlink
\usepackage[colorlinks,linkcolor=blue]{hyperref} 
\usepackage{mathrsfs}
% add package for \mathbb
\usepackage{amsfonts}
\usepackage{amsmath}
\usepackage[flushleft]{threeparttable}
\usepackage{tablefootnote}
\usepackage{multirow}
\usepackage{graphicx}
\usepackage{subfigure}
\usepackage{hanging}
\usepackage{color}
\usepackage{tikz}
\usetikzlibrary{calc,arrows,decorations.markings}
\usepackage{balance}
\usepackage{cite}
\usepackage{booktabs}
% \usepackage{caption}% http://ctan.org/pkg/caption
\usepackage{xpatch}
\usepackage{mathrsfs}
\usepackage{amsmath}
\usepackage{float}
% \usepackage[title]{appendix}
% \usepackage[toc,title,page]{appendix}
\DeclareMathOperator*{\argmax}{arg\,max}
\DeclareMathOperator*{\argmin}{arg\,min}
\makeatletter
\patchcmd\@makecaption{\\}{.~}{}{\fail}
\makeatletter



% \graphicspath{{../},{../image}}
% \graphicspath{{./image/}}

%In case you encounter the following error:
%Error 1010 The PDF file may be corrupt (unable to open PDF file) OR
%Error 1000 An error occurred while parsing a contents stream. Unable to analyze the PDF file.
%This is a known problem with pdfLaTeX conversion filter. The file cannot be opened with acrobat reader
%Please use one of the alternatives below to circumvent this error by uncommenting one or the other
%\pdfobjcompresslevel=0
%\pdfminorversion=4

% See the \addtolength command later in the file to balance the column lengths
% on the last page of the document

% The following packages can be found on http:\\www.ctan.org
%\usepackage{graphics} % for pdf, bitmapped graphics files
%\usepackage{epsfig} % for postscript graphics files
%\usepackage{mathptmx} % assumes new font selection scheme installed
%\usepackage{times} % assumes new font selection scheme installed
%\usepackage{amsmath} % assumes amsmath package installed
%\usepackage{amssymb}  % assumes amsmath packag e installed

\title{\LARGE \bf

GraspGPT: Leveraging Semantic Knowledge from \\ a Large Language Model for Task-Oriented Grasping}


\author{Chao Tang$^{1, 2}$, Dehao Huang$^{1, 2}$, Wenqi Ge$^{1, 2}$, Weiyu Liu$^{3}$, Hong Zhang$^{1, 2}$ \emph{Fellow, IEEE}
% <-this % stops a space
\thanks{$^{1}$Shenzhen Key Laboratory of Robotics and Computer Vision, Southern University of Science and Technology, Shenzhen, China.}%
\thanks{$^{2}$Department of Electronic and Electrical Engineering, Southern University of Science and Technology, Shenzhen, China.}%
\thanks{$^{3}$Institute for Robotics and Intelligent Machines, Georgia Institute of
Technology, Atlanta, United States.}%
}


\begin{document}

\maketitle

%%%%%%%%%%%%%%%%%%%%%%%%%%%%%%%%%%%%%%%%%%%%%%%%%%%%%%%%%%%%%%%%%%%%%%%%%%%%%%%%


\begin{abstract}

The Fast Reciprocal Square Root Algorithm is a well-established approximation technique consisting of two stages: first, a coarse approximation is obtained by manipulating the bit pattern of the floating point argument using integer instructions, and second, the coarse result is refined through one or more steps, traditionally using Newtonian iteration but alternatively using improved expressions with carefully chosen numerical constants found by other authors. The algorithm was widely used before microprocessors carried built-in hardware support for computing reciprocal square roots. At the time of writing, however, there is in general no hardware acceleration for computing other fixed fractional powers. This paper generalises the algorithm to cater to all rational powers, and to support any polynomial degree(s) in the refinement step(s), and under the assumption of unlimited floating point precision provides a procedure which automatically constructs provably optimal constants in all of these cases. It is also shown that, under certain assumptions, the use of monic refinement polynomials yields results which are much better placed with respect to the cost/accuracy tradeoff than those obtained using general polynomials. Further extensions are also analysed, and several new best approximations are given.

\end{abstract}


% Figure environment removed

\section{Introduction}
Automatic 3D reconstruction of clothed humans using image inputs has gained increasing significance due to its potential applications in a wide array of AR/VR scenarios. High-fidelity reconstructions typically depend on sophisticated capture systems, which are developed with dense camera arrays~\cite{collet2015high,joo2015panoptic,joo2018total}, programmable light-stages~\cite{Vlasic2009, guo2019relightables}, and depth sensors~\cite{newcombe2011kinectfusion,DoubleFusion,BodyFusion,dou2016fusion4d,newcombe2015dynamicfusion}. However, stringent capture environments equipped with complex hardware pose significant challenges for consumer-level applications.


In this context, considerable research effort has been dedicated to developing methods that allow for more flexible capture configurations, such as utilizing a few RGB inputs. Among these works, learning implicit functions \cite{iccv2020PIFu, saito2020pifuhd, hong2021stereopifu} has proven effective in achieving highly detailed reconstructions by integrating the advancements of deep neural networks. These methods employ large multi-layer perceptrons (MLPs) to predict the occupancy probability or truncated signed distance function (TSDF) value of every queried 3D point based on its associated local feature, which is extracted from images. They can recover a continuous surface at arbitrary resolutions without topology restrictions.


However, in typical MLP-based implicit networks, the occupancy or TSDF value at each location is solved independently with planar image features, rendering them less capable of addressing challenging cases such as occlusions. Consequently, these methods suffer from generalization and robustness issues, particularly when tackling strong occlusions caused by large motion or multiple interacting humans. 
Some follow-up studies  \cite{zheng2021deepmulticap,zheng2021pamir,huang2020arch} utilize an extra geometric model, SMPL~\cite{Loper2015}, to improve robustness by introducing strong shape priors. 
Their success typically relies on the assumption of geometrical similarity \cite{huang2020arch} between the shape prior and target reconstruction, making them intractable for handling complex cases with loose clothes and sensitive to errors in SMPL model fitting.



%\ping{this paragraph sounds like `TSDF is better than MLP/SMPL, and we use TSDF to solve the problem'. But in Sec 3, we are telling a different story, saying `MLP needs a 3D convolutional encoder'. We need to make these two sections consistent.}\sicong{I think in this paragraph we claim that the TSDF}


%We opt for Trucated Signed Distance Funtion (TSDF) volumetric representations as they are naturally suitable for convolution operations, which have shown remarkable performance for learning hierarchical features on 2D visual perception tasks \cite{SunXLW19}. 
%Meanwhile, TSDF also describes the gradual geometry change around shape surface, which is not reflected by occupancy volume. 

We instead revisit the 3D volumetric representation and resort to 3D convolutional neural networks (CNNs) for feature learning, due to their impressive performance in feature learning and the ability to incorporate spatial context. However, volumetric methods and 3D convolution involve discretization, which might raise concerns regarding whether a discretized volume can preserve subtle geometric details as continuous representations learned in implicit functions. We investigate the relationship between volume resolution and quantization error on synthetic data by converting target mesh objects to TSDF volumes, as shown in Figure~\ref{fig:quantization_error}. We observe that the quantization errors are significantly reduced by increasing volume resolution and become nearly negligible when reaching a relatively high resolution (e.g., 512 or higher). In other words, achieving fine-detailed reconstruction is not supposed to be restricted by the use of volume representations as long as a proper volume resolution is utilized. Therefore, we present a method with high-resolution feature volumes, e.g., 256 and 512, while traditional volumetric methods \cite{varol18_bodynet,gilbert2018volumetric} are often limited to much lower resolutions, such as 32 or 128.



On the other hand, an increase in volume resolution may lead to a cubic growth of memory overhead \cite{8100085}. Reducing memory costs while guaranteeing the granularity of volumetric representations is necessary for pursuing high-quality reconstruction. Thus, we adopt a coarse-to-fine approach and cull away irrelevant voxels to build a sparse high-resolution feature volume. At the coarse level, the network computes an initial TSDF by applying a U-Net with sparse 3D CNN \cite{3DSemanticSegmentationWithSubmanifoldSparseConvNet} on the sparse feature volume, which is carved by a visual hull. Through our experiments, it turns out that more than 95\% of the volume grids are discarded by the visual hull culling, making the sparse 3D CNN efficient. At the fine level, the network focuses on a narrow band near the zero-level set of the initial TSDF and discretizes the narrow band with smaller voxels. By employing this narrow-band culling, we further shrink the sampling space, resulting in a relatively small range of grid numbers (usually 300K--500K in our experiments) even with a high volume resolution of 512. The remaining voxels in the narrow band are associated with features that fuse high-frequency information from the computed normal maps upon the low-frequency shape from the coarse level to compute the TSDF at high resolution. The final mesh is then extracted from the TSDF using the Marching-Cube algorithm ~\cite{Lorensen87marchingcubes}.
% Different from the u-net sturcture to preserve global topology context, we then apply a shallow 3dcnn to compute the final TSDF $D_{final}$ which contain more local geometry detail.




% \ping{this paragraph can be expanded. It is an important contribution and often ignored by other works. stress on the novel idea of regressing blending weights instead of colors}

In addition to geometry, high-quality mesh texture is also a crucial factor contributing to visual appearance. Directly computing a color field in 3D space, as in \cite{iccv2020PIFu}, struggles to capture high-frequency texture details, while the neural radiance field (NeRF) \cite{yu2020pixelnerf} or the DoubleField~\cite{shao2022doublefield} require expensive per-instance optimization and are often unstable for sparse input images. In contrast, we adopt an image-based rendering approach to compute a texture atlas map, which is efficient and widely supported in existing computer graphics tools. 
Specifically, we compute a blending weight at each 3D point on the mesh surface to determine its color as a weighted average of the colors at its image projections. The blending weights can be computed at a relatively coarse resolution, e.g., 512 volume resolution in our case, and leave texture details to the high-resolution images, such as 1K or 2K. Unlike previous methods that generate blurry texturing results under sparse input, our method generalizes well on both synthetic and real data with just a few input views. 
Figure~\ref{fig:teaser} shows two examples reconstructed by our method. Despite the challenging garment, pose, and occlusion, our method recovers faithful shape, normal, and texture on the right.

%with a wide variety of poses and clothing styles, and it is also adaptive to handle input image with arbitrary resolutions.
%\sicong{For this concern we claim that when the resolution of dicretized volume meets certain threshold (which is 256 in our experiment), the quantization error can be neglected.} 



In summary, the main contributions of this paper are as follows:
\begin{itemize}
\vspace{-0.1in}
  \item 
  We revisit the 3D volumetric representation and demonstrate that it can support clothed human reconstruction with equal or even better performance compared to implicit representation. 
  \item 
  We develop a memory and computation-efficient method for high-resolution volumetric reconstruction using sophisticated sparse 3D CNN, coarse-to-fine estimation, and voxel culling by visual hull and narrow bands. 
  \item 
  We introduce a novel method to compute a texture atlas map, which captures rich appearance details from high-resolution input images.
  \item 
  We achieve impressive results on standard benchmark datasets Twindom and MultiHuman, significantly reducing the point-2-surface (P2S) precision to approximately 0.2cm from just six input views, with more than $50\%$ error reduction compared to the state-of-the-art methods, including DoubleField~\cite{shao2022doublefield} and PIFuHD~\cite{saito2020pifuhd}.
\end{itemize}

\section{Related Work}
\label{appsec: related work}
Bayesian causal discovery literature has primarily focused on inference in linear models with closed-form posteriors or marginalized parameters. Early works considered sampling directed acyclic graphs (DAGs) for discrete~\cite{cooper1992bayesian, madigan1995bayesian, heckerman2006bayesian} and Gaussian random variables~\cite{friedman2003being, tong2001active} using Markov chain Monte Carlo (MCMC) in the DAG space. However, these approaches exhibit slow mixing and convergence~\cite{eaton2012bayesian,grzegorczyk2008improving}, often requiring restrictions on number of parents~\cite{kuipers2017partition}. %Alternative exact dynamic programming methods are limited to small settings~\cite{koivisto2012advances}. 

Recent advances in variational inference~\cite{zhang2018advances} have facilitated graph inference in DAG space, with gradient-based methods employing the NOTEARS DAG penalty \cite{zheng2018dags}.\cite{annadani2021variational} samples DAGs from autoregressive adjacency matrix distributions, while \cite{lorch2021dibs} utilizes Stein variational approach \cite{liu2016stein} for DAGs and causal model parameters. \cite{cundy2021bcd} proposed a variational inference framework on node orderings using the gumbel-sinkhorn gradient estimator \cite{mena2018learning}. \cite{deleu2022bayesian,nishikawa2022bayesian} employ the GFlowNet framework \cite{bengio2021gflownet} for inferring the DAG posterior. Most methods, except\cite{lorch2021dibs} are restricted to linear models, while \cite{lorch2021dibs} has high computational costs and lacks DAG generation guarantees compared to our method.
% at least quadratic scaling complexity, both with respect to the number of nodes (due to the DAG penalty) as well as number of posterior samples. Our proposed approach instead has linear complexity with respect to number of posterior samples and does not require any additional DAG penalty.     

In contrast, \emph{quasi-Bayesian} methods, such as DAG bootstrap \cite{friedman2013data}, demonstrate competitive performance. DAG bootstrap resamples data and estimates a single DAG using PC \cite{spirtes2000causation}, GES \cite{chickering2002optimal}, or similar algorithms, weighting the obtained DAGs by their unnormalized posterior probabilities. Recent neural network-based works employ variational inference to learn DAG distributions and point estimates for nonlinear model parameters \cite{charpentier2022differentiable,geffner2022deep}.

% \lstMakeShortInline[columns=fixed]@
% Figure environment removed
\lstDeleteShortInline@

In this section, we describe how we collect examples for learning repair strategies without any version-controlled data. Specifically, we first detect \safeprogs and corresponding witnesses using \sawitnessfull (witnesses are sanitizers and guards that protect from vulnerabilities)  in Section~\ref{subsec:sa-witness}. Using these witness annotations, we generate unsafe programs and \textit{edits} from the \safeprog using a \textbf{witness-removal} step (Section ~\ref{subsec:witness-removal}). In the following, we define terminology for the \astree  data-structure we operate on. 


\astree refers to the abstract syntax tree representation of programs, augmented with data flow edges and annotations for sources, sinks, sanitizers, guards, witnesses etc. 
An \astree is a five-tuple 
$\langle \mathcal{N},\mathcal{V},\mathcal{T},\mathcal{E}, \mathcal{A} \rangle$, where:
\begin{enumerate}
\item
$\mathcal{N}=\{\mathit{id}_0,\ldots\mathit{id}_n\}$  is a set of nodes, where  $\mathit{id_i}\in\mathbb{N}$ for 
$ 0 \leq i \leq n$.
\item
$\mathcal{V}$ is a map from nodes to program snippets
represented as strings. For a node $n$, we have that $\mathcal{V}(n)$ is a string representing the code snippet associated with $n$
\item
$\mathcal{T}$ is a map from nodes to their types defined by 
 \sa~\cite{codeqlast}. For example, \callexpr is the type of a node representing a function call, \indexexpr is the type of a node representing an array index, and \blockstmt is the type of a node representing a basic block of statements.
\item
$\mathcal{E}$ is a set of directed edges.
Each edge is of the form $(n_1,n_2,\edgetype,z)$, where
$n_1$ is a source node, $n_2$ is a target node, 
$\edgetype \in \{\T{SynParent}, \T{SynChild}, \T{SemParent},
\T{SemChild} \}$ denotes the relationship from 
$n_1$ to $n_2$, as one of syntactic parent, syntactic child, semantic parent or semantic child,
and $z\in\mathbb{Z}$ is the index of $n_2$ among $n_1's$ children if this edge is a child edge, and $-1$ if the edge is a parent edge. 
\item
$\mathcal{A}$ is a set of annotations associated with each node. The annotations are from the set $\{\T{source},
\T{sink},\T{sanitizer},\T{guard}$,\T{witness}\}. We also refer to annotations using predicates or relations. For instance, for a node $n$, if an annotation  $\T{source}$ is present, we say that
the predicate $\T{source}(n)$ is true.
\end{enumerate}

%\setlength{\grammarindent}{5em} % increase separation between LHS/RHS

% Figure environment removed



A {\em traversal} or a {\em path} in an \astree is a sequence of edges $e_0,\ldots,e_{i-1},e_i,\ldots ,e_k$ such that the target node of $e_{i-1}$ is also the source node of $e_i$, for all $i\in\{1,\ldots,k\}$. That is, $e_{i-1}$ is of the form $(\_,n,\_,\_)$ and $e_i$ is of the form $(n,\_,\_,\_,\_)$. The source node of $e_0$ is the source of this path and the target node of $e_k$ is the target of the path.


\lstMakeShortInline[columns=fixed]@
%Note that these additional edges can capture long-range dependencies in programs. E.g. edge 4 in Figure ~\ref{fig:unsafememberex} links two nodes across the function boundaries. 
Figure~\ref{fig:example1-pdg} depicts a partial \pdg corresponding to the unsafe program in Figure~\ref{fig:unsafememberex}. Each oval corresponds to an \astree-node containing a type $\tau$ and an associated value. The dark edges denote the syntactic child edges. For example, the oval with value @foo(data)@ is an \astree-node with type \callexpr and has two children -- @foo@ and @data@, both with the type \varexpr. 
%Similarly, the \blockstmt node on the top refers to the function body between Line~\ref{lst:line:handlers-run} and Line~\ref{lst:line:handlers-run-end} in Figure ~\ref{fig:unsafememberex}. As the body of a function block can contain a variable number of children, we link to @handlers[callerId](data);@ as the k-th child of the \blockstmt. 
The semantic child edges are at the bottom in cyan. These edges correspond to the ones depicted in cyan in Figure ~\ref{fig:unsafememberex}. 
\lstDeleteShortInline@

%TODO:FIX THIS

%With this simplification, 
If $\prog$ is an \pdg then
we use  $\prog.\mathtt{source}$ to denote the source node, $\prog.\mathtt{sink}$ to denote the sink node, and $\prog.\mathtt{witness}$ to denote the witness node.
If the program has several sources, sinks and sanitizers then we generate a separate \pdg for each $(\mathtt{source},\mathtt{witness},\mathtt{sink})$ triple.
For a node $n$, its syntactic parent is $n.\mathtt{parent}$, syntactic children are $n.\mathtt{children}$, semantic parent is $n.\mathtt{semparent}$, and semantic children are $n.\mathtt{semchildren}$.

%\input{ql.tex}

\subsection{Static Analysis Witnessing}
\label{subsec:sa-witness}

\newcommand{\DMethodjudge}[1]{\texttt{#1(}\checknextarga}

% Figure environment removed

%\naman{TODO - sell this more as technique to work with any \sa tool ; our master query is a general framework implemented in \codeql that can work for any vulnerability -- easily extendable to other languages }
In this section, we show how to repurpose \sa tools to generate witnesses.
\sa tools perform dataflow analysis to check for rule-violations in programs. They use pattern matching to identify known sources, sinks, sanitizers, and guards. For commercial tools, these patterns are implemented (and continuously updated) manually by developers and encode this domain knowledge. Next, 
%these patterns are used to detect sources, sinks, sanitizers, and guards in programs and
\sa checks if there exists a flow between a source and a sink that does not cross a sanitizer or guard. We capture this formally in Figure~\ref{fig:judgements} (top two rules), and explain the notation used in it below.

\sa tools encode domain knowledge about the vulnerability by annotating nodes as \T{Source}, \T{Sink}, \T{Sanitizer}, and \T{Guard}. %These relations operate on the set of dataflow nodes in the programs.
So \DMethod{Source}{\I{n}}\ is true iff the node \I{n} is a \textit{source} node for a vulnerability. Next, \sa tools perform dataflow analysis by defining the relation \DMethod{SemChild}{$n_1$}{$n_2$}\ which is true iff there is a \taintpropedge between $n_1$ and $n_2$. Then the \DMethod{Vulnerability}{$n_1$}{$n_2$}\ relation can be defined as:
\begin{enumerate}
    \item $n_1$ and $n_2$ are source and sink nodes (\DMethod{Source}{$n_1$}\ and \DMethod{Sink}{$n_2$}\ are true)
    \item There exists a \textit{path} between $n_1$ and $n_2$ which is free of sanitizers or guards (\DMethod{SanGuardFree*}{$n_1$}{$n_2$}\ is true). A path is free of sanitizers and guards iff every \textit{edge} in the \textit{path} is free of sanitizers and guards. An edge between $n_1$ and $n_2$ is considered free of sanitizers and guards (\DMethod{SanGuardFree}{$n_1$}{$n_2$}\ is true) iff $(n_1, n_2, \_, \T{SemChild}) \in \mathcal{E}$ and neither of $n_1$ or $n_2$ is a sanitizer or a guard
\end{enumerate}

Here, we make the following observation - \emph{this domain knowledge present in these annotations and relations is helpful beyond just detecting vulnerabilities}. For instance, simply using the sanitizer relation allows us to query the different kinds of sanitizers domain experts have specified. We use this observation to discover \emph{\safeprogs} i.e., programs having a source, sink, and a sanitizer or guard that \textit{blocks} the \taintprop or, in simpler terms, make the program safe. In addition, we also detect the corresponding sanitizers or guards in the programs and refer to them as \textit{witnesses} because they serve as the evidence of making the program safe. We call this procedure \sawitnessfull (abbreviated as \sawitness). 
We define this as the \T{Witness} relation in Figure~\ref{fig:judgements} (bottom two rules). Specifically, \DMethod{Witness}{$n_1$}{$n_3$}{$n_2$}\ is defined as:
\begin{enumerate}
    \item $n_1$ and $n_2$ are source and sink nodes (\DMethod{Source}{$n_1$}\ and \DMethod{Sink}{$n_2$}\ are true)
    \item There exists a node $n_3$ such that it satisfies \DMethod{SanGuardInMid}{$n_1$}{$n_3$}{$n_2$}. \DMethod{SanGuardInMid}{$n_1$}{$n_3$}{$n_2$}\ is true iff there exists a \T{SemChild}
    %\naga{notation for flow inconsistent with (2) above} 
    path between $n_1$, $n_3$, between $n_3$ and $n_2$, with the additional constraint of $n_3$ being a sanitizer or guard. 
\end{enumerate}

The difference between the \T{Vulnerability} relation (which \sa populates) and \T{Witness} relations (which we want to find) is highlighted in {\color{red} red} and {\color{ForestGreen} green}. Notice that while defining the \T{Witness} relation, we simply use the existing relations that define the \T{Vulnerability} relation. Thus, we argue that \sawitness can be implemented on top of \sa by using the intermediate relations that \sa is computing.
%for every pair of source and sink, they track taint through a taint-flow analysis. If there is a flow from a source to a sink that does not go through a sanitizer or guard, then the source-sink pair is reported as vulnerable.

%We make the following observation - \emph{the patterns defined by experts encodes domain knowledge which can be used for use cases beyond just detecting vulnerabilities}. For instance, we can use the sanitizer patterns to search for all sanitizers in source-code. In this work, we use this idea to detect \safeprogs, which we define as programs having a source, sink, and a sanitizer or guard that blocks the \unsure{flow} or in other words, makes the program safe.  \naman{highlighted part of Figure somethings shows the difference between semantics of witnessing vs traditional semantics}

%We realize the following -- the set of patterns of sources, sinks, and sanitizers are useful beyond detecting vulnerabilities. We override the existing static analysis query that detects unsafe programs and use these encoded sanitizers for detecting sanitizers and guards in programs. Specifically, in the existing query that detects unsafe programs, we modify the taint-propagation steps to propagate taints through sanitizers and guards and then use static analysis to then find these dataflows containing sanitizers and guards. Thus, we can directly find the safe programs containing these \textit{witnesses} of safety. 
%Once such a dataset is collected, we use these witnesses to convert safe  to unsafe  and thus obtain paired examples for learning repair strategies (Section~\ref{subsec:witness-removal}). 

\lstMakeShortInline[columns=fixed]@
%We instantiate our \sawitness technique using \codeql~\cite{a}. It is an open-source \sa tool that allows implementing custom static analysis as queries in a high-level object-oriented extension of datalog. These queries usually contain a \Verb|select from where| statement that allows querying the program database. \codeql maintains these patterns of sources, sinks, sanitizers, and guards using \Verb"Configuration" classes. Consider an example of a simplified \Verb"Configuration" for \xss vulnerability in Figure~\ref{fig:configuration}. It defines a set of predicates @isSource@, @isSink@, @isSanitizer@, and @isGuard@. These predicates are written manually by \codeql authors and improved through rich community support\footnote{\url{https://github.com/github/codeql}}. With this configuration, vulnerabilities are reported by selecting source-sink pairs such that the @cfg.hasFlow@ predicate is true for the source, and the sink. This predicate is internally defined by \codeql and uses the patterns defined in the configuration to check for the presence of vulnerability-causing dataflows. %\spsays{Showing corresponding programs will be useful}

%Now, we demonstrate the static-analysis-witnessing approach for collecting examples of \safeprog and witnesses in Figure~\ref{fig:safe-configuration}. Specifically, we inherit from the existing configuration, using the same @isSource@ and @isSink@ predicates while overriding the @isSanitizer@ and @isGuard@ predicates to @none()@. This ensures that all the source and sink pairs are detected independent of the presence of sanitizers/guards between them. Finally, to detect our witnesses, we define the @isWitness@ predicate which uses the @isSanitizer@ and @isGuard@ predicates from the original configuration. Specifically, witnesses are defined as sanitizers/guards that lie between a source-sink pair. Finally, to report \safeprog and witnesses, the @cfg.hasFlow@ predicate is used to select all valid source-sink pairs and the corresponding witnesses are detected via the @isWitness@ predicate. Note that Figure~\ref{fig:configuration-vs-safe-configuration} depicts the key idea behind our approach in a simplified view. In practice, additional measures need to block the taint propagation internally and we share the actual \codeql queries used as part of the Appendix~\ref{app:codeql-queries}.


\subsection{Witness Removal}
\label{subsec:witness-removal}

We obtain \safeprogs and witnesses by applying \sawitness to a snapshot of a codebase. Recall that the witnesses block the flow between a source and a sink and thus help make programs  \textit{safe}. Hence, removing these witnesses will make the programs unsafe. Recall also that the witnesses are either sanitizing functions of the form @sanitize(taintedVar)@ or guards of the form @if checkSafe(taintedVar) {executeSink(taintedVar)}@. %Usually, they are used only for ensuring the safety of programs and are not critical to the functionality of programs. Therefore, 
We implement witness-removal perturbations  that precisely remove the guard-checks and sanitizer-functions. Note that our goal here is to generate unsafe programs and corresponding edits that enable learning repair strategies that insert such witnesses. So, while we generate the unsafe programs by perturbation, they should look structurally similar to natural unsafe programs written by the developers, otherwise the repair strategies learned on this artificially generated data through perturbations would not generalize to code in the wild. 
%At the same time, minor syntactic-semantic issues in parts of unsafe programs not directly relevant to the vulnerability or repair do not impact learning.
\lstDeleteShortInline@

% Figure environment removed

\lstMakeShortInline[columns=fixed]@

\input{witnessremoval.tex}

We use \rmSan and \rmGuard functions to programmatically remove the witnesses. A high-level sketch of these functions is illustrated in Figure~\ref{fig:remove-functions}. The functions use the structure of the corresponding \astree (node types $\tau$) to decide how to remove witnesses. Consider the \rmGuard function. It first computes the parent (\witnesspar) and grand-parent (\witnessparpar) of the witness guard condition. Then if the type of \witnesspar is \ifstmt (i.e., program is of the form @if (witness) body@ then we modify the \astree edge from \witnessparpar and \witnesspar to instead point to the body of the \ifstmt (index 1 child is body of \ifstmt). Similarly, if the type of \witnesspar is \binaryexpr with operator @&&@ (i.e. of the form @if (otherCond && guard)@ or @if (guard && otherCond)@) then we again modify the edge from \witnessparpar and \witnesspar to instead point to the non-guard child of \binaryexpr (@otherCond@ in the example). Note that since \binaryexpr has 3 children, the index of non-guard child is index of guard-child subtracted from 2. 
Figure~\ref{fig:witness-removal} depicts this removal on the \astree level, where the syntactic edges in red are removed and the syntactic edges in green are inserted.
In the end, the functions returns a tuple of the \pdg of the unsafe program ($\prog_{unsafe}$), \pdg of the safe program ($\prog_{safe}$)
and an edit object (\edit) which stores


\begin{enumerate}
    \item \astree for the removed witness (referred to as \editprog)
    \item location in the \pdg where the witness is removed (referred to as editloc
    %\naga{shouldn't it be editloc to be consistent with (1)?} 
    or \editloc)
    %\item an enum (\insertsc or \replace) depending on whether \concedit is inserted or replaced 
\end{enumerate}

Since $\prog_{unsafe}$ and edit-object can generate the safe program, we only propagate the unsafe programs and edits as the output of this step. Applying \rmGuard function to the safe program in Figure~\ref{fig:safememberex} removes the \ifstmt on Line~\ref{lst:line:fix-start} while preserving the @handlers[callerId](data);@ statement and in fact produces the unsafe program in Figure~\ref{fig:unsafememberex}. Additionally, it  returns the removed witness guard  @if handlers.hasOwnProperty(data.id){ ... }@ as the \editprog and \blockstmt (blue oval in Figure~\ref{fig:example1-pdg}) as the edit location \edit.editloc. Figure~\ref{fig:example1-editprog} shows the \astree for the \editprog containing the \ifstmt. 
The dashed line and dark circle correspond to the \textit{removed} \astree edge between the \blockstmt and the \expr @handlers[callerId](data)@. 

Note that Figure~\ref{fig:remove-functions} provides a high-level sketch of witness-removal and elides over implementation details that are required to make it work for real \js programs. We discuss these issues in the implementation section (Section~\ref{subsec:impl:witness-removal}).% and include the full implementation as part of supplementing source code\naga{we should make sure we are doing these, else remove this sentence}. 
%. In practice, we need implement such decisions more carefully to cover other traditional cases in which guards occur and we document them in the supplementing source code.
\lstDeleteShortInline@

%\naman{add examples $\dots$ } \spsays{do we re-run codeql on this generated bad program? -- NO (naman)}



In this section, we describe how to learn repair strategies from the  unsafe programs and edits collected in Section~\ref{sec:data}. We define a \dsl (Section~\ref{subsec:dsl}) to express repair strategies that take an \pdg of an unsafe program  as input and generate a safe program as output. The DSL is expressive and can even express bad strategies that don't generalize well to programs in the wild. We provide examples of such bad strategies and good strategies that generalize well  (Section~\ref{subsec:examples}). We learn good repair strategies  in a data-driven manner using an example-based synthesis algorithm (Section~\ref{subsec:synthesis}). %Finally, given a new unsafe program and a set of learned repair strategies, we apply these strategies and generate  candidate repairs (Section~\ref{subsec:applying}).


%Our goal is to use the collected data to learn high-level general repair strategies. We learn these repair strategies over a joint representation of the \astree with the annotations inferred from the \sa tool (the representation referred to as \pdg ahead).  These inferred \sa tool annotations allow us to take the advantage of rich semantic information while performing \unsure{repairs}. Figure ~\ref{fig:example1-pdg} shows an example \pdg corresponding to the unsafe code shown in Figure ~\ref{fig:unsafememberex}. We develop a powerful \dsl that can utilize the annotations in the \pdg structure and learns repair strategies using a deductive synthesis algorithm. More specifically, strategies in this \dsl operate over the \pdg structure of unseen code-snippets and suggest appropriate edits correspondingly. \aksays{The following sentence can be removed if space becomes a constraint.} Section~\ref{subsec:dsl} describes the \dsl, Section~\ref{subsec:synthesis} talks about the synthesis algorithm, and Section~\ref{subsec:applying} demonstrates strategies in this \dsl can be applied. 


\subsection{\dsl for repair strategies}
\label{subsec:dsl}
We introduce a novel \dsl to express repair strategies in Figure~\ref{fig:fixing-dsl}.
%that use the knowledge of program semantics annotated on \pdg instead of just using the syntactic program structure and in-turn are more expressive and generalize better. These strategies take the an unsafe-program as input and return candidate repair programs by performing tree-edit-operations.
At a high level, the strategies define a three-step process where  they provide a computation to identify the edit-location node \editloc, a computation to identify the child index $\editindex$ of \editloc where repair happens, and a computation to generate the AST that must be placed at  index $\editindex$ of \editloc for the repair. The main part of these computations involve traversing paths of the input unsafe program \prog.
%The edit-operation can either be inserting a syntactic-child at \editloc (\insertsc) or replacing a syntactic-child with another tree at \editloc (\replace). The index at the \astree-node $\editloc$ where the insertion or replacement occurs is called the edit-index (\editindex). The tree that is inserted or replaces another existing tree at the \editloc is materialized hierarchically for the given example by defining abstract program structure using a combination of constant structure and references to \astree-nodes in the existing program \prog. These \astree-nodes are called reference-locations (\refloc). To find these locations (\editloc, \refloc) in a given program, the strategies abstractly store \textit{traversals} which materialize into a \textit{concrete} \astree-node in the given programs.  

%\naman{todo - talk about traversal in the introduction, background etc.}
\input{dsl}

%We present our \dsl in Figure ~\ref{fig:fixing-dsl}. 
%The DSL is a list of definitions for various non-terminals in the grammar. For each non-terminal, we define a corresponding type and a set of production rules. Each production rule is either a fixed expression, or an operator applied to other non-terminals or fixed-expressions in the grammar.  
The top-level production rule of the DSL defines strategies, \strategy, with type \newtextsc{Strategy}. 
%A \node is either the source node (\prog.source) or an application of \traversal on another \node. 
\gettraversal, \getclauses, and \getindex are all functions that take a \node $n$ as input and return a \node, \bool, and \integer as output respectively. The edit-AST, \eastree, is similar to a syntactic variant of \astree (i.e. no semantic edges) which we define in Section~\ref{sec:data} with one addition. It has reference nodes that, when applying the strategy to the input \pdg of \prog, are materialized from sub-trees of this \pdg, where the root nodes of these sub-trees are identified by traversing paths in the input. 
%Finally, edge-type (denoted by \edgetype) is an enumeration describing the type of edge, i.e. syntactic or semantic, and parent or child, as defined in Section~\ref{sec:data}. 

% Given these types, we now define the operators used in our \dsl. 
The strategy \strategy is of two types, \insertsc or \replace. \DMethod{Insert}{\I{L}}{\I{I}}{\I{O}}\ declaratively expresses the computation that computes the edit-location \editloc by traversing the path supplied in \I{L}, then computes \editindex, the index of edit-location,  by evaluating \I{I}(\editloc), and inserts the materialization of \I{O} as a syntactic-child \astree at index \editindex of the edit-location \editloc. \DMethod{Replace}{\I{L}}{\I{I}}{\I{O}}\ is similar and performs a replacement instead of an insertion.
%computes the \editloc and \editindex, and replaces syntactic-child of \editloc at \editindex with \I{O}.
%The insertion and replacement operations modify the nodes $\mathcal{N}$ and edges $\mathcal{E}$ of the \astree (Figure~\ref{fig:astsyntax}) appropriately. 

\node (\I{L}) is either the node corresponding to the source of vulnerability (\prog.source) or the target of the path corresponding to the traversal \DMethod{ApplyTraversal}{L}{\I{F}$_k\ o\ $\I{F}$_{k-1}\ o\ \cdots$\I{F}$_0$}. 
Here, each \I{F}$_i$ 
is a function that takes a node 
$n$ as input, performs a traversal from $n$, and returns the traversal's target node $n'$. 
Thus, \T{ApplyTraversal} can be recursively defined as \DMethod{ApplyTraversal}{\I{F}$_0$(L)}{\I{F}$_k\ o\ $\I{F}$_{k-1}\ o\ \cdots$\I{F}$_1$}\ if $k>0$ and \I{F}$_0$(L) otherwise. 

\newtextsc{GetTraversal} (\I{F}) defines a function that takes a node $n$ and returns a node $n'$ reachable from $n$ and can be of two types. Given $n$, the \DMethod{GetEdge}{\I{ET}}{\I{I}}\ operator first finds the possible single-edge traversals of type \I{ET} and indexes it using \I{I}. Specifically, if edge type \I{ET} is a parent then it returns the parent of $n$. Otherwise, 
it finds a set of $N$ of nodes that are connected with $n$ via the edge type \I{ET}, i.e., $N = \mathcal{E}(n, \I{ET})$, and returns the node $N[I(\I{n})]$ at the index given by $I$. In contrast, $\DMethod{GetKleeneStar}{\I{ET}}{\I{C}}(n)$  performs a \newtextsc{KleeneStarTraversal} that iteratively traverses edges of type \I{ET}, staring from input node $n$, until it reaches an edge whose target  node $n^{i}$ satisfies the condition defined by the clause \I{C}. Formally, \newtextsc{KleeneStarTraversal} can be defined recursively as $KE(n_1,ET,C) = \I{C}(n_1)? n_1 : \left(let\ t\in\mathcal{E}(n_1,ET)\ in\ KE(t,ET,C)\right)$. Here, the node $t$, which is target of an edge with source $n_1$ and type $ET$,  is chosen non-deterministically and our implementation resolves this non-determinism through a breadth-first search.
%A \traversal is a relation between nodes $n_1$ and $n_2$ such that there is an edge or a sequence of edges between them. 

\newtextsc{GetIndex} (\I{I}) defines a function that takes a node $n$ and returns a \integer. It is either a constant function that returns a fixed integer $z$ or a \DMethod{GetOffsetIndex}{\I{L}, \I{z}}. \DMethod{GetOffsetIndex}{\I{L}, \I{z}}\ takes a node $n$ as input and returns an integer $DO(n,L)+z$, where $DO(n_1,n_2)$ returns the index of syntactic child of $n_2$ who is a syntactic ancestor of $n_1$. 

\eastree (\I{O}) defines the edit \astree with reference nodes which, given an input program \prog, are materialized to a concrete \astree. The \eastree can either be a \T{ConstantAST} or a \T{ReferenceAST}. Specifically, \DMethod{ConstantAST}{$\tau$}{\I{value}}{\I{O}$_1$}{\I{O}$_2$}{$\cdots$}{{\I{O}$_k$}}\ returns an \eastree that has a type $\tau$, string representation \I{value}, and is recursively constructed with sub-trees \I{O}$_1 \cdots$ \I{O}$_k$ as syntactic children, each of which can either be a \T{ConstantAST} or a \T{ReferenceAST}. The \DMethod{ReferenceAST}{\I{L}}, when applying the strategy, finds a node $n$ in \prog by traversing the path described in \I{L} and returns a copy of the (syntactic) sub-tree of \prog rooted at $n$. %Next, we show examples of strategies written in this DSL and how to learn them automatically.

% Finally, note that the traversals can be composed by applying multiple \T{ApplyTraversal} operators sequentially. We use this key insight into developing our learning from examples setup. 
\newcommand{\newwrapbox}[2]{\adjustbox{margin=1pt 1.3pt, bgcolor=white, frame=1pt, cframe=#1, color=#1}{#2}}
% Figure environment removed


\lstMakeShortInline[columns=fixed]@
\subsection{Example of strategies in our \dsl}
\label{subsec:examples}
Figure~\ref{fig:repair-strategy-ex1} describes   two possible repair strategies that are sufficient to repair the motivating example in Figure~\ref{fig:vulnerabilty-example1}. We first describe the good strategy in Figure~\ref{fig:strat1}, referred to as \strategyone,  and then compare it with the bad strategy \strategytwo in Figure~\ref{fig:strat2}. 

Given the program \prog in Figure~\ref{fig:vulnerabilty-example1}(a) as input, the strategy \strategyone
performs a replacement at index \I{I} of edit-location $L_e$ with the materialization of \I{O} (line 20 of \strategyone).
This process requires first finding the "semantic location" node \semloc. %The semantic location for \prog is shown in red in Figure~\ref{fig:example1-pdg}.
To this end, the strategy 
first  traverses a path from the node annotated as \T{source} by \sa  using \DMethod{GetKleeneStar}\ in Line~\ref{lst:line:semkleene} of \strategyone.  This \newtextsc{KleeneStarTraversal} starts from \T{source}, traverses semantic dataflow edges, and stops at a node 
corresponding to an identifier being used as the function name in a function call. 
 For the input program $P$, the traversal takes the semantic-child-edges 1-7 (Figure~\ref{fig:example1-pdg}) and stops at @foo@ in Line~\ref{lst:line:callerId-sink} of Figure~\ref{fig:vulnerabilty-example1}(a). Next, to reach the edit-location $L_e$, the strategy uses a \newtextsc{KleeneStarTraversal} that starts from \semloc, traverses syntactic parent edges,  and stops when it reaches a \blockstmt. For $P$, this traversal sets $L_e$  as the node corresponding to the \blockstmt between Lines~\ref{lst:line:handlers-run} and ~\ref{lst:line:handlers-run-end} of Figure~\ref{fig:vulnerabilty-example1}(a). Next, in Line~\ref{lst:line:offseteditindex} of \strategyone, the index \I{I} is set to the index corresponding to the  syntactic child of the edit-location $L_e$ who is an ancestor of the semantic location $L_s$ . For $P$, this index  materializes into $13$; the edge  outgoing from blue \blockstmt in Figure~\ref{fig:example1-pdg} to an ancestor of semantic location (shown in red) has label \T{ch:13}. Next, we materialize the \eastree defined in Line~\ref{lst:line:eastree} of \strategyone by  materializing the  reference-nodes. The \eastree \I{O} serializes into @if (REF1.hasOwnProperty(REF2)) { REF3 } @ where @REF1@, @REF2@, and @REF3@ correspond to \T{ReferenceAST} operators with locations as \reflocone, \refloctwo, and \reflocthree. \refloctwo traverses semantic-parent edges  from \semloc (Line~\ref{lst:line:goodref}) and materialize into @data.id@. Similarly, \reflocone and \reflocthree traverse syntactic children edges and materialize into @handlers@ and @foo(data);@ respectively. Thus, the \eastree \I{O} materializes  into @if (handlers.hasOwnProperty(data.id)) { foo(data); }@, which is the required repair. 

%When \strategyone is given the program \prog in Figure~\ref{fig:vulnerabilty-example1}(a) as input, then it first  traverses a path from the source node to the "semantic location"  \semloc using \DMethod{GetKleeneStar}{"SemChild"}{\DMethod{GetClause}{"Expr"}}\ in Line~\ref{lst:line:semkleene}. This leads to a \newtextsc{KleeneStarTraversal} with the stopping condition $\lambda n.\mathcal{T}[n] = \text{"CallExpr"}$. For the input program, the traversal skips through the semantic-child-edges 1-7 and reaches @foo@ in Line~\ref{lst:line:callerId-sink}. Next, it applies another \newtextsc{KleeneStarTraversal} starting from \semloc to reach \editloc in Line~\ref{lst:line:synkleene}. This traversal skips over syntactic-parent-edges and reaches the \blockstmt between Lines~\ref{lst:line:handlers-run} and ~\ref{lst:line:handlers-run-end}. Next, in Line~\ref{lst:line:offseteditindex}, the index \I{I} is computed as \DMethod{GetOffsetIndex}{Ls, 0}\ which means to pick the child-index of \editloc that has \semloc as its descendent. For our example, this index would materialize into the statement number in the block statement containing @foo@, which turns out to be $13$. Next, we instantiate the \eastree in Line~\ref{lst:line:eastree} which hierarchically defines the children-nodes or reference-nodes. The \eastree \I{O} deserializes into @if (REF1.hasOwnProperty(REF2)) { REF3 } @ where @REF1@, @REF2@, and @REF3@ correspond to \T{ReferenceAST} operators with locations as \reflocone, \refloctwo, and \reflocthree. \reflocone and \refloctwo use the semantic-parent edge traversals from \semloc (Line~\ref{lst:line:goodref}) and materialize into @handlers@ and @data.id@. \reflocthree performs a syntactic-child edge traversal from \editloc and materializes into @foo(data);@ thus materializing the entire \eastree \I{O} into @if (handlers.hasOwnProperty(data.id)) { foo(data); }@, i.e. the required repair. 

Now consider the repair strategy \strategytwo in Figure~\ref{fig:strat2}. This strategy shares a similar structure with the earlier strategy but differs in the way traversals and the index $\I{I}$ are computed. There are four key differences
\begin{enumerate}
    \item In order to reach \semloc from \prog.source, \strategytwo performs the \T{EdgeTraversal} using semantic-child edge seven times in Line~\ref{lst:line:nosemkleene}. The number of semantic edges varies widely across programs and prevents generalization to other scenarios. \T{KleeneStarTraversal} operator instead uses \newtextsc{Clauses} over nodes to find the edit-location.
    \item To reach $L_e$ from \semloc, \strategytwo performs the \T{EdgeTraversal} using syntactic-parent edge seven times in Line~\ref{lst:line:nosynkleene}. Consider a program that instead assigns output of the function-call @let out = foo(data)@. \strategytwo will find \assignexpr as the edit-location and fail to generalize whereas \strategyone will appropriately adjust and take four parent steps.
    \item In order to compute the index at which replacement needs to occur, \strategytwo uses a \DMethod{ConstantIndex}{13}\ in Line~\ref{lst:line:consteditindex} of Figure~\ref{fig:strat2}, which effectively assumes that replacement should always occur at 13$^{th}$ child of $L_e$ and again doesn't generalize. \strategyone on the other hand uses of \T{GetOffsetIndex} operator to instead compute index dynamically for a given input program
    \item In order to materialize reference nodes, \strategytwo uses syntactic edge traversals (Line~\ref{lst:line:badref} of Figure~\ref{fig:strat2}) which assume definite structure about the structure of the program (@GetConstant(7)@ used as syntactic child index to solve a long-ranged-dependency). \strategyone instead uses semantic-parent edges to capture the semantics here and produces a better generalizing repair.
\end{enumerate} 

\lstDeleteShortInline@

\noindent These programs highlight that our \dsl is expressive enough to perform complicated non-local repairs in a generic manner. At the same time, while many strategies can repair a given program, all applicable strategies are not equally good. A key realization is that we \emph{prefer shorter traversal functions} (\newtextsc{KleeneStarTraversal}\ over a long sequence of \newtextsc{EdgeTraversal}). Similarly, we \textit{prefer the traversals with none or small constants}. For example, we prefer \DMethod{GetOffsetIndex}{\semloc}{0}\  over \DMethod{GetConstant}{13}\ and semantic-parent traversal over syntactic-parent traversal with index \DMethod{GetConstant}{7}. %Finally, we also \emph{prefer strategies that share traversals across localizing \editloc and \refloc}.
We use these insights to guide the search in our synthesis algorithm.

% The strategy (\strategy) is defined by 
% performs this localization using an edit path (\editpath). We define a path (\genpath) in the strategy as a sequence of edges in the \prog. An edge is either a syntactic \astree edge or a semantic \taintpropedge in either direction (i.e. towards parent or child). 
% The localized node in the \prog is called edit location (\editloc). Next, the strategy either inserts or replaces a child of the edit location with a new \astree. This new \astree can either be a constant node or reference a node in the original \prog using a reference path (\rfpath). Figure ~\ref{fig:approach-notations} summarizes the notations 

% This \dsl was created so we can use the \sa annotations seamlessly and is guided by how humans fix such vulnerabilities. A line of previous works~\cite{} manually write repair patterns for fixing code. Our \dsl-based approach is strictly more general as it can perform various kinds of repairs and the exact repair strategies are learned from data. Moreover, we make effective use of high-level patterns and domain insights, and annotations. So while these other approaches tend to be simplistic and \textbf{either do not generalize well or over-generalize (generate a large number of false positives)}, concrete instantiations of strategies in our \dsl are better at capturing the high-level repair intent better. Following we describe the terminologies used in the repair \dsl.

% \lstMakeShortInline[columns=fixed]@
% The top level rule in our \dsl defines the Repair Strategy (denoted by \strategy). It is parameterized by the type of vulnerability the strategy fixes and the edit \edit. We consider two kinds of edits, either an insert operation or a replace operation. This means that the edit \edit either inserts an \astree child or replaces an \astree child with another \astree. Since we are fixing taint-flow vulnerabilities, we found these two operations to be sufficient. However, our \dsl can be expanded to also handle delete operations \aksays{Why can't we say that delete is replacement with an empty tree?}. In Figure ~\ref{fig:vulnerabilty-example1}, the fix applied in replaces the \astree corresponding to @handlers[callerId](data)@ (line ~\ref{lst:line:callerId-sink}, Figure ~\ref{fig:unsafememberex}) with the if statement in lines ~\ref{lst:line:fix-start}-\ref{lst:line:fix-end}, (Figure ~\ref{fig:safememberex}) and depicts a replace edit. 

% %\paragraph{Edit (\edit)} Since we are solving source-sink-sanitizer vulnerabilities, our \edit either inserts child \astree at edit-locations (denoted by \editloc) or replaces a child with another \astree (at \editloc). %This \editloc is a node in the \pdg which is reachable from the vulnerability source (as provided by the \astree) by traversing syntactic (\astree) or semantic edges in \pdg. %Once the \editloc is found, the new \astree (either replacing the existing child being inserted as a child) can be   

% Notice that in the \pdg, @handlers[callerId](data)@ is a child of the \blockstmt (parenthesis block between lines ~\ref{lst:line:handlers-run}-\ref{lst:line:handlers-run-end} and marked in blue in Figure ~\ref{fig:example1-pdg}). So while applying the fix, we replace the \textit{$k^{th}$} child of \blockstmt with the \ifstmt. We call the node in the \pdg where the edit operation applies as the edit location (\editloc). When a strategy applies, it has to determine this edit location based on the \pdg structure. Our \dsl defines an edit path (denoted by \editpath) to find edit location. In Figure ~\ref{fig:example1-pdg}, starting from the source-node @event@ (marked in orange), we take 7 semantic edges (reaching @callerId@) and then after hopping four synactic parent edges we reach the edit location \aksays{The notion of semantic edges should be defined and explained before this.}. This sequence of edge traversal defines our edit path. More generally, our \dsl considers the \editpath to be a set of semantic edges followed by a set of syntactic edges. This constraint on the paths allows expressivity to learn general strategies while also keeping the search space small. The semantic edges in \editpath allows navigating to ``somewhere close'' to sink location. Next once semantic edges are traversed, \editloc is reached by traversing a set of syntactic edges. Additionally, since the number of semantic edges might vary across examples, our \dsl allows a powerful  operator that navigates an indefinite number of semantic edges. This formulation helps our strategies to generalize well across widely different sets of programs. 

% %\paragraph{Edit Location (\editloc)} Edit location is the node in the \pdg where the edit operation (i.e. insertion or replacement of a child node) applies. \editloc is reachable from the vulnerability source found by traversing the edit-path (\editpath) in the \pdg.  So, in our running example, "$\dots$ function (data){$\dots$}" node (marked in blue) is the edit root and is reachable from the source via first traversing the semantic edges followed by traversing to "syntactic parent" four times. 
% Our kleene-star operator navigates indefinite semantic child edges until a ``stopping node'' (parameterized by a stopping condition) is reached. This stopping-condition is defined by a set of predicates applied on a \aksays{an} \astree node. We find that simplistic predicates about \newtextsc{ASTType} or \newtextsc{ASTValue} of \astree node and its neighbours suffice in locating this stopping node. For our running example, the stopping condition is the conjunction of the predicates @ASTType(node.parent) = IndexExpr@, @ASTType(node.parent.parent) = MethodCallExpr@. The stopping node lies on the taint-flow path from source to sink and therefore is quite relevant to the insert or replace operations (being a proxy for the semantic information of the vulnerability). Therefore, we call the stopping node as the semantic location (\semloc). In our running example, @calledId@, the sink-node is also the \semloc.

% As described above, our \dsl either inserts a new \astree, or it replaces an existing \astree with another \astree. An \astree is defined by three properties -- type, value and an array of children \astree. One can construct such an \astree by concretely initializing it using specific types and values for the tree and its descendants. However, a constant \astree cannot generalize because the fix depends on existing variables in the source code. Therefore, in addition to a constant \astree, our \dsl also allows referring to any existing node in the \astree. This referral is computed by traversing a path from the semantic location (\semloc defined above) to the node to reference \aksays{What is node to reference?}. The corresponding path is known as reference path. For e.g. the condition @handers.hasOwnProperty(callerId)@ which is used in the fix refers to @handlers@ and @callerId@ nodes in the tree and combines them in a constant \callexpr \astree. 

% We described edit paths and reference paths above. More generally a path is a sequence of edges in the \pdg where the edge can be one of syntactic or semantic or ancestral. When selecting a child edge, we also need to store \textit{which child} to select and it is determined by an index. Figure ~\ref{fig:repair-strategy-ex1} shows the entire strategy that fixes the example in Figure ~\ref{fig:vulnerabilty-example1}.
% \lstDeleteShortInline@

% \paragraph{Semantic Location (\semloc)} The node at which Kleene-Star traversal of semantic edges stops is called semantic location. This node is a key component in the repair since this node is a proxy for the semantically important values that would be necessary for making the edit. The sink node, callerId, is also the \semloc in our example. %Additionally, the \editloc is near this node and 

% \paragraph{Paths (\dslpath)} A path is described as a sequence of edges in the \pdg. The edges can be syntactic parent or child edges, semantic child edges, and ancestor edges in the \pdg.

% \paragraph{Index (\dslindex)}
% While selecting a child edge or while determining where the \concinsertcode needs to be inserted or replaced with, we need some index of which child to follow. Generally, this index is a constant value however it can be computed as an offset from the \semloc descendent direction as computed with the \newtextsc{OffsetFrom} operator. \naman{explain the requirement of offset with example}

%\paragraph{\astree} \astree is the tree-representation of the editcode that will replace some existing child of \editloc or will be inserted at some child indices of \editloc. One possible way to construct this \astree is to concretely initialize it using specific values and types. However, a constant \astree cannot generalize because the inserted code almost always depends on specific variables and the structures in the code. Therefore, in addition to a constant structure, the \astree can also refer to existing elements in the \pdg. This referral is again found using a path (\dslpath) traversal from the \semloc, the intuition being that it is a good \textit{proxy for the semantics of the vulnerability} and necessary variables to refer would be close to it.  


\subsection{Synthesizing \dsl strategies from examples}
\label{subsec:synthesis}
%\naman{The discussion about anti-unification would go in related work I presume?}

Given this high-level \dsl, we will now describe our example-based synthesis algorithm. 
We take as input a set of unsafe programs and edits generated as output at the end of data collection step (Section~\ref{sec:data}). 
Let $\{(\prog_{1},\edit_1),(\prog_{2},\edit_2),\dots$ $,(\prog_{n},\edit_n)\}$. 
Here, $\prog_{i}$ is the $i^{th}$ unsafe program and $\edit_i$ is the corresponding edit. Edit ($\edit$) contains the \astree-node of the edit-location ($\edit$.loc), the \textit{concrete} \astree of the edit-program ($\edit$.editprog), and the type of edit i.e. \insertsc or \replace ($\edit$.type). We use these to learn high-level repair strategies in our \dsl. 
\lstMakeShortInline[columns=fixed]@
Our goal is to combine specific paths, learned over examples that share similar repairs in different semantic and syntactic contexts, to obtain general strategies. Our repair strategies abstractly learn the following:
% \begin{enumerate}
%     \item the traversal for localizing edit-locations (\editloc) and reference-locations (\refloc). 
%     \item template-repair-program-representations using the reference-traversals . 
% \end{enumerate}
\begin{enumerate}
    \item Traversals for localizing edit-locations (\editloc) and reference-locations (\refloc). For example, @Ls@ in Line~\ref{lst:line:semkleene} (Strategy \strategyone) depicts a \T{KleeneTraversal} abstraction we can learn from examples having a variable number of semantic-edges. Similarly, @I@ in Line~\ref{lst:line:offseteditindex} (of \strategyone) depicts a generalized index expression we can learn from examples.
    \item \eastree which use reference-traversals. For example, @O@ in Line~\ref{lst:line:goodstratO} demonstrates templated-program-structure that we can learn from examples (say by generalizing from the witnessed guards @handlers.has(data)@  and @events.storage.has(event.name)@).
\end{enumerate}
\lstDeleteShortInline@


% In particular, we wish to abstract over examples that share similar repairs in different semantic or syntactic contexts. Consider the example abstractions below: 
% \begin{enumerate}
%     \item  
%     \item Line~\ref{lst:line:eastree} in Figure~\ref{fig:strat1} depicts 
%     the guard condition in Figure~\ref{fig:safememberex} @handlers.hasOwnProperty(data.id)@ can be abstracted with another guard @eventHandlers._storage.hasOwnProperty(event.name)@ into an abstract template @REF1.hasOwnProperty(REF2)@ where @REF1@ and @REF2@ are \T{ReferenceAST} have use traversals
% \end{enumerate}
% For example, . 
%Consider the example in Figures~\ref{fig:static-witnessing-1},~\ref{fig:static-witnessing-2}, and ~\ref{fig:static-witnessing-3}. They describe three different kinds of syntactic repairs and are not candidates to merge. Instead, 

We depict our synthesis algorithm in Figure~\ref{fig:strategy-learning}. At a high-level, our synthesis algorithm, first pre-processes the inputs, storing the required \textit{concrete} traversals. Next, it performs ranked pair-wise merging over the processed edits to synthesize strategies.
%We merge non-terminals recursively by deductively choosing production rules to enumerate and merging the non-terminals appearing in the productions. %During this recursion, it learns the \textit{traversals} and program templates abstractly. 

\noindent \textbf{Pre-processing.} In this step, given the programs and edits, we store the concrete traversals required for learning \editloc and \refloc (Line~\ref{algo:line:preprocess}). Naively computing all such traversals is very expensive and also leads to \textit{bad strategies}. Here, based on the insights from Section~\ref{subsec:examples}, we only compute the traversals that lead to shorter  
traversals
%\textit{abstract traversals}
which generalize better. In addition, we also share traversals between between \editloc and \refloc. Pre-processing has following three key steps:
\begin{enumerate}
    \item \textbf{Edit Traversals.} We compute the traversals between \prog.source and \editloc (Line~\ref{algo:line:conceditloc} of Figure~\ref{fig:strategy-learning}) that have the form of a sequence of semantic-edges followed by a sequence of syntactic-edges. This allows abstracting variable-length sequences of semantic-edge traversals as a \kleeneedge (corresponding to an abstract \newtextsc{KleeneTraversal}). We implement this using \newtextsc{BiDirecBFS} method at Line ~\ref{algo:line:bidirecbfs}. For every edit-traversal ($\I{T}_e$), we define {\em semantic-location} (\semloc for brevity) as the last-node on the semantic (dataflow) traversal before traversing the syntactic-edges.
    \item \textbf{Compressing Edit Traversals.} We compress these edit-traversals using the \newtextsc{Compress} method in Line~\ref{algo:line:compress}. It takes in a sequence of (syntactic or semantic) edges as input, greedily combines the consecutive edges with the same edge-type (\edgetype) into a \kleeneedge. The \kleeneedge is constructed using the edge type \edgetype, and a set of clauses $\clause_i$ that satisfy the target node of \kleeneedge. These clauses are either $\lambda n. \mathcal{T}(n) = \tau$ that check the type  or $\lambda n. \mathcal{T}(F_i(n)) = \nu$ that check the type of a neighbor. \newtextsc{Compress} returns a sequence of edges or \kleeneedges as output. 
    \item \textbf{Reference Traversals.} For every node of the edit-program, we locate nodes in the \pdg with the same \textit{value} using a \newtextsc{LevelOrderBFS} until a max-depth (Line~\ref{algo:line:maxlevel}). We perform this traversal from \semloc (defined in (1) above). We thus share parts of traversals between locating \editloc and \refloc which helps in learning \textit{better strategies}. The motivation behind using \semloc is that the expressions necessary for repair will be close to \semloc as it lies on the information-flow path. 
\end{enumerate}
%Specifically, for \editloc, we find the traversals between \prog.source and \editloc (Line~\ref{algo:line:conceditloc}) that first navigate a set of semantic-edges followed by a set of syntactic-edges. We implement this using the \newtextsc{BiDirecBFS} function at Line ~\ref{algo:line:bidirecbfs}. %It traverses semantic-edges from the source, syntactic-edges from the edit-location, and returns the intersecting traversals. 
%For every edit-traversal ($\I{T}_e$), we define semantic-location (\semloc for brevity) as the last-node on the semantic (dataflow) traversal before navigating a syntactic-edge. 
%Next, we compress the traversals greedy by combining consecutive edges of the same edge-type (\edgetype) into a \kleeneedge using the \newtextsc{Compress} method in Line~\ref{algo:line:compress}. Every \kleeneedge stores the \edgetype, and a set of clauses $\clause_i : i \in {1,\dots,n}$ that satisfy the end-node of \kleeneedge. These clauses are either $\lambda n. \mathcal{T}(n) = \nu$, i.e. a clause on the type of \semloc or $\lambda n. \mathcal{T}(F_i(n)) = \nu$, i.e. a clause on the type of a syntactic-neighbour of \semloc. We then compute traversals for finding reference locations. Here, instead of computing traversals from the source-node, we instead compute traversals from the semantic-locations. The expressions referenced in repairs are usually close to the \semloc (as it lies on the information-flow path and thus is affiliated with variables likely necessary for building the repair). This traversal-sharing optimizes the search and generalization of our strategies.

\noindent \textbf{Strategy Synthesis.} Given the edits and the associated traversal meta-data, we synthesize the strategy by pair-wise merging  (Line~\ref{algo:line:callmerge}). \newtextsc{MergeEdits}, the top-level synthesis method, takes a pair of edits as inputs and returns a list of strategies satisfying the example edits. We synthesize the strategies recursively using a deductive search over the non-terminals of the DSL (Figure~\ref{fig:fixing-dsl}). Specifically, to synthesize an expression corresponding to a non-terminal, we deduce which production to use and recursively synthesize the non-terminals given by its production-rule. This has the following key components: 
\begin{enumerate}
    \item \newtextsc{MergeEdits}: It takes pairs of edits as inputs and returns the strategy. It recursively synthesizes the traversal (for \editloc), index, and \eastree. It combines and returns them using the edit-type. 
    \item \newtextsc{MergeTraversal}: It takes two concrete traversals (sequence of edges or \kleeneedges) as inputs and returns the abstracted traversal. by merging elements in the sequence.
    \item \newtextsc{MergeEdge}: It takes two edges or \kleeneedges as inputs and returns a \T{GetKleeneTraversal} or \T{GetEdgeTraversal}. We combine two \kleeneedges using their edge-types and intersecting the clauses stored during pre-processing. We combine two edges using their edge-types, and recursively combining their indices.
    \item \newtextsc{MergeIndex}: It takes two integer indices as inputs and returns an abstracted index. If the two input indices are equal, we return a \T{GetConstant} operator with the input index value. Otherwise, we compute offset as the difference between input-index and index of child of $n$ which has \semloc as descendent (computed by $DO(n, \semloc)$). We return this offset if they are equal and an empty-list otherwise. 
    \item \newtextsc{MergeProg}: It takes two programs as input and returns a list of \eastree, where each list element can materialize into the input programs. If the top-level node in the programs have equal values and types, we combine them as a \T{ConstantAST}. Otherwise, we recursively combine their children. Finally, we merge the reference-traversals corresponding to the input programs and combine them into \T{ReferenceAST}.
\end{enumerate}

Our synthesis procedure is inspired by anti-unification~\cite{anti-unification} and we abstract the paths and edit-programs across different examples. Specifically, our \T{KleeneTraversal} and \T{OffsetIndex} functions allow generalization across paths having different number of edges and different indices where naive abstractions fail. Similarly, \eastree also resemble anti-unification over tree-edits. However, again we use traversals over syntactic and longer-context semantic-edges, for better generalizations  and repairs. 

Finally, note that while we perform pair-wise merges over the edits, the strategy synthesis algorithm can be extended to merge bigger cluster of edits together as well. However, from our experience, we find that the pair-wise merging performs well and is sufficient for our experiments. 
%by recursively synthesizing values corresponding to the non-terminals in our \dsl. 
%Thus, to synthesize a strategy, we synthesize traversals for the \editloc. To synthesize a traversal, we check if the two traversals contain an equal number of edges. Next, we try to merge the corresponding pairs of edges on each traversal. To merge an edge, if it is simply a syntactic or semantic edge, we instantiate a \T{GetEdge} operator appropriately. However, if the two edges are \kleeneedge, then we instantiate a \T{GetKleeneEdge} operator where the clauses are constructed by intersecting the clauses computed in \newtextsc{Compress} step. It additionally ensures that the number of clauses after the intersection is more than $1$ to prevent over-generalizing strategies. In order to merge strategies, we next merge the edit-programs (\edit.editprog). We first combine the programs as a \T{ConstantAST} by checking whether the type and value match and then recursively merge the children and finally take the cartesian product of children. Next, based on the concrete reference traversals, we merge them and construct \T{ReferenceAST} \eastree. 



%Note that in order to learn repair strategies, we need to synthesize paths corresponding to all locations (\location) used in the strategy. We use \location in three production rules in our \dsl (\locationone, \locationtwo, \locationthree in Rules~\ref{dslrule:strategy},~\ref{dslrule:index},~\ref{dslrule:node}). Naively, trying to synthesize these paths is very expensive and will lead to non-generalizing strategies.
%Here, based on common fix-patterns for these vulnerabilities, we reduce the search space by enforcing structure over the paths we learn. Additionally, we also share paths between the three locations. 

%We find that performing these 

%At a high-level, our synthesis algorithm takes these unsafe-programs and edits as inputs, preprocesses them to store relevant meta-data, clusters them and then recursively enumerates over the non-terminals in the \dsl. How


% \aksays{This section is very dense. It would be good to take a small example and illustrate the key steps visually.}
\input{synthesisalgo2}

% Given this high-level \dsl, we will now describe our synthesis algorithm. We build a top-down synthesis algorithm that learns strategies in this \dsl through a \pbe approach. We receive a set of unsafe codes and concrete edits \unsure{the terminology of concrete edits can be confusing for the paper. Basically, anologous to all things in \dsl, we have concrete edits, paths, etc.}. Let $\{(\prog_{1},\concedit_1),(\prog_{2},\concedit_2),\dots$ $,(\prog_{n},\concedit_n)\}$ be the data we collect from our data collection step where $\prog_{i}$ is the ith unsafe code snippets and $\concedit_i$ is the corresponding concrete edit. $\concedit_i$ contains the concrete edit-location $\conceditloc_i$, the \astree of the editcode $\concinsertcode_i$, and the type of edit i.e. \insertsc or \replace.

% Given this data, we instantiate our algorithm to learn repair strategies. Our algorithm takes in these set of examples and learns a set of ($\{\strategy_1,\strategy_1,\dots,\strategy_k\}$) that are supposed to cover the training examples. Later, these repair strategies, when given an unsafe program \pdg will generate the edit that needs to be applied. The sketch of our synthesis and learning algorithm is presented in Figure~\ref{fig:strategy-learning}.

% %%O := \{\(\strategy\sb{1},\strategy\sb{2},\dots,\strategy\sb{k}\)\}
\HUGE DONT USE THIS - USE synthesisalgo2
% Figure environment removed

% \unsure{\textbf{Terminology for reference!!:} Letters with overlines are concrete elements ($\concedit$ is a concrete edit, $\conceditpath$ is a concrete edit path) while the letters without lines are abstract elements that can generalize over examples ($\edit$ is an abstract edit, $\editpath$ is an abstract edit path). Following, we again define the various abbreviations used in the algorithm
% \begin{itemize}
%     \item \prog is the \pdg containing \sa annotations
%     \item \concedit is the concrete edit which itself contains editcode, edit-location, edit-type, and indices
%     \item \concinsertcode is the concrete editcode that is either inserted or replaces some existing region in the unsafe code. \concinsertcode itself can be represented as an \astree
%     \item \conceditloc is the concrete edit location i.e. where the edit takes place in a \prog
%     \item \concpath is a concrete path as a sequence of edges
%     \item \conceditpath is a concrete edit path from source to \conceditloc
%     \item semLoc or semantic location is the stopping node of \conceditpath
%     \item \concrefpath is a concrete reference path from semantic location to a particular node matching value of a \astree node in \concinsertcode
%     \item edit-type refers to whether edit is \insertsc or \replace
% \end{itemize}}

% \spsays{Consider breaking these into subsections and have a running example. for instance, first section can be running bidirectional bfs, other could be pairwise merging, and so on..} 

% Our top-level \newtextsc{Learn} method receives the programs and concrete edits as inputs (Line~\ref{algo:line:learn}). This method first invokes the \newtextsc{PreProcessConcEdit} method which computes edit-paths and reference-paths in the $\concedit$. Next, \newtextsc{Learn} method ranks edits based on the similarity of their $\concinsertcode$ and in that order tries to combine edits together in a pairwise manner using the \newtextsc{MergeEdit} method.

% \newtextsc{PreProcessConcEdit} method (in Line~\ref{algo:line:preprocess}) stores the relevant paths in the $\concedit$ structure that will be useful during pairwise-merging. It first computes a set of edit paths from source to sink using a bi-directional breadth-first search (\newtextsc{BiDirectBFS}) (Line~\ref{algo:line:conceditloc}) and stores it in the edit. Note that during this \newtextsc{BiDirectBFS}, it traverses only semantic edges from the source and only syntactic edges from the edit location. This naturally leads to paths that follow the required pattern of a set of semantic edges followed by a set of syntactic edges. Note that edit-paths also contain the semantic-locations in the semLoc field. Then for every edit-node in the edit code and every semantic location in the edit path, it stores reference paths from semantic locations to nodes in the \astree having the same value as edit-node (Line~\ref{algo:line:concrefpath}. These paths are computed using a \newtextsc{MaxLevelBFS} until a certain depth and storing the satisfying nodes. Note that during implementation, we memoize the path-finding steps to avoid repeating computations.

% \newtextsc{PairSimilar} method computes a score for every pair of edits in the edit set. To compute the similarity for a given pair of edits it performs \newtextsc{ASTSimilarity} on their editcodes. 

% \newtextsc{MergeEdit} is the top-level method of our deductive top-down synthesis algorithm. The merging procedure is intuitive. For every element in the edit, it recursively calls \newtextsc{Merge} operation on the elements and then assembles an edit using their outputs. Specifically, this method first ensures that edits are of the same type (\insertsc or \replace. Then it obtains a set of candidate edit-paths by calling the \newtextsc{MergeEditPaths} method on the concrete edit paths stored in the input concrete edits. Next for every candidate edit-path, it finds a candidate editcode using the \newtextsc{MergeEditCode} method. Finally, for every editpath and editcode pair, it assembles the final edit (Line~\ref{algo:line:assembleedit})
    
% \newtextsc{MergeEditPath} method tries to merge two concrete editpaths. It first computes a set of intersecting predicates over the semantic-locations of the two paths. Using the predicates, it builds the \semkleene edge. Finally, over the remaining set of edges in the edit-path, it calls the \newtextsc{MergePath} method which inturn merges all the edges successively (Line~\ref{algo:line:mergepath})


% \subsection{Applying the learned strategies}
% \label{subsec:applying}
% Given an unsafe-program and a set of repair strategies, we apply each strategy to the program to generate candidate repairs. 
% To apply a single repair strategy, we use the definitions of operators described in Section~\ref{subsec:dsl} to generate candidate repair programs. We find that, in practice, we obtain a few distinct repairs and we return them as the output of our system. 

%Once these high level repair strategies are learnt, applying them is natural. For an unsafe program, strategy \strategy ingests the \pdg of the program. Then it tries to build an edit, by first locating the edit location (using the edit path \editpath) and building the \astree recursively depending on whether it is a constant or reference tree. If edit location and \astree are generated, the edit operation is applied appropriately based on whether it is an \insertsc or \replace edit. Otherwise, if either of location or \astree is not generated then no edit is applied. 
% \subsection{Scribblings}
% Each node $n$ of the AST has an identifier $\mathit{id}\in\mathbb{N}$. The AST is characterized by a set $\mathcal{N}$ of node $\mathit{id}$s, i.e., $\mathcal{N}=\{\mathit{id}_0,\ldots\mathit{id}_k\}$. We have a map $\mathcal{T}$ from nodes to their types, i.e., $\mathcal{T}(n)=\tau$, where the primitive types $\tau$ include {\sc MethodCallExpr}, {\sc IndexExpr}, etc. We also have a set $\mathcal{E}$ of edges, where each edge is $(n_1,n_2,ET,z)$. Here, $n_1$ is a source node, $n_2$ is a target node, $ET$ is the type of edge (syntactic parent, syntactic child, semantic parent, or semantic child), and $z$ is a child's index (set to $-1$ if the edge is a parent edge). 

% The strategy $S$ is of two types, insert an AST $O$ at index $I$ of location $L$, $\mathit{Insert}(L(\mathit{source}),I,O)$, and replace the AST at index $I$ of location $L$ with $O$, $\mathit{Replace}(L(\mathit{source}),I,O)$. Each $L(n)$ takes a node $n$ and traverses a path to reach a location, i.e.,  $L(n)=\mathit{ApplyPath}(n,F_k\circ\ldots\circ F_0)$, where the output node is $F_k(F_{k-1}(\ldots F_0(n)\ldots)$. Each $F$ instantiates the edge traversal function $TE[I,ET,C]$ with an index $I$, an edge type $ET$, and an optional clause $C$ (relevant for KleeneEdge). We define $TE[I,ET](n)$ as $let\ i=I(n)\ in\ let\ N=\mathcal{E}(n,ET)\ in\ N[i]$, which gets an integer index $i$ of children of $n$, gets a set $N$ of nodes by dereferencing edges of type $ET$ from $n$ and returns the $i^{th}$ child of $n$.
% $F$ can also be $TE[ET,C](n)\equiv KE(n,ET,C)$.
%  The KleeneEdge $KE$ keeps dereferencing edges of type $ET$ till it hits a node where a clause $C$ holds, i.e., $KE(n,ET,C)$ is defined as $C(n)? n : \left(let\ t=\mathcal{E}(n,ET)\ in\ KE(t,ET,C)\right)$. The node $t$ which is target of an edge with type $ET$ and $n$ as a source here is chosen non-deterministically and our implementation resolves this non-determinism through a breadth-first search. A clause is a conjunction of predicates of the form $\lambda n. \mathcal{T}(L(n))=\tau$. The index $I$ is either an integer $z$ or of the form $\lambda n.DO(n,L(\mathit{source}))+z$, where $DO(n_1,n_2)$ returns the index of syntactic child of $n_2$ who is a syntactic ancestor of $n_1$. 
 
%  A strategy can fail to apply if in $DO(n_1,n_2)$ there is no path from $n_2$ to $n_1$,

\section{Experimental Setup} \label{exp_setup}
% To demonstrate the superiority of GraspGPT over existing methods, we conduct both perception experiments and real-robot experiments. The goal is to evaluate the performance of generalizing learned TOG skills to novel object classes and tasks out of the training set. 

\subsection{Perception Experiments} \label{perception_exp_setup}
\noindent \textbf{Baselines} \  We compare GraspGPT to the following methods: (1) \textbf{Random}, which represents the method in \cite{sundermeyer2021contact} that focuses on grasp stability only and ignores task constraints (i.e., task-agnostic grasping method). (2) \textbf{Semantic Grasp Network (SGN)} \cite{liu2020cage}, which learns class-task-grasp relations without incorporating external semantic knowledge. (3) \textbf{GCNGrasp} \cite{murali2021same}, which is the state-of-the-art TOG algorithm introduced earlier and whose main limitation is its inability to generalize to novel concepts out of the graph. During inference, we connect the novel concept node to its nearest neighbor in the KG. The nearest neighbor search is based on the cosine similarity between the concepts' pre-trained word embeddings provided by ConceptNet \cite{liu2004conceptnet}. \\

% To evaluate its generalization ability to novel concepts, 



\noindent \textbf{Dataset} \ GraspGPT and three baselines are evaluated on the LA-TaskGrasp dataset, which augments the TaskGrasp dataset with language data. The original TaskGrasp dataset contains 250K task-oriented grasp pose annotations for 56 tasks, 75 object classes, and 191 object instances. Each instance is a partial point cloud of a real household object \textbf{with multi-view RGB-D fusion}. TaskGrasp provides three types of held-out settings: held-out (object) class, held-out task, and held-out instance. We focus on the former two settings in this letter. For language data, LA-TaskGrasp contains 80 language description records for each object class and 40 records for each task, resulting in 6000 object class description records and 2240 task description records. We combine these descriptions to generate 750 object class description paragraphs and 560 task description paragraphs. LA-TaskGrasp dataset also includes 53 language instruction templates, resulting in 222600 possible language instruction sentences. \\

% % Figure environment removed 

% Figure environment removed 

\noindent \textbf{Metrics} \  We use the same set of evaluation metrics used by GCNGrasp. Specifically, we compute the Average Precision (AP) score for each object class, task, and instance, and then compute the mean AP (mAP) averaged over all object classes, tasks, and instances (i.e., class mAP, task mAP, and instance mAP).

\subsection{Real-Robot Experiments}
% Real-robot experiments evaluate the performance of physical grasping subject to the task constraints. 
The real-robot experiment platform comprises a 7 DoF Kinova Gen3 robotic arm with a parallel jaw gripper and an Intel RealSense D435 RGB-D camera with eye-in-hand calibration. For each test object, we first apply SAM\cite{kirillov2023segment} to extract the object point cloud \textbf{captured from a single view} and then apply Contact-GraspNet\cite{sundermeyer2021contact} to generate 50 grasp pose candidates. Single-view setup is used here because it is more practical for real-world robotic applications. We collect test objects from our laboratory and YCB dataset. More details on the experimental setup can be found in the appendix.

% Finally, all the candidates are evaluated and the one with the highest score are executed. 
The physical grasping pipeline is divided into three stages: Perception, Planning, and Action, and the statistics of each stage is reported separately for clarity. A trial succeeds if the test object is grasped subject to the natural language instruction and lifted stably by the robot. We additionally combine GraspGPT with three pre-defined skills (pouring, handover, and scooping) in the form of motion primitive to showcase its practicality in tool manipulation. 

% We place each object (without
% clutter) on a table in front of the robot. After table plane segmentation to obtain the object point
% cloud, 600 stable grasps are sampled and 50 candidates are selected using farthest point sampling
% for evaluation. We evaluate the grasps on our best performing GCNGrasp model from the held-out
% task ablations (Table 2). Our hardware setup comprises of a 7-DOF Sawyer Robot with a 2-fingered
% Robotiq gripper and a Intel Realsense D415 RGB-D camera mounted on the gripper wrist. Inference
% for the 50 grasps takes around 3s on a desktop with an NVIDIA GTX 1080 Ti GPU and the grasp with
% the best score is executed on the robot. Fig 4 shows the executed task-oriented grasps on unknown
% objects. Even though our dataset objects were collected only in one canonical pose, our approach
% is able to generalize to new grasps and in unknown poses due to data augmentation during training.
% Based on the grasp evaluator scores from Fig 4, our model is also able to interpolate between modes
% in the continuous SE(3) space to reason about task-oriented grasping. One failure mode of our work
% is that it does not generalize to categories (like the spray bottle in Fig 4 in the bottom right) with
% limited training data. A future work is to balance the dataset in terms of object categories.

% \noindent \textbf{Baselines} \  xxx \\

% \noindent \textbf{Metrics} \  xxx \\

\subsection{Implementation Details}\label{detail}
All the experiments are conducted on a desktop PC with a single Nvidia RTX 3090 GPU. GraspGPT is optimized with an Adam optimizer\cite{kingma2014adam} with a weight decay of 0.0001. The learning rate is set to 0.0001 initially and decays subject to a customized function as in GCNGrasp. We train GraspGPT for 50 epochs with a batch size of 32. Each point cloud is downsampled to 4096 points before being fed into the model. 

% We provide language descriptions to each concept during both training and testing. 

For the choice of an LLM, we select the OpenAI GPT-3 model, specifically the \textit{text-davinci-003} version. GraspGPT is capable of incorporating any current LLM, such as OpenAI GPT-4 and Google Bard, or using an ensemble of LLMs. We leave this ensemble approach for our future work. For the language encoder, we choose the Google pre-trained \textit{BERT-Base} model provided by Hugging Face.

% This ensemble approach is expected to provide complementary semantic knowledge from each LLM, further enhancing the overall performance of GraspGPT. 

% learning rate 1e-4, LambdaLR, Adam optimizer, weight decay 0.0001, 50 epochs, batch size 32, , 4096 points, loss

% GPT-3 (text-davinci-003), GPT-3.5/4, PalM, bard, subset of 10 each concept, pre-embedding encoded description and instruction





\begin{table}[t]
	\centering
	\caption{Preallocation strategy results with $3$ machines per tool group and $10$ operations per lot}
	\label{tab:table}
	\figspace\scriptsize
	%	\resizebox{15.5cm}{!}{
		\begin{tabular}{|l%r
				cl||rr|rr|rr|rr|}
			%			\hline
			%			&                    &                      & %        &
			%			 \multicolumn{8}{c}{\textbf{M = 9}} \\
			\hline
			& \multicolumn{1}{@{\hspace{-3mm}}c@{\hspace{-3mm}}}{\textbf{9 Machines}}                   &                      & % &
			\multicolumn{2}{r|}{\textbf{70 Operations}}                 & \multicolumn{2}{r|}{\textbf{80 Operations}}                 & \multicolumn{2}{r|}{\textbf{90 Operations}}                 & \multicolumn{2}{r|}{\textbf{100 Operations}}                 \\
			& Size % \multicolumn{2}{c}{\textbf{Parameters}}            
			&        &
			Lot                         & Step                        & Lot                         & Step                        & Lot          & Step         & Lot          & Step         \\
			%			& size              % & setup % idx
			%			                  &         & 0                           & 1                           & 0                           & 1                           & 0            & 1            & 0            & 1            \\
			%			&                    & setup                &         &                             &                             &                             &                             &              &              &              &              \\
			\hline\hline
			\multirow{3}{*}{\textbf{Fixed}}    & \multirow{3}{*}{1} & % \multirow{3}{*}{0/1} &
			Makespan    & 483                         & 428                         & 489                         & 440                         & 486          & 531          & 592          & 553         \\
			&                    & %                     &
			Setup/Batch & 6/12                        & 2/12                        & 5/14                        & 0/13                        & 5/14         & 3/12         & 3/12         & 0/16         \\
			&                    & %                     &
			1\ts{st}/2\ts{nd} Stage & 2/1                         & TO/27                          & 6/2                        & TO/13                          & 11/13         & TO           & TO/78           & TO           \\
			\midrule
			\multirow{6}{*}{\textbf{Flexible}} & \multirow{3}{*}{2} & % \multirow{6}{*}{0}   &
			Makespan    & 483                         & 475                         & 592                         & 592                         & 592          & 539          & 745          & 698          \\
			&                    & %                     &
			Setup/Batch & 2/8                        & 0/9                        & 1/8                        & 1/8                        & 1/10         & 0/11          & 0/12          & 0/15          \\
			&                    & %                     &
			1\ts{st}/2\ts{nd} Stage & 5/1                         & TO                          & TO/114                          & TO/1                          & TO/130           & TO           & TO           & TO          \\
			\cline{2-11}
			%			& & & & & & & & & & &   \\
			& \multirow{3}{*}{3} & %                     &
			Makespan    & 559                         & --                          & 815                         & --                          & 1357 & -- & 1486 & -- \\ % \multicolumn{4}{c|}{\multirow{3}{*}{Assignment issue}}     \\
			&                    & %                     &
			Setup/Batch & 0/8                         & --                          & 0/8                        & --                          & 0/10 & -- & 10/18 & -- \\ %\multicolumn{4}{c|}{}                                      \\
			&                    & %                     &
			1\ts{st}/2\ts{nd} Stage & TO                       & --                          & TO/140                          & --                          & TO/79 & -- & TO & -- \\ %\multicolumn{4}{c|}{}                                      \\
			\midrule
			\multirow{6}{*}{\textbf{Setup}}    & \multirow{3}{*}{2} & % \multirow{6}{*}{1}   &
			Makespan    & 483                         & 475                         & 592                         & 592                         & 592          & 536          & 745          & 683          \\
			&                    & %                     &
			Setup/Batch & 2/8                        & 0/9                        & 1/8                        & 1/8                        & 1/10         & 0/12          & 0/13          & 0/16          \\
			&                    & %                     &
			1\ts{st}/2\ts{nd} Stage & 2/1                        & TO                          & TO/21                          & TO/25                          & TO/22           & TO           & TO/76           & TO           \\
			%			& & & & & & & & & & &   \\
			\cline{2-11}
			& \multirow{3}{*}{3} & %                     &
			Makespan    & \textbf{334}                         & --                          & \textbf{345}                         & --                          & \textbf{434}          & --           & \textbf{555}          & --           \\
			&                    & %                     &
			Setup/Batch & 0/8                         & --                          & 0/8                         & --                          & 0/11          & --           & 0/12          & --           \\
			&                    & %                     &
			1\ts{st}/2\ts{nd} Stage & TO/20                       & --                          & TO/123                          & --                          & TO           & --           & TO/73           & --           \\
			\hline
		\end{tabular}
		%	}
\end{table}
%
We constructed a scalable set of benchmark instances, focusing on sub-routes of
$10$ production operations for two product types from the SMT2020 simulation scenario~\cite{kopp2020smt2020}.
The $10$ operations in both sub-routes are processed by machines
belonging to three tool groups and do thus involve re-entrant flow,
as a lot visits the same tool group multiple times.
Moreover, the operations incorporate batching and specific setups, and machines undergo periodic maintenance operations.
In the following, we concentrate on instances with $9$ machines, i.e., $3$ per
tool group, and gradually increasing number of lots.
Further smaller- and larger-scale instances along with our implementation are
available online.\footref{foo:online}

We ran our experiments with \clingodl\ (version 1.4.0) on an Intel® Core™i7-8650U CPU Dell Latitude 5590 machine under Windows 10, imposing two time limits per run:
the first stage for makespan minimization is aborted at $450$ seconds, in which case the best schedule found so far % (if any) 
is taken as upper bound on the makespan for proceeding to minimize setup and batch violations with 
another $150$ seconds time limit.

Table~\ref{tab:table} reports the quality of best schedules obtained within the time limits for both optimization stages, split into `Makespan' and `Setup/Batch'
values, while two runtimes or `TO' for a timeout, respectively, are given in the
`1\ts{st}/2\ts{nd} Stage' rows, only listing a single `TO' entry in case both stages timed out.
The `Size' column provides the value taken for the constant \lstinline{sub_size},
limiting the number of machines in subgroups to which the operations are preallocated.
For the latter, the `Lot' columns include results with value \lstinline{0} for the constant \lstinline{lot_step}, where a common subgroup takes all operations for a lot, or for value \lstinline{1} in the `Step' columns, leading to their distribution among subgroups.

The `Size' value 1 necessarily leads to a fixed machine assignment, for which the
quality indicators clearly show that the `Step' strategy yields better schedules,
although it incurs more timeouts and thus fewer certain optima because operations on different lots increase the flexibility of execution sequences and thus search complexity.
While flexibility within subgroups by setting their `Size' to 2 or 3 in principle allows for improved schedules, we observe a deterioration due to sharply increasing instantiation size and search effort, as already observed in \cite{ali2023flexible}.
The setup strategy to differentiate operations and machines within subgroups,
activated by changing the constant \lstinline{by_setup},
aims to cut down the scheduling complexity in line with the optimization objectives by reducing the need for setup changes.
This leads to significantly improved schedules with `Size' 3, where the
`Lot' and `Step' preallocation strategies are indifferent and redundant results for the latter are omitted, up to a critical size reached with $100$~operations.

With our preliminary approach~\cite{ali2023flexible}, using a more naive and less feature-rich encoding of either fixed or fully flexible machine assignments, the
threshold at which problem size and combinatorics get prohibitive was reached at less than $50$ operations already.
Despite gearing up to double that size, our benchmark instances still represent small excerpts of the large-scale semiconductor fabs with more than $100$ tool groups and from $242$ to $543$ production operations per lot modeled by~\cite{kopp2020smt2020}.
%
The elevated complexity in comparison to basic settings like the traditional FJSP is mainly due to sophisticated setup and maintenance operations, requiring a detailed analysis of execution sequences on machines for SMSP.
We conjecture that similar scalability limits would also be encountered with MIP or CP encodings, yet the first-order modeling language of ASP with difference logic facilitates rapid prototyping and experimentation.
In fact, our performance evaluation aims to explore the feasibility of search and optimization, in order to come up with strategies for breaking down large SMSP instances into more manageable portions, e.g., focusing on some bottleneck tool groups or re-entrant flow of operations.

% This section will show the experimental results performed by applying the machine assignment strategies mentioned before, with several instances ranging from $30$ to $130$ steps and $6$ to $12$ machines. All experiments are run using an Intel\textsuperscript{\textregistered} Core\texttrademark{} i7-8650U CPU Dell Latitude 5590 machine under Windows 10. Our timeout limit is $600$ seconds, splitted to $450$ seconds for the makespan and $150$ seconds for the setup and batching. 

% We considered three tool groups for all generated instances in which batch processing, time/counter-based maintenance, and setup are considered. For generating the instances, we started with a small instance containing $30$ steps and $6$ machines where each tool group has $2$ machines and then we generate the next instance by adding one more lot, which has $10$ steps. We kept the tool group size till the fixed machine assignment strategy could not reach the optimum within the time limit. We created $3$ parameters \textit{size, idx} and \textit{setup} to activate a specific machine assignment strategy. The size determines the size of a sub-group in each tool group. The $idx$ defines the Job/Step-based indexing of all steps in the same tool group where all steps of the same lot will have the same index if the $idx = 0$ and Hence, they are assigned to the same sub-group/machine. If $idx = 1$, then each step in the tool group will have an identical index. The last parameter setup is to activate the setup strategy or not. If the $setup = 1$, then the setup strategy is applied; if $setup = 0$ then it's not applied.

% % To continue tomorrow isA :)
% Table \ref{tab:table01} shows the results of the instances with $2$ machines in each toll group. The first column refers to the strategy applied for the machine assignment. The second and third columns show the parameters for selecting a particular strategy. The assignment is fully flexible if the \textit{size} is greater than or equal to the number of machines in a tool group. Otherwise, the assignment is partially flexible. In the fourth column, we list our optimization criteria and the time limit for the makespan and setup/batching represented by 1st/2nd call. Each following two consecutive columns illustrate the results of an instance when the Job/Step-based indexing is selected. From the \ref{tab:table01}, we observed that the best-obtained results were achieved by the full flexible assignment in the first three instances and for the last instance, the setup strategy was the best. The fixed/setup strategies terminated within the time limit except for only one case.

% \begin{table}[h]
% 	\centering
% 	\caption{Comparison between the allocation strategies with 2 machines per tool group}
% 	\label{tab:table01}
% %	\resizebox{15.5cm}{!}{
% 		\begin{tabular}{|l%r
% 			cl||rr|rr|rr|rr|}
% 			\hline
% %			&                    &                      &         & \multicolumn{8}{c}{\textbf{M = 6}} \\
% %			\hline
% 			& \textbf{M = 6}                   & %                     &
% 			  & \multicolumn{2}{r|}{\textbf{Instance 01}}                 & \multicolumn{2}{r|}{\textbf{Instance 02}}                 & \multicolumn{2}{r|}{\textbf{Instance 03}}                 & \multicolumn{2}{r|}{\textbf{Instance 04}}                 \\
% 			& Size % \multicolumn{2}{c}{\textbf{Parameters}}            
% 			 &			         & Job                         & Step                        & Job                         & Step                        & Job          & Step         & Job          & Step         \\
% 			\hline
% %			& size               & setup %idx
% %			                  &         & 0                           & 1                           & 0                           & 1                           & 0            & 1            & 0            & 1            \\
% %			&                    & setup                &         &                              &                             &                             &                             &              &              &              &              \\
% 			\hline
% 			\multirow{3}{*}{\textbf{Fixed}}    & \multirow{3}{*}{1} & % \multirow{3}{*}{0/1} &
% 			 Makespan    & 409                         & 353                         & 409                         & 409                         & 525          & 424          & 525          & 493          \\
% 			&                    & %                     &
% 			 Setup/Batch & 5/6                         & 4/6                         & 4/8                         & 4/8                         & 4/9          & 1/9          & 3/11          & 2/10          \\
% 			&                    & %                     &
% 			 1\ts{st}/2\ts{nd}-Call & \textless{}1/\textless{}1 & \textless{}1/\textless{}1 & \textless{}1/\textless{}1 & \textless{}1/\textless{}1 & 31/1         & 137/6        & 37/11          & TO/53           \\
% 			\midrule
% 			\multirow{3}{*}{\textbf{Flexible}} & \multirow{3}{*}{2} & % \multirow{3}{*}{0}   &
% 			 Makespan   & \textbf{233}                         & --                          & \textbf{281}                         & --                          & \textbf{365}          & --           & 587          & --           \\
% 			&                    & %                     &
% 			 Setup/Batch & 0/5                         & --                          & 0/6                         & --                          & 0/8          & --           & 3/9          & --           \\
% 			&                    & %                     &
% 			 1\ts{st}/2\ts{nd}-Call & 7/0                         & --                          & TO/6                          & --                          & TO/83           & --           & TO           & --           \\
% 			\midrule
% 			\multirow{3}{*}{\textbf{Setup}}    & \multirow{3}{*}{2} & % \multirow{3}{*}{1}   &
% 			 Makespan  & 277                         & --                          & 321                         & --                          & 381          & --           & \textbf{419}          & --           \\
% 			&                    & %                     &
% 			 Setup/Batch & 0/4                         & --                          & 0/6                         & --                          & 0/8          & --           & 0/9          & --           \\
% 			&                    & %                     &
% 			 1\ts{st}/2\ts{nd}-Call & \textless{}1/\textless{}1 & --                          & 25/1                         & --                          & TO/12        & --           & TO/122           & -- \\
% 			 \hline
% 		\end{tabular}
% %	}
% \end{table}

% Table~\ref{tab:table02} summarizes the results of the subsequent $4$ instances where each tool group has $3$ machines. In this instances group, we can split the machines into sub-group by setting the \textit{size} parameter to $2$; in that case, we have two sub-groups in each tool group. The experiments demonstrated that the fixed strategy has the same or better performance than the flexible. In addition, the flexible strategy could not find a feasible solution for instances $7$ and $8$ when all machines were in the same group. On the other hand, the setup strategy performed better than the other two strategies when all machines were in one group, in addition to reaching the optimal value of the setup for all instances. 

% \begin{table}[h]
% 	\centering
% 	\caption{Comparison between the allocation strategies with 3 machines per tool group}
% 	\label{tab:table02}
% %	\resizebox{15.5cm}{!}{
% 		\begin{tabular}{|l%r
% 			cl||rr|rr|rr|rr|}
% %			\hline
% %			&                    &                      & %        &
% %			 \multicolumn{8}{c}{\textbf{M = 9}} \\
% 			\hline
% 			& \textbf{M = 9}                   &                      & % &
% 			 \multicolumn{2}{r|}{\textbf{Instance 05}}                 & \multicolumn{2}{r|}{\textbf{Instance 06}}                 & \multicolumn{2}{r|}{\textbf{Instance 07}}                 & \multicolumn{2}{r|}{\textbf{Instance 08}}                 \\
% 			& Size % \multicolumn{2}{c}{\textbf{Parameters}}            
% 			&        &
% 			 Job                         & Step                        & Job                         & Step                        & Job          & Step         & Job          & Step         \\
% %			& size              % & setup % idx
% %			                  &         & 0                           & 1                           & 0                           & 1                           & 0            & 1            & 0            & 1            \\
% %			&                    & setup                &         &                             &                             &                             &                             &              &              &              &              \\
% 			\hline\hline
% 			\multirow{3}{*}{\textbf{Fixed}}    & \multirow{3}{*}{1} & % \multirow{3}{*}{0/1} &
% 			 Makespan    & 525                         & 433                         & 525                         & 452                         & 525          & 521          & 643          & \textbf{559}          \\
% 			&                    & %                     &
% 			 Setup/Batch & 6/13                        & 1/13                        & 5/15                        & 0/14                        & 5/16         & 6/16         & 6/12         & 3/12         \\
% 			&                    & %                     &
% 			 1\ts{st}/2\ts{nd}-Call & 30/3                         & TO/153                          & 24/8                        & TO/63                          & 231/81         & TO           & TO           & TO           \\
% 			\midrule
% 			\multirow{6}{*}{\textbf{Flexible}} & \multirow{3}{*}{2} & % \multirow{6}{*}{0}   &
% 			 Makespan    & 525                         & 475                         & 650                         & 650                         & 650          & 595          & 745          & 742          \\
% 			&                    & %                     &
% 			 Setup/Batch & 2/11                        & 0/11                        & 1/12                        & 1/12                        & 6/13         & 4/14          & 3/17          & n/a          \\
% 			&                    & %                     &
% 			 1\ts{st}/2\ts{nd}-Call & 26/7                         & TO                          & TO/12                          & TO                          & TO           & TO           & TO           & TO           \\
% 			\cline{2-11}
% %			& & & & & & & & & & &   \\
% 			& \multirow{3}{*}{3} & %                     &
% 			 Makespan    & 744                         & --                          & 1206                         & --                          & 1698 & -- & n/a & -- \\ % \multicolumn{4}{c|}{\multirow{3}{*}{Assignment issue}}     \\
% 			&                    & %                     &
% 			 Setup/Batch & 2/12                         & --                          & n/a                        & --                          & 8/15 & -- & n/a & -- \\ %\multicolumn{4}{c|}{}                                      \\
% 			&                    & %                     &
% 			 1\ts{st}/2\ts{nd}-Call & TO                       & --                          & TO                          & --                          & TO & -- & TO & -- \\ %\multicolumn{4}{c|}{}                                      \\
% 			\midrule
% 			\multirow{6}{*}{\textbf{Setup}}    & \multirow{3}{*}{2} & % \multirow{6}{*}{1}   &
% 			 Makespan    & 525                         & 475                         & 650                         & 650                         & 643          & 553          & 745          & 642          \\
% 			&                    & %                     &
% 			 Setup/Batch & 2/11                        & 0/11                        & 1/12                        & 1/12                        & 1/14         & 0/13          & 1/14          & 1/16          \\
% 			&                    & %                     &
% 			 1\ts{st}/2\ts{nd}-Call & 44/2                        & TO                          & TO/4                          & TO/2                          & TO           & TO/7           & TO           & TO           \\
% %			& & & & & & & & & & &   \\
% 			\cline{2-11}
% 			& \multirow{3}{*}{3} & %                     &
% 			 Makespan    & \textbf{346}                         & --                          & \textbf{373}                         & --                          & \textbf{429}          & --           & 820          & --           \\
% 			&                    & %                     &
% 			 Setup/Batch & n/a                         & --                          & n/a                         & --                          & n/a          & --           & n/a          & --           \\
% 			&                    & %                     &
% 			 1\ts{st}/2\ts{nd}-Call & TO                       & --                          & TO                          & --                          & TO           & --           & TO           & --           \\
% 			\hline
% 		\end{tabular}
% %	}
% \end{table}

% Table~\ref{tab:table03} considers $4$ machines in each tool group and the flexible strategy obtained the best result for the first instance. However, it had the same feasibility issue when all machines were in the same group. For the rest instances, the setup strategy dominated when the machines were equally distributed into sub-groups. 

% From the conducted experiments, we can conclude that 
% \begin{itemize}
% 	\item The flexible assignment performed well on the small-scale.
% 	\item While increasing the scale, the setup strategy dominates in the most cases
% 	\item Assigning the steps of the same lot independently with the fixed assignment leads to better performance
% 	\item The Setup strategy has a significant impact in minimizing the setup objective through all instances
% 	\item The full flexible assignment has an assignment issue while increasing the number of machines
% \end{itemize}

% \begin{table}[h]
% 	\centering
% 	\caption{Comparison between the allocation strategies with 4 machines per tool group}
% 	\label{tab:table03}
% %	\resizebox{15.5cm}{!}{%
% 		\begin{tabular}{|l%r
% 			cl||rr|rr|rr|rr|}
% 			\hline
% %			&                    &                      &  &  \multicolumn{8}{c}{\textbf{M = 12}} 
% %			\\ \hline
% 			& \textbf{M = 12}                   & %                     & 
% 			 & \multicolumn{2}{r|}{\textbf{Instance 09}}                 & \multicolumn{2}{r|}{\textbf{Instance 10}}                 & \multicolumn{2}{r|}{\textbf{Instance 11}}                 & \multicolumn{2}{r|}{\textbf{Instance 12}}                 \\
% 			& Size % \multicolumn{2}{l}{\textbf{Parameters}}            
% 			 &			 &			 Job                    & Step                   & Job                    & Step                   & Job                    & Step                   & Job                    & Step                   \\
% %			& Size               & setup % idx
% %			                  &  & 0                      & 1                      & 0                      & 1                      & 0                      & 1                      & 0                      & 1                      \\
% %			&                    & setup                &  &  &                        &                        &                        &                        &                        &                        &                                               \\
% 			\hline\hline
% 			\multirow{3}{*}{\textbf{Fixed}}    & \multirow{3}{*}{1} & % \multirow{3}{*}{0/1} &
% 			 Makespan                 & 525                    & 453                    & 525                    & 452                    & 525                    & 493                    & 643                    & 561                    \\
% 			&                    & %                     &
% 			 Setup/Batch              & 7/19                   & 3/20                   & 7/20                  & n/a                   & 6/22                   & 4/20                   & 4/22                   & n/a                   \\
% 			&                    & %                     &
% 			 1\ts{st}/2\ts{nd}-Call              & 124/5                 & TO & 25/17                 & TO & 25/53                 & TO/142 & TO & TO \\
% 			\midrule
% 			\multirow{9}{*}{\textbf{Flexible}} & \multirow{3}{*}{2} & % \multirow{9}{*}{0}   &
% 			 Makespan                 & \textbf{373}                    & 503                    & 491                    & 778                    & 569                    & 569                    & 765                    & 1673                   \\
% 			&                    & %                     &
% 			 Setup/Batch              & n/a                    & 6/17                    & n/a                   & n/a                    & n/a                    & n/a                   & n/a                    & 12/24                  \\
% 			&                    & %                     &
% 			 1\ts{st}/2\ts{nd}-Call              & TO & TO & TO & TO & TO & TO & TO & TO \\
% 			\cline{2-11}
% %			& & & & & & & & & & &   \\
% 			& \multirow{3}{*}{3} & %                     &
% 			 Makespan                 & 709                    & 688                    & 800                    & 907                    & 876                    & 876                    & 905                    & 1643                   \\
% 			&                    & %                     &
% 			 Setup/Batch              & 5/17                    & n/a                   & 3/18                   & 5/19                   & n/a                   & n/a                   & n/a                  & 15/24                    \\
% 			&                    & %                     &
% 			 1st/2nd              & TO & TO & TO & TO & TO & TO & TO & TO \\
% 			\cline{2-11}
% %			& & & & & & & & & & &   \\
% 			& \multirow{3}{*}{4} & %                     &
% 			 Makespan                 & n/a & -- & n/a & -- & n/a & -- & n/a & -- \\ %\multicolumn{8}{c|}{\multirow{3}{*}{Assignment issue}}                                                                                                                                                 \\
% 			&                    & %                     &
% 			 Setup/Batch              & n/a & -- & n/a & -- & n/a & -- & n/a & -- \\ %\multicolumn{8}{c|}{}                                                                                                                                                                                  \\
% 			&                    & %                     &
% 			 1\ts{st}/2\ts{nd}-Call              & TO & -- & TO & -- & TO & -- & TO & -- \\ %\multicolumn{8}{c|}{}                                                                                                                                                                                  \\
% 			\midrule
% 			\multirow{9}{*}{\textbf{Setup}}    & \multirow{3}{*}{2} & % \multirow{9}{*}{1}   &
% 			 Makespan                 & 401                    & 396                    & 419                    & \textbf{416}                    & \textbf{419}                    & \textbf{419}                    & \textbf{457}                    & 471                    \\
% 			&                    & %                     &
% 			 Setup/Batch              & 0/15                   & 0/14                   & 0/16                   & 0/16                   & n/a                   & n/a                   & 0/21                    & n/a                    \\
% 			&                    & %                     &
% 			 1\ts{st}/2\ts{nd}-Call              & TO & TO/92 & TO & TO & TO & TO & TO & TO \\
% 			\cline{2-11}
% %			& & & & & & & & & & &   \\
% 			& \multirow{3}{*}{3} & %                     &
% 			 Makespan                 & 706                    & 642                    & 792                    & 753                    & 942                    & 942                    & 939                    & 894                    \\
% 			&                    & %                     &
% 			 Setup/Batch              & 1/14                    & n/a                    & 2/16                    & n/a                   & n/a                   & n/a                    & n/a                    & 1/22                    \\
% 			&                    & %                     &
% 			 1\ts{st}/2\ts{nd}-Call              & TO & TO & TO & TO & TO & TO & TO & TO \\
% 			\cline{2-11}
% %			& & & & & & & & & & &   \\
% 			& \multirow{3}{*}{4} & %                     &
% 			 Makespan                 & 679                    & -- & 1725                    & -- & n/a                    & -- & n/a                    & -- \\
% 			&                    & %                     &
% 			 Setup/Batch              & n/a                   & -- & n/a                    & -- & n/a                   & -- & n/a                   & -- \\
% 			&                    & %                     &
% 			 1st/2nd              & TO & -- & TO & -- & TO & -- & TO & -- \\
% 			\hline
% 		\end{tabular}%
% %	}
% \end{table}

\section{Conclusion and Future Work}
In this work, I design corruption-robust algorithms for the Lipschitz contextual search problem. I present the \emph{agnostic checking} technique and demonstrate its effectiveness in designing corruption-robust algorithms. There are several open problems for future research. First, in the algorithm I propose for pricing loss, the schedule for agnostic checks is fixed upfront. Can the learner design an adaptive checking schedule for the pricing loss? Second, this work assumes the learner has knowledge of the Lipschitz constant $L$. Can the learner design efficient no-regret algorithms without knowledge of $L$? 





%%%%%%%%%%%%%%%%%%%%%%%%%%%%%%%%%%%%%%%%%%%%%%%%%%%%%%%%%%%%%%%%%%%%%%%%%%%%%%%%




% \addtolength{\textheight}{-12cm}   % This command serves to balance the column lengths
%                                   % on the last page of the document manually. It shortens
%                                   % the textheight of the last page by a suitable amount.
%                                   % This command does not take effect until the next page
%                                   % so it should come on the page before the last. Make
%                                   % sure that you do not shorten the textheight too much.

%%%%%%%%%%%%%%%%%%%%%%%%%%%%%%%%%%%%%%%%%%%%%%%%%%%%%%%%%%%%%%%%%%%%%%%%%%%%%%%%


% \section*{ACKNOWLEDGMENT}

% TBD



%%%%%%%%%%%%%%%%%%%%%%%%%%%%%%%%%%%%%%%%%%%%%%%%%%%%%%%%%%%%%%%%%%%%%%%%%%%%%%%%

\bibliographystyle{IEEEtran}
\balance
\bibliography{main}

\newpage

\begin{comment}
\section{System Architecture}
\label{appendix:architecture}
\system has a novel modularized system architecture with three key components: 
\emph{StreamManager}, 
\emph{TxnManager} and \emph{TxnScheduler}. 
These components are instantiated in each thread locally.
The execution outline of \system is presented in Algorithm~\ref{alg:algo}.
Transactional stream processing is continuous and potentially never ends (Line 1$\sim$8).
The dependency resolution and execution of state transactions are separated into two non-overlapping phases by punctuations~\cite{Tucker:2003:EPS:776752.776780} (Line 2 and 5), which guarantees that no subsequent input event will have a smaller timestamp. 
Effectively, a batch of state transactions is collected during the first phase, and processed during the second phase.

In the first phase (i.e., stream processing phase), 
the \emph{StreamManager} conducts preprocessing for every input event ($e$). Similar to some prior works~\cite{tstream}, state transactions may be issued but not immediately processed during preprocessing (Line 3).
The \emph{pre\_processing} and \emph{post\_processing} functions are exposed as APIs to users.
The \emph{TxnManager} handles dependency resolution (Line 4) among state transactions and insert decomposed operations to construct a \tpg. We discuss the detailed two-phase \tpg construction process in Section~\ref{subsec:construction}.

In the second phase  (i.e., transaction processing phase), 
the \emph{TxnManager} is first involved again to refine (Line 6) the constructed \tpg with further dependency resolution.
The \emph{TxnScheduler} 
schedules operations for concurrent execution based on the constructed \tpg according to the three dimensions of scheduling decisions (Line 7). 
In particular, a scheduling decision model $M$ is instantiated based on the constructed \tpg (Line 14).
\textbf{\circled{1}} Guided by $M$, execution threads adopt an exploration strategy (Section~\ref{subsec:explore}) to explore the constructed \tpg for operations available to be scheduled constrained by dependencies. 
\textbf{\circled{2}} 
During exploration, one or multiple operations may be treated as the 
% basic 
unit of scheduling (Section~\ref{subsec:granularity}). 
Subsequently, \textbf{\circled{3}} every thread executes operation(s) in the unit of scheduling with various abort handling mechanisms (Section~\ref{subsec:abort_handling}).
Only when state transactions are processed (i.e., committed or aborted) can the associated input events be postprocessed (Line 8) by the \emph{StreamManager} based on transaction processing results.
\end{comment}

\begin{comment}
\begin{algorithm}
\footnotesize
    \KwData{$e$ \tcp{Input event}}
    \KwData{$txn_{ts}$ \tcp{State transaction}}
    \KwData{$G$ \tcp{The currently constructed TPG}}
    \While{!finish processing of input streams}{
        \eIf(\tcp*[h]{Phase 1}){\text{$e$ is not a $punctuation$}}{
                $txn_{ts}$ $\gets$ PRE\_Processing($e$)\;
                \textbf{TPG\_Construction}($G$, $txn_{ts}$)\; 
          }(\tcp*[h]{Phase 2}){
                \textbf{TPG\_Refinement}($G$)\; 
                \textbf{TXN\_Scheduling}($G$)\; 
                POST\_Processing()\;
          }
    }
    
    \SetKwFunction{FMain}{TPG\_Construction}
    \SetKwProg{Fn}{Function}{:}{}
    \Fn{\FMain{$G$, $txn_{ts}$}}{
        $O_{1..k}$ $\gets$ \textbf{Partition} $txn_{ts}$\;
        \ForEach{\text{operation $O_{i}$ $\in$ $O_{1..k}$}}{
            \textbf{Identify} its \ld\;
            $G$ $\gets$ $G$ + $O_{i}$ \;
        }
    }
    \SetKwFunction{FMain}{TPG\_Refinement}
    \SetKwProg{Fn}{Function}{:}{}
    \Fn{\FMain{$G$}}{
        \ForEach{\text{vertex $e_{i}$ $\in$ $G$}}{
            \textbf{Identify} its \td, \pd\;
        }
    }
    
    \SetKwFunction{FMain}{TXN\_Scheduling}
    \SetKwProg{Fn}{Function}{:}{}
    \Fn{\FMain{$G$}}{
        $M$ $\gets$ Instantiated with $G$;\tcp{A decision model}
        \While{!finish scheduling of $G$
        }{
          \textbf{\circled{2}} $Scheduling Unit$ $\gets$ \textbf{\circled{1}} \emph{Explore}($G$, $M$)\; 
            \textbf{\circled{3}} \emph{Execute with Abort Handling} ($Scheduling Unit$)\; 
        }
    }
  \caption{Execution Outline of \system}
  \label{alg:algo}
\end{algorithm}
\end{comment}

% \onecolumn

% \begin{appendices}

% \section{}

% \subsection{Language Augmented TaskGrasp (LA-TaskGrasp) Dataset}

% In this section, we begin by presenting the natural language prompts utilized to generate object class and task descriptions within the LA-TaskGrasp dataset. Additionally, we provide detailed examples of object class descriptions (\textit{``Mug"}, \textit{``Spoon"}, and \textit{``Hammer"}) and task descriptions (\textit{``Sweep"}, \textit{``Screw"}, and \textit{``Slice"}). Following that, we present a list of 53 language instruction templates, consisting of 11 templates derived from a previous work and 42 templates generated through data augmentation using the LLM. Lastly, we offer 18 LA-TaskGrasp dataset examples, each of which includes 6 DoF task-oriented grasp poses, a language instruction, an object class description paragraph, and a task description paragraph.\\

% \subsubsection{Object Class Description Prompts} 

% \begin{itemize}
%     \setlength{\itemsep}{5pt}
%     \item ``\textit{Describe the shape/geometry of household object [obj] in a detailed and scientific response:}"
%     \item ``\textit{Describe the common use/function of household object [obj] in a detailed and scientific response:}"
%     \item ``\textit{Describe what household objects have similar shapes/geometries to [obj] in a detailed and scientific response:}"
%     \item ``\textit{Describe what household objects have similar uses/functions to [obj] in a detailed and scientific response:}" \\
% \end{itemize} 

% \subsubsection{Object Class Description Examples}\

% \begin{table}[h]
% \centering
% \renewcommand\arraystretch{1.3}
% \setlength\tabcolsep{5pt}%调列距
% \begin{tabular}{lcc}
% \hline
% \multicolumn{1}{c}{\textbf{Class}}                  & \textbf{Property Description}                                                                                                                                                                                                                                                                                                                             & \textbf{Similarity Description}                                                                                                                                                                                                                                                                                                                     \\ \hline \specialrule{0em}{3pt}{3pt}
% \multirow{6}{*}{\textit{Mug}}                       & \begin{tabular}[c]{@{}c@{}}(shape)\textit{``The mug is cylindrical in shape, with a slightly rounded} \\ \textit{base leading  up to straight walls which eventually taper} \\ \textit{slightly towards the rim."}\end{tabular}                                                                                                                                             & \begin{tabular}[c]{@{}c@{}}(similar shape)\textit{``Mugs typically have a cylindrical shape with a} \\ \textit{slightly tapered top and a curved handle; objects with similar} \\ \textit{shapes include bottles and vases."}\end{tabular}                                                                                                                             \\ \specialrule{0em}{3pt}{3pt}
%                                            & \begin{tabular}[c]{@{}c@{}}(use)\textit{``A mug is a cylindrical drinking vessel typically used to} \\ \textit{hold  hot beverages such as coffee, tea, hot chocolate,} \\ \textit{or soup. The curved shape of the mug allows liquids} \\ \textit{to be held and consumed while reducing splashes} \\ \textit{and spills."}\end{tabular}                                                      & \begin{tabular}[c]{@{}c@{}}(similar use)\textit{``Mugs and other household objects, such as glasses,} \\ \textit{jars, and other containers, can all be used to hold and contain} \\ \textit{liquids, such as hot or cold drinks. In addition, certain mugs,} \\ \textit{such as those with handles, can also be used to stir and mix} \\ \textit{drinks and other items."}\end{tabular} \\ \specialrule{0em}{3pt}{3pt} \hline \specialrule{0em}{3pt}{3pt}
% \multirow{5}{*}{\textit{Spoon}}                     & \begin{tabular}[c]{@{}c@{}}(shape)\textit{``The spoon has a slightly curved bowl shape with a} \\ \textit{long, narrow handle extending diagonally from the bowl in} \\ \textit{the opposite direction to the spoon’s curved edge."}\end{tabular}                                                                                                                            & \begin{tabular}[c]{@{}c@{}}(similar shape)\textit{``A teaspoon and a spoon have a similar shape,} \\ \textit{with both utensils curved at the bowl with a long, thin handle.} \\ \textit{In addition, a soup spoon has a shape similar to a spoon."}\end{tabular}                                                                                                      \\ \specialrule{0em}{3pt}{3pt}
%                                            & \begin{tabular}[c]{@{}c@{}}(use)\textit{``A spoon is a unit of measurement for dry and liquid} \\ \textit{ingredients that equals approximately three teaspoons,} \\ \textit{and is used for serving, measuring, and mixing foods."}\end{tabular}                                                                                                                            & \begin{tabular}[c]{@{}c@{}}(similar function)\textit{``A spoon is a utensil commonly used for} \\ \textit{measuring, stirring, and serving food. Household objects} \\ \textit{similar in function include a measuring cup, and soup ladle."}\end{tabular}                                                                                                             \\ \specialrule{0em}{3pt}{3pt} \hline \specialrule{0em}{3pt}{3pt}
% \multirow{7}{*}{\textit{Hammer}}                    & \begin{tabular}[c]{@{}c@{}}(shape)\textit{``The hammer has an elongated cylindrical shape with a} \\ \textit{flattened striking face at oneend and a handle at the other end."}\end{tabular}                                                                                                                                                                        & \begin{tabular}[c]{@{}c@{}}(similar shape)\textit{``A hammer has a similar shape as a meat} \\ \textit{tenderizer, an ice pick, and a nail puller."}\end{tabular}                                                                                                                                                                                             \\ \specialrule{0em}{3pt}{3pt}
%                                            & \begin{tabular}[c]{@{}c@{}}(use)\textit{``The hammer is a tool used to deliver an impulse force to an} \\ \textit{object. It works by transferring kinetic energy from the handle} \\ \textit{to the striking head of the hammer, allowing the user to drive} \\ \textit{nails, shape and flatten objects, and otherwise manipulate} \\ \textit{the material being worked upon."}\end{tabular} & \begin{tabular}[c]{@{}c@{}}(similar functions)\textit{``A household object that serves a similar} \\ \textit{function to that of a hammer is a kitchen mallet, which is} \\ \textit{used to pound, tenderize, and flatten food. Another} \\ \textit{common household item that serves a similar} \\ \textit{purpose is a rubber mallet.}\end{tabular}                                    \\  \specialrule{0em}{3pt}{3pt}  \hline
% \end{tabular}
% \caption{Examples of object class descriptions}
% \end{table}

% \newpage

% \subsubsection{Task Description Prompts}
% \begin{itemize}
%     \setlength{\itemsep}{5pt}
%     \item ``\textit{Describe what household objects can be used to [task] in a detailed and scientific response:}"
%     \item ``\textit{Describe what household objects support the function of [task] in a detailed and scientific response:}"
%     \item ``\textit{Describe what verbs are semantically close to [task] in a detailed and scientific response:}"
%     \item ``\textit{Describe what verbs achieve similar effects to ’[task] an object’ in a detailed and scientific response:}" \\
% \end{itemize} 

% \subsubsection{Task Description Examples}\

% \begin{table}[h]
% \centering
% \renewcommand\arraystretch{1.3}
% \setlength\tabcolsep{5pt}%调列距
% \begin{tabular}{lcc}
% \hline
% \multicolumn{1}{c}{\textbf{Task}}                   & \textbf{Affordance Description}                                                                                                                                                                                                                                                                                                                           & \textbf{Relevance Description}                                                                                                                                                                                                                                                                                                                      \\ \hline \specialrule{0em}{3pt}{3pt}
% \multicolumn{1}{c}{\multirow{5}{*}{\textit{Sweep}}} & \begin{tabular}[c]{@{}c@{}}(use)\textit{``Household objects that can be used to sweep include brooms,} \\ \textit{dustpans and mops.} "\end{tabular}                                                                                                                                                                                                                & \begin{tabular}[c]{@{}c@{}}(closeness)\textit{``Verbs that are semantically close to sweep} \\ \textit{include cleanse, purify, and eradicate."}\end{tabular}                                                                                                                                                                                                 \\  \specialrule{0em}{3pt}{3pt}
% \multicolumn{1}{c}{}                       & \begin{tabular}[c]{@{}c@{}}(function)\textit{``Household objects such as a broom, dustpan, and mop help} \\ \textit{to support the sweeping function by providing a tool with which to} \\ \textit{sweep away dust and debris. The broom helps to physically remove} \\ \textit{dirt and dust from the floors and other surfaces."}\end{tabular}                                      & \begin{tabular}[c]{@{}c@{}}(similar effect)\textit{``The action of sweeping an object can be} \\ \textit{described as a type of displacement, in which the object is} \\ \textit{moved across a surface in a steady, sweeping motion. This} \\ \textit{is similar to actions like pushing, dragging, and gliding."}\end{tabular}                                                \\  \specialrule{0em}{3pt}{3pt} \hline  \specialrule{0em}{3pt}{3pt}
% \multirow{6}{*}{\textit{Screw}}                     & \begin{tabular}[c]{@{}c@{}}(use)\textit{``Household objects that can be used to screw in objects} \\ \textit{include screwdrivers, managed screwdrivers, and various tools} \\ \textit{with rotating handles, such as adjustable wrenches, pliers,} \\ \textit{and socket wrenches."}\end{tabular}                                                                                    & \begin{tabular}[c]{@{}c@{}}(closeness)\textit{``Verbs that are semantically close to ‘screw’} \\ \textit{might include ‘twist’, ‘fasten’, or ‘tighten’, as these all} \\ \textit{fall within the general semantic domain of attaching or} \\ \textit{fastening objects or components together."}\end{tabular}                                                                   \\  \specialrule{0em}{3pt}{3pt}
%                                            & \begin{tabular}[c]{@{}c@{}}(function)\textit{``Household objects such as screwdrivers and} \\ \textit{power drills provide the necessary torque required to} \\ \textit{turn a screw and hold it in place, providing support} \\ \textit{for the function of the screw."}\end{tabular}                                                                                                & \begin{tabular}[c]{@{}c@{}}(similar effect)\textit{``Verbs such as ’fasten’, ’attach’, and ’secure’} \\ \textit{can have similar effects to ’screw an object’, by joining} \\ \textit{two pieces of material together and providing} \\ \textit{a strong connection."}\end{tabular}                                                                                             \\  \specialrule{0em}{3pt}{3pt} \hline  \specialrule{0em}{3pt}{3pt}
% \multirow{5}{*}{\textit{Slice}}                     & \begin{tabular}[c]{@{}c@{}}(use)\textit{``Household objects that can be used to slice include knives,} \\ \textit{graters, mandolines, and vegetable peelers. These objects} \\ \textit{typically have blades that are designed to cut through a} \\ \textit{variety of food items."}\end{tabular}                                                                                    & \begin{tabular}[c]{@{}c@{}}(closeness)\textit{``Verbs semantically close to slice include} \\ \textit{chop, divide, section, mince, dice and shred, as} \\ \textit{they all involve cutting an object into smaller pieces,} \\ \textit{either directly or through an intermediary tool."}\end{tabular}                                                                          \\  \specialrule{0em}{3pt}{3pt}
%                                            & \begin{tabular}[c]{@{}c@{}}(function)\textit{``Household objects that support the function of} \\ \textit{slicing include knives with sharp edges and fine serrations,} \\ \textit{as well as manual slicers that use an adjustable blade} \\ \textit{to create uniform slices."}\end{tabular}                                                                                        & \begin{tabular}[c]{@{}c@{}}(similar effect)\textit{``Verbs such as cleave, cut, and divide} \\ \textit{can also achieve the same effect as slicing an object,} \\ \textit{by physically splitting the object into two} \\ \textit{or more distinct parts."}\end{tabular}                                                                                                        \\  \specialrule{0em}{3pt}{3pt} \hline
% \end{tabular}
% \caption{Examples of task descriptions}
% \end{table}

% \newpage

% \subsubsection{Language Instruction Templates}\


% \begin{table}[h]
% \centering
% \renewcommand\arraystretch{2.0}
% \begin{tabular}{cc}
% \hline
% \multicolumn{2}{c}{\textbf{Language Instruction Templates}}                                                                                                                                                                       \\ \hline
% \textit{``use the \textless{}obj\textgreater to \textless{}task\textgreater{}"}                                   & \textit{``use the \textless{}obj\textgreater to perform \textless{}tasking\textgreater{}"}                            \\
% \textit{``\textless{}task\textgreater things with the \textless{}obj\textgreater{}"}                                & \textit{``use the \textless{}obj\textgreater to \textless{}task\textgreater something"}                               \\
% \textit{``executing \textless{}tasking\textgreater with the \textless{}obj\textgreater{}"}                          & \textit{``use the \textless{}obj\textgreater to conduct \textless{}tasking\textgreater{}"}                            \\
% \textit{``utilize the \textless{}obj\textgreater to \textless{}task\textgreater{}"}                                 & \textit{``just use the \textless{}obj\textgreater to \textless{}task\textgreater{}"}                                  \\
% \textit{``using a \textless{}obj\textgreater to \textless{}task\textgreater{}"}                                     &\textit{``do \textless{}tasking\textgreater with the \textless{}obj\textgreater{}"}                                   \\
% \textit{``perform \textless{}tasking\textgreater with the \textless{}obj\textgreater{}"}                            & \textit{``perform \textless{}tasking\textgreater using the \textless{}obj\textgreater{}"}                             \\
% \textit{``bring the \textless{}obj\textgreater out to \textless{}task\textgreater{}"}                               & \textit{``to \textless{}task\textgreater{}, get the \textless{}obj\textgreater{}" }                                   \\
% \textit{``find the \textless{}obj\textgreater so that you can \textless{}task\textgreater{}"}                       & \textit{``get the \textless{}obj\textgreater and start \textless{}tasking\textgreater{}"},                            \\
% \textit{``bring out the \textless{}obj\textgreater to \textless{}task\textgreater{}"}                               & \textit{``perform \textless{}tasking\textgreater with the \textless{}obj\textgreater{}"}                              \\
% \textit{``using a \textless{}obj\textgreater to do \textless{}tasking\textgreater{}"}                               & \textit{``make use of the \textless{}obj\textgreater to \textless{}task\textgreater{}"}                               \\
% \textit{``grab the \textless{}obj\textgreater to \textless{}task\textgreater{}"}                                    & \textit{``pick up the \textless{}obj\textgreater to \textless{}task\textgreater{}"}                                   \\
% \textit{``to \textless{}task\textgreater{}, hold the \textless{}obj\textgreater in your hand"}                      & \textit{``hold the \textless{}obj\textgreater in your hand and \textless{}task\textgreater{}" }                       \\
% \textit{``in order to \textless{}task\textgreater{}, grasp the \textless{}obj\textgreater{}"}                       & \textit{``grasp the \textless{}obj\textgreater in order to \textless{}task\textgreater{}"}                            \\
% \textit{``if you want to \textless{}task\textgreater{}, hold the \textless{}obj\textgreater{}"}                     & \textit{``grip the \textless{}obj\textgreater to \textless{}task\textgreater{}"}                                      \\
% \textit{``\textless{}task\textgreater with the \textless{}obj\textgreater{}"}                                       & \textit{``grasp the \textless{}obj\textgreater in a way that allows for \textless{}tasking\textgreater{}"}            \\
% \textit{``holding the \textless{}obj\textgreater in a \textless{}tasking\textgreater{}-friendly manner"}            & \textit{``take a \textless{}tasking\textgreater{}-friendly hold of the \textless{}obj\textgreater{}" }                \\
% \textit{``make sure you have a \textless{}tasking\textgreater{}-friendly grip on the \textless{}obj\textgreater{}"} & \textit{``hold the \textless{}obj\textgreater in a \textless{}tasking\textgreater manner"}                            \\
% \textit{``grip the \textless{}obj\textgreater in a \textless{}tasking\textgreater{}-friendly manner"}               & \textit{``ensure you have a \textless{}tasking\textgreater{}-friendly grip on the \textless{}obj\textgreater{}"}      \\
% \textit{``use the \textless{}obj\textgreater to \textless{}task\textgreater things"}                                & \textit{``performing \textless{}tasking\textgreater with the \textless{}obj\textgreater{}"}                           \\
% \textit{``use the \textless{}obj\textgreater to accomplish \textless{}tasking\textgreater{}"}                       & \textit{``\textless{}tasking\textgreater with the \textless{}obj\textgreater{}"}                                      \\
% \textit{``do \textless{}tasking\textgreater using the \textless{}obj\textgreater{}"}                                & \textit{``fetch the \textless{}obj\textgreater to \textless{}task\textgreater{}"}                                     \\
% \textit{``find the \textless{}obj\textgreater and then \textless{}task\textgreater{}"}                              & \textit{``obtain the \textless{}obj\textgreater for \textless{}tasking\textgreater{}"}                                \\
% \textit{``use the \textless{}obj\textgreater to conduct \textless{}tasking\textgreater{}"}                          & \textit{``grasp the \textless{}obj\textgreater to \textless{}task\textgreater{}"}                                     \\
% \textit{``taking hold of the \textless{}obj\textgreater{}, \textless{}task\textgreater{}"}                          & \textit{``to \textless{}task\textgreater{}, grasp the \textless{}obj\textgreater{}"}                                  \\
% \textit{``to \textless{}task\textgreater{}, take hold of the \textless{}obj\textgreater{}"}                         & \textit{``get the \textless{}obj\textgreater to \textless{}task\textgreater{}"}                                       \\
% \textit{``grasp the \textless{}obj\textgreater in a way that allows you to \textless{}task\textgreater{}"}          & \textit{``ensure you grasp the \textless{}obj\textgreater in a way that allows for \textless{}tasking\textgreater{}"} \\
% \textit{``hold the \textless{}obj\textgreater in a \textless{}tasking\textgreater position"}                        &                                                                                                             \\ \hline
% \end{tabular}
% \caption{Language instruction templates}
% \end{table}

% \newpage

% \subsubsection{Dataset Examples} \ 



% % Figure environment removed

% \newpage




% \subsection{Additional Experimental Setup}
% In this section, we aim to provide further information regarding the experimental setup employed for both model training and real-robot experiments. \\

% \subsubsection{Training Hyper-Parameters} \ 


% \begin{table}[h]
% \renewcommand\arraystretch{1.5}
% \centering
% \begin{tabular}{lc}
% \hline
% \multicolumn{2}{l}{\textbf{Basic Setting}}                      \\ \hline
% Batch Size           & 32                              \\
% \# of Points         & 4096                            \\
% \# of Epochs         & 50                              \\ \hline
% \multicolumn{2}{l}{\textbf{Optimization Setting}}               \\ \hline
% Optimizer            & Adam                            \\
% Learning Rate                  & 0.0001                          \\
% Learning Rate Decay             & 0.7                             \\
% Decay Step           & 2e4                             \\
% Weight Decay         & 0.0001                          \\
% Learning Rate Clip              & 1e-5                            \\ \hline
% \multicolumn{2}{l}{\textbf{PointNet++ Setting}}                 \\ \hline
% \# of SA Layers      & 3                               \\
% \# of Sampled Points & 512, 128, 1                     \\
% Embedding Sizes      & 320, 640, 1024                  \\ \hline
% \multicolumn{2}{l}{\textbf{Data Preprocessing}}                 \\ \hline
% Scaling              & True                            \\
% Mean Centering       & True                            \\
% Random Rotation      & True                            \\
% Random Jitter        & True                            \\
% Random Dropout       & True                            \\ \hline
% \multicolumn{2}{l}{\textbf{Hardware Resource}}                  \\ \hline
% CPU                  & 12th Gen Intel® Core™ i9-12900K \\
% \# of  CPU Cores     & 24                              \\
% GPU                  & Nvidia RTX 3090                 \\ \hline
% \multicolumn{2}{l}{\textbf{LLM Setting}}                  \\ \hline
% Model              & OpenAI GPT-3     \\
% Engine             & text-davinci-003 \\
% Prompt Type             & text             \\
% Temperature        & 1.0              \\
% Max Tokens        & 256              \\
% Top P             & 1.0              \\
% Frequency Penalty & 0.0              \\
% Presence Penalty  & 0.0              \\ \hline
% \end{tabular}
% \caption{Training hyper-parameter setting}
% \end{table}


% \newpage

% \subsubsection{Real-Robot Experiment} \ 

% % Figure environment removed


% % Figure environment removed

% % Figure environment removed






% \newpage

% % \subsection{Additional Results}

% % \subsubsection{Task-Oriented Grasping Experiments} \ 


% % \subsubsection{Task-Oriented Manipulation Experiments} \ 


% % \newpage


% \subsection{Discussion}
% In this section, we discuss the limitations of GraspGPT. Potential solutions are also provided as part of our future work.
% \begin{itemize}
%     \setlength{\itemsep}{5pt}
%     \item \textbf{LLM Knowledge Filtering}  

%     As previously mentioned, we do not perform specific preprocessing on the language data returned by an LLM, which can result in language descriptions containing imprecise or false commonsense knowledge. We identify two typical errors in the generated language descriptions: (1) Part-of-speech error. Since certain words have multiple uses as nouns and verbs, LLM occasionally returns object class knowledge even when prompted with task description prompts, or vice versa. For instance, when we prompt the LLM for the task description of verb \textit{``ladle"} (e.g., \textit{``Describe what household objects support the function of `ladle' in a detailed and scientific response:"}), the LLM might respond with: \textit{``A ladle is a utensil that is typically long-handled, with a deep dish or scoop at the end. It is usually made of metal or plastic, and is used to serve or measure hot liquids such as soup, sauce, or gravy."} (2) Mismatching descriptions. Due to significant intra-class variances, the LLM-generated object class descriptions may not precisely match the properties of actual object instances. For example, the object class description of \textit{``frying pan"} could be:  \textit{``The geometry of a frying pan is generally cylindrical, with sloping sides and a flat base to evenly disperse heat while cooking."} However, the actual object instance might be square-shaped. To address the first error, GraspGPT should be able to inspect semantic meaning of the generated language description and verify if it meets the prompt's intention. To tackle the second error, a potential solution would involve incorporating a large multimodal model, such as \href{https://arxiv.org/abs/2301.12597}{BLIP-2}, which can generate language descriptions based on the visual content provided. This integration would require GraspGPT to process additional visual inputs, such as RGB images.
        
%     \item \textbf{Task-Oriented Pick and Place} 
    
%     Our current work focuses on addressing the challenge of task-oriented grasping/picking. However, to achieve successful tool manipulation, the robot must also anticipate the subsequent motion of the tool and effectively interact with the target object. For instance, when inserting a nail into a slot, the robot needs to perform the following steps: (1) securely grasp the hammer by its handle, (2) guide the hammer towards the nail, assuming the nail is initially positioned halfway inside the slot, and (3) forcefully pound the nail into the slot. While GraspGPT has made progress in addressing the initial task-oriented grasping step, the subsequent steps are currently simplified with pre-defined motion primitives. Recent works, such as \href{https://arxiv.org/abs/2112.05124}{Neural Descriptor Fields}, approach tool manipulation as a pick-and-place task. This involves predicting the grasp point on the tool object (i.e., task-oriented picking) and determining the effect point on the target object (i.e., task-oriented placing). Since we currently use rule-based heuristics to determine the effect point, the robot is not able to model the relative pose between the tool object and the target object. Ideally, the robot should adjust the effect point depending the grasp point adaptively. A failure case can be found in the supplementary video. Our plan is to expand the capabilities of GraspGPT from task-oriented grasping to encompass task-oriented pick and place, leveraging both grasping and placement knowledge from an LLM.
        
%     \item \textbf{Single-Stage Architecture} 
    
%     GraspGPT currently follows a two-stage, sample-and-evaluate approach, similar to previous works. While this design choice simplifies the complexity of constructing GraspGPT, it introduces a reliance on a pre-trained task-agnostic grasp sampler. As the grasp sampler solely considers geometry information without incorporating semantic priors about the tool to be grasped, it uniformly samples over the given point cloud. However, during robot interaction, only specific functional/affordance regions of a tool are engaged, while non-functional regions remain untouched. This uniform sampling approach makes GraspGPT inefficient for real-time inference. Moreover, GraspGPT assumes all the candidate grasp poses generated by the grasp sampler are stable. However, the sampler may output marginal/unstable grasp poses which are susceptible to perturbances such as unpredicted contact or calibration error. An example can be found in the supplementary video. Future research aims to integrate task-agnostic sampling and task-oriented evaluation within an end-to-end architecture, enabling the direct prediction of task-oriented grasp poses from given point clouds. This approach are anticipated to consider both the stability and task-compatibility at the same time. 
    
%     \item \textbf{Simultaneous Affordance Learning} 
    
%      A closely related task to task-oriented grasping is affordance recognition, where the robot identifies specific regions on an object for various types of interactions. Previous studies, such as the \href{https://ieeexplore.ieee.org/abstract/document/9364360}{Affordance Keypoint Detection Network} (\textit{AffKP}), have demonstrated the benefits of joint learning of affordance segmentation for task-oriented grasping and manipulation. In future work, we aim to incorporate affordance learning into the GraspGPT framework. We anticipate that simultaneously learning these two objectives would mutually enhance their performance. On one hand, affordance learning would assist the robot in identifying the relevant regions to grasp for a given task. On the other hand, the supervision provided by task-oriented grasping could serve as weak supervision for affordance recognition. Equipping the robot with affordance recognition capability also opens up possibilities for other tasks, such as task-driven object retrieval/selection and semantic scene understanding.
    
    
    
    
   










    
% \end{itemize}



% \end{appendices}




\end{document}
