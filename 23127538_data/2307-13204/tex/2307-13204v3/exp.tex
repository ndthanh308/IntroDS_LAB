% \begin{table*}[th]
% \centering
% \renewcommand\arraystretch{1.6}
% \setlength\tabcolsep{20pt}%调列距
% \begin{tabular}{ccccccc}
% \toprule
% \multirow{2}{*}{\textbf{Method}} & \multicolumn{3}{c}{\textbf{Held-out Task Performance (mAP)}} & \multicolumn{3}{c}{\textbf{Held-out Class Performance (mAP)}} \\ \cline{2-7} 
%                         & Instance          & Class          & Task           & Instance           & Class          & Task           \\ \hline
% Random                  & 59.06             & 58.24          & 52.37          & 59.32              & 58.73          & 52.72          \\
% SGN                     & 75.17             & 71.59          & 63.35          & 74.20              & 72.95          & 62.55          \\ \hline
% GCNGrasp (open-world)   & 57.48             & 47.21          & 33.17          & 72.92              & 72.45          & 67.58          \\
% GCNGrasp (closed-world) & 80.43             & 76.06          & 76.11          & 79.35              & 76.88          & 72.97          \\ \hline
% GraspGPT (full model)   & 79.32             & 76.90          & 72.34          & 79.70              & 77.88          & 72.84          \\
% GraspGPT (w/o D)        & 74.66             & 70.85          & 68.14          & 74.10              & 74.33          & 66.38          \\
% GraspGPT (w/o TD)       & 75.00             & 71.21          & 68.38          & 80.95              & 77.74          & 73.76          \\
% GraspGPT (w/o OD)       & 78.26             & 74.71          & 71.60          & 76.04              & 74.71          & 71.60          \\ \bottomrule
% \end{tabular}
% % \caption{\textcolor{red}{result of perception experiments \textcolor{red}{MERGE THIS TABLE I AND TABLE II}}
% \caption{quantitative results of perception experiments}
% \label{tab:in-out-eval}
%   \vspace*{-0.15in}
% \end{table*}

\begin{table*}[th]
\centering
\renewcommand\arraystretch{1.6}
\setlength\tabcolsep{20pt}%调列距
\begin{tabular}{ccccccc}
\toprule
\multirow{2}{*}{\textbf{Method}} & \multicolumn{3}{c}{\textbf{Held-out Class Performance (mAP)}} & \multicolumn{3}{c}{\textbf{Held-out Task Performance (mAP)}} \\ \cline{2-7} 
                                                & Instance           & Class          & Task           & Instance          & Class          & Task           \\ \hline
Random                  & 59.32              & 58.73          & 52.72          & 59.06             & 58.24          & 52.37          \\
SGN                     & 74.20              & 72.95          & 62.55          & 75.17             & 71.59          & 63.35          \\ \hline
GCNGrasp (open-world)   & 72.92              & 72.45          & 67.58          & 57.48             & 47.21          & 33.17          \\
GCNGrasp (closed-world) & 79.35              & 76.88          & 72.97          & 80.43             & 76.06          & 76.11          \\ \hline
GraspGPT (full model)   & 79.70              & 77.88          & 72.84          & 79.32             & 76.90          & 72.34          \\
GraspGPT (w/o D)        & 74.10              & 74.33          & 66.38          & 74.66             & 70.85          & 68.14          \\
GraspGPT (w/o TD)       & 80.95              & 77.74          & 73.76          & 75.00             & 71.21          & 68.38          \\
GraspGPT (w/o OD)       & 76.04              & 74.71          & 71.60          & 78.26             & 74.71          & 71.60          \\\bottomrule
\end{tabular}
% \caption{\textcolor{red}{result of perception experiments \textcolor{red}{MERGE THIS TABLE I AND TABLE II}}
\caption{quantitative results of perception experiments}
\label{tab:in-out-eval}
  \vspace*{-0.15in}
\end{table*}





% \begin{table*}[t]
% \centering
% \renewcommand\arraystretch{1.6}
% \setlength\tabcolsep{12pt}%调列距
% \begin{tabular}{ccccccccc}
% \toprule
% \multirow{2}{*}{\textbf{Method}} & \multicolumn{3}{c}{\textbf{Held-out Task Performance}} & \multirow{2}{*}{Success} & \multicolumn{3}{c}{\textbf{Held-out Class Performance}} & \multirow{2}{*}{Success} \\ \cline{2-4} \cline{6-8}
%                         & Perception          & Planning         & Action         &                          & Perception          & Planning         & Action         &                          \\ 
% \hline
% GraspGPT   & 86/100             & 77/100            & 71/100           & 71.00\%                       & 91/100             & 85/100             & 77/100           & 77.00\%                      \\ \bottomrule
% \end{tabular}
% \caption{results of task-oriented grasping experiments}
% \label{tab:real_grasp}
%   \vspace*{-0.15in}
% \end{table*}

\begin{table*}[t]
\centering
\renewcommand\arraystretch{1.6}
\setlength\tabcolsep{12pt}%调列距
\begin{tabular}{ccccccccc}
\toprule
\multirow{2}{*}{\textbf{Method}} & \multicolumn{3}{c}{\textbf{Held-out Class Performance}} & \multirow{2}{*}{Success} & \multicolumn{3}{c}{\textbf{Held-out Task Performance}} & \multirow{2}{*}{Success} \\ \cline{2-4} \cline{6-8}
                        & Perception          & Planning         & Action         &                          & Perception          & Planning         & Action         &                          \\ 
\hline
GraspGPT   & 91/100             & 85/100            & 77/100           & 77.00\%                       & 86/100             & 77/100             & 71/100           & 71.00\%                      \\ \bottomrule
\end{tabular}
\caption{results of task-oriented grasping experiments}
\label{tab:real_grasp}
  \vspace*{-0.15in}
\end{table*}

\begin{table*}[t]
\centering
\renewcommand\arraystretch{1.6}
\setlength\tabcolsep{8pt}%调列距
\begin{tabular}{cccccccccc}
\toprule
\multirow{2}{*}{\textbf{Method}} & \multicolumn{2}{c}{\textbf{Pouring}} & \multirow{2}{*}{Success} & \multicolumn{2}{c}{\textbf{Handover}} & \multirow{2}{*}{Success} & \multicolumn{2}{c}{\textbf{Scooping}} & \multirow{2}{*}{Success} \\ \cline{2-3} \cline{5-6} \cline{8-9}
                        & Grasping   & Manipulation   &                          & Grasping    & Manipulation   &                          & Grasping    & Manipulation   &                          \\ \hline
GraspGPT                & 15/20      & 12/20         & 60.00\%                  & 17/20       & 16/20          & 80.00\%                  & 18/20       & 13/20          & 65.00\%                  \\ \bottomrule
\end{tabular}
\caption{results of task-oriented manipulation experiments}
\label{tab:real_mani}
  \vspace*{-0.3in}
\end{table*}

\section{Results} \label{exp}

% \begin{table*}[th]
% \centering
% \renewcommand\arraystretch{1.6}
% \setlength\tabcolsep{15pt}%调列距
% \begin{tabular}{ccccccc}
% \toprule
% \multirow{2}{*}{\textbf{Method}} & \multicolumn{3}{c}{\textbf{Held-out Task Performance (mAP)}} & \multicolumn{3}{c}{\textbf{Held-out Class Performance (mAP)}} \\ \cline{2-7} 
%                         & Instance       & Class       & Task       & Instance          & Class          & Task          \\ \midrule
% Random                  & 59.06          & 58.24       & 52.37      & 59.32             & 58.73          & 52.72         \\
% SGN                     & 75.17          & 71.59       & 63.35      & 74.20             & 72.95          & 62.55         \\ \hline
% GCNGrasp (out-of-graph) & 57.48          & 47.21       & 33.17      & 79.92             & 72.45          & 67.58         \\
% GCNGrasp (in-graph)     & 80.43          & 76.06       & 76.11      & 79.35             & 76.88          & 72.97         \\ \hline
% GraspGPT-MLP (ours)     & 76.57          & 73.05       & 72.61      & 78.20             & 76.57          & 69.50         \\
% GraspGPT-ATN (ours)     & 79.32          & 76.90       & 72.34      & 79.70             & 77.88          & 72.84         \\ \bottomrule
% \end{tabular}
% \caption{result of perception experiments}
% \label{tab:in-out-eval}
% \end{table*}

% \begin{table*}[th]
% \centering
% \renewcommand\arraystretch{1.6}
% \setlength\tabcolsep{15pt}%调列距
% \begin{tabular}{ccccccc}
% \toprule
% \multirow{2}{*}{\textbf{Method}} & \multicolumn{3}{c}{\textbf{Held-out Task Performance (mAP)}} & \multicolumn{3}{c}{\textbf{Held-out Class Performance (mAP)}} \\ \cline{2-7} 
%                         & Instance       & Class       & Task       & Instance          & Class          & Task          \\ \midrule
% Random                  & 59.06          & 58.24       & 52.37      & 59.32             & 58.73          & 52.72         \\
% SGN                     & 75.17          & 71.59       & 63.35      & 74.20             & 72.95          & 62.55         \\ \hline
% GCNGrasp (open-world) & 57.48          & 47.21       & 33.17      & 72.92             & 72.45          & 67.58         \\
% GCNGrasp (closed-world)     & 80.43          & 76.06       & 76.11      & 79.35             & 76.88          & 72.97         \\ \hline
% GraspGPT (ours)     & 79.32          & 76.90       & 72.34      & 79.70             & 77.88          & 72.84         \\ \bottomrule
% \end{tabular}
% \caption{result of perception experiments \textcolor{red}{MERGE THIS TABLE I AND TABLE II}}
% \label{tab:in-out-eval}
% \end{table*}

% Please add the following required packages to your document preamble:
% \usepackage{multirow}


% % Please add the following required packages to your document preamble:
% % \usepackage{multirow}
% \begin{table*}[th]
% \centering
% \renewcommand\arraystretch{1.6}
% \setlength\tabcolsep{15pt}%调列距
% \begin{tabular}{ccccccc}
% \hline
% \multirow{2}{*}{\textbf{Model}} & \multicolumn{3}{c}{\textbf{Held-out Task Performance (mAP)}} & \multicolumn{3}{c}{\textbf{Held-out Class Performance (mAP)}} \\ \cline{2-7} 
%                        & Instance          & Class          & Task           & Instance           & Class          & Task           \\ \hline
% w/o D                  & 74.66             & 70.85          & 68.14          & 74.10              & 74.33          & 66.38          \\
% w/o TD                 & 75.00             & 71.21          & 68.38          & \textbf{80.95}              & 77.74          & \textbf{73.76}          \\
% w/o OD                 & 78.26             & 74.71          & 71.60          & 76.04              & 74.71          & 71.60          \\ \hline
% Full model             & \textbf{79.32}             & \textbf{76.06}          & \textbf{72.34}          & 79.70              & \textbf{77.88}          & 72.84          \\ \hline
% \end{tabular}
% \caption{ablation on semantic knowledge}
% \label{tab:ablation_sn}
% \end{table*}



% \begin{table}[th]
% \centering
% \renewcommand\arraystretch{1.6}
% \setlength\tabcolsep{12pt}%调列距
% \begin{tabular}{cccc}
% \toprule
% \multirow{2}{*}{\textbf{Model}} & \multicolumn{3}{c}{\textbf{Held-out Task Performance (mAP)}} \\ \cline{2-4} 
%                        & Instance      & Class      & Task      \\ \hline
% w/o D                  & 74.66             & 70.85           & 68.14         \\
% w/o TD                  & 75.00             & 71.21           & 68.38         \\
% w/o OD                  & 78.26             & 74.71           & 71.60         \\ \hline
% Full model            & \textbf{79.32}          & \textbf{76.06}        & \textbf{72.61}         \\ \bottomrule
% \end{tabular}
% \caption{\textcolor{red}{ablation on semantic knowledge (held-out task)}}
% \label{tab:ablation_sn_task}
% \end{table}

% \begin{table}[th]
% \centering
% \renewcommand\arraystretch{1.6}
% \setlength\tabcolsep{12pt}%调列距
% \begin{tabular}{cccc}
% \toprule
% \multirow{2}{*}{\textbf{Model}} & \multicolumn{3}{c}{\textbf{Held-out class Performance (mAP)}} \\ \cline{2-4} 
%                        & Instance      & Class      & Task      \\ \hline
% w/o D                  & 74.10            & 74.33           & 66.38         \\
% w/o TD                  & \textbf{80.95}             & 77.74           & \textbf{73.76}         \\
% w/o OD                  & 76.04             &   74.14         &  68.42        \\ \hline
% Full model            & 79.70          & \textbf{77.88}        & 72.85         \\ \bottomrule
% \end{tabular}
% \caption{\textcolor{red}{ablation on semantic knowledge (held-out class)}}
% \label{tab:ablation_sn_obj}
% \end{table}


\subsection{Results of Perception Experiments}
To highlight the difference between our approach and GCNGrasp, we investigate perception experiments from two perspectives: open-world generalization and closed-world generalization. The former evaluates the generalization performance to novel concepts out of the knowledge graph of GCNGrasp. In the latter evaluation, GCNGrasp has access to all the concepts in the LA-TaskGrasp dataset and the ground truth relations between them in its pre-defined graph. Although this assumption is impractical in real-world robotic applications, we still want to explore how our approach compares to GCNGrasp even though GraspGPT does not assume access to concepts out of the training set. Since GraspGPT and the other two baselines do not rely on a pre-defined graph structure, their results for the two evaluations are the same. The quantitative results of perception experiments are reported in Table \ref{tab:in-out-eval}. 

\noindent \textbf{Open-World Generalization} \ For both held-out settings, Random achieves approximate mAPs of 50-60\%, indicating that the distribution of positive and negative samples in the dataset is even. By considering task constraints, SGN achieves consistent improvements (10\%+) over Random under two held-out settings. For GCNGrasp, we observe a significant performance difference between the held-out task setting and the held-out class setting. GCNGrasp even falls behind Random by 1.58\%, 11.03\%, and 19.20\% on three metrics in the held-out task setting. This suggests that the pre-trained word embeddings of ConceptNet are good at capturing the linguistic relations between object classes but perform poorly on relating task concepts. Therefore, GCNGrasp cannot fully exploit the power of semantic knowledge encoded in its graph. Since GraspGPT does not rely on a pre-defined KG but instead leverages the open-end semantic knowledge from an LLM, \textbf{GraspGPT outperforms all three baselines when generalizing to concepts out of the training set}. It outperforms GCNGrasp by 21.84\%, 29.69\%, and 39.17\% on held-out task setting and by 6.78\%, 5.43\%, and 5.26\% on held-out class setting. The qualitative results are presented in Fig. \ref{fig:qualitative}. \\

% \textcolor{red}{Specifically, GraspGPT-ATN performs better than GraspGPT-MLP in most cases. It is because the multi-head attention component in GraspGPT-ATN dynamically assigns weights to different input tokens, better filtering out irrelevant or redundant information in language descriptions. On the contrary, GraspGPT-MLP uses fixed weights to process language descriptions once trained.} \\ 





% \begin{table}[h]
% \centering
% \renewcommand\arraystretch{1.5}
% \begin{tabular}{cccc}
% \toprule
% \multirow{2}{*}{\textbf{Method}} & \multicolumn{3}{c}{\textbf{Test Performance (mAP)}} \\ \cline{2-4} 
%                         & Instance     & Class      & Task    \\ \hline
% Random                  & 59.06          & 58.24        & 52.37      \\
% SGN                     & 75.17          & 71.59        & 63.35       \\
% GCNGrasp                & 57.48          & 47.21        & 33.17      \\ \hline
% GraspGPT-MLP (ours)     & 76.57          & 73.05           & \textbf{72.61}         \\
% GraspGPT-ATN (ours)     & \textbf{79.32}          & \textbf{76.90}        & 72.34 \\ \bottomrule
% \end{tabular}
% \caption{Out-of-Graph Generalization Evaluation (Held-out Task)}
% \label{tab:out-task}
% \end{table}

% \begin{table}[h]
% \centering
% \renewcommand\arraystretch{1.5}
% \begin{tabular}{cccc}
% \toprule
% \multirow{2}{*}{\textbf{Method}} & \multicolumn{3}{c}{\textbf{Test Performance (mAP)}} \\ \cline{2-4} 
%                        & Instance      & Class      & Task     \\ \hline
% Random                  & 59.32          & 58.73        & 52.72      \\
% SGN                     & 74.20         & 72.95        & 62.55       \\
% GCNGrasp                & 72.92           & 72.45         & 67.58        \\ \hline
% GraspGPT-MLP (ours)     & 78.20             & 76.57           & 69.50         \\
% GraspGPT-ATN (ours)     & \textbf{79.70}             & \textbf{77.88}           & \textbf{72.84}         \\ \bottomrule

% \end{tabular}
% \caption{Out-of-Graph Generalization Evaluation (Held-out Object Class)}
% \label{tab:out-object}
% \end{table}

% Please add the following required packages to your document preamble:
% \usepackage{multirow}
% \begin{table*}[th]
% \centering
% \renewcommand\arraystretch{1.5}
% \begin{tabular}{ccccccc}
% \toprule
% \multirow{2}{*}{\textbf{Method}} & \multicolumn{3}{c}{\textbf{Held-out Task}}                                                           & \multicolumn{3}{c}{\textbf{Held-out  Object Class}}                                                  \\ \cline{2-7} 
%                         & Instance                        & Class                           & Task                            & Instance                        & Class                           & Task                            \\ \hline
% Random                  & 59.06                           & 58.24                           & 52.37                           & 59.32                           & 58.73                           & 52.72                           \\
% SGN                     & 75.17                           & 71.59                           & 63.35                           & 74.20                           & 72.95                           & 62.55                           \\
% GCNGrasp                & 57.48                           & 47.21                           & 33.17                           & 72.92                           & 72.45                           & 67.58                           \\ \hline
% GraspGPT-MLP (ours)     & 76.57                           & 73.05                           & \textbf{72.61} & 78.20                           & 76.57                           & 69.50                           \\
% GraspGPT-ATN (ours)     & \textbf{79.32} & \textbf{76.90} & 72.34                           & \textbf{79.70} & \textbf{77.88} & \textbf{72.84} \\ \bottomrule
% \end{tabular}
% \caption{out-of-graph generalization evaluation}
% \label{tab:out}
% \end{table*}

% \begin{table*}[t]
% \centering
% \renewcommand\arraystretch{1.6}
% \setlength\tabcolsep{12pt}%调列距
% \begin{tabular}{ccccllllll}
% \toprule
% \multirow{2}{*}{Method} & \multicolumn{3}{c}{Task-Oriented Grasping Performance} & \multirow{2}{*}{Success} & \multicolumn{4}{l}{Task-Oriented Manipulation Performance}                                & \multirow{2}{*}{Success} \\ \cline{2-4} \cline{6-9}
%                         & Perc.            & Plan.            & Act.             &                          & \multicolumn{1}{c}{Pouring} & \multicolumn{1}{c}{Handover} & \multicolumn{2}{c}{Scooping} &                          \\ \toprule
% GraspGPT                & xx/xx            & xx/xx            & xx/xx            & xx.xx\%                  & xx/xx                       & xx/xx                        & \multicolumn{2}{l}{xx/xx}    & xx.xx\%                  \\ \bottomrule
% \end{tabular}
% \caption{\textcolor{red}{XXXXXXXXXXXXXXXX}}
% \label{tab:real_exp}
% \end{table*}

\noindent \textbf{Closed-World Generalization} \ Compared to open-world generalization, GCNGrasp performs consistently better on closed-world generalization since all the concepts and the ground truth relations between them have been pre-defined in its graph. GraspGPT and GCNGrasp outperform both Random and SGN due to the incorporation of semantic knowledge. For the held-out task setting, GraspGPT achieves comparable performance with GCNGrasp on instance mAP and class mAP but falls behind by 3.77\% on task mAP. For held-out class setting, GraspGPT outperforms GCNGrasp on two metrics. Overall, \textbf{GraspGPT achieves comparable performance with GCNGrasp on closed-world generalization} even though it does not assume access to all concepts and their relations as GCNGrasp does.

% % Please add the following required packages to your document preamble:
% % \usepackage{multirow}
% \begin{table}[h]
% \centering
% \renewcommand\arraystretch{1.5}
% \begin{tabular}{cccc}
% \toprule
% \multirow{2}{*}{\textbf{Method}} & \multicolumn{3}{c}{\textbf{Test Performance (mAP)}} \\ \cline{2-4} 
%                         & Instance      & Class      & Task      \\ \hline
% Random                  & 59.06          & 58.24        & 52.37      \\
% SGN                     & 75.17          & 71.59        & 63.35       \\
% GCNGrasp                & \textbf{80.43}        & 76.06       & \textbf{76.11}   \\ \hline
% GraspGPT-MLP (ours)     & 76.57          & 73.05           & 72.61         \\
% GraspGPT-ATN (ours)     & 79.32          & \textbf{76.90}         & 72.34 \\ \bottomrule
% \end{tabular}
% \caption{In-Graph Generalization Evaluation \\ (Held-out Task)}
% \label{tab:in-task}
% \end{table}

% \begin{table}[h]
% \centering
% \renewcommand\arraystretch{1.5}
% \begin{tabular}{cccc}
% \toprule
% \multirow{2}{*}{\textbf{Method}} & \multicolumn{3}{c}{\textbf{Test Performance (mAP)}} \\ \cline{2-4} 
%                          & Instance    & Class      & Task      \\ \hline
% Random                  & 59.32          & 58.73        & 52.72      \\
% SGN                     & 74.20         & 72.95        & 62.55       \\
% GCNGrasp                & 79.35          & 76.88        & \textbf{72.97}      \\ \hline
% GraspGPT-MLP (ours)     & 78.20             & 76.57           & 69.50         \\
% GraspGPT-ATN (ours)     & \textbf{79.70}             & \textbf{77.88}           & 72.84         \\ \bottomrule
% \end{tabular}
% \caption{In-Graph Generalization Evaluation \\ (Held-out Object Class)}
% \label{tab:in-object}
% \end{table}

% Please add the following required packages to your document preamble:
% \usepackage{multirow}

% \begin{table*}[th]
% \centering
% \renewcommand\arraystretch{1.5}
% \begin{tabular}{ccccccc}
% \hline
% \multirow{2}{*}{Method} & \multicolumn{3}{c}{Held-out Task Setting} & \multicolumn{3}{c}{Held-out  Object Class Setting} \\ \cline{2-7} 
%                         & Instance       & Class       & Task       & Instance         & Class           & Task          \\ \hline
% Random                  & 59.06          & 58.24       & 52.37      & 59.32            & 58.73           & 52.72         \\
% SGN                     & 75.17          & 71.59       & 63.35      & 74.20            & 72.95           & 62.55         \\
% GCNGrasp                & \textbf{80.43}          & 76.06       & \textbf{76.11}      & 79.35            & 76.88           & \textbf{72.97}         \\ \hline
% GraspGPT-MLP (ours)     & 76.57          & 73.05       & 72.61      & 78.20            & 76.57           & 69.50         \\
% GraspGPT-ATN (ours)     & 79.32          & \textbf{76.90}       & 72.34      & \textbf{79.70}            & \textbf{77.88}         & 72.84         \\ \hline
% \end{tabular}
% \caption{in-graph generalization evaluation}
% \label{tab:}
% \end{table*}




% Please add the following required packages to your document preamble:
% \usepackage{multirow}
% \begin{table*}[t]
% \centering
% \renewcommand\arraystretch{1.6}
% \setlength\tabcolsep{16pt}%调列距
% \begin{tabular}{ccccccccc}
% \hline
% \multirow{2}{*}{\textbf{Method}} & \multicolumn{3}{c}{\textbf{Held-out Task Performance}} & \multirow{2}{*}{Success} & \multicolumn{3}{c}{\textbf{Held-out Class Performance}} & \multirow{2}{*}{Success} \\ \cline{2-4} \cline{6-8}
%                         & Perc.          & Plan.         & Act.         &                          & Perc.          & Plan.          & Act.         &                          \\ \hline
% Random                  & xx/xx             & xx/xx            & xx/xx           & xx.xx\%                       & xx/xx             & xx/xx             & xx/xx           & xx.xx\%                       \\
% GCNGrasp                & xx/xx             & xx/xx            & xx/xx           & xx.xx\%                       & xx/xx             & xx/xx             & xx/xx           & xx.xx\%                        \\ \hline
% GraspGPT (ours)     & xx/xx             & xx/xx            & xx/xx           & xx.xx\%                       & xx/xx             & xx/xx             & xx/xx           & xx.xx\%                      \\ \hline
% \end{tabular}
% \caption{result of real-robot experiments}
% \label{tab:real_exp}
% \end{table*}

% Please add the following required packages to your document preamble:
% \usepackage{multirow}
% Please add the following required packages to your document preamble:
% \usepackage{multirow}
% \begin{table}[th]
% \centering
% \renewcommand\arraystretch{1.6}
% \setlength\tabcolsep{12pt}%调列距
% \begin{tabular}{cccc}
% \hline
% \multirow{2}{*}{\textbf{Method}} & \multicolumn{3}{c}{\textbf{Task-Oriented Manipulation Performance}} \\ \cline{2-4} 
%                        & Pouring           & Handover           & Scooping          \\ \hline
% GraspGPT          & xx.xx\%             & xx.xx\%              & xx.xx\%             \\ \hline
% \end{tabular}
% \caption{result of task-oriented manipulation experiments \textcolor{red}{GRASPING+MANIPULATION STATISTICS}}
% \label{tab:mani_exp}
% \end{table}

% Please add the following required packages to your document preamble:
% \usepackage{multirow}


% Please add the following required packages to your document preamble:
% \usepackage{multirow}
% Please add the following required packages to your document preamble:
% \usepackage{multirow}


\subsection{Results of Real-Robot Experiments}

\noindent \textbf{Task-Oriented Grasping} \ We conduct 100 trials on each held-out setting, with ten trials per object class or task. As presented in Table \ref{tab:real_grasp}, GraspGPT achieves high success rates (86.00\% and 91.00\%) in the perception stage, even though the object point clouds are captured from a single view. The performance drop from the perception stage to the action stage (71.00\% and 77.00\%) can be attributed to three primary reasons: (1) marginal grasp candidates generated by the grasp sampler; (2) incorrect evaluation by GraspGPT; (3) motion planning failure. The qualitative results of three test objects are shown in Figure \ref{fig:qualitative} (right). \\

\noindent \textbf{Task-Oriented Manipulation} \ To support task-oriented manipulation (refer to Figure \ref{fig:real-robot}), we first utilize GraspGPT to generate task-oriented grasp poses for tool objects. Then, we design rule-based heuristics to determine the operating direction and effect points \cite{qin2020keto} on the target objects. As presented in Table \ref{tab:real_mani}, GraspGPT performs well in task-oriented grasping, achieving success rates of 75.00\%, 85.00\%, and 90.00\% in three tasks, respectively. However, due to its inability to adaptively model the relative pose \cite{pan2023tax} between the tool object and the target object, the success rates of task-oriented manipulation decrease, especially for pouring and scooping. 


% Future work includes extending GraspGPT to support task-oriented pick and place.


% Figure environment removed



\subsection{Ablation Study}\label{augment_exp}
To gain further insights into the effectiveness of each component of GraspGPT, we perform two sets of ablation studies, aiming to answer two questions:
\begin{itemize}
    \item Does the incorporation of semantic knowledge from an LLM help to better generalize to novel concepts out of the training set?
    \item How does the selection of a pre-trained language encoder affect the overall performance of GraspGPT?    
\end{itemize}



% \textcolor{red}{Ablation study is conducted on GraspGPT-ATN. We analyze the result of held-out task setting here, and leave the result of held-out class setting in the supplementary material.}\\  

\noindent \textbf{Ablation on Semantic Knowledge} \ We compare GraspGPT to three ablations: (1) no semantic knowledge (i.e., w/o D); (2) object class description only (i.e., w/o TD); (3) task description only (i.e., w/o OD). The results for two held-out settings are reported in Table \ref{tab:in-out-eval}. For the held-out task setting, the full model outperforms all three ablations. Specifically, the comparison between w/o D - w/o TD and w/o D - w/o OD demonstrates that the incorporation of task knowledge is more important for novel task generalization. For the held-out class setting, we observe that object class knowledge is more important for generalizing to novel classes out of the training set. Using object class knowledge only (w/o TD) even slightly outperforms the full model. We argue that object class descriptions have already provided sufficient knowledge for novel object class generalization. Arbitrarily incorporating extra task knowledge may lead to adversarial/conflicting effects in some cases. In our current implementation, we do not preprocess the language data from the LLM. Future work will be done on knowledge filtering and selection. Overall, the result verifies the hypothesis that \textbf{the incorporation of semantic knowledge helps achieve better generalization to novel concepts.} \\




% This phenomenon can be explained by the observation the LLM responses with longer object class descriptions (averaging 145.86 tokens per answer) compared to task descriptions (averaging 62.52 tokens per answer), which may lead to more imprecise or false commonsense knowledge. 





% \begin{table}[h]
% \centering
% \renewcommand\arraystretch{1.6}
% \setlength\tabcolsep{12pt}%调列距
% \begin{tabular}{cccc}
% \toprule
% \multirow{2}{*}{\textbf{Model}} & \multicolumn{3}{c}{\textbf{Held-out Task Performance (mAP)}} \\ \cline{2-4} 
%                        & Instances      & Classes      & Tasks      \\ \hline
% w/o D                  & 74.66             & 70.85           & 68.14         \\
% w/o TD                  & 75.00             & 71.21           & 68.38         \\
% w/o OD                  & 78.26             & 74.71           & 71.60         \\ \hline
% Full model            & \textbf{79.32}          & \textbf{76.06}        & \textbf{72.61}         \\ \bottomrule
% \end{tabular}
% \caption{Ablation on Semantic Knowledge}
% \label{tab:ablation_sn}
% \end{table}

\noindent \textbf{Ablation on Language Encoder} \ To validate the design choice of using a large pre-trained language encoder, we equip GraspGPT with pre-trained BERTs of three sizes and compare their resulting mAPs. Since the conclusions for the two held-out settings are similar, we only report the result of the held-out task setting for simplicity. The result is presented in Table \ref{tab:ablation_le}, where $L$ denotes the number of transformer layers. It is clear that \textbf{\textit{BERT-Base} outperforms two smaller models}, but the gaps are insignificant. We argue that the three models are equally pre-trained on a large corpus of text data, so they achieve a similar level of knowledge understanding capability despite their differences in model complexity. 

\begin{table}[t]
\centering
\renewcommand\arraystretch{1.6}
\setlength\tabcolsep{12pt}%调列距
\begin{tabular}{cccc}
\toprule
\multirow{2}{*}{\textbf{Model}} & \multicolumn{3}{c}{\textbf{Held-out Task Performance (mAP)}} \\ \cline{2-4} 
                         & Instance      & Class      & Task     \\ \hline
BERT-Small ($L$=4)               & 78.06             & 74.48           & 71.90         \\
BERT-Medium ($L$=8)              & 78.47             & 75.93           & 72.02         \\ \hline
BERT-Base ($L$=12)                & \textbf{79.32}          & \textbf{76.06}        & \textbf{72.34} \\ \bottomrule
\end{tabular}
\caption{ablation on pre-trained language encoder}
\label{tab:ablation_le}
  \vspace*{-0.3in}
\end{table}


% Please add the following required packages to your document preamble:
% \usepackage{multirow}
% \begin{table}[h]
% \centering
% \renewcommand\arraystretch{1.6}
% \setlength\tabcolsep{12pt}%调列距
% \begin{tabular}{cccc}
% \toprule
% \multirow{2}{*}{\textbf{Model}} & \multicolumn{3}{c}{\textbf{Held-out Task Performance (mAP)}} \\ \cline{2-4} 
%                          & Instances      & Classes      & Tasks      \\ \hline
% BERT-Small ($L$=4)               & 78.06             & 74.48           & 71.90         \\
% BERT-Medium ($L$=8)              & 78.47             & 75.93           & 72.02         \\ \hline
% BERT-Base ($L$=12)                & \textbf{79.32}          & \textbf{76.06}        & \textbf{72.34} \\ \bottomrule
% \end{tabular}
% \caption{Ablation on Pre-trained Language Encoder}
% \label{tab:ablation_le}
% \end{table}






% % Please add the following required packages to your document preamble:
% % \usepackage{multirow}
% \begin{table}[h]
% \centering
% \renewcommand\arraystretch{1.5}
% \begin{tabular}{ccccl}
% \toprule
% \multirow{2}{*}{\textbf{Method}} & \multicolumn{3}{c}{\textbf{Test Performance}} & \multicolumn{1}{c}{\multirow{2}{*}{Success}} \\ \cline{2-4}
%                         & Perc.         & Plan.        & Act.        & \multicolumn{1}{c}{}                              \\ \hline
% Random                  & xx            & xx           & xx          &                                                   \\
% GCNGrasp                & xx            & xx           & xx          &                                                   \\ \hline
% GraspGPT-ATN (ours)           & xx            & xx           & xx          &                                                   \\ \bottomrule
% \end{tabular}
% \caption{held-out class}
% \end{table}

% \begin{table}[h]
% \centering
% \renewcommand\arraystretch{1.5}
% \begin{tabular}{ccccl}
% \toprule
% \multirow{2}{*}{\textbf{Method}} & \multicolumn{3}{c}{\textbf{Test Performance}} & \multicolumn{1}{c}{\multirow{2}{*}{Success}} \\ \cline{2-4}
%                         & Perc.         & Plan.        & Act.        & \multicolumn{1}{c}{}                              \\ \hline
% Random                  & xx            & xx           & xx          &                                                   \\
% GCNGrasp                & xx            & xx           & xx          &                                                   \\ \hline
% GraspGPT-ATN (ours)           & xx            & xx           & xx          &                                                   \\ \bottomrule
% \end{tabular}
% \caption{held-out task}
% \end{table}
















