\documentclass[12pt,reqno]{amsart}
\usepackage{amsmath}
\numberwithin{equation}{section}
\usepackage{hyperref}
\hypersetup{nesting=true,debug=true,naturalnames=true}
\usepackage{graphicx,amssymb,upref}
\usepackage{enumerate}
\usepackage[all]{xy}
\usepackage{color}
\usepackage{mathrsfs}
\usepackage{amsfonts}
\usepackage{caption}
\usepackage{paralist}
\usepackage{color}
\usepackage{float}
\usepackage{tikz,pgf}
\usetikzlibrary{calc}
\usetikzlibrary{intersections,decorations.markings}
\usepackage{tikz-cd}



\theoremstyle{definition}
\newtheorem{theorem}{\bf Theorem}[section]
\newtheorem{prop}[theorem]{\bf Proposition}
\newtheorem{lemma}[theorem]{\bf Lemma}
\newtheorem{corollary}[theorem]{\bf Corollary}
\newtheorem{conjecture}[theorem]{\bf Conjecture}

\theoremstyle{definition}
\newtheorem{example}[theorem]{\bf Example}
\newtheorem{examples}[theorem]{\bf Examples}
\newtheorem{definition}[theorem]{\bf Definition}
\newtheorem{remark}[theorem]{\bf Remark}
\newtheorem{remarks}[theorem]{\bf Remarks}
\newtheorem{proposition}[theorem]{\bf Proposition}
     % Please give any \input statements here:
\newcommand{\mm}[1]{\mathrm{#1}}
\newcommand{\mb}[1]{\mathbb{#1}}
\setcounter{tocdepth}{1}




\begin{document}

\today




\title[ Topological classification of Bazaikin spaces]
{Topological classification of Bazaikin spaces}

\author[F. Fang]{Fuquan Fang}
\thanks{The first author was supported by
NSF Research Group Grant 11821101 of China and National Center of Applied Mathematics, Shenzhen}
\address{Department of Mathematics, Capital Normal University,
Beijing, P.R.China}
  \email{ffang@nankai.edu.cn}
\author[Wen Shen]{Wen Shen}
\address{Department of Mathematics, Capital Normal University,
Beijing, P.R.China  }
\email{shenwen121212@163.com}

\begin{abstract}
Geometry of manifolds with positive sectional curvature has been a central object dates back to the beginning of Riemannian geometry. Up to homeomorphism, there are only finitely many examples of simply connected positively curved manifolds in all dimensions except in dimension $7$ and $13$, namely, the Aloff-Wallach spaces and the Eschenburg spaces in dimension $7$, and the Bazaikin spaces in dimension $13$. The topological classification modelled on the $7$-dimensional examples has been carried out by Kreck-Stolz which leads to a complete topological classification for the Aloff-Wallach spaces. The main goal of this paper is to provide the topological classification of $13$-dimensional manifolds modelled on the Bazaikin spaces.
\end{abstract}



\maketitle
\tableofcontents

\section{Introduction}

Manifolds with positive sectional curvature have been a central object dates back to the beginning of Riemannian geometry. Except the rank one symmetric spaces, there are very examples of compact manifolds with positive sectional curvature
which exists only in dimensions $6$, $7$, $12$, $13$ and $24$ due to Wallach \cite{Wallach} and Berger \cite{Be}, and two infinite families, one in dimension $7$ (Aloff-Wallach spaces and Eschenburg spaces; see \cite{Aloff} \cite{E1} \cite{E2}) and the other in dimension $13$ (Bazaikin spaces; see \cite{Ba}).
The first infinite family in dimension 7 consists of the Eschenburg biquotients
$$\mathcal{E}_{k,l}=diag(z^{k_1},z^{k_2},z^{k_3})\backslash \mm{SU}(3)/diag(z^{l_1},z^{l_2},z^{l_3})^{-1},$$
with $\Sigma k_i=\Sigma l_i$ which includes the infinite sub-family of homogeneous Aloff-Wallach spaces
$$ \mathcal{W}_l= \mm{SU}(3)/diag(z^{l_1},z^{l_2},z^{l_3})$$
$\Sigma l_i=0.$ The second one exists in dimension 13 and consists
of the Bazaikin biquotients (as described in \cite{E3})
$$\mathcal{B}_\bold{q}=diag(z^{q_0},z^{q_1},z^{q_2},z^{q_3},z^{q_4},z^{q_5})\backslash \mm{SU}(6)/\mm{Sp}(3)$$
where $\bold{q}= (q_0, . . . , q_5)$ is a $6$-tuple of odd integers satisfying $\Sigma q_i = 0$. When the restrictions on $\bold{q}$ that are necessary for this biquotient to be a manifold are satisfied, we call $\mathcal{B}_\bold{q}$ a Bazaikin space.

Based on the modified surgery theory developed by Kreck \cite{K}, Kreck-Stolz \cite{KS1}\cite{KS} obtained a complete topological classification of $7$-manifolds modelled on the Aloff-Wallach spaces. As an important by product,  it was shown that the space of positively curved metrics on a certain Aloff-Wallach space is not path-connected. Further classification results on Eschenburg spaces $\mathcal{E}$ were obtained in \cite{kb1}, \cite{kb2}.

%In \cite{K}, the method about constructing the invariant in \cite{KS1} has been generalized, which is as the following theorem.
%\begin{theorem}\label{krec}\cite{K}
%Let $l\ge [n/2]-1$. A normal bordism $W$ of dimension $n +1 >4$ between two normal $l$-smoothings on manifolds $M_0$ and $M_1$ with the same Euler characteristics is bordant to an s-cobordism if and only if an algebraic obstruction %$\theta(W)$ is elementary. Thus $M_0$ and $M_1$ are diffeomorphic, if $n > 4$.
%\end{theorem}
%The theorem can also be proved in the topological (PL) category with appropriate modifications  from the basic results of \cite{k-s}.

In \cite{FL} Florit and Ziller studied the topology of Bazaikin spaces. Among others,  the cohomology rings, the link forms and the Pontryagin classes were calculated. Based on these basic calculation, the following finiteness assertions hold:
\begin{enumerate}
\item Given a cohomology ring, there are only finitely many positively curved Bazaikin spaces.
\item Given a homeomorphism type, there are only finitely many Bazaikin spaces.
\end{enumerate}

On the other hand, Kerin  \cite{KerM} presented infinitely many pairwise non-homeomorphic Bazaikin spaces admitting quasi-positive curvature with the same cohomology ring.

The main purpose of this paper addresses to the topological classification of the Bazaikin space $\mathcal{B}_\bold{q}$ where $\bold{q} = (q_0, . . . , q_5)$.
Throughout the rest of the paper let $H^i(-)$ (or $H_i(-)$) denote the integral cohomology (or homology) group $H^i(-;\mathbb{Z})$ (or $H_i(-;\mathbb{Z})$) unless other statements are given.


To state our main result, let us recall some basic facts on the topology of the Bazaikin space.
By \cite{Ba}, $\mathcal{B}_\bold{q}$ is simply connected and all of the non-vanishing cohomology groups are given by
$$H^0(\mathcal{B}_\bold{q})\cong   H^2(\mathcal{B}_\bold{q}) \cong   H^4(\mathcal{B}_\bold{q}) \cong   H^9(\mathcal{B}_\bold{q}) \cong   H^{11}(\mathcal{B}_\bold{q}) \cong   H^{13}(\mathcal{B}_\bold{q}) \cong  \mathbb{ Z},$$
$$and \quad H^6(\mathcal{B}_\bold{q}) \cong   H^8(\mathcal{B}_\bold{q}) \cong  \mathbb{ Z}_s$$
where $s=\frac 18 |\sigma _3(\bold{q}) |\in \Bbb Z$ which equals $\pm 1$ mod $6$, and $\sigma _3(\bold{q})$ is the $3$-rd elementary symmetric polynomial of $\bold{q}$. Moreover, if $\zeta\in H^2(\mathcal{B}_\bold{q})$ is a generator, then $\zeta^i$ is a generator of $H^{2i}(\mathcal{B}_{\bold q})$ for $i = 2$, $3$, $4$. This completely determines the cohomology ring structure of $\mathcal{B}_\bold{q}$. By \cite{FL}, the first Pontryagin class $p_1(\mathcal{B}_\bold{q}) = 7$ or $15$ mod $24$. In particular, note that replacing $\bold q$
with $-\bold q$, one obtains the same manifold. Thus we can take appropriate $\bold q$ so that $\sigma _3(\bold{q})>0$.

Now we are ready to present the main theorem of this paper.


\begin{theorem}\label{classify}
If $s\ne 0\mod 5$, then Bazaikin spaces $\mathcal{B}_{\bold q}$ and $\mathcal{B}_{\bold q^\prime}$ are homeomorphic if and only if $\sigma_2({\bold q})=\sigma_2({\bold q}^\prime)$ and $\sigma_3({\bold q})=\sigma_3({\bold q}^\prime)=8s$, and $$\quad \sigma_4({\bold q}) =\sigma_4({\bold q}^\prime)\mod s  \text{ and } \sigma_5({\bold q})=\sigma_5({\bold q}^\prime)\mod s. $$
\end{theorem}

Indeed, $\sigma _2({\bold q})$, $\sigma _4({\bold q})$ and $\sigma _5({\bold q})$ reflect the topological data of the Pontryagin classes $p_1$, $ p_2$ and the linking form of  $\mathcal{B}_\bold{q}$ \cite{FL}.



For $s=0\mod 5$, it is not clear whether $\sigma_i(\bold{q})$ mod $s$ for $2\le i\le 5$ are sufficient to distinguish the homeomorphic types of the Bazaikin spaces, but we have the following

\begin{theorem}\label{tops0}
	If $s=0\mod 5$, there are $25$ homeomorphic types at most for the Bazaikin spaces with the same $\sigma_2({\bold q})$, $\sigma_3({\bold q})$ and the modulo $s$ values of $ \sigma_4({\bold q})$ as well as $   \sigma_5({\bold q}).$
\end{theorem}

For the diffeomorphic types, we  have the following
\begin{theorem}\label{diff}
	There are $12$ diffeomorphic structures at most on a given homeomorphism type of Bazaikin spaces.
\end{theorem}	

The organization of this paper is as follows. In Section \ref{Normalbg} we set up a suitable bordism theory for Bazaikin space in smooth category and present  the necessary results of this paper. In Section \ref{cohomologybarB}, \ref{pprimary}, \ref{2primary}, \ref{3primary} we mainly focus on the computation for certain bordism group. The bordism invariants of certain direct summands of the bordism group are given in Section \ref{Zs}. In Section \ref{oriented}, we compute some basic homotopy propositions for Bazaikin space. In Section \ref{Characteristic}, \ref{koMketa}, \ref{Detect} we deal with the bordism invariants of other direct summands of the bordism group. In Section \ref{PL} the bordism theory in topological category will be discussed. In Section \ref{obstruction}, \ref{mainproof} we make using the modified surgery theory to prove the main theorems announced above.








\section{Normal bordism group}\label{Normalbg}




Let $\pi: \mathscr{B}\to \mm{BO}$ be a fibration over the classifying space $\mm{BO}$. Given a smooth manifold $M$ with Gauss map $\mathscr{N}_{M} :M\to \mm{BO}$, according to \cite{K},  a normal $\mathscr{B}$-structure on $M$ is a lifting $ \bar{\mathscr{N}}_{M}: M\to \mathscr{B}$ of $ {\mathscr{N}}_{M}$. Moreover, $\bar {\mathscr{N}}_{M}$ is called a normal $\ell$-smoothing when  it is a $(\ell + 1)$ homotopy equivalence.

The first key step for the classification is to find a model $\pi: \mathscr{B}\to \mm{BO}$ suitable for all Bazaikin spaces, then we will compute the bordism group by using a very delicate computation of the Adams spectral sequences. The second step is to analyze which topological invariants can characterize the elements of the bordism group. The last step is to determine if two $\mathscr{B}$-bordant Bazaikin spaces are homeomorphic or diffeomorphic, for which we use the modified surgery theory of Kreck \cite{K}. It turns out that there is no additional surgery obstruction which is in contrast to a very different phenomenon from the $7$-dimensional models for the Eschenburg spaces.




\subsection{The normal $5$-smoothing of Bazaikin spaces}\label{normal5}

%Now we construct a normal $5$-smoothing of $\mathcal{B}_{\bold q}$.

Let $x$ be a generator of $H^2(\mathrm{CP}^\infty)$. The map $g:\mathrm{CP}^\infty\to \mm{K}(\mathbb{ Z},6)$ represents the cohomology class $g^\ast(l_6)=sx^3\in H^6(\mm{CP}^\infty)$ where $s$ is a positive integer, $l_6$ is a generator of $H^6(\mm{K}(\mb{Z},6))$. Consider the fibration $\mm{K}(\mb{Z},5) \to B \to \mm{K}(\Bbb Z, 6)$ given by the classifying map $g$. Then we have the commutative diagram \ref{fib}
\begin{equation}\begin{tikzcd}
 &    &\mm{K}(\mathbb{Z},5) \ar[rr] \ar[d,"i"]&  & \mm{K}(\mathbb{Z},5)  \ar[d]\\
&     & {B}\ar[rr] \ar[d,"p"]& &\mm{PK}(\mathbb{Z},6)\ar[d]\\
\mathcal{B}_{\bold{q}}\ar[rr,""]\ar[rru,dotted,"{f}"]&  &{\mathrm{CP}}^\infty \ar[rr,"g"] &  & \mm{K}(\mathbb{Z},6)
\end{tikzcd}
\label{fib}
\end{equation}
where $ f $ is a lifting of the map $\mathcal{B}_{\bold{q}}\to \mm{CP}^\infty$ represented by the generator $\zeta \in H^2( \mathcal{B}_{\bold q})$.
Note that  the first Pontryagin class $p_1(\mathcal{B}_{\bold{q}}) =k\zeta^2$ for some positive integer $k$, the second Stiefel-Whitney class  $w_2(\mathcal{B}_{\bold{q}})=k \zeta\pmod{2}$.
Let $\mathcal{H}$ denote the Hopf bundle over $\mathrm{CP}^\infty$ and let $\xi=-k\mathcal{H}$ be the virtual complementary bundle of the Whitney sum $k\mathcal{H}$ (which is a vector bundle over any finite dimensional skeleton),  let $\eta$ denote $p^\ast(\xi)$ where $p: B\to \mathrm{CP}^\infty$ is as in the diagram.

Considering the classifying map
$$\pi: \mathscr{B}=\mm{BO}\langle 8 \rangle \times  B\to \mm{BO}$$ of the bundle $\gamma_8\times \eta$ where $\gamma_8$ is the universal bundle over $\mm{BO}\langle 8 \rangle$,
we claim that the Bazaikin space $\mathcal{B}_{\bold q}$ has a normal $\mathscr{B}$-structure, i.e., the Gauss map $\mathscr{N}_{\mathcal{B}_\bold{q}}$ lifts to $\bar{\mathscr{N}_{\mathcal{B}_\bold{q}}}: \mathcal{B}_{\bold q}\to \mathscr{B}$ which is furthermore a $5$-smoothing.


Note that  the Chern classes
$$c_1(\xi)=-kx\quad c_2(\xi)=\frac{k(k+1)}{2}x^2,$$
the first Pontryagin class $p_1(\xi)=-kx^2$. Moreover, the Stiefel-Whitney classes $w_2(\xi)\ne 0$ and $w_4(\xi)=0$ since $k$ is determined by $p_1(\mathcal{B}_{\bold{q}}) $ which is equal to $7$ or $15$ mod $24$ by \cite{FL}.  Recall that the map $p$ is a $5$-equivalence, we get that
$$p_1(\eta)=-kx^2\quad w_2(\eta)\ne 0\quad  w_4(\eta)=0$$
with the same notion $x$ denoting the generator of $H^2( B)$.
Thus the difference bundle $\mathscr{N}_{\mathcal{B}_{\bold q}}-f^\ast(\eta)$ has a $\mm{BO}\langle 8 \rangle$-structure where $\mathscr{N}_{\mathcal{B}_{\bold q}}$ is the stable normal bundle of the Bazaikin space, i.e.,  its classifying map $\mathcal{B}_{\bold q} \to \mm{BO}$ lifts to $ \nu : \mathcal{B}_{\bold q} \to \mm{BO}\langle 8 \rangle$. In other words, the stable normal bundle $\mathscr{N}_{\mathcal{B}_{\bold q}}$ satisfies
$$\mathscr{N}_{\mathcal{B}_{\bold q}}=\nu^{\ast} \gamma_8\oplus f^\ast(\eta),$$
$\bar{\mathscr{N}}_{\mathcal{B}_{\bold q}}: \mathcal{B}_{\bold q}\to \mathscr{B}$ is the composite
$$\mathcal{B}_{\bold q}\stackrel{\triangle}{\rightarrow}\mathcal{B}_{\bold q}\times \mathcal{B}_{\bold q}\stackrel{({\nu},f)}{\rightarrow} \mathscr{B}=\mm{BO}\langle 8\rangle\times {B}.$$

In additional we note that the $5$-smoothing of $\mathcal{B}_{\bold q}$ is unique up to homotopy since $H^1(\mathcal{B}_{\bold q};\mathbb{Z}_2)=0,$ $ H^3(\mathcal{B}_{\bold q})=H^5(\mathcal{B}_{\bold q})=0.$





\subsection{Statements of technical results}

Note that manifolds with normal $\mathscr{B}$-structure form a bordism group $\Omega_\ast^{\mm{O}\langle 8 \rangle}(\eta)$ which is isomorphic to the homotopy group $\pi_\ast(\mm{M\mathscr{B}})$ by the Pontryagin-Thom isomorphism where $\mm{M\mathscr{B}}=\mm{MO}\langle 8 \rangle\wedge \mm{M}\eta $,  $\mm{M}\eta$ and $\mm{MO}\langle 8\rangle$ are the Thom spectra of the bundles $\eta$ and $\gamma_8$.

In order to compute the homotopy group $\pi_\ast(\mm{M\mathscr{B}})$ we have to compute first the cohomology groups of $\mm{M}\eta$ (see section \ref{cohomologybarB}). The main tools are the Atiyah-Hirzebruch spectral sequence (AHSS) with $E_2$-terms  $E_2^{p,q}=H_p(\mm{M}\eta;\pi_q(\mm{MO}\langle 8\rangle))$, and the Adams spectral sequence (ASS), in particular for the computations of the $p$-primary part of $\pi_{13}(\mm{M\mathscr{B}})$ for odd prime number $p>3$ and $2$, $3$-primary parts respectively (see sections \ref{pprimary}, \ref{2primary}, \ref{3primary}). The results for the computation of the group $\pi_{13}(\mm{M\mathscr{B}})$ reads as follows
\begin{theorem}	
\label{bordismgroup}
$\Omega_{13}^{\mm{O}\langle 8 \rangle}(\eta)\cong G\oplus T_2\oplus T_3$
\newline where $G$ satisfies the exact sequence $0\to \mb{Z}_s\to G\to \mb{Z}_s\oplus \mb{Z}_5\to 0$, in particular $G=\mb{Z}_s\oplus \mb{Z}_s\oplus \mb{Z}_5$ when $s\ne 5$;
 $T_2$ is the $2$-primary part and $T_3$ is  the $3$-primary part.
$$T_2=\mathbb{Z}_8\oplus \mathbb{Z}_4\oplus \mathbb{Z}_2^n$$
$\mathbb{Z}_2^n$ denotes the direct sum of $n$ cyclic groups of order $2$, $n=1$ or $2$.
\[
T_3=\left\{
\begin{array}{lr}
\mathbb{Z}_3\oplus \mathbb{Z}_3\oplus \mathbb{Z}_3, & k=0\mod 3 \\
\mathbb{Z}_9\quad or \quad \mathbb{Z}_3\oplus \mathbb{Z}_3  & k\ne 0\mod 3
\end{array}
\right.
\]
$k$ is the integer as above given by the first Pontryagin class  $p_1(\mathcal{B}_\bold{q})$.
\end{theorem}

Note that the uncertainty in $T_2$ and $T_3$ is due to that an Adams differential (see Section \ref{2primary}) and an extension as $a_0$-module are not determined, but it does not affect the homeomorphism classifications for the Bazaikin spaces.
%Remarks: The ambiguity of $"or"$ should be clarified.

\begin{lemma}\label{Zsinvar}
Assume $s\ne 0 \mod 5$. Then the component in $\mb{Z}_s\oplus \mathbb{Z}_s$  of the bordism class of a Bazaikin space is determined by the second Pontryagin class and the  link form.
\end{lemma}

Indeed, when $s\ne 0\mod 5$, the group $\mathbb{Z}_s\oplus \mathbb{Z}_s \subset \Omega_{13}^{\mm{O}\langle 8 \rangle}(\eta)\cong \pi_{13}(\mm{M\mathscr{B}})$ is mapped to $H_{13}(\mm{M\mathscr{B}})$ injectively by the Hurewicz homomorphism, thus
the bordism class in $\mathbb{Z}_s\oplus \mathbb{Z}_s$ can be detected by the image of orientable class of the manifold $M$ with $\mathscr{B}$-structure under the map $\bar{\mathscr{N}}_M$.  Unfortunately, some elements of $T_2,T_3$ are in the kernel of the Hurewicz homomorphism which forces us to use the $ko$-homology theory.


We will consider the bordism group $\Omega_{13}^{\mm{Top}\langle 8\rangle}(\eta)$ in topological  category in Section \ref{Detect} and \ref{PL} by proving the following

\begin{lemma}\label{topcobordism} Given Bazaikin spaces $\mathcal{B}_{\bold{q}}$ and $\mathcal{B}_{\bold{q}^\prime}$,  if $s\ne 0 \mod 5$ and $\sigma_2({\bold q})=\sigma_2({\bold q}^\prime)$, $\sigma_3({\bold q})=\sigma_3({\bold q}^\prime)=8s$ and moreover
 $$ \sigma_4({\bold q}) =\sigma_4({\bold q}^\prime)\mod s  \quad \text{ and } \quad \sigma_5({\bold q})=\sigma_5({\bold q}^\prime)\mod s,$$
 then $[\mathcal{B}_{\bold{q}}]=[\mathcal{B}_{\bold{q}^\prime}]$ in $\Omega_{13}^{\mm{Top}\langle 8\rangle}(\eta)$.
\end{lemma}

In Section \ref{obstruction} we will develop the modified surgery theory to prove that

\begin{lemma}\label{chongfen1} Under the same condition as in the lemma above, the Bazaikin spaces $\mathcal{B}_{\bold{q}}$ and $\mathcal{B}_{\bold{q}^\prime}$ are homeomorphic.
\end{lemma}

We remark that, for the case of $s=0\mod 5$, the second Pontryagin class can not determine  a certain summand $\mb{Z}_s\subset \Omega_{13}^{\mm{O}\langle 8\rangle}(\eta)$. More details will be discussed in Section  \ref{mainproof}.

\section{The cohomology ring of ${B}$}\label{cohomologybarB}

In this section, we use the Serre spectral sequence for the fibration $$\mm{K}(\mathbb{Z},5)\stackrel{i}{\rightarrow} {B}\stackrel{p}{\rightarrow} {\mathrm{CP}}^\infty$$
to compute the cohomology ring $H^\ast({B})$, the module $H^\ast(B;\mb{Z}_2)$ and $H^\ast(B;\mb{Z}_3)$ over the Steenrod algebra. Besides we give the preparatory computations of partial cohomology rings $H^\ast(\mm{K}(\mb{Z},n))$ for $3\le n \le 6$.
\subsection{$H^\ast(B;\mathbb{Z}_2)$ and its module over the Steenrod algebra}
\quad

From the fibration diagram \ref{fib}, we have the Serre spectral sequence (SSS) with $\mathbb{Z}_2$-coefficient for the fibration $\mm{K}(\mathbb{Z},5)\stackrel{i}{\rightarrow} B\stackrel{}{\rightarrow} {\mathrm{CP}}^\infty$
$$ E_2^{p,q}(B)=H^p(\mm{CP}^\infty;H^q(\mm{K}(\mb{Z},5);\mb{Z}_2))\Longrightarrow H^{p+q}(B;\mb{Z}_2) .$$
Recall that $s=\pm 1 \mod 6$ for a Bazaikin space being a manifold and moreover, $ E_2^{p,q}(B) \cong E_6^{p,q}(B)$ and the differential $$d_6(l_5)=x^3\in H^6({\mathrm{CP}}^\infty;\mathbb{Z}_2)$$
by the definition  of the fibration and the naturality of the spectral sequence, where $l_5\in H^5(\mm{K}(\mathbb{Z},5);\mathbb{Z}_2)$.



\begin{lemma}\label{vanish0}
$E_7^{p,q}(B)=0$ for $p\ge 5.$
\end{lemma}
\begin{proof}
Obviously $E_2^{p,q}(B)=0$ when $p$ is odd. For $p\geq 6$, note that every element of $E_2^{p,q}(B)=E_6^{p,q}(B)$ is of the form $x^3a$ where $x^3\in E_6^{6,0}(B)$, $a\in E_6^{p-6,q}(B)$.  To prove the claim, we consider the differential $d_6$. If $x^3a$ is in the kernel of some $d_6$, i.e., $d_6(x^3a)=0$, it follows that $d_6(a)=0$ by the product of the Serre spectral sequence. We asserts that $x^3a$ is also in the image of a certain $d_6$. Indeed, $d_6(l_5a)=x^3a$. The desired result follows.
\end{proof}

\begin{prop}\label{convergence}
{$E_\infty^{p,q}(B)=E_7^{p,q}(B)$.}
\end{prop}
\begin{proof}
The desired result follows since  all differentials $d_r$ ($r\ge 7$) are trivial by
Lemma \ref{vanish0}.
\end{proof}

Recall that $H^\ast(\mm{K}(\mathbb{Z},n);\mathbb{Z}_2)$ is the polynomial ring with generators $l_n$ and $\mathrm{Sq}^\mm{I} l_n$, where $\mm{I}=(i_1,i_2,\cdots,i_r)$ is the admissible sequence with excess
$$ e(\mm{I})=2i_1-(i_1+\cdots +i_r )< n$$
and $i_r>1.$ In the Serre spectral sequence for the fibration
$$\mm{K}(\mathbb{Z},5)\to \mm{PK}(\mathbb{Z},6)\simeq \ast \to \mm{K}(\mathbb{Z},6)$$
 with $\mathbb{Z}_2$-coefficient,
$d_6(E_6^{0,q})=0$
 obviously for $5<q\le 10$, then for any element $a\in E_6^{0,q}$,
$d_6(l_5a)=l_6a$. By the naturality of the Serre spectral sequence, for $q\le 15$, $E_7^{p,q}(B)$ is a truncated polynomial ring generated by $x$, $l_5^2$ and
$\mathrm{Sq}^\mm{I} l_5$ where $e(\mm{I})<5$, $0<d(\mm{I})=i_1+\cdots +i_r\le 10$. In summary we have the following  table for the cohomology ring $H^\ast(B;\mathbb{Z}_2)$ in the range up to dimension $15$ where $u \in H^7(B;\mathbb{Z}_2)$ and $v \in   H^9(B;\mathbb{Z}_2)$ are generators so that $i^\ast (u)=\mathrm{Sq}^2l_5$, $i^\ast (v)=\mathrm{Sq}^4l_5$.

\begin{center}
\begin{tabular}{|c|c|c|c|c|c|c|c|c|c|c|c|c|c|c|c|}
\hline
$n$ & 1 & 2 & 3 &4& 5& 6&7 & 8& 9 & 10 \\
\hline
$H^n(B;\mathbb{Z}_2)$ &0& $x$&0& $x^2$& 0&0& $u$& $\mathrm{Sq}^1u$&$ux$, $v$&$\mathrm{Sq}^2\mathrm{Sq}^1u$, x$\mathrm{Sq}^1u$\\
\hline
\end{tabular}
\end{center}

\begin{center}
\begin{tabular}{|c|c|c|c|c|c|c|c|c|c|c|c|c|c|c|c|}
\hline
 11 & 12 & 13  \\
\hline
 $ux^2$, $vx$, $\mathrm{Sq}^4 u$& $x^2\mathrm{Sq}^1 u$, $x\mathrm{Sq}^2\mathrm{Sq}^1 u$, $\mathrm{Sq}^5u$&$vx^2$, $x\mathrm{Sq}^4 u$, $\mathrm{Sq}^6 u$\\
\hline
\end{tabular}
\end{center}

\begin{center}
\begin{tabular}{|c|c|c|c|c|c|c|c|c|c|c|c|c|c|c|c|}
\hline
14& 15 \\
\hline
 $u^2$, $x^2\mathrm{Sq}^2\mathrm{Sq}^1 u$, $x\mathrm{Sq}^5u$, $\mathrm{Sq}^6 \mathrm{Sq}^1 u$& $x^2\mathrm{Sq}^4 u$, $ x\mathrm{Sq}^6 u$, $u\mathrm{Sq}^1u$, $\mathrm{Sq}^7\mathrm{Sq}^1 u$\\
\hline
\end{tabular}
\end{center}

Note that $\mathrm{Sq}^1 (ux)=x\mathrm{Sq}^1u$ and $i^\ast(\mathrm{Sq}^1v)=\mathrm{Sq}^1i^\ast(v)=l_5^2=i^\ast(\mathrm{Sq}^2\mathrm{Sq}^1u)$. It follows that
$\mathrm{Sq}^1:H^9(B,\mathbb{Z}_2)\to H^{10}(B,\mathbb{Z}_2)$ is an  isomorphism. Therefore we may choose in additional  $v$ so that $\mathrm{Sq}^1v=\mathrm{Sq}^2\mathrm{Sq}^1u$.

 The main technical tool in this paper to calculate the $2$-primary part of $\pi_\ast(\mm{M\mathscr{B}})$ is the Adams Spectral sequence (Ass in brief)  with $E_2$-term
$Ext_{\mathscr{A}}^{s,t}(H^\ast(\mm{M\mathscr{B}};\mathbb{Z}_2),\mathbb{Z}_2)$
where $\mathscr{A}$ is the mod $(2)$ Steenrod algebra. By \cite{Giambalvo}, as a module over the Steenrod algebra in dimensions $<16$, $$H^\ast(\mm{MO}\langle 8 \rangle; \mathbb{Z}_2)\cong \mathscr{A}\slash\slash\mathscr{A}_2$$ where $\mathscr{A}_2$ is the Hopf-subalgebra of $\mathscr{A}$ generated by $\mathrm{Sq}^1$, $\mathrm{Sq}^2$ and $\mathrm{Sq}^4$. In our case, we need to determine $Ext_{\mathscr{A}}^{s,t}(H^\ast(\mm{M\mathscr{B}};\mathbb{Z}_2),\mathbb{Z}_2)$ when $t-s<15$. Note that the following modules  are isomorphic in this range
$$H^\ast(\mm{M\mathscr{B}};\mathbb{Z}_2)\cong H^\ast(\mm{M}\eta ;\mathbb{Z}_2)\otimes \mathscr{A}//\mathscr{A}_2$$
$$Ext_{\mathscr{A}}^{s,t}(H^\ast(\mm{M}\eta;\mathbb{Z}_2)\otimes \mathscr{A}//\mathscr{A}_2,\mathbb{Z}_2)\cong Ext_{\mathscr{A}_2}^{s,t}(H^\ast(\mm{M}\eta ;\mathbb{Z}_2),\mathbb{Z}_2).$$
Thus we have to determine the $\mathscr{A}_2$-module structure of $H^\ast(B;\mathbb{Z}_2)$ in dimensions $<16$.

\begin{proposition}
$\mathrm{Sq}^2u=ux,$ $\mathrm{Sq}^2v\ne 0.$
\end{proposition}
\begin{proof}
Consider the Serre spectral sequence of the homotopy fibration $B\to \mathrm{CP}^\infty\stackrel{g}{\rightarrow} \mm{K}(\mathbb{Z},6)$
$$E^{p,q}_2=H^p(\mm{K}(\mb{Z},6);H^q(B;\mb{Z}_2))\Longrightarrow H^{p+q}(\mm{CP}^\infty;\mb{Z}_2).$$
%Since $H^7(\mm{CP}^\infty;\mb{Z}_2)=0$, there exists a nontrivial differential from $E_r^{0,7}$ to $E_r^{r,7-r+1}$ for some $r\ge 2$.
Note that $E_6^{0,7}=E_2^{0,7}=\mb{Z}_2$ generated by $u$. We claim that
$d_6(u)=l_6\cdot x\in E_6^{6,2}$. It suffices to prove that $d_6$ is nontrivial. If not,
since $H^7(\mm{CP}^\infty;\mb{Z}_2)=0$ so $E_\infty ^{0,7}=0$. Then we get that $d_8 :E_8^{0,7}\to E_8^{8,0}$ is nontrivial.  On the other hand,
since  $g^\ast(l_6)=x^3$, we have
$g^\ast(\mathrm{Sq}^2l_6)=\mathrm{Sq} ^2(x^3)=x^4$. Therefore, $E_\infty^{8,0}=E_2^{8,0}$ [\cite{Sw}, 15.29].  A contradiction.
By the Steenrod squares on Serre spectral sequence [\cite{Si},  theorem 2.17 and formulas (2.79)],
$$d_6(\mathrm{Sq}^2u)=\mathrm{Sq}^2d_6(u)=\mathrm{Sq}^2(l_6\cdot x)=l_6\cdot x^2\in E_6^{6,4},$$
so $\mathrm{Sq}^2u\ne 0$. Moreover, it is not hard to check that $\mathrm{Sq}^2u=ux \in H^9(B;\Bbb Z_2)$ since $i^\ast(u)=\mathrm{Sq}^2l_5$ and $i^\ast(v)=\mathrm{Sq}^4l_5$.

Note that $E_6^{0,9}=E_2^{0,9}=\mb{Z}_2\oplus \mb{Z}_2$  generated by $ux$ and $v$. The differential
 $$d_6:E_6^{0,9}\to E_6^{6,4}=\mb{Z}_2$$ satisfies that $d_6(ux)=l_6\cdot x^2$ since $d_6(u)=l_6\cdot x$ and $d_6(x)=0$. Therefore  $E_8^{0,9}=E_7^{0,9}=\mathbb{Z}_2$ generated by either $v$ or $v+ux$.


Next we discuss the cases accordingly.  In the case of $E_8^{0,9}=\mb{Z}_2\langle v\rangle$,
by the same argument as in the proof of the first assertion it follows that the  differential $d_8:E_8^{0,9}\to E_8^{8,2}$ is nontrivial, i.e.,
$d_8(v)=\mathrm{Sq}^2l_6\cdot x$.  Then
$$d_8(\mathrm{Sq}^2v)=\mathrm{Sq}^2d_8(v) =\mathrm{Sq}^2(\mathrm{Sq}^2l_6\cdot x)=\mathrm{Sq}^2l_6\cdot x^2\in E_8^{8,4},$$
so $\mathrm{Sq}^2v\ne 0.$ The other case is similar.
\end{proof}

Next we determine $\mathrm{Sq}^2v$ and $\mathrm{Sq}^4v$ by considering the diagram of pullback fibration \ref{fib2}.
\begin{equation}
\begin{tikzcd}
 \mm{K}(\mathbb{Z},5) \ar[rr] \ar[d,""]&  & \mm{K}(\mathbb{Z},5)  \ar[d,"i"]\\
  \mm{K}(\mathbb{Z},5)\times \mathrm{CP}^2\ar[rr," h"] \ar[d]& &  B\ar[d]\\
{\mathrm{CP}}^2 \ar[rr] &  & {\mathrm{CP}}^\infty
\end{tikzcd}
\label{fib2}
\end{equation}

By comparing the Serre Spectral sequences of the fibration we get


\begin{lemma}\label{injective}
$h^\ast:H^\ast(B;\mathbb{Z}_2)\to H^\ast(\mm{K}(\mathbb{Z},5)\times \mathrm{CP}^2;\mathbb{Z}_2)$ is injective.
\end{lemma}



\begin{proposition}\label{huv}
$h^\ast(u)=\mathrm{Sq}^2l_5+xl_5$, $h^\ast(v)=\mathrm{Sq}^4l_5+x^2l_5$.
\end{proposition}
\begin{proof}
Since $i^\ast(v)=\mathrm{Sq}^4l_5$ and $i^\ast(u)=\mathrm{Sq}^2l_5$, we may assume
 $$h^\ast (v)=\mathrm{Sq}^4l_5+ax^2l_5+bx\mathrm{Sq}^2l_5,$$
$$h^\ast (u)=\mathrm{Sq}^2l_5+cxl_5$$
for some parameters $a, b,c \in \Bbb Z_2$. By 3.3 and  \ref{injective}, we know that $c=1$ since
$0\ne h^\ast(\mm{Sq}^2u)=\mm{Sq}^2h^\ast(u)=\mm{Sq}^2(\mathrm{Sq}^2l_5+cxl_5)=c\mm{Sq}^2(xl_5).$

On the other hand, from the equation
$$l_5^2+bx\mathrm{Sq}^3l_5=h^\ast(\mathrm{Sq}^1v)=h^\ast (\mathrm{Sq}^2\mathrm{Sq}^1u)=l_5^2$$
we get that $b=0$. The proof for  $a=1$ is similar.
\end{proof}

By applying the above proposition it is easy to see

\begin{proposition}
$\mathrm{Sq}^2v=ux^2$, $\mathrm{Sq}^4v=\mathrm{Sq}^6u$.
\end{proposition}

\subsection{The integral cohomology rings of $\mm{K}(\mathbb{Z}, n)$ and $B$}
\quad

Let $l_n$ also denote the generator of $H^n(\mm{K}(\mathbb{Z}, n))=\mb{Z}$. From [\cite{WhiteGW1978}, Theorem 7.8], $H^{n+1}(\mm{K}(\mathbb{Z},n))=0.$
 For the fibration
$$\mm{K}(\mathbb{Z}, n)\to \mm{PK}(\mathbb{Z}, n+1)\to \mm{K}(\mathbb{Z}, n+1),$$
we have the Serre spectral sequence
$$E_2^{p,q}=H^p(\mm{K}(\mathbb{Z}, n+1);H^{q}(\mm{K}(\mathbb{Z},n)))\Longrightarrow H^{p+q}(\mm{PK}(\mathbb{Z}, n+1)) .$$
 As is known well, $H^{i}(\mm{PK}(\mathbb{Z}, n+1))=0$ for $i>0$, $H^\ast(\mm{K}(\mathbb{Z}, 2))=\mathbb{Z}[l_2]$ is a polynomial algebra. Using the above spectral sequence for $n=2,3,4,5$, we have the following results for the integral cohomology rings of $\mm{K}(\mathbb{Z}, n+1)$. The symbols $a_i,b_i,c_i,d_i$ denote the generators of $H^i(\mm{K}(\mb{Z},n+1))$ respectively for $n=2,3,4,5$.  %We consider firstly the above spectral sequence when $n=2$.  Since $H^i(\mm{PK}(\mathbb{Z}, n+1);\mathbb{Z})=0$ for $i>0$, we only determine the nontrivial differentials for $E_r^{p,0}$ in order to get the cohomology ring of $\mm{K}(\mathbb{Z}, n+1)$.

\begin{lemma}\label{Z-3}
The nontrivial integral cohomology groups of $\mm{K}(\mathbb{Z}, 3)$ up to degree $14$ are showed as the following
\end{lemma}
\begin{center}
\begin{tabular}{|c|c|c|c|c|c|c|c|c|c|c|c|c|c|c|c|}% ͨ | ʾǷҪ
\hline  %ڵһк͵ڶ֮ƺ
 i & 3& 6 &8 &9 &10 &11 \\
\hline  % ڱϷƺ
$H^i$ &$\mathbb{Z}\langle l_3 \rangle$ &$\mathbb{Z}_2\langle l_3^2 \rangle$&$\mathbb{Z}_3\langle a_8 \rangle$&$\mathbb{Z}_2\langle l^3_3 \rangle$&$\mathbb{Z}_2\langle a_{10} \rangle$&$\mathbb{Z}_3\langle a_8l_3 \rangle$\\
\hline
\end{tabular}

\begin{tabular}{|c|c|c|c|c|c|c|c|c|c|c|c|c|c|c|c|}% ͨ | ʾǷҪ
\hline  %ڵһк͵ڶ֮ƺ
 i & 12 &13 \\
\hline  % ڱϷƺ
$H^i$ &$\mathbb{Z}_2\langle l^4_3 \rangle\oplus \mathbb{Z}_5\langle a_{12} \rangle$&$\mathbb{Z}_2\langle a_{10}l_3 \rangle$\\
\hline
\end{tabular}
\end{center}



\begin{lemma}\label{Z-4}
The nontrivial integral cohomology groups of $\mm{K}(\mathbb{Z}, 4)$ up to degree $15$ are as  the following
\end{lemma}
\begin{center}
\begin{tabular}{|c|c|c|c|c|c|c|c|c|c|c|c|c|c|c|c|}% ͨ | ʾǷҪ
\hline  %ڵһк͵ڶ֮ƺ
 i & 4& 7 &8 &9 &11 & 12  \\
\hline  % ڱϷƺ
$H^i$ &$\mathbb{Z}\langle l_4 \rangle$ &$\mathbb{Z}_2\langle b_7 \rangle$&$\mathbb{Z}\langle l_4^2 \rangle$&$\mathbb{Z}_3\langle b_9 \rangle$&$\mathbb{Z}_2\langle b_7l_4 \rangle\oplus \mathbb{Z}_2\langle b_{11} \rangle$&$\mathbb{Z}\langle l^3_4 \rangle$\\
\hline
\end{tabular}

\begin{tabular}{|c|c|c|c|c|c|c|c|c|c|c|c|c|c|c|c|}
\hline
 13 &14 &15\\
\hline
$\mathbb{Z}_3\langle b_9l_4 \rangle\oplus \mathbb{Z}_2\langle b_{13,1} \rangle\oplus \mathbb{Z}_2\langle b_{13,2} \rangle\oplus\mathbb{Z}_5\langle b_{13,3} \rangle$ &$\mathbb{Z}_2\langle b_{14} \rangle$ &$\mathbb{Z}_2\langle b_{7} l_4^2\rangle\oplus \mathbb{Z}_2\langle b_{11} l_4\rangle$\\
\hline
\end{tabular}
\end{center}

\begin{lemma}\label{Z-5}
The nontrivial integral cohomology groups of $\mm{K}(\mathbb{Z}, 5)$ up to degree $15$ are as  the following
\end{lemma}
\begin{center}
\begin{tabular}{|c|c|c|c|c|c|c|c|c|c|c|c|c|c|c|c|}% ͨ | ʾǷҪ
\hline  %ڵһк͵ڶ֮ƺ
 i & 5& 8 &10 &12 &13   \\
\hline  % ڱϷƺ
$H^i$ &$\mathbb{Z}\langle l_5 \rangle$ &$\mathbb{Z}_2\langle c_8 \rangle$&$\mathbb{Z}_3\langle c_{10} \rangle \oplus \mathbb{Z}_2\langle l_5^2 \rangle$&$\mathbb{Z}_2\langle c_{12} \rangle$&$\mathbb{Z}_2\langle c_8l_5 \rangle$\\
\hline
\end{tabular}

\begin{tabular}{|c|c|c|c|c|c|c|c|c|c|c|c|c|c|c|c|}% ͨ | ʾǷҪ
\hline  %ڵһк͵ڶ֮ƺ
 i   &14 &15\\
\hline  % ڱϷƺ
$H^i$ &$\mathbb{Z}_2\langle c_{14,1} \rangle \oplus \mathbb{Z}_3\langle c_{14,2} \rangle \oplus \mathbb{Z}_5\langle c_{14,3}\rangle$ &$\mathbb{Z}_2\langle l_{5}^3 \rangle\oplus \mathbb{Z}_3\langle c_{10} l_5\rangle \oplus \mathbb{Z}_2\langle c_{15}\rangle$\\
\hline
\end{tabular}
\end{center}

\begin{lemma}\label{Z-6}
The nontrivial integral cohomology groups of $\mm{K}(\mathbb{Z}, 6)$ up to degree $16$ are showed as follow.
\end{lemma}
\begin{center}
\begin{tabular}{|c|c|c|c|c|c|c|c|c|c|c|c|c|c|c|c|}% ͨ | ʾǷҪ
\hline  %ڵһк͵ڶ֮ƺ
 i & 6& 9 &11 &12 &13   \\
\hline  % ڱϷƺ
$H^i$ &$\mathbb{Z}\langle l_6 \rangle$ &$\mathbb{Z}_2\langle d_9 \rangle$&$\mathbb{Z}_3\langle d_{11,1} \rangle \oplus \mathbb{Z}_2\langle d_{11,2} \rangle$&$\mathbb{Z}\langle l^2_{6} \rangle$&$\mathbb{Z}_2\langle d_{13} \rangle$\\
\hline
\end{tabular}
\begin{tabular}{|c|c|c|c|c|c|c|c|c|c|c|c|c|c|c|c|}% ͨ | ʾǷҪ
\hline  %ڵһк͵ڶ֮ƺ
 i & 15& 16   \\
\hline  % ڱϷƺ
$H^i$ &$\mathbb{Z}_3\langle d_{15,1} \rangle \oplus \mathbb{Z}_2\langle d_{15,2} \rangle \oplus \mathbb{Z}_5\langle d_{15,3} \rangle\oplus \mathbb{Z}_2\langle d_{9}l_6 \rangle$&$\mathbb{Z}_2\langle d_{16} \rangle$\\
\hline
\end{tabular}
\end{center}

Most details for the computations of above lemmas are showed in Appendix \ref{SSSforintegral}.
Next we calculate $H^\ast(B)$ by the SSS
$$E_2^{p,q}=H^p(\mm{CP}^\infty
;H^{q}(\mm{K}(\mathbb{Z},5)))\Longrightarrow H^{p+q}(B). $$



\begin{proposition}\label{Z-B}
If $s=\pm 1 \pmod{ 6}$, the nontrivial integral cohomology groups of $B$ up to degree $15$ are showed as follow
\end{proposition}
\begin{center}
\begin{tabular}{|c|c|c|c|c|c|c|c|c|c|c|c|c|c|c|c|}
\hline
 i & 2& 4 &6 &8 & 10  \\
\hline
$H^i$ &$\mathbb{Z}\langle x \rangle$ &$\mathbb{Z}\langle x^2 \rangle$& $\mathbb{Z}_s\langle x^3 \rangle $& $\mathbb{Z}_2 \oplus \mathbb{Z}_s\langle x^4\rangle$ &$\mathbb{Z}_s\langle x^5\rangle \oplus \mathbb{Z}_2^2 \oplus \mathbb{Z}_3$\\
\hline
\end{tabular}
\begin{tabular}{|c|c|c|c|c|c|c|c|c|c|c|c|c|c|c|c|}
\hline
 i & 12& 14 &15   \\
\hline
$H^i$ &$\mathbb{Z}_s\langle x^6\rangle \oplus \mathbb{Z}^3_2 \oplus \mathbb{Z}_3$ &$\mathbb{Z}_s\langle x^7\rangle  \oplus  \mathbb{Z}^3_2 \oplus \mathbb{Z}^2_3 \oplus \mathbb{Z}_5$& $\mathbb{Z}_2$\\
\hline
\end{tabular}
\end{center}

We refer to Appendix \ref{SSSforintegral} for the details.



\subsection{$H^\ast(B;\mathbb{Z}_3)$ and its module over the Steenrod
algebra}
\quad

By the same method as in the calculation of $H^\ast(B;\mathbb{Z}_2)$, we have the following lemmas for the SSS with $\mathbb{Z}_3$-coefficient
$$ E_2^{p,q}(B)=H^p(\mm{CP}^\infty;H^q(\mm{K}(\mb{Z},5);\mb{Z}_3))\Longrightarrow H^{p+q}(B;\mb{Z}_3) .$$

\begin{lemma}\label{vanish1}
$E_6^{p,q}(B)=E_2^{p,q}(B)$ for $p\ge 0$; $E_7^{p,q}(B)=0$ for $p\ge 5.$
\end{lemma}

\begin{proposition}\label{convergence1}
$E_\infty^{p,q}(B)=E_7^{p,q}(B)$ for $p\ge 0$.
\end{proposition}

\begin{lemma}\label{injective1}
$h^\ast:H^\ast(B;\mathbb{Z}_3)\to H^\ast(\mm{K}(\mathbb{Z},5)\times \mathrm{CP}^2;\mathbb{Z}_3)$ is injective.
\end{lemma}

Since that the cohomology ring with $\mathbb{Z}_3$-coefficient is relatively sparse, we consider the module structure of $H^\ast(B;\mathbb{Z}_3)$ over the mod $(3)$ Steenrod algebra $\mathcal{A}$ directly by Lemma  \ref{Z-5} and Proposition \ref{Z-B}.

By Lemma \ref{Z-4}, $l_4^3\in H^{12}(\mm{K}(\mathbb{Z},4);\mathbb{Z}_3)$. Let $\mathcal{P}^i$ be the reduced {\it i}-th power. Note that $\mathcal{P}^2l_4=l_4^3$. So $\mathcal{P}^1l_4\ne 0$ by $\mathcal{P}^1\mathcal{P}^1=-\mathcal{P}^2.$ By the homomorphism $\sigma$
$$H^{i}(\mm{K}(\mathbb{Z},5);\mathbb{Z}_3)\to H^{i-1}(\Omega \mm{K}(\mathbb{Z},5);\mathbb{Z}_3)\cong H^{i-1}( \mm{K}(\mathbb{Z},4);\mathbb{Z}_3)$$
which is commutative with Steenrod operation and satisfies $\sigma (l_5)=l_{4}$, $\mathcal{P}^1l_5\ne 0$ and $\mathcal{P}^2l_5\ne 0$. Then $H^\ast(\mm{K}(\mathbb{Z},5);\mathbb{Z}_3)$ up to $14$ dimension follows by Lemma \ref{Z-5}.
\begin{center}
\begin{tabular}{|c|c|c|c|c|c|c|c|c|c|c|c|c|c|c|c|}
\hline
 i & 5& 9 &10 &13 &14   \\
\hline
$H^i(\mm{K}(\mathbb{Z},5);\mathbb{Z}_3)$ &$ l_5 $ &$\mathcal{P}^1l_5$&$\beta\mathcal{P}^1l_5$&$\mathcal{P}^2l_5$&$\beta\mathcal{P}^2l_5$, $l_5\mathcal{P}^1l_5$\\
\hline
\end{tabular}
\end{center}
 where $\beta$ is the Bockstein operation associated to the short exact sequence $0\to \mathbb{Z}_3\to \mathbb{Z}_{3^2}\to\mathbb{Z}_3\to 0$ unless otherwise specified.
By the proposition \ref{Z-B}, $H^\ast(B;\mathbb{Z}_3)$ up to $14$ dimension is listed as  in following table where $i^\ast(z_9)=\mathcal{P}^1l_5.$

\begin{center}
\begin{tabular}{|c|c|c|c|c|c|c|c|c|c|c|c|c|c|c|c|}
\hline
 i   &2 &4&9&10 & 11& 12 & 13& 14\\
\hline
$H^i(B;\mathbb{Z}_3)$ &$x$ &$x^2$& $ z_9 $ &$\beta z_9 $& $ xz_9$& $x\beta z_9$ &$z_{13}$, $x^2z_{9}$& $\beta z_{13} $, $x^2\beta z_9$\\
\hline
\end{tabular}
\end{center}

By Lemma \ref{injective1},  we may write $h^\ast(z_9)=\mathcal{P}^1l_5+mx^2l_5$ where $m\in \mathbb{Z}_3$. Since
$$h^\ast(\mathcal{P}^1z_9)=\mathcal{P}^1(\mathcal{P}^1l_5+mx^2l_5)=-\mathcal{P}^2l_5+mx^2\mathcal{P}^1l_5\ne 0,$$
$\mathcal{P}^1z_9\ne 0$ and $\mathcal{P}^1z_9\ne z_9x^2$. Let us denote $\mathcal{P}^1z_9$ by $z_{13}$.

From the equations
$$h^\ast(\mathcal{P}^1\beta z_9)=\mathcal{P}^1(\beta\mathcal{P}^1l_5)=\beta \mathcal{P}^2l_5,$$
$$h^\ast(\beta z_{13})=-\beta \mathcal{P}^2l_5+mx^2\beta \mathcal{P}^1l_5,$$
$$h^\ast(mx^2\beta z_9)=mx^2\beta \mathcal{P}^1l_5,$$
it follows that
\begin{lemma}\label{3module}
 $\mathcal{P}^1\beta z_9=-\beta z_{13}+mx^2z_9$.
\end{lemma}

\section{Some basic homotopy propositions for Bazaikin space}\label{oriented}

 It is easy to see that  $w _2(\mathcal{B}_{\bold{q}})\ne 0$ and $w_4(\mathcal{B}_{\bold{q}})\ne 0$, thus the Wu class $v_2(\mathcal{B}_{\bold{q}})\ne 0$, $v_4(\mathcal{B}_{\bold{q}})= 0$.  Note that   %From \cite{MilnorStasheff}
 \begin{equation}
 \mathrm{Sq}^2(b_{11})=b_{13}, \quad \mathrm{Sq}^2(b_9)=0, \quad \mathrm{Sq}^4(b_9)=0 \label{SqBq}	
 \end{equation}
where $b_n$ is the generator of $H^n(\mathcal{B}_{\bold{q}};\mb{Z}_2)$.

Recall the $6$-equivalence $f:\mathcal{B}_{\bold{q}}\to B$ so that $f^\ast(x)=\zeta \in H^2(\mathcal{B}_{\bold{q}})$ where $x\in H^2({B})$ is a generator.
There is a principal $S^1$-bundle $\theta$ over $B$ whose Euler class is $x\in H^2(B)$,
\[
\xymatrix{
 S^1 \ar[rr]^-{} \ar[d]^-{}&  & S^1  \ar[d]^-{}\\
  P\ar[rr]^-{\bar{f}} \ar[d]^-{\bar{p}}& & \mm{K}(\mathbb{Z},5)\ar[d]^-{i}\\
\mathcal{B}_{\bold{q}}\ar[rr]^-{f} &  & B
}
\]
then the Euler class of the $S^1$ bundle $f^\ast \theta$ over $\mathcal{B}_{\bold{q}}$ is $\zeta\in H^2(\mathcal{B}_{\bold{q}})$. Recall that $\mathcal{B}_{\bold{q}}=S^1\backslash\mm{SU(6)}/\mm{Sp}(3)$, $\mm{SU(6)}/\mm{Sp}(3)$ has a free $S^1\cong S^1/\{\mb{Z}_2\}$ action. Hence $P\cong \mm{SU}(6)/\mm{Sp}(3).$
By the Gysin sequence, we have two short exact sequences
$$0\to H^5(P)=\mathbb{Z} \to H^4(\mathcal{B}_{\bold{q}})=\mathbb{Z}\to H^6(\mathcal{B}_{\bold{q}})=\mathbb{Z}_s\to 0$$
$$0\to \mathbb{Z}=H^9(\mathcal{B}_{\bold{q}})\stackrel{\times s}{\rightarrow}H^9(P)\to H^8(\mathcal{B}_{\bold{q}})=\mathbb{Z}_s\to 0,$$
thus $\bar{p}^\ast:H^9(\mathcal{B}_{\bold{q}}; \mathbb{Z}_p)\to H^9(P; \mathbb{Z}_p)$ is an isomorphism for $p=2$ or $3$. From the naturality for Gysin sequence, $\bar{f}$ induces the isomorphism
$$\bar{f}^\ast: H^5(\mm{K}(\mathbb{Z},5))\to H^5(P).$$

Besides one can describe $P$ as a CW-complex
$$P = S^5 \cup e^9 \cup \cdots ,$$
with cells in dimension $5,$ $9$, $\cdots$. The attaching map $\partial(e^9)
 = S^8 \to S^5$
is $\alpha \in \pi_8(S^5) = \mathbb{Z}_{24}$ \cite{Toda} and hence $\pi_8(P) = \mathbb{Z}_{24}/\langle \alpha \rangle$. From \cite{FL}, $\pi_8(P) = 0$, thus $\alpha$ is a generator of $\pi_8(S^5)=\pi^s_3(S^0)$.

\begin{lemma}\label{4module}
$\mathrm{Sq}^4:H^5( P;\mathbb{Z}_2)\to H^9( P;\mathbb{Z}_2)$ and
$$\mathcal{P}^1:H^5( P;\mathbb{Z}_3)\to H^9( P;\mathbb{Z}_3)$$ are isomorphisms.
\end{lemma}

\begin{proof}
Note that the CW complex $X={S}^4\cup_a e^8$ is the $8$-skeleton of $\mm{HP}^\infty$ if $a=1\in \mb{Z}\subset \pi_7(S^4)$. It is clear that the $9$-skeleton of $P$ has the homotopy type of $\Sigma X$ for some $a\in \mb{Z}\subset\pi_7(S^4)$ so that $a=\alpha \mod 24$. For $a,b\in  \pi_7(S^{4})$, let us compare the mapping cone ${S}^4\cup_{a+b} e^8$ with ${S}^4\cup_{a\vee b} e^8$.
The desired results follow by the $\mm{Sq^4}$ and $\mathcal{P}^1$ actions on $H^4(\mm{HP}^\infty;\mb{Z}_2)$ and $H^4(\mm{HP}^\infty;\mb{Z}_3)$ respectively.
\end{proof}

\begin{proposition}\label{image}
$f_{\ast}([\mathcal{B}_{\bold{q}}]_2)=(vx^2)_\ast$ and $f_{\ast}([\mathcal{B}_{\bold{q}}]_3)=(z_9x^2)_\ast$ where $(vx^2)_\ast$, $(z_9x^2)_\ast$ are the homology  duals of $vx^2\in H^{13}(B;\mathbb{Z}_2)$, $z_9x^2\in H^{13}(B;\mathbb{Z}_3)$, $[\mathcal{B}_{\bold{q}}]_2$ and $[\mathcal{B}_{\bold{q}}]_3$ are the mod $2$ and $3$ reduction of the orientable class. %of Bazaikin space.
\end{proposition}
\begin{proof}
By Poincare duality
$\langle b_2^2\cup b_9, [\mathcal{B}_{\bold{q}}]_2\rangle=1.$
By Lemma \ref{4module},
$$\bar{f}^\ast(\mathrm{Sq}^4l_5)=\mathrm{Sq}^4\bar{f}^\ast(l_5)=x_9\in H^9(P;\mathbb{Z}_2).$$
From the computation about $H^\ast(B;\mathbb{Z}_2)$, $i^\ast(v)=\mathrm{Sq}^4l_5$, we get
$$\bar{f}^\ast i^\ast (v)=\bar p^\ast f^\ast (v)=x_9$$
$$f^\ast (v)=b_9\in H^9(\mathcal{B}_{\bold{q}};\mathbb{Z}_2).$$
Hence
$\langle x^2\cup v,f_{\ast} ( [\mathcal{B}_{\bold{q}}]_2)\rangle=\langle f^\ast (x^2\cup v), [\mathcal{B}_{\bold{q}}]_2\rangle=\langle b_2^2\cup b_9, [\mathcal{B}_{\bold{q}}]_2\rangle=1$,
$$f_{\ast}([\mathcal{B}_{\bold{q}}]_2)=(vx^2)_\ast.$$

The proof for $[\mathcal{B}_{\bold{q}}]_3$ is analogous.
\end{proof}

\section{Atiyah-Hirzebruch spectral sequence}\label{secAHSS}

 For connective spectral $E$ and $X$, i.e., $\pi_n(E)=\pi_n(X)=0$ for $n<0$, by \cite{Adams1974} we have the Atiyah-Hirzebruch spectral sequence (AHSS)  for $E_\ast(X)$
 $$E_2^{p,q}\cong H_p(X;\pi_q(E))\Longrightarrow E_\ast(X)\cong \pi_\ast(E\wedge X).$$
   By the construction of the AHSS, the spectral sequence has the naturality for both $E$ and $X$. Now we give three lemmas used in the sequel.
\begin{lemma} \cite{Peter1993} \label{d2}
Assume that the unit $\iota: S^0\to E$ induces the isomorphisms $\pi_i(S^0)\cong \pi_i(E)$ for $ i\le 2$. Then
the differential $d_2:E_2^{p+2,1} \to E_2^{p,2}$ is the dual of
$$\mathrm{Sq}^2: H^p(X; \mathbb{Z}_2)\to H^{p+2}(X ; \mathbb{Z}_2);$$
\item $d_2:E_2^{p+2,0} \to E_2^{p,1}$ is the reduction mod $2$ composed with the dual of $\mathrm{Sq}^2$.
\end{lemma}
Here we give a terminologie for the AHSS.
\begin{definition}\label{provided}
 	Suppose that an AHSS with $E_2^{p}$ converges to a filtered module $N$, $x \in E_2^{p}$ and $z\in F^pN$ which is the $p$-dimensional filtration.
We call that $z$ is {\it provided} by $x$  if $x$ survives to $ E_\infty^{p}$ corresponding to $z\in F^p N/F^{p-1} N$  under the identification $E^p_\infty \cong F^pN/F^{p-1}N$.
 \end{definition}

Moreover, for the Adams spectral sequence with identification $E^s_\infty \cong F^sN/F^{s+1}N$, the definition \ref{provided} is also used.

 For a connective spectrum $E$ with $\pi_0(E)=\mathbb{Z}$, let $\chi:E\to H$ be the $0$-th Postnikov tower where $H$ is the Eilenberg spectrum with integral coefficient. %Remark that $H(\mb{Z}_p)$ denotes the Eilenberg spectrum with mod $p$ coefficient for $p\in \mb{Z}$.
\begin{lemma}\label{chi}
	For connective spectral $E$ and $X$, $z\in E_n(X)$ is provided by
	$x\in E_2^{n,0}=H_n(X;\pi_0(E))$ in the $\mm{AHSS}$ if and only if $\chi_\ast(z)=x $ $$ \chi_\ast : E_n(X)\to H_n(X;\mathbb{Z}).$$
\end{lemma}
Lemma \ref{chi} is obvious by the naturality for the map $\chi:E\to H$.

\begin{lemma}\label{ahsspro}
	Let $f_r:E_r^{p,q} \to \bar{E}_r^{p,q}$ be the morphism between two $\mm{AHSS}$ with $E_2^{p,q}=\bar{E}_2^{p,q}=0$ for all $p<0$. If  $E_2^{p,q}\cong\bar{E}_2^{p,q}$ for $0\le p<n$, then
	\item [(5.4.1)] $E_l^{n-k,q}\cong \bar{E}_l^{n-k,q}$ for any $2\le  l\le  k\le n$,
	
	$E_l^{n-a,q}\to \bar{E}_l^{n-a,q}$ is an epimorphism for $1\le a<l$;
	
	\item [(5.4.2)] For an $x\in  {E}_{2}^{n,m}$ with $f_2(x)=\bar{x}\ne 0$ for which
$\bar{x}\in \bar{E}_{2}^{n,m}$ survives in $\bar{E}_{n+1}^{n,m}$, then $x$ survives to ${E}_{n+1}^{n,m}$ at least.
\end{lemma}
\begin{proof}
 Proof of (5.4.1): It is obvious for $l=2$ and for any $1\le k\le n$.
 We prove by induction. Assume that (5.4.1) is true when $l=l_0>2$.  Consider the chain map
 $$f_{ls}:\{E_{l}^{n-a-sl,\ast},d_{ls}\}\to \{\bar{E}_l^{n-a-sl,\ast},\bar{d}_{ls}\} $$
 \[
\xymatrix@C=.8cm{
 E_l^{n-a,\ast}\ar[d]^-{f_{l0}}  \ar[r]^-{d_{l0}}& E_l^{n-a-l,\ast}\ar[d]^-{f_{l1}}  \ar[r]^{{d}_{l1}} & E_l^{n-a-2l,\ast}\ar[d]^-{}  \ar[r]^-{}& \cdots \ar[r]&0\\
\bar{E}_l^{n-a,\ast}  \ar[r]^-{\bar{d}_{l0}}& \bar{E}_l^{n-a-l,\ast}  \ar[r]^-{\bar{d}_{l1}} & \bar{E}_l^{n-a-2l,\ast}  \ar[r]^-{}& \cdots \ar[r]&0
}
\]
for $1\le a\le l$, $f_{ls}$ is an isomorphism for $s>0$, an epimorphism for $s=0$. Thus
$f_{ls}|_{ker}:ker(d_{ls})\to ker(\bar{d}_{ls})$
 is an epimorphism for $s\ge 0$,
 $$f_{(l+1)s}:E_{l+1}^{n-a-sl,\ast}\to \bar{E}_{l+1}^{n-a-sl,\ast}$$
 is an epimorphism for $s\ge 0$ and $1\le a\le l.$

 Consider the kernel of $f_{(l+1)s}$ for $s>0$ and $1\le a\le l$. %Let $f_{(l+1)s}(\{x\})=0$ where $x\in E_{l}^{n-a-sl,\ast}$, which means that $f_{ls}(x)=\bar{d}_{l(s-1)}(\bar{y})$ where $\bar{y}\in \bar{E}_{l}^{n-a-(s-1)l,\ast}$. Thus there is $y\in E_{l}^{n-a-(s-1)l,\ast}$ such that $f_{l(s-1)}(y)=\bar{y}$ and $d_{l(s-1)}(y)=x$. Hence $\{x\}=0$,
 By the chasing diagram, we have $ker(f_{(l+1)s})=0$ for $s>0$. Hence (5.4.1) is true for $l=l_0+1$.
		
Proof of (5.4.2): If $x$ survives to $E_r^{n,m}$ for some $2\le r <n+1$
then
$$f_r\circ d_r(x)=\bar{d}_r\circ f_r(x)=\bar{d}_r(\bar{x})=0$$
thus $d_r(x)=0$ by (5.4.1). If $d_r(y)=x$, then $\bar{d}_r(f_r(y))=\bar{x}\in \bar{E}_r^{n,m}$ which is a contradiction. Therefore $x$ survives to $E_{n+1}^{n,m}$.	
\end{proof}

In the next subsection we will calculate a certain part of $\Omega_{13}^{\mm{O}\langle 8 \rangle}(\eta)\cong \pi_{13}(\mm{M\mathscr{B}})$, such as $\mb{Z}_s\oplus \mb{Z}_s\subset \pi_{13}(\mm{M\mathscr{B}})$ whose characterizing invariants will be discussed in Section \ref{Zs}.

\subsection{The $p$-primary part of $\Omega_{13}^{\mm{O}\langle 8 \rangle}(\eta)$}\label{pprimary}
\quad
$$\pi_\ast(\mm{M\mathscr{B}})=\pi_\ast(\mm{MO}\langle 8 \rangle\wedge \mm{M}\eta)\cong \mm{MO}\langle 8 \rangle_\ast (\mm{M}\eta)$$
Since $\mm{M}\eta$ is a connective spectrum, we have the spectral sequence
$$E_2^{p,q}=H_p(\mm{M}\eta;\pi_q(\mm{MO}\langle 8 \rangle))\Longrightarrow \pi_\ast(\mm{M\mathscr{B}}).$$
 Recall that in dimension $i \le 14$, $\pi_i(\mm{MO} \langle  8 \rangle )$ is as follows \cite{Giambalvo,HoRa1995}.

\begin{center}
\begin{tabular}{|c|c|c|c|c|c|c|c|c|c|c|c|c|c|c|c|}
\hline
$i$ & 0&1& 2& 3&4& 5& 6&7&8&9\\
\hline
$\pi_i(\mm{MO} \langle  8\rangle )$ & $\mathbb{Z}$ & $\mathbb{Z}_2$ & $\mathbb{Z}_2$ & $\mathbb{Z}_{24}$& 0& 0& $\mathbb{Z}_2$ & 0& $ \mathbb{Z}\oplus \mathbb{Z}_2$ & $\mathbb{Z}_2\oplus \mathbb{Z}_2 $\\
\hline

\end{tabular}
\end{center}
\begin{center}
\begin{tabular}{|c|c|c|c|c|c|c|c|c|c|c|c|c|c|c|c|}
\hline
$i$ &10&11&12&13&14\\
\hline
$\pi_i(\mm{MO} \langle  8 \rangle )$  & $\mathbb{Z}_6 $ & 0& $\mathbb{Z}$ & $\mathbb{Z}_3$ & $\mathbb{Z}_2$\\
\hline

\end{tabular}
\end{center}

By Proposition \ref{Z-B} and the Thom isomorphism, all of the nontrivial $E_2$-terms of the AHSS  for $p+q=12,$ $13$ and $14$ are as follows
\begin{center}
\begin{tabular}{|c|c|c|c|c|c|c|c|c|c|c|c|c|c|c|c|}
\hline
$(p+q=13)$ $p$ & 0&4& 5& 7&10 &11 &12 &13\\
\hline
 $E_2^{p,q}$ & $\mathbb{Z}_3$ & $\mathbb{Z}_2^2$ & $\mathbb{Z}_{s}$& $\mathbb{Z}_2$ & $\mathbb{Z}_2^2\oplus \mathbb{Z}_3$& $\mathbb{Z}_2^3$ & $\mathbb{Z}_2^3$& $\mathbb{Z}_2^3\oplus \mathbb{Z}_3^2\oplus \mathbb{Z}_5\oplus \mathbb{Z}_s$ \\
\hline
\end{tabular}
\end{center}

\begin{center}
\begin{tabular}{|c|c|c|c|c|c|c|c|c|c|c|c|c|c|c|c|}
\hline
$(p+q=14)$ $p\ge 7$ & 8&11& 12&13&14 \\
\hline
 $E_2^{p,q}$ & $\mathbb{Z}_2$ & $\mathbb{Z}_2^3\oplus \mathbb{Z}_3$ & $\mathbb{Z}_{2}^3$& $\mathbb{Z}_2^3$ & $\mathbb{Z}_2$ \\
\hline
\end{tabular}
\end{center}

\begin{center}
\begin{tabular}{|c|c|c|c|c|c|c|c|c|c|c|c|c|c|c|c|}
\hline
$(p+q=12)$ $p$ & 0&2&4&9&10&11 \\
\hline
 $E_2^{p,q}$ & $\mathbb{Z}$ & $\mathbb{Z}_6$ & $\mathbb{Z}_{2}\oplus \mathbb{Z}$& $\mathbb{Z}_2^2\oplus \mathbb{Z}_3$ & $\mathbb{Z}_2^2$ &$\mathbb{Z}_2^3$\\
\hline
\end{tabular}
\end{center}

From above tables, $\pi_{13}(\mm{M\mathscr{B}})$ is a finite group and the elements of order $n=\pm 1\mod 6>0$ are provided by $E_2^{13,0}$ and $E_2^{5,8}$.
	Consider the differentials of the AHSS directly,
 $E_\infty^{5,8}=E_2^{5,8}=\mathbb{Z}_s$, $\mathbb{Z}_s\oplus \mathbb{Z}_5\subset E_2^{13,0}$ survives to $E_\infty^{13,0}$.
 Then we discuss the extension between $\mathbb{Z}_s\oplus \mathbb{Z}_5\subset E_\infty^{13,0}$ and $ \mathbb{Z}_s=E_\infty^{5,8} $  by the Hurewicz homomorphism. First we have

\begin{lemma}
	The Hurewicz homomorphism
$$\iota:\mathbb{Z}\subset \pi_8(\mm{MO}\langle 8\rangle)\to H_8(\mm{MO}\langle 8\rangle;\mathbb{Z})$$
is the $\times 240$ homomorphism.
\end{lemma}
\begin{proof}
	Note that $H_8(\mm{MO}\langle 8\rangle)= \mathbb{Z}$ and $H_i(\mm{MO}\langle 8\rangle)=0$ for $0<i\le 7$.
	Consider the AHSS
	$$E_2^{p,q}=H_p(\mm{MO}\langle 8\rangle;\pi_q(S^0))\Longrightarrow \pi_{p+q}(\mm{MO}\langle 8\rangle),$$
 $E_8^{8,0}= E_2^{8,0}$ and $E_8^{0,7}= E_2^{0,7}$. Since	$\pi_7(\mm{MO}\langle 8\rangle)=0$ , the differential
$$d_8:E_8^{8,0}=\mb{Z}\to E_8^{0,7}=\mb{Z}_{240}$$
is an epimorphism. The unit $\iota:S^0\to {H}(\mb{Z})$ is the 0th Postnikov tower,
 thus  the desired result follows by Lemma \ref{chi}.
 %the Hurewicz homomorphism
%$$\iota:\mathbb{Z}\subset \pi_8(\mm{MO}\langle 8\rangle)\to H_8(\mm{MO}\langle 8\rangle;\mathbb{Z})$$
%is the $\times 240$ homomorphism.
\end{proof}

\begin{proposition}\label{ZsHure}
In the filtartion of the AHSS for $\pi_{13}(\mm{M\mathscr{B}})$ above, the extension between
$ \mathbb{Z}_s=E_\infty^{5,8} $ and $ \mathbb{Z}_s\oplus\mathbb{Z}_5\subset E_\infty^{13,0}$
is trivial if $s\ne 5$. Moreover, if $s\ne 0\mod 5$,
 	 $\mathbb{Z}_s\oplus \mathbb{Z}_s\oplus \mathbb{Z}_5\subset \pi_{13}(\mm{M\mathscr{B}})$
 has an isomorphic image in $H_{13}(\mm{M\mathscr{B}})$ under the Hurewicz homomorphism.
 \end{proposition}
 \begin{proof}
By the Thom isomorphism and the $K\ddot{u}nneth$ formula, the generators for the direct summands
$$\mathbb{Z}_s\oplus \mathbb{Z}_s \oplus \mathbb{Z}_5\subset H_{13}(\mm{BO}\langle 8 \rangle\times B)\cong H_{13}(\mm{MO}\langle 8 \rangle\wedge \mm{M}\eta)=H_{13}(\mm{M\mathscr{B}})$$
are respectively $m_8\otimes y_5$ where
 $y_5\in \mathbb{Z}_s\subset  H_{5}( \mm{M}\eta)$,
 $m_8\in H_{8}(\mm{MO}\langle 8 \rangle);$
$$y_{13,s}\in \mathbb{Z}_s\subset  H_{13}( \mm{M}\eta);\quad
 y_{13,5}\in \mathbb{Z}_5\subset  H_{13}( \mm{M}\eta).$$
In the AHSS for $H_\ast(\mm{MO}\langle 8 \rangle\wedge \mm{M}\eta)$ with
$E_2^{p,q}(H)=H_p(\mm{M}\eta;H_q(\mm{MO}\langle 8 \rangle))$,
$y_{13,s}$ and $y_{13,5}$ are provided by $E_2^{13,0}(H)$, $m_8\otimes y_5$ is provided by  $E_2^{5,8}(H)$.
By the morphism of spectral sequences
\[
\xymatrix@C=.2cm{
  {E}_2^{p,q}={H}_p(\mm{M}\eta;\pi_q(\mm{MO}\langle 8\rangle))\ar@{=>}[rr]^{}\ar[d]&  & \pi_{p+q}(\mm{MO}\langle 8\rangle\wedge \mm{M}\eta)\ar[d]^-{\iota_\ast} \\
{E}_2^{p,q}(H)={H}_p(\mm{M}\eta;H_q(\mm{MO}\langle 8\rangle))\ar@{=>}[rr]^-{}& & H_{p+q} ( \mm{MO}\langle 8\rangle\wedge \mm{M}\eta)
}
\]
the images of the generators of $\mb{Z}_s=E_\infty^{5,8}$ and $\mathbb{Z}_s \oplus \mathbb{Z}_5\subset E_\infty^{13,0}$ can be regarded as
 $240m_8\otimes y_5, y_{13,s}, y_{13,5}$ respectively under $\iota_\ast$.

If $s\ne 5$, then $240m_8\otimes y_5\ne 0$ which implies that the extension between $\mb{Z}_s=E_\infty^{5,8}$ and $\mathbb{Z}_s \oplus \mathbb{Z}_5\subset E_\infty^{13,0}$ is trivial.

 If $(s,5)=1$,
 then $(s,240)=1$, $ks+240l=1$ for some $k,l\in \mathbb{Z}$,  $$0\ne 240lm_8\otimes y_5={m}_8\otimes y_5$$
where $(m,n)$ denotes the maximum common divisor of $m$ and $n$.  So the desired result follows.
\end{proof}
The first conclusion in Theorem \ref{bordismgroup} is contained in Proposition \ref{ZsHure}.

 \subsection{The bordism invariants of characterizing $\mathbb{Z}_s\oplus \mathbb{Z}_s \subset \Omega_{13}^{\mm{O}\langle 8 \rangle}(\eta)$}\label{Zs}
 \quad

If $(s,5)=1$, from the classical Thom construction and Proposition \ref{ZsHure}, the summand in $\mathbb{Z}_s\oplus \mathbb{Z}_s\oplus \mathbb{Z}_5\subset \Omega_{13}^{\mm{O}\langle 8 \rangle}(\eta)$ of the bordism class of a manifold $M$ with $\mathscr{B}$-structure can be detected by the image of the orientation class $[M]$  under the map
$\bar{\mathscr{N}}_M:M\stackrel{\triangle}{\rightarrow}M\times M\stackrel{(\nu,f)}{\rightarrow} \mathscr{B}=\mm{BO}\langle 8\rangle\times B .$

At first, we discuss the linking form. For the homological Bazaikin space $M$ equipped with the same cohomology ring as Bazaikin space,  the Bockstein homomorphism
$$\beta : H^5(M; \mathbb{Z}_s) =\mathbb{Z}_s\to H^6(M; \mathbb{Z}) = \mathbb{Z}_s$$
associated with the short exact sequence $0\to \mathbb{Z}\to \mathbb{Z}\to\mathbb{Z}_s\to 0$, is an isomorphism since $H^5(M) = 0$. The linking form is then given by
$$L: H^6(M) \times H^8(M)\to \mathbb{Z}_s , \quad L(a, b) = (\beta^{-1}(a) \cup r_s(b))([M]_s),$$
for a given choice of orientation class $[M]_s$ for $\mathbb{Z}_s$-coefficient where $r_s$ is the mod $s$ reduction, $[M]_s=r_s([M])$. $L$ is determined up to sign by $\ell k(M)=L(\zeta^3, \zeta^4) \in\mathbb{Z}_s$, where $\zeta$ is the generator of $H^2(M)$.

By Proposition \ref{Z-B},
$$\beta : H^5(B; \mathbb{Z}_s) =\mathbb{Z}_s\to H^6(B; \mathbb{Z}) = \mathbb{Z}_s$$
 is an isomorphism. Note that $f:M\to B$ is a $6$-equivalence, $f^\ast(x)=\zeta$ where $x$ is the generator of $H^2(B)$.
$$f^\ast(\beta^{-1}(x^3) \cup r_s(x^4))=\beta^{-1}(\zeta^3) \cup r_s(\zeta^4)$$
Note that another Bockstein homomorphism
$$\beta_s : \mathbb{Z}_s\subset H^i(B; \mathbb{Z}_s) \to \mathbb{Z}_s\subset H^{i+1}(B; \mathbb{Z}_s) $$
associated with the short exact sequence $0\to \mathbb{Z}_s\to \mathbb{Z}_{s^2}\to\mathbb{Z}_s\to 0$, is also an isomorphism for $i=5$ or $13$. Since $\beta^{-1}(x^3)$ is a generator of $H^5(B; \mathbb{Z}_s)$,
$\beta_s(\beta^{-1}(x^3))$ is a generator of $H^6(B; \mathbb{Z}_s)$. Denote $\beta_s(\beta^{-1}(x^3))$ by $m r_s(x^3)$ where $(m,s)=1$. Since
$$\beta_s(\beta^{-1}(x^3)\cup r_s(x^4))=\beta_s(\beta^{-1}(x^3))\cup r_s(x^4)= m r_s(x^7)$$
is a generator of $H^{14}(B; \mathbb{Z}_s)$, $\beta^{-1}(x^3) \cup r_s(x^4)$ is a generator of $H^{13}(B;\mathbb{Z}_s)$. Choosing appropriate $y_{13,s}$ such that $r_s(y_{13,s})$ is the dual of $\beta^{-1}(x^3) \cup r_s(x^4)$, then we have
$$\langle \beta^{-1}(x^3) \cup r_s(x^4), f_\ast([M]_s)\rangle=\langle f^\ast(\beta^{-1}(x^3) \cup r_s(x^4)), [M]_s\rangle=L(\zeta^3,\zeta^4).$$
Indeed $\mathbb{Z}_{s}\langle y_{13,s}\rangle \subset H_{13}(B)$, so the component in $\mathbb{Z}_{s}\langle y_{13,s}\rangle$ of the image of the orientation class $[M]$ under $\bar{\mathscr{N}}_M$ is determined by $\ell k(M)$.

Next we consider the role played by the second Pontryagin class. Recall $$H^\ast(\mm{BSO};\mathbb{Z})=\mathbb{Z}[p_1,p_2,\cdots]\oplus A$$
 where $p_i$ is the $i$-th Pontryagin class, $A$ contains the image of the Bockstein homomorphism
 $\beta :H^i(\mm{BSO};\mathbb{Z}_2)\to H^{i+1}(\mm{BSO};\mathbb{Z}).$ By the Serre spectral sequences for the fibrations,
$$\mm{K}(\mathbb{Z}_2,1)\to \mm{BSpin}\to \mm{BSO}$$
$$\mm{K}(\mathbb{Z},3)\to \mm{BO}\langle 8 \rangle \to \mm{BSpin}$$
the second Pontryagin class $p_2(\gamma_8)$ of the universal bundle $\gamma_8$ over $\mm{BO}\langle 8 \rangle$ is $2^n3^tm_8^\ast$ for some $n,t\ge 0$ where $m_8^\ast$ is a generator of $H^8(\mm{BO}\langle 8 \rangle; \mathbb{Z})=\mathbb{Z}$. Since the stable normal bundle $\mathscr{N}_M$ of
the homological Bazaikin space $M$ with the $\mathscr{B}$-structure satisfies $\mathscr{N}_M\cong {\nu}^{\ast}\gamma_8 \oplus f^{\ast}\eta$,
$$p_2(\mathscr{N}_M)={\nu}^{\ast}p_2(\gamma_8)+f^{\ast}p_2(\eta).$$
 Since $f$ is a $6$-equivalence, $f^{\ast}p_2(\eta)$ is fixed. Thus $p_2(\mathscr{N}_M)$ is only relevant to ${\nu}^{\ast}p_2(\gamma_8)$. Besides, $p_1(\mathscr{N}_M)$ is also fixed, equal to $-k\zeta^2$. Hence the second Pontryagin class $p_2(M)$ of the tangent bundle of $M$ determines  the image ${\nu}^{\ast}p_2(\gamma_8)$ and thus the image ${\nu}^{\ast}m_8^\ast$ by ($6,s$)$=1$.
Let $r_s(m_8\otimes y_5)$ be the dual of $\beta^{-1}(x^3)\cup r_s(p_2(\gamma_8))\in H^{13}(\mathscr{B};\mathbb{Z}_s)$, ${\nu}^{\ast}p_2(\gamma_8)$ denoted by $n\zeta^4$, then
$$\langle \beta^{-1}(x^3)\cup r_s(p_2(\gamma_8)), \bar{\mathscr{N}}_{M\ast}([M]_s)\rangle=\langle \beta^{-1}(\zeta^3) \cup nr_s(\zeta^4), [M]_s\rangle=n\ell k(M).$$
Hence the component in $\mathbb{Z}_{s}\langle m_8\otimes y_{5}\rangle$ of the image of the orientation class $[M]$ under $\bar{\mathscr{N}}_M$ is determined by $\ell k(M)$ and $p_2(M)$.

The Lemma \ref{Zsinvar} follows by above discussions. Conversely, if two Bazaikin spaces, saying $\mathcal{B}_{\bold q_ 1}$ and $\mathcal{B}_{\bold q _2}$  admit the same normal $5$-smoothing and their bordism classes have the same summand in $\mb{Z}_s\oplus \mathbb{Z}_s\subset \Omega_{13}^{\mm{O}\langle 8 \rangle}(\eta)$, then $\ell k(\mathcal{B}_{\bold q _1})=\ell k(\mathcal{B}_{\bold q _2})\mod s$, $$n_1\ell k(\mathcal{B}_{\bold q _1})=n_2\ell k(\mathcal{B}_{\bold q _2})\mod s.$$
Since $(\ell k(\mathcal{B}_{\bold q_1 }),s)=(\ell k(\mathcal{B}_{\bold q_2 }),s)=1$ (see \cite{FL}), we get $n_1=n_2\mod s$.

%\begin{lemma}\label{Zsinvar}
%Let $(s,5)=1$. If $M_0$ and $M_1$ are homological Bazaikin spaces with the same 5-smoothing, link form and the second Pontryagin class, then the image $\bar{\nu}_\ast([M_0])$ located in $\mathbb{Z}_s\oplus \mathbb{Z}_s$ is equal to $\bar{\nu}_\ast([M_1])$
%when the generators of $\mathbb{Z}_s\oplus \mathbb{Z}_s$ are fixed.
%\end{lemma}

\section{The $2$-primary part of $\Omega_{13}^{\mm{O}\langle 8 \rangle}(\eta)$}\label{2primary}

Let $U\in H^0(\mm{M}\eta;\mathbb{Z}_2) $ be the stable Thom class of $\eta$. Note $H^i(\mm{M}\eta;\mathbb{Z}_2) \cong H^i(B;\mathbb{Z}_2)\cup U $ as $\mathbb{Z}_2$ vector space for any $i > 0$. Thus one can compute the $\mathscr{A}_2$-module structure of $H^\ast(\mm{M}\eta;\mathbb{Z}_2)$ with the $\mathscr{A}_2$-module structure of $H^\ast(B;\mathbb{Z}_2)$, Cartan formula and the fact $\mm{Sq}^iU=w_i(\eta)U$ where
 $w_i(\eta) $ is the $i$-th Stiefel-Whitney class of $\eta$. From Section \ref{normal5},
$ w_2(\eta)\ne 0, w_4(\eta)= 0.$

Before computing the $E_2$-term, $\mm{Ext}_{\mathscr{A}_2}^{s,t}(H^\ast(\mm{M}\eta;\mathbb{Z}_2),\mathbb{Z}_2)$, of the Adams spectral sequence (ASS) for $\mm{M\mathscr{B}}$, we recall  some propositions in homological algebra.

\begin{lemma}\cite{Adams1960} \label{E-T}
Let $K$ be a field, $A$ a graded and locally finite-dimensional algebra over $K$, $N$ a graded and locally finitely-generated module over $A$. Then
$\mm{Ext}_{A}^{s,t}(N, K)\cong \mm{Tor}_{s,t}^A(K,N)$
as the $K$ vector space.
\end{lemma}


By the lemma \ref{E-T}, our goal turns to $\mm{Tor}^{\mathscr{A}_2}_{s,t}(\mathbb{Z}_2, H^\ast (\mm{M}\eta;\mathbb{Z}_2)).$ From \cite{Adams1960} we know that the bar construction gives us a standard resolution of $H^\ast (\mm{M}\eta;\mathbb{Z}_2)$ over $\mathscr{A}_2$. Let $A\otimes B$ be $A\otimes_{\mathbb{Z}_2} B$, $N$ denote $H^\ast (\mm{M}\eta;\mathbb{Z}_2)$.
$$\overline{\mathscr{A}_2}=\mathscr{A}_2/(\mathscr{A}_2)_0,\quad (\overline{\mathscr{A}_2})^0=\mathbb{Z}_2, \quad (\overline{\mathscr{A}_2})^s=\overline{\mathscr{A}_2}\otimes(\overline{\mathscr{A}_2})^{s-1}(s>0),$$
$$\overline{B}(\mathscr{A}_2)=\Sigma_{i\ge 0}(\overline{\mathscr{A}_2})^i,\quad \quad\quad B(N)=\overline{B}(\mathscr{A}_2)\otimes N.$$
We write the elements of bar complexes $(\overline{\mathscr{A}_2})^s$ and $(\overline{\mathscr{A}_2})^s\otimes N$ in the forms
$$[a_1|a_2|\cdots| a_s], \quad \quad [a_1|a_2|\cdots| a_s]b.$$
 The boundaries $\bar{\partial}$ in $\overline{B}(\mathscr{A}_2)$ and $\partial$ in $B(N)$ are as follows
$$\bar{\partial}[a_1|a_2|\cdots| a_s]=\Sigma_{1\le i<s}[a_1|\cdots|a_ia_{i+1}|\cdots| a_s]$$
$$\partial[a_1|a_2|\cdots| a_s]b=\Sigma_{1\le i<s}[a_1|\cdots|a_ia_{i+1}|\cdots| a_s]b+[a_1|a_2|\cdots| a_{s-1}]a_sb.$$
$$H_s(\overline{B}(\mathscr{A}_2))=\mm{Tor}^{\mathscr{A}_2}_{s,t}(\mathbb{Z}_2, \mathbb{Z}_2)\cong \mm{Ext}_{\mathscr{A}_2}^{s,t}(\mathbb{Z}_2, \mathbb{Z}_2)$$
$$H_s(B(N))=\mm{Tor}^{\mathscr{A}_2}_{s,t}(\mathbb{Z}_2, N)\cong \mm{Ext}_{\mathscr{A}_2}^{s,t}(N, \mathbb{Z}_2)$$

% Figure environment removed

   The cobar complexes of $\mathscr{A}_2^\ast$ and $N^\ast$ are the vector-space duals of $B(\mathscr{A}_2)$ and $B(N)$ where $\mathscr{A}_2^\ast$ is the dual of $\mathscr{A}_2,$ $N^\ast=H_\ast (\mm{M}\eta;\mathbb{Z}_2)$. Indeed $\mathscr{A}_2^\ast$ is a subalgera of $\mathscr{A}^\ast=\mb{Z}_2[\xi_1,\xi_2,\xi_3,\cdots]$,  the dual of the mod (2) Steenrod algebra $\mathscr{A}$.
    The cobar complexes are denoted by $cB(\mathscr{A}^\ast_2)$ and $cB(N^\ast)$ with coboundaries $\bar{\delta}$ and $\delta$. By \cite{Giambalvo}, $H^\ast(cB(\mathscr{A}^\ast_2))$ is the $E_2$-term of the ASS for $\mm{MO}\langle 8 \rangle$  (see Fig.\ref{Fig.1}) in the range $t-s\le  15$. From (\cite{Adams1960} Theorem 2.4.1), $h_i=\xi_1^{2^i}\in \mathscr{A}^\ast$ duals to $\mathrm{Sq}^{2^i}\in \mathscr{A}$. Thus we have a representative in bar complex $B(\mathscr{A}_2)$ for every cohomology class of $\mm{Ext}^{s,t}_{\mathscr{A}_2}(\mathbb{Z}_2, \mathbb{Z}_2)$, in particular,
$h_0^s$ is represented by $[\mathrm{Sq}^1| \cdots | \mathrm{Sq}^1];$
$h_1^2$ by $[\mathrm{Sq}^2|\mathrm{Sq}^2]+[\mathrm{Sq}^3| \mathrm{Sq}^1]$; $h_0h_2$  by $[\mathrm{Sq}^1| \mathrm{Sq}^4]+[\mathrm{Sq}^2| \mathrm{Sq}^3]+[\mathrm{Sq}^4| \mathrm{Sq}^1];$ $h_0^2h_2$ by
$$[\mathrm{Sq}^1| \mathrm{Sq}^1|\mathrm{Sq}^4]+[\mathrm{Sq}^1| \mathrm{Sq}^2| \mathrm{Sq}^3]+[\mathrm{Sq}^1|\mathrm{Sq}^4| \mathrm{Sq}^1]+[\mathrm{Sq}^2|\mathrm{Sq}^3|\mathrm{Sq}^1]+$$
$$[\mathrm{Sq}^4|\mathrm{Sq}^1|\mathrm{Sq}^1]+[\mathrm{Sq}^3|\mathrm{Sq}^1|\mathrm{Sq}^2]+[\mathrm{Sq}^2|\mathrm{Sq}^2|\mathrm{Sq}^2];$$
$h_2^2$ by $[\mathrm{Sq}^4| \mathrm{Sq}^4]+[\mathrm{Sq}^7| \mathrm{Sq}^1]+[\mathrm{Sq}^6| \mathrm{Sq}^2].$

Let $N^n=H^n(\mm{M}\eta;\mathbb{Z}_2)$, $N_n=\oplus_{i\ge n}H^i(\mm{M}\eta;\mathbb{Z}_2)$. For the sake of simplicity we abbreviate $\mm{Ext}^{ s,t} _{\mathscr{A}_2}(N, \mathbb{Z}_2)$ by $\mm{Ext}^{s,t}(N)$, $\mm{Tor}^{\mathscr{A}_2}_{s,t}(\mathbb{Z}_2, N)$ by $\mm{Tor}_{s,t}(N)$. The short exact sequence of chain complexes
$$0\to B(N_{n+1})\stackrel{i}{\rightarrow} B(N_n)\stackrel{j}{\rightarrow} B(N^n)\to 0$$
induces the long exact homology sequence
$$\cdots \to \mm{Tor}_{s,t}(N_{n+1})\stackrel{i_\ast}{\rightarrow}\mm{Tor}_{s,t}(N_n)\stackrel{j_\ast}{\rightarrow}\mm{Tor}_{s,t}(N^n)\stackrel{d_\ast}{\rightarrow}\mm{Tor}_{s-1,t}(N_{n+1})\to \cdots,$$
which gives rise to an exact couple ($D_1, E_1; i_\ast, j_\ast,d_\ast$)
\[
\xymatrix{
 D_1^{s,m,n} \ar[rr]^-{i_\ast}& &D_1^{s,m,n}\ar[ld]^-{j_\ast}\\
     &  E_1^{s,m,n} \ar[lu]^-{d_\ast}     &
}
\]
with $E_1^{s,m,n}=\mm{Tor}^{\mathscr{A}_2}_{s,s+m}(\mathbb{Z}_2,\mathbb{Z}_2)\otimes N^n$, $D_1^{s,m,n}=\mm{Tor}_{s,n+m+s}(N_n)$ and the differential
$$d_1=j_\ast d_\ast:E_1^{s,m,n}\to E_1^{s-1,m,n+1},\quad d_r:E_r^{s,m,n}\to E_r^{s-1,m+1-r,n+r}$$
where $n+m+s=t$.
Given $s$, $m$ and $n$, there exists $n_0$ such that $$D_1^{s-1,m^\prime,n^\prime}=0, \quad s-1+m^\prime+n^\prime=t$$
for $n^\prime> n_0$. Let $r_0=$max$\{n_0-n,n\}$,
$E_{r_0+1}^{s,m,n}=E_{r_0+2}^{s,m,n}=\cdots=E_{\infty}^{s,m,n}$
which means that the spectral sequence called Algebraic Atiyah-Hirzebruch spectral sequence (AAHSS), converges finitely to $\mm{Tor}_{s,t}(N)$ \cite{HiltonStammbach1997}.
The details of the computation of AAHSS are showed in appendix \ref{AAHSSmod2}. Here we give the final result (see Fig.\ref{Fig.2}). One dot in Fig.\ref{Fig.2} denotes a generator of $\mm{Ext}^{s,t}(N)$ as $\mb{Z}_2$-vector space. These generators of $\mm{Ext}^{s,t}(N)$ are represented by the elements in the $E_1$-term of the AAHSS for $\mm{Tor}_{s,t}(N)$ e.g. $h_1x^2\mm{Sq}^1uU$, $h_1^2vxU$. Next we consider the  $h_0$-module structure of $\mm{Ext}^{s,t}(N)$ denoted by the line $|$ between two dots in Fig.\ref{Fig.2}.

% Figure environment removed


By Lemma \ref{E-T}, the generators of $\mm{Ext}^{s,t}(N)$ can be  represented by elements in either the bar complex $B(N)$, or the cobar complex $cB(N^\ast)$. From the computations for AAHSS in appendix \ref{AAHSSmod2}, we get the following representatives in $B(N)$ for some generators of $\mm{Ext}^{s,t}(N)$. For convenience, we will use the same symbols to denote elements in $N$ and its  dual in $N^\ast$.

The cohomology class $h_1 x^2 \mathrm{Sq}^1uU\in \mm{Ext}^{1,14}(N)$ is represented by $$[\mathrm{Sq}^2] x^2\mathrm{Sq}^1uU+[\mathrm{Sq}^1] vx^2U$$ in the bar complex.
$\mathrm{Sq}^2( x^2\mathrm{Sq}^1uU)=\mathrm{Sq}^1( vx^2U)=x^2\mathrm{Sq}^2\mathrm{Sq}^1uU$. Thus
$$\delta(x^2\mathrm{Sq}^2\mathrm{Sq}^1uU)=\xi_{1}\otimes vx^2 U+\xi^2_{1}\otimes x^2 \mathrm{Sq}^1uU+\cdots$$
$$\delta(\xi_{1}\otimes vx^2 U)=0$$
in the cobar complex, $h_{0}vx^2 U=h_{1} x^2 \mathrm{Sq}^1uU\in \mm{Ext}^{1,14}(N)$.


$h_1^2 ux^2U$ is represented by %$[h_1^2 ux^2U]=$
$$[\mathrm{Sq}^2| \mathrm{Sq}^2] ux^2U+[\mathrm{Sq}^3| \mathrm{Sq}^1] ux^2U+[\mathrm{Sq}^1| \mathrm{Sq}^2] x^2\mathrm{Sq}^1uU+[\mathrm{Sq}^1| \mathrm{Sq}^1] vx^2U$$ in the bar complex,
%$$\partial(\mathrm{Sq}^2\otimes \mathrm{Sq}^2\otimes ux^2U+\mathrm{Sq}^3\otimes \mathrm{Sq}^1\otimes ux^2U+\mathrm{Sq}^1\otimes \mathrm{Sq}^2\otimes x^2\mathrm{Sq}^1uU)=\partial(\mathrm{Sq}^1\otimes \mathrm{Sq}^1\otimes vx^2U),$$
hence there exists $\alpha\in cB^1(N^\ast)$ such that
$$\delta(\alpha)=\xi_{1}\otimes \xi_{1}\otimes vx^2 U+\xi^2_{1}\otimes \xi^2_{1}\otimes ux^2U+\cdots$$
$$\delta(\xi_{1}\otimes \xi_{1}\otimes vx^2 U)=0$$
in the cobar complex, $h^2_0vx^2 U=h^2_1 ux^2U\in \mm{Ext}^{2,15}(N)$. Indeed
$$\alpha=\xi_{1}\otimes x^2\mm{Sq}^2\mm{Sq}^1uU+\xi_{2}\otimes x^2\mm{Sq}^1uU+\xi_{1}\xi_{2}\otimes ux^2U.$$

$h_1^2 vxU$ is represented by $$[h_1^2 vxU]=[\mathrm{Sq}^2| \mathrm{Sq}^2] vxU+[\mathrm{Sq}^3| \mathrm{Sq}^1] vxU+[\mathrm{Sq}^1| \mathrm{Sq}^2] x\mathrm{Sq}^2\mathrm{Sq}^1uU$$ in the bar complex.
$h_1^3$ is represented by
$$[\mathrm{Sq}^2\otimes \mathrm{Sq}^2\otimes \mathrm{Sq}^2]=[\mathrm{Sq}^2| \mathrm{Sq}^2| \mathrm{Sq}^2]+[\mathrm{Sq}^3| \mathrm{Sq}^1| \mathrm{Sq}^2]+[\mathrm{Sq}^4| \mathrm{Sq}^1| \mathrm{Sq}^1]$$
$$+[\mathrm{Sq}^2| \mathrm{Sq}^3| \mathrm{Sq}^1]+[\mathrm{Sq}^1| \mathrm{Sq}^4| \mathrm{Sq}^1]+[\mathrm{Sq}^1| \mathrm{Sq}^2| \mathrm{Sq}^3]+[\mathrm{Sq}^1| \mathrm{Sq}^1| \mathrm{Sq}^4]$$
%\in {B}^3(\mathscr{A}_2)
in the bar complex.
Since
$$\partial([\mathrm{Sq}^2\otimes \mathrm{Sq}^2\otimes \mathrm{Sq}^2]\otimes (ux+v)U)=[h_1^2 vxU]+$$
$$[\mathrm{Sq}^1| \mathrm{Sq}^4] (\mathrm{Sq}^2\mathrm{Sq}^1u+x\mathrm{Sq}^1u )U+\cdots$$
in the bar complex, $h^2_1vx U$ and $h_0h_2 (\mathrm{Sq}^2\mathrm{Sq}^1u+x\mathrm{Sq}^1u )U$ represent the same class in $\mm{Ext}^{2,15}(N)$ which implies $h_0h^2_1vx U=h^2_0h_2 (\mathrm{Sq}^2\mathrm{Sq}^1u+x\mathrm{Sq}^1u )U$.



Now we consider some ASS differentials and have that
\begin{lemma}
	As in Fig.\ref{Fig.2}, $vx^2U$, $h_1x^2\mathrm{Sq}^1uU$, $h_1^2ux^2U$, $h_1^2vxU$ and $h_1vx^2U$ survive to $E_\infty$-term in the ASS for $\pi_\ast(\mm{MO}\langle 8  \rangle\wedge \mm{M}\eta)$.
\end{lemma}
\begin{proof}
	From \cite{FangWang2010}, $c_0x^2U\in \mm{Ext}_{\mathscr{A}_2}^{3,15}(H^\ast(\mm{M}\xi; \mathbb{Z}_2), \mathbb{Z}_2)$ survives in the ASS for $\mm{MO}\langle 8  \rangle\wedge \mm{M}\xi$ where $\mm{M}\xi$ is the Thom spectra of the bundle $\xi$ over $\mathrm{CP}^\infty$. By the morphism between the Adams spectral sequences for $\mm{M}\eta$ and $ \mm{M}\xi$ induced by
$p:B\to \mathrm{CP}^\infty$,   $c_0x^2U$ can not be killed by any Adams differentials in the ASS for $\mm{MO}\langle 8  \rangle\wedge \mm{M}\eta$, which implies $d_3(vx^2U)=0$. Thus, from the Fig.\ref{Fig.2}, $d_r(vx^2U)=0$ for $r\ge 2$. By the extension as $h_0$ module, $d_r(h_1x^2\mathrm{Sq}^1uU) =d_r(h_1^2ux^2U)=0$ for $r\ge 2$.  Moreover, $d_r(h_1vx^2U)=0$ for $r\ge 2$  by the product of the ASS. So the desired results follows by $\mm{Ext}_{\mathscr{A}_2}^{0,14}(H^\ast(\mm{M}\eta; \mathbb{Z}_2), \mathbb{Z}_2)=0$.
\end{proof}


Next we consider the filtrations in the AHSS of the above elements in $\pi_\ast(\mm{M\mathscr{B}})$ through comparing different spectral sequences.
The following proposition is necessary to describe the morphism between spectral sequences more conveniently. We will use it frequently, such as in the proof of Proposition \ref{221}.
\begin{proposition}\label{cosetspectralsequence}
	For a spectral sequence $E_2^{s}\Longrightarrow N$ with the short exact sequence
	$0\to F^{s\pm1}\to F^{s}\to E_\infty^{s}\to 0$, if $z_1,z_2\in N$ are {\it provided} (cf. Definition \ref{provided}) by the same $x\in E_2^{s,t}$, then $z_1-z_2=z_3\in F^{s\pm1}$.
\end{proposition}

 The symbol $\pm 1$ means the spectral sequence with decreasing or increasing filtrations. Indeed $x\in E_\infty^{s,t}$ represents a coset $x+F^{s\pm1}$. We use $\{x\}$ to denote an element provided by $x\in E_2^{s,t}$ in the ASS.

\begin{proposition}\label{221}
  $\{h_1^2ux^2U\}, \{h_1^2vxU\} \in \pi_{13}(\mm{M\mathscr{B}})$  are provided by
$$E_2^{p,13-p}=H_{p}(\mm{M}\eta;\pi_{13-p}(\mm{MO}\langle 8 \rangle))$$ for some $11\le p\le 13$ in the AHSS. Moreover, $\{h_1vx^2U\}\in \pi_{14}(\mm{M\mathscr{B}})$ is provided by
$$E_2^{13,1}=H_{13}(\mm{M}\eta;\pi_{1}(\mm{MO}\langle 8 \rangle)).$$
\end{proposition}
\begin{proof}
Let $\Pi$ be an element provided by $h_1^2ux^2U$ in the ASS. Assume $\Pi$ is provided by $E_2^{p,13-p}$ for some $p\le 10$ in AHSS. The discussion for $p<10$ is same as $p=10$, so we may let $p=10$. By Proposition \ref{Z-B}, $H_{10}(\mm{M}\eta;\mathbb{Z})=0$, thus we can take partial $10$-skeleton of $\mm{M}\eta$ denoted by $\mm{M}\eta_{10}$ such that
$i_\ast:H_n(\mm{M}\eta_{10};\mathbb{Z})\to H_n(\mm{M}\eta;\mathbb{Z}) $
is an isomorphism for $n\le 10$, $H_n(\mm{M}\eta_{10};\mathbb{Z})=0$ for $n>10$. Consider the morphism of following spectral sequences
\[
\xymatrix@C=.3cm{
{E}_2^{p,q}(|_{10})=H_p(\mm{M}\eta_{10};\pi_q(\mm{MO}\langle 8\rangle))\ar@{=>}[rr]^-{}\ar[d]^-{}& & \pi_{p+q} ( \mm{MO}\langle 8\rangle \wedge \mm{M}\eta_{10})\ar[d]^-{}\\
{E}_2^{p,q}=H_p(\mm{M}\eta;\pi_q(\mm{MO}\langle 8\rangle))\ar@{=>}[rr]^-{}& & \pi_{p+q} ( \mm{MO}\langle 8\rangle\wedge \mm{M}\eta)
}
\]
By Lemma \ref{ahsspro}, ${E}_\infty^{10,3}(|_{10})\cong {E}_\infty^{10,3}$. Thus there exists $\bar{\Pi}\in \pi_{13}(\mm{MO}\langle 8\rangle \wedge \mm{M}\eta_{10})$ provided by
${E}_2^{10,3}(|_{10})$ such that
$$i_\ast:\pi_{13}(\mm{MO}\langle 8\rangle \wedge \mm{M}\eta_{10})\to \pi_{13}(\mm{MO}\langle 8\rangle \wedge \mm{M}\eta)$$
$$i_\ast (\bar{\Pi})={\Pi}+\Pi^1$$
where $\Pi^1$ denotes an element provided by ${E}_2^{p_1,13-p_1}$ where $p_1<10$.

Let $m$ be the dimension of the Adams filtration of $i_\ast(\bar{\Pi})$, $\bar{m}$ be the dimension of the Adams filtration of $\bar{\Pi}$. ${m}\ge \bar{m}$ follows by the naturality of ASS. By the AAHSS for
$Ext_{\mathscr{A}_2}^{\ast,\ast}(H^\ast(\mm{M}\eta_{10}; \mathbb{Z}_2), \mathbb{Z}_2),$
$$Ext_{\mathscr{A}_2}^{s,13+s}(H^\ast(\mm{M}\eta_{10}; \mathbb{Z}_2), \mathbb{Z}_2)=0$$
for $s=0,1$.
Although $h_2^2uU\in Ext_{\mathscr{A}_2}^{2,15}(H^\ast(\mm{M}\eta_{10}; \mathbb{Z}_2), \mathbb{Z}_2)$,
$$0=i_\ast (h_2^2uU)\in Ext_{\mathscr{A}_2}^{2,15}(H^\ast(\mm{M}\eta; \mathbb{Z}_2), \mathbb{Z}_2).$$
Hence $\bar{m}\ge 2$, $m>2$. Since that the Adams filtration of $\Pi$ is $2$ and the Adams filtration of $i_\ast(\bar{\Pi})$ is greater than $2$,
 ${\Pi}^1=i_\ast(\bar{\Pi})-\Pi$ is also provided by $h_1^2ux^2U$ in the ASS for $\pi_{13}(\mm{MO}\langle 8\rangle \wedge \mm{M}\eta)$.

Note that ${\Pi}^1\in \pi_{13}(\mm{MO}\langle 8\rangle \wedge \mm{M}\eta)$ is provided by
${E}_2^{p_1,13-p_1}$ where $p_1<10$. By Lemma \ref{ahsspro}, $i_\ast:E_\infty^{p_1,13-p_1}(|_{10})\to E_\infty^{p_1,13-p_1}$ is an epimorphism, thus there exists $\bar{\Pi}^1\in \pi_{13}(\mm{MO}\langle 8\rangle \wedge \mm{M}\eta_{10})$ provided by
${E}_2^{p_1,13-p_1}(|_{10})$ such that
$i_\ast (\bar{\Pi}^1)={\Pi}^1+\Pi^2$ where $\Pi^2$ is provided by $E_2^{p_2,13-p_2}$ in the AHSS for $p_2<p_1$. Due to the dimension $m^1>2$ of the Adams filtration of $i_\ast(\bar{\Pi}^1)$, $\Pi^2$ is also provided by $h_1^2ux^2U$ in the ASS.   Repeating the process, for some $n$, we get the element $\bar{\Pi}^n\in   \pi_{13}(\mm{MO}\langle 8\rangle \wedge \mm{M}\eta_{10})$ such that $i_\ast (\bar{\Pi}^n)={\Pi}^n$ is provided by $h_1^2ux^2U$ since $10>p_1>\cdots >p_n\ge 0$ and $E_2^{p,q}=0$ for $p< 0$. However the Adams filtration of $i_\ast(\bar{\Pi}^n)=\Pi^n$ is ${m}^n> 2$
which contradicts with the summary that the Adams filtration of $i_\ast (\bar{\Pi}^n)={\Pi}^n$ is $2$. The desired results for $\{h_1^2ux^2U\}$ and $\{h_1^2vxU\}$ follow by repeating above discussion for each $p\le 10$.


%Assume $\Pi_1$ and $\Pi_2$ are provided by $E_2^{p,13-p}(\mm{AHSS})$ for $p<10$, we can also derive contradictions as above.

Similarly, we can prove that  $\{h_1vx^2U\}$ is provided by
$E_2^{p,14-p}$ for some $13\le p\le 14$ in the AHSS. By Lemma \ref{d2},
$E_3^{14,0}=0.$
So $\{h_1vx^2U\}$ is provided by $E_2^{13,1}$.
\end{proof}

\begin{corollary}\label{011}
 $\{vx^2U\}$ is provided by $E_2^{13,0}$ and
  $\{h_1x^2\mathrm{Sq}^1uU\}$ is provided by $E_2^{12,1}$. Moreover,
 $\{h_1^2ux^2U\}, \{h_1^2vxU\}$ are provided by $E_2^{11,2}$.
\end{corollary}
\begin{proof}
From the extension relation, the results for $\{vx^2U\}$, $\{h_1x^2\mathrm{Sq}^1uU\}$ and $\{h_1^2ux^2U\}$ follow by Proposition \ref{221}. Since that $E_3^{13,0}$ and $E_3^{12,1}$ have only one subgroup with order 2 respectively by Lemma \ref{d2}, $\{h_1^2vxU\}$ is provided by $E_2^{11,2}$.
\end{proof}

Next we discuss other elements with higher Adams filtrations.

\begin{proposition}
	$d_2(\tau xU)=h_2\omega xU$ in the $\mm{ASS}$ for $\mm{MO}\langle 8\rangle\wedge \mm{M}\eta$.
\end{proposition}
\begin{proof}
$\mm{CP}^1$ is the $2$-skeleton of $B$. Let $\mm{M}\eta|_{\mathrm{CP}^1}$ be the Thom spectrum associated with the restriction bundle $\eta|_{\mm{CP}^1}$.
Consider the AHSS
 $$E_2^{p,q}(|_{\mathrm{CP}^1})=H_p(\mm{M}\eta|_{\mathrm{CP}^1};\pi_{q}(\mm{MO} \langle 8  \rangle))\Longrightarrow\pi_\ast(\mm{MO}\langle 8\rangle\wedge \mm{M}\eta|_{\mathrm{CP}^1})$$
Since $E_2^{2,11}(|_{\mathrm{CP}^1})=0 $ and $E_2^{0,13}(|_{\mathrm{CP}^1})=\mb{Z}_3$, $_2\pi_{13}(\mm{MO} \langle 8  \rangle\wedge \mm{M}\eta|_{\mathrm{CP}^1})=0.$

By the AAHSS, we get $ Ext_{\mathscr{A}_2}^{s,t}(H^\ast(\mm{M}\eta|_{\mathrm{CP}^1}; \mathbb{Z}_2), \mathbb{Z}_2)$ showed in Fig.\ref{Fig.3}. Since $_2\pi_{13}(\mm{MO} \langle 8  \rangle\wedge \mm{M}\eta|_{\mathrm{CP}^1})=0$, $d_2(\tau xU)=h_2\omega xU$ in the ASS for $\mm{MO}\langle 8 \rangle\wedge \mm{M}\eta|_{\mathrm{CP}^1}$.
By the map $\mm{M}\eta|_{\mathrm{CP}^1}\to \mm{M}\eta$ induced by $ \mathrm{CP}^1\to B$,
$$d_2(\tau xU)=h_2\omega xU$$
 in the ASS for $\mm{MO}\langle 8 \rangle\wedge \mm{M}\eta$.
\end{proof}

% Figure environment removed

\begin{lemma}\label{CP1}
 (1) $_2\pi_{12}(\mm{MO}\langle 8\rangle \wedge \mm{M}\eta|_{\mathrm{CP}^1})=\mathbb{Z}$ is provided by the extension of $\mathbb{Z}_2\subset E_2^{2,10}(|_{\mathrm{CP}^1})$ and $E_2^{0,12}(|_{\mathrm{CP}^1})=\mb{Z}$ in the $\mm{AHSS}$ for $\mm{MO}\langle 8\rangle \wedge \mm{M}\eta|_{\mathrm{CP}^1}$.
	
	(2) The differential $d_2:E_2^{2,9}(|_{\mathrm{CP}^1})\to E_2^{0,10}(|_{\mathrm{CP}^1})$ is nontrivial.
	
(3) $\{h_1c_0xU\}$ is provided by $E_2^{2,9}(|_{\mathrm{CP}^1}).$
\end{lemma}
\begin{proof}
From Fig.\ref{Fig.3}, $_2\pi_{12}(\mm{MO}\langle 8\rangle \wedge \mm{M}\eta|_{\mathrm{CP}^1})=\mathbb{Z}$. By the $E_\infty$-terms
\begin{equation}
E_\infty^{2,10}(|_{\mathrm{CP}^1})=E_2^{2,10}(|_{\mathrm{CP}^1})=\mathbb{Z}_2\oplus \mathbb{Z}_3, \label{1CP210}
\end{equation}
 \begin{equation}
 	E_\infty^{0,12}(|_{\mathrm{CP}^1})=E_2^{0,12}(|_{\mathrm{CP}^1})=\mathbb{Z}, \label{1CP012}
 \end{equation}
 the desired extension follows.

Since that $Ext^{s,s+10}_{\mathscr{A}_2}(H^\ast(\mm{M}\eta|_{\mathrm{CP}^1};\mathbb{Z}_2),\mathbb{Z}_2)$ is nontrivial only for $s\ge 5$ and has generator $h_0^{s-4}\omega xU$ which survives to $E_\infty^{s,s+10}$ in the ASS for $\mm{MO}\langle 8\rangle \wedge \mm{M}\eta|_{\mathrm{CP}^1}$, we get that
${_2}\pi_{10}(\mm{MO}\langle 8\rangle \wedge \mm{M}\eta|_{\mathrm{CP}^1})=\mathbb{Z}$ and $h_1c_0xU$ survives to $E_\infty$ in the ASS for $\mm{MO}\langle 8\rangle \wedge \mm{M}\eta|_{\mathrm{CP}^1}$.
Thus $\mathbb{Z}_2\subset E_2^{0,10}(|_{\mathrm{CP}^1})=\mb{Z}_6$ is killed by the differential
$$d_2:E_2^{2,9}(|_{\mathrm{CP}^1})=\mb{Z}_2\oplus\mb{Z}_2\to E_2^{0,10}(|_{\mathrm{CP}^1}).$$
 Moreover, $\{h_1c_0xU\}$ is provided by $\mb{Z}_2=E_\infty^{2,9}(|_{\mathrm{CP}^1})$.
\end{proof}

\begin{corollary}\label{210012}
	There is a free subgroup $\mb{Z}$
of ${\pi}_{12}(\mm{MO}\langle 8\rangle \wedge \mm{M}\eta)$ provided by the extension of $\mathbb{Z}_2\subset E_2^{2,10}$ and $E_2^{0,12}$ in the $\mm{AHSS}$ for $\mm{MO}\langle 8\rangle \wedge \mm{M}\eta$.
\end{corollary}
\begin{proof}
	Note that $h_0^sh_2\omega U\in Ext_{\mathscr{A}_2}^{s,s+11}(H^\ast(\mm{M}\eta; \mathbb{Z}_2), \mathbb{Z}_2)$ $0\le s\le 1$, $h_0^2h_2\omega U=0$,
	$h_0^s\tau U\in Ext_{\mathscr{A}_2}^{s,s+12}(H^\ast(\mm{M}\eta; \mathbb{Z}_2), \mathbb{Z}_2)$ $ s\ge 0.$
	By the map $\mm{M}\eta|_{\mm{CP}^1}\to \mm{M}\eta$, $d_2(\tau U)=h_2\omega U$ in the ASS for $\mm{M}\eta$, $h_0^2\tau U$ represents the generator of $\mb{Z}\subset {_2\pi}_{12}(\mm{MO}\langle 8\rangle \wedge \mm{M}\eta)$. By Lemma \ref{CP1}, the desired result follows.
\end{proof}

\begin{corollary}\label{CP2}
	In the $\mm{AHSS}$ for $\pi_\ast(\mm{MO}\langle 8\rangle \wedge \mm{M}\eta|_{\mathrm{CP}^2})$ with
	$$E_2^{p,q}(|_{\mathrm{CP}^2})=H_p(\mm{M}\eta|_{\mathrm{CP}^2};\pi_{q}(\mm{MO}\left \langle 8 \right \rangle)),$$
	the differential $d_2:E_2^{4,8}(|_{\mathrm{CP}^2})\to E_2^{2,9}(|_{\mathrm{CP}^2})$ is trivial.
\end{corollary}
\begin{proof}
	By $\{h_1c_0xU\}\in {_2\pi}_{11}(\mm{MO}\langle 8\rangle \wedge \mm{M}\eta|_{\mathrm{CP}^1})$, $h_1c_0xU$ is a permanent cycle in the ASS for $\mm{MO}\langle 8\rangle\wedge \mm{M}\eta|_{\mathrm{CP}^2}$. Since $$Ext_{\mathscr{A}_2}^{s,s+12}(H^\ast(\mm{M}\eta|_{\mm{CP}^2}; \mathbb{Z}_2), \mathbb{Z}_2)=0,\quad 0\le s\le 2$$
$h_1c_0xU$ survives to $E_\infty$ in the ASS.	
	 So the desired result follows by the naturality of AHSS and (2), (3) in Lemma \ref{CP1}.
\end{proof}

\begin{proposition}\label{4948}
 If $h_1\omega x^2U$ or $h_1c_0 x^2U$ survives in the ASS, $\{h_1\omega x^2U\}$ or $\{h_1c_0 x^2U\}$ is provided by
$E_2^{4,9}=H_{4}(\mm{M}\eta;\pi_{9}(\mm{MO}\langle 8 \rangle)).$
Moreover, $c_0x^2$ survives in the ASS, and $\{c_0x^2U\}$ is provided by
$E_2^{4,8}=H_{4}(\mm{M}\eta;\pi_{8}(\mm{MO}\langle 8 \rangle)).$
\end{proposition}
\begin{proof}
%By the proof proposition 4.2 in \cite{FangWang2010},
$c_0x^2U$ survives in the ASS for $\mm{MO}\langle 8\rangle \wedge \mm{M}\eta$ since it is a permanent cycle and
$d_3(vx^2U)=0.$
From \cite{FangWang2010}, $\{c_0x^2U\}$ is either
provided by $E_2^{4,8}$ or $E_2^{2,10}$ in the AHSS for $\mm{MO}\langle 8\rangle \wedge \mm{M}\eta$. By Corollary \ref{210012}, $\{c_0x^2U\}$ is provided by
$E_2^{4,8}.$

$h_1\omega x^2U$ and $h_1c_0 x^2U$  survive in the ASS for $\mm{MO}\langle 8\rangle \wedge \mm{M}\eta|_{\mathrm{CP}^2}$, thus $\{h_1\omega x^2U\}$ and $\{h_1c_0 x^2U\}$ are provided by
$E_2^{4,9}(|_{\mathrm{CP}^2})$
due to
$$E_2^{2,11}(|_{\mathrm{CP}^2})=0 \quad\quad  E_2^{0,13}(|_{\mathrm{CP}^2})=\mathbb{Z}_3.$$
Assume $h_1\omega x^2U$ and $h_1c_0 x^2U$ survive in the ASS for $\mm{MO}\langle 8\rangle \wedge \mm{M}\eta$, then the desired result follows by the map $\mm{M}\eta|_{\mathrm{CP}^2}\to \mm{M}\eta$ .
\end{proof}

\begin{proposition}\label{Z4103}
$h_0^2h_2(\mathrm{Sq}^2\mathrm{Sq}^1u+x\mathrm{Sq}^1u)U$ survives in the ASS for $\mm{MO}\langle 8 \rangle \wedge \mm{M}\eta$, and $\{h_0^2h_2(\mathrm{Sq}^2\mathrm{Sq}^1u+x\mathrm{Sq}^1u)U\}$ is provided by
$E_2^{10,3}$ in the AHSS.
\end{proposition}
\begin{proof}
In the AHSS for $\mm{MO}\langle 8 \rangle \wedge \mm{M}\eta$, $E_2^{10,3}$ has two summands with order $2$, $\mathbb{Z}_2\oplus \mathbb{Z}_2.$ By Lemma \ref{d2} and the $\mathscr{A}_2$-module $H^\ast(\mm{M}\eta;\mathbb{Z}_2)$,
$$E_2^{12,2}/im(d_2)=\mathbb{Z}_2,\quad E_3^{14,0}=0.$$
By Proposition \ref{221}, $E_3^{13,1}=E_\infty^{13,1}$. By Proposition \ref{4948}, $\mathbb{Z}_2\subset E_2^{4,8}$ survives in $E_\infty^{4,8}$.
By Corollary \ref{210012}, $\mathbb{Z}_2\subset E_2^{2,10}$ survives in $E_\infty^{2,10}$.
 Note that $E_2^{8,4}=E_2^{7,5}=E_2^{6,6}=0$, $E_2^{0,12}=\mathbb{Z}$ and $E_2^{4,8}=\mathbb{Z}\oplus \mathbb{Z}_2$.
 Summarizing these conclusions,
  we know that the element with order 2 in $E_3^{10,3}$ survives to $E_\infty^{10,3}$. By Fig.\ref{Fig.2}, Proposition \ref{4948} and Corollary \ref{011}, $E_\infty^{10,3}$ only contains one $\mb{Z}_2$. Hence
 there is  a nontrivial differential
$d_2:E_2^{12,2}\to E_2^{10,3}$, and thus
 the element provided by $\mathbb{Z}_2\subset E_\infty^{10,3}$ in the AHSS is provided by $$Ext_{\mathscr{A}_2}^{3,16}(H^\ast(\mm{M}\eta;\mathbb{Z}_2),\mathbb{Z}_2)$$
 in the ASS for $\mm{MO} \langle 8  \rangle\wedge \mm{M}\eta$. The desired result follows.
\end{proof}

The second conclusion in Theorem \ref{bordismgroup} follows by the above discussions,
$$_2\pi_{13}(\mm{MO} \langle 8  \rangle\wedge \mm{M}\eta)=\mathbb{Z}_8\oplus \mathbb{Z}_4\oplus \mathbb{Z}^n_2$$
where $n=1$ or $2$. The value of $n$ is determined by the ASS differential $d_r$ about $h_2vxU$. If $d_3(h_2vxU)=h_1c_0x^2U$ or $d_4(h_2vxU)=h_1\omega x^2U$, then $n=1$. Otherwise $n=2$.

Finally we conclude this section by the following lemma.

\begin{lemma}\label{x2}
	$x^2U$ survives to $E_\infty^{0,4}$ in the $\mm{ASS}$ for $\mm{M}\eta$.
\end{lemma}
\begin{proof}
Note that $x^2U\in Ext_\mathscr{A}^{0,4}(H^\ast(\mm{M}\eta;\mathbb{Z}_2),\mathbb{Z}_2)$,
	$$h_0^sh_2U\in Ext_\mathscr{A}^{s+1,s+4}(H^\ast(\mm{M}\eta;\mathbb{Z}_2),\mathbb{Z}_2)\quad s=0,1.$$


Consider the bordism group
$\Omega_4^{\mm{Spin}}(\eta)=\pi_4(\mm{MSpin}\wedge \mm{M}\eta)$
$$\hat{\mathscr{B}}=\mm{BSpin}\times B\to \mm{BSO}$$
with $p_1(\eta)=-kx^2$, $w_2(\eta)\ne 0$ and $w_4(\eta)=0$. From the ASS,
$$3(1\wedge x^2_\ast U)\in H_4(\mm{MSpin}\wedge \mm{M}\eta;\mathbb{Z})$$
is in the image of Hurewicz homomorphism
$$\pi_4(\mm{MSpin}\wedge \mm{M}\eta)\to H_4(\mm{MSpin}\wedge \mm{M}\eta;\mathbb{Z}).$$
By the Thom construction, there is a manifold $M^4$ with $\hat{\mathscr{B}}$-structure
 $\hat{\bar{\mathscr{N}}}_M:M\to \hat{\mathscr{B}}$
 such that
 $$\hat{\bar{\mathscr{N}}}_{M\ast}([M])=3(1\otimes x^2_\ast)\in H_4(\mm{BSpin} \times B;\mathbb{Z})$$
  and $p_1(M)=3k$. Since that the first Pontrjagin class of the normal bundle of $M$ is provided by $\eta$ totally, $M$ has $\mathscr{B}$-structure and
  $$\bar{\mathscr{N}}_{M\ast}([M]_2)=(1\otimes x^2_\ast)\in H_4(\mm{BO}\langle 8\rangle \times B;\mathbb{Z}_2)\cong H_4(\mathscr{B};\mb{Z}_2),$$
 thus $x^2U\in E_\infty^{0,4}$ in the ASS for ${_2\pi}_\ast(\mm{MO}\langle 8\rangle\wedge \mm{M}\eta)$. Due to
 $$h_0^sh_2U\in Ext_{\mathscr{A}_2}^{s+1,s+4}(H^\ast(\mm{M}\eta;\mathbb{Z}_2),\mathbb{Z}_2)\quad s=0,1,$$
 the desired result follows.
\end{proof}


\section{The $3$-primary part of $\Omega_{13}^{\mm{O}\langle 8 \rangle}(\eta)$ }\label{3primary}

Follow the computation in Section \ref{2primary}, there is the AAHSS for
$$\mm{Ext}_{\mathcal{A}}^{s,t}(N,\mathbb{Z}_3)\cong \mm{Tor}^{\mathcal{A}}_{s,t}(\mathbb{Z}_3,N)$$
$$N=H^\ast(\mm{MO} \langle  8 \rangle \wedge \mm{M}\eta;\mathbb{Z}_3)\cong H^\ast(\mm{MO} \langle  8 \rangle;\mathbb{Z}_3)\otimes  H^\ast(\mm{M}\eta;\mathbb{Z}_3)$$
$$N_n= H^\ast(\mm{MO} \langle  8 \rangle;\mathbb{Z}_3)\otimes \oplus_{i\ge n}H^i(\mm{M}\eta;\mathbb{Z}_3)$$
$$N^n=H^\ast(\mm{MO} \langle  8 \rangle;\mathbb{Z}_3)\otimes H^n(\mm{M}\eta;\mathbb{Z}_3)$$
with
$E_1^{s,m,n}=\mm{Tor}^{\mathcal{A}}_{s,s+m}(\mathbb{Z}_3, H^\ast(\mm{MO} \langle  8 \rangle;\mathbb{Z}_3))\otimes  H^n(\mm{M}\eta;\mathbb{Z}_3)$ where $\mathcal{A}$ is the mod $3$ Steenrod algebra.
 At first, we determine the $$\mm{Ext}_{\mathcal{A}}^{s,s+m}(H^\ast(\mm{MO} \langle  8 \rangle;\mathbb{Z}_3),\mathbb{Z}_3).$$

From \cite{Aikawa} and Vanishing Theorem \cite{Ravenel1986}, the nontrivial $\mm{Ext}_{\mathcal{A}}^{s,t}(\mb{Z}_3,\mathbb{Z}_3)$ is as follows for $0<t-s<14$.
$$a^s_0\in \mm{Ext}_{\mathcal{A}}^{s,s}(\mathbb{Z}_3,\mathbb{Z}_3)$$
\begin{center}
\resizebox{0.9\textwidth}{!}{
\begin{tabular}{|c|c|c|c|c|c|c|c|c|c|c|c|c|c|c|c|}
\hline
$s$ & $1$ & $1$ &$2$ &$2$&$2$&$3$&$3$&$3$\\
\hline
$t-s$ & $3$ & $11$ &$7$ &$10$&$11$&$10$&$11$&$13$\\
\hline
$\mm{Ext}$ & $h_{1,0}$ &$h_{1,1}$&$\Pi_0h_0$&$b_{0}$ &$a_0h_{1,1}$&$a_0b_{0}$&$a^2_0h_{1,1}$&$b_0h_{1,0}$\\
\hline
\end{tabular}
}
\end{center}
where $a_0$ is the dual of Bockstein $\beta$, $h_{1,0}$ is the dual of $\mathcal{P}^1\in \mathcal{A}$, $h_{1,1}$ is the dual of $\mathcal{P}^3$.
From \cite{Giambalvo}, $H^l(\mm{MO} \langle  8 \rangle; \mathbb{Z}_3)$ ($0< l \le 13$) is nontrivial if $l=8$ and $12$, with generators $m_8$ and $m_{12}$.
From \cite{HoRa1995} and AAHSS, the nontrivial $\mm{Ext}_{\mathcal{A}}^{s,s+l}(H^\ast(\mm{MO} \langle  8 \rangle;\mathbb{Z}_3),\mb{Z}_3)$ are as the following for $0\le l\le 14$
\begin{center}
\resizebox{1.0\textwidth}{!}{
\begin{tabular}{|c|c|c|c|c|c|c|c|c|c|c|c|c|c|c|c|}
\hline
$s$ &$s\ge 0$ & $1$ & $1$ &$2$ &$2$& $3$&$s\ge3$&$3$\\
\hline
$t-s$ &$8$ & $3$ & $11$ &$7$ &$10$&$10$&$12$ &$13$\\
\hline
Ext &$a_0^sm_8$ & $h_{1,0}$ &$h_{1,0}m_8$&$\Pi_0h_0$&$b_{0}$ & $a_0b_{0}=h_{1,0}\Pi_0h_0$&$a_0^sm_{12}$&$b_0h_{1,0}$\\
\hline
\end{tabular}
}
\end{center}

Besides, $d_2(m_8)=\Pi_0h_0U$ in the ASS for $\mm{MO} \langle  8 \rangle $.
The details of the computation of AAHSS for $\mm{Tor}^{\mathcal{A}}_{s,t}(\mathbb{Z}_3,N)$ are showed in appendix \ref{AAHSSmod3}. Here we give the final result (see Fig.\ref{Fig.4}). Next we compute some Adams differentials.

% Figure environment removed

\begin{lemma}
$d_2(h_{1,0}m_8xU)=h_{1,0}\Pi_0h_0xU=a_0b_0xU$ in the $\mm{ASS}$ for $\pi_\ast(\mm{M\mathscr{B}})$.
\end{lemma}
\begin{proof}
Let $x$ be a generator of $H^2(\mm{M\eta};\mb{Z}_3)$. Since
$$\mm{Ext}_{\mathcal{A}}^{0,2}(\mm{M}\eta,\mathbb{Z}_3)=\{xU\}\quad \mm{Ext}^{s,s+1}_{\mathcal{A}}(\mm{M}\eta,\mathbb{Z}_3)=0\quad s\ge 0,$$
$\{xU\}\in {_3\pi}_2(\mm{M}\eta)$. From the product of ASS,
$$h_{1,0}m_8\circ xU=h_{1,0}m_8xU\quad h_{1,0}\Pi_0h_0U\circ xU=h_{1,0}\Pi_0h_0xU$$
$d_2(h_{1,0}m_8xU)=h_{1,0}\Pi_0h_0xU=a_0b_0xU$ by $d_2(h_{1,0}m_8)=h_{1,0}\Pi_0h_0$.
\end{proof}

From Fig \ref{Fig.4}, $h_{1,0}\beta z_9U$ survives in the ASS.
  By
the image of the orientable class of Bazaikin space under the map $\bar{\mathscr{N}}_{\mathcal{B}_{\bold q}}$ (Proposition \ref{image}), $z_9x^2U$ survives to $E_\infty$ in the ASS. Then we determine the filtrations in the AHSS for some elements of ${_3\pi}_\ast(\mm{M\mathscr{B}})$.

\begin{proposition}\label{013}
If $k=0$ mod $3$, $b_0h_{1,0}U$ survives in the ASS, and $\{b_0h_{1,0}U\}\in \pi_{13}(\mm{M\mathscr{B}})$ is provided by
$E_2^{0,13}=H_{0}(\mm{M\eta};\pi_{13}(\mm{MO}\langle  8 \rangle))$.

\end{proposition}
\begin{proof}
$b_0h_{1,0}U$ survives in the ASS for $\mm{MO}\langle  8 \rangle$,  represents the generator of $\pi_{13}(\mm{MO}\langle  8 \rangle)=\mathbb{Z}_3$. If $k=0\mod 3$, $0\ne b_0h_{1,0}U\in \mm{Ext}_{\mathcal{A}}^{s,t}(N,\mb{Z}_3)$ by Corollary \ref{313}. From Fig.\ref{Fig.4}, $b_0h_{1,0}U$ survives to $E_\infty$ in the ASS for    $_3\pi_\ast(\mm{M\mathscr{B}})$. Then the map $\iota: \mm{MO}\langle  8 \rangle\wedge S^0\to \mm{MO}\langle  8 \rangle\wedge \mm{M}\eta$ induces the monomorphism
$\pi_{13}(\mm{MO}\langle  8 \rangle)\to \pi_{13}(\mm{MO}\langle  8 \rangle\wedge \mm{M}\eta).$
So the proposition follows by the naturality of AHSS for the map $\iota$.
\end{proof}
 Note that $b_0h_{1,0}U=0\in \mm{Ext}_{\mathcal{A}}^{s,t}(N,\mb{Z}_3)$ when $k\ne 0\mod 3$.
\begin{proposition}\label{topmod3}
 $\{z_9x^2U\}$ is provided by $E_2^{13,0}=H_{13}(\mm{M\eta};\pi_{0}(\mm{MO}\langle  8 \rangle))$. Moreover,
	 $\{h_{1,0}\beta z_9U\}$ is provided by
	$E_2^{10,3}=H_{10}(\mm{M\eta};\pi_{3}(\mm{MO}\langle  8 \rangle)).$
\end{proposition}
\begin{proof}
	 $E_2^{p,13-p}$ has $3$-torsion only for $p=13,10,0$. By proposition \ref{image}, $\{z_9x^2U\}$
	 is provided by $E_2^{13,0}$. Due to Proposition \ref{013}, $\{h_{1,0}\beta z_9U\}$ is provided by
	$E_2^{10,3}$.
\end{proof}

\begin{proposition}\label{b0x}
	$b_0xU$ survives to $E_\infty$ in the ASS for $_3\pi_\ast(\mm{M\mathscr{B}})$, and $\{b_0xU\}$ is provided by $E_2^{2,10}=H_2(\mm{M\eta};\pi_{10}(\mm{MO}\langle 8\rangle))$ in the AHSS.
\end{proposition}
\begin{proof}
	By AAHSS, $\mm{Ext}_{\mathcal{A}}^{s,s+11}(N,\mb{Z}_3)=0$ for $s\ge 4$. $b_0xU\in E_\infty$ in the ASS follows by $d_2(z_9x^2U)=0$. Then the desired result follows by the fact that $\{b_0xU\}\in \pi_{12}(\mm{MO}\langle 8\rangle \wedge \mm{M}\eta|_{\mm{CP}^1})$ is provided by $E_2^{2,10}(|_{\mm{CP}^1})$.
\end{proof}


\section{Characteristic classes of Bazaikin manifolds}\label{Characteristic}

Recall that
$\mathcal{B}_{\bold{q}}=S^1\backslash \mm{SU(6)/Sp(3)},$
 $P=\mm{SU(6)/Sp(3)}$. $A[\mm{Sp}(3)]$ represents an element in $P$ where $A\in \mm{SU}(6)$. %$H^\ast(P;\mathbb{Z})\cong H^\ast(S^5\times S^9;\mathbb{Z})$.
The left $S^1$-action on $P$ is given by
$$S^1 \curvearrowright P: z\circ A[\mm{Sp}(3)]=\{diag(z^{q_0},z^{q_1},\cdots,z^{q_5})A\}[\mm{Sp}(3)],$$
where ${q_0}+{q_1}+\cdots +{q_5}=0$, $z\in S^{1}$. Let $\bold{q}=({q_0},{q_1},\cdots ,{q_5})$. Note that under a reordering of the integers $q_i$, or replacing $\bold{q}$ with $-\bold{q}$, one obtains the same manifold. Thus we can take the appropriate $\bold{q}$ so that the elementary symmetric polynomial $\sigma_3(\bold{q})>0.$

Consider the bundle
$$P\to S^\infty\times_{S^1}P \to \mathrm{CP}^\infty,$$
denoted by $\Theta_\bold{q}(P,S^1)$, associated with the universal principal $S^1$-bundle
$$S^1\to S^\infty \to \mathrm{CP}^\infty.$$
Note that $S^\infty\times_{S^1} P\simeq \mathcal{B}_{\bold{q}}$.

The action of  $S^1\subset \mm{SU}(6)$ on $P$ is a sub-action of the transitive $\mm{SU}(6)$ action
$$\mm{SU}(6)\curvearrowright P: A\circ B[\mm{Sp}(3)]=AB[\mm{Sp}(3)]$$
where $A,B\in \mm{SU}(6)$, $AB$ is the product in $\mm{SU}(6).$ Therefore $\Theta_\bold{q}(P,S^1)$ can be regarded as the fibre bundle with fiber $P$ and structure group $\mm{SU}(6)$ associated
with the unique principal $\mm{SU}(6)$-bundle ${\Theta}_\bold{q}$ as below
$$\mm{SU}(6)\to E_\bold{q} \to \mathrm{CP}^\infty$$ i.e.,   $S^{\infty}\times_{S^1} P\cong E_\bold{q}\times_{\mm{SU}(6)} P.$

\begin{lemma}
$E_\bold{q}= S^{\infty}\times_{S^1} \mm{SU}(6)$ where the $S^1$ acts on $S^{\infty}$ by  the right Hopf action, on $\mm{SU}(6)$ by the left $S^1\curvearrowright \mm{SU}(6)$ multiplication.
\end{lemma}
\begin{proof}
There is a right $\mm{SU}(6)$ action on $S^{\infty}\times_{S^1} \mm{SU}(6)$ induced by the right $\mm{SU}(6)$ action on $S^{\infty}\times \mm{SU}(6)$
$$S^{\infty}\times \mm{SU}(6)\curvearrowleft \mm{SU}(6): (x,A)\circ B=(x,AB)$$
for $x\in S^{\infty}$, $A,B\in \mm{SU}(6).$

Given an element $\{(x,A)\}\in S^{\infty}\times_{S^1} \mm{SU}(6)$, if $\{(x,A)\}=\{(x,A)\}\circ B=\{(x,AB)\}$ for some  $B\in \mm{SU}(6)$, then there exists an $h\in S^1$ such that
$$(xh^{-1},hA)=(x,AB)\in S^\infty \times \mm{SU}(6).$$
Since the action of $S^1$ is free on $S^\infty$, we get that $h=1$, so $A=AB$ meaning $B=1$ the unit. Hence the $\mm{SU}(6)$-action is free and we get a principal $\mm{SU}(6)$-bundle
$$\mm{SU}(6)\to S^{\infty}\times_{S^1} \mm{SU}(6) \to \mathrm{CP}^\infty .$$


Next we define two maps
$$\phi: (S^{\infty}\times_{S^1} \mm{SU}(6))\times_{\mm{SU}(6)} P\to S^{\infty}\times_{S^1} P$$
$$ \phi(\{\{x,A\},B[\mm{Sp}(3)]\})=\{x,AB[\mm{Sp}(3)]\}$$
$$\psi: S^{\infty}\times_{S^1} P\to (S^{\infty}\times_{S^1} \mm{SU}(6))\times_{\mm{SU}(6)} P$$
$$ \psi(\{x,A[\mm{Sp}(3)]\})=\{\{x,1\},A[\mm{Sp}(3)]\}.$$
It is easy to see that $\phi$ and $\psi$ are continuous, $\phi \circ \psi=id$, $\psi \circ \phi=id.$ The desired result follows.
\end{proof}

%In fact, $\phi: (S^{\infty}\times_{S^1} \mm{SU}(6))\times_{\mm{SU}(6)} P\to S^{\infty}\times_{S^1} P$ is a bundle map. So the fibre bundle $\Theta_q(P,S^1)$ is determined by the principal $\mm{SU}(6)$-bundle $\Theta_q$ over $\mathrm{CP}^\infty$ with the classifying map $\kappa_q:\mathrm{CP}^\infty\to \mm{BSU}(6)$.

Let $S^1$ be a subgroup of $\mm{SU}(6)$ given  by the embedding
$$z\in S^1\to diag(z^{q_0},z^{q_1},z^{q_2},z^{q_3},z^{q_4},z^{q_5}),\quad \Sigma_{0\le i\le 5} q_i=0$$
there is a fibration sequence
$$S^1\to \mm{SU}(6)\to S^1 \setminus \mm{SU}(6)\to \mm{BS^1}=\mathrm{CP}^\infty\stackrel{\kappa_\bold{q}}{\longrightarrow} \mm{BSU}(6)$$
where the map $\kappa_\bold{q}$ induced by the embedding $S^1\hookrightarrow \mm{SU}(6)$ is the classifying map of the principal $\mm{SU}(6)$-bundle $\Theta_\bold{q}$. From classical theory of Lie group, the Borel's Theorem is useful for the computation of the homomorphism
$$\kappa_\bold{q}^\ast:H^\ast(\mm{BSU}(6))=\mathbb{Z}[c_2,\cdots , c_6]\to H^\ast(\mm{CP}^\infty).$$

Let $G$ be a Lie group; and $T^n \subset G$ be a maximal torus in $G$ with induced map $\kappa: \mm{BT}^n\to \mm{BG}$. Let $I_G$ be the ring of polynomials in $H^\ast(\mm{BT}^n)$ invariant under the Weyl group $W(G)$.
\begin{theorem}
[Borel's Theorem \cite{Borel1953}] If $H^\ast(G)$ and $H^\ast(G/T^n)$ are torsion-free, then $\kappa^\ast : H^\ast(\mm{BG}) \to H^\ast(\mm{BT}^n)$ is a monomorphism with range $I_G$.
\end{theorem}
As was shown in \cite{Borel1953}, the conditions of the theorem are satisfied for all classical groups. Let $G=\mm{SU}(6)$, the standard maximal torus is
$$T^5=\{diag(e^{i\theta_0},\cdots ,e^{i\theta_5})\quad |\quad \theta_0+\cdots +\theta_5=0 \mod (2\pi)\}.$$
The homomorphism
$$H^\ast(\mm{BSU}(6))\to H^\ast(\mm{BT}^5)$$
maps $c_i$ to the elementary symmetric polynomial $\sigma_i(x_0,\cdots,x_5)$ of degree $i$, $x_j\in H^2(\mm{BT}^5)$ is a generator for $0\le j\le 5.$
Since the embedding $S^1\hookrightarrow \mm{SU}(6)$ is decomposed by
\[
\xymatrix{
 S^1\ar[r]^-{}  & T^5\ar[r]^-{} & \mm{SU}(6),
}
\]
 we have
$ \kappa^\ast_\bold{q}(c_i)=\sigma_i(q_0,\cdots,q_5)x^i$
for $2\le i\le 5$, where $x$ is the generator of $H^2(\mathrm{CP}^\infty)$, i.e.,  the Chern class $C_i(\Theta_\bold{q})$ of the principle bundle $\Theta_\bold{q}$ is
$$C_i(\Theta_\bold{q})=\sigma_i(q_0,\cdots,q_5)x^i=\sigma_i(\bold{q})x^i.$$

Now we can give the final construction and present a crucial proposition.
Let ${\Theta}^3_\bold{q}(P,S^1)$ be the restriction of the bundle ${\Theta}_\bold{q}(P,S^1)$ on $\mm{CP}^3\subset \mm{CP}^\infty  ,$
$${\Theta}^3_\bold{q}(P,S^1):P\to X_\bold{q}=S^7\times_{S^1}P \to \mathrm{CP}^3$$
which can be regarded as the associated bundle of the unique principal $\mm{SU}(6)$-bundle ${\Theta}^3_\bold{q}$
$${\Theta}_\bold{q}|_{\mm{CP}^3}={\Theta}^3_\bold{q}:\mm{SU}(6)\to S^7\times_{S^1} \mm{SU}(6) \to \mathrm{CP}^3.$$

There is  a commutative diagram of fibrations

\[
\xymatrix{
 P\ar[r]^-{} \ar[d]^-{} & P\ar[d]^-{} \\
X_\bold{q}=S^7\times_{S^1}P\ar[r]^-{j} \ar[d]^-{}&S^\infty \times_{S^1}P\simeq  \mathcal{B}_{\bold{q}} \ar[d]^-{}\\
\mathrm{CP}^3\ar[r]^-{} & \mathrm{CP}^\infty
}
\]
Similarly, we have the principal bundle ${\Theta}^3_{\bold{q}^\prime}$ for another Bazaikin space $\mathcal{B}_{\bold{q}^\prime}$.
\begin{proposition}\label{Xq}
If $\sigma_2(\bold{q})=\sigma_2(\bold{q}^\prime)$ and $\sigma_3(\bold{q})=\sigma_3(\bold{q}^\prime)$, then ${\Theta}^3_\bold{q}\cong {\Theta}^3_{\bold{q}^\prime}$.
\end{proposition}
\begin{proof}
	  The Chern character defines a ring homomorphism \cite{AtiyahTodd1960}
$$ch: K(X) \to H^\ast(X;\Bbb Q),$$
$\Bbb Q$ denotes the rational field, $K(X)$ is complex $K$-group. Its image will be denoted by $ch(X)$, and if $H^\ast(X)$ has no torsion, $ch$ is a monomorphism. Hence the $\mm{SU}(6)$-bundle ${\Theta}^3_\bold{q}$ is determined by the second and third Chern classes, i.e. the elementary symmetric polynomials $\sigma_2(\bold{q})$ and $\sigma_3(\bold{q}).$
\end{proof}

It is easy to compute the singular cohomology of $X_\bold{q}$ by the Serre spectral sequence whose differentials can be induced by the Serre spectral sequence for the fibration $P\to \mathcal{B}_{\bold{q}} \to \mathrm{CP}^\infty$. %we have $d_6(p_5)=sx^3$ thus $d_6(x^2p_5)=sx^5$, $E_{10}^{10,0}\cong E_7^{10,0}=\mathbb{Z}_s$, $d_{10}(p_9)$ is the generator of $E_{10}^{10,0}$. The homomorphism $H^9(\mathcal{B}_{\bold{q}})\to H^9(P)$ is the $\times s.$
Here we give the needed results.
%By the naturality, for the fibration $P\to X_q \to \mathrm{CP}^3$, $j$ is the $7$-equivalence, besides $E_2^{2,5}=E_\infty^{2,5}$, $E_2^{0,9}=E_\infty^{0,9}$, $E_2^{4,5}=E_\infty^{4,5}$, $E_2^{2,9}=E_\infty^{2,9}$, $E_2^{6,5}=E_\infty^{6,5}$, $E_2^{4,9}=E_\infty^{4,9}$.
\begin{center}
\resizebox{1.0\textwidth}{!}{
\begin{tabular}{|c|c|c|c|c|c|c|c|c|c|c|c|c|c|c|c|}
\hline
$n$ & 2 & 4 & 6 &7& 9& 11&13  \\
\hline
 $H^n(X_\bold{q})$ &$\mathbb{Z}\langle x\rangle$ & $\mathbb{Z}\langle x^2\rangle$ & $\mathbb{Z}_s\langle x^3\rangle$ & $\mathbb{Z}\langle x_7\rangle$ & $\mathbb{Z}\langle x_7x\rangle$, $\mathbb{Z}\langle x_9\rangle$&$\mathbb{Z}\langle x_7x^2\rangle$, $\mathbb{Z}\langle x_9x\rangle$& $\mathbb{Z}\langle x_9x^2\rangle$\\
\hline
\end{tabular}
}
\end{center}

Let $b_i\in H^i(\mathcal{B}_{\bold{q}})$ be the generator such that
\begin{equation}
j^\ast(b_9)=sx_9. \label{j-co}
\end{equation}
 Let $x_{i}\in H^i(X_\bold{q})$ also denote the element in $H^i(X_\bold{q};\mathbb{Z}_2)$. From the theory about the Steenrod squares in Serre spectral sequence, i.e. formulas (2.25), (2.79) and theorem 2.16 in \cite{Si},
\begin{equation}
\mathrm{Sq}^2(x_7x^2)=0. \label{sq}
\end{equation}

\section{The $ko$ homology of $\mm{M}\eta$}\label{koMketa}
As is well known, $KO$ is the classical real $K$-theory. We can regard $KO$ as the spectrum of period $8$ (i.e. $KO_{q+8}=KO_q$). By [\cite{Rudyak}, \uppercase\expandafter{\romannumeral2} theorem 4.15], for the spectrum $KO$, there exists a $0$-connective covering
$q^{KO}:ko\to KO$
 such that $\pi_i(ko)=0$ for $i<0$ and $$q^{KO}_\ast:\pi_i(ko)\to \pi_i(KO)$$ is an isomorphism for $i\ge 0$. The $ko$ is also called the connective real $K$-theory.
$H^\ast(ko;\mathbb{Z}_2)$ and $H^\ast(ko;\mathbb{Z}_3)$ have been computed in \cite{Stong1963} and \cite{AdamsPriddy}.
\begin{equation}
	H^\ast(ko;\mathbb{Z}_2)=\mathscr{A}//(\mathrm{Sq}^1,\mathrm{Sq}^2) \label{ko2}
\end{equation}
\begin{equation}
	H^\ast(ko;\mathbb{Z}_3)=\mathcal{A}/(\mathcal{A}Q_0+\mathcal{A}Q_1) \label{ko3}
\end{equation}
 $Q_0=\beta$, $Q_1=\beta \mathcal{P}^1- \mathcal{P}^1\beta$. In the following we will give some necessary  properties on $H_\ast(ko;\mb{Z})$ .
 \begin{proposition}\label{hko}	
	\item (1) $H_{n}(ko)$ has a torsion free abelian subgroup only when $n=0\mod 4$.
	\item (2) $H_{n}(ko)=0$ for $1\le n \le 3$ and $n=5$.
	\item (3) $H_{n}(ko)$ is a finite group of order $2^m$, $m\ge 0$ when $n=6,7$.
	\item (4) $H_{4}(ko)=\mathbb{Z}$ and the morphism $\iota_\ast:\pi_4(ko)\to H_4(ko)$ is $\times 24$.	
\end{proposition}
\begin{proof}
Proof of (1): If $\mathbb{Z}\subset H_n(ko)$, then $\mathbb{Z}\subset \pi_n(ko)$ by the AHSS with
\begin{equation}
	E_2^{p,q}=H_p(ko;\pi_q(S^0)). \label{AHSSko}
\end{equation}
From \cite{Bott1959}, $\pi_n(ko)$ has a  free abelian subgroup if and only if $n=0\mod 4$.

Proof of (2): The unit $S^0\to ko$ induces isomorphisms $\pi_i(S^0)\cong \pi_i(ko)$ for $i=0,1,2$, so $E_\infty^{0,1}=E_2^{0,1}$ and $ E_\infty^{0,2}=E_2^{0,2}$ in the AHSS $($\ref{AHSSko}$)$ follows by the naturality of AHSS. By the differentials of AHSS $($\ref{AHSSko}$)$, $H_{n}(ko)=0$ when $1\le n \le 3$ or $ n=5$.

Proof of (3): If $H_{n}(ko)$ has odd-torsion for $n=6,7$, then $\pi_n(ko)$ is nontrivial by the differentials of AHSS for $E_r^{n,0}$ $r\ge 2$, which contradicts with $\pi_6(ko)=\pi_7(ko)=0.$

Proof of (4): By equations \ref{ko2} and \ref{ko3},
$$H^4(ko;\mathbb{Z}_2)=\mathbb{Z}_2\langle \mathrm{Sq}^4 \rangle \quad H^4(ko;\mathbb{Z}_3)=\mathbb{Z}_3\langle \mathcal{P}^1 \rangle.$$
Assume $H_4(ko)=\mb{Z}\oplus \mb{Z}_u$ where $(u,2)=(u,3)=1$, then $\pi_4(ko)=\mb{Z}\oplus \mb{Z}_u$ by the differentials of AHSS $($\ref{AHSSko}$)$ for $E_r^{4,0}$ $r\ge 2$. Since $\pi_4(ko)=\mb{Z}$ and $\pi_3(ko)=0$, we have $u=1$ and
 the differential   $d_4:E_4^{4,0}=\mb{Z}\to E_4^{0,3}=\mb{Z}_{24}$ is an epimorphism. The desired result follows by Lemma \ref{chi}.
\end{proof}

From the Bott periodicity, $\pi_\ast(ko)$ is well-known. But for the convenience of later discussion, we explain $\pi_\ast(ko)$ through
the ASS with $E_2$-term
$$Ext_\mathscr{A}^{s,t}(H^\ast(ko;\mb{Z}_2),\mb{Z}_2)\cong Ext_{\mathscr{A}(1)}^{s,t}(\mb{Z}_2,\mb{Z}_2)$$
where ${\mathscr{A}(1)}$ is generated by $\mm{Sq}^1$ and $\mm{Sq}^2$.
Especially
$$h_1\in Ext_{\mathscr{A}(1)}^{1,2},\quad  v\in Ext_{\mathscr{A}(1)}^{3,7},\quad \omega\in Ext_{\mathscr{A}(1)}^{4,12}$$
represent the generators of ${_2\pi}_1(ko)$, ${_2\pi}_4(ko)$ and ${_2\pi}_8(ko)$ respectively \cite{Ravenel1986}.

After the preparation on $ko$-theory above, we consider the composition map $\mm{\alpha F}: \mm{MO}\langle 8\rangle \to ko$ of $F : \mm{MO}\langle 8\rangle \to \mm{MSpin}$ and
$\alpha :  \mm{MSpin} \to  ko$, where $\alpha: \mm{MSpin} \to  ko$ is the Atiyah-Bott-Shapiro orientation \cite{AtiyahBottShapiro} (cf.\cite{HebeJoach} also)  which is an 8-equivalence.
 The composite $\alpha \circ F=\mm{\alpha F}$ induces a natural homomorphism of $Ext$-groups
 $$\mm{\alpha F}_\ast: Ext_{\mathscr{A}}^{s,t}(H^\ast(\mm{MO}\langle 8\rangle;\mb{Z}_2),\mb{Z}_2)\to Ext_{\mathscr{A}}^{s,t}(H^\ast(ko;\mb{Z}_2),\mb{Z}_2).$$
 By the computations of the $Ext$-groups in \cite{Giambalvo} and \cite{Ravenel1986}, we know
 \begin{equation}
 \mm{\alpha F}_\ast(h^s_1)=h^s_1  \label{h1}
\end{equation}
 \begin{equation}
 \mm{\alpha F}_\ast(h_1^s\omega)=h_1^s\omega  \label{h1o}
\end{equation}
for $0\le s\le 2$. Then we get the following lemma by comparing the ASS.
\begin{lemma}
	The homomorphism $\mm{\alpha F}_\ast :\pi_n(\mm{MO}\langle 8\rangle)\to \pi_n(ko)$ is an isomorphism for $n=0$, $1$, $2$, an epimorphism for $n=9,10$. Moreover, the summand $\mb{Z}\subset \pi_8(\mm{MO}\langle 8\rangle)$ is mapped injectively to $\mb{Z}=\pi_8(ko)$  by $\mm{\alpha F}_\ast$.
\end{lemma}







%\begin{corollary}The condition is same as lemma \ref{ahsspro}. If $x\in E_2^{s,m}$ $s>n$ survives to $E_r^{s,m}$ $r>s-n$, and $f_2(x)=\bar{x}\ne 0$,$\bar{x}\in \bar{E}_{2}^{s,m}$ survives to $\bar{E}_{\infty}^{s,m}$, then $x$ survives to ${E}_{\infty}^{s,m}$.\end{corollary}
Next
 we compute the homomorphism $(\mm{\alpha F}\wedge 1)_\ast: \pi_\ast(\mm{MO}\langle 8\rangle \wedge \mm{M}\eta)\to \pi_\ast(ko \wedge \mm{M}\eta)$ through two following AHSS
 \begin{equation}
 	E_2^{p,q}=H_p(\mm{M}\eta;\pi_q(\mm{MO}\langle 8\rangle))\stackrel{}{\Longrightarrow}\pi_\ast(\mm{MO}\langle 8\rangle\wedge \mm{M}\eta) \label{E2pq}
 \end{equation}
\begin{equation}
	E_2^{p,q}(ko)=H_p(\mm{M}\eta;\pi_q(ko))\stackrel{}{\Longrightarrow}\pi_\ast(ko\wedge \mm{M}\eta) \label{E2pqko}
\end{equation}
For the AHSS (\ref{E2pqko}), there are three formulas about its differentials showed generally in theorem \ref{dr}.
\begin{theorem}\label{dr}
Let $X$ be a connective spectrum. The $\mm{AHSS}$ for $ko_\ast(X)$ with $E_2^{p,q}(ko,X)=H_p(X;\pi_q(ko))$ satisfies that
\item [(1)] $d_2:E_2^{p+2,0}(ko,X) \to E_2^{p,1}(ko,X)$ is the reduction mod $2$ composed with the dual of $\mathrm{Sq}^2: H^p(X; \mathbb{Z}_2)\to H^{p+2}(X ; \mathbb{Z}_2).$
\item [(2)] $d_2:E_2^{p+2,1}(ko,X) \to E_2^{p,2}(ko,X)$ is the dual of
$\mathrm{Sq}^2.$

\item [(3)] $d_3:E_3^{p+3,2}(ko,X) \to E_3^{p,4}(ko,X)$ is the dual of $\mathrm{Sq}^2\mathrm{Sq}^1.$

%(4) If $\mathrm{Sq}^4\mathrm{Sq}^1(x^\ast)\ne0$, then for the differential $d_5:E^5_{p+5,4}(ko,X) \to E^5_{p,8}(ko,X)$, $d_5(z)=x$ where $z\in H_{p+5}(X;\mathbb{Z})$, $r_2(z)$ is the dual of $\mathrm{Sq}^4\mathrm{Sq}^1(x^\ast)$, $r_2$ is the mod 2 reduction.

\end{theorem}
\begin{proof}
(1) and (2) are obvious by lemma \ref{d2}. By equation \ref{ko2}, we have
$$ko_4(H(\mathbb{Z}_2))\cong H_4(ko;\mathbb{Z}_2)=\mathbb{Z}_2.$$
%by $E_n (F) =\pi_n(E\wedge F)\cong F_n(E)$.
Consider the AHSS
$$E_2^{p+n,q}(ko,\Sigma^p H(\mathbb{Z}_2))\Longrightarrow ko_{\ast}(\Sigma^p H(\mathbb{Z}_2)).$$
 By (1) and (2),
 $$E_2^{p+4,0}(ko,\Sigma^p H(\mathbb{Z}_2))\cong E_3^{p+4,0}(ko,\Sigma^p H(\mathbb{Z}_2))$$  $$E_3^{p+1,2}(ko,\Sigma^p H(\mathbb{Z}_2))=0,$$  thus $E_2^{p+4,0}(ko,\Sigma^p H(\mathbb{Z}_2))\cong E_\infty^{p+4,0}(ko,\Sigma^p H(\mathbb{Z}_2))\cong ko_{p+4}(\Sigma^p H(\mathbb{Z}_2))$.
%Due to $E_2^{p,3}(ko,\Sigma^p H(\mathbb{Z}_2))=0$, we get that
%$$\mb{Z}_2=E_2^{p+4,0}(ko,\Sigma^p H(\mathbb{Z}_2))\cong E_\infty^{p+4,0}(ko,\Sigma^p H(\mathbb{Z}_2))\cong ko_{p+4}(\Sigma^p H(\mathbb{Z}_2)).$$
So the element in $E_2^{p,4}(ko,\Sigma^p H(\mathbb{Z}_2))$ can be killed by a differential. Using (1) and (2) again, we have that  $E_3^{p+3,2}(ko,\Sigma^p H(\mathbb{Z}_2))$ has a generator $(\mathrm{Sq}^2\mathrm{Sq}^1)_\ast$ which is the dual of $\mathrm{Sq}^2\mathrm{Sq}^1$, and $E_3^{p+4,1}(ko,\Sigma^p H(\mathbb{Z}_2))=0$.
So the needed differential is $d_3: E_3^{p+3,2}(ko,\Sigma^p H(\mathbb{Z}_2))\to E_3^{p,4}(ko,\Sigma^p H(\mathbb{Z}_2))$
$$d_3(\mathrm{Sq}^2\mathrm{Sq}^1)_\ast=1.$$
Assume that  $x\in H_p(X;\mathbb{Z})$ is an element of order $2$, $\mm{Sq^2Sq^1 }x^\ast\ne 0$ where $x^\ast\in H^p(X;\mathbb{Z}_2)$ is the dual of the mod $2$ reduction $r_2(x)$  of $x$ and represents the map $x^\ast:X\to \Sigma^p H(\mathbb{Z}_2)$ inducing the homomorphism:
$$\mathscr{X}: H_i(X ; \mathbb{Z}_2)\to H_i(\Sigma^p H(\mathbb{Z}_2) ; \mathbb{Z}_2)$$
for any $y\in H_i(X ; \mathbb{Z}_2)\cong \pi_i(H(\mb{Z}_2)\wedge X)$, $\mathscr{X}(y)$ is represented by
$$S^i\stackrel{y}{\to }H(\mb{Z}_2)\wedge X\stackrel{1\wedge x^\ast}{\to }H(\mb{Z}_2)\wedge \Sigma^pH(\mb{Z}_2).$$
\[
\xymatrix{
H_i(X ; \mathbb{Z}_2)\ar[rr]^-{\mathscr{X}}\ar[d]^-{\cong}& &H_i(\Sigma^p H(\mathbb{Z}_2) ; \mathbb{Z}_2)\ar[d]^-{\cong}\\
  [S^i,X\wedge H(\mb{Z}_2)]\ar[rr]^-{x^\ast\wedge 1}   &      &[S^i,\Sigma^p H(\mathbb{Z}_2)\wedge H(\mb{Z}_2)]
}
\]
From the definition of Kronecker product \cite{Sw}
$$\langle -,-\rangle : H^i(X;\mb{Z}_2)\otimes H_i(X;\mb{Z}_2)\to \pi_{0}(H(\mb{Z}_2))=\mb{Z}_2$$$\mathscr{X}(r_2(x))=1$ for $i=p$. If $y$ is the dual of $\mathrm{Sq}^2\mathrm{Sq}^1(x^\ast)$, then $y\in E_3^{p+3,2}(ko,X)$ by (2). Next we claim that for $i=p+3$
$$\mathscr{X}(y)=(\mathrm{Sq}^2\mathrm{Sq}^1)_\ast\in H_{p+3}(\Sigma^p H(\mathbb{Z}_2) ; \mathbb{Z}_2).$$
By [\cite{Sw}, Theorem 17.5], $\mm{Sq}^2\mm{Sq}^1x^\ast\in H^{p+3}(X;\mb{Z}_2)$ is represented by the composite of $x^\ast :X\to \Sigma^pH(\mb{Z}_2)$ and $\mm{Sq}^2\mm{Sq}^1:\Sigma^pH(\mb{Z}_2)\to \Sigma^{p+3}H(\mb{Z}_2)$.
So the diagram as follows is commutative up to homotopy.
\[
\xymatrix{
 S^{p+3}\ar[r]^-{y}&X\wedge H(\mathbb{Z}_2)\ar[d]_-{\mathrm{Sq}^2\mathrm{Sq}^1x^\ast\wedge 1}\ar[r]^-{x^\ast\wedge 1} &\Sigma^p H(\mathbb{Z}_2)\wedge  H(\mathbb{Z}_2)\ar[ld]^-{\mathrm{Sq}^2\mathrm{Sq}^1 \wedge 1}\\
     &   \Sigma^{p+3} H(\mathbb{Z}_2)\wedge  H(\mathbb{Z}_2)    &
}
\]
By the definition of Kronecker product $$(\mathrm{Sq}^2\mathrm{Sq}^1x^\ast\wedge 1)\circ y=1\in\pi_0(H(\mathbb{Z}_2)\wedge H(\mb{Z}_2))=\mb{Z}_2.$$
By the definition of Kronecker product and the following equation
 $$(\mathrm{Sq}^2\mathrm{Sq}^1 \wedge 1)\circ ((x^\ast\wedge 1)\circ y)=(\mathrm{Sq}^2\mathrm{Sq}^1x^\ast\wedge 1)\circ y=1 \in\pi_0(H(\mathbb{Z}_2)\wedge H(\mb{Z}_2)),$$ we have that $(x^\ast\wedge 1)\circ y$ is the dual of $\mathrm{Sq}^2\mathrm{Sq}^1$.
So the claim follows by the definition of $\mathscr{X}$.
\[
\xymatrix{
y\in E_3^{p+3,2}(ko,X)\ar[rr]^-{d_3}\ar[d]^-{\mathscr{X}}& &x\in E_3^{p,4}(ko,X)\ar[d]^-{\mathscr{X}}\\
  (\mathrm{Sq}^2\mathrm{Sq}^1)_\ast\in E_3^{p+3,2}(ko,\Sigma^pH(\mb{Z}_2))\ar[rr]^-{d_3}   &      &1\in E_3^{p,4}(ko,\Sigma^pH(\mb{Z}_2))
  }
\]

 By the naturality of AHSS for the map $x^\ast:X\to \Sigma^p H(\mb{Z}_2)$, $d_3(y)=x$.
\end{proof}

\begin{lemma}\label{koketa94}
$ E_\infty^{9,4}(ko)\cong E_4^{9,4}(ko)\cong \mb{Z}_2\oplus \mb{Z}_3\oplus \mb{Z}_s$ in the AHSS  (\ref{E2pqko}).	
\end{lemma}
\begin{proof}
Recall the AHSS (\ref{E2pqko})
	$$E_2^{p,q}(ko)=H_p(\mm{M}\eta;\pi_q(ko))\stackrel{}{\Longrightarrow}ko_\ast (\mm{M}\eta)\cong  \pi_\ast(ko\wedge \mm{M}\eta).$$
By Proposition \ref{Z-B}, $E_2^{9,4}(ko)\cong \mb{Z}_2^2\oplus \mb{Z}_3\oplus \mb{Z}_s$.
Computing the differentials by theorem \ref{dr}, we get that $$E_4^{9,4}(ko)\cong  \mb{Z}_2\oplus \mb{Z}_3\oplus \mb{Z}_s,\quad E_3^{14,0}(ko)=0.$$ Applying  Proposition \ref{221} and Lemma \ref{d2} for the spectral sequence
$$E_2^{p,q}=H_p(\mm{M}\eta;\pi_q(\mm{MO}\langle 8\rangle))\stackrel{}{\Longrightarrow}\pi_\ast(\mm{MO}\langle 8\rangle\wedge \mm{M}\eta)$$
we get that
 $E_\infty^{13,1}=E_3^{13,1}$. Considering the morphism of AHSS $$\mm{\alpha F}:E_r^{p,q}\to E_r^{p,q}(ko),$$ we have $E_3^{13,1}\cong E_3^{13,1}(ko)$ and thus
$ E_3^{13,1}(ko)=E_\infty^{13,1}(ko).$
Note that $E_5^{4,8}(ko)=\mb{Z}$ and $E_3^{14,0}(ko)=0$. Hence $E_7^{9,4}(ko)\cong E_4^{9,4}(ko)$.

Next we compute $E_\infty^{9,4}(ko)$ by comparing spectral sequences as follows
	\[
\xymatrix@C=.3cm{
E_2^{p,q}(ko)=H_p(\mm{M}\eta;\pi_q(ko))\ar@{=>}[rr]^-{}\ar[d]^-{}& & \pi_{p+q} ( ko \wedge \mm{M}\eta)\ar[d]^-{}\\
\bar{E}_2^{p,q}(ko)=H_p(\mm{M}\xi;\pi_q(ko))\ar@{=>}[rr]^-{}& & \pi_{p+q} ( ko\wedge \mm{M}\xi)
}
\]
${E}_3^{2,10}(ko)\cong \bar{E}_3^{2,10}(ko)$ follows by ${E}_2^{4,9}(ko)\cong \bar{E}_2^{4,9}(ko)$, $ {E}_2^{2,10}(ko)\cong \bar{E}_2^{2,10}(ko)$ and $ {E}_2^{0,11}(ko)=\bar{E}_2^{0,11}(ko)=0$. By simple computations, $\bar{E}_3^{2,10}(ko)\cong \bar{E}_8^{2,10}(ko)$. By lemma \ref{ahsspro}, ${E}_8^{2,10}(ko)={E}_3^{2,10}(ko)$, thus the differential $d_7:{E}_7^{9,4}(ko)\to {E}_7^{2,10}(ko)$ is trivial. Since $E_2^{0,12}(ko)\cong \mb{Z}$, and $E_2^{13-q,q}(ko)$ is a torsion group for $q\le 11$, we have
$E_\infty^{0,12}(ko)=E_2^{0,12}(ko).$
Therefore $ E_\infty^{9,4}(ko)=E_7^{9,4}(ko)= \mb{Z}_2\oplus \mb{Z}_3\oplus \mb{Z}_s$.
\end{proof}

Now we take the cofibration sequence
\begin{equation}
	\mm{MO}\langle 8\rangle \stackrel{\mm{\alpha F}}{\rightarrow}ko\stackrel{h}{\rightarrow} W\stackrel{\partial}{\rightarrow}\Sigma \mm{MO}\langle 8\rangle. \label{MO8koWcofib}
\end{equation}
Note that  $\pi_n(W)=0$ for $n\le 3$ and the sequence
\begin{equation}
	0\to \pi_4(ko)=\mathbb{Z}\stackrel{h_{4\ast}}{\longrightarrow} \pi_4(W)\to \pi_3(\mm{MO}\langle 8\rangle)=\mathbb{Z}_{24}\to 0 \label{exactkow}
\end{equation}
is exact. By proposition \ref{hko}, $H_4(ko)=\mb{Z}$. Smashing the integral Eilenberg spectrum $H$ on the cofibration \ref{MO8koWcofib}, we have the following  cofibration $$H\wedge \mm{MO}\langle 8\rangle \stackrel{}{\rightarrow}H\wedge ko\stackrel{}{\rightarrow} H\wedge W.$$
Thus $H_4(W)\cong \mb{Z}$ follows. Then $\pi_4(W)\cong \mathbb{Z}$ follows by the Hurewicz theorem. By the exact sequence \ref{exactkow}, the homomorphism $h_{4\ast}$ is $\times 24$. Moreover,
\begin{itemize}
	\item $\pi_5(W)=\pi_6(W)=0$;
	\item $\pi_7(W)\cong \mb{Z}_2$;
	\item the images of the generators of $\pi_9(W)\cong \mb{Z}_2$, $\pi_{10}(W)\cong \mb{Z}_2$ under the homomorphisms induced by $\partial$ are $ c_0$, $h_1c_0$ respectively;
	\item $\pi_{11}(W)\cong \mb{Z}_3\to \pi_{10}(\mm{MO}\langle 8\rangle)=\mb{Z}_6$ is a monomorphism;
	\item there is an exact sequence
	\begin{equation}
	0\to \pi_{13}(W)\stackrel{\partial_\ast}{\to } \pi_{12}(\mm{MO}\langle 8\rangle)\cong \mb{Z}\to \pi_{12}(ko)\cong \mb{Z}\to \pi_{12}(W)\to 0. \label{exactWMO8}	
	\end{equation}
\end{itemize}
 \begin{lemma}\label{pFZ4}
 The element provided by $\mb{Z}_2\oplus \mb{Z}_3\subset  E_\infty^{9,4}(ko)$ is in the kernel of the homomorphism $h_\ast:ko_\ast(\mm{M}\eta)\to W_\ast(\mm{M}\eta)$.
 \end{lemma}
\begin{proof}
	There are two morphisms of AHSS as following diagram.
\[
\xymatrix@C=.3cm{
E_2^{p,q}(ko)=H_p(\mm{M}\eta;\pi_q(ko)\ar@{=>}[rr]^-{}\ar[d]^-{h_\ast}& & \pi_{p+q} ( ko \wedge \mm{M}\eta)\ar[d]^-{h_\ast}\\
{E}_2^{p,q}(W)=H_p(\mm{M}\eta;\pi_{q}(W))\ar[d]^-{\partial_\ast}\ar@{=>}[rr]^-{}& & \pi_{p+q} (W \wedge \mm{M}\eta)\ar[d]^-{\partial_\ast}\\
{E}_2^{p,q-1}=H_p(\mm{M}\eta;\pi_{q-1}(\mm{MO}\langle 8\rangle))\ar@{=>}[rr]^-{}& & \pi_{p+q-1} (\mm{MO}\langle 8\rangle \wedge \mm{M}\eta)
}
\]

We discuss ${E}_2^{p,13-p}(W)$ firstly. Note that $h_\ast:\mb{Z}_2\oplus \mb{Z}_2\oplus \mb{Z}_3\subset E_2^{9,4}(ko)\to E_2^{9,4}(W)$
is the $0$-homomorphism since $h_{4\ast}=\times 24$.
Note that $ E_2^{5,8}(W)=\mb{Z}_s\otimes \pi_8(W)$, $E_2^{2,11}(W)\cong \mb{Z}_3$, $E_2^{4,9}(W)\cong \mb{Z}_2$, $E_2^{8,5}(W)=E_2^{7,6}(W)=E_2^{6,7}(W)=0$.
Moreover,
  $ E_2^{0,13}(W)$ is either $0$ or $  \mb{Z}$ by the exact sequence \ref{exactWMO8}.
  In case of $E_2^{0,13}(W)\cong \mb{Z}$, $\partial_\ast: E_2^{0,13}(W)\to  E_2^{0,12}$ is an isomorphism. By Corollary \ref{210012}, $E_\infty^{0,12}=E_2^{0,12}$. So $E_\infty^{0,13}(W)=E_2^{0,13}(W)$ follows by Lemma \ref{ahsspro}.

Since that $E_2^{p,13-p}(ko)$ is a torsion group for each $p<9$, every element in $ko_{13}(\mm{M\eta})$ provided by $\mb{Z}_2\oplus \mb{Z}_3\subset E_\infty^{9,4}(ko)$ is a torsion element.
Assume that $\mathtt{k}\in ko_{13}(\mm{M\eta})$ is  provided by $\mb{Z}_2\subset E_\infty^{9,4}(ko)$ and $0\ne h_\ast(\mathtt{k})\in \pi_{13}(W\wedge \mm{M\eta})$. By the discussion for $E_2^{p,13-p}(W)$ as above, $h_\ast(\mathtt{k})$ is provided by $ E_\infty^{4,9}(W)$.
By proposition \ref{4948}, $\mb{Z}_2\subset {E}_2^{4,8}$ survives to ${E}_\infty^{4,8}$. Thus the  homomorphism $\partial_\ast:\pi_9(W)\to \pi_8(\mm{MO}\langle 8\rangle)$ implies that
$\partial_\ast: {E}_\infty^{4,9}(W)\to  {E}_\infty^{4,8}$
is nontrivial which contradicts with the exact sequence i.e. $\partial_\ast h_\ast(\mathtt{k})=0$.

Similarly, if the image of an element provided by $\mb{Z}_3\subset E_\infty^{9,4}(ko)$ is nontrivial under $h_\ast$, then the image is provided by $E_\infty^{2,11}(W)$ which also contradicts with the exact sequence.
\end{proof}

\section{Detection of certain bordism classes in $\Omega_{13} ^{\mm{O}\langle 8 \rangle}(\eta)$ }\label{Detect}

This section will involve many Atiyah-Hirzebruch spectral sequences for different homology theories and spectral. We firstly give a general notation as follows.

For two connective spectral $E$ and $Y$, the $E_r^{p,q}$-terms of the AHSS converging to $E_\ast(Y)\cong \pi_\ast(E\wedge Y)$ is denoted by $E_r^{p,q}(E,Y)$. In particular $E_2^{p,q}(E,Y)=H_p(Y;\pi_q(E))$.

From the Thom construction, a bordism class in $\Omega_n^{\mm{O}\langle 8 \rangle}(\eta)$ is  the stable homotopy class $\mm{M}(\bar{\mathscr{N}}_M) \circ \ell \in \pi_n(\mm{M\mathscr{B}})$ of the composite
$$S^n\stackrel{\ell }{\to}   \mm{M}(\mathscr{N}_M)\stackrel{\mm{M}(\bar{\mathscr{N}}_M)}{\to}  \mm{M\mathscr{B}}=\mm{MO}\langle 8\rangle \wedge \mm{M}\eta$$
where $\mm{M}(\mathscr{N}_M)$ denotes the Thom spectrum of the stable normal bundle $\mathscr{N}_M$ of manifold $M$, $\bar{\mathscr{N}}_M:M\to \mathscr{B}$ is the lifting of the Gauss map. The image of $\ell$ is $[M]U_\ast \in H_n(\mm{M}(\mathscr{N}_M))$ under the Hurewicz homomorphism where $[M]$ is the orientable class of $M$, $U_\ast$ is the dual of the stable Thom class of $\mathscr{N}_M$.


For a Bazaikin manifold  $\mathcal{B}_\bold{q}$, recall that the stable normal bundle $\mathscr{N}_{\mathcal{B}_\bold{q}} \cong \nu^{\ast} \gamma_8\oplus f^\ast\eta$. For the sake of simplicity, let $\psi$ and $\varphi$ denote $\nu^{\ast} \gamma_8$ and $f^\ast\eta$ respectively.  Note that $w_2(\varphi)\ne 0$ and $w_4(\varphi)=0$. We also use $U_\ast$ to denote the dual of the stable Thom class of $\varphi$ for convenience.

By definition we know that
$\mm{M}(\bar{\mathscr{N}}_{\mathcal{B}_{\bold q}})$    is the composite of the diagonal
$\mm{M}(\vartriangle) : \mm{M}(\mathscr{N}_{\mathcal{B}_\bold{q}}) \to \mm{M\psi } \wedge \mm{M\varphi }$ and $\mm{M}{\nu}   \wedge \mm{M}f : \mm{M\psi } \wedge \mm{M\varphi } \to \mm{MO} \langle 8\rangle\wedge \mm{M}\eta$.
By the decomposition of the map ${\nu}\times f$
$$\mathcal{B}_{\bold{q}}\times \mathcal{B}_{\bold{q}}\stackrel{{\nu}\times 1}{\longrightarrow}\mm{BO}\langle 8 \rangle \times \mathcal{B}_{\bold{q}}\stackrel{1\times f}{\longrightarrow} \mm{BO}\langle 8 \rangle \times B$$
we have the decomposition of $\mm{M}({\nu}   )\wedge \mm{M}(f)$ as follows
$$\mm{M\psi } \wedge \mm{M\varphi }\stackrel{\mm{M}{\nu}   \wedge 1}{\longrightarrow} \mm{MO\langle 8\rangle\wedge \mm{M\varphi }}\stackrel{1\wedge \mm{M}f}{\longrightarrow}\mm{MO\langle 8\rangle \wedge M\eta}.$$

Next we consider the  homomorphism $(1\wedge \mm{M}f)_\ast$ of homotopy groups. To start we need some preliminary computations for $\pi_\ast(\mm{MO}\langle 8\rangle\wedge \mm{M}\varphi )$.
 \begin{lemma} \label{d2S0}
Assume that the unit $\iota: S^0\to E$ induces the isomorphisms $\pi_i(S^0)\cong \pi_i(E)$ for $ i\le 3$, $X$ is a connective spectrum with $H_n(X;\mb{Z})=\mb{Z}$ for some $n\ge 0$. Then in the $\mm{AHSS}$,
 $d_2:E_2^{n+2,2}(E,X) \to E_2^{n,3}(E,X)$ is nontrivial if and only if
$\mathrm{Sq}^2: H^n(X; \mathbb{Z}_2)\to H^{n+2}(X ; \mathbb{Z}_2)$ is nontrivial.

\end{lemma}
\begin{proof}
	We firstly consider the AHSS for the integral Eilenberg spectrum $H$.
	 $$E_2^{n+p,q}(S^0,\Sigma^n H)\Longrightarrow\pi_{n+p+q}(\Sigma^n H) $$
Note that $H_5(H;\mb{Z})=0,$ $H^4(H;\mb{Z}_2)=\mb{Z}_2\langle \mm{Sq}^4 \rangle$, $H^2(H;\mb{Z}_2)=\mb{Z}_2\langle \mm{Sq}^2 \rangle$. By $\pi_4(H)=0$ and Theorem \ref{d2},
$$d_2:E_2^{n+2,2}(S^0,\Sigma^n H)=\mb{Z}_2 \to E_2^{n,3}(S^0,\Sigma^n H)=\mb{Z}_{24}$$
is nontrivial. There is a AHSS for $E_{p+n+q}(\Sigma^nH)$
$$E_2^{p+n,q}(E,\Sigma^n H)=H_{p+n}(\Sigma^nH;\pi_q(E))\Longrightarrow E_{p+n+q}(\Sigma^nH).$$
By the hypothesis and the naturality of AHSS for $\iota:S^0\to E$,
$$d_2:E_2^{n+2,2}(E,\Sigma^n H)=\mb{Z}_2 \to E_2^{n,3}(E,\Sigma^n H)=\mb{Z}_{24}$$
is also nontrivial.
Let $x^\ast:X\to \Sigma^nH$ be the map representing the generator $x^\ast\in \mb{Z}\subset H^n(X;\mb{Z})$, we have the following commutative diagram
\[
\xymatrix@C=.2cm{
 E_2^{n+2,2}(E,X)\ar[rr]^-{d_2}\ar[d]^-{x^\ast}&  & E_2^{n,3}(E,X)\ar[d]^-{x^\ast}\\
E_2^{n+2,2}(E,\Sigma^n H)\ar[rr]^-{d_2}& &E_2^{n,3}(E,\Sigma^n H)
}
\]
Then the lemma follows by the analogous approach in theorem \ref{dr}.
\end{proof}
\begin{corollary}
	
	$ {E}_3^{11,2}(\mm{MO}\langle 8\rangle,\mm{M}\varphi)=0 $
\end{corollary}
\begin{proof}
	By equation \ref{SqBq}, $\mm{Sq}^2(b_9U)=xb_9U$. Then the desired result follows by Theorem \ref{d2S0}.
	\end{proof}

Indeed, ${E}_\infty^{13-q,q}(\mm{MO}\langle 8\rangle,\mm{M}\varphi)=0$ for $1\le q\le 3$.
Next, after recalling some ${E}_\infty^{p,q}(\mm{MO}\langle 8\rangle,\mm{M}\eta)$-terms we compute ${E}_\infty^{13,0}(\mm{MO}\langle 8\rangle,\mm{M}\varphi)$.

\begin{lemma}\label{keta12}
${E}_\infty^{p,12-p}(\mm{MO}\langle 8\rangle,\mm{M}\eta)={E}_2^{p,12-p}(\mm{MO}\langle 8\rangle,\mm{M}\eta)$ for $p=0,2,4.$
\end{lemma}
\begin{proof}
The desired results are summarized directly by Corollaries \ref{210012}, \ref{CP2} and Propositions \ref{4948}, \ref{b0x}.	
\end{proof}

\begin{lemma}
 ${E}_\infty^{13,0}(\mm{MO}\langle 8\rangle,\mm{M}\varphi)={E}_2^{13,0}(\mm{MO}\langle 8\rangle,\mm{M}\varphi)$.
\end{lemma}
\begin{proof}
From the decomposition of $\mm{M}(\bar{\mathscr{N}}_{\mathcal{B}_\bold{q}})$, we have
$$H_{13}(\mm{M}(\mathscr{N}(\mathcal{B}_\bold{q})))\stackrel{\mm{M}(\vartriangle)}{\longrightarrow} H_{13} ( \mm{M\psi\wedge M\varphi})\stackrel{\mm{M}(\nu)\wedge \mm{M}(f)}{\longrightarrow} H_{13}(\mm{MO\langle 8\rangle \wedge M\eta}).$$
By the Thom isomorphism and the diagonal map for prime number $p$
$$\vartriangle_\ast:H_{13}(\mathcal{B}_\bold{q};\mb{Z}_p)\to \sum_{0\le n\le 13} H_{n}(\mathcal{B}_\bold{q};\mb{Z}_p)\otimes H_{13-n}(\mathcal{B}_\bold{q};\mb{Z}_p), $$
we have $\mm{M}(\vartriangle)_\ast([\mathcal{B}_\bold{q}]U_{\ast})=1\wedge [\mathcal{B}_\bold{q}] U_{\ast 2}\oplus \cdots $ where $\wedge$ is the product
$$\wedge:H_0(\mm{M}\psi)\otimes H_{13}(\mm{M}\varphi)\to H_{13}(\mm{M}\psi \wedge \mm{M}\varphi).$$
 By proposition \ref{image} i.e.
	\begin{equation}
\mm{M}(\bar{\mathscr{N}}_{\mathcal{B}_\bold{q}})_\ast([\mathcal{B}_\bold{q}]U_{\ast})=1\wedge(vx^2)_\ast U_\ast\oplus 1\wedge(z_9x^2)_\ast U_\ast\oplus\cdots,
\label{Hnu}
	\end{equation}
  we claim that the bordism class $\mm{M}(\bar{\mathscr{N}}_{\mathcal{B}_\bold{q}})\circ \ell$ has a nontrivial component in $\mb{Z}_2\oplus \mb{Z}_3$ provided by ${E}_2^{13,0}(\mm{MO}\langle 8\rangle,\mm{M}\eta)$.

Assume that there are nontrivial differentials $d_r:\bar{E}_r^{13,0}\to  \bar{E}_r^{13-r,r-1}$ for $2\le r\le 4$. Since $\pi_1(\mm{MO}\langle 8\rangle)=\mathbb{Z}_2$, $\pi_2(\mm{MO}\langle 8\rangle)=\mathbb{Z}_2$ and $\pi_3(\mm{MO}\langle 8\rangle)=\mathbb{Z}_{24},$
 we get that $n[\mathcal{B}_\bold{q}]U_{\ast 2}$ is a generator of ${E}_5^{13,0}(\mm{MO}\langle 8\rangle,\mm{M}\varphi)$ where $n=0\mod 2$ or $3$, which contradicts with the claim. So ${E}_5^{13,0}(\mm{MO}\langle 8\rangle,\mm{M}\varphi)={E}_2^{13,0}(\mm{MO}\langle 8\rangle,\mm{M}\varphi)$.
	By Lemma \ref{ahsspro} and \ref{keta12}, $${E}_\infty^{p,12-p}(\mm{MO}\langle 8\rangle,\mm{M}\varphi)={E}_2^{p,12-p}(\mm{MO}\langle 8\rangle,\mm{M}\varphi)$$ for $p=0,2,4.$
 Hence the desired result follows.
\end{proof}

Now we introduce a definition in order to describe certain direct summands of $ \Omega_{13}^{\mm{O}\langle 8 \rangle}(\eta) $ and give characteristic classes for these summands.
\begin{definition}\label{topandleft}
A direct summand  $T\subset T_2$ or $T_3$ is called {\it top} if it is provided by ${E}_2^{p,13-p}(\mm{MO}\langle 8\rangle,\mm{M}\eta)$ with  $p\ge 10$. In this fashion every summand $\mb{Z}_5\subset \Omega_{13}^{\mm{O}\langle 8 \rangle}(\eta)$ is always {\it top}.
\end{definition}

It is easy to know that both $\pi_{13}(\mm{MO\langle 8\rangle \wedge M\varphi})$ and $H_{13}(\mm{MO\langle 8\rangle \wedge M\varphi})$ have only one free subgroup.
Let $\lambda\in \pi_{13}(\mm{MO\langle 8\rangle \wedge {M\varphi}})$
be a generator provided by $[\mathcal{B}_\bold{q}]U_{\ast}\in {E}_2^{13,0}(\mm{MO}\langle 8\rangle,\mm{M}\varphi)$, $1\wedge [\mathcal{B}_\bold{q}]U_{\ast }$ denote the free generator of $H_{13}(\mm{MO\langle 8\rangle \wedge M\varphi})$ arising from the product
$$\wedge:H_{0}(\mm{MO}\langle 8\rangle)\otimes H_{13}(\mm{M}\varphi)\to H_{13}(\mm{MO\langle 8\rangle \wedge M\varphi}).$$
\begin{lemma}\label{r08}
The bordism class of $\mathcal{B}_{\bold{q}}$ in each top summand of the bordism group $\Omega_{13}^{\mm{O}(8)}(\eta)$  is determined by the image $(1\wedge \mm{M} f)_\ast (\lambda)$, where
$$(1\wedge \mm{M} f)_\ast :\pi_{13}(\mm{MO\langle 8\rangle \wedge {M\varphi}})\to \pi_{13}(\mm{MO\langle 8\rangle \wedge {M\eta}})$$
  \end{lemma}
\begin{proof}
By the following  morphism of spectral sequences,
\[
\xymatrix@C=.2cm{
  {E}_2^{p,q}(\mm{MO}\langle 8\rangle,\mm{M}\varphi)\ar@{=>}[rr]^{}\ar[d]&  & \pi_{p+q}(\mm{MO}\langle 8\rangle \wedge \mm{M\varphi})\ar[d]^-{\iota_\ast} \\
{E}_2^{p,q}(H\wedge \mm{MO}\langle 8\rangle,\mm{M}\varphi)\ar@{=>}[rr]^-{}& & H_{p+q} (\mm{MO}\langle 8\rangle \wedge \mm{M\varphi})
}
\]
$\iota_\ast(\lambda)=1\wedge [\mathcal{B}_\bold{q}]U_{\ast }+T\in H_{13}(\mm{MO\langle 8\rangle \wedge M\varphi}) $ where $T$ is a torsion element of $H_{13}(\mm{MO\langle 8\rangle \wedge M\varphi})$.
Note that
$$(\mm{M}\nu\wedge 1)_\ast \circ\mm{M}(\vartriangle)_\ast([\mathcal{B}_\bold{q}]U_{\ast})=1\wedge [\mathcal{B}_\bold{q}] U_{\ast }+T^\prime\in H_{13}(\mm{MO\langle 8\rangle \wedge M\varphi}) $$
where $T^\prime$ is a torsion element in $H_{13}(\mm{MO\langle 8\rangle \wedge M\varphi})$.
By the naturality  of the Hurewicz homomorphism $\iota_\ast$,
we have
$$(\mm{M}({\nu})\wedge 1)_\ast\circ \mm{M}(\vartriangle)_\ast (\ell)=\lambda + R\in \pi_{13}(\mm{MO}\langle 8\rangle \wedge \mm{M\varphi}).$$
 where $R$ is provided by ${E}_2^{p,13-p}(\mm{MO}\langle 8\rangle,\mm{M}\varphi)$ for some $p<8$ due to
 $${E}_\infty^{p,13-p}(\mm{MO}\langle 8\rangle,\mm{M}\varphi)=0$$ for $8\le p\le 12$. Thus the desired result follows by Definition \ref{topandleft}.
\end{proof}

By the map $\mm{\alpha F}$, we can transform {\it top} summands of the bordism group into $ko_{13}(\mm{M}\eta)$ so as to use the $ko$ homology theory.
\[
\xymatrix@C=.7cm{
 \pi_{13} ( \mm{MO}\langle 8\rangle \wedge \mm{M \varphi })\ar[d]^-{(\mm{\alpha F}\wedge 1)_\ast}  \ar[rr]^-{(1\wedge \mm{M}f)_\ast}&  & \pi_{13}(\mm{MO}\langle 8\rangle \wedge \mm{M}\eta)\ar[d]^-{(\mm{\alpha F}\wedge 1)_\ast} \\
\pi_{13} ( ko \wedge \mm{M \varphi }) \ar[rr]^-{(1\wedge \mm{M}f)_\ast}& & \pi_{13} ( ko\wedge \mm{M}\eta)
}
\]
\begin{lemma}\label{RFPI}
Under the map $\mm{\alpha F}$, all {\it top} summands have the isomorphic images in $\pi_{13}(ko \wedge \mm{M}\eta)$ provided by
$E_2^{13-q,q}(ko,\mm{M}\eta)$ for $0\le q\le 4.$
\end{lemma}
\begin{proof}
	For certain {\it top} summands, i.e. $\mb{Z}_8,$ $\mb{Z}_5$, the generator of $\mb{Z}_4$, the $\mb{Z}_3$ provided by $E_2^{13,0}(\mm{MO}\langle 8\rangle
	,\mm{M}\eta)$, the lemma is obvious due to the isomorphisms $\mm{\alpha F}:\pi_n(\mm{MO}\langle 8\rangle)\to \pi_n(ko)$ when $n=0,1,2$. For $\mb{Z}_2\oplus \mb{Z}_3\subset  E_\infty^{10,3}(\mm{MO}\langle 8\rangle
	,\mm{M}\eta)$, the desired results follow by Lemma \ref{pFZ4}.
\end{proof}
\begin{proposition}\label{koM0}
${E}_\infty^{13,0}(ko,\mm{M}\varphi)={E}_2^{13,0}(ko,\mm{M}\varphi)=\mb{Z}$.
\end{proposition}
The Proposition \ref{koM0} follows by the following morphism for AHSS
\[
\xymatrix@C=.2cm{
  {E}_2^{p,q}(\mm{MO}\langle 8\rangle,\mm{M}\varphi)\ar@{=>}[rr]^{}\ar[d]&  & \pi_{p+q}(\mm{MO}\langle 8\rangle \wedge \mm{M\varphi})\ar[d]^-{(\mm{\alpha F}\wedge 1)_\ast} \\
{E}_2^{p,q}(ko,\mm{M}\varphi)\ar@{=>}[rr]^-{}& & \pi_{p+q} (ko \wedge \mm{M\varphi})
}
\]
By Lemma \ref{r08}, \ref{RFPI} and the above morphism for AHSS, we have
\begin{lemma}\label{kod}
(1) $\rho=(\mm{\alpha F}\wedge 1)_\ast(\lambda)\in ko_{13} ( \mm{M\varphi})$ is a free generator provided by ${E}_2^{13,0}(ko,\mm{M}\varphi)$. (2) The bordism class of $\mathcal{B}_{\bold{q}}$ in each {\it top} summand of the bordism group $\Omega_{13}^{\mm{O}(8)}(\eta)$  is determined by the image $(1\wedge \mm{M}f)_\ast ( \rho)$.
\end{lemma}

 Next we make a deeper computation for $ko_{13}(\mm{M}\varphi)$.
\begin{lemma}\label{koM4}
${E}_\infty^{13-q,q}(ko,\mm{M}\varphi)={E}_2^{13-q,q}(ko,\mm{M}\varphi)$ for $q=2,4$.
 Moreover, there is a nontrivial extension between ${E}_\infty^{11,2}(ko,\mm{M}\varphi)$ and ${E}_\infty^{9,4}(ko,\mm{M}\varphi)$.
\end{lemma}
\begin{proof}
Consider the following  morphism of spectral sequences.
\[
\xymatrix@C=.2cm{
  {E}_2^{p,q}(ko,\mm{M}\varphi)\ar@{=>}[rr]^{}\ar[d]^-{\mm{M}f_\ast}&  & \pi_{p+q}(ko \wedge \mm{M\varphi})\ar[d]^-{(1\wedge \mm{M}f)_\ast} \\
{E}_2^{p,q}(ko,\mm{M}\eta)\ar@{=>}[rr]^-{}& & \pi_{p+q} (ko\wedge \mm{M\eta})
}
\]
Since ${E}_2^{p,13-p}(ko,\mm{M}\eta)$ is a torsion group for $2\le p\le 13$, ${E}_\infty^{4,8}(ko,\mm{M}\eta)={E}_5^{4,8}(ko,\mm{M}\eta)$ and ${E}_\infty^{0,12}(ko,\mm{M}\eta)={E}_2^{0,12}(ko,\mm{M}\eta)$.
Note that ${E}_5^{4,8}(ko,\mm{M}\varphi)={E}_5^{4,8}(ko,\mm{M}\eta)=\mb{Z}$ and ${E}_2^{0,12}(ko,\mm{M}\varphi)={E}_2^{0,12}(ko,\mm{M}\eta)$.
 Thus
  $${E}_\infty^{4,8}(ko,\mm{M}\varphi)={E}_5^{4,8}(ko,\mm{M}\varphi)\quad {E}_\infty^{0,12}(ko,\mm{M}\varphi)={E}_2^{0,12}(ko,\mm{M}\varphi)$$
follow by Lemma \ref{ahsspro}. Next we compute ${E}_\infty^{2,10}(ko,\mm{M}\varphi)$ through the map
\begin{equation}
	\mm{M}\varphi\stackrel{\mm{M}f}{ \to} \mm{M}\eta \stackrel{\mm{M}p}{ \to} \mm{M}\xi \label{phitoxi}
\end{equation}
induced by $\mathcal{B}_\bold{q}\to B\to \mathrm{CP}^\infty$ and the AHSS ${E}_r^{p,q}(ko,\mm{M}\xi)$ for
$ko_\ast(\mm{M}\xi)$.
Note that ${E}_2^{2,10}(ko,\mm{M}\varphi)={E}_2^{2,10}(ko,\mm{M}\xi)$ and ${E}_{10}^{2,10}(ko,\mm{M}\xi)={E}_2^{2,10}(ko,\mm{M}\xi)$.
Thus ${E}_{10}^{2,10}(ko,\mm{M}\varphi)={E}_2^{2,10}(ko,\mm{M}\varphi)$ follows by Lemma \ref{ahsspro}. By Lemma \ref{d2} and equation \ref{SqBq}, ${E}_{3}^{11,2}(ko,\mm{M}\varphi)={E}_{2}^{11,2}(ko,\mm{M}\varphi)$.
Summary of the above computations is that there is not nontrivial differential about ${E}_{r}^{9,4}(ko,\mm{M}\varphi)$ and ${E}_{r}^{11,2}(ko,\mm{M}\varphi)$ for any $r\ge 2$ i.e. the first conclusion.


From the computation in section \ref{oriented}, $\mm{M}f_\ast:H_n(\mm{M}{\varphi })\to H_n(\mm{M}\eta)$  is nontrivial for $n=9$ and $11$.
By computing the ASS for $\pi_\ast(ko\wedge \mm{M}\varphi )$, we know that $h_1^2b_9xU\in Ext_{\mathscr{A}}^{2,15}(H^\ast(ko\wedge \mm{M}{\varphi };\mb{Z}_2),\mb{Z}_2)$ survives in the ASS. Then
$$(1\wedge \mm{M}f)_\ast:\pi_\ast(ko\wedge \mm{M}{\varphi })\to \pi_\ast(ko\wedge \mm{M}\eta)$$
$$(1\wedge \mm{M}f)_\ast(\{h_1^2b_9xU\})=\{h_1^2vxU\}.$$
It follows by Fig.\ref{Fig.2}, Corollary \ref{011} and Lemma \ref{RFPI} that $\{h_1^2vxU_{\ast 2}\}$ is provided by ${E}_2^{11,2}(ko,\mm{M}\eta)$ and there is an extension between $\mb{Z}_2\subset {E}_\infty^{11,2}(ko,\mm{M}\eta)$ and $\mb{Z}_2\subset {E}_\infty^{9,4}(ko,\mm{M}\eta)$.
 So $\{h_1^2b_9xU_{\ast 2}\}$ is provided by $ {E}_2^{11,2}(ko,\mm{M}\varphi)$,
 and the second conclusion follows.
\end{proof}

Different from $\pi_{13}(\mm{MO}\langle 8\rangle \wedge \mm{M}\varphi)$, $ko_{13}(\mm{M}\varphi)$ has a rank $2$ free abelian   subgroup.
Indeed ${E}_\infty^{11,2}(ko,\mm{M}\varphi)=\mb{Z}_2$ and ${E}_\infty^{9,4}(ko,\mm{M}\varphi)=\mb{Z}$. By the extension in Lemma \ref{koM4}, $ko_{13}(\mm{M}\varphi)$ has another free generator denoted by $\bar{\rho}$ and provide by
${E}_2^{11,2}(ko,\mm{M}\varphi)$. Moreover $\rho$ is linearly independent to $\bar \rho$.

Recall the map $j:X_{\bold{q}}\to \mathcal{B}_\bold{q}$ (cf. section \ref{Characteristic}).
Let $\omega$ be the pull-back bundle $j^\ast (f^\ast \eta)$ over $X_\bold{q}$, $\mm{M}\omega$ be the Thom spectrum, $U_{\ast}$ be the dual of the stable Thom class. Now we consider the AHSS ${E}_{r}^{p,q}( ko,\mm{M}\omega)$ for   $ko_{13}(\mm{M}{\omega})$.
 \begin{lemma}\label{koXq}
 	$ {E}_{\infty}^{13-q,q}(ko,\mm{M}\omega)={{{E}}}_{2}^{13-q,q}(ko,\mm{M}\omega)$ for $q=0,2,4$.
 	Moreover, $ko_{13}(\mm{M \omega })$ has a free subgroup provided by the extension of $$\mb{Z}_2\langle (x_9x)_\ast U_{\ast}\rangle\subset {{{E}}}_{\infty}^{11,2}(ko,\mm{M}\omega)\quad\text{and}\quad \mb{Z}\langle(x_9)_\ast U_{\ast}\rangle\subset {{{E}}}_{\infty}^{9,4}(ko,\mm{M}\omega)$$
 	where $x\in H^2(X_\bold q)$ and $x_9\in H^9(X_\bold q)$, $(-)_\ast$ denotes the homology dual.
 \end{lemma}
 \begin{proof}
 	From the proof of lemma \ref{koM4}, ${E}_\infty^{4,8}(ko,\mm{M}\varphi)={E}_5^{4,8}(ko,\mm{M}\varphi)$ and
 	 $${E}_\infty^{0,12}(ko,\mm{M}\varphi)={E}_2^{0,12}(ko,\mm{M}\varphi) \quad {E}_{10}^{2,10}(ko,\mm{M}\varphi)={E}_{7}^{2,10}(ko,\mm{M}\varphi).$$
 	 By ${E}_{\infty}^{13,0}(ko,\mm{M}\varphi)={E}_{2}^{13,0}(ko,\mm{M}\varphi)$ and ${E}_{2}^{12,1}(ko,\mm{M}\varphi)=0$, ${E}_{\infty}^{2,10}(ko,\mm{M}\varphi)={E}_{7}^{2,10}(ko,\mm{M}\varphi)$.
 	By the morphism for AHSS induced by $\mm{M}j:\mm{M}\omega\to \mm{M}\varphi$, ${{{E}}}_{5}^{4,8}(ko,\mm{M}\omega)={{{E}}}_5^{4,8}(ko,\mm{M}\varphi)$ and ${{{E}}}_{7}^{2,10}(ko,\mm{M}\omega)={{{E}}}_7^{2,10}(ko,\mm{M}\varphi)$.  Then it follows by Lemma \ref{ahsspro} that
 	${{E}}_{\infty}^{4,8}(ko,\mm{M}\omega)={{E}}_{5}^{4,8}(ko,\mm{M}\omega)$,
 	 ${{E}}_{\infty}^{0,12}(ko,\mm{M}\omega)={{E}}_{2}^{0,12}(ko,\mm{M}\omega) $ and $ {{E}}_{\infty}^{2,10}(ko,\mm{M}\omega)={{E}}_{7}^{2,10}(ko,\mm{M}\omega)$. Hence
${{E}}_{\infty}^{13-q,q}(ko,\mm{M}\omega)={{E}}_{2}^{13-q,q}(ko,\mm{M}\omega)$ for $q=0,2,4$ follows by equation \ref{sq} and theorem \ref{dr}.
By lemma \ref{koM4} and equation \ref{j-co}, we get the extension.
 \end{proof}

By Lemma \ref{koXq}, $ko_{13}(\mm{M}\omega)$ has a rank $3$ free abelian subgroup. One of free generators is $\theta$ provided by $(x_9x^2)_\ast U_\ast\in E_2^{13,0}(ko,\mm{M}\omega)$. Another is $\bar \theta$ provided by
 the extension in Lemma \ref{koXq}.
Then we compute $H_{13}(ko\wedge \mm{M}{\varphi })$ and $H_{13}(ko\wedge \mm{M\omega})$
by the AHSS.
\begin{lemma}\label{H1394}
	 (1) ${E}_{\infty}^{13-q,q}(H\wedge ko,\mm{M}\varphi)\cong {E}_{2}^{13-q,q}(H\wedge ko,\mm{M}\varphi)$ for $q=0,4$;

	(2) ${E}_{\infty}^{13-q,q}(H\wedge ko,\mm{M}\omega)\cong {E}_{2}^{13-q,q}(H\wedge ko,\mm{M}\omega)$ for $q=0,4$.
\end{lemma}
\begin{proof}
	Let us prove (1) at first. By the morphism of spectral sequences
	\[
\xymatrix@C=.2cm{
  {E}_{2}^{p,q}( ko,\mm{M}\varphi)\ar@{=>}[rr]^{}\ar[d]^-{\iota_\ast}&  & \pi_{p+q}(ko \wedge \mm{M\varphi})\ar[d]^-{\iota_\ast} \\
{E}_2^{ p,q}(H\wedge ko,\mm{M}\varphi)\ar@{=>}[rr]^-{}& & H_{p+q} (ko\wedge \mm{M\varphi})
}
\]
and Proposition \ref{koM0},
	we have ${E}_\infty^{13,0}(H\wedge ko,\mm{M}\varphi)={E}_2^{13,0}(H\wedge ko,\mm{M}\varphi)$.
	
	Next we use the map (\ref{phitoxi}) $\mm{M}{\varphi }\to \mm{M}\xi$ again
	  for the computation of
	$ {E}_\infty^{9,4}(H\wedge ko,\mm{M}\varphi).$
	Since that $H_\ast(\mm{M}\xi)$ is  free, the product
	$$\wedge:H_q(ko)\otimes H_p(\mm{M}\xi)\to H_{p+q}(ko\wedge \mm{M}\xi)$$
	is an isomorphism. Hence, in the AHSS $E_r^{p,q}(H\wedge ko,\mm{M}\xi)$ for $H_{\ast}(ko\wedge \mm{M}\xi)$,
$E_\infty^{p,q}(H\wedge ko,\mm{M}\xi)=E_2^{p,q}(H\wedge ko,\mm{M}\xi)$.
Note that $H_p(\mm{M}{\varphi })\cong H_p(\mm{M}\xi)$ for $p\le 4$. Then $E_\infty^{p,12-p}(H\wedge ko,\mm{M}\varphi)=E_2^{p,12-p}(H\wedge ko,\mm{M}\varphi)$ for $p=0,2,4$ follows by Lemma \ref{ahsspro}.
  By Proposition \ref{hko}, $E_2^{p,12-p}(H\wedge ko,\mm{M}\varphi)=0$ for $p=5,6,7$.
 So all differentials for $E_r^{9,4}(H\wedge ko,\mm{M}\varphi)$ are trivial,  and thus  $E_\infty^{9,4}(H\wedge ko,\mm{M}\varphi)\cong E_2^{9,4}(H\wedge ko,\mm{M}\varphi)$.
The proof of (2) is same as (1).
\end{proof}	

By the (1) in Lemma \ref{H1394}, $H_{13}(ko\wedge \mm{M}{\varphi })$ has a rank $2$ free abelian subgroup. It follows by the naturality of the product $\wedge$ that $1\wedge [\mathcal{B}_{\bold{q}}]U_{\ast}$ is a free generator provided by $E_2^{13,0}(H\wedge ko,\mm{M}\varphi)$. The other free generator is $\mathtt{k}_4 \otimes (b_9)_\ast U_{\ast}$ provided by $E_2^{9,4}(H \wedge ko, \mm{M}\varphi)$.

 By the (2) in Lemma \ref{H1394}, $H_{13}(ko\wedge \mm{M\omega})$ has a rank $3$ free abelian  subgroup.
    By the naturality of the product $\wedge$, $1\wedge (x_{9}x^2)_\ast U_{\ast }$ is a free generator provided by ${E}_{2}^{13,0}(H\wedge ko,\mm{M}\omega)$. Other free generators are $\mathtt{k}_4 \otimes (x_9)_\ast U_{\ast }$ and  $\mathtt{k}_4 \otimes (x_7x)_\ast U_{\ast }$ provided by ${E}_{2}^{9,4}(H\wedge ko,\mm{M}\omega)$.

Recall that $ko_{13}( \mm{M}\varphi) $ has a rank $2$ free abelian subgroup with generators ${\rho}$ and $\bar{\rho}$; $ko_{13}( \mm{M}\omega) $ has a rank $3$ free abelian subgroup, two of its generators are $\theta$ and $\bar \theta$.
Next we consider the Hurewicz homomorphism.

\begin{lemma}\label{Hure-rhophi}
	Under the Hurewicz homomorphisms,
	$$\iota_\ast({\rho})=1\wedge [\mathcal{B}_{\bold{q}}]U_{\ast }+\mathcal{T}(\varphi) \in H_{13}(ko\wedge \mm{M \varphi })$$
where $\mathcal{T}$ is a torsion element in $H_{13}(ko\wedge \mm{M \varphi })$.
\end{lemma}
\begin{proof}
 From the proof of lemma \ref{r08},
$$\iota_\ast:\pi_{13}(\mm{MO}\langle 8\rangle \wedge \mm{M\varphi})\to H_{13}(\mm{MO}\langle 8\rangle \wedge \mm{M\varphi})$$
\begin{equation}
	\iota_\ast(\lambda)=1\wedge [\mathcal{B}_{\bold q} ]U_{\ast }+T. \label{HureMo8phi}
\end{equation}
By the naturality of $\wedge$,
\begin{equation}
	(\mm{\alpha F}\wedge 1)_\ast(1\wedge [\mathcal{B}_{\bold{q}}]U_{\ast })=1\wedge [\mathcal{B}_{\bold{q}}]U_{\ast }\in H_{13}(ko\wedge \mm{M \varphi }). \label{alphaFiota}
\end{equation}
By $\rho=(\mm{\alpha F}\wedge 1)_\ast(\lambda)$, the lemma follows.
\end{proof}

\begin{lemma}
Under the Hurewicz homomorphism
	$$\iota_\ast(\theta)=1\wedge (x_9x^2)_\ast U_{\ast}+t_1\mathtt{k}_4\otimes (x_7x)_\ast U_{\ast }\quad \quad \quad \quad $$
\begin{equation}
	\quad \quad \quad \quad \quad\qquad +t_2\mathtt{k}_4 \otimes (x_9)_\ast U_{\ast }+\mathcal{T}(\omega) \in H_{13}( ko\wedge \mm{M}{\omega}) \label{HureKomega}
\end{equation}
for some $t_1,t_2\in \mb{Z}$ where $\mathcal{T}(\omega)$ is a torsion element.
\end{lemma}
\begin{proof}
$\chi:ko\to H$
	is the $0$th Postnikov tower of $ko$. We have the following commutative diagram.
\[
\xymatrix@C=.2cm{
 \pi_{13}(ko\wedge \mm{M}{\omega})\ar[d]^-{\iota_\ast}  \ar[rr]^-{\chi \wedge 1}&  & \pi_{13}(\bar{H} \wedge \mm{M}{\omega})\ar[d]^-{\iota_\ast } \\
 \pi_{13} (H\wedge ko \wedge \mm{M}{\omega}) \ar[rr]^-{1\wedge \chi \wedge 1}& & \pi_{13} ( H \wedge \bar{H}\wedge \mm{M}{\omega})
}
\]
Although $H=\bar{H}$ is the integral Eilenberg spectrum, we also discriminate them for convenience.
By Lemma \ref{chi} and ${E}_{\infty}^{13,0}(ko,\mm{M}\omega)={E}_{2}^{13,0}(ko,\mm{M}\omega)$ (see Lemma \ref{koXq}),
$(\chi\wedge 1)_\ast (\theta)=(x_9x^2)_\ast U_{\ast}. $ If we take the Eilenberg spectrum $H(\mb{Z}_p)$ with $\mathbb{Z}_p$-coefficient,
$$\iota_\ast :\bar{H}(\mathbb{Z}_p)_\ast(\mm{M}{\omega})\to  H(\mathbb{Z}_p)_\ast(\bar{H}(\mathbb{Z}_p)\wedge \mm{M}{\omega})\cong \mathscr{A}^\ast_{p}\otimes H(\mathbb{Z}_p)_\ast(\mm{M}\omega)$$
\begin{equation}
	(x_9x^2)_\ast U_{\ast }\to 1\otimes (x_9x^2)_\ast U_{\ast}+\cdots \in \mathscr{A}^\ast_{p}\otimes H_{\ast}(\mm{M}\omega;\mb{Z}_p) \label{comoduleoperation}
\end{equation}
  is the $\mathscr{A}^\ast_{p}$-comodule operation \cite{Sw} where $p$ is a prime number, $\mathscr{A}^\ast_{p}$ is the dual of the mod $p$ Steerod algebra. By the product $\wedge$ and the equation \ref{comoduleoperation},
$$\iota_\ast((x_9x^2)_\ast U_{\ast})=1\wedge (x_9x^2)_\ast U_{\ast }+\mathcal{T}(H\omega) \in H_{13}(H(\mathbb{Z})\wedge \mm{M}{\omega}).$$
 Since that $H_{13}( H(\mathbb{Z}) \wedge \mm{M}{\omega})$ has only one free subgroup with a generator $1\wedge (x_9x^2)_\ast U_{\ast}$,
 $\mathcal{T}(H\omega)$ is a torsion element. So the desired result follws by the commutative diagram.
\end{proof}

\begin{lemma}\label{Hure-rho94}
(1)  Under the Hurewicz homomorphism
$$\iota_\ast(\bar{\rho})=12\mathtt{k}_4\otimes (b_9)_\ast U_{\ast }+\mathcal{T}^\prime(\varphi) \in H_{13}(ko\wedge \mm{M \varphi })$$
where $\mathcal{T}^\prime(\varphi)\in H_{13}(ko\wedge \mm{M}{\varphi })$ is a torsion element.

(2) Under the Hurewicz homomorphism $$\iota_\ast(\bar \theta)=12\mathtt{k}_4\otimes (x_9)_\ast U_{\ast }+\mathcal{T}^\prime(\omega)\in H_{13}(ko\wedge \mm{M \omega })$$
  where $\mathcal{T}(\omega)\in H_{13}(ko\wedge \mm{M}\omega)$ is a torsion element.
\end{lemma}
\begin{proof}
	Let us prove (1). Consider the morphism of spectral sequences
	\[
\xymatrix@C=.2cm{
  {E}_2^{p,q}(ko,\mm{M}\varphi)={H}_p(\mm{M}{\varphi };\pi_q(ko))\ar@{=>}[rr]^{}\ar[d]&  & \pi_{p+q}(ko\wedge \mm{M}{\varphi })\ar[d]^-{\iota_\ast} \\
{E}_2^{p,q}(H\wedge ko,\mm{M}\varphi)={H}_p(\mm{M}{\varphi };H_q(ko))\ar@{=>}[rr]^-{}& & H_{p+q} ( ko\wedge \mm{M}{\varphi })
}
\]
${E}_2^{p,q}(ko,\mm{M}\varphi)\to {E}_2^{p,q}(H\wedge ko,\mm{M}\varphi)$ is $\times 24$ due to the Hurewicz homomorphism $ \pi_4(ko)\to H_4(ko)$ is $\times 24$ (see proposition \ref{hko}). By lemma \ref{koM4} and \ref{H1394}, we have $$\iota_\ast (\bar{\rho})=12\mathtt{k}_4\otimes (b_9)_\ast U_{\ast }+\mathcal{T}^\prime(\varphi).$$
 By Proposition \ref{hko}, $E_2^{p,13-p}(H\wedge ko,\mm{M}\varphi)$ is a torsion group for $p<9$, thus $\mathcal{T}^\prime(\varphi)$ is a torsion element.
The proof of (2) is same as (1).
\end{proof}



 Now we consider the following commutative diagram
\[
\xymatrix@C=.2cm{
 \theta\in \pi_{13}(ko\wedge \mm{M}{\omega})\ar[d]^-{\iota_\ast}  \ar[rrr]^-{(1 \wedge \mm{M}j)_\ast}& & & \pi_{13}(ko \wedge \mm{M}{\varphi})\ar[d]^-{\iota _\ast} \\
 H_{13} ( ko \wedge \mm{M}{\omega}) \ar[rrr]^-{(1 \wedge \mm{M}j)_\ast}&& & H_{13} ( ko\wedge \mm{M}\varphi)
}
\]
By equation \ref{HureKomega} and equation \ref{j-co},
$$(1 \wedge \mm{M}j)_\ast\circ \iota_\ast (\theta)=s\wedge [\mathcal{B}_\bold q] U_{\ast}+st_2\mathtt{k}_4 \otimes (b_9)_\ast U_{\ast 2}+T(\omega)$$
where $T(\omega)$ is a torsion element. Recall that $ko_{13}(\mm{M\varphi})$ has a rank two free subgroup with generators ${\rho}$ and $\bar{\rho}$, assume
$$(1 \wedge \mm{M}j)_\ast (\theta)=n_1{\rho}+n_2\bar{ \rho}+R(\varphi) \in \pi_{13}(ko\wedge \mm{M}\varphi)$$
where $R(\varphi)$ is a torsion element.
 By Lemma \ref{Hure-rhophi}, \ref{Hure-rho94} and the above commutative diagram, we have $n_1=s$ and $12n_2=st_2$. By $s=\pm 1\mod 6$,
 $n_2=sd$ and $t_2=12d$ $d\in \mb{Z}$. Hence
$$(1 \wedge \mm{M}j)_\ast\circ \iota_\ast (\theta -d\bar{\theta})=s\wedge [\mathcal{B}_\bold q] U_{\ast }+\mathcal{T}^{\prime \prime}(\varphi) \in H_{13}(ko\wedge \mm{M}\varphi)$$
where $\mathcal{T}^{\prime \prime}(\varphi)$ is a torsion element. Let $\widehat{\theta}={\theta}-d\bar{\theta}\in ko_{13}(\mm{M}\omega)$,
\begin{equation}
	(1\wedge \mm{M}j)_\ast (\widehat{\theta})=s\rho+R^\prime(\varphi) \in \pi_{13}(ko\wedge \mm{M}{\varphi }) \label{lastbor}
\end{equation}
where $R^\prime(\varphi)$ denotes a torsion element provided by ${{E}}_2^{p,q}(ko,\mm{M}\varphi)$ for $q>4$ due to Lemma \ref{koM4}. Especially, the $d$ only depends on the $\mm{M}{\omega}$.

\begin{lemma}\label{topbor}
 Given Bazaikin spaces $\mathcal{B}_{\bold{q}}$ and $\mathcal{B}_{\bold{q}^\prime}$ with the same cohomology rings and the  first Pontrjagin classes. If $s\ne 0\mod 5$, then their bordism classes  have the same components in every {\it top} summand of $\Omega_{13} ^{\mm{O}\langle 8 \rangle}(\eta)$.
\end{lemma}
\begin{proof}
By the assumption we know that $\sigma_2(\bold{q})=\sigma_2(\bold{q}^\prime)$ and $\sigma_3(\bold{q})=\sigma_3(\bold{q}^\prime)$. By Proposition \ref{Xq}, ${\Theta}^3_\bold{q}\cong {\Theta}^3_{\bold{q}^\prime}$.
\[
\xymatrix@C=.2cm{
& \mathcal{B}_{\bold{q}} \ar[drr]_-{f} & &\mm{K}(\mathbb{Z},5)\ar[d]^-{}\\
  X_\bold{q}\cong X_{\bold{q}^\prime} \ar[ru]^-{j}\ar[rd]^-{j^\prime}&&  &B\ar[d]^-{}\\
& \mathcal{B}_{\bold{q}^\prime}\ar[urr]^-{f^\prime}& & \mathrm{CP}^\infty
}
\]
Since $H^5(X_\bold{q};\mathbb{Z})=0$, the diagram is homotopy commutative, i.e., $f^\prime \circ j^\prime\simeq f\circ j$, and thus
$$(1\wedge \mm{M}(f\circ j))_\ast(\widehat{\theta})=(1\wedge \mm{M}(f^\prime \circ j^\prime))_\ast(\widehat{\theta})$$
Therefore   $(1\wedge \mm{M}f)_\ast({\rho})$ and $(1\wedge \mm{M}f^\prime)_\ast({\rho}^{\prime})$ have the same components in every {\it top} summand of the bordism group when $(s,6)= (s,5)=1$.
The desired result follows by Lemma \ref{kod}.
\end{proof}

\section{Bordism in PL and TOP category }\label{PL}

In section \ref{Detect} we have determined the {\it top} summands of $\Omega_{13}^{\mm{O}\langle 8\rangle}(\eta)$. It is still difficult to detect the whole bordism group by topological calculable invariants. In this section, we  prove that the trouble disappear under the forgetful map  $$T:\mm{M O}\langle 8\rangle\to \mm{M PL}\langle 8\rangle \simeq \mm{M TOP}\langle 8\rangle.$$
We start the proof with the represented elements
of the torsion   of $\pi_{n}(\mm{M O}\langle 8 \rangle)$ for $n=8,$ $9$ and $13$.
\begin{lemma}\label{top}
The torsion elements of $\Omega_{n}^{\mm{ O}\langle 8 \rangle}$ are represented by the exotic spheres for $n=8,$ $9$ and $13$.
\end{lemma}
\begin{proof}
By \cite{Adams1966} and \cite{Quillen},  the  $coker(J)$ has the following generators
$$\phi \in \mathbb{Z}_2\subset \pi_8(S^0),\quad \eta\phi  \in \mathbb{Z}_2\subset \pi_9(S^0),$$
$$\kappa  \in \mathbb{Z}_2\subset \pi_9(S^0),\quad \sigma \in \mathbb{Z}_3\subset \pi_{13}(S^0)$$
where $J:\pi_n(\mathit{SO})\to \pi_n(S^0)$, $\eta\in \pi_1(S^0)$ is represented by $h_1$ in the ASS for $S^0$. $\iota:S^0\to \mm{MO}\langle 8 \rangle$ is the unit, $\iota_\ast (\phi)$ is represented by $c_0$, $\iota_\ast(\kappa)$ is represented by $h_1\omega $, $\iota_\ast(\sigma)$ is represented by $b_0h_{1,0}$ in the ASS for $\mm{M O}\langle 8 \rangle$.
Besides, from the Kervaire-Milnor exact sequence \cite{KervaMilnor} \cite{Brumfiel}
$$\Theta_m\cong {bP}_{m+1}\oplus
coker(J)_m,$$
$\phi $, $\eta\phi $, $\kappa$ and $\sigma$ denote the exotic spheres $\Sigma^n$, thus $\Sigma^n\in \Omega_n^{fr}$ by $\pi_n(S^0)\cong \Omega_n^{fr}$. By the unit $\iota$, the torsion elements of $\pi_{n}(\mm{M O}\langle 8 \rangle)\cong \Omega_n^{\mm{O}\langle 8\rangle}$ are the exotic spheres for $n=8,$ $9$ and $13$.
\end{proof}

\begin{corollary}\label{top13}
The elements $\{h_1c_0x^2U\}$, $\{h_1\omega x^2U\}$ and $\{b_{0}h_{1,0}U\}$ of $\Omega_{13}^{\mm{ O}\langle 8 \rangle}(\eta)$ are in the kernel of the map
$$T\wedge 1 :\mm{ MO}\langle 8 \rangle \wedge \mm{M}\eta\to \mathrm{MTOP}\langle 8 \rangle \wedge \mm{M}\eta . $$
\end{corollary}
\begin{proof}
By proposition  \ref{013}, the element $\{b_{0}h_{1,0}U\}$ in $\Omega_{13}^{\mm{ O}\langle 8 \rangle}(\eta)$   is provided by
$E_2^{0,13}=H_0(\mm{M}\eta;\pi_{13}(\mm{MO}\langle 8 \rangle))$
in the AHSS. By lemma \ref{top}, the image of $\{b_{0}h_{1,0}U\}$ is $0$ under $(T\wedge 1)_\ast$.

By lemma \ref{x2}, $\{x^2U\}\in {_2\pi}_4(\mm{M}\eta)$. Since $\{h_1c_0\},\{h_1\omega\}\in {_2\pi}_9(\mm{MO}\langle 8\rangle)$, $\{h_1c_0x^2U\}$ and $\{h_1\omega x^2U\}$ are the products of $\{h_1c_0\},\{h_1\omega\}$ and $\{x^2U\}$. Thus they can be represented by the product manifolds $M^4\times \Sigma^9$
due to Lemma \ref{top}. Hence $\{h_1c_0x^2 U\}$ and $\{h_1\omega x^2U\}$ are mapped to $0$.
\end{proof}

Next we compute $\Omega_n^{\mm{Top}\langle 8\rangle}$ in order to consider the images of other elements of $\Omega_{13}^{\mm{ O}\langle 8 \rangle}(\eta)$ under the map $\mm{MO}\langle 8 \rangle \wedge \mm{M}\eta\to \mathrm{MTOP}\langle 8 \rangle \wedge \mm{M}\eta . $



\begin{lemma} \label{top8}
 $\Omega_n^{\mathrm{Top}\langle 8\rangle}\cong \Omega_n^{\mathrm{PL}\langle 8\rangle}$ are given by the following table.
\end{lemma}
\begin{center}
\resizebox{0.7\textwidth}{!}{
\begin{tabular}{|c|c|c|c|c|c|c|c|c|c|c|c|c|c|c|c|}
\hline
$n$ & 1 & 2 & 3 &4&5& 6&7  &8\\
\hline
 $\Omega_n^{\mathrm{Top}\langle 8\rangle}$ &$\mathbb{Z}_2$ & $\mathbb{Z}_2$ & $\mathbb{Z}_{24}$ & $0$ & $0$&$\mathbb{Z}_2$& $0$&$\mathbb{Z}\oplus \cdots$\\
\hline
\end{tabular}
}
\end{center}
\begin{proof}
The fiber of the natural map $\mathrm{BPL} \to \mathrm{BTOP}$ is $\mm{K}(\mathbb{Z}_2,3)$, then for the $7$-connected covers, $\mathrm{BPL}\langle 8\rangle \to \mathrm{BTOP}\langle 8\rangle$ is the homotopy equivalence. The natural map $\mathrm{BO}\langle 8\rangle\to \mathrm{BPL}\langle 8\rangle$ induces the isomorphism
$$\pi_n(\mathrm{BO}\langle 8\rangle)\cong \pi_n(\mathrm{BPL}\langle 8\rangle)=0$$ for $n<8$. Hence
$\pi_n(\mm{MO}\langle 8\rangle)\cong \pi_n(\mathrm{MPL}\langle 8\rangle)\cong \pi_n(\mathrm{MTOP}\langle 8\rangle)$ for $n<7$ by AHSS. From \cite{Hir1963}, $\pi_k(\mm{BPL},\mm{BO})$ is isomorphic to the group $\Gamma_{k-1}$ which is finite for all values of $k$. By the exact sequence $$0\to \pi_{8}(\mathrm{BO})\to \pi_{8}(\mathrm{BPL})\to \Gamma_7\to 0,$$
  $\pi_8(\mm{BO}\langle 8\rangle)\to \pi_8(\mathrm{BPL}\langle 8\rangle)$ is a monomorphism, and thus
$$H_8(\mm{MO}\langle 8\rangle)=\mb{Z}\to H_8(\mathrm{MTOP}\langle 8\rangle)$$ is a monomorphism. Hence $\pi_7(\mm{MO}\langle 8\rangle)\cong \pi_7(\mathrm{MTOP}\langle 8\rangle)=0$ and
$$\mb{Z}\subset\pi_8(\mm{MO}\langle 8\rangle)\to \pi_8(\mathrm{MTOP}\langle 8\rangle)$$ is a monomorphism by AHSS.
\end{proof}

\begin{corollary}\label{topbordism}
Every top summand (as well as $\mb{Z}_s^2$) of $\Omega_{13} ^{\mm{O}\langle 8 \rangle}(\eta)$  has isomorphic image under the forgetful map
$\mm {MO}\langle 8 \rangle \wedge \mm{M}\eta\to \mathrm{MTOP}\langle 8 \rangle \wedge \mm{M}\eta . $
\end{corollary}
%\begin{proof}The {\it top} bordism groups are provided by
	%$$E_2^{13-q,q}=H_p({M}\eta;\pi_q({MO}\langle 8\rangle))$$
	%for $q\le 3$. By the lemma \ref{top8}, we have the lemma.\end{proof}

\section{Surgery obstruction}\label{obstruction}
Assume that the $W^{14}$ has the $ B={BO} \langle 8 \rangle \times B$-structure with the boundary $\mathcal{B}_\bold{q}\sqcup -\mathcal{B}_{\bold{q}^\prime}$ which is the disjoint union of two homological Bazaikin spaces. From surgery theory \cite{Wall1999}, we can take appropriate $W$ such that $\phi:W\to  B$ is a $7$-equivalence. Thus
$i:\mathcal{B}_\bold{q}\to W$
is a $6$-equivalence, $H_n(W,\mathcal{B}_\bold{q})=0$ for $0\le n\le 6$. From the long exact sequence for the pair $(W, \mathcal{B}_\bold{q})$ we see that
$H^7(W,\mathcal{B}_\bold{q})=H^7(W).$ By the $7$-equivalence $\phi$, $H_6(W)=0,$ thus $H^7(W)$ is torsion free. From the long exact sequence for the triple $(W,\partial W, \mathcal{B}_\bold{q}),$
assume $H^7(W,\mathcal{B}_\bold{q})=\mathbb{Z}^{2n}$, then $H_7(W)=H^7(W,\partial W)=\mathbb{Z}^{2n}\oplus \mathbb{Z}_s.$

Considering the surgery obstruction for replacing $W$ by an h-cobordism, we transform the elements in $H_7(W,\mathcal{B}_\bold{q})$ into the kernel of the homomorphism $\phi_\ast:\pi_7(W)\to \pi_7(B)$ at first.
 \[
\xymatrix@C=.7cm{
&\pi_7(\mathcal{B}_\bold{q})\ar[r]^-{} \ar[d]^-{}& \pi_7(W) \ar[r]^-{} \ar[d]^-{}& \pi_7(W,\mathcal{B}_\bold{q})  \ar[r]^-{} \ar[d]^-{\cong}&\pi_6(\mathcal{B}_\bold{q}) \ar[d]^-{}\ar[r]^-{}&0\\
 0\ar[r]^-{}&H_7(\mathcal{B}_\bold{q})\ar[r]^-{} &H_7(W) \ar[r]^-{}& H_7(W,\mathcal{B}_\bold{q})\ar[r]^-{} &H_6(\mathcal{B}_\bold{q})=0&
}
\]
 By the $6$-equivalence $i:\mathcal{B}_\bold{q}\to W$, $\pi_7(W,\mathcal{B}_\bold{q})\cong H_7(W,\mathcal{B}_\bold{q})$.
 $$H_7(W)\cong H_7(W,\mathcal{B}_\bold{q})\oplus H_7(\mathcal{B}_\bold{q})=\mathbb{Z}^{2n}\oplus \mathbb{Z}_s$$
 \[
\xymatrix@C=.3cm{
\pi_8({BO}\langle 8\rangle) =\pi_8(B)\ar[r]^-{} \ar[d]^-{\cong}& \pi_8(B,W) \ar[r]^-{} \ar[d]^-{\cong}& \pi_7(W)  \ar[r]^-{\phi_\ast} \ar[d]^-{h_\ast}&\pi_7(B)=0 \ar[d]^-{}&\\
 H_8({BO}\langle 8\rangle)= H_8(B)\ar[r]^-{} &H_8(B,W) \ar[r]^-{}& H_7(W)\ar[r]^-{\rho_\ast} &H_7(B;\mathbb{Z})\ar[r]^-{}& 0
}
\]
Obviously, $ker\phi_\ast=\pi_7(W)$.
\begin{lemma}\label{surgery1}
	$ker\rho_\ast\cong \pi_7(W)=ker\phi_\ast.$
\end{lemma}
\begin{proof}
	It is easy to prove the isomorphism $h_\ast:\pi_7(W)\to  ker\rho_\ast$ by chasing the commutative diagram.
\end{proof}

From proposition \ref{Z-B},
$$H^8(B)=\mathbb{Z}_2\oplus \mathbb{Z}_s,\quad H_7(B)=H_7(B)=\mathbb{Z}_2\oplus \mathbb{Z}_s$$
where $x^4$ is the generator of $\mathbb{Z}_s\subset H^8(B).$
By universal coefficient theorem and the diagram as follow,
\[
\xymatrix{
H^8(B)\ar[rr]^-{}\ar[rd]^-{} & & H^8(W)\ar[ld]^-{}\\
 & H^8(\mathcal{B}_\bold{q})&
}
\]
$\mathbb{Z}_s\nsubseteq ker\rho_\ast$, and the free elements in $H_7({W})$ are mapped to $\mathbb{Z}_2\subset H_7({B})$.

By the Poincar$\acute{e}$ Duality, $H^7(W,\mathcal{B}_\bold{q})\cong H_7(W,-\mathcal{B}_\bold{q^\prime})$.
$$ \langle H^7(W,\mathcal{B}_\bold{q})\cup H^7(W,-\mathcal{B}_\bold{q^\prime}), [W,\partial W]  \rangle$$ and $H^7(W,\partial W) \to H^7(W,\mathcal{B}_\bold{q})$ (or $H^7(W,-\mathcal{B}_\bold{q^\prime}))$
induce
$$\langle H^7(W,\partial W)\cup H^7(W,\partial W), [W,\partial W]  \rangle$$
$$\lambda:H^7(W,\partial W)\otimes H^7(W,\partial W)\to \mathbb{Z}$$
which is an antisymmetric quadratic form, so the free part of $H^7(W,\partial W)$ can be divided into two parts $A$ and $B$ generated by $\{a_1,\cdots, a_{{n}}\}$ and $\{b_1,\cdots, b_{n}\}$ such that
$$ \langle a_i\cup b_i, [W,\partial W]  \rangle=1\quad \langle a_i\cup b_j, [W,\partial W]  \rangle=0$$
$$ \langle a_i\cup a_k, [W,\partial W]  \rangle=0\quad \langle b_k\cup b_i, [W,\partial W]  \rangle=0$$
where $j\ne i$, $1\le k\le {n}$.

Due to $H^7(W,\partial W)\cong H_7(W)$, we can regard $\rho_\ast$ as
$$\rho_\ast :H^7(W,\partial W)\cong H_7(W)\to H_7(B).$$

For every $1\le i\le {n}$, if $\rho_\ast a_i=0$ ( or $\rho_\ast b_i=0$), we take $u_i=a_i$ ( or $u_i=b_i$); if $\rho_\ast a_i=1$ and $\rho_\ast b_i=1$, we take $u_i=a_i+b_i$. Let $U=\mathbb{Z}^{{n}}$ generated by $\{u_1,\cdots, u_{{n}}\}$, then
$$U\subset ker\rho, \quad \lambda(u_i,u_k)=0$$
 for any $1\le i,k \le n$. By lemma \ref{surgery1}, $U$ is a direct summand in $ ker\phi_\ast$.

  By smooth or PL embedding theorem \cite{Milnor1965} \cite{RourkeSanderson}, any $u_i\in U$ can be represented by the embedding $u_i:S^7\to W$. The stable normal bundle of this embedding $u_i$ is trivial. From \cite{KervaMilnor} and [\cite{KirbSieb} First stability theorem],
  any $7$-dimensional vector $($PL or TOP$)$ bundle  over $S^7$ is trivial.
  Thus any $u_i\in U$ can be represented by the embedding $f_i:S^7\times D^7\to W$. Let $F$ be the homotopy fiber of $\xi:B\to {BO}$.
  Due to $\pi_7(F)=0$, the embedding $f_i$ is a compatible embedding. Hence we can replace the $W$ by an $h$-cobordism.

  In the PL category, we take $B^\prime=\mathrm{BPL}\langle 8\rangle \times B$ with the natural map
  $$B=\mathrm{BO}\langle 8\rangle \times B\to B^\prime=\mathrm{BPL}\langle 8\rangle \times B.$$
  The PL bundle over $B^\prime$ is the bundle $\gamma_8 \times \eta$ forgetting the structure of vector bundle
  with the classifying map $\xi^\prime:B^\prime\to \mathrm{BPL}$
  where $\gamma_8$ is the universal bundle  over $\mathrm{BO}\langle 8 \rangle$. Thus the PL manifold $\mathcal{B}_\bold{q}$ have $B^\prime$-structure. Analogously, if $\mathcal{B}_\bold{q}$ is $B^\prime$-coborant to $\mathcal{B}_\bold{q^\prime}$, then the cobordism can be replaced by $h$-cobordism through the surgery in PL category.

  \begin{lemma}\label{hcobordism}
   If $\mathcal{B}_\bold{q}$ is $B-$ $(B^\prime-)$ coborant to $\mathcal{B}_\bold{q^\prime}$	in smooth $(\mathrm{PL})$ category, then there is a smooth $(\mathrm{PL})$ $h$-cobordism
  $(W,\mathcal{B}_\bold{q},\mathcal{B}_\bold{q^\prime}).$
  \end{lemma}

\section{Proofs of Theorem 1.1, 1.2 and 1.3}\label{mainproof}

\begin{proof}[Proof of Theorem \ref{classify}] The sufficient of theorem \ref{classify} arises from Lemma \ref{Zsinvar}, \ref{topbor}, \ref{hcobordism}  and Corollary \ref{top13} obviously. Here we only prove the necessity.

As well known, the cohomology ring and link form are homotopy invariants. Besides the rational Pontryagin classes are not only diffeomorphism invariants, but also homeomorphism invariants. For our case, $H^4(\mathcal{B}_{\bold{q}};\mathbb{Z})=\mathbb{Z}$,  $H^8(\mathcal{B}_{\bold{q}};\mathbb{Z})=\mathbb{Z}_s$. Outwardly, the second Pontryagin class is not homeomorphism invariant. However, for the Bazaikin spaces, it is homeomorphism invariant from the view of bordism invariant.

From the previous discussion in Section \ref{Zs}, the second Pontryagin class determines the bordism group $\mathbb{Z}_s\subset \Omega_{13}^{\mm{O}\langle 8\rangle}(\eta)$ provided by $E_2^{5,8}$. Assume that two Bazaikin spaces $\mathcal{B}_\bold{q}$ and $\mathcal{B}_{\bold{q}^\prime}$ are homeomorphic, then they correspond to the same bordism class in $\Omega_{13}^{\mm{TOP}\langle 8\rangle}(\eta)$. By Corollary \ref{topbordism}, the bordism classes represented by them in $\Omega_{13}^{\mm{O}\langle 8\rangle}(\eta)$ have same component in $\mb{Z}_s\subset\Omega_{13}^{\mm{O}\langle 8\rangle}(\eta)$ provided by $E_2^{5,8}$.
Hence $p_2(\mathcal{B}_{\bold{q}})=p_2(\mathcal{B}_{\bold{q}^\prime})$.
\end{proof}

\begin{proof}[Proof of Theorem \ref{tops0}] Let $s=0\mod 5$. The second Pontryagin class can not determine the bordism invariant of above $\mb{Z}_s\subset \Omega_{13}^{\mm{O}\langle 8\rangle}(\eta)$ sufficiently since that
we do not have a conclusion analogue to corollary \ref{ZsHure}. It is unknown for the bordism invariant of one subgroup with order $5$ of the $\mb{Z}_s$. Besides, since the $s$-times in the equation \ref{lastbor}, two Bazaikin spaces may have different components in summand $\mb{Z}_5\subset \Omega_{13}^{\mm{O}\langle 8\rangle}(\eta)$ even if satisfying the conditions of lemma \ref{topbor}, which means that the bordism invariant of $\mb{Z}_5$ is unclear. Hence we have theorem \ref{tops0} by lemma \ref{hcobordism}.
\end{proof}


\begin{proof}[Proof of Theorem \ref{diff}] From the computations in section \ref{2primary} and \ref{3primary}, there are $12$ elements at most whose bordism invariants determine the differential structures for Bazaikin spaces by lemma \ref{hcobordism}. So theorem \ref{diff} follows.
\end{proof}

\appendix
\section{The Serre spectral sequence}\label{SSSforintegral}
 The computations for Lemma \ref{Z-3}:
It is obvious $d_3(l_2)=l_3$ in the  SSS $$E_2^{p,q}=H^p(\mm{K}(\mathbb{Z}, 3);H^{q}(\mm{K}(\mathbb{Z},2)))\Longrightarrow H^{p+q}(\mm{PK}(\mathbb{Z}, 3)) .$$
By the product of SSS, $d_3(l_2^2)=2l_3l_2\in E_3^{3,2}$. Note that $E_\infty^{3,2}\cong E_4^{3,2}=0$ and $E_\infty^{6,0}\cong E_4^{6,0}=0$. Thus there exists a nontrivial differential $d_3:E_3^{3,2}\to E_3^{6,0}$ such that the following sequence
$$0\to E_3^{0,4}=\mb{Z}\stackrel{d_3}{\to } E_3^{3,2}=\mb{Z}\stackrel{d_3}{\to }E_3^{6,0}\to 0$$
is exact. Hence $E_3^{6,0}\cong E_2^{6,0}=H^6(\mm{K}(\mb{Z},6))=\mb{Z}_2\langle l_3^2 \rangle$.

Then we consider successively whether the differential $d_r:E_r^{p,r-1}\to E_r^{p+r,0}$ is nontrivial when $p+r=i\ge 7$ as above. Here we give two examples.

 Consider $p+r=7$, all differentials $d_r:E_r^{7-r,r-1}\to E_r^{7,0}$ for $r\ge 2$ are trivial, so $H^7(\mm{K}(\mb{Z},3))=0$.

Consider $p+r=8$, the only possible nontrivial differential  for $E_r^{8,0}$ is $d_{5}:E_{5}^{3,4}\to E_{5}^{8,0}.$ From the following chain complex
$$0\to E_3^{0,6}=\mb{Z}\stackrel{d_3}{\to } E_3^{3,4}=\mb{Z}\stackrel{d_3}{\to }E_3^{6,2}=\mb{Z}_2$$
$d_3(l_2^3)=3l_3l_2^2\in E_3^{3,4}$. If $0\ne d_3(l_3l_2^2)\in  E_3^{6,2}$, then $d_3(3l_3l_2^2)\ne 0$ which contradicts with $d_3\circ d_3=0$. Thus $E_5^{3,4}\cong E_4^{3,4}=\mb{Z}_3$. Since $E_6^{3,4}\cong E_\infty^{3,4}=0$, the differential $d_{5}:E_{5}^{3,4}\to E_{5}^{8,0}$ is an isomorphism. $H^8(\mm{K}(\mb{Z},3))=\mb{Z}_3$ follows by $E_{5}^{8,0}=E_{2}^{8,0}$.

The computations for Lemma \ref{Z-4} is easy, so is left to the readers.


The computations for Lemma \ref{Z-5}:
Consider the SSS
$$E_2^{p,q}=H^p(\mm{K}(\mathbb{Z}, 5);H^{q}(\mm{K}(\mathbb{Z},4)))\Longrightarrow H^{p+q}(\mm{PK}(\mathbb{Z}, 5)) .$$
All differentials about $E_{r}^{i,0}$ $(i\le 11)$ can be determined easily.

We claim that $d_{12}: E_{12}^{0,11}\to E_{12}^{12,0}$ is the only nontrivial differential for $E_r^{12,0}$. By Lemma \ref{Z-4}, $E_{5}^{0,11}=E_{2}^{0,11}=\mb{Z}_2\langle b_7l_4,b_{11}\rangle$. Since $d_5(b_7l_4)=l_5\circ b_7\in E_5^{5,7}$, $E_{8}^{0,11}=E_{6}^{0,11}=\mb{Z}_2\langle b_{11}\rangle$. Note that the mod $2$ reduction of $c_8$ is $\mm{Sq^3}l_5$. By the cup product of $H^\ast(\mm{K}(\mb{Z},5);\mb{Z}_2)$, $0\ne c_8\cdot l_5\in H^{13}(\mm{K}(\mb{Z},5))$, thus the differential $d_5: E_5^{8,4}=\mb{Z}_2\to E_5^{13,0}$ is nontrivial which implies $E_8^{8,4}=E_6^{8,4}=0$. So $E_{12}^{0,11}=E_{8}^{0,11}$. The claim follows by $E_{\infty}^{0,11}=E_{13}^{0,11}=0.$ By $E_{12}^{12,0}=E_{2}^{12,0}$, $H^{12}(\mm{K}(\mb{Z},5))=\mb{Z}_2$.

Note that $d_{5}: E_{5}^{8,4}\to E_{5}^{13,0}$ is the only nontrivial differential for $E_r^{13,0}$. So $ H^{13}(\mm{K}(\mb{Z},5))=\mb{Z}_2$ by $E_{5}^{13,0}=E_{2}^{13,0}$.

The possible nontrivial differentials for $E_r^{14,0}$ are $d_6: E_6^{5,8}\to E_6^{14,0}$ and $d_{14}: E_{14}^{0,13}\to E_{14}^{14,0}$. By the complex $0\to E_5^{0,12}\stackrel{d_5}{\to }E_5^{5,8}\stackrel{d_5}{\to } E_5^{10,4}$, we have $E_6^{5,8}=\mathbb{Z}_3$. Since $E_\infty^{5,8}=E_7^{5,8}=0$, $d_6: E_6^{5,8}\to E_6^{14,0}$ is nontrivial. So $\mathbb{Z}_3 \subset E_6^{14,0}=E_2^{14,0}.$
By the differential $d_{5}: E_{5}^{0,13}\to E_{5}^{5,9}$ i.e. $d_5(b_9l_4)=-l_5\circ b_9$, $E_7^{0,13}=E_6^{0,13}=\mathbb{Z}_2^2 \oplus \mathbb{Z}_5$. Note that $H^{13}(\mm{K}(\mathbb{Z},5);\mathbb{Z}_2)=\mathbb{Z}_2^2$ and $H^{13}(\mm{K}(\mathbb{Z},5))=\mathbb{Z}_2$. By the universal coefficient theorem, $H^{14}(\mm{K}(\mathbb{Z},5))$ has only one subgroup with order $2$. We claim that the differential $d_7: E_{7}^{0,13}\to E_{7}^{7,7}=\mathbb{Z}_2$ is nontrivial. If the differential $d_7: E_{7}^{0,13}\to E_{7}^{7,7}$ is trivial, then  $E_{10}^{0,13}=E_{7}^{0,13}$.
Note that $E_{5}^{10,4}=E_{2}^{10,4}=\mb{Z}_2\langle l_5^2\circ l_4 \rangle \oplus \mb{Z}_3\langle c_{10}\circ l_4 \rangle$. By the cup product of $H^\ast(\mm{K}(\mb{Z},5);\mb{Z}_2)$, $0\ne  l^3_5\in H^{15}(\mm{K}(\mb{Z},5))$, thus $d_5(l_5^2\circ l_4)=l_5^3$. So $d_{10}: E_{10}^{0,13}\to E_{10}^{10,4}$ is trivial, and $E_{14}^{0,13}=E_{10}^{0,13}$.
 By $E_{\infty}^{0,13}=E_{15}^{0,13}=0$, $d_{14}: E_{14}^{0,13}\to E_{14}^{14,0}$ is isomorphic, thus $\mb{Z}_2^2\subset H^{14}(\mm{K}(\mathbb{Z},5))$ which is a contradiction.  Hence the claim follows, then $E_{14}^{14,0}\cong E_{14}^{0,13}=E_{8}^{0,13}=\mathbb{Z}_2 \oplus \mathbb{Z}_5$, $H^{14}(\mm{K}(\mathbb{Z},5))=\mathbb{Z}_2\oplus \mathbb{Z}_3 \oplus \mathbb{Z}_5$.

The possible nontrivial differentials for $E_r^{15,0}$ are $d_{5}:E_{5}^{10,4}\to E_{5}^{15,0}$ and $d_{15}:E_{15}^{0,14}\to E_{15}^{15,0}$. From above discussion,  $d_5( E_5^{5,8})=0$ and $d_{10}( E_{10}^{0,13})=0$. Hence
  $d_5: E_5^{10,4}\to E_5^{15,0}$ is injective which means $\mathbb{Z}_2\langle l_{5}^3 \rangle\oplus \mathbb{Z}_3\langle c_{10} l_5\rangle \subset E_5^{15,0}=E_2^{15,0}$. By the cup product of $H^\ast(\mm{K}(\mb{Z},5);\mb{Z}_2)$, $0\ne  c^2_8\in H^{15}(\mm{K}(\mb{Z},5))$.
  For the differential $d_8: E_8^{8,7}\to E_8^{16,0}$, $d_8(c_8\circ b_7)=c_8^2$ which means $E_{15}^{0,14}=E_{2}^{0,14}=\mb{Z}_2$. So the differential $d_{15}:E_{15}^{0,14}\to E_{15}^{15,0}$ is nontrivial. Note that $H^{14}(\mm{K}(\mathbb{Z},5);\mathbb{Z}_2)=\mathbb{Z}_2^3$ and $H^{14}(\mm{K}(\mathbb{Z},5))=\mathbb{Z}_2\oplus \mathbb{Z}_3 \oplus \mathbb{Z}_5$. By the universal coefficient theorem, $H^{15}(\mm{K}(\mathbb{Z},5))$ has two subgroups with order $2$. Hence $H^{15}(\mm{K}(\mathbb{Z},5))=\mb{Z}_2^2\oplus \mb{Z}_3.$

The computations for Lemma \ref{Z-6}: Consider the SSS
$$E_2^{p,q}=H^p(\mm{K}(\mathbb{Z}, 6);H^{q}(\mm{K}(\mathbb{Z},5)))\Longrightarrow H^{p+q}(\mm{PK}(\mathbb{Z}, 6)) .$$
All differentials about $E_{r}^{i,0}$ $i\le 15$ can be determined easily.

For the differential $d_{6}:E_{6}^{0,15}\to E_{6}^{6,10}$, $d_6(l_5^3)=l_6\circ l_5^2$, $d_6(c_{10}l_5)=l_6\circ c_{10}$. So $E_{7}^{0,15}=\mb{Z}_2$. By
$H^{15}(\mm{K}(\mathbb{Z}, 6);\mathbb{Z}_2)=\mathbb{Z}_2^3$ and $ H^{15}(\mm{K}(\mathbb{Z}, 6))=\mathbb{Z}_2^2\oplus \mathbb{Z}_3\oplus \mathbb{Z}_5,$
$H^{16}(\mm{K}(\mathbb{Z},6))$ has a subgroup with order $2$. Note that there is only one nontrivial differential for $E_r^{16,0}$, i.e. $d_{16}:E_{16}^{0,15}\to E_{16}^{16,0}$.
So $E_{16}^{16,0}\cong E_{16}^{0,15}=E_{7}^{0,15}=\mathbb{Z}_2$. By $E_{16}^{16,0}=E_{2}^{16,0}$, $H^{16}(\mm{K}(\mathbb{Z},6))=\mb{Z}_2$. %Hence $H^{16}(\mm{K}(\mathbb{Z},6);\mathbb{Z})=\mathbb{Z}_2.$

The computations for Lemma \ref{Z-B}:
In order to calculate $H^\ast(B)$ by the fibration diagram \ref{fib}, we list all nontrivial differentials for $E_r^{0,q}$ of the SSS $E_2^{p,q}=H^p(\mm{K}(\mathbb{Z}, 6);H^{q}(\mm{K}(\mathbb{Z},5)))\Longrightarrow H^{p+q}(\mm{PK}(\mathbb{Z}, 6)) .$
$$d_6:E_{6}^{0,5}\to E_{6}^{6,0}\quad d_9:E_{9}^{0,8}=E_{2}^{0,8}\to E_{9}^{9,0}\quad d_{11}:E_{11}^{0,10}=E_{2}^{0,10}\to E_{11}^{11,0}$$
$$d_{13}:E_{13}^{0,12}=E_{2}^{0,12}\to E_{13}^{13,0}\quad d_{6}:E_{6}^{0,13}=E_{2}^{0,13}\to E_{6}^{6,8}$$ $$d_{15}:E_{15}^{0,14}=E_{2}^{0,14}\to E_{15}^{15,0}$$

$d_{6}:E_{6}^{0,15}=E_{2}^{0,15}\to E_{6}^{6,10}$,  $d_6(l_5^3)=l_6\circ l_5^2,$  $d_6(c_{10}l_5)=l_6\circ c_{10}$,
 $$d_{16}:E_{16}^{0,15}=E_{7}^{0,15}\to E_{16}^{16,0}.$$
Let $\{ \bar{E}_r^{p,q}, \bar{d}_r    \}$ denote the SSS for the fibration
$\mm{K}(\mathbb{Z},5)\to B\to \mathrm{CP}^\infty$ where $\bar{E}_2^{p,q}=H^p(\mm{CP}^\infty;H^q(\mm{K}(\mb{Z},5)))=H^p(\mm{CP}^\infty)\otimes H^q(\mm{K}(\mb{Z},5))$.
By the fibration diagram \ref{fib}, $\bar{d}_6(l_5)=sx^3$. Let $n\ge 0$.
For the differential $\bar{d}_6: \bar{E}_6^{2n,5}\to \bar{E}_6^{2n+6,0}$, $\bar{d}_6(x^n\circ l_5)=sx^{n+3}$, thus $\bar{E}_7^{2n+6,0}=\mathbb{Z}_s$. Any differential onto $\bar{E}_r^{2n+6,0}$ for $7\le r\le 15$ is the $0$-homomorphism by $s=\pm 1 \mod 6$ and $\bar{E}_{2}^{2n-9,14}=0$. So $\bar{E}_\infty^{2n+6,0}=\mathbb{Z}_s\langle x^{n+3} \rangle$ for $0\le n \le 5$.

By the listed differentials about $E_r^{0,q}$, other differentials about $\bar{E}_r^{p,q}$ $(0<p+q\le 15)$ is obvious.

\section{Algebraic Atiyah-Hirzebruch spectral sequence}

\subsection{The part of mod 2}\label{AAHSSmod2}
Recall that
$$E_1^{s,m,n}=\mm{Tor}^{\mathscr{A}_2}_{s,s+m}(\mathbb{Z}_2,\mathbb{Z}_2)\otimes N^n \Rightarrow \mm{Tor}^{\mathscr{A}_2}_{s,s+m+n}(\mathbb{Z}_2,N)$$
where $N^n=H^n(\mm{M}\eta;\mathbb{Z}_2)$, $N=H^\ast(\mm{M}\eta;\mb{Z}_2)$. And $N_n=\oplus_{i\ge n}H^i(\mm{M}\eta;\mathbb{Z}_2)$.
$$d_1=j_\ast\partial_\ast:E_1^{s,m,n}\to E_1^{s-1,m,n+1}\quad d_r:E_r^{s,m,n}\to E_r^{s-1,m+1-r,n+r}$$

The nontrivial $E_1^{s,m,n}$-terms for $12\le n+m \le 14$ are listed as follow.
$$E_1^{s,0,13}=\{h_0^svx^2U, h_0^sx\mathrm{Sq}^4uU, h_0^s\mathrm{Sq}^6uU\}$$ $$E_1^{s,0,12}=\{h_0^sx^2\mathrm{Sq}^1uU, h_0^sx\mathrm{Sq}^2\mathrm{Sq}^1uU, h_0^s\mathrm{Sq}^5uU\}$$
$$E_1^{s,0,14}=\{h_0^su^2U,h_0^sx^2\mathrm{Sq}^2\mathrm{Sq}^1uU, h_0^sx\mathrm{Sq}^5uU, h_0^s\mathrm{Sq}^6\mathrm{Sq}^1uU\}\quad s\ge 0;$$
$$E_1^{1,1,12}=\{h_1x^2\mathrm{Sq}^1uU, h_1x\mathrm{Sq}^2\mathrm{Sq}^1uU, h_1\mathrm{Sq}^5uU\}$$
$$E_1^{1,1,11}=\{h_1ux^2U , h_1vxU, h_1\mathrm{Sq}^4uU\},$$
$$ E_1^{1,1,13}=\{h_1vx^2U, h_1x\mathrm{Sq}^4uU, h_1\mathrm{Sq}^6uU\};$$
$$E_1^{2,2,11}=\{h_1^2ux^2U , h_1^2vxU, h_1^2\mathrm{Sq}^4uU\}\quad E_1^{2,2,10}=\{h_1^2\mathrm{Sq}^2\mathrm{Sq}^1uU , h_1^2x\mathrm{Sq}^1uU\}$$
$$E_1^{2,2,12}=\{h_1^2x^2\mathrm{Sq}^1uU, h_1^2x\mathrm{Sq}^2\mathrm{Sq}^1uU, h_1^2\mathrm{Sq}^5uU\};$$
$$E_1^{s+1,3,10}=\{h_0^sh_2\mathrm{Sq}^2\mathrm{Sq}^1uU , h_0^sh_2x\mathrm{Sq}^1uU\}\quad E_1^{s+1,3,9}=\{h_0^sh_2uxU , h_0^sh_2vU\}$$
$$E_1^{s+1,3,11}=\{h_0^sh_2ux^2U , h_0^sh_2vxU, h_0^sh_2\mathrm{Sq}^4uU\}\quad 0\le s\le 2;$$
$$E_1^{2,6,7}=\{h_2^2uU\}\quad E_1^{2,6,8}=\{h_2^2\mathrm{Sq}^1uU\}$$
$$E_1^{3,8,4}=\{c_0x^2U\}, \quad E_1^{s+4,8,4}=h_0^s\omega x^2U\}\quad s\ge 0;$$ $$E_1^{s,9,4}=\{h_1c_0x^2U, h_1\omega x^2U\}\quad s=4,5;$$
$$E_1^{6,10,2}=\{h_1^2\omega xU\}\quad E_1^{6,10,4}=\{h_1^2\omega x^2U\}$$
$$E_1^{s+5,11,2}=\{h_0^sh_2\omega xU\}\quad 0\le s\le 2;$$
$$E_1^{s+3,12,0}=\{h_0^s\tau U\}\quad E_1^{s+3,12,2}=\{h_0^s\tau xU\}\quad s\ge 0;$$
$$E_1^{s+4,14,0}=\{h_0^sd_0U\}\quad 0\le s\le 2.$$

\begin{lemma}\label{pert}
Each element in $E_1^{s,m,4}$ or $E_1^{s,m,2}$ is a permanent cycle.
\end{lemma}
\begin{proof}
 For any $\Sigma [a_1|a_2|\cdots|a_s] x^2U\in E_1^{s,m,4}$, $\Sigma [a_1|a_2|\cdots|a_s] x^2U$ is a cycle in $B(N_4)$ since $\mm{Sq}^\mm{I}x^2U=0$ for any admissible sequence $\mm{I}=(i_1,i_2,\cdots,i_r)$. Hence $\partial_\ast\Sigma [a_1|a_2|\cdots|a_s] x^2U=0.$ Similarly, for any $\Sigma [a_1|a_2|\cdots|a_s] xU\in E_1^{s,m,2}$, $\Sigma [a_1|a_2|\cdots|a_s] xU$ is a cycle in $B(N_2)$, thus $\partial_\ast\Sigma [a_1|a_2|\cdots|a_s] xU=0.$
The boundary homomorphisms $\partial_\ast$ imply the lemma.
\end{proof}


\begin{lemma}
$E_2^{s,0,12}=0$ for $s\ge 0;$ $E_2^{0,0,13}=\{vx^2U,x\mathrm{Sq}^4uU, \mathrm{Sq}^6uU\}$, $E_2^{s,0,13}=0$ for $s> 0;$ $E_2^{s,0,14}=0$ for $s>0,$ $E_2^{0,0,14}=\{\mathrm{Sq}^6\mathrm{Sq}^1uU\}$.
\end{lemma}
\begin{proof}
Note that $\mm{Tor}^{\mathscr{A}_2}_{s,s}(\mathbb{Z}_2,\mathbb{Z}_2)$ is generated by $\mathrm{Sq}^1\otimes \cdots \otimes \mathrm{Sq}^1$. The desired results follow by the $\mm{Sq}^1$ action on $H^{n}(\mm{M}\eta;\mathbb{Z}_2)$ for $11\le n\le 14$.
\end{proof}

\begin{lemma}
$E_2^{1,1,n}=E_1^{1,1,n}$ and $E_2^{2,2,n}=E_1^{2,2,n}$ for $n=11$, $12$ or $13$.
\end{lemma}
\begin{proof}
The desired results follow by the degree of the differential $d_1$.
\end{proof}

\begin{lemma}
$E_2^{1,3,9}=E_1^{1,3,9}$, $E_2^{s,3,9}=0$ for $s=2$ or $3;$ $E_2^{3,3,10}=E_1^{3,3,10}$, $E_2^{s,3,10}=0$ for $s=1$ or $2;$ $E_2^{1,3,11}=E_1^{1,3,11}$, $E_2^{s,3,11}=0$ for $s=2$ or $3$.
\end{lemma}
\begin{proof}
By the differential $d_1$
$$E_1^{n,3,8}\stackrel{d_1}{\rightarrow}E_1^{n-1,3,9}\stackrel{d_1}{\rightarrow}E_1^{n-2,3,10}\stackrel{d_1}{\rightarrow}E_1^{n-3,3,11}\stackrel{d_1}{\rightarrow}E_1^{n-4,3,12}$$
and the $\mathrm{Sq}^1$ action on $H^\ast(\mm{M}\eta;\mathbb{Z}_2)$, the desired results follow.
\end{proof}

The nontrivial $E_2^{s,m,n}$-terms for $12\le n+m \le 14$ are listed as follow.
$$E_2^{0,0,13}=\{vx^2U, x\mathrm{Sq}^4uU, \mathrm{Sq}^6uU\}\quad E_2^{0,0,14}=\{\mathrm{Sq}^6\mathrm{Sq}^1uU\};$$
$$E_2^{1,1,12}=\{h_1x^2\mathrm{Sq}^1uU, h_1x\mathrm{Sq}^2\mathrm{Sq}^1uU, h_1\mathrm{Sq}^5uU\}$$
$$E_2^{1,1,11}=\{h_1ux^2U , h_1vxU, h_1\mathrm{Sq}^4uU\}$$ $$E_2^{1,1,13}=\{h_1vx^2U, h_1x\mathrm{Sq}^4uU, h_1\mathrm{Sq}^6uU\};$$
$$E_2^{2,2,11}=\{h_1^2ux^2U , h_1^2vxU, h_1^2\mathrm{Sq}^4uU\}\quad E_2^{2,2,10}=\{h_1^2\mathrm{Sq}^2\mathrm{Sq}^1uU , h_1^2x\mathrm{Sq}^1uU\}$$
$$E_2^{2,2,12}=\{h_1^2x^2\mathrm{Sq}^1uU, h_1^2x\mathrm{Sq}^2\mathrm{Sq}^1uU, h_1^2\mathrm{Sq}^5uU\};$$
$$E_2^{3,3,10}=\{h_0^2h_2\mathrm{Sq}^2\mathrm{Sq}^1uU , h_0^2h_2x\mathrm{Sq}^1uU\}\quad E_2^{1,3,9}=\{h_2uxU , h_2vU\}$$
$$E_2^{1,3,11}=\{h_2ux^2U , h_2vxU, h_2\mathrm{Sq}^4uU\};$$
$$E_2^{2,6,7}=\{h_2^2uU\}\quad E_2^{2,6,8}=\{h_2^2\mathrm{Sq}^1uU\}$$
$$E_2^{3,8,4}=\{c_0x^2U\}; \quad E_2^{s+4,8,4}=h_0^s\omega x^2U\}\quad s\ge 0;$$ $$E_2^{s,9,4}=\{h_1c_0x^2U, h_1\omega x^2U\}\quad s=4,5;$$
$$E_2^{6,10,2}=\{h_1^2\omega xU\}\quad E_2^{6,10,4}=\{h_1^2\omega x^2U\}$$
$$E_2^{s+5,11,2}=\{h_0^sh_2\omega xU\}\quad 0\le s\le 2;$$
$$E_2^{s+3,12,0}=\{h_0^s\tau U\}\quad E_2^{s,12,2}=\{h_0^s\tau xU\}\quad s\ge 0;$$
$$E_2^{s+4,14,0}=\{h_0^sd_0U\}\quad 0\le s\le 2.$$


\begin{lemma}\label{0015}
$E_3^{1,1,11}=0,$ $E_3^{0,0,13}=\{vx^2U, \mathrm{Sq}^6uU=x\mathrm{Sq}^4uU\},$ $$E_3^{1,1,13}=\{h_1vx^2U\},\quad
 E_3^{2,2,10}=0,\quad E_3^{2,2,11}=\{h_1^2ux^2U, h_1^2vxU\} $$
 $$ E_3^{1,1,12}=\{h_1x^2\mathrm{Sq}^1uU\},\quad E_3^{0,0,14}=0,$$
 $$E_3^{3,3,10}=\{h_0^2h_2(x\mathrm{Sq}^1uU+\mathrm{Sq}^2\mathrm{Sq}^1uU)\},\quad E_3^{2,2,12}=0,$$
$$ E_3^{0,0,15}=\{x^2\mathrm{Sq}^4uU, u\mathrm{Sq}^1uU\},\quad E_3^{1,1,14}=\{h_1u^2U\}.$$
\end{lemma}
\begin{proof}
By the $\mm{Sq}^1$ action, we have $E_2^{3,3,8}=E_1^{3,3,8}=\{h_0^2h_2\mathrm{Sq}^1uU\},$  $E_2^{2,2,9}=E_1^{2,2,9}$, $E_2^{1,1,14}=E_1^{1,1,14}$,  $E_2^{0,0,15}=\{x^2\mathrm{Sq}^4uU, x\mathrm{Sq}^6uU, u\mathrm{Sq}^1uU\}.$ Then consider the following differentials
$$E_2^{2,2,9}\stackrel{d_2}{\rightarrow}E_2^{1,1,11}\stackrel{d_2}{\rightarrow}E_2^{0,0,13}, \quad 0=E_2^{3,3,9}\stackrel{d_2}{\rightarrow}E_2^{2,2,11}\stackrel{d_2}{\rightarrow}E_2^{1,1,13}\stackrel{d_2}{\rightarrow}E_2^{0,0,15}$$
$$E_2^{3,3,8}\stackrel{d_2}{\rightarrow}E_2^{2,2,10}\stackrel{d_2}{\rightarrow}E_2^{1,1,12}\stackrel{d_2}{\rightarrow}E_2^{0,0,14}, \quad E_2^{3,3,10}\stackrel{d_2}{\rightarrow}E_2^{2,2,12}\stackrel{d_2}{\rightarrow}E_2^{1,1,14}.$$
Note that the $\mm{Sq}^2$ action on $H^\ast(\mm{M}\eta;\mb{Z}_2)$ determines these $d_2$ differentials. So the desired results follow by the $\mm{Sq}^2$ action.
\end{proof}

\begin{lemma}\label{pert2}
$E_3^{6,10,2}=0,$ $E_\infty^{s+5,11,2}=E_1^{s+5,11,2},$ $E_\infty^{s+3,12,2}=E_1^{s+3,12,2}$ $s\ge 0$.
\end{lemma}
\begin{proof}
 By $\mathrm{Sq}^2U=x$, we have $d_2(h_1^3\omega U)=h_1^2\omega xU$ and $E_3^{6,10,2}=0.$ In the cobar complex $cB^\ast(N^\ast)$,
$\delta(h_0^sh_2\omega xU)=h_0^sh_2\omega  h_1 U$. Since that $h_0^sh_2\omega  h_1$ is not a generator of $Ext^{s,t}_{\mathscr{A}_2}(\mathbb{Z}_2, \mathbb{Z}_2)$, there exists $\alpha$ such that $\delta\alpha=h_0^sh_2\omega  h_1$ in $cB(\mathscr{A}^\ast_2)$. Then $h_0^sh_2\omega xU+\alpha U$ represents a cycle in $cB(N^\ast)$, thus its dual is not a image of $\partial$ in the bar complex $B(N)$. By lemma \ref{pert}, $E_\infty^{s+5,11,2}=E_1^{s+5,11,2}$. The proof for $E_1^{s,12,2}$ is similar to $E_1^{s+5,11,2}$.
\end{proof}

\begin{lemma}\label{pert3}
$E_\infty^{3,8,4}=E_1^{3,8,4},$ $E_\infty^{s+4,8,4}=E_1^{s+4,8,4}$ for $s\ge 0$, $E_\infty^{s,9,4}=E_1^{s,9,4}$ for $s=4$ and $5$, $E_\infty^{6,10,4}=E_1^{6,10,4}$.
\end{lemma}
\begin{proof}
In the cobar complex $cB(N^\ast)$, $\delta(c_0 x^2U)=0$ by $\mathrm{Sq}^4U=0$ and $\mathrm{Sq}^2xU=0$. Then $c_0 x^2U$ represents a cycle in $cB(N^\ast)$, thus its dual is not a image of $\partial$ in the bar complex $B(N)$. By lemma \ref{pert}, $E_\infty^{3,8,4}=E_1^{3,8,4}$. The proofs of other cases $E_1^{s,m,4}$ are similar to $E_1^{3,8,4}$.
\end{proof}

\begin{lemma}\label{pert4}
$E_\infty^{s+3,12,0}=E_1^{s+3,12,0}$ for $s\ge 0$, $E_\infty^{s+4,14,0}=E_1^{s+4,14,0}$ for $0\le s\le 2$.
\end{lemma}
\begin{proof}
By Lemma \ref{pert2}, $E_3^{s+3,12,0}=E_1^{s+3,12,0}$ and $E_3^{s+4,14,0}=E_1^{s+4,14,0}$. Note that $\mm{Sq}^\mm{I}U=0$ for each admissible sequence $\mm{I}=(i_1,i_2,\cdots,i_r)$ with $i_1+i_2+\cdots +i_r\ge 3$. So $h_0^s\tau U$ and $h_0^sd_0U$ are nontrivial cycles in $B(N)$, and the desired results follow.
\end{proof}

The nontrivial $E_3^{s,m,n}$-terms for $12\le n+m \le 14$ are listed as follow.
$$E_3^{0,0,13}=\{vx^2U, x\mathrm{Sq}^4uU=\mathrm{Sq}^6uU\}\quad E_3^{1,1,12}=\{h_1x^2\mathrm{Sq}^1uU\} $$
$$E_3^{1,1,13}=\{h_1vx^2U\}\quad E_3^{2,2,11}=\{h_1^2ux^2U , h_1^2vxU\}$$
$$E_3^{3,3,10}=\{h_0^2h_2(\mathrm{Sq}^2\mathrm{Sq}^1uU+x\mathrm{Sq}^1uU)\}\quad E_3^{1,3,9}=\{h_2uxU , h_2vU\}$$
$$E_3^{1,3,11}=\{h_2ux^2U , h_2vxU, h_2\mathrm{Sq}^4uU\}$$
$$E_3^{2,6,7}=\{h_2^2uU\}\quad E_3^{2,6,8}=\{h_2^2\mathrm{Sq}^1uU\}$$
$$E_\infty^{3,8,4}=E_1^{3,8,4}=\{c_0x^2U\}; \quad E_\infty^{s+4,8,4}=E_1^{s+4,8,4}=\{h_0^s\omega x^2U\}\quad s\ge 0;$$
$$E_\infty^{s,9,4}=E_1^{s,9,4}=\{h_1c_0x^2U, h_1\omega x^2U\}\quad s=4,5;$$
$$ E_\infty^{6,10,4}=E_1^{6,10,4}=\{h_1^2\omega x^2U\};$$
$$E_\infty^{s+5,11,2}=E_1^{s+5,11,2}=\{h_0^sh_2\omega xU\}\quad 0\le s\le 2;$$
$$E_\infty^{s+3,12,0}=E_1^{s+3,12,0}=\{h_0^s\tau U\}\quad E_\infty^{s+3,12,2}=E_1^{s+3,12,2}=\{h_0^s\tau xU\}\quad s\ge 0;$$
$$E_\infty^{s+4,14,0}=E_1^{s+4,14,0}=\{h_0^sd_0U\}\quad 0\le s\le 2.$$
By the degree of $d_3$, $E_4^{s,m,n}=E_3^{s,m,n}.$

\begin{lemma}\label{infty}
$E_\infty^{0,0,13}=E_5^{0,0,13}=\{vx^2\}$,  $E_\infty^{1,3,9}=E_5^{1,3,9}=\{h_2(ux+v)U\}$, $$E_5^{2,6,7}=0, \quad E_\infty^{1,3,11}=E_5^{1,3,11}=\{h_2vxU\},$$

$E_5^{2,6,8}=\{h_2^2\mathrm{Sq}^1uU\}$, but $E_7^{2,6,8}=0$.
\end{lemma}
\begin{proof}
By the degree of the differential $d_r$ $(r\ge 5)$, we have $E_\infty^{1,3,11}=E_5^{1,3,11}$, $E_\infty^{1,3,9}=E_5^{1,3,9}$ and $E_\infty^{0,0,13}=E_5^{0,0,13}$.
By lemma \ref{0015}, $E_4^{0,0,15}=\{x^2\mathrm{Sq}^4uU,u\mathrm{Sq}^1uU\}$. By the $\mathrm{Sq}^1$ action, $E_2^{1,3,12}=0$. Now consider the following differentials
$0\to E_4^{1,3,9}\stackrel{d_4}{\rightarrow}E_4^{0,0,13},$ $ 0\to E_4^{2,6,7}\stackrel{d_4}{\rightarrow}E_4^{1,3,11}\stackrel{d_4}{\rightarrow} E_4^{0,0,15}$ and
$0\to E_4^{2,6,8}\stackrel{d_4}{\rightarrow}E_4^{1,3,12}=0.$
Note that the $\mm{Sq}^4$ action on $H^\ast(\mm{M}\eta;\mb{Z}_2)$ determines these $d_4$ differentials. By the $\mm{Sq}^4$ action and the listed $E_4^{s,m,n}$-terms, we have
 $E_5^{1,3,9}=\{h_2(ux+v)U\}$, $E_5^{0,0,13}=\{vx^2\}$, $E_5^{2,6,7}=0$, $E_5^{1,3,11}=\{h_2vxU\}$ and
$E_6^{2,6,8}=E_3^{2,6,8}$.

Indeed, a differential $d_r$ of AAHSS implies a part of a boundary homomorphism in the bar complex $B(N)$. Consider the following boundary homomorphism in the bar complex $B(N)$
$$(\mathrm{Sq}^4\otimes\mathrm{Sq}^4+\mathrm{Sq}^7\otimes\mathrm{Sq}^1+\mathrm{Sq}^6\otimes\mathrm{Sq}^2)\otimes \mathrm{Sq}^1uU+$$
$$(\mathrm{Sq}^2\otimes\mathrm{Sq}^4+\mathrm{Sq}^5\otimes\mathrm{Sq}^1) \otimes (\mathrm{Sq}^2\mathrm{Sq}^1u+x\mathrm{Sq}^1u)U+ $$
$$(\mathrm{Sq}^4\otimes\mathrm{Sq}^1+\mathrm{Sq}^2\otimes\mathrm{Sq}^3+\mathrm{Sq}^1\otimes\mathrm{Sq}^4)\otimes (\mathrm{Sq}^4u+ux^2)U\stackrel{\partial}{\rightarrow}$$
$$\mathrm{Sq}^2 \otimes (u^2+x^2\mathrm{Sq}^2\mathrm{Sq}^1u)U \quad \mod \quad M_{15}.$$
Note that $E_6^{1,1,14}=\{h_1u^2U\}$. Hence $d_6(h_2^2\mathrm{Sq}^1uU)=h_1u^2U$, $E_7^{2,6,8}=0$.
\end{proof}

\begin{lemma}
All elements in $E_4^{3,3,10}$, $E_4^{1,1,13}$, $E_4^{1,1,12}$ and $E_4^{2,2,11}$ survive to $E_\infty$.
\end{lemma}
\begin{proof}
The desired results follow by the degrees of the differentials $d_r$ ($r\ge 4$) about these $E_4$-terms in the lemma.
\end{proof}

\subsection{The part of mod 3}\label{AAHSSmod3}
Recall that
$$E_1^{s,m,n}=\mm{Tor}^{\mathcal{A}}_{s,s+m}(\mathbb{Z}_3,\mathbb{Z}_3)\otimes N^n\Rightarrow \mm{Tor}^{\mathcal{A}}_{s,s+m}(\mathbb{Z}_3,N)$$
$$d_1=j_\ast\partial_\ast:E_1^{s,m,n}\to E_1^{s-1,m,n+1},\quad d_r:E_r^{s,m,n}\to E_r^{s-1,m+1-r,n+r}$$
where $N^n=H^n(\mm{M}\eta;\mathbb{Z}_3)$, $N=H^\ast(\mm{M}\eta;\mb{Z}_3)$.

For $12\le m+n\le 14$, the nontrivial $E_1$-terms are listed as follows.
$$E_1^{s,0,12}=\{a_0^sx\beta z_9U\},\quad E_1^{s,8,4}=\{a_0^sx_8x^2U\},\quad  E_1^{1,3,9}=\{h_{1,0}z_9U\},$$
$$E_1^{2,10,2}=\{b_0xU\}, \quad E_1^{3,10,2}=\{a_0b_0xU\},\quad E_1^{s,12,0}=\{a_0^sx_{12}U\}$$
$$E_1^{1,3,10}=\{h_{1,0}\beta z_9U\},\quad E_1^{1,11,2}=\{h_{1,0}x_8xU\} ,\quad E_1^{3,13,0}=\{b_0h_{1,0}U\} ,$$
$$ E_1^{s,0,13}=\{a_0^sz_{13}U, a_0^sx^2z_{9}U\}, $$
$$E_1^{1,3,11}=\{h_{1,0}xz_{9}U\},\quad E_1^{2,10,4}=\{b_0x^2U\}, \quad E_1^{3,10,4}=\{a_0b_0x^2U\} ,$$
$$E_1^{s,12,2}=\{a_0^sx_{12} xU\} \quad s\ge 3, \quad E_1^{s,0,14}=\{a_0^s\beta z_{13}U, a_0^sx^2\beta z_{9}U\}.$$

\begin{lemma}\label{s014}
 $E_{2}^{s+1,0,13}=E_{2}^{s,0,14}=E_{2}^{s,0,12}=0$ for $s\ge 0$.
\end{lemma}
\begin{proof}
By the $\beta$ action on $H^\ast(\mm{M}\eta;\mathbb{Z}_3)$, $d_{1}:E_{1}^{s+1,0,13}\to E_{1}^{s,0,14}$ and $d_{1}:E_{1}^{s+1,0,11}\to E_{1}^{s,0,12}$ are isomorphisms.
\end{proof}

By the $E_1$-terms and Lemma \ref{s014}, $ \mm{Tor}^{\mathcal{A}}_{s,s+13}(\mathbb{Z}_3,N)=0$
 for $s>3$. Thus we only calculate $\mm{Tor}^{\mathcal{A}}_{s,s+14}(\mathbb{Z}_3,N)$ for $0\le s \le 1$, and $\mm{Tor}^{\mathcal{A}}_{s,s+12}(N,\mathbb{Z}_3)$ for $ s\ge 2$ due to the degree of the Adams differential.

\begin{lemma}\cite{MilnorStasheff}
Let $U$ be the stable Thom class of $\mm{M}\eta$, then
$$\mathcal{P}^1U=r_3(p_1(\eta))U$$  where $r_3$ is the mod $3$ reduction.
\end{lemma}

\begin{corollary}\label{313}
If $p_1(\eta)= 0 \mod 3$, then $E_\infty^{3,13,0}=E_1^{3,13,0}$. Otherwise,  $E_\infty^{3,13,0}=0$.
\end{corollary}
\begin{proof}
	The desired result follows by the fact that $\mathcal{P}^1U$ determines the differential $d_4:E_4^{3,13,0}\to E_4^{2,10,4}$.
\end{proof}

\begin{lemma}\label{1112}
  $E_\infty^{1,11,2}=E_1^{1,11,2}$, $E_\infty^{2,10,2}=E_1^{2,10,2}$,
 $E_\infty^{3,10,2}=E_1^{3,10,2},$

 $E_\infty^{s,12,0}=E_1^{s,12,0}$ where $s\ge 3$, $E_\infty^{s,8,4}=E_1^{s,8,4}$ where $s\ge 1$.
\end{lemma}
\begin{proof}
By $\mathcal{P}^1x=x^3=0$ and $\beta x=0$, the element of $E_1^{s,m,2}$ is a permanent cycle; since $E_1^{2,m,14-m}=0$ for $ 11\le m \le 14$, $E_\infty^{1,11,2}=E_1^{1,11,2}$.
By $E_1^{3,m,11-m}=0$ for $ 10\le m \le 11$, $E_\infty^{2,10,2}=E_1^{2,10,2}$. Similarly $E_\infty^{3,10,2}=E_1^{3,10,2}$.
By $\beta U=0$, the element of $E_1^{s,12,0}$ is a permanent cycle, thus $E_\infty^{s,12,0}=E_1^{s,12,0}$. By $\beta x^2=0$, the element of $E_1^{s,8,4}$ is a permanent cycle$;$ by $E_1^{s+1,m,11-m}=0$ for $ 8\le m \le 11$, $E_\infty^{s,8,4}=E_1^{s,8,4}$.
\end{proof}

\begin{lemma}\label{0013}
   $E_\infty^{0,0,13}=E_5^{0,0,13}=\{x^2z_9U\}$,
 $E_5^{1,3,11}=0,$ $ E_\infty^{1,3,10}=E_1^{1,3,10}.$
\end{lemma}
\begin{proof}
Since that the $\mathcal{P}^1$ action on $H^\ast(\mm{M}\eta;\mb{Z}_3)$ determines the following differentials
$d_4: E_4^{1,3,9}{\rightarrow}E_4^{0,0,13}$ and $d_4: E_4^{1,3,11}{\rightarrow}E_4^{0,0,15}$, we have
 $E_5^{0,0,13}=\{z_{13}U=kx^2z_9U\}$ and $E_5^{1,3,11}=0.$

Finally the desired results follow by the degree of the differential $d_r$.
\end{proof}

\begin{thebibliography}{100}


\bibitem{Adams1960} {Adams}, J. F. On the non-existence of element of Hopf invariant one. Ann. of Math. 72 (1960), 20-104.

\bibitem{Adams1966} {Adams}, J. F. On the groups J(x), IV. Topology 5 (1966), 21-71.

\bibitem{Adams1974} {Adams}, J. F. Stable homotopy and generalised homology. Chicago: University of Chicago Mathematics Lecture Notes 1971.


\bibitem{AdamsPriddy} {Adams}, J. F.; Priddy, S. B. Uniqueness of BSO. Math. Proc. Cambridge Philos. Soc. 80 (1976), no. 3, 475-509.

\bibitem{Aikawa} {Aikawa}, T. 3-dimensional cohomology of the mod $p$ Steenrod algebra, Math. Scand. 47 (1980), 91-115.

\bibitem{Aloff} { Aloff}, S., Wallach, N. An infinite family of 7-manifolds admitting positively curved Riemannian structures. Bull. Am. Math. Soc. 81 (1975), 93-97.

\bibitem{ABP1967} {Anderson}, D. W.; Brown, E. H., Jr.; Peterson, F. P. The structure of the Spin cobordism ring. Ann. of Math. (2) 86 (1967), 271-298.

\bibitem{AtiyahBottShapiro} {Atiyah}, M. F.; Bott, R.; Shapiro, A. Clifford modules.
Topology 3 (1964), no. suppl, suppl. 1, 3-38.

\bibitem{AtiyahTodd1960} {Atiyah}, M. F.; Todd, J. A. On complex Stiefel manifolds. Proc. Cambridge Philos. Soc. 56 (1960), 342-353.


\bibitem{Ba} {Bazaikin}, Y.V. On a certain family of closed 13-dimensional Riemannian manifolds of positive curvature. Sib. Math. J. 37, No. 6 (1996), 1219-1237.

\bibitem{Be} { Berger, M.} Les vari$\acute{e}$t$\acute{e}$s riemanniennes homog$\grave{e}$nes normales simplement connexes $\grave{a}$ courbure strictement positive. Ann. Scuola Norm. Sup. Pisa 15 (1961), 179-246.

\bibitem{Borel1953} {Borel}, A. Sur la cohomologie des espaces fibr$\grave{e}$s principaux et des espaces homog$\acute{e}$nes de groupes de Lie compacts. (French) Ann. of Math. (2) 57 (1953), 115-207.


\bibitem{Bott1959} {Bott}, R. The stable homotopy of the classical groups. Ann. of Math. (2) 70 (1959), 313-337



\bibitem{Brumfiel} {Brumfiel}, G.
On the homotopy groups of BPL and PL/O. Ann. of Math. (2) 88 (1968), 291-311.




\bibitem{E1} { Eschenburg}, J. Freie isometrische Aktionen auf kompakten Liegruppen mit positiv gekr$\ddot{u}$mmten Orbitr$\ddot{a}$umen. Schriftenreihe Math. Inst. Univ. M$\ddot{u}$nster (2) 32 (1984)

\bibitem{E2} {Eschenburg}, J. Cohomology of biquotients.
Manuscripta Math. 75 (1992), no. 2, 151-166.

\bibitem{E3} {Eschenburg}, J, Kollross, A and Shankar, K. Free isometric circle actions on compact symmetric spaces. Geom. Dedicata 103 (2003), 35-44.

\bibitem{FangWang2010} {Fang}, F., Wang, J. Homeomorphism classification of complex projective complete intersections of dimensions 5, 6 and 7. Math. Z. 266 (2010), no. 3, 719-746.



\bibitem{FL} {Florit}, L., Ziller, W. On the topology of positively curved Bazaikin spaces.
J. Eur. Math. Soc. 11 (2009), no. 1, 189-205.

\bibitem{Giambalvo} {Giambalvo}, V. \emph{On $\left \langle 8 \right \rangle$-cobordism}. Ill. J. Math. 15 (1971), 533-541.

\bibitem{HebeJoach} {Hebestreit}, F., Joachim, M.
Twisted spin cobordism and positive scalar curvature.
J. Topol. 13 (2020), no. 1, 1-58.

\bibitem{HiltonStammbach1997} {Hilton}, P. J., Stammbach, U. A course in homological algebra. Second edition. Graduate Texts in Mathematics, 4. Springer-Verlag, New York, 1997.

\bibitem{Hir1963} Hirsch, M.
Obstruction theories for smoothing manifolds and maps.
Bull. Amer. Math. Soc. 69 (1963), 352-356.

\bibitem{HoRa1995} {Hovey}, M.A., Ravenel, D.C.
The 7-connected cobordism ring at p=3.
Trans. Amer. Math. Soc. 347 (1995), no. 9, 3473-3502.



\bibitem{KerM} {Kerin}, M. On the curvature of biquotients. Math. Ann. 352 (2012), no. 1, 155-178.

\bibitem{KervaMilnor} {Kervaire}, M. A.; Milnor, J. W. Groups of homotopy spheres.  I. Ann. of Math. (2) 77 (1963), 504-537.

\bibitem{KirbSieb} {Kirby}, R. C.; Siebenmann, L. C. Foundational Essays on Topological Manifolds, Smoothings and Triangulations. Ann. of Math. Studies 88, Princeton Univ. Press, Princeton, NJ, 1977.

\bibitem{K} {Kreck}, M. Surgery and duality. Ann. of Math. (2) 149 (1999), no. 3, 707-754.

\bibitem{KS1} {Kreck}, M.; Stolz, S. A diffeomorphism classification of 7-dimensional homogeneous Einstein manifolds with $SU(3)\times SU(2)\times U(1)$-symmetry.
Ann. of Math. (2) 127 (1988), no. 2, 373-388.

\bibitem{KS} {Kreck}, M.; Stolz, S. Some nondiffeomorphic homeomorphic homogeneous 7-manifolds with positive sectional curvature. J. Differential Geom. 33 (1991), no. 2, 465-486.

\bibitem{kb1} {Kruggel}, B. Kreck-Stolz invariants, normal invariants and the homotopy classification of generalized Wallach spaces. Quart. J. Math. Oxford Ser. (2) 49 (1998), 469-485.

\bibitem{kb2} {Kruggel}, B. Homeomorphism and diffeomorphism classification of Eschenburg spaces. Quart. J. Math. Oxford Ser. (2) 56 (2005), 553-577.


\bibitem{MilnorStasheff} {Milnor}, J. W. Characteristic classes, Mimeographed Notes, Princeton Univ, Princeton, N. J., 1957.


\bibitem{Milnor1965} {Milnor}, J. W.
Lectures on the cobordism theorem.
Notes by L. Siebenmann and J. Sondow Princeton University Press, Princeton, N.J. 1965

\bibitem{Quillen} {Quillen}, D. G. The Adams conjecture. Topology 10 (1971), 1-10.



\bibitem{Ravenel1986} {Ravenel}, D. C. Complex cobordism and stable homotopy groups of spheres. Pure and Applied Mathematics, 121. Academic Press, Inc., Orlando, FL, 1986.

\bibitem{RourkeSanderson}  {Rourke}, C. P.; Sanderson, B. J. Introduction to piecewise-linear topology. Reprint Springer Study Edition Springer-Verlag, Berlin-New York, 1982.

\bibitem{Rudyak} { Rudyak}, Y. B. On Thom Spectra, Orientability, and Cobordism, Springer Mongraphs in Mathematics,Corrected 2nd printing, Springer, 2008.

\bibitem{Si} {Singer}, W. M. Steenrod Squares in Spectral Sequences. Mathematical Surveys and Monographs 129, American Mathematical Society, 2006.

 \bibitem{Stong1963} {Stong}, R. E. Determination of $H^\ast(BO(k,\cdots , \infty),\mathbb{Z}_2)$ and $H^\ast(BU(k,\cdots , \infty),\mathbb{Z}_2)$. Trans. Amer. Math. Soc. 107 (1963), 526-544.

\bibitem{Sw} { Switzer}, R. W.
 Algebraic Topology-Homotopy and Homology. Springer, Berlin Heidelberg New York 1975.

\bibitem{Peter1993} {Teichner}, P. On the signature of four-manifolds with universal covering spin. Math. Ann. 295 (1993), no. 4, 745-759.

\bibitem{Toda} Toda, H. Composition Methods in Homotopy Groups of Spheres, Ann. of Math. Studies 49, 1962.

\bibitem{Wall1999} {Wall}, C. T. C. Surgery on Compact Manifolds, London Math. Soc. Monographs 1, Academic Press, New York, 1970.

\bibitem{Wallach} { Wallach, N.} Compact homogeneous Riemannian manifolds with strictly positive curvature. Ann. of Math. (2) 96 (1972), 277-295.


\bibitem{WhiteGW1978} {Whitehead}, G. W. Elements of Homotopy Theory, Grad. Texts in Math. 61, Springer-Verlag, New York, 1978.
\end{thebibliography}


\end{document}











