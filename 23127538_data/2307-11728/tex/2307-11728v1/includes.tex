
\usepackage[margin=1in]{geometry}  
\usepackage{amsmath}               
\usepackage{amsfonts}              
\usepackage{amsthm}      
%\usepackage{mathrsfs}                          
\usepackage{amssymb}
\usepackage{bbm}

\usepackage[utf8]{inputenc}
\usepackage{mathtools}
%\usepackage{cite}
%\usepackage{mathabx}


\usepackage{color}   %May be necessary if you want to color links
\usepackage{hyperref}
\hypersetup{
    colorlinks=true, %set true if you want colored links
    linktoc=all,     %set to all if you want both sections and subsections linked
    linkcolor=black,  %choose some color if you want links to stand out
}

\newcommand{\sslash}{\mathbin{/\mkern-6mu/}}	%for double slash
\def\acts{\curvearrowright}
\DeclarePairedDelimiter\abs{\lvert}{\rvert}%
\DeclarePairedDelimiter\norm{\lVert}{\rVert}%

% Swap the definition of \abs* and \norm*, so that \abs
% and \norm resizes the size of the brackets, and the 
% starred version does not.
\makeatletter
\let\oldabs\abs
\def\abs{\@ifstar{\oldabs}{\oldabs*}}
%
\let\oldnorm\norm
\def\norm{\@ifstar{\oldnorm}{\oldnorm*}}


\makeatother


%\theoremstyle{definition}
%\newtheorem{thm}{Theorem}
%\newtheorem{lem}{Lemma}
%\newtheorem{prop}{Proposition}
%\newtheorem*{claim}{Claim}
%\newtheorem{cor}{Corollary}
%\newtheorem*{conj}{Conjecture}
%\newtheorem{defn}{Definition}
%\newtheorem{example}{Example}
%\newtheorem*{fact}{Fact}
%\newtheorem{remark}{Remark}
%\newtheorem*{notation}{Notation}
%\newtheorem*{exercise}{Exercise}
%\newtheorem{question}{Question}

\theoremstyle{definition}
\newtheorem{thm}{Theorem}[section]
\newtheorem{lem}{Lemma}[section]
\newtheorem{prop}{Proposition}[section]
\newtheorem*{claim}{Claim}
\newtheorem{cor}{Corollary}[section]
\newtheorem*{conj}{Conjecture}
\newtheorem{defn}{Definition}[section]
\newtheorem{example}{Example}[section]
\newtheorem*{fact}{Fact}
\newtheorem{remark}{Remark}[section]
\newtheorem*{notation}{Notation}
\newtheorem*{exercise}{Exercise}
\newtheorem{question}{Question}




\DeclareMathOperator{\id}{id}
\DeclareMathOperator{\cost}{cost}
\DeclareMathOperator{\gcost}{gcost}

\DeclareMathOperator{\PSL}{PSL}
\DeclareMathOperator{\SL}{SL}

\DeclareMathOperator{\Aut}{Aut}
\DeclareMathOperator{\Isom}{Isom}
\DeclareMathOperator{\stab}{stab}
\DeclareMathOperator{\support}{support}

\DeclareMathOperator{\Sym}{Sym}
\DeclareMathOperator{\Cay}{Cay}
\DeclareMathOperator{\Sch}{Sch}


\DeclareMathOperator{\ima}{im}

\let\oldphi\phi
\let\phi\varphi
\let\oldemptyset\emptyset
\let\empt\varnothing


\newcommand{\EE}{\mathbb{E}}      % for Real numbers

\newcommand{\RR}{\mathbb{R}}      % for Real numbers


\newcommand{\e}{\varepsilon} 
\newcommand{\ZZ}{\mathbb{Z}}      % for Integers
\newcommand{\QQ}{\mathbb{Q}}      % for Integers
\newcommand{\PP}{\mathbb{P}}      % for Integers
\newcommand{\MM}{\mathbb{M}}      % for Integers
\newcommand{\Maper}{\mathbb{M}^\text{aper}} 
\newcommand{\Mfin}{\mathbb{M}^{<\infty}} 
\newcommand{\model}{\mathscr{M}} 
\newcommand{\DD}{\mathscr{D}}      % for Integers

\newcommand{\NN}{\mathbb{N}}      % for Integers
\newcommand{\HH}{\mathbb{H}}      % for Integers
\newcommand{\dprok}{d_{\text{prok}}}      

\newcommand{\1}{\mathbbm{1}}

\newcommand\restr[2]{{% we make the whole thing an ordinary symbol
  \left.\kern-\nulldelimiterspace % automatically resize the bar with \right
  #1 % the function
  \vphantom{\big|} % pretend it's a little taller at normal size
  \right|_{#2} % this is the delimiter
  }}

\usepackage{color}
\usepackage{mathrsfs}
\newcommand{\Rel}{\mathcal{R}} 
\newcommand{\MMo}{{\mathbb{M}_0}}
\newcommand{\Marrow}{\overrightarrow{\mathbb{M}_0}} 
\newcommand{\muarrow}{\overrightarrow{\mu_0}} 
\newcommand{\eNet}{{}_\e \mathcal{N}} 
\newcommand{\Narrow}{\overrightarrow{\eNet_0}} 

\newcommand{\UH}{\mathcal{U}(\mathcal{H})} 
\newcommand{\SH}{S(\mathcal{H})} 
\newcommand{\OH}{O(\mathcal{H})} 

\newcommand{\HHH}{\mathcal{H}}

\newcommand{\PPP}{\mathscr{P}} 
\newcommand{\law}{\mathcal{L}} 

\newcommand{\eh}{\mathcal{E}} 

\DeclareMathOperator{\cgraph}{CGraph}
\DeclareMathOperator{\aper}{Aper}
\DeclareMathOperator{\graph}{Graph}
\DeclareMathOperator{\covol}{covol}
\DeclareMathOperator{\intensity}{int}
\DeclareMathOperator{\diam}{diam}
\DeclareMathOperator{\Var}{Var}
\DeclareMathOperator{\RG}{RG}
