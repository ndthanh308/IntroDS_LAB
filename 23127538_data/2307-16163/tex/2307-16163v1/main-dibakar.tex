\documentclass[pre,aps,twocolumn,superscriptaddress,floatfix]{revtex4-2}
\usepackage{graphicx,amsmath,amssymb}
\usepackage[usenames]{color}
\usepackage{cancel}

\begin{document}
\title{Universality of synchronization in weighted simplicial complexes}

\author{S. Nirmala Jenifer}
\affiliation{Department of Physics, Bharathidasan University, Tiruchirappalli 620 024, Tamil Nadu. India}

\author{Dibakar Ghosh}
\affiliation{Physics and Applied Mathematics Unit, Indian Statistical Institute, 203 B. T. Road, Kolkata-700108, India}

\author{Paulsamy Muruganandam}
\affiliation{Department of Physics, Bharathidasan University, Tiruchirappalli 620 024, Tamil Nadu. India}

\date{\today}

\begin{abstract}

We present a universal formula for determining the synchronizability of large randomized weighted simplicial complexes. We analyze the synchronizability of simplicial complexes with various network topologies and intensity distributions by calculating the eigenratio and cost. Our analysis shows that it is possible to determine the synchronizability of a large randomized weighted simplicial complex based on the mean degrees, coupling strengths, and intensities without explicitly constructing connectivity matrices and finding their eigenvalues. Further, this universal formula provides an efficient way to manipulate the synchronizability of complex systems with higher-order interactions by adjusting the degrees, weights, and coupling strengths.

\end{abstract}

\maketitle

Complex systems are formed when a large number of dynamical units interact with each other, giving rise to emergent properties that are distinct from those of the individual subsystems \cite{Boccaletti2006}. Such systems are ubiquitous and occur naturally in entities such as the brain or are man-made, such as the internet or financial markets \cite{Kwapien2012}. Researchers have long focused on modelling and studying these systems~\cite{Kuramoto1975, Iacopini2019, Strogatz1993}. Complex networks represent the dynamical units, and their interactions as nodes and links are a common way to model such systems. However, these networks typically only account for pairwise interactions, whereas many complex systems involve interactions among three or more units. To gain a more comprehensive understanding of these systems, one must incorporate higher-order networks, such as hypergraphs and simplicial complexes, to capture these higher-order interactions \cite{Majhi2022}.

Simplicial complexes are used to model complex systems with higher-order interactions \cite{Iacopini2019, Battiston2021, Battiston2020, Boccaletti2023}. A simplicial complex consists of $d$-simplices, where $d$ is the dimension of the simplex. A $0$-dimensional simplex is a node, a $1$-dimensional simplex is a link, a $2$-dimensional simplex is a triangle, and so on. A $d$-dimensional simplicial complex consists of simplices up to dimension $d$. They provide a good representation of complex systems with higher-order interactions \cite{Grilli2017, Mayfield2017, Battiston2021, Alvarez2021}. In most studies, these simplicial complexes are unweighted, meaning all the links and triangles have the same weight \cite{Anwar2022, Anwar2023}. However, it has been observed that in most real-world systems, this is not feasible. For instance, in collaboration networks, if there is a three-author paper, not every pair of authors has to produce research papers by themselves. But in unweighted simplicial complexes, if there is a three-author paper, then there is also the presence of all the possible two-author research papers, and this leads to information loss, which one could avoid by introducing proper weights for links \cite{Baccini2022}. Bianconi et al.  have developed a model to distribute weights to links in terms of bare affinity weights and topological weights \cite{Courtney2017}. Every simplex has bare-affinity weights, while the topological weights represent whether a particular link has a contribution other than being part of a triangle. Therefore, weighted simplicial complexes provide a more realistic representation of higher-order networks.

After accurately modelling complex systems, we can better understand their emergent properties, such as synchronization. As a result, we can control the synchronization that occurs in natural or artificial systems. Synchronization occurs when individual dynamical systems adjust their properties to have common dynamics \cite{Kuramoto1975}. Neuronal synchronization can cause epilepsy, while synchronization from a healthy to an infected state can cause an epidemic outbreak, and so on. Synchronization can be both constructive and destructive. However, it is possible to promote synchronization in desired systems and inhibit it where it leads to destruction. This can be achieved by determining under what circumstances the system goes into synchronization and desynchronization and whether synchronization is possible. Two main parameters are used to calculate the synchronizability of the system: eigenratio and the cost of the connections of nodes in the system. The lower the eigenratio and cost, the system has more synchronizability \cite{Zhou2006}.

To calculate the eigenratio and costs, we need to construct the connectivity matrices, such as adjacency matrices and Laplacian matrices, and then find the eigenvalues of these matrices. Writing connectivity matrices for large, complex systems, such as connection networks, collaboration networks, and neuronal networks, can be challenging, and collecting the necessary data is not an easy task. It becomes even more difficult when we include higher-order interactions, as we need to construct the adjacency tensors for these interactions. Even for three-body interactions, the adjacency tensor contains $N \times N \times N$ elements. In 2006, Zhou et al. \cite{Zhou2006} developed an approach to calculate eigenratios and costs based on the heterogeneity of the system intensities. This approach significantly reduced the computational cost of constructing connectivity matrices and evaluating their eigenvalues. We extend this approach to randomized weighted simplicial complexes. As a result, we can determine the synchronizability parameters solely from the heterogeneity of the intensities, specifically the maximum and minimum intensities of a node as part of links ($d=1$), triangles ($d=2$), and so on and from their coupling strengths. In this letter, we shall discuss the mathematical formulation, derive universal formulas for determining the synchronization without explicitly computing the eigenvalues of huge matrices and corroborate with numerical results.

In order to derive the universal formulas for the eigenratio and the cost, we first write the dynamical equations for the weighted simplicial complex. Then, we deduce the corresponding variational equations and modify these equations in terms of the effective matrix $M$. The eigenvalues of the effective matrix determine the stability of the synchronized state. Since the effective matrix is a zero row sum matrix, the first eigenvalue will be zero, corresponding to the mode along the synchronization manifold. We can find the synchronizability of the system from the ratio between $\lambda_N$ and $\lambda_2$.  For simplicity, we consider the simplicial complex of dimension $2$, which we can extend to any dimension.

The equation of motion for a 2d-weighted simplicial complexes can be written as,
\begin{align}
  \mathbf{ \dot{x}}_i = & \mathbf{f}(\mathbf{x}_i) + \sigma_1 \sum_{j=1}^N a_{ij}^{(1)} \omega_{ij}^{(1)} \mathbf{g}^{(1)}(\mathbf{x}_i,\mathbf{x}_j)  \notag \\ & + \sigma_2{\sum_{j=1}^N}{\sum_{k=1}^N} a_{ijk}^{(2)} \omega_{ijk}^{(2)} \mathbf{g}^{(2)}(\mathbf{x}_i,\mathbf{x}_j,\mathbf{x}_k). \label{eq:1}
\end{align}
Here, $\mathbf{f}(\mathbf{x}_i)$ represents the dynamics of the uncoupled oscillators, where $\mathbf{x}_i$ is the $i$th oscillator's state vector of dimension $m$. ${\omega_{ij}}^{(1)}$ and $\omega_{ijk}^{(2)}$ are the topological weights of links and triangles, respectively. $a_{ij}^{(1)}$ and $a_{ijk}^{(2)}$ are the elements of the adjacency matrix $A^{(1)}$ and adjacency tensor $A^{(2)}$. $a_{ij}^{(1)} = 1$ if the nodes $i$ and $j$ form a link, or $a_{ij}^{(1)} = 0$ otherwise. Also $a_{ijk}^{(2)} = 1$ if the nodes $i$, $j$, and $k$ form a triangle, or $a_{ijk}^{(2)} = 0$ otherwise. Here $\sigma_1$ and $\sigma_2$ are the coupling strengths of the links (pairwise) and triangles (non-pairwise), respectively. $\mathbf{g}^{(1)}(\mathbf{x}_i,\mathbf{x}_j)$ and $\mathbf{g}^{(2)}(\mathbf{x}_i,\mathbf{x}_j,\mathbf{x}_k)$ are the synchronization noninvasive functions \cite{Battiston2021}. At the synchronization state, they tend to zero, making the dynamics of the system resemble that of the uncoupled oscillators.

We can write the above equation as 
\begin{align}
\mathbf{ \dot{x}}_i = & \, \mathbf{f}(\mathbf{x}_i) + \sigma_1\sum_{j=1}^N a_{ij}^{(1)} \omega_{ij}^{(1)} \left[\mathbf{h}^{(1)} (\mathbf{x}_j) - \mathbf{h}^{(1)}(\mathbf{x}_i) \right] \notag \\  & \,  + \sigma_2{\sum_{j=1}^N}{\sum_{k=1}^N}a_{ijk}^{(2)} \omega_{ijk}^{(2)}\left[\mathbf{h}^{(2)}(\mathbf{x}_j, \mathbf{x}_k) - \mathbf{h}^{(2)}(\mathbf{x}_i, \mathbf{x}_i) \right], \label{eq:2:0}
\end{align}
where $\mathbf{h}^{(1)}(\mathbf{x}_j)$ and $\mathbf{h}^{(2)}(\mathbf{x}_j, \mathbf{x}_k) $ are coupling functions that couples nodes in links and triangles, respectively. The choice of coupling functions affects the synchronizability of the system \cite{Battiston2021}. So we have to choose the coupling functions in a way that the oscillators tend to synchronize and we can write,
\begin{align}
S_i^{(1)} = \sum_{j=1}^N{a_{ij}}^{(1)} \omega_{ij}^{(1)}\, \;\; \mathrm{and}\;\; 
S_i^{(2)}=\frac{1}{2}\sum_{j=1}^N\sum_{k=1}^N a_{ijk}^{(2)}\omega_{ijk}^{(2)}. \notag
\end{align}
Here $S_i^{(1)}$ and $S_i^{(2)} $ are the intensities of node $i$ to form links and triangles, i.e., the number of weighted links and weighted triangles incident on node $i$.  For randomized simplicial complexes with $K^{(1)}_{\mathrm{min}} \gg 1$, we can use mean-field approximation and this equation can be rewritten as,
\begin{align}
\dot{\mathbf x}_i = & \, \mathbf{f}(\mathbf{x}_i) + 
   \sigma_1\frac{S_i^{(1)}}{K_i^{(1)}}\sum_{j=1}^N{a_{ij}}^{(1)}\left(\mathbf{h}^{(1)}(\mathbf{x}_j) - \mathbf{h}^{(1)}(\mathbf{x}_i)\right) 
\notag \\ & + \sigma_2\frac{S_i^{(2)}}{K_i^{(2)}} 
   {\sum_{j=1}^N}{\sum_{k=1}^N}a_{ijk}^{(2)} \left(\mathbf{h}^{(2)}(\mathbf{x}_j,\mathbf{x}_k) - \mathbf{h}^{(2)}(\mathbf{x}_i,\mathbf{x}_i) \right). \label{eq:intensity} 
\end{align}
Now, we consider pairwise and non-pairwise local mean fields as
\begin{align}
\mathbf{\overline{H}}^{(1)}{(\mathbf{x}_i)} = \frac{1}{K_i^{(1)}}\sum_{j=1}^N{a_{ij}}^{(1)} \left(\mathbf{h}^{(1)}(\mathbf{x}_j) \right) \notag
\end{align}
due to the interaction between the nodes through links  and 
\begin{align}
\mathbf{\overline{H}}^{(2)}{(\mathbf{x}_i,\mathbf{x}_i})= \frac{1}{K_i^{(2)}}{\sum_{j=1}^N}{\sum_{k=1}^N}a_{ijk}^{(2)} \left(\mathbf{h}^{(2)}(\mathbf{x}_j,\mathbf{x}_k)\right) \notag
\end{align}
due to the interaction between nodes through triangles.
Eq.~\eqref{eq:intensity} can be written as,
\begin{align} 
\mathbf{\dot{x}}_i = &\, \mathbf{f}(\mathbf{x}_i) + \sigma_1S_i^{(1)}\left(\mathbf{\overline{H}}^{(1)}(\mathbf{x_i})-\mathbf{h^{(1)}}(\mathbf{x}_i)\right) \notag \\ & + \sigma_2{S_i^{(2)}}\left( \mathbf{\overline{H}}^{(2)}{(\mathbf{x}_i,\mathbf{x}_i}) - \mathbf{h}^{(2)}(\mathbf{x}_i,\mathbf{x}_i) \right). \label{eq:2}
\end{align}
So, the interaction between nodes in a large randomized weighted simplicial complex can be approximated as the interaction of a node with a mean field. This mean field is not only generated by the interaction of links, as in the case of complex networks, but also by the interaction of nodes as triangles.

For natural coupling, close to the synchronized state, the interaction between nodes in links and in triangles will be similar \cite{Battiston2021} and we can write $
\mathbf{h}^{(2)}(\mathbf{x},\mathbf{x}) =  \mathbf{h}^{(1)}(\mathbf{x})$ and $\mathbf{\overline{H}}^{(2)}{(\mathbf{x}, \mathbf{x})} =  \mathbf{\overline{H}}^{(1)}{(\mathbf{x})} = \mathbf{H(x)}.$
Eq.~\eqref{eq:2} can be written as,
\begin{align} \label{eq:3}
\mathbf{ \dot{x}}_i = \mathbf{f}(\mathbf{x}_i) + \left(\sigma_1S_i^{(1)} + \sigma_2S_i^{(2)}\right) \left(\mathbf{H}(\mathbf{x})-\mathbf{h}^{(1)}(\mathbf{x}_i) \right).
\end{align}
The variational equation of the above Eq.~\eqref{eq:3} will be of the form
\begin{align}
%\delta\mathbf{ \dot{x}}_i = & J\mathbf{f}(\mathbf{x}^s)\delta\mathbf{x}_i  -\left (\sigma_1S_i^{(1)} + \sigma_2S_i^{(2)} \right)J \left(\mathbf{h}^{(1)}(\mathbf{x}^s)\right) \delta\mathbf{x}_i, \notag \\
\delta\mathbf{ \dot{x}}_i = & \left[ J\mathbf{f}(\mathbf{x}^s) - \left(\sigma_1S_i^{(1)} + \sigma_2S_i^{(2)} \right)J\left(\mathbf{h}^{(1)}(\mathbf{x}^s)\right) \right] \delta\mathbf{x}_i, \label{variational} 
\end{align}
where $\mathbf{x}^s$ is the synchronization state.
In matrix form,
\begin{align}
\delta\mathbf{ \dot{X}} = & \left[J\mathbf{F}(\mathbf{x}^s) - \left(\sigma_1S_i^{(1)} + \sigma_2S_i^{(2)}\right)J\left(\mathbf{H}(\mathbf{x}^s)\right) \right]\delta\mathbf{X}, \label{eq:4}
\end{align}
where $\mathbf{X}=(\mathbf{x}_1, \mathbf{x}_2, \dots, \mathbf{x}_N)^T$ and $T$ denotes the transpose of a matrix.
The eigenvalues are approximately equal to $\sigma_1S_i^{(1)} + \sigma_2S_i^{(2)}$,  $i = 1, 2,  \ldots, N$.
From the above equation, we can write the eigenratio \cite{Zhou2006} as,
\begin{align}
R \approx \frac{\sigma_1S_{\mathrm{max}}^{(1)} + \sigma_2S_{\mathrm{max}}^{(2)}}{\sigma_1S_{\mathrm{min}}^{(1)} + \sigma_2S_{\mathrm{min}}^{(2)}}.
\end{align}
For the case of pairwise interaction only ($\sigma_2=0.0$), the eigenratio $R$ is very similar to the previous result \cite{Zhou2006}. For non-zero pairwise coupling strength ($\sigma_1\not=0.0,$), we can rewrite the above equation as,
\begin{align} \label{eq:5}
% R = \frac{\sum_{d=1}^N(\sigma_dS_{\mathrm{max}}^{(d)}) }{\sum_{d=1}^N(\sigma_dS_{\mathrm{min}}^{(d)} )}
R \approx \frac{S_{\mathrm{max}}^{(1)} + \frac{\sigma_2}{\sigma_1}S_{\mathrm{max}}^{(2)}}{S_{\mathrm{min}}^{(1)} + \frac{\sigma_2}{\sigma_1}S_{\mathrm{min}}^{(2)}}.
\end{align}
In the similar way, we can derive the eigenratio for the $d$-dimensional simplical complex as,
\begin{align}\label{eq:5.1}
R \approx \frac{S_{\mathrm{max}}^{(1)} + \frac{\sigma_2}{\sigma_1}S_{\mathrm{max}}^{(2)}+...+ \frac{\sigma_d}{\sigma_1}S_{\mathrm{max}}^{(d)}}{S_{\mathrm{min}}^{(1)} + \frac{\sigma_2}{\sigma_1}S_{\mathrm{min}}^{(2)} +\ldots +\frac{\sigma_d}{\sigma_1}S_{\mathrm{min}}^{(d)}}.
\end{align} %
Thus, the eigenratio of a sufficiently randomized simplicial complex depends on the coupling strengths (pairwise and non-pairwise), and maximum and minimum intensities of the nodes. If we assume all the non-pairwise coupling strengths are identical to the pairwise coupling strength (i.e., $\sigma_2=\sigma_3=\cdots=\sigma_d=\sigma_1$), then the eigenratio $R$ is independent on the coupling strengths and depends on the maximum and minimum intensities as observed for pairwise weighted random networks \cite{Zhou2006}. 

To find the tight bounds of the above mean-field approximation, we can write Eq.~\eqref{eq:3} as,
\begin{align} \label{eq:6}
\mathbf{ \dot{X}_i} = & \, \mathbf{F}(\mathbf{x}_i) - \sigma_1\sum_{j=1}^N G_{ij}^{(1)}\mathbf{H}^{(1)}(\mathbf{x}_j) %\notag \\ & 
-  \sigma_2\sum_{j=1}^N G_{ij}^{(2)}\mathbf{H}^{(2)}(\mathbf{x}_j),
\end{align}
where $G_{ij}^{(1)}$ and $G_{ij}^{(2)}$ are the elements of matrices $G^{(1)}= S^{(1)}D^{(1)-1}L^{(1)}$ and $G^{(2)}=S^{(2)}D^{(2)-1}L^{(2)}$, respectively. 
Here $L^{(1)}$ and $L^{(2)}$ are generalized Laplacian matrices, while $S$ and $D$ are diagonal matrices of strengths and generalized degrees, respectively. It is a good point that we have separated weights from topology.
%Here, $L^{(1)}$ and  $L^{(2)}$ are generalized Laplacian matrices. $S$ and $D$ are diagonal matrices of strengths and generalized degrees  Thus, we have sepereated weights from topology. 
We can write normalized Laplacian matrices as $
    \overline{L}^{(1)} = D^{-1}L^{(1)}$ and $
    \overline{L}^{(2)} = D^{-1}L^{(2)}$, then $
    G^{(1)} = S^{(1)}\overline{L}^{(1)}$ and $
    G^{(2)} = S^{(2)}\overline{L}^{(2)}$.
Thus, the largest and smallest eigenvalues of  matrices $G^{(1)}$ and $G^{(2)}$ are bounded by the eigenvalues of $\overline{L}^{(1)}$ and $\overline{L}^{(1)}$.
%As we follow the procedure in \cite{gambuzza2021stability}, we can write the linearized variational equation of \eqref{eq:6} as,
We can write the linearized variational Eq.  \eqref{eq:6} by following the procedure outlined in \cite{Gambuzza2021} and is given by
\begin{align} 
\delta\dot{\mathbf x}_i = \left(J\mathbf{f}(\mathbf{x}^s) - \sum_{j=1}^N\left[\sigma_1 G_{ij}^{(1)} + \sigma_2 G_{ij}^{(2)}\right]J\left(\mathbf{h}^{(1)}(\mathbf{x}^s) \right)\right)\delta{\mathbf x}_i. \label{eq:7}
\end{align}
We can define an effective matrix $M$ as,
\begin{align} \label{eq:8}
     M_{ij}= G^{(1)}_{ij} + \frac{\sigma_2}{\sigma_1}G^{(2)}_{ij}.
\end{align}
Equation~\eqref{eq:7} can be rewritten, in terms of effective matrix $M$, as 
\begin{align} \label{eq:9}
    \delta\mathbf{ \dot{x}}_i = \left[J\mathbf{f}\left(\mathbf{x}^s\right) - \sigma_1\sum_{j=1}^N M_{ij} J\left(\mathbf{h}^{(1)}\mathbf{x}^s\right)\right]\delta\mathbf{x}_i.
\end{align}
The eigenvalues of the matrix $M$ depend on the ratio of coupling strengths, degrees and intensities, and the eigenratio of this matrix can be expressed as $R=\lambda_N / \lambda_2,$ where $\lambda_2$ and $\lambda_N$ are the 2nd and $N$-th eigenvalues of $M$ such that $0=\lambda_1\le \lambda_2 \le \lambda_3\le \cdots \le \lambda_N$.

Another measure of synchronizability is the cost $C$ involved in the coupling of nodes in a simplicial complex, and it is the total strength of connections of all nodes. It can be written as,
%\begin{align}
%  C & = \sigma_1\sum_{i=1}^N\sum_{j=1}^N{a_{ij}}^{(1)}{\omega_{ij}}^{(1)} + \sigma_2{\sum_{i=1}^N\sum_{j=1}^N}{\sum_{k=1}^N}a_{ijk}^{(2)}\omega_{ijk}^{(2)}
%  \notag\\ & =\sigma_1\sum_{i=1}^N{S_{i}}^{(1)} + \sigma_2\sum_{i=1}^NS_{i}^{(2)}.
%\end{align}
\begin{align}
C = \sigma_1\sum_{i=1}^N{S_{i}}^{(1)} + \sigma_2\sum_{i=1}^NS_{i}^{(2)},
\end{align}
%where ${S_{i}}^{(1)} = \sum_{j=1}^N  {a_{ij}}^{(1)}{\omega_{ij}}^{(1)}$ and ${S_{i}}^{(2)} = \sum_{j=1}^N  \sum_{k=1}^N a_{ijk}^{(2)} \omega_{ijk}^{(2)}$.
%\begin{align} \label{eq:11}
 % C = \sigma_1\sum_{i=1}^N{S_{i}}^{(1)} + \sigma_2\sum_{i=1}^NS_{i}^{(2)}.
%\end{align}
By using Eq.~\eqref{eq:9}, the normalized cost can be written as,
\begin{align} \label{eq:12}
    C_0 = \frac{C}{N\alpha_1} = \frac{\Omega}{\lambda_2},
\end{align}
where $\alpha_1 = \sigma_1\lambda_2(M) $ and mean intensity
\begin{align} \label{eq:12:1}
   \Omega = \frac{1}{N}\left( \sum_{i=1}^N S_{i}^{(1)} + \frac{\sigma_2}{\sigma_1}\sum_{i=1}^NS_{i}^{(2)} \right).
\end{align}
From Eq.~\eqref{eq:4}, the normalized cost will be,
\begin{align} \label{eq:12:2}
C_0 \approx \frac{\Omega}{{S_{\mathrm{min}}^{(1)} + r S_{\mathrm{min}}^{(2)}}},
\end{align}
where $r={\sigma_2}/{\sigma_1}$.
We can verify Eq.~\eqref{eq:5} and Eq.~\eqref{eq:12:2} numerically by computing the eigenvalues of the effective matrix $M$, and we can obtain an explicit bound from the eigenvalues of $\overline{L}^{(1)}$ and $\overline{L}^{(2)}$. Using this bound, we can derive a universal formula for $R$ and $C_0$ as a function of $S_{\mathrm{max}}^{(1)}$, $S_{\mathrm{min}}^{(1)}$ and the generalized degrees.

We can demonstrate that the upper and lower bounds of the nonzero eigenvalues of the effective matrix $M$ are given by the eigenvalues $\mu^{(1)}_l$ and $\mu^{(2)}_l$ of the matrices $G^{(1)}$ and $G^{(2)}$, respectively, as follows,
\begin{align} 
& {S_{\mathrm{min}}^{(1)}\mu_2^{(1)} + rS_{\mathrm{min}}^{(2)}\mu_2^{(2)}} \le \lambda_2 \le  {S_{\mathrm{min}}^{(1)} + r S_{\mathrm{min}}^{(2)}},
 \label{eq:13} \\
%\end{align}
%\begin{align} 
&  {S_{\mathrm{max}}^{(1)} + r S_{\mathrm{max}}^{(2)}}  \le \lambda_N \le {S_{\mathrm{max}}^{(1)}\mu_N^{(1)} + rS_{\mathrm{max}}^{(2)}\mu_N^{(2)}}.\label{eq:14}
\end{align}
For sufficiently random simplicial complexes, the spectra of the matrices $G^{(1)}$ and $G^{(2)}$ tend to follow the semicircle law \cite{Zhou2006} and we can write $\mu_2^{(1)} \approx 1 - 2/\sqrt{K^{(1)}}$, $\mu_N^{(1)} \approx 1 + 2/\sqrt{K^{(1)}}$, $\mu_2^{(2)} \approx 1 - 2/\sqrt{K^{(2)}}$ and $\mu_N^{(2)} \approx 1 + 2/\sqrt{K^{(2)}}$, 
provided that $ k_\mathrm{min}^{(d)} \gg \sqrt{K^{(d)}} $. Here $K^{(d)}$ is the mean degree of the $d$-simplex. By introducing these in Eq. \eqref{eq:13} and Eq. \eqref{eq:14}, we get the following approximations for the bounds of $R$ and $C_0$ as
\begin{align} \label{eq:15}
  \frac{S_{\mathrm{max}}^{(1)} +r S_{\mathrm{max}}^{(2)}}{S_{\mathrm{min}}^{(1)} + rS_{\mathrm{min}}^{(2)}} \le R \le \frac{1 + \frac{2}{\sqrt{K^{(1)}}}}{1 - \frac{2}{\sqrt{K^{(1)}}}} \left( \frac{S_{\mathrm{max}}^{(1)} +r\frac{1 + \frac{2}{\sqrt{K^{(2)}}}}{1 + \frac{2}{\sqrt{K^{(1)}}}}S_{\mathrm{max}}^{(2)}}{S_{\mathrm{min}}^{(1)} + r\frac{1 - \frac{2}{\sqrt{K^{(2)}}}}{1 - \frac{2}{\sqrt{K^{(1)}}}}S_{\mathrm{min}}^{(2)}} \right).
\end{align}
\begin{align} \label{eq:15:2} 
  \frac{\Omega}{S_{\mathrm{min}}^{(1)} + r S_{\mathrm{min}}^{(2)}} \le C_0 \le \frac{1}{1 - \frac{2}{\sqrt{K^{(1)}}}} \left( \frac{\Omega}{S_{\mathrm{min}}^{(1)} + r\frac{1 - \frac{2}{\sqrt{K^{(2)}}}}{1 - \frac{2}{\sqrt{K^{(1)}}}}S_{\mathrm{min}}^{(2)}} \right).
\end{align}
For a given value of $K$, the synchronizability of randomly weighted simplicial complexes with a large value of $k_\mathrm{min}^{(1)}$ is expected to follow a universal formula as follows,
\begin{align} \label{eq:15:3}
R = A_R\left(\frac{S_{\mathrm{max}}^{(1)} + B_{R_1} S_{\mathrm{max}}^{(2)}}{S_{\mathrm{min}}^{(1)} +  B_{R_2}S_{\mathrm{min}}^{(2)}} \right),
\end{align}
and 
\begin{align} \label{eq:15:4}
C_0 = A_C\left(\frac{\Omega}{S_{\mathrm{min}}^{(1)} +  B_{R_2}S_{\mathrm{min}}^{(2)}} \right),
\end{align}
where $A_R = \frac{1 + 2/\sqrt{K^{(1)}}}{1 - 2/\sqrt{K^{(1)}}}$ and $ A_C =\frac{1}{1 - 2/\sqrt{K^{(1)}}}$. Here, $A_R \rightarrow 1$ and $A_C \rightarrow 1$ in the limit $K^{(1)} \rightarrow \infty $. Also, $B_{R_1} = r\frac{1 + 2/\sqrt{K^{(2)}}}{1 + 2/\sqrt{K^{(1)}}} $ and $ B_{R_2} = r\frac{1 - 2/\sqrt{K^{(2)}}}{1 - 2/\sqrt{K^{(1)}}} $. $B_{R_1} \rightarrow r$ and $ B_{R_2} \rightarrow r$ in the limit $K^{(d)} \rightarrow \infty (d=1,2)$. They are well agreed with Eqs. \eqref{eq:5} and \eqref{eq:12:2}. 

For $N$-dimensional simplicial complex, 
 \begin{align} \label{eq:15:5}
  R = A_R\left(\frac{S_{\mathrm{max}}^{(1)} +\sum_{d=2}^N B_{R_1}^{(d)} S_{\mathrm{max}}^{(N)}}{S_{\mathrm{min}}^{(1)} + \sum_{d=2}^N B_{R_2}^{(d)} S_{\mathrm{min}}^{(N)}} \right)
 \end{align}
and 
\begin{align} \label{eq:15:6}
  C_0 = A_C\left(\frac{\Omega}{S_{\mathrm{min}}^{(1)} + \sum_{d=2}^N B_C^{(d)} S_{\mathrm{min}}^{(N)}} \right),
\end{align}
where 
\begin{align}
B_{R_1}^{(d)} = r\frac{1 + 2/\sqrt{K^{(d)}}}{1 + 2/\sqrt{K^{(1)}}},\;\; B_{R_2}^{(d)} = r\frac{1 - 2/\sqrt{K^{(d)}}}{1 - 2/\sqrt{K^{(1)}}}.
\end{align}
We compare the expression of $R$ as a function of eigenvalue ratio ${\lambda_N}/{\lambda_2}$ and $C_0$ as a function of the ratio $\Omega/\lambda_2$ and found that they are almost identical when $K^{(1)}_{\mathrm{max}} \gg 1$.

To conduct our analysis, the parameter values are set as follows: $N = 10^3$, $\sigma_1 = 0.001$, $\sigma_2 = 0.01$. We arbitrarily fix the values of the minimum and maximum intensities of the node for pairwise $(S^{(1)}_{\mathrm{min}}, S^{(1)}_{\mathrm{max}})$ and non-pairwise  $(S^{(2)}_{\mathrm{min}}, S^{(2)}_{\mathrm{max}})$ interactions. In each iteration of the simulation, we add $100$ to $S^{(1)}_{\mathrm{max}}$ and keep all other values constant. The simulation is run $10^3$ times. We first consider a simplicial complex with a node $i$ connected to all other nodes, resulting in $K^{(1)}_{\mathrm{max}} = N - 1$. The maximum number of triangles incident on a node is $(N-1) \times (N-2)$, and the maximum number of triangles incident on a link is $(N-2)$. Using this structure, we generate the Laplacian matrices $L^{(1)}$ and $L^{(2)}$. Finally, we uniformly distribute $S$ among the nodes. All the triangles are considered as nodes having three-body interactions. 

We calculate the eigenratio $R$ of the effective matrix $M$ using Eq.~\eqref{eq:8} and compare it with Eq.~\eqref{eq:15:3}. We then plot the values of $R$ against the corresponding values of the eigenratio in terms of $S$ calculated from the universal formula, which reveals a linear relationship between $R$ and the eigenratio as a function of $S$. %
% Figure environment removed
Similarly, we calculate the normalized cost $C_0$ using Eq. \eqref{eq:8} and compare it with Eq. ~\eqref{eq:15:4}. We then plot the values of $\Omega/\lambda_2$ against the corresponding values of $C_0$, which a reveals a linear relationship between them, which is shown dashed-purple line with diamond symbol (AA) in Fig.~\ref{fig:1}. 

%The universal formula gives the upper limit of the $R$ and the approximated formula gives the lower limit. In this case, the calculated values of $R$ and $C_0$ are close to the lower limit.

One may note that the most realistic networks are scale-free networks, where the degrees are distributed according to a power law \cite{Barabasi2003, Adamic2000}. This power-law distribution means that only a small portion of nodes has very high degrees, while a majority of nodes have relatively small degrees. Therefore, we construct a simplicial complex with power-law distributed degrees and intensities; however, they are not related. We achieve this by distributing degrees and intensities to nodes using a power-law distribution. In Fig. \ref{fig:1}, we also show these results by the dotted-red line with stars (PL) and the green-dashed line with empty circles (PLICD).
%One may note that the most realistic networks are scale-free networks, where the degrees are distributed according to a power law~\cite{Barabasi2003}. This power-law distribution means that only a small portion of nodes have very high degrees, while the majority of nodes have relatively small degrees. Therefore, we construct a simplicial complex with power-law distributed degrees and intensities; however, they are not related. We achieve this by distributing degrees and intensities to nodes using a power-law distribution. We show the results by the dotted-red line with star symbols (PLICD) in Fig. \ref{fig:1}.

Next, we consider the scale-free distribution with intensities correlated to degrees. We construct this by distributing the intensities, which are proportional to degrees. Hence, the node with the maximum degree will receive the maximum intensity, and the node with the minimum degree will receive the minimum intensity. We apply this procedure to both three-body interactions and two-body interactions. Subsequently, we plot the graphs for $R$ and $C_0$ corresponding to this distribution.
The results demonstrate a linear relationship between the values of the eigenratio and cost calculated from the universal formula. This linear relationship also holds for the uniform distribution, as depicted by the dash-dotted pink line with square symbols (Uni) in Fig.~\ref{fig:1}.
From Fig.~\ref{fig:1}, we can conclude that the universal formula holds true regardless of the distribution of degrees and intensities. Further, we observe that the values obtained from the universal formula are nearly identical to that obtained by finding the eigenvalues of the effective matrix $M$ (blue dashed-line in Fig.~\ref{fig:1}).

As we can see from Fig.~\ref{fig:1}(b), the cost of connection is dependent on the distribution. The cost is very low for the power law distribution of degrees and power law distribution of intensities correlated to degrees. It is the network topology of scale free networks that arise spontaneously in natural and man made systems \cite{Barabasi2003}. So, the cost of connection is low for the more realistic networks. The cost of connection varies as the topology changes. 

Next, we study effect of the coupling strengths of pairwise and three-body interactions. We fix the values of the intensities as $S^{(1)}_{\mathrm{min}} = 10 $, $S^{(1)}_{\mathrm{max}} = 1000$,  $S^{(2)}_{\mathrm{min}} = 10 $ and $S^{(2)}_{\mathrm{max}} = 20$, and  and change the values of $r={\sigma_2}/{\sigma_1}$. Here, to change the value of $r$, we fix the pairwise interaction coupling strength $\sigma_1=0.001$ and vary the non-pairwise interaction coupling strength $\sigma_2$. 
It is evident from Fig.~\ref{fig:2}(a) that as the coupling strength for three-body interactions increases, the eigenratio decreases and approaches smaller values and increases the synchronizability of the system. When $r \approx 10$, the universal formula gives the exact values for the eigenratio. Then, for larger values of $r$, $R$ from both the approximated and universal formulas converge. %
% Figure environment removed%
One can easily study the effect of other higher-order interactions using Eq.~\eqref{eq:5.1}. From Fig.~\ref{fig:2}(a) , at the small values of $r$, (i.e., $\sigma_2$), the value of the eigenratio approaches to ${S^{(1)}_{\mathrm{max}}}/{S^{(1)}_{\mathrm{min}}}$. As we increase the $r$, by increasing $\sigma_2$, we can drastically reduce the values of $R$. In Fig.~\ref{fig:2}(a), ${S^{(1)}_{\mathrm{max}}}/{S^{(1)}_{\mathrm{min}}} = 100 $ and it falls down to $3$ as we increase the three-body interactions. Therefore, higher-order interactions promote synchronization when the first order intensities (degrees in the case of unweighted networks) are very heterogeneous. The same effect is shown in the values of cost as we can see from Fig.~\ref{fig:2}(c).
Then we analyze the synchronizability when the intensity of three-body interaction becomes more heterogeneous than the pairwise interaction, i.e., when $S^{(2)}_{\mathrm{max}} > S^{(1)}_{\mathrm{max}}$. The values are chosen as follows as $S^{(2)}_{\mathrm{max}} = 1000$ and $S^{(1)}_{\mathrm{max}} = 20 $.  In this case, the three-body interaction coupling strength has opposite effect. For a given value of $S^{(2)}{\mathrm{max}} $, the eigenratio increases as we increase the three-body interaction coupling strength $\sigma_2$. Hence, decrease the synchronizability. The values of eigenratio from the approximated formula and the universal formula converges when the eigenratio is minimum.

The effect of coupling strength is the same for both cost and the eigenratio [cf. Fig.~\ref{fig:2}(b) and \ref{fig:2}(d)]. The more heterogeneity there is in the second-order intensities, the faster the cost increases. Hence, the synchronizability decreases if we increase the value of $\sigma_2$.
The effect of coupling strengths on synchronization depends on the intensity of the network topology. Based on the above results, if $S^{(1)}_{\mathrm{max}} > S^{(2)}_{\mathrm{max}}$, an increase in $\sigma_2$ enhances synchronizability. However, the opposite scenario occurs if $S^{(2)}_{\mathrm{max}} > S^{(1)}_{\mathrm{max}}$, where a larger $\sigma_2$ reduces synchronizability. 

In general, higher-order interactions do not always promote synchronization \cite{Zhang2023}. They inhibit the synchronization of the system when the weights of the non-pairwise interactions are higher than those of the pairwise interactions. With this knowledge, we can promote or inhibit synchronization by manipulating the weights and coupling strengths of the pairwise and non-pairwise interactions. 

Next, we analyze the behaviour of the constants $A_R$, $A_C$, $B_{R_1}$, and $B_{R_2}$ as the mean degree changes for a simplicial complex with a mean degree of $K$. For this purpose, we choose the values of the coupling strengths as $\sigma_1 = 0.001$ and $\sigma_2 = 0.01$ and vary ${S^{(1)}{\mathrm{max}}/{S^{(1)}{\mathrm{min}}}} = 1, 2, 10, 100$ by fixing all nodes to the same degree as in the case of $K$-regular networks. Using Eqs. \eqref{eq:15:3} and \eqref{eq:15:4}, we calculate the values of $A_R$, $A_C$, $B_{R_1}$, and $B_{R_2}$ for large values of mean degree $K$.  %

As we can see from Figs.~\ref{fig:3}(a) and \ref{fig:3}(b), the heterogeneity of the intensities has a small effect on $A_R$, and the behaviour of $A_C$ appears independent of the intensities' heterogeneity as $K$ increases. %
% Figure environment removed%
Both $A_R$ and $A_C$ quickly approach unity as $K$ increases.

Further, from Figs.~\ref{fig:3}(c) and \ref{fig:3}(d), we observe that the values of $B_{R_1}$ and $B_{R_2}$ are highly dependent on the heterogeneity of the intensities. $B_{R_1}$ remains close to $r=10$ when the intensities are not very heterogeneous. As the heterogeneity increases, the values decrease for low $K$s and slowly approach $r$. The values of $B_{R_2}$ remain close to the upper bound when the intensities are heterogeneous and slowly approach $r$. When the intensities are homogeneous, i.e., ${S^{(1)}\mathrm{max}}/{S^{(1)}\mathrm{min}}=1$, the values of $B_{R_2}$ remain close to $r$ for all $K$.

From the behaviours of $A_R$, $A_C$, $B_{R_1}$ and $B_{R_2}$, we can observe that for $K \gg 1$ and not very heterogeneous networks, the universal formula approaches the approximated formula. With this, we can easily analyze the synchronizability of the weighted simplicial complexes in the presence of other higher-order interactions, such as four-body interactions, five-body interactions etc., from the knowledge of mean degrees, coupling strengths, and maximum and minimum intensities for the interaction networks. 

In this letter, we have analyzed the universality of synchronization in randomized weighted simplicial complexes. We derived the universal formulas for the eigenratio and the cost. We verified the theoretical results for different network typologies and various distributions of intensities. We concluded that we can determine the synchronizability of large randomized weighted simplicial complexes using the mean degrees, coupling strengths, and intensities without constructing the connectivity matrices and explicitly finding the eigenvalues of these matrices. These formulas can drastically reduce the computation time and cost while attempting to determine whether the system tends to synchronize and, if so, when. Furthermore, we observed that the effect of coupling strengths varies depending on the largest intensity. If $S^{(1)}{\mathrm{max}} > S^{(2)}{\mathrm{max}}$, the increase in $\sigma_2$ enhances synchronizability. The opposite situation arises when $S^{(2)}{\mathrm{max}} > S^{(1)}{\mathrm{max}}$, i.e., synchronization decreases when $\sigma_2$ increases. From the behaviours of constants $A_R$, $A_C$, $B_{R_1}$, and $B_{R_2}$, we can deduce that we can use the approximated formula, which contains only the values of coupling strengths and intensities. In conclusion, with these formulas, one can predict and control the synchronizability of complex networks with higher-order interactions by manipulating degrees, weights, and coupling strengths.


The work of P.M. and S.N.J. is supported by MoE RUSA 2.0 (Physical Sciences).

\input ref
%\bibliography{ref}
\end{document}
