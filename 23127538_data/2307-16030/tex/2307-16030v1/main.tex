\documentclass{scrartcl}
\usepackage[margin=3cm]{geometry}
\usepackage[utf8]{inputenc}
\usepackage{amsmath, amsthm, amssymb, amsfonts}
\usepackage[T1]{fontenc}
\usepackage{tikz-cd} 
\usepackage{enumitem}
\setcounter{secnumdepth}{4}
%\usepackage{bm}
%\usepackage{libertine}
\usepackage[italic]{mathastext}
\usepackage[english]{babel} 
\usepackage{url}
\usepackage{amsmath}


\usepackage{tcolorbox}
\usepackage{xcolor}
\usepackage{hyperref}
\theoremstyle{plain}
\newtheorem{thm}{Theorem}[section]
\newtheorem{fact}{Fact}

\newtheorem{lemma}[thm]{Lemma}
\newtheorem{prop}[thm]{Proposition}
\newtheorem{corollary}[thm]{Corollary}
\theoremstyle{definition}
\newtheorem{rmk}[thm]{Remark}
\newtheorem{defi}[thm]{Definition}
\newtheorem{example}[thm]{Example}
\newtheorem{question}[thm]{Question}
\newtheorem{properties}[thm]{Properties}
\newtheorem{pbm}[thm]{Problem}
\usepackage{upgreek}
\usepackage{mathtools}
\usepackage[all]{xy}
\usepackage{filecontents}
\DeclareMathOperator{\br}{Br}
\DeclareMathOperator{\inv}{inv}
\DeclareMathOperator{\ev}{ev}
\DeclareMathOperator{\ch}{ch}
\DeclareMathOperator{\Gal}{Gal}
\DeclareMathOperator{\Hom}{Hom}
\DeclareMathOperator{\spec}{Spec}
\DeclareMathOperator{\rsw}{rsw}
\DeclareMathOperator{\M}{M}
\DeclareMathOperator{\U}{U}
\DeclareMathOperator{\Pic}{Pic}
\DeclareMathOperator{\Ev}{Ev}
\DeclareMathOperator{\im}{\mathfrak{m}}
\DeclareMathOperator{\ip}{\mathfrak{p}}
\DeclareMathOperator{\gr}{gr}
\DeclareMathOperator{\h}{H}
\DeclareMathOperator{\tildefil}{\widetilde{fil}}
\DeclareMathOperator{\fil}{fil}
\newcommand{\numberset}{\mathbb}
\newcommand{\p}{\numberset{P}}
\newcommand{\F}{\numberset{F}}
\newcommand{\N}{\numberset{N}}
\newcommand{\Z}{\numberset{Z}}
\newcommand{\Q}{\numberset{Q}}
\newcommand{\R}{\numberset{R}}
\newcommand{\C}{\numberset{C}}
\newcommand{\Os}{\mathcal{O}}
\newcommand{\Ad}{\mathbf{A}}
\newcommand{\B}{\mathcal{B}}
\newcommand{\A}{\mathcal{A}}
\newcommand{\proj}{\text{Proj}}
\newcommand{\et}{\text{ét}}
\usepackage{dsfont}
\usepackage{commath}
\newcommand{\thistheoremname}{}
\newtheorem*{genericthm*}{\thistheoremname}
\newenvironment{namedthm*}[1]
  {\renewcommand{\thistheoremname}{#1}%
   \begin{genericthm*}}
  {\end{genericthm*}}


\makeatletter

\definecolor{mame}{rgb}{0.0, 0.5, 0.5}
\definecolor{bibi}{rgb}{0.79, 0.08, 0.48}

\title{The role of primes of good reduction in the Brauer--Manin obstruction}
\author{Margherita Pagano}
\date{}
\begin{document}
\maketitle

\begin{abstract}
    \noindent \textbf{Abstract:} We discuss the role of primes of good reduction in the existence of the Brauer--Manin obstruction to weak approximation for varieties defined over number fields. Following Bright and Newton, we give some necessaries conditions on the ramification index that the prime ideal of the number field should satisfy in order to be involved in the Brauer--Manin obstruction to weak approximation. To support the results, many examples of transcendental Brauer--Manin obstruction to weak approximation on K$3$ surfaces are given. 
\end{abstract}
\section{Introduction}
Let $k$ be a number field and $\Ad_k$ be the ring of adèles of $k$, i.e.\ the restricted product of $k_\nu$ for all places $\nu$ of $k$, taken with respect to the rings of integers $\Os_\nu\subseteq k_\nu$. Let $X$ be a smooth, proper, geometrically irreducible variety over $k$. In order to study the rational points on $X$ it is helpful to look at the image of $X(k)$ in the set of the adèlic points $X(\Ad_k)$. We say that $X$ satisfies \emph{weak approximation} if the image of $X(k)$ in $X(\Ad_k)$ is dense. Manin \cite{Manin} introduces the use of the Brauer group to study weak approximation. More precisely, he showed that there exists a pairing
\[
    \langle-,-\rangle\colon \br(X)\times X(\Ad_k)\rightarrow \Q/\Z
\]
such that the rational points of $X$ lie in the image of the right kernel of the pairing, denoted by $X(\Ad_k)^{\br}$. If $X(\Ad_k)^{\br}$ is not equal to the whole $X(\Ad_k)$ we say that there is a \emph{Brauer--Manin obstruction to weak approximation} on $X$.

For every non-archimedean place $\ip$ and for every element $\A\in \br(X)$ we denote by $\ev_\A$ the map $\langle \A, -\rangle\mid_{X(k_{\ip})}\colon X(k_{\ip})\rightarrow \Q/\Z$.
\begin{defi}
    We say that a prime $\ip$ of $k$ \emph{plays a role} in the Brauer--Manin obstruction to weak approximation on $X$ if there exists an element $\A\in \br(X)$ such that the corresponding evaluation map $\ev_\A\colon X(k_{\ip})\rightarrow \br(k_{\ip})$ is non-constant.  
\end{defi}
In this paper, we are addressing the following question: 
\begin{question}
    Assume $\Pic(\bar{X})$ to be torsion-free and finitely generated. Which primes can play a role in the Brauer--Manin obstruction to weak approximation on $X$?
\end{question}
This question is inspired by a question asked by Swinnerton-Dyer \cite[Question 1]{colliotskogoodred}. He asked whether only primes of bad reduction and archimedean places could play a role in the Brauer--Manin obstruction to weak approximation. 
 Bright and Newton \cite{BrightNewton} prove the following theorem, which leads to a negative answer to Swinnerton-Dyer's question.
\begin{namedthm*}{Theorem C}\label{thm: Theorem C}
    Assume that $\h^0(X,\Omega^2_X)\ne 0$. Let $\ip$ be a prime at which $X$ has good ordinary reduction and of residue characteristic $p$. Then there exists a finite field extension $k'/k$, a prime $\ip'$ of $k'$ lying over $\ip$ and an element $\A\in \br(X_{k'})\{p\}$ such that the corresponding evaluation map is non-constant on $X(k'_{\ip'})$. 
\end{namedthm*}
Equivalently, under the assumption of Theorem C, up to a base change to a finite field extension $k'/k$ we can always find a prime $\ip'$ of good reduction that plays a role in the Brauer--Manin obstruction to weak approximation on $X_{k'}$.

\vspace{5mm}

The main focus of this paper is on the field extension $k'/k$ appearing in Bright and Newton's result, with a particular emphasis on K$3$ surfaces. We are going to analyse how the reduction type and the absolute ramification index are involved in the possibility for a prime of good reduction to play a role in the Brauer--Manin obstruction to weak approximation. 
\begin{thm}\label{thm: ordinary good reduction}
    Let $\ip$ be a prime of good ordinary reduction for $X$ of residue characteristic $p$. Assume that the special fibre $Y$ has no non-trivial global $1$-forms, $\h^1(\bar{Y},\Z/p\Z)=0$ and $(p-1)\nmid e_{\ip}$. Then the prime $\ip$ does not play a role in the Brauer--Manin obstruction to weak approximation on $X$.
\end{thm}
In particular, as pointed out in remark \cite[Remark~11.5]{BrightNewton}, if $X$ is a K$3$ surface, then for every prime of good reduction the special fibre is still a K$3$ surface and hence has no non-trivial global $1$-forms and satisfies $\h^1(\overline{Y},\Z/p\Z)=0$. It follows from Theorem \ref{thm: ordinary good reduction} that for K$3$ surfaces over number fields, in order for a prime of good ordinary reduction to be involved in the Brauer--Manin obstruction to weak approximation it is necessary that $(p-1)\mid e_{\ip}$. In the case of K$3$ surfaces, we analyse also what happens for primes of good non-ordinary reduction.
\begin{thm}\label{thm: good non-ordinary reduction}
    Let $X$ be a K$3$ surface and $\ip$ be a prime of good non-ordinary reduction for $X$ with $e_{\ip} \leq (p-1)$. Then the prime $\ip$ does not play a role in the Brauer--Manin obstruction to weak approximation on $X$.
\end{thm}
Furthemore, in Section~\ref{subsubsection: On the existence of a Brauer Manin obstruction over a field extension} we improve the result of Theorem~C. In particular, we show that it is always possible to find (over a finite field extension $k'/k$) an element whose order is exactly $p$ and whose evaluation map is non-constant.

We provide several examples showing that the results in Theorem \ref{thm: ordinary good reduction} and \ref{thm: good non-ordinary reduction} are optimal. In particular, we exhibit K$3$ surfaces $X$ over number fields such that:
\begin{enumerate}[label=(\alph*)]
    \item $X$ has good ordinary reduction at a prime $\ip$ with ramification index $e_{\ip}=p-1$ and there is an element $\A\in \br(X)[p]$ whose evaluation map is non-constant on $X(k_{\ip})$;
    \item $X$ has good ordinary reduction at a prime $\ip$ with $e_{\ip}=p-1$ and $\ip$ does not play a role in the Brauer--Manin obstruction to weak approximation; 
    \item $X$ has good non-ordinary reduction at a prime $\ip$ with $e_{\ip}\geq p$ and there is an element $\A\in \br(X)[p]$ whose evaluation map is non-constant on $X(k_{\ip})$.
\end{enumerate}
More precisely: from (a) we get that the condition $(p-1)\nmid e_{\ip}$ in Theorem~\ref{thm: ordinary good reduction} is necessary; from (b) we get that the inverse of Theorem~\ref{thm: ordinary good reduction} does not hold in general; from (c) we get that the bound in Theorem~\ref{thm: good non-ordinary reduction} is optimal.

Finally, we point out that in all the examples of K$3$ surfaces in which a prime of good reduction plays a role in the Brauer--Manin obstruction of weak approximation, the corresponding element in the Brauer group is of \emph{transcendental} nature, i.e. it does not belong to the algebraic Brauer group, which is defined as the kernel of the natural map from $\br(X)$ to $\br(\bar{X})$, where $\bar{X}$ is the base change of $X$ to an algebraic closure of $k$ (cfr. Lemma~\ref{lemma: transcendental nature of A}). 
The first example of a transcendental element in the Brauer group of a K$3$ surface defined over a number field was given by Wittenberg in \cite{Wittenberg}. Other examples of transcendental elements that obstruct weak approximation can be found in \cite{HasVarWAorder2}, \cite{IeronymouOrd2},\cite{PreuOrd3}, \cite{NewtonOrd3} and \cite{BergVarOrd3}. In all these articles, the obstruction to weak approximation comes from the fact that either the transcendental algebra has non-constant evaluation at the place at infinity or at a prime of bad reduction for the K$3$ surface taken into account. The first example of a K$3$ surface in which a prime of good reduction plays a role in the Brauer--Manin obstruction to weak approximation can be found in \cite{Pagano}. 
\subsection{Notation} If $G$ is an abelian group and $n$ a positive integer, then $G[n]$ denotes the kernel of multiplication by $n$ on $G$. If $p$ is a prime, then $G\{p\}$ denotes the $p$-power torsion subgroup of $G$. For any smooth scheme $X$ over a field $k$, we denote by $\Z/n\Z(r)$ the sheaves on $X_\et$ defined as
\[
\Z/n\Z(r):=
\begin{cases}
    \mu_n^{\otimes r}, \text{ if } k \text{ has characteristic }0;\\
    \mu_n^{\otimes r}\oplus W_s \Omega^r_{X,\log}[-r], \text{ if }k\text{ has characteristic }p>0\text{ and }n=mp^r, \, p\nmid m.
\end{cases}
\]
For the definition of $W_s \Omega^r_{Y,\log}$ see \cite[Chapter~I]{Illusie}. Finally, if $R$ is a $k$-algebra and either $n$ is invertible in $k$ or $R$ is smooth over $k$, we define
\[
    \h^q_n(R):=\h^q(R_\et, \Z/n\Z(q-1)) \, \text{ and } \, \h^q(R):=\varinjlim_n \h^q_n(R). 
\]

\subsection{Outline}
In Section~\ref{Section: General Setting} we describe the general setting we will be working in during the whole paper. Section~\ref{Section: the refined Swan conductor} is devoted to the notion and the properties of the refined Swan conductor. In Section~\ref{section:ordinarygoodred} we prove Theorem~\ref{thm: ordinary good reduction} and we prove a stronger version of \cite[Theorem~C]{BrightNewton} for K$3$ surfaces. Section~\ref{section: ordinary case examples} is devoted to the explanation of several examples of K$3$ surfaces in which primes of good ordinary reduction are (or are not) involved in the Brauer--Manin obstruction to weak approximation. In Section~\ref{Section: non-ordinary good reduction} we give a proof of Theorem~\ref{thm: good non-ordinary reduction}. Finally,  in Section~\ref{Section: family of examples} we exhibit a family of examples where, depending on a parameter $\alpha$, both ordinary and non-ordinary reduction occur and we analyse what happens to the Brauer--Manin obstruction to weak approximation depending on the parameter $\alpha$.
\subsection{Acknowledgements}
I would like to thank Martin Bright for all the useful conversations, suggestions and carefully reading the first version of this paper. I am thankful to Evis Ieronymou for pointing out to me Example~\ref{subsection: non ordinary case Example}.

\section{General setting}\label{Section: General Setting}
We begin this section with the following observation: let $X$ be a smooth, proper, geometrically irreducible variety over a number field $k$. Assume that there is a prime $\ip$ of $k$ that plays a role in the Brauer--Manin obstruction to weak approximation on $X$; then there is an element $\A\in \br(X)$ such that the evaluation map $\ev_\A$ attached to $\A$ is non-constant on $X(k_{\ip})$. If we denote by $\mathrm{res}\colon \br(X)\rightarrow\br(X_{\ip})$ the natural restriction map form $\br(X)$ to $\br(X_{\ip})$, then for every point $P\in X(k_{\ip})$ we have a corresponding point $P_{\ip}\in X_{\ip}(k_{\ip})$ and
\[
    \ev_{\mathrm{res}(\A)}(P_{\ip})=\ev_\A(P).
\]
Hence, there is an element $\mathrm{res}(A)\in \br(X_{\ip})$ whose evaluation map on $X_{\ip}(k_{\ip})$ is non-constant. By what we just said together with the fact that we are interested in primes of good reduction, we will work under the general setting described below.

Let $p$ be a prime number and $L$ a finite field extension of $\Q_p$, with ring of integers $\Os_L$, uniformiser $\pi$ and residue field $\ell$. Let $X$ be a smooth and geometrically irreducible $L$-variety having good reduction (i.e. there exists a smooth proper $\Os_L$-scheme $\mathcal{X}$ whose generic fibre is isomorphic to $X$). We assume furthermore the special fibre, $Y:=\mathcal{X}\times_{\spec(\Os_L)}\spec(\ell)$ to be geometrically irreducible. 
\begin{equation}\label{eq:diagram1}
    \begin{tikzcd}
    X\arrow[d]\arrow[r,"j"] &\mathcal{X}\arrow[d] &Y\arrow[d] \arrow[l,"i"'] \\
    \spec(L)\arrow[r] &\spec(\Os_L) &\spec(\ell).\arrow[l]
    \end{tikzcd}
\end{equation}
In \cite{BrightNewton} Bright and Newton define the following filtration, called the \emph{evaluation filtration}, on the Brauer group of $X$:
 \begin{alignat*}{2}
     &\Ev_n \br X:=\{ \mathcal{B} \in \br(X) \mid \forall L'/L \text{ finite, } \forall P\in \mathcal{X}(\Os_{L'})\\
     &\qquad \qquad \qquad \qquad \quad \ev_{\mathcal{B}}  \text{ is constant on }B(P,e_{L'/L}(n+1))\}, \qquad (n\geq 0)\\ 
     &\Ev_{-1} \br X :=\{ \mathcal{B} \in \br(X) \mid \forall L'/L \text{ finite, } \ev_{\mathcal{B}} \text{ is constant on }\mathcal{X}(\Os_{L'})\}\\
     &\Ev_{-2} \br X :=\{ \mathcal{B} \in  \br(X) \mid \forall  L'/L  \text{ finite, } \ev_{\mathcal{B}}  \text{ is zero on }\mathcal{X}(\Os_{L'})\}
 \end{alignat*}
For every positive integer $m$ we denote by $\Ev_n \br(X)[m]$ the restriction of $\Ev_n \br(X)$ to $\br(X)[m]$, i.e. $\Ev_n \br(X)[m]:=\Ev_n \br(X) \cap \br(X)[m]$. If $(m,p)=1$ Colliot-Th\'{e}l\`{e}ne--Skorobogatov \cite{colliotskogoodred} and Bright \cite{BrightBadReduction} proved that $\Ev_0\br(X)[m]=\br(X)[m]$. Moreover, the residue map coming from the Purity Theorem for the Brauer group 
\begin{equation}\label{eqPurityTheoremCoprimeToP}
    0\rightarrow \br(\mathcal{X})[m]\rightarrow \br(X)[m]\xrightarrow{\partial_m} H^1(Y,\Z/m\Z)
\end{equation}
is such that 
\begin{align*}
    &\Ev_{-1}\br(X)[m]=\{\A \in \br(X)[m]\mid \partial_m(\A)\in H^1(\ell,\Z/m\Z)\} \\
    &\Ev_{-2} \br(X)[m]=\{ \A \in \br(X)[m] \mid \partial_m (\A)=0\}.
\end{align*}
In order to give a description of the the Evaluation filtration also on the $p$-power torsion part of $\br(X)$, Bright and Newton introduce another filtration $\{\fil_n \br(X)\}_{n\geq 0}$ on $\br(X)$, defined through the notion (introduced by Kato \cite{Kato}) of Swan conductor on discrete henselian valuation fields, see Section \ref{Section: the refined Swan conductor}. The interaction between the two filtration is described in the theorem that follows.
%Let $K^h$ be the field of fractions of a henselisation of the discrete valuation ring $\Os_{\mathcal{X},Y}$. For every element $\A\in \br(K^h)$ there exists a non-negative integer $n$ called the \textbf{Swan conductor} of $\A$; we  denote by $\{\fil_n \br(K^h)\}_{n\geq 0}$ the corresponding increasing filtration. Moreover, for every positive integer $n$, Kato defines a map, called \textbf{refined Swan conductor} 
%\[
%    \rsw_{n,\pi}:\fil_n \br(K^h)\rightarrow \Omega^2_F\oplus \Omega^1_F
%\]
%whose kernel coincide with $\fil_{n-1}\br(K^h)$.
%The natural map $\br(X)\rightarrow \br(K^h)$ allows to pull back the definition of Swan conductor and the corresponding filtration to $\br(X)$. Bright and Newton prove that for every $n\geq 1$, and for every $\A\in \fil_n\br(X)$, $(\alpha,\beta)\in \Omega^2_F\oplus \Omega^1_F$ such that $\rsw_{n,\pi}(\A)=(\alpha,\beta)$ are global forms on $Y$. The link bewteen the Evaluation filtration and the Swan conductor filtration is given by the following theorem. 
\begin{thm}\label{thm: evaluation fil and refined Swan conductor}
There exists a residue map $\partial\colon \fil_0\br(X)\rightarrow \varinjlim_n H^1(Y,\Z/n\Z)=:H^1(Y,\Q/\Z)$ such that
    \begin{enumerate}[label=(\alph*)]
    \item $\fil_0\br(X)$ coincides with $\Ev_0\br(X)$; 
    \item $\Ev_{-1} \br(X)=\{\A \in \br(X)\mid \partial(\A)\in H^1(\ell,\Q/\Z)\}$;
    \item $\Ev_{-2} \br(X)=\{ \A \in \br(X) \mid \partial (\A)=0\}.$
    \end{enumerate} 
    For every $n \geq 1$ there is a map, called refined Swan conductor
    \[
    \rsw_{n,\pi}\colon\fil_n \br(X)\rightarrow \h^0(Y,\Omega^2_Y)\oplus \h^0(Y,\Omega^1_Y)
    \]
    such that
    \[
    \Ev_n \br(X)=\left\{
        \A\in \fil_{n+1}\br(X)\mid \rsw_{n+1,\pi}(\A)\in \h^0(Y,\Omega^2_Y)\oplus 0\right\}.
    \]
\end{thm}
\begin{proof}
    This is a reformulation of \cite[Theorem~A]{BrightNewton}.
\end{proof}
It is clear that in order to understand the evaluation filtration on the Brauer group of $X$ we need to understand the residue map $\partial$ and the refined Swan conductor maps $\rsw_{n,\pi}$, the next section is fully devoted to it. 
\section{The refined Swan conductor}\label{Section: the refined Swan conductor}
In this section we gather some of the main results proven in \cite{BrightNewton}. 
Through this section we denote by $K$ an henselian discrete valuation field of characteristic zero with ring of integers $\Os_K$ and residue field $F$ of characteristic $p$. Let $\pi$ be a uniformiser in $\Os_K$ and $\mathfrak{m}$ the maximal ideal of $\Os_K$.
Let $A$ be a ring over $\Os_K$, and $i,j$ the inclusions of the special and generic fibers into $\spec(A)$:
\begin{center}
    \begin{tikzcd}
     \spec(A\otimes_{\Os_K} K)\arrow[r,"j"] &\spec(A) &\spec(A/\im A).\arrow[l,"i"']
    \end{tikzcd}
\end{center}
Define
\[
    V_n^q(A):=\h^q\left((A/\im A)_\et, i^*\mathrm{R}j_* \Z/n\Z(q-1)\right)
\]
and $V^q(A):=\varinjlim_n V_n^q(A)$. In particular, $\h^q(K)=V^q(\Os_K)$. For all positive integers $r,q$, there is a well defined map (see \cite[Section~1.7]{Kato} for the details)
    \begin{align*}
        V_n^q(A)\times &((A\otimes_{\Os_K} K)^\times)^r \rightarrow V_n^{q+r}(A)\\
        (\chi,&a_1,\dots,a_r) \mapsto \{\chi, a_1,\dots,a_r\}.
\end{align*}
\begin{defi}[Swan conductor]\label{def: Swan Conductor}
The increasing filtration $\{\fil_n H^q(K)\}_{n\geq 0}$ on $\h^q(K)$ is defined by 
$$\chi \in \fil_n \h^q(K) \Leftrightarrow \{\chi, 1+\pi^{n+1} T\}=0 \text{ in }V^{q+1}(\Os_K[T]). $$
We say that $\chi \in \h^q(K)$ has \emph{Swan conductor} $n$, if $\chi\in \fil_n\h^q(K)$ and $\chi \notin \fil_{n-1} \h^q(K)$. By \cite[Lemma 2.2]{Kato} $\h^q(K)=\cup_n \fil_n\h^q(K)$, hence the Swan conductor is defined for every element $\chi\in \h^q(K)$.
\end{defi}


In \cite{Kato} and \cite{BrightNewton} some maps $\lambda_\pi\colon \h^{q}_n(A/\im A)\oplus \h^{q-1}_n (A/\im A)\rightarrow V^q_n(A)$ are defined under additional assumptions on the $\Os_K$-algebra $A$. We give an overview of these maps, that with abuse of notation we will always call $\lambda_\pi$.
\begin{rmk}[The map $\lambda_\pi$]\label{rmk: different definition of lambda_pi}
    \leavevmode
    \begin{enumerate}[label=(\alph*)]
        \item In \cite[Section~1.4]{Kato} Kato defines for every $n$ an injective map 
        \[
        \lambda_\pi\colon \h^q_n(F)\oplus\h^{q-1}_n(F) \rightarrow \h^q_n(K).
        \]
        This collection of maps induces an injective map
        \[
            \lambda_\pi\colon \h^q(F)\oplus \h^{q-1}(F)\rightarrow \h^q(K)
        \]
        whose image coincides with $\fil_0 \h^q(K)$, see \cite[Proposition~6.1]{Kato}.
        \item Kato \cite[Section~1.9]{Kato} extends the definition of $\lambda_\pi\colon \h^q_p(F)\oplus \h^{q-1}_p(F)\rightarrow \h^{q}_p(K)$ to any smooth $\Os_K$-algebra $A$. In particular, he defines a map 
        \[
            \lambda_\pi\colon \h^q_p(A/\im A)\oplus \h^{q-1}_p(A/\im A)\rightarrow V^q_p(A).
        \]
        \item Bright and Newton generalise the map $\lambda_\pi\colon \h^q_p(A/\im A)\oplus \h^{q-1}_p(A/\im A)\rightarrow V^q_p(A)$ by defining 
        \[
        \lambda_\pi\colon  \h^q_{p^r}(A/\im A)\oplus \h^{q-1}_{p^r}(A/\im A)\rightarrow V^q_{p^r}(A)
        \]
        for any $r\geq 1$.
    \end{enumerate}
    In \cite[Section~1.3]{Kato} Kato shows that for every $r\geq 1$ there is a surjection $\delta_r$ from $W_r \Omega^q_{A/\im A}$ to $\h^q_{p^r}(A/\im A)$. Following \cite{Kato} and \cite{BrightNewton} we sometimes use $\lambda_\pi$ also to denote the composition
    \[
    W_r \Omega^q_{A/\im A}\oplus W_r \Omega^{q-1}_{A/\im A}\xrightarrow{\delta_r} V^q_{p^r}(A/\im A)\oplus V^{q-1}_{p^r}(A/\im A)\xrightarrow{\lambda_\pi} V^q_{p^r}(A)
    \]
\end{rmk}
The following theorem allows Kato to define the refined Swan conductor of an element $\chi\in \fil_n\h^q(K)$.
\begin{thm}\label{Thm: existence of refined Swan conductor}
    Let $\chi\in \fil_n \h^q(K)$, with $n\geq 1$; then there exists a unique pair $(\alpha,\beta)$ in $\Omega^q_F\oplus \Omega^{q-1}_F$ such that 
    \begin{equation}
        \{\chi,1+\pi^n T\}=\lambda_\pi(T\alpha,T\beta) \quad \text{in }V^{q+1}_p(\Os_K[T]).
    \end{equation}
\end{thm}
\begin{proof}
    See \cite[Section ~5]{Kato}.
\end{proof}
As a consequence of the uniqueness of the pair $(\alpha,\beta)$ in the previous theorem, we get that for all $n\geq 1$ there exists a morphism 
\[
    \rsw_{n,\pi}\colon\fil_n \h^q(K)\rightarrow \Omega^q_F\oplus \Omega^{q-1}_F
\]
whose kernel is $\fil_{n-1}\h^q(K)$. The pair $(\alpha,\beta)$ is called the \emph{refined Swan conductor} of $\chi \in \fil_n \h^q(K)$.

From now on we will focus attention on the refined Swan conductor of elements in the Brauer group of $K$. Note that 
\[
\h^2(K)=\varinjlim_n \h^2(K_\et,\Z/n\Z(1))=\br(K).
\]
Finally, also the definition of the residue map $\partial\colon \br(K)\rightarrow \h^1(F)$ uses the map $\lambda_\pi$ (cfr. \cite[Section~2.5]{BrightNewton}, \cite[Section~7.5]{Kato}).
\begin{defi}[Residue map]\label{defi: residue map on fields K,F}
    The \emph{residue map} 
    \[
    \partial\colon \fil_0 \br(K)\rightarrow \h^1(F)
    \]
    is defined as the projection on the second component of the inverse of the isomorphism $\lambda_\pi$ from $\h^2(F)\oplus \h^1(F)$ to $\fil_0 \br(K)$ (cfr. Remark~\ref{rmk: different definition of lambda_pi}(a)). 
\end{defi}


\subsection{Differential forms in positive characteristic and image of the refined Swan conductor}\label{subsubsection: differential forms in positive characteristic and image of the refined Swan conductor}
To describe further the refined Swan conductor maps, we need to recap some properties of differential forms over fields of positive characteristic.
Through this section we assume that $F$ is a field of characteristic $p>0$ finitely generated over a perfect field $\ell$. Let
\[
    Z^q_F:=\ker(d\colon\, \Omega^q_F\rightarrow \, \Omega^{q+1}_F) \quad \text{ and } \quad B^q_F:=\mathrm{im}(d\colon\, \Omega^{q-1}_F\rightarrow \, \Omega^{q}_F).
\]
\begin{lemma}[Inverse Cartier operator]\label{lemma:inverse cartier operator}
    There exists a unique morphism of groups 
    $$C^{-1}_F\colon \Omega^1_F \rightarrow \, \Omega^1_F/B^1_F$$
    satisfying $C_F^{-1}(da)=a^{p-1}da \mod B^1_F$ for all $a\in F$ and $C^{-1}_F(\lambda \omega)=\lambda^p C^{-1}_F(\omega)$ for all $\lambda\in F$. Moreover, we have $d\circ C^{-1}_F=0$.
\end{lemma}
\begin{proof}
    See \cite[Lemma $9.2.1$]{GilleSzamuely}.
\end{proof}
\begin{rmk}\label{remark: on the F^p structure on B^q and Z^q}
    The subgroup $B^1_F$ has a natural structure of $F$-vector space, which is given by $\alpha \cdot d\beta=\alpha^p d\beta=d(\alpha^p \beta)$. If we denote by $^p\Omega^1_F$ the $F$-vector space structure on $\Omega^1_F$ given by $\alpha \cdot \omega=\alpha^p \omega$, then the condition $C_F^{-1}(\lambda \omega)=\lambda^p C_F^{-1}(\omega)$ is equivalent to asking that $C_F^{-1}\colon\Omega^1_F\rightarrow \,^p\Omega^1_F/^pB^1_F$ is a morphism of $F$-vector spaces. 
\end{rmk}
We can extend the definition of $C^{-1}_F$ to higher differential forms by setting 
$$C^{-1}_F(\omega_1\wedge \dots \wedge \omega_q):=C^{-1}_F(\omega_1)\wedge\dots \wedge C^{-1}_F(\omega_q).$$
\begin{thm}\label{thm: higher dimension Cartier isomorphism}
    The morphism
    $$C^{-1}_F\colon\Omega^q_F\rightarrow \, Z^q_F/B^q_F$$ 
    is an isomorphism for all $q\geq 0$. We will denote by $C_F$ its inverse, which is called the {Cartier operator}.
\end{thm}
\begin{proof}
    See \cite[Theorem $9.4.3$]{GilleSzamuely}.
\end{proof}

The following proposition gives a way to characterise exact differential forms in terms of the Cartier operator.
\begin{prop}\label{prop: Exact Forms in terms of Cartier Op}
A $q$-form $\omega\in \Omega^q_F$ is exact if and only if $d(\omega)=0$ and $C_F(\omega)=0$.
\end{prop}

\begin{proof}
    See \cite[Corollary $9.4.4$]{GilleSzamuely}
\end{proof}
The last object we need to define is the subgroup of \emph{logarithmic} $q$-differential forms on $F$, which will play a crucial role in this paper. 
\begin{defi}\label{def: logarithmic forms}
    The \emph{logarithmic} $q$-differential forms on $F$, denoted by $\Omega^q_{F,\log}$ are defined as the kernel of the map
    $$C^{-1}_F-\mathrm{id}\colon \Omega^q_F\rightarrow \Omega^q_F/B^q_F.$$
    
\end{defi}
The following result gives a way to write down logarithmic differential forms explicitly.
\begin{thm}\label{thm: Log forms on field F}
    The logarithmic differential $q$-forms $\Omega^q_{F,\log}$ is the subgroup of $\Omega^q_F$ generated by elements of the form 
    $$\frac{dy_1}{y_1}\wedge \dots \wedge\frac{dy_q}{y_q}, \text{ with }y_i\in F^\times.$$
\end{thm}
\begin{proof}
    It follows from surjectivity in the Bloch--Gabber--Kato Theorem \cite[Theorem $9.5.2$]{GilleSzamuely}.
\end{proof}
\subsection{Properties of the image of the refined Swan conductor}\label{subsection: properties of the image of the refined Swan conductor}
We are now ready to state and prove some properties about the image of the refined Swan conductor. We denote by $e$ be the absolute ramification index of $K$ and by $e':=ep(p-1)^{-1}$. We start by recalling the following Lemma.
\begin{lemma}\label{lemma: rsw d(alpha)=0 and d(beta)=(alpha)}
    Let $\A$ be an element in $\fil_n \br(K)$ with $\rsw_{n,\pi}(\A)=(\alpha,\beta)\in \Omega^2_F\oplus \Omega^1_F$. Then $d\alpha=0$ and $d\beta=n \alpha$.
\end{lemma}
\begin{proof}
    See \cite[Lemma~2.8]{BrightNewton}.
\end{proof}
\begin{rmk}\label{rmk: rsw comp proj_2}
    We get that:
    \begin{enumerate}[label=(\alph*)]
        \item if $p\mid n$, then $d\alpha=0$ and $d\beta=0$, meaning that $(\alpha,\beta)\in Z^2_F\oplus Z^1_F$;
        \item if $p\nmid n$, then $\alpha=\bar{n}^{-1}d\beta$, meaning that the composition
        \[
        \fil_n \br(K)\xrightarrow{\rsw_{n,\pi}} \Omega^2_F\oplus \Omega^1_F \xrightarrow{\mathrm{pr}_2} \Omega^1_F
        \] 
        has also kernel equal to $\fil_{n-1}\br(K)$.
    \end{enumerate}
\end{rmk}
In \cite[Lemma~2.19]{BrightNewton} Bright and Newton are able to link the refined Swan conductor of an element $\A\in \fil_n\br(K)$ with the refined Swan conductor of $\A^{\otimes p}$, whenever $n\geq e'$. More precisely, assume that $\rsw_{n,\pi}(\A)=(\alpha,\beta)$ and let $\bar{u}$ be the reduction modulo $\pi$ of $p\cdot \pi^{-e}$, then 
\begin{equation}\label{eq: rsw n with n>e'}
    \rsw_{n-e,\pi}(A^{\otimes p})=\begin{cases}
        (\bar{u}\alpha,\bar{u}\beta) \text{ if }n>e';\\
        (\bar{u}\alpha+C(\alpha),\bar{u}\beta+C(\beta)) \text{ if }n=e'.
    \end{cases}
\end{equation}
The following lemma gives a result analogous to the one proven by Bright and Newton for elements $\A\in \fil_{np}\br(K)$, when $np<e'$.
\begin{lemma}\label{lemma: refined np with np<e'}
    Let $\A\in \fil_{np} \br(K)$, with $np<e'$. Then $\A^{\otimes p}\in \fil_n\br(K)$ and if $\rsw_{np,\pi}(\A)=(\alpha,\beta)$, then $d\alpha=0$, $d\beta=0$ and 
    \[
        \rsw_{n,\pi}(\A^{\otimes p})=(C(\alpha),C(\beta)).
    \]
\end{lemma}

\begin{proof}
    From Remark~\ref{rmk: rsw comp proj_2}(a) we know that $(\alpha,\beta)\in Z^2_F\oplus Z^1_F$, since clearly $p\mid np$. The condition $e'>np$ implies $ \frac{p}{p-1} \cdot (e-np+n)>0$, which implies 
    \begin{equation}\label{eq: e-np+n>0}
        e-np+n>0
    \end{equation}
    Let $u\in \Os_L^\times$ be such that $p=u\cdot \pi^e$. 
    
    %First, recall that for every $1<k<p$, we have that $p\nmid k!$ and $p\nmid (p-k)!$, hence
    %\begin{equation}\label{eq:binomialcoeff}
    %    \binom{p}{k}=\frac{p\cdot (p-1)!}{k!\cdot (p-k)!}=\pi^e \cdot a_k
    %\end{equation}
    %with $a_k \in \Os_L$. Hence
    \[
        \{ \A^{\otimes p}, 1+\pi^{n+1} T \}=\{ \A, (1+\pi^{n+1} T)^p \}=\{\A, 1+\pi^{np+1}b(T)\}
    \]
    where
    \[
        b(T)=\frac{(1+\pi^{n+1} T)^p-1}{\pi^{np+1}}.
    \]
    %Using equation~\eqref{eq:binomialcoeff} 
    We can rewrite $b(T)$ as 
    \[
        \sum_{k=1}^{p-1} \pi^{e+(n+1)k-(np+1)} a_k T^k+\pi^{p(n+1)-(np+1)} T^p.
    \]
    Note that, for every $1\leq k \leq p$, we have that from equation~\eqref{eq: e-np+n>0} 
    \[
        e+(n+1)k-np-1=(e-np)+(n+1)k-1\geq 0.
    \]
    Therefore, $b(T)\in \Os_L [T]$. Now, since by assumption $\A\in \fil_{np} \br(K)$, we have that $\{\A, 1+\pi^{np+1}b(T)\}=0$ for all $b(T)\in \Os_{K}[T]$, thus $\A^{\otimes p} \in \fil_n\br(K)$. In a similar way,
    \[
        \{ \A^{\otimes p}, 1+\pi^n T \}=\{ \A, (1+\pi^n T)^p \}=\{\A, 1+\pi^{np}c(T)\}
    \]
    where 
    \[
        c(T)=\frac{(1+\pi^n T)^p-1}{\pi^{np}}.
    \]
    %Again, using equation~\eqref{eq:binomialcoeff} 
    We can rewrite $c(T)$ as 
    \[
       \sum_{k=1}^{p-1} \pi^{e+nk-np} a_k T^k+ T^p.
    \]
    In this case, again equation \ref{eq: e-np+n>0} together with $1\leq k\leq p-1$, implies $e+nk-np>0$. Therefore, $c(T)\in \Os_L[T]$ and its reduction modulo $\pi$ is equal to $T^p$. It follows from \cite[(6.3.1)]{Kato} that 
    \[
        \{\A, 1+\pi^{np} c(T)\}=\lambda_\pi(\bar{c}(T)\alpha,\bar{c}(T) \beta)=\lambda_\pi(T^p\alpha,T^p \beta)=\lambda_\pi(TC(\alpha),TC(\beta))
    \]
    where the last equality follows from \cite[Lemma~2.9(2)]{BrightNewton}.
\end{proof}

For every non-negative integer $d$, we denote by $\fil_n \br(K)[d]:=\fil_n \br(K) \cap \br(K)[d]$. Kato proves that for every non-negative integer $d$ coprime to $p$,
$\br(K)[d]=\fil_0\br(K)[d]$, see \cite[Proposition~6.1]{Kato}. We will now prove some useful properties of the refined Swan conductor on $p$-power order elements. Part (b) and (c) are already proven with different techniques by Kato in \cite[Sections 4 and 5]{Kato}. 
\begin{enumerate}[label=(\alph*)]
    \item The filtration $\fil_n \br(K)[p^m]$ is finite. For $m=1$ it is proven in \cite[Proposition~4.1]{Kato} that $\br(K)[p]=\fil_{e'}\br(K)[p]$. Assume that $m>1$, $\A\in \fil_n\br(K)[p^m]$ with $n\geq e'+(m-1)e$ and $\rsw_{n,\pi}(\A)=(\alpha,\beta)$. From equation~\eqref{eq: rsw n with n>e'} we know that $\rsw_{m-e,\pi}(\A^{\otimes p})=(\bar{u}\alpha,\bar{u}\beta)$, hence the result is proven by an inductive argument on $m$. Note that, this result was essentially already proved in \cite[Proposition~17]{ieronymou2021evaluation}.
    \item Let $\A\in \fil_{np}\br(K)[p]$ with $np<e'$ and $\rsw_{np,\pi}(\A)=(\alpha,\beta)$. We know from Lemma~\ref{lemma: refined np with np<e'} that $(\alpha,\beta)\in Z^2_F\oplus Z^1_F$ and since $\A$ has order $p$
    \[
        (C(\alpha),C(\beta))=\rsw_{n,\pi}(\A^{\otimes p})=(0,0)
    \]
    Equivalently, from Proposition \ref{prop: Exact Forms in terms of Cartier Op} we get that for $np<e'$, the refined Swan conductor on the $p$-torsion takes image in $B^2_F\oplus B^1_F$.
    \item Assume $e'$ to be an integer, $\A\in \fil_{e'}\br(K)[p]$ and $\rsw_{e',\pi}(\A)=(\alpha,\beta)$. From equation~\eqref{eq: rsw n with n>e'} we get that
    \[
        -\bar{u}\alpha=C(\alpha) \; \text{ and } \; -\bar{u}\beta=C(\beta).
    \]
    Let $\zeta$ be a primitive $p$-root of unity and $c=(\zeta-1)^p\pi^{-e'}$, then $c=(c_1)^p$, with $c_1=(\zeta-1)\pi^{-e/(p-1)}$. By the properties of the Cartier operator $C(\bar{c} \alpha)=\bar{c}_1C(\alpha)=-\bar{u}\bar{c}_1 \alpha$.
    Since %\textcolor{red}{add ref? pag 132 Bloch and Kato mention it} 
    $(\zeta-1)^{p-1}\equiv -p \mod \pi^{e+1}$, there is a $v\in \Os_L$ such that
    $(\zeta-1)^{p-1}=-p+\pi^{e+1} v$. Hence
    \[
        c=\frac{(\zeta-1)^{p}}{\pi^{e'}}=\frac{\zeta-1}{\pi^{e/(p-1)}}\cdot \frac{(\zeta-1)^{p-1}}{\pi^e}=c_1\cdot\left(\frac{-p+\pi^{e+1} u}{\pi^e}\right)=c_1\cdot (-u+\pi v)=-c_1 u+\pi c_1 v.
    \]
    Thus $\bar{c}=-\bar{u}\bar{c}_1$ and hence
    \[
        \mathrm{mult}_{\bar{c}}\circ \rsw_{e',\pi}(\br(K)[p])\subseteq \Omega^2_{F,\log} \oplus \Omega^1_{F,\log}.
    \]
\end{enumerate}

\subsection{Swan conductor on Br(X)}\label{subsection: refined Swan conductor on br(X)}
Let $K^h$ be the field of fractions of the henselisation of the discrete valuation ring $\Os_{\mathcal{X},Y}$. Bright and Newton \cite{BrightNewton} define the filtration $\{\fil_n\br(X)\}_{n\geq 0}$ on $\br(X)$ as the pullback via the natural map $\br(X)\rightarrow \br(K^h)$ of the filtration $\{\fil_n\br(K^h)\}_{n\geq0}$ on $\br(K^h)$. Therefore it is possible to extend the definition of the residue map and the refined Swan conductor to elements in $\br(X)$ simply as the residue map and refined Swan conductor of the image of $\A$ in $\br(K^h)$. A priori these maps take values in $\h^1(F)$ and $\Omega^2_F\oplus \Omega^1_F$ respectively. However, Bright and Newton prove that the residue map of an element in $\fil_0\br(X)$ takes values in $\h^1(Y,\Q/\Z)\subseteq \h^1(F)$ (see \cite[Proposition~3.1(1)]{BrightNewton}) and that the refined Swan conductor on $\fil_n\br(X)$ has image in $\h^0(Y,\Omega^2_Y)\oplus \h^0(Y,\Omega^1_Y)\subseteq \Omega^2_F\oplus \Omega^1_F$ (a proof following \cite[Theorem~7.1]{Kato} can be found in \cite[Theorem~B]{BrightNewton}). 

The aim of this section is to transfer the results we got in Section~\ref{subsubsection: differential forms in positive characteristic and image of the refined Swan conductor} on the refined Swan conductor on $\br(K^h)$ to the refined Swan conductor on $\br(X)$. 
\subsubsection{Cartier operator on the sheaf of q-forms on Y}

It is possible to generalise the definition of inverse Cartier operator and Cartier operator to the sheaf of $q$-forms on $Y$. Following Illusie \cite{Illusie}, we will denote by $F_Y$ the absolute Frobenius endomorphism of $Y$ and by $Y^{(p)}$ the base change of $Y$ via the absolute Frobenius $\sigma_\ell$ of the base field $\ell$, namely
\begin{center}
    \begin{tikzcd}
        Y \arrow[dr] \arrow[r,"F_{Y/\ell}"] &Y^{(p)}\arrow[d] \arrow[r,"W"] &Y\arrow[d]\\
        &\spec(\ell) \arrow[r,"\sigma_\ell"] &\spec(\ell)
    \end{tikzcd}
\end{center}
where $W\circ F_{Y/\ell}=F_Y$; we call $F_{Y/\ell}$ the relative Frobenius of $Y$ over $\ell$.
%\begin{example} 
%If $Y=\numberset{A}_\ell^n=\spec(\ell[T_1,\dots,T_n])$, then $Y^{(p)}=\spec(\ell[T_1,\dots,T_n)\otimes_{\ell,F_\ell} \ell)=\spec(\ell[T_1,\dots,T_n])$. Moreover, 
%\begin{align*}
    %W: \ell[T_1,\dots,T_n] &\rightarrow \ell[T_1,\dots,T_n]\\
    %\sum \lambda_I \cdot T_1^{i_1}\dots T_n^{i_n} &\mapsto \sum \lambda_I^p \cdot T_1^{i_1}\dots T_n^{i_n} \\
   % F_{Y/\ell}: \ell[T_1,\dots,T_n] &\rightarrow \ell[T_1,\dots,T_n]\\
    %\sum \lambda_I \cdot T_1^{i_1}\dots T_n^{i_n} &\mapsto \sum \lambda_I \cdot T_1^{p\cdot i_1}\dots T_n^{p\cdot i_n} \\
    %F_Y: \ell[T_1,\dots,T_n] &\rightarrow \ell[T_1,\dots,T_n]\\
    %\sum \lambda_I \cdot T_1^{i_1}\dots T_n^{i_n} &\mapsto \sum \lambda_I^p \cdot T_1^{p\cdot i_1}\dots T_n^{p \cdot i_n}.
%\end{align*}
%\end{example}
Furthermore, we denote by $\Omega^{\bullet}_{Y/\ell}$ the De Rham complex of $Y/\ell$.
\begin{rmk}
    We are following \cite[Section $0.2$]{Illusie}. In his paper, Illusie works more generally with $S$-schemes, where $S$ is a scheme of positive characteristic. In this paper, we are only interested in varieties over perfect fields. In this case, the absolute De Rham complex $\Omega^\bullet_Y$ coincides with the relative De Rham complex $\Omega^\bullet_{Y/\ell}$ (since $\ell$ being perfect implies $\Omega^q_{\ell}=0$). 
\end{rmk}

For every $q\geq 0$, the differential $d\colon\Omega^q_{Y}\rightarrow \Omega^{q+1}_{Y}$ is $\Os_{Y^{(p)}}$-linear, hence 
$$Z^i_{Y}:=\ker(d\colon \Omega^i_{Y}\rightarrow \Omega^{i+1}_{Y}) \quad \text{and} \quad B^i_{Y}:=\mathrm{im}(d\colon\Omega^{i-1}_{Y}\rightarrow \Omega^{i}_{Y})$$
are $\Os_{Y^{(p)}}$-modules and the abelian sheaf $\mathcal{H}^i(\Omega^{\bullet}_{Y})$ is also a sheaf of $\Os_{Y^{(p)}}$-modules.
%This is the generalisation for varieties of what happens for differential forms over a field $F$ of positive characteristic. 
\begin{defi}[Inverse Cartier operator]
    For every $q\geq 0$ there is a morphism of $\Os_Y$-modules, called the \emph{inverse Cartier operator}
$$C^{-1}_Y\colon\Omega^q_{Y}\rightarrow W^* \mathcal{H}^q(\Omega^{\bullet}_{Y}).$$

\end{defi}
\begin{rmk}
    Since $\Omega^q_{Y^{(p)}}=W^* \Omega^q_{Y}$, by adjunction we get a morphism of $\Os_{Y^{(p)}}$-modules
    $$C^{-1}_{Y/\ell}\colon \Omega^q_{Y^{(p)}}\rightarrow \mathcal{H}^q(\Omega^{\bullet}_{Y}).$$
\end{rmk}

\begin{thm}
If $Y$ is a smooth variety over $\ell$, then $C_{Y/\ell}^{-1}$ is an isomorphism.
\end{thm}
\begin{proof}
    See \cite[Theorem $0.2.1.9$]{Illusie}.
\end{proof}
From now on we are assuming that $Y$ is a smooth and proper variety over $\ell$. In this case, we denote by $C_{Y/\ell}$ the inverse of $C^{-1}_{Y/\ell}$. 
We are ready to define the sheaf of logarithmic forms on $Y$. 
\begin{defi}[Logarithmic forms]
    For every non-negative integer $q$, we denote by 
    $$\Omega^q_{Y,\log}:=\ker(W^*-C_{Y/\ell} \colon Z^q_{Y}\rightarrow \Omega^q_{Y^{(p)}}).$$
    The sheaf $\Omega^q_{Y,\log}$ is called the sheaf of \emph{logarithmic} $q$-forms on $Y$.
\end{defi}
The following theorem is the analogue of Theorem \ref{thm: Log forms on field F} for fields of positive characteristic. 
\begin{thm}\label{thm: log forms on variety Y}
The sheaf $\Omega^q_{Y,\log}$ is the subsheaf of $\Omega^q_{Y}$ generated $\et$ale-locally by the logarithmic differentials, i.e. the sections of the form 
$$\frac{dy_1}{y_1}\wedge \dots \wedge \frac{dy_q}{y_q} \text{ with }y_i\in \Os_Y^*.$$ 
\end{thm}
\begin{proof}
    See \cite[Theorem $0.2.4.2$]{Illusie}.
\end{proof} 
The Cartier operator induces an exact sequence
\[
0\rightarrow \h^0(Y,B^q_Y) \rightarrow \h^0(Y,Z^q_Y)\xrightarrow{C_{Y}} \h^0(Y,\Omega^q_Y)
\]
The following lemma describes the interaction between global differential forms on $Y$ and their image in $\Omega^q_F$. 
\begin{lemma}\label{lemma: (a) global form is log (b) global form is exact}
    Let $\omega \in \h^0(Y,\Omega^q_Y)$, then:
    \begin{enumerate}[label=(\alph*)]
        \item $\omega\in \h^0(Y,\Omega^q_{Y,\log})$ if and only if the image of $\omega$ in $\Omega^q_F$ lies in $\Omega^q_{F,\log}$;
        \item $\omega\in \h^0(Y,B^q_Y)$ if and only if the image of $\omega$ in $\Omega^q_F$ lies in $B^q_F$.
    \end{enumerate}
\end{lemma}
\begin{proof}
    For all non-negative integers $q$, we have natural inclusions $\h^0(Y,\Omega^q_Y)\hookrightarrow \Omega^q_F$.
    These inclusions are compatible with the differential maps and with the Cartier operator. The proof of the first part of the lemma follows from the definition of logarithmic forms, while the second part is an immediate consequence of Proposition \ref{prop: Exact Forms in terms of Cartier Op}. 
\end{proof}

\begin{corollary}[On the image of $\rsw_{n,\pi}$]\label{cor: on the image of the refined swan conductor}
    Let $\A\in \fil_n \br(X)$ and $u=p \pi^{-e}\in \Os_L^\times$; then one of the following cases occurs:
    \begin{enumerate}[label=(\alph*)]
        \item If $n\nmid p$, then 
        \[
            \mathrm{pr}_2\circ \rsw_{n,\pi}\colon \fil_n \br(X)\rightarrow \h^0(Y,\Omega^1_Y) 
        \]
        has kernel equal to $\fil_{n-1}\br(X)$.
         \item If $p\mid n$ and $n<e'$ we write $n=mp$, then $(\alpha,\beta)$ lies in $ \h^0(Y,Z^2_Y)\oplus \h^0(Y,Z^1_Y)$ and the following diagram
        \begin{equation*}
            \begin{tikzcd}
                \fil_n \br(X) \arrow[d,"(-)^{\otimes p}"]\arrow[r,"\rsw_{n,\pi}"] &\h^0(Y,Z^2_Y)\oplus \h^0(Y,Z^1_Y)\arrow[d,"C"]\\
                \fil_m \br(X) \arrow[r,"\rsw_{m,\pi}"] &\h^0(Y,\Omega^2_Y)\oplus \h^0(Y,\Omega^1_Y).
            \end{tikzcd}
        \end{equation*}
        commutes.
        \item If $n=e'$, then $(\alpha,\beta)$ lies in $ \h^0(Y,Z^2_Y)\oplus \h^0(Y,Z^1_Y)$ and the following diagram
        \begin{equation*}
            \begin{tikzcd}
                \fil_{e'} \br(X) \arrow[d,"(-)^{\otimes p}"]\arrow[r,"\rsw_{n,\pi}"] &\h^0(Y,Z^2_Y)\oplus \h^0(Y,Z^1_Y)\arrow[d,"\mathrm{mult}_{\bar{u}}+C"]\\
                \fil_{e'-e} \br(X) \arrow[r,"\rsw_{n-e,\pi}"] &\h^0(Y,\Omega^2_Y)\oplus \h^0(Y,\Omega^1_Y).
            \end{tikzcd}
        \end{equation*}
        commutes. Moreover, there exists a constant $\bar{c}\in \ell^\times$ such that 
        \begin{equation*}
                \mathrm{mult}_{\bar{c}}\left(\rsw_{e',\pi} \fil_{e'}\br(X)[p]\right)\subseteq \h^0(Y,\Omega^2_{Y,\log})\oplus \h^0(Y,\Omega^1_{Y,\log}).
        \end{equation*}
        \item if $p\mid n$ and $n>e'$, then $(\alpha,\beta)$ lies in $\h^0(Y,Z^2_Y)\oplus \h^0(Y,Z^1_Y)$ and the following diagram
        \begin{equation*}
            \begin{tikzcd}
                \fil_n \br(X) \arrow[d,"(-)^{\otimes p}"]\arrow[r,"\rsw_{n,\pi}"] &\h^0(Y,Z^2_Y)\oplus \h^0(Y,Z^1_Y)\arrow[d,"\mathrm{mult}_{\bar{u}}"]\\
                \fil_{n-e} \br(X) \arrow[r,"\rsw_{n-e,\pi}"] &\h^0(Y,\Omega^2_Y)\oplus \h^0(Y,\Omega^1_Y).
            \end{tikzcd}
        \end{equation*}
        commutes.
    \end{enumerate}
\end{corollary}
\begin{proof}
    All these properties are a consequence of Section~\ref{subsubsection: differential forms in positive characteristic and image of the refined Swan conductor}. More precisely: $(a)$ follows from Remark \ref{rmk: rsw comp proj_2}$.2$; $(b)$ follows from Lemma \ref{lemma: refined np with np<e'}; $(c)$ and $(d)$ are a direct consequence of \cite[Lemma~$2.10$]{BrightNewton}.
\end{proof}
\subsection{Computation with the refined Swan conductor}
The aim of this section is to collect some results that will allow us to compute in the examples presented in this paper the Swan conductor and the refined Swan conductor of elements in $\br(X)$.

We are in the general setting of Section \ref{Section: General Setting}. Recall that we denote by $K^h$ the fraction field of the henselianisation $R$ of the discrete valuation ring $\Os_{\mathcal{X},Y}$. In this section we work under the additional assumption that $L$ contains a primitive $p$-root of unity $\zeta$. In this case the map from $\Z/p\Z(1)$ to $\Z/p\Z(2)$ sending $1$ to $\zeta$ induces an isomorphism\begin{equation}\label{eqIsomoprhismRootUnity}
    \br(K^h)[p]\simeq \h^2\left(K^h,\Z/p\Z(2)\right)=:h^2(K^h).
\end{equation}
For any two non-zero elements $x,y\in K^h$ we will denote by $\{x,y\}$ the cup product of $\delta(x)$ with $\delta(y)$, where $\delta\colon (K^h)^\times \rightarrow \h^1(K^h,\Z/p\Z(1))$ is the connecting morphism coming from the Kummer sequence.
In \cite[Proposition~4.7.1]{GilleSzamuely} it is proven that the isomorphism~\eqref{eqIsomoprhismRootUnity} sends a symbol $\{x,y\}$ to the class of the cyclic algebra $(x,y)_p$, which is defined as 
\[
    (x,y)_p:=\langle a,b\mid a^p-a=x, \, b^p=y, \, ab=b(a+1)\rangle.
\]
See \cite[Section~2.5]{GilleSzamuely} for more details. Bloch and Kato \cite{BlochKatoEtale} define a decreasing filtration $\{U^m h^2(K^h)\}_{m\geq 0}$ on $h^2(K^h)$ as follows: $U^0h^2(K^h)=h^2(K^h)$ and for $m\geq 1, \, U^m h^2(K^h)$ is the subgroup of $h^2(K^h)$ generated by symbols of the form $$\{1+\pi^m x, y\}, \text{ with }x\in R \text{ and }y\in (K^h)^\times.$$
This filtration is strictly related to the filtration $\{\fil_n\br(K^h)[p]\}$ on the $p$-torsion of $\br(K^h)$. More precisely, we have the following result.
\begin{prop}\label{prop: explicit description of fil_{e'-m}}
Let $e=\text{ord}_K(p)$ and $e'=ep(p-1)^{-1}$. For $0\leq m \leq e'$ the isomorphism of equation $(\ref{eqIsomoprhismRootUnity})$ induces isomorphism
\[
\fil_{e'-m} \br(K^h)[p]\simeq U^mh^2(K^h).
\]
Thus, $\fil_{e'-m}\br(K^h)[p]$ is the subgroup of $\br(K^h)[p]$ generated by the class of cyclic algebras of the form $(1+\pi^{e'-m} x, y)_p$ with $x\in R$ and $y\in (K^h)^\times$.
\end{prop}
\begin{proof}
The first isomorphism follows from \cite[Proposition 4.1(6)]{Kato}. While, the description of the cyclic algebras generating $\fil_{e'-m}\br(K^h)[p]$ is an immediate consequence of the explicit description of the isomorphism~\eqref{eqIsomoprhismRootUnity}.
\end{proof}
Bloch and Kato give a description of the graded pieces $\gr^m$ of the filtration $\{U^m h^2(K^h)\}_{m\geq 0}$ in terms of differential forms $\Omega^q_F$. 
\begin{prop}\label{propGradedPieces}
We have the following description of the grading pieces $\gr^m:=\frac{U^mh^2(K^h)}{U^{m+1}h^2(K^h)}.$
\begin{enumerate}[label=(\alph*)]
    \item $U^m h^2(K^h)=\{0\}$ for $m> e'$; $U^{e'}h^2(K^h)$ coincides with the image of the injective map
    \begin{align*}
        \lambda_\pi \colon \h^2_p(F)\oplus \h^{1}_p(F)&\rightarrow \gr^{e'}=h^2(K^h)\\
        \left(\left[ \bar{x}\frac{d\bar{y}}{\bar{y}}\right],0\right) &\mapsto \{1+(\zeta-1)^px,y\}\\
        \left(0, \left[ \bar{x}\right] \right)&\mapsto \{1+(\zeta-1)^px,\pi\}
    \end{align*}
    where $x$ and $y$ are any lifts of $\bar{x}$ and $\bar{y}$ to $K^h$.  
    \item Let $0<m<e'$ and $p\nmid m$. Then we have an isomorphism 
    \begin{align*}
        \rho_m\colon  \Omega^{1}_F &\xrightarrow{\simeq} \gr^m\\
        \bar{x}\frac{d\bar{y}}{\bar{y}} &\mapsto \{1+\pi^m x,y\}
    \end{align*}
    where $x$ and $y$ are any lifts of $\bar{x}$ and $\bar{y}$ to $K^h$.  
    \item Let $0<m<e'$ with $p\mid m$. Then we have an isomorphism 
    \begin{align*}
        \rho_m\colon \Omega^{1}_F/Z^1_F\oplus \Omega^0_F/Z^0_F &\xrightarrow{\simeq} \gr^m\\
        \left(\left[\bar{x}\frac{d\bar{y}}{\bar{y}}\right],0\right) &\mapsto \{1+\pi^m x,y\}\\
        \left(0,\left[\bar{x}\right]\right)&\mapsto \{1+\pi^m x,\pi\}
    \end{align*}
    where $x$ and $y$ are any lifts of $\bar{x}$ and $\bar{y}$ to $K^h$.  
    \item We have an isomorphism 
    \begin{align*}
        \rho_0\colon \Omega^{2}_{F,\log}\oplus \Omega^{1}_{F,\log} &\xrightarrow{\simeq} \gr^0\\
        \left(\frac{d\bar{y}_1}{\bar{y}_1}\wedge \frac{d\bar{y}_{2}}{\bar{y}_{2}},0\right) &\mapsto \{y_1,y_2\}\\
        \left(0,\frac{d\bar{y}}{\bar{y}}\right)&\mapsto \{ y,\pi\}
    \end{align*}
    where $y,y_1$ and $y_2$ are any lifts of $\bar{y},\bar{y}_1$ and $\bar{y}_2$ to $K^h$.  
\end{enumerate}
\end{prop}
\begin{proof}
See \cite[Section 4]{BlochKatoEtale}.
\end{proof}
As a consequence of the Merkurjev-Suslin Theorem \cite[Theorem~8.6.5]{GilleSzamuely} we know that $\h^2_p(K^h)$ is generated by the symbols $\{x,y\}$ with $x,y\in K^h$. By the isomorphism~\eqref{eqIsomoprhismRootUnity} $\br(K^h)[p]$ is also generated by the classes of cyclic algebras of the form $(x,y)_p$ with $x,y\in K^h$. The previous proposition gives a way to detect to which piece of the filtration $\{\fil_n\br(K^h)[p]\}_{n\geq 0}$ does a given cyclic algebra belong.

In \cite{Kato} Kato links the maps $\rho_m$ appearing in Proposition \ref{propGradedPieces} with the computation of the refined Swan conductor on elements in $\br(K^h)[p]$.
\begin{lemma}\label{lemma: rsw and map rho_m}
    Let $\A$ be an element in $\fil_n \br(K^h)[p]$ and $i:=e'-n$, let $(\alpha,\beta)\in \Omega^2_F\oplus \Omega^{1}_F$ be such that $$\rho_{i}(\alpha,\beta)=[\A] \in \gr^i.$$
    The map $\rsw_{n,\pi}\colon \fil_n\br(K^h)[p]\rightarrow \Omega^2_F\oplus \Omega^1_F$ sends $\A$ to $(\bar{c}^{-1}\cdot \alpha,\bar{c}^{-1}\cdot \beta)$, where $\bar{c}$ reduction modulo $\pi$ of $c=\pi^{-e'}(\zeta-1)^{p}$. 
\end{lemma}
\begin{proof}
    See \cite[Lemma 4.3]{Kato}.
\end{proof}
 
\subsubsection{The case of K3 surfaces}\label{subsubsection: the case of K3 surfaces}
In this paper the examples will always be about K$3$ surfaces with good reduction. The special fibre of a K$3$ surface with good reduction is still a K$3$ surface, see for example \cite[Remark~11.5]{BrightNewton}. We start by stating the following well known result, of which we include the proof as we could not find it in standard literature.
\begin{lemma}\label{lemma: criterium ordinary k3}
    Let $p$ be a prime number and $Y$ a K$3$ surface over the finite field $\F_{p^n}$ for some non-negative $n$. Then $Y$ is ordinary if and only if $|Y(\F_{p^n})| \not \equiv 1 \mod p$.
\end{lemma}
\begin{proof}
    The proof is an almost immediate consequence of \cite[Section~1]{BogomolovZarhin}. Let $\bar{Y}$ be the base change of $Y$ to an algebraic closure of $\F_{p^n}$ and $l$ be a prime different from $p$. The Frobenius endomorphism $F$ of $\bar{Y}$ acts by functoriality on $22$-dimensional $\Q_l$-vector space  
    \[
        \h^2_\et(\bar{Y},\Q_l):=\h^2_\et(\bar{Y},\Z_l)\otimes_{\Z_l} \Q_l. 
    \]
    Let $\lambda_i$ with $i=1,\dots,22$ be the corresponding eigenvalues. From the Lefschetz trace formula \cite[Section~1]{KatzCrystallineCohomology} we get
    \begin{equation}\label{eq: Lefschetz trace formula for k3}
        |Y(\F_{p^n})|=\sum (-1)^i \mathrm{Tr}(\mathrm{F},\h^i_\et(\bar{Y},\Q_l)=1+\sum_{i=1}^{22} \lambda_i + p^{2n}.
    \end{equation}
    The last equality follows from the fact that for K$3$ surfaces both the first and the third Betti numbers are trivial and $\h^0_\et(\bar{Y},\Q_l)$ and $\h^4_\et(\bar{Y},\Q_l)$ are $1$-dimensional $\Q_l$-vector space with Frobenius eigenvalue equal to $1$ and $p^{2n}$ respectively \cite[Theorem~1.6]{Deligne}.

    %Let $L:=\Q_l(\{\lambda_i\})$ and $\mathfrak{p}$ be a maximal ideal in $\Os_L$ that lies above $\ip$, assume that the eigenvalues are order such that \[\mathrm{ord}_{\mathfrak{p}}(\lambda_i)\leq \mathrm{ord}_{\ip}(\lambda_j), \quad \text{for all }i<j.\]
    It is proven in \cite[Lemma~1.1]{BogomolovZarhin} that a K$3$ surface $Y$ is ordinary if and only if $\sum_{i=1}^{22} \lambda_i$ is not divisible by $p$. It is therefore clear from~\eqref{eq: Lefschetz trace formula for k3} that
    \[
        |Y(\F_{p^n})|\equiv 1+\sum_{i=1}^{22} \lambda_i\not \equiv 1\mod p.
    \]
    if and only if $Y$ is ordinary.
\end{proof}
\begin{lemma}\label{lemma: explicit global 2-form}
    Let $\ell$ be a field of characteristic $2$ and $f(x_0,x_1,x_2,x_3)\in \ell[x_0,x_1,x_2,x_3]$ be a homogeneous polynomial of degree $4$. Assume that the corresponding projective variety $Y$ is smooth. Then $Y$ is a K$3$ surface and the $1$-dimensional $\ell$-vector space of global $2$-forms is generated by the $2$-form
    \[
        \omega=\frac{d\left(\frac{y}{x}\right)\wedge d\left(\frac{z}{x}\right)}{\frac{1}{x^3}\cdot \frac{\partial f}{\partial w}}.
    \]
\end{lemma}
\begin{proof}
    The first part follows from \cite[Example 1.3(i)]{Huybrechts}. 
     For every permutation $\{p,q,i,j\}$ of $\{0,1,2,3\}$ we define $W_{p,q}\subseteq Y$
    as the open subset of $Y$ where $x_p\cdot  \frac{\partial f}{\partial x_q}$ does not vanish. Moreover, we set
    $$\omega_{p,q}:=\frac{d\left(\frac{x_i}{x_p}\right)\wedge d\left(\frac{x_j}{x_p}\right)}{\frac{1}{x_p^3}\cdot \frac{\partial f}{\partial x_q}}\in \h^0(W_{p,q},\Omega^2_Y).$$
    Since $Y$ is smooth, the open sets $\{W_{p,q}\}$ cover it. It is easy to check that, since we are working over a field of characteristic $2$, for every $(p,q)\ne (p',q')$, $\omega_{p,q}=\omega_{p',q'}$ on $W_{p,q}\cap W_{p',q'}$.
\end{proof}
If the K$3$ surface has good ordinary reduction, then the remark that follows shows that there is a strong link between global logarithmic $2$-forms on $Y$ and $p$-torsion elements on $X$ having non-constant evaluation map.
\begin{rmk}[K$3$ surfaces with good ordinary reduction]\label{rmk: K3 with good reduction global 2 form }
    Let $X$ be a K$3$ surface with good ordinary reduction defined over a $p$-adic field $L$ having absolute ramification index $p-1$. Assume that there is an element $\A\in \br(X)[p]$ that does not belong to $\fil_0\br(X)$. Then, $\A\in \fil_{n}\br(X)$ for some $n\geq 1$. From Section~\ref{subsection: properties of the image of the refined Swan conductor} we can assume $n\leq e' =\frac{e p}{p-1}=p$, since $\fil_{e'}\br(K)[p]=\br(K)[p]$. Moreover, for $n<e'=p$ we have from Corollary~\ref{cor: on the image of the refined swan conductor}(a) together with the fact that for K$3$ surfaces we have $\h^0(Y,\Omega^1_Y)=0$ $\fil_n\br(X)[p]=\fil_0\br(X)[p]$. By Corollary~\ref{cor: on the image of the refined swan conductor}(c) we know that there is a constant $\bar{c}\in \ell^\times$ such that 
    \[
        \mathrm{mult}_{\bar{c}}\left(\fil_p\br(X)[p]\right)\subseteq \h^0(Y,\Omega^2_{Y,\log})\subseteq \h^0(\bar{Y},\Omega^2_{\bar{Y},\log}).
    \]
    From Lemma~\ref{lemma: rsw and map rho_m} we know that the class of $\A$ in $\frac{\fil_p\br(K^h)[p]}{\fil_{p-1}\br(K^h)[p]}\simeq \gr^0$ has to be such that 
    \[
        [\A]=\rho_0(\omega,0)
    \]
    where $\omega\in \Omega^2_{F,\log}$ is the image in $\Omega^2_F$ of a non-trivial logarithmic form in $\h^0(Y,\Omega^2_{Y,\log})$. Moreover, $\h^0(\bar{Y},\Omega^2_{\bar{Y},\log})$ is a $1$-dimensional $\F_p$-vector space (cfr~\eqref{eq: cohomology of M^q_1 ordinary case}). 
\end{rmk}
We point out that in \cite[Proposition~2.3]{ieronymou2023odd} it is proven that for K$3$ surface an element $\A$ lies in $\fil_0 \br(X)$ if and only if $\ev_\A\colon X(L)\rightarrow \br(L)$ is constant. It is already known from \cite[Lemma~11.3]{BrightNewton} that, since for K$3$ surface $\h^1(\bar{Y},\Z/p\Z)=0$, then $\fil_0\br(X)=\Ev_0\br(X)=\Ev_{-1}\br(X)$. However, from the result proven by Ieronymou we know that in order to detect whether $\A$ belongs to $\fil_0\br(X)$ it is enough to look at the corresponding evaluation map on the $L$-points, $X(L)$. 
\begin{lemma}\label{lemma: transcendental nature of A}
    Let $X$ be a K$3$ surface and $\A\in \br(X)$ be such that $\A\notin \fil_0\br(X)$. Then $\A \notin \br_1(X)$.
\end{lemma}
\begin{proof}
    As we just pointed out, if $\A\notin\fil_0\br(X)$, then the evaluation map attached to $\A$ in non-constant on $X(L)$. The result now follows from the fact that Colliot-Thélène and Skorobogatov proved \cite[Proposition~2.3]{colliotskogoodred} that for every element in the algebraic Brauer group the associated evaluation map at a prime with good reduction has to be constant.
\end{proof}
\subsubsection{Refined Swan conductor and extension of the base field}
In this section we want to analyse what happens to the refined Swan conductor when we take a field extension $L'/L$ of the base field $L$. Bright and Newton prove the following result.
\begin{lemma}\label{lemma: rsw of field extension K'/K}
    Let $K'/K$ be a finite extension of Henselian discrete valuation fields of ramification index $e$. Let $\pi'$ be a uniformiser in $K'$, $F'$ be the residue field of $K'$ and define $\bar{a}\in F'$ to be the reduction of $\pi(\pi')^{-e}$. Let $\chi \in \fil_n \br(K)$, and let 
    \[
    \mathrm{res}\colon \br(K)\rightarrow \br(K')
    \]
    be the restriction map. Then $\mathrm{res}(\chi)\in \fil_{en}\br(K')$ and if $\rsw_{n,\pi}(\chi)=(\alpha,\beta)$, then 
    \[
        \rsw_{en,\pi'}(\mathrm{res}(\chi))=(\bar{a}^{-n}(\alpha+\beta\wedge d\log(\bar{a}), \bar{a}^{-n}e \beta).
    \]
\end{lemma}
\begin{proof}
    See \cite[Lemma~2.7]{BrightNewton}.
\end{proof}
We are back to the general setting of Section~\ref{Section: General Setting}. The aim of this section is to use this result to prove the following Lemma. 
\begin{lemma}[Base change]\label{lemma: base change L'/L}
Let $L'/L$ be a finite field extension, with ramification index $e_{L'/L}$. Let $\pi'$ be an uniformiser in $L'$ and $\ell'$ its residue field.  Let $\A\in \br(X)$ and let 
\[
    \mathrm{res}\colon \br(X)\rightarrow \br(X_L)
\]
be the restriction map. Then $\mathrm{res}(\A)\in \fil_{e_{L'/L}n}\br(X_L)$ and if $\rsw_{n,\pi}(\A)=(\alpha,\beta)$ with $(\alpha,\beta)\in \h^0(Y,\Omega^2_Y)\oplus \h^0(Y,\Omega^1_Y)$, then 
\[
   \rsw_{e_{L'/L}n,\pi'}(\mathrm{res}(\A))=(\bar{a}^{-n}\alpha, \bar{a}^{-n}e_{L'/L}\beta)\in \h^0(Y_{\ell'},\Omega^2_{Y_{\ell'}})\oplus\h^0(Y_{\ell'},\Omega^1_{Y_{\ell'}}) 
\]
with $\bar{a}\in \ell'$ reduction of $\pi(\pi')^{-e_{L'/L}}$.
\end{lemma}
The refined Swan conductor of an element $\A\in \br(X)$ is defined through the refined Swan conductor of its image in the discrete henselian valuation field $K^h$. Namely, we have the following commutative diagram
\[
    \begin{tikzcd}
        \fil_n\br(X) \arrow[d] \arrow[r,"\rsw_{n,\pi}"] &\h^0(Y,\Omega^2_Y)\oplus \h^0(Y,\Omega^1_Y) \arrow[d,hookrightarrow]\\
        \fil_n\br(K^h)\arrow[r,"\rsw_{n,\pi}"] &\Omega^2_F\oplus \Omega^1_F.
    \end{tikzcd}
\]We recall the construction of $K^h$: let $\eta$ be the generic point of $Y\subseteq\mathcal{X}$, then we define $R$ as henselianisation of the discrete valuation ring $\Os_{\mathcal{X},\eta}$ and $K^h$ as the fraction field of $R$. 
%By assumption $\mathcal{X}$ is smooth, and $Y$ is of co-dimension $1$, hence $\Os_{\mathcal{X},\eta}$ is a discrete valuation ring. 
The construction of $\Os_{\mathcal{X},\eta}$ (and hence of $K^h$) is local on $\mathcal{X}$. From now on we will therefore assume $\mathcal{X}=\spec(A)$, with $A$ smooth $\Os_L$-algebra, $Y=\spec(A/\pi A)$; hence $\eta=(\pi)\in \spec(A)$ and $\Os_{\mathcal{X},\eta}=A_{(\pi)}$. We can re-write diagram~\eqref{eq:diagram1} as:
\begin{equation}\label{eq:diagramAffine}
    \begin{tikzcd}
        A\otimes_{\Os_L} L & \arrow[l] A \arrow[r] &A/\pi A\\
        L \arrow[u] & \arrow[l] \Os_L \arrow[r]\arrow[u] & \ell. \arrow[u]
    \end{tikzcd}
\end{equation}
\begin{lemma}
     The uniformiser $\pi$ is also a uniformiser for $K^h$. Moreover, $\mathrm{ord}_{K^h}(p)=\mathrm{ord}_L(p)$.
\end{lemma}
\begin{proof}
    The uniformiser $\pi$ is also the generator of the maximal ideal of $A_{(\pi)}$, hence of its henselianisation $R$. The equality $(p)=(\pi)^e$ as ideals on $\Os_L$ implies that $p\in (\pi)^e R$, hence $e_1:=\mathrm{ord}_{K^h}(p)\leq e$. The equality between the two orders follows from the fact that for every $m\geq 1$, $(\pi)^{m}R \cap \Os_L=(\pi)^{m}\Os_L$.
\end{proof}
We denote by $L'$ a finite field extension of $L$, by $\Os_{L'}$ its ring of integers with uniformiser $\pi'$ and residue field $\ell'$. Moreover, we denote by $X'$, $\mathcal{X}'$ and $Y'$ the base change of $X$, $\mathcal{X}$ and $Y$ to $\spec(L'), \spec(\Os_{L'})$ and $\spec(\ell')$ respectively.  Let $(K')^h$ be the fraction field of $R'$, where $R'$ is the henselianisation of the discrete valuation ring $\Os_{\mathcal{X}',Y'}$. In this setting
\[
    \frac{A}{\pi A}\otimes_{\ell} \ell' = A\otimes_{\Os_{L}} \ell \otimes_\ell \ell' =A\otimes_{\Os_L} \Os_{L'} \otimes_{\Os_{L'}} \ell' =\frac{A\otimes_{\Os_L} \Os_{L'}}{(1\otimes \pi')}.
\]
Thus, the generic point $\eta'$ of $Y'$ is the ideal generated by $1\otimes \pi'$ and $\Os_{\mathcal{X}',Y'}$ becomes the ring $(A\otimes_{\Os_L} \Os_{L'})_{(1\otimes \pi')}.$
\begin{lemma}
The field extension $(K')^h/K^h$ is finite with ramification index $e_{L'/L}$.
\end{lemma}
\begin{proof}
We start by noticing that
\[
    \Os_{\mathcal{X}',\eta'}\simeq \Os_{\mathcal{X},\eta} \otimes_{\Os_L}\Os_{L'}.
\]
We have that $\Os_{\mathcal{X},\eta} \otimes_{\Os_L}\Os_{L'}=S^{-1}( A\otimes_{\Os_L}\Os_{L'})$, with $S=(A\setminus (\pi))\cdot A\otimes_{\Os_L}\Os_{L'}$, while $\Os_{\mathcal{X}',\eta'}=T^{-1}(A\otimes_{\Os_L} \Os_{L'})$, with $T=(A\otimes_{\Os_L} \Os_{L'})\setminus (1\otimes \pi')$. The isomorphism follows from the equality $(1\otimes \pi')^{e_{L'/L}}=(\pi \otimes 1)$ together with the fact that $\Os_{L'}$ is a free $\Os_L$-module with basis $\{1,\pi',\dots,(\pi')^{e_{L'/L}}\}$.

As a second step we show that 
\[
    R'\simeq R\otimes_{\Os_L} \Os_{L'}.
\]
The discrete valuation ring $\Os_{L'}$ is a finite $\Os_L$-module, hence we get that the natural map 
\[
    R \rightarrow R\otimes_{\Os_L}\Os_{L'}
\]
is finite; therefore \cite[Lemma $10.153.4$]{Stacks} implies that $R\otimes_{\Os_L} \Os_{L'}$ is henselian and therefore by \cite[Lemma $10.156.1$]{Stacks} 
\[
    R'=R\otimes_{\Os_L}\Os_{L'}.
\]
As a final step, we notice that 
\[
    (K')^h=R'\left[\frac{1}{\pi'}\right]=R'\left[\frac{1}{\pi}\right]=R\left[\frac{1}{\pi}\right]\otimes_{L} L'=K^h\otimes_L L'.
\]
\end{proof}
\begin{proof}[proof of Lemma~\ref{lemma: base change L'/L}]
    It follows immediately from the previous lemma together with Lemma \ref{lemma: rsw of field extension K'/K} and the fact that since $\bar{a}\in \ell'$, which is a finite field, $d\log(\bar{a})=0$.
\end{proof}
\begin{corollary}\label{cor: refined Swan conductor and algebraic elements}
    Assume that $\A\in \fil_n \br(X)$ for some $n\geq 1$ is such that $\rsw_{n,\pi}(\A)=(\alpha,\beta)$ with $\alpha \ne 0$, then $\A\notin \br_1(X)$, i.e. $\A$ is a transcendental element in the Brauer group of $X$.
\end{corollary}
\begin{proof}
    Assume $\A$ to be in $\br_1(X)$; then by definition of $\br_1(X)$ there is a finite field extension $L'/L$ such that $\mathrm{res}(\A)=0$ in $\br(X_{L'})$, where $\mathrm{res}$ is the restriction map from $\br(X)$ to $\br(X_{L'})$. Let $e_{L'/L}$ be the ramification index of the extension, $\pi'$ be a uniformiser of $L'$ and $\ell'$ its residue field. We know from Lemma~\ref{lemma: base change L'/L} that 
        \[
        \rsw_{e(L'/L)n,\pi_{L'}}(\mathrm{res}(\A))=(\bar{a}^{-n}\cdot \alpha,\bar{a}^{-n}e_{L'/L}\cdot \beta)
        \]
        where $\bar{a}^{-n}\in (\ell')^\times$. Hence, $\rsw_{e(L'/L)n,\pi_{L'}}(\mathrm{res}(\A))\ne (0,0)$ and therefore $\mathrm{res}(\A)$ can not be the trivial element.
\end{proof}
We end this section with a lemma that shows how the evaluation map behaves under base change without the assumption of good reduction for $X$. 
\begin{lemma}\label{lemma: base change no good-reduction}
    Let $X$ be a variety over a p-adic field $L$, not necessarily having good reduction, and let $\A\in \br(X)\{p\}$ be such that $\ev_\A\colon X(L)\rightarrow \br(L)$ is non-constant. Then for every field extension $L'/L$ with order co-prime to $p$ we have that $\mathrm{res}(\A)\in \br(X_{L'})$ has also a non-constant evaluation map $\ev_{\mathrm{res}(\A)}\colon X_{L'}(L')\rightarrow \br(L')$.
\end{lemma}
\begin{proof}
   Let $P,Q\in X(L)$ be such that 
    \[
        \ev_{\A}(P)\ne \ev_{\A}(Q).
    \]
    Denote by $P'$ and $Q'$ the base change of $P$ and $Q$ to $L'$, i.e. we have the following commutative diagrams
    \[
    \begin{tikzcd}
        X_{L'} \arrow[d] \arrow[r,"\psi"] &X_0 \arrow[d]\\
        \spec(L') \arrow[r,"\varphi"] &\spec(L)
    \end{tikzcd}
    \quad 
    \begin{tikzcd}
        \spec(L') \arrow[d,"\varphi"] \arrow[r,"P'"] &X_{L'}\arrow[d,"\psi"]\\
        \spec(L) \arrow[r,"P"] &X
    \end{tikzcd}
    \quad \begin{tikzcd}
        \spec(L') \arrow[d,"\varphi"] \arrow[r,"Q'"] &X_{L'}\arrow[d,"\psi"]\\
        \spec(L) \arrow[r,"Q"] &X.
    \end{tikzcd}
    \]
    Then 
    \[
    \ev_{\mathrm{res}(\A)}(P')=\br(P')(\br(\psi_L)(\A_0))=\br(\varphi_L)\br(P)(\A)=\br(\varphi_L)(\ev_{\A}(P)).
    \]
    Finally since $L'/L$ has degree co-prime to $p$ the map $\br(\varphi)\colon \br(L)\rightarrow \br(L')$ is injective on elements of $p$-order; hence
    \[
    \ev_{\mathrm{res}(A)}(P')\ne \ev_{\mathrm{res}(A)}(Q').
    \]
\end{proof}
\section{Ordinary good reduction }\label{section:ordinarygoodred}
We start by giving the definition of ordinary variety. 
\begin{defi}\label{def: ordinary var}
    Let $Y$ be a smooth, proper and geometrically integral variety over a perfect field $\ell$ of positive characteristic. We say that $Y$ is \emph{ordinary} if $\h^i(Y,B^q_Y)=0$ for every $i,q$.
\end{defi}
\subsection{Proof of Theorem~\ref{thm: ordinary good reduction}}\label{subsection: proof of good ordinary reduction theorem}
Throughout this section, we will assume that $X$ is such that its special fibre $Y$ is smooth, ordinary and such that both $\h^0(Y,\Omega^1_Y)$ and $\h^1(\bar{Y},\Z/p\Z)$ are trivial. In this case, the Cartier operator gives a bijection between the global closed $q$-forms and the global $q$-forms on $Y$. In fact, for every $q$ the short exact sequence $0\rightarrow B^q_Y\rightarrow Z^q_Y\xrightarrow{C}\Omega^q_Y\rightarrow 0$ induces a long exact sequence in cohomology
\[
0 \rightarrow \h^0(Y,B^q_Y)\rightarrow \h^0(Y,Z^q_Y)\xrightarrow{C}\h^0(Y,\Omega^2_Y)\rightarrow \h^1(Y,B^q_Y)\rightarrow \dots .
\]
The ordinary condition assures the vanishing of $\h^0(Y,B^q_Y)$ and $\h^1(Y,B^q_Y)$ and hence the bijectivity of the Cartier operator $C\colon\h^0(Y,Z_Y^q)\rightarrow \h^0(Y,\Omega^q_Y)$.
\begin{lemma}\label{LemmaOrdinaryK3Fil_n<e'}
    Let $n$ be an integer such that $0<n<e'$; then for every $\A\in \fil_n \br(X)$ we have
    \[
        \rsw_{n,\pi}(\A)=0
    \]
\end{lemma}
\begin{proof}
    The assumption $\h^0(Y,\Omega^1_Y)=0$ together with Lemma \ref{cor: on the image of the refined swan conductor}$(a)$ assure us that if $p\nmid n$, then $\rsw_{n,\pi}(\A)=0$. Moreover, as already pointed out in Section~\ref{subsection: properties of the image of the refined Swan conductor} if $\A\in \br(X)$ has order coprime to $p$, then $\A\in \fil_0\br(X)$. Therefore we are reduced to the case $\A\in \fil_{n} \br(X)[p^r]$, with $r\geq 1$ and $p\mid n$. We will prove the lemma by induction on $r$. If $r=1$ and $\rsw_{n,\pi}(\A)=(\alpha,\beta)$, by Corollary \ref{cor: on the image of the refined swan conductor}$(b)$ 
    \[
        (0,0)=\rsw_{n/p,\pi}(\A^{\otimes p})=(C(\alpha),C(\beta)).
    \]
    By Lemma \ref{lemma: (a) global form is log (b) global form is exact}$(b)$ it follows that $(\alpha,\beta)\in \h^0(Y,B^2_Y)\oplus \h^0(Y,B^1_Y)$, and we are done since by ordinary assumption there are no non-trivial global exact $1$ and $2$-forms. Assume the result to be true for $r-1$ and let $\A\in \br(X)[p^r]$, then again by Corollary \ref{cor: on the image of the refined swan conductor}$(b)$ together with the induction hypothesis we have that 
    \[
        (0,0)=\rsw_{n/p,\pi}(\A^{\otimes p})=(C(\alpha),C(\beta))
    \]
    and once again we get that $(\alpha,\beta)\in \h^0(Y,B^2_Y)\oplus \h^0(Y,B^1_Y)$ and therefore both $\alpha$ and $\beta$ are zero.
\end{proof}

\begin{lemma}
    Assume that $(p-1)\nmid e$; then for every $\A\in \fil_n \br(X)$ we have
    \[
        \rsw_{n,\pi}(\A)=0.
    \]
\end{lemma}
\begin{proof}
    Let $\A\in \fil_n \br(X)[p^r]$. By the previous lemma, we already know that the result is true if $n<e'$ and we know that the result is true if $p\nmid n$. Again, we work by induction on $r$. For $r=1$ the result follows from what we just said and the fact that $\fil_{e'}\br(X)[p]$ is the whole $\br(X)[p]$ (see Section~\ref{subsection: properties of the image of the refined Swan conductor}). Let $\A\in \fil_n \br(X)[p^r]$ and $\rsw_{n,\pi}(\A)=(\alpha,\beta)$. By assumption we know that $p\nmid e'$, hence if $p\mid n$ and $n\geq e'$ then $n>e'$. From Corollary \ref{cor: on the image of the refined swan conductor}$(c)$ 
    \[
        \rsw_{n-e}(\A^{\otimes p})=(\bar{u}\alpha,\bar{u}\beta)
    \]
    with $\bar{u}\in \ell^\times$ and the result follows from the induction hypothesis. 
\end{proof}
\begin{thm}
        Assume that $Y$ is an ordinary variety with no global $1$-forms and that $(p-1)\nmid e$. Then $\br(X)=\fil_{0}\br(X).$ 
\end{thm}
\begin{proof}
    It follows from the previous two lemmas.
\end{proof}
The assumption that $\h^1(\bar{Y},\Z/p \Z)=0$ implies that $\Ev_{-1}\br(X)\{p\}=\fil_0\br(X)\{p\}$, see \cite[Lemma~11.3]{BrightNewton}. Hence, combining this result with the previous theorem we get the proof of Theorem~\ref{thm: ordinary good reduction}.
\subsection{On the existence of a Brauer--Manin obstruction over a field extension}\label{subsubsection: On the existence of a Brauer Manin obstruction over a field extension}
In this section we prove a slightly stronger version of \cite[Theorem~C]{BrightNewton} for K$3$ surfaces having good ordinary reduction. We show that it is always possible (after a finite field extension) to find an element of order \emph{exactly} $p$ with non-constant evaluation map.  
%under the additional assumption that $X$ (and hence also $Y$) is a K$3$ surface.
\begin{thm}\label{thm: obstruction after field extension}
    Let $X$ be K$3$ surface whose special fibre $Y$ is ordinary. Then there exists a finite field extension $L'/L$ and an element $\A\in \br(X_{L'})[p]$ that obstructs weak approximation on $X_{L'}$.
\end{thm}
 Without loss of generality we can assume that $L$ contains a primitive $p$-root of unity. We fix, for every $q$, an isomorphism on $X_\et$ between $\Z/p\Z$ and $\Z/p\Z(q)$. Define 

\[
    {\mathcal{M}}_1^q:={i}^*\text{R}^q{j}_*\left(\Z/p\Z(1)\right)
\]
where ${i}$ and ${j}$ denote the inclusion of the generic and special fibre (${X}$ and ${Y}$, respectively) in ${\mathcal{X}}$ (cfr. Section \ref{Section: General Setting}). 
\begin{rmk}
    The main reference for this section is \cite{BlochKatoEtale}. In \cite{BlochKatoEtale} Bloch and Kato link a “twisted” version of the sheaves $\bar{\mathcal{M}}^q_1$ with the sheaf of logarithmic forms on $\bar{Y}$. In particular, they work with the sheaves $\bar{M}^1_q:=\bar{i}^* R^q \bar{j}_*\left(\Z/p\Z(q)\right)$. Since for every $q$ we fixed an isomorphism between $\Z/p\Z(1)$ and $\Z/p\Z(q)$, the same results apply to the sheaves $\bar{\mathcal{M}}^q_1$.
\end{rmk}
The sheaves $\bar{\mathcal{M}}^q_1$ build the spectral sequence of vanishing cycles \cite[0.2]{BlochKatoEtale}:
\[
    E^{s,t}_2:=\h^s({Y},{\mathcal{M}}^t_1)\Rightarrow \h^{s+t} \left({X},\Z/p\Z(1)\right).
\]
By comparing the spectral sequence of vanishing cycles for $X$ and for $K^h$ (which we recall to be defined as the fraction field of the henselisation of the discrete valuation ring $\Os_{\mathcal{X},Y}$) we get the following commutative diagram:
\[
\begin{tikzcd}
    \h^2\left(X, \Z/p\Z(1)\right) \arrow[r,"f"] \arrow[d] & \h^0(Y,\mathcal{M}^1_2) \arrow[d,"g"]\\
    \br(K^h)[p] \arrow[r,"\text{res}"] &\h^0(K^h,\br(K^h_{nr})[p]).
\end{tikzcd}
\]
In \cite[Lemma~3.4]{BrightNewton} it is proven that the map $g$ is injective. The map $f$ is defined as the composition of the projection $\h^2(X,\Z/p\Z(1))\twoheadrightarrow E^{0,2}_\infty$ with inclusion map $E^{0,2}_\infty \hookrightarrow E^{0,2}_2$. Furthermore, the injectivity of $g$ allows us to define a map $f_1\colon \br(X)[p]\rightarrow \h^0(Y,\mathcal{M}^1_2)$ such that the diagram 
\[
\begin{tikzcd}
    \h^2(X,\Z/p\Z(1)) \arrow[d]\arrow[r,"f"] &\h^0(Y,\mathcal{M}^1_2) \\
    \br(X)[p] \arrow[ur,"f_1"']
 \end{tikzcd}
\]
commutes, where the vertical arrow comes from the Kummer exact sequence \cite[Section~1.3.4]{BGgroupTheleneSkoro}. In fact, every element in $\h^2\left(X,\Z/p\Z(1)\right)$ coming from an element $\delta\in \Pic(X)$ is sent to $0$ by $f$, since its image in $\br(K^h)[p]$ comes from an element in $\Pic(K^h)$, which is trivial by Hilbert's theorem 90 \cite[Theorem~1.3.2]{BGgroupTheleneSkoro}.

By \cite[Proposition~6.1]{Kato} we also know that $\ker({\mathrm{res}})=\fil_0\br(K^h)[p]$, hence we have the diagram
\begin{equation}\label{diagram: vanishing cycles}
\begin{tikzcd}
    \fil_0\br(X)[p]\arrow[r]\arrow[d] &\br(X)[p] \arrow[r,"f_1"] \arrow[d] & \h^0(Y,\mathcal{M}^1_2) \arrow[d,"g"]\\
    \fil_0\br(K^h)[p] \arrow[r] &\br(K^h)[p] \arrow[r,"\text{res}"] &\h^0(K^h,\br(K^h_{nr})[p]).
\end{tikzcd}
\end{equation}
Note that we can build a diagram analogous to~\eqref{diagram: vanishing cycles} for every finite field extension $L'/L$.

We are left to prove the existence (over a field extension $L'$ of $L$) of an element $\A$ in $\br(X_{L'})[p]$ such that $f_{L'}(\A)\ne 0$. In order to do this, we analyse the spectral sequence of vanishing cycles over an algebraic closure $\bar{L}$ of $L$. Let $\Lambda$ the integral closure of $\Os_L$ in $\bar{L}$ and $\bar{\ell}$ the residue field of $\Lambda$. Let $\bar{X}$, $\bar{\mathcal{X}}$ and $\bar{Y}$ be the base change of $X,\mathcal{X}$ and $Y$ to $\bar{L},\Lambda$ and $\bar{\ell}$ respectively. We define $\bar{\mathcal{M}}_1^q:=\bar{i}^*\text{R}^q\bar{j}_*\left(\Z/p\Z(1)\right)$, where $\bar{i}$ and $\bar{j}$ denote the inclusion of the generic and special fibre ($\bar{X}$ and $\bar{Y}$, respectively) in $\bar{\mathcal{X}}$. These sheaves build a spectral sequence of vanishing cycles: 
\[
    \bar{E}^{s,t}_2:=\h^s(\bar{Y},\bar{\mathcal{M}}^t_1)\Rightarrow \h^{s+t} \left(\bar{X},\Z/p\Z(1)\right).
\]
We are interested in the map $\bar{f}_1\colon\br(\bar{X})[p]\rightarrow \bar{E}^{0,2}_\infty$, defined in the same way as the map $f_1$.

Combining \cite[Theorem~8.1]{BlochKatoEtale} with the short exact sequence~(8.0.1) and Proposition~7.3 of \cite{BlochKatoEtale} we get that if $Y$ is ordinary, then 
\begin{equation}\label{eq: cohomology of M^q_1 ordinary case}
    \h^n(\bar{Y},\bar{\mathcal{M}}^q_1)\simeq \h^n(\bar{Y},\Omega^q_{\bar{Y},\log}) \, \text{ and } \, \h^n(\bar{Y},\Omega^q_{\bar{Y},\log})\otimes_{\F_p} \bar{\ell}\xrightarrow{\sim}\h^n(\bar{Y},\Omega^q_{\bar{Y}}).
\end{equation}
for every $q,n$. 
From equation~\eqref{eq: cohomology of M^q_1 ordinary case} $\bar{E}_2^{2,1}=\h^2(\bar{Y},\bar{\mathcal{M}}_1^1)\simeq \h^2(\bar{Y},\Omega^1_{\bar{Y},\log})=0$, where the last equality follows from the fact that for K$3$ surfaces $\h^2(\bar{Y},\Omega^1_{\bar{Y}})=0$. Moreover, since for K$3$ surfaces $\h^3(\bar{Y},\Omega^0_{\bar{Y}})=0$, we get from equation~\eqref{eq: cohomology of M^q_1 ordinary case} that also $\bar{E}_2^{3,0}$ vanishes. Thus, \[
    \bar{E}_\infty^{0,2}=\ker(\bar{E}_2^{0,2}\rightarrow \bar{E}_2^{2,1})=\bar{E}_2^{0,2}
\]
    and hence (by construction) the map $\bar{f}_1\colon \br(\bar{X})[p]\rightarrow \h^0(\bar{Y},\bar{\mathcal{M}}_1^2)$ is surjective. Finally, using again equation~\eqref{eq: cohomology of M^q_1 ordinary case} we get that $\h^0(\bar{Y},\bar{\mathcal{M}}_1^2)$ is not trivial. 

\begin{proof}[Proof of Theorem~\ref{thm: obstruction after field extension}]               Let $\A\in \br(\bar{X})[p]$ be such that $\bar{f}_1(\A)\ne 0$, then there is a finite field extension $L'/L$  such that $\A$ is defined over $L'$ (i.e. $\A\in \br(X_{L'})[p]$) and $f_{1,L'}(\A)\ne 0$. Namely, by what we said above $\A\notin \fil_0 \br(X_{L'})[p]$.
\end{proof}
Let $X$ be a K$3$ surface over a number field     $k$ that contains a primitive $p$-root of unity $\zeta$ and let $\ip$ be a prime of good ordinary reduction for $X$ of residue characteristic $p$. Then, since $\br(\bar{X})[p]\simeq \br(\bar{X_{\ip}})[p]$ there exists $\A\in \br(\bar{X})$ such that $\bar{f}_1(\A)=0$. Hence, if $k'/k$ is a finite field extension such that $\A$ is defined over $k'$, then we know from what we have showed above that there is a prime $\ip'$ above $\ip$ such that $\ev_\A$ is non-constant on $X_{k'}(k'_{\ip'})$.
\section{Ordinary case: examples}\label{section: ordinary case examples}
In this section we give examples of Brauer--Manin obstructions on K$3$ surfaces, defined over a number fields, that comes from primes of good ordinary reduction. Moreover, we prove that if $X$ is a Kummer K$3$ surface coming from a product of elliptic curves defined over $\Q$ with good ordinary reduction at the prime $2$ and full $2$-torsion defined over $\Q_2$, then $\br(X)[2]=\Ev_{-1}\br(X)[2]$ (cfr. Theorem~\ref{thm: Kummer}). This theorem proves what whose already claimed by Ieronymous after some computational evidences, see \cite[Remark~2.6]{ieronymou2023odd}.

\subsection{Kummer K3 surfaces over 2-adic fields}
Let $L$ be a $2$-adic field. The recent papers \cite{LazdaSkoro}, \cite{matsumoto2023supersingular} allow us to know whether the Kummer K$3$ surface attached to an abelian variety $A/L$ with good reduction is still a K$3$ surface with good reduction (this was already known for K$3$ surfaces over $p$-adic fields with $p\ne 2$). In \cite{SkoroZarhin} Skorobogatov and Zarhin link the transcendental part of the Brauer group of a Kummer K$3$ surface to the one of the corresponding abelian variety (cft. Section~\ref{subsubsection: Kummer K3 generalities}). All these results open up the possibility of building examples of K$3$ surfaces with good reduction at the prime $2$ and of which we are able to study the Brauer group.
In this section, we are going to show that for every pair of elliptic curves $E_1,E_2$ over $\Q$ with good ordinary reduction at $p=2$ and full $2$-torsion defined over $\Q_2$, the $2$-torsion elements in the Brauer group of the corresponding Kummer K$3$ surface $X$ do not play a role in the Brauer--Manin obstruction to weak approximation. In particular, this shows that the field extension in Theorem \ref{thm: obstruction after field extension} is needed. We will then use these computations to exhibit an example of a K$3$ surface over $\Q_2$ with good ordinary reduction and such that $\br(X)=\Ev_{-1}\br(X)$, showing that the inverse of Theorem \ref{thm: ordinary good reduction} does not hold in general.    
\subsubsection{Kummer K3 surfaces and their Brauer group: generalities}\label{subsubsection: Kummer K3 generalities}
Let $A$ be an abelian surface over a field $k$ of characteristic different from $2$ and $X=\mathrm{Kum}(A)$ the corresponding Kummer surface, Skorobogatov and Zarhin \cite{SkoroZarhin} prove that there is a well-defined map
\[
\pi^*\colon \br(X)\rightarrow \br(A)
\]
that induces an injection of $\br(X)/\br_1(X)$ into $\br(A)/\br_1(A)$. They also prove that this injection is an isomorphism on the $p$-torsion for all odd primes, see \cite[Theorem 2.4]{SkoroZarhin}. 
We say that an element $\A\in \br(A)$ \emph{descends} to $\br(X)$ if there exists $\mathcal{C}\in \br(X)$ such that $\pi^*(\mathcal{C})=\A$. 
\begin{lemma}\label{lemma: pi^*(Ev_-1br(X)) is in Ev_-1br(A)}
    Let $X=\mathrm{Kum}(A)$, $\mathcal{C}\in \br(X)$ and $\mathcal{B}:=\pi^*(\mathcal{C})\in \br(A)$; if $\mathcal{C}$ lies in $ \Ev_{-1}\br(X)$ then $\mathcal{B}$ lies in $ \Ev_{-1}\br(A)$.
\end{lemma}
\begin{proof}
    The result follows from the fact that any finite field extension $M/L$ and $P\in A(M)$ we have 
    \[
        \ev_{\mathcal{B}}(P)=\ev_{\pi^*(\mathcal{C})}(P)=\ev_{\mathcal{C}}(\pi(P)).
    \]
\end{proof}
Moreover, Skorobogatov and Zarhin \cite{SkoroZarhin} show that given two elliptic curves $E_1$ and $E_2$ with Weierstrass equations
\[
E_1: \, v_1^2=u_1\cdot (u_1-\gamma_{1,1})\cdot(u_1-\gamma_{1,2}), \quad E_2: \,v_2^2=u_2\cdot (u_2-\gamma_{2,1}) \cdot (u_2-\gamma_{2,2})
\]
the quotient $\br(E_1\times E_2)[2]/\br_1 (E_1\times E_2)[2]$ is generated by the classes of the four Azumaya algebras 
\[
    \mathcal{A}_{\epsilon_1,\epsilon_2}=((u_1-\epsilon_1)(u_1-\gamma_{1,2}),(u_2-\epsilon_2)(u_2-\gamma_{2,2})) \; 
    \text{ with } \; \epsilon_i\in \{0,\gamma_{i,1}\}. 
\]
Finally, if $M$ is the matrix
    \[
    M=
    \begin{pmatrix}
    1 &\gamma_{1,1}\cdot \gamma_{1,2} &\gamma_{2,1}\cdot \gamma_{2,2} &-\gamma_{1,1}\cdot \gamma_{2,1}\\
    \gamma_{1,1}\cdot \gamma_{1,2} &1 &\gamma_{1,1}\cdot \gamma_{2,1} &\gamma_{2,1}\cdot (\gamma_{2,1}-\gamma_{2,2}) \\
    \gamma_{2,1}\cdot \gamma_{2,2} &\gamma_{1,1}\cdot \gamma_{2,1} &1 &\gamma_{1,1}\cdot(\gamma_{1,1}-\gamma_{1,2})\\
    -\gamma_{1,1}\cdot \gamma_{2,1} &\gamma_{2,1}\cdot (\gamma_{2,1}-\gamma_{2,2}) &\gamma_{1,1}\cdot(\gamma_{1,1}-\gamma_{1,2}) &1
    \end{pmatrix}
    \]
    then by \cite[Lemma 3.6]{SkoroZarhin}:
    \begin{enumerate}
        \item $\A_{\gamma_{1,1},\gamma_{2,1}}$ descends to $\br(X)$ if and only if the entries of the first row of $M$ are all squares;
        \item $\A_{\gamma_{1,1},0}$ descends to $\br(X)$ if and only if the entries of the second row of $M$ are all squares;
        \item $\A_{0,\gamma_{2,1}}$ descends to $\br(X)$ if and only if the entries of the third row of $M$ are all squares;
        \item $\A_{0,0}$ descends to $\br(X)$ if and only if the entries of the last row of $M$ are all squares.
    \end{enumerate}


\subsubsection{Product of elliptic curves with good reduction at $2$ and full $2$-torsion}
In order to use the results summarised in the previous section we need to analyse how the $2$-torsion points of an elliptic curve with good ordinary reduction at $2$ look like. Let $E/\Q$ be the elliptic curve defined by the minimal Weierstrass equation 
\begin{equation}\label{eqOrdinaryEllipticCurve}
    y^2+xy+\delta y= x^3+ax^2+bx+c
\end{equation}
with $\delta \in \{0,1\}$ and $a,b,c\in \Z$ such that $E$ has good reduction at $2$. Assume furthermore that the $2$-torsion of $E$ is defined over $\Q_2$, i.e. $E(\Q_2)[2]=E(\bar{\Q}_2)[2]$. Let $\alpha_i,\beta_i\in \Q_2$ be such that $E(\Q_2)[2]=\{\mathcal{O},(\alpha_1,\beta_1),(\alpha_2,\beta_2),(\alpha_3,\beta_3)\}$, with $\mathcal{O}$ point at infinity of $E$. 
\begin{lemma}\label{lemma:2adicValOf2TorsionPoint}
    Assume that $\beta_1,\beta_2,\beta_3$ are ordered as
    \[
    \mathrm{ord}_2(\beta_1)\leq \mathrm{ord}_2(\beta_2) \leq \mathrm{ord}_2(\beta_3).
    \]
    Then $\mathrm{ord}_2(\alpha_1)=-2$ and $\alpha_2,\alpha_3\in \Z_2$. 
\end{lemma}
\begin{proof}
    The $2$-torsion points on $E$ can be computed through the $2$-division polynomial of $E$, which is $\psi_2(x,y)=2y+x+\delta$.
In particular, $\alpha_i=-2\beta_i-\delta$ with $\beta_i$ solution of 
\begin{equation*}
    \Phi(y):=y^2+(-2y-\delta)y+\delta y -((-2y-\delta)^3+a(-2y-\delta)^2+b(-2y-\delta)+c).
\end{equation*}
The polynomial $\Phi(y)$ can be rewritten as 
%\[
%y^2-2y^2-\delta y+\delta y + 8y^3+12\delta y^2+6\delta^2 y-4ay^2-4a\delta y-a\delta^2 +2by+b\delta -c
%\]
\begin{equation}
    \Phi(y)=8y^3-(1-12\delta+4a)y^2-(-6\delta^2+4a\delta -2b)y+\delta^3-a\delta^2+b\delta-c.
\end{equation}
 Looking at the coefficients of $\Phi(y)$ we get that 
\[
\begin{cases}
    \mathrm{ord}_2(\beta_1+\beta_2+\beta_3)=\mathrm{ord}_2 (1-12\delta +4a)-\mathrm{ord}_2(8)\\
    \mathrm{ord}_2(\beta_1\beta_2+\beta_1\beta_3+\beta_2\beta_3)=\mathrm{ord}_2(-6\delta^2+4a\delta -2b)-\mathrm{ord}_2(8)\\
    \mathrm{ord}_2(\beta_1\beta_2\beta_3)=\mathrm{ord}_2(\delta^3-a\delta^2+b\delta-c)-\mathrm{ord}_2(8).
\end{cases}
\]
From the first equation, we get 
$\mathrm{ord}_2(\beta_1)\leq -3$
that combined with \cite[Theorem~VIII.7.1]{Silverman} tells us that $\mathrm{ord}_2(\beta_1)=-3$. From the third equation, we get 
$\mathrm{ord}_2(\beta_2)+\mathrm{ord}_2(\beta_3)\geq 0.$
Hence, if $\mathrm{ord}_2(\beta_2)=\mathrm{ord}_2(\beta_3)$ then $\beta_2$ and $\beta_3$ have both non-negative $2$-adic valuation; otherwise, if $\mathrm{ord}_2(\beta_2)<\mathrm{ord}_2(\beta_3)$ then, from the second equation, we get
$\mathrm{ord}_2(\beta_2)\geq 1$
which implies again that both $\beta_2$ and $\beta_3$ have non-negative $2$-adic valuation. The result now follows from the fact that $\alpha_i=-2\beta_i-\delta$, with $\delta\in \{0,1\}$. 
\end{proof}



\begin{lemma}\label{lemma:changeOfVarElliptic}
    The change of variables given by 
\begin{equation}\label{eqChangeVar}
    \begin{cases}
        u=4x-4\alpha_1\\
        v=4(2y+x+\delta)
    \end{cases}
\end{equation}
induces an isomorphism between $E$ and the elliptic curve given by the equation
    \begin{equation}
        v^2=u(u-\gamma_1)(u-\gamma_2)
    \end{equation}
    where $\gamma_1=4\cdot (\alpha_2-\alpha_1)$ and $\gamma_2=4\cdot(\alpha_3-\alpha_1)$.
\end{lemma}
\begin{proof}
    The change of variables 
    \begin{equation*}
        \begin{cases}
            u_1=4x\\
            v_1=4(2y+x+\delta)
        \end{cases}
    \end{equation*}
    sends the elliptic curve given by the equation 
    \begin{equation}\label{eqIntermediate}
        v_1^2=u_1^3 + (4a+1) u_1^2 +(16b+8\delta) u_1 + 16 c+16\delta^2
    \end{equation}
    to the elliptic curve given by equation~\eqref{eqOrdinaryEllipticCurve}.
    %In fact, 
    %\[
    %    [4(2y+x+\delta)]^2=(4x)^3 + (4a+1) (4x)^2 + (16b+8\delta) (4x) + 16 c+16\delta^2=0
    %\]
    %namely, 
    %\[
        %4y^2+x^2+\delta^2+4xy+4y\delta+2x\delta=4x^3+4ax^2+x^2+4bx+2\delta x +16 c-\delta^2
    %\]
    %therefore
    %\[
    %y^2+xy+\delta y=x^3+ax^2+bx+c.
    %\]
    Moreover, the $2$-division polynomial of $E$ is given by $2y+x+\delta$. Hence the non-trivial $2$-torsion points on $E$ are sent to non-trivial $2$-torsion points on the elliptic curve given by equation~\eqref{eqIntermediate}. It is therefore enough to consider the extra translation $u=u_1-4\alpha_1$ and $v=v_1$ to get the desired equation. 
\end{proof}
Let $E_1$ and $E_2$ be two elliptic curves with equations of the form~\eqref{eqOrdinaryEllipticCurve}. We denote by $(\delta_i,a_i,b_i,c_i)$ the parameters that determine the equation attached to $E_i$, by $(\alpha_{j,i},\beta_{j,i})$, $j\in \{1,2,3\}$ the non-trivial $2$-torsion points of $E_i$ and by $A$ the abelian surface given by the product of $E_1$ with $E_2$. We denote by $\langle \Ev_{-1}\br(A)[2], \br_1(A)[2] \rangle$ the subgroup of $\br(A)[2]$ generated by $\Ev_{-1}\br(A)[2]$ and $\br_1(A)[2]$, where $\br_1(A)[2]$ is the algebraic Brauer group of $A$.

\begin{lemma}\label{lemma: A_epsilon_1,epsilon_2 in Ev_-1}
    Assume that $\epsilon_1$ and $\epsilon_2$ are as in Section \ref{subsubsection: Kummer K3 generalities}; then the class of the quaternion algebra $\mathcal{A}_{\epsilon_1,\epsilon_2}$ lies in $\langle \Ev_{-1}\br(A)[2], \br_1(A)[2] \rangle$ if and only if at least one among $\epsilon_1$ and $\epsilon_2$ is different from $0$.
\end{lemma}
\begin{proof}
    We fix $\pi=2$ as a uniformiser and $\xi=-1$ as a primitive $2$-root of unity. We start by assuming that at least one among $\epsilon_1$ and $\epsilon_2$ is different from $0$. By the symmetry of the statement, we can assume without loss of generality that $\epsilon_1\ne 0$. Then 
    \[
    \A_{\epsilon_1,\epsilon_2}=((u_1-\gamma_{1,1})\cdot (u_1-\gamma_{1,2}) \, , \,(u_2-\gamma_{2,1})\cdot (u_2-\gamma_{2,2}) )=(u_1,f_{\epsilon_2}(u_2))
    \]
    where $f_{\epsilon_2}(u_2)=(u_2-\epsilon_2)\cdot (u_2-\gamma_{2,2})$.
    The quaternion algebra $\A_{\epsilon_1,\epsilon_2}$ corresponds via the change of variables of Lemma \ref{lemma:changeOfVarElliptic} to 
    \[
    \mathcal{A}_{\epsilon_1,\epsilon_2}=(4\cdot(x_1-\alpha_{1,1}) \, , \, f_{\epsilon_2}(4x_2-4\alpha_{2,1}))=(x_1-\alpha_{1,1} \, ,\,f_{\epsilon_2}(4x_2-4\alpha_{2,1})).
    \]
    We define 
    \[
    g_{\epsilon_2}(x_2):=\begin{cases}
        (x_2-\alpha_{2,2})\cdot (x_2-\alpha_{2,3}) \text{ if }\epsilon_2=\gamma_{2,1};\\
        (4x_2-\alpha_{2,1})\cdot (x_2-\alpha_{2,3}) \text{ if }\epsilon_2=0.
    \end{cases}
    \]
    Then, $4\cdot g_0(x_2)=f_0(4x_2-4\alpha_{2,1})$ and $16 \cdot g_{\gamma_{2,1}}(x_2)=f_{\gamma_{2,1}}(4x_2-4\alpha_{2,1})$. Thus we can rewrite $\A_{\epsilon_1,\epsilon_2}$ as
    \[
        (-(\alpha_{1,1})^{-1} \, ,\, g_{\epsilon_2}(x_2))\otimes (1+\alpha_{1,1}^{-1}\cdot x_1\, ,\, g_{\epsilon_2}(x_2)).
    \]
  Since $(-(\alpha_{1,1})^{-1}\, ,\, g_{\epsilon_2}(x_2))$ lies in $\br_1(A)[2]$, we are left to show that the class of the quaternion algebra $(1+\alpha_{1,1}^{-1}\cdot x_1\, ,\, g_{\epsilon_2}(x_2))$ lies in $ \Ev_{-1}\br(A)[2]$. By Lemma \ref{lemma:2adicValOf2TorsionPoint} we know that $\mathrm{ord}_2(\alpha_{1,1}^{-1})=2$ and therefore by Proposition \ref{prop: explicit description of fil_{e'-m}}
  \[
    (1+\alpha_{1,1}^{-1}\cdot  x_1 \, , \, g_{\epsilon_2}(x_2))\in \fil_{0}\br(A)[2].
  \]
  By \cite[Theorem~C]{BrightNewton} in order to establish if  $(1+\alpha_{1,1}^{-1}\cdot  x_1 \, , \, g_{\epsilon_2}(x_2))$ belongs to $\Ev_{-1}\br(A)[2]$ we need to compute $\partial(1+\alpha_{1,1}^{-1}\cdot  x_1 \, , \, g_{\epsilon_2}(x_2))$. We have that $g_{\epsilon_2}(x_2)\not \equiv 0 \mod 2$ and again by combining the isomorphism of Proposition \ref{prop: explicit description of fil_{e'-m}} and Proposition \ref{propGradedPieces}$(a)$ we get  
  \[
  \lambda_\pi \left(\bar{g}_{\epsilon_2}(\bar{x}_2)\cdot \frac{d\bar{x_1}}{\bar{x_1}},0\right)=\left(1+\alpha_{1,1}^{-1} \cdot x_1 \, , \, g_{\epsilon_2}(x_2)\right )
  \]
  since $1+\alpha_{1,1}^{-1}\cdot x_1=1+4 \cdot (s^{-1} \cdot x_1)$ with $s=4\cdot \alpha_{1,1}\in \Z_2^\times$ and hence $s^{-1}\cdot x_1$ is a lift to characteristic $0$ of $\bar{x}_1$. 
  Therefore, by definition of the residue map $\partial$, we get
  \[
  \partial((1+\alpha_{1,1}^{-1} \cdot x_1 \, ,\,g_{\epsilon_2}(x_2)))=0
  \]
    which by Theorem \ref{thm: evaluation fil and refined Swan conductor} implies that $(1+\alpha_{1,1}^{-1} \cdot x_1 \, , \, g_{\epsilon_2}(x_2))\in \Ev_{-2}\br(A)[2]\subseteq \Ev_{-1}\br(A)[2]$.
    In order to end the proof we are left to show that $\A_{0,0}\notin \langle \Ev_{-1}\br(A)[2],\br_1(A)[2]\rangle$.
    The change of variables of Lemma \ref{lemma:changeOfVarElliptic} sends the class of the quaternion algebra 
    \[
        \A_{0,0}=(u_1\cdot (u_1-\gamma_{1,2})\, , \, u_2\cdot(u_2-\gamma_{2,2}))=(u_1-\gamma_{1,1} \, ,\, u_2-\gamma_{2,1})
    \]
    to the class of the quaternion algebra
    \[
    (4\cdot(x_1-\alpha_{1,2}) \, ,\, 4\cdot(x_2-\alpha_{2,2}))=(x_1+2\beta_{1,2}+\delta \, , \, x_2+2\beta_{2,2}+\delta).
    \]
    From Proposition~\ref{propGradedPieces}(d) the latter is such that  
    \[
    \rho_0\left(\frac{d(\bar{x}_1+\delta)}{\bar{x}_1+\delta}\wedge \frac{d(\bar{x}_2+\delta)}{\bar{x}_2+\delta}\right)=\left[\left\{x_1+2\beta_{2,1}+\delta,x_2+2\beta_{2,2}+\delta\right\}\right]\in \gr^0.
    \]
    In fact, $x_1+2\beta_{2,1}+\delta,x_2+2\beta_{2,2}+\delta$ and $x_2+2\beta_{2,2}+\delta$ are lifts to characteristic $0$ of $\bar{x}_1+\delta$ and $\bar{x}_2+\delta$ respectively. Note that, $\frac{d(\bar{x}_1+\delta)}{\bar{x}_1+\delta}\wedge \frac{d(\bar{x}_2+\delta)}{\bar{x}_2+\delta}$ comes from a global $2$-form on the special fibre $Y$ of $A$ and hence it is non-zero in its function field. Finally, using Lemma~\ref{lemma: rsw and map rho_m} we get that 
    \[
        \rsw_{2,\pi}((x_1+2\beta_{2,1}+\delta,x_2+2\beta_{2,2}+\delta))=\left(\frac{d(\bar{x}_1+\delta)}{\bar{x}_1+\delta}\wedge \frac{d(\bar{x}_2+\delta)}{\bar{x}_2+\delta},0\right)\ne(0,0)
    \]
    and hence $(x_1+2\beta_{1,2}+\delta , x_2+2\beta_{2,2}+\delta)\notin \fil_1\br(A)[2]\supseteq \Ev_{-1}\br(A)[2]$. Moreover, as a consequence of Corollary~\ref{cor: refined Swan conductor and algebraic elements} we get that $\A\notin \langle \Ev_{-1}\br(A)[2],\br_1(A)[2]\rangle$. In fact, otherwise, there would be an element $\A_1\in \Ev_{-1}\br(A)[2]$ such that $\A\otimes \A_1 \in \br_1(A)[2]$, but $\A\otimes \A_1$ has the same refined Swan conductor as $\A$.
\end{proof}
%\begin{rmk}    Note that, $\frac{d(\bar{x}_1+\delta)}{\bar{x}_1+\delta}\wedge \frac{d(\bar{x}_2+\delta)}{\bar{x}_2+\delta}$ comes from a global $2$-form on the special fibre $Y$ of $A$; in fact, it is the (local) expression of a global $2$-logarithmic form on the special fibre $Y$ of $A$.\end{rmk}
\subsubsection{No Brauer--Manin obstruction from 2-torsion elements in Kum(A)}
In this section, we show how, from the results of the previous section, we can deduce information on the $2$-torsion elements in the Brauer group of the corresponding Kummer surface $X=\mathrm{Kum}(A)$. With abuse of notation, we denote by $A$ and $X$ also the base change of the abelian surface $A$ and the corresponding Kummer surface $X$ to $\Q_2$. By Section \ref{subsubsection: Kummer K3 generalities} we know that $\A_{0,0}$ descends to $X$ if and only if 
\[
 \left[ -\gamma_{1,1}\cdot \gamma_{2,1}, \gamma_{2,1}(\gamma_{2,1}-\gamma_{2,2}), \gamma_{1,1} (\gamma_{1,1}-\gamma_{1,2}),1\right] \in (\Q_2^{\times 2})^4.
\]
By construction, $\gamma_{1,1}=4\cdot(\alpha_{1,2}-\alpha_{1,1})=8\beta_{1,2}-8\beta_{1,1}$ and therefore
\[
\gamma_{1,1}\equiv -8\beta_{1,1} \equiv -1+4\delta_1-4a_1 \equiv -1+4(\delta_1 -a_1) \mod 8
\]
since from Lemma \ref{lemma:2adicValOf2TorsionPoint} $\mathrm{ord}_2(8 \beta_{1,2})>0$.
Similarly, 
\[
\gamma_{2,1}\equiv -8\beta_{2,1} \equiv -1+4\delta_2-4a \equiv -1+4(\delta_2 -a_2) \mod 8.
\]
In particular, both $\gamma_{1,1}$ and $\gamma_{2,1}$ are either $-1$ or $3$ modulo $8$; hence $-\gamma_{1,1}\cdot \gamma_{2,1}$ is either $-1$ or $3$ and therefore it is never a square. We are ready to prove the main theorem of this section.
\begin{thm}\label{thm: Kummer}
    Let $X=\mathrm{Kum}(A)$, where $A=E_1\times E_2$ is as in Section \ref{subsubsection: Kummer K3 generalities}; then $\br(X)[2]=\Ev_{-1}\br(X)[2]$.
\end{thm}
\begin{proof}
Let $\epsilon_1,\epsilon_2$ and $\A_{\epsilon_1,\epsilon_2}\in \br(A)[2]$ be as in the previous section. If $\A_{\epsilon_1,\epsilon_2}$ descends to $\br(X)[2]$, we denote by $\mathcal{C}_{\epsilon_1,\epsilon_2}$ the corresponding element in $\br(X)[2]$, i.e. $\mathcal{C}_{\epsilon_1,\epsilon_2}$ is such that $\pi^*(\mathcal{C}_{\epsilon_1,\epsilon_2})=\A_{\epsilon_1,\epsilon_2}$. 
   
Let $L$ be a field extension of $\Q_2$ such that all elements appearing in the matrix $M$ of Section~\ref{subsubsection: Kummer K3 generalities} are squares, i.e. the injective map 
\[
    \pi^*\colon \frac{\br(X)[2]}{\br_1(X)[2]}\hookrightarrow \frac{\br(A)[2]}{\br_1(A)[2]}
\]
is an isomorphism. With abuse of notation, we denote by $\mathrm{res}$ both the restriction map from $\br(A)$ to $\br(A_L)$ and the one from $\br(X)$ to $\br(X_L)$. We denote by $(\mathcal{C}_{\epsilon_1,\epsilon_2})_L$ the pre-image of $\mathrm{res}(\A_{\epsilon_1,\epsilon_2})\in \br(A_L)[2]$.

     From \cite[Theorem~3]{LazdaSkoro} we know that $X$ is an ordinary K$3$ surface, hence by Corollary \ref{cor: on the image of the refined swan conductor}(c) for every field extension $L$ of $\Q_2$ with ramification index $e=e(L/\Q_2)$ and uniformiser $\pi_L$ we have an injection 
    \[
        \mathrm{mult}_{\overline{c}}\circ \rsw_{e',\pi_L}:\frac{\br(X_L)[2]}{\Ev_{-1}\br(X_L)[2]} \hookrightarrow \h^0(Y_\ell,\Omega^2_{Y_\ell,\log})
    \]
    where $e'=2e$. Let $\ell$ be the residue field of $L$, as $X$ (and hence all its base change) has good ordinary reduction, we know that $\h^0(\bar{Y},\Omega^2_{\bar{Y},\log})\otimes_{\F_2}\bar{\ell}$ is a one dimensional $\bar{\ell}$-vector space (cfr.   equation~\eqref{eq: cohomology of M^q_1 ordinary case}) and hence $\br(X_L)[2]/\Ev_{-1}\br(X_L)[2]$ is an $\F_2$-vector space of dimension at most $1$.
    
     From Lemma \ref{lemma: A_epsilon_1,epsilon_2 in Ev_-1} we know that $\A_{0,0}\notin \fil_1\br(A)[2]$ and by Lemma~\ref{lemma: base change L'/L} we have that $\mathrm{res}(\A_{0,0})\notin \fil_{e}\br(A_L)[2]$. Hence by Lemma \ref{lemma: pi^*(Ev_-1br(X)) is in Ev_-1br(A)} $(\mathcal{C}_{0,0})_L\notin \Ev_{-1}\br(X_L)[2]$ and therefore \[
     \langle [(\mathcal{C}_{0,0})_L] \rangle =\frac{\br(X_L)[2]}{\Ev_{-1}\br(X_L)[2]}.
     \] 

Assume now that there exists $(\epsilon_1,\epsilon_2)\ne (0,0)$ such that $\A_{\epsilon_1,\epsilon_2}$ descends to $\br(X)$ and the corresponding element $\mathcal{C}_{\epsilon_1,\epsilon_2}\notin \Ev_{-1}\br(X)[2]$; then again by Lemma \ref{lemma: base change L'/L} $\mathrm{res}(\mathcal{C}_{\epsilon_1,\epsilon_2})$ in $\br(X_L)$ does not lie in $\Ev_{-1}\br(X_L)[2]$ and therefore, $\mathrm{res}(\mathcal{C}_{\epsilon_1,\epsilon_2})\otimes (\mathcal{C}_{0,0})_L\in \Ev_{-1}\br(X)[2]$, which implies that $\mathrm{res}(\A_{\epsilon_1,\epsilon_2}\otimes \A_{0,0})\in \Ev_{-1}\br(A_L)[2]$ that is a contradiction.
\end{proof}
Finally, we give an example of a K$3$ surface over $\Q$ with good ordinary reduction at $2$ and such that $\br(X)=\Ev_{-1}\br(X)$. The existence of such an example shows that the inverse of Theorem \ref{thm: ordinary good reduction} is not true, i.e. it is not enough to have that $p-1\mid e$ in order to find an element in $\br(X)$ that does not lie in $\Ev_{-1}\br(X)$.
\begin{example}
Let $A=E\times E$, where $E$ is the elliptic curve given by the minimal Weierstrass equation 
\[
y^2+xy+y=x^3-7\cdot x+5.
\]
Then, with the same notation as in the previous sections $\beta_1=-11/8$, $\beta_2=-1$ and $\beta_3=1$. Hence 
\[
\alpha_1=7/4, \quad \alpha_2=1, \quad \alpha_3=-3 \quad \text{ and } \quad \gamma_{1}=-3, \quad \gamma_2=-21.
\]
The matrix $M$ is of the form
\[
    \begin{pmatrix}
    1 &3\cdot 21 &3\cdot 21 &-9\\
    3\cdot 21 &1 &9 &-3\cdot 18 \\
    3\cdot 21 &9 &1 &-3\cdot 18\\
    -9 &-3\cdot 18 &-3\cdot 18 &1
    \end{pmatrix}.
    \]
    In particular, all the rows of $M$ have at least one term which does not lie in $\Q_2^{\times 2}$. Moreover, using \cite[Proposition~3.7]{SkoroZarhin} we can compute the dimension as an $\F_2$-vector space of the quotient of $\br(X)[2]$ by $\br(\Q_2)[2]$ and in this case particular example:
    \[
        \mathrm{dim}_{\F_2}\left(\frac{\br(X)[2]}{\br(\Q_2)[2]}\right)=0.
    \]
    We want to show that $\br(X)\{2\}=\br(\Q_2)\{2\}$. We work by induction on $n$; let $\A$ be in $\br(X)[2^n]$, then 
    \[
        \A^{\otimes 2^{n-1}}\in \br(X)[2]= \br(\Q_2)[2].
    \]
    In particular, given $P\in X(\Q_2)$, we have that
    \[
    (\A \otimes \ev_{\A}(P))^{\otimes 2^{n-1}}=\A^{\otimes 2^{n-1}}\otimes \ev_{\A^{\otimes 2^{n-1}}}(P)=0
    \]
    hence $\A\otimes \ev_{\A}(P)\in \br(X)[2^{n-1}]$, and by induction hypothesis that $\A\in \br(\Q_2)[2^n]$.
\end{example}
\subsubsection{Examples of Brauer--Manin obstruction}
In this section, we give examples that show that in Theorem~\ref{thm: ordinary good reduction} the assumption that the absolute ramification index $e$ is not divisible by $(p-1)$ is needed. We start with recalling the following example, which is the central result of \cite{Pagano}.
\begin{example}[\cite{Pagano}]\label{example: BM obstruction over Q_2}
Let ${X}\subseteq \p^3_{\Q}$ be the projective K$3$ surface defined by the equation 
\begin{equation}\label{eq}
    x^3y+y^3z+z^3w+w^3x+xyzw=0.
\end{equation}
Then $X$ has good ordinary reduction at $2$ and
the class of the quaternion algebra 
\[
    \A=\left(\frac{z^3+w^2x+xyz}{x^3},-\frac{z}{x}\right)\in \br \Q(X)
\]
defines an element in $\br(X)$. The evaluation map $\ev_\A \colon {X}(\Q_2)\rightarrow \br(\Q_2)$ is non-constant, and therefore gives an obstruction to weak approximation on $X$. This example is explained in detail in \cite{Pagano}. In particular, this example is the first example showing that a prime of good (ordinary) reduction can play a role in the Brauer--Manin obstruction to weak approximation. In \cite{Pagano} it is explained how the element $\A$ is related to the existence of a global logarithmic $2$-form on $Y$. More precisely, $\h^0(Y,\Omega^2_{Y,\log})$ is a $1$-dimensional $\F_2$-vector space and the non-zero global logarithmic $2$-form $\omega$ can be written (locally) as: 
\[
    \omega=\left(\frac{z^3+w^2x+xyz}{x^3}\right)\wedge\left(\frac{z}{x}\right).
\]
If we denote by $f$ and $g$ the functions $\frac{z^3+w^2x+xyz}{x^3}$ and $\frac{z}{x}$ seen as element in the function field of $Y,\, F=\F_2(Y)$, then we see that the two functions appearing in the definition of $\A$ are lifts to characteristic $0$ of $f$ and $g$, and hence from Proposition~\ref{propGradedPieces}
\[
\rho_0(\omega,0)=\left[\left\{ \frac{z^3+w^2x+xyz}{x^3},-\frac{z}{x}\right\}\right]\in \gr^0.
\]
Using Lemma~\ref{lemma: rsw and map rho_m}, $\rsw_{2,\pi}(\A)=(\omega,0)\ne (0,0)$ and $\A\notin \fil_1\br(X)[2]\supseteq \Ev_{-1}\br(X)[2]$. Note that, by Remark~\ref{rmk: K3 with good reduction global 2 form } we already know that since the K$3$ surface has good ordinary reduction at $2$, the only way for the prime $2$ to play a role in the Brauer--Manin obstruction to weak approximation via an $2$-torsion element $\A\in \br(X)[2]$ is if $\A$ comes from a logarithmic $2$-form through $\rho_0$.
\end{example}
We continue this section by giving new examples of this phenomenon for $p=3$ and $p=5$.
\begin{example}
    Let $L=\Q_3(\zeta)$ with $\zeta$ primitive $3$-root of unity. Let $\pi$ be a uniformiser of $\Os_L$; then $e=e(L/\Q_3)=2$ and the residue field $\ell$ is equal to $\F_3$.

    We define $X$ to be the Kummer K$3$ surface over $L$ attached to the abelian surface $A=E\times E$, with $E$ the elliptic curve over $L$ defined by the Weierstrass equation
    \[
        y^2=x^3+4\cdot x^2+3\cdot x+1.
    \]
    The elliptic curve $E$ (and hence $A$ and $X$) has good ordinary reduction at the prime $\ip=(\pi)$. We will denote by $\{x,y,z\}$ and $\{u,v,w\}$ the variables corresponding to the embedding of respectively the first and the second copy of $E$ in $\p^2_L$. We define the cyclic algebra
    \[
    \A:=\left(\frac{v-u}{w},\frac{y-x}{z}\right)_{\zeta} \in \br(L(A))[3].
    \]
    \textbf{Claim 1:} $\A$ belongs to $\br(A)[3]$.  
    \begin{proof}
        First of all, notice that from $y^2z=x^3+4x^2z+3xz^2+z^3$ we get
    \[
        z(y-x)(y+x)=(x+z)^3 \quad \text{and} \quad z(y^2-4x^2-3xz-z^2)=x^3.
    \]
    Then:
    \begin{itemize}
        \item[-] if $z=0$, then $x=0$ and therefore $y^2-4x^2-3xz-z^2\ne 0$ and $x\ne y$ and from the equation above $\A$ is equivalent to 
        \[
        \left(\frac{v-u}{v^2-4u^2-3uw-w^2},\frac{y-x}{y^2-4x^2-3xz-z^2}\right)_{\zeta};
        \]
        \item[-] if $x=y$, then $x\ne -y$ and $z\ne 0$ (since $z=0$ implies $x\ne y$) and from the equation above $\A$ is equivalent to 
        \[
            \left(\frac{w}{v+u},\frac{z}{x+y}\right)_{\zeta}.
        \]
    \end{itemize}  
    Thus, we see that along all the divisors over which $\A$ is not well defined we are able to find an equivalent Azumaya algebra which on those divisors is well defined. Hence, $\A$ defines an element in $\br(A)[3]$.
    \end{proof}
    \noindent \textbf{Claim 2:} $\A$ and does not lie in $\Ev_{-1}\br(A)[3]$.
    \begin{proof}
        The regular global $1$-form on the reduction of $E$ modulo $\ip$ is given by the (local) formula
    \[
        \frac{dx}{2y}=-\frac{1}{2}\cdot \frac{dx\cdot(x-y)}{y(y-x)}=-\frac{1}{2}\cdot \frac{dx \cdot \frac{x}{y}-dx}{y-x}=\frac{d(y-x)}{y-x}
    \]
    where the last equality follows from the fact that on the special fibre $\frac{dx}{y}=\frac{dy}{x}$ and since we are in characteristic $3$, $\frac{1}{2}=-1$.
    Hence, if we denote by $Y$ the reduction modulo $\ip$ of $A$, we have that the global $2$-form on $Y$ is given by
    \[
    \omega=\frac{d\left(\frac{v-u}{w}\right)}{\left(\frac{v-u}{w}\right)}\wedge \frac{d\left(\frac{y-x}{z}\right)}{\left(\frac{y-x}{z}\right)}.
    \]
    
    Finally, $\rho_0(\omega,0)=\left[\left\{\frac{v-u}{w},\frac{y-x}{z}\right\}\right]$ and hence again by Lemma~\ref{lemma: rsw and map rho_m} we get that
    \[
        \rsw_{3,\pi}(\A)\ne (\bar{c}^{-1}\cdot \omega,0)
    \]
    and therefore $\A\notin \fil_2\br(A)[3]\supseteq \Ev_{-1}\br(A)[3]$.
    \end{proof}
    \noindent \textbf{Claim 3:} The cyclic algebra $\A$ in $\br(A)[3]$ is not algebraic, i.e. $\A\notin \br_1(A)[3]$.
    \begin{proof}
        If follows directly from Corollary~\ref{cor: refined Swan conductor and algebraic elements}.
    \end{proof}
    Finally, from \cite[Theorem~2.4]{SkoroZarhin} the map 
    \[
    \pi^*\colon \frac{\br(X)[3]}{\br_1(X)[3]}\hookrightarrow \frac{\br(A)[3]}{\br_1(A)[3]}
    \]
    is an isomorphism. Let $\mathcal{B}\in \br(X)[3]$ be such that $\pi^*(\mathcal{B})=\A\in \br(A)[3]$. Then, from Lemma~\ref{lemma: pi^*(Ev_-1br(X)) is in Ev_-1br(A)} we get that $\mathcal{B}\notin \Ev_{-1}\br(X)[3]$, namely (since for K$3$ surfaces $\Ev_0\br(X)=\Ev_{-1}\br(X)$, see \ref{subsubsection: the case of K3 surfaces}) the corresponding evaluation map on $X(L)$ is non-constant.
\end{example}
\begin{example}\label{ex: diagonal quartic ordinary}
    Let $X$ be the diagonal quartic surface over $\Q_5$ defined by the equation:
    \[
    5x^4-4y^4=z^4+w^4.
    \]
    Skorobogatov and Ieronymou prove \cite[Theorem~1.1]{IeronymouSkoro}, \cite[Proposition~5.12]{IeronymouSkoro} that there exists an element $\A\in \br(X)[5]$ with surjective evaluation map. Let $L=\Q_5(\sqrt[4]{5})$, $e(L/\Q_5)=4$ and $\alpha\in L$ be such that $\alpha^4=5$. Then the change of variables:
    \[
    (x,y,z,w)\mapsto \left(\frac{x_1}{\alpha},y_1,z_1,w_1\right)
    \]
    sends $X_{L}$ to the diagonal quartic $\tilde{X}/L$ given by the equation
    \[
        x_1^4-4y_1^4=z_1^4+w_1^4.
    \]
    The surface $\tilde{X}$ has good ordinary reduction over $L$. Finally, by Lemma \ref{lemma: base change no good-reduction} we know that $\mathrm{res}(\A)\in \br(X_{L'})=\br(\tilde{X})$ has non-constant evaluation map.
    
    Note that, since in \cite{IeronymouSkoro} they do not write the algebra $\A$ as a cyclic algebra, we are not able to show explicitly the link with the global logarithmic $2$-forms on the special fibre and hence compute the refined Swan conductor of $\A$, even if from Remark~\ref{rmk: K3 with good reduction global 2 form } we know that $\A\in \fil_p\br(X)[5]$. 
\end{example}
\section{Non-ordinary good reduction}\label{Section: non-ordinary good reduction}
We have seen in the examples of Section \ref{section: ordinary case examples} how for an ordinary K$3$ surface, the existence of a $p$-torsion element obstructing weak approximation is strictly related to the existence of a global logarithmic $2$-form on the special fibre. In this section we want to understand what can happen when the special fibre is non-ordinary.
\subsection{De Rham--Witt complex and logarithmic forms}
 The aim of this section is to briefly introduce the De Rham--Witt complex $W\Omega^{\bullet}_Y$ of a scheme $Y$ over a perfect field $\ell$ of characteristic $p$ and see how the ordinary condition can be read from it. We will not go into the details of the De Rham--Witt complex nor of crystalline cohomology, we will just use them to study the global sections the sheaf of logarithmic $2$-forms on $Y$.

In \cite{Illusie}, Illusie defines the De Rham--Witt complex of $Y$:
$$W\Omega^\bullet_Y\colon \quad W\Omega^0_Y=W\Os_Y\xrightarrow{d}W\Omega^1_Y\xrightarrow{d}\dots\xrightarrow{d} W\Omega^q_Y \xrightarrow{d}\dots .$$
To define the De Rham--Witt complex, Illusie begins by building a projective system $\{W_m \Omega^q_Y\}_{m\geq 0}$ equipped with the Frobenius and the Verschiebung maps
\[
    F\colon W_{m+1} \Omega^q_Y\rightarrow W_m \Omega^q_Y \quad \text{and} \quad V\colon W_m \Omega^q_Y \rightarrow W_{m+1} \Omega^q_Y
\]
for every $m\geq 0$. This projective system has transaction maps given by the projection maps $R\colon W_{m+1}\Omega^j_Y\rightarrow W_m \Omega^j_Y$.
The sheaves $W\Omega^q_Y$ are defined as the inverse limit of the projective system $\{W_m \Omega^q_Y\}_{m\geq 0}$, which is the initial object in a suitable category, see \cite[Chapter~I]{Illusie}. We will state here some properties of the De Rham--Witt complex, without going into the details.
\begin{enumerate}[label=(\alph*)]
    \item For every non-negative $i,j$, the De Rham--Witt cohomology groups $\h^i(Y,W\Omega^j_{Y})$ are finitely generated $W$-modules modulo torsion, where $W=W(\ell)$ is the ring of Witt vectors of $\ell$. 
     \item The De Rham--Witt cohomology groups are strictly related to the crystalline cohomology groups of $Y$ (see \cite{BerhelotOgus} for the definition of the crystalline site and crystalline cohomology). More precisely, there is a spectral sequence, called the the slope spectral sequence: 
     $$E_1^{i,j}:=\h^j(Y,W\Omega^i_Y)\Longrightarrow H^{i+j}_{\mathrm{cris}}(Y/W)$$
     and the following degeneracy results \cite{Illusie}:
     \begin{itemize}
         \item[-] the slope spectral sequence always degenerates modulo torsion at $E_1$;
         \item[-] the slope spectral sequence degenerates at $E_1$ if and only if $\h^j(Y, W \Omega^i_Y)$ is a finitely generated $W$-module for all $i,j$.
     \end{itemize}
\end{enumerate}

Let $\sigma$ be the lift of the Frobenius of $\ell$ to $W$. The complex $W\Omega^\bullet_Y$ is endowed with a natural $\sigma$-semilinear endomorphism  $F\colon W\Omega^\bullet_Y\rightarrow W\Omega^\bullet_Y$, induced by the Frobenius $F\colon W_{m+1}\Omega^q_Y\rightarrow W_m \Omega^q_Y$.
Ordinary varieties (see Definition~\ref{def: ordinary var}) can also be defined in terms of the action of $F$ on the cohomology groups $\h^j(Y,W\Omega^i_Y)$. 
\begin{lemma}
    A proper smooth variety $Y$ is ordinary if and only if 
    \[
        F\colon \h^{j}(Y,W\Omega^i_Y)\rightarrow\h^{j}(Y,W\Omega^i_Y)
    \]
    is bijective for all $i,j$.
\end{lemma}
\begin{proof}
    See \cite[Section~7]{BlochKatoEtale}.
\end{proof}
In order to understand how being ordinary is related to logarithmic forms on $Y$, we first need to introduce the logarithmic Hodge--Witt sheaf.
\begin{defi}\label{defiLogForms}
Given two positive integers $i,m$, the \emph{logarithmic Hodge--Witt sheaf}
$W_m \Omega^i_{X,\log}$ is defined as
$$W_m \Omega^i_{X,\log}:=\mathrm{im}(s\colon (\Os_X^\times)^{\otimes i}\rightarrow W_m \Omega^i_X),$$
where $s$ is defined by 
$$s(x_i\otimes\dots\otimes x_i):=d\log \underline{x}_1\wedge \dots \wedge d\log \underline{x}_i$$
with $\underline{x}=(x,0,\dots,0)\in W_m \Os_X$ the Teichmuller representative of $x\in \Os_X$. The system $\{W_m \Omega^i_{Y,\log}\}_{m\geq 0}$ is a sub-system of $\{W_m \Omega^i_Y\}_{m\geq 0}$. Moreover, for every $i,q$ 
\[
    \h^i(Y,W\Omega^q_{Y,\log}):=\varprojlim_{m} \h^i(Y,W_m \Omega^q_{Y,\log})
\]
where the limit is take over the projection maps $R\colon W_{m+1} \Omega^j_{Y,\log}\rightarrow W_m \Omega^j_{Y,\log}$.
\end{defi}
\subsection{Mittag-Leffler systems}
Assume $Y$ to be smooth and proper over $\ell$. A crucial fact that we will use several time in the next section is that for every $i,j$ the inverse systems of abelian groups $\{\h^i(Y,W_m \Omega^j_Y)\}_{m\geq 0}$ and $\{\h^i(Y,W_m \Omega^j_{Y,\log})\}_{m\geq 0}$ both satisfy the Mittag-Leffler (ML) condition \cite[Corollary~3.5, Pag.~194]{IllusieRaynaud}, \cite[proof of Proposition~$2.1$, page $607$]{Illusie} and
\[
\h^i(Y,W\Omega^j_Y)=\varprojlim_m \h^i(Y,W_m\Omega^j_Y) \quad \text{\cite[Proposition~2.1]{Illusie}}.
\]
The aim of this section is to briefly recap what Mittag-Leffler systems (ML) are and prove some auxiliary lemmas that we will need afterwords. The main reference for this section is \cite[Section~12.31]{Stacks}. 

Let $\mathcal{C}$ be an abelian category. An inverse system consists of a pair $(A_i,\varphi_{ji})$, where for every $i\in \N$, $A_i\in \mathrm{Ob}(\mathcal{C})$ and for each $j>i$ there is a map $\varphi_{ji}\colon A_j\rightarrow A_i$ such that $\varphi_{ji}\circ \varphi_{kj}=\varphi_{ki}$ for all $k>j>i$. We will often omit the transition maps from the notation. We will denote by $\varprojlim_i A_i$ the inverse limit of the system $(A_i)$; this limit always exists and can be described as follow
\[
    \varprojlim_i A_i = \left \{ (a_i)\in \prod_i A_i \, \mid \, \varphi_{i+1,i}(a_{i+1})=a_i, \, i=0,1,2,\dots \right\}.
\]
See \cite[Section~12.31]{Stacks} for more details. 
\begin{defi}
     We say that $(A_i)$ satisfies the \emph{Mittag-Leffler (ML)} condition if for every $i$ there is a $c=c(i)\geq i$ such that 
    \[
        \mathrm{Im}(A_k \xrightarrow{\varphi_{ki}} A_i)=\mathrm{Im}(A_c\xrightarrow{\varphi_{ci}} A_i) 
    \]
    for all $k\geq c$.
\end{defi}
\begin{lemma}\label{lemma: ML condition Stacks}
    Let 
    \[
    0 \rightarrow (A_i) \rightarrow (B_i)\rightarrow (C_i)\rightarrow 0
    \]
    be a short exact sequence of inverse systems of abelian groups.
    \begin{enumerate}[label=(\alph*)]
        \item If $(B_i)$ is ML, then also $(C_i)$ is ML.
        \item If $(A_i)$ is ML, then 
        \[
        0 \rightarrow \varprojlim_i A_i \rightarrow \varprojlim_i B_i \rightarrow \varprojlim_i C_i \rightarrow 0
        \]
        is exact. 
    \end{enumerate}
\end{lemma}
\begin{proof}
    This is part $(2)$ and $(3)$ of \cite[Lemma~12.31.3]{Stacks}.
\end{proof}
\begin{lemma}\label{lemma: ML for A_i and B_i}
    Let $(A_i,\varphi_{ji})$ and $(B_i,\psi_{ji})$ be two inverse systems of abelian groups. Assume that there are maps of inverse systems $(\pi_i)\colon (A_i)\rightarrow (B_i)$ and $(\lambda_{i})\colon (B_i)\rightarrow (A_{i-1})$ such that $\varphi_{i(i-1)}=\lambda_i \circ \pi_i$ and $\psi_{i(i-1)}=\pi_{i-1}\circ \lambda_i$. Then both $(\pi_i)$ and $(\lambda_i)$ induce an isomorphism between $\varprojlim_i A_i $ and $\varprojlim_i B_i$.
    Moreover, $(A_i)$ is ML if and only if $(B_i)$ is ML.
\end{lemma}
\begin{proof}
    Given an inverse system $(A_i)$ and a positive integer $n$, we denote by $(A[n]_i)$ the inverse system given by 
    \[
    A[n]_i=A_{i-n} \text{ and }\varphi[n]_{(i+1)i}=\varphi_{(i-n+1)(i-n)}
    \]
    whenever $i>n$ and the trivial group otherwise. 
    
   The inverse limits of the systems $A[n]_i$ and $A[m]_i$ are isomorphic. In fact, without loss of generality, we can assume $m>n$. The transaction maps $\varphi_{(i-n)(i-m)}\colon A[n]_i \rightarrow A_{i}[m]$ induce a map between inverse systems
   \[
        (\varphi[n,m]_i)\colon (A[n]_i) \rightarrow (A[m]_i).
   \]
  This map induces an isomorphism $\varphi[n,m]$ between $\varprojlim_i A[n]_i$ and $\varprojlim_i A[m]_i$ sending an element $(a_{i-n})\in \varprojlim_i A[n]_i$ to $(\varphi_{i-n,i-m}(a_i))=(a_{i-m})\in \varprojlim_i A[m]_i$. 
  
  Similarly, for every $n,m$ with $m>n$ we can build maps 
  \begin{align*}
      &\psi[n,m]\colon \varprojlim_i B[n]_i \rightarrow \varprojlim_i B[m]_i\\
      &\lambda[n]\colon \varprojlim_i B[n]_i\rightarrow \varprojlim_i A[n+1]_i\\
      &\pi[n]\colon \varprojlim_i A[n]_i\rightarrow \varprojlim B[n]_i
  \end{align*}
  that fit in following commutative diagram 
   \[
    \begin{tikzcd}[row sep=large,column sep=huge]
        \varprojlim_{i} A[1]_{i} \arrow[dr,"\pi{[}1{]}"']\arrow[r,"\varphi{[}1{,}2{]}"]  & \varprojlim_i A[2]_i\arrow[r,"\varphi{[}2{,}3{]}"] \arrow[dr,"\pi{[}2{]}"'] &\varprojlim_{i} A[3]_{i} \\
        \varprojlim_{i} B_{i} \arrow[r,"\psi{[}0{,}1{]}"'] \arrow[u,"\lambda{[}0{]}"] &\varprojlim_{i} B[1]_{i} \arrow[u,"\lambda{[}1{]}"] \arrow[r,"\psi{[}1{,}2{]}"'] & \varprojlim_i B[2]_i. \arrow[u,"\lambda{[}2{]}"]
    \end{tikzcd}
    \]
    As the horizontal maps are isomorphism, the maps $(\pi_i)$ and $(\lambda_{(i+1)i})$ induce the desired isomorphisms. 
    
   Assume now that $(A_i)$ is ML. By definition for every $i$, there is a $c=c(i)$ such that for all $k\geq c$
   \[
        \mathrm{Im}(A_k \xrightarrow{\varphi_{ki}} A_i)=\mathrm{Im}(A_c\xrightarrow{\varphi_{ci}} A_i). 
    \]
    Assume that $k\geq c+1$. Since $(B_i)$ is a inverse system, $\mathrm{Im}(\psi_{ki})\subseteq \mathrm{Im}(\psi_{(c+1) i})$ always.
    
    At the same time, from the following commutative diagram 
    \[
    \begin{tikzcd}
        A_k \arrow[d,"\pi_k"] \arrow[r,"\varphi_{k c}"] &A_{c} \arrow[r,"\varphi_{ci}"] &A_i \arrow[d,"\pi_i"]\\
        B_k \arrow[r,"\psi_{k (c+1)}"'] &B_{c+1} \arrow[u,"\lambda_{c+1}"']\arrow[r,"\psi_{(c+1)i}"'] &B_i\\
    \end{tikzcd}
    \]
    we get that for every $b\in B_{c+1}$ 
    \[
        \psi_{(c+1) i}(b)=\pi_i \circ \varphi_{c i}\circ \lambda_{c+1}(b).
    \]
    Using that the system $(A_i)$ is ML we get that there is an element $a$ in $A_k$ such that $\varphi_{k i}(a)=\varphi_{ci}\circ \lambda_{c+1}(b)$. Putting everything together we get that 
    \[
        \psi_{ki}(\pi_k(a))=\psi_{(c+1)i}\circ \psi_{k(c+1)}(\pi_k(a))=\pi_i \circ  \varphi_{ki}(a)=\pi_i \circ \varphi_{(c+1)i}\circ \lambda_{c+1}(b)=\psi_{(c+1)i}(b).
    \]
    Hence, for all $k\geq c+1$ every element in the image of $\psi_{(c+1)i}$ lies also in the image of $\psi_{ki}$, namely $(B_i)$ is ML. Finally, $(B_i)$ being ML implies that $(A_i)$ is ML as well just follows from replacing $(B_i)$ with $(B[1]_i)$ and switching the roles of $(A_i)$ and $(B_i)$ in what we just proved.
\end{proof}
\begin{lemma}\label{lemma: ML long exact sequences}
    Let 
    \[
    0\rightarrow (A_{1,i}) \rightarrow (A_{2,i})\rightarrow \dots
    \]
    be a long exact sequence of inverse systems of abelian groups. Assume that for every $n$, the inverse system $(A_{n,i})$ is ML. Then we have an induced long exact sequence of abelian groups
    \[
        0\rightarrow \varprojlim_i A_{1,i} \rightarrow \varprojlim_i A_{2,i} \rightarrow \dots 
    \]
\end{lemma}
\begin{proof}
    For every $n$, we define $B_{n,i}:=\mathrm{Im}(A_{n-1,i} \rightarrow  A_{n,i})=\mathrm{Ker}(A_{n,i}\rightarrow A_{n+1,i})$. For every $n$ we get short exact sequences
    \[
    0 \rightarrow (B_{n,i})\rightarrow (A_{n,i}) \rightarrow (B_{n+1,i})\rightarrow 0.
    \]
    From Lemma~\ref{lemma: ML condition Stacks}(a) together with the assumption on the $(A_{n,i})$'s we get that, for every $n$, $(B_{n,i})$ is ML. Using Lemma~\ref{lemma: ML condition Stacks}(b) we get, for every $n$, a short exact sequence of abelian groups 
    \[
    0 \rightarrow \varprojlim_i B_{n,i}\rightarrow \varprojlim_i A_{n,i} \rightarrow \varprojlim_i  B_{n+1,i}\rightarrow 0.
    \]
    The result now follows from putting all these short exact sequences together.
\end{proof}
\subsection{Global logarithmic 2-forms}\label{subsection: global log 2-forms}
We are now ready to go back to the inverse systems given by the cohomology groups associated to $\{W_m \Omega^q_Y\}_{m\geq 0}$ and $\{W_m \Omega^q_{Y,\log}\}_{m\geq 0}$
\begin{lemma}\label{LemmaLongExSeq(1-F)}
   % For every positive integers $i,m$, there is a short exact sequence of inverse systems
    %\begin{equation}\label{eqProSheaves(1-F)}
    %    0\rightarrow W_\bullet \Omega^i_{Y,\log}\rightarrow W_\bullet \Omega^i_Y \xrightarrow{1-F} W_\bullet \Omega^i_Y \rightarrow 0.
    %\end{equation}
    For every positive integer $i$ there is a long exact sequence in cohomology
    \begin{center}
        \begin{tikzcd}
  \h^0(Y,W\Omega^i_{Y,\log}) \arrow[r,hookrightarrow] & \h^0(Y,W\Omega^i_{Y}) \arrow[r,"1-F"] & \h^0(Y,W\Omega^i_{Y}) \ar[out=-30, in=150]{dll} \\
  \h^1(Y,W\Omega^i_{Y,\log}) \arrow[r] & \h^1(Y,W           \Omega^i_{Y}) \arrow[r,"1-F"] & \h^1(Y,W\Omega^i_{Y})     \ar[out=-30, in=150]{dll} \\
    \h^2(Y,W\Omega^i_{Y,\log}) \arrow[r] & \dots .
    \end{tikzcd}
\end{center}
\end{lemma}

\begin{proof}
    Following Illusie \cite[page~567]{Illusie} we define for every non-negative $m$ 
    \[
        \mathrm{Fil}^m W_{m+1}\Omega^i_Y:=\ker(R\colon W_{m+1}\Omega^i_Y\rightarrow W_m \Omega^i_Y).
    \]
     It is proven in \cite[Proposition~3.2, page 568]{Illusie} that $\mathrm{Fil}^m W_{m+1}\Omega^i_Y=V^m \Omega^i_Y+dV^{m-1} \Omega^{i-1}_Y$ and hence
     \[
        F(\mathrm{Fil}^m W_{m+1}\Omega^i_Y)= dV^{m-1} \Omega_Y^{i-1}
     \]
     where the last equality follows from the fact that $FV=p$ and $FdV=d$, see \cite[Chapter~I, page~541]{Illusie}. Therefore, the map $F\colon W_{m+1}\Omega^i_Y\rightarrow W_m \Omega^i_Y$ induces a map
     \[
        F'\colon W_m\Omega^i_Y\rightarrow W_m \Omega^i_Y/ dV^{m-1}\Omega^{i-1}_Y.
     \]
    By \cite[Lemme~2]{TheleneSansucSoule} for every non-negative integer $m$ there is a short exact sequence
     \[
     0 \rightarrow W_m \Omega^i_{Y,\log} \rightarrow W_m\Omega^i_Y \xrightarrow{1-F'} W_m \Omega^i_Y/dV^{m-1} \Omega^{i-1}_Y \rightarrow 0.
     \]
    This sequence (as $m$ varies) induce long exact sequences of inverse systems of abelian groups
    \[
    \begin{tikzcd}[column sep = small]
      \dots \arrow[r] & \{\h^j(Y,W_m\Omega^i_{Y})\}_{m\geq 0} \arrow[r,"1-F'"]  & \{\h^j(Y,W_m \Omega^i_Y/ dV^{m-1}\Omega^{i-1}_Y)\}_{m\geq 0} \ar[out=-30, in=150]{dl}\\
      &\{\h^{j+1}(Y,W_m\Omega^i_{Y,\log})\}_{m\geq 0}  \arrow[r] &\dots 
    \end{tikzcd}
    \]
The projection map $R\colon W_m\Omega^i_Y\rightarrow W_{m-1}\Omega^i_Y$ factors as \[
    W_m\Omega^i_Y\xrightarrow{\pi_m}  W_m\Omega^i_Y/dV^{m-1}\Omega^{i-1}_Y \xrightarrow{\lambda_m} W_{m-1}\Omega^i_Y,
 \]
 see \cite[Proposition~3.2, page 568]{Illusie}.
 Hence, by Lemma~\ref{lemma: ML for A_i and B_i} also the inverse system given by $\{\h^j(Y,W_m \Omega^i_Y/ dV^{m-1}\Omega^{i-1}_Y)\}_{m\geq 0}$ satisfies the Mittag--Leffler condition and $\pi_m$ induces an isomorphism $\pi$
 \[
    \h^j(Y,W\Omega^i_Y) \simeq \varprojlim_m \h^j(Y,W_m \Omega^i_Y/ dV^{m-1}\Omega^{i-1}_Y). 
 \]
Finally, in order to conclude the proof it is enough to apply Lemma~\ref{lemma: ML long exact sequences} and to notice (using Lemma~\ref{lemma: ML for A_i and B_i}) that the isomorphism $R$ induced by the transition maps of the system $\{\h^j(Y,W_m\Omega^i_Y)\}_{m\geq 0}$ 
\[
    R\colon \varprojlim_m \h^j(Y,W_{m+1} \Omega^i_Y)\xrightarrow{\sim} \varprojlim_m \h^j(Y,W_m \Omega^i_Y) 
\]
is such that $(1-F')\circ R= \pi \circ(1-F)$.
\end{proof}
%\textcolor{blue}{Second version of the proof:}
%\begin{proof}    Let $F:W_{m+1}\Omega^i_Y\rightarrow W_m \Omega^i_Y$ and $F':W_m\Omega^i_Y\rightarrow W_m \Omega^i_Y/ dV^{m-1}\Omega^{i-1}_Y$ be as in \cite[2.17, page~562]{Illusie} and \cite[Prop. 3.3, page~569]{Illusie}. By \cite[Lemme~2]{TheleneSansucSoule} for every non-negative integer $m$ there are short exact sequences such that the following diagram\[\begin{tikzcd}0 \arrow[r] &M_{m+1} \arrow[d] \arrow[r] &W_{m+1}\Omega^i_Y \arrow[d,"R"] \arrow[r,"1-F"] &W_m \Omega^i_Y \arrow[d] \arrow[r] &0\\0 \arrow[r] &W_m \Omega^i_{Y,\log} \arrow[r] &W_m\Omega^i_Y \arrow[r,"1-F'"] &W_m \Omega^i_Y/dV^{m-1} \Omega^{i-1}_Y \arrow[r] &0.\end{tikzcd}\]commutes. These sequences (as $m$ varies) induce long exact sequences of projective systems of abelian groups such that the following diagram commutes\[\begin{tikzcd}[column sep = small]\dots \arrow[r] & \{\h^j(Y,W_{m+1}\Omega^i_{Y})\}_{m\geq 0} \arrow[r,"1-F"] \arrow[d,"R"] & \{\h^j(Y,W_m \Omega^i_Y)\}_{m\geq 0} \arrow[r] \arrow[d] &\{\h^{j+1}(Y,M_{m+1})\}_{m\geq 0} \arrow[d] \arrow[r] &\dots\\\dots \arrow[r] & \{\h^j(Y,W_m\Omega^i_{Y})\}_{m\geq 0} \arrow[r,"1-F'"]  & \{\h^j(Y,W_m \Omega^i_Y/ dV^{m-1}\Omega^{i-1}_Y)\}_{m\geq 0} \arrow[r]  &\{\h^{j+1}(Y,W_m\Omega^i_{Y,\log})\}_{m\geq 0}  \arrow[r] &\dots\end{tikzcd}\]All the projective system of abelian groups appearing in the diagram above satisfy the Mittag--Leffler condition \cite[Corollary~3.5, Pag.~194]{IllusieRaynaud}, \cite[Pag.~784]{TheleneSansucSoule}. Hence, taking the inverse limit is preserving the long exact sequences.Finally, the isomorphisms \cite[Pag.~864]{TheleneSansucSoule}\[\varprojlim_m \h^j(Y,W_m \Omega^i_Y/ dV^{m-1}\Omega^{i-1}_Y)\simeq \varprojlim_m \h^j(Y,W_m \Omega^i_Y) \; \text{ and } \; \varprojlim_m \h^j(Y,W_{m+1} \Omega^i_Y)\simeq \varprojlim_m \h^j(Y,W_m \Omega^i_Y) \]prove the result. \end{proof}
The following Lemma gives a way to relate the $\Z_p$-module $\h^j(Y,W\Omega^i_{Y,\log})$ to the $\Z/p^m\Z$-modules $\h^j(Y,W_m \Omega^i_{Y,\log})$.
\begin{lemma}\label{LemmaLongExactSeq(p^m)}
    For every positive integer $n$ we have a long exact sequence in cohomology
    \begin{center}
        \begin{tikzcd}
  \h^0(Y,W\Omega^i_{Y,\log}) \arrow[r,hookrightarrow] & \h^0(Y,W\Omega^i_{Y,\log}) \arrow[r,"p^n"] & \h^0(Y,W_n\Omega^i_{Y,\log}) \ar[out=-30, in=150]{dll} \\
  \h^1(Y,W\Omega^i_{Y,\log}) \arrow[r] & \h^1(Y,W           \Omega^i_{Y,\log}) \arrow[r,"p^n"] & \h^1(Y,W_n\Omega^i_{Y,\log})     \ar[out=-30, in=150]{dll} \\
    \h^2(Y,W\Omega^i_{Y,\log}) \arrow[r] & \dots
    \end{tikzcd}
\end{center}
\end{lemma}
\begin{proof}
In \cite[Lemme~3]{TheleneSansucSoule} the authors show that there is, for every non-negative $m$, a short exact sequence of sheaves
\[
    0\rightarrow W_{m+n} \Omega^i_{\log}\xrightarrow{p^n} W_m \Omega^i_{\log} \rightarrow W_n \Omega^i_{\log}\rightarrow 0
\]
Again, the result follows from taking the long exact sequence of inverse systems of abelian groups in cohomology, then using that the systems involved satisfy the Mittag--Leffler condition and hence apply Lemma~\ref{lemma: ML long exact sequences}.
\end{proof}

{\begin{prop}\label{prop:vanishing_global_log_forms}
    Let $Y$ be a smooth, proper variety over a a perfect field $\ell$ of positive characteristic. Assume that $\h^0(Y,W\Omega^q_{Y})=0$ and $\h^1(Y,W\Omega^q_Y)$ is torsion-free. Then for all $m\geq 1$
    $$\h^0(Y,W_m\Omega^q_{Y,\log})=0.$$
\end{prop}
}
\begin{proof}
    The long exact sequence in cohomology of Lemma \ref{LemmaLongExSeq(1-F)} together with the vanishing of $\h^0(Y,W\Omega^q_Y)$ imply that 
    $$\h^0(Y,W\Omega^q_{Y,\log})=0 \quad \text{and} \quad \h^1(Y,W\Omega^q_{Y,\log})\subseteq \h^1(Y,W\Omega^q_Y).$$ 
    In particular, $\h^1(Y,W\Omega^q_{Y,\log})$ is torsion-free as well. Finally, we can use the long exact sequence of Lemma \ref{LemmaLongExactSeq(p^m)} to get that
    $$\frac{\h^0(Y,W\Omega^q_{Y,\log})}{p^m\cdot \h^0(Y,W\Omega^q_{Y,\log})}\simeq \h^0(Y,W_m\Omega^q_{Y,\log}),$$
    which proves the result.
\end{proof}
The idea is to use the lemma we just proved to show that for non-ordinary K$3$ surfaces perfect fields there are no non-trivial global logarithmic $2$-forms. We will see at the end of the section how a proof of this fact follows immediately from the following lemma and some results proven by Illusie \cite{Illusie}.
\begin{lemma}\label{lemma: on the 2-forms on Y}
    Assume that $\h^0(Y,W\Omega^2_Y)$ is trivial and $\h^1(Y,W\Omega^2_Y)$ is torsion free. Then
    \[
        \h^0(Y,\Omega^2_{Y,\log})=0.
    \]
    If $\h^0(Y,\Omega^2_Y)$ is a $1$-dimensional $\ell$-vector space, then the Cartier operator is zero on the space of global $2$-forms. 
\end{lemma}

\begin{proof}
    The first part is a direct consequence of Proposition \ref{prop:vanishing_global_log_forms} with $q=2$.
    
    If $\h^0(Y,\Omega^2_Y)$ is a $1$-dimensional $\ell$-vector space, for every $\omega \in \h^0(Y,\Omega^2_Y)$ there exists $\lambda\in \ell$ such that 
    \[
        C(\omega)=\lambda \cdot \omega.
    \]
    Assume that $\lambda\ne 0$, then there exists a finite field extension $\ell'/\ell$ that contains an element $\lambda_0$ such that $\lambda=\lambda_0^{p-1}$. In particular, we get 
    \[
        C(\lambda_0^p \cdot \omega)=\lambda_0 \cdot C(\omega)=(\lambda_0\lambda)\cdot \omega =\lambda_0^p \cdot \omega. 
    \]
    This implies, by definition of logarithmic forms, that $\lambda_0^p \cdot \omega \in \h^0(Y_{\ell'},\Omega^2_{Y_{\ell'},\log})$, which gives the desired contradiction (in fact the first part of the Lemma applies also to the base change of $Y$ to $\ell'$). 
\end{proof}
We go back to the general setting of Section~\ref{Section: General Setting} and we work under the extra assumption that the special fibre $Y$ has good non-ordinary reduction and no global $1$-forms. 
\begin{lemma}
    Assume that the special fibre $Y$ of the $\Os_L$-model $\mathcal{X}$ is such that there are no non-trivial global logarithmic $2$-forms and $\h^0(Y,\Omega^2_Y)$ is a $1$-dimensional $\ell$-vector space. Then, if the absolute ramification index $e\leq p-1$, we have $\br(X)=\fil_0 \br(X)$.
\end{lemma}
\begin{proof}
    If $e<p-1$ the result is proven by Bright and Newton, \cite[Lemma~11.2]{BrightNewton} without any assumption on the global section of the sheaves $\Omega^2_{Y,\log}$ and $\Omega^2_Y$. If $e=p-1$, let $\A\in \fil_n \br(X)$ with $\rsw_{n,\pi}=(\alpha,\beta)$. In order to prove the Lemma it is enough to show that $(\alpha,\beta)=(0,0)$.
    \begin{itemize}
        \item[-] If $n<e'$ then $p\nmid n$ and hence by Corollary \ref{cor: on the image of the refined swan conductor}(a) and the assumption that $\h^0(Y,\Omega^1_Y)=0$ we get $\rsw_{n,\pi}(\A)=(0,0)$.
        \item[-] If $n>e'$ then $p\nmid n$ or $p\nmid n-e$ and hence either $\rsw_{n,\pi}(\A)$ or $\rsw_{n-e,\pi}(\A^{\otimes p})$ is zero and therefore by Corollary \ref{cor: on the image of the refined swan conductor}(b) $\rsw_{n,\pi}(\A)=(0,0)$. 
        \item[-] If $n=e'$, then by Corollary \ref{cor: on the image of the refined swan conductor}(c) we have 
        \[
            \rsw_{e'-e,\pi}(\A^{\otimes p})=(C(\alpha)-\bar{u} \alpha,C(\beta)-\bar{u} \beta).
        \]
        However, by assumption on $Y$ there are no non-trivial global $1$-forms, hence $\beta=0$. Moreover, by Lemma \ref{lemma: on the 2-forms on Y} $C(\alpha)=0$. Hence, 
        \[
            \rsw_{e'-e,\pi}(\A^{\otimes p})=(-\bar{u} \alpha,0).
        \]
        Finally, $e'-e<e'$ and hence $\rsw_{e'-e,\pi}(\A^{\otimes p})=0$, which implies that $\alpha=0$ and thus also in this case $\rsw_{e',\pi}(\A)=(0,0)$.
    \end{itemize} 
\end{proof}
In \cite[Chapter~7]{Illusie} Illusie shows that if $Y$ is a non-ordinary K$3$ surface, then $Y$ satisfies the assumptions of Lemma \ref{lemma: on the 2-forms on Y}. Moreover, as already mentioned at the end of \ref{subsubsection: the case of K3 surfaces} from \cite[Proposition~2.3]{ieronymou2021evaluation} we know that $\A\notin \fil_0\br(X)$ if and only if $\ev_\A$ is non-constant on $X(L)$. This shows how the ordinary assumption was needed in the examples that we gave in Section \ref{section: ordinary case examples}.  

\subsection{Non-ordinary case: example}\label{subsection: non ordinary case Example}
This example was suggested by Evis Ieronymou. 

In \cite{ErrataIeronymouSkoro} Skorobogatov and Ieronymou prove that for the diagonal quartic surface $X/\Q_3$ defined by the equation
\begin{equation}
    x^4-y^4=12\cdot z^4-9\cdot w^4
\end{equation}
there exists an element $\A \in \br(X)[3]$ such that the corresponding evaluation map is non-constant on $X(\Q_3)$. In a similar way to what we have shown in Example~\ref{ex: diagonal quartic ordinary} we have that $X$ has potentially good reduction. Let $L=\Q_3(\zeta_4,\alpha)$, with $\alpha^4=3$, then $e(L/\Q_3)=4$; the change of variables 
\[
    (x,y,z,w)\mapsto \left(x_1,y_1,\frac{z_1}{\alpha},\frac{w_1}{\alpha^2}\right)
\]
sends $X_L$ to the diagonal quartic $\tilde{X}/L$ given by the equation
\[
    x_1^4-y_1^4=4\cdot z_1^4-w_1^4.
\]
Using Lemma~\ref{lemma: criterium ordinary k3} it is possible to check that the K$3$ surface $\tilde{X}$ has good non-ordinary reduction over $L$. By Lemma~\ref{lemma: base change no good-reduction} we know that $\mathrm{res}(\A)\in \br(X_L)=\br(\tilde{X})$ has non-constant evaluation map. Hence, $\tilde{X}$ is a K$3$ surface over a $3$-adic field having ramification index $4$, good non-ordinary reduction and such that $\Ev_{-1}\br(\tilde{X})[3]\subsetneq \br(\tilde{X})[3]$. Moreover, $e'=6$ and we proved that $\rsw_{6,\pi}=0$ on the $3$-torsion of $\br(\tilde{X})$ and hence $\mathrm{res}(\A)$ has to lie in $\fil_3\br(\tilde{X})[3]$ and be such that $\rsw_{3,\pi}(\mathrm{res}(\A))\ne (0,0)$.

\section{Family of examples}\label{Section: family of examples}
We end this paper by giving an example of a family of K$3$ surfaces.

Let $\alpha \in \bar{\Q}$ be such that $\alpha^2\in \Z$ and let $X_\alpha$ be the K$3$ surface over $k:=\Q(\alpha)$ defined by the equation 
\begin{equation}\label{eq: example X_a}
        x^3y + y^3 z+ z^3 w- w^4+\alpha^2\cdot  xyzw -2\cdot \alpha^{-1} \cdot x z w^2=0.
    \end{equation}
    \begin{lemma}
        The class of the quaternion algebra 
        \[
         \A:= \left(\frac{z^2+\alpha^2 \cdot  xy}{z^2},-\frac{z}{x}\right)\in \br(k(X_\alpha))
        \]
        lies in $\br(X_\alpha)[2]$. 
    \end{lemma}
\begin{proof}
    Let $f:=z^2+\alpha^2 xy$ and $C_x$,$C_z$,$C_{f}$ be the closed subsets of $X$ defined by the equations $x=0$, $z=0$ and $f=0$ respectively. The quaternion algebra $\A$ defines an element in $\br(U)$, where $U:=X\setminus (C_x\cup C_z \cup C_{f})$.
The purity theorem for the Brauer group \cite[ Theorem 3.7.2]{BGgroupTheleneSkoro}, assures us of the existence of the exact sequence 
\begin{equation}\label{purity}
    0 \rightarrow \br(X)[2] \rightarrow \br(U)[2] \xrightarrow{\oplus \partial_{D}} \bigoplus_{D} \h^1(k(D),\Z/2)
\end{equation}
where $D$ ranges over the irreducible divisors of $X$ with support in $X\setminus U$ and $k(D)$ denotes the residue field at the generic point of $D$. 

In order to use the exact sequence (\ref{purity}) we need to understand how the prime divisors of $X$ with support in $X\setminus U=C_x \cup C_z \cup C_{f}$ look like. Using MAGMA \cite{magma} it is possible to check the following:
\begin{itemize}
    \item $C_x$ has one irreducible component $D_1$ defined by the equations $\{x=0,y^3z+z^3w+w^4=0\}$;
    \item $C_z$ has one irreducible component $D_2$, defined by the equations $\{z=0,x^3y-w^4=0\}$;
    \item $C_{f}$ has one irreducible component $D_3$, defined by the equations $\{\alpha^2xy + z^2=0, x^3z^2 + y^2z^3 + 2\alpha x^2zw^2 + \alpha^2 xw^4=0,
        \alpha^2 y^3z -x^2z^2 - 2\alpha xzw^2 -\alpha^2 w^4=0
        \}$.
\end{itemize}
Therefore, we can rewrite (\ref{purity}) in the following way:
\begin{equation}\label{purity2}
    0 \rightarrow \br(X)[2] \rightarrow \br(U)[2] \xrightarrow{\oplus \partial_{D_i}} \bigoplus_{i=1}^3 H^1(k(D_i),\Z/2).
\end{equation}
Moreover, we have an explicit description of the residue map on quaternion algebras: for an element $(a,b)\in \br(U)[2]$ we have 
\begin{equation}\label{eqResidueMap}
    \partial_{D_i}(a,b)=\left[(-1)^{\nu_i(a)\nu_i(b)}\frac{a^{\nu_i(b)}}{b^{\nu_i(a)}}\right]\in \frac{k(D_i)^\times}{k(D_i)^{\times2}}\simeq H^1(k(D_i),\Z/2) 
\end{equation}
where $\nu_i$ is the valuation associated to the prime divisor $D_i$. This follows from the definition of the tame symbols in Milnor $K$-theory together with the compatibility of the residue map with the tame symbols given by the Galois symbols (see \cite{GilleSzamuely}, Proposition $7.5.1$).

We can proceed with the computation of the residue maps $\partial_{D_i}$ for $i=1,\dots,3$:

\begin{enumerate}
    \item $\nu_1(x)=1$ and $\nu_1(f)=\nu_1(z)=0$. Hence,
    \[
         \partial_{D_1}\left(\frac{f}{z^2},-\frac{z}{x}\right)=\left[\left(\frac{f}{z^2}\right)^{-1}\right]=1\in \frac{k(D_1)^\times}{k(D_1)^{\times 2}}
    \]
    where the last equality follows from the fact that $x=0$ on $D_1$, thus $f\mid_{D_1}=z^2$.
    \item $\nu_2(z)=1$ and $\nu_2(f)=\nu_2(x)=0.$ Hence,
    \[
        \partial_{D_2}\left(\frac{f}{z^2},-\frac{z}{x}\right)=\left[\left(\frac{f}{z^2}\right)\left(-\frac{z}{x}\right)^{2}\right]=\left[\frac{f}{x^2}\right]=1\in \frac{k(D_2)^\times}{k(D_2)^{\times 2}}
    \]
    where the last equality follows from the fact that $z=0$ and $x^3y=w^4$ on $D_2$ , thus $f\mid_{D_2}=\alpha^2 xy$ and $\frac{\alpha^2 y}{x}=\alpha^2 \left(\frac{w}{x}\right)^4=\left(\alpha \frac{w^2}{x^2}\right)^2$.
    \item $\nu_3(f)=1$ and $\nu_3(x)=\nu_3(z)=0$. Hence, 
    \[
        \partial_{D_3}\left(\frac{f}{z^2},-\frac{z}{x}\right)=\left[-\frac{z}{x}\right]=\left[ \left(\frac{\alpha^3 x}{z^3} \left(w^2+\frac{xz}{\alpha}\right)\right)^2\right]=1\in \frac{k(D_3)^\times}{k(D_3)^{\times 2}}
    \]
    where the last equality follows from the following equalities on $D_3$:
    \begin{itemize}
        \item $y^3z=w^4+\frac{2}{\alpha} xz w^2+\frac{x^2z^2}{\alpha^2}=\left(w^2+\frac{xz}{\alpha}\right)^2$;
        \item $xy=-\frac{1}{\alpha^2}z^2$ implies that $y^3z=(xy)^3 \frac{z}{x}\frac{1}{x^2}=-\frac{z}{x}\left(\frac{z^2}{\alpha^2}\right)^3 \frac{1}{x^2}=\frac{-z}{x}\left(\frac{z^3}{\alpha^3 x}\right)^2$
    \end{itemize}
\end{enumerate}
Therefore, $\partial_{D_i}(\A)=0$ for all $i\in \{1,2,3\}$, hence $\A\in \br(X_\alpha)$.
    \end{proof}
    Let $\ip$ be a prime above $2$ and $\Os_{\ip}$ be the valuation ring of $k_{\ip}$. We have that $2\cdot \alpha^{-1}\in \Os_{\ip}$ if and only if $\alpha^2 \not \equiv 0 \mod 8$; we can define $\mathcal{X}_\alpha$ to be the $\Os_{\ip}$-scheme defined by equation \eqref{eq: example X_a}. If $\alpha^2\not\equiv 0 \mod 8$, then $\mathcal{X}_\alpha$ is smooth and hence $X_\alpha$ has good reduction at $\ip$. 
    \begin{thm}
        Assume that $\alpha^2\not\equiv 0 \mod 8$. Then, $\mathcal{X}_\alpha$ has good ordinary reduction if and only if $\alpha^2\equiv 1 \mod 2$. The evaluation map attached to $\A$ 
        \[
            \ev_\A\colon X_\alpha (k_{\ip})\rightarrow \Q/\Z
        \]
        is non-constant if and only if 
         \[
            \alpha^2 \not \equiv 0\mod 4.
        \]
    \end{thm}
    \begin{proof}
       Recall that we know that for K$3$ surfaces the evaluation map attached to $\A$ is non-constant if and only if $\A \notin \fil_0\br(X_\alpha)$ (cfr.~\ref{subsubsection: the case of K3 surfaces}). 
        \begin{itemize}
            \item If $\alpha^2\equiv 1\mod 2$, then the special fibre $Y_\alpha$ is defined by the equation 
        \[
        x^3y + y^3 z+z^3 w+w^4+xyzw.
        \]
        From Lemma~\ref{lemma: criterium ordinary k3} we get that $Y_\alpha$ is an ordinary K$3$ surface. From Lemma~\ref{lemma: explicit global 2-form} we know that a generator (as $k(\ip)$-vector space) of $\h^0(Y_\alpha,\Omega^2_{Y_\alpha})$ is given by the global $2$-form $\omega$ that can be written (locally) as 
        \[
            \frac{d\left(\frac{y}{x}\right)\wedge d\left(\frac{z}{x}\right)}{\frac{z^3+xyz}{x^3}}=\frac{x^2}{z^2+xy}d\left(\frac{y}{x}\right)\wedge d\left(\frac{z}{x}\right)\frac{x}{z}=\left(\frac{x^2}{z^2+xy}\right)\cdot d\left(\frac{z^2+xy}{x^2}\right)\wedge \left(\frac{x}{z}\right)\cdot d\left(\frac{z}{x}\right)
        \]
        where the last equality follows from 
        $d\left(\frac{z^2+xy}{x^2}\right)\wedge d\left(\frac{z}{x}\right)=d\left(\frac{y}{x}\right)\wedge d\left(\frac{z}{x}\right)$. Hence, we can write $\omega$ as $\frac{df}{f}\wedge \frac{dg}{g}$,
        with $f=\frac{z^2+xy}{z^2}$, $g=\frac{z}{x}$. Finally, we see that, by Proposition~\ref{propGradedPieces}
        \[
        [\A]=\left[\left\{\frac{z^2+\alpha^2 \cdot  xy}{z^2},-\frac{z}{x}\right\}\right]=\left[\left\{\frac{z^2+\alpha^2 \cdot  xy}{x^2},-\frac{z}{x}\right\}\right]=\rho_0(\omega,0)
        \]
        since $\frac{z^2+\alpha^2 \cdot  xy}{x^2}$ and $-\frac{z}{x}$ are lifts to characteristic $0$ of $f$ and $g$, respectively. Hence, using Lemma~\ref{lemma: rsw and map rho_m} we get that $\rsw_{e',\pi}(\A)\ne (0,0)$ and $\A\notin \fil_{e'-1}\br(X_\alpha)\supseteq \fil_0\br(X_\alpha)$.
       \item If $\alpha^2\equiv 2\mod 4$, then the special fibre $Y_\alpha$ is defined by the equation 
        \begin{equation*}
            x^3y + y^3 z+z^3 w+w^4
        \end{equation*}
        From Lemma~\ref{lemma: criterium ordinary k3} we get that $Y_\alpha$ is a non-ordinary K$3$ surface over $k(\ip)$. From Lemma~\ref{lemma: explicit global 2-form} we know that $\h^0(Y_\alpha,\Omega^2_{Y_\alpha})$ is generated (as a $k(\ip)$-vector space) by the $2$-form $\omega$ that can be written (locally) as
        \[
            \frac{d\left(\frac{y}{x}\right)\wedge d\left(\frac{z}{x}\right)}{\frac{z^3}{x^3}}=\left(\frac{x^2}{z^2}\right)\cdot d\left(\frac{y}{x}\right)\wedge \left(\frac{x}{z}\right)\cdot  d\left(\frac{z}{x}\right)= d\left(\frac{xy}{z^2}\right)\wedge \left(\frac{x}{z}\right)\cdot d\left(\frac{z}{x}\right)
        \]
        where the last equality follow from the fact that, since we are working over a field of characteristic $2$, $\left(\frac{x}{z}\right)^2 d\left(\frac{y}{x}\right)=d\left(\frac{xy}{z^2}\right)$. Hence, in this case we can write $\omega$ as $d\left(f \cdot \frac{dg}{g}\right)$,
        with $f=\frac{xy}{z^2}$, $g=\frac{z}{x}$. Since $\alpha^2\equiv 2\mod 4$, the prime ideal $(2)$ is ramified in the field extension $\Q_2(\alpha)/\Q_2$ and $\pi=\alpha$ is a uniformiser. From Proposition~\ref{propGradedPieces} we get that 
        \[
            \left[\left\{\frac{z^2+\alpha^2 \cdot  xy}{z^2},-\frac{z}{x}\right\}\right]=\left[\left\{1+\alpha^2 \frac{xy}{z^2},-\frac{z}{x}\right\}\right]=\rho_2(\omega,0)
        \]
        since $\frac{xy}{z^2}$ and $-\frac{z}{x}$ are two lifts to characteristic $0$ of $f$ and $g$, respectively.
        Hence, by Lemma~\ref{lemma: rsw and map rho_m} $\rsw_{2,\pi}(\A)\ne (0,0)$ and hence $\A\notin \fil_1\br(X_\alpha)\supseteq \fil_0\br(X_\alpha)$. 
        \item If $\alpha^2\equiv 0 \mod 4$, then the special fibre $Y_\alpha$ is defined by the equation 
        \[
            x^3 y+y^3z +z^3 w+w^4+xzw^2.
        \]
        Again, from Lemma~\ref{lemma: criterium ordinary k3} we get that $Y_\alpha$ is a non-ordinary K$3$ surface over $k(\ip)$. If $\Q_2(\alpha)/\Q_2$ is unramified then we are done by Theorem~\ref{thm: good non-ordinary reduction}. If the field extension is ramified and $\pi$ is a uniformiser, then since $\alpha^2\equiv 0 \mod 4$ and $\alpha^2\not\equiv 0 \mod 8$, we have that $\alpha^2=\pi^4 \beta$ with $\beta\in \Os_{\ip}^\times$. Hence, if we look at the corresponding element in $k^2(K^h)$ via the isomorphism of equation~\eqref{eqIsomoprhismRootUnity}, we have that 
         \[
            \A\mapsto \left\{\frac{z^2+\alpha^2 \cdot  xy}{z^2},-\frac{z}{x}\right\}=\left\{1+\pi^4\cdot \frac{\beta\cdot xy}{z^2},-\frac{z}{x}\right\} \in U^{4} k^2(K^h).
        \]
         Hence, using again Lemma~\ref{lemma: rsw and map rho_m}, $\A\in \fil_0\br(X_\alpha)[2]$.
         \end{itemize}
    \end{proof}
   \begin{rmk}
       The case $\alpha^2\equiv 2\mod 4$ proves that the bound appearing in Theorem~\ref{thm: good non-ordinary reduction} is optimal. In fact, we are able to find examples of K$3$ surface $X$ over a quadratic field extension of $\Q$ such that there is a prime above $2$ whose ramification index is $e(\ip/2)=2$ and that plays a role in the Brauer--Manin obstruction to weak approximation.
       
       When $\alpha^2\equiv 0 \mod 4$ and $e(\ip/2)=1$, we know from Theorem~\ref{thm: good non-ordinary reduction} that there is an equality $\Ev_{-1}\br(X_\alpha)=\br(X_\alpha)$. However, if $e(\ip/2)=2$, we just showed that the element $\A$ of the previous theorem lies in $\Ev_{-1}\br(X_\alpha)[2]$ and not that $\br(X_\alpha)=\Ev_{-1}\br(X_\alpha)$.
   \end{rmk}

\newpage
\bibliographystyle{plain}
\bibliography{bib.bib}
\textsc{Mathematisch Instituut, Niels Bohrweg 1, 2333 CA Leiden, Netherlands}\\
\textit{Email Address:} \textbf{m.pagano@math.leidenuniv.nl}
\end{document}