
%% bare_jrnl_compsoc.tex
%% V1.4b
%% 2015/08/26
%% by Michael Shell
%% See:
%% http://www.michaelshell.org/
%% for current contact information.
%%
%% This is a skeleton file demonstrating the use of IEEEtran.cls
%% (requires IEEEtran.cls version 1.8b or later) with an IEEE
%% Computer Society journal paper.
%%
%% Support sites:
%% http://www.michaelshell.org/tex/ieeetran/
%% http://www.ctan.org/pkg/ieeetran
%% and
%% http://www.ieee.org/

%%*************************************************************************
%% Legal Notice:
%% This code is offered as-is without any warranty either expressed or
%% implied; without even the implied warranty of MERCHANTABILITY or
%% FITNESS FOR A PARTICULAR PURPOSE! 
%% User assumes all risk.
%% In no event shall the IEEE or any contributor to this code be liable for
%% any damages or losses, including, but not limited to, incidental,
%% consequential, or any other damages, resulting from the use or misuse
%% of any information contained here.
%%
%% All comments are the opinions of their respective authors and are not
%% necessarily endorsed by the IEEE.
%%
%% This work is distributed under the LaTeX Project Public License (LPPL)
%% ( http://www.latex-project.org/ ) version 1.3, and may be freely used,
%% distributed and modified. A copy of the LPPL, version 1.3, is included
%% in the base LaTeX documentation of all distributions of LaTeX released
%% 2003/12/01 or later.
%% Retain all contribution notices and credits.
%% ** Modified files should be clearly indicated as such, including  **
%% ** renaming them and changing author support contact information. **
%%*************************************************************************


% *** Authors should verify (and, if needed, correct) their LaTeX system  ***
% *** with the testflow diagnostic prior to trusting their LaTeX platform ***
% *** with production work. The IEEE's font choices and paper sizes can   ***
% *** trigger bugs that do not appear when using other class files.       ***                          ***
% The testflow support page is at:
% http://www.michaelshell.org/tex/testflow/


\documentclass[10pt,journal,compsoc]{IEEEtran}
%
% If IEEEtran.cls has not been installed into the LaTeX system files,
% manually specify the path to it like:
% \documentclass[10pt,journal,compsoc]{../sty/IEEEtran}





% Some very useful LaTeX packages include:
% (uncomment the ones you want to load)


% *** MISC UTILITY PACKAGES ***
%
%\usepackage{ifpdf}
% Heiko Oberdiek's ifpdf.sty is very useful if you need conditional
% compilation based on whether the output is pdf or dvi.
% usage:
% \ifpdf
%   % pdf code
% \else
%   % dvi code
% \fi
% The latest version of ifpdf.sty can be obtained from:
% http://www.ctan.org/pkg/ifpdf
% Also, note that IEEEtran.cls V1.7 and later provides a builtin
% \ifCLASSINFOpdf conditional that works the same way.
% When switching from latex to pdflatex and vice-versa, the compiler may
% have to be run twice to clear warning/error messages.






% *** CITATION PACKAGES ***
%
\ifCLASSOPTIONcompsoc
  % IEEE Computer Society needs nocompress option
  % requires cite.sty v4.0 or later (November 2003)
  \usepackage[nocompress]{cite}
\else
  % normal IEEE
  \usepackage{cite}
\fi
% cite.sty was written by Donald Arseneau
% V1.6 and later of IEEEtran pre-defines the format of the cite.sty package
% \cite{} output to follow that of the IEEE. Loading the cite package will
% result in citation numbers being automatically sorted and properly
% "compressed/ranged". e.g., [1], [9], [2], [7], [5], [6] without using
% cite.sty will become [1], [2], [5]--[7], [9] using cite.sty. cite.sty's
% \cite will automatically add leading space, if needed. Use cite.sty's
% noadjust option (cite.sty V3.8 and later) if you want to turn this off
% such as if a citation ever needs to be enclosed in parenthesis.
% cite.sty is already installed on most LaTeX systems. Be sure and use
% version 5.0 (2009-03-20) and later if using hyperref.sty.
% The latest version can be obtained at:
% http://www.ctan.org/pkg/cite
% The documentation is contained in the cite.sty file itself.
%
% Note that some packages require special options to format as the Computer
% Society requires. In particular, Computer Society  papers do not use
% compressed citation ranges as is done in typical IEEE papers
% (e.g., [1]-[4]). Instead, they list every citation separately in order
% (e.g., [1], [2], [3], [4]). To get the latter we need to load the cite
% package with the nocompress option which is supported by cite.sty v4.0
% and later. Note also the use of a CLASSOPTION conditional provided by
% IEEEtran.cls V1.7 and later.





% *** GRAPHICS RELATED PACKAGES ***
%
\ifCLASSINFOpdf
  % \usepackage[pdftex]{graphicx}
  % declare the path(s) where your graphic files are
  % \graphicspath{{../pdf/}{../jpeg/}}
  % and their extensions so you won't have to specify these with
  % every instance of \includegraphics
  % \DeclareGraphicsExtensions{.pdf,.jpeg,.png}
\else
  % or other class option (dvipsone, dvipdf, if not using dvips). graphicx
  % will default to the driver specified in the system graphics.cfg if no
  % driver is specified.
  % \usepackage[dvips]{graphicx}
  % declare the path(s) where your graphic files are
  % \graphicspath{{../eps/}}
  % and their extensions so you won't have to specify these with
  % every instance of \includegraphics
  % \DeclareGraphicsExtensions{.eps}
\fi
% graphicx was written by David Carlisle and Sebastian Rahtz. It is
% required if you want graphics, photos, etc. graphicx.sty is already
% installed on most LaTeX systems. The latest version and documentation
% can be obtained at: 
% http://www.ctan.org/pkg/graphicx
% Another good source of documentation is "Using Imported Graphics in
% LaTeX2e" by Keith Reckdahl which can be found at:
% http://www.ctan.org/pkg/epslatex
%
% latex, and pdflatex in dvi mode, support graphics in encapsulated
% postscript (.eps) format. pdflatex in pdf mode supports graphics
% in .pdf, .jpeg, .png and .mps (metapost) formats. Users should ensure
% that all non-photo figures use a vector format (.eps, .pdf, .mps) and
% not a bitmapped formats (.jpeg, .png). The IEEE frowns on bitmapped formats
% which can result in "jaggedy"/blurry rendering of lines and letters as
% well as large increases in file sizes.
%
% You can find documentation about the pdfTeX application at:
% http://www.tug.org/applications/pdftex

\usepackage{epsfig}
\usepackage{graphicx}
\usepackage{booktabs}       % professional-quality tables
\usepackage{amsfonts}       % blackboard math symbols
\usepackage{amsthm}
\usepackage{nicefrac}       % compact symbols for 1/2, etc.
\usepackage{microtype}      % microtypography
\usepackage{xcolor,pifont}
\usepackage{epsfig}
%\usepackage{caption}
\usepackage{subfigure}
%\usepackage{subcaption}
\usepackage{amsmath}
\usepackage{array}% http://ctan.org/pkg/array
\usepackage{pifont}
\usepackage{wrapfig}
\usepackage{hhline}
\usepackage{mathrsfs} 
\usepackage{algorithm}
\usepackage{listings}
\usepackage{xcolor,amsmath}

%----------------------------------------------------------
% this is for adding footnote after algorithm
\usepackage{etoolbox}
\makeatletter
\AfterEndEnvironment{algorithm}{\let\@algcomment\relax}
\AtEndEnvironment{algorithm}{\kern2pt\hrule\relax\vskip3pt\@algcomment}
\let\@algcomment\relax
\newcommand\algcomment[1]{\def\@algcomment{\footnotesize#1}}
\renewcommand\fs@ruled{\def\@fs@cfont{\bfseries}\let\@fs@capt\floatc@ruled
  \def\@fs@pre{\hrule height.8pt depth0pt \kern2pt}%
  \def\@fs@post{}%
  \def\@fs@mid{\kern2pt\hrule\kern2pt}%
  \let\@fs@iftopcapt\iftrue}
\makeatother

\usepackage{color, colortbl}
\usepackage{bm}
\usepackage{amssymb}
\usepackage{multirow, tabularx}
\usepackage{dsfont}
\usepackage[normalem]{ulem}
\usepackage[pagebackref=true,breaklinks=true,colorlinks,bookmarks=false, allcolors=blue]{hyperref}
\usepackage[capitalize]{cleveref}
\useunder{\uline}{\ul}{}
\usepackage[misc]{ifsym}
% Include other packages here, before hyperref.

% If you comment hyperref and then uncomment it, you should delete
% egpaper.aux before re-running latex.  (Or just hit 'q' on the first latex
% run, let it finish, and you should be clear).
% \usepackage[pagebackref=true,breaklinks=true,letterpaper=true,colorlinks,bookmarks=false]{hyperref}

\def\eg{\emph{e.g}. } \def\Eg{\emph{E.g}. }
\def\ie{\emph{i.e}. } \def\Ie{\emph{I.e}. }
\def\cf{\emph{cf}. } \def\Cf{\emph{Cf}.}
\def\etc{\emph{etc}. } \def\vs{\emph{vs}. }
\def\wrt{w.r.t. } \def\dof{d.o.f. }
\def\iid{i.i.d. } \def\wolog{w.l.o.g. }
\def\etal{\emph{et al}. }
\makeatother

\usepackage{xcolor}
\definecolor{COLOR_MEAN}{HTML}{f0f0f0}
\definecolor{ROW_COLOR}{HTML}{C9F7F4}
\definecolor{best}{HTML}{48cae4}
\definecolor{second_best}{HTML}{ffd166}
\definecolor{light_blue}{HTML}{0096c7}
\definecolor{contrast_purple}{HTML}{7b2cbf}
\definecolor{cool_orange}{HTML}{eb5e28}

\newcommand{\lht}[1]{{\color{orange}[HT: #1]}}
\newcommand{\chunyl}[1]{{\color{blue}[CY: #1]}}
\newcommand{\bo}[1]{{\color{red}[Bo: #1]}}

\newcommand{\cler}{\textbf{CLER}} 
\newcommand{\gb}{\textbf{GenBench}}
\newcommand*\rot{\rotatebox{90}}
\newcommand{\common}{% Figure removed{~Common}}
\newcommand{\finegrained}{% Figure removed{~Fine-grained}}
\newcommand{\rare}{% Figure removed{~Rare}}
\newcommand{\stars}{% Figure removed{}}

\usepackage{xcolor}
\definecolor{COLOR_MEAN}{HTML}{f0f0f0}
\definecolor{ROW_COLOR}{HTML}{C9F7F4}
\definecolor{best}{HTML}{48cae4}
\definecolor{second_best}{HTML}{ffd166}
\definecolor{light_blue}{HTML}{0096c7}
\definecolor{contrast_purple}{HTML}{7b2cbf}
\definecolor{cool_orange}{HTML}{eb5e28}

% *** MATH PACKAGES ***
%
%\usepackage{amsmath}
% A popular package from the American Mathematical Society that provides
% many useful and powerful commands for dealing with mathematics.
%
% Note that the amsmath package sets \interdisplaylinepenalty to 10000
% thus preventing page breaks from occurring within multiline equations. Use:
%\interdisplaylinepenalty=2500
% after loading amsmath to restore such page breaks as IEEEtran.cls normally
% does. amsmath.sty is already installed on most LaTeX systems. The latest
% version and documentation can be obtained at:
% http://www.ctan.org/pkg/amsmath





% *** SPECIALIZED LIST PACKAGES ***
%
%\usepackage{algorithmic}
% algorithmic.sty was written by Peter Williams and Rogerio Brito.
% This package provides an algorithmic environment fo describing algorithms.
% You can use the algorithmic environment in-text or within a figure
% environment to provide for a floating algorithm. Do NOT use the algorithm
% floating environment provided by algorithm.sty (by the same authors) or
% algorithm2e.sty (by Christophe Fiorio) as the IEEE does not use dedicated
% algorithm float types and packages that provide these will not provide
% correct IEEE style captions. The latest version and documentation of
% algorithmic.sty can be obtained at:
% http://www.ctan.org/pkg/algorithms
% Also of interest may be the (relatively newer and more customizable)
% algorithmicx.sty package by Szasz Janos:
% http://www.ctan.org/pkg/algorithmicx




% *** ALIGNMENT PACKAGES ***
%
%\usepackage{array}
% Frank Mittelbach's and David Carlisle's array.sty patches and improves
% the standard LaTeX2e array and tabular environments to provide better
% appearance and additional user controls. As the default LaTeX2e table
% generation code is lacking to the point of almost being broken with
% respect to the quality of the end results, all users are strongly
% advised to use an enhanced (at the very least that provided by array.sty)
% set of table tools. array.sty is already installed on most systems. The
% latest version and documentation can be obtained at:
% http://www.ctan.org/pkg/array


% IEEEtran contains the IEEEeqnarray family of commands that can be used to
% generate multiline equations as well as matrices, tables, etc., of high
% quality.




% *** SUBFIGURE PACKAGES ***
%\ifCLASSOPTIONcompsoc
%  \usepackage[caption=false,font=footnotesize,labelfont=sf,textfont=sf]{subfig}
%\else
%  \usepackage[caption=false,font=footnotesize]{subfig}
%\fi
% subfig.sty, written by Steven Douglas Cochran, is the modern replacement
% for subfigure.sty, the latter of which is no longer maintained and is
% incompatible with some LaTeX packages including fixltx2e. However,
% subfig.sty requires and automatically loads Axel Sommerfeldt's caption.sty
% which will override IEEEtran.cls' handling of captions and this will result
% in non-IEEE style figure/table captions. To prevent this problem, be sure
% and invoke subfig.sty's "caption=false" package option (available since
% subfig.sty version 1.3, 2005/06/28) as this is will preserve IEEEtran.cls
% handling of captions.
% Note that the Computer Society format requires a sans serif font rather
% than the serif font used in traditional IEEE formatting and thus the need
% to invoke different subfig.sty package options depending on whether
% compsoc mode has been enabled.
%
% The latest version and documentation of subfig.sty can be obtained at:
% http://www.ctan.org/pkg/subfig




% *** FLOAT PACKAGES ***
%
%\usepackage{fixltx2e}
% fixltx2e, the successor to the earlier fix2col.sty, was written by
% Frank Mittelbach and David Carlisle. This package corrects a few problems
% in the LaTeX2e kernel, the most notable of which is that in current
% LaTeX2e releases, the ordering of single and double column floats is not
% guaranteed to be preserved. Thus, an unpatched LaTeX2e can allow a
% single column figure to be placed prior to an earlier double column
% figure.
% Be aware that LaTeX2e kernels dated 2015 and later have fixltx2e.sty's
% corrections already built into the system in which case a warning will
% be issued if an attempt is made to load fixltx2e.sty as it is no longer
% needed.
% The latest version and documentation can be found at:
% http://www.ctan.org/pkg/fixltx2e


%\usepackage{stfloats}
% stfloats.sty was written by Sigitas Tolusis. This package gives LaTeX2e
% the ability to do double column floats at the bottom of the page as well
% as the top. (e.g., "\begin{figure*}[!b]" is not normally possible in
% LaTeX2e). It also provides a command:
%\fnbelowfloat
% to enable the placement of footnotes below bottom floats (the standard
% LaTeX2e kernel puts them above bottom floats). This is an invasive package
% which rewrites many portions of the LaTeX2e float routines. It may not work
% with other packages that modify the LaTeX2e float routines. The latest
% version and documentation can be obtained at:
% http://www.ctan.org/pkg/stfloats
% Do not use the stfloats baselinefloat ability as the IEEE does not allow
% \baselineskip to stretch. Authors submitting work to the IEEE should note
% that the IEEE rarely uses double column equations and that authors should try
% to avoid such use. Do not be tempted to use the cuted.sty or midfloat.sty
% packages (also by Sigitas Tolusis) as the IEEE does not format its papers in
% such ways.
% Do not attempt to use stfloats with fixltx2e as they are incompatible.
% Instead, use Morten Hogholm'a dblfloatfix which combines the features
% of both fixltx2e and stfloats:
%
% \usepackage{dblfloatfix}
% The latest version can be found at:
% http://www.ctan.org/pkg/dblfloatfix




%\ifCLASSOPTIONcaptionsoff
%  \usepackage[nomarkers]{endfloat}
% \let\MYoriglatexcaption\caption
% \renewcommand{\caption}[2][\relax]{\MYoriglatexcaption[#2]{#2}}
%\fi
% endfloat.sty was written by James Darrell McCauley, Jeff Goldberg and 
% Axel Sommerfeldt. This package may be useful when used in conjunction with 
% IEEEtran.cls'  captionsoff option. Some IEEE journals/societies require that
% submissions have lists of figures/tables at the end of the paper and that
% figures/tables without any captions are placed on a page by themselves at
% the end of the document. If needed, the draftcls IEEEtran class option or
% \CLASSINPUTbaselinestretch interface can be used to increase the line
% spacing as well. Be sure and use the nomarkers option of endfloat to
% prevent endfloat from "marking" where the figures would have been placed
% in the text. The two hack lines of code above are a slight modification of
% that suggested by in the endfloat docs (section 8.4.1) to ensure that
% the full captions always appear in the list of figures/tables - even if
% the user used the short optional argument of \caption[]{}.
% IEEE papers do not typically make use of \caption[]'s optional argument,
% so this should not be an issue. A similar trick can be used to disable
% captions of packages such as subfig.sty that lack options to turn off
% the subcaptions:
% For subfig.sty:
% \let\MYorigsubfloat\subfloat
% \renewcommand{\subfloat}[2][\relax]{\MYorigsubfloat[]{#2}}
% However, the above trick will not work if both optional arguments of
% the \subfloat command are used. Furthermore, there needs to be a
% description of each subfigure *somewhere* and endfloat does not add
% subfigure captions to its list of figures. Thus, the best approach is to
% avoid the use of subfigure captions (many IEEE journals avoid them anyway)
% and instead reference/explain all the subfigures within the main caption.
% The latest version of endfloat.sty and its documentation can obtained at:
% http://www.ctan.org/pkg/endfloat
%
% The IEEEtran \ifCLASSOPTIONcaptionsoff conditional can also be used
% later in the document, say, to conditionally put the References on a 
% page by themselves.




% *** PDF, URL AND HYPERLINK PACKAGES ***
%
%\usepackage{url}
% url.sty was written by Donald Arseneau. It provides better support for
% handling and breaking URLs. url.sty is already installed on most LaTeX
% systems. The latest version and documentation can be obtained at:
% http://www.ctan.org/pkg/url
% Basically, \url{my_url_here}.





% *** Do not adjust lengths that control margins, column widths, etc. ***
% *** Do not use packages that alter fonts (such as pslatex).         ***
% There should be no need to do such things with IEEEtran.cls V1.6 and later.
% (Unless specifically asked to do so by the journal or conference you plan
% to submit to, of course. )


% correct bad hyphenation here
\hyphenation{op-tical net-works semi-conduc-tor}


\begin{document}
%
% paper title
% Titles are generally capitalized except for words such as a, an, and, as,
% at, but, by, for, in, nor, of, on, or, the, to and up, which are usually
% not capitalized unless they are the first or last word of the title.
% Linebreaks \\ can be used within to get better formatting as desired.
% Do not put math or special symbols in the title.
\title{Benchmarking and Analyzing Generative Data for Visual Recognition}
%
%
% author names and IEEE memberships
% note positions of commas and nonbreaking spaces ( ~ ) LaTeX will not break
% a structure at a ~ so this keeps an author's name from being broken across
% two lines.
% use \thanks{} to gain access to the first footnote area
% a separate \thanks must be used for each paragraph as LaTeX2e's \thanks
% was not built to handle multiple paragraphs
%
%
%\IEEEcompsocitemizethanks is a special \thanks that produces the bulleted
% lists the Computer Society journals use for "first footnote" author
% affiliations. Use \IEEEcompsocthanksitem which works much like \item
% for each affiliation group. When not in compsoc mode,
% \IEEEcompsocitemizethanks becomes like \thanks and
% \IEEEcompsocthanksitem becomes a line break with idention. This
% facilitates dual compilation, although admittedly the differences in the
% desired content of \author between the different types of papers makes a
% one-size-fits-all approach a daunting prospect. For instance, compsoc 
% journal papers have the author affiliations above the "Manuscript
% received ..."  text while in non-compsoc journals this is reversed. Sigh.

\author{Bo Li, Haotian Liu, Liangyu Chen, Yong Jae Lee \protect\\ Chunyuan Li, Ziwei Liu
% <-this % stops a space
\IEEEcompsocitemizethanks{\IEEEcompsocthanksitem Bo Li, Liangyu Chen and Ziwei Liu are with the S-Lab, Nanyang Technological University \protect \\
% note need leading \protect in front of \\ to get a newline within \thanks as
% \\ is fragile and will error, could use \hfil\break instead.
E-mail: \{libo0013, liangyu.chen, ziwei.liu\}@ntu.edu.sg
\IEEEcompsocthanksitem Haotian Liu and Yong Jae Lee are with University of Wisconsin-Madison.
E-mail: \{lht, yongjaelee\}@cs.wisc.edu
\IEEEcompsocthanksitem Chunyuan Li is with Microsoft Research, Redmond. \protect \\
E-mail: chunyuan.li@microsoft.com}}

% <-this % stops an unwanted space
% \thanks{Manuscript received April 19, 2005; revised August 26, 2015.}}

% note the % following the last \IEEEmembership and also \thanks - 
% these prevent an unwanted space from occurring between the last author name
% and the end of the author line. i.e., if you had this:
% 
% \author{....lastname \thanks{...} \thanks{...} }
%                     ^------------^------------^----Do not want these spaces!
%
% a space would be appended to the last name and could cause every name on that
% line to be shifted left slightly. This is one of those "LaTeX things". For
% instance, "\textbf{A} \textbf{B}" will typeset as "A B" not "AB". To get
% "AB" then you have to do: "\textbf{A}\textbf{B}"
% \thanks is no different in this regard, so shield the last } of each \thanks
% that ends a line with a % and do not let a space in before the next \thanks.
% Spaces after \IEEEmembership other than the last one are OK (and needed) as
% you are supposed to have spaces between the names. For what it is worth,
% this is a minor point as most people would not even notice if the said evil
% space somehow managed to creep in.



% The paper headers
\markboth{Journal of \LaTeX\ Class Files,~Vol.~14, No.~8, August~2015}%
{Shell \MakeLowercase{\textit{et al.}}: Bare Demo of IEEEtran.cls for Computer Society Journals}
% The only time the second header will appear is for the odd numbered pages
% after the title page when using the twoside option.
% 
% *** Note that you probably will NOT want to include the author's ***
% *** name in the headers of peer review papers.                   ***
% You can use \ifCLASSOPTIONpeerreview for conditional compilation here if
% you desire.



% The publisher's ID mark at the bottom of the page is less important with
% Computer Society journal papers as those publications place the marks
% outside of the main text columns and, therefore, unlike regular IEEE
% journals, the available text space is not reduced by their presence.
% If you want to put a publisher's ID mark on the page you can do it like
% this:
%\IEEEpubid{0000--0000/00\$00.00~\copyright~2015 IEEE}
% or like this to get the Computer Society new two part style.
%\IEEEpubid{\makebox[\columnwidth]{\hfill 0000--0000/00/\$00.00~\copyright~2015 IEEE}%
%\hspace{\columnsep}\makebox[\columnwidth]{Published by the IEEE Computer Society\hfill}}
% Remember, if you use this you must call \IEEEpubidadjcol in the second
% column for its text to clear the IEEEpubid mark (Computer Society jorunal
% papers don't need this extra clearance.)



% use for special paper notices
%\IEEEspecialpapernotice{(Invited Paper)}



% for Computer Society papers, we must declare the abstract and index terms
% PRIOR to the title within the \IEEEtitleabstractindextext IEEEtran
% command as these need to go into the title area created by \maketitle.
% As a general rule, do not put math, special symbols or citations
% in the abstract or keywords.
\IEEEtitleabstractindextext{%
\begin{abstract}
Advancements in large pre-trained generative models have expanded their potential as effective data generators in visual recognition. This work delves into the impact of generative images, primarily comparing paradigms that harness external data (\ie generative \vs retrieval \vs original). Our key contributions are:
\textbf{1) GenBench Construction:} We devise \textbf{GenBench}, a broad benchmark comprising 22 datasets with 2548 categories, to appraise generative data across various visual recognition tasks.
\textbf{2) CLER Score:} To address the insufficient correlation of existing metrics (\eg, FID, CLIP score) with downstream recognition performance, we propose \textbf{CLER}, a training-free metric indicating generative data's efficiency for recognition tasks prior to training.
\textbf{3) New Baselines:} Comparisons of generative data with retrieved data from the same external pool help to elucidate the unique traits of generative data.
\textbf{4) External Knowledge Injection:} By fine-tuning special token embeddings for each category via Textual Inversion, performance improves across 17 datasets, except when dealing with low-resolution reference images. Our exhaustive benchmark and analysis spotlight generative data's promise in visual recognition, while identifying key challenges for future investigation. Our code is available at: \href{https://github.com/Luodian/Genbench}{GenBench}.
\end{abstract}

% Note that keywords are not normally used for peerreview papers.
\begin{IEEEkeywords}
Visual Recognition, Generative Models, Data-Centric AI
\end{IEEEkeywords}}


% make the title area
\maketitle


% To allow for easy dual compilation without having to reenter the
% abstract/keywords data, the \IEEEtitleabstractindextext text will
% not be used in maketitle, but will appear (i.e., to be "transported")
% here as \IEEEdisplaynontitleabstractindextext when the compsoc 
% or transmag modes are not selected <OR> if conference mode is selected 
% - because all conference papers position the abstract like regular
% papers do.
\IEEEdisplaynontitleabstractindextext
% \IEEEdisplaynontitleabstractindextext has no effect when using
% compsoc or transmag under a non-conference mode.



% For peer review papers, you can put extra information on the cover
% page as needed:
% \ifCLASSOPTIONpeerreview
% \begin{center} \bfseries EDICS Category: 3-BBND \end{center}
% \fi
%
% For peerreview papers, this IEEEtran command inserts a page break and
% creates the second title. It will be ignored for other modes.
\IEEEpeerreviewmaketitle
\section{Introduction}
Large-scale pretrained generative models~\cite{rombach2022high,nichol2021glide,ramesh2022hierarchical,saharia2022photorealistic} have advanced the synthesis of realistic images, opening possibilities for improving visual recognition with generative data. Some studies~\cite{he2022synthetic,bansal2023leaving,shipard2023boosting} have explored the use of generative data for training visual systems with positive results. In addition, the development of cost-efficient and controllable generative models~\cite{zhang2023adding,li2023gligen,hu2021lora,chen2022analog,song2020denoising,kong2021fast,blattmann2022retrieval} has gained more and more attention, offering the potential for richer and more cost-effective external data. To determine whether large pre-trained generative models can serve as efficient data generators and to understand the unique characteristics of generative data in comparison to human-annotated and retrieval data (as depicted in \cref{fig:teaser}), a comprehensive study is required.

% Figure environment removed

To address it, we introduce \gb, a benchmark with 22 datasets and 2548 categories, to evaluate the effectiveness of generative data in a broad range of scenarios. Formally, we measure the quality of generative data on \gb by its test accuracy after linear probing on CLIP~\cite{radford2021learning} model and compare it with other types of external data. The datasets on \gb are grouped into three, namely \common, \finegrained, and \rare~concepts, facilitating our analysis of the generative model's generation capabilities on different types of data.

% Figure environment removed

To validate the quality of generative images as well as other types of external data, we propose the Class-centered Recognition (\textbf{CLER}) score. It is directly correlated with the test accuracy from linear probing CLIP~\cite{radford2021learning} and is an efficient training-free measure for assessing the \textit{improvements} of external data.

In~\cref{sec:analysis}, we provide a detailed analysis of generative data. In~\cref{subsec:tradeoffs}, we compare the performance and cost (in USD) of different data types with a maximum of 500 shots per category, totaling over 1M images. Our results in~\cref{fig:scaling_effect} show that generative data is cost-effective and performs well for common concepts. However, it offers no apparent advantage over original data for fine-grained and rare concepts. 

% Combining~\cref{fig:rel_improvement} and~\cref{fig:scaling_effect}, we demonstrate the advantages of generative data in comparison to CLIP's strong zero-shot baseline performance on various concepts. ~\cref{fig:rel_improvement} reveals that while generative data may not significantly enhance CLIP's performance on common concepts, it exhibits distinct benefits concerning fine-grained concepts and rare concepts that present challenges for CLIP's effective transfer learning. ~\cref{fig:scaling_effect} offers valuable insights into the scaling effect of different types of external data. Notably, generative data emerges as highly advantageous and cost-effective when compared to both original data and retrieval data. The incorporation of generative data showcases its potential to bolster model performance effectively. However, it is important to acknowledge the limitations associated with rare concepts, as they are infrequently represented in stable diffusion's training data, thereby impacting the efficacy of generative data on such concepts.

Our integration of~\cref{fig:rel_improvement} and~\cref{fig:scaling_effect} illustrates generative data's superiority over CLIP's zero-shot baseline in varied conceptual realms. ~\cref{fig:rel_improvement} suggests that although generative data scarcely improves CLIP's aptitude for common concepts, it significantly bolsters its handling of nuanced and rare concepts. ~\cref{fig:scaling_effect} emphasizes generative data's cost-effectiveness and advantage over both original and retrieval data. Notwithstanding, limitations surface with rare concepts due to their sparse representation in stable diffusion's training data, impacting the generative data's efficacy.

We then conducted an examination to delineate the possible scenarios for generative data usage and the particular reasons for its infeasibility in certain circumstances. We first delved deeper into the susceptibility of present text-to-image generative models to different prompt strategies, as detailed in ~\cref{subsec:exp_prompt_strategy}. The evidence unearthed highlights the imperative for custom prompt strategies tailored to specific datasets due to the variability of the optimal strategy. Then our investigation in ~\cref{subsec:domain_gap} shines a light on the potential relationship between the amplification effect of generative data and the average text resemblance between the dataset and the pre-training dataset of generative models. We hypothesize the existence of an inherent bias in generative models, which favors categories that either frequently occur or display high query similarity in their pre-training dataset.

Aiming for the augmentation of generative data, we incorporated external knowledge into pre-trained models by fine-tuning specialized token embeddings for each dataset category via Textual Inversion ~\cite{gal2022image}. Upon injecting varied types of reference data, a consistent performance improvement was observed across most of the 17 datasets examined. Despite these advances, certain limitations were observed that the method did not improve generative data performance on specific datasets with only low-resolution reference images.

In summary, our contributions are as follows:
\vspace{-\topsep}
\begin{enumerate}
  \setlength{\parskip}{1pt}
  \setlength{\itemsep}{0pt plus 1pt}
    \item[(1)] A benchmark, \gb, that comprehensively evaluates the benefits of generative data across a diverse range of visual concepts.
    \item[(2)] A analytical training-free metric \textbf{CLER} score for fast evaluation of the image quality of different types of external data.
    \item[(3)] A detailed examination on the potential applications for generative data, along with the explicit rationales for its infeasibility in specified situations.
    \item[(4)] A comprehensive investigation into the effectiveness of generative data through the injection of external knowledge into pretrained generative models.
\end{enumerate}
\vspace{-\topsep}

% Figure environment removed
\section{On the Evaluation of Generative Data}
\label{sec:genbench}
In this section, we investigate how generative data could improve pre-trained visual recognition models on a wider range of downstream datasets, with \gb as our evaluation benchmark.

\subsection{Datasets}
Inherited from existing benchmarks such as Elevater~\cite{li2022elevater} and VTAB~\cite{zhai2019large}, \gb encompasses a total of 22 datasets with a broad spectrum of visual recognition concepts consisting of 2548 categories. These datasets are categorized into the following three main groups. More statistical information on these datasets is listed in the appendix.

\noindent \textbf{\common~Concepts} mainly cover generic objects that are commonly seen in daily life and in the Internet data. In this group, we can evaluate whether the generative models could cover those most common categories effectively. This group includes: ImageNet-1K~\cite{russakovsky2015imagenet}, CIFAR-10~\cite{krizhevsky2009learning}, CIFAR-100~\cite{krizhevsky2009learning}, Caltech-101~\cite{fei2004learning}, VOC-2007~\cite{everingham2009pascal}, MNIST~\cite{lecun1998gradient} and SUN-397~\cite{xiao2010sun}. 


\noindent \textbf{\finegrained~Concepts} cover subcategories within a meta-category, such as different breeds of dogs within the dog category or different models of airplanes within the airplane category.
In this group, we can assess the ability of generative models to capture the subtle visual differences between different subcategories and their capacity to generate objects with more specialized names. This task can help identify the limitations of generative models in capturing the complexity and variability of fine-grained object categories. This group includes: Food-101~\cite{bossard2014food}, Oxford Pets~\cite{parkhi2012cats}, Oxford Flowers~\cite{nilsback2008automated}, Standford Cars~\cite{krause20133d}, FGVC Aircraft~\cite{maji2013fine}, Country-211~\cite{radford2021learning}. 

\noindent \textbf{\rare~Concepts} covers less common objects in the real world, such as medical images (e.g., Patch-Camelyon~\cite{veeling2018rotation}) and remote sensing images (e.g., RESISC-45~\cite{cheng2017remote}, EuroSAT~\cite{helber2017eurosat}). Obtaining such data can be challenging, making it important to assess whether generative models can effectively synthesize this type of data, which presents a unique challenge to computer vision research. This group includes PatchCamelyon~\cite{veeling2018rotation}, EuroSAT~\cite{helber2017eurosat}, GTSRB~\cite{stallkamp2011german}, Rendered-SST2~\cite{radford2021learning}, FER 2013~\cite{fer2013}, RESISC-45~\cite{cheng2017remote}, Hateful Memes~\cite{kiela2020hateful}, Describable Textures~\cite{cimpoi2014describing}, and KITTI Distance~\cite{fritsch2013new}.

\subsection{External Data} 
\noindent \textbf{Generative Data.} Although there exists various large-scale pretrained text-to-image generative models~\cite{saharia2022photorealistic,ramesh2022hierarchical}. Our investigation in ~\gb is centered around two readily available and accessible generative models, namely GLIDE ~\cite{nichol2021glide} and Stable Diffusion 2.1~\cite{rombach2022high}. We use the names of categories in the dataset and combine them with various prompt strategies in~\cref{subsec:prompt_strategy} as textual prompts to generative model and obtain corresponding sampled images. If the number of required generated images exceeds the initial prompts, we randomly augment the prompt list to reach the required number.
% We choose the prompt strategies to be repeated based on the number of images required to be generated for each category.

\noindent \textbf{Retrieval Data.} Similar to retrieval data, we utilize category names along with defined templates specific to each dataset as retrieval queries, without employing additional prompt strategies as used in generative data. We employ text-to-text retrieval on LAION-400M to obtain the top-$k$ results using these queries and select the required number of images based on their similarity to each result. This approach aligns with generative data acquisition to facilitate our evaluation and analysis of data quality. It may differ from other retrieval-based works~\cite{liu2023react}, which we will discuss in \cref{sec: ext_data}. 

\noindent \textbf{Original Data.}  It refers to the training data of each individual dataset. This data type is considered the highest quality among all different types of external data and serves as the upper bound when evaluating on the same amount of shots.
However, it is also the most expensive to acquire.


To compare the cost-effectiveness of different types of data, we also analyze the cost of external data. For generative and retrieval data, it is calculated based on the time needed for sampling generative models or performing top-$k$ text-to-text search and the cost of running a CPU/GPU instance on Azure\footnote{\href{https://instances.vantage.sh/azure/}{Azure Pricing}}. For original data, we provide a reference value for human labeling cost obtained from Amazon Turk\footnote{\href{https://aws.amazon.com/sagemaker/data-labeling/pricing/}{Amazon Turk}}.
We list the estimated cost for each type of external data in ~\cref{tab:data_collection}.

% Overall, the acquisition methods and the corresponding cost per image for each type of external data are listed in ~\cref{tab:data_collection}.

\begin{table}[ht]
\centering
\caption{Data acquisition cost for different types of external data (measured in US dollars per image). See appendix for details.}
\label{tab:data_collection}
\resizebox{\columnwidth}{!}{%
\setlength{\tabcolsep}{10pt}
\renewcommand{\arraystretch}{1.2}
\begin{tabular}{c|c|c}
\toprule
\rowcolor{COLOR_MEAN}
\textbf{Data Type} & \textbf{Collection Source} & \textbf{Est. Cost / Image} \\ \midrule
Generative Data & Model Inference & $2.54 \times 10^{-4}$ USD \\
Retrieval Data & Web/Database Query & $3.93 \times 10^{-5}$ USD \\
Original Data & Human Label & $1.20 \times 10^{-2} $ USD \\ \bottomrule
\end{tabular}%
}
\end{table}

% \textcolor{red}{Bo: definition of original data}
% Recently, REACT~\cite{liu2023react} proposes to learn customized visual models by finetuning pretrained vision-language models on retrieved data from large image-text database according to task definition. There are two major distinctions between REACT and our approach: (1) REACT finetunes pretrained vision language models on the retrieved data, while our CLER score does not require finetuning; (2) REACT directly uses the retrieved data for customizing visual models, while our approach can use the retrieved data as image prompts for the generative model and prompt for new data for classification.

% \footnote{\href{https://github.com/openai/glide-text2im}{OpenAI/GLIDE}}
% \footnote{\href{https://github.com/huggingface/diffusers/tree/main/examples/text_to_image}{Diffusers/StableDiffusion}}

\subsection{Prompt Strategy} 
\label{subsec:prompt_strategy}
Initially, we convert the category name into a textual prompt and feed it into the generative models to produce the corresponding image. The different specific descriptions added to the prompt to generate better images are referred to as prompt strategies. In our benchmark, we use several prompt strategies, which we present below:

\noindent \textbf{Simple Template (ST)} refers the basic prompt format consisting of \textit{a photo of \{\}}, where the category name is inserted into the brackets. 

\noindent \textbf{Defined Template (DT)} refers using simple category names as prompts, we also followed  the practice of CLIP\footnote{\href{https://github.com/openai/CLIP/blob/main/data/prompts.md}{CLIP Prompts}}, by defining dataset-specific prompt formats for each dataset. For example, in FVGC-Aircraft dataset, we extend the simple prompt with a description \textit{a type of aircraft}. 

\noindent \textbf{Category Enhancement (CE)} was introduced in~\cite{he2022synthetic}, where an off-the-shelf word-to-sentence T5~\cite{raffel2020exploring} model is utilized to expand each category into a complete sentence, such as expanding \textit{airplane} into \textit{A large, sleek, white airplane with dual engines is soaring through the clear blue skies, leaving behind a trail of white clouds.}. In the appendix, we will provide examples of generated sentences for different datasets.

\noindent \textbf{Restrictive Description (RD)} are additional, specific phrases added to the prompt in order to guide the generative model to produce images of higher quality. Examples of such phrases include \textit{hi-res, highly detailed, sharp focus}. Additionally, some special restrictive symbols, such as enclosing the category with parentheses, \eg \textit{((airplane))}, can be added to the prompt to make the generative model more focused on the category rather than other descriptive words. These ideas mainly come from the Stable Diffusion community, and the results show that this approach can lead to better generation results, but there is still no rigorous conclusion.

\noindent \textbf{Negative Prompts (NP)} are input arguments that guide the Stable Diffusion model to deliberately exclude particular objects, styles, or abnormalities from the generated image. This significant feature empowers users to eliminate undesired or redundant elements from the final output. Moreover, there are several quality constraints that are applicable to all datasets, such as restricting words like \textit{bad shape} and \textit{misfigured} to prevent the generative model from producing low-quality data.

% \noindent \textbf{Retrieval Augmented (RA)} refers to obtaining few-shot retrieval images from a retrieval system (specified in \cref{subsec:tradeoffs}), and using them as image prompts for the generative model. These images will serve as a starting point for the generative model's sampling process to better guide the model in generating images in combination with text prompts. This strategy is similar to TIG and we specify it in \cref{sec:tag}.

All prompt strategies aforementioned will be presented with more details for each dataset in the appendix.

\subsection{Evaluation Protocol}

We aim to measure the \textit{improvement} of generative data over the baseline zero-shot accuracy for a diverse range of categories. However, existing metrics such as FID primarily assess the aesthetic quality and visual coherence. While current diffusion-based generative models excel in these metrics, they are inadequate for evaluating generative data's downstream task quality. Typically, the accuracy of a model on test data is used to measure this improvement. Our experiments primarily use CLIP VIT/B-32 as the baseline visual recognition model.
% , and leave the results with larger-scale CLIP models in the appendix.

\begin{algorithm}[t]
\caption{Pseudocode of CLER in PyTorch style.}
\label{alg:code}
\algcomment{\fontsize{7.2pt}{0em}\selectfont \texttt{mm}: matrix multiplication; \texttt{T}: transpose.
%\vspace{-1.em}
}
\definecolor{codeblue}{rgb}{0.25,0.5,0.5}
\lstdefinestyle{python}{
  language=python,
  morekeywords={mm,encode_image,encode_text,group,mean,argmax},
  keywordstyle={\fontsize{7.2pt}{7.2pt}\selectfont\textcolor{blue!90!black}}
}
\lstset{
  backgroundcolor=\color{white},
  basicstyle=\fontsize{7.2pt}{7.2pt}\ttfamily\selectfont,
  columns=fullflexible,
  breaklines=true,
  captionpos=b,
  commentstyle=\fontsize{7.2pt}{7.2pt}\color{codeblue},
  mathescape
}
\begin{lstlisting}[style=python]
def CLER_score(gen_images, gen_labels, test_images, test_labels, class_names):
    n_classes = len(class_names)
    gen_emb = CLIP.encode_image(gen_images) $~~$# [M,F]
    test_emb = CLIP.encode_image(test_images) $$# [N,F]
    text_emb = CLIP.encode_text(class_names) $~$# [C,F]
    gen_emb_grp = group(gen_emb, gen_labels) $~$# [C,K,F]
    class_centers = gen_emb_grp.mean(dim=1) $~~$# [C,F]

    preds_CLER = mm(test_emb, class_centers.T)
    preds_CLIP = mm(text_emb, text_features.T)
    preds_ensemble = (preds_CLER + preds_CLIP) / 2

    CLER = sum(preds_CLER.argmax(dim=1) == test_labels)
    CLER_ensemble = sum(preds_ensemble.argmax(dim=1) == test_labels)
\end{lstlisting}
\end{algorithm}

Motivated by CLIP zero-shot evaluation, we propose a training-free metric, Class-centered Recognition (CLER) score, an approximation to the improvement of generative data over a given downstream task. The core idea is to replace the averaged language embeddings of each class in CLIP zero-shot evaluation, with the averaged image embeddings from each class in the target downstream dataset. Specifically, given a set of images $\mathcal{I}$ with corresponding class labels $\mathcal{C}$, a downstream dataset $\mathcal{D}$ containing $C$ classes, CLER score measures the improvement $\mathcal{I}$ can bring to the classification task on $\mathcal{D}$.

First, we extract embeddings of images $\mathcal{I}$ using CLIP $\mathcal{E}_i = \text{CLIP}(\mathcal{I}_i)$.
Then, we group embeddings $\mathcal{E}$ according to labels $\mathcal{T}$, and compute the average embedding within each group: $\mathcal{M}_j = \text{average}(\{\mathcal{E}_i; \mathcal{T}_i = j\})$, where $1~\leq j \leq~C$.
For each test image $\mathcal{D}_k$, we obtain its image embedding $\mathcal{E}_k$ and class prediction $\mathcal{P}_k = \arg\max_{j}(\mathcal{E}_k^\top \cdot \mathcal{E}_j)$.  Then we calculate the mean accuracy of class prediction $\mathcal{P}_k$ for each test image in \cref{eq:cler_score}, where $\mathcal{L}_k$ is the ground truth label of the test image $\mathcal{D}_k$ and $\mathds{1}$ is the indicator function that returns $1$ if $\mathcal{P}_k = \mathcal{L}_k$ or $0$ otherwise.
\begin{equation}
    \text{CLER} = \frac{1}{N} \sum^{1}_{N}\mathds{1}(\mathcal{P}_k = \mathcal{L}_k)
    \label{eq:cler_score}
\end{equation}
 If we specify the CLER score evaluation with the CLIP model,  we can further enhance to a more accurate version, by ensembling the prediction with CLIP contrastive prediction: $\hat{\mathcal{P}}_k = \arg\max_{j}(\mathcal{E}_k^\top \cdot (\mathcal{E}_j + \mathcal{L}_j) / 2)$, where $\mathcal{L}_j$ is the text embeddings obtained from CLIP text encoder for class $j$. We provide the pseudo code for CLER score in ~\cref{alg:code}.
 
 In our experiments and analysis, we primarily use CLIP ViT-B/32 as the recognition model for evaluation. We will also discuss the performance of finetuning the CLIP model on generative data, as well as the evaluation of other backbones (\eg ResNet, larger ViTs, \etc) in the appendix.

% Initially, we focus on conducting experiments and analyzing the results using the CLIP VIT/B-32 model. We generated up to 500 images per category and assessed their accuracy by training a linear classifier on a frozen CLIP model and evaluating its performance on the same test set. 

% Figure environment removed

We assess the correlation between CLER score, CLIP score, and FID with linear probing accuracy using a correlation analysis on 18 datasets with 100-shot generative data per category, excluding four datasets with a small number of categories. Results in~\cref{fig:metric_correlation} show that CLER score has the best positive correlation with linear probing accuracy, followed by CLIP score. FID, which is often used to assess image quality, does not exhibit a clear correlation with linear probing accuracy. \textcolor{black}{Moreover, it is important to note that CLER computes dot products between cluster centers, significantly reducing complexity compared to computing dot products for individual images. Consequently, evaluating the CLER score is considerably more cost-effective than obtaining linear probing accuracy, which requires tuning models on generated data, as demonstrated in~\cref{fig:efficient_evaluation}.}

In addition, CLER score can be employed to assess the performance improvement of other types of data, such as original training data or retrieval data from the Internet. In the appendix, we provide a correlation analysis for original and retrieval data. 

Given the advantages of CLER score, we primarily rely on this metric in subsequent analyses, it enables us to efficiently draw indicative conclusions.

% Figure environment removed

\begin{table*}[tp]
\centering
\caption{Results of GLIDE and Stable Diffusion, along with various corresponding strategies, on the \textbf{GenBench} dataset, are presented in the table. We report the CLER scores of the 20-shot generated images on each dataset. For simplicity, we use the following abbreviations: ST for Simple Templates, CE for Category Enhancement, DT for Defined Template, NP for Negative Prompts, RD for Restrictive Descriptions, RA for Retrieval Augmented. We denote the \colorbox{best}{best} and \colorbox{second_best}{second best} with specified colors.}
\label{tab:strategy_comparison}
\resizebox{\textwidth}{!}{%
\setlength{\tabcolsep}{5pt}
\renewcommand{\arraystretch}{2}
\begin{tabular}{c|c|cccccccccccccccccccccc|c}
\toprule
\rowcolor{COLOR_MEAN}
\multicolumn{1}{c|}{\rot{\textbf{Model}}} & \rot{\textbf{Prompt Strategy}} & \rot{\textbf{ImageNet-1K}} & \rot{\textbf{Caltech-101}} & \rot{\textbf{CIFAR-10}} & \rot{\textbf{CIFAR-100}} & \rot{\textbf{Country-211}} & \rot{\textbf{Desc. Textures}} & \rot{\textbf{EuroSAT}} & \rot{\textbf{FER-2013}} & \rot{\textbf{FGVC-Aircraft}} & \rot{\textbf{Food-101}} & \rot{\textbf{GTSRB}} & \rot{\textbf{Hateful Memes}} & \rot{\textbf{KITTI Distance}} & \rot{\textbf{MNIST}} & \rot{\textbf{Oxford Flowers}} & \rot{\textbf{Oxford-IIIT Pets}} & \rot{\textbf{PatchCamelyon}} & \rot{\textbf{Rendered-SST2}} & \rot{\textbf{RESISC-45}} & \rot{\textbf{Stanford Cars}} & \rot{\textbf{SUN-397}} & \rot{\textbf{VOC-2007}} & \rot{\textbf{Mean}} \\ \midrule
\multirow{3}{*}{GLIDE~\cite{nichol2021glide}} & ST & 43.66 & 78.75 & 85.70 & 45.54 & 11.07 & 22.07 & \colorbox{best}{51.36} & \colorbox{second_best}{35.30} & 15.86 & 71.33 & 18.05 & \colorbox{best}{58.61} & 37.27 & 18.98 & 62.23 & 64.24 & \colorbox{best}{62.53} & 52.94 & 49.52 & 51.20 & 49.95 & 80.34 & \textbf{48.48} \\
 & CE & 57.35 & 81.35 & 82.35 & 47.83 & 10.16 & 29.84 & 40.48 & 22.82 & 13.35 & 70.74 & 10.48 & 51.87 & \colorbox{best}{43.32} & 16.29 & 49.15 & 56.89 & \colorbox{second_best}{61.98} & 54.97 & 48.87 & 45.27 & \colorbox{best}{58.57} & 78.62 & \textbf{46.93} \\ 
 & DT & 57.55 & 85.38 & 88.64 & 57.64 & 13.17 & 36.38 & 41.64 & 23.79 & \colorbox{best}{18.96} & 79.58 & 17.75 & \colorbox{second_best}{54.62} & 40.79 & 15.91 & \colorbox{second_best}{65.66} & \colorbox{best}{83.96} & 58.36 & \colorbox{second_best}{55.57} & \colorbox{best}{54.87} & 54.33 & \colorbox{second_best}{58.56}  & 81.73 & \colorbox{second_best}{\textbf{52.04}} \\ \hline
\multirow{6}{*}{Stable Diffusion~\cite{rombach2022high}} & ST & 47.90 & 85.79 & 85.20 & 58.63 & 13.34 & 38.14 & 48.42 & 19.17 & 15.09 & \colorbox{best}{82.39} & 20.99 & 46.05 & 8.30 & 19.80 & 64.87 & 77.41 & 51.72 & 47.83 & 51.31 & 58.35 & 54.35 & 79.20 & \textbf{48.83} \\
 & CE & 60.04 & 81.30 & 88.22 & 56.09 & 11.27 & 30.69 & \colorbox{second_best}{50.00} & 29.81 & 13.24 & 73.95 & 8.73 & 50.67 & 24.47 & 22.15 & 45.95 & 57.01 & 54.70 & 50.58 & 49.15 & 54.11 & 55.90 & 80.97 & \textbf{47.68} \\
 & DT & \colorbox{second_best}{61.07} & 86.31 & 88.90 & 60.19 & 13.85 & \colorbox{best}{46.12} & 40.72 & 27.81 & 16.30 & \colorbox{second_best}{82.29} & \colorbox{best}{30.09} & 49.02 & 9.14 & 19.70 & \colorbox{best}{66.13} & \colorbox{second_best}{83.36} & 53.09 & 50.08 & 52.73 & \colorbox{second_best}{58.45} & 57.48 & \colorbox{best}{82.35} & \textbf{51.60} \\
 & w/ NP & 60.19 & \colorbox{second_best}{86.70} & \colorbox{second_best}{89.43} & \colorbox{second_best}{62.95} & \colorbox{best}{14.25} & 44.36 & 35.70 & 28.11 & 15.95 & 81.64 & 22.03 & 44.75 & 22.08 & 17.49 & 63.31 & 80.87 & 50.54 & 53.27 & 52.34 & 58.41 & 57.55 & \colorbox{second_best}{82.32} & \textbf{51.10} \\
 & w/ NP, RD & \colorbox{best}{61.45} & \colorbox{best}{87.26} & \colorbox{best}{89.22} & \colorbox{best}{64.90} & \colorbox{second_best}{14.14} & \colorbox{second_best}{45.69} & 27.94 & 27.42 & 15.23 & 80.99 & 22.03 & 43.30 & 37.27 & \colorbox{second_best}{26.22} & 64.89 & 80.10 & 53.04 & 55.41 & 50.83 & 57.29 & 57.43 & 82.16 & \textbf{52.01} \\  
% & w/ RA & 59.37 & 84.15 & 87.17 & 57.92 & 12.19 & 40.74 & 44.46 & \colorbox{best}{38.17} & \colorbox{second_best}{17.55} & 68.56 & \colorbox{second_best}{25.82} & 43.10 & \colorbox{second_best}{38.82} & \colorbox{best}{29.68} & 63.93 & 74.79 & 53.13 & \colorbox{best}{55.74} & \colorbox{second_best}{54.11} & \colorbox{best}{58.58} & 57.92 & 79.66 & \colorbox{best}{\textbf{52.07}} \\ 
\bottomrule
% \hline
% \rowcolor{ROW_COLOR}
% \multicolumn{2}{c|}{Max.} &  & 87.26 & 82.35 & 64.90 & 13.34 & 46.12 & 49.90 & 28.11 & 16.30 & 82.39 & 30.09 & 51.87 & 43.32 & 26.22 & 66.13 & 83.36 & 61.98 & 55.41 & 52.73 & 58.45 &  & 82.35 \\ 
\end{tabular}%
}
\end{table*}
\section{Systematic Analysis on Generative Data}
\label{sec:analysis}
In this section, we analyze: \textbf{(1)} the effectiveness and cost trade-offs of generative data compared to other external data, \textbf{(2)} the performance of different generative models with various prompt strategies on a wide range of categories, and \textbf{(3)} the advantages and disadvantages of using generative data on different datasets and their reasons.

\subsection{Trade-offs between Performance and Cost}
\label{subsec:tradeoffs}
Data efficiency refers to the amount of labeled data required for a model to achieve a desired performance. While more data generally improves recognition accuracy and generalization ability, it also means higher annotation costs. Thus, obtaining higher-quality training data through efficient means is a goal of ML practitioners.

With the information of estimated cost per image in~\cref{tab:data_collection}, in~\cref{fig:scaling_effect}, we demonstrate the impact of adding different amount of external data on performance across different concept groups. This figure provides a detailed breakdown and suggests that there are significant differences in the scaling effect across different concept groups. 

For example, in the \common~concepts, the gap between original and generation is gradually reduced with an increasing amount of data per category. Meanwhile, generative data also significantly outperforms the usage of retrieval data. From the cost perspective, as the amount of generative data increases, the advantage of using generative data over original data becomes more significant in \common~concepts. As the figure suggests, obtaining 500-shot original data on \common~concepts costs around 10k USD while generating the same amount of data would only cost 208 USD. The cost difference between the two is around 9,800 USD, but the performance gap is not significant.

However, in the other two groups, \finegrained~and \rare~concepts, using original data is still advantageous, neither generative data nor retrieval data can reach the performance of original data in these concepts.

% why failed in providing analysis on proxy-a-distance.

\subsection{Performance of Various Prompt Strategies}
\label{subsec:exp_prompt_strategy}
We carried out a comprehensive assessment of the generative data obtained from GLIDE and Stable Diffusion using various prompt strategies on 22 datasets. Table~\ref{tab:strategy_comparison} presents the CLER scores of the 20-shot generative data obtained from different prompt strategies with GLIDE and Stable Diffusion on 22 datasets. 

Our findings indicate that while Stable Diffusion 2.1 with dataset-specific prompts generally yielded better results, the optimal strategy varied across different datasets. Therefore, it is important to consider prompt strategy on a case-by-case basis, as there is no universal optimal strategy. To summarize, our observations are as follows:

% This difference in performance may be attributed to the varying capability of the generative models to generate images of different categories. In some object-centric datasets such as CIFAR-10 and ImageNet-1k, we found that the restrictive description strategy was particularly effective. By adding limiting words such as \textit{sharp focus} the generative model was better able to generate object-focused images. 

First, Stable Diffusion and GLIDE perform comparably well with a simple prompt strategy. Despite Stable Diffusion using a larger pretraining dataset and being considered to create higher-quality images, the ability on creating useful generative data for downstream tasks is not much more significant than GLIDE.


Second, the category enhancement strategy in~\cite{he2022synthetic} may result in performance degradation when evaluated on more datasets due to the introduced noise from expanding a category name into a sentence containing other categories. This approach can enhance data diversity but may lead to images corresponding to multiple concepts but a single label. Compared to GLIDE, Stable Diffusion may be better at handling this kind of noise because it incorporates Open CLIP's Text encoder, which has a better ability to attend to information related to the original category in the sentence.

% Second, the category enhancement strategy that was considered useful in~\cite{he2022synthetic} appears to result in some performance degradation in when evaluated on more datasets. This may be due to that simply expanding a category name into a sentence may result in the sentence containing other categories, leading to images that contain multiple concepts but still correspond to a single label. This approach can enhance data diversity to some extent, but it also introduces more noise. Compared to GLIDE, Stable Diffusion may be better at handling this kind of noise because it incorporates Open CLIP's Text encoder, which has better ability to attend to information related to the original category in the sentence. Compared to GLIDE, Stable Diffusion may be better at handling this kind of noise because it incorporates Open CLIP's Text encoder, which has better ability to attend to information related to the original category in the sentence.

Third, advanced prompt strategies can further enhance Stable Diffusion's generative ability. Negative prompts and restrictive descriptions both perform well on average. The latter is especially effective in object-centric datasets like CIFAR-10 and ImageNet-1K, where limiting words such as \textit{sharp focus} improve the model's ability to generate object-focused images.

Fourth, the retrieval augmented strategy improves performance for fine-grained and rare concept datasets such as FGVC-Aircraft, Stanford-Cars, and RESISC45. The categories in these datasets are less frequent in the pretraining dataset of the generative models. Retrieval images as image prompt provide more hints for the generative models to generate high-quality data for these rare categories.

Overall, the above analysis mainly stems from an observational perspective. In \cref{subsec:domain_gap}, we will quantitatively analyze the correlation between the CLER score of the generated data and the dataset-level mean text similarity on a specific dataset.

\subsection{Correlation with Domain Gap}
\label{subsec:domain_gap}
From Figure~\ref{fig:rel_improvement} and Figure~\ref{fig:scaling_effect}, it can be seen that the effectiveness of generative data varies across different groups. As these groups were defined based on empirical observations, in this section, we will further quantify the correlation between the performance improvement and the properties of each specific dataset.

We speculate that the capability of a generative model may differ across different categories, which mainly depends on whether the pre-training data it was trained on covers these categories and their related information. We consider measuring the capability of a generative model when facing different categories with {\it mean text similarity} (MTS). 

For a specific downstream dataset, we used text-to-text retrieval on LAION-400M, which is a subset of LAION-5B, to retrieve the top-$k$ results using category names as queries. Then, we computed the average similarity among these $k$ retrieval results to obtain the MTS for each category. The category-level MTS measures the similarity between the top-$k$ retrieval results obtained from LAION-400M and the category name query. Common categories such as \textit{airplane} tend to have higher MTS scores, whereas rare categories such as \textit{lymph node} tend to have significantly lower MTS scores. We averaged MTS scores for all categories within a dataset to obtain the dataset-level MTS. From~\cref{fig:mean_text_similarity}, we observed the positive correlation between the changes in CLER score and the dataset-level mean text similarity.

% Figure environment removed

\section{Injecting External Knowledge to Generative Models}
\label{sec:tag}

After observing the improvement that simple data generation can bring to various datasets, we remain interested in exploring whether a more effective method for generating higher quality data can be achieved through efficient fine-tuning using a small amount of data.

Building upon this, we consider injecting (1) retrieval data and (2) original data into generative model using Textual Inversion~\cite{gal2022image}. As shown in the conclusion drawn from \cref{fig:scaling_effect}, the cost of obtaining retrieval data is low, while the cost of acquiring original data is relatively high. Therefore, we consider conducting our experiments using a larger quantity of retrieval data and a limited number of original data.

We adopt the method of Textual Inversion~\cite{gal2022image} to finetune the token embeddings $V*$ (a continous vector representations) for each category using external data.
%
For example in aeroplane category from VOC-2007 dataset, for retrieval data, we retrieve 100 images related to \textit{aeroplane} from LAION-400M using text-to-text similarity matching. We then use BLIP-2~\cite{li2023blip} to caption these images, and calculate the similarity between the BLIP-2 generated captions and the word \textit{aeroplane} using BERT~\cite{devlin2018bert}. We select the subset of images whose similarity scores are above threshold of 0.8 (the number of remaining images may vary slightly for each category, typically ranging from 5-30 images). The data will be used to train the token embeddings specific to VOC-2007 aeroplane. During training, the conditional text prompt will follow the naming rule of \texttt{\textless|dataset  category|\textgreater} , etc. During sampling, we add defined prompts on those special tokens, forming the conditional sampling prompts \textit{a photo of \texttt{\textless|voc-2007 aeroplane|\textgreater}}.

The visual results are provided in \cref{fig:demo}. From a direct comparison between the generation results and Stable Diffusion direct sampling, the results generated after Textual Inversion fine-tuning are closer to the style of the reference data. This approach has the potential to gradually move the sampling space of Stable Diffusion towards the target reference data, thus obtaining data that is more similar to the downstream task.

In \cref{tab:add_external_data}, we evaluated 17 datasets from GenBench. It can be seen that adding retrieval data and original data through Textual Inversion finetuning in Stable Diffusion consistently improves performance across the 17 datasets. We highlighted the column for the datasets where the inclusion of external data further improved the performance. Due to space limit, we provide more detailed analysis in appendix by listing the visual examples of generated images for datasets with significant improvement. We also provide an analysis for why finetuning on original images was not able to improve model performance on low-resolution datasets (\eg for CIFAR-10). These results are presented to provide readers with a reference.

Moreover, Textual Inversion is a very lightweight fine-tuning method. We trained 1000 steps for each category, and the training process takes less than 10 minutes on average per category on a A100 GPU.

% Afterwards, we integrate the obtained token embeddings as external knowledge into a Stable Diffusion framework for sampling.


% As generative data's effectiveness is highly correlated with the mean text similarity (MTS) of datasets, it varies significantly in different scenarios, especially in those with large domain gaps. 

% To address this issue, we discuss a practical strategy in this section on how to adapt the generation model to produce images closer to downstream tasks.

% Previous works, like Textual Inversion \cite{gal2022image} and Dreambooth \cite{ruiz2022dreambooth}, fine-tune generative models with a small subset of downstream task images to guide the model to generate relevant images. However, this method requires thousands of training steps and showed limited applicability in our scenario with no significant improvements in our preliminary experiments. 

% Motivated by recent image-to-image generation techiniques~\cite{saharia2022palette,rombach2022high,bansal2023leaving}, we propose Target-Initialized Generation for generating images closer to the target task. Specifically, our method involves using a few target images $\mathbf{x}_{t}$ as the starting point for the generative model sampling process, which produces images that are closer to the target images. To achieve this, we first encode the target images with a pretrained encoder $\tau_{\theta}(\mathbf{x}_{t})$ and run them through a forward diffusion process for $T$ steps, resulting in an approximately normal distribution $\hat{z}_{T}(\mathbf{x}_{t})$. We then generate new images by denoising $\hat{z}_{T}(\mathbf{x}_{t})$ while conditioning on the category name. This approach is equivalent to using target images as the starting point for sampling in the generative model, resulting in the generated images being closer to the target images.

\begin{table*}[htp]
\centering
% \caption{Incorporating retrieval and original data.}
\caption{Analysis of injecting different types of external knowledge by finetuning special token embedding for each category through Textual Inversion across 17 datasets. The \colorbox{best}{entries} indicate improved performance while others indicate degraded performance.}
\label{tab:add_external_data}
\resizebox{\textwidth}{!}{%
\setlength{\tabcolsep}{5pt}
\renewcommand{\arraystretch}{2}
\begin{tabular}{c|cccc|ccccccccccccccccc}
\toprule
\rowcolor{COLOR_MEAN}
\textbf{Model} & \textbf{FT Data} & \textbf{FT Shot} & \textbf{Gen. Shot} & \textbf{Mean} & \rot{\textbf{Caltech-101}} & \rot{\textbf{Country-211}} & \rot{\textbf{Desc. Textures}} & \rot{\textbf{EuroSAT}} & \rot{\textbf{FER-2013}} & \rot{\textbf{FGVC-Aircraft}} & \rot{\textbf{Food-101}} & \rot{\textbf{GTRSB}} & \rot{\textbf{Hateful Memes}} & \rot{\textbf{Kitti Distance}} & \rot{\textbf{Oxford Flowers}} & \rot{\textbf{Oxford Pets}} & \rot{\textbf{PatchCamelyon}} & \rot{\textbf{Rendered-SST2}} & \rot{\textbf{RESISC-45}} & \rot{\textbf{Stanford Cars}} & \rot{\textbf{VOC-2007}} \\ \hline
Stable Diffusion & - & - & 5-shot & 45.92 & 84.68 & \cellcolor[HTML]{48cae4}11.20 & \cellcolor[HTML]{48cae4}40.27 & 37.88 & \cellcolor[HTML]{48cae4}21.04 & \cellcolor[HTML]{48cae4}14.26 & 72.97 & 23.29 & \cellcolor[HTML]{48cae4}44.16 & \cellcolor[HTML]{48cae4}28.69 & \cellcolor[HTML]{48cae4}52.03 & \cellcolor[HTML]{48cae4}72.94 & 51.61 & 49.54 & \cellcolor[HTML]{48cae4}50.48 & \cellcolor[HTML]{48cae4}50.76 & \cellcolor[HTML]{48cae4}74.79 \\ \hline
 & retrieval & 5$\sim$30-shot & 5-shot & 46.46 & 83.70 & \cellcolor[HTML]{48cae4}9.90 & \cellcolor[HTML]{48cae4}36.91 & 23.82 & \cellcolor[HTML]{48cae4}39.78 & \cellcolor[HTML]{48cae4}13.46 & 69.33 & 22.40 & \cellcolor[HTML]{48cae4}49.02 & \cellcolor[HTML]{48cae4}30.80 & \cellcolor[HTML]{48cae4}58.72 & \cellcolor[HTML]{48cae4}74.63 & 50.32 & 50.08 & \cellcolor[HTML]{48cae4}45.64 & \cellcolor[HTML]{48cae4}51.43 & \cellcolor[HTML]{48cae4}79.94 \\ \cline{2-22} 
\multirow{-2}{*}{+ TI} & original & 5-shot & 5-shot & 50.52 & 80.04 & \cellcolor[HTML]{48cae4}14.65 & \cellcolor[HTML]{48cae4}45.05 & 28.56 & \cellcolor[HTML]{48cae4}35.00 & \cellcolor[HTML]{48cae4}17.43 & 69.93 & 16.93 & \cellcolor[HTML]{48cae4}52.10 & \cellcolor[HTML]{48cae4}42.90 & \cellcolor[HTML]{48cae4}71.06 & \cellcolor[HTML]{48cae4}85.50 & 54.45 & 51.13 & \cellcolor[HTML]{48cae4}56.71 & \cellcolor[HTML]{48cae4}57.25 & \cellcolor[HTML]{48cae4}80.16 \\ \hline
Stable Diffusion & - & - & 100-shot & 48.64 & 86.23 & \cellcolor[HTML]{48cae4}12.65 & \cellcolor[HTML]{48cae4}46.38 & 40.04 & \cellcolor[HTML]{48cae4}28.25 & \cellcolor[HTML]{48cae4}13.16 & 74.42 & 26.98 & \cellcolor[HTML]{48cae4}45.20 & \cellcolor[HTML]{48cae4}25.74 & \cellcolor[HTML]{48cae4}57.42 & \cellcolor[HTML]{48cae4}81.82 & 50.27 & 50.80 & \cellcolor[HTML]{48cae4}52.19 & \cellcolor[HTML]{48cae4}55.07 & \cellcolor[HTML]{48cae4}80.29 \\ \hline 
 & retrieval & 5$\sim$30-shot & 100-shot & 47.75 & 85.68 & \cellcolor[HTML]{48cae4}10.79 & \cellcolor[HTML]{48cae4}37.50 & 24.12 & \cellcolor[HTML]{48cae4}46.58 & \cellcolor[HTML]{48cae4}11.88 & 69.24 & 24.45 & \cellcolor[HTML]{48cae4}49.73 & \cellcolor[HTML]{48cae4}32.22 & \cellcolor[HTML]{48cae4}59.42 & \cellcolor[HTML]{48cae4}76.53 & 50.63 & 51.81 & \cellcolor[HTML]{48cae4}47.68 & \cellcolor[HTML]{48cae4}53.21 & \cellcolor[HTML]{48cae4}80.22 \\ \cline{2-22} 
\multirow{-2}{*}{+ TI} & original & 5-shot & 100-shot & 53.90 & 84.40 & \cellcolor[HTML]{48cae4}19.20 & \cellcolor[HTML]{48cae4}48.41 & 34.74 & \cellcolor[HTML]{48cae4}45.86 & \cellcolor[HTML]{48cae4}17.85 & 70.62 & 25.98 & \cellcolor[HTML]{48cae4}55.51 & \cellcolor[HTML]{48cae4}47.65 & \cellcolor[HTML]{48cae4}72.57 & \cellcolor[HTML]{48cae4}84.98 & 55.51 & 52.19 & \cellcolor[HTML]{48cae4}57.85 & \cellcolor[HTML]{48cae4}61.20 & \cellcolor[HTML]{48cae4}81.83 \\ \bottomrule
\end{tabular}%
}
\end{table*}

% Figure environment removed

% % Figure environment removed


% In \cref{fig:demo}, we directly compare the images generated by Stable Diffusion with and without the TIG method. The second row in the figure (\textit{wo/ TIG}) shows images generated directly by Stable Diffusion using the given textual prompt at the bottom. The third row (\textit{w/ TIG}) displays images generated by Stable Diffusion using TIG with 5-shot original images (as shown in the first row) as reference. The TIG-generated images are noticeably closer to the original images.

% In ~\cref{fig:tag_comparison}, we compare the performance improvement of different methods on four datasets where vanilla generative data (\textit{wo/ TIG}) underperforms. To ensure a fair comparison, we also include the results of using 5 original images per category. The figure shows that TIG significantly improves the quality of generated images and, in some datasets, outperforms the performance of 5-shot original data (gray dotted line).
\section{Related Works}
\label{sec:related_works}

Image OD methods can be categorized by their detection mechanism into three main types: density estimation, image reconstruction, and self-supervised classification. We provide a brief introduction of each in the following.

\subsection{Density Estimation}
Density estimation relies on a set of inlier images to build a representation of the inlier distribution. During test time, images are compared to the representation to determine their outlier scores. While some methods have suggested to explicitly model the inlier distribution \cite{book-prml-bishop, a_surve_of_one_class_khan, 2018_dae_gmm}, others have explored non-parametric methods. 

One strategy is to combine tabular non-parametric methods with neural networks to address outlier images. In one study \cite{iForest_on_NN}, image embeddings extracted by a neural network were fed into the Isolation Forest \cite{isolation_forest} and Local Outlier Factor (LOF) \cite{local_outlier_factor} algorithms for outlier score prediction. Other studies \cite{knn-distance-algo, 2019_deep_feature_for_oneclass_cls} have considered the use of nearest neighbors between image embeddings as an outlier score measure. 

Deep learning-based density estimation methods rely on optimizing a neural network to learn characteristics specific to inlier images. DeepSVDD \cite{2018_deepsvdd} is a method inspired by \cite{oc-svm, 2004_svdd} that learns a mapping from the inlier image distribution to a minimum volume hypersphere in latent space. At test time, images that are mapped to points far from the center of the hypersphere are considered likely outliers. OC-CNN \cite{2018_oneclass_cnn} suggested a similar approach, but it leveraged a pre-trained convolutional neural network that can be trained in an end-to-end fashion. Adversarial techniques \cite{sogann} that combine synthetic outlier images with real inlier images to train a discriminator network have also been presented. Given the wide variety of density estimation methods, we refer to \cite{ImageOD-survey, od_survey} for a complete overview.

\subsection{Image Reconstruction}
Image reconstruction refers to a class of methods that measures the outlier score of an image by its reconstruction \cite{Outlier_Analysis_aggarwal}. Auto-encoders and generative adversarial networks (GAN's) are two common options given their ability to generate realistic images using latent representations \cite{2014_anomaly_detection, 2014_gans}. In \cite{2015_vae_anomaly_det, 2015_autoencoder_reconstruction_loss}, auto-encoders were trained on a set of inlier images, and the outlier scores of test images were measured by comparing their reconstruction loss with those of known inlier images. This method was further improved in \cite{2019_as_ae, 2019_mem_ae} by including additional latent heuristics as part of the comparison. Other works \cite{2020_attr, 2020_puzzle_ae} have suggested the addition of image augmentations to increase the distinction between latent representations.  

GAN-based reconstruction relies on the generative ability of a generator network to mimic the inlier image space. One approach \cite{deecke2018anomaly} is to search for a latent representation of a test image from a trained generator. At test time, images that fail to find a corresponding latent representation are considered outliers. Another method \cite{anogan} attempts to generate a replication of a given image using a trained generator. The image is considered an outlier if the generator fails to replicate a similar reconstruction.

\subsection{Self-Supervised Classification}
Self-supervised classification is a type of method that follows the same principles from self-supervised learning \cite{ssl_cookbook}. In self-supervised learning, a model is optimized on one or more auxiliary tasks using transformed data from an unlabeled dataset. The purpose is to learn meaningful representations that can potentially benefit downstream tasks by training on the auxiliary tasks. 

GEOM \cite{geom} proposed an auxiliary objective to train a classification model in identifying geometric transforms on unlabeled inlier images. During inference, the same transformations are applied to the test image, and by comparing the model's output with those seen during training, the outlier score can be determined. Additional transforms were introduced by \cite{2019_using_ssl_can_improve_robustness}, which have demonstrated improved OD performance. GOAD \cite{goad} builds upon GEOM by generalizing the types of transforms to include non-image data. Additional methods \cite{2020_ssl_ae, 2020_csi_ssl_ae, 2021_ssd_od, 2021_neural_transformation} have also shown promising performance using self-supervised classification methods.

% As demonstrated in \cite{geom}, training a model to identify geometric transforms on unlabeled inlier images has proved to be an effective auxiliary task. \cite{2019_using_ssl_can_improve_robustness} improved upon this by adding an additional translation task, which was shown to benefit the overall robustness. \cite{goad} expanded upon the types of transformation by including 



% The objective of self-supervised learning is to leverage auxiliary tasks in an attempt to extract available supervision information from large amounts of unsupervised data. 


% \begin{itemize}
  % \item Image OD Survey Paper: \cite{ImageOD-survey}
  % \item One-class classification
  %   \begin{itemize}
  %     \item Learn a decision boundary around normal images in feature space. Detect anomalies by checking if test images fall outside the boundary.
  %     \item Classical Paper: One class SVM \cite{2001_oneclass_svm}, SVDD \cite{2004_svdd}
  %     \item \cite{2019_deep_feature_for_oneclass_cls} FT the CNN to extract image features and then takes the nearest neighbor classification method to construct the one-class classifier
  %     \item \cite{2019_dl_nd} extend it to the problem of detecting anomalies for multiple known classes instead of one class
  %     \item \cite{2018_deepsvdd} proposed end-to-end deep support vector description model
  %     \item \cite{2018_oneclass_cnn} proposed one-class cnn (OCCNN)
  %   \end{itemize}
  % \item Image reconstruction (learn features from normal cases
    % \begin{itemize}
    %   \item Map images to latent vectors and reconstruct them. Assume reconstruction error is larger for anomalies.
      % \item Autoencoder for OD: \cite{1995_novelty_det_cls}, \cite{2014_anomaly_detection}, \cite{2015_vae_anomaly_det}
      % \item Autoencoder (+latent) for OD: \cite{2018_dae_gmm}, \cite{2019_as_ae}, \cite{2019_mem_ae}
      % \item Autoencoder (+data augmentation): \cite{2020_attr}, \cite{2020_puzzle_ae}
      % \item GAN \cite{2017_ad_gan}
      % \item GAN+Reconstruction \cite{2019_skip_ganomaly}, \cite{2018_gan_ae}, \cite{2019_ocgan}, \cite{2020_old_is_god}
    % \end{itemize}
%   \item SSL (learn features from normal cases
%     \begin{itemize}
%       \item Train model on pretext task using only normal images. Anomalies stand out as model fails on pretext task.
%       \item Example methods: Predict image rotations, translations
%       \item papers: \cite{2018_geometric_trans_ae}, \cite{2020_ssl_ae}, \cite{2020_csi_ssl_ae}, \cite{2021_ssd_od}
%   \end{itemize}
% \end{itemize}

% The problem of OD has been studied in a wide range of machine learning domains, including tabular data, time series, and graph networks. In this paper, we introduce some well-established methods that have targeted image data.

% \subsection{Tabular Methods}
% Prior to deep learning, tabular methods were the primary solutions for image OD. 
% Isolation Forest (iForest) \cite{isolation_forest} is an unsupervised OD algorithm that identifies outliers by partitioning the feature space. The objective is to build \textit{isolation trees} that isolate every data sample by their feature attributes. Densely-packed inliers would require trees of deeper depth for isolation, whereas shallow trees are enough to isolate most outliers. The outlier likelihood of a sample can then be measured by the tree depth for which it is isolated. Another unsupervised algorithm is Local Outlier Factor (LOF) \cite{local_outlier_factor}. In LOF, the \textit{local density} of each sample is computed by measuring its average distance to its \textit{k}-nearest neighbors. Samples with low \textit{local densities} are likely outliers.
% By comparing each sample's \textit{local density} to those of its neighbors, outlier samples can be identified as those with low \textit{local densities}.

% LODA \cite{loda} is a semi-supervised OD algorithm that utilizes random sparse projections to model the distribution of an inlier set. During training, input samples are projected onto a fixed set of sparse projection vectors, and a set of 1D histograms learns the projected distributions. During inference, input samples are projected in the same manner, and the outlier likelihood can be approximated by taking the joint probability of the projections. 

% Other existing works have suggested similar measures to handle outliers in the tabular domain.
% Despite their success, tabular methods often struggle with detecting outlier images due to the differences in data structure and dimensionality. 
% To this extent, some have explored the use of deep learning models to extract meaningful representations in replacement of raw image pixels.

% In a study by Luan \textit{et al.} \cite{iForest_on_NN}, the authors demonstrate significant improvements by feeding features extracted by neural networks into iForest and LOF. Another semi-supervised strategy called KNN-Distance \cite{knn-distance-algo} uses the cosine similarity between a sample and its \textit{k}-nearest neighbors as a measure of outlier likelihood. As demonstrated in their experiments, the transition from raw pixels to deep features has led to improved performance when adopting tabular OD methods. In this paper, we consider such an approach when evaluating existing tabular methods on image OD tasks. 

% \subsection{Deep Learning Methods}
% A fundamental aspect of deep learning methods is that the data is assumed to be homogeneous. Therefore, deep learning algorithms often involve a model that learns inlier data characteristics before generalizing to unseen data. One classic example of such models are auto-encoders for semi-supervised OD. 

% In the auto-encoder setup, an encoder is trained to map high-dimensional inlier samples to low-dimensional features. Using the encoded features, a decoder learns to reconstruct the original sample in its original dimensions. The assumption is that outliers are harder to encode, which leads to poor reconstructions. Thus, the reconstruction error can be an effective measure to determine the outlier likelihood of a sample.

% DeepSVDD \cite{2018_deepsvdd} is semi-supervised algorithm whose objective is to learn a neural network that maps the inlier data into a minimum volume hypersphere. To achieve this, the network learns a center point of a hypersphere such that inliers are distanced closer to the center. During inference, outlier samples that are distant from the center can be identified by their large radial distances. In comparison to previous kernel-based approaches \cite{oc-svm, 2004_svdd}, DeepSVDD offers the advantage of avoiding explicit feature engineering while ensuring scalability to high dimension signals. 


% Generative methods such as adversarial networks or diffusion models has also demonstrated potential for dealing with OD. In the work by Liu \textit{et al.}, SO-GAAL and MO-GAAL \cite{sogann} were presented with the objective of generating artificial outliers for classification training. The idea is to have a generator produce potential outliers that are similar to the inlier data. A discriminator is trained alongside the generator to classify real from fake data. During the optimization process, the discriminator indirectly learns a decision boundary around the inlier distribution, which can be used to evaluate future samples. Similar approaches \cite{anogan} have been demonstrated using other forms of generative methods.







% BELOW ARE OLD OLD TEXTS


% Auto-encoders are often considered in the context of reconstruction-based methods. In this setting, a model is trained to encode inlier samples to a lower dimension, and a reconstruction is generated using the encoded representation. Since outliers have distinct characteristics compared to inliers, the reconstruction loss can be a meaningful measure. 

% Others have suggested more sophisticated methods that include information compression approaches to boundary-based methods. Most information compression algorithms rely on auto-encoders, which generate a reconstruction of an input using the encoded latent variables. By training an auto-encoder on a set of inlier images, outliers can be detected by an abnormally large reconstruction loss. 
% Boundary-based image OD has also been demonstrated using generative adversarial networks. When the discriminator is trained to distinguish real inliers from generated outliers, it forms a decision boundary that encapsulates the inlier distribution. Additional methods have been proposed in the literature, and they all demonstrate promising performance in their respective settings.



% Deep learning OD methods differ from tabular methods in that 
% Common OD methods in deep learning have involved auto-encoders and generative models. 


% In the work by Golan \textit{et al.}, GEOM \cite{geom} was presented as a semi-supervised approach to building an anomaly classifier. During training, GEOM applies multiple distinct transforms to the inlier images, and a model learns to identify features useful for detecting anomalies. GOAD \cite{goad} further extends upon this approach by generalizing the transforms to include non-image data. As demonstrated in their paper, the addition of transformations has allowed the model to learn spatially-preserving features that are inherent among inliers. 

% Deep learning methods have the advantage of learning inlier characteristics directly from data. Once the model has been trained, future samples can be easily determined by simply applying the model regardless of dataset size. Unfortunately, the challenge with such methods is the difficulty in obtaining a quality training set with enough inlier samples. In our experiments, we demonstrate that outlier contamination can severely degrade the performance in both deep learning and tabular methods. 
% To mitigate the influence of outlier contamination, we introduce RANSAC-NN, an unsupervised image OD algorithm that can detect outliers in heavy contaminated datasets. 

% the goal of RANSAC-NN is to provide an unsupervised image OD algorithm that maintains strong performance under heavy contaminated settings. 


% graph networks, and 

% time-series data, graph networks, tabular data, etc.)

% OD in machine learning have spanned domains including time-series data, graph networks, tabular data, and images.   


% Due to the extent of OD methods in the ML literature, 

% OD methods throughout the machine learning literature have covered domains including time-series data, graph-based networks, tabular data, and 

% The vast majority of OD algorithms proposed throughout the literature can categorized by their domain of application and the amount of supervision required. Common domains include tabular, time-series, 

% Numerous OD methods have been proposed throughout the machine learning literature. 

% The vast majority of OD methods in machine learning can be categorized by their domain of application and level of supervision. Common domains include time-series, tabular, 

% Types of data The OD domain span 


% Here we highlight a few well-established methods....




% The scope of OD in machine learning encompass various domains including time-series data, 

% The scope of OD methods in machine learning have encompassed 
% Many different modalities time series, graph based, 

% OD algorithms within the machine learning literature can be categorized into 

% The goal of OD 

% One Class Classification definition
\section{Discussions}

\paragraph{Takeaway Messages}
In this paper, we present the following key findings:

\noindent \textbf{(1)} Generative data can improve downstream tasks on the most common categories. Obtaining generative data is not significantly more expensive than retrieval data, and it can lead to better performance in downstream tasks.

\noindent \textbf{(2)} The effectiveness of generative data is uncertain for fine-grained and rare categories, and careful selection of prompt strategies is required in these scenarios.

\noindent \textbf{(3)} We found that the effectiveness of generative data is closely related to the mean text similarity (MTS) of downstream tasks. In scenarios with low MTS, using Target-Initialized Generation (TIG) can generate images that are more suitable for downstream tasks.

\noindent \textbf{(4)} Using a few target images as the starting point for generation in TIG can significantly improve the quality of generated images and even outperform the use of original images in some cases.

\vspace{-4mm}
\paragraph{Future Directions and Limitations}
We view generative models as a cost-effective and controllable approach for obtaining high-quality external data. With retrieval augmentation and other methods, we can enhance the generative model's ability to enhance its performance on fine-grained and rare concepts, which promotes trustworthiness and fairness. Our method potentially improves learning performance in the long run, e.g., as the initial query in active learning~\cite{chen2022making}.

In our study, we explored the use of up to 500-shot per category (totaling over 1 million images on GenBench) and observed an upward trend in performance in~\cref{fig:scaling_effect}. It would be worthwhile to investigate the scaling law in this direction with more computing resources and identify key factors in generative data and explore their characteristics, as well as the best scenarios for training downstream models. We also believe that increasing the prompt diversity of the images will enhance the usability of the generative data for training models. However, ensuring that the generative data still contain the main semantic information while increasing this diversity is a challenging and worthwhile research problem.
\begin{comment}
\section{System Architecture}
\label{appendix:architecture}
\system has a novel modularized system architecture with three key components: 
\emph{StreamManager}, 
\emph{TxnManager} and \emph{TxnScheduler}. 
These components are instantiated in each thread locally.
The execution outline of \system is presented in Algorithm~\ref{alg:algo}.
Transactional stream processing is continuous and potentially never ends (Line 1$\sim$8).
The dependency resolution and execution of state transactions are separated into two non-overlapping phases by punctuations~\cite{Tucker:2003:EPS:776752.776780} (Line 2 and 5), which guarantees that no subsequent input event will have a smaller timestamp. 
Effectively, a batch of state transactions is collected during the first phase, and processed during the second phase.

In the first phase (i.e., stream processing phase), 
the \emph{StreamManager} conducts preprocessing for every input event ($e$). Similar to some prior works~\cite{tstream}, state transactions may be issued but not immediately processed during preprocessing (Line 3).
The \emph{pre\_processing} and \emph{post\_processing} functions are exposed as APIs to users.
The \emph{TxnManager} handles dependency resolution (Line 4) among state transactions and insert decomposed operations to construct a \tpg. We discuss the detailed two-phase \tpg construction process in Section~\ref{subsec:construction}.

In the second phase  (i.e., transaction processing phase), 
the \emph{TxnManager} is first involved again to refine (Line 6) the constructed \tpg with further dependency resolution.
The \emph{TxnScheduler} 
schedules operations for concurrent execution based on the constructed \tpg according to the three dimensions of scheduling decisions (Line 7). 
In particular, a scheduling decision model $M$ is instantiated based on the constructed \tpg (Line 14).
\textbf{\circled{1}} Guided by $M$, execution threads adopt an exploration strategy (Section~\ref{subsec:explore}) to explore the constructed \tpg for operations available to be scheduled constrained by dependencies. 
\textbf{\circled{2}} 
During exploration, one or multiple operations may be treated as the 
% basic 
unit of scheduling (Section~\ref{subsec:granularity}). 
Subsequently, \textbf{\circled{3}} every thread executes operation(s) in the unit of scheduling with various abort handling mechanisms (Section~\ref{subsec:abort_handling}).
Only when state transactions are processed (i.e., committed or aborted) can the associated input events be postprocessed (Line 8) by the \emph{StreamManager} based on transaction processing results.
\end{comment}

\begin{comment}
\begin{algorithm}
\footnotesize
    \KwData{$e$ \tcp{Input event}}
    \KwData{$txn_{ts}$ \tcp{State transaction}}
    \KwData{$G$ \tcp{The currently constructed TPG}}
    \While{!finish processing of input streams}{
        \eIf(\tcp*[h]{Phase 1}){\text{$e$ is not a $punctuation$}}{
                $txn_{ts}$ $\gets$ PRE\_Processing($e$)\;
                \textbf{TPG\_Construction}($G$, $txn_{ts}$)\; 
          }(\tcp*[h]{Phase 2}){
                \textbf{TPG\_Refinement}($G$)\; 
                \textbf{TXN\_Scheduling}($G$)\; 
                POST\_Processing()\;
          }
    }
    
    \SetKwFunction{FMain}{TPG\_Construction}
    \SetKwProg{Fn}{Function}{:}{}
    \Fn{\FMain{$G$, $txn_{ts}$}}{
        $O_{1..k}$ $\gets$ \textbf{Partition} $txn_{ts}$\;
        \ForEach{\text{operation $O_{i}$ $\in$ $O_{1..k}$}}{
            \textbf{Identify} its \ld\;
            $G$ $\gets$ $G$ + $O_{i}$ \;
        }
    }
    \SetKwFunction{FMain}{TPG\_Refinement}
    \SetKwProg{Fn}{Function}{:}{}
    \Fn{\FMain{$G$}}{
        \ForEach{\text{vertex $e_{i}$ $\in$ $G$}}{
            \textbf{Identify} its \td, \pd\;
        }
    }
    
    \SetKwFunction{FMain}{TXN\_Scheduling}
    \SetKwProg{Fn}{Function}{:}{}
    \Fn{\FMain{$G$}}{
        $M$ $\gets$ Instantiated with $G$;\tcp{A decision model}
        \While{!finish scheduling of $G$
        }{
          \textbf{\circled{2}} $Scheduling Unit$ $\gets$ \textbf{\circled{1}} \emph{Explore}($G$, $M$)\; 
            \textbf{\circled{3}} \emph{Execute with Abort Handling} ($Scheduling Unit$)\; 
        }
    }
  \caption{Execution Outline of \system}
  \label{alg:algo}
\end{algorithm}
\end{comment}


% Can use something like this to put references on a page
% by themselves when using endfloat and the captionsoff option.
\ifCLASSOPTIONcaptionsoff
  \newpage
\fi



% trigger a \newpage just before the given reference
% number - used to balance the columns on the last page
% adjust value as needed - may need to be readjusted if
% the document is modified later
%\IEEEtriggeratref{8}
% The "triggered" command can be changed if desired:
%\IEEEtriggercmd{\enlargethispage{-5in}}

% references section

% can use a bibliography generated by BibTeX as a .bbl file
% BibTeX documentation can be easily obtained at:
% http://mirror.ctan.org/biblio/bibtex/contrib/doc/
% The IEEEtran BibTeX style support page is at:
% http://www.michaelshell.org/tex/ieeetran/bibtex/
%\bibliographystyle{IEEEtran}
% argument is your BibTeX string definitions and bibliography database(s)
%\bibliography{IEEEabrv,../bib/paper}
%
% <OR> manually copy in the resultant .bbl file
% set second argument of \begin to the number of references
% (used to reserve space for the reference number labels box)
% \clearpage
{\small
\bibliographystyle{plain}
\bibliography{egbib}
}

% biography section
% 
% If you have an EPS/PDF photo (graphicx package needed) extra braces are
% needed around the contents of the optional argument to biography to prevent
% the LaTeX parser from getting confused when it sees the complicated
% \includegraphics command within an optional argument. (You could create
% your own custom macro containing the \includegraphics command to make things
% simpler here.)
%\begin{IEEEbiography}[{% Figure removed}]{Michael Shell}
% or if you just want to reserve a space for a photo:

% \begin{IEEEbiography}{Michael Shell}
% Biography text here.
% \end{IEEEbiography}

% if you will not have a photo at all:
% \begin{IEEEbiographynophoto}{John Doe}
% Biography text here.
% \end{IEEEbiographynophoto}

% insert where needed to balance the two columns on the last page with
% biographies
%\newpage

% \begin{IEEEbiographynophoto}{Jane Doe}
% Biography text here.
% \end{IEEEbiographynophoto}

% You can push biographies down or up by placing
% a \vfill before or after them. The appropriate
% use of \vfill depends on what kind of text is
% on the last page and whether or not the columns
% are being equalized.

%\vfill

% Can be used to pull up biographies so that the bottom of the last one
% is flush with the other column.
%\enlargethispage{-5in}



% that's all folks
\end{document}


