\section{CCZ gate}
\label{Sec4A}

\begin{center}
% Figure environment removed
\end{center}
\vspace{-.7 cm}

We performed numerous simulations of the $CC\Phi(\phi)$ gate protocol for a wide range of $\phi$. The simulations were implemented in an extended basis, which also included the logical states of qubits in order to take into account the Rydberg interactions already arising during the excitation step. For the first demonstration of the protocol operation, we consider the implementation of the $CC\Phi(\pi)$ gate, commonly referred to as $CCZ$ gate. Results for several alternative values of $\phi$ are shown in Section \ref{AppA}.

The population and phase dynamics of the three-atom system during gate operation is shown in Figure \ref{CCZ}. To perform a gate according to the protocol described above, we use the asterisk-marked Floquet sideband shown in Figure \ref{Res1}(c). Thus, the atoms are excited into Rydberg states $\ket{70P_{3/2}}$ in an external DC field $F_S=0.1805$ V/cm. Then, the resonant interaction is activated by applying the inducing RF radiation of $\nu = 50$ MHz. In order to demonstrate the relevant phase dynamics over the whole gate sequence, we describe the evolution of the weighted phase $\Phi_{R_i}^{l_i}$ defined by Eq. (\ref{eq6}). Here $l_i$ is one of the register collective logical states, $R_i$ is the Rydberg state, which is coupled with $l_i$ by the excitation pulses 1-3, and $a_{i}$ denotes the complex amplitude of the corresponding state. Weighted phase allows one to accommodate the phase changes of both logical and Rydberg qubit states within a single meaningful smoothly changing variable, providing a convenient analytical tool.
\begin{equation}
    \label{eq6}
\Phi_{R_i}^{l_i}=\frac{|a_{R_i}|^2 \phi_{R_i}+|a_{l_i}|^2 \phi_{l_i}}{|a_{l_i}|^2+|a_{R_i}|^2}
\end{equation}
Note that here we use standard qubit state notation order for the register states, related to the qubit role in the gate and not to its physical location in the array. Thus, a $\ket{Control} \otimes \ket {Control} \otimes \ket{Target}$ state representation is used, although in the described gate configuration the target qubit is the central one.

If all three atoms were initialized in logical state $\ket{1}$, the collective Rydberg state $\ket{RRR}$ is excited during the time $T_{ex}$ (represented by left $\Omega_{ex}$ symbol in Fig.\ref{CCZ}(a)). In this case, we observe resonant Rabi-like population oscillations (Fig.\ref{CCZ}(a)), accompanied by fast phase dynamics (Fig.\ref{CCZ}(b)). Additional slow phase evolution also occurs during the excitation and deexcitation steps, as well as during the waiting times $T^{1(2)}_{wait}$. We optimize the waiting time periods such that the overall evolution results in a $\pi$ phase change after the interaction time. This corresponds to the controlled phase shift for the initial state $\ket{111}$ in Fig.\ref{Scheme}(a). This phase shift is sensitive to both DC electric field and RF pulse amplitude, which act directly on the F\"orster defect. The population of initial state after the RF-induced resonance is $96.2\%$ due to the finite Rydberg lifetimes and the leakage of population to other collective levels by Rydberg interactions. These effects are found to be the major sources of the gate error.

Alternatively, if two atoms out of three are excited into Rydberg states, only off-resonant two-body interactions $\ket{70P_{3/2}\left(m=1/2\right)}^{\otimes 2}\to \ket{70S_{1/2}\left(m=1/2\right);71S_{1/2}\left(m=1/2\right)}$ are possible due to the selection rule $\Delta M=0$. As discussed in \cite{Ashkarin2022}, these interactions only lead to substantial dynamics for closely-spaced atoms due to the fast decrease of van der Waals interaction strength with distance. Thus, for state $\ket{RR0}$ (coupled to the register state $\ket{110}$), which correspond to the case of excitation of two corner atoms, the population dynamics is mainly limited by the Ryderg state decay, accompanied by modest phase evolution (see Figs.\ref{CCZ}(e,f))  Nevertheless, for states $\ket{0RR}$ and $\ket{R0R}$ one can observe fast phase evolution provoked by off-resonant two-body transitions (Fig.\ref{CCZ}(d)). Note that since these two states are analogous in chosen register configuration, they experience the same time dynamics. The accumulated phase shift is sensitive to the external electric field and can be compensated to zero during the interaction time $T_{RF}+T^1_{wait}+T^2_{wait}$, taking into account the slow phase dynamics for partially-excited Rydberg states and additional evolution during the waiting times. Two-body interactions also have a residual impact on the population of the initial state, leading to weak Rabi-like oscillations (amplitude $\sim1-2\%$) (Fig.\ref{CCZ}(c)). A detailed analysis of two-body interactions in three-body systems of Rydberg atoms is given in our previous papers~\cite{Ashkarin2022, Cheinet2020, Ryabtsev2018, Beterov2018}.

Finally, when only one atom in the ensemble is temporarily excited (forming Rydberg states $\ket{00R}$, $\ket{0R0}$ or $\ket{00R}$), the $\pi$ and $-\pi$ pulses, shown in Fig.\ref{Scheme}(a), will return the system into the initial state with zero phase shift after the interaction time $T_{RF}+T^{1}_{wait}+T^{2}_{wait}$. However, temporary Rydberg excitation will result in population loss due to the finite lifetimes of Rydberg states, with the loss magnitude dependent on the environmental temperature $T$. The trivial case is when no Rydberg atoms are excited (state $\ket{000}$). We assume that the pulses 1-6 will have no effect in this instance, leaving the ground states unperturbed. 

Considering the phase and population dynamics described above, we come to the following conclusions: as a result of the algorithm implementation, the phase of the $\ket{111}$ state changes by $\pi$ during the gate execution time (see Fig.\ref{CCZ}(b)). In contrast, the phase shifts for the other basis states of the quantum register remain negligibly small, or are compensated for by weak non-resonant interactions. Thus, the algorithm performs the quantum gate operation $CC\Phi(\pi) \equiv CCZ$.

To perform additional verification of the protocol functionality, we numerically simulated the Toffoli gate scheme implementation based on proposed $CCZ$ gate. We apply two additional Hadamard gates to the target qubit, thus constructing the doubly-controlled-NOT operator. These Hadamard gates can be performed by two-photon Raman $Y_{\pi/2}$ pulses \cite{Zeng2017}, and are considered perfect during the gate simulation.

Figure~\ref{Toffoli} shows the behavior of the relevant collective states of the system in the case when the initial state of the register is $\ket{111}$. Due to the application of the Hadamard gate, the population is first evenly distributed between states $\ket{111}$ and $\ket{110}$. Then the atoms are excited into the Rydberg state $\left(\ket{RRR} + \ket{RR0}\right)/ \sqrt{2}$. The subsequent RF pulse results in the application of $CCZ$ gate, shown in Fig.\ref{CCZ}, thus leading to the final population interchange between $\ket{111}$ and $\ket{110}$ states. The stages of de-excitation of atoms and the second application of the Hadamard gate complete the gate protocol.

\begin{center}
% Figure environment removed
\end{center}
\vspace{-1 cm}