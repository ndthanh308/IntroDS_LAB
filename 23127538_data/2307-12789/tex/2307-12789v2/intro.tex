\section{Introduction}
\label{Sec1}

Reconfigurable arrays of neutral atoms arranged in optical tweezers present a prospective platform for quantum computing \cite{Madjarov2020, Morgado2021, Graham2022, Cuadra2022, Bluvstein2022, Graham2022}. Broad possibilities of interatomic interactions control via Rydberg excitation \cite{Saffman2010, Adams2019, Levine2019, Shi2022, Evered2023}, complemented by long atomic lifetimes in individual traps \cite{Browaeys2020, Schymik2021} and high scalability of quantum registers \cite{Ebadi2022, Scholl2021, Schymik2022, Pause2024, Norcia2024, Manetsch2024} underscore the pivotal virtues of this approach. Recent advances in neutral-atom-based quantum computing include simulations of quantum phase transitions \cite{De2019, Scholl2021, Ebadi2021}, as well as implementation of high-fidelity parallel quantum gates in large-scale registers and generation of entangled multi-qubit states \cite{Graham2019, Levine2019, Evered2023, Shaw2023}.

An outstanding challenge for neutral-atom-based quantum information processing is the implementation of high-fidelity multi-qubit controlled quantum gates. Leveraging multi-qubit operations allows to significantly reduce the total gate count for complex quantum algorithms \cite{Dlaska2022, Zhang2023, Tang2022}, thus paving a way to the practical applications of near-term noisy intermediate-scale quantum (NISQ) devices \cite{Daley2022, Lanthaler2023}. In this regard, a variety of many-body quantum gate schemes has been recently designed, including protocols based on the dipole blockade effect \cite{Isenhower2011, Levine2019, Young2021, Yu2022, Li2022, Jandura2023, Evered2023, Ma2023}, electromagnetically induced transparency \cite{McDonnell2022, Farouk2022} and Rydberg antiblockade \cite{Yin:20, Wu2021, Yang2021, Wu2022}. Nevertheless, utilization of the described approaches is limited by the van der Waals character of the interatomic interaction, which imposes significant constraints on the distances between qubits involved in the gate protocol ($ \sim 2-5$ \textmu m for typical experimental realizations) \cite{Levine2019, Evered2023}. In turn, the ability to perform gates between remote qubits is important for creating entangled states in large quantum registers and realizing inter-register transport of quantum information \cite{Adams2019, Morgado2021}. To circumvent this bottleneck, an alternative technique based on three-body Stark-induced Förster resonant interactions \cite{Faoro2015, Tretyakov2017} has been proposed for the implementation of fast high-fidelity three-qubit quantum gates \cite{Beterov2018a, Ashkarin2022}.
The enhancement of dipole coupling achieved via activation of Förster transitions allows the realization of controlled interactions between distant qubits ($\sim~10-30$ \textmu m), thus providing vast possibilities to achieve increased interconnectivity in the atom-based devices~\cite{Hollerith2022, Saffman2010, Ramette2022}.

%The Förster-based gates implementation complexity due to the technical limits on the control accuracy of the external inducing fields is an essential difficulty for the proposed approach \cite{Ryabtsev2010, Beterov2015, Beterov2016a}. In particular, previously proposed protocols required the application of Stark-switching techniques during the gate process, assuming the external dc electric field control as precise as $\sim 10^{-6}$ V/cm \cite{Beterov2018a}. Significant obstacle is also presented by the strong disturbance of gate fidelity due to interatomic distance variations \cite{Ashkarin2022,Farouk2022}. In this regard, the search for new interaction control and resonance activation methods in order to increase both stability and fidelity of multi-qubit Förster-based gates poses an important challenge.

The Förster-based gates sensitivity to accuracy level of the external inducing fields is an essential difficulty for the proposed approach \cite{Ryabtsev2010, Beterov2015, Beterov2016a}. In particular, previously designed protocols required changing the resonance-inducing DC electric field from a non-resonant value to the exact resonance during the gate process by applying Stark-switching techniques, while assuming the external field control as precise as $\sim 10^{-6}$ V/cm \cite{Beterov2018a}. Significant obstacle is also presented by the strong disturbance of gate fidelity due to interatomic distance variations \cite{Ashkarin2022,Farouk2022}. In this regard, the search for new interaction control and resonance activation methods in order to increase both robustness and fidelity of multi-qubit Förster-based gates poses an important challenge.

One of the effective techniques of Förster resonance induction is the application of external radio-frequency (RF) radiation. Previously, RF-induced two-body resonances were thoroughly investigated in Rydberg ensembles \cite{Tauschinsky2008, Ditzhuijzen2009, Tretyakov2014, Yakshina2016, Lee2017}. High-fidelity two-qubit gate protocols, robust to the experimental imperfections were also proposed based on this approach \cite{Huang2018, Beterov2018}. Thus, extended research on RF induction of many-body resonant interactions presents an interest for potential applications in quantum computing.

In this paper, we propose a protocol for arbitrary doubly-controlled $CC\Phi$ quantum phase gate based on a novel RF-induced three-body F\"{o}rster resonant interaction. The developed protocol exhibits high fidelity for arbitrary phase values ($\sim 99.2 - 99.7 \%$). We explore three-body resonant transfers in ordered arrays of three $^{87}$Rb atoms and show that under the action of a composite electric field consisting of a constant (DC) and a variable (AC) parts, these transfers acquire replicas that give access to coherent population and phase dynamics.  We demonstrate that RF pulse duration along with the radiation frequency and amplitude provide sufficient interaction control parameters to supervise the gate behaviour. Another important feature of the proposed protocol is the possibility to activate the gate solely by applying an RF pulse, thus eliminating the necessity to use Stark-switching techniques and significantly increasing the potential robustness of the gate to electric field deviations compared to the previous proposals \cite{Beterov2018a, Ashkarin2022}.

%In this paper, we describe for the first time three-body RF-induced F\"{o}rster resonant transitions. The resonant transfers in ordered arrays of three $^{87}$Rb atoms are numerically studied, providing a demonstration of the interaction properties. We show that under the action of a composite electric field consisting of a constant (DC) and a variable (AC) parts, three-body resonances acquire replicas that give access to coherent population and phase dynamics. RF radiation thus provides new opportunities to facilitate the experimental realization of multi-qubit Rydberg gates, as well as it gives new degrees of freedom in interaction control, allowing us to significantly increase the gate fidelity compared to the previous proposals \cite{Beterov2018a, Ashkarin2022}.

%We then propose a protocol for arbitrary doubly-controlled $CC\Phi$ quantum phase gates based on described three-body F\"{o}rster resonances, tailored by external radio frequency field. The developed protocol exhibits high fidelity for arbitrary phase values ($\sim 99.2 - 99.7 \%$). Due to the high flexibility of control attainable using RF induction, we perform gates for different phase values at a single three-body transition by adjusting only the RF radiation amplitude and frequency, leaving the key system parameters (interatomic distance, gate time, external dc field) unchanged. Another important feature of the proposed protocol is the possibility to activate the gate solely by applying an RF pulse, thus eliminating the necessity to use Stark-switching techniques and significantly increasing the potential robustness of the gate to electric field deviations.

%We then propose a protocol for universal doubly-controlled quantum phase gates based on described three-body F\"{o}rster resonant interactions, tailored by external radio-frequency field. The developed protocol can be implemented for a wide range of experimental parameters, while exhibiting high fidelity for arbitrary phase values ($\sim 99.2 - 99.7 \%$). We demonstrate that RF pulse duration along with the radiation frequency and amplitude provide sufficient interaction control parameters to supervise the gate behaviour. Another important feature of the proposed protocol is the possibility to activate the gate solely by applying an RF pulse, thus eliminating the necessity to use Stark-switching techniques and significantly increasing the potential robustness of the gate to electric field deviations.