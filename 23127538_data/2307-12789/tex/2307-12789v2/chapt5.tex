\section{Arbitrary $CC\Phi(\phi)$ gates}
\label{AppA}

\begin{center}
% Figure environment removed
\end{center}

\vspace{-.5 cm}

Section \ref{Sec4A} described the implementation of $CC\Phi (\pi) \equiv CCZ$ gate in a linear $^{87}$Rb quantum register. In this section, we present exemplar numerical implementations of the $CC\Phi(\phi)$ gate for alternative values of the phase $\phi = \{\pi/4, \pi/2, 3\pi/4\}$. During the simulations, we aimed to achieve the maximum gate fidelity while keeping all basic experimental parameters unchanged compared to $CC\Phi (\pi)$ gate protocol. We leave fixed the interatomic distances $R=10$ \textmu m in the register, thus effectively retaining the interaction strength. Accordingly, the gate for any phase value is realised using the asterisk-marked Floquet peak in Figure \ref{Res1}(c), at a fixed value of the DC electric field $F_S=0.1805$ V/cm. To adjust the population and phase evolution patterns, we vary the frequency of inducing RF radiation $\nu$ and its amplitude $F_{RF}$. Note that all the timings are also preserved during the gate implementation.

Figure \ref{CCPh} illustrates the phase and population dynamics of the quantum register states during the $CC\Phi(\phi)$ gate implementation. Since the evolution of the system in the case of two-atom excitation does not change noticeably for different values of the accomodated gate phase, the dynamics of the $\ket{0RR}$ ($\ket{R0R}$) and $\ket{RR0}$ states depicted in Figs.\ref{CCZ}(c,d,e,f) remain relevant for all the gates presented in this section. Thus, Fig.\ref{CCPh} only depicts the evolution of the logical state $\ket{111}$ and its Rydberg coupled state $\ket{RRR}$.

In Figure \ref{CCPh}(a), the time dependencies of the logical and Rydberg states populations during the $CC\Phi(3\pi/4)$ gate implementation are shown. We observe a significant decrease in the amplitude of Rabi oscillations during the RF-induced transition for the $\ket{RRR}$ state due to the presence of a weak detuning from the exact resonance necessary to slow down the phase dynamics. As a result of this slowdown, phase evolution allows to reach a value of $\phi=3\pi/4$ during the protocol implementation period (see Fig.\ref{CCPh}(b)). 

Increasing the resonance detuning by varying the RF radiation parameters, we slow down the phase evolution rate during the three-body transfer. Thus, for the population dynamics during the $CC\Phi(\pi/2)$ and $CC\Phi(\pi/4)$ gate implementations shown in Figs.\ref{CCPh}(c,e), the Rabi oscillations amplitude decrease is clearly noticeable, which lead to a significant phase rate decrease (Figs.\ref{CCPh}(d,f)).

Note that the described gate protocol can be realised for arbitrary atomic system parameters. Thus, if certain values of the principal quantum number $n$, interatomic distance $R$ and DC electric field $F_S$ have been chosen due to experimental requirements, one will be able to select a suitable radio-frequency pulse to realize the $CC\Phi(\phi)$ gate, given that strong enough three-body transfer can be induced by RF radiation.