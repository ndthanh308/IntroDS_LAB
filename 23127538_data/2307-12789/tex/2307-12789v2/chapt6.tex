\vspace{.5 cm}
\section{Gate fidelity and optimization}
\label{Sec4B}

To estimate the individual gate fidelity, the method proposed in~\cite{Bowdrey2002} is used. We consider 6 single-qubit configuration states: $\ket{0}$, $\ket{1}$, $\left( \ket{0}+\ket{1} \right)/ \sqrt{2}$, $\left( \ket{0}-\ket{1} \right)/ \sqrt{2}$, $\left( \ket{0}+i\ket{1} \right)/ \sqrt{2}$ and $\left( \ket{0}-i\ket{1} \right)/ \sqrt{2}$. We form a set of three-qubit states as all $6^3=216$ combinations of three single-qubit basis states. We simulate the density matrices $\rho_{sim}$ of all final states after $CC\Phi(\phi)$ gate is applied to each initial state. Then we calculate the fidelity of each final state comparing to the reference state $\rho_{et}$, which is the final state of the ensemble after the perfect $CC\Phi(\phi)$ gate is performed~\cite{Nielsen2011}:
\begin{eqnarray}
\label{eq7}
F=\textrm{Tr}\sqrt{\sqrt{\rho_{et}}\rho_{sim}\sqrt{\rho_{et}}}
\end{eqnarray}
Averaging over all 216 states, we calculate $CC\Phi(\phi)$ gate fidelities for different values of $\phi$. Several exemplar results are shown in Table \ref{tab}. Note that the characteristics of the applied RF radiation are the only controlling parameters of the gate. To perform different phase gates, we only change $F_{RF}$ and $\nu$, and keep all the other parameters fixed (see Sec. \ref{AppA}).

\begin{table}
\caption{\label{tab} Estimated $CC\Phi(\phi)$ gate fidelities for different values of $\phi$. Results for room temperature ($T=300$ K) and cryogenic setup ($T=4$ K) are presented. System parameters: $R=10$ \textmu m, $F_S=0.1805$ V/cm, $T_{RF}=1.27$ \textmu s, $T_{wait}^1=T_{wait}^2=T_{ex}=T_{deex}=20$ ns. RF parameters are shown in the table for each gate.}
\begin{ruledtabular}

\begin{tabular}{c c c c c}
 $T$ (K)&$CC\Phi(\frac{\pi}{4})$ & $CC\Phi(\frac{\pi}{2})$ & $CC\Phi(\frac{3\pi}{4})$ & $CC\Phi(\pi)$ \\
 300 & 99.27\% & 99.26\% & 99.22\% & 99.31\% \\
 4 & 99.64\% & 99.63\% & 99.6\% & 99.69\% \\
 \hline
 $F_{RF}$ (V/cm) & 0.0389 & 0.03409 & 0.0416 & 0.05 \\
 $\nu$ (MHz) & 46.75 & 47.1 & 48.35 & 50 \\
\end{tabular}
\end{ruledtabular}
\end{table}

For the described experimental conditions of resonance (Eq. \ref{eq5}), the average fidelity of the gate implementation is $99.27\%$ for the room-temperature environment. The main source of fidelity losses ($\sim 0.51\%$) is the finiteness of the lifetimes of Rydberg states. Compensation for these losses down to $0.13\%$ can be achieved using a cryostat at 4 K \cite{Ximenez2023, Schymik2021} temperature for the proposed parameter values, increasing the average fidelity to $99.65\%$. Additional compensation for Rydberg lifetime losses can be achieved by using higher Rydberg levels \cite{Wu2021a, Saffman2010}. The losses associated with the non-optimal choice of parameter values are estimated as $\sim 0.22\%$ after the application of multiparametric optimization routines based on simulated annealing technique \cite{Suman2006}. However, this optimization is preliminary and does not guarantee to have reached the global optimum for parameter values due to the complexity of the model. Remaining losses could be compensated for by applying additional multi-parametric optimization routines. For practical applications, we propose to use simulated annealing method in conjunction with Nelder-Mead pre-processing \cite{Ali2014} or GRAPE-based optimization routines \cite{Jandura2023}. QAOA optimization techniques are also applicable \cite{Dlaska2022, Nguyen2023}.

To implement correct phase gates experimentally, it is necessary to pay attention to the required accuracy of the parameter values control. Assuming that the allowable fidelity deviation from the maximum value cannot exceed $0.1\%$ we found the following requirements for parameter accuracy thresholds: the interatomic distance must be controlled with an accuracy of $20$ nm; the interaction time~-~$0.025$ \textmu s; DC electric field amplitude~-~$7 \cdot 10^{-5}$ V/cm, AC electric field amplitude~-~$9.2 \cdot 10^{-5}$ V/cm, RF frequency - $20$ kHz. While the control of the interaction time, as well as the frequency of RF radiation, do not present any experimental challenges \cite{Vankka2001}, the limiting factors are the complexity of the DC and AC field amplitudes control, along with the interatomic distance control. As shown in our previous work \cite{Ashkarin2022}, one can use closer interatomic distances for the chosen value of $n$ to reduce the possible errors in field amplitude control. Nevertheless, this proposal implies an interchange between different error sources, depending on the experimental requirements. A complete analysis on electric field recording was proposed both for DC \cite{Facon2016, Bowden2017} and AC \cite{Meyer2021, Sedlacek2012, Liao2020} fields in Rydberg systems. A proper level of interatomic distance control can be attained with modern holographic and AOD-based tweezer techniques \cite{Chew2022, Xu2022, Chen2022}. Appropriate composite pulse sequences \cite{Torosov2011, Torosov2011a} and coherent control techniques \cite{Warren1993, Larrouy2020} may also be applied to reduce sensitivity to the experimental parameters.