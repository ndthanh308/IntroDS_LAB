\section{Resonance model}
\label{Sec2}
Previously, we have investigated many-body Förster resonant interactions in systems of ultracold Rydberg atoms both theoretically \cite{Ryabtsev2018, Beterov2018, Cheinet2020} and experimentally \cite{Gurian2012, Faoro2015, Tretyakov2017}. It was shown that three-body resonances can be represented as two concurrent two-body dipole transitions \cite{Ryabtsev2018, Beterov2018a} between collective three-atom states of the form $\ket{n_{1} l_{1} j_{1} \left(m_{j1}\right) ;n_{2} l_{2} j_{2} \left(m_{j2}\right) ;n_{3} l_{3} j_{3} \left(m_{j3}\right)} $. In turn, the dipole-dipole interaction (DDI) operator between two neighboring atoms positioned along the quantization axis (Z) can be expressed as~\cite{Walker2008}:
\begin{equation}
\label{eq1}
    \begin{aligned}
V_{dd} &= \frac{e^{2} }{4\pi \varepsilon _{0} R^{3} } \left(\textbf{a}\cdot \textbf{b}-3a_{z} b_{z} \right)=\\
&= -\frac{\sqrt{6} e^{2} }{4\pi \varepsilon _{0} R^{3} } \sum _{q=-1}^{1}C_{1q \; 1-q}^{20} a_{q} b_{-q}.  
\end{aligned}
\end{equation}
Here $\varepsilon_0$ is the vacuum dielectric constant; $e$ is the electron charge; \textbf{a} and \textbf{b} are the vectorial positions of the Rydberg electrons. The radial matrix elements of the dipole moment are calculated using a quasiclassical approximation~\cite{Kaulakys1995}. Note that due to positioning of atoms along the Z axis, the operator (Eq. \ref{eq1}) only couples collective atomic states with $\Delta M = 0$, where $M= \sum_i m_i$ is the total momentum projection of the state \cite{Ryabtsev2018, Beterov2018a}.
\begin{equation}
    \begin{aligned}
\label{eq2}
\ket{nP_{3/2}}^{\otimes 3} \longleftrightarrow &\ket{nS_{1/2};(n+1)S_{1/2};nP_{3/2}} \longleftrightarrow \\ &\ket{nS_{1/2};nP_{1/2};(n+1)S_{1/2}}
\end{aligned}
\end{equation}

We consider three-body fine-structure-state-changing (FSSC) F\"{o}rster resonant transitions of the form (Eq.~\ref{eq2}) in a linear ensemble of three $^{87}$Rb atoms, isolated in individual optical tweezers at a distance $R$ from each other along the direction of the control electric field~\cite{Cheinet2020}. Being initially excited into identical Rydberg states $\ket{nP_{3/2}}$, the atoms pass into a collective state $\ket{nS_{1/2};nP_{1/2};(n+1)S_{1/2}}$ as a result of a Stark-induced resonant transfer. A distinctive feature of this scheme is that one of the atoms passes to the $nP$ state with $J=1/2$ during non-resonant two-body $SP$ excitation hopping. The negative zero-field energy defect of the two-body transfer $\ket{nP_{3/2}}^{\otimes 2} \leftrightarrow \ket{nS;(n+1)S}$, in turn, guarantees the absence of two-body resonant interactions in the vicinity of the considered three-body transition. Note that the described resonances (Eq. \ref{eq2}) exist for arbitrary value of the principal quantum number $n$, which thus can be chosen depending on the experimental requirements.

%Using RF induction, resonances (Eq. \ref{eq2}) can be activated for arbitrary values of DC electric field. We define the F\"orster energy defect $\Delta_F$ as the difference between the energies of the final and initial collective states. The resonance induction results from the compensation of the Förster defect by the energy provided by the RF photons of frequency $\omega$ under the condition $\Delta_F=m \omega$ for integer $m$.

We define the F\"orster energy defect $\Delta_F$ as the difference between the energies of the final and initial collective states in external DC electric field. Using additional RF pulse, resonances (Eq. \ref{eq2}) can be activated for arbitrary values of DC field. The resonance induction results from the compensation of the Förster defect by the energy provided by the RF photons of frequency $\nu$ under the condition $\Delta_F=s \nu$ for a number of photons $s$.

Numerical approach to the simulation of resonant interactions induced by a time-dependent periodic electric field was analysed in detail in \cite{Ditzhuijzen2009}. Following the course of this work in combination with our previous study~\cite{Beterov2018a}, we solve numerically the non-Hermitian Hamiltonian based Schr\"{o}dinger equations for the complex amplitudes of the collective basis states taking into account Rydberg lifetimes~\cite{Beterov2009, Beterov2018a}. The basis is represented by combinations of magnetic sublevels of the corresponding $nS$, $(n+1)S$, and $nP$ states. For simplicity, we consider an open system and neglect the population redistribution due to decay processes, assuming that finite lifetimes only lead to irrecoverable population losses.

During the simulation, we take into account the dipole-dipole interatomic interactions \cite{Walker2008}, as well as the interaction of atoms with an external electric field $F = F_S+F_{RF}\cos\left(2\pi \nu t\right)$ \cite{Ditzhuijzen2009}. Here $F_S$ denotes the static part of electric field (DC field), while $F_{RF}$ is the amplitude of RF field (AC field), polarized linearly along the DC field component \cite{Sevincli2014}. We consider a range of small electric fields, which do not lead to significant mixture between low-orbit states, thus allowing us to deal with bare zero-field states. The simple form of quadratic Stark shifts $\sim~\alpha_{nL}^m F^2/2$ is assumed for all Rb Rydberg states included in our model.

To provide a clear analytical description of our results, we present additional simulations based on Floquet approach \cite{Ditzhuijzen2009, Sevincli2014, Yakshina2016}. In a composite "DC+AC" electric field, the Rydberg states wavefunctions $\Psi_{nL}^m(\textbf{r},t)$ are described as compositions of an infinite number of Floquet sidebands with relative amplitudes $a_{nL,s}$. This leads to the formation of an infinite number of lines in the spectrum of individual Rydberg atoms, separated by the frequency of the applied field $\nu$.
%\begin{equation}
 %   \begin{aligned}
%\label{eq3}
 %   \Psi_{nL}(\textbf{r},t) &= \psi_{nL}(\textbf{r}) e^{i\alpha_{nL}(F^2_S + F^2_{RF}/2)t/2} \cdot \\ &\cdot \sum_{m=-\infty}^{\infty}a_{nL,m}e^{im\omega t}
%\end{aligned}
%\end{equation}

\begin{equation}
    \label{eq3}
    \Psi_{nL}^m(\textbf{r},t) = \psi_{nL}^m(\textbf{r}) e^{i\alpha_{nL}^m(F^2_S + F^2_{RF}/2)t/2} \sum_{s=-\infty}^{\infty}a_{nL,s}e^{is\omega t}
\end{equation}

\begin{equation}
    \label{eq4}
    a_{nL,s} = \sum_{k=-\infty}^{\infty}J_{s-2k}\left(\frac{\alpha_{nL}^mF_{S}F_{RF}}{\omega}\right) J_{k}\left(\frac{\alpha _{nL}^mF_{RF}^2}{8\omega}\right)
\end{equation}
    

%\begin{equation}
 %   \begin{aligned}
%\label{eq4}
 %   a_{nL,m} = \sum_{k=-\infty}^{\infty}&J_{m-2k}\left(\frac{\alpha_{nL}F_{S}F_{RF}}{\omega}\right) \cdot \\ &\cdot J_{k}\left(\frac{\alpha _{nL}F_{RF}^2}{8\omega}\right)
%\end{aligned}
%\end{equation}
According to \cite{Yakshina2016}, the intersection of collective states Floquet sidebands in the external electric field results in the appearance of a Förster resonant transition. Note that both DC and AC components are necessary to effectively drive the resonances according to the Floquet model, since at $F_S=0$ the odd sidebands disappear in Eq. (\ref{eq4}), while the even sidebands are weak \cite{Ditzhuijzen2009}.