\section{Three-body resonance simulation}

This section provides a detailed description of our simulation results for three-body F\"{o}rster resonant interactions (Eq. \ref{eq2}) induced by RF radiation in a three-atom system. According to the previous studies \cite{Beterov2018a, Cheinet2020, Pham2020}, a good compromise between robustness to small fluctuations of DC electric field and individual Rydberg lifetimes during the resonance process can be demonstrated for $65\leq n \leq 85$ in $^{87}$Rb ensemble. Following the course of our previous work \cite{Ashkarin2022}, we choose $n=70$ to expose the described resonances. To reduce the number of basis states, we limit the collective state spectra within $\pm~2$~GHz range relative to the $\ket{70P_{3/2}(m=1/2)}^{\otimes 3}$ initial state energy, thus considering only atomic states with a momentum value $l~\le~1$. Also, due to the selection rule $\Delta M = 0$, raised by the linear configuration of the atomic register, we can exclude states whose total momentum projection differs from $M$ of the initial state \cite{Ryabtsev2018}. Thus, we significantly reduce the basis of states under consideration, down to $\sim 60$ collective three-atom states, presented by combinations of the magnetic sublevels $\ket{70S_{1/2} (m=\pm1/2)}, \ket{71S_{1/2}(m=\pm1/2)},\ket{70P_{1/2}(m=\pm1/2)}, \ket{70P_{3/2}(m=\pm1/2;\pm3/2)}$.
%%%%%%%%%%%%%%%%%
\begin{eqnarray}
\label{eq5}
&\ket{70P_{3/2}}^{\otimes 3} \to \ket{70S_{1/2};70P_{1/2};71S_{1/2}}
\end{eqnarray}
%%%%%%%%%%%%%%%%

Figure~\ref{Res1}(a) depicts the three-body Förster resonant transitions in a form of Stark diagram, calculated for non-interacting atomic system. The original resonance (Eq. \ref{eq5}) is represented by the intersection of solid blue (initial $\ket{70P_{3/2}}^{\otimes 3}$ state) and red (final $\ket{70S_{1/2};70P_{1/2};71S_{1/2}}$ state) lines, and is indicated by the symbol (I). This resonance naturally exists in the corresponding DC electric field ($F_S=0.135$ V/cm) even in absence of RF radiation. In turn, upon activation of the AC electric field component, one can observe the occurrence of additional 1$^{st}$-order induced resonance transfers displayed by the violet arrows in Fig.~\ref{Res1}(a). Thus, the left ($F_S = 0.074$ V/cm) and right ($F_S = 0.178$ V/cm) arrows correspond to the emission and absorption of an RF photon with frequency $\nu=50$ MHz, respectively. Note that resonances of higher orders can also arise with absorption/emission of a larger number of photons. Nevertheless, such resonances possess essentially smaller amplitudes (see Eq. \ref{eq4}) that, together with limited strength of three-body interactions, makes them poorly suitable for quantum operations realization \cite{Tretyakov2014, Yakshina2016}. We keep the modulation depth on the order of 1 during the simulations in order to get a strong $1^{st}$-order sideband and avoid unwanted spurious effects due to further sidebands.

\label{Sec3}
\begin{center}
% Figure environment removed
\end{center}
\vspace{-.5 cm}

Alternatively, we can describe the RF induction process using the Floquet approach. Thus, Figure~\ref{Res1}(b)  shows the initial and final collective Rydberg states, accompanied by the corresponding Floquet sidebands. Intersections (1) and (1') correspond to the left arrow in Fig~\ref{Res1}(a), while intersections (2) and (2') correspond to the right arrow. In turn, intersections (I') and (I'') emerge for the same DC field value as the original resonance (I). Notably, when considered in dressed state formalism, the radio-frequency sidebands only differ by the number of RF photons included. Thus, when found at the same DC field, the sidebands crossings represent the same resonant transfers and do not create additional interaction channels.

Figure~\ref{Res1}(c) presents the numerically calculated dependence of the fraction $\rho$ of atoms found in $\ket{71S_{1/2}}$ state on the DC electric field, thus demonstrating the efficiency of the three-body Förster resonant transfer (Eq. \ref{eq5}). When RF radiation is turned off, only original resonance (I) remains (black line). Note that in the resonant process, we cannot attribute various two-step transfers $\ket{70P_{3/2}} \to \ket{70S_{1/2}}$, $\ket{70P_{3/2}} \to \ket{71S_{1/2}}$ and $\ket{70P_{3/2}} \to \ket{70P_{1/2}}$ to a specific atom in the ensemble. In this regard, several resonant interaction channels are formed, from which only two are allowed due to symmetry reasons \cite{Ryabtsev2018}. Ultimately, this process is similar to the Autler–Townes effect, and leads to the splitting of resonant peaks into 2 satellites, as it is shown in Figure \ref{Res1}(c). Alternatively, when the RF radiation is present in the system, it induces additional first-order peaks (green line). The relative distances between the resonances correspond to the applied radiation frequency of 50 MHz. The displacement of the doublet centers relative to the expected resonant positions provided in Figures~\ref{Res1}(a,b), is due to the presence of the interatomic DDI taken into account in the complete simulation. The expected DC field of the resonance peak is also dependent on the AC Stark shift produced by the RF field applied. This effect is clearly noticeable for the original three-body resonance peak (black line in Fig.\ref{Res1}(c)), which shifts when the AC field component is switched on (central doublet, green line in Fig.\ref{Res1}(c)).

The slight difference in the shape and amplitude of the central peak in the presence/absence of RF radiation (Fig.\ref{Res1}(c)) is due to the difference in the interaction strength for these two cases. The population oscillation frequency of the DC-induced resonance ($\Omega_{DC}$) exceeds that of the RF-induced counterpart ($\Omega_{RF}$). In turn, the interaction time value $T_{int}=0.635$ \textmu s applied during simulation was chosen to maximize the amplitude of the RF-induced peaks under study, providing the pulse area $\Omega_{RF} T_{int} = \pi$. Accordingly, the pulse area of the DC-induced peaks $\Omega_{DC} T_{int} > \pi$, thus leading to the resonance amplitude reduction.

A great decrease of intensity can be seen for the left sideband (two leftmost peaks in Fig.\ref{Res1}(c), green line), when compared with the right sideband. This decrease is associated with two effects. First, the amplitude is influenced by two-body exchange interactions between atoms in $S$ and $P$ states, accompanied by exchange of momentum projection $m$. These interactions are resonant in the zero field due to the spectra degeneracy, thus provoking a significant leakage of the initial state population. The influence of such processes decreases quadratically when the DC field increases and becomes negligibly small for the original resonance \cite{Cheinet2020}. Second, the modulation depth (or equivalently the amplitude of the Floquet sideband) depends on the DC electric field. Since the Stark effect is quadratic, the modulation depth increases fast with the DC field component (see Eq. \ref{eq4}). Thus, a significant decrease in peak amplitudes can be expected for small values of $F_S$ compared to larger values. To avoid the unwanted influence of these two effects, we opt for the transitions arising for the DC field range $0.15 \leq F_{S} \leq 0.2$ V/cm when considering any resonant dynamics.

When tuning the DC field to the rightmost resonant sideband $F_S = 0.1805$ V/cm (marked with asterisk in Fig.\ref{Res1}(c)), we observe coherent Rabi-like population dynamics of the collective Rydberg states (Fig.\ref{Rabi}). The oscillation period of $T_{osc}=1.27$ \textmu s corresponds to the RF-induced interaction strength. The amplitude decrease is mainly caused by the limited lifetimes of the Rydberg states. Note that the Rabi oscillations are also accompanied by fast phase dynamics, which can be directly controlled by the external field adjustment (see Sec. \ref{Sec4A}).

\begin{center}
% Figure environment removed
\end{center}

\vspace{-1.5 cm}

