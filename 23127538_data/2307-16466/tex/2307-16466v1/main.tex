\documentclass[journal]{IEEEtran}
\usepackage{xcolor}
\usepackage{gensymb}

\usepackage{capekCommands}

\usepackage[]{hyperref}
\hypersetup{
	colorlinks,
	linkcolor={red!50!black}, 
	citecolor={blue!50!black},
	urlcolor={blue!20!black}
}

% Path to figures
\usepackage{graphicx}

\DeclareRobustCommand*{\IEEEauthorrefmark}[1]{\raisebox{0pt}[0pt][0pt]{\textsuperscript{\footnotesize #1}}}




% *** SUBFIGURE PACKAGES ***
\ifCLASSOPTIONcompsoc
\usepackage[caption=false,font=normalsize,labelfont=sf,textfont=sf]{subfig}
\else
\usepackage[caption=false,font=footnotesize]{subfig}
\fi


\hyphenation{}

\newacro{VSWE}{vector spherical wave expansion}
\newacro{VSW}{vector spherical wave}
\newacro{QCQP}{quadratically constrained quadratic program}
\newacro{AToM}{Antenna Toolbox for Matlab~\cite{atom}}
\newacro{FEKO}{FEKO~\cite{feko}}
\newacro{CST}{CST~\cite{cst}}

\begin{document}
	
	
	
	\setcounter{page}{1}
	\setcounter{figure}{0}
	\setcounter{section}{0}
	\newpage
	\pagestyle{headings}
	\twocolumn
	
	% ========================================================================
	\title{Upper Bound on Implantable Antennas\\Considering Ohmic Loss}
	\author{Jakub~Liska,
		Mingxiang~Gao,
		Lukas~Jelinek,
		Erik~R.~Algarp,
		Anja~K.~Skrivervik,
		Miloslav~Capek,~\IEEEmembership{Senior~Member,~IEEE}
		\thanks{Manuscript received \today; revised \today.}
		\thanks{This work was supported by the Czech Science Foundation under project~\mbox{No.~21-19025M}, and by the Grant Agency of the Czech Technical University in Prague under project \mbox{No.~SGS22/162/OHK3/3T/13}.}% <-this % stops a space
		\thanks{J.~Liska, L. Jelinek and M. Capek are with the Czech Technical University in Prague, Prague, Czech Republic (e-mails: \{jakub.liska; lukas.jelinek; miloslav.capek\}@fel.cvut.cz).}% <-this % stops a space
		\thanks{M.~Gao, E.~Algarp and A.~K.~Skrivervik are with École Polytechnique Fédérale de Lausanne, Lausanne, Switzerland (e-mails: \{mingxiang.gao; erik.algarp; anja.skrivervik\}@epfl.ch).}% <-this % stops a space
		\thanks{Color versions of one or more of the figures in this paper are
			available online at http://ieeexplore.ieee.org.}% <-this % stops a space
		%\thanks{Digital Object Identifier XXX}% <-this % stops a space
	}
	
	\maketitle
	
	% As a general rule, do not put math, special symbols or citations
	% in the abstract or keywords.
	\begin{abstract}
		A hybrid approach for the computation of performance limitations for implanted antennas is presented, taking radiation efficiency as the optimized metric. Ohmic loss in the antenna and surrounding tissue are both considered. The full-wave interaction of all parts of the system is taken into account. The method is thoroughly tested on a realistic implanted antenna design that is treated both experimentally and as a model in a commercial electromagnetic solver. A great agreement is reported. In addition, the fundamental bounds on radiation efficiency are compared to the performance of a loop and a dipole antenna showing the importance of various loss mechanisms during the designs. The trade-off between tissue loss and antenna ohmic loss indicates critical points in which the optimal solution drastically changes and in which the real designs should change their topology.
	\end{abstract}
	
	% Note that keywords are not normally used for peerreview papers.
	\begin{IEEEkeywords}
		Implanted biomedical devices, hybrid solution methods, antenna efficiency, fundamental bounds, tissue loss, ohmic losses, numeric methods, quadratic programming.
	\end{IEEEkeywords}
	
	
	
	
	
	
	% For peer review papers, you can put extra information on the cover
	% page as needed:
	% \ifCLASSOPTIONpeerreview
	% \begin{center} \bfseries EDICS Category: 3-BBND \end{center}
	% \fi
	%
	% For peerreview papers, this IEEEtran command inserts a page break and
	% creates the second title. It will be ignored for other modes.
	\IEEEpeerreviewmaketitle
	
	
	
	\section{Introduction}
	% The very first letter is a 2 line initial drop letter followed
	% by the rest of the first word in caps.
	% 
	% form to use if the first word consists of a single letter:
	% \IEEEPARstart{A}{demo} file is ....
	% 
	% form to use if you need the single drop letter followed by
	% normal text (unknown if ever used by the IEEE):
	% \IEEEPARstart{A}{}demo file is ....
	% 
	% Some journals put the first two words in caps:
	% \IEEEPARstart{T}{his demo} file is ....
	% 
	% Here we have the typical use of a "T" for an initial drop letter
	% and "HIS" in caps to complete the first word.
	% You must have at least 2 lines in the paragraph with the drop letter
	% (should never be an issue)
	
	% needed in second column of first page if using \IEEEpubid
	%\IEEEpubidadjcol
	
	\IEEEPARstart{T}{elemedicine} has experienced a significant boom during the past few decades. An important enabler of this progress is wireless communication and wireless power transfer for medical implants via radiofrequency fields, which renders the antenna a crucial element. Laws of physics, however, bound the performance of antennas, and the knowledge of these limits is essential to improve the development of telemedicine devices.
	
	Performance limitations on antennas are of key interest to the antenna designers~\cite{Chu_PhysicalLimitationsOfOmniDirectAntennas, Wheeler_FundamentalLimitationsOfSmallAntennas,Fano_TheoreticalLimitationsOnTheBroadbandMatchingOfArbitraryImpedances,McLean_AReExaminationOfTheFundamentalLimitsOnTheRadiationQofESA,YaghjianBest_ImpedanceBandwidthAndQOfAntennas,Harrington_OnTheGainAndBWofDirectAntennas} even outside telemedicine and their determination has evolved to a discipline on its own. For antennas radiating into free space, they can nowadays be evaluated for arbitrary metrics and shapes~\cite{Gustafsson_OptimalAntennaCurrentsForQsuperdirectivityAndRP,2020_Molesky_PRR,GustafssonCapek_MaximumGainEffAreaAndDirectivity}.
	
	Nevertheless, the aspects of wireless communication with implanted devices~\cite{wearableImplantableAntennasReview,Nikolayev_AntennaIngestibleAndImplantableApp} differ significantly from antennas operating in free space. Of particular interest for this paper are physical bounds on the maximum power density of \ac{EM} waves reaching free space from implanted antennas~\cite{Skrivervik_Bosiljevac_Sipus_FundamentalBoundsForImplantedAntennas,Bosiljevac_PropagationInFiniteLossyMedia,Gao_RulesOfThumbToAssessLossesOfImplantedAntennasEuCAP21,Gao_OnTheMaximumPowerDensityofImplantedAntennasWithinSimplifiedBodyPhantomsEuCAP22, Nikolayev+etal2019}. Generally, the size of an implanted antenna is typically a few millimeters, and the operating frequency ranges from hundreds of megahertz to a few gigahertz~\cite{Nikolayev2018}. This indicates that implanted antennas are usually electrically small. Contrary to conventional antennas, the implant is firstly surrounded by a small volume of ideally lossless medium representing the implant encapsulation. Beyond this, the \ac{EM} waves radiated by the implant must pass through a lossy tissue before escaping to free space.
	
	Until recently, fundamental bounds for implanted antennas have only been treated considering elementary sources~\cite{Skrivervik_Bosiljevac_Sipus_FundamentalBoundsForImplantedAntennas} with no considerations on true physical dimensions. Finite geometry, cylindrical shell, was considered in~\cite{Nikolayev+etal2019} and compared to elementary infinitesimally small sources studied before. An initial attempt to find the radiation efficiency limitation of an implanted antenna with finite conductivity and arbitrary shape of the current supporting region was presented in~\cite{Jelinek-MaxRadEffImplantedAntennaHybrid, Liska_MaxRadEffImplantableAntennaEuCAP2023}. It was followed by a feasibility study of performance limitations on implanted antennas considering arbitrarily-shaped and lossy current-supporting regions~\cite{Algarp2022}.
	
	A popular methodology~\cite{Gustafsson_OptimalAntennaCurrentsForQsuperdirectivityAndRP, Liska_etal_FundamentalBoundsEvaluation} for evaluating fundamental bounds is based on optimal current density, integral equation formulation, and the employment of the method of moments. This presents a major obstacle for implanted devices which demand the interaction of an electrically small antenna and a potentially electrically large body\footnote{The radius of the smallest sphere circumscribing the body is much larger than a typical operating wavelength in the free space.}. A direct approach using the volumetric meshing of inhomogeneous tissues would be unbearably computationally demanding~\cite{Polimeridis_OnComputationOfPowerInVolumeIntegralEquationFormulations}. To overcome the computation difficulties, a hybrid method~\cite{Kim-ImplantedAntennasInsideHumanBody} is used in this work, which consists of a combination of two frequency domain schemes: the \ac{MoM} formulation of the electric field integral equation is used for the antenna and its closest neighborhood, while a \ac{VSWE} is applied to numerically evaluate the interaction of the \ac{EM} field with the host body~\cite{2022_Losenicky_TAP}. This allows treating this complex problem with relatively few unknowns~\cite{2022_Losenicky_TAP}. The hybrid method requires a spherical boundary on which the two underlying numerical methods, \ac{MoM} and \ac{VSWE}, are coupled. This boundary is defined naturally as a spherical encapsulation of the current support, see~Fig.~\ref{fig:setupS}.
	% Figure environment removed
	
	Though the hybrid method is known, it remains to determine how fundamental bounds can be formulated in this framework. The basic idea was introduced in~\cite{Jelinek-MaxRadEffImplantedAntennaHybrid, Algarp2022, Liska_MaxRadEffImplantableAntennaEuCAP2023} and is developed in this paper. It also employs some previously developed optimization schemes~\cite{Liska_etal_FundamentalBoundsEvaluation,Liska-CompFunBoAntennas-EuCAP22}. Similarly to antennas in the free space~\cite{Gustafsson_OptimalAntennaCurrentsForQsuperdirectivityAndRP, Liska_etal_FundamentalBoundsEvaluation}, the computation of fundamental bounds~\cite{Liska_etal_FundamentalBoundsEvaluation,Liska-CompFunBoAntennas-EuCAP22} is based on convex optimization, particularly~\ac{QCQP}~\cite{NocedalWright_NumericalOptimization,Liska_etal_FundamentalBoundsEvaluation}. The approach is applied to a basic scenario with a spherical single-layered phantom and commonly used medical frequencies of 403\,MHz and 2.45\,GHz. Results are verified by measurement and by simulations using independent commercially available solvers: \ac{CST} simulation based on FDTD and \ac{FEKO} simulation based on \ac{MoM}.
	
	This paper is organized as follows. Section~\ref{sec:formulation} presents the used model and the basis of the numerical tools used later. Their description is necessary to understand the following sections. Section~\ref{sec:losses} stresses the importance of finite metal conductivity of the current supporting region and consequences. The method verification is done in Section~\ref{sec:verification}, which also presents results of simulations in commercially available solvers and measurement results. Further investigation and discussion in the setup for verification are given in section~\ref{sec:investigation}.
	
	%%%%%%%%%%%%%%%%%%%%%%%%%%%%%%%%%%%%%%%%%%%%%%%%%%%%%%%%%%%%%%%%%%%%%%%%%%%%%%%%%%%%%%%%%%%%%%%%%%%%%%%%%%%%%%%%%%%%%%%%%%
	\section{Computational Tools}\label{sec:formulation}
	Modeling the wireless link between an implanted antenna and an external node is a challenging problem, resolved here using the advantages of two distinct approaches, \ac{MoM} and \ac{VSWE}, which are combined within the hybrid method. \Ac{MoM} with piece-wise basis functions~\cite{RaoWiltonGlisson_ElectromagneticScatteringBySurfacesOfArbitraryShape} is applied to characterize the antenna or current supporting region in a great level of spatial details. Conversely, \ac{VSWE} serves as a set of entire-domain basis functions to determine the influence of the surrounding tissues. Computational complexity is significantly reduced~\cite{Moerlein2021AntennaDe-EmbFDTDusinfSWFarXiv,Berkelmann2022AntennaOptimWBANbasedSWFde-EmbTAP} as compared to the case, in which the entire setup is solved just with a single method with entire-domain discretization.
	
	Within the quadratic representation of power-like quantities in \ac{MoM} and \ac{VSWE}, the physical limitations are readily evaluated as in previous studies related to \ac{MoM}~\cite{Gustafsson_OptimalAntennaCurrentsForQsuperdirectivityAndRP,Liska-CompFunBoAntennas-EuCAP22,Liska_etal_FundamentalBoundsEvaluation}.The evaluation yields the value of the fundamental bound on a chosen metric and an associated optimal current distribution, which may serve as an initial inspiration for the antenna designers.
	
	%%%%%%%%%%%%%%%%%%%%%%%%%%%%%%%%%%%%%%%%%%%%%%%%%%%%%%%%%%%%%%%%%%%%%%%%%%%%%%%%%%%%%%%%%%%%%%%%%%%%%%%%%%%%%%%%%%%%%%%%%%%
	\subsection{Method of Moments and T-matrix Hybrid}\label{sec:hybrid}
	\Ac{MoM} and T-matrix methods are described in~\cite{2022_Losenicky_TAP} and only the essential steps to hybridize them are presented here. As mentioned above, the method applies \ac{MoM} for the antenna description and \ac{VSWE} with T-matrix for the surrounding tissues.
	
	\subsubsection{Method of Moments}
	Within \ac{MoM}, the current supporting region\footnote{Support for optimized current density in the sense of fundamental bounds defined in Section~\ref{sec:FunBo} or simply the antenna metalization in the case of a design.}, $\varOmega_Z$ region in Figs.~\ref{fig:setupS} and~\ref{fig:setupS1}, is discretized in a set of elementary cells together with current density, which is approximated by a sum of weighted basis functions~$\left\{\basisFcn_n\right\}_{n=1}^N$ with expansion coefficients~$I_n$,
	\begin{equation} \label{eq:currentExpansion}
		\V{J} (\V{r}) \approx \sum \limits_{n=1}^N I_n \basisFcn_n (\V{r}).
	\end{equation}
	Substitution of~\eqref{eq:currentExpansion} in the appropriate \ac{EFIE} and the use of Galerkin method~\cite{Harrington_FieldComputationByMoM} results in a system of linear equations
	\begin{equation} \label{eq:MoM}
		\left( \Zmat_0 + \Zmat_\rho \right) \Ivec = \Vvec_\T{i}
	\end{equation}
	where $\Zmat_0$ is the free-space impedance matrix, $\Zmat_\rho$ is the material impedance matrix\footnote{The free-space impedance matrix~$\Zmat_0$ gives the properties of the current as it flows in the free space, meaning the radiation (real part) and reactive power (imaginary part). The material impedance matrix~$\Zmat_\rho$ gives the properties of the current flowing in the chosen material, meaning the losses (real part) and reactive power of the material (imaginary part).}, $\Vvec_\T{i}$ is the excitation vector, and $\Ivec$ is the unknown vector of current expansion coefficients~$I_n$. This allows for the characterization of the antenna in the free space. It should also be noted that the addition of dielectric inclusions or the use of non-spherical encapsulation can be resolved by the use of a volumetric electric field integral equation in the same way as described above.
	% Figure environment removed
	
	\subsubsection{Vector Spherical Wave Expansion}
	The idea of field expansion into a set of basis functions is also applied in the T-matrix method. In this case, entire-domain basis functions are used, and the electric and magnetic fields are expanded in vector spherical harmonics
	\begin{align}
		\V{E} (\V{r}) &= k \sqrt{Z} \sum \limits_{\alpha} a_\alpha \M{u}_\alpha^{(1)} (k \V{r}) + f_\alpha \M{u}_\alpha^{(4)} (k \V{r}), \\
		\V{H} (\V{r}) &= \J \dfrac{k}{\sqrt{Z}} \sum \limits_{\alpha} a_\alpha \M{u}_{\bar{\alpha}}^{(1)} (k \V{r}) + f_\alpha \M{u}_{\bar{\alpha}}^{(4)} (k \V{r}),
	\end{align}
	where $k$ and $Z$ are wave number and background wave impedance, respectively, $a_\alpha,~f_\alpha$ are the expansion coefficients, and $\M{u}_\alpha^{(1)},~\M{u}_\alpha^{(4)}$ are vector spherical waves~\cite[Appendix A]{2022_Losenicky_TAP} of regular and outgoing type, respectively. To describe the scenario of implanted antenna, the relation between the field outside the body and in the implant (antenna) encapsulation is given in matrix form as
	\begin{equation}\label{eq:VSWE}
		\begin{bmatrix}
			\M{f}_1 \\ \M{a}_2
		\end{bmatrix} = \begin{bmatrix}
			\M{T} & \M{\Psi} \\
			\M{\Psi}^\trans & \M{\Gamma}
		\end{bmatrix} \begin{bmatrix}
			\M{a}_1 \\ \M{f}_2
		\end{bmatrix},
	\end{equation}
	with $\M{T},~\M{\Gamma}$ being transmission matrices outside and inside, respectively, matrices $\M{\Psi},~\M{\Psi}^\trans$ accounting for field penetrating from inside to outside or vice-versa, and $\M{a},~\M{f}$ gathering expansion coefficients. Sub-indices ``1'' and ``2'' corresponds to expansion outside and inside, respectively; see Figs.~\ref{fig:setupS1} and~\ref{fig:layer}.
	% Figure environment removed
	In this paper, neither an incident field from infinity nor an external antenna is considered, therefore, $\M{a}_1 = \M{0}$ is used for simplicity in further derivations.
	
	In the case of a spherically layered host body, the interaction matrices in~\eqref{eq:VSWE} can be evaluated analytically~\cite{Merli-TheEffectOfInsulatingLayersOnThePerformanceOfImplantedAntennas}, which is also done in this paper. A representative example is a spherical encapsulation with an antenna centered in a phantom of a human head. The human head can be sufficiently approximated by a spherical multilayered host body~\cite{Kim-ImplantedAntennasInsideHumanBody}, where tighter bounds on implanted antennas, as compared to~\cite{Skrivervik_Bosiljevac_Sipus_FundamentalBoundsForImplantedAntennas,Gao_OnTheMaximumPowerDensityofImplantedAntennasWithinSimplifiedBodyPhantomsEuCAP22}, can be evaluated in terms of seconds using the technique proposed here. The method can be easily adapted to other scenarios like ingestible devices and is not restricted to medical telemetry, but T-matrix has to be obtained numerically by some standard solver in the case of a non-spherical model.
	
	\subsubsection{Hybridization}
	The coupling of \ac{MoM} and T-matrix method is done via projection matrix~$\M{U}_1$
	\begin{equation}
		U_1^{\alpha n} = k \sqrt{Z} \left\langle \M{u}_\alpha^{(1)}, \basisFcn_n \right\rangle
	\end{equation}
	which maps vector spherical waves to basis functions from \ac{MoM} as
	\begin{align}
		\M{f}_2 &= - \M{U}_1 \Ivec, \\
		\Vvec &= \Vvec_\T{i} + \M{U}_1^\trans \M{a}_2.
	\end{align}
	The above relations are written for a particular case of outgoing waves and complete excitation within \ac{MoM}, where $\Vvec_\T{i}$ is the direct excitation in the current supporting region (for example delta-gap source). This connection allows combining the methods from \eqref{eq:MoM} and \eqref{eq:VSWE} into a hybrid method
	\begin{equation}\label{eq:system}
		\begin{bmatrix}
			\Zmat_0 + \Zmat_\rho & - \M{U}_1^\trans & \M{0} & \M{0} \\
			-\M{U}_1  & \M{0} & -\M{1} & \M{0} \\
			\M{0}  & -\M{1} & \M{\Gamma} & \M{0} \\
			\M{0}  & \M{0} & \M{\Psi} & -\M{1}
		\end{bmatrix}  \begin{bmatrix}
			\Ivec \\ \M{a}_2 \\ \M{f}_2 \\ \M{f}_1
		\end{bmatrix} = \begin{bmatrix}
			\Vvec_\T{i} \\ \M{0} \\ \M{0}\\ \M{0}
		\end{bmatrix}.
	\end{equation}
	
	%%%%%%%%%%%%%%%%%%%%%%%%%%%%%%%%%%%%%%%%%%%%%%%%%%%%%%%%%%%%%%%%%%%
	\subsection{Upper Bound on Radiation Efficiency}\label{sec:FunBo}
	Concerning implanted antennas, radiation efficiency is one of the most important performance metrics. The performance is generally low due to the lossy tissues surrounding the antenna. The previously studied limits~\cite{Skrivervik_Bosiljevac_Sipus_FundamentalBoundsForImplantedAntennas, Gao_RulesOfThumbToAssessLossesOfImplantedAntennasEuCAP21} considered cycle-mean radiated power $P_\T{rad}$ and cycle-mean power absorbed by the lossy biological tissues $P_\T{tis}$.
	
	Within the hybrid method shown in the previous section~\ref{sec:hybrid}, ohmic losses in the antenna can also be considered and separated from the total loss. Similarly to~\cite{Jelinek-MaxRadEffImplantedAntennaHybrid, Algarp2022, Liska_MaxRadEffImplantableAntennaEuCAP2023}, the upper bound on radiation is here formulated as the optimal current density, which realizes the highest radiation efficiency. The expansion coefficients~$I_n$ are taken as the degrees of freedom in this optimization problem. Explicitly, the radiated power~$P_\T{rad}$ over the total power~$P_\T{tot}$ supplied to the system is maximized
	\begin{equation}\label{eq:optim}
		\eta_\T{rad}^\T{ub} = \max \limits_\Ivec \eta_\T{rad},
	\end{equation}
	where
	\begin{equation}
		\eta_\T{rad} = \dfrac{P_\T{rad}}{P_\T{tot}} = \dfrac{P_\T{rad}}{P_\T{rad} + P_\T{ant} + P_\T{tis}},
	\end{equation}
	with~$P_\T{ant}$ being the cycle-mean lost power in the current supporting region~$\varOmega_Z$ and $P_\T{tis}$ being the cycle-mean lost power in region~$\varOmega_T$.
	
	The individual power terms are defined within the hybrid method as follows
	\begin{align}
		P_\T{rad} &= \dfrac{1}{2} \M{f}_1^\herm \M{f}_1, \label{eq:Prad} \\
		P_\T{ant} &= \dfrac{1}{2} \Ivec^\herm \RE \left[ \Zmat_\rho \right] \Ivec, \label{eq:Pant} \\
		P_\T{tis} &= \dfrac{1}{2} \left(\M{f}_2^\herm \M{f}_2 + \RE \left[ \M{a}_2^\herm \M{f}_2 \right] \right) - P_\T{rad}. \label{eq:Ptis}
	\end{align}
	Note that the first term on the right-hand side of~\eqref{eq:Ptis} is the net cycle-mean outward power flow at the inner boundary of~$\varOmega_T$, see~\cite[Appendix~E]{2022_Losenicky_TAP} for derivation. 
	
	As mentioned above, the current density expansion coefficients~$I_n$ are the optimization variables, and the first row of system~\eqref{eq:system} $\left( \Zmat_0 + \Zmat_\rho \right) \Ivec - \M{U}_1^\trans \M{a}_2 = \Vvec_\T{i}$ is therefore not imposed\footnote{Note that equation system~\eqref{eq:system} has a unique solution, which binds the excitation vector with current vector. In order to set up an upper limit to radiation efficiency of all antennas fitting the prescribed current supporting region, this unicity must be relaxed. One of the possible ways is to skip the first row of equation system~\eqref{eq:system}, which gives the necessary freedom to the current vector~$\Ivec$. But, particular lines in the first row can be used as power constraints, to further tighten the bounds and to impose constraints on input impedance, the shape of the feed, \textit{etc}.}~\cite{JelinekCapek_OptimalCurrentsOnArbitrarilyShapedSurfaces}. Optimization only relies on the boundary conditions between different regions, which are represented by other rows. Specifically, the equation system
	\begin{equation}\label{eq:linCon}
		\begin{bmatrix}
			-\M{U}_1  & \M{0} & -\M{1} & \M{0} \\
			\M{0}  & -\M{1} & \M{\Gamma} & \M{0} \\
			\M{0}  & \M{0} & \M{\Psi} & -\M{1}
		\end{bmatrix}  \begin{bmatrix}
			\Ivec \\ \M{a}_2 \\ \M{f}_2 \\ \M{f}_1
		\end{bmatrix} = \begin{bmatrix}
			\M{0} \\ \M{0} \\ \M{0}
		\end{bmatrix}
	\end{equation}
	is used as an affine constraint in the optimization problem, while the current density vector~$\Ivec$ is optimized so as to maximize radiation efficiency. 
	
	As seen in~\eqref{eq:Prad}~--~\eqref{eq:Ptis}, all the power terms are represented as quadratic functions of expansion coefficients. Converting all terms~\eqref{eq:Prad}~--~\eqref{eq:Ptis} to the functionals of vectors~$\Ivec$ via~\eqref{eq:linCon}, optimization problem~\eqref{eq:optim} is transformed to \acf{QCQP}. Its solution can be approached numerically~\cite{Liska_etal_FundamentalBoundsEvaluation}, but in this particular case, it can be solved more efficiently. After several algebraic manipulations and substitutions from~\eqref{eq:linCon} to~\eqref{eq:Prad}--\eqref{eq:Ptis}, the optimization problem~\eqref{eq:optim} reduces to \ac{GEP}
	\begin{equation}
		\left( \M{U}_1^\trans \M{\Psi}^\trans \M{\Psi} \M{U}_1 \right) \Ivec = \eta \left(\RE \left[\Zmat_0 +  \Zmat_\rho \right] + \M{U}_1^\trans \M{\Gamma} \M{U}_1 \right) \Ivec,
	\end{equation}
	where the left-hand-side matrix is proportional to the cycle-mean radiated power~\eqref{eq:Prad} and the right-hand side to the cycle-mean total supplied power. The maximum radiation efficiency~$\eta_\T{rad}^\T{ub}$ is given by the maximum eigenvalue~$\eta$, and the optimal current density corresponds to the associated eigenvector~$\Ivec$.
	
	\section{Different Kind of Losses}\label{sec:losses}
	Implanted antennas suffer from dissipation of power in two different regions: the antenna itself and the host body. So far, only idealized \ac{EM} dipole sources were considered in~\cite{Skrivervik_Bosiljevac_Sipus_FundamentalBoundsForImplantedAntennas, Nikolayev+etal2019}, concluding that elementary electric dipole suffers more from tissue losses than the magnetic source, mainly due to the coupling of the near field and the lossy tissues. In this section, we show that the metal losses of a given current supporting region should also be taken into account.
	
	Let us consider two frequencies within standard medical bands: $f_1 = 403\,\T{MHz}$ and $f_2 = 2.45\,\T{GHz}$. Further, let us consider the following example: the current supporting region is a copper disc with radius~$r$ swept from 0.75\,mm to 7.5\,mm  placed in a centered spherical air bubble with radius $a = 7.5\,\T{mm}$ and implanted in spherical muscle ($\varepsilon_\T{r,m} = 57 - 36\J$ at frequency~$f_1$ and~$\varepsilon_\T{r,m} = 53 - 13\J$ at frequency~$f_2$) phantom with radius $R = 61.5\,\T{mm}$. The current supporting region is supposed to be made of copper with a thickness much higher than skin depth and, therefore, can be modeled by a sheet of surface resistivity~\cite{Pozar_MicrowaveEngineering}
	$R_\T{s} = \sqrt{(2 \pi f \mu_0)/(2 \sigma)}$,
	where conductivity~$\sigma = 5.56 \cdot 10^7 \, \T{S/m}$ and permeability of vacuum~$\mu_0$ are assumed. The upper bound on radiation efficiency is computed for different radii $r$ and is shown in Fig.~\ref{fig:effMuscle}.
	% Figure environment removed
	
	Fig.~\ref{fig:effMuscle} depicts, at each frequency, the optimal radiation efficiency as a function of normalized radius~$r/a$. This efficiency cannot be overcome by any real antenna fitting the considered disc region. An interesting phenomenon is an abrupt growth in radiation efficiency at frequency~$f_1$ for support sizes larger than $r/a=0.32$. For current supports up to a radius of $r/a = 0.5$, it is convenient to use frequency $f_2$ with stable performance among different radii. In contrast, for larger sizes, the first frequency is preferred.
	
	A better physical insight into the aforementioned phenomenon is attained by analyzing the power quantities in~\eqref{eq:Prad}~--~\eqref{eq:Ptis} normalized to the radiated power~$P_\T{rad}$, i.e., dissipation factors of the current support/tissue
	\begin{equation}
		\delta_\T{ant/tis} = \dfrac{P_\T{ant/tis}}{P_\T{rad}}, \label{eq:deltaAntTis}
	\end{equation}
	and the total dissipation factor
	\begin{equation}\label{eq:deltaRad}
		\delta_\T{rad} = \delta_\T{ant} + \delta_\T{tis} = \dfrac{1 - \eta_\T{rad}}{\eta_\T{rad}}.
	\end{equation}
	The comparison of all dissipation factors is shown in Fig.~\ref{fig:disMuscle}.
	
	At frequency~$f_1$, the dissipation factors $\delta_\T{ant}$ and $\delta_\T{tis}$ are discontinuous, with the critical point being the radius $r/a=0.32$. For smaller sizes, tissue loss dominates, while the antenna loss is comparable for larger sizes. At frequency~$f_2$, all curves are smooth, and antenna loss is negligible as compared to tissue loss.
	% Figure environment removed
	
	The behavior of power losses can be explained by observing the optimal current density for the case of small and large current supporting regions. For the smaller size, the optimal current densities are shown in the left panels of Fig.~\ref{fig:optCurrentsMuscle}.
	% Figure environment removed
	The current densities are both $\T{TM}_{1m}$-like\footnote{Current density is denoted as $\T{TE}_{1m}$-like (magnetic dipole), $\T{TM}_{1m}$-like (electric dipole), or $\T{TM}_{2m}$-like (electric quadrupole) if its projection to \ac{VSWE} generates only $\T{TE}_{1m}$~mode, $\T{TM}_{1m}$~mode, or $\T{TM}_{2m}$~mode, respectively. Subscript $m$ corresponds to the azimuthal number and only reflects a rotation of a coordinate system.}. For the larger size of the current supporting region, the optimal current density is, see the right panels in Fig.~\ref{fig:optCurrentsMuscle}, $\T{TE}_{1m}$-like at frequency~$f_1$ and is $\T{TM}_{1m}$-like at frequency~$f_2$. This explains the discontinuity in dissipation factors at frequency~$f_1$, where the optimal current must dramatically switch its character to maximize radiation efficiency. The reason for this is that $\T{TE}_{1m}$-like current couples less to the lossy dielectric host body~\cite{Skrivervik_Bosiljevac_Sipus_FundamentalBoundsForImplantedAntennas} and therefore become favorable when the boarders of the current supporting region come closer to the tissue. While small-size $\T{TE}_{1m}$-like currents can be seen as an analogy to a short circuit~\cite{Algarp2022}. At frequency~$f_2$, this is not necessary since the electrical size of the current support is big enough to accommodate $\T{TM}_{1m}$-like current with an optimal trade-off antenna loss due to dense current in the center and tissue loss due to short distance from the tissues. But, in general, antenna loss decreases with the increasing $r$ for $\T{TM}_{1m}$-like and $\T{TE}_{1m}$-like currents because more spreading of the current is allowed.
	
	%%%%%%%%%%%%%%%%%%%%%%%%%%%%%%%%%%%%%%%%%%%%%%%%%%%%%%%%%%%%%%%%%%%%%%%%%%%%%%%%%%%%%%%%%%%%%%%%%%%%%
	\section{Experimental Verification}\label{sec:verification}
	The following two sections show how to use the previously developed methodology in the real design of an implanted antenna. This provides an experimental validation of the method and also demonstrates some new features that were not observed previously.
	
	To allow for a simple experimental setup, the host body is simulated by a glass bottle filled with distilled water, see Fig.~\ref{fig:photo} and Fig.~\ref{fig:setupWater}. Distilled water has well-determined \ac{EM} properties ($\varepsilon_\T{r,w} = 78 - 12\J$ at frequency~$f_2$) and similarly well defined are \ac{EM} properties of glass ($\varepsilon_\T{r,g} \approx 6.7$). These values are also used by commercial \ac{EM} solver \ac{CST}, which is employed as a virtual version of the experiment. The outer radius of the bottle is $R_1 = 63.5\, \T{mm}$ and glass thickness is $t = 2.0\,\T{mm}$.
	% Figure environment removed
	% Figure environment removed
	
	The implanted antenna, see Fig.~\ref{fig:ant_parameter}, is designed and measured for frequency $f_2$. It consists of a $100$\,µm thick polyimide disc ($\varepsilon_\T{r} = 3.5$) with $18$\,µm thick copper cladding and two ceramic hemispheres as its encapsulation. A meander dipole antenna is etched in the copper. The antenna is connected to a copper jacketed semi-rigid cable (EZ-34, EZ Form Cable) through a chip balun (2450BL15B050, Johanson Technology) and a short coplanar stripline. A slot in the top ceramic hemisphere is machined to accommodate the balun and the cable. The ceramic hemispheres are made of $\T{Al_2O_3}$ (alumina, $\varepsilon_\T{r} = 12$ and $\T{tan}\delta = 0.0003$, \ac{CST} model) that is bio-compatible and also provides good matching of the antenna.
	% Figure environment removed
	
	At first, the above-mentioned antenna setup is used to validate the hybrid method used to define the fundamental bound. For this aim, the gain pattern of the antenna was measured in an anechoic chamber, see Fig.~\ref{fig:gainPattern} and appendix~\ref{app:meas} for details. Apart from the measurement and the hybrid method, the gain patterns were also evaluated in commercial solvers: \ac{CST} and \ac{FEKO}. All simulations omit the thin, flexible substrate and losses in the ceramic encapsulation. All the gain patterns are depicted in Fig.~\ref{fig:gainPattern} and show excellent agreement. The maximum gain values $G(\pi/2, 0)$ are $-16.9 \, \T{dB}$ for simulated pattern by the hybrid method and using \ac{AToM}, $-16.9 \, \T{dB}$ in \ac{CST},  $-16.5 \, \T{dB}$ in \ac{FEKO}, and $-17.6 \, \T{dB}$ for the measurement. The computational requirements of the commercial solutions (using the FDTD method in CST and the surface equivalence formulation in FEKO) were considerable due to the use of high dielectric constants and overall large electric size. The data presented in Fig.~\ref{fig:gainPattern} demanded hours of computer time. Compared to this, the hybrid method demanded only a few seconds and order in magnitude less computer memory for the same task with the same discretization of the dipole antenna. Even for the maximum possible meshing allowed by the used computer, the results from the commercial solvers were still slightly mesh dependent, and the hybrid method is therefore considered as more reliable for this task.
	% Figure environment removed
	
	The radiation efficiency of the experimental setup, see~Fig.~\ref{fig:effWater}, is also compared with the performance of a loop and dipole antenna, which were made of a copper strip of width $w = 0.5 \, \T{mm}$ and fed by an ideal delta-gap voltage source. The results once more show good agreement between several numerical simulations, with only slight variations attributed to different numerical schemes. Importantly, the figure also shows the fundamental bound on radiation efficiency with a current supporting region in the form of a disc accommodating all used antenna models.
	% Figure environment removed
	
	The performance of the experimental setup lies only approximately $1\,\T{dB}$ below the fundamental bound. This means that for this material composition, and shape of the design region (disc), not much can be done to improve the design. The hybrid method shows that the interaction with the host material is mostly dipolar and therefore does not depend much on the detailed composition of the host body. It can be expected that this design will similarly perform also in real-body conditions.
	
	%%%%%%%%%%%%%%%%%%%%%%%%%%%%%%%%%%%%%%%%%%%%%%%%%%%%%%%%%%%%%%%%%%%%%%%%%%%%%%%%%%
	\section{Discussion}\label{sec:investigation}
	In order to understand the design principles of implanted antennas in more detail, it is worth studying the simple cases of loop and dipole antennas in Fig.~\ref{fig:effWater}. The dipole antenna has a positive trend in radiation efficiency with increasing length, while the loop antenna achieves its maximum performance around the radius of~$r = 0.6a$. These trends can be explained by separating dissipation factors into three contributions as depicted in Fig.~\ref{fig:disWaterDL}. With the increasing length of the dipole antenna, the tissue dissipation factor is constant, but on the contrary, the antenna dissipation factor declines, which is inherited by the total dissipation factor. The antenna dissipation factor of the circular loop antenna completely dominates for the small radius up to $r/a = 0.3$. Antenna dissipation factor decay is faster with the increasing radius for the circular loop than for the straight dipole antenna due to the increase of the impedance. However, with a circumscribing sphere radius of $0.5a$, the circumference of the circular loop antenna is comparable to the wavelength in the surrounding medium, and a further increase stops the decrease of the antenna dissipation factor and increase of tissue dissipation factor.
	% Figure environment removed
	
	Due to high antenna losses, the performance gap (distance to fundamental bound) for small electrical sizes is significant for straight dipole antennas (being comparable to tissue losses) and even wider for loop antennas. To see the reason for this behavior, it is instructive to plot how the dissipation factors, see Fig.~\ref{fig:disWater}, corresponding to the fundamental bound, behave when changing the size of the current supporting region.
	% Figure environment removed
	The total dissipation factor is approximately constant, and it is dominated by the tissue dissipation factor. While the antenna dissipation factor is discontinuous at $r/a \approx 0.35$. As can be seen in Fig.~\ref{fig:optCurrentsWater}, for small electric sizes, $\T{TM}_{1m}$-like current realizes the optimal current density. In contrast, $\T{TM}_{2m}$-like current is optimal at higher electric sizes.  A switch between these two current profiles is represented by an abrupt jump seen in~Fig.~\ref{fig:disWater}. This critical point at $r/a \approx 0.35$ happens because of a slight improvement in radiation efficiency: a small drop in tissue dissipation factor and an increase in antenna dissipation factor; the radiation on an implanted antenna must always be seen as a trade-off between the antenna and tissue dissipation factors. The $\T{TM}_{2m}$-like current shows that in certain scenarios, antennas using higher-order modes can perform better than the ones using only the dipole currents.
	% Figure environment removed
	
	%%%%%%%%%%%%%%%%%%%%%%%%%%%%%%%%%%%%%%%%%%%%%%%%%%%%%%%%%%%%%%%%%%%%%%%%%%%%
	\section{Conclusion}
	The use of the hybrid method for the computation of radiation efficiency limitations for implanted antennas is shown in this paper. The hybrid method not only allows for a dramatic reduction of computational demands (order in magnitude faster than commercial solutions of the same setup) but also reveals different contributions of losses in the problem of an antenna radiating through lossy media. The technique is applied to a disc-like current supporting region that is encapsulated in a dielectric and placed in the host body, i.e., a setup able to accommodate many of the practical implanted antenna designs. The paper shows how the upper bound on the radiation efficiency of all these designs can simply be evaluated. Furthermore, the paper reveals the importance of ohmic loss in the antenna proper compared to tissue loss.
	
	Varying the size of the current supporting region, the trade-off between ohmic losses in the antenna and the tissue drastically changes the shape of the optimal current density. The critical points were identified where the $\T{TE}_{1m}$-like current density performs better than a $\T{TM}_{1m}$-like current and vice-versa. This proves the importance of the fundamental limitations on the performance of implanted antennas and the usefulness to evaluate them for the scenario before starting the design to know the bounds.
	
	The method is verified by measurement as well as computation in commercial solvers, with the example being a meander dipole with a balun immersed in a phantom filled with distilled water. Great agreement is achieved for the results from the hybrid method, commercial simulators, and measurement. A detailed study of the fundamental bound on radiation efficiency in this scenario has also been done, concluding that ohmic loss in an implanted antenna must be considered in the design.
	
	% if have a single appendix:
	%\appendix[Proof of the Zonklar Equations]
	% or
	%\appendix  % for no appendix heading
	% do not use \section anymore after \appendix, only \section*
	% is possibly needed
	
	% use appendices with more than one appendix
	% then use \section to start each appendix
	% you must declare a \section before using any
	% \subsection or using \label (\appendices by itself
	% starts a section numbered zero.)
	%
	
	
	\appendices
	\section{Measurement} \label{app:meas}
	The implanted meander dipole antenna within a glass bottle filled with distilled water gain pattern is measured at a distance of 2.1\,m from the reference antenna (a quad-ridge horn antenna, QH400, MVG~Industries), see Fig.~\ref{fig:photo}. The meander dipole antenna is oriented along the $y$ axis.
	
	The reflection and transmission coefficients were measured from 2\,GHz to 3\,GHz, see Fig.~\ref{fig:mS11S12}, evidencing good matching of the meander dipole antenna at frequency~$f_2$.
	% Figure environment removed
	At studied frequency~$f_2$, measured magnitudes of reflection and transmission coefficients are $|s_{11}| = -18 \, \T{dB}$ and $|s_{12}| = -54 \, \T{dB}$, respectively. The transmission coefficient was measured in the direction of gain pattern maximum $\theta = \pi/2$ and $\varphi = 0$. In all measurements, the vector network analyzer was calibrated to the reference plane lying at coaxial ports placed at the end of a semi-rigid cable feeding the antenna.
	
	The gain pattern of the implanted meander dipole antenna, $G(\theta, \varphi) = \eta_\T{rad} D(\theta, \varphi)$, is obtained from the measured transmission coefficient by excluding the gain of the reference antenna and the path loss between the two antennas, which was calibrated as $-36.10$\,dB in total at frequency~$f_2$ in the measurement setup. The measured gain pattern was further corrected by the balun insertion loss of 0.30\,dB and the cable loss of 0.35\,dB (cable connected from the measurement reference plane to the balun).
	
	% you can choose not to have a title for an appendix
	% if you want by leaving the argument blank
	%\section{}
	%Appendix two text goes here.
	
	
	% Can use something like this to put references on a page
	% by themselves when using endfloat and the captionsoff option.
	\ifCLASSOPTIONcaptionsoff
	\newpage
	\fi
	
	
	
	% trigger a \newpage just before the given reference
	% number - used to balance the columns on the last page
	% adjust value as needed - may need to be readjusted if
	% the document is modified later
	%\IEEEtriggeratref{8}
	% The "triggered" command can be changed if desired:
	%\IEEEtriggercmd{\enlargethispage{-5in}}
	
	% references section
	
	% can use a bibliography generated by BibTeX as a .bbl file
	% BibTeX documentation can be easily obtained at:
	% http://mirror.ctan.org/biblio/bibtex/contrib/doc/
	% The IEEEtran BibTeX style support page is at:
	% http://www.michaelshell.org/tex/ieeetran/bibtex/
	\bibliographystyle{IEEEtran}
	% argument is your BibTeX string definitions and bibliography database(s)
	\bibliography{references.bib}
	
	
	% BIOGRAPHY
	% =============================================================================
	\begin{IEEEbiography}[{% Figure removed}]{Jakub Liska}
		received his B.Sc. and M.Sc. degrees in electrical engineering from the Czech Technical University in Prague, Czech Republic, in 2019 and 2021, respectively. He is currently continuing towards a Ph.D. degree in electrical engineering at the same university.
		
		His research interests include \ac{EM} field theory, fundamental bounds, computational electromagnetics, numerical and convex optimization, numerical techniques, and eigenproblems.
	\end{IEEEbiography}
	
	\begin{IEEEbiography}[{% Figure removed}]{Mingxiang Gao}
		received the double B.Sc. degree in electrical engineering and business administration and the M.Sc. degree in electrical engineering from the Xi’an Jiaotong University, Xi’an, China, in 2016 and 2019, respectively, and the M.Sc. degree (summa cum laude) in electrical engineering from the Politecnico di Milano, Milan, Italy, in 2019. He is currently working towards the Ph.D. degree in electrical engineering at the École Polytechnique Fédérale de Lausanne, Lausanne, Switzerland. His current research interests include the theory and design of implantable antennas, wireless techniques for bioelectronics, and bioelectromagnetics.
	\end{IEEEbiography}
	
	\begin{IEEEbiography}[{% Figure removed}]{Lukas Jelinek}
		received his Ph.D. degree from the Czech Technical University in Prague, Czech Republic, in 2006. In 2015 he was appointed Associate Professor at the Department of Electromagnetic Field at the same university.
		
		His research interests include wave propagation in complex media, electromagnetic field theory, metamaterials, numerical techniques, and optimization.
	\end{IEEEbiography}
	
	\begin{IEEEbiography}[{% Figure removed}]{Erik R. Algarp} received his M.Sc. degree in Electrical Engineering from KTH Royal Institute of Technology, Sweden, in 2022. He is currently pursuing a Ph.D degree in Electrical Engineering within the Microwaves and Antennas group at École polytechnique fédérale de Lausanne. 
	\end{IEEEbiography}
	
	\begin{IEEEbiography}[{% Figure removed}]{Anja K. Skrivervik} received the master’s and Ph.D. degrees in electrical engineering from the École Polytechnique Fédérale de Lausanne (EPFL), Lausanne, Switzerland, in 1986 and 1992, respectively.
		
		After a stay at the University of Rennes, Rennes, France, as an Invited Research Fellow and two years in the industry, she returned part-time to EPFL as an Assistant Professor in 1996, where she is currently a Professeur Titulaire, the Head of the Microwave and Antenna Group. She was the Director of the EE Section from 1996 to 2000, and she is currently the Director of the EE Doctoral School at EPFL. She is also a Visiting Professor at the University of Lund. Her teaching activities include courses on microwaves and antennas, and she teaches at bachelor’s, master’s, and Ph.D. levels. She is the author or co-author of more than 200 peer-reviewed scientific publications. Her research interests include electrically small antennas, antennas in biological media, periodic structures, reflect and transmit arrays, and numerical techniques for electromagnetics.
		
		Dr. Skrivervik is a Board Member of the European School on Antennas and is frequently requested to review research programs and centers in Europe. She has been a member of the Board of Directors of the European Association on Antennas and Propagation (EurAAP) since 2017. She received the Latsis Award. She is very active in European collaboration and European projects. She was the Chairperson of the Swiss URSI until 2012. She was the General Chair of the Loughborough Antenna and Propagation Conference in 2015, the Vice-Chair and Technical Program Committee-Chair of the EuCAP 2016 Conference, and the Financial Chair of EuCAP 2017 to EuCAP 2022. 
	\end{IEEEbiography}
	
	\begin{IEEEbiography}[{% Figure removed}]{Miloslav Capek}
		(M'14, SM'17) received the M.Sc. degree in Electrical Engineering 2009, the Ph.D. degree in 2014, and was appointed Associate Professor in 2017, all from the Czech Technical University in Prague, Czech Republic.
		
		He leads the development of the AToM (Antenna Toolbox for Matlab) package. His research interests are in the area of electromagnetic theory, electrically small antennas, antenna design, numerical techniques, and optimization. He authored or co-authored over 130~journal and conference papers.
		
		Dr. Capek is the Associate Editor of IET Microwaves, Antennas \& Propagation. He was a regional delegate of EurAAP between 2015 and 2020 and an associate editor of Radioengineering between 2015 and 2018.
	\end{IEEEbiography}
	
\end{document}


