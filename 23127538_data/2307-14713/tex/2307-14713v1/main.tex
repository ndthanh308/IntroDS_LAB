\documentclass[10pt,twocolumn,letterpaper]{article}

%\usepackage{ijcb}
\usepackage{ijcb}
%\usepackage{cvpr}
\usepackage{times}
\usepackage{epsfig}
\usepackage{graphicx}
\usepackage{amsmath}
\usepackage{amssymb}
\usepackage{multirow}
\usepackage{multicol}

\usepackage{color, colortbl}
\definecolor{LightCyan}{rgb}{0.88,1,1}

% Include other packages here, before hyperref.

% If you comment hyperref and then uncomment it, you should delete
% egpaper.aux before re-running latex.  (Or just hit 'q' on the first latex
% run, let it finish, and you should be clear).
%\usepackage[pagebackref=true,breaklinks=true,letterpaper=true,colorlinks,bookmarks=false]{hyperref}

\ijcbfinalcopy % *** Uncomment this line for the final submission

\def\ijcbPaperID{183} % *** Enter the IJCB Paper ID here
\def\httilde{\mbox{\tt\raisebox{-.5ex}{\symbol{126}}}}

\usepackage{algorithm}
\usepackage{algpseudocode}
\usepackage{mathtools}
\usepackage{pifont}
\usepackage{booktabs}
\newcommand{\cmark}{\ding{51}}%
\newcommand{\xmark}{\ding{55}}%
\usepackage{todonotes}
\usepackage[bb=boondox]{mathalfa}
\usepackage{multirow}

\DeclareMathOperator*{\argmin}{arg\,min} % Jan Hlavacek
% \DeclareMathOperator*{\arg}{arg} % Jan Hlavacek
\DeclareMathOperator*{\argmax}{arg\,max} % Jan Hlavacek
\algnewcommand{\LeftComment}[1]{\Statex \(\triangleright\) #1}

% Pages are numbered in submission mode, and unnumbered in camera-ready
\ifijcbfinal\pagestyle{empty}\fi

\begin{document}

%%%%%%%%% TITLE
\title{GaitMorph: Transforming Gait by Optimally Transporting Discrete Codes}

\author{Adrian Cosma, Emilian Rădoi\\
University Politehnica of Bucharest, Bucharest, Romania\\
{\tt\small cosma.i.adrian@gmail.com,  emilian.radoi@upb.ro}
% For a paper whose authors are all at the same institution,
% omit the following lines up until the closing ``}''.
% Additional authors and addresses can be added with ``\and'',
% just like the second author.
% To save space, use either the email address or home page, not both
% \and
% Emilian Radoi\\
% Politehnica University of Bucharest\\
% Bucharest, Romania\\
% {\tt\small emilian.radoi@upb.ro} \\
}



\maketitle
\thispagestyle{empty}

%%%%%%%%% ABSTRACT
\begin{abstract} 
   Gait, the manner of walking, has been proven to be a reliable biometric with uses in surveillance, marketing and security. A promising new direction for the field is training gait recognition systems without explicit human annotations, through self-supervised learning approaches. Such methods are heavily reliant on strong augmentations for the same walking sequence to induce more data variability and to simulate additional walking variations. Current data augmentation schemes are heuristic and cannot provide the necessary data variation as they are only able to provide simple temporal and spatial distortions. In this work, we propose GaitMorph, a novel method to modify the walking variation for an input gait sequence. Our method entails the training of a high-compression model for gait skeleton sequences that leverages unlabelled data to construct a discrete and interpretable latent space, which preserves identity-related features. Furthermore, we propose a method based on optimal transport theory to learn latent transport maps on the discrete codebook that morph gait sequences between variations. We perform extensive experiments and show that our method is suitable to synthesize additional views for an input sequence.
\end{abstract}

%%%%%%%%% BODY TEXT
\section{Introduction}
\section{Introduction}
%the Introduction section need to be improved (writting & arguments, and reduce non-sense words)
Air quality forecasting using data-driven models has gained significant attention in recent years, thanks to the proliferation of data collection infrastructures such as sensor stations and advancements of telecommunication technologies. These infrastructures are typically managed by national institutes (e.g., AirParif\footnote{https://www.airparif.asso.fr/}, EPA\footnote{https://www.epa.gov/air-quality}) or large companies (e.g., PurpleAir\footnote{https://www2.purpleair.com/}) that specialize in air quality monitoring or forecasting services and products. Leveraging existing data collection infrastructures proves beneficial for initial research exploration or validating product prototypes.
However, reliance on fixed infrastructures presents practical constraints when customization is required for specific tasks. For instance, certain monitoring areas may be inadequately covered or completely absent from the existing infrastructures, or the density of coverage may not be sufficient. This issue particularly affects small or mid-sized industrial and academic players who face budget limitations that prevent them from investing in their own infrastructure from scratch, but have specific customization needs.

% give the motivation from another aspect: incrementally built infrastructure, no need to re-train the model
In addition to data collection, air quality forecasting models trained solely with data from public fixed infrastructures may not perform well for users' specific scenarios, such as forecasting at a higher spatial resolution. Deploying additional sensors as a cost-effective solution can enrich the data and improve forecasting performance without the need to build infrastructures from scratch. 
Subsequently, this targeted solution leads us to consider the practical question: \textit{how we can make use of the data collected from existing infrastructures, when integrating new sensor infrastructures?} 
%which can be equipped on fixed sensor stations or moving objects (e.g., drones) with a higher flexibility.

% Figure environment removed

As depicted in Figure \ref{fig:research_background}, the topological sensor network may change as the urban infrastructure evolves, resulting in varying network structures of air quality sensors. The data collected from the network $G_{\tau}$ needs to be augmented with enriched data from newly installed sensors $\Delta G_{\tau'}$ and $\Delta G_{\tau''}$. Training a model solely on recent data with $G_{\tau''}$ would overlook valuable information contained in the historical data with $G_{\tau}$ and $G_{\tau'}$.

In this paper, we propose an expandable graph attention network (EGAT) that effectively integrates data with various graph structures. This approach is versatile and can be seamlessly embedded into any existing air quality forecasting model. Furthermore, it applies to scenarios where sensors are not installed, enabling accurate forecasting in such areas.
We summarize our approach's main advantages as follows:
\begin{itemize}
    %\item \textbf{Air quality forecasting in real scenarios:} we consider the complex data quality issues, e.g., missing values

    \item \textbf{Less is more:} With fewer installed sensors, we can directly predict the air quality of other unknown area where sensors are not installed and achieve comparable performance to models relying on extensive data collection infrastructures with more sensors.
    \item \textbf{Continual learning with self-adaptation:} The proposed model enables continuous learning from newly collected data with expanded sensor networks, demonstrating self-adaptability to different topological sensor networks.
    \item \textbf{Embeddable module with scalability:} The proposed module can be seamlessly integrated into any air quality forecasting model, enhancing its ability to forecast in real-world scenarios.

\end{itemize}

The rest of this paper starts with a review of the most related work. Then, we formulate the problems of the paper. Later, we present in detail our proposal, which is followed by the experiments on real-life datasets and the conclusion.



\section{Related Work}
\section{Related Work}
\subsection{Air Quality Forecasting}
% Air quality monitoring and forecasting: data-driven and model-driven approaches
Data-driven models for air quality forecasting has gained a huge popularity recently. Recent work \cite{zuo2023graph,liang2022airformer} studies graph-based representations of the air quality data by considering the sensor network as a graph structure, which extracts decent structural features between sensor data from a topological view. The air quality forecasting can be then formulated as a spatio-temporal forecasting problem.

% work on general spatio-temporal forecasting tasks, and what make Air Quality forecasting different? 
Works like DCRNN~\cite{li2018diffusion}, STGCN~\cite{yu2018spatio} and Graph WaveNet~\cite{wu2019graph}, have shown promising results in traffic forecasting tasks. These models can be adapted to air quality forecasting tasks owing to the shared spatio-temporal features present in the data. 
%The key differences lie in two aspects: i) Air quality exhibits continuous state changes between areas, while traffics change abruptly between nodes; ii) External factors impacting air quality are more complex, which can be from human activities, or cased by natural environment. 
% the typical research problems considered in air quality forecasting: multiple data sources -> data fusion -> forecasting target 
%DeepAir~\cite{yi2018deep}: spatial transformation for various sensor readings, a fusion network to model the relationships between different factors and AQIs.
%AirFormer~\cite{liang2022airformer}: self-attention for learning spatio-temporal representations; capture the intrinsic uncertainty of air quality data.
% Challenges/constrains of previous air quality forecasting models
However, in practice, the above-mentioned models often overlook the evolving nature of sensor networks as more data collection infrastructures are incrementally built. Consequently, these models require re-training from scratch on the most recent data that reflects the evolved sensor network. It may result in the loss of valuable information contained in outdated data collected from different network configurations.
%or using the previous model to do approximate inference of unlearned areas relying on neighbor predictions. 

% Learning models
%Some work models the air pollution in the whole city with an image-based representation. However, this representation may not be ideal, as air pollution and other impact factors have natural graph structures. 
% Challenges: huge amount of sensors are required

\subsection{Expandable Graph Neural Networks}
%Dynamic graph structures, meaning at different time stamps, the graph structure can be different. Although various works have studied the dynamic graphs with a focus on the dynamic edge weight but over a fixed set of graph nodes, they barely consider a dynamic graph with evolving number of nodes. 

In the field of graph learning, several works, such as ContinualGNN \cite{wang2020streaming} and ER-GNN \cite{zhou2021overcoming}, have incorporated the concept of Continual Learning to capture the evolving patterns within graph nodes.
While these approaches are valuable, it is important to consider spatio-temporal features in air quality forecasting tasks.
Designed for traffic forecasting, TrafficStream~\cite{chen2021trafficstream} considers evolving patterns on both temporal and spatial axes; ST-GFSL~\cite{lu2022spatio} introduces a meta-learning model for cross-city spatio-temporal knowledge transfer. However, these works primarily focus on shared (meta-)knowledge between nodes, and give less attention to expandable graph structures. 
Basically, spectral-based graph neural networks (GNNs) face challenges when scaling to graphs with different structures due to the complexity of reconstructing the Laplacian matrix. To address this issue, our paper explores the use of spatial-based GNNs, such as Graph Attention Networks (GAT) \cite{velivckovic2018graph}, for expandable graph learning in air quality forecasting tasks.
%ContinualGNN~\cite{wang2020streaming}: streaming GNN considering continual learning of new patterns and preservation of existing patterns.
%ER-GNN~\cite{zhou2021overcoming}: consider catastrophic forgetting problems using Memory Replay.

%ST-GFSL~\cite{lu2022spatio}: meta-learning for cross-city knowledge transfer, the graphs of different cities can be different. 

%GAT~\cite{velivckovic2018graph}; 

%TrafficStream~\cite{chen2021trafficstream}: evloving patterns on both temporal axis and spatial axis (expanded graphs) 




\section{Method}
The use of a VQ-VAE \cite{van2017neural} for learning a latent walking representation for skeletons is motivated by the discrete nature of the latent embeddings, which simplifies the constraint optimization for morphing between walking variations. While the VQ-VAE is widely used in generative modelling \cite{zhang2023t2m,xie2022vector,liu2022learning}, other algorithms such as diffusion models \cite{zhang2022motiondiffuse} might offer higher quality reconstructions. However, our aim is not to generate walking sequences, but to manipulate the latent space to change existing walks into desired variations.

In this section, we describe the main components of GaitMorph: we describe the pretraining dataset used for training the VQ-VAE, the architecture and training procedure for the VQ-VAE, and the morphing algorithm based on optimal transport between the latent codes. 

\subsection{Pretraining Dataset}

\label{sec:dataset}

In order to train a sufficiently large and general autoencoder model, we assess that current gait datasets are too small. Even though datasets such as DenseGait \cite{cosma22gaitformer} and GREW \cite{grew} are collected "in-the-wild" outdoor environments using surveillance cameras, they nonetheless lack some walking registers such as treadmill walking, more aggressive camera angles and indoor environments. However, by combining the major large-scale gait datasets into a single dataset, we can ensure more diversity of walking registers. In Table \ref{tab:walkpile} we showcase the existing datasets that comprises our pretraining dataset. We used \textbf{DenseGait} \cite{cosma22gaitformer} and \textbf{GREW} \cite{grew}, two similar in-the-wild datasets for their diverse walking sequences in outdoor environments, \textbf{OU-ISIR} \cite{ouisir} for more controlled walking in indoor and treadmill registers, and \textbf{Gait3D} \cite{zheng2022gait3d}, and indoor "in-the-wild" dataset collected in a supermarket setting. After concatenation of all skeleton sequences from the datasets, we obtain 875,543 walking sequences, totalling 1220.06 hours. To increase the size as much as possible, we also included the testing / distractor splits of each dataset whenever possible. We purposely did not include controlled, small scale datasets such as CASIA-B \cite{casia}, as we use them for downstream evaluation.

All walking sequences in this dataset are 2D skeletons in COCO pose format. We chose 2D poses to unify all datasets, as every dataset is providing 2D poses by default, while only some are also providing silhouettes or body meshes. Even though many gait processing models have good results using and appearance-based approach with silhouettes \cite{chao2019gaitset,lin2022gaitgl}, pose sequences only encode movement and abstract away any appearance information, preserving the privacy of walking individuals \cite{gaitgraph,cosma22gaitformer}. Skeleton sequences are a more interpretable and a plethora of models employ them for motion synthesis \cite{siyao2022bailando,zhang2023t2m,tevet2022motionclip,temos-eccv-2022}.

Skeleton sequences from each datasets are pre-processed in the same way. We filtered out skeletons that have too small or too large joint variance, which corresponds to static or erratic movement, respectively. We found that this procedure ensures that only properly moving skeletons are kept in the dataset. Furthermore, skeleton sequences are normalized and aligned at the pelvis, using the following formulae, considering that each of the $J = 18$ joints have $(x, y)$ coordinates:

\begin{equation*}
    \hat{x}_{joint} = \frac{x_{joint} - x_{pelvis}}{|x_{R.shoulder} - x_{L.shoulder}|}
    \label{eq:eq1}
\end{equation*}
\begin{equation*}
    \hat{y}_{joint} = \frac{y_{joint} - y_{pelvis}}{|y_{neck} - y_{pelvis}|}
    \label{eq:eq2}
\end{equation*}

\begin{table}[hbt!]
    \centering
    \resizebox{0.85\linewidth}{!}{
        \begin{tabular}{c|ccc}
            \textbf{Dataset} & \textbf{Split} & \textbf{\# Sequences} & \textbf{Duration (hr.)}\\
            \midrule\midrule
            \multirow{2}{*}{DenseGait \cite{cosma22gaitformer}} & Train & 217,954 & 614.75\\
            & Validation & 10,733 & 36.53 \\
            \midrule
    
            \multirow{3}{*}{GREW \cite{grew}} & Train & 102,888 & 175.92 \\
            & Test & 24,000 & 65.37\\
            & Distractor & 226,588 & 154.82\\
            \midrule
    
            \multirow{2}{*}{OU-ISIR \cite{ouisir}} & Train & 133,872 & 57.82 \\
            & Test & 134,199 & 57.92 \\
            \midrule
    
            \multirow{2}{*}{Gait3D \cite{zheng2022gait3d}} & Train & 18,940 & 42.70 \\
            & Test & 6,369 & 14.23 \\
            \midrule
    
            \textbf{Total} & & \textbf{875,543} &  \textbf{1220.06} \\
            
        \end{tabular}
    }
    \caption{Datasets that make up our pretraining dataset. We combined all the major in-the-wild and controlled datasets (including all splits) into a single, large-scale and diverse dataset. The dataset contains gait samples from a diverse set of walking registers, environments and camera angles. }
    \label{tab:walkpile}
\end{table}

In this manner, every skeleton sequence is aligned spatially and the differences in height and width of individuals are essentially eliminated. Consequently, only movement is encoded irrespective of the screen coordinates, distance to camera or appearance cues. We employ minimal augmentations to the skeleton sequences, adopting only random temporal cropping and walking pace modifications \cite{9721551,wang2020self,cosma22gaitformer}. We crop each skeleton sequence to be $T = 64$ frames long.

% Figure environment removed

\subsection{Learning the Walking Codebook}

In order to learn an informative and context-rich walking codebook, we leverage the expressive power of a Vector Quantized Variational AutoEncoder model (VQ-VAE) \cite{van2017neural}. The VQ-VAE model has been shown to be effective for a range of tasks, including image compression and generation \cite{esser2020taming,razavi2019generating}, and speech recognition \cite{van2017neural}. It is particularly useful in situations where the input data has a high degree of variability, and where traditional continuous latent space models may struggle to capture the underlying structure of the data. Furthermore, a discrete latent space enables a high degree of data compression, and allows the input data to be further processed as a sequence of discrete tokens. 

To properly encode skeleton sequences, we construct a skeleton autoencoder based on the MS-G3D \cite{liu2020disentangling} model. Figure \ref{fig:architecture} showcases the overall architecture of our method. MS-G3D is a powerful graph convolutional model that has state-of-the-art results in skeleton action recognition, surpassing other graph-based methods \cite{yan2018spatial,shi2019two} by a large margin. Graph convolutional models are well established in the field of skeleton sequence processing \cite{gupta2021quo} and were developed to properly handle spatial and temporal variation of the skeleton graph.

\subsubsection{MS-G3D Encoder-Decoder}

The encoder and decoder models for the skeleton autoencoder are both based on the MS-G3D architecture \cite{liu2020disentangling}. For simplicity, we did not perform any graph subsampling \cite{yan2019convolutional}, and only used temporal pooling to compress the skeleton sequence. We follow the official model implementation \cite{liu2020disentangling}, and adapt it for gait processing. Specifically, we changed all activations to GeLU \cite{hendrycks2016gaussian}, we removed the initial data batch-normalization since the skeletons were already normalized. Initial experiments showed that the default model was not large enough to reconstruct sequences other than the mean skeleton. Consequently, we doubled each convolution - batch normalization - activation block to increase model capacity. A MS-G3D model is composed of multiple Spatial-Temporal Graph Convolution (ST-GC) blocks. Each block consists of a Multi-scale Graph Convolution block (MS-GCN) and two Multi-Scale Temporal Convolutional blocks \cite{liu2020disentangling}. We used the default 6 G3D scales and 13 GCN scales for both the encoder and decoder models.

For the MS-G3D encoder $E(\cdot)$, we used 20 ST-GC encoder blocks. We used a feature map size of 64 for the first 5 blocks, 128 for the next 5 and 256 for the final 10. Temporal pooling is performed every 5 blocks. Therefore, for an initial skeleton sequence $x \in \mathbb{R}^{T \times J \times 2}$ consisting of $T = 64$ skeletons (i.e. frames) with $J = 18$ joints, the sequence is temporally downsampled to $\hat{z} \in \mathbb{R}^{\frac{T}{4} \times J \times n_{\hat{z}}}$, where, in our case, $n_{\hat{z}} = 256$ the encoder embedding size. 

For the MS-G3D decoder $G(\cdot)$, we opted for a slightly smaller model, since we experimentally observed that the encoder size is more negatively correlated to the final reconstruction error than the decoder size. Moreover, having a smaller decoder is more computationally efficient, and enables faster reconstruction of latent codes. The overall constituent decoder blocks are identical to the encoder blocks, but we replaced the strided convolution with a strided deconvolutional block for the temporal upsampling. We used 16 ST-GC blocks, with feature maps of size 32 for the first 4 blocks, 16 for the next 4 and 8 for the final 8. Temporal upsampling was performed every 4 blocks. 

\subsubsection{Skeleton Vector Quantization}

Instead of utilizing a continuous latent space to encode the skeleton sequences, we quantize each latent embedding into a fixed-length learnable codebook $\mathcal{Z} = \{z_k\}^K_{k=1} \subset \mathbb{R}^{n_z}$. Any skeleton sequence $x \in \mathbb{R}^{T \times J \times 2}$ is encoded using the MS-G3D encoder described above into a temporally-compressed representation $\hat{z} \in \mathbb{R}^{\frac{T}{4} \times J \times n_{\hat{z}}}$, which is then quantized into $z_{\textbf{q}} \in \mathbb{R}^{\frac{T}{4} \times J \times n_{z}}$, where $n_{\textbf{z}}$ is the codebook dimensionality, not necessarily equal to the encoder embedding size. Each $\hat{z}$ is encoded using a nearest neighbor search in the codebook (see Eq. \ref{eq:quantization-nn}).

\begin{equation}
    z_{\mathbf{q}} = \mathbf{q}(\hat{z}) \coloneqq \argmin_{z_k \in \mathcal{Z}} \lVert \hat{z}_{ij} - z_k \rVert \in \mathbb{R}^{T \times J \times n_z}
    \label{eq:quantization-nn}
\end{equation}

After quantization, skeletons are reconstructed using the MS-G3D decoder: $\hat{x} = G(z_{\textbf{q}}) = G(\textbf{q}(E(x)))$. The model is trained end-to-end using a stop-gradient operation (see Eq. \ref{eq:vq-loss}) since the dictionary look-up is not differentiable. For more details regarding training, readers are referred to the work of Van Den Oord et al. \cite{van2017neural}.

\begin{equation}
\begin{split}
    \mathcal{L}_{VQ} (E, G, \mathcal{Z}) = \lVert x - \hat{x} \rVert &+ \lVert \text{sg}[E(x)] - z_{\mathbf{q}} \rVert_2^2 \\
    & + \lVert \text{sg}[z_{\mathbf{q}}] - E(x) \rVert_2^2
\end{split}
\label{eq:vq-loss}
\end{equation}

In practice, instead of the $l_2$ loss for the reconstruction error, we employed a Smooth $l_1$ with $\beta = 0.25$ \cite{girshick2015fast} to further penalize small reconstruction errors. VQ-VAE models are notoriously hard to train \cite{lancucki2020robust}, primarily due to the dictionary collapse problem, in which most of the codebook entries are not utilized in reconstruction, yielding poor performance. To deal with this problem, we employed a standard array of "bag-of-tricks" to increase codebook usage. We used K-means initialization of the codebook \cite{zeghidour2021soundstream}, we used a lower codebook dimensionality of $n_z = 16$ by linearly projecting down the encoder embedding, we used cosine similarity for codebook search \cite{yu2021vector}, expiring stale codes \cite{zeghidour2021soundstream} and orthogonal regularization \cite{shin2021translation} of the codebook vectors to encourage linear independence. The codebook is learned using an exponential moving average approach with a decay rate of $\gamma = 0.9$. Autoencoder warm-up \cite{fu-etal-2019-cyclical} was not necessary. We experimented with using a separate codebook for each limb, similar to Xie et al. \cite{xie2022vector}, but did not observe a substantial improvement.

Training was performed on a single NVIDIA RTX 3060, using mixed-precision training, with a batch size of 48. The network was updated for 50k steps, using AdamW \cite{kingma2014adam} optimizer using a cyclical learning rate schedule \cite{cyclical-lr} which varies the learning rate between 0.0025 and 0.0075. The model has 4.8M non-embedding parameters. The training duration for the VQVAE is approximately 15 hours.

\subsection{Learning Optimal Transport Mappings}

In order to exploit the expressive power of the learned gait tokens, we posit that only specific tokens from a tokenized gait sequence are responsible for encoding the gait viewpoint and variation. Therefore, for a set of walks from a particular variation $\mathcal{T}$, we can learn a set of transport maps $\Gamma = \{\gamma_j^{\textbf{*}} | j \in 1 \dots (\frac{T}{4} \times J)\}$, for each encoded position $j$, that transform the target quantized gait representation into a quantized representation of a baseline walk $\mathcal{B}$. The transformed walk $\mathcal{T}$ is then decoded by the generator: $\mathcal{T}^* = G(\Gamma(\textbf{q}(E(\mathcal{T}))))$. The walks $\mathcal{B}$ and $\mathcal{T}^*$ should be from the same walking variation.  We propose to learn the transport maps $\Gamma$ by utilizing optimal transport theory \cite{villani2009optimal}. We learn a transport map $\gamma^*_j$ by minimizing the Earth Mover's Distance (EMD) between the histograms of two quantized gaits. EMD assumes there is a cost for moving one quantity to another, which is encoded into a cost matrix $C$. In general, EMD is defined as:

\begin{equation}
\begin{split}
    \gamma^* = \argmin_{\gamma \in \mathbb{R}_{+}^{m \times n}} \sum_{i,j} \gamma_{i,j} C_{i,j} \\
    \text{s.t.} \gamma 1 = a; \gamma^T 1 = b; \gamma \geq 0
\end{split}
\label{eq:ot}
\end{equation}

In our case, $a$ and $b$ are histograms of the token occurrences in each gait sequence, and the cost matrix $C$ is given by the pairwise distances between the token embeddings. To account for multiple occurrence of the same token in a quantized gait sequence, we scale the corresponding vector embedding by the number of occurrences. We describe our method in Algorithm \ref{alg:gait-morphing}. The algorithm is an instance of an assignment problem for each token position, and is similar to finding the minimum flow between the two token distributions. In practice, we use the algorithm proposed by Bonneel et al. \cite{bonneel2011displacement} implemented in the PyOT \cite{flamary2021pot} library.

\begin{algorithm}
    \caption{Finding the optimal transport maps between walking variations. } 
    \label{alg:gait-morphing}
    \begin{algorithmic}
     \scriptsize
        \Require \\
        $E$ - Trained MS-G3D gait encoder \\
        $\mathcal{B} \in \mathbb{R}^{B^{(b)} \times T \times J \times 2}$ - baseline variation walks \\
        $\mathcal{T} \in \mathbb{R}^{B^{(t)} \times T \times J \times 2}$ - target walks \\
        $\mathcal{Z}$ - learned codebook vectors \\
        $s$ - token sequence length\\
        
        \State $k^{(b)} \gets \arg (\textbf{q}(E(\mathcal{B})))$ \Comment{\textit{Baseline token indices.}}
        % \in \{1 \dots |\mathcal{Z}|\}$ 
        \State $k^{(t)} \gets \arg (\textbf{q}(E(\mathcal{T})))$ \Comment{\textit{Target token indices.}}
        % \in \{1 \dots |\mathcal{Z}|\}$ 

        \State $\Gamma \gets \emptyset$
        \For{$j \gets 1 \dots s$} 
            \LeftComment \textit{Count occurrences of each baseline and target tokens.}

            \State $c^{(b)} \gets \{\sum_l^{B^{(b)}} \mathbb{1}[k^{(b)}_{l, j} = r] | r \in 1 \dots |\mathcal{Z}|\}$
            \State $c^{(t)} \gets \{\sum_l^{B^{(t)}} \mathbb{1}[k^{(t)}_{l, j} = r] | r \in 1 \dots |\mathcal{Z}|\}$

            \LeftComment \textit{Increase codebook embedding magnitude.}
            \State $C^{(b)} \gets \mathcal{Z} \odot c^{(b)}$
            \State $C^{(t)} \gets \mathcal{Z} \odot c^{(t)}$

            \LeftComment \textit{Compute cost matrix as pairwise distances between scaled token embeddings.}
            \State $C \gets C^{(b)} \cdot (C^{(t)})^\top$
            
            \LeftComment \textit{Find optimal transport map for position \textit{j}}
            \State $\gamma^{\textbf{*}}$ $\gets$ $\argmin_{\gamma} \sum \gamma C$ \Comment{Eq. \ref{eq:ot}}
            \State $\Gamma_j \gets \gamma^{\textbf{*}}$ 
        \EndFor
        
        \State \textbf{return} $\Gamma$
    \end{algorithmic}
\end{algorithm}


In practical scenarios where the gait variation is not known beforehand, domain-expert models such as pedestrian attribute identification models \cite{wang2022pedestrian} can be used to estimate particular walking attributes, similar to the approach of Cosma and Radoi \cite{cosma22gaitformer}, to inform the morphing target. This method can also be used as data augmentation to generate multiple views for the same walking sequence, for use in contrastive self-supervised training \cite{jaiswal2020survey,cosma22gaitformer}.

\section{Experiments and Results}
% Figure environment removed

For our experiments, we used CASIA-B \cite{casia} and Front-View Gait (FVG) \cite{fvg} to evaluate the performance of our proposed method. CASIA-B is a popular gait recognition dataset, widely used to test the robustness of gait analysis model across multiple viewpoints and walking variations. It contains gait sequences from 124 subjects, captured in 11 different viewpoints, under three walking variations: normal walking (NM), clothing walking (CL) and walking with a bag (BG). For a walker, 6 sessions are captured under the normal walking variation, 2 sessions with clothing change and 2 sessions under the bag carrying variation. Each walk has its variation / viewpoint known. We use the standard \cite{casia} training / testing split, consisting of the first 62 training subjects, with all available walking sessions. FVG is another popular dataset for gait recognition that features walks from only the front facing viewpoint, which is considered the most challenging due to the reduced perceived movement variation. It is comprised of 226 subjects captured in 6 walking variations: normal walk (NM), walk speed (WS), change in clothing (CL), carrying bag (CB), cluttered background (CBG) and ALL. In our experiments, we omit the "ALL" variation to properly isolate confounding factors. We used the first 136 subjects as the training split, and the rest for testing. It is important to note that the VQ-VAE model is trained on the dataset described in Section \ref{sec:dataset}, and remains frozen throughout the rest of the experiments. Moreover, the transport maps are learned only on the training split of each dataset (CASIA-B / FVG) and are utilized as-is on the testing split.

\subsection{The effect of dictionary size on the reconstructed gait sequences}

% Figure environment removed

% Figure environment removed

We trained several VQ-VAE models with increasingly larger dictionary sizes ($|\mathcal{Z}| \in \{2, 8, 16, 32, 128, 512, 2048\}$) to gauge the effect on the reconstructed sequences. In Figure \ref{fig:casia-recons}, we showcase the reconstruction error for each model on CASIA-B evaluation set, for each walking variation. For $|\mathcal{Z}| = 2$, the results are significantly worse than other dictionary sizes, due to the extreme compression. The model with $|\mathcal{Z}| = 8$ achieves the best overall performance. We showcase qualitative reconstruction samples in Figure \ref{fig:recons} - the model can reliably reconstruct skeleton sequences and acts as a low-pass filter which dampens exaggerated movements caused by inaccurate pose extractions.

Our model achieves a high degree of compression for skeleton sequences - a skeleton sequence represented as a float32 sequence of 2304 numbers is equivalent to storing 73728 bits of information, but using a VQ-VAE approach, the storage space is reduced to only 144 bits for $|\mathcal{Z}| = 2$ and 432 bits for $|\mathcal{Z}| = 8$. This compression level potentially allows on-device storage of massive amounts of skeleton sequences. 

Figure \ref{fig:casia-usage} showcases the dictionary usage for each dictionary size. Increasing the dictionary size slightly decreases dictionary usage, which implies that some tokens are underutilized by the model. This effect is more pronounced for $|\mathcal{Z}| = 2048$, especially for non-normal walking variations. This is most likely due to the fact that the pretraining dataset for the VQ-VAE mostly contains in-the-wild walks, which make the BG and CL variations easier to reconstruct.  The reconstructed gait sequences are not detrimental to downstream gait recognition models. To gauge the faithfulness of the reconstructed skeletons to the real walking skeletons, in Table \ref{tab:recog-recons} we showcase gait recognition results for CASIA-B using reconstructed skeletons as training data. The performance loss by using reconstructed skeletons is marginal, and even beneficial in some cases. This result can be attributed to the fact that the VQ-VAE acts as a low-pass filter and can slightly improve data quality across training. Furthermore, performance on gait recognition is not correlated with the reconstruction error of the VQ-VAE: the model with $|\mathcal{Z}| = 2$ achieves comparable results with the baseline method using real skeletons.

\begin{table}[hbt!]
    \centering
    \resizebox{0.85\linewidth}{!}{
        \begin{tabular}{llll}
         % & \multicolumn{3}{c}{\textbf{Variation}} \\
         &           \textbf{NM} &           \textbf{BG} &           \textbf{CL} \\
        \midrule
        Baseline                &  0.79 ± 0.06 &  0.46 ± 0.08 &  0.24 ± 0.06 \\
        \midrule
        $|\mathcal{Z}| = 2$  &   0.76 ± 0.09 &  0.46 ± 0.09 &  0.24 ± 0.07 \\
        $|\mathcal{Z}| = 8$     &  0.75 ± 0.13 &  0.44 ± 0.08 &  0.21 ± 0.08 \\
        $|\mathcal{Z}| = 16$    &  0.75 ± 0.12 &  0.44 ± 0.08 &  0.22 ± 0.08 \\
        $|\mathcal{Z}| = 32$    &  0.76 ± 0.12 &   0.47 ± 0.1 &  0.23 ± 0.09 \\
        $|\mathcal{Z}| = 128$   &   0.78 ± 0.1 &  0.46 ± 0.09 &  0.22 ± 0.07 \\
        $|\mathcal{Z}| = 512$   &  0.76 ± 0.12 &  0.46 ± 0.09 &  0.22 ± 0.09 \\
        $|\mathcal{Z}| = 2048$ &  0.78 ± 0.11 &  0.47 ± 0.11 &   0.24 ± 0.1 \\
        \end{tabular}
    }
    \caption{Accuracy on CASIA-B for GaitFormer \cite{cosma22gaitformer} trained with reconstructed skeletons. We report the mean and standard deviation of recognition accuracy across 4 distinct runs and all viewpoints.}
    \label{tab:recog-recons}
\end{table}
 

\subsection{Evaluation of morphed gait sequences}

% Figure environment removed

% Figure environment removed

We first present a qualitative evaluation for gait morphing. Figure \ref{fig:casia-morphs} showcases selected gait sequences from three different viewpoints morphed to a common NM-36 variation. The model is able to morph sequences into the baseline sequence, properly handling limb switching (left and right limbs are properly swapped when the viewpoint is from behind the walker). For similar baseline / target pairs, the transport maps exhibit fewer changes. Transport maps from VQ-VAE models with a larger dictionary size exhibit more token changes, but the earth movers distance between variations is comparatively smaller. This implies that many smaller changes are performed with same effect. In Figure \ref{fig:mass} we showcase the average moved distance, the number of changes and the total mass moved across the walking variations for both CASIA-B and FVG. We define mass as the number of changes multiplied by the average change cost. The number of changes is larger when the dictionary size is larger, but the total mass moved remains constant after the $|\mathcal{Z}| = 128$. This implies that the tokens from the model with $|\mathcal{Z}| = 2048$ are more disentangled, since less mass is moved to achieve the same outcome. 

In terms of numeric evaluation, our goal is to compare the morphed walks to the real walks of a particular variation. In our experiments, our comparisons are made with regard to NM-36 variation for CASIA-B and NM for FVG. The most straightforward comparison is to use mean squared error between skeletons, but we have no guarantees of sequence alignment between variations in either dataset. As such, we propose a metric between walking distributions, similar to the FID \cite{heusel2017gans} distance. The Frechet Inception Distance (FID) was introduced by Heusel et al. \cite{heusel2017gans} to measure the generation quality of GANs compared to real images. The FID score is based on the Frechet Distance \cite{dowson1982frechet}, and measures the distance $d(\cdot)$ between two gaussian distributions $\phi = (\textbf{m}, \textbf{C})$ and $\phi_w = (\textbf{m}_w, \textbf{C}_w)$, corresponding to a distribution of real and synthetic samples, respectively: $d^2(\phi, \phi_w) = ||\textbf{m} - \textbf{m}_w||^2 + Tr(\textbf{C} + \textbf{C}_w - 2(\textbf{CC}_w)^{\frac{1}{2}})$. The means and standard deviations of the Gaussians are, for images, the means and standard deviations of a set of embedding vectors of an Inception network \cite{inception} pretrained on ImageNet. The metric captures levels of perceived disturbance between real and synthetic samples \cite{heusel2017gans}. 

For gait synthetisation, we propose a specialized variant of the FID score, which we name "\textit{Frechet Gait Distance (FGD)}", in which walks are processed by a pretrained GaitFormer network on DenseGait \cite{cosma22gaitformer}. FGD stands as a automatic measure of walking "naturalness", by measuring the similarity to a given real gait distribution. Variants have been proposed for measuring motion naturalness and are geared towards general action synthesis \cite{gopinath2020fairmotion,siyao2022bailando,maiorca2022evaluating}, but a specialized variant for gait has not yet been adopted.

\begin{table}[]
    \centering
    \resizebox{\linewidth}{!}{
        \begin{tabular}{l|lllllll}
             &  &      \textbf{0$^\circ$} &     \textbf{72$^\circ$} &     \textbf{90$^\circ$} &    \textbf{126$^\circ$} &    \textbf{162$^\circ$} &   \textbf{ 180$^\circ$} \\
            \midrule
\multirow{8}{*}{\textbf{NM}} &  \textit{Baseline (vs real NM-36)} &  \textit{\textbf{0.045532}} &  \textit{\underline{0.070282}} &  \textit{\underline{0.111757}} &  \textit{0.138415} &   \textit{0.19525} &  \textit{0.265378} \\
             & \textit{Heuristic Aug. (vs real NM-36)} &  0.047659 &  0.076971 &  0.115972 &  0.138536 &  0.195943 &   0.27324 \\
             &  $|\mathcal{Z}| = $ 2 &  0.754251 &  0.320832 &  0.315892 &  0.195995 &  0.339743 &   0.74698 \\
             & $|\mathcal{Z}| = $ 8 &  0.379271 &  0.200036 &  0.104054 &  0.207928 &  0.673472 &  0.549586 \\
             & $|\mathcal{Z}| = $ 16 &  0.136429 &   0.11086 &   0.22199 &  0.434574 &  0.530231 &  0.128465 \\
             & $|\mathcal{Z}| = $ 32 &  0.091645 &  0.293709 &  0.400562 &  0.557834 &   0.63797 &   0.20792 \\
            & $|\mathcal{Z}| = $ 128 &   0.08184 &  0.465142 &  0.597904 &  0.766841 &  0.697738 &  0.268043 \\
            & $|\mathcal{Z}| = $ 512 &  0.074983 &   0.11027 &  0.117966 &  \underline{0.110944} &  \underline{0.140733} &  \textbf{0.107217} \\
           & $|\mathcal{Z}| = $ 2048 &  \underline{0.046048} &  \textbf{0.060231} &  \textbf{0.082002} &  \textbf{0.102774} & \textbf{0.104749} &  \underline{0.135883} \\
            \midrule
  \multirow{8}{*}{\textbf{BG}} & \textit{Baseline (vs real NM-36)} &  \textit{\textbf{ 0.05295}} &  \textit{\underline{0.074746}} &  \textit{0.114694} & \textit{ 0.150358} &  \textit{0.211948} &  \textit{0.274384} \\
               & \textit{Heuristic Aug. (vs real NM-36)} &  0.055826 &  0.083356 &  0.119362 &  0.152413 &  0.209289 &  0.283982 \\
              & $|\mathcal{Z}| = $ 2 &  0.716497 &  0.304378 &  0.243575 &  0.193439 &  0.599968 &  0.743501 \\
              & $|\mathcal{Z}| = $ 8 &  0.320088 &  0.184071 &  0.177406 &  0.190533 &  0.653437 &  0.639077 \\
             & $|\mathcal{Z}| = $ 16 &  0.169745 &  0.129582 &  0.232889 &  0.455946 &  0.536296 &  0.185364 \\
             & $|\mathcal{Z}| = $ 32 &  0.088956 &  0.341674 &  0.407447 &  0.589189 &  0.635381 &  0.203955 \\
            & $|\mathcal{Z}| = $ 128 &   0.07735 &  0.343908 &  0.514205 &  0.592576 &  0.527332 &  0.307418 \\
            & $|\mathcal{Z}| = $ 512 &  0.080196 &  \textbf{0.062556} &  \textbf{0.070378} &  \textbf{0.094266} &  \textbf{0.115324} &  \textbf{0.136357} \\
           & $|\mathcal{Z}| = $ 2048 &  \underline{0.056126} &  0.081991 &  \underline{0.106456} &  \underline{0.131166} &  \underline{0.137103} &  \underline{0.161214} \\
            \midrule
  \multirow{8}{*}{\textbf{CL}} & \textit{Baseline (vs real NM-36)} & \textit{ 0.110895} &  \textit{0.140185} &  \textit{0.189128} &  \textit{0.230226} &  \textit{0.320092} &  \textit{0.411968} \\
            & \textit{Heuristic Aug. (vs real NM-36)} &  0.120972 &  0.147726 &  0.197784 &  0.235666 &  0.318236 &  0.420584 \\
              & $|\mathcal{Z}| = $ 2 &  0.726999 &  0.312846 &   0.30105 &  0.376383 &  0.338582 &  0.670051 \\
              & $|\mathcal{Z}| = $ 8 &  0.261182 &  0.191699 &  0.129651 &  0.219145 &  0.656104 &  0.521065 \\
             & $|\mathcal{Z}| = $ 16 &  0.142326 &  0.194611 &  0.273543 &   0.50944 &  0.566114 &  0.177739 \\
             & $|\mathcal{Z}| = $ 32 &   0.07656 &  0.316372 &  0.418504 &  0.572725 &  0.589525 &  0.217498 \\
            & $|\mathcal{Z}| = $ 128 &  \textbf{0.064639} &  0.380276 &  0.514381 &  0.531558 &    0.4364 &  0.284238 \\
            & $|\mathcal{Z}| = $ 512 &  0.084125 &  \textbf{0.057824} &  \textbf{0.063853} &  \textbf{0.095734} &  \textbf{0.128801} &  \textbf{0.147653} \\
           & $|\mathcal{Z}| = $ 2048 &  \underline{0.075194} &  \underline{0.096743} &  \underline{0.128168} &  \underline{0.148594} &  \underline{0.159419} &  \underline{0.192654} \\
        \end{tabular}
    }
    \caption{FGD values between the morphed gait to the NM-36 variation and the real NM-36 for CASIA-B validation set. Baseline values corresponds to the FGD between the real unmodified gait and NM-36. In most variations, the morphed walk is much closer to the real NM-36 than the unmodified walk, especially for extreme viewpoints. We denote with \textbf{bold} the smallest distance and with \underline{underline} the second smallest distance.}
    \label{tab:fgd-casia}
\end{table}

\begin{table}[]
    \centering
    \resizebox{\linewidth}{!}{   
        \begin{tabular}{l|llll}
         & \textbf{WS} &        \textbf{CB} &        \textbf{CL} &       \textbf{CBG} \\
        \midrule
                      \textit{Baseline (vs real NM)} &  \textit{\textbf{0.001754}} & \textit{\textbf{ 0.039509}} &  \textit{\textbf{0.014785}} &  \textit{\textbf{0.001582}} \\
                      \textit{Heuristic Aug. (vs real NM-36)}  &  0.002493 &  0.042682 &   0.01474 &  0.001812 \\
                      $|\mathcal{Z}| = $ 2 &  0.051189 &  0.081181 &  0.189054 &  0.077997 \\
                      $|\mathcal{Z}| = $ 8 &  0.055022 &  0.100748 &  0.107133 &  0.082222 \\
                     $|\mathcal{Z}| = $ 16 &  0.032046 &   0.04954 &  0.067667 &  0.053239 \\
                     $|\mathcal{Z}| = $ 32 &  0.031136 &  0.058868 &  0.059492 &   0.04834 \\
                    $|\mathcal{Z}| = $ 128 &  0.032589 &  0.059443 &  0.056675 &  0.034228 \\
                   $|\mathcal{Z}| = $ 512 &   0.03081 &  0.050809 &  0.035253 &  0.028172 \\
                   $|\mathcal{Z}| = $ 2048 &  \underline{0.022087} &  \underline{0.043005} &  \underline{0.019479} &  \underline{0.025608} \\
        \end{tabular}
}
    \caption{FGD values between the morphed gait to the NM variation and the real NM for FVG validation set. Baseline values corresponds to the FGD between the real unmodified gait and NM. The morphed walk is similar to the real NM variations, but the effect is not pronounced due to the same underlying viewpoint for all variations. We denote with \textbf{bold} the smallest distance and with \underline{underline} the second smallest distance. 
    }
    \label{tab:fgd-fvg}
\end{table}

In Tables \ref{tab:fgd-casia} and \ref{tab:fgd-fvg}, we present our results for gait morphing for CASIA-B and FVG, respectively. We utilized the proposed FGD metric to compare the distance between the distribution of the morphed walks to the real baseline walking variation (NM-36 for CASIA-B and NM for FVG). For CASIA-B we focus our evaluation in terms of viewpoint, since it is the principal confounding factor, especially for 2D poses. Results show that the morphed walks are properly generated and are closer to the real NM-36 walking variation compared to the unmodified walk and for more extreme viewpoints, the effect is larger. Results are more correlated with the dictionary usage for each dictionary size, rather than reconstruction error (which is low for every dictionary size). Additionally, we compared morphed gaits with standard array of heuristic skeleton augmentations present in other works\cite{cosma22gaitformer,gaitgraph}: random pace with a time multiplier sampled from \{0.5, 0.75, 1, 1.25, 1.5, 1.75, 2.0\}, joint and point noise with standard deviation of 0.001, random mirroring and reversing the walk. While heuristic augmentations provide some variation in the vicinity of the original walk, the FGD across views are similar to the non-augmented walks. These results show that the morphed walks with our method represent a good way to augment existing walks to synthesize novel views. Since all the walks in FVG are from the same viewpoint, the differences between walking variations are not as evident. Consequently, the distance between distributions is comparatively smaller than in CASIA-B. 

% Figure environment removed

It is clear from results in Tables \ref{tab:fgd-casia} and \ref{tab:fgd-fvg} that models operating with a low dictionary size are not appropriate to be used for morphing. This is most likely due to the latent embeddings being severely entangled. Figure \ref{fig:failure} showcases a selected failure case for morphing a NM-180 walk from CASIA-B into NM-36 using a VQ-VAE with $|\mathcal{Z}| = 8$. The generated walk has severe artifacts and cannot be considered appropriate for downstream model training. Inherently, there is a trade-off between dictionary size and the manipulability of the latent codes: larger dictionary sizes have more disentangled representations which allow for more informed changes at the expense of lower data compression.

\section{Conclusions}
In this work, we presented GaitMorph, a novel method for modifying gait sequences into new walking variations. Our proposed approach entails firstly training a discrete latent model (in our case, a VQ-VAE) that compresses the walking sequences into a sequence of interpretable tokens, and learning an optimal latent transport map across variations. Our extensive experiments show that the trained VQ-VAE model preserves the walker's identity, achieving a marginal loss in performance when utilizing reconstructed sequences in gait recognition scenarios. Furthermore, we showed that the distribution of morphed sequences is similar to the real walk distribution. 
Our approach has the potential to be applied to self-supervised learning scenarios for gait recognition \cite{cosma22gaitformer}, which are heavily reliant \cite{chen2020simple,tian2020makes} on multiple strong augmentations / views for the same input.

{\small
\bibliographystyle{ieee}
\bibliography{egbib}
}

\end{document}
