\documentclass[conference]{IEEEtran}
\IEEEoverridecommandlockouts
% The preceding line is only needed to identify funding in the first footnote. If that is unneeded, please comment it out.
\usepackage{cite}
\usepackage{amsmath,amssymb,amsfonts}
\usepackage{algorithmic}
\usepackage{graphicx}
\usepackage{textcomp}
\usepackage{xcolor}
\usepackage{hyperref}
\usepackage{multirow}
\usepackage{rotating}
\usepackage{subfigure}
\usepackage{float}
\usepackage{adjustbox}

\def\BibTeX{{\rm B\kern-.05em{\sc i\kern-.025em b}\kern-.08em
    T\kern-.1667em\lower.7ex\hbox{E}\kern-.125emX}}
\begin{document}

\title{ELiOT : End-to-end Lidar Odometry using Transformer Framework 
% {\footnotesize \textsuperscript{*}Note: Sub-titles are not captured in Xplore and
% should not be used}
\thanks{\textsuperscript{1} Department of Electrical Engineering, Korea Advanced Institute of Science and Technologies (KAIST), Daejeon, Republic of Korea
        {\texttt{\{lee.dk, hcshim\}@kaist.ac.kr}}}
\thanks{\textsuperscript{2} Robotics Program, KAIST, Daejeon, Republic of Korea
        {\texttt{menu107@kaist.ac.kr}}
        }
\thanks{*Corresponding Author.}
\thanks{This work was supported by the Technology Innovation Program (RS-2023-00256794, Development of drone-robot cooperative multimodal delivery technology for cargo with a maximum weight of 40kg in urban areas) funded By the Ministry of Trade, Industry \& Energy(MOTIE, Korea).}
\thanks{This work has been submitted to the IEEE for possible publication. Copyright may be transferred without notice, after which this version may no longer be accessible.}
}

\makeatletter
\newcommand{\linebreakand}{%
  \end{@IEEEauthorhalign}
  \hfill\mbox{}\par
  \mbox{}\hfill\begin{@IEEEauthorhalign}
}
\makeatother

\author{\IEEEauthorblockN{Daegyu Lee\textsuperscript{1}}
\and
\IEEEauthorblockN{Hyunwoo Nam\textsuperscript{2}}
\and
\IEEEauthorblockN{D.Hyunchul Shim\textsuperscript{1*}}
% \linebreakand
% Department of electrical engineering, Korea Advanced Institute of Science and Technology\\
% Daejeon, South Korea\\
% \IEEEauthorblockA{\{\texttt{lee.dk, menu107, hcshim}\}@\texttt{kaist.ac.kr}}
}

\maketitle

% \begin{abstract}
% In recent years, deep-learning-based point-cloud registration methods have shown promising results.
% Additionally, learning-based 3D detectors enable LiDAR data to be represented in a way that encodes semantic information effectively.
% In this paper, we propose end-to-end LiDAR odometry framework using Transformer framework, dubbed \textbf{ELiOT}.
% We propose a Self-Attention Flow Embedding Network to implicitly represent the motion of sequential LiDAR scenes. 
% Our 3D transformer encoder-decoder-based network pipeline that eliminates the need for 3D-2D projections demonstrates effective in predicting poses on urban datasets.
% Our proposed method achieves promising performances resulting 7.59$\%$/2.67$\%$ in terms of translational/rotational errors on the KITTI odometry dataset based-on end-to-end approach that foregoes the need for conventional geometric concepts.
% We also propose a data-driven framework in which the performance of a LiDAR odometry improves gradually as a robot accumulates more and more data operating autonomously in real environments.
% \end{abstract}

\begin{abstract}
In recent years, deep-learning-based point cloud registration methods have shown significant promise. 
Furthermore, learning-based 3D detectors have demonstrated their effectiveness in encoding semantic information from LiDAR data. 
In this paper, we introduce \textbf{ELiOT}, an end-to-end LiDAR odometry framework built on a transformer architecture. 
Our proposed Self-attention flow embedding network implicitly represents the motion of sequential LiDAR scenes, bypassing the need for 3D-2D projections traditionally used in such tasks. 
The network pipeline, composed of a 3D transformer encoder-decoder, has shown effectiveness in predicting poses on urban datasets. 
In terms of translational and rotational errors, our proposed method yields encouraging results, with 7.59$\%$ and 2.67$\%$ respectively on the KITTI odometry dataset. 
This is achieved with an end-to-end approach that foregoes the need for conventional geometric concepts. 
% Additionally, we propose a data-driven framework where LiDAR odometry performance incrementally improves as a robot autonomously accumulates data from real environments.
\end{abstract}


\begin{IEEEkeywords}
Localization, LiDAR odometry, Deep learning
\end{IEEEkeywords}

\section{Introduction}
Localization is one of the most essential modules in autonomous mobile robot. In recent years, LiDAR (Light Detection and Ranging)-based and vision-based \cite{qin2018vins, cao2022gvins, cvivsic2022soft2} localization methods have gained substantial interest because the localization quality of global navigation satellite system (GNSS) deteriorates in urban canyons due to the signal multi-path and weak signal strength \cite{kamat2018survey, sun2020robust}.
Especially, point-clouds obtained from a LiDAR or a RGB-D camera can be utilized to estimate the robot motion between two consecutive sweeps.
Therefore, this application called the LiDAR odometry (LO) \cite{jonnavithula2021lidar} is studied using geometry-based method \cite{zhang2014loam, shan2018lego, shan2020lio} or learning-based method \cite{yin2018locnet, lu2019l3, barsan2020learning, horn2020deepclr}. 
\par
The goal of the LO is to minimize the error of translation and rotation while accumulating the sequential robot motions. Typically, traditional LO methods are based on the point registration, which iteratively finds  point-to-geometry correspondences, and minimizes the pose transformation error.
Iterative Closest Point (ICP) \cite{besl1992method}, G-ICP \cite{segal2009generalized}, Normal Distributions Transform (NDT) \cite{biber2003normal} and its variants \cite{censi2008icp, das2012scan, das2014scan} are well-known examples utilizing the point registration algorithm. 
% Most registration methods are computationally expensive, feature-based LO is hence implemented by calculating geometry constraints \cite{shan2020lio, shan2018lego,xu2021fast}. 
Rather than employing computationally expensive dense registration methods, feature-based LO utilizes efficient geometry constraint calculations for its implementation as exemplified in the previous works \cite{shan2020lio, shan2018lego,xu2021fast} 
This method has shown high performance on various benchmarks including the KITTI dataset \cite{geiger2012we}.
% Despite these progresses, geometry methods lack memory considerations, whereas learning-based methods take advantage of data-driven approaches \cite{xue2020deep}.
Despite these advancements, geometry-based methods may not adequately account for memory considerations, whereas learning-based methods leverage data-driven approaches, as demonstrated in the work of Xue et al \cite{xue2020deep}.
\par
In addition, learning-based approaches have a strength in handling the point-cloud acquired in deteriorated conditions because deep learning for feature extraction is more promising than handcrafted feature extractions by modeling robust and effective cross-domain generalizability as in \cite{almalioglu2022deep}.
Recent state-of-the-art learning-based LiDAR odometry methods harness the advantages of a cylindrical projection-based approach \cite{wang2021efficient, liu2023translo}. This form of projection representation has the benefit of allowing the use of image-based network pipelines and concepts from recent architectural studies.
However, we note that emerging 3D data detection studies indicates that 2D represented methods cannot fully reflect the characteristics of 3D data and easy to loose features.
\par
% Figure environment removed
In this work, we propose end-to-end LO using 3-D Transformer frameworks, \textbf{ELiOT}, as illustrated in Fig. \ref{fig:simple}. 
The main contributions of this paper is as follows.
\begin{itemize}
    \item We present an end-to-end approach that foregoes the need for conventional geometric concepts such as K-nearest neighbors (\textit{KNN}) and clustering algorithms.
    \item We introduce a self-attention flow embedding network designed to implicitly represent the motion within sequential LiDAR scenes.
    \item Our network is capable of efficiently learning both global and local features by leveraging the positional embedding features of consecutive data sets.
    \item To the best of our knowledge, we are the first to propose a 3D transformer-based LiDAR Odometry (LO) framework that eliminates the need for 3D-2D projection.
\end{itemize}
% paper structure
The remainder of this paper is organized as follows. Section \ref{sec:related} introduces related 3D object detectors and deep LO studies. 
Section \ref{sec:methods} describes our problem definition, network and its modules for LO.
Section \ref{sec:implement} describes our implementation details.
Experiments conducted on the KITTI odometry dataset and our high-speed racecar dataset are discussed in Section \ref{sec:experiment}. 
Finally, Section \ref{sec:conclusion} concludes this study.

\section{Related work}
\label{sec:related}
\subsection{Feature extraction of 3-D detector}
In recent years, 3-D detectors are studied to handle the sparse and unordered characteristic of points.
Before directly processing point clouds in 3D, there has been a lot of research that has rasterized data into 2D by converting it to a BEV (Bird's Eye View) in order to utilize image-based recognition models for object recognition \cite{Yang_2018_CVPR, lang2019pointpillars}. 
This method of representing 3D data in 2D does not fully reflect the characteristics of 3D data and has difficulty considering the sparse data characteristics. As a result, research has emerged to directly process sparse, unordered set of 3D points as in \cite{qi2017pointnet, qi2017pointnet++}.
\par
However, because processing data arrays with tens of millions of points requires a large amount of computation, research has been proposed to efficiently detect objects by voxelizing the 3D space \cite{zhou2018voxelnet, deng2021voxel}. 
However, this voxelization method can decrease the accuracy of recognizing small objects, leading to the emergence of research that combines the benefits of both point-wise and voxel-wise approaches through a hybrid method of point-voxel feature extraction \cite{shi2020pv, shi2022pv}.
\par
In addition, the Transformer architecture by Vaswani \textit{et al.} \cite{vaswani2017attention} has been successfully adopted to image recognition studies \cite{carion2020end, beal2020toward}.
Recently, there has been a study on the 3D data Transformer backbone, incorporating a self-attention algorithm to enhance the receptive field. 
The main focus of these studies is to enhance the performance of learning-based detection methods by leveraging both efficient feature extraction and a multi-head self-attention modules \cite{misra2021end, mao2021voxel}. 
These techniques are aimed at improving the ability of the model to capture important spatial features and relationships within 3D data, ultimately leading to more accurate and robust detection results \cite{misra2021end, mao2021voxel}.
\par
Our idea is to apply self-attention mechanisms to sequential tasks, specifically pose prediction using LiDAR data. 
This utilization aims to capture long-range dependencies and temporal relationships, improving the accuracy of pose prediction for applications like autonomous vehicles and robotics.

\subsection{Lidar odometry}
Lidar odometry algorithms can be represented with two categories : geometry-based methods and learning-based methods.
The most straightforward method is point-cloud registration algorithm as known as ICP-variant \cite{besl1992method, segal2009generalized, censi2008icp} and NDT-variant \cite{biber2003normal, takeuchi20063,das2012scan, das2014scan} algorithms. 
Owing to registration algorithm's large computing consumption, feature-based LO methods \cite{shan2020lio, shan2018lego,xu2021fast} implement a handcrafted feature extraction and constraint for solution space by utilizing inertial measurement unit (IMU) values. 
These geometry-based methods are more intuitive for estimating robot locomotion, and they have demonstrated high accuracy in various benchmarks \cite{geiger2012we}.
\par
Recently, CNN-based odometry methods have been proposed, and these learning-based studies have originated from research in visual odometry \cite{wang2017deepvo, zhou2017unsupervised}.
To make use of this pipeline for visual odometry, research has also been conducted on rasterizing 3D points \cite{cho2020unsupervised}.
In the wake of the PointNet proposal \cite{qi2017pointnet, qi2017pointnet++}, 3D CNN-based methods have been proposed for end-to-end registration \cite{Yew_2018_ECCV, Lu_2019_ICCV,wang2019deeppco}.
Deep point-cloud registration has been previously proposed as a solution to suppress the substantial noise that can pervade real-world data \cite{horn2020deepclr}. However, conventional geometric concepts such as the \textit{KNN} algorithm are still utilized for motion embedding part \cite{Liu_2019_CVPR}.
In addition, A learning-based registration algorithm can be extended for domain adaptation work, as it has the advantage of being able to solve localization problems without a prior map by utilizing extracted feature information \cite{cho2022openstreetmap}.
Our study builds upon the foundation provided by the open-source project referenced in \cite{deepclr}.

\section{Proposed method}
\label{sec:methods}
% In this paper, we propose the DLOP-VT, which stands for 3D end-to-end LO using point-voxel feature extraction and Transformer blocks.
% voxel CNN-based and PointNet-based methods to effectively pool its corresponding features, as well as to increase the detecting accuracy.  
% In addition, because the self-attention mechanism of Transformers models all pairwise interactions for sequential data, it is suitable to apply Transformer blocks to train a spatially sequential data, which is the main goal of LO tasks.
% \\
% Our ELiOT improves the the advantages of two type of networks. 

% the DLOP-VT is composed of 3D voxel CNN with sparse convolution for efficient feature representation.
% To find a corresponding points between sequential points, we implement grouping module using \textit{k}-nearest neighbors algorithm.
% Subsequently, we integrate Transformer-based encoder-decoder modules for flow embedding module to understand the locomotion from corresponding points between groups. 
% Our mini-PointNet head enable us to estimate the transformation matrix $\mathbf{T} \in $ SE(3), consisting of translation $\mathbf{t} \in \mathbb{R}^3$ and rotation $\mathbf{R} \in 
% $ SO(3).
% Lastly, DLOP-VT architecture is modified from DeepCLR \cite{horn2020deepclr, deepclr} architecture which consists of set abatraction, flow embedding, and output modules.

% In the following section, we describe our network architecture to deeply integrate the point-voxel feature extraction and Transformer blocks for LO task.
% Our network architecture is composed of set abstraction, flow embedding with Transformer encoder-decoder, and output modules. 

\subsection{Problem definition}
Input of LO problem is unordered set of 3D points $\mathcal{P} = \{\mathbf{p}_1, \dots, \mathbf{p}_{n}\}$ where $n$ is the size of the query point-cloud.
Let us define target set as $\mathcal{Q} = \{\mathbf{q}_1, \dots, \mathbf{q}_{m}\}$ where $m$ is the size of the target point-cloud.
Moreover, Individual points in the point sets are $\mathbf{p}_i = \{\mathbf{x}_i, \mathbf{f}_i\}$ and $\mathbf{q}_i = \{\mathbf{y}_i, \mathbf{g}_i\}$ where $\mathbf{x}_i, \mathbf{y}_i \in \mathbb{R}^3$ denote its a vector of position $(x, y, z)$ and $\mathbf{f}_i, \mathbf{g}_i \in \mathbb{R}^c$ are optical feature such as reflection intensity, color or surfaces where $c$ is the number of feature dimension. 
The objective of \textbf{ELiOT} is to learn $\mathbf{T}$ from the sequential set $\{\mathcal{P}, \mathcal{Q} \}$. 
Let us define $\hat{\mathbf{x}}_t \in $ SE(3) be the odometry at time $t$.
Then, we can transform $\hat{\mathbf{x}}_t$ to $\hat{\mathbf{x}}_{t+1}$ using $\mathbf{T}_t$, which denotes estimated transformation matrix at time $t$ as $\hat{\mathbf{x}}_{t+1} = \mathbf{T}_t\hat{\mathbf{x}}_t$.
Therefore, if the initial pose $\hat{\mathbf{x}}_{0}$ is obtained from $\mathbf{T}_0 = \mathbf{I}$, we can formulate the odometry pose at time $t$ as following : 
\begin{equation}
\begin{aligned}
    \hat{\mathbf{x}}_{t} = \prod_{i = 0}^{t}{\mathbf{T}_i}\mathbf{x}_{0}.
\end{aligned}
\label{eq_definition}
\end{equation}
In this study, we train our model using supervised learning, with the aim of minimizing the error between the outcome predicted by Eq. \ref{eq_definition} and the ground truth odometry. Furthermore, we utilize dual-quaternions to represent the translation and rotation components of $\mathbf{T}_t$ \cite{kenwright2012beginners, schneider2017regnet, horn2020deepclr}.


\subsection{Overall Network architecture}
% Figure environment removed
% Our network architecture is composed of set abstraction, Implicitly Represented Self-Attention(IRSA) layers, and Transformer encoder-decoder modules. 
Figure \ref{fig:architecture} presents an overview of the proposed network architecture. 
We utilize a PointNet{++} backbone module \cite{qi2017pointnet++, qi2017pointnet} in order to extract geometric features via a learning-based approach. 
This module processes two consecutive point-cloud frames, $\mathcal{P}$ and $\mathcal{Q}$, generating keypoints and their corresponding set of abstracted features.
\par
Subsequently, these keypoints and features are input into the implicitly represented flow embedding (\textit{IRFE}) layers.
This allows for the learning of global and local features for motion flows. 
Finally, Transformer-based encoder-decoder modules are employed to predict the dual-quaternion of the consecutive point-cloud frames.

\subsection{Feature extraction \& Set abstraction}
\label{subsec:sb}
The first module of our framework is designed to extract features, thereby generating a subsampled set of abstracted data.
Because the size of the $\mathbf{P}$ is too large for a training model to process directly, sampling of point clouds is necessary.
More precisely, we adopt the Furthest-Point-sampling (FPS) algorithm to extract a small number of $n_{key}$ keypoints $\mathcal{K}_{\mathcal{P}} = \{\mathbf{p}_1, \dots, \mathbf{p}_{n_{key}}\} \subset \mathbf{P}$.
As previous deep LO \cite{Liu_2019_CVPR, horn2020deepclr} extended multi-scale grouping set abstraction \cite{qi2017pointnet++} for deep registration algorithm, 
we also generate a feature set $\mathcal{F}_{\mathcal{P}} = \{\mathbf{f}_1, \dots, \mathbf{f}_{n_{key}}\}$ with $n_{r}$ radii in multi-scale grouping, where $\mathcal{F}$ is fed into PointNet \cite{qi2017pointnet} with nonlinear function $h_{sa} : \mathbb{R}^{3+c} \xrightarrow{} \mathbb{R}^{c'}$, realized as $\text{MLP}_{sa}$, and element-wise max pooling.
\par
% Having keypoints uniformly distributed throughout the entire space can be advantageous for finding the $\mathbf{T}$ between corresponding points, we hence design the set abstraction module to include abundant global features and local features.

% To end this, we integrate \textit{Voxel Set Abstraction} (VSA) module to encode the multi-scale semantic features effectively.
% Let $\mathcal{C}$ represents the 3D overall space with sizes of $(D, W, H)$ along the axes of x , y and z . 
% The space is divided first into equal-sized cells $c_i$, with sizes of $(D/d_{D} \times W/d_{W} \times H/d_{H})$. 
% Then, the points $\mathbf{p}_i\in \mathcal{P}$ are distributed to the respective cells, which are referred to as voxels. 
% A the set of voxel-wise feature vectors in $k$-th level of 3D voxel CNN is adopted \cite{shi2020pv} where each voxel-wise feature vectors can be represented as $\mathcal{V}^{k}_{\mathcal{P}} = \{\mathbf{v}_1^{k}, \dots, \mathbf{v}_{n_{key}}^{k}\}$ where also realized as $\text{MLP}_{voxel}$, and element-wise max pooling.
% Subsequently, we concatenate features keypoints and a series of $k$-th voxel representation.
As a result, we generate an feature aggregation set for input $\mathcal{S}^{\mathcal{P}}$ which contains $\mathcal{K}_{\mathcal{P}}$, $\mathcal{F}_{\mathcal{P}}$ and identical mechanism for target set as follows:
\begin{equation}
\begin{aligned}
    &\mathcal{S}^{\mathcal{P}}_{sa, i} = \{ \mathbf{p}_i, \mathbf{f}_i | \forall \mathbf{p}_i \in \mathcal{K}_{\mathcal{P}}, \forall \mathbf{f}_i \in \mathcal{F}_{\mathcal{P}} \}, 
    \\
    &\mathcal{S}^{\mathcal{Q}}_{sa, i} = \{ \mathbf{q}_i, \mathbf{g}_i | \forall \mathbf{q}_i \in \mathcal{K}_{\mathcal{Q}}, \forall \mathbf{g}_i \in \mathcal{F}_{\mathcal{Q}} \}.
\end{aligned}
\label{eq_sa}
\end{equation}
Then, we pass through each $\mathcal{S}^{\mathcal{P}}_{sa, i}$ and $\mathcal{S}^{\mathcal{Q}}_{sa, i}$ to batch normalization of linear layers to generate feature aggregation.
Input and target point cloud with dimensions $[n, 3+c]$ is pooled and represented with dimension $[n_{key}, 3+c']$ where $c'$ is depth of $\text{MLP}_{sa}$. 
% Thus, let us define point-voxel feature abstracted set as $\mathcal{S}^{\mathcal{P}}_{sa, i} = \{\mathbf{p}_i, \mathbf{f'}_i \}$ and $\mathcal{S}^{\mathcal{Q}}_{sa, i} = \{\mathbf{q}_i, \mathbf{g'}_i \}$, respectively as to input and target points set.    

\subsection{Implicitly represented flow embedding (\textit{IRFE})}
Our reference studies \cite{Liu_2019_CVPR, horn2020deepclr} used flow embedding modules to concatenate consecutive LiDAR frames.
Specifically, they take the sequential feature aggregation set $\mathcal{S}^{\mathcal{P}}_{sa, i}$ and $\mathcal{S}^{\mathcal{Q}}_{sa, i}$ for flow embedding module.
Using the $(x, y, z)$ position vector of $\mathbf{p}_i \in \mathcal{S}^{\mathcal{P}}_{sa, i}$ and $\mathbf{q}_i \in \mathcal{S}^{\mathcal{Q}}_{sa, i}$, they implement the $k$-nearest neighbor (\textit{KNN}) and clustering algorithms to index $n_{group}$ of corresponding points between two sequential point clouds.
Therefore, they can generate a flow embedding set $\mathcal{S}^{fe}$ by concatenate abstracted sets as follow:
\begin{equation}
\begin{aligned}
    &\mathcal{S}^{fe}_i = \{ \mathbf{q}_{i} - \mathbf{p}_{i}, \mathbf{f}_i, \mathbf{g}_i \}.
\end{aligned}
\label{eq_fe}
\end{equation}
\par
However, conventional approaches such as \textit{KNN} and clustering algorithms still have inherent limitations. 
Specifically, these methods constrain the model to correspond only with the nearest features, potentially omitting valuable distant feature interactions.
\par
To enable our model to learn uniformly distributed keypoints across the entire space, we have designed the \textit{IRFE} layer. 
This layer is intended to learn both global and local features, providing a more holistic understanding of the data.
Drawing inspiration from NeRF \cite{mildenhall2021nerf}, we employ positional encoding to transform a 3D pose into a higher dimensional space. 
This process stimulates our model to encapsulate more complex and high-frequency data about the surroundings. 
We define $\gamma(\mathbf{p})$ as the Fourier positional encoding : 
\begin{equation}
\begin{aligned}
    \gamma(\mathbf{p}) = [\mathbf{p}, &sin(2^{0}\mathbf{p}), cos(2^{0}\mathbf{p}), ..., \\
                         & sin(2^{L-1}\mathbf{p}), cos(2^{L-1}\mathbf{p}) ].
\end{aligned}
\label{eq_position_encode}
\end{equation}
where $L$ is the dimension of Transformer's decoder.
\par
Therefore, we propose the \textit{IRFE} set, denoted as $\mathcal{S}^{irfe}$. 
This set, $\mathcal{S}^{irfe}_i$, is built by concatenating the abstracted positional encoding sets with the abstracted features as follows:
\begin{equation}
\begin{aligned}
    &\mathcal{S}^{irfe}_i = \{\gamma(\mathbf{p}_i), \gamma(\mathbf{q}_i), \mathbf{f}_i, \mathbf{g}_i \},
\end{aligned}
\label{eq_irfe}
\end{equation}
where $\mathcal{S}^{irfe}_{i}$ signifies the sequence of the flattened feature set with dimensions $[n_{key}, 4 \times c']$. 


% As we implemented in \ref{subsec:sb}, $\mathcal{S}^{fe1}_i$ is fed into PointNet $\text{MLP}_{fe}$, and element-wise max pooling, which maps $h_{fe1} : \mathbb{R}^{3+2c'} \xrightarrow{} \mathbb{R}^{3+c''}$.
% Let us define feature-follow embedded set $\mathcal{S}^{fe2}_i$ being the result of $\text{MLP}_{fe}$.
% For the positional embedding, we merge $(x, y, z)$ position vector of $\mathbf{p}_i$ to $\mathcal{S}^{fe2}_i$, its dimension hence becomes $[n_{key}, 3+c'']$. 
\subsection{3-D Transformer-based pose prediction}
%Transformer block explain here.
Here, we propose a encoder-decoder transformer block for self-attentioning flow embedding set.
We adopt and extend the framework of the 3D Transformer \cite{misra2021end} to suit the LO task.
\par
We feed $\mathcal{S}^{irfe}_i$, $\mathbf{p}_{i}$, and $\mathbf{q}_{i}$ into the Transformer layer. 
% Taking inspiration from the additional learnable \textit{classification token} concept \cite{devlin2018bert, dosovitskiy2020image}, we allow the sampled keypoint tokens $\mathbf{p}_{i}$ and $\mathbf{q}_{i}$ to be passed through the Transformer layers. 
% This is to preserve the geometric characteristics of consecutive LiDAR frames.
% In our implementation of the Transformer's encoder and decoder, only $\mathbf{p}{i}$ is fed into the encoder layer, while queries for the decoder are generated from $\mathbf{q}{i}$.
Motivated by the concept of learnable \textit{classification tokens} \cite{devlin2018bert, dosovitskiy2020image}, our Transformer network follows a similar approach. 
Specifically, we enable the passage of sampled keypoint tokens $\mathbf{p}_{i}$ and $\mathbf{q}_{i}$ through the Transformer layers. 
As a result, the encoder effectively processes $\mathbf{p}_{i}$ to capture spatial features, while the decoder handles $\mathbf{q}_{i}$ to focus on sequential dependencies, preserving the geometric characteristics of consecutive LiDAR frames. 
Our implementation utilizes the encoder to handle $\mathbf{p}_{i}$ and generates queries for the decoder from $\mathbf{q}_{i}$. This design choice optimizes the model's ability to accurately predict pose using LiDAR data.
\par
Finally, we employ an output multilayer perceptron using conventional Conv1d and Linear layers to predict the pose using a dual-quaternion representation. 

\subsection{Training losses}
% Dual quaternions are often used to represent the pose (i.e., position and orientation) of an object in 3D space. They have a number of advantages over other representations, such as Euler angles or rotation matrices, including the ability to interpolate smoothly between poses and the ability to represent reflections. However, they can be more difficult to work with than other representations, due to the need to handle both the real and dual quaternions separately.
The objective of LO is to minimize minimize the error between Eq. \ref{eq_definition} and ground truth odometry.
Here, we can denote the predictive transformation as $\mathbf{T}$, and ground truth transformation as $\mathbf{T}^{gt}$, respectively.
There is a intuitive loss function using geometric approach \cite{kendall2017geometric} such as Euler angles, SO(3) rotation matrices or quaternions.
\par
In this study, as \cite{schneider2017regnet, horn2020deepclr} proposed, we adopt dual-quaternions $\mathbf{\sigma} = \mathbf{p}_{r} + \epsilon\mathbf{q}_{d}$ for representing rigid transformation where $\mathbf{p}_{r}$, $\mathbf{q}_{d}$ are real part, dual part, respectively.
The details of dual-quaternions can be found here \cite{kenwright2012beginners}.
Therefore, we implement the dual loss as follows : 
\begin{equation}
\begin{aligned}
    &{L}_{d} = \mathbb{E}\left[\left\| \mathbf{q}_{d}^{pred} - \mathbf{q}_{d}^{gt}\right\|_2\right],
\end{aligned}
\label{eq_dual_loss}
\end{equation}
where $\mathbf{q}_{d}^{pred}$, $\mathbf{q}_{d}^{gt}$ are denoted as predicted transformation, ground-truth transformation represented by dual part in dual-quaternions form, respectively.
Also, the real loss is given by
\begin{equation}
\begin{aligned}
    &{L}_{r} = \mathbb{E}\left[\left\| \frac{\mathbf{p}^{pred}_{r}}{\left\|\mathbf{p}^{pred}_{r} \right\|} - \mathbf{p}^{gt}_{r} \right\|_2\right],
\end{aligned}
\label{eq_real_loss}
\end{equation}
where $\mathbf{p}_{r}^{pred}$, $\mathbf{p}_{r}^{gt}$ are denoted as predicted rotation, ground-truth rotation represented by real part in dual-quaternions form, respectively.

\begin{table}[t!]
\caption[]{Hyperparameters for KITTI Odometry dataset}
\label{tab:implement}
\centering
\begin{tabular}{c|c|c}
Module & Parameter for KITTI & Value \\ 
\hline 
\multirow{3}{*}{\begin{turn}{0}Feature extraction\end{turn}} 
& $n_{key}$  & 1024 \\  
% SA_LAYER
& $\text{MLP}_\text{raw}$ & [[16, 16, 32], [16, 16, 32]] \\
& SA radii & [0.5, 1.0] \\
% & $\text{MLP}_\text{v=1}$ & [[16, 16], [16, 16]] \\
% & 1st VSA radii & [0.4, 0.8] \\
% & $\text{MLP}_\text{v=2}$ & [[32, 32], [32, 32]] \\
% & 2nd VSA radii & [0.8, 1.2] \\
\hline
% Flow Embedding
\multirow{4}{*}{\begin{turn}{0}\begin{tabular}[c]{@{}c@{}}Flow embedding \& \\ Positional encoding \end{tabular}\end{turn}}
& $L$ & 64 \\
& PE method & Fourier \\
& PE normalization & True \\
& Gauss scale & 1.0 \\
% Transformer
\hline
\multirow{13}{*}{\begin{turn}{0}Transformer\end{turn}}
% encoder
& Enc. layer & 3 \\
& Enc. dim. & 256 \\
& Enc. head number & 4 \\
& Enc. ffn dim.& 16 \\
& Enc. dropout& 0.1 \\
& Enc. activation & relu \\
% decoder
& Dec. layer & 3 \\
& Dec. dim. & 256 \\
& Dec. head number & 2 \\
& Dec. ffn dim.& 16 \\
& Dec. dropout& 0.1 \\
& Num. of queries & 1024 \\
\hline
\multirow{2}{*}{\begin{turn}{0}Head\end{turn}}
& $\text{MLP}_\text{pn}$ & [256, 512, 1024] \\
& $\text{MLP}_\text{fc}$ & [1024, 512, 256, 8] \\
\hline
\end{tabular}
\end{table} 
\section{Implementation details}
\label{sec:implement}
In our implementation, raw points are consecutively sampled using FPS methods, and aggregated into feature abstracted set $\mathcal{S}_{sa, i}^{\mathcal{P}}$ and $\mathcal{S}_{sa, i}^{\mathcal{Q}}$.
\par
% For the grouping module, we depart from conventional grouping approaches such as the \textit{KNN} algorithm, and instead, employ an \textit{IRFE} layer. 
% To accomplish this, we flatten $\gamma(\mathbf{p}_i), \gamma(\mathbf{q}_i), \mathbf{f}i, \mathbf{g}i $, each of which has the same dimension of $[n_{key}, c']$. 
% This process generates the sequence of the flattened feature set with dimensions $[n_{key}, 4 \times c']$.
When it comes to the flow embedding module, we opt for a departure from conventional methods such as the \textit{KNN} algorithm. 
Instead, we have opted for an \textit{IRFE} layer, which involves flattening $\gamma(\mathbf{p}_i)$, $\gamma(\mathbf{q}_i)$, $\mathbf{f}_i$, and $\mathbf{g}_i$, each having a consistent dimension of $[n_{key}, c']$. 
This flattening process yields a sequence of the flattened feature set with dimensions of $[n_{key}, 4 \times c']$.
\par
Lastly, our transformer block is composed of position embedding, encoder, decoder, and query embedding modules.
An overview of the network parameter is represented in Table \ref{tab:implement}.
\par
During learning process, random translations and rotations noises are applied to the input points set for data augmentation.
% Lastly, our full implementation is available here: 
% \href{https://github.com/Leedk3/dlopvt}{https://github.com/Leedk3/dlop\_vt}.


\begin{table}[htb!]
\caption[]{Comparison with the state-of-the-arts on KITTI odometry dataset. We list the classic methods and learning-based odometry methods.}
\label{tab:result}
\centering
\resizebox{0.5\textwidth}{!}{%
\begin{tabular}{c|c|cccccccccc}
\multirow{2}{*}{Method} & \multirow{2}{*}{Seq.} 
& \multicolumn{2}{|c}{7} & \multicolumn{2}{|c}{8} & \multicolumn{2}{|c}{9} & \multicolumn{2}{|c}{10} & \multicolumn{2}{|c}{Testing Avg.} \\ \cline{3-12}
& & \multicolumn{1}{|c}{$t_{rel}$} & \multicolumn{1}{|c}{$r_{rel}$} & \multicolumn{1}{|c}{$t_{rel}$} & \multicolumn{1}{|c}{$r_{rel}$} & \multicolumn{1}{|c}{$t_{rel}$} & \multicolumn{1}{|c}{$r_{rel}$} & \multicolumn{1}{|c}{$t_{rel}$} & \multicolumn{1}{|c}{$r_{rel}$} & \multicolumn{1}{|c}{$t_{rel}$} & \multicolumn{1}{|c}{$r_{rel}$} \\
\hline \hline 
% classic
\multirow{4}{*}{\begin{turn}{90}Classic\end{turn}}
& ICP-PO2PO & $5.17$ & $3.35$ & $10.04$ & $4.93$ & $6.93$ & $2.89$ & $8.91$ & $4.74$ & $7.76$ & $3.98$\\
& ICP-PO2PL & $1.55$ & $1.42$ & $4.42$ & $2.14$ & $3.95$ & $1.71$ & $6.13$ & $2.60$ & $4.01$ & $1.97$\\
& GICP\cite{segal2009generalized}  & $0.64$ & $0.45$ & $1.58$ & $0.75$ & $1.97$ & $0.77$ & $1.31$ & $0.62$ & $1.38$ & $0.65$\\
& LOAM\cite{zhang2014loam}  & $2.98$ & $1.55$ & $4.89$ & $2.04$ & $6.04$ & $1.79$ & $3.65$ & $1.55$ & $4.39$ & $1.73$\\
\hline
% learning-based
\multirow{4}{*}{\begin{turn}{90}\begin{tabular}[c]{@{}c@{}} Learning-\\ based \end{tabular} \end{turn}}
& SelfVoxeLO\cite{xu2021selfvoxelo} & $3.09$ & $1.81$ & $3.16$ & $1.14$ & $3.01$ & $1.14$ & $3.48$ & $1.11$ & $3.19$ & $1.30$\\
& Efficient-LO\cite{wang2021efficient} & $0.46$ & $0.38$ & $1.14$ & $0.41$ & $0.78$ & $0.33$ & $0.80$ & $0.46$ & $0.80$ & $0.40$\\
& DEEPCLR\cite{horn2020deepclr} & $4.82$ & $3.47$ & $6.37$ & $2.27$ & $3.87$ & $1.63$ & $8.10$ & $2.77$ & $5.79$ & $2.54$\\
& \textbf{Proposed} & $7.20$ & $3.39$ & $5.30$ & $1.85$ & $5.42$ & $2.09$ & $12.46$ & $3.35$ & $7.59$ & $2.67$\\
\hline
\end{tabular}
}
\begin{flushleft}
* Units for $t_{rel}$ and $r_{rel}$ are the average translational RMSE (\%) and rotational RMSE ($\deg$/100m) respectively on all possible subsequences in the length of 100, 200, ..., 800 m.
\end{flushleft}
\end{table}


% \begin{table*}[hbt!]
% \caption[]{Comparison with the state-of-the-arts on KITTI odometry dataset. We list the classic methods and learning-based odometry methods.}
% \label{tab:result}
% \centering
% \begin{tabular}{c|c|cccccccccccc}
% \multirow{2}{*}{Method} & \multirow{2}{*}{Seq.} & 
% \multicolumn{2}{c}{Training Seq.} & \multicolumn{2}{|c}{7} & \multicolumn{2}{|c}{8} & \multicolumn{2}{|c}{9} & \multicolumn{2}{|c}{10} & \multicolumn{2}{|c}{Testing Avg.} \\ \cline{3-14}
% & & {$t_{rel}$} & \multicolumn{1}{|c}{$r_{rel}$} & \multicolumn{1}{|c}{$t_{rel}$} & \multicolumn{1}{|c}{$r_{rel}$} & \multicolumn{1}{|c}{$t_{rel}$} & \multicolumn{1}{|c}{$r_{rel}$} & \multicolumn{1}{|c}{$t_{rel}$} & \multicolumn{1}{|c}{$r_{rel}$} & \multicolumn{1}{|c}{$t_{rel}$} & \multicolumn{1}{|c}{$r_{rel}$} & \multicolumn{1}{|c}{$t_{rel}$} & \multicolumn{1}{|c}{$r_{rel}$} \\
% \hline \hline 
% % classic
% \multirow{3}{*}{\begin{turn}{90}Classic\end{turn}}
% & ICP-PO2PO & $t_{rel}$ & $r_{rel}$ & $t_{rel}$ & $r_{rel}$ & $t_{rel}$ & $r_{rel}$ & $t_{rel}$ & $r_{rel}$ & $t_{rel}$ & $r_{rel}$ & $t_{rel}$ & $r_{rel}$\\
% & ICP-PO2PL & $t_{rel}$ & $r_{rel}$ & $t_{rel}$ & $r_{rel}$ & $t_{rel}$ & $r_{rel}$ & $t_{rel}$ & $r_{rel}$ & $t_{rel}$ & $r_{rel}$ & $t_{rel}$ & $r_{rel}$\\
% & GICP & $t_{rel}$ & $r_{rel}$ & $t_{rel}$ & $r_{rel}$ & $t_{rel}$ & $r_{rel}$ & $t_{rel}$ & $r_{rel}$ & $t_{rel}$ & $r_{rel}$ & $t_{rel}$ & $r_{rel}$\\
% \hline
% % learning-based
% \multirow{5}{*}{\begin{turn}{90}\begin{tabular}[c]{@{}c@{}} Learning-\\ based \end{tabular} \end{turn}}
% & DEEPCLR & $t_{rel}$ & $r_{rel}$ & $t_{rel}$ & $r_{rel}$ & $t_{rel}$ & $r_{rel}$ & $t_{rel}$ & $r_{rel}$ & $t_{rel}$ & $r_{rel}$ & $t_{rel}$ & $r_{rel}$\\
% & DEEPPVO & $t_{rel}$ & $r_{rel}$ & $t_{rel}$ & $r_{rel}$ & $t_{rel}$ & $r_{rel}$ & $t_{rel}$ & $r_{rel}$ & $t_{rel}$ & $r_{rel}$ & $t_{rel}$ & $r_{rel}$\\
% & SelfVoxeLO & $t_{rel}$ & $r_{rel}$ & $t_{rel}$ & $r_{rel}$ & $t_{rel}$ & $r_{rel}$ & $t_{rel}$ & $r_{rel}$ & $t_{rel}$ & $r_{rel}$ & $t_{rel}$ & $r_{rel}$\\
% & Efficient-LO & $t_{rel}$ & $r_{rel}$ & $t_{rel}$ & $r_{rel}$ & $t_{rel}$ & $r_{rel}$ & $t_{rel}$ & $r_{rel}$ & $t_{rel}$ & $r_{rel}$ & $t_{rel}$ & $r_{rel}$\\
% & \textbf{Proposed} & $t_{rel}$ & $r_{rel}$ & $t_{rel}$ & $r_{rel}$ & $t_{rel}$ & $r_{rel}$ & $t_{rel}$ & $r_{rel}$ & $t_{rel}$ & $r_{rel}$ & $t_{rel}$ & $r_{rel}$\\
% \hline
% \end{tabular}
% \end{table*}


% Figure environment removed


% Figure environment removed

\section{Experiments}
\label{sec:experiment}
% \subsection{LiDAR Odometry Dataset}
\subsection{KITTI Odometry}
We validate the proposed end-to-end LO method with KITTI odometry dataset \cite{geiger2012we}.
There are 22 LiDAR sequences and corresponding RGB/gray images in the KITTI odometry dataset.
It provides ground-truth poses derived from IMU/GPS fusion algorithms for sequences 00-10.
Also, ground-truth poses aren't provided by the remaining sequences, which are for benchmark testing.
There are different types of road environments in this dataset, as well as pedestrians, cyclists, and different types of vehicles. 
Data-collection vehicle drives from 0 $km/h$ to $90km/h$ in different areas.
Therefore, we can evaluate whether the trained model can handle noisy real-world data with a large-scale point cloud.
% \subsubsection{Mulran}

\subsection{Evaluation}
All methods are tested on a machine equipped with an Intel Core i9-10900X CPU @ 3.70GHz, memory of 128GB, and a GPU of 24GB.
The implementation is written in PyTorch.
\par
% \subsubsection{KITTI Odometry}
As our training dataset consists of seqeunces 00-06, our method is compared against other competitive methods using sequences 07-10 from the KITTI odometry dataset. 
We compare our method with the classic method \cite{censi2008icp, segal2009generalized} ---i.e. to demonstrate the classic methods, point-to-point ICP, point-to-plane ICP, and GICP are implemented based on the Open3D library.   
Additionally, we compare the proposed method against other CNN-based odometry methods \cite{horn2020deepclr, wang2019deeppco}, and we discover that our method achieves competitive performance.
We train a model with 00-06 sequences and evaluate with 07-10. 
The results were evaluated using the \textit{KITTI Odometry Devkit}, which provides the average relative translation ($t_{rmse}$) and rotation errors ($r_{rmse}$) as part of its output metrics.
\par
The evaluation results, which include a comparison with classic and learning-based methods, are presented in Table \ref{tab:result}. 
For a more intuitive understanding of the performance of our proposed method, we have visualized the generated trajectories on KITTI odometry sequences 07 through 10, as depicted in Fig. \ref{fig:kitti_07_10}. This visualization also includes each sequence's translation and rotation error segregated by speed and path length segments in Fig. \ref{fig:kitti_error}.
\par
While comparing our method with traditional techniques and referencing learning-based studies, we acknowledge the potential for further enhancements. 
Specifically, we understand that our methodology requires more development to achieve the performance benchmarks set by 3D to 2D projection-based techniques and conventional approaches. 
As we continue our research, we aim to address these areas of improvement and strive for better results in future iterations.
\par
One possible reason for the lack of generalization could be that the KITTI dataset used for training only covered limited sequences from 0 to 6. 
As a result, our model may not have fully generalized to other unseen sequences. Addressing this limitation will be an important focus for future work.

% Figure environment removed

\subsection{Visualization}
In Fig. \ref{fig:attention}, we provide a visual representation of the sampling points and their respective flow attention results. 
These results highlight how the attention is primarily directed towards edge and corner features. 
These features provide a richer source of motion information between consecutive LiDAR frames, thereby significantly contributing to the efficacy of our method.

% \subsubsection{Mulran}


% \subsection{Ablation study}
% We conduct an ablation study to illustrate how our learning-based approach can enhance the model's speed compared to the use of conventional geometric-based \textit{KNN} layers for identifying corresponding points. 
% This study underscores the efficiency gains offered by our methodology, contributing to its real-time applicability and performance optimization.
% \begin{table}[hbt!]
\caption[]{Comparison of runtimes with other methods and ablation study}
\label{tab:runtime}
\centering
\begin{tabular}{c|c}
Methods & Time($ms$) \\ 
\hline 
Number of layer & 3 \\
Number of head & 4 \\
Enc. layer & 3 \\
\hline
\end{tabular}
\end{table}


% \section{Conclusion}
% \label{sec:conclusion}
% In this work, we have presented \textbf{ELiOT}, a novel architecture that harmoniously integrates point feature abstraction with a motion flow self-attentioning 3D transformer framework. 
% we have offered a end-to-end LiDAR odometry approach unlocking new potential for efficient and accurate data interpretation without conventional geometric methods.
% \par
% While comparing our method with traditional techniques and referenced learning-based studies, we acknowledge the room for further enhancements. 
% Particularly, we understand that our methodology requires more development to reach the performance benchmarks set by 3-D to 2-D projection-based methodologies and conventional approaches.
% \par
% Nonetheless, our novel deep learning-based LiDAR odometry method holds significant promise. 
% It capitalizes on the potential of positional embedding networks to identify motion flow and pays special attention to key features. 
% In doing so, it obtains geometric motion through an end-to-end approach. This method offers a unique and promising perspective in the field, making it a valuable contribution to the ongoing advancements in autonomous navigation systems.

\section{Conclusion}
\label{sec:conclusion}
In this work, we have presented \textbf{ELiOT}, a novel architecture that seamlessly integrates point feature abstraction with a motion flow self-attentioning 3D transformer framework. Our end-to-end LiDAR odometry approach unlocks new potential for efficient and accurate data interpretation without relying on conventional geometric methods.
\par
While comparing our method with traditional techniques and referencing learning-based studies, we acknowledge the potential for further enhancements. We understand that our methodology requires more development to achieve the performance benchmarks set by 3D to 2D projection-based techniques and conventional approaches.
\par
Nonetheless, our novel deep learning-based LiDAR odometry method holds significant promise. It leverages the potential of positional embedding networks to identify motion flow and gives special attention to key features. By adopting an end-to-end approach, our method obtains geometric motion in a seamless manner, making it a valuable contribution to the field and offering a unique and promising perspective for advancements in autonomous navigation systems.
\par
\textbf{Future Directions:} Moving forward, we plan to address the generalization issue by extending our training dataset to encompass a more diverse range of sequences. This should enable our model to perform better on unseen data and improve its overall performance. Additionally, we aim to explore the integration of other novel techniques and architectural enhancements to further enhance the capabilities of \textbf{ELiOT} and push the boundaries of LiDAR odometry accuracy and efficiency. By continuously refining and advancing our approach, we hope to contribute significantly to the field of autonomous navigation and robotic perception.

% \section*{Acknowledgment}

% The preferred spelling of the word ``acknowledgment'' in America is without 
% an ``e'' after the ``g''. Avoid the stilted expression ``one of us (R. B. 
% G.) thanks $\ldots$''. Instead, try ``R. B. G. thanks$\ldots$''. Put sponsor 
% acknowledgments in the unnumbered footnote on the first page.
% 7

\bibliographystyle{unsrt}
\bibliography{cite}

% \vspace{12pt}
% \color{red}
% IEEE conference templates contain guidance text for composing and formatting conference papers. Please ensure that all template text is removed from your conference paper prior to submission to the conference. Failure to remove the template text from your paper may result in your paper not being published.

\end{document} 
