\section{Linearity of the glucose rate of appearance}\label{sec:LinearGlucoseRateOfAppearance}
Here, we show that the glucose rate of appearance in the blood, as a function of time, is linear in the total meal carbohydrate content, $D$, for linear state space models and the model developed by~\citet{DallaMan:etal:2006, DallaMan:etal:2007}, provided that the meal input, $d$, is linear in $D$. That is the case when $d$ is an impulse or step function as described in Section~\ref{sec:simulation:meal:inputs}. Consequently, the meal response can be computed for $D = 1$ and scaled in order to obtain the response for any other value of $D$. For brevity of notation, we do not explicitly indicate the time-dependence of the states.

\begin{remark}
There exist meal input functions, $d$, that are not linear in $D$. For instance, if $d$ is a step function, the meal duration may increase linearly in $D$ while the glucose flow rate remains constant.
\end{remark}

\subsection{Linear models}
The linear meal models are in the form
%
\begin{subequations}
\begin{align}
    \dot x &= A_c x + B_c d, \\
    y &= C_c x.
\end{align}
\end{subequations}
%
We introduce the normalized states, inputs, and outputs
%
\begin{subequations}
\begin{align}
    \tilde x &= x/D, &
    \tilde d &= d/D, &
    \tilde y &= y/D.
\end{align}
\end{subequations}
%
Note that, by assumption, $\tilde d$ is independent of $D$. Then,
%
\begin{align}
    \dot{\tilde x} &= \dot x/D = A_c x/D + B_c d/D = A_c \tilde x + B_c \tilde d, \\
    \tilde y &= y/D = C_c x/D = C_c \tilde x,
\end{align}
%
and, given a simulation of this normalized system, the glucose rate of appearance can be obtained for any meal by scaling the normalized response, i.e., $y = \tilde y D$.

\subsection{Dalla Man model}
As for the linear state space models, we introduce the normalized state variables
%
\begin{subequations}
\begin{align}
    q_{sto, 1} &= Q_{sto, 1}/D, &
    q_{sto, 2} &= Q_{sto, 2}/D, \\
    q_{gut}    &= Q_{gut}   /D, &
    q_{sto}    &= Q_{sto}   /D,
\end{align}
\end{subequations}
%
and the normalized meal input $\tilde d = d/D$. Note that $q_{sto} = q_{sto, 1} + q_{sto, 2}$.
%
The flow rates $R_{12}$ and $R_{gut, pla}$ and the glucose rate of appearance, $R_A$, are linear in their arguments, and they do not depend directly on $D$, i.e.,
%
\begin{align}
    R_{12}(Q_{sto, 1}) &= R_{12}(q_{sto, 1}) D, \\
    R_{gut, pla}(Q_{gut}) &= R_{gut, pla}(q_{gut}) D, \\
    R_A(Q_{gut}) &= R_A(q_{gut}) D.
\end{align}
%
In contrast, $R_{sto, gut}$ depends on the gastric emptying rate, $k_{empt} = k_{empt}(Q_{sto}, D)$, which 1)~depends directly on $D$ and 2)~is nonlinear in both its arguments. However, the arguments are not independent, and we show that $k_{empt}$ is independent of $D$ when $Q_{sto} = q_{sto} D$, i.e., that $k_{empt}(q_{sto} D, D) = k_{empt}(q_{sto})$. First, we note that
%
\begin{align}
    \alpha D &= \frac{5}{2 (1 - b)}, &
    \beta  D &= \frac{5}{2 c}.
\end{align}
%
Next, using these expressions and substituting $Q_{sto} = q_{sto} D$,
%
\begin{subequations}
\begin{align}
    \alpha (Q_{sto} - b D)
    &= \alpha (q_{sto} D - b D) = \alpha D (q_{sto} - b) \nonumber \\
    &= \frac{5}{2}\frac{q_{sto} - b}{1 - b}, \\
    %
    \beta (Q_{sto} - c D)
    &= \beta (q_{sto} D - c D) = \beta D (q_{sto} - c) \nonumber \\
    &= \frac{5}{2}\frac{q_{sto} - c}{c}.
\end{align}
\end{subequations}
%
Finally, we insert into the expression for the gastric emptying rate:
%
\begin{align}
    k_{empt}
    =&\, k_{min} + \frac{k_{max} - k_{min}}{2}\Bigg(\tanh\left(\alpha (Q_{sto} - b D)\right) \nonumber \\
    &- \tanh\left(\beta (Q_{sto} - c D)\right) + 2\Bigg) \nonumber \\
    % =&\, k_{min} + \frac{k_{max} - k_{min}}{2}\Bigg(\tanh\left(\alpha D (q_{sto} - b)\right) \nonumber \\
    % &- \tanh\left(\beta D(q_{sto} - c)\right) + 2\Bigg) \nonumber \\
    =&\, k_{min} + \frac{k_{max} - k_{min}}{2}\Bigg(\tanh\left(\frac{5}{2}\frac{q_{sto} - b}{1 - b}\right) \nonumber \\
    &- \tanh\left(\frac{5}{2}\frac{q_{sto} - c}{c}\right) + 2\Bigg).
\end{align}
%
Clearly, $k_{empt}$ is independent of $D$.
%
Consequently, $R_{sto, gut}$ is linear in $D$ for $Q_{sto, 1} = q_{sto, 1} D$ and $Q_{sto, 2} = q_{sto, 2} D$:
%
\begin{align}
    R_{sto, gut}&(Q_{sto, 1}, Q_{sto, 2}, D) \nonumber \\
    &= k_{empt}(Q_{sto}, D) Q_{sto, 2} \nonumber \\
    % &= R_{sto, gut}(q_{sto, 1} D, q_{sto, 2} D, D) \nonumber \\
    &= k_{empt}(q_{sto} D, D) q_{sto, 2} D \nonumber \\
    &= k_{empt}(q_{sto}) q_{sto, 2} D \nonumber \\
    &= R_{sto, gut}(q_{sto, 1}, q_{sto, 2}) D.
\end{align}
%
In conclusion, the normalized variables are described by
%
\begin{subequations}
\begin{align}
    \dot q_{sto, 1}
    &=  \dot Q_{sto, 1}/D = d/D - R_{12}(Q_{sto, 1})/D \nonumber \\
    &= \tilde d - R_{12}(q_{sto, 1}), \\
    %
    \dot q_{sto, 2}
    &= \dot Q_{sto, 2}/D \nonumber \\
    &= R_{12}(Q_{sto, 1})/D - R_{sto, gut}(Q_{sto, 1}, Q_{sto, 2}, D)/D \nonumber \\
    &= R_{12}(q_{sto, 1}) - R_{sto, gut}(q_{sto, 1}, q_{sto, 2}), \\
    %
    \dot q_{gut}
    &= \dot Q_{gut}/D \nonumber \\
    &= R_{sto, gut}(Q_{sto, 1}, Q_{sto, 2}, D)/D \nonumber \\
    & \phantom{=} - R_{gut, pla}(Q_{gut})/D \nonumber \\
    &= R_{sto, gut}(q_{sto, 1}, q_{sto, 2}) - R_{gut, pla}(q_{gut}).
\end{align}
\end{subequations}
%
Given a simulation of this system, the glucose rate of appearance for any meal carbohydrate content, $D$, can be computed as $R_A(q_{gut}) D$.

% \subsection{Hovorka model}

% Note to TOBK: Perhaps show it for the general linear state space model instead of for each of the linear models.

% \begin{subequations}
% \begin{align}
%     d_1 &= D_1/D, \\
%     d_2 &= D_2/D
% \end{align}
% \end{subequations}

% \begin{subequations}
% \begin{align}
%     \dot d_1 &= d(t)/D - R_{12}(d_1), \\
%     \dot d_2 &= R_{12}(d_1) - R_2(d_2)
% \end{align}
% \end{subequations}

% \begin{align}
%     R_A &= R_A(d_2) D
% \end{align}æø

% \subsection{SIMO model}

% \begin{subequations}
% \begin{align}
%     s &= S/D, \\
%     j &= J/D, \\
%     r &= R/D, \\
%     l &= L/D
% \end{align}
% \end{subequations}

% \begin{subequations}
% \begin{align}
%     \dot s &= d(t)/D - R_{SJ}(s), \\
%     \dot j &= R_{SJ}(s) - R_{JR}(j) - R_{A, J}(j), \\
%     \dot r &= R_{JR}(j) - R_{RL}(r), \\
%     \dot l &= R_{RL}(r) - R_{A, P}(l)
% \end{align}
% \end{subequations}

% \begin{align}
%     R_A(J, L) = R_A(j, l) D
% \end{align}

% \subsection{CSTR-PFR model}