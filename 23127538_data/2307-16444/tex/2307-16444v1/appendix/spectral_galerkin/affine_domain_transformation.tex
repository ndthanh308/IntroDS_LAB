\subsection{Domain transformation}\label{sec:affine:domain:transformation}
If the physical spatial domain is not $[z_0, z_f] = [-1, 1]$, we transform the system by introducing the spatial coordinate $\xi = \xi(z) = 2 \frac{z - z_0}{z_f - z_0} - 1 \in [-1, 1]$. We use the inverse transformation, $z = z(\xi) = \frac{1}{2}(\xi + 1)(z_f - z_0) + z_0$, to express the PDE and the boundary condition in terms of $\xi$. For simplicity, in this appendix, we assume that the velocity, $v$, and the diffusion coefficient, $D_c$, are independent of space, $z$, and concentration, $c$. First, we use the chain rule to derive the partial derivatives of the concentration with respect to $\xi$:
%
\begin{subequations}
	\begin{align}
		\partial_\xi c
		&= \partial_z c \diff{z}{\xi}, \\
		%
		\partial_{\xi\xi} c
		&= \partial_{zz} c \left(\diff{z}{\xi}\right)^2 + \partial_z c \ndiff[2]{z}{\xi} = \partial_{zz} c \left(\diff{z}{\xi}\right)^2.
	\end{align}
\end{subequations}
%
We have exploited that $z$ is linear in $\xi$, i.e., $\ndiff[2]{z}{\xi} = 0$. Consequently,
%
\begin{align}
	\partial_z c &= \partial_\xi c \left(\diff{z}{\xi}\right)^{-1}, &
	%
	\partial_{zz} c &= \partial_{\xi\xi} c \left(\diff{z}{\xi}\right)^{-2}.
\end{align}
%
Next, we use the chain rule to express the partial derivatives of the flux:
%
\begin{subequations}
	\begin{align}
		\partial_\xi N
		&= \partial_z N \diff{z}{\xi}
		= \left(v \partial_z c + \partial_z J\right) \diff{z}{\xi} 
		= v \partial_\xi c + \partial_\xi J, \\
		%
		\partial_\xi J
		&= \partial_z J \diff{z}{\xi}
		= - D_c \partial_{zz} c \diff{z}{\xi} 
		= -D_c \partial_{\xi\xi} c\left(\diff{z}{\xi}\right)^{-1}.
	\end{align}
\end{subequations}
%
Consequently,
%
\begin{align}
	\partial_t c
	&= -\partial_z N + Q = -\partial_\xi N \left(\diff{z}{\xi}\right)^{-1} + Q,
\end{align}
%
and
%
\begin{subequations}
	\begin{align}
		%
		\partial_\xi N \left(\diff{z}{\xi}\right)^{-1} &= v \left(\diff{z}{\xi}\right)^{-1} \partial_\xi c + \partial_\xi J \left(\diff{z}{\xi}\right)^{-1}, \\
		%
		\partial_\xi J \left(\diff{z}{\xi}\right)^{-1}
		&= -D_c \left(\diff{z}{\xi}\right)^{-2} \partial_{\xi\xi} c.
	\end{align}
\end{subequations}
%
Therefore, the transformed system
%
\begin{align}
    \partial_t c &= -\partial_\xi \bar N + Q,
\end{align}
%
where
%
\begin{subequations}
	\begin{align}
		\bar N &= \bar v c + \bar J, \\
		\bar J &= -\bar D_c \partial_\xi c, \\
		\bar v &= v \left(\diff{z}{\xi}\right)^{-1}, \\
		\bar D_c &= D_c \left(\diff{z}{\xi}\right)^{-2},
	\end{align}
\end{subequations}
%
is in the form~\eqref{eq:general:pde}, and $\xi \in [-1, 1]$. The two main differences between the original and the transformed system are the velocity and the diffusion coefficient. Note that we have exploited that $\diff{z}{\xi}$ is independent of $\xi$ (i.e., it is constant).
%
The boundary condition in the transformed system is
%
\begin{align}\label{eq:spectral:galerkin:transformed:boundary:condition}
	A \tilde N \rvert_{z = z_0} &= F,
\end{align}
%
where
%
\begin{subequations}
	\begin{align}
		\tilde N &= v c + \tilde J, \\
		\tilde J &= -D_c \partial_z c = -D_c \partial_\xi c \left(\diff{z}{\xi}\right)^{-1} = -\tilde D_c \partial_\xi c, \\
		\tilde D_c &= D_c \left(\diff{z}{\xi}\right)^{-1}.
	\end{align}
\end{subequations}
%
Finally, we reformulate the integral in~\eqref{eq:duodenum:glucose} in the CSTR-PFR model with Alsk{\"{a}}r's feedback mechanism. We use that $\incr z = \diff{z}{\xi} \incr \xi$:
%
\begin{align}
	m_d &= A \int_{z_0}^{z_d} c \incr z = A \int_{\xi_0}^{\xi_d} c \diff{z}{\xi} \incr \xi,
\end{align}
%
where $\xi_0 = -1$ and $\xi_d = \xi(z_d)$.
%
\begin{remark}
    As $\diff{z}{\xi}$ is constant, the boundary condition~\eqref{eq:spectral:galerkin:transformed:boundary:condition} can also be formulated in terms of the transformed flux $\bar N$, i.e.,
    %
    \begin{align}
        A \bar N \rvert_{z=z_0} = \bar F,
    \end{align}
    %
    where
    %
    \begin{align}
        \bar F = F \left(\diff{z}{\xi}\right)^{-1}.
    \end{align}
\end{remark}