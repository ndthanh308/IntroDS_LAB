\subsection{Jacobi polynomials and quadrature}\label{sec:spectral:galerkin:jacobi:polynomials}
We denote by $P_k^{(\alpha, \beta)}$ a general $k$'th order Jacobi polynomial~\citep{Kopriva:2009}, and we describe two special cases: 1)~Legendre polynomials and 2)~Chebyshev polynomials.
%
Both polynomials can be used in a Gauss or Gauss-Lobatto quadrature rule:
%
\begin{align}
    \int_{z_0}^{z_f} f(z) w(z) \incr z \approx \sum_{l=0}^M f(z_l) w_l.
\end{align}
%
The weight function $w$ and the weights $\{w_l\}_{l=0}^M$ are specific to each Jacobi polynomial. Gauss quadrature rules are exact for polynomials of up to order $2 M + 1$, but do not include the endpoints, $z_0$ and $z_f$. The endpoints are included in Gauss-Lobatto quadrature rules, which are only exact for polynomials of up to order $2 M - 1$.
%
\begin{remark}
    The Sturm-Liouville problem consists of the following differential equation combined with boundary conditions on $u$ (not shown).
    %
    \begin{align}
    	-\diff{}{z}\left(p(z) \diff{u}{z}\right) + q(z) u &= \lambda w(z) u, & a &< z < b.
    \end{align}
    %
    Jacobi polynomials, $P_k^{(\alpha, \beta)}(z)$, are eigenfunctions of the specific Sturm-Liouville problem
    %
    \begin{align}
    	-\diff{}{z}\left((1 - z)^{1 + \alpha} (1 + z)^{1 + \beta} \diff{u}{z}\right) \nonumber \\
    	= \lambda (1 - z)^{\alpha} (1 + z)^{\beta} u,
    \end{align}
    %
    where $\alpha, \beta > -1$ and $-1 < z < 1$.
\end{remark}

\subsubsection{Legendre polynomials}
The $k$'th order Legendre polynomial, $L_k = P_k^{(0, 0)}$, is obtained with $\alpha = \beta = 0$, and it is defined recursively starting with $L_0(z) = 1$ and $L_1(z) = z$. Subsequently,
%
\begin{align}\label{eq:legendre}
    L_{k+1}(z) &= \frac{2k+1}{k+1} z L_k(z) - \frac{k}{k+1} L_{k-1}(z),
\end{align}
%
and the weight function is
%
\begin{align}\label{eq:legendre:w}
	w(z) &= 1.
\end{align}
%
The Legendre polynomials also satisfy
%
\begin{align}
    (2k + 1) L_k(z) &= \diff{L_{k+1}}{z}(z) - \diff{L_{k-1}}{z}(z).
\end{align}
%
The Legendre Gauss collocation points, $\{z_l\}_{l=0}^M$, are the zeros of $L_{M+1}$, and the weights are
%
\begin{align}
	w_l &= \frac{2}{(1 - z_l^2)\left(\diff{L_{M+1}}{z}(z_l)\right)^2}, & l &= 0, \ldots, M.
\end{align}
%
The Legendre Gauss-Lobatto collocation points, $\{z_l\}_{l=0}^M$, are -1, 1, and the zeros of $\diff{L_M}{z}$, and the weights are
%
\begin{align}\label{eq:quadrature:gauss:lobatto:legendre:w}
	w_l &= \frac{2}{M(M + 1)} \frac{1}{\left(L_M(z_l)\right)^2}, & l &= 0, \ldots, M.
\end{align}

\subsubsection{Chebyshev polynomials}
For Chebyshev polynomials, $\alpha = \beta = -1/2$ and the $k$'th order polynomial is denoted by $T_k = P_k^{(-1/2, -1/2)}$. Chebyshev polynomials are given by the explicit expression
%
\begin{align}
	T_k(z) &= \cos\left(k \cos^{-1}(z)\right),
\end{align}
%
but they also satisfy a recursion. It starts with $T_0(z) = 1$ and $T_1(z) = z$ and is followed by
%
\begin{align}
	T_{k+1}(z) &= 2 z T_k(z) - T_{k-1}(z).
\end{align}
%
They also satisfy
%
\begin{align}
	2 T_k(z) &= \frac{1}{k+1} \diff{T_{k+1}}{z}(z) - \frac{1}{k-1} \diff{T_{k-1}}{z}(z),
\end{align}
%
and the weight function is
%
\begin{align}
	w(z) &= \frac{1}{\sqrt{1 - z^2}}.
\end{align}
%
The Chebyshev Gauss collocation points and weights are given by
%
\begin{subequations}
	\begin{align}
		z_l &= \cos\left(\frac{2l + 1}{2M + 2} \pi\right), & l &= 0, \ldots, M, \\
		w_l &= \frac{\pi}{M + 1}, & l &= 0, \ldots, M,
	\end{align}
\end{subequations}
%
and the Chebyshev Gauss-Lobatto collocation points and weights are
%
\begin{subequations}\label{eq:quadrature:gauss:lobatto:chebyshev}
	\begin{align}
		\label{eq:quadrature:gauss:lobatto:chebyshev:x}
		z_l &= \cos\left(\frac{l\pi}{M}\right), \\
		%
		\label{eq:quadrature:gauss:lobatto:chebyshev:w}
		w_l &=
		\begin{cases}
			\frac{\pi}{2 M}, & l \in \{0, M\}, \\
			\frac{\pi}{M}, & l = 1, \ldots, M-1,
		\end{cases}
	\end{align}
\end{subequations}
%
for $l = 0, \ldots, M$.