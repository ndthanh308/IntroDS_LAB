\section{Finite volume discretization}\label{sec:finite:volume}
In this appendix, we present a finite volume discretization of the PFR model~\eqref{eq:pfr},
%
\begin{align}\label{eq:finite:volume:pde}
    \partial_t c &= -\partial_z N + Q,
\end{align}
%
with the boundary condition
%
\begin{align}\label{eq:finite:volume:boundary:condition}
    A N \rvert_{z = z_0} &= F.
\end{align}
%
For simplicity, we ignore the reaction term, $R$, which is treated in the same way as the source term, $Q$.
%
We discretize the cylindrical domain, $\Omega$, as shown in Fig.~\ref{fig:grid}. That is, we split it into $M$ smaller non-overlapping volumes (also cylinders) such that $\Omega = \bigcup_{i = 0}^{M-1} \Omega_i$, where $\Omega_i = \{s = (z, r, \theta) | z\in[z_i, z_{i+1}], r \in [0, r_c], \theta \in [0, 2\pi[\}$ and $r_c$ is the radius of the cylinder.
%
First, we integrate over each volume, i.e.,
%
\begin{align}
	\int_{\Omega_i} \partial_t c \incr s &= -\int_{\Omega_i} \partial_z N \incr s + \int_{\Omega_i} Q \incr s, \nonumber \\
	%
	&= -A \int_{z_i}^{z_{i+1}} \partial_z N \incr z + A \int_{z_i}^{z_{i+1}} Q \incr z,
\end{align}
%
for $i = 0, \ldots, M-1$. We have interchanged integration and differentiation on the left-hand side, and exploited that the concentration is identical in the plane perpendicular to the motion through the cylinder.
%
Next, we 1)~use the definition of concentration to define the glucose mass $m_i = \int_{\Omega_i} c \incr s$, 2)~apply Gauss' divergence theorem to the first term on the right-hand side, and 3)~approximate $Q$ as constant in each volume:
%
\begin{align}
	\dot m_i &= -A (N_{i+1} - N_i) + F_i, & i &= 0, \ldots, M-1.
\end{align}
%
Here, $N_i$ is an approximation of the flux on the left boundary of the $i$'th volume, and
%
\begin{align}
	F_i &= A \Delta z_i Q_i, & i &= 0, \ldots, M-1,
\end{align}
%
where $\Delta z_i = z_{i+1} - z_i$ and $Q_i = Q(c_i)$. Next, we 1)~use the boundary condition~\eqref{eq:finite:volume:boundary:condition}, 2)~use an upwind scheme to approximate the advection term, 3)~use a first-order finite difference approximation of the spatial derivative in the diffusion term, and 4)~assume that there's no diffusion at the end of the cylinder:
%
\begin{subequations}
\begin{align}
	N_0 &= F/A, \\
	N_i &= N_{a, i} + N_{d, i}, & i &= 1, \ldots, M, \\
	N_{a, i} &= v c_{i-1}, & i &= 1, \ldots, M, \\
	N_{d, i} &= -D_c \frac{c_i - c_{i-1}}{\Delta z_{c, i-1}}, & i &= 1, \ldots, M-1, \\
	N_{d, M} &= 0.
\end{align}
\end{subequations}
%
The center of the $i$'th volume is
%
\begin{align}
	z_{c, i} &= z_i + \frac{1}{2} \Delta z_i = z_i + \frac{1}{2}(z_{i+1} - z_i) \nonumber \\
	&= \frac{z_{i+1} + z_i}{2}, \qquad i = 0, \ldots, M-1,
\end{align}
%
and the distance between the $i$'th and $i+1$'th cell center is
%
\begin{align}
	\Delta z_{c, i} &= z_{c, i+1} - z_{c, i}, & i &= 0, \ldots, M-2.
\end{align}

For completeness, we also describe the discretization of
%
\begin{align}
    m_d &= A \int_{z_0}^{z_d} c \incr z,
\end{align}
%
which is used in the CSTR-PFR model with Alsk{\"{a}}r's feedback mechanism~\eqref{eq:duodenum:glucose}.
%
Let $K$ be the number of volumes for which $z_{K} \leq z_d < z_{K + 1}$. Then, assuming that the glucose is evenly distributed in each volume,
%
\begin{align}
	m_d
	&= A\left(\int_{z_K}^{z_d} c \incr z + \sum_{i=0}^{K-1} \int_{z_i}^{z_{i+1}} c \incr z\right) \nonumber \\
	&\approx \frac{z_d - z_{K}}{z_{K + 1} - z_{K}} m_K + \sum_{i=0}^{K-1} m_i.
\end{align}
%
First, we have split up the integral, and then, we have used the definition of concentration and the assumption of even distribution of the glucose.
%
% Figure environment removed