\section{Spectral Galerkin discretization}\label{sec:spectral:galerkin}
In this appendix, we describe a spectral Galerkin discretization~\citep{Kopriva:2009} of the PFR model~\eqref{eq:pfr}. As with the finite volume discretization, we disregard the reaction term, $R$, because it is discretized in the same way as the source term, $Q$. That is, we discretize the system
%
\begin{subequations}\label{eq:general:pde}
	\begin{align}
		\partial_t c &= -\partial_z N + Q, \\
		A N \rvert_{z = z_0} &= F.
	\end{align}
\end{subequations}
%
In this appendix, we assume that $z\in[-1, 1]$ (referred to as the computational domain). That is typically not the case. However, the actual physical domain can be mapped onto the computational domain and the system can be transformed accordingly, as described in Appendix~\ref{sec:affine:domain:transformation}.

We approximate the solution, $c$, as a sum of products between time-dependent functions and space-dependent polynomials:
%
\begin{align}\label{eq:spectral:galerkin:approximate:concentration}
	c(t, z)
	&\approx \hat c(t, z) = \sum_{m=0}^M \hat c_m \ell_m.
\end{align}
%
Here, $\hat c_m = \hat c_m(t) = \hat c(t, z_m)$ is the $m$'th time-dependent coefficient, $\{z_m\}_{m=0}^M$ is a set of collocation points, and $\ell_m = \ell_m(z)$ is the $m$'th Lagrange polynomial (see Appendix~\ref{sec:spectral:galerkin:lagrange:polynomials}). An important property of such polynomials is that $\ell_m(z_i) = \delta_{im}$, i.e., it is equal to one when evaluated in the $m$'th collocation point and zero when evaluated in any other collocation point.
%
For brevity of notation, we often omit the dependency on time and space when functions are evaluated at $t$ and $z$.
%
Furthermore, for clarity, we explicitly indicate the arguments of the flux, $N$, and the source term, $Q$.

We require that the approximate solution, $\hat c$, satisfies the PDE weakly. Specifically,
%
\begin{align}\label{eq:spectral:galerkin:weak:form}
	\int_{z_0}^{z_f} \left(\partial_t \hat c + \partial_z N(\hat c) - Q(\hat c)\right) \phi \incr z &= 0
\end{align}
%
must be satisfied for any test function $\phi$ that belongs to the same function space as the approximate solution, i.e., for any polynomial.
%
As for the solution, we write the test function as a Lagrange polynomial:
%
\begin{align}\label{eq:spectral:galerkin:weak:form:test:polynomial}
	\phi(t, z) &= \sum_{n=0}^M \phi_n \ell_n.
\end{align}
%
Here, $\phi_n = \phi_n(t) = \phi_n(t, z_n)$ is the $n$'th time-dependent coefficient. We substitute the expression for the test function in~\eqref{eq:spectral:galerkin:weak:form}:
%
\begin{align}\label{eq:spectral:galerkin:weak:form:2}
	\sum_{n=0}^M \phi_n \int_{z_0}^{z_f} (\partial_t \hat c + \partial_z N(\hat c) - Q(\hat c)) \ell_n \incr z &= 0.
\end{align}
%
This equality must be satisfied for any polynomial $\phi$, which means that it must be satisfied for any combination of values of $\{\phi_n\}_{n=0}^m$ at any point in time. Therefore, the integral must equal zero, i.e.,
%
\begin{align}\label{eq:spectral:galerkin:weak:form:3}
	\int_{z_0}^{z_f} (\partial_t \hat c + \partial_z N(\hat c) - Q(\hat c)) \ell_n \incr z &= 0,
\end{align}
%
for $n = 0, \ldots, M$. Next, we use integration by parts to rewrite the integral of the flux term. The result is
%
\begin{align}
	\int_{z_0}^{z_f} \partial_z N(\hat c) \ell_n \incr z =\,& \left[N(\hat c) \ell_n\right]_{z_0}^{z_f} \nonumber \\
	&- \int_{z_0}^{z_f} N(\hat c) \diff{\ell_n}{z} \incr z,
\end{align}
%
for $n = 0, \ldots, M$. We insert this expression and the expression for the approximate concentration~\eqref{eq:spectral:galerkin:approximate:concentration} into~\eqref{eq:spectral:galerkin:weak:form:3} in order to obtain
%
\begin{align}\label{eq:spectral:galerkin:before:quadrature}
	&\sum_{m=0}^M \diff{\hat c_m}{t} \int_{z_0}^{z_f} \ell_m \ell_n \incr z + \left[N(\hat c) \ell_n\right]_{z_0}^{z_f} \nonumber \\
	&- \int_{z_0}^{z_f} N(\hat c) \diff{\ell_n}{z} \incr z - \int_{z_0}^{z_f} Q(\hat c) \ell_n \incr z = 0,
\end{align}
%
for $n = 0, \ldots, M$. We approximate the integrals using quadrature (see Appendix~\ref{sec:spectral:galerkin:jacobi:polynomials}). Consequently,
%
\begin{align}
	&\sum_{m=0}^M \diff{\hat c_m}{t} \sum_{l=0}^M \ell_m(z_l) \ell_n(z_l) w_l + \left[N(\hat c) \ell_n\right]_{z_0}^{z_f} \nonumber \\
	&- \sum_{l=0}^M N(\hat c(t, z_l)) \diff{\ell_n}{z}(z_l) w_l \nonumber \\
	&- \sum_{l=0}^M Q(\hat c(t, z_l)) \ell_n(z_l) w_l = 0,
\end{align}
%
for $n = 0, \ldots, M$. We exploit that $\ell_m(z_l) = \delta_{ml}$ and that $\hat c(t, z_l) = \hat c_l(t)$:
%
\begin{align}
	&\diff{\hat c_n}{t} w_n + \left[N(\hat c) \ell_n\right]_{z_0}^{z_f} - \sum_{l=0}^M N(\hat c_l) \diff{\ell_n}{z}(z_l) w_l \nonumber \\
	&- Q(\hat c_n) w_n = 0, \quad n = 0, \ldots, M.
\end{align}
%
Finally, we rearrange terms in order to obtain the ODEs for each of the coefficients,
%
\begin{align}\label{eq:spectral:galerkin:final}
	\diff{\hat c_n}{t} =\,& -\frac{1}{w_n} \left[N(\hat c) \ell_n(z)\right]_{z_0}^{z_f} \nonumber \\
	&+ \frac{1}{w_n} \sum_{l=0}^M N(\hat c_l) \diff{\ell_n}{z}(z_l) w_l + Q(\hat c_n),
\end{align}
%
for $n = 0, \ldots, M$.
%
\begin{remark}
When using Gauss-Lobatto quadrature, the boundary contribution in~\eqref{eq:spectral:galerkin:final} (i.e., the first term on the right-hand side) is only nonzero for the boundary coefficient (i.e., for $n=0$) because $z_0$ and $z_f$ are collocation points. For Gauss quadrature, the boundary contribution is nonzero for all of the differential equations because $z_0$ and $z_f$ are not collocation points.
\end{remark}

\subsection{Jacobi polynomials and quadrature}\label{sec:spectral:galerkin:jacobi:polynomials}
We denote by $P_k^{(\alpha, \beta)}$ a general $k$'th order Jacobi polynomial~\citep{Kopriva:2009}, and we describe two special cases: 1)~Legendre polynomials and 2)~Chebyshev polynomials.
%
Both polynomials can be used in a Gauss or Gauss-Lobatto quadrature rule:
%
\begin{align}
    \int_{z_0}^{z_f} f(z) w(z) \incr z \approx \sum_{l=0}^M f(z_l) w_l.
\end{align}
%
The weight function $w$ and the weights $\{w_l\}_{l=0}^M$ are specific to each Jacobi polynomial. Gauss quadrature rules are exact for polynomials of up to order $2 M + 1$, but do not include the endpoints, $z_0$ and $z_f$. The endpoints are included in Gauss-Lobatto quadrature rules, which are only exact for polynomials of up to order $2 M - 1$.
%
\begin{remark}
    The Sturm-Liouville problem consists of the following differential equation combined with boundary conditions on $u$ (not shown).
    %
    \begin{align}
    	-\diff{}{z}\left(p(z) \diff{u}{z}\right) + q(z) u &= \lambda w(z) u, & a &< z < b.
    \end{align}
    %
    Jacobi polynomials, $P_k^{(\alpha, \beta)}(z)$, are eigenfunctions of the specific Sturm-Liouville problem
    %
    \begin{align}
    	-\diff{}{z}\left((1 - z)^{1 + \alpha} (1 + z)^{1 + \beta} \diff{u}{z}\right) \nonumber \\
    	= \lambda (1 - z)^{\alpha} (1 + z)^{\beta} u,
    \end{align}
    %
    where $\alpha, \beta > -1$ and $-1 < z < 1$.
\end{remark}

\subsubsection{Legendre polynomials}
The $k$'th order Legendre polynomial, $L_k = P_k^{(0, 0)}$, is obtained with $\alpha = \beta = 0$, and it is defined recursively starting with $L_0(z) = 1$ and $L_1(z) = z$. Subsequently,
%
\begin{align}\label{eq:legendre}
    L_{k+1}(z) &= \frac{2k+1}{k+1} z L_k(z) - \frac{k}{k+1} L_{k-1}(z),
\end{align}
%
and the weight function is
%
\begin{align}\label{eq:legendre:w}
	w(z) &= 1.
\end{align}
%
The Legendre polynomials also satisfy
%
\begin{align}
    (2k + 1) L_k(z) &= \diff{L_{k+1}}{z}(z) - \diff{L_{k-1}}{z}(z).
\end{align}
%
The Legendre Gauss collocation points, $\{z_l\}_{l=0}^M$, are the zeros of $L_{M+1}$, and the weights are
%
\begin{align}
	w_l &= \frac{2}{(1 - z_l^2)\left(\diff{L_{M+1}}{z}(z_l)\right)^2}, & l &= 0, \ldots, M.
\end{align}
%
The Legendre Gauss-Lobatto collocation points, $\{z_l\}_{l=0}^M$, are -1, 1, and the zeros of $\diff{L_M}{z}$, and the weights are
%
\begin{align}\label{eq:quadrature:gauss:lobatto:legendre:w}
	w_l &= \frac{2}{M(M + 1)} \frac{1}{\left(L_M(z_l)\right)^2}, & l &= 0, \ldots, M.
\end{align}

\subsubsection{Chebyshev polynomials}
For Chebyshev polynomials, $\alpha = \beta = -1/2$ and the $k$'th order polynomial is denoted by $T_k = P_k^{(-1/2, -1/2)}$. Chebyshev polynomials are given by the explicit expression
%
\begin{align}
	T_k(z) &= \cos\left(k \cos^{-1}(z)\right),
\end{align}
%
but they also satisfy a recursion. It starts with $T_0(z) = 1$ and $T_1(z) = z$ and is followed by
%
\begin{align}
	T_{k+1}(z) &= 2 z T_k(z) - T_{k-1}(z).
\end{align}
%
They also satisfy
%
\begin{align}
	2 T_k(z) &= \frac{1}{k+1} \diff{T_{k+1}}{z}(z) - \frac{1}{k-1} \diff{T_{k-1}}{z}(z),
\end{align}
%
and the weight function is
%
\begin{align}
	w(z) &= \frac{1}{\sqrt{1 - z^2}}.
\end{align}
%
The Chebyshev Gauss collocation points and weights are given by
%
\begin{subequations}
	\begin{align}
		z_l &= \cos\left(\frac{2l + 1}{2M + 2} \pi\right), & l &= 0, \ldots, M, \\
		w_l &= \frac{\pi}{M + 1}, & l &= 0, \ldots, M,
	\end{align}
\end{subequations}
%
and the Chebyshev Gauss-Lobatto collocation points and weights are
%
\begin{subequations}\label{eq:quadrature:gauss:lobatto:chebyshev}
	\begin{align}
		\label{eq:quadrature:gauss:lobatto:chebyshev:x}
		z_l &= \cos\left(\frac{l\pi}{M}\right), \\
		%
		\label{eq:quadrature:gauss:lobatto:chebyshev:w}
		w_l &=
		\begin{cases}
			\frac{\pi}{2 M}, & l \in \{0, M\}, \\
			\frac{\pi}{M}, & l = 1, \ldots, M-1,
		\end{cases}
	\end{align}
\end{subequations}
%
for $l = 0, \ldots, M$.
\subsection{Lagrange polynomials}\label{sec:spectral:galerkin:lagrange:polynomials}
Here, we describe the Lagrange polynomials which we use several times in the derivation of the spectral Galerkin method presented in this section.
%
For arbitrary $z$, the $m$'th Lagrange polynomial of order $M+1$ and its derivatives are~\citep{Berrut:Trefethen:2004}
%
\begin{subequations}
	\begin{align}
		\ell_m(z) &= \prod_{\substack{l = 0\\l \neq m}}^M \frac{z - z_l}{z_m - z_l} = \frac{1}{s(z)} \frac{\tilde w_m}{z - z_m}, \\
		%
		\diff{\ell_m}{z}(z) &= \frac{1}{s(z)}\left(\frac{-\tilde w_m}{(z - z_m)^2} - \ell_m(z) \diff{s}{z}(z)\right), \\
		%
		\ndiff[2]{\ell_m}{z}(z)
		&= \frac{1}{s(z)}\bigg(\frac{2 \tilde w_m}{(z - z_m)^3} - 2 \diff{\ell_m}{z}(z) \diff{s}{z}(z) \nonumber \\
		&- \ell_m(z) \ndiff[2]{s}{z}(z)\bigg),
	\end{align}
\end{subequations}
%
where the corresponding weight is
%
\begin{align}
	\tilde w_m &= \prod_{\substack{l = 0\\l \neq m}}^M \frac{1}{z_m - z_l},
\end{align}
%
and the auxiliary function, $s$, and its derivatives are given by
%
\begin{subequations}
	\begin{align}
		s(z) &= \sum_{l=0}^M \frac{\tilde w_l}{z - z_l}, \\
		%
		\diff{s}{z}(z) &= \sum_{l=0}^M \frac{-\tilde w_l}{(z - z_l)^2}, \\
		%
		\ndiff[2]{s}{z}(z) &= \sum_{l=0}^M \frac{2 \tilde w_l}{(z - z_l)^3}.
	\end{align}
\end{subequations}
%
The above expressions for the derivative of $\ell_m$ cannot be evaluated in the collocation points, $\{z_m\}_{m=0}^M$, because it would result in division by zero. In the collocation points, the Lagrange polynomials and their derivatives are
%
\begin{subequations}
	\begin{align}
		\ell_m(z_l) &= \delta_{ml}, \\
		%
		\diff{\ell_m}{z}(z_l) &=
		\begin{cases}
			\frac{\tilde w_m}{\tilde w_l (z_l - z_m)}, & l \neq m, \\
			-\sum\limits_{\substack{j = 0\\j\neq l}}^M \diff{\ell_j}{z}(z_i), & l = m,
		\end{cases} \\
		%
		\ndiff[2]{\ell_m}{z}(z_l) &=
		\begin{cases}
			-2 \diff{\ell_m}{z}(z_l)\left(-\diff{\ell_l}{z}(z_l) + \frac{1}{z_l - z_m}\right), & l \neq m, \\
			-\sum\limits_{\substack{j = 0\\j\neq l}}^M \ndiff[2]{\ell_j}{z}(z_i), & l = m.
		\end{cases}
	\end{align}
\end{subequations}
%
\begin{remark}
There is a sign error in the last term in the parenthesis on the right-hand side of~(9.4) in the paper by~\cite{Berrut:Trefethen:2004}.
\end{remark}
\subsection{Domain transformation}\label{sec:affine:domain:transformation}
If the physical spatial domain is not $[z_0, z_f] = [-1, 1]$, we transform the system by introducing the spatial coordinate $\xi = \xi(z) = 2 \frac{z - z_0}{z_f - z_0} - 1 \in [-1, 1]$. We use the inverse transformation, $z = z(\xi) = \frac{1}{2}(\xi + 1)(z_f - z_0) + z_0$, to express the PDE and the boundary condition in terms of $\xi$. For simplicity, in this appendix, we assume that the velocity, $v$, and the diffusion coefficient, $D_c$, are independent of space, $z$, and concentration, $c$. First, we use the chain rule to derive the partial derivatives of the concentration with respect to $\xi$:
%
\begin{subequations}
	\begin{align}
		\partial_\xi c
		&= \partial_z c \diff{z}{\xi}, \\
		%
		\partial_{\xi\xi} c
		&= \partial_{zz} c \left(\diff{z}{\xi}\right)^2 + \partial_z c \ndiff[2]{z}{\xi} = \partial_{zz} c \left(\diff{z}{\xi}\right)^2.
	\end{align}
\end{subequations}
%
We have exploited that $z$ is linear in $\xi$, i.e., $\ndiff[2]{z}{\xi} = 0$. Consequently,
%
\begin{align}
	\partial_z c &= \partial_\xi c \left(\diff{z}{\xi}\right)^{-1}, &
	%
	\partial_{zz} c &= \partial_{\xi\xi} c \left(\diff{z}{\xi}\right)^{-2}.
\end{align}
%
Next, we use the chain rule to express the partial derivatives of the flux:
%
\begin{subequations}
	\begin{align}
		\partial_\xi N
		&= \partial_z N \diff{z}{\xi}
		= \left(v \partial_z c + \partial_z J\right) \diff{z}{\xi} 
		= v \partial_\xi c + \partial_\xi J, \\
		%
		\partial_\xi J
		&= \partial_z J \diff{z}{\xi}
		= - D_c \partial_{zz} c \diff{z}{\xi} 
		= -D_c \partial_{\xi\xi} c\left(\diff{z}{\xi}\right)^{-1}.
	\end{align}
\end{subequations}
%
Consequently,
%
\begin{align}
	\partial_t c
	&= -\partial_z N + Q = -\partial_\xi N \left(\diff{z}{\xi}\right)^{-1} + Q,
\end{align}
%
and
%
\begin{subequations}
	\begin{align}
		%
		\partial_\xi N \left(\diff{z}{\xi}\right)^{-1} &= v \left(\diff{z}{\xi}\right)^{-1} \partial_\xi c + \partial_\xi J \left(\diff{z}{\xi}\right)^{-1}, \\
		%
		\partial_\xi J \left(\diff{z}{\xi}\right)^{-1}
		&= -D_c \left(\diff{z}{\xi}\right)^{-2} \partial_{\xi\xi} c.
	\end{align}
\end{subequations}
%
Therefore, the transformed system
%
\begin{align}
    \partial_t c &= -\partial_\xi \bar N + Q,
\end{align}
%
where
%
\begin{subequations}
	\begin{align}
		\bar N &= \bar v c + \bar J, \\
		\bar J &= -\bar D_c \partial_\xi c, \\
		\bar v &= v \left(\diff{z}{\xi}\right)^{-1}, \\
		\bar D_c &= D_c \left(\diff{z}{\xi}\right)^{-2},
	\end{align}
\end{subequations}
%
is in the form~\eqref{eq:general:pde}, and $\xi \in [-1, 1]$. The two main differences between the original and the transformed system are the velocity and the diffusion coefficient. Note that we have exploited that $\diff{z}{\xi}$ is independent of $\xi$ (i.e., it is constant).
%
The boundary condition in the transformed system is
%
\begin{align}\label{eq:spectral:galerkin:transformed:boundary:condition}
	A \tilde N \rvert_{z = z_0} &= F,
\end{align}
%
where
%
\begin{subequations}
	\begin{align}
		\tilde N &= v c + \tilde J, \\
		\tilde J &= -D_c \partial_z c = -D_c \partial_\xi c \left(\diff{z}{\xi}\right)^{-1} = -\tilde D_c \partial_\xi c, \\
		\tilde D_c &= D_c \left(\diff{z}{\xi}\right)^{-1}.
	\end{align}
\end{subequations}
%
Finally, we reformulate the integral in~\eqref{eq:duodenum:glucose} in the CSTR-PFR model with Alsk{\"{a}}r's feedback mechanism. We use that $\incr z = \diff{z}{\xi} \incr \xi$:
%
\begin{align}
	m_d &= A \int_{z_0}^{z_d} c \incr z = A \int_{\xi_0}^{\xi_d} c \diff{z}{\xi} \incr \xi,
\end{align}
%
where $\xi_0 = -1$ and $\xi_d = \xi(z_d)$.
%
\begin{remark}
    As $\diff{z}{\xi}$ is constant, the boundary condition~\eqref{eq:spectral:galerkin:transformed:boundary:condition} can also be formulated in terms of the transformed flux $\bar N$, i.e.,
    %
    \begin{align}
        A \bar N \rvert_{z=z_0} = \bar F,
    \end{align}
    %
    where
    %
    \begin{align}
        \bar F = F \left(\diff{z}{\xi}\right)^{-1}.
    \end{align}
\end{remark}