\section{Mathematical models and simulation}\label{sec:simulation}
We consider mathematical models of the meal nutrient absorption in the gastrointestinal tract, which are in the form of initial value problems involving ODEs:
% 
% To describe the time evolution of the systemic absorption from the gastro-intestinal tract, we  consider initial value problems described by a system of ordinary differential equations
% 
\begin{align}\label{eq:IVP:ODE}
    \dot x(t) &= f(x(t), d(t), p_f), & x(t_0) &= x_0.
\end{align}
%
Here, $t$ is time, $x$ are the states, $d$ are the meal inputs, and $p_f$ are the parameters in the model, $f$. $x_0$ are the states at time $t_0$. The states constitute the minimal amount of information necessary for simulating the future evolution of the system~\eqref{eq:IVP:ODE}.

The outputs, $y$, are described by the function
%
\begin{equation}
\label{eq:NonlinearOutputFunction}
    y(t) = g(x(t), p_g),
\end{equation}
%
where $p_g$ is a parameter vector. The purpose of the model is to describe the relation between the outputs, $y$, and 1)~the meal inputs, $d$, and 2)~the parameters, $p_f$ and $p_g$, which are specific to each person (and possibly also to each meal).

\begin{remark}
Models that contain PDEs or DDEs can be approximated by models that only contain ODEs by means of spatial discretizations and delay approximations, respectively.
\end{remark}

\subsection{Typical structure of nonlinear models}
Nonlinear models of meal nutrient absorption are often more structured than the system~\eqref{eq:IVP:ODE}. Specifically, many nonlinear models are affine in the meal inputs:
%
\begin{equation}
\begin{split}
    \dot x(t) &= f(x(t), d(t), p_f) \\
    &= f_x(x(t), p_{f_x}) + f_d(x(t), p_{f_d}) d(t), \quad x(t_0) = x_0.
\label{eq:IVP:ODE:NonlinearAffineInputModel}
\end{split}
\end{equation}
%
The first term describes the internal dynamics of the meal absorption and the second term describes the direct effect of the meal inputs on the states, e.g., the relation between the amount of glucose in the meal and in the stomach. The parameter vectors $p_{f_x}$ and $p_{f_d}$ contain the same parameters as $p_f$.

\subsection{Linear models}
Several meal models are linear in the states, $x$, and the meal inputs, $d$:
%
\begin{subequations}
\label{eq:LinearStateSpaceModel}
\begin{align}
    \dot x(t) &= A_c(p_{f_x}) x(t) + B_c(p_{f_d}) d(t), \quad x(t_0) = x_0, \label{eq:LinearStateSpaceModel:Dynamics} \\
    y(t) &=  C_c(p_g) x(t).
\end{align}
\end{subequations}
%
The subscript $c$ on the system matrices, $A_c$, $B_c$, and $C_c$, indicate that it is a \emph{continuous-time} linear state space model (as opposed to a \emph{discrete-time} state space model).

\begin{remark}
The linear state space model~\eqref{eq:LinearStateSpaceModel} is a special case of the nonlinear model \eqref{eq:IVP:ODE}--\eqref{eq:NonlinearOutputFunction} where
%
\begin{subequations}
\begin{align}
    f(x(t), d(t), p_f) &= A_c(p_{f_x}) x(t) + B_c(p_{f_d}) d(t), \\
    g(x(t), p_g) &= C_c(p_g) x(t).
\end{align}
\end{subequations}
%
The dynamical equation in the linear state space model~\eqref{eq:LinearStateSpaceModel:Dynamics} is also a special case of the meal input-affine model~\eqref{eq:IVP:ODE:NonlinearAffineInputModel} where
%
\begin{subequations}
\begin{align}
    f_x(x(t), p_{f_x}) &= A_c(p_{f_x}) x(t), \\
    f_d(x(t), p_{f_d}) &= B_c(p_{f_d}).
\end{align}
\end{subequations}
\end{remark}

\subsection{Meal inputs}\label{sec:simulation:meal:inputs}
Some models of meal nutrient absorption represent the meal inputs as flow rates, i.e., as step functions, and others represent them as instantaneous, i.e., as impulses.

\subsubsection{Step inputs}
When the meal inputs are represented using step functions, they are described by
% The meal consumption can be simulated using piecewise constant inputs, i.e., using a finite flow rate,
%
\begin{equation}
    d(t) = d_k = D_k/\Delta t, \quad t_k \leq t < t_{k+1},
\end{equation}
%
where $D_k$ is the total meal size ingested in the interval $[t_k, \,t_{k+1}[$ and $\Delta t = t_{k+1}-t_k$. The response to a sequence of piecewise constant meal inputs of sizes $\set{D_k}_{k=0}^{M-1}$ may be simulated by setting $x(t_0) = x_0$ and solving the $M$ initial value problems
%
\begin{subequations}
\begin{align}
    x(t_k) &= x_k, \\
    \dot x(t) &= f(x(t), d_k, p_f), & t_{k} &\leq t < t_{k+1}, \\
    x_{k+1} &= x(t_{k+1}),
\end{align}
\end{subequations}
%
for $k = 0,1, \ldots, M-1$. The result is the sequence of states $\set{x_k}_{k=0}^M$ that may be used to compute the corresponding sequence of outputs $\set{y_k}_{k=0}^M$ from \eqref{eq:NonlinearOutputFunction}.

\subsubsection{Impulse inputs}
For this type of meal input model, we only consider the input-affine model~\eqref{eq:IVP:ODE:NonlinearAffineInputModel}.
%
A single meal of size $D$, which is consumed instantaneously at time $t_0$, can be represented as an impulse,
%
\begin{equation}\label{eq:single:meal:impulse}
    d(t) = D \delta(t - t_0),
\end{equation}
%
using the Dirac delta function, $\delta(t)$. 
%
We denote by $t_0^-$ the time $t_0$ before the impulse and by $t_0^+$ the time $t_0$ immediately after the impulse. The impulse function has three relevant properties:
%
\begin{subequations}
\begin{align}
    d(t_0) &= \infty, \\
    d(t) &= 0, & t_0 &< t,  \\
    \int_{t_0^-}^{t_0^+} d(t) \incr t &= D.
\end{align}
\end{subequations}
%
Consequently, the states immediately before and after the impulse are
%
\begin{subequations}
\begin{align}
x(t_0^-) &= x_0^- = x_0, \\
x(t_0^+) &= x_0^+ = x_0^- + f_d(x_0^-, p_{f_d}) D.    
\end{align}
\end{subequations}
%
Therefore, the initial value problem~\eqref{eq:IVP:ODE:NonlinearAffineInputModel} with the meal input function~\eqref{eq:single:meal:impulse} can be simulated by solving the initial value problem
%
\begin{equation}
    \dot x(t) = f(x(t), 0, p_f) = f_x(x(t), p_{f_x}), \quad x(t_0^+) = x_0^+,
\end{equation}
%
for $t \geq t_0^+$. The corresponding output, $y(t)$, computed by \eqref{eq:NonlinearOutputFunction} is called the impulse response of the system~\eqref{eq:IVP:ODE:NonlinearAffineInputModel} and~\eqref{eq:NonlinearOutputFunction} to the meal impulse, $D$, provided that $x(t_0) = x_0 = x_{ss}$ is a steady state, i.e., that $x_{ss}$ satisfies $f(x_{ss}, 0, p_f) = f_x(x_{ss}, p_{f_x}) = 0$.

Multiple instantaneous meals (i.e., impulses) of sizes $\set{D_k}_{k=0}^{M-1}$ at times $\set{t_k}_{k=0}^{M-1}$ can be represented by the input function
%
\begin{equation}
\label{eq:MultipleImpulseInputFunction}
    d(t) = \sum_{k=0}^{M-1} D_k \delta(t - t_k).
\end{equation}
%
This input function has the three properties
%
\begin{subequations}
\begin{align}
    d(t_k) &= \infty, \\
    d(t) &= 0, & t_k &< t < t_{k+1},  \\
    \int_{t_{k}^-}^{t_k^+} d(t) \incr t &= D_k.
\end{align}
\end{subequations}
%
The definitions of $t_k^-$ and $t_k^+$ are analogous to those of $t_0^-$ and $t_0^+$, respectively. Because of these properties, the system~\eqref{eq:IVP:ODE:NonlinearAffineInputModel} with the multiple meal impulse input function \eqref{eq:MultipleImpulseInputFunction} may be simulated by using that $x_0^- = x_0$ and solving the $M$ initial value problems
%
\begin{subequations}
\begin{align}
    x(t_k^+)  &= x_k^- + f_d(x_k^-,p_{f_d}) D_k, \\
    \dot x(t) &= f_x(x(t),p_{f_x}),  & t_k^+ &< t < t_{k+1}^-, \\
    x_{k+1}^- &= x(t_{k+1}^-),
\end{align}
\end{subequations}
%
for $k= 0, \ldots, M-1$. As mentioned previously,
%
\begin{equation}
    f_x(x(t), p_{f_x}) = f(x(t), 0, p_f),
\end{equation}
%
which means that the simulation can be carried out with the general dynamic model~\eqref{eq:IVP:ODE} using $d(t) = 0$ for $t_k < t < t_{k+1}$.