\section{Discussion}\label{sec:discussion}
Table~\ref{tab:model:comparison} shows the main characteristics of the models described in Section~\ref{sec:meal:models}: 1)~the types of equations in the model, 2)~the number of states, 3)~whether it is a linear state space model or not, and 4)~whether or not the glucose rate of appearance is linear in the total meal carbohydrate content, $D$. It is more straightforward to simulate models that only contain ODEs. The reason is that PDEs are typically approximated by a set of ODEs using spatial discretization (this is called the method of lines). However, this approximation is derived analytically, and it is problem-specific. In contrast, there exists general-purpose software for simulating models that only contain ODEs. In Appendix~\ref{sec:finite:volume} and~\ref{sec:spectral:galerkin}, we describe two spatial discretization schemes that are relevant to meal models containing PDEs. The approximation often results in a large number of ODEs. Consequently, it is more computationally intensive to simulate models that contain PDEs because the computation time depends strongly on the number of states. Next, linear state space models are simpler to analyze than nonlinear models, and, in important special cases, explicit expressions for their solutions can be derived. Similarly, it can be exploited in both analysis and simulation if a model is linear in the total meal carbohydrate content, $D$. Specifically, if $R_A^{(1)}$ is the glucose rate of appearance over time for $D = 1$, the rate of appearance for any meal carbohydrate content is $R_A^{(1)} D$ if the model is linear in $D$.

Only the CSTR-PFR model contains a PDE (describing the glucose transport in the small intestine). Consequently, when discretized, it will contain more states than the other models, and it will be more computationally intensive to simulate. All the other models contain a small number of states. Furthermore, Hovorka's model and the SIMO model are linear. The CSTR-PFR model is also linear if the pylorus sphincter is modeled as always being open, i.e., if there is no feedback mechanism. The remaining models are nonlinear. Finally, Hovorka's and Dalla Man's models, the SIMO model, and the CSTR-PFR model without feedback are linear in the total meal carbohydrate content, $D$, (see Appendix~\ref{sec:LinearGlucoseRateOfAppearance}).

Fig.~\ref{fig:glucoseRateOfAppearanceComparisonMealSizes} shows the response to meals with different carbohydrate contents. The meal consumption is modeled as instantaneous (as described in Section~\ref{sec:simulation:meal:inputs}). The parameter values used in the various models (see Appendix~\ref{sec:parameter:values}) do not represent the same individual. Therefore, we show the glucose rate of appearance in the blood normalized with body weight. The meal responses predicted by the linear models, i.e., Hovorka's model, the SIMO model, and the CSTR-PFR model without feedback, are qualitatively similar. After an initial rise, the glucose rate of appearance slowly decays to zero. In contrast, Dalla Man's model predicts two peaks. After the initial rise, the rate of appearance decreases and then increases again before decaying to zero. The second peak represents the delayed carbohydrate absorption caused by, e.g., fat and protein in the meal. None of the other models predict more than one peak. In Alsk{\"{a}}r's model, there is a pronounced saturation effect, and the larger meals do not lead to significantly higher glucose absorption rates. Instead, the absorption is prolonged for larger meals. For the largest meal, the CSTR-PFR model using Moxon's feedback mechanism shows a similar saturation effect. However, for the two smaller meals, the saturation threshold is not reached and the simulations are almost identical to those obtained without a feedback mechanism. When Alsk{\"{a}}r's feedback mechanism is used in the CSTR-PFR model, the glucose rate of appearance does not saturate. Instead, after a fast but short rise where the duodenum is filled, it increases slowly. For larger meals, it increases for a longer time. Finally, the simulations clearly demonstrate that the glucose rate of appearance is linear in $D$ for Hovorka's and Dalla Man's models, the SIMO model, and the CSTR-PFR model without feedback.

In Fig.~\ref{fig:glucoseRateOfAppearanceComparisonImpulseAndStep}, we compare the two meal input models discussed in Section~\ref{sec:simulation:meal:inputs}, i.e., 1)~the instantaneous model using an impulse function and 2)~the constant flow rate model using a step function. The two representations are almost identical if the meal is consumed over 5~min. However, there is a pronounced lag for almost all of the models if the meal is consumed over 30~min. The exceptions are Alsk{\"{a}}r's model and the CSTR-PFR model using Alsk{\"{a}}r's feedback mechanism. The reason is that the feedback limits the amount of glucose that can enter into the duodenum. Consequently, the rate at which the stomach is filled has a smaller impact than in the other models.

% \begin{remark}
% A mathematical model of the glucoregulatory system can predict multiple peaks in the blood glucose concentration . For instance, people with type 1 diabetes inject exogenous insulin because their bodies do not produce it themselves. Consequently, the effect of the injected insulin might wear off before the glucose rate of appearance begins to decrease. In particular, that could be the case when using Alsk{\"{a}}r's model or the CSTR-PFR model with Moxon's or Alsk{\"{a}}r's feedback mechanism.
% \end{remark}

\begin{table*}
    \centering
    \caption{Main characteristics of the meal models described in Section~\ref{sec:meal:models}. For the CSTR-PFR model, the number of states depends on the discretization of the PDE (resulting in $M$ ODEs). The models are considered linear if they are in the form of a linear state space model~\eqref{eq:linear:state:space:meal:model}. The CSTR-PFR model is linear if $k_{sd}$ in~\eqref{eq:cstr:pfr:stomach:fsd} is constant and nonlinear if it is described by~\eqref{eq:moxon:pyloric:sphincter} or~\eqref{eq:alskar:pyloric:sphincter}. In Appendix~\ref{sec:LinearGlucoseRateOfAppearance}, we show that the glucose rate of appearance is linear in the total meal carbohydrate content, $D$, for some of the models.}
    \label{tab:model:comparison}
    \begin{tabular}{lcccc}
        \hline
         Model & Types of equations & Number of states & Linear & Linear in $D$ \\
         \hline
         \citet{Hovorka:etal:2004}                      & ODEs          & 2        & Yes       & Yes       \\
         \citet{DallaMan:etal:2006, DallaMan:etal:2007} & ODEs          & 3        & No        & Yes       \\
         \citet{DeGaetano:etal:2013}                    & ODEs          & 4        & Yes       & Yes       \\
         \citet{Alskar:etal:2016}                       & ODEs          & 4        & No        & No        \\
         CSTR-PFR~\citep{Moxon:etal:2016}               & ODEs and PDEs & 1 + $M$  & Yes/no    & Yes/no    \\
        \hline
    \end{tabular}
\end{table*}

% Figure environment removed
%
% % Figure environment removed
% 
% Figure environment removed