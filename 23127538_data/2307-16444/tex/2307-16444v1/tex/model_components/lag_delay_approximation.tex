\subsubsection{Lag approximation}
The transfer function in~\eqref{eq:Laplace:TimeDelayModelYU:G} can be approximated by the transfer function of a lag process, i.e.,
%
\begin{align}
    G(s) &\approx \frac{1}{\tau_d s + 1} = \frac{1/\tau_d}{s + 1/\tau_d} = \frac{P(s)}{Q(s)} = \tilde G(s).
\end{align}
%
The system matrices in the corresponding linear state space realization, in observable canonical form~\citep[Chap.~3.9]{Hendricks:etal:2008}, are
%
\begin{subequations}
\begin{align}
    A_c &= - 1/\tau_d, & B_c &= 1/\tau_d, \\
    C_c &= 1, & D_c &= 0.
\end{align}
\end{subequations}

We apply the same approximation to the system~\eqref{eq:Laplace:TimeDelayModelYU:Multiple}:
%
\begin{align}
    G_i(s)
    &\approx \frac{1}{(\tau_d/M) s + 1} = \frac{M/\tau_d}{s + M/\tau_d} = \frac{P_i(s)}{Q_i(s)} \nonumber \\
    &= \tilde G_i(s).
\end{align}
%
Again, we consider the corresponding state space realization in observable canonical form. In this case, the system matrices are
%
\begin{subequations}
\label{eq:lag:multiple:system:matrices}
\begin{align}
    A_{c, ij} &=
    \begin{cases}
        -M/\tau_d, & i = j, \\
         M/\tau_d, & i = j-1, \\
         0, & \text{otherwise},
    \end{cases} \\
    %
    B_{c, i} &=
    \begin{cases}
        M/\tau_d, & i = 1, \\
        0, & \text{otherwise},
    \end{cases} \\
    %
    C_{c, i} &=
    \begin{cases}
        1, & i = M, \\
        0, & \text{otherwise},
    \end{cases} \\
    %
    D_c &= 0,
\end{align}
\end{subequations}
%
for $i = 1, \ldots, M$ and $j = 1, \ldots, M$.