\subsection{PFR}\label{sec:models:components:pfr}
The parts of the gastrointestinal tract where advective and diffusive transport phenomena are significant can be described as PFRs. A PFR is cylindrical and the concentration, $c = c(t, z, r, \theta) = c(t, z)$, only changes along the transport direction, $z$, i.e., it is constant along the radial and angular coordinates, $r$ and $\theta$.

The spatiotemporal evolution of the concentration is described by the PDE
%
\begin{equation}\label{eq:pfr}
    \partial_t c = - \partial_z N + R + Q,
\end{equation}
%
where $N$ is flux, $R$ is the production rate, and $Q$ is a source term. The flux is the sum of an advection term, $N_a$, and a diffusion term, $N_d$:
%
\begin{equation}
    N = N_a + N_d.
\end{equation}
%
These terms are
%
\begin{subequations}\label{eq:PFR:Flux:Terms}
\begin{align}
    \label{eq:PFR:Flux:Terms:Advection}
    N_a &= v c, \\
    %
    \label{eq:PFR:Flux:Terms:Diffusion}
    N_d &= -D_c \partial_z c,
\end{align}
\end{subequations}
%
where $v$ is velocity and $D_c$ is the diffusion coefficient. The expression~\eqref{eq:PFR:Flux:Terms:Diffusion} is called Fick's law.

% \begin{equation}
%     q(t) = \int_{0}^{L} Q(t, z) \incr z
% \end{equation}

% % Figure environment removed