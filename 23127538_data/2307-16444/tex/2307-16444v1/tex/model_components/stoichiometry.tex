\subsection{Stoichiometry and reaction kinetics}\label{sec:stoichiometry}
Consider a set of molecules $\mathcal{C}$ which are involved in a set of reactions $\mathcal{R}$ in the human metabolism. Let $S \in \R^{n_r \times n_c}$ be the matrix of stoichiometric coefficients for this set of reactions and molecules. $n_c$ is the number of molecules and $n_r$ is the number of reactions. Let $c$ be the vector of concentrations such that we can express the rate vector, $r$, for this set of reactions as the function 
%
\begin{equation}
    r = r(c).
\end{equation}
%
Consequently, the production rate vector for the molecules can be expressed as
%
\begin{equation}
    R = S' r.
\end{equation}
%
This general way of expressing the production rate, $R$, is useful because it only requires the specification of the chemical reaction stoichiometry (and the corresponding stoichiometric matrix, $S$) as well as the corresponding expression for the reaction rates, $r = r(c)$.