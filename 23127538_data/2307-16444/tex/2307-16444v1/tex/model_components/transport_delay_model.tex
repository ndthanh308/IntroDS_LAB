\subsubsection{Physical transport delay model}
Delays can also be represented using transport processes. The input signal, $u$, constitutes the boundary condition,
%
\begin{align}
    c_{in}(t) = u(t),
\end{align}
%
and the initial boundary value problem
%
\begin{subequations}
\label{eq:PDE:timedelay}
\begin{align}
    c(t,0) &= c_{in}(t), \\
    \partial_t c &= - v \partial_z c, & t &\geq 0, & 0 &\leq z \leq L,
\end{align}
\end{subequations}
%
has the analytical solution $y(t) = c(t, L) = c_{in}(t - \tau_d) = u(t - \tau_d)$ with the delay $\tau_d = L/v$.
% Consequently, \eqref{eq:PDE:timedelay} may be used to represent time delays.

\begin{remark}
    A left-sided first-order finite difference discretization of the PDE~\eqref{eq:PDE:timedelay}, based on an equidistant grid with $M+1$ nodes, is equivalent to the linear state space model~\eqref{eq:linear:state:space:model} with the system matrices~\eqref{eq:lag:multiple:system:matrices} obtained using a series of $M$ lag approximations.
\end{remark}