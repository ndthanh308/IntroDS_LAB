\section{Introduction}
Mathematical modeling and simulation of the human metabolism are central to treating and preventing two of the major pandemics of the 21\textsuperscript{st} century; diabetes and obesity~\citep{Pattaranit:vandenBerg:2008}. Model-based simulation can both support scientific developments within physiology, help to improve drug development processes~\citep{Huang:etal:2009}, and be used directly in support tools, e.g., in automated insulin delivery systems for people with diabetes~\citep{Lal:etal:2019}.
%
Specifically, modeling is a key component of virtual clinical trials~\citep{Reenberg:etal:2022b, Ritschel:etal:2022}, rigorous mathematical analysis~\citep{Cohen:Li:2021}, and of model-based algorithms for, e.g., monitoring, prediction, control, and optimization~\citep{Boiroux:etal:2018b}.

The human metabolism is a complex set of chemical reactions that are responsible for sustaining life. Their purposes are to 1)~digest food, 2)~convert the energy in the food into a form that can be used in cellular processes, 3)~convert food into building blocks for nucleic acids, proteins, carbohydrates, and lipids, 4)~transport substances into and between cells, and 5)~eliminate waste from metabolic processes.
%
Food digestion and absorption are particularly important to mathematical modeling in diabetes and obesity~\citep{Gouseti:etal:2019, LeFeunteun:etal:2020}.
%
Many models describe the dynamics of glucose and insulin and disregard other macronutrients (fat and protein) and hormones (e.g., glucagon, ghrelin, and incretins). Furthermore, it is common to describe meal glucose absorption using simple algebraic relations~\citep{Silber:etal:2010} or to only consider intravenous glucose injection~\citep{Silber:etal:2007}. However, models that include the dynamics of glucagon~\citep{Adams:Lasseigne:2018}, ghrelin~\citep{Barnabei:etal:2022}, and incretins~\citep{Jauslin:etal:2007}, as well as the absorption of other macronutrients~\citep{Sicard:etal:2018}, have also been proposed. Recently, \citet{Pompa:etal:2021} compared three models commonly used in diabetes using a simulation study. However, they conclude that it is not possible to determine which of the models that is more physiologically accurate based on simulations alone. \citet{Noguchi:etal:2014} propose a model which accounts for the digestion and absorption of carbohydrates based on the glycemic index and carbohydrate bioavailability. \citet{Moxon:etal:2016} propose three models which include transport along the small intestine. Later, both \citet{Noguchi:etal:2016} and \citet{Moxon:etal:2017} extended their respective models with an upper bound on the glucose rate of appearance in the blood stream.
%
We refer to the reviews by~\citet{Smith:etal:2009}, \citet{Palumbo:etal:2013}, and~\citet{Huard:Kirkham:2022} for further information on models in the literature.

Apart from the physiological phenomena included in the models, there are several differences between the underlying mathematical formulations. Some models are purely compartmental~\citep{DeGaetano:etal:2013} and described only by ordinary differential equations~(ODEs). Others also use partial differential equations~(PDEs), e.g., to describe the transport through the small intestine~\citep{Moxon:etal:2016}.
%
Similarly, there are models that represent delays exactly~\citep{Contreras:etal:2020, Cohen:Li:2021} using delay differential equations~(DDEs), and, in other cases, they are approximated~\citep{Alskar:etal:2016}.
%Some models use delay differential equations~(DDEs) to represent delays exactly~\citep{Contreras:etal:2020, Cohen:Li:2021} and others approximate them, e.g., using a simple algebraic relation~\citep{Alskar:etal:2016}.
%
Furthermore, meal consumption can either be represented as a finite flow rate of nutrients~\citep{Hovorka:etal:2004} or as instantaneous~\citep{DallaMan:etal:2014}. Finally, while most models are deterministic, some also include stochasticity (uncertainty), e.g., to model variations in the meal size and consumption time~\citep{Chudtong:DeGaetano:2021}. These different mathematical formulations are also discussed in the review by~\citet{Makroglou:etal:2006}.

In this work, we present a critical discussion of five commonly used mathematical models of meal glucose absorption: 1)~the model proposed by~\citet{Hovorka:etal:2004}, 2)~the UVA/Padova model presented by~\citet{DallaMan:etal:2006, DallaMan:etal:2007}, 3)~the SIMO model described by~\citet{DeGaetano:etal:2013} and used in the revised Sorensen model by~\citet{Panunzi:etal:2020}, 4)~the model by~\citet{Alskar:etal:2016}, and 5)~a model which represents the stomach as a continuous stirred-tank reactor (CSTR) and the small intestine as a plug-flow reactor (PFR)~\citep{Moxon:etal:2016, Moxon:etal:2017}. In the last model, we compare different models of the opening and closing of the pylorus valve, which connects the stomach to the duodenum in the small intestine. Furthermore, we discuss general aspects of mathematical modeling relevant to the human metabolism (representation of meals, delays, and general modeling components).

The remainder of the paper is structured as follows. In Section~\ref{sec:simulation}, we discuss the simulation of mathematical meal models, and in Section~\ref{sec:models:components}, we present modeling components that are relevant to modeling of the human metabolism in general. In Section~\ref{sec:meal:models}, we present the five models of meal glucose absorption mentioned above, and in Section~\ref{sec:discussion}, we discuss and compare them based on simulations. Finally, we present conclusions in Section~\ref{sec:conclusions}.