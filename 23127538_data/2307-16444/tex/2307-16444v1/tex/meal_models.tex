\section{Meal models}\label{sec:meal:models}
In this section, we present five commonly used models of glucose absorption in the gastrointestinal tract: The model by~\citet{Hovorka:etal:2004}, the model by~\citet{DallaMan:etal:2006, DallaMan:etal:2007}, the SIMO model by~\citet{DeGaetano:etal:2013}, the model by~\citet{Alskar:etal:2016}, and a CSTR-PFR model based on the ones proposed by~\citet{Moxon:etal:2016, Moxon:etal:2017}. For the CSTR-PFR model, we consider different descriptions of the pylorus sphincter (or valve) which connects the stomach to the small intestine.

Several of the models are linear. Specifically, they are in the form
%
\begin{subequations}
\label{eq:linear:state:space:meal:model}
\begin{align}
    \dot x(t) &= A_c x(t) + B_c d(t), \\
         y(t) &= C_c x(t),
\end{align}
\end{subequations}
%
where $d$ is the meal input and $y$ is the glucose rate of appearance in the blood plasma.
%
Furthermore, in Appendix~\ref{sec:LinearGlucoseRateOfAppearance}, we show that, for some of the models, $y$ is a linear function of the total meal carbohydrate content, $D$.
%
For brevity of notation, we omit the time dependency in the remainder of this section.

\subsection{Hovorka's model}
The model by~\citet{Hovorka:etal:2004} contains two compartments, as illustrated in Fig.~\ref{fig:hovorka:diagram}.
%
% Figure environment removed
%
The first compartment, $D_1$, describes the amount of glucose in the stomach, and the second compartment, $D_2$, describes the amount of glucose in the small intestine:
%
\begin{subequations}
\begin{alignat}{3}
    \dot D_1 &= A_G d - R_{12}, \\
    \dot D_2 &= R_{12} - R_2.
\end{alignat}
\end{subequations}
%
Here, $A_G$ describes the bioavailability of the carbohydrates in the meal, and $R_{12} = R_{12}(D_1)$ is the glucose flow rate between the stomach and the small intestine. Furthermore, $R_2 = R_2(D_2)$ describes the glucose absorption, and the glucose rate of appearance, $R_A = R_A(D_2)$, is a fraction, $f$, of $R_2$:
%
\begin{subequations}
\begin{align}
    R_{12} &= D_1/\tau_D, \\
    R_2 &= D_2/\tau_D, \\
    R_A &= f R_2.
\end{align}
\end{subequations}
%
The parameter $\tau_D$ is a time constant, and the model is a linear state space model in the form~\eqref{eq:linear:state:space:meal:model}, where the system matrices are
%
\begin{align}
    A_c &= \begin{bmatrix} \frac{-1}{\tau_D} & 0 \\ \frac{1}{\tau_D} & \frac{-1}{\tau_D} \end{bmatrix}, &
    B_c &= \begin{bmatrix} A_G \\ 0 \end{bmatrix}, &
    C_c &= \begin{bmatrix} 0 & \frac{f}{\tau_D}\end{bmatrix}.
\end{align}
\subsection{Dalla Man's model}
The model by~\citet{DallaMan:etal:2006, DallaMan:etal:2007} is sketched in Fig.~\ref{fig:uvapadova:diagram}, and it contains three compartments: The glucose in the solid and liquid phases of the stomach content, $Q_{sto, 1}$ and $Q_{sto, 2}$, respectively, and the amount of glucose in the small intestine, $Q_{gut}$.
%
% Figure environment removed
%
The compartments are described by
%
\begin{subequations}
\begin{align}
    \dot Q_{sto,1} &= d - R_{12}, \\
    \dot Q_{sto,2} &= R_{12} - R_{sto,gut},  \\
    \dot Q_{gut}   &= R_{sto,gut} - R_{gut, pla},
\end{align}
\end{subequations}
%
where $R_{12} = R_{12}(Q_{sto, 1})$ is the glucose flow rate between the liquid and solid phase in the stomach, $R_{sto, gut} = R_{sto, gut}(Q_{sto, 1}, Q_{sto, 2}, D)$ is the flow rate between the stomach and the small intestine, and $D$ is the total carbohydrate content of the meal. Furthermore, $R_{gut, pla} = R_{gut, pla}(Q_{gut})$ is the glucose absorption rate, and the glucose rate of appearance in the blood plasma, $R_A = R_A(Q_{gut})$, is a fraction, $f$, of the glucose absorption rate:
%
\begin{subequations}
\begin{align}
    R_{12} &= k_{gri} Q_{sto, 1}, \\
    R_{sto, gut} &= k_{empt} Q_{sto, 2}, \\
    R_{gut, pla} &= k_{abs} Q_{gut}, \\
    R_A &= f R_{gut, pla}.
\end{align}
\end{subequations}
%
Here, $k_{gri}$ and $k_{abs}$ are the inverses of time constants, and the gastric emptying rate, $k_{empt} = k_{empt}(Q_{sto}, D)$, is
%
\begin{subequations}
\begin{align}
    \label{eq:dalla:man:gastric:emptying:rate}
    k_{empt}
    =&\, k_{min} + \frac{k_{max} - k_{min}}{2}\Bigg(\tanh\left(\alpha(Q_{sto} - b D)\right) \nonumber \\
    &- \tanh\left(\beta(Q_{sto} - c D)\right) + 2\Bigg), \\
    %
    \alpha
    =&\, \frac{5}{2 D (1 - b)}, \\
    %
    \beta
    =&\, \frac{5}{2 D c},
\end{align}
\end{subequations}
%
where $Q_{sto} = Q_{sto}(Q_{sto, 1}, Q_{sto, 2})$ is the total amount of glucose in the stomach:
%
\begin{equation}
    Q_{sto} = Q_{sto,1} + Q_{sto,2}.
\end{equation}
%
Furthermore, the parameters $k_{min}$ and $k_{max}$ are the minimum and maximum gastric emptying rates, and $b$ and $c$ are the percentages of $D$ where the magnitude of the derivative of $k_{empt}$ is $\frac{1}{2}(k_{max} - k_{min})$, i.e., at $Q_{sto} = bD$ and $Q_{sto} = cD$. Finally, as $k_{empt}$ in~\eqref{eq:dalla:man:gastric:emptying:rate} is nonlinear in $Q_{sto}$, the model is not in the linear form~\eqref{eq:linear:state:space:meal:model}.

% \begin{subequations}
% \begin{align}
%     A_c &=
%     \begin{bmatrix}
%         -k_{gri} & 0 & 0 \\
%          k_{gri} & -k_{empt} & 0 \\
%          0 & k_{empt} & -k_{abs}
%     \end{bmatrix}, \\
%     %
%     B_c &=
%     \begin{bmatrix}
%         1 \\ 0 \\ 0
%     \end{bmatrix}, \\
%     %
%     C_c &=
%     \begin{bmatrix}
%         0 & 0 & f k_{abs}
%     \end{bmatrix}
% \end{align}
% \end{subequations}
\subsection{The SIMO model}
The SIMO model by~\citet{DeGaetano:etal:2013} contains four compartments, and it is sketched in Fig.~\ref{fig:simo:diagram}. The compartments represent the amounts of glucose in 1)~the stomach, $S$, 2)~the jejunum, $J$, 3)~an artificial delay compartment, $R$, and 4)~the ileum, $L$:
%
% Figure environment removed
%
\begin{subequations}
\begin{align}
    \dot S &= d - R_{SJ}, \\
    \dot J &= R_{SJ} - R_{JR} - R_{A, J}, \\
    \dot R &= R_{JR} - R_{RL}, \\
    \dot L &= R_{RL} - R_{A, P}.
\end{align}
\end{subequations}
%
Here, the glucose flow rates between the stomach and the jejunum, $R_{SJ} = R_{SJ}(S)$, between the jejunum and the delay compartment, $R_{JR} = R_{JR}(J)$, and between the delay compartment and the ileum, $R_{RL} = R_{RL}(R)$, as well as the glucose absorption rates in the jejunum, $R_{A, J} = R_{A, J}(J)$, and the ileum, $R_{A, L} = R_{A, L}(L)$, are given by
%
\begin{subequations}
\begin{align}
    R_{SJ}   &= k_{js} S, \\
    R_{JR}   &= k_{rj} J, \\
    R_{RL}   &= k_{lr} R, \\
    R_{A, J} &= k_{gj} J, \\
    R_{A, L} &= k_{gl} L.
\end{align}
\end{subequations}
%
The coefficients, $k_{js}$, $k_{rj}$, $k_{lr}$, $k_{gj}$, and $k_{gl}$ are the inverses of time constants, and the glucose rate of appearance, $R_A = R_A(J, L)$, is a fraction, $f$, of the total glucose absorption:
%
\begin{align}
    R_A &= f (R_{A, J} + R_{A, L}).
\end{align}
%
This model is in the linear form~\eqref{eq:linear:state:space:meal:model}, and the system matrices are given by
%
\begin{subequations}
\begin{align}
    A_c &=
    \begin{bmatrix}
        -k_{js} &                  0 &       0 &       0 \\
         k_{js} & -(k_{gj} + k_{rj}) &       0 &       0 \\
              0 &            k_{rj}  & -k_{lr} &       0 \\
              0 &                  0 &  k_{lr} & -k_{gl}
    \end{bmatrix}, \\
    %
    B_c &=
    \begin{bmatrix}
        1 \\ 0 \\ 0 \\ 0
    \end{bmatrix}, \\
    %
    C_c &=
    \begin{bmatrix}
        0 & f k_{gj} & 0 & f k_{gl}
    \end{bmatrix}.
\end{align}
\end{subequations}
\subsection{Alsk{\"{a}}r's model}\label{sec:meal:models:alskar}
The model by~\citet{Alskar:etal:2016} contains four compartments representing the amounts of glucose in the stomach, $G_S$, the duodenum, $G_D$, the jejunum, $G_J$, and the ileum, $G_I$, as illustrated in Fig.~\ref{fig:alskar:diagram}. They are described by
%
% Figure environment removed
%
\begin{subequations}
\begin{align}
    \dot G_S &= d - R_{SD}, \\
    \dot G_D &= R_{SD} - R_{DJ} - R_{A, D}, \\
    \dot G_J &= R_{DJ} - R_{JI} - R_{A, J}, \\
    \dot G_I &= R_{JI} - R_{A, I},
\end{align}
\end{subequations}
%
where the glucose flow rates between the stomach and duodenum, $R_{SD} = R_{SD}(G_S)$, between the duodenum and jejunum, $R_{DJ} = R_{DJ}(G_D)$, and between the jejunum and ileum, $R_{JI} = R_{JI}(G_J)$, are
%
\begin{subequations}
\begin{align}
    R_{SD} &= k_{SD} \tau G_S, \\
    R_{DJ} &= k_{DJ} G_D, \\
    R_{JI} &= k_{JI} G_J.
\end{align}
\end{subequations}
%
The inverses of the time constants, $k_{SD} = k_{SD}(G_D)$, $k_{DJ}$, and $k_{JI}$, are
%
\begin{subequations}\label{eq:alskar:k}
\begin{align}
    \label{eq:alskar:k:sd}
    k_{SD} &= k_w\left(1 - \frac{G_D^\gamma}{IG_{D50}^\gamma + G_D^\gamma}\right), \\
    %
    \label{eq:alskar:k:df}
    k_{DJ} &= \frac{1}{L_D T}, \\
    %
    \label{eq:alskar:k:ji}
    k_{JI} &= \frac{1}{L_J T}.
\end{align}
\end{subequations}
%
Here, $k_{SD}$ represents the pylorus sphincter, and it is a function of the amount of glucose in the duodenum described using the Hill expression. For $G_D = 0$, $k_{SD}$ is equal to its nominal value, $k_w$, and, as $G_D$ increases, $k_{SD}$ approaches zero. For large values of the Hill coefficient, $\gamma$, $k_{SD}$ has a steep decrease around $IG_{D50}$. Furthermore, $L_D$ and $L_J$ are the relative lengths of the duodenum and jejunum (i.e., fractions of the total length of the small intestine), and $T$ is the transit time through the small intestine. The lag coefficient (used to approximate a time delay) is given by
%
\begin{align}
    \tau &= \frac{1}{1 + \exp(-\sigma(t - t_{50}))},
\end{align}
%
as described in Section~\ref{sec:models:components:delay:algebraic}. The parameter $\sigma$ determines the steepness, and $t_{50}$ is the time at which $\tau$ is 0.5. The glucose absorption rates in the duodenum, $R_{A, D} = R_{A, D}(G_D)$, jejunum, $R_{A, J} = R_{A, J}(G_J)$, and ileum, $R_{A, I} = R_{A, I}(G_I)$, are described using Michaelis-Menten expressions, i.e.,
%
\begin{subequations}
\begin{align}
    R_{A, D} &= \frac{R_{D, \max} G_D}{K_{mG} + G_D}, \\
    R_{A, J} &= \frac{R_{J, \max} G_J}{K_{mG} + G_J}, \\
    R_{A, I} &= \frac{R_{I, \max} G_I}{K_{mG} + G_I},
\end{align}
\end{subequations}
%
where $K_{mG}$ is the Michaelis constant and $R_{D, \max}$, $R_{J, \max}$, and $R_{I, \max}$ are the maximum glucose absorption rates in the duodenum, jejunum, and ileum, respectively. Finally, the glucose rate of appearance in the blood plasma, $R_A = R_A(G_D, G_J, G_I)$, is a fraction, $F_P$, of the total glucose absorption:
%
\begin{align}
    R_A &= F_P (R_{A, D} + R_{A, J} + R_{A, I}).
\end{align}
\subsection{CSTR-PFR model}\label{sec:cstr:pfr:meal:model}
The CSTR-PFR model presented here consists of a CSTR representing the stomach and a PFR representing the small intestine, as shown in Fig.~\ref{fig:cstr:pfr:diagram} (see also Section~\ref{sec:models:components:cstr} and~\ref{sec:models:components:pfr}). It is based on the second model presented by~\citet{Moxon:etal:2016}, and we describe three ways of modeling the opening and closing of the pylorus sphincter, which connects the stomach to the small intestine.
%
% Figure environment removed

The amount of glucose in the stomach, $m_s$, is given by
%
\begin{subequations}\label{eq:cstr:pfr:stomach}
\begin{align}
    \label{eq:cstr:pfr:stomach:ms}
	\dot m_s &= F_m - F_{sd}, \\
	%
	\label{eq:cstr:pfr:stomach:fsd}
	F_{sd} &= k_{sd} m_s,
\end{align}
\end{subequations}
%
where $F_m = d$ is the meal input, $F_{sd}$ is the glucose flow rate from the stomach to the duodenum, and $k_{sd}$ is the inverse of a time constant. We either consider $k_{sd}$ to be 1)~constant (the pylorus sphincter is always completely open), 2)~a function of the glucose rate of appearance in the blood (see Section~\ref{sec:cstr:pfr:meal:model:moxon}), or 3)~a function of the amount of glucose in the duodenum (see Section~\ref{sec:cstr:pfr:meal:model:alskar}).
%
The glucose concentration in the small intestine is described by the PDE
%
\begin{align}
	\partial_t c_{si} &= -\partial_z N_p - Q_a, & z &\in [z_0, z_f],
\end{align}
%
where $z$ is the spatial coordinate along the small intestine, and the positions $z_0$ and $z_f$ denote the beginning and end of the small intestine.
%
The flux $N_p$ describes the peristaltic movement in the small intestine, and it consists of an advection term, $N_{ap}$, and a diffusion term, $N_{dp}$:
%
\begin{subequations}
\begin{align}
	N_p     &= N_{ap} + N_{dp}, \\
	N_{ap}  &= v_p c_{si}, \\
	N_{dp}  &= -D_p \partial_z c_{si}.
\end{align}
\end{subequations}
%
The velocity, $v_p$, and the diffusion coefficient, $D_p$, are constant. The glucose absorption, $Q_a$, is given by
%
\begin{subequations}
\begin{align}
	Q_a &= \frac{2f}{r_{si}} q_a, \\
	q_a &= v_a c_{si},
\end{align}
\end{subequations}
%
where $r_{si}$ is the radius of the small intestine, and $f$ is a factor describing 1)~the increase in surface area (compared to that of a cylinder) due to villi, microvilli, and plicae circulares, and 2)~the fact that glucose is only absorbed from a fraction of the surface. Furthermore, $v_a$ is the glucose absorption rate.
%
The flow rate from the stomach to the duodenum is represented as a boundary condition, i.e., the flux at the beginning of the small intestine times the cross-sectional area, $A_{si}$, must equal the glucose flow rate $F_{sd}$:
%
\begin{align}\label{eq:cstr:pfr:bc}
	A_{si} N_p \rvert_{z = z_0} &= F_{sd}.
\end{align}
%
Finally, the glucose rate of appearance is the cross-sectional area times the integral of the glucose absorption rate over the length of the small intestine:
%
\begin{align}
    R_A &= A_{si} \int_{z_0}^{z_f} Q_a \incr z.
\end{align}

\subsubsection{Moxon's feedback mechanism}\label{sec:cstr:pfr:meal:model:moxon}
\citet{Moxon:etal:2017} propose that the glucose flow rate between the stomach and duodenum is equal to 1)~zero if the glucose rate of appearance, $R_A$, is above a certain threshold, $R_{A, max}$, and 2)~$k_{sd}^{max}$ otherwise. This is approximated by
%
\begin{align}\label{eq:moxon:pyloric:sphincter}
    k_{sd} &= k_{sd}^{max} \frac{1}{1 + \exp(\sigma (R_A - R_{A, max}))},
\end{align}
%
where the parameter $\sigma$ determines the accuracy of the approximation, i.e., the steepness of $k_{sd}$ around $R_{A, max}$.

\subsubsection{Alsk{\"{a}}r's feedback mechanism}\label{sec:cstr:pfr:meal:model:alskar}
\citet{Alskar:etal:2016} propose that the glucose flow rate can be described using a Hill expression with a high Hill coefficient, $\gamma$, i.e., it approximates an on/off mechanism where the glucose flow rate is equal to zero if the amount of glucose in the duodenum, $m_d$, is above a threshold value, $m_{d, 50}$, and $k_{sd}^{max}$ otherwise. In addition to the original model of the feedback mechanism, we introduce a minimum value, $k_{sd}^{min}$:
%
\begin{align}\label{eq:alskar:pyloric:sphincter}
	k_{sd} &= k_{sd}^{min} + (k_{sd}^{max} - k_{sd}^{min})\left(1 - \frac{m_d^\gamma}{m_{d, 50}^\gamma + m_d^\gamma}\right).
\end{align}
%
Finally, the duodenum constitutes the first part of the small intestine (from $z_0$ to $z_d$). Consequently, the amount of glucose in the duodenum is given by
%
\begin{align}\label{eq:duodenum:glucose}
	m_d &= A_{si} \int_{z_0}^{z_d} c_{si} \incr z.
\end{align}
%
\begin{remark}
    If $k_{sd}^{min} = 0$ in~\eqref{eq:alskar:pyloric:sphincter}, the glucose flow rate may become close to zero even though the duodenum is almost entirely empty. The reasons are that the velocity of the peristaltic movement, $v_p$, is relatively low and that it is independent of the glucose concentration. Consequently, a very short plug of chyme with a high glucose concentration will move through the duodenum, and once it enters into the jejunum, the duodenum again becomes empty, and the process repeats itself.
\end{remark}