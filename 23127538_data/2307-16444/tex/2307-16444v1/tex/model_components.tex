\section{Model components}\label{sec:models:components}
As the human metabolism is a set of chemical reactions, the gastrointestinal tract can be modeled mathematically using modeling techniques from chemical reaction engineering. In this section, we briefly outline a systematic modeling approach using stoichiometry and reaction kinetics in combination with ideal CSTRs and PFRs. Furthermore, delays play an important role in metabolic modeling~\citep{Voit:2017}, and we describe several mathematical models and approximations of delayed signals.

% CITE THESE INSTEAD OF MIXRAJI:ETAL:1988
% https://www.ncbi.nlm.nih.gov/pmc/articles/PMC5643013/pdf/nihms869909.pdf
% https://www.eolss.net/sample-chapters/c02/E6-188-14.pdf <- THIS DOESN'T SEEM TO BE PROPERLY PUBLISHED: MAYBE IT'S A TECHNICAL REPORT

\subsection{Stoichiometry and reaction kinetics}\label{sec:stoichiometry}
Consider a set of molecules $\mathcal{C}$ which are involved in a set of reactions $\mathcal{R}$ in the human metabolism. Let $S \in \R^{n_r \times n_c}$ be the matrix of stoichiometric coefficients for this set of reactions and molecules. $n_c$ is the number of molecules and $n_r$ is the number of reactions. Let $c$ be the vector of concentrations such that we can express the rate vector, $r$, for this set of reactions as the function 
%
\begin{equation}
    r = r(c).
\end{equation}
%
Consequently, the production rate vector for the molecules can be expressed as
%
\begin{equation}
    R = S' r.
\end{equation}
%
This general way of expressing the production rate, $R$, is useful because it only requires the specification of the chemical reaction stoichiometry (and the corresponding stoichiometric matrix, $S$) as well as the corresponding expression for the reaction rates, $r = r(c)$.
\subsection{CSTR}\label{sec:models:components:cstr}
Any part of the gastrointestinal tract where transport phenomena (i.e., advection and diffusion) are negligible can be represented as a CSTR. The mass balance for a CSTR is
%
\begin{equation}
    V \dot c = (c_{in} - c) F + R V,
\end{equation}
%
where $V$ is volume, $c$ is concentration, $c_{in}$ is the inflow concentration, $F$ is the volumetric in- and outflow rate, and $R$ is the production rate. The volume is assumed to be constant, and the model can be reformulated as an ODE:
%
\begin{equation}
    \dot c = (c_{in} - c) F/V + R.
\end{equation}
\subsection{PFR}\label{sec:models:components:pfr}
The parts of the gastrointestinal tract where advective and diffusive transport phenomena are significant can be described as PFRs. A PFR is cylindrical and the concentration, $c = c(t, z, r, \theta) = c(t, z)$, only changes along the transport direction, $z$, i.e., it is constant along the radial and angular coordinates, $r$ and $\theta$.

The spatiotemporal evolution of the concentration is described by the PDE
%
\begin{equation}\label{eq:pfr}
    \partial_t c = - \partial_z N + R + Q,
\end{equation}
%
where $N$ is flux, $R$ is the production rate, and $Q$ is a source term. The flux is the sum of an advection term, $N_a$, and a diffusion term, $N_d$:
%
\begin{equation}
    N = N_a + N_d.
\end{equation}
%
These terms are
%
\begin{subequations}\label{eq:PFR:Flux:Terms}
\begin{align}
    \label{eq:PFR:Flux:Terms:Advection}
    N_a &= v c, \\
    %
    \label{eq:PFR:Flux:Terms:Diffusion}
    N_d &= -D_c \partial_z c,
\end{align}
\end{subequations}
%
where $v$ is velocity and $D_c$ is the diffusion coefficient. The expression~\eqref{eq:PFR:Flux:Terms:Diffusion} is called Fick's law.

% \begin{equation}
%     q(t) = \int_{0}^{L} Q(t, z) \incr z
% \end{equation}

% % Figure environment removed

\subsection{Delays}
Here, we describe different formulations and approximations of a model, where $y$ is equal to the input signal~$u$ delayed by $\tau_d$, i.e.,
%
\begin{align}
\label{eq:TimeDelayModelYU}
    y(t) &= u(t - \tau_d).
\end{align}
%
The Laplace transform of \eqref{eq:TimeDelayModelYU} is
%
\begin{subequations}\label{eq:Laplace:TimeDelayModelYU}
\begin{align}
    \label{eq:Laplace:TimeDelayModelYU:YGU}
    Y(s) &= G(s) U(s), \\
    %
    \label{eq:Laplace:TimeDelayModelYU:G}
    G(s) &= e^{-\tau_d s}.
\end{align}
\end{subequations}

Alternatively, the system~\eqref{eq:TimeDelayModelYU} can be formulated as a series of $M$ systems with smaller time delays:
%
\begin{align}
    y_i(t) &= y_{i-1}(t - \tau_d/M), & i &= 1, \ldots, M,
\end{align}
%
where
%
\begin{subequations}
\begin{align}
    y_0(t) &= u(t), \\
    y(t) &= y_M(t).
\end{align}
\end{subequations}
%
The Laplace transform of this series of systems are
%
\begin{subequations}
\label{eq:Laplace:TimeDelayModelYU:Multiple}
\begin{align}
    Y_i(s) &= G_i(s) Y_{i-1}(s), & i &= 1, \ldots, M, \\
    G_i(s) &= e^{-(\tau_d/M) s},
\end{align}
\end{subequations}
%
where
%
\begin{subequations}
\begin{align}
    Y_0(s) &= U(s), \\
    Y(s)   &= Y_M(s).
\end{align}
\end{subequations}
%
Approximating $G_i$ in~\eqref{eq:Laplace:TimeDelayModelYU:Multiple} will typically result in a lower error than approximating $G$ in~\eqref{eq:Laplace:TimeDelayModelYU} because the delay is smaller. However, the increased accuracy comes at the expense of higher computational cost.
%
Below, we show different approximations based on a dynamical system in the form
%
\begin{subequations}
\label{eq:linear:state:space:model}
\begin{align}
    \dot   x(t) &= A_c x(t) + B_c u(t), \\
    \tilde y(t) &= C_c x(t) + D_c u(t).
\end{align}
\end{subequations}

\subsubsection{Lag approximation}
The transfer function in~\eqref{eq:Laplace:TimeDelayModelYU:G} can be approximated by the transfer function of a lag process, i.e.,
%
\begin{align}
    G(s) &\approx \frac{1}{\tau_d s + 1} = \frac{1/\tau_d}{s + 1/\tau_d} = \frac{P(s)}{Q(s)} = \tilde G(s).
\end{align}
%
The system matrices in the corresponding linear state space realization, in observable canonical form~\citep[Chap.~3.9]{Hendricks:etal:2008}, are
%
\begin{subequations}
\begin{align}
    A_c &= - 1/\tau_d, & B_c &= 1/\tau_d, \\
    C_c &= 1, & D_c &= 0.
\end{align}
\end{subequations}

We apply the same approximation to the system~\eqref{eq:Laplace:TimeDelayModelYU:Multiple}:
%
\begin{align}
    G_i(s)
    &\approx \frac{1}{(\tau_d/M) s + 1} = \frac{M/\tau_d}{s + M/\tau_d} = \frac{P_i(s)}{Q_i(s)} \nonumber \\
    &= \tilde G_i(s).
\end{align}
%
Again, we consider the corresponding state space realization in observable canonical form. In this case, the system matrices are
%
\begin{subequations}
\label{eq:lag:multiple:system:matrices}
\begin{align}
    A_{c, ij} &=
    \begin{cases}
        -M/\tau_d, & i = j, \\
         M/\tau_d, & i = j-1, \\
         0, & \text{otherwise},
    \end{cases} \\
    %
    B_{c, i} &=
    \begin{cases}
        M/\tau_d, & i = 1, \\
        0, & \text{otherwise},
    \end{cases} \\
    %
    C_{c, i} &=
    \begin{cases}
        1, & i = M, \\
        0, & \text{otherwise},
    \end{cases} \\
    %
    D_c &= 0,
\end{align}
\end{subequations}
%
for $i = 1, \ldots, M$ and $j = 1, \ldots, M$.
\subsubsection{Pad{\'{e}} approximation}
The Pad{\'{e}} approximation~\citep{Wei:Hu:Dai:Wang:2016} is another classical way used to approximate time delays. The first-order Pad{\'{e}} approximation of $G(s) = e^{-\tau_d s}$ is
%
\begin{align}
\label{eq:Laplace:FirstOrderPadeApproximation}
    G(s) &\approx \frac{-(\tau_d/2)s +1}{(\tau_d/2) s + 1} = \frac{- s + 2/\tau_d}{s+2/\tau_d} = \frac{P(s)}{Q(s)} = \tilde{G}(s).
\end{align}
%
The first-order Pad{\'{e}} approximation~\eqref{eq:Laplace:FirstOrderPadeApproximation} may be used to approximately realize~\eqref{eq:Laplace:TimeDelayModelYU} as the linear state space model~\eqref{eq:linear:state:space:model}, in observable canonical form, with the system matrices
%
\begin{subequations}
\begin{align}
    A_c &= -2/\tau_d, & B_c &= 4/\tau_d, \\
    C_c &= 1, & D_c &= -1.
\end{align}
\end{subequations}

The Pad{\'{e}} approximation of $G_i$ in~\eqref{eq:Laplace:TimeDelayModelYU:Multiple} is
% The approximation is more accurate for small time delays than for large time delays. In case of larger time delays, $\tau_d$, it may approximated by $M$ first order Pad{\'{e}} approximations in series, each with the time constant $\tau_d/M$:
%
\begin{align}
    G_i(s)
    &\approx \frac{-(\tau_d/(2M))s +1}{(\tau_d/(2M)) s + 1} = \frac{-s + 2M/\tau_d}{s + 2M/\tau_d} \nonumber \\
    &= \frac{P_i(s)}{Q_i(s)} = \tilde{G}_i(s),
\end{align}
%
and the system matrices in the corresponding state space realization (in observer canonical form) are
%
\begin{subequations}
\begin{align}
    A_{c, ij} &=
    \begin{cases}
        -2M/\tau_d, & i = j, \\
         (-1)^{i+j+1}4M/\tau_d, & i > j, \\
         0, & \text{otherwise},
    \end{cases} \\
    %
    B_{c, i} &= (-1)^{i+1} 4M/\tau_d, \\
    %
    C_{c, i} &= (-1)^{M+i}, \\
    %
    D_c &= (-1)^M,
\end{align}
\end{subequations}
%
for $i = 1, \ldots, M$ and $j = 1, \ldots, M$.
\subsubsection{Physical transport delay model}
Delays can also be represented using transport processes. The input signal, $u$, constitutes the boundary condition,
%
\begin{align}
    c_{in}(t) = u(t),
\end{align}
%
and the initial boundary value problem
%
\begin{subequations}
\label{eq:PDE:timedelay}
\begin{align}
    c(t,0) &= c_{in}(t), \\
    \partial_t c &= - v \partial_z c, & t &\geq 0, & 0 &\leq z \leq L,
\end{align}
\end{subequations}
%
has the analytical solution $y(t) = c(t, L) = c_{in}(t - \tau_d) = u(t - \tau_d)$ with the delay $\tau_d = L/v$.
% Consequently, \eqref{eq:PDE:timedelay} may be used to represent time delays.

\begin{remark}
    A left-sided first-order finite difference discretization of the PDE~\eqref{eq:PDE:timedelay}, based on an equidistant grid with $M+1$ nodes, is equivalent to the linear state space model~\eqref{eq:linear:state:space:model} with the system matrices~\eqref{eq:lag:multiple:system:matrices} obtained using a series of $M$ lag approximations.
\end{remark}
\subsubsection{Algebraic delay approximation}\label{sec:models:components:delay:algebraic}
For completeness, we also describe an algebraic delay approximation which is used in the literature, e.g., by~\citet{Alskar:etal:2016}. However, unlike the previous approximations, it is an algebraic expression rather than a linear state space model in the form~\eqref{eq:linear:state:space:model}. Furthermore, it specifically approximates a step in the input function, $u$, whereas the other approximations can be used for arbitrary input functions.

Let $t_s$ denote the time at which the step in $u$ occurs, i.e., $u(t) = 1$ for $t \geq t_s$ and $u(t) = 0$ otherwise. Then, the approximation is
% If the step in $u$ occurs at time $t_s$, the approximation is
%
\begin{align}
    y(t) &\approx \frac{1}{1 + \exp(-\sigma(t - t_{50}))} = \tilde y(t),
\end{align}
%
where $t_{50} = t_s + \tau_d$ is the time at which $\tilde y$ is halfway between the value of $u$ before and after the step.

% Figure environment removed