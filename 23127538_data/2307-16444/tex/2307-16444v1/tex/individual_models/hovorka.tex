\subsection{Hovorka's model}
The model by~\citet{Hovorka:etal:2004} contains two compartments, as illustrated in Fig.~\ref{fig:hovorka:diagram}.
%
% Figure environment removed
%
The first compartment, $D_1$, describes the amount of glucose in the stomach, and the second compartment, $D_2$, describes the amount of glucose in the small intestine:
%
\begin{subequations}
\begin{alignat}{3}
    \dot D_1 &= A_G d - R_{12}, \\
    \dot D_2 &= R_{12} - R_2.
\end{alignat}
\end{subequations}
%
Here, $A_G$ describes the bioavailability of the carbohydrates in the meal, and $R_{12} = R_{12}(D_1)$ is the glucose flow rate between the stomach and the small intestine. Furthermore, $R_2 = R_2(D_2)$ describes the glucose absorption, and the glucose rate of appearance, $R_A = R_A(D_2)$, is a fraction, $f$, of $R_2$:
%
\begin{subequations}
\begin{align}
    R_{12} &= D_1/\tau_D, \\
    R_2 &= D_2/\tau_D, \\
    R_A &= f R_2.
\end{align}
\end{subequations}
%
The parameter $\tau_D$ is a time constant, and the model is a linear state space model in the form~\eqref{eq:linear:state:space:meal:model}, where the system matrices are
%
\begin{align}
    A_c &= \begin{bmatrix} \frac{-1}{\tau_D} & 0 \\ \frac{1}{\tau_D} & \frac{-1}{\tau_D} \end{bmatrix}, &
    B_c &= \begin{bmatrix} A_G \\ 0 \end{bmatrix}, &
    C_c &= \begin{bmatrix} 0 & \frac{f}{\tau_D}\end{bmatrix}.
\end{align}