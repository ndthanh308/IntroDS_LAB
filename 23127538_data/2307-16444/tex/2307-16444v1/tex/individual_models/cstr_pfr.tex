\subsection{CSTR-PFR model}\label{sec:cstr:pfr:meal:model}
The CSTR-PFR model presented here consists of a CSTR representing the stomach and a PFR representing the small intestine, as shown in Fig.~\ref{fig:cstr:pfr:diagram} (see also Section~\ref{sec:models:components:cstr} and~\ref{sec:models:components:pfr}). It is based on the second model presented by~\citet{Moxon:etal:2016}, and we describe three ways of modeling the opening and closing of the pylorus sphincter, which connects the stomach to the small intestine.
%
% Figure environment removed

The amount of glucose in the stomach, $m_s$, is given by
%
\begin{subequations}\label{eq:cstr:pfr:stomach}
\begin{align}
    \label{eq:cstr:pfr:stomach:ms}
	\dot m_s &= F_m - F_{sd}, \\
	%
	\label{eq:cstr:pfr:stomach:fsd}
	F_{sd} &= k_{sd} m_s,
\end{align}
\end{subequations}
%
where $F_m = d$ is the meal input, $F_{sd}$ is the glucose flow rate from the stomach to the duodenum, and $k_{sd}$ is the inverse of a time constant. We either consider $k_{sd}$ to be 1)~constant (the pylorus sphincter is always completely open), 2)~a function of the glucose rate of appearance in the blood (see Section~\ref{sec:cstr:pfr:meal:model:moxon}), or 3)~a function of the amount of glucose in the duodenum (see Section~\ref{sec:cstr:pfr:meal:model:alskar}).
%
The glucose concentration in the small intestine is described by the PDE
%
\begin{align}
	\partial_t c_{si} &= -\partial_z N_p - Q_a, & z &\in [z_0, z_f],
\end{align}
%
where $z$ is the spatial coordinate along the small intestine, and the positions $z_0$ and $z_f$ denote the beginning and end of the small intestine.
%
The flux $N_p$ describes the peristaltic movement in the small intestine, and it consists of an advection term, $N_{ap}$, and a diffusion term, $N_{dp}$:
%
\begin{subequations}
\begin{align}
	N_p     &= N_{ap} + N_{dp}, \\
	N_{ap}  &= v_p c_{si}, \\
	N_{dp}  &= -D_p \partial_z c_{si}.
\end{align}
\end{subequations}
%
The velocity, $v_p$, and the diffusion coefficient, $D_p$, are constant. The glucose absorption, $Q_a$, is given by
%
\begin{subequations}
\begin{align}
	Q_a &= \frac{2f}{r_{si}} q_a, \\
	q_a &= v_a c_{si},
\end{align}
\end{subequations}
%
where $r_{si}$ is the radius of the small intestine, and $f$ is a factor describing 1)~the increase in surface area (compared to that of a cylinder) due to villi, microvilli, and plicae circulares, and 2)~the fact that glucose is only absorbed from a fraction of the surface. Furthermore, $v_a$ is the glucose absorption rate.
%
The flow rate from the stomach to the duodenum is represented as a boundary condition, i.e., the flux at the beginning of the small intestine times the cross-sectional area, $A_{si}$, must equal the glucose flow rate $F_{sd}$:
%
\begin{align}\label{eq:cstr:pfr:bc}
	A_{si} N_p \rvert_{z = z_0} &= F_{sd}.
\end{align}
%
Finally, the glucose rate of appearance is the cross-sectional area times the integral of the glucose absorption rate over the length of the small intestine:
%
\begin{align}
    R_A &= A_{si} \int_{z_0}^{z_f} Q_a \incr z.
\end{align}

\subsubsection{Moxon's feedback mechanism}\label{sec:cstr:pfr:meal:model:moxon}
\citet{Moxon:etal:2017} propose that the glucose flow rate between the stomach and duodenum is equal to 1)~zero if the glucose rate of appearance, $R_A$, is above a certain threshold, $R_{A, max}$, and 2)~$k_{sd}^{max}$ otherwise. This is approximated by
%
\begin{align}\label{eq:moxon:pyloric:sphincter}
    k_{sd} &= k_{sd}^{max} \frac{1}{1 + \exp(\sigma (R_A - R_{A, max}))},
\end{align}
%
where the parameter $\sigma$ determines the accuracy of the approximation, i.e., the steepness of $k_{sd}$ around $R_{A, max}$.

\subsubsection{Alsk{\"{a}}r's feedback mechanism}\label{sec:cstr:pfr:meal:model:alskar}
\citet{Alskar:etal:2016} propose that the glucose flow rate can be described using a Hill expression with a high Hill coefficient, $\gamma$, i.e., it approximates an on/off mechanism where the glucose flow rate is equal to zero if the amount of glucose in the duodenum, $m_d$, is above a threshold value, $m_{d, 50}$, and $k_{sd}^{max}$ otherwise. In addition to the original model of the feedback mechanism, we introduce a minimum value, $k_{sd}^{min}$:
%
\begin{align}\label{eq:alskar:pyloric:sphincter}
	k_{sd} &= k_{sd}^{min} + (k_{sd}^{max} - k_{sd}^{min})\left(1 - \frac{m_d^\gamma}{m_{d, 50}^\gamma + m_d^\gamma}\right).
\end{align}
%
Finally, the duodenum constitutes the first part of the small intestine (from $z_0$ to $z_d$). Consequently, the amount of glucose in the duodenum is given by
%
\begin{align}\label{eq:duodenum:glucose}
	m_d &= A_{si} \int_{z_0}^{z_d} c_{si} \incr z.
\end{align}
%
\begin{remark}
    If $k_{sd}^{min} = 0$ in~\eqref{eq:alskar:pyloric:sphincter}, the glucose flow rate may become close to zero even though the duodenum is almost entirely empty. The reasons are that the velocity of the peristaltic movement, $v_p$, is relatively low and that it is independent of the glucose concentration. Consequently, a very short plug of chyme with a high glucose concentration will move through the duodenum, and once it enters into the jejunum, the duodenum again becomes empty, and the process repeats itself.
\end{remark}