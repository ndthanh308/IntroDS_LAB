\subsection{Alsk{\"{a}}r's model}\label{sec:meal:models:alskar}
The model by~\citet{Alskar:etal:2016} contains four compartments representing the amounts of glucose in the stomach, $G_S$, the duodenum, $G_D$, the jejunum, $G_J$, and the ileum, $G_I$, as illustrated in Fig.~\ref{fig:alskar:diagram}. They are described by
%
% Figure environment removed
%
\begin{subequations}
\begin{align}
    \dot G_S &= d - R_{SD}, \\
    \dot G_D &= R_{SD} - R_{DJ} - R_{A, D}, \\
    \dot G_J &= R_{DJ} - R_{JI} - R_{A, J}, \\
    \dot G_I &= R_{JI} - R_{A, I},
\end{align}
\end{subequations}
%
where the glucose flow rates between the stomach and duodenum, $R_{SD} = R_{SD}(G_S)$, between the duodenum and jejunum, $R_{DJ} = R_{DJ}(G_D)$, and between the jejunum and ileum, $R_{JI} = R_{JI}(G_J)$, are
%
\begin{subequations}
\begin{align}
    R_{SD} &= k_{SD} \tau G_S, \\
    R_{DJ} &= k_{DJ} G_D, \\
    R_{JI} &= k_{JI} G_J.
\end{align}
\end{subequations}
%
The inverses of the time constants, $k_{SD} = k_{SD}(G_D)$, $k_{DJ}$, and $k_{JI}$, are
%
\begin{subequations}\label{eq:alskar:k}
\begin{align}
    \label{eq:alskar:k:sd}
    k_{SD} &= k_w\left(1 - \frac{G_D^\gamma}{IG_{D50}^\gamma + G_D^\gamma}\right), \\
    %
    \label{eq:alskar:k:df}
    k_{DJ} &= \frac{1}{L_D T}, \\
    %
    \label{eq:alskar:k:ji}
    k_{JI} &= \frac{1}{L_J T}.
\end{align}
\end{subequations}
%
Here, $k_{SD}$ represents the pylorus sphincter, and it is a function of the amount of glucose in the duodenum described using the Hill expression. For $G_D = 0$, $k_{SD}$ is equal to its nominal value, $k_w$, and, as $G_D$ increases, $k_{SD}$ approaches zero. For large values of the Hill coefficient, $\gamma$, $k_{SD}$ has a steep decrease around $IG_{D50}$. Furthermore, $L_D$ and $L_J$ are the relative lengths of the duodenum and jejunum (i.e., fractions of the total length of the small intestine), and $T$ is the transit time through the small intestine. The lag coefficient (used to approximate a time delay) is given by
%
\begin{align}
    \tau &= \frac{1}{1 + \exp(-\sigma(t - t_{50}))},
\end{align}
%
as described in Section~\ref{sec:models:components:delay:algebraic}. The parameter $\sigma$ determines the steepness, and $t_{50}$ is the time at which $\tau$ is 0.5. The glucose absorption rates in the duodenum, $R_{A, D} = R_{A, D}(G_D)$, jejunum, $R_{A, J} = R_{A, J}(G_J)$, and ileum, $R_{A, I} = R_{A, I}(G_I)$, are described using Michaelis-Menten expressions, i.e.,
%
\begin{subequations}
\begin{align}
    R_{A, D} &= \frac{R_{D, \max} G_D}{K_{mG} + G_D}, \\
    R_{A, J} &= \frac{R_{J, \max} G_J}{K_{mG} + G_J}, \\
    R_{A, I} &= \frac{R_{I, \max} G_I}{K_{mG} + G_I},
\end{align}
\end{subequations}
%
where $K_{mG}$ is the Michaelis constant and $R_{D, \max}$, $R_{J, \max}$, and $R_{I, \max}$ are the maximum glucose absorption rates in the duodenum, jejunum, and ileum, respectively. Finally, the glucose rate of appearance in the blood plasma, $R_A = R_A(G_D, G_J, G_I)$, is a fraction, $F_P$, of the total glucose absorption:
%
\begin{align}
    R_A &= F_P (R_{A, D} + R_{A, J} + R_{A, I}).
\end{align}