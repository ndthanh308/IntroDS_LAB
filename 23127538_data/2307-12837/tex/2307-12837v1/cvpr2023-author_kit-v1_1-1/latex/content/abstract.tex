\begin{abstract}
This report presents the technical details of our approach for the EPIC-Kitchens-100 Unsupervised Domain Adaptation (UDA) Challenge in Action Recognition.
Our approach is based on the idea that the order in which actions are performed is similar between the source and target domains. Based on this, we generate a modified sequence by randomly combining actions from the source and target domains. As only unlabelled target data are available under the UDA setting, we use a standard pseudo-labeling strategy for extracting action labels for the target. We then ask the network to predict the resulting action sequence. This allows to integrate information from both domains during training and to achieve better transfer results on target. 
Additionally, to better incorporate sequence information, we use a language model to filter unlikely sequences.
Lastly, we employed a co-occurrence matrix to eliminate unseen combinations of verbs and nouns.
Our submission, labeled as `sshayan', can be found on the leaderboard, where it currently holds the 2nd position for `verb' and the 4th position for both `noun' and 'action'.
\end{abstract}