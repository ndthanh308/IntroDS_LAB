%% ****** Start of file template.aps ****** %
%%
%%
%%   This file is part of the APS files in the REVTeX 4 distribution.
%%   Version 4.0 of REVTeX, August 2001
%%
%%
%%   Copyright (c) 2001 The American Physical Society.
%%
%%   See the REVTeX 4 README file for restrictions and more information.
%%
%
% For Phys. Rev. appearance, change preprint to twocolumn.
% Choose pra, prb, prc, prd, pre, prl, prstab, or rmp for journal
%  Add 'draft' option to mark overfull boxes with black boxes
%
%
% To compile: latex prl.tex ; dvips prl.dvi -o prl.ps ; ps2pdf prl.ps
%
\documentclass[aps,prl,twocolumn,showpacs,superscriptaddress,groupedaddress]{revtex4}
\usepackage{graphicx}  % needed for figures
\usepackage{dcolumn}   % needed for some tables
\usepackage{bm}        % for math
\usepackage{amssymb}   % for math

\usepackage{xcolor}
\usepackage{hyperref}
\usepackage{ulem} % pour barrer du texte avec \sout{texte}


\newcommand{\Jishen}[1]{\textcolor{blue}{#1}}
\newcommand{\steph}[1]{\textcolor{red}{#1}}
\newcommand{\Benoit}[1]{\textcolor{cyan}{#1}}

\begin{document}

\title{Large-Scale Turbulent Pressure Fluctuations Revealed by Ned Kahn's Artwork}

\author{J.~Zhang}
\author{S.~Perrard}
\affiliation{PMMH, CNRS, ESPCI Paris, Universit\'e PSL, Sorbonne Universit\'e, Universit\'e de Paris, F-75005, Paris, France}
\date{\today}

\begin{abstract}
We investigate the dynamics of pendulum chains immersed in turbulent boundary layers. We combine laboratory experiments and video analysis of the kinetic facade exhibits by the artist Ned Kahn, composed of large-scale clusters of centimeter-sized plates oscillating freely in the wind. At the laboratory scale, we show that a one-dimensional pendulum chain immersed in a wind tunnel exhibits a wave dispersion relation derived from a Sine-Gordon equation. Under the wind action, the dynamical response is either dominated by a resonance phenomenon, or a linear response to pressure fluctuations. From amateur video analysis on large-scale kinetic facades, we show that the plate oscillation is driven by the same resonant response mechanisms and the apparent wavy pattern corresponds to the most energetic Fourier mode propagating at the advection speed of pressure fluctuations.
%Indeed, the Artist Ned Kahn designed building facades, composed of decimetric scale oscillating plates, that exhibit complex behaviour when the wind blows.
%a 1D chain of coupled pendulums in a laboratory-scale turbulent flow. The system is composed of rectangular 3D printed thin plates, aligned with the air flow direction and elastically coupled with nylon fishing wires. In the case of external force being purely mechanical, we have successfully predicted the dispersion relation derived from a 1D discrete Sine-Gordon equation. As the wind blows over the chain, modifications have been observed as regards to the morphology in the spatiotemporal Fourier space: energy input from the wind is distributed around two branches, one branch refers to the dispersion relation with wind coupling induced shift, the other bears upon the large scale structures induced imprints of pressure fluctuations. By analogy, we extend the terminology to dynamics of 2D free pendulum facades in open field conditions using video analysis. Similar behaviors have been depicted. We quantitatively deduced the convection velocity of the large scale structures being identical to the phase velocity of the most excited wave trains. The current investigation suggests that such system might give alternative approach to large-scale turbulence measurements which remain challenging today.

%We investigate the dynamics of a 1D chain of coupled pendulums in a laboratory-scale turbulent flow. The system is composed of rectangular 3D printed thin plates, aligned with the air flow direction and elastically coupled with nylon fishing wires. We highlight a resonance mechanism between the pendulum oscillation and pressure fluctuations around the dispersion relation in the Fourier space. Our observations reveal equally some modifications of the dispersion relation as a result of additional aerodynamic coupling. The current investigation suggests that such system might give alternative approach to large-scale turbulence measurement which remains challenging today.
\end{abstract}

\maketitle

\textit{Introduction.}--- Ned Kahn is an American artist who constructs numerous exhibits inspired by the ephemera of nature. Amongst his works is the kinetic facade, a regular assembly of small aluminum plates covering entire facades of buildings in various countries (US, Scotland, Netherland, Switzerland). As the wind blows along the wall, the plates oscillate freely, creating wave-like large-scale patterns (Fig.~\ref{fig:schema}a). The spatiotemporal structures could be revealed by the kinetic facades are the grail of any fluid mechanician, for whom measuring spatiotemporal structures of boundary layers in shear flows has been a long-standing challenge, and has mainly been investigated in numerical simulations of channel flows (\cite{chung2010large, jimenez2018coherent,towne2020resolvent}) of increasing dimensions, from idealized to specific configurations for application to wind farms~(\cite{churchfield2012numerical}). Spatiotemporal fluctuations of velocity at large scale~\cite{He_2017}  are also key elements in wind farm design in order to minimize the global power fluctuations of a plant affected by long range fluctuations, and have been characterized by two point measurements in the far wake region~\cite{sorensen2002wind, wetz2023analyses, vermeer2003wind}. A direct visualisation of spatiotemporal pressure fluctuations in turbulent boundary layers is of great interest, and has been investigated experimentally ocassionally in laboratory experiments since the pioneering work of Dinkelacker~\cite{Dinkelacker_1977} who measured the spatiotemporal structures of pressure fluctuations using a Mylar$\textsuperscript{\textregistered}$ plate and interference fringes. It has recently been shown that indirect insight can be obtained from the pattern imprints on a liquid surface below the onset of wind-wave generation on an air-water interface; these imprints are also the signature of turbulent pressure fluctuations~\cite{perrard2019turbulent}. However, to the best of our knowledge, the origin of the kinetic-facade motion has not been investigated. It could be the result of numerous phenomena, such as fluid-structure instabilities observed in flapping wings~\cite{shelley2011flapping} or canopies~\cite{de2008effects}, or bistability of pendulums in turbulent flows~\cite{Gayout_2021}.
%willmarth1962measurements
%{\bf Question} : what properties of the turbulent air boundary layer can be extracted from these visuals ? 

%\Jishen{ contexte industriel à mettre quelque part ? measurements of the tip-vortices convection velocity in far-wake region in wind farms }

% Figure environment removed


\textit{Experimental set-up.}--- To investigate the origin of these dynamical patterns, we designed a one-dimensional experimental model at the laboratory scale, composed of a chain of elastically coupled pendulums, as sketched in fig.~\ref{fig:schema}b. The chain consists of $36$ 3D-printed plates of height $l=48.5$~mm, plate distance $L=32.5$~mm, and width $28.5$~mm, for a total chain length of $1.1$~m . The plates are connected by a nylon fishing wire to control the coupling constant between adjacent elements. The \textit{i}-th pendulum of inclination angle $\theta_i$ then interacts with its neighbors ($i-1$) and ($i+1$) with a coupling force linearly proportional to the angle difference. In the absence of wind, the motion of each pendulum follows the one-dimensional discrete Sine-Gordon equation:
\begin{equation}
  \partial_{tt}\theta_i - L^2\omega_w^2 \Delta \theta_i + \lambda \partial_t \theta_i + \omega_0^2\sin\theta_i= 0%\Jishen{A}_i(t),
  \label{eq:force_balance}
\end{equation}
where $\Delta \theta_i = (\theta_{i+1}-2\theta_i+\theta_{i-1})/L^2$ is the discrete second spatial derivative of $\theta$ on site $i$, $\lambda$ is the damping coefficient, and $\omega_0 = \sqrt{g/\ell}$ is the natural angular frequency with $g$ the acceleration of gravity and $\ell=2/3l$ is the effective pendulum length of a thin plate. The velocity $L\omega_w$ originates from the pendulum coupling through the wire elastic torsion, with $\omega_w$ the wire elastic torsion frequency. From elastica, we have $\omega_w=\sqrt{GJ_w/(LI_p)}$ with $G$ the wire shear modulus, $J_w=\pi a^4/32$ the polar second moment of area of the wire and $I_p = ml^2/3$ the moment of inertia of the pendulum with respect to $x$ axis. In the presence of an air flow, a torque-induced angular acceleration $A_i=T_i/I_p$ applied to each pendulum appears as a source term on the right-hand-side of eq.~\ref{eq:force_balance}.
%The air flow may also modify the constants $\lambda$, $\omega_0$, and $2L\omega_w$ respectively through additional flow dissipation, added-mass effect, and flow-induced coupling.

The chain is placed at the symmetry plane of an open circuit suction wind tunnel with a free-stream wind speed $U$ ranging from 0 to $8$ m/s. Fishing wire of two different diameters $a=0.2$ and $0.8$ mm was used to vary the coupling between plates. The pendulum dynamics are recorded from below, and a white spot on each pendulum is used to track the instantaneous oscillation angles. For each wind speed, $90000$ images are recorded at a frame rate of $300.03$ Hz. From the sets of images, a spot-recognition code with sub-pixel accuracy is used to extract the pendulum angles $\theta_i(t)$. Fig.~\ref{fig:schema}c,d show respectively a snapshot of the pendulum chain, and the spatiotemporal diagram of the inclination angles $\theta_i(t)$. 

% Figure environment removed

To validate the Sine-Gordon model, we first perform a set of experiments without wind. The first pendulum element is mechanically excited with a motorized hammer with a beat frequency ranging from $1.3-6.5$ Hz. We record all $\theta_i$ as a function of time, and filtered the signals at frequencies ranging from $f_0$ to 9 Hz to extract the complex mode $\tilde \theta(x,f)$. We extract the wavenumber $k$ and the spatial damping factor $\Lambda$ using a fit of the form $\theta_0 \cos(kx+\phi) \exp(-\Lambda x)$. Figure~\ref{fig:RD_kappa}a shows the dispersion relation at rest, for 36 plates (blue) and 18 plates (red) separated by $2L$. In both cases, we recover the expected minimum cut-off frequency $f_{min}=f_0=2.77$~Hz, and the maximum cut-off frequency $f_{max}$ near $kL = \pi$ reduces from $9.5$~Hz to $7$~Hz as the spacing is doubled. The dispersion relation 
\begin{equation}
  \omega_r^2 = \omega_0^2 + 4\omega_w(a)^2\sin^2 \left (\frac{kL}{2}\right ),
  \label{eq:dispersionOld}
\end{equation}
of eq.~\ref{eq:force_balance} is superimposed (black dashed line) without any fitting parameter, and shows excellent quantitative agreement with the experimental data. Using the group velocity $c_g = \rm{d}\omega/\rm{d}k$, we converted the spatial damping coefficient $\Lambda$ into a temporal damping coefficient $\lambda=\Lambda c_g$, shown in figure~\ref{fig:RD_kappa}b as a function of the wavenumber $k$. The data for the standard chain (blue) and the spaced out chain (red) collapse onto a single master curve, showing that the coupling between adjacent pendulum plays a negligible role in the dissipation of pendulum motions. The upper frequency cutoff is also influenced by the wire diameter. For a weaker elastic coupling ($a=0.2$~mm, not shown), the maximum cutoff reduces drastically from $9.5$~Hz to $2.83$~Hz: the dispersion relation becomes almost independent of the wave number.
%between adjacent plates provides us with a tunable parameter for the wave dispersion relation. In particular, the wire diameter 

%A temporal Fast-Fourier Transform is performed on the $3rd$ pendulum to obtain the full spectrum of the oscillating angles and a signal demodulation ($\tilde{\theta}=\Re(\Phi e^{-i\Psi})$ where $\Phi_{x, f_r}= \langle \theta_{x,t} e^{i2\pi t f_r} \rangle_t$ and $\Psi = \tan^{-1}(\Im(\Phi_x)/\Re(\Phi_x))$) is proceeded on resonant frequencies including fundamental ones and sub-harmonics. The spatial dissipation rate $\kappa$ is obtained by using the exponential decay fit on $|\Phi|=ce^{-\kappa x}$ and a damped-monochromatic wave model is used to determine the associated wavenumber $k$. Fig.~\ref{fig:RD_kappa}a shows the dispersion relation for both pendulum spacing. An excellent agreement is achieved with theoretical predictions~(Eq.~\ref{eq:dispersion})
%\Jishen{nombre N ? Zero padding params ?}
%\begin{equation}
%  c^2\partial_{xx}\theta - \partial_{tt}\theta - \omega_0^2\sin\theta=\lambda \partial_t \theta - F \Jishen{- F_C\partial_{xx}\theta}
%  \label{eq:force_balance}
%\end{equation}

We now blow wind over the pendulum chain. As above, we measure each $\theta_i$ as a function of time, and we compute the Fourier transform $\hat{\theta}_{k,f}$ of $\theta$ both in $x$ and time. The colormaps of $\hat{\theta}_{k,f}$ at the wind speed of $U=2.6$ m/s are shown in figure~\ref{fig:RD2D} for a wire diameter $a=0.8$~mm (a) and  $a=0.2$~m (b). In both cases, the prominent feature is the strong energy localization along two main branches. A first branch is found surrounding $f = kU_c/2\pi$, where $U_c$ is a characteristic velocity extracted from a linear fit of amplitude maxima (solid red line). A second branch surrounding the red dashed line is the wave dispersion relation slightly upshifted from the rest case. We then introduce an additional term in the dispersion relation (eq.~\ref{eq:dispersion}):
\begin{equation}
  \omega_r^2 = \omega_0^2 + 4\left (\omega_w^2 + \frac{u_s^2}{L^2} \right)\sin^2 \left (\frac{kL}{2}\right ),
  \label{eq:dispersion}
\end{equation}
where the factor $4u_s^2/L^2$ is associated to the wind-induced modification of the dispersion relation, assuming a linear dependence of this extra term on the pendulums' local curvature, we have $u_s^2 \sim \Delta \theta$. This phenomenological wind-induced coupling velocity $u_s$ is shown as a function of wind speed (inset figure~\ref{fig:RD2D}b). The coupling velocity $u_s$ is a small fraction of $U_c$. With the new dispersion relation, we find an excellent agreement between eq.~\ref{eq:dispersion} and the experimental data, with $\omega_0$ and $\omega_w$ set fixed to their values at rest. %going from $0.03$~m/s to $0.20$~m/s as the convection velocity moves from $0.54$~m/s to $6.38$~m/s. of the order of the friction velocity at the plate surfaces

% Figure environment removed

\textit{Resonant response in Fourier space.}--- The accumulation of energy surrounding two branches, one being the wave dispersion relation, and one being the maximum energy of the forcing recalls the spatiotemporal resonance mechanism introduced by Phillips~\cite{phillips1957generation} and others~\cite{perrard2019turbulent,nove2020effect} in the context of wind-generated water waves by pressure fluctuations. Assuming a statistically stationary state, eq.~\ref{eq:force_balance} can be transformed into Fourier space to express the Fourier transform $\theta(k,\omega)$ as:
\begin{equation}
|\hat{\theta}|^2=\frac{|\hat{\mathcal{A}}|^2}{(\omega^2-\omega_r^2(k))^2+\lambda^2(k) \omega^2},
\label{eq:fourier}
\end{equation}
where the denominator is a convolution kernel, and $\hat{\mathcal{A}}$ is the Fourier transform of the torques-induced acceleration applied to each pendulum by the turbulent fluctuations. The maximum response in Fourier space is therefore either located where the forcing $\hat{\mathcal{A}}$ is maximum (branch I) or along the dispersion relation ($\omega \sim \omega_r$) (branch II). At a given wavenumber and an angular frequency $\omega \sim \omega_r(k)$, the pendulum chain resonates with the spectral modes of $\hat{\mathcal{A}}$, the growth being limited by dissipative effects. To test the validity of eq.~\ref{eq:fourier}, we compute the energy spectrum $|\hat{\theta}|^2$ as a function of $f$, as shown for $kL=0.72$ in fig.~\ref{fig:RD2D} (red line) for a wire diameter $a=0.8$~mm (c) and $a=0.2$~mm respectively (d). We fit the source term $\hat{\mathcal{A}}$ by $\hat{\mathcal{A}}=\hat{\mathcal{A}}_r+\gamma (\omega-\omega_r)$ with two adjustable parameters $\hat{\mathcal{A}}_r$ and $\gamma$, corresponding to a first-order Taylor expansion of $\hat{\mathcal{A}}$ in the vicinity of the resonant frequency $\omega_r$. We find a quantitative agreement between eq.~\ref{eq:fourier} (blue dashed line) and the experimental measurements.

%\steph{La suite reste à reprendre : intégrer déjà les figures en inversant 4 et 5. On parle d'abord des TFs 2d, on montre que c'est la même physique, et on embraye sur l'extraction de valeur à partir d'une mesure en temps ($f_{M}$) ou d'une mesure en espace ($k_M$).}
The branch (I) is located away from the resonant curve and corresponds to a direct response to wind-induces acceleration, and the slope $2\pi f_I/k_I$ is  the advection speed at the scale $k_I$~\cite{wills1964convection,del2009estimation, moin2009revisiting}. We determine the convection velocity by measuring the wavenumber giving the maximum amplitude at each frequency~(\cite{hussain1981measurements,goldschmidt1981turbulent}), away from the resonant branch ($f_c<f_0$). These structures correspond to the signature of pressure fluctuations integrated over each plate, advected at a constant speed independent of their size (see red solid lines in Fig.~\ref{fig:RD2D}a,b).

%\textit{Convection velocity of vortex structures.}--- Turbulent wind field can be considered as a superposition of the harmonic travelling vortices at a convective speed $U_c=\omega_c/k_c$~(\cite{wills1964convection}). Though the very terminology for convection velocity of vortices being still under debate~(\cite{del2009estimation, moin2009revisiting}), 

%The calculation is carried out at a zone away from the resonant one ($f_c<f_0$) which corresponds physically to a zone where large eddies are advected at a constant speed thus the convection velocity should be independent of both $f_c$ and $k_c$, this refers to Taylor's frozen-flow hypothesis and is confirmed by our observations (see red solid lines in Fig.~\ref{fig:RD2D}a,b). 

%\steph{It is worth noticing that the spatiotemporal structure of this region spans over a relatively large frequency range including the ones outside the frequency cutoffs, this reveals the non-propagative behavior that is imminent to the wall turbulence.} 
%\Jishen{a deplacer ou supprimer: No systematic variation in $U_c$ emerges as the wire coupling constant varies,} \Jishen{which means the retro-action of the plate dynamics on the vortex structure is of no significance.} and the ratio of $U_c$ to the free-stream velocity $U_a$ reads $0.36$ for the smallest wind speed (i.e: $1$~m/s) and reaches almost $1$ at $2.5$~m/s. 

The pendulum chain response is then the superposition of the large-scale imprints of turbulent fluctuations along $\omega = U_c k$, and the resonant response near $\omega = \omega_r(k)$. For vanishing mechanical coupling, the angular frequency $\omega_r$ becomes almost independent of $k$. The advection speed of the large-scale structures can thus be approximated by $2\pi f_0/k_{max}$, where $k_{max}$ is the wavenumber of the most energetic Fourier mode, located at the intersection of the two branches. Using our experiment of observations, we have developed a theoretical framework that we will now use to analyze the recordings of the kinetic facades.

% Figure environment removed

\textit{Survey data collection.}--- Let now examine the 2D dynamics of Ned Kahn's large-scale kinetic facades using amateurs' online videos. We gathered data on YouTube.com and Vimeo.com, with frame rate ranging from $25$ to $29$ Hz, and we investigate on the exhibits' dimensions from various sources. Most facades are covered with square plates of flapping length size from $51$~mm to $127$~mm, and separated by a distance ranging from $1.2$ to $1.6$ times the pendulum length. To account for the variability in viewing angles, we apply a quadratic transformation that maps the quadrilateral distorted plate images to rectangles. The spatial scale is recovered from the plate dimensions and their known spacing distances. A video snapshot and a corrected image of the \textit{Swiss Science Center Technorama}~\citep{technoramaF} are shown in fig.~\ref{fig:terrain}a. Eventually, A total of 18 videos taken on 6 different facades are studied.

%Some spectral analyses on Ned Kahn's large-scale façades are achieved based on amateurs' videos from the Internet. \Jishen{To make simple physical interpretations, only videos taken with fixed visual field of pendulums in steady wind flow being coaxial to the pendulum grids are examined.} 

We perform a spatiotemporal FFT on the corrected images. Fig.~\ref{fig:terrain} shows two typical examples from two facades~(left: \textit{Glacial Facade}~\citep{glacialF}, right: \textit{Digitized Field}~\citep{digitizedF}). The spatiotemporal diagrams of the image intensity along a line parallel to the direction of the wave propagation are shown in Fig.~\ref{fig:terrain}b,c and the corresponding spectra in Fourier space, averaged over the direction perpendicular to the wave propagation (d,e). Similar to our laboratory results, the fluctuation intensity peaks around two branches, which we consider to be the wave dispersion relation (branch II) and the convection velocity of large scale structures (branch I). In Ned Kahn's facades, since plates are free to swing with no elastic coupling between them, the slight inclination of the dispersion relation is assumed to be wind-driven, as observed for small wire diameters. The similarity between spectral maps of fluctuations and our laboratory experiments can then be used to deduce the traveling speed of turbulent fluctuations on the kinetic facades. 

% Figure environment removed

For each sequence, we measured the convection velocity $U_c$ from a linear fit of the branch I of the spatio-temporal spectrum. As the dispersion relation does not significantly depend on the wavenumber $k$, this convection speed can be obtained from a single image analysis. Indeed, assuming that the frequency of maximum response $f_M$ corresponds to the natural oscillation frequency $f_0$ of the pendulum, we estimate the convection velocity as $\omega_0/k_M$, where the wavenumber $k_M$ is measured from the wavelength of the dominant patterns of the Ned Kahn's facade.  Fig.~\ref{fig:Uc_terrain} shows the estimated convection velocity $\omega_0/k_M$ (blue square) as a function of the convection velocity $U_c$ obtained from spectral analysis. We eventually compared to a third measurement technique. From a series of image, we measured the traveling speed $C$ of the dominant wave crest, as shown in fig~\ref{fig:Uc_terrain} (red circle). We found a quantitative agreement between the two alternative methods, which do not require prior determination of the spatiotemporal spectrum, and the advection speed $U_c$.

%Right $y$-axis shows the scale for measuring the displacement of a single wave crest over ten frames and we obtain direct measurements of the phase velocity $C$ (blue diamonds) of the most energetic mode Left $y$-axis shows the scale for the ratio of $\omega_M$ to $k_M$, with $\omega_M$ and $k_M$ the peaks in $k$-averaged and $f$-averaged spatiotemporal spectrum (red circles), $\omega_M$ and $k_M$ the peaks of temporal and spatial FFT spectra (green squares), and same as for green squares with $f_M=f_0$ approximation (black triangles). 
The relation in Fourier space between the pendulum angles and the turbulent forcing thus provides an original, cost-effective and promising pathway to characterize spatio-temporal turbulent fluctuations at large scales, in particular for atmospheric turbulent boundary layers.

%\steph{Describe both axis of figure 5 sequentially, and all colors of data points.}

%by applying a temporal Fourier transform on the image sequence, this gives rise to a much simpler approach to the convection velocity measurements. As for the maximum wavenumber $f_M$, multiple ways are considered. 1. direct measurements of the wave pattern travelling speed by crest tracking on the image sequence, since the phase speed of the most excited wave train being identical to the convection velocity. 2. one can deduce the maximum wavenumber by a direct application of the spatial FFT on the image sequence, yet the result quality might be sensible to both spatial resolution of the image and also to the induced noise by the large vortex motion. 3. direct measurements of the wavelength on an image snapshot. \sout{3. to improve the measurement quality from the above mentioned short comes, a band-pass filter around $f_M$ should be applied first on the target image set, and repeat 2.} Fig.~\ref{fig:us}}  Blue dots represent ...

%\textit{Conclusion.}--- \Jishen{We promote quantitative measurements of near-wall large-scale convective motions by means of a 1D chain of coupled pendulum model in a laboratory environment. In the case of external force being purely mechanical, we have successfully predicted the dispersion relation by means of wave demodulation. As the wind blows over the chain, modifications have been observed as regards to the morphology in the spatiotemporal Fourier space: energy input from the wind is distributed around two branches, one branch refers to the dispersion relation with wind coupling induced shift, the other bears upon the large scale structures induced imprints of pressure fluctuations. By analogy, we extend the terminology to dynamics of 2D free pendulum facades in open field conditions using video analysis. Similar behaviors have been depicted. We quantitatively deduced the convection velocity being identical to the phase velocity of the most excited wave trains. After all, we provide an alternative approach to measurements of convection velocity of large structure induced vortices in atmospheric turbulent layer.}

%Wind turbulence ... Two branches... We give experimental evidence of a wind-induced coupling in the dispersion relation derived from the 1D discrete Sine-Gordon equation... Examinations on large-scale kinetic facades reveal the existence of propagating waves in an initially non-propagative medium. The frequency and wavenumber of the most excited waves are given by the intersection point of the two branches... relatively simple way to estimate the convection velocity of near-wall vortices

We acknowledge fruitful discussions with Ned Kahn, Antonin Eddi, Laurette Tuckerman, Marc Rabaud, Frédéric Moisy, Aliénor Rivière and Philippe Bourianne. We are specially grateful to Amaury Fourgeaud for technical support, Gauthier Bertrand for the wind tunnel design and Marc Fermigier for scientific support. We also acknowledge Jörg Moor from \textit{Swiss Science Center Technorama} and John Gray from \textit{Dundee City Council} for providing Ned Kahn's facade dimensions. This work was supported by a PSL Junior Fellow Starting Grant 2022.% Agence Nationale de la Recherche (Grant No. ANR-XX-XX-XX).


\bibliographystyle{apsrev4-1} % Tell bibtex which bibliography style to use
\bibliography{main} % Tell bibtex which .bib file to use (this one is some example file in TexLive's file tree)


\end{document}
