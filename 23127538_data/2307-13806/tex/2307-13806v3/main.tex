%% ****** Start of file template.aps ****** %
%%
%%
%%   This file is part of the APS files in the REVTeX 4 distribution.
%%   Version 4.0 of REVTeX, August 2001
%%
%%
%%   Copyright (c) 2001 The American Physical Society.
%%
%%   See the REVTeX 4 README file for restrictions and more information.
%%
%
% For Phys. Rev. appearance, change preprint to twocolumn.
% Choose pra, prb, prc, prd, pre, prl, prstab, or rmp for journal
%  Add 'draft' option to mark overfull boxes with black boxes
%
%
% To compile: latex prl.tex ; dvips prl.dvi -o prl.ps ; ps2pdf prl.ps
%
\documentclass[aps,onecolumn,10pt]{revtex4-2}
\usepackage{graphicx}  % needed for figures
\usepackage{dcolumn}   % needed for some tables
\usepackage{bm}        % for math
\usepackage{amssymb}
\usepackage{amsmath} 

\usepackage{xcolor}
\usepackage{hyperref}
\usepackage{ulem} % pour barrer du texte avec \sout{texte}



\newcommand{\Force}{\mathcal{A}}
\newcommand{\Jishen}[1]{\textcolor{blue}{#1}}
\newcommand{\steph}[1]{\textcolor{red}{#1}}
\newcommand{\dd}{\textrm{d}}
\newcommand{\tomega}{\tilde{\omega}}


\begin{document}

\title{Large-Scale Turbulent Pressure Fluctuations Revealed by Ned Kahn's Artwork}

\author{J.~Zhang}
\author{S.~Perrard}
\affiliation{PMMH, CNRS, ESPCI Paris, Universit\'e PSL, Sorbonne Universit\'e, Universit\'e de Paris, F-75005, Paris, France}
\date{\today}

\begin{abstract}

We investigate the dynamics of pendulum chains immersed in turbulent boundary layers. We combine qualitative video analysis of the artist Ned Kahn's \textit{kinetic façades} of buildings, and laboratory experiments on a unidimensional weakly coupled chain of pendulums. We performed analysis of both the façades and the laboratory scale model. We show that pendulum waves travelled in the direction of the wind. These waves originate from the excitation by the spatio-temporal pressure fluctuations. We identified two dispersion relation branches. The first branch corresponds to a resonant response of each pendulum at its natural {frequency of oscillations}. The second branch corresponds to an excitation by the advected pressure fluctuations along the chain. Using local pressure sensors, we show a quantitative agreement between the convection velocity of the pendulum chain waves and the pressure fluctuations. Eventually, we propose a model in Fourier space to describe the magnitude of each branch. We show that at a small wind speed, the pendulum response is dominated by the resonance at their natural frequency. At larger wind speed, the response becomes dominated by the advected pressure fluctuations.

%Indeed, the Artist Ned Kahn designed building facades, composed of decimetric scale oscillating plates, that exhibit complex behaviour when the wind blows.
%a 1D chain of coupled pendulums in a laboratory-scale turbulent flow. The system is composed of rectangular 3D printed thin plates, aligned with the air flow direction and elastically coupled with nylon fishing wires. In the case of external force being purely mechanical, we have successfully predicted the dispersion relation derived from a 1D discrete Sine-Gordon equation. As the wind blows over the chain, modifications have been observed as regards to the morphology in the spatiotemporal Fourier space: energy input from the wind is distributed around two branches, one branch refers to the dispersion relation with wind coupling induced shift, the other bears upon the large scale structures induced imprints of pressure fluctuations. By analogy, we extend the terminology to dynamics of 2D free pendulum facades in open field conditions using video analysis. Similar behaviors have been depicted. We quantitatively deduced the convection velocity of the large scale structures being identical to the phase velocity of the most excited wave trains. The current investigation suggests that such system might give alternative approach to large-scale turbulence measurements which remain challenging today.

%We investigate the dynamics of a 1D chain of coupled pendulums in a laboratory-scale turbulent flow. The system is composed of rectangular 3D printed thin plates, aligned with the air flow direction and elastically coupled with nylon fishing wires. We highlight a resonance mechanism between the pendulum oscillation and pressure fluctuations around the dispersion relation in the Fourier space. Our observations reveal equally some modifications of the dispersion relation as a result of additional aerodynamic coupling. The current investigation suggests that such system might give alternative approach to large-scale turbulence measurement which remains challenging today.
\end{abstract}

\maketitle

\section{Introduction} \label{sec:intro} Ned Kahn is an American artist who constructs numerous exhibits inspired by the ephemera of nature. Amongst his works is the kinetic façade, a regular assembly of small aluminum plates hinged to wall-attached grid covering the entire facade of buildings in various countries (US, Scotland, Netherland, Switzerland, France). As the wind blows along the wall, the plates oscillate freely, creating propagative wave-like large-scale patterns (fig.~\ref{fig:terrain}). The generation of such complex structures could be the result of numerous mechanisms, such as fluid-structure instabilities observed in flapping wings~\cite{shelley2011flapping}, flags~\cite{ristroph2008anomalous,connell2007flapping} and canopies~\cite{de2008effects}, or bistability of pendulums in turbulent flows~\cite{Gayout_2021}. Recent study on an air flow above a viscous liquid surface otherwise quiescent~\cite{paquier2016viscosity} suggests that these patterns might be the signature of the wall-pressure fluctuations induced by the turbulent boundary layer near the plate surface (TBL)~\cite{perrard2019turbulent}.

Turbulent pressure fluctuations are ubiquitous in natural and industrial fluid systems. In the scenario of a TBL adjacent to a flexible solid interface, the spatiotemporal pressure fluctuations interact with the eigenmodes of the structure and contribute to the resonance-induced vibration in many industrial applications leading to a crack propagation in aircraft wings or pressure vessel nozzles~\cite{blake2017mechanics}. In atmospheric TBL, knowledge of the spatiotemporal pressure fluctuations at large scale~\cite{He_2017} are also key elements in wind farm design to minimize the global power fluctuations of a plant affected by long-range fluctuations. These fluctuations have been characterized by two-point measurements in the far wake region~\cite{sorensen2002wind, wetz2023analyses, vermeer2003wind}. However, simultaneous measurements of both the spatial and temporal components of the turbulent structures remain still a long-standing challenge. Numercially, it has been shown~\cite{del2009estimation} that the wall pressure fluctuations travel at a speed $U_c(k)$ smaller than the bulk mean speed, which depends on the mode wavenumber $k$.
%To characterize spatial structures from point like measurements, G. I. Taylor introduced the frozen turbulent hypothesis~\cite{taylor1938spectrum}, that all fluctuations are advected at the mean turbulent flow speed $U$, far from any boundary. However, in a TBL, 

In this paper, we aim to characterize the waves that are observed on Ned Kahn's kinetic façades. To do so, we performed the analysis of kinetic facades video recording. To test the sensitivity in wind speed, we design a reduced one-dimensional model, made of a chain of weakly coupled pendulums. We study the response of the pendulum chain to turbulent fluctuations as a function of the wind speed. We perform in particular a component-by-component analysis in the space and time Fourier space of the pendulum chain.
%willmarth1962measurements
%{\bf Question} : what properties of the turbulent air boundary layer can be extracted from these visuals ? 

%\Jishen{ contexte industriel à mettre quelque part ? measurements of the tip-vortices convection velocity in far-wake region in wind farms }

% Figure environment removed

\section{Survey data collection} \label{sec:Kineticfacade} 

To characterize the dynamic patterns of the kinetic façades, we rely on video extracts recorded by amateurs. A total of 18 videos taken on six different facades have been gathered from YouTube.com and Vimeo.com, with typical frame rates ranging from $25$ to $29$ Hz. Most facades are covered with square plates of flapping length $l$ ranging from $51$~mm to $127$~mm, and separated by a distance ranging from $1.2$ to $1.6$ times the flapping length. The dimensions of the plates were collected by direct communications with Ned Kahn and technical services of the corresponding buildings. The spatial dimensions were also used to recover the scale factor from the amateur videos. To account for the variability in viewing angles of the camera, we apply a quadratic transformation mapping the quadrilateral distorted plate images to rectangles.  A video snapshot and a corrected image of the facade from the \textit{Swiss Science Center Technorama}~\citep{technoramaF} are shown in fig.~\ref{fig:terrain} with wavy patterns at three different instants. These moving patterns are observed with the reflecting light on the oscillating plates that leads to a change of pixel brightness as the pendulum angle changes. The $x$ axis is parallel to the ground surface, and the $y$ axis is vertical, pointing downward. To facilitate the data analysis, image sequences are horizontally flipped to ensure that the patterns propagate always toward the positive $x$ direction. 

In the absence of quantitative measure of the pendulum's instantaneous angle of rotation $\theta(x,y,t)$, we rely on the temporal variation of pixel brightness $I(x,y,t)$ by assuming a monotonous relation between $\theta$ and $I$. By doing so, we can qualitatively study the spatio-temporal behaviors of the moving patterns. As the videos are taken in natural conditions, both the wind amplitude and its direction changes over time (fig.~\ref{fig:terrain}b). The wavy patterns generated propagate mainly in $x$ direction but also exhibit some propagative bursts with a vertical component. From our analysis, we speculate that this fluctuating direction of propagation comes from the fluctuations of the wind direction, but we have not studied in details this phenomenon. To study the propagation along $x$, we truncated the video clip to focus on video extract in which the waves propagate mainly along $x$. We decompose the image sequence into two-dimensional space vectors $I_y(x,t)$ containing the pattern propagation in $x$ and $t$ for each row in $y$. To examine the frequency–wave number spectrum of the kinetic façade patterns, we perform the two-dimensional discrete Fourier transform in space and time that converts $I_y(x,t)$ from physical space into spectral space $\hat{I}_y(k,\omega)$:

\begin{gather}
    \hat{I}_y(k,\omega) = \mathcal{F} \left \{ I_y(x,t) \right \} = \int d^2x dt I_y(x,t) e^{-i(kx-\omega t)}\label{eq:Fourier1} \\
     I_y(x,t) = \mathcal{F}^{-1}\left \{ \hat{I}_y(k,\omega) \right \}=(2\pi)^{-3} \int d^2 k d\omega \hat{I}_y(k,\omega) e^{i(kx-\omega t)}
    \label{eq:Fourier2}
\end{gather}

with $k$ the wave number of the patterns in the $x$ direction and $\omega$ the angular frequency of the plate oscillation. Figure~\ref{fig:spec_terrain} shows the pattern propagations in physical space ($x, t$) and the corresponding frequency-wave number spectra averaged over the $y$ direction for two facades~(left: \textit{Glacial facade}~\citep{glacialF}, right: \textit{Digitized Field}~\citep{digitizedF}). The wave number $k$ is made dimensionless using the spacing $L$ between two adjacent pendulums.

% Figure environment removed

In Fourier space, we found that the pixel brightness is localized along two main branches. The branch II is located around $f = U_c k/(2 \pi)$ (dashed red line), where $U_c$ is a typical velocity. This velocity corresponds to the convection of pendulum fluctuations at a constant speed $U_c$, almost independent of the wave number. 
 %The convection velocity $U_c=2\pi f(k_c)/k_c$ of these modes is defined considering maximum intensity amplitude of the wave number spectrum at each given frequency $[\partial \hat{I}(k_c,f)/\partial k]_{k=k_c} =0$~\cite{comte1971simple,hussain1981measurements,goldschmidt1981turbulent}.  

The branch I is nearly horizontal, located around a typical frequency $f_0$, that corresponds to the natural oscillation frequency of each pendulum given by $f_0=\sqrt{1.5g/l}/(2\pi)$, where $l$ is the flapping length. This first branch can be interpreted as a resonant response at the natural flapping frequency of each pendulum to the turbulent fluctuations. 
In the absence of coupling between adjacent pendulum, we expect this first branch to be horizontal, corresponding to $f=f_0$ for all wave numbers $kL$. 
However, we found a slight increase of $f$ with $kL$ in all cases. This weak trend can be interpreted as a pendulum coupling induced by the wind blow. Similar behaviors have been observed for flags in an air flow. In this context, the inertia of the air flow brings about a linear contribution $m_AU\partial^2A/\partial x^2$ to the flag normal force balance, known as an added stiffness effect~\cite{de2001fluides, ramananarivo2014propulsion}. The mass of the otherwise air volume occupied by the flag is $m_A$ and $A$ is local out-of-plane amplitude of the flag. 
This term being proportional to the local curvature of the flag interface and the air flow velocity, tends to destabilize the flag motion. However, here we observe an increase of the natural frequency with the wave number, reminiscent of a stabilizing effect. Overall, this wind-induced coupling effect remains small, with less than 10 $\%$ of frequency increase at $kL=1$. It can be attributed to the presence of holes between pendulums, that prevent the build-up of large pressure differences between the two faces. We can therefore consider that the wave propagation is marginally affected. An image of branch II appears at larger wave numbers ($kL>2$) {in fig.~\ref{fig:spec_terrain}d}, and it originates from the secondary maximum at higher wave number of the Fourier transform of the rectangular plates brightness. For all the analysed videos, we have observed these two main branches.

To summarize, spatiotemporal spectral analysis on kinetic façades suggest that the facades are excited by two mechanisms: a resonant response at all wave numbers around $f=f_0$ (branch I) and a direct response to turbulent fluctuations traveling at constant convection speed $U_c$ (branch II). The maximum response is reached at the intersection between the two branches, which meets both criteria: the pendulum responds to spatiotemporal turbulent fluctuations that excite their natural frequency, corresponding to a wave number $k_{max} = 2\pi f_0/U_c$. To investigate the origin of the plate motions and better identify the mechanism at play, we built a one dimensional laboratory model.
%The lack of knowledge of certain relevant information such as the instantaneous pendulum angles and wind mean $\&$ turbulent velocities prevents us from testing the Taylor's hypothesis and giving further quantitative comparison between the pendulum response and the turbulent wind forcing. 

%To summarize, spectral analyses on kinetic façades using amateurs' videos suggest a two-branch energy redistribution of the pendulum dynamics in Fourier space: The 

\section{Laboratory set-up} \label{sec:setup} We design a one-dimensional experimental model at the laboratory scale, composed of a chain of pendulum plates, as sketched in fig.~\ref{fig:schema}. The chain measures $1.1$~m long, and it consists of $N=36$ 3D-printed pendulum thin plates with flapping length $l=48.5$~mm; width $w=28.5$~mm, thickness $h=1$~mm, mass $m=2.33$~g and density $\rho_p=1040$~kg/$m^3$, equally spaced by a distance $L=32.5$~mm. We use screws to attach each pendulum center to a nylon wire, sufficiently thin to prevent significant mechanical coupling between plates. To limit the chain warping induced by gravity, a sustaining pillar holds the wire every six pendulums while leaving the wire free to rotate. At both chain ends the wire is clamped onto the pillar's tip, imposing a zero rotating angle boundary condition~(fig.~\ref{fig:schema} inset) at a distance $L$ from the first and last plates. A guardrail wire spanning the whole length of the chain is attached to the pillars, this prevents the pendulums to make a full lap around the rotation axis thus keeps the rotation angle of each pendulum limited between $-\pi$ and $\pi$. The pendulum connection using the nylon wire of $a = 0.2$ mm in diameter introduces a weak elastic coupling constant between adjacent plates. 
In the approximation of zero elastic coupling constant, each pendulum is free to oscillate at a natural frequency $\omega_0=\sqrt{1.5g/l}=17.4~\mathrm{rad./s}$. In the absence of any external applied force and with negligible mechanical coupling between plates the motion of each pendulum follows:
\begin{equation}
  \partial_{tt}\theta_i + \lambda \partial_t \theta_i + \omega_0^2\sin\theta_i= 0.%\Jishen{A}_i(t),
  \label{eq:force_balance}
\end{equation}
In practice, we limit ourself to pendulum angles of the order of 0.4 rad. The equation of motion can therefore be linearized, $\sin \theta_i \sim \theta_i$. As a consequence, we will assume in particular that the {frequency of oscillations} $\omega_0$ is independent of the pendulum angle. 
In the absence of flow, the damping coefficient $\lambda$ is of the order of $0.3$ s$^{-1}$, which is much smaller than the natural angular frequency $\omega_0$. However, this damping should never be neglected, as it limits the amplitude growth near the resonance. 


%a torque-induced angular acceleration $A_i=T_i/I_p$ is applied to each pendulum. It appears as a source term on the right-hand-side of eq.~\ref{eq:force_balance}, with $I_p = ml^2/3$ the moment of inertia of the pendulum with respect to $x$ axis. {Discuss this torque term in greater detail : With the air flow direction being colinear with the pendulum chain, the turbulent torque $T_i$ is originated from the turbulent pressure fluctuations}. 

%can be further expressed as $T_i=\int_{S_i}\Delta p_i(x_i,r)\textbf{n}\times \textbf{r}$ds which is the cross product of the instantaneous pressure difference between both side of the plate surface $\Delta p_i(x,r)\textbf{n}$ applied on a given point $(x,r)_i$ and the point vector $\textbf{r}$, integrated over the total surface of plate $S_i$. Given the thickness of the plate $h \ll L$ ($h=1$ mm), the pressure difference $\Delta p_i(x,r)$ can be approximated by the pressure gradient $\nabla p_i(x,r)h$

%The air flow may also modify the constants $\lambda$, $\omega_0$, and $2L\omega_w$ respectively through additional flow dissipation, added-mass effect, and flow-induced coupling.

The pendulum chain is immersed in a turbulent flow, generated by an open circuit suction wind tunnel. The chain is placed at the symmetry plane of the wind tunnel. We limit the free-stream wind speed $U$ between 0 to $4.6$~m/s. Above 4.6~m/s, the pendulum starts to swing around the wire episodically. The wind tunnel has a measuring section of 720 $\times$ 720 $\times$ 1400 mm and a contraction ratio of 1.4. This low contraction design leads to a significant turbulent intensity in the incoming flow~\cite{bell1988contraction} and the grid-free open entrance makes the inlet flow susceptible to the room ambient turbulence where the large-scale structures are preferably prominent. All experiments were performed in a closed room, with no external current. Independent measurements on the turbulent statistics and the convection velocity of the pressure fluctuations in the wind tunnel were performed using a hot-wire probe and two pressure sensors. The hot-wire probe was calibrated using the King's law and a resistor anemometer Testo 425.

We define the typical {velocity fluctuations} as $u'= \sqrt{\langle u_x - \langle u_x \rangle \rangle}$. The turbulent intensity $u'/U = 0.094 \pm 0.01$ is constant over the full range of explored wind speed. We estimate the integral length scale $L_{int}$ from the temporal longitudinal auto-correlation function of velocity fluctuations, using Taylor hypothesis. We found that $L_{int}$ increases linearly at low wind speed, ranging from $L_{int} = 2.8$ cm for $U$=1.5~m/s to $L_{int} = 6.2$cm for $U$=4.6~m/s. This increase of the integral length scale with the wind speed is anomalous. It can be attributed to the natural injection of velocity fluctuations from upstream in the absence of a meshed grid. With a grid placed upstream, we would have expected an approximately constant integral length scale with wind speed. A first estimate of the dissipation rate $\epsilon$ can be obtained from the dissipation law $\epsilon = C u'^3/L_{int}$ for homogeneous and isotropic turbulence\cite{Pope_2000, vassilicos2015dissipation}, with $C\approx0.5$. We find a quadratic increase of $\epsilon$ with wind speed, with an upper value of about $\epsilon$ = 0.8 m$^2$/s$^{-3}$ for $U$= 4.6~m/s. From the second longitudinal structure function $S_2(r) = \langle (u(x+r,t)-u(x,t))^2 \rangle_t$, we estimate the range of inertial scale that follows K41 scaling\cite{kolmogorov1941local, Pope_2000}, $S_2(r) = C_2\epsilon^{2/3} r^{2/3}$, with $C_2 \approx 2$ from the literature\cite{Pope_2000}. We do observe a plateau for the compensated structure function $(S_2(r)/C_2)^{3/2} r^{-1}$ at all wind speeds. For $U=4.6$~m/s, we extract the value of the dissipation rate $\epsilon = 0.72$ m$^2$/s$^{-3}$ from the plateau. This value is compatible with our estimate from the large scale dissipation law. Assuming isotropic turbulence, we can also estimate the Taylor microscale $\lambda_T$, the scale at which the velocity gradients are maximum, from the dissipation rate and the velocity fluctuations $u'$. We have $\lambda_T = (15 \nu/\epsilon)^{1/2} u' \approx 8$~mm, and a corresponding Taylor Reynolds number $Re_\lambda = u' \lambda_T/\nu \approx 250$ for $U = 4.6$~m/s and $Re_\lambda = 140$ for $U = 1.5$ m/s. Overall, this wind tunnel facility provides a simple flow, homogeneous in the central region, to immerse objects of centimetric sizes, of size comparable with the integral length scale. Note that the pendulum chain length is much larger than the integral length scale: the pendulum chain oscillations will probe structures that are larger than the integral scale.

Immersed at the center of the channel, the pendulum motions are recorded from below and a white spot painted on each pendulum tip is used to track the instantaneous pendulum angles of orientation. For each wind speed, $90000$ images are recorded at a frame rate of $300.03$ images per second. From the sets of images, a spot-recognition code with sub-pixel accuracy is used to extract the pendulum angles $\theta_i(t)$. Figure~\ref{fig:timeshot} shows respectively a snapshot of the pendulum chain, and the spatiotemporal diagram of the inclination angles $\theta_i(t)$ for a wind speed of 3.38 m/s. Similarly to Ned Kahn's facade, we observe the propagation of patterns at an almost constant speed, traveling downstream along the pendulum chain. 

% Figure environment removed

% Figure environment removed

%between adjacent plates provides us with a tunable parameter for the wave dispersion relation. In particular, the wire diameter 

%A temporal Fast-Fourier Transform is performed on the $3rd$ pendulum to obtain the full spectrum of the oscillating angles and a signal demodulation ($\tilde{\theta}=\Re(\Phi e^{-i\Psi})$ where $\Phi_{x, f_r}= \langle \theta_{x,t} e^{i2\pi t f_r} \rangle_t$ and $\Psi = \tan^{-1}(\Im(\Phi_x)/\Re(\Phi_x))$) is proceeded on resonant frequencies including fundamental ones and sub-harmonics. The spatial dissipation rate $\kappa$ is obtained by using the exponential decay fit on $|\Phi|=ce^{-\kappa x}$ and a damped-monochromatic wave model is used to determine the associated wave number $k$. Fig.~\ref{fig:RD_kappa}a shows the dispersion relation for both pendulum spacing. An excellent agreement is achieved with theoretical predictions~(Eq.~\ref{eq:dispersion})
%\Jishen{nombre N ? Zero padding params ?}
%\begin{equation}
%  c^2\partial_{xx}\theta - \partial_{tt}\theta - \omega_0^2\sin\theta=\lambda \partial_t \theta - F \Jishen{- F_C\partial_{xx}\theta}
%  \label{eq:force_balance}
%\end{equation}

We measure each $\theta_i$ as a function of time, and we compute the Fourier transform $\hat{\theta}(k,f)$ of $\theta$ both in $x$ and time. The colormaps of $\hat{\theta}(k,f)$ are shown in fig.~\ref{fig:RD2D} for two different wind speeds of $U=1.16$~m/s and $U=3.16$~m/s. With the weakly coupled pendulum chain, we observe the same two main branches previously observed on the building facades.

The branch I corresponds to the propagation of waves at a frequency close to the pendulum natural frequency in the absence of flow. The theoretical prediction of a natural frequency $\omega_0$ independent of the wave number is illustrated in red dashed line in fig.~\ref{fig:RD2D}a,b. For the small wind speed $U=1.16$~m/s, the prediction using constant frequency $\omega_0$ shows excellent agreement with the experimental values (fig.~\ref{fig:RD2D}a). At higher wind speed condition $U=3.16$~m/s, we observe a slight but significant increase of the resonant frequency with the wave number $kL$. This trend was also observed on the kinetic façades, and we attribute it to a wake-induced aerodynamic coupling between adjacent pendulums. In the following sections, we refer $\omega_0$ to be the overall resonant frequency that includes both the natural frequency and the weak coupling effects.
% Figure environment removed

The branch II, on the other hand, corresponds to a continuum of Fourier modes located around a line $\omega = U_c k$, with a characteristic velocity $U_c$. This characteristic velocity increases with wind speed (fig.~\ref{fig:RD2D}), and can be interpreted as a convection speed of spatiotemporal structures along the pendulum chain. Compared to the field study, we have now access independently to the properties of the surrounding turbulent flow, as a function of the wind speed.

% Figure environment removed



%In our spatiotemporal Fourier spectrum, the convection velocity $U_c(k)$ for each mode $k_c$ is defined as the slope $\omega_c/k_c$ at the scale $k_c$~\cite{wills1964convection,del2009estimation, moin2009revisiting}. We first measure the wave number $k=k_c$ giving the maximum amplitude at each frequency $f(k_c)$, we then extract the convection velocity from a linear interpolation passing through zero of all $k_c$ and $f(k_c)$ distributions. Figure~\ref{fig:cv}a shows the normalized convection velocity as a function of the mean velocity.
%The branch I region spans over a large range of frequencies. It corresponds to the signature of pressure fluctuations, that are transported by the mean flow at a convection velocity $U_c$. 
We then compare the convection speed of turbulent structures on the pendulum chain to the convection speed of pressure fluctuations in the absence of the pendulum chain. From the spatiotemporal spectrum of pendulum angles, for each wind speed, we apply a linear fit on the local maxima of $\hat{\theta}$ along the branch II. Doing so, we extract the slope of branch II as a function of the wind speed. To check that the convection speed $U_c$ scales with the wind speed $U$, we represent in fig.~\ref{fig:cv} the ratio {$c_U=U_c/U$} as a function of the wind speed $U$. This ratio increases with wind speed, and converge to a constant value {near} $0.8$ above $U>2$~m/s. This ratio is reminiscent of the convection speed of wall pressure fluctuations in a zero pressure gradient TBL~\cite{choi1990space,caiazzo2023effect}. %results~\cite{choi1990space,caiazzo2023effect} for large scale modes in a zero pressure gradient TBL, and the ratio reduces to $0.6$ at high wave numbers. 
To compare $U_c$ with the convection speed of pressure fluctuations in the turbulent channel, we performed measurements of two points spatiotemporal pressure fluctuations, in the absence of pendulum chain. The measurements were conducted with two acoustic pressure probes (PCB 103B01 with $\pm15\%$ sensibility) at the center line of the channel, spaced by $36$ cm which is a distance larger than the integral scale. We measured the convection speed of pressure fluctuations from temporal delay of the correlation peak between the two probe signals. The ratio {$c_U$} for pressure fluctuations have been superimposed on fig.~\ref{fig:cv} (black stars). We find a quantitative agreement with the convection speed of patterns along the pendulum chain, showing that branch II can indeed be interpreted as the signature of turbulent fluctuations traveling downstream, at a speed close but smaller than the wind speed. Note that the ratio {$c_U$}, however, can be sensitive to the type of turbulent flow, and we do not expect this curve to be universal. Another prediction from the literature of pressure fluctuations in TBL is the decrease of the convection speed with the wave number at large enough Reynolds numbers: the large modes (small $kL$) travel faster than the smaller ones, as observed numerically by Choi $\&$ Moin~\cite{choi1990space} and experimentally by Willmarth $\&$ Wooldridge~\cite{willmarth1962measurements}. From the motion of the pendulum chain, we have in theory access to the convection speed of patterns as a function of the wave number. In Fourier space, we extract the local slope of branch II as a function of the wave number $kL$. Figure~\ref{fig:cv}b shows the corresponding ratio {$c_U(k) = U_c(k)/U$} as a function of the wave number $kL$ for two wind speeds (blue and red). The scatter is due to the limited resolution in $k$. For small wind speed (red), we do not observe a significant trend with the wave number. For higher wind speed (blue), we see a slight decrease in the convection speed with the wave number, even though we are close to our limit of resolution. {Note that considering the pendulum's spacing $L$, the maximum dimensionless wave number $k_{max}L=2\pi$ introduces a wave number cutoff $k_{max} = 193~\mathrm{m^{-1}}$}:
%Note that the pendulum's dimension and their spacing $L$ introduces a wave number cutoff $k_{max} = 193~\mathrm{m^{-1}}$
 the structures smaller than $L$ will not be resolved by the chain oscillations.
%for small scale structures, yet the good agreement with results from two-point correlation suggested that the flow in the wind tunnel is dominated by large scale modes. The convection velocity per wave number from the pendulum dynamics measurements is shown in Figure~\ref{fig:cv}b for the two mean bulk velocities. The mean convection velocity for each case is represented by horizontal dotted lines. 

The analysis of the two branches in Fourier space shows that the pendulum chain responds to the turbulent fluctuations as a collection of pendulums randomly pushed and pulled. To describe the pendulum angle statistics, we consider that the motion of each pendulum can still be described by an oscillator, and we model the acceleration exerted by the flow on the pendulum by an effective term $\Force_i(t)$:
\begin{equation}
  \partial_{tt}\theta_i + \Lambda \partial_t \theta_i + \Omega_0^2\theta_i= \Force_i(t),
  \label{eq:force_balance_full}
\end{equation}
where $\Lambda$ and $\Omega_0$ are respectively the damping coefficient and the resonant frequency of each pendulum in the presence of an external flow, that may depend on the wind speed. We have limited the analysis to the linearized equation, valid for small angles of oscillation. From the analysis of the spatiotemporal spectrum, we have seen that $\Omega_0=\omega_0$ holds for all wind speeds. For the sake of simplicity, we also assume no wind-induced coupling between adjacent pendulums. The acceleration $A_i$ is related to the normal stresses integrated over the plate surface by:
\begin{equation}
  \Force_i = \frac{1}{I_p}\int [{\bf \tau} {\bf n}] r ~ \textrm{d}S,
  \label{eq:effective_force}
\end{equation}
where $I_p = \rho_p w h l^3/3 = 10.9~$g cm$^2$ is the moment of inertia of the pendulum, ${\bf \tau}$ is the fluid stress tensor on the plate surface, ${\bf n}$ is the unit vector normal to the plate surface and the bracket $[\cdot]$ stands for the difference between the two plate surfaces. There is a priori no simple model for the forcing term $\Force_i$. Indeed, the presence of the pendulum chain introduces mixed boundary conditions at the plate surface, that will modify the flow structure near the plate. In the following, we will consider the plates as infinitely thin. Before investigating the pendulum chain case, it is interesting to briefly review two main limit cases: vanishing inertia (i) and rigid wall (ii). In limit (i), the plates do not modify the turbulent flow, and their motions follow the velocity fluctuations along the normal direction of the place. We then expect the normal stress difference $[{\bf \tau} {\bf n}]$ to scale with the instantaneous momentum flux perpendicular to the plate surface, {\textit{i.e.}}, $[{\bf \tau} {\bf n}] \sim \rho_a |u_\theta|u_\theta$, where $u_\theta$ is the velocity fluctuations along ${\bf n}$. From Taylor hypothesis~\cite{Pope_2000}, the velocity turbulent fluctuations travel at the mean speed of the flow, and the acceleration therefore scales as $\Force_i \sim 3 \rho_a/(2 \rho_p L h) u_\theta^2$. In limit (ii), the normal stress will be that of the wall-normal stress in growing turbulent boundary layers on both sides of the plate. The convection speed of the forcing would then be that of pressure fluctuations, which is typically smaller than the mean flow velocity, and the normal stress would scale as $\rho_a {u^*}^2$, where $u^*$ is the friction velocity. In practice, we can estimate if the pendulum freely follows the air, by looking at the magnitude of the relative pendulum velocity. At the tip of each plate, the pendulum velocity is of the order of $\theta_{rms} \omega_0 l$. 
% Figure environment removed

Figure~\ref{fig:thetaRMS} shows the root mean square $\theta_{rms}$ of the pendulum motion as a function of the wind speed. We observe that $\theta_{rms}$ is {roughly} proportional to the wind speed $U$ for $U < 3$~m/s, and increases faster than linear for $U \geq 3$~m/s. For $U<3$~m/s, a linear fit gives $\theta_{rms} = \alpha~U$ where $\alpha = 0.06~\pm~0.005~\mathrm{rad.m^{-1}s}$. To investigate if the pendulum surface moves with the flow velocity, we compare the azimuthal velocity at the plate tip $U_{plate}$  scales as $U_{plate} \sim \omega_0 l \theta_{rms} \sim 0.05 U$, which is of the order of but smaller than the velocity fluctuations $u' \sim 0.1 U$. The pendulum chain motion therefore lies in an intermediate regime, in which the plates can be approximated neither by a rigid wall nor a freely moving boundary. For $U>3$~m/s, we found a quadratic growth of $\theta_{rms}$  with a fitting coefficient $\gamma=2.18\times 10^{-2}~\mathrm{rad.m^{-2}s^2}$. 
%\sout{From the study of the convection velocity $U_c$, we have seen that the traveling speed of turbulent structures along the pendulum chain corresponds to the convection speed of pressure fluctuations.} 

%A naive scaling $\theta$
 To understand the origin of these two scaling laws, we will perform a component-by-component analysis in Fourier space. %From the spatiotemporal spectrum, we now aim at describing the pendulum oscillation in this flow.

%\section{Results $\&$ discussions} \label{sec:results} 
\subsection{Resonant response in Fourier space} 

{A mechanism of spatio-temporal resonance between turbulent fluctuations and waves was introduced by Phillips~\cite{phillips1957generation} in the context of wind waves induced by turbulent wall-pressure fluctuations. More recently, Perrard \textit{et al.}~\cite{perrard2019turbulent,nove2020effect} combine component-by-component Fourier analysis with direct numerical simulation of wall pressure statistics to describe the statistics of surface waves below the onset of wind wave instability. A linear response theory was shown to describe accurately the observed behaviors. Here we proceed with a similar approach, to model the response of the pendulum chain in Fourier space.}
{Under the following assumptions:
\begin{itemize}
    \item A statistically stationary steady state is reached for both the turbulent flow and the pendulum chain oscillations.
    \item The angle of oscillations are small enough to describe the dynamics with a linear response theory.
    \item The pendulum oscillation dynamics is described by Eq.~\ref{eq:force_balance_full}.    
\end{itemize}
We Fourier transform in space and in time eq.~\ref{eq:force_balance_full} to express $\hat{\theta}(k,\omega)$ as:}
\begin{equation}
|\hat{\theta}|^2=\frac{|\hat{\Force}|^2}{(\omega^2-\omega_0^2(k))^2+\Lambda^2(k) \omega^2},
\label{eq:fourier}
\end{equation}
where $\hat{\mathcal{A}}$ is the Fourier transform of the effective acceleration, and the denominator is minimum at the pendulum resonant frequency $\omega = \omega_0$. The maximum response in Fourier space is therefore either located in the vicinity of the dispersion relation ($\omega \sim \omega_0$) where the denominator is minimum (branch I) or located where the forcing $\hat{\Force}$ is maximum (branch II). At a given wave number and an angular frequency $\omega \sim \omega_0$, the pendulum chain resonates with the spectral modes of $\hat{\mathcal{A}}$, and the amplitude growth is {only} limited by dissipative effects {which depends on the damping coefficient $\Lambda$}.

% Figure environment removed
To test the validity of eq.~\ref{eq:fourier}, we compute {from the experimental data} the energy spectrum $|\hat{\theta}|^2$. Fig.~\ref{fig:reson}a,b shows the energy spectrum {for $kL=0.75$ as a function of the frequency $f$ for two different} wind speeds (red line). We observe resonant curves {near the resonant frequency $\omega_0$ in the absence of wind}, for all small wave numbers, $kL < 1$, and all wind speeds. To describe the shape of the resonant response near $\omega_0$, we fit the expression of eq.~\ref{eq:fourier} using two adjustable parameters, the effective {acceleration} $\hat{\Force}(\omega_0)$ at the resonance, and the damping coefficient $\Lambda$. We {then} neglect the variation of $\hat{\Force}$ with $\omega$, which is valid for sharp resonance, $\textit{i.e.}$ {for} $\Lambda \ll \omega_0$. We {eventually} find a quantitative agreement between eq.~\ref{eq:fourier} (blue dashed line) and the experimental measurements for $kL <1$ and all wind speeds. {From the fit of the resonant curve, we extract the values of $\hat{\Force}(\omega_0,kL)$ and $\Lambda$ as a function of the dimensionless wavenumber $kL$ and the wind speed.}
%both the value of the source term $\hat{\Force}$ at $\omega=\omega_0$ by $\hat{\mathcal{A}}=\hat{\mathcal{A}}_r+\gamma (\omega-\omega_r)$ with two adjustable parameters $\hat{\mathcal{A}}_r$ and $\gamma$, corresponding to a first-order Taylor expansion of $\hat{\mathcal{A}}$ in the vicinity of the resonant frequency $\omega_r$.
The effective acceleration $\hat{\Force}(\omega_0,kL)$ at the resonance is shown in fig.~\ref{fig:fitcoef}(a) as a function of the dimensionless wave number $kL$ for different wind speed (green color bar). The effective acceleration $\hat{\Force}(\omega_0,kL)$ exhibits a smooth maximum around $k_0 = \omega_0/U_c$, which corresponds to the cross-over between the branch I and II. The magnitude of $\hat{\Force}(\omega_0,kL)$ decreases both for smaller and larger wave numbers. Figure~\ref{fig:fitcoef}b shows the effective acceleration $\hat{\Force}_{max} = \hat{\Force}(\omega_0,L\omega_0/U_c)$, {located in Fourier space} at the intersection between branches I and II, as a function of the wind speed $U$. We find that the effective acceleration goes as $\hat{\Force}_{max} = c_\Force U^2$ with $c_\Force = 0.23~\mathrm{rad.m^{-3/2}s^{1/2}}$. {This effective acceleration originates from the torque of pressure fluctuations integrated over the plate surface (eq.~\ref{eq:effective_force}), which scales as $\hat{\Force}_{max} \sim U^2$ as expected for inertial forces}. %However, the scaling holds only for the highest amplitude Fourier mode. Indeed, the entire pendulum response described by $\theta_{rms}$ only increases linearly with wind speed. 

%To understand the full pendulum response, we will now integrate eq.~\ref{eq:Fourier2} over Fourier space. To do so, {a model for the forcing term and the damping coefficient $\Lambda$ are required.} as a function of the wave number $k$ in the vicinity of the resonant curve, $\omega \sim \omega_0$ where most of the energy is located. %We also compute the damping coefficient $\Lambda$ at the resonance for all wave numbers. 

% Figure environment removed

{The fitted values of the damping coefficient $\Lambda$} are shown in fig.~\ref{fig:fitcoef}c as a function of the dimensionless wave number $kL$, for different wind speeds (color-coded). For small wave numbers ($kL<0.2$), the damping coefficient is almost constant and then decreases with $kL$. 
{Note that the fit is performed with a constant forcing term $\hat{\Force}$ which is a valid assumption only in the vicinity of the resonance.} {The damping coefficient  $\Lambda_{max} = \Lambda(\omega_0,L\omega_0/U_c)$ at the intersection between branches I and II is shown in figure~\ref{fig:fitcoef}d as a function of the wind speed $U$}. We find that $\Lambda_{max}$ increases linearly with the wind speed $U$. This increase of dissipation with the wind speed could either originate from an enhancement of the momentum transfer from the plate oscillation to the fluid by the turbulent flow, or by an increase of dissipation in the boundary layers, as observed recently for bubble oscillations in turbulence~\cite{riviere2024bubble}.

%Considering that this transfer is driven by momentum transport by eddies of size comparable to the plate dimensions, the dissipative torque integrated over both faces writes $\rho_a \nu_{eff} w l^2 \partial_t \theta$, and the damping coefficient is given by $\Lambda = \rho_a W L^2\nu_{eff}/I_p$ where $\nu_{eff}$ is an effective viscosity that depends on the air-flow properties.
%Using a model of turbulent viscosity, we estimate the effective viscosity as $\nu_{eff} \sim u' \ell$, where $u' = u_{rms}$ is the typical magnitude of velocity fluctuations, and $\ell$ is a characteristic length of the plate dimensions. If we take $\ell = \kappa l$ assuming Prandtl's mixing layer hypothesis, where $\kappa=0.41$ is the von Kármán constant and $l$ the length of the pendulum, the associated damping coefficient $\Lambda$ then scales as 
%\begin{equation}
%\Lambda \sim \frac{\rho_a}{\rho_p}\frac{3 \kappa \beta U}{h},
%\end{equation}
%where $u' = \beta U$ is the velocity fluctuation at the center of the wind tunnel. Eventually, we find a fit coefficient $c = 3 \rho_a w \kappa \ell \beta/m = 0.14~\mathrm{m^{-1}}$, of the order of the measured fit coefficient shown in fig.~\ref{fig:fitcoef}b, $c_\Lambda = 0.34~\mathrm{m^{-1}}$. Note that the associated length scale $1/c_\Lambda$ is larger than the plate dimension, which originates from the large density ratio between the air and the plate. The scaling for the turbulent dissipation, however, would require additional measurements to be confirmed, in particular by varying the turbulent intensity $\beta$, as well as the plate dimensions.

The maximum response is located in Fourier space at the intersection between the two branches. {From the values of the fitted coefficients $\Force$ and $\Lambda$}, we {now} estimate the contribution {of the resonant response near $\omega=\omega_0$ and $k_0 = \omega_0/U_c$ to $\theta_{rms}$}. To do so, we consider the integral of eq.~\ref{eq:fourier} over the wave numbers $k$ and the angular frequency $\omega$:
\begin{equation}
\theta_{rms}^2 = \int \dd k \int \dd \omega  \frac{|\hat{\Force}|^2}{(\omega^2-\omega_0^2(k))^2+\Lambda^2(k) \omega^2}.
\label{eq:theta_integral}
\end{equation}

%{Check values of fig9b : the forcing should be given in unit of observation time, and size of the system, so we don't have to divide by 1/L and 1/T to get $\theta_{rms}$}

We estimate the resonant response $\theta_r$ by considering wave numbers in the range $k \in [\omega_0/U_c-\Delta k/2,~\omega_0/U_c+\Delta k/2]$, {where the spectral width $\Delta k \sim \omega_0/U_c$ is associated to the typical fluctuations of the} convection speed of turbulent structures. In the limit of large quality factor $\omega_0/\Lambda \gg 1$, we approximate the numerator by its value at {the} resonance, $\hat{\Force} = \hat{\Force}(\omega_0, \omega_0/U_c)$, and we obtain an estimate of the resonant response as:
\begin{equation}
\theta_{r}^2 = \frac{2 \Delta k {c_U}^2 U^4}{\omega_0^3} \int_{-\infty}^{+\infty} \dd \tomega  \frac{1}{(\tomega^2-1)^2+\Lambda^2(k)/\tomega_0^2~\tomega^2},
\label{eq:theta_resonance}
\end{equation}
where the integral $I(\Lambda) = \int \dd \tomega~ ((\tomega^2-1)^2+\Lambda^2 \tomega^2)^{-1} = \pi/\Lambda$ with $\tomega$ the normalised frequency. We eventually obtain an estimate for the resonant response :
\begin{equation}
\theta_{r}^2 = \frac{2 c_\Force^2 U^4}{\omega_0^2 U_c}\pi \frac{\omega_0}{\Lambda}.
\label{eq:thetar_scaling}
\end{equation}
{Using the fitted expression $\Lambda = c_\Lambda U$ and the expression of the advection speed $U_c = c_U U$, we eventually obtain an expression for the resonant response}:
\begin{equation}
\theta_{r} = \sqrt{\frac{2 \pi c_\Force^2}{c_\Lambda c_U \omega_0}} U.
\label{eq:thetar_scaling}
w\end{equation}
We eventually found that the resonant response {yields an amplitude of oscillation proportional to the wind speed $U$}, as observed experimentally for $U<3$~m/s. Using the fitted values  $c_{\Lambda}$,$c_\Force$ and $c_U\sim=0.8$, we obtain $\theta_r = c_\theta U$ with $c_\theta = 0.23~\mathrm{rad.m^{-1}s}$. This coefficient is of the same order of magnitude but larger than the fitted value of the slope $\alpha = 0.06~\mathrm{rad.m^{-1}s}$ shown in fig.~\ref{fig:thetaRMS}. 
Note that at higher wind speeds, {the damping coefficient increases and the assumption of sharp resonance is not fulfilled. Consequently, the pendulum response far from the resonance cannot be neglected. The integral over Fourier space of Eq.~\ref{eq:theta_integral} is eventually dominated by the energy along branch II}. Assuming a constant forcing term {along the line $\omega = k U_c$} for $k L_{int} <1$, the amplitude of pendulum oscillation $\theta_{rms}$ given by eq.~\ref{eq:theta_integral} {scales} quadratically with $U$, as $\theta_{rms} \sim c_\Force U^2/(\omega_0^{3/2}L_{int}^{1/2})$. 
Using the fitted coefficient $c_\Force$ and a constant integral length $L_{int} = 5$ cm for large wind speeds, we obtain $\gamma_{th} = 1.41\times 10^{-2}~\mathrm{rad.m^{-2}s^2}$, which is in fair agreement with $\gamma_{exp}$ previously found for the $\theta_{rms}$ fit in the quadratic regime. 
To summarize, we find that in the explored range of wind speed, we observed two regimes of pendulum response. At low wind speed, the oscillations are dominated by the resonant response to pressure fluctuations, while at higher wind speed, the response is dominated by the direct forcing from the pressure fluctuations traveling along the pendulum chain.

\section{Conclusion} \label{sec:conclusion} Inspired by Ned Kahn's kinetic façade artwork, we conducted a qualitative analysis of structures propagating on the building facades. To explain the physical processes at play, we studied a one dimensional chain of weakly coupled pendulums immersed in a turbulent flow. We performed analysis in Fourier space, and we showed that both the natural system and the laboratory analog exhibit energy along two main branches in Fourier space. These two branches correspond to two different mechanisms. The branch I corresponds to the resonant response of each pendulum at its natural oscillation frequency. We measured the associated damping rate as a function of the wind speed, as well as the magnitude of the forcing at the resonance. We showed that the dissipation is proportional to the wind speed. The forcing term increases as $U^2$ with the wind speed, as expected from inertial forces. For small damping ($U<3$~m/s for the laboratory analog), the response is dominated by this mechanism, and the oscillation amplitude $\theta_{rms}$ scales as the wind speed $U$. The second mechanism (branch II) corresponds to the response to turbulent fluctuations traveling downstream along the wire at an almost constant convection speed independent of the wave number. This convection speed turns out to be equal to the convection speed of pressure fluctuations measured with pressure probes in the absence of the pendulum chain. This convection speed is of the order of the wind speed, but smaller (typically 80\% of wind speed). This pendulum system, either in one dimension or two dimensions, naturally responds at large wave number and small frequencies, revealing some large scale structures of the flow. However, the resonant response generates a filter, which preferentially amplifies the fluctuations at the pendulum frequency. 


%\textit{Convection velocity of vortex structures.}--- Turbulent wind field can be considered as a superposition of the harmonic travelling vortices at a convective speed $U_c=\omega_c/k_c$~(\cite{wills1964convection}). Though the very terminology for convection velocity of vortices being still under debate~(\cite{del2009estimation, moin2009revisiting}), 

%The calculation is carried out at a zone away from the resonant one ($f_c<f_0$) which corresponds physically to a zone where large eddies are advected at a constant speed thus the convection velocity should be independent of both $f_c$ and $k_c$, this refers to Taylor's frozen-flow hypothesis and is confirmed by our observations (see red solid lines in Fig.~\ref{fig:RD2D}a,b). 

%{It is worth noticing that the spatiotemporal structure of this region spans over a relatively large frequency range including the ones outside the frequency cutoffs, this reveals the non-propagative behavior that is imminent to the wall turbulence.} 
%\Jishen{a deplacer ou supprimer: No systematic variation in $U_c$ emerges as the wire coupling constant varies,} \Jishen{which means the retro-action of the plate dynamics on the vortex structure is of no significance.} and the ratio of $U_c$ to the free-stream velocity $U_a$ reads $0.36$ for the smallest wind speed (i.e: $1$~m/s) and reaches almost $1$ at $2.5$~m/s. 


%Some spectral analyses on Ned Kahn's large-scale facades are achieved based on amateurs' videos from the Internet. \Jishen{To make simple physical interpretations, only videos taken with fixed visual field of pendulums in steady wind flow being coaxial to the pendulum grids are examined.} 


%Right $y$-axis shows the scale for measuring the displacement of a single wave crest over ten frames and we obtain direct measurements of the phase velocity $C$ (blue diamonds) of the most energetic mode Left $y$-axis shows the scale for the ratio of $\omega_M$ to $k_M$, with $\omega_M$ and $k_M$ the peaks in $k$-averaged and $f$-averaged spatiotemporal spectrum (red circles), $\omega_M$ and $k_M$ the peaks of temporal and spatial FFT spectra (green squares), and same as for green squares with $f_M=f_0$ approximation (black triangles). 


%{Describe both axis of figure 5 sequentially, and all colors of data points.}

%by applying a temporal Fourier transform on the image sequence, this gives rise to a much simpler approach to the convection velocity measurements. As for the maximum wave number $f_M$, multiple ways are considered. 1. direct measurements of the wave pattern travelling speed by crest tracking on the image sequence, since the phase speed of the most excited wave train being identical to the convection velocity. 2. one can deduce the maximum wave number by a direct application of the spatial FFT on the image sequence, yet the result quality might be sensible to both spatial resolution of the image and also to the induced noise by the large vortex motion. 3. direct measurements of the wavelength on an image snapshot. \sout{3. to improve the measurement quality from the above mentioned short comes, a band-pass filter around $f_M$ should be applied first on the target image set, and repeat 2.} Fig.~\ref{fig:us}}  Blue dots represent ...

%\textit{Conclusion.}--- \Jishen{We promote quantitative measurements of near-wall large-scale convective motions by means of a 1D chain of coupled pendulum model in a laboratory environment. In the case of external force being purely mechanical, we have successfully predicted the dispersion relation by means of wave demodulation. As the wind blows over the chain, modifications have been observed as regards to the morphology in the spatiotemporal Fourier space: energy input from the wind is distributed around two branches, one branch refers to the dispersion relation with wind coupling induced shift, the other bears upon the large scale structures induced imprints of pressure fluctuations. By analogy, we extend the terminology to dynamics of 2D free pendulum facades in open field conditions using video analysis. Similar behaviors have been depicted. We quantitatively deduced the convection velocity being identical to the phase velocity of the most excited wave trains. After all, we provide an alternative approach to measurements of convection velocity of large structure induced vortices in atmospheric turbulent layer.}

%Wind turbulence ... Two branches... We give experimental evidence of a wind-induced coupling in the dispersion relation derived from the 1D discrete Sine-Gordon equation... Examinations on large-scale kinetic façades reveal the existence of propagating waves in an initially non-propagative medium. The frequency and wave number of the most excited waves are given by the intersection point of the two branches... relatively simple way to estimate the convection velocity of near-wall vortices

\begin{acknowledgments}

We acknowledge fruitful discussions with Ned Kahn, Antonin Eddi, Laurette Tuckerman, Guowei He, Marc Rabaud, Frédéric Moisy, Aliénor Rivière, and Philippe Bourianne. We are specially grateful to Amaury Fourgeaud for technical support, Gauthier Bertrand for the wind tunnel design, and Marc Fermigier for scientific support. We also acknowledge Jörg Moor from \textit{Swiss Science Center Technorama} and John Gray from \textit{Dundee City Council} for providing Ned Kahn’s facade dimensions. This work was supported by a PSL Junior Fellow Starting Grant 2022 (No. 2022–305) and by the Agence Nationale de la Recherche with grants ANR Lascaturb (reference ANR-23-CE30-0043).

\end{acknowledgments}

%\bibliographystyle{apsrev4-1} % Tell bibtex which bibliography style to use
\bibliographystyle{unsrt} % Tell bibtex which bibliography style to use
\bibliography{biblioWV.bib} % Tell bibtex which .bib file to use (this one is some example file in TexLive's file tree)


\end{document}
