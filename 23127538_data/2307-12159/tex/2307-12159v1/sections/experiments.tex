\section{Results}
\label{s.res}

The experimental results were obtained for each subtask separately. To provide a more in-depth evaluation of the proposed approach, we considered three evaluation measures: accuracy, sensitivity, and specificity. Table~\ref{t.results} presents the results for each subtask accordingly.

\begin{table}[!htb]
\caption{FPG results for each subtask in speech and non-speech data.}
\label{t.results}
\scalebox{0.91}{
\begin{tabular}{|cc|c|c|c|c|}
\hline
\multicolumn{2}{|c|}{TASK}                                                                                                        & Classification                              & Accuracy                                & Specificity                             & Sensitivity                             \\ \hline
\multicolumn{1}{|c|}{}                                                                        &                                   & \cellcolor[HTML]{FFFFFF}\textbf{Repetition} & \cellcolor[HTML]{FFFFFF}\textbf{80,7\%} & \cellcolor[HTML]{FFFFFF}\textbf{79,6\%} & \cellcolor[HTML]{FFFFFF}\textbf{81,8\%} \\ \cline{3-6} 
\multicolumn{1}{|c|}{}                                                                        & \multirow{-2}{*}{\textbf{SPREAD}} & \cellcolor[HTML]{F9B0E1}\textbf{Subject}    & \cellcolor[HTML]{F9B0E1}\textbf{81,8\%} & \cellcolor[HTML]{F9B0E1}\textbf{81,8\%} & \cellcolor[HTML]{F9B0E1}\textbf{81,8\%} \\ \cline{2-6} 
\multicolumn{1}{|c|}{}                                                                        &                                   & \cellcolor[HTML]{FFFFFF}Repetition          & \cellcolor[HTML]{FFFFFF}68,1\%          & \cellcolor[HTML]{FFFFFF}80,7\%          & \cellcolor[HTML]{FFFFFF}55,9\%          \\ \cline{3-6} 
\multicolumn{1}{|c|}{}                                                                        & \multirow{-2}{*}{KISS}            & \cellcolor[HTML]{F9B0E1}Subject             & \cellcolor[HTML]{F9B0E1}68,1\%          & \cellcolor[HTML]{F9B0E1}81,8\%          & \cellcolor[HTML]{F9B0E1}54,5\%          \\ \cline{2-6} 
\multicolumn{1}{|c|}{}                                                                        &                                   & \cellcolor[HTML]{FFFFFF}Repetition          & \cellcolor[HTML]{FFFFFF}77,0\%          & \cellcolor[HTML]{FFFFFF}78,1\%          & \cellcolor[HTML]{FFFFFF}75,9\%          \\ \cline{3-6} 
\multicolumn{1}{|c|}{}                                                                        & \multirow{-2}{*}{OPEN}            & \cellcolor[HTML]{F9B0E1}Subject             & \cellcolor[HTML]{F9B0E1}81,8\%          & \cellcolor[HTML]{F9B0E1}81,8\%          & \cellcolor[HTML]{F9B0E1}81,8\%          \\ \cline{2-6} 
\multicolumn{1}{|c|}{}                                                                        &                                   & \cellcolor[HTML]{FFFFFF}Repetition          & \cellcolor[HTML]{FFFFFF}37,1\%          & \cellcolor[HTML]{FFFFFF}51,2\%          & \cellcolor[HTML]{FFFFFF}19,3\%          \\ \cline{3-6} 
\multicolumn{1}{|c|}{\multirow{-8}{*}{\begin{tabular}[c]{@{}c@{}}Non-\\ speech\end{tabular}}} & \multirow{-2}{*}{BLOW}            & \cellcolor[HTML]{F9B0E1}Subject             & \cellcolor[HTML]{F9B0E1}38,4\%          & \cellcolor[HTML]{F9B0E1}57,1\%          & \cellcolor[HTML]{F9B0E1}16,6\%          \\ \hline
\multicolumn{1}{|c|}{}                                                                        &                                   & \cellcolor[HTML]{FFFFFF}Repetition          & \cellcolor[HTML]{FFFFFF}49,0\%          & \cellcolor[HTML]{FFFFFF}63,0\%          & \cellcolor[HTML]{FFFFFF}32,6\%          \\ \cline{3-6} 
\multicolumn{1}{|c|}{}                                                                        & \multirow{-2}{*}{BBP}             & \cellcolor[HTML]{F9B0E1}Subject             & \cellcolor[HTML]{F9B0E1}50,0\%          & \cellcolor[HTML]{F9B0E1}63,6\%          & \cellcolor[HTML]{F9B0E1}33,3\%          \\ \cline{2-6} 
\multicolumn{1}{|c|}{}                                                                        &                                   & \cellcolor[HTML]{FFFFFF}Repetition          & \cellcolor[HTML]{FFFFFF}64,2\%          & \cellcolor[HTML]{FFFFFF}64,5\%          & \cellcolor[HTML]{FFFFFF}64,0\%          \\ \cline{3-6} 
\multicolumn{1}{|c|}{}                                                                        & \multirow{-2}{*}{PA}              & \cellcolor[HTML]{F9B0E1}Subject             & \cellcolor[HTML]{F9B0E1}57,1\%          & \cellcolor[HTML]{F9B0E1}54,5\%          & \cellcolor[HTML]{F9B0E1}60,0\%          \\ \cline{2-6} 
\multicolumn{1}{|c|}{}                                                                        &                                   & \cellcolor[HTML]{FFFFFF}Repetition          & \cellcolor[HTML]{FFFFFF}67,3\%          & \cellcolor[HTML]{FFFFFF}65,7\%          & \cellcolor[HTML]{FFFFFF}69,3\%          \\ \cline{3-6} 
\multicolumn{1}{|c|}{\multirow{-6}{*}{Speech}}                                                & \multirow{-2}{*}{PATAKA}          & \cellcolor[HTML]{F9B0E1}Subject             & \cellcolor[HTML]{F9B0E1}66,6\%          & \cellcolor[HTML]{F9B0E1}63,6\%          & \cellcolor[HTML]{F9B0E1}70,0\%          \\ \hline
\end{tabular}}
\end{table}

According to previous studies, the SPREAD subtask also appears to be the most discriminative one, with an accuracy of 80.7\% during repetition-based classification and 81.8\% concerning the subject-based classification in our model approach. As described in Section~\ref{s.methodology}, the experiments were conducted by first splitting the dataset into training and test folds. The former was partitioned into a smaller training set to generate a validation fold, whose size was limited to the data available for training. 

We anticipate that the results obtained using SVM and Logistic Regression may differ significantly from the findings presented by Bandini et al.~\cite{bandini2018automatic}, for they employed a slightly different approach. Although Toronto Neuroface contains the same speech and non-speech tasks as those in the study conducted by Bandini et al.~\cite{bandini2018automatic}, our approach has several differences. Firstly, the participants in the dataset we had access to were not the same, and there were variations in terms of quantity. Moreover, we manually cropped the frames containing repetitions, for only the entire video was made available. Additionally, we did not have access to the videos containing samples from the REST subtask, which was used for normalization in both SVM and Regression models. Furthermore, our videos only included color information and did not incorporate three-dimensional depth features.

Figure~\ref{f.non_speech} compares FPG against the baselines inspired in Bandini et al.~\cite{bandini2018automatic} work. One can observe that our model consistently outperforms others in the majority of tasks, e.g., SPREAD, KISS, PA, and PATAKA. However, SVM-RBF stands out as the top-performing model in the BLOW subtask. However, SVM-RBF stands out as the top-performing model in the BLOW subtask, which was the most challenging, as also observed by Bandini et al.~\cite{bandini2018automatic}.

% Figure environment removed


% The experimental results were obtained for each subtask separately. As evaluation metrics of our model's performance, accuracy (the proportion of samples correctly classified out of the total samples), sensitivity (the proportion of ALS samples correctly classified as ALS), and specificity (the proportion of HC samples correctly classified as HC) were employed Table~\ref{t.results}.

% In this regard, the experimentation was conducted by dividing the dataset into three parts: two individuals for validation, one for testing, and the remaining data for training the GAT model. 

% It is worth noting that the number of experiments conducted would equal the number of patients for each subtask, meaning each subject would participate in the test once. 

% In this sense, as evaluation metrics for the model, accuracy (the proportion of samples correctly classified out of the total samples), sensitivity (the proportion of ALS samples correctly classified as ALS), and specificity (the proportion of HC samples correctly classified as HC) were employed as described in Table~\ref{t.results}. These metrics provide a refined assessment, determining the usefulness and comprehensiveness of the classifier.



