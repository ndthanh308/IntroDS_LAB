\section{Discussion and Conclusion}
\label{s.discussion_conclusion}

To the best of our knowledge, the current study is the first to evaluate Graph Neural Networks for ALS identification based on facial expression. As the main finding, we showed state-of-the-art results in all subtasks of the Toronto Neuroface dataset but one.

%Although we don't have directly related work as a benchmark of ALS classification in this dataset to compare with our model's evaluation, using a similar approach as Bandini et al.\cite{bandini2018automatic} approach, we achieve relatively better results given to the model's high degree of freedom, using just coordinates of some geometric points of the face, without handcrafted calculation.

The two highest accuracies are observed in SPREAD and OPEN subtasks, achieving results above 80\%. We can observe similar values for the specificity and sensitivity in both subtasks, showing the model's robustness in distinguishing ALS patients from healthy ones.

% As duas melhores acurácias obtidas foram na classificação das tasks SPREAD e OPEN ambas alcançando resultados superiores a 80\%. É válido notar que, a especificidade e sensitividade tiveram valores semelhantes em ambas as tasks. Isto demonstra a criação de um modelo robusto capaz de diferenciar em aproximadamente 82\% dos pacientes com ALS aqueles que realmente tinham a doença. Da mesma forma ocorreu para o grupo de controle com base nas duas tasks.

We attribute the high accuracy in the SPREAD task to the pure lip movement not involving the jaw muscles~\cite{bandini2018automatic}, allowing the detection of the loss of lip muscle extension exhibited by bulbar ALS patients. Additionally, as shown in previous studies, the jaw muscles decline in bulbar ALS patients~\cite{bandini2018automatic}. Consequently, the extension of this movement was distinguished with high accuracy by the model during the OPEN task. OPEN considers the greatest extent of jaw muscle movement among all other tasks, justifying the model's accuracy.

% A acurácia elevada na classificação da task SPREAD deve-se ao movimento puramente labial não conduzido pela musculatura da mandíbula~\cite{bandini2018automatic}. Isto permitiu a detecção da perda da extensão do movimento da musculatura labial tida pelos pacientes com ALS bulbar. Além disso, a musculatura da mandíbula também decai para pacientes com ALS do tipo bulbar~\cite{bandini2018automatic}. Neste sentido, a extensão deste movimento foi diferenciada com uma acurácia elevada pelo modelo durante a task OPEN. Esta, entre todas as outras tasks, é a que considera a maior extensão de movimento dos músculos da mandíbula o que justifica a acurácia alcançada pelo modelo.


The exchange of information among the graph nodes during the learning iterations allowed for better differentiation of facial points between individuals with ALS or HC. It is also noteworthy that, except for PA and PATAKA tasks, the model showed inferior or equal results in repetition classification compared to subject-base classification, indicating that most repetitions were correctly classified, as the mode of labeled repetitions ended in the correct classification of the subject.

% Da maneira como foi dividido o dataset, os métodos de machine learning de separação linear obtiveram, no geral, resultados inferiores aos do modelo proposto. A troca de informação entre os nós do grafo durante as iterações da GNN permitiu criar uma melhor diferenciação dos pontos da face de indíviduos com ALS ou HC. Pode-se observar ainda, que com exceção da task PA e PATAKA o modelo apresentou a classificação da repetição com resultados inferiores ou iguais aos obtidos na classificação do sujeito. Este fato, mostra que a maioria das repetitions foram classificadas corretamente, visto que a moda das repetições resultaram na classificação correta do sujeito.


One of the major limitations and challenges in training deep learning models is the limited number of videos available in the dataset. Deep models typically require a substantial amount of data to learn effectively. However, FPG showcased exceptional performance despite being a deep approach. Remarkably, it achieved high accuracy without data augmentation during training.

Such outcomes highlight the effectiveness of GNN models, showcasing their inherent structural characteristics and information propagation capabilities. GNNs demonstrate their ability to capture complex patterns and relationships within the data, even when dealing with a limited dataset, underscoring GNNs as a powerful approach in this particular domain.

% A quantidade diminuta de vídeos no banco de dados dificulta o aprendizado de modelos profundos. Modelos profundos são conhecidos por utilizarem extensas quantidades de dados. Entretanto, o modelo construído com GNN, mesmo sendo considerado profundo, alcançou alta acurácia sem a necessidade augmentation dos dados de treinamento.

This study did not consider the order of repetitions. Therefore, exploring temporal information in the Facial Point Graph as a future work would be interesting, particularly the changes observed in facial movements in the presence of neurodegenerative diseases. We shall further investigate the impact of fatigue found in ALS patients during speech tasks like 'PA' and 'PATAKA' and potentially improve the model's performance in capturing these variations.

% Nesse trabalho não foi considerado a ordem das repeticoes, sendo assim como trabalho futuro explorar  o approach Temporal Facial Point Graph, considerando as caracteristicas temporal, principalmente levando em consideração a mudança causada nos movimentos faciais na presença de doenças neurodegenerativas,como por exemplo a fadiga apresentapor alguns pacientes ALS durante a execução de determiandas subtasks devido a fadiga.


% "Despite a number of limitations, the overall results of this study are promising as they may lead to the development of a cheap, easy-to-use, and clinically feasible system that will support clinicians in the assessment and diagnosis of the bulbar form of ALS."

Despite the limitations and complexity of the problem, the proposed approach achieved significant results when compared to similar works, introducing the Facial Point Graph for ALS diagnosis. In addition, the results were achieved without handcrafted features and with a lightweight model, enabling the development of affordable systems capable of supporting clinicians in automatic ALS diagnosis.



