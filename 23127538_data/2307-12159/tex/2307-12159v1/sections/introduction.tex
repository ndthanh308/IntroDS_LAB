\section{Introduction}
\label{s.introducao}

A gradual decline in the structure and functioning of the central nervous system marks Neurodegenerative Diseases (NDDs). The incidence and prevalence of these diseases exhibit a sharp increase with age, which means that life expectancy continues to rise in many parts of the world. Consequently, the number of cases is projected to grow in the future~\cite{checkoway2011neurodegenerative}. Despite the availability of certain treatments that can relieve the physical or mental symptoms linked to neurodegenerative diseases, there is currently no known method to slow down their progression or achieve a complete cure. 

Amyotrophic lateral sclerosis (ALS) is an NDD that causes the gradual deterioration of motor functions of the nervous system. Worldwide, the annual incidence of ALS is about 1.9 per 100,000 inhabitants~\cite{arthur2016projected}. In addition, patients face a delay in disease diagnosis by approximately 18 months~\cite{bandini2018automatic} and an average survival of 2 to 4 years after diagnosis~\cite{xu2021considerations}. Since effective treatments are currently unavailable, early and precise diagnosis is crucial in maintaining patients' quality of life.

Evaluating the facial expression of people is one effective way to diagnose neurodegenerative diseases, for the subject may lose a significant amount of verbal communication ability~\cite{yolcu2019facial}. It is worth noting that all types of NDDs affect the oro-facial musculature\footnote{Musculature related to communication and critical to functions such as chewing, swallowing, and breathing.} with significant impairments in speech, swallowing, and oro-motor skills, as well as emotion expression~\cite{bandini2020new}. Therefore, analyzing a patient's facial expression in an image or video can be crucial for diagnosing ALS.

The geometry-based characteristics derived from an individual's face describe the shape of its components, such as the eyes or mouth, which are very important for facial analysis~\cite{wu2019facial}. Based on these landmarks, Bandini et al.~\cite{bandini2018automatic} proposed an approach that predicts the patient's healthy state based on features representing motion, asymmetry, and face shape through video analysis. Such an inference was accomplished using well-known machine learning techniques, i.e., Support Vector Machines (SVM)~\cite{Cortes:95} and Logistic Regression~\cite{wright1995logistic}. Although reasonable results have been reported, there is still the need to deal with handcrafted features. Our work circumvents such a shortcoming by introducing Facial Point Graphs (FPGs) to learn motion information from facial expressions automatically. Our model is based on Graph Neural Networks (GNNs) and first constructs a graph with the most important facial points for ALS diagnosis to fulfill that purpose for further training. Later, each frame is classified as positive or negative to the disease. The majority voting then assigns the final label to individual.

As far as we know, no method employs Facial Point Graphs for ALS identification. We firmly believe that the landmarks extracted from frames can be better encoded in a non-Euclidean space, enabling the precise definition and representation of their distinct features. Therefore, the main contributions of this paper are twofold:
\begin{itemize}
    \item To introduce Facial Point Graphs to identify ALS.
    \item To employ a deep learning approach to the same context, thus not requiring handcrafted features.
\end{itemize}

%Graphs are structures that can represent abstract concepts, such as interactions between users in social networks, the connection between websites on the internet, molecules, or even important points for an individual's facial expression. In this sense, the method utilized in this paper does the graph analysis and diagnosis of patients by capturing facial landmarks during non-speech and speech actions. Notably, the mouth region holds significant diagnostic value for ALS~\cite{bandini2018automatic}, as it is often affected by motor movement loss in patients. Consequently, we constructed a graph specifically for this region and employed Graph Neural Networks (GNNs)~\cite{scarselli2008graph} to classify the patients accordingly.

% In this paper, we propose a method that involves GNN to distinguish patients diagnosed with ALS from a control group of healthy individuals through facial expressions. The geometry-based characteristics extracted from the face can describe the shape of the face and its components, such as the eyes or mouth. We performed a  graph of computational structures to identify the features in facial expressions without pre-calculation. For that, our model first constructed a graph with the most important facial points to diagnose ALS in patient videos. As far as we know, no method utilizes facial point graphs for ALS identification.

The remainder of this paper is structured as follows: Sections~\ref{s.related_works} and~\ref{s.theoretical} present the literature review and theoretical background, respectively. Section~\ref{s.methodology} presents an explanation regarding the employed dataset, the used models to crop images and extract facial features, the proposed approach, and the classification method. Finally, Section~\ref{s.res} presents the experimental results and Section~\ref{s.discussion_conclusion} states the discussions about the results, conclusions, and future works.