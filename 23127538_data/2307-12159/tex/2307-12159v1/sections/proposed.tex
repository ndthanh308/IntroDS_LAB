% \section{Proposed model}
% \label{s.proposed}

% Initially, the proposed model uses fifteen equally spaced frames for each of the repetitions performed by the patients. In this context, it receives a graph of twenty-six nodes representing the face landmarks, where each node has a feature vector with two pieces of information: the x and y positions in the two-dimensional plane. In addition, each of the graph's edges stores its length determined by the Euclidean distance between the two connected points.

% In this regard, each of the fifteen frames proceeds through six layers of GAT and two linear layers. It is important to note that before the information enters the linear layers, pooling is performed using all the feature vectors of the graph, i.e., all the graph information contained in the nodes is transformed into a single vector. To achieve this, the average of the graph's feature vectors is calculated, as shown in Figure~\ref{f.proposed_model}.

% Eventually, the result obtained after pooling goes through two linear layers, which generate the model's output. It is worth noting that the error is calculated based on the class of that graph (frame) and the results are reached for the Repetition classification by identifying the mode among the frames used. In other words, classifying an individual's repetition is based on the majority consensus among the frames, as represented in Figure~\ref{f.evaluation}. Likewise, when classifying the subject, the majority mode derived from the classifications of each repetition determines whether the patient has ALS or not.


