\documentclass[prd,aps,a4,onecolumn,superscriptaddress,preprintnumbers,nofootinbib]{revtex4-1}
% ---------------------
% Load special packages
% ---------------------

\usepackage{pslatex}
\usepackage[pdftex]{graphicx}
\usepackage{psfrag}
\usepackage{epsfig}
\usepackage{color}
\usepackage{cancel}
\usepackage{slashed}
\usepackage{amssymb}
\usepackage{amsmath}
\usepackage{hyperref}
\usepackage{enumerate}
\usepackage{multirow}
\usepackage{ulem}
\usepackage{float}
\usepackage{comment}
\usepackage{diagbox}
\usepackage{amsmath}
\usepackage{soul, xcolor}
\usepackage{makecell}
% ---------------------
% New commands
% ---------------------
\newcommand\subsetsim{\mathrel{\substack{
  \textstyle\subset\\[-0.2ex]\textstyle\sim}}}
  
\newcommand{\dk}[1]{\textcolor{blue}{#1}}
\newcommand{\dkc}[1]{\textbf{\textcolor{blue}{(#1 --DK)}}}

\newcommand{\LS}[1]{{\textcolor{red}{[LS: #1]}}}

\newcommand{\YZ}[1]{{\textcolor{blue}{[YZ: #1]}}}

\newcommand{\LJ}[1]{{\textcolor{purple}{[LJ: #1]}}}

%%%%%%%%%%%%%%%%%%%%%
% to get script r
\usepackage{calligra}
\DeclareMathAlphabet{\mathcalligra}{T1}{calligra}{m}{n}
\DeclareFontShape{T1}{calligra}{m}{n}{<->s*[2.2]callig15}{}
\newcommand{\scriptr}{\mathcalligra{r}\,}
%%%%%%%%%%%%%%%%%%%%%


\def\bse{\begin{eqnarray*}}
	\def\ese{\end{eqnarray*}}
\def\pr{\hbox{pr}}
\def\wh{\widehat}
\def\D{\mathcal{D}}
\def\T{\mathcal{T}}
% --------------------------
% Load bibliography style
% --------------------------
\bibliographystyle{apsrev}
\allowdisplaybreaks
%\begin{document}

\begin{document}
\setstcolor{red}
%\begin{flushright}
%MI-TH-XXX
%\end{flushright}

\title{Detection of astrophysical neutrinos at prospective locations of dark matter detectors}

\author{Yi Zhuang}
\affiliation{Department of Physics and Astronomy, Mitchell Institute
  for Fundamental Physics and Astronomy, Texas A\&M University,
  College Station, Texas 77843, USA}
\author{Louis E. Strigari}%
%\email{strigari@tamu.edu}%
\affiliation{Department of Physics and Astronomy, Mitchell Institute
  for Fundamental Physics and Astronomy, Texas A\&M University,
  College Station, Texas 77843, USA}

\author{Lei Jin}%
\affiliation{Department of Mathematics and Statistics, Texas A\&M University-Corpus Christi,  Corpus Christi,  TX 78412, USA}

\author{Samiran Sinha}%
\affiliation{Department of Statistics, Texas A\&M University,
  College Station, Texas 77843, USA}
  
\date{\today}
\begin{abstract}
We study the prospects for detection of solar and atmospheric neutrino fluxes at future large-scale dark matter detectors through both electron and nuclear recoils. We specifically examine how the detection prospects change for several prospective detector locations (SURF, SNOlab, Gran Sasso, CJPL, and Kamioka), and improve upon the statistical methodologies used in previous studies. Due to its ability to measure lower neutrino energies than other locations, we find that the best prospects for the atmospheric neutrino flux are at the SURF location, while the prospects are weakest at CJPL because it is restricted to higher neutrino energies. On the contrary, the prospects for the diffuse supernova neutrino background (DSNB) are best at CJPL, due largely to the reduced atmospheric neutrino background at this location. Including full detector resolution and efficiency models, the CNO component of the solar flux is detectable via the electron recoil channel with exposures of $\sim 10^3$ ton-yr for all locations. These results highlight the benefits for employing two detector locations, one at high and one at low latitude. 
\end{abstract}


\keywords{Direct Dark Matter Detection}

\maketitle

\section{Introduction \label{sec:introduction}}

\par Over the past several decades, direct dark matter detection experiments have made tremendous progress in constraining weak-scale particle dark matter~\cite{XENON:2018voc,LUX-ZEPLIN:2022qhg}. Future larger-scale detectors will be sensitive to not only particle dark matter, but also astrophysical neutrinos and various other rare-event phenomenology~\cite{Aalbers:2022dzr}. The most prominent of the neutrino signals are from the Sun, the atmosphere, and the diffuse supernova neutrino background (DSNB)~\cite{Billard:2013qya}. Understanding these signals has important implications for the future of particle dark matter searches, and also for understanding the nature of the sources and the properties of neutrinos~\cite{Dutta:2019oaj}. 

\par Various methods have been proposed to distinguish neutrinos and a possible dark matter signal. These include exploiting the energy distribution of nuclear recoils between neutrinos and dark matter~\cite{Dent:2016iht,Dent:2016wor}, the differences in arrival directions~\cite{OHare:2015utx}, and the differences in the periodicities of the signal~\cite{Davis:2014ama}. New physics in the neutrino sector may also change the nature of the predicted neutrino signal~\cite{Cerdeno:2016sfi,AristizabalSierra:2019ykk}, and provide a method to discriminate from dark matter. 

\par In this paper, we examine the prospects for detecting all of these neutrino flux components at large-scale, next generation detectors. We consider detection through both the electron recoil and the nuclear recoil channels. We present a principled statistical methodology for extracting all flux components, and compare it to to previous methods that attempted to extract some of the flux components that we consider~\cite{2019Jayden,2021Jayden}. We mainly focus on how the detection prospects for all flux components depend on detector location, considering 5 specific detectors locations: China Jinping Underground Laboratory (CJPL), Kamioka, Laboratori Nazionali del Gran Sasso (LNGS), SNOlab and the Sanford Underground Research Facility (SURF). The location of the detector is crucial because the atmospheric neutrino flux depends strongly on the detector's latitude and longitude. Including this effect is important not only for detecting the atmospheric neutrino flux itself but also for other subdominant components, such as the DSNB, for which the atmospheric neutrino component is a background. 

\par This paper is organized as follows. In Section~\ref{sec:SandB}, we describe the nature of the signals and backgrounds that we use in our analysis. In Section~\ref{sec:NEST}, we describe the simulations of detector properties to interpret the signals. In Section~\ref{sec:statistical}, we introduce the statistical methodologies used in our analysis, and compare to previous analysis methods. Then in Section~\ref{sec:results}, we present our resulting projections, and in Section~\ref{sec:conclusions}, the discussion and conclusions. 

\section{Signal and Backgrounds \label{sec:SandB}}
\par Figures~\ref{fig:recoil_pdf_cdf_ER} and   ~\ref{fig:recoil_pdf_cdf_NR} show the electron and nuclear recoil spectra, respectively, for the solar, atmospheric, and DSNB spectra. We show results for both for Xenon and Argon targets, which are the most likely target nuclei for large scale detectors with size $\sim 10-100$ ton. The nuclear recoil spectrum uses the neutral current coherent elastic neutrino-nucleus scattering (CE$\nu$NS) channel, and the electron recoil channel uses charged current neutrino-electron elastic scattering (ES). The differential cross section and minimum neutrino energies are 
\begin{equation}
\begin{split}
    \frac{d\sigma (E_r, E_{\nu})}{dE_r} &=\left\{
    \begin{array}{cr}
     \frac{G_f^2 m_e}{2\pi}\sum_{\nu_{\mu}, \nu_e} 
 \left[(c_v + c_a)^2 + (c_v - c_a)^2 \left(1-\frac{E_r}{E_{\nu}}\right)^2 + ({c_a}^2 - {c_v}^2) \frac{m_e E_r}{E_{\nu}^2}\right] & \textrm{ES}\\
     \frac{G_f^2}{4\pi} Q^2 m_N  \left(1- \frac{m_N  E_r}{2E_{\nu}^2}\right) F^2(E_r) & \textrm{CE$\nu$NS}
    \end{array} \right.\\
    E_{\nu,min} &=\left\{
    \begin{array}{cr}
     \frac{1}{2}\left[E_r +\sqrt{E_r (E_r+2m_e)}\right]&\mbox{ES}\\
     \sqrt{\frac{m_N E_r}{2}}&\mbox{CE$\nu$NS}
    \end{array}\right.
\end{split}
%\label{eqn:cross_section}
\nonumber 
\end{equation}
where $E_r$ is the electron or nuclear recoil kinetic energy and $m_N$ is the mass of the nucleon. Assuming pure Standard Model interactions, for ES, $g_v = 2 \sin^2\theta_w-1/2$, $g_a = -1/2$. For $\nu_{\mu}$, $c_v = g_v$, $c_a = g_a$. For $\nu_e$, $c_v = g_v + 1$ and $c_a = g_a + 1$. For  CE$\nu$NS, \mbox{$Q = N - (1-4\sin^2\theta_w)Z$} and $\sin^2 \theta_w = 0.231$~\cite{ParticleDataGroup:2018ovx}, where $Z$ is the number of protons, $N$ is the number of neutrons, and the mass number is $A = N + Z$. We take $F(E_r)$ to be modelled by the Helm form factor~\citep{Lewin:1995rx}. Defining the neutrino flux as $d \phi /d E_\nu$, the rates are then
\begin{equation}
\frac{dR_{\nu}(E_r)}{dE_r} = 
\frac{N_A}{A} \int_{E_{\nu,min}} dE_{\nu} \frac{d\phi}{dE_{\nu}}\frac{d\sigma (E_r, E_{\nu})}{dE_r}
\nonumber 
\end{equation}
with \mbox{$N_A$/A} being number of nuclei per mass of the target nuclei. 

\par We use solar neutrino flux normalizations from GS98-SFII (high metallicity) and AGSS09-SFII (low metallicity)~\citep{Haxton:2012wfz}. For the electron recoil component, we apply the electron binding energy correction which adds the step-like features at the lowest recoil energies~\citep{2017Chen}. Also shown in the electron channel are the projections for the $^{85}$Kr, $^{222}$Rn, and 2$\nu\beta\beta$ backgrounds~\cite{Franco:2015pha,2019Jayden}. 


% Figure environment removed

% Figure environment removed

\section{NEST simulation and Detector Efficiency Models}
\label{sec:NEST}
\par To simulate the detection of the signal in Xenon time-projection chambers (TPC), we use the Noble Element Simulation Technique (NEST)~\citep{2011NEST} code. Neutrinos (or dark matter particles) interact with the gas in the detector, producing a scintillation signal, $S1$, and ionization electrons, which then drift along the electric field to produce a signal, $S2$. The NEST code simulates the detection of events in the space of $S1$ and $S2$. For the NEST configuration, we choose all enhanced parameters, similar to previous studies ~\citep{2021Jayden,Tang:2023xub}. For the detector geometry, we take the radius to be $1300$ mm, and the $z$ position to range from $75.8$ mm to $1536.5$ mm. 

\par For comparison to the analysis in $S1/S2$ space, we will perform an analysis directly in electron and nuclear recoil space.  With the efficiency modeled as a function of recoil energy, the modified event rate is 
\begin{equation}
    \frac{dR}{dE_{r}} = \epsilon (E_{r}) \int dE_{r}^{'} \frac{dR(E_{r}^{'})}{dE_{r}^{'}}
     \frac{1}{\sqrt{2\pi \sigma^{2} (E_{r}^{'})} } e^{-\frac{(E_{r} - E_{r}^{'})^2}{ 2\sigma ^{2}  (E_{r}^{'})}}, 
     \nonumber 
    %\label{eqn: dNdEr_sigma}
\end{equation}
where $E_{r}^{'}$ is the true recoil energy, $E_{r}$ is the detected recoil energy, $\sigma(E_{r}^{'})$ is the resolution at the true recoil energy, and $\epsilon (E_{r})$ is the detector efficiency. 

\par For Xenon and Argon, we consider four models  defined by how the detector efficiency and energy resolution are modeled. These models are defined as follows: 
\begin{enumerate}
    \item Ideal Xenon: Characterized by perfect efficiency and energy resolution. 
    \item Xenon $S1/S2$: Full analysis in $S1/S2$ space for Xenon, including efficiency and energy resolution corrections via the NEST simulations. Due to the minimum work function in the NEST setup (13.469 eV) to excite the atom, nuclear recoils below this energy do not generate $S1/S2$ signals. So the solar CE$\nu$NS components with neutrino energies at this scale do not contribute to the rate in this scenario. 
    \item Ideal Argon: Characterized by perfect efficiency and energy resolution.  
    \item Argon resolution + efficiency: Characterized by an energy resolution of 
    \mbox{$\sigma(E_{r}) = 0.1E_{r}$}. The nuclear recoil efficiency is taken from DarkSide-50~\citep{2016PhRvD..93h1101A}. For electron recoils, we simply assume a threshold energy of $E_{th} = 1$ keV.   
\end{enumerate}

\section{Statistical Methods \label{sec:statistical}}
\par In this section, we establish our statistical methods for the detection of solar and atmospheric neutrinos. For related previous analysis, see Refs.~\citep{OHare:2021utq,2014Ruppin} based on the statistical methodology in Ref.~\citep{2011Cowan}. As described above, there are two components to our analysis. The first involves an Argon-based and idealized Xenon-based analysis in the nuclear and electron recoil energy ($E_r$) space, and the second involves a Xenon-based analysis in the $S1/S2$ space. 

\par We start with construction of the function applicable to both $E_r$ and $S1/S2$ space. Let $\Omega$ represent the entire domain of interest, which can exist in one-dimensional, two-dimensional, or ultimately event higher-dimensional space, depending on the observable. For instance, in the analysis in recoil energy space in Argon and Xenon, $\Omega$ is one-dimensional, and in the analysis of Xenon in $S1/S2$ space, $\Omega$ is two-dimensional. To detect the presence of a given neutrino flux component, multiple non-overlapping subdomains $\Omega_i$, $i=1,2,\cdots, M$ are selected from $\Omega$. Let the detector have a volume of $\D$ and a run-time of $\T$. In the $i$-th subdomain $\Omega_i$, let $\eta_i^\kappa$ be the mean event rate for a flux component, measured in units ton$^{-1}$ yr$^{-1}$, so that the expected number of events for all components is 
\begin{equation}
\mu_{i} = \D\T \times \sum_{\kappa}\eta_i^\kappa. 
\label{eq:events}
\end{equation}
Defining the observed number of events as $n_i$ at $\Omega_i$, the likelihood for the observed number of events over $M$ subdomains is
\begin{equation}
               \mathcal{L} = \prod_{i=1}^{M} \frac{\exp(-\mu_{i} )\mu_{i} ^{n_{i}} }{n_i!},
   \nonumber% \label{eq:poissonS1S2} 
\end{equation} 
where the observations across the subdomains are assumed to be independent. 


\subsection{Analysis in E$_{r}$ and S1$/$S2 space }

\par We start with an analysis directly in $E_r$ space for the detected electron and nucleus. We use this $E_r$ space for the idealized Xenon, idealized Argon, and Argon resolution+efficiency analysis methods. In this case, the domain is one-dimensional, and the $i$-th subdomain $\Omega_i$ corresponds to an energy range in the one-dimensional $E_r$ space. The event rate in the $\imath^{th}$ subdomain from the $\kappa^{th}$ flux component is 
\begin{equation}
\eta_i^\kappa = \int dE_r \frac{dR_{i}^\kappa}{dE_r} 
\nonumber
\end{equation}
where the integral is over the energy range covered by the $\imath ^{th}$ subdomain. The expected number of events is then derived using Equation~\ref{eq:events}.  

\par We then move on to a Xenon analysis in $S1/S2$ space. In this case, the domain $\Omega$ is two-dimensional, and has total $N_{S1}$ S1 bins and $N_{S2}$ S2 bins (in log$_{10}$ space). We construct a likelihood function in terms of the detector-related $S1/S2$ variables. For the $\kappa^{th}$ flux component, we can schematically represent the event rate in the $i$-th subdomain $\Omega_i$ in the two-dimensional space as 
\begin{equation}
    \eta_{i}^\kappa = \int d S1 d S2 \frac{d^2 R_{i}^\kappa}{d S1 d S2} {\rm pdf} (S1,S2 |E_r).  
    \label{eq:dnds1ds2} 
\end{equation}
In Equation~\ref{eq:dnds1ds2},  
 ${\rm pdf} (S1,S2 |E_r)$ is the probability density function for $S1$ and $S2$ for a given $E_r$.  The pdf converts a recoil energy spectrum to $S1/S2$ space after the interaction process in the detector, and is calculated from the NEST simulations described above. To determine the pdf for each component, $10^7$ energy depositions are simulated and converted to $S1/S2$ signals. Since each component has its own recoil spectrum, this leads to a unique event rate in the subdomain $\Omega_i$ of $S1/S2$ space.

\subsection{Test of significance}
\label{sec:TestSig}

\par Our goal will be to detect the solar, atmospheric, and DSNB flux components. To achieve this, we define components of the flux as the background to be contained in the null hypothesis, $H_0$. We then add a flux component $\eta_{i}^\gamma$ on top of the background in the null to define the alternative hypothesis, $H_1$. As an example, if we are attempting to detect hep neutrinos via nuclear recoils, the null hypothesis $H_0$ includes $^{8}$B, DSNB, and atmospheric neutrinos. The alternative hypothesis $H_1$ includes $^{8}$B, DSNB, atmospheric, and hep neutrinos.

\par Now define the total number of events over all subdomains of interest or being selected as $N=\sum_{i}n_{i}$; here index $i$ is used to denote different sub-regions. 
Then the expectation of the random variable $N$ is 
\begin{equation}\label{test:simple}
    E(N) = \left\{
    \begin{array}{lr}
     {\cal D}{\cal T} \sum_{i} \sum_{\kappa} \eta_{i}^\kappa  &\mbox{under }H_0,\\
      {\cal D}{\cal T}  \sum_{i} (\sum_{\kappa} \eta_{i}^\kappa +\eta_{i}^\gamma) &\mbox{under }H_1. 
    \end{array}\right.
\end{equation}
The goal of the following exercise is to find the exposure $\D\T$ that ensures  $H_0$  is rejected when $H_1$ holds with a desired probability while the test's significance level is set to $\alpha$. 
Moreover, here we assume that $\eta^\kappa_i$
and $\eta^\gamma_i$ are known for all $i$ and $\kappa$. The index $\kappa$ is used to to denote different background components. The level or the significance level $\alpha$ denotes the probability of Type-I error, and Type-I error signifies rejection of $H_0$ when $H_0$ holds. In the context of this 
problem, the Type-I error signifies the situation where based on the data and statistical test, we declare the presence of a particular neutrino signal $\eta^\gamma_i$, but in reality there were no such signals other than the background noise.
On the other hand, the probability of a type-II error denoted by $\beta$ signifies the probability of failing to reject $H_0$ when $H_1$ holds. In the context of this problem, Type-II error signifies the situation where based on the data and statistical test, we failed to find the presence of a neutrino signal $\eta^\gamma_i$, but in reality, it was present along with the background noise. The power of the test is one minus the probability of Type-II error (i.e., power$=1-\beta$). The probabilities of a Type-I error and a Type-II error are in inverse relation. Usually, we want a large power of a test, say 90\%; to obtain 90\% power, we set  $\beta=0.1$.  On the other hand, we set the probability of Type-I error, $\alpha$ to a small number. 

Given that $\eta_{i}^\gamma \geq 0$ for all 
$i$, the most powerful test for testing the simple null against the simple alternative as given in Equation~\ref{test:simple} at level $\alpha$ is reject $H_0$ if $N>N_\alpha=N_\alpha({\cal D}{\cal T})$, the smallest integer of the set 
\begin{equation}\nonumber\label{eq:foralpha} 
\left[\scriptr:  \pr \{N> \scriptr \mbox{ where }N\sim {\rm Poisson}({\cal D}{\cal T} \sum_{i} \sum_{\kappa} \eta_{i}^\kappa) \}\leq \alpha\right]. 
\end{equation}
This form of the test is obtained by using the Neyman-Pearson lemma and the monotone likelihood ratio property of the Poisson distribution \citep[p.~388, 391]{CB}. Note that there is no closed form analytical expression for $N_\alpha$ in terms of ${\cal D}{\cal T} \sum_{i} \sum_{\kappa} \eta_{i}^\kappa$, so it must be computed numerically. However, there is no need to do a simulation to compute $N_\alpha$. Moreover, this test is uniformly the most powerful test for testing the null $H_0: E(N)= \D\T\sum_i\sum_\kappa \eta^\kappa_i$ against the alternative $H_1: E(N)> \D\T\sum_i\sum_\kappa \eta^\kappa_i$. For a minimum power of $1-\beta$ for the alternative $E(N)= \D\T\sum_i(\sum_\kappa \eta^\kappa_i+\eta^\gamma_i)$, the desired ${\cal D}{\cal T}$ is 
\bse 
{\inf}\left[ {\cal D}{\cal T}:   \pr \left\{N> N_{\alpha} \mbox{ where }N\sim {\rm Poisson}\left({\cal D}{\cal T}  \sum_{i} (\sum_{\kappa} \eta_{i}^\kappa +\eta_{i}^\gamma)\right) \right\}
\geq  (1-\beta) 
\right].
\ese 
To find this exposure $\D\T$, we first fix $\alpha$ and $(1-\beta)$, then for a grid of values of $\D\T$, find $N_\alpha$. For each $\D\T$ and the corresponding $N_\alpha$, we compute 
$$
\pr \left\{N> N_{\alpha} \mbox{ where }N\sim {\rm Poisson}\left(\D\T \sum_{i} (\sum_{\kappa} \eta_{i}^\kappa +\eta_{i}^\gamma)\right) \right\},
$$
and check if the above probability is at least $(1-\beta)$. The infimum of the set of $\D\T$ that yields the above probability to be at least $(1-\beta)$ is the desired exposure. 


\subsection{Choice of the subdomain} 

\par Given the nature of the distribution of the events in the subdomains, the null and alternative hypotheses may be difficult to distinguish, especially when a large portion or almost all of the domain with a very small signal is on top of a relatively large background. This is especially true for the CE$\nu$NS channel since the atmospheric, hep and DSNB event rates are low. It is less of a concern for the ES channel due to their relatively large event rates. For both cases, we wish to choose a set of subdomains that maximize the ability of the test to differentiate $H_0$ and $H_1$. With this motivation, we now discuss choices of subdomain for each of the analysis methods. One recommended principle for selecting subdomains is to enhance the signal-to-background noise ratio within the chosen subdomains when compared to using the entire domain without any selection while keeping sufficient exposure.

\subsubsection{Choice of two-dimensional subdomain in S1/S2 space for ES Xenon}\label{sec:S1/S2_ERbinning}

\par For the detection of the CNO and pep components via elastic scattering in Xenon, we divide into 30 $S1$ bins within the range $[2-12000]$, and 30 log$_{10}$S2 bins in the range $[2-6.9]$. For each component, the specific backgrounds contained in the null hypothesis are shown in Table~\ref{tab:ER_bg}. 

\begin{table}[!htbp]
\caption{Flux components $\eta^\kappa$ that are included in the null hypothesis, listed in the $\kappa$ column, when considering the detection of the CNO and pep fluxes in the electron scattering channel. This considers Xenon as the nuclear target and a full analysis in the S1/S2 space. The two rows of the second column indicate whether the background includes only $2\nu\beta\beta$ or all of $2\nu\beta\beta$, $^{85}$Kr and $^{222}$Rn. 
}
\begin{tabular}{c|c|c}
\hline
$\gamma$ & Detector backgrounds & Background components ($\kappa$)
\\
%\hline
\hline
\multirow{2}{*}{CNO} & $2\nu\beta\beta$ & pp, $^{7}$Be 861, $^{7}$Be 384, pep, $2\nu\beta\beta(f_{2\nu\beta\beta})$, $^{8}$B,   hep, DSNB, atm  \\
%
& all  & pp, $^{7}$Be 861, $^{7}$Be 384, pep, $2\nu\beta\beta(f_{2\nu\beta\beta})$,$^{85}$Kr, $^{222}$Rn, $^{8}$B, hep, DSNB, atm\\
%
\cline{1-3}
\multirow{2}{*}{pep} & $2\nu\beta\beta$  &pp, $^{7}$Be 861, $^{7}$Be 384, CNO, $2\nu\beta\beta(f_{2\nu\beta\beta})$, $^{8}$B,   hep, DSNB, atm \\
%
& all  & pp, $^{7}$Be 861, $^{7}$Be 384,  CNO, $2\nu\beta\beta(f_{2\nu\beta\beta})$, $^{85}$Kr, $^{222}$Rn, $^{8}$B, hep, DSNB, atm  \\
%
\hline
\end{tabular}
\label{tab:ER_bg}
\end{table}

\subsubsection{Choice of two-dimensional subdomain in S1/S2 space for CE$\nu$NS Xenon}
\label{sec:S1/S2_NRbinning}

\par For the detection of atmospheric, hep and DSNB components through the nuclear recoil channel, we take additional steps to define the optimal set of subdomains in $S1/S2$ space. This is because the backgrounds contained in the null hypothesis can be significant, and care must be taken to identify the signal. 

\par We start by dividing $S1/S2$ space into a large number of small regions, and then combine the small regions into a subdomain. More specifically, for a fixed $S1$, we combine the consecutive log$_{10}$S2 regions in which the signal is greater than the background into one rectangular log$_{10} S2$ subdomain. When scanning over the entire $S1/S2$ space, we keep only the subdomains with similar event rates so that the total rate summed over all subdomains is nearly uniform. We do not include bins with a very small event rate, corresponding to a threshold of signal $>$ 10$^{-4}$ ton$^{-1}$ yr$^{-1}$. 

\par As an example of this procedure, consider the detection of the atmospheric neutrino signal. In this case, the null hypothesis is given by the background components in Table~\ref{tab:optimized_NR_bg}. We simulate 10$^7$ $S1/S2$ events using NEST for each CE$\nu$NS and ES component. We then divide the events into many small S1/S2 grids, where number of the grids and the S1/S2 range are specifically shown in Table~\ref{tab:optimized_NR_bg}. This example of atmospheric neutrinos highlights the motivation for starting with small regions in log$_{10}$S2 space, which in this case is to avoid a significant contamination of events from leakage due to ES of pp solar neutrinos. The resulting set of subdomains for atmospheric signal are shown in Figure~\ref{fig:atmNu_method2_region} at each detector location. 

\begin{table}[!htbp]
\caption{Flux components $\eta^\kappa$ that are included in the null hypothesis, listed in the $\kappa$ column, when considering the detection of the atmospheric, hep and DSNB fluxes in the CE$\nu$NS channel at detector locations CJPL, Kamioka, LNGS, SURF, SNOlab. Xenon is the nuclear target and the analysis is in $S1/S2$ space. The last two columns indicate the $S1/S2$ range used to generate the grid and the number of grids. }
\begin{tabular}{c|c|cc|cc}
\hline
%\hline
$\gamma$ &  Background components ($\kappa$) & \# grid  & S1 range & \# grid & log$_{10}$S2 range\\
\hline
atm &  pp, $^{7}$Be 861, $^{7}$Be 384, CNO, pep, $2\nu\beta\beta$, $^{85}$Kr, $^{222}$Rn, $^{8}$B, hep, DSNB & 100  & $[1-401]$ & 100 & $[2-4.7]$ \\


hep  &  pp, $^{7}$Be 861, $^{7}$Be 384, CNO, pep, $2\nu\beta\beta$, $^{85}$Kr, $^{222}$Rn, $^{8}$B, DSNB,  atm & 10 & $[1-25]$ & 10 & $[1.9-4]$\\



DSNB &  pp, $^{7}$Be 861, $^{7}$Be 384, CNO, pep, $2\nu\beta\beta$,$^{85}$Kr, $^{222}$Rn, $^{8}$B, hep, atm& 90 & $[1-361]$ & 80 & $[1.9-4.6]$\\


\hline
\end{tabular}
\label{tab:optimized_NR_bg}
\end{table}

% Figure environment removed 


% Figure environment removed 

\par For the hep and DNSB components, we slightly adjust our method for determining the set of subdomains. This is because there is no region in the $S1/S2$ space over which these components are larger than the background components. In particular, for the DSNB, the atmospheric signal dominates over the DSNB where the latter is most prominent. Therefore, in choosing the set of subdomains to search for the DSNB signal, we follow the same steps as above, except that we exclude the atmospheric signal. Note that in the full analysis for the DSNB, we do include the atmospheric component, we only exclude it to define the subdomains. 

\par For hep, we follow a similar strategy as for the DSNB, in this case excluding the more dominant $^{8}$B signal over that region. Then for the full analysis, we include the $^{8}$B component. The full background models considered for DSNB and hep are shown in Table~\ref{tab:optimized_NR_bg}. The results for the subdomains are shown in Figure~\ref{fig:hepdsnb_method2_region}. 

\subsubsection{Choice of one-dimensional subdomain in recoil energy space for ES and CE$\nu$NS}\label{sec:Er_ERbinning}

\par For the analyses in recoil energy space, for ES, the background components used for the detection of CNO and pep neutrinos are shown in Table~\ref{tab:Er_ER_bg}. This table may be compared to Table~\ref{tab:ER_bg} which shows the background components used to detect CNO and pep neutrinos in the $S1/S2$ Xe analysis. 

\par For CE$\nu$NS, similar to the $S1/S2$ analysis described above, the event rate is low in $E_r$ space. Upon optimizing the $E_r$ range as above, Figure~\ref{fig:NR_Er_region} shows the $E_r$ range used for the SURF detector location. Similarly, the criteria for selection of the $E_r$ ranges for all detector locations are shown in Table~\ref{tab:Er_NR_bg}. 

\begin{table}[!htbp]
\caption{Flux components that are included in the null hypothesis, listed in the $\kappa$ column, when considering the detection of CNO and pep fluxes in the electron scattering channel, with the observable being the electron recoil energy. The second column gives the nuclear target, Xenon or Argon. The third column indicates whether efficiency and resolution are included. The energy threshold is the end point of pp, which removes two background components, pp and $^{7}$Be 384. 
}
\begin{tabular}{c|c|c|c|c}
\hline
$\gamma$ & Nucleon & Analysis method & $E_r$ threshold [keV] & Background components ($\kappa$)\\
%\hline
\hline
\multirow{4}{*}{CNO} & Xe & ideal & 291 & pep, $^{7}$Be 861\\
%~\footnote{Energy threshold is the maximum end point of pp and $^{7}$Be 384 to remove these two background components \label{footnote 1}}
\cline{2-5}
&\multirow{2}{*}{Ar} & ideal & 284 & pep, $^{7}$Be 861\\
\cline{3-5}
&& resolution+efficiency& 414 & pep, $^{7}$Be 861, $^{222}$Rn\\
\hline
\multirow{4}{*}{pep} & Xe & ideal & 291 & CNO, $^{7}$Be 861\\


\cline{2-5}
& \multirow{2}{*}{Ar}  & ideal& 284 & CNO, $^{7}$Be 861\\
\cline{3-5}
& & resolution+efficiency& 414 & CNO, $^{7}$Be 861, $^{222}$Rn\\
\hline

\end{tabular}
\label{tab:Er_ER_bg}
\end{table}

% Figure environment removed 

\begin{table}[!htbp]
\caption{Flux components that are included in the null hypothesis, listed in the $\kappa$ column, when considering the detection of the atmospheric, hep and DSNB fluxes in the CE$\nu$NS channel in nuclear recoil space. The third column gives the nuclear target, Xenon or Argon. The fourth column indicates whether efficiency and resolution are included in the electron recoil energy space. The last column indicates how the energy range is selected. 
}
\begin{tabular}{c|c|c|c|c}
\hline
$\gamma$ & $\kappa$ & Nucleon & Analysis Method & Condition for the subdomain selection \\
\hline
\multirow{3}{*}{atm} &\multirow{3}{*}{$^{8}$B, hep, DSNB}  & Xe  & ideal & \multirow{3}{*}
{$\eta_{i}^\gamma$ $>$ $\sum_{\kappa} \eta_{i}^\kappa $} \\

\cline{3-4}
&&  \multirow{2}{*}{Ar} & ideal & \\
\cline{4-4}
&&& resolution+efficiency &\\
\hline

\multirow{3}{*}{hep} & \multirow{3}{*}{$^{8}$B, atm, DSNB} & Xe  &  ideal &  \multirow{2}{*}{ remove all $^{8}$B from pdf  simulations}\\
\cline{3-4}
&& \multirow{2}{*}{Ar} & ideal &   \\

 \cline{4-5}
&&& resolution+efficiency & keep all hep\\

\hline

\multirow{3}{*}{DSNB } & \multirow{3}{*}{ $^{8}$B, atm, hep}& Xe  & ideal   & \multirow{3}{*}{$\eta_{i}^\gamma$ $>$ $\sum_{\kappa = hep, ^8B} \eta_{i}^\kappa $ and $\eta_{i}^\gamma$ $>$ $1/4 \eta_{i}^{atm}$ %\makecell{signal $>$ bg without atm \\and signal $>$ 1/4 atm}
}\\

\cline{3-4}
&&\multirow{2}{*}{Ar} & ideal  \\

\cline{4-4}
&&& resolution+efficiency  & \\
 
\hline
\end{tabular}

\label{tab:Er_NR_bg}
\end{table}









%\newpage 
\section{Results \label{sec:results}}
\par We start by presenting the results for the analysis in the electron recoil channel, focusing on the CNO and pep neutrino fluxes. Figure~\ref{fig:method2_DTvsALPHA_ER} shows the detector exposure which satisfies $\alpha = 0.001, 0.01, 0.1$ and $\beta=0.1$, as a function of $f_{2\nu \beta \beta}$. For $f_{2\nu \beta \beta} = 1$, we have a full contribution from the $2\nu \beta \beta$ background, while $f_{2\nu \beta \beta} = 0$ corresponds to a complete reduction of this background. Shown are the results using the likelihood in the $S1/S2$ space, and also in the recoil electron energy space, for both Argon and Xenon. As $\alpha$ increases the test becomes less stringent, consequently the desired exposure $\D\T$ tends to decrease. For the ideal case with a perfect energy resolution and  $f_{2\nu \beta \beta} = 0$, we find that both the pep and CNO signals may be extracted for exposures $\lesssim 100$ ton-years. For a full analysis in the $S1/S2$ space and with $f_{2\nu \beta \beta} = 0$, the exposures for signal identification reach $\gtrsim 10^3$ ton-years. Note that in all cases the CNO detection significance is more sensitive to the assumed metallicity due to the differences between the high and low metallicity models.


% Figure environment removed 


\par We now move onto the nuclear recoil channel. Figure~\ref{fig:DTvsALPHA_NR} shows the detection significances for the hep, DSNB and atmospheric neutrino components for the five detector locations under \mbox{$\alpha$ = 0.001, 0.01, 0.1} and \mbox{$\beta$ = 0.1}, corresponding to 90\% power. For all locations, we compare the ideal cases with no background and perfect efficiency to those in which detector efficiency and backgrounds are modelled. 

% Figure environment removed 


% Figure environment removed
%\newpage 



\par There are several interesting points to be noted in Figure~\ref{fig:DTvsALPHA_NR}, in particular how the detection prospects change as a function of detector location. For atmospheric neutrinos, the top panel in Figure~\ref{fig:DTvsALPHA_NR} shows how the optimal exposure (the exposure determined by fixing $\alpha$ and $\beta$) changes across the detectors' location; the best detection prospects come from SURF because the flux is the largest at this location, while the most pessimistic detection prospects are from CJPL because of a relatively lower flux. Depending on detector location, background model and the criteria of $\alpha$ and $\beta$, the atmospheric flux will be detectable with anywhere from $\sim 200-2000$ ton-years of exposure. The hep flux is detectable with $\gtrsim 10^3$ ton-yr exposure, while in all cases the DSNB requires larger exposures, $\gtrsim 10^4$ ton-yr. Interestingly, the prospect for detecting the DSNB at CJPL are slightly better than other locations, since the atmospheric flux is the lowest at these locations. 

\par  The above results show that the most drastic decrease in  power occurs when including backgrounds and resolution is for the hep and atmospheric components. This can be seen specifically by comparing the Xe ideal to the Xe $S1/S2$ model, and the Ar ideal to the Ar resolution $+$ efficiency model. The hep component is significantly affected by the energy smearing from the ${}^8$B component. On the other hand, the atmospheric significance is strongly affected by leakage from the electron recoil band, due to scattering events from pp neutrinos and from the 2$\nu \beta \beta$ background. 

\par As discussed above, our statistically methodology introduced in Section~\ref{sec:statistical} is different than previous studies~\cite{2021Jayden,AristizabalSierra:2021kht,Tang:2023xub}, so it is interesting to compare methodologies where possible. These previous studies provided a test statistic and expected significance for atmospheric neutrino signals under different parameters and exposures. For comparison, we also calculated this significance under our optimally determined detector exposures for each of the different solar, atmospheric, and DSNB flux components that we consider. To compute this significance, we first calculated the test statistic defined in~\citet{2021Jayden} as the ratio of the likelihoods under the null and alternative hypothesis. Then we calculated the square root of the test statistic, and the average of this square root over 10,000 simulated data sets is referred to as the expected significance. 
 Before the square root calculation, we replaced the negative value of the test statistic by a zero. The results for each of the fluxes are shown in Figure~\ref{fig:cf_Z_Asimov}. For atmospheric neutrinos, the expected significance generally increases for the high latitude detector locations such as SURF, which implies better prospects for detecting atmospheric neutrinos, and decreases for lower latitude detector locations such as CJPL, which implies worse detection prospects. On the other hand, for the DSNB the expected significance is larger for CJPL than for SURF, implying better detection prospects for the former location. This is the same trend as shown in our primary analysis method.  

\section{Conclusions \label{sec:conclusions}}

\par In this paper we have studied the prospects for detecting solar and atmospheric neutrinos in future large-scale Xenon and Argon dark matter detectors. We have in particular focused on how the prospects change as a function of detector location. This is an important consideration which has not been specifically addressed in previous studies, especially because the low-energy atmospheric neutrino flux depends strongly on the detector location. Moreover, we have employed a principled statistical   methodology that allows calculating the detector exposures as a function of the Type-I and Type-II error rates. 

\par Our analysis shows that the best prospects for the detection of the atmospheric neutrino flux are at the SURF location, while  the least pessimistic chances are at CJPL. In addition to the atmospheric neutrino flux, we examine the prospects for detection of the diffuse supernova neutrino background (DSNB) and all components of the solar neutrino flux. For the DSNB, we find that the prospects are best at CJPL, due largely to the reduced atmospheric neutrino background at this location. We find that the CNO component of the solar flux is detectable via the electron recoil channel with exposures of $\sim 10^3$ ton-yr for all locations.

\par Dark matter detectors provide a unique to measure the neutrino fluxes that we have discussed. The measurements would be complementary to those at future large-scale neutrino detectors. For example, a precise characterization of the solar neutrino flux would allow for novel methods to measure properties of the Sun~\cite{Cerdeno:2017xxl} and neutrino mass matrix parameters via dark matter detectors~\cite{Mishra:2023jlq} and at future detectors such as DUNE, JUNO, and Hyper-Kamiokande~\cite{Brdar:2023ttb}. In addition,  DUNE~\citep{DUNE:2020ypp,Kelly:2023ugn} and JUNO~\citep{JUNO:2021tll,Suliga:2023pve} will be sensitive to primarily charged current channels, and different neutrino energy ranges, for low-energy atmospheric neutrinos. Because of the relatively low event rates for atmospheric neutrinos, even for the large-scale detector that we consider here, a measurement of the event rates at different detectors through different channels would provide an important calibration in order to better understand the systematics in the measurement of the low-energy flux. This is important for the potential extraction of neutrino parameters and new physics from this data~\citep{Kelly:2019itm,Dutta:2020che}. 

\section*{Acknowledgements} 
Y.~Z. and L.~E.~S. are supported by the DOE Grant No. DE-SC0010813.

\bibliography{apssamp}


\end{document}

