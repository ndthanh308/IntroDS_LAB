\section{LIGHT Game Map}
\label{game_maps}
% Figure environment removed

%% Figure environment removed
%% Figure environment removed

\section{Test Results between Baseline and StoryShaping with Dialogue}
\label{test_results}
% Figure environment removed

% Figure environment removed

\section{ChatGPT Prompts}
\todo{I don't know how to eliminate space between tables...}
\begin{table}[h]
\caption{An example prompt used to trigger ChatGPT to act as NPC in game 1. Goal and prerequisites are different for each game.}
\label{tab:npc_prompt}
\footnotesize
\setlength\tabcolsep{6pt}
\toprule
You are an NPC in a text-adventure game. You and the agent are both in the game. For each step, waits for the agent to ask questions, then you should provide a correct answer based on the information about the game given as follow: \\
\textbf{Layout:} [room1]-east-[room2], [room2]-east-[room3] [room3]-north-[room4], [room1]-west-[room5], etc. (A-east-B means A is to the east of B, and B is to the west of A) \\
\textbf{Goal and prerequisite:} A dragon is in the dungeon. The only way to kill the dragon is to use a sword and there is no other way. \\
\textbf{Object information:} [object1], [object2] is in [room1]. [object3], [object4], [object5] is in [room2]. [room3] has no objects...
\bottomrule
%\caption{The first paragraph provides the context. The second paragraph essentially encodes the game map layout as graph edges in order to fit ChatGPT's text-only input. The third paragraph tells the NPC what the goal of the agent is and how to achieve it. The final paragraph provides information about what objects are available and where to find them.}
\end{table}
\begin{table}[t]
\caption{An example prompt used to trigger ChatGPT to act as the agent in game 1. Goal is different for each game.}
\label{tab:agent_prompt}
\footnotesize
\setlength\tabcolsep{6pt}
\toprule
You are an agent in a text-adventure game. You and the NPC are both in the game. Your goal is to kill the dragon. For each step, you should ask questions to the NPC in order to get the information on how to kill the dragon. \\
Ask a new set of questions based on the current observation and answers given to the previous set of questions according to the following rule: 1. ask similar and follow-up questions to previous questions that have a "yes" answer. 2. Avoid asking similar and follow-up questions to previous questions that have a "no" answer.
\bottomrule
%\caption{The first paragraph provides the context and the goal of the agent. The second paragraph provides instructions on how to ask the optimal questions. Except from the goal, the prompt does not contain any information about the game because we expect the ChatGPT agent should be able to retrieve them through conversation with ChatGPT NPC.}
\end{table}

\begin{table}[t]
\caption{Examples of an efficient dialogue and an inefficient dialogue between ChatGPT agent and ChatGPT NPC.}
\label{tab:dialogue}
\footnotesize
\setlength\tabcolsep{6pt}
\toprule
\textbf{Desired Dialogue} \\
\textit{Agent:} Do I need an object to kill the dragon? Can the dragon be lured out? Can the dragon be defeated without using a weapon? \\
\textit{NPC:} Yes. No. No. \\
\textit{Agent:} What object I should get to kill the dragon? \\
\textit{NPC:} A sword. \\
\textit{Agent:} Where can I find the sword? \\
\textit{NPC:} The Artillery room. \\
\textit{Dialogue ends. Enough information is gathered on how to kill the dragon.}
\midrule
\textbf{Undesired Dialogue} \\
\textit{Agent:} Do I need an object to kill the dragon? Can the dragon be lured out? Can the dragon be defeated without using a weapon?
\textit{NPC:} Yes. No. No. \\
\textit{Agent:} Can the dragon be trapped or immobilized temporarily? Is there any way to communicate with the dragon or understand its intentions? \\
\textit{NPC:} No. No. \\
\textit{Agent:} Is there any information about the dragon's origin or history? Can the dragon be reasoned with or persuaded to leave peacefully?
\textit{NPC:} No. No. \\
$\bullet \bullet \bullet$
\bottomrule
\end{table}

\begin{table}[t]
\caption{The prompt used to trigger the ChatGPT agent to generate the target knowledge graph, which will later be used in the training of KGA2C agent with storyshaping.}
\label{tab:kg_prompt}
\footnotesize
\setlength\tabcolsep{6pt}
\toprule
Output a textual knowledge graph that contains the game information required to reach the goal. Output it in the format of edges (entity1 $--$direction or verbs$\rightarrow$ entity2). For example, you$--$have$\rightarrow$rugs, town center $--$west$\rightarrow$ the bar
\bottomrule
%\caption{Prompt used to ask ChatGPT agent to generate a knowledge graph representing the steps it need to take to reach the goal. The knowledge graph generated will be used as the target knowledge graph in KGA2C agent's training.}
\end{table}