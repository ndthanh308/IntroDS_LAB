\documentclass[12pt]{amsart}

\usepackage{amsmath,amssymb}
\usepackage{bm}
\usepackage{graphicx}
\usepackage{ascmac}
\usepackage{amsthm}
\usepackage{amsfonts}
\usepackage{mathrsfs}
\usepackage{mathtools}
\usepackage[all]{xy}
%\mathtoolsset{showonlyrefs=true}

\renewcommand{\labelenumi}{\textrm(\roman{enumi})}

\theoremstyle{definition}
\newtheorem{thm}{Theorem}[section]
\newtheorem{dfn}[thm]{Definition}
\newtheorem{ex}[thm]{Example}
\newtheorem{cor}[thm]{Corollary}
\newtheorem{prop}[thm]{Proposition}
\newtheorem{lem}[thm]{Lemma}
\newtheorem{rem}[thm]{Remark}
\newtheorem{qst}[thm]{Question}
\newtheorem*{nota}{Notation}
\newtheorem*{claim}{Claim}
\newtheorem*{sotp}{Strategy of our proof}
\newtheorem*{oftp}{Organization of this paper}
\newtheorem*{ack}{Acknowledgments}
\renewcommand{\qed}{\hfill$\blacksquare$\par}

\numberwithin{thm}{section}
\numberwithin{equation}{section}

\newcommand{\Zpn}{\mathbb{Z}_{>0}}
\newcommand{\Znn}{\mathbb{Z}_{\geq 0}}
\newcommand{\Z}{\mathbb{Z}}
\newcommand{\Zp}{{\mathbb{Z}}_p}
\newcommand{\Zpt}{{\mathbb{Z}}_{p}^{\times}}
\newcommand{\Zps}{{\mathbb{Z}}_{p^2}}
\newcommand{\Zpst}{{\mathbb{Z}}_{p^2}^{\times}}
\newcommand{\Zpb}{\breve{\Z}_p}
\newcommand{\Q}{\mathbb{Q}}
\newcommand{\Qbar}{\overline{\Q}}
\newcommand{\Qp}{{\Q}_p}
\newcommand{\Qpt}{{\Q}_{p}^{\times}}
\newcommand{\Qps}{{\Q}_{p^2}}
\newcommand{\Qpb}{\breve{\Q}_p}
\newcommand{\Qpbt}{\breve{\Q}_p^{\times}}
\newcommand{\Qpbar}{\overline{\Q}_p}
\newcommand{\Zl}{{\mathbb{Z}}_{\ell}}
\newcommand{\Ql}{{\mathbb{Q}}_{\ell}}
\newcommand{\Qlbar}{\Qbar_{\ell}}
\renewcommand{\O}{\mathcal{O}}
\newcommand{\Zhat}{\widehat{\mathbb{Z}}}
\newcommand{\R}{\mathbb{R}}
\newcommand{\C}{\mathbb{C}}
\newcommand{\F}{\mathbb{F}}
\newcommand{\Fp}{\mathbb{F}_p}
\newcommand{\Fps}{\mathbb{F}_{p^2}}
\newcommand{\Fpbar}{\overline{\mathbb{F}}_p}
\newcommand{\A}{\mathbb{A}}
\newcommand{\Sbar}{\overline{S}}
\newcommand{\D}{\mathbb{D}}
\newcommand{\G}{\mathbb{G}}
\newcommand{\bS}{\mathbb{S}}
\newcommand{\bfS}{\mathbf{S}}
\newcommand{\bD}{\mathbf{D}}
\newcommand{\bE}{\mathbf{E}}
\newcommand{\bG}{\mathbf{G}}
\newcommand{\bH}{\mathbf{H}}
\newcommand{\M}{\mathcal{M}}
\newcommand{\bM}{\mathbf{M}}
\newcommand{\bI}{\mathbf{I}}
\newcommand{\sI}{\mathscr{I}}
\newcommand{\bP}{\mathbf{P}}
\newcommand{\sS}{\mathscr{S}}
\newcommand{\sShat}{\widehat{\sS}}
\newcommand{\sSbar}{\overline{\sS}}
\newcommand{\T}{\mathbb{T}}
\newcommand{\bT}{\mathbf{T}}
\newcommand{\bV}{\mathbf{V}}
\newcommand{\X}{\mathbb{X}}
\newcommand{\bX}{\mathbf{X}}
\newcommand{\sX}{\mathscr{X}}
\newcommand{\bZ}{\mathbf{Z}}
\newcommand{\be}{\mathbf{e}}
\newcommand{\Lambdabar}{\overline{\Lambda}}
\newcommand{\Omegabar}{\overline{\Omega}}
\newcommand{\gbar}{\overline{g}}
\newcommand{\sbar}{\overline{s}}
\renewcommand{\hbar}{\overline{h}}
\renewcommand{\H}{\mathbb{H}}
\renewcommand{\L}{\mathbf{L}}
\newcommand{\E}{\mathbf{E}}
\renewcommand{\P}{\mathbb{P}}
\newcommand{\Fpt}{\mathbb{F}^{\times}_p}
\newcommand{\bK}{\mathbf{K}}
\newcommand{\bc}{\mathbf{c}}
\newcommand{\bi}{\mathbf{i}}
\newcommand{\bu}{\mathbf{u}}
\newcommand{\bv}{\mathbf{v}}
\newcommand{\bx}{\mathbf{x}}
\newcommand{\cA}{\mathcal{A}}
\newcommand{\cB}{\mathcal{B}}
\newcommand{\cC}{\mathcal{C}}
\newcommand{\cD}{\mathcal{D}}
\newcommand{\cG}{\mathcal{G}}
\newcommand{\cI}{\mathcal{I}}
\newcommand{\cL}{\mathcal{L}}
\newcommand{\cN}{\mathcal{N}}
\newcommand{\cR}{\mathcal{R}}
\newcommand{\cS}{\mathcal{S}}
\newcommand{\cV}{\mathcal{V}}
\newcommand{\cZ}{\mathcal{Z}}
\newcommand{\fA}{\mathfrak{A}}
\newcommand{\fS}{\mathfrak{S}}
\newcommand{\Eb}{\breve{E}}
\newcommand{\Fb}{\breve{F}}
\newcommand{\Kb}{\breve{K}}
\newcommand{\fp}{\mathfrak{p}}
\renewcommand{\sc}{\mathrm{sc}}

\DeclareMathOperator{\Ker}{Ker}
\DeclareMathOperator{\Ima}{Im}
\DeclareMathOperator{\Coker}{Coker}
\DeclareMathOperator{\Hom}{Hom}
\DeclareMathOperator{\End}{End}
\DeclareMathOperator{\id}{id}
\DeclareMathOperator{\Aut}{Aut}
\DeclareMathOperator{\Lie}{Lie}
\DeclareMathOperator{\Frac}{Frac}
\DeclareMathOperator{\tr}{tr}
\DeclareMathOperator{\trd}{Trd}
\DeclareMathOperator{\nrd}{Nrd}
\DeclareMathOperator{\ord}{ord}
\DeclareMathOperator{\Gal}{Gal}
\DeclareMathOperator{\ev}{ev}
\DeclareMathOperator{\GL}{GL}
\DeclareMathOperator{\GSp}{GSp}
\DeclareMathOperator{\GU}{GU}
\DeclareMathOperator{\GSpin}{GSpin}
\DeclareMathOperator{\SL}{SL}
\DeclareMathOperator{\Sp}{Sp}
\DeclareMathOperator{\SU}{SU}
\DeclareMathOperator{\Spin}{Spin}
\DeclareMathOperator{\SO}{SO}
\DeclareMathOperator{\PSL}{PSL}
\DeclareMathOperator{\ab}{ab}
\DeclareMathOperator{\N}{N}
\DeclareMathOperator{\Nilp}{Nilp}
\DeclareMathOperator{\Sets}{Sets}
\DeclareMathOperator{\Spec}{Spec}
\DeclareMathOperator{\Spf}{Spf}
\DeclareMathOperator{\loc}{loc}
\DeclareMathOperator{\sml}{sim}
\DeclareMathOperator{\red}{red}
\DeclareMathOperator{\length}{length}
\DeclareMathOperator{\Int}{int}
\DeclareMathOperator{\inv}{inv}
\DeclareMathOperator{\Res}{Res}
\DeclareMathOperator{\Cor}{Cor}
\DeclareMathOperator{\Inf}{Inf}
\DeclareMathOperator{\df}{def}
\DeclareMathOperator{\Proj}{Proj}
\DeclareMathOperator{\diag}{diag}
\DeclareMathOperator{\rk}{rank}
\DeclareMathOperator{\Br}{Br}
\DeclareMathOperator{\ch}{char}
\DeclareMathOperator{\Pic}{Pic}
\DeclareMathOperator{\Fr}{Fr}
\DeclareMathOperator{\ad}{ad}
\DeclareMathOperator{\Irr}{Irr}
\DeclareMathOperator{\Sh}{Sh}
\DeclareMathOperator{\val}{val}
\DeclareMathOperator{\den}{den}
\DeclareMathOperator{\sep}{sep}
\DeclareMathOperator{\der}{der}
\DeclareMathOperator{\Ver}{Ver}
\DeclareMathOperator{\Stab}{Stab}
\DeclareMathOperator{\disc}{disc}
\DeclareMathOperator{\Ext}{Ext}
\DeclareMathOperator{\Ind}{Ind}
\DeclareMathOperator{\pr}{pr}
\DeclareMathOperator{\spl}{spl}
\DeclareMathOperator{\an}{an}
\DeclareMathOperator{\cd}{cd}
\DeclareMathOperator{\Tor}{Tor}
\DeclareMathOperator{\tor}{tor}
\DeclareMathOperator{\Ram}{Ram}
\DeclareMathOperator{\Mod}{Mod}
\DeclareMathOperator{\rec}{rec}
\DeclareMathOperator{\Ad}{Ad}
\DeclareMathOperator{\Unr}{Unr}
\DeclareMathOperator{\ur}{ur}
\DeclareMathOperator{\rH}{H}
\DeclareMathOperator{\et}{\acute{e}t}
\DeclareMathOperator{\nor}{nor}

\usepackage[OT2,T1]{fontenc}
\DeclareSymbolFont{cyrletters}{OT2}{wncyr}{m}{n}
\DeclareMathSymbol{\Sha}{\mathalpha}{cyrletters}{"58}

\usepackage{geometry}
\geometry{left=20mm,right=20mm,top=25mm,bottom=25mm}

\title[Hasse norm principle]{The Hasse norm principle for some non-Galois extensions of square-free degree}
\author[Y.~Oki]{Yasuhiro Oki}
\address{Department of Mathematics, Faculty of Science, Hokkaido University, Kita 10, Nishi 8, Kita-ku, Sapporo, Hokkaido, 060-0810, Japan. }
\email{oki@math.sci.hokudai.ac.jp}
\subjclass[2010]{Primary 11E72; Secondary 20C10}

\begin{document}
\maketitle
\begin{abstract}
In this paper, we study the Hasse norm principle for some non-Galois extensions of number fields. Our main theorem is that for any square-free composite number $d$ which is divisible by at least one of $3$, $55$, $91$ or $95$, there exists a finite extension of degree $d$ for which the Hasse norm principle fails. To accomplish it, we determine the structure of the Tate--Shafarevich groups of norm one tori for finite extensions of degree $d$ under the normality of $p$-Sylow subgroups of the Galois groups of their Galois closures for a square-free prime factor $p$ of $d$. Moreover, we reduce the assertion to an investigation of $2$-dimensional $\mathbb{F}_p$-representations of some groups of order coprime to $p$. 
\end{abstract}

\tableofcontents

\section{Introduction}\label{intr}

Let $k$ be a global field, and $K/k$ a finite separable field extension. Put
\begin{equation*}
\Sha(K/k):=(k^{\times}\cap \N_{K/k}(\A_{K}))/\N_{K/k}(K^{\times}),
\end{equation*}
where $\A_{K}^{\times}$ is the id{\`e}le group of $K$. We say that the \emph{Hasse norm principle} holds for $K/k$ holds if $\Sha(K/k)=0$. The study of $\Sha(K/k)$ is one of the most classical problem in number theory. It is known by Hasse that the Hasse norm principle for cyclic extensions of $k$ hold, and the biquadratic extension $\Q(\sqrt{13},\sqrt{-3})/\Q$ does not satisfy the Hasse norm principle (\cite{Hasse1931}). The following question is quite simple, however still open in general. 
\begin{itemize}
\item[($\ast$)] Determine the condition on $d\in \Zpn$ such that there is a finite separable extension $K/k$ of degree $d$ satisfying $\Sha(K/k)\neq 0$. 
\end{itemize}
If $d=p$ is a prime number, then Bartels proved that all finite extensions of degree $p$ satisfy the Hasse norm principle (\cite{Bartels1981a}). On the other hand, the global class field theory and Chebotarev's density theorem imply the existence of the failure of the Hasse norm principle if $d$ has a square factor. This is pointed out by Katayama (\cite{Katayama1985}) when $k$ is a number field. See also the appendix of \cite{Liang2022} for the function field case. Hence we need only investigate the question ($\ast$) in the case that $d$ is a square-free composite number. 

It is essentially known by Endo and Miyata (\cite{Endo1975}) that all Galois extensions of square-free degree satisfy the Hasse norm principle. However, it is much less known when we omit the condition that $K/k$ is Galois. In 1987, Drakokhrust and Platonov gave the existence of a number field $k$ which admits a finite extension $K/k$ of degree $6$ so that $\Sha(K/k)\neq 0$ (\cite{Drakokhrust1987}). Recently, the same assertion as above has been proven by Macedo and Newton in the case $d\in \{30,70,210\}$ (\cite{Macedo2021}), and by Hoshi, Kanai and Yamasaki for all composite numbers $d\leq 15$ (\cite{Hoshi2022}) and $d=110$ (\cite{Hoshi2023b}). 

In this paper, we study the Hasse norm principle for some finite field extensions of square-free degrees of number fields, and give another partial solution to the problem ($\ast$) for \emph{any} number field. 

\subsection{Main theorems: the Hasse norm principle}

For a finite separable field extension $K$ of $k$, we write $K^{\Gal}/k$ for the Galois closure of $K/k$. Let $p$ be a prime number which is distinct from the characteristic of $k$, and $\bE(k,p)$ the set of all finite separable field extensions $K$ of $k$ such that $p$-Sylow subgroups of $\Gal(K^{\Gal}/k)$ are normal. Here we give a description of the set
\begin{equation*}
\bfS(k,p):=\{d\in \Zpn \mid \text{there is $K\in \bE(k,p)$ so that $\Sha(K/k)[p^{\infty}]\neq 0$}\}. 
\end{equation*}
Here $\Sha(K/k)[p^{\infty}]$ is the $p$-primary torsion part of $\Sha(K/k)$. To archive it, we define subsets of $p\Z$ as follows: 
\begin{align*}
D_{1}(p)&:=\{n\in p\Zpn \mid \gcd(n,p-1)\geq 3\},\\
D_{2}(p)&:=\{n\in p\Zpn \mid \gcd(n,p+1)\text{ is not a power of }2\}.
\end{align*}

\begin{thm}[{Theorem \ref{ptnt}}]\label{mth1}
\emph{Let $k$ be a global field, and $p$ a prime number which is distinct from the characteristic of $k$. Then one has an equality
\begin{equation*}
\bfS(k,p)=p^{2}\Z \cup D_{1}(p)\cup D_{2}(p). 
\end{equation*}
In particular, we have
\begin{equation*}
\min \bfS(k,p)=
\begin{cases}
3p & \text{if }p>2,\\
4&\text{if }p=2. 
\end{cases}
\end{equation*}}
\end{thm}

\begin{rem}
We can also prove the vanishing of $\Sha(K/k)[p^{\infty}]$ for any finite extension $K$ of $k$ with $[K:k]<\min \bfS(k,p)$. See Corollary \ref{pttr}. 
\end{rem}

Theorem \ref{mth1} implies an infinitely existence of the failure of the Hasse norm principle for finite extensions of number fields of square-free degree. 

\begin{thm}[{Corollary \ref{sfhf}}]\label{mth2}
\emph{Assume that $k$ is a number field. Let $d$ be a square-free positive composite number which is divisible by at least one of $3$, $55$, $91$ or $95$. Then there is a finite extension $K$ of $k$ of degree $d$ for which the Hasse norm principle does not hold. }
\end{thm}

We also obtain more precise results on the group $\Sha(K/k)$ for $K\in \bE(k,p)$ when $[K:k]$ is a multiple of $p$ and at most two prime numbers which are different from $p$. 

\begin{thm}[Theorem \ref{3pdt}]\label{mth3}
\emph{Let $p$ and $\ell$ be two distinct prime numbers, and $K\in \bE(k,p)$ with $[K:k]=p\ell$. 
\begin{enumerate}
\item Suppose that one of the following is satisfied: 
\begin{itemize}
\item[(1)] $p>2=\ell$,
\item[(2)] $\ell \nmid p^{2}-1$. 
\end{itemize}
Then the Hasse principle holds for $K/k$. 
\item Assume $p\neq 3=\ell$. Then the Hasse principle holds for $K/k$ unless $p$, $\Gal(K^{\Gal}/k)$ and $\Gal(K^{\Gal}/K)$ fulfill at least one of the following: 
\begin{itemize}
\item[($\alpha$)] $\Gal(K^{\Gal}/k)\cong (\Z/p)^{2}\rtimes_{\varphi_{1}} \Z/3$ and $\Gal(K^{\Gal}/K)\cong \langle(\overline{1},\overline{0})\rangle \rtimes_{\varphi_{1}} \{0\}$, where
\begin{equation*}
\varphi_{1} \colon \Z/3 \rightarrow \Aut((\Z/p)^{2})\cong \GL_{2}(\Fp); 1\mapsto 
\begin{pmatrix}
0&-1\\
1&-1
\end{pmatrix},
\end{equation*}
\item[($\beta$)] $p\geq 5$, $\Gal(K^{\Gal}/k)\cong (\Z/p)^{2}\rtimes_{\varphi_{2}} \fS_{3}$ and $\Gal(K^{\Gal}/K)\cong \langle(\overline{1},\overline{1})\rangle \rtimes_{\varphi_{2}} \langle (1\ 2)\rangle$, where
\begin{equation*}
\varphi_{2}\colon \fS_{3}\rightarrow \Aut((\Z/p)^{2})\cong \GL_{2}(\Fp); (1\ 2\ 3)^{i}(1\ 2)^{j}\mapsto 
\begin{pmatrix}
-1&-1\\
1&0
\end{pmatrix}^{i}
\begin{pmatrix}
0&1\\
1&0
\end{pmatrix}^{j}. 
\end{equation*}
\end{itemize}
Moreover, if ($\alpha$) or ($\beta$) holds, then $\Sha(K/k)=0$ if and only if $(\Z/p)^{2}$ is contained in a decomposition group at some place of $K^{\Gal}$. Otherwise, there is an isomorphism $\Sha(K/k)\cong \Z/p$. 
\end{enumerate}}
\end{thm}

For example, two distinct prime numbers $p,\ell \in\{5,7\}$ satisfy the condition (2) in Theorem \ref{mth3} (i). Hence the Hasse principle holds for $K/k$ with $K\in \bE(k,5)\cup \bE(k,7)$. 

\begin{rem}
\begin{enumerate}
\item If $\{p,\ell\}=\{2,3\}$, then Theorem \ref{mth3} is contained in \cite[\S 10]{Drakokhrust1987}. 
\item Theorem \ref{mth3} for $\{p,\ell\}=\{3,5\}$ also follows from a result of Hoshi--Kanai--Yamasaki \cite[Theorem 1.15]{Hoshi2022} for $[K:k]=15$. Note that their proof heavily relies on computer, and hence Theorem \ref{mth2} gives another proof of a part of the their result.  
\end{enumerate}
\end{rem}

\begin{thm}[Theorem \ref{npex}]\label{mth4}
\emph{Let $k$ be a global field, and $p\neq 3$ a prime number which does not coincide with the characteristic of $k$.  
\begin{enumerate}
\item There is $K\in \bE(k,p)$ which has degree $9p$ over $k$ so that
\begin{equation*}
\Sha(K/k)\cong \Z/3p. 
\end{equation*}
\item Let $\ell \nmid 3p$ be a prime number which is invertible in $k$. Then there is $K\in \bE(k,p)$ which satisfies $[K:k]=3p\ell$ and
\begin{equation*}
\Sha(K/k)\cong \Z/p\ell. 
\end{equation*}
\end{enumerate}}
\end{thm}

\begin{rem}
\begin{enumerate}
\item If $K/k$ is Galois, then the exponent of $\Sha(K/k)$ is not a power of a prime number only if $[K:k]$ has at least two distinct square factors. 
\item The field $K$ in Theorem \ref{mth4} (i) for $p=2$ is the smallest degree finite separable field extension of $k$ such that the exponent of $\Sha(K/k)$ has at least two distinct prime divisors. Moreover, if $p\geq 5$, it turns out that $\Sha(K/k)$ is annihilated by a power of a prime number for every $K\in \bE(k,p)$ with $p\mid [K:k]$ and $[K:k]<6p$. See Proposition \ref{expp}. 
\end{enumerate}
\end{rem}

\subsection{Structure of second cohomology groups}

Here let $G$ be a finite group. For a subgroup $H$ of $G$, we define a finitely generated free abelian group $J_{G/H}$ equipped with an action of $G$ by the following exact sequence: 
\begin{equation*}
0\rightarrow \Z \rightarrow \Ind_{H}^{G}\Z \rightarrow J_{G/H}\rightarrow 0. 
\end{equation*}
Pick a set of subsets of $G$ containing all its cyclic subgroups. For $i\in \Znn$, put
\begin{equation*}
\Sha_{\cD}^{2}(G,J_{G/H}):=\Ker \left(H^{2}(G,J_{G/H})\xrightarrow{(\Res_{G/D})_{D \in \cD}} \bigoplus_{D\in \cD}H^{2}(D,J_{G/H})\right). 
\end{equation*}
If $\cD$ is the set of all cyclic subgroups of $G$, then we also denote $\Sha_{\cD}^{2}(G,J_{G/H})$ by $\Sha_{\omega}^{2}(G,J_{G/H})$. 

Assume that $G$ is the Galois group of a finite Galois extension $\widetilde{K}$ of $k$, and let $K$ be the intermediate field of $\widetilde{K}/k$ corresponding to $H$. Moreover, denote by $\cD$ the set of decomposition group of $\widetilde{K}/k$. It is known by \cite{Ono1963} and the Poitou--Tate duality that there is an isomorphism
\begin{equation*}
\Sha(K/k)\cong \Sha_{\cD}^{2}(G,J_{G/H})^{\vee}, 
\end{equation*}
where $\Sha_{\cD}^{2}(G,J_{G/H})^{\vee}$ is the Pontryagin dual of $\Sha_{\cD}^{2}(G,J_{G/H})$. See also Proposition \ref{onhn}. Note that $J_{G/H}$ is isomorphic to the character group of the \emph{norm one torus} associated to $K/k$, which is refered as Proposition \ref{mnjg}. This implies that the analysis of the group $\Sha_{\cD}^{2}(G,J_{G/H})$ is essential for proofs of Theorems \ref{mth1}, \ref{mth3} and \ref{mth4}. 

\vspace{3mm}

We prepare some notations on finite groups. Under the above notations, denote by $N_{G}(H)$ and $Z_{G}(H)$ the normalizer and the centralizer of $H$ in $G$ respectively. Moreover, put $N^{G}(H):=\bigcap_{g\in G}gHg^{-1}$, which is a normal subgroup of $G$. On the other hand, write $[H,G]$ for the subgroup of $G$ generated by $hgh^{-1}g^{-1}$ for all $g\in G$ and $h\in H$. 

Let $A$ be a finite abelian group, and $p$ a prime number. Then set
\begin{align*}
A[p^{\infty}]&:=\{a\in A\mid p^{n}a=0\text{ for some }n\in \Zpn \},\\
A^{(p)}&:=\{a\in A\mid np=0\text{ for some }n\in \Z\setminus p\Z \}. 
\end{align*}

\begin{thm}[Theorem \ref{tspt}]\label{mth5}
\emph{Let $G$ be a finite group, and $p$ a prime divisor of $\#G$. Assume that a $p$-Sylow subgroup $S_{p}$ of $G$ is normal. Take a subgroup $H$ of $G$ satisfying $N^{G}(H)=\{1\}$ and $\ord_{p}(G:H)=1$, and pick a set $\cD$ of subgroups of $G$ containing $\cC_{G}$. 
\begin{enumerate}
\item If one has $\Sha_{\cD}^{2}(G,J_{G/S_{p}H})=\Sha_{\omega}^{2}(G,J_{G/S_{p}H})$, 
then there is an isomorphism
\begin{equation*}
\Sha_{\cD}^{2}(G,J_{G/S_{p}H})^{(p)} \cong \Sha_{\cD}^{2}(G,J_{G/S_{p}H}). 
\end{equation*}
\item The abelian group $\Sha_{\omega}^{2}(G,J_{G/H})[p^{\infty}]$ is non-zero only if the following are satisfied: 
\begin{itemize}
\item[(a)] there is an isomorphism $S_{p}\cong (\Z/p)^{2}$,
\item[(b)] $[S_{p},G]=S_{p}$, 
\item[(c)] $N_{G}(S_{p}\cap H)=Z_{G}(S_{p}\cap H)$. 
\end{itemize}
\item If the abelian group $\Sha_{\omega}^{2}(G,J_{G/H})[p^{\infty}]$ is non-zero, then there is an isomorphism 
\begin{equation*}
\Sha_{\cD}^{2}(G,J_{G/H})[p^{\infty}] \cong 
\begin{cases}
0&\text{if there is $D \in \cD$ which contains $S_{p}$,}\\
\Z/p &\text{otherwise. }
\end{cases}
\end{equation*}
\end{enumerate}}
\end{thm}

We explain that Theorem \ref{mth5} imply Theorems \ref{mth1}, \ref{mth3} and \ref{mth4}. If a $p$-Sylow subgroup $S_{p}$ of $G$ is normal, then there are a subgroup $G'$ of $G$ and an isomorphism $G\cong S_{p}\rtimes G'$. Put $L:=S_{p}\cap H$. Then the conditions (b) and (c) are rephrased as (B) and (C) respectively. Here (B) and (C) are as follows: 
\begin{itemize}
\item[(B)] the $G'$-fixed part of $S_{p}$ is trivial,
\item[(C)] the stabilizer of $L$ in $G'$ acts trivially on $L$. 
\end{itemize}
First, we give an outline of a proof of Theorem \ref{mth1}. It is not so difficult to prove inclusions
\begin{equation*}
p^{2}\Z \subset \bfS(k,p) \subset p\Z. 
\end{equation*}
For the remaining assertion, take an integer $n\in \Zpn \setminus p\Z$. If $pn\in D_{1}(p)\cup D_{2}(p)$, then we can take a $2$-dimensional representation of $\Z/n$ fulfilling (B) and (C). Since the group $S_{p}\rtimes \Z/n$ is solvable, the proof of Shafarevich's theorem implies the existence of finite Galois extension $\widetilde{K}/k$ with Galois group $S_{p}\rtimes \Z/n$ in which all decomposition groups are cyclic. Then the subextension $K/k$ of $\widetilde{K}/k$ corresponding to $L\rtimes \{0\}$, where $L$ is a  certain line in $S_{p}$, does not satisfy the Hasse norm principle. Therefore we obtain that $pn$ is contained in $\bfS(k,p)$. On the other hand, if $pn\notin D_{1}(p)\cup D_{2}(p)$, we can prove that all $2$-dimensional representations of a group $G'$ do not satisfy (B) or (C) if there is a subgroup of $G'$ of index $n$. If $p>2$, this follows from the structure of a $2$-Sylow subgroup of $\GL_{2}(\Fp)$, which is clarified by Carter--Fong (\cite{Carter1964}). The case $p=2$ is much easier than the case $p>2$. This implies that $\bfS(k,p)$ does not contain $pn$, and hence the proof of Theorem \ref{mth1} (i) is completed. In addition, Theorems \ref{mth3} (ii) and \ref{mth4} follow from the positivity of the Hasse norm principle for finite extensions of prime degree (\cite{Bartels1981a}) and a more precise analysis of the argument in the proof of Theorem \ref{mth1}. 

\vspace{3mm}

In the following, we sketch a proof of Theorem \ref{mth5}. One has a short exact sequence
\begin{equation}\label{itex}
0\rightarrow J_{G/S_{p}H}\rightarrow J_{G/H}\rightarrow \Ind_{S_{p}H}^{G}J_{S_{p}H/S_{p}}\rightarrow 0. 
\end{equation}
This sequence can be interpreted as a sequence of $k$-tori. Indeed, suppose that $K/k$ is a finite extension, $G$ is a Galois group of $K^{\Gal}/k$ and $H$ is the subgroup of $G$ corresponding to $K$. If we denote by $K_{0}$ the $S_{p}H$-fixed part of $G$, then the above exact sequence is induced by the canonical exact sequence 
\begin{equation*}
1\rightarrow \Res_{K_{0}/k}T_{K/K_{0}}^{1} \rightarrow T_{K/k}^{1} \xrightarrow{\N_{K/K_{0}}} T_{K_{0}/k}^{1} \rightarrow 1. 
\end{equation*}
Here $\N_{K/K_{0}}$ is induced by the norm homomorphism of $K/K_{0}$. 

It turns out that there is an isomorphism $S_{p}\cong (\Z/p)^{m}$ for some $m\in \Zpn$. The assertion on the prime-to-$p$ torsion part of $\Sha_{\cD}^{2}(G,J_{G/H})$ follows from the exact sequence \eqref{itex} and a relation between the restriction and the corestriction on group cohomology. On the other hand, we analyze the structure of $\Sha_{\cD}^{2}(G,J_{G/H})[p^{\infty}]$ separately when $m=2$ and when it is not. In the case $m=2$, we introduce a subgroup $H^{2}(G,J_{G/S_{p}H})^{(\cD)}$ of $H^{2}(G,J_{G/S_{p}H})$, which fits into an exact sequence
\begin{equation*}
0\rightarrow H^{1}(G,\Ind_{S_{p}H}^{G}J_{S_{p}H/S_{p}})\xrightarrow{\delta_{G}} H^{2}(G,J_{G/S_{p}H})^{(\cD)} \xrightarrow{\widehat{\N}_{G}} \Sha_{\cD}^{2}(G,J_{G/H}) \rightarrow 0. 
\end{equation*}
This method is inspired by the study on \emph{CM tori} developed by Liang, Yang, Yu and the author (\cite{Liang2022}). We give a description of $H^{2}(G,J_{G/S_{p}H})^{(\cD)}[p^{\infty}]$ as a subquotient of $S_{p}^{\vee}$ (Proposition \ref{lciv}), which is archived by using the Mackey decomposition. This description implies the desired assertion for $m=2$. On the other hand, in the case $m\neq 2$, it would be difficult to make the argument in a similar case in $m=2$. However, we can derive the vanishing of $\Sha_{\omega}^{2}(G,J_{G/H})$ from the result of Bayer-Fluckiger, Lee and Parimala (\cite{BayerFluckiger2019}), which analyzes the structure of the Tate--Shafarevich groups of \emph{multinorm one tori}. 

\begin{oftp}
First, we review basic facts on group cohomology and flasque resolutions in Section \ref{prlm}. Second, in Section \ref{mnot}, we specify some known results on the cohomology groups of character groups of multinorm one tori. Section \ref{thts} is the technical heart of this paper. More precisely, we give a proof of Theorem \ref{mth5}. In Section \ref{fnrp}, we study $2$-dimensional $\Fp$-representations of finite groups of order coprime to $p$, which will be needed for the next section. Finally, we give proofs of main theorems in Section \ref{hnmt} by using Theorem \ref{mth5}. Appendix \ref{apdx} gives a group-theoretic interpretation of the structure of Tate--Shafarevich groups of charcter groups of multinorm one tori obtained by \cite[Proposition 8.5]{BayerFluckiger2019}. 
\end{oftp}

\begin{ack}
I expresses my gratitude to Seidai Yasuda for helpful comments. Moreover, I would like to thank Noriyuki Abe for answering my questions on representation theory of finite groups. This work was carried out with the support from the JSPS Research Fellowship for Young Scientists and KAKENHI Grant Number 22KJ0041.
\end{ack}


\section{Preliminaries}\label{prlm}

\subsection{Cohomology of finite groups}\label{chfg}

Let $G$ be a finite group. For a $G$-module, we mean a finitely generated abelian groups equipped with left actions of $G$. Moreover, a $G$-module which is torsion-free over $\Z$ is said to be a $G$-lattice. On the other hand, for a subgroup $H$ of $G$ and an $H$-module $M$, put $\Ind_{H}^{G}M:=\Hom_{\Z[H]}(\Z[G],M)$. We define a left action of $G$ on $\Ind_{H}^{G}M$ by the map
\begin{equation*}
G \times \Ind_{H}^{G}M \rightarrow \Ind_{H}^{G}M; (g,\varphi) \mapsto [g'\mapsto \varphi(g'g)]. 
\end{equation*}

\begin{prop}[{Mackey decomposition; cf.~\cite[Section 7.3, Proposition 22]{Serre1977}}]\label{mcky}
\emph{Let $G$ be a finite group, and $H$ and $D$ subgroups of $G$. Take a complete representative $R(D,H)$ of $D\backslash G/H$ in $G$. Consider an $H$-module $(\rho,M)$. Then the homomorphism of abelian groups
\begin{equation*}
\pi_{D,g}\colon \Ind_{H}^{G}M\rightarrow \Ind_{D\cap gHg^{-1}}^{D}M^{g};\varphi \mapsto [d\mapsto \varphi(g^{-1}d)]
\end{equation*}
for all $g\in R(D,H)$ induces an isomorphism of $D$-modules
\begin{equation*}
\Ind_{H}^{G}M \cong \bigoplus_{g\in R(D,H)}\Ind_{D\cap gHg^{-1}}^{D}M^{g}. 
\end{equation*}
Here $M^{g}$ is the abelian group $M$ equipped with the action of $gHg^{-1}$ defined by
\begin{equation*}
gHg^{-1}\rightarrow \Aut(M); h\mapsto (\rho \circ \Ad(g^{-1}))(h). 
\end{equation*}}
\end{prop}

The independence of the choice of $R(D,H)$ in Proposition \ref{mcky} follows from Lemma \ref{doub} as follows. 

\begin{lem}\label{doub}
\emph{Let $G$ be a finite group, and $H$ and $D$ are subgroups of $G$. 
\begin{enumerate}
\item Let $g\in G$. For $d,d'\in D$, we have $dgH=d'gH$ if and only if $d'd^{-1}\in D\cap gHg^{-1}$. In particular, the subset $DgH$ of $G$, which is an element of $D\backslash G/H$, has order $(\#D \cdot \#H)/\#(D\cap gHg^{-1})$. 
\item Assume that $g_1,g_2 \in G$ satisfies $Dg_{1}H=Dg_{2}H$. Then the subgroups $D \cap g_{1}Hg_{1}^{-1}$ and $D\cap g_{2}Hg_{2}^{-1}$ of $D$ are conjugate in $D$. 
\end{enumerate}}
\end{lem}

\begin{proof}
(i): Let $d,d'\in D$. The equality $dgH=d'gH$ is equivalent to the condition $g^{-1}d^{-1}d'g\in H$. Moreover, Since $d$ and $d'$ lie in $D$, the condition $g^{-1}d^{-1}d'g \in H$ holds if and only if the desired condition $d'd^{-1}\in D\cap gHg^{-1}$ is satisfied. 

(ii): By assumption, there are $h\in H$ and $d\in D$ such that $g_2=dg_{1}h$. Then we have
\begin{equation*}
D\cap g_{2}Hg_{2}^{-1}=dDd^{-1}\cap dg_{1}(hHh^{-1})g_{1}^{-1}d^{-1}=d(D\cap g_{1}Hg_{1}^{-1})d^{-1}. 
\end{equation*}
Hence the assertion holds. 
\end{proof}

We also need a description of the restriction map on the cohomology of $\Ind_{H}^{G}M$. 

\begin{lem}\label{hmts}
\emph{Let $G$ be a finite group, $H$ a subgroup of $G$, and $M$ a $G$-module. 
\begin{enumerate}
\item Consider a homomorphism of abelian groups
\begin{equation*}
\Phi_{G/H,M}\colon \Ind_{H}^{G}M \xrightarrow{\cong} \Z[G]\otimes_{\Z[H]}M; \varphi \mapsto \sum_{g\in R}g^{-1}\otimes \varphi(g),
\end{equation*}
where $R$ is a complete representative of $H\backslash G$ in $G$. Then the map $\Phi_{G/H,M}$ is independent of the choice of $R$, and is a homomorphism of $G$-modules. 
\item Under the notations in Proposition \ref{mcky}, the composite of $D$-homomorphisms
\begin{align*}
\Ind_{D\cap gHg^{-1}}^{D}M^{g} & \xrightarrow{\Phi_{D/(D\cap gHg^{-1}),M^{g}}} \Z[D]\otimes_{\Z[D\cap gHg^{-1}]}M^{g} \xrightarrow{1\otimes x\mapsto g\otimes x}\Z[G]\otimes_{\Z[H]}M \\
& \xrightarrow{\Phi_{G/H,M}^{-1}} \Ind_{H}^{G}M \xrightarrow{\pi_{D,g}} \Ind_{D\cap gHg^{-1}}^{D}M^{g}
\end{align*}
coincides with the indentity map on $\Ind_{D\cap gHg^{-1}}^{D}M^{g}$. 
\end{enumerate}}
\end{lem}

\begin{proof}
The assertions can be proved by direct computation. 
\end{proof}

\begin{prop}\label{gpac}
\emph{Let $G$ be a finite group, and $M$ a $G$-module. Pick a subgroup $D$ of $G$ and a direct summand $M_{0}$ of $M$, and denote by $\pr_{0}\colon M\rightarrow M_{0}$ the projection as $H$-modules. Then, for any $g\in G$ and $j\in \Znn$, one has a commutative diagram
\begin{equation*}
\xymatrix@C=60pt{
H^{j}(G,M)\ar[r]^{\Res_{G/D}}\ar@{=}[d] & H^{j}(D,M)\ar[r]^{\pr_{0*}}\ar[d]^{\cong} & H^{j}(D,M_{0})\ar[d]^{\cong}\\
H^{j}(G,M)\ar[r]^{\Res_{G/gDg^{-1}}\hspace{5mm}}& H^{j}(gDg^{-1},M)\ar[r]^{g\circ \pr_{0*}\circ g^{-1}}& H^{j}(gDg^{-1},g(M_{0})). 
}
\end{equation*}
Here the right vertical homomorphisma are induced by the inner automorphism $\Ad(g^{-1})$ on $G$ and the homomorphism $M\rightarrow M;x\mapsto gx$. }
\end{prop}

\begin{proof}
This is contained in \cite[Chapter 1, \S 5, 1.~Conjugation, p.~46]{Neukirch2000}. 
\end{proof}

\begin{lem}\label{qzcz}
\emph{Let $G$ be a finite group, and $\psi \colon G' \rightarrow G$ a group homomorphism. For any $j\in \Zpn$, there is a commutative diagram
\begin{equation*}
\xymatrix{
H^{j}(G,\Q/\Z)\ar[r]^{\cong}\ar[d]^{\psi^{*}}& H^{j+1}(G,\Z)\ar[d]^{\psi^{*}}\\
H^{j}(G',\Q/\Z)\ar[r]^{\cong}& H^{j+1}(G',\Z). }
\end{equation*}}
\end{lem}

\begin{proof}
There is a canonical exact sequence of trivial $G$-modules
\begin{equation*}
0\rightarrow \Z \rightarrow \Q \rightarrow \Q/\Z \rightarrow 0. 
\end{equation*}
Applying the cohomology to this sequence, we obtain the desired assertion. 
\end{proof}

For a finite group $A$, put $A^{\vee}:=\Hom(A,\Q/\Z)$. Note that it is the Pontryagin dual if $A$ is abelian.  

\begin{cor}\label{rsch}
\emph{Let $G$ be a finite group, and $H$ and $D$ subgroups of $G$. Then the following diagram is commutative: 
\begin{equation*}
\xymatrix@C=90pt{
H^{2}(G,\Ind_{H}^{G}\Z)\ar[d]^{\cong}\ar[r]^{\Res_{G/D}}& H^{2}(D,\Ind_{H}^{G}\Z) \ar[d]^{\cong}\\
H^{\vee}\ar[r]^{f\mapsto ((f\circ \Ad(g^{-1}))\mid_{D\cap gHg^{-1}})_{g}\hspace{20mm}}& \bigoplus_{g\in R(D,H)}(D\cap gHg^{-1})^{\vee}. 
}
\end{equation*}
Here $R(D,H)$ is a complete representative of $D\backslash G/H$ in $G$. }
\end{cor}

\begin{proof}
Note that $\Z$ is a direct summand of $\Ind_{H}^{G}\Z$ as an $H$-module, which is a consequence of Lemmas \ref{hmts} (ii). Take $g\in R(D,H)$, then we have a commutative diagram as follows by Lemma \ref{gpac}: 
\begin{equation*}
\xymatrix@C=60pt{
H^{2}(G,\Ind_{H}^{G}\Z)\ar[r]^{\Res_{G/H}}\ar@{=}[d] & H^{2}(H,\Ind_{H}^{G}\Z)\ar[r]^{\varphi \mapsto \varphi(1)} \ar[d]^{\cong} & H^{2}(H,\Z)\ar[d]^{\cong}\\
H^{2}(G,\Ind_{H}^{G}\Z)\ar[r]^{\Res_{G/gHg^{-1}}\hspace{5mm}}& H^{2}(gHg^{-1},\Ind_{H}^{G}\Z)\ar[r]^{\varphi \mapsto \varphi(g^{-1})}& H^{2}(gHg^{-1},\Z). 
}
\end{equation*}
Here the vertical isomorphisms are induced by $\Ad(g^{-1})$ and the action of $g$ on $\Ind_{H}^{G}\Z$. Moreover, the composite of the upper horizontal maps is an isomorphism by Shapiro's lemma. On the other hand, the following is commuative: 
\begin{equation*}
\xymatrix@C=60pt{
H^{2}(gHg^{-1},\Ind_{H}^{G}\Z)\ar[r]^{\varphi \mapsto \varphi(g^{-1})} \ar[d]^{\pi_{D,g*}} & H^{2}(gHg^{-1},\Z)\ar[d]^{\Res_{gHg^{-1}/(D\cap gHg^{-1})}} \\
H^{2}(D\cap gHg^{-1},\Ind_{D\cap gHg^{-1}}^{D}\Z)\ar[r]^{\hspace{8mm}\varphi \mapsto \varphi(1)} & H^{2}(D\cap gHg^{-1},\Z). 
}
\end{equation*}
Here the lower horizontal map is an isomorphism by Shapiro's lemma. In summary, we obtain a commutative diagram
\begin{equation*}
\xymatrix@C=60pt{
H^{2}(G,\Ind_{H}^{G}\Z)\ar[d]^{\cong} \ar[r]^{\pi_{D,g*}} & H^{2}(D,\Ind_{D\cap gHg^{-1}}^{D}\Z)\ar[d]^{\Res_{gHg^{-1}/(D\cap gHg^{-1})}} \\
H^{2}(H,\Z)\ar[r]^{\Ad(g^{-1})^{*}} & H^{2}(D\cap gHg^{-1},\Z). 
}
\end{equation*}
Furthermore, Lemma \ref{qzcz} implies that the lower horizontal homomorphism can be written as 
\begin{equation*}
\Ad(g^{-1})^{*}\colon H^{1}(H,\Q/\Z)\rightarrow H^{1}(D\cap gHg^{-1},\Q/\Z). 
\end{equation*}
This homomorphism is described as follows by direct calculation: 
\begin{equation*}
H^{\vee}\rightarrow (D\cap gHg^{-1});f\mapsto f\circ \Ad(g^{-1})\mid_{D\cap gHg^{-1}}. 
\end{equation*}
Therefore the assertion follows from Proposition \ref{mcky}. 
\end{proof}

%\begin{proof}
%By Proposition \ref{mcky}, Lemma \ref{qzcz} for $j=1$ and Shapiro's lemma, it suffices to prove the commutativity of the following for every $g\in R(D,H)$: 
%\begin{equation*}
%\xymatrix@C=75pt{
%H^{1}(H,\Q/\Z)\ar[r]^{\cong}\ar[d]^{\cong} & H^{1}(D\cap gHg^{-1},\Q/\Z)\ar[d]^{\cong}\\
%H^{\vee} \ar[r]^{f\mapsto f\circ \Ad(g^{-1})\mid_{D\cap gHg^{-1}}\hspace{10mm}}& (D\cap gHg^{-1})^{\vee}. }
%\end{equation*}
%This follows from Proposition \ref{gpac} for $j=1$ and direct computation. 
%\end{proof}

\begin{dfn}
\begin{enumerate}
\item We write $\cC_{G}$ for the set of cyclic subgroups of $G$. 
\item Let $M$ be a $G$-module, and $\cD$ a set of subgroups of $G$. For $j\in \Znn$, put
\begin{equation*}
\Sha_{\cD}^{j}(G,M):=\Ker\left(H^{j}(G,M)\xrightarrow{(\Res_{G/D})_{D}} \bigoplus_{D\in \cD}H^{j}(D,M) \right). 
\end{equation*}
Here $\Res_{G/D} \colon H^{j}(G,M)\rightarrow H^{j}(D,M)$ is the restriction map. In the case $\cD=\cC_{G}$, we denote the above set by $\Sha_{\omega}^{j}(G,M)$. 
\end{enumerate}
\end{dfn}

\begin{lem}\label{qtts}
\emph{Under the same notations as Corollary \ref{rsch}, the group $\Sha_{\omega}^{2}(G,\Ind_{H}^{G}\Z)$ is trivial. }
\end{lem}

\begin{proof}
Take a complete representative $R(D,H)$ of $D\backslash G/H$ which contains $1$. By Corollary \ref{rsch}, it suffices to prove the following: 
\begin{itemize}
\item If $f\in H^{\vee}$ satisfies $(f\circ \Ad(g^{-1}))\!\mid_{D\cap gHg^{-1}}$ is trivial for any $D\in \cC_{G}$ and $g\in R(D,H)$, then $f=0$. 
\end{itemize}
Take $f\in H^{\vee}$. For each $h\in H$, the assumption for $D=\langle h\rangle$ and $1 \in R(D,H)$ implies $f(h)=0$. Hence we obtain $f=0$ as desired. 
\end{proof}

\begin{prop}\label{ifts}
\emph{Let $\widetilde{G} \twoheadrightarrow G$ be a surjective homomorphism of finite groups, and $M$ a $G$-lattice. Take sets of subsets $\cD$ and $\widetilde{\cD}$ of $G$ and $\widetilde{G}$ respectively. Assume the  following: 
\begin{itemize}
\item $\cC_{\widetilde{G}}\subset \widetilde{\cD}$ (in particular, we have $\cC_{G}\subset \cD$),
\item the homomorphism $\widetilde{G}\twoheadrightarrow G$ induces a surjection $q_{\widetilde{G}/G}\colon \widetilde{\cD} \twoheadrightarrow \cD$. 
\end{itemize}
Then the inflation map $\Inf_{\widetilde{G}/G}\colon H^{2}(G,M)\rightarrow H^{2}(\widetilde{G},M)$ induces an isomorphism
\begin{equation*}
\Sha_{\cD}^{2}(G,M)\cong \Sha_{\widetilde{\cD}}^{2}(\widetilde{G},M). 
\end{equation*}}
\end{prop}

\begin{proof}
We follow the proof of \cite[Lemma 3.2 (ii)]{Huang2022}. Let $N$ be the kernel of the homomorphism $\widetilde{G} \twoheadrightarrow G$. Take $D \in \cD$ and $\widetilde{D} \in q_{\widetilde{G}/G}^{-1}(D)$, and put $D':=\widetilde{D} \cap N$. Since $N$ acts on $M$ trivially, we have the following: 
\begin{equation*}
H^{1}(N,M)=\Hom(N,M),\quad H^{1}(D',M)\cong \Hom(D',M). 
\end{equation*}
Moreover, the assumption that $M$ is torsion-free implies that $\Hom(N,M)$ and $\Hom(D',M)$ are trivial. Hence, by the Hochshild--Serre spectral sequence, we obtain commutative diagrams
\begin{gather*}
\xymatrix@C=30pt{
0\ar[r] & H^{1}(G,M)\ar[d]^{\Res_{G/D}}\ar[r]^{\Inf_{\widetilde{G}/G}}& H^{1}(\widetilde{G},M)\ar[d]^{\Res_{\widetilde{G}/\widetilde{D}}}\ar[r]& 0\\
0\ar[r] & H^{1}(D,M)\ar[r]^{\Inf_{\widetilde{D}/D}}& H^{1}(\widetilde{D},M) \ar[r]& 0,
}\\
\xymatrix@C=40pt{
0\ar[r] & H^{2}(G,M)\ar[d]^{\Res_{G/D}}\ar[r]^{\Inf_{\widetilde{G}/G}}& H^{2}(\widetilde{G},M)\ar[d]^{\Res_{\widetilde{G}/\widetilde{D}}} \ar[r]^{\Res_{\widetilde{G}/N}}& H^{2}(N,M) \ar[d]^{\Res_{N/D'}}\\
0\ar[r] & H^{2}(D,M)\ar[r]^{\Inf_{\widetilde{D}/D}}& H^{2}(\widetilde{D},M) \ar[r]^{\Res_{\widetilde{D}/D'}}& H^{2}(D',M). 
}
\end{gather*}
Here all the horizontal sequences are exact. Hence the assertion for $i=1$ holds. On the other hand, the set $\{D'\cap N'\mid D'\in \cD\}$ contains the set $\cC_{N'}$, which follows from the inclusion $\cC_{G'}\subset \cD'$. Hence, if $\Sha_{\omega}^{2}(N',M)=0$, then we obtain the desired isomorphism for $i=2$. It suffices to prove the triviality of $\Sha_{\omega}^{2}(N',\Z)$. However, this follows from Lemma \ref{qtts} for $G=H=N'$. 
\end{proof}

\begin{dfn}\label{ptpt}
For an abelian group $A$ and a prime number $p$, put
\begin{align*}
A[p^{\infty}]&:=\{a\in A\mid p^{m}a=0\text{ for some }m\in \Zpn\},\\
A^{(p)}&:=\{a\in A\mid na=0\text{ for some $n \in \Z\setminus p\Z$}\}. 
\end{align*}
\end{dfn}

\begin{prop}\label{tsan}
\emph{Let $G$ be a finite group, and $M$ a $G$-lattice. Take a subgroup $H$ of $G$. Let $\cD$ be a set of subgroups of $G$, and put $\cD_{H}:=\{D\cap H\subset H\mid D\in \cD\}$. Fix $j\in \Zpn$
\begin{enumerate}
\item If the abelian group $\Sha_{\cD_{H}}^{j}(H,M)$ is trivial, then $\Sha_{\cD}^{j}(G,M)$ is annihilated by $(G:H)$. 
\item Let $p$ be a prime number which does not divide $(G:H)$. Then the composite
\begin{equation*}
\Sha_{\cD}^{j}(G,M)[p^{\infty}]\hookrightarrow \Sha_{\cD}^{j}(G,M)\xrightarrow{\Res_{G/H}} 
\Sha_{\cD_{H}}^{j}(H,M)
\end{equation*}
is injective. 
\item Suppose that $(G:H)$ is a power of a prime number $p$. Then the composite
\begin{equation*}
\Sha_{\cD}^{j}(G,M)^{(p)}\hookrightarrow \Sha_{\cD}^{j}(G,M)\xrightarrow{\Res_{G/H}} \Sha_{\cD_{H}}^{j}(H,M). 
\end{equation*}
is injective. 
\end{enumerate}}
\end{prop}

\begin{proof}
These follow from \cite[Chap.~VII, \S 7, Proposition 6]{Serre1979}, which asserts that the composite
\begin{equation*}
H^{j}(G,M)\xrightarrow{\Res_{G/H}}H^{j}(H,M)\xrightarrow{\Cor_{G/H}}H^{j}(G,M)
\end{equation*}
is the multiplication by $(G:H)$ for every $j\in \Zpn$. 
\end{proof}

\subsection{Flasque resolutions}

Keep the notations in Section \ref{chfg}. 

\begin{dfn}
Let $G$ be a finite group. 
\begin{enumerate}
\item We say that $G$-lattice $P$ is \emph{induced} if $P=\bigoplus_{i=1}^{r}\Ind_{H_i}^{G}\Z$ for some subgroups $H_1,\ldots,H_r$ of $G$. 
\item A $G$-lattice $Q$ is said to be \emph{flasque} if $\widehat{H}^{-1}(H,Q)=0$ for any subgroup $H$ of $G$. 
\item A \emph{flasque resolution} of $M$ is an exact sequence of $G$-modules
\begin{equation*}
0 \rightarrow M \rightarrow P \rightarrow Q \rightarrow 0,
\end{equation*}
where $P$ is induced and $Q$ is flasque. 
\end{enumerate}
\end{dfn}

Note that a flasque resolution exists for any $G$-lattice, which is asserted by \cite[Lemme 3]{ColliotThelene1977}. 

\begin{lem}\label{wrfl}
\emph{Let $G$ be a finite group, $H$ a subgroup of $G$, and $M$ a $G$-lattice. Then a flasque resolution $0 \rightarrow M\rightarrow P \rightarrow Q \rightarrow 0$ of $M$ induces a flasque resolution of $\Ind_{H}^{G}M$: 
\begin{equation*}
0 \rightarrow \Ind_{H}^{G}M \rightarrow \Ind_{H}^{G}P \rightarrow \Ind_{H}^{G}Q \rightarrow 0. 
\end{equation*}}
\end{lem}

\begin{proof}
This follows from Proposition \ref{mcky} and Shapiro's lemma. 
\end{proof}

\begin{prop}\label{enmy}
\emph{Let $G$ be a finite group, and $p$ a divisor of $G$. Assume that a $p$-Sylow subgroups of $G$ is cyclic. Then we have $H^{1}(G,Q)[p^{\infty}]=0$ for any flasque $G$-lattice $Q$. }
\end{prop}

\begin{proof}
Take a $p$-Sylow subgroup $S_{p}$ of $G$, which is cyclic. Then $H^{1}(S_{p},Q)$ is trivial by \cite[Theorem 1.5]{Endo1975}. Since $(G:S_{p})$ is not divisible by $p$, the assertion follows from Proposition \ref{tsan} (ii). 
\end{proof}

\begin{prop}\label{snsc}
\emph{Let $G$ be a finite group, and $M$ a $G$-lattice. Take a flasque resolution
\begin{equation*}
0 \rightarrow M \rightarrow P\rightarrow Q \rightarrow 0
\end{equation*}
of $M$, and a set $\cD$ of subsets of $G$ containing $\cC_{G}$. 
\begin{enumerate}
\item There is an isomorphism of finite abelian groups
\begin{equation*}
\Sha_{\cD}^{1}(G,Q) \cong {\Sha}_{\cD}^{2}(G,M). 
\end{equation*}
\item If $\cD=\cC_{G}$, then there is an isomorphism of finite abelian groups
\begin{equation*}
H^{1}(G,Q) \cong {\Sha}_{\omega}^{2}(G,M). 
\end{equation*}
\end{enumerate}}
\end{prop}

\begin{proof}
We follow the argument in \cite[8.3, pp.~97--98]{Voskresenskii1998}. For any subgroup $D$ of $G$, the given flasque resolution of $M$ induces a commutative diagram
\begin{equation*}
\xymatrix{
0\ar[r]& H^{1}(G,Q)\ar[d]^{\Res_{G/D}} \ar[r]& H^{2}(G,M)\ar[d]^{\Res_{G/D}} \ar[r]& H^{2}(G,P)\ar[d]^{\Res_{G/D}}\\
0\ar[r]& H^{1}(D,Q) \ar[r]& H^{2}(D,M) \ar[r]& H^{2}(D,P). }
\end{equation*}

(i): It suffices to prove that $\Sha_{\omega}^{2}(k,P)$ is trivial. This assertion follows from Lemma \ref{qtts} since $P$ is induced. 

(ii): If $D$ is cyclic, then we have $H^{1}(D,Q)=0$ by Proposition \ref{enmy}. Combining this with (i), we obtain the desired isomorphism. 
\end{proof}

\begin{cor}\label{cyan}
\emph{Let $G$ be a finite group of which all Sylow subgroups are cyclic. Then, for a $G$-lattice $M$, we have $\Sha_{\omega}^{2}(G,M)=0$. }
\end{cor}

\begin{proof}
Take a flasque resolution $0 \rightarrow M \rightarrow P \rightarrow Q \rightarrow 0$ of $M$. Then we have $H^{1}(G,Q)=0$ by Proposition \ref{enmy}. Hence Proposition \ref{snsc} follows the desired assertion. 
\end{proof}

\begin{cor}\label{ids2}
\emph{Let $G$ be a finite group, $H$ a subgroup of $G$. For a $H$-lattice $M$, there is an isomorphism
\begin{equation*}
\Sha_{\omega}^{2}(G,\Ind_{H}^{G}M)\cong \Sha_{\omega}^{2}(H,M). 
\end{equation*}}
\end{cor}

\begin{proof}
Take a flasque resolution $0\rightarrow M \rightarrow P \rightarrow Q \rightarrow 0$ of $M$. By Lemma \ref{wrfl}, this induces a flasque resolution of $\Ind_{H}^{G}M$: 
\begin{equation*}
0 \rightarrow \Ind_{H}^{G}M \rightarrow \Ind_{H}^{G}P \rightarrow \Ind_{H}^{G}Q \rightarrow 0. 
\end{equation*}
Moreover, Proposition \ref{snsc} (ii) implies isomorphisms
\begin{equation*}
H^{1}(G,\Ind_{H}^{G}Q)\cong \Sha_{\omega}^{2}(G,\Ind_{H}^{G}M),\quad 
H^{1}(H,Q)\cong \Sha_{\omega}^{2}(H,M). 
\end{equation*}
On the other hand, we have $H^{1}(G,\Ind_{H}^{G}Q)\cong H^{1}(H,Q)$ by Shapiro's lemma, and hence the assertion holds. 
\end{proof}

\section{Multinorm one tori and their character groups}\label{mnot}

Let $k$ is a field, and fix a separable closure  $k^{\sep}$ of $k$. For a finite Galois extension $k'/k$, denote by $\Gal(k'/k)$ the Galois group of $k'/k$. 

Denote by $\G_m$ the multiplicative group scheme over $k$. Recall that a torus over $k$ is a linear algebraic group $T$ satisfying $T\otimes_{k}k^{\sep}\cong \G_{m,k_0}^{d}$ for some $d\in \Zpn$. It is known that we can take a finite Galois extension $K/k$ so that $T\otimes_{k}K\cong \G_{m,K}^{d}$. 

For a torus $T$ over $k$, put
\begin{equation*}
X^{*}(T):=\Hom_{k^{\sep}}(T\otimes_{k}k^{\sep},\G_{m,k^{\sep}}). 
\end{equation*}
If $T$ splits over a finite Galois extension $K$ of $k$, then $X^{*}(T)$ is a $\Gal(K/k)$-lattice. 

\begin{dfn}
Let $\bK$ be a finite separable algebra over $k$. Then put
\begin{equation*}
T_{\bK/k}^{1}:=\{t\in \Res_{\bK/k}\G_{m}\mid \N_{\bK/k}(t)=1\},
\end{equation*}
which is a $k$-torus. It is called a \emph{mutinorm one torus}. Moreover, we say it a \emph{norm one tori} if $\bK$ is a field. 
\end{dfn}

We give a description of character groups of multinorm one tori. 

\begin{dfn}\label{jgdf}
Let $G$ be a finite group, and $H_1,\ldots,H_r$ subgroups of $G$. Then we define a $G$-module $J_{G}$ by the exact sequence
\begin{equation*}
0\rightarrow \Z \rightarrow \bigoplus_{i=1}^{r}\Ind^{G}_{H_{i}}\Z \rightarrow J_{G/(H_1,\ldots,H_r)} \rightarrow 0. 
\end{equation*}
Here the homomorphism $\Z \rightarrow \Ind_{H_i}^{G}\Z$ is given by $1\mapsto \sum_{g_i \in G/H_i}g_i$ for each $i$. If $r=1$ and $H_{1}=\{1\}$, we simply denote $J_{G/(H_1,\ldots,H_r)}$ by $J_{G}$. 
\end{dfn}

\begin{prop}\label{mnjg}
\emph{Let $\bK=K_1\times \cdots \times K_r$, where $K_1,\ldots,K_r$ are finite separable field extensions of $k$. Take a finite Galois extension $\widetilde{K}$ of $k$ which contains $K_{i}$ for all $i\in \{1,\ldots,r\}$. Put $G:=\Gal(\widetilde{K}/k)$ and $H_{i}:=\Gal(\widetilde{K}/K_{i})$ for each $i\in \{1\,\ldots,r\}$. Then there is an isomorphism
\begin{equation*}
X^{*}(T_{\bK/k}^{1}) \cong J_{G/(H_1,\ldots,H_r)}. 
\end{equation*}}
\end{prop}

\begin{prop}\label{mnex}
\emph{Keep the notations in Definition \ref{jgdf}. 
\begin{enumerate}
\item There is an isomorphism $H^{0}(G,J_{G/(H_1,\ldots,H_r)})\cong \Z^{\oplus r-1}$.  
\item There is an exact sequence
\begin{align*}
0\rightarrow H^{1}(G,J_{G/(H_1,\ldots,H_r)})\rightarrow G^{\vee} &\xrightarrow{(f\mapsto f\mid_{H_i})_{i}}\bigoplus_{i=1}^{r}H_{i}^{\vee} \rightarrow H^{2}(G,J_{G/(H_1,\ldots,H_r)})\\
\rightarrow H^{2}(G,\Q/\Z)&\xrightarrow{(\Res_{G/H_{i}})_{i}}\bigoplus_{i=1}^{r}H^{2}(H_{i},\Q/\Z). 
\end{align*}
\end{enumerate}}
\end{prop}

\begin{proof}
(i): Consider the exact sequence of abelian groups
\begin{equation*}
0\rightarrow H^{0}(G,\Z)\rightarrow \bigoplus_{i=1}^{r}H^{0}(G,\Ind_{H_{i}}^{G}\Z) \rightarrow H^{0}(G,J_{G/(H_1,\ldots,H_r)})\rightarrow H^{1}(G,\Z). 
\end{equation*}
Then the group $H^{1}(G,\Z)$ is zero. Moreover, Shapiro's lemma implies
\begin{equation*}
H^{0}(G,\Ind_{H_i}^{G}\Z)\cong H^{0}(H_i,\Z)\cong \Z. 
\end{equation*}
Hence $H^{0}(G,J_{G/(H_1,\ldots,H_r)})$ has rank $r-1$. Moreover, it is torsion-free since $H^{0}(G,J_{G/(H_1,\ldots,H_r)})$ is contained in $J_{G/(H_1,\ldots,H_r)}$. Consequently, we obtain the desired assertion. 

(ii): The exact sequence
\begin{equation*}
0\rightarrow \Z \rightarrow \bigoplus_{i=1}^{r}\Ind_{H_i}^{G}\Z \rightarrow J_{G/(H_1,\ldots,H_r)} \rightarrow 0,
\end{equation*}
which is a consequence of Proposition \ref{mnjg}, induces an exact sequence
\begin{gather*}
\bigoplus_{i=1}^{r}H^{1}(G,\Ind_{H_{i}}^{G}\Z)\rightarrow H^{1}(G,J_{G/(H_1,\ldots,H_r)})\rightarrow H^{2}(G,\Z)\rightarrow \bigoplus_{i=1}^{r}H^{2}(G,\Ind_{H_i}^{G}\Z)\\ 
\rightarrow H^{2}(G,J_{G/(H_1,\ldots,H_r)})\rightarrow H^{3}(G,\Z)\rightarrow \bigoplus_{i=1}^{r}H^{3}(G,\Ind^{G}_{H_i}\Z). 
\end{gather*}
For any $i\in \{1,\ldots,r\}$ and $j\in \Zpn$, we have $H^{j}(G,\Ind_{H_i}^{G}\Z)\cong H^{j}(H_i,\Z)$. Moreover, $H^{1}(H_{i},\Z)$ is trivial for any $i$. On the other hand, \cite[(1.6.5) Proposition]{Neukirch2000} implies the composite
\begin{equation*}
H^{j}(G,\Z)\rightarrow H^{j}(G,\Ind_{H}^{G}\Z)\cong H^{j}(H,\Z),
\end{equation*}
where the second homomorphism follows from Shapiro's lemma, coincides with the restriction map $\Res_{G/H_{i}}$. In addition, by Lemma \ref{qzcz}, one has a commuative diagram as follows: 
\begin{equation*}
\xymatrix@C=46pt{
H^{j-1}(G,\Q/\Z)\ar[d]^{\cong}\ar[r]^{(\Res_{G/H_{i}})_{i}\hspace{6mm}}& \bigoplus_{i=1}^{r}H^{j-1}(H_{i},\Q/\Z)\ar[d]^{\cong}\\
H^{j}(G,\Z)\ar[r]^{(\Res_{G/H_{i}})_{i}\hspace{6mm}}& \bigoplus_{i=1}^{r}H^{j}(H_{i},\Z). }
\end{equation*}
In the case $j=2$, Corollary \ref{rsch} implies that the upper horizontal sequence are written as the direct sum of the restriction maps
\begin{equation*}
G^{\vee}\rightarrow \bigoplus_{i=1}^{r}H_{i}^{\vee};f\mapsto (f\!\mid_{H_i})_{i}. 
\end{equation*}
Therefore the assertion holds. 
\end{proof}

\begin{lem}[{\cite[Proposition 1.3]{Endo2011}}]\label{endo}
\emph{Under the same notations as Definition \ref{jgdf}, we further assume that $H_{r-1}$ contains $H_{r}$. Then there is an isomorphism
\begin{equation*}
J_{G/(H_1,\ldots,H_r)} \cong J_{G/(H_1,\ldots ,H_{r-1})} \oplus \Ind^{G}_{H_r}\Z. 
\end{equation*}}
\end{lem}

\begin{cor}\label{andg}
\emph{Under the same notations as Definition \ref{jgdf}, the abelian group $\Sha_{\omega}^{2}(G,J_{G/(H_1,\ldots,H_r)})$ is annihilated by the great common divisor of $(G:H_{i})$ for all $i\in \{1,\ldots,r\}$. }
\end{cor}

\begin{proof}
Fix $j\in \{1,\ldots,r\}$. For each $i\in \{1,\ldots,r\}$, take a complete representative $R(H_{j},H_{i})$ of $H_{j}\backslash G/H_{i}$ in $G$ containing $1$. By Proposition \ref{mcky}, there is an isomorphism of $H_{j}$-modules
\begin{equation*}
\bigoplus_{i=1}^{r}\Ind^{G}_{H_i}\Z \cong \bigoplus_{i=1}^{r}\bigoplus_{g\in R(H_{j},H_{i})}\Ind^{H_{j}}_{H_{j}\cap gH_{i}g^{-1}}\Z. 
\end{equation*}
Note that we have $H_{j}\cap gH_{i}g^{-1}\subset H_{j}$ for any $i\in \{1,\ldots,r\}$ and $g\in G$. Moreover, it is an equality if $i=j$ and $g\in H_{j}$. Therefore, Lemma \ref{endo} gives an isomorphism of $H_{j}$-modules
\begin{equation*}
J_{G/(H_{1},\ldots,H_{r})} \cong \left(\bigoplus_{g\in R(H_{j},H_{j})\setminus \{1\}}\Ind^{H_{j}}_{H_{j}\cap gH_{j}g^{-1}}\Z \right) \oplus \left(\bigoplus_{i\neq j}\bigoplus_{g\in R(H_{j},H_{i})}\Ind^{H_{j}}_{H_{j}\cap gH_{i}g^{-1}}\Z \right). 
\end{equation*}
In particular, $\Sha_{\omega}^{2}(H_{j},J_{G/(H_1,\ldots,H_r)})$ vanishes by Lemma \ref{qtts}. Combining this consequence with Proposition \ref{tsan}, the abelian group $\Sha_{\omega}^{2}(G,J_{G/(H_1,\ldots,H_r)})$ is annihilated by $(G:H_{j})$. This concludes the proof of Corollary \ref{andg}. 
\end{proof}

It is known that there are much stronger assertions as Corollary \ref{andg} for certain $(H_{1},\ldots,H_{r})$. 

\begin{prop}\label{bart}
\emph{Let $G$ be a finite group, and $H$ a subgroup of $G$ of prime index. Then the abelian group $\Sha_{\omega}^{2}(G,J_{G/H})$ is trivial. }
\end{prop}

\begin{proof}
This follows from \cite[Proposition 9.1]{ColliotThelene1987}, which claims that $J_{G/H}$ is retract rational. Here we give an another proof for reader's convenience. Put $p:=(G:H)$, which is a prime number. Then the group $G$ admits an injection into the symmetric group of $p$-letters. In particular, a $p$-Sylow subgroup $S_{p}$ of $G$ is isomorphic to $\Z/p$. This concludes the triviality of $\Sha_{\omega}^{2}(S_{p},J_{G/H})$ by Corollary \ref{cyan}. Moreover, we obtain that $(G:S_{p})$ annihilates $\Sha_{\omega}^{2}(G,J_{G/H})$ by Proposition \ref{tsan} (i). On the other hand, Corollary \ref{andg} implies that $\Sha_{\omega}^{2}(G,J_{G/H})$ is a $p$-group, and hence the assertion holds. 
\end{proof}

\begin{prop}\label{abts}
\emph{Put $G:=\Z/n_1 \times \Z/n_2$, where $n_1,n_2\in \Zpn$ and $n_1\mid n_2$. Then there is an isomorphism
\begin{equation*}
\Sha_{\omega}^{2}(G,J_{G})\cong \Z/n_1. 
\end{equation*}}
\end{prop}

\begin{proof}
By Proposition \ref{mnex} (ii), we have $H^{2}(G,J_{G})\cong H^{2}(G,\Q/\Z)$. On the other hand, for any $D\in \cC_{G}$, we have $H^{2}(D,\Q/\Z)\cong H^{1}(D,\Z)$, and hence it is trivial. This induces an isomorphism
\begin{equation*}
\Sha_{\omega}^{2}(G,J_{G/H})\cong H^{2}(G,\Q/\Z)\cong H^{3}(G,\Z). 
\end{equation*}
Therefore the assertion follows from \cite[Lemma 6.4, Lemma 6.5]{Frei2018}. 
\end{proof}

\begin{thm}[Theorem \ref{apth}]\label{blpl}
\emph{Let $G$ be a finite group, and $H_1,\ldots,H_r$ normal subgroups of $G$ of index a prime number $p$ that satisfy $H_{i}\not\subset H_{j}$ for distinct $i,j\in \{1,\ldots,r\}$. Put $N:=\bigcap_{i=1}^{r}H_{i}$ and $m:=\ord_{p}(G:N)$. Take $e_1,\ldots,e_r\in \Zpn$. Then there is an isomorphism
\begin{equation*}
\Sha_{\omega}^{2}\left(G,J_{G/(H_1^{(e_1)},\ldots,H_r^{(e_r)})}\right)\cong 
\begin{cases}
(\Z/p)^{\oplus r-2}&\text{if $m=2$ and $r\geq 3$},\\
0&\text{otherwise}. 
\end{cases}
\end{equation*}}
\end{thm}

We give a proof of Theorem \ref{blpl} in Appendix \ref{apdx}, which will be reduced to the structure of Tate--Shafarevich groups of multinorm one tori (\cite[Proposition 8.5]{BayerFluckiger2019}). 

\begin{rem}
Theorem \ref{blpl} for $m=1$ also follows from Corollary \ref{cyan}. Moreover, Theorem \ref{blpl} in the case $m=r=2$ is known by H{\"u}rlimann (\cite{Hurlimann1984}). 
\end{rem}

\section{Computation of second cohomology groups}\label{thts}

Let $G$ be a group, and $H$ a subgroup of $G$. Then put
\begin{equation*}
N^{G}(H):=\bigcap_{g\in G}gHg^{-1},\quad Z_{G}(H):=\{g\in G\mid gh=hg\text{ for any }h\in H \}. 
\end{equation*}
By definition, the subgroup $N^{G}(H)$ is normal in $G$. Moreover, for another subgroup $D$ of $G$, denote by $[H,D]$ the subgroup of $G$ generated by $hdh^{-1}d^{-1}$ for all $h\in H$ and $d\in D$. Note that $[H,D]$ is normal in $G$ if $D$ is abelian and normal in $G$. In particular, if $H=D$, then write $[H,H]$ for $H^{\der}$. 

\subsection{Exact sequences}\label{fcex}

\begin{lem}\label{jgcg}
\emph{Let $G$ be a finite group, and $H\subset H'$ subgroups of $G$. Then there is an exact sequence
\begin{equation*}
\xymatrix{
0\ar[r] & \Ind_{H'}^{G}\Z \ar[d] \ar[r]& \Ind_{H}^{G}\Z \ar[d] \ar[r]& \Ind_{H'}^{G}J_{H'/H} \ar@{=}[d] \ar[r]& 0\\
0\ar[r] & J_{G/H'} \ar[r]& J_{G/H} \ar[r]& \Ind_{H'}^{G}J_{H'/H} \ar[r]& 0, }
\end{equation*}
where the horizontal sequences are exact. }
\end{lem}

\begin{proof}
We define the upper horizontal sequence by taking $\Ind_{H'}^{G}$ to the exact sequence in Definition \ref{jgdf} for $H\subset H'$. On the other hand, by definition, one has a commutative diagram of $G$-lattices
\begin{equation*}
\xymatrix{
0\ar[r] & \Z \ar@{=}[d] \ar[r]& \Ind_{H'}^{G}\Z \ar[d] \ar[r]& J_{G/H'} \ar[d] \ar[r]& 0\\
0\ar[r] & \Z \ar[r]& \Ind_{H}^{G}\Z \ar[r]& J_{G/H} \ar[r]& 0. }
\end{equation*}
This implies that the composite $\Z \rightarrow \Ind_{H}^{G}\Z \rightarrow \Ind_{H'}^{G}J_{H'/H}$ is zero, and hence the upper horizontal sequence induces a lower exact sequence. 
\end{proof}

\begin{rem}
The exact sequence in Lemma \ref{jgcg} can be interpreted as a sequence of tori. Let $k$ be a field, $K/k$ a finite field extension and $K'/k$ a subextension of $K/k$. Put $G:=\Gal(K^{\Gal}/k)$, $H:=\Gal(K^{\Gal}/K)$ and $H':=\Gal(K^{\Gal}/K')$. Then we obtain a commutative diagram of $k$-tori
\begin{equation*}
\xymatrix@C=30pt{
1\ar[r]& \Res_{K'/k_0}T_{K/K'}^{1}\ar@{=}[d] \ar[r]& T_{K/k}^{1}\ar[d] \ar[r]^{\N_{K/K'}}& T_{K'/k}^{1}\ar[d] \ar[r]& 1\\
1\ar[r]& \Res_{K'/k_0}T_{K/K'}^{1} \ar[r]& \Res_{K/k}\G_m \ar[r]^{\N_{K/K'}}& \Res_{K'/k}\G_m \ar[r]& 1. 
}
\end{equation*}
Here the horizontal sequences are exact. Then exact sequence of $G$-lattices induced by Proposition \ref{mnjg} is the desired one. 
\end{rem}

\begin{lem}\label{sbsd}
\emph{Let $G$ be a finite group of which a $p$-Sylow subgroup $S_{p}$ is normal for a prime divisor $p$ of $\#G$. 
\begin{enumerate}
\item There exist a subgroup $G'$ of $G$ and an isomorphism $G\cong S_{p}\rtimes G'$. 
\item Let $D$ be a subgroup of $G$. Then the subgroup $S_{p}\cap gDg^{-1}$ of $S_{p}$ has index $p^{\ord_{p}(G:D)}$ for any $g\in G$. 
\item Let $D$ be a subgroup of $G$. If $S_{p}$ is abelian, then there are subgroup $D'$ of $G'$ and $s\in S_{p}$ such that the isomorphism in (i) induces $sDs^{-1}\cong (S_{p}\cap D)\rtimes D'$ in $G$. 
\item Let $H$ be a subgroup of $G$ satisfying $N^{G}(H)=\{1\}$. 
\begin{itemize}
\item[(a)] Let $R(S_{p},H)$ be a complete representative of $S_{p}\backslash G/H$ in $G$. Then
\begin{equation*}
\bigcap_{g\in R(S_{p},H)}(S_{p}\cap gHg^{-1})=\bigcap_{g\in G}(S_{p}\cap gHg^{-1})=N^{G}(S_{p}\cap H)=\{1\}. 
\end{equation*}
\item[(b)] If $S_{p}$ is abelian, then the exponent of $S_{p}$ is equal to that of $S_{p}/(S_{p}\cap H)$. 
\end{itemize}
\item Under the same assumption as (iv), we further assume $\ord_{p}(G:H)=1$. Then the group $S_{p}$ is abelian. In particular, there is an isomorphism $S_{p}\cong (\Z/p)^{m}$ for some $m\in \Zpn$. 
\end{enumerate}}
\end{lem}

\begin{proof}
(i): This is contained in the Schur--Zassenhaus theorem. 

(ii): Since the normality of $S_{p}$ in $G$ implies $S_{p}\cap gDg^{-1}=g^{-1}(S_{p}\cap D)g$ for every $g\in G$, we may assume $g=1$. Put $r:=\ord_{p}(G:D)$, and take a $p$-Sylow subgroup $S'_{p}$ of $H$, which has order $\#S_{p}/p^{r}$. Using the assumption $S_{p}\triangleleft G$ and Sylow's theorem, we have $S'_{p}\subset S_{p}$. Hence $S'_{p}$ is contained in $S_{p}\cap D$. This inclusion is an equality by the assumption on $S'_{p}$, which implies the desired equality $(S_{p}:S_{p}\cap D)=p^{r}$. 

(iii): Let $D'$ be the image of $S_{p}D/S_{p}$ under the isomorphism $G/S_{p}\cong G'$. Then, for each $d \in D'$ there is $\widetilde{f}(d)\in S_{p}$ so that $\widetilde{f}(d)d \in D$. Moreover, if $\widetilde{f'}(d)\in S_{p}$ also satisfies $\widetilde{f'}(d)d \in D$, then one has
\begin{equation*}
\widetilde{f}(d)\widetilde{f'}(d)^{-1}=(\widetilde{f}(d)d)(\widetilde{f'}(d)d)^{-1}\in S_{p}\cap D. 
\end{equation*}
Hence we obatin a map
\begin{equation*}
f\colon D'\rightarrow S_{p}/(S_{p}\cap D);d \mapsto \widetilde{f}(d)\bmod S_{p}\cap D. 
\end{equation*}
By direct computation, it turns out that $f$ is a $1$-cocycle. On the other hand, we have
\begin{equation*}
H^{1}(D',S_{p}/(S_{p}\cap D))=0
\end{equation*}
since $\#D'$ is coprime to $p$. Hence there is $s\in S_{p}$ such that $f(d)=s^{-1}d(s)\bmod S_{p}\cap D$ for all $d\in D'$. Then we obtain that the subgroup $sDs^{-1}$ is equal to $(S_{p}\cap D)\rtimes D'$. 

(iv): First, we give a proof of (a). By Lemma \ref{doub} (ii) and the normality of $S_{p}$ in $G$, we have
\begin{equation*}
\bigcap_{g\in R(S_{p},H)}(S_{p}\cap gHg^{-1})=\bigcap_{g\in G}(S_{p}\cap gHg^{-1})=\bigcap_{g\in G}g^{-1}(S_{p}\cap H)g=N^{G}(S_{p}\cap H). 
\end{equation*}
Furthermore, since $N^{G}(S_{p}\cap H)$ is contained in $N^{G}(H)$, the triviality of $N^{G}(S_{p}\cap H)$ follows from that of $N^{G}(H)$. This completes the proof of (a). 

Second, for a proof of (b), recall that the subgroup $S_{p}\cap H$ of $S_{p}$ has index $p^{\ord_{p}(G:H)}$, which is a consequence of (ii) for $D=H$. This implies that the quotient $S_{p}/N^{G}(S_{p}\cap H)$ is an abelian group whose exponent coincides with that of $S_{p}/(S_{p}\cap H)$. Hence the assertion follows from (a). 

(v): This follows from the same argument as (iv) (b) since all subgroups of $S_{p}$ of index $p$ are normal, and hence they contains $S_{p}^{\der}$. 
\end{proof}

Let $G$, $S_{p}$ and $H$ be as in Lemma \ref{sbsd}. Then the exact sequence
\begin{equation*}
0\rightarrow J_{G/S_{p}H}\rightarrow J_{G/H}\rightarrow \Ind_{S_{p}H}^{G}J_{S_{p}H/H}\rightarrow 0,
\end{equation*}
which follows from Lemma \ref{jgcg}, induces the diagram as follows for any subgroup $D$ of $G$: 
\begin{equation}\label{res1}
\xymatrix{
H^{1}(G,\Ind_{S_{p}H}^{G}J_{S_{p}H/H})\ar[r]^{\hspace{7mm}\delta_{G}} \ar[d]^{\Res_{G/D}}& H^{2}(G,J_{G/S_{p}H})\ar[d]^{\Res_{G/D}}\\
H^{1}(D,\Ind_{S_{p}H}^{G}J_{S_{p}H/H})\ar[r]^{\hspace{7mm}\delta_{D}} & H^{2}(D,J_{G/S_{p}H}). 
}
\end{equation}

\begin{dfn}
Keep the notations in Lemma \ref{sbsd}. For a set of subgroups $\cD$ of $G$, put
\begin{equation*}
H^{2}(G,J_{G/S_{p}H})^{(\cD)}:=\{f\in H^{2}(G,J_{G/S_{p}H})\mid \Res_{G/D}(f)\in \Ima(\delta_{D})\text{ for any }D\in \cD \}. 
\end{equation*}
\end{dfn}

\begin{prop}\label{h2s2}
\emph{Keep the notations in Lemma \ref{sbsd}. Let $\cD$ be a set of subgroups of $G$ containing $\cC_{G}$. There is an exact sequence
\begin{equation*}
H^{1}(G,\Ind_{S_{p}H}^{G}J_{S_{p}H/H})\xrightarrow{\delta_{G}} H^{2}(G,J_{G/S_{p}H})^{(\cD)}\xrightarrow{\widehat{\N}_{G}} \Sha_{\cD}^{2}(G,J_{G/H}). 
\end{equation*}
Moreover, the homomorphism $\widehat{\N}_{G}$ is surjective if $\ord_{p}(G:H)=1$. }
\end{prop}

\begin{proof}
By the definition of $H^{2}(G,J_{G/S_{p}H})^{(\cD)}$, the exact sequence
\begin{equation*}
H^{1}(G,\Ind_{S_{p}H}^{G}J_{S_{p}H/H})\xrightarrow{\delta_{G}} H^{2}(G,J_{G/S_{p}H})\xrightarrow{\widehat{\N}_{G}} H^{2}(G,J_{G/H})\rightarrow H^{2}(G,\Ind_{S_{p}H}^{G}J_{S_{p}H/H})
\end{equation*}
induces an exact sequence
\begin{equation*}
H^{1}(G,\Ind_{S_{p}H}^{G}J_{S_{p}H/H})\xrightarrow{\delta_{G}} H^{2}(G,J_{G/S_{p}H})^{(\cD)}\xrightarrow{\widehat{\N}_{G}} \Sha_{\cD}^{2}(G,J_{G/H})\rightarrow \Sha_{\cD}^{2}(G,\Ind_{S_{p}H}^{G}J_{S_{p}H/H}), 
\end{equation*}
which implies the desired exactness. 

In the following, we further assume $\ord_{p}(G:H)=1$. It suffices to give the triviality of $\Sha_{\omega}^{2}(G,\Ind_{S_{p}H}^{G}J_{S_{p}H/H})$. Since $(S_{p}H:H)=p$, one has an isomorphism
\begin{equation*}
\Sha_{\omega}^{2}(G,\Ind_{S_{p}H}^{G}J_{S_{p}H/H}) \cong \Sha_{\omega}^{2}(S_{p}H,J_{S_{p}H/H}) \end{equation*}
by Corollary \ref{ids2}. Hence Proposition \ref{bart} implies that $\Sha_{\omega}^{2}(S_{p}H,J_{S_{p}H/H})$ is zero as desired
\end{proof}

\begin{lem}\label{exh1}
\emph{Keep the notations and assumptions in Lemma \ref{sbsd} (iii) (in particular, $S_{p}$ is abelian). 
\begin{enumerate}
\item There is a commutative diagram
\begin{equation*}
\xymatrix@C=35pt{
0\ar[r]& H^{1}(G,J_{G/S_{p}H})\ar[d]^{\cong} \ar[r]& H^{1}(G,J_{G/H})\ar[d]^{\cong}\ar[r]& 
H^{1}(G,\Ind_{S_{p}H}^{G}J_{S_{p}H/H}) \ar[d]^{\cong} \\
0\ar[r]& \Ker(G^{\vee}\rightarrow (S_{p}H)^{\vee}) \ar[r]& \Ker(G^{\vee}\rightarrow H^{\vee}) \ar[r]^{f\mapsto f\mid_{S_{p}H}\hspace{5mm}}& 
\Ker((S_{p}H)^{\vee}\rightarrow H^{\vee}). }
\end{equation*}
\item The homomorphism $\delta_{G}\colon H^{1}(G,\Ind_{S_{p}H}^{G}J_{S_{p}H/H})\rightarrow H^{2}(G,J_{G/S_{p}H})$ is injective if $S_{p}$ is generated by $[S_{p},G]$ and $S_{p}\cap H$. 
\item The inclusion $S_{p}\subset G$ induces is an isomorphism
\begin{equation*}
\Ker((S_{p}H)^{\vee}\rightarrow H^{\vee})\cong (S_{p}/[S_{p},H]\cdot (S_{p}\cap H))^{\vee}. 
\end{equation*}
In particular, the abelian group $H^{1}(G,\Ind_{S_{p}H}^{G}J_{S_{p}H/H})$ is annihilated by the exponent of $S_{p}$. 
\end{enumerate}}
\end{lem}

\begin{proof}
(i): By Lemma \ref{jgcg}, we have an exact sequence of abelian groups
\begin{align*}
H^{0}(G,\Ind_{S_{p}H}^{G}J_{S_{p}H/H})&\rightarrow H^{1}(G,J_{G/S_{p}H})\\
&\rightarrow H^{1}(G,J_{G/H})\rightarrow H^{1}(G,\Ind_{S_{p}H}^{G}J_{S_{p}H/H}). 
\end{align*}
The assertion follows from Proposition \ref{mnex}. 

In the following, we give proofs of (ii) and (iii). By Lemma \ref{sbsd} (iii), we may assume
\begin{equation*}
H=(S_{p}\cap H)\rtimes H', 
\end{equation*}
where $H'$ is a subgroup of $G'$. Note that we have $S_{p}H=S_{p}\rtimes H'$. 

(ii): It suffices to prove the isomorphy of the homomorphism
\begin{equation*}
\Ker(G^{\vee}\rightarrow (S_{p}H)^{\vee})\rightarrow \Ker(G^{\vee}\rightarrow H^{\vee}), 
\end{equation*}
which is a consequence of (i) and Proposition \ref{h2s2}. By definition, we have isomorphisms
\begin{equation*}
\Ker(G^{\vee}\rightarrow (S_{p}H)^{\vee})\cong \Ker(G/G^{\der}S_{p}H)^{\vee},\quad \Ker(G^{\vee}\rightarrow H^{\vee})\cong \Ker(G/G^{\der}H)^{\vee}. 
\end{equation*}
Moreover, direct computation implies the following: 
\begin{equation*}
G^{\der}S_{p}H=S_{p}\rtimes ((G')^{\der}H'),\quad G^{\der}H=([S_{p},G]\cdot (S_{p}\cap H))\rtimes ((G')^{\der}H'). 
\end{equation*}
Hence the assumption $S_{p}=[S_{p},G]\cdot (S_{p}\cap H)$ follows the desired isomorphy. 

(iii): Since the derived group of $S_{p}H$ equals $[S_{p},H]\rtimes (H')^{\der}$, one has an isomorphism
\begin{align*}
\Ker((S_{p}H)^{\vee}\rightarrow H^{\vee})&\cong ((S_{p}\rtimes H')/(([S_{p},H]\rtimes (H')^{\der})\cdot ((S_{p}\cap H)\rtimes H')))^{\vee}\\
&\cong (S_{p}/([S_{p},H]\cdot (S_{p}\cap H)))^{\vee}. 
\end{align*}
Hence the assertion holds. 
\end{proof}

\begin{lem}\label{shpp}
\emph{Under the same notations and assumptions as Lemma \ref{sbsd} (iii), let $\cD$ be a set of subgroups of $G$ containing $\cC_{G}$. 
\begin{enumerate}
\item The group $H^{2}(G,J_{G/S_{p}H})^{(\cD)}$ contains $\Sha_{\cD}^{2}(G,J_{G/S_{p}H})$. 
\item The composite
\begin{equation*}
\Sha_{\cD}^{2}(G,J_{G/S_{p}H})\hookrightarrow H^{2}(G,J_{G/S_{p}H})^{(\cD)}\xrightarrow{\widehat{\N}_{G}}\Sha_{\cD}^{2}(G,J_{G/H})
\end{equation*}
is injective, and factors through $\Sha_{\cD}^{2}(G,J_{G/H})^{(p)}$ (see Definition \ref{ptpt} for this group). 
\end{enumerate}}
\end{lem}

\begin{proof}
(i): This follows from the definition of $H^{2}(G,J_{G/S_{p}H})^{(\cD)}$. 

(ii): By Lemma \ref{exh1} (iii), the abelian group $H^{1}(G,\Ind_{S_{p}H}^{G}J_{S_{p}H/H})$ is annihilated by a power of $p$. Hence so is the kernel of the homomorphism $\widehat{\N}_{G}\colon H^{2}(G,J_{G/S_{p}H})^{(\cD)}\rightarrow \Sha_{\cD}^{2}(G,J_{G/H})$ by Proposition \ref{h2s2}. Since $\Sha_{\cD}^{2}(G,J_{G/S_{p}H})$ is annihilated by $n=(G:S_{p}H)\notin p\Z$, which follows from Corollary \ref{andg}, the assertion holds. 
\end{proof}

\begin{thm}\label{upts}
\emph{Keep the notations and assumptions in Lemma \ref{sbsd} (iii). Let $\cD$ be a set of subsets of $G$ containing $\cC_{G}$. Assume that we have $\Sha_{\cD}^{2}(G,J_{G/S_{p}H})=\Sha_{\omega}^{2}(G,J_{G/S_{p}H})$. Then there is an isomorphism
\begin{equation*}
\Sha_{\cD}^{2}(G,J_{G/H})^{(p)} \cong \Sha_{\cD}^{2}(G,J_{G/S_{p}H}). 
\end{equation*}}
\end{thm}

\begin{proof}
By Lemma \ref{shpp} (ii), we obtain an injection
\begin{equation*}
\widehat{\N}_{G}\colon \Sha_{\cD}^{2}(G,J_{G/S_{p}H})\hookrightarrow \Sha_{\cD}^{2}(G,J_{G/H})^{(p)}. 
\end{equation*}
In the following, we prove the surjectivity of this map. We may assume $N^{G}(H)$ is trivial by Proposition \ref{ifts}. Moreover, by Lemma \ref{sbsd} (iii), we may further assume that the isomorphism $G\cong S_{p}\rtimes G'$ induces an isomorphism $H\cong (S_{p}\cap H)\rtimes H'$ for some subgroup $H'$ of $G'$. Since $\#S_{p}$ is a power of $p$, Proposition \ref{tsan} (iii) gives an injection
\begin{equation*}
\Res_{G/G'}\colon \Sha_{\cD}^{2}(G,J_{G/H})^{(p)}\hookrightarrow \Sha_{\omega}^{2}(G',J_{G/H}). 
\end{equation*}
Consider an isomorphism of $G'$-modules in Proposition \ref{mcky}: 
\begin{equation*}
\Ind_{H}^{G}\Z \cong \bigoplus_{g\in R(G',H)}\Ind_{G'\cap gHg^{-1}}^{G'}\Z. 
\end{equation*}
Here we may assume $1\in R(G',H)\subset S_{p}\rtimes \{1\}$. Then one has $G' \cap sHs^{-1}\subset H'$ for any $s\in R(G',H)$. Moreover, it is an equality if $s=1$. Hence we obtain an isomorphism of $G'$-modules
\begin{equation*}
J_{G/H} \cong J_{G'/H'} \oplus \left(\bigoplus_{g\in R(G',H)\setminus \{1\}}\Ind_{G'\cap gHg^{-1}}^{G'}\Z \right),
\end{equation*}
which follows from Lemma \ref{endo}. Combining this with Lemma \ref{qtts} and Proposition \ref{ifts}, one has
\begin{equation*}
\Sha_{\omega}^{2}(G',J_{G/H})\cong \Sha_{\omega}^{2}(G',J_{G'/H'})\cong \Sha_{\omega}^{2}(G,J_{G/S_{p}H}). 
\end{equation*}
Therefore, the assumption $\Sha_{\cD}^{2}(G,J_{G/S_{p}H})=\Sha_{\omega}^{2}(G,J_{G/S_{p}H})$ implies that the composite of $\Res_{G/G'}$ and $\widehat{\N}_{G}$ is an isomorphism. This follows the desired surjectivity of $\widehat{\N}_{G}$. 
\end{proof}

\subsection{The $p$-primary torsion parts, I}\label{ptp1}

Here we keep the notations in Lemma \ref{sbsd} (iv). The precise notations are as follows: 
\begin{itemize}
\item $G=S_{p}\rtimes G'$, where $S_{p}$ is a unique $p$-Sylow subgroup of $G$ which is abelian (in particular, $\#G'$ is not a multiple of $p$),
\item $H$ is a subgroup of $G$ satisfying $N^{G}(H)=\{1\}$. 
\end{itemize}

\begin{thm}\label{2dpt}
\emph{Under the above notations, we further assume $S_{p}\cong (\Z/p)^{2}$ and $\ord_{p}(G:H)=1$. 
\begin{enumerate}
\item The abelian group $\Sha_{\omega}^{2}(G,J_{G/H})[p^{\infty}]$ is trivial or isomorphic to $\Z/p$. Moreover, it is non-zero if and only if the following are satisfied: 
\begin{itemize}
\item[(b)] $[S_{p},G]=S_{p}$, 
\item[(c)] $N_{G}(S_{p}\cap H)=Z_{G}(S_{p}\cap H)$. 
\end{itemize}
\item Assume that the conditions (b) and (c) in (i) are satisfied. Let $\cD$ be a set of subsets of $G$ containing $\cC_{G}$. Then there is an isomorphism
\begin{equation*}
\Sha_{\cD}^{2}(G,J_{G/H})[p^{\infty}] \cong 
\begin{cases}
0&\text{if there is $D \in \cD$ which contains $S_{p}$,}\\
\Z/p&\text{otherwise. }
\end{cases}
\end{equation*}
\end{enumerate}}
\end{thm}

\begin{lem}\label{smtr}
\emph{Retain the notations and assumptions in the beginning of Section \ref{ptp1}. For any subgroup $D$ of $G$, there is an isomorphism
\begin{equation*}
H^{2}(D,J_{G/S_{p}H})[p^{\infty}] \cong \Coker\left(D^{\vee}\rightarrow \bigoplus_{g\in R(D,S_{p}H)}(D\cap g(S_{p}H)g^{-1})^{\vee}\right)[p^{\infty}],
\end{equation*}
where $R(D,S_{p}H)$ is a complete representative of $D\backslash G/S_{p}H$ in $G$. }
\end{lem}

\begin{proof}
By Proposition \ref{mcky}, there is an isomorphism of $D$-modules
\begin{equation*}
\Ind_{S_{p}H}^{G}\Z \cong \bigoplus_{g\in R(D,S_{p}H)}\Ind_{D\cap g(S_{p}H)g^{-1}}^{D}\Z. 
\end{equation*}
Combining this isomorphism with Proposition \ref{mnex} (ii), we obtain an exact sequence 
\begin{align*}
D^{\vee}&\rightarrow \bigoplus_{g\in R(D,S_{p}H)}(D\cap g(S_{p}H)g^{-1})^{\vee}\rightarrow H^{2}(D,J_{G/S_{p}H}) \\
&\rightarrow H^{2}(D,\Q/\Z) \rightarrow \bigoplus_{g\in R(D,S_{p}H)}H^{2}(D\cap g(S_{p}H)g^{-1},\Q/\Z).  
\end{align*}
On the other hand, $D\cap S_{p}$ is the unique $p$-Sylow subgroup of $D\cap g(S_{p}H)g^{-1}$ for any $g\in R(D,S_{p}H)$. Hence it suffices to prove that the composite
\begin{equation*}
H^{2}(D,\Q/\Z)[p^{\infty}] \hookrightarrow H^{2}(D,\Q/\Z) \xrightarrow{\Res_{D/(D\cap S_{p})}} H^{2}(D\cap S_{p},\Q/\Z)
\end{equation*}
is injective. By Lemma \ref{sbsd} (iii) and Proposition \ref{gpac}, we may assume $D=(D\cap S_{p})\rtimes D'$, where $D'$ is a subgroup of $G'$. Put
\begin{equation*}
H^{2}(D,\Q/\Z)_{2}:=\Ker\left(H^{2}(D,\Q/\Z) \xrightarrow{\Res_{D/D'}}H^{2}(D',\Q/\Z)\right). 
\end{equation*}
It contains $H^{2}(D,\Q/\Z)[p^{\infty}]$ since $p$ does not divide the order of $D'$. On the other hand, \cite[Theorem 2 (I\hspace{-0.5mm}I)]{Tahara1972} gives the following exact sequence: 
\begin{equation*}
H^{1}(D',(D\cap S_{p})^{\vee})\rightarrow H^{2}(D,\Q/\Z)_{2} \xrightarrow{\Res_{D/(D\cap S_{p})}} H^{2}(D\cap S_{p},\Q/\Z). 
\end{equation*}
The group $H^{1}(D',(D\cap S_{p})^{\vee})$ vanishes by the assumption $p\nmid \#G'$. Hence we obtain the desired injectivity. 
\end{proof}

\begin{prop}\label{dlgc}
\emph{Keep the notations in Lemma \ref{smtr}. Let $D$ be a subgroup of $G$. 
\begin{enumerate}
\item There is a commutative diagram
\begin{equation*}
\xymatrix{
H^{1}(D,\Ind_{S_{p}H}^{G}J_{S_{p}H/H})[p^{\infty}] \ar[d]^{\cong}\ar[r] & H^{2}(D,\Ind_{S_{p}H}^{G}\Z)[p^{\infty}] \ar[d]^{\cong}\ar[r] & H^{2}(D,J_{G/S_{p}H})[p^{\infty}] \ar[d]^{\cong} \\
\bigoplus_{g\in R(D,S_{p}H)}C_{D,g}[p^{\infty}] \ar[r]^{\hspace{8mm}\iota_{D}}& C_{D,S_{p}H}[p^{\infty}] \ar[r]& \Coker(D^{\vee}\rightarrow C_{D,S_{p}H})[p^{\infty}]. }
\end{equation*}
Here we use the notations as follows: 
\begin{itemize}
\item $R(D,S_{p}H)$ is a complete representative of $D\backslash G/S_{p}H$ in $G$,
\item $C_{D,S_{p}H}:=\bigoplus_{g\in R(D,S_{p}H)}(D\cap g(S_{p}H)g^{-1})^{\vee}$,
\item for each $g\in R(D,S_{p}H)$, a subset $R(D,g)$ of $S_{p}H$ is a complete representative of $(g^{-1}Dg \cap S_{p}H)\backslash S_{p}H/H$, and $C_{D,g}$ is the kernel of the canonical homomorphism
\begin{equation*}
(D\cap g(S_{p}H)g^{-1})^{\vee}\rightarrow \bigoplus_{h \in R(D,g)}(D\cap \Ad(gh)H)^{\vee};f\mapsto (f\!\mid_{D\cap \Ad(gh)H})_{h}. 
\end{equation*}
\end{itemize}
Furthermore, the composite of the upper horizontal maps coincides with $\delta_{D}$. 
\item Assume that $D$ is contained in $S_{p}$. Then there is a commutative diagram
\begin{equation*}
\xymatrix@C=40pt{
H^{2}(G,\Ind_{S_{p}H}^{G}\Z)[p^{\infty}] \ar[d]^{\cong}\ar[r]^{\Res_{G/D}} & H^{2}(D,\Ind_{S_{p}H}^{G}\Z)[p^{\infty}] \ar[d]^{\cong} \\
C_{G,S_{p}H}[p^{\infty}] \ar[r]^{f\mapsto (\Ad(g^{-1})\mid_{D})_{g}}& C_{D,S_{p}H}[p^{\infty}]. }
\end{equation*}
\end{enumerate}}
\end{prop}

\begin{proof}
(i): By Lemma \ref{jgcg}, the homomorphism $\delta_{D}$ is factored as
\begin{equation*}
H^{1}(D,\Ind_{S_{p}H}^{G}J_{S_{p}H/H})\rightarrow H^{2}(D,\Ind_{S_{p}H}^{G}\Z)\rightarrow H^{2}(D,J_{G/S_{p}H}). 
\end{equation*}
First, we prove the left-hand side of the commutative diagram. By Proposition \ref{mcky}, there are isomorphisms of $D$-modules
\begin{align*}
\Ind_{S_{p}H}^{G}\Z&\cong \bigoplus_{g\in R(D,S_{p}H)}\Ind_{D\cap g(S_{p}H)g^{-1}}^{D}\Z, \\
\Ind_{S_{p}H}^{G}J_{S_{p}H/H}&\cong \bigoplus_{g\in R(D,S_{p}H)}\Ind_{D\cap g(S_{p}H)g^{-1}}^{D}(J_{S_{p}H/H})^{g}\\
&\cong \bigoplus_{g\in R(D,S_{p}H)}\Ind_{D\cap g(S_{p}H)g^{-1}}^{D}J_{g(S_{p}H)g^{-1}/gHg^{-1}}. 
\end{align*}
Combining these with Shapiro's lemma, we obtain a commutative diagram
\begin{equation*}
\xymatrix{
H^{1}(D,\Ind_{HS_{p}}^{G}J_{S_{p}H/H})\ar[r] \ar[d]^{\cong} & H^{2}(D,\Ind_{S_{p}H}^{G}\Z)\ar[d]^{\cong}\\
\bigoplus_{g\in R(D,S_{p}H)}H^{1}(D\cap g(S_{p}H)g^{-1},J_{g(S_{p}H)g^{-1}/gHg^{-1}})\ar[r] & \bigoplus_{g\in R(D,S_{p}H)}H^{2}(D\cap g(S_{p}H)g^{-1},\Z). }
\end{equation*}
Here the lower homomoprhism is the direct sum of the connecting homomorphism induced by the exact sequence of $D\cap g(S_{p}H)g^{-1}$-modules
\begin{equation*}
0 \rightarrow \Z \rightarrow \Ind_{gHg^{-1}}^{g(S_{p}H)g^{-1}}\Z \rightarrow J_{g(S_{p}H)g^{-1}/gHg^{-1}} \rightarrow 0. 
\end{equation*}
See Definition \ref{jgdf}. Hence it suffices to prove the commutativity of the following diagram for any $g\in R(D,S_{p}H)$: 
\begin{equation}\label{dcf1}
\xymatrix{
H^{1}(D\cap g(S_{p}H)g^{-1},J_{g(S_{p}H)g^{-1}/gHg^{-1}})\ar[r] \ar[d]^{\cong} & H^{2}(D\cap g(S_{p}H)g^{-1},\Z)\ar[d]^{\cong}\\
C_{D,g}\ar[r] & (D\cap g(S_{p}H)g^{-1})^{\vee}. }
\end{equation}
Fix $g\in R(D,S_{p}H)$. Since every $h\in R(D,g)$ satisfies
\begin{equation*}
\Ad(ghg^{-1})(gHg^{-1})=(gh)H(gh)^{-1}\subset g(S_{p}H)g^{-1},
\end{equation*}
we obtain an isomorphism of $(D\cap g(S_{p}H)g^{-1})$-modules by Proposition \ref{mcky}: 
\begin{equation*}
\Ind_{gHg^{-1}}^{g(S_{p}H)g^{-1}}\Z \cong \bigoplus_{h \in R(D,g)}\Ind_{D\cap (gh)H(gh)^{-1}}^{D\cap g(S_{p}H)g^{-1}}\Z. 
\end{equation*}
Hence Proposition \ref{mnex} (ii) induces the desired commutative diagram \eqref{dcf1}. 

On the other hand, the right-hand side of the diagram is a consequence of Proposition \ref{mnex} (ii) for the exact sequence of $D$-modules
\begin{equation*}
0\rightarrow \Z \rightarrow \Ind_{S_{p}H}^{G}\Z \rightarrow J_{G/S_{p}H} \rightarrow 0. 
\end{equation*}
This completes the proof of (i). 

(ii): This follows from Corollary \ref{rsch}. 
\end{proof}

\begin{lem}\label{rdsp}
\emph{Under the same assumption in the beginning of Section \ref{ptp1}, let $\cD$ be a set of subgroups of $G$ containing $\cC_{G}$. Assume that every $D\in \cD$ does not contain $S_{p}$. Then the abelian group $H^{2}(G,J_{G/S_{p}H})^{(\cD)}[p^{\infty}]$ coincide with the subgroup of $H^{2}(G,J_{G/S_{p}H})$ as follows: 
\begin{equation*}
M_{S_{p}H}:=\!\left\{f\in H^{2}(G,J_{G/S_{p}H})[p^{\infty}] \,\middle|\, \Res_{G/D}(f)\in \Ima(\delta_{D})\text{ for any }D\in \cD_{S_{p},H}\right\}\!.
\end{equation*}
Here $\cD_{S_{p},H}:=\{D\cap S_{p}\subset S_{p}\mid D\in \cD,D\cap (S_{p}\cap H)\neq \{1\}\}$. }
\end{lem}

\begin{proof}
It is clear that $H^{2}(G,J_{G/H})[p^{\infty}]$ is contained in $M_{S_{p}H}$. For the reverse inclusion, it suffices to prove the following. 
\begin{itemize}
\item[(I)] For $f\in H^{2}(G,J_{G/H})[p^{\infty}]$, we have $\Res_{G/D}(f)\in \Ima(\delta_{D})$ if and only if $\Res_{G/(D\cap S_{p})}(f)\in \Ima(\delta_{D\cap S_{p}})$. 
\item[(I\hspace{-0.5mm}I)] Let $f\in H^{2}(G,J_{G/H})[p^{\infty}]$, $D\in \cD$ and $g\in G$. Then we have $\Res_{G/D}(f)\in \Ima(\delta_{D})$ if and only if $\Res_{G/gDg^{-1}}(f)\in \Ima(\delta_{gDg^{-1}})$. 
\item[(I\hspace{-0.5mm}I\hspace{-0.5mm}I)] If $D\cap g^{-1}(S_{p}\cap H)g=\{1\}$ for any $g\in G$, then the map $\delta_{D\cap S_{p}}$ is surjective. 
\end{itemize}

\textbf{Proof of (I): }It suffices to prove the assertion $\Res_{G/D}(f)\in \Ima(\delta_{D})$ under the assumption $\Res_{G/(D\cap S_{p})}(f)\in \Ima(\delta_{D\cap S_{p}})$. Suppose that an element $f_{D\cap S_{p}}$ of $H^{1}(D\cap S_{p},\Ind_{S_{p}H}^{G}J_{S_{p}H/H})$ satisfies $\delta_{D\cap S_{p}}(f_{D\cap S_{p}})=\Res_{G/(D\cap S_{p})}(f)$. On the other hand, we have $\#(D\cap S_{p})=p$ and $(D:D\cap S_{p})\notin p\Z$ since $S_{p}$ is normal in $G$. Therefore we can define an element
\begin{equation*}
\widetilde{f}_{D\cap S_{p}}:=(D:D\cap S_{p})^{-1}\sum_{d\in D/(D\cap S_{p})}d(f_{D\cap S_{p}})\in H^{1}(D\cap S_{p},\Ind_{S_{p}H}^{G}J_{S_{p}H/H}). 
\end{equation*}
It is fixed under the action of $D$ and $\delta_{D}(\widetilde{f}_{D\cap S_{p}})=\Res_{G/(D\cap S_{p})}(f)$. 

Now we prove that the homomorphism
\begin{equation}\label{hcsc}
\Res_{D/(D\cap S_{p})}\colon H^{1}(D,\Ind_{S_{p}H}^{G}J_{S_{p}H/H})[p^{\infty}]\rightarrow H^{1}(D\cap S_{p},\Ind_{S_{p}H}^{G}J_{S_{p}H/H})^{D}. 
\end{equation}
is an isomorphism. By the Hochshild--Serre spectral sequence, we obtain an exact sequence
\begin{align*}
0&\rightarrow H^{1}(D/(D\cap S_{p}),(\Ind_{S_{p}H}^{G}J_{S_{p}H/H})^{D\cap S_{p}})\xrightarrow{\Inf_{D/(D\cap S_{p})}} H^{1}(D,\Ind_{S_{p}H}^{G}J_{S_{p}H/H})\\
&\xrightarrow{\Res_{D/(D\cap S_{p})}} H^{1}(D\cap S_{p},\Ind_{S_{p}H}^{G}J_{S_{p}H/H})^{D}\rightarrow H^{2}(D/(D\cap S_{p}),(\Ind_{S_{p}H}^{G}J_{S_{p}H/H})^{D\cap S_{p}}). 
\end{align*}
Note that the abelian group $H^{1}(D\cap S_{p},\Ind_{S_{p}H}^{G}J_{S_{p}H/H})^{D}$ has order a power of $p$ since $D\cap S_{p}$ is a $p$-group. On the other hand, for any $j\in \{1,2\}$, the integer $(D:D\cap S_{p})$ annihilates $H^{j}(D/(D\cap S_{p}),(\Ind_{S_{p}H}^{G}J_{S_{p}H/H})^{D\cap S_{p}})$. Hence the isomorphy of \eqref{hcsc} follows from the fact that $(D:D\cap S_{p})$ is not divisible by $p$, which is a consequence of Lemma \ref{sbsd} (ii). 

By the isomorphy of \eqref{hcsc}, there is a unique $\widetilde{f}_{D}\in H^{1}(D,\Ind_{S_{p}H}^{G}J_{S_{p}H/H})[p^{\infty}]$ so that
\begin{equation*}
\Res_{D/(D\cap S_{p})}(\widetilde{f}_{D})=\widetilde{f}_{D\cap S_{p}}. 
\end{equation*}
Then we have the following: 
\begin{align*}
(\Res_{D/(D\cap S_{p})}\circ \delta_{D})(\widetilde{f}_{D})&=(\delta_{D\cap S_{p}}\circ \Res_{D/(D\cap S_{p})})(\widetilde{f}_{D})\\
&=\Res_{G/(D\cap S_{p})}(f)=(\Res_{D/(D\cap S_{p})}\circ \Res_{G/D})(f). 
\end{align*}
On the other hand, since $p\nmid (D:D\cap S_{p})$, we obtain the injectivity of the composite
\begin{equation*}
H^{1}(D,\Ind_{S_{p}H}^{G}\Z)[p^{\infty}]\hookrightarrow H^{1}(D,\Ind_{S_{p}H}^{G}\Z) \xrightarrow{\Res_{D/(D\cap S_{p})}}H^{1}(D\cap S_{p},\Ind_{S_{p}H}^{G}\Z)
\end{equation*}
by Proposition \ref{tsan} (ii). Consequently, the equality $\delta_{D}(\widetilde{f}_{D})=\Res_{G/D}(f)$ holds. 

\textbf{Proof of (I\hspace{-0.5mm}I): }
This is a consequence of Proposition \ref{gpac} for $j=1$ and the exact sequence
\begin{equation*}
0\rightarrow J_{G/S_{p}H}\rightarrow J_{G/H}\rightarrow \Ind_{S_{p}H}^{G}J_{S_{p}H/H}\rightarrow 0. 
\end{equation*}

\textbf{Proof of (I\hspace{-0.5mm}I\hspace{-0.5mm}I): }
The assertion follows from Proposition \ref{dlgc} (i) since we have
\begin{equation*}
(D\cap S_{p})\cap (gh)H(gh)^{-1}=D\cap g(H\cap S_{p})g^{-1}=\{1\}
\end{equation*}
for any $g\in R(D,HS_{p})$ and $s\in R(D\cap S_{p},g)$. 
\end{proof}

\begin{lem}\label{csds}
\emph{Let $G$, $H$ and $S_{p}$ be as in the beginning of Section \ref{ptp1}. Then there are isomorphisms
\begin{equation*}
\xymatrix{
C_{G,1} \ar[d]^{\cong} \ar[r]^{\iota_{G}}& C_{G,S_{p}H}[p^{\infty}] \ar[d]^{\cong} \ar[r]& \Coker(G^{\vee}\rightarrow C_{G,S_{p}H})[p^{\infty}] \ar[d]^{\cong}\\
(S_{p}/[S_{p},H]\cdot (S_{p}\cap H))^{\vee} \ar[r]& (S_{p}/[S_{p},H])^{\vee} \ar[r]& (S_{p}/[S_{p},H])^{\vee}/(S_{p}/[S_{p},G]^{\vee}). }
\end{equation*}
Here the vertical isomorphisms are induces by the natural inclusion $S_{p}\subset S_{p}H$. }
\end{lem}

\begin{proof}
Recall Lemma \ref{exh1} (iii) that there is an isomorphism $C_{G,1}\cong (S_{p}/[S_{p},H]\cdot (S_{p}\cap H))$. On the other hand, we have $C_{G,S_{p}H}=(S_{p}H)^{\vee}$ by definition, and this induces an equality $\Coker(G^{\vee}\rightarrow C_{G,S_{p}H})=\Coker(G^{\vee}\rightarrow (S_{p}H)^{\vee})$. Moreover, the inclusion $S_{p}\subset S_{p}H$ induces isomorphisms
\begin{equation*}
(S_{p}H)^{\vee}[p^{\infty}]\cong (S_{p}/[S_{p},H])^{\vee},\quad \Coker(G^{\vee}\rightarrow (S_{p}H)^{\vee})[p^{\infty}]\cong (S_{p}/[S_{p},H])^{\vee}/(S_{p}/[S_{p},G])^{\vee}. 
\end{equation*}
Hence the assertion (ii) holds. 
\end{proof}

\begin{prop}\label{lciv}
\emph{Keep the notations and assumptions in the beginning of Section \ref{ptp1}. Let $\widetilde{M}_{S_{p}H}^{(\cD)}$ the preimage of $H^{2}(G,J_{G/S_{p}H})^{(\cD)}$ under the homomorphism $H^{2}(G,\Ind_{S_{p}H}^{G}\Z)[p^{\infty}]\rightarrow H^{2}(G,J_{G/S_{p}H})$ in Proposition \ref{dlgc} (i). 
\begin{enumerate}
\item There is an isomorphism of abelian groups
\begin{equation*}
H^{2}(G,J_{G/S_{p}H})^{(\cD)}[p^{\infty}]\cong \widetilde{M}_{S_{p}H}^{(\cD)}/(S_{p}/[S_{p},G])^{\vee}. 
\end{equation*}
\item The isomorphism $H^{2}(G,J_{G/S_{p}H})\cong (S_{p}/[S_{p},H])^{\vee}$, which follows from Proposition \ref{dlgc} (i) and Lemma \ref{csds}, induces an isomorphism
\begin{equation*}
\widetilde{M}_{S_{p}H}^{(\cD)} \cong \{f\in (S_{p}/[S_{p},H])^{\vee}\mid ((f\circ \Ad(g^{-1}))\!\mid_{D})_{g}\in \Ima(\iota_{D})+\Delta_{D}\text{ for all }D\in \cD_{S_{p},H}\}. 
\end{equation*}
Here $\Delta_{D}$ is the image of the homomorphism
\begin{equation*}
D^{\vee} \rightarrow \bigoplus_{g\in R(D,S_{p}H)}D^{\vee}; f\mapsto (f\!\mid_{D})_{g}
\end{equation*}
%\item If $\cD_{S_{p},H}$ contains $S_{p}$, then the right-hand side of \eqref{mtch} is trivial. 
\item Assume that there is an isomorphism $S_{p} \cong (\Z/p)^{2}$ and $\cD_{S_{p},H}=\{S_{p}\cap H\}$. Then the abelian group $\widetilde{M}_{S_{p}H}^{(\cD)}$ coincides with $(S_{p}/[S_{p},H]\cdot [S_{p}\cap H,N_{G}(S_{p}\cap H)])^{\vee}$. 
\end{enumerate}}
\end{prop}

\begin{proof}
(i): By Lemma \ref{csds} and Proposition \ref{dlgc} (i), there is an exact sequence
\begin{equation*}
0\rightarrow (S_{p}/[S_{p},G])^{\vee}\rightarrow H^{2}(G,\Ind_{S_{p}H}^{G}\Z)[p^{\infty}]\rightarrow H^{2}(G,J_{G/S_{p}H})[p^{\infty}] \rightarrow 0. 
\end{equation*}
Since the order of $(S_{p}/[S_{p},G])^{\vee}$ is a power of $p$, we obtain the desired isomorphism in (i). 

(ii): We identify $H^{2}(G,\Ind_{H}^{G}\Z)$ and $(S_{p}/[S_{p},H])^{\vee}$. Then Proposition \ref{dlgc} (ii), Lemmas \ref{rdsp}, \ref{csds} and (i) imply that $f\in (S_{p}/[S_{p},H])^{\vee}$ is contained in $\widetilde{M}_{S_{p}H}^{(\cD)}$ if and only if $(f\circ \Ad(g^{-1})\mid_{D\cap g(S_{p}H)g^{-1}})$ is contained in the sum of $\Ima(\iota_{D})$ and the kernel of the homomorphism
\begin{equation*}
D^{\vee}\rightarrow (D\cap g(S_{p}H)g^{-1})^{\vee};f\mapsto (f\!\mid_{D\cap g(S_{p}H)g^{-1}})
\end{equation*}
for every $D\in \cD_{S_{p},H}$. On the other hand, any $D\in \cD_{S_{p},H}$ satisfies $D\cap g(S_{p}Hg^{-1})=D$ for all $g\in G$ since $D$ is contained in $S_{p}$. In particular, one has an equality $C_{D,S_{p}H}=\bigoplus_{R(D,HS_{p})}D^{\vee}$ under the notation in Proposition \ref{dlgc} (i). Moreover, Proposition \ref{mnex} implies an exact sequence of $p$-abelian groups
\begin{equation*}
0\rightarrow D^{\vee} \xrightarrow{f\mapsto (f)_{R(D,S_{p}H)}}\bigoplus_{R(D,S_{p}H)}D^{\vee}\rightarrow \Coker\left(D^{\vee}\rightarrow \bigoplus_{R(D,S_{p}H)}D^{\vee}\right)\rightarrow 0. 
\end{equation*}
Therefore the assertion (ii) holds. 

(iii): Since $S_{p}\cong (\Z/p)^{2}$ and $S_{p}\cap H \cong \Z/p$, one has an equality
\begin{equation*}
C_{S_{p}\cap H,g}=
\begin{cases}
(S_{p}\cap H)^{\vee}&\text{if }g\in N_{G}(S_{p}\cap H),\\
\{0\}&\text{if }g\notin N_{G}(S_{p}\cap H). 
\end{cases}
\end{equation*}
This implies an equality
\begin{equation*}
\Ima(\iota_{S_{p}\cap H})+\Delta_{S_{p}\cap H}=\{f\in (S_{p}/[S_{p},H])^{\vee}\mid f\circ \Ad(g^{-1})\mid_{S_{p}\cap H}=f\!\mid_{S_{p}\cap H}\}. 
\end{equation*}
The right-hand side is isomorphic to $S_{p}/([S_{p},H]\cdot [S_{p}\cap H,N_{G}(S_{p}\cap H)])$, which implies (iii). 
\end{proof}

\begin{lem}\label{msck}
\emph{Continue the notations in the beginning of Section \ref{ptp1}. We further assume that $S_{p}$ has exponent $p$. If $[S_{p},H]$ contains $S_{p}\cap H$, then we have $N_{G}(S_{p}\cap H)\neq Z_{G}(S_{p}\cap H)$. }
\end{lem}

\begin{proof}
Recall that there is an isomorphism $G\cong S_{p}\rtimes G'$ and $p\nmid \#G'$. Moreover, there is an isomorphism $S_{p}\cong (\Z/p)^{m}$ for some $m\in \Zpn$ by Lemma \ref{sbsd} (v). We regard $S_{p}$ as a representation of $G'$ over $\Fp$. By Lemma \ref{sbsd} (iii), we may assume $H\cong (S_{p}\cap H)\rtimes H'$, where $H'$ is a subgroup of $G'$. Since $\#H'\notin p\Z$, Maschke's theorem (cf.~\cite[Chap.~1, Theorem 1]{Serre1977}) implies that there is a subgroup $S'_{p}$ of $S_{p}$ which is stable under $H'$ such that $S_{p}=(S_{p}\cap H)\times S'_{p}$. Under this decomposition, we have the following: 
\begin{equation}\label{cidc}
[S_{p},H]=[S_{p}\cap H,H]\times [S'_{p},H]. 
\end{equation}

Assume $S_{p}\cap H\subset [S_{p},H]$. By \eqref{cidc}, one has an equality $[S_{p}\cap H,H]=S_{p}\cap H$. This implies the existence of $h\in H$ and $s\in S_{p}\cap H$ so that $hsh^{-1}s^{-1}\neq 1$. Hence we have $h\in H \setminus Z_{G}(S_{p}\cap H)$, in particular the desired assertion holds. 
\end{proof}

\begin{proof}[Proof of Theorem \ref{2dpt}]
It suffices to prove the following: 
\begin{itemize}
\item[(I)] if $S_{p}\subset D$ for some $D\in \cD$, then $\Sha_{\cD}^{2}(G,J_{G/H})[p^{\infty}]=0$. 
\item[(I\hspace{-0.5mm}I)] if $\cD$ does not satisfy the assumption in (I), then one has
\begin{equation*}
\Sha_{\cD}^{2}(G,J_{G/H})[p^{\infty}] \cong 
\begin{cases}
\Z/p &\text{if (b) and (c) hold, }\\
0 &\text{otherwise. }
\end{cases}
\end{equation*}
\end{itemize}

\textbf{Proof of (I): }Assume that $D\in \cD$ contains $S_{p}$. It suffices to prove that the composite
\begin{equation*}
H^{2}(G,J_{G/S_{p}H})[p^{\infty}] \hookrightarrow H^{2}(G,J_{G/S_{p}H})\xrightarrow{\Res_{G/S_{p}}} 
H^{2}(S_{p},J_{G/S_{p}H})
\end{equation*}
is injective. This follows from Proposition \ref{tsan} (ii) for $\cD=\emptyset$. 

\textbf{Proof of (I\hspace{-0.5mm}I): }By Lemma \ref{sbsd} (iv) (a) and $S_{p}\cong (\Z/p)^{2}$, the group $[S_{p},G]$ is non-trivial and $S_{p}=[S_{p},G]\cdot (S_{p}\cap H)$. Hence, Proposition \ref{h2s2} and Lemma \ref{exh1} (ii) imply an exact sequence
\begin{equation*}
0\rightarrow H^{1}(G,\Ind_{S_{p}H}^{G}J_{S_{p}H/H})[p^{\infty}]\xrightarrow{\delta_{G}} H^{2}(G,J_{G/S_{p}H})^{(\cD)}[p^{\infty}] \xrightarrow{\widehat{\N}_{G}} \Sha_{\cD}^{2}(G,J_{G/H})[p^{\infty}] \rightarrow 0. 
\end{equation*}
This sequence can be written as follows by Lemma \ref{exh1} (iii) and Proposition \ref{lciv} (i): 
\begin{equation*}
0\rightarrow (S_{p}/[S_{p},H]\cdot (S_{p}\cap H))^{\vee} \rightarrow \widetilde{M}_{S_{p}H}/(S_{p}/[S_{p},G])^{\vee} \rightarrow \Sha_{\cD}^{2}(G,J_{G/H})[p^{\infty}] \rightarrow 0. 
\end{equation*}
Furthermore, since we have $\cD_{S_{p}\,H}=\{S_{p}\cap H\}$ by assumption, one has an isomorphism by Proposition \ref{lciv} (iii): 
\begin{equation*}
\widetilde{M}_{S_{p}H}\cong (S_{p}/[S_{p},H]\cdot [S_{p}\cap H,N_{G}(S_{p}\cap H)])^{\vee}. 
\end{equation*}

\textbf{Case 1.~$N_{G}(S_{p}\cap H)\neq Z_{G}(S_{p}\cap H)$. }

Since $S_{p}\cap H\cong \Z/p$, the assumption implies an equality $[S_{p}\cap H,N_{G}(S_{p}\cap H)]=S_{p}\cap H$. Therefore the homomorphism $\delta_{G}$ is an isomorphism. This follows the desired assertion
\begin{equation*}
\Sha_{\cD}^{2}(G,J_{G/H})[p^{\infty}]=0. 
\end{equation*}

\textbf{Case 2.~$N_{G}(S_{p}\cap H)=Z_{G}(S_{p}\cap H)$. }

In this case, the group $[S_{p}\cap H,N_{G}(S_{p}\cap H)]$ is trivial. Moreover, the non-triviality of $[S_{p},G]$ induce an isomorphism
\begin{equation*}
(S_{p}/[S_{p},G])^{\vee} \cong 
\begin{cases}
0&\text{if }[S_{p},G]=S_{p},\\
\Z/p&\text{if }[S_{p},G]\neq S_{p},
\end{cases}
\end{equation*}

First, suppose that $[S_{p},H]$ is trivial. Then there is an isomorphism
\begin{equation*}
(S_{p}/[S_{p},H]\cdot (S_{p}\cap H))^{\vee}\cong \Z/p. 
\end{equation*}
On the other hand, the following holds: 
\begin{equation*}
\widetilde{M}_{S_{p}H}/(S_{p}/[S_{p},G])^{\vee}\cong S_{p}^{\vee}/(S_{p}/[S_{p},G])^{\vee} \cong 
\begin{cases}
(\Z/p)^{2} &\text{if }[S_{p},G]=S_{p},\\
\Z/p &\text{if }[S_{p},G]\neq S_{p}. 
\end{cases}
\end{equation*}
Therefore, we obtain the desired isomorphism on $\Sha_{\cD}^{2}(G,J_{G/H})$. 

Second, assume that $[S_{p},H]$ is non-trivial. By Lemma \ref{msck}, the group $[S_{p},H]$ does not contain $S_{p}\cap H$. In particular, the group $[S_{p},H]$ has order $p$ and $S_{p}/([S_{p},H]\cdot (S_{p}\cap H))$ is trivial. Moreover, one has an isomorphism
\begin{equation*}
\widetilde{M}_{S_{p}H}/(S_{p}/[S_{p},G])^{\vee}\cong 
(S_{p}/[S_{p},H])^{\vee}/(S_{p}/[S_{p},G])^{\vee}
\cong 
\begin{cases}
\Z/p&\text{if }[S_{p},G]=S_{p},\\
0&\text{if }[S_{p},G]\neq S_{p}. 
\end{cases}
\end{equation*}
This induces the desired isomorphism on $\Sha_{\cD}^{2}(G,J_{G/H})$, which completes the proof in Case 2. 
\end{proof}

\subsection{The $p$-primary torsion parts, I\hspace{-0.5mm}I}\label{ptp2}

Here we prove the vanishing of the $p$-primary torsion part of $\Sha_{\omega}^{2}(G,J_{G/H})$ for certain pair of finite groups $G$ and their subgroups $H$. 

\begin{thm}\label{pptt}
\emph{Let $G$ be a finite group, and $H$ a subgroup of $G$. Assume that at least one of the following is satisfied: 
\begin{enumerate}
\item $p>2$ and $(G:H)=2p$, 
\item a $p$-Sylow subgroup $S_{p}$ of $G$ is normal, $\ord_{p}(G:H)=1$ and $\ord_{p}\#S_{p}\neq 2$. 
\end{enumerate}
Then the abelian group $\Sha^{2}_{\omega}(G,J_{G/H})[p^{\infty}]$ is trivial. }
\end{thm}

In the following, we give a proof of Theorem \ref{pptt}. We need a preparation on finite groups for the assumption (i). 

\begin{lem}\label{cspp}
\emph{Let $G$ be a finite group, and $p$ a prime divisor of $\#G$. Take a $p$-Sylow subgroup $S_{p}$ of $G$, and let $H$ be a subgroup of $G$ with $\ord_{p}(G:H)=1$. If $(G:H)\leq p(p-1)$, then one has $\#(S_{p}\backslash G/H)=(G:H)/p$ and $S_{p}\cap gHg^{-1}$ is a normal subgroup of $S_{p}$ of index $p$ for any $g\in G$. }
\end{lem}

\begin{proof}
Take any $g\in G$. Then the condition $p\mid (G:H)$ implies that $(S_{p}:S_{p}\cap gHg^{-1})$ is divisible by $p$. Moreover, it is a power of $p$ since $S_{p}$ is a $p$-group. Hence we obtain that the order of $S_{p}gH$ is a multiple of $\#H$ and a positive power of $p$. Combining this with the equality
\begin{equation*}
G=\coprod_{g\in R(S_{p},H)}S_{p}gH,
\end{equation*}
one has $\#(S_{p}\backslash G/H)=(G:H)/p$ and $\#(S_{p}gH)=p\#H$ for all $g\in G$. In particular, by Lemma \ref{doub}, we obtain an equality
\begin{equation*}
(S_{p}:S_{p}\cap gHg^{-1})=\frac{\#(S_{p}gH)}{\#H}=\frac{p\#H}{\#H}=p
\end{equation*}
for every $g\in G$. This completes the proof of Lemma \ref{cspp}. 
\end{proof}

\begin{proof}[Proof of Theorem \ref{pptt}]
Take a $p$-Sylow subgroup $S_{p}$ of $G$. By Proposition \ref{tsan} (i), it suffices to prove that $\Sha_{\cD}^{2}(S_{p},J_{G/H})$ is zero. 

First, assume the condition (i). We may assume $\cD=\cC_{G}$. In this case, Lemma \ref{cspp} implies that there is $g\in G$ which satisfies the following: 
\begin{itemize}
\item $G=S_{p}H\sqcup S_{p}gH$, 
\item $(S_{p}:S_{p}\cap H)=(S_{p}:S_{p}\cap gHg^{-1})=p$. 
\end{itemize}
In particular, there is an isomorphism of $S_{p}$-modules
\begin{equation*}
J_{G/H} \cong J_{S_{p}/(S_{p}\cap H,S_{p}\cap gHg^{-1})}. 
\end{equation*}
Hence we have $\Sha_{\omega}^{2}(S_{p},J_{G/H})=0$ by Theorem \ref{blpl} for $r\leq 2$. 

Second, we prove the assertion under the condition (ii). By Lemma \ref{sbsd} (iii), there is an isomorphism $S_{p}\cong (\Z/p)^{m}$, where $m:=\ord_{p}\#G$. Take a complete representative $\{g_1,\ldots,g_r\}$ of $S_{p}\backslash G/H$ in $G$. For each $i\in \{1,\ldots,r\}$, the subgroup $S_{p}\cap g_iHg_i^{-1}$ of $S_{p}$ has index $p$ by Lemma \ref{sbsd} (ii). Moreover, one has an isomorphism of $S_{p}$-modules by Proposition \ref{mcky}: 
\begin{equation*}
\Ind_{H}^{G}\Z \cong \bigoplus_{i=1}^{r}\Ind_{S_{p}\cap g_iHg_i^{-1}}^{S_{p}}\Z. 
\end{equation*}
In particular, there is an isomorphism of $S_{p}$-modules
\begin{equation*}
J_{G/H} \cong J_{S_{p}/(S_{p}\cap g_1Hg_1^{-1},\ldots,S_{p}\cap g_rHg_r^{-1})}. 
\end{equation*}
Consequently, the assertion follows from Theorem \ref{blpl}. 
\end{proof}

As a summarization of Theorems \ref{upts}, \ref{2dpt} and \ref{pptt}, we obtain the full structure of the abelian group $\Sha_{\cD}^{2}(G,J_{G/H})$ under the normality of a $p$-Sylow subgroup of $G$. 

\begin{thm}\label{tspt}
\emph{Let $G$ be a finite group, and $p$ a prime divisor of $\#G$. Assume that a $p$-Sylow subgroup $S_{p}$ of $G$ is normal. Take a subgroup $H$ of $G$ satisfying $N^{G}(H)=\{1\}$ and $\ord_{p}(G:H)=1$, and pick a set $\cD$ of subgroups of $G$ containing $\cC_{G}$. 
\begin{enumerate}
\item If one has $\Sha_{\cD}^{2}(G,J_{G/S_{p}H})=\Sha_{\omega}^{2}(G,J_{G/S_{p}H})$, 
then there is an isomorphism
\begin{equation*}
\Sha_{\cD}^{2}(G,J_{G/H})^{(p)} \cong \Sha_{\cD}^{2}(G,J_{G/S_{p}H}). 
\end{equation*}
\item The abelian group $\Sha_{\omega}^{2}(G,J_{G/H})[p^{\infty}]$ is non-zero if and only if the following is satisfied: 
\begin{itemize}
\item[(a)] there is an isomorphism $S_{p}\cong (\Z/p)^{2}$,
\item[(b)] $[S_{p},G]=S_{p}$, 
\item[(c)] $N_{G}(S_{p}\cap H)=Z_{G}(S_{p}\cap H)$. 
\end{itemize}
\item If the abelian group $\Sha_{\omega}^{2}(G,J_{G/H})$ is non-zero, then there is an isomorphism 
\begin{equation*}
\Sha_{\cD}^{2}(G,J_{G/H})[p^{\infty}] \cong 
\begin{cases}
0&\text{if there is $D \in \cD$ which contains $S_{p}$,}\\
\Z/p &\text{otherwise. }
\end{cases}
\end{equation*}
\end{enumerate}}
\end{thm}

\section{Representations of finite groups over finite fields}\label{fnrp}

Here let $p$ be a prime number. We study $2$-representations of finite groups of order coprime to $p$ over $\Fp$. 

\begin{lem}\label{frrp}
\emph{Let $G'$ be a finite group, and $V$ an irreducible $\F_{q}$-representation of $G'$ of finite dimension. Here $q$ is a power of $p$. Then there is a surjection $\F_{q}[G'] \twoheadrightarrow V$. }
\end{lem}

\begin{proof}
This follows from the Frobenius reciprocity. %See \cite{Neukirch2000, p.~63}. 
\end{proof}

\begin{dfn}\label{cdbc}
Let $G'$ be a finite group of order prime to $p$, and $H'$ a subgroup of $G'$. Take a $2$-dimensional $\Fp$-representation $V$ of $G'$ and a $1$-dimensional subspace $L$ of $V$ which is fixed under $H'$. Then we consider the two conditions as follows: 
\begin{itemize}
\item[(B)] $V^{G'}=\{0\}$,
\item[(C)] the stabilizer of $L$ in $G'$ acts trivially on $L$. 
\end{itemize}
\end{dfn}

\begin{prop}\label{bcit}
\emph{Let $G'$, $H'$, $V$ and $L$ be as in Definition \ref{cdbc}. Denote by $G:=L\rtimes G'$ the associated semi-direct product. Furthermore, put $H:=L\rtimes H'$ and $S_{p}:=V\rtimes \{1\}$. 
\begin{enumerate}
\item the representation $V$ of $G'$ satisfies (B) and (C),
\item the finite group $G'$ and its subgroup $H$ and $S_{p}$ satisfy (b) and (c) in Theorem \ref{tspt} (i). 
\end{enumerate}}
\end{prop}

\begin{proof}
First, we prove that (B) and (b) are equivalent. By \cite[Chap.~VIII, \S 1]{Serre1979}, one has an exact sequence
\begin{equation*}
0\rightarrow \widehat{H}^{-1}(G',V)\rightarrow H_{0}(G',V)\rightarrow H^{0}(G',V)\rightarrow \widehat{H}^{0}(G',V)\rightarrow 0. 
\end{equation*}
By definition, we have $H_{0}(G',V)=V/[V,G']$ and $H^{0}(G',V)=V^{G'}$. On the other hand, since $\#G'$ is prime to $p$, the abelian groups $\widehat{H}^{-1}(G',V)$ and $\widehat{H}^{0}(G',V)$ vanish. Therefore the desired equivalence holds. 

Second, we give an equivalence between (C) and (c). Denote by $N'$ the stabilizer of $L$ under the action of $G'$. Then we have $N_{G}(S_{p}\cap H)=S_{p}\rtimes N'$. On the other hand, if we put
\begin{equation*}
Z':=\{g'\in G'\mid g'(v)=v\text{ for any }v\in V\}, 
\end{equation*}
then $Z_{G}(S_{p}\cap H)=S_{p}\rtimes Z'$. Hence the condition (C) holds if and only if so does (c). 
\end{proof}

We give some existence of representations satisfying (B) and (C), which will be used in Section \ref{pfmt}. 

\begin{dfn}\label{d1d2}
We define two sets $D_{1}(p)$ and $D_{2}(p)$ as follows: 
\begin{align*}
D_{1}(p)&:=\{n\in p\Zpn \mid \gcd(n,p-1)\geq 3\},\\
D_{2}(p)&:=\{n\in p\Zpn \mid \gcd(n,p+1)\text{ is not a power of }2\}.
\end{align*}
By definition, these are closed under the multiplication by any positive integer. 
\end{dfn}

\begin{prop}\label{chsm}
\emph{Let $q>1$ be a power of a prime number, and $n$ be a positive integer dividing $q-1$. Pick a primitive $n$-th root of unity $\zeta_{n}$ in $\F_{q}^{\times}$. For each $j \in \Z/n$, put
\begin{equation*}
\chi_{q,n}^{j}\colon \Z/n \rightarrow \F_{q}^{\times}; 1 \bmod n \mapsto \zeta_{n}^{j}. 
\end{equation*}
We regard $\chi_{q,n}^{j}$ as a $1$-dimensional $\F_{q}$-representation of $\Z/n$. Then every $2$-dimensional $\Fp$-representation of $\Z/n$ is isomorphic to $\chi_{q,n}^{j_1}\oplus \chi_{q,n}^{j_2}$ for some $j_1,j_2\in \Z/n$. }
\end{prop}

\begin{proof}
Since $n \mid q-1$, the polynomial $X^{n}-1$ is decomposed as the product of $X-\zeta_{n}^{i}$ for all $i\in \Z/n$, the homomorphism $\F_{q}[x]\rightarrow \F_{q}[G']$, which sends $x$ to $1\bmod n$, induces an isomorphism
\begin{equation*}
\F_{q}[G']\cong \prod_{i\in \Z/n}\F_{q}[X]/(X-\zeta_{n}^{i}). 
\end{equation*}
Hence, Lemma \ref{frrp} implies that any irreducible representation of $G'$ is isomorphic to $\chi_{q,n}^{i}$ for some $i\in \Z/n$. Combining this result with Maschke's theorem, we obtain the desired assertion. 
\end{proof}

\begin{prop}\label{cyrd}
\emph{Assume $p\geq 5$ is a prime number and $n\in \Zpn$ be a divisor of $p-1$. 
\begin{enumerate}
\item Assume that $n=\ell$ is an odd prime number (in particular, $p\ell \in D_{1}(p)$). Then, for $j_1,j_2\in \Z$, the representation $\chi_{\ell}^{j_1}\oplus \chi_{\ell}^{j_2}$ of $\Z/\ell$ satisfies (B) and (C) for $H'=\{0\}$ and $L=\langle (1,a)\rangle_{\Fp}$ for any (or, some) $a\in \F_{p}^{\times}$ if and only if $i,j,i-j\neq 0$ in $\Z/\ell$. 
\item If $p\equiv 1 \bmod 4$ and $n=4$, then the $2$-dimensional representation $\chi_{p,4}^{j_1}\oplus \chi_{p,4}^{j_2}$ of $\Z/4$ satisfies (B) and (C) for $H=\{0\}$ and $L=\langle (1,a)\rangle_{\Fp}$ for any (or, some) $a\in \F_{p}^{\times}$ if and only if $j_1,j_2\neq 0$ and $j_1-j_2\notin \{0,2\}$ in $\Z/4$. 
\end{enumerate}}
\end{prop}

\begin{proof}
By definition, the condition (B) is equivalent to $i,j\neq 0$. Moreover, the condition (C) holds for $H'=\{0\}$ and $L=\langle(1,a)\rangle_{\Fp}$ for only if $i\neq j$. On the other hand, we have
\begin{equation*}
(\chi_{n}^{j_1}\oplus \chi_{n}^{j_2})(d)\cdot (1,a)=(\zeta_{n}^{dj_1},\zeta_{n}^{dj_2})=\zeta_{n}^{j_1}(1,\zeta_{n}^{d(j_2-j_1)}a)
\end{equation*}
for any $d\in \Z/n$. 

First, we give a proof of (i). It suffices to prove the condition (C) for $H'=\{0\}$ and $L=\langle(1,a)\rangle_{\Fp}$ under the assumption $j_1 \neq j_2$. Take $d\in \Stab_{G'}(L)$, then one has $d(j_2-j_1)=0$ in $\Z/\ell$. Hence we have $d=0$. This means the triviality of $\Stab_{G'}(L)$, which implies the condition (C). 

Second, we prove the assertion (ii). First, assume $j_1-j_2 \in 2\Z \setminus 4\Z$. We prove that $\chi_{4}^{i}\oplus \chi_{4}^{j}$ does not satisfy (B) or (C). In this case, the element $2\bmod 4$ in $\Z/4$ stabilizes $L$. Moreover, the assumption $j_1-j_2 \in 2\Z \setminus 4\Z$ implies that $(\chi_{4}^{j_1}\oplus \chi_{4}^{j_2})(d)$ is trivial if and only if $j_1,j_2\in 4\Z$. Hence we obtain the desired assertion. On the other hand, the same proof as (i) implies that the condition (C) holds for $H'=\{0\}$ and $L=\langle(1,a)\rangle_{\Fp}$ if $j_1-j_2\notin 2\Z$. This completes the proof of (ii). 
\end{proof}

\begin{prop}\label{cyir}
\emph{Let $p$ be a prime number, and $\ell$ an odd prime number satisfying $p\ell \in D_{2}(p)$. Fix a primitive $\ell$-th root of unity $\zeta_{\ell}$, which is contained in $\F_{p^2}^{\times}\setminus \F_{p}^{\times}$. For $j\in (\Z/\ell)^{\times}$, let $V_{p,\ell}^{j}=\F_{p}^{\oplus 2}$ be the $\Fp$-representation of $G'$ defined by the homomorphism
\begin{equation*}
\Z/\ell \rightarrow \GL_{2}(\Fp); 1 \bmod \ell \mapsto 
\begin{pmatrix}
0&-1\\
1&\zeta_{\ell}^{j}+\zeta_{\ell}^{pj}
\end{pmatrix}. 
\end{equation*}
\begin{enumerate}
\item Every $v\in V_{p,\ell}^{j}\setminus \{0\}$ satisfies $V_{p,\ell}^{j}=\langle v,(1\bmod \ell)\cdot v\rangle_{\Fp}$ and a non-trivial linear relation
\begin{equation*}
(2\bmod \ell)\cdot v=(\zeta_{\ell}^{j}+\zeta_{\ell}^{pj})((1\bmod \ell)\cdot v)-v. 
\end{equation*}
\item For $j_1,j_2\in (\Z/\ell)^{\times}$, there is an isomorphism $V_{j_1}\cong V_{j_2}$ if and only $j_1=j_2$ or $j_1=pj_2$. 
\item Every non-trivial $2$-dimensional $\Fp$-representation of $\Z/\ell$ is isomorphic to $V_{p,\ell}^{j}$ for some $j\in (\Z/\ell)^{\times}$. 
\item For any $j\in (\Z/\ell)^{\times}$, the representation $V_{p,\ell}^{j}$ of $\Z/\ell$ satisfies (B) and (C) for $H'=\{0\}$ and any $1$-dimensional subspace $L$ of $V$. 
\end{enumerate}}
\end{prop}

\begin{proof}
(i): This follows from direct computation. 

(ii): By definition, there is an isomorphism $V_{p,\ell}^{j}\cong V_{pj}$ for any $j\in (\Z/\ell)^{\times}$. On the other hand, since $V_{p,\ell}^{j}\otimes_{\Fp}\F_{p^2}\cong \chi_{p^{2},\ell}^{j}\oplus \chi_{p^{2},\ell}^{pj}$, there is an isomorphism $V_{j_1}\otimes_{\Fp}\F_{p^2}\cong V_{j_2}\otimes_{\Fp}\F_{p^2}$ only if $j_1=j_2$ or $j_1=pj_2$. Hence the assertion holds. 

(iii): The homomorphism $\Fp[x] \rightarrow \Fp[G']$, which maps $x$ to $1\bmod \ell$, induces an isomorphism
\begin{equation*}
\Fp[G']\cong \Fp[x]/(x-1)\times \prod_{j\in (\Z/\ell)^{\times}/(j\sim pj)}\Fp[x]/(x^{2}-(\zeta_{\ell}^{j}+\zeta_{\ell}^{pj})x+1). 
\end{equation*}
Here all factors of the right-hand side are fields. Hence any irreducible representation of $G'$ is isomorphic to a trivial representation or $V_{p,\ell}^{j}$ for $j\in (\Z/\ell)^{\times}$, which is a consequence of Lemma \ref{frrp}. The assertion follows from this result and Maschke's theorem. 

(iv): This follows from (i). 
\end{proof}

In the sequel of this section, for $n\in \Zpn$, we write $\fS_{n}$ and $D_{n}$ for the symmetric group of $n$-letters and the dihedral group of order $2n$ respectively. 

\begin{prop}\label{s3bc}
\emph{Let $p\geq 5$ be a prime number. 
\begin{enumerate}
\item The $\Fp$-representation $V_{p,0}:=J_{\fS_{3}/\langle (1\ 2)\rangle}\otimes_{\Z}\Fp$ of $\fS_{3}$ is the unique one which is irreducible and $2$-dimensional. Moreover, $L:=\langle 2-(1\ 3)-(2\ 3)\rangle \otimes_{\Z}\Fp$ is the unique $1$-dimensional subspace of $V_{p,0}$ for which $H':=\langle (1 \ 2)\rangle$ acts trivially. 
\item Under the notations in (i), the representation $V_{p,0}$ satisfies (B) and (C) for $H'$ and $L'$. 
\end{enumerate}}
\end{prop}

\begin{proof}
For any field $F$, we denote by $R_{F}(\fS_{3})$ be the Grothendieck group of finite-dimensional $F$-representations of $\fS_{3}$. Recall that the $2$-dimensional $F$-representation $J_{\fS_{3}/\langle (1\ 2)\rangle}\otimes_{\Z}F$ of $\fS_{3}$ is irreducible. Moreover, the abelian group $R_{\C}(\fS_{3})$ is generated by characters and $J_{\fS_{3}/\langle (1\ 2)\rangle}\otimes_{\Z}\C$. This is a consequence of \cite[Section 5.3, pp.~37--38]{Serre1977} and \cite[Proposition 40, Section 14.1]{Serre1977}. Note that we use an isomorphism $\fS_{3}\cong D_{3}$. Now fix an injection $\Qp \hookrightarrow \C$. As pointed out in \cite[Section 14.6, p.~122]{Serre1977}, the canonical homomorphism $R_{\Qp}(\fS_{3})\rightarrow R_{\C}(\fS_{3})$ is injective. Hence it is an isomorphism, which implies that $R_{\Qp}(\fS_{3})$ is generated by characters and $J_{\fS_{3}/\langle (1\ 2)\rangle}\otimes_{\Z}\Qp$. 

(i): The the irreduciblity and the uniqueness on $V_{p,0}$ follow from \cite[Section 15.5, Proposition 43, p.~128]{Serre1977}. The positivity of the conditions (B) and (C) are clear by the above assertion. 

(ii): The satisfaction of the condition (B) is a consequence of Proposition \ref{mnex} (i). On the other hand, for the condition (C), it suffices to prove that $\Stab_{G'}(L')$ has order $2$. However, this follows from the equality $\#G'=6$ and the irreducibility of $V$. 
\end{proof}

In the following, we give a non-existence of $2$-dimensional $\Fp$-representation of certain groups. 

\begin{dfn}
We define two subgroups of $\GL_{2}(\Fp)$ as follows: 
\begin{equation*}
\bT(\Fp):=\{\diag(a,b)\in \GL_{2}(\Fp)\mid a,b\in \F_{p}^{\times}\},\quad \bM(\Fp):=\{\diag(1,b)\in \GL_{2}(\Fp)\mid b\in \F_{p}^{\times}\}. 
\end{equation*}
\end{dfn}

\begin{lem}\label{cmtv}
\emph{Let $g\in \GL_{2}(\Fp)$ and $h=\diag(1,e)\in \bM(\Fp)$, where $e \neq 1$. If $ghg^{-1}\in \bM(\Fp)$, then $g$ is contained in $\bT(\Fp)$. }
\end{lem}

\begin{proof}
Since the characteristic polynomials of $h$ and $ghg^{-1}$ coincide, we obtain an equality $ghg^{-1}=h$. Write $g=\begin{pmatrix}
a&b\\
c&d
\end{pmatrix}$, where $a,b,c,d\in \Fp$. Then the equality $gh=hg$ is equivalent to the following: 
\begin{equation*}
\begin{pmatrix}
a&eb\\
c&ed
\end{pmatrix}
=
\begin{pmatrix}
a&b\\
ec&ed
\end{pmatrix}. 
\end{equation*}
Hence $e\neq 1$ implies $b=c=0$, that is, $g$ is contained in $\bT(\Fp)$. 
\end{proof}

\begin{cor}\label{diag}
\emph{Assume $p>2$. Let $G'$ be a subgroup of $\GL_{2}(\Fp)$, and $H'$ a subgroup of $G'\cap \bM(\Fp)$. 
\begin{enumerate}
\item If $\{1\}\neq N^{G'}(H')\subset \bM(\Fp)$, then $G'$ is contained in $\bT(\Fp)$. 
\item We further assume that $H'$ has index $2$ in $G'$. Then we have one of the following: 
\begin{itemize}
\item $G'$ is contained in $\bT(\Fp)$,
\item $\#G'=2$ and $(\F_{p}^{\oplus 2})^{H'}\neq \{0\}$. 
\end{itemize}
\end{enumerate}}
\end{cor}

\begin{proof}
(i): Take a non-trivial element $h$ of $N^{G'}(H')$. Write $h=\diag(1,e)$, where $e\neq 1$. Take any $g\in G'$. Then $ghg^{-1}$ is contained in $N^{G'}(H')\subset \bM(\Fp)$. Hence the assertion follows from Lemma \ref{cmtv}. 

(ii): First, assume that $H'$ is non-trivial. Since $(G':H')=2$, we have $N^{G'}(H')=H'$. Hence Lemma \ref{cmtv} implies that $G'$ is contained in $\bT(\Fp)$. Second, suppose that $H'$ is trivial, that is, $\#G'=2$. If  $\bT(\Fp)$ does not contain $G'$, then let $g$ the unique non-trivial element of $G'$. Since $g^{2}=1$ and $g\notin \bT(\Fp)$, the minimal polynomial of $g$ must be $X^{2}-1$. Hence there is $x\in \F_{p}^{\oplus 2}$ which is fixed under $g$. This is equivalent to the desired non-triviality of $(\F_{p}^{\oplus 2})^{H'}$. 
\end{proof}

\begin{cor}\label{trtr}
\emph{Let $G'$ be a finite group of order prime to $p$, and $H'$ a subgroup of $G'$. Take a $2$-dimensional $\Fp$-representation $V$ of $G'$ and an $H'$-stable $1$-dimensional subspace $L$ of $V$. Assume that (B) and (C) hold for $V$. Then the action on $N^{G'}(H')$ on $V$ is trivial. }
\end{cor}

\begin{proof}
Suppose that the action of $N^{G'}(H')$ on $V$ is non-trivial and satisfies (C) for an $H'$-stable $1$-dimensional subspace of $V$. Then Corollary \ref{diag} implies the stability of $L$ under $G'$. Hence, the $G'$-fixed part contains $L$ by the assumption (C). In particular, the condition (B) does not hold. 
\end{proof}

\begin{lem}[{\cite[I, pp.~142--143]{Carter1964}}]\label{cf2s}
\emph{
\begin{enumerate}
\item Assume $p\equiv 1\bmod 4$. Put $s:=\ord_{2}(p-1)$, and let $\zeta_{2^s}$ be a primitive $2^{s}$-root of unity. Set
\begin{equation*}
X:=\begin{pmatrix}
\zeta_{2^s}&0\\
0&1
\end{pmatrix},\quad
Y:=\begin{pmatrix}
1&0\\
0&\zeta_{2^s}
\end{pmatrix},\quad
Z:=\begin{pmatrix}
0&1\\
1&0
\end{pmatrix}. 
\end{equation*}
Then the subgroup $S$ generated by $X$, $Y$ and $Z$ is a $2$-Sylow subgroup of $\GL_{2}(\Fp)$. 
\item Suppose $p\equiv 3\bmod 4$. Put $s:=\ord_{2}(p+1)$. Let $\zeta_{2^{s+1}}$ be a $2^{s+1}$-th root of unity, which is contained in $\F_{p^2}$. Put
\begin{equation*}
X:=\begin{pmatrix}
0&1\\
1&\zeta_{2^{s+1}}+\zeta_{2^{s+1}}^{p}
\end{pmatrix},\quad Y:=
\begin{pmatrix}
0&1\\
-1&0
\end{pmatrix}
\begin{pmatrix}
0&1\\
1&\zeta_{2^{s+1}}+\zeta_{2^{s+1}}^{p}
\end{pmatrix}. 
\end{equation*}
Then the matrices $X$ and $Y$ generates a $2$-Sylow subgroup $S$ of $\GL_{2}(\Fp)$. Moreover, there are relations
\begin{equation*}
X^{2^{s}}=\diag(-1,-1),\quad Y^{2}=1,\quad YXY^{-1}=X^{2^{s}-1}. 
\end{equation*}
\end{enumerate}}
\end{lem}

\begin{cor}\label{-1id}
\emph{Assume $p>2$, and let $G'$ be a subgroup of $\GL_{2}(\Fp)$ of order $4$. Then at least one of the following is satisfied: 
\begin{enumerate}
\item $G'$ contains the matrix $\diag(-1,-1)$. 
\item $p\equiv 1\bmod 4$ and $G'\cong \Z/4$. 
\end{enumerate}}
\end{cor}

\begin{proof}
\textbf{Case 1.~$p\equiv 1\bmod 4$. }

We may assume that $G'$ is contained in $S$ in Lemma \ref{cf2s} (i). It suffices to prove that $G'$ contains $\diag(-1,-1)$ if $G'$ is not cyclic. In the case $G'\subset \langle X,Y\rangle$, we have $G'=\langle \diag(-1,1),\diag(1,-1)\rangle$. Hence the condition (i) holds. Otherwise, the order of $G'\cap \langle X,Y\rangle$ must be $2$. Now take the unique non-trivial element $Y_{0}\in G'\cap \langle X,Y\rangle$ and an element $Z_{0}\in G'\setminus \langle X,Y\rangle$. Write
\begin{equation*}
Y_{0}=
\begin{pmatrix}
y_1&0\\
0&y_2
\end{pmatrix},\quad 
Z_{0}=
\begin{pmatrix}
0&z_1\\
z_2&0
\end{pmatrix},
\end{equation*}
where $y_1,y_2,z_1,z_2\in \F_{p}^{\times}$. Since $Y_{0}^{2}=Z_{0}^{2}=1$, we obtain $y_1^2=y_2^2=1$ and $z_1z_2=1$. Moreover, since
\begin{equation*}
Y_{0}Z_{0}=
\begin{pmatrix}
0&y_1z_1\\
y_2z_2&0
\end{pmatrix},
\end{equation*}
one has $y_1y_2z_1z_2=1$. This is equivalent to $y_1y_2=1$ since $z_1z_2=1$. On the other hand, the non-triviality of $Y_{0}$ implies that either $y_1$ or $y_2$ is not equal to $1$. Hence we obtain $y_1=y_2=-1$, which implies the condition (i). 

\textbf{Case 1.~$p\equiv 3\bmod 4$. }

We may assume that $G'$ is contained in $S$ in Lemma \ref{cf2s} (ii). First, suppose that $G'=\langle Z \rangle$ is cyclic. Write $Z=X^{m_0}Y^{n_0}$, where $m_0,n_0 \in \Z$. Note that the following holds for any $m,n\in \Z$, which follows from Lemma \ref{cf2s} (ii): 
\begin{equation*}
(X^{m}Y^{n})^{2}=
\begin{cases}
X^{2m}&\text{if }2\mid n,\\
X^{2^{s}m}&\text{if }2\nmid n. 
\end{cases}
\end{equation*}
This implies the condition as follows: 
\begin{itemize}
\item $m_{0} \in 2^{s-1}\Z \setminus 2^{s}\Z$ if $n_0$ is even, 
\item $m_{0}$ is odd if $n_{0}$ is so. 
\end{itemize}
Hence we have $\diag(-1,-1)=Z^{2}\in G'$ as desired. Second, if $G'=\langle Z_1,Z_2 \rangle$ is not cyclic, write $Z_1=X^{m_1}Y^{n_1}$ and $Z_2=X^{m_2}Y^{n_2}$, where $m_1,m_2,n_1,n_2 \in \Z$. If $n_1\equiv n_2 \bmod 2$, then $Z_1Z_2^{-1}\in G'$ is contained in $\langle X\rangle$ and has order $2$. This gives an equality $Z_1Z_2^{-1}=\diag(-1,-1)$, and hence the condition (i) holds. Otherwise, after exchanging $Z_1$ and $Z_2$ if necessary, we may assume $n_1=0$. Then we have $Z_{1}=\diag(-1,-1)$, and this implies $\diag(-1,-1)\in G'$ as desired. 
\end{proof}

\begin{lem}\label{idx4}
\emph{Assume $p>2$, and let $G'$ be a subgroup of $\GL_{2}(\Fp)$. Assume that there is a subgroup $H'$ of $G'$ of index $4$ such that $N^{G'}(H')=\{1\}$. Then the following are equivalent: 
\begin{enumerate}
\item all subgroups of $G'$ of order $4$ are cyclic,
\item there is an isomorphism $G'\cong \Z/4$. 
\end{enumerate}}
\end{lem}

\begin{proof}
Take a $2$-Sylow subgroup $S'$ of $G'$, which has order a multiple of $4$. By assumption, there is an injection from $G'$ into $\fS_{4}$. Since a $2$-Sylow subgroup of $\fS_{4}$ is isomorphic to $D_{4}$, we obtain an injection $S'\hookrightarrow D_{4}$. Hence the group $S'$ is isomorphic to one of $\Z/4$, $(\Z/2)^{2}$ or $D_{4}$. 

First, if $S'$ is not isomorphic to $\Z/4$. Then it is clear that both (i) and (ii) fail. Second, suppose $S'\cong \Z/4$, which implies that the condition (i) holds. Since there is an injection $G'\hookrightarrow \fS_{4}$, the order of $G'$ is $4$ or $12=4\cdot 3$. If $\#G'=12$, the we necessarily have the normality of a $3$-Sylow subgroup of $G'$, which is a consequence of \cite[Chap II, Proposition 6.4, p.~98]{Hungerford1974}. Hence $H'$ must be normal in $G'$, which contradicts the triviality of $N^{G'}(H')$. Therefore the order of $G'$ coincides with $4$, which implies the condition (ii). 
\end{proof}

\begin{prop}\label{nerp}
\emph{Let $G'$ be a finite group of order prime to $p$, and $V$ a $2$-dimensional $\Fp$-representation of $G'$. Take a subgroup of $H'$ of $G'$ and a $1$-dimensional subspace $L$ of $V$ which is stable under $H'$. We assume at least one of the following: 
\begin{enumerate}
\item $p\cdot (G':H')$ is not contained in $D_{1}(p)\cup D_{2}(p)\cup p^{2}\Zpn$,
\item $(G':H')=4$ and $G'\not\cong \Z/4$. 
\end{enumerate}
Then any $2$-dimensional $\Fp$-representation $(\rho,V)$ does not satisfy one of (B) or (C). }
\end{prop}

\begin{proof}
Assume that a given representation $(\rho,V)$ satisfies (C). Take $\be_1 \in L\setminus \{0\}$ and $\be_2\in V\setminus L$, which induces an isomorphism $\GL(V)\cong \GL_{2}(\Fp)$. Let $G'_{1}$ and $H'_{1}$ be the images of $G'$ and $H'$ respectively under the homomorphism
\begin{equation*}
G'\xrightarrow{\rho} \GL(V)\cong \GL_{2}(\Fp). 
\end{equation*}
Then the assumption (C) implies that $H'_{1}$ is contained in $\bM(\Fp)$. 

\textbf{Case 1.~$\ord_{2}(G'_1:H'_1)\leq 1$. }

This happens only when (i) holds. Moreover, we have $\ord_{2}(G'_1:H'_1)=0$ if $p=2$. In particular, we have $(G'_1:H'_1)\in \{1,2\}$ since $\#G'_1$ is divisible by $\#\GL_{2}(\Fp)=p(p+1)(p-1)^{2}$. If $(G:H)=1$, then it is equivalent to $G'_1=H'_1$, which follows that (B) is negative. On the other hand, if $(G'_1:H'_1)=2$, then Corollary \ref{diag} (ii) implies that at least one of the following is satisfied: 
\begin{itemize}
\item[(1)] $G'_1 \subset \bT(\Fp)$,
\item[(2)] $\#G'_1=2$ and $V^{G'}\neq \{0\}$. 
\end{itemize}
If (1) holds, then $L$ becomes stable under $G'$. Combining this with the condition (C), we obtain that $V^{G'}$ is contains $L$. This implies that $(\rho,V)$ does not satisfy (B). If the condition (2) is true, it is immediate that (B) is not satisfied. Therefore the assertion holds in this case. 

\textbf{Case 2.~$\ord_{2}(G'_1:H'_1)\geq 2$. }

In this case, the assumption implies that $(G'_1:H'_1)$ is a power of $2$. Moreover, by Corollary \ref{trtr}, we may assume that $N^{G'}(H')$ is trivial. Take a subgroup $S'$ of $G'_1$ of order $4$. Then it contains the matrix $\diag(-1,-1)$. This is a consequence of Corollary \ref{-1id} since we assume $G'\not\cong \Z/4$ if $p\equiv 1\bmod 4$. Therefore the stabilizer of $L$ under $G'_1$ contains $\diag(-1,-1)$, which contradicts the assumption (C). 
\end{proof}

\section{The Hasse norm principle}\label{hnmt}

In this section, let $k$ be a global field. Fix a separable closure $k^{\sep}$ of $k$. For a finite Galois extension $K$ of $k$, write $\Sigma_{K}$ for the set of places of $K$. For each $w \in \Sigma_{K}$, denote by $D_{K/k,w}$ the decomposition group of $K/k$ at $w$. Moreover, set $\cD_{K/k}:=\{D_{K/k,w}\subset \Gal(K/k)\mid w\in \Sigma_{K}\}$. 

\subsection{Relation with second cohomology groups}

\begin{dfn}\label{gfnt}
Let $M$ be an abelian group equipped with an action of $\Gal(k^{\sep}/k)$. For $j\in \Znn$, put
\begin{equation*}
\Sha^{j}(k,M):=\Ker\left(H^{j}(k.M)\xrightarrow{(\Res_{k_v/k})_{v}}\prod_{v\in \Sigma_{k}}H^{j}(k_{v},M)\right). 
\end{equation*}
Here $\Res_{k_v/k}\colon H^{j}(k,M)\rightarrow H^{j}(k_{v},M)$ denotes the restriction map for each $v\in \Sigma_{k}$. 
\end{dfn}

\begin{lem}\label{almt}
\emph{Under the above notations, let $T$ be a $k$-torus which splits over a finite Galois extension $K$ of $k$. Put $G:=\Gal(K/k)$, and denote by $\cD_{K/k}$ the set of decomposition groups of $K/k$. Then there is an isomorphism
\begin{equation*}
\Sha^{2}(k,X^{*}(T))\cong \Sha_{\cD_{K/k}}^{2}(G,X^{*}(T)). 
\end{equation*}}
\end{lem}

\begin{proof}
Let $\widetilde{K}/k$ be a finite Galois extension which contains $K$. Put $\widetilde{G}:=\Gal(\widetilde{K}/k)$. Then Proposition \ref{ifts} implies an isomorphism $\Sha_{\cD_{K/k}}^{2}(G,X^{*}(T))\cong \Sha_{\cD_{\widetilde{K}/k}}^{2}(\widetilde{G},X^{*}(T))$. Hence this isomorphism implies the desired assertion. 
\end{proof}

For a finite separable field extension $K/k$, put
\begin{equation*}
\Sha(K/k):=(k^{\times}\cap \N_{K/k}(\A_{K}^{\times}))/\N_{K/k}(K^{\times}),
\end{equation*}
where $\A_{K}^{\times}$ is the id{\`e}le group of $K$. We say that the \emph{Hasse norm principle} holds for $K/k$ holds if $\Sha(K/k)=0$. 

\begin{prop}\label{onhn}
\emph{Let $K/k$ be a finite separable field extension of a global field. Put $G:=\Gal(K^{\Gal}/k)$ and $H:=\Gal(K^{\Gal}/K)$. Then there is an isomorphism
\begin{equation*}
\Sha(K/k)\cong \Sha_{\cD_{K^{\Gal}/k}}^{2}(G,J_{G/H})^{\vee}. 
\end{equation*}}
\end{prop}

\begin{proof}
There is an isomorphism of abelian groups
\begin{equation*}
\Sha(K/k)\cong {\Sha}^{1}(k,T_{K/k}),
\end{equation*}
which is a consequence of \cite[p.~70]{Ono1963}. On the other hand, the Poitou--Tate duality (see \cite[(8.6.8) Proposition]{Neukirch2000}) implies an isomorphism
\begin{equation*}
\Sha^{1}(k,T_{K/k})\cong \Sha^{2}(k,X^{*}(T_{K/k}))^{\vee}. 
\end{equation*}
Hence the assertion follows from Lemma \ref{almt}. 
\end{proof}

\begin{cor}\label{shan}
\emph{Let $K/k$ be a finite separable field extension of a global field. 
\begin{enumerate}
\item \emph{(cf.~\cite[Corollary 2.7]{Macedo2021})} The abelian group $\Sha(K/k)$ is annihilated by $[K:k]$. 
\item \emph{(\cite[Lemma 4]{Bartels1981a})} If $[K:k]$ is a prime number, then $\Sha(K/k)$ is trivial. 
\item Assume $[K:k]=2p$, where $p$ is an odd prime number. Then $\Sha(K/k)$ is annihilated by $2$. 
\end{enumerate}}
\end{cor}

\begin{proof}
(i): We use the notations in Proposition \ref{onhn}. Since $[K:k]=(G:H)$, it suffices to prove that $(G:H)$ annihilates $\Sha_{\omega}^{2}(G,J_{G/H})$. However, Proposition \ref{tsan} (i) implies the desired assertion. 

(ii): This is a consequence of Propositions \ref{onhn} and \ref{bart}. 

(iii): The claim follows from (i) and Proposition \ref{onhn} and Theorem \ref{pptt}. 
\end{proof}

\begin{cor}\label{pttr}
\emph{Let $K/k$ be a finite separable field extension of global fields. Then the abelian group $\Sha(K/k)[p^{\infty}]$ is trivial if
\begin{equation*}
[K:k]<
\begin{cases}
3p &\text{if }p>2,\\
4 &\text{if }p=2. 
\end{cases}
\end{equation*}}
\end{cor}

\begin{proof}
First, if $p\nmid [K:k]$, then Corollary \ref{shan} (i) implies the triviality of $\Sha(K/k)[p^{\infty}]$. Second, the assertion in the case $[K:k]=p$ follows from Corollary \ref{shan} (ii). Finally, for the remaining case $p>2$ and $[K:k]=2p$, the claim is a consequence of Corollary \ref{shan} (iii). 
\end{proof}

\subsection{Proofs of main theorems}\label{pfmt}

\begin{prop}\label{shaf}
\emph{Let $k$ be a global field, and $G$ a finite group. Assume that at least one of the following holds:  
\begin{enumerate}
\item $G \cong \Z/\ell_{0} \times \Z/n_{0}$, where $\ell_{0}$ is a prime number, and $\ell_{0} \mid n_{0}$, 
\item $G$ is a solvable group whose $p$-Sylow subgroup is normal, where $p$ is a prime divisor of $\#G$ which is invertible in $k$. 
\end{enumerate}
Then there is a finite Galois extension $\widetilde{K}/k$ with Galois group $G$ in which all decomposition groups of $\widetilde{K}/k$ are cyclic. }
\end{prop}

\begin{proof}
\textbf{Case 1.~the condition (i) holds. }
First, suppose that $k$ is a number field. Write $n_{0}=\prod_{i=0}^{r-1}\ell_{i}^{e_i}$, where $r$ is a positive integer, $\ell_{0},\ldots,\ell_{r-1}$ are prime numbers which are distinct from each other and $e_{0},\ldots,e_{r-1}\in \Zpn$. Let $\delta_{k}$ be the discriminat of $k$. By Dirichlet's prime number theorem, there are $r$ distinct prime numbers $\ell'_{0},\ldots,\ell'_{r-1}$ which are not divisors of $\delta_{k}$ and are congruent to $1$ modulo $\ell_{i}^{e_{i}}$ for every $i\in \{0,\ldots,r-1\}$. Then the finite extension $\Q(\zeta_{\ell'_{i}})/\Q$ is abelian with Galois group $\Z/(\ell'_{i}-1)$. By assumption, there is a unique subfield $K_{i}$ of $\Q(\zeta_{\ell'_{i}})$ which has degree $\ell_{i}^{e_i}$ over $\Q$. Hence, if we denote by $K$ the composite field of $K_{0},\ldots,K_{r-1}$, then there is an isomorphism $\Gal(K'/\Q)\cong \Z/n_{0}$. On the other hand, take another prime number $\ell'$ which splits totally in $K(\zeta_{\ell_{0}},\sqrt[\ell_{0}]{\ell'_{0}},\ldots, \sqrt[\ell_{0}]{\ell'_{r-1}})$. Note that this is possible by Chebotarev's density theorem. Now let $K'$ be the unique subfield of $\Q(\zeta_{\ell'})$ of degree $\ell$ over $\Q$, which satisfies $K \cap K'=\Q$. Hence there is an isomorphism $\Gal(KK'/\Q)\cong G$. Moreover, the assumption on $\ell'$ implies that all decomposition groups of $KK'/\Q$ are cyclic, and the fields $KK'$ and $k$ are linearly disjoint over $\Q$. Therefore the composite field of $KK'$ and $k$ gives a desired assertion. 

Second, we consider the case that $k$ is a function field. Let $\F$ is the field of constants in $k$, and denote by $\F'$ the unique finite field extension of $\F$ of degree $n_{0}$. Then the field extension $\F'k/k$ is unramified and satisfies $\Gal(\F'k/k)\cong \Z/n_{0}$. On the other hand, fix a finite separable field extension $k/\Fp(t)$. Then Chebotarev's density theorem implies that there exists a place $v$ of $\Fp(t)$ which splits totally in $\F'k$. Moreover, by the proof of \cite[Theorem A.2]{Liang2022}, we can take a cyclic field extension $K_{0}/\Fp(t)$ of degree $\ell_{0}$ which is ramified only at $v$. Then the fields $K_{0}$ and $\F'k$ are linearly disjoint over $\Fp(t)$. Hence, if we set $\widetilde{K}:=K_{0}\F'k$, then we obtain an isomorphism $\Gal(\widetilde{K}/k)\cong G$. Moreover, the assumption on $v$ implies that every decomposition groups of $\widetilde{K}/k$ is cyclic. This completes the proof in this case. 

\textbf{Case 2.~the condition (ii) holds. }This follows from a proof of Shafarevich's theorem, which states the inverse Galois theory holds for finite solvable groups. See \cite[(9.6.7) Theorem]{Neukirch2000}. See also \cite[Theorem 5.1]{Kim2021} in the case that $k$ is a number field. 
\end{proof}

\begin{cor}\label{cpts}
\emph{
\begin{enumerate}
\item Suppose $G'=\Z/n_{1} \times \Z/n_{2}$, where $n_1$ is $1$ or a prime number, and $n_1\mid n_2$. Then there is a finite Galois extension $K'/k$ with Galois group $G'$ which admits an isomorphism
\begin{equation*}
\Sha(K'/k)\cong \Z/n_1. 
\end{equation*}
\item Let $G$ be a finite group of which a $p$-Sylow subgroup $S_{p}$ is normal for a prime divisor $p$ of $\#G$. Take a subgroup of $H$ of $G$ satisfying $N^{G}(H)=\{1\}$ and $\ord_{p}(G:H)=1$. We further assume that $G$ is solvable. Then there is a finite extension $K/k$ which satisfies (1) and (2) as follows. 
\begin{itemize}
\item[(1)] There is an isomorphism $\Gal(K^{\Gal}/k)\cong G$ in which $K$ corresponds to $H$. 
\item[(2)] There is an isomorphism of abelian groups
\begin{equation*}
\Sha(K/k)\cong 
\begin{cases}
\Z/p \oplus \Sha_{\omega}^{2}(G,J_{G/S_{p}H})^{\vee}&\text{if (a), (b) and (c) in Theorem \ref{tspt} hold,}\\
\Sha_{\omega}^{2}(G/S_{p}H,J_{G/S_{p}H})^{\vee}&\text{otherwise.} 
\end{cases}
\end{equation*}
\end{itemize}
\end{enumerate}}
\end{cor}

\begin{proof}
(i): By Proposition \ref{shaf}, there is a finite Galois extension $K'/k$ with Galois group $G'$ for which all decomposition groups of $K'/k$ are cyclic. Then one has an isomorphism
\begin{equation*}
\Sha_{\cD}^{2}(G',J_{G'})=\Sha_{\omega}^{2}(G',J_{G'})\cong \Z/n_1
\end{equation*}
by Proposition \ref{abts}. Hence Proposition \ref{onhn} gives an isomorphism
\begin{equation*}
\Sha(K/k)\cong \Sha_{\omega}^{2}(G',J_{G'})^{\vee}\cong (\Z/n_1)^{\vee}\cong \Z/n_1. 
\end{equation*}
This completes the proof of (i). 

(ii): Since $G$ is solvable, Proposition \ref{shaf} gives the existence of finite Galois extension $\widetilde{K}/k$ with Galois group $G$ for which all decomposition groups of $\widetilde{K}/k$ are cyclic. Now let $K$ be the intermediate field of $\widetilde{K}/k$ corresponding to $H$. Then the triviality of $N^{G}(H)$ implies the equality $K^{\Gal}=\widetilde{K}$. In particular, the condition (1) holds. On the other hand, the condition (2) follows from Proposition \ref{onhn} and Theorem \ref{tspt}. 
\end{proof}

We denote by $\bE(k,p)$ the set of finite separable field extensions $K$ of $k$ such that $p$-Sylow subgroups of $\Gal(K^{\Gal}/k)$ are normal. First, we give an explicit description of the set
\begin{equation*}
\bfS(k,p):=\{d\in \Zpn \mid \text{there is $K\in \bE(k,p)$ so that $\Sha(K/k)[p^{\infty}]\neq 0$}\}. 
\end{equation*}

\begin{thm}\label{ptnt}
\emph{Let $k$ be a global field, and $p$ a prime number which is distinct from the characteristic of $k$. Then one has an equality
\begin{equation*}
\bfS(k,p)=p^{2}\Z \cup D_{1}(p)\cup D_{2}(p). 
\end{equation*}
Here $D_{1}(p)$ and $D_{2}(p)$ are in Definition \ref{d1d2}. In particular, we have
\begin{equation*}
\min \bfS(k,p)=
\begin{cases}
3p &\text{if }p>2,\\
4 &\text{if }p=2. 
\end{cases}
\end{equation*}}
\end{thm}

\begin{proof}
First, note that the equality $\bfS(k,p)=p^{2}\Z \cup D_{1}(p)\cup D_{2}(p)$ follows the assertion by the definitions of $D_{1}(p)$ and $D_{2}(p)$. 

In the following, we give a proof of the equality $\bfS(k,p)=p^{2}\Z \cup D_{1}(p)\cup D_{2}(p)$. 

\textbf{Step 1.~$\bfS(k,p)\subset p\Zpn$. }

Assume that a finite $K$ satisfies $\Sha(K/k)[p^{\infty}]\neq 0$. Since Proposition \ref{onhn} induces a surjection from $\Sha_{\omega}^{2}(k,X^{*}(T_{K/k}^{1}))^{\vee}$ onto $\Sha(K/k)$, Corollary \ref{andg} follows that $[K:k]$ is divisible by $p$. Hence the assertion holds. 

\textbf{Step 2.~$D_{1}(p)\subset \bfS(k,p)$. }

Let $pn\in D_{1}(p)$. We may assume $p\nmid n$. First, assume that $n$ has an odd prime divisor $\ell$ which divides $p-1$. Then there is a $2$-dimensional $\Fp$-representation $(\rho,V)$ of $\Z/\ell$ satisfying (B) and (C), which is a consequence of Proposition \ref{cyrd} (i). This induces a representation of $\Z/n$, and hence Lemma \ref{bcit} and Theorem \ref{ptnt} imply that $\Sha_{\omega}(G,J_{G/H})[p^{\infty}]$ is non-trivial. Therefore Corollary \ref{cpts} (ii) implies the assertion $pn\in \bfS(k,p)$. 

Second, suppose that $n$ is a power of $2$. This happens only when $p\equiv 1\bmod 4$. In this case, the assertion follows from the same argument as the previous case by using Lemma \ref{cyrd} (ii). 

\textbf{Step 3.~$D_{2}(p)\subset \bfS(k,p)$. }

We can prove the same argument as Step 2 by using Lemma \ref{cyir} (iv). 

\textbf{Step 4.~$p^{2}\Zpn \subset \bfS(k,p)$. }

This is a consequence of Corollary \ref{cpts} (i) for $n_1=p$. 

\textbf{Step 5.~$pn\notin \bfS(k,p)$ if $pn\notin D_{1}(p)\cup D_{2}(p)\cup p^{2}\Zpn$. }

Take $K\in \bE(k,p)$ with $[K:k]=pn$. Let $G$, $H$ and $S_{p}$ be as in Corollary \ref{cpts} (without assuming that $G$ is solvable). By Corollary \ref{cpts}, it suffices to prove that at least one of (a), (b) or (c) in Theorem \ref{tspt} is not satisfied. Assume that (a) holds. Write $G=S_{p}\rtimes G'$ and $S_{p}H=S_{p}\rtimes H'$, where $G'$ is a subgroup of $G$ with $p\nmid \#G'$, and $H'$ is a subgroup of $G'$. We regard $S_{p}$ as a $2$-dimensional representation of $G'$. Since $(G':H')=n$, the assumption on $n$ and Proposition \ref{nerp} implies that $S_{p}$ does not satisfy (B) and (C) for $H'$. Therefore at least one of (b) or (c) fails by Proposition \ref{bcit}. 
\end{proof}

\begin{cor}\label{sfhf}
\emph{Let $d$ be a square-free positive composite number which is divisible by at least one of $3$, $55$, $91$ or $95$. Then there is a finite extension $K$ of $k$ of degree $d$ for which the Hasse norm principle does not hold. }
\end{cor}

\begin{proof}
First, assume $3\mid d$. By assumption, there is a prime divisor $p \neq 3$ of $d$. Then $d$ lies in $D_{1}(p)$ if $p\equiv 1\bmod 3$; otherwise we have $d\in D_{2}(p)$. Hence the assertion follows from Theorem \ref{ptnt}. 

On the other hand, the facts $55\in D_{1}(11)$, $91\in D_{2}(13)$ and $95\in D_{2}(19)$ imply the assertion in the case $d\in 55\Zpn \cup 91\Zpn \cup 95\Zpn$. This is a consequence of Theorem \ref{ptnt}. 
\end{proof}

Second, we study a group-theoretic equivalent condition on the triviality of $\Sha(K/k)$ for $K\in \bE(k,p)$ with $[K:k]=p\ell$, where $\ell \neq p$ is a prime number. 

\begin{lem}\label{prpn}
\emph{Let $\ell$ be a prime number satisfying $p\ell \in D_{1}(p)\cup D_{2}(p)$, and $K\in \bE(k,p)$ with $[K:k]=p\ell$. Put $G:=\Gal(K^{\Gal}/k)$, and write $S_{p}$ for the unique $p$-Sylow subgroup of $G$. Then there is an isomorphism
\begin{equation*}
\Sha(K/k)\cong 
\begin{cases}
\Z/p &\text{if (a), (b) and (c) hold and all $D\in \cD_{K^{\Gal}/k}$ do not contain $S_{p}$,}\\
0 &\text{otherwise.} 
\end{cases}
\end{equation*}}
\end{lem}

\begin{proof}
Put $H:=\Gal(K^{\Gal}/K)$. By Proposition \ref{onhn}, there is an isomorphism of finite abelian groups $\Sha(K/k)\cong \Sha_{\cD}^{2}(G,J_{G/H})^{\vee}$, where $\cD:=\cD_{K^{\Gal}/k}$. On the other hand, since $(G:S_{p}H)=\ell$ is a prime number, Proposition \ref{bart} gives the triviality of $\Sha_{\omega}^{2}(G,J_{G/S_{p}H})$. In particular, we have $\Sha_{\cD}^{2}(G,J_{G/S_{p}H})=\Sha_{\omega}^{2}(G,J_{G/S_{p}H})$. Combining this with Theorem \ref{tspt}, one has an isomorphism 
\begin{equation*}
\Sha_{\cD}^{2}(G,J_{G/H})\cong 
\begin{cases}
\Z/p &\text{if (a), (b) and (c) hold and all $D\in \cD$ do not contain $S_{p}$,}\\
0 &\text{otherwise.} 
\end{cases}
\end{equation*}
Therefore the assertion holds. 
\end{proof}

\begin{thm}\label{3pdt}
\emph{Let $p$ and $\ell$ be two distinct prime numbers, and $K\in \bE(k,p)$ with $[K:k]=p\ell$. 
\begin{enumerate}
\item Suppose that one of the following is satisfied: 
\begin{itemize}
\item[(1)] $p>2=\ell$,
\item[(2)] $\ell \nmid p^{2}-1$. 
\end{itemize}
Then the Hasse principle holds for $K/k$. 
\item Assume $p\neq 3=\ell$. Then the Hasse principle holds for $K/k$ unless $p$, $\Gal(K^{\Gal}/k)$ and $\Gal(K^{\Gal}/K)$ fulfill at least one of the following: 
\begin{itemize}
\item[($\alpha$)] $\Gal(K^{\Gal}/k)\cong (\Z/p)^{2}\rtimes_{\varphi_{1}} \Z/3$ and $\Gal(K^{\Gal}/K)\cong \langle(\overline{1},\overline{0})\rangle \rtimes_{\varphi_{1}} \{0\}$, where
\begin{equation*}
\varphi_{1} \colon \Z/3 \rightarrow \Aut((\Z/p)^{2})\cong \GL_{2}(\Fp); 1\mapsto 
\begin{pmatrix}
0&-1\\
1&-1
\end{pmatrix},
\end{equation*}
\item[($\beta$)] $p\geq 5$, $\Gal(K^{\Gal}/k)\cong (\Z/p)^{2}\rtimes_{\varphi_{2}} \fS_{3}$ and $\Gal(K^{\Gal}/K)\cong \langle(\overline{1},\overline{1})\rangle \rtimes_{\varphi_{2}} \langle (1\ 2)\rangle$, where
\begin{equation*}
\varphi_{2}\colon \fS_{3}\rightarrow \Aut((\Z/p)^{2})\cong \GL_{2}(\Fp); (1\ 2\ 3)^{i}(1\ 2)^{j}\mapsto 
\begin{pmatrix}
-1&-1\\
1&0
\end{pmatrix}^{i}
\begin{pmatrix}
0&1\\
1&0
\end{pmatrix}^{j}. 
\end{equation*}
\end{itemize}
Moreover, if ($\alpha$) or ($\beta$) holds, then $\Sha(K/k)=0$ if and only if $(\Z/p)^{2}$ is contained in a decomposition group at some place of $K^{\Gal}$. Otherwise, there is an isomorphism $\Sha(K/k)\cong \Z/p$. 
\end{enumerate}}
\end{thm}

\begin{proof}
(i): This follows from Lemma \ref{prpn} and Theorem \ref{ptnt}. 

(ii): By Lemma \ref{prpn}, it suffices to prove the following: 
\begin{itemize}
\item Under the notations in Corollary \ref{cpts} (without assuming that $G$ is solvable), we further assume $(G:H)=3p$. Then (a), (b) and (c) in Theorem \ref{tspt} hold if and only if at least one of ($\alpha$) or ($\beta$) is satisfied. 
\end{itemize}
Denote $G=S_{p}\rtimes G'$ and $H=L\rtimes H'$, where $G'$ is a subgroup of $G$ with $p\nmid \#G'$, $H'$ is a subgroup of $G'$ and $L$ is a subgroup of $S_{p}$ of order $p$. We regard $S_{p}$ as a $2$-dimensional representation of $G'$. By Proposition \ref{bcit}, the validity of (b) and (c) in Theorem \ref{tspt} is equivalent to the condition that (B) and (C) hold for $H'$ and $L$. In particular, this implies that the group $N^{G'}(H')$ must be trivial by Corollary \ref{trtr}. Hence the prime number $p$ and the pair $H'\subset G'$ satisfy at least one of the following: 
\begin{itemize}
\item[(1)] $G'=\Z/3$ and $H'=\{0\}$,
\item[(2)] $p\geq 5$, $G'=\fS_{3}$ and $H'=\langle (1\ 2)\rangle$. 
\end{itemize}

\textbf{Case (1) with $p\equiv 1\bmod 3$: }First, note that there is an isomorphism
\begin{equation*}
(\Z/p)^{2}\rtimes_{\varphi_{1}}\Z/3 \cong (\chi_{p,3}\oplus \chi_{p,3}^{-1})\rtimes G'. 
\end{equation*}
Moreover, by Proposition \ref{cyrd} (i), all the conditions (a), (b) and (c) are true if ($\alpha$) holds. On the other hand, if (a), (b) and (c) are satisfied, then Proposition \ref{chsm} implies an isomorphism of $G'$-representations $S_{p}\cong \chi_{p,3}\oplus \chi_{p,3}^{-1}$. This induces an isomorphism $L\cong \langle(\overline{1},a)\rangle$ for some $a\in \F_{p}^{\times}$ by the positivity of the condition (C). Consequently, we obtain an isomorphism
\begin{equation*}
\tau \colon G=S_{p}\rtimes G' \xrightarrow{\cong} (\chi_{p,3}\oplus \chi_{p,3}^{-1})\rtimes G'
\end{equation*}
satisfying the equality $\tau(L)=\langle(\overline{1},a)\rangle$. This implies the desired assertion ($\alpha$). 

\textbf{Case (1) with $p\equiv 2 \bmod 3$: }Since $\zeta_{3}+\zeta_{3}^{-1}=-1$, there is an isomorphism
\begin{equation*}
(\Z/p)^{2}\rtimes_{\varphi_{1}}\Z/3 \cong V_{p,3}^{1}\rtimes G'. 
\end{equation*}
Hence ($\beta$) implies (a), (b) and (c) by Proposition \ref{cyir} (iv). Conversely, assume (a), (b) and (c). There is an isomorphism of $G'$-representations $\tau \colon S_{p}\xrightarrow{\cong} V_{p,3}^{1}$ by Proposition \ref{cyir} (iii). Take a generator $v$ of $\tau(L)$. By Proposition \ref{cyir} (iv), one has an homomorphism of $G'$-module
\begin{equation*}
V_{p,3}^{1}\xrightarrow{\cong} V_{p,3}^{1};v\mapsto (\overline{1},\overline{0}). 
\end{equation*}
Therefore we obtain an isomorphism $G\cong V_{p,3}^{1}$ which sends $H$ to $\langle(\overline{1},\overline{0}) \rangle \rtimes \{0\}$, which implies the validity of ($\alpha$). 

\textbf{Case (2): }Since there is an isomorphism
\begin{equation*}
(\Z/p)^{2}\rtimes_{\varphi_{2}}G'\cong V_{p,0}\rtimes \fS_{3},
\end{equation*}
Proposition \ref{s3bc} (ii) implies that the condition (a), (b) and (c) are satisfied if we assume ($\beta$). On the other hand, if we suppose (a), (b) and (c), then one has an isomorphism of $G'$-modules $\tau \colon S_{p}\xrightarrow{\cong} V_{p,0}$ by Proposition \ref{s3bc} (i). Moreover, since $H'$ fixes all elements of $L$, we obtain an equality $\tau(L)=\langle 2-(1\ 2)-(2\ 3)\rangle_{\Fp}$ by Proposition \ref{s3bc} (ii). Therefore the condition ($\beta$) holds. 
\end{proof}

\begin{thm}\label{npex}
\emph{Let $k$ be a global field, and $p\neq 3$ a prime number which does not coincide with the characteristic of $k$.  
\begin{enumerate}
\item There is $K\in \bE(k,p)$ which has degree $9p$ over $k$ so that
\begin{equation*}
\Sha(K/k)\cong \Z/3p. 
\end{equation*}
\item Let $\ell \nmid 3p$ be a prime number which is invertible in $k$. Then there is $K\in \bE(k,p)$ which satisfies $[K:k]=3p\ell$ and
\begin{equation*}
\Sha(K/k)\cong \Z/p\ell. 
\end{equation*}
\end{enumerate}}
\end{thm}

\begin{proof}
Recall that $3$ is contained in $D_{1}(\ell)\cup D_{2}(\ell)$ for any prime number $\ell \neq 3$. Hence, Propositions \ref{cyrd} (i) (in the case $\ell \equiv 1\bmod 3$) and \ref{cyir} (iv) (in the case $\ell \equiv 2\bmod 3$) imply that there is a $2$-dimensional $\F_{\ell}$-representation $V_{\ell}$ of $\Z/3$ satisfying (B) and (C) for the trivial subgroup $\{0\}$ of $\Z/3$ and a $1$-dimensional subspace $L_{\ell}$ of $V_{\ell}$ which is not stable under $\Z/3$. 

(i): We regard $(\rho_{p},V_{p})$ as a representation of $G':=(\Z/3)^{2}$ by the first projection from $G'$ to $\Z/3$, and let $G:=V_{p}\rtimes G'$ be associated the semi-direct product. Then $V_{p}$ satisfies (B) and (C) for $H'$ and the above $L_{p}$. Since $G'$ is solvable, Corollary \ref{cpts} (ii) gives an existence of a finite extension $K/k$ which satisfies an isomorphism 
\begin{equation*}
\Sha(K/k)\cong \Z/p \oplus \Sha_{\omega}^{2}(G',J_{G'})^{\vee}. 
\end{equation*}
On the other hand, one has an isomorphism 
\begin{equation*}
\Sha_{\omega}^{2}(G',J_{G'})^{\vee}\cong (\Z/3)^{\vee}\cong \Z/3. 
\end{equation*}
by Corollary \ref{cpts} (i), and hence the assertion holds. 

(ii): Let $G'$ be the semi-direct product associated to $V_{\ell}$. Endow $V_{p}$ as a representation of $G'$ by the natural surjection $G'=V_{\ell}\rtimes \Z/3 \twoheadrightarrow \Z/3$, and set $G:=V_{p}\rtimes G'$. Then $V_{p}$ satisfies (B) and (C) for $H'=V_{\ell}\rtimes \{0\}$ and $L_{p}$ as above. Since $G'$ is solvable, Corollary \ref{cpts} (ii) gives an existence of a finite extension $K/k$ which satisfies an isomorphism 
\begin{equation*}
\Sha(K/k)\cong \Z/p \oplus \Sha_{\omega}^{2}(G',J_{G'})^{\vee}. 
\end{equation*}
On the other hand, one has an isomorphism 
\begin{equation*}
\Sha_{\omega}^{2}(G',J_{G'/H'})^{\vee}\cong (\Z/\ell)^{\vee}\cong \Z/\ell. 
\end{equation*}
by Corollary \ref{cpts} (ii), and hence the assertion holds. 
\end{proof}

\begin{prop}\label{expp}
\emph{Let $K/k$ be a finite field extension satisfying at least one of the following: 
\begin{enumerate}
\item $[K:k]<18$, 
\item $K\in \bE(k,p)$, $[K:k]\in p\Z$ and $[K:k]<6p$, where $p\geq 5$ is a prime number. 
\end{enumerate}
Then the exponent of $\Sha(K/k)$ is a power of a prime number. }
\end{prop}

\begin{proof}
First, assume the condition (i). By Corollary \ref{shan}, it suffices to prove the assertion in the case $[K:k]\in \{12,15\}$. The case $[K:k]=12$ is a consequence of \cite[Theorem 1.1]{Hoshi2023}. In the case $[K:k]=15$, the assertion follows from \cite[Theorem 1.15]{Hoshi2022}. 

Second, suppose that (ii) holds. If $[K:k]\neq 4p$, then the assertion is a consequence of Corollary \ref{shan} and Lemma \ref{prpn}. From now on, we assume $[K:k]=4p$. We use the notations in Lemma \ref{prpn}. By Proposition \ref{onhn}, it suffices to prove the assertion with $\Sha(K/k)$ replaced by $\Sha_{\omega}^{2}(G,J_{G/H})$. Suppose that $\Sha_{\omega}^{2}(G,J_{G/H})[p^{\infty}]$ is non-zero. Then Theorem \ref{ptnt} implies the validity of  (a), (b) and (c) and an isomorphism
\begin{equation*}
\Sha_{\omega}^{2}(G,J_{G/H})^{(p)}\cong \Sha_{\omega}^{2}(G,J_{G/S_{p}H})\cong \Sha_{\omega}^{2}(G'/J_{G'/H'}). 
\end{equation*}
We regard $S_{p}$ as an $\Fp$-representation of $G'$. Then it satisfies (B) and (C) by Lemma \ref{cyir} (ii). Moreover, we have $(G':H')=4$. Therefore Proposition \ref{nerp} gives an isomorphism $G'\cong \Z/4$. This implies that $\Sha_{\omega}^{2}(G',J_{G'/H'})$ is trivial, which is a consequence of Propositions \ref{tsan} (i) and \ref{enmy}. 
\end{proof}

\appendix
\section{Second cohomology of multinorm one tori}\label{apdx}

In this appendix, we give a proof of the following: 

\begin{thm}[{Theorem \ref{blpl}}]\label{apth}
\emph{Let $G$ be a finite group, and $H_1,\ldots,H_r$ normal subgroups of $G$ of index a prime number $p$ that satisfy $H_{i}\not\subset H_{j}$ for distinct $i,j\in \{1,\ldots,r\}$. Put $N:=\bigcap_{i=1}^{r}H_{i}$ and $m:=\ord_{p}(G:N)$. Take $e_1,\ldots,e_r\in \Zpn$. Then there is an isomorphism
\begin{equation*}
\Sha_{\omega}^{2}\left(G,J_{G/(H_1^{(e_1)},\ldots,H_r^{(e_r)})}\right)\cong 
\begin{cases}
(\Z/p)^{\oplus r-2}&\text{if $m=2$ and $r\geq 3$},\\
0&\text{otherwise}. 
\end{cases}
\end{equation*}}
\end{thm}

\begin{proof}
By Proposition \ref{ifts}, we may assume that $\bigcap_{i=1}^{r}H_{i}$ is trivial. In particular, there is an isomorphism $G\cong (\Z/p)^{m}$. Moreover, we may further assume $e_1=\cdots =e_r=1$ by Lemma \ref{endo}. Since $G$ is solvable, Proposition \ref{shaf} (ii) implies that there exists a finite abelian extension $\widetilde{K}/\Q$ with Galois group $G$ in which all decomposition groups are cyclic. For each $i\in \{1,\ldots,r\}$, let $K_i$ be the subfield of $\widetilde{K}$ corresponding to $H_{i}$ under the isomorphism $\Gal(\widetilde{K}/\Q)\cong G$. Furthermore, put $\bK:=\prod_{i=1}^{r}K_{i}$. Then we have $X^{*}(T_{\bK/\Q}^{1})\cong J_{G/(H_1,\ldots,H_r)}$ by Proposition \ref{mnjg}. Since $\cD_{\widetilde{K}/\Q}=\cC_{G}$, Lemma \ref{almt} and Proposition \ref{onhn} imply an isomorphism
\begin{equation*}
\Sha_{\omega}^{2}(G,J_{G/(H_1,\ldots,H_r)})\cong \Sha^{2}(k,X^{*}(T_{\bK/\Q}^{1})). 
\end{equation*}
Hence the assertion follows from \cite[Proposition 8.5]{BayerFluckiger2019}. 
\end{proof}

\begin{thebibliography}{99}
\bibitem[Bar81]{Bartels1981a}
H.~J.~Bartels, \emph{Zur Arithmetik von Konjugationsklassen in algebraischen Gruppen}, J.~Alg.~\textbf{70} (1981), 179--199. 
\bibitem[BLP19]{BayerFluckiger2019}
E.~Bayer-Fluckiger, T.~-Y.~Lee, R.~Parimala, \emph{Hasse principle for multinorm equations}, Adv.~Math.~\text{356} (2019), 106818. 
\bibitem[CF64]{Carter1964}
R.~Carter, P.~Fong, \emph{The $2$-Sylow subgroups of classical groups over finite fields}, J.~Algebra \text{1} (1964), 139--151. 
\bibitem[CS77]{ColliotThelene1977}
J.-L.~Colliot--Th{\'e}l{\`e}ne, J-J.~Sansuc, \emph{La $R$-{\'e}quivalence sur les tores}, Ann.~sci.~de l'{\'E}.N.S.~$4^{e}$ s{\'e}rie, \textbf{10} no.~2 (1977), 175--229. 
\bibitem[CS87]{ColliotThelene1987}
J.-L.~Colliot-Th{\'e}l{\`e}ne, J-J.~Sansuc, \emph{Principal homogeneous spaces under flasque tori: applications}, J.~Algebra \textbf{106} (1987), no.~1, 148--205. 
\bibitem[DP87]{Drakokhrust1987}
Y.~A.~Drakokhrust, V.~P.~Platonov, \emph{The Hasse norm principle for algebraic number fields} (Russian) Izv.~Akad.~Nauk SSSR Ser.~Mat.~\textbf{50} (1986), 946--968; translation in Math.~USSR-Izv.~\textbf{29} (1987), 299--322. 
\bibitem[EM75]{Endo1975}
S.~Endo, T.~Miyata, \emph{On a classification of the function fields of algebraic tori}, Nagoya Math.~J.~\textbf{56} (1975), 85--104. 
\bibitem[End11]{Endo2011}
S.~Endo, \emph{The rationality problem for norm one tori}, Nagoya Math.~J.~\textbf{202} (2011), 83--106. 
\bibitem[FLN18]{Frei2018}
C.~Frei, D.~Loughran, R.~Newton, \emph{The Hasse norm principle for abelian extensions}, Amer.~J.~Math.~\textbf{140} (2018), 1639--1685. 
\bibitem[Has31]{Hasse1931}
H.~Hasse, \emph{Beweis eines Satzes und Wiederlegung einer Vermutung {\"u}ber das allgemeine Normenrestsymbol}, Nschrichten von der Gesellschaft der Wissenschaften zu G{\"o}ttingen, Mathematisch-Physikalische Klasse (1931), 64--69.  
\bibitem[HKY22]{Hoshi2022}
A.~Hoshi, K.~Kanai, A.~Yamasaki, \emph{Norm one tori and Hasse norm principle}, Math.~Comput.~\textbf{91} (2022), 2431--2458. 
\bibitem[HKY23a]{Hoshi2023}
A.~Hoshi, K.~Kanai, A.~Yamasaki, \emph{Norm one tori and Hasse norm principle II: degree $12$ case}, J.~Number Theory \textbf{245} (2023), 84--110. 
\bibitem[HKY23b]{Hoshi2023b}
A.~Hoshi, K.~Kanai, A.~Yamasaki, \emph{Hasse norm principle for $M_{11}$ and $J_{1}$ extensions}, preprint, arXiv:2210.09119v2, 2023. 
\bibitem[HHLY22]{Huang2022}
J.-H.~Huang, F.-Y.~Hung, P.-X.~Liang, C.-F.~Yu, \emph{Computing Tate-Shafarevich groups of multinorm one tori of Kummer type}, preprint, arXiv:2211.11249, 2022. 
\bibitem[Hun74]{Hungerford1974}
T.~W.~Hungerford, \emph{Algebra}, Grad.~Text.~in Math., Vol.~73, Springer, 1974. 
\bibitem[H{\"u}r84]{Hurlimann1984}
W.~H{\"u}rlimann, \emph{On algebraic tori of norm type}, Comment.~Math.~Helv., \textbf{59}, no.~4 (1984), 539--549. 
\bibitem[Kat85]{Katayama1985}
S.~Katayama, \emph{On the Tamagawa numbers of algebraic tori}, S\={u}gaku, \textbf{37}, no.~1 (1985), 81--83. 
\bibitem[KK21]{Kim2021}
K.-S.~Kim, J.~K{\"o}nig, \emph{On Galois extensions with prescribed decomposition groups}, J.~Number Theory \textbf{220} (2021), 266--294. 
\bibitem[LOYY22]{Liang2022}
P.-X.~Liang, Y.~Oki, H.~-Y.~Yang, C.-F.~Yu, \emph{On Tamagawa numbers of CM tori}, to appear in Alg.~Number Theory, arXiv:2109.04121v3, 2022. 
\bibitem[MN21]{Macedo2021}
A.~Macedo, R.~Newton, \emph{Explicit methods for the Hasse norm principle and applications to $A_{n}$ and $S_{n}$-extensions}, Math.~Proc.~Phil.~Soc.~\textbf{172}, Issue 3 (2021), 489--529. 
\bibitem[NSW00]{Neukirch2000}
J.~Neukirch, A.~Schmidt, K.~Wingberg, \emph{Cohomology of number fields}, Springer Verlag, 2000. 
\bibitem[Ono63]{Ono1963}
T.~Ono, \emph{On Tamagawa numbers of algebraic tori}, Ann.~Math. (1963). 
\bibitem[Ser77]{Serre1977}
J.-P.~Serre, \emph{Linear representations of finite groups}, Springer-Verlag, New York-Heidelberg, 1977, translated from the second French edition by L.~L.~Scott, Gard.~Texts in Math., Vol.~42. 
\bibitem[Ser79]{Serre1979}
J.-P.~Serre, \emph{Local Fields}, Grad.~Texts in Math.,Vol.~67, Springer, 1979. 
\bibitem[Tah72]{Tahara1972}
K.~Tahara, \emph{On the second cohomology groups of semidirect products}, Math.~Z.~\textbf{129} (1972), 365--379. 
\bibitem[Vos98]{Voskresenskii1998}
V.~E.~Voskresenskii, \emph{Algebraic groups and their birational invariants}, translated from the Russian manuscript by B.~Kunyavskii, Trans.~Math.~Monographs, 179, Amer.~Math.~Soc.~Providence, RI, 1998. 
\end{thebibliography}

\end{document}