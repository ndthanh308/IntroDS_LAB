\documentclass[aps,prd,amsmath
,nofootinbib         % Stops RevTeX moving footnotes to the references/bibliography
%,twocolumn         % for two columns per page, as in the journal.  Only do this when close to submission.
]{revtex4}
\usepackage{amsmath}
\usepackage{amsfonts}
\usepackage{amssymb}
% \usepackage{multicol} % allows a choice of column number, cannot be used with revtex4

\allowdisplaybreaks[4]     % allows page breaks in the middle of an equation group.


\begin{document}

\title{Interior spacetimes sourced by stationary differentially rotating irrotational cylindrical fluids. III. Azimuthal pressure}

\author{M.-N. C\'el\'erier}
\affiliation{Laboratoire Univers et Th\'eories, Observatoire de Paris, Universit\'e PSL, Universit\'e Paris Cit\'e, CNRS, F-92190 Meudon, France}
\email{marie-noelle.celerier@obspm.fr}

\date{25 July 2023}

\begin{abstract}

In a recent series of papers, new exact analytical solutions to field equations of General Relativity representing interior spacetimes sourced by stationary rigidly rotating cylinders of fluids with various equations of state have been displayed. This work is currently extended to the case of differentially rotating irrotational fluids. The results are presented in a new series of papers considering, in turn, a perfect fluid source, arXiv:2305.11565 [gr-qc], as well as the three anisotropic pressure cases already studied in the rigidly rotating configuration. The axially directed pressure case has already been developed in arXiv:2307.07263. Here, a fluid with an azimuthally directed pressure is considered. A general method for generating the corresponding new mathematical solutions to the field equations when the ratio $h=$pressure/energy density varies with the radial coordinate is proposed, and a class of solutions exemplifying this recipe is derived. Then, the case where $h=const.$ is solved. It splits into two subclasses depending on the value of $h$. The mathematical and physical properties of these three classes are analyzed which provides some constraints on $h$, different for each class and subclass. Their matching to an exterior Lewis-Weyl vacuum and the conditions for avoiding an angular deficit are discussed. A comparison with the rigidly rotating fluid case is provided.

\end{abstract}

\maketitle

\section{Introduction} \label{intro}

Exact solutions for interior spacetimes are very valuable in General Relativity (GR), since their field equations are more difficult to integrate than for vacuum, and therefore their scarcity offset the usual simplifications made in addition. This is indeed the case when cylindrical symmetric and stationarity are involved, while physically standard sources are assumed.

In a recent series of papers new exact analytical solutions to field equations of GR representing interior spacetimes sourced by stationary rigidly rotating cylinders of fluids with different equations of state have been displayed. In the first two \cite{C21,C23a}, the fluid exhibits axially directed pressure. Then, we have considered a perfect fluid \cite{C23b}. In the following paper\cite{C23c}, we have analyzed a fluid with an azimuthally directed pressure. Finally, in the last paper of this rigidly rotating fluid series \cite{C23d}, the anisotropic pressure has been radially oriented.

 This work is currently extended to the case of differentially rotating irrotational fluids. The results are presented in a new series of papers considering, in turn, a perfect fluid source \cite{C23e}, as well as the three anisotropic pressure cases already studied in the rigidly rotating configuration. The axially directed pressure case has already been achieved \cite{C23f}. 
 
 In the present paper, a fluid with an azimuthally directed pressure is considered. A general method for generating the corresponding new mathematical solutions to the field equations when the ratio $h=$pressure/energy density varies with the radial coordinate is proposed, and a class of solutions exemplifying this recipe, denoted class A, is derived. Then, the case where $h=const.$, denoted class B, is solved. It splits into two subclasses: class Bi, for $h \neq 1$, and class Bii, where $h=1$. The mathematical and physical properties of these three classes are analyzed which provides some constraints on $h$, different for each class and subclass. Their matching to an exterior Lewis-Weyl vacuum and the conditions for avoiding an angular deficit are discussed. A comparison with the rigidly rotating fluid case is provided.
 
 The paper is organized as follows. The statement of the problem and the main equations to be solved are displayed in Sec.\ref{gen}. A general method for solving the field equations in the case where $h=h(r)$ is proposed in Sec.\ref{solvingb}. It is exemplified in Sec.\ref{solvinga}, where the class A solutions are derived. In Sec.\ref{solvingc}, both subclasses of class B where $h=cst.$ are described and discussed. A comparison with the rigidly rotating solutions with azimuthally directed pressure is displayed in Sec.\ref{comp}. Section \ref{concl} is devoted to the conclusions.

\section{Statement of the problem} \label{gen}

The general properties of interior spacetimes sourced by stationary rotating fluids have been established by C\'el\'erier and Santos \cite{CS20} and specialized to the case of differential rotation by C\'el\'erier \cite{C23e,C23f}. In particular, it has been shown that the problem is under-determined. One degree of freedom has been employed there such as to simplify the set of equations to be solved. This ansatz amounts to making the rotation tensor of the fluid vanishing. It is still used here where irrotational fluids are also considered. Since part of the equations displayed in the first papers of the series will be needed in the present article, they are recalled below and specialized to fluids with azimuthally directed pressure.

\subsection{Spacetime inside the source} \label{inside}

The stress-energy tensor of a stationary cylindrically symmetric anisotropic nondissipative fluid reads
\begin{equation}
T_{\alpha \beta} = (\rho + P_r) V_\alpha V_\beta + P_r g_{\alpha \beta} + (P_\phi - P_r) K_\alpha K_\beta + (P_z - P_r)S_\alpha S_\beta, \label{setens1}
\end{equation}
where $\rho$ is the energy density of the fluid, $P_r$, $P_z$ and $P_\phi$ are the principal stresses and $V_\alpha$, $K_\alpha$ and $S_\alpha$ are 4-vectors satisfying
\begin{equation}
V^\alpha V_\alpha = -1, \quad K^\alpha K_\alpha = S^\alpha S_\alpha = 1, \quad V^\alpha K_\alpha = V^\alpha S_\alpha = K^\alpha S_\alpha =0. \label{fourvec}
\end{equation}
For a fluid with azimuthal pressure, whose equation of state reads $P_z = P_r = 0$, the stress-energy tensor simplifies as
\begin{equation}
T_{\alpha \beta} = \rho V_\alpha V_\beta + P_\phi K_\alpha K_\beta. \label{setens2}
\end{equation}
The fluid being bounded by a cylindrical hyper-surface $\Sigma$, the spacelike Killing vector, $\partial_z$, is assumed to be hypersurface orthogonal, in view of its subsequent matching to an exterior Lewis metric. Hence, the  line element can be written as
\begin{equation}
\textrm{d}s^2=-f \textrm{d}t^2 + 2 k \textrm{d}t \textrm{d}\phi +\textrm{e}^\mu (\textrm{d}r^2 +\textrm{d}z^2) + l \textrm{d}\phi^2, \label{metric}
\end{equation}
where $f$, $k$, $\mu$ and $l$ are real functions of the radial coordinate $r$ only. Owing to cylindrical symmetry, the coordinates conform to the following ranges
\begin{equation}
-\infty \leq t \leq +\infty, \quad 0 \leq r, \quad -\infty \leq z \leq +\infty, \quad 0 \leq \phi \leq 2 \pi, \label{ranges}
\end{equation}
with the two limits of the coordinate $\phi$ topologically identified. These coordinates are numbered $x^0=t$, $x^1=r$, $x^2=z$ and $x^3=\phi$.

The 4-velocity of the fluid, satisfying (\ref{fourvec}), can be written as
\begin{equation}
V^\alpha = v \delta^\alpha_0 + \Omega  \delta^\alpha_3, \label{4velocity}
\end{equation}
where $v$ and $\Omega$ are functions of $r$ only. The timelike condition for $V^\alpha$ displayed in (\ref{fourvec}) becomes therefore
\begin{equation}
fv^2 - 2kv\Omega - l\Omega^2 -1= 0. \label{timelike}
\end{equation}
The spacelike 4-vector $K^{\alpha}$ defining the stress-energy tensor and verifying (\ref{fourvec}) is chosen as
\begin{equation}
K^\alpha = -\frac{1}{D}\left[(kv+l\Omega)\delta^\alpha_0 + (fv - k\Omega)\delta^\alpha_3\right], \label{kalpha}
\end{equation}
with
\begin{equation}
D^2 \equiv fl + k^2. \label{D2}
\end{equation}

\subsection{Irrotational assumption and related equations} \label{feq}

It has been previously shown \cite{C23e,C23f} that the irrotational motion assumption, besides reducing the degrees of freedom displayed by the problem, allows us to simplify its mathematics, since it yields
\begin{equation}
kv + l\Omega=0. \label{partint8}
\end{equation}
Indeed, (\ref{partint8}) implies a vanishing rotation tensor. Inserting (\ref{partint8}) into (\ref{timelike}), one obtains
\begin{equation}
v^2 = \frac{l}{D^2}, \label{partint9}
\end{equation}
which, moreover, imposes the positiveness of the metric function $l$. This choice implies \cite{C23e,C23f}
\begin{equation}
kl' - lk' = 2c D, \label{partint11}
\end{equation}
where $c$ is an integration constant and the factor 2 is retained for further convenience. Considered as a first-order ordinary differential equation in $k$, (\ref{partint11}) possesses, as a general solution,
\begin{equation}
k = l \left(c_k - 2c \int^r_{r_1} \frac{D(v)}{l^2(v) } \textrm{d} v \right), \label{partint12}
\end{equation}
where $r_1$ denotes an arbitrary value of the radial coordinate located between $r=0$ and $r_{\Sigma}$, its value at the boundary. Indeed, any limiting value of $r$ can be chosen here, a change of limit amounting merely to a redefinition of the constant of integration $c_k$. Then, from (\ref{partint11}), we can obtain \cite{C23e,C23f}
\begin{equation}
\frac{f'l' +k'^2}{2D^2} = \frac{l'D'}{lD} - \frac{l'^2}{2l^2} + \frac{2c^2}{l^2}. \label{partint14}
\end{equation}

\subsection{Conservation of the stress-energy tensor}

The conservation of the stress-energy tensor is implemented by the Bianchi identity. With $V^\alpha$ given by (\ref{4velocity}), and the space-like vector $K^\alpha$ given by (\ref{kalpha}), inserted into (\ref{setens2}), while using (\ref{metric}) and (\ref{timelike}), this identity reduces to
\begin{equation}
T^\beta_{1;\beta} = - (\rho + P_\phi) \Psi - P_\phi\frac{D'}{D}  = 0, \label{Bianchi1}
\end{equation}
with
\begin{equation}
\Psi = fvv' - k(v\Omega)' - l\Omega\Omega' = -\frac{1}{2} (v^2f' - 2v\Omega k' - \Omega^2l'), \label{psidef}
\end{equation}
which becomes, with the choice (\ref{partint8}),
\begin{equation}
\Psi = \frac{l'}{2l} - \frac{D'}{D}, \label{psi8}
\end{equation}
which, once inserted into the Bianchi identity (\ref{Bianchi1}), yields
\begin{equation}
(\rho + P_\phi)\left( \frac{l'}{2l} - \frac{D'}{D}\right) + P_\phi\frac{D'}{D} = 0. \label{Bianchi2}
\end{equation}
With $h(r)$ defined henceforth as $h(r)\equiv P_{\phi}(r)/\rho(r)$, (\ref{Bianchi2}) becomes
\begin{equation}
(1+h) \frac{l'}{2l}- \frac{D'}{D} = 0. \label{Bianchi4a}
\end{equation}

\subsection{Junction conditions} \label{junct}

These conditions have already been displayed in \cite{CS20,C21,C23a,C23b,C23c,D06} and fully examined by C\'el\'erier \cite{C23d} for the metric (\ref{metric}). As they have been discussed at length in those papers, only their main resulting constraint is recalled here and it is shown to be perfectly compatible with the systems described in the present work. 
 
Since the motion of the source is stationary, the Lewis metric \cite{L32} is well suited to represent the exterior spacetime, and the Weyl class of real metrics \cite{W19} is chosen such as to obtain junction conditions for a physically well-behaved fluid. 

In accordance with Darmois' junction conditions \cite{D27}, the coefficients of the interior metric (\ref{metric}) and of the Lewis-Weyl exterior together with their derivatives must be continuous across the boundary surface $\Sigma$. It has been shown by Debbasch et al.  \cite{D06} that these constraints imply that the radial coordinate, $P_r$, of the pressure vanishes on the boundary. Now, owing to the equations of state applying to the fluids under consideration here and displayed in Sec.\ref{inside}, $P_r$ vanishes at any point of the corresponding spacetimes. Therefore, it is indeed zero on the boundary and Darmois' junction conditions are satisfied.

\subsection{Hydrodynamical properties} \label{genhydro}

The hydrodynamical properties of such differentially rotating fluids have been analyzed by C\'el\'erier \cite{C23e,C23f}. We recall here the main results which will be needed in the following of the paper.

These hydrodynamical features have been obtained with the use of a quantity named $\Psi$ and defined here by (\ref{psidef}). This quantity is used in the determination of the only nonzero component of the acceleration vector that reads
\begin{equation}
\dot{V}_1 = -\Psi = \frac{D'}{D} -\frac{l'}{2l} . \label{hydr2}
\end{equation}
Hence the modulus of this acceleration vector comes easily as
\begin{equation}
\dot{V}^{\alpha}\dot{V}_{\alpha}= \textrm{e}^{-\mu} \left(\frac{D'}{D} -\frac{l'}{2l}\right)^2 . \label{hydr3}
\end{equation}

All the components of the rotation or twist tensor vanish, which must be viewed as a confirmation of the irrotational motion of the fluid.

The nonzero components of the shear tensor have also been determined as
\begin{equation}
2 \sigma_{01} = \frac{2ck}{l\sqrt{l}}, \label{hydr9a}
\end{equation}
\begin{equation}
2 \sigma_{13} = \frac{2c}{\sqrt{l}}. \label{hydr10}
\end{equation}
The shear scalar is therefore given by
\begin{equation}
\sigma^2 = c^2 \frac{\textrm{e}^{-\mu}}{l^2}. \label{hydr12}
\end{equation}

\subsection{Field equations} \label{fea}

By inserting (\ref{partint8}) and the equations of state displayed in Sec.\ref{inside} into (10)-(14) of C\'el\'erier and Santos \cite{CS20}, the five field equations for the spacetimes inside $\Sigma$ can be written as
\begin{equation}
G_{00} = \frac{\textrm{e}^{-\mu}}{2} \left[-f\mu'' - 2f\frac{D''}{D} + f'' - f'\frac{D'}{D} + \frac{3f(f'l' + k'^2)}{2D^2} \right] = \frac{\kappa}{l} \left(D^2 \rho + k^2 P_{\phi} \right), \label{G00a}
\end{equation}
\begin{equation}
G_{03} =  \frac{\textrm{e}^{-\mu}}{2} \left[k\mu'' + 2 k \frac{D''}{D} -k'' + k'\frac{D'}{D} - \frac{3k(f'l' + k'^2)}{2D^2}\right] = \kappa P_{\phi} k, \label{G03a}
\end{equation}
\begin{equation} 
G_{11} = \frac{\mu' D'}{2D} + \frac{f'l' + k'^2}{4D^2} = 0, \label{G11a}
\end{equation}
\begin{equation}
G_{22} = \frac{D''}{D} -\frac{\mu' D'}{2D} - \frac{f'l' + k'^2}{4D^2} = 0, \label{G22a}
\end{equation}
\begin{equation}
G_{33} =  \frac{\textrm{e}^{-\mu}}{2} \left[l\mu'' + 2l\frac{D''}{D} - l'' + l'\frac{D'}{D} - \frac{3l(f'l' + k'^2)}{2D^2}\right] =  \kappa P_{\phi} l. \label{G33a}
\end{equation}

\section{Solving the field equations for $h=h(r)$. A general method} \label{solvingb}

The addition of both field equations (\ref{G11a}) and (\ref{G22a}) yields
\begin{equation}
D'' = 0, \label{sol1a}
\end{equation}
which can be integrated as
\begin{equation}
D = c_1r+c_2, \label{sol2a}
\end{equation}
where $c_1$ and $c_2$ are integration constants.

The coordinate $r$ can be rescaled from a factor $c_1$, giving
\begin{equation}
D = r+c_2. \label{sol3a}
\end{equation}
This implies therefore
\begin{equation}
\frac{D'}{D} = \frac{1}{r+c_2}. \label{sol4a}
\end{equation}

The problem being to solve nine equations, namely, (\ref{timelike}), (\ref{partint8}), (\ref{G00a})--(\ref{G33a}) and the double equation of state $P_r = P_z = 0$, for ten unknown functions of $r$, which are $f$, $k$, $\mu$, $l$, $v$, $\Omega$, $\rho$ and $P_{\phi}$, one more equation determining the system is thus needed. 

A special form of the $l$ function, designed such as to allow an easy integration of the Bianchi identity (\ref{Bianchi4a}), is defined by
\begin{equation}
l' = \frac{\textrm{d}l}{\textrm{d}h} h', \label{solA}
\end{equation}
and chosen such as to be integrable with respect to $h$, giving $l(h)$. We insert (\ref{solA}) divided by $l(h)$ into the Bianchi identity (\ref{Bianchi4a}) and obtain
\begin{equation}
\frac{D'}{D} = \frac{(1+h)}{2}\frac{\frac{\textrm{d}l}{\textrm{d}h} h'}{l}, \label{solB}
\end{equation}
which is designed such as to be also integrable with respect to $r$ under the form 
\begin{equation}
D=D(h). \label{solC}
\end{equation}
Now, by equalizing both expressions for $D$, (\ref{sol3a}) and (\ref{solC}), we obtain an equation for $h$ as a function of $r$ under the form
\begin{equation}
r+c_2 = D(h). \label{solD}
\end{equation}
By differentiating (\ref{solD}) with respect to $r$, we obtain
\begin{equation}
h'(h) = \frac{1}{\frac{\textrm{d}D(h)}{\textrm{d}h}}. \label{solE}
\end{equation}

Then, by substituting (\ref{partint14}) into (\ref{G11a}), we obtain
\begin{equation}
\frac{\mu' D'}{D} = -\frac{l'D'}{lD} + \frac{l'^2}{2l^2} - \frac{2c^2}{l^2}, \label{partint15}
\end{equation}
where we substitute (\ref{solA}), (\ref{solB}), and $l(h)$ which gives
\begin{equation}
\mu' = {\frac{\textrm{d}\mu(h)}{\textrm{d}h}}h'. \label{solF}
\end{equation}
If the choice of $l(h)$ has been judiciously made, $\textrm{d}\mu(h) /\textrm{d}h$ happens to be an integrable function of $h$ and, therefore, $\mu(h)$ follows.

Knowing $D(h)$, $l(h)$, and $h'(h)$, the metric function $k$ proceeds from (\ref{partint12}). Then $f$ is obtained from the definition (\ref{D2}) of $D^2$.

The field equation (\ref{G33a}), with (\ref{G11a}), (\ref{G22a}), and the metric functions, $D$ and derivatives inserted, yields the expression for the pressure $P_{\phi}$, from which $\rho= P_{\phi}/h$ proceeds. The differential angular velocity $\Omega$ and the velocity $v$ of the fluid follow from (\ref{partint8}) and (\ref{partint9}).

Provided that the $l$ function has been chosen so that the integrations can be completed at each stage of the reasoning, we have thus displayed a general method for deriving different classes of solutions to the problem. 

To exemplify this method, a particular class of solutions is now displayed in Sec. \ref{solvinga}.

\section{Solving the field equations for $h=h(r)$. The class A example} \label{solvinga}

A particular form of the $l$ function, designed such as to allow an easy integration of the Bianchi identity (\ref{Bianchi4a}), is chosen as
\begin{equation}
l = c_l^2 \frac{h^2}{(1-h)^2}, \label{sol5a}
\end{equation}
from which we derive
\begin{equation}
\frac{l'}{l} = \frac{2h'}{h(1-h)}, \label{sol7a}
\end{equation}
which we substitute into the Bianchi identity (\ref{Bianchi4a}), so as to obtain
\begin{equation}
\frac{D'}{D} = \frac{(1+h)h'}{h(1-h)}, \label{sol8a}
\end{equation}
which can be integrated by
\begin{equation}
D = c_D\frac{h}{(1-h)^2}, \label{sol9a}
\end{equation}
where $c_D$ is an integration constant. By equalizing the expression for $D$ as given by (\ref{sol3a}) with that given by (\ref{sol9a}), we obtain
\begin{equation}
r +c_2= c_D\frac{h}{(1-h)^2}, \label{sol10a}
\end{equation}
which is a quadratic equation in $h$ whose double solution reads, for $r\neq0$,
\begin{equation}
h = 1 +\frac{c_D}{2(r+c_2)} + \epsilon \sqrt{\left[1 + \frac{c_D}{2(r+c_2)} \right]^2 -1}, \label{sol11a}
\end{equation}
where $\epsilon = \pm 1$ allows for two solutions.

Differentiating (\ref{sol10a}) with respect to $r$ gives
\begin{equation}
h' = \frac{(1-h)^3}{c_D(1+h)}. \label{sol12a}
\end{equation}

Now, (\ref{sol5a}), (\ref{sol7a}) and (\ref{sol8a}) are inserted into (\ref{partint15}), giving
\begin{equation}
\mu' = - \frac{2h'}{(1-h)(1+h)} - \frac{2c^2 c_D^2}{c_l^4} \frac{(1+h)h'}{h^3(1-h)}, \label{sol13a}
\end{equation}
which can be integrated as
\begin{equation}
\textrm{e}^{\mu} = c_{\mu}\frac{(1-h)^{\frac{4c^2c_D^2}{c_l^4} +1}}{h^{\frac{4c^2c_D^2}{c_l^4}}(1+h)} \exp\left[\frac{2c^2c_D^2}{c_l^4}\left(\frac{2}{h} + \frac{1}{2h^2}\right)\right], \label{sol14a}
\end{equation}
where the integration constant $c_{\mu}$ can be absorbed into a rescaling of the $r$ and $z$ coordinates. 

The metric function $k$ is obtained by substituting the functions $l$ and $D$, given by (\ref{sol5a}) and (\ref{sol9a}), into (\ref{partint12}), which yields
\begin{equation}
k = c_l^2 \frac{h^2}{(1-h)^2} \left[c_k - \frac{2c c_D}{c_l^4} \int^r_{r_{\Sigma}} \frac{(1-h(v))^2}{h(v)^3 } \textrm{d} v \right], \label{sol16a}
\end{equation}
where we multiply the integrand in the right hand side by $1= c_D (1+h)h'/(1-h)^3$, owing to (\ref{sol12a}), and carry out a change of integration variable to obtain
\begin{equation}
k = c_l^2 \frac{h^2}{(1-h)^2} \left[c_k - \frac{2c c_D ^2}{c_l^4} \int^h_{h_{\Sigma}} \frac{(1+v)}{v^3 (1-v)} \textrm{d} v \right]. \label{sol17a}
\end{equation}
After integration and absorption of the new integration constant into $c_k$, one obtains
\begin{equation}
k = c_l^2 \frac{h^2}{(1-h)^2} \left[c_k + \frac{2c c_D^2}{c_l^4} \left(\frac{2}{h} + \frac{1}{2h^2} + \ln \frac{(1-h)^2}{h^2} \right) \right]. \label{sol18a}
\end{equation}
Finally, the metric function $f$ can be calculated, using (\ref{D2}), as 
\begin{equation}
f = \frac{1}{c_l^2 (1-h)^2} - c_l^2 \frac{h^2}{(1-h)^2}\left[c_k  + \frac{2c c_D^2}{c_l^4} \left(\frac{2}{h} + \frac{1}{2h^2} + \ln \frac{(1-h)^2}{h^2} \right) \right]^2.  \label{sol19a}
\end{equation}

\subsection{Axisymmetry and so-called ``regularity'' conditions} \label{regcond}

To satisfy the axisymmetry condition, the spacetimes under study must verify $l\stackrel{0}{=}0$ \cite{C23a,S09}, where $l\stackrel{0}{=}$ denotes that the quantities are evaluated at the axis. Now, (\ref{sol5a}) evaluated at $r=0$ where the value of $h$ is denoted $h_0$ gives
\begin{equation}
l \stackrel{0}{=}  c_l^2 \frac{h_0^2}{(1-h_0)^2}, \label{arc1}
\end{equation}
which vanishes provided
\begin{equation}
h_0 = 0. \label{arc2}
\end{equation}
Since (\ref{sol10a}) evaluated  on the axis yields
\begin{equation}
c_2 =c_D \frac{h_0}{(1-h_0)^2}, \label{arc3}
\end{equation}
a consequence of (\ref{arc2}) is
\begin{equation}
c_2 =0. \label{arc4}
\end{equation}
Then, absorbing $c_D$ into a new rescaling of the radial coordinate $r$, $c_D=1$ can be substituted everywhere in the relevant expressions.

The so-called ``regularity'' condition which is indeed a mere condition for avoiding an angular deficit in the vicinity of the symmetry axis \cite{C23f} and which reads
\begin{equation}
\frac{\textrm{e}^{-\mu}l'^2}{4l} \stackrel{0}{=} 1. \label{arc4a}
\end{equation}
becomes, in this case,
\begin{equation}
\frac{c_l^2 h_0^{\frac{4c^2}{c_l^4}}}{(1-h_0)^{\frac{4c^2}{c_l^4}-1}(1+ h_0)} \exp\left[- \frac{2c^2}{c_l^4}\left(\frac{2}{h_0} + \frac{1}{2h_0^2}\right)\right]=1, \label{arc5}
\end{equation}
whose left hand side vanishes for $h_0=0$. Therefore this condition is not verified by such spacetimes.

\subsection{Energy density.} \label{endens}

We calculate first the component $P_{\phi}$ of the pressure. The field equation (\ref{G33a}), with (\ref{G11a}) and (\ref{G22a}) inserted, becomes
\begin{equation}
\mu'' - \frac{l''}{l} +\frac{l'D'}{lD} + \frac{3\mu' D'}{D} = 2 \kappa P_{\phi} \textrm{e}^{\mu}, \label{sol20a}
\end{equation}
where the metric functions, the $D$ function and their derivatives are inserted, which yields
\begin{equation}
\kappa P_{\phi} \textrm{e}^{\mu} = 2 \frac{(1-h)^5}{(1+h)^4} - \frac{4 c^2}{c_l^4}\frac{(1-h)^4}{h^3 (1+h)}. \label{sol21a}
\end{equation}
The energy density is obtained through the definition of the function $h$, by substituting $P_{\phi} = h \rho$ and (\ref{sol14a}) into (\ref{sol21a}) that gives
\begin{equation}
\rho = \frac{2 h^{\frac{4c^2}{c_l^4} -1}}{\kappa (1-h)^{\frac{4c^2}{c_l^4} -3}} \left[\frac{(1-h)}{(1+h)^3} - \frac{2c^2}{c_l^4 h^3}\right] \exp\left[-\frac{2c^2}{c_l^4}\left(\frac{2}{h} + \frac{1}{2h^2}\right)\right]. \label{sol22a}
\end{equation}

\subsection{Components of the four-velocity}

Inserting (\ref{sol5a}) and (\ref{sol9a}) into (\ref{partint9}) gives
\begin{equation}
v^2 = c_l^2 (1-h)^2. \label{sol23a}
\end{equation}
Taking the square root while choosing the plus sign for $v$ yields
\begin{equation}
v = c_l (1-h). \label{sol24a}
\end{equation}
Now, (\ref{sol5a}), (\ref{sol18a}) and (\ref{sol24a}) are inserted into (\ref{partint8}) so as to obtain
\begin{equation}
\Omega = -c_l (1-h)  \left[c_k + \frac{2c}{c_l^4} \left(\frac{2}{h} + \frac{1}{2h^2} + \ln \frac{(1-h)^2}{h^2} \right) \right]. \label{sol25a}
\end{equation}
Therefore, for $h \neq const.$, $\Omega$ depends on $r$ through (\ref{sol11a}), and the rotation is actually differential.

\subsection{Physical properties and final form of the solution}

\subsubsection{Hydrodynamical tensors, vectors and scalars.} \label{hydra}

The general expressions displayed in section \ref{genhydro} are now specialised to the present case. The quantity $\Psi$, as given by the second equality in (\ref{hydr2}), becomes, with (\ref{sol5a}) and (\ref{sol9a}) inserted together with their derivatives, 
\begin{equation}
\Psi = - \frac{(1-h)^2}{1+h}. \label{psia}
\end{equation}
The nonzero component of the acceleration vector is obtained by implementing the first equality in (\ref{hydr2}), which yields
\begin{equation}
\dot{V_1} = \frac{(1-h)^2}{1+h}. \label{V1a}
\end{equation}
Its modulus follows as
\begin{equation}
\dot{V^{\alpha}}\dot{V_{\alpha}} = \frac{h^{\frac{4c^2}{c_l^4}}}{(1-h)^{\frac{4c^2}{c_l^4} -3}(1+h)}\exp\left[-\frac{2c^2}{c_l^4}\left(\frac{2}{h} + \frac{1}{2h^2}\right)\right]. \label{VVa}
\end{equation}

As assumed, the rotation tensor vanishes.

The shear tensor exhibits two nonzero components which can be calculated using (\ref{hydr9a}) and (\ref{hydr10}). We obtain thus
\begin{equation}
2\sigma_{01}= \frac{2c}{c_l} \frac{(1-h)}{h}\left[c_k + \frac{2c}{c_l^4} \left(\frac{2}{h} + \frac{1}{2h^2} + \ln \frac{(1-h)^2}{h^2} \right) \right], \label{shear1}
\end{equation}
\begin{equation}
2\sigma_{13}= \frac{2c}{c_l} \frac{(1-h)}{h}. \label{shear2}
\end{equation}
The shear scalar proceeds from (\ref{hydr12}) and reads
\begin{equation}
\sigma^2= \frac{ c^2 h^{4\left(\frac{c^2}{c_l^4} -1\right)}(1+h)}{c_l^4 (1-h)^{\frac{4c^2}{c_l^4} -3}} \exp\left[-\frac{2c^2}{c_l^4}\left(\frac{2}{h} + \frac{1}{2h^2}\right)\right] . \label{shear3}
\end{equation}

\subsubsection{Behaviour of the $h(r)$ function and sign constraints} \label{signconsa}

It is easy to see, from the expression of $h(r)$ given by (\ref{sol11a}), that $h(r) >0$ whatever the sign of $\epsilon$. 

The first derivative of $h(r)$, as given by (\ref{sol12a}) with $c_D=1$, is $1$ on the axis where $h=h_0=0$ and vanishes for $h=+1$ which is the upper limit of the definition interval of $h$. It is therefore positive in the full interval. Hence, the function $h$ keeps growing from $h_0=0$ towards its maximum. This implies that the ratio $h$ of the pressure over the energy density is everywhere positive in such spacetimes where the interval of definition of $h$ is reduced to
\begin{equation}
0<h<+1. \label{sol22b}
\end{equation}
With this constraint accounted for, the weak energy condition, $\rho>0$, imposes
\begin{equation}
2 \frac{c^2}{c_l^4}<\frac{h^3 (1-h)}{(1+h)^3}. \label{sol22bb}
\end{equation}
A straightforward mathematical analysis of the expression at the right hand side of (\ref{sol22bb}) shows that its first derivative with respect to $h$ vanishes for two numerical values located outside the interval of definition (\ref{sol22b}). This derivative keeps thus the same positive sign at all points of the interior spacetimes. The corresponding expression is therefore monotonically increasing from the axis, where it vanishes, towards the boundary. Thus, for any couple of parameters $\{c,c_l\}$, the weak energy condition is verified at all point characterized by a ratio $h$ such that $h_1<h<h_{\Sigma}<1$, with $h_1$ defined by
\begin{equation}
2 \frac{c^2}{c_l^4}=\frac{h_1^3 (1-h_1)}{(1+h_1)^3}. \label{sol22c}
\end{equation}
The larger $c_l$ with respect to $c$, the closer from the axis this condition is satisfied.

Then, considering the first derivative of the function $h(r)$ given by (\ref{sol11a}), written as
\begin{equation}
h' = -\frac{1}{2r^2}\left(1 + \epsilon\frac{1 + 2r}{\sqrt{1+4r}} \right), \label{sol22aa}
\end{equation}
two cases can be distinguished. The first one, when $\epsilon=+1$ that implies $h'<0$, is ruled out by the above discussion. The second one, when $\epsilon=-1$, implies $h'>0$ provided that
\begin{equation}
\sqrt{1+4r}< 1+2r, \label{sol23b}
\end{equation}
which is always satisfied since the coordinate $r$ is positive definite. Any well-behaved solution implies, therefore, $\epsilon = -1$. Recall that (\ref{sol11a}) and (\ref{sol22aa}) apply only outside the axis, i.e., for $r \neq 0$, since they are obtained after dividing by $r$ during the calculations.

\subsection{Metric signature}

To preserve the Lorentzian signature of the metric, the metric functions are compelled to obey the constraints first displayed by Davidson \cite{D96}, then simplified by C\'el\'erier \cite{C23e,C23f}, and implying that $f$ and $\textrm{e}^\mu$ must have the same sign.

For $h$ fulfilling constraint (\ref{sol22b}), $\textrm{e}^\mu$ given by (\ref{sol14a}), with $c_{\mu} = 1$, exhibits a positive sign. Hence $f$ should be positive definite.

Now, owing to (\ref{sol19a}), a positive metric function $f$ is obtained if and only if
\begin{equation}
1 > c_k c_l^2h + \frac{2c}{c_l^2} \left[2 + \frac{1}{2h} +h \ln \frac{(1-h)^2}{h^2}\right]. \label{sol25b}
\end{equation}
At both extrema of the definition interval, i. e., on the axis where $h=0$ and for $h=+1$, the term inside the brackets diverges. Thus, this inequality can be fulfilled only if $c<0$. Hence, $c$ will be denoted $-c^2$ in the following. Moreover, either $c_k$ is negative and the inequality is verified without any other requirement, or $c_k$ is positive and (\ref{sol25b}) constitutes a constraint on the three parameters $c^2$, $c_l^2$ and $c_k$, depending on the value of $h_{\Sigma}$ on the boundary, since the maximum $h=1$ is not reached inside the cylinder of fluid, as shown in Sec.\ref{signconsa}.

\subsection{Final form of the solution} \label{final}

Implementing the rescalings, the renamings, and the constraints on the parameters described above leads to the final form of the solution, which can be summarized as
\begin{equation}
l = c_l^2 \frac{h^2}{(1-h)^2}, \label{fin1}
\end{equation}
\begin{equation}
\textrm{e}^{\mu} = \frac{(1-h)^{\frac{4c^4}{c_l^4} +1}}{h^{\frac{4c^4}{c_l^4}}(1+h)} \exp\left[\frac{2c^4}{c_l^4}\left(\frac{2}{h} + \frac{1}{2h^2}\right)\right], \label{fin2}
\end{equation}
\begin{equation}
k = c_l^2 \frac{h^2}{(1-h)^2} \left[c_k - \frac{2c^2}{c_l^4} \left(\frac{2}{h} + \frac{1}{2h^2}+ \ln \frac{(1-h)^2}{h^2} \right) \right], \label{fin3}
\end{equation}
\begin{equation}
f = \frac{1}{c_l^2 (1-h)^2} - c_l^2 \frac{h^2}{(1-h)^2}\left[c_k  - \frac{2c^2}{c_l^4} \left(\frac{2}{h} + \frac{1}{2h^2} + \ln \frac{(1-h)^2}{h^2} \right) \right]^2,  \label{fin4}
\end{equation}
\begin{equation}
D = \frac{h}{(1-h)^2} = r, \label{fin5}
\end{equation}
\begin{equation}
h = 1 +\frac{1}{2r} - \sqrt{\frac{1}{r} + \frac{1}{4r^2}}, \label{fin6}
\end{equation}
\begin{equation}
h' = \frac{(1-h)^3}{(1+h)}= -\frac{1}{2r^2}\left(1 - \frac{1 + 2r}{\sqrt{1+4r}} \right), \label{fin7}
\end{equation}
\begin{equation}
\rho =\frac{2 h^{\frac{4c^4}{c_l^4} -1}}{\kappa (1-h)^{\frac{4c^4}{c_l^4} -3}} \left[\frac{(1-h)}{(1+h)^3} - \frac{2c^4}{c_l^4 h^3}\right] \exp\left[-\frac{2c^4}{c_l^4}\left(\frac{2}{h} + \frac{1}{2h^2}\right)\right], \label{fin8}
\end{equation}
\begin{equation}
P_{\phi} =\frac{2 h^{\frac{4c^4}{c_l^4}}}{\kappa (1-h)^{\frac{4c^4}{c_l^4} -3}} \left[\frac{(1-h)}{(1+h)^3} - \frac{2c^4}{c_l^4 h^3}\right] \exp\left[-\frac{2c^4}{c_l^4}\left(\frac{2}{h} + \frac{1}{2h^2}\right)\right], \label{fin8a}
\end{equation}
\begin{equation}
v = c_l (1-h), \label{fin9}
\end{equation}
\begin{equation}
\Omega = -c_l (1-h)  \left[c_k - \frac{2c^2}{c_l^4} \left(\frac{2}{h} + \frac{1}{2h^2} + \ln \frac{(1-h)^2}{h^2} \right) \right], \label{pfin10}
\end{equation}
\begin{equation}
\dot{V^{\alpha}}\dot{V_{\alpha}} = \frac{h^{\frac{4c^4}{c_l^4}}}{(1-h)^{\frac{4c^4}{c_l^4} -3}(1+h)}\exp\left[-\frac{2c^4}{c_l^4}\left(\frac{2}{h} + \frac{1}{2h^2}\right)\right], \label{fin11}
\end{equation}
\begin{equation}
\omega =0, \label{fin12}
\end{equation}
\begin{equation}
\sigma^2= \frac{c^4}{c_l^4} \frac{h^{4\left(\frac{c^4}{c_l^4} -1\right)}(1+h)}{(1-h)^{\frac{4c^4}{c_l^4} -3}} \exp\left[-\frac{2c^4}{c_l^4}\left(\frac{2}{h} + \frac{1}{2h^2}\right)\right]. \label{fin13}
\end{equation}

\subsection{Singularities}

The solutions displayed here seem, at first sight, to involve two possible singular loci.

One occurs for $h=+1$ where the metric functions diverge. The radial coordinate $r$, given by (\ref{fin5}), diverges also for this value of $h$. However, as pointed out in section \ref{signconsa}, since $r$ is bounded by the radius of the cylinder $r_{\Sigma}$, $h=+1$, is never reached inside the fluid and does not affect the interior spacetimes under consideration.

Another singularity seems to occur for $h=r=0$, i.e., on the axis. For such a value of $h$, the metric functions $k$, $\textrm{e}^{\mu}$ and $l$ vanish and the density diverges. The axis could be thus regarded as a physical singularity. However, this alleged singularity is a mere mathematical decoy, since it emerges from repeated divisions by $h$  during the calculations. One can convince oneself of this by examining, for instance, (\ref{fin6}), where $h$ diverges for $r=0$, instead of vanishing.

\subsection{Interpretation of the parameters} \label{parametersa}

The mathematical and physical properties of these differentially rotating spacetimes with purely azimuthal pressure depend on three parameters, the integration constants $c^2$, $c_l^2$, and $c_k$.

At variance with the parameters of the formerly displayed rigidly rotating solutions \cite{C21,C23a,C23b,C23c}, their individual interpretation is not straightforward since no ``regularity'' condition, nor the ``rotation scalar theorem'' \cite{C23a} neither, can be used here and most of the physical quantities diverge or vanish on the axis whatever the values of the parameters.

However, it has been shown in section \ref{signconsa} that the ratio $2c^4/c_l^4$ drives the weak energy condition in the vicinity of the axis. The smaller this ratio, the closer to the axis this energy condition is fulfilled.

A deeper analysis of the properties of these solutions will have to be made in the future to obtain a detailed interpretation of these parameters.

\section{Solving the field equations for $h=const.$ Class B} \label{solvingc}

We consider now the case where the ratio $h$ of the equation of state of the fluid is a constant, since this is known to have a number of physical applications in astrophysics.

The beginning of the calculations, which is independent of $h$, is the same as in Sec.\ref{solvingb} and yields similarly (\ref{sol1a})-(\ref{sol4a}).

Then, we insert (\ref{sol4a}) into the Bianchi identity (\ref{Bianchi4a}) and obtain
\begin{equation}
\frac{l'}{l} = \frac{2}{(1+h)(r+c_2)}, \label{sol1b}
\end{equation}
which can be integrated by
\begin{equation}
l = c_l (r+ c_2)^{\frac{2}{1+h}}, \label{sol2b}
\end{equation}
where $c_l$ is an integration constant.

The axisymmetry condition $l\stackrel{0}{=}0$ is satisfied provided that $c_2 = 0$, which implies
\begin{equation}
l = c_l r^{\frac{2}{1+h}}, \label{sol3b}
\end{equation}
and
\begin{equation}
D = r. \label{sol4b}
\end{equation}

By implementing (\ref{sol3b}) and (\ref{sol4b}) into (\ref{partint12}), we obtain
\begin{equation}
k = c_l r^{\frac{2}{1+h}} \left[c_k + \frac{c(1+h)}{c_l^2(1-h)} r^{\frac{2(h-1)}{h+1}} \right], \label{sol5b}
\end{equation}
where $c_k$ is another integration constant. Then, the metric function $f$ follows from inserting (\ref{sol3b})-(\ref{sol5b}) into (\ref{D2}), which yields
\begin{equation}
f = \frac{r^{\frac{2h}{1+h}}}{c_l} - c_l r^{\frac{2}{1+h}} \left[c_k + \frac{c(1+h)}{c_l^2(1-h)} r^{\frac{2(h-1)}{h+1}} \right]^2. \label{sol6b}
\end{equation}

Now, we insert the derivatives of the above metric functions, together with $D$ and $D'$, into (\ref{G11a}) and obtain
\begin{equation}
\mu' = - \frac{2h}{(1+h)^2 r} - \frac{2 c^2}{c_l^2} r^{\frac{h-3}{h+1}}, \label{sol7b}
\end{equation}

Here, we see that we have two different classes of solutions to (\ref{sol7b}), depending on the value taken by $h$.

\subsection{Subclass i: $h \neq 1$} \label{classA}

In the case where $h \neq 1$, which implies $(h-3)/(h+1) \neq -1$, $\mu'$ can be integrated such as to give
\begin{equation}
\textrm{e}^{\mu} =  \frac{1}{r^{\frac{2h}{(1+h)^2}}} \exp\left[ \frac{ c^2(1+h)}{c_l^2(1-h)} r^{\frac{2(h-1)}{h+1}} \right], \label{sol8b}
\end{equation}
after having rescaled the $r$ and $z$ coordinates, so as to get rid of the integration constant.

\subsubsection{Energy density}

Now that we have the whole metric, as well as $D$, as explicit functions of $r$, it is easy to derive the pressure $P_{\phi}$ by inserting the terms of interest into (\ref{G33a}). This gives
\begin{equation}
P_{\phi} = - \frac{4 c^2 h}{\kappa c_l^2 (1+h) r^{\frac{2(2+h)}{(1+h)^2}}} \exp\left[ - \frac{ c^2(1+h)}{c_l^2(1-h) r^{\frac{2(1-h)}{1+h}}} \right]. \label{sol9b}
\end{equation}
Then, the energy density follows as $\rho = P_{\phi}/h$, which yields
\begin{equation}
\rho = - \frac{4 c^2}{\kappa c_l^2 (1+h) r^{\frac{2(2+h)}{(1+h)^2}}} \exp\left[ - \frac{ c^2(1+h)}{c_l^2(1-h) r^{\frac{2(1-h)}{1+h}}} \right]. \label{sol10b}
\end{equation}

Note that the energy density vanishes at the axis where $r=0$, and such does the pressure. Hence, the axis of rotation is devoid of physical drawbacks.

Moreover, we have $\rho >0$ if $h< -1$, while, in this case, the pressure $P_{\phi}$ is negative. However, a negative pressure is encountered in various physical configurations \cite{C23c} and is therefore acceptable here.

\subsubsection{So-called ``regularity'' condition} \label{regi}

By implementing the solutions of the current subclass into (\ref{arc4a}), we obtain the constraint:
\begin{equation}
\frac{\textrm{e}^{-\mu}l'^2}{4l} = \frac{c_l}{(1+h)^2 r^{\frac{2h^2}{(1+h)^2}}} \exp\left[ - \frac{ c^2(1+h)}{c_l^2(1-h) r^{\frac{2(1-h)}{1+h}}} \right] \stackrel{0}{=} 1. \label{reguli}
\end{equation}
Now, this expression diverges or vanishes depending on the value of $h_0$, but can never match unity for $r=0$. Therefore, this subclass of solutions exhibits an angular deficit in the vicinity of the axis.

\subsubsection{Components of the four-velocity}

Now, we substitute (\ref{sol3b}) and (\ref{sol4b}) into (\ref{partint9}) to obtain
\begin{equation}
v^2 = \frac{c_l}{r^{\frac{2h}{1+h}}}. \label{sol11b}
\end{equation}
Taking the square root while choosing the plus sign for $v$ yields
\begin{equation}
v = \frac{\sqrt{c_l}}{r^{\frac{h}{1+h}}}. \label{sol12b}
\end{equation}
Then, the differential angular velocity follows by inserting (\ref{sol3b})-(\ref{sol5b}) into (\ref{partint8}), so as to obtain
\begin{equation}
\Omega = - \frac{\sqrt{c_l}}{r^{\frac{h}{1+h}}} \left[c_k +  \frac{c(1+h)}{c_l^2(1-h) r^{\frac{2(1-h)}{1+h}}} \right]. \label{sol13b}
\end{equation}
In this case also, $\Omega$ depends on $r$, and the rotation is actually differential.

\subsubsection{Hydrodynamical properties}

The nonzero component of the acceleration vector is obtained by implementing (\ref{hydr2}), which yields
\begin{equation}
\dot{V_1} = \frac{h}{(1+h) r}. \label{V1b}
\end{equation}
Its modulus follows as
\begin{equation}
\dot{V^{\alpha}}\dot{V_{\alpha}} = \frac{h^2}{(1+h)^2} r^{\frac{2h}{(1+h)^2} - 2} \exp\left[-\frac{c^2(1+h)}{c_l^2(1-h) r^{\frac{2(1-h)}{1+h}}}\right]. \label{VVb}
\end{equation}

As assumed, the rotation tensor vanishes.

The shear tensor exhibits two nonzero components which can be calculated using (\ref{hydr9a}) and (\ref{hydr10}). We obtain thus
\begin{equation}
2\sigma_{01}= \frac{2c^2(1+h)}{c_l^{\frac{5}{2}}(1-h)} r^{\frac{2h-3}{1+h}}, \label{shear1b}
\end{equation}
\begin{equation}
2\sigma_{13}= \frac{2c}{\sqrt{c_l} r^{\frac{1}{1+h}}}. \label{shear2b}
\end{equation}
The shear scalar proceeds from (\ref{hydr12}) and reads
\begin{equation}
\sigma^2= \frac{c^2}{c_l^2 r^{\frac{2(2+h)}{(1+h)^2}}} \exp\left[-\frac{c^2(1+h)}{c_l^2(1-h) r^{\frac{2(1-h)}{1+h}}}\right]. \label{shear3b}
\end{equation}

\subsubsection{Singularities}

The metric functions $k$ and $f$ diverge for $h=1$, but not for $h \neq 1$, which is the case here. The solutions belonging to this subclass are, therefore, singularity-free.

\subsubsection{Metric signature}

The metric function $\textrm{e}^{\mu}$, as given by (\ref{sol8b}), is obviously positive. So must be $f$, given by (\ref{sol6b}). A straightforward analysis of this expression shows that $f>0$ implies
\begin{equation}
\frac{1 - 2c c_k \frac{1+h}{1-h} - \sqrt{\left[2 c c_k \frac{1+h}{1-h} - 1\right]^2 - 4 c^2 c_k^2 c_l \frac{(1+h)^2}{(1-h)^2}}}{\frac{2 c^2 (1+h)^2}{c_l (1-h)^2}} < r^{\frac{2(h-1)}{h+1}} < \frac{1 - 2c c_k \frac{1+h}{1-h} + \sqrt{\left[2 c c_k \frac{1+h}{1-h} - 1\right]^2 - 4 c^2 c_k^2 c_l \frac{(1+h)^2}{(1-h)^2}}}{\frac{2 c^2 (1+h)^2}{c_l (1-h)^2}}. \label{sol19b}
\end{equation}

We must consider two cases:

\hfill

Case (a): $\frac{h-1}{h+1} > 0$, which corresponds either to $h>1$ or to $h< -1$. In this case, $r^{\frac{2(h-1)}{h+1}}$ is an increasing function of $r$. For $r=0$, $r^{\frac{2(h-1)}{h+1}} = 0$, which implies therefore
\begin{equation}
1 - 2c c_k \frac{1+h}{1-h} - \sqrt{\left[2 c c_k \frac{1+h}{1-h} - 1\right]^2 - 4 c^2 c_k^2 c_l \frac{(1+h)^2}{(1-h)^2}} = 0, \label{sol20b}
\end{equation}
which we square after having switched the square root to the right-hand side. We obtain
\begin{equation}
\left[1 - 2c c_k \frac{1+h}{1-h}\right]^2 = \left[1 - 2c c_k \frac{1+h}{1-h}\right]^2 - 4 c^2 c_k^2 c_l \frac{(1+h)^2}{(1-h)^2}. \label{sol200b}
\end{equation}
This equality imposes that the last term on the right-hand side vanishes. This can be performed through the vanishing of one of the three parameters occurring there. However, since only the vanishing of $c_k$ does not cause any damage to the solutions, we choose $c_k=0$ to deal with the metric signature, in this case.

\hfill

Case (b): $\frac{h-1}{h+1} < 0$, which corresponds to $-1<h<+1$. Thus, $r^{\frac{2(h-1)}{h+1}}$ is a decreasing function of $r$, which gives, for $r \rightarrow 0$, $r^{\frac{2(h-1)}{h+1}} \rightarrow \infty$. This is a pathological feature which rules out case (b).

\subsection{Subclass ii: $h=1$}

Implementing $h=1$ into (\ref{sol7b}), we obtain
\begin{equation}
\mu' = - \left(\frac{1}{2} + \frac{2 c^2}{c_l^2}\right) \frac{1}{r}, \label{sol14b}
\end{equation}
which can be integrated by
\begin{equation}
\textrm{e}^{\mu} = \frac{c_{\mu}}{r^{\frac{1}{2}+\frac{2 c^2}{c_l^2}}}, \label{sol15b}
\end{equation}
where we set $c_{\mu} = 1$ through a rescaling of the $r$ and $z$ coordinates, which gives
\begin{equation}
\textrm{e}^{\mu} = \frac{1}{r^{\frac{1}{2}+\frac{2 c^2}{c_l^2}}}. \label{sol16b}
\end{equation}

Now, $l$ given by (\ref{sol2b}), becomes, with $h=1$,
\begin{equation}
l = c_l r, \label{sol16ab}
\end{equation}
which, inserted into (\ref{partint12}), together with (\ref{sol4b}), gives
\begin{equation}
k = c_l r \left(c_k - \frac{2c}{c_l^2} \ln r \right). \label{sol16bb}
\end{equation}
Then, the metric function $f$ follows from inserting (\ref{sol4b}), (\ref{sol16ab}), and (\ref{sol16bb}), into (\ref{D2}), which yields
\begin{equation}
f = \frac{r}{c_l} - c_l r \left(c_k - \frac{2c}{c_l^2} \ln r \right)^2. \label{sol6cb}
\end{equation}

Now, we substitute (\ref{sol4b}) and (\ref{sol16ab}) into (\ref{partint9}) to obtain
\begin{equation}
v^2 = \frac{c_l}{r}. \label{sol16db}
\end{equation}
Taking the square root while choosing the plus sign for $v$ yields
\begin{equation}
v = \sqrt{\frac{c_l}{r}}. \label{sol16eb}
\end{equation}
Then, the differential angular velocity follows by inserting (\ref{sol16ab}), (\ref{sol16bb}), and (\ref{sol16eb}) into (\ref{partint8}), so as to obtain
\begin{equation}
\Omega = - \sqrt{\frac{c_l}{r}} \left(c_k - \frac{2c}{c_l^2} \ln r \right). \label{sol13b}
\end{equation}
In this case also, $\Omega$ depends on $r$, and the rotation is actually differential.

\subsubsection{Energy density}

Since we have obtained the whole metric, the pressure can be calculated by inserting the functions of interest and their derivatives into (\ref{G33a}), which yields
\begin{equation}
P_{\phi} = - \frac{2 c^2}{\kappa c_l^2} r^{\frac{2 c^2}{c_l^2} - \frac{3}{2}}. \label{sol17b}
\end{equation}

With $h=1$, the energy density follows readily as
\begin{equation}
\rho = P_{\phi} = - \frac{2 c^2}{\kappa c_l^2} r^{\frac{2 c^2}{c_l^2} - \frac{3}{2}}. \label{sol18b}
\end{equation}
Therefore, this energy density happens to be negative which compromises the physical behaviour of this subclass of solutions. However, since such exotic matter might be considered for some theoretical applications, we will study below the properties of this subclass.

\subsubsection{So-called ``regularity'' condition} \label{regii}

By implementing (\ref{sol15b}) and (\ref{sol16ab}) into (\ref{arc4a}), we obtain the constraint:
\begin{equation}
\frac{\textrm{e}^{-\mu}l'^2}{4l} = \frac{c_l}{4} r^{\frac{2c^2}{c_l^2} - \frac{1}{2}} \stackrel{0}{=} 1, \label{regulii1}
\end{equation}
which can be verified provided that
\begin{equation}
\frac{2c^2}{c_l^2} - \frac{1}{2} = 0  \quad \textrm{and} \quad \frac{c_l}{4}=1, \label{regulii2}
\end{equation}
which implies
\begin{equation}
c = \pm 2 \quad \textrm{and} \quad c_l=4. \label{regulii2}
\end{equation}
Within the subclass ii, only the two corresponding spacetimes are devoid of angular deficit in the vicinity of the axis.

\subsubsection{Hydrodynamical properties}

The nonzero component of the acceleration vector is obtained by implementing (\ref{hydr2}), which yields
\begin{equation}
\dot{V_1} = \frac{1}{2 r}. \label{V1c}
\end{equation}
Its modulus follows as
\begin{equation}
\dot{V^{\alpha}}\dot{V_{\alpha}} = \frac{r^{\frac{2c^2}{c_l^2} - \frac{3}{2}}}{4}. \label{VVc}
\end{equation}

As assumed, the rotation tensor vanishes.

The shear tensor exhibits two nonzero components which can be calculated using (\ref{hydr9a}) and (\ref{hydr10}). We obtain thus
\begin{equation}
2\sigma_{01}= \frac{2c}{\sqrt{c_l r}} \left(c_k - \frac{2c}{c_l^2} \ln r \right), \label{shear1c}
\end{equation}
\begin{equation}
2\sigma_{13}= \frac{2c}{\sqrt{c_l r}}. \label{shear2c}
\end{equation}
The square of the shear scalar proceeds from (\ref{hydr12}) and reads
\begin{equation}
\sigma^2= \frac{c^2}{c_l^2} r^{\frac{2c^2}{c_l^2} - \frac{3}{2}}. \label{shear3c}
\end{equation}

\subsubsection{Singularities}

The metric functions $\textrm{e}^{\mu}$, $k$, and $f$ diverge on the axis where $r=0$. However, the energy density and the pressure are either constant or vanish there when
\begin{equation}
\frac{2c^2}{c_l^2} \geq \frac{3}{2}. \label{singc}
\end{equation}
When this inequality is satisfied, we can infer that the potential axis singularity is merely a coordinate singularity. We will thus impose the constraint (\ref{singc}) on the parameters of this subclass.

\subsubsection{Metric signature}

The  metric function $\textrm{e}^{\mu}$ as given by (\ref{sol16b}) is obviously positive definite. Therefore, to yield a proper metric signature, $f$ must be also positive, which implies
\begin{equation}
\frac{1}{c_l} \geq c_l\left(c_k - \frac{2c}{c_l^2} \ln r \right). \label{signc}
\end{equation}
It is obvious that, in the vicinity of the axis, where $r=0$, the right end side of this inequality diverges towards $+ \infty$ if $c>0$, and towards $- \infty$ if $c<0$. This inequality imposes therefore $c<0$, for the metric signature to be Lorentzian. Hence, $c$ will be denoted $-c^2$ thereafter.

\subsection{Final forms of the solutions}

\subsubsection{Subclass i}

Implementing $c_k=0$, we obtain

\begin{equation}
l = c_l r^{\frac{2}{1+h}}, \label{fin1b}
\end{equation}
\begin{equation}
\textrm{e}^{\mu} = \frac{1}{r^{\frac{2h}{(1+h)^2}}} \exp\left[ \frac{ c^2(1+h)}{c_l^2(1-h)r^{\frac{2(1-h)}{1+h}}}  \right], \label{fin2b}
\end{equation}
\begin{equation}
k = \frac{c(1+h)}{c_l(1-h)} r^{\frac{2h}{1+h}}, \label{fin3b}
\end{equation}
\begin{equation}
f = \frac{r^{\frac{2h}{1+h}}}{c_l} - \frac{c^2(1+h)^2}{c_l^3(1-h)^2} r^{\frac{2(2h-1)}{1+h}}, \label{fin4b}
\end{equation}
\begin{equation}
D = r, \label{fin5b}
\end{equation}
\begin{equation}
\rho = - \frac{4 c^2}{\kappa c_l^2 (1+h) r^{\frac{2(2+h)}{(1+h)^2}}} \exp\left[ - \frac{ c^2(1+h)}{c_l^2(1-h) r^{\frac{2(1-h)}{1+h}}} \right], \label{fin6b}
\end{equation}
\begin{equation}
P_{\phi} = - \frac{4 c^2 h}{\kappa c_l^2 (1+h) r^{\frac{2(2+h)}{(1+h)^2}}} \exp\left[ - \frac{ c^2(1+h)}{c_l^2(1-h) r^{\frac{2(1-h)}{1+h}}} \right]. \label{fin7b}
\end{equation}
\begin{equation}
v = \frac{\sqrt{c_l}}{r^{\frac{h}{1+h}}}, \label{fin8b}
\end{equation}
\begin{equation}
\Omega = - \frac{c(1+h)}{c_l^{\frac{3}{2}}(1-h) r^{\frac{2-h}{1+h}}}, \label{fin9b}
\end{equation}
\begin{equation}
\dot{V^{\alpha}}\dot{V_{\alpha}} = \frac{h^2}{(1+h)^2} r^{\frac{2h}{(1+h)^2} - 2} \exp\left[-\frac{c^2(1+h)}{c_l^2(1-h) r^{\frac{2(1-h)}{1+h}}}\right], \label{fin10b}
\end{equation}
\begin{equation}
\omega = 0, \label{fin11b}
\end{equation}
\begin{equation}
\sigma^2= \frac{c^2}{c_l^2 r^{\frac{2(2+h)}{(1+h)^2}}} \exp\left[-\frac{c^2(1+h)}{c_l^2(1-h) r^{\frac{2(1-h)}{1+h}}}\right]. \label{fin12b}
\end{equation}

\subsubsection{Subclass ii}

\begin{equation}
l = c_l r, \label{fin13b}
\end{equation}
\begin{equation}
\textrm{e}^{\mu} = \frac{1}{r^{\frac{1}{2}+\frac{2 c^4}{c_l^2}}}, \label{fin14b}
\end{equation}
\begin{equation}
k = c_l r \left(c_k + \frac{2c^2}{c_l^2} \ln r \right), \label{fin15b}
\end{equation}
\begin{equation}
f = \frac{r}{c_l} - c_l r \left(c_k + \frac{2c^2}{c_l^2} \ln r \right)^2. \label{fin16b}
\end{equation}
\begin{equation}
D = r, \label{fin17b}
\end{equation}
\begin{equation}
\rho = P_{\phi} = - \frac{2 c^4}{\kappa c_l^2} r^{\frac{2 c^4}{c_l^2} - \frac{3}{2}}, \label{fin18b}
\end{equation}
\begin{equation}
v = \sqrt{\frac{c_l}{r}}, \label{fin19b}
\end{equation}
\begin{equation}
\Omega = - \sqrt{\frac{c_l}{r}} \left(c_k + \frac{2c^2}{c_l^2} \ln r \right), \label{fin20b}
\end{equation}
\begin{equation}
\dot{V^{\alpha}}\dot{V_{\alpha}} = \frac{r^{\frac{2c^4}{c_l^2} - \frac{3}{2}}}{4}, \label{fin21b}
\end{equation}
\begin{equation}
\omega = 0, \label{fin22b}
\end{equation}
\begin{equation}
\sigma^2= \frac{c^4}{c_l^2} r^{\frac{2c^4}{c_l^2} - \frac{3}{2}}. \label{fin23b}
\end{equation}

\section{Comparisons with the rigidly rotating solutions} \label{comp}

The aim of this section is to discuss and compare the main features of the six classes of exact solutions exhibited up to now and describing the interior spacetimes of a cylinder of fluid with azimuthally directed pressure: three for rigidly rotating matter \cite{C23c} and the three differentially rotating ones analyzed in the present article.

\subsection{Axisymmetry and regularity conditions} \label{axreg}

Since these cylindrically symmetric spacetimes, all defined by the same form of the metric (\ref{metric}), are axisymmetric by construction , the axisymmetry condition \cite{S09,vdB85} has been applied to each of them. For the three classes with rigid rotation, the constraints issuing from this condition have proven compatible with the ``no angular deficit condition'' usually applied under the name ``regularity condition'' to the vicinity of the rotation axis of axially symmetric spacetimes and given by (\ref{arc4a}). However, it has been shown here, respectively in Secs.\ref{regcond}, \ref{regi} and \ref{regii}, that this regularity condition is not realized in the differentially rotating spacetimes, save for two particular couples of values attached to the $c$ and $c_l$ parameters of subclass Bii.

Now, this ``regularity'' condition is obtained by a normalisation of the $\phi$ coordinate provided elementary flatness is assumed in the vicinity of the axis. However, it has been shown by Lake and Musgrave \cite{L94} that elementary flatness by no means guarantees regularity. Moreover, it has been claimed by Wilson and Clarke \cite{W96} that, regularity being defined in Cartesian type coordinates, a violation of the ``regularity'' condition might be linked to some issue with the use of polar type coordinates. The adaptation of polar type coordinates to the study of features of the axis has also been discussed by Carot \cite{C00}, while the presence of some matter on the axis has been refered to by Pereira, Santos, and Wang \cite{P96} to deal with the ``nonregularity'' problem.

It is therefore interesting to note that passing from rigid to differential rotation of the gravitational source generally affects the properties exhibited by the spacetimes in the vicinity of the axis. Indeed such a fact might help improving, in the future, our understanding of this issue in axisymmetry.

It can also be stressed that in both cases, rigid and differential rotation, the purely azimuthal pressure vanishes on the axis. This property might be ascribed to the fact that the azimuthal coordinate $\phi$ is undefined there, implying the lack of proper determination of $P_{\phi}$ and, therefore, its vanishing. The validity of this argument is reinforced by its verification in both the rigidly and the differentially rotating cases. 

\subsection{Hydrodynamical properties.} \label{hydro}

It has been known for a while \cite{CS20,D06} that rigid rotation implies a vanishing shear. This had been confirmed by the exhibition and the analysis of three classes of rigidly rotating anisotropic fluids \cite{C21,C23a,C23c,C23d}. As expected, it has been shown here that this is not the case for anisotropic fluids with differential rotation. The shear can either vanish or diverge on the axis depending on the value of the parameters, $c/c_l$ for class A, corresponding to $-c^2/c_l$ for class Bii, and $h$ for class Bi.

Now, the modulus of the acceleration vector, vanishes (class A and Bii) or diverges (class Bi) on the axis in the case of differential rotation, while it is unity in the rigidly rotating case \cite{C23c}. No general rule can therefore be set out for this modulus at this stage.

As regards the property of local rotation of the fluid, owing to the irrotational assumption retained here, the fluid is forced to be twist-free, which has indeed been verified for each of the three differentially rotating classes, while this is not the case for rigid rotation. 

\subsection{Singularities} \label{sing}

In the case of rigid rotation, the three displayed solutions are singularity-free for any values of the parameters, while for differential rotation, this is a property of classes A and Bi. The other class, Bii, is deprived of any singularity provided that the parameters satisfy the constraint $2c^4/c_l^2 \geq 3/2$.

All the classes of spacetimes thus considered are well-behaved from this point of view.

\subsection{The parameters} \label{param}

However, discrepancies occurring between the six classes of spacetimes compared here concern the parameters of the solutions which are the independent integration constants remaining after the satisfaction of the constraints issued from the various mathematical and physical requirements pertaining to each case. These discrepancies arise from the fact that more constraints on the parameters appear when the rigid rotation is involved since the corresponding solutions happen to satisfy both the ``singularity'' condition and the ``rotation scalar theorem'' \cite{C23a}, which is the case of no differentially rotating spacetime.

The rigidly rotating cases depend each, therefore, on a single parameter which has been chosen to be the value $h_0$ of the ratio $h$ on the axis for classes 1 and 2, and any other integration constant for the class with $h=const.$, among which the easiest to interpret, and therefore the best choice, might be $c$, the rotation scalar on the axis.

For differential rotation, class A involves two positive and one with no sign constraint independent parameters. It consists therefore of three-parameter solutions, the parameters being $c^2$, $c_l^2$, and $c_k$. It has been identified in Sec.\ref{parametersa} that the ratio $2 c^4/c_l^4$ runs the fulfillment of the weak energy condition in the vicinity of the axis. The smaller this ratio, the closer to the axis this energy condition is verified. Moreover, the value of this very ratio determines also the vanishing or divergence property of the shear on the axis. It seems therefore to play a key role as regards the physical properties of these solutions in the vicinity of the axis. Class Bi depends on only two parameters, $c$ and $c_l$, since $c_k$ has been led to vanish by the requirement of a Lorentzian metric. Conversely, class Bii is three-parameter dependent, since no additional constraint could be imposed on it.

\section{Conclusions}\label{concl}

The investigations concerning interior spacetimes gravitationally sourced by stationary anisotropic fluids with cylindrical symmetry have been extended here to the differentially rotating irrotational counterparts of three rigidly rotating configurations exhibiting azimuthally directed pressure and previously studied in \cite{C21,C23c}. Three new classes or subclasses of exact solutions to Einstein field equations have thus been displayed and analyzed here.
This follows a study of an analogous issue realized for fluids with axial pressure \cite{C23f}, and will be followed by the analysis of the radial pressure case. 

General properties pertaining to differential rotation have been first recalled. It has been shown that the field equations generated by fluids submitted to such a rotating regime can be integrated provided that they are twist-free, i. e., irrotational. It has also be confirmed that they do not exhibit a vanishing shear, which characterises rigid rotation.

The particular case of nonrigidly rotating fluid with equation of state $P_r=P_z=0$ has, therefore, been considered here. Two classes of solutions have been displayed. For class A, the metric of the interior spacetime has been exhibited as a set of functions of the ratio $h \equiv P_{\phi}/ \rho$. This ratio $h$ has been given as an explicit function of $r$ and the different quantities representing the physical properties of the system have been provided as functions of $h$, easily convertible into functions of $r$ through $h(r)$. Class B includes the solutions where $h=const.$ It splits into two subclasses, class Bi where $h\neq 1$, and class Bii for $h=1$. 

Provided that the weak energy condition is imposed, which is generally the case for standard physical applications, class A alone allows  a positive pressure, while class Bi exhibits a negative pressure. However, a negative pressure is encountered in various physical systems which has allowed us to accept this class as a set of proper solutions. Now, class Bii displays a negative energy density which is less standard in physics. Anyhow, we have provided it for completeness, and as possibly useful for exotic applications. To reinforce the potential of these solutions to be used as tools designed to represent physical objects encountered in the Universe, they have been matched on their cylindrical boundary to an exterior vacuum solution. To account with stationarity, a Lewis-Weyl metric has been chosen to represent the exterior spacetime. A proper matching has thus been implemented which apply perfectly to all the classes of spacetimes considered here.

A comparison of the main properties identified for the six spacetimes with azimuthal pressure already displayed (three with rigid rotation \cite{C23c}, and the three with differential rotation analyzed here) has been provided. It has been stressed that, contrary to the rigidly rotating configurations, the differentially rotating ones do not verify the ``regularity'' condition as generally formulated in the literature \cite{S09,D06}, which we have renamed ``no angular deficit'' condition. Since the application of this condition to the here presented systems is questionable, a discussion of this point has been delivered in the comparison section. The hydrodynamical properties and the singularities have been confronted. It has been pointed out that, owing to the lack of ``regularity'' condition and of applicability of the ``rotation scalar theorem'' \cite{C23a}, the independent parameters are more numerous and less easy to interpret in the differential rotation cases.

Now, these two new classes of solutions to field equations of GR verifying particular equations of state, together with the ones previously displayed by C\'el\'erier \cite{C21,C23a,C23b,C23c,C23d,C23e,C23f}, must be viewed as belonging to a step by step discovery of interior spacetimes equipped with thorough mathematical and physical analyses and matching to a robust vacuum exterior. Their examination would therefore benefit of the reading of the other papers in the series.

\begin{thebibliography}{10}

\bibitem {C21} M.-N. C\'el\'erier, ``New classes of exact interior nonvacuum solutions to the GR field equations for spacetimes sourced by a rigidly rotating stationary cylindrical anisotropic fluid,'' Phys. Rev. D {\bf 104,} 064040 (2021).
\bibitem {C23a} M.-N. C\'el\'erier, ``Study of stationary rigidly rotating anisotropic cylindrical fluids with new exact interior solutions of GR. II. More about axial pressure,'' J. Math. Phys. {\bf 64,} 032501 (2023).
\bibitem {C23b} M.-N. C\'el\'erier, ``Fully integrated interior solutions of GR for stationary rigidly rotating cylindrical perfect fluids,'' J. Math. Phys. {\bf 64,} 022501 (2023).
\bibitem {C23c} M.-N. C\'el\'erier, ``Study of stationary rigidly rotating anisotropic cylindrical fluids with new exact solutions of GR. III. Azimuthal pressure,'' J. Math. Phys. {\bf 64,} 042501 (2023).
\bibitem{C23d} M.-N. C\'el\'erier, ``Study of stationary rigidly rotating anisotropic cylindrical fluids with new exact interior solutions of GR. IV. Radial pressure,'' J. Math. Phys. {\bf 64,} 052502 (2023).
\bibitem {C23e} M.-N. C\'el\'erier, ``Interior spacetimes sourced by stationary differentially rotating irrotational cylindrical fluids. Perfect fluids,'' arXiv:2305.11565 [gr-qc].
\bibitem {C23f} M.-N. C\'el\'erier, ``Interior spacetimes sourced by stationary differentially rotating irrotational cylindrical fluids. II. Axial pressure,'' arXiv:2307.07263 [gr-qc].
\bibitem {CS20} M.-N. C\'el\'erier and N. O. Santos, ``Stationary cylindrical anisotropic fluid and new purely magnetic GR solutions,'' Phys. Rev. D {\bf 102,} 044026 (2020).
\bibitem {D06} F. Debbasch, L. Herrera, P. R. C. T. Pereira, and N. O. Santos, ''Stationary cylindrical anisotropic fluid,'' Gen. Relativ. Gravitation {\bf 38,} 1825 (2006).
\bibitem {L32} T. Lewis, ``Some special solutions of the equations of axially symmetric gravitational fields,'' Proc. R. Soc. London, Ser. A {\bf 136,} 176 (1932).
\bibitem {W19} H. Weyl, ``Bemerkung uber die axisymmetrischen Losungen der einsteinschen gravitationsgleichungen,'' Ann. Physik {\bf 59,} (364), 185 (1919).
\bibitem {D27} G. Darmois, ``Les \'equations de la gravitation einsteinienne,'' in {\it M\'emorial des Sciences Mathématiques Fascicule XXV} (Gauthier-Villars, Paris, 1927).
\bibitem {D96} W. Davidson, ``A Petrov type I cylindrically symmetric solution for perfect fluid in steady rigid body rotation,''  Classical Quantum Gravity {\bf 13,} 283 (1996).
\bibitem {S09} H. Stephani, D. Kramer, M. MacCallum, C. Honselaers, and E. Herlt, {\it Exact Solutions to Einstein's Field Equations,} Cambridge Monographs on Mathematical Physics (Cambridge University Press, Cambridge, England, 2009).
\bibitem {vdB85} N. Van den Bergh and P. Wils, ``The rotation axis for stationary and axisymmetric space-times,'' Classical Quantum Gravity {\bf 2,} 229 (1985).
 \bibitem {L94} K. Lake and P. Musgrave, ``The Regularity of Static Spherically Cylindrically and Plane Symmetric Spacetimes at the Origin,'' Gen. Relativ. Gravitation {\bf 26,} 917 (1994).
\bibitem {W96} J. P. Wilson and C. J. S. Clarke, ``'Elementary flatness' on a symmetry axis,'' Classical Quantum Gravity {\bf 13,} 2007 (1996).
\bibitem {C00} J. Carot, ``Some developments on axial symmetry,'' Classical Quantum Gravity {\bf 17,} 2675 (2000).
\bibitem {P96} P. R. C. T. Pereira, N. O. Santos, and A. Z. Wang, ``Geodesic motion and confinement in Lanczos spacetimes,'' Classical Quantum Gravity {\bf 13,} 1641 (1996).

\end{thebibliography}


\end{document}



