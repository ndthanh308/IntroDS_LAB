\subsection{\aclp{PMT}}
For \ac{OSIRIS}, 76 dynode photomultipliers of the type R15343 from Hamamatsu Photonics were prepared. \remove{Due to unforeseen problems with the {HV} unit, they need to be repaired and will be installed in a future upgrade of {OSIRIS} }.\R{A90}

The PMTs are comparable to the Hamamatsu R12860HQ regarding the dimensions and performance.
%
% Figure environment removed
%
\ac{OSIRIS}'s \acp{PMT}  can be separated into two groups, the \acp{PMT} looking inwards towards the acrylic vessel, and those, which monitor the water pool and serve as a muon veto.
%
%\subsection{Selection of the Veto PMTs}
% SELECT * FROM `PMT` WHERE `TTS` >= 2.8 ORDER BY `TTS` DESC  -> 8 PMTs
% SELECT * FROM `PMT` WHERE PMT_ID Not In (SELECT PMT_ID FROM `PMT` WHERE `TTS` >= 2.8) ORDER BY `DarkCountRate` DESC LIMIT 4  -> additional 4
%
12~\acp{PMT} are selected for the muon veto. The selection is based on the data given by Hamamatsu (cf. Figure~\ref{fig:pmt_dist}). There are eight \acp{PMT} with \ac{TTS} of \qty{2.8}{\nano\second} or more. These \acp{PMT} are selected for the veto. The remaining four \acp{PMT} are the ones, which have the highest dark count rate.
