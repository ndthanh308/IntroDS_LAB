\subsubsection*{VULCAN}
\emph{VULCAN}~\cite{MuralidharanPhd, PhDParkalian} is a dedicated \ac{PMT} frontend and digitization chip developed by the ZEA-2 of the Forschungszentrum Jülich. 
%This \ac{ASIC} is based on the \qty{65}{\nano\metre} \ac{CMOS} process.
It can be configured via the \ac{JTAG} interface. VULCAN hosts a dedicated \ac{PLL}, which reduces the phase noise \add{of the internal clocks}~\cite{PhDParkalian}\R{A123}. \replace{Although VULCAN is designed to run at a speed of {1}{GSps}, in {OSIRIS} it is used at half the speed.}{Inside \ac{OSIRIS} VULCAN is used with a sampling rate of \qty{500}{\mega\Sps}.} \remove{This reduces the noise of the digitiser.}
%{By this the optimal configuration regarding sampling spacing and input noise is used.}

VULCAN features three identical receivers. They are configured to cover three ranges: \ac{HG}, \ac{MG} and \ac{LG}. Using these three ranges, VULCAN provides a dynamic range up to \qty{1000}{\photoelectrons} while maintaining a high resolution at lower charges. The \ac{HG} and \ac{MG} channel are using a \ac{TIA} -- with the \ac{HG} channel is configured with a larger amplification. The \ac{LG} channel measures the voltage across the \ac{TIA} input.

Each receiver channel serves \remove{four {6}{bit} {ADC}, which are concatenated and form a single} \add{one} \qty{8}{\bit} \ac{ADC}. All three channels are digitised in parallel and the \replace{processor}{data processing logic}\R{A132} inside VULCAN selects the unsaturated samples with the highest resolution. These samples are forwarded to the \ac{FPGA} via a \ac{LVDS} connection. %(cf. to~\cite{MuralidharanPhd} for more details on VULCAN).
