\section{Slowcontrol of the \acs{iPMT} system}
\label{sec:slow_control}
In order to control all different parts of the \ac{iPMT}, an implementation based on \ac{EPICS}~\cite{EPICS} version {3.14.12.7} has been chosen.

For each subsystem, like \ac{I2C} or \ac{HV}, a dedicated \ac{IOC} handles the communication to the \ac{iPMT} using a custom implementation. Most of the endpoints are handled by the \ac{SCCU}. 
The conversion of the raw data into physical values is done by \ac{EPICS} conversion settings. 
The whole \ac{iPMT} can be monitored and controlled via \ac{EPICS}. It is possible to use a graphical interface as well as scripts to interact with the \acp{iPMT}.

The initial values, limits, and alarm settings of each record are derived from a \emph{MySQL} database. Each \ac{iPMT} can be configured individually. Most of the limits require tuning during the commissioning of \ac{OSIRIS}. Hence there is a default -- fail safe -- configuration in the database, which is used in case that no dedicated values are set for an \ac{iPMT}.