% commands for the measured / calculated values to have one place to change them

% CUT parameters
\newcommand{\CUTbaselineN}{18.55}
\newcommand{\CUTbaselineP}{20.70}

% BASELINE parameters
\newcommand{\baselineAVGstart}{54}  % Samples
\newcommand{\baselineAVGend}{84}    % Samples

% integration region
\newcommand{\IntegrationStart}{92}  % Samples
\newcommand{\IntegrationEnd}{122}   % Samples

% AMPLITUDE characteristics
% TreeViewer -> define an expression: merged_wv&0x3FF and a cut 54<=Iteration$<84
% from file ipmt-paper-materials/raw_data/20230210/charge_spectrum_script_37ns_1_566V_dly74_1_5M_bl_hg_20/20230210_090747_hg_0_merged.root
\newcommand{\amplBASELINEmean}{19.59}
\newcommand{\amplBASELINEsigma}{2.407}

\newcommand{\amplCalcSPEheight}{\fpeval{\amplSPEmean - \amplBASELINEmean}}
\newcommand{\amplCalcSPEheightSigma}{\fpeval{sqrt(\amplSPEsigma * \amplSPEsigma - \amplBASELINEsigma * \amplBASELINEsigma)}}

% calculations for getting the error on the NPE in HG
\newcommand{\NPEinHG}{\fpeval{(255 - \amplBASELINEmean) / (\amplSPEmean - \amplBASELINEmean)}}
\newcommand{\NPEinHGsigma}{ \fpeval{ sqrt(\termC + \termF)} }

% forward the error on the amplitude
\newcommand{\termA}{\fpeval{\amplSPEmean - \amplBASELINEmean} }
\newcommand{\termB}{\fpeval{ (255 - \amplBASELINEmean) / (\termA * \termA)}}
\newcommand{\termC}{\fpeval{ \termB * \termB * \amplSPEsigma * \amplSPEsigma }}
% forward the error on the baseline
\newcommand{\termD}{\fpeval{ \amplBASELINEmean - \amplSPEmean}}
\newcommand{\termE}{\fpeval{ (\amplSPEmean - 255) / ( \termD * \termD ) }}
\newcommand{\termF}{\fpeval{ \termE * \termE * \amplBASELINEsigma * \amplBASELINEsigma}}

% from figure 9
\newcommand{\amplSPEmean}{51.8}
\newcommand{\amplSPEsigma}{6.0}

% CHARGE characteristics
\newcommand{\sigmaNoise}{28.05} % width of the noise peak
\newcommand{\sigmaNoiseUncert}{0.03} % uncertainty on the noise peak from fit

\newcommand{\sigmaSPE}{116.6}   % width of the SPE peak
\newcommand{\sigmaSPEUncert}{1.3} % uncertainty on the SPE peak from fit

\newcommand{\gainSPE}{415.2}    % gain of the PMT
\newcommand{\gainSPEUncert}{1.2} % uncertainty on the gain from fit

\newcommand{\PeakToValley}{3.55}% peak to valley

\newcommand{\chargeSNR}{\fpeval{\gainSPE / \sigmaNoise }}
\newcommand{\chargeSNRUncert}{\fpeval{\chargeSNR * sqrt((\gainSPEUncert/\gainSPE)^2 + (\sigmaNoiseUncert / \sigmaNoise)^2)}}

\newcommand{\chargeNoiseWidth}{\fpeval{ \sigmaNoise / \gainSPE}}

\newcommand{\chargeResolution}{\fpeval{100*\sigmaSPE / \gainSPE} }
\newcommand{\chargeResolutionUncert}{\fpeval{\chargeResolution * sqrt( (\sigmaSPEUncert/\sigmaSPE)^2 + (\gainSPEUncert/\gainSPE)^2)}}