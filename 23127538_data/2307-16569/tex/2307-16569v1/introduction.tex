\section{Introduction}
The \ac{JUNO} experiment is a dedicated neutrino detector currently under construction in southern China~\cite{JUNO_YellowBook}. The detection mechanism is based on neutrino interactions with \ac{LS}. A highly radiopure \ac{LS} is of foremost importance for the success of \ac{JUNO}~\cite{JUNO_radioActivity2021}.
%\todo{Remove the very similar sentence from OSIRIS section?}
%Therefore, the dedicated \ac{OSIRIS} (Section~\ref{sec:osiris}) is build to monitor the internal radiopurity of the scintillator during filling of \ac{JUNO}.
Therefore, the dedicated \ac{OSIRIS} experiment monitors the radiopurity of the scintillator during filling of \ac{JUNO}.

\ac{OSIRIS} uses a novel \ac{PMT} integration scheme -- called \ac{iPMT} -- to detect light from radioactive decays.
The concept, realisation and performance of the \ac{iPMT} system is presented in this paper.

\R{A322_fix}Due to unforeseen problems and the necessary repairs of the \acp{iPMT}, they are not installed in the first phase of \ac{OSIRIS}. They will be used in the future upgrade of \ac{OSIRIS} focusing on the search for solar $pp$ neutrinos \cite{SERAPPISconcept}.
%The concept and the realisation of a single \ac{iPMT} is explained in Section~\ref{sec:concept}.
%In order to combine multiple \acp{iPMT} to one detector, a synchronization as explained in Section~\ref{sec:surface_board} is required.
%For detector control and monitoring an \ac{EPICS} based slow control is implemented and described (Section~\ref{sec:slow_control}).
%\todo{Would it make sense to integrate Potting and Holder into the iPMT concept section?}
%The mechanical integration and electromagnetic shielding concept is explained in Section~\ref{sec:potting_holder}.
%Section~\ref{sec:performance} reports the performance of a single \ac{iPMT}.