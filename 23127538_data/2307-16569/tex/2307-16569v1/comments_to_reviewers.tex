\newpage
\section*{Answers to review comments}
\textbf{Comments from a A. Triossi}
{\parindent0pt

L.8: "it is not evident to me why it should be valid expecially for neutrino experiments"\\
\textbf{Answer:} We reformulated this sentence to keep the focus on neutrino experiments, which have a huge volume but only few sensors. \lr{A8} You are absolutely right, the cable length for other detectors increase as well, but for example for CMS at LHC, the signals are digitised as early as possible and sent to the DAQ via fiber. 

\vspace{1em}

L. 68: afford -> effort\\
\textbf{Answer:} Fixed

\vspace{1em}

L. 73: "the digital signal degradates as well as the analog, but the analog shape is not anymore carrying the information. Nevertheless a degradation of the rising/falling edge (slope) of the signal limits bandwidth and transmission rate."\\
\textbf{Answer:} This is why we have included, that the length of the cable is limited as well. 

\vspace{1em}

L. 80: "please specify the mechanism for discrimination: leading edge, constant fraction..."\\
\textbf{Answer:}
added "by e.g. a simple threshold discriminator or more advanced pulse detection techniques" \lr{A80}. It is not necessarily a simple discriminator since more advanced pulse detection circuits can be implemented in the FPGA as well.

\vspace{1em}

L. 83: "What is "concept"? Electronics, detector, block diagram, etc..."  \\
\textbf{Answer:} added "readout" to specify which concept was reused.\lr{A83}

\vspace{1em}

L. 83: "It is not explained why this detector electronics it is not used in the central detector and it is not clear to me what do you mean with "iteration". I would expand this sentence giving more explanations or remove it at all. "\\
\textbf{Answer:} I'm not sure, whether there is a citeable document, giving the explanation, why the BX concept has not made it into JUNO. \\
We would like to mention it here, because the readout concept has not been developed by us, but we have iterated it once again in order to tune / adjust some parameters of the hardware design to our needs and use VULCAN as PMT frontend + digitizer.

\vspace{1em}

L. 90: "It is not clear if the R15343 is used in OSIRIS or not. Did the problem on the voltage unit force to change PMT?"\\
\textbf{Answer:} Exactly, we were forced to change the PMTs from iPMTs to LPMTs for OSIRIS. We removed the sentence from \lr{A90} and \lr{A322} and inserted it into the introduction \lr{A322_fix}.

\vspace{1em}

L. 123: "Of what?" -> added "of the internal clocks" \lr{A123} 

\vspace{1em}

L. 124: "It would be nice to write here why 500Mbps is enough for OSIRIS"\\
\textbf{Answer:} We removed the part that explains that VULCAN can also run with 1GSps since this information is not relevant for the paper.

\vspace{1em}

L. 131: "It is not clear to me what you mean with concatenated"\\
\textbf{Answer:} We removed this part of the sentence. This avoids confusion of the reader and since this part of information is not used any more in the paper, we can remove it without loosing information.

\vspace{1em}

L. 132: "maybe the logic? it should not be a real processor" \lr{A132}\\
\textbf{Answer:} Inside VULCAN it is called data processor \cite[p. 45]{MuralidharanPhd}. We clarified this by using the term "data processing logic".

\vspace{1em}

L. 139: "a" -> included in \lr{A139}

\vspace{1em}

L. 142: "the SCCU itself" -> included in \lr{A142}

\vspace{1em}

L. 145: "It is a modification of UDP or it is a custom protocol over UDP?"\\
\textbf{Answer:} Good point, we clarified this in the text \lr{A145}

\vspace{1em}

L. 165: "You could add that this syncronous link will be presented later in the paper"\\
\textbf{Answer:} The synchronous link has been already mentioned in the ROB section.

\vspace{1em}

L. 167: "a" -> fixed

\vspace{1em}

L. 204: "separated"\\
\textbf{Answer:} We reformulated this to "passively separated" \lr{A204}. For us it is important to emphasize that the splitting / separation is done in a passive way.

\vspace{1em}

L. 210: "...by injecting a pulse on the dowlink and measuring the time of flight when loopback to the uplink in the iPMT?"\\
\textbf{Answer:} We added this explanation how to measure the cable length.\lr{A210}

\vspace{1em}

L. 211: "integrity?" -> replaced \lr{A211}

\vspace{1em}

L. 212: remove "of" -> done

\vspace{1em}

L. 212: remove "the" -> done

\vspace{1em}

L. 213: "one" -> replaced "a" by "one"

\vspace{1em}

L. 216: "periodic trigger" -> added periodic

\vspace{1em}

L. 216: "Rate" -> replaced

\vspace{1em}

L. 258: "A" \\
\textbf{Answer:} "An LED" should be correct since "LED" starts with a vowel sound

\vspace{1em}

L. 261: "no comma" -> removed

\vspace{1em}

L. 278: "Please, cite also "Qualification Tests of 474 Photomultiplier Tubes for the Inner Detector of the Double Chooz Experiment" where this technique is used as well"\\
\textbf{Answer:} Added a reference to the paper.

\vspace{1em}

L. 307: "Figure 14 and Figure 13 should be swapped"\\
\textbf{Answer:} The reason to have a nice take home message, is, why we have swapped Figure 14 and Figure 13. So we end with a nice spectrum and not with the waveform, which shows a noise event. 
Doubling an image from section 7 is what we wanted to avoid, because the paper is not too long.

\vspace{1em}

L. 322: "In my opinion, I would remove this part from the conclusion. Maybe adding one of the nice figure of merit found in section 7, it would be a nice take home message at the end of the paper \lr{A322}"\\
\textbf{Answer:} We removed the part from the summary and added it in the introduction \lr{A322_fix}. In the conclusion we added a part stating the performance of the system, see \lr{A322_concl}.

\vspace{1em}

L. 325: "Probably this should go in a dedicated section
Forn JINST author's maunal:
Acknowledgments. The command \verb|\acknowledgments| starts a new non-numbered section
where the acknowledgments can be placed. It usually resides before the bibliography, or at the end of the introduction."\\
\textbf{Answer:} Thank you for the advice. We included the \verb|\acknowledgments| section \lr{A325}

}