\section{Performance of a single \acs{iPMT}}
\label{sec:performance}
In this section the representative performance of a single \ac{iPMT} is shown.
%
For the following measurements, the \ac{iPMT} has been placed in a magnetic shielding box. An LED illuminates the photo cathode with short pulses of low intensity. LED and \ac{iPMT} share a common trigger. The response of the \ac{iPMT} to the light emission is recorded and analysed. For studies of the single photo electron signal, the light intensity of the LED has been tuned such\replace{,}{} that a pulse is observable in about \qty{10}{\%} of all recorded events. The synchronous detection of two or more photons is highly suppressed.

% Figure environment removed

Only those waveforms are processed whose average baseline in the baseline window (cf. Figure~\ref{fig:exem_wv_pmt}) -- from \qtyrange[evaluate-expression]{2*\baselineAVGstart}{2*\baselineAVGend}{\nano\second} -- is in a range of $\pm 1\sigma$ around the mean average baseline. The accepted average baseline values are \qtyrange{\CUTbaselineN}{\CUTbaselineP}{\LSB}.
%
% Note, here the cut is based on the histogram of the baselines (average over 30 samples)
% below in the amplitude characteristics the baseline is derived from the histogram of the ADC samples (no averaging before histogramming)
%
By using this cut, the simple charge integration method works reliably and baseline fluctuations from earlier hits are rejected. 
Using a more advanced waveform reconstruction algorithm would supersede this cut.

\subsection{Charge characteristics}
To analyse the charge, the waveform is corrected for the baseline, first. The average of the samples in the baseline window is subtracted from all samples. The charge integration window ranges from 
\qtyrange[evaluate-expression]{2*\IntegrationStart}{2*\IntegrationEnd}{\nano\second}
(cf. Figure~\ref{fig:exem_wv_pmt}). The histogram of these waveform integrals are shown in Figure~\ref{fig:exem_charge_spectrum}. It can be seen, that a tiny fraction of events with two \ac{PE} exists in the measurement as well.

% Figure environment removed

The charge histogram is fitted with Equation~\ref{eq:ptv_fit_func} to extract the gain and other parameters.
\begin{align}
\label{eq:ptv_fit_func}
f_{\text{fit}}(x) &= f_{\text{ped}}(x)+f_{\text{pe1}}(x)+f_{\text{pe2}}(x)+f_{\text{exp}}(x)\\
%\end{align}
%
%\begin{align}
f_{\text{ped}}(x) &= s_{\text{ped}}\cdot e^{-0.5\cdot ((x-\mu_{\text{ped}})/\sigma_{\text{ped}})^2}\\
f_{\text{pe1}}(x) &= s_{\text{pe1}}\cdot e^{-0.5\cdot ((x-(\mu_{\text{ped}}+\text{gain}_{\text{pe1}}))/\sigma_{\text{pe1}})^2}\\
f_{\text{pe2}}(x) &= s_{\text{pe2}}\cdot e^{-0.5\cdot ((x-(\mu_{\text{ped}}+2\cdot \text{gain}_{\text{pe1}}))/(\sqrt{2}\sigma_{\text{pe1}}))^2}\\
f_{\text{exp}}(x) &= 
\begin{dcases*}
e^{R_{0}+\tau \cdot (x-x_v)} & if $x > x_v$\\
0 & \textrm{otherwise}
\end{dcases*}.
\end{align}
%
$f_{\text{ped}}(x)$ is a Gaussian component to model the noise, also denoted as pedestal or noise peak. 
$f_{\text{pe1}}(x)$ and $f_{\text{pe2}}(x)$ are Gaussian functions that describe the charge signals corresponding to one, respective two \aclp{PE}.
Since the \ac{PMT} is a linear amplifier for this low number of \ac{PE}, the respective Gaussian mean values are correlated via the variables $\text{gain}_{\text{pe1}}$ and $\sigma_{\text{pe1}}$.
$f_{\text{exp}}(x)$ is an empirical approach to include other effects into the model to better describe especially the region between pedestal peak and single \ac{PE} peak.
The variable $x_v$ is the starting point of this fit on the falling edge of the pedestal peak~\cite{WysotzkiPhD,DC_pmt_test_2011}.
The \ac{SNR} is given by
\begin{equation}
\text{SNR}_\text{charge} = \frac{ \text{gain}_{\text{pe1}} }{ \sigma_{\text{ped}} } = \add{\num[round-mode=uncertainty]{\chargeSNR +- \chargeSNRUncert}}\, .
\end{equation}
The noise -- in terms of charge -- is $\frac{1}{\text{SNR}} = \qty[evaluate-expression, round-mode=places,round-precision=2]{\chargeNoiseWidth}{\ac{PE}}$ wide. The charge resolution is calculated as
\begin{equation}
\text{Resolution}_\text{charge} = \frac{ \sigma_{\text{pe1}}} { \text{gain}_{\text{pe1}} } = \add{\qty[round-mode=uncertainty]{\chargeResolution +- \chargeResolutionUncert}{\%}}\, .
\end{equation}
The charge resolution is not fully determined by the readout electronics. The amplification process inside the \ac{PMT} limits  this value~\cite{BscKuhlbusch}.

From the fit (Equation \ref{eq:ptv_fit_func}), the peak to valley ratio can be determined. It is calculated from the height of the single \ac{PE} peak divided by the number of entries in the valley between the single \ac{PE} peak and the noise peak. For this measurement the value is \num{\PeakToValley}, compared to the value of \num{3.08} measured by the vendor.

\subsection{Characteristics of the waveform maximum}
This study is done using the raw \ac{ADC} waveforms without any additional calibration. A distribution of the maximum \ac{ADC} value within the integration window is shown in Figure~\ref{fig:amp_dist}. Those waveforms which have a charge of $\pm 1\sigma$ around the single \ac{PE} peak (see Figure~\ref{fig:exem_charge_spectrum}) are plotted in red. From this distribution it can be seen that the  single \ac{PE} maximum is on average \qty{ \amplSPEmean +- \amplSPEsigma}{\LSB}. With the baseline of \qty[round-mode=uncertainty]{ \amplBASELINEmean +- \amplBASELINEsigma }{\LSB}, 
the baseline corrected maximum of a single \ac{PE} is \qty[round-mode=uncertainty,round-precision=3]{ \amplCalcSPEheight +- \amplCalcSPEheightSigma }{\LSB}. Using these values, the dynamic range of the high gain receiver with a full scale value ${\text{FS}_\text{HG} = \qty{255}{\LSB}}$ can be calculated as
\begin{equation}
    \text{N}_\text{PE, HG} = \frac{ \text{FS}_\text{HG} - \text{baseline}_\text{HG} }{\text{maximum}_\text{SPE, HG} - \text{baseline}_\text{HG} } 
    %= \frac{ 255 - \qty{\amplBASELINEmean}{LSB} }{\qty{\amplSPEmean}{LSB} - \qty{\amplBASELINEmean}{LSB} } 
    = \qty[round-mode=uncertainty,round-precision=3]{ \NPEinHG +- \NPEinHGsigma }{\ac{PE}}
\end{equation}
% output of error calculation:
%A \termA \\
%B \termB \\ 
%C \termC \\
%D \termD \\
%E \termE \\
%F \termF \\
%
% Figure environment removed
As for the charge, the \ac{SNR} can be calculated accordingly to
\begin{equation}
\text{SNR}_\text{maximum} = \frac{ \text{maximum}_\text{SPE, HG} - \text{baseline}_\text{HG} }{ \sigma_{\text{baseline}} } 
= \num[evaluate-expression, round-mode=places,round-precision=2]{ (\amplSPEmean - \amplBASELINEmean) / \amplBASELINEsigma }\, .
\end{equation}
This value is comparable to the charge based value. The resolution can also be derived from the measurement shown in Figure~\ref{fig:amp_dist} as
\begin{equation}
\text{Resolution}_\text{maximum} = \frac{ \sigma_{\text{ampl}}} { \text{maximum}_\text{SPE, HG} - \text{baseline}_\text{HG} } = \qty[evaluate-expression, round-mode=places,round-precision=2]{100 * ( \amplSPEsigma / (\amplSPEmean - \amplBASELINEmean) )}{\%}\, .
\end{equation}
The resolution of the waveform maximum is less than the charge resolution. But for the maximum only the highest value inside the integration window is used. For the charge, all information from the whole pulse is used. Using a more dedicated waveform reconstruction algorithm might increase the maximum resolution as well.

\subsection{Self triggered charge spectrum}
In \ac{OSIRIS}, the \acp{iPMT} operate in self-triggered mode. They send out a waveform with a timestamp once their local trigger condition is met. With this acquisition mode, a charge spectrum can be recorded as well. Figure~\ref{fig:self_trg_charge_spectrum} depicts multiple charge spectra recorded with increasing trigger thresholds. It starts at the lowest trigger threshold, which can be recorded without data loss. The charge spectrum obtained with the external trigger is plotted for comparison, too.
%
% Figure environment removed
%
 The trigger thresholds are given in raw \ac{ADC} counts. After subtraction of the baseline~(\qty{\amplBASELINEmean}{\LSB}), the minimal pulse height can be derived, which is accepted with this setting.
%
% Figure environment removed
%
Figure~\ref{fig:self_trg_charge_spectrum} shows that the noise suppression increases with the trigger threshold. However, there are still few noise events remaining, which cause a peak between \qtyrange{0}{100}{\LSB \cdot 2\nano\second}. These events have a damped oscillation signature of unknown source as shown in Figure~\ref{fig:self_trg_noisy_wv}.
%
With increasing trigger threshold, the detection threshold in terms of light intensity increases as well.