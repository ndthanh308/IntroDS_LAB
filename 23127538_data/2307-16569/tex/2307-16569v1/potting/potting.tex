\subsection{Potting \& Holder}
\label{sec:potting_holder}
The \acp{iPMT} will be submerged in a pool of ultra-pure water. The electronics need to be encapsulated to protect it from contact with water and to keep the water clean. 
The maximum water depth in \ac{OSIRIS} is \qty{9}{\meter}, which is equal to a maximum water pressure of \qty{0.9}{\bar} the \ac{iPMT} has to withstand. Since this pressure is low and the \acp{PMT} are separated sufficiently, implosion protection is not necessary.

The electronics-stack is encapsulated inside a stainless steel shell. To ensure proper cooling of all electronics boards, this shell is filled with white oil. Via convection, the oil transports the heat from the electronics to the shell, where it gets transferred to the surrounding water. The water in the \ac{OSIRIS} tank is kept at a stable temperature of about \qty{21}{\celsius}.

A \ac{PMMA} ring  is glued to the neck of the \ac{PMT}. The electronics is mounted on the back of the \ac{PMT}, the shell is glued to the \ac{PMMA} ring and filled with oil (cf. to~\cite{PhDFengGao} for details of the potting). The oil expands under heating. If no countermeasures are taken, this expansion will damage the electronics. An air-filled \ac{HDPE} bottle is placed inside the shell to absorb the expansion of the oil.

Special measures have been taken to seal the joint between the cable and the shell. In the region, where the cable enters the shell, all wires are stripped down to the bare copper. For isolation, the cable is embedded in a low viscosity epoxy. It stops creeping of oil or water along the wires in both directions.

% Figure environment removed

The location of the items described above can be seen in Figure~\ref{fig:iPMTassemblyOverview}. The whole \ac{iPMT} is mounted together with its electromagnetic shield~\cite{MagneticShieldingPaper} in a stainless steel holder. This holder fixes the position and orientation of the \ac{iPMT} within \ac{OSIRIS}. For further support of the \ac{PMT}, four clamps, made from \ac{HDPE}, are located around the equator.
