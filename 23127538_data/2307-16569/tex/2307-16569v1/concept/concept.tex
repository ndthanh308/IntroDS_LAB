\section{Concept of the \acl{iPMT}}
\label{sec:concept}
For the readout of large liquid scintillator detectors, long analog cables have been used in the past. For example, Borexino uses a cable length of \qty{57}{\meter} for all \acp{PMT} \cite[p. 12]{TheBorexinoDetector}. A comparable cable length is used in the SuperKamiokande detector \cite{SuperKelectronicsCalibration}. If the size of the detector gets larger, the length of those cables increases. Thus, the attenuation along these cables increases as well. 

Assuming an {RG58} cable with typical attenuation at \qty{100}{\mega\hertz} of \mbox{\qty{12}{\dB}/\qty{100}{\meter}}. At a length of \qty{60}{\meter} the output voltage is reduced to \qty{43.6}{\percent} at \qty{100}{\mega\hertz}.
% 20 log (V/V_0) dB => V/V_0 = 10^(db/20) => V/V_0 = 10^(-12/20) = 0.2511
% for 60 meters
% V/V_0 = 10^(-12*0.6/20) = 0.436
The attenuation depends on the frequency. Typically, the high frequencies are attenuated more, resulting in a shaping of the signal. Recovery of the original signal is only possible with enormous \replace{afford}{effort}. 
%The attenuated high frequencies have to be amplified more than the lower frequencies in exactly the inverse attenuation. 
The attenuated frequencies have to be amplified to compensate the attenuation. 
In case the amplification and the attenuation do not match, the original signal is not recovered.

The idea behind the concept of the \acp{iPMT} is to move digitiser and readout as close as possible to the \ac{PMT}. By this, the analog signal path can be reduced to a few centimeters. With the digital output, the cable length no longer has an impact on the quality of the data. The length of the digital cable is limited as well, but the digital signals can be refreshed without loss of information.

In order to handle the digitised data, a signal processing unit is inserted at the back of each single \ac{PMT} as illustrated in Figure~\ref{fig:ipmt_electronics}. The processing unit packs the digitised data into waveforms, which are transmitted digitally via Ethernet. In addition, this unit provides resources for data reduction or reconstruction. Also high level algorithms, e.g. a stabilisation of the gain of the \ac{PMT}, can be implemented on the processing unit.

% Figure environment removed
\todo[inline, disable]{Do we want to have an overview of the iPMT electronics (this graph needs obviously some extension)}

The waveforms are transmitted on a self-triggered basis. If the processing unit detects a pulse in the waveform \add{by e.g. a simple threshold discriminator or more advanced pulse detection techniques}\R{A80}, this waveform packet is sent to the \ac{DAQ}. Inside the \ac{DAQ} software, a global trigger and event building is implemented and noise hits are discarded.

The whole \add{readout}\R{A83} concept was initially developed for usage in the \ac{JUNO} central detector \cite{Bellato_2021}. After it became clear that the concept is not going to be used for this detector, the whole concept was iterated again. This second iteration is presented in this paper.