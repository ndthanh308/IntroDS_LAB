\section{\acl{SB}}
\label{sec:surface_board}
The main purpose of the \acf{SB} is the synchronisation of all \acp{iPMT}.
%As a first step the Ethernet pairs of the cable get split from the synchronous ones passively. The Ethernet pairs are further forwarded to a commercial Ethernet Switch.
Each \ac{SB} hosts eight cards (Connector Boards) which can connect to six \acp{iPMT} each (cf. Figure~\ref{fig:sb_backplane}). In total, up to \qty{48}{\acp{iPMT}} can be connected to a single \ac{SB}. For \ac{OSIRIS}, two \acp{SB} are needed. One \ac{SB} acts as a \emph{master} and provides the \emph{slave} \ac{SB} with a clock and a datastream, which is forwarded without any further processing.

% Figure environment removed

After the cables from the \acp{iPMT} arrive at the \ac{SB}, their Ethernet pairs are \remove{split} passively \add{separated}\R{A204} from the synchronous ones and further forwarded to a commercial Ethernet switch.
The synchronous connection to the \ac{iPMT} can be divided into a downlink (\ac{SB} to \ac{iPMT}) and an uplink with the reverse direction. All downlinks are fed with the same signal, which is generated by a single output of a \Xilinx~\ZYNQ on the \ac{SB}. This signal is distributed by clock splitters to the cable drivers of all \acp{iPMT}. For the uplinks, the signal of the \ac{iPMT} is routed from the cable receiver to a dedicated input for each \ac{iPMT}. This creates the possibility to measure the length of the cable \add{ by injecting a pulse on the downlink and measuring the travel time, when looped back inside the \ac{iPMT}}.\R{A210} The link \replace{quality}{integrity}\R{A211} can be measured using \ac{PRBS}. 

Programming \remove{of} the \ZYNQ and monitoring \remove{the} temperatures, voltages etc. on the \ac{SB}, is done by \replace{a}{one} \ac{SCCU} using \ac{JTAG} and \ac{I2C}. In addition, the \acp{UART} of the \ac{SCCU} are connected to the \ZYNQ for monitoring and debug purposes.

\subsection{LED \& Laser Trigger}
Each \ac{SB} provides two independent \add{periodic} trigger outputs via a Trigger Board. The trigger \replace{repetition frequency}{rate} can be adjusted in a range from \qty[exponent-to-prefix = true, scientific-notation=engineering]{0.12}{\hertz} to \qty[exponent-to-prefix = true, scientific-notation=engineering]{2.6e6}{\hertz}. The width of the output pulse can be configured from \qtyrange{16}{496}{\nano\second}. The outputs are used to trigger light emission from the Laser and LED calibration systems within \ac{OSIRIS}. A maximal rate of \qty{10}{\kilo\hertz} for the laser trigger is expected~\cite{OSIRISDesign}.
%
Each output pulse is tagged with a timestamp. This information is sent via Ethernet to the \ac{DAQ}, where this information is used to form calibration events.

\subsection{Synchronisation}
All \acp{iPMT} are supplied with the identical synchronous datastream with a bitrate of \qty{125}{\mega\bit\per\second}. A Manchester code is used on the stream which reduces the datarate to \qty{62.5}{\mega\bit\per\second}. A \qty{125}{\mega\hertz} clock is recovered from the datastream on the \ac{ROB} by the \ac{CDR}. The recovered clock is used as reference clock for VULCAN. By using the same frequency on all \acp{iPMT}, a frequency synchronisation is achieved (syntonization). Via the embedded datastream the \acp{iPMT} are partially synchronised. After this partial synchronisation, a time offset, which is dominated by the cable length difference between the \acp{iPMT}, remains. This offset can be determined by measurements of the electrical cable length or using pulsed Laser illumination events. A combination of both methods will provide a full synchronisation of all \acp{iPMT}.

% synchronisation of all iPMTs by using synchronous datastream
% - this regards (only) frequency synchronization (also called syntonization)
% - a complete synchronization possible in combination with calibration 
% Figure environment removed
For testing the partial synchronisation, two half stacks (\ac{POE}-Board, \ac{SCCU} and \ac{ROB}) are connected to a pulse generator as a signal source. The cable length to both stacks is not matched (cf. Figure \ref{fig:sync_setup}). When the signal generator creates a pulse, this gets recorded by both stacks. Each stack tags this event with its own timestamp.
Figure \ref{fig:sb_sync_single_pulse} shows part of a single recorded waveform. To determine the proper timestamp, a linear function is fitted to the rising edge. The time when this interpolation crosses \qty{100}{\LSB} is used as the event time $t_{s0}$ and $t_{s1}$. Figure~\ref{fig:sb_time_diff_vs_time} shows the difference between both event times against the measurement time. The total measurement time shown in Figure~\ref{fig:sb_time_diff_vs_time} is divided into six times \qty{2}{\hour} due to file size constrains. The mean synchronisation difference between both used half-stacks is \qty{20.84 \pm 0.28}{\nano\second}.
The difference of \qty{20.84}{\nano\second} is determined by the different cable length. More relevant is the variation of the synchronisation over time, characterised by a standard deviation of \qty{0.28}{\nano\second}. It is well below the \ac{PMT}'s intrinsic \ac{TTS} of about \qty{1.1}{\nano\second}\footnote{Figure~\ref{fig:pmt_dist_tts} states a mean \ac{TTS} of \qty{2.6}{\nano\second} as \ac{FWHM}. With the assumption of a Gaussian distribution this number converts to a standard deviation of \qty{1.1}{\nano\second}.}.
% Figure environment removed
