%  \vspace{-0.1em}
\section{Conclusion}
% Our goal in this work is to draw the community’s attention to role attribution analysis can/should play in interpreting policies.  Specifically, in the common case of cross-modal fusion … We demonstrate this using ALFRED as an example domain …
In this work, we draw the community's attention to attribution analysis for interpreting multimodal policies.
We provide the framework MAEA for attribution analysis to gain insights into multimodal embodied AI policies. We compare seq2seq (baseline, MOCA) and transformer-based (ET, HiTUT) policies trained on the ALFRED dataset and highlight the modality biases in these models. We also analyze how the focus center in language instructions moves as the episode progresses, and discuss how visual attributions can be used for analyzing successful/unsuccessful action predictions.
Note while we use this gradient-based attribution, the ideas of multimodal attribution can be generally applicable to other kinds of attributions. 
% In future work, attribution analysis can be utilized to rank and group the failure scenarios, to investigate modeling and dataset biases, and to critically analyze multimodal EAI policies for robustness and user trust before deployment, thereby improve the success and robustness of the multimodal policies. 
This technique could also be used to better understand the correlation between certain types of biases and failure cases in certain in-distribution as well as out-of-distribution scenarios.

