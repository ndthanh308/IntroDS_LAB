\section{Conclusion}
The move from \as{hdd} to flash \as{ssd}, has been one of the most fundamental shifts in the storage hierarchy. The increased performance of flash \as{ssd} over \as{hdd}, capable of achieving single several GB/s bandwidth with millions of \as{iops}, however required adaptations in the software stack, changing the design of file systems and storage software to integrate with these devices. In this literature study we evaluate the current state of file systems for flash storage devices, how these file systems design to align with introduced flash characteristics, and how the integration of flash storage into file system design has affected storage stack design.

Iterating the considered research objectives for this literature study, the main findings show that \textbf{(RQ1)} NAND flash storage has influenced file system design particular in the aspects of no in-place updates and requiring erasing of flash blocks prior to rewriting data. Resulting file system design has largely focused on \as{lfs}, aligning with the sequential write requirement of the storage, however introducing complexity through required \as{gc} and effective \as{wl} across the flash. We present numerous effective methods, showing that dealing with \as{gc} through efficient \as{gc} policies, write buffering, reducing write traffic, and effective data grouping techniques is a widely adopted and research field. Numerous file systems present different approaches and policies. Furthermore, the increased performance of flash storage is largely dependent on exploiting the internal parallelism, of which favored methods involve data striping across parallel flash units and increasing the parallelism. An additional major change is the reduction of the software stack, which was originally designed for slower storage devices (\as{hdd}), and is showing its shortcomings in integrating flash storage and providing and scalable and performant storage systems. User-space file systems, bypassing a significant amount of the software stack layers are increasingly popular adoption for eliminating software stack layers.

The main challenges of integrating with NAND flash storage \textbf{(RQ2)} revolve around dealing with introduced \as{gc} overheads, limiting its introduced \as{io} amplifications, exploiting the possible parallelism capabilities of flash storage, and lowering the storage software stack overheads. We showcase the different methods adopted by flash file systems for dealing with these challenges, evaluating developments of the past decades on how to effectively integrate flash storage in storage software design. While we emphasize the design for file systems, the majority of the evaluated methods are applicable to the larger spectrum of storage software. We therefore include methods from \as{ftl} design, which have not yet been adopted by file system designers, providing the possible methods for future flash optimized file system design.

Given the increasing rise in adoption for flash storage, and the introduction of new flash-based storage devices and interfaces, future implications of flash storage \textbf{(RQ3)} provide a promising ground of better integrating the flash storage with software, leveraging the increased performance capabilities further. Through new interfaces exposing a larger amount of flash characteristics, the host software gets an increasing level of possibility to design application specific flash management, integrating the application design with flash management. Future storage software developments are likely to continue integrating the closer integration of flash storage into host storage software design, optimizing the flash utilizing for full leveraging of flash performance. We envision the semantic gap between storage hardware and software to slowly decrease over time, allowing to build more performant, efficient, reliable, and responsible storage systems. Such efforts align with the grand challenges of future storage~\cite{2022-compsys-manifesto}, particularly with increasing demand of systems due to the digitalization of the world, and the push towards a more sustainable future. 
