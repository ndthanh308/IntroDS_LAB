\section{Introduction}\label{sec:introduction}
With the increasing amount of data, estimated to reach 200 Zettabytes by the year 2025~\cite{2020-Morgan-Data_Attack_Surface}, efficient storage systems are becoming imperative. A large contribution factor to increased data generation is the gain in popularity for big data~\cite{Khan2014BigDS,2019-Baig-Big_Data,2015-Abaker-Big_data_on_cloud} and cloud services~\cite{rimal2011architectural,antonopoulos2010cloud}. While there exist a plethora of different storage technologies, the most prevalent type is \af{hdd}~\cite{arpaci2018operating,2011-Deng-Disk_Drives_Future}, which are now largely being replaced by \af{ssd}~\cite{cornwell2012anatomy}. \as{hdd} is one of the cheapest forms of storage, however is limited in performance due to requiring on mechanical movement to access data on the disk. This results in high latency for random access patterns~\cite{ding2007diskseen,jiang2005dulo} and additionally increases power demand~\cite{2006-Feng-Disk_Energy,gurumurthi2003drpm}. While \as{ssd} is more expensive than \as{hdd}, it is becoming more affordable~\cite{mohan2012refresh} and provides increased performance over \as{hdd}~\cite{kasavajhala2011solid}, resulting in a growing adoption for enterprise businesses~\cite{daim2008forecasting,2008-Lee-Flash_in_enterprise_db}. 

One of the most fundamental mechanisms of storing and organizing data on \as{hdd}, \as{ssd}, and other storage technologies, is through the use of file systems, enabling the structural organization of data on persistent storage media. Building efficient and performant file systems for the evolving storage media technologies and progressing with future demands of data storage is of paramount importance. With \as{hdd} having been the prevailing storage technology for decades, file system and application design revolved around the intrinsic characteristics of these devices. In particular, aiming to limit access patterns to sequential accesses~\cite{chang2008bigtable,oneil1996log}, in an effort to minimize mechanical movement on the disk and thus optimize their performance.

The most widely adopted type of \as{ssd} is based on \textit{flash storage}, having different characteristics than traditional \as{hdd}. Performance of flash storage achieves several GB/s, with millions of \af{iops}~\cite{2022-samsung-zand,2022-intel-p}, and access latency as low as single digit $\mu$-second latency. However, flash storage has its own characteristics different from \as{hdd}. In particular, flash storage does not support in-place updates, requiring data to be erased at a larger unit in order to be written again. Additionally, the cost of erase operations is substantially higher than read and write operations~\cite{stoica2009evaluating,2022-intel-p}. In order to hide these constraints from host systems, flash \as{ssd} employs firmware, called the \af{ftl}, that exposes a sector-addressable interface. This allows \as{ssd} to be addressed in the same way as conventional \as{hdd}, requiring no changes in host software for accessing the different storage technologies.

While \as{ssd} and \as{hdd} utilize the same interfaces to be addressed, in order to exploit the increased performance benefits of flash storage, software must integrate with the characteristics of flash storage. Adapting software design to align with flash storage characteristics helps minimize \as{ftl} overheads to manage the flash storage. With the increasing adoption of flash \as{ssd} in enterprise, a plethora of applications and file systems have been proposed aiming at integrating software design with flash storage characteristics. In this survey we evaluate the changes in software, particularly in file systems, caused by integrating with flash storage characteristics, and how these changes have affected file system design. We additionally assess the future implications of evolving flash storage technologies to file system and software design. In order to evaluate the various work on flash storage implications for file system design, we devise three key \af{srq} that aim at analyzing past, current, and future trends.

\begin{itemize}[left=0.9cm]
    \item [\textbf{SRQ1.}] \textbf{What are the main challenges arising from NAND flash characteristics and its integration into file system design?} \\
        Flash storage has particular characteristics, such as sequential writing, no in-place updates, and requiring erasing of flash blocks. This \as{srq} aims at analyzing what particular challenges arise for storage software from the flash-specific constraints and resulting effects of on-device operations. Devising a list of key challenges provides the foundation based on which relevant work in this literature study is selected, and the final report is structured.
    \item[\textbf{SRQ2.}] \textbf{How has NAND flash storage influenced the design and development of file system and the storage software stack?} \\
        Using the identified challenges in \textbf{\as{srq}1}, this \as{srq} evaluates for each of the challenges, how file system design has changed to integrate with it. As file systems are commonly built on top of existing storage software layers, such as the Linux Block \as{io} layer, we include methods and mechanisms in the storage software stack particularly devised for file systems and flash storage integration. As a result, this \as{srq} evaluates how the depicted challenges are addressed throughout the various software stack layers, up to the file system.
    \item [\textbf{SRQ3.}] \textbf{How will NAND flash storage and newly introduced NAND flash-based storage devices and interfaces affect future file system design and development?} \\
        With a particular goal of this literature study being to evaluate the validity of data structures, algorithms, and mechanisms of flash, and understanding the applicability to \as{zns}, a newly arising storage technology, this research question furthermore aims at evaluating future challenges that may arise from new technology.
\end{itemize}

Furthermore, this literature study makes the following contributions:

\begin{itemize}
    \item We devise six key challenges for storage software arising from the integration of flash-based \as{ssd}, particularly focusing on leveraging its capabilities and enhancing device utilization.
    \item For each of the six devised flash integration challenges, we summarize the main methods of relevant work on dealing with and integration the particular challenge(s) into file system design.
    \item Based on the findings of this literature study, we present a discussion on the future applicability of the presented methods during this study, and evaluate the effects of newly arising flash-based \as{ssd} devices.
\end{itemize}
