\section{Research Questions}\label{sec:lit_study_rqs}
In order to evaluate the various work on flash storage implications for file system design, we devise three key \af{srq} that aim at analyzing past, current, and future trends.

\begin{itemize}
    \item [\textbf{SRQ1.}] \textbf{What are the main challenges arising from NAND flash characteristics and its integration into file system design?} \\
        Flash storage has particular characteristics, such as sequential writing, no in-place updates, and requiring erasing of flash blocks. This \as{srq} aims at analyzing what particular challenges arise for storage software from the flash-specific constraints and resulting effects of on-device operations. Devising a list of key challenges provides the foundation based on which relevant work in this literature study is selected, and the final report is structured.
    \item[\textbf{SRQ2.}] \textbf{How has NAND flash storage influenced the design and development of file system and the storage software stack?} \\
        Using the identified challenges in \textbf{\as{srq}1}, this \as{srq} evaluates for each of the challenges, how file system design has changed to integrate with it. As file systems are commonly built on top of existing storage software layers, such as the Linux Block \as{io} layer, we include methods and mechanisms in the storage software stack particularly devised for file systems and flash storage integration. As a result, this \as{srq} evaluates how the depicted challenges are addressed throughout the various software stack layers, up to the file system.
    \item [\textbf{SRQ3.}] \textbf{How will NAND flash storage and newly introduced NAND flash-based storage devices and interfaces affect future file system design and development?} \\
        With a particular goal of this literature study being to evaluate the validity of data structures, algorithms, and mechanisms of flash, and understanding the applicability to \as{zns}, a newly arising storage technology, this research question furthermore aims at evaluating future challenges that may arise from new technology.
\end{itemize}
