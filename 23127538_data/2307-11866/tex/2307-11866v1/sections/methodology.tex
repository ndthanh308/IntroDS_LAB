\section{Literate Study Research Methodology}\label{sec:lit_methodology}
Several methodologies for conducting literature studies exist such as an unguided traversal of the
literature~\cite{2018-Hegeman-Survey_Graph_Analysis} and using
\textit{snowballing}~\cite{Webster2002AnalyzingTP,2014-Wohlin-Snowballing}. However, we find that said methods lack
systematic mechanisms in their evaluation, where the search space of relevant literature with unguided is not clearly
defined and can become incomprehensibly large. Snowballing on the other hand allows limiting the search space by
evaluation going forward, studies that reference the seed paper, and backwards, studies that the seed paper references,
in the citations from a set of seed papers systematically. However, it can introduce possible bias from the seed paper
selection. Therefore, we utilize a combination of approaches and additionally apply the \af{slr} presented by Kitchenham
et al.~\cite{kitchenham2007guidelines}. The \as{slr} approach relies on three separate stages to construct a systematic
literature review, depicted in \cref{tab:SLR_process}. As we do not inclusively apply every possible stage of the
\as{slr} method, the table additionally depicts which methods we apply in this study (indicated with a \cmark), and
where relevant information can be found in this literature study.

\begin{table}[t]
    % Update reference format to fit table into column
    \crefformat{section}{\S#2#1#3}
    \crefformat{subsection}{\S#2#1#3}
    \centering
    \begin{tabular}{c c c}
        \hline
        & \textbf{Planning the review} & \\
        \hline
        \cmark & Identification of the need for a review & (\cref{sec:introduction}) \\
        \xmark & Commissioning a review & - \\
        \cmark & Specifying the research question(s) & (\cref{sec:introduction}) \\
        \cmark & Developing a review protocol & (\cref{sec:review_protocol}) \\
        \xmark & Evaluating the review protocol & - \\
        \hline \hline
        & \textbf{Conducting the review} & \\
        \hline
        \cmark & Identification of research & (\cref{sec:search_space_selection}) \\
        \cmark & Selection of primary studies & (\cref{sec:study_selection}) \\
        \xmark & Study quality assessment & - \\
        \cmark & Data extraction and monitoring & (\cref{sec:study_analysis}) \\
        \cmark & Data synthesis & (\cref{sec:study_analysis}) \\
        \hline \hline
        & \textbf{Reporting the review} & \\
        \hline
        \xmark & Specifying dissemination mechanism & - \\
        \cmark & Formatting the main report & - \\
        \xmark & Evaluating the report & - \\
        \hline
    \end{tabular}
    % Reset reference format to before
    \crefformat{section}{Section #2#1#3}
    \crefformat{subsection}{Section #2#1#3}
    \caption{Outline of the Systematic Literature Review approach presented by Kitchenham et al.~\cite{kitchenham2007guidelines}, inspired by the structure of the literature study on Graph Analysis by Hegeman and Iosup~\cite{2018-Hegeman-Survey_Graph_Analysis}.}
    \label{tab:SLR_process}
\end{table}

With these three stages, clear protocol and process definitions are established prior to conducting the review
(discussed in \cref{sec:review_protocol}), limiting possible reviewer bias and additionally enhancing reproducibility.
The initial stage consists of planning the review, which includes establishing the need for this certain review and
developing the review protocol. The protocol encompasses the inclusion and exclusion criteria, as well as research
question definition, and the establishment of the review process. Based on an established protocol we conduct the
review, establish the selection of studies to evaluate, and proceed to extract and analyze the studies. Lastly, with all
collected data on selected studies, we format this survey to present the review in a comprehensible manner.

\subsection{Review Protocol}\label{sec:review_protocol}
By making use of the \as{slr} process, we initially define a review protocol and review processes that we detail for this study. With emphasis on making this study reproducible, we provide detailed descriptions of each phase in the review process. A visual representation of the application of the review protocol phases is depicted in \cref{fig:review_protocol}. It revolves around three phases, firstly establishing the search space from which we extract relevant studies for this literature study. In the second phase we collect the studies and apply the defined selection criteria (explained in \cref{sec:study_selection}). Lastly, we analyze the collected relevant studies. The following sections explain each of the phases in detail, \cref{sec:search_space_selection} explains methods used for establishing the search space of literature. Next, \cref{sec:study_selection} depicts the selection criteria for extracting relevant studies from the defined search space. Lastly, \cref{sec:study_analysis} provides the methods of data extraction from studies and gives the organization of studies for this literature review.

\input{figs/review_protocol.tex}

\subsection{Search Space Selection}\label{sec:search_space_selection}
The first stage of our review protocol defines the definition of the search space from which relevant studies are extracted. We make use of several approaches for identifying relevant studies. Firstly, we apply the snowballing method on a set of seed papers. Seed papers selected for this literature review are depicted in \cref{tab:seed_papers}. With these seed papers, we analyze the studies from forward and backward citations of the seed paper. To further expand the search space of relevant studies, we examine the publications of numerous conferences, workshops, and journals which are focused on the area of systems and storage research. These conferences are analyzed in the range of 2010-2022, if present, as some are bi-annual or may not have been established in the given time range. The following venues are checked in this literature review: 

\begin{itemize}
    \item USENIX Annual Technical Conference (\textit{USENIX ATC})
    \item USENIX Conference on File and Storage Technologies (\textit{FAST})
    \item Networked Systems Design \& Implementation (\textit{NSDI})
    \item European Conference on Computer Systems (\textit{EuroSys})
    \item USENIX Symposium on Operating Systems Design and Implementation (\textit{OSDI})
    \item Symposium on Operating Systems Principles (\textit{SOSP})
    \item ACM International Systems and Storage Conference (\textit{SYSTOR})
    \item ACM Workshop on Hot Topics in Storage and File Systems (\textit{HotStorage})
    \item Architectural Support for Programming Languages and Operating Systems (\textit{ASPLOS})
    \item ACM Special Interest Group on Management of Data (\textit{SIGMOD})
    \item International Conference on Very Large Data Bases (\textit{VLDB})
    \item IEEE International System-on-Chip Conference (\textit{SOCC})
    \item International Conference on Distributed Computing Systems (\textit{ICDCS})
    \item ACM/IFIP Middleware Conference (\textit{Middleware})
    \item ACM Transactions on Storage (\textit{TOS})
    \item International Conference for High Performance Computing, Networking, Storage, and Analysis (\textit{SC})
    \item International Conference on Massive Storage Systems and Technology (\textit{MSST})
    \item IEEE International Conference on Computer Design (\textit{ICCD})
\end{itemize}

\begin{table}[!t]
    \centering
    \begin{tabular}{||c c c||}
        \hline
        Title & Venue & Publication Year \\
        \hline
        \hline
        F2FS~\cite{2015-Changman-f2fs} & FAST & 2016 \\
        JFFS~\cite{woodhouse2001jffs} & OLS & 2001 \\
        LogFS~\cite{engel2005logfs} & Linux Kongress & 2005 \\
        \hline
    \end{tabular}
    \caption{Seed papers used for this literature review. Titles are shortened.}
    \label{tab:seed_papers}
\end{table}

Lastly, we run individual queries on academic search engines for scholarly literature; Google Scholar, Semantic Scholar, and dblp. We utilize two types of queries; \af{rsq} for finding of relevant studies for this literature study, and \af{rlsq} for finding related work to this literature review. Related work encompasses surveys on file systems for flash, flash specific algorithms and data structures, and additional studies of flash related application and system integration. For each query, with each search engine, we analyze the 100 most relevant results (or less if there are fewer query results). The keyword queries for finding relevant studies and relevant related literature studies are, respectively:

\begin{itemize}[align=left,leftmargin=1.3cm]
    \item[\textbf{RSQ1.}] Flash File System
    \item[\textbf{RSQ2.}] NVM File System
    \item[\textbf{RSQ3.}] SSD File System
    \item[\textbf{RSQ4.}] File System \as{spdk}~\footnote{\af{spdk}~\cite{2017-Yang-SPDK} provides a number of tools and libraries for building high performant user-level storage software over NVMe, making it applicable to file system development on flash SSDs.}
\end{itemize}
\begin{itemize}[align=left,leftmargin=1.3cm]
    \item[\textbf{RLSQ1.}] Flash File System Survey
    \item[\textbf{RLSQ2.}] NVM File System Survey
    \item[\textbf{RLSQ3.}] SSD File System Survey
\end{itemize}

\noindent Finding of related surveys is not limited to the established queries, but additionally during the snowballing and synthesis of conference, workshop, and journal publications we identify related work based on the prior defined classifications. Inclusion of relevant studies outside of the time range from 2010-2022 is only applicable when the study is retrieved using snowballing from seed papers, or the study is present in one of the respective query results. The timing constraint is thus only applied to the extraction of relevant studies by analyzing publications at conferences, workshops, and journals.

\subsection{Study Selection Criteria}\label{sec:study_selection}
The second phase of the review protocol defines the study selection, which extracts the relevant studies from the defined search space with a set of established criteria. We define a specific Inclusion/Exclusion criteria, with which studies are selected for this literature review based on the appropriateness for the criteria. These criteria are based on the defined research questions and are aimed to narrow down the search space to a particular set of studies of interest in this review, and enforce only relevant work is included. While studies do not have to exclusively meet all the inclusion requirements to be included in this review, any if any of the exclusion criteria is present, the study is not included in this review.

\begin{itemize}
    \item[\textbf{I1.}] The work is novel.
    \item[\textbf{I2.}] The work designs a file system specifically for NAND flash storage.
    \item[\textbf{I3.}] The work adapts an existing file system to integrate with NAND flash storage.
    \item[\textbf{E1.}] The work designs or adapts a file system not specifically for NAND flash storage.
    \item[\textbf{E2.}] The work designs or adapts a hybrid/tiered file system that utilizes various storage technologies, not focusing file system design to NAND flash storage.
    \item[\textbf{E3.}] The work designs or adapts a file system which is evaluated on NAND flash storage, but not particularly built for NAND flash storage.
    \item[\textbf{E4.}] The work designs or adapts a file system for \as{ssd}, but not specifically for NAND flash storage.
\end{itemize}

While meeting inclusion requirements does not mean a study is guaranteed to be included, it is more likely to be included. In the case of exclusion criteria, exceptions are made in cases of using papers to establish background knowledge or building context, however they are not the core focus of the respective section where its content is discussed.

Furthermore, there has been a plethora of flash-based file systems which utilize hybrid/tiered storage devices, including NOR-based flash~\cite{2008-Chul-Hybrid_NOR_NAND_FS}, \af{scm}~\cite{2013-Sheng-NVMFS,2008-Youngwoo-PFFS,2010-Jaemin-Frash,2016-Mazumder-WSN_FS,2014-Li-PCM_NAND_Fusion}, byte addressable \gls{nvm}~\cite{2019-Lee-Parallel_LFS}, and methods that expose byte addressable NVRAM on SSDs with custom firmware for metadata placement~\cite{2021-Zhou-Remap_SSD,2017-Jin-Byte_Adressable_SSD}. Our focus is on block address storage, which is the most prevalent for storage. Additionally, combining of multiple storage technologies, such as \as{ssd} and \as{hdd}, commonly utilize \as{ssd} as a cache for the file system on the \as{hdd}, similarly shifting focus away from flash storage integration for the file system. For this reason we exclude such file system designs from the core study of this literature review. Similarly, file systems that are not designed specifically for flash storage, but have flash-friendly characteristics are also excluded. This mainly includes log-structured file systems that are intended for HDD, with a focus on writing sequentially to minimize arm movement on the disk, which coincidentally matches the sequential write requirement of flash.

\subsection{Study Analysis}\label{sec:study_analysis}
The last stage of the review protocol defines the extraction of data from the relevant studies, and establishing the final report. During evaluation of the different studies selected for this literature review, we disseminate the information presented based on their answering of the defined research questions. For this, we define the various key integration challenges of flash storage integration (\cref{sec:flash_challenges}), based on the hardware and software characteristics of flash storage. Using the defined integration challenges, we divide the contributions of the studies evaluated in this literature study into the respective challenge, and discuss its mechanisms for solving the particular flash integration challenge. These challenges are evaluated in \cref{sec:ssd_rw_asym,sec:gc,sec:io_amplification,sec:flash_parallelism,sec:wear_leveling,sec:io_sched,sec:failure_consistency}. Lastly, we discuss on the findings of the main findings from this literature study, followed by the relevant related literature studies for flash storage in \cref{sec:related_work}.

\subsection{Limitations}
Albeit this literature study being extensively defined and established through clear protocol definitions, there are several limitations that remain. Firstly, the search space selection is only based on studies that have scientific literature. Therefore, file systems for flash which may be in the mainline Linux kernel, are not guaranteed to be included in this survey, if no scientific literature on it exists. Secondly, the search space is limited to only scientific literature written in English. This additionally limits the inclusion of conferences, workshops, and journals to venues with proceedings in English. Thirdly, given that snowballing uses a manual selection of meeting inclusion/exclusion criteria, possible bias is inherent. Lastly, as the resulting queries on the selected literature search engines produces several hundred thousand results, and we select to evaluate the first 100 results, we rely on the sorting of results based on relevance that each search engine provides. While we utilize multiple search engines in an effort to avoid bias from each search engine, it does not completely eliminate it.

