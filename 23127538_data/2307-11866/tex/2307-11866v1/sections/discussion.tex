\section{Discussion}
With the multitude of methods for dealing with flash integration, there are several that are of key concern. In particular, \as{gc} and mechanisms of dealing with \as{gc} have shown to be applicable, and enhance other flash integration challenges. However, there is no panacea for solving all flash integration challenges. Incorporating a particular mechanism causes difficulty of integration with other flash challenges. Especially, depending on the level of integration of flash storage, there is limited possibility of integrating particular mechanisms. Therefore, the closer integration is largely necessary in order to more optimally integrate flash, however coming with the additional flash management complexity. Therefore, a tradeoff between the level of integratin and the required complexity must be made. A higher level of management allows the application to more optimally be designed for the flash. This however comes at the expense of increased complexity. Furthermore, generic file systems working on numerous integration levels, require more generic interface management, limiting their customizability for different integrations. 

% While we discussed methods for integrating flash at the file system and software level, there exist other flash-based storage devices and interfaces that can provide adapted integration for host software. Such include a flash design that allows multi-write coding of the flash pages in order to reduce erasing and garbage collection~\cite{jagmohan2010write}. With multi-write coding, the bits in a flash page can be reprogrammed, without a required erase, to a bit value higher than it currently is. As an erase on a flash page resets the bits to a 1, any bit that is set to a 0 can be set to a 1 without requiring to erase the entire flash page. Similar flexible storage interfaces such as \as{ocssd}, and Prism-SSD~\cite{2022-Shen-PrismSSD} provide increased possibility for flash management. Such interfaces aim at reducing the mismatch between storage devices and their interfaces, which allow to better integrate flash characteristics, and the devised flash integration challenges, into file system and storage software design.

% \section{Future Predictions}
Clear trends in the storage community are becoming apparent, focusing on eliminating the hiding of flash management idiosyncrasies, and exposing its characteristics to the host. Therefore, allowing the host to integrate and optimize the software for its access patterns with increased knowledge of the underlying storage device characteristics~\cite{bjorling2017lightnvm,2012-asplos-moneta,kang2014multi,bhimani2017enhancing,2010-Josephson-DFS,2021-Bjorling-ZNS}. As stated in the CompSys Manifesto~\cite{2022-compsys-manifesto}, ''the grand challenge in storage systems is to combine heterogeneous storage layers, leveraging their programmability and capabilities to deliver a new class of cost, data, and performance efficiency for all kinds of applications''. Reducing the semantic gap between storage hardware and software must be a driving concern in future storage system design, leveraging the existing hardware capabilities and enhancing integration for more efficient, effective, sustainable, and reliable storage systems.

The recent addition of \af{zns} \as{ssd}~\cite{2021-Bjorling-ZNS,2022-Tehrany-Understanding_ZNS}, standardized in the \as{nvme} 2.0 specification~\cite{2022-nvme-spec}, similarly provides host the control over \as{gc} management on the storage device, and matches the interface to the underlying flash characteristics. 
% It exposes flash storage through a number of \textit{zones} which must be written sequentially, and must be erased prior to being overwritten~\cite{2021-ZNS-ZBD_docs}. 
Leveraging such device interfaces allows the file system development to partially take control of on-device operations. In addition to eliminating duplicate work of file system and \as{gc} on the device, where the increased coordination provides better performance capabilities, the higher-level control of the file system allows it to apply its data knowledge, such as particular grouping based on file characteristics, without a significant increase in complexity through the zone interface of \as{zns} \as{ssd}. Such integration presents a well-defined interface aimed at eliminating the interface mismatch between flash storage and storage software, leaving a plethora of possibility for storage software to enhance flash integration. Exposing more storage device control to the host system, particularly in the configuration of \as{zns}, allows not only data placement to be better integrated into storage software, but furthermore allows to fully utilize the parallel units of the storage device. With \as{zns} devices, the parallelism unit of the storage device is clearly defined~\cite{2022-Bae-ZNS_Parallelism}, and can be leveraged by the storage software.

% With \as{zns} having been a recent proposal at the time of writing, file system support is limited. F2FS provides capabilities to run on \as{zns} \as{ssd}~\cite{2021-Bjorling-ZNS}, however requires an additional randomly writable device for its metadata. ZenFS~\cite{2021-Bjorling-ZNS,2022-zenfs-git} is a domain specific file system for \as{zns} that provides the backend for the RocksDB key-value store. Therefore, currently there does not exist a purely \as{zns} based file system, which we envision to be a prospect of future developments in storage. 

We imagine interfaces exposing an increasing level of flash hardware characteristics to the host software, to begin appearing, in order to better coordinate and integrate the storage. Similarly, we envision future efforts to aim at further decreasing the semantic gap between storage hardware and software, and leverage a higher degree of coordination across storage software/hardware layers. The gain in popularity of user-space based applications, as we discussed in \cref{sec:user_space_fs}, presents prominent possibility for future development and limiting kernel involvement, which has become an increasing overhead in the software stack. Therefore, we furthermore picture an increase in user-space applications, and particularly file systems.
