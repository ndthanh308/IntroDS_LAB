\section{Flash Storage Integrations}
As there are various methods for integrating flash into storage devices, in addition building full \as{ssd} devices, ranging from directly attaching the flash chip to the motherboard, as is common with embedded mobile and \as{iot} devices, or custom integrations of flash chips, we evaluate the file systems based on their level of integration. Given that a different integration exposes a different interface, the possibility to enhance particular operations is highly dependent on the integration.

Therefore, throughout this literature study, we divide the relevant work based on the type of flash integration. \cref{fig:flash_integrations} shows three integration levels for flash, where \cref{fig:flash_integration_ssd} depicts the conventional integration with a \as{ssd}. \cref{fig:flash_integration_custom} shows a custom integration of flash storage for devices such as \af{ocssd}~\cite{2019-Lu-Sync_IO_OCSSD,bjorling2017lightnvm}, multi-stream SSD~\cite{bhimani2017enhancing,kang2014multi}, and \af{sdf}~\cite{2014-Ouyang-SDF}. The main benefit of these types of integration is that the flash characteristics are no longer hidden behind the device, giving the host an increasing level of storage control. \as{ocssd} is a type of \as{ssd} that exposes the device geometry to the host, allowing the host to manage device parallelism and allocation. While such a device allows increased data management for the host software, it comes at increased complexity for managing the device constraints. 

Lastly, \cref{fig:flash_integration_embedded} shows flash integration at the embedded level, such as is commonly used in
mobile devices and \as{iot} devices. In embedded flash configuration the flash chip is commonly directly attached to the
motherboard, giving the host system full control over the underlying flash storage. Throughout this literature study, we group file system design and mechanisms based on these three integration levels, as different levels of integration allow different degrees of flash management and ranging possibility for flash integration.

% Over time the device contains an increasing number of invalid pages within blocks, requiring to run \af{gc} to read out still valid pages in the block, move these pages to a free block on the SSD, and erase the original block. An additional requirement for performing \as{gc} is the need for extra space, such that the \as{ftl} can move valid pages to a free space before erasing the block. This extra space is called the \af{ops}, and typically ranges between 10-28\% of the device capacity. Due to the limited life of flash cells~\cite{mohan2010learned,2016-Schroeder-Flash_Reliability}, the FTL has to additionally ensure even wear across the device, such that no particular region of the flash storage is burnt out quicker than other regions. This process is referred to as \af{wl}. 

% Figure environment removed


