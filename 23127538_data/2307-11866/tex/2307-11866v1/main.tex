%%%%%%%%%%%%%%%%%%%%%%%%%%%%%%%%%%%%%%%%%%%%%%%%%%%%%%%%%%%%%%%%%%%%%%%%%%%%%%%%
% Template for USENIX papers.
%
% History:
%
% - TEMPLATE for Usenix papers, specifically to meet requirements of
%   USENIX '05. originally a template for producing IEEE-format
%   articles using LaTeX. written by Matthew Ward, CS Department,
%   Worcester Polytechnic Institute. adapted by David Beazley for his
%   excellent SWIG paper in Proceedings, Tcl 96. turned into a
%   smartass generic template by De Clarke, with thanks to both the
%   above pioneers. Use at your own risk. Complaints to /dev/null.
%   Make it two column with no page numbering, default is 10 point.
%
% - Munged by Fred Douglis <douglis@research.att.com> 10/97 to
%   separate the .sty file from the LaTeX source template, so that
%   people can more easily include the .sty file into an existing
%   document. Also changed to more closely follow the style guidelines
%   as represented by the Word sample file.
%
% - Note that since 2010, USENIX does not require endnotes. If you
%   want foot of page notes, don't include the endnotes package in the
%   usepackage command, below.
% - This version uses the latex2e styles, not the very ancient 2.09
%   stuff.
%
% - Updated July 2018: Text block size changed from 6.5" to 7"
%
% - Updated Dec 2018 for ATC'19:
%
%   * Revised text to pass HotCRP's auto-formatting check, with
%     hotcrp.settings.submission_form.body_font_size=10pt, and
%     hotcrp.settings.submission_form.line_height=12pt
%
%   * Switched from \endnote-s to \footnote-s to match Usenix's policy.
%
%   * \section* => \begin{abstract} ... \end{abstract}
%
%   * Make template self-contained in terms of bibtex entires, to allow
%     this file to be compiled. (And changing refs style to 'plain'.)
%
%   * Make template self-contained in terms of figures, to
%     allow this file to be compiled. 
%
%   * Added packages for hyperref, embedding fonts, and improving
%     appearance.
%   
%   * Removed outdated text.
%
%%%%%%%%%%%%%%%%%%%%%%%%%%%%%%%%%%%%%%%%%%%%%%%%%%%%%%%%%%%%%%%%%%%%%%%%%%%%%%%%


\documentclass[letterpaper,twocolumn,10pt]{article}
\usepackage{usenix-2020-09}

\usepackage{times,url,color,soul,xspace,enumitem}
\usepackage{xurl}
\usepackage{breakurl}
\usepackage{graphicx}
\usepackage{caption}
\usepackage{subcaption}
\usepackage{comment}
\usepackage{xspace}

\usepackage{booktabs}
\usepackage{tabularx}
\usepackage{multirow}
\usepackage{makecell}
\usepackage{threeparttable}
\newcommand{\mtnote}[1]{\textsuperscript{\TPTtagStyle{#1}}}

\usepackage{xcolor,colortbl}

\usepackage{tcolorbox}
\tcbuselibrary{skins}
\definecolor{BgBlue}{HTML}{336699}

\newcommand{\sbi}{\textit{sbi}}
\newcommand{\crs}{\texttt{curseg\_array}}

\usepackage{tcolorbox}
\definecolor{C0}{RGB}{85,105,151}
\definecolor{C1}{RGB}{187,129,92}
\definecolor{C2}{RGB}{96,145,101}
\definecolor{C3}{RGB}{161,88,86}

\newcommand{\highlight}[2]{\colorbox{#1!17}{$\displaystyle #2$}}
\newcommand{\highlightdark}[2]{\colorbox{#1!47}{$\displaystyle #2$}}
% to be able to draw some self-contained figs
\usepackage{tikz}
\usepackage{amsmath}
\usepackage{xcolor}
\usepackage[inline,draft,nomargin,index,author=]{fixme}

\usepackage[noabbrev,capitalize]{cleveref}
\crefformat{section}{Section #2#1#3}
\crefformat{subsection}{Section #2#1#3}
\crefformat{subsubsection}{Section #2#1#3}

\FXRegisterAuthor{nty}{anty}{\textcolor{orange}{Nty}}
\newcommand{\nty}[1]{\ntynote{\textcolor{orange}{#1}}}

\usepackage{todonotes}
% fixme theme to disable note comment, just want the counter to work for custom todo command
\fxsetup{theme=color}
\newcommand{\ntytodo}[1]{\todo[inline, color=orange!30]{TODO \textcolor{red}{\textbf{nty}}: #1\fxnote{}}}

\usepackage{adjustbox}
\usepackage{tikz}
\usetikzlibrary{positioning,calc,fit,backgrounds}

% custom tikz commands for subplot reuse
\pgfdeclarelayer{bg0}
\pgfdeclarelayer{bg1}
\pgfdeclarelayer{bg2}
\pgfdeclarelayer{bg3}
\pgfdeclarelayer{foreground}
\pgfsetlayers{bg0,bg1,bg2,bg3,main,foreground}
% colors
\definecolor{lightblue}{HTML}{B1DDF0}
\definecolor{lightorange}{HTML}{FFE6CC}
\definecolor{lightred}{HTML}{FAD9D5}
\definecolor{lightgreen}{HTML}{B9E0A5}

\tikzset{
    title/.style={font=\fontsize{10pt}{10pt}\color{black!90}\bfseries,anchor=center,text centered,align=center},
    outerbox/.style={rectangle, draw=black!80, rounded corners,line width=.9pt, minimum size=0.5cm, minimum width=1cm,minimum height=2cm},
    box/.style={rectangle,font=\fontsize{10pt}{10pt}\color{black!90}\bfseries, draw=black!80, line width=.9, fill=white!100, line width=.5pt, minimum size=1cm, minimum width=2.5cm,minimum height=.5cm},
    dot/.style = {circle, fill, minimum size=3pt, inner sep=0pt, outer sep=0pt},
    fp/.style={rectangle,font=\fontsize{10pt}{10pt}\color{black!90}\bfseries, draw=black!80, line width=.9, fill=lightgreen, line width=.5pt, minimum size=1cm, minimum height=1cm, minimum width=1cm,align=center},
}

\newcommand{\plothdots}[1]{
    \node (dt#1) [dot] [below=.1cm and 0cm of #1] {};
    \node (dt1#1) [dot] [below=.1cm and 0cm of dt#1] {};
    \node (dt2#1) [dot] [below=.1cm and 0cm of dt1#1] {};
}

\newcommand{\plotvdots}[1]{
    \node (dt#1) [dot] [right=0cm and 0.6cm of #1] {};
    \node (dt1#1) [dot] [right=0cm and .2cm of dt#1] {};
    \node (dt2#1) [dot] [right=0cm and .2cm of dt1#1] {};
}

\newcommand{\block}[5]{
    \node (blk#1) [title] [#4] {Block #5};
    \node (pg1#1) [box,fill=gray!15] [below=.15cm and 0cm of blk#1] {Page 1};
    \node (pg2#1) [box,fill=gray!15] [below=-0.5pt and 0cm of pg1#1] {Page 2};
    \node (dt#1) [dot] [below=.1cm and 0cm of pg2#1] {};
    \node (dt1#1) [dot] [below=.1cm and 0cm of dt#1] {};
    \node (dt2#1) [dot] [below=.1cm and 0cm of dt1#1] {};
    \node (pgn#1) [box,fill=gray!15] [below=.1cm and 0cm of dt2#1] {Page $n$};
    \begin{pgfonlayer}{bg3}
        \node (#2) [outerbox,rounded corners=0,fill=lightorange] [fit=(blk#1) (pg1#1) (pg2#1) (dt#1) (dt1#1) (dt2#1) (pgn#1),#3] {};
    \end{pgfonlayer}
}

\newcommand{\plane}[2]{
    \block{0p#1}{blkbox0p#1}{}{}{0}
    \plothdots{blkbox0p#1}
    \block{mp#1}{blkboxmp#1}{below=0.1cm and 0cm of dt2blkbox0p#1}{below=0.1cm and 0cm of dt2blkbox0p#1}{$m$}
    \node (titlep#1) [title] [above=0.4cm and 0cm of blk0p#1]{Plane #1};
    \node (drp#1) [box,fill=lightblue] [below=.2cm and 0cm of blkboxmp#1] {Data Register};
    \node (crp#1) [box,fill=lightred] [below=.15cm and 0cm of drp#1] {Cache Register};
    \begin{pgfonlayer}{bg2}
        \node (p#2) [outerbox,fill=lightgreen] [fit=(titlep#1) (blkbox0p#1) (blkboxmp#1) (drp#1) (crp#1)] {};
    \end{pgfonlayer}
}


\usepackage{amssymb}
\usepackage{pifont}
\newcommand{\cmark}{\ding{51}}
\newcommand{\xmark}{\ding{55}}
\newcommand{\qmark}{\textbf{?}}

%-------------------------------------------------------------------------------
% glossary of acronyms

\usepackage[acronym,automake,nonumberlist]{glossaries}
\makeglossaries

\chapter*{Glossary of Sets and Constructions}
\label{chap:Glossary}
% next resets the equation numbers to start at 1 at the start of the chapter
\setcounter{equation}{0}
\renewcommand{\theequation}{\thechapter.\arabic{equation}}

We give a table that details all of the major sets and operators that are used in this thesis, for convenience and for reference.

%\begin{table}[h]
%\begin{tabular}{p{2.5cm}|p{8cm}|p{2cm}}
\begin{longtable}{p{2.5cm}|p{8.25cm}|p{1.75cm}}
\textbf{Name} & \textbf{Description} & \textbf{Thesis Ref.} \\
 \midrule
$m$-reducibility & Given two sets $A$ and $B$, $A$ is $m$-reducible to $B$, written $A \leq_m B$, if there exists some computable function $f: \omega \rightarrow \omega$ such that for all $x \in \omega$, $x \in A \iff f(x) \in B$ & \ref{def:mred}\\
\hline
Weihrauch Reducibility & Given two operators $f$ and $g$ on represented spaces, we say $f \leq_W g$, if there exist computable $H,K :\subseteq \omega^\omega \rightarrow \omega^\omega$ such that for any realizer $G \vdash g$, $F = K\langle id_{\omega^\omega}, GH \rangle$ is a realizer for $f$. & \ref{def:Weihrauch} \\
\hline
$WELL$ & The set of all indices $e$ such that $\varphi_e$ is the characteristic function of a well-founded tree $T \subseteq \omega^{<\omega}$. & \ref{def:WELL}\\
\hline
$ILL$ & The set of all indices $e$ such that $\varphi_e$ is the characteristic function of an ill-founded tree $T \subseteq \omega^{<\omega}$. & \ref{def:ILL}\\
\hline
$TILE$ & The set of all indices $e$ such that $\varphi_e$ is the characteristic function of an infinite Wang prototile set whose tilings are total in the plane. & \ref{def:TILE} \\
\hline
$WTILE$ & The set of all indices $e$ such that $\varphi_e$ is the characteristic function of an infinite Wang prototile set whose tilings are infinite, connected, but not necessarily total in the plane. & \ref{def:WTILE} \\
\hline
$SNT$ & The set of all indices $e$ such that $\varphi_e$ is the characteristic function of an infinite Wang prototile set whose connected tilings are all finite. & \ref{def:SNT}\\
\hline
$ATile$ & Set of all $e$ such that $\varphi_e$ is the characteristic function for a set of prototiles who planar tilings are all total and aperiodic. & \ref{def:ATile} \\
\hline
$PTile$ & Set of all $e$ such that $\varphi_e$ is the characteristic function for a set of prototiles who planar tilings are all total and periodic. & \ref{def:PTile} \\
\hline
$ATile_{FIN}$ & Set of all $e$ such that $\varphi_e$ is the characteristic function for a \emph{finite} set of prototiles who planar tilings are all total and aperiodic. & \ref{def:ATileFIN}\\
\hline
$PTile_{FIN}$ & Set of all $e$ such that $\varphi_e$ is the characteristic function for a \emph{finite} set of prototiles who planar tilings are all total and periodic. & \ref{def:PTileFIN}\\
\hline
AIT & The construction found in the proof of theorem \ref{thm:TILE-ILL} that creates an aperiodic prototile set given an ill-founded tree. & \ref{def:AITPIT}\\
\hline
PIT & The construction found in the proof of theorem \ref{thm:ILL-PTile} that creates an aperiodic prototile set given an ill-founded tree. & \ref{def:AITPIT}\\
\hline
$CT$ & The operator that takes some set of Wang prototiles as input and returns a total tiling of the plane. & \ref{def:ChooseTiling} \\
\hline
$CWPT$ & An operator that takes a set of Wang prototiles and returns a connected planar, but not necessarily total tiling. & \ref{ref:CWPT} \\
\hline
$CIPT$ & An operator that takes a prototile set $S$ that has total planar tilings, and returns an infinite `slice' of this tiling as a tiling of an infintie region of $\mathbb{Z}^2$. & \ref{def:CIPT} \\
\hline
$WIPT$ & An operator that takes a set of prototiles and return a tiling that has an infinite patch tiled within it, but we do not know where it is. & \ref{def:WIPT} \\
\hline
$DPW$ & The $DPW$ operator takes some set of prototiles and return a tiling that has an infinite connected patch within it. & \ref{def:DPW} \\
\hline
$C_{\omega^\omega}$ & Closed choice on Baire space - equivalent to finding a path through an ill-founded Baire space tree. & \ref{def:ClosedChoice} \\
\hline
$C_{2^\omega}$ & Closed choice on Cantor space - equivalent to Weak K\"onig's Lemma. & Sec \ref{sec:FurtherWeakTilingProblems} \\
\hline
$C_{\omega}$ & closed choice on the natural numbers - this takes a function $f: \omega \rightarrow \omega$ such that $range(f) \neq \omega$, and returns some point $n \notin range(f)$. & Sec. \ref{sec:FurtherWeakTilingProblems} \\
\end{longtable}
%\end{tabular}
%\end{table}

%-------------------------------------------------------------------------------

%-------------------------------------------------------------------------------
\begin{document}
%-------------------------------------------------------------------------------

%don't want date printed
\date{}

% make title bold and 14 pt font (Latex default is non-bold, 16 pt)

\title{\Large \bf A Survey on the Integration of NAND Flash Storage \\ in the Design of File Systems and the Host Storage Software Stack \\ { \normalsize Survey done: July 2022}} 
% \title{\Large \bf The Past, Present, and Future of File Systems for \\ NAND Flash SSD Storage Devices: A Survey}
%for single author (just remove % characters)
\author{
{\rm Nick Tehrany}\\
Delft University of Technology\\
n.a.tehrany@vu.nl
\and
{\rm Krijn Doekemeijer}\\
Vrije Universiteit Amsterdam\\
k.doekemeijer@vu.nl
\and
{\rm Animesh Trivedi}\\
Vrije Universiteit Amsterdam\\
a.trivedi@vu.nl
}

\maketitle

%-------------------------------------------------------------------------------
\begin{abstract}
    With the ever-increasing amount of data generate in the world, estimated to reach over 200 Zettabytes by 2025, pressure on efficient data storage systems is intensifying. The shift from \as{hdd} to flash-based \as{ssd} provides one of the most fundamental shifts in storage technology, increasing performance capabilities significantly. However, flash storage comes with different characteristics than prior \as{hdd} storage technology. Therefore, storage software was unsuitable for leveraging the capabilities of flash storage. As a result, a plethora of storage applications have been design to better integrate with flash storage and align with flash characteristics.

In this literature study we evaluate the effect the introduction of flash storage has had on the design of file systems, which providing one of the most essential mechanisms for managing persistent storage. We analyze the mechanisms for effectively managing flash storage, managing overheads of introduced design requirements, and leverage the capabilities of flash storage. Numerous methods have been adopted in file systems, however prominently revolve around similar design decisions, adhering to the flash hardware constrains, and limiting software intervention. Future design of storage software remains prominent with the constant growth in flash-based storage devices and interfaces, providing an increasing possibility to enhance flash integration in the host storage software stack.
\end{abstract}


\section{Introduction}
Current quantum hardware is unable to carry out universal quantum computations due to the buildup of errors that occur during the computation. 
The magnitude of the individual error is currently above the value that the Threshold Theorem requires in order to kick-start quantum error correction and fault-tolerant quantum computation~\cite[Section 10.6]{nielsen_chuang_2010}. 
Although the experimentally achieved fidelity rates are promising and the error bounds are inching closer to the required threshold, we will have to work for the foreseeable future with quantum hardware with errors that build-up during the computation.  This implies that we can only do a limited number of steps before the output of the computation has become completely uncorrelated with the intended one.

For fault-tolerant quantum computing, we repeat four steps: 
1) We apply a number of single and two-qubit quantum gates, in parallel whenever possible; 
2) We perform a syndrome measurement on a subset of the qubits; 
3) We perform fast classical computations to determine which errors have occurred and how to correct them; 
and, 4) We apply correction terms based on the classical computations.
We then repeat these four steps with a next sequence of gates. 
These four steps are essential to fault-tolerant quantum computing. 


The starting point of this work is to use the four steps outlined above, not to carry out error correction and fault-tolerant computation, but to enhance short, constant-depth, {\em uncorrected} quantum circuits that perform single qubit gates and {\em nearest-neighbor} two qubit gates. 
Since in the long run we will have to implement error-correction and fault-tolerant computation anyhow, and this is done by such a four-step process, why not make other use of this architecture? Moreover, on some of the quantum hardware platforms, these operations are already in place.
Embracing this idea we naturally arrive at the question: what is the computational power of \textit{low-depth} quantum-classical circuits organized as in the four steps outlined above? 
We thus investigate circuits that execute a small, ideally constant, number of stages, where at each stage we may apply, in parallel, single qubit gates and {\em nearest-neighbor} two qubit gates, followed by measurements, followed by low-depth classical computations of which the outcome can control quantum gates in later stages. 
It is not clear, at first, whether such circuits, especially with constant depth, can do anything remotely useful. 
But we will see that this is indeed the case: many quantum computations can be done by such circuits in constant depth. 
By parallelizing quantum computations in this way, we improve the overall computational capabilities of these circuits, as we do not incur errors on qubits that are idle, simply because qubits are not idle for a very long time. 
Furthermore, reducing the depth of quantum circuits, at the cost of increasing width, allows the circuit to be run faster even if errors occur.

The first usage of such a four-step layout, not to do error correction, but to perform computations, can be found in the paradigm of measurement-based quantum computing~\cite{gottesman1999demonstrating,raussendorf2001one,jozsa2006introduction,clark2007generalised}: 
A universal form of quantum computing where a quantum state is prepared and operations are performed by measuring qubits in different bases, depending on previous measurements and intermediate measurements.

\citeauthor{PhamSvore2013} were the first to formalize the four-step protocol for performing computations~\cite{PhamSvore2013}. They included specific hardware topologies by considering two-dimensional graphs for imposing constraints on qubit interactions. In their model, they develop circuits for particularly useful multi-qubit gates, including specifying costs in the width, number of qubits, depth, number of concurrent time steps, size, and total number of non-Identity operations.
As a result, they find an algorithm that factors integers in polylogarithmic depth.
\citeauthor{Browne:2011} showed that the main tool in the work by \citeauthor{PhamSvore2013}, the fan-out gate, can also be replaced by additional log-depth classical computations in the measurement-based quantum computing setting~\cite{Browne:2011}.

More recently, \citeauthor{Cirac:2021} introduced a scheme to implement unitary operations involving quantum circuits combined with Local Operations and Classical Communication ($\mathsf{LOCC}$) channels: $\mathsf{LOCC}$-assisted quantum circuits~\cite{Cirac:2021}. Similarly to the four-step scheme we just described, they allow for a short depth circuit to be run on the qubits, followed by one round of $\mathsf{LOCC}$, in which ancilla qubits are measured and local unitaries are applied based on the measurement outcomes. They show that in this model any 1D transitionally invariant matrix-product state (MPS) with fixed bond dimension is in the same phase of matter as the trivial state. Similar ideas can be found in~\cite{TVV_NonAbelianTopologicalOrder_2022, tantivasadakarn2021long}.

In this work, we introduce a new model, called \textit{Local Alternating Quantum-Classical Computations} ($\LAQCC$). In this model we alternate between running quantum circuits (constrained by locality), ending in the measurement of a subset of qubits, and fast classical computations based on the measurement results. The outcome of the classical computations are then used to control future quantum circuits. We allow for flexibility in this model, by giving different constraints to the power of both the quantum circuits and the classical circuits as well as the number of alternations between them. 
Most attention will be given to $\LAQCC$ containing quantum circuits of constant depth, classical circuits of logarithmic depth and at most a constant number of alternations between them. 
Any circuit constructed in this model is considered to be of constant depth. 
We restrict ourselves to logarithmic depth classical computations, as this is the first natural and non-trivial extension beyond constant-depth classical computations. 
Constant-depth classical computations do however also have an equivalent constant-depth quantum implementation.

The definition of $\LAQCC$ sharpens the original definition of \citeauthor{PhamSvore2013} by adding constraints to the intermediate classical computations. This allows us to bound the power of $\LAQCC$ from above. 

The main result of \citeauthor{Cirac:2021}, that 1D translational invariant MPS with fixed bond dimension can be prepared by $\mathsf{LOCC}$-assisted circuits, relies on local symmetries of the MPS. These symmetries allow them to prepare local states (on a constant number of qubits) and glue them together by doing one round of the appropriate entangling measurement and corrections, after which they run a round of local unitaries to get the desired result. This general scheme for preparing states that exhibit an MPS description with the appropriate local symmetries requires only geometrically local unitaries and one round of measurement and corrections an therefore is accessible in $\LAQCC$. Studying different local symmetries, known as Symmetry Protected Topological (SPT) phases of matter, to find measurement-based constant depth circuits for states is a broad ongoing field of research~\cite{TVV_NonAbelianTopologicalOrder_2022, tantivasadakarn2021long, smith2023deterministic}. 
All these schemes have a $\LAQCC$ implementation.

%$\LAQCC$-circuits also exist for general schemes of preparing local states, based on the local tensors, and gluing them together using one round of entangled measurement and corrections, based on the local symmetry. 
%The main result of \citeauthor{Cirac:2021}, that 1D translational invariant MPS with fixed bond dimension can be prepared by $\mathsf{LOCC}$-assisted circuits, relies heavily on local symmetries of the MPS and as a result also has an equivalent $\LAQCC$ implementation. 
%The corrections applied after the measurement round are local unitaries depending on the local symmetries of the MPS. 

 

%This general scheme of preparing local states, based on the local tensors, and gluing it together by doing one round of entangled measurement and corrections, based on the local symmetry, is accessible in $\LAQCC$.
Note however that \citeauthor{Cirac:2021} also suggest a circuit for the $W$-state.
This circuit uses sequentially and dependent measurement-based corrections of the ancilla qubits. 
These dependent measurements translate to sequential alternations between the quantum and classical circuits and therefore increase the total depth to linear depth, exceeding the constant-depth constraints imposed by $\LAQCC$-circuits. 

We study the power of the $\LAQCC$ model with respect to state preparation, showing that even with only constant quantum-depth and logarithmic classical depth it remains possible to prepare states with long-range entanglement.
Another surprising result is that it is unlikely that $\LAQCC$ circuits are classically simulatable. We show that any instantaneous quantum polynomial-time (IQP) circuit~\cite{Bremner2010,Shepherd2009} has an $\LAQCC$ implementation.
Classical simulation of IQP circuits implies the collapse of the polynomial hierarchy to the third level, which is not believed to be true~\cite{Bremner2017}. Therefore, we expect that $\LAQCC$ circuits are unlikely to be classically simulatable. We bound the power of $\LAQCC$ by showing that it is contained in $\QNC^1$, the class of polynomial-size, log-depth circuits.

Next, we also study the power that intermediate classical calculations can add to quantum computations, by considering a new model that alternates between polynomially many polynomial-depth quantum circuits and unbounded classical computations
We study this model by doing a complexity theoretical analysis, where we draw inspiration from the notions of complexity given by \citeauthor{RosenthalYuen:2022}, \citeauthor{MetgerYuen:2023}, and \citeauthor{Aaronson:2004}.
All three complexity notions are based on the notion of state preparation, instead of more traditional definition of complexity such as the decidability of a computational problem. 
The first two consider classes based on sequences of quantum states preparable by a polynomial-sized quantum circuit, where the circuits are uniformly generated by a computational class, for instance, the class $\mathsf{PSPACE}$, which results in the complexity class $\mathsf{StatePSPACE}$~\cite{RosenthalYuen:2022,MetgerYuen:2023}.
The third notion considers a relative complexity, where the complexity is measured between two given states, and is measured by the number of gates, from a given gate-set, required to transform one state in another state~\cite{Aaronson:2004}. 
For our definition of state preparation complexity, we drop the uniformity constraint from~\cite{RosenthalYuen:2022,MetgerYuen:2023} and define a class as $\mathsf{StateX}$, which refers to states preparable by circuits of type $\mathsf{X}$. 
As an example, if $\mathsf{X} = \QNC^0$, this results in the class $\mathsf{StateQNC^0}$, which is the set of states preparable from the $\ket{0}^n$ state by poly-size constant-depth circuits. 
This notion is similar to the relative complexity from~\cite{Aaronson:2004}, where one state is the  $\ket{0}^n$ state and instead of counting the number of gates we consider the set of states preparable by a fixed number of gates. Using this notion of complexity we show that any state preparable by an $\LAQCC^*$ circuit is also preparable by a $\mathsf{PostQPoly}$ circuit, the class of circuits of polynomial depth with an additional post-selection gate. 

All Clifford circuits have a constant-depth $\LAQCC$ implementation, implying that any stabilizer state can be implemented by a constant-depth $\LAQCC$ circuit, see Section~\ref{sec:clifford_circuits} for a proof of this statement. 
Efficient circuits for stabilizer states have been known already through measurement-based quantum computing. Therefore this paper focuses on the preparation of non-stabilizer states, and as a surprising result we find novel constant-depth protocols for four very natural classes of non-stabilizer states.
Despite the extensive research into these four classes of non-stabilizer states and the many applications of them, no efficient constant- or low-depth state preparation protocols are known yet. We specifically consider these four classes as they are all often used as initial states in other algorithms.

The first state is a uniform superposition over an arbitrary number of states. 
This state finds applications in many quantum algorithms, as they often start with a uniform superposition over multiple states. 
This superposition is often achieved by applying Hadamard gates to every qubit due to its simplicity to prepare. 
Yet, the analysis of many algorithms, such as Shor's algorithm~\cite{Shor:1997}, would benefit from a different initial superposition. 
The circuit to prepare the uniform superposition over an arbitrary number of states uses an exact version of Grover search as a subroutine, that turns a probabilistic circuit, with a known constant probability of success, into a deterministic circuit. 
We use the circuit for preparing a uniform superposition over an arbitrary number of states as a subroutine in the next two quantum state preparation protocols. 

The second state is the $W$-state, the uniform superposition over all computational basis states of Hamming-weight~$1$, a natural long-ranged entangled state that displays a fundamentally nonequivalent type of entanglement from the Greenberger–Horne–Zeilinger state~\cite{WState:2000}, for which $\LAQCC$-type constant-depth circuits were previously known~\cite{PhamSvore2013, Cirac:2021}. 
The $W$-state is often used as benchmark for new quantum hardware~\cite{Haffner2005,Neeley2010,GarciaPerez:2021}. 
A novel way to prepare the $W$-state therefore gives a new way to benchmark different quantum devices with each other. 
A circuit for preparing the $W$-state was given in~\cite{Cirac:2021}, but this implementation requires sequentially alternating measurements followed by local unitaries, which in the $\LAQCC$ model is not considered to be of constant depth. 
We improve this protocol by giving an $\LAQCC$ implementation of the $W$-state, based on a compress-uncompress method that links the one-hot and binary encoding of integers.

The third state considered is the Dicke state, a generalization of the $W$-state, a superposition over all computational basis states with Hamming-weight $k$~\cite{Dicke:1954}. 
Dicke states have relevance in various practical settings.
For instance, for quantum game theory~\cite{zdemir2007}, quantum storage~\cite{Bacon_Compress:2006,Plesch:2010}, quantum error correction~\cite{ouyang2014permutation}, quantum metrology~\cite{toth2012multipartite}, and quantum networking~\cite{prevedel2009experimental}. 
Dicke states have been used as a starting state for variational optimization algorithms, most notably Quantum Alternating Operator Ansatz (QAOA)~\cite{Hadfield2019}, to find solutions to problems such as Maximum k-vertex Cover~\cite{Brandhofer2022,cook2020quantum}.
The ground states of physical Hamiltonians describing one-dimensional chains tend to show a resemblance to Dicke states such as states resulting from the Bethe ansatz, making them an ideal starting state when investigating the ground state behavior of these Hamiltonians~\cite{TDL_BetheAnsatzDerivation:2010,B_ExcitedStateQuantumPhaseTransitions:2013,DickeTransitions:2021}. 
For instance, the algorithm by \citeauthor{van2021preparing}, who give an algorithm to prepare the Bethe ansatz eigenstates of the spin-1/2 XXZ spin chain, starts by first preparing a Dicke state~\cite{van2021preparing}. 
A Dicke-state preparation protocol based on the compress-uncompress methodology used in the $W$-state furthermore finds applications in entanglement distillation, where the entanglement of a large state is concentrated on only a few qubits. 
Efficient deterministic circuits for preparing Dicke states have been proposed by \citeauthor{bartschi2019deterministic}~\cite{bartschi2019deterministic, bartschi2022deterministic_short_depth}. 
They provide a quantum circuit of depth $\mathO(k \log(\frac{n}{k}))$, allowing arbitrary connectivity, to prepare a Dicke state, which they conjecture to be optimal when $k$ is constant. 
In this work, we provide a constant-depth $\LAQCC$ circuit below their conjectured bound already for constant $k$. 
However, this does not directly disprove their conjecture, as we allow for intermediate measurements and classical computations. 
More significantly, we even construct constant-depth $\LAQCC$ circuits for $k = \mathO(\sqrt{n})$ greatly improving their bound.
This construction extends the compress-uncompress method for the $W$-state combined with additional subroutines. 

We continue with a log-depth state preparation protocol for the Dicke-state for arbitrary $k$. 
This protocol implements an efficient transformation between the factoradic number representation and the combinatorial number representation of a positive integer. 
The combinatorial number representation relates directly to the Dicke state. 
The provided efficient transformation between number representation systems might be of independent interest. 

We conclude by modifying our protocol for preparing a Dicke-state to a protocol that prepares quantum many-body scar states in constant-depth. 
These states have low entanglement and longer coherence times than states with similar energy density.
These characteristics make many-body scar states interesting to analyze and relevant within physics.
Many-body scar states appear for instance in the AKLT model~\cite{AKLT:1987,MRBAR:2018,MRB:2018} and different spin models~\cite{SI:2019,MOBFR:2020}.
Known methods for preparing these states have polynomial-depth~\cite{Gustafson:2023}, whereas our circuit has constant depth. 

% We conclude by studying the power that intermediate classical calculations can add to quantum computations. 
% In this study, we define a new model that relaxes constant-depth quantum circuits to polynomial depth quantum circuits, log-depth classical calculations to unbounded classical computations and a constant number of alternations to a polynomial number of alternations. 
% We call this model $\LAQCC^*$. 
% We study this model by doing a complexity theoretical analysis, where we draw inspiration from the notions of complexity given by \citeauthor{RosenthalYuen:2022}, \citeauthor{MetgerYuen:2023}, and \citeauthor{Aaronson:2004}.
% All three complexity notions are based on the notion of state preparation, instead of more traditional definition of complexity such as the decidability of a computational problem. 
% The first two consider classes based on sequences of quantum states preparable by a polynomial-sized quantum circuit, where the circuits are uniformly generated by a computational class, for instance, the class $\mathsf{PSPACE}$, which results in the complexity class $\mathsf{StatePSPACE}$~\cite{RosenthalYuen:2022,MetgerYuen:2023}.
% The third notion considers a relative complexity, where the complexity is measured between two given states, and is measured by the number of gates, from a given gate-set, required to transform one state in another state~\cite{Aaronson:2004}. 
% For our definition of state preparation complexity, we drop the uniformity constraint from~\cite{RosenthalYuen:2022,MetgerYuen:2023} and define a class as $\mathsf{StateX}$, which refers to states preparable by circuits of type $\mathsf{X}$. 
% As an example, if $\mathsf{X} = \QNC^0$, this results in the class $\mathsf{StateQNC^0}$, which is the set of states preparable from the $\ket{0}^n$ state by poly-size constant-depth circuits. 
% This notion is similar to the relative complexity from~\cite{Aaronson:2004}, where one state is the  $\ket{0}^n$ state and instead of counting the number of gates we consider the set of states preparable by a fixed number of gates. Using this notion of complexity we show that any state preparable by an $\LAQCC^*$ circuit is also preparable by a $\mathsf{PostQPoly}$ circuit, the class of circuits of polynomial depth with an additional post-selection gate. 

\paragraph{Summary of results}
\begin{itemize}
    \item We give a new definition of a computational model that captures the power of the four step process: applying a constant number of layers of one- and two-qubit gates; performing a syndrome measurement; perform a fast classical computation determining corrections; apply corrections. We call this model \emph{Local Alternating Quantum Classical Computations}, or $\LAQCC$ for short. In this model we bound the allowed quantum operations, intermediate classical calculations, and number of rounds separately. In Section~\ref{sec:LAQCC_model} we define this model and give a list of operations based on results from literature contained in this computational model. In some of these operations we explicitly use that we allow for multiple, but at most constant, rounds  of corrections.
    \item  We show show that there exist $\LAQCC$ circuits that can not be weakly simulated in Section~\ref{sec:IQP_in_LAQCC}. We further show that for every $\LAQCC$ circuit there exists a $\QNC^1$ circuit simulating it perfectly, in Section~\ref{sec:LAQCC_in_QNC1}.
    \item We introduce a new type computational complexity for preparing states and show that the extension of $\LAQCC$ where we allow a polynomial number of rounds and unbounded classical computation, is contained in $\mathsf{PostQPoly}$, the class of polynomial circuits with post-selection, in Section~\ref{sec:Complexity results}.
    \item We show a protocol to prepare the uniform superposition state of size $q$ in $\LAQCC$ using $\mathO(\ceil{\log_2(q)}^2)$ qubits in Section~\ref{sec:superposition_modulo_q}. 
    \item We show a protocol to prepare the $W_n$ state in $\LAQCC$ using $\mathO(n\log(n))$ qubits in Section~\ref{sec:W_state_in_LAQCC}.
    \item We show two ways of preparing the Dicke-$(n,k)$ state. The first method is in $\LAQCC$, works up to $k = \mathO(\sqrt{n})$, uses $\mathO(n^2\log(n))$ qubits, and is found in Section~\ref{sec:dicke:small_k}. The second method is in $\LAQCC\text{-}\mathsf{LOG}$ (an extension of $\LAQCC$ allowing for logarithmic number of alterations instead of constant), works for any $k$, uses $\mathO(\text{poly}(n))$ qubits, and is found in Section~\ref{sec:Dicke_in_LAQCC_LOG}. 
    \item We extend on our $\LAQCC$ method of generating Dicke-$(n,k)$ states for $k = \mathO(\sqrt{n})$ and show a protocol to generate many-body scar states for a particular Hamiltonian in $\LAQCC$ (Section~\ref{sec:many_body_scar}). 
\end{itemize}
Summarized in a table, we provide the following state generation protocols:
\begin{table}[htb]
\centering
\begin{tabular}{l|l|l|l}
\textbf{State description} & \textbf{Width} & \textbf{Depth} & \textbf{Implementation}\\
\hline 
Uniform superposition mod $q$: $\frac{1}{\sqrt{q}} \sum_{i = 0}^{q-1}\ket{i}$ & $\mathO(\ceil{\log^2 q})$ & $\mathO(1)$ & Section~\ref{sec:superposition_modulo_q}\\

$W$-state: $\frac{1}{\sqrt{n}}\sum_{i = 0}^{n-1}\ket{e_i}$ & $\mathO(n \log n)$ & $\mathO(1)$ & Section~\ref{sec:W_state_in_LAQCC}\\

Dicke-$(n,k)$, $k = \mathO(\sqrt{n})$: $\binom{n}{k}^{-1/2}\sum_{x \in \{0,1\}^n: |x| = k} \ket{x}$ &  $\mathO(n^2\log n)$ & $\mathO(1)$ 
&Section~\ref{sec:dicke:small_k}\\

Dicke-$(n,k)$: $\binom{n}{k}^{-1/2}\sum_{x \in \{0,1\}^n: |x| = k} \ket{x}$ & $\mathO(\text{poly}(n))$ & $\mathO(\log n)$ &Section~\ref{sec:Dicke_in_LAQCC_LOG}\\

QMBS: $\ket{S_k} = \frac{1}{k! \sqrt{\mathcal N(n,k)}}(Q^\dagger)^k \ket{\Omega}$ &  $\mathO(n^2\log n)$ & $\mathO(1)$  &  Section~\ref{sec:many_body_scar}
\end{tabular}
\caption{Summary of state preparation protocols given in this paper.}
\label{tab:sate_prep}
\end{table}
In the entry for the quantum many-body scar state $Q$ denotes the raising operator and $\mathcal N(n,k)=\binom{n-k-1}{k}$. 
Section~\ref{sec:many_body_scar} will provide more details on the variables and the implementation. 

\paragraph{Organization of the paper}
\noindent We first introduce relevant preliminaries in Section~\ref{sec:preliminaries}. 
In Section~\ref{sec:LAQCC_model} we formally define the class of Local Alternating Quantum-Classical Computations ($\LAQCC$). We also show that any Clifford circuit can be implemented in constant depth $\LAQCC$ (a result based on a result from measurement-based quantum computing~\cite{jozsa2006introduction}). 
This result allows us to give many useful multi-qubit gates and routines in Section~\ref{sec:gates_created_in_LAQCC}. 
Beyond that we show that constant depth $\LAQCC$ circuits are contained in $\QNC^1$ and that any $\mathsf{IQP}$ circuit has an $\LAQCC$ implementation.
We conclude this section with an analysis of a more powerful instantiation of $\LAQCC$ and show an inclusion with respect to the class $\mathsf{PostQPoly}$, which is the class of circuits of polynomial depth with one additional post-selection gate. 
In Section~\ref{sec:state_prep_in_LAQCC} we give $\LAQCC$ circuit implementations for preparing the uniform superposition over an arbitrary number of states, the $W$-state and the Dicke state up to $k = \mathO(\sqrt{n})$. We furthermore give a log-depth circuit implementation for preparing the Dicke state for any $k$. We conclude by showing a $\LAQCC$ circuit for generating many body scar states of a particular type of Hamiltonian.


% \section{Research Questions}\label{sec:lit_study_rqs}
In order to evaluate the various work on flash storage implications for file system design, we devise three key \af{srq} that aim at analyzing past, current, and future trends.

\begin{itemize}
    \item [\textbf{SRQ1.}] \textbf{What are the main challenges arising from NAND flash characteristics and its integration into file system design?} \\
        Flash storage has particular characteristics, such as sequential writing, no in-place updates, and requiring erasing of flash blocks. This \as{srq} aims at analyzing what particular challenges arise for storage software from the flash-specific constraints and resulting effects of on-device operations. Devising a list of key challenges provides the foundation based on which relevant work in this literature study is selected, and the final report is structured.
    \item[\textbf{SRQ2.}] \textbf{How has NAND flash storage influenced the design and development of file system and the storage software stack?} \\
        Using the identified challenges in \textbf{\as{srq}1}, this \as{srq} evaluates for each of the challenges, how file system design has changed to integrate with it. As file systems are commonly built on top of existing storage software layers, such as the Linux Block \as{io} layer, we include methods and mechanisms in the storage software stack particularly devised for file systems and flash storage integration. As a result, this \as{srq} evaluates how the depicted challenges are addressed throughout the various software stack layers, up to the file system.
    \item [\textbf{SRQ3.}] \textbf{How will NAND flash storage and newly introduced NAND flash-based storage devices and interfaces affect future file system design and development?} \\
        With a particular goal of this literature study being to evaluate the validity of data structures, algorithms, and mechanisms of flash, and understanding the applicability to \as{zns}, a newly arising storage technology, this research question furthermore aims at evaluating future challenges that may arise from new technology.
\end{itemize}

\section{Cross-Lingual Diffusion Language Model}
\label{sec:XDLMusion}

% In this section, we present our proposed language modeling objectives designed specifically for diffusion and the diffusion model applied for cross-lingual translation. These objectives cater to both monolingual and multilingual data, and they are situated within the diffusion model framework for facilitating cross-lingual translation.

In this section, we present the Cross-lingual Diffusion Language Model (XDLM), which incorporates a pre-training phase on cross-lingual data, utilizing diffusion techniques for the purpose of non-autoregressive machine translation, and a fine-tuning phase generating corresponding text from one language to another language based on the pre-trained model.

% \subsection{Preliminary}
% \subsubsection{Cross-lingual translation}
% (\irene{combine 3.1.1 and 3.1.2 as NAR machine translation, and, there is no such term called \textit{Cross-lingual translation}, all translation is cross-lingual, it should be either \textit{machine translation} or \textit{cross-lingual language model}})

% Cross-lingual translation typically involves generating an output sequence $Y=\{y_1, y_2,…, y_{|Y|}\}$ from a given input sequence $X=\{x_1,x_2,…,x_{|X|}\}$, with each sequence being in a different language. Three common generative paradigms exist for cross-lingual translation: AutoRegressive (AR) generation, Non-AutoRegressive (NAR) generation, and semi-NAR generation. Ordinarily, diffusion models employ the NAR approach for generation tasks.

% \subsubsection{Non-AutoRegressive(NAR) generation}
% The NAR generation follows the conditional probality: 
% $$
% p_{\theta}(Y|X)=\prod_{i=1}^{|Y|} p_{\theta}(y_i|X)
% $$

% Unlike AutoRegressive (AR) generation, all tokens $y_i$$(0\leq i \leq |Y|)$ in the generated sequence Y are predicted concurrently. The generation solely depends on the input sequence X, without any dependency on preceding tokens. This attribute presents a challenge in determining the length of the generated sequence. To address this issue, the prediction of the output sequence is introduced as an auxiliary task \cite{gu2017non}.

\textbf{Non-AutoRegressive (NAR) Machine Translation}
In machine translation, given the input sequence from a source language $X=\{x_1,x_2,…,x_{|X|}\}$, the task is to generate the output sequence of the translation in the target language $Y=\{y_1, y_2,…, y_{|Y|}\}$. In this work, we focus on the Non-AutoRegressive (NAR) translation setting with the diffusion model. Typically, it has the following conditional probability:  
$$
p_{\theta}(Y|X)=\prod_{i=1}^{|Y|} p_{\theta}(y_i|X).
$$

Unlike AutoRegressive (AR) text generation, all tokens $y_i$$(0\leq i \leq |Y|)$ in the generated sequence $Y$ are predicted concurrently. The generation solely depends on the input sequence $X$, without any dependency on preceding tokens. This attribute presents a challenge in determining the length of the generated sequence. To address this issue, the length prediction of the output sequence is introduced as an auxiliary task \cite{gu2017non}. And the training loss is defined as a weighted sum between the translation loss and the length prediction loss.

\textbf{Diffusion Models}
The Denoising Diffusion Probabilistic Model (DDPM) \cite{ho2020denoising} is a parametrized Markov chain, and it is trained using variational inference to generate samples that match the original input data. 
% a diffusion process for generative tasks was introduced by \cite{ho2020denoising}, yielding impressive results.
The diffusion process comprises a noise-adding forward process and a noise-removing backward process, both of which can be viewed as discrete-time Markov processes. During the forward process, the model gradually introduces random noise with different scheduled variance $\beta_1,...,\beta_t$, with the aim of generating a standard Gaussian noise $x_t$ after $t$ turns. This can be formalized as follows:
$$
q(x_{t+1}|x_t)=\mathcal{N}(x_{t+1};\sqrt{1-\beta_{t+1}}x_t,\beta_{t}\mathbf{I}).
$$

The backward process, the reverse of the forward process, attempts to reconstruct the target sequence from the standard noise. Like the forward process, this procedure is also applied incrementally and can be formalized as follows:

$$
    p(x_{t-1}|x_t)=\mathcal{N}(x_{t-1};\mu_{\theta}^{t-1},\sigma_{\theta}^{t-1}),
$$
$$
    \mu_{\theta}^{t-1}=\frac{1}{\sqrt{\alpha_{t}}}(x_t-\frac{\beta_{t}}{\sqrt{1-\overline(\alpha_{t})}}z_{\theta}(x_{t},t)), 
$$
$$
    \sigma_{\theta}^{{t-1}^2}=\frac{1-\overline{\alpha_{t-1}}}{1-\overline{\alpha_{t}}}\dot \beta_{t},
$$

where $\alpha_t=1-\beta_t, \overline{\alpha_{t}}=\prod_{i=1}^t \alpha_{i}$ and $z_\theta$ comes from the prediction of model parameterized by $\theta$. 
In this work, we apply discrete diffusion for text generating and cross-lingual translation. Based on \citet{zheng2023reparameterized}, we follow the proposed discrete diffusion model with the following routing mechanism.

$
    x_{t-1}, v_{t-1} \sim q(x_{t-1},v_{t-1}|x_t,x_0) \\
    q(v_{t-1}|x_t,x_0)=q(v_{t-1})=Bernoulli(\lambda) \\
    q(x_{t-1}|v_{t-1},x_t,x_0)= \\
    v_{t-1}x_t+(1-v^{(1)}_{t-1})q_{noise}, \quad if \quad x_t = x_0 \\
    v_{t-1}x_0+(1-v_{t-1}^{(2)})q_{noise} (x_t), \quad if \quad x_t \neq x_0 \\
$


Which models the joint distribution over both $x$ and $v$. The sampling process here also takes the reparameterized method, which improves flexibility and expressiveness compared to the original process.

% Figure environment removed
\textbf{Translation Diffusion Language Modeling (TDLM)}
% Contrary to previous language modeling objectives for diffusion models, which primarily focus on monolingual data and neglect the potential to harness cross-lingual modeling capabilities from parallel datasets, we propose a pretraining process for parallel language pairs along with a corresponding modeling objective.
Unlike previous diffusion model objectives for language modeling that primarily concentrate on monolingual data, we target to exploit cross-lingual modeling capabilities from parallel datasets. Consequently, we propose a pretraining process named Translation Diffusion Language Modeling (TDLM), aiming at enhancing cross-lingual pretraining with diffusion models. As illustrated in Figure 1, we first concatenate both source and target sentences and generate the corresponding language and position embedding sequences, and then stack them as the input to a diffusion model. 
% we select both source and target sentences, generate their corresponding language and position embedding series, and concatenate them to form the input text stream. 
In a similar vein to \citet{lin2023text}, we random mask 15\% of the tokens to the input as \cite{lample2019cross} designed, tasking the model with predicting the noise and its surrounding text based on the cross-lingual context. This denoising setting assists the model in grasping the cross-lingual context.



% TODO: merge the prior background section from the survey?
\vspacebeforesection
\section{Background}
\label{sec:background}

In this section, we provide the necessary background information to ensure a comprehensive understanding of the attack described in this paper. We start with a description of the Distributed Hash Table (DHT) used by IPFS, followed by its content resolution mechanisms. We also detail techniques for network size estimation, necessary for our attack detection and mitigation mechanisms.

\vspacebeforesection
\subsection{IPFS DHT}
\label{sec:kad_dht}

We review the features of the Kademlia DHT~\cite{maymounkov2002kademlia} and its \texttt{libp2p} implementation~\cite{libp2p_github} that are the most relevant to our attack.
To participate in the DHT, each peer generates a public/private key pair and derives an identity $\peerid \in \{0,1\}^{256}$ as the hash of its public key.
Ideally, each peer generates a random key pair and, therefore, peer IDs are distributed uniformly and independently over the space $\{0,1\}^{256}$.
While honest nodes follow this rule, malicious nodes may generate and choose from an arbitrary number of key pairs.
Each peer maintains a routing table consisting of $m=256$ buckets.
The $i$-th bucket contains the addresses of up to $k=20$ peers whose peer IDs share a common prefix of exactly $i$ bits with the peer's own peer ID. 

%
A new participant node joins the IPFS network by contacting one of the hardcoded bootstrap nodes. This bootstrap node provides the new node with some initial peers allowing it to join the DHT. The new node uses this information to perform a walk through the DHT towards its own peer ID.
The walk allows to: \textit{(i)}~make sure that there is no other node in the network with the same ID; \textit{(ii)}~discover new peers and fill the newcomer's DHT routing table. At the same time, the newcomer establishes \bitswap~\cite{de2021accelerating} connections to a subset of encountered peers (usually around 300 of them). The core role of the \bitswap protocol is to enable bilateral content transfer and to play the role of a cache for recently-accessed content.

The main DHT operation $\Call{GetClosestPeers}{\key}$ returns the $k=20$ closest peers to $\key$. 
%
In Kademlia, the distance between two keys $x$ and $y$ in the key space is given by $x \oplus y \in \{0,...,2^{256}-1\}$, where $\oplus$ denotes the bitwise XOR operation on the keys; the resulting binary string is interpreted as an integer.
%
When a client wants to find the peers with IDs closest to $\key$, it sends a request to the $\alpha=3$ peers in its routing table whose peer IDs are closest to $\key$. Each of these peers returns the $k$ closest peers to $\key$ in its own routing table and the addresses of these peers. 
%
The client again sends a request to the $\alpha$ peers closest to $\key$, among peers in its routing table and those whose addresses it just received. This process repeats until the client does not find any more peers closer to $\key$.
Due to network churn and imperfect routing tables, we observed in our experiments that successive calls to $\Call{GetClosestPeers}{\key}$ do not always return the same set of $k=20$ peers (we provide more details in \Cref{sec:evaluation}, \Cref{fig:20closest}). This is an important limitation affecting our attack.

\vspacebeforesection
\subsection{Content Resolution in IPFS}
\label{sec:ipfs}

IPFS is a content-centric network.
It allows its participant to request files without specifying their location. 
%
Content is indexed by content IDs $\cid \in \{0,1\}^{256}$ that are derived from a hash of that content.
Both peer IDs and CIDs are used as keys in the DHT.
Each node can play the role of a \provider, \downloader, or \resolver. 
The process of content advertisement and resolution is illustrated in \Cref{fig:add_get_provider}.

%
When a \provider wishes to publish content with a given $\cid$ on IPFS, it creates a \emph{provider record} that contains $cid$ and the \provider's address.
During a $\Call{Provide}{\cid}$ operation, the \provider first uses $\Call{GetClosestPeers}{\cid}$ to locate the $k=20$ peers with their peer IDs closest to $\cid$, 
%
and then sends them a $\mathsf{PutProvider}$ message including the provider record (\Cref{fig:add_get_provider}(a)).
We call the peers that hold provider records for $\cid$ the \emph{resolvers} for $\cid$.

Each CID can have several \providers. In fact, by default, each IPFS client becomes a provider for each piece of content it downloads for a fixed amount of time (12h, 24h, or 48h depending on the client version or custom configuration). As a result, the system provides an auto-scaling feature with supply automatically rising with demand.

%
When a \downloader wishes to fetch a piece of content, it first sends a request to all its \bitswap peers. If none of them has the content, the \downloader uses the DHT-based resolution system. We stress that the \bitswap protocol plays the supporting role of a cache in the dissemination of popular files. However, the mechanism does not provide reliable content resolution, in particular for new or less popular content. %

When \bitswap unstructured search fails, the \downloader resolves $\cid$ using $\Call{FindProviders}{\cid}$. This operation uses a DHT walk identical to that of $\Call{GetClosestPeers}{\cid}$ to find $k$ \resolvers but also queries encountered nodes for a provider record for $\cid$ (\Cref{fig:add_get_provider}(b)). The process terminates when either 20 \providers have been found, or all \resolvers have been asked. Querying all encountered nodes (\ie, not only the designated \resolvers) is useful because some of the encountered nodes may have a provider record in their cache.
%

Upon receiving a provider record, the client connects to the address specified in the provider record to retrieve the actual content (\Cref{fig:add_get_provider}(c)).
Provider records are not authenticated, and therefore malicious \providers may respond with incorrect provider records (or may not respond at all). However, the integrity of the content is preserved because the hash of the retrieved content can be verified against its $\cid$.
%


%

\input{img/add_get_provider.tex}

\vspacebeforesection
\subsection{Network Size Estimator}
\label{sec:netsize}

The number of nodes in a decentralized system is generally unknown due to the avoidance of centralized membership management.
This number is nonetheless useful for optimizations, deciding on individual node configurations, or security mechanisms.
Various methods were proposed for the decentralized estimation of unstructured and structured networks~\cite{eli-sohl-dht-size-estimation,kostoulas2005decentralized, manku2003symphony}.
We use in this work a mechanism developed initially by Protocol Labs as part of a mechanism for decreasing the latency of publishing content in IPFS~\cite{network-size-estimation-notion,network-size-estimation-github-pr}.

%
%
%
%
%
%
%
%
%
%

Each node in the DHT refreshes its routing table periodically (every $10$ minutes in \texttt{libp2p}). 
For this, the node samples $m$ random keys (one for each bucket of its routing table)
%
and queries the DHT to obtain the $k=20$ closest peer IDs to each key.
Using these, the node then computes the average distance between each one of these keys $\key_j$ for $j=1,\dots,m$ and their $i$-th closest peer ID for $i=1,...,k$ (with $m=256$ and $k=20$).
\begin{equation}
    \label{equ:avg-dist}
    \overline{D}_i = \frac{1}{m} \sum_{j=1}^m \operatorname{dist}(\key_j, \peerid_{j}^{(i)})
\end{equation}
where $\peerid_{j}^{(i)}$ is the $i$-th closest peer ID to $\key_j$.
With $N$ peers in the DHT and peer IDs uniformly distributed in the hash space, the expected distance between a $\key$ and its $i$-th closest peer ID is $\frac{2^{256}i}{N+1}$. The node then runs a least square regression to compute the value of $N$ for which the expected distances best fit the empirical average distances, \ie,
\begin{equation}
    \label{equ:netsize-least-squares}
    \hat{N} = \arg\min_{N} \sum_{i=1}^k \left(\overline{D}_i - \frac{2^{256}i}{N+1}\right)^2.
\end{equation}
The resulting estimate $\hat{N}$ can be computed in closed form.
%

When a node starts running, it must perform DHT queries for a few random keys to initialize its network size estimate. 
Since a larger number of queries will result in higher accuracy, making more queries than what is needed to initialize one's routing table is recommended.
Thereafter, keeping the estimate up-to-date does not require any excess DHT queries beyond what is already used for refreshing the routing table as this is done frequently (every 10 minutes).

While the network size estimate has a stochastic variance resulting from the probability distribution of the honest peer IDs, it is hard for an attacker to bias the estimate significantly. Since the estimator uses the density of peer IDs around keys chosen uniformly at random, the adversary would require numerous Sybil nodes (on the order of the whole network size) to significantly affect the peer ID density around those keys.

\section{Flash Storage Integrations}
As there are various methods for integrating flash into storage devices, in addition building full \as{ssd} devices, ranging from directly attaching the flash chip to the motherboard, as is common with embedded mobile and \as{iot} devices, or custom integrations of flash chips, we evaluate the file systems based on their level of integration. Given that a different integration exposes a different interface, the possibility to enhance particular operations is highly dependent on the integration.

Therefore, throughout this literature study, we divide the relevant work based on the type of flash integration. \cref{fig:flash_integrations} shows three integration levels for flash, where \cref{fig:flash_integration_ssd} depicts the conventional integration with a \as{ssd}. \cref{fig:flash_integration_custom} shows a custom integration of flash storage for devices such as \af{ocssd}~\cite{2019-Lu-Sync_IO_OCSSD,bjorling2017lightnvm}, multi-stream SSD~\cite{bhimani2017enhancing,kang2014multi}, and \af{sdf}~\cite{2014-Ouyang-SDF}. The main benefit of these types of integration is that the flash characteristics are no longer hidden behind the device, giving the host an increasing level of storage control. \as{ocssd} is a type of \as{ssd} that exposes the device geometry to the host, allowing the host to manage device parallelism and allocation. While such a device allows increased data management for the host software, it comes at increased complexity for managing the device constraints. 

Lastly, \cref{fig:flash_integration_embedded} shows flash integration at the embedded level, such as is commonly used in
mobile devices and \as{iot} devices. In embedded flash configuration the flash chip is commonly directly attached to the
motherboard, giving the host system full control over the underlying flash storage. Throughout this literature study, we group file system design and mechanisms based on these three integration levels, as different levels of integration allow different degrees of flash management and ranging possibility for flash integration.

% Over time the device contains an increasing number of invalid pages within blocks, requiring to run \af{gc} to read out still valid pages in the block, move these pages to a free block on the SSD, and erase the original block. An additional requirement for performing \as{gc} is the need for extra space, such that the \as{ftl} can move valid pages to a free space before erasing the block. This extra space is called the \af{ops}, and typically ranges between 10-28\% of the device capacity. Due to the limited life of flash cells~\cite{mohan2010learned,2016-Schroeder-Flash_Reliability}, the FTL has to additionally ensure even wear across the device, such that no particular region of the flash storage is burnt out quicker than other regions. This process is referred to as \af{wl}. 

% Figure environment removed



\section{Challenges of Flash Storage Integration}\label{sec:flash_challenges} 
While \as{ssd} uses the same block interface that is used with \as{hdd}, flash has different characteristics that software must account for to better integrate flash storage. This section details the challenges that arise from integrating flash storage into systems, providing the guidelines along which we dictate the bottom up view of changes in the host storage software stack, up to the file system. We define the challenges to account for the characteristics of flash storage, as well as enhance its integration into host systems. In the case of \as{ssd} devices, these challenges often largely depend on the underlying \as{ftl}, as it is making the final decision, independent of what data placement the host implements, however aiding the \as{ftl} can increase the performance. Embedded devices provide a higher level of host data placement by eliminating the \as{ftl} and directly attaching flash chips to the motherboard. Each challenge is assigned a specific identifier with the \af{fic}, in order to refer back to the specific challenge throughout this literature review. \cref{tab:flash-challenges} summarizes the 6 key challenges arising from flash storage devices.

\begin{table*}[!t]
    \centering
    \begin{tabular}{||p{1cm}|p{35mm}|p{95mm}||}
        \hline 
        ID & Flash Integration Challenge & Description \\
        \hline
        \hline
        \textbf{\as{fic} 1} & Asymmetric Read and Write Performance & Write operations require more time than read operations~\cite{stoica2009evaluating,2022-intel-p,chen2009understanding,2021-osdi-zns+,2017-parity-stream}\\
        \hline
        \textbf{\as{fic} 2} & Garbage Collection & The lack of in-place updates results in flash storage running garbage collection to free space and clear invalid pages. \\
        \hline
        \textbf{\as{fic} 3} & I/O Amplification & The lack of in-place and the required garbage collection introduce write amplification, writing more flash pages than the size of the I/O issued by the host. \\
        \hline
        \textbf{\as{fic} 4} & Flash Parallelism & The architecture of flash utilizes a high degree of parallelism (channels/chips/planes) to be utilized to enhance performance. \\
        \hline
        \textbf{\as{fic} 5} & Wear Leveling & Limited lifetime of flash cells requires careful consideration during writes to ensure flash is worn out evenly across the storage space. \\
        \hline
        \textbf{\as{fic} 6} & I/O Management & Optimizations on the I/O requests, such as merging, aims at leveraging the flash storage capabilities and reducing I/O latency. \\
        \hline
    \end{tabular}
    \caption{Overview of the challenges arising from integrating flash storage. The identifier corresponds to the respective \af{fic} referred to throughout this literature study.}
    \label{tab:flash-challenges}
\end{table*}

\noindent \textbf{\as{fic}1: Asymmetric Read and Write Performance.} On flash storage write operations require more time than read operations~\cite{stoica2009evaluating,2022-intel-p,chen2009understanding,2017-parity-stream,2021-osdi-zns+}, making it important for software to limit write operations. Particularly, frequent small writes that are smaller than the allocation unit, referred to as \textit{microwrites} incur significant performance penalties, and should be avoided where possible. Similarly, methods for enhancing the write performance are important to account for the lower write performance, compared to read performance.

\noindent \textbf{\as{fic}2: Garbage Collection (GC).} While the \as{ftl} hides the flash access constraints from host applications, providing seemingly in-place data updates, it adds the cost of performing garbage collection to free up space. \as{gc} overheads have unpredictable performance penalties for the host system~\cite{2015-Kim-SLO_complying_ssds,2014-yang-dont_stack_log_on_log}, resulting in large tail latency~\cite{2013-Dean-tail_at_scale}. Dealing with, and aiming to minimize required garbage collection for the flash device is a key challenge in integrating flash storage.

\noindent \textbf{\as{fic}3: \as{io} Amplification.} Due to the characteristics of flash avoiding in-place updates of flash pages, writes often encounter \af{wa}. With this the amount of data that is written on the flash storage is larger than the write that is issued by the host system. For example a 4KiB issued write may increase to 16KiB being written on the device, due to possible garbage collection requiring to copy data, resulting in a \as{wa} factor of 4x. \as{wa} furthermore adds to an increase in wear on the flash cells~\cite{2013-Lu-SSD_WA_Lifetime}. \af{ra} similarly is caused by requiring to read a larger amount of data than is issued in the read \as{io} request. \as{ra} most commonly happens when reading metadata in order to locate data, thus requiring an additional read of metadata on top of the request read \as{io} request for the data. This is most often inevitable, as all data requires metadata for management, however this should be kept to a minimum at application-level. Furthermore, minimization of \as{wa} is more important than \as{ra}, since write requests have a higher latency than read requests, and writing has a more significant impact on the flash storage, resulting in increased flash wear. While read requests also incur wear on the flash cell, called read disturbance~\cite{2015-Liu-Read_leveling}, it is not as significant as for write requests. 

% \textbf{\as{fic}2.} \textit{Auxiliary Write Amplification (AWA):} Similar to on-device write amplification, applications can also encounter write amplification, which is referred to as auxiliary write amplification~\cite{2014-Leonardo-NVMKV}. An example of this happens when the application is writing more data than was intended, when for instance required to run application-level garbage collection.

% \textbf{\as{fic}4.} \textit{Auxiliary Read Amplification (ARA):} Similar to read amplification and auxiliary write amplification, when an application is required to read more data than the read size issued. Such an amplification is not caused by the FTL but rather represents application introduced amplification. An example of this is an application required to issue more read requests to locate data, in for instance a file system required to read inodes prior to locating valid data blocks. Therefore, a 4KiB read for a data block requires multiple inode (data directory and file inode) to be located.

\noindent \textbf{\as{fic}4: Flash Parallelism.} With the various possible levels of parallelism on flash storage devices (discussed in \cref{sec:ssd-performance}), exploiting of the various possibilities requires software design consideration to aligning with these. Although the \as{io} scheduling of on-device parallelism, such for channel-level parallelism, is responsibility of the \as{ftl} (on devices at \as{ssd} integration level), the \as{ftl} implements particular parallelism, given that the host \as{io} requests aligning with the possibility of parallelizing the request, such as with large enough \as{io}s to stripe across channels and dies. Embedded device and custom flash integrations have more possibility to manage flash device parallelism at the host software level.

\noindent \textbf{\as{fic}5: Wear Leveling (WL).} Given limited program/erase cycles for flash cells, even wear over the entire device is required to ensures that no specific areas of the device are burnt out faster than others. Similar to flash parallelism, this largely depends on the flash integration level, as the \as{ftl} at the \as{ssd} integration level ensures \as{wl}, however embedded flash integration and custom flash integration is required to place more significance on ensuring even wear across the flash cells. Strongly related to prior flash integration challenges, wear is commonly a result of \as{gc}, which in turn increases the \as{io} amplification, and particularly the  \as{wa} and \as{ra}~\cite{2014-Desnoyers-Analytic_Models_SSD,2009-Hu-WA_SSD}. 

% Figure environment removed

\noindent \textbf{\as{fic}6: \as{io} Management.} As \as{ssd} ships with integrated firmware to expose the flash storage as a block addressable storage device, integration into the current software stack is seamless. \cref{fig:storage_stack} shows the integration of a flash \as{ssd} into the Linux kernel storage stack. Since flash storage devices are significantly faster than prior storage technologies, such as \as{hdd}, the storage software stack becomes the dominating factor in \as{io} latency~\cite{2010-Caulfield-Moneta,2012-Caulfield-Fast_User_Space_Access}. One particular optimization for performance of \as{io} requests to flash storage devices is provided by an \as{io} scheduler, deciding when to issue \as{io} requests to the storage device. As is visible in \cref{fig:storage_stack}, the block \as{io} layer implements various schedulers with different functionality. Providing a ranging degree of optimizations for \as{io} requests, such as varying scheduling policies and merging of \as{io} requests, or possible reordering, specific configurations are favorable to increase performance with flash storage. Particularly the utilization of multiple queues, with multiple software and hardware dispatch queues (visible in the \texttt{blk-mq} configuration of the block \as{io} layer), allows better exploitation of flash storage capabilities, and avoids certain Linux kernel overheads. Furthermore, evaluating mechanisms that reduce the latency of \as{io} operations, and particularly write \as{io} operations.

\subsection{Flash Integration Organization}
\begin{table}[!t]
    \centering
    \begin{tabular}{||p{35mm}|p{38mm}||}
        \hline 
        Integration Level & File Systems \\
        \hline
        \hline
        \as{ssd} Flash Integration & \cite{2015-Changman-f2fs,2021-Gwak-SCJ,2012-Min-SFS,2014-Lu-ReconFS,2020-Tu-URFS,2019-Yoshimura-EvFS,2022-Jiao-BetrFS,jannen2015betrfs,huang2013improve,2019-Liu-fs_as_process,2018-Kannan-DevFS,2021-Liao-Max,2015-Kang-SpanFS,2022-Oh-exF2FS,rodeh2013btrfs} \\
        \hline
        Custom Flash Integration & \cite{kawaguchi1995flash,josephson2011direct,2020-Wu-DualFS,2018-Rho-Fstream,dubeyko2019ssdfs,lee2014refactored,2019-Lu-Sync_IO_OCSSD,2016-Lee-AMF,2018-Yoo-OrcFS,2016-Zhang-ParaFS,2021-Qin-Atomic_Writes} \\
        \hline
        Embedded Flash \newline Integration & \cite{2006-Lim-NAND_fs,hunter2008brief,2009-Sungjin-FlexFS,2008-Jung-ScaleFFS,woodhouse2001jffs,park2013enffis,nahill2015flogfs,schildt2012contiki,2015-Park-Suspend_Aware_cleaning-F2FS,2009-Zuck-NANDFS,2009-Park-Multimedia_NAND_Fs,2007-Hyun-LeCramFS,2021-Ji-F2FS_compression,2011-Park-Multi_NAND,manning2010yaffs,aleph2001yaffs,manning2002yaffs,engel2005logfs,2008-Kim-DFFS,park2006flash,2020-Yang-F2FS_Framentation,2022-Lee-F2FS_Address_Remapping,2011-Lim-DeFFS,2014-Hyunchan-O1FS,2022-Zhang-ELOFS,2020-Zhang-LOFFS,kim2006mnfs,2009-Tsiftes-Coffee_FS} \\
        \hline
    \end{tabular}
    \caption{Classification on the flash integration level utilised for the file systems evaluated in this literature study.}
    \label{tab:integration_levels}
\end{table}

With the different possible levels for integration of flash storage (recall \cref{fig:flash_integrations}), and while mechanisms for solving flash integration challenges are frequently applicable at various integration levels, several of the mechanisms we present require deeper integration of flash storage, levering increased control, in order to be implemented. For instance, the incorporation of the various levels of on-device parallelism is not directly possible at the \as{ssd} integration level, as the \as{ftl} hides the parallelism on the physical device from the host system. The custom and embedded level flash integration provide the host with more possibility to manage these. In order to separate the possibility of mechanisms to be implemented with a particular flash integration level, \cref{tab:integration_levels} provides a classification of each evaluated file system in this literature study to the respective integration level. During the evaluation we present the different mechanisms to solve a \as{fic}, and indicate which file systems utilize these. Therefore, when considering the feasibility of a mechanism for a particular flash integration consult this table to see its applicability. Note that file systems are not limited to the classification we provide, as for instance file systems designed for \as{ssd} flash integration also work on some embedded flash integration. However, we utilize only a single classification for each file system to avoid confusion. Exceptions are made only in specific cases where an existing file system is adapted for a different flash integration level.

We divide the discussion of mechanism by the \as{fic} for which the evaluated study presents a novel solution. This implies that mechanisms that solve multiple \as{fic} are discussed in detail in the first section they appear in, however are also mentioned in all latter sections for the \as{fic} that the mechanism solves. Therefore, each \as{fic} section contains a table of the respective mechanisms presented to solve that particular \as{fic}, along with a reference to the corresponding section of its detailed discussion.


% % Core sections for flash integration challenges
\input{sections/rw_asym.tex}
\section{Garbage Collection}\label{sec:gc}
A significant challenge of flash storage is \as{gc} overheads having unpredictable performance penalties for the host system~\cite{2015-Kim-SLO_complying_ssds,2014-yang-dont_stack_log_on_log}, resulting in large tail latency~\cite{2013-Dean-tail_at_scale}. Dealing with, and aiming to minimize required garbage collection for the flash device is a key challenge in integrating flash storage. Naturally, as flash storage does not provide in-place updates, data is written in a log-based fashion, sequentially in the flash blocks. Therefore, over time as data is overwritten, the blocks contain an increasing number of invalid pages that must be erased to free space. However, as the block also contains valid data, and the erase unit is a block, the still valid pages are moved to a new block, such that the old black can be erased. 

% Figure environment removed

In a \as{lfs} updates to file data are written at the head of the log, resulting in the parts of the file that are updated to be located in a different flash block than the parts of the original file data. Furthermore, \as{gc} causes file data to moved around the storage space as well, resulting in scattering of parts of files, referred to as \textit{fragmentation}. Figure~\ref{fig:fragmentation} shows the resulting fragmentation for several files that occurs over time from file updates and \as{gc}. Fragmentation results in increased read time, due to files being in non-contiguous regions, and introduces increased garbage collection overheads due to the failed grouping of data. Ji et al.~\cite{2016-Ji-Framgentation_empirical_study} show an empirical study on fragmentation in mobile devices, identifying that fragmentation introduces performance degradation due to increased \as{io} requests, and it further produces increased pressure on caching. The effects on caching are due to increased difficulty of prefetching data, since it no longer is in contiguous physical ranges, but scattered throughout the physical space. Therefore, prefetching cannot bring correct data into the caches, resulting in an increase in cache misses. 

In addition to increased \as{io} requests for reading and rising cache pressure, increased garbage collection is caused when frequently modified files or file fragments, referred to as the \textit{hot data}, are in the same block as rarely accessed files, referred to as the \textit{cold data}. When hot data is modified, its flash pages are invalidated. Once enough flash pages in a block are invalidated, it can be erased. However, if it is co-located with cold data, the cold must be copied to a free space, as it is still valid. If this cold data is then again co-located with hot data, the modifications of the hot data cause the block to be cleared during \as{gc}, which requires the cold data to be moved again. Therefore, co-locating hot and cold data in the same physical erase unit results in significant \as{gc} increase due to the unnecessary moving of the cold data.

\begin{table}[!t]
    \centering
    % Update reference format to fit table into column
    \crefformat{section}{\S#2#1#3}
    \crefformat{subsection}{\S#2#1#3}
    \crefformat{subsubsection}{\S#2#1#3}
    \begin{tabular}{||p{40mm}|p{35mm}||}
        \hline 
        Mechanism & File Systems \\
        \hline
        \hline
        Reducing Write Traffic (\cref{sec:reduce_write_traffic}) & \cite{huang2013improve,2011-Lim-DeFFS,woodhouse2001jffs,dubeyko2019ssdfs,2021-Ji-F2FS_compression,2007-Hyun-LeCramFS,ning2011design,2009-Zuck-NANDFS,2022-Lee-F2FS_Address_Remapping} \\
        \hline
        Aligning the Allocation Unit (\cref{sec:align_alloc}) & \cite{2020-Tu-URFS,2009-Park-Multimedia_NAND_Fs} \\
        \hline
        Data Grouping (\cref{sec:data_grouping}) & \cite{2006-Lim-NAND_fs,2022-Zhang-ELOFS,2020-Zhang-LOFFS,2015-Changman-f2fs,2018-Rho-Fstream} \\
        \hline
        \as{gc} Policies (\cref{sec:gc_policy}) & \cite{2012-Min-SFS,2018-Yoo-OrcFS,2021-Gwak-SCJ,2015-Park-Suspend_Aware_cleaning-F2FS} \\
        \hline
        Coordinating the Software Stack Layers (\cref{sec:software_stack}) & \cite{2020-Wu-DualFS,2016-Zhang-ParaFS,2016-Lee-AMF,lee2014refactored} \\
        \hline
    \end{tabular}
    % Reset reference format to before
    \crefformat{section}{Section #2#1#3}
    \crefformat{subsection}{Section #2#1#3}
    \crefformat{subsubsection}{Section #2#1#3}
    \caption{Mechanisms for file systems to deal with \as{gc} overheads from flash storage, and the respective file systems that implement a particular mechanism.}
    \label{tab:gc}
\end{table}

Reducing the write traffic to the storage device, an effective method to handle the asymmetric flash performance (see \cref{sec:reduce_write_traffic}), is a solution to minimize fragmentation and reduce possible future \as{gc} overheads. However, additional mechanisms are required for effective \as{gc} management. With fragmentation being a significant contribution to increased garbage collection, avoiding it is a key objective to reducing garbage collection overheads. Fragmentation is classified into three different types~\cite{2007-Sato-defrag}. Firstly, \textit{single file fragmentation}, where data in a single file is dispersed over the storage (as is shown in Figure~\ref{fig:fragmentation}). Secondly, \textit{relevant file fragmentation}, where files that are relevant to each other and should be grouped together are split over the storage, such as co-locating hot and cold data in the same erase unit. Lastly, \textit{free space fragmentation}, where the file system has a large amount of small free space, because of deletion of dispersed small files. The cause of fragmentation occurring over time is referred to as \textit{file system aging}~\cite{smith1997file}. While several tools exist that implement \textit{defragmentation}~\cite{hahn2017improving,f2fs_defrag_tool,park2021fragpicker}, additional mechanism can be utilized to avoid fragmentation and \as{gc} overheads. \cref{tab:gc} depicts the mechanisms for file systems to deal with and minimize garbage collection overheads. The process of countering the different types of fragmentation is commonly referred to as \textit{storage gardening}~\cite{2020-Kesavan-Fragmentation,2019-Kesavan-Storage_Gardening}, and for file system development various aging tools exist in order to generate real-world file system workloads and simulate file system aging~\cite{2019-Conway-FS_aging,2018-Kadekodi-Geriatrix}. 

\subsection{Aligning the Allocation Unit}\label{sec:align_alloc}
\as{gc} is a result of having to move valid blocks in the erase unit to a free space, in order erase the flash block. While data grouping allows to align the validity inside the block, such that blocks are likely to be updated within close proximity, multiple files may be co-located in the same block. Therefore, a similar method is to align the allocation unit of data blocks for a single file to the erase unit, resulting in only a single file being located in a block. Such a mechanism is implemented in URFS~\cite{2020-Tu-URFS}, which aligns the data allocation unit for large files to the flash erase unit, allowing files to be erased as a single unit, limiting required \as{gc}. However, as the erase unit of flash can be several hundreds MBs, resulting in significant over-allocation for small files, it makes such a mechanism only beneficial with large files. NAMU~\cite{2009-Park-Multimedia_NAND_Fs} similarly showcases a file system that aligns its content with the requirement of large files. Focused on the multimedia domain, where files have the particular characteristics of rarely being modified, and if removed all file data blocks are erased in one unit, \as{gc} in NAMU is done at the granularity of a file. In addition to improving on \as{gc}, the memory requirements for mapping tables are also minimized. For generic file systems that vary in file characteristics, \af{ars}~\cite{2020-Yang-F2FS_Framentation} minimizes the issue of over-allocating space by allocating in a smaller unit of 2MB (a single segment). File data is written to the space until it is exhausted, upon which a new segment is allocated. While this does not map entire files to the flash erase unit, it allows writing of file data sequentially for each segment, eliminating fragmentation to a degree. 

\subsection{Data Grouping}\label{sec:data_grouping}
A key circumvention method for fragmentation relies on grouping of related data. Most commonly this is applied in the type grouping data by its access and modification frequency into hot and cold data, however other less commonly used groupings based on \textit{death-time prediction} exist~\cite{2021-Chakraborttii-DT_LBA}. Commonly more classifications than simply hot and cold are utilized for more effective grouping. A plethora of methods for grouping data in such a way have been proposed, which we split by its application type into several groups.

\subsubsection{Data Type Grouping}
Common write patterns in storage systems follow a bi-modal distribution, where many very small write requests and a numerous very large write requests are issued~\cite{chang2008hybrid}. This stems from the fact that small changes are caused by metadata updates, which occur the most frequently, whereas large changes are file updates. Given that metadata is more likely to be updated frequently, separating metadata from data improves on required garbage collection. Specifically, since metadata is often updated even if the data is not updated, for example in scenarios where the file attributes (access time, permissions, etc.) are updated, or the file is moved. Therefore, based on the request size, the data can be classified to be metadata and be grouped accordingly. F2FS also groups based on the data type, where metadata is considered to be always hot data, which is implemented by similar file systems~\cite{2006-Lim-NAND_fs}.

Dividing of file system data is similarly applied in ELOFS~\cite{2022-Zhang-ELOFS,2020-Zhang-LOFFS}, which splits the flash storage two partitions, where a directory partition contains the data of directory entries, and a data partition contains the file system data, which is compacted with the inodes. Jung et al.~\cite{jung2010process} propose the addition of classifying data based on the \af{pid}, as a process is likely to generate similar access patterns and data types throughout its lifetime, classifying by the \as{pid} allows to indirectly infer a data type. A similar data type separation is implemented by Fstream~\cite{2018-Rho-Fstream}, for which the authors modify ext4~\cite{cao2007ext4} and xfs~\cite{sweeney1996scalability} to map different operations to different streams on a stream \as{ssd}. Ext4Stream, the modified ext4 to support streams, maps different metadata operations to different streams, including the journal writes for consistency, the inode writes, as well as different streams for the directory blocks and the bitmaps (inode and block). Furthermore, it utilizes different streams that can be created for different files and for different file extensions. The goal of such streams is to map particular files, such as LOG files for key-value stores for example to a particular stream, separating its access patterns from that of other files and file system data. Similarly, the modified XFStream utilizes different streams for the log, inodes, and specific files.

\subsubsection{Dynamic Grouping}\label{sec:dynamic_grouping}
While grouping is an effective method for minimizing \as{gc}, they rely on static definition on classification targets for the number of hot/cold degrees to classify to. Shafaei et al.~\cite{2016-Shafaei-WA_extent_temp} identify that the majority of hot/cold data grouping methods fail to account for the accuracy in the hot/cold grouping mechanism, as well as relying on an individual classification of each LBA, making the management of increasingly larger flash storage difficult. Therefore, Shafaei et al.~\cite{2016-Shafaei-WA_extent_temp} propose an extent-based temperature identification mechanism. It is based on the density stream clustering problem~\cite{jia2008grid,chen2007density,forestiero2013single,isaksson2012sostream}, which is a common approach of classification in artificial intelligence and stream processing, however has not been applied to storage before. The density-based stream clustering groups data in a one dimensional space as the data arrives, hence its applicability for stream processing. 

Applying this method to storage, the one-dimensional space is the range of \as{lba}s, and extent-based clustering splits the available space into a number of extents to group by. Initially, the entire space is a single extent and as writes occur the extent is split into smaller extents with different classifications. Over time as more writes are issued, extents are expanded and merged (merging of extents with the same classification). Such a grouping allows a more detailed grouping due to the increase in classification targets, compared to binary hot/cold grouping. However, an evaluation by Yang and Zhu~\cite{2015-Yang-algebric_WA_modeling} on a configurable garbage collection policy, where the number of hotness classification targets is evaluated, shows that various hotness classification targets can significantly increase the write amplification during garbage collection. 

\subsubsection{\as{lba} Hotness Classification}\label{sec:lba_classification}
Different to grouping data based on its type, hotness can be classified on the \as{lba} based on its access frequency. The naive approach at modeling hotness for each \as{lba} is with table-based classification model~\cite{hsieh2005efficient}. This however comes at a high overhead cost, as an entry for each LBA is needed, which becomes increasingly expensive as flash storage grows.  Therefore, a more non-trivial method is based on two-level \as{lru} classification~\cite{chang2002adaptive}, with two LRU lists. Upon an initial \as{lba} access, the \as{lba} is stored in the first list, and a subsequent access moves it to the next list, which is referred to as the \textit{hot list}. Therefore, if an \as{lba} is in the hot list, it is considered to be frequently accessed. 

A different approach is implemented by \af{mbf}~\cite{jagmohan2010write,park2011hot}, which uses bloom filters to identify if an \as{lba} is hot. Bloom filters rely on a hash function that, given an input such as the \as{lba}, provide an output which is mapped to a bit array, and sets the bit to true. Therefore, if an \as{lba} is accessed, applying the hash function sets the respective bit in the array to true, and checking if an \as{lba} is hot simply applies the hash function and checks if the bit is set. However, depending on the length of the array, multiple \as{lba}s can map to the same bit location, as proven by the pigeonhole principle~\cite{ajtai1994complexity}, resulting in false positive hotness classifications for an \as{lba}. To avoid frequent false positive classifications, \as{mbf} utilizes multiple bloom filters at the same time. With multiple bloom filters, the same amount of arrays exist, applying all bloom filter hash functions to the \as{lba}, and setting the respective bit in each of the arrays. As a result, collisions on all bloom filter hash functions are less likely, minimizing the possibility for false positives.

Similar to \as{mbf}, Kuo et al.~\cite{kuo2006configurability} present a hot data identification method by using multiple hash functions and a hash table. Upon a write, the \as{lba} is hashed by multiple hash functions, and a counter for each hash function is incremented in a hash table. To check if an \as{lba} is hot, the \as{lba} is hashed and a configured $H$ most significant bits of the resulting hash table indicate if the \as{lba} is hot if they are non-zero, as the counter is increased on accesses the most significant bits are only non-zero if the \as{lba} is frequently accessed. Multiple hash functions are used for the same reason multiple bloom filters are used in \as{mbf}, to avoid false positive classifications. Lee and Kim~\cite{2013-Lee-data_grouping_empirical_study} provide a study into comparing performance of two-level \as{lru}, \as{mbf}, and \af{dac}, which are similar to the density-based stream clustering by Shafaei et al.~\cite{2016-Shafaei-WA_extent_temp} (discussed in the prior \cref{sec:dynamic_grouping}). The authors show that on the evaluated synthetic workloads \as{dac} provides the highest reduction in write amplification factor, which in turn leads to a decrease in \as{gc} overheads.

Unlike all prior approaches basing classification on the access frequency directly to identify hotness, Chakraborttii and Litz~\cite{2021-Chakraborttii-DT_LBA} propose a temporal convolutional network that predicts the \textit{death-time} of an \as{lba}, based on modification history. This allows to more optimally group data based on the death-time of individual \as{lba}, which has been shown to be an effective grouping mechanism~\cite{2017-He-SSD-Unwritten-Contract}. Grouping related death-time LBA reduces the required garbage collection, as blocks containing \as{lba}s with similar death-times are erased together, which in turn reduces the write amplification (solving \textbf{\as{fic}3}).

\subsection{Garbage Collection Policies}\label{sec:gc_policy}
While data grouping provides benefits of co-locating data based on their update likelihood, an additional essential part of garbage collection is the policy of victim selection for segments to clean. Since during garbage collection a segment to clean is required to be selected, where all still valid data in the segment is moved to a free space, selecting a victim becomes non-trivial. With the importance of data grouping with hotness for effective \as{gc}, conventional \as{gc} policies, such as greedy and cost-benefit lack their inclusion. SFS~\cite{2012-Min-SFS} proposes the \textit{cost-hotness} policy to account for the hotness of segments instead of the segment age, better incorporating the data grouping into victim selection. Similar to the cost-benefit policy (recall \cref{sec:bg_lfs}), the cost-hotness is calculated as
\begin{equation*}
    \text{cost-hotness}=\frac{\text{free space generated}}{\text{cost}*\text{segment hotness}}=\frac{(1-u)*\text{age}}{2u*h}
\end{equation*}
where the cost considers reading and writing the valid blocks (equivalent to $2u$) with the segment hotness. A further \as{gc} policy that is based on the cost-benefit policy, is the \af{cat} policy~\cite{chiang1999cleaning}. It extends cost-benefit by including an erase count for each block, improving wear leveling (solving \textbf{\as{fic}5}).

The majority of \as{gc} policies have a fixed algorithm, limiting configuration possibility. The \textit{$d$-Choice} algorithm~\cite{van2013mean} is a configurable \as{gc} policy that combines greedy selection with random selection. The tunable parameter $d$ defines the number of blocks to be selected randomly out of the $N$ total blocks. Therefore, configuring $d=1$ results in fully random victim selection, as a single block is randomly selected from the total blocks, providing effective wear leveling through randomness~\cite{2015-Yang-algebric_WA_modeling} (solving \textbf{\as{fic}5}). Configuring $d\rightarrow \infty$ selects a larger subset of blocks to use greedy selection on, such that $d=N$ is equivalent to fully greedy victim selection, allowing to provide the lowest cleaning latency. Configuration of $d$ thus allows to define the tradeoff between wear leveling and performance. 

An evaluation by Yang and Zhu~\cite{2015-Yang-algebric_WA_modeling} of the algorithm shows the significance of the number of hotness classification targets that are utilized, and the configuration of the $d$ parameter, where various hotness classification targets can significantly increase the \as{wa} during \as{gc}. Similarly, \af{fagc}~\cite{yan2014efficient} is a \as{gc} policy that maintains an \af{uft} for each \as{lba} of a file, in order to group valid pages in the victim block based on the access frequency when these are copied to a new block during \as{gc}. This is similar to grouping hot and cold data, but at the level of \as{gc} for each \as{lba} in a file. SFS~\cite{2012-Min-SFS} implements a \as{gc} policy that accounts for data grouping, with a lower overhead of having maintaining a \as{uft} for each file. It maintains a hotness classification for each block, and combines blocks with k-means clustering~\cite{hartigan1979algorithm}, into groups with similar hotness classification. 

The \as{gc} cost of foreground cleaning in F2FS can take up to several seconds~\cite{2018-Yoo-OrcFS}, as it is not a preemptive task. Implementing preemptive scheduling in Kernel file systems is challenging, as during the preemption the Kernel has to store the file system state to continue after higher priority tasks have finished, and the state being restored when returning depicts possibly outdated data. Due to the continued writing on the file system, restoring the prior state may no longer be valid, as ongoing writes may have changed file system metadata. OrcFS~\cite{2018-Yoo-OrcFS} implements \af{qpsc}, which sets a maximum time interval $T_{max}$ (default of 100ms). After cleaning a segment it checks if the timer has expired, and if so it checks if outstanding writes are present from the host. If there are outstanding writes, the locks are released and the write is executed, and if there are no outstanding writes the next segment can be cleaned and the timer is reset. This allows any host write command to not encounter a segment cleaning overhead higher than $T_{max}$.

A further drawback of segment cleaning with \as{lfs}, in particular F2FS, is that the modification of metadata during segment cleaning requires a checkpoint to be created after each segment clean. This constitutes to a significant overhead for the segment cleaning process. To avoid the excessive checkpointing after segment cleaning, \af{scj}~\cite{2021-Gwak-SCJ} adds support to F2FS to journal metadata updates made during segment cleaning, instead of creating a checkpoint. This journal is stored in a journal area, which delays the updating of the original metadata until the journal becomes large enough or the checkpointing time interval is reached. However, metadata still points to old invalid data blocks (referred to as \textit{pre-invalid blocks}), which requires that data only be invalidated once the metadata is updated by the \as{scj}. Therefore, \as{scj} implements an adaptive checkpointing that evaluates the cost of checkpointing to flush the metadata updates and the accumulation of pre-invalid blocks, and checkpoints if its cost is lower. 

In addition to utilizing \as{gc} policies to reduce its overheads, the policy can further be used to incorporate management of fragmentation. Park el at.~\cite{2016-Park-LFS_Defragmentation} propose to use a \af{vbq} in which valid blocks are sorted during garbage collection. Typically, a victim segment in the \as{lfs} is selected for cleaning, valid data is copied to the free space at the log head, and the old segment is erased. The \as{vbq} is added such that after a victim segment is selected, it is copied into the \as{vbq}, where the blocks it contains are sorted by the \as{inode} number. Then are the valid blocks written to a new segment and the old one is erased. This sorting allows to maintain file associated blocks together based on their \as{inode} number. 

A different approach to mitigate \as{gc} overheads is to design the \as{gc} procedure such that accesses do not suffer from high tail latency when \as{gc} is running. TTFlash~\cite{2017-Yan-Tiny_tail} achieves this through several mechanisms. It implements \textit{plane-blocking GC}, which limits any resources that are blocking \as{io}s to only the affected planes on the flash. However, this leads to blocking of requests to the \as{gc} affected planes. Therefore, TTFlash implements \textit{rotating \as{gc}} that only runs at most one \as{gc} operation on a \textit{plane group}. The plane group assignment is based on the possible parallelism of the device, such as plane- and channel-level parallelism. This ensures that a plane group is never blocked for more than a single \as{gc} operation, implying that any request will not be blocked for more than one \as{gc} operation. 

On embedded devices, and in particular mobile devices, in order to save energy the device suspends all threads when not needed (e.g., when the mobile screen is turned off). This implies that all file system threads are also suspended, which means the file system cannot run the background \as{gc} during these inactive sessions. Therefore, \af{sac}~\cite{2015-Park-Suspend_Aware_cleaning-F2FS} is an addition to \as{lfs}, which is the time between the suspend initiation and the suspending of the file system threads, also referred to as the \textit{slack time}. It uses this slack time to run background \as{gc}, however it does not write any data on the flash storage, but instead selects a victim block and brings the still valid pages in the page cache and marks these as dirty. As a result, all pages in the victim block to be dirty, which allows it to be erased. This process is referred to as \textit{virtual segment cleaning}, which however is not run every time the screen is turned off, but rather based on the device utilization, which is called \textit{utilization based segment cleaning}. 

\subsection{Coordinating the Software Stack Layers}\label{sec:software_stack}
% Figure environment removed

The various layers in the storage software stack introduces several redundant operations, such as \as{gc} that is run in the \as{lfs} and \as{gc} run on the storage device. This duplicate work leads to significant performance impacts~\cite{2014-yang-dont_stack_log_on_log}, requiring a co-design of the storage device software and the file system. Qiu et al.~\cite{2013-Qiu-Codesign_FTL_FS} show that the co-design of \as{ftl} and the file system show benefits of reduced memory requirements for \as{l2p} mappings, and increased device parallelism that can be exploited with better file system knowledge of the storage characteristics. In addition to duplicated work, information about data characteristics are lost across the various, since in order to integrate communication across the layers, each utilizes an interface for the other layers to communicate with. However, the expressiveness of the interfaces limits the capability to communicate data characteristics across the layers, as \cref{fig:software_stack} illustrates. Applications have the highest knowledge of data characteristics, how it is best allocated on the flash and managed in order to reduce \as{gc}, however cannot forward all of this information to the file system. Similarly, the file system may group specific data together, however cannot forward this information to the block layer, and similarly the \as{ftl} may take the submitted \as{io} requests and organize these different on the flash storage. This \textit{semantic gap} between the storage device and storage software is a result of the device integrating into the existing block \as{io} interface~\cite{zhang2017flashkv}, however failing to represent the flash-specific characteristics. Such mismatch between storage device and its accessing interface, requires storage software to enhance capabilities to pass information across the layers to avoid increasing the semantic gap across the layers.

DualFS~\cite{2020-Wu-DualFS} utilizes the custom integration with \as{ocssd} to merge the garbage collection of the file system with that of the \as{ftl}, and present this scheme as \textit{global garbage collection}. ParaFS~\cite{2016-Zhang-ParaFS} implements a similar coordination of file system \as{gc} and \as{ftl} \as{gc}. A different approach taken by Lee at al.~\cite{2016-Lee-AMF} is to modify the block interface with the flash characteristics, moving responsibility directly to the file system, or other application built on top of it. The resulting interface called \af{amf} exposes a block interface that does not allow overwriting unless an explicit erase is issued for the blocks. This matches the flash requirement that prior to overwriting data it has to be erased. The interface is implemented as a custom \as{ftl}, called AFTL, on top of which the ALFS file system is built. This avoids the duplicate garbage collection of the \as{ftl} and the file system, as the garbage collection of the ALFS erases blocks during garbage collection, informing the \as{ftl} to erase the physical block.

Co-designing the FTL and the file system allows removing uncoordinated duplicate work, and coordinate the flash management. To this end, Lee et al.~\cite{lee2014refactored} present a redesigned \as{io} architecture, called REDO, which avoids the duplicate operations from file system and \as{ftl} by implemented the new framework directly as the storage controller, and building the \af{rfs} on top of the new controller interface. By combining the file system operations with the storage controller, the file system is responsible for running \as{gc} and managing the storage by maintaining the \as{l2p} mappings.


\section{FIC-3: \as{io} Amplification}\label{sec:io_amplification}
Several of the applied mechanisms for dealing with asymmetric flash performance and garbage collection are key mechanisms to eliminate \as{io} amplification. \cref{tab:io_amplification} shows the different mechanisms that can be applied to lower the various types of \as{io} amplification, including \af{wa}, \af{ra}, and \af{sa}. These include the benefit of write buffering, which in the case of write requests that are smaller than the flash allocation unit, avoids the unnecessary \as{wa} to fill the flash allocation unit. Similar buffering as is employed in \as{wods}, such as the $\text{B}^\varepsilon$-trees, allows decreasing the \as{wa}. Another key mechanism is reducing the generated write traffic, which minimizes \as{wa}, \as{ra}, and \as{sa} through deduplication, compression, delta-encoding, and virtualization. Similarly, the discussed methods of grouping data, avoiding fragmentation, allows reducing the \as{ra} to locate data, and furthermore limits \as{gc} overheads and \as{gc} caused \as{wa}. Throughout this section we evaluate the additional methods for limiting the various types of \as{io} amplification.

\begin{table}[!t]
    \centering
    % Update reference format to fit table into column
    \crefformat{section}{\S#2#1#3}
    \crefformat{subsection}{\S#2#1#3}
    \crefformat{subsubsection}{\S#2#1#3}
    \begin{tabular}{||p{30mm}|p{20mm}|p{20mm}||}
        \hline 
        Mechanism & Amplification Type & File Systems \\
        \hline
        \cellcolor{lightgreen!75} Write Optimized Data Structures (\cref{sec:wods}) & \cellcolor{lightgreen!75}\as{wa} & \cellcolor{lightgreen!75}\cite{2020-Tu-URFS,hunter2008brief,rodeh2013btrfs,2022-Jiao-BetrFS,dubeyko2019ssdfs} \\
        \hline
        \cellcolor{lightgreen!75} Write Buffering (\cref{sec:write_buffering}) & \cellcolor{lightgreen!75}\as{wa} & \cellcolor{lightgreen!75}\cite{park2006cflru,jo2006fab,park2013enffis,2018-Kannan-DevFS,josephson2011direct,2019-Lu-Sync_IO_OCSSD,2010-Josephson-DFS,2011-Park-Multi_NAND} \\
        \hline
        \cellcolor{lightgreen!75} Reducing Write Traffic (\cref{sec:reduce_write_traffic}) & \cellcolor{lightgreen!75}\as{wa}, \as{ra}, \as{sa} & \cellcolor{lightgreen!75}\cite{huang2013improve,2011-Lim-DeFFS,woodhouse2001jffs,dubeyko2019ssdfs,2021-Ji-F2FS_compression,2007-Hyun-LeCramFS,ning2011design,2009-Zuck-NANDFS,2022-Lee-F2FS_Address_Remapping} \\
        \hline
        \cellcolor{lightgreen!75} Data Grouping (\cref{sec:data_grouping}) & \cellcolor{lightgreen!75}\as{wa}, \as{ra} & \cellcolor{lightgreen!75}\cite{2006-Lim-NAND_fs,2022-Zhang-ELOFS,2020-Zhang-LOFFS,2015-Changman-f2fs,2018-Rho-Fstream} \\
        \hline
        \cellcolor{lightgreen!75} \as{gc} Policies (\cref{sec:gc_policy}) & \cellcolor{lightgreen!75}\as{wa}, \as{ra} & \cellcolor{lightgreen!75}\cite{2012-Min-SFS,2018-Yoo-OrcFS,2021-Gwak-SCJ,2015-Park-Suspend_Aware_cleaning-F2FS} \\
        \hline
        Space Optimized Data Structure (\cref{sec:space_optimized_ds}) & SA & \cite{2014-Lu-ReconFS,2015-Changman-f2fs,2021-Liao-Max} \\
        \hline
        \as{wa} with Coarse Granularity Flash Mappings (\cref{sec:coarse_gran}) & \as{wa} & \cite{2018-Yoo-OrcFS,kim2006mnfs,2018-Yoo-OrcFS,lee2007log} \\
        \hline
        Reverse Indexing (\cref{sec:reverse_indexing}) & \as{wa} & \cite{2014-Lu-ReconFS}\\
        \hline
    \end{tabular}
    % Reset reference format to before
    \crefformat{section}{Section #2#1#3}
    \crefformat{subsection}{Section #2#1#3}
    \crefformat{subsubsection}{Section #2#1#3}
    \caption{Mechanisms for file systems to deal with \as{io} amplification caused by flash storage integration, and the respective file systems that implement a particular mechanism. Green highlighted table cells depict previously discussed mechanisms with their respective section.}
    \label{tab:io_amplification}
\end{table}

\subsection{Space Optimized Data Structures}\label{sec:space_optimized_ds}
Similar to the design of a \as{wods}, data structures with particular focus on optimally utilizing available space are a mechanism to deal with \as{sa}. A commonly applied method for file systems to optimize space utilization is to possibly embed file data in the \as{inode}. Commonly the allocation of the file system \as{inode} occupies at least a block, such as 4KiB in F2FS, which however is more space than file metadata requires. Therefore, several bytes ($\sim$3.4KB in F2FS) are free, which are used to inline file data in the inode. This particularly allows for small files to entirely fit into the \as{inode}, avoiding writing the \as{inode} and leave the unused space empty, and additionally write an additional data block, which also has free space. Different to inline data, ReconFS~\cite{2014-Lu-ReconFS} uses an \as{inode} with a size of 128B, allowing to place numerous inodes in a single flash page. With such an \as{inode} size, writing each \as{inode} change directly requires filling the flash page with unnecessary data. Therefore, ReconFS implements a \textit{metadata persistent log}, in which metadata changes are logged and compacted to align with pages, and are only written back to the storage when evicted or checkpointed, in order for the file system to remain consistent.

Similar to effective tree-based \as{wods}, the radix tree is a space optimized tree variant, that is commonly used as the directory and inode tree for file systems~\cite{2015-Changman-f2fs,2021-Liao-Max}. The directory tree is commonly constructed and maintained in memory and written to the persistent flash storage. Its space optimization revolves around merging of nodes that have a single child with that child node. This eliminates the need for an individual node that is assigned to each child, lowering the space requirement. As a result, the radix tree is also referred to as a compressed tree, due to the compression of single child nodes.

\subsection{Coarse Granularity Flash Mappings}\label{sec:coarse_gran}
A similar goal of file systems is to reduce the amount of memory that is required for the mapping table to maintain the \as{l2p} mappings. \as{l2p} mappings are commonly persisted periodically, from the storage device \as{ram} to the flash storage, such that in the case of system shutdown or the device is unplugged, upon reconnection the mapping information can be recovered. Hence, the mapping information similarly requires flash pages to be stored. A common solution is to increase the granularity of the mapping table (e.g., block-level mapping instead of page-level mapping), requiring fewer mappings. MNFS~\cite{kim2006mnfs} manages flash storage with page-level and block-level mapping, depending on the update frequency. Metadata is updated more frequently and therefore utilizes a page-level mapping compared to larger mapping granularity for data. OrcFS~\cite{2018-Yoo-OrcFS} similarly utilizes a page-level for metadata, and a superblock-level mapping for data, which represents several flash blocks. The allocation unit is called a superblock as it consists of multiple blocks (not to be confused with the file system superblock). Furthermore, logical addresses are mapped to the same physical addresses in the data partition, requiring no mapping table, and file system sections are aligned to the superblock unit. Therefore, OrcFS only requires a block allocation information for each file in the superblock, which are stored in the inode block in the metadata area. 

However, this comes at the cost of having a larger allocation unit, and if a host write is smaller than the allocation unit it causes \as{wa}, due to the partial flash page write when the flash page size and the allocation unit are not aligned~\cite{2018-Yoo-OrcFS,lee2007log}. OrcFS~\cite{2018-Yoo-OrcFS,lee2007log} implements \textit{block patching} to solve this issue. It takes write requests that are smaller than the flash page size and pads the remaining space with dummy data to align the write request to flash page size. This mechanism avoids copying data if a flash page is partially written, and the next \as{lba} in the same flash page is written, which triggers a copy of all \as{lba}s in the flash page followed by writing the \as{lba} for the new write. For instance, for a flash page containing 4 \as{lba}s, if \as{lba}s 1-3 are written by one request, the first 3 \as{lba}s are mapped to the data and the fourth holds dummy data, such that the page is fully filled. If a second request to \as{lba} 4 is issued, it cannot fill the flash page as it has already been written. Therefore, to write the newly written data after the already written \as{lba}s, it must copy \as{lba}s 1-3, append the new write to \as{lba} 4, and write the 4 \as{lba}s to a new flash page. Adding of dummy data to fill pages reduces the \as{wa}, which would be caused by copying of all data in the flash page, as it now avoids copying the added dummy data on consecutive writes. While reduction in \as{wa} are presented, the adding of dummy data nonetheless adds \as{wa} to fill the flash page. However, as latter updates require less data written, and the importance of data grouping indicates, maintaining related data in the same flash page is more beneficial and possibly decreases future \as{wa}. Related data remains in the same flash page, as only the valid data in flash pages is copied on writes, as opposed to copying the entire flash page, introducing copied dummy data.

\subsection{Reverse Indexing}\label{sec:reverse_indexing}
As file system metadata is commonly maintained in a tree-based data structure, updates to metadata in the leaf nodes can propagate changes to the root node, known as the Wandering Tree Problem~\cite{bityutskiy2005jffs3}. Due to the update of leaf metadata, such that when file data is modified, the metadata points to the new location of the file data, causing new metadata to be written, which in turn requires its parent to be updated to point to the new location of the metadata. This propagates up to the root note, causing significant \as{wa}. F2FS utilizes a table based indexing, with the \as{nat}, such that only a table entry is required to change to update the data location, and metadata points to the table entry to locate the data. ReconFS~\cite{2014-Lu-ReconFS} utilizes an inverted indexing tree, which also avoids the wandering tree problem. With such a tree, each node points to its parent node, instead of the parent node pointing to a child node. Therefore, upon address change of a child node, the parent does not need to be modified, since the child node points upwards to the parent node. Similar mechanisms are utilized in \as{ftl} design~\cite{2013-Lu-Flash_Lifetime_Reduce_With_WA}, where indexing data is written in the \as{oob} space of the flash page from the data, in order to locate its metadata. In order to avoid increased scan times on failure recovery, which can no longer traverse the tree from the root, the updated pages are tracked to locate the most recently updated valid page, which is then periodically included in the checkpointing to ensure consistency. 

\subsection{Summary}
The introduced \as{io} amplification of flash storage, particularly a result of \as{gc}, requires careful consideration to reduce the write requests, such that the flash device lifetime can be extended. Several of the previously discussed mechanisms, such as reducing write traffic and utilizing effective \as{gc} policies, aid in reducing the \as{io} amplification, however furthermore particular data structures optimized for space utilization similarly provide efficient methods for reducing \as{io} amplification.

\section{FIC-4: Flash Parallelism}\label{sec:flash_parallelism}
With the capabilities of flash storage relying largely on increased parallelism, several existing mechanisms are leveraging these. Depending on the level of flash integration, different mechanisms are possible, where at the \as{ssd} integration the host has not control over the possible physical parallel utilization of flash, as the \as{ftl} controls this, however deeper flash integration at the custom and embedded levels provide more possibility. \cref{tab:flash_parallelism} depicts the various mechanisms to aid the utilization of flash parallelism and exploit the physical characteristics of flash.

\begin{table}[!t]
    \centering
    % Update reference format to fit table into column
    \crefformat{section}{\S#2#1#3}
    \crefformat{subsection}{\S#2#1#3}
    \crefformat{subsubsection}{\S#2#1#3}
    \begin{tabular}{||p{40mm}|p{35mm}||}
        \hline 
        Mechanism & File Systems \\
        \hline
        \hline
        \cellcolor{lightgreen!75}Aligning the Allocation Unit (\cref{sec:align_alloc}) & \cellcolor{lightgreen!75}\cite{2020-Tu-URFS,2009-Park-Multimedia_NAND_Fs} \\
        \hline
        Clustered Allocation \& Striping (\cref{sec:clustered_alloc}) & \cite{2012-Min-SFS,dubeyko2019ssdfs,2018-Yoo-OrcFS,2016-Zhang-ParaFS,2011-Park-Multi_NAND,manning2010yaffs,aleph2001yaffs,manning2002yaffs} \\
        \hline
        Concurrency (\cref{sec:concurrency}) & \cite{2021-Liao-Max,2018-Kannan-DevFS,2015-Kang-SpanFS,2019-Lee-Parallel_LFS,2016-Lee-AMF,2016-Zhang-ParaFS} \\
        \hline
    \end{tabular}
    % Reset reference format to before
    \crefformat{section}{Section #2#1#3}
    \crefformat{subsection}{Section #2#1#3}
    \crefformat{subsubsection}{Section #2#1#3}
    \caption{Mechanisms for file systems to exploit flash parallelism capabilities, and the respective file systems that implement a particular mechanism. Green highlighted table cells depict previously discussed mechanisms with their respective section.}
    \label{tab:flash_parallelism}
\end{table}

\subsection{Clustered Allocation \& Striping}\label{sec:clustered_alloc}
As host software has no direct access to flash storage with \as{ssd}, the \as{ftl} implements and manages all device-level parallelism. The possibility for the host to utilize flash parallelism comes from aiding the \as{ftl} in providing large enough \as{io}s such that the \as{ftl} can stripe data across flash chips and channels. This is achieved with \textit{clustered blocks/pages}~\cite{2012-Kim-clustered_blocks}, where blocks or pages on different units (such as blocks on different planes) are accessed in parallel. The \as{ftl} can possibly stripe data across these clustered blocks, given that the \as{io} request is large enough to fill the clustered unit. Such a mechanism aligns with prior discussed aligning of the allocation unit (\cref{sec:align_alloc}) to a physical unit to reduce \as{io} amplification, such as making the file system segment unit a multiple of the flash allocation unit. SFS~\cite{2012-Min-SFS} takes advantage of the achieved device-level parallelism with clustered blocks by aligning segments to a multiple of the clustered block size. During garbage collection SFS ensures that cleaning of segments, that do not have enough blocks to fill the clustered block size, is delayed until enough data is present.

SSDFS~\cite{dubeyko2019ssdfs} utilizes the custom flash integration to map data allocation of segments to the unit of a \af{peb}, where the \as{peb} is split over the parallel unit on the device, such as previously mentioned parallel erasing of flash blocks over channels. Therefore, utilizing \as{peb}s over the parallel unit allows striping writes into a segment over the varying channels, increasing the device parallelism. In order to achieve this, \as{io} requests have to be large enough such that they can be striped across the channel and fill the \as{peb}s mapped to the segment. For this, SSDFS utilizes aggressive merging of \as{io} requests to achieve the larger \as{io}s that can be striped across the parallel units (solving \textbf{\as{fic}7}). Instead of merging \as{io} requests, OrcFS~\cite{2018-Yoo-OrcFS} increases its file system allocation unit to a \textit{superblock} (not to be confused with the file system superblock), which represents a set of flash blocks. These flash blocks are then split over the parallel units of the flash storage by the file system for increased parallelism. The file system utilizes a custom flash integration, and hence is capable of managing the parallelism of the flash storage.

The large allocation unit however introduces increased \as{gc} overheads, and block-level striping has lower performance than page-level striping~\cite{2016-Zhang-ParaFS}. Therefore, ParaFS~\cite{2016-Zhang-ParaFS} implements a 2-D allocation scheme, with page-level striping over the flash channels, where striping is also based on the data hotness, hence having a 2-dimensional allocation scheme. Different groups are assigned for the hotness levels, where writes are issued to the corresponding hotness group striped over the flash channels. Several other file systems implement variations of striping across different parallel units on flash storage~\cite{2011-Park-Multi_NAND,manning2010yaffs,aleph2001yaffs,manning2002yaffs}. These mechanisms are also present in the design of storage applications, such as key-value stores~\cite{zhang2017flashkv,wang2014efficient}.

\subsection{Concurrency}\label{sec:concurrency}
Similar to increasing the data allocation unit, concurrency is a mechanism to exploit the flash parallelism. \as{lfs} design relies on a single append point at the head of the log, in the simplistic implementations, depending on methods such as locking to ensure only one write is issued at the log head. This has a significant impact on the performance where other \as{io}s are idle while a single \as{io} completes. F2FS however suffers from severe lock contention overheads, where the performance of the multi-headed logging is nearly fully deprecated due to the serialization of data updates~\cite{2021-Liao-Max}. In particular, as data has to be written persistently before \as{inode} and other metadata can be written. Furthermore, F2FS suffers from contention of the in-memory data structures, for which it uses reader-writer semaphores for read and write operations from the user (termed \textit{external \as{io} operations}), and reader-writer locks for writing of checkpoints and other metadata (termed \textit{internal operations})~\cite{2021-Liao-Max}. As lock counters are shared among all cores, cache coherence adds a significant overhead that increases with more cores. Max~\cite{2021-Liao-Max} extends F2FS to increase the concurrency scalability with three main modifications.

Firstly, in order to eliminate cache coherence overheads, it introduces a \af{rps} that uses a per-process counter. Secondly, the shared data structures in memory are partitioned by the \as{inode}, such that concurrent accessing does not require locking on parts of the radix tree, but instead on an \as{inode} basis. Lastly, it utilizes multiple independent logs, called a \af{mlog}, which are accessed concurrently. The different between \as{mlog} and multi-headed logging in F2FS is that atomic data blocks are mapped to the same \as{mlog}, eliminating the need to ensure concurrency control across different logs. Ordering for persistence, ensuring data blocks are written before metadata, is delegated to the recovery mechanism using a global versioning number in each inode to identify ordering across \as{mlog}s, and recover the most recent version number in case of a system crash. These mechanisms eliminate much of the needed concurrency control, which sequentialized major parts of operations and hindered multicore scalability.

With this increased concurrency capabilities, the file system can issue more \as{io} requests to the device, allowing to leverage a higher degree of on-device parallelism. Similarly, DevFS~\cite{2018-Kannan-DevFS} utilizes the parallel capabilities by exploiting the high number of \as{io} queues supported by \as{nvme}. It maps \as{io} queues to individual files, allowing single file operations to submit \as{io}s concurrently without interfering on the \as{io} queue, therefore increasing the per-file concurrency as well. Likewise to the concept of \as{mlog}, SpanFS~\cite{2015-Kang-SpanFS} maps files and directories to different \textit{domains}, such that individual domains can be accessed in parallel. Such methods have a higher lock granularity, where concurrency below the lock granularity is not possible, as a single process is holding the lock. Therefore, instead of holding locks for individual inodes or files, preventing concurrent writing to the same inode or file, Lee et al.~\cite{2019-Lee-Parallel_LFS} extend F2FS to utilize \af{rl}, in which ranges of a file are locked, and different ranges can be written concurrently. Therefore, it provides the possibility for intra-file parallelism. 

ALFS~\cite{2016-Lee-AMF}, exploits the flash parallelism by mapping consecutive file system segments to the flash channels and utilizing different \as{io} queues for each flash channel. Similarly, ParaFS~\cite{2016-Zhang-ParaFS} implements \textit{parallelism-aware scheduling}, which also maintains different \as{io} queues for each channel. However, it extends this concept by using a \textit{dispatching phase} and a \textit{request scheduling phase}. The dispatching phase optimizes write \as{io}s by scheduling \as{io} requests to the respective channels based on the utilization of the channel, such that the least utilized channels receives more requests. All requests are assigned a weight, which indicates their priority in the queue, where read requests weight is lower than that of write requests, because of the asymmetric performance of flash storage. During the request scheduling phase the scheduler assigns slices to the read and write/erase operations in the individual queues, such that if the time from the slice of a read operation is up and the queue contains no other read requests, a write or erase is scheduled, based on a fairness formula that incorporates the amount of free space in the block and concurrently active erase operations on other channels. This allows to minimize the erase operations on the flash, giving always free channels to utilize and maintain a fair schedule between write and erase operations.

\subsection{Summary}
Due to the architecture of flash storage providing a high degree of parallelism, numerous methods are employed to leverage these parallel units in order to maximize the performance. Depending on the level of flash integration, particular design choices can be made, such as clustered allocation and striping can be achieved with flash \as{ssd} integration, by providing large write \as{io} requests such that the \as{ftl} can stripe the data across parallel units. With a higher degree of control over the flash storage, file systems can directly rely on utilizing concurrency to leverage the parallelism of flash storage.

\section{FIC-5: Wear Leveling}\label{sec:wear_leveling}
As flash cells wear out over time, it is important to utilize the flash evenly to avoid burning out particular flash cells faster than others. The possibility for ensuring even wear at the different levels of integration is limited, as at the \as{ssd} flash integration level, the \as{ftl} handles all wear leveling, without host considerations. However, similar to prior flash integration challenges, several mechanisms are nonetheless applicable. In particular, reducing the write traffic to the flash device, as less writing incurs less flash wear, and particularly flash-specific data structures inherently provide a degree of wear leveling. Based on the sequential write requirement, data structures, such as the \as{lfs} must write sequentially in an append-only fashion, which evenly writes the space. At closer to flash integrations, where host systems have more control over the flash management, there are particular mechanisms to ensure better flash wear. \cref{tab:wl} depicts the methods we discuss in this section for enabling increased wear leveling for file systems.

\begin{table}[!t]
    \centering
    % Update reference format to fit table into column
    \crefformat{section}{\S#2#1#3}
    \crefformat{subsection}{\S#2#1#3}
    \crefformat{subsubsection}{\S#2#1#3}
    \begin{tabular}{||p{50mm}|p{25mm}||}
        \hline 
        Mechanism & File Systems \\
        \hline
        \hline
        \cellcolor{lightgreen!75} Write Optimised Data Structures (\cref{sec:wods}) & \cellcolor{lightgreen!75} \cite{2020-Tu-URFS,hunter2008brief,rodeh2013btrfs,2022-Jiao-BetrFS,dubeyko2019ssdfs} \\
        \hline
        \cellcolor{lightgreen!75} Reducing Write Traffic (\cref{sec:reduce_write_traffic}) & \cellcolor{lightgreen!75} \cite{huang2013improve,2011-Lim-DeFFS,woodhouse2001jffs,dubeyko2019ssdfs,2021-Ji-F2FS_compression,2007-Hyun-LeCramFS,ning2011design,2009-Zuck-NANDFS,2022-Lee-F2FS_Address_Remapping} \\
        \hline
        \cellcolor{lightgreen!75} \as{gc} Policies (\cref{sec:gc_policy}) & \cellcolor{lightgreen!75} \cite{2012-Min-SFS,2018-Yoo-OrcFS,2021-Gwak-SCJ,2015-Park-Suspend_Aware_cleaning-F2FS} \\
        \hline
        Write \& Read Leveling (\cref{sec:write_leveling,sec:read_leveling}) & \cite{2006-Lim-NAND_fs,2016-Lee-AMF,2020-Wu-DualFS,2009-Sungjin-FlexFS} \\
        \hline
    \end{tabular}
    % Reset reference format to before
    \crefformat{section}{Section #2#1#3}
    \crefformat{subsection}{Section #2#1#3}
    \crefformat{subsubsection}{Section #2#1#3}
    \caption{Mechanisms for file systems to deal with wear leveling of the flash storage, and the respective file systems that implement a particular mechanism. Green highlighted table cells depict previously discussed mechanisms with their respective section.}
    \label{tab:wl}
\end{table}

\subsection{Write Leveling}\label{sec:write_leveling}
Several flash integration challenges proved data grouping to be an effective method for dealing with \as{gc} overheads and \as{io} amplification. However, this can have an effect on the flash storage. In particular the hot data, such as file system metadata which is more frequently updated and written, must be moved across the flash space more than cold data. CFFS~\cite{2006-Lim-NAND_fs} therefore switches the allocation areas for metadata and data blocks, such that an erased metadata block becomes a free data block. Therefore, cold data should be placed in blocks that have been written more frequently, whereas hot data should be placed in blocks with a lower write count history. The principle of migrating cold data from less written blocks, which are also referred to as \textit{younger blocks}, to more frequently written, \textit{older blocks}, is referred to as \textit{cold-data migration}, and similarly moving hot data from old blocks to younger blocks is known as \textit{hot-data migration}~\cite{chang2007efficient}. These methods are commonly used in \as{ftl} implementations due to their simplicity and effectiveness, and similarly in file systems such as ALFS~\cite{2016-Lee-AMF}.

Wear leveling is an increasingly vital concern on file systems that utilize flash dual mode~\cite{2020-Wu-DualFS,2009-Sungjin-FlexFS}, where it switches the flash level to increase performance for critical \as{io} requests. Due to the lowering in flash cell level, the same amount of written data requires a larger amount of space, where a switch from \as{mlc} to \as{slc} divides the capacity in half, requiring double the space for the same \as{io} request. Therefore, these file systems include a \textit{write budget} that is maintained for the areas, and dynamically resizes the available lower cell level area, such as decreasing the space if the wear is reaching a threshold. This switches the cell level back to a higher number, allowing to write more with less wear. Additionally, the file systems utilize the wear budget in order to identify if an \as{io} request should be redirected to the larger cell area, instead of being written to the lower cell level area.

\subsection{Read Leveling}\label{sec:read_leveling}
While write operations are the major cause of flash cells burning out, read operations also pay a toll on flash cells, as the current flash cell technology utilizes flash cells that are only capable of holding very few electrons (determining the charge of the gate) due to their size~\cite{lu2009future,shin2005non}. This makes the cells increasingly susceptible to \textit{read disturbance}~\cite{2015-Liu-Read_leveling}, where reading of a page results in shifting of voltages in nearby cells (typically in the same block), requiring frequent rewriting to ensure the charge stays consistent. In order to control read disturbance, Liu et al.~\cite{2015-Liu-Read_leveling} propose to read-leveling mechanisms in the FTL. While their proposal is aimed at \as{ftl} implementations, the ideas are applicable to file systems for flash storage devices. With the proposed read-leveling, the read-hot data, that is read more frequently, is isolated from other data pages by placing the hot pages into \textit{shadow blocks}, which contain no valid data, in order to avoid disturbing that data. However, this requires to identify the read-hot pages, where a tracking of read counters for each page would require significant resources. Therefore, a \textit{second-chance monitoring strategy} is proposed, which initially tracks the reads for each block, therefore requiring counters at a higher block granularity, and upon reaching a threshold indicating the block contains read-hot pages, the individual pages in the block are tracked on their read counters. Finally, the pages in these blocks that reach a certain threshold are copied to the shadow blocks. Therefore, this avoids the tracking of read counters for individual pages and only copies read-hot pages into the shadow blocks. While this strategy requires copying of read-hot pages to shadow blocks, it minimizes read disturbance which in turn minimizes the \as{wa} it causes.

\subsection{Summary}
In addition to previously discussed mechanisms to reduce \as{io} amplification and \as{gc}, resulting in decreased wear of the flash storage, write leveling, ensuring that write requests are spread across the available storage space, and read leveling are important mechanisms for ensuring the longevity of the flash storage.

\section{\as{io} Scheduling}\label{sec:io_sched}
Given that flash storage has the capabilities to achieve single digit $\mu$-second latency, whereas overheads in the software stack, such as context switching in the Kernel caused by system calls, can already require $\mu$-seconds to complete~\cite{soares2010flexsc}, making software the dominating factor in overheads~\cite{2010-Caulfield-Moneta,2012-Caulfield-Fast_User_Space_Access,foong2010towards,seppanen2010high,vasudevan2012using}. In addition, \textit{interrupts} cause significant overheads for systems. Aimed at slow storage devices, the \as{io} request is submitted to the device, the context is switched, such that the process can continue with other work, and upon completion of the \as{io}, the host is interrupted, and the context is switched again. Any added interrupt on the \as{io} path can cause significant delays~\cite{2014-Shin-OS_IO_Path}. Cache effects are another drawback of context switching, since other work is continued, replacing data in the caches, it requires bringing the replaced data back into the caches after the interrupt and resuming of the prior context. Similarly, it also causes \af{tlb} pollution on the host system. A different approach to submitting \as{io} requests is with \textit{polling}, which eliminates the need for context switches. With polling, the \as{io} request is submitted and instead of continuing other work, the process regularly checks the \as{io} for completion. Using polling for \as{io} requests has been shown to be a favored method of building application for fast storage devices~\cite{kourtis2019reaping,yang2012poll,didona2022understanding}.

Modern systems largely rely on \textit{asynchronous \as{io}} over \textit{synchronous \as{io}}. With synchronous \as{io}, a process submits a single \as{io} request and waits for its completion. In order to saturate the storage device performance, additional threads are required, which submit \as{io} requests to the numerous \as{io} queues in the of the storage device. However, this mechanism does not scale efficiently, where each thread must wait until the \as{io} request is completed. Therefore, with asynchronous \as{io} the threads do not wait for completion, but instead submit a larger number of \as{io} requests each, allowing to fill the device \as{io} queues more effectively. The \as{io} requests for which a thread as submitted a request, but have not completed, are referred to as \textit{outstanding} or \textit{in-flight} \as{io} requests. \cref{tab:io_sched} shows the mechanisms for host systems to better leverage the flash storage performance and minimize overheads. In addition to file systems implementing particular mechanisms, we discuss more general methods applicable to all applications for benefiting from flash storage and enhancing performance.

\begin{table}[!t]
    \centering
    % Update reference format to fit table into column
    \crefformat{section}{\S#2#1#3}
    \crefformat{subsection}{\S#2#1#3}
    \crefformat{subsubsection}{\S#2#1#3}
    \begin{tabular}{||p{40mm}|p{35mm}||}
        \hline 
        Mechanism & File Systems \\
        \hline
        \hline
        \as{io} Operations (\cref{sec:io_ops}) & \cite{2020-Tu-URFS} \\
        \hline
        \as{io} Scheduler (\cref{sec:io_scheduler}) & \cite{2021-Qin-Atomic_Writes} \\
        \hline
        \as{io} Path - User-Space File Systems & \cite{2020-Tu-URFS,2019-Yoshimura-EvFS,2018-Kannan-DevFS,2019-Liu-fs_as_process} \\
        \hline
    \end{tabular}
    % Reset reference format to before
    \crefformat{section}{Section #2#1#3}
    \crefformat{subsection}{Section #2#1#3}
    \crefformat{subsubsection}{Section #2#1#3}
    \caption{\as{io} scheduling mechanisms to exploit performance capabilities of flash storage, and the respective file systems that implement a particular mechanism.}
    \label{tab:io_sched}
\end{table}

\subsection{\as{io} Operations}\label{sec:io_ops}
Given that particular \as{io} patterns can have degrading affects on the \as{ssd} performance, such as mixing read and write operations, as they share resources on the device, including the mapping table and \as{ecc} engine, and furthermore possibly invalidating the cached data in the \as{ssd} \as{ram}. Similarly, mixing \as{io} operations with different block sizes can result in increased fragmentation~\cite{2020-Tu-URFS}. As the Linux Kernel relies on a submission and completion queue for \as{io}, user-space frameworks such as \as{spdk} and NVMeDirect provide more flexibility for user-space file systems to design different queues, depending on the requirements. URFS~\cite{2020-Tu-URFS} utilizes this possibility to create adaptive queues that can better optimize \as{io} submissions to the device. Based on the workload characteristics URFS dynamically creates flexible \as{io} queues (e.g., group by size, read/write operation) to increase \as{ssd} performance. Similarly, Borge et al.~\cite{2019-Borge-SSD_read_variability} show with a case study on HDFS performance with \as{ssd}, that in order to leverage the capabilities of flash \as{ssd}, direct \as{io}, and increased parallel requests with buffered \as{io} are needed. 

\subsection{\as{io} Scheduler}\label{sec:io_scheduler}
With the possibility for asynchronous \as{io} to merge and reorder requests, the Linux Kernel implements several schedulers, such as \textit{NOOP}, \textit{deadline}, and \textit{CFQ}~\cite{sun2014exploring,pratt2004workload,moallem2008study,heger2010linux}. NOOP being the least intrusive scheduler only merges \as{io} request, but does not reorder them, which is beneficial on devices such as \as{ocssd}, that require consecutive LBAs. Son et al.~\cite{2015-Son-Optimizing_FS} showcase the benefits of merging random write requests, regardless of contiguity of the LBAs, in order to better enhance performance with fast storage devices. The deadline scheduler adds to NOOP by utilizing merging and reordering, however also applies a deadline for each \as{io} request to ensure requests are submitted to the device eventually. Two separate queues, one for read requests and an additional one for write requests are utilized, which are both ordered by the deadline of the request. Another scheduler variant of deadline exists, called \textit{mq-deadline}, which is aimed at multi-queue devices, such as \as{nvme} \as{ssd}. \af{cfq} implements a round-robin based queue that assigns time slices to each in order to prevent starvation and provide fairness. While these are the common schedulers in the Linux Kernel, to see details on all schedulers present in the Linux Kernel consult~\cite{2019-ubuntu_wiki_io_schedulers}. Such scheduling configuration begs the question on which scheduler is best suited for file systems on flash storage. Several studies into performance of the schedulers exist~\cite{sun2014exploring,yu2014optimizing}, showcasing that merging of read \as{io} requests in synchronous \as{io} provides beneficial performance gains, and similarly the merging of write \as{io}s in asynchronous \as{io} shows performance gains.

Qin et al.~\cite{2021-Qin-Atomic_Writes} argue that \as{io} ordering limits exploiting the parallelism of flash devices. Especially as the Linux block layer does not guarantee particular ordering, flags such as \af{fua}, indicating that \as{io} completion is signaled upon arrival of data on the storage, and \textit{PREFLUSH}, which before completing the request flushes the volatile caches, have to be set in order to ensure a specific ordering~\cite{2021-Qin-Atomic_Writes}. With file systems, the \as{io} of metadata and data has a particular ordering, such that metadata can only point to data that exists, needing to ensure that data is written prior to metadata. Removing of \as{io} ordering allows eliminating this need and better utilize the flash parallelism. Utilizing the \as{oob} area on flash pages, the file system developed by Qin et al., called NBFS, maintains versioning in order to identify out of order updates. Furthermore, updates are done using atomic writes (discussed in Section~\ref{sec:failure_consistency}). The issuing of \as{fua} requests further implies that its \as{io}s cannot be merged in the scheduler~\cite{2021-Qin-Atomic_Writes}, implying that if a smaller than flash page size \as{fua} \as{io} request is issued, it is padded to the page size, causing \as{wa}. NBFS solves this with its atomic writes that imply that the \as{fua} request does not immediately have to be written to the flash, but instead wait for all data blocks to arrive, which are then used to fill the pages, allowing to reduce the \as{wa} (solving \textbf{\as{fic}3}).

\subsection{\as{io} Path - User-Space File Systems}\label{sec:user_space_fs}
A mechanism that is gaining significant attention in the research community is the utilization of user-space file systems, bypassing the Kernel layers and avoid its associated overheads. These file systems run only in the user-space, as opposed to commonly used file systems (e.g., F2FS) running in Kernel space. In addition to the benefit of avoiding Kernel overheads, user-space file systems are easier to develop, have increased reliability and security by avoiding Kernel bugs and vulnerabilities, and provide increased portability~\cite{2015-Tarasov-User_space_fs_practicality}. A widely adopted framework for building user-space storage applications is \af{fuse}~\cite{szeredi2010fuse}. It is implemented over a Kernel module with which it exports a virtual file system from the Kernel, where data and metadata are provided by a user-space process, hence allowing user-space applications to interact with it. Since \as{fuse} is implemented with a Kernel module, \as{fuse} based file systems suffer significant performance penalties, requiring more \as{cpu} cycles than file systems in Kernel space. Particularly contributing to overheads is the need to copy memory between user-space and kernel-space, caused by the way \as{fuse} handles \as{io} requests~\cite{2019-Vangoor-Fuse_performance,vangoor2017fuse}. Furthermore, \as{fuse} still suffers from context switching~\cite{vangoor2017fuse,2019-Vangoor-Fuse_performance,rajgarhia2010performance} overheads and \af{ipc} between the \as{fuse} kernel module and \as{fuse} user-space daemon~\cite{zhu2018direct}. 

A similar framework for building user-space applications with direct storage access is \as{nvme}Direct~\cite{2016-Kim-NVMe_Direct}. However, it also relies on a Kernel driver to provide enhanced \as{io} policies. \as{spdk}~\cite{2017-Yang-SPDK} is another framework for building user-space storage applications, however it provides the mechanisms to bypass the Kernel and submit \as{io} directly to the device, by implementing a user-space driver for the storage device. Such a framework allows building high performance storage applications in user-space, which eliminate the overheads coming from the Kernel \as{io} stack. 

URFS~\cite{2020-Tu-URFS} provides increased concurrency performance by implementing a multi-process shared cache in memory, in order to avoid the Kernel overheads of copying data as is present in \as{fuse}. It furthermore helps avoid contention on the storage device. Eliminating of data copy is also addressed in ZUFS (zero-copy user-mode file system)~\cite{2019-Harrosh-zufs}, which is a
user-space file system for persistent memory, which completes \as{io} requests by requesting exact data locations instead of copying data into the own address space. A similar user-space file system that implements a shared cache for process is EvFS~\cite{2019-Yoshimura-EvFS}, which is \as{spdk}-based. While this file system can also support multiple readers/writers in the page cache, it only supports these for a single user process. User-space frameworks often provide capabilities to either expose the storage device as a block device, which the user-space application then accesses, or build a custom block device module (e.g., with \as{spdk}, which also has default driver modules such as \as{nvme}). For \as{nvme} devices that support \as{nvme} controller memory buffer management, the file system can manage parts of the device memory. DevFS~\cite{2018-Kannan-DevFS} utilize such an integration to manage the device memory for file metadata and \as{io} queues.

Different from prior discussed development frameworks, \af{fsp}~\cite{2019-Liu-fs_as_process} provides a storage architecture design for user-space file systems. The emphasis of \as{fsp} is to scale with the arrival of faster storage, and similarly to other user-space frameworks, minimize the software stack. For this it bases development on running file systems as processes, providing safer metadata integrity, data sharing coordination, and consistency control, as the process running the file system is in control of everything, instead of trusting libraries. Furthermore, \as{fsp} relies on \as{ipc} for fast communication, which unlike FUSE has a low overhead since it does not require context switching. Inter-core message passing comes at a low overhead and cache-to-cache transfers on multi-core systems can complete in double-digit cycles~\cite{soares2010flexsc}. DashFS~\cite{2019-Liu-fs_as_process} is built with \as{fsp}, providing a safe user-space file system with isolation of different user processes, and efficient \as{ipc}.

% Figure environment removed

\section{Failure Cases}
\label{sec:failure}
In this section, we provide some failure cases and investigate possible reasons. We perform the analysis on single-agent predictions on Argoverse, since the miss rate in single-agent prediction is relatively higher than multi-agent predictions on Interaction. We identify three sources of error for failure cases: erroneous data, missing rare behaviors, and inaccurate predictions. 



\boldparagraph{Erroneous Data} The accuracy of the provided input trajectories in the past directly affects the future predictions, since the future predictions are trained to be consistent with the past ones. Thus, defective or unstable history data causes incorrect future predictions as shown in \figref{fig:argo_fail_input_supp}. Moreover, defects in the future steps result in inaccurate evaluations of the predictions~(\figref{fig:argo_fail_output_supp}). Some problems such as id-switch and position-oscillation, resulting in unstable and incorrect ground truth future locations, are addressed in previous works as well~\cite{Ye2021CVPR, Song2021CoRL}. 



Additionally, accurate map information plays an important role in future predictions because it directly affects the reasoning of the model about drivable areas. In the example shown in~\figref{fig:argo_fail_lane_supp}, predictions are intensified on a single mode \ie left turn, because of the missing lane that the ground truth trajectory follows.


\boldparagraph{Missing a Peculiar Mode} Despite the large number of scenarios on Argoverse, some behaviors are less frequently observed such as a u-turn or an abrupt lane change. These behaviors that are rare on the training set cause the model to miss the relevant mode at test time as shown in~\figref{fig:argo_fail_behaviour_supp}.


\boldparagraph{Precision of Predictions} Some predictions result in an error due to a lack of precision in the predicted trajectories despite correctly identifying intention. For example, in~\figref{fig:argo_fail_inaccurate_supp}, all possible paths are covered by the predictions but the difference between the closest endpoint and the ground truth endpoint is higher than the miss rate threshold.



\section{Discussion}
\label{sec: discussion}
\kmsdelete{In this work} We study \kmsreplace{Fairness-Aware PAC learning}{Fair-ERM} in the malicious noise model, and  in some cases allow 
the learner to maintain optimal overall accuracy despite the signal in Group $B$ being almost entirely washed out.
%when we allow learners to use the
%$\PQ$ randomized expansion of the hypothesis class $\mathcal{H}$
In particular we show that different fairness constraints have fundamentally different behavior in the presence of Malicious Noise, in terms of the amount of accuracy loss that a given level of Malicious Noise could cause a fairness-constrained learner to incur. 
The key to achieving our results, which are more optimistic than those in \cite{lampert}, is allowing for improper learners using the (P,Q)-randomized expansions of the given class $\mathcal{H}$.
%We \kmsreplace{present a picture of the}{prove upper and lower bounds on}
%accuracy loss for a range of fairness notions, given \kmsreplace{this simple randomization step.}{learning over $\PQ$.
%In general our results indicate Fair-ERM (given learning over $\PQ$) is more robust than claimed in \cite{lampert}.
The type of smoothness we create by using $\PQ$ seems to be a natural property that is likely shared by many natural hypothesis classes.

Fairness notions are motivated as a response to learned disparities when there is \kmsdelete{data corruption or} systemic error affecting \kmsdelete{the data for}
one group. 
Fairness notions are supposed to mitigate this by ruling out classifiers that have worse performance on a sub-group. 
This can peg both classifiers at a lower level of performance \kmsdelete{(e.g that the lower subgroup)} in order to \emph{motivate} \cite{hardt16} improving the data collection or labelling process to obtain more reliable performance. 
%So in \kmsreplace{some}{a} sense, sensitivity of the fairness notion to poor sub-group performance caused by malicious noise is the \textit{point} of fairness constraints! 
However, it also desirable that fairness constraints perform gracefully when subject to Malicious Noise because fairness constraints will be used in contexts where the data is unreliable and noisy and this might not be known to the learner.
This tension, exposed by our work, motivates 
%a revisiting of fairness notions from first principles approach and trying to axiomatize the 
%desired properties of a fairness intervention a la cryptography and privacy. \footnote{Work in multi-calibration \cite{multicalib} is a viable direction for this problem but it is unclear how 
%that and related notions behave with unreliable data. }
on going work studying the sensitivity level of fairness constraints. 
%If we we are to take a view, if a classifier is deployed 

\section{Related Work}
\label{appsec: related work}
Bayesian causal discovery literature has primarily focused on inference in linear models with closed-form posteriors or marginalized parameters. Early works considered sampling directed acyclic graphs (DAGs) for discrete~\cite{cooper1992bayesian, madigan1995bayesian, heckerman2006bayesian} and Gaussian random variables~\cite{friedman2003being, tong2001active} using Markov chain Monte Carlo (MCMC) in the DAG space. However, these approaches exhibit slow mixing and convergence~\cite{eaton2012bayesian,grzegorczyk2008improving}, often requiring restrictions on number of parents~\cite{kuipers2017partition}. %Alternative exact dynamic programming methods are limited to small settings~\cite{koivisto2012advances}. 

Recent advances in variational inference~\cite{zhang2018advances} have facilitated graph inference in DAG space, with gradient-based methods employing the NOTEARS DAG penalty \cite{zheng2018dags}.\cite{annadani2021variational} samples DAGs from autoregressive adjacency matrix distributions, while \cite{lorch2021dibs} utilizes Stein variational approach \cite{liu2016stein} for DAGs and causal model parameters. \cite{cundy2021bcd} proposed a variational inference framework on node orderings using the gumbel-sinkhorn gradient estimator \cite{mena2018learning}. \cite{deleu2022bayesian,nishikawa2022bayesian} employ the GFlowNet framework \cite{bengio2021gflownet} for inferring the DAG posterior. Most methods, except\cite{lorch2021dibs} are restricted to linear models, while \cite{lorch2021dibs} has high computational costs and lacks DAG generation guarantees compared to our method.
% at least quadratic scaling complexity, both with respect to the number of nodes (due to the DAG penalty) as well as number of posterior samples. Our proposed approach instead has linear complexity with respect to number of posterior samples and does not require any additional DAG penalty.     

In contrast, \emph{quasi-Bayesian} methods, such as DAG bootstrap \cite{friedman2013data}, demonstrate competitive performance. DAG bootstrap resamples data and estimates a single DAG using PC \cite{spirtes2000causation}, GES \cite{chickering2002optimal}, or similar algorithms, weighting the obtained DAGs by their unnormalized posterior probabilities. Recent neural network-based works employ variational inference to learn DAG distributions and point estimates for nonlinear model parameters \cite{charpentier2022differentiable,geffner2022deep}.
\section{Conclusion and Future Work}
In this work, I design corruption-robust algorithms for the Lipschitz contextual search problem. I present the \emph{agnostic checking} technique and demonstrate its effectiveness in designing corruption-robust algorithms. There are several open problems for future research. First, in the algorithm I propose for pricing loss, the schedule for agnostic checks is fixed upfront. Can the learner design an adaptive checking schedule for the pricing loss? Second, this work assumes the learner has knowledge of the Lipschitz constant $L$. Can the learner design efficient no-regret algorithms without knowledge of $L$? 

\printglossaries

% 
% %-------------------------------------------------------------------------------
% \section{Footnotes, Verbatim, and Citations}
% %-------------------------------------------------------------------------------
% 
% Footnotes should be places after punctuation characters, without any
% spaces between said characters and footnotes, like so.%
% \footnote{Remember that USENIX format stopped using endnotes and is
%   now using regular footnotes.} And some embedded literal code may
% look as follows.
% 
% \begin{verbatim}
% int main(int argc, char *argv[]) 
% {
%     return 0;
% }
% \end{verbatim}
% 
% Now we're going to cite somebody. Watch for the cite tag. Here it
% comes. Arpachi-Dusseau and Arpachi-Dusseau co-authored an excellent OSuse
% book, which is also really funny~\cite{arpachiDusseau18:osbook}, and
% Waldspurger got into the SIGOPS hall-of-fame due to his seminal paper
% about resource management in the ESX hypervisor~\cite{waldspurger02}.
% 
% The tilde character (\~{}) in the tex source means a non-breaking
% space. This way, your reference will always be attached to the word
% that preceded it, instead of going to the next line.
% 
% And the 'cite' package sorts your citations by their numerical order
% of the corresponding references at the end of the paper, ridding you
% from the need to notice that, e.g, ``Waldspurger'' appears after
% ``Arpachi-Dusseau'' when sorting references
% alphabetically%~\cite{waldspurger02,arpachiDusseau18:osbook}.
% 
% reference testing~\cite{2012-osdi-spanner}.  
% 
% It'd be nice and thoughtful of you to include a suitable link in each
% and every bibtex entry that you use in your submission, to allow
% reviewers (and other readers) to easily get to the cited work, as is
% done in all entries found in the References section of this document.
% 
% Now we're going take a look at Section~\ref{sec:figs}, but not before
% observing that refs to sections and citations and such are colored and
% clickable in the PDF because of the packages we've included.
% 
% %-------------------------------------------------------------------------------
% \section{Floating Figures and Lists}
% \label{sec:figs}
% %-------------------------------------------------------------------------------
% 
% 
% %---------------------------
% % Figure environment removed
% %% %---------------------------
% 
% 
% Here's a typical reference to a floating figure:
% Figure~\ref{fig:vectors}. Floats should usually be placed where latex
% wants then. Figure\ref{fig:vectors} is centered, and has a caption
% that instructs you to make sure that the size of the text within the
% figures that you use is as big as (or bigger than) the size of the
% text in the caption of the figures. Please do. Really.
% 
% In our case, we've explicitly drawn the figure inlined in latex, to
% allow this tex file to cleanly compile. But usually, your figures will
% reside in some file.pdf, and you'd include them in your document
% with, say, \textbackslash{}includegraphics.
% 
% Lists are sometimes quite handy. If you want to itemize things, feel
% free:
% 
% \begin{description}
%   
% \item[fread] a function that reads from a \texttt{stream} into the
%   array \texttt{ptr} at most \texttt{nobj} objects of size
%   \texttt{size}, returning returns the number of objects read.
% 
% \item[Fred] a person's name, e.g., there once was a dude named Fred
%   who separated usenix.sty from this file to allow for easy
%   inclusion.
% \end{description}
% 
% \noindent
% The noindent at the start of this paragraph in its tex version makes
% it clear that it's a continuation of the preceding paragraph, as
% opposed to a new paragraph in its own right.
% 
% 
% \subsection{LaTeX-ing Your TeX File}
% %-----------------------------------
% 
% People often use \texttt{pdflatex} these days for creating pdf-s from
% tex files via the shell. And \texttt{bibtex}, of course. Works for us.
% 
% %-------------------------------------------------------------------------------
% \section*{Acknowledgments}
% %-------------------------------------------------------------------------------
% 
% The USENIX latex style is old and very tired, which is why
% there's no \textbackslash{}acks command for you to use when
% acknowledging. Sorry.

%-------------------------------------------------------------------------------
\bibliographystyle{plain}
%\bibliography{\jobname}
\bibliography{main}


%%%%%%%%%%%%%%%%%%%%%%%%%%%%%%%%%%%%%%%%%%%%%%%%%%%%%%%%%%%%%%%%%%%%%%%%%%%%%%%%
\end{document}
%%%%%%%%%%%%%%%%%%%%%%%%%%%%%%%%%%%%%%%%%%%%%%%%%%%%%%%%%%%%%%%%%%%%%%%%%%%%%%%%

%%  LocalWords:  endnotes includegraphics fread ptr nobj noindent
%%  LocalWords:  pdflatex acks
