\section{Flash Integration Organization}
\ntytodo{coordinate this section}
With conventional SSDs, the FTL has full control over the underlying flash media. As a result, the possibility for file systems to fully incorporate each flash challenge is limited. However, particular design decisions in the file system aid the performance of the FTL and help leverage the flash characteristics. Therefore, this section covers the file systems that are particularly built on conventional flash-based SSDs (recall Figure~\ref{fig:flash_integration_ssd}), with an FTL exposing a conventional block interface. Table~\ref{tab:ssd_integration_comparison} shows a comparison of all file systems meeting this criteria, evaluating if the file system provides novel methods for handling of certain flash characteristics. As all file systems are log-structured file systems, which is a write optimized data structure due to its append only nature, which further inherently provides basic wear leveling by writing everything sequentially, the WL column is only marked on file systems that further extend this mechanism with additional novel methods.

\begin{table*}[!t]
    \centering
    \begin{tabular}{||p{15mm}||p{8mm}|p{15mm}|p{5mm}|p{20mm}|p{15mm}|p{5mm}|p{15mm}|p{15mm}||}
        \hline 
        & & \multicolumn{7}{|c||}{Flash Challenge} \\
        \hline
        File \newline System & Year & Read/Write \newline Asymmetry & GC & I/O \newline Amplification & Flash \newline Parallelism & WL & Software \newline Stack & I/O \newline Scheduling \\
        \hline
        \hline
        F2FS\cite{2015-Changman-f2fs} & 2015 & \cmark & \cmark & \xmark & \cmark & \xmark & \xmark & \xmark \\
        \hline
        F2FS + SCJ\cite{2021-Gwak-SCJ} & 2021 & \cmark & \cmark & \xmark & \cmark & \xmark & \xmark & \xmark \\
        \hline
        SFS\cite{2012-Min-SFS} & 2014 & \xmark & \cmark & \xmark & \cmark & \xmark & \xmark & \xmark \\
        \hline
        ReconFS\cite{2014-Lu-ReconFS} & 2014 & \xmark & \xmark & WA \& SA & \cmark & \xmark & \xmark & \xmark \\
        \hline
        URFS\cite{2020-Tu-URFS} & 2020 & \cmark & \cmark & WA & \xmark & \xmark & \cmark & \cmark \\
        \hline
        EvFS\cite{2019-Yoshimura-EvFS} & 2020 & \xmark & \xmark & \xmark & \xmark & \xmark & \cmark & \cmark \\
        \hline
        BetrFS\cite{2022-Jiao-BetrFS} & 2022 & \cmark & \xmark & \xmark & \xmark & \xmark & \xmark & \xmark \\
        \hline
        CSA-FS\cite{huang2013improve} & 2013 & \cmark & \xmark & SA & \xmark & \xmark & \xmark & \xmark \\
        \hline
        DashFS\cite{2019-Liu-fs_as_process} & 2019 & \xmark & \xmark & \xmark & \xmark & \xmark & \cmark & \cmark \\
        \hline
        DevFS\cite{2018-Kannan-DevFS} & 2018 & \xmark & \xmark & \xmark & \cmark & \xmark & \cmark & \cmark \\
        \hline
        % From here on papers added from original FS Design Considerations classification
        Max\cite{2021-Liao-Max} & 2021 & \cmark & \cmark & \cmark & \cmark & \xmark & \xmark & \cmark \\
        \hline
        SpanFS\cite{2015-Kang-SpanFS} & 2015 & \cmark & \cmark & \xmark & \cmark & \xmark & \xmark & \cmark \\
        \hline
        exF2FS\cite{2022-Oh-exF2FS} & 2022 & \cmark & \cmark & WA & \cmark & \xmark & \xmark & \xmark \\
        \hline
    \end{tabular}
    \caption{Evaluated file systems for flash-based SSD integration (with conventional FTL) and the flash challenges (Section~\ref{sec:opt_flash}) the file system solves. A \cmark\hspace{0.5mm} indicates that a file system implements a mechanism to handle the particular flash challenge, however it does not indicate to what extent it is handled or how successful it is handled. If a file system does not have a name and/or it is an extension on an existing file system, the authors' names are used as the file system name. As the majority of file systems are LFS-based designs, the table indicates if the file system employs a novel method in addition to what is required of LFS. For instance, the GC column contains a \xmark\hspace{0.5mm} if the file system simply implements greedy or cost-benefit GC, and a \cmark\hspace{0.5mm} if a new GC policy or new mechanisms of managing GC are presented.}
    \label{tab:ssd_integration_comparison}
\end{table*}

\textbf{Custom Flash Integration-Level.} Different types of flash-based storage devices allow for varying levels of flash management for the file system. With OCSSD, the device geometry of the underlying flash storage is exposed to the file system, allowing for specific data placement and optimizations. Multi-streamed SSDs can leverage the stream hint for the FTL to leverage increased parallelism and better data placement. Custom FTLs further extend this concept by giving the file system developers advanced interfaces and exposing distinct flash characteristics, allowing the file system to better manage data and optimize for the flash characteristics. Custom device drivers similarly allow increased device management. 

This section evaluates such storage devices, giving the host file system more flash management responsibility. We analyze the various file system implementations at this integration level, evaluating the additional flash integration modifications, which are not possible at the SSD level integration. Thus, we evaluate what additional benefits the deeper integration of custom flash devices can provide, and how file systems deal with these. We do not repeat the concepts that are discussed in the SSD integration, as they are all also applicable at this integration, instead we look only at the enhancement made by this integration on the concepts and new concepts made possible by the integration level. Table~\ref{tab:custom_integration_comparison} shows a comparison of the file systems evaluated in this section, indicating which flash challenges the particular proposed file system incorporates. Note, not all file systems in the table are discussed in the text, if they did not contribute a mechanism that is not already discussed.

\begin{table*}[!ht]
    \centering
     \begin{tabular}{||p{19mm}||p{8mm}|p{12mm}|p{15mm}|p{5mm}|p{20mm}|p{15mm}|p{5mm}|p{15mm}|p{15mm}||} \hline & & & \multicolumn{7}{|c||}{Flash Challenge} \\
        \hline
         File \newline System & Year & Type & Read/Write \newline Asymmetry & GC & I/O \newline Amplification & Flash \newline Parallelism & WL & Software \newline Stack & I/O \newline Scheduling \\
        \hline
        \hline
        Kawaguchi et al.\cite{kawaguchi1995flash} & 1995 & CD & \xmark & \cmark & \xmark & \xmark & \xmark & \xmark & \xmark \\
        \hline
        DFS\cite{josephson2011direct} & 2011 & CD & \xmark & \xmark & \xmark & \cmark & \xmark & \cmark & \cmark \\
        \hline
        DualFS\cite{2020-Wu-DualFS} & 2020 & OCSSD & \cmark & \cmark & \xmark & \xmark & \cmark & \xmark & \xmark \\
        \hline
         Ext4Stream\cite{2018-Rho-Fstream} & 2018 & MS-SSD & \xmark & \cmark & \xmark & \cmark & \xmark & \xmark & \xmark \\
        \hline
         XFStream\cite{2018-Rho-Fstream} & 2018 & MS-SSD & \xmark & \cmark & \xmark & \cmark & \xmark & \xmark & \xmark \\
        \hline
         SSDFS\cite{dubeyko2019ssdfs} & 2019 & OCSSD & \xmark & \cmark & WA \& SA & \cmark & \xmark & \xmark & \xmark \\
        \hline
         RFS\cite{lee2014refactored} & 2014 & CSC & \xmark & \xmark & \xmark & \cmark & \xmark & \cmark & \xmark \\
        \hline
         StageFS\cite{2019-Lu-Sync_IO_OCSSD} & 2019 & OCSSD & \cmark & \cmark & WA & \cmark & \xmark & \xmark & \cmark \\
        \hline
         ALFS\cite{2016-Lee-AMF} & 2016 & CFTL & \cmark & \cmark & \cmark & \cmark & \cmark & \cmark & \cmark \\
        \hline
         OrcFS\cite{2018-Yoo-OrcFS} & 2018 & CFW & \cmark & \cmark & \cmark & \cmark & \xmark & \cmark & \xmark \\
        \hline
        ParaFS~\cite{2016-Zhang-ParaFS} & 2016 & CFTL & \cmark & \cmark & \xmark & \cmark & \xmark & \cmark & \cmark \\
        \hline
        % From here on papers added from original FS Design Considerations classification
         NBFS\cite{2021-Qin-Atomic_Writes} (based on F2FS) & 2021 & OCSSD & \cmark & \cmark & WA & \cmark & \xmark & \xmark & \cmark \\
        \hline
    \end{tabular}
    \caption{Evaluated file systems for flash-based storage with custom FTL, OCSSD, custom device driver, and multi-streamed SSD under the flash challenges (Section~\ref{sec:opt_flash}) the file system solves. A \cmark\hspace{0.5mm} indicates that a file system implements a mechanism to handle the particular flash challenge, however it does not indicate to what extent it is handled or how successful it is handled. The type column indicates if the file system is built on OCSSD, custom FTL (abbreviation C-FTL), custom driver (abbreviation CD), custom storage controller (abbreviation CSC), custom FTL (CFTL), custom firmware (CFW), or multi-streamed SSD (abbreviation MS-SSD). If a file system does not have a name and/or it is an extension on an existing file system, the authors' names are used as the file system name. For instance, the GC column contains a \xmark\hspace{0.5mm} if the file system simply implements greedy or cost-benefit GC, and a \cmark\hspace{0.5mm} if a new GC policy or new mechanisms of managing GC are presented.}
    \label{tab:custom_integration_comparison}
\end{table*}

\textbf{Embedded Flash Integration-Level.} The last group of flash-based storage systems is the embedded systems, which give the file system full control over the underlying flash storage. This allows the file system to fully manage the storage, however comes at the cost of increased complexity. This section evaluates the file systems for flash-based embedded systems, analyzing how different file systems solve the particular design challenges of flash, leveraging the direct management of flash media. Again, we only discuss concepts that are not mentioned in the prior integration levels or methods of which further optimization is possible as a result of the embedded flash integration. Table~\ref{tab:embedded_integration_comparison} shows a comparison of the various file systems discussed in this section, evaluating the flash challenge(s) that a file system addresses. Note, also not all file systems in the table are discussed in this section if they do not contribute any new methods not already discussed.

\begin{table*}[!ht]
    \centering
    \begin{threeparttable}
        \begin{tabular}{||p{15mm}||p{8mm}|p{15mm}|p{5mm}|p{20mm}|p{15mm}|p{5mm}|p{15mm}|p{15mm}||}
            \hline 
            & & \multicolumn{7}{|c||}{Flash Challenge} \\
            \hline
            File \newline System & Year & Read/Write \newline Asymmetry & GC & I/O \newline Amplification & Flash \newline Parallelism & WL & Software \newline Stack & I/O \newline Scheduling \\
            \hline
            \hline
            CFFS\cite{2006-Lim-NAND_fs} & 2006 & \cmark & \cmark & \xmark & \xmark & \cmark & \xmark & \xmark \\
            \hline
            UBIFS\cite{hunter2008brief} & 2008 & \cmark & \cmark & SA & \xmark & \cmark & \xmark & \xmark \\
            \hline
            FlexFS\cite{2009-Sungjin-FlexFS} & 2009 & \cmark & \cmark & WA & \xmark & \cmark & \xmark & \xmark \\
            \hline
            ScaleFFS\cite{2008-Jung-ScaleFFS} & 2008 & \cmark & \xmark & WA & \xmark & \xmark & \xmark & \xmark \\
            \hline
            JFFS\tnote{†}\cite{woodhouse2001jffs} & 2001 & \xmark & \cmark & \xmark & \xmark & \cmark & \cmark & \cmark \\
            \hline
            EnFFiS\cite{park2013enffis} & 2013 & \cmark & \xmark & \xmark & \cmark & \cmark & \xmark & \xmark \\
            \hline
            FlogFS\cite{nahill2015flogfs} & 2015 & \cmark & \xmark & \xmark & \xmark & \xmark & \xmark & \xmark \\
            \hline
            RFS\cite{schildt2012contiki} & 2012 & \cmark & \cmark & \xmark & \xmark & \cmark & \xmark & \xmark \\
            \hline
            F2FS + SAC\cite{2015-Park-Suspend_Aware_cleaning-F2FS} & 2015 & \cmark & \cmark & \xmark & \cmark & \xmark & \xmark & \xmark \\
            \hline
            NANDFS\cite{2009-Zuck-NANDFS} & 2009 & \cmark & \cmark & WA & \xmark & \cmark & \xmark & \xmark \\
            \hline
            NAMU\cite{2009-Park-Multimedia_NAND_Fs} & 2009 & \cmark & \cmark & \xmark & \xmark & \cmark & \xmark & \xmark \\
            \hline
            LeCramFS\cite{2007-Hyun-LeCramFS} & 2007 & \cmark & \xmark & SA & \xmark & \xmark & \xmark & \xmark \\
            \hline
            F2FS + FPC\cite{2021-Ji-F2FS_compression} & 2021 & \cmark & \cmark & SA & \cmark & \xmark & \xmark & \xmark \\
            \hline
            NAFS\cite{2011-Park-Multi_NAND} & 2011 & \cmark & \xmark & WA & \cmark & \cmark & \xmark & \xmark \\
            \hline
            YAFFS\cite{manning2010yaffs,aleph2001yaffs,manning2002yaffs}\tnote{‡} & 2010 & \cmark & \cmark & \xmark & \cmark & \xmark & \xmark & \xmark \\
            \hline
            % From here on papers added from original FS Design Considerations classification
            LogFS\cite{engel2005logfs} & 2005 & \xmark & \cmark & SA & \xmark & \xmark & \xmark & \xmark \\
            \hline
            DFFS\cite{2008-Kim-DFFS} & 2008 & \cmark & \xmark & \xmark & \xmark & \xmark & \xmark & \xmark \\
            \hline
            RFFS\cite{park2006flash} & 2006 & \xmark & \xmark & \xmark & \xmark & \xmark & \xmark & \xmark \\
            \hline
            F2FS + ARS\cite{2020-Yang-F2FS_Framentation} & 2020 & \cmark & \cmark & WA \& RA & \cmark & \xmark & \xmark & \xmark \\
            \hline
            F2FS + RM-IPU\cite{2022-Lee-F2FS_Address_Remapping} & 2022 & \cmark & \cmark & WA \& RA & \cmark & \xmark & \xmark & \xmark \\
            \hline
        \end{tabular}
        \begin{tablenotes}
            \item[†] Focusing on JFFS V2.
            \item[‡] References include all YAFFS publications, however we focus on YAFFS2.
        \end{tablenotes}
    \end{threeparttable}
    \caption{Evaluated file systems for embedded systems with flash storage under the flash challenges (Section~\ref{sec:opt_flash}) the file system solves. A \cmark\hspace{0.5mm} indicates that a file system implements a mechanism to handle the particular flash challenge, however it does not indicate to what extent it is handled or how successful it is handled. If a file system does not have a name and/or it is an extension on an existing file system, the authors' names are used as the file system name. For instance, the GC column contains a \xmark\hspace{0.5mm} if the file system simply implements greedy or cost-benefit GC, and a \cmark\hspace{0.5mm} if a new GC policy or new mechanisms of managing GC are presented. Continued in Table~\ref{tab:embedded_integration_comparison_cont}.}
    \label{tab:embedded_integration_comparison}
\end{table*}

\begin{table*}[!ht]
    \centering
    \begin{tabular}{||p{15mm}||p{8mm}|p{15mm}|p{5mm}|p{20mm}|p{15mm}|p{5mm}|p{15mm}|p{15mm}||}
        \hline 
        & & \multicolumn{7}{|c||}{Flash Challenge} \\
        \hline
        File \newline System & Year & Read/Write \newline Asymmetry & GC & I/O \newline Amplification & Flash \newline Parallelism & WL & Software \newline Stack & I/O \newline Scheduling \\
        \hline
        \hline
        DeFFS\cite{2011-Lim-DeFFS} & 2011 & \cmark & \xmark & WA & \xmark & \xmark & \xmark & \xmark \\
        \hline
        O1FS\cite{2014-Hyunchan-O1FS} & 2014 & \cmark & \xmark & \xmark & \xmark & \xmark & \xmark & \xmark \\
        \hline
        ELOFS\cite{2022-Zhang-ELOFS,2020-Zhang-LOFFS} & 2022 & \xmark & \cmark & WA & \cmark & \cmark & \xmark & \xmark \\
        \hline
        MNFS\cite{kim2006mnfs} & 2006 & \xmark & \xmark & \xmark & \xmark & \xmark & \xmark & \xmark \\
        \hline
        Coffee FS\cite{2009-Tsiftes-Coffee_FS} & 2009 & \cmark & \cmark & \xmark & \xmark & \xmark & \xmark & \xmark \\
        \hline
    \end{tabular}
    \caption{Table~\ref{tab:embedded_integration_comparison} continued.}
    \label{tab:embedded_integration_comparison_cont}
\end{table*}

