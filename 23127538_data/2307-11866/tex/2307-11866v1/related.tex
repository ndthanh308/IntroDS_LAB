\section{Related Work}\label{sec:related_work}
With the growing adoption of flash based storage systems~\cite{daim2008forecasting,2008-Lee-Flash_in_enterprise_db}, there has been a plethora of proposed systems to optimize for flash characteristics. We focus on the existing file systems for flash storage, but other aspects such as key-value stores are another popular use case for flash storage.

\noindent \textbf{Flash Optimized File Systems.} Egger~\cite{egger2010file} provide a survey on the file systems for flash storage at the time (2010), comparing the file systems in key features important for flash storage. This includes the feature set of the file system, time complexity of operations such as mount time, and space requirements for memory. However, evaluated file systems are not limited to flash specific file systems. While the survey presents an insightful summary of flash file systems, it was published in 2010, thus limiting the number of available file systems significantly. Similarly, Gal and Toledo~\cite{gal2005algorithms} present a survey of algorithms and data structures for flash storage, which encompasses flash mappings and flash-specific file systems.

Jaffer~\cite{jafferevolution} provide a comparison of five file systems for flash storage, analyzing the feature set of each and discussing limitations. In particular, the author focuses on the evolution of design trends for flash storage over the past decades, where the earliest file system included in the comparison was presented in 1994. The author additionally includes a discussion on file system optimizations for data management with Streams~\cite{kang2014multi} and \as{ocssd}~\cite{picoli2020open,bjorling2017lightnvm}. While the author presents an insightful analysis into design trends for flash storage, the literature review is limited to only the five discussed file systems and does not differentiate in the file system application domain.

Dubeyko~\cite{dubeyko2019ssdfs} presents SSDFS, a file system designed for SSDs. Albeit not being a literature review of flash file systems, the author presents an extensive comparison of related work that proposes flash file systems. Including a discussion on flash-friendly and flash-oriented file systems, and summarizing the available methods for optimizing flash specific operations and storage management. While the discussion of flash file systems provides a comparison of flash file systems, it is similarly not differentiating between application domain of the file system. Munegowda et al.~\cite{munegowda2014evaluation} showcase a study into several file systems on Windows and Linux for SSD and flash devices. Furthermore, the authors present a comparison of features for the varying file systems, including flash support and FTL integration, as well as a high level performance comparison. However, the study is focused on the main adopted file systems, lacking the inclusion of less adopted flash file systems.

Ramasamy and Karantharaj~\cite{2014-Ramasamy-Flash_FS_Challenges} survey the challenges of building file systems for flash storage, including the performance implications, caching techniques, and implications of the FTL. The authors additionally discuss available solutions to the presented design challenges for storage systems on flash memory. Similarly, Di Carlo et al.~\cite{2011-DiCarlo-Design_Issues_FS_Flash} present a study into the design issues and challenges of flash memory file systems. The authors analyze the inherent implications of the flash storage from the type of cell type used and required wear leveling, error correction, and bad block management. A comparison of several available flash file systems at the time of publication (2011) is presented, with focus on the discussed design challenges.

\noindent \textbf{Flash Optimized Applications.} Not focusing of file systems for flash storage, there have been several surveys into flash characteristics. Luo and Carey~\cite{2018-Luo-LSM_Survey} present a survey into log-structured merge-tree (LSM-tree) design techniques for storage systems. With LSM-trees being a widely adopted and popular choice of database and key-value store design to optimize for flash storage, the presented survey showcases relevant flash storage optimizations for data management. Similarly, Doekemeijer and Trivedi~\cite{2022-Doekemeijer-KV_Flash_Survey} present a study into key-value stores optimized for flash storage, showcasing several techniques that can likewise be applied to file system design for flash storage.

\noindent \textbf{Flash Translation Layer.} Chung et al.~\cite{chung2009FTL_survey} provide a survey into the various FTL algorithms, discussing the design issues of various algorithms. Flash file system performance will largely depend on the FTL implementation for SSDs, making optimal FTL design an important aspect of optimizing file system performance. A similar survey on FTL algorithms is presented by Kwon et al.~\cite{kwon2011ftl}. As embedded devices and possible custom integrations require management of flash at the file system level, the concepts for efficient and performant FTL algorithms are applicable to file system level management of flash storage.
