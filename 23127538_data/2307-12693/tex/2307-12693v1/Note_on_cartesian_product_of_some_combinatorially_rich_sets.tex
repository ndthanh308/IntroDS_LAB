%% LyX 2.3.0 created this file.  For more info, see http://www.lyx.org/.
%% Do not edit unless you really know what you are doing.
\documentclass[oneside,english]{amsart}
\usepackage[T1]{fontenc}
\usepackage[latin9]{inputenc}
\usepackage{mathrsfs}
\usepackage{amstext}
\usepackage{amsthm}
\usepackage{amssymb}

\makeatletter
%%%%%%%%%%%%%%%%%%%%%%%%%%%%%% Textclass specific LaTeX commands.
\numberwithin{equation}{section}
\numberwithin{figure}{section}
\theoremstyle{plain}
\newtheorem{thm}{\protect\theoremname}
\theoremstyle{definition}
\newtheorem{defn}[thm]{\protect\definitionname}
\theoremstyle{plain}
\newtheorem{cor}[thm]{\protect\corollaryname}
\theoremstyle{plain}
\newtheorem{lem}[thm]{\protect\lemmaname}

\makeatother

\usepackage{babel}
\providecommand{\corollaryname}{Corollary}
\providecommand{\definitionname}{Definition}
\providecommand{\lemmaname}{Lemma}
\providecommand{\theoremname}{Theorem}

\begin{document}

\title{{\large{}Note on cartesian product of some combinatorially rich sets}}

\author{Pintu Debnath}

\address{{\large{}Pintu Debnath, Department of Mathematics, Basirhat College,
Basirhat -743412, North 24th parganas, West Bengal, India.}}

\email{{\large{}pintumath1989@gmail.com}}

\keywords{{\large{}Stone-\v{C}ech compactification, closed subsets of $\beta\mathbb{N}$,
Ramsay families.}}
\begin{abstract}
{\large{}D. De, N. Hindman and D. Strauss have introduced $C$-set
in \cite{key-5}, satisfying the strong central set theorem. Using
the algebraic structure of the Stone-\v{C}ech compactification of
a discrete semigroup, N. Hindman and D. Strauss proved that the product
of two $C$-sets is a $C$-set. In \cite{key-7}, S. Goswami has proved
the same result using elementary characterization of $C$-set. In
this paper we prove that the product of two $C$-sets is a $C$-set,
using dynamical characterization of $C$-set.}{\large\par}
\end{abstract}

\maketitle

\section*{{\large{}introduction}}

{\large{}A collection $\mathcal{F}\in\mathcal{P}(S)\setminus\left\{ \emptyset\right\} $
is upward hereditary if whenever $A\in\mathcal{F}$ and $A\subseteq B\subseteq S$
then it follows that $B\in\mathcal{F}$. A nonempty and upward hereditary
collection $\mathcal{F}\in\mathcal{P}(S)\setminus\left\{ \emptyset\right\} $
will be called a family. If $\mathcal{F}$ is a family, the dual family
$\mathcal{F}^{*}$ is given by,
\[
\mathcal{F}^{*}=\{E\subseteq S:\forall A\in\mathcal{F},E\cap A\neq\emptyset.\}
\]
A family $\mathcal{F}$ possesses the Ramsey property if whenever
$A\in\mathcal{F}$ and $A=A_{1}\cup A_{2}$ there is some $i\in\left\{ 1,2\right\} $
such that $A_{i}\in\mathcal{F}$.}{\large\par}

{\large{}In this paper we consider universal family $\mathcal{F}$,
every set of which is infinite or contains a configuration which is
defined for any semigroup. and $\mathcal{F}_{S}$ is the family $\mathcal{F}$
in the semigroup $S$. Let us give some definitions: }{\large\par}
\begin{defn}
{\large{}Let $\left(S,\cdot\right)$ be a discrete semigroup.}{\large\par}
\end{defn}

\begin{enumerate}
\item {\large{}A set $A\subseteq S$ is $IP$ set if and only if there exists
a sequence $\left\{ x_{n}\right\} _{n=1}^{\infty}$ in $S$ such that
$FP\left(\left\{ x_{n}\right\} _{n=1}^{\infty}\right)\subseteq A$.
Where 
\[
FP\left(\left\{ x_{n}\right\} _{n=1}^{\infty}\right)=\left\{ \prod_{n\in F}x_{n}:F\in\mathcal{P}_{f}\left(\mathbb{N}\right)\right\} 
\]
 and $\prod_{n\in F}x_{n}$ to be the product in increasing order.}{\large\par}
\item {\large{}A set $A\subseteq S$ is piecewise syndetic if and only if
\[
\left(\exists H\in\mathcal{P}_{f}\left(S\right)\right)\left(\forall F\in\mathcal{P}_{f}\left(S\right)\right)\left(\exists x\in S\right)\left(Fx\subseteq\cup_{t\in H}t^{-1}A\right)
\]
}{\large\par}
\item {\large{}Let $\left(S,\cdot\right)$ be a semigroup. Given $l\in\mathbb{N}$,
a set $B\subseteq S$ is a leangth $l$ progression if and only if
there exist , $a\in S^{2}$ and $d\in S$ such that 
\[
B=\left\{ a\left(1\right)d^{t}a\left(2\right):t\in\left\{ 1,2,\ldots,l\right\} \right\} .
\]
}{\large\par}
\item {\large{}A is a Prog-set if and only if for each $l\in\mathbb{N}$,
$A$ contains a length $l$ progression.}{\large\par}
\item {\large{}Let $\left(S,\cdot\right)$ be a semigroup. Given $l\in\mathbb{N}$,
a set $B\subseteq S$ is a leangth $l$ weak progression if and only
if there exist $m\in\mathbb{N}$, $a\in S^{m+1}$ and $d\in S$ such
that 
\[
B=\left\{ a\left(1\right)d^{t}a\left(2\right)d^{t}\cdots a\left(m\right)d^{t}a\left(m+1\right):t\in\left\{ 1,2,\ldots,l\right\} \right\} .
\]
}{\large\par}
\item {\large{}A is a weak Prog(wProg) set if and only if for each $l\in\mathbb{N}$,
$A$ contains a length $l$ weak progression.}{\large\par}
\item {\large{}Let $\left(S,\cdot\right)$ be a semigroup and let $A\subseteq S$.
Then $A$ is a $J$-set if and only if $F\in\mathcal{P}_{f}\left(S^{\mathbb{N}}\right)$,
there exist $m\in\mathbb{N}$, $a\in S^{m+1}$, and $t\left(1\right)<t\left(2\right)<\ldots<t\left(m\right)$
in $\mathbb{N}$ such that for each $f\in F$, 
\[
a\left(1\right)f\left(t\left(1\right)\right)a\left(2\right)\cdots a\left(m\right)f\left(t\left(m\right)\right)a\left(m+1\right)\in A
\]
}{\large\par}
\end{enumerate}
{\large{}Let $S$ and $T$ be semigroups, let $A$ be a $J$-set in
$S$ and $B$ be a $J$-set in $T$. Then from \cite[Theorem 2.11]{key-10},
we get $A\times B$ is a $J$-set in $S\times T$. We also know from\cite[Theorem 2.4]{key-7}
that the product of two piecewise syndetic-sets is a piecewise syndetic-set.
It is trivial from the definition of Prog-set that the product of
two Prog-set is also a Prog-set. But we do not know, whether the product
of two wProg-sets is wProg-set or not. To discuss further and express
our aim in this paper, we need a brief review of algebraic structure
of the Stone-\v{C}ech compactification.}{\large\par}

{\large{}The set $\{\overline{A}:A\subset S\}$ is a basis for the
closed sets of $\beta S$. The operation `$\cdot$' on $S$ can be
extended to the Stone-\v{C}ech compactification $\beta S$ of $S$
so that$(\beta S,\cdot)$ is a compact right topological semigroup
(meaning that for any \  is continuous) with $S$ contained in its
topological center (meaning that for any $x\in S$, the function $\lambda_{x}:\beta S\rightarrow\beta S$
defined by $\lambda_{x}(q)=x\cdot q$ is continuous). This is a famous
Theorem due to Ellis that if $S$ is a compact right topological semigroup
then the set of idempotents $E\left(S\right)\neq\emptyset$. A nonempty
subset $I$ of a semigroup $T$ is called a $\textit{left ideal}$
of $S$ if $TI\subset I$, a $\textit{right ideal}$ if $IT\subset I$,
and a $\textit{two sided ideal}$ (or simply an $\textit{ideal}$)
if it is both a left and right ideal. A $\textit{minimal left ideal}$
is the left ideal that does not contain any proper left ideal. Similarly,
we can define $\textit{minimal right ideal}$ and $\textit{smallest ideal}$.}{\large\par}

{\large{}Any compact Hausdorff right topological semigroup $T$ has
the smallest two sided ideal}{\large\par}

{\large{}
\[
\begin{array}{ccc}
K(T) & = & \bigcup\{L:L\text{ is a minimal left ideal of }T\}\\
 & = & \,\,\,\,\,\bigcup\{R:R\text{ is a minimal right ideal of }T\}.
\end{array}
\]
}{\large\par}

{\large{}Given a minimal left ideal $L$ and a minimal right ideal
$R$, $L\cap R$ is a group, and in particular contains an idempotent.
If $p$ and $q$ are idempotents in $T$ we write $p\leq q$ if and
only if $pq=qp=p$. An idempotent is minimal with respect to this
relation if and only if it is a member of the smallest ideal $K(T)$
of $T$. Given $p,q\in\beta S$ and $A\subseteq S$, $A\in p\cdot q$
if and only if the set $\{x\in S:x^{-1}A\in q\}\in p$, where $x^{-1}A=\{y\in S:x\cdot y\in A\}$.
See \cite{key-12} for an elementary introduction to the algebra of
$\beta S$ and for any unfamiliar details.}{\large\par}

{\large{}It will be easy to check that the family$\mathcal{F}$ has
the Ramsey property iff the family $\mathcal{F}^{*}$ is a filter.
For a family $\mathcal{F}$ with the Ramsey property, let $\beta(\mathcal{F})=\{p\in\beta S:p\subseteq\mathcal{F}\}$.
Then the following from\cite[Theorem 5.1.1]{key-3}.}{\large\par}
\begin{thm}
{\large{}Let $S$ be a discrete set. For every family $\mathcal{F\subseteq P}\left(S\right)$
with the Ramsay property, $\beta\left(\mathcal{F}\right)\subseteq\beta S$
is closed. Furthermore, $\mathcal{F}=\cup\beta\left(\mathcal{F}\right)$.
Also if $K\subseteq\beta S$ is closed, $\mathcal{F}_{K}=\left\{ E\subseteq S:\overline{E}\cap K\neq\emptyset\right\} $
is a family with the Ramsay property and $\overline{K}=\beta\left(\mathcal{F}_{K}\right)$.}{\large\par}
\end{thm}

{\large{}Let $S$ be a discrete semigroup, then for every family $\mathcal{F\subseteq P}\left(S\right)$
with the Ramsay property, $\beta\left(\mathcal{F}\right)\subseteq\beta S$
is closed. If $\beta\left(\mathcal{F}\right)$ be a subsemigroup of
$\beta S$, then $E\left(\beta\mathcal{F}\right)\neq\emptyset$. But
may not be subsemigroup. For example, let $\mathcal{F}=\mathcal{IP}$,
the family of IP- sets. It is easy to show that $\beta\left(\mathcal{F}\right)=\beta\left(\mathcal{IP}\right)=E\left(\beta S\right)$.
But $E\left(\beta S\right)$ is not a subsemigroup of $\beta S$.}{\large\par}
\begin{defn}
{\large{}Let $\mathcal{F}$ be a family with Ramsay property such
that $\beta(\mathcal{F})$ is a subsemigroup of $\beta S$ and $p$
be an idempotent in $\beta(\mathcal{F})$, then each member of $p$
is called essential $\mathcal{F}$-set. And $A\subset S$ is called
essential $\mathcal{F}^{\star}$-set if $A$ intersects with all essential
$\mathcal{F}$-sets.}{\large\par}
\end{defn}

{\large{}The family $\mathcal{F}$ is called left (right) shift-invariant
if for all $s\in S$ and all $E\in\mathcal{F}$ one has $sE\in\mathcal{F}(Es\in\mathcal{F})$.
The family $\mathcal{F}$ is called left (right) inverse shift-invariant
if for all $s\in S$ and all $E\in\mathcal{F}$ one has $s^{-1}E\in\mathcal{F}(Es^{-1}\in\mathcal{F})$.
From \cite[Theorem 5.1.2]{key-3}:}{\large\par}
\begin{thm}
{\large{}If $\mathcal{F}$ is a family having the Ramsey property
then $\beta\mathcal{F}\subseteq\beta S$ is a left ideal if and only
if $\mathcal{F}$ is left shift-invariant. Similarly, $\beta\mathcal{F}\subseteq\beta S$
is a right ideal if and only if $\mathcal{F}$ is right shift-invariant.}{\large\par}
\end{thm}

{\large{}From\cite[Theorem 5.1.10]{key-3}, we can identifie those
families $\mathcal{F}$ with Ramsey property for which $\beta\left(\mathcal{F}\right)$
is a subsemigroup of $\beta S$ . The condition is a rather technical
weakening of left shift-invariance.}{\large\par}
\begin{thm}
{\large{}Let $S$ be any semigroup, and let $\mathcal{F}$ be a family
of subsets of $S$ having the Ramsey property. Then the following
are equivalent:}{\large\par}

{\large{}(1) $\beta\left(\mathcal{F}\right)$ is a subsemigroup of
$\beta S$.}{\large\par}

{\large{}(2) $\mathcal{F}$ has the following property: If $E\subseteq S$
is any set, and if there is $A\in\mathcal{F}$ such that for all finite
$H\subseteq A$ one has $\left(\cap_{x\in H}x^{-1}E\right)\in\mathcal{F}$,
then $E\in\mathcal{F}$.}{\large\par}
\end{thm}

{\large{}Let us abbreviate the family of infinite sets as $\mathcal{IF}$,
the family of piecewise syndetic sets as $\mathcal{PS}$, the family
of Prog-sets as $\mathcal{P}$, the family of wProg-sets as $\mathcal{WP}$
and the family of $J$-sets as $\mathcal{J}$. The family of Prog-sets
and wProg-sets are Ramsey follows from \cite[Theorem2.6(8,9)]{key-8}.So
it is clear from the above discussion that $\beta\left(\mathcal{IF}\right)$,
$\beta\left(\mathcal{PS}\right)$, $\beta\left(\mathcal{P}\right)$,
$\beta\left(\mathcal{WP}\right)$ and $\beta\left(\mathcal{J}\right)$
are closed subsemigroups of $\beta\left(S\right)$.}{\large\par}

{\large{}From the above definition together with the abbreviations,
we get quasi central set is an essential $\mathcal{PS}$-set and $C$-set
is an essential $\mathcal{J}$-set. In this paper we prove that the
product of two essential $\mathcal{F}$-sets is an essential $\mathcal{F}$-set
under some certain conditions. We use algebraic, dynamical and elementary
techniques to prove our desired result.}{\large\par}

\section*{{\large{}Algebraic proof}}

{\large{}We start this section with the following theorem.}{\large\par}
\begin{thm}
{\large{}Let $S$ be a discrete set. For every family $\mathcal{F\subseteq P}\left(S\right)$
with the Ramsay property, $\beta\left(\mathcal{F}\right)\subseteq\beta S$,
$A$ is a $\mathcal{F}$-set if and only if $\overline{A}\cap\beta\left(\mathcal{F}\right)\neq\emptyset$.}{\large\par}
\end{thm}

\begin{proof}
{\large{}Now $\mathcal{F}$ is partition regular. So the theorem follows
from \cite[Theorem 3.11]{key-12} .}{\large\par}
\end{proof}
{\large{}From the above theorem, we get the following corollary.}{\large\par}
\begin{cor}
{\large{}Let $S$ be a discreate semigroup.}{\large\par}

{\large{}(a) $A$ is a Prog-set if and only if $\overline{A}\cap\beta\left(\mathcal{P}\right)\neq\emptyset$}{\large\par}

{\large{}(b) $A$ is a wProg-set if and only if $\overline{A}\cap\beta\left(\mathcal{WP}\right)\neq\emptyset$}{\large\par}

{\large{}(b) $A$ is infinite set if and only if $\overline{A}\cap\beta\left(\mathcal{\mathcal{IF}}\right)\neq\emptyset$}{\large\par}

{\large{}(c) $A$ is a $PS$-set if and only if $\overline{A}\cap\beta\left(\mathcal{\mathcal{PS}}\right)\neq\emptyset$}{\large\par}

{\large{}(e) $A$ is $J$-set if and only if $\overline{A}\cap\beta\left(\mathcal{\mathcal{J}}\right)\neq\emptyset$}{\large\par}
\end{cor}

{\large{}We now turn our attention to proving that product of two
essential $\mathcal{F}$-sets is essential $\mathcal{F}$-set with
a certain condition. To accomplish this we consider a universal family
$\mathcal{F}$ which is of the form $\mathcal{F}_{S}$ in the semigroup
$S$. We assume that $\beta\left(\mathcal{F}\right)$ is closed subsemigroup
of $\beta\left(S\right)$.}{\large\par}
\begin{lem}
{\large{}Let $S$ and $T$ be semigroup. let $p$ be an idempotent
in $\beta\left(\mathcal{F}_{S}\right)$ and let $q$ be an idempotent
in $\beta\left(\mathcal{F}_{T}\right)$. For all $A\in\mathcal{F}_{S}$
and $B\in\mathcal{F}_{T}$ , $A\times B\in\mathcal{F}_{S\times T}$
. Let $\widetilde{\imath}:\beta\left(S\times T\right)\rightarrow\beta S\times\beta T$
be the continuous extension of the identity function. Then $\widetilde{\imath}^{-1}\left[\left\{ \left(p,q\right)\right\} \right]\cap\beta\left(\mathcal{F}_{S\times T}\right)\neq\emptyset$. }{\large\par}
\end{lem}

\begin{proof}
{\large{}For all $A\in\mathcal{F}_{S}$and $B\in\mathcal{F}_{T}$,
it is given that $A\times B$ is a $\mathcal{F}_{S\times T}$-set.
So by Theorem 6, there is an ultrafilter $r$ on $S\times T$ such
that $A\times B\in r$ and every member of $r$ is a $\mathcal{F}_{S\times T}$-set,
that is, $r\in\overline{A\times B}\cap\beta\left(\mathcal{F}_{S\times T}\right)$. }{\large\par}
\end{proof}
\begin{thm}
{\large{}Let $\left(S,\cdot\right)$ and $\left(T,\cdot\right)$ be
semigroups, for all $A\in\mathcal{F}_{S}$and $B\in\mathcal{F}_{T}$
, $A\times B\in\mathcal{F}_{S\times T}$. Let $A$ be an essential
$\mathcal{F}_{S}$-set and $B$ be an essential $\mathcal{F}_{T}$-set,
then $A\times B$ is an essential $\mathcal{F}_{S\times T}$-set.}{\large\par}
\end{thm}

\begin{proof}
{\large{}Pick idempotents $p\in\beta\left(\mathcal{F}_{S}\right)$
and $q\in\beta\left(\mathcal{F}_{T}\right)$ such that $A\in p$ and
$B\in q$. By Lemma 8, $\widetilde{\imath}^{-1}\left[\left\{ \left(p,q\right)\right\} \right]\cap\beta\left(\mathcal{F}_{S\times T}\right)$
is a compact subsemigroup of $\beta\left(S\times T\right)$ so pick
an idempotent $r\in\widetilde{\imath}^{-1}\left[\left\{ \left(p,q\right)\right\} \right]\cap\beta\left(\mathcal{F}_{S\times T}\right)$.
Hence $A\times B$ is an essential $\mathcal{F}_{S\times T}$-set.}{\large\par}
\end{proof}

\section*{{\large{}Dynamical proof}}

{\large{}In the previous section we have proved our main result using
the algebraic characterizations of essential $\mathcal{F}$-set and
in this section we prove same result using the dynamical characterizations
of essential $\mathcal{F}$-set. To do so, we first establish the
dynamical characterizations of essential $\mathcal{F}$-set by using
results of \cite{key-13}. Now we start with the following definitions.}{\large\par}
\begin{defn}
{\large{}Let $S$ be a nonempty discrete semigroup and $\mathcal{K}$
is a filter on $S$.}{\large\par}

{\large{}(a) $\overline{\mathcal{K}}=\left\{ p\in\beta S:\mathcal{K}\subseteq p\right\} $.}{\large\par}

{\large{}(b) $\mathcal{L}\left(\mathcal{K}\right)=\left\{ A\subseteq S:S\backslash A\notin\mathcal{K}\right\} $.}{\large\par}
\end{defn}

{\large{}Following is the definition of dynamical system.}{\large\par}
\begin{defn}
{\large{}A pair $\left(X,\langle T_{s}\rangle_{s\in S}\right)$ is
a dynamical system if and only if it satisfies the following conditions:}{\large\par}

{\large{}(1) $X$ is a compact Hausdorff space.}{\large\par}

{\large{}(2) $S$ is a semigroup.}{\large\par}

{\large{}(3) $T_{s}:X\rightarrow X$ is continuous for every $s\in S$.}{\large\par}

{\large{}(4) For every $s,t\in S$ we have $T_{st}=T_{s}\circ T_{t}$.}{\large\par}
\end{defn}

{\large{}Now, we define the jointly $\mathcal{K}$-recurrent.}{\large\par}
\begin{defn}
{\large{}Let $\left(X,\langle T_{s}\rangle_{s\in S}\right)$ be a
dynamical system, $x$ and $y$ points in $X$, and $\mathcal{K}$
a filter on $S$. The pair $\left(x,y\right)$ is called jointly $\mathcal{K}$-
recurrent if and only if for every neighborhood $U$ of $y$ we have
$\left\{ s\in S:T_{s}\left(x\right)\in U\,\text{and}\,T_{s}\left(y\right)\in U\right\} \in\mathcal{L}\left(\mathcal{K}\right)$.}{\large\par}
\end{defn}

{\large{}We have the following important theorem \cite[Theorem 3.3]{key-13}
relating the above definitions.}{\large\par}
\begin{thm}
{\large{}Let $\left(S,\cdot\right)$ be a semigroup, let $\mathcal{K}$
be a filter on $S$ such that $\overline{\mathcal{K}}$ is a compact
subsemigroup of $\beta S$, and let $A\subseteq S$. Then $A$ is
a member of an idempotent in $\overline{\mathcal{K}}$ if and only
if there exists a dynamical system $\left(X,\langle T_{s}\rangle_{s\in S}\right)$
with points $x$ and $y$ in $X$ and there exists a neighbourhood
$U$ of $y$ such that the pair $\left(x,y\right)$ is jointly $\mathcal{K}$-
recurrent and $A=\left\{ s\in S:T_{s}\left(x\right)\in U\right\} $.}{\large\par}
\end{thm}

{\large{}The following theorem is essential to apply the above theorem.}{\large\par}
\begin{thm}
{\large{}Let $\left(S,\cdot\right)$ be a semigroup and $\mathcal{F}$
be family of $S$ with Ramsey property and $\mathcal{K}=\left\{ A\subseteq S:S\backslash A\,\text{is}\,\text{not}\,\text{a}\,\mathcal{F}\text{-}\text{set}\right\} $.
Then $\mathcal{K}$ is a filter on $S$ with $\beta\left(\mathcal{F}\right)=\overline{\mathcal{K}}$. }{\large\par}
\end{thm}

\begin{proof}
{\large{}We have $\mathcal{K}$ is nonempty as $\emptyset\notin\mathcal{F}$,
and is closed under supersets. The fact that $\mathcal{K}$ is closed
under finite intersection follows from the fact that $\mathcal{F}$
is Ramsey family. Hence $\mathcal{K}$ is a filter, $\mathcal{L\left(K\right)}=\left\{ A\subseteq S:A\,\text{is}\,\text{a}\,\mathcal{F}\text{-}\text{set}\right\} $.
Hence it follows from \cite[Theorem 3.11]{key-12} that $\beta\left(\mathcal{F}\right)=\overline{\mathcal{K}}$.}{\large\par}
\end{proof}
{\large{}Now, we are in the position of dynamical characterization
of essential $\mathcal{F}$-set. }{\large\par}
\begin{thm}
{\large{}Let $\left(S,\cdot\right)$ be a semigroup and $A\subseteq S$.
Then $A$ is an essential $\mathcal{F}$-set if and only if there
exists a dynamical system $\left(X,\langle T_{s}\rangle_{s\in S}\right)$
with points $x$ and $y$ in $X$ and there exists a neighbourhood
$U$ of $y$ such that 
\[
\left\{ s\in S:T_{s}\left(x\right)\in U\,\text{and}\,T_{s}\left(y\right)\in U\right\} 
\]
 is a $\mathcal{F}$-set and $A=\left\{ s\in S:T_{s}\left(x\right)\in U\right\} $.}{\large\par}
\end{thm}

\begin{proof}
{\large{}Let $\mathcal{K}=\left\{ A\subseteq S:S\backslash A\,\text{is}\,\text{not}\,\text{a}\,\mathcal{F}\text{-}\text{set}\right\} $
and by Theorem 14, $\beta\left(\mathcal{F}\right)=\overline{\mathcal{K}}$
and $\mathcal{L\left(K\right)}=\left\{ A\subseteq S:A\,\text{is}\,\text{a}\,\mathcal{F}\text{-}\text{set}\right\} $
and apply Theorem 13 to prove our statement.}{\large\par}
\end{proof}
{\large{}After getting dynamical characterization of essential $\mathcal{F}$-set,
we can prove the main theorem in dynamical approach.}{\large\par}
\begin{thm}
{\large{}Let $\left(S,\cdot\right)$ and $\left(T,\cdot\right)$ be
semigroups, for all $A\in\mathcal{F}_{S}$and $B\in\mathcal{F}_{T}$
, $A\times B\in\mathcal{F}_{S\times T}$. Let $A$ be an essential$\mathcal{F}_{S}$-set
and $B$ be an essential$\mathcal{F}_{T}$-set, then $A\times B$
is an essential$\mathcal{F}_{S\times T}$-set.}{\large\par}
\end{thm}

\begin{proof}
{\large{}Given that $A$ be an essential$\mathcal{F}_{S}$-set, then
there exists a dynamical system $\left(X^{A},\langle T_{s}^{A}\rangle_{s\in S}\right)$
with points $x^{A}$ and $y^{A}$ in $X$ and there exists a neighbourhood
$U_{A}$ of $y^{A}$ such that 
\[
\left\{ s\in S:T_{s}^{A}\left(x^{A}\right)\in U^{A}\,\text{and}\,T_{s}^{A}\left(y^{A}\right)\in U^{A}\right\} 
\]
 is a $\mathcal{F}_{S}$-set and $A=\left\{ s\in S:T_{s}^{A}\left(x^{A}\right)\in U^{A}\right\} $.
It is also given that $B$ be an essential $\mathcal{F}_{T}$-set,
then there exists a dynamical system $\left(X^{B},\langle T_{s}^{B}\rangle_{s\in S}\right)$
with points $x^{B}$ and $y^{B}$ in $X^{A}$ and there exists a neighbourhood
$U_{B}$ of $y^{B}$ such that 
\[
\left\{ t\in T:T_{t}^{B}\left(x^{B}\right)\in U^{B}\,\text{and}\,T_{t}^{B}\left(y^{B}\right)\in U^{B}\right\} 
\]
 is a $\mathcal{F}_{T}$-set and $A=\left\{ t\in T:T_{t}^{B}\left(x^{B}\right)\in U^{B}\right\} $. }{\large\par}

{\large{}To show that $A\times B$ is an essential $\mathcal{F}_{S\times T}$-set,
consider the dynamical system $\left(X^{A}\times X^{B},\langle T_{s}^{A}\times T_{t}^{B}\rangle_{\left(s,t\right)\in S\times T}\right)$. }{\large\par}

{\large{}Now, }{\large\par}

{\large{}$\left\{ \left(s,t\right)\in S\times T:\left(T_{s}^{A}\left(x^{A}\right),T_{t}^{B}\left(x^{B}\right)\right)\in U^{A}\times U^{B}\,\text{and}\,\left(T_{s}^{A}\left(y^{A}\right),T_{t}^{B}\left(y^{B}\right)\right)\in U^{A}\times U^{B}\right\} $}{\large\par}

{\large{}$=\left\{ s\in S:T_{s}^{A}\left(x^{A}\right)\in U^{A}\,\text{and}\,T_{s}^{A}\left(y^{A}\right)\in U^{A}\right\} \times\left\{ t\in T:T_{t}^{B}\left(x^{B}\right)\in U^{B}\,\text{and}\,T_{t}^{B}\left(y^{B}\right)\in U^{B}\right\} $. }{\large\par}

{\large{}Hence, 
\[
\left\{ \left(s,t\right)\in S\times T:T_{s}^{A}\times T_{t}^{B}\left(x^{A},x^{B}\right)\in U^{A}\times U^{B}\,\text{and}\,T_{s}^{A}\times T_{t}^{B}\left(y^{A},y^{B}\right)\in U^{A}\times U^{B}\right\} 
\]
 is a $\mathcal{F}_{S\times T}$-set.}{\large\par}

{\large{}Now $\left\{ \left(s,t\right)\in S\times T:\left(T_{s}^{A}\left(y^{A}\right),T_{t}^{B}\left(y^{B}\right)\right)\in U^{A}\times U^{B}\right\} $}{\large\par}

{\large{}$=$$\left\{ \left(s,t\right)\in S\times T:T_{s}^{A}\times T_{t}^{B}\left(y^{A},y^{B}\right)\in U^{A}\times U^{B}\right\} $}{\large\par}

{\large{}$=\left\{ s\in S:T_{s}^{A}\left(x^{A}\right)\in U^{A}\right\} \times\left\{ t\in T:T_{t}^{B}\left(x^{B}\right)\in U^{B}\right\} $}{\large\par}

{\large{}$=A\times B$. Hence $A\times B$ is an essential $\mathcal{F}_{S\times T}$-set.}{\large\par}
\end{proof}

\section*{{\large{}Elementary proof}}

{\large{}In the last two sections, we have proved our main result
using algebraic and dynamical characterizations of essential $\mathcal{F}$-set.
As we promised to prove the same theorem by elementary approach, we
need elementary characterization of essential $\mathcal{F}$-set.
Although elementary characterization of quasi central-sets and $C$-sets
are known from \cite[Theorem 3.7]{key-9} and \cite[Theorem 2.7]{key-11}
respectively. Since quasi central sets and $C$ sets are coming from
by the setting of essential $\mathcal{F}$-set and this fact confines
the fact that essential $\mathcal{F}$-sets might have elementary
characterization and elementary characterization of essential $\mathcal{F}$-sets
are known from\cite[Theorem 5]{key-4}.}{\large\par}

{\large{}Let $\omega$ be the first infinite ordinal and each ordinal
indicates the set of all it's predecessor. In particular, $0=\emptyset,$
for each $n\in\mathbb{N},\:n=\left\{ 0,1,...,n-1\right\} $.}{\large\par}
\begin{defn}
{\large{}(a) If $f$ is a function and $dom\left(f\right)=n\in\omega$,
then for all $x$, $f^{\frown}x=f\cup\left\{ \left(n,x\right)\right\} $.}{\large\par}

{\large{}(b) Let $T$ be a set functions whose domains are members
of $\omega$. For each $f\in T$, $B_{f}\left(T\right)=\left\{ x:f^{\frown}x\in T\right\} .$}{\large\par}

{\large{}We get the following theorem from\cite[Theorem 5]{key-4}
which play a vital role in this section.}{\large\par}
\end{defn}

\begin{thm}
{\large{}Let $\left(S,.\right)$ be a semigroup, and assume that $\mathcal{F}$
is a family of subsets of $S$ with the Ramsay property such that
$\beta\left(\mathcal{F}\right)$ is a subsemigroup of $\beta S$.
Let $A\subseteq S$. Statements (a), (b) and (c) are equivalent.}{\large\par}

{\large{}(a) $A$ is an essential $\mathcal{F}$-set.}{\large\par}

{\large{}(b)There is a non empty set $T$ of function such that}{\large\par}
\end{thm}

{\large{}(i) For all $f\in T$,$\text{domain}\left(f\right)\in\omega$
and $rang\left(f\right)\subseteq A$;}{\large\par}

{\large{}(ii) For all $f\in T$ and all $x\in B_{f}\left(T\right)$,
$B_{f^{\frown}x}\subseteq x^{-1}B_{f}\left(T\right)$; and}{\large\par}

{\large{}(iii) For all $F\in\mathcal{P}_{f}\left(T\right)$, $\cap_{f\in F}B_{f}(T)$
is a $\mathcal{F}$-set.}{\large\par}
\begin{thm}
{\large{}(c) There is a downward directed family $\left\langle C_{F}\right\rangle _{F\in I}$
of subsets of $A$ such that}{\large\par}
\end{thm}

{\large{}(i) for each $F\in I$ and each $x\in C_{F}$ there exists
$G\in I$ with $C_{G}\subseteq x^{-1}C_{F}$ and}{\large\par}

{\large{}(ii) for each $\mathcal{F}\in\mathcal{P}_{f}\left(I\right),\,\bigcap_{F\in\mathcal{F}}C_{F}$
is a $\mathcal{F}$-set. }{\large\par}

{\large{}Now get an elementary proof of our main theorem, exactaly
in the same fashion what the author did in \cite[Theorem 2.4]{key-7}.}{\large\par}
\begin{thm}
{\large{}Let $\left(S,\cdot\right)$ and $\left(T,\cdot\right)$ be
semigroups, for all $A\in\mathcal{F}_{S}$and $B\in\mathcal{F}_{T}$
, $A\times B\in\mathcal{F}_{S\times T}$. Let $A$ be an essential
$\mathcal{F}_{S}$-set and $B$ be an essential $\mathcal{F}_{T}$-set,
then $A\times B$ is an essential $\mathcal{F}_{S\times T}$-set.}{\large\par}
\end{thm}

\begin{proof}
{\large{}Given that $A$ be an essential $\mathcal{F}_{S}$-set, then
$\left\langle C_{F}\right\rangle _{F\in I}$ be as guaranteed by theorem
18(c) for $A$. It is also given that $B$ be an essential $\mathcal{F}_{T}$-set,
then $\left\langle D_{G}\right\rangle _{G\in J}$ be as guaranteed
by theorem 18(c) for $B$. Direct $I\times J$ by agreeing that $\left(F,G\right)\geq\left(F^{\prime},G^{\prime}\right)$
if and only if $F\geq F^{\prime}$ and $G\geq G^{\prime}$. We claim
that $\left\langle C_{F}\times D_{G}\right\rangle _{\left(F,G\right)\in I\times J}$
is as required by theorem 18(c) to show that $A\times B$ is an essential
$\mathcal{F}_{S\times T}$-set. Let $\left(F,G\right)\in I\times J$
and let $\left(x,y\right)\in C_{F}\times D_{G}$. Pick $H\in I$ and
$K\in J$ such that $\text{\ensuremath{C_{H}\subseteq x^{-1}C_{F}}}$
and $D_{K}\subseteq y^{-1}D_{G}$. Then $\left(H,K\right)\in I\times J$
, $C_{H}\times D_{K}\subseteq\left(x,y\right)^{-1}\left(C_{F}\times D_{G}\right)$
and $C_{H}\times D_{K}$ and $C_{F}\times D_{G}$ are $\mathcal{F}_{S\times T}$
by the given condition.}{\large\par}
\end{proof}
\begin{thebibliography}{10}
{\large{}\bibitem[1]{key-1} V. Bergelson and T. Downarowicz, Large
sets of integers and hierarchy of mixing properties of measure preserving
systems, Colloq. Math. 110 (2008), 117-150.}{\large\par}

{\large{}\bibitem[2]{key-2} V. Bergelson and N. Hindman, Nonmetrizable
topological dynamics and Ramsey theory, Trans. Amer. math. soc. 320(1990),
293320.}{\large\par}

{\large{}\bibitem[3]{key-3} C.Christopherson, Closed ideals in the
Stone-\v{C}ech compactification of a countable semigroup and some
application to ergodic theory and topological dynamics, PhD thesis,
Ohio State University, 2014.}{\large\par}

{\large{}\bibitem[4]{key-4} D. De, P. Debnath and S. Goswami, Elementary
characterization of essential $\mathscr{F}-$sets and its combinatorial
consequences, Semigroup Forum volume 104, pages 45--57 (2022).}{\large\par}

{\large{}\bibitem[5]{key-5} D. De, N. Hindman and D. Strauss, A new
and stronger central set theorem, Fundamenta mathematicae, 199 (2008):
155-175.}{\large\par}

{\large{}\bibitem[6]{key-6} H. Furstenberg. Recurrence in ergodic
theory and combinatorial number theory, Princeton university press,
Princeton, NJ, 1981.}{\large\par}

{\large{}\bibitem[7]{key-7} S. Goswami, Cartesian product of some
combinatorially rich sets, INTEGERS 20(2020)\#A64.}{\large\par}

{\large{}\bibitem[8]{key-8} N. Hindman, L. L. Jones and D. Strauss,
The ralationship among many notions of largness for subsets of a semigroup,
Semigroup Forum 99 (2019), 9-31.}{\large\par}

{\large{}\bibitem[9]{key-9} N. Hindman, A. Maleki and D. Strauss.
Central sets and their combinatorial characterization. Journal of
Combinatorial Theory Series A, archive Volume 74 Issue 2, May 1996
Pages 188-208.}{\large\par}

{\large{}\bibitem[10]{key-10} N. Hindman and D. Strauss, Cartesian
product of sets satisfying the central sets theorem, Topology Proceedings
35 (2010).}{\large\par}

{\large{}\bibitem[11]{key-11} N. Hindman and D. Strauss, A simple
characterization of sets satisfying the Central Sets Theorem, New
York J. Math. 15 (2009), 405-413.}{\large\par}

{\large{}\bibitem[12]{key-12} N.Hindman, D.Strauss, Algebra in the
Stone-\v{C}ech Compactification: theory and applications,2nd edition,
Walter de Gruyter \& Co., Berlin, 2012. }{\large\par}

{\large{}\bibitem[13]{key-13} John H. Johnson, A dynamical characterization
of $C$ sets, arxive:1112.0715v1 {[}math.DS{]} 4 Dec 2011.}{\large\par}

{\large{}\bibitem[14]{key-14} Jain Li, Dynamical Characterization
of $C-$sets and its applications, Fundamental Mathematicae 216(3):259-286.}{\large\par}
\end{thebibliography}

\end{document}
