\section{Challenge Results}
\label{sec:challenge_results}

\subsection{Evaluation Metrics}
In the RoboDepth Challenge, the two most conventional metrics were adopted: 1) error rate, including \texttt{Abs Rel}, \texttt{Sq Rel}, \texttt{RMSE}, and \texttt{log RMSE}; and 2) accuracy, including $\delta_1$, $\delta_2$, and $\delta_3$.

\noindent\textbf{Error Rate}.
The Relative Absolute Error (\texttt{Abs Rel}) measures the relative difference between the pixel-wise ground-truth (\texttt{gt}) and the prediction values (\texttt{pred}) in a depth prediction map $D$, as calculated by the following equation:
\begin{equation}
\text{Abs Rel} = \frac{1}{|D|}\sum_{pred\in D}\frac{|gt - pred|}{gt}~.
\end{equation}
The Relative Square Error (\texttt{Sq Rel}) measures the relative square difference between \texttt{gt} and \texttt{pred} as follows:
\begin{equation}
\text{Sq Rel} = \frac{1}{|D|}\sum_{pred\in D}\frac{|gt - pred|^2}{gt}~.
\end{equation}
\texttt{RMSE} denotes the Root Mean Square Error (in meters) of a scene (image), which can be calculated as $\sqrt{\sum|gt - pred|^2}$; while \texttt{log RMSE} is the log-normalized version of \texttt{RMSE}, \textit{i.e.}, $\sqrt{\sum|\log(gt) - \log(pred)|^2}$.


\noindent\textbf{Accuracy}.
The $\delta$ metric is the depth estimation accuracy given the threshold:
\begin{equation}
\delta_t = \frac{1}{|D|}|\{\ pred\in D | \max{(\frac{gt}{pred}, \frac{pred}{gt})< 1.25^t}\}| \times 100\%~,
\end{equation}
where $\delta_1 = \delta<1.25, \delta_2 = \delta<1.25^2, \delta_3 = \delta<1.25^3$ are the three conventionally used accuracy scores among prior works \cite{godard2019monodepth2,lidepthtoolbox2022}.

Following the seminar work MonoDepth2 \cite{godard2019monodepth2}, the \texttt{Abs Rel} metric was selected as the major indicator to compare among submissions in the first track of the RoboDepth Challenge.

Based on the Monocular-Depth-Estimation-Toolbox\footnote{\url{https://github.com/zhyever/Monocular-Depth-Estimation-Toolbox}.}, the $\delta_1$ score was used to rank different submissions in the second track of the RoboDepth Challenge.

\subsection{Track \# 1 Results}

In the first track of the RoboDepth Challenge, we received $684$ valid submissions. The top-performing teams in this track include \texttt{OpenSpaceAI}, \texttt{USTC-IAT-United}, and \texttt{YYQ}. The shortlisted submissions are shown in Table~\ref{tab:track1_results}; the complete results can be found on our evaluation server.

Specifically, the team \texttt{OpenSpaceAI} achieved a \texttt{Abs Rel} score of $0.121$, which is $0.100$ higher than the baseline MonoDepth2 \cite{godard2019monodepth2}. They also ranked first on the \texttt{log RMSE}, $\delta_1$, and $\delta_3$ metrics. Other top-ranked submissions are from: the team \texttt{USTC-IAT-United} (\texttt{Abs Rel}$=0.123$, $\delta_1=0.861$), team \texttt{YYQ} (\texttt{Abs Rel}$=0.123$, $\delta_1=0.848$), team \texttt{zs\_dlut} (\texttt{Abs Rel}$=0.124$, $\delta_1=0.852$), and team \texttt{UMCV} (\texttt{Abs Rel}$=0.124$, $\delta_1=0.847$). We refer readers to the solutions presented in Section~\ref{sec:track1} for additional comparative and ablation results and more detailed analyses.

\begin{table*}[t]
\caption{Leaderboard of Track \# 1 (robust self-supervised depth estimation) in the RoboDepth Challenge. The \textbf{best} and \underline{second best} scores of each metric are highlighted in \textbf{bold} and \underline{underline}, respectively. Only entries better than the baseline are included in this table. In Track \# 1, MonoDepth2 \cite{godard2019monodepth2} was adopted as the baseline. See our evaluation server for the complete results.}
\centering\scalebox{0.78}{
\begin{tabular}{c|c|cccc|cccc}
    \toprule
    \textbf{\#} & \textbf{Team Name} & \textbf{Abs Rel~$\downarrow$} & \textbf{Sq Rel~$\downarrow$} & \textbf{RMSE~$\downarrow$} & \textbf{log RMSE~$\downarrow$} & $\delta<1.25$~$\uparrow$ & $\delta<1.25^2$~$\uparrow$ & $\delta<1.25^3$~$\uparrow$
    \\\midrule\midrule
    \textcolor{robo_red}{\textbf{1}} & \textcolor{robo_red}{OpenSpaceAI} & $\mathbf{0.121}$ & $0.919$ & $4.981$ & $\mathbf{0.200}$	& $\mathbf{0.861}$ & \underline{$0.953$} & $\mathbf{0.980}$
    \\
    \textcolor{robo_blue}{\textbf{2}} & \textcolor{robo_blue}{USTC-IAT-United} & \underline{$0.123$} & $0.932$ & $\mathbf{4.873}$ & $0.202$ & $\mathbf{0.861}$ & $\mathbf{0.954}$ & \underline{$0.979$}
    \\
    \textcolor{robo_green}{\textbf{3}} & \textcolor{robo_green}{YYQ} & \underline{$0.123$} & $0.885$ & $4.983$ & \underline{$0.201$} & $0.848$ & $0.950$ & \underline{$0.979$}
    \\\midrule
    4 & zs\_dlut & $0.124$ & $0.899$ & $4.938$ & $0.203$ & $0.852$ & $0.950$ & \underline{$0.979$}
    \\
    5 & UMCV & $0.124$ & $\mathbf{0.845}$ & \underline{$4.883$} & $0.202$ & $0.847$ & $0.950$ & $\mathbf{0.980}$
    \\
    6 & THU\_ZS & $0.124$ & $0.892$ & $4.928$ & $0.203$ & $0.851$ & $0.951$ & $\mathbf{0.980}$
    \\
    7 & THU\_Chen & $0.125$ & \underline{$0.865$} & $4.924$ & $0.203$ & $0.846$ & $0.950$ & $\mathbf{0.980}$
    \\
    8 & seesee & $0.126$ & $0.990$ & $4.979$ & $0.206$ & $0.857$ & $0.952$ & $0.978$
    \\
    9 & namename & $0.126$ & $0.994$ & $4.950$ & $0.204$ & \underline{$0.860$} & \underline{$0.953$} & \underline{$0.979$}
    \\
    10 & USTCxNetEaseFuxi & $0.129$ & $0.973$ & $5.100$ & $0.208$ & $0.846$ & $0.948$ & $0.978$
    \\
    11 & Tutu & $0.131$ & $0.972$ & $5.085$ & $0.207$ & $0.835$ & $0.946$ & \underline{$0.979$}
    \\
    12 & Cai & $0.133$ & $1.017$ & $5.282$ & $0.214$ & $0.837$ & $0.945$ & $0.976$
    \\
    13 & Suzally & $0.133$ & $1.023$ & $5.285$ & $0.215$ & $0.835$ & $0.943$ & $0.976$
    \\
    14 & waterch & $0.137$ & $0.904$ & $5.276$ & $0.214$ & $0.813$ & $0.941$ & \underline{$0.979$}
    \\
    15 & hust99 & $0.139$ & $1.057$ & $5.302$ & $0.220$ & $0.826$ & $0.939$ & $0.975$
    \\
    16 & panzer & $0.141$ & $0.953$ & $5.429$ & $0.221$ & $0.804$ & $0.936$ & $0.976$
    \\
    17 & lyle & $0.142$ & $0.981$ & $5.590$ & $0.225$ & $0.806$ & $0.936$ & $0.974$
    \\
    18 & SHSCUMT & $0.142$ & $1.064$ & $5.155$ & $0.215$ & $0.821$ & $0.943$ & $0.977$
    \\
    19 & hanchenggong & $0.142$ & $1.064$ & $5.155$ & $0.215$ & $0.821$ & $0.943$ & $0.977$
    \\
    20 & king & $0.160$ & $1.230$ & $5.927$ & $0.244$ & $0.769$ & $0.921$ & $0.966$
    \\
    21 & xujianyao & $0.172$ & $1.340$ & $6.177$ & $0.258$ & $0.743$ & $0.910$ & $0.963$
    \\
    22 & Wenhui\_Wei & $0.172$ & $1.340$ & $6.177$ & $0.258$ & $0.743$ & $0.910$ & $0.963$
    \\
    23 & jerryxu & $0.192$ & $1.594$ & $6.506$ & $0.279$ & $0.709$ & $0.895$ & $0.956$
    \\\midrule
    - & MonoDepth2 \cite{godard2019monodepth2} & $0.221$ & $1.988$ & $7.117$ & $0.312$ & $0.654$ & $0.859$ & $0.938$
\\\bottomrule
\end{tabular}
}
\label{tab:track1_results}
\end{table*}

\subsection{Track \# 2 Results}

In the second track of the RoboDepth Challenge, we received $453$ valid submissions. The top-performing teams in this track include \texttt{USTCxNetEaseFuxi}, \texttt{OpenSpaceAI}, and \texttt{GANCV}. The shortlisted submissions are shown in Table~\ref{tab:track2_results}; the complete results can be found on our evaluation server.

Specifically, the team \texttt{USTCxNetEaseFuxi} achieved a $\delta_1$ score of $0.940$, which is $0.285$ higher than the baseline DepthFormer-SwinT \cite{li2022depthformer}. They also ranked first on the \texttt{Abs Rel} and \texttt{log RMSE} metrics. Other top-ranked submissions are from: the team \texttt{OpenSpaceAI} (\texttt{Abs Rel}$=0.095$, $\delta_1=0.928$), team \texttt{GANCV} (\texttt{Abs Rel}$=0.104$, $\delta_1=0.898$), team \texttt{shinonomei} (\texttt{Abs Rel}$=0.123$, $\delta_1=0.861$), and team \texttt{YYQ} (\texttt{Abs Rel}$=0.125$, $\delta_1=0.851$). We refer readers to the solutions presented in Section~\ref{sec:track2} for additional comparative and ablation results and more detailed analyses.
