\section{Challenge Summary}
\label{sec:challenge_summary}

\subsection{Overall Statistics}
This is the first edition of the RoboDepth Challenge. The official evaluation servers\footnote{We built servers on CodaLab. More details of this platform are at \url{https://codalab.lisn.upsaclay.fr}.} of this competition were launched on 01 January 2023. During the five-month period of competition, $226$ teams registered on our servers; among them, $66$ teams attempted to make submissions. Finally, we received $1137$ valid submissions and selected six winning teams (three teams for each track) and three innovation prize awardees. The detailed information of the winning teams and innovation prize awardees is shown in Table~\ref{tab:summary} and Table~\ref{tab:summary_innovation}, respectively.

% Figure environment removed

% Figure environment removed

\subsection{Track \# 1: Robust Self-Supervised Depth Estimation}
\noindent\textbf{Evaluation Server}. The first track of the RoboDepth Challenge was hosted at \url{https://codalab.lisn.upsaclay.fr/competitions/9418}. The participants were requested to submit their depth disparity maps to our server for evaluation. Such depth predictions were expected to be generated by a learning-based model, in a self-supervised learning manner, trained on the official \textit{training} split of the KITTI dataset \cite{geiger2012kitti}.

\noindent\textbf{Statistics}. In the first track of the RoboDepth Challenge, a total number of $137$ teams registered at our evaluation server. We received $684$ valid submissions during the competition period. The top-three best-performing teams are \texttt{OpenSpaceAI}, \texttt{USTC-IAT-United}, and \texttt{YYQ}. Additionally, we selected the teams \texttt{Ensemble} and \texttt{Scent-Depth} as the innovation prize awardees of this track.


\subsection{Track \# 2: Robust Supervised Depth Estimation}
\noindent\textbf{Evaluation Server}. The second track of the RoboDepth Challenge was hosted at \url{https://codalab.lisn.upsaclay.fr/competitions/9821}. The participants were requested to submit their depth disparity maps to our server for evaluation. Such depth predictions were expected to be generated by a learning-based model, in a fully-supervised learning manner, trained on the official \textit{training} split of the NYUDepth V2 dataset \cite{silberman2012nyu2}.

\noindent\textbf{Statistics}. In the second track of the RoboDepth Challenge, a total number of $89$ teams registered at our evaluation server. We received $453$ valid submissions during the competition period. The top-three best-performing teams are \texttt{USTCxNetEaseFuxi}, \texttt{OpenSpaceAI}, and \texttt{GANCV}. Additionally, we selected the team \texttt{AIIA-RDepth} as the innovation prize awardee of this track.

\begin{table*}[t]
\caption{Summary of the top-performing teams in each track of the RoboDepth Challenge.}
\centering\scalebox{1}{
\begin{tabular}{c|p{5cm}|p{5cm}}
\toprule
\textbf{Rank} & \textbf{\#1: Robust Self-Supervised MDE} & \textbf{\#2: Robust Supervised MDE}
\\\midrule\midrule
\multirow{13}{*}{\textcolor{robo_blue}{\textbf{1st Place}}} & \textbf{Team Name} & \textbf{Team Name}
\\
& \textcolor{robo_blue}{OpenSpaceAI} & \textcolor{robo_blue}{USTCxNetEaseFuxi}
\\
\cmidrule{2-3}
& \textbf{Team Members} & \textbf{Team Members}
\\
& Ruijie Zhu$^1$, Ziyang Song$^1$, Li Liu$^1$, Tianzhu Zhang$^{1,2}$ & Jun Yu$^1$, Mohan Jing$^1$, Pengwei Li$^1$, Xiaohua Qi$^1$, Cheng Jin$^2$, Yingfeng Chen$^2$, Jie Hou$^2$
\\
\cmidrule{2-3}
& \textbf{Affiliations} & \textbf{Affiliations}
\\
& $^1$University of Science and Technology of China, $^2$Deep Space Exploration Lab & $^1$University of Science and Technology of China, $^2$NetEase Fuxi
% \\
% \cmidrule{2-3}
% & \textbf{Approach} & \textbf{Approach}
% \\
% & IRUDepth with MPViT as depth encoder and PoseNet for camera poses and depth maps with AugMix& <...>
\\\cmidrule{2-3}
& \textbf{Contact} $\textrm{\Letter}$ & \textbf{Contact} $\textrm{\Letter}$
\\
& \texttt{ruijiezhu@mail.ustc.edu.cn} & \texttt{USTC\_IAT\_United@163.com}
\\\midrule\midrule
\multirow{17}{*}{\textcolor{robo_red}{\textbf{2nd Place}}} & \textbf{Team Name} & \textbf{Team Name}
\\
& \textcolor{robo_red}{USTC-IAT-United} & \textcolor{robo_red}{OpenSpaceAI}
\\
\cmidrule{2-3}
& \textbf{Team Members} & \textbf{Team Members}
\\
& Jun Yu$^1$, Xiaohua Qi$^1$, Jie Zhang$^2$, Mohan Jing$^1$, Pengwei Li$^1$, Zhen Kan$^1$, Qiang Ling$^1$, Liang Peng$^3$, Minglei Li$^3$, Di Xu$^3$, Changpeng Yang$^3$ & Li Liu$^1$, Ruijie Zhu$^1$, Ziyang Song$^1$, Tianzhu Zhang$^{1,2}$
\\
\cmidrule{2-3}
& \textbf{Affiliations} & \textbf{Affiliations}
\\
& $^1$University of Science and Technology of China, $^2$Central South University, $^3$Huawei Cloud Computing Technology Co., Ltd & $^1$University of Science and Technology of China, $^2$Deep Space Exploration Lab
\\
\cmidrule{2-3}
& \textbf{Contact} $\textrm{\Letter}$ & \textbf{Contact} $\textrm{\Letter}$
\\
& \texttt{USTC\_IAT\_United@163.com} & \texttt{liu\_li@mail.ustc.edu.cn}
\\\midrule\midrule
\multirow{11}{*}{\textcolor{robo_green}{\textbf{3rd Place}}} & \textbf{Team Name} & \textbf{Team Name}
\\
& \textcolor{robo_green}{YYQ} & \textcolor{robo_green}{GANCV}
\\
\cmidrule{2-3}
& \textbf{Team Members} & \textbf{Team Members}
\\
& Yuanqi Yao$^1$, Gang Wu$^1$, Jian Kuai$^1$, Xianming Liu$^1$, Junjun Jiang$^1$ & Jiamian Huang$^1$, Baojun Li$^1$
\\
\cmidrule{2-3}
& \textbf{Affiliations} & \textbf{Affiliations}
\\
& $^1$Harbin Institute of Technology & $^1$Individual Researcher
\\
\cmidrule{2-3}
& \textbf{Contact} $\textrm{\Letter}$ & \textbf{Contact} $\textrm{\Letter}$
\\
& \texttt{yuanqiyao@stu.hit.edu.cn} & \texttt{huang176368745@gmail.com}
\\\bottomrule
\end{tabular}
}
\label{tab:summary}
\end{table*}
\begin{table*}[t]
\caption{Summary of innovation prize awardees (across two tracks) in the RoboDepth Challenge.}
\centering\scalebox{1}{
\begin{tabular}{p{4.1cm}|p{4.1cm}|p{4.1cm}}
\toprule
\textbf{Team 1} & \textbf{Team 2} & \textbf{Team 3}
\\\midrule\midrule
\textbf{Team Name} & \textbf{Team Name} & \textbf{Team Name}
\\
\textcolor{robo_blue}{Ensemble} & \textcolor{robo_red}{Scent-Depth} & \textcolor{robo_green}{AIIA-RDepth}
\\\midrule
\textbf{Track} & \textbf{Track} & \textbf{Track}
\\
\#1: Robust Self-Supervised MDE & \#1: Robust Self-Supervised MDE & \#2: Robust Supervised MDE
\\\midrule
\textbf{Team Members} & \textbf{Team Members} & \textbf{Team Members}
\\
Jiale Chen$^1$, Shuang Zhang$^1$ & Runze Chen$^{1,2}$, Haiyong Luo$^1$, Fang Zhao$^2$, Jingze Yu$^{1,2}$ & Sun Ao$^1$, Gang Wu$^1$, Zhenyu Li$^1$, Xianming Liu$^1$, Junjun Jiang$^1$
\\\midrule
\textbf{Affiliations} & \textbf{Affiliations} & \textbf{Affiliations}
\\
$^1$Tsinghua University & $^1$Beijing University of Posts and Telecommunications, $^2$Institute of Computing Technology, Chinese Academy of Sciences & $^1$Harbin Institute of Technology
\\\midrule
\textbf{Contact} $\textrm{\Letter}$ & \textbf{Contact} $\textrm{\Letter}$ & \textbf{Contact} $\textrm{\Letter}$
\\
cjl20@mails.tsinghua.edu.cn & chenrz925@bupt.edu.cn & sunao\_sz@163.com
\\\bottomrule
\end{tabular}
}
\label{tab:summary_innovation}
\end{table*}

\subsection{The RoboDepth Workshop}
We hosted the online workshop at ICRA 2023 on 02 June 2023 after the competition was officially concluded. Six winning teams and three innovation prize awardees attended and presented their approaches.

The video recordings of this workshop are publicly available at \url{https://www.youtube.com/watch?v=mYhdTGiIGCY&list=PLxxrIfcH-qBGZ6x_e1AT2_YnAxiHIKtkB}.

The slides used can be downloaded from \url{https://ldkong.com/talks/icra23_robodepth.pdf}.

\subsection{Terms and Conditions}
The RoboDepth Challenge is made freely available to academic and non-academic entities for non-commercial purposes such as research, teaching, scientific publications, or personal experimentation. Permission is granted to use the related public resources given that the participants agree:
\begin{itemize}
    \item That the data in this challenge comes “AS IS”, without express or implied warranty. Although every effort has been made to ensure accuracy, the challenge organizing team is not responsible for any errors or omissions.
    \item That the participants may not use the data in this challenge or any derivative work for commercial purposes as, for example, licensing or selling the data, or using the data with the purpose to procure a commercial gain.
    \item That the participants include a reference to RoboDepth (including the benchmark data and the specially generated data for this academic challenge) in any work that makes use of the benchmark. For research papers, please cite our preferred publications as listed on our webpage and GitHub repository.
\end{itemize}