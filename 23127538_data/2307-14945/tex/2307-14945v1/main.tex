\documentclass[prl,superscriptaddress,showpacs,longbibliography,reprint]{revtex4-2} 

\usepackage{packages}

\newcommand{\id}{\mathds{1}}
\newcommand{\tubingen}{Institut f\"{u}r Theoretische Physik,  Universit\"{a}t T\"{u}bingen, Auf der Morgenstelle 14, 72076 T\"{u}bingen, Germany}
\newcommand{\trento}{Physics Department, University of Trento, via Sommarive, 14 I-38123 Trento, Italy}
\newcommand{\nottingham}{School of Physics and Astronomy and Centre for the Mathematics and Theoretical Physics of Quantum Non-Equilibrium Systems, The University of Nottingham, Nottingham, NG7 2RD, United Kingdom}

\begin{document}
\raggedbottom

\title{Large-scale universality in Quantum Reaction-Diffusion from Keldysh field theory}

\author{Federico Gerbino}
\email{federicogerbino99@gmail.com}
\affiliation{\tubingen}
\affiliation{\trento}

\author{Igor Lesanovsky}
\affiliation{\tubingen}
\affiliation{\nottingham}

\author{Gabriele Perfetto}
\affiliation{\tubingen}

\begin{abstract}
We consider the quantum reaction-diffusion dynamics in $d$ spatial dimensions of a Fermi gas subject to binary annihilation reactions $A+A \to \emptyset$. These systems display collective nonequilibrium long-time behavior, which is signalled by an algebraic decay of the particle density. Building on the Keldysh formalism, we devise a field theoretical approach for the reaction-limited regime, where annihilation reactions are scarce. By means of a perturbative 
expansion of the dissipative interaction, we derive a description in terms of a large-scale universal kinetic equation.
Our approach shows how the time-dependent generalized Gibbs ensemble assumption, which is often employed for treating low-dimensional nonequilibrium systems, emerges from systematic diagrammatics. It also allows to exactly compute --- for arbitrary spatial dimension --- the decay exponent of the particle density. The latter is based on the large-scale description of the quantum dynamics and it differs from the mean-field prediction even in dimension larger than one. We moreover consider spatially inhomogeneous setups involving an external potential. In confined systems the density decay is accelerated towards the mean-field algebraic behavior, while for deconfined scenarios the power-law decay is replaced by a slower non-algebraic decay. 
\end{abstract}

\maketitle

%%%%%%%%%%%%%%%%%%%%%%%%%%%%%%%%%%%%%%%%%%%%%%%%%%%%%%%%%%%%%%%%%%%%
\textbf{Introduction.---} 
The identification of large-scale universal behavior in many-body nonequilibrium systems is a demanding task. 
Reaction-diffusion (RD) systems \cite{henkel2008non, hinrichsen2000non, vladimir1997nonequilibrium}, where particles diffuse and react upon meeting, 
are ideal systems for the investigation of dynamical universal behavior. 
For example, for binary annihilation reactions $A+A\to \emptyset$, the late time decay of the particle density takes a universal power-law form. In the ``diffusion-limited'' regime  \cite{OVCHINNIKOV78, kang84scaling, kang84fluct, kang85,spounge88,torney1983diffusion,privman94,toussaint1983particle,racz1985}, where diffusion is weak, the origin of this dynamical behavior are spatial density fluctuations. Here, mean-field approaches cannot be applied and field-theorical and renormalization group analyses \cite{doi1976, peliti1985,peliti1986renormalisation, Lee1994, cardy1996, mattis1998,grassberger1980fock,tauber2002dynamic, tauber2005, tauber2014critical} correctly predict the observed power-law. The mean-field approximation is, however, valid in more than one dimension and/or in the ``reaction-limited'' regime of fast hopping mixing \cite{hinrichsen2000non,vladimir1997nonequilibrium,kang84fluct,fastdiffusion1992}. 

% Figure environment removed

In quantum many-body systems large-scale properties are even harder to uncover than in the classical realm, already in the one-dimensional case, since they entail the simulations of large sizes and long times \cite{van_horssen2015,gillman2019,gillman20,jo2021,carollo2019,carollo2022nonequilibrium,carollo2022quantum}. In this regard, quantum RD systems have moved into the focus of attention. They follow simple dynamical rules \cite{lossth1,Rosso2022,lossth3,lossth4,lossth5,lossth6,lossth7,lossth8,lossth9,lossth10}, which connect to cold-atomic experiments \cite{lossexp1, lossexp2, lossexp3, lossexp4, lossexp5, lossexp6, lossexp7}, and they allow for novel forms of particle-density decay beyond mean-field. This has been shown in Refs.~\cite{lossth1,Rosso2022,lossth3,lossth6,perfetto22,perfetto23,lossth11,lossexpF1,lossexpF2,lossexpF3} for one-dimensional systems in the reaction-limited regime. In this limit, analytical predictions can be obtained under the assumption that the systems relaxes to a time-dependent generalized Gibbs ensemble (TGGE) \cite{tGGE1,tGGE2,tGGE3,tGGE4}. The systematic classification of these universal properties, and the study of their dependence on the spatial dimensions $d$, requires, however, the development of a field-theory.

In this manuscript, we accomplish this by exploiting the Keldysh path integral representation of the quantum master equation \cite{schwinger1961brownian, keldysh1965, kamenev2011, altland2010,sieberer2016,tonielli2016,thompson2023field}. We investigate as a paradigmatic example the Fermi gas in $d$ spatial dimensions subject to binary annihilation reactions $A+A\to\emptyset$. From the Keldysh-field theory, we perform a diagrammatic expansion of the dissipative interaction vertices. At leading order in space-time derivatives, this expansion acquires the universal form of a kinetic Boltzmann equation. In $d=1$, this systematic derivation naturally yields the TGGE ansatz for the reaction-limited regime. Our results therefore bridge the gap between the much-employed TGGE relaxation assumption, diagrammatic expansions and large-scale dynamics in dissipative quantum systems. Crucially, from the field-theory description, we exactly compute the density decay exponent in arbitrary dimensions, which is found to deviate from the mean-field prediction even in $d>1$. This result is in contrast with the classical case and it is rooted into the large-scale universal description of the underlying quantum dynamics. We also consider the case of inhomogeneous systems where we study quenches of a trapping potential confining %that confines
the fermions. For a quench from a double to single well potential, we find an acceleration of the particle decay, which diverts the decay exponent towards the mean-field one. For a trap-release quench, we, instead, find a qualitatively different scenario: 
the algebraic decay first slows down on an intermediate time window, and then it gets replaced at long times by a slower non-algebraic decay.

%%%%%%%%%%%%%%%%%%%%%%%%%%%%%%%%%%%%%%%%%%%%%%%%%%%%%%%%%%%%%%%%%%%%
\textbf{Quantum RD Keldysh action.---}
The dynamics of the considered Fermi gas in $d$ spatial dimensions is governed by the quantum master equation \cite{lindblad1976generators,gorini1976completely,breuer2002,gardiner2004quantum} with a Lindblad map $\mathcal{L}$
\begin{equation}
\label{eq:master}
    \dot{\rho}(t)=\mathcal{L}[\rho(t)]=-\frac{i}{\hbar}[H,\rho(t)]+\mathcal{D}[\rho(t)].
\end{equation}
The Hamiltonian $H$ describes coherent free motion, see Fig.~\ref{fig:diffusion}(a), which is the quantum mechanical counterpart of classical diffusion. We further allow for the presence of an external trapping potential $V(\vec{x})$:
\begin{equation}
    H=\int d^d\vec{x} \,  \psi^\dagger(\vec{x})[-D\nabla^2+V(\vec{x})]\psi(\vec{x}).
\label{eq:Hamiltonian_diff_potential}
\end{equation}
Here $D=\hbar^2/(2m)$, while $\psi$, $\psi^\dagger$ are fermionic field operators satisfying canonical anticommutation relations $\{\psi(\vec{x}),\psi^\dagger(\vec{x}')\}\,=\,\delta(\vec{x}-\vec{x}')$. The dissipator $\mathcal{D}[\rho]$ embodies irreversible reaction processes
\begin{align}
    \mathcal{D}[\rho(t)]=\sum_{\alpha}\int d^d\vec{x}\,\Big[\,& L_\alpha(\vec{x})\rho(t)L^\dagger_\alpha(\vec{x})\nonumber\\
    &-\frac{1}{2}\{\,L^\dagger_\alpha(\vec{x})L_\alpha(\vec{x}),\rho(t)\}\Big],
\end{align}
with $\alpha=1,2\dots d$. We focus on binary annihilation reactions $A+A\,\rightarrow\,\emptyset$, see Fig.~\ref{fig:diffusion}(a), modelled by the jump operators
\begin{equation}
L_\alpha(\vec{x})=\sqrt{\Gamma}\,\psi(\vec{x})\, \partial_{x_\alpha}\psi(\vec{x}).
\label{eq:annihilation_continuum_limit}
\end{equation}
The constant $\Gamma$ (units: $\mbox{length}^{d+2}/\mbox{time}$) characterizes the annihilation reactions. In the Supplemental Material \cite{SM}, the jump operator \eqref{eq:annihilation_continuum_limit} is obtained by taking the continuum limit of nearest neighbours annihilation. The latter is the natural annihilation decay to consider for fermionic particles, where onsite reactions are forbidden. 

Nonequilibrium universal behavior manifests in the power-law decay of the density $n(x,t)=\braket{\psi^{\dagger}(\vec{x}) \psi(\vec{x})}_t$ in time. Here, we characterise this decay in the reaction-limited regime, where $\hbar n(x,t) \Gamma/D \ll 1$ throughout the dynamics. In this limit, reactions are weak and the density slowly changes in time. We further allow for weak space inhomogeneities due the presence of the trapping potential $V(\vec{x})$, which we assume to vary on macroscopic length scales. In this limit, the quasiparticle dispersion relation $\epsilon_{k}(\vec{x})=D\vec{k}^2+V(\vec{x})$ of $H$ in Eq.~\eqref{eq:Hamiltonian_diff_potential}, with $\vec{k}$ the momentum, is locally modified by the external potential. In this regard, the quantum reaction-limited regime is particularly suitable for the study of large space-time universality, i.e., taking place over long times and large distances, in the density decay. 

In order to do this, we exploit the Keldysh quantum field theory description \cite{kamenev2011, altland2010,sieberer2016,thompson2023field,tonielli2016} of the open quantum RD dynamics \eqref{eq:master}-\eqref{eq:annihilation_continuum_limit} \cite{SM}. A general feature of Keldysh field theory is the doubling of the $\psi,\,\bar{\psi}$ fields into four fields --- $\psi_+,\,\bar{\psi}_+$ and $\psi_-,\,\bar{\psi}_-$ --- evolving along a ``forward'' ($+$) and a ``backward'' ($-$) contour of the considered time interval, respectively, as sketched in Fig.~\ref{fig:diffusion}(b).
\enquote{Plus} and \enquote{minus} fields are usually rewritten in terms of Keldysh-rotated fields \cite{keldysh1965,SM, larkin1975} $\phi_1,\,\bar{\phi}_1,\,\phi_2,\,\bar{\phi}_2$.
The Keldysh partition function $Z(t)\!\!=\!\!\mathrm{tr}[\rho(t)] \!\!=\!\! \int \mathbf{D}[\phi_1,\bar{\phi}_1,\phi_2,\bar{\phi}_2]\,\mathrm{exp}\{iS[\phi_1,\bar{\phi}_1,\phi_2,\bar{\phi}_2]\}$
includes full information on the system's microscopic dynamics. The action $S=S_{\mathrm{diff}}+S_{\mathcal{D}}$ is composed of two sectors. A quadratic sector $S_{\mathrm{diff}}$ describes coherent motion \eqref{eq:Hamiltonian_diff_potential}:
\begin{equation}
    S_{\mathrm{diff}}=\int d^dx \, dt'\,[\bar{\phi}_{1}(G_0^R)^{-1}\phi_{1}+\bar{\phi}_{2}(G_0^A)^{-1}\phi_{2}\,],
\label{eq:diffusion_action}
\end{equation}
with $(G_0^{R/A})^{-1}\,=\,i\partial_{t'}+(D\nabla^2-V(\vec{x}))/\hbar\pm i\delta$ the inverse retarded/advanced bare propagator, respectively. The bare Keldysh Green's function $G_0^{K}\equiv -i\braket{\phi_1\bar{\phi}_2}$ associated to $S_{\mathrm{diff}}$ is a regularization factor. The Green's function $\braket{\phi_2 \bar{\phi}_1}=0$ is generically zero as a consequence of trace preservation. The retarded/advanced propagator $G_0^{R/A}$ (with $V(\vec{x})=0$) in $S_{\mathrm{diff}}$ has a structure similar to the classical RD quadratic counterpart  \cite{peliti1985,peliti1986renormalisation, Lee1994, cardy1996, tauber2005,tauber2002dynamic, tauber2014critical}, the difference between the two being in the factor $i$ in front of the time derivative in \eqref{eq:diffusion_action}. The second part $S_{\mathcal{D}}$ of the action contains the interaction vertices of the theory due to annihilation reactions \eqref{eq:annihilation_continuum_limit}:
\begin{align}
    S_{\mathcal{D}} &=\frac{i\Gamma}{4}\int d^dx \, dt'  \Big[ 2(\vec{\nabla}\Bar{\phi}_1\Bar{\phi}_2 + \vec{\nabla}\Bar{\phi}_2\Bar{\phi}_1)\cdot(\phi_1\vec{\nabla}\phi_2+ \nonumber \\
    &+\phi_2\vec{\nabla}\phi_1)+(\vec{\nabla}\Bar{\phi}_1\Bar{\phi}_2 + \vec{\nabla}\Bar{\phi}_2\Bar{\phi}_1)\cdot
    (\phi_1\vec{\nabla}\phi_1+\phi_2\vec{\nabla}\phi_2) -\nonumber \\
    &- (\vec{\nabla}\Bar{\phi}_1\Bar{\phi}_1 + \vec{\nabla}\Bar{\phi}_2\Bar{\phi}_2)\cdot
    (\phi_1\vec{\nabla}\phi_2+\phi_2\vec{\nabla}\phi_1)\Big]\,.
\label{eq:interaction_vertex}    
\end{align}
All the interaction vertices are quartic in the fields, differently from the classical RD field theory, where both cubic and quartic interaction vertices are present. Furthermore, spatial gradients of the fields appear as a consequence of fermionic statistics. 

%%%%%%%%%%%%%%%%%%%%%%%%%%%%%%%%%%%%%%%%%%%%%%%%%%%%%%%%%%%%%%%%%%%%
\textbf{Kinetic equation.---}
In the reaction-limited regime $\hbar n \Gamma/D\ll 1$, the dressed Green's functions $\hat{G}$ can be rewritten in terms of the bare Green's functions $\hat{G}_0$ as the interaction vertices \eqref{eq:interaction_vertex} are expanded perturbatively around the quadratic sector \eqref{eq:diffusion_action}. The sum of all internal one-particle-irreducible contributions to the Feynman diagrams results in the entries of the ``self-energy'' Keldysh-space matrix $\hat{\Sigma}$ \cite{schwinger1961brownian, keldysh1965,kamenev2011, altland2010,sieberer2016,tonielli2016,thompson2023field,buchhold2015kinetic,Duval2023}. The ensuing Dyson equation $(\hat{G}_0^{-1}-\hat{\Sigma})\,\circ\,\hat{G}\,=\,\mathds{1}$ is pictorially shown in Fig.~\ref{fig:diffusion}(c). 

The Keldysh component of the Dyson equation for $G^K(\vec{x}_1,t_1,\vec{x}_2,t_1)$ determines the kinetic equation \cite{SM}. The large-scale universal limit of this equation is best extracted by performing a Fourier transform in the relative space [time] variable $\vec{x}'=\vec{x}_1-\vec{x}_2$ [$t'=t_1-t_2$], $G^K(\vec{x},t,\vec{k},\epsilon)$  \footnote{In the whole manuscript, for lightness of notation, we use the same symbol for a function and its Wigner transform. The two functions can be distinguished from the corresponding arguments.}, i.e., the so-called ``Wigner transform''\cite{weyl1931theory,WignerRev,case2008wigner}. The set of Wigner center-of-mass coordinates $\vec{x}\,=\,(\vec{x_1}+\vec{x_2})/2$ [$t=(t_1+t_2)/2$], as well as momentum $\vec{k}$ and energy $\epsilon$, characterise the effective macroscopic evolution of Green's functions.

% Figure environment removed

% Figure environment removed

In the regime of weak space-time modulations, $G^K(\vec{x},t,\vec{k},\epsilon)$ is a slow function of its arguments and hence it is possible to perform a derivative expansion of the kinetic equation, akin to the Moyal expansion carried on in hydrodynamics \cite{Moyal1,Moyal2,Moyal3,Moyal4,Moyal5,Moyal6}. Variations in time are also slow in the reaction-limited regime $\hbar n \Gamma/D\ll 1$. To leading order in the interaction $\Gamma$, indeed, the quasi-particle lifetime $\sim (n\Gamma)^{-1}$, given by the width of $\mbox{Im}(G^{R})$, is much larger than the coherent time scale set by $(D/\hbar)^{-1}$. This implies that $G^{K}(\vec{x},t,\vec{k},\epsilon)$ is a sharply peaked function of $\epsilon$ around $\epsilon_{k}(\vec{x})$ and therefore quasiparticles are well-defined even throughout the time evolution [see Fig.~\ref{fig:diffusion}(c)]. This allows to identify the equal-time Keldysh Green's function $G^K(\vec{x},t,\vec{k},t)$ as the emergent large-scale degree of freedom:
\begin{equation} \label{eq:GK}
    iG^K(\vec{x},\vec{k},t,t)\,=\,1\,-\,2\,n(\vec{x},\vec{k},t).
\end{equation}
Here, $n(\vec{x},\vec{k},t)$ is the one-body Wigner function, i.e., the semiclassical phase-space $(\vec{x},\vec{k})$ occupation function \cite{wigner3F,wigner1F,wigner2F}. At leading order in the space time-derivatives, it obeys the quantum Boltzmann-like equation 
\begin{equation} 
\label{eq:boltzmann}
\begin{split}
    \Big[\,\partial_t &+ \vec{v}_g(\vec{k}) \cdot \vec{\nabla}_x-\vec{\nabla}_x V/\hbar\cdot \vec{\nabla}_k \,\Big]n(\vec{x},\vec{k},t) =\\
    &=-\Gamma \int \frac{d^dq}{(2\pi)^d}(\vec{k}-\vec{q})^2 \,n(\vec{x},\vec{k},t)\,n(\vec{x},\vec{q},t).
\end{split}
\end{equation}
Crucially, at first order in $\Gamma$, we find that $\hat{\Sigma}$ contributes via purely imaginary terms, which determine the r.h.s., named collision integral. The latter is computed in terms of tadpole Feynman diagrams [depicted in Fig.~\ref{fig:diffusion}(c)], which we consider at lowest order in the derivatives $\vec{\nabla}$.
The appearing factor $(\vec{k}-\vec{q})^2$ stems from the fermionic statistics. Moreover, the dispersion relation $\epsilon_k(\vec{x})$ ($\vec{v}_g(\vec{k})\,=\,2 D \vec{k}/\hbar$ is the group velocity) is not renormalized. In $d=1$, the r.h.s. of Eq.~\eqref{eq:boltzmann} has the same form derived in Refs.~\cite{lossth1,Rosso2022,perfetto22,perfetto23,lossth11} assuming the systems relaxes to a TGGE state in-between consecutive reactions. Our Keldysh field-theory analysis thus shows how the TGGE ansatz naturally emerges as the large-scale limit of tadpole Feynman diagrams generated by the interaction vertices \eqref{eq:interaction_vertex}. At the same time, it allows us to consider higher dimensional systems.

%%%%%%%%%%%%%%%%%%%%%%%%%%%%%%%%%%%%%%%%%%%%%%%%%%%%%%%%%%%%%%%%%%%%
\textbf{Homogeneous decay in $d$ dimensions.---}
For homogeneous initial states and no trapping potential $V(\vec{x})=0$, the Wigner function $n(\vec{x},\vec{k},t)$ reduces to the momentum-occupation function $n(\vec{k},t)$. We consider a Fermi-sea initial state, with equally populated modes up to some Fermi momentum $k_d^F\,\sim\,n_0^{1/d}$ and a total initial density $n_0$. The asymptotics of the particle density decay can then be worked out for generic $d$ by reducing the $d$-dimensional problem to a radial one, i.e., $n(\vec{k},t)=n(k,t)$. Moreover, it is convenient to introduce the adimensional re-scaled density $\tilde{n}\,=\,n/n_0$ and time $\tilde{t}\,=\,n_0^{1+2/d}\Gamma t$. The long-time asymptotic for $\tilde{n}(\tilde{t})$ is given by the power law \footnote{$\alpha_d$ is a geometric factor arising form angular integration. $\alpha_d\,=\,1$ for $d$ odd, $\alpha_d\,=\,\sqrt{\pi/2}$ for $d$ even.}
\begin{equation}
\label{eq:homo}
    \tilde{n}(\tilde{t})\,\sim\,\Bigg[\,\frac{(\alpha_d \, \Theta_d)^2}{2^d (d+1)^d (2\pi)^2}\,\Bigg]^{\frac{1}{d+1}}\,\tilde{t}^{-\frac{d}{d+1}},
\end{equation}
with $\Theta_d$ the $d$-dimensional solid angle. The solution of the homogeneous Boltzmann equation for $d=1,2,3$ from the Fermi-sea initial state is shown in Fig.~\ref{fig:homo}. In $d=1$, the decay exponent is $1/2$ in agreement with the TGGE prediction \cite{lossth1,Rosso2022,perfetto22, perfetto23,lossth11}. We find that the algebraic decay is different from the mean-field $\tilde{t}^{-1}$ even for $d>1$ and it approaches the latter only for $d\to \infty$. This result is surprising and fundamentally different from the classical $A+A\to\emptyset$ RD dynamics. Therein, non mean-field algebraic decay is possible only in $d=1$ in the diffusion limited regime $\hbar \Gamma n/D \sim 1$ \cite{peliti1985,peliti1986renormalisation, Lee1994, cardy1996,tauber2005,tauber2002dynamic, tauber2014critical}, as a consequence of spatial fluctuations. Conversely, the different dimensional dependence of the exponent in Eq.~\eqref{eq:homo} does not emerge due to spatial fluctuations but from the universal large-scale limit \eqref{eq:boltzmann} of the quantum dynamics. 

%%%%%%%%%%%%%%%%%%%%%%%%%%%%%%%%%%%%%%%%%%%%%%%%%%%%%%%%%%%%%%%%%%%%
\textbf{Inhomogeneous decay in one dimension.---}
To show that our approach allows to treat inhomogeneous settings, we consider a one-dimensional quantum quench of the trapping potential from the prequench $V_0(x)=Ax^4/4 - m\omega^2 x^2/2$ double-well to the postquench harmonic $V(x)=m\omega^2 x^2/2$ form. A similar setting has been consider in Ref.~\cite{Rosso2022}, where also the prequench potential $V_0(x)$ is harmonic. The initial condition is the local-density approximation of the ground state of the Hamiltonian $H$ in Eq.~\eqref{eq:Hamiltonian_diff_potential} with potential $V_0(x)$. We set the initial particle number to $N_0$ and we rescale variables through the parameters of the harmonic potential $V(x)$: for the particle number $\tilde{N}=N/N_0$, time $\tilde{t}=t\Gamma\big(2N_0 m \omega/\hbar\big)^{3/2}/(2\pi)$, space $\tilde{x}=x\sqrt{m \omega/(2N_0 \hbar)}$, momentum $\tilde{k}=k\sqrt{\hbar/(2N_0 m \omega)}$ and density $\tilde{n}=n \sqrt{\hbar/(2N_0 m \omega)}$. The adimensional parameter $\Omega=2\pi (D/\hbar N_0)^{3/2}/(\Gamma \sqrt{\omega})$ quantifies the interplay between reactions ($\Gamma)$, diffusion ($\hbar/m$) and confinement ($\omega$). 

In Fig.~\ref{fig:plots}(a), the decay of $\tilde{N}$ as a function of $\tilde{t}$ is shown. The decay accelerates periodically as $\Omega$ is increased since particles bounce off the potential walls %at about $\tilde{t}\,=\,\pi/\Omega$ 
and gather up at the centre of the well, as shown in Fig.~\ref{fig:plots}(b) for the density $\tilde{n}(\tilde{x},\tilde{t})$. As a consequence of such breathing motion all decay profiles converge from the short-time $\tilde{N}(\tilde{t})\sim \tilde{t}^{-1/2}$ algebraic behavior towards the mean-field asymptotic decay $\tilde{N}(\tilde{t})\sim \tilde{t}^{-1}$. 

Next, we consider the long-time decay for the deconfinement dynamics of an harmonic trap-release quench of the Fermi gas from $V_0(x)=m \omega^2 x^2/2$ to $V(x)=0$. We rescale also in this case variables with respect to the harmonic potential $V_0(x)$ parameters. In Fig.~\ref{fig:plots}(c) the decay of $\tilde{N}$ in time $\tilde{t}$ is reported. One first observes \cite{SM} an approximate algebraic decay $\tilde{N}(\tilde{t})\sim \tilde{t}^{-\xi}$, with an exponent $\xi$ continuously decreasing as $\Omega$ is increased (from the value $\xi=1/2$ at $\Omega=0$). At longer times the algebraic decay is replaced by a much slower non-algebraic decay. As the particle density $\tilde{n}(\tilde{x},\tilde{t})$ spreads  in free space, shown in Fig.~\ref{fig:plots}(d), the most relevant reactions, indeed, take place between particles with same momenta. The fermionic statistics, manifest in the factor $(\tilde{k}-\tilde{q})^2$ in Eq.~\eqref{eq:boltzmann}, thereby suppress these reactions and determine the unexpectedly slow large-scale behavior of Figs.~\ref{fig:plots}(c)-(d).

%%%%%%%%%%%%%%%%%%%%%%%%%%%%%%%%%%%%%%%%%%%%%%%%%%%%%%%%%%%%%%%%%%%%
\textbf{Summary.---} 
We provided a Keldysh field-theory description of quantum RD dynamics of binary annihilation $A+A\to\emptyset$. We analytically derived the universal large-scale Boltzmann equation in arbitrary dimension $d$ describing the reaction-limited regime of slow reactions $\hbar n \Gamma/D\ll 1$. In $d=1$, our results are equivalent to the TGGE ansatz, connecting the latter to the field-theoretical diagrammatic expansions. For homogeneous systems, we analytically showed that the density algebraic decay exponent features an unexpected dependency on $d$ and it deviates from mean-field value even in $d>1$, in contrast with classical RD dynamics. In one-dimensional inhomogeneous setups involving a trapping potential, we found that the decay is either accelerated towards the mean-field value (confined systems), or severely slowed down (deconfined systems). Our results find a natural application in cold-atomic experiments of fermionic gases involving two-body losses \cite{lossexpF1,lossexpF2,lossexpF3}. From the formulation here proposed, several relevant questions can be addressed. As an example, one can assess the impact of elastic-Hamiltonian collisions on the decay exponent \cite{lossexpF2,lossexpF3,FHth1,FHth2,FHth3}. Away from the reaction-limited, it is also crucial to study the quantum diffusion-limited regime $\hbar n \Gamma/D \sim 1$ via renormalization group schemes, as done for the classical RD \cite{cardy1996,tauber2005,tauber2002dynamic,tauber2014critical}.   

\textbf{Aknowledgments.---}
We aknowledge fruitful discussion with M. Buchhold, S. Diehl and J.P. Garrahan. F.G. acknowledges financial support from the University of Trento and Collegio Clesio. G.P. acknowledges support from the Alexander von Humboldt Foundation through a Humboldt research fellowship for postdoctoral researchers. We acknowledge financial support in part from EPSRC Grant no.\ EP/R04421X/1 and EPSRC Grant no.\ EP/V031201/1. We are also grateful for funding from the Deutsche Forschungsgemeinsschaft (DFG, German Research Foundation) under Project No. 435696605, the Research Unit FOR 5413/1, Grant No. 465199066 and the Research Unit FOR 5522/1, Grant No. 499180199.

%\pagebreak

%%%%%%%%%%%%%%%%%%%%%%%%%%%%%%%%%%%%%%%%%%%%%%%%%%%%%%%%%%%%%%%%%%%%
\bibliography{refs}

\setcounter{equation}{0}
\setcounter{figure}{0}
\setcounter{table}{0}
\renewcommand{\theequation}{S\arabic{equation}}
\renewcommand{\thefigure}{S\arabic{figure}}

\makeatletter

\onecolumngrid
\newpage

\setcounter{secnumdepth}{3}
\pagestyle{plain}

\setcounter{page}{1}

\begin{center}
{\Large SUPPLEMENTAL MATERIAL}
\end{center}
\begin{center}
\vspace{0.8cm}
{\Large Large-scale universality in Quantum Reaction-Diffusion from Keldysh field theory}
\end{center}

\begin{center}
Federico Gerbino,$^{1,2}$ Igor Lesanovsky,$^{1,3}$ and Gabriele Perfetto$^{1}$
\end{center}
\begin{center}
$^1${\em Institut f\"ur Theoretische Physik, Universit\"at T\"ubingen,}\\
{\em Auf der Morgenstelle 14, 72076 T\"ubingen, Germany}\\
$^2${\em Physics Department, University of Trento, via Sommarive, 14 I-38123 Trento, Italy}\\
$^3${\em School of Physics and Astronomy and Centre for the Mathematics}\\
{\em and Theoretical Physics of Quantum Non-Equilibrium Systems,}\\
{\em  The University of Nottingham, Nottingham, NG7 2RD, United Kingdom}\\
\end{center}

%%%%%%%%%%%%%%%%%%%%%%%%%%%%%%%%%%%%%%%%%%%%%%%%%%%%%%%%%%%%%%%%%%%%
This Supplemental Material provides additional information on the model and the calculations at the basis of the results presented in the main text. In Sec.~\ref{sec:continuum}, we derive the continuum-space Lindblad dynamics, Eqs.~(1)-(4) of the main text, by taking the continuum limit of the corresponding lattice model. 
In Sec.~\ref{sec:open_keldysh}, we briefly recall the basic aspects of the Keldysh field theory for open quantum systems, which are needed for the understanding of the action formulation in Eqs.~(5) and (6) of the main text. In Sec.~\ref{sec:self_energy_calculations}, we briefly report the main steps underlying the diagrammatic calculation of the self-energy Keldysh matrix at first order in $\Gamma$ and the associated derivation of the Boltzmann equation in Eq.~(8) of the main text. In Sec.~\ref{sec:effective_exp_supp}, we compute the effective exponent for the deconfining dynamics ensuing from the trap-release quench of the trapping potential (see Fig.~3 of the main text).
	
%%%%%%%%%%%%%%%%%%%%%%%%%%%%%%%%%%%%%%%%%%%%%%%%%%%%%%%%%%%%%%%%%%%%
\section{Lindblad dynamics in the space-continuum limit} 
\label{sec:continuum}

We start by introducing, for the sake of illustrative purposes, a one-dimensional lattice model, which is pictorially represented in Fig.~\ref{fig:continuum}(a). Let us consider an infinite chain whose sites labelled by index $j\in\mathbb{Z}$ can be either occupied by one fermion $n_j\ket{...\bullet_j...}=\ket{...\bullet_j...}$ or empty $n_j\ket{...\circ_j...}=0$. The number operator $n_j=c_j^{\dagger}c_j$ is written in terms of fermionic destruction ($c_j$) and creation ($c_j^{\dagger}$) operators satisfying the anticommutation relation $\{c_i,c_j^\dagger\}=\delta_{i,j}$. We also introduce the lattice spacing $a$, i.e., the shortest length scale of the problem. The hopping Hamiltonian reads
\begin{equation}
    H\,=\,-\,\frac{D}{a^2}\,\sum_j\,(\,c_{ja}^\dagger c_{(j+1)a}+c^\dagger_{(j+1)a} c_{ja}\,)\,+\,\sum_j\,c^\dagger_{ja} V_{ja} c_{ja}\,.
\label{eq:hopping_Hamiltonian}
\end{equation}
We remark that we here follow the convention of Ref.~\cite{tauber2005} for the hopping amplitude notation so that $D/\hbar$ has units [$\mbox{length}^{2}/\mbox{time}$] of a diffusion constant, while $D/(a^2 \hbar)$ has units [$\mbox{time}^{-1}$] of a rate. We further include the effect of a position-dependent trapping potential $V_{ja}$. In the absence of the latter, the fermionic hopping Hamiltonian \eqref{eq:hopping_Hamiltonian} has been studied in Refs.~\cite{van_horssen2015,perfetto22} for quantum RD models on a chain. The reaction part is encoded into the jump operators of the Lindbladian and it is given by two-body annihilation process $A+A\to\emptyset$. This destroys two neighbouring fermions at a rate $\Gamma/a^3$ ($\Gamma$ has units [$\mbox{length}^{3}/\mbox{time}$]) \begin{equation}
    \dot{\rho}(t)=-i[H,\rho(t)]+\sum_{j} \left[L_{ja}\rho {L_{ja}}^\dagger-\frac{1}{2}\left\{{L_{ja}}^\dagger L_{ja},\rho \right\}\right], \quad \mbox{with} \quad L_{ja}=\sqrt{\frac{\Gamma}{a^{3}}}\,c_{ja} c_{(j+1)a}. 
\label{eq:Lindblad_1d}
\end{equation}
Clearly, for fermions, one cannot define reactions occurring on the same site as this would violate the exclusion principle. The annihilation process between neighbouring sites is sketched in Fig.~\ref{fig:continuum}(a). 

The generalization of Eqs.~\eqref{eq:hopping_Hamiltonian} and \eqref{eq:Lindblad_1d} to generic space dimension $d$ can be simply accomplished by introducing a $d-$dimensional hypercubic lattice. Lattice points on this lattice are identified by a $d-$dimensional vector $\textbf{j}=(j_1,\dots j_{\alpha},\dots j_d)$ of integer numbers $j_{\alpha}\in\mathbb{Z}$ ($\alpha=1,2\dots d$). Then one replaces $c_{ja}$ and $c_{(j+1)a}$ (the same applying for the corresponding creation operators) with $c_{\textbf{j}a}$ and $c_{\textbf{j}a+a\textbf{e}_\alpha}$, $\textbf{e}_\alpha$ being the unit-vector pointing to direction $\alpha\,=\,1,...,d$. The Hamiltonian and the Lindblad master equation then read 
\begin{equation}
    H=-\,\frac{D}{a^2}\,\sum_{\alpha=1}^d\,\sum_{\textbf{j}}\,(\,c_{\textbf{j}a}^{\dagger}c_{\textbf{j}a+a\textbf{e}_\alpha}+c_{\textbf{j}a+a\textbf{e}_\alpha}^{\dagger}c_{\textbf{j}a})\,+\,\sum_{\alpha=1}^d\,\sum_{\textbf{j}}\,c_{\textbf{j}a}^\dagger V_{\textbf{j}a} c_{\textbf{j}a},
\label{eq:Hamiltonian_hopping_d_dimension}
\end{equation}
and
\begin{equation}
\dot{\rho}(t)=-i[H,\rho(t)]+\sum_{\alpha=1}^{d}\sum_{\textbf{j}} \left[L_{\textbf{j}a}^{(\alpha)}\rho {L_{\textbf{j}a}^{(\alpha)}}^\dagger-\frac{1}{2}\left\{{L_{\textbf{j}a}^{(\alpha)}}^\dagger L_{\textbf{j}a}^{(\alpha)},\rho \right\}\right], \quad \mbox{with} \quad
L_{\textbf{j}\alpha}^{(\alpha)}=\sqrt{\frac{\Gamma}{a^{d+2}}}\,c_{\textbf{j}a} c_{\textbf{j}a+a\textbf{e}_\alpha}\,.
\label{eq:lindblad_d_dimension}
\end{equation}
Equation \eqref{eq:Hamiltonian_hopping_d_dimension} shows that particles can hop isotropically (with the same amplitude) to one of the neighbouring sites in any direction $\alpha$. Annihilation reactions, according to Eq.~\eqref{eq:lindblad_d_dimension}, also take place isotropically.

Let us take the continuum limit of this model defined by the substitutions $\textbf{j}a\,\rightarrow\,\vec{x}$,  $\sum_\textbf{j}\,\rightarrow\,\int\,d^dx/a^d$ and by considering $a\,\rightarrow\,0$.
We may then write:
\begin{equation}
    \lim_{a\rightarrow 0}\,\frac{L_{\textbf{j}a}^{(\alpha)}}{a^{d/2}}=\lim_{a\rightarrow 0}\,\sqrt{\frac{\Gamma}{a^d}}\,\frac{c_{\textbf{j}a}c_{\textbf{j}a+a\textbf{e}_\alpha}}{a^{d/2+1}}=\lim_{a\rightarrow 0}\,\sqrt{\frac{\Gamma}{a^d}}\,\frac{c_{\textbf{j}a}}{a^{d/2}}\,\frac{c_{\textbf{j}a+a\textbf{e}_\alpha}-c_{\textbf{j}a}}{a}.
\label{eq:continuum_limit_L_d_dimensions}
\end{equation}
Introducing the field operators
\begin{equation}
\psi(\vec{x})=\frac{c_{\textbf{j}a}}{a^{d/2}}, \quad \mbox{with} \quad \{\psi(\vec{x}),\psi^{\dagger}(\vec{x}')\}=\delta(\vec{x}-\vec{x}'),
\label{eq:field_operators}
\end{equation}
one recognises that Eq.~\eqref{eq:continuum_limit_L_d_dimensions} is the definition of a partial derivative in direction $x_\alpha$, namely:
\begin{equation}
    \lim_{a\rightarrow 0}\,\frac{L_{\textbf{j}a}^{(\alpha)}}{a^{d/2}}=\sqrt{\Gamma}\,\psi(\vec{x})\,\frac{\partial \psi}{\partial x_\alpha}=L_{\alpha}(\vec{x})\,.
\label{supeq:continuum_limit_ann_jump}
\end{equation}
Repeating the same procedure for the Hamiltonian \eqref{eq:Hamiltonian_hopping_d_dimension}, the definition of a second partial derivative $\frac{\partial^2}{\partial x_\alpha^2}$ appears
\begin{equation}
    H=\int\,d^dx\,\psi^\dagger(\vec{x})\,[\,-D\nabla^2+V(\vec{x})\,]\,\psi(\vec{x}).
\label{eq:Hamiltonian_continuum_sup}
\end{equation}
This equation is readily recognized as Eq.~(2) of the main text. For the dissipator in Eq.~\eqref{eq:lindblad_d_dimension}, we also have 
\begin{equation}
\dot{\rho}(t)=-i[H,\rho(t)]+\Gamma\,\int\,d^dx\,\Big[\,\psi\vec{\nabla}\psi\,\rho\,\cdot\,\vec{\nabla}\psi^\dagger\psi^\dagger\,-\,\{\,\vec{\nabla}\psi^\dagger\psi^\dagger\,\cdot\,\psi\vec{\nabla}\psi,\,\rho\,\}\,\Big].
\label{eq:lindblad_d_dimension_continuum_supp}
\end{equation}
Writing the dot product $\cdot$ explicitly in terms of the associated $\alpha=1,2\dots d$ components, Eq.~\eqref{eq:lindblad_d_dimension_continuum_supp} is recognized as Eq.~(3) of the main text. In Fig.~\ref{fig:continuum}(b), we pictorially show the continuum formulation of the model (in one dimension for the sake of simplicity in the illustration).

% Figure environment removed

%%%%%%%%%%%%%%%%%%%%%%%%%%%%%%%%%%%%%%%%%%%%%%%%%%%%%%%%%%%%%%%%%%%%
\section{Keldysh path integral formulation of open systems}
\label{sec:open_keldysh}
We report here the main steps underlying the construction of Keldysh field theory for dissipative many-body quantum systems \cite{altland2010,sieberer2016,tonielli2016, kamenev2011,thompson2023field}. We then report the main definitions concerning the associated Green functions, which are necessary to understand the derivation of the quantum Boltzmann equation in the main text.

Let us consider the formal solution to a generic quantum Master equation of the form, e.g., in Eq.~\eqref{eq:lindblad_d_dimension}. We start by considering a zero-dimensional system with no spatial structure (no sum over lattice sites $\mathbf{j}$). We can Trotter-decompose the time-evolution operator in infinitely many time slices, thus writing:
\begin{equation}
    \rho(t)= e^{\mathcal{L}t}[\rho(0)]=\lim_{N\rightarrow \infty}\,(\,1+\mathcal{L}\delta t\,)^N[\rho(0)],
\end{equation}
with $\delta t\,=\,t/N$. Hence, at any time instant $t_n\,=\,n\delta t$, with $n=1,2\dots N$, the subsequent density matrix operator evolves via:
\begin{equation}
    \rho_{n+1}=(\,1+\mathcal{L}\delta t\,)[\rho_n]+\,O(\delta t^2).
\label{sup:step_1_path}
\end{equation}
It is now possible to introduce two sets of fermionic coherent states, labeled by Grassmann fields $\psi_n^{\pm}$ which are eigenvalues of the fermionic destruction operator. From the resolution for the identity operator one has:
\begin{equation}
    \rho_n=\int\,d\psi_n^+\,d\bar{\psi}_n^+ d\psi_n^-\,d\bar{\psi}_n^-\,e^{-\bar{\psi}^+_n \psi^+_n -\bar{\psi}^-_n \psi^-_n}\,\bra{\psi_n^+}\rho_n\ket{-\psi_n^-}\,\ket{\psi_n^+}\bra{-\psi_n^-}.
\label{sup:step2_path}
\end{equation}
We label $\ket{\psi_n^+}$ the states acting at time slice $n$ on the density operator from the left, hence evolving the latter in the forward direction along the so-called forward (+) contour. Conversely, states $\ket{\psi_n^-}$ act on the density operator from the right and thus evolve backward in time along a backward (-) contour (cf. the central panel of Fig.~(1) of the main text). 
Doubling the field variables from two to four fields evolving along anti-parallel time axes is a peculiarity of out-of-equilibrium quantum field theories \cite{altland2010,sieberer2016,tonielli2016, kamenev2011,thompson2023field}. Besides, an additional minus sign in front of $\ket{-\psi_n^-}$ and $\bra{-\psi_n^-}$ coherent states has been added with no effect on the result of the calculation.
Inserting Eq.~\eqref{sup:step2_path} into \eqref{sup:step_1_path}, we arrive to:
\begin{equation}
\begin{split}
    \bra{\psi_{n+1}^+}\rho_{n+1}\ket{-\psi_{n+1}^-}=\int\,d\psi_n^+d\bar{\psi}_n^+\,&d\psi_n^-d\bar{\psi}_n^-\,e^{-\bar{\psi}^+_n \psi^+_n -\bar{\psi}^-_n \psi^-_n}\\
    &\bra{\psi_n^+}\rho_n\ket{-\psi_n^-}\,
    \bra{\psi_{n+1}^+}\big\{ (1+\mathcal{L}\delta t)\left[\ket{\psi_{n}^+}
    \bra{-\psi_n^-}\right]\big\}\ket{-\psi_{n+1}^-}\,+\,O(\delta t^2)\,.
\end{split}
\end{equation}
Evaluation of the matrix element in squared brackets, together with the normalisation factors of coherent states completeness relations $e^{-\bar{\psi}^+_n \psi^+_n -\bar{\psi}^-_n \psi^-_n}$,  leads to:
\begin{equation}
\begin{split}
    \bra{\psi_{n+1}^+}\rho_{n+1}\ket{-\psi_{n+1}^-}=\int\,d\psi_n^+d\bar{\psi}_n^+\,&d\psi^-_n d\bar{\psi}^-_n
    \,\bra{\psi_n^+}\rho_n\ket{-\psi_n^-}\\
    &\mathrm{exp}\Big\{ i\delta t\Big[ -i\partial_t\bar{\psi}^+_n\psi^+_n
    -i\bar{\psi}^-_n\partial_t\psi^-_n -i\mathcal{L}[ \psi_n^+,\bar{\psi}_{n+1}^+,\psi_{n+1}^-,\bar{\psi}_n^-]\Big]\Big\}\,+\,O(\delta t^2).
\label{supeq:path_integral_3}
\end{split}
\end{equation}
In the previous equation, the function $\mathcal{L}[ \psi_n^+,\bar{\psi}_{n+1}^+,\psi_{n+1}^-,\bar{\psi}_n^-]$ is the Lindbladian evaluated in terms of Grassmann fields after the Hamiltonian and the jump operators have been normally ordered so that creation operators always lie to the left of destruction ones. We also introduced the notation $\partial_t \psi_n\,=\,\frac{\psi_{n+1}-\psi_n}{\delta t}$. 

Repeated iteration of the previous equation from time slice $n\,=\,1$ to $n\,=\,N$ connects the initial time to the final time density matrix. One can then take the limit $N\,\rightarrow\,\infty$, thus neglecting the $O(\delta t^2)$ terms and performing the double substitution $\sum_n \delta t\,\rightarrow\,\int dt$, $\prod_n d\psi_n\,\rightarrow\,\mathbf{D}\psi$. The trace operation joins the forward and backward contour leading to the Keldysh partition function:
\begin{equation}
    Z(t)= \int \mathbf{D}[\psi_+,\bar{\psi}_+,\psi_-,\bar{\psi}_-]\,\bra{\psi_+(0)}\rho(0)\ket{-\psi_-(0)}\,e^{iS[\psi_+,\bar{\psi}_+,\psi_-,\bar{\psi}_-]}\,.
\end{equation}
In the following, we will neglect the boundary term $\bra{\psi_+(0)}\rho(0)\ket{-\psi_-(0)}$ as it does not affect the large-scale Boltzmann equation, which is solely determined by the bulk part of the action. We therefore focus on the bulk part of the action only and we require the associated normalisation constraint
\begin{equation}
    Z(t)=\mathrm{tr}[\rho(t)]= \int \mathbf{D}[\psi_+,\bar{\psi}_+,\psi_-,\bar{\psi}_-]\,e^{iS[\psi_+,\bar{\psi}_+,\psi_-,\bar{\psi}_-]}=1.
\end{equation}
The bulk part of the Keldysh action is denoted as $S$ and it reads
\begin{equation}
    S[\psi_+,\bar{\psi}_+,\psi_-,\bar{\psi}_-]= \int_{0}^{t}\,dt'\, (\, \bar{\psi}_+i\partial_{t'}\psi_+ - \bar{\psi}_-i\partial_{t'}\psi_- - i \mathcal{L}(\psi_+,\bar{\psi}_+,\psi_-,\bar{\psi}_-),
\label{supeq:keldysh_action_1}
\end{equation}
with the Lindbladian:
\begin{equation}
    \mathcal{L}(\psi_+,\bar{\psi}_+,\psi_-,\bar{\psi}_-)\,=\, -i(H_+-H_-)\,+\,\sum_{\alpha}\,\big[\, \bar{L}_{\alpha,-}L_{\alpha,+}-\frac{1}{2}(\bar{L}_{\alpha,+}L_{\alpha,+}+\bar{L}_{\alpha,-}L_{\alpha,-})\,\big].
\label{supeq:keldysh_action_2}
\end{equation}
Functions $H_{\pm}=H(\bar{\psi}_{\pm},\psi_{\pm})$, $L_{\alpha,\pm}=L_{\alpha}(\bar{\psi}_{\pm},\psi_{\pm})$ and $\bar{L}_{\alpha,\pm} L_{\alpha,\pm}=L^{\dagger}_{\alpha}L_{\alpha}(\bar{\psi}_{\pm},\psi_{\pm})$ are evaluated on the forward (+) or backward (-) contour after normal ordering, as explained also after Eq.~\eqref{supeq:path_integral_3}. 

Clearly, the field theory described by the preceding Keldysh action applies to zero-dimensional systems without a space structure. However, one can generalise the derivation to spatially-extended systems, with the representation of the identity via a set of coherent states $\ket{\{\psi_n\}}=\ket{\psi_{n,\mathbf{1}}}\ket{\psi_{n,\mathbf{2}}} \dots \ket{\psi_{n,\mathbf{j}}} \dots$ labelled on a time slice $n$ by the lattice site coordinate $\mathbf{1},\mathbf{2}\dots \mathbf{j}\dots$ (the bold face notation representing a vector of integer coordinates with the notation introduced after Eq.~\eqref{eq:Lindblad_1d}):
\begin{equation}
    \mathbb{I}=\int \prod_\textbf{j} d\psi_{n,\textbf{j}} d\bar{\psi}_{n,\textbf{j}} e^{-\sum_\textbf{j} \bar{\psi}_{n,\textbf{j}} \psi_{n,\textbf{j}}} \ket{\{\psi_n\}}\bra{\{\psi_n\}}.
\end{equation}
In this case, the continuum-space limit turns the sum over lattice sites $\textbf{j}$ into a space integral, while it includes in the functional measure $\mathbf{D}[\psi,\bar{\psi}]$ an infinite product over the spatial field configurations evaluated at each time slice. Accordingly, the Keldysh action for the fermionic binary annihilation dynamics, described by the Lindblad equation in Eqs.~\eqref{eq:Hamiltonian_continuum_sup} and \eqref{eq:lindblad_d_dimension_continuum_supp}, in the $\pm$ basis reads:
\begin{align}
        S =S_{\mathrm{diff}}+S_{\mathcal{D}}&=\int d^dx dt\, \frac{1}{\hbar}\,\Big[ \,\bar{\psi}_{+}(i\hbar\partial_t +D\nabla^2-V(\vec{x}))\psi_{+} 
        - \bar{\psi}_{-} (i\hbar\partial_t +D\nabla^2-V(\vec{x}))\psi_{-}\,\Big]+ \nonumber \\
        &+ i\Gamma \int  d^dx\,dt \,\Big[\,\frac{1}{2}\vec{\nabla}\bar{\psi}_+\Bar{\psi}_+ \cdot \psi_+\vec{\nabla}\psi_+ + \frac{1}{2}\vec{\nabla}\bar{\psi}_-\Bar{\psi}_- \cdot \psi_- \vec{\nabla}\psi_- -
       \vec{\nabla}\bar{\psi}_-\Bar{\psi}_- \cdot \psi_+\vec{\nabla}\psi_+ \,\Big]\,,
\label{supeq:keldysh_pm}
\end{align}
where the r.h.s. of the second equality on the first line is the quadratic action $S_{\mathrm{diff}}$ describing coherent motion in a continuous $d$-dimensional space \eqref{eq:Hamiltonian_continuum_sup}. The quartic action $S_{\mathcal{D}}$ on the second lines represents, instead, the interaction vertices due to nearest-neighbour reactions \eqref{supeq:continuum_limit_ann_jump}. By means of the Keldysh rotation of Eq.~\eqref{eq:rotation_K_fermions} which we shall later introduce, the former line gives rise to the diffusion action in Eq.~(5), while the latter generates the dissipative sector discussed in Eq.~(6). The two-point Green's functions associated to the Keldysh action are of central relevance. We list them here as they will be needed in the derivation of the quantum Boltzmann equation in Sec.~\ref{sec:self_energy_calculations}. We use henceforth in the Supplemental Material, the compact 4-vector notation $x=(\vec{x},t)$:
\begin{subequations}
\begin{align}
        & iG^{T}(x_1,x_2)=\langle \psi_+(x_1)\bar{\psi}_+(x_2) \rangle= \braket{T\left[c(x_1)c^{\dagger}(x_2)\right]}, \label{eq:pmgreens_ferm_1} \\% -\,\langle \, c^\dagger (x') c(x)\,\rangle   \,+\,\Theta(x-x')\,\langle\,\{\, c(x),c^\dagger (x')\,\}\,\rangle \\
        & iG^<(x_1,x_2)=\langle \psi_+(x_1)\bar{\psi}_-(x_2) \rangle=-\langle c^\dagger (x_2) c(x_1)\rangle,\label{eq:pmgreens_ferm_2}  \\
        & iG^>(x_1,x_2)=\langle \psi_-(x_1)\bar{\psi}_+(x_2) \rangle = \braket{c(x_1)c^{\dagger}(x_2)},\label{eq:pmgreens_ferm_3}\\ %-\langle  c^\dagger (x') c(x)\rangle +\langle\{\, c(x),c^\dagger (x')\}\ \rangle \\
        & iG^{\tilde{T}}(x_1,x_2)=\langle \psi_-(x_1)\bar{\psi}_-(x_2) \rangle=\braket{\widetilde{T}\left[c(x_1)c^{\dagger}(x_2)\right]}. \label{eq:pmgreens_ferm_4} %-\langle  c^\dagger (x') c(x)\rangle +\Theta(x'-x)\langle\{ c(x),c^\dagger (x')\} \rangle.
\end{align}
\label{eq:pmgreens}%
\end{subequations}
In Eqs.~\eqref{eq:pmgreens_ferm_1} and \eqref{eq:pmgreens_ferm_4}) the symbols $T$ and $\widetilde{T}$ denote time ordering and anti-time ordering along the Keldysh contour, respectively (the operator evaluated at the larger/smaller time goes to the left, respectively, possibly taking a minus sign if a permutation is needed). Equation \eqref{eq:pmgreens} also shows the direct connection between the Green functions and dynamical correlation functions of creation and destruction operators. We refer the reader to Section $5.2$ of Ref.~\cite{gardiner2004quantum} for a detailed discussion of correlation functions in dissipative systems. Because of probability conservation, the four functions are not independent, i.e., they satisfy the identity:
\begin{equation}
\label{eq:probcons+-}
    G^{T}(x_1,x_2)+G^{\Tilde{T}}(x_1,x_2)- G^< (x_1,x_2)- G^>(x_1,x_2)\,=\,0\,.
\end{equation}
It is then customary to introduce the Keldysh rotation of the fields, which automatically deletes the spurious degrees of freedom, namely, one correlation function is identically vanishing after the rotation. Following the convention by Larkin and Ovchinnikov \cite{larkin1975}, we define:
\begin{equation}
    \phi_1=\frac{\psi_++\psi_-}{\sqrt{2}},\;\;\;
    \phi_2=\frac{\psi_+-\psi_-}{\sqrt{2}},\;\;\;
    \bar{\phi}_1=\frac{\bar{\psi}_+-\bar{\psi}_-}{\sqrt{2}},\;\;\;
    \bar{\phi_2}=\frac{\bar{\psi}_++\bar{\psi}_-}{\sqrt{2}}.
\label{eq:rotation_K_fermions}    
\end{equation}
Applying this linear transformation of the fields $\psi_{\pm}$ to the Keldysh action in Eq.~\eqref{supeq:keldysh_pm} gives the action reported in Eqs.~(5) and (6) of the main text. The rotation also clarifies the role of the remaining Green's functions
\begin{subequations}
\begin{align}
    iG^R(x_1,x_2)&=\langle \phi_1(x_1)\bar{\phi}_1(x_2) \rangle=\Theta(t-t')\langle\{\,c(x_1), c^\dagger(x_2)\}\rangle,\label{eq:greens_rotated_1} \\
    iG^A(x_1,x_2)&=\langle \phi_2(x_1)\bar{\phi}_2(x_2) \rangle=-\Theta(t'-t)\langle\{\,c(x_1), c^\dagger(x_2)\}\rangle, \label{eq:greens_rotated_2} \\
    iG^K(x_1,x_2)&=\langle \phi_1(x_1)\bar{\phi}_2(x_2) \rangle= \braket{[c(x_1),c^{\dagger}(x_2)]}=-2\langle c^\dagger(x_2)c(x_1)\rangle +\langle\{\,c(x_1),c^\dagger(x_2)\}\rangle, \label{eq:greens_rotated_3}
\end{align}
\label{eq:greens_rotated}%
\end{subequations}
with $\braket{\phi_2(x_1)\bar{\phi}_1(x_2)}=0$ being zero.
In fact, the retarded and the advanced Green's functions $G^R$ \eqref{eq:greens_rotated_1} and $G^A$ \eqref{eq:greens_rotated_2}, respectively, give information about the quasi-particle spectrum of the system, while the Keldsyh Green's function $G^K$ \eqref{eq:greens_rotated_3} gives information on the statistical occupation of the quasi-particle spectrum. The retarded and advanced Green functions are related by Hermitian adjoint $G^R(x_1,x_2)=[G^A(x_1,x_2)]^{\dagger}$ (complex conjugate and swap of the space-time variables $x_1$ and $x_2$), while $G^K$ is anti-hermitian $[G^K(x_1,x_2)]^{\dagger}=-G^{K}(x_1,x_2)$. The three Green functions \eqref{eq:greens_rotated} can be organised in the following 2x2 Keldysh-space matrix:
\begin{equation} 
\label{eq:propmat}
    \hat{G}(x_1,x_2) =
    \begin{pmatrix}
    G^R(x_1,x_2) & G^K (x_1,x_2)\\
    0 & G^A(x_1,x_2) 
    \end{pmatrix}.
\end{equation}
It is eventually customary to parametrise the Keldysh Green's function in terms of the hermitian distribution function $F(x_1,x_2)=[F(x_1,x_2)]^{\dagger}$, which is defined from the relation
\begin{equation}
\label{eq:distrib}
    G^K=G^R\circ F-F\circ G^A.
\end{equation}
In the previous equation, and in Sec.~\ref{sec:self_energy_calculations}, the symbol $\circ$ denotes the convolution product in the space-time variables. Whenever the symbol $\circ$ appears together with a propagator $\hat{G}$, as in Eq.~\eqref{eq:propmat}, contraction of the Keldysh 2x2 indices is also implied. The distribution function $F$ is the object of the kinetic Boltzmann equation, whose derivation we detail in the next section.  

%%%%%%%%%%%%%%%%%%%%%%%%%%%%%%%%%%%%%%%%%%%%%%%%%%%%%%%%%%%%%%%%%%%%
\section{Self-energy and Boltzmann equation}
\label{sec:self_energy_calculations}

The evolution equation for the distribution function $F$ follows from the Dyson equation $[\hat{G}^{-1}_0-\hat{\Sigma}] \circ \hat{G}\,=\,\mathds{1}$. Here $\hat{G}_0$ denotes the bare propagators of the quadratic action $S_{\mathrm{diff}}$ in Eq.~(5) of the main text (we do not report their expressions as they are not needed in the derivation), while $\hat{G}$ the propagators dressed by the interaction via the self-energy $\hat{\Sigma}$. The latter has the same structure in Keldysh indices as the propagator $\hat{G}$ in Eq.~\eqref{eq:propmat} 
\begin{equation} 
\label{eq:sigmamat}
    \hat{\Sigma}(x_1,x_2) =
    \begin{pmatrix}
    \Sigma^R(x_1,x_2) & \Sigma^K (x_1,x_2)\\
    0 & \Sigma^A(x_1,x_2) 
    \end{pmatrix}.
\end{equation}
Taking the Keldysh component of the Dyson equation and parametrizing $G^K$ in terms of $F$ as in Eq.~\eqref{eq:distrib}, one has
\begin{equation} 
\label{eq:kinetic}
    \left[-i(\partial_{t_1} +\partial_{t_2})-D(\nabla_{\vec{x_1}}^2-\nabla_{\vec{x_2}}^2)+(V(\vec{x_1})-V(\vec{x_2}))\right]F(x_1,x_2)=I[F]=\Sigma^K\circ\mathds{1}-(\,\Sigma^R\circ F-F\circ \Sigma^A\,).
\end{equation}
The latter is named kinetic equation, and the right-hand side $I[F]$ is called the collision integral. The self-energy matrix $\hat{\Sigma}$ encodes the effect of the interaction. Within this description, the computation of $F$ and of correlation functions necessarily amounts to calculating $\hat{\Sigma}$. The latter cannot be evaluated exactly and therefore one needs to develop an approximation scheme. In this manuscript, we consider the so-called Born perturbation theory approximation \cite{altland2010,sieberer2016,kamenev2011}. Within this approximation, $\hat{\Sigma}$ is evaluated by perturbatively expanding the interaction vertices around the quadratic part and thereby expressing the resulting Feynman diagrams in terms of the bare Green functions $\hat{G}_0$. 

As summarised in the main text, we consider, in particular, first-order contributions in $\Gamma$ to $\hat{\Sigma} \approx \hat{\Sigma}_{(1)}$. Such contributions are encoded into tadpole diagrams, where internal loops are connected to the external legs via a four-fielded vertex. There are four possible choices of the external legs, i.e., two possible fields for each of the two legs. Hence, we can sum up all internal contributions, and organise them in the 2x2 Keldysh-space matrix \eqref{eq:sigmamat}. Besides, at first-order in perturbation theory $\hat{\Sigma}(x_1,x_2) \approx \hat{\Sigma}_{(1)}(x_1,x_2)=\hat{\Sigma}_{(1)}(x)$ as tadpole graphs are defined at one space-time point $x_1=x_2=x$ only. Its components read:
\begin{equation}   \label{eq:self_en_ferm}
    \hat{\Sigma}_{(1)}(x)\,=\,
    \begin{pmatrix}
        \Sigma^R_{(1)}(x) & \Sigma^K_{(1)}(x) \\
        0 & \Sigma^A_{(1)}(x)
    \end{pmatrix}\,=\,\frac{\Gamma}{4}\,
    \begin{pmatrix}
        [f(G^A_0)+f(G^K_0)] & [2f(G^K_0)+f(G^A_0)-f(G^R_0)] \\
        0 & [f(G^R_0)-f(G^K_0)]
    \end{pmatrix}\,.
\end{equation}
Since the interaction vertices of the theory contain gradients of the fields $\phi_{1,2}$, the self-energy $\hat{\Sigma}$ is a differential operator acting on Feynman diagrams' external legs. In the previous equation, this is expressed by the appearance of the operator $f(G_0)$, which is defined as:
\begin{equation}
    f(G_0)=[\,G_0(x,x) \overleftarrow{\nabla}\,]\cdot\overrightarrow{\nabla} -[\,\overrightarrow{\nabla}G_0(x,x)\cdot\overleftarrow{\nabla}\,] + \overleftarrow{\nabla}\cdot [\,\overrightarrow{\nabla}G_0(x,x)\,] - \overleftarrow{\nabla}\cdot G_0(x,x) \overrightarrow{\nabla}.
\label{supeq:f_G_0}    
\end{equation}
In Eq.~\eqref{supeq:f_G_0}, the over-script arrows indicate the direction of the gradient operator. In particular, the notation $(G_0(x_1,x_2)\overleftarrow{\nabla})$ is used to denote differentiation of the Green's function $G_0(x_1,x_2)$ with respect to their second variable, i.e., $(G_0(x_1,x_2)\overleftarrow{\nabla})=(\vec{\nabla}_{x_2}G_0(x_1,x_2))$. Conversely, the notation $(\overrightarrow{\nabla}G_0(x_1,x_2))$ indicates differentiation with respect to the first variable of $G_0$, i.e., $(\overrightarrow{\nabla}G_0(x_1,x_2))=(\vec{\nabla}_{x_1}G_0(x_1,x_2))$. The differential operators in Eq.~\eqref{supeq:f_G_0} not enclosed into square brackets act on the diagrams' external legs, with $\overleftarrow{\nabla}$ differentiating with respect to the second variable of the external leg and $\overrightarrow{\nabla}$ with respect to the first one. From Eq.~\eqref{eq:self_en_ferm}, using the Hermitian adjoint properties of $G_0^{R,A,K}$ recalled after Eq.~\eqref{eq:greens_rotated}, it follows that $\Sigma^R_{(1)}=[\Sigma^A_{(1)}]^{\dagger}$ and $[\Sigma^{K}_{(1)}]^{\dagger}=-\Sigma^{K}_{(1)}$. The self-energy therefore shares the same structure as $\hat{G}$, as anticipated before Eq.~\eqref{eq:sigmamat} (this is expected to hold at any order in perturbation theory in $\Gamma$). We now derive a more useful expression for \eqref{eq:self_en_ferm} and \eqref{supeq:f_G_0}, which can be better used into the kinetic equation \eqref{eq:kinetic}. Since the self-energy matrix elements $\Sigma^K$, $\Sigma^R$ are convoluted with the identity and with $F$, respectively, we now want to turn operator $f(G_0)$ into a standard differential operator which acts on functions in subsequent positions on the right. We manage to do so by introducing a space-time variable $x_2$ which is constrained to take the same value of $x_1$ via a delta $\delta(x_1-x_2)$ (due to the tadpole structure of the diagrams). With this trick, we can write:
\begin{equation}
    \Sigma_{(1)}^R(x_1,x_2)=\frac{\Gamma}{4}\,\Big[\,\tilde{f}(G^A_0)+\tilde{f}(G^K_0)\,\Big], \quad \mbox{and} \quad     \Sigma_{(1)}^K(x_1,x_2)=\frac{\Gamma}{4}\,\Big[\,2\,\tilde{f}(G^K_0)+\tilde{f}(G^A_0)-\tilde{f}(G^R_0)\,\Big],
\label{supeq:sigma_1_R}    
\end{equation}
with
\begin{align}
    \tilde{f}(G_0)&=
    (\vec{\nabla}_{x_1}\delta(x_1-x_2)) \cdot [\,G_0(x_1,x_2) \vec{\nabla}_{x_2}-(\vec{\nabla}_{x_1} G_0(x_1,x_2))\,]+\nonumber \\
    &+\delta(x_1-x_2)\,[\,(\vec{\nabla}_{x_1} G_0(x_1,x_2))\cdot \vec{\nabla}_{x_2} + (\vec{\nabla}_{x_2}G_0(x_1,x_2))\cdot\vec{\nabla}_{x_2} 
    -(\nabla^2_{x_1} G_0(x_1,x_2))-(\vec{\nabla}_{x_1}\cdot\vec{\nabla}_{x_2} G_0(x_1,x_2))\,].
\end{align}
Derivatives of the Dirac deltas are understood in the sense of distribution, i.e., they have to be evaluated via integration by parts when inserted in the convolution products of the kinetic equation. We can also derive an analogous expression for $\Sigma^A$, which acts on the distribution function from the right, c.f. Eq.~\eqref{eq:kinetic}. 
where gradient operators act on the distribution function from the right only:
\begin{equation}
    \Sigma^A_{(1)}(x_1,x_2)=\frac{\Gamma}{4}\,\Big[\,\tilde{\tilde{f}}(G^R_0)-\tilde{\tilde{f}}(G^K_0)\,\Big],
\end{equation}
with
\begin{align}
    \tilde{\tilde{f}}(G_0)&=\delta(x_1-x_2)\,[\,(\nabla^2_{x_1} G_0(x_1,x_2))+(\nabla^2_{x_2} G_0(x_1,x_2))\,]\,-\,
    \overleftarrow{\nabla}_{x_1}\delta(x_1-x_2)\cdot[\,(\vec{\nabla}_{x_1} G_0(x_1,x_2))+(\vec{\nabla}_{x_2}G_0(x_1,x_2))\,]- \nonumber\\
    &+\,[\,(\vec{\nabla}_{x_2}G_0(x_1,x_2)) -\overleftarrow{\nabla}_{x_1} G_0(x_1,x_2)\,] \cdot (\vec{\nabla}_{x_2}\delta(x_1-x_2)).
\label{supeq:tilde_tilde_f}    
\end{align}

In order to extract the large-scale universal limit of the kinetic equation \eqref{eq:kinetic} and explicitly evaluate the convolution products appearing in the collision term, we resort to the Wigner transformation (WT). We first move to the set of Wigner coordinates $x=(x_1+x_2)/2$, $x'=x_1-x_2$, again using the 4-vector notation $x'=(\vec{x}',t')$ and $x=(\vec{x},t)$. These set of variables clearly distinguishes between the slow, macroscopic, evolution in $x$ and the fast dynamics in $x'$. The chain rule yields $\vec{\nabla}_{x_1}\,=\,\vec{\nabla}_x/2+\vec{\nabla}_{x'}$. The Wigner transform $A(x,k)$ corresponds to the Fourier transform of $A(x_1,x_2)$ with respect to the set of relative coordinates $x'$, which introduces the conjugated 4-momentum $k=(\vec{k},\epsilon)$:
\begin{equation}
    A(x,k)=\int d^dx'\,dt'\,A(x+x'/2,x-x'/2)\,e^{-i\vec{k}\cdot\vec{x}'+i\epsilon t'}, \, \, \, 
    A(x_1,x_2)=\int\,\frac{d^dk}{(2\pi)^d}\,\frac{d\epsilon}{2\pi}\,A\left(\frac{x_1+x_2}{2},k\right)\,e^{i\vec{k}\cdot(\vec{x_1}-\vec{x_2})-i\epsilon_k(t_1-t_2)}.
\label{supeq:WT_definition} 
\end{equation}
The main advantage of this procedure is to turn convolution products of two functions $A$ and $B$ in Eq.~\eqref{eq:kinetic} into Moyal products \cite{Moyal1,Moyal2,Moyal3,Moyal4,Moyal5} of their Wigner transforms
\begin{equation}
C(x_1,x_2)=(A \circ B)(x_1,x_2)=\int d^d x_3 A(x_1,x_3) B(x_3,x_2) \,\xlongrightarrow{\text{WT}}\,C(x,p)=A(x,k)\, \mathrm{exp}\left\{\frac{i}{2}[\overleftarrow{\partial}_x\overrightarrow{\partial}_k - \overleftarrow{\partial}_k\overrightarrow{\partial}_x]\right\}\,B(x,k).
\label{eq:Moyal_expansion}
\end{equation}
Crucially, in the reaction-limited regime $\hbar n \Gamma/D\ll 1$, time variations due to reactions are slow. Space modulations, induced by trapping potential $V$, can be also considered weak as long as $V$ varies on macroscopic length scales. In this limit, one can assume that the dependency of $G^{R,A,K}(x,x')$ on the macroscopic variable $x$ is much slower than the one on the relative one $x'$, entailing that 
\begin{equation}
\vec{\nabla}_{x_1} G^{R,A,K}(x_1,x_2)=\frac{1}{2}\vec{\nabla}_{x}G^{R,A,K}(x,x')+\vec{\nabla}_{x'}G^{R,A,K}(x,x')\simeq \vec{\nabla}_{x'} G^{R,A,K}(x,x')\xlongrightarrow{WT}i\vec{k} G^{R,A,K}(x,k).
\label{eq:derivative_approximation}
\end{equation}
Moreover, for slow space-time variations, one can simplify the Moyal product in Eq.~\eqref{eq:Moyal_expansion} by expanding the complex exponential. This leads to a derivative expansion, which can be truncated at the lowest order. From this expansion, the left-hand-side of the kinetic equation \eqref{eq:kinetic} becomes
\begin{equation} \label{eq:kinetic_term}
         \left[-i(\partial_{t_1} +\partial_{t_2})-D(\nabla_{\vec{x_1}}^2-\nabla_{\vec{x_2}}^2)+(V(\vec{x_1})-V(\vec{x_2}))\right]F(x_1,x_2) \xlongrightarrow{\text{WT}}\,i\,\Big[\,
        %\partial_\epsilon \epsilon\,
        \partial_t+
        %D\,\vec{\nabla}_k k^2\,
        v_g(\vec{k})
        \cdot \vec{\nabla}_x
        - \frac{1}{\hbar} \,\vec{\nabla}_x V(\vec{x})\cdot \vec{\nabla}_k\,\Big] F(x,k).
\end{equation}
We remark that Eq.~\eqref{eq:kinetic_term} is exact for harmonic potentials. For anharmonic potentials, the first correction to \eqref{eq:kinetic_term} is proportional to the cubic-space derivative of the potential \cite{Moyal5}. To deal with the WT of the collision integral, we first write $G_0^K=F(G_0^{R}-G_0^{A})$ in Eqs.~\eqref{supeq:sigma_1_R}-\eqref{supeq:tilde_tilde_f}. The previous equation follows from the WT of Eq.~\eqref{eq:distrib} at lowest order in derivatives. Here, albeit $G_0^{R,A,K}$ are bare Green functions, $F$ does not obey the free/quadratic-action dynamics, but it must be determined self-consistently from the kinetic equation. Truncating the Moyal expansion \eqref{eq:Moyal_expansion} of convolution products at lowest order in derivatives and using Eq.~\eqref{eq:derivative_approximation}, we eventually obtain for the collision integral [$q=(\vec{q},\omega)$] 
\begin{equation}
    I[F]= 
    -\frac{i\Gamma}{2}\int\frac{d^dq}{(2\pi)^d}\frac{d\omega}{2\pi}(\vec{k}-\vec{q})^2[1-F(x,k)-F(x,q)+F(x,k)F(x,q)]A_0(x,q),
\label{supeq:collision_wigner}   
\end{equation}
where we introduced the bare spectral function $A_0(x,q)=-2\mbox{Im}(G^R_0(x,q))=i[G^R_0(x,q)-G^A_0(x,q)]$. One can also replace the bare green functions $G_0^{R,A}$ in \eqref{supeq:collision_wigner} with the dressed ones $G^{R,A}$, and therefore $A_0(x,q)\to A(x,q)$, within the self-consistent Born approximation. One then writes an equation for $G^{R,A}$ which has to be solved self-consistently together with the equation for $F$. The self-consistent Born approximation is necessary when the perturbative Born series is divergent, as shown in Ref.~\cite{buchhold2015kinetic}. In our case, however, the perturbative expression (of order $\Gamma$) \eqref{supeq:collision_wigner} is finite and the self-consistent Born approximation therefore only gives sub-leading corrections. For this reason, we do not consider it here.

From Eq.~\eqref{supeq:collision_wigner}, one can see that the collision integral $I[F]$ is a purely imaginary quantity: $F(y,k)$ is real since it is the WT of the hermitian distribution function $F(x_1,x_2)$. This result implies that the self-energy does not generate any renormalization of the quasi-particle energy spectrum $\epsilon_k(\vec{x})=D\vec{k}^2+V(\vec{x})$, i.e., the kinetic term in Eq.~\eqref{eq:kinetic_term} is left unchanged. On the contrary, the imaginary part of the collision integral represents the finite particle lifetime $\sim (n\Gamma)^{-1}$ introduced by the dissipative interactions. In the reaction-limited regime of weak dissipation, this lifetime $\sim (n\Gamma)^{-1}$ is much longer than the coherent time-scale set by $(D/\hbar)^{-1}$ and therefore quasi-particles remain well defined during time evolution. The spectral function $A(x,k) $ is accordingly sharply peaked in $\epsilon$ around $\epsilon_k(\vec{x})$, with a width $\sim \Gamma$. In the non-interacting limit $\Gamma=0$, one exactly has $A_0(x,k)=2\pi \delta(\epsilon-\epsilon_{k}(\vec{x}))$. 
This allows us to consider the on-shell distribution function $\widetilde{F}(\vec{x},\vec{k},t)\equiv F(\vec{x},\vec{k},t,\epsilon=\epsilon_{k}(\vec{x}))$, which is given by 
\begin{equation}
    \widetilde{F}(\vec{x},\vec{k},t)=\int\frac{d\epsilon}{2\pi}\,F(x,k)A(x,k)=i\int \frac{d\epsilon}{2\pi} G^K(x,k)= iG^K(\vec{x},\vec{k},t,t)=1-2n(\vec{x},\vec{k},t).
\label{eq:F_on_shell}
\end{equation}
In the first equality, we used the peaked structure of the spectral function $A(x,k)$, in the second the expression of $G^K(x,k)$, in the third the inverse WT definition in $\epsilon$ \eqref{supeq:WT_definition}, and in the fourth equality the WT in space of Eq.~\eqref{eq:greens_rotated_3} (evaluated at equal times). The evolution equation for the one-body Wigner function $n(\vec{x},\vec{k},t)$ is thus equivalent to that for $\widetilde{F}(\vec{x},\vec{k},t)$. The latter is obtained by multiplying \eqref{eq:kinetic_term} and \eqref{supeq:collision_wigner} by $A(x,k)$ and integrating in $\epsilon$. The integral in $\omega$ in Eq.~\eqref{supeq:collision_wigner} is similarly handled by exploiting the Dirac delta form of $A_0(x,q)=2\pi \delta(\omega-\epsilon_k(\vec{x}))$. Applying the substitution $\widetilde{F}(\vec{x},\vec{k},t)=1-2 n(\vec{x},\vec{k},t)$, the Boltzmann equation (8) of the main text eventually follows. 

%%%%%%%%%%%%%%%%%%%%%%%%%%%%%%%%%%%%%%%%%%%%%%%%%%%%%%%%%%%%%%%%%%%%
\section{Effective exponent for the trap-release protocol}
\label{sec:effective_exp_supp}

% Figure environment removed

We give here additional details concerning the decay of the particle number in the trap-release protocol discussed in Fig.~3(c)-(d) of the main text. As we presented in the main text, the behavior of the re-scaled particle number $\tilde{N}(\tilde{t})=N(\tilde{t})/N_0$ at short rescaled times $\tilde{t}$ approaches the power-law $\tilde{N}(\tilde{t})\sim\tilde{t}^{-\xi}$, where $\xi$ is an exponent depending on parameter $\Omega$. In order to quantify $\xi$ and to see how the asymptotic decay changes as $\Omega$ is tuned, we compute the effective decay exponent $\xi_{\mathrm{eff}}$ \cite{hinrichsen2000non}
\begin{equation}
    \xi_{\mathrm{eff}}=-\,\frac{\log\,\Big[\tilde{N}(b\tilde{t})/\tilde{N}(\tilde{t})\Big]}{\log(b)},
    \label{supeq:effective_exp}
\end{equation}
with $b$ an arbitrary adimensional positive parameter. One can check that if $\tilde{N}(\tilde{t})$ asymptotically approaches a power law, i.e., $\tilde{N}(\tilde{t})\sim\tilde{t}^{-\xi}$ at long times, then $\lim_{t\rightarrow\infty}\xi_{\mathrm{eff}}=\xi$, $\forall b$.
In order to evaluate the effective decay exponent in a neighborhood of a given time coordinate, we choose $b=1.1$. 

The plot in Fig.~\ref{fig:eff_exponent} shows the effective exponent $\xi_{\mathrm{exp}}(\tilde{t})$ in the interval $\tilde{t}\in[0,10]$ for various values of $\Omega$ (the same used in Fig.~3(c) of the main text). For very small $\Omega=0.01$ value, one can identify a power-law behavior with effective exponent $\xi_{\mathrm{eff}}(\tilde{t})=\xi \approx 1/2$ throughout the considered domain, since $\xi_{\mathrm{eff}}$ converges to that value. Conversely, by increasing $\Omega$, an approximate algebraic decay $\tilde{N}(\tilde{t})\sim \tilde{t}^{-\xi_M}$ is observed, where the effective exponent reaches a maximum value $\xi_M$ and remains constant within an intermediate rescaled time window. For instance, the selected values $\Omega=0.05$, $\Omega=0.1$ are characterized by $\xi_M\approx 0.45$, $\xi_M\approx 0.4$ within the rescaled time interval $\tilde{t}\sim [3,6]$, $\tilde{t}\sim[2,4]$, respectively. For longer times the effective exponent $\xi_{\mathrm{eff}}$ drifts away from the aforementioned values and slowly decreases. This signals the onset of a non-algebraic (since $\xi_{\mathrm{eff}}$ is not constant) slow decay. As $\Omega$ is increased, the approximate decay exponent $\xi_{M}$ therefore decreases, the time window where it is observed is anticipated and shrinks. For $\Omega=0.01$, the non-algebraic behavior takes place for longer times than those shown in the plot ($\tilde{t}<10$). Conversely, for large values $\Omega=1$, the intermediate power law decay is completely wiped out, no exponent can be identified and the non-algebraic decay immediately sets in. 

\end{document}
