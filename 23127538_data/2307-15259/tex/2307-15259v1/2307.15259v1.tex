

\documentclass{article}
\usepackage{amsfonts}
\usepackage{amsmath}
\usepackage{amssymb}
\usepackage{amsthm}
%\usepackage{nopageno}
%\usepackage{mathtools}

%\usepackage{fontspec}
%\usepackage{unicode-math}

\renewcommand\labelenumi{(\roman{enumi})}

%\usepackage{graphicx,float}
\usepackage{enumerate}  
%\usepackage{color}

%\hoffset  = -5pt
%\topmargin = 5pt  %20pt
%\headheight = -5pt %12pt
%\headsep = 0pt %25pt
%\textheight = 660pt %592pt
%\setlength{\oddsidemargin}{2pt}    %{15.5pt} 
%\setlength{\evensidemargin} {2pt}    %{15.5pt}
%\marginparwidth = 2pt  %35pt
%\textwidth = 500pt    %390pt


% math help

\newtheorem{thm}{Theorem}[section]
\newtheorem{lemma}[thm]{Lemma}
\newtheorem{prop}[thm]{Proposition}
\newtheorem{cor}[thm]{Corollary}
\newtheorem{defi}[thm]{Definition}
\newtheorem{example}[thm]{Example}
\newtheorem{notation}[thm]{Notation}
\newtheorem{note}[thm]{Note}
\newtheorem{cond}[thm]{Set up}



\newcounter{mycounter}

\numberwithin{equation}{section}   % {thm} or subsection


\theoremstyle{definition}
\newtheorem{defn}{Definition}[section]
\newtheorem{conj}{Conjecture}[section]
\newtheorem{exmp}{Example}[section]
\theoremstyle{remark}
\newtheorem*{rem}{Remark}
%\newtheorem*{note}{Note}\newtheorem{case}{Case}
\newtheorem*{open}{Open question}



%----------------------------------------------------------------------------------------
%	TITLE SECTION
%----------------------------------------------------------------------------------------



%\runningheads{J. Hults \& K. Reinhold}{. }

\begin{document}
\title{Square Functions for Ritt Operators in $L^1$}
 
\author{Jennifer Hults and Karin Reinhold-Larsson}

%\address{
%Department of Mathematics and Statistics\\
%University at Albany, SUNY, Albany, NY 12222 \\} % Your institution
%\email{jhults@albany.edu,reinhold@albany.edu} % Your email address

%\author{Karin Reinhold-Larsson } 
%\address{
%Department of Mathematics and Statistics\\
%University at Albany, SUNY, Albany, NY 12222 } % Your institution
%\email{reinhold@albany.edu} % Your email address


%----------------------------------------------------------------------------------------

\maketitle


\begin{abstract}
$T$ is a Ritt operator in $L^p$ if $\sup_n n\|T^n-T^{n+1}\|<\infty$. From \cite{LeMX-Vq}, if $T$ is a positive contraction and a Ritt operator in $L^p$, $1<p<\infty$, the square function 
 $\left( \sum_n n^{2m+1} |T^n(I-T)^{m+1}f|^2 \right)^{1/2}$ is bounded. We show that if $T$ is a Ritt operator in $L^1$,
\[Q_{\alpha,s,m}f=\left( \sum_n n^{\alpha} |T^n(I-T)^mf|^s \right)^{1/s}\] is bounded $L^1$ when $\alpha+1<sm$, and examine related questions on variational and oscillation norms.
\end{abstract}




\section{Introduction}
Let $(X,\beta,m)$ be a non--atomic, separable probability space and $\tau$ an invertible measure--preserving transformation on $(X,\beta,m)$. 
A probability measure $\mu$ on $\mathbb Z$ defines the contraction operator
$\tau_{\mu}f(x) = \sum_k \mu(k) f(\tau^k x)\label{eq:1}$, for $x\in X$, $f\in L^p(X)$, with $p\ge 1$. 
In \cite{BJR-conv}, Below, Jones and Rosenblatt  showed that if the measure $\mu$ satisfies the bounded angular condition 
\[\sup_{|t|<1/2} \frac {|1-\hat\mu(t)|}{1-|\hat\mu(t)|} <\infty,\] 
then both, the maximal function $Mf=\sup_{n\ge 1} |\tau_{\mu}^n f|$ and the square function  $Sf=(\sum_n n |\tau_{\mu}^n-\tau_{\mu}^{n+1}f|^2)^{1/2}$, are bounded operators in $L^p$; and the powers $\tau_{\mu}^nf(x)$ converge almost everywhere in $L^p$ ($1<p<\infty$).  Additional control on convergent sequences can be established with variation and oscillation norms. Jones and Reinhold \cite{JR} proved that, for measures satisfying the bounded angular condition, the variation and oscillation  norms 
\begin{align*} \| &\{\tau_{\mu}^nf\} \|_{v(s)}=\sup_{\{n_k\}} \left(\sum_k |\tau_{\mu}^{n_k} f-\tau_{\mu}^{n_{k+1}} f|^s \right)^{1/s} \mbox{ and } \\
\| &\{\tau_{\mu}^nf\} \|_{o(s)}=\left(\sum_k  \sup_{\{m_k\le n,m \le m_{k+1}\}} |\tau_{\mu}^{n} f-\tau_{\mu}^{m} f|^s \right)^{1/s},\end{align*}
where the supremum in the later is taken over all non decreasing sequences $\{m_k\}$ of positive integers,  
are bounded in $L^2$ for any $s>2$. 


The natural question was whether properties of the contraction operator $\tau_{\mu}$ also hold for other type of contractions. 
Let $Y$ be a Banach space. We say that  $T\in  \mathcal L(Y)$  satisfies  the {\bf resolvent condition}  if there exists a constant $C$ such that  
\[ |z-1|\|(T-z)^{-1}\| \le C  \mbox{ for all } |z|>1.\] 
Ritt \cite{Ritt} proved that the resolvent condition implies the operator is power bounded, $\sup_{n\ge 1} \|T^n\|<\infty$. 
Nagy and Zem\'anek \cite{NZ}, and independently Lyubich \cite{Lyub}, (see also \cite{Nev,LeM-H}) showed that 
the resolvent condition and $\sigma(T)\subset $ unit disk,
 is equivalent to the operator $T$ being power bounded and satisfying  $\sup_n n\|T^n-T^{n+1}\| <\infty$.  

\begin{defi} $T\in \mathcal L(Y)$ is a Ritt operator if $T$ is power bounded and $\sup_n n\|T^n-T^{n+1}\| <\infty$.
\end{defi}


It turns out that $T$ is a Ritt operator if there is $\gamma\in (0,\pi/2)$ such that its 
spectrum is included in the closure of a Stolz domain $B_{\gamma}$ of the unit disk, that is, the interior of the convex hull of 1 and the disc $D(0, \sin \gamma)$. In other words, $\sigma(T)$ satisfies the bounded angular condition. See \cite{LeM-H,Lyub,Nev,NZ}.


\begin{lemma}
$T$ is a Ritt operator if and only if there exists an angle $\gamma\in (0,\pi/2)$ such that $\sigma(T)\subset B_{\gamma}$ 
and for any $\beta\in(\gamma,\pi/2)$, the set
$ \{(z-1)(z-T)^{-1}:z \not\in \bar B_{\beta}\}$ is bounded.
\end{lemma}

This characterization allowed Blunke \cite{Bl} to prove an interpolation theorem for Ritt operators. 
\begin{thm} (Theorem 1.1 \cite{Bl}) Let $p,q\ge 1$ and $T \in \mathcal L(L^p)$ be power bounded and Ritt in $L^p$.
If $T$ is power bounded on $L^q$ then $T$ is power bounded and Ritt on
$L^r$ for any $r$ strictly between $p$ and $q$.
\end{thm}

Thus, for $T$ and $L^1-L^{\infty}$ contraction, If $T$ is Ritt in one $L^p$, $p\ge 1$, it is Ritt in all $L^q$'s $1<q<\infty$. 
In particular, if $T$ is Ritt in $L^2$ it is Ritt in all $L^p$, $1<p<\infty$. Thus, the operator being Ritt in $L^1$ does not immediately follow from being Ritt in another $L^p$.


With the resolvent condition, Le Merdi \cite{LeM-H} developed $H^{\infty}$ functional calculus for Ritt operators, 
yielding new insights regarding the convergence of powers and maximal and square functions estimates \cite{LeM-H,LeMX-max,LeMX-Vq}. 
In particular,
LeMerdi and Xu \cite{LeMX-Vq} (Prop. 4.1, Thm. 4.4 \& Thm. 5.6) established the following implications for Ritt operators in $L^p$, $1<p<\infty$, connecting the Ritt property with the convergence of a square function.
\begin{thm} \label{thm:LeM} Let $(X,m)$ be a $\sigma$--finite measure space, $1<p<\infty$, and $T$ a positive contraction of $L^p(X,m)$. 
If $\sup_n n \|T^n-T^{n+1}\|<\infty$, then, for any fixed integer $m\ge 0$ and any real number $s>2$, 
\begin{itemize}
\item $\left( \sum_n (n+1)^{2m+1} |T^n(I-T)^{m+1}f|^2 \right)^{1/2}$, 
\item $\| \{T^nf\} \|_{v(s)}$, and more generally $\| \{n^m T^n(I-T)^mf\} \|_{v(s)}$, 
\item for any increasing sequence $\{n_k\}$, $\| \{T^nf\} \|_{o(2)}$, and more generally $\| \{n^m T^n(I-T)^mf\} \|_{o(2)}$, 
\end{itemize}
are bounded in $L^p$.
\end{thm} 


Cohen, Cuny and Lin  \cite{CCL}  completed the study by providing equivalent conditions between the Ritt, spectral conditions and the square functions.
\begin{thm}\label{thm:CCL}
Let $(X,m)$ be a $\sigma$-finite measure space, $1 <p <\infty$ and $T$ a positive contraction on $L^p(X, m)$. Then the following are equivalent:
\begin{enumerate}
\item  $\sup_n n\|T^n-T^{n+1}\| <\infty$,
\item there exists a constant $C_p>0$ such that $\|(\sum_n n |T^n-T^{n+1}f|^2)^{1/2}\|\le C_p \|f\|_p$,
\item there exists a closed Stolz region $\Sigma$ and a constant $K>0$ such that
$\|u(T)\| \le K \sup_{z\in \Sigma} u(z)$ for every rational function $u$ with poles outside $\Sigma$.
\end{enumerate}
\end{thm}



Results in $L^1$ turned out to be more elusive. 
 Bellow and Calderon   \cite{BC}  showed that for $\tau_{\mu}$, the maximal function $Mf$ is weak (1,1) for centered measures $\mu$ with finite second moment. Such measures have bounded angular ratio. Losert \cite{Losert1,Losert2} constructed measures $\mu$ without bounded angular ratio for which pointwise convergence of $\tau^n_{\mu}$ failed.  Wedrychowicz  \cite{Chris} gave conditions for $Mf$ to be weak (1,1) for measures with bounded angular ratio but without a finite second moment.  
 
Next we extend $\tau_{\mu}$ to a broader setting. 
\begin{defi}
Let $T\in \mathcal L(Y)$ be a power bounded linear operator and $\mu$ a finite signed measure on the integers.  If $supp(\mu)\not\subset \mathbb Z^+\cup\{0\}$, we assume $T$ is invertible and $\sup_{n\in \mathbb Z} \|T^n\|<\infty$.  We define the operator induced by $\mu$ as
\[T_{\mu}f=\sum_k \mu(k) T^kf.\] 
\end{defi}
Note that with the restrictions on the operator $T$, $T_{\mu}$ is well defined for any $f\in Y$. %$f\in L^1$. 
In most of the paper, the Banach is $L^1$ or $L^p$ depending on the context.
 
 Dungey  \cite{Dunn} proved that $T_{\mu}$ is Ritt in $L^1$ for measures supported on the positive integers satisfying a spectral property (M1) that guarantee bonded angular ratio.  
\begin{thm} (Dungey Theorem 4.1  \cite{Dunn}) \label{thm:D}
Let $\mu$ be a probability measure on $\mathbb Z^+$ for which there exists $0<\alpha<1$ such that (i) $|\mbox{Re} \, \hat\mu(t)|\le 1-c |t|^{\alpha}$, and 
(ii)  $|\hat\mu'(t)|\le c\, |t|^{\alpha-1}$ for $0<|t|\le 1/2$.
Then $T_{\mu}$ is Ritt in $L^1$.
\end{thm}
%Thm 1.5
 
 
 Inspired by \cite{Chris} and % Dungey (2011) 
 \cite{Dunn}, Cuny  \cite{Cuny-weak} considered more general spectral conditions (M2) which apply to measures with support on $\mathbb Z$ and yield week type inequalities.

\begin{thm} \label{thm:weak} %Prop 2.3 and 2.5,, 2.6
\cite{Cuny-weak} Let $\mu$ be a probability measure on $\mathbb Z$ such that its Fourier transform $\hat\mu(t)=\sum_k \mu(k) e^{-2\pi i kt}$
is twice continuously differentiable on $0<|t|<1$  and such that there exists a continuous function $h(t)$ on $|t|\le 1$ with $h(0)=0$, $h(-t)=h(t)$, continuously differentiable on $0<|t|<1$ satisfying the following conditions:
(i)  $|\hat\mu(t)|\le 1-c\, h(t)$,
(ii)  $|t \, \hat\mu'(t)|\le c\, h(t)$,
(iii)  $|\hat\mu'(t)|\le c\, h'(t)$, and
(iv)  $|t \, \hat\mu''(t)|\le c\, h'(t)$.\\
Let $T$ be the shift in $\mathcal l^1(\mathbb Z)$. Then 
$m(x\in \mathbb Z: \sup_n |T_{\mu}^nf(x)|>\lambda) \le \frac{C}{\lambda} \|f\|_1$
and $T_{\mu}$ is weak-$\mathcal l^1$-Ritt:  \\
$ m(x\in \mathbb Z: \sup_n n^m |(T_{\mu}^n-T_{\mu}^{n+m})f(x)|>\lambda) \le \frac{C_m}{\lambda} \|f\|_1$
for any $f\in \mathcal l^1 (\mathbb Z)$.\\
If in addition, $h$ satisfies (v) $h(t)\le c th'(t)$ for $0<t<1$,
then  $\sup_n n\|\mu^n-\mu^{n+1}\|_{l^1}<\infty$. 
\end{thm} 


 
If $0<\alpha<1$, we  define $(I-T)^{\alpha}$ by considering the series expansion for 
\[(1-x)^{\alpha}=1-\sum_{k\ge 1} g(\alpha,k) x^k,\] 
for $|x|\le 1$. That is, $(I-T)^{\alpha} = I - \sum_{l\ge 1} g(\alpha,k) T^k$. The coefficients satisfy $g(\alpha,k)=\frac{\alpha |\alpha-1| \ldots |\alpha-k+1|}{k!} \ge 0$ and $\sum_k g(\alpha,k)=1$. See \cite{Dunn}, \cite{LinDer}. In other words, $I-(I-T)^{\alpha}=T_{\nu_{\alpha}}$ where $\nu_{\alpha}$ is the probability measure on $\mathbb Z^+$ with \begin{equation}\nu_{\alpha}(k)= g(\alpha,k) \label{eqn:valpha}.\end{equation}
By separating the integer part from the fractional part, we can define  $(I-T)^m$ for any real $m>0$ . 
In the particular case of $T_{\mu}$, we note that
\[T_{\mu}^n(I-T_{\mu})^m f = T_{\nu_{n,m}} f\]
where $\nu_{n,m}$ is a signed measure on $\mathbb Z$ satisfying $\hat \nu_{n,m}(t) = \hat \mu^n(t) (1-\hat\mu(t))^m$.

\begin{defi}
For $s,m>0$, define the "generalized square functions" (associated with the operator $T$) as:
\[Q_{\alpha,s,m}f =Q^{T}_{\alpha,s,m}f = \left(\sum_{n=1}^{\infty} n^{\alpha} |T^n(I-T)^m f|^{s} \right)^{1/s}.\]
\end{defi}

To prove the boundenes of $Q^{T}_{\alpha,s,m}f$ for general Ritt operators in $L^1$, we study first the case when $T=T_{\mu}$. In this case, we work with probability measures for which $T_{\mu}$ is Ritt in $L^1$.

We say that a measure $\nu$ on $\mathbb Z$ satisfies condition M1 if $\hat\nu$ is continuous differentiable on $0<|t|<1$, and there exists $a>0$ such that $|\hat\nu(t)|\le 1-c |t|^{a}$ (for some $c>0$) and  $|\hat\nu'(t)|\lesssim |t|^{a-1}$, for $0<|t|<1$. We say $\nu$ satisfies condition M2 if it satisfies conditions (i) to (v) of Theorem \ref{thm:weak}.

Either condition guarantees the measure has bounded angular ratio. Since, if $|\hat\mu(t)|<1-ch(t)$,
\[ |1-\hat\mu(t)| \le | \int_0^t \hat\mu'(u) du | \le \int_0^{|t|} h'(u) du = h(t) \le \frac{1-|\hat\mu(t)|}{c}.\]

There are many examples of measure satisfying M1 or M2.
If $\mu$ is a centered measure with finite second moment, then it satisfies M1 with $a=2$. 
The next example, due to Dungey \cite{Dunn}, exhibits a non-centered measure  without finite first moment.

\begin{example} For fixed $0<\alpha<1$, let $\nu_{\alpha}$ be the probability measure on $\mathbb Z$ defined in (\ref{eqn:valpha}). 
Then $\hat\nu_{\alpha}(t) = 1-(1-e^{2\pi i t})^{\alpha}$. 
By Proposition 3.3 of \cite{Cuny-weak}, 
$\mu$ has bounded angular ratio and satisfies property M1. There are $c, c'>0$ such that
\[ |\hat\mu(t)|\le 1-\frac{|1-\hat\mu(t)|}{c} \le 1-c' |t|^{\alpha}.\] 

Moreover, with this property and  Theorem \ref{thm:D}, (Theorem 1.1 \cite{Dunn}),
 $T_{\nu_{\alpha}} = I-(I-T)^{\alpha}$ is a Ritt operator. 
\end{example}

More examples of measures satisfying M2 can be found in \cite{Cuny-weak}. Now we are ready for the main result.


%Thm 1.9
\begin{thm}\label{thm:main}
Let $\mu$ be a probability measure on $\mathbb Z$ satisfying condition M1 or M2.
 If $m>0$, and  $sm>\alpha+1$,
then $Q_{\alpha,s,m}^{T_{\mu}}f$ is a bounded operator in $L^1$.
\end{thm} 

\begin{rem}
Note that, by Theorem \ref{thm:CCL},  $Q_{1,2,1}f$  is bounded in $L^p$ for any $p>1$, but in $L^1$, $Q_{1,s,1}f$ is bounded for $s>2$. And in general, when $m=\alpha$, $Q_{m,s,m}f$ is bounded in $L^1$ for $s>1+1/m$.

Also, 
by Theorem \ref{thm:LeM}, $Q_{2m+1,2,(m+1)}f$ is bounded in $L^p$ for any $p>1$. But in $L^1$,
 $Q_{2m+1,s,(m+1)}f$ bounded  for any $s>2$,  and $Q_{\alpha,2,(m+1)}f$ bounded in $L^1$ for any $\alpha<2m+1$.
When $m=1$,  $Q_{\alpha,s,1}f$ is bounded in $L^1$ for $s>1+\alpha$.
\end{rem}

\begin{open} Is $Q_{\alpha,s,s}f$ bounded in $L^1$ for $sm=\alpha+1$? Is it weak (1,1) in $\mathcal l^1$?
\end{open}

Proposition 6.1 and Corollary 6.2 in \cite{CCL} show that for a power bounded Ritt operator $T$ in $L^p$ ($1<p<\infty$), 
$\sup_{n\ge 1} n^m \| T^n(I-T)^m\|<\infty$. And, by Theorem \ref{thm:weak}, $\sup_n n^m |T_{\mu}^n(I-T_{\mu})^mf|$ is weak (1,1) (when $T$ is the shift operator in $\mathcal l^1$).
The next corollary shows that for a factor slightly below $n^m$, the maximal function is bounded.


%Cor 1.10
\begin{cor} \label{cor:sup} Let $\mu$ be as in \ref{thm:main} and $0\le \alpha<m$. Then 
\begin{enumerate}
\item $\sup_n n^{\alpha} |T_{\mu}^n (I- T_{\mu})^{m}f|$ is bounded in $L^1$. 
\item Let $1 <p <\infty$ and $T$ a positive contraction on $L^p(X)$.
If $T$ a Ritt operator in $L^p$, then $\sup_n \frac{n^{m}}{\sqrt{\ln n}} |T^n (I- T)^{m}f|$ is bounded in $L^p$. % for any $\alpha<m$. 
\end{enumerate}
In either case, $\lim_{n\to \infty} n^{\alpha} |T_{\mu}^n (I- T_{\mu})^{m}f| = 0$ in norm and a.e..
\end{cor}



Theorem \ref{thm:LeM} showed that $\|n^{m} T_{\mu}^n(I-T_{\mu})^mf\|_{v(s)}$ and $\|n^{m} T_{\mu}^n(I-T_{\mu})^mf\|_{o(s)}$ are bounded in $L^p$, for $s>2$ and $p>1$ , even in the case $m=0$.
In $L^1$, we obtained the following variations and oscillations results.
%Prop 1.11
\begin{prop} \label{prop:genvar}
Let $\mu$ be as in \ref{thm:main}.  Let $s\ge 1$, $m>0,$ and  $\beta\ge 0$  be fixed. If  $\beta=0$ or if $s(m-\beta)>1$, then both 
 $\|n^{\beta} T_{\mu}^n(I-T_{\mu})^mf\|_{v(s)}$ and $\|n^{\beta} T_{\mu}^n(I-T_{\mu})^mf\|_{o(s)}$ are bounded in $L^1$.
\end{prop}

%1.13
\begin{open} Are there values of $\beta>-1$ and $s>0$ for which
$\|n^{\beta} T_{\mu}^nf\|_{v(s)}$ and $\|n^{\beta} T_{\mu}^nf\|_{o(s)}$ are bounded in $L^1$.
\end{open}


The result in Proposition \ref{prop:genvar} stays shy of the case $m=0$. The next result handles cases of differences along subsequences with increasing gaps.

%Prop 1.12
\begin{prop} \label{prop:longvar}
Let $\mu$ be as in \ref{thm:main}, $\{n_k\}$ an increasing sequence such that $ n_{k+1} -n_k \sim n_k^{\alpha}$ for some $0<\alpha<1$, and $0\le \beta<1-\alpha$. Then
\begin{enumerate}
\item 
$\left( \sum_k n_k^{\beta s} |T_{\mu}^{n_k}f-T_{\mu}^{n_{k+1}}f|^s \right)^{1/s}$
is bounded in $L^1$ for $s>(1-\alpha)/(1-\alpha-\beta)$.
 In particular,  $\left( \sum_k  |T_{\mu}^{n_k}f-T_{\mu}^{n_{k+1}}f|^s \right)^{1/s}$ is bounded in $L^1$ for $s>1$;
\item 
$\left( \sum_k n_k^{\beta s}  \|\{T_{\mu}^n f:n_k\le n<n_{k+1}\} \|_{v(s)}^s\right)^{1/s}$
is bounded in $L^1$ for $s>1/(1-\alpha-\beta)$; \\
and $\left( \sum_k n_k^{\beta s}  \max_{ n_k\le n<n_{k+1} } |T_{\mu}^n f-T_{\mu}^{n_k} f|^s \right)^{1/s} $  is
bounded in $L^1$ for $s>(1-\alpha)/(1-\alpha-\beta)$.
\end{enumerate}
\end{prop}


We conclude noticing that any Ritt operator in $L^1$ inherits the same properties as those obtained for $T_{\mu}$. 



%1.14
\begin{thm} Let $(X,m)$ be a $\sigma$--finite measure space and $S$ a Ritt operator in $L^1(X)$, $m>0$ and $s\ge 1$. Then 
\begin{enumerate}
\item $Q^S_{\alpha,s,m}f$ %=\left( \sum_n n^{\alpha} |S^n(I-S)^mf|^s \right)^{1/s}$ 
is bounded in $L^1$ for $sm>\alpha+1$;
\item $\sup_n n^{\alpha} |S^n (I- S)^{m}f|$ is bounded in $L^1$ for any  $\alpha<m$;
\item   both  $\|n^{\beta} S^n(I-S)^mf\|_{v(s)}$ and $\|n^{\beta} S^n(I-S)^mf\|_{o(s)}$ are bounded in $L^1$ if $\beta=0$ or $\beta>0$ and $s(m-\beta)>1$;
\item for any increasing sequence $\{n_k\}$ such that $ n_{k+1} -n_k \sim n_k^{\alpha}$ for some $\alpha\in(0,1)$, then 
(i) 
$\left( \sum_k n_k^{\beta s} |S^{n_k}f-S^{n_{k+1}}f|^s \right)^{1/s}$ and \\
$\left( \sum_k n_k^{\beta s}  \max_{ n_k\le n<n_{k+1} } |S^n f-S^{n_k} f|^s \right)^{1/s} $  
are bounded in $L^1$ for $s>(1-\alpha)/(1-\alpha-\beta)$; and\\
(ii)  
$\left( \sum_k n_k^{\beta s}  \|\{S^n f:n_k\le n<n_{k+1}\} \|_{v(s)}^s\right)^{1/s}$
is bounded in $L^1$ for $s>1/(1-\alpha-\beta)$.
\end{enumerate}
\end{thm}

\begin{proof}
By Theorem 1.3. in \cite{Dunn}, there exists  a power bounded %Kreiss 
operator $T$ and $\gamma\in (0,1)$ 
such that 
$S=I-(I-T)^{\gamma} = T_{\nu_{\gamma}}$, where $\nu_{\gamma}$ is the probability measure on $\mathbb Z$ defined in (\ref{eqn:valpha}).  
Then Theorem \ref{thm:main}, Corollary \ref{cor:sup}, Propositions \ref{prop:genvar} and \ref{prop:longvar} apply to $S$.
\end{proof}




%%%%%%%%%%%%%%%%%%%%%%%%
\section{Proofs of Results}
In these notes, $c$ and $C$ denote constants whose values may change from one instance to the next. We simplify $e(x)=e^{2\pi i x}$. And for  $0\le x,y$, we say $x \lesssim y$ if there exists a constant $c>0$ such that $x \le c y$.

\begin{lemma}\label{lem:basic} 
Let $\Delta_n$ be a sequences of (finite) signed measures on $\mathbb Z$. 
Let
\begin{align*} 
A=  \int_{|t|<1/2} \frac 1{|t|} \left( \sum_n |\hat \Delta_n(t)|^s \right)^{1/s} dt, & \
\quad C=\sum_{k\neq 0} \frac 1{|k|} \left( \sum_n |\hat \Delta_n(1/|k|)|^s \right)^{1/s}, 
\\
B=\int_{|t|<1/2} |t| \left( \sum_n |\hat \Delta''_n(t)|^s \right)^{1/s} dt, & \ D= \sum_{k\neq 0} \frac 1{|k|^2} \left( \sum_n |\hat \Delta'_n(1/|k|)|^s \right)^{1/s} \\
\tilde B=\int_{|t|<1/2} \ln |t|^{-1} \left( \sum_n |\hat \Delta'_n(t)|^s \right)^{1/s} dt,
& \quad E=\left( \sum_n  \left|   \Delta_n(0)   \right|^{s} \right)^{1/s}.
\end{align*}
If $A, B, C, D$ and $E$ are all finite, or $A, \tilde B, C$ and $E$ are all finite, then, for any $f\in X$,
\[ \left\| \left( \sum_n |T_{\Delta_n}f|^s \right)^{1/s} \right\| \lesssim \|f\|.\]
\end{lemma}

\begin{proof}
Let $\kappa = \sup_{n\ge 1} \|T^n\|$. Without loss of generality, we assume $\kappa=1$.

\begin{align*}   \left\| \left( \sum_n |T_{\Delta_n}f|^s \right)^{1/s} \right\|
= & \left\| \left(\sum_n  \left| \sum_k \Delta_n(k) T^kf\right|^{s} \right)^{1/s} \right\| \\
\le &  c \left\|  \left( \sum_n  \left|  \sum_{k\neq 0} \int_{|t|<1/2|k| } \hat \Delta_n(t) e(kt) dt  \,  T^kf \right|^{s} \right)^{1/s} \right\| \\
 & + c \left\|   \left( \sum_n  \left|  \sum_{k\neq 0}   \int_{1/2|k|<|t|<1/2} \hat \Delta_n(t) e(kt) dt  \,  T^kf \right|^{s} \right)^{1/s} \right\| \\
& + c \left\|   \left( \sum_n  \left|   \Delta_n(0)   \right|^{s} \right)^{1/s} |f| \right\| \\
 =  & c (  \mbox{I} + \mbox{II}+  E \|f\|).
\end{align*}


For the first term we have,
\begin{align*}
  \mbox{I} 
=  & \left\|  \left( \sum_n  \left|  \sum_{k\neq 0} \int_{|t|<1/2|k|} \hat \Delta_n(t) e(kt) dt  \,  T^kf\right|^{s} \right)^{1/s} \right\| \\
\le & \|f\| \sum_{k\neq 0} \left( \sum_n \left|   \int_{|t|<1/ 2|k| } \hat \Delta_n(t) e(kt) dt   \right|^{s} \right)^{1/s}\\
\le &   \|f\| \int_{|t|<1/2} \frac1{|t|}  \left( \sum_n  \left| \hat \Delta_n(t) \right|^s \right)^{1/s}  \, dt
= A \|f\|.
  \end{align*}

For the second term we have,
\begin{align*}
\mbox{II} \le & \left\|  \left( \sum_n  \left|  \sum_{k\neq 0}   \int_{1/2|k| <|t|<1/2} \hat \Delta_n(t) e(kt) dt  \,  T^kf \right|^{s} \right)^{1/s} \right\| \\
\le & \left\| \sum_{k\neq 0}  \left( \sum_n  \left |  \int_{1/2|k|<|t|<1/2} \hat \Delta_n(t) e(kt) dt \right |^s \right)^{1/s} |T^kf | \right\|\\
\le & \|f\|  \sum_{k\neq 0}  \left( \sum_n  \left |  \int_{1/2|k|<|t|<1/2} \hat \Delta_n(t) e(kt) dt \right |^s \right)^{1/s}.
\end{align*}

The integrand decomposes as
\begin{align*}
\left| \int_{1/|k| <|t|<1/2} \hat \Delta_n(t) e(kt) dt \right| \le & 
 \left|\int_{1/|k| <|t|<1/2} \hat \Delta'_n(t) \frac{e(kt)}{2\pi k} dt \right| \\
& + \left|\frac{\hat \Delta_n(1/|k| ) e(k/|k| )}{2\pi k} - \frac{\hat \Delta_n(-1/|k| ) e(-k/|k| )}{2\pi k} \right|\\  
\le &  \left|\int_{1/|k| <|t|<1/2} \hat \Delta''_n(t) \frac{e(kt)}{4\pi^2 k^2} dt \right| \\
 & + \left|\frac{\hat \Delta_n(1/|k| ) e(k/|k| )}{2\pi k} - \frac{\hat \Delta_n(-1/|k| ) e(-k/|k| )}{2\pi k} \right|\\  
 &  +  \left|\frac{\hat \Delta'_n(1/|k| ) e(k/|k| )}{4\pi^2 k^2} - \frac{\hat \Delta'_n(-1/|k| ) e(-k/|k| )}{4\pi^2 k^2} \right| \\
 = & \mbox{II}_1 + \mbox{II}_2 + \mbox{II}_3.
 \end{align*}

Using the first inequality
\begin{align*}
\sum_{k\neq 0}  &  \left( \sum_n   \left |  \int_{1/2|k|<|t|<1/2} \hat \Delta'_n(t)  \frac{e(kt)}{2\pi k}  dt \right |^s \right)^{1/s} \\
\lesssim & \sum_{k\neq 0} \frac 1{k}  \int_{1/2|k|<|t|<1/2} \left( \sum_n  \left | \hat \Delta'_n(t) \right |^s \right)^{1/s} \ dt  \\
\lesssim & \int_{0<|t|<1/2}  \ln |t|^{-1} \,  \left( \sum_n  \left | \hat \Delta'_n(t) \right |^s \right)^{1/s}\, dt  = \tilde B,
\end{align*}
and using the second one,
\begin{align*}
\sum_{k\neq 0}  & \left( \sum_n  \left |  \int_{1/2|k|<|t|<1/2} \hat \Delta''_n(t) \frac{e(kt)}{4\pi^2 k^2}  dt \right |^s \right)^{1/s} \\
\lesssim  & \sum_{k\neq 0} \frac 1{k^2}  \int_{1/2|k|<|t|<1/2} \left( \sum_n  \left | \hat \Delta''_n(t) \right |^s \right)^{1/s} \ dt  \\
\lesssim & \int_{0<|t|<1/2}  |t| \,  \left( \sum_n  \left | \hat \Delta''_n(t) \right |^s \right)^{1/s}\, dt  = B .
\end{align*}

Thus,
$\mbox{II}\lesssim (\tilde B+C) \|f\|$  in the first case, and $\mbox{II}\lesssim (B+C+D) \|f\|$ in the second case.

\end{proof}

The proof of this lemma can be adapted for the following setting.


\begin{lemma}\label{lem:basic2} 
Let $\Delta_n$ and $T$ as in Lemma \ref{lem:basic}, $\{n_k\}$ an increasing sequence and $I_k=[n_k,n_{k+1})$.
Let
\begin{align*}
A=  \int_{|t|<1/2} \frac 1{|t|}        \left( \sum_k n_k^{\beta} \max_{n\in I_k} |\hat \Delta_{n}(t)|^s \right)^{1/s} dt, & \
C=\sum_{l\neq 0} \frac 1{|l|}       \left( \sum_k n_k^{\beta} \max_{n\in I_k} |\hat \Delta_{n}(1/|l|)|^s \right)^{1/s}, 
\\
B=\int_{|t|<1/2} |t|     \left( \sum_k n_k^{\beta} \max_{n\in I_k} |\hat \Delta''_{n}(t)|^s \right)^{1/s} dt, & \
D= \sum_{l\neq 0} \frac 1{|l|^2}       \left( \sum_k n_k^{\beta} \max_{n\in I_k} |\hat \Delta'_{n}(1/|l|)|^s \right)^{1/s},
\\
\tilde B=\int_{|t|<1/2} \ln |t|^{-1} \left( \sum_n |\hat \Delta'_n(t)|^s \right)^{1/s} dt, & \
 E=      \left( \sum_k n_k^{\beta} \max_{n\in I_k}  \left|   \Delta_{n}(0)   \right|^{s} \right)^{1/s}.
\end{align*}
If either $A, B, C, D, E$ are all finite, or $A, \tilde B, C, E$ are all finite, then, for any $f\in X$,
\[ \left\|    \left( \sum_k n_k^{\beta} \max_{n\in I_k} |T_{\Delta_{n}}f|^s \right)^{1/s} \right\| \lesssim  \|f\|.\]
\end{lemma}

%%%%%%%%%%%%%%%%%%%%%%%%%%%%%%%%
\bigskip
%1.9
\noindent {\it Proof of Theorem \ref{thm:main}:}\\
Let's assume $\mu$ satisfies condition M2.
We'll apply Lemma \ref{lem:basic} with $\Delta_n$ the measure on the integers defined by $\hat\Delta_n=n^{\alpha/s} \hat\mu^n \hat(1-\hat\mu)^m$, that is 
$T_{\Delta_n}=n^{\alpha/s} T_{\mu}^n (I-T_{\mu})^m$. 
The case for $\alpha<-1$ is immediate. We'll address the cases $\alpha\ge -1$. 

For $\alpha>-1$,
\[\sum_n n^{\alpha}  |\hat\mu(t)|^{ns} \le \sum_n n^{\alpha}  (1-c \, h(t))^{ns}    \lesssim   \frac 1{h(t)^{\alpha+1}}.\]%\frac1{(1-|\hat\mu(t)|)^{\alpha+1}} ,\]
For $\alpha=-1$,
\[\sum_n \frac 1n  |\hat\mu(t)|^{ns} \le \sum_n \frac 1n  (1-ch(t))^{ns}= | \ln(1-(1-ch(t))^s) | % \lesssim \frac 1{(1-|\hat\mu(t)|)^{\gamma}}
 \lesssim \frac 1{h(t)^{\gamma}}\]
for any $\gamma>0$. 

Note that 
\[|1-\hat\mu(t)|  \le c \int_0^{|t|} h(s)' ds = c h(t). \] % \le c (1-|\hat\mu(t)|).\]

Following Lemma \ref{lem:basic}, we estimate
\begin{align*}
A=\int_{|t|<1/2} \frac1{|t|} &  \left( \sum_n  \left| \hat\Delta_n(t) \right|^s \right)^{1/s}  \\
= &    \int_{|t|<1/2}  \left( \sum_n n^{\alpha} | \hat\mu(t)|^{ns} \right)^{1/s} \frac{|1-\hat\mu(t)|^{m}}{|t|}  \\
\lesssim  & \begin{cases} \int_{0<t<1/2} h(t)^{m-(\alpha+1)/s-1}  \, h'(t) \, dt & \mbox{ for } \alpha>-1 \\
 \int_{0<t<1/2} h(t)^{m-\gamma/s-1}  \, h'(t) \, dt & \mbox{ for } \alpha=-1.
 \end{cases}
  \end{align*}
When $\alpha>-1$, the integral is finite for $sm>\alpha+1$, and when $\alpha=-1$, the integral is finite for $m>0$ since we can choose $\gamma$ arbitrarily small, say $\gamma=sm/2$.

% III
Similarly for E, with $sm>\alpha+1$ (or $0<\gamma<sm/2$ when $\alpha=-1$),
\begin{align*}
E= \left( \sum_n  \left|   \Delta_n(0)   \right|^{s} \right)^{1/s} \le &
\left( \sum_n n^{\alpha}    \int_{|t|<1/2} |\hat\mu(t)|^{ns} |1-\hat\mu(t)|^{sm} dt \right)^{1/s} \\
\lesssim  &\begin{cases} \left(  \int_{|t|<1/2}  h(t)^{ (sm-(\alpha+1))} dt \right)^{1/s} <\infty, & \mbox{ for } \alpha>-1 \\
 \left(  \int_{|t|<1/2}  h(t)^{ (sm-\gamma)} dt \right)^{1/s} <\infty, & \mbox{ for } \alpha=-1. 
  \end{cases}
\end{align*}


With $B_n=T^n (I-T)^m$,
\[
|\hat B'_n(t)| = | n \hat\mu^{n-1}(t) (\hat \mu'(t)) (1-\hat\mu(t))^m 
     - \hat\mu^{n}(t) \hat \mu'(t)(1-\hat\mu(t))^{m-1}|, \]
and 
\begin{align*}
|\hat B''_n(t)| = &| n(n-1) \hat\mu^{n-2}(t) (\hat \mu'(t))^2 (1-\hat\mu(t))^m 
     +n \hat\mu^{n-1}(t) \hat \mu''(t)(1-\hat\mu(t))^m \\
& - 2n m \hat\mu^{n-1}(t) (\hat \mu'(t))^2 (1-\hat\mu(t))^{m-1}  \\
&+ m (m-1) \hat\mu^{n}(t) (1-\hat\mu(t))^{m-2}  (\hat\mu'(t))^2\\
& - m \hat\mu^{n}(t) (1-\hat\mu(t))^{m-1}  \hat \mu''(t)| \\
\lesssim & n^2 |\hat\mu^{n-2}(t)| \frac{h'(t)}{|t|} h^{m+1}(t) + 
n |\hat\mu^{n-1}(t)| \frac{h'(t)}{|t|} h^{m}(t) \\
& + |\hat\mu^{n}(t)| \frac{h'(t)}{|t|} h^{m-1}(t).
\end{align*}
Thus, for $\alpha>-1$,
\[
\left(\sum_n   |\hat \Delta''_n (t)|^s=  \sum_n n^{\alpha}  |\hat B''_n (t)|^s\right)^{1/s}  
\lesssim    
h(t)^{(m-1)-(\alpha+1)/s} \frac{|h'(t)|}{|t|}.\]
% II_1
Since $(1+\alpha)<sm$,
\begin{align*}
B=   \int_{|t|<1/2} |t| \, \left( \sum_n   |\hat\Delta''_n(t)|^s  \right)^{1/s} dt   
 \lesssim &  \int_{0<t<1/2}   h(t)^{(m-1)-(\alpha+1)/s } \, h'(t) \, dt <\infty.   \label{eq:II1} 
\end{align*}

When $\alpha=-1$, choosing $0<\gamma<m$, the estimate is
\[   \int_{|t|<1/2} |t| \, \left( \sum_n   |\hat\Delta''_n(t)|^s  \right)^{1/s} dt   
 \lesssim   \int_{0<t<1/2}   h(t)^{m-\gamma-1 } \, h'(t) \, dt <\infty.  \]


For the remaining terms, we show the case $\alpha>-1$ and note that the estimates for $\alpha=-1$ follow similar arguments.
% II_2
For $(1+\alpha)<sm$, we have
\begin{align*}
 C= \sum_{k\neq 0} \frac 1{|k|} &
\left(    \sum_n   |\hat\Delta_n(1/|k| )|^s  \right)^{1/s}  \\
\lesssim   & \sum_{k\neq 0} \frac 1{|k|} \left(    \sum_n n^{\alpha} |\hat\mu(1/|k|)|^{ns} 
|1-\hat\mu(1/|k|)|^{sm}  \right)^{1/s} \nonumber  \\
\lesssim & \sum_{k> 0} \frac 1k  h(1/k)^{m-(\alpha+1)/s}
\le c+ \int_0^{1/2} \frac{h(t)^{m-(\alpha+1)/s}}t \, dt <\infty.
\label{eq:II2}
\end{align*}
Since
\[ \left( \sum_n  |\hat\Delta'_n(1/|k| )|^s\right)^{1/s} = \left( \sum_n n^{\alpha}  |\hat B'_n(1/|k| )|^s\right)^{1/s}
 \lesssim  h(1/k)^{m-(\alpha+1)/s} h'(1/k) .\]
 Then 
\begin{align*}
D = & \sum_{k\neq 0} \frac 1{|k|^2}
\left(    \sum_n   |\hat\Delta'_n(1/|k| )|^s  \right)^{1/s} \nonumber \\
 & 
  \lesssim     \sum_{k\neq 0} \frac 1{|k|^2}   h(1/k)^{m-(\alpha+1)/s} h'(1/k) <\infty.   \label{eq:II3}
\end{align*}

If instead of condition M2, we require M1, that is, $|\hat\mu(t)|\le 1-c |t|^{a}$ and $|\mu'(t)|\lesssim t^{a-1}$,
 we need only to use estimates for the first derivative and use $\tilde B$ in Lemma \ref{lem:basic},
\begin{align*}
\tilde B= &  \int_{|t|<1/2} \ln |t|^{-1} \, \left( \sum_n   |\hat\Delta'_n(t)|^s  \right)^{1/s} dt   \\
 \lesssim &  \int_{0<t<1/2}  \ln |t|^{-1} t^{a(m-(\alpha+1)/s)-1 } \,  dt <\infty,
\end{align*}
as long as $ms>\alpha+1$. The computations of the other terms (A,C and E) are the same as above but substituting $h(t)$ with $t^a$ and $h'(t)$ with $t^{a-1}$.
The case $\alpha=-1$ also follows from similar arguments.
~\hfill$\square$
  \bigskip
  
  %%%%. corolary 1.10
  \noindent {\it Proof of Corollay \ref{cor:sup}:}\\
An application of Abel's summation yields
\begin{align*}
n^{\alpha } |T_{\mu}^{n}(I-T_{\mu})^{m}f| \le & \sum_{k=0}^{n-1} ((k+1)^{\alpha }-k^{\alpha } )|T_{\mu}^k(I - T_{\mu})^{m}f| \\
& + \sum_{k=1}^n k^{\alpha } |T_{\mu}^{k-1}(1-T_{\mu})^{m+1}f|\\
\lesssim & \sum_{k=0}^{n-1} (k+1)^{\alpha  -1} |T_{\mu}^k (I- T_{\mu})^{m}f| \\
& + \sum_{k=1}^n k^{\alpha } |T_{\mu}^{k-1}(1-T_{\mu})^{m+1}f|\\
\lesssim &  \ Q_{\alpha -1,1,m}f + Q_{\alpha ,1,m+1}f.
\end{align*}
By Theorem \ref{thm:main},
 both generalized square functions on the right are bounded in $L^1$ for $\alpha<m$ and $m>0$. Therefore 
$\sup_n n^{\alpha } |T_{\mu}^n(I - T_{\mu})^{m})f|$ is also bounded in $L^1$.


When $1<p<\infty$ we have
\begin{align*}
n^{\alpha } & |T^{n}(I-T)^{m}f|  \\
\le & \left(\sum_{k=0}^{n-1} (k+1)^{2m-1} |T^k (I- T)^{m}f|^2 \right)^{1/2} \, \left( \sum_{k=1}^n \frac 1{k^{1+2(m-\alpha)}}\right)^{1/2}\\
& + \left(\sum_{k=1}^n k^{2m+1} |T^{k-1}(1-T)^{m+1}f|^2 \right)^{1/2} \, \left( \sum_{k=1}^n  \frac 1{k^{1+2(m-\alpha)}}\right)^{1/2}. \\
\end{align*}
If $\alpha<m$, 
\[n^{\alpha }  |T^{n}(I-T)^{m}f|  \lesssim  Q_{2m-1,2,m}f + Q_{2m+1 ,2,m+1}f,\]
which, by Theorem \ref{thm:LeM},  are bounded in $L^p$.\\
If $\alpha = m$,
\[n^{\alpha }  |T^{n}(I-T)^{m}f|  \lesssim  \sqrt{\ln n} (Q_{2m-1,2,m}f + Q_{2m+1 ,2,m+1}f),\]
it follows that
$\sup_{n >1} \frac{ n^{m} |T^{n}(I-T)^{m}f| }{\sqrt{\ln n}}$ is bounded in $L^p$.

%\end{proof}
~\hfill$\square$ \bigskip



%Variation. 1.11
\noindent {\it Proof of Proposition \ref{prop:genvar}:}\\

Let $\Delta_{n,m}f=T_{\mu}^n(I-T_{\mu})^mf$, and 
let $\{n_k\}$ any increasing sequence. 
%\begin{align*}
\[D_{k,\beta}=  n_k^{\beta} \Delta_{n_k,m} - n_{k+1}^{\beta} \Delta_{n_{k+1},m} 
=  n_k^{\beta} \sum_{r=n_k}^{n_{k+1}-1} \Delta_{r,m+1} - (n_{k+1}^{\beta}-n_k^{\beta} ) \Delta_{n_{k+1},m} 
%\end{align*}
\]
%Choose $1>\alpha>1/s'$,  %$1-\alpha<1/s$
\begin{align*}
\left( \sum_k |D_{k,\beta}f|^s \right)^{1/s}
\lesssim &  \sum_k  \sum_{r=n_k}^{n_{k+1}-1}  r^{\beta } |\Delta_{r,m+1}f|  
+ \left( \sum_k n_k^{s\beta}  |\Delta_{n_{k},m}f|^s  \right)^{1/s}
\end{align*}
Thus, 
\[ \| n^{\beta}\Delta_{n,m}f \|_{v(s)}  \lesssim Q_{\beta ,1,m+1}f + Q_{\beta s,s,m}f\]
which, by Theorem \ref{thm:main}, are bounded in $L^1$ for $s(m-\beta)>1$.


For $n_k\le n \le n_{k+1}$
\[ \max_{n_k\le n \le n_{k+1}}
\left| n^{\beta} \Delta_{n,m} - n_{k}^{\beta} \Delta_{n_{k},m}f \right|^s \le 
 \sum_{r=n_k}^{n_{k+1}-1} r^{\beta } |\Delta_{r,m+1}f| 
+  \sum_{r=n_k+1}^{n_{k+1}} n^{s\beta}  |\Delta_{n,m}f|^s
\]
Thus, 
\[ \| n^{\beta}\Delta_{n,m}f \|_{o(s)}  \lesssim Q_{\beta ,1,m+1}f + Q_{\beta s,s,m}f.\]
is bounded in $L^1$ for $s(m-\beta)>1$.

When $\beta=0$,
\[ \| \Delta_{n,m}f \|_{v(s)}  \lesssim Q_{0,1,m+1}f \quad
\mbox{ and }\quad
 \| \Delta_{n,m}f \|_{o(s)}  \lesssim Q_{0,1,m+1}f,\]
are bounded in $L^1$ for $m> 0$. 
\hfill$\square$

\bigskip
%%%%%%%%%%%%%%%%%%%%%%%
\noindent {\it Proof of Proposition \ref{prop:longvar}:}  %1.12
%Variation on special sequence: 
Let $\mu$ satisfy condition M2. The case for M1 is similar.
We'll apply Lemma \ref{lem:basic} with $\Delta_k = n_k^{\beta } (T_{\mu}^{n_k} f - T_{\mu}^{n_{k+1}})$.

We estimate, for $\gamma>\alpha$,
 \[\sum_k n_k^{s\gamma} (n_{k+1}-n_k) |\hat\mu(t)|^{n_k s}\lesssim \frac1{h(t)^{s\gamma+1}} .\]
 

 

\begin{align*}
 \mbox{A} = &
\int_{|t|<1/2} \frac 1{|t|} \left( \sum_k  n_k^{\beta s}  |\hat\mu(t)|^{sn_k} \, |1-\hat\mu(t)^{(n_{k+1}-n_k)}|^s \right)^{1/s} dt\\
\lesssim & \int_{|t|<1/2} \frac {|1-\hat\mu(t)|}{|t|} \left( \sum_k n_k^{\beta s} (n_{k+1}-n_k)^s |\hat\mu(t)|^{s n_k}  \right)^{1/s} \,  dt\\
\lesssim & \int_{|t|<1/2} \frac {|1-\hat\mu(t)|}{|t|} \left( \sum_k n_k^{(\alpha+\beta-\alpha/s) s} (n_{k+1}-n_k) |\hat\mu(t)|^{s n_k}  \right)^{1/s} \,  dt\\
\lesssim & \int_{|t|<1/2} \frac {h(t)}{|t|} \left(\frac 1{h(t)^{s(\alpha+\beta-\alpha/s)+1}}\right)^{1/s} \, dt 
 \lesssim  \int_{|t|<1/2} \frac {h'(t)}{h(t)^{(\alpha+\beta)+(1-\alpha)/s}} \, dt   <\infty,
  \end{align*}
for $\alpha+\beta+(1-\alpha)/s<1, s>1$. %, that is $\beta s'<1-\alpha$.
% II
Similarly, 
\begin{align*}
\mbox{E} \le &  \left(  \sum_k n_k^{(\alpha+\beta)s}   \int_{|t|<1/2} |\hat\mu(t)|^{n_k s}  |1-\hat\mu(t)|^s dt    \right)^{1/s} \\
\lesssim &  \left(  \int_{|t|<1/2}   h(t)^{s(1-(\alpha+\beta)-(1-\alpha)/s)} dt   \right)^{1/s}<\infty.
\end{align*}

For the next term,
\begin{align*}
n_k^{-\beta}  \hat\Delta''_k(t) = &  n_k(n_k-1) \hat\mu^{n_k-2}(t) (\hat \mu'(t))^2 - n_{k+1}(n_{k+1}-1) 
         \hat\mu^{n_{k+1}-2}(t)(\hat \mu'(t))^2 \\
&  +n_k \hat\mu^{n_k-1}(t) \hat \mu''(t) - n_{k+1} \hat\mu^{n_{k+1}-1}(t) \hat \mu''(t)  \\
= & n_k(n_k-1)\hat\mu^{n_k-2}(t) (1-\hat\mu^{n_{k+1}-n_k}(t)) (\hat \mu'(t))^2 \\
& +[n_k(n_k-1)-n_{k+1} (n_{k+1}-1)] \hat\mu^{n_{k+1}} (\hat \mu'(t))^2 \\
& +  n_k  \hat\mu^{n_k-1}(t) (1-\hat\mu^{n_{k+1}-n_k}(t)) \hat \mu''(t) \\
& +(n_k-n_{k+1}) \hat\mu^{n_{k+1}-1}(t) \hat \mu''(t).
\end{align*}
Estimating
\begin{align*}
n_k(n_k-1)-n_{k+1}(n_{k+1}-1)= & n_k^2-n_{k+1}^2+(n_{k+1}-n_k) \sim n_k^{\alpha}(n_{k+1}+n_k) + n_k^{\alpha} \\
\le  & n_k^{\alpha}(2n_k+n_k^{\alpha})  \lesssim n_k^{1+\alpha},
\end{align*}
we have
\begin{align*}
|\hat\Delta''_k(t)| \lesssim & \left[   n_k^{2+\alpha+\beta}  \hat\mu^{n_k-2}(t) h(t)^2 
     + n_k^{1+\alpha+\beta} |\hat\mu(t)|^{n_{k+1}-1}  h(t)  \right. \\
 & \left. + n_{k}^{1+\alpha+\beta} |\hat\mu(t)|^{n_{k+1}-2}  h(t) 
 +n_{k}^{\alpha+\beta} |\hat\mu(t)|^{n_{k+1}-1}  \right] \frac{h'(t)}{|t|}.
\end{align*}
Thus
\begin{align*}
\left(   \sum_k   |\hat\Delta''_k(t)|^s    \right)^{1/s} 
  \lesssim    \frac{1}{h(t)^{(\alpha+\beta)+(1-\alpha)/s) } }   \frac{h'(t)} {|t|}.
\end{align*}
% 
%
% III
Then,  for $\alpha+\beta+(1-\alpha)/s<1$,
\begin{equation*}
\mbox{B} =  \int_{|t|<1/2} |t|\,  \left( \sum_k  |\hat\Delta''_{k}(t)|^s  \right)^{1/s} \, dt 
  \lesssim   \int_{0<t<1/2} 
\frac {h'(t)}{h(t)^{(\alpha +\beta)+(1-\alpha)/s)) }}\, dt<\infty,
 \label{eq:II4}
\end{equation*}
and
%##############
% IV
\begin{align*}
\mbox{C} =&  \sum_{l\neq 0} \frac 1{|l|}
\left(    \sum_k |\hat\Delta_{k}(1/|l| )|^s  \right)^{1/s} 
\lesssim   \sum_{l\neq 0} \frac 1{|l|} \left(    \sum_k  n_k^{(\alpha+\beta) s} |\hat\mu(1/|l|)|^{n_k s} 
|1-\hat\mu(1/|l|)|^s  \right)^{1/s} \\
\lesssim  &  \sum_{l\neq 0} \frac {h(1/|l|)^{1-(\alpha+\beta)-(1-\alpha)/s)}}{|l|}   \label{eq:II5} 
<\infty.
\end{align*}


% V
For the last term, we have
\begin{align*}
n_k^{-\beta} |\hat\Delta'_{k}(1/|l| )| \le &
 n_k | \hat \mu^{n_k-1}(1/|l|)| \,  
\left| 1-\hat\mu(1/|l|)^{n_{k+1}-n_k} \right| \,|\hat\mu'(1/|l|) | \\
& +  (n_{k+1}-n_k)  \left| \hat\mu(1/|l|) \right|^{n_{k+1}-1}  \, |\hat\mu'(1/|l|)|\\
 \lesssim & n_k^{1+\alpha} | \hat \mu^{n_k-1}(1/|l|)| \, |\hat\mu'(1/|l|)| |1-\hat\mu(1/|l|)| \\
& + n_k^{\alpha} | \hat \mu^{n_{k+1}-1}(1/|l|)| \, |\hat\mu'(1/|l|)|.
 \end{align*}
Then
\[ \left( \sum_k  |\hat\Delta'_{k}(1/|l| )|^s\right)^{1/s}
 \lesssim  \frac {h'(1/|l|)}{h(1/l)^{(\alpha+\beta)+(1-\alpha)/s )}} , \]
 and
\[
\mbox{D} \lesssim        \sum_{l>0} \frac 1{l^2} \frac {h'(1/|l|)}{h(1/l)^{(\alpha+\beta)+(1-\alpha)/s }} <\infty
\]
for $0<\alpha+\beta+(1-\alpha)/s<1$. 
%~\hfill$\square$
  
  
  
 \bigskip
%%%%%%%%%%%%%%%%%%%%%%%


Now, let $I_k=[n_k,n_{k+1})$. 
\begin{align*}
\sum_k  n_k^{\beta s}  \|\{T_{\mu}^n f-T_{\mu}^{n_k}f:n\in I_k \} \|_{v(s)}^s     \le  & c \sum_k  n_k^{\beta s} \sum_{n\in I_k} |T_{\mu}^n f-T_{\mu}^{n_k}f|^s \\
\le  &c \sum_k n_k^{\beta s}  (n_{k+1}-n_k) \max_{n\in I_k} |T_{\mu}^n f-T_{\mu}^{n_k}f|^s .  
 \end{align*}
 Using Lemma \ref{lem:basic2} and arguments similar to the above, \[\sum_k  n_k^{\beta s}  \|\{T_{\mu}^n f-T_{\mu}^{n_k}f:n\in I_k \} \|_{v(s)}^s \]
is bounded in $L^1$ for $1<s(1-\alpha-\beta)$, and \[\sum_k n_k^{\beta s} \max_{n\in I_k} |T_{\mu}^n f-T_{\mu}^{n_k}f|^s \]
is bounded in $L^1$ for $1-\alpha<s(1-\alpha-\beta)$.

~\hfill$\square$


%%%%%%%%%%%%%%%%%%%%%%%%%%%%%%%%%%%%%%%%%%

\begin{thebibliography}{99}

\bibitem{BC} A. Bellow and A. P. Calder\'on, {\it A weak-type inequality for convolution products}, Harmonic analysis
and partial differential equations (Chicago, IL, 1996), 41--48, Chicago Lectures in Math., Univ.
Chicago Press, Chicago, IL, 1999.


\bibitem{BJR-conv} A. Bellow, R. Jones and J. Rosenblatt, {\it Almost everywhere convergence of convolution powers}, Ergodic
Theory Dynam. Syst. {\bf 14} (1994), 415-432.
  
\bibitem{Bl} S. Blunck, {\it Analyticity and Discrete Maximal Regularity on $L^p$--Spaces}, J. Func. Anal. {\bf 183} (2001), 211--230.

\bibitem{CCL} 
G. Cohen, C. Cuny, M. Lin, {\it Almost everywhere convergence of powers of some positive $L^p$ contractions}, J. Math. Anal. Appl. {\bf 420} (2014), 1129--1153.


\bibitem{Cuny-weak} C. Cuny, {\it On the Ritt property and weak type maximal inequalities for convolution powers on $l^1(\mathbb Z)$}. Studia Mathematica {\bf 235} (1), (2016), 47--85.

\bibitem{Dunn} N. Dungey, {\it Subordinated discrete semigroups of operators}, Trans.\ AMS 363 (2011), 1721--1741.

\bibitem{JR} R.L. Jones and K. Reinhold, {\it Oscillation and variation inequalities for convolution powers}, Ergod.\ Th.\  \& Dynam.\ Sys.\ (2001), {\bf 21}, 1809--1829.

\bibitem{LeM-H} C. LeMerdy, $H^{\infty}$ {\it Functional calculus and square function estimates for Ritt operators}. Rev.\ Mat.\ Iberoam.\  {\bf 30} (2014), no. 4, pp. 1149--1190. 
DOI 10.4171/RMI/811

\bibitem{LeMX-max} C. Le Merdy, and Q. Xu, {\it Maximal theorems and square functions for analytic operators on $L^p$-spaces}, J. Lond.\, Math.\, Soc.\, (2) {\bf 86} (2012), no. 2, 343--365.

\bibitem{LeMX-Vq}C. Le Merdy, and Q. Xu, {\it Strong q-variation inequalities for analytic semigroups}, Ann. Inst. Fourier
(Grenoble) {\bf 62} (2012), no. 6, 2069--2097 (2013).

\bibitem{LinDer} Y. Derriennic, M. Lin, {\it Fractional Poisson equations and ergodic theorems for fractional coboundaries}, Israel J. Math. 123 (2001) 93--130.

\bibitem{Losert1} V. Losert, {\it A remark on almost everywhere convergence of convolution powers}, Illinois J. Math. {\bf 43}
(1999), 465-479.

\bibitem{Losert2} V. Losert, {\it The strong sweeping out property for convolution powers}, Ergodic Theory Dynam. Systems
{\bf 21} (1) (2001), 115--119.

\bibitem{Lyub}  Y. Lyubich, {\it The single--point spectrum operators satisfying Ritt's resolvent condition}, Studia Math.\ {\bf 145} (2001), 135--142.

\bibitem{NZ} B. Nagy, J. Zemanek, {\it A resolvent condition implying power boundness}, Studia Math {\bf 134} 2 (1999), 143--151.

\bibitem{Nev} O. Nevanlinna, {\it Convergence of Iterations for Linear Equations}, Birkhauser, Basel, 1993.

\bibitem{Ritt} R.K. Ritt, {\it A condition that $\lim_{n\to\infty} n^{-1} T^n = 0$}, Proc. of the AMS {\bf 4} (1953), 898-899.

\bibitem{Chris} C. Wedrychowicz, {\it Almost everywhere convergence of convolution powers without finite second moment}, Ann.\, Inst.\, Fourier (Grenoble) {\bf 61} (2) (2011), 401--415.

\end{thebibliography}

\end{document}

%%%%%%%%%%%%%%%%%%%%%%
