
\documentclass[lettersize,journal]{IEEEtran}
\usepackage{amsmath,amsfonts}
\usepackage{algorithmic}
\usepackage{algorithm}
\usepackage{array}
\usepackage{amsmath}
\usepackage{amssymb}
\usepackage[caption=true,font=footnotesize]{subfig}
\usepackage{textcomp}
\usepackage{stfloats}
\usepackage{url}
\usepackage[hidelinks]{hyperref} % make the reference hyperlink
\usepackage{verbatim}
\usepackage{graphicx}
\usepackage{cite}
\usepackage{upgreek}
\usepackage{comment} 
\usepackage{multirow}
\usepackage{graphicx}
\usepackage[table,xcdraw]{xcolor}
 
\hyphenation{op-tical net-works semi-conduc-tor IEEE-Xplore}
\usepackage{bm} 
\begin{document}

%\title{A Sample Article Using IEEEtran.cls\\ for IEEE Journals and Transactions}
\title{A Model Predictive Capture Point Control Framework for Robust Humanoid Balancing via Ankle, Hip, and Stepping Strategies}
% \title{A Robust Humanoid Walking Control \\Framework based on Capture Point Control via \\ Ankle, Hip, and Stepping Strategies}
%\author{IEEE Publication Technology,~\IEEEmembership{Staff,~IEEE,}
\author{Myeong-Ju Kim$^{1}$, Daegyu Lim$^{1}$, Gyeongjae Park$^{1}$, {Kwanwoo Lee$^{1}$}, and Jaeheung Park$^{1,2}$,~\IEEEmembership{Member,~IEEE}
        % <-this % stops a space
%\thanks{This paper was produced by the IEEE Publication Technology Group. They are in Piscataway, NJ.}% <-this % stops a space
\thanks{This work was supported by the National Research Foundation of Korea (NRF) grant funded by the Korea government (MSIT) (No. 2021R1A2C3005914).}
\thanks{$^{1}$Myeong-Ju Kim, Daegyu Lim, Gyeongjae Park, Kwanwoo Lee and Jaeheung Park are with the Department of Intelligence and Information, Seoul National University, Republic of Korea. Contact : {\tt\small park73@snu.ac.kr}}%
\thanks{$^{2}$Jaeheung Park is also with the Advanced Institutes of Convergence Technology, Republic of Korea and with ASRI, RICS, Seoul National University, Republic of Korea.}
%\thanks{Manuscript received April 19, 2021; revised August 16, 2021.}
}

% The paper headers
\markboth{Journal of \LaTeX\ Class Files,~Vol.~14, No.~8, August~2021}%
{Shell \MakeLowercase{\textit{et al.}}: A Sample Article Using IEEEtran.cls for IEEE Journals}

\IEEEpubid{0000--0000/00\$00.00~\copyright~2021 IEEE}
% Remember, if you use this you must call \IEEEpubidadjcol in the second
% column for its text to clear the IEEEpubid mark.

\maketitle

\begin{abstract}
% The robust balancing capability of humanoid robots against disturbances has been considered as one of the crucial requirements for their practical mobility in real-world environments. In particular, many studies have been devoted to the efficient implementation of the three balance strategies, inspired by human balance strategies involving ankle, hip, and stepping strategies, to endow humanoid robots with human-level balancing capability. In this paper, a robust balance control framework for humanoid robots is proposed. Firstly, a novel Model Predictive Control (MPC) framework is proposed for Capture Point (CP) tracking control, enabling the integration of ankle, hip, and stepping strategies within a single framework. Additionally, a variable weighting method is introduced that adjusts the weighting parameters of the Centroidal Angular Momentum (CAM) damping control over the time horizon of MPC to improve the balancing performance. Secondly, a hierarchical structure of the MPC and a stepping controller was proposed, allowing for the step time optimization. The robust balancing performance of the proposed method is validated through extensive simulations and real robot experiments. Furthermore, a superior balancing performance is demonstrated, particularly in the presence of disturbances, compared to a state-of-the-art Quadratic Programming (QP)-based CP controller that employs the ankle, hip, and stepping strategies.

% The robust balancing capability of humanoids has been considered one of the crucial requirements for their mobility in real environments. In particular, many studies have been devoted to the efficient implementation of human-inspired ankle, hip, and stepping strategies, to endow humanoids with human-level balancing capability. 
The robust balancing capability of humanoids is essential for mobility in real environments. Many studies focus on implementing human-inspired ankle, hip, and stepping strategies to achieve human-level balance. In this paper, a robust balance control framework for humanoids is proposed. Firstly, a Model Predictive Control (MPC) framework is proposed for Capture Point (CP) tracking control, enabling the integration of ankle, hip, and stepping strategies within a single framework. Additionally, a variable weighting method is introduced that adjusts the weighting parameters of the Centroidal Angular Momentum damping control. Secondly, a hierarchical structure of the MPC and a stepping controller was proposed, allowing for the step time optimization. 
The robust balancing performance of the proposed method is validated through simulations and real robot experiments. Furthermore, a superior balancing performance is demonstrated compared to a state-of-the-art Quadratic Programming-based CP controller that employs the ankle, hip, and stepping strategies.
\end{abstract}

\begin{IEEEkeywords}
Humanoid walking control, capture point (CP) control, and model predictive control (MPC). 
%IEEEtran, journal, \LaTeX, paper, template, typesetting.
\end{IEEEkeywords}

\section{Introduction}\label{Section/Introduction}
% \IEEEPARstart{O}{ne} of the primary objectives of humanoids is to operate effectively in environments that are designed for human work activities. However, the presence of irregular terrain and the possibility of external collisions in the work environment can make it a challenging and unpredictable place for humanoids to navigate freely.
% As a result, it is imperative to develop robust balance control strategies to enable humanoid robots to overcome these disturbances and maintain stable locomotion. 
% % Figure environment removed  

%\IEEEPARstart{H}{umanoid} robots have been researched to achieve the human-like walking capability, especially in environments designed for human activities. However, their ability to walk with adaptability in human-centered environments remains a significant challenge due to various disturbances caused by uneven terrain and potential collisions with humans or obstacles. To successfully navigate such environments, humanoid robots must possess the ability to maintain balance in the face of disturbances. Therefore, the development of a robust balance control strategy is crucial to overcome disturbances and ensure stability during locomotion.
\IEEEPARstart{H}{umanoid} robots have been studied to achieve human-like walking capability in environments designed for human activities. However, their ability to adapt to disturbances caused by uneven terrain and potential collisions remains a challenge. To navigate these environments successfully, a robust balance control strategy that overcomes disturbances and ensures safe locomotion is essential.

% Three balance strategies
{In order to obtain robust balancing performance in humanoid robots}, researchers have been inspired by human balance strategies, and several studies are conducted to explore them \cite{nashner1985organization, winter1995human, maki1997role}. The imitation and analysis of human balance strategies, including ankle, hip, and stepping strategies, have been explored using a simple model \cite{barin1989evaluation, kuo1993human, park2004postural}. The lessons learned from these investigations have been applied to humanoid robot research, with each strategy being implemented in a distinct manner. Specifically, the ankle strategy has been implemented through the Zero Moment Point (ZMP) control and the hip strategy has been implemented using Centroidal Angular Momentum (CAM) control. Lastly, the stepping strategy has been realized through the stepping control, which adjusts footstep position or step time.

% Ankle stategy
\IEEEpubidadjcol In the early days of humanoid walking research, many studies were conducted to control ZMP, which is frequently referred to as an ankle strategy, by utilizing a simplified model. The Linear Inverted Pendulum Model (LIPM) \cite{kajita20013d, kajita2003biped} was introduced, which simplifies the dynamics of complex humanoid robots, and the linear relationship between the Center of Mass (CoM) and ZMP \cite{vukobratovic2004zero, popovic2005ground} is established. Many studies aimed to minimize the error between the reference ZMP and the actual ZMP, premised on the concept that if the ZMP is positioned within the support region, the robot will remain stable \cite{kajita2003preview, choi2006walking, kim2006experimental, kajita2010biped, joe2019robust}. Kajita et al. \cite{kajita2003preview} proposed a preview control method that considers a future reference ZMP trajectory, which compensates for ZMP errors by generating CoM trajectories. Choi et al. \cite{choi2006walking} calculates the desired CoM velocity to regulate ZMP and CoM errors, and utilizes the CoM Jacobian to generate the desired CoM velocity. Kim et al. \cite{kim2006experimental} proposed a state observer-based ZMP feedback controller that accounts for the joint elasticity of the robot. 
%Joe et al. \cite{joe2019robust} proposed a comprehensive ZMP control framework, which integrates the contact wrench control method in \cite{kajita2010biped} and the state feedback control method in \cite{kim2006experimental}. 
Kajita et al. \cite{kajita2010biped} calculates contact wrenches to track the reference ZMP and utilized an admittance control method to generate the calculated contact wrenches. Joe et al. \cite{joe2019robust} proposed a comprehensive ZMP control framework that combines the state feedback controller suggested in \cite{kim2006experimental} with the contact wrench controller proposed in \cite{kajita2010biped}.
While these studies have greatly contributed to the research on balancing in humanoid walking, the robot's ability to maintain balance relies only on ZMP control whose balancing capacity is limited by the size of the supporting polygon. Therefore, additional strategies are necessary to withstand strong disturbances.

% Hip stategy
Studies have been conducted to enhance the balancing capabilities of humanoid robots by exploring not only the ankle strategy utilizing ZMP control but also the hip strategy utilizing the robot's upper body. The hip strategy is commonly used in humanoids to control the CAM with the upper body. To account for the angular momentum that was disregarded in the LIPM, simple models such as Angular Momentum inducing inverted Pendulum Model (AMPM) or Linear Inverted Pendulum Plus Flywheel Model (LIPFM) have been proposed \cite{komura2005feedback, komura2005simulating, pratt2006capture}, and the dynamic relationship between the CoM and Centroidal Moment Pivot (CMP) has been defined \cite{pratt2006capture}. Based on this, many CAM control frameworks have been proposed to overcome external disturbances \cite{stephens2007humanoid, yi2016whole, wiedebach2016walking, schuller2021online, kim2022humanoid, ding2022dynamic}. 
Yi et al. \cite{yi2016whole} proposed a CAM control framework that generates a desired CAM through the hip joint and sequentially recovers the initial pose at a predetermined time when the CoM is perturbed by external disturbances. Schuller et al. \cite{schuller2021online} proposed a CAM control framework based on whole-body dynamics, in which the CAM tracking control and initial pose return control are operated by soft hierarchy-based Quadratic Programming (QP) optimization. In our previous CAM control approach \cite{kim2022humanoid}, we addressed the issue of degraded balancing performance caused by the conflict between initial pose control and CAM tracking control arising from the soft hierarchy in \cite{schuller2021online} by utilizing a control framework based on Hierarchical Quadratic Programming (HQP).
Ding et al. \cite{ding2022dynamic} proposed a CAM controller that plans arm trajectories to improve balancing performance using the Model Predictive Control (MPC) approach.
These methods have been shown to improve balancing performance. However, the amount of CAM generation in a robot is constrained by the joint limits of the robot and the self-collision avoidance, which in turn affects the robot's ability to maintain balance.

% 5. 미리 정해진 발자국과 스텝 시간을 기반으로 보행하는 기존의 standard 형태의 보행이 아닌 disturbance에 adaptive도록 발자국 위치와 스텝 시간을 조정하는 Stepping 제어가 휴머노이드 밸런싱에 적용되면서, 외란에 대한 밸런싱 성능이 크게 향상되었다. LIPM dyanmics를 기반으로 CP의 위치에 CoM의 위치가 안정적으로 수렴한다는 개념을 도입하여 많은 스테핑 알고리즘들이 제안되었으며, 또한 MPC와 같은 최적화 알고리즘의 발전과 함께, 외란에 대해 발 위치를 조정하는 프레임워크들도 많이 제안되었다. 이러한 스테핑 연구의 끊임없는 발전은 휴머노이드 밸런스 제어에 크게 기여하였다. 
The stepping control, which adjusts the footstep position or step time in an adaptive manner to disturbances, has greatly improved the balancing performance of humanoid robots compared to the conventional walking control based on pre-determined footsteps and step time\cite{herdt2010online, khadiv2016step, jeong2017biped, joe2018balance, jeong2019robust1, khadiv2020walking, kim2023foot}. Under the notion that the position of the CoM converges to the Capture Point (CP) using LIPM dynamics, several stepping algorithms have been proposed \cite{khadiv2016step, jeong2017biped, jeong2019robust1,khadiv2020walking}. 
Khadiv et al. \cite{khadiv2016step, khadiv2020walking} proposed a framework that employs QP optimization based on the CP end-of-step dynamics to calculate the footstep position and step time. This approach tracks the pre-planned CP offset during walking by adjusting the footstep position and step time. Jeong et al. \cite{jeong2017biped, jeong2019robust1} proposed a stepping control method that also utilizes CP end-of-step dynamics, where the ankle torque is pre-calculated in response to external disturbances, and the step time and footstep position are optimized accordingly.
Furthermore, many control frameworks have been proposed for adjusting footstep position against disturbances by utilizing MPC optimization \cite{herdt2010online, joe2018balance, kim2023foot}. Herdt et al. \cite{herdt2010online} proposed an MPC framework that extends the framework proposed by Wieber et al. \cite{wieber2006trajectory}, by automatically adjusting the footstep position to control the CoM velocity and the ZMP error. Joe et al. \cite {joe2018balance} proposed a framework for adjusting footstep position when the desired ZMP generated to reduce ZMP error is limited by ZMP constraints. 
The continuous evolution of stepping algorithms has significantly contributed to humanoid balance control.  

% 6. 각 전략이 발전함에 따라 3가지 전략을 통합하는 전략들이 개발되었다. 
% 6-1 MPC 기반으로 3가지 전략을 통합한 제어 전략들 소개 - 하지만 이 방법들은 스텝시간 조정 안한다.
% 6-2 스텝 시간 조정까지 하면서 3가지 전략 통합 - 원스텝까지밖에 고려못한다.
% 7. 본 논문에서는, 계층적 구조에 기반한 강건한 균형 제어 프레임워크가 제안된다. 먼저 CP 추종 제어를 위해 세가지 전략을 동작되는 MPC 프레임워크가 먼저 제안된다. 다음으로, 제안하는 MPC의 변수들을 기반한 스테핑 시간 최적화 접근법이 제안된다.

With the advancement of each balance control strategy, many studies have been proposed with the aim of integrating these strategies to enhance balancing capability \cite{shafiee2017robust, ding2021versatile, ding2022robust, romualdi2022online}.
Shafiee-Ashtiani et al. \cite{shafiee2017robust} proposed a linear MPC framework that builds upon the MPC-based stepping controller suggested by \cite{herdt2010online} to combine the three balance strategies (ankle, hip, and stepping strategies).
In this approach, the ZMP and CoM velocity control problem proposed in \cite{herdt2010online} was changed to a ZMP and CP control problem. Furthermore, the control inputs were expanded to include the change of the centroidal moment, resulting in improved control performance.
Ding et al. \cite{ding2021versatile, ding2022robust} proposed a nonlinear MPC framework using a nonlinear Inverted Pendulum Flywheel (IPF) model. By considering the nonlinear relationship between CoM and ZMP in IPF as a quadratic constraint, the method adjusts the ZMP and body angles, as well as the footstep position to compensate for large disturbances.
Romualdi et al. \cite{romualdi2022online} proposed a nonlinear MPC framework for disturbance rejection based on centroidal dynamics, which involves the control of contact wrench, CAM, and footstep position. 
These studies developed a robust walking framework to cope with disturbances by integrating the three balance strategies through MPC. However, they did not consider step time adjustment algorithms, which play a crucial role in withstanding disturbances.

Studies that included the three balance strategies and a step time adjustment were also proposed \cite{aftab2012ankle, nazemi2017reactive, jeong2019robust}. 
Aftab et al. proposed an MPC-based balance control framework that integrates the three balance strategies and step time adjustment to address disturbances \cite{aftab2012ankle}. To avoid a biased output of the smallest step time during step time optimization, a swing foot acceleration cost was introduced to adjust the step time appropriately. However, due to the nonlinearity arising from step time optimization, the algorithm could not be used in real-time and was only applicable to standing situations, not during walking control.
Nazemi et al. proposed a reactive walking pattern generator based on hierarchical structure \cite{nazemi2017reactive}. In this approach, the stepping controller proposed by \cite{khadiv2016step, khadiv2020walking} is first used to determine the footstep position and step time for the stepping strategy. Subsequently, the CP trajectory and body angle were adjusted using MPC to track the reference ZMP trajectory determined by the pre-planned footstep position and step time.
Jeong et al. \cite{jeong2019robust} proposed a QP-based optimization method to achieve the three balance strategies with step time adjustment for controlling the CP end-of-step. In this method, to avoid the variable coupling during the linearization for QP, the ZMP control input for ankle strategy was pre-computed using the instantaneous CP control method \cite{englsberger2011bipedal}, rather than through optimization. Based on this pre-computed ZMP control input, the CAM control and stepping control are performed through QP optimization.  
In \cite{nazemi2017reactive,jeong2019robust}, the robust balancing performance was validated against disturbances through simulations or experiments. However, the three balance strategies cannot be computed from a single framework in \cite{nazemi2017reactive,jeong2019robust}. Additionally, these methods are unable to consider future states and constraints beyond the current walking step, which can affect the balancing performance. This limitation arises from the control framework relying on CP end-of-step dynamics, which restricts the prediction horizon to the current step duration. 

% evolved paper version
% In this paper, a robust balance control framework is proposed to overcome disturbances through the integration of the three balance strategies and step time optimization (as shown in Fig. \ref{figure/Thumbnail}).
In this paper, a robust balance control framework is proposed to overcome disturbances through the integration of the three balance strategies and step time optimization. In contrast to \cite{nazemi2017reactive,jeong2019robust}, our research does not restrict the horizon time of MPC based on the current step duration. The proposed framework expands upon our prior works \cite{kim2022humanoid, kim2023foot} that were developed for the same goal, i.e., CP tracking control. We combined the hip strategy in \cite{kim2022humanoid} and the ankle and stepping strategies in \cite{kim2023foot} into a single MPC framework. 
With novel ideas, we addressed several technical problems that arose during the integration process, thereby enhancing the coherence between control hierarchies in the proposed control framework.
The primary contributions of this paper can be summarized as follows:

% 1) An MPC framework that integrates the three balance strategies for CP tracking control is proposed. The proposed MPC framework is an extension of the MPC framework proposed in our prior work \cite{kim2023foot}. Unlike \cite{kim2023foot}, the proposed framework includes MPC optimization to handle not only ZMP control and stepping control but also CAM control. Furthermore, this framework enables more effective CAM control through MPC, unlike the heuristic CAM calculation method based on CMP decomposition proposed in our previous study \cite{kim2022humanoid}. 
% 2) A novel variable weighting method for CAM control is proposed. This method adjusts the weighting parameters of CAM damping control during the time horizon of the MPC to enhance the CP control performance and demonstrates better balancing performance compared to the conventional constant weighting method.
% 3) A hierarchical structure of the proposed MPC and stepping controller enables optimization of step time. Furthermore, an approach for determining stepping control parameters based on the MPC is suggested, which achieves better control performance than the previous parameter selection approach \cite{khadiv2016step, khadiv2020walking, kim2023foot}.
% 4) {\color{blue} A novel HQP-based whole-body IK controller is proposed, focusing on nominal pose tracking and CAM control performance, while ensuring a smooth transition into pose recovery after CAM control.}
% 5) The proposed method is validated through extensive simulations and real robot experiments using our humanoid robot, TOCABI. Additionally, when compared to a state-of-the-art QP-based CP controller that incorporates the three balance strategies \cite{jeong2019robust}, the proposed method demonstrates superior balancing performance in the presence of disturbances. 
\begin{enumerate}
    \item An MPC framework that integrates three balance strategies for CP tracking control is proposed as an extension of our prior work \cite{kim2023foot}. Unlike \cite{kim2023foot}, this framework optimizes not only ZMP and stepping control but also CAM control. Additionally, it enhances CAM control through MPC, rather than using the heuristic CAM calculation method proposed in our previous study \cite{kim2022humanoid}. 
    \item A novel variable weighting method for CAM control is proposed. This method adjusts the weighting parameters of CAM damping control during the time horizon of the MPC to enhance the CP control performance.
    \item A hierarchical structure of the proposed MPC and stepping controller enables optimization of step time. Furthermore, an approach for determining stepping control parameters based on the MPC is suggested, which achieves better control performance than the previous parameter selection approach \cite{khadiv2016step, khadiv2020walking, kim2023foot}.
    \item { A novel HQP-based whole-body IK controller is proposed, focusing on nominal pose tracking and CAM control, while ensuring a smooth transition into initial pose recovery.}
    \item 
    % The proposed method is validated through extensive simulations and real robot experiments. Additionally, when compared to a state-of-the-art QP-based CP controller that incorporates the three balance strategies \cite{jeong2019robust}, the proposed method demonstrates superior balancing performance in the presence of disturbances. 
    The proposed method is validated through extensive simulations and real robot experiments, showing superior balancing performance compared to a state-of-the-art QP-based CP controller \cite{jeong2019robust} that incorporates the three balance strategies.
\end{enumerate}

% This paper is organized as follows. In Section \ref{Section/Fundamentals}, we briefly introduce the LIPFM and CP dynamics. In Section \ref{Section/Overall Framework}, an overview of the walking control framework is introduced.  
% Section \ref{Section/MPC Framework} details the proposed MPC framework, while Section \ref{Section/Stepping Controller} describes the MPC-based stepping controller. \textcolor{blue}{Section \ref{Section/WBIK} covers the HQP-based WBIK.}
% In Section \ref{Section/Results}, the results of simulations and real robot experiments conducted to validate the proposed algorithm are presented. \textcolor{red}{In Section \ref{Section/Discussion}, discussions on the proposed framework and implementation details are provided.} Finally, in Section \ref{Section/Conclusion}, the conclusion of this paper is presented.
% Subsequently, we shall refer to the proposed MPC framework as CP--MPC for the remainder of this paper.
This paper is organized as follows. Section \ref{Section/Fundamentals} introduces the LIPFM and CP dynamics, followed by an overview of the walking control framework in Section \ref{Section/Overall Framework}. The proposed MPC framework is detailed in Section \ref{Section/MPC Framework}, and the MPC-based stepping controller is described in Section \ref{Section/Stepping Controller}. {Section \ref{Section/WBIK} presents the HQP-based WBIK.} Simulation and experimental results validating the proposed algorithm are provided in Section \ref{Section/Results}, while {Section \ref{Section/Discussion} discusses the framework and implementation details.} Finally, the conclusion is presented in Section \ref{Section/Conclusion}. Henceforth, the proposed MPC framework is referred to as CP--MPC.

\section{Fundamentals}
\label{Section/Fundamentals}
\subsection{Linear Inverted Pendulum Plus Flywheel Model}
% \label{Section/Fundamentals/LIPFM}
% % Figure environment removed  

The LIPFM is a linear abstract model that was developed to address the CAM of humanoid robots \cite{pratt2006capture}.
The LIPFM assumes that the total mass of the robot is concentrated at the CoM, and the height of the CoM from the ground is considered to be constant. Unlike the LIPM \cite{kajita20013d, kajita2003biped}, the LIPFM is capable of handling reaction torque by means of the rotational motion of a flywheel located at the CoM. 
The dynamic equation of the LIPFM is expressed in terms of the relationship between the CoM and the CMP, and is defined as follows,
\begin{equation}
{\Ddot{c}_x}={\omega^2}(c_x-p_{x}),
\label{eq/COM-CMP}
\end{equation}
\begin{equation}
\label{eq/CMP-ZMP-Moment}
{p_{x}} = {z_{x}} + \frac{{\tau_y}}{mg},
\end{equation}
where $c_x$, $p_{x}$, and $z_{x}$ denote the positions of CoM, CMP, and ZMP in the x-direction, respectively. $\omega=\sqrt{g/c_z}$ is the natural frequency, ${g}$ is the gravitational acceleration, and $c_z$ is the height of the CoM from the ground. $\tau_y$ represents the reaction torque of the flywheel in the y-direction. 
%It is noteworthy from (\ref{eq/COM-CMP})-(\ref{eq/CMP-ZMP-Moment}) that, unlike LIPM, the acceleration of CoM is generated in proportion to the difference between CoM and CMP, where CMP is expressed as a linear combination of ZMP and reaction torque. This means that the greater acceleration of CoM can be handled by modulating CMP compared to the ZMP control using LIPM, which is confined inside a finite-size foot. 
A detailed derivation of LIPFM dynamics is presented in \cite{pratt2006capture}.
% Reaction torque 없으면 ZMP=CMP 같다. -> 없어도 될듯.

The dynamics of LIPFM in the y-direction can be derived in the same way as the x-direction, and the dynamics in each direction can be dealt with independently. 
% This decoupling property of the system simplifies the analysis and control of the system and makes it easier to handle each direction separately. %Therefore, in this study, only the dynamics in the x-direction are derived to simplify the equations without loss of generality. 
Therefore, in this study, the dynamics were derived only in the x-direction for conciseness of the paper, except for Section \ref{Section/Stepping Controller}, where the step time variable couples both x- and y-direction variables.%both the x- and y-direction variables need to be considered simultaneously because a step time variable is involved that couples both directions.

\subsection{Capture Point dynamics based on LIPFM}
\label{Section/Fundamentals/CP dynamics}
This section provides a brief introduction to the CP dynamics based on LIPFM.
The concept of CP, introduced in \cite{pratt2006capture, hof2008extrapolated}, has been extensively utilized in various studies as a control variable for stabilizing the CoM of a robot and as an indicator of its balance state \cite{englsberger2011bipedal, englsberger2016combining, jeong2017biped, joe2018balance, morisawa2012balance, wiedebach2016walking, griffin2017walking, joe2019robust}.
The dynamics of the CP is expressed as a linear combination of horizontal position and velocity of the CoM as below,
\begin{equation}
\xi_x = c_x + \frac{{\Dot{c}_x}}{\omega}, 
\label{eq/CP-COM}
\end{equation}
where $\xi_x$ represents the CP in the x-direction. 
The LIPFM also allows the CP to be expressed as a dynamic relationship with the CMP. By combining (\ref{eq/COM-CMP}) with the time derivative of (\ref{eq/CP-COM}), the dynamics of the CP--CMP can be derived as   
\begin{equation} 
\Dot{\xi}_x = \omega(\xi_x-p_{x})=\omega(\xi_x-({z_{x}} + \frac{{\tau_y}}{mg})).
\label{eq/CP-CMP}
\end{equation}
Assuming the constant CMP $p_x$ within a step duration $T$, and considering the time elapsed after the start of the swing phase as $t$, the behavior of the CP can be defined as follows,
\begin{equation}
{\xi}_{x,T} = (\xi_{x}-p_x)e^{\omega (T-t)} + p_x.
\label{eq/CP eos}
\end{equation}
Equation (\ref{eq/CP eos}), known as the CP end-of-step dynamics \cite{englsberger2011bipedal, jeong2019robust, jeong2019robust1}, provides a means to predict the CP at the end of a current step by using the current CP, current time, and CMP as inputs.

In Section \ref{Section/MPC Framework}, the CP--CMP dynamics (\ref{eq/CP-CMP}) is employed as a prediction model in the proposed CP--MPC. Additionally, the CP end-of-step dynamics (\ref{eq/CP eos}) is utilized for the stepping controller presented in Section \ref{Section/Stepping Controller}. 


% \section{Overall Walking Control Structure}
% \label{Section/Overall Framework}
% % Figure environment removed 
\section{Overall Walking Control Structure}
\label{Section/Overall Framework}
% Figure environment removed 

This section provides an overview of the proposed walking control framework. The schematic diagram of the overall walking control framework is illustrated in Fig. \ref{figure/overallframe}. First, the footstep planner determines the positions of the footsteps $\mathbf{F}^{ref}=[\mathbf{F}_x^{ref}\,\mathbf{F}_y^{ref}]$ based on the target velocity command. Using the footstep positions as input, the ZMP trajectory generator creates reference ZMP trajectories $\mathbf{Z}^{ref}=[\mathbf{Z}_x^{ref}\,\mathbf{Z}_y^{ref}]\in\mathbb{R}^{N\times2}$. The walking pattern generator \cite{herdt2010online} then takes these reference ZMP trajectories as input and produces reference CP trajectories $\mathbf{\Xi}^{ref}=[\mathbf{\Xi}^{ref}_x\,\mathbf{\Xi}^{ref}_y]\in\mathbb{R}^{N\times2}$. {Note that our walking pattern generator does not include the footstep adjustment algorithm from \cite{herdt2010online}.}

The main goal of CP--MPC is to track the reference CP trajectories.  
The CP--MPC receives the reference CP trajectories as its control target and generates three optimized MPC variables, including ZMP control inputs $\mathbf{Z}=[\mathbf{Z}_x\,\mathbf{Z}_y]\in\mathbb{R}^{N\times2}$, centroidal moment control inputs $\bm{\uptau}=[\bm{\uptau}_y\,-\bm{\uptau}_x]\in\mathbb{R}^{N\times2}$, and footstep adjustments $\Delta \mathbf{F}=[\Delta\mathbf{F}_{x}\,\Delta\mathbf{F}_{y}]\in\mathbb{R}^{m\times2}$ from $\mathbf{F}^{ref}$ to track these trajectories. %Note that
$N$ represents the prediction horizon for the CP--MPC, while $m$ denotes the number of pre-designed next footsteps within the prediction horizon $N$. 

%In order to track the desired ZMP, $\mathbf{z}^{des}=[z_{x}^{des}\,z_{y}^{des}]\in\mathbb{R}^{2}$, which is the first element of the ZMP control inputs, $\mathbf{Z}$, the ZMP controller \cite{kajita2010biped} calculates desired contact wrenches to achieve the desired ZMP. 
The ZMP controller \cite{kajita2010biped} computes desired contact wrenches to achieve the desired ZMP $\mathbf{z}^{des}=[z_{x}^{des}\,z_{y}^{des}]\in\mathbb{R}^{2}$, which is the first element of the ZMP control inputs $\mathbf{Z}$. 
% Subsequently, the admittance controller compares the measured contact wrench $\mathbf{w}_{L,R}\in\mathbb{R}^{6}$ with the desired contact wrench to determine the changes of the position and orientation of the support foot with respect to the pelvis $\Delta \mathbf{e}_{L,R}\in\mathbb{R}^{6}$. 
{Subsequently, the admittance controller compares the measured contact wrench $\mathbf{w}_{L,R}\in\mathbb{R}^{6}$ with the desired contact wrench to determine the changes in the position and orientation of the support foot with respect to the pelvis $\Delta \mathbf{e}_{L,R}\in\mathbb{R}^{6}$, where both wrenches reflect only the z-direction position and roll/pitch  orientation components.}
The subscript $L$ and $R$ signify the left and right, respectively.

%%%%%%%%%%%%%%%%%%%%%%%%%%%%%%%%%%%%%%%%%%%%
%%%%% MJ Original Writings (212 - 218) %%%%%
% The stepping controller calculates the next footstep position $\mathbf{f}=[{f}_{x}\,{f}_{y}]\in\mathbb{R}^{2}$ and the step time $T$ to achieve the footstep adjustment $\Delta \mathbf{f}_1=[\Delta {f}_{x,1}\, \Delta{f}_{y,1}]\in\mathbb{R}^{2}$ planned by CP--MPC. 
% Specifically, the controller primarily adjusts the step time $T$ based on CP end-of-step dynamics (\ref{eq/CP eos}) to ensure that the $\mathbf{f}$ achieves $\Delta \mathbf{f}_1$ as closely as possible.
% Here, $\Delta\mathbf{f}_1$ refers to the first element of $\Delta \mathbf{F}$, representing the adjustment of the next footstep. %The detailed explanation is described in Section \ref{Section/Stepping Controller}. 
% After determining $\mathbf{f}$ and $T$, the foot trajectory generator creates reference foot trajectories $\mathbf{e}^{ref}_{L,R}\in\mathbb{R}^{6}$. Consequently, the lower body inverse kinematics is solved for the desired joint angles of legs $\mathbf{q}^{des}_{lb}$ required to achieve the reference CoM trajectories $\mathbf{c}^{ref}\in\mathbb{R}^{3}$, reference foot trajectories $\mathbf{e}^{ref}_{L,R}\in\mathbb{R}^{6}$, and displacement of the support foot pose $\Delta \mathbf{e}_{L,R}\in\mathbb{R}^{6}$. In addition, when a step change occurs, the footstep planner generates footsteps based on the target velocity command relative to the supporting foot.

% The desired centroidal moment $\bm{\tau}^{des}=[\tau_{y}^{des}\,-\tau_{x}^{des}]\in\mathbb{R}^{2}$, which is the first element of the centroidal moment control inputs $\bm{\uptau}$ is controlled by the HQP-CAM controller proposed in our previous study \cite{kim2022humanoid}. The desired centroidal moment $\bm{\tau}^{des}$ is integrated to yield the desired CAM, and the HQP-CAM controller computes the desired upper body joint angles $\mathbf{q}^{des}_{ub}$ to achieve the desired CAM. \textcolor{red}{In this study, only the CAM in the x- and y-directions is addressed based on the CP--CMP dynamics (\ref{eq/CP-CMP}), while the CAM in the z-direction is not considered.} 

% In the final stage, the joint PD controller is employed to calculate the joint torques required to track the upper and lower body motions. These computed joint torques $\bm{\Gamma}_{pd}$ are then commanded to the joints in conjunction with the gravity compensation torque $\bm{\Gamma}_{g}$.
%%%%%%%%%%%%%%%%%%%%%%%%%%%%%%%%%%%%%%
%%%%% KW modifications (220-228) %%%%%
The stepping controller calculates the next footstep position $\mathbf{f}=[{f}_{x}\,{f}_{y}]\in\mathbb{R}^{2}$ and the step time $T$ to achieve the footstep adjustment $\Delta \mathbf{f}_1=[\Delta {f}_{x,1}\, \Delta{f}_{y,1}]\in\mathbb{R}^{2}$ planned by CP--MPC. 
Specifically, the controller primarily adjusts the step time $T$ based on CP end-of-step dynamics (\ref{eq/CP eos}) to ensure that the $\mathbf{f}$ achieves $\Delta \mathbf{f}_1$ as closely as possible.
Here, $\Delta\mathbf{f}_1$ refers to the first element of $\Delta \mathbf{F}$, representing the adjustment of the next footstep. 
After determining $\mathbf{f}$ and $T$, the foot trajectory generator creates reference foot trajectories $\mathbf{e}^{ref}_{L,R}\in\mathbb{R}^{6}$.  
In addition, when a step change occurs, the footstep planner generates footsteps based on the target velocity command relative to the supporting foot.

{
% The desired centroidal moment in the horizontal plane, $\bm{\tau}^{des}_{xy}=[\tau_{y}^{des}\,-\tau_{x}^{des}]\in\mathbb{R}^{2}$, which is the first element of the centroidal moment control inputs $\bm{\uptau}$, is integrated to yield the desired CAM. The desired CAM~{$\bm{h}^{des}\in\mathbb{R}^3$} in the x- and y-directions are addressed based on the CP--CMP dynamics, while the CAM in the z-direction is controlled to maintain zero, preventing yaw slip over the support foot.
The desired centroidal moment $\bm{\tau}^{des}=[\tau_{y}^{des},-\tau_{x}^{des}]\in\mathbb{R}^{2}$, which is the first element of the centroidal moment control input $\bm{\uptau}$, is integrated to obtain the desired CAM. In this study, only the CAM in the x- and y-directions is computed from the CP--MPC based on CP--CMP dynamics (\ref{eq/CP-CMP}), while the CAM in the z-direction is controlled to remain zero to compensate for momentum induced by the swing foot motion and to prevent yaw slip at the support foot.

% In the final stage, the HQP-based whole-body IK is solved to determine the desired joint angles $\mathbf{q}^{des}$ and joint velocities $\mathbf{\dot{q}}^{des}$ required to achieve the reference CoM trajectories $\mathbf{c}^{ref}\in\mathbb{R}^{3}$, the reference foot trajectories $\mathbf{e}^{ref}_{L,R}\in\mathbb{R}^{6}$, the displacement of the support foot pose $\Delta \mathbf{e}_{L,R}\in\mathbb{R}^{6}$, and the desired CAM $\mathbf{h}^{des}\in\mathbb{R}^3$.
{In the final stage, the HQP-based whole-body IK receives feedback from the FK results, represented as $\mathbf{X}_{FK}=[\mathbf{c}~\mathbf{e}~\mathbf{x}_{pelv}~\mathbf{x}_{chest}~\mathbf{x}_{hand}]^T$, and determines the desired joint angles $\mathbf{q}^{des}$ and joint velocities $\mathbf{\dot{q}}^{des}$ required to achieve the reference CoM trajectory $\mathbf{c}^{ref}\in\mathbb{R}^{3}$, reference foot trajectories $\mathbf{e}^{ref}_{L,R}\in\mathbb{R}^{6}$, support foot pose displacement $\Delta \mathbf{e}_{L,R}\in\mathbb{R}^{6}$, and desired CAM $\mathbf{h}^{des}\in\mathbb{R}^3$.
% Then, a joint PD controller is employed to calculate the joint torques required to track the desired joint trajectory. These joint torques $\bm{\Gamma}_{pd}$ are then applied to the joints in conjunction with the gravity compensation torque $\bm{\Gamma}_{g}$.
Then, a joint PD controller computes the torques $\bm{\Gamma}_{pd}$ to track the desired joint trajectory, applied alongside the gravity compensation torque $\bm{\Gamma}_{g}$ computed using the contact-consistent control framework \cite{park2006contact}.}}

\section{CP--MPC: Model Predictive Control Framework for Capture Point Tracking}
\label{Section/MPC Framework}
% In this section, we extend the framework of our previous study \cite{kim2023foot} to propose CP--MPC, an MPC framework that incorporates the three balance strategies: ankle, hip, and stepping strategies.
In this section, we extend our previous framework \cite{kim2023foot} to propose CP--MPC, an MPC framework integrating ankle, hip, and stepping strategies for balance control.
CP--MPC leverages the ZMP, footstep position, and centroidal moment to enhance CP control under strong external disturbances. The following subsections provide a detailed description of the proposed framework.
% The proposed CP--MPC utilizes not only the ZMP and footstep position but also the centroidal moment to enhance the CP control performance against strong external disturbances. Detailed explanations of the proposed MPC framework are provided in the following subsections.

\subsection{Prediction of Future Trajectory}
\label{Section/MPC Framework/Prediction model} 
In order to formulate the CP--MPC, the CP--CMP dynamics described in (\ref{eq/CP-CMP}) is utilized as a prediction model. Equation (\ref{eq/CP-CMP}) can be discretized with the piecewise ZMP, centroidal moment, and sampling time $T_s$ as below,
\begin{equation} 
\label{eq/discrete CP-CMP}
\xi_{x,k+1} = A\xi_{x,k} + \mathbf{B}\left[z_{x,k}\quad\tau_{y,k}\right]^T
\end{equation}
where $A = e^{\omega T_s}, \mathbf{B} = \left[1-e^{\omega T_s}\quad \frac{1-e^{\omega T_s}}{mg}\right]$.
The CMP $p_{x}$ can be decomposed into ZMP $z_{x}$ and centroidal moment $\tau_{y}$ as shown in (\ref{eq/CP-CMP}), and two variables are independently treated as control inputs of the CP--MPC.
By recursive application of (\ref{eq/discrete CP-CMP}), the predicted future trajectories of CP over the time horizon $N$ that emanate from the current CP $\xi_{x,k}$ can be expressed as below,
\begin{equation}
\label{eq/reculsive dynamics}
\mathbf{\Xi}_x=\bm{\Phi}_\xi \xi_{x,k} + \bm{\Phi}_p\mathbf{P}_x ,
\end{equation}
\begin{equation}
\bm{\Phi}_\xi=
\begin{bmatrix}
A \\ \vdots \\ A^N 
\end{bmatrix}
,\quad
\bm{\Phi}_p=
\begin{bmatrix}
A^{0}B & \cdots & 0 \\ \vdots & \ddots & \vdots \\ A^{N-1}B & \cdots & A^{0}B 
\end{bmatrix}
,
\end{equation}
\begin{equation} 
\mathbf{\Xi}_x = 
\begin{bmatrix}
\xi_{x,k+1} \\ \vdots \\ \xi_{x,k+N} 
\end{bmatrix}
,\quad
\mathbf{P}_x = 
\begin{bmatrix}
z_{x,k} \\ \tau_{y,k} \\ \vdots \\ z_{x,k+N-1} \\ \tau_{y,k+N-1}
\end{bmatrix}\label{eq/reculsive dynamics2}
.
\end{equation}
$\bm{\Xi}_x \in \mathbb{R}^{N}$ represents the predicted future CP trajectory in the x-direction and $\mathbf{P}_x \in \mathbb{R}^{2N}$ denotes the future inputs of the ZMP and centroidal moment. The matrix $\bm{\Phi}_{\xi}\in \mathbb{R}^{N}$ defines the dynamic relationship between the current CP $\xi_{x,k}$ and future CP trajectory $\bm{\Xi}_x$. The matrix $\bm{\Phi}_{p}\in \mathbb{R}^{N\times 2N}$ defines the relationship between future inputs $\mathbf{P}_{x}$ and future CP trajectory $\bm{\Xi}_x$. 
% In $\mathbf{P}_{x}$, to handle each control input independently, we represent the series of ZMP vectors as $\mathbf{Z}_x=[z_{x,k}\,z_{x,k+1}\cdots z_{x,k+N-1}]^T\in \mathbb{R}^{N}$, and the series of centroidal moment vectors is represented as $\bm{\uptau}_y=[\tau_{y,k}\,\tau_{y,k+1}\cdots\tau_{y,k+N-1}]^T\in \mathbb{R}^{N}$.
{In $\mathbf{P}_{x}$, the elements of ZMP $z_x$ and centroidal moment $\tau_y$ are sequentially incorporated. For intuitive representation, we separately define the series of ZMP vectors as $\mathbf{Z}_x=[z_{x,k}\,z_{x,k+1}\cdots z_{x,k+N-1}]^T\in \mathbb{R}^{N}$ and the series of centroidal moment vectors as $\bm{\uptau}_y=[\tau_{y,k}\,\tau_{y,k+1}\cdots\tau_{y,k+N-1}]^T\in \mathbb{R}^{N}$.}   
 
\subsection{Problem Setup for MPC Optimization}
\label{Section/MPC Framework/Cost function}
This section presents the problem formulation for MPC optimization. 
The cost function and constraints of CP--MPC are formulated {using (\ref{eq/reculsive dynamics})-(\ref{eq/reculsive dynamics2})} and are as follows,
\begin{align}
\label{eq/CP-MPC cost function}
& \underset{\mathbf{P}_{x},\Delta \mathbf{F}_x}{\text{min}} & & 
\|\mathbf{\Xi}_{x}-\mathbf{\Xi}^{ref}_{x}\|^2_{\mathbf{w}_{\xi}} + 
\|\boldsymbol{\uptau}_{y} + \mathbf{K}_d \mathbf{h}_y\|^2_{\mathbf{w}_{\tau}}  \\
& & +&\|\Delta\mathbf{F}_{x}\|^2_{\mathbf{w}_{F}} + \|\Delta\mathbf{P}_{x}\|^2_{\mathbf{w}_{p}} \nonumber\\
& \text{s. t.} & & \underline{\mathbf{Z}}_x\leq \begin{bmatrix}
\mathbf{I}_N & -\mathbf{S}
\end{bmatrix}
\begin{bmatrix} 
\mathbf{Z}_x\\ \Delta\mathbf{F}_x \label{eq/CP-MPC constraint}
\end{bmatrix}
 \leq \overline{\mathbf{Z}}_x\\
& & &\underline{\boldsymbol{\uptau}}_y\leq \boldsymbol{\uptau}_y \leq \overline{\boldsymbol{\uptau}}_y \nonumber\\
& & & \Delta\underline{\mathbf{F}}_{x}\leq \Delta\mathbf{F}_x \leq \Delta\overline{\mathbf{F}}_{x} \nonumber \\
& & & {\xi^{ref}_{x,k+N}=\xi_{x,k+N}.} \nonumber 
\end{align}
% with 
% \begin{equation}
%     $\mathbf{A}_p=
% \begin{bmatrix}
% \mathbf{I}_N & -\mathbf{S}
% \end{bmatrix}$.
% ,\quad \mathbf{S}=
% \begin{bmatrix}
% {s}_{1,1} & {s}_{1,2} & \cdots & {s}_{1,m} \\
% \vdots & \vdots &\ddots & \vdots \\
% {s}_{N,1} & {s}_{N,2} & \cdots & {s}_{N,m} \\
% \end{bmatrix}, \nonumber
% \end{equation}
% \begin{equation} \resizebox{\columnwidth}{!}
% { s_{k,m} = \left \{
%     \begin{array}{ll} 
%       \text{\textcolor{red}{1; If time step $k$ is included in the $m$-th footstep}}\\
%       \text{\textcolor{red}{0; If time step $k$ is not included in the $m$-th footstep}} 
%     \end{array}
%   \right.
% }\nonumber
% \end{equation}

The cost function in (\ref{eq/CP-MPC cost function}) is composed of four cost terms in total. The first cost term functions as a reference CP trajectory tracking control and is weighted by a positive diagonal weighting matrix $\mathbf{w}_{\xi}\in\mathbb{R}^{N\times N}$. The second cost term plans the {centroidal moment $\boldsymbol{\uptau}_y$ that drives the CAM $\mathbf{h}_y$} to zero when the disturbance is small. Here, $\mathbf{K}_d\in\mathbb{R}^{N\times N}$ is a damping matrix which is positive and diagonal and $\mathbf{h}_y\in\mathbb{R}^{N}$ is the CAM in y-direction which is an integration of {centroidal moment} $\boldsymbol{\uptau}_y$. This term is weighted by a positive diagonal matrix, $\mathbf{w}_{\tau}\in\mathbb{R}^{N\times N}$. The diagonal terms of $\mathbf{w}_{\tau}$ are adjusted based on the magnitude of the ZMP control inputs to improve the CP control performance. This adjustment algorithm is explained in Section \ref{Section/Variable Weighting}. The third cost term regulates the additional footstep adjustments from the pre-planned footsteps. This term is weighted by a positive diagonal matrix $\mathbf{w}_F\in\mathbb{R}^{m\times m}$. The last cost term regulates the change of the control input to prevent rapid changes and generate a smooth control input. This term is weighted by positive diagonal matrix $\mathbf{w}_{p}\in\mathbb{R}^{2N\times 2N}$. The MPC variables consist of $\mathbf{P}_{x}$ and $\Delta\mathbf{F}_x$, where {$\mathbf{P}_{x}\in\mathbb{R}^{2N}$ represents the sequence of ZMP $z_{x,k}$ and centroidal moment $\tau_{y,k}$, treated as control inputs}, while $\Delta\mathbf{F}_x=[\Delta{f}_{x,1} \Delta{f}_{x,2} \cdots \Delta{f}_{x,m}]^{\textrm{T}}\in\mathbb{R}^m$ represents a vector consisting of additional footstep adjustments in the x-direction. {Here, although $\Delta\mathbf{F}_x$ is not defined as a variable in CP--CMP dynamics (\ref{eq/discrete CP-CMP}), it is utilized as an optimization variable to relax the ZMP constraint.}

{The constraints in (\ref{eq/CP-MPC constraint}) consist of three inequality conditions and one equality condition.}
The first constraint refers to the ZMP constraint, which confines the ZMP control input $\mathbf{Z}_x\in\mathbb{R}^{N}$ within the support polygon. The vectors $\overline{\mathbf{Z}}_x\in\mathbb{R}^{N}$ and $\underline{\mathbf{Z}}_x\in\mathbb{R}^{N}$ indicate the upper and lower limits of ZMP inputs, respectively.
{These limits are approximated by modeling the foot as a rectangle, defined by the foot's width and length relative to the reference ZMP $\mathbf{Z}^{ref}$, while taking yaw rotation into account.}
{To prevent the desired ZMP from being generated outside the support polygon due to approximation errors, we approximate the rectangular foot size such that the approximated support area falls within a range smaller than the actual foot shape-based support area.}
% {To prevent the desired ZMP from being generated in infeasible areas outside the support polygon or near the foot's edges due to approximation errors, we define a smaller support area than the actual support area.}
% \textcolor{red}{and these limits are determined by reflecting the size of both feet, {\color{blue}with their yaw orientation considered using the planned footstep}, approximated as rectangles based on the reference ZMP trajectory $\mathbf{Z}^{ref}$}. 
The matrix $\mathbf{I}_N\in\mathbb{R}^{N\times N}$ refers to an $N$ size identity matrix and the matrix $\mathbf{S}\in\mathbb{R}^{N\times m}$ represents a selection matrix whose element ${s}_{k,m}$ is 1 or 0. %The matrix, $\mathbf{S}$, activates $\Delta\mathbf{F}_x$ within the time horizon after the current Single Support Phase (SSP). 
{The element ${s}_{k,m}$ is set to 1 to activate $\Delta{f}_{x,m}$ when time step $k$ is included in the $m$-th footstep after the current Single Support Phase (SSP), and it is set to 0 when time step $k$ is not included in the $m$-th footstep, including the current SSP.}
Using $\Delta\mathbf{F}_x$ to relax the ZMP constraint in situations where ZMP control inputs are limited improves the performance of CP control by increasing flexibility in generating ZMP control inputs.
{The second constraint refers to the limit on the amount of centroidal moment that can be produced by the robot's motion. The vectors $\overline{\boldsymbol{\uptau}}_y\in\mathbb{R}^{N}$ and $\underline{\boldsymbol{\uptau}}_y\in\mathbb{R}^{N}$ indicate the upper and lower bounds of the centroidal moment, respectively. These constraints were experimentally determined by considering joint position and velocity limits.} The third constraint denotes the kinematic constraint of the additional footsteps to prevent singularities and self-collisions {between legs}. The vectors $\Delta\overline{\mathbf{F}}_x\in\mathbb{R}^{m}$ and $\Delta\underline{\mathbf{F}}_x\in\mathbb{R}^{m}$ refer to the upper and lower bounds of additional footsteps adjustment. {The last constraint signifies the terminal constraint imposed to ensure the convergence of CP tracking.}

{The number of footsteps $m$ in the optimization is set sufficiently large to accommodate the minimum walking duration and the time horizon. If the footsteps in the time horizon is less than $m$, the unnecessary footstep constraints are dynamically excluded from MPC by setting zero for the elements in $\mathbf{S}$ and constraining the footstep adjustment.}

{In (\ref{eq/CP-MPC cost function}), the generation of $\boldsymbol{\uptau}_y$ and $\Delta\mathbf{F}_x$ is regulated by the second and third cost terms, respectively. Therefore, for small disturbances, CP control is primarily achieved through ZMP control. For large disturbances, the variable weighting approach in Section \ref{Section/Variable Weighting} facilitates the generation of $\boldsymbol{\uptau}_y$ to reduce CP errors. Additionally, $\Delta\mathbf{F}_x$ is generated to relax the ZMP constraints of the next footsteps, leading to larger ZMP inputs and a reduction in CP errors. The weighting ratio of $\mathbf{w}_{\tau}$ and $\mathbf{w}_{F}$ can determine the preferred balance strategy (hip or stepping strategy) in the presence of large disturbances.}
The values of weighting parameters and constraints used in the experiments are summarized in Table \ref{table/Parameters}.

\begin{table}[t]
\caption{{Parameters used in the controllers for simulations and real robot experiments.}}
\label{table/Parameters}
\resizebox{\columnwidth}{!}{%
\begin{tabular}{|lll|}
\hline

\multicolumn{3}{|c|}{\cellcolor[HTML]{c6eefe}CP--MPC}                \\ \hline
\multicolumn{1}{|c|}{Parameters} & \multicolumn{2}{c|}{Value}                   \\ \hline
% \multicolumn{1}{|l|}{$\mathbf{w}_{\xi,x,y}$}        & \multicolumn{2}{l|}{$10(i=1) , 5(i=2 \sim N-10), 100(i = N-10 \sim N)$} \\ \hline
\multicolumn{1}{|l|}{$\mathbf{w}_{\xi,x,y}$}        & \multicolumn{2}{l|}{\begin{tabular}[c]{@{}l@{}}$10.0\,(i=1), 5.0\,(i=2 \sim N-10),$\\ $100.0\,(i = N-10 \sim N)$\end{tabular}} \\ \hline
\multicolumn{1}{|l|}{$\mathbf{w}_{p,x,y}$}        & \multicolumn{2}{l|}{\begin{tabular}[c]{@{}l@{}}$0.1\,(i=1), 10.0\,(i=2 \sim N-10),$\\ $0.1\,(i = N-10 \sim N)$\end{tabular}} \\ \hline
% \multicolumn{1}{|l|}{$\mathbf{w}_{p,x,y}$}        & \multicolumn{2}{l|}{$0.1(i=1) , 10(i=2 \sim N-10), 0.1(i = N-10 \sim N)$} \\ \hline
\multicolumn{1}{|l|}{$\mathbf{w}_{F,x,y}$}        & \multicolumn{2}{l|}{$0.001$} \\ \hline
\multicolumn{1}{|l|}{$\mathbf{w}_{\tau,y}$}        & \multicolumn{2}{l|}{$1.0\times10^{-6} \,(\Delta z_{min,x} : 0.05) \xrightarrow{} 0 \,(\Delta z_{max,x} : 0.1)$} \\ \hline
\multicolumn{1}{|l|}{$\mathbf{w}_{\tau,x}$}        & \multicolumn{2}{l|}{$1.0\times10^{-6} \,(\Delta z_{min,y} : 0.04) \xrightarrow{} 0 \,(\Delta z_{max,y} : 0.07)$} \\ \hline
%\multicolumn{1}{|l|}{$\mathbf{w}_{\tau,y}$}    & \multicolumn{1}{l|}{$1.0\times10^{-6} \,(\Delta z_{min,x} : 0.05)$} & $0 \,(\Delta z_{max,x} : 0.1)$ \\ \hline
%\multicolumn{1}{|l|}{$\mathbf{w}_{\tau,x}$}    & \multicolumn{1}{l|}{$1.0\times10^{-6} \,(\Delta z_{min,y} : 0.04)$} & $0 \,(\Delta z_{max,y} : 0.07)$\\ \hline
\multicolumn{1}{|l|}{${\mathbf{K}}_{d}$} & \multicolumn{2}{l|}{$50.0\,\mathbf{I}_N$} \\ \hline
\multicolumn{1}{|l|}{$\underline{\mathbf{Z}}_{x},\overline{\mathbf{Z}}_{x}\,[m]$}        & \multicolumn{2}{l|}{$(\mathbf{Z}_x^{ref}-0.09,\,\mathbf{Z}_x^{ref}+0.12)$} \\ \hline
\multicolumn{1}{|l|}{$\underline{\mathbf{Z}}_{y},\overline{\mathbf{Z}}_{y}\,[m]$}        & \multicolumn{2}{l|}{$(\mathbf{Z}_y^{ref}-0.07,\,\mathbf{Z}_y^{ref}+0.07)$} \\ \hline
\multicolumn{1}{|l|}{$\Delta\underline{\mathbf{F}}_x,\Delta\overline{\mathbf{F}}_x\,[m]$}        & \multicolumn{2}{l|}{$(-0.2,0.2)$ w.r.t$\;$support foot} \\ \hline
\multicolumn{1}{|l|}{$\Delta\underline{\mathbf{F}}_y,\Delta\overline{\mathbf{F}}_y\,[m]$}        & \multicolumn{2}{l|}{$(-0.1,0.03)$/$(-0.03,0.1)$ w.r.t$\,$ left/right support foot} \\ \hline
\multicolumn{1}{|l|}{$\underline{\boldsymbol{\uptau}}_{y,x},\overline{\boldsymbol{\uptau}}_{y,x}\,[N\cdot m]$} & \multicolumn{2}{l|}{$(-15.0,15.0)$ (Simulation) /$\,(-7.0,7.0)$ (Experiment)} \\ \hline

\multicolumn{3}{|c|}{\cellcolor[HTML]{FFE6FC}Stepping controller}  \\ \hline
\multicolumn{1}{|c|}{Parameters} & \multicolumn{2}{c|}{Value}                   \\ \hline
\multicolumn{1}{|l|}{$\mathbf{w}_{f,x,y}$}        & \multicolumn{2}{l|}{$1000.0$}                    \\ \hline
\multicolumn{1}{|l|}{$\mathbf{w}_{b,x,y}$}        & \multicolumn{2}{l|}{$3000.0$}                    \\ \hline
\multicolumn{1}{|l|}{$w_\gamma$}     & \multicolumn{2}{l|}{$1.0$}                 \\ \hline
\multicolumn{1}{|l|}{$\underline{{f}}_{x,y},\overline{{f}}_{x,y}\,[m]$}         & \multicolumn{2}{l|}{$({f}_{nom,x,y}-0.05,{f}_{nom,x,y}+0.05$)}                \\ \hline
\multicolumn{1}{|l|}{$\underline{{b}}_{x,y},\overline{{b}}_{x,y}\,[m]$}         & \multicolumn{2}{l|}{$({b}_{nom,x,y}-0.10,{b}_{nom,x,y}+0.10)$}                \\ \hline 
\multicolumn{1}{|l|}{$\underline{T},\overline{T}\,[s]$}  & \multicolumn{2}{l|}{$(T_{nom}-0.2,T_{nom}+0.2)\;$ /$\;\gamma = e^{\omega T}$}                \\ \hline

\multicolumn{3}{|c|}{\cellcolor[HTML]{C6FDC6}HQP--WBIK}                \\ \hline
\multicolumn{1}{|c|}{Parameters}              & \multicolumn{2}{c|}{Value}                   \\ \hline
\multicolumn{1}{|l|}{$\mathbf{w}_{J}$}        & \multicolumn{2}{l|}{\begin{tabular}[c]{@{}l@{}}$120.0$ \text{(Foot pose)}, $10.0$ \text{(CoM)}, $0.1$ \text{(Hand pose)}, \\ $1.0$ \text{(Chest orientation)}, $5.0$ \text{(Pelvis orientation)}\end{tabular}}                    \\ \hline
\multicolumn{1}{|l|}{$\mathbf{w}_{A}$}        & \multicolumn{2}{l|}{$1.0$}                    \\ \hline
\multicolumn{1}{|l|}{$\mathbf{w}_{\dot{q}}$}  & \multicolumn{2}{l|}{$0.0001$} \\ \hline%, $q-q_{prev}: 1.5$}                    \\ \hline
\multicolumn{1}{|l|}{$\mathbf{K}_{p}$}         & \multicolumn{2}{l|}{\begin{tabular}[c]{@{}l@{}}$30.0$ \text{(Foot pose)}, $30.0$ \text{(CoM)}, $10.0$ \text{(Hand pose)}, \\ $10.0$ \text{(Chest orientation)}, $10.0$ \text{(Pelvis orientation)}\end{tabular}}                     \\ \hline
\multicolumn{1}{|l|}{$\mathbf{K}_{q}$}        & \multicolumn{2}{l|}{$50.0$}                    \\ \hline
% \multicolumn{1}{|l|}{$\varepsilon$}           & \multicolumn{2}{l|}{$5.0 \ (h^{des}=1) \ \xrightarrow{} 0 \ (h^{des}=3)$}                    \\ \hline
\end{tabular}%
}
\end{table}

% 현재 tick에서 미래까지 봤을때 지지발에 위치한 cp 레퍼런스를 제일 잘 추종하는 스윙 발의 위치가 del F, 그래서 거기다가 스윙발을 놓았다. 발 랜딩 이후에는 예전에 예측한대로 안움직여도 된다. 행동이 바뀐거니까 Local Target velocity에 맞춰서 다시 CP reference를 바꾼다. 어짜피 매틱 푸는거니 Reference가 바뀌어도 상관없다. 미래 Input까지 다 사용하는게 아니라서.
\begin{comment}
   

\subsection{Methodology for Generating the Footstep Displacement Variable}

% Figure environment removed  
This section describes the generation principle of the footstep displacements, $\Delta\mathbf{F}_x$, in the CP--MPC. 
Fig. \ref{figure/delF explanation} illustrates the behavior of the algorithm during prediction horizon in response to a disturbance while a robot is walking in place. Additionally, Fig. \ref{figure/delF explanation} focuses solely on the $\Delta{f}_{x,1}$ component of the $\Delta\mathbf{F}_x$, which represents the adjustment for the next footstep. The green dot indicates the current CP and the blue and black dots indicate the predicted, and reference CP trajectories, respectively. The red dots indicate the ZMP inputs used to control the CP over the time horizon. % CP ref 는 스윙 도중에 안바뀌고 지지발 중앙이라고 가정.

Generally, small CP errors can be caused by various factors, including foot-ground impacts during walking, model uncertainties, and system noises. It is typically possible to control the small CP error by ZMP control inputs within the ZMP constraint. However, in the event of large disturbances from external forces applied to the robot or uneven terrain, the ZMP control inputs within the ZMP constraint may be insufficient to control the CP error, as shown in Fig. \ref{figure/delF explanation}(a). This can result in the robot losing balance. %As for Fig. \ref{figure/delF explanation}(b), the activation of the $\Delta{f}_{x,1}$ is possible after the current SSP within the remaining time horizon in (\ref{eq/CP-MPC constraint}). This indicates that the ZMP constraint can be relaxed by adjusting the next footstep position.
On the other hand, if the footstep can be used along with the zmp control as shown in Fig.\ref{figure/delF explanation}(b), the robot can step forward by $\Delta{f}_{x,1}$. This footstep adjustment generates larger ZMP inputs than using only the ZMP control and enables better CP tracking control performance.

In terms of the overall cost reduction in (\ref{eq/CP-MPC cost function}), $\Delta\mathbf{F}_x$ increases the third term of (\ref{eq/CP-MPC cost function}), which is the regulation term of footstep displacement. On the other hand, $\Delta\mathbf{F}_x$ can relax the ZMP constraint of the next footstep, resulting in a larger ZMP input and less CP error. Therefore, if the additional $\Delta\mathbf{F}_x$ can decrease the CP error cost (the first term of (\ref{eq/CP-MPC cost function})) more than the increased regulation cost by relaxing the ZMP input constraint, CP--MPC generates $\Delta\mathbf{F}_x$.

In conclusion, the adjusted ZMP constraint enables the generation of more ZMP inputs, which can control the CP error and maintain the balance of the robot.
\end{comment}
%%%
\subsection{Variable Weighting Parameter for CAM control}
\label{Section/Variable Weighting}
The CAM control approaches have been widely utilized in many studies to overcome large disturbances that affect robots. In general, a desired CAM of humanoids is initially generated to overcome external disturbances, but it {should converge} to zero in steady-state. 
%This is because a constant desired CAM cannot be generated continuously by the joint limits of the robot, and constantly deviating from its initial pose can lead to walking instability.
This is because there is no need to generate constant CAM for the balancing purpose in steady-state, and furthermore, a constant desired CAM cannot be generated continuously due to the joint limits of the robot.
The convergence process of desired CAM can be implemented heuristically during a specific time period \cite{yi2016whole} or under certain conditions \cite{kim2022humanoid, schuller2021online, kanamiya2010ankle}. However, this issue is more commonly addressed through optimization methods by implementing a CAM regulation term or CAM damping control term in the cost function\cite{jeong2019robust, khazoom2022humanoid, romualdi2022online, ding2021versatile, park2021whole}.
In our case, the convergence process of desired CAM is also addressed through the CAM damping control term in MPC optimization. However, these approaches commonly suffer from a trade-off between generating the desired CAM to overcome the disturbance and the term that drives the CAM to zero, which can have a negative impact on the balancing performance.  

% Figure environment removed 
To address this problem, a novel variable weighting approach based on MPC is proposed. 
The purpose of this approach is not to compromise the CP control performance when the robot is subjected to large disturbances, by reducing the weighting parameter $\mathbf{w}_{\tau}$ to reduce the influence of the damping cost term in (\ref{eq/CP-MPC cost function}).
This allows the robot to generate more control input $\bm{\uptau}$ in the presence of disturbances, thereby improving balance performance. Furthermore, in the steady-state when the disturbances diminish, the weighting parameter can be increased to dampen the generation of CAM.

Fig. \ref{figure/VariableWeighting} presents the scheme of the decision process of the variable weighting parameters in the x-direction.
In Fig. \ref{figure/VariableWeighting}, when a disturbance is applied to the robot, ZMP control inputs $\mathbf{Z}_x$ increase in CP--MPC to control the perturbed CP.
{Afterward, the weighting parameters $\mathbf{w}_{\tau}$ for the next MPC cycle are determined through a one-to-one mapping using a cubic function based on the magnitude of the delta ZMP inputs $|\Delta \mathbf{Z}_x|=|\mathbf{Z}_{x}-\mathbf{Z}_x^{ref}|\in\mathbb{R}^{N}$ in the current MPC cycle.}
% Then, weighting parameters, $\mathbf{w}_{\tau}$, are determined through a one-to-one mapping using cubic function according to the magnitude of the delta ZMP inputs, $|\Delta \mathbf{Z}_x| = |\mathbf{Z}_{x}-\mathbf{Z}_x^{ref}|\in\mathbb{R}^{N}$. 
{Note that the coefficients of the cubic function $f$ are determined by the following four conditions: $f(|\Delta z_x|_{max})=w_{\tau,min},\;f(|\Delta z_x|_{min})=w_{\tau,max},\;f'(|\Delta z_x|_{max})=f'(|\Delta z_x|_{min})=0$,} and $\mathbf{Z}_x^{ref}\in\mathbb{R}^{N}$ represents the reference ZMP trajectory in the x-direction based on the pre-planned footstep.
%Note that $\mathbf{Z}_x^{ref}\in\mathbb{R}^{N}$ represents the reference ZMP trajectory in the x-direction based on the pre-planned footstep. This mapping enables a reduction in $\mathbf{w}_{\tau}$ as the $|\Delta \mathbf{Z}_x|$ for controlling the CP increases.
As a result, the damping cost term in (\ref{eq/CP-MPC cost function}) becomes less influential, and the centroidal moment $\bm{\uptau}_y\in\mathbb{R}^{N}$ is primarily generated to balance the robot against disturbances. 
%The operational principle of this algorithm is based on the observation that situations where the ZMP inputs, $\mathbf{Z}$, increase significantly require more control inputs to control the CP trajectory.
The operational principle of this algorithm is based on the observation that the increased ZMP control inputs $\mathbf{Z}_x$ indicate that the robot is subjected to a substantial disturbance, thereby requiring more control inputs to accurately track the CP trajectory.
Therefore, the capability of CP control can be increased by increasing the generation of another control input $\bm{\uptau}_y$. The same process is applied in the y-direction, independent of the x-direction.

%In terms of the approach to varying the control parameters in the optimization problem, it may also be helpful to reduce the diagonal element of the damping matrix, $\mathbf{K}_d$, in addition to adjusting the variable weighting. However, it should be noted that even with these modifications, $\uptau$ is still subject to a regulation term.
% Finally, the desired centroidal moment $\bm{\tau}^{des}=[\tau_{y}^{des}\,-\tau_{x}^{des}]$ is integrated into the desired CAM at each time step as follows:
{Finally, the desired centroidal moment $\bm{\tau}^{des}=[\tau_{y}^{des}\,-\tau_{x}^{des}]$ is computed and integrated at each time step, accumulating into the desired CAM as follows,}
\begin{equation}
{h}_i^{des} = \int{\tau_{i}^{des}}\,dt,\quad i = x,y %\\
% {h}_x^{des} = \int{\tau_{x,0}}\,dk.
\label{eq/tauToCAM}
\end{equation}
{Here, the z-axis component is set to zero, and the desired CAM $\mathbf{h}^{des}\in\mathbb{R}^3$ is realized through the HQP-based WBIK described in Section \ref{Section/WBIK}.}
% , which prioritizes CAM control first and performs the initial pose return task with secondary priority, resulting in improved CAM control performance.
% The effectiveness of the variable weighting approach will be validated in Section \ref{Section/Results/Simulation/VariableWeighting}.



\section{Stepping Controller based on CP--MPC}
\label{Section/Stepping Controller}
While footstep position adjustment has been extensively studied in MPC approaches \cite{herdt2010online, joe2018balance, romualdi2022online, ding2022dynamic, kim2023foot}, optimizing the step time in the MPC formulation is still considered a challenging problem due to its non-convex nature. However, since step time adjustment improves balancing performance compared to adjusting footstep position alone, various methods such as heuristic approaches \cite{griffin2017walking, ding2022dynamic} or optimization based on LIPM assumptions \cite{kryczka2015online, nazemi2017reactive, jeong2019robust, khadiv2020walking, kim2023foot} have been conducted in addition to MPC approaches.

In this study, we developed a hierarchical control structure of CP--MPC and the stepping controller, enabling step time optimization based on MPC variables. Specifically, CP--MPC utilizes ZMP, CAM, and footstep adjustment to track the CP trajectory, while the stepping controller determined the footstep position and step time to control the CP offset at the end of the step.

\subsection{Overview of the Stepping Controller}
In this section, an overview is provided for the adjustment of footstep position and step time to control the CP offset in the stepping controller.
Our stepping controller is designed based on the QP optimization proposed by Khadiv et al. \cite{khadiv2016step, khadiv2020walking} using LIPFM-based CP end-of-step dynamics. The dynamics of CP end-of-step based on LIPFM can be expressed as follows, 
\begin{equation}
\label{eq/stepping_cpeos}
\bm{\xi}_{T}=\mathbf{f}^{} + \mathbf{b} = (\bm{\xi}-\mathbf{P}_{ssp})e^{-\omega t}\gamma^{}_{} + \mathbf{P}_{ssp}. 
\end{equation}
In \cite{khadiv2016step, khadiv2020walking}, the stepping controller is primarily designed to adjust the next footstep position $\mathbf{f}=[f_x\,f_y]$ and step time term $\gamma=e^{\omega T}$ ensuring that the CP offset $\mathbf{b}=[b_x\,b_y]$ closely aligns with the planned CP offset. The CP offset in LIPM-based walking has been identified as a critical factor that determines the initial point of the exponential growth of the CP in the next step, and has been emphasized in numerous studies \cite{englsberger2011bipedal, khadiv2016step, jeong2017biped, jeong2019robust, khadiv2020walking} as a key factor that affects the walking stability. Based on these concepts, we also adjust the footstep position and step time to control the CP offset.

Fig. \ref{figure/Steppingcontrol} presents the overall concept of our approach in the presence of disturbances. When the disturbance is applied to the robot, the CP end-of-step (purple point) $\bm{\xi}^{'}_{T}$ is predicted using (\ref{eq/stepping_cpeos}) to deviate significantly due to the disturbance. In this case, the stepping controller adjusts the footstep position $\mathbf{f}$ and the step time term $\gamma$ to control the CP offset $\mathbf{b}$ at the end of the step. %Unlike \cite{khadiv2016step, khadiv2020walking}, in our case, $\mathbf{f}^{nom}$ is calculated based on $\Delta\mathbf{f}_1$ planned from CP--MPC, and $\mathbf{f}$ is adjusted to prevent significant deviations from $\mathbf{f}^{nom}$. Consequently, $\gamma$ is primarily adjusted to control $\mathbf{b}$. 
% gamma가 어떻게 조정되는지와 gamma가 바뀌면 왜 좋아지는지에 대해 설명.
\subsection{Parameter Decision based on CP--MPC}
\label{Section/Stepping Controller/Parameter decision}
%CP MPC는 보행패턴 생성기로부터 생성된 CP 궤적을 추종해서 전반적인 보행 안정성을 가지는게 목적, 스테핑 제어기는 보행패턴생성기로부터 미리 plan된 CP offset을 제어해서 다음 스텝을 안정적으로 가져가면서 정해진 velocity를 따르는게 목적. 
% 스테핑 제어기 없이 CP-MPC만 썼을때 (외란 큰 경우): 스테핑 제어기가 없는데 외란이 크면 CP-MPC의 del f1에 의해 발의 위치는 최대한 바꿔볼테지만, CP eos error는 크기때문에 다음 스텝에서 원하는 CP offset이 아니라 보행 불안정해질 수 있다.
% 스테핑 제어기 없이 CP-MPC만 썼을때 (del f1 조금 생길정도의 외란): 마찬가지로 CP offset 제어는 안되지만, CP eos ref와 CP eos는 거의 비슷할거고 del f1 사이 값이 적당히 작아서 다음스텝에서 안정.
% 스테핑 제어기 (외란 큰 경우): 시간과 발을 조정해서 CP offset을 제어함. 우리 구조에서는, CP-MPC에서 CP 에러를 리밋안에서 최적으로 제어하기 위해 del f1을 생성했지만, CP 제어가 완벽하지 않아서 CP eos를 예측해보니 xi_T 에러가 크다. 이때, 스테핑 제어기에서 발의 위치를 바꿔서 CP offset을 제어하려고해도 del f1에서 이미 kinematic 제한까지 걸렸으니, 스테핑 제어기에서 발을 더 많이 뻗는건 제한적일거고, 그래서 스텝시간이 조절되면서 CP offset이 제어된다. 
% 스테핑 제어기 (외란 작은 경우): 외란이 작은 경우에는 del f1이 작을테고, 스텝 시간이나 발 위치 조정 거의 안해도 CP offset 제어 됨.
% Figure environment removed
This section describes the parameter decision for the nominal values of the next foot position $\mathbf{f}$, CP offset $\mathbf{b}$, step time term $\gamma$, and CMP parameters $\mathbf{P}_{ssp}$ in (\ref{eq/stepping_cpeos}).
% 
The nominal footstep position $\mathbf{f}^{nom}=[{f}^{nom}_{x}\,{f}^{nom}_{y}]$ is defined as the sum of the reference next footstep position $\mathbf{f}_1^{ref}$ determined by the footstep planner and the footstep adjustment $\Delta \mathbf{f}_1$ optimized by CP--MPC to minimize the CP error over the future horizon, as follows, 
\begin{equation}
\mathbf{f}^{nom} = \mathbf{f}_1^{ref} + \Delta \mathbf{f}_1. 
\label{eq/F_nom}
\end{equation}
Here, $nom$ and $1$ refer to the nominal value and the next footstep, respectively.
% By incorporating $\Delta \mathbf{f}_1$ into the calculation of $\mathbf{f}^{nom}$ instead of solely relying on $\mathbf{f}_1^{ref}$, it leads to a reduction in the cost of adjusting $\mathbf{f}$ (see Fig. \ref{figure/Steppingcontrol}) and consequently reduces the CP offset optimization error.

Next, the nominal CP offset $\mathbf{b}^{nom}=[{b}^{nom}_x\,{b}^{nom}_y]$ is defined as the difference between the reference CP end-of-step $\bm{\xi}_{T}^{ref}=[{\xi}^{ref}_{T,x}\,{\xi}^{ref}_{T,y}]$ planned by the walking pattern generator and the reference next footstep position $\mathbf{f}^{ref}_1$, {which is equivalent to reference CP offset $\mathbf{b}^{ref}$}, to adhere to the target velocity command. 
\begin{equation}
\mathbf{b}^{nom} = \bm{\xi}^{ref}_{T} - \mathbf{f}^{ref}_1 = \mathbf{b}^{ref}. 
\label{eq/b_nom}
\end{equation} 
%Note that $\bm{\xi}_{T}$ represents the predicted CP at time $T$, which is the end of the current step, among the predicted CP trajectory, $\bm{\Xi}=[\bm{\xi}_{k+1}\,\bm{\xi}_{k+2}\,\cdots\,\bm{\xi}_{T}\,\cdots\bm{\xi}_{k+N}]^T$. 
The nominal step time term $\gamma^{nom}=e^{\omega T^{ref}}$ is determined by a pre-defined step time based on the target velocity command.

\begin{comment}
The CMP parameter, $\mathbf{P}_{ssp}=[p_{ssp,x}\,p_{ssp,y}]$, (as shown in Fig. \ref{figure/Steppingcontrol}) are calculated based on the average of CP--MPC control inputs (ZMP and centroidal moment control inputs) corresponding to the remaining SSP time within the prediction horizon. The decision to utilize the average CP--MPC control inputs in (\ref{eq/stepping_cpeos}) is based on two considerations. Firstly, by approximating the CP--MPC control inputs as a single point, it becomes possible to reflect the tendencies of where the robot's CMP acts. This approach enables more accurate prediction of the CP end-of-step compared to using a fixed CMP at the center of the support foot or only the CMP control input for the current time step. Secondly, it can improve the overall stability of the stepping controller by smoothing out the noisy input at the current time step. The equation of the CMP parameter is as follows, 
\end{comment}
{The CMP parameters $\mathbf{P}_{ssp}=[p_{ssp,x}\,p_{ssp,y}]$ (as shown in Fig. \ref{figure/Steppingcontrol}) are calculated as a single point accurately representing CP--MPC control inputs (ZMP and centroidal moment control inputs) corresponding to the remaining SSP within the prediction horizon. This approach enables a more precise reflection of the future behaviors of CP--MPC. In contrast to assuming that the CMP is fixed either at the center of the supporting foot or at the current position during the SSP, this method allows for a more accurate prediction of the CP end-of-step based on the inputs of CP--MPC.} The equation of the CMP parameter is as follows, 
% \begin{equation}
% \mathbf{P}_{ssp} = \sum_{i=k}^{T}\frac{\mathbf{z}_i+\bm{\tau}_i/mg}{T-k+1} 
% \label{eq/stepping_CMP input}
% \end{equation}
\begin{equation}
{
\mathbf{P}_{ssp} = \frac{1}{1-A^{T_k-k}}\sum_{i=k}^{T_k-1}A^{T_k-i-1}\mathbf{B} 
\begin{bmatrix}
\mathbf{z}_{i} \\ \bm{\tau}_{i}
\end{bmatrix}}
\label{eq/stepping_CMP input}
\end{equation}
where $T_k$ represents the time step at which SSP ends within the MPC horizon, $\mathbf{z}_i=[z_{x,i}\,z_{y,i}]$ and $\bm{\tau}_i=[\tau_{y,i}\,-\tau_{x,i}]$ denote the ZMP and centroidal moment control input at each time step $i$, respectively. {The derivation for $\mathbf{P}_{ssp}$ is provided in Appendix \ref{section/appendix}}.
 
\subsection{Formulating a QP problem for Stepping Control}
\label{Section/Stepping Controller/QP formulation}
% 2. (a) QP formulation for Stepping Control 
The QP formulation for adjusting the footstep position and the step time to control the CP offset is as follows.
\begin{align}
\label{eq/stepping_cost function}
& \underset{\mathbf{f}, \gamma, \mathbf{b}}{\text{min}} & & {\mathbf{w}_{f}}\| \mathbf{f}^{}-\mathbf{f}^{nom}\|^{2} + {w_{\gamma}}(\gamma-\gamma^{nom})^{2} \\ 
& & +&{\mathbf{w}_{b}}\|\mathbf{b}-\mathbf{b}^{nom}\|^{2} \nonumber \\ \label{eq/stepping_constraint}
& \text{s. t.} & & \underline{\mathbf{f}}^{}\leq  \mathbf{f}^{} \leq \overline{\mathbf{f}}^{}\\ 
& & &\underline{\gamma}^{}_{}\leq \gamma^{}_{} \leq \overline{\gamma}^{}_{} \nonumber\\
& & &\underline{\mathbf{b}}\leq \mathbf{b} \leq \overline{\mathbf{b}} \nonumber
\\ 
& & & \mathbf{f}^{} + \mathbf{b} = (\bm{\xi}-\mathbf{P}_{ssp})e^{-\omega t}\gamma^{}_{} + \mathbf{P}_{ssp} \nonumber 
\end{align}

The cost function (\ref{eq/stepping_cost function}) is composed of the footstep position error term, step time error term, and CP offset error term, and each cost is weighted by $\mathbf{w}_f$, $w_\gamma$, and $\mathbf{w}_b$, respectively. In each of the optimization variables, $\mathbf{f}=[f_x\,f_y]$ represents the next footstep position, $\gamma$ represents the step time term, and $\mathbf{b}=[b_x\,b_y]$ represents the CP offset. The variable $\gamma = e^{\omega T}$ is introduced to linearize the equality constraint in (\ref{eq/stepping_constraint}). The weighting parameter was set primarily to adjust the step time $T$ while also enhancing the consistency of $\mathbf{f}$ with the CP--MPC output. To achieve this, we set a smaller weighting value for the step time than the other weighting values as listed in Table. \ref{table/Parameters}.

In (\ref{eq/stepping_constraint}), the inequality constraint is composed of the upper and lower bounds for each optimization variable, and the equality constraint is defined by the LIPFM-based CP end-of-step dynamics in (\ref{eq/stepping_cpeos}).
The variable $\bm{\xi}$ represents the current CP, $t$ denotes the elapsed time since the start of the swing phase, and $\mathbf{P}_{ssp}$ refers to the CMP parameter used in the equality constraint.

%%%%%%%%%%%%%%%%%%%%%%%%%%%%%%%%%%%%%%%%%%%%%%%%%%%%%%%%%%%%%%%%%%%%%%%%%%%%%%%%%%%%%%%%%%%%%%%%%%%%%%%%%%%%%%%%%%%%%%
\subsection{Characteristic of Our Parameter Selection Approach compared to Previous Methods}
\label{Section/Stepping Controller/Characteristic}
In this section, we introduce improvements in our parameter selection approach compared to previous studies \cite{khadiv2016step, nazemi2017reactive, khadiv2020walking, kim2023foot}.

% In \cite{khadiv2016step, khadiv2020walking}, since the reference CoM trajectory is not defined, ensuring stable walking is reliant on precise control of the CP offset through adjusting footstep position and step time. In \cite{nazemi2017reactive}, the stepping controller proposed in \cite{khadiv2016step, khadiv2020walking} was employed to calculate the next footstep position and CP offset. The resulting next footstep position and CP offset were then utilized in the reference CP trajectory generation. 
% The CP offset used in \cite{khadiv2016step, khadiv2020walking, nazemi2017reactive} is calculated as follows,
Controlling the CP offset is crucial as it determines the initial velocity of the CP in the next step, thereby significantly affecting walking stability. Previous studies \cite{khadiv2016step, khadiv2020walking, nazemi2017reactive} rigorously controlled the CP offset. However, in these studies, the CP offset was derived under the assumption of a constant walking velocity, as follows,
\begin{align}
\label{eq/cp_offset_bolt_x}
{b}^{nom}_x &= \frac{L}{e^{\omega T^{nom}}-1},\\ {b}^{nom}_y &= (-1)^{n}\frac{D}{e^{\omega T^{nom}}-1}-\frac{W}{e^{\omega T^{nom}}-1}. 
\label{eq/cp_offset_bolt_y}
\end{align}
Here, $L$ represents the step length, $D$ denotes the default step width during walking, and $W$ represents the deviation with respect to the default step width. %In other words, the forward walking speed is determined by $L$, while the lateral walking speed is determined by $W$. 
When $n$ is equal to 1, it indicates the right foot support phase, and when $n$ is equal to 2, it indicates the left foot support phase.

The nominal CP offset in (\ref{eq/cp_offset_bolt_x}) and (\ref{eq/cp_offset_bolt_y}) can be derived simply based on the CP end-of-step dynamics, and a detailed derivation is presented in \cite{khadiv2020walking}. However, these derivations include the assumption that the current step and the next step have the same values for $L$, $W$, and $T^{nom}$. Therefore, this assumption is violated when the walking velocity between the current and next steps are different (e.g., walking with different speeds for each step or sudden changes in walking direction). On the contrary, in our approach, while using the same QP formulation (\ref{eq/stepping_cost function}) as in previous studies, the CP offset is determined from (\ref{eq/b_nom}), taking into account the variations in walking velocity at each step. 
%However, the nominal value for the CP offset used in these studies \cite{khadiv2016step, nazemi2017reactive, khadiv2020walking} is derived under the assumption that the CP offset for the next step equals that of the current step, i.e., the robot is assumed to walk at a constant velocity. Therefore, this assumption is violated when the walking velocity changes between the current and next steps are different (e.g., walking with gradually increasing speed, sudden changes in walking direction, or lateral walking). 

% In our previous study \cite{kim2023foot}, the nominal value for CP offset was calculated as the difference between the pre-planned CP end-of-step based on the walking pattern generator (i.e., using $\bm{\xi}^{ref}_{T}$ instead of $\bm{\xi}_{T}$) and the adjusted next footstep position for CP control. However, in such cases, although the next footstep position is adjusted to control the CP error, the pre-planned CP end-of-step, $\bm{\xi}^{ref}_{T}$, cannot reflect the perturbation of the CP. Consequently, controlling the CP offset calculated from this approach can lead to the initial CP error at the beginning of the next step.
% In this paper, to address these issues, the CP offset is calculated based on the CP end-of-step, $\xi_{T}$, and $\mathbf{f}^{nom}$ derived from the CP--MPC. This approach is based on the consideration that if the next footstep position is calculated based on the CP--MPC, CP end-of-step should also be predicted based on CP--MPC to more accurately reflect the behavior of CP--MPC. 

Next, in previous studies \cite{khadiv2016step, nazemi2017reactive, khadiv2020walking, kim2023foot}, it was assumed that the ZMP or CMP is fixed at the center of the support foot ($\mathbf{P}_{ssp} = \mathbf{P}^{ref}$), based on the point foot assumption. However, in most humanoid robots, the foot has a finite size and the ground reaction forces do not act exactly at the center of the support foot. 
Indeed, the larger the robot's feet, the further the robot's ZMP or CMP can deviate from the center of its support foot. This can potentially violate the point foot assumption on the CP dynamics based on LIPM or LIPFM. Therefore, in this paper, the CMP parameter obtained from CP--MPC is used to consider the behavior of CMP by CP--MPC instead of the point foot assumption.

In Section \ref{Section/Results/Simulation/SteppingController}, the effectiveness of the parameter selection approach is analyzed through a comparison with the previous methods.
{
\section{HQP--based Whole-Body Inverse Kinematics}
\label{Section/WBIK}
% [TODO]: 섹션 3도 수정해야 함. -> LOWER BODY IK, HQP-CAM CONTROLLER - 완료! [2024-08-14]
% [TODO]: Figure 2 수정. - 완료!  [2024-08-14]
This section introduces an HQP--based WBIK to achieve the outputs from the walking pattern generator, CP--MPC, and stepping controller. The HQP-based WBIK extends our previous HQP--CAM controller \cite{kim2022humanoid}, which was limited to upper-body CAM control, into a whole-body control framework. 

The main objective of HQP--WBIK is to track the reference foot and CoM trajectories, as well as the desired CAM, while integrating initial pose recovery after CAM tracking motion into the hierarchical control framework. HQP--WBIK prioritizes CAM control over initial pose recovery by establishing strict task hierarchies to enhance balance performance.

The formulation of the HQP--WBIK is defined as follows.
\begin{align}
\label{eq/1st WBIK cost}
& \underset{\mathbf{\dot{q}_1}}{\text{min}}  & & \sum_j{ 
\| \mathbf{J}_{1,j}\mathbf{\dot{q}}_{1} -\mathbf{\dot{x}}^{des}_{1,j}\|^{2}_{\mathbf{w}_{\mathbf{J}}} +
\| \mathbf{A}\mathbf{\dot{q}}_{1}-\mathbf{h}^{des}\|^{2}_{\mathbf{w}_{\mathbf{A}}} +\| \mathbf{\dot{q}}_{1}\|^{2}_{\mathbf{w}_{\mathbf{\dot{q}}}}
} \\
\label{eq/1st WBIK constraint}
& \text{s. t.} & & \underline{\mathbf{\dot{q}}}_{1}\leq  \mathbf{\dot{q}}_{1} \leq \overline{\mathbf{\dot{q}}}_{1} \\ 
& & &  \mathbf{K}_q (\underline{\mathbf{q}}_{1} - \mathbf{q}_{1}) \leq  \dot{\mathbf{q}}_{1} \leq \mathbf{K}_q (\overline{\mathbf{q}}_{1} - \mathbf{q}_{1}) \nonumber \\
\label{eq/2nd WBIK cost}
& \underset{\mathbf{\dot{q}_2}}{\text{min}} & &\sum_j{\| \mathbf{J}_{2,j}\mathbf{\dot{q}}_{2} -\mathbf{\dot{x}}^{des}_{2,j}\|^{2}_{\mathbf{w}_{\mathbf{J}}}} 
+ \| \mathbf{\dot{q}}_{2}\|^{2}_{\mathbf{w}_{\mathbf{\dot{q}}}} \\
\label{eq/2nd WBIK constraint}
& \text{s. t.} & &  \mathbf{J}_{1,j} \mathbf{\dot{q}}_{2} = \mathbf{J}_{1,j} \mathbf{\dot{q}}_{1} \\
& & &  \mathbf{A} \mathbf{\dot{q}}_{1} - \mathbf{\boldsymbol{\varepsilon}} \leq \mathbf{A} \mathbf{\dot{q}}_{2} \leq \mathbf{A} \mathbf{\dot{q}}_{1} + \mathbf{\boldsymbol{\varepsilon}}   \nonumber \\
& & & \underline{\mathbf{\dot{q}}}_{2}\leq  \mathbf{\dot{q}}_{2} \leq \overline{\mathbf{\dot{q}}}_{2} \nonumber   \\ 
& & &  \mathbf{K}_q (\underline{\mathbf{q}}_{2} - \mathbf{q}_{2}) \leq  \dot{\mathbf{q}}_{2} \leq \mathbf{K}_q (\overline{\mathbf{q}}_{2} - \mathbf{q}_{2}) \nonumber 
\end{align} 
The optimization variable for the hierarchy~\(i\in\{1,2\}\) is the joint velocity \(\mathbf{\dot{q}}_i\in \mathbb{R}^{n+6}\), where \(n\) and \(6\) represent the number of actuated joints and virtual joints, respectively.
$\mathbf{J}_{i,j}$ and $\mathbf{\dot{x}}^{des}_{i,j}$ denote the Jacobian matrix and the desired velocity command for \(j\)-th task in the hierarchy $i$, respectively. 
The desired velocity command is defined as $\mathbf{\dot{x}}^{des}_{i,j} = \mathbf{\dot{x}}^{ref}_{i,j} + \mathbf{K}_p({\mathbf{x}}^{ref}_{i,j} - {\mathbf{x}}_{i,j})$ to minimize position or orientation error of the \(j\)-th task in the hierarchy $i$, where $\mathbf{K}_p$ is a positive feedback gains.
%%%%% 1st version %%%%%
% where $\mathbf{\dot{q}}_i\in\mathbf{R}^{n+6}$ and $\mathbf{q}_i\in \mathbf{R}^{n+6}$ denote the optimization variables, i.e., the joint velocity and joint position for each hierarchy~$i\in \{1, 2\}$, where the \(n\) is the number of actuated joints and \(6\) represents the number of virtual joints.
% $\mathbf{J}_{i,j}$, $\mathbf{A}$, $\mathbf{\dot{x}}^{cmd}_{i,j}$, $\mathbf{h}^{des}$, and $\mathbf{\boldsymbol{\varepsilon}}$ denote the Jacobian matrix, centroidal momentum matrix, desired velocity command, CAM from~(\ref{eq/tauToCAM}) and the tolerance for each hierarchy $i$ and each desired motion $j$, respectively. 
% $\mathbf{w}_{\mathbf{J}}$, $\mathbf{w}_{\mathbf{A}}$ are the weight matrices for the pose and CAM control objective, respectively.
% $\underline{\mathbf{\dot{q}}}_{i}$ and $\overline{\mathbf{\dot{q}}}_{i}$ are lower and upper bounds for joint velocities.
% $\underline{\mathbf{{q}}}_{i}$ and $\overline{\mathbf{{q}}}_{i}$ are lower and upper bounds for joint positions, while $\mathbf{K}_q$ is the weight matrix related to joint position limits.

% In the first hierarchy, the desired motions are composed of the CoM position, foot position, and orientation, which are crucial for balancing. In the second hierarchy, the desired motions are composed of the pelvis orientation, chest orientation, hand position, and orientation, which are needed for designing the walking behaviors or task implementation.
% By setting the tolerance as a function of the 2-norm of the desired CAM, $\boldsymbol{\varepsilon}(\mathbf{h}^{des})$, CAM control can be prioritized over pose recovery control, and each control target can also be transitioned smoothly~\cite{kim2022humanoid}.
% Joint velocity is computed from the HQP-WBIK, reducing the position and orientation errors of the desired motion as $\mathbf{\dot{x}}^{cmd}_{i,j} = \mathbf{\dot{x}}^{ref}_{i,j} + \mathbf{K_p}({\mathbf{x}}^{ref}_{i,j} - {\mathbf{x}}_{i,j}) $ where the $\mathbf{K_p}$ is the positive diagonal matrix. Joint position is numerically integrated using the optimized joint velocity command from the $2^{nd}$ hierarchy.
% Finally, the joint PD controller calculates the joint torques required to track the optimized joint position and velocity. 
% These joint torques are then commanded to the joints in conjunction with the gravity compensation torque.
%%%%% 2nd version %%%%%
In the first hierarchy, the cost function~(\ref{eq/1st WBIK cost}) is composed of three cost terms.
The first cost term serves as the desired velocity command tracking control, ensuring that the robot follows the reference CoM position $\mathbf{c}^{ref}\in\mathbb{R}^3$ and desired foot pose as $\mathbf{e}^{des}_{L,R} = \mathbf{e}^{ref}_{L,R} + \triangle\mathbf{e}_{L,R}\in\mathbb{R}^6$.
The second cost term serves as the desired CAM tracking control.
Here, \(\mathbf{A}\in\mathbb{R}^{3 \times (n+6)}\) denotes the angular component of the centroidal momentum matrix, and \(\mathbf{h}^{des}\in\mathbb{R}^3\) is the desired CAM calculated from the CP--MPC.
The third cost term regulates the optimization variable~\(\mathbf{\dot{q}}_{1}\).
Each term is weighted by its respective positive diagonal weighting matrix:~\({\mathbf{w}_{\mathbf{J}}}\), \({\mathbf{w}_{\mathbf{A}}}\), and \({\mathbf{w}_{\mathbf{\dot{q}}}}\).
The constraints in~(\ref{eq/1st WBIK constraint}) consist of two inequality conditions, which limit the joint velocity~\(\mathbf{\dot{q}}_1\) and joint position~\(\mathbf{q}_1\) within their respective boundaries. 
The vectors $\overline{\mathbf{\dot{q}}}_{1}\in \mathbb{R}^{n+6}$ and $\underline{\mathbf{\dot{q}}}_{1}\in \mathbb{R}^{n+6}$ represent upper and lower bounds for joint velocities, respectively.
$\overline{\mathbf{{q}}}_{1}\in \mathbb{R}^{n+6}$ and $\underline{\mathbf{{q}}}_{1}\in \mathbb{R}^{n+6}$ represent upper and lower bounds for joint positions, where $\mathbf{K}_q$ is a positive diagonal matrix.

In the second hierarchy, the cost function in (\ref{eq/2nd WBIK cost}) consists of two cost terms, similar to those in~(\ref{eq/1st WBIK cost}).
The first cost term serves as the desired velocity command tracking control, required to maintain the initial pose such as pelvis orientation, chest orientation, and both hand poses. 
% The second term regulates the optimization variable~\(\mathbf{\dot{q}}_{2}\).
The constraints in (\ref{eq/2nd WBIK constraint}) include three inequality conditions and one equality condition.
The first and second constraints are set to ensure a strict task hierarchy, where lower-priority tasks do not affect higher-priority tasks.
If $\mathbf{\boldsymbol{\varepsilon}} = 0$, CAM control has higher priority than the initial pose recovery since the optimal solution of (\ref{eq/2nd WBIK cost}) does not affect the result from the first QP (\ref{eq/1st WBIK cost}).
% However, the transition between CAM control in (\ref{eq/1st WBIK cost}) and initial pose recovery in (\ref{eq/2nd WBIK cost}) results in decreased balance performance as both cost terms conflict with each other.
However, it is impossible to generate the return motion without violating the prior CAM control calculated from the higher-level hierarchy.
To address this problem, a tolerance $\mathbf{\boldsymbol{\varepsilon}}$ was introduced in \cite{kim2022humanoid} to relax the strict hierarchy by a specified magnitude.
The tolerance $\mathbf{\boldsymbol{\varepsilon}}$ is adjusted online according to the 2-norm of the desired CAM.
% By setting the tolerance as a function of the 2-norm of the desired CAM, $\boldsymbol{\varepsilon}(\mathbf{h}^{des})$, CAM control can be prioritized over initial pose recovery control based on the magnitude of the desired CAM.
% Lastly, the joint PD controller calculates~the joint torques required to track the desired joint positions and velocities.
{Lastly, the optimized $\mathbf{\dot{q}}$, excluding the underactuated virtual joint, becomes $\mathbf{\dot{q}}^{des}$, while $\mathbf{{q}}^{des}$ is obtained by integrating $\mathbf{\dot{q}}^{des}$ from the current time step.} %Both $\mathbf{{q}}^{des}$ and $\mathbf{\dot{q}}^{des}$ are then provided as inputs to the joint PD controller.}

{Note that in our framework, we assigned a higher weighting parameter to maintain the initial pelvis pose compared to the hand and chest pose maintenance, which is a design choice to limit excessive pelvic movement under small disturbances.}
}
% Figure environment removed
\section{Results of Simulations and Experiments}
\label{Section/Results} 
\subsection{System Overview}
\label{Section/Results/System overview}
% In this section, a comprehensive system overview for both the simulations and real robot experiments is presented. The humanoid robot, TOCABI, utilized in the simulations and experiments comprises a total of 33 degrees of freedom (DOFs): 16 for the arms, 12 for the legs, 3 for the waist, and 2 for the neck. Its physical dimensions are approximately 1.8 m in height, weighing around 100 kg, and with a foot size of 15 cm $\times$ 30 cm. The upper body actuators are comprised of Parker's BLDC motors and harmonic gears, and the lower body actuators are composed of Kollmorgen's BLDC motors and harmonic gears. 
This section provides a system overview for both simulations and real robot experiments. The humanoid robot, TOCABI, utilized in the simulations and experiments comprises a total of 33 degrees of freedom (DOFs): 16 for the arms, 12 for the legs, 3 for the waist, and 2 for the neck. Its physical dimensions are approximately 1.8 m in height, weighing around 100 kg, and with a foot size of 15 cm $\times$ 30 cm. The upper-body actuators comprise Parker BLDC motors with harmonic gears, while the lower-body actuators incorporate Kollmorgen BLDC motors with harmonic gears.
The current control of the robot is performed by Elmo Motion Control's Gold Solo Whistle servo controller, and the communication between the servo controller and the main PC is achieved via EtherCAT communication.
MicroStrain's 3DM-GX5-25 IMU is attached to the pelvis, and ATI's mini85 F/T sensor is mounted on each foot. The algorithmic operation and torque command frequency of the robot are set to 2 kHz, and the walking pattern generator and CP--MPC operate at 50 Hz, due to the high computational load involved, through parallel threads. The walking pattern generator and CP--MPC have respective MPC time horizons of 2.5 s and 1.5 s. It should be noted that the positive direction of the x-axis represents the robot's forward direction, while the positive direction of the z-axis is the opposite direction of the gravity vector. The simulator employed in this study is the MuJoCo simulator \cite{todorov2012mujoco}. {The qpOASES \cite{ferreau2014qpoases} was employed to solve QP optimization problems, while RBDL \cite{felis2017rbdl} was utilized for calculating the kinematics and dynamics of the robot.} 

% Figure environment removed

\subsection{{Simulations to Validate Robustness of the Proposed Method under External Forces}}
\label{Section/Results/Simulation/extforce}

Simulations were conducted to evaluate the balancing capability of the robot against external forces using the proposed method. The simulations involved a scenario where the robot walked in the place and external forces are applied to the robot's pelvis link in the x- and y-directions during an SSP (left foot support). The step duration was configured to consist of an SSP of 0.6 s and a DSP of 0.3 s. %Furthermore, adjustments of the footstep position and step time are not allowed 0.1 s before the end of SSP to prevent jerky motions in the swing leg.

In the first simulation, an external force of 240 N was applied to the robot's pelvis in the negative x-direction for 0.2 s. Fig. \ref{figure/Simulation_1_Snapshot_Graph_}(a) shows snapshots and data of the robot's response to the external force. 
% The snapshots illustrates the overall balancing process. 
% \textcolor{blue}{The robot maintains balance by employing ZMP control (ankle strategy), CAM control (hip strategy), and performing back steps (stepping strategy).}
{The snapshots illustrate the overall balancing process, where the robot maintains balance through a combination of ZMP control (ankle strategy), CAM control (hip strategy), and back steps (stepping strategy).}

{The graph in Fig. \ref{figure/Simulation_1_Snapshot_Graph_}(a) represents the outputs of the CP--MPC and stepping controller in the x-direction. 
At approximately 5.9 s, the robot was pushed backward, causing the CP error $\xi_{err,x}$ (green line) to increase in the negative x-direction due to the external force. To reduce the CP error, the desired ZMP $z_x^{des}$ (red line) is generated by the CP--MPC. Despite the attempts to reduce the CP error through the desired ZMP, it remained high due to the significant external force and the ZMP constraint of -9 cm. As a result, the CP error increased and reached up to -18.3 cm. Then, footstep adjustment $\Delta f_{x,1}$ (blue line) was generated to relax the ZMP constraint and the desired centroidal moment $\tau_y^{des}$ (yellow line) was also generated to further reduce the CP error. 
The desired ZMP was generated up to the relaxed ZMP constraint until approximately 7.6 s, while the centroidal moment was generated up to the limit of -15 Nm until approximately 7.2 s.
Meanwhile, the stepping controller performed a total of two back-steps (magenta and cyan line) based on $\Delta f_{x,1}$, moving approximately -23.5 cm in conjunction with an optimal step time $T$ (orange line). As a result, the CP error is reduced to approximately zero, and the robot maintains a balance against the external force.}

In the following simulation, an external force of 500 N was applied to the robot's pelvis in the negative y-direction for 0.2 s. Fig. \ref{figure/Simulation_1_Snapshot_Graph_}(b) presents snapshots and y-direction data of the robot's response to the external force. Similar to the previous simulation, the robot maintained balance by stepping in the direction of the external force while performing ZMP control and CAM control. 
% Similar to the previous simulation, the robot achieved balance through the three balance strategies. 

Fig. \ref{figure/Simulation_1_Snapshot_Graph_}(b) represents the robot's data, including the CP--MPC and stepping controllers' y-direction outputs. 
{At approximately 5.9 s, the external force was applied to the robot in the negative y-direction, leading to a significant deviation in the CP error $\xi_{err,y}$. The CP--MPC attempted to decrease the increasing CP error by generating the desired ZMP $z_y^{des}$ up to the ZMP constraint. At approximately 6 s, footstep adjustment of -8.6 cm $\Delta f_{y,1}$ was generated to relax the ZMP constraint, while a desired centroidal moment $\tau_x^{des}$ of 15 Nm was also generated to reduce the CP error. Based on the generated $\Delta f_{y,1}$, the stepping controller adjusted the swing foot trajectory and reduced the step time $T$ to allow for stepping at approximately 6.1 s. Subsequently, continuous CP control using each strategy reduced the CP error from approximately -25.7 cm initially, gradually decreasing until the robot was able to maintain balance against external forces.}

\subsection{{Simulation to Validate Robustness of the Proposed Method on Uneven Terrain}}
\label{Section/Results/Simulation/uneven}
% Although the robustness of the proposed method was validated against external forces, it is imperative to ensure its ability to maintain balance against disturbances arising from uneven terrain. Therefore, in this simulation, the robot walks over objects placed on the ground, demonstrating the performance of the algorithm on uneven terrain. 
In this simulation, the robot walks over objects placed on the ground to validate the balancing performance of the proposed method on uneven terrain. Four objects, each with a length and width of 5 cm, were positioned in front of the robot with a distance of 30 cm between them. To make the robot step on each object alternately, two objects were placed on the left and two on the right with respect to the robot's center. 
The thickness of the objects ranged from 1.0 cm to 2.5 cm with an increase of 0.5 cm. The robot walked a distance of 2.0 m with a step length of 0.1 m, and the step duration consisted of 0.6 s of SSP and 0.3 s of DSP. 
%In the second row of Fig. \ref{figure/Simulation_1_2_uneven_Snapshot_Graph_}, the left y-axis of the graph represents the displacements of variables related to ZMP. Conversely, the right y-axis, indicated by the purple color, displays the CP error $\xi_{err}$ and footstep adjustment $\Delta \mathbf{f}_{1}$. This arrangement is intended to enhance the visibility of variations in CP error and footstep adjustment. 
% In the second row of Fig. \ref{figure/Simulation_1_2_uneven_Snapshot_Graph_}, the right y-axis, highlighted in purple, depicts the CP error $\xi_{err}$ and footstep adjustment $\Delta \mathbf{f}_{1}$. This arrangement is intended to enhance the visibility of variations in $\xi_{err}$ and $\Delta \mathbf{f}_{1}$.

Fig. \ref{figure/Simulation_1_2_uneven_Snapshot_Graph_} shows simulation snapshots and the corresponding output data. 
%The snapshots provide a reaction of the robot to each object before and after stepping on it. 
In the snapshots, the robot was able to navigate forward while stepping on the objects in its path. The first and second objects were overcome using ZMP and stepping control (ankle and stepping strategies). For the third and fourth objects, CAM control (hip strategy) was additionally employed to navigate past them. The analysis of this process is explained based on the data presented in the graphs. 
Note that in the second row of Fig. \ref{figure/Simulation_1_2_uneven_Snapshot_Graph_}, the purple right y-axis highlights the CP error $\xi_{err}$ and footstep adjustment $\Delta \mathbf{f}_{1}$ to enhance the visibility of their variations.
% In the second row of Fig. \ref{figure/Simulation_1_2_uneven_Snapshot_Graph_}, the purple right y-axis highlights the CP error  $\xi_{err}$ and footstep adjustment $\Delta \mathbf{f}_{1}$ to improve visibility of their variations.

{First, at approximately 7.2 s, which is the start of the 6th DSP, the robot steps on the object with a thickness of 1.0 cm diagonally in the y-direction, causing external disturbances.
However, the robot was not significantly disturbed by the object of relatively low thickness, and the reduction in CP error was solely achieved by generating the desired ZMP until approximately 8.4 s when the robot passed through the object.}

{At the start of the 9th DSP, approximately 9.9 s, the robot encountered the 1.5 cm object. To overcome the resulting disturbance, the CP--MPC generated a footstep adjustment $\Delta f_{y,1}$ of 3.8 cm at approximately 10.8 s, in contrast to the initial encounter with 1.0 cm object. By utilizing $\Delta f_{y,1}$ as a nominal value in the stepping controller, the reference swing foot position $e^{ref}_{L,R,y}$ was adjusted, enabling the robot to overcome the disturbance.}

{Next, at the start of the 12th DSP, approximately 12.6 s, the robot stepped on the 2 cm object. To alleviate the CP errors, the desired ZMP was generated up to the constraint in both the x- and y-directions. Additionally, $\Delta f_{y,1}$ of -7.0 cm was generated at approximately 13.2 s and $\Delta f_{x,1}$ of -11.2 cm was also generated at 13.5 s. Notably, the large CP error in the y-direction led to a centroidal moment $\tau^{des}_x$ of 15 Nm. The robot was able to overcome disturbances by adjusting footstep position and step time to a greater extent than in the second disturbance response, while also implementing CAM control.} 

{Finally, at the start of the 15th DSP, approximately 15.3 s, the robot stepped on the thickest object of 2.5 cm. 
Due to the large disturbance, the desired ZMP in the x- and y-directions was generated up to each ZMP constraint in three steps.
$\Delta f_{y,1}$ of 7.5 cm was generated at approximately 15.9 s, and $\Delta f_{x,1}$ was generated up to -11.4 cm at approximately 16.2 s. The centroidal moment $\tau^{des}_x$ in the x-direction was also generated up to the constraint. With footstep position adjustment, step time was also reduced by up to 0.46 s, allowing the robot to overcome the disturbance. In the x-direction, due to the larger range of ZMP control and footstep adjustments compared to the y-direction, only a small amount of centroidal moment $\tau^{des}_y$ was generated in overcoming all objects.}

In conclusion, this simulation showed that appropriate strategies were executed to mitigate disturbances of various magnitudes caused by uneven terrain.
% \textcolor{red}{The sequence in our approach, where ankle and stepping strategies are employed first, followed by the use of the hip strategy, resembles the preferred order of balance strategies in human behavior \cite{park2004postural, mcilroy1993changes, mansfield2011training}}.

\subsection{{Simulation to Evaluate Balancing Performance based on the Different Combinations of Each Balance Strategy}}
Comparative simulations were conducted to evaluate how the balancing performance varies based on different combinations of balancing strategies in response to disturbances. Presented below is a summary of the implementation methods employed to combine each strategy {(see Fig. \ref{figure/Simulation1_DP})}.
\begin{itemize}
    \item Method 2) is composed of a combination of ZMP control and stepping control (footstep position and step time adjustment), excluding the CAM control. To achieve this, %the prediction model of CP--MPC in (\ref{eq/discrete CP-CMP}) was modified to include LIPM dynamics but not LIPFM, and 
    all terms related to the centroidal moment $\bm{\uptau}$ in (\ref{eq/CP-MPC cost function}) and (\ref{eq/CP-MPC constraint}) were excluded. %, as well as the CAM controller. %All other controllers were kept the same. 
    \item Method 3) has the same structure as method 2), but it does not consider adjusting the step time $T$ in the stepping controller. Therefore, only the next footstep position $\mathbf{f}$ is adjusted in the stepping controller.
    \item Method 4) is a combination of ZMP control and CAM control, excluding the stepping control. To achieve this, all terms related to $\Delta \mathbf{F}$ in (\ref{eq/CP-MPC cost function}) and (\ref{eq/CP-MPC constraint}) were excluded, and the stepping controller was also excluded.
    \item Method 5) only performs ZMP control. All terms related to $\uptau$ and $\Delta \mathbf{F}$ in (\ref{eq/CP-MPC cost function}) and (\ref{eq/CP-MPC constraint}) were excluded, as well as the stepping controller.
\end{itemize}
% Figure environment removed 
In order to analyze the robustness of each method against disturbances from various directions, the maximum external impulse that the robot could withstand was analyzed by varying the direction of the external force in increments of 30 deg.
Similar to previous simulations, an external force was applied to the pelvis of the robot for 0.2 s while the robot walked in place. In Fig. \ref{figure/Simulation1_DP}, the maximum external impulse that the robot can withstand is represented by the form of a polygon, which is referred to as the disturbance polygon (DP) in this paper. The direction of the external force vector starts from the robot and points outward, and a force of 90 deg represents a force that is directed from back to front (+x-axis).

As expected, method 5), which only tracks the desired ZMP, exhibited the lowest balancing performance compared to other methods. This is because CP control is achieved only through ZMP control, and ZMP control performance is limited within the support polygon.
Method 4), which tracks the desired ZMP and centroidal moment, extended the range of the control input that controls CP beyond the support polygon compared to method 5), resulting in a {15.8 $\%$} increase in the average maximum external impulses that the robot can withstand.
% Method 3), which tracks the desired ZMP and adjusts the footstep position, significantly expands the maximum endurable impulses compared to 4) and 5) at 0, 30, and 330 degrees, where the support area can be greatly widened through footstep position adjustment. 
Method 3), which tracks the desired ZMP and adjusts the footstep position, significantly increases the maximum endurable impulses compared to methods 4) and 5) at 0, 30, and 330 degrees, where the support area is greatly expanded through footstep position adjustments.
% In cases where the footstep position adjustment is limited due to self-collision, i.e. near 180 degrees, the contribution of the footstep position adjustment to the robustness decreases compared to the other directions and becomes similar to that of method 4). 
However, when footstep position adjustment is restricted due to self-collision (i.e., near 180 degrees), its contribution to robustness decreases relative to other directions, becoming comparable to that of method 4).
In method 2), which simultaneously tracks the desired ZMP and adjusts footstep position and step time, it was possible to adjust the footstep position more quickly in response to disturbances compared to method 3). In particular, this led to a noticeable improvement in balancing performance, especially at 0, 30, and 330 degrees. %Additionally, the average maximum external impulses that the robot could withstand increased by approximately \textcolor{red}{15.5 $\%$} compared to method 3). 
Our proposed method, which combines all strategies, exhibited superior balancing performance across all directions compared to other sub-combinations. This is because all limbs of the robot (supporting leg, upper body, and swing leg) participated in CP control through the three balance strategies.
As a result, our method led to an increase of {14.5 $\%$, 27.7 $\%$, 48.8 $\%$, and 72.2 $\%$} in the average maximum external impulses that the robot could withstand compared to methods 2), 3), 4), and 5), respectively.
 
\subsection{{Simulation to Validate the Effectiveness of the Variable Weighting Method}}
\label{Section/Results/Simulation/VariableWeighting}
% Figure environment removed 
% Figure environment removed 
In order to evaluate the effectiveness of the variable weighting method in the presence of disturbance, a comparative simulation with the constant weighting method was performed. The simulation setup was identical to that in Sec \ref{Section/Results/Simulation/extforce}, where an external force was applied to the robot's pelvis in the negative x-direction during walking in place. %In this simulation, only the effects of external forces in the x-direction were analyzed, as the results in the y-direction were found to be similar.
{In this simulation, ZMP control and CAM control without stepping control were employed, and only the effects of external forces in the x-direction were analyzed. In the constant weighting method, $\mathbf{w}_{\tau}$ is set equal to $\mathbf{w}_{\tau,max}$ in the variable weighting method, aiming to restrict CAM generation for small disturbances, as in the variable weighting method.} 
% Fig. \ref{figure/Simulation_2_graph} presents the results of comparing the performance of the two methods. The first graph shows the relationship between CP error and the magnitude of the current delta ZMP $|\Delta z_x|$, indicating an increasing trend in delta ZMP as the CP error increases. The delta ZMP refers to the additional ZMP input needed to control the CP error, which is calculated as the difference between the desired ZMP and the reference ZMP, as explained in Section \ref{Section/Variable Weighting}.The second graph shows how the weighting parameter varies with the series of the delta ZMP inputs $|\Delta \mathbf{Z}_x|$ using a color-map. The third and fourth graphs compare the centroidal moment and CAM generated by each weighting method.
% 
% In the first graph of Fig. \ref{figure/Simulation_2_graph}, both methods exhibit an increase in CP error (green and black dotted line) when the external force is applied to the robot at approximately 5.9 s. As a result, the magnitude of delta ZMP input (red and blue dotted line) also increases. Note that the weighting parameters changes according to $N$ series of delta ZMP inputs, but the graph depicts only the first element of delta ZMP inputs to show the overall behavior with respect to CP error. 
% 
% The second graph depicts the time sequence of $N$ weighting parameters based on the magnitude of $N$ series of delta ZMP inputs $|\Delta \mathbf{Z}_x|$ over the time horizon (vertical axis). In the variable weighting method, the graph appears white when no disturbance is applied to the robot. 
% However, as the delta ZMP increases due to disturbances after approximately 6 s, the weighting parameters of the CAM damping cost decrease, causing the color-map to turn dark.
% In contrast, the constant weighting method always has a constant weighting value regardless of delta ZMP inputs.
% 
% As evident from the third and fourth graphs, the difference between the two methods impacts the generation $\tau_y^{des}$ for disturbances. As a result, it directly affects the desired CAM $h^{des}_y$, which is generated by integrating $\tau_y^{des}$.
% Although the delta ZMP was almost the same (as shown in the graph of the first row), using constant weighting method resulted in the insufficient generation of CAM compared to variable weighting method. This led to the robot falling over after approximately 6.8 s when the constant weighting is used.   

Fig. \ref{figure/Simulation_2_graph} presents the results of comparing the performance of the two methods. The first graph shows the relationship between CP error and the magnitude of the current delta ZMP $|\Delta z_x|$. The delta ZMP refers to the additional ZMP input needed to control the CP error, which is calculated as the difference between the desired ZMP and the reference ZMP, as explained in Section \ref{Section/Variable Weighting}. When the external force is applied to the robot at approximately 5.9 s, both methods exhibit an increasing trend in CP error (green line and black dotted line), leading to an increase in the magnitude of delta ZMP input (red line and blue dotted line).  
Note that the weighting parameters changes according to $N$ series of delta ZMP inputs, but the graph depicts only the first element of delta ZMP inputs to show the overall behavior with respect to CP error. 

The second graph depicts the variation of $N$ weighting parameters (vertical axis) based on the magnitude of delta ZMP inputs $|\Delta \mathbf{Z}_x|$ using a color-map. In the variable weighting method, the graph appears white when no disturbance is applied to the robot. 
However, as the delta ZMP increases due to disturbances after approximately 6 s, the weighting parameters of the CAM damping cost decrease, causing the color-map to turn dark.
In contrast, the constant weighting method always has a constant weighting value regardless of delta ZMP inputs.

As evident from the third and fourth graphs, the difference between the two methods impacts the generation $\tau_y^{des}$ for disturbances. As a result, it directly affects the desired CAM $h^{des}_y$, which is generated by integrating $\tau_y^{des}$.
Although the delta ZMP was almost the same (as shown in the graph of the first row), using constant weighting method resulted in the insufficient generation of CAM compared to the variable weighting method. This led to the robot falling over after approximately 6.8 s when the constant weighting is used.
% % Figure environment removed 

%Similar to the simulation in \ref{Section/Results/Simulation/extforce}, the performance of the variable weighting method was tested against various directions of disturbances, and the results were illustrated in Fig. \ref{figure/Simulation_2_DP}. The use of the variable weighting method led to a \textcolor{red}{6.8 $\%$} increase in the average maximum external impulses that the robot can withstand compared to the constant weighting method.
{The performance of the variable weighting method was compared with that of the constant weighting method using DP, and the results were illustrated in Fig. \ref{figure/Simulation_2_DP}. }
{With the variable weighting method, the average maximum external impulses that the robot can withstand increased by 9.88 $\%$ and 9.71 $\%$ in the ZMP+CAM method and the ZMP+CAM+Stepping method, respectively, compared to the constant weighting method.}

% \subsection{Simulation to Validate the Suitability of Parameter Decision for Stepping Controller}
\subsection{Simulation to evaluate walking performance based on different parameter selections of the stepping controller}
\label{Section/Results/Simulation/SteppingController}
In this section, simulations were conducted to analyze the impact of parameter selection in the stepping controller on walking stability. Specifically, the CP tracking error $\bm{\xi}_{err}$ and CP offset error $\mathbf{b}_{err}$ were analyzed according to the two parameters (nominal CP offset $\mathbf{b}^{nom}$ and CMP parameter $\mathbf{P}_{ssp}$). Fig. \ref{figure/Simulation3_footstep} illustrates the walking scenarios conducted in this simulation. Fig. \ref{figure/Simulation3_footstep}(a) represents forward and backward walking with step length $L$, and in this case, the deviation of the step width $W$ is zero during walking. Fig. \ref{figure/Simulation3_footstep}(b) represents the lateral walking with step width deviation $W$, where the default step width $D$ is 20.5 cm. Additionally, in this case, the step length $L$ is zero during walking.

% Figure environment removed 
First, in order to analyze the effects of the nominal CP offset $\mathbf{b}^{nom}$ on walking stability, simulations were conducted in two walking scenarios depicted in Fig. \ref{figure/Simulation3_footstep}.
For comparative simulations, an analytical CP offset calculation method proposed in \cite{khadiv2016step, khadiv2020walking} was adopted. 
Here, both methods used the CMP parameter $\mathbf{P}_{ssp}$ calculated by (\ref{eq/stepping_CMP input}).
\begin{table}[]
\caption{{Analysis of CP offset errors and CP tracking errors according to the selection of nominal CP offset $\mathbf{b}^{nom}$.}}
\label{table/stepping b nom}
\centering
\resizebox{\columnwidth}{!}{%
\begin{tabular}{|l|c|cccc|}
\hline
\multicolumn{1}{|c|}{} &  & \multicolumn{4}{c|}{RMS error [cm]} \\ \cline{3-6} 
\multicolumn{1}{|c|}{\multirow{-2}{*}{Scenario}} & \multirow{-2}{*}{Method} & \multicolumn{1}{c|}{$\xi_x$} & \multicolumn{1}{c|}{$\xi_y$} & \multicolumn{1}{c|}{$b_x$} & $b_y$\\ \hline
 & \multicolumn{1}{l|}{\begin{tabular}[c]{@{}c@{}} {Analytical CP offset} \\ {calculation}  \end{tabular}} & \multicolumn{1}{c|}{1.27} & \multicolumn{1}{c|}{3.24} & \multicolumn{1}{c|}{4.04} & 6.47 \\ \cline{2-6} 
\multirow{-2}{*}{\begin{tabular}[c]{@{}l@{}}(a) Forward and \\ backward walking\end{tabular}} & Our approach & \multicolumn{1}{c|}{\cellcolor[HTML]{D7D7D7}0.81} & \multicolumn{1}{c|}{\cellcolor[HTML]{D7D7D7}0.98} & \multicolumn{1}{c|}{\cellcolor[HTML]{D7D7D7}1.14} & \cellcolor[HTML]{D7D7D7}0.66 \\ \hline
 & \multicolumn{1}{l|}{\begin{tabular}[c]{@{}c@{}} {Analytical CP offset} \\ {calculation} \end{tabular}} & \multicolumn{1}{c|}{0.47} & \multicolumn{1}{c|}{4.06} & \multicolumn{1}{c|}{1.37} & 8.23 \\ \cline{2-6} 
\multirow{-2}{*}{(b) Lateral walking} & Our approach & \multicolumn{1}{c|}{\cellcolor[HTML]{D7D7D7}0.24} & \multicolumn{1}{c|}{\cellcolor[HTML]{D7D7D7}1.40} & \multicolumn{1}{c|}{\cellcolor[HTML]{D7D7D7}0.14} & \cellcolor[HTML]{D7D7D7}1.05 \\ \hline
\end{tabular}%
}
\end{table}
Table \ref{table/stepping b nom} shows the root mean square (RMS) error of the CP offset and the CP tracking when different nominal CP offsets are selected in each scenario. When adopting $\mathbf{b}^{nom}$ calculated using (\ref{eq/cp_offset_bolt_x}) and (\ref{eq/cp_offset_bolt_y}), substantial CP tracking errors and CP offset errors were observed in both scenarios (a) and (b) compared to our method. As mentioned in Section \ref{Section/Stepping Controller/Characteristic}, this method assumes that both $L$ in (\ref{eq/cp_offset_bolt_x}) and $W$ in (\ref{eq/cp_offset_bolt_y}) remain constant for the current and the next step. However, as shown in Fig. \ref{figure/Simulation3_footstep}, this assumption is violated in scenario (a) where $L$ changes at each step, or in scenario (b) where $W$ changes at each step. As a result, unintended CP velocity is induced at the beginning of each step, leading to a deterioration in walking stability. %This causes significant CP tracking errors and CP offset errors during walking. 
When our method is used, the CP offset is calculated based on (\ref{eq/b_nom}) according to the walking velocity command without the constant walking velocity assumption. Consequently, the CP offset is controlled to follow the walking velocity command, allowing the robot to minimize CP tracking error and start the next step with a small CP error.

\begin{table}[t]
\caption{{Analysis of CP offset errors and CP tracking errors according to the selection of CMP parameter $\mathbf{P}_{ssp}$.}}
\label{table/stepping cmp}
\centering
\resizebox{\columnwidth}{!}{%
\begin{tabular}{|l|c|cccc|}
\hline
\multicolumn{1}{|c|}{} &  & \multicolumn{4}{c|}{RMS error [cm]} \\ \cline{3-6} 
\multicolumn{1}{|c|}{\multirow{-2}{*}{Scenario}} & \multirow{-2}{*}{Method} & \multicolumn{1}{c|}{$\xi_x$} & \multicolumn{1}{c|}{$\xi_y$} & \multicolumn{1}{c|}{$b_x$} & $b_y$\\ \hline
 & \multicolumn{1}{l|}{\begin{tabular}[c]{@{}c@{}}{Fixed CMP with} \\ point foot assumption\end{tabular}} & \multicolumn{1}{c|}{1.02} & \multicolumn{1}{c|}{1.35} & \multicolumn{1}{c|}{1.66} & 1.22 \\ \cline{2-6} 
\multirow{-2}{*}{\begin{tabular}[c]{@{}l@{}}(a) Forward and \\ backward walking\end{tabular}} & Our approach & \multicolumn{1}{c|}{\cellcolor[HTML]{D7D7D7}0.81} & \multicolumn{1}{c|}{\cellcolor[HTML]{D7D7D7}0.98} & \multicolumn{1}{c|}{\cellcolor[HTML]{D7D7D7}1.14} & \cellcolor[HTML]{D7D7D7}0.66 \\ \hline
 & \multicolumn{1}{l|}{\begin{tabular}[c]{@{}c@{}}{Fixed CMP with} \\ point foot assumption \end{tabular}} & \multicolumn{1}{c|}{0.43} & \multicolumn{1}{c|}{1.74} & \multicolumn{1}{c|}{0.38} & 1.60 \\ \cline{2-6} 
\multirow{-2}{*}{(b) Lateral walking} & Our approach & \multicolumn{1}{c|}{\cellcolor[HTML]{D7D7D7}0.24} & \multicolumn{1}{c|}{\cellcolor[HTML]{D7D7D7}1.40} & \multicolumn{1}{c|}{\cellcolor[HTML]{D7D7D7}0.14} & \cellcolor[HTML]{D7D7D7}1.05 \\ \hline
\end{tabular}%
}
\end{table}

Next, simulations were performed to analyze the impact of the CMP parameter $\mathbf{P}_{ssp}$ on walking stability for both scenarios (a) and (b). Here, our CP offset calculation method (\ref{eq/b_nom}) was used.
For the comparison simulations, the fixed CMP that assumed the point foot \cite{khadiv2016step, khadiv2020walking, kim2023foot} was compared with our method, and the results are presented in Table \ref{table/stepping cmp}. In both scenarios (a) and (b), the fixed CMP under the point foot assumption resulted in larger errors compared to our method. 
% The reason is that the point foot assumption fails to accurately reflect the CMP position of the robot with large feet, thus ignoring the causal relationship between the predicted CP end-of-step and the CMP, which operates based on LIPFM dynamics. As a result, the CP end-of-step is inaccurately predicted, and CP offset control based on this CP end-of-step leads to a deterioration of walking stability. On the other hand, our approach calculates the CMP parameter based on the control inputs of CP--MPC in (\ref{eq/stepping_CMP input}), which accurately reflects the robot's CMP, enabling accurate prediction of CP end-of-step.
{The reason is that the point foot assumption assumes the CMP to act at the center of the foot during SSP, which fails to accurately reflect the CMP position for the robot with large feet. As a result, the CP end-of-step is inaccurately predicted, and CP offset control based on this prediction leads to a deterioration of walking stability.
On the other hand, our approach calculates the CMP parameter by accurately approximating series of CP--MPC control inputs during the SSP to a single point. This enables the accurate prediction of CP end-of-step by reflecting the tendencies of the CMP acting on the robot.}

In conclusion, it is important to adopt appropriate parameters for the stepping controller to enhance the overall stability of walking.

% Therefore, it is important to adopt the parameters of the stepping controller from CP--MPC to operate the stepping controller parallel with CP--MPC.  

\subsection{{Simulations to Compare Robustness against Disturbances with QP-based CP controller [39]}}
\label{Section/Results/Simulation/Jeong}

In this section, simulations were conducted to compare the robustness against disturbances between the proposed method and state-of-the-art QP-based CP controller \cite{jeong2019robust}. For the comparative simulations, the algorithm proposed by \cite{jeong2019robust} was implemented in our walking control framework (see Fig. \ref{figure/overallframe}). In the implemented algorithm, the QP-based CP controller replaced our CP--MPC and stepping controller, while all other local controllers and trajectory planners described in Section \ref{Section/Overall Framework} remained unchanged.

The reasons for selecting the QP-based {CP controller} \cite{jeong2019robust} as a suitable comparison for our method are the following. First, our method and the QP-based {CP controller} are based on the same CP dynamics of the LIPFM and integrate ankle, hip, and stepping strategies that include step time optimization. Second, the QP-based {CP controller \cite{jeong2019robust}} is considered state-of-the-art research in this field. 

% While the QP-based method only considers the robot's CP for the current step duration, our CP--MPC takes into account not only the current states but also the future constraints and future states of the robot.
The QP formulation proposed by \cite{jeong2019robust} is defined as follows:
\begin{align}
\label{eq/jeong's method Costfunction}
& \underset{\mathbf{f}^{}, \gamma, \bm{\tau}, \mathbf{b}}{\text{min}} & & \mathbf{w}_{f}\| \mathbf{f}_{err}\|^{2} + w_{\gamma}(\gamma^{}_{err})^{2} + \mathbf{w}_{{\tau}}\|\bm{\tau}+\mathbf{K}_{d}\mathbf{h}\|^{2} \nonumber\\ 
& & +&\mathbf{w}_{b}\|\mathbf{b}_{err}\|^{2}  \\ \label{eq/jeong's method constraint}
& \text{s. t.} & & \mathbf{f}_{err} + \mathbf{b}_{err} - \bm{\xi}^{ref}e^{-\omega t}\gamma_{err} -(1-e^{-\omega t}\gamma^{'}){\frac{\bm{\tau}_{err}}{mg}}\nonumber\\ 
& & &=(\bm{\xi}_{err}-\Delta\mathbf{z}_{})e^{-\omega t}\gamma_{err}  + \Delta\mathbf{z}_{}  
\end{align}
where the subscript $err$ signifies the difference between the optimization variables and the pre-designed nominal variables. The delta ZMP $\Delta \mathbf{z}=[\Delta z_x\,\Delta z_y$] denotes the difference between the desired ZMP for controlling the CP and the pre-designed reference ZMP. The cost function (\ref{eq/jeong's method Costfunction}) comprises the step position error term, step time error term, damping term of the CAM, and CP offset error term. The equality constraint in (\ref{eq/jeong's method constraint}) is derived from the error dynamics of the CP end-of-step dynamics in LIPFM. Detailed derivation and explanations for each equation are provided in \cite{jeong2019robust}. 

% Figure environment removed
The QP formulation in (\ref{eq/jeong's method Costfunction}) and (\ref{eq/jeong's method constraint}) addresses the ankle strategy through delta ZMP $\Delta \mathbf{z}\in\mathbb{R}^{2}$, the hip strategy through centroidal moment $\bm{\tau}\in\mathbb{R}^{2}$, and the stepping strategy through footstep position $\mathbf{f}\in\mathbb{R}^{2}$ and step time $\gamma$.
However, due to the variable coupling between $\Delta \mathbf{z}$ and $\gamma$, as well as between $\gamma$ and $\bm{\tau}$, the original equation of (\ref{eq/jeong's method constraint}) resulted in a non-linear constraint. To address this issue in \cite{jeong2019robust}, $\Delta \mathbf{z}$ is considered as a constant using the pre-computed value from the {instantaneous CP end-of-step controller} proposed in \cite{englsberger2011bipedal} without optimization. Furthermore, to avoid variable coupling between $\bm{\tau}$ and $\gamma$, a constant $\gamma'$ was used, which corresponds to the $\gamma$ from the previous control cycle. In contrast, our approach optimizes ZMP, centroidal moment, and footstep position simultaneously using CP--MPC, and the step time is separately optimized in the stepping controller without any constant assumptions.
 % Figure environment removed

Simulations were conducted to compare the robustness of the two algorithms against external forces. The external forces were applied to the robot during walking in place and forward walking, and the maximum impulse that the robot could withstand was analyzed using DP. The weighting parameter in (\ref{eq/jeong's method Costfunction}) was the same as that used in \cite{jeong2019robust}. Fig. \ref{figure/Simulation4_DP} shows the DP obtained when using both methods. Overall, our method exhibited more robust balancing performance, especially at 0, 30, and 330 degrees, compared to the {QP-based CP controller}. The overall difference in balancing performance may first be attributed to the ankle strategy. 
In \cite{jeong2019robust}, the desired ZMP for instantaneous CP control is calculated as follows,
\begin{align}
\label{eq/jeong_ankle}
\mathbf{z}^{des} &= \mathbf{z}^{ref} + \Delta\mathbf{z} = \mathbf{z}^{ref} - \frac{e^{\omega(T-t)}}{1-e^{\omega(T-t)}}\bm{\xi}_{err}.
\end{align}
However, in (\ref{eq/jeong_ankle}), since the CP control feedback gain is determined based on the remaining step time $(T-t)$, the CP control performance cannot be increased as needed. Next, as analyzed in Section \ref{Section/Results/Simulation/extforce}, stepping control was the most helpful in withstanding external forces at 0, 30, and 330 degrees. 
{The QP-based CP controller \cite{jeong2019robust} considers only the robot's CP for the current step duration, performing instantaneous control to reduce the current CP error. In contrast, our CP--MPC not only considers the current state but also extends its consideration to future constraints and the future state of the robot. This enables proactive stepping control against disturbances by anticipating future CP errors. }%This enables proactive stepping control against disturbances by taking into account not only the current CP error but also anticipating future CP errors.}
Therefore, the response to disturbances through stepping control was slower compared to our method.
The start time of the stepping control for each method was measured 10 times when the robot was subjected to the maximum external impulse that our method could withstand at 0, 30, and 330 degrees (in Fig. \ref{figure/Simulation4_DP}). Our method, on average, outpaced the {QP-based CP controller} by 0.078 s, 0.086 s, and 0.084 s for each direction, respectively. 
Considering the SSP time (0.6 s), this means that stepping control in our method was performed 13$\%$, 14.3$\%$, and 14$\%$ faster than the {QP-based CP controller}, respectively. 
In conclusion, when using our method compared to the {QP-based CP controller} \cite{jeong2019robust}, the average maximum external impulses that the robot could withstand increased by {27.3 $\%$ for walking in place and 31.8 $\%$ for forward walking.} %The video of these comparison simulations can be found in the supplementary material. 
% While the QP-based method only considers the robot's CP for the current step duration, our CP--MPC takes into account not only the current states but also the future constraints and future states of the robot.
% \subsection{Simulation to Validate Optimality of Hierarchical structure of CP--MPC and Stepping controller}
% Figure environment removed
\subsection{{Real Robot Experiments to Validate Robustness of the Proposed Method under External Forces}}
\label{Section/Results/Experiment}

% In order to ensure the robustness of the proposed method to disturbances, it is important to validate its balancing performance not only through simulation but also in real robot experiments. To this end, 
Real robot experiments were conducted where the robot was subjected to external forces in both x- and y-directions, replicating the scenarios in the simulation presented in Section \ref{Section/Results/Simulation/extforce}. {The external force was applied directly to the robot by a human using a tool equipped with an F/T sensor to measure the timing and magnitude of the force.}
The step duration includes an SSP of 0.6 s and a DSP of 0.3 s which is consistent with the simulation setup. The control parameters used in both the simulation and experiment are identical, except for the constraints of the centroidal moment (as shown in Table. \ref{table/Parameters}). %The difference in centroidal moment constraints arises from the conservative assignment of upper body motion constraints in real robot experiments. This is done to prevent self-collisions with equipment attached to the robot when the upper body moves. 
The difference in centroidal moment constraints results from conservative upper body motion limits in real robot experiments, applied to prevent collisions with equipment attached to the robot.
Consequently, this implies that if the low-level control of the real robot is already adjusted to suit the hardware, there is no need for tuning of control parameters during sim-to-real implementation.

{In the first experiment, the robot was subjected to an external push of 44.98 Ns in the negative x-direction while in the right foot support. The experimental snapshots and the corresponding data are shown in Fig. \ref{figure/Experiment_1_Snapshot_Graph}(a). 
The snapshot illustrates the robot maintaining balance against external forces. As in Section \ref{Section/Results/Simulation/extforce}, the robot successfully overcame the external force using a combination of ZMP control, CAM control, and stepping control. At approximately {6.5 s}, when the external force is applied to the robot, the CP error $\xi_{err,x}$ (green line) increases in the negative x-direction, and the desired ZMP $z^{des}_x$ (red line) is rapidly generated in the direction that reduces the CP error in the CP--MPC. The desired ZMP is then constrained by the ZMP constraint of -9 cm, and the CP--MPC generates a footstep adjustment $\Delta f_{x,1}$ (blue line) with a peak value of approximately {-11.5 cm to relax the ZMP constraint at approximately 7.0 s}. Additionally, the desired centroidal moment $\tau^{des}_y$ (yellow line) is generated up to -7.0 Nm to further reduce the CP error at approximately {7.1 s.} Despite these efforts, the robot was unable to overcome the strong disturbance. As a result, CP--MPC continuously generated footstep adjustment to reduce the CP error while the desired ZMP was generated up to the constraint until {approximately 8.5 s}. In conclusion, a total of two back-step controls (magenta and cyan lines) were performed, involving a movement of {approximately -15 cm}, along with step time optimization $T$ (orange line) and the robot maintained its balance. In contrast, when the QP-based CP controller proposed in \cite{jeong2019robust} was used in the experiment, despite an external push of 38.7 Ns in the negative x-direction, the CP of the robot could not be recovered and the robot lost its balance.}

{In the following experiment, the robot was subjected to an external push of 94.71 Ns in the negative y-direction during left foot support. The snapshots and the corresponding experimental data are presented in Fig. \ref{figure/Experiment_1_Snapshot_Graph}(b). %Similar to the previous experiment, the robot maintained its balance through the three balance strategies. 
{At approximately 7.4 s,} the external force caused significant perturbation to the robot's CP, resulting in an increase in CP error $\xi_{err,y}$ in the negative y-direction. To reduce the CP error, the desired ZMP $z^{des}_y$ was generated up to the ZMP constraint at {approximately 7.6 s.} The CP--MPC then generated a footstep adjustment $\Delta f_{y,1}$ of approximately {-8.6 cm} and a maximum centroidal moment $\tau_x^{des}$ of 7.0 Nm. Additionally, the stepping controller reduced the step time $T$ {from 0.6 s to 0.48 s.} 
Subsequently, the desired ZMP was continuously generated up to the ZMP constraint to reduce the CP error until {approximately 8.1 s. However, centroidal moment and footstep adjustment were reduced after 7.8 s,} and CP control was handled mainly through ZMP control. Finally, the robot maintained balance against external forces. 
On the other hand, when using the QP-based CP controller \cite{jeong2019robust}, the robot could not withstand an external force of approximately 75.43 Ns and lost balance.
Overall, the magnitude of external impulses applied to the robot was similar to those in the simulation, although it was hard to make them identical due to the human factor. }


\subsection{{Real Robot Experiment to Validate Robustness of the Proposed Method on Uneven Terrain}}

To validate the robustness of the proposed method in uneven terrain, a real robot experiment was conducted. In a scenario similar to the simulation in Section \ref{Section/Results/Simulation/uneven}, three objects, each with a length of 15 cm and width of 20 cm, were placed in front of the robot. The distance between the first and second objects was 25 cm, and the distance between the second and third objects was 40 cm. The objects were placed alternately on the left and right sides of the robot's center.
{The thickness of the objects ranged from 2.0 cm to 3.0 cm in increments of 0.5 cm.} The robot walked a distance of 1.3 m with a step length of 0.1 m, and the walking duration consisted of 0.6 s of SSP and 0.3 s of DSP. 
% In the second row of Fig. \ref{figure/Experiment_2_Snapshot_Graph}, the left y-axis represents the displacements of variables associated with the ZMP. In contrast, the right y-axis, denoted by the purple color, displays the CP error $\bm{\xi}_{err}$ and footstep adjustment $\Delta \mathbf{f}_{1}$.

Fig. \ref{figure/Experiment_2_Snapshot_Graph} presents experimental snapshots and the corresponding output data. The snapshots provide a reaction of the robot to each object before and after stepping on it. 
{The first object was primarily overcome using ZMP control, while the second and third objects were addressed through a combination of stepping and CAM control in addition to ZMP control.}
%Note that the robot does not step on each object just once, but rather repeatedly steps on them at various positions and angles, leading to continuous exposure to diverse external disturbances. 

{Based on the data presented in the graph, the balancing process can be explained. During forward walking, the robot stepped onto a 2.0 cm object at the start of the 5th DSP, approximately 6.3 s. However, the robot’s CP was not significantly perturbed, allowing it to overcome the obstacle primarily through ZMP control until approximately 7.5 s.}

% 9번째 DSP의 시작인 약 9초에, 로봇은 2.0 cm object를 비스듬히 밟게 되면서 첫번째 물체를 밟았을때와는 달리 로봇이 크게 기울며  CP에러가 크게 증가됩니다. 발생된 외란을 극복하기 위해, CP-MPC는 X와 Y방향으로  constraint까지 desired ZMP를 생성할 뿐만 아니라, 발 위치 조정과 Centroidal moment까지 로봇이 물체를 지나칠때까지 생성합니다. 또한 스테핑 제어기는 풋스텝 조정을 기반으로 스윙발의 위치와 스텝시간까지 조정하였습니다. 첫번째 물체를 직면했을때와는 달리 가능한 모든 전략들을 활발히 사용하며 외란를 극복하였습니다.

{At the start of the 8th DSP, approximately 9.0 s, the robot stepped at an angle on a 2.5 cm object, resulting in a significant increase in CP error in both the x- and y-directions until the 11th SSP. To reduce this error, CP--MPC not only generates the ZMP up to the ZMP constraints in both x- and y-directions but also generates footstep adjustments $\Delta f_{x,y,1}$ and the desired centroidal moment $\tau_{x,y}^{des}$. The stepping controller then adjusts the swing foot position $e^{ref}_{L,R}$ based on these footstep adjustments $\Delta f_{x,y,1}$ and the step time $T$. Unlike the disturbance from the initial object, all three balance strategies are actively employed.}
% However, during walking on the 2 cm object after 10.2 s, the robot experienced continuous disturbances such as contact impacts and late landings, leading to instability in its walking.
% To overcome this disturbance, starting from approximately 10.2 s, the desired ZMP in the y-direction began to be generated up to the ZMP constraint, and the centroidal moment $\tau_{x}^{des}$ (yellow line) in the x-direction was also generated up to 7 Nm at approximately 10.8 s. Additionally, based on the generated $\Delta f_{y,1}$ of -3 cm at approximately 10.7 s, the stepping controller adjusts the swing foot position (magenta and cyan lines) while optimizing the step time $T$ (orange line). On the other hand, to control the CP error in the x-direction, CP--MPC primarily generates the desired ZMP $z^{des}_x$ with little utilization of centroidal moment control $\tau_{y}^{des}$ or footstep position control $\Delta f_{x,1}$.

% Finally, from the beginning of the 11th DSP at 11.6 s to the end of the 14th DSP at 14.6 s, the robot consistently steps on a 3.0 cm object obliquely in the y-direction. As a result, a significant disturbance occurs, particularly in the y-direction compared to the x-direction. Initially, the robot reduces CP error using only ZMP control, but notably from around 13.5 s, the three balance strategies are simultaneously employed to control substantial y-direction CP error, ultimately enabling the robot to maintain balance.
{Immediately after the balance control performed until the 11th SSP, from the start of the 11th DSP at 11.7 s to the end of the 14th DSP at 14.6 s, the robot continuously stepped obliquely on a 3.0 cm object in the y-direction. This caused a significantly larger disturbance in the y-direction compared to the x-direction. In response, the robot actively employed the three balance strategies to control the y-direction CP error, ultimately enabling it to maintain balance.}


Similar to the results in Section \ref{Section/Results/Simulation/uneven}, where the robot primarily overcame disturbances in the x-direction using ZMP control with a long support foot length, in the y-direction, both stepping and CAM control were actively utilized. When using the {QP-based CP controller} \cite{jeong2019robust}, however, the robot could not withstand the disturbance from a 3.0 cm object and fell over.

% \section{Discussion}
% \subsection{Features of the proposed method}
% \begin{enumerate}
%     \item \textit{Computational load for CP--MPC:} abc
%     \item d 
% \end{enumerate}
% \subsection{Limitation of the Proposed method}

\section{{Discussion}}
\label{Section/Discussion}
\subsection{{Hierarchical Structure of CP--MPC and Stepping Controller}} 
\label{Section/Discussion/Hierarchical}
\subsubsection{Optimality}\label{Section/Discussion/Hierarchical/Optimality}
{In the proposed walking framework, the CP--MPC and the stepping controller are structured hierarchically. 
The CP--MPC is not re-optimized within the same control cycle based on adjusted step time $T$ and footstep position $\mathbf{f}$ from the stepping controller. In order to analyze the impact of the adjusted step time and footstep position on the optimality of CP--MPC, we re-optimized CP--MPC by incorporating the adjusted footstep $\mathbf{f}$ and step time $T$. We then analyzed the cost values (\ref{eq/CP-MPC cost function}) of CP--MPC during this re-optimization process to examine the impact of these adjustments on its optimality.}

{Fig. \ref{figure/Simulation_8_rev_cost} depicts the iteration--cost graph in a scenario where a robot experiences disturbances from the front to the back during forward walking. Fig. \ref{figure/Simulation_8_rev_cost}(a) displays the values of the average cost (\ref{eq/CP-MPC cost function}) over the iteration count during the period from 5.9 s, right after the disturbance was applied to the robot, to approximately 7.4 s when the robot stabilizes (two walking steps). Fig. \ref{figure/Simulation_8_rev_cost}(b) illustrates the value of the cost over time, where $i$ denotes the iteration count.} 

{In Fig. \ref{figure/Simulation_8_rev_cost}(a), a decreasing trend in the average cost is evident with iterative re-optimization compared to the original structure ($i=1$). After two iterations ($i=2$), the average cost is reduced by 5.1 $\%$, indicating a noticeable reduction in the second iteration. However, subsequent iterations show diminishing returns in terms of cost reduction. Furthermore, in Fig. \ref{figure/Simulation_8_rev_cost}(b), a difference in cost values between the original structure and $i\geq 2$ is observed at 6.2 s with a magnitude of around 19.1.
This implies that changes in the step time and footstep position from the stepping controller lead to variations in the cost of CP--MPC and affect the optimality of CP--MPC. Therefore, re-optimizing CP--MPC twice by reflecting the adjusted step time and footstep position yields values closer to the optimum for CP--MPC.}

{However, more than two iterations incurs a significant computational load, making it difficult to maintain a 50 Hz operation with our 1.5 s horizon setup. To achieve a two-time iteration, it is necessary to adjust the CP--MPC's operation to 40 Hz and reduce the MPC horizon to 1.0 s.}% However, more than two iterations significantly increase the computational load, making it difficult to maintain a 50 Hz operation with our 1.5 s horizon setup.

{We compared the robustness against disturbances between the two-iteration case and the original framework. 
The robot exhibited a 21.6 $\%$ and 23.2 $\%$ reduction in its maximum external force tolerance during in-place and forward walking, respectively, compared to the original control structure.}
% Figure environment removed
{Consequently, the shortened horizon and reduced control frequency negatively impacted the robot's robustness during walking. From this perspective, as suggested in \cite{scianca2020mpc}, implementing an intrinsically stable MPC that is less dependent on horizon length could be one of the promising ideas for improving the overall performance of MPC.}

\subsubsection{Adavantages and limitations}\label{Section/Discussion/Hierarchical/Advantages}
{The proposed hierarchical structure has the advantage of handling step time adjustment as a linear optimization problem, avoiding the nonlinearity associated with considering step time as an MPC variable. Previous studies have either utilized Sequential Quadratic Programming (SQP) for nonlinear MPC calculations \cite{aftab2012ankle, choe2023seamless} or linearized nonlinear constraints by defining specific bounds for the step time variable \cite{bohorquez2017adaptive}. However, optimization of step time within the MPC, as reported in \cite{khadiv2020walking}, may increase computation time and risk getting stuck in undesired local minima. The increased computational load may compromise robustness against disturbances by MPC optimizations with short horizon length and low control frequencies. In contrast, our approach defines the step time optimization as a convex problem, making it suitable for real-time applications.}

{However, unlike conventional NMPC-based methods \cite{aftab2012ankle, kryczka2015online, bohorquez2017adaptive, choe2023seamless}, our framework optimizes step time without explicitly considering future states and constraints, unlike MPC. Additionally, it may have a limitation where the CP--MPC solution may not be entirely optimal when the step time undergoes significant modifications. Despite this limitation, our framework can enhance robustness against disturbances by operating CP--MPC and the stepping controller with a higher control frequency and sufficient CP--MPC horizon.}
\subsubsection{Time horizon}
{In \cite{zaytsev2015two}, it is reported that considering a future horizon beyond two walking steps in an MPC framework is unnecessary for humanoid balancing, and balancing can be achieved within mostly two steps when disturbances occur. Based on a 0.9 s step duration, we consider a 1.5 s MPC horizon for CP--MPC and a 0.9 s horizon for the stepping controller. Similar to the horizon length suggested in \cite{zaytsev2015two}, our horizon length considers 1 to 2 walking steps for both controllers. Note that, unlike CP--MPC, the stepping controller considers only a one-step horizon, as it relies on CP end-of-step dynamics and the CP--MPC control input for a single walking step. }

\subsection{{Sequence for Ankle, Hip, and Stepping Strategies in Response to Disturbances}}
\label{Section/Discussion/Sequence}
{
% 일반적으로 로봇이 보행하는 동안 foot-ground 충격, 모델 불확실성, 시스템 노이즈 등으로부터 유발되는 작은 CP 에러는 ZMP constraint 안에서 ZMP 제어만으로 줄이는것이 가능하지만, 외부 충돌이나 uneven terrain으로 부터 발생되는 큰 CP 에러는 ZMP constraint 내에서 ZMP 제어만으로 불충분하며 hip 그리고 stepping strategy가 추가적으로 수행됩니다. CP--MPC는 (10)의 규제항 ($\mathbf{w}{\tau}, \mathbf{w}{F}$)의 가중치를 기반으로 하여 어떤 전략을 더 활발하게 활용할 지 최적의 행동을 유도합니다. 
% 힙 전략에서는 외란에 대한 균형 회복을 위한 모션을 수행한 이후 로봇의 모션을 중지하고 초기 자세로 복귀하는 과정이 있으며, 이에 따라 힙 전략은 재 사용되기까지 일정한 시간이 필요합니다. 또한 작은 CP 에러에 대해 과도한 힙전략의 사용은 부자연스러운 상체모션을 유발 할 수 있습니다. 따라서 우리는 섹션 4.C의 가변 웨이팅 방법을 통해 작은 CP 에러에 대해서는 hip strategy를 억제하고, 큰 CP 에러에 대해 hip strategy의 사용을 자유롭게하여 균형을 회복합니다. 결과적으로 외란의 크기에 따라 발목 및 스테핑 전략을 사용하여 외란에 먼저 대응하며, 외란이 더 클 경우 힙 전략까지 사용하여 CP 에러를 줄이도록 합니다. 이는 실험섹션에서 언급했듯이 ankle 전략 이후 stepping 전략을 선호하는 인간의 행동 [1-3]과 유사한 결과입니다.
During robot walking, small CP errors caused by foot-ground impacts, model uncertainties, etc., can be reduced through ankle strategy (ZMP control). 
However, larger CP errors from external impacts or uneven terrain may exceed the ankle strategy’s capability, requiring additional use of hip (CAM control) and stepping strategies (stepping control).
% However, large CP errors arising from external collisions or uneven terrain may not be adequately addressed by ankle strategy alone. In such cases, hip (CAM control) and stepping strategies (stepping control) are additionally employed. 
CP--MPC guides optimal behavior by leveraging the weights of the regulation terms in (\ref{eq/CP-MPC cost function}) ($\mathbf{w}_{\tau}, \mathbf{w}_{F}$), determining which strategy to employ more actively.}

{
% In the hip strategy, after executing motions for balance recovery, there is a process of halting the robot's motion and returning to the initial posture. Consequently, the hip strategy requires a certain amount of time to be reused.
In the hip strategy, after balance recovery motions, the robot halts and returns to its initial posture, requiring time to reuse.
Moreover, excessive use of the hip strategy for small CP errors can lead to unnatural motions. Therefore, we employ a variable weighting approach to suppress the use of the hip strategy for small CP errors and allow its flexibility for large CP errors to facilitate balance recovery. 
% As a result, depending on the magnitude of disturbances, the ankle and stepping strategies are employed initially, with the hip strategy used subsequently. This aligns with the preference for the stepping strategy following the ankle strategy in human behavior \cite{park2004postural, mcilroy1993changes, mansfield2011training}.}
As a result, the ankle and stepping strategies are employed first, with the hip strategy used as needed, similar to human behavior \cite{park2004postural, mcilroy1993changes, mansfield2011training}.}

\subsection{{CoM State Estimation}}
\label{Section/Discussion/Com}
\subsubsection{CoM position}
{The position of the CoM is estimated by utilizing global orientation information from the IMU and conducting forward kinematics calculations. } 
\subsubsection{CoM velocity}
{The linear velocity of the pelvis is initially estimated using a complementary filter, integrating the linear acceleration from the IMU and the pelvis's linear velocity that is kinematically calculated using joint angular velocities. The angular velocity of the pelvis is directly measured from the IMU. Subsequently, the velocity of the CoM is estimated by utilizing the spatial velocity of the pelvis, joint angular velocities, and the CoM Jacobian.}
% \begin{itemize}
%     \item \textcolor{red}{CoM position: The position of the CoM is estimated by utilizing global orientation information from the IMU and conducting forward kinematics calculations. } 
%     \item \textcolor{red}{CoM velocity: The linear velocity of the pelvis is initially estimated using a complementary filter, integrating the linear acceleration from the IMU and the pelvis's linear velocity that is kinematically calculated using joint angular velocities. The angular velocity of the pelvis is directly measured from the IMU. Subsequently, the velocity of the CoM is estimated by utilizing the spatial velocity of the pelvis, joint angular velocities, and the CoM Jacobian.}
% \end{itemize}
% \subsection{\textcolor{red}{CAM Control using Upper Body Joints}}
% \textcolor{red}{In our framework, CAM control is performed using only upper body joints, and CAM generated from the lower body during walking is not explicitly considered. The CAM from the lower body is treated as a disturbance in CP--MPC and indirectly reflected in the CP error, which is typically compensated by ZMP control.}

% \textcolor{red}{In the selection of upper body joints for CAM control, the yaw joint of the waist, neck joints, and wrist joints of both arms have been excluded, as they contribute minimally to CAM generation while introducing unnecessary computational load.} 

\subsection{{Selection of IK over ID Control}} 
\label{Section/Discussion/IKID}
{Our research group is exploring high-performance whole-body ID control {for humanoids using only proprioceptive sensors, without joint torque sensors,} even in the presence of challenges such as friction and elasticity in high-reduction-ratio actuators. In this study, however, we opted for whole-body IK to develop the walking algorithm in parallel, without the presence of such uncertainty factors.}

\subsection{{Adaptation to Early or Late Foot Contacts}} 
{Early or late foot contacts are managed through the ZMP control in \cite{kajita2010biped}. In cases of early or late foot contact, a significant difference arises between the desired contact wrench for achieving the desired ZMP and the measured contact wrench. The ZMP controller adjusts the ankle orientations and the leg length difference between both legs to track the desired contact wrench using admittance control, enabling the robot to maintain balance amidst unexpected early or late foot contacts.}  

\subsection{{Limitations of balancing performance}} 
{From the external force experiments, we found that when addressing disturbances directed toward the stance foot, the kinematic constraints designed to prevent leg collisions hinder the adjustment of far footstep placements, degrading balancing performance. In such cases, strategies like rapid stepping or cross-leg movements could be beneficial.}

{In the uneven terrain experiments, our system, which employs position control with high gear ratio actuators, struggled to quickly respond to early contact with unexpected objects. As a result, unless the experimental walking duration is further extended, the success rate for overcoming objects over 3.5 cm diminishes. We plan to address this issue by researching whole-body ID control, as mentioned in Section \ref{Section/Discussion/IKID}.}
% \subsection{\textcolor{red}{Hierarchical Structure of CP--MPC and Stepping Controller}} 
% \label{Section/Discussion/Hierarchical}
% \subsubsection{Optimality}\label{Section/Discussion/Hierarchical/Optimality}
% \textcolor{red}{In the proposed walking framework, CP--MPC and the stepping controller are structured hierarchically. 
% CP--MPC is not iteratively re-optimized within the same control cycle based on adjusted step time $T$ and footstep position $\mathbf{f}$ from the stepping controller. In order to analyze the impact of the adjusted step time and footstep position on the optimality of CP--MPC, we re-optimized CP--MPC by incorporating the adjusted footstep $\mathbf{f}$ and step time $T$. We then analyzed the cost values (\ref{eq/CP-MPC cost function}) of CP--MPC during this re-optimization process to examine the impact of these adjustments on its optimality.}

% \textcolor{red}{Fig. \ref{figure/Simulation_8_rev_cost} depicts the iteration--cost graph in a scenario where a robot experiences disturbances from the front to the back during forward walking. Fig. \ref{figure/Simulation_8_rev_cost}(a) displays the values of the average cost (\ref{eq/CP-MPC cost function}) with respect to the iteration count, and Fig. \ref{figure/Simulation_8_rev_cost}(b) illustrates the variation of the cost over time, where $i$ denotes the iteration count. In Fig. \ref{figure/Simulation_8_rev_cost}(a), a decreasing trend is evident in the average cost with iterative re-optimization; however, the cost values show relatively minor fluctuations with respect to the iteration count. Furthermore, as depicted in Fig. \ref{figure/Simulation_8_rev_cost}(b), it is difficult to distinguish the cost values between the original control structure and iteratively re-optimized structures ($i\geq 2$) when the robot is walking, whether disturbances are present or not. However, a difference in the cost values between the original control structure and $i\geq 2$ could be observed at 6.2 s.
% % As evident from the Fig. \ref{figure/Simulation_8_rev_cost}, it is difficult to distinguish between the value of costs over iterations when the robot is walking, whether disturbances are present or not. However, a difference in the cost values between $i=0$ and $i> 1$ could be observed at 6.2 s. 
% This implies that changes in the step time and footstep position from the stepping controller lead to variations in the cost of CP--MPC and affect the optimality of CP--MPC. Therefore, re-optimizing CP--MPC twice by reflecting the adjusted step time and footstep position yields values closer to the optimum for CP--MPC.}

% \textcolor{red}{However, performing more than two iterations incurs a significant computational load, making it challenging to achieve a 50 Hz operation with our 1.5 s horizon setup. Based on the simulation results, to achieve a two-time iteration, it is necessary to adjust the CP--MPC's operation to 40 Hz and reduce the MPC horizon to 1.0 s. A shorter horizon time and a decrease in control frequency can degrade MPC performance. Consequently, this leads to weakened balancing performance against disturbances.} 

% \textcolor{red}{We compared the robustness against disturbances between the single iteration case and the original framework. Ultimately, the robot demonstrated a reduction of 21.6 $\%$ and 23.2 $\%$ in its maximum external force tolerance during in-place and forward walking, respectively, compared to when the original control structure was employed.} 
% % Figure environment removed
% \subsubsection{Adavantages and limitations}\label{Section/Discussion/Hierarchical/Advantages}
% \textcolor{red}{The proposed hierarchical structure has the advantage of handling step time adjustment as a linear optimization problem, avoiding the nonlinearity associated with considering step time as an MPC variable. Previous studies have either utilized Sequential Quadratic Programming (SQP) for nonlinear MPC calculations \cite{aftab2012ankle, choe2023seamless} or linearized nonlinear constraints by defining specific bounds for the step time variable \cite{bohorquez2017adaptive}. However, optimization of step time within the MPC, as reported in \cite{khadiv2020walking}, may increase computation time and risk getting stuck in undesired local minima. The increased computational load may compromise robustness against disturbances by MPC optimizations with short horizon length and low control frequencies. In contrast, our approach defines the step time optimization as a convex problem, making it suitable for real-time applications.}

% \textcolor{red}{However, unlike conventional NMPC-based methods \cite{aftab2012ankle, kryczka2015online, bohorquez2017adaptive, choe2023seamless}, our framework optimizes step time without explicitly considering future states and constraints, unlike MPC. Additionally, it may have a limitation where the CP--MPC solution may not be entirely optimal when the step time undergoes significant modifications. Despite this limitation, our framework can enhance robustness against disturbances by operating CP--MPC and the stepping controller with a higher control frequency and sufficient CP--MPC horizon.}

% \subsubsection{Time horizon}
% \textcolor{red}{In \cite{zaytsev2015two}, it is reported that considering a future horizon beyond two walking steps in an MPC framework is unnecessary for humanoid balancing, and balancing can be achieved within mostly two steps when disturbances occur. Based on a 0.9 s step duration, we consider a 1.5 s MPC horizon for CP--MPC and a 0.9 s horizon for the stepping controller. Similar to the horizon length suggested in \cite{zaytsev2015two}, our horizon length considers 1 to 2 walking steps for both controllers. Note that, unlike CP--MPC, the stepping controller considers only a one-step horizon, as it relies on CP end-of-step dynamics and the CP--MPC control input for a single walking step. }

% % Figure environment removed

% Zaytsev : 다른 그룹들은 자신들의 모델에 대해 두 단계 또는 세 단계를 미리 계획하는 것이 충분한 안정성을 제공한다는 것을 발견했습니다 [7], [25], [26]. 예를 들어, Nishiwaki 등 [26]는 중심 압력의 주어진 경로(ZMP 경로)를 기반으로 로봇 링크의 보행 궤적을 온라인으로 생성하는 문제를 연구했습니다. 각 궤적은 경계 값 문제(BVP)의 해로 생성되며 여러 제약 조건(초기 및 최종 조건 및 주어진 ZMP 경로를 따르기)을 충족시키려고 두 단계, 세 단계 또는 그 이상을 계획합니다. Nishiwaki 등은 세 단계 이상을 계획하는 것이 생성된 궤적을 거의 개선하지 않는다는 결론을 내렸습니다(그들의 부드러움 및 제약 조건을 충족하지 못한 오류). 이 모델은 각 스텝마다 두 개의 제어 매개변수를 갖고 있으며, 이는 스텝 크기와 푸시 오프 충격의 양입니다. 우리는 모델이 적절한 제어를 통해 특정한 목표 속도에 n 단계 안에 도달할 수 있는 모든 중간 상태를 계산했습니다. 우리는 일반적으로 모델이 목표에 도달할 수 있다면 대부분 두 단계 이하로 이루어질 수 있음을 보였습니다. 이 결과는 어떤 제어도 일부 실제적인 한계 내에서 제한된 경우에도 변함이 없었습니다. 또한, 이 능력이 보다 복잡한 모델에 대해서도 유효하며, 인간들에 의해 사용되는 것으로 보이는 것을 입증하기 위해 실제 로봇 제어 및 여러 인간 보행 실험에서 증거를 제시했습니다. 상기 고려 사항들은 우리를 다음의 명제로 이끌었습니다: 두 단계는 거의 모든 것이다. 즉, 모델이 어떤 것이든 할 수 있는 경우, 대부분은 단 두 단계 안에 할 수 있다는 것입니다. 실제로, 이는 컨트롤러 설계에서 로봇의 움직임을 계획하는 데, 적어도 균형 목적으로는 두 단계 이상 앞으로 계획할 필요가 없다는 것을 시사합니다. 왜냐하면, 인간과 유사한 비율을 갖는 로봇의 경우, 상반신 동작의 복잡성에 관계없이 대부분의 제어 권한이 스텝 위치와 푸시 오프에 있다고 여겨지기 때문에, 이러한 결과는 거의 모든 상황에서 인간 및 모든 인류형 로봇의 보행에 적용될 것으로 믿습니다. 2 step , 1-2step 안에 스테핑을 통해 밸런싱을 못하면 넘어지는것이기 때문에 1, 2스텝 정도의 짧은 horizon time만 봐도 되는것은 아님. 미래를 고려해서 현재의 제어입력을 만들어 내는것이기 때문에, 짧은 horizon time을 기반으로 현재의 제어 입력을 계산하면 1 step 밸런싱도 못할 수 있음. 

% Bohorquez : 이 논문에서는 Bolt나 Griffin의 논문을 두고 스텝 시간 최적화를 위해 1 step 호라이즌에 제한시키는것을 두고 너무 짧다라고 말함 (limit the number of steps in the future to one/ one proposes a preview horizon that might be too short). LIPM을 기반으로 스텝 시간 조정을 할때 직면하는 특정한 비선형성을 제시하고, 발생하는 비선형 constraint를 베지어 곡선을 이용하여 스텝 시간 변수에 대해 state들이 특정 bound에 포함되도록 정의하여 기구학, 동역학 등이 고려되는 비선형 constraint를 선형화하여 사용하였다. 즉, 스텝 시간이 변수가 될때 발생하는 비선형 constraint를 선형 constraint로 변환하였다. 
% Bolt 논문에서 언급한 Bohorquez 방법 : In [3], a robust approach is proposed to deal with the nonlinearity introduced by adapting step timing in a receding horizon controller. 저희의 접근 방식은 기존 접근 방식 [1], [3], [6], [34], [35]보다 최소 한 차원 덜한 결정 변수와 볼록 형태의 공식을 사용하여 두 가지 문제를 우회하고 실시간 응용 프로그램에 신뢰성 있는 도구를 제공합니다.
% [8] investigated nonlinear constraints due to step timing adaptation. However, 1) they do not account for the closed-loop tracking errors due to disturbances, 2) there are no robustness guarantees of constraints satisfaction in the presence of different disturbances, which is critical for generating safe walking motions
% A nonlinear MPC is presented in [5] for adapting step timings via time scaling at each iteration

\section{Conclusion}
\label{Section/Conclusion}
This paper presents a novel balance control framework for humanoid robots to achieve robust balancing performance. The framework employs a linear MPC to drive three balance strategies, namely ankle, hip, and stepping strategies, for CP tracking control, effectively calculating ZMP, CAM, and footstep positions. Additionally, a variable weighting method adjusts MPC weighting parameters over the time horizon for CAM damping control, reducing the influence of damping action on CAM and improving balancing performance. %compared to the conventional constant weighting method.
% Next, a stepping controller based on the CP--MPC is proposed. The stepping controller optimizes the step time while satisfying the output of CP--MPC as closely as possible.

% Next, a hierarchical structure combining CP--MPC and a stepping controller is proposed to enable step time optimization.
Next, a hierarchical structure combining CP--MPC and a stepping controller is proposed, enabling step time optimization. Specifically, the stepping controller adjusts the step time while closely following the footstep adjustments planned by CP--MPC to control the CP offset. The parameter selection method in the stepping controller is validated against methods from previous studies \cite{khadiv2016step, khadiv2020walking, kim2023foot}.
% The suitability of the parameter selection method used in the stepping controller is validated in comparison to the methods proposed in previous studies \cite{khadiv2016step, khadiv2020walking, kim2023foot}.

Finally, the balancing performance of the proposed method has been validated through simulations and real robot experiments under various scenarios with disturbances. The robot can traverse uneven terrain and overcome external forces.  
% Furthermore, the superior balancing performance of the proposed method was experimentally verified, compared with a state-of-the-art QP-based CP controller \cite{jeong2019robust}.

{In the current walking framework, LIPFM dynamics are defined under the assumption of flat terrain, with no adjustments made for ground inclination affecting LIPFM dynamics, foot, or CoM trajectories. Our future work will adapt these dynamics and trajectories for sloped or uneven terrain. %Additionally, to enhance the generality of the walking framework, we plan to investigate the implementation of a state machine to generate steps in response to disturbances, even after a predefined walking sequence has ended or while standing still.
Additionally, to enhance generality, we plan to implement a state machine for step generation in response to disturbances, even after a predefined walking sequence or while standing still.}
% \textcolor{blue}{In the proposed framework, due to the absence of JTS and the friction caused by high gear ratios, whole-body IK control is utilized. Our future work includes implementing whole-body control based on inverse dynamics rather than IK.}
% In this study, we addressed the optimization of step time by developing a hierarchical structure comprising CP--MPC and a stepping controller. Our future work will focus on creating an MPC-based balance control framework that includes step time as an MPC variable. This framework will enable the simultaneous determination of ZMP, CAM, footstep position, and step time, providing a comprehensive solution for balancing control of humanoid robots.
% To enhance the practicality of the proposed balance strategy in various environments, it is necessary to incorporate the vision system, enabling the robot to employ suitable balancing strategies based on the surrounding environment. For instance, when disturbances are applied to the robot in terrain where it is challenging to adjust the foot placement, it would be advisable to minimize the use of stepping strategies and prioritize ankle and hip strategies. Therefore, our future work will focus on integrating a vision system to further enhance the balance strategy. By utilizing the vision system to perceive the surrounding environment, the robot can apply appropriate strategies that effectively overcome disturbances.
% 본 연구에서는 Ankle, hip, stepping strategies와 스텝 시간 최적화를 이용하여 외란에 대해 강건한 균형 제어기를 개발하였다. 본 제어 전략이 다양한 환경에서 더 활용도 높게 사용되기 위해서는, 비전 시스템과의 결합을 통해 각 지형 지물에 따라 적절한 보행 전략이 적절하게 사용되어야 한다. 예를 들어 좁은 지형에서 외란이 가해지면, Stepping 전략을 최소화하며 Ankle과 Hip 제어를 통한 밸런싱이 우선시 되어야 한다. 따라서 우리의 미래 연구를 비전시스템과의 결합을 통해 더 적절한 전략을 통해 다양한 지형에서의 외란을 극복하며, 일상에서 활용될 수 있어야한다.
% \section*{Acknowledgments}
% This should be a simple paragraph before the References to thank those individuals and institutions who have supported your work on this article.

{\appendix[A. Derivation of CMP Parameter $\mathbf{P}_{ssp}$] \label{section/appendix}
{According to the definition of (\ref{eq/reculsive dynamics}), the predicted CP end-of-step within the MPC can be expressed as follows,
% \begin{align} \label{eq/CPeos_CPMPC}
%     {\xi}_{T_k,x} = {A}^{T_k-k}\xi_{x,k} + 
%     \begin{bmatrix}
%     {A}^{T_k-k-1}\mathbf{B} \cdots A^{0}\mathbf{B}
%     \end{bmatrix}
%     \begin{bmatrix}
%     z_{x,k} \\ \tau_{y,k} \\ \vdots \\ z_{x,T_k-1} \\ \tau_{y,T_k-1} 
%     \end{bmatrix}
% \end{align} 
\begin{equation} \label{eq/CPeos_CPMPC}
    \resizebox{0.89\linewidth}{!}{%
        $
        {\xi}_{T_k,x} = {A}^{T_k-k}\xi_{x,k} + 
        \begin{bmatrix}
        {A}^{T_k-k-1}\mathbf{B} \cdots A^{0}\mathbf{B}
        \end{bmatrix}
        \begin{bmatrix}
        z_{x,k} \\ \tau_{y,k} \\ \vdots \\ z_{x,T_k-1} \\ \tau_{y,T_k-1} 
        \end{bmatrix}
        $
    }
\end{equation}
where $T_k$ represents the time step at which SSP ends within the MPC horizon. To incorporate the predicted CP end-of-step (\ref{eq/CPeos_CPMPC}) into the equality constraint (\ref{eq/stepping_constraint}) of the stepping controller, it is necessary to represent it in the discretized form of the CP end-of-step dynamics (\ref{eq/stepping_cpeos}) as follows,
\begin{align}
\label{appendix/CPeos}
    {\xi}_{T_k,x} &= {A}^{T_k-k}(\xi_{x,k}-{p}_{ssp,x,k})+{p}_{ssp,x,k} \\ \nonumber
    &= {A}^{T_k-k}\xi_{x,k}+(1-{A}^{T_k-k})\,{p}_{ssp,x,k}. 
\end{align} 
Here, $A=e^{\omega T_s}$, and $p_{ssp,x,k}$ denotes the CMP parameter in the x-direction. 
Therefore, to make (25) identical to (24), the CMP parameter ${p}_{ssp,x}$ is defined as follows:
% \begin{align}
%     {p}_{ssp,x} = \frac{1}{1-{A}^{T_k-k}} 
%     \begin{bmatrix}
%     {A}^{T_k-k-1}\mathbf{B} \cdots {A}^{0}\mathbf{B}
%     \end{bmatrix}
%     \begin{bmatrix}
%     z_{x,k} \\ \tau_{y,k} \\ \vdots \\ z_{x,T_k-1} \\ \tau_{y,T_k-1}
%     \end{bmatrix}.
% \end{align} 
\begin{align}
    \resizebox{0.89\linewidth}{!}{  
        ${p}_{ssp,x} = \frac{1}{1-{A}^{T_k-k}} 
        \begin{bmatrix}
        {A}^{T_k-k-1}\mathbf{B} \cdots {A}^{0}\mathbf{B}
        \end{bmatrix}
        \begin{bmatrix}
        z_{x,k} \\ \tau_{y,k} \\ \vdots \\ z_{x,T_k-1} \\ \tau_{y,T_k-1}
        \end{bmatrix} $
    }
\end{align} 
}
% {\appendices
% \section*{Proof of the First Zonklar Equation}
% Appendix one text goes here.
% You can choose not to have a title for an appendix if you want by leaving the argument blank
% \section*{Proof of the Second Zonklar Equation}
% Appendix two text goes here.}



% \section{References Section}
% You can use a bibliography generated by BibTeX as a .bbl file.
%  BibTeX documentation can be easily obtained at:
%  http://mirror.ctan.org/biblio/bibtex/contrib/doc/
%  The IEEEtran BibTeX style support page is:
%  http://www.michaelshell.org/tex/ieeetran/bibtex/
 
 % argument is your BibTeX string definitions and bibliography database(s)
%\bibliography{IEEEabrv,../bib/paper}
%
% \section{Simple References}
% You can manually copy in the resultant .bbl file and set second argument of $\backslash${\tt{begin}} to the number of references
%  (used to reserve space for the reference number labels box).

% \begin{thebibliography}{1} 이거랑 \bibliographystyle{BIB/IEEEtran} 얘 같이쓰면 버그생김.
\bibliographystyle{BIB/IEEEtran}
% \bibliography{BIB/references}
\bibliography{BIB/Manuscript}
% \end{thebibliography}

\begin{comment} 
\bibitem{ref1}
{\it{Mathematics Into Type}}. American Mathematical Society. [Online]. Available: https://www.ams.org/arc/styleguide/mit-2.pdf

\bibitem{ref2}
T. W. Chaundy, P. R. Barrett and C. Batey, {\it{The Printing of Mathematics}}. London, U.K., Oxford Univ. Press, 1954.

\bibitem{ref3}
F. Mittelbach and M. Goossens, {\it{The \LaTeX Companion}}, 2nd ed. Boston, MA, USA: Pearson, 2004.

\bibitem{ref4}
G. Gr\"atzer, {\it{More Math Into LaTeX}}, New York, NY, USA: Springer, 2007.

\bibitem{ref5}M. Letourneau and J. W. Sharp, {\it{AMS-StyleGuide-online.pdf,}} American Mathematical Society, Providence, RI, USA, [Online]. Available: http://www.ams.org/arc/styleguide/index.html

\bibitem{ref6}
H. Sira-Ramirez, ``On the sliding mode control of nonlinear systems,'' \textit{Syst. Control Lett.}, vol. 19, pp. 303--312, 1992.

\bibitem{ref7}
A. Levant, ``Exact differentiation of signals with unbounded higher derivatives,''  in \textit{Proc. 45th IEEE Conf. Decis.
Control}, San Diego, CA, USA, 2006, pp. 5585--5590. DOI: 10.1109/CDC.2006.377165.

\bibitem{ref8}
M. Fliess, C. Join, and H. Sira-Ramirez, ``Non-linear estimation is easy,'' \textit{Int. J. Model., Ident. Control}, vol. 4, no. 1, pp. 12--27, 2008.

\bibitem{ref9}
R. Ortega, A. Astolfi, G. Bastin, and H. Rodriguez, ``Stabilization of food-chain systems using a port-controlled Hamiltonian description,'' in \textit{Proc. Amer. Control Conf.}, Chicago, IL, USA,
2000, pp. 2245--2249.
\end{comment}




% \newpage

% \section{Biography Section}
% If you have an EPS/PDF photo (graphicx package needed), extra braces are
%  needed around the contents of the optional argument to biography to prevent
%  the LaTeX parser from getting confused when it sees the complicated
%  $\backslash${\tt{includegraphics}} command within an optional argument. (You can create
%  your own custom macro containing the $\backslash${\tt{includegraphics}} command to make things
%  simpler here.)
 
 

% \bf{If you will not include a photo:}\vspace{-33pt}
% \begin{IEEEbiographynophoto}{John Doe}
% Use $\backslash${\tt{begin\{IEEEbiographynophoto\}}} and the author name as the argument followed by the biography text.
% \end{IEEEbiographynophoto}




\vfill


\begin{comment}
    \section{Where to Get \LaTeX \ Help --- User Groups}
The following online groups are helpful to beginning and experienced \LaTeX\ users. A search through their archives can provide many answers to common questions.
\begin{list}{}{}
\item{\url{http://www.latex-community.org/}} 
\item{\url{https://tex.stackexchange.com/} }
\end{list}

\section{Some Common Elements}
\subsection{Lists}
In this section, we will consider three types of lists: simple unnumbered, numbered, and bulleted. There have been many options added to IEEEtran to enhance the creation of lists. If your lists are more complex than those shown below, please refer to the original ``IEEEtran\_HOWTO.pdf'' for additional options.\\

\subsection{Figures}
Fig. 1 is an example of a floating figure using the graphicx package.
 Note that $\backslash${\tt{label}} must occur AFTER (or within) $\backslash${\tt{caption}}.
 For figures, $\backslash${\tt{caption}} should occur after the $\backslash${\tt{includegraphics}}.

% Figure environment removed

Fig. 2(a) and 2(b) is an example of a double column floating figure using two subfigures.
 (The subfig.sty package must be loaded for this to work.)
 The subfigure $\backslash${\tt{label}} commands are set within each subfloat command,
 and the $\backslash${\tt{label}} for the overall figure must come after $\backslash${\tt{caption}}.
 $\backslash${\tt{hfil}} is used as a separator to get equal spacing.
 The combined width of all the parts of the figure should do not exceed the text width or a line break will occur.
%
% Figure environment removed

Note that often IEEE papers with multi-part figures do not place the labels within the image itself (using the optional argument to $\backslash${\tt{subfloat}}[]), but instead will
 reference/describe all of them (a), (b), etc., within the main caption.
 Be aware that for subfig.sty to generate the (a), (b), etc., subfigure
 labels, the optional argument to $\backslash${\tt{subfloat}} must be present. If a
 subcaption is not desired, leave its contents blank,
 e.g.,$\backslash${\tt{subfloat}}[].


 

\section{Tables}
Note that, for IEEE-style tables, the
 $\backslash${\tt{caption}} command should come BEFORE the table. Table captions use title case. Articles (a, an, the), coordinating conjunctions (and, but, for, or, nor), and most short prepositions are lowercase unless they are the first or last word. Table text will default to $\backslash${\tt{footnotesize}} as
 the IEEE normally uses this smaller font for tables.
 The $\backslash${\tt{label}} must come after $\backslash${\tt{caption}} as always.
 
\begin{table}[!t]
\caption{An Example of a Table\label{tab:table1}}
\centering
\begin{tabular}{|c||c|}
\hline
One & Two\\
\hline
Three & Four\\
\hline
\end{tabular}
\end{table}

\section{Algorithms}
Algorithms should be numbered and include a short title. They are set off from the text with rules above and below the title and after the last line.

\begin{algorithm}[H]
\caption{Weighted Tanimoto ELM.}\label{alg:alg1}
\begin{algorithmic}
\STATE 
\STATE {\textsc{TRAIN}}$(\mathbf{X} \mathbf{T})$
\STATE \hspace{0.5cm}$ \textbf{select randomly } W \subset \mathbf{X}  $
\STATE \hspace{0.5cm}$ N_\mathbf{t} \gets | \{ i : \mathbf{t}_i = \mathbf{t} \} | $ \textbf{ for } $ \mathbf{t}= -1,+1 $
\STATE \hspace{0.5cm}$ B_i \gets \sqrt{ \textsc{max}(N_{-1},N_{+1}) / N_{\mathbf{t}_i} } $ \textbf{ for } $ i = 1,...,N $
\STATE \hspace{0.5cm}$ \hat{\mathbf{H}} \gets  B \cdot (\mathbf{X}^T\textbf{W})/( \mathbb{1}\mathbf{X} + \mathbb{1}\textbf{W} - \mathbf{X}^T\textbf{W} ) $
\STATE \hspace{0.5cm}$ \beta \gets \left ( I/C + \hat{\mathbf{H}}^T\hat{\mathbf{H}} \right )^{-1}(\hat{\mathbf{H}}^T B\cdot \mathbf{T})  $
\STATE \hspace{0.5cm}\textbf{return}  $\textbf{W},  \beta $
\STATE 
\STATE {\textsc{PREDICT}}$(\mathbf{X} )$
\STATE \hspace{0.5cm}$ \mathbf{H} \gets  (\mathbf{X}^T\textbf{W} )/( \mathbb{1}\mathbf{X}  + \mathbb{1}\textbf{W}- \mathbf{X}^T\textbf{W}  ) $
\STATE \hspace{0.5cm}\textbf{return}  $\textsc{sign}( \mathbf{H} \beta )$
\end{algorithmic}
\label{alg1}
\end{algorithm}

Que sunt eum lam eos si dic to estist, culluptium quid qui nestrum nobis reiumquiatur minimus minctem. Ro moluptat fuga. Itatquiam ut laborpo rersped exceres vollandi repudaerem. Ulparci sunt, qui doluptaquis sumquia ndestiu sapient iorepella sunti veribus. Ro moluptat fuga. Itatquiam ut laborpo rersped exceres vollandi repudaerem. 
\section{Mathematical Typography \\ and Why It Matters}

Typographical conventions for mathematical formulas have been developed to {\bf provide uniformity and clarity of presentation across mathematical texts}. This enables the readers of those texts to both understand the author's ideas and to grasp new concepts quickly. While software such as \LaTeX \ and MathType\textsuperscript{\textregistered} can produce aesthetically pleasing math when used properly, it is also very easy to misuse the software, potentially resulting in incorrect math display.

IEEE aims to provide authors with the proper guidance on mathematical typesetting style and assist them in writing the best possible article. As such, IEEE has assembled a set of examples of good and bad mathematical typesetting \cite{ref1,ref2,ref3,ref4,ref5}. 

Further examples can be found at \url{http://journals.ieeeauthorcenter.ieee.org/wp-content/uploads/sites/7/IEEE-Math-Typesetting-Guide-for-LaTeX-Users.pdf}

\subsection{Display Equations}
The simple display equation example shown below uses the ``equation'' environment. To number the equations, use the $\backslash${\tt{label}} macro to create an identifier for the equation. LaTeX will automatically number the equation for you.
\begin{equation}
\label{deqn_ex1}
x = \sum_{i=0}^{n} 2{i} Q.
\end{equation}

\noindent is coded as follows:
\begin{verbatim}
\begin{equation}
\label{deqn_ex1}
x = \sum_{i=0}^{n} 2{i} Q.
\end{equation}
\end{verbatim}

To reference this equation in the text use the $\backslash${\tt{ref}} macro. 
Please see (\ref{deqn_ex1})\\
\noindent is coded as follows:
\begin{verbatim}
Please see (\ref{deqn_ex1})\end{verbatim}

\subsection{Equation Numbering}
{\bf{Consecutive Numbering:}} Equations within an article are numbered consecutively from the beginning of the
article to the end, i.e., (1), (2), (3), (4), (5), etc. Do not use roman numerals or section numbers for equation numbering.

\noindent {\bf{Appendix Equations:}} The continuation of consecutively numbered equations is best in the Appendix, but numbering
 as (A1), (A2), etc., is permissible.\\

\noindent {\bf{Hyphens and Periods}}: Hyphens and periods should not be used in equation numbers, i.e., use (1a) rather than
(1-a) and (2a) rather than (2.a) for subequations. This should be consistent throughout the article.

\subsection{Multi-Line Equations and Alignment}
Here we show several examples of multi-line equations and proper alignments.

\noindent {\bf{A single equation that must break over multiple lines due to length with no specific alignment.}}
\begin{multline}
\text{The first line of this example}\\
\text{The second line of this example}\\
\text{The third line of this example}
\end{multline}

\noindent is coded as:
\begin{verbatim}
\begin{multline}
\text{The first line of this example}\\
\text{The second line of this example}\\
\text{The third line of this example}
\end{multline}
\end{verbatim}

\noindent {\bf{A single equation with multiple lines aligned at the = signs}}
\begin{align}
a &= c+d \\
b &= e+f
\end{align}
\noindent is coded as:
\begin{verbatim}
\begin{align}
a &= c+d \\
b &= e+f
\end{align}
\end{verbatim}

The {\tt{align}} environment can align on multiple  points as shown in the following example:
\begin{align}
x &= y & X & =Y & a &=bc\\
x' &= y' & X' &=Y' &a' &=bz
\end{align}
\noindent is coded as:
\begin{verbatim}
\begin{align}
x &= y & X & =Y & a &=bc\\
x' &= y' & X' &=Y' &a' &=bz
\end{align}
\end{verbatim}





\subsection{Subnumbering}
The amsmath package provides a {\tt{subequations}} environment to facilitate subnumbering. An example:

\begin{subequations}\label{eq:2}
\begin{align}
f&=g \label{eq:2A}\\
f' &=g' \label{eq:2B}\\
\mathcal{L}f &= \mathcal{L}g \label{eq:2c}
\end{align}
\end{subequations}

\noindent is coded as:
\begin{verbatim}
\begin{subequations}\label{eq:2}
\begin{align}
f&=g \label{eq:2A}\\
f' &=g' \label{eq:2B}\\
\mathcal{L}f &= \mathcal{L}g \label{eq:2c}
\end{align}
\end{subequations}

\end{verbatim}

\subsection{Matrices}
There are several useful matrix environments that can save you some keystrokes. See the example coding below and the output.

\noindent {\bf{A simple matrix:}}
\begin{equation}
\begin{matrix}  0 &  1 \\ 
1 &  0 \end{matrix}
\end{equation}
is coded as:
\begin{verbatim}
\begin{equation}
\begin{matrix}  0 &  1 \\ 
1 &  0 \end{matrix}
\end{equation}
\end{verbatim}

\noindent {\bf{A matrix with parenthesis}}
\begin{equation}
\begin{pmatrix} 0 & -i \\
 i &  0 \end{pmatrix}
\end{equation}
is coded as:
\begin{verbatim}
\begin{equation}
\begin{pmatrix} 0 & -i \\
 i &  0 \end{pmatrix}
\end{equation}
\end{verbatim}

\noindent {\bf{A matrix with square brackets}}
\begin{equation}
\begin{bmatrix} 0 & -1 \\ 
1 &  0 \end{bmatrix}
\end{equation}
is coded as:
\begin{verbatim}
\begin{equation}
\begin{bmatrix} 0 & -1 \\ 
1 &  0 \end{bmatrix}
\end{equation}
\end{verbatim}

\noindent {\bf{A matrix with curly braces}}
\begin{equation}
\begin{Bmatrix} 1 &  0 \\ 
0 & -1 \end{Bmatrix}
\end{equation}
is coded as:
\begin{verbatim}
\begin{equation}
\begin{Bmatrix} 1 &  0 \\ 
0 & -1 \end{Bmatrix}
\end{equation}\end{verbatim}

\noindent {\bf{A matrix with single verticals}}
\begin{equation}
\begin{vmatrix} a &  b \\ 
c &  d \end{vmatrix}
\end{equation}
is coded as:
\begin{verbatim}
\begin{equation}
\begin{vmatrix} a &  b \\ 
c &  d \end{vmatrix}
\end{equation}\end{verbatim}

\noindent {\bf{A matrix with double verticals}}
\begin{equation}
\begin{Vmatrix} i &  0 \\ 
0 & -i \end{Vmatrix}
\end{equation}
is coded as:
\begin{verbatim}
\begin{equation}
\begin{Vmatrix} i &  0 \\ 
0 & -i \end{Vmatrix}
\end{equation}\end{verbatim}

\subsection{Arrays}
The {\tt{array}} environment allows you some options for matrix-like equations. You will have to manually key the fences, but there are other options for alignment of the columns and for setting horizontal and vertical rules. The argument to {\tt{array}} controls alignment and placement of vertical rules.

A simple array
\begin{equation}
\left(
\begin{array}{cccc}
a+b+c & uv & x-y & 27\\
a+b & u+v & z & 134
\end{array}\right)
\end{equation}
is coded as:
\begin{verbatim}
\begin{equation}
\left(
\begin{array}{cccc}
a+b+c & uv & x-y & 27\\
a+b & u+v & z & 134
\end{array} \right)
\end{equation}
\end{verbatim}

A slight variation on this to better align the numbers in the last column
\begin{equation}
\left(
\begin{array}{cccr}
a+b+c & uv & x-y & 27\\
a+b & u+v & z & 134
\end{array}\right)
\end{equation}
is coded as:
\begin{verbatim}
\begin{equation}
\left(
\begin{array}{cccr}
a+b+c & uv & x-y & 27\\
a+b & u+v & z & 134
\end{array} \right)
\end{equation}
\end{verbatim}

An array with vertical and horizontal rules
\begin{equation}
\left( \begin{array}{c|c|c|r}
a+b+c & uv & x-y & 27\\ \hline
a+b & u+v & z & 134
\end{array}\right)
\end{equation}
is coded as:
\begin{verbatim}
\begin{equation}
\left(
\begin{array}{c|c|c|r}
a+b+c & uv & x-y & 27\\
a+b & u+v & z & 134
\end{array} \right)
\end{equation}
\end{verbatim}
Note the argument now has the pipe "$\vert$" included to indicate the placement of the vertical rules.


\subsection{Cases Structures}
Many times cases can be miscoded using the wrong environment, i.e., {\tt{array}}. Using the {\tt{cases}} environment will save keystrokes (from not having to type the $\backslash${\tt{left}}$\backslash${\tt{lbrace}}) and automatically provide the correct column alignment.
\begin{equation*}
{z_m(t)} = \begin{cases}
1,&{\text{if}}\ {\beta }_m(t) \\ 
{0,}&{\text{otherwise.}} 
\end{cases}
\end{equation*}
\noindent is coded as follows:
\begin{verbatim}
\begin{equation*}
{z_m(t)} = 
\begin{cases}
1,&{\text{if}}\ {\beta }_m(t),\\ 
{0,}&{\text{otherwise.}} 
\end{cases}
\end{equation*}
\end{verbatim}
\noindent Note that the ``\&'' is used to mark the tabular alignment. This is important to get  proper column alignment. Do not use $\backslash${\tt{quad}} or other fixed spaces to try and align the columns. Also, note the use of the $\backslash${\tt{text}} macro for text elements such as ``if'' and ``otherwise.''

\subsection{Function Formatting in Equations}
Often, there is an easy way to properly format most common functions. Use of the $\backslash$ in front of the function name will in most cases, provide the correct formatting. When this does not work, the following example provides a solution using the $\backslash${\tt{text}} macro:

\begin{equation*} 
  d_{R}^{KM} = \underset {d_{l}^{KM}} {\text{arg min}} \{ d_{1}^{KM},\ldots,d_{6}^{KM}\}.
\end{equation*}

\noindent is coded as follows:
\begin{verbatim}
\begin{equation*} 
 d_{R}^{KM} = \underset {d_{l}^{KM}} 
 {\text{arg min}} \{ d_{1}^{KM},
 \ldots,d_{6}^{KM}\}.
\end{equation*}
\end{verbatim}

\subsection{ Text Acronyms Inside Equations}
This example shows where the acronym ``MSE" is coded using $\backslash${\tt{text\{\}}} to match how it appears in the text.

\begin{equation*}
 \text{MSE} = \frac {1}{n}\sum _{i=1}^{n}(Y_{i} - \hat {Y_{i}})^{2}
\end{equation*}

\begin{verbatim}
\begin{equation*}
 \text{MSE} = \frac {1}{n}\sum _{i=1}^{n}
(Y_{i} - \hat {Y_{i}})^{2}
\end{equation*}
\end{verbatim}
\end{comment}

\end{document}


