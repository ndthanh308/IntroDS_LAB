%%
%% This is file `sample-sigconf.tex',
%% generated with the docstrip utility.
%%
%% The original source files were:
%%
%% samples.dtx  (with options: `sigconf')
%% 
%% IMPORTANT NOTICE:
%% 
%% For the copyright see the source file.
%% 
%% Any modified versions of this file must be renamed
%% with new filenames distinct from sample-sigconf.tex.
%% 
%% For distribution of the original source see the terms
%% for copying and modification in the file samples.dtx.
%% 
%% This generated file may be distributed as long as the
%% original source files, as listed above, are part of the
%% same distribution. (The sources need not necessarily be
%% in the same archive or directory.)
%%
%%
%% Commands for TeXCount
%TC:macro \cite [option:text,text]
%TC:macro \citep [option:text,text]
%TC:macro \citet [option:text,text]
%TC:envir table 0 1
%TC:envir table* 0 1
%TC:envir tabular [ignore] word
%TC:envir displaymath 0 word
%TC:envir math 0 word
%TC:envir comment 0 0
%%
%%
%% The first command in your LaTeX source must be the \documentclass
%% command.
%%
%% For submission and review of your manuscript please change the
%% command to \documentclass[manuscript, screen, review]{acmart}.
%%
%% When submitting camera ready or to TAPS, please change the command
%% to \documentclass[sigconf]{acmart} or whichever template is required
%% for your publication.
%%
%%
\documentclass[sigconf,screen]{acmart}
% \documentclass[sigconf,anonymous,review,screen]{acmart}



\usepackage{times}
\usepackage{epsfig}
\usepackage{graphicx}
\usepackage{amsmath}
% \usepackage{amssymb}
\usepackage{booktabs}
\usepackage{tikz}
\usepackage{comment}
\usepackage{color}
% \usepackage{cite}
\usepackage{subfigure}
\usepackage{multirow}
\usepackage{diagbox}
\usepackage{rotating}
\usepackage{booktabs}
\usepackage{colortbl}
\usepackage{overpic}
\usepackage{makecell}
\usepackage{textcomp}
\usepackage{contour}
\usepackage{algorithm}
\usepackage{algorithmic}

%%
%% \BibTeX command to typeset BibTeX logo in the docs
\AtBeginDocument{%
  \providecommand\BibTeX{{%
    Bib\TeX}}}

%% Rights management information.  This information is sent to you
%% when you complete the rights form.  These commands have SAMPLE
%% values in them; it is your responsibility as an author to replace
%% the commands and values with those provided to you when you
%% complete the rights form.
\setcopyright{acmcopyright}
\copyrightyear{2023}
\acmYear{2023}
\acmDOI{XXXXXXX.XXXXXXX}

% These commands are for a PROCEEDINGS abstract or paper.
\acmConference[MM'23]{Proceedings of the 31st ACM International Conference on Multimedia (MM '23}{October 2023}{Ottawa, Canada}
 % 29 -- November 03

\acmBooktitle{Proceedings of the 31st ACM International Conference on Multimedia (MM '23),October 29 -- November 03,2023, Ottawa, Canada}

\acmPrice{15.00}
\acmISBN{978-1-4503-XXXX-X/18/06}


%%
%% Submission ID.
%% Use this when submitting an article to a sponsored event. You'll
%% receive a unique submission ID from the organizers
%% of the event, and this ID should be used as the parameter to this command.
\acmSubmissionID{506}


\newcommand{\pjl}[1]{{\textcolor{red}{#1}}}

\newcommand{\sm}[1]{{\textcolor{magenta}{#1}}}
\newcommand{\tickYes}{\bullet}
\newcommand{\cmark}{\checkmark}
\newcommand{\tickNo}{\texttimes}

\newcommand{\figref}[1]{Fig. \ref{#1}}
\newcommand{\tabref}[1]{Table \ref{#1}}
\newcommand{\eqnref}[1]{Eq. (\ref{#1})}
\newcommand{\secref}[1]{Sec. \ref{#1}}
\newcommand{\algref}[1]{Alg. \ref{#1}}

\newcommand{\ourmodel}{UDUN}
\newcommand{\Rows}[1]{\multirow{3}{*}{#1}}
\newcommand{\Cols}[1]{\multicolumn{4}{c|}{#1}}

\renewcommand{\tabcolsep}{.5mm}


\newcommand{\tabincell}[2]{
\begin{tabular}{@{}#1@{}}#2\end{tabular}
}

\definecolor{mygray}{gray}{.95}
\definecolor{myRed}{RGB}{219, 68, 55}
\definecolor{myGreen}{RGB}{15, 157, 88}
\definecolor{myBlue}{RGB}{66, 133, 244}
\newcommand{\tr}[1]{{\textcolor{myRed}{\textbf{#1}}}}
\newcommand{\tg}[1]{{\textcolor{myGreen}{\textbf{#1}}}}
\newcommand{\tb}[1]{{\textcolor{myBlue}{\textbf{#1}}}}


\newcommand{\sArt}{state-of-the-art}
\newcommand{\eg}{\emph{e.g.}}
\newcommand{\ie}{\emph{i.e.}}
\newcommand{\etal}{\emph{et al.}}
\newcommand{\AddText}[3]{\put(#1,#2){\contour{white}{\textbf{\textcolor{black}{#3}}}}}
\newcommand{\AddAttr}[3]{\put(#1,#2){\contour{black}{\textbf{\textcolor{white}{#3}}}}}
\graphicspath{{./Imgs/}}


\renewcommand\footnotetextcopyrightpermission[1]{}
\settopmatter{printacmref=false} %remove ACM reference format


%%
%% For managing citations, it is recommended to use bibliography
%% files in BibTeX format.
%%
%% You can then either use BibTeX with the ACM-Reference-Format style,
%% or BibLaTeX with the acmnumeric or acmauthoryear sytles, that include
%% support for advanced citation of software artefact from the
%% biblatex-software package, also separately available on CTAN.
%%
%% Look at the sample-*-biblatex.tex files for templates showcasing
%% the biblatex styles.
%%

%%
%% The majority of ACM publications use numbered citations and
%% references.  The command \citestyle{authoryear} switches to the
%% "author year" style.
%%
%% If you are preparing content for an event
%% sponsored by ACM SIGGRAPH, you must use the "author year" style of
%% citations and references.
%% Uncommenting
%% the next command will enable that style.
%%\citestyle{acmauthoryear}


%%
%% end of the preamble, start of the body of the document source.
\begin{document}

%%
%% The "title" command has an optional parameter,
%% allowing the author to define a "short title" to be used in page headers.
\title{Unite-Divide-Unite: Joint Boosting Trunk and Structure for High-accuracy Dichotomous Image Segmentation}

%%
%% The "author" command and its associated commands are used to define
%% the authors and their affiliations.
%% Of note is the shared affiliation of the first two authors, and the
%% "authornote" and "authornotemark" commands
%% used to denote shared contribution to the research.

\author{Jialun Pei}
\authornote{Both authors contributed equally to this research.}
%\orcid{0000-0002-2630-2838}
%\authornotemark[1]
 \affiliation{School of Computer Science and Engineering, 
   \institution{The Chinese University of Hong Kong}
% \streetaddress{P.O. Box 1212}
  \city{Shatin}
  \state{NT}
  \country{Hong Kong}
%  \postcode{43017-6221}
}
\email{jialunpei@cuhk.edu.hk}

\author{Zhangjun Zhou}
\authornotemark[1]
 \affiliation{School of Software Engineering, 
   \institution{Huazhong University of Science and Technology}
\streetaddress{Wuhan, China}
  \city{Wuhan}
%  \state{NT}
  \country{China}
%  \postcode{43017-6221}
}
\email{zhouzhangjun@hust.edu.cn}

\author{Yueming Jin}
 \affiliation{School of Electrical and Computer Engineering, 
   \institution{National University of Singapore}
% \streetaddress{P.O. Box 1212}
  \city{}
%  \state{NT}
  \country{Singapore}
%  \postcode{43017-6221}
}
\email{ymjin@nus.edu.sg}

\author{He Tang}
\authornote{Corresponding author: He Tang (E-mail: hetang@hust.edu.cn)}
 \affiliation{School of Software Engineering, 
   \institution{Huazhong University of Science and Technology}
% \streetaddress{P.O. Box 1212}
  \city{Wuhan}
%  \state{NT}
  \country{China}
%  \postcode{43017-6221}
}
\email{hetang@hust.edu.cn}

\author{Pheng-Ann Heng}
%\orcid{0000-0002-2630-2838}
%\authornotemark[1]
 \affiliation{School of Computer Science and Engineering, 
   \institution{The Chinese University of Hong Kong}
% \streetaddress{P.O. Box 1212}
  \city{Shatin}
  \state{NT}
  \country{Hong Kong}
%  \postcode{43017-6221}
}
\email{pheng@cse.cuhk.edu.hk}

%%
%% By default, the full list of authors will be used in the page
%% headers. Often, this list is too long, and will overlap
%% other information printed in the page headers. This command allows
%% the author to define a more concise list
%% of authors' names for this purpose.
% \renewcommand{\shortauthors}{Trovato et al.}






% % % %图1得展示内部区域结构识别清晰的例子(如椅子)对比实验的定性比较也是(pipeline的摩天轮好像看不出内部结构预测的困难,但是展示了trunk和structure的解耦学习)。
% % Figure environment removed



%%
%% The abstract is a short summary of the work to be presented in the
%% article.
\begin{abstract}

High-accuracy Dichotomous Image Segmentation (DIS) aims to pinpoint category-agnostic foreground objects from natural scenes. 
%Unlike high-resolution semantic-specific binary image segmentation, DIS involves identifying the accurate foreground area while rendering detailed object structure.
The main challenge for DIS involves identifying the highly accurate dominant area while rendering detailed object structure. 
However, directly using a general encoder-decoder architecture may result in an oversupply of high-level features and neglect the shallow spatial information necessary for partitioning meticulous structures. 
To fill this gap, we introduce a novel \textbf{Unite-Divide-Unite Network (\ourmodel)} that restructures and bipartitely arranges complementary features to simultaneously boost the effectiveness of trunk and structure identification.
The proposed~\ourmodel~proceeds from several strengths.
First, a dual-size input feeds into the shared backbone to produce more holistic and detailed features while keeping the model lightweight.
Second, a simple \emph{Divide-and-Conquer Module (DCM)} is proposed to decouple multiscale low- and high-level features into our structure decoder and trunk decoder to obtain structure and trunk information respectively.
Moreover, we design a \emph{Trunk-Structure Aggregation module (TSA)} in our union decoder that performs cascade integration for uniform high-accuracy segmentation.
As a result, \ourmodel~performs favorably against state-of-the-art competitors in all six evaluation metrics on overall DIS-TE, \ie, achieving 0.772 weighted F-measure and 977 HCE. Using 1024$\times$1024 input, our model enables real-time inference at 65.3 \emph{fps} with ResNet-18. 
% The source code will be made available.
The source code is available at \sm{https://github.com/PJLallen/UDUN.}

\end{abstract}



%%
%% The code below is generated by the tool at http://dl.acm.org/ccs.cfm.
%% Please copy and paste the code instead of the example below.
%%
% \begin{CCSXML}
% <ccs2012>
%   <concept>
%       <concept_id>10010147</concept_id>
%       <concept_desc>Computing methodologies</concept_desc>
%       <concept_significance>300</concept_significance>
%       </concept>
%   <concept>
%       <concept_id>10010147</concept_id>
%       <concept_desc>Computing methodologies</concept_desc>
%       <concept_significance>500</concept_significance>
%       </concept>
%   <concept>
%       <concept_id>10010147.10010371.10010382.10010383</concept_id>
%       <concept_desc>Computing methodologies~Image processing</concept_desc>
%       <concept_significance>500</concept_significance>
%       </concept>
%  </ccs2012>
% \end{CCSXML}

% \ccsdesc[300]{Computing methodologies}
% \ccsdesc[500]{Computing methodologies}
% \ccsdesc[500]{Computing methodologies~Image processing}
\begin{CCSXML}
<ccs2012>
   <concept>
       <concept_id>10010147.10010178.10010224.10010245.10010247</concept_id>
       <concept_desc>Computing methodologies~Image segmentation</concept_desc>
       <concept_significance>500</concept_significance>
       </concept>
 </ccs2012>
\end{CCSXML}

\ccsdesc[500]{Computing methodologies~Image segmentation}
%%
%% Keywords. The author(s) should pick words that accurately describe
%% the work being presented. Separate the keywords with commas.
\keywords{high-resolution detection, dichotomous image segmentation, fully convolutional network}
%% A "teaser" image appears between the author and affiliation
%% information and the body of the document, and typically spans the
%% page.
% \begin{teaserfigure}
%   % Figure removed
%   \caption{Seattle Mariners at Spring Training, 2010.}
%   \Description{Enjoying the baseball game from the third-base
%   seats. Ichiro Suzuki preparing to bat.}
%   \label{fig:teaser}
% \end{teaserfigure}

% \received{20 February 2007}
% \received[revised]{12 March 2009}
% \received[accepted]{5 June 2009}



% Challenges of high-accuracy dichotomous image segmentation (DIS) task. 
% (a) Comparisons of our UDUN with other representative models in terms of both weighted F-measure and HCE on DIS-TE. The area of each circle corresponds to the number of parameters of the respective model. Our method yields superior performance on both the weighted F-measure and HCE metrics.
% (b) Visual comparison of corresponding metrics. UDUN can take into account both the detailed structure and integrity of targets.

%\begin{teaserfigure}
%  % Figure removed
%  \caption{Challenges of high-accuracy dichotomous image segmentation (DIS) task. (a) It is difficult to obtain high performances in both weighted F-measure and HCE. For instance, IS-Net \cite{qin2022highly} excels in HCE but underperforms in weighted F-measure, while PGNet \cite{xie2022pyramid} shows the opposite pattern, performing well in weighted F-measure but falling behind in HCE. The area of each circle corresponds to the number of parameters of the respective model. (b) The examples of previous works \cite{qin2022highly} \cite{xie2022pyramid} that produce inaccurate segmentations for the objects with complex internal structures. In contrast, the proposed UDUN performs state-of-the-art in terms of both weighted F-measure and HCE.}
%  \Description{E}
%  \label{fig:motivation}
%\end{teaserfigure}

\begin{teaserfigure}
  % Figure removed
  \put(-405,-10){{(a)}}
  \put(-160,-10){{(b)}}
  \caption{Dichotomous image segmentation (DIS) encourages the methods to boost both weighted F-measure and HCE. (a) Comparisons of the proposed~\ourmodel\ with competitors in terms of weighted F-measure, HCE and number of parameters on the DIS-TE test set. The area of each circle represents the number of parameters of the respective model. UDUN achieves superior performance in all metrics. (b) Visual comparison of corresponding metrics. UDUN can take into account both the detailed structure and integrity of targets.}
  \Description{E}
  \label{fig:motivation}
\end{teaserfigure}

% Dichotomous image segmentation (DIS) encourages the methods to boost both weighted F-measure and HCE. (a) Comparisons of the proposed~\ourmodel\ with other competitors in terms of weighted F-measure, human correction efforts (HCE) and number of parameters on the DIS-TE test set. The area of each circle corresponds to the number of parameters of the respective model. Our method achieves superior performance in all metrics. (b) Visual comparison of corresponding metrics. UDUN can take into account both the detailed structure and integrity of targets.

%%
%% This command processes the author and affiliation and title
%% information and builds the first part of the formatted document.
\maketitle

\section{Introduction}
%With the significant improvement in the representational capacity of deep neural networks, modern semantic segmentation \cite{} and class-agnostic segmentation \cite{} methods have obtained high accuracy at predict the general mask of the foreground. However, high-demand scenarios like intelligent healthcare and intelligent construction expect the segmentation results with both trunk and structure preserving. Recently, Qin $\etal$ \cite{qin2022highly} introduced a new category-agnostic dichotomous image segmentation (DIS) task to produce highly accurate mask of foreground area. Moreover, they proposed a new human correction efforts (HCE) metric to evaluate structural accuracy of segmentation other than traditional trunk accuracy metrics like maximal F-measure \cite{achanta2009frequency}, mean absolute error \cite{perazzi2012saliency}, mean enhanced alignment measure \cite{fan2018enhanced}, etc. The study of DIS is promising for extensive applications like medical image analysis \cite{}, AR/VR \cite{}, etc.
With the prominent developments in the representational capacity of deep neural networks, modern semantic segmentation~\cite{shen2022high,guo2022isdnet} and class-agnostic segmentation methods~\cite{tang2021disentangled,pang2022zoom} have yielded high accuracy in covering desired areas of the foreground. 
However, high-demand scenarios like intelligent healthcare and smart construction expect segmentation to preserve both the integrity of trunk and detailed structure~\cite{zhuge2022salient}. As two sides of a coin, the trunk involves in the dominant area and the structure focuses on fine-grained internal and external edges of objects.
Recently, Qin \etal~\cite{qin2022highly} introduced a new class-agnostic dichotomous image segmentation (DIS) task to produce a highly accurate mask of foreground objects.
Moreover, they proposed a new human correction efforts (HCE) metric that measures the difference between predictions and realistic applications by estimating the need for human interventions to calibrate interior structures and exterior boundaries.
It indicates that the HCE metric is more sensitive to the structural refinement of the segmentation map other than traditional accuracy metrics like weighted F-measure~\cite{achanta2009frequency}, mean absolute error~\cite{perazzi2012saliency}, and mean enhanced alignment measure~\cite{fan2018enhanced}. 
Therefore, compared to general task-specific object segmentation~\cite{fan2022salient,fan2021concealed}, DIS has a greater challenge since it demands simultaneous attention to the integrity of the target and the finer structure.
Research into DIS is promising for extensive applications such as robotic intelligence~\cite{zhao2022trasetr}, medical image analysis~\cite{valanarasu2022unext}, and AR/VR applications~\cite{tian2022kine}.

% Figure environment removed

% %图1得展示内部区域结构识别清晰的例子(如椅子)对比实验的定性比较也是(pipeline的摩天轮好像看不出内部结构预测的困难,但是展示了trunk和structure的解耦学习)。
% % Figure environment removed

%Various works \cite{} have been proposed to promote high-accuracy image segmentation with respect to the encoder-decoder framework.
%(Fig. \ref{fig:Cmp}(a)). IS-Net (Fig. \ref{fig:Cmp}(b)) \cite{qin2022highly} trains the network in a two-stage manner with high-dimensional intermediate supervision. PGNet (Fig. \ref{fig:Cmp}(c)) \cite{xie2022pyramid} adopts dual-input and dual-encoder to extract features from different resolution images, then interacts the features with a cross-model grafting module. 
%Though some recently proposed methods, like IS-Net \cite{qin2022highly} and PGNet \cite{xie2022pyramid} have achieved good performance for DIS task, they still suffer from two major drawbacks as shown in Fig. \ref{fig:motivation}: 1) the segmentations with complex internal structures are inaccurate; 2) the weighted F-measure and HCE of non-convex objects can hardly be simultaneously boosted. This can be attributed to: 1) the features of later stages are high-level and lack of details for structure segmentation; 2) segmenting trunk and structure in high-accuracy are conflict in some extent.
Various prominent works~\cite{cheng2020cascadepsp,zhang2021looking} have been spawned to facilitate high-accuracy image segmentation with respect to the encoder-decoder framework. 
IS-Net~\cite{qin2022highly} provides the first solution to the DIS task by cascading multiple U-Net structures~\cite{ronneberger2015u} with intermediate supervision.
PGNet~\cite{xie2022pyramid} adopts a dual-branch architecture and embraces a cross-model grafting module to detect high-resolution salient objects.
Despite the satisfactory performance achieved by previous works, two major issues remain unresolved for high-accuracy DIS:
(1) as demonstrated in~\figref{fig:motivation}(a), it is difficult to simultaneously improve the quality of the holistic segmentation of the target (refer to weighted F-measure) and detailed structural segmentation of non-convex objects (refer to HCE); (2) as shown in~\figref{fig:motivation}(b), high-resolution targets with intricate structures are challenging to segment properly and entirely.
The underlying reasons may be: (i) existing methods often struggle to simultaneously reconcile the dominant area and internal details of the object for high-accuracy segmentation; (ii) features extracted from deeper layers of the FCN-based encoder have a higher level of aggregation, resulting in a lack of low-level structural and spatial information.

%To address the above-mentioned issues, we propose a dual-size input unite-divide-unite network (\ourmodel) to disentangle the trunk and structure segmentation for high-accuracy dichotomous image segmentation. Union encoder is proposed to extract multi-level features with shared backbone for parameter-saving. Thereinto, Divide-and-conquer Module (DCM) splits multi-level features from the dual-size input, then recombine the features to prevent high-level features for structure segmentation. The trunk decoder and structure decoder are designed to segment core mask and interior structures of objects respectively, this design alleviates the conflict between trunk and structure segmentations. Finally, Union Decoder is proposed to unite both structure and trunk prediction for the final dichotomous image segmentation.
To overcome the above-mentioned issues, we propose a dual-size input unite-divide-unite network (\ourmodel) to disentangle the trunk and structure segmentation for high-accuracy DIS. 
Concretely, we introduce a union encoder to extract multi-layer features by feeding larger and smaller inputs to a shared backbone for parameter-saving. 
Thereinto, the divide-and-conquer module (DCM) recombines two groups of multi-level features for efficient divisional optimization. 
The trunk decoder and structure decoder are designed to separately refine and fuse the dominant area and detailed structure information. Finally, a union decoder is proposed to integrate and unify the structure and trunk information for the final dichotomous segmentation. 
The unite-divide-unite strategy effectively alleviates the conflict between trunk and structure segmentation. 
% As shown in Fig. \ref{fig:Cmp}, the network of high-accuracy image segmentation can be build upon a general FCN-based encoder-decoder model (Fig. \ref{fig:Cmp}(a)). Compare with IS-Net \cite{qin2022highly} (Fig. \ref{fig:Cmp}(b)), the proposed \ourmodel (Fig. \ref{fig:Cmp}(d)) requires only one-stage training and not heavily rely on intermediate supervision, which makes \ourmodel more easier training. Different from PGNet \cite{xie2022pyramid} (Fig. \ref{fig:Cmp}(c)), \ourmodel (Fig. \ref{fig:Cmp}(d)) requires single united encoder and with disentangled decoders, which makes the model more parameter-saving and higher accuracy.
\figref{fig:Cmp} shows a comparison of high-accuracy image segmentation architectures built upon a general FCN-based encoder-decoder model (\figref{fig:Cmp}(a)). 
Compared to IS-Net~\cite{qin2022highly} (\figref{fig:Cmp}(b)), the proposed~\ourmodel~(\figref{fig:Cmp}(d)) framework requires only one stage for training and relies not heavily on intermediate supervision, which makes the~\ourmodel~ easier to converge.
Different from PGNet~\cite{xie2022pyramid} (\figref{fig:Cmp}(c)), \ourmodel~ employs a single union encoder with disentangled decoders, which contributes to the efficiency and higher accuracy of our model. 
Extensive experiments demonstrate that~\ourmodel~achieves state-of-the-art performance on the DIS-TE test set~\cite{qin2022highly}, in terms of both the integrity of the object and the fineness of the structure, \eg, 0.831, 0.892, and 977 on the $F_\beta^\text{max}$, $E_{\phi}^\text{m}$, and HCE$_{\gamma}$ respectively.

% Overall framework of the proposed~\ourmodel. \ourmodel is organized as: a union encoder with the divide-and-conquer module (DCM), a structure decoder, a trunk decoder, and a union decoder with the trunk-structure aggregation module (TSA) and the mask-structure aggregation module (MSA).

%贡献
%单阶段的DIS解决方案:解决了前人工作忽略的问题
%我们采用了UEncoder 以减少参数的方式去获取丰富的多尺度的特征
%并且,使用Divide-and-Conquer Module 的策略重组高级特征和低级特征,分别用于预测前景目标的trunk   和 structure。moreover,TSA被设计用来融合两种互补的特征,以此得到高精度的预测图。
%所有DISTE测试集上,我们的方法在六个评价指标上attains a superior performance,并且都达到了sota 和实时的速度with不同的backbone ResNet-18 ResNet-34 ResNet-50。
Our main contributions can be summarized as follows:
\begin{itemize}
    %    \item We propose a dual-size input unite-divide-unite network (\ourmodel) for high-accuracy dichotomous image segmentation. It not only produces finer segmentation for complex objects, but also alleviates the conflict between trunk and structure segmentation.
    \item We propose a novel unite-divide-unite network (\ourmodel) for high-accuracy DIS. This framework contains a union encoder to efficiently obtain rich multiscale features, a trunk and structure decoder to refine and merge multi-level trunk and structure features respectively, and a union decoder to aggregate cross-structural information. \ourmodel~provides finer segmentation of complex objects and joint optimization of trunks and structures.
    % \item UDUN disentangles the trunk and structure segmentation from DIS to boost both accuracies. In this way, we introduce a union encoder to obtain rich multiscale features in a parameter-saving manner, it contains a Divide-and-Conquer Module to recombine high- and low-level features, which are used for predicting the complementary trunk and structure of the foreground, respectively. Moreover, Trunk-Structure Aggregation module is designed to fuse these complementary features, resulting in high-accuracy segmentation maps.
    \item A divide-and-conquer module (DCM) is embedded in our union encoder to recombine high- and low-level features for capturing complementary trunk and structure cues respectively. Moreover, we design a trunk-structure aggregation module (TSA) to sufficiently integrate these complementary features for a unified mask feature.
    \item \ourmodel~achieves state-of-the-art performance in terms of all six metrics on DIS-TE at real-time speed, demonstrating consistent advancements in trunk and structure accuracy.
\end{itemize} 

\section{Related Works}

\subsection{High-resolution Semantic Segmentation}
% 说明高分辨语义分割任务的目的,总结几种相关模型的处理方式(基于全卷积网络模型,由粗到细,级联方式等等)。然而,他们都是基于语义相关的目标来进行精细化分割,但都忽略了在语义区域中结构化的细节。为此,我们提出了一个XX模型,针对结构复杂度较高的目标,可以进一步提取更加精细化的细节特征以达到更高质量的分割。

%高分辨率语义分割是为了在高分辨率的样本上识别出语义相关的目标。
%一些方法,利用了降采样或者裁剪成patch的方法得到低分辨率的特征来高效率的定位语义。另一方面。
%为了直接处理高分辨率样本,一些轻量化的模型[,,]被设计出来以便处理高分辨率特征。然而,前者为了追求语义定位而牺牲了空间细节,后者往往存在感受野不足的问题.
%BiSeNetV1  v2,ISDNet(这些都是实时的,不该提增加了参数量)
% 使用一个浅层分支来提取高分辨率输入低级特征,一个深层分支从低分辨率输入提取高级特征。这些工作采用了双输入的策略来提高准确率并达到实时的速度。
%然而,他们都是基于语义相关的目标来进行精细化分割,但都忽略了在语义区域中结构化的细节。
% 因此,我们设计了一个简洁的双输入高分辨率分割模型,它具有轻量化的结构并可以准确的同时捕捉语义特征和细节特征。

High-resolution semantic segmentation aims to identify the classes of each pixel in high-resolution images. 
Numerous approaches~\cite{zeng2019towards,chen2019collaborative,li2021contexts}  leverage downsampling or patch cropping techniques to efficiently locate semantics using low-resolution features. In order to directly handle high-resolution samples, some lightweight models~\cite{pohlen2017full,romera2017erfnet,chaurasia2017linknet}  have been designed to handle and preserve high-resolution features. However, the former sacrifices spatial details in pursuit of semantic localization, while the latter often suffers from insufficient receptive field. Recently, several methods~\cite{yu2018bisenet,fan2021rethinking,guo2022isdnet} employed a lightweight branch to extract low-level features from the high-resolution input together with a heavyweight branch to extract high-level features from the low-resolution input. 
These works adopted a dual-input strategy to improve accuracy and achieve real-time speeds. 
For example, ISDNet~\cite{guo2022isdnet} integrated deep and shallow networks to construct an efficient segmentation model for remote sensing fields.
% in remote sensing field.
%However, they are all based on semantically-predefined objectives for fine-grained segmentation, while ignoring structural details within semantic regions. Therefore, we propose a concise dual-input high-resolution segmentation model with a lightweight structure that capture high-level semantic features and low-level detail features simultaneously.

Nevertheless, such methods all rely on semantically-defined objectives for fine-grained segmentation, while ignoring structural details within semantic regions. 
Therefore, we develop a concise dual-input high-resolution segmentation model with a lightweight architecture that collaboratively optimizes high-level semantic features and low-level detail features. 

\subsection{High-quality Class-agnostic Segmentation}
% 从High-quality object detection的task展开,说明一下这些任务和模型的特点,然后引出DIS任务(ECCV2022),然后从模型结构的角度说明之前的模型的架构怎么怎么样,与他们不同的是,我们通过两个size的输入通过特征解耦可以同时精确的分割出目标的区域和结构化细节信息。
%类无关的高质量图像分割、抠图等作为一类,突出需要refine  coarse prediction 到high-quality
%注重boundary或texture的一些SOD/COD方法作为一类,如BASNet等,也是突出需要refine,PGNet就不要了,因为说不清楚
% 2.2.1依赖于coarse  prediction   并且  主要关注是边缘局部区域的高质量分割
% eg:   BASNet HQSOD BackgroundMattingV2 将refine集成在网络内部(两个网络) 
% BASNet~\cite{qin2019basnet} 
% BackgroundMattingV2~\cite{lin2021real}  
% HQSOD~\cite{tang2021disentangled}
% 2.2.2直接关注refiner 的 CascadePSP   MGMatting    CRM
% CascadePSP~\cite{cheng2020cascadepsp}
% MGMatting~\cite{yu2021mask}
% CRM~\cite{shen2022high}
% 一些工作~\cite{qin2019basnet,tang2021disentangled,lin2021real}将任务分解为粗定位和优化细节两个任务,具体来说,使用一个网络得到cosrse prediction 紧接着再使用一个网络去refine  error region。为了进一步提高图像分割的质量,一些分割细化技术~\cite{cheng2020cascadepsp,yu2021mask,shen2022high}被提出,他们直接专注于refine  得到一个更高质量的预测图,通过拼接原图和来自其他网络cosrse prediction
% High-quality class-agnostic segmentation predicts both core mask and fine structure of the foreground, without identify the class of each pixel.
% Some works~\cite{qin2019basnet,tang2021disentangled,lin2021real} decompose the task into two sequential tasks: coarse localization and fine-grained optimization. Specifically, they employ one network to obtain the coarse prediction and then use another network to refine the erroneous regions. To further enhance the quality of image segmentation, some refinement techniques~\cite{cheng2020cascadepsp,yu2021mask,shen2022high} have been proposed, which focus directly on refining the prediction to obtain a higher-quality segmentation result. This is achieved by concatenating the original image and the coarse prediction from other networks.
% % However, these methods rely  on the accuracy of the coarse prediction and may amplify errors during the refinement process.
% However, the aforementioned task focuses solely on producing high-quality predictions of foreground boundaries, without considering the fine details and complicated structures within the boundaries. Recently, Qin~\etal ~\cite{qin2022highly} proposed a new task called Highly Accurate Dichotomous Image Segmentation (DIS), which presents challenges to existing refinement strategies. The DIS task demands finer details and has higher structural and edge complexities than high-quality object detection. To tackle these challenges, they developed a novel solution called IS-Net, which optimized the structure of the U$^\text{2}$-Net~\cite{qin2020u2} to handle high-resolution inputs and adopted an intermediate detection strategy to reduce the risk of overfitting.
High-quality class-agnostic segmentation identifies both the task-specific foreground area and fine boundaries of the object, without predicting the class of each pixel~\cite{tang2021disentangled}.
Several works~\cite{qin2019basnet,tang2021disentangled,lin2021real} decomposed the task into two sequential stages: coarse localization and fine-grained optimization.
Generally, they employed one network to obtain coarse predictions and then another network to refine the erroneous regions. 
Toward higher-quality segmentation maps, consequential refinement techniques~\cite{cheng2020cascadepsp,yu2021mask,shen2022high} have been proposed that focus directly on refining coarse predictions. For instance, CRM~\cite{shen2022high} continuously aligned the feature map with the refinement target and aggregated features to reconstruct these image details.

% it can be implemented by concatenating the original image with the coarse prediction from other networks.}

The aforementioned task focuses solely on producing high-quality segmentations of foreground boundaries, without considering the fine details and complicated structures within the object. 
Recently, Qin~\etal~\cite{qin2022highly} proposed a new task, called highly accurate Dichotomous Image Segmentation (DIS). This task requires splitting finer details and more complex internal and external structures. 
To tackle these challenges, they developed a novel solution called IS-Net, which is based on the structure of the U$^\text{2}$-Net~\cite{qin2020u2} optimized to handle the high-resolution input and adopted an intermediate supervision strategy to reduce the risk of overfitting. 
Unlike the coarse-to-fine paradigm~\cite{qin2019basnet,tang2021disentangled,lin2021real} or the two-stage training style~\cite{qin2022highly}, we introduce a one-stage DIS model that can simultaneously boost accuracy in terms of trunk and structure (refer to \figref{fig:Cmp}(d)).
%we propose a novel "unite-divide-unite" approach. Specifically, our method involves aggregating dual-resolution inputs, passing them through a shared backbone, and then separating low-level features and high-level features. The trunk features and structure features are then predicted separately using the respective feature sets, followed by a structure features purify operation that enables precise segmentation of both the region and structured details of the target simultaneously.

% Figure environment removed

\section{Proposed \ourmodel}

\subsection{Overall Architecture}

As illustrated in \figref{fig:model}, the main components of the proposed~\ourmodel\ include:
(1) A union encoder that contains a shared backbone with the dual-size input to efficiently extract multiscale low- and high-level feature representations and a divide-and-conquer module (DCM) to assemble desired features to optimize the input of different decoders. 
(2) A trunk decoder to refine trunk information through multiscale cascade upsampling.
(3) A structure decoder to capture and integrate detailed structure features.
(4) A union decoder that is designed to merge multi-level trunk and structure features to generate the binary mask. Within this, the trunk-structure aggregation module (TSA) is proposed to robustly integrate cross-structure features.

\subsection{Union Encoder}

To better extract the trunk and structure information more adequately, our union encoder adopts a dual-size input to produce multiscale features over a wider range.
In accordance with IS-Net~\cite{qin2022highly}, the larger-size input is set to 1024$\times$1024. 
The input image, denoted as $I_{h}\in\mathbb{R}^{1024\times 1024\times 3}$, is directly resized to obtain a smaller-size input $I_{l}\in\mathbb{R}^{256\times 256\times 3}$. 
To mitigate the computational cost incurred by two resolution inputs, we utilize a shared backbone (defaulted to ResNet-50~\cite{he2016deep}) to extract double group multiscale features, denoted as $\{HRi|i=1, 2, 3, 4, 5\}$ from $I_{h}$ and $\{LRi|i=1, 2, 3, 4, 5\}$ from $I_{l}$. 
It is worth noting that we apply two convolution blocks (1$\times$1 convolution, 3$\times$3 convolution, and batch normalization (BN)) to reduce the channels of $HRi$ and $LRi$ to 32 and 64 respectively for lightening the model parameters.

Considering the tailor-made treatment of structure and trunk cues, it is essential to divide and recombine two sets of features $HRi$ and $LRi$ for joint optimization.
Toward this goal, we introduce a concise divide-and-conquer module (DCM) to reorganize two groups of features. 
As shown in the DCM part of \figref{fig:model}, the main principles of division in DCM include two aspects. 
First, higher-level features with smaller scales are gathered to process trunk information, as they contain more semantic knowledge that is desirable for locating the trunk region of the object. 
Second, identifying object structures requires lower-level features with high resolution to provide more detailed texture information.
Therefore, we redivide features from the backbone into two groups: $\{HR3, HR4, HR5, LR4, LR5\}$ for the trunk decoder and $\{HR1, HR2, LR1, LR2, LR3\}$ for the structure decoder.
Additionally, to supplement larger resolution low-level features without increasing the computational burden, we directly apply a convolution block to the larger-size input $I_{h}$ to obtain the feature $HR0$, without passing it through the backbone. 
The lowest-level feature $HR0$ with a size of 1024 is crucial for the enhanced structure perception in~\ourmodel\ (discussed in \secref{ablation}).

\subsection{Trunk and Structure Decoder}

Following the unite-divide-unite theme, after passing the union encoder, the two sets of recombined features are processed by the structure and trunk decoder respectively to fully exploit holistic trunk information and detailed structure information.
Inspired by~\cite{wei2020label}, we decouple the ground-truth label into the trunk and structure labels for bilateral supervision.

\subsubsection{Trunk Decoder}
As shown in~\figref{fig:model}, the features from deeper layers with 64 channels  $\{Ti|i=1, 2, 3, 4, 5\}$ are taken as input for the trunk decoder. 
Here, a dense cascade fusion strategy is applied to spotlight the trunk location and minimize the loss of spine features. 
Specifically, we perform progressive fusion at each upsampling stage.
For instance, the feature $T1$ is up-sampled by a factor of 2 and passed through a 1$\times$1 convolution block. Meanwhile, $T2$ is also processed through a 1$\times$1 convolution block and then summed with $T1$ to generate $T21$. 
We repeat this process in parallel operations to obtain $T31$, $T41$, and $T51$. 
After four upsampling dense fusions, we reach the unified trunk feature $T54$. 
Finally, the predicted trunk map is obtained by a 3$\times$3 convolution and a sigmoid function, with supervision using the trunk label. In addition, $T32$, $T43$, and $T54$ will be fed into the union decoder for incorporation with multiscale structure features.

\subsubsection{Structure Decoder}
Different from trunk feature fusion, mining structure information requires large-scale holistic features and specific low-level texture cues. 
To this end, the structure decoder receives additional large-scale coarse feature $HR0$ processed directly by a set of 1$\times$1 and 3$\times$3 convolutional blocks without passing through the backbone. 
To reduce the influence of smaller-scale input features $(LR1, LR2, LR3)$ on the identification of exterior details and hollow regions, we adopt a feature filtering operation to concentrate on interior and exterior edges and suppress undesired trunk noise. 
Given the smaller-size feature $LR3$ and the intermediate feature $T32$ from the trunk decoder with the same scale, we first reduce the feature $T32$ from 64 to 32 channels by 1$\times$1 convolution and BN, which is followed by a subtraction operation to obtain the filtered feature $S1$. This operation can be formulated as
\begin{equation}\label{equ:filter}
S1=LR3-\mathcal{C}_{1}(T32),
\end{equation}
where $\mathcal{C}_{1}$ is 1$\times$1 convolution with BN. Similarly, we can obtain the filtered features $S2$ and $S3$. As described at the top of~\figref{fig:model}, we employ a simple upsampling fusion to generate the integrated structure feature $S65$. Each step of the fusion is the same as the operation in the trunk decoder. Correspondingly, the structure label is used to supervise the structure map.

\subsection{Union Decoder}

% Figure environment removed

The union decoder aims to integrate multiscale cross-structural feature maps from both trunk and structure decoders. 
In this regard, we propose the trunk-structure aggregation module (TSA) and the mask-structure aggregation module (MSA) to combine trunk features and structure features of the same scale respectively, followed by layer-by-layer residual fusion to generate the unified feature.

The proposed TSA is developed to supplement detailed structural features while preserving sufficient trunk information. 
The workflow of TSA is illustrated in~\figref{fig:TSA}. The trunk feature $Ti$ is processed by 1$\times$1 convolution and a sigmoid function to produce the attention map. 
This trunk attention map is element-wise multiplied with the structural feature map to highlight border features and suppress background noise.
Then, we apply the residual operation separately to embed $Ti$ and $Si$ to enhance the coherence of the filtered features. 
After passing a 3$\times$3 convolution block, we attain the integrated mask feature $Fi$. 
The entire process of TSA can be formulated as follows:
\begin{equation}
Fi=\mathcal{C}_{3}(\mathcal{C}_{3}(\mathcal{C}_{3}(Si\otimes{Sig(\mathcal{C}_{1}(Ti))})+\mathcal{C}_{1}(Si))+\mathcal{C}_{1}(Ti)),
\end{equation}
where $\mathcal{C}_{3}$ is 3$\times$3 convolution followed by BN and ReLU. $Sig(\cdot)$ denotes the sigmoid function. 
Next, we upsample $Fi$ and add it to the mask feature $F(i+1)$ of the subsequent TSA output. 
When receiving $F3$, the unified feature already contains abundant trunk information. 

To further improve the accuracy of the target structure, we merge the mask feature directly with the structure feature using the MSA module at larger scales. 
Specifically, we replace the trunk feature input with the mask feature while keeping all other operations the same. 
Thanks to MSA, the noise from larger-scale structure features is further suppressed by the guidance of the mask feature. 
In parallel, the mask feature can be enriched with additional detailed information. 
Passing through three MSAs with residual operations, the union decoder yields the integrated mask feature $F$ and generates the final high-accuracy map by two 3$\times$3 convolution blocks.

\subsection{Loss Function}

According to the trunk label supervision, structure label supervision, and mask label supervision in our three decoders, the total loss function of~\ourmodel~can be defined as:
\begin{equation}
\mathcal{L}_{total}=\mathcal{L}_{trunk}+\mathcal{L}_{struc}+\mathcal{L}_{mask},
\end{equation}
where $L_{trunk}$ represents the loss of the trunk map in the trunk decoder and $L_{struc}$ indicates the loss of the structure map in the structure decoder.
Both $L_{trunk}$ and $L_{struc}$ are calculated using the Binary Cross Entropy (BCE) loss $L_{bce}$. 
Additionally, $L_{mask}$ is the loss of supervision on the final prediction map, which consists of the BCE loss and the Intersection over Union (IoU) loss~\cite{qin2019basnet}:
\begin{equation}
\mathcal{L}_{mask}=\mathcal{L}_{bce}+\mathcal{L}_{iou}.
\end{equation}
$L_{iou}$ is employed to enhance the supervision of the overall similarity of the prediction map, which is formulated as:
\begin{equation}
\mathcal{L}_{iou}=1-\frac{\sum_{(x,y)}f(x,y)g(x,y)}{\sum_{(x,y)}[f(x,y)+g(x,y)-f(x,y)g(x,y)]},
\end{equation}
where $f(x,y)$ denotes as the foreground probability of the predicted object and $g(x,y)$ is the ground-truth label. 


\begin{table*}[t!]
\centering
	\caption{Quantitative comparison with 16 representative methods on the DIS5K dataset. $\uparrow$ / $\downarrow$ represents the higher/lower the score, the better. The best three scores are highlighted in \tr{red}, \tg{green}, and \tb{blue}, respectively. R is ResNet~\cite{he2016deep} and R2 is Res2Net~\cite{gao2019res2net}.
	}\label{tab:sota}
    \resizebox{\textwidth}{!}{
	\begin{tabular}{c|r|ccccccccccccccc|cc}
		\hline
        Dataset	&	Metric	&	        	
        \tabincell{c}{BASNet\\\cite{qin2019basnet}}	&	
         \tabincell{c}{GateNet\\\cite{zhao2020suppress}}	&	
        \tabincell{c}{U$^2$Net\\\cite{qin2020u2}}	&
        \tabincell{c}{HRNet\\\cite{wang2020deep}} &
         \tabincell{c}{LDF\\\cite{wei2020label}}	&	
        \tabincell{c}{F$^3$Net\\\cite{wei2020f3net}}	&	
        \tabincell{c}{GCPANet\\\cite{chen2020global}}	&
        \tabincell{c}{SINetV2\\\cite{fan2021concealed}}	&	
        \tabincell{c}{PFNet\\\cite{mei2021camouflaged}}	&	

        \tabincell{c}{MSFNet\\\cite{zhang2021auto}}	&	
        \tabincell{c}{CTDNet\\\cite{zhao2021complementary}}	&
        \tabincell{c}{BSANet\\\cite{zhu2022can}}	&


        \tabincell{c}{ISDNet\\\cite{guo2022isdnet}}	&	
        \tabincell{c}{IFA\\\cite{hu2022learning}}	&	
        \tabincell{c}{PGNet\\\cite{xie2022pyramid}}	&

       \tabincell{c}{IS-Net\\\cite{qin2022highly}}	&	
        \textbf{\ourmodel}	\\

% 注意zoomnet输入512  上采样1024  下采样256
        
        \hline 
        \multirow{4}{*}{\begin{sideways}\tabincell{c}{\textbf{Attr.}}\end{sideways}}

         &	Volume	&	CVPR19  & ECCV20 & PR20  & TPAMI20 & CVPR20 & AAAI20 & AAAI20& TPAMI21 & CVPR21 & MM21  & MM21 & AAAI22 & CVPR22 & ECCV22 & CVPR22 & ECCV22 & - \\

        &	Backbone &	R-50                        & R-50   & -     & -       & R-50   & R-50   & R-50                         & R2-50   & R-50   & R-50  & R-50                         & R2-50  & R-50   & R-50   & R-50                         & -                           & R-50                         \\

        &	Input Size	&	$1024^2$	&	$1024^2$	&	$1024^2$	&	$1024^2$	&	$1024^2$	&	$1024^2$ 	&	$1024^2$ &	$1024^2$		  &  $1024^2$    &$1024^2$    &$1024^2$&	$1024^2$   & $1024^2$	     &	$1024^2$&$1024^2$&	$1024^2$&$1024^2$	\\


        &	Size (MB)	&359.0                       & 515.0  & 176.3 & 264.4   & 101.0  & 102.6  & 268.7                        & 108.5   & 186.6  & 113.9 & 98.9                         & 131.1  & 111.5  & 111.4  & 150.1                        & 176.6                       & 100.2                        \\


        &	FPS    		&79.0                        & 76.7   & 50.0  & 24.7    & 72.6   & 61.0   & 59.0                         & 50.1    & 43.9   & 64.0  & 78.0                         & 35.0   & 78.6   & 33.7   & 64.3                         & 51.3                        & 45.5                         \\


       
\hline 
        
\multirow{6}{*}{\begin{sideways}\textbf{DIS-VD}\end{sideways}}	



&	$maxF_\beta\uparrow$ &  0.737                       & 0.790  & 0.753 & 0.726   & 0.780  & 0.783  & {\cellcolor{myGreen!30} 0.798} & 0.748   & 0.793  & 0.714 & {\cellcolor{myBlue!30} 0.795} & 0.738  & 0.763  & 0.749  & {\cellcolor{myGreen!30} 0.798} & 0.791                       & {\cellcolor{myRed!30} 0.823} \\
&	$F^w_\beta\uparrow$ &0.656                       & 0.716  & 0.656 & 0.641   & 0.715  & 0.714  & {\cellcolor{myGreen!30} 0.736} & 0.632   & 0.724  & 0.620 & 0.729                        & 0.615  & 0.691  & 0.653  & {\cellcolor{myBlue!30} 0.733} & 0.717                       & {\cellcolor{myRed!30} 0.763} \\
&	$~M~\downarrow$ &0.094                       & 0.072  & 0.089 & 0.095   & 0.071  & 0.072  & {\cellcolor{myGreen!30} 0.066} & 0.093   & 0.075  & 0.105 & {\cellcolor{myBlue!30} 0.067} & 0.100  & 0.080  & 0.088  & {\cellcolor{myBlue!30} 0.067} & 0.074                       & {\cellcolor{myRed!30} 0.059} \\
&	$S_{\alpha}\uparrow$&0.781                       & 0.816  & 0.785 & 0.767   & 0.813  & 0.813  & {\cellcolor{myGreen!30} 0.826} & 0.796   & 0.818  & 0.759 & 0.823                        & 0.786  & 0.803  & 0.785  & {\cellcolor{myBlue!30} 0.824} & 0.813                       & {\cellcolor{myRed!30} 0.838} \\
&	$E_{\phi}^{m}\uparrow$&0.809                       & 0.868  & 0.809 & 0.824   & 0.861  & 0.860  & {\cellcolor{myBlue!30} 0.878} & 0.814   & 0.868  & 0.800 & 0.873                        & 0.807  & 0.852  & 0.829  & {\cellcolor{myGreen!30} 0.879} & 0.856                       & {\cellcolor{myRed!30} 0.892} \\
&	$HCE_\gamma\downarrow$&{\cellcolor{myBlue!30} 1132 }                      & 1189   & 1139  & 1 560   & 1314   & 1291   & 1264                         & 1756    & 1350   & 1376  & 1354                         & 1731   & 1549   & 1437   & 1326                         & {\cellcolor{myGreen!30} 1116} & {\cellcolor{myRed!30} 1097}  \\




	\hline 
\multirow{6}{*}{\begin{sideways}\textbf{DIS-TE1}\end{sideways}}

&	$maxF_\beta\uparrow$ &  0.663                       & 0.737  & 0.701 & 0.668   & 0.727  & 0.726  & {\cellcolor{myBlue!30} 0.741} & 0.695   & 0.740  & 0.658 & 0.738                        & 0.683  & 0.717  & 0.673  & {\cellcolor{myGreen!30} 0.754} & 0.740                       & {\cellcolor{myRed!30} 0.784} \\
&	$F^w_\beta\uparrow$ &0.577                       & 0.654  & 0.601 & 0.579   & 0.655  & 0.655  & {\cellcolor{myBlue!30} 0.676} & 0.568   & 0.665  & 0.559 & 0.668                        & 0.545  & 0.643  & 0.573  & {\cellcolor{myGreen!30} 0.680} & 0.662                       & {\cellcolor{myRed!30} 0.720} \\
&	$~M~\downarrow$ &0.105                       & 0.076  & 0.085 & 0.088   & 0.074  & 0.076  & {\cellcolor{myBlue!30} 0.070} & 0.088   & 0.075  & 0.101 & 0.072                        & 0.098  & 0.077  & 0.088  & {\cellcolor{myGreen!30}0.067} & 0.074                       & {\cellcolor{myRed!30} 0.059} \\
&	$S_{\alpha}\uparrow$&0.741                       & 0.786  & 0.762 & 0.742   & 0.783  & 0.783  & {\cellcolor{myBlue!30} 0.797} & 0.767   & 0.791  & 0.734 & 0.792                        & 0.754  & 0.782  & 0.746  & {\cellcolor{myGreen!30} 0.800} & 0.787                       & {\cellcolor{myRed!30} 0.817} \\
&	$E_{\phi}^{m}\uparrow$&0.756                       & 0.826  & 0.783 & 0.797   & 0.822  & 0.820  & 0.834                        & 0.785   & 0.830  & 0.771 & {\cellcolor{myBlue!30} 0.837} & 0.773  & 0.824  & 0.785  & {\cellcolor{myGreen!30} 0.848} & 0.820                       & {\cellcolor{myRed!30} 0.864} \\
&	$HCE_\gamma\downarrow$&155                         & 164    & 165   & 262     & 155    & 150    & {\cellcolor{myGreen!30} 145}   & 320     & 164    & 203   & 162                          & 314    & 214    & 229    & 162                          & {\cellcolor{myBlue!30} 149}  & {\cellcolor{myRed!30} 140}   \\


	\hline 
\multirow{6}{*}{\begin{sideways}\textbf{DIS-TE2}\end{sideways}}	

&	$maxF_\beta\uparrow$ &  0.738                       & 0.795  & 0.768 & 0.747   & 0.784  & 0.789  & {\cellcolor{myBlue!30} 0.799} & 0.761   & 0.796  & 0.736 & {\cellcolor{myBlue!30} 0.799} & 0.752  & 0.783  & 0.758  & {\cellcolor{myGreen!30} 0.807} & 0.799                       & {\cellcolor{myRed!30} 0.829} \\
&	$F^w_\beta\uparrow$ &0.653                       & 0.721  & 0.676 & 0.664   & 0.719  & 0.719  & {\cellcolor{myBlue!30} 0.741} & 0.646   & 0.729  & 0.642 & 0.731                        & 0.628  & 0.714  & 0.666  & {\cellcolor{myGreen!30} 0.743} & 0.728                       & {\cellcolor{myRed!30} 0.768} \\
&	$~M~\downarrow$ &0.096                       & 0.074  & 0.083 & 0.087   & 0.073  & 0.075  & {\cellcolor{myBlue!30} 0.068} & 0.090   & 0.073  & 0.096 & 0.070                        & 0.098  & 0.072  & 0.085  & {\cellcolor{myGreen!30} 0.065} & 0.070                       & {\cellcolor{myRed!30} 0.058} \\
&	$S_{\alpha}\uparrow$&0.781                       & 0.818  & 0.798 & 0.784   & 0.813  & 0.814  & {\cellcolor{myBlue!30} 0.830} & 0.805   & 0.821  & 0.780 & 0.823                        & 0.794  & 0.817  & 0.793  & {\cellcolor{myGreen!30} 0.833} & 0.823                       & {\cellcolor{myRed!30} 0.843} \\
&	$E_{\phi}^{m}\uparrow$&0.808                       & 0.864  & 0.825 & 0.840   & 0.862  & 0.860  & {\cellcolor{myBlue!30} 0.874} & 0.823   & 0.866  & 0.816 & 0.872                        & 0.815  & 0.865  & 0.835  & {\cellcolor{myGreen!30} 0.880} & 0.858                       & {\cellcolor{myRed!30} 0.886} \\
&	$HCE_\gamma\downarrow$&{\cellcolor{myBlue!30} 341}  & 368    & 367   & 555     & 370    & 358    & 345                          & 672     & 389    & 456   & 382                          & 660    & 494    & 479    & 375                          & {\cellcolor{myGreen!30} 340}  & {\cellcolor{myRed!30} 325}   \\


    \hline 
\multirow{6}{*}{\begin{sideways}\textbf{DIS-TE3}\end{sideways}}	

&	$maxF_\beta\uparrow$ &  0.790                       & 0.835  & 0.813 & 0.784   & 0.828  & 0.824  & {\cellcolor{myGreen!30} 0.844} & 0.791   & 0.835  & 0.763 & 0.838                        & 0.783  & 0.817  & 0.797  & {\cellcolor{myBlue!30} 0.843} & 0.830                       & {\cellcolor{myRed!30} 0.865} \\
&	$F^w_\beta\uparrow$ &0.714                       & 0.769  & 0.721 & 0.700   & 0.770  & 0.762  & {\cellcolor{myGreen!30} 0.789} & 0.676   & 0.771  & 0.674 & 0.778                        & 0.660  & 0.747  & 0.705  & {\cellcolor{myBlue!30} 0.785} & 0.758                       & {\cellcolor{myRed!30} 0.809} \\
&	$~M~\downarrow$ &0.080                       & 0.062  & 0.073 & 0.080   & 0.061  & 0.063  & {\cellcolor{myGreen!30} 0.056} & 0.084   & 0.062  & 0.089 & {\cellcolor{myBlue!30} 0.059} & 0.090  & 0.065  & 0.077  & {\cellcolor{myGreen!30} 0.056} & 0.064                       & {\cellcolor{myRed!30} 0.050} \\
&	$S_{\alpha}\uparrow$&0.816                       & 0.847  & 0.823 & 0.805   & 0.844  & 0.841  & {\cellcolor{myGreen!30} 0.855} & 0.823   & 0.847  & 0.793 & {\cellcolor{myBlue!30} 0.850} & 0.814  & 0.834  & 0.815  & 0.844                        & 0.836                       & {\cellcolor{myRed!30} 0.865} \\
&	$E_{\phi}^{m}\uparrow$&0.848                       & 0.901  & 0.856 & 0.869   & 0.896  & 0.892  & {\cellcolor{myBlue!30} 0.909} & 0.845   & 0.901  & 0.845 & 0.903                        & 0.840  & 0.893  & 0.861  & {\cellcolor{myGreen!30}0.911} & 0.883                       & {\cellcolor{myRed!30} 0.917} \\
&	$HCE_\gamma\downarrow$&{\cellcolor{myGreen!30} 681}  & 737    & 738   & 1049    & 782    & 767    & 742                          & 1219    & 816    & 901   & 807                          & 1204   & 994    & 937    & 797                          & {\cellcolor{myBlue!30} 687}  & {\cellcolor{myRed!30} 658}   \\



	\hline 
 
\multirow{6}{*}{\begin{sideways}\textbf{DIS-TE4}\end{sideways}}	

&	$maxF_\beta\uparrow$ &  0.785                       & 0.826  & 0.800 & 0.772   & 0.818  & 0.815  & {\cellcolor{myGreen!30} 0.831} & 0.763   & 0.816  & 0.743 & 0.826                        & 0.757  & 0.794  & 0.790  & {\cellcolor{myGreen!30} 0.831} &   {\cellcolor{myBlue!30} 0.827}                     & {\cellcolor{myRed!30} 0.846} \\
&	$F^w_\beta\uparrow$ &0.713                       & 0.766  & 0.707 & 0.687   & 0.762  & 0.753  & {\cellcolor{myGreen!30} 0.776} & 0.649   & 0.755  & 0.660 & 0.766                        & 0.640  & 0.725  & 0.700  & {\cellcolor{myBlue!30} 0.774} & 0.753                       & {\cellcolor{myRed!30} 0.792} \\
&	$~M~\downarrow$ &0.087                       & 0.067  & 0.085 & 0.092   & 0.067  & 0.070  & {\cellcolor{myGreen!30} 0.064} & 0.101   & 0.072  & 0.102 & 0.066                        & 0.107  & 0.079  & 0.085  & {\cellcolor{myBlue!30} 0.065} & 0.072                       & {\cellcolor{myRed!30} 0.059} \\
&	$S_{\alpha}\uparrow$&0.806                       & 0.839  & 0.814 & 0.792   & 0.832  & 0.826  & {\cellcolor{myGreen!30} 0.841} & 0.799   & 0.831  & 0.775 & {\cellcolor{myBlue!30} 0.840} & 0.794  & 0.815  & 0.811  & {\cellcolor{myGreen!30} 0.841} & 0.830                       & {\cellcolor{myRed!30} 0.849} \\
&	$E_{\phi}^{m}\uparrow$&0.844                       & 0.895  & 0.837 & 0.854   & 0.888  & 0.883  &{\cellcolor{myBlue!30}  0.898}                        & 0.816   & 0.885  & 0.825 & 0.895                        & 0.815  & 0.873  & 0.847  & {\cellcolor{myGreen!30} 0.899} & 0.870                       & {\cellcolor{myRed!30} 0.901} \\
&	$HCE_\gamma\downarrow$&{\cellcolor{myGreen!30} 2852} & 2965   & 2898  & 3864    & 3364   & 3291   & 3229                         & 4050    & 3391   & 3425  & 3447                         & 4014   & 3760   & 3554   & 3361                         & {\cellcolor{myBlue!30} 2888} & {\cellcolor{myRed!30} 2785}  \\


        \hline 
	\hline
\multirow{6}{*}{\begin{sideways}\tabincell{c}{\textbf{Overall}\\\textbf{DIS-TE (1-4)}}\end{sideways}}

&	$maxF_\beta\uparrow$ &  0.744                       & 0.798  & 0.771 & 0.743   & 0.789  & 0.789  & {\cellcolor{myBlue!30} 0.804} & 0.753   & 0.797  & 0.725 & 0.800                        & 0.744  & 0.778  & 0.755  & {\cellcolor{myGreen!30} 0.809} & 0.799                       & {\cellcolor{myRed!30} 0.831} \\
&	$F^w_\beta\uparrow$ &0.664                       & 0.728  & 0.676 & 0.658   & 0.727  & 0.722  & {\cellcolor{myGreen!30} 0.746} & 0.635   & 0.730  & 0.634 & {\cellcolor{myBlue!30} 0.736} & 0.618  & 0.707  & 0.661  & {\cellcolor{myGreen!30} 0.746} & 0.726                       & {\cellcolor{myRed!30} 0.772} \\
&	$~M~\downarrow$ &0.092                       & 0.070  & 0.082 & 0.087   & 0.069  & 0.071  & {\cellcolor{myBlue!30} 0.065} & 0.091   & 0.071  & 0.097 & 0.067                        & 0.098  & 0.073  & 0.084  & {\cellcolor{myGreen!30} 0.063} & 0.070                       & {\cellcolor{myRed!30} 0.057} \\
&	$S_{\alpha}\uparrow$&0.786                       & 0.823  & 0.799 & 0.781   & 0.818  & 0.816  & {\cellcolor{myGreen!30} 0.831} & 0.799   & 0.823  & 0.771 & 0.826                        & 0.789  & 0.812  & 0.791  & {\cellcolor{myBlue!30} 0.830} & 0.819                       & {\cellcolor{myRed!30} 0.844} \\
&	$E_{\phi}^{m}\uparrow$&0.814                       & 0.872  & 0.825 & 0.840   & 0.867  & 0.864  & {\cellcolor{myBlue!30} 0.879} & 0.817   & 0.871  & 0.814 & 0.877                        & 0.811  & 0.864  & 0.832  & {\cellcolor{myGreen!30}0.885} & 0.858                       & {\cellcolor{myRed!30} 0.892} \\
&	$HCE_\gamma\downarrow$&{\cellcolor{myGreen!30} 1007} & 1059   & 1042  & 1432    & 1167   & 1141   & 1115                         & 1565    & 1190   & 1246  & 1200                         & 1548   & 1365   & 1299   & 1173                         & {\cellcolor{myBlue!30}1016} & {\cellcolor{myRed!30} 977}  \\


        \hline
        
	\end{tabular}
	}
\end{table*}

% &\cellcolor{myBlue!30} 1009      &\cellcolor{myGreen!30} 999   &\cellcolor{myRed!30}977  \\


\section{Experiments}

\subsection{Datasets and Evaluation Metrics}

\subsubsection{Datasets}  
%DIS5K Datasets.   DIS5K有5470张图像,包好三个子集:dis tr(3000)、DIS-VD(470)和DIS-TE(2000),分别用于训练、验证和测试。根据数据集的对象形状和结构复杂性,2000张DIS-TE图像被进一步分为四个子集(DIS-TE1∼DIS-TE4),形状复杂性(结构复杂度和边界复杂性的乘法)递增,每个子集中有500张图像,代表四个测试难度级别。
%The DIS5K dataset consists of 5470 images, which are divided into three subsets: DIS-TR (3000 images) for training, DIS-VD (470 images) for validation, and DIS-TE (2000 images) for testing. Based on the shape and structural complexity of the objects in the dataset, the 2000 images in DIS-TE are further divided into four subsets (DIS-TE1 to DIS-TE4), with increasing shape complexity (multiplication of structural complexity and boundary complexity). Each subset contains 500 images, representing four levels of testing difficulty.
We conduct experiments on DIS5K~\cite{qin2022highly} with a total of 5,470 images, which are divided into three subsets: DIS-TR (3,000 images for training), DIS-VD (470 images for validation), and DIS-TE (2,000 images for testing). 
Depending on the shape and structural complexity of the objects, DIS-TE are further divided into four subsets (DIS-TE1 to DIS-TE4). 
Each subset contains 500 images, representing four levels of testing difficulty.
 
 %为了从不同角度评估高精度二义分割的性能,我们采用五种广泛使用的指标以及HCE来评估所有比较方法: 包括最大↑)[2],测量(Fβ mx加权↑)[2],测量(Fβ↑)[64],平均绝对误差(M↓)[73],结构测量(Sα↑)[22],意味着增强对齐测量(Eφ↑)[23,25]和人工矫正量(HCEγ↓),作为前面5个指标的补充,HCE近似于纠正错误预测所需的人工努力,以满足现实应用中的特定精度要求,它对目标的细节和整体性非常敏感,其中γ代表误差容忍度(表示被忽略的小故障区域的大小),我们设置γ = 5和[xx]保持一致 

\subsubsection{Evaluation Metrics}  
% In order to evaluate the performance of high- accuracy Dichotomous segmentation from different perspectives, we adopt five commonly used metrics as well as HCE (Human Correction Efforts) to evaluate all compared methods: including maximal F-measure ($F_\beta^\text{max}\uparrow$) [2], weighted F-measure ($F_\beta^\omega \uparrow$) [64], mean absolute error $(M\downarrow$) [73], structural measure ($S_\alpha \uparrow$ ) [22], mean enhanced alignment measure ($E_{\phi}^\text{m} \uparrow$) [23, 25] and human correction efforts ($HCE_{\gamma} \downarrow$). As a complement to the previous five metrics, HCE approximates the human effort required to correct prediction errors in order to meet specific accuracy requirements in real-world applications. It is highly sensitive to the details and integrity of the object. Here, $\gamma$ represents the error tolerance (indicating the size of small error regions to be ignored), and we set $\gamma$ = 0.5 to be consistent with [xx].
To comprehensively assess the performance of high-accuracy DIS, we employ a total of six evaluation metrics to evaluate compared models in terms of both foreground region accuracy and structural granularity.
To assess the accuracy of foreground areas, we adopt five metrics that are widely used in category-agnostic segmentation tasks~\cite{wang2021salient, fan2021concealed}. These metrics include maximal F-measure ($F_\beta^\text{max}\uparrow$)~\cite{achanta2009frequency}, weighted F-measure ($F_\beta^\omega\uparrow$)~\cite{margolin2014evaluate}, Mean Absolute Error (MAE, $M\downarrow$)~\cite{perazzi2012saliency}, Structural measure (S-measure, $S_\alpha\uparrow$)~\cite{fan2017structure}, and mean Enhanced alignment measure (E-measure, $E_{\phi}^\text{m}\uparrow$)~\cite{fan2018enhanced}. 
For further evaluating
 the quality of structural details and integrity, 
following~\cite{qin2022highly}, we also utilize Human Correction Efforts (HCE$_{\gamma}\downarrow$) to specifically focus on the details and integrity of the object. 
Here, $\gamma$ represents the error tolerance, which is set to 5 to maintain consistent with~\cite{qin2022highly}.

\subsection{Implementation Details} 
%该模型在DIS-TR上进行了训练在DIS-VD上验证,并在上述4个测试集上进行了测试。我们使用 Pytorch [25]来实现我们的模型,一个RTX 3090 gpu用于训练.对于数据增强,我们使用水平翻转,随机裁剪。在ImageNet上预先训练的ResNet-50用于初始化主干,其他参数被随机初始化。值得注意的是,尺度为1024的特征是独立于backbone的分支,它直连解码器,没有预训练权重。我们将ResNet-50主干的最大学习率设置为0.005,将其他部分设置为0.05。采用了热身策略和线性衰减策略。采用随机梯度下降法(SGD)对整个网络进行端到端训练。动量和重量衰减分别设置为0.9和0.0005。批数大小设置为8,最大历元设置为48。训练和测试期间,每张图像都是简单地调整大小到1024 x 1024x3和256 x 256 x3,得到双尺度输入,然后输入网络,无需进行任何后处理即可获得预测.
% The model was trained on DIS-TR and validated on DIS-VD, and tested on the aforementioned DIS-TE1 to DIS-TE4. We implemented our model using PyTorch and trained it on an RTX 3090 GPU. For data augmentation, we use horizontal flipping and random cropping. ResNet-X (e.g., ResNet-18, ResNet-34, ResNet-50)~\cite{he2016deep}, pretrained on ImageNet, is used to initialize the backbone, while other parameters are randomly initialized. It's worth noting that the feature at a scale of 1024 is a separate branch from backbone and directly connected to the decoder, without any pre-trained weights. The maximum learning rate for the ResNet-X backbone is set to 0.005, while other parts are set to 0.05. Warm-up strategy and linear decay strategy are employed. Stochastic gradient descent (SGD) is used for end-to-end training of the entire network. Momentum and weight decay are set to 0.9 and 0.0005 respectively. Batch size is set to 8, and the maximum number of epochs is set to 48. During training and testing, each image is simply resized to 1024$\times$ 1024$\times$ 3 and 256$\times$ 256$\times$ 3, obtaining the dual-si'ze input, and then fed into the network to obtain predictions  without any post-processing.

Experiments are implemented in PyTorch on a single RTX 3090 GPU. 
During the training phase, the images are resized to both 1024$\times$1024 and 256$\times$256 to create the dual-size input. 
Horizontal flipping and random cropping are applied for data augmentation. 
The ResNet~\cite{he2016deep} is used as the backbone with the pre-trained weights on ImageNet~\cite{deng2009imagenet}, while other parameters of~\ourmodel~are initialized randomly. 
Notably, the low-level feature at a scale of 1024 is processed directly by a set of convolutional operations without any pre-trained weights. 
The maximum learning rate for the backbone is set to 0.005, while other parts are set to 0.05. 
To optimize the training process, we utilize the Stochastic gradient descent (SGD) optimizer with a warm-up strategy and linear decay strategy. 
The batch size is set to 8, and the maximum number of epochs is set to 48. 
During inference, the input image with a scale of 1024$\times$1024 is automatically resized to 256$\times$256 to form a double-size input, which is then fed into the network for prediction without any post-processing.

\subsection{Comparison with State-of-the-arts}
%这里说后续加的几个sota 能处理高分辨率的都处理成1024 r50?

% 我们和18个不同....

% 值得注意的是,我们在~\cite{qin2022highly}establish DIS benchmark 基础上,新增加了最新的几个SOTA,比如LDF~\cite{wei2020label}(CVPR2020),PGNet~\cite{xie2022pyramid}(CVPR2022)
% Zoom-Net~\cite{pang2022zoom}(CVPR2022)
% IFA~\cite{hu2022learning}(ECCV2022) and ISDNet~\cite{guo2022isdnet}(CVPR2022),为了公平比较,对于如果可以直接处理高分辨率输入的模型,我们将模型最大输入尺度统一为1024,对于多encoder 网络最重的backbone使用resnet50 。


% Tab 1 shows the quantitative results.AS we can seen,
% 我们的方法在六个评价指标上产生最佳的性能,特别的,对比专门为DIS任务设计的ISnet 我们的三版本(with不同backbone)的模型在overall DIS-TE(1-4) 都取得了更好的性能,其中使用Resnet18的那个ones,模型大小只有49MB,而ISNet达到了176.6MB.
% 关键的是,从表格中可以发现以往的模型在前5个指标和HCE指标的性能冲突,往往不能同时提升,而我们的模型Joint Boosting Structure and Trunk精度,使得所有指标都有一个显著提升。
% 从表中还可以看出,ZOOMNET和PGNet在S-measure 也上展示出比较优异的精度,这验证了多尺度输入对结构完整度的必要性。对比他们,我们采用union encoder的方式可以更全面提升所有指标精度,以更少的参数。
\subsubsection{Quantitative Comparisons}
%Notably, based on the DIS benchmark established in~\cite{qin2022highly}, we have added several latest SOTA models, such as LDF~\cite{wei2020label} (CVPR2020), PGNet~\cite{xie2022pyramid} (CVPR2022), ZoomNet~\cite{pang2022zoom} (CVPR2022), IFA~\cite{hu2022learning} (ECCV2022) and ISDNet~\cite{guo2022isdnet} (CVPR2022). 
%For fair comparison, if the models were capable of directly processing high-resolution inputs, we standardized the maximum input size of the models to 1024 $\times$1024. For multi-encoder networks, we used ResNet50 as the heaviest backbone.
We compare the proposed \ourmodel~with the DIS-only model IS-Net~\cite{qin2022highly} and other 15 well-known task-related models, including BASNet~\cite{qin2019basnet}, GateNet~\cite{zhao2020suppress}, $\rm U^\text{2}$Net~\cite{qin2020u2}, HRNet~\cite{wang2020deep}, LDF~\cite{wei2020label}, $\rm F^\text{3}Net$~\cite{wei2020f3net}, GCPANet~\cite{chen2020global}, SINet-V2~\cite{fan2021concealed}, PFNet~\cite{mei2021camouflaged}, MSFNet~\cite{zhang2021auto}, CTDNet~\cite{zhao2021complementary},   BSANet~\cite{zhu2022can}, ISDNet~\cite{guo2022isdnet}, IFA~\cite{hu2022learning}, and PGNet~\cite{xie2022pyramid}.
For a fair comparison, we standardize the input size of the comparison models to 1024$\times$1024. 
Besides, most models use the ResNet-50~\cite{he2016deep} as the default backbone with ImageNet~\cite{deng2009imagenet} pre-trained weights except for those using Res2Net-50~\cite{gao2019res2net} (\eg, SINetV2 and BSANet) and random initialization (\eg, IS-Net, $\rm U^\text{2}$-$Net$, and HRNet). 
% All models are trained on the DIS-TR set and evaluated on the sets of DIS-VD, DIS-TE1, DIS-TE2, DIS-TE3, DIS-TE4, and overall DIS-TE.

The comparison results are shown in the \tabref{tab:sota}. We can explicitly observe that our UDUN outperforms the compared models by a large margin across all test sets. In particular, we gain significant improvements in five general metrics compared to the second-place performance on the overall DIS-TE, \ie, 2.2\% on the $F_\beta^\text{max}$, 2.6\% on the $F_\beta^\omega$, 0.6\% on the $M$, 1.3\% on the $S_\alpha$, and 0.7\% on the $E_{\phi}^\text{m}$. This achievement is mainly attributed to the proposed unite-divide-unite strategy for precise identification of the integrity of targets. More importantly, for the HCE metric, focusing on a detailed structure assessment of predictions, our model also outperforms the current best performer, BASNet, by about 30 points. It demonstrates the ample extraction of structure information in our structure decoder and the effective integration in our union decoder.

\subsubsection{Qualitative Comparisons}
%得挑一下可视化结果,各个sota的  可以用python脚本GPT is ok     展示FPFN 区域  统计HCE(FP FN)
% image  GT our  IS-Net  PGnet   IFAnet  ISDnet
% 我们的模型不仅可以预测出trunk 区域  还可以保持结构的完整性,特别是interior structure,如图中的椅子....啥的
\figref{fig:visual_comp} exhibits visual comparisons of our method with four cutting-edge methods. 
On a macroscopic level, \ourmodel~provides a more complete segmentation of high-resolution targets than other competitors. 
On a microscopic level, our model is capable of handling complex structures and slender areas with higher accuracy. 
More visualization results can be found in the \sm{supplementary material}. 

% % Figure environment removed

% Figure environment removed




% % Figure environment removed

\subsection{Ablation Studies}\label{ablation}
%我们在DIS-VD数据集上评估评估每个提出的组件。采用Fβ max,MAE,HCE作为主要指标。结果分析如下。
To evaluate the effectiveness of each design of our~\ourmodel, we conduct a range of ablations on the DIS-VD validation set.

\subsubsection{Effectiveness of DCM}
%As widely acknowledged, an encoder such as ResNet extracts features R1-R5, which range from low-level to high-level and decrease in resolution. Large-scale low-level features typically contain texture and structural information, while small-scale high-level features usually capture semantic and contextual information. In a classical decoder, these features R1-R5 are directly processed, resulting in the decoder needing to simultaneously handle both high-level and low-level features. However, a single set of parameters may not efficiently perceive both low-level and high-level features after the encoder has obtained high-level semantics. For tasks like Dichotomous Image Segmentation (DIS) that require high structural accuracy, this can lead to neglect of internal structural details of the objects, thereby impacting the quality of the interior structure.
%Our DCM strategy involves separating and reassembling two sets of features with different resolutions and levels to allow the trunk decoder to solely process high-level features (e.g., R3-R5) for perceiving the network's trunk, while the structure decoder exclusively handles low-level features (e.g., R1-R2) for perceiving the network's structure.
%The DCM approach effectively retains the accuracy of both trunk and structure by disentangling low-level and high-level features. The results, as shown in Table 5, demonstrate the effectiveness of DCM in preserving the accuracy of both trunk and structure.
%DCM的作用是将两组从共享backbone中输出的multiscale cross-level特征进行分离重组。分离出的较低尺寸的高级别特征适用于trunk解码器,较高尺寸的低级别特征适配于structure解码器。如表3所示,将特征进行重组可以有针对性的refine特征以提高模型的性能。
The role of DCM is to divide and reassemble the two groups of multiscale cross-level features produced by the shared backbone. 
The decoupled higher-level features with lower sizes are suitable for our trunk decoder, whereas the lower-level features with higher sizes are appropriate for our structure decoder. 
% Referring to \tabref{tab:dcm}, regrouping features through DCM results in effective refinement and integration of decoupled features, thereby enhancing the accuracy of prediction maps.
Referring to \tabref{tab:dcm}, the regrouping of features by DCM effectively improves the capabilities of subsequent decoders.

\begin{table}
\footnotesize
  \renewcommand{\arraystretch}{1}
  \renewcommand{\tabcolsep}{2.8mm}
  \caption{Effectiveness of DCM in union encoder.}
  \label{tab:dcm}
  \begin{tabular}{c|cccccc}
    \toprule
    DCM & $F_\beta^\text{max} \uparrow$ &  $F_\beta^\omega \uparrow$ & $M \downarrow$ &
    $S_\alpha \uparrow$ & $E_{\phi}^\text{m} \uparrow$ & $HCE_{\gamma} \downarrow$ \\
    \midrule
       &0.815 & 0.753 & 0.064 & 0.833 & 0.886 & 1112\\
    \rowcolor[RGB]{235,235,235}
     \checkmark &\textbf{0.823} & \textbf{0.763} & \textbf{0.059} & \textbf{0.838} & \textbf{0.892} & \textbf{1097} \\
  \bottomrule
\end{tabular}
\end{table}

%使用HR作为structural  feature    LR作为Trunk feature  作为对比

\subsubsection{Trunk and Structure Decoders}
% To validate the effectiveness of our Structure decoder,  we set up a decoder named sum with an Element-wise Sum way like\cite{liu2021scg}to fusion multiscale Structure feature as a comparison group.
% More specifically, an addition operation is performed on all features, which are upsampled to match the resolution of the highest resolution feature. A convolution operation is then applied to obtain the structure map, which is supervised by the structure map of the ground truth.
% The results are shown in the first row of Table 1.
% Similarly, we replaced the trunk decoder with a sum-decoder to verify its performance, and the results are shown in the second row of Table 1. The results of our~\ourmodel using both trunk decoder and structure decoder are shown in the last row.
% In order to visually demonstrate the trunk and structure decoders, as well as the features identified by our model, we visualized the last layer features of the trunk decoder, structure decoder, and Union Decoder (UFD). The visualization results can be found in Figure 3.
% According to Figure 3(b), it can be observed that the structure decoder focuses on the overall fine structure of the Ferris wheel, while the trunk region of the Ferris wheel is captured by the trunk decoder, as shown in Figure 3(c). Furthermore, as shown in Figure 3(d), the Union Decoder effectively integrates the features captured by both the trunk and structure decoders, demonstrating its efficient fusion capability.
%为了验证提出的trunk和结构解码器的有效性,我们在表3中进行了消融实验。值得注意的是在该实验中我们直接对所有输入的特征上采样并相加来融合多尺度特征以代替相应的解码器。如表3所示,trunk解码器和结构解码器都对结果的提升有显著的帮助。同时使用两个解码器后,我们的模型达到了更好的性能。为了更加直观的展示trunk解码器和结构解码器对各自特征的提纯和整合效果,我们还在图3中分别展示了两个解码器输出的特征。图3(b)展示了目标的主干区域被trunk解码器识别,而结构解码器更侧重于检测具体的目标结构。
We conduct an ablation study to assess the effectiveness of the proposed trunk and structure decoders of our~\ourmodel~in \tabref{tab:decoder}. 
It should be noted that, in this experiment, we directly upsample and sum all multiscale features instead of the corresponding decoder that is removed. 
As demonstrated in \tabref{tab:decoder}, both the trunk and the structure decoders have a remarkable contribution to the segmentation results. When embedding two decoders simultaneously, our model achieves superior performance. 
To further intuitively illustrate the capability of trunk and structure decoders for refining and integrating features, we also provide the last layer of features from the corresponding decoders in \figref{fig:cam}. 
As displayed in \figref{fig:cam}(b) and \figref{fig:cam}(c), the trunk areas are identified by the trunk decoder, while the structure decoder concentrates on detecting detailed object structures.

\begin{table}
\footnotesize
  \renewcommand{\arraystretch}{1}
  \renewcommand{\tabcolsep}{1.9mm}
  \caption{Ablations for trunk and structure decoders.}
  \label{tab:decoder}
  \begin{tabular}{c|c|cccccc}
    \toprule
    Structure & Trunk & $F_\beta^\text{max} \uparrow$ &  $F_\beta^\omega \uparrow$ & $M \downarrow$ &
    $S_\alpha \uparrow$ & $E_{\phi}^\text{m} \uparrow$ & $HCE_{\gamma} \downarrow$ \\
    \midrule
     \checkmark  &  & 0.807 & 0.743 & 0.064 & 0.826 & 0.880  & 1110 \\
       & \checkmark & 0.812 & 0.752 & 0.062 & 0.833 & 0.886 & 1136 \\
    \rowcolor[RGB]{235,235,235}
    \checkmark  &  \checkmark & \textbf{0.823} & \textbf{0.763} & \textbf{0.059} & \textbf{0.838} & \textbf{0.892} & \textbf{1097} \\
  \bottomrule
\end{tabular}
\end{table}


\subsubsection{Effectiveness of TSA}
% The Trunk-Structure Aggregation module (TSA) is designed to fuse the two complementary semantic features captured by the trunk and structure decoders. We compare three different fusion methods, including addition, multiplication, and concatenation, including TSA, as shown in Table 2. It is worth noting that directly adding the two semantically complementary features did not yield satisfactory results, which may be attributed to the uncertainty in feature extraction by neural networks. On the other hand, concatenation of features achieved better results in terms of the HCE metric, indicating that this method may focus more on the structure of the target. However, it also weakens some trunk features, possibly due to the equal treatment of features in concatenation, which may not be suitable for fusing two different semantics, resulting in a bias towards one of the semantics.On the contrary, our TSA is capable of simultaneously improving both trunk-related and structure-related metrics such as HCE. This is achieved by guiding the structure decoder with trunk features, which strengthens the internal structure while weakening the background noise in the structure features. The residual from this process is then supplemented into the structure features, followed by the addition of trunk features in a step-wise manner to fuse complementary semantic features and obtain a complete and integrated feature representation.
%TSA被设计用于整合来自Trunk和Structure解码器的多尺度跨结构的特征。我们将TSA模块与直接相加和级联的融合方式在表5中进行了对比。我们可以看到,直接相加两种语义互补的特征的效果并不好,而采用级联的方式会好于相加操作。相比于相加和级联,提出的TSA模块可以更充分的融合trunk和structure信息以获得更好的unified特征。这得益于我们在TSA中我们通过Trunk特征对structure的背景噪声进行了抑制,并加强了主体和细节信息。$图6(d)$ 也展示了从union decoder输出的融合后的mask feature。
The proposed TSA module is intended to combine multiscale cross-structure features from the trunk and structure decoders. 
In \tabref{tab:TSA}, we compare our TSA with the summation and concatenation operations.
The results show that it is not desirable to directly add two spatially complementary features, and adopting concatenation performs better than the summation operation.
Compared to summation and concatenation, TSA adequately integrates the trunk and structure information to generate a unified mask feature. 
This suggests that TSA suppresses the background noise of structure features via trunk features and enhances the the representation of body and detail information. 
\figref{fig:cam}(d) also visualizes the fused mask feature generated by  the union decoder.

% Figure environment removed

\begin{table}
\footnotesize
  \renewcommand{\arraystretch}{1}
  \renewcommand{\tabcolsep}{2.6mm}
  \caption{Effectiveness of TSA in union decoder.}
  \label{tab:TSA}
  \begin{tabular}{c|cccccc}
    \toprule
    Modules & $F_\beta^\text{max} \uparrow$ &  $F_\beta^\omega \uparrow$ & $M \downarrow$ &
    $S_\alpha \uparrow$ & $E_{\phi}^\text{m} \uparrow$ & $HCE_{\gamma} \downarrow$ \\
    \midrule
    Add  &0.802 & 0.742 & 0.066 & 0.827 & 0.880 & 1109      \\
    Concat  &0.813 & 0.751 & 0.063 & 0.832 & 0.886 & 1104\\
    \rowcolor[RGB]{235,235,235}
    TSA  & \textbf{0.823} & \textbf{0.763} & \textbf{0.059} & \textbf{0.838} & \textbf{0.892} & \textbf{1097} \\
  \bottomrule
\end{tabular}
\end{table}

\subsubsection{Dual- \emph{vs.} Single-size Input}
% As shown in the previous work \cite{qin2022highly}, using the dual-size input is essential for handling high-resolution images. In our analysis, we compare the performance of our model using single-scale inputs, including low-resolution (size: 256) and high-resolution (size: 1024) inputs, with our model using the dual-size input, as presented in Table 3.
% Indeed, it is intuitive that the HCE metric shows a significant improvement with high-resolution inputs as they retain more structural details. However, the maximal F-measure ($F_\beta^\text{max}\uparrow$) \cite{achanta2009frequency} and Mean Absolute Error (MAE, M \downarrow$) \cite{perazzi2012saliency} metrics do not show significant improvements, likely due to the limited receptive field of the model.
% By adopting the approach of using the dual-size input and sharing the backbone, we can retain the advantages of both scales while achieving comprehensive improvements in all metrics, with only a small increase in computational cost while still maintaining real-time inference speed.
%使用双尺度输入对于处理高分辨率图像是非常有必要的,因为它可以提供跨越更大范围的多尺度特征。我们在表3中比较了在UDUN中使用单尺度输入和双尺度输入时模型的性能。对比前两行我们发现,在高分辨率输入的情况下HCE得分有明显的改善,这归功于大分辨率包含更多的细节信息。通过双尺度输入,模型的表现在所有性能指标上都有明显的提升。尽管牺牲了一些推理速度,但是在共享主干的设计下使得模型只增加了少量的GFLOPs同时仍然保持实时的推理速度。
The dual-size input is essential for processing high-resolution segmentation because it enables the encoder to extract multiscale features over a wider range.
\tabref{tab:input} compares the performance of~\ourmodel~when using the single-size input and dual-size input. The results in the first two rows illustrate a significant improvement in HCE$_{\gamma}$ when using higher-resolution input, which is attributed to the greater amount of detailed information contained in the higher-resolution features. 
By utilizing the dual-size input, the performance of~\ourmodel~is further improved in terms of $F_\beta^\text{max}$, $M$, and HCE$_{\gamma}$. 
Despite sacrificing some inference speed, the shared backbone design enables our model to increase only a few GFLOPs while still keeping real-time inference.

% Figure environment removed

\begin{table}
\footnotesize
  \renewcommand{\arraystretch}{1}
  \renewcommand{\tabcolsep}{3.2mm}
  \caption{Comparisons of single- and dual-size inputs.}
  \label{tab:input}
  \begin{tabular}{c|cccccc}
    \toprule
    Input Sizes & $F_\beta^\text{max} \uparrow$ &  $M \downarrow$ & $HCE_{\gamma} \downarrow$ &
    GFLOPs & FPS \\
    \midrule
    256   &0.758 & 0.083 & 1486 & \textbf{16.4} & \textbf{80.2}     \\
    1024  &0.786 & 0.072 & 1138 & 131.9 &70.5    \\
    \rowcolor[RGB]{235,235,235}
    256\&1024  &\textbf{0.823} & \textbf{0.059} & \textbf{1097} & 142.3 &45.5  \\
  \bottomrule
\end{tabular}
\end{table}

%\noindent\textbf{Effectiveness of Decoupled Interior \$ Exterior label Supervision.}
%考虑放补充材料


\subsubsection{Large-scale Shallow Features}
%如何引出这个动机?IS-Net并没有1024,而且去掉1024HCE比不过IS-Net
% 
% 对于DIS任务而言,一些backbone 如resnet50对捕获很精细结构是不足的 due to continuous downsampling ,为了弥补这个缺陷
% 为了保留大尺度的结构完整性,所以我们设计了一个独立的large-scale(size:1024)路径,不需要经过heavy backbone仅仅几个卷积,可以以低代价获取高精度的细节。性能表现如表4所示,由于可以承载了更多结构信息,各个指标都显著提升。
% Based on our observations, we have found that large-scale features are essential for achieving ultra-high accuracy in Dichotomous Image Segmentation (DIS) tasks, as they enable retention of fine structures. Conversely, low-scale features are unable to effectively capture high-precision details. 
% To preserve the integrity of large-scale structures, we have designed a dedicated large-scale pathway with a size of 1024. This pathway does not require a heavy backbone and can achieve high-precision details through only a few convolutions, at a low computational cost. The performance results, as shown in Table \ref{tab: Large-scale Features}, demonstrate significant improvements across various metrics, due to the increased capacity to capture more structural information.
% 对于High-accuracy DIS任务而言,尽可能的保留大尺度的特征来捕捉低级的空间和纹理线索是非常有必要的。为了保留最大分辨率的特征并维持模型的低成本,我们嵌入了一个独立通道来直接从1024尺度的输入图像提取原尺度的浅层特征以输入到Structure解码器中而不经过backbone。如表6所示,该设计显著的提升了模型在所有指标上的表现,特征是在HCE指标上。这证明了大尺度低级别的特征包含的丰富的空间和纹理信息其可以帮助识别高分辨率目标中的精细结构。
For achieving highly accuracy masks, it is crucial to extremely preserve large-scale features that contain a wealth of low-level spatial and texture information. 
To ensure that~\ourmodel~retains the highest resolution features for feeding into the structure decoder while maintaining the lower computational cost, we embedded a stand-alone path that extracts the shallow features directly from the input image with a scale of 1024 without passing through the backbone (see \figref{fig:model}).
As shown in \tabref{tab:shallow_feat}, this design significantly boosts the performance of our model on all metrics, especially on the HCE$_{\gamma}$. 
It demonstrates the rich spatial and textural information available from large-scale low-level features that can facilitate the discrimination of fine-grained structures in high-resolution objects.

% 在structure decoder中,我们使用了purify operation 利用高级的trunk特征,它往往更聚焦前景的内部区域,除低级特征中的内部噪声和背景噪声,让网络更加关注前景结构信息.table6中展示了purify operation的性能,维持HCE指标的同时,增强了trunk的结构完整度从前5个指标可以显著看出。
% 同时,我们可视化了purify operation前后的特征,见图5,从图中可知,purify operation 之前网络只是关注整体轮廓,关注点在一些梯度大的边缘区域并夹杂者背景噪声,而使用了purify operation 之后,网络更加关注前景目标的structure包括前景的内部结构。
\subsubsection{Effectiveness of Filtering Operation}
% In the structure decoder, we have employed a filtering operation that utilizes high-level trunk features, which tend to focus more on the internal regions of the foreground. This operation removes internal noise and background noise from the low-level features, allowing the network to better focus on foreground structural information. The performance of the filtering operation is shown in \tabref{tab:purification}, where it is evident that while maintaining the HCE metric, the integrity of the trunk structure is significantly enhanced, as evident from the first 5 metrics.
% We also visualized the features before and after applying the filtering operation, as shown in  \figref{fig:purify}. From the figure, it can be observed that before the filtering operation, the network only focuses on the overall contours, with attention to areas with high gradients and background noise. However, after applying the filtering operation, the network pays more attention to the structure of the foreground targets, including the internal structure of the foreground.
The structure decoder embraces a simple feature filtering operation that leverages the intermediate trunk features in the trunk decoder to suppress background noise in lower-level features and focus on dominant information (refer to \eqnref{equ:filter}). 
\tabref{tab:filter} demonstrates the benefit of this operation.
It significantly enhances the holistic perception effect on the foreground area while keeping a superior HCE score. 
We also compare the visual features before and after adding the filtering operation. As shown in \figref{fig:filter}, after embedding the filtering operation, the feature maps are focused on the internal structure of the targets and effectively suppress substantial undesired background noise.

\begin{table}
\footnotesize
  \renewcommand{\arraystretch}{1}
  \renewcommand{\tabcolsep}{1.8mm}
  \caption{Effectiveness of large-scale shallow features for the structure decoder.}
  \label{tab:shallow_feat}
  \begin{tabular}{c|cccccc}
    \toprule
    Large-scale Features & $F_\beta^\text{max} \uparrow$ &  $F_\beta^\omega \uparrow$ & $M \downarrow$ &
    $S_\alpha \uparrow$ & $E_{\phi}^\text{m} \uparrow$ & $HCE_{\gamma} \downarrow$ \\
    \midrule
       &0.802 & 0.743 & 0.066 & 0.828 & 0.878 & 1157 \\
    \rowcolor[RGB]{235,235,235}
     \checkmark  &\textbf{0.823} & \textbf{0.763} & \textbf{0.059} & \textbf{0.838} & \textbf{0.892} & \textbf{1097} \\
  \bottomrule
\end{tabular}
\end{table}

\begin{table}
\footnotesize
  \renewcommand{\arraystretch}{1}
  \renewcommand{\tabcolsep}{1.9mm}
  \caption{Effectiveness of the filtering operation for the input features of the structure decoder.}
  \label{tab:filter}
  \begin{tabular}{c|cccccc}
    \toprule
    Filtering Operation & $F_\beta^\text{max} \uparrow$ &  $F_\beta^\omega \uparrow$ & $M \downarrow$ &
    $S_\alpha \uparrow$ & $E_{\phi}^\text{m} \uparrow$ & $HCE_{\gamma} \downarrow$ \\
    \midrule
       &0.813 & 0.752 & 0.062 & 0.835 & 0.887 & 1108\\
    \rowcolor[RGB]{235,235,235}
    \checkmark   &\textbf{0.823} & \textbf{0.763} & \textbf{0.059} & \textbf{0.838} & \textbf{0.892} & \textbf{1097} \\
  \bottomrule
\end{tabular}
\end{table}


\subsubsection{Performance of Different Backbones}
% 为了探究Shared and  Non-shared Backbone对性能的影响,我们将双尺度书入分别输入到两个同样的R50中得到两组分辨率的特征,如(Fig. 1(c)),性能展示在table7中,使用非共享方式达到了一个比较好的HCE指标,但是Fmax M指标却有所下降。这可能是因为输入高分辨率的backbone占据网络优化的主导,但是其感受野不足而导致目标定位不精确。采用共享的方式,可以使得一组backbone参数同时接受两种尺度的输入like zoom in and out 在定位目标trunk的同时不丢失精细的结构,同时减少了近乎一半的参数量使得网络更轻量。
% Shared and Non-shared Backbone are two different ways to use two scales of features in the network. In the Non-shared Backbone approach, each scale of feature is input into a separate ResNet50 backbone, and in the Shared Backbone approach, both scales of features are input into the same ResNet50 backbone.
% According to the results shown in Table 8, using the Non-shared Backbone approach achieved a good HCE score, but the Fmax M score decreased. This may be because the high-resolution backbone dominates the network optimization, but its receptive field is insufficient, leading to imprecise target localization. In contrast, using the Shared Backbone approach allows one set of backbone parameters to simultaneously process both scales of features, similar to zooming in and out, thus preserving fine-grained structures while localizing the trunk. Moreover, using the Shared Backbone approach reduces the number of parameters by almost half, making the network more lightweight.
\ourmodel~adopts a shared backbone to minimize the impact of model parameters from the dual-size input. 
To investigate the effect of the performance with shared and non-shared backbones, we compare the segmentation results on overall DIS-TE in the third and fourth rows of \tabref{tab:backbone}. 
Although the non-shared backbone style achieves a better HCE score, it has a decreasing trend in terms of $F_\beta^\text{max}$ and $M$ performance. 
In contrast, the shared backbone reduces the number of parameters by almost 50\%, resulting in a lightweight framework.
Moreover, we observe that the number of backbone parameters accounts for a large proportion of our model. 
As a result, we evaluate the results of our model with lighter backbones, including ResNet-18~\cite{he2016deep} and ResNet-34~\cite{he2016deep}. 
\tabref{tab:backbone} illustrates that when using ResNet-18, UDUN reaches a real-time inference speed of 65.3 \emph{fps} with only 12.33M parameters, while still outperforming IS-Net~\cite{qin2022highly} in all metrics on overall DIS-TE. 
The experimental results highlight the potential of our method for real-world applications.



% 消融实验是在验证集
%要不要展示R18 R34 R59在overall TE1-4六个指标都超过IS-net?  放补充材料?

\begin{table}
\footnotesize
  \renewcommand{\arraystretch}{1}
  \renewcommand{\tabcolsep}{2.05mm}
  \caption{Performance of~\ourmodel\ with diverse backbones on the overall DIS-TE test set.}
  \label{tab:backbone}
  \begin{tabular}{c|c|ccccc}
    \toprule
    Backbones &  Shared  & $F_\beta^\text{max} \uparrow$ &  $M \downarrow$ &
    $HCE_{\gamma} \downarrow$ & Params & FPS \\
    \midrule
          ResNet-18~\cite{he2016deep}  & \checkmark   &0.807  & 0.065 & 1009 & \textbf{12.33}M  &\textbf{65.3}  \\
          ResNet-34~\cite{he2016deep} & \checkmark   &0.816  & 0.061 & 995 & 22.45M  &55.0  \\
      
         % Res-50~\cite{he2016deep} (Non-shared) &0.810  & 0.063 & \textbf{1072} & 48.65  &42.3  \\
         ResNet-50~\cite{he2016deep} &    &0.826  &0.058  & \textbf{964} & 48.65M  &42.3  \\
         
    \rowcolor[RGB]{235,235,235}
    ResNet-50~\cite{he2016deep} &  \checkmark   &\textbf{0.831} & \textbf{0.057} & 977 & 25.05M &  45.5    \\
  \bottomrule
\end{tabular}
\end{table}


% \begin{table*}
%   \caption{Some Typical Commands}
%   \label{tab:commands}
%   \begin{tabular}{ccl}
%     \toprule
%     Command &A Number & Comments\\
%     \midrule
%     \texttt{{\char'134}author} & 100& Author \\
%     \texttt{{\char'134}table}& 300 & For tables\\
%     \texttt{{\char'134}table*}& 400& For wider tables\\
%     \bottomrule
%   \end{tabular}
% \end{table*}

% \begin{math}
%   \lim_{n\rightarrow \infty}x=0
% \end{math},

% \begin{equation}
%   \lim_{n\rightarrow \infty}x=0
% \end{equation}

% % Figure environment removed


% \section{Acknowledgments}

% Identification of funding sources and other support, and thanks to individuals and groups that assisted in the research and the
% preparation of the work should be included in an acknowledgment
% section, which is placed just before the reference section in your document.

\section{Conclusion}

The main purpose of this paper is to explore how to segment highly accurate objects with intricate outlines and cumbersome structures in the DIS task. 
Unlike high-resolution task-specific binary image segmentation, the difficulty for DIS lies in simultaneously segmenting dominant regions and detailed structures with high accuracy. 
To address this challenge, we contribute a one-stage unite-divide-unite network (\ourmodel) that adopts a unite encoder to extract two sets of features at different scales from a shared backbone, then a divide-and-conquer strategy to refine and fuse the trunk and structure information in our trunk and structure decoders, respectively. 
Finally, a union decoder is introduced to integrate these two features for efficient segmentation of high-accuracy objects. 
Qualitative and quantitative experimental results demonstrate the effectiveness
and robustness of the proposed~\ourmodel.
We hope our model can be widely applied to various real-life scenarios, such as medical image analysis, image matting, and the art field. 

%% The next two lines define the bibliography style to be used, and
%% the bibliography file.

% 加了一个blance  参考文献排版就对齐了,但是警告还是在
\balance


\bibliographystyle{ACM-Reference-Format}
\bibliography{reference}


%%
%% If your work has an appendix, this is the place to put it.
% \appendix

% \section{Research Methods}



\end{document}
\endinput
%%
%% End of file `sample-sigconf.tex'.


