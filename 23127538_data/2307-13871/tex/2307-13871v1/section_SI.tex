\clearpage
\pagebreak
\widetext
\begin{center}
\textbf{\large Supplemental Material: The Scientific Value Agent}
\end{center}

\setcounter{equation}{0}
\setcounter{section}{0}
\setcounter{figure}{0}
\setcounter{table}{0}
\setcounter{page}{1}
\makeatletter
\renewcommand{\thesection}{S\arabic{section}}
\renewcommand{\thesubsection}{S\arabic{section}.\arabic{subsection}}
\renewcommand{\theequation}{S\arabic{equation}}
\renewcommand{\thefigure}{S\arabic{figure}}
% \renewcommand{\bibnumfmt}[1]{[S#1]}
% \renewcommand{\citenumfont}[1]{S#1}


% Figure environment removed

% Figure environment removed

% Figure environment removed


% Figure environment removed



% Figure environment removed

\section{Calculation of crystallographic residual}

Normalized weighted residue ($\text{norm.\,} R_{wp}$) were calculated for each fit of a refined phase by extracting the calculated $R_{wp}$ from the Rietveld refinements using the TOPAS software \cite{coelho2018topas}, and then normalizing by the average $R_{wp}$ for each refined phase, and such that the result spans 0 to 1 for visual clarity.

\begin{equation}
    \text{norm.\,} R_{wp} = \frac{(R_{wp} - <R_{wp}>) - \min(R_{wp} - <R_{wp}>)}{\max(R_{wp} - <R_{wp}> - \min(R_{wp}-<R_{wp}>))}
\end{equation}

In this description, a normalized  $R_{wp}$  of 0 would correspond to a phase existence. Subsequent re-normalization was accomplished by applying Gibb's phase rule, and only allowing for 2 phases with non-zero composition, $1-R_{wp}$, then again normalized on $[0, 1]$ for output compositions. 