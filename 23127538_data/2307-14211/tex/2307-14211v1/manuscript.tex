%This file serves as a draft for the publication, the materials selected and organized should show clearly the goal of our paper and why it's interesting. A logic order for the presentation is essential to achieve these.

\documentclass[preprint,a4paper,nofootinbib,showkeys,noindent,superscriptaddress]{revtex4-1}%a4paper is used for the geometry package, nofootinbib is used to avoid footnotes appeared in the references, as it would be by default, showkeys is used to display keywords

\usepackage{amsmath,amssymb}
\usepackage{enumerate}
\usepackage{amsfonts}

\usepackage{subfigure}
\usepackage[hyperindex,colorlinks]{hyperref}

\usepackage{allpurpose}

\usepackage{tikz}
\usepackage{multirow}

\usepackage{array}
\newcolumntype{P}[1]{>{\centering\arraybackslash}p{#1}} %to get a centering text in a cell with a given width, see https://tex.stackexchange.com/questions/157389/how-to-center-column-values-in-a-table

% to provide the command \slashed for Dirac slash notation
\usepackage{slashed} 


%to use symbols instead of numbers to enumerate footnotes 
% \usepackage[marginal,perpage,symbol]{footmisc}
% \DefineFNsymbols*{dagfirst}{\dag\ddag*{**}{\dag\dag}{\ddag\ddag}}
% \setfnsymbol{dagfirst}

\usepackage{color}
\hypersetup{hidelinks,
%            colorlinks=true,
%           linkcolor=black,       
%            anchorcolor=black  
%            citecolor=black, 
%            urlcolor=black
        }


%to define several colors to be used in the manuscript
%\newcommand{\red}{\color{red}}
%\newcommand{\blue}{\color{blue}}
%\newcommand{\purple}{\color{purple}}
%\newcommand{\orange}{\color{orange}}


\begin{document}

%\title{AR-like matter from Perturbations on the Boulatov model}
\title{Generalised Amit-Roginsky model from perturbations of 3d quantum gravity}
\date{\today}


\author{Victor Nador}
\email[Email: ]{victor.nador@u-bordeaux.fr}
\affiliation{LaBRI, Univ. Bordeaux, 351 cours de la Lib\'eration, 33405, Talence, France}

\author{Daniele Oriti}
\email[Email: ]{daniele.oriti@physik.lmu.de}
\affiliation{Arnold Sommerfeld Center for Theoretical Physics, Ludwig-Maximilians-Universit\"at, Theresienstra\ss e 37, 80333, M\"uchen, Germany}

\author{Xiankai Pang}
\email[Email: ]{Xiankai.Pang@physik.uni-muenchen.de}
\affiliation{Arnold Sommerfeld Center for Theoretical Physics, Ludwig-Maximilians-Universit\"at, Theresienstra\ss e 37, 80333, M\"uchen, Germany}

\author{Adrian Tanasa}
\email[Email: ]{ntanasa@u-bordeaux.fr}
\affiliation{LaBRI, Univ. Bordeaux, 351 cours de la Lib\'eration, 33405, Talence, France}
\affiliation{LIPN, Univ. Sorbonne Paris Nord, Villetaneuse, France}
\affiliation{H. Hulubei Nat. Inst. Phys. Nucl. Engineering, P.O.Box MG-6, 077125, Magurele, Romania}

\author{Yi-Li Wang}
\email[Email: ]{Wang.Yili@physik.uni-muenchen.de}
\affiliation{Arnold Sommerfeld Center for Theoretical Physics, Ludwig-Maximilians-Universit\"at, Theresienstra\ss e 37, 80333, M\"uchen, Germany}


\begin{abstract}
  %A new model is constructed in this article, where an Amit-Ronginsky-like model is obtained as fluctuations around a classical solution to the Boulatov model. The melonic dominance in Amit-Ronginsky model at large $N$ limit is unclear in the new model, but can be retained with further constraints on the classical solutions of the Boulatov model. Both theories share common features, namely a large $N$ expansion yielding a melonic dominance in that limit. To our knowledge, this is the first result connecting vector and tensor melonic theories.
  A generalised Amit-Roginsky vector model in flat space is obtained as the effective dynamics of pertubations around a classical solution of the Boulatov group field theory for 3d euclidean quantum gravity, extended to include additional matter degrees of freedom. By further restricting the type of perturbations, the original Amit-Roginsky model can be obtained. This result suggests a general link (and possibly a unified framework) between two types of tensorial quantum field theories: quantum geometric group field theories and tensorial models for random geometry, on one hand, and melonic-dominated vector and tensorial models in flat space, such as the Amit-Roginsky model (and the SYK model), on the other hand.
\end{abstract}

%\keywords{Amit-Roginsky model; Boulatov model; Matter and Quantum gravity}  

\maketitle
\tableofcontents

\section{Introduction}

Random matrix models ~\cite{David:1985nj,Ambjorn:1985az,DiFrancesco:2004qj, DiFrancesco:1993cyw} are, in their simplest formulation, 
$0$-dimensional 
field theories of an $N\times N$ (Hermitian) matrix $M_{ij}$ successfully employed to define $2$-dimensional euclidean quantum gravity, based on the fact that their perturbative expansion generates a sum over random surfaces weighted by purely combinatorial amplitudes corresponding to a simplicial gravity path integral on the triangulation dual to each matrix Feynman diagram. They have been generalized to matrix field theories in flat space, by the addition of suitable flat space coordinates, and used to describe, for example, large-N regimes of non-abelian gauge theories. Both finite matrix models and matrix field theories have found innumerable applications in mathematical and theoretical physics. 

A different kind of  %straightforward 
generalization is to define tensorial models producing, in their perturbative expansion, a sum over higher-dimensional lattices. Tensorial models in $d$ dimensions are obtained by replacing the matrix field $M$ by a tensor field with $d$ indices $M_{ij}\to T_{{i_1}..{i_d}}$ 

Such tensorial generalization was proposed already 30 years ago in a random geometry context ~\cite{Ambjorn:1990ge,Sasakura:1990fs,Godfrey:1990dt}, and soon adapted to the quantum geometric one for the description of topological quantum field theories \cite{Boulatov:1992vp,Ooguri:1992eb}, with 3d quantum gravity being a special case. The same quantum geometric models, under the label of group field theories, became central to formulate 4d quantum gravity in the context of spin foam models and canonical loop quantum gravity \cite{DP-F-K-R,P-R, P-R2}.
In this quantum gravity context, both as purely combinatorial random geometric models, and as richer quantum geometric ones, they represent nowadays a very promising and quickly developing area of research \cite{Freidel:06, Oriti:2011jm, Krajewski:2011zzu, Carrozza:2016vsq, Oriti:2016acw, Gielen:2016dss}.
The simplest example of such quantum geometric tensorial field theories is the so-called Boulatov model ~\cite{Boulatov:1992vp}, where
%Boulatov generalised the matrix model in another way where 
the r\^ole of the matrix indices is played here by group elements $g_1,g_2,g_3 \in SU(2)$. In these theories, the tensors $T_{i_1...i_d}$ of simple tensor models are replaced by fields $T(g_1,...,g_d)$ on a Lie group manifold $G^d$, having the local symmetries of gravitational theories in mind. 

More recently, tensorial field theories have proven to define very rich and interesting quantum field theories in flat space, again via the addition of suitable embedding coordinates; in particular, they define new conformal field theories, with many potential applications, e.g. to the AdS/CFT context \cite{Benedetti:2020yvb, Benedetti:2017fmp}.

The key mathematical fact that spurred much development in these models was the availability of analytic tools that allowed control over their perturbative expansion, despite the combinatorial intricacies.
Tensorial models, just like matrix models, admit a large $N$ expansion \cite{Gurau:2010ba, Gurau:2011xq, Dartois:2013he, Carrozza:2015adg, Tanasa:2015uhr, Tanasa:2012pm, Gurau:2019qag}.
%, however it is no longer topological. 
The leading order in the tensor large $N$ expansion is given by a particular family of Feynman graphs called the \emph{melonic graphs}, which correspond to (special triangulations of) spherical topology. 
This analytic control has made possible the wealth of results on the renormalization group flow, both perturbative and non-perturbative, of tensorial field theories and group field theories \cite{Carrozza:2016vsq, Carrozza2017}, as well as the statistical analysis of critical behaviour \cite{Bonzom:2011, Baratin:2013rja}.
%These graphs are discretizations of the $d$-sphere and are in bijection with branched polymers~\cite{Gurau:2013cbh}. 

It is worth emphasizing here that the %celebrated holographic 
Sachdev-Ye-Kitaev (SYK) model \cite{Maldacena:2016hyu,Rosenhaus:2018dtp}
also enjoys the same melonic dominance in the large $N$ limit \cite{Bonzom:2018jfo}, with $N$ the number of fermionic fields of the SYK model.
%(see \cite{Bonzom:2018jfo} for an explicit proof of the melonic dominance property of the SYK model).

%However it is difficult to relate the models in~\cite{Ambjorn:1990ge,Sasakura:1990fs,Godfrey:1990dt} to simplicial gravity, so 
%The Boulatov model has a similar interpretation in terms of simplicial $3$-manifold. 
%with the addition of additional dependency on group elements $g_1,g_2,g_3\in SU(2)$. 
%The Feynman graphs of Boulatov-Ooguri model are assigned with a BF-weight~\cite{Dijkgraaf:1989pz}, which is an equivalent theory of $3$D general relativity (GR) in the classical limit.


The Amit-Roginsky (AR) model~\cite{Amit:1979ev} (see also \cite{Benedetti:2020iku}) describes a vector field theory 
%on $3$-dimensional spacetime 
whose coupling constant is proportional to an $SU(2)$ $3j$-symbol. 
%Its action reads
%\begin{align}
%  S_{AR}[\phi]=&\int\dd^dx\left\{\frac{1}{2}\sum_m(-1)^{j-m}\left[(\nabla \phi^j_m)(\nabla \phi^j_{-m})+\mu\phi^j_m\phi^j_{-m}\right]\right. \nonumber \\
 % &+\left.\sum_{m_1,m_2,m_3}\frac{\lambda}{3!}\sqrt{2j+1}\mat{ccc}{j&j&j\\m_1&m_2&m_3}\phi^j_{-m_1}\phi^j_{-m_2}\phi^j_{-m_3}\right\}
%\end{align}
This model also has a large $N$ expansion and one can prove that it exhibits a melonic limit, just like tensor models, where $N=2j+1$ is the dimension of the irreducible vector representation, and can thus be understood as a special (and particularly simple) element of tensorial vector field theories. 

%These common features and t
%where the matter fields enter as a quantum fluctuation around classical solutions of a Boulatov model. 


Together, tensorial models of random and quantum geometry, and tensorial field theories in flat space, can be seen as part of a broader framework of {\it tensorial group field theories} (TGFTs), sharing key mathematical features and techniques, while remaining flexible enough to allow for a large variety of possible physical applications. However, the two classes of models have remained quite separate, so far.
The present work establishes the first explicit link between them.

A crucial ingredient will be, from the quantum geometric side of the story, the addition of matter degrees of freedom to the quantum geometric ones, also inspired by recent work on the extraction of a relational cosmological dynamics from group field theory \cite{Oriti:2016qtz, Marchetti:2020qsq, Marchetti:2021gcv}.
As a candidate of quantum gravity, the inclusion of matter is of course crucial for TGFTs. Work in this direction has followed two main routes, rather disconnected. On the one hand, non-commutative scalar field theories have been extracted, by interpreting the Lie group domain of quantum geometric models as a curved momentum space, as perturbations over classical solutions of group field theories \cite{Fairbairn:2007sv, Girelli:2009yz}, producing an \lq emergent matter\rq description from the same quantum degrees of freedom having a pre-geometric interpretation. On the other hand, matter degrees of freedom (or, maybe more properly, \lq pre-matter\rq degrees of freedom) have been added to the quantum geometric ones, so to produce a lattice path integrals for the coupling of gravity and matter at the level of the Feynman amplitudes of GFT models \cite{Li:2017uao, Oriti:2006jk, Fairbairn:2006dn}. These additional degrees of freedom are also instrumental for the definition of relational observables with a local spacetime interpretation, in group field theory cosmology, as mentioned. 
Both strategies turn out to be relevant for linking the quantum geometric Boulatov model to the Amit-Roginsky model, in this work, with the latter arising as the effective dynamics of quantum geometric perturbations, but with the additional degrees of freedom interpret as matter frames in the quantum geometric setting playing the role of flat space coordinates in the resulting Amit-Rogisnky model.

%The group dependency appearing in both Boulatov and AR model suggests to look for a common framework unifying these two types of models. A natural idea for such mechanism comes from effective field theory of additional scalar degrees of freedom coupled to the Boulatov model~\cite{Freidel_06,Freidel_2006,Fairbairn:2007sv}.


\medskip

In this paper, the classical solutions of the equation of motion of the Boulatov model regularised via a heat-kernel approach are investigated. We exhibit an explicit solution of these equations with the $3j$ symbol of $SU(2)$ and study $2$-dimensional perturbations around this solution. We then give explicit conditions on these perturbations to give rise to an AR-like effective action - with additional summation over spin indices with respect to the original AR model. This shows that the AR model can be seen as a 
perturbation around classical solutions of the Boulatov model, thus giving the anticipated explicit link between two types of tensorial models. 
%exhibiting a large $N$ melonic limit.


The paper is organised as follows. A brief review of the Boulatov model is given in section~\ref{sec:bolatovreview}. In the following section we recall the definition of the AR model; then, we study the condition on the perturbations of classical solutions of the equations of motion of the Boulatov model necessary to recover an AR-like action as an effective action, which is then explicitly derived. Section~\ref{melonicdominance} discusses the existence of a melonic dominance for our effective action. While it is unsettled whether melonic dominance is preserved in the most general setting, we exhibit additional conditions that ensure this property. Finally, we offer some conclusions and perspectives.


\section{Boulatov model} \label{sec:bolatovreview}
\subsection{A short review on the Boulatov GFT model}

Quantum geometric TGFTs, or GFTs, \cite{Freidel:2005qe,Oriti:2006se,Krajewski:2011zzu,Oriti:2011jm,Oriti:2013aqa} are field theories whose dynamical field depends on $n$ points $g_i$ of a Lie group $G$. 
The group elements $g_i$ can be interpreted as discrete parallel transports of a gravitational connection, {\it i.e.} of a $G$-vector bundle. 
The Boulatov model~\cite{Boulatov:1992vp} is a $3$D GFT model with field $T(g_1,g_2,g_3) :G^{3}\to \mathbb{C}$, where $G=SU(2)$. 
The field is invariant under 
%global right diagonal action of $SU(2)$
\begin{equation}
    T(g_1h,g_2h,g_3h)=T(g_1,g_2,g_3)\hspace{10pt}\forall h\in SU(2). 
    \label{right_inv}
\end{equation}
and satisfies the reality condition \cite{Fairbairn:2007sv}
\begin{equation}
    T(g_1,g_2,g_3)=\bar{T}(g_3,g_2,g_1). \label{reality}
\end{equation}
%Note that i
The original Boulatov model~\cite{Boulatov:1992vp} further requires cyclic symmetry in the group elements $g_i$. But this property plays no role in this paper, so we do not discuss it further.

The action of the Boulatov model is non-local and it writes~\cite{Boulatov:1992vp}
\begin{align}
          S[T]&=\frac{\mu^2}{2}\int \dd g_1 \dd g_2 \dd g_3T(g_1,g_2,g_3)\bar{T}(g_1,g_2,g_3) \nonumber \\
      &-\frac{\lambda}{4!}\int \prod_{i=1}^6 \dd g_i T(g_1,g_2,g_3) T(g_3,g_5,g_4)T(g_4,g_2,g_6)T(g_6,g_5,g_1), \label{eq:actionTgroup}
\end{align}
where $\mu$ is the \lq mass\rq of the field (simply the coupling of the quadratic non-derivative term) and $\lambda$ is the coupling constant of the quartic interaction. The connection to simplicial geometries is elucidated by a suitable graphical interpretation of the elements in the action.
The field $T(g_1,g_2,g_3)$ represents a triangle, with three group elements associated with its three edges, and the interaction contains four triangles glued along shared edges (thus sharing the same group element) forming a tetrahedron, which is the building block of a $3D$ simplicial lattice, likes those generated as dual to the Feynman diagrams of the model in its perturbative expansion.

The equation of motion of the field $T(g_1,g_2,g_3)$ reads
\begin{equation}
      \mu^2T(g_3,g_2,g_1)=\frac{\lambda}{3!}\int\dd g_4\dd g_5\dd g_6T(g_3,g_5,g_4)T(g_4,g_2,g_6)T(g_6,g_5,g_1). \label{eq:boulatoveqTgroup}
\end{equation}
This provides a description of the GFT model in a \emph{group representation}. By generalised Fourier transforms, GFTs can also be written in terms of a \emph{spin representation}.

As a function of $SU(2)^{\otimes3}$, the field $T$ can be expanded in terms of Wigner matrices $D^{j_i}_{m_in_i}(g_i)$ via the Peter-Weyl theorem \cite{Makinen:2019rou,Martin-Dussaud:2019ypf}. Considering the invariance~\eqref{right_inv}, this decomposition takes the form 
\begin{equation}
      T(g_1,g_2,g_3)=\sum_{\{j,m,n\}}T^{m_1m_2m_3}_{j_1j_2j_3}\prod_{i=1}^{3}\sqrt{2j_i+1}D^{j_i}_{m_in_i}(g_i)\mat{ccc}{j_1&j_2&j_3\\n_1&n_2&n_3}, \label{eq:Tdecompos}
\end{equation}
with $\displaystyle \mat{ccc}{j_1&j_2&j_3\\n_1&n_2&n_3}$ the Wigner's $3j$ symbol of $SU(2)$. 
The sum on ${\{j\}}$ denotes the summation over $j_1,~j_2$ and $j_3$ (resp. for $\{m\}$ and $\{n\}$).
The coefficients $T^{m_1m_2m_3}_{j_1j_2j_3}$ can be computed using the orthogonality of Wigner matrices as
\begin{equation}
      T^{m_1m_2m_3}_{j_1j_2j_3}=\int \dd g_1 \dd g_2 \dd g_3 \sum_{\{n\}}T(g_1,g_2,g_3)\prod_{i=1}^3\sqrt{2j_i+1}\bar{D}^{j_i}_{m_in_i}(g_i)\mat{ccc}{j_1&j_2&j_3\\n_1&n_2&n_3}, \label{eq:Tdecomposcoeff}
\end{equation}
Using this decomposition, the integral over the Wigner matrix can be performed explicitly and the Boulatov action~\eqref{eq:actionTgroup} in spin representation reads \cite{Boulatov:1992vp}
\begin{equation}
      S_B[T] = \sum_{j_1,j_2,j_3} \frac{\mu^2}{2}|T^{m_1,m_2,m_3}_{j_1,j_2,j_3}|^2 - \frac{\lambda}{4!} \sum_{j_1,..,j_6} \sixj{j_1}{j_2}{j_3}{j_4}{j_5}{j_6}  T^{4_{6j}}, \label{eq:actionTspin}
\end{equation}
where the kinetic term is
\begin{equation}
|T^{m_1,m_2,m_3}_{j_1,j_2,j_3}|^2=\sum_{\substack{j_1,j_2,j_3\\m_1,m_2,m_3}}(-1)^{\sum_{i=1}^3(j_i-m_i)}T^{m_1,m_2,m_3}_{j_1,j_2,j_3}T^{-m_1,-m_2,-m_3}_{j_1,j_2,j_3},
\end{equation}
and the term $T^{4_{6j}}$ encodes the contraction of the magnetic indices $m_i$ of the field paralleling the contraction pattern of 3j-symbols to give the $6j$ symbol, {\it i.e.}
\begin{equation}
      T^{4_{6j}}=\sum_{\{j,m\}}(-1)^{\sum_{i=1}^{6}(j_i-m_i)}T^{-m_1,-m_2,-m_3}_{j_1j_2j_3}T^{m_3,m_5,-m_4}_{j_3j_5j_4}T^{m_4,m_2,-m_6}_{j_4j_2j_6}T^{m_1,-m_5,m_1}_{j_6j_5j_1}. \label{eq:T6jdef}
\end{equation}
In this form, the equation of motion~\eqref{eq:boulatoveqTgroup} now becomes
\begin{equation}
      \mu^2T^{m_1,m_2,m_3}_{j_1,j_2,j_3} = \frac{\lambda}{3!} \sum_{j_4,j_5,j_6} \sixj{j_1}{j_2}{j_3}{j_4}{j_5}{j_6} T^{4_{6j}}_{\backslash \{m_1,m_2,m_3\}}, \label{eq:boulatoveqTspin}
\end{equation}
where
\begin{equation}
      T^{4_{6j}}_{\backslash \{m_1,m_2,m_3\}}=\sum_{m_4,m_5,m_6}(-1)^{\sum_{i=4}^6(j_i-m_i)}T^{m_3,m_5,-m_4}_{j_3j_5j_4}T^{m_4,m_2,-m_6}_{j_4j_2j_6}T^{m_6,-m_5,m_1}_{j_6j_5j_1}.
\end{equation}
is the field $T$ where the three magnetic indices $m_1$,$m_2$ and $m_3$ are not summed on.

In the rest of this article we will use this spin representation~\eqref{eq:actionTspin} of the Boulatov action.
%The perturbative expansion of the Boulatov model
%has a clear interpretation as a discrete $3D$ pseudo-manifold
%\footnote{For the definition of pseudo-manifold and its differences with manifold, see for example \cite{David:1992jw,Krajewski:2012aw}} from this point of view. The interaction term encodes a tetrahedron whose vertices are copies of $T$ and whose edges are labeled by representations of $SU(2)$. The kinetic term then encodes how these tetrahedra are glued together. Thus, the Boulatov model can be viewed as a toy model of three dimensional lattice gravity.

Finally, before we discuss how matter degrees of freedom are included in the Boulatov model, let us recall some facts concerning its interpretation as a model for 3d euclidean quantum gravity. Its Feynman amplitudes are given by lattice gravity path integrals corresponding to a discretization of 1st order Palatini 3d gravity on the lattices dual to the Feynman diagrams. Equivalently, they correspond to the Ponzano-Regge spin foam amplitudes, known to correspond to a state sum formulation of the same quantum theory. In absence of matter, this quantum theory only describe flat 3d geometries and the partition function, for given lattice, counts the moduli space of flat connections for the given topology. The Boulatov model extends thus this quantum geometric content with a sum over lattices of all topologies (all possible gluings of 3-simplices), including pseudomanifold configurations. The quantum geometric effect of this additional sum is not fully understood. While the sum over lattices with the same topology is most likely irrelevant from the physical point of view, and, once controlled, should give at most a rescaling of the amplitudes, the sum over different topologies may have more interesting physical consequences. Tree level amplitudes, however, should not encode such topological effects, thus it is natural to interpret classical solutions of the Boulatov model as still corresponding to flat space. Clearly, further work is needed to improve our understanding of these issues. 

\subsection{Matter degrees of freedom}

GFTs are not usual QFTs describing a theory \emph{on} spacetime, but QFTs \emph{of} spacetime, tentatively describing its quantum building blocks and their dynamics \cite{Oriti:2011jm}. Their dynamical fields do not live, accordingly, on a manifold interpreted as spacetime, and on which the usual metric and matter fields of GR and standard model live. Such spacetime manifold simply does not appear in the fundamental formulation of the theory, as one does not find coordinates and directions on such manifold. 

According to the relational strategy for the construction of diffeomorphism-invariant observables in classical and quantum gravity~\cite{Rovelli:1990ph,Rovelli:2004tv}, spacetime localization should be defined in terms of appropriately chosen dynamical degrees of freedom, {\it internal} to the theory, rather than absolute external directions. For example, matter coupled to gravity can play the role of a physical reference frame~\cite{Rovelli:1990ph}, i.e. of rods and clocks. While different choices of matter can be used to fill that role, the simplest framework is to use free massless (minimally coupled) scalar fields $\chi_i$~\cite{Oriti:2016qtz,Oriti:2017}. 

In three dimensions, one needs three scalar fields, and they can be combined into a vector $\vec{\chi}=(\chi_1,\chi_2,\chi_3)$, to be added to the GFT data to localize in space and time, in a continuum approximation, GFT observables and their dynamics.

We now exhibit a specific construction extending the Boulatov model to include such matter degrees of freedom. Other constructions can be found in the cited GFT literature.
Requiring the theory to be invariant under translations $\chi_i\to\chi_i+a_i$ allows for a kinetic term in the action~\eqref{eq:actionTgroup} defined as $\displaystyle \nabla =\left(\frac{\partial }{\partial \chi_1},\frac{\partial }{\partial \chi_2},\frac{\partial }{\partial \chi_3}\right)$ and thus extends $T(g_1,g_2,g_3)$ to $T(g_1,g_2,g_3;\vec{\chi}):SU(2)^3\times\mathbb{R}^3\to\mathbb{C}$. The new action writes~\footnote{Note that this action should not be confused with that of a dynamical Boulatov model of~\cite{Geloun:2013} where a Laplace-Beltrami operator acts on the group manifold.}
\begin{align}
    S[T]&=\int [\dd g]^3\dd^3\vec{\chi}\left[\frac{1}{2}\nabla T(g_1,g_2,g_3;\vec{\chi})\nabla \bar{T}(g_1,g_2,g_3;\vec{\chi})+ \frac{\mu^2}{2}T(g_1,g_2,g_3;\vec{\chi})\bar{T}(g_1,g_2,g_3;\vec{\chi})\right] \nonumber \\
      &-\frac{\lambda}{4!}\int \prod_{i=1}^6 \dd g_i\dd^3\vec{\chi} T(g_1,g_2,g_3;\vec{\chi}) T(g_3,g_5,g_4;\vec{\chi})T(g_4,g_2,g_6;\vec{\chi})T(g_6,g_5,g_1;\vec{\chi}).\label{eq:actionTchigroup}
\end{align}
   This yields a modified equation of motion: 
\begin{align}
      &\nabla^2 T(g_3,g_2,g_1;\vec{\chi})+\mu^2T(g_3,g_2,g_1;\vec{\chi})\nonumber\\
  &=\frac{\lambda}{3!}\int\dd g_4\dd g_5\dd g_6T(g_3,g_5,g_4;\vec{\chi})T(g_4,g_2,g_6;\vec{\chi})T(g_6,g_5,g_1;\vec{\chi}). \label{eq:boulatoveqTchigroup} 
\end{align}
The corresponding action in spin representation writes
\begin{align}
        S_B[T(\vec{\chi})] &= \sum_{j_1,j_2,j_3}\int\dd^3\vec{\chi}\left[\frac{1}{2}\left|\nabla T^{m_1,m_2,m_3}_{j_1,j_2,j_3}(\vec{\chi})\right|^2+ \frac{\mu^2}{2}\left|T^{m_1,m_2,m_3}_{j_1,j_2,j_3}(\vec{\chi})\right|^2\right. \nonumber \\
    & \left.- \frac{\lambda}{4!} \sum_{j_1,..,j_6} \sixj{j_1}{j_2}{j_3}{j_4}{j_5}{j_6}  \int\dd^3\vec{\chi}T(\vec{\chi})^{4_{6j}}\right], \label{eq:actionTchispin}
\end{align}
leading to the following equation of motion:
\begin{align}
      \nabla^2 T^{m_1,m_2,m_3}_{j_1,j_2,j_3}(\vec{\chi})+\mu^2T^{m_1,m_2,m_3}_{j_1,j_2,j_3}(\vec{\chi}) = \frac{\lambda}{3!} \sum_{j_4,j_5,j_6} \sixj{j_1}{j_2}{j_3}{j_4}{j_5}{j_6} T(\vec{\chi})^{4_{6j}}_{\backslash \{m_1,m_2,m_3\}}. \label{eq:boulatoveqTchispin}
\end{align}

Before we take our next step in the derivation, we point out that TGFT models of the above \lq extended\rq type, including both local and non-local (tensorial) directions have also been analysed, recently, from the point of view fo their renormalization group flow \cite{Geloun:2023ray} and their critical behaviour (at mean field level) \cite{Marchetti:2020xvf, Marchetti:2022nrf}.

\subsection{Classical homogeneous solutions to the Boulatov model}

We first exhibit a homogeneous classical solution of the Boulatov model, independent of $\vec{\chi}$. In the homogeneous restriction, equation~\eqref{eq:boulatoveqTchigroup} reduces to equation~\eqref{eq:boulatoveqTgroup}. The dependence on the scalar matter degrees of freedom $\chi_i$ will only enter perturbatively around this solution. A one-parameter family of solutions parametrized by normalised class functions $f:SU(2)\to \mathbb{C}$ was proposed in~\cite{Fairbairn:2007sv}, with associated field $T_f$ is given by
\begin{equation}
     T_f(g_1,g_2,g_3)=\mu\sqrt{\frac{3!}{\lambda}}\int \dd h\delta(g_1h)f(g_2h)\delta(g_3h), \label{eq:Tfsolgroup}
\end{equation}
where $\delta(g)$ is the Dirac delta function over the group $SU(2)$ such that
\begin{equation}
      \int \dd h \delta(h)=1, ~\int \dd h\delta(h)f(h)=f(I),
\end{equation}
with $I$ is the identity of $SU(2)$ group. 

The function $f(g)$ is normalised, {\it i.e.}
\begin{equation}
  \int \dd h f(h)^2=1. \label{eq:fnorcon}
\end{equation}
We can also write this solution in spin representation, substituting the solution \eqref{eq:Tfsolgroup} into the general Peter-Weyl coefficients \eqref{eq:Tdecomposcoeff}, to obtain
\iea{
  (T_f)_{j_1,j_2,j_3}^{m_1,m_2,m_3} = \mu\sqrt{\frac{3!}{\lambda}} \sqrt{d_{j_1}d_{j_3}} \sum_{l_2} f^{j_2}_{m_2,l_2} \threej{j_1}{j_2}{j_3}{m_1}{l_2}{m_3}, \label{eq:Tfsolspin}
}
where $f^{j}_{mn}$ is the coefficients in the Peter-Weyl decomposition of $f(g)$
\begin{equation}
     f^{j}_{mn}=\sqrt{2j+1}\int \dd g f(g)\bar{D}^{j}_{mn}(g), \label{eq:fdecomposcoeff}
\end{equation}
and the corresponding normalisation condition becomes
\iea{
  \sum_{j,m,n} (-1)^{m-n}f^j_{mn}f^j_{-m,-n}=1. \label{eq:fnorspin}
}

Before we move on, let us give some remark on this class of solutions and its special form which is regularised by ``heat kernel''. First, the classical solution\eqref{eq:Tfsolgroup} is asymmetrical in the group elements $g_i$ since $g_2$ plays a preferential role through $f$. Restricting attention to this special asymmetric solutions is thus a form of spontaneous symmetry breaking of the model. Second, the presence of Dirac delta function in ~\eqref{eq:Tfsolgroup} leads to divergences. For example, the action~\eqref{eq:boulatoveqTchigroup} is divergent when evaluated on this solution due to the factor $\delta(I)$ appearing. This can also be seen from its Peter-Weyl expansion
\begin{equation}
     \delta(g)= \sum_{j,m} (2j+1) D^j_{mm}(g).
\end{equation}
Thus we need to regularize our solution. This can be achieved by different methods. For example, one strategy is to introduce a sharp cut-off parameter $J$ in the Peter-Weyl expansion of $T(g_1,g_2,g_3)$, thus making the action finite. Here, we will instead use a \emph{heat kernel} regularization to make all quantities well-defined, at the cost of only having an approximate solution to the equations of motion. To do so, we introduce a new real parameter $\varepsilon$. For any function $f$ of $SU(2)$ with coefficients $f^j_{mn}$ in its Peter-Weyl expansion, we define its heat kernel regularization as ($d_j=2j+1$)
\begin{equation}
    f_\epsilon(g) = \sum\limits_{j,m,n} \sqrt{d_j} f^j_{mn} D^j_{mn}(g) \ee^{-\varepsilon C_j}
\end{equation}
with $C_j$ is the Casimir of the spin $j$ representation of $SU(2)$. This function is well-defined for any $\epsilon>0$ and its leading order when $\varepsilon \rightarrow 0$ is the initial function $f$. In particular, for the Dirac delta function of $SU(2)$, its heat kernel regularization is
\begin{equation}
    \delta_\varepsilon(g)=\sum_{j,m}d_j D^j_{mm}(g)\ee^{-\varepsilon C_j}.
\end{equation}

Note that this function is not normalised. If we denote its norm as $\alpha_\epsilon^{-2}$, the normalised function associated to $\delta_\varepsilon$ is ($d_j=2j+1$)
\begin{equation}
    \Delta_\varepsilon(g) =  \alpha_\varepsilon \sum\limits_{j,m,n} \sqrt{d_j}(\Delta_\varepsilon)^j_{mn} D^j_{mn}(g) \ee^{-\varepsilon C_j},
\end{equation}
where the Peter-Weyl coefficients $(\Delta_\varepsilon)^j_{mn}$ has the form
\begin{equation}
     (\Delta_\varepsilon)^j_{mn}=\alpha_\varepsilon \sqrt{d_j} \delta_{mn} \ee^{-\varepsilon C_j}. \label{eq:Deltavarepsiloncoef}
\end{equation}

Using $\Delta_\varepsilon(g)$, we can build now a regularized and symmetric field
\begin{equation}
     T_\varepsilon(g_1,g_2,g_3)=\mu\sqrt{\frac{3!}{\lambda}}\int \dd h\delta_\varepsilon(g_1h)\Delta_\varepsilon(g_2h)\delta_\varepsilon(g_3h)=\mu\alpha_\varepsilon\sqrt{\frac{3!}{\lambda}}\int \dd h\delta_\varepsilon(g_1h)\delta_\varepsilon(g_2h)\delta_\varepsilon(g_3h).\label{eq:Tepsilonsolgroupreg}
\end{equation}

However, $T_\varepsilon(g_1,g_2,g_3)$ is only an approximate solution of the homogeneous equation of motion, i.e. it is a solution at leading order in $\varepsilon$. The coefficients of its Peter-Weyl expansion are given by 
\begin{equation}
    \left(T_\varepsilon\right)^{m_1m_2m_3}_{j_1j_2j_3}= \mu\alpha_{\varepsilon}\sqrt{\frac{3!}{\lambda}}\prod_{i=1}^3\sqrt{d_{j_i}}\ee^{-\varepsilon C_{j_i}} \mat{ccc}{j_1&j_2&j_3\\m_1&m_2&m_3}. \label{eq:PW_eps}
\end{equation}
In particular, when $\varepsilon \rightarrow 0$ the coefficients of $T$ are given by the $3j$ symbol, which is a (regularized) classical solution to the Boulatov model.\\

In the following calculation, we will use the solution (\ref{eq:Tfsolgroup}) and will briefly illustrate the special case (\ref{eq:Tepsilonsolgroupreg}) separately.

\section{Amit-Roginsky-like model from perturbations around classical Boulatov solutions}
\label{sec:emergenceAR}

In this section, we obtain an AR-like action from the Boulatov GFT action by considering specific perturbations around the classical solution constructed in the previous section.

The AR model~\cite{Amit:1979ev} is a cubic field theory of a vector field $\phi$ self-coupled through the $3j$ symbol for a fixed value of the spin $j$. Its action is
\begin{align}
  S_{AR}[\phi]=&\int\dd^dx\left\{\frac{1}{2}\sum_m(-1)^{j-m}\left[(\nabla \phi^j_m)(\nabla \phi^j_{-m})+\mu\phi^j_m\phi^j_{-m}\right]\right. \nonumber \\
  &+\left.\sum_{m_1,m_2,m_3}\frac{\lambda}{3!}\sqrt{2j+1}\mat{ccc}{j&j&j\\m_1&m_2&m_3}\phi^j_{-m_1}\phi^j_{-m_2}\phi^j_{-m_3}\right\}, \label{eq:actionAR}
\end{align}
where $\displaystyle \nabla$ is the gradient operator.

It was recently pointed out in~\cite{Benedetti:2020iku} that the large $N(=2j+1)$ limit of the AR model is given by the~\emph{melonic} graphs. As mentioned in the introduction, this feature is shared with $0$-dimensional tensor models~\cite{Dartois:2013he, Carrozza:2015adg} and topological GFTs as well. 

\subsection{Perturbations over homogeneous Boulatov solution}

Following~\cite{Fairbairn:2007sv}, we consider two-dimensional perturbations over the Boulatov model, which depend on matter reference frame $\vec{\chi}$. The field becomes
\begin{equation}
    T_\psi(g_1,g_2,g_3;\vec{\chi})=T_f(g_1,g_2,g_3)+\xi \psi(g_1,g_3;\vec{\chi}), \label{eq:2dpert_solution}
\end{equation}
where $T_f(g_1,g_2,g_3)$ is the solution to the equation of motion given by equation~\eqref{eq:Tfsolgroup} with \eqref{eq:Tepsilonsolgroupreg} a special case, and $\psi(g_1,g_3;\vec{\chi})$ is a $2$D-perturbation with 
$\xi$ a real parameter 
$0<\xi\ll 1$. The Peter-Weyl coefficients of the perturbation are given by
\begin{align}
      \psi^{m_1m_2m_3}_{j_1j_2j_3}(\vec{\chi})&=\sum_{\{n\}}\int[\dd g]^3\psi(g_1,g_3;\vec{\chi})\prod_{i=1}^3\sqrt{2j_i+1}\bar{D}^{j_i}_{m_in_i}\mat{ccc}{j_1&j_2&j_3\\n_1&n_2&n_3} \nonumber \\
  &\equiv \delta^{j_2,0}\delta_{m_2,0}\delta^{j_1,j_3}\sqrt{2j_1+1}\psi^{j_1}_{m_1,m_3}(\vec{\chi}).
\end{align}
 In order to obtain the equation above, we used the fact that $j_2=0$ (see equation~\eqref{eq:3jj2eq0} in the appendix~\ref{sec:su2recoupling}). The scaling factor $\sqrt{2j_1+1}$ is introduced for later convenience. The Peter-Weyl coefficients of the perturbed solution write
\begin{equation}
      (T_\psi)^{m_1m_2m_3}_{j_1j_2j_3}(\vec{\chi})=T^{m_1m_2m_3}_{j_1j_2j_3}+\xi \delta^{j_2,0}\delta_{m_2,0}\delta^{j_1,j_3}\psi^{j_1}_{m_1,m_3}(\vec{\chi}). \label{eq:Tpsispin}
\end{equation}

Substituting~\eqref{eq:Tpsispin} into the action~\eqref{eq:actionTchispin}, we get the action for the perturbed solution
\begin{equation}
      S_B[T_\psi(\vec{\chi})]=S_B[T] + \xi^2\cdot S_{\mathrm{eff}}[\psi]+\mathcal{O}(\xi^4),
\end{equation}
where the first order in $\xi$ vanishes since $T_f$ is a solution to the equation of motion. The action $S_{\mathrm{eff}}[\psi]$ represents the effective action of the perturbation field $\psi^j_{mn}$ and contains corrections up to $\xi$. Therefore, $\xi^2 S_{\mathrm{eff}}[\psi]$ contains corrections up to order $\xi^3$. 

In the following subsection, we develop each term arising from the Boulatov model in the effective action and give sufficient conditions on the coefficients $(T_\epsilon)^{m_1m_2m_3}_{j_1j_2j_3}$ such that the effective action $S_{\mathrm{eff}}[\psi]$ takes the form of an AR-like action. Since the AR model involves a vector field transforming in a representation of $SU(2)$ and thus carrying only one magnetic index $m$. Hence, we will specialize the perturbations to
\begin{equation}
      \psi^{j_1}_{m_1m_3}(\vec{\chi})=\sum_{m}\sqrt{2j_1+1}\phi^{j_1}_{m}(\vec{\chi})\mat{ccc}{j_1&j_1&j_1\\m_1&m&m_3}.
      \label{eq:psiphirelation}
\end{equation}
and check that this particular choice of perturbations satisfies all the required conditions.

\subsection{Conditions for the emergence of an Amit-Roginsky-like model} 

In order to simplify the notations, we omit from now on to explicitly write the dependency on the vector $\vec{\chi}$, which should always be assumed. 

\subsubsection{Quadratic terms}

Substituting perturbation~\eqref{eq:Tpsispin} into the Boulatov action~\eqref{eq:actionTchispin}, the quadratic term in $\xi$ receives three kinds of contributions.
The kinetic term of Boulatov model gives rise to one contribution of the form $\psi\psi$. Then, the interaction term gives two distinct type of contributions, depending on how the two perturbation fields are connected in the action. 

Schematically, these two terms can be represented as $TT\psi\psi$ when the two perturbation fields $\psi_{m_am_b}$ share one magnetic index, and $T\psi T\psi$ represents the terms that share none. They yield different contributions to the effective action.

\vspace{10pt}
\paragraph{Term $\psi\psi$\\}

The kinetic term $\sum_{j_1,j_2,j_3}\left|(T_\psi)^{m_1,m_2,m_3}_{j_1,j_2,j_3}(\vec{\chi})\right|^2$ of the Boulatov action
gives the following contribution to the effective action:
\begin{align}
      &\sum_{\substack{j_1,j_2,j_3\\m_1,m_2,m_3}}(-1)^{\sum_{i=1}^3(j_i-m_i)}\left[\delta^{j_2,0}\delta_{m_2,0}\delta^{j_1,j_3}\psi^{j_1}_{m_1,m_3}\right]\left[\delta^{j_2,0}\delta_{-m_2,0}\delta^{j_1,j_3}\psi^{j_1}_{-m_1,-m_3}\right]  \nonumber \\
  &=\sum_{\substack{j_1,m_1,m_3\\m,m'}}(-1)^{2j_1-m_1-m_3}\phi^{j_1}_{m}\phi^{j_1}_{m'}(2j_1+1)\mat{ccc}{j_1&j_1&j_1\\m_1&m&m_3}\mat{ccc}{j_1&j_1&j_1\\-m_1&m'&-m_3} \nonumber\\
  &=\sum_{j_1,m_1}(-1)^{j_1-m_1}\phi^{j_1}_{m_1}\phi^{j_1}_{-m_1}. 
\end{align}
This term is simply the quadratic term of the AR action~\eqref{eq:actionAR}.
Note that this contribution is independent of the solution $T^{m_1m_2m_3}_{j_1j_2j_3}$ and therefore it does not impose any restriction on the homogeneous solution to be considered.

\vspace{10pt}
\paragraph{Terms $TT\psi\psi$\\}

There are four terms of type $TT\psi\psi$. Each of them contributes to the effective action as
\begin{align}
     &\sum_{\substack{m_1,\cdots,m_6\\j_1,\cdots,j_6}}\sixjsim (-1)^{\sum_i(j_i-m_i)}T^{-m_1,-m_2,-m_3}_{j_1j_2j_3}T^{m_3,m_5,-m_4}_{j_3j_5j_4} \nonumber\\
	  &\times\delta^{j_2,0}\delta_{m_2,0}\delta^{j_4,j_6}\psi^{j_4}_{m_4,-m_6}\delta^{j_5,0}\delta_{m_5,0}\delta^{j_6,j_1}\psi^{j_6}_{m_6,m_1} \nonumber\\
	  =& \sum_{\substack{m_1,m_3,m_4,m_6\\j_1,j_3,j_4,j_6}}\matcurl{ccc}{j_1&0&j_3\\j_4&0&j_6}(-1)^{\sum_{i\neq2,5}(j_i-m_i)}T^{-m_1,0,-m_3}_{j_1,0,j_3}T^{m_3,0,-m_4}_{j_3,0,j_4}\delta^{j_4,j_6}\delta^{j_6,j_1}\psi^{j_1}_{m_4,-m_6}\psi^{j_1}_{m_6,m_1} \nonumber\\
	  %=& \sum_{\substack{m_1,m_3,m_4,m_6\\j_1}}(-1)^{6j_1-\sum_{i\neq 2,5}m_i}T^{-m_1,0,-m_3}_{j_1,0,j_1}T^{m_3,0,-m_4}_{j_1,0,j_1}\psi^{j_1}_{m_4,-m_6}\psi^{j_1}_{m_1,m_6} \nonumber\\
	  =&\sum_{j_1,m_1,m_6,m_4}\left[\sum_{m_3}(-1)^{-m_3-m_4}T^{-m_1,0,-m_3}_{j_1,0,j_1}T^{m_3,0,-m_4}_{j_1,0,j_1}\right](-1)^{2j_1-m_1-m_6}\psi^{j_1}_{m_4,-m_6}\psi^{j_1}_{m_1,m_6} 
\end{align}

%Terms of the form $\displaystyle \sum_{j_1,m_1,m_3}(-1)^{2j_1-m_1-m_3}\psi^{j_1}_{m_1,m_3}\psi^{j_1}_{-m_1,-m_3}$ can take the form of a contribution to the AR kinetic term with proper choice of the perturbation~\eqref{eq:psiphirelation}.
Thus if the homogeneous solution $T^{m_1m_2m_3}_{j_1j_2j_3}$ is such that
\begin{equation}
    \sum_{m_3}(-1)^{-m_3-m_4}T^{-m_1,0,-m_3}_{j_1,0,j_1}T^{m_3,0,-m_4}_{j_1,0,j_1}=c_{1,j_1} \delta_{m_1,-m_4} \label{eq:TTpsipsicon}
\end{equation}
for some coefficients $c_{1,j_1}$ then we get
\begin{align}
    	  &\sum_{\substack{m_1,\cdots,m_6\\j_1,\cdots,j_6}}\sixjsim (-1)^{\sum_i(j_i-m_i)}T^{-m_1,-m_2,-m_3}_{j_1j_2j_3}T^{m_3,m_5,-m_4}_{j_3j_5j_4}\nonumber \\
	  &\times\delta^{j_2,0}\delta_{m_2,0}\delta^{j_4,j_6}\psi^{j_4}_{m_4,-m_6}\delta^{j_5,0}\delta_{m_5,0}\delta^{j_6,j_1}\psi^{j_6}_{m_6,m_1} \nonumber\\
	  &=\sum_{j_1,m_1,m_6,m_4}\left[\sum_{m_3}(-1)^{-m_3-m_4}T^{-m_1,0,-m_3}_{j_1,0,j_1}T^{m_3,0,-m_4}_{j_1,0,j_1}\right](-1)^{2j_1-m_1-m_6}\psi^{j_1}_{m_4,-m_6}\psi^{j_1}_{m_1,m_6}\nonumber\\
\end{align}

And we specializing to the perturbation~\eqref{eq:psiphirelation} we get
\begin{align}
    &\sum_{\substack{m_1,\cdots,m_6\\j_1,\cdots,j_6}}\sixjsim (-1)^{\sum_i(j_i-m_i)}T^{-m_1,-m_2,-m_3}_{j_1j_2j_3}T^{m_3,m_5,-m_4}_{j_3j_5j_4} \nonumber\\
	  &\times\delta^{j_2,0}\delta_{m_2,0}\delta^{j_4,j_6}\psi^{j_4}_{m_4,-m_6}\delta^{j_5,0}\delta_{m_5,0}\delta^{j_6,j_1}\psi^{j_6}_{m_6,m_1} \nonumber\\
   &=\sum_{j_1,m_1}c_{1,j_1}(-1)^{j_1-m_1}\phi^{j_1}_{m_1}\phi^{j_1}_{-m_1}.
\end{align}
which is the kinetic term of the AR model.

Given a homogeneous solution, the proportionality coefficient $c_{1,j_1}$ can be explicitly computed. Later, we will obtain another condition given by equation~\eqref{eq:tpsipsipsicon} that will be stronger than condition~\eqref{eq:TTpsipsicon} obtained here. Thus, this condition will be automatically satisfied when Equation~\eqref{eq:tpsipsipsicon} is. 
%We will see later that to get a proper interaction term for $\psi$ field, the condition~\eqref{eq:tpsipsipsicon} is required, and then condition~\eqref{eq:TTpsipsicon} would be satisfied automatically.
%on $T\psi\psi\psi$ term, this condition will be satisfied automatically. We will see that this condition is automatically satisfied , which is given by Equation~\eqref{eq:tpsipsipsicon}.

\vspace{10pt}
\paragraph{Term $T\psi T\psi$\\}

The remaining two quadratic contributions from the interaction term of the Boulatov model are of the form $T\psi T\psi$. Each of these terms contributes to the effective action as
\begin{align}
    &\sum_{\substack{m_1,\cdots,m_6\\j_1,\cdots,j_6}}(-1)^{\sum_i(j_i-m_i)}T^{-m_1,-m_2,-m_3}_{j_1j_2j_3}\delta^{j_5,0}\delta_{m_5,0}\delta^{j_3,j_4}\psi^{j_3}_{m_3,-m_4} \nonumber\\
	  & \times T^{m_4,m_2,-m_6}_{j_4j_2j_6}\delta^{j_5,0}\delta_{m_5,0}\delta^{j_1,j_6}\psi^{j_6}_{m_6m_1}\sixjsim \nonumber\\
  %&=\sum_{\substack{m_1,m_2,m_3,m_4,m_6\\j_1,j_2,j_3}}(-1)^{2j_1+2j_3+j_2-\sum_{i\neq5}m_i}T^{-m_1,-m_2,-m_3}_{j_1,j_2,j_3}T^{m_4,m_2,-m_6}_{j_3,j_2,j_1}\psi^{j_3}_{m_3,-m_4}\psi^{j_1}_{m_6,m_1}\matcurl{ccc}{j_1&j_2&j_3\\j_3&0&j_1} \nonumber\\
  %&=\sum_{\substack{m_1,m_2,m_3,m_4,m_6\\j_1,j_2,j_3}}(-1)^{2j_1+2j_3+j_2-\sum_{i\neq5}m_i}T^{-m_1,-m_2,-m_3}_{j_1,j_2,j_3}T^{m_4,m_2,-m_6}_{j_3,j_2,j_1}\psi^{j_3}_{m_3,-m_4}\psi^{j_1}_{m_6,m_1} \nonumber\\
  %&\times\frac{1}{\sqrt{(2j_1+1)(2j_3+1)}}(-1)^{j_1+j_2+j_3}\{j_1,j_2,j_3\} \nonumber\\  
  &=\sum_{\substack{m_1,m_3,m_4,m_6\\j_1,j_3}}(-1)^{j_1+j_3-m_4-m_6}\psi^{j_3}_{m_3,-m_4}\psi^{j_1}_{m_6,m_1} \frac{1}{\sqrt{(2j_1+1)(2j_3+1)}} \nonumber\\
  &\times \sum_{j_2,m_2}(-1)^{\sum_{i=1}^3(2j_i-m_i)}T^{-m_1,-m_2,-m_3}_{j_1,j_2,j_3}T^{m_4,m_2,-m_6}_{j_3,j_2,j_1}. 
\end{align}
For a general solution of the equation of motion, this term leads to a non-diagonal kinetic term for the $\psi$ field. If the homogeneous solution satisfies the condition
\begin{equation}
      \sum_{j_2,m_2}(-1)^{\sum_{i=1}^3(2j_i-m_i)}T^{-m_1,-m_2,-m_3}_{j_1,j_2,j_3}T^{m_4,m_2,-m_6}_{j_3,j_2,j_1}=c_{2,j_1}c_{2,j_3}\delta_{m_1,-m_6}\delta_{m_3,m_4},
  \label{eq:tpsitpsicon}
\end{equation}
then this contribution becomes
\begin{align}
       &\sum_{\substack{m_1,m_3,m_4,m_6\\j_1,j_3}}(-1)^{j_1+j_3-m_4-m_6}\psi^{j_3}_{m_3,-m_4}\psi^{j_1}_{m_6,m_1} \frac{1}{\sqrt{(2j_1+1)(2j_3+1)}} \nonumber\\
  &\times \sum_{j_2,m_2}(-1)^{\sum_{i=1}^3(2j_i-m_i)}T^{-m_1,-m_2,-m_3}_{j_1,j_2,j_3}T^{m_4,m_2,-m_6}_{j_3,j_2,j_1} \nonumber\\
  %=& \sum_{\substack{m_1,m_3,m_4,m_6\\j_1,j_3}}(-1)^{j_1+j_3-m_4-m_6}\psi^{j_3}_{m_3,-m_4}\psi^{j_1}_{m_6,m_1} \frac{1}{\sqrt{(2j_1+1)(2j_3+1)}} c_{2,j_1}c_{2,j_3}\delta_{m_1,-m_6}\delta_{m_3,m_4}, \\
  =&\left[\sum_{j_1,m_1}(-1)^{j_1-m_1}\frac{c_{2,j_1}}{\sqrt{2j_1+1}}\psi^{j_1}_{m_1,-m_1}\right]^2.
\end{align}

When specializing to the type of perturbation given by equation~\eqref{eq:psiphirelation}, we get
\begin{align}
      \sum_{j_1,m_1}(-1)^{j_1-m_1}\frac{c_{2,j_1}}{\sqrt{2j_1+1}}\psi^{j_1}_{m_1,-m_1}=&\sum_{j_1,m_1,m}(-1)^{j_1-m_1}c_{2,j_1}\phi^{j_1}_{m}\mat{ccc}{j_1&j_1&j_1\\m_1&m&-m_1}, \nonumber\\
  %=&\sum_{j_1}c_{2,j_1}\phi^{j_1}_{0}\sum_{m_1}(-1)^{j_1-m_1}\mat{ccc}{j_1&j_1&j_1\\m_1&0&-m_1}, \nonumber\\
  =&\sum_{j_1}c_{2,j_1}\phi^{j_1}_{0}\delta_{j_1,0}\sqrt{2j_1+1}, \nonumber\\
  =&c_{2,0}\phi^0_0.
\end{align}
where we used the equation~\eqref{eq:3jm3eq0}. Therefore, the quadratic term obtained from the $T\psi T\psi$ term can also be made diagonal under the right choice of homogeneous solution and perturbations.

\subsubsection{Cubic terms} 

There is only one type of cubic contribution which comes from the interaction term of the Boulatov model. These terms take the form $T\psi\psi\psi$; there are four such terms and they each contribute as follows:
\begin{align}
     & \sum_{\{j,m\}}(-1)^{\sum\limits_i (j_i-m_i)}T^{-m_1,-m_2,-m_3}_{j_1,j_2,j_3}\delta^{j_5,0}\delta_{m_5,0}\psi^{j_3}_{m_3,-m_4}\delta^{j_2,0}\delta_{m_2,0}\psi^{j_4}_{m_4,-m_6}\delta^{j_5,0}\delta_{m_5,0}\psi^{j_6}_{m_6,m_1}\sixjsim \nonumber\\
 %&= \sum_{\substack{m_1,m_3,m_4,m_6\\j_1}}(-1)^{4j_1-\sum_{i\neq 2,5}m_i}T^{-m_1,0,-m_3}_{j_1,0,j_1}\psi^{j_1}_{m_3,-m_4}\psi^{j_1}_{m_4,-m_6}\psi^{j_1}_{m_6,m_1} \matcurl{ccc}{j_1&0&j_1\\j_1&0&j_1} \nonumber\\
 &= \sum_{\substack{m_1,m_3,m_4,m_6\\j_1}}(-1)^{-\sum\limits_{i\neq 2,5}-m_i}T^{-m_1,0,-m_3}_{j_1,0,j_1}\frac{(-1)^{2j_1}}{2j_1+1}\psi^{j_1}_{m_3,-m_4}\psi^{j_1}_{m_4,-m_6}\psi^{j_1}_{m_6,m_1}.
\end{align}
If we impose that the homogeneous solution $T$ satisfies
\begin{equation}
      T^{-m_1,0,-m_3}_{j_1,0,j_1} = c_{3,j_1}(-1)^{-m_3}\delta_{m_1,-m_3}, \label{eq:tpsipsipsicon}
\end{equation}
for some coefficient $c_{3,j_1}$, this contribution becomes
\begin{align}
     &\sum_{\substack{m_1,m_3,m_4,m_6\\j_1}}(-1)^{-\sum_{i\neq 2,5}m_i}T^{-m_1,0,-m_3}_{j_1,0,j_1}\frac{(-1)^{2j_1}}{2j_1+1}\psi^{j_1}_{m_3,-m_4}\psi^{j_1}_{m_4,-m_6}\psi^{j_1}_{m_6,m_1} \nonumber\\
 &=\sum_{\substack{m_3,m_4,m_6\\j_1}}(-1)^{2j_1-m_3-m_4-m_6}\frac{c_{3,j_1}}{2j_1+1}\psi^{j_1}_{m_3,-m_4}\psi^{j_1}_{m_4,-m_6}\psi^{j_1}_{m_6,-m_3}.\\
\end{align}

And when specializing to perturbation~\eqref{eq:psiphirelation}, this contribution becomes
\begin{align}
    &\sum_{\substack{m_1,m_3,m_4,m_6\\j_1}}(-1)^{-\sum_{i\neq 2,5}m_i}T^{-m_1,0,-m_3}_{j_1,0,j_1}\frac{(-1)^{2j_1}}{2j_1+1}\psi^{j_1}_{m_3,-m_4}\psi^{j_1}_{m_4,-m_6}\psi^{j_1}_{m_6,m_1} \\
     &= \sum_{\substack{m_3,m_4,m_6\\j_1}}(-1)^{2j_1-m_3-m_4-m_6}\frac{c_{3,j_1}}{2j_1+1}\sum_{m,m',m''}\phi^{j_1}_{m}\phi^{j_1}_{m'}\phi^{j_1}_{m''} \nonumber\\
 &\times\mat{ccc}{j_1&j_1&j_1\\m_3&m&-m_4}\mat{ccc}{j_1&j_1&j_1\\m_4&m'&-m_6}\mat{ccc}{j_1&j_1&j_1\\m_6&m''&-m_3} \nonumber\\
 %&=\sum_{\substack{m,m',m''\\j_1}}\frac{c_{3,j_1}}{2j_1+1}\phi^{j_1}_{m}\phi^{j_1}_{m'}\phi^{j_1}_{m''}(-1)^{2j_1} \nonumber\\
 &\times (-1)^{j_1}\sum_{m_3,m_4,m_6}(-1)^{3j_1-m_3-m_4-m_6}\mat{ccc}{j_1&j_1&j_1\\m&-m_4&m_3}\mat{ccc}{j_1&j_1&j_1\\m_6&m''&-m_3}\mat{ccc}{j_1&j_1&j_1\\-m_6&m_4&m'} \nonumber\\
% &=& \sum_{\substack{m,m',m''\\j_1}}\frac{(-1)^{3j_1}c_{3,j_1}}{2j_1+1}\matcurl{ccc}{j_1&j_1&j_1\\j_1&j_1&j_1}\phi^{j_1}_{m}\phi^{j_1}_{m'}\phi^{j_1}_{m''}\mat{ccc}{j_1&j_1&j_1\\m&m''&m'} \\
 &=\sum_{\substack{m,m',m''\\j_1}}\frac{c_{3,j_1}}{2j_1+1} \matcurl{ccc}{j_1&j_1&j_1\\j_1&j_1&j_1}\phi^{j_1}_{m}\phi^{j_1}_{m'}\phi^{j_1}_{m''}\mat{ccc}{j_1&j_1&j_1\\m&m'&m''}.
\end{align}
Where we have used equations~\eqref{eq:3jpermu} and~\eqref{eq:3jsum6j}, and the fact that $(-1)^{2j_1}=1$ since $j_1$ here has to be an integer for the $3j$ symbol not to vanish. Thus, when imposing the condition~\eqref{eq:tpsipsipsicon}, we get a contribution which corresponds to the interaction term of the AR model. 

Furthermore, as mentioned above, when comparing the two conditions~\eqref{eq:TTpsipsicon} and~\eqref{eq:tpsipsipsicon}, we see that the former will be automatically satisfied when the later is as the two coefficients are related through the relation
\begin{equation}
      c_{1,j_1}=c_{3,j_1}^2. \label{eq:c1c3rel}
\end{equation}

\subsection{Emergence of the Amit-Roginsky-like model}

Now we are ready to extract AR model from the Boulatove action~\eqref{eq:actionTchispin}, based on the two conditions \eqref{eq:tpsitpsicon} and~\eqref{eq:tpsipsipsicon} we discussed in the last subsection. Our main result is the effective action~\eqref{eq:actionphi0epsilon} and~\eqref{eq:actionphijepsilon} for each mode $\phi^j_m$ of the perturbation field (defined through equation \eqref{eq:psiphirelation}). We can see that the form of these actions is the same as the AR one~\cite{Amit:1979ev,Benedetti:2020iku}. 

\paragraph{The effective action for the perturbation $\psi$\\} 

For a perturbation of the form given by equation~\eqref{eq:psiphirelation}, it follows from the previous paragraph that the conditions~\eqref{eq:tpsitpsicon}and~\eqref{eq:tpsipsipsicon} are satisfied. The effective action for the vector perturbation $\phi^{j}_{m}(\vec{\chi})$ then becomes
\begin{equation}
      S[\phi^{j}_{m}]= S_0[\phi^{0}_{0}]+\sum_{j>0}S_j[\phi^{j}_{m}],
      \label{eq:action_phi}
\end{equation}
where 
\begin{equation}
      S_0[\phi^0_0]= \int\dd^3\vec{\chi}\left(\frac{1}{2}\left\{(\nabla \phi^{0}_{0})^2+\left[\mu^2+\frac{\lambda}{3!}(2c_{3,0}^2+c_{2,0}^2)\right](\phi^{0}_{0})^2\right\}-\frac{\xi\lambda}{3!} c_{3,0}\left(\phi^0_0\right)^3 \right),
\end{equation}
and
\begin{align}
      S_j[\phi^{j}_{m}] &=\int\dd^3\vec{\chi}\left\{\frac{1}{2}\left[|\nabla\phi^{j}_{n}|^2+\left(\mu^2+\frac{\lambda}{3!}c_{3,j}^2\right)|\phi^{j}_{n}|^2\right]\right. \nonumber \\
  &\left.-\frac{c_{3,j_1}}{2d_{j}}\frac{\xi\lambda}{3!}\matcurl{ccc}{j&j&j\\j&j&j}\sum_{m_1,m_2,m_3}\phi^{j}_{m_1}\phi^{j}_{m_2}\phi^{j}_{m_3}\mat{ccc}{j&j&j\\m_1&m_2&m_3}\right\}, \label{eq:actionphij}
\end{align}
where $\sum_n|\phi^j_n|^2=\sum_n(-1)^{j-n}\phi^j_n\phi^j_{-n}$. 
The vector fields with different spin labels $j$ decouple and each of them has the form of an AR action with $j$-dependent mass term and coupling. And again, the coefficients $c_{2,j}$ and $c_{3,j}$ can be given explicitly after substituting solutions \eqref{eq:Tfsolspin} and \eqref{eq:PW_eps}.

\vspace{10pt}
\paragraph{Computing coefficients $c_i$ and checking compatibility conditions.\\}

We compute explicitly here the coefficients $c_{1,j}$,$c_{3,j}$ and $c_{2,j}$ for the homogeneous solution~\eqref{eq:Tfsolgroup} to check that these conditions are compatible with our homogeneous solution. Substituting~\eqref{eq:Tfsolgroup} into condition~\eqref{eq:tpsipsipsicon}, we have
\begin{equation}
      \mu\sqrt{\frac{3!}{\lambda}}d_{j_1}f^0_{00}\mat{ccc}{j_1&0&j_1\\-m_1&0&-m_3}=\mu \sqrt{\frac{3!d_{j_1}}{\lambda}}f^0_{00}(-1)^{j_1+m_3}\delta_{m_1,-m_3}=c_{3,j_1}(-1)^{-m_3}\delta_{m_1,-m_3},
\end{equation}
which leads to
\begin{equation}
  c_{3,j}= \begin{cases} (-1)^j \mu \sqrt{\frac{3!d_{j}}{\lambda}}f^0_{00} \hspace{5pt}&\text{if} \hspace{5pt} j\in\mathbb{N} \\ 0 \hspace{5pt}&\text{otherwise}\end{cases}.
\end{equation}

On the other hand, condition~\eqref{eq:tpsitpsicon} yields
\begin{align}
     &\frac{3!\mu^2}{\lambda} \sum_{j_2,m_2}(-1)^{\sum_{i=1}^3(4j_i-m_i)}d_{j_1}d_{j_3}\sum_{n_2,l_2}f^{j_2}_{-m_2,-n_2}f^{j_2}_{m_2,l_2}\mat{ccc}{j_1&j_3&j_2\\m_1&m_3&n_2} \mat{ccc}{j_1&j_3&j_2\\-m_6&m_4&l_2} \nonumber\\
  &=c_{2,j_1}c_{2,j_3}\delta_{m_1,-m_6}\delta_{m_3,m_4},
\end{align}
which leads to the condition for $f^{j_2}_{m_2n_2}$
\begin{equation}
  \sum_{m_2}(-1)^{n_2-m_2}f^{j_2}_{-m_2,-n_2}f^{j_2}_{m_2,l_2}=d_{j_2}c_{f,j_2}^2\delta_{n_2,l_2},
\end{equation}
for some new constants $c_{f,j_2}$. Together with the normalisation condition~\eqref{eq:fnorcon} for $f^j_{mn}$, we get the condition that these new constants should satisfy
\begin{align}
	1&= \sum_{j_2,m_2,n_2,l_2}(-1)^{n_2-m_2}f^{j_2}_{-m_2,-n_2}f^{j_2}_{m_2,l_2} \delta_{n_2,l_2}, \nonumber\\
	&=\sum_{j_2}d^2_{j_2} c_{f,j_2}^2,
\end{align}
and we can get the explicit form \eqref{eq:cfepsilon} of $c_{f,j_2}$ by substituting the heat kernel regularized solution \eqref{eq:PW_eps}.
%with $c_{f,j_2}$ the square of the coefficients of the Peter-Weyl decomposition of~\eqref{eq:Tfsolgroup}.

\vspace{10pt}
\paragraph{The heat kernel regularized solution\\}

The check on the extra conditions performed above on the homogeneous solution~\eqref{eq:PW_eps} still holds at first order in $\epsilon$ when considering the heat kernel regularized solution~\eqref{eq:Tepsilonsolgroupreg}. Using its Peter-Weyl coefficients \eqref{eq:Deltavarepsiloncoef}, we see that
%its Peter-Weyl coefficients are
%\begin{equation}
%      (f_\varepsilon)^j_{mn}=\alpha_\varepsilon \sqrt{d_j} \delta_{mn} \ee^{-\varepsilon C_j}.
%\end{equation}
the constant $c_{3,j}$ is simply
\begin{equation}
      c_{3,j}=(-1)^{j} \mu \sqrt{\frac{3!d_j}{\lambda}}(\Delta_\varepsilon)^0_{00}=(-1)^{j}\mu\sqrt{\frac{3!d_j}{\lambda}}\alpha_\varepsilon.
\end{equation}
And the coefficients $c_{f,j}$ would have the form
\begin{equation}
      c_{f,j}=\alpha_\varepsilon  \ee^{-\varepsilon C_j}. \label{eq:cfepsilon}
\end{equation}

Similarly, the condition~\eqref{eq:tpsitpsicon} is only satisfied approximately at first order in $\varepsilon$. Indeed at first order in $\varepsilon$  the Equation~\eqref{eq:3jsumdelta} gives
\begin{equation}
    \sum_{j,m}d_j\ee^{-2\varepsilon C_j}\mat{ccc}{j_1&j_2&j\\m_1&m_2&m}\mat{ccc}{j_1&j_2&j\\m_1'&m_2'&m}\approx\delta_{m_1'm_1}\delta_{m_2'm_2}. \label{eq:3jsumdeltaappro}
\end{equation}
Hence the coefficients $c_{2,j}$ of the condition~\eqref{eq:tpsitpsicon} can then be determined as 
\begin{equation}
 c_{2,j}=\mu d_j\alpha_\varepsilon\sqrt{\frac{3!}{\lambda}}.
\end{equation}
It follows that the effective action for the heat kernel regularized homogeneous solution is
\begin{align}
       S_0[\phi^0_0]&= \int\dd^3\vec{\chi}\left\{\frac{1}{2}\left[(\nabla \phi^{0}_{0})^2+\mu^2\left(1+3\alpha_\varepsilon^2\right)(\phi^{0}_{0})^2\right]-\frac{\sqrt{\lambda}\xi\mu\alpha_\varepsilon}{\sqrt{3!}} \left(\phi^0_0\right)^3 \right\}, \label{eq:actionphi0epsilon}\\
   S_j[\phi^{j}_{m}]  &=\int\dd^3\vec{\chi}\left\{\frac{1}{2}\left[|\nabla\phi^{j}_{n}|^2+\mu^2\left(1+d_j\alpha_\varepsilon^2 \right)|\phi^{j}_{n}|^2\right]\right. \nonumber \\
  &\left.-\frac{(-1)^j}{\sqrt{3!}}\frac{\sqrt{\lambda}\xi\mu\alpha_\varepsilon}{2\sqrt{d_{j}}}\matcurl{ccc}{j&j&j\\j&j&j}\sum_{m_1,m_2,m_3}\phi^{j}_{m_1}\phi^{j}_{m_2}\phi^{j}_{m_3}\mat{ccc}{j&j&j\\m_1&m_2&m_3}\right\}, \label{eq:actionphijepsilon}
\end{align}
where the second equation is exactly the AR action for spin $j$, with mass and interaction coupling dependent on the fundamental GFT coupling and on the spin index $j$. 

This shows that the AR model can be obtained as a particular two dimensional perturbation around classical solutions of the Boulatov model, provided that the classical solution satisfies the conditions given by Equations~\eqref{eq:tpsitpsicon} and~\eqref{eq:tpsipsipsicon}. This is our main result.

Before analysing the resulting generalized AR model further, let us add a few comments on our result. As recalled earlier, the AR model is a vector model on flat euclidean space. From a quantum gravity point of view, the two key ingredients of the model that one would consider challenging to reproduce from the fundamental quantum dynamics are the background flat space it lives on and its local nature. This is because the fundamental formulation of the theory, here the extended Boulatov model with its simplicial quantum gravity underpinning, does not feature continuum spacetime manifold nor local fields defined on it, so both have to be somehow reconstructed in the continuum limit, and the whole framework is diffeomorphism invariant. In our derivation, these issues are apparently bypassed in few simple steps: the continuum limit is encoded in the mean field treatment of the Boulatov model, effectively resumming an infinite series of perturbative, lattice-dependent amplitudes; the desired flat geometry is provided by the homogeneous background solution we expand around; the local characterization of the GFT field perturbations, interpreted as a local vector field in that flat space, is allowed by the extra frame degrees of freedom, in turn coming from scalar matter in the discrete gravity picture, thus a material reference frame. While this ensures some coherence between the interpretation of all the various formal ingredients in our derivation and its result, it is clear that each of them requires further analysis.  

\section{Melonic dominance}\label{melonicdominance}

As already mentioned above, an important feature of the AR model is the dominance of melonic graphs in the large $N=2j+1$ limit.
%~\cite{Benedetti:2020iku}. 
However, the main difference between the effective action~\eqref{eq:action_phi} and the original AR action
is the presence of the sum over spins $j$. Thus we have to check whether or not this new summation spoils the existence of a melonic regime.  Even though the general behaviour of $\{3nj\}$ symbols as functions of $j$ is an open issue \cite{Haggard:2010,Costantino,Bonzom:2012,Don:2018}, one can qualitatively study the behaviour of the Feynman amplitudes of the model and give additional constraints to ensure the existence of such melonic regime.

\subsection{Feynman amplitudes for the non-regularized solution}

For simplicity, we will drop  below  the heat kernel regularisation and work with the actions given by Equation~\eqref{eq:actionphij}, including the sum over spin labels $j$. As in the AR model, each Feynman diagram $\gamma$ of our new model consists of isoscalar part $I_{\gamma}$ and isospin part $A_{\gamma}$~\cite{Amit:1979ev,Benedetti:2020iku}:
\begin{equation}
    \mathcal{A}_{\gamma}=\sum_{j}c_{\gamma}\left(\frac{\lambda \{6j\}}{3!(2j+1)}\right)^{v}I_{\gamma}A_{\gamma},
\end{equation}
where $c_{\gamma}$ is the combinatorial factor of the diagram. The isoscalar part yields a space integral, so one needs to study the isospin part to find how the Feynman amplitude depends on $N$.\\

The melonic graphs are Fully $2$-Particle Reducible (F2PR) diagrams, i.e. they always admit a $2$-cut which gives another melonic graph with fewer vertices, until the trivial graph is reached. Their contribution writes
\begin{equation}
    \mathcal{A}_{F2PR}\sim\sum_{j}(2j+1)^{1-3n}\{6j\}^{2n}\equiv\bar{\mathcal{A}}_{F2PR},
\end{equation}
with for a graph with $v=2n$ vertices. For a graphs which is not F2PR, the Feynman amplitude can be factorized as a product of $2$-particule irreducible graphs
\begin{equation}
    \mathcal{A}_{NF2PR}\sim\sum_{j}(2j+1)^{-n_0-2n}\prod_{i=1}^k A_{\{3n_i j\}}\{6j\}^{2n}
    \equiv\bar{\mathcal{A}}_{NF2PR},
\end{equation}
where
\begin{equation}
    n=1+n_0-k+\sum_{i=1}^k n_i,
\end{equation}
and $A_{\{3n_i j\}}$ is the amplitude of a three-particle irreducible diagrams with $2n_i$ vertices.

When $N=2j+1$ goes to infinity, the amplitudes $\bar{\mathcal{A}}_{NF2PR}$ is conjectured to obey the following bound~\cite{Amit:1979ev}
\begin{equation}\label{ln}
    \bar{\mathcal{A}}_{NF2PR}\leq \sum_j (2j+1)^{1-3n-\alpha}\{6j\}^{2n},
\end{equation}
for some real number $\alpha>0$. Asymptotically, when $N\rightarrow \infty$, both $n\geq 1$ and the $6j$ symbol are small with respect to $N$. Therefore we get the following bound
\begin{equation}
    \bar{\mathcal{A}}_{F2PR}<\sum_jN^{1-3n}=\sum_j (2j+1)^{1-3n}=\left(1-2^{1-3n}\right)\zeta(3n-1),
\end{equation}
where $\zeta$ is the Riemann zeta function, which is a monotonically decreasing finite function of $n$. 

If one assumes that the bound~\eqref{ln} holds for any value of $N$, then the amplitude of a $NF2PR$ graphs is also finite. If the bound~\eqref{ln} fails to hold for values of $N$ satisfying $N<N_t$ for some bound $N_t$, then the sum from $N=3$ ($j$ can only be an integer no smaller than $1$, so $N\geq 3$) to $N=N_t$ is still a finite number, while the sum from $N=N_t$ is finite as well. Therefore, it is possible that $\mathcal{A}_{NF2PR}$ is comparable with $\mathcal{A}_{F2PR}$ since the maximal value of $\zeta(3n-1)$ is only $\pi^2/6\simeq 1.645$. 

One can thus conclude that the sum over $j$ can dramatically change the amplitude of a Feynman graphs of the AR model and spoil the melonic limit at large $N$. However, one can find ways to rule out this possibility and ensure that the melonic dominance is preserved. This will be illustrated in the following subsection.

\subsection{Restoring the melonic dominance}

One na\"ive way to  restore the melonic dominance is of course to further specialize the form of the perturbation~\eqref{eq:Tpsispin} in order to enforce the selection of one single value for the spin $j$, thus getting rid of the sum over spin labels and leading to the original AR model:
\begin{equation}
      (T_\psi)^{m_1m_2m_3}_{j_1j_2j_3}(\vec{\chi})=T^{m_1m_2m_3}_{j_1j_2j_3}+\delta^{j_1j}\delta^{j_2,0}\delta_{m_2,0}\psi^{j_1}_{m_1,m_3}(\vec{\chi}), 
\end{equation}

Another, more interesting, way to recover melonic dominance is to work with the approximate solution~\eqref{eq:Tepsilonsolgroupreg}. Indeed, when $j_2=0$ the solution has the form:
\begin{equation}
      (T_{\varepsilon})^{m_1m_2m_3}_{j_1j_2j_3}=\mu\alpha_{\varepsilon}\sqrt{\frac{3!}{\lambda}}\ee^{-2\varepsilon C_{j_1}} \sqrt{2j_1+1}(-1)^{j_1-m_1}\delta_{j_1,j_3}\delta_{m_1,-m_3}.
\end{equation}
For $\varepsilon=(2j_{\mathrm{max}}(j_{\mathrm{max}}+1))^{-1}$, the expression
above scales as $\sqrt{2j_1+1}$ for $j_1<j_{\mathrm{max}}$, with $j_{\mathrm{max}}$ a large number. Hence, in the Peter-Weyl expansion, the coefficients with larger $j$ are dominant, and the coefficients $(T_{\varepsilon})^{m_1m_2m_3}_{j_1j_2j_3}$ with $j_i< j_{\mathrm{min}}$ for some threshold $j_{\mathrm{min}}$ can be neglected. We require that $j_{\mathrm{min}}$ is also a large number so that the bound~\eqref{ln} is valid.
%Then the perturbations fields would be a
At first order in $\epsilon$ one then has:
\begin{equation}
      (T_\psi)^{m_1m_2m_3}_{j_1j_2j_3}(\vec{\chi})\simeq
      \left\{\begin{array}{cc}    T^{m_1m_2m_3}_{j_1j_2j_3}+\delta^{j_1j}\delta^{j_2,0}\delta_{m_2,0}\psi^{j_1}_{m_1,m_3}(\vec{\chi}), & j_{\mathrm{min}}\leq j_i\leq j_{\mathrm{max}}\nonumber\\
 0, &\text{otherwise}\end{array}  \right. .
\end{equation}
Such perturbations $\phi^j_m$ will lead to the amplitude
\begin{equation}
      \mathcal{A}_{\gamma}=\sum_{j=j_{\mathrm{min}}}^{j=j_{\mathrm{max}}}c_{\gamma}\left(\frac{\lambda \{6j\}}{3!\sqrt{2j+1}}\right)^{v}I_{\gamma}A_{\gamma},
\end{equation}
which becomes an infinitesimal again for large $j_{\mathrm{min}}$ and $j_{\mathrm{max}}$, while the non-F2PR graphs are higher order infinitesimals as in the original AR model. The melonic dominance is thus restored.

\section{Concluding remarks} \label{sec:summary}

In this work, we have obtained a generalised version of the Amit-Roginski model as a (two-dimensional) perturbation around a classical homogeneous solution of the Boulatov group field theory model for 3d quantum gravity, extended to include (what plays the role, in the discrete gravity picture, of) scalar matter degrees of freedom, which end up providing a material frame and embedding coordinates in the resulting AR model. 
This is an interesting result from a physical point of view, first of all, since it connects 3d quantum gravity, in a well-studied and mathematically rich formulation, and the AR model, itself of great mathematical interest.
%To our knowledge, this is the first result unifying two different type of field theories which admit a melonic limit. 
%The idea of extracting additional scalar degrees of freedom from perturbations of quantum gravity dynamics is not new \cite{Fairbairn:2007sv,Girelli:2009yz,DiMare:2010zp}, but now we realised that the emerged matter can be related to the AR model, which means that a scalar field theory can be extracted from a tensorial group field theory (i.e. the Boulatov model).
%On one hand, the Boulatov model, a tensorial group field theory, and on the other hand the AR model, a scalar field theory. 
The main difference between our effective action for the perturbation and the usual AR model is the presence of the summation on the spin index $j$. While it is still unclear whether this summation could spoil the dominance of melonic diagram in the most general framework, it is possible to preserve this  melonic limit also in this generalised AR model by making use of the heat kernel regularization and taking a double scaling limit. %where $\epsilon \rightarrow 0$ as $N \rightarrow \infty $. 
%While little is known about classical solution to the Boulatov model beyond the framework studied here, we have given sufficient condition on the solutions to the equation of motion to recover an AR-like dynamic for the effective action of the perturbation around the classical solution. 
%Note that the lost of melonic dominance in the general case is an non-perturbative feature of GFT around the non-trivial solution \eqref{eq:Tfsolgroup} \cite{Fairbairn:2007sv}, as perturbatively the amplitudes of GFT is dominant by melonic graphs around the Gaussian fixed point \cite{Gurau:2010ba}, even include the low spin mode. 
%For the non-trivial solution however, the melonic dominance can only be restored if we only focus on high modes, which is also justified as their coefficients is large compared to the low spin modes in the Peter-Weyl decomposition.

It is also an interesting result from a more conceptual point of view, since it shows an example of the emergence of a local field theory (in flat space) from a background independent quantum gravity formalism based on non-spatiotemporal structures (meaning, not corresponding directly to quantized continuum spacetime-based fields), which is an outstanding challenge for most quantum gravity approaches. In fact, our result resonates (especially in the key role played by scalar matter used as a relational frame) with recent work on GFT cosmology \cite{Marchetti:2021gcv}.

\medskip

This result opens the way for at least two different generalisations. Firstly, a natural follow-up would be to find other classical solutions to the Boulatov model and to study perturbations around these solutions to see if they also admit Amit-Roginski-like perturbations. 

Secondly, as already mentioned in the Introduction, the holographic SYK model is another type of field theory that is known to enjoy a melonic limit. 
It thus appears interesting to us to investigate how the SYK model as well can be obtained within a GFT setup, again in terms of fluctuations over non-perturbative quantum gravity configurations.

Once more, while exploring generalizations of our results, many elements in our derivation deserve a deeper and more extensive analysis. Among these, we mention again: the quantum geometric interpretation and effects of the sum over topologies in the GFT construction; the continuum physical interpretation of the classical solutions of GFT equations of motion; the renormalization group flow and continuum limit of the extended TGFT models, with both local and tensorial directions, that were the starting point of our analysis.  
%However, note that due to the difference in nature of the coupling between the two theory, i.e. a group theoretic related coupling constant with the $3j$ symbol of the AR model and a Gaussian random tensor for the SYK model, obtaining the SYK model would a priori require a different mechanism than a perturbation around a classical solution as studied in this article~\footnote{Mention Johannes proejct woth FRG here ? (V)}. Finally, since $0$-dimensional tensorial field theories can be seen as a TGFT stripped from its group-theoretic aspects, this would show that all melonic theory known to this day can be obtained as a particular regime of TGFT.





\acknowledgements

The authors acknowledge financial support from the Bordeaux-LMU collaboration grant.
DO acknowledges financial support from the Deutsche Forschung Gemein-schaft (DFG). XP is supported by China Scholarship Council. 
%VN and AT are partially supported by the ANR ....
VN and AT have been partially supported by the ANR-
20-CE48-0018 “3DMaps” grant. 
AT is partially supported by the PN 23 21 01 01/2023 grant.
 DO, VN and AT acknowledge support of the Institut Henri Poincaré (UAR 839 CNRS-Sorbonne Université), and LabEx CARMIN (ANR-10-LABX-59-01)
%DO, VN and AT are grateful to IHP Paris for hospitality 
during the 2023 IHP trimester "Quantum gravity, random geometry and holography".

\appendix

\section{Definitions and identities from $SU(2)$ recoupling theory} \label{sec:su2recoupling}

We give several definitions and properties related to $SU(2)$ recoupling theory used in the article. All those properties are classical results on recoupling theory of $SU(2)$, and we refer the interested reader to Ilkka M\"akinen's introduction~\cite{Makinen:2019rou} as well as Pierre Martin-Dussaud's lively note \cite{Martin-Dussaud:2019ypf} on this topic for more details. 

\subsection{Haar measure and Wigner matrices}

From the Peter-Weyl theorem, the Wigner matrices $D^j_{mn}(g)$ form an orthogonal basis of the functions $f:SU(2) \rightarrow \mathbb{C}$. This orthogonality relation is encoded in the Haar measure via the relation
\begin{equation}
      \int\dd g D^j_{mn}(g)\bar{D}^{j'}_{m'n'}(g) =\frac{1}{(2j+1)}\delta^{jj'}\delta_{mm'}\delta_{nn'},
\end{equation}
where the Wigner matrices satisfy
\begin{equation}
      D^j_{mn}(g)=(-1)^{m-n}\bar{D}^j_{-m,-n}(g).
\end{equation}

\subsection{$3j$-symbol and its properties}

The $3j$ symbol is invariant under the action of $SU(2)$ group, 
\begin{equation}
D^{j_1}_{m_1n_1}D^{j_2}_{m_2n_2}D^{j_3}_{m_3n_3}\mat{ccc}{j_1&j_2&j_3\\n_1&n_2&n_3}=\mat{ccc}{j_1&j_2&j_3\\m_1&m_2&m_3}.
\end{equation}

It's also invariant under the even permutations of indices, while it acquires an additional phase under odd permutations
\begin{equation}
       \mat{ccc}{j_1&j_2&j_3\\ m_1&m_2&m_3}=(-1)^{j_1+j_2+j_3}\mat{ccc}{j_2&j_1&j_3\\ m_2&m_1&m_2}. \label{eq:3jpermu}
\end{equation}

The same phase also appear if we replace $m_i$ by their negative
\begin{equation}
       \mat{ccc}{j_1&j_2&j_3\\ -m_1&-m_2&-m_3}=(-1)^{j_1+j_2+j_3}\mat{ccc}{j_1&j_2&j_3\\ m_1&m_2&m_3}. 
\end{equation}

The $3j$ symbols satisfy two orthonormal relations
\begin{align}
     &(2j_3+1)\sum_{m_1,m_2}\mat{ccc}{j_1&j_2&j_3\\m_1&m_2&m_3}\mat{ccc}{j_1&j_2&j_3'\\m_1&m_2&m_3'}=\delta_{j_3,j_3'}\delta_{m_3,m_3'}, \\
  &\sum_{j_3,m_3}(2j_3+1)\mat{ccc}{j_1&j_2&j_3\\m_1&m_2&m_3}\mat{ccc}{j_1&j_2&j_3\\m_1'&m_2'&m_3}=\delta_{m_1,m_2'}\delta_{m_2,m_2'},  \label{eq:3jsumdelta}
\end{align}


Finally, when one of the magnetic moment (say $m_3$) vanishes, then the $3j$ symbol vanishes unless $m_1 = -m_2$ and we have
\begin{equation}
       \sum_m (-1)^{j-m}\mat{ccc}{j&j&k\\m&-m&0}=\sqrt{2j+1}\delta_{k,0}. \label{eq:3jm3eq0}
\end{equation}

And in particular for $k=0$ we have
\begin{equation}
      \mat{ccc}{j_1&0&j_3\\n_1&0&n_3}=\delta^{j_1,j_3}\frac{1}{\sqrt{2j_1+1}}(-1)^{j_1+n_1}\delta_{n_1,-n_3} \label{eq:3jj2eq0} \\
\end{equation}

\subsection{$6j$-symbol and its properties}

The $6j$ symbol is defined as
\begin{align}
      \left\{\begin{array}{ccc}j_1&j_2&j_3\\j_4&j_5&j_6\end{array}\right\}
  =&\sum_{j_i,m_i} (-1)^{\sum_{a=1}^6(j_a-m_a)} \mat{ccc}{j_1&j_2&j_3\\-m_1&-m_2&-m_3}\mat{ccc}{j_1&j_5&j_6\\m_1&-m_5&m_6} \nonumber \\
  &\cdot \mat{ccc}{j_4&j_2&j_6\\m_4&m_2&-m_6}\mat{ccc}{j_4&j_5&j_3\\-m_4&m_5&m_3}.   \label{eq:6jdef}
\end{align}
It enjoys several symmetries properties that we do not make use of in the main body. We refer the interested reader to~\cite{Makinen:2019rou} where they are explicitly mentioned.

Using the $6j$ symbol we have
\begin{align}
       &\sum_{n_1,n_2,n_3}(-1)^{\sum_{a=1}^{3}(k_a-n_a)} \mat{ccc}{j_1&k_2&k_3\\m_1&-n_2&n_3}\mat{ccc}{k_1&j_2&k_3\\n_1&m_2&-n_3}\mat{ccc}{k_1&k_2&j_3\\-n_1&n_2&m_3} \nonumber \\
   &=\left\{\begin{array}{ccc}j_1&j_2&j_3\\k_1&k_2&k_3\end{array}\right\} \mat{ccc}{j_1&j_2&j_3\\ m_1&m_2&m_3}. \label{eq:3jsum6j} 
\end{align}

Finally when one of the spin index (say $j_6$) vanishes we have
\begin{equation}
      \left\{\begin{array}{ccc}j_1&j_2&j_3\\j_4&j_5&0\end{array}\right\}=\frac{\delta_{j_1,j_5}\delta_{j_2,j_4}}{\sqrt{d_{j_1}d_{j_2}}}(-1)^{j_1+j_2+j_3}\{j_1~j_2~j_3\}. \label{eq:6jj6eq0}
\end{equation}

 % \bibliographystyle{JHEP}
 % \bibliography{./bib_library/library}
% %Version 2.1 April 2023
% See section 11 of the User Manual for version history
%
%%%%%%%%%%%%%%%%%%%%%%%%%%%%%%%%%%%%%%%%%%%%%%%%%%%%%%%%%%%%%%%%%%%%%%
%%                                                                 %%
%% Please do not use \input{...} to include other tex files.       %%
%% Submit your LaTeX manuscript as one .tex document.              %%
%%                                                                 %%
%% All additional figures and files should be attached             %%
%% separately and not embedded in the \TeX\ document itself.       %%
%%                                                                 %%
%%%%%%%%%%%%%%%%%%%%%%%%%%%%%%%%%%%%%%%%%%%%%%%%%%%%%%%%%%%%%%%%%%%%%

%%\documentclass[referee,sn-basic]{sn-jnl}% referee option is meant for double line spacing

%%=======================================================%%
%% to print line numbers in the margin use lineno option %%
%%=======================================================%%

%%\documentclass[lineno,sn-basic]{sn-jnl}% Basic Springer Nature Reference Style/Chemistry Reference Style

%%======================================================%%
%% to compile with pdflatex/xelatex use pdflatex option %%
%%======================================================%%

%%\documentclass[pdflatex,sn-basic]{sn-jnl}% Basic Springer Nature Reference Style/Chemistry Reference Style


%%Note: the following reference styles support Namedate and Numbered referencing. By default the style follows the most common style. To switch between the options you can add or remove �Numbered� in the optional parenthesis. 
%%The option is available for: sn-basic.bst, sn-vancouver.bst, sn-chicago.bst, sn-mathphys.bst. %  
 
\documentclass[sn-mathphys,Numbered,iicol]{sn-jnl}% Style for submissions to Nature Portfolio journals
%%\documentclass[sn-basic]{sn-jnl}% Basic Springer Nature Reference Style/Chemistry Reference Style
% \documentclass[sn-mathphys,Numbered]{sn-jnl}% Math and Physical Sciences Reference Style
%%\documentclass[sn-aps]{sn-jnl}% American Physical Society (APS) Reference Style
%%\documentclass[sn-vancouver,Numbered]{sn-jnl}% Vancouver Reference Style
%%\documentclass[sn-apa]{sn-jnl}% APA Reference Style 
%%\documentclass[sn-chicago]{sn-jnl}% Chicago-based Humanities Reference Style
%%\documentclass[default]{sn-jnl}% Default
%%\documentclass[default,iicol]{sn-jnl}% Default with double column layout

%%%% Standard Packages
%%<additional latex packages if required can be included here>

\usepackage{graphicx}%
\usepackage{multirow}%
\usepackage{amsmath,amssymb,amsfonts}%
\usepackage{amsthm}%
\usepackage{mathrsfs}%
\usepackage[title]{appendix}%
\usepackage{xcolor}%
\usepackage{textcomp}%
\usepackage{manyfoot}%
\usepackage{booktabs}%
\usepackage{algorithm}%
\usepackage{algorithmicx}%
\usepackage{algpseudocode}%
\usepackage{listings}%
\usepackage{subfigure}
\usepackage{bm}
\usepackage{multicol}

%%%%

%%%%%=============================================================================%%%%
%%%%  Remarks: This template is provided to aid authors with the preparation
%%%%  of original research articles intended for submission to journals published 
%%%%  by Springer Nature. The guidance has been prepared in partnership with 
%%%%  production teams to conform to Springer Nature technical requirements. 
%%%%  Editorial and presentation requirements differ among journal portfolios and 
%%%%  research disciplines. You may find sections in this template are irrelevant 
%%%%  to your work and are empowered to omit any such section if allowed by the 
%%%%  journal you intend to submit to. The submission guidelines and policies 
%%%%  of the journal take precedence. A detailed User Manual is available in the 
%%%%  template package for technical guidance.
%%%%%=============================================================================%%%%

%\jyear{2021}%

%% as per the requirement new theorem styles can be included as shown below
\theoremstyle{thmstyleone}%
\newtheorem{theorem}{Theorem}%  meant for continuous numbers
%%\newtheorem{theorem}{Theorem}[section]% meant for sectionwise numbers
%% optional argument [theorem] produces theorem numbering sequence instead of independent numbers for Proposition
\newtheorem{proposition}[theorem]{Proposition}% 
%%\newtheorem{proposition}{Proposition}% to get separate numbers for theorem and proposition etc.

\theoremstyle{thmstyletwo}%
\newtheorem{example}{Example}%
\newtheorem{remark}{Remark}%

\theoremstyle{thmstylethree}%
\newtheorem{definition}{Definition}%

\raggedbottom
%%\unnumbered% uncomment this for unnumbered level heads

\begin{document}

\title[Article Title]{Non-Markovian Quantum Gate Set Tomography}

%%=============================================================%%
%% Prefix	-> \pfx{Dr}
%% GivenName	-> \fnm{Joergen W.}
%% Particle	-> \spfx{van der} -> surname prefix
%% FamilyName	-> \sur{Ploeg}
%% Suffix	-> \sfx{IV}
%% NatureName	-> \tanm{Poet Laureate} -> Title after name
%% Degrees	-> \dgr{MSc, PhD}
%% \author*[1,2]{\pfx{Dr} \fnm{Joergen W.} \spfx{van der} \sur{Ploeg} \sfx{IV} \tanm{Poet Laureate} 
%%                 \dgr{MSc, PhD}}\email{iauthor@gmail.com}
%%=============================================================%%

\author[1,3,4]{\fnm{Ze-Tong} \sur{Li}}

\author[1,3,4]{\fnm{Cong-Cong} \sur{Zheng}}

\author[5]{\fnm{Fan-Xu} \sur{Meng}}

\author[2,3,4,6]{\fnm{Zai-Chen} \sur{Zhang}}

\author*[1,3,4,6]{\fnm{Xu-Tao} \sur{Yu}}\email{yuxutao@seu.edu.cn}

% \equalcont{These authors contributed equally to this work.}
\affil[1]{\orgdiv{State Key Laboratory of Millimeter Waves}, \orgname{Southeast University}, \orgaddress{\city{Nanjing}, \postcode{210096}, \country{China}}}

\affil[2]{\orgdiv{National Mobile Communications Research Laboratory}, \orgname{Southeast University}, \orgaddress{\city{Nanjing}, \postcode{210096}, \country{China}}}

\affil[3]{\orgdiv{Frontiers Science Center for Mobile Information Communication and Security}, \orgname{Southeast University}, \orgaddress{\city{Nanjing}, \postcode{210096}, \country{China}}}

\affil[4]{\orgdiv{Quantum Information Center}, \orgname{Southeast University}, \orgaddress{\city{Nanjing}, \postcode{210096}, \country{China}}}

\affil[5]{\orgdiv{College of Artificial Intelligence}, \orgname{Nanjing Tech University,}, \orgaddress{\city{Nanjing}, \postcode{211800}, \country{China}}}

\affil[6]{\orgname{Purple Mountain Lab}, \orgaddress{\city{Nanjing}, \postcode{211111}, \country{China}}}


%%==================================%%
%% sample for unstructured abstract %%
%%==================================%%

\abstract{Engineering quantum devices requires reliable characterization of the quantum system including qubits, quantum operations (aka instruments) and the quantum noise. Recently, quantum gate set tomography (GST) has emerged as a promissing technique to self-consistently describe the quantum states, gates and measurements. However, non-Markovian correlations between the quantum system and environment cause the reliability regression of GST. It is essential to simultaneously describe the gate set and non-Markovian correlations. To this end, we first propose a self-consistent operational method, named instrument set tomography (IST), for non-Markovian GST. Based on the stochastic quantum process, the instrument set is defined to describe instruments, the initial state, and non-Markovian system-environment (SE) correlations. First, we propose a linear inversion IST (LIST) to detect and describe the disharmony of linear relationship of instruments and SE correlations with gauge freedom. 
However, LIST cannot always determine physical implementable instrument set because of the absence of constraints. Then, a physically constrained statistical method based on the miximum likelihood estimation for IST (MLE-IST) is proposed with polynomial number of parameters with respect to the Markovian order. It shows significant flexibility that suit for different types of device, e.g. noisy intermediate-scale quantum (NISQ) devices, by adjusting the model and constraints. The experimental results show the effectiveness of describing instruments and the non-Markovian quantum system. As a result, the IST provides an essential method for benchmarking and developing quantum devices in the aspect of instrument set.}

%%================================%%
%% Sample for structured abstract %%
%%================================%%

% \abstract{\textbf{Purpose:} The abstract serves both as a general introduction to the topic and as a brief, non-technical summary of the main results and their implications. The abstract must not include subheadings (unless expressly permitted in the journal's Instructions to Authors), equations or citations. As a guide the abstract should not exceed 200 words. Most journals do not set a hard limit however authors are advised to check the author instructions for the journal they are submitting to.
% 
% \textbf{Methods:} The abstract serves both as a general introduction to the topic and as a brief, non-technical summary of the main results and their implications. The abstract must not include subheadings (unless expressly permitted in the journal's Instructions to Authors), equations or citations. As a guide the abstract should not exceed 200 words. Most journals do not set a hard limit however authors are advised to check the author instructions for the journal they are submitting to.
% 
% \textbf{Results:} The abstract serves both as a general introduction to the topic and as a brief, non-technical summary of the main results and their implications. The abstract must not include subheadings (unless expressly permitted in the journal's Instructions to Authors), equations or citations. As a guide the abstract should not exceed 200 words. Most journals do not set a hard limit however authors are advised to check the author instructions for the journal they are submitting to.
% 
% \textbf{Conclusion:} The abstract serves both as a general introduction to the topic and as a brief, non-technical summary of the main results and their implications. The abstract must not include subheadings (unless expressly permitted in the journal's Instructions to Authors), equations or citations. As a guide the abstract should not exceed 200 words. Most journals do not set a hard limit however authors are advised to check the author instructions for the journal they are submitting to.}

\keywords{non-Markovian correlation, gate set tomography, quantum tomography}

%%\pacs[JEL Classification]{D8, H51}

%%\pacs[MSC Classification]{35A01, 65L10, 65L12, 65L20, 65L70}

\maketitle

\section{Introduction}
Quantum computing requires engineering reliable and controllable quantum devices that manipulate the quantum states with high fidelity. However, recent quantum devices suffer the non-ignorable quantum noise introduced by the imperfect implementations of quantum gates and the system-environment (SE) correlations \cite{papic2023Error}. Characterization of qubits, operations, and entire processors to analyse the influence of quantum noise plays a significant role in the quantum characterization, verification, and validation (QCVV) and offers basic information for the device manufacturing and calibration. 

Based on different assumptions, many protocols have been proposed for this task under the common skeleton of quantum tomography \cite{banaszek2013Focus,smolin2012Efficient,blume-kohout2010Optimal,koutny2022Neuralnetwork,riebe2006Process,mohseni2008Quantumprocess,surawy-stepney2022Projected,greenbaum2015Introduction,nielsen2021Gate}: (1) prepare a set of experiments described by quantum states, circuits and measurements; (2) gather data by executing the prepared experiments; (3) yeild the target result of quantum states, processes and/or measurements by performing estimation algorithms. Among these tomographic methods, gate set tomography (GST) \cite{greenbaum2015Introduction,nielsen2021Gate} is the most powerful and comprehensive method to operationally and self-consistently characterize quantum gates, 
state preparations and measurements (SPAM) without assuming any component of the experiments to be known previously, while the quantum state tomography (QST) \cite{banaszek2013Focus,smolin2012Efficient,blume-kohout2010Optimal,koutny2022Neuralnetwork} and quantum process tomography (QPT) \cite{riebe2006Process,mohseni2008Quantumprocess,surawy-stepney2022Projected} generally require the full knowledge of not-target parts in the experiments. The GST successfully describes two-time noisy quantum gates by completely positive trace-preserving (CPTP) maps under the Markovian assumption. However, no system is isolated \cite{pollock2018NonMarkovian}. There is sufficient evidence that the non-Markovian multiple time correlation nonnegligibly impacts current generation quantum devices \cite{blume-kohout2017Demonstration,proctor2022Measuring,white2020Demonstration,sarovar2020Detecting}. It not only disturbs the tomography under Markovian model that operations in the past influence the behavior of current operation and result in the theoretical violation of CPTP constraints \cite{proctor2022Measuring,milz2021Quantum}. Moreover, the effectiveness of quantum error-correcting codes can degrade or vanish with the appearance of the non-Markovian correlation \cite{nickerson2019Analysing,clader2021Impact}. Therefore, Markovian two-time CPTP maps are not sufficient to describe entire dynamics of the quantum device. Correlations across multiple time scales should be considered while characterizing the device. 

Based on the quantum stochastic process \cite{milz2021Quantum} representing the multiple time correlation, the non-Markovian system dynamics can be modeled by the system, environment, instruments act on the system, and unitaries act on the system and environment simultaneously. For an experimenter, the only accessible part is the instruments representing interventions on the system including quantum gates and measurements. Hence, an instrument can be represented by completely positive trace-non-increasing (CPTNI) maps. Aiming at operationally describing the time-dependent SE correlations, process tensor tomography (PTT) \cite{pollock2018NonMarkovian,guo2022Reconstructing,milz2018Reconstructing,white2022NonMarkovian} relaxes the Markovian constraint to perform the non-Markovian quantum process tomography. It construct well defined CP process tensor with unit trace by interventions of known instruments. However, the differences between the knowledge and the practical performance of instruments may disturb the reconstruction of the process tensor \cite{white2022NonMarkovian}. A simple example is that PTT may generate inconsistent two process tensors using two set of faulty state-informationally complete instruments (that are sufficient to span the space of quantum state). Consequently, the characterization of real quantum devices requires a self-consistent method to tomographically describe the non-Markovian SE correlation and faulty instruments, which directly motivate this work.

To tackle these issues, we first propose a self-consistent method to perform GST under non-Markovian situation. We call the method instrument set tomography (IST). We first propose the linear inversion IST (LIST) a simple, closed-form algorithm to estimate the instruments as well as the SE correlations represented by the process tensor. Unsurprisingly, the IST still exhibits the gauge freedom as GST. Hence the gauge optimization is required at the end of LIST. Although the estimated result may not satisfy the physical constraints since we introduce no constraint in the gauge optimization, it is consistent to the probability measurement data.
Then, we propose a statistical IST method based on the maximum likelihood estimation (MLE) trying to extract more information from overcomplete measurement data. The MLE-IST models the instruments and SE correlations via a flexible way that can suite for different assumption. By introduce constraints, the results are guaranteed to be physical. It also enables the explicit estimation of unitaries representing the non-Markovian SE correlation and evolution instead of the process tensor. Particularly, we also demonstrate how to implement IST on the current noisy quantum intermediate-scale quantum (NISQ) devices. The experimental results show the effectiveness of characterization of instruments, initial states, and non-Markovian correlations. As a result, the IST provides an essential, self-consistent, and reliable method for benchmarking and developing a quantum device under non-Markovian situation in the aspect of instrument set.

\section{Result}

\subsection{Quantum Stochastic Process and Instrument Set}
Before moveing on to present the IST, we first recall the quantum stochastic process representing the non-Markovian quantum correlation and give definitions for instrument set. For a $d$-dimensional quantum system with non-Markovian correlations, the experimenter intervenes the quantum system at $k$ time steps by CPTNI instruments from 
\begin{align}
  \mathcal{J}^{(t)} := \left\{\mathcal{A}^{(t)}_{0}, \mathcal{A}^{(t)}_{1},\dots, \mathcal{A}^{(t)}_{m_t-1}\right\},
\end{align}
where $t=1,\dots,k$ and $m_t$ is the number of valid instruments at time step $t$. Each intervention of the instrument output a value and transform the quantum state for the next time step. The available instruments at different time steps may be different. Then, the operational open quantum process can be described by a $d$-dimensional system and a $d$-dimensional environment with interventions of instruments on the system at $k$ time steps and SE unitaries evolutions between time steps as depicted in Fig.~\ref{fig:qsp} \cite{pollock2018NonMarkovian}. Note that there is a boundary between the accessible and inaccessible parts of the open quantum dynamics to an experimenter. Specifically, an experimenter can not access the quantum state directly. All information of the quantum state the experimenter obtained should with the help of output values of interventions of instruments. The probability to get a sequence of output values $\bm{x}$ is
\begin{equation}\label{eq:se_evo_prob_tr}
  p_{\bm{x}}=\mathrm{Tr}\left[\mathcal{A}^{(k)}_{x_{k-1}}\bigcirc_{t=0}^{k-2}\left(\mathcal{U}_{t:t+1}\mathcal{A}^{(t)}_{x_t}\right)\left(\rho_{SE}^{(0)}\right)\right],
\end{equation}
where $x_t$ is the output value at time step $t$. Without loss of generality, we refer the output value $x_t$ to be the indexes of intruments instead of the actual output value in the following text. Besides, we use the note $\mathcal{A}^{(t)}_{x_t}$ instead of $\mathcal{A}^{(t)}_{x_t}\otimes \mathcal{I}$ for simplicity without confusing. 

From Eq.~\eqref{eq:se_evo_prob_tr}, it is quite clear that the probability can be determined when the instruments, the SE unitary dynamics, and the initial state are given. Therefore, the instrument set describing the operational open quantum dynamics of the quantum device can be defined as
\begin{align}\label{eq:instrument_set_full}
  \mathfrak{I}_{\mathrm{full}} := \left\{\mathcal{J}, \mathcal{U},\rho^{(0)}_{SE}\right\},
\end{align}
where $ \mathcal{J}:=\left\{\mathcal{J}^{(t)}\right\}_{t=0}^{k-1}$ and $\mathcal{U}:=\left\{\mathcal{U}_{t:t+1}\right\}_{t=0}^{k-2}$.
This full definition explicitly depends on the inaccessible initial state and the SE unitary evolution between time steps in which the experimenter may be insterested. However, explicitly characterization of the initial state and the SE unitarie is difficult.

Benifit from the process tensor $\mathcal{T}$ representing the inaccessible parts \cite{pollock2018NonMarkovian,milz2021Quantum,white2022NonMarkovian}, the probability to get $\bm{x}$ can be described as 
\begin{align}\label{eq:pt_def}
  p_{\bm{x}} = \mathcal{T}\left(\mathcal{A}^{(0)}_{x_0},\dots,\mathcal{A}^{(k-1)}_{x_{k-1}}\right),
\end{align}
implying the sufficiency to determine the measurement probability by given $\mathcal{T}$ and $\mathcal{J}$. Therefore, the reduced instrument set can be defined as 
\begin{align}\label{eq:instrument_set_reduced}
  \mathfrak{I}_{\mathrm{reduced}}:=\left\{\mathcal{J}, \mathcal{T}\right\}.
\end{align}

These two definitions of instrument set will be used to propose the IST with clear declaration. In the following, we always use the pauli transfer matrix (PTM) representation to describe instruments, quantum states and process tensors. Particularly, notations $A$ and $\vert \rho\rangle\!\rangle$ are used to indicate the instrument $\mathcal{A}$ and the quantum state $\rho$. Moreover, the PTM representation of the process tensor $\Upsilon_{\mathcal{T}}$ is defined as
\begin{align}\label{eq:se_evo_prob_pt}
  p_{\bm{x}} = \mathrm{Tr}\left[\Upsilon_{\mathcal{T}}^\dagger\left(\begin{matrix}A^{(0)}_{x_0}\\\vdots\\A^{(k-1)}_{x_{k-1}}\end{matrix}\right)\right],
\end{align}
where terms in parentheses are defined as
\begin{align}
  \left(\begin{matrix}X_1\\ \vdots \\X_n\end{matrix}\right) = \left(\begin{matrix}X_1, \dots, X_n\end{matrix}\right):=X_1\otimes\dots\otimes X_n
\end{align}
for clearness and simplicity, instead of directly applying the Choi-Jamiołkowski isomorphism (CJI) representation for the notation consistency. It is easy to verify that the PTM and CJI representation of process tensor is equivalent.

% $\chi^{(t)}_{x_t}$ and $\Upsilon_{\mathcal{T}}$ are the Choi-Jamiołkowski isomorphism (CJI) of $\mathcal{A}^{(t)}_{x_t}$ and $\mathcal{T}$ respectively. Therefore, the reduced instrument set can be defined as 
% \begin{align}\label{eq:instrument_set_reduced}
%   \mathfrak{I}_{\mathrm{reduced}}=\left\{\mathcal{J}, \mathcal{T}\right\}.
% \end{align}
% The reduced instrument set is sufficient to describe the non-Markovian quantum dynamics with the interventions of instruments as the full instrument set, since the process tensor are sufficient to describe the initial state and SE unitary evolution that 
% \begin{align}\label{eq:pt_u_convert}
%   \Upsilon_{\mathcal{T}} = \mathrm{Tr}_E \left[\mu_{k-2:k-1}\star \dots \star \mu_{0:1} \star \rho_{SE}^{(0)}\right],
% \end{align}
% where $\mu_{t:t+1}$ is the CJI of $\mathcal{U}_{t:t+1}$ and $\star$ is the link product defined in \cite{chiribella2009Theoretical}. These two definitions of instrument set will be used to propose the IST with clear declaration. 

% Figure environment removed

\subsection{Linear Inversion IST}\label{sec:LIST}
We first propose the linear inversion IST (LIST) based on the reduced instrument set as defined in Eq.~\eqref{eq:instrument_set_reduced}. Focusing on the time step $t$, the measurement probability can be described as
\begin{equation}\label{eq:inst_decomp_tr_basis}
  p^{(t)}_{\alpha,{x_t}} = \mathrm{Tr}\left[B_\alpha^{(t)} A^{(t)}_{x_t}\right],
\end{equation}
where $A^{(t)}_{x_t}$ is the PTM a $d^2\times d^2$ matrix that completely represent the instrument $\mathcal{A}^{(t)}_{x_t}$, $B_\alpha^{(t)}$ is a $d^2\times d^2$ real basis matrix indexed by $\alpha$. Let $\bm{x}^+$ and $\bm{x}^-$ denote the output values before and after time step $t$ in a $k$-time step non-Markovian experiment, respectively. The LIST constructs a bijection $\alpha = f(\bm{x}^+, \bm{x}^-)$ between integer $\alpha$ and the concatenation of vectors $(\bm{x}^+,\bm{x}^-)$ by the adjustment of $\bm{x}^+$ and $\bm{x}^-$ such that $\mathbb{B}^{(t)} = \{B_0^{(t)}, B_1^{(t)},\dots, B_{d^4-1}^{(t)}\}$ is a linear independent basis set. See Method for detail. 

This implies the decomposition of $A^{(t)}_{x_t}$ on the non-orthogonal process-informationally complete basis $\mathbb{B}^{(t)}$, 
\begin{equation}\label{eq:inst_decomp_vec}
  \bm{p}_{x_t}^{(t)} = \begin{bmatrix}
    (\bm{b}_0^{(t)})^\dagger\\
    (\bm{b}_1^{(t)})^\dagger\\
    \vdots\\
    (\bm{b}_{d^4-1}^{(t)})^\dagger\\ 
  \end{bmatrix}\bm{a}^{(t)}_{x_t} = B^{(t)}\bm{a}^{(t)}_{x_t},
\end{equation}
where $\bm{a}^{(t)}_{x_t}$ and $\bm{b}_\alpha^{(t)}$ represent the vectorization of the $A^{(t)}_{x_t}$ and $B_\alpha^{(t)}$, respectively. Note that instruments at time step $t$ share the same $B^{(t)}$. If $B^{(t)}$ is invertible, we can get instruments
\begin{gather}\label{eq:list_recover}
    \Xi^{(t)} =\left(B^{(t)}\right)^{-1} \Gamma^{(t)},
\end{gather}
where $\Xi^{(t)} = [\bm{a}^{(t)}_{0},\bm{a}^{(t)}_{1}, \dots, \bm{a}^{(t)}_{m_t-1}]$ and $\Gamma^{(t)} = [\bm{p}_{0}^{(t)},\bm{p}_{1}^{(t)},\dots,\bm{p}_{m_t-1}^{(t)}]$. PTMs of instruments can be recovered by devectorization of determined $\bm{a}^{(t)}_{x_t}$. 

The instruments are reconstructed by repeating this for each time step. Then, we choose the maximum linear independent set of the instruments at each time step to formulate the process tensor
\begin{align}
  \Upsilon_\mathcal{T} =\sum_{\bm{x}}p_{\bm{x}}\left(\begin{matrix}D^{(0)}_{x_0}\\\vdots\\D^{(k-1)}_{x_{k-1}}\end{matrix}\right),
\end{align}
where $\left\{D^{(t)}_{x_{t}}\right\}$ is the dual set of maximum linear independent set $\left\{A^{(t)}_{x_t}\right\}$ such that $\mathrm{Tr}\left[\left(D^{(t)}_{i}\right)^\dagger A^{(t)}_{j}\right] = \delta_{ij}$. 

The tomography of the instrument set shows gauge freedom up to a set of invertible matrices $\{B^{(t)}\}$ because of the inaccessible initial state and SE unitaries. We can not distinguish the quantum operations up to $\{B^{(t)}\}$ by the probability measurement, because we can obtain a set of instruments and process tensor without violations of measurement probabilities $p_{\bm{x}}$ for each given set of gauge matrices $\{B^{(t)}\}$. See Method for detail.

The gauge optimization is required to provide a reasonable gauge matrices set to determine the tomographic result of instrument set. We assume that the quantum instruments are implemented well that are close to the ideal instruments. Then, the gauge matrix can be optimized by
\begin{equation}\label{eq:gauge_opt_obj_fn}
  B^{(t)} = \arg\min_X \sum_{t} \left\|X\Gamma^{(t)} - \Xi^{(t)}_{\mathrm{knowledge}}\right\|_F,
\end{equation}
where $\Xi^{(t)}_{\mathrm{knowledge}}$ is the knowledge of instruments to the experimenter. Consequently, the tomographic result is
\begin{gather}
  \hat{\mathfrak{I}}=\left\{\hat{\mathcal{J}}, \hat{\mathcal{T}}\right\},\label{eq:result_list}\\
  \hat{\mathcal{J}}=\left\{\left\{A^{(t)}_{0},\dots,A^{(t)}_{m_t-1}\right\}\right\}_{t=0}^{k-1},\\
  \hat{\mathcal{T}}=\Upsilon_\mathcal{T}.
\end{gather}

A few points are worth mentioning. First, it can be seen from Eq.\eqref{eq:list_recover} and Eq.\eqref{eq:gauge_opt_obj_fn} that the tomographic result of instruments is always the prior knowledge at the time step the instruments are linear independent and not overcomplete. In this case, the LIST degrades to the linear inversion PTT and all imperfect implementation of instruments are represented by the process tensor. The LIST shows the power detecting the disharmony of linear relationship when the instruments at a time step are not linear independent. Moreover, the linear inverse method actually determines a self-consistent tomographic result of instruments and the process tensor, but they may not physically implementable. These characteristics result from the absence of constraints in the gauge optimization. We actually can constrain each $B_\alpha^{(t)}$ and/or $A_{x_t}^{(t)}$ to be CPTNI\footnote{The constraints to the instruments are corresponding to the completely positive trace non-increasing assumption of instruments. This can be adjusted along with the instruments' assumptions (CPTP at intermediate time steps on NISQ devices, for example).}. Nevertheless, this will increase computational complexity. Instead, we optimize $B^{(t)}$ over the entire group of real, invertible matrices to strenuously fit the data. This is similar to the linear inverse GST (LGST) \cite{greenbaum2015Introduction} under Markovian situation.

Second, the objective function in Eq.\eqref{eq:gauge_opt_obj_fn} is not convex and may has nonunique global minima especially the instruments at time step $t$ are not process-informationally complete. This indeterminacy is generic in quantum tomography, and appears in QST, QPT and GST as well. Therefore, we adopt the reduced definition of instrument set using the process tensor to avoid explicit introducing of the SPAM gauge freedom at each time step. See Method for detail. However, this gauge freedom objectively exists that we cannot determine the initial state and actual SE unitaries by the LIST but a set of consistent ones. In this case, fixing the gauge provides no additional information about the initial state and S-E unitaries. Therefore, the LIST also derives the consequence that a initialization error can not be distinguished from a faulty measurement (as described in GST) at each time step $t$ \cite{greenbaum2015Introduction}. Moreover, the non-Markovian SE correlation before the intervention of the instrument can not be distinguished from the non-Markovian SE correlation after.

Third, the LIST at each time step requires the a process-informationally complete basis by combining the instrument at other time steps rather than that the system state and the measurement simultaneously and respectively form state-informationally complete basis. This is because the environment carries the information by non-Markovian SE evolutions. See Method for detail. However, it is difficult to confirm that the entanglements before and after a time step are enough to carry the information such that the specified set of time steps has the ability to construct process-informationally complete basis for tomography at the time step, because the SE dynamics are inaccessible for experimenters. In other word, the non-Markovian effect may not so severe that has high possibility to satisfy the condition of composing process-informationally complete basis. Therefore, we still recommend constructing state-informationally complete basis before and after the time step, respectively. Note that this challenge becomes intractable when conducting tomography at time steps closed to the time edge, especially $0$ and $k$, that the former or the later instruments can not form a state-informationally complete basis. The proposed LIST do not tackle this problem. However, the tomographic result are still compatible with the measurement probabilities.

\subsection{Maximum Likelihood Estimation based IST}

LIST provide a quick method to estimate the instruments and the non-Markovian quantum system. However, it may not always give a physical result and is incompatible of working with overcomplete data for constructing basis of decomposition, which could be used to improve the estimate. Moreover, experimenters may interested in more characteristics, for example, the S-E evolutions themself, requiring high flexibility of the model. To tackle these issue, we propose a statistical framework for IST via maximum likelihood estimation (MLE-IST). As a result, the likelihood function of instrument set is derived as
\begin{align}
  l(\hat{\mathfrak{I}})=\sum_{\bm{x}}{\left(\tilde{p}_{\bm{x}}-\hat{p}_{\bm{x}}\right)^2}/{\sigma_{\bm{x}}^2},
\end{align}
where $\tilde{p}_{\bm{x}}$ denote the measurement probability of getting $\bm{x}$ obtained by the experiment, $\sigma_{\bm{x}}^2$ is the sampling variance of $\tilde{p}_{\bm{x}}$, and $\hat{p}_{\bm{x}}$ is the estimator of measurement probability which is modeled by parameters. The MLE-IST exhibit high flexibility estimating the instrument set with physical constraints based on the various assumptions, such as CPTNI for generality or CPTP on NISQ.

Based on the full definition of instrument set in Eq.\eqref{eq:instrument_set_full}, each instrument $\mathcal{A}_{x_t}^{(t)}$ is modeled by a real matrix $\hat{R}_{x_t}^{(t)}\in[-1,1]^{d^2\times d^2}$ as the PTM representation with the CPTNI constraints. More specifically, the CP requires the Choi state of $\hat{R}_{x_t}^{(t)}$ to be positive semidefinite as 
\begin{align}
  \hat{\rho}_{x_t} = \frac{1}{d^2}\sum_{i,j=0}^{d^2-1}[\hat{R}_{x_t}^{(t)}]_{i,j} \left(\begin{matrix}P_j^T\\ P_i\end{matrix}\right) \succcurlyeq 0,
\end{align}
where $P_i$ represents the $i$-th Pauli matrix. The TNI requires the first entry to be $0\le[\hat{R}_{x_t}^{(t)}]_{0,0} \le 1$. The initial S-E state $\vert\hat{\rho}^{(0)}_{SE}\rangle\!\rangle\in[-1,1]^{d^2\times 1}$ is modeled as a real vector with CP and unit-trace constraints. In other word, the corresponding density matrix is positive semidefinite, and the first entry of $\vert\hat{\rho}^{(0)}_{SE}\rangle\!\rangle$ is $1/\sqrt{d}$. Without loss of generality, we assume that the S-E evolutions do not include the operation on the system dimension, which means any evolution on the system only are absorbed into the instruments. Hence, we can use $\bm{\alpha}^{(t:t+1)} \in [-\pi,\pi]^{d^4(d^4-1)}$ to model each $U_{t}$ corresponding to rotation angles of Pauli operators in $\mathbb{P} = \{P^{S}_i \otimes P^{E}_j|i=1,2,\dots,d^4-1,~j=0,1,\dots,d^4-1\}$, where $P_0 = I$ is the identity matrix. Letting $\bm{\sigma}$ represent the vector of Pauli operators, the recovered unitary $\hat{V}_{t:t+1}(\bm{\alpha})$ is defined as the PTM of unitary $\exp\left(\iota (\bm{\alpha}^{(t:t+1)})^T\bm{\sigma}\right)$, where $\iota^2 = -1$. We use the notation $\hat{V}_{{t:t+1}}$ to indicate $\hat{V}_{{t:t+1}}(\bm{\alpha}^{(t:t+1)})$ in the following for simplicity. Hence, the estimator of the probability is given by
\begin{align}
  \hat{p}_{\bm{x}}=\langle\!\langle 0_{SE}\rvert\!\left(\!\begin{matrix}\hat{R}^{(k-1)}_{x_{k-1}}\\I\end{matrix}\!\right) \!\!\prod_{t=0}^{k-2} \!{\hat{V}_{t:t+1}\left(\!\begin{matrix}\hat{R}^{(t)}_{x_t}\\ I\end{matrix}\right)} \!\lvert\hat{\rho}^{(0)}_{SE}\rangle\!\rangle.
\end{align}

Then, the optimization problem describing MLE-IST based on the full definition of instrument set is given by
\begin{align}\label{eq:mle_ist_opt_prob_full}
  \min&_{\substack{\lvert \hat{\rho}^{(0)}_{SE}\rangle\!\rangle,\hat{R}^{(t)}_{x_t}, \bm{\alpha}^{(t:t+1)},
  \forall x_t, t}}~l(\hat{\mathfrak{I}}),\\
  s.t.~& \hat{\rho}_{x_t} = \frac{1}{d^2}\sum_{i,j=0}^{d^2-1}[\hat{R}_{x_t}^{(t)}]_{i,j}\left(\begin{matrix}P_j^T\\ P_i\end{matrix}\right) \succcurlyeq 0, \forall x_t,\tag{C1} \label{eq:full_def_cp_costraint}\\
  &0\le[\hat{R}_{x_t}^{(t)}]_{1,1} \le 1, \forall x_t,\tag{C2} \label{eq:full_def_tni_costraint}\\
  &-1\le[\hat{R}_{x_t}^{(t)}]_{i,j} \le 1, \forall x_t,i,j,\tag{C3} \label{eq:full_def_ptm_val_constraint}\\
  &\hat{\rho}_{SE}^{(0)} = \frac{1}{\sqrt{d}} \sum_{i=0}^{d^4-1} \langle\!\langle i \vert\hat{\rho}^{(0)}_{SE}\rangle\!\rangle P_i \succcurlyeq 0, \tag{C4} \label{eq:full_def_init_state_cp_constraint}\\
  &[\vert\hat{\rho}^{(0)}_{SE}\rangle\!\rangle]_0 = 1/\sqrt{d}, \tag{C5} \label{eq:full_def_init_state_tr1_constraint}\\
  &-\pi \le [\bm{\alpha}^{(t)}]_i \le \pi, \tag{C6} \label{eq:full_def_upval_constraint}
\end{align}
where \eqref{eq:full_def_cp_costraint} and \eqref{eq:full_def_tni_costraint} constraint the instruments to be CP and TNI, respectively, \eqref{eq:full_def_ptm_val_constraint} defines the range of PTM entries, \eqref{eq:full_def_init_state_cp_constraint} and \eqref{eq:full_def_init_state_tr1_constraint} restrict the initial state to be CP and with unit trace, respectively, and \eqref{eq:full_def_upval_constraint} limits the range of paramters of SE unitaries. Consequently, the MLE-IST estimate the insturment set as 
\begin{gather}\label{eq:result_mleist_full}
  \hat{\mathfrak{I}}:= \left\{\hat{\mathcal{J}}, \hat{\mathcal{U}},\vert\hat{\rho}^{(0)}_{SE}\rangle\!\rangle\}\right\}\\
  \hat{\mathcal{J}}= \left\{\left\{\hat{R}^{(t)}_{0},\dots, \hat{R}^{(t)}_{m_t-1}\right\}\right\}_{t=0}^{k-1},\\
  \hat{U} = \left\{\hat{V}_{t:t+1}\right\}_{t=0}^{k-2}
\end{gather}
with $\sum_{t=0}^{k-1}m_t d^4 + (k-1)d^4(d^4-1) + d^4-1$ parameters which is linear with respect to the non-Markovian order $k$.

The model described above makes an isolation of the instruments and SE unitary dynamics that the S-E unitaries include nothing act on the system dimension only. All evolutions on the local system dimension are absorbed into the instruments. Therefore, the result data explicitly link the instruments and the transformaton of the state on system dimension. This may helps the calibration of quantum operations. Moreover, the models of instruments and the SE unitaries are flexible to be manipulated depending on assumptions the experimenter takes and the characteristics of the instrument set the experimenter interested in. For example, the constraints of instruments can be replaced by the CPTP for quantum gates on NISQ devices, while the instruments of measurements are assumed to be vectors. The SE unitary can also be modeled as a CPTP real orthonormal matrix.

It is obvious that the optimization problem is non-convex and may have multiple global optima, because each estimator $\hat{p}_{\bm{x}}$ consists of mulplications of variable matrices resulting in $(k+2)$-order of polynomial with $k$-order of exponential parameters. Hence, a reasonable initialization of parameters is significant for the optimization. We recommend conducting the LIST (or regular MLE-GST under the Markovian assumption if the LIST generates a nonphysical result) for the initialization of the MLE-IST with identity initialization of $\hat{V}_t$.

Additionally, the MLE-IST can also work with reduced instrument set. However, it requires $\mathcal{O}(d^{4k})$ parameters which is exponential with respect to the non-Markovian order $k$. It is intractable to solve the problem with exponentially increasing number of parameters. Therefore, we propose the reduced instrument set MLE-IST framework but do not implement it for simulations and experiments.

% \subsubsection*{MLE-IST with reduced instrument set}

% While performing MLE-IST with reduced instrument set (reduced MLE-IST), an instrument $\mathcal{A}_{x_t}^{(t)}$ is modeled as it in the full instrument set MLE-IST with CPTNI constraints. The process tensor is modeled by $\hat{\Upsilon}_\mathcal{T} \in [-1,1]^{d^{2k}\times d^{2k}}$ with CP and casuality constraints. Specifically, the casuality requires $\langle\!\langle \hat{\Upsilon} \vert 0\rangle\!\rangle = 1$ and
% \begin{gather}
%   \langle\!\langle \hat{\Upsilon} \vert P_{\mathrm{ban}}\rangle\!\rangle = 0, \\
%   \forall P_{\mathrm{ban}} := I^{\otimes{2t+1}} \otimes \left(\begin{matrix} \tilde{Q}_{2t+2} \\Q_{2t+3}\\\vdots\\ Q_{2k-1}\end{matrix}\right), \forall t,\\
%   \tilde{Q} \in \left\{P_1,\dots,P_{d^2-1}\right\},\\
%   Q \in \left\{P_0,P_1,\dots,P_{d^2-1}\right\}.
% \end{gather}
% Then, the estimator is modeled by
% \begin{align}
%   \hat{p}_{\bm{x}}=\langle\!\langle \hat{\Upsilon} \vert\left(\begin{matrix}\vert\hat{\chi}^{(0)}_{x_0}\rangle\!\rangle\\\vdots\\\vert\hat{\chi}^{(k-1)}_{x_{k-1}}\rangle\!\rangle\end{matrix}\right).
% \end{align}

% Consequently, the optimization problem of the reduced MLE-IST is given by
% \begin{align}\label{eq:mle_ist_opt_prob_reduced}
%   \min&_{\substack{\langle\!\langle \hat{\Upsilon} \vert, \vert\hat{\chi}^{(t)}_{x_t}\rangle\!\rangle,\forall x_t, t
%   }}~l(\hat{\mathfrak{I}})\\
%   s.t.~& \hat{\chi}^{(t)}_{x_t} = \frac{1}{\sqrt{d}} \sum_{i=0}^{d^4-1} \langle\!\langle i \vert\hat{\chi}^{(t)}_{x_t}\rangle\!\rangle P_i \succcurlyeq 0, ~ \forall{t,x_t},\tag{C7} \label{eq:reduced_def_inst_cp}\\
%   & 0 \le [\vert\hat{\chi}^{(t)}_{x_t}\rangle\!\rangle]_0 \le 1, ~\forall{t,x_t} \tag{C8} \label{eq:reduced_def_inst_tni}\\
%   & \langle\!\langle \hat{\Upsilon} \vert P_{\mathrm{ban}}\rangle\!\rangle = 0, ~\forall P_{\mathrm{ban}},\tag{C9} \label{eq:reduced_def_casuality_1}\\
%   & \langle\!\langle \hat{\Upsilon} \vert 0\rangle\!\rangle = 1, \tag{C10} \label{eq:reduced_def_casuality_2}
% \end{align}
% where \eqref{eq:reduced_def_inst_cp} and \eqref{eq:reduced_def_inst_tni} constraint the CPTNI of instruments, \eqref{eq:reduced_def_casuality_1} and \eqref{eq:reduced_def_casuality_2} guarantee the casuality of process tensor. See Method for detail. The reduced MLE-IST requires $\sum_{t=0}^{k-1}m_t d^4 + d^{4k} - \frac{d^{2k}-d^2}{d+1}$ parameters to estimate the instrument set as described in Eq.~\eqref{eq:result_list} after the devectorization of instruments.

% Obviously, this formulation constraints the instrument set in a quite simple form that all inaccessible initial state and SE unitary dynamics are modeled in a vector represented process tensor. However, it requires $\mathcal{O}(d^{4k})$ parameters which is exponential with respect to the non-Markovian order $k$. It is intractable to solve the problem with exponentially increasing number of parameters. Therefore, we propose the reduced instrument set MLE-IST framework but do not implement it for simulations and experiments. The flexibility to adjust constraints to fit the various assumptions still holds. Moreover, this optimization problem is still non-convex and may have multiple global optima, since each estimator consists of $(k+1)$-order of polynomial parameters.

\subsection{Performing IST on NISQ Devices}
A typical NISQ device execute a given quantum circuit consisting of CPTP intermidiate operations and a measurement at the end. Hence, instruments at time step $t$ consist of a set of CPTP operations and a set of measurements,
\begin{align}\label{eq:nisq_inst}
  \mathcal{J}^{(t)} := \left\{\left\{\mathcal{A}^{(t)}_{x_t}\right\}, \left\{\mathcal{M}_{x_t}^{(t)}\right\}\right\}.
\end{align}

Hence, the measurement probability for a $k$-time step non-Markovian quantum circuit is given by
\begin{align}\label{eq:nisq_exp}
  p_{\bm{x}} = \left(\begin{matrix}\langle\!\langle M^{(k-1)}_{x_{k-1}}\vert\\\langle\!\langle 0_E\vert\end{matrix}\right)\prod_{t=0}^{k-2} {U_{t:t+1}\left(\begin{matrix}A^{(t)}_{x_t}\\ I\end{matrix}\right)} \vert\rho^{(0)}_{SE}\rangle\!\rangle.
\end{align}
Associating with Eq.\eqref{eq:nisq_exp} and Eq.\eqref{eq:se_evo_prob_tr}, the measurement $\langle\!\langle M_{x_t}^{(t)}\rvert$ is the first row of the PTM of a CP and trace decreasing (TD) instrument with other entries $0$ at the end. Therefore, LIST can be performed as regular non-Markovian situation by representing measurements as regular instruments at the last time step and takes the first row as the result, when the non-Markovian correlation is sufficient to construct process-informationally complete basis by justing the instruments before the time step. However, it is intractable to perform ordinary LIST in the other case. This results from that the measurement should be the last instrument of the circuit. Hence, every time step are considered as the last time step in LIST.

Based on the observation that the PTM matrix of a (CPTD) measurement is always linear independent with the CPTP maps, the LIST can be performed by separately conducting the LIST subroutine for CPTP maps and measurements. The first row of CPTP maps are omitted in the vectorization in Eq.~\eqref{eq:inst_decomp_vec}, leading to the requirement of $d^2(d^2-1)$ measured probabilities per CPTP map and $d^2(d^2-1)\times d^2(d^2-1)$ dimensional gauge matrix. Then, the tomography of measurements are conducted by $d^2$ measured probabilities per measurement and $d^2\times d^2$ dimensional gauge matrix. Probability data and gauge matrix are measured and optimized in the two subroutine independently. Other steps are the same as ordinary LIST.

As for MLE-IST, the model can be simplified to enhance the efficiency. Each measurement can be modeled by a $d^2$ dimensional real row vector $\langle\!\langle \hat{E}_{x_t}\vert\in[-1, 1]^{1\times d^2}$ with positive constraints, i.e., both the matrix $\hat{E}_{x_t}$ the $\langle\!\langle \hat{E}_{x_t}\vert$ represents and $I-\hat{E}_{x_t}$
are positive semidefinite. Each intermediate instrument can be modeled by $d^2\times (d^2-1)$ parameters with CP constraint, since the TP constraint implies that the first row of $\hat{R}_{x_t}^{(t)}$ is $[1,0,0,\dots,0]$. Then, the estimator of probability is given by
\begin{align}\label{eq:nisq_estimator}
  \hat{p}_{\bm{x}} = \left(\begin{matrix}\langle\!\langle \hat{E}^{(k-1)}_{x_{k-1}}\vert\\\langle\!\langle 0_E\vert\end{matrix}\right)\prod_{t=0}^{k-2} {\hat{V}_{t:t+1}\left(\begin{matrix}\hat{R}^{(t)}_{x_t}\\ I\end{matrix}\right)} \vert\hat{\rho}^{(0)}_{SE}\rangle\!\rangle.
\end{align}

Consequently, the optimization problem for MLE-IST on NISQ devices can be described as
\begin{align}
  \min&_{\substack{\lvert \rho^{(0)}_{SE}\rangle\!\rangle, \langle\!\langle E_{x_t}^{(t)} \vert,R^{(t)}_{x_t},V_{t:t+1}, 
  \forall x_t, t}}~l(\hat{\mathcal{I}}),\\
  s.t.~& \eqref{eq:full_def_cp_costraint}, \eqref{eq:full_def_ptm_val_constraint},\eqref{eq:full_def_init_state_cp_constraint},\eqref{eq:full_def_init_state_tr1_constraint}, \eqref{eq:full_def_upval_constraint}\notag\\
  &[R_{x_t}^{(t)}]_{0,i} = \delta_{0,i}, \forall x_t, i,\tag{C7}\label{eq:cptp_constraint}\\
  &\hat{E}^{(t)}= \frac{1}{\sqrt{d}} \sum_{i=0}^{d^2-1} \langle\!\langle E_{x_t}^{(t)} \vert i\rangle\!\rangle P_i \succcurlyeq 0,~ \forall t,\tag{C8}\label{eq:mea_positive}\\
  &I-\hat{E}^{(t)} \succcurlyeq 0,~ \forall t,\tag{C9}\label{eq:comp_mea_positive}
\end{align}
where Eq.~\eqref{eq:cptp_constraint} is the CPTP constraint, Eq.~\eqref{eq:mea_positive} and Eq.~\eqref{eq:comp_mea_positive} are positive constraints of measurements.

\subsection{Experiment Result}

We first conduct a $5$-time-step single-qubit LIST simulation with overcomplete instruments and SE unitaries $R_{ZZ}(0.2)$ for all time step. Details of instruments and SE unitaries are given in the Method and depicted in Fig.~\ref{fig:ideal_inst_set}. Note that the knowledge and the basic implementation of $\mathcal{A}_4$ and $\mathcal{A}_5$ are not identical. 

The tomographic result of CPTP maps are depicted in the Fig.~\ref{fig:u1_list_ptm}. The difference between knowledge and implementation of $\mathcal{A}_4$ and $\mathcal{A}_5$ are detected. However, the disharmony not only influence the tomographic results of $\mathcal{A}_4$ and $\mathcal{A}_5$, since the LIST cannot distinguish which instrument is correctly implemented. Besides, the nonunique global optima leads to the results consistent with the probability but different from corresponding PTMs in Fig.~\ref{fig:ideal_inst_set}. 

% Figure environment removed

% Figure environment removed


Then, a $5$-time-step single-qubit MLE-IST is simulated with perfect and imperfect implemented complete instruments and SE unitaries $R_{ZZ}(0.2)$ for all time step. As depicted in Fig.~\ref{fig:perfect_inst_set} and Fig.~\ref{fig:imperfect_inst_set}, the tomographic results shows that the IST methods effectively reconstruct the instrument set. However, there are non-ignorable differences between the setups and the results of unitaries, initial state and measurement at the start and the end time step in the imperfect scenario. The sufficiency of constructing process-informationally complete decomposition basis influences the tomographic result in a subtle way. There are more results of unitary evolutions and initial quantum states that fit the probability data when the sufficiency is not satisfied. As a consequence, the output may not meet the experimenter's expectation, but is loyal to the data. 

Moreover, we demonstrate the instrument set of $4$-time-step single-qubit MLE-IST on the real quantum device of IBM Quantum Experience (QX) with complete instruments in Fig~\ref{fig:ibm_lima_inst_set}. Each time step consists of 10 time slots of single qubit gate depending on the quantum hardware. The result shows the potentiality guiding the quantum device engineering.

% Figure environment removed
% Figure environment removed
% Figure environment removed

\section{Method}

\subsection{Decomposition of Instruments for LIST}

By Pauli transfer matrix (PTM) representation defined in \cite{greenbaum2015Introduction}, the probability of getting $\bm{x}$ as described in Eq.\eqref{eq:se_evo_prob_tr} can be reformed as
\begin{align}
  p_{\bm{x}} =& \langle\!\langle0_{SE}\vert \!\left(\!\!\begin{matrix}A^{(k-1)}_{x_{k-1}}\\ I\!\end{matrix}\right)\! \prod_{t=0}^{k-2}U_{t:t+1}\!\left(\begin{matrix}\!\!A^{(t)}_{x_t}\\ I\end{matrix}\right) \!\vert \rho^{(0)}_{SE}\rangle\!\rangle,
  \label{eq:nm_exp_ptm}
\end{align}
where $\vert \bullet \rangle\!\rangle$ and $\langle\!\langle \bullet\vert$ are the superoperators of a quantum state and a positive operator-valued measurement (POVM) operator, $A^{(t)}_{x_t}$ and $U_{t:t+1}$ are the PTM representations of $\mathcal{A}^{(t)}_{x_t}$ and $\mathcal{U}_{t:t+1}$, respectively, and $I$ is the identity. When performing tomography at time step $t$, the probability of getting $x_t$ with $\bm{x}^{+}$ and $\bm{x}^{-}$ is
\begin{align}
  &p_{\bm{x^+}\bm{x^-}}(x_t)\\
  &\begin{aligned}=&\langle\!\langle 0_{SE}\vert \left[\prod_{i=1}^{k-t-2}\left(\begin{matrix}A^{(t+i)}_{x^{+}_i}\\ I\end{matrix}\right)U_{t+i-1:t+i}\right]  \\
    &\left(\begin{matrix}A^{(t)}_{x_t}\\ I\end{matrix}\right)\left[\prod_{j=1}^{t-1}U_{j:j+1}(A^{(j)}_{x^{-}_j}\otimes I)\right] \vert \rho^{(0)}_{SE}\rangle\!\rangle\end{aligned}\\
  =&\langle\!\langle 0_{SE}\vert F_{\bm{x}^+} \left(\begin{matrix}A^{(t)}_{x_t}\\ I\end{matrix}\right)F_{\bm{x}^-}\vert \rho^{(0)}_{SE}\rangle\!\rangle\\
  =&\langle\!\langle F_{\bm{x}^+}^{SE}\vert  \left(\begin{matrix}A^{(t)}_{x_t}\\ I\end{matrix}\right)\vert F_{\bm{x}^-}^{SE}\rangle\!\rangle\\
  =&\sum_{ij}\langle\!\langle F^S_{\bm{x}^+,i}\vert A^{(t)}_{x_t}\vert F^S_{\bm{x}^-,j}\rangle\!\rangle \langle\!\langle F^E_{\bm{x}^+,i}\vert F^E_{\bm{x}^-,j}\rangle\!\rangle\\
  =&\mathrm{Tr}\left[\!\sum_{ij}\langle\!\langle F^E_{\bm{x}^+\!,i}\vert F^E_{\bm{x}^-\!,j}\rangle\!\rangle \vert F^S_{\bm{x}^-\!,j}\rangle\!\rangle\langle\!\langle F^S_{\bm{x}^+\!,i}\vert A^{(t)}_{x_t}\!\right],\label{eq:inst_decomp_detail}
\end{align}
which corresponds to Eq.~\eqref{eq:inst_decomp_tr_basis} implying the decomposition of $A^{(t)}_{x_t}$ on the non-orthogonal basis $\mathbb{B}^{(t)} = \left\{B_{f(\bm{x}^+,\bm{x}^-)} := \sum_{ij}\langle\!\langle F^E_{\bm{x}^+,i}\vert F^E_{\bm{x}^-,j}\rangle\!\rangle \vert F^S_{\bm{x}^-,j}\rangle\!\rangle\langle\!\langle F^S_{\bm{x}^+,i}\vert\right\}$. Reconstruction of $A^{(t)}_{x_t}$ requires $\mathbb{B}^{(t)}$ to be process-informationally complete that there exist at least $d^4$ linear independent basis matrices in it. Then we can obtain the decomposition in Eq.\eqref{eq:inst_decomp_vec} via the vectorization of the matrices. 

% The vectorization method is specified as the superoperator of the Choi state of the operand without loss of generality for the simplicity of process tensor representation. Specifically, let 
% \begin{align}
%   \vert \chi_{x_t}^{(t)}\rangle\!\rangle = \sum_i s_i A_{x_t}^{(t)} \vert i \rangle\!\rangle \otimes \vert i \rangle\!\rangle,\\
%   \langle\!\langle B_{\alpha}^{(t)}\vert = \sum_j s_j \langle\!\langle j \vert B_{\alpha}^{(t)} \otimes \langle\!\langle j\vert,
% \end{align}
% where $\sum_i s_i \vert i \rangle\!\rangle \otimes \vert i \rangle\!\rangle$ is the superoperator of maximum entangled state, $s_i \in \{-1, 1\}$. Then, the measurement probability is represented by 
% \begin{align}
%   \left(\vert p_{x_t}^{(t)}\rangle\!\rangle\right)_\alpha =& \langle\!\langle B_{\alpha}^{(t)}\vert \chi_{x_t}^{(t)}\rangle\!\rangle \\
%   =& \sum_{i,j} s_i s_j \langle\!\langle j \vert B_{\alpha}^{(t)} A_{x_t}^{(t)} \vert i \rangle\!\rangle \otimes \langle\!\langle j\vert i \rangle\!\rangle\\
%   =& \sum_{i} \langle\!\langle i \vert B_{\alpha}^{(t)} A_{x_t}^{(t)} \vert i \rangle\!\rangle\\
%   =& \mathrm{Tr}\left[B_{\alpha}^{(t)} A_{x_t}^{(t)}\right] = p_{\alpha,x_t}^{(t)},
% \end{align}
% which brings about the Eq.\eqref{eq:inst_decomp_ptm_choi}.


\subsection{Gauge Freedom}
The tomography of the instrument set shows gauge freedom up to a set of invertible matrices $\{B^{(t)}\}$ because of the inaccessible initial state and SE unitaries. We can not distinguish the quantum operations by probability measurement up to $\{B^{(t)}\}$. This is because the probability is given by
\begin{align}
  p_{\bm{x}} &= \mathrm{Tr}\left[\Upsilon_{\mathcal{T}}^\dagger\left(\begin{matrix}A^{(0)}_{x_0}\\\vdots\\A^{(k-1)}_{x_{k-1}}\end{matrix}\right)\right]\\
  &=\sum_{\bm{x}}p_{\bm{x}}\prod_{t=0}^{k-1}\mathrm{Tr}\left[D^{(t)\dagger}_{x_t}A^{(t)}_{x_t}\right]\\
  &=\sum_{\bm{x}}p_{\bm{x}}\prod_{t=0}^{k-1}{\bm{d}}^{(t)\dagger}_{x_t}\bm{a}^{(t)}_{x_t}\\
  &=\sum_{\bm{x}}p_{\bm{x}}\prod_{t=0}^{k-1}{\bm{q}}^{(t)\dagger}_{x_t}\left(B^{(t)}\right)^{-1} B^{(t)}\bm{p}^{(t)}_{x_t},
\end{align}
where $\left\{ \bm{q}^{(t)}_{x_{t}}\right\}$ is the dual set of $\left\{\bm{p}^{(t)}_{x_{t}}\right\}$ corresponding to $\left\{A^{(t)}_{x_{t}}\right\}$. This indicates that, for any gauge $\{B^{(t)}\}$, we can obtain a set of instruments and process tensor without violations of measurement probabilities $p_{\bm{x}}$.

Another kind of gauge freedom is the indeterminacy of SE unitaries and initial states. Specifically, we can not distinguish $\langle\!\langle 0_{SE}\vert F_{\bm{x}^+} U(A^{(t)}_{x_t}, I)F_{\bm{x}^-}\vert\rho^{(0)}_{SE}\rangle\!\rangle$ and $\langle\!\langle 0_{SE}\vert F_{\bm{x}^+} ( A^{(t)}_{x_t}, I ) F_{\bm{x}^-} U\vert\rho^{(0)}_{SE}\rangle\!\rangle$, where U is an arbitrary operation that commutes with $(A, I)$. For example, the depolarizing noise on the system with an arbitrary operation on the environment. This generally results in different sets $\{F_{\bm{x}^+}, F_{\bm{x}^-}, \vert\rho^{(0)}_{SE}\rangle\!\rangle\}$ that consistent with the data. We adopt the reduced definition of instrument set using process tensor instead of discussing $F_{\bm{x}^+}$, $F_{\bm{x}^-}$ and $\vert\rho^{(0)}_{SE}\rangle\!\rangle$ themselves to avoid explicit introducing of this type of gauge freedom.

\subsection{Likelihood Function}

Specifying a sequence $\bm{x}$, the probability defined in Eq.~\eqref{eq:se_evo_prob_tr} is measured by repeating the experiment $n_s$ times and record $n_{\bm{x}}$ how many times the desired outputs occur. Therefore, we use the general likelihood function of instrument set
\begin{align}
  \mathcal{L}(\hat{\mathfrak{I}}) = \prod_{\bm{x}}(\hat{p}_{\bm{x}})^{n_{\bm{x}}}(1-\hat{p}_{\bm{x}})^{n_{s}-n_{\bm{x}}},
\end{align}
where $\hat{p}_{\bm{x}}$ is the probability estimator modeled by parameters.

By exploiting the central limit theorem, each term of the likelihood can be rewritten as a normal distribution,
\begin{align}
  \mathcal{L}(\hat{\mathfrak{I}}) = \prod_{\bm{x}}\exp\left[-\frac{\left(\tilde{p}_{\bm{x}}-\hat{p}_{\bm{x}}\right)^2}{\sigma_{\bm{x}}^2}\right],
\end{align}
where $\tilde{p}_{\bm{x}}=n_{bm{x}}/n_s$ represents the measured probability, $\sigma_{\bm{x}}^2=\tilde{p}_{\bm{x}}(1-\tilde{p}_{\bm{x}})/n_s$ is the sampling variance in the measurement $m_{\bm{x}}$. Exploiting the the monotonic logarithm function, maximizing $\mathcal{L}$ is equivalent to minimizing the weighted mean square error (MSE)
\begin{align}
  l(\hat{\mathfrak{I}})=& -\log(\mathcal{L}(\hat{\mathfrak{I}})) = \sum_{\bm{x}}\frac{\left(\tilde{p}_{\bm{x}}-\hat{p}_{\bm{x}}\right)^2}{\sigma_{\bm{x}}^2}.
\end{align}

\subsection{Detail of Numerical Simulation}
We conduct the numerical simulations on classical computers by Python with Pennylane package. The probability $\tilde{p}$ is analytically computed without sampling error appling the model as shown in Fig.~\ref{fig:qsp}. We specify the number of qubits to simulating the environment is the same as the system. Moreover, we introduce the $U_{\textnormal{-}1:0}$ to generate the initial state as $\vert\rho^{(0)}_{SE}\rangle\!\rangle=U_{\textnormal{-}1:0}\vert \rho^{(\textnormal{-}1)}_{SE}\rangle\!\rangle$, where $\vert \rho^{(\textnormal{-}1)}_{SE}\rangle\!\rangle$ is the superoperator of $\vert0_{SE}\rangle\langle0_{SE}\vert$.

Instruments consist of a measurement $\mathcal{M}$ and CPTP operations $\mathcal{A}_i$, $i=0,1,\dots,5$. 
The knowledge of measurements and CPTP maps known to the experimenter is defined as
\begin{align}
  \mathcal{M} &:=\vert 0 \rangle\langle 0\vert,~ \mathcal{A}_0 := I,~ \mathcal{A}_1 := X,\\
  \mathcal{A}_{2} &:=\sqrt{X}R_Z\left(\frac{\pi}{2}\right),~
  \mathcal{A}_3 := R_Z\left(\frac{\pi}{2}\right)\sqrt{X},\\
  \mathcal{A}_4 &:=\sqrt{X}R_Z\left(\frac{\pi}{3}\right),~ \mathcal{A}_5 := R_Z\left(\frac{\pi}{3}\right)\sqrt{X}.
\end{align}
However, the basic implementation of the $\mathcal{A}_4$ and $\mathcal{A}_5$ are $\sqrt{X}R_Z\left(\frac{\pi}{4}\right)$ and $R_Z\left(\frac{\pi}{4}\right)\sqrt{X}$, respectively.

The perfect implementations of instruments are the basic implementations described above. For the imperfect implementations, there is a depolarizing channel and an amplitude damping channel after the basic implementation of each CPTP operation. The parameters of depolarizing and amplitude damping are set increasingly with respect to the time step as $0.05(t+1)$. 

Simulations uses both the complete instruments and overcomplete instruments, where the complete instruments are given by
\begin{equation}\label{eq:complete_instruments}
  \mathcal{J}^{(t)} := \left\{\begin{aligned}&\left\{\mathcal{A}_0, ..., \mathcal{A}_3, \mathcal{M}\right\}, & t\ne k-1,\\
    &\left\{\mathcal{M}\right\}, & t=k-1,\end{aligned}\right.\\
\end{equation}
and the overcomplete instruments are defined as
\begin{equation}\label{eq:overcomplete_instruments}
  \mathcal{J}^{(t)} := \left\{\begin{aligned}&\left\{\mathcal{A}_0, ..., \mathcal{A}_5, \mathcal{M}\right\}, & t\ne k-1,\\
    &\left\{\mathcal{M}\right\}. & t=k-1.\end{aligned}\right.\\
\end{equation}


% For all simulations, SE unitary evolutions are selected from unitaries defined as
% \begin{align}
%   U_0 &:= R_{ZZ}(0.2)R_{YY}(0.2)R_{XX}(0.2),\\
%   U_1 &:= R_{ZZ}(0.2),\\
%   U_2 &:= R_{IX}(0.2)R_{ZZ}(0.2),\\
%   U_3 &:=  R_{IX}(0.2)R_{ZZ}(0.2)R_{XI}(0.2),
% \end{align}
% where $R_{P^SP^E}(\theta):=\exp(\frac{-\iota \theta P^SP^E}{2})$. Specifically, $U_1$, $U_2$, and $U_3$ introduce the non-Markovian quantum correlations that are insufficient to construct process-informationally complete decomposition basis for the first and last time steps but sufficient for intermediate time steps. On the contrary, the non-Markovian correlations of $U_0$ are sufficient to construct the basis for all time steps. Besides, there is no component of $U_0$ and $U_1$ on the system or the environment only. $U_2$ and $U_3$ have the components on the environment and both the system and environment, respectively.


\section{Discussion}\label{sec:discussion}

In this paper, we proposed a framework the instrument set tomography (IST) for quantum gate set tomography under the non-Markovian situation. Based on the quantum stochastic process operationally representing the non-Markovian quantum correlation and evolution, the instrument set is defined in the full and reduced formation. We first proposed a quick linear inversion method based on the reduced instrument set for IST, aka LIST. Consequently, both the disharmony of linear relationship of instruments and the non-Markovian quantum correlations are detected and described with gauge freedom by LIST. However, because of the absence of constraints in the gauge optimization, the result of linear independent instruments is always the prior knowledge when the probability matrix is full rank. Moreover, the result of LIST is not guaranteed to be physical implementable. Then, a statistical method based on the miximum likelihood estimation for IST is proposed as MLE-IST with the ability utilizing overcomplete data. Based on the full instrument set, the MLE-IST tries to explicitly describe the detail of SE correlations with polynomial number of parameters with respect to the Markovian order. The results of MLE-IST is guaranteed to be physical implementable with constraints based on the assumptions of the quantum device. Specifically, we demonstrate how to implement IST on the current noisy quantum intermediate-scale quantum (NISQ) devices. The results of simulations and experiments shows the effectiveness of describing instruments and the non-Markovian quantum system including the initial state and the SE correlations. The IST provide an essential method for benchmarking and developing a quantum device under non-Markovian situation in the aspect of instrument set.
\backmatter

\bmhead{Acknowledgments}

This work is supported by NSFC projects 61960206005 and the Fundamental Research Funds for the Central Universities 2242022k60001, and in part by the National Science Foundation of China under Grant 61871111.


%%===========================================================================================%%
%% If you are submitting to one of the Nature Portfolio journals, using the eJP submission   %%
%% system, please include the references within the manuscript file itself. You may do this  %%
%% by copying the reference list from your .bbl file, paste it into the main manuscript .tex %%
%% file, and delete the associated \verb+\bibliography+ commands.                            %%
%%===========================================================================================%%

\bibliography{citations}% common bib file
%% if required, the content of .bbl file can be included here once bbl is generated
%%\input sn-article.bbl


\end{document}

\begin{thebibliography}{10}

\bibitem{David:1985nj}
F.~David, \emph{{A Model of Random Surfaces with Nontrivial Critical
  Behavior}}, \href{https://doi.org/10.1016/0550-3213(85)90363-3}{\emph{Nucl.
  Phys. B} {\bfseries 257} (1985) 543}.

\bibitem{Ambjorn:1985az}
J.~Ambjorn, B.~Durhuus and J.~Frohlich, \emph{{Diseases of Triangulated Random
  Surface Models, and Possible Cures}},
  \href{https://doi.org/10.1016/0550-3213(85)90356-6}{\emph{Nucl. Phys. B}
  {\bfseries 257} (1985) 433}.

\bibitem{DiFrancesco:2004qj}
P.~Di~Francesco, \emph{{2D quantum gravity, matrix models and graph
  combinatorics}},  in \emph{{NATO Advanced Study Institute: Marie Curie
  Training Course: Applications of Random Matrices in Physics}}, pp.~33--88, 6,
  2004 [\href{https://arxiv.org/abs/math-ph/0406013}{{\ttfamily
  math-ph/0406013}}].

\bibitem{DiFrancesco:1993cyw}
P.~Di~Francesco, P.H.~Ginsparg and J.~Zinn-Justin, \emph{{2-D Gravity and
  random matrices}},
  \href{https://doi.org/10.1016/0370-1573(94)00084-G}{\emph{Phys. Rept.}
  {\bfseries 254} (1995) 1}
  [\href{https://arxiv.org/abs/hep-th/9306153}{{\ttfamily hep-th/9306153}}].

\bibitem{Ambjorn:1990ge}
J.~Ambjorn, B.~Durhuus and T.~Jonsson, \emph{{Three-dimensional simplicial
  quantum gravity and generalized matrix models}},
  \href{https://doi.org/10.1142/S0217732391001184}{\emph{Mod. Phys. Lett. A}
  {\bfseries 6} (1991) 1133}.

\bibitem{Sasakura:1990fs}
N.~Sasakura, \emph{{Tensor model for gravity and orientability of manifold}},
  \href{https://doi.org/10.1142/S0217732391003055}{\emph{Mod. Phys. Lett. A}
  {\bfseries 6} (1991) 2613}.

\bibitem{Godfrey:1990dt}
N.~Godfrey and M.~Gross, \emph{{Simplicial quantum gravity in more than
  two-dimensions}},
  \href{https://doi.org/10.1103/PhysRevD.43.R1749}{\emph{Phys. Rev. D}
  {\bfseries 43} (1991) 1749}.

\bibitem{Boulatov:1992vp}
D.~Boulatov, \emph{{A Model of Three-Dimensional Lattice Gravity}},
  \href{https://doi.org/10.1142/s0217732392001324}{\emph{Mod. Phys. Lett. A}
  {\bfseries 07} (1992) 1629} [\href{https://arxiv.org/abs/9202074}{{\ttfamily
  9202074}}].

\bibitem{Ooguri:1992eb}
H.~Ooguri, \emph{{Topological lattice models in four-dimensions}},
  \href{https://doi.org/10.1142/S0217732392004171}{\emph{Mod. Phys. Lett. A}
  {\bfseries 7} (1992) 2799}
  [\href{https://arxiv.org/abs/hep-th/9205090}{{\ttfamily hep-th/9205090}}].

\bibitem{DP-F-K-R}
R.~De~Pietri, L.~Freidel, K.~Krasnov and C.~Rovelli, \emph{{Barrett-Crane model
  from a Boulatov-Ooguri field theory over a homogeneous space}},
  \href{https://doi.org/10.1016/S0550-3213(00)00005-5}{\emph{Nucl. Phys. B}
  {\bfseries 574} (2000) 785}
  [\href{https://arxiv.org/abs/hep-th/9907154}{{\ttfamily hep-th/9907154}}].

\bibitem{P-R}
A.~Perez and C.~Rovelli, \emph{{A Spin foam model without bubble divergences}},
  \href{https://doi.org/10.1016/S0550-3213(01)00030-X}{\emph{Nucl. Phys. B}
  {\bfseries 599} (2001) 255}
  [\href{https://arxiv.org/abs/gr-qc/0006107}{{\ttfamily gr-qc/0006107}}].

\bibitem{P-R2}
A.~Perez and C.~Rovelli, \emph{{Spin foam model for Lorentzian general
  relativity}}, \href{https://doi.org/10.1103/PhysRevD.63.041501}{\emph{Phys.
  Rev. D} {\bfseries 63} (2001) 041501}
  [\href{https://arxiv.org/abs/gr-qc/0009021}{{\ttfamily gr-qc/0009021}}].

\bibitem{Freidel:06}
L.~Freidel and E.R.~Livine, \emph{Ponzano{\textendash}regge model revisited:
  {III}. feynman diagrams and effective field theory},
  \href{https://doi.org/10.1088/0264-9381/23/6/012}{\emph{Classical and Quantum
  Gravity} {\bfseries 23} (2006) 2021}.

\bibitem{Oriti:2011jm}
D.~Oriti, \emph{{The microscopic dynamics of quantum space as a group field
  theory}},  \href{https://arxiv.org/abs/1110.5606}{{\ttfamily 1110.5606}}.

\bibitem{Krajewski:2011zzu}
T.~Krajewski, \emph{{Group field theories}},
  \href{https://doi.org/10.22323/1.140.0005}{\emph{PoS} {\bfseries QGQGS2011}
  (2011) 005} [\href{https://arxiv.org/abs/1210.6257}{{\ttfamily 1210.6257}}].

\bibitem{Carrozza:2016vsq}
S.~Carrozza, \emph{{Flowing in Group Field Theory Space: a Review}},
  \href{https://doi.org/10.3842/SIGMA.2016.070}{\emph{SIGMA} {\bfseries 12}
  (2016) 070}.

\bibitem{Oriti:2016acw}
D.~Oriti, \emph{{The universe as a quantum gravity condensate}},
  \href{https://doi.org/10.1016/j.crhy.2017.02.003}{\emph{Comptes Rendus Phys.}
  {\bfseries 18} (2017) 235}
  [\href{https://arxiv.org/abs/1612.09521}{{\ttfamily 1612.09521}}].

\bibitem{Gielen:2016dss}
S.~Gielen and L.~Sindoni, \emph{{Quantum Cosmology from Group Field Theory
  Condensates: a Review}},
  \href{https://doi.org/10.3842/SIGMA.2016.082}{\emph{SIGMA} (2016) }
  [\href{https://arxiv.org/abs/1602.08104}{{\ttfamily 1602.08104}}].

\bibitem{Benedetti:2020yvb}
D.~Benedetti, R.~Gurau and K.~Suzuki, \emph{{Conformal symmetry and composite
  operators in the $O(N)^{3}$ tensor field theory}},
  \href{https://doi.org/10.1007/JHEP06(2020)113}{\emph{JHEP} {\bfseries 06}
  (2020) 113} [\href{https://arxiv.org/abs/2002.07652}{{\ttfamily
  2002.07652}}].

\bibitem{Benedetti:2017fmp}
D.~Benedetti, S.~Carrozza, R.~Gurau and A.~Sfondrini, \emph{{Tensorial
  Gross-Neveu models}},
  \href{https://doi.org/10.1007/JHEP01(2018)003}{\emph{JHEP} {\bfseries 01}
  (2018) 003} [\href{https://arxiv.org/abs/1710.10253}{{\ttfamily
  1710.10253}}].

\bibitem{Gurau:2010ba}
R.~Gurau, \emph{{The 1/N expansion of colored tensor models}},
  \href{https://doi.org/10.1007/s00023-011-0101-8}{\emph{Annales Henri
  Poincare} {\bfseries 12} (2011) 829}
  [\href{https://arxiv.org/abs/1011.2726}{{\ttfamily 1011.2726}}].

\bibitem{Gurau:2011xq}
R.~Gurau, \emph{{The complete 1/N expansion of colored tensor models in
  arbitrary dimension}},
  \href{https://doi.org/10.1007/s00023-011-0118-z}{\emph{Annales Henri
  Poincare} {\bfseries 13} (2012) 399}
  [\href{https://arxiv.org/abs/1102.5759}{{\ttfamily 1102.5759}}].

\bibitem{Dartois:2013he}
S.~Dartois, V.~Rivasseau and A.~Tanasa, \emph{{The $1/N$ expansion of
  multi-orientable random tensor models}},
  \href{https://doi.org/10.1007/s00023-013-0262-8}{\emph{Annales Henri
  Poincare} {\bfseries 15} (2014) 965}
  [\href{https://arxiv.org/abs/1301.1535}{{\ttfamily 1301.1535}}].

\bibitem{Carrozza:2015adg}
S.~Carrozza and A.~Tanasa, \emph{{$O(N)$ Random Tensor Models}},
  \href{https://doi.org/10.1007/s11005-016-0879-x}{\emph{Lett. Math. Phys.}
  {\bfseries 106} (2016) 1531}
  [\href{https://arxiv.org/abs/1512.06718}{{\ttfamily 1512.06718}}].

\bibitem{Tanasa:2015uhr}
A.~Tanasa, \emph{{The Multi-Orientable Random Tensor Model, a Review}},
  \href{https://doi.org/10.3842/SIGMA.2016.056}{\emph{SIGMA} {\bfseries 12}
  (2016) 056} [\href{https://arxiv.org/abs/1512.02087}{{\ttfamily
  1512.02087}}].

\bibitem{Tanasa:2012pm}
A.~Tanasa, \emph{{Tensor models, a quantum field theoretical
  particularization}}, {\emph{Proc. Rom. Acad. A} {\bfseries 13} (2012) 225}
  [\href{https://arxiv.org/abs/1211.4444}{{\ttfamily 1211.4444}}].

\bibitem{Gurau:2019qag}
R.G.~Gurau, \emph{{Notes on tensor models and tensor field theories}},
  \href{https://doi.org/10.4171/aihpd/117}{\emph{Ann. Inst. H. Poincare D Comb.
  Phys. Interact.} {\bfseries 9} (2022) 159}
  [\href{https://arxiv.org/abs/1907.03531}{{\ttfamily 1907.03531}}].

\bibitem{Carrozza2017}
S.~Carrozza, V.~Lahoche and D.~Oriti, \emph{{Renormalizable group field theory
  beyond melonic diagrams: An example in rank four}},
  \href{https://doi.org/10.1103/PhysRevD.96.066007}{\emph{Phys. Rev. D}
  {\bfseries 96} (2017) 066007}
  [\href{https://arxiv.org/abs/1703.06729}{{\ttfamily 1703.06729}}].

\bibitem{Bonzom:2011}
V.~Bonzom, R.~Gurau, A.~Riello and V.~Rivasseau, \emph{Critical behavior of
  colored tensor models in the large n limit},
  \href{https://doi.org/10.1016/j.nuclphysb.2011.07.022}{\emph{Nuclear Physics
  B} {\bfseries 853} (2011) 174}.

\bibitem{Baratin:2013rja}
A.~Baratin, S.~Carrozza, D.~Oriti, J.~Ryan and M.~Smerlak, \emph{{Melonic phase
  transition in group field theory}},
  \href{https://doi.org/10.1007/s11005-014-0699-9}{\emph{Lett. Math. Phys.}
  {\bfseries 104} (2014) 1003}
  [\href{https://arxiv.org/abs/1307.5026}{{\ttfamily 1307.5026}}].

\bibitem{Maldacena:2016hyu}
J.~Maldacena and D.~Stanford, \emph{{Remarks on the Sachdev-Ye-Kitaev model}},
  \href{https://doi.org/10.1103/PhysRevD.94.106002}{\emph{Phys. Rev. D}
  {\bfseries 94} (2016) 106002}
  [\href{https://arxiv.org/abs/1604.07818}{{\ttfamily 1604.07818}}].

\bibitem{Rosenhaus:2018dtp}
V.~Rosenhaus, \emph{{An introduction to the SYK model}},
  \href{https://doi.org/10.1088/1751-8121/ab2ce1}{\emph{J. Phys. A} {\bfseries
  52} (2019) 323001} [\href{https://arxiv.org/abs/1807.03334}{{\ttfamily
  1807.03334}}].

\bibitem{Bonzom:2018jfo}
V.~Bonzom, V.~Nador and A.~Tanasa, \emph{{Diagrammatic proof of the large $N$
  melonic dominance in the SYK model}},
  \href{https://doi.org/10.1007/s11005-019-01194-8}{\emph{Lett. Math. Phys.}
  {\bfseries 109} (2019) 2611}
  [\href{https://arxiv.org/abs/1808.10314}{{\ttfamily 1808.10314}}].

\bibitem{Amit:1979ev}
D.J.~Amit and D.V.I.~Roginsky, \emph{{EXACTLY SOLUBLE LIMIT OF PHI**3 FIELD
  THEORY WITH INTERNAL POTTS SYMMETRY}},
  \href{https://doi.org/10.1088/0305-4470/12/5/017}{\emph{J. Phys. A}
  {\bfseries 12} (1979) 689}.

\bibitem{Benedetti:2020iku}
D.~Benedetti and N.~Delporte, \emph{{Remarks on a melonic field theory with
  cubic interaction}},
  \href{https://doi.org/10.1007/jhep04(2021)197}{\emph{JHEP} {\bfseries 2021}
  (2020) } [\href{https://arxiv.org/abs/2012.12238}{{\ttfamily 2012.12238}}].

\bibitem{Oriti:2016qtz}
D.~Oriti, L.~Sindoni and E.~Wilson-Ewing, \emph{{Emergent Friedmann dynamics
  with a quantum bounce from quantum gravity condensates}},
  \href{https://doi.org/10.1088/0264-9381/33/22/224001}{\emph{Class. Quant.
  Grav.} {\bfseries 33} (2016) 224001}
  [\href{https://arxiv.org/abs/1602.05881}{{\ttfamily 1602.05881}}].

\bibitem{Marchetti:2020qsq}
L.~Marchetti and D.~Oriti, \emph{{Quantum fluctuations in the effective
  relational GFT cosmology}},
  \href{https://arxiv.org/abs/2010.09700}{{\ttfamily 2010.09700}}.

\bibitem{Marchetti:2021gcv}
L.~Marchetti and D.~Oriti, \emph{{Effective dynamics of scalar cosmological
  perturbations from quantum gravity}},
  \href{https://doi.org/10.1088/1475-7516/2022/07/004}{\emph{JCAP} {\bfseries
  07} (2022) 004} [\href{https://arxiv.org/abs/2112.12677}{{\ttfamily
  2112.12677}}].

\bibitem{Fairbairn:2007sv}
W.~Fairbairn and E.R.~Livine, \emph{{3d Spinfoam Quantum Gravity: Matter as a
  Phase of the Group Field Theory}},
  \href{https://doi.org/10.1088/0264-9381/24/20/021}{\emph{Class. Quant. Grav.}
  {\bfseries 24} (2007) 5277}
  [\href{https://arxiv.org/abs/gr-qc/0702125}{{\ttfamily gr-qc/0702125}}].

\bibitem{Girelli:2009yz}
F.~Girelli, E.R.~Livine and D.~Oriti, \emph{{4d Deformed Special Relativity
  from Group Field Theories}},
  \href{https://doi.org/10.1103/PhysRevD.81.024015}{\emph{Phys. Rev. D}
  {\bfseries 81} (2010) 024015}
  [\href{https://arxiv.org/abs/0903.3475}{{\ttfamily 0903.3475}}].

\bibitem{Li:2017uao}
Y.~Li, D.~Oriti and M.~Zhang, \emph{{Group field theory for quantum gravity
  minimally coupled to a scalar field}},
  \href{https://doi.org/10.1088/1361-6382/aa85d2}{\emph{Class. Quant. Grav.}
  {\bfseries 34} (2017) 195001}
  [\href{https://arxiv.org/abs/1701.08719}{{\ttfamily 1701.08719}}].

\bibitem{Oriti:2006jk}
D.~Oriti and J.~Ryan, \emph{{Group field theory formulation of 3-D quantum
  gravity coupled to matter fields}},
  \href{https://doi.org/10.1088/0264-9381/23/22/027}{\emph{Class. Quant. Grav.}
  {\bfseries 23} (2006) 6543}
  [\href{https://arxiv.org/abs/gr-qc/0602010}{{\ttfamily gr-qc/0602010}}].

\bibitem{Fairbairn:2006dn}
W.J.~Fairbairn, \emph{{Fermions in three-dimensional spinfoam quantum
  gravity}}, \href{https://doi.org/10.1007/s10714-006-0395-x}{\emph{Gen. Rel.
  Grav.} {\bfseries 39} (2007) 427}
  [\href{https://arxiv.org/abs/gr-qc/0609040}{{\ttfamily gr-qc/0609040}}].

\bibitem{Freidel:2005qe}
L.~Freidel, \emph{{Group field theory: An Overview}},
  \href{https://doi.org/10.1007/s10773-005-8894-1}{\emph{Int. J. Theor. Phys.}
  {\bfseries 44} (2005) 1769}
  [\href{https://arxiv.org/abs/hep-th/0505016}{{\ttfamily hep-th/0505016}}].

\bibitem{Oriti:2006se}
D.~Oriti, \emph{{The Group field theory approach to quantum gravity}},
  \href{https://arxiv.org/abs/gr-qc/0607032}{{\ttfamily gr-qc/0607032}}.

\bibitem{Oriti:2013aqa}
D.~Oriti, \emph{{Group field theory as the 2nd quantization of Loop Quantum
  Gravity}}, \href{https://doi.org/10.1088/0264-9381/33/8/085005}{\emph{Class.
  Quant. Grav.} {\bfseries 33} (2016) 85005}
  [\href{https://arxiv.org/abs/1310.7786}{{\ttfamily 1310.7786}}].

\bibitem{Makinen:2019rou}
I.~M{\"{a}}kinen, \emph{{Introduction to SU(2) recoupling theory and graphical
  methods for loop quantum gravity}}, {\emph{arXiv} (2019) }
  [\href{https://arxiv.org/abs/1910.06821}{{\ttfamily 1910.06821}}].

\bibitem{Martin-Dussaud:2019ypf}
P.~Martin-Dussaud, \emph{{A primer of group theory for Loop Quantum Gravity and
  spin-foams}}, \href{https://doi.org/10.1007/s10714-019-2583-5}{\emph{Gen.
  Rel. Grav.} {\bfseries 51} (2019) 110}
  [\href{https://arxiv.org/abs/1902.08439}{{\ttfamily 1902.08439}}].

\bibitem{Rovelli:1990ph}
C.~Rovelli, \emph{{What Is Observable in Classical and Quantum Gravity?}},
  \href{https://doi.org/10.1088/0264-9381/8/2/011}{\emph{Class. Quant. Grav.}
  {\bfseries 8} (1991) 297}.

\bibitem{Rovelli:2004tv}
C.~Rovelli, \emph{{Quantum gravity}}, Univ. Pr., Cambridge, UK (2004).

\bibitem{Oriti:2017}
D.~Oriti, L.~Sindoni and E.~Wilson-Ewing, \emph{Bouncing cosmologies from
  quantum gravity condensates},
  \href{https://doi.org/10.1088/1361-6382/aa549a}{\emph{Classical and Quantum
  Gravity} {\bfseries 34} (2017) 04LT01}.

\bibitem{Geloun:2013}
J.B.~Geloun, \emph{On the finite amplitudes for open graphs in abelian
  dynamical colored boulatov-ooguri models},
  \href{https://doi.org/10.1088/1751-8113/46/40/402002}{\emph{Journal of
  Physics A: Mathematical and Theoretical} {\bfseries 46} (2013) 402002}.

\bibitem{Geloun:2023ray}
J.B.~Geloun, A.G.A.~Pithis and J.~Th\"urigen, \emph{{QFT with Tensorial and
  Local Degrees of Freedom: Phase Structure from Functional Renormalization}},
  \href{https://arxiv.org/abs/2305.06136}{{\ttfamily 2305.06136}}.

\bibitem{Marchetti:2020xvf}
L.~Marchetti, D.~Oriti, A.G.A.~Pithis and J.~Th\"urigen, \emph{{Phase
  transitions in tensorial group field theories: Landau-Ginzburg analysis of
  models with both local and non-local degrees of freedom}},
  \href{https://doi.org/10.1007/JHEP12(2021)201}{\emph{JHEP} {\bfseries 21}
  (2020) 201} [\href{https://arxiv.org/abs/2110.15336}{{\ttfamily
  2110.15336}}].

\bibitem{Marchetti:2022nrf}
L.~Marchetti, D.~Oriti, A.G.A.~Pithis and J.~Th\"urigen, \emph{{Mean-Field
  Phase Transitions in Tensorial Group Field Theory Quantum Gravity}},
  \href{https://doi.org/10.1103/PhysRevLett.130.141501}{\emph{Phys. Rev. Lett.}
  {\bfseries 130} (2023) 141501}
  [\href{https://arxiv.org/abs/2211.12768}{{\ttfamily 2211.12768}}].

\bibitem{Haggard:2010}
H.M.~Haggard and R.G.~Littlejohn, \emph{Asymptotics of the wigner 9$j$ symbol},
  \href{https://doi.org/10.1088/0264-9381/27/13/135010}{\emph{Classical and
  Quantum Gravity} {\bfseries 27} (2010) 135010}.

\bibitem{Costantino}
F.~Costantino and J.~Marche, \emph{Generating series and asymptotics of
  classical spin networks},  2011.
\newblock 10.48550/ARXIV.1103.5644.

\bibitem{Bonzom:2012}
V.~Bonzom and P.~Fleury, \emph{Asymptotics of wigner $3nj$-symbols with small
  and large angular momenta: an elementary method},
  \href{https://doi.org/10.1088/1751-8113/45/7/075202}{\emph{Journal of Physics
  A: Mathematical and Theoretical} {\bfseries 45} (2012) 075202}.

\bibitem{Don:2018}
P.~Don{\`{a} }, M.~Fanizza, G.~Sarno and S.~Speziale, \emph{{SU}(2) graph
  invariants, regge actions and polytopes},
  \href{https://doi.org/10.1088/1361-6382/aaa53a}{\emph{Classical and Quantum
  Gravity} {\bfseries 35} (2018) 045011}.

\end{thebibliography}

\end{document}
