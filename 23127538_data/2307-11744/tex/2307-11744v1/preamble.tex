\newif\ifsiamart
\makeatletter%
\@ifclassloaded{siamart171218}%
  {\siamarttrue}%
  {\siamartfalse}%
\makeatother%

\usepackage[margin=1in]{geometry}
\usepackage{titlesec}
\usepackage{color}
\usepackage{graphicx}
\usepackage{float}
\usepackage{amsmath}
\usepackage{amssymb}
\usepackage{amsfonts}
\usepackage{mathtools}
\usepackage{dsfont}
\usepackage{todonotes}
\usepackage{enumitem}
\usepackage[
    style=trad-abbrv,
    maxcitenames=2,
    doi=false,
    url=false,
    isbn=false,
    backend=biber]{biblatex}
\usepackage{csquotes}
\usepackage[english]{babel}
\usepackage{bold-extra}

\ifsiamart\else
\usepackage{amsthm}
\usepackage{hyperref}
\hypersetup{%
    linktocpage=true,
    colorlinks=true,
    citecolor=magenta,
    linkcolor=blue,
}
\usepackage[nameinlink,capitalise]{cleveref}
\newcommand{\email}[1]{\href{mailto:#1}{#1}}
\fi

\usepackage{setspace}
\usepackage{mathrsfs}

\DeclarePairedDelimiter{\floor}{\lfloor}{\rfloor}
\DeclarePairedDelimiter{\ceil}{\lceil}{\rceil}
\DeclarePairedDelimiter\abs{\lvert}{\rvert}
\DeclarePairedDelimiter\ip{\langle}{\rangle}
\DeclarePairedDelimiter\norm{\lVert}{\rVert}
\DeclareMathOperator*{\argmin}{arg\,min}
\DeclareMathOperator*{\argmax}{arg\,max}
\DeclareMathOperator{\sgn}{sgn}

\makeatletter
\renewcommand\section{\@startsection {section}{1}{\z@}%
                               {-3.5ex \@plus -1ex \@minus -.2ex}%
                               {2.3ex \@plus.2ex}%
                               {\normalfont\large\bfseries}}
\renewcommand\subsection{\@startsection{subsection}{2}{\z@}%
                                 {-3.25ex\@plus -1ex \@minus -.2ex}%
                                 {1.5ex \@plus .2ex}%
                                 {\normalfont\bfseries}}
\makeatother

% \setlength\parindent{0pt}

\ifsiamart
\newtheorem{remark}[theorem]{Remark}
\newtheorem{assumption}{Assumption}
\newtheorem{example}[theorem]{Example}
\else
\newtheorem{theorem}{Theorem}
\newtheorem{lemma}[theorem]{Lemma}
\newtheorem{corollary}[theorem]{Corollary}
\newtheorem{proposition}[theorem]{Proposition}
\newtheorem{assumption}[theorem]{Assumption}
\newtheorem{specialassumption}{Assumption}
\theoremstyle{remark}
\newtheorem{remarkx}[theorem]{Remark}
\theoremstyle{definition}
\newtheorem{definition}[theorem]{Definition}
\newtheorem{examplex}[theorem]{Example}
% Add marker at the end of remark / example
\newenvironment{remark}
  {\pushQED{\qed}\renewcommand{\qedsymbol}{$\triangle$}\remarkx}
  {\popQED\endremarkx}

\newenvironment{example}
  {\pushQED{\qed}\renewcommand{\qedsymbol}{$\triangle$}\examplex}
  {\popQED\endexamplex}
\fi

\ifsiamart\else
\Crefname{figure}{Figure}{Figures}
\Crefname{examplex}{Example}{Examples}
\Crefname{example}{Example}{Examples}
\crefname{lemma}{Lemma}{Lemmas}
\crefname{remark}{Remark}{Remarks}
\crefname{assumption}{Assumption}{Assumptions}
\crefname{proposition}{Proposition}{Propositions}
\crefname{section}{Section}{Sections}
\crefname{subsection}{Subsection}{Subsections}
\crefname{equation}{}{}
\Crefname{equation}{Equation}{Equations}
\fi


\ifsiamart\else
\newcommand{\orcidcolor}{ORC\textcolor{orcidlogocol}{ID}}
\newcommand{\orcid}[1]{\href{https://orcid.org/#1}{% Figure removed}}
\fi

\DeclareMathOperator{\Span}{Span}
\DeclareMathOperator{\supp}{supp}
\DeclareMathOperator{\closure}{cl}
\DeclareMathOperator{\range}{Ran}
\DeclareMathOperator{\kernel}{Ker}
\DeclareMathOperator{\diag}{diag}
\renewcommand{\d}{\mathrm d}
\newcommand{\e}{\mathrm e}
\newcommand{\dummy}{\,\cdot\,}
\newcommand{\placeholder}{\,\cdot\,}
\newcommand{\normal}{\mathcal N}
\newcommand{\vect}[1]{\boldsymbol{\mathbf #1}}
\newcommand{\mat}{\mathit}
\newcommand{\real}{\mathbb R}
\newcommand{\integer}{\mathbb Z}
\newcommand{\nat}{\mathbb N}
\newcommand{\torus}{\mathbb T}
\newcommand{\domain}{\mathbb D}
\newcommand{\grad}{\nabla}
\newcommand{\laplacian}{\Delta}
\newcommand{\expect}{\mathbb E}
\newcommand{\hessian}{\nabla^2}
\newcommand{\var}{\mathbb V}
\newcommand{\id}{\mathcal I}
\newcommand{\matid}{\mathrm{id}}
\renewcommand{\t}{\mathsf T}
\newcommand{\red}[1]{{\color{red}#1}}
\newcommand{\Z}[1]{Z[#1]}
\newcommand{\ZN}[1]{Z_N[#1]}
\newcommand{\step}{\delta}
\newcommand{\bigo}{O}
\renewcommand{\leq}{\leqslant}
\renewcommand{\geq}{\geqslant}
\newcommand{\smoothcompact}{C^{\infty}_{\rm c}}

\DeclareFieldFormat{volume}{volume \textbf{#1}}
\DeclareFieldFormat[article]{volume}{\textbf{#1}}
\renewcommand*{\bibfont}{\small}

\ifsiamart\else
\onehalfspacing
\fi

\renewcommand*{\theassumption}{\Alph{assumption}}
