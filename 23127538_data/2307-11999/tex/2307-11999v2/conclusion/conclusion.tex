%auto-ignore
\pdfoutput=1

\documentclass[../main.tex]{subfiles}

\begin{document}

\section{Conclusion} \label{sec:conclusion}

In this paper we have explored a general method for survey design and estimation in the presence of big data, in a way that accommodates many different weights, sampling approaches and statistics. Our method leads to an estimator that is asymptotically unbiased (both under the design and jointly with the superpopulation) and where both design-based and joint variance estimators are available. Using the method to integrate big and survey data reduces the design and joint variance of the estimator, provided that dependence between the variables of different units is small. All results are validated and illustrated in a simulation study that examines the performance of estimates of the median and Gini index in the case of Australian personal income.

Several directions of inquiry for future research are evident. First, note that our approach for combining big and survey data to produce nonlinear statistics has been explored theoretically and in simulation above, but the benefits of this approach in practice can only be precisely quantified by applying it to real-world data.

Second, recall that in Theorem \ref{thm:desvar} we assume that the weights are design unbiased in order to produce valid variance estimators, but many popular estimators that take advantage of auxiliary characteristics do not satisfy this property. The generalised regression estimator is a leading example, and is not design unbiased because the coefficients must be estimated. While such estimators may be asymptotically design unbiased, this is not likely to be sufficient because of the $N$ terms in \eqref{eqn:designvar} and \eqref{eqn:uncondvar}. One possible solution might be to incorporate into $\theta$ the coefficients relating the auxiliary characteristics to the target variable, and use \textit{two-step} estimation (e.g.\ Section 6 of \citealppref{Newey1994}) to recover the marginal asymptotic variance of the elements in $\hat{\theta}_s$ that are of interest. This might even extend to cases where models and auxiliary characteristics are used to correct for biases such as measurement error, or incorporate information from big data for which only uncertain linkage to the survey is available (e.g.\ \citealppref{Fellegi1969, Samuels2012}).

Third, we saw in \eqref{eqn:desvarest} and \eqref{eqn:dotpsiest} that variance estimation of nonlinear statistics often involves taking derivatives that can be tedious and error prone to do manually. As a result, existing software packages for survey design and variance estimation advance on their predecessors in part by making better use of automatic differentiation facilities to broaden the class of estimators supported by their variance estimation routines. For example, CLAN \citeppref{Andersson1994} advances on traditional computing environments (e.g.\ SAS) by supporting variance estimation for rational functions of linear estimators of totals. In turn, ReGenesees \citeppref{Zardetto2015} advances on CLAN by extending this support to functions whose derivatives may be computed via the `deriv' routine provided in base R \citeppref{R2020}. In this case, supported functions are formed by composing `$+$', `$-$', `$\times$', `$/$', `exp', `log', `sin', and several others. Such software packages (see references in \citealp{Zardetto2015}) predate recently developed automatic diffferentiation libraries (e.g.\ \citealppref{Carpenter2015, Paszke2017, Paszke2019}) that better support a greater range of functions. Further, in Section 9, \citepref{Zardetto2015} proposes exploring functional derivatives as a way of again broadening the class of estimators supported by variance estimation routines, and the estimating equations framework discussed in the current paper provides an alternative approach. We therefore contend that the time is ripe for the development of a new software library for survey design and variance estimation that takes advantage of these recent developments and supports a greater range of nonlinear statistics incorporating information from both surveys and big data.

\if0\wholepaper {
	\bibliographypref{../library}
} \fi

\end{document}