%auto-ignore
\pdfoutput=1

\documentclass[../main.tex]{subfiles}

\begin{document}

\section{Introduction}

Big data is taking on an increasingly dominant role in both official statistics and empirical research, but big data also introduces many problems for estimation and inference that are not present when relying only on probability surveys or censuses \citeppref{Christen2022}. Despite concerns about the representativeness, sustainability, relevance and interpretability of big data, a number of factors are nevertheless increasing the adoption of big data by National Statistics Offices (NSOs) \citeppref{Holmberg2012, Citro2014, Tam2015, Meng2018}. These factors include declining response rates, the high respondent burden of surveys, the affordability (to the NSO) of big data collection relative to survey collection and the demand for more timely and comprehensive statistical releases servicing both traditional and emerging data needs. There are also many practical challenges for NSOs to overcome in order to effectively link big and survey data so that their information may be combined; see \citepref{Lothian2019}.

In this paper, we consider estimation and inference using \textit{estimating equations} that integrate big and survey data to improve accuracy but preserve asymptotic unbiasedness. We cater to a broad class of statistics that includes the mean, the median and other quantiles, the Gini coefficient, linear regression coefficients, maximum likelihood estimators, and many others; see Chapter 5 of \citepref{vanderVaart1998} for a text-book introduction. We extend \citepref{Binder1983} and \citepref{Godambe1986} to provide estimators of the variance under the survey design alone (\textit{design variance}, for finite population inference) and jointly with the unknown superpopulation (\textit{joint variance}, for inference about superpopulation parameters) produced in part by standard estimators of the design variance for sums and averages (e.g.\ \citealppref{Sarndal1992}). Our approach is therefore applicable to the same complex survey designs for which standard design-based variance estimators are available. The joint variance is equal to the anticipated variance of \citepref{Isaki1982} when the latter is taken with respect to the true superpopulation, and we accommodate superpopulation models that are incorrectly specified. We also incorporate the weights into our joint variance estimator to allow for informative sampling with nonignorable designs; see \citepref{Pfeffermann1993}.

Extending the approach of \citepref{Kim2021}, our integrated estimator is asymptotically unbiased with a smaller variance than the corresponding estimator produced without big data. Following \citepref{Lohr2021}, we treat the big data as a completely enumerated stratum to show how our design-based variance estimators can be used for optimal survey design in the presence of big data to target nonlinear parameters, leading to more accurate statistics with a lower cost. Further, we accommodate modifications to the standard Horvitz-Thompson weights, for example to account for nonresponse (e.g. \citealppref{Brick2013}) and of course to incorporate the big data.

A key advantage of our method is its general applicability. Existing approaches tend to apply to a narrower class of statistics; address variance from the design or superpopulation, but not both; cater only to particular sample designs; assume Horvitz-Thompson weights; or provide only informal justification: see references above, \citepref{Binder1995}, \citepref{Imbens1996}, \citepref{Kovacevic1997}, \citepref{Wooldridge1999, Wooldridge2001, Wooldridge2002}, \citepref{Bhattacharya2007} and \citepref{Lumley2017}. Aside from the integrated median of \citepref{Covey2023}, it appears to us that the literature on integrating big and survey data has so far ignored nonlinear statistics; see \citepref{Rao2021} and \citepref{Wu2022} for reviews.

In Section \ref{sec:intest}, we introduce our integrated estimator alongside its unintegrated counterpart, and outline some desirable properties that define the broader class of estimators we consider. Theoretical results are present in Section \ref{sec:theory}, where we: 1) show that these estimators are close to their population and superpopulation counterparts for large population sizes, 2) provide estimators for both the design and joint variance based on large-population central limit theorems, and 3) show that normalising the weights so that they sum to one often (but not always) has no effect on the asymptotic behaviour of the estimator. These results are applied to produce variance estimators for examples in Section \ref{sec:eg}, where we consider Horvitz-Thompson and integrated weights, sample designs that use stratified simple random sampling without replacement, quantiles, the Gini index, linear regression coefficients and maximum likelihood estimators. In Section \ref{sec:sim}, we compare the performance of integrated, survey-only and big-data-only estimates of the median and Gini index in a simulation of Australian incomes. We finish with some concluding remarks in Section \ref{sec:conclusion}.

We will observe the following notational conventions. The indicator function $I(E)$ evaluates to one or zero according to whether or not the event $E$ is true. All probability statements, expectations and variances are under the joint distribution spanning both the survey design and superpopulation. This means that standard results in asymptotic statistics can be applied immediately, as written and without translation; see for example \citepref{vanderVaart1998} and \citepref{Davidson2021}. Probabilities, expectations and variances with respect to the design alone are then obtained via conditional probability; see Section \ref{subsec:theoryoverview}. We use $X_N \overset{P}{\to} X$ to denote convergence in probability of $X_N$ to $X$ as $N \to \infty$, and use $X_N = o_P(r_N)$ and $X_N = O_P(r_N)$ to express that $X_N$ converges to zero in probability or is bounded in probability according to the ``rate'' $r_N$; see Chapter 2 of \citepref{vanderVaart1998}.

\if0\wholepaper {
	\bibliographypref{../library}
} \fi

\end{document}