%auto-ignore
\pdfoutput=1

\documentclass[../main.tex]{subfiles}

\begin{document}

\section{Simulation}\label{sec:sim}

\subsection{Overview}

In this simulation exercise, we study the performance of integrated estimation of the median and Gini index in the context of Australian personal income. We shall compare the bias and variance of big-data-only, survey-only and integrated estimators across varying population sizes, maintaining a fixed sampling fraction. Estimators are compared under the joint distribution spanning both the design and superpopulation. 

In this case, our population consists of $12$ strata stratified by age and sex (males and females aged 24 and under, 25 to 34, 35 to 44, 45 to 54, 55 to 64 and 65 and over), where each stratum is comprised of scalar observations representing personal income. Our big-data set will emulate an administrative tax-return data set as could be sourced from a taxation department such as the Australian Taxation Office. Because not all Australian residents earning under the tax-free threshold of \$18,200 are required to submit a tax-return, we expect that such a data set would underrepresent this demographic, introducing selection bias that will be emulated by the big-data sampling mechanism in the simulation. Our simulated survey data set will represent an official survey conducted by a national statistics office, such as the Australian Bureau of Statistics' Survey of Income and Housing (SIH), and use stratified simple random sampling without replacement.\footnote{Note that the SIH typically uses a survey design different from what we consider here; see \citepref{ABSMD}.} We consider two survey sampling mechanisms: one in which units are sampled from the entire population, and a second in which only non-big-data units are sampled. 

\subsection{Simulation Design}

Firstly, we construct a superpopulation distribution from which an `Australia-like' population may be generated. This superpopulation distribution is a finite mixture with c.d.f.
\begin{equation}
F(y) = \sum_{h = 1}^{12} p_h F_h(y),
\end{equation}
where $p_h$ represents the proportion of stratum $h$ in the Australian population, and $F_h(y)$ the c.d.f.\ of stratum $h$. These proportions are derived from Table 4.1 of \citepref{ABSPI} by calculating the ratio of the counts (Column I) of each stratum for a given age (Column C) and sex (Column D) to the total (Column I, Row 28). Each c.d.f.\ $F_h(y)$ is constructed by interpolating a monotonic cubic smoothing spline through income frequency data from Graph 1 of \citepref{ABSDB}, using the method described in the `Interpolated CDFs' section of \citepref{Hippel2017}. Specifically, for each stratum $h$, we fit a cubic smoothing spline with knots $(y, F_h(y))$ at the points 
\begin{equation}
( 0, 0), \bigg ( \frac{52\eta_h b_1}{\eta}, \frac{ r_1}{100} \bigg ),  \bigg ( \frac{52\eta_h b_2}{\eta}, \frac{ \sum_{i=1}^2 r_i}{100} \bigg ), \ldots,  \bigg ( \frac{52\eta_h b_{57}}{\eta}, 1 \bigg ),
\end{equation}
where $\eta_h$ is the reported median of stratum $h$ (Table 4.1 Col.\ N of \citealppref{ABSPI}), $\eta$ is the reported population median (Table 4.1 Col.\ N of \citealppref{ABSPI}), $b_i$ is the upper bound of the $i$\textsuperscript{th} income bracket ($x$-axis) of Graph 1 of \citepref{ABSDB}\footnote{Note that \citepref{ABSDB} uses equivalised household income instead of personal income. We assume that the distribution is comparable. } and $r_i$ is the corresponding frequency from the 2019-20 financial year ($y$-axis). Bracket $b_{57}$ is set such that $\mathbb{E}[Y_h]= \mu_h$, where $\mu_h$ is the reported mean of stratum $h$ (Table 4.1 Col.\ S of \citealppref{ABSPI}). This approach also ensures the median of each stratum $h$ falls within the income bracket of the reported median $\eta_h$. The coefficients of each spline are computed by the Hyman method using the splinebins function of the R package `Binsmooth' (\citealppref{binsmooth}).

From this c.d.f.\ $F$, we calculate the superpopulation median, given by $F^{-1}(0.5)$, and the superpopulation Gini index, given by $\mathbb{E}[Y]^{-1}\int_0^\infty F(y)(1 - F(y)) \mathop{dy}$. We also estimate the asymptotic joint variance (see Theorem \ref{thm:desvar}) of the integrated median and Gini index estimators by generating $10^8$ observations from the superpopulation distribution and using these observations to calculate $\hat V ^\prime$ based on (\ref{eqn:vhatprimestrat}) and (\ref{eqn:vhatprimestratintegratedsrswor}).

We then perform a simulation consisting of the following steps:
\begin{enumerate}
	\item Generate $2 \times 10^6$ observations from the superpopulation distribution, given above.
	\item Construct $10$ populations of increasing sizes, the sizes equally spaced along the interval $[ 5 \times 10^5,  \ldots, 2 \times 10^6]$, where the population $U_k = \{1, 2, \ldots, N_k\}$ consists of the first $N_k$ individuals whose incomes were generated in Step 1. Denote $Y_{i}$ the income of the $i$\textsuperscript{th} individual.
	\item For each population $U_k$, sample $N_k/2$ individuals to comprise big data set $B_k$, where the probability of individual $i \in U_k$ being selected into $B_k$ is given by
	\begin {equation}
	\pi_i^b =
    \begin{cases}
        (|H_k| + 0.05|L_k| )^{-1} &  i \in H_k\\
        0.05(|H_k| + 0.05|L_k|)^{-1} &  i \in L_k
    \end{cases},
    \end{equation}
    where $ L_k = \{i \in U_k  \mid Y_i < 18200 \} $ and $ H_k = U_k \setminus L_k$. Note that we  $\pi_i^b$ would be unknown in practice and is not used to compute any statistics in this simulation.
	\item For each population $U_k$ and stratum $h$, sample
	\begin{equation}
	n_{k, h} =  n_k \frac{N_{k,h} S_{Y_{k,h}}}{\sum_{h=1}^{12} N_{k, h} S_{Y_{k, h}}}
	\end{equation}
	units without replacement to produce the set of sampled stratum-$h$ units units $A_{k,h}$ and the set of all surveyed units $A_k = \cup_{h=1}^{12} A_{k,h}$,  where $N_{k, h}$ is the population size of stratum $h$ from population $U_k$, $n_k = 10^{-3}N_k$,  and $S_{Y_{k, h}}^2$ is the sample variance of $Y_i$ across $i$ in stratum $h$ and population $U_k$. Within each stratum, units are selected with equal probability, yielding a first-order inclusion probability of $ \pi_i = n_{k,h}/N_{k, h}$ for a unit $i$ in stratum $h$. Given a fixed total sample size $n_{k}$, this allocation minimises the variance of the survey-only mean (see Section 3.7.4.\ i.\ of \citealppref{Sarndal1992}).
	\item Define $ U_k^{\prime} = U_k \setminus B_k$. Repeat Step 4., sampling instead from $U_k^{\prime}$ to form $A_k^{\prime}$. This is equivalent to treating the big data sample as a completely enumerated stratum, as described in Section \ref{eg:strat}.
	\item For $k \in \{1,2,\ldots,10\}$, use \eqref{eqn:pquantile} with $p = 0.5$ to calculate the following estimates of the median:
	\begin{itemize}
		\item An unweighted estimate using only $B_k$. 
		\item A Horvitz-Thompson-weighted estimate using only $A_k$.
		\item An integrated estimate using both $B_k$ and $A_k$, using the integrated weights in \eqref{eqn:wdi} alongside Horvitz-Thompson survey weights.
		\item An integrated estimate using both $B_k$ and $A_k^{\prime}$, using the integrated weights in \eqref{eqn:wdi} alongside Horvitz-Thompson survey weights.
	\end{itemize}
	\item Repeat Step 6.\ for the Gini index given by $\eqref{eqn:gini}$.
	\item Produce $10^4$ Monte-Carlo draws for each population size $N_k$ by repeating Steps 1 to 7.
	\item Across incomes given each $N_k$:
	\begin{enumerate}
	\item Calculate the sample variance of the four estimates for both the median and Gini index using the Monte-Carlo draws.
	\item Calculate the sample bias of the four estimates using the Monte-Carlo draws and the superpopulation median and Gini index. 
	\item Calculate approximate $95 \%$ confidence intervals for the bias and variance estimates, using the standard asymptotics for i.i.d.\ draws.
	\end{enumerate}
\end{enumerate}

\subsection{Results}

% Figure environment removed

% Figure environment removed

% Figure environment removed

% Figure environment removed

The sample bias of the median and Gini index estimates as a function of the population size are depicted in Figures \ref{fig:median_bias} and \ref{fig:gini_bias} respectively. Observe that the big-data-only estimate exhibits bias due to the sampling mechanism of the big data set outlined in Step 3. This mechanism overrepresents high-income earners, resulting in an overestimate of the median and an underestimate of the Gini index. Importantly, both of the integrated estimators yield a statistically insignificant bias for large populations, which is consistent with Theorem \ref{thm:normsup} and Remark \ref{rmk:clt}.

Figures \ref{fig:median_var} and \ref{fig:gini_var} depict the size-adjusted variance estimates for the Gini index and median, respectively. We see that the Monte-Carlo variance of the integrated estimators calculated in Step 9.\ are consistent with the asymptotic variance estimates described in Section \ref{sec:varest}. This demonstrates the utility of the asymptotic variance estimates for practical applications, particularly given that the asymptotic variances only very occasionally fall outside the confidence intervals for the corresponding Monte-Carlo variances for population and sample sizes consistent with those of surveys such as the SIH.

Figures \ref{fig:median_var} and \ref{fig:gini_var} also highlight the efficiency of the integrated estimators over the survey-only estimators in that they exhibit significantly lower variance. Additionally, we can see from both Figure \ref{fig:median_var} and Figure \ref{fig:gini_var} that the integrated estimate using probability sample $A_k^{\prime}$ outperforms in terms of variance the estimate using $A_k$ (with a variance of less than half). This illustrates that, in the presence of a pre-existing big data set, significant efficiency gains in terms of variance can be achieved by tailoring the survey design to exclude those units present in the big data set.

\if0\wholepaper {
	\bibliographypref{../library}
} \fi

\end{document}