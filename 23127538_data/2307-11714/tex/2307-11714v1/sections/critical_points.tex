\section{Properties of the Optimisation Landscapes of \texorpdfstring{$\SWY$}{S} and \texorpdfstring{$\SWpY$}{Sp}}\label{sec:crit}

The goal of this section is to study the respective landscapes of $\SWY$ and $\SWpY$, their critical points and the links between them. 

\subsection{Optimising \texorpdfstring{$\SWY$}{S}}

\subsubsection{Global optima of \texorpdfstring{$\SWY$}{S}}

As its name suggests, the SW distance is indeed a distance on $\mathcal{P}_2(\R^d)$ (this result can be proven in the same manner for the $q$-SW distances, for $q \geq 1$).

\begin{prop}[Bonnotte~\cite{bonnotte}, Theorem 5.1.2]\label{prop:SW_distance} SW is a distance on $\mathcal{P}_2(\R^d)$.
\end{prop}
As a consequence, the global optima of $\SWY$ are exactly the points $Y^*$ such that $\gamma_{Y^*} = \gamma_{Z}$, or said otherwise the points such that $(y_1^*, \cdots, y_\npoints^*)$ is a permutation of $(z_1, \cdots, z_\npoints)$. 

%

\subsubsection{Critical points of \texorpdfstring{$\SWY$}{SWY}}\label{sec:E_crit}

A first step in studying the landscape $\SWY$ is to determine its critical points, which we define as the set of points $Y$ where $\SWY$ is differentiable and $\nabla \SWY(Y) = 0$.
%
Thanks to \ref{thm:bonneel_diff}, these critical points can be shown to satisfy a fixed point equation. 
%


\begin{corollary}[Equation characterising the critical points of $\SWY$]\label{cor:crit_points_S}\
  Let $Y \in \mathcal{U}$ (defined in \ref{eqn:U}). For $(k,l) \in \llbracket 1, \npoints \rrbracket$, define $\Theta_{k,l}^{Y,Z} := \left\lbrace \theta \in \SS^{d-1}\ |\ \sort{Z}{\theta} \circ (\sort{Y}{\theta})^{-1} (k) = l \right\rbrace \subset \SS^{d-1}$ and $S_{k,l}^{Y,Z} := d\Int{\Theta_{k,l}^{Y,Z}}{}\theta \theta^T \dd \bbsigma \in S_d^+(\R)$.
  $Y$ is a critical point of $\SWY$ iif $Y$ satisfies
	\begin{equation}\label{eqn:crit_fixed_point}
		\forall k \in \llbracket 1, \npoints \rrbracket, \; y_k = \Sum{l=1}{\npoints}S_{k,l}^{Y,Z}z_l.
	\end{equation}
\end{corollary}

\begin{proof}
	Let $k \in \llbracket 1, \npoints \rrbracket$. We have $\SS^{d-1} = \Reu{l=1}{\npoints}\Theta_{k,l}^{Y, Z}$, where the union is disjoint, therefore one may write
	$$\cfrac{\partial \SWY}{\partial y_k}(Y) = \cfrac{2}{\npoints}\Int{\SS^{d-1}}{}\theta \theta^T (y_k- z_{\sort{Z}{\theta} \circ (\sort{Y}{\theta})^{-1}(k)})  \dd \bbsigma(\theta) =  \cfrac{2}{\npoints}\Sum{l=1}{\npoints}\Int{\Theta_{k,l}}{}\theta \theta^T (y_k - z_l) \dd \bbsigma(\theta) = \cfrac{2}{d\npoints} y_k - \cfrac{2}{d\npoints}\Sum{l=1}{\npoints}S_{k,l}^{Y,Z}z_l,$$
where we have used $\Int{\SS^{d-1}}{}\theta \theta^T \dd \bbsigma(\theta) = I/d$ in the last equality.
%
	Equating the partial differential to 0 yields~\ref{eqn:crit_fixed_point}.
\end{proof}
Equation~\ref{eqn:crit_fixed_point} shows that the critical points can be written as combinations of the points $(z_l)$, "weighted" by the normalised conditional covariance matrices $S_{k,l}^{Y,Z} = d\mathbb{E}_{\theta \sim \bbsigma}\left[\mathbbold{1}(\theta \in \Theta_{k,l}^{Y,Z})\theta\theta^T\right]$.
Note that with $\Psi := \app{\mathcal{U}}{\R^{\npoints\times d}}{Y}{\left(\begin{array}{c}
		\Sum{l=1}{\npoints}z_l^TS_{1,l}^{Y,Z} \\
		\vdots \\
		\Sum{l=1}{\npoints}z_l^TS_{\npoints,l}^{Y,Z}
	\end{array}\right)}$,~\ref{eqn:crit_fixed_point} writes as a fixed-point equation $Y = \Psi(Y)$.

Further notice that $\Psi$ cannot be properly defined on $\mathcal{U}^c$, for instance if $\npoints=2$, and if $Y = (y,y)$, the two possible sorting choices $\sort{Y}{\theta} \in \lbrace (1, 2), (2, 1)\rbrace$ yield two different values for $\Psi(Y)$ (the first value is the second with the indices exchanged).
%
We show below that $\Psi$ is continuous on $\mathcal{U}$. Unfortunately, $\Psi$ cannot be extended to the whole space $\R^{\npoints \times d}$, since the restrictions $\Psi|_{C_\config}$ may have distinct limits at the borders of the cells.

\begin{prop}[Regularity of $\Psi$]\label{prop:Fcont}\  $\Psi$ is continuous on $\mathcal{U}$ (defined in \ref{eqn:U}).
\end{prop}
\begin{proof}
  It is sufficient to prove the continuity of $G := Y \rightarrow S_{k,l}^{Y,Z}$ on $ \mathcal{U}$, for  $k,l$ fixed.
  %
	Let $Y \in \mathcal{U}$ and $\varepsilon > 0$. Define $\Theta_\varepsilon(Y) := \left\lbrace \theta \in \SS^{d-1} \ |\ \forall \delta Y \in B(0, \varepsilon),\ \left(\theta \cdot y_{\sort{Y}{\theta}(k)} + \theta \cdot \delta y_{\sort{\delta Y}{\theta}(k)}\right)_{k \in \llbracket 1, \npoints \rrbracket} \in \mathcal{U}_{\npoints, 1} \right\rbrace$, with $\mathcal{U}_{\npoints, 1}$ the open set of lists $(x_1, \cdots, x_\npoints) \in \R^{\npoints}$ with distinct entries.
	By Bonneel et al.~\cite{bonneel2015sliced}, Appendix A, Lemma 2, $\forall \theta \in \Theta_\varepsilon(Y),\; \forall \delta Y \in B(0, \varepsilon),\; \sort{Y}{\theta} = \sort{Y+\delta Y}{\theta}$.

	Let $\varepsilon$ small enough such that $\forall \delta Y \in B(0,\varepsilon),\; Y+\delta Y \in \mathcal{U}$. Let $\delta Y \in B(0,\varepsilon)$. Separating the integral yields:

	$$G(Y+\delta Y) = \Int{\Theta_{k,l}^{Y+\delta Y,Z}}{}\theta\theta^T \dd\bbsigma(\theta)  =\Int{\Theta_{k,l}^{Y+\delta Y,Z} \bigcap \Theta_\varepsilon(Y)}{}\theta\theta^T \dd\bbsigma(\theta) + \Int{\Theta_{k,l}^{Y+\delta Y,Z} \bigcap \Theta_\varepsilon(Y)^c}{}\theta\theta^T \dd\bbsigma(\theta).$$
Using the fact that $\Theta_{k,l}^{Y+\delta Y,Z} \bigcap \Theta_\varepsilon(Y) = \Theta_{k,l}^{Y,Z} \bigcap \Theta_\varepsilon(Y)$, and denoting $\|\cdot\|_{\mathrm{op}}$ the $\|\cdot\|_2$-induced operator norm on $\R^{d \times d}$, we get
%
$$G(Y+\delta Y) - G(Y) = \Int{\Theta_{k,l}^{Y+\delta Y,Z} \bigcap \Theta_\varepsilon(Y)^c}{}\theta\theta^T \dd\bbsigma(\theta) - \Int{\Theta_{k,l}^{Y,Z} \bigcap \Theta_\varepsilon(Y)^c}{}\theta\theta^T \dd\bbsigma(\theta), \quad \text{ thus }$$  
\begin{align}\|G(Y+\delta Y) - G(Y)\|_{\mathrm{op}} &\leq \Int{\Theta_{k,l}^{Y+\delta Y,Z} \bigcap \Theta_\varepsilon(Y)^c}{}\left\|\theta\theta^T\right\|_{\mathrm{op}} \dd\bbsigma + \Int{\Theta_{k,l}^{Y,Z} \bigcap \Theta_\varepsilon(Y)^c}{}\left\|\theta\theta^T\right\|_{\mathrm{op}} \dd\bbsigma \\
  &\leq 2\Int{ \Theta_\varepsilon(Y)^c}{}1 \dd\bbsigma = 2\bbsigma(\Theta_\varepsilon(Y)^c).
\end{align}
By Bonneel et al.~\cite{bonneel2015sliced}, Appendix A, Lemma 3, there exists a constant $C$ such that $\bbsigma(\Theta_\varepsilon(Y)^c) \leq C \varepsilon$, which  proves the continuity of $G$ on $\mathcal{U}$.
\end{proof}

\subsection{Optimising \texorpdfstring{$\SWpY$}{Sp}}

\subsubsection{Global optima of \texorpdfstring{$\SWpY$}{Sp}}

We saw in~\ref{prop:SW_distance} that SW is a distance. Unfortunately, its discretised version $\SWMC$ is only a pseudo-distance: the arguments for non-negativity, symmetry and the triangular inequality still hold, but separation comes to a fault.

For generic measures, a measure-theoretic way of seeing this is through characteristic functions. Given $\mu, \nu \in \mathcal{P}_2(\R^d)$ and $(\theta_1, \cdots, \theta_p) \in (\SS^{d-1})^p$, the condition $\SWMC(\mu, \nu) = 0$ is equivalent to $\forall i \in \llbracket 1, p \rrbracket,\; \forall t \in \R,\; \phi_\mu(t\theta_i) = \phi_\nu(t\theta_i)$, where $\phi_\mu$ (resp. $\phi_\nu$) is the characteristic function of $\mu$ (resp. $\nu$). 
This condition only constrains the characteristic functions on $p$ radial lines, and Bochner or P\'olya-type criteria may be considered to find a characteristic function $\phi$ which equals $\phi_\mu$ on these lines but differs on a non-null set.

%

%

The discrete case pertains more to our setting. As shown in~\cite{tanguy2023reconstructing}, for $p$ large enough, almost-sure separation holds. This result can be proven by leveraging the geometrical consequences of the constrains $P_{\theta_i}\#\gamma_Y = P_{\theta_i}\#\gamma_Z$, and determining the a.s. solution set using random affine geometry.

\begin{theorem}[\cite{tanguy2023reconstructing}, Theorem 4]\label{thm:SW_insufficient_projections}
	Let {$\gamma_Z := \Sum{l=1}{\npoints}b_l\delta_{z_l}$}, where the $(z_l)$ are fixed and distinct.	Assuming $\theta_1, \cdots, \theta_p \sim \bbsigma^{\otimes p}$, we have
	\begin{itemize}
		\item if $p \leq d$, there exists $\bbsigma$-a.s. an infinity of measures $\gamma \neq \gamma_Z \in \mathcal{P}_2(\R^d)$ s.t. $\widehat{\mathrm{SW}}_p(\gamma, \gamma_Z) = 0$.

		\item if $p > d$, we have $\bbsigma$-almost surely $\lbrace\gamma_Z\rbrace = \underset{\gamma \in \mathcal{P}_2(\R^d)}{\argmin}\; \widehat{\mathrm{SW}}_p(\gamma, \gamma_Z)$.
	\end{itemize}
\end{theorem}

With a sufficient amount of projections, $\SWMC(\gamma_Y, \gamma_Z) = 0 \Rightarrow \gamma_Y = \gamma_Z$ (a.s.), hence when minimising $\SWMC(\gamma_Y, \gamma_Z)$ in $Y$, there is some hope of recovering $\gamma_Z$. Unfortunately, this does not guarantee that the (unique) solution will be attained numerically. This practical reality motivates the study of eventual local optima of $\SWpY$.

The computation of the critical points of $\SWpY$ can be done using the cell decomposition of~\ref{sec:cells}. %
We show that the critical points of $\SWpY$ are exactly the local optima of $\SWpY$, and correspond to "stable cells", which is to say cells that contain the minimum of their quadratic.

\subsubsection{Critical points of \texorpdfstring{$\SWpY$}{Sp} and cell stability}

The objective of this section is to confirm theoretically some of the intuitions provided by the illustrations of~\ref{sec:L2}, namely that the critical points of $\SWpY$ correspond to stable cells. Since the union of cells is exactly the differentiability set of $\SWpY$, any critical point $Y$ of $\SWpY$ is necessarily within a cell $\mathcal{C}_\config$. Since $\SWpY$ is quadratic on $\mathcal{C}_\config$, then a critical point $Y$ is the minimum of the cell's quadratic $q_\config$. As a consequence, the critical points of $\SWpY$ are exactly the "stable cell optima", i.e. the $Y \in \mathcal{U}$ (see the definition \ref{eqn:U}) such that $Y = \underset{Y' \in \R^{\npoints \times d}}{\argmin}\ q_{\config(Y)}(Y')$.

The following theorem shows that there are no local optima of $\SWpY$ outside of $\mathcal{U}$, and therefore that the set of local optima of $\SWpY$, the set of critical points of $\SWpY$ and the set of stable cell optima coincide. As previously, we define the set of critical points of $\SWpY$ as the set of points $Y$ where $\SWpY$ is differentiable and $\nabla \SWpY(Y) = 0$.

\begin{theorem}[The local optima of $\SWpY$ are within cells]\label{thm:Sp_crit_optloc_stable}\
 Assume that $(\theta_1, \cdots, \theta_p) \sim \bbsigma^{\otimes p}$, then the following results hold $\bbsigma$-almost surely.
  Let $Y \in \R^{\npoints\times d}$ a local optimum of $\SWpY$, then $\exists \config\in \mathfrak{S}_\npoints^p$ such that $ Y \in \mathcal{C}_\config$.
	As a consequence, we have the equality between the three sets:
	\begin{itemize}
		\item Local optima of $\SWpY$;
		\item Critical points of $\SWpY$;
		\item Stable cell optima: $\left\lbrace Y \in \mathcal{U}\ |\ Y = \underset{Y' \in \R^{\npoints \times d}}{\argmin}\ q_{\config(Y)}(Y') \right\rbrace$.
	\end{itemize}
\end{theorem}

\begin{proof}

	Let $Y \in \R^{\npoints\times d}$ a local optimum of $\SWpY$. Let $M := \lbrace \config \in \mathfrak{S}_\npoints^p\ |\ Y \in \overline{\mathcal{C}_\config} \rbrace$.

	Let $\config \in M$. Let us show that $\nabla q_\config(Y) = 0$ by contradiction: suppose $\nabla q_\config(Y) \neq 0$. For $t$ positive and small enough, %
        
	$\SWpY(Y) \leq \SWpY\left(Y - t\cfrac{\nabla q_\config(Y)}{\|\nabla q_\config(Y)\|}\right) \leq q_\config\left(Y - t\cfrac{\nabla q_\config(Y)}{\|\nabla q_\config(Y)\|}\right) = q_\config(Y) - t \|\nabla q_\config(Y)\| + o(t) = \SWpY(Y) -t \|\nabla q_\config(Y)\| +o(t).$

	Therefore, for $t>0$ sufficiently small, we have $\SWpY(Y) < \SWpY(Y),$ which is a contradiction. We now prove that $\# M = 1$.
	Using the notations of \ref{rem:quadratics}, for $\config \in M$, we have $\nabla q_\config(Y) = 0$, thus $B\vec{y} = a_\config$.
	For $(\theta_1, \cdots, \theta_p) \sim \bbsigma^{\otimes p}$, we have $\bbsigma$-almost surely that $B$ is invertible and that $\config \neq \config' \Longrightarrow a_\config \neq a_{\config'}$, thus $\bbsigma$-almost surely, $\# M = 1$, proving that in fact $Y$ belongs to $\mathcal{C}_\config$ and not to its boundary.

\end{proof}

\subsubsection{Closeness of critical points of \texorpdfstring{$\SWpY$}{Sp} and \texorpdfstring{$\SWY$}{S}}\label{sec:closeness}

In practice, all numerical optimisation methods converge towards a local optimum. %
One may wonder what is the link between the critical points of $\SWpY$, which we reach in practice, and the critical points of $\SWY$, among which are the theoretical solutions we would like to reach.

The following theorem shows that at the limit $p \rightarrow +\infty$, any sequence of critical points of $\SWpY$ satisfies become fixed points of $\Psi$ \ref{eqn:crit_fixed_point} in probability, which is to say that they exhibit similar properties to the critical points of $\SWY$.

\begin{theorem}[Approximation of the fixed-point equation]\label{thm:cv_fixed_point_distance}\

	For $p > d$, let $Y_p$ any critical point of $\SWpY$. Then we have the convergence in probability:
	\begin{equation}\label{eqn:cv_fixed_point_distance}
		Y_p - \Psi(Y_p) \xrightarrow[p \longrightarrow +\infty]{\P} 0.
	\end{equation}

	Specifically (see \ref{cor:simplified_condition_concentration}), in order to reach a precision of $\varepsilon$, we have $\|Y_p - \Psi(Y_p)\|_{\infty, 2} \leq \varepsilon$ with probability exceeding $1 - \eta$ if $p \geq \mathcal{O}\left(d^3\npoints\log(1/\eta)/\varepsilon^3\right)$ and $p \geq \mathcal{O}\left(d^3\npoints^2\log(1/\eta)/\varepsilon^2\right)$, omitting logarithmic multiplicative terms in $d$ and $\npoints$.
\end{theorem}

We provide the proof in \ref{p:cv_fixed_point_distance}, where we also estimate more precisely the convergence rate. The idea behind this result stems from computing the minima of the quadratics. Let $Y^* := \underset{Y}{\argmin}\ q_\config(Y)$, we have
\begin{equation}\label{eqn:next_opt_pos}
	y_k^* = A^{-1}\left(\cfrac{1}{p}\Sum{i=1}{p}\theta_i \theta_i^Tz_{\config_i(k)}\right) = \cfrac{A^{-1}}{p}\Sum{l \in \llbracket 1, \npoints \rrbracket}{}\Sum{\substack{i \in \llbracket 1, p \rrbracket \\ \config_i(k)=l}}{}\theta_i \theta_i^Tz_l,
\end{equation}
with $A = \cfrac{1}{p}\Sum{i=1}{p}\theta_i\theta_i^T$ which approaches the covariance matrix of $\theta \sim \bbsigma$, i.e. $I/d$.
Likewise, $\cfrac{1}{p}\Sum{\substack{i \in \llbracket 1, p \rrbracket \\ \config_i(k)=l}}{}\theta_i \theta_i^T$ can be seen as an empirical conditional covariance, and it approaches $S_{k,l}^{Y,Z} / d$.
We then apply matrix concentration inequalities to quantify the approximation error.

\subsubsection{Critical points of \texorpdfstring{$\SWpY$}{Ep} and Block Coordinate Descent}\label{sec:Ep_crit_and_bcd}

Leveraging on the cell structure of $\SWpY$, we present an algorithm alternatively solving for the transport matrices and for the positions. Writing $\U$ the set of valid transport plans between two uniform measures with $\npoints$ points, we minimise the following energy (with $(\theta_1,\cdots, \theta_p)$ fixed)
\begin{equation}
	J := \app{\U^p \times \R^{\npoints \times d}}{\R_+}{(\pi^{(1)}, \cdots, \pi^{(p)}), Y}{\cfrac{1}{ p}\Sum{i=1}{p}\Sum{k=1}{\npoints}\Sum{l=1}{\npoints}(\theta_i\cdot y_k - \theta_i\cdot z_l)^2\pi_{k,l}^{(i)}}.
\end{equation}
Observe that minimising $J$ amounts to minimising $\SWpY$.
%

% Figure environment removed

The computation in \ref{alg:BCD}, line 3 is done using standard 1D OT solvers, and the update on the positions at line 4 can be computed in closed form. BCD can be seen as a walk from cell to cell (see \ref{sec:cells}), as illustrated in \ref{fig:tessellation}. BCD moves from cell to cell and converges towards a stable cell optimum, and thus towards a local optimum of $\SWpY$ (since these two sets are equal by \ref{thm:Sp_crit_optloc_stable}). This behaviour is further studied in the experimental section. 

% Figure environment removed

