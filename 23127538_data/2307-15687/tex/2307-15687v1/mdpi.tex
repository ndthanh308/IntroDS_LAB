
%\documentclass[prd,preprint,tightenlines,floatfix,showpacs,preprintnumbers,nofootinbib,eqsecnum]{revtex4}
\documentclass[prd,tightenlines,floatfix,showpacs,preprintnumbers,nofootinbib,eqsecnum]{revtex4}
 \usepackage[dvips,final]{graphicx}
  \usepackage{amssymb}
   \usepackage{amsmath} % Advanced math typesetting
    \usepackage{amsfonts}
     \usepackage{epsfig}
     \usepackage{color}
      \usepackage{bm} % bold math
        \usepackage{xcolor}
        
\usepackage{booktabs}
\usepackage{multirow}
\usepackage{slashed}
\usepackage{comment}



%%%%%%%%%%%%%%%%%%%%%%%%%%%%%%%%%%%%%%%%%%%%%%%%%%%%%%%%%%%%%%%%%%%%%%%%%
%\documentclass[aps,12pt]{revtex4}
%\usepackage{graphicx}
%\usepackage{placeins}
%
%\textheight     9.2in
%\textwidth      6.5in
%\topmargin     -0.8in
%\oddsidemargin  0.0in
%\evensidemargin 0.0in
%%%%%%%%%%%%%%%%%%%%%%%%%%%%%%%%%%%%%%%%%%%%%%%%%%%%%%%%%%%%%%%%%%%%%%%%%
\def\doublespace{\def\baselinestretch{1.6}\large\normalsize}
\def\normalspace{\def\baselinestretch{1.0}\normalsize}
\def\PSfig#1#2{\scalebox{#1}{% Figure removed}}
\def\Ref#1{(\ref{#1})}
\def\Caption#1{
  \normalspace
  \vskip-1mm\caption{\sl#1}\vskip-1mm
  \doublespace
}

\newcommand\la{\langle}
 \newcommand\ra{\rangle}
 \newcommand\beq{\begin{equation}}
 \newcommand\noi{\noindent}
 \newcommand\eeq{\end{equation}}
 \newcommand\beqn{\begin{eqnarray}}
 \newcommand\eeqn{\end{eqnarray}}
\def\mb{\,\mbox{mb}}
\def\fm{\,\mbox{fm}}
\def\MeV{\,\mbox{MeV}}
\def\GeV{\,\mbox{GeV}}
\def\TeV{\,\mbox{TeV}}

%%%%%%%%%%%%%%%%%
\begin{document}
%%%%%%%%%%%%%%%%%

%================================================
\title{
\vspace*{-2.0cm}
Unconventional mechanisms 
of heavy quark fragmentation
%
%Heavy quarkonium production \\
%in ultra-peripheral nuclear collisions
}
%================================================

\author{B. Z. Kopeliovich$^1$}
\email{boris.kopeliovich@usm.cl}

\author{J. Nemchik$^{2,3}$}
\email{jan.nemcik@fjfi.cvut.cz}
%\email{nemcik@saske.sk}

\author{I. K. Potashnikova$^1$}
\email{irina.potashnikova@usm.cl}

\author{Ivan Schmidt$^1$}
\email{ivan.schmidt@usm.cl}

\vspace*{0.5cm}

\affiliation{
\vspace*{0.2cm}
$^1$Departamento de F\'{\i}sica,
Universidad T\'ecnica Federico Santa Mar\'{\i}a,\\
Avenida Espa\~na 1680, Valpara\'iso, Chile}
\affiliation{$^2$
Czech Technical University in Prague, FNSPE, \\B\v rehov\'a 7, 11519
Prague, Czech Republic}
\affiliation{$^3$
Institute of Experimental Physics SAS,\\ Watsonova 47, 04001 Ko\v sice, Slovakia
}



\vspace*{2.0cm}
\date{\today}
%%%%%%%%%%%%%%%%%%%%%%%%%
\begin{abstract}
%%%%%%%%%%%%%%%%%%%%%%%%%
\vspace*{5mm}
Heavy and light quarks produced in high-$p_T$ partonic collisions radiate differently.
Heavy quarks regenerate their color field,  stripped-off in the hard reaction, much faster than the light ones
and radiate a significantly smaller fraction of the initial quark energy.
%
This peculiar feature of heavy-quark jets leads to a specific shape of the fragmentation functions observed in $e^+e^-$ annihilation. Differently from light flavors, the heavy quark fragmentation function strongly   peaks at large fractional momentum $z$, i.e. the produced heavy-light mesons, $B$ or $D$, carry the main fraction of the jet momentum. This is a clear evidence of the dead-cone effect, 
and of a short production time of a heavy-light mesons.
%
Contrary to propagation of a small $q\bar q$ dipole, which survives in the medium due to color transparency, a heavy-light $Q\bar q$ dipole promptly expands to a large size. Such a big dipole has no chance to remain intact in a dense medium produced in relativistic heavy ion collisions. On the other hand, a breakup of such a dipole does not affect much the production rate of $Q\bar q$ mesons, differently from the case of light $q\bar q$ meson production.%%%%%%%%%%%%%%%%%%%%%%%%
\end{abstract}
%%%%%%%%%%%%%%%%%%%%%%%%

\pacs{12.38.Bx, 12.38.Lg, 12.38.Mh, 12.38.-t, 13.85.Ni, 13.87.Ce}


\maketitle
\section{Introduction}

High-$p_T$ parton scattering leads to 
formation of four cones of gluon radiation:
(i)-(ii) backward-forward jets formed by the color field of the colliding partons 
shaken off in in the hard collision;
(iii)-(iv) the scattered partons carry
 no field up to transverse momenta $k_T<p_T$.
These partons 
are regenerating the lost color field via gluon radiation 
forming the up-down jets, as is illustrated in Fig.~\ref{fig1}
%\setcounter{figure}{-1}
% Figure environment removed   
%%%%%%%%%%%%%%%%%%%%%%%
The radiation process is ordered in time or path length according to \cite{lp},
\beq
l_c=\frac{2Ex(1-x)}{k_T^2+x^2m_q^2}\ .
\label{lc}
\eeq
Here $x$ is the fractional light-cone momentum of the radiated gluon; $k_T$ is its transverse momentum relative to the initial quark direction. The radiated gluons subsequently hadronize forming a jet of hadrons. For heavy quarks the second term in the denominator play important role leading to the so called dead-cone effect \cite{troyan}.

In terms of the Fock state representation all radiated
gluons pre-exist in the initial
bare parton, and are liberated  on mass shell successively in 
accordance with their coherence length/time Eq.~(\ref{lc}).
First  are radiated gluons with small longitudinal and large transverse momenta.

\subsection{Radiational energy loss in vacuum}

How much energy is radiated over the path length L? 
Only gluons with radiation length $l_c<L$ contribute \cite{similar},
\beq
\Delta E_{rad}(L) =
\int\limits_{\lambda^2}^{Q^2}
dk_T^2\int\limits_0^1 dx\,\omega\,
\frac{dn_g}{dx\,dk_T^2}\,
\Theta(L-l_c),
\label{130}
\eeq
 where $\omega$ is the gluon energy; the soft cut-off parameter $\lambda=0.2\GeV$.
The perturbative radiation spectrum reads,
\beq
\frac{dn_g}{dx\,dk_T^2} =
\frac{2\alpha_s(k_T^2)}{3\pi\,x}\,
\frac{k_T^2[1+(1-x)^2]}{[k_T^2+x^2m_Q^2]^2}\,.
\label{145}
\eeq
We see that radiation by light and heavy quarks behave quite differently at small $k_T$:\\
(i) Light quarks: $dn_g/dk_T^2\propto 1/k_T^2$\\
(ii) Heavy quarks: $dn_g/dk_T^2\propto k_T^2/m_Q^2$

Dead-cone effect: gluons with $k_T^2<x^2m_Q^2$ 
are suppressed \cite{troyan,similar}. Heavy quarks radiate less energy
compared with the light ones. They promptly restore their color
field and  stop radiating. The amount of radiated energy for light and heavy flavors is depicted in Fig.~\ref{dE-E} vs radiation length for different jet energies.
%%%%%%%%%%%%%
% Figure environment removed   
We see that heavy quarks radiate only a small fraction
10-20\% of their initial momentum. In particular, this explains the unusual shape of the
experimentally observed fragmentation function $D_{b/B}(z)$ of b-quarks, presented in Fig.~\ref{ff} \cite{bottom} (and similar for charm \cite{charm}).
%%%%%%%%%%%%%
% Figure environment removed   
%%%%%%%%%%%%
Indeed, most of $B$-mesons carry a large fraction 
$z\sim 80\%$, of the $b$-quark momentum.

We conclude that such a specific shape of the fragmentation
function of heavy quarks is a direct manifestation of the dead-cone effect.

\section{Production length}

The process of gluon radiation by a heavy quark $Q$ ends up
with color neutralization by a light antiquark and
production of a $Q\bar q$ dipole.
As far as we are able to calculate the radiated fraction of the light-cone momentum (e.g. for $b$-quark) $\Delta p_+^b(L)/p_+^b$, the production length $L_p$ distribution $W(L_p)$ can be extracted directly from data on $D_{b/B}(z)$,
\beq
\frac{dW}{dL_p}=
\left.\frac{\partial \Delta p_+^b(L)/p_+^b}{\partial L}\right|_{L=L_p}
\!D_{b/B}(z)\, ,
\label{155}
\eeq
The results for the differential distribution $dW/dL_p$ are depicted in Fig.~\ref{Log} at several values of momenta $p_T$.
%%%%%%%%%%%%%
% Figure environment removed   
%%%%%%%%%%%%

Remarkably, the mean value of $L_p$ is extremely short
and shrinks with rising $p_T$. This sounds counter-intuitive,
however, the process has maximal hard scale allowed by the kinematics $p_T=E_{c.m.}/2$.

The production length $L_p$ turns out to
be much shorter than the confinement radius,
indicating that the fragmentation mechanism is pure perturbative.
At $L=L_p$, a small-size dipole $b\bar q$ is produced, with no certain mass, but with a certain radius. It is to be projected on the $B$-meson wave function, giving $\Psi_B(0)$ (compare with \cite{berger}).

\section{Fragmentation in a dense medium}

\subsection{Formation length of a $Q\bar q$ meson}

The light antiquark in the B-meson carries a tiny fraction of its momentum, $x\approx m_q/m_Q$, i.e. about 5\%.
The produced $b\bar q$ dipole has a small transverse separation, but it expands with a high speed, enhanced by 1/x, i.e. is an order of magnitude faster than symmetric $\bar qq$ or $\bar QQ$ dipoles. 
\beq
l_f\sim {1\over2}x(1-x)\la r_T^2\ra p_T,
\label{lf}
\eeq
where $\la r_T^2\ra = {8/3}\la r_{ch}^2\ra$, and $\la r_{ch}^2\ra_B=0.378\fm^2$ as was evaluated in the potential model \cite{radius}. The $B$ meson is nearly as big as the pion, since its radius is controlled by the mass of the light antiquark.

According to (\ref{lf}) the dipole heavy-light $Q\bar q$ dipole separation promptly reaches the large hadronic size. This is confirmed by comparison data, for $J/\psi$ detected in $Pb-Pb$ nuclear collisions.
Data demonstrate a color opacity for $B$-mesons (prompt production) 
and color transparency effect for $J/\psi$ decaying to $B$ (non-prompt production). The nuclear suppression factors $R_{AA}$ for these two channels are compared in Fig.~\ref{ATLAS} \cite{atlas-b}.
%%%%%%%%%%%%%
% Figure environment removed   
%%%%%%%%%%%%
 

While Eq.~(\ref{lf}) describes the early, perturbative stage of the dipole expansion,
the further evolution filters out the states with large relative phase shifts.
The longest time takes discrimination between the two lightest hadrons,
the ground state $B$ and the first radial excitation $B^\prime$, which concludes the
formation process. Correspondingly, the full formation path length can be evaluated as,
\beq
l_f=\frac{2p_T}{m_{B^\prime}^2-m_B^2}.
\label{Lf}
\eeq
E.g. for the oscillatory potential $m_{B^\prime}-m_B=0.6\GeV$, so $l_f=0.06\fm[p_T/1\GeV]$ is extremely short for medium-large transverse momenta.
 
 \subsection{Attenuation of dipoles propagating in a dense medium}
 
 The mean free path of a $Q-\bar q$ meson in a hot medium characterizing by the transport coefficient (the rate of broadening) $\hat q$,
 \beq
 \lambda_{Q\bar q}\sim\frac{1}{\hat q\la r_T^2\ra} = \frac{3}{8\hat q\la r_{ch}^2\ra_{Q\bar q}}.
 \label{lambda}
 \eeq
 E.g. at $\hat q=1\GeV^2/\fm$ $\lambda_B=0.04\fm$, so a formed $B$-meson breaks up in the medium nearly instantaneously.
 
 A $b$-quark propagating through the hot medium, easily picks up and loses accompanying light antiquarks without an essential reduction of its momentum. Meanwhile the $b$-quark keeps dissipating its energy with a rate, slightly enhanced by medium induced radiative energy loss \cite{dk} 
effects. Eventually the detected $B$-meson is produced in the dilute periphery of the medium.

The heavy quark keeps losing energy even inside a colorless $Q\bar q$   dipole sharing its momentum with the light quark, as is illustrated in Fig.~\ref{regge-cut} presenting a unitarity cut of a $\bar qq$ Reggeon,
 %%%%%%%%%%%%%
% Figure environment removed   
%%%%%%%%%%%%
Thus, the heavy quark $Q$ dissipates a part of its energy on a long path from the hard collision point to the medium periphery.
\beq
\frac{dE}{dL}=\frac{dE_{rad}}{dL}-\kappa(T),
\label{e-loss}
\eeq
where $\kappa(T)$ is temperature dependent string tension in the medium \cite{string}
$\kappa(T)=\kappa_0(1-T/T_c)^{1/3}$; the vacuum string tension $\kappa_0=1\GeV/\fm$; The critical temperature is fixed at $T_c=200\MeV$.

\subsection{Medium modified production rate}

The cross section of a heavy-light meson $M$ production in $pp$ collisions can be presented in the factorized form,
\beq
\frac{d^2\sigma_{pp\to M}}{d^2p_T}=\frac{1}{2\pi p_T E_T}
\int d^2 q_T\,\frac{d^2\sigma_{pp\to Q}}{d^2q_T}
\int\limits_0^\infty dL_p \frac{dW}{dL_p}\,\frac{\Delta E(L_p)}{E}\,
\delta\left(1-z-\frac{\Delta E(L_p)}{E}\right)
%z\,D_{Q/M}(z),
\label{890}
\eeq
We replaced the $b\to B$ fragmentation function by the differential expression (\ref{155}).
The medium-modified $L_p$ distribution is given by,
\beq
\frac{dW^{AA}}{dL_p}={1\over2}\la r_B^2\ra \hat q(L_p)\,
\exp\left[-{1\over2}\la r_B^2\ra\int\limits_{L_p}^\infty dL\,\hat q(L)\right]
\label{WAA}
\eeq
Here, for the sake of simplicity, we fixed the $Q\bar q$ dipole separation at the mean value. This approximation is rather accurate due to shortness of $l_f$. Otherwise, one can calculate the attenuation factor in (\ref{WAA}) exactly, applying the path integral technique \cite{kst1,kst2}.

Eventually, the production rate of heavy-light mesons in $AA$ collisions with impact parameter $\vec s$ reads,
\beq
\int d^2 q_T\,\frac{d^2\sigma_{pp\to Q}}{d^2q_T}
\int d^2\tau\, T_A(s)T_A(\vec s-\vec\tau)
\int\limits_0^\infty dL_p \frac{dW^{AA}}{dL_p}\,\frac{\Delta E(L_p)}{E}\,
\delta\left(1-z-\frac{\Delta E(L_p)}{E}\right)
\label{sigmaAA}
\eeq

The effective production length $\tilde L_p$ in the medium turns out to be much longer than in vacuum, because the heavy-light meson is produced mainly at the medium periphery, long distance from the hard collision point.

\subsection{Data analysis}

Now we are in a position to calculate the nuclear ratio
\beq
R_{AA}(\vec{s},\vec p_T) 
=
\frac{d^2\sigma_{AA}(s)/d^2p_Td^2s}
{T_{AA}(s)\,d^2\sigma_{pp}/d^2p_T},
\label{850}
\eeq
to be compared with data. Here
\beq 
T_{AA}(s)=\int d^2\tau T_A(\tau) T_A(\vec s-\vec\tau),
\label{TAA}
\eeq
and $T_A(s)$ is the nuclear thickness function.


The model cannot fully predict (as well as any other model) the nuclear ratio, because the medium density is not known, but is rather the goal of the research. We embedded this information into the broadening rate (transport coefficient) following the popular model \cite{wang}
 \beq
\hat q(l,\vec s,\vec\tau,\phi)=\frac{\hat
q_0\,t_0}{t}\, \frac{n_{part}(\vec s,\vec\tau + l\,\vec p_T/p_T)}{n_{part}(0,0)}
\,\Theta(t-t_0)\, ,
\label{900}
\eeq
%
where 
 $n_{part}(\vec s,\vec\tau)$ is the number of participants at transverse coordinates
 $\vec s$ and $\vec \tau$ relative to the centers of the colliding nuclei.
 The falling time dependence, $1/t$ is due to longitudinal expansion of the produced medium. 
 The time interval $t_0$ required for equilibrated medium production. We fixed it at the frequently used value $t_0=1\fm$.

The only fitted parameter is $\hat q_0$, which is
 the maximal value of the brodening rate (transport coefficient) at $s=\tau=0$ and $t=t_0$. In fact, measurement of this parameter is our goal.
Comparison with ATLAS \cite{atlas-b} and CMS data \cite{Sirunyan:2017oug}  for $B$-meson production (non-prompt $J/\psi$) in lead-lead collisions at $\sqrt{s}=5.02\TeV$ is presented in Fig.~\ref{Bmeson}.
 %%%%%%%%%%%%%
% Figure environment removed   
%%%%%%%%%%%%
We see that data are described pretty well, either for $p_T$, or  $N_{part}$ dependences.
The adjusted parameter $\hat q_0$ ranges within $\hat q_0= 0.2-0.25\GeV^2/\fm$.
This magnitude is considerably smaller compared with the values usually measured for light quarks. See discussion below.

We successfully described data on $D$-meson production as well, as is demonstrated in 
Fig.~\ref{Dmeson}.
%%%%%%%%%%%%%
% Figure environment removed   
%%%%%%%%%%%%
Notice that c-quarks radiate in vacuum more energy than b-quarks, while the effects of absorption of $c\bar q$ and $b\bar q$
dipoles in the medium are similar. Therefore $D$-mesons are suppressed in $AA$ collisions more than $B$-mesons. $R_{AA}(p_T)$ for $D$-mesons steeply rises with $p_T$ due to color transparency. Since $b\bar q$ dipoles expand much faster than $c\bar q$, no color transparency effects are seen in $R_{AA}(p_T)$ for $B$-mesons, as was demonstrated in the right pane of Fig.~\ref{ATLAS}.

Interesting that the found broadening rate parameter for $c$-quarks $\hat q_0= 0.45-0.55\GeV^2/\fm$, significantly exceeds the value  $\hat q_0= 0.2-0.25\GeV^2/\fm$ we found for $b$-quarks, while is quite less than  $\hat q_0\approx 2\GeV^2/\fm$ for light quarks (see below). Such a hierarchy of broadening rates for different quark flavors might look puzzling, if $\hat q$ were a real transport coefficient in terms of statistical medium properties. It coincides with the rate of broadening \cite{bdmps} only within the Born approximation, i.e. single gluon exchange for an inelastic process. In reality, broadening is subject to strong higher-order corrections and usually considerably exceeds the Born approximation estimate. The rate of broadening reads \cite{jkt,mutual},
\beq
\hat q=\frac{2\pi^2}{3}\,\alpha_s(\mu^2)\,xg(x,\mu^2)\,\rho_2,
\label{1057}
\eeq
where $g(x,\mu^2)$ is the gluon density; $\rho_2$ is the medium density per unit of length.
The characteristic scale of the process $\mu$ is related to the mean transverse momentum of the radiated gluons. For light quarks it is given by the non-perturbative effective gluon mass, 
$m_g\sim 0.7\GeV$ \cite{kst2,spots}. For heavy quarks gluon radiation is subject to the dead-cone effect and the scale is much larger $\mu^2\approx m_Q^2$. This is why the rate of broadening for heavy quarks is significantly reduced. This is another manifestation of the dead-cone effect.

The left plot in Fig.~\ref{Dmeson} shows a considerable disagreement with data at small transverse momenta $p_T\lesssim 10\GeV$. While the measured $R_{AA}(p_T)$ is steeply falling with $p_T$,  our calculations predict a nearly constant value. Such kind of disagreement has been observed earlier for light quarks, as is displayed in Fig.~\ref{Hydro}
%%%%%%%%%%%%%
% Figure environment removed   
%%%%%%%%%%%%
Apparently a bump at small $p_T$ is presented in $R_{AA}(p_T)$ for $D$-mesons as well, while our calculations in Fig.~\ref{Dmeson} disregard the hydrodynamic component.

\section{Summary}

Our observations and results can be summarized as follows.
\begin{itemize}
\item
Heavy and light quarks originated from hard collisions radiate differently.
The former is subject to the dead-cone effect, suppressing radiation of low-$k_T$ gluons.
Consequently heavy quarks regenerate their color field much faster than light ones
and radiate a significantly smaller fraction of the initial energy. The heavier is a quark, the less it radiates.
\item
The fragmentation function usually depends on two variables $D_{M/q}(z,Q^2)$, fractional light-cone momentum of produced meson, and the scale $Q^2$. However, we consider here the case of "maximal" scale, when the jet energy and the hard scale coincide. This happens e.g. in $e^+e^-$ annihilation, or high-$p_T$ jet production at Feynman $x_F=0$. 

\item
The dead-cone effect suppressing bremsstrahlung of heavy quarks, 
explains the unusual shape of the fragmentation function of heavy quarks $D_{M/Q}(z)$,
observed at LEP and SLAC. It peaks at large fractional momentum $z$, i.e. the produced heavy-light mesons, $B$ or $D$, carry the main fraction of the jet momentum. 
On the contrary, the fragmentation function of light quarks is falling steadily with $z$ towards $z=1$.

\item
Differently from propagation of a small $q-\bar q$ dipole, which survives in the medium due to color
transparency, a $Q-\bar q$ dipole promptly expands to a large transverse size, controlled by  the small mass of the light quark. Such a big dipole has no
chance to remain intact in a hot medium. On the other hand, a breakup of such a dipole
hardly affects the production rate of $Q-\bar q$ mesons.

\item
We successfully described data on $p_T$ and centrality dependence of the production rate of $B$ and $D$ mesons in heavy ion collisions. The only unavoidable parameter of such analyses  is the broadening rate (usually called transport coefficient) of the quark  in the medium. Its maximal value $\hat q_0$ was found $0.2-0.25\GeV^2/fm$, $0.4-0.45\GeV^2/fm$ and $2\GeV^2/fm$ for $b$, $c$ and light quarks respectively. Such hierarchy of the broadening rates is related to the same dead-cone effect.Suppression of bremsstrahlung leads to a considerable reduction of broadening.
\end{itemize}

\acknowledgments{
%\begin{acknowledgments}
 This work was supported in part by grants ANID - Chile FONDECYT 1231062 and 1230391,  by  ANID PIA/APOYO AFB220004, and by ANID - Millennium Science  Initiative Program 
ICN2019\_044.
%
 The work of J.N. was partially supported by
Grant No. LTT18002 of the Ministry of Education, Youth and Sports of
the Czech Republic, by the project of the European Regional Development
Fund No. CZ.02.1.01/0.0/0.0/16\_019/0000778 and by the Slovak Funding
Agency, Grant No. 2/0020/22.
%\end{acknowledgments}
}

%\reftitle{References}


\begin{thebibliography}{999}

\bibitem{lp} 
Landau, L.D.; Pomeranchuk, I. Limits of applicability of the theory of bremsstrahlung electrons and pair production at high-energies,
{\em Dokl. Akad. Nauk Ser. Fiz.}, \textbf{1953},  92, 535.

\bibitem{troyan} 
Dokshitzer, Y.L.; Khoze, V.A.; Troian, S.I.
 On specific QCD properties of heavy quark fragmentation ('dead cone'). 
   {\em J.\ Phys.\ G} \textbf{1991}, 17, 1602-1604,
  doi:10.1088/0954-3899/17/10/023.

\bibitem{similar}
Kopeliovich, B.Z.; Potashnikova, I.K.; Schmidt, I.
Why heavy and light quarks radiate energy with similar rates.
{\em Phys. Rev. C} \textbf{2010}, 82, 037901-037904,
doi:10.1103/PhysRevC.82.037901

\bibitem{bottom}  
Kniehl, B.A.; Kramer, G; Schienbein, I.; Spiesberger, H.
Finite-mass effects on inclusive $B$ meson hadroproduction.
{\em Phys. Rev. D} \textbf{2008}, 77, 014011-014038,
doi:10.1103/PhysRevD.77.014011.

\bibitem{charm} Kneesch, T.; Kniehl, B.A.; Kramer, G; Schienbein, I.
Charmed-meson fragmentation functions with finite-mass corrections,
{\em   Nucl. Phys. B} \textbf{2008}, 799, 34-69,
 doi:10.1016/j.nuclphysb.2008.02.015.

\bibitem{berger}
Kopeliovich, B.Z.; Pirner, H.-J.; Potashnikova, I.K.; Schmidt, I.; Tarasov A.V.
Perturbative fragmentation,
{\em Phys. Rev. D} \textbf{2008}, 77, 054004-054014,
doi: 10.48550/arXiv.0801.0251.

\bibitem{radius} 
Hwang, C.W.
  Charge radii of light and heavy mesons,
{\em   Eur. Phys. J. C} \textbf{2002}, 23, 585-596,
  doi:10.1007/s100520200904.

 \bibitem{atlas-b} Aaboud, M.; Aad, G.; Abbott, B.; et al. Prompt and non-prompt $J/\psi$ and $\psi(2S)$ suppression at high transverse momentum in 5.02 TeV Pb+Pb collisions with the ATLAS experiment,
{\em  Eur. Phys. J. C} \textbf{2018}, 78, 762-797,
doi.org/10.1140/epjc/s10052-018-6219-9.

\bibitem{dk} 
Dokshitzer, Y.L.; Kharzeev, D.E.
 Heavy quark colorimetry of QCD matter,
{\em   Phys. Lett. B} \textbf{2001}, 519, 199-214,
  doi:10.1016/S0370-2693(01)01130-3.
  
\bibitem{string} 
Toki, H.; Sasaki, S.; Ichie, H.; Suganuma, H.
Chiral symmetry breaking in the dual Ginzburg-Landau theory,
{\em  Austral. J. Phys.} \textbf{1997}, 50, 199-206,
doi:10.1071/P96051.  
  
\bibitem{kst1}  
Kopeliovich, B.Z.; Sch\"afer, A.; Tarasov, A.V.
Bremsstrahlung of a quark propagating through a nucleus,
{\em Phys. Rev. C} \textbf{1999}, 59, 1609-1619,
doi: 10.1103/PhysRevC.59.1609.
  
 \bibitem{kst2} 
 Kopeliovich, B.Z.; Sch\"afer, A.; Tarasov, A.V.
 Nonperturbative effects in gluon radiation and photo-production of quark pairs,
 {\em  Phys. Rev. D} \textbf{2000}, 62, 054022-054080,
 doi: 10.1103/PhysRevD.62.054022.
  
\bibitem{wang} 
Chen, X.F.; Greiner, C.; Wang, E.; Wang, X.N.; Xu, Z.
Bulk matter evolution and extraction of jet 
     transport parameter in heavy-ion collisions at RHIC,'
{\em   Phys. Rev. C} \textbf{2010}, 81, 064908-064921,
  doi:10.1103/PhysRevC.81.064908.

\bibitem{Sirunyan:2017oug}
Sirunyan, A.M.; et al.
Measurement of the ${B}^{\pm}$ Meson Nuclear Modification Factor in Pb-Pb Collisions at $\sqrt{{s}_{NN}}=5.02\TeV$,
{\em Phys. Rev. Lett.} \textbf{2017}, 119, no.15, 152301-152305,
doi:10.1103/PhysRevLett.119.152301.

\bibitem{D-alice} 
Acharya, S.; et al.
Measurement of $D^0$, $D^+$,
     $D^{*+}$ and $D_{s}^+$ production in $Pb$-$Pb$ collisions at
       $\sqrt{s_{NN}}= 5.02\TeV$,
{\em    JHEP} \textbf{2018}, 1810, 174-209,
doi:10.1007/JHEP10(2018)174.

\bibitem{cms-D-pp}
Sirunyan, A.M.; et al.
Nuclear modification factor of $D^0$ mesons in PbPb collisions at  $\sqrt{s_{NN}} = 5.02\TeV$,
{\em   Phys.\ Lett.\ B} \textbf{2018}, 782, 474-496,
doi:10.1016/j.physletb.2018.05.074.

\bibitem{bdmps} 
 Baier, R.; Dokshitzer, Y.L.; Mueller, A.H.; Peigne, S.; Schiff, D.
Radiative energy loss and p(T) broadening of high-energy partons in nuclei,
{\em  Nucl.\ Phys.\ B} \textbf{1997}, 484, 265-283,
doi:10.1016/S0550-3213(96)00581-0.

\bibitem{mutual}
Kopeliovich, B.Z.; Pirner, H.-J.; Potashnikova, I.K.; Schmidt, I.
Mutual boosting of the saturation scales in colliding nuclei,
{\em Phys. Lett. B}, \textbf{2011}, 697, 333-344,
doi:10.48550/arXiv.1007.1913.

\bibitem{jkt} 
Johnson, M.B.; Kopeliovich, B.Z.; Tarasov, A.V.
 Broadening of transverse momentum of partons propagating through a medium,
{\em   Phys.\ Rev.\ C} \textbf{2001}, 63, 035203-035230,
 doi:10.1103/PhysRevC.63.035203.

 \bibitem{spots}
 Kopeliovich, B.Z.; Potashnikova, I.K.; Povh, B.; Schmidt, I.
 Evidences for two scales in hadrons,
{\em    Phys.\ Rev.\  D} \textbf{2007}, 76, 094020-094035,
doi: 10.1103/PhysRevD.76.094020.

\bibitem{pert}  
Kopeliovich, B.Z.; Nemchik, J; Potashnikova, I.K.; Schmidt, I.
Quenching of high-pT hadrons: Energy Loss vs Color Transparency,
{\em Phys. Rev. C} \textbf{2012}, 86, 054904-054921,
doi: 10.48550/arXiv.1208.4951.

\bibitem{sinyukov}
Nemchik, J.; Karpenko, I.A.; Kopeliovich, B.Z.; Potashnikova, I.K.; Sinyukov, Y.M.,
High-pT hadrons from nuclear collisions: Unifying pQCD with hydrodynamics,
\textbf{2013}, 
Proceedings, 15th conference on Elastic and Diffractive scattering (EDS Blois 2013), 9-13 September 2013. Saariselka, Lapland, Finland
[arXiv:1310.3455].

\bibitem{alice1} Abelev, B.; et al. 
Centrality dependence of charged particle production at large transverse momentum in $Pb-Pb$ collisions at $\sqrt{s}=2.76\TeV$,
{\em Phys. Lett. B} \textbf{2013}, 720, 52-62,
doi: 10.1016/j.physletb.2013.01.051.

\bibitem{cms1} Lee, Y.-J.; (for the CMS Collaboration), 
Quarkonia Measurements by the CMS Experiment in pp and Pb-Pb Collisions,
{\em J. Phys. G} \textbf{2011}, 38, 124015-124023, 
doi:10.1088/0954-3899/38/12/124015.

\bibitem{cms2} Yoon, A.S.; (for the CMS Collaboration), 
Centrality and pT dependence of charged particle $R_{AA}$ in Pb-Pb collisions at $\sqrt{s}=2.76\TeV$,
{\em J. Phys. G} \textbf{2011}, 38, 124116-124122,
doi: 10.1088/0954-3899/38/12/124116.



\end{thebibliography}



\end{document}

\vspace*{-0.25cm}


%=============================================================
\section{Introduction}
\label{Sec:intro}
%=============================================================
%
% 
%


Ultra-peripheral collisions (UPC) of nuclei at high energies correspond to collisions with impact parameter much larger than the sum of nuclear radii. Correspondingly, the total cross section of UPC is much larger than that of collisions with nuclear overlap. Besides, UPC is the dominant source of heavy vector mesons. Collisions with double rapidity gaps can be interpreted as photon-Pomeron fusion into a heavy vector meson $V=\Jpsi(1S)$, $\Y(1S)$, or their radial excitations.
In impact parameter space the Pomeron is short range exchange, while the radius of photon exchange is infinitely large.


UPC provide a unique access to photon-nucleus interactions. Weizs\"acker-Williams photons, originated from one of the nuclei, can interact diffractively with another one. The mechanisms of photo-production of vector mesons on nuclei have been studied pretty well beyond the Glauber approximation, either in the multi-channel approach \cite{Hufner:1997jg}, or employing the color dipole model \cite{Kopeliovich:1991pu} .
Having a comprehensive understanding of photo-production mechanisms, one can switch to description of nuclear UPC, as was done in Ref.~\cite{Ivanov:2007ms}.


Within the popular light-front (LF) color dipole approach
\cite{Kopeliovich:1991pu,Kopeliovich:1993gk,Nemchik:1996pp,Kopeliovich:1993pw,Nemchik:1994fp,Nemchik:1994fq,Nemchik:1996cw,Hufner:2000jb,Nemchik:2000de,Nemchik:2000dd,Kopeliovich:2001xj,Kopeliovich:2007wx}
%%%%%%%%%%%%%%%%%%%%%%%%%%%%%%%%%%%%%%%%%%%%%%%%%%%%%%%%%%%%%%%%%%%%%%%%%%%%%%%%%%%%%%%%%%%%%%%%%%%%%%%
{\color{red}
(for recent studies of quarkonium photo- and electro-production, see also Refs.~\cite{Lappi:2020ufv,Mantysaari:2021ryb,Sambasivam:2019gdd,Goncalves:2020vdp}),
}
%%%%%%%%%%%%%%%%%%%%%%%%%%%%%%%%%%%%%%%%%%%%%%%%%%%%%%%%%%%%%%%%%%%%%%%%%%%%%%%%%%%%%%%%%%%%%%%%%%%%%%%
the effect, known as color transparency (CT), significantly affects heavy quarkonium photo-production off nuclei.
It is controlled by the {\it formation length} $l_f$, characterizing evolution of the $Q\bar Q$ pair, evolving from the small initial size $\sim1/m_Q$, where $m_Q$ is the heavy quark mass, up to a larger non-perturbative size of the quarkonium.
The corresponding expression for $l_f$
can be obtained in the nuclear rest frame from the condition that the relative phase shift between the two lowest levels, $V$ and $V^\prime$ becomes of the order of unity \cite{Kopeliovich:1991pu,Kopeliovich:2001xj}, 
%
%======================================================
 \BE
l_f = \frac{2\,k}
{M_{V^\prime}^2 - M_V^2}\,
%-----------
\label{tf}
%----------
 \EE
%======================================================
%
where $k$ is the photon energy
and $M_V$ and $M_{V^\prime}$ are the quarkonium masses in $1S$ and $2S$ states, respectively.


Another important length scale, called {\it coherence length} (CL) \cite{Kopeliovich:1991pu,Kopeliovich:2001xj}, characterizes the phase shift between $Q\bar Q$ photo-production  amplitudes with different longitudinal coordinates of interaction.
It has the following form,
%
%=========================
 \BE
l_c = \frac{2\,k}{M_V^2} ,
%----------
\label{lc}
%----------
 \EE
%==============================%
The amplitudes are coherent, provided that the length interval $\Delta l\ll l_c$.
The coherence length $l_c$ can also be interpreted as a lifetime (or path) of the $Q\bar Q$ fluctuation of the photon.


In our calculations of nuclear effects we rely, for the sake of simplicity, on the eikonal approximation for in-medium propagation of long-lived $Q\bar Q$ photon fluctuations, which is relevant at rapidity $y=0$.
However, at forward or backward rapidities at the Large Hadron Collider (LHC),
as well as at Relativistic Heavy Ion Collider (RHIC),
such an approximation cannot be applied anymore because the coherence length Eq.~(\ref{lc}) becomes too short at least for one of the colliding nuclei. Here, for the first time, we apply the rigorous path-integral technique
as is described in Sect.~\ref{fcor}. 


Another source of nuclear suppression, called {\it gluon shadowing} (GS), in terms of parton model looks like reduction of gluon density in nuclei at small Bjorken $x$. In the infinite momentum frame of the nucleus this occurs due to longitudinal overlap and fusion of gluons, originated from different bound nucleons. This effect is difficult to evaluate, and usually extracted from the global fits to data. The result is unreliable and is known only for the density integrated over impact parameter.

The parton model description is not Lorentz invariant (only observables are), it depends on reference frame. What looks like gluon fusion in the infinite momentum frame of the nucleus, corresponds to usual Glauber-like shadowing of the photon fluctuations in the nuclear rest frame. Namely, gluon reduction corresponds to shadowing of higher Fock components of the photon, which contain one or more gluons (corrected for the effect for the lowest Fock state $|Q\bar Q\ra$) \cite{Kopeliovich:2022jwe}. The transverse size of such fluctuations depends on the hard scale ($m_Q$) logarithmically, so is the leading twist effect.


Notice that onset of gluon shadowing requires much smaller $x$ in comparison with the $Q\bar Q$ component. That happens due to a specifically short coherence length for higher Fock states, $l_c^G\ll l_c$. They differ by an order of magnitude \cite{Kopeliovich:2000ra,Kopeliovich:2022jwe}.
Since the transverse size of $Q\bar Q-G$ dipoles fluctuates during propagation through the nucleus,
even at very high energies of the LHC, one cannot rely on the "frozen" eikonal approximation,
$l_c^G\gg R_A$, where $R_A$ is the nuclear radius.
For this reason, our calculations of the GS effect are performed relying
on the Green function formalism \cite{Kopeliovich:1999am,Ivanov:2002kc,Kopeliovich:2022jwe}.


In the present paper we incorporate various improvements in theoretical description of quarkonium photo-production off nuclei, performed in Ref.~\cite{Kopeliovich:2022jwe}, and apply them also to UPC at RHIC and the LHC. We include  
the higher- and leading-twist shadowing corrections corresponding to the $|Q\bar Q\ra$ and
$|Q\bar QG\ra$ Fock state of the photon, respectively. 
The multi-gluon Fock components have too short coherence length to generate a significant shadowing effects even at very high energies of the LHC.
Moreover, as is discussed in Ref.~\cite{Kopeliovich:2022jwe} (see also Refs.~\cite{Krelina:2018hmt,Cepila:2019skb,Krelina:2019egg,Krelina:2020bxt}), we ignore the frequently used unjustified model of the photon-like $V\to Q\bar Q$ transition,
which would lead to an exaggerated weight of the $D$-wave in the rest frame quarkonium wave function.


The paper is organized as follows. We present expressions for calculations of differential cross sections $d\sigma/dy$ corresponding to coherent (elastic) and incoherent (quasi-elastic) heavy quarkonium production in UPC  in Sections~\ref{sub-coh} and \ref{sub-inc}, respectively. 
%
Incorporation of a small real part of the production amplitude, as well as spin rotation effects
is explained in Sec.~\ref{sub-sr}.
%
In the following Section~\ref{fcor} we discuss the two main effects, which influence on 
nuclear effects in UPC: 
(i) the corrections for a finite coherence length and 
(ii) gluon shadowing. 
The former effect is calculated for the first time within the LF dipole approach based on the Green function formalism, leading to the results, that are substantially different from the standard vector dominance model (VDM). 
%
The following Sec.~\ref{res} is devoted to comparison of model predictions with available data and to the analysis of particular nuclear effects in coherent and incoherent quarkonium production in UPC. 
%
Finally, the last Sec.~\ref{conclusions} contains a summary and concluding remarks.


%%%%%%%%%%%%%%%%%%%%%%%%%%%%%%%%%%%%%%%%%%%%%%%%%%%%%%%%%%%%%%%%%%%%%
%%%%%%%%%%%%%%%%%%%%  S E C T I O N    II  %%%%%%%%%%%%%%%%%%%%%%%%%%
%%%%%%%%%%%%%%%%%%%%%%%%%%%%%%%%%%%%%%%%%%%%%%%%%%%%%%%%%%%%%%%%%%%%%
%
%
%
%============================================================
\section{Quarkonium production cross section in 
ultra-peripheral collisions}
\label{Sec:cross-sec}
%============================================================
%
%
%

The large charge $Z$ of colliding heavy nuclei gives rise to strong electromagnetic
fields: in a heavy-ion UPC, the photon field of one nucleus can 
produce a photo-nuclear reaction in the other. Then the cross section for photo-production of a vector meson $V$ by the Weizs\"acker-Williams photons
can be written in the rest
frame of the target nucleus $A$ as follows \cite{Bertulani:2005ru}:
%
%====================================================================
\BE
  k\frac{d\sigma}{dk} = \int\,d^2\tau \int\,d^2b \,\,
  n(k,\vec b-\vec\tau,y)\, 
  \frac{d^2
  \sigma_A(s,b)}{d^2b}
  ~~ + ~~
  \Bigl \{ y\rightarrow -y \Bigr \}
  \,,
%--------------
\label{cs-upc}
%--------------
\EE
%====================================================================
%
where the rapidity variable $y = \ln \bigl [s / (M_V \sqrt{s_N}~) \bigr] \approx \ln\bigl [(2 k M + M^2) / (M_V \sqrt{s_N}~)\bigr ]$.

The formula (\ref{cs-upc}) is derived in the one-photon-exchange approximation. Here
the variable $\vec\tau$ is the relative impact parameter of a nuclear collision, and $\vec{b}$ is the impact parameter of the photon-nucleon collision
relative to the center of one of the nuclei. 
%
Particularly, 
the collision of identical nuclei with the nuclear radius $R_A$ in UPC
leads to a condition that the impact parameter $\tau > 2 R_A$ \cite{Bertulani:2005ru}.


The variable
$n(k,\vec b)$ in Eq.~(\ref{cs-upc}) represents the photon flux induced by the
projectile nucleus with Lorenz factor $\gamma$ and has the following form, 
%
%=================================================================
\BE
  n(k,\vec b) = \frac{\aem Z^2 k^2}{\pi^2\gamma^2}
  \Biggl [
  K_1^2\left(\frac{bk}{\gamma}\right)
  +
  \frac{1}{\gamma^2} K_0^2\left(\frac{bk}{\gamma}\right)
  \Biggr ]
  \,,
%--------------
  \label{flux}
%--------------
\EE
%==================================================================
%
where $\aem= 1/137.036$ is the fine-structure constant,
$K_{0,1}$ are the modified Bessel functions of the second kind and the Lorentz factor $\gamma = 2 \gamma_{col}^2 - 1$ with $\gamma_{col} = \sqrt{s_N}/2 M$.
%
The first and the second
term in Eq.~(\ref{flux}) corresponds to the flux of photons transversely and 
longitudinally polarized to the ion direction, respectively. The former photon flux
dominates in ultra-relativistic collisions with $\gamma\gg 1$. Consequently, in heavy-ion UPC
at RHIC and the LHC one can safely neglect the second term in Eq.~(\ref{flux}) treating the photons as almost real due to very small virtuality, $-q^2 = Q^2 < 1/R_A^2$.



%%%%%%%%%%%%%%%%%%%%%%%%%%%%%%%%%%%%%%%%%%%%%%%%%%%%%%%%%%%%%%%%%%%%%
%%%%%%%%%%%%%%%%%%%%  S E C T I O N    II  A  %%%%%%%%%%%%%%%%%%%%%%%
%%%%%%%%%%%%%%%%%%%%%%%%%%%%%%%%%%%%%%%%%%%%%%%%%%%%%%%%%%%%%%%%%%%%%
%
%
%
%%%%%%%%%%%%%%%%%%%%%%%%%%%%%%%%%%%%%%%%%%%%%
\subsection{Coherent production of quarkonia}
\label{sub-coh}
%%%%%%%%%%%%%%%%%%%%%%%%%%%%%%%%%%%%%%%%%%%%%
%
%
%

To calculate the cross sections for coherent ($coh$) 
quarkonium production, $\gamma A\to V\!A$, we use the light-front dipole approach
\cite{Kopeliovich:1991pu}, which has been applied to describe $\Jpsi$ photo-production
off nucleons \cite{Hufner:2000jb,Krelina:2018hmt,Cepila:2019skb} and nuclei 
\cite{Ivanov:2002kc,Nemchik:2002ug}. In this approach, 
assuming sufficiently large photon energies, corresponding to most of the kinematic regions studied in the present paper when the CL (\ref{lc}) 
$l_c\gg R_A$,
the nuclear cross section takes a simple asymptotic form,


%
%============================================================================
 \BA
 \frac{d^2\sigma_{A}^{coh}(s,b)}{d^2b} \Biggr|_{l_c \gg R_A} 
&=&
\Biggl |
\int d^2r\int_0^1 d\alpha\,
\Psi_{V}^{*}(\vec r,\alpha)\,
\Biggl (1 -
\exp\left[-\frac{1}{2}\,\sqq(r,s)\,T_A(b)\right]
\Biggr )
\Psi_{\gamma}(\vec r,\alpha)
\Biggr |^2
\nonumber\\
&\equiv&
\Biggl |
\int d^2r\int_0^1 d\alpha\,
\Sigma_{A}^{coh}(r,\alpha,s,b)\,
\Biggr |^2\,.
%---------------
\label{coh-lcl}
%---------------
 \EA
%============================================================================
%
Here we rely on the optical approximation, assuming the elastic dipole amplitude pure imaginary, which is rather accurate for heavy nuclei.
Expression (\ref{coh-lcl}) is frequently called ``frozen'' approximation, assuming that the transverse separation of the $|Q\bar Q\ra$ Fock state of the photon does not change
during propagation through a nuclear medium. It represents the higher twist shadowing correction since the $Q-\bar Q$ transverse separation diminishes as $1/m_Q$.


In Eq.~(\ref{coh-lcl}) 
$T_A(b) = \int_{-\infty}^{\infty} dz\,\rho_A(b,z)$ is the nuclear thickness function normalized as $\int d^2 b\,T_A(b) = A$, 
where $\rho_A(b,z)$ is the nuclear density function, for which we employ the realistic Wood-Saxon form with parameters taken from \cite{DeJager:1987qc};
$\Psi_V(r,\alpha)$ is the LF wave function for heavy quarkonium;
%
$\Psi_{\gamma}(r,\alpha)$ is the LF distribution or the wave function 
of the $Q\bar Q$ Fock component of the quasi-real (transversely polarized) photon,
where the $Q\bar Q$ fluctuation (dipole) has the transverse size $\vec{r}$ and the variable $\alpha = p_Q^+/p_{\gamma}^+$ is the boost-invariant fraction of the photon momentum carried by a heavy quark (or antiquark).


The universal dipole-nucleon total cross section $\sqq(r,s)$ depends on transverse dipole separation $r$ and c.m. energy squared $s = M_V\,\sqrt{s_N}\,\exp[y]$. Energy dependence of the dipole cross section can be alternatively included also via variable $x = M_V^2/s = M_V\,\exp[-y]/\sqrt{s_N}$.

Notice that the coherent cross section, Eq.~(\ref{coh-lcl}), is different from the usual Glauber expression \cite{Bauer:1977iq} due to presence of the dipole cross section \cite{Kopeliovich:1981pz}. It effectively includes the 
Gribov inelastic shadowing corrections \cite{Gribov:1968jf,Kopeliovich:2016jjx} in all orders for the $Q\bar Q$ Fock component of the photon.



%%%%%%%%%%%%%%%%%%%%%%%%%%%%%%%%%%%%%%%%%%%%%%%%%%%%%%%%%%%%%%%%%%%%%
%%%%%%%%%%%%%%%%%%%%  S E C T I O N    II  B  %%%%%%%%%%%%%%%%%%%%%%%
%%%%%%%%%%%%%%%%%%%%%%%%%%%%%%%%%%%%%%%%%%%%%%%%%%%%%%%%%%%%%%%%%%%%%
%
%
%
%%%%%%%%%%%%%%%%%%%%%%%%%%%%%%%%%%%%%%%%%%%%%%%%
\subsection{Incoherent production of quarkonia}
\label{sub-inc}
%%%%%%%%%%%%%%%%%%%%%%%%%%%%%%%%%%%%%%%%%%%%%%%%
%
%
%



Besides ``elastic'' coherent photo-production $\gamma A\to V A$, where the nucleus remains intact, the vector meson can be produced in a quasi-elastic process $\gamma A\to V A^*$, where the nucleus is excited and decays to fragments. Important is that additional meson production is excluded. In this case, one can sum over different products of  nuclear excitation and employ the conditions of completeness. Of course, one channel of elastic photo-production must be subtracted. It is instructive to see the result within the Glauber approximation \cite{Hufner:1996dr},
%
%===================================================================================
 \BA
 \!\!\!\!\!
 \frac{d^2\sigma_A^{inc}(s,b)}{d^2b}
\Biggr|_{l_c \gg R_A}^{Gl} 
\propto
\exp\left[-\sigma^{VN}_{in}\,T_A(b)\right] -
\exp\left[-\sigma^{VN}_{tot}\,T_A(b)\right]
%\nonumber\\
%&= &
=
\exp\left[-\sigma^{VN}_{tot}\,T_A(b)\right]
\Bigl\{\exp\left[-\sigma^{VN}_{el}\,T_A(b)\right]-1 \Bigr\}\,.
%------------------
\label{glauber1}
%-----------------
\EA
%==================================================================================
%
Here the inelastic $V-N$ cross section $\sigma^{VN}_{in}=
\sigma^{VN}_{tot}-\sigma^{VN}_{el}$, where the elastic cross section
%
%======================================================================
\BA 
\sigma^{VN}_{el} \approx\frac{(\sigma^{VN}_{tot})^2}{16\,\pi\,B^{VN}}
%-----------------
\label{glauber2}
%----------------
\EA
%=====================================================================
%
and $B^{VN}$ is the slope of the differential elastic $V-N$ cross section.

The cross section of incoherent ($inc$) photo-production has the form, analogous to (\ref{glauber1}), but with additional integrations over the dipole size (see derivation in Sect.~VII of Ref.~\cite{Kopeliovich:2005us}),
%
%====================================================================================================================
 \BA
\frac{d^2\sigma_A^{inc}(s,b)}{d^2b}
\Bigr|_{l_c \gg R_A} 
&= &
\int d^2r_1\int_0^1 d\alpha_1\,
\Psi_{V}^{*}(\vec r_1,\alpha_1)\,\Psi_{\gamma}(\vec r_1,\alpha_1)\,\exp\left[-\frac{1}{2} \sqq(r_1,s) T_A(b)\right]
\nonumber\\
&\times &
\int d^2r_2\int_0^1 d\alpha_2\,
\Psi_{V}^{*}(\vec r_2,\alpha_2)\,\Psi_{\gamma}(\vec r_2,\alpha_2)\,\exp\left[-\frac{1}{2} \sqq(r_2,s)\, T_A(b)\right]
\nonumber\\
&\times &
\left\{\exp\left[\frac{\sqq(r_1,s)\,\sqq(r_2,s)}{16\pi B(s)}\,T_A(b)\right] - 1\right\}\,.
%-----------
\label{inc}
%-----------
\EA 
%===================================================================================================================
%
The elastic cross section of a heavy quarkonium on a nucleon is rather small and the exponential in the last row of Eq.~(\ref{inc}) can be expanded. Then we arrive at a simple result \cite{Ivanov:2002kc},
%
%===========================================================================
\BA 
\frac{d^2\sigma_A^{inc}(s,b)}{d^2b}
\Biggr|_{l_c \gg R_A} 
&\approx&
\frac{T_A(b)}{16 \pi B(s)}\,
\Biggl |
\int d^2r\int_0^1 d\alpha\,
\Psi_{V}^{*}(\vec r,\alpha)\,
\Psi_{\gamma}(\vec r,\alpha)\,
\sqq(r,s)\,
%\nonumber\\ &\times& 
\exp\left[-\frac{1}{2} \sqq(r,s)\, T_A(b)\right]\,
\Biggr |^2
%
\nonumber\\
&\equiv&
\frac{T_A(b)}{16 \pi B(s)}\,
\Biggl |
\int d^2r\int_0^1 d\alpha\,
\Sigma_{A}^{inc}(r,\alpha,s,b)\,
\Biggr |^2.
%---------------
\label{inc-lcl}
%---------------
 \EA
%===========================================================================
%


%%%%%%%%%%%%%%%%%%%%%%%%%%%%%%%%%%%%%%%%%%%%%%%%%%%%%%%%%%%%%%%%%%%%%
%%%%%%%%%%%%%%%%%%%%  S E C T I O N    II  C  %%%%%%%%%%%%%%%%%%%%%%%
%%%%%%%%%%%%%%%%%%%%%%%%%%%%%%%%%%%%%%%%%%%%%%%%%%%%%%%%%%%%%%%%%%%%%
%
%
%
%%%%%%%%%%%%%%%%%%%%%%%%%%%%%%%%%%%%%%%%%%%%%%%%%%%%%%%%%%%%%%%%%%%%%%%%%%%%
\subsection{Real part of the production amplitude and spin rotation effects}
\label{sub-sr}
%%%%%%%%%%%%%%%%%%%%%%%%%%%%%%%%%%%%%%%%%%%%%%%%%%%%%%%%%%%%%%%%%%%%%%%%%%%%
%
%
%



The both Eqs.~(\ref{coh-lcl}) and (\ref{inc-lcl}) contain a small correction due to 
the real part of the $\gamma N\to V N$ amplitude applying 
the following replacement \cite{Bronzan:1974jh,Nemchik:1996cw,Forshaw:2003ki},
%
%=======================================================
\BA
\sqq(r,s)
\Rightarrow
\sqq(r,s)
\,
\left(1 - i\,\frac{\pi}{2}\,\frac
{\partial
 \,\ln\,{\sqq(r,s)}}
{\partial\,\ln s} \right)\ .
%--------------
  \label{re/im}
%--------------
\EA
%========================================================
%
\\

The advantage of the $S$-wave heavy quarkonia, considered in the present paper, is based on a simple factorization of the radial and spin-dependent components of their wave functions. Here is well defined in the $Q\bar Q$ rest frame and can be obtained by solving the Schr\"odinger equation for various realistic interaction potentials between $Q$ and $\bar Q$ proposed in the literature. 
%
In our calculations, we choose two of them, the power-like (POW) \cite{Martin:1980jx,Barik:1980ai} and Buchm\"uller-Tye (BT) \cite{Buchmuller:1980su} potentials, which provide the best description of available data on charmonium electroproduction off protons, as was demonstrated in Ref.~\cite{Cepila:2019skb}.
%
For the dipole cross section, we employ three popular parametrizations - Kopeliovich-Sch\"afer-Tarasov (KST) from 
Ref.~\cite{Kopeliovich:1999am},  
Golec-Biernat-W\"usthoff (GBW) from
Refs.~\cite{GolecBiernat:1998js,GolecBiernat:1999qd} and 
Bartels-Golec-Biernat-Kowalski (BGBK) from Ref.~\cite{Bartels:2002cj}.

Treating the structure of the $V\to Q\bar Q$ vertex from Refs.~\cite{Ivanov:2002kc,Ivanov:2002eq,Ivanov:2007ms,Krelina:2018hmt,Cepila:2019skb,Kopeliovich:2022jwe}
the Melosh spin transformation \cite{Melosh:1974cu} is incorporated performing following substitutions in Eqs.~(\ref{coh-lcl}) and (\ref{inc-lcl}),
%
%========================================================================
\BA
\Sigma_{A}^{coh}(r,\alpha,s,b)
\Rightarrow
\Sigma_{A}^{coh}(r,\alpha,s,b)\cdot 
     \Bigl [\Sigma^{(1)}(r,\alpha) + \Sigma^{(2)}(r,\alpha) \Bigr ]
\nonumber\\
\Sigma_{A}^{inc}(r,\alpha,s,b)
\Rightarrow
\Sigma_{A}^{inc}(r,\alpha,s,b)\cdot
     \Bigl [\Sigma^{(1)}(r,\alpha) + \Sigma^{(2)}(r,\alpha) \Bigr ],
%------------------------
\label{sr-coh-inc}
%------------------------
\EA
%========================================================================
%
where
%
%=================================================================
\BA
  \Sigma^{(1,2)}(r,\alpha)
   &=&  
   N\,
   K_{0,1}(m_Q r) \int\limits_{0}^{\infty} dp_T\,p_T\,
   J_{0,1}(p_T r) \Psi_V (\alpha,p_T) \,
    \mathcal{R}^{(1,2)}(p_T)
%-------------------
\label{sigma-sr1}
%-------------------
\EA
%==================================================================
%
with
%
%===============================================================
\BA
\mathcal{R}^{(1)}(p_T) 
%&=& 
=
\frac{2\,m_Q^2(m_L+m_T)+m_L\,p_T^2}{ m_T (m_L + m_T)} \,,
%
\qquad\qquad\quad
%
\mathcal{R}^{(2)}(p_T) 
&=&
\frac{m_Q^2(m_L+2m_T)-m_T\,m_L^2}{m_Q\,m_T (m_L+m_T)}\, p_T\,.
%------------------
\label{sigma-sr2}
%------------------
\EA
%==============================================================
%
Here $N=Z_Q\,\sqrt{2 N_c \,\alpha_{em}}/4\,\pi$, where the factor $N_c=3$ represents the number of colors in QCD, $Z_Q=\frac{2}{3}$ and $\frac{1}{3}$  
are the charge-isospin factors for the production of charmonia and bottomonia, respectively, and
$J_{0,1}$ are the Bessel functions of the first kind.
The variables $m_{T,L}$ in the above formulas have the following form,
%
%------------------------------------------------------
\BE
m_T = \sqrt{m_Q^2 + p_T^2} \,, 
\qquad m_L = 2\, m_Q\,\sqrt{\alpha(1-\alpha)}\,.
\EE
%------------------------------------------------------
%

Note that the new form of Eqs.~(\ref{sr-coh-inc})-(\ref{sigma-sr2})
does not require performing the so-called {``\it resummation procedure''} proposed in \cite{Ivanov:2002kc} in order to include properly the spin rotation effects in 
nuclear photo-production cross sections (\ref{coh-lcl}) and (\ref{inc-lcl}).

%%%%%%%%%%%%%%%
%Fir 
%%%%%%%%%%%%%%%
For numerical calculations, following the results from Refs.~
\cite{Ivanov:2002kc,Ivanov:2002eq,Ivanov:2007ms,Krelina:2018hmt,Cepila:2019skb},
we relied on the charm and the bottom quark masses, corresponding to the values used with  the realistic phenomenological models for the $Q-\bar Q$ interaction potential, such as POW and BT.
Consequently, the LF quarkonium wave functions $\Psi_V(\alpha,r)$ have been obtained adopting the Lorentz boosting procedure as described in Ref.~\cite{Terentev:1976jk} and justified in Ref.~\cite{Kopeliovich:2015qna}.



%%%%%%%%%%%%%%%%%%%%%%%%%%%%%%%%%%%%%%%%%%%%%%%%%%%%%%%%%%%%%%%%%%%%%
%%%%%%%%%%%%%%%%%%%%  S E C T I O N    II  D  %%%%%%%%%%%%%%%%%%%%%%%
%%%%%%%%%%%%%%%%%%%%%%%%%%%%%%%%%%%%%%%%%%%%%%%%%%%%%%%%%%%%%%%%%%%%%
%
%
%
%=======================================================
\subsection{Corrections for a finite coherence length and the gluon shadowing}
\label{fcor}
%=======================================================
%
%
%

As was already mentioned above, the Green function approach allows to include directly the effects of quantum coherence without any restrictions for the magnitude of CL, Eq.~(\ref{lc}). However, as an alternative and a more simple way,
instead of such a complicated method, one can use expressions ({\ref{coh-lcl}) and (\ref{inc-lcl})) for nuclear cross sections in the limit of long CL, $l_c\gg R_A$, 
and then provide additional corrections for a finite CL when $l_c\lsim R_A$. 
Such an incorporation of finite-$l_c$ effects via the effective correction factors (form factors), $F^{coh}(s,l_c)$ and $F^{inc}(s,l_c)$ 
based on VDM has been suggested in  Ref.~\cite{Hufner:1996jw} and employed in Ref.~\cite{Ivanov:2002kc} for calculations of charmonium photo-production off nuclei. 
Then nuclear cross sections corrected to the finite coherence length effects are given as 
%
%==================================================================================================
\BA
\frac{d^2
  \sigma_A^{coh}(s,b)}{d^2 b} 
  %&=&
  =
\frac{d^2
  \sigma_A^{coh}(s,b)}{d^2 b}\Biggl|_{l_c\gg R_A} \cdot F^{coh}\bigl (s,l_c(s)\bigr )\,,
  %\\
  \qquad
\frac{d^2   
  \sigma_A^{inc}(s,b)}{d^2 b} 
  %&=&
=
\frac{d^2
  \sigma_A^{inc}(s,b)}{d^2 b}\Biggr|_{l_c\gg R_A} \cdot F^{inc}\bigl (s,l_c(s)\bigr )\,.
\EA
%=================================================================================================
%

In the present paper, as the further improvement, instead of VDM 
the corresponding factors $F(s,l_c)$
have been calculated within a more sophisticated Green function formalism and
have the following form,
%
%====================================================================================
\BA
  F^{coh}(s,l_c) &=& \int\!d^2b
    \left|
    \,\int\limits_{-\infty}^\infty\!dz\,\rho_A(b,z)\,F_1(s,b,z,l_c)\,
    \right|^2 \bigl / \,\Bigl (...\Bigr )\Bigl |_{l_c \to \infty} \,, 
%------------------------
    \label{fcoh}
    \\
%------------------------
  F^{inc}(s,l_c) &=& \int\!d^2b
    \,\int\limits_{-\infty}^\infty\!dz\,\rho_A(b,z)
    \,\Bigl |F_1(s,b,z,l_c)-F_2(s,b,z,l_c)\Bigr |^2
    \bigl / \,\Bigl (...\Bigr )\Bigl |_{l_c \to \infty} \,,
%------------------------
    \label{finc}
%----------------------- 
\EA
%======================================================================================
%
where the functions $F_1$ and $F_2$ read,
%
%==================================================================================!!!!
 \BA
F_1(s,b,z,l_c) &=&
\int\limits_0^1 d\alpha
\int d^{2} r_{1}\,d^{2} r_{2}\,
\Psi^{*}_{V}(\vec r_{2},\alpha)\,
G_{Q\bar Q}(z^\prime,\vec r_{2};z,\vec r_{1};l_c)\,
\sqq(r_{1},s)\,
\Psi_{\gamma}(\vec r_{1},
\alpha)
\Bigl|_{z^\prime\to\infty}
%------------
\label{f1}
%------------
\\
%
F_{2}(s,b,z,l_c) &=& \frac{1}{2}\,
\int\limits_{-\infty}^{z} dz_{1}\,\rho_{A}(b,z_1)\,
\int\limits_0^1 d\alpha\int d^2 r_1\,
d^2 r_{2}\,d^2 r\,
\Psi^*_V (\vec r_2,\alpha)\nonumber \\
&\times&
G_{Q\bar Q}(z^{\prime}\to\infty,\vec r_2;z,\vec r;l_c)\,
\sqq(\vec r,s)\,
G_{Q\bar Q}(z,\vec r;z_1,\vec r_1;l_c)\,
\sqq(\vec r_1,s)\,
\Psi_{\gamma}(\vec r_1,\alpha)\, .
%------------
\label{f2}
%------------
 \EA
%==================================================================================
%


%\!\!\!\!\!\!\!\!
Here $\Psi_{\gamma}(\vec r,\alpha) = K_0(m_Q r)$, and the Green function 
$G_{Q\bar Q}(z^\prime,\vec r_2;z,\vec r_1;l_c)$ describes the propagation of an interacting $Q\bar Q$ pair in a nuclear medium between points with longitudinal coordinates $z$ and $z^\prime$ and with initial and final separations $\vec r_1$ and $\vec r_2$.
%
In calculations for the sake of simplicity, we employed the quadratic form for the dipole cross section $\sqq(r,s) = C(s)\,r^2$,
and the harmonic oscillatory form for the LF $Q-\bar Q$ interaction potential in the evolution equation for the Green function (e.g. see Ref.~\cite{Kopeliovich:2001xj}). 
Another simplification is related to
a constant nuclear density, $\rho_A(b,z) = \rho_0 \Theta (R_A^2 - b^2 - z^2)$, which is rather accurate for heavy nuclei used in our analysis. Consequently, for the LF quarkonium wave functions, we obtained the following Gaussian shape \cite{Kopeliovich:2001xj} for the $1S$ and $2S$ states,
%
%=======================================================================================
\BA
\Psi_V(r,\alpha) 
&=& 
C_V\,a^2(\alpha)\,f(\alpha)\,\exp\Bigl [- \frac{1}{2} a^2(\alpha) r^2\Bigr ]
\\
\Psi_{V^{\prime}}(r,\alpha) 
&=& 
C_{V^{\prime}}\,a^2(\alpha)\,f(\alpha)\,\exp\Bigl [- \frac{1}{2} a^2(\alpha) r^2\Bigr ]
\Bigl \{1 + 4 h(\alpha) - \beta\, 2 a^2(\alpha) r^2 \Bigr \}\,,
%--------------
\label{psi-lc}
%--------------
\EA
%=======================================================================================
%
where
%
%==================================================================================
\BE
f(\alpha) = \exp\Bigl [ - h(\alpha)\Bigr ]
          =
\exp\biggl [ - \frac{m_Q^2}{2 a^2(\alpha)}
                        + \frac{4 \alpha (1-\alpha) m_Q^2}{2 a^2(\alpha)} \biggr ]\, ,
%---------------
\label{fa}
%---------------
\EE
%=================================================================================
%
and the parameter $\beta$, controlling the position of the node, has been determined from the orthogonality condition $\int d^2 r\,d \alpha \Psi_V(r,\alpha)\,\Psi_{V^{\prime}}(r,\alpha)= \delta_{VV^{\prime}}$. We have found $\beta = 0.908$ and $0.963$ for production of $\psip$ and $\Yp$, respectively. The function $a^2(\alpha) = 2\alpha(1-\alpha)\,m_Q\,\omega$, where the oscillatory frequency $\omega = (\left.M_{V^\prime}\right. - \left.M_V\right.)/2\approx 0.3\,\GeV$. 


The above approximations substantially simplify the calculations of the $l_c$-correction factors $F^{coh}$ and $F^{inc}$
since allow to obtain an explicit analytical harmonic oscillatory form for the Green function \cite{fg},  
%
%============================================================
 \BA
G_{Q\bar Q}(\vec{r_2},z_2;\vec{r_1},z_1;l_c) 
=
\frac{b(\alpha)}{2 \pi i\,
{\rm sin}(\Omega\Delta z)}\, {\rm exp}
\left\{
\frac{i\,b(\alpha)}{{2\,\rm sin}(\Omega \Delta z)}\,
\Bigl[(r_1^2+r_2^2)\,{\rm cos}(\Omega \Delta z) -
2 \vec{r_1}\cdot\vec{r_2}\Bigr]\right\}\,
{\rm exp}\left[-
\frac{i \Delta z}
{l_c}\right]\,,
%-------------
\label{GF-HO}
%-------------
 \EA
%============================================================
with $\Delta z = z_2-z_1$, $l_c = 2 k \alpha (1-\alpha)/m_Q^2$ and
%
%============================================================
\BA
\Omega
=      
\frac{b(\alpha)}{k\, \alpha (1 - \alpha)}
&=&
\frac{
\sqrt{a^4(\alpha) - i\,k\,\alpha\,(1-\alpha)\, 
C_{eff}(s,\alpha)\,\rho_A({b},z_2)
}}
{k\,\alpha\,(1 - \alpha)}\,.
%---------------
\label{GF-HO-s}
%---------------
\EA
%============================================================
%
Here,
considering the standard saturated shape of the dipole cross section,
%
%==============================================
\BA
\sqq(\vec r,s) =
\sigma_0\,
\biggl (1 - \exp \Bigl [ - \frac{r^2}{r_0^2(s)}\Bigr ] \biggr )
\,,
%-----------
\label{gbw}
%-----------
\EA
%==============================================
the factor $C_{eff}$ in Eq.~(\ref{GF-HO-s}) can be expressed in the following form,
%
%========================================================
\BA
C_{eff}(s,\alpha) 
= 
C(s)\, \biggl\{1 - \exp\biggl [-\,\frac{1} { a^2(\alpha)\,r_0^2(s)}\,\biggr ] \biggr \}\, a^2(\alpha)\,r_0^2(s)\,,
\qquad\qquad\qquad
C(s) = \sigma_0/r_0^2(s)\,.
\EA
%=========================================================

%%%%%%%%%%%%%%%%%%%%%%%%%%%%%%%%%%%%%%%%%%%%%%%%%%%%%%%%%%%%%%%%%%%%%%%%%%%%%%%%%%%%%%%
      %%%%%%%%%%%%%%%%%%%%%%%%%%%%%%%%%% FIG. 1 %%%%%%%%%%%%%%%%%%%%%%%%%%%%%%%%
%%%%%%%%%%%%%%%%%%%%%%%%%%%%%%%%%%%%%%%%%%%%%%%%%%%%%%%%%%%%%%%%%%%%%%%%%%%%%%%%%%%%%%%
\BF
%%%%%%%%%%%%%%%%
\hspace*{0.00cm}
\PSfig{1.30}{037-FF-final}
\vspace*{-0.6cm}
\Caption
   {
%-------------------
  \label{Fig-lc-gf}
%-------------------
  (left panels) - 
  the $l_c$-correction factors for coherent and incoherent production of $\Jpsi$ (solid lines) and $\psip$ (dashed lines) in $Pb-Pb$ UPC obtained within the color dipole approach based on the Green function technique, 
  Eqs.~(\ref{fcoh})-(\ref{f2}). The quarkonium wave functions are generated by the BT potential.
  (right panels) -
  the same as left panels but for production of $\Y$ and $\Yp$.
  The top and bottom panels correspond to calculations using KST and GBW models for the dipole cross section, respectively.
  }
\EF
%%%%%%%%%%%%%%%%%%%%%%%%%%%%%%%%%%%%%%%%%%%%%%%%%%%%%%%%%%%%%%%%%%%%%%%%%%%%%%%%

The $l_c$-correction factors $F(s)$, obtained from Eqs.~(\ref{fcoh})-(\ref{f2}),
are depicted in Fig.~\ref{Fig-lc-gf} 
as a function of the square of c.m. energy $s$
for coherent and incoherent $\Jpsi$ (solid lines) and $\psip$ (dashed lines) photo-production on the lead target (left panels).
Analogous results for $F(s)$ for photo-production of $\Y$ and $\Yp$ are depicted on the right panels of the same Fig.~\ref{Fig-lc-gf}. 
Here the top and bottom panels correspond to calculations using KST and GBW parametrization for the dipole cross section, respectively.
%

One can see from Fig.~\ref{Fig-lc-gf} that the effects of a finite CL are important for 
energies $s\lsim 10^3\,\GeV^2$ and $s\lsim 10^4\,\GeV^2$ in the production of charmonia and bottomonia, respectively. This is a direct consequence of the CL dependence on the quarkonium mass as given by Eq.~(\ref{lc}).
Figure~\ref{Fig-lc-gf} also demonstrates that contraction of the CL at smaller values of $s$ leads to a significant reduction of the coherent cross sections for the $1S$ quarkonium states. However, the corresponding incoherent cross sections are enhanced  by $\sim 20\%-30\,\%$ and $\sim 7\%-12\,\%$ for production of $\Jpsi(1S)$ and $\Y(1S)$, respectively.  

For radially excited $2S$ quarkonia, the manifestations of finite-$l_c$ 
is affected by the nodal structure of quarkonium wave functions. Its influence is stronger for  $\psip(2S)$ in comparison with $\Yp(2S)$, leading to a more complicated non-monotonic behaviour of the factor $F^{inc}(s)$ at small values of $s$. On the other hand, a stronger energy dependence from the region of small $Q\bar Q$ transverse separations below the node position compared to large $Q\bar Q$ dipole sizes above the node position causes a weakening of the node effect with rising energy, resulting in a gradual convergence of factors $F^{inc}_{V^{\prime}}(s)$ and $F^{inc}_{V}(s)$ 
towards large $s$.

\vspace*{0.2cm}
The leading twist gluon shadowing was introduced within the dipole representation in \cite{Kopeliovich:1999am} and applied to photo-production of vector mesons on nuclei in  Refs.~\cite{Kopeliovich:2001xj,Ivanov:2002kc,Nemchik:2002ug,Kopeliovich:2007wx}.
In the present paper, we include only  one gluon Fock state $|Q\bar QG\ra$,  because
higher multi-gluon components give negligible contribution to nuclear shadowing within the 
kinematic regions of present UPC experiments at the LHC (see analysis and discussion in Ref.~\cite{Kopeliovich:2022jwe}).
%
Since the dipole cross section $\sqq(r,s)$ at small dipole sizes $\vec r$
depends on the gluon distribution in the target, nuclear shadowing of the
gluon distribution can be interpreted within the parton model as a reduction of $\sqq(r,s)$ in nuclear reactions with respect to processes on the nucleon,
%
%------------------------------------------------------------
\BE
  \sqq(r,x) \Rightarrow \sqq(r,x) \cdot R_G(x,b)\,.
\label{eq:dipole:gs:replace}
\EE
%------------------------------------------------------------
%
Here the Gribov correction factor $R_G(x,b)$, related to the $Q\bar QG$ component of the photon, was calculated at impact parameter $b$ using the Green function
formalism \cite{Kopeliovich:1999am,Kopeliovich:2001xj,Kopeliovich:2001ee,Ivanov:2002kc,Nemchik:2002ug,Kopeliovich:2008ek,Krelina:2020ipn}\footnote{An alternative estimation of gluon shadowing effects in charmonium production in UPC can be found in \cite{guzey-1,guzey-2}, for example.}
(see also Fig.1 and a discussion in Ref.~\cite{Kopeliovich:2022jwe}).
\vspace*{0.2cm}

%%%%%%%%%%%%%%%%%%%%%%%%%%%%%%%%%%%%%%%%%%%%%%%%%%%%%%%%%%%%%%%%%%%%%%%%%%%%%%%%%%%%
       %%%%%%%%%%%%%%%%%%%%%%%%%%%%% FIG. 2 %%%%%%%%%%%%%%%%%%%%%%%%%%%%%%%%
%%%%%%%%%%%%%%%%%%%%%%%%%%%%%%%%%%%%%%%%%%%%%%%%%%%%%%%%%%%%%%%%%%%%%%%%%%%%%%%%%%%%
\BF
%%%%%%%%%%%%%%%%
\PSfig{1.1}{104-200-psi1S-coh.pdf}~~~~
\PSfig{1.1}{105-200-psi1S-inc.pdf}~~~~\\
\PSfig{1.1}{100-2760-psi1S-coh.pdf}~~~~
\PSfig{1.1}{101-2760-psi1S-inc.pdf}~~~~\\
\PSfig{1.1}{102-5020-psi1S-coh.pdf}~~~~
\PSfig{1.1}{103-5020-psi1S-inc.pdf}~~~~\\
\vspace*{-0.30cm}
\Caption{
%------------------
\label{Fig-UPC1psi}
%------------------
  Rapidity distributions of coherent (left panels) and incoherent (right panels) charmonium
  photo-production in UPC at RHIC collision energy $\sqrt{s_N}=200\,\GeV$ (top panels) and at LHC energies $\sqrt{s_N} = 2.76\,\TeV$ (middle panels) and $\sqrt{s_N} = 5.02\,\TeV$ (bottom panels). The nuclear cross sections are calculated with charmonium wave functions generated by the POW (thin lines) and BT (thick lines) potentials and with  GBW (solid lines), KST (dashed lines) and BGBK (dotted lines) models for the dipole cross section. The data are taken from
  PHENIX \cite{Afanasiev:2009hy},  
  CMS \cite{Khachatryan:2016qhq}, ALICE \cite{Abelev:2012ba,Abbas:2013oua,Adam:2015sia,Acharya:2019vlb,ALICE:2021gpt} and  LHCb \cite{LHCb:2018ofh,LHCb:2022ahs} collaborations.
  }
\EF
%%%%%%%%%%%%%%%%%%%%%%%%%%%%%%%%%%%%%%%%%%%%%%%%%%%%%%%%%%%%%%%%%%%%%%%%%%%%%%%%%%%%%%%
{\color{red} 
It is worth emphasizing that modification of the dipole cross section Eq.~(\ref{eq:dipole:gs:replace}) is a $b$-dependent procedure and can be done only theoretically. Indeed, the magnitude of the gluon shadowing factor $R_G(x)$ extracted from global analyses of nuclear parton distributions is $b$-integrated, so cannot be used in Eq.~(\ref{eq:dipole:gs:replace}), like it was done in Ref.~\cite{Henkels:2020kju} without any justification. 

%%%%%%%%%%%%%%%%%%%%%%%%%%%%%%%%%%%%%%%%%%%%%%%%%%%%%%%%%%%%%%%%%%%%%%%
Another difference between our calculation and those done in Ref.~\cite{Henkels:2020kju} is related to the effect of
the finite coherence length, which are unavoidable due to specifics of the UPC kinematics.
The approximation used in  \cite{Henkels:2020kju} was the same as in the previous studies \cite{Hufner:1996jw,Ivanov:2002kc,Ivanov:2002eq,Ivanov:2007ms} and was based on unjustified extrapolation of the small $q_T$ behavior to long $q_L$.
 In the present paper such an effect is calculated exactly in the rigorous quantum-mechanical description based on the Green function formalism.
%
Moreover, 
the photon energy given by Eq.~(3.1) in \cite{Henkels:2020kju} cannot
provide us with the $l_c$-values in the target rest frame as follows from the subsequent Eq.~(3.4). 
This fact makes questionable the correctness of calculations of the finite-$l_c$ effects.
}
%%%%%%%%%%%%%%%%%%%%%%%%%%%%%%%%%%%%%%%%%%%%%%%%%%%%%%%%%%%%%%%%%%%%%%






%%%%%%%%%%%%%%%%%%%%%%%%%%%%%%%%%%%%%%%%%%%%%%%%%%%%%%%%%%%%%%%%%%%%%
%%%%%%%%%%%%%%%%%%%%  S E C T I O N    III    %%%%%%%%%%%%%%%%%%%%%%%
%%%%%%%%%%%%%%%%%%%%%%%%%%%%%%%%%%%%%%%%%%%%%%%%%%%%%%%%%%%%%%%%%%%%%
%
%
%
%====================================================
\section{Model predictions vs available data}
\label{res}
%====================================================
%
%
%


We have calculated rapidity distributions $d\sigma/dy$ for
the coherent and incoherent heavy quarkonium photo-production in UPC according to Eq.~(\ref{cs-upc}). 
% 
Here the $|Q\bar Q\ra$ Fock component of the photon was treated in the ``frozen'' approximation,
$l_c\gg R_A$ (see Eqs.(\ref{coh-lcl}) and (\ref{inc-lcl})).
%
Moreover,
in order to calculate the nuclear cross sections (\ref{inc-lcl}) for the incoherent (quasi-elastic) production, one should know also the slope parameter for the elastic process $\gamma N\to V N$. Here we rely on the standard Regge form,
%
%*************************************
%\BE
$B_{\Jpsi}(s)=B_0 + 2\,\alpha'(0) \ln\big(s/s_0\big)$,
%B_{\Jpsi}(s)=B_0 + 2\,\alpha'(0) \ln\Big(\frac{s}{s_0}\Big)
%\,, 
%%%%%%%%%%%
%\label{BW}
%%%%%%%%%%%
%\EE
%*************************************
%
where the parameters $\alpha'=0.171\GeV^{-2}$, the slope of the Pomeron trajectory, and
$B_{0}=1.54\,\GeV^{-2}$ were fitted in \cite{Cepila:2019skb} to data on $\Jpsi$ photo-production with $s_0=1\GeV^2$. 
%
The slope for $1S$ bottomonium photo-production was fitted to data in \cite{Cepila:2019skb} and found to have a smaller value than for $\Jpsi$,
$B_{\Y}(s)\approx B_{\Jpsi}(s) -1\GeV^{-2}$.
%
For the production of radially excited $2S$ state of bottomonia, the node effect is negligibly small, and one can safely use the same magnitudes of the slope parameter for both $1S$ and $2S$ states, i.e. $B_{\Yp}(s)\sim B_{\Y}(s)$. Not so for the production of radially excited charmonia where the difference of diffraction slopes $\Delta_B(s)=B_{\Jpsi}(s)-B_{\psip}(s)$ cannot be neglected. 
Here we adopt a parametrization of the factor $\Delta_B(s)$ 
from Ref.~\cite{Cepila:2019skb} (see also Ref.~\cite{Nemchik:1997xb}).

Besides nuclear suppression of the lowest Fock component, $|Q\bar Q\ra$,
we included in our predictions two main phenomena affecting the nuclear cross sections:
the gluon shadowing and the finite-$l_c$ corrections. Whereas the former dominates at large photon energies, the latter is prominent at smaller energies, 
when $l_c\lsim R_A$.
Here the $l_c$-correction factors are
calculated for the first time within a rigorous Green function formalism as described in Sect.~\ref{fcor}.


In Fig.~\ref{Fig-UPC1psi} we present 
our results for the rapidity distributions $d\sigma/dy$ of coherent (left panels) and incoherent (right panels) charmonium photo-production in UPC obtained for $\sqrt{s_N} = 200\,\GeV$ (top panels),
$\sqrt{s_N} = 2.76\,\TeV$ (middle panels) and $\sqrt{s_N} = 5.02\,\TeV$ (bottom panels).
Calculations have been performed for charmonium wave functions generated by two distinct $Q\bar Q$ potentials, POW (thin lines) and BT (thick lines).  
For the dipole cross sections $\sqq$ we adopted three different parametrizations,
GBW (solid lines), KST (dashed lines) and BGBK (dotted lines).
Here the model predictions are tested
by the RHIC data from the PHENIX experiment \cite{Afanasiev:2009hy},
by the LHC data from the CMS \cite{Khachatryan:2016qhq} and ALICE \cite{Abelev:2012ba,Abbas:2013oua,Adam:2015sia} collaborations at c.m. collision energy $\sqrt{s_N} = 2.76\,\TeV$, as well as by the 
ALICE \cite{Acharya:2019vlb,ALICE:2021gpt} and 
LHCb \cite{LHCb:2018ofh,LHCb:2022ahs} data at $\sqrt{s_N} = 5.02\,\TeV$.


In Fig.~\ref{Fig-UPC1psi} one can see that the values of $d\sigma/dy$ 
strongly correlate with the shape of the quarkonium wave functions determined with various models for $Q-\bar Q$ interaction potentials (compare thin and thick lines).
While in charmonium production in UPC, the POW (thin lines) and BT (thick lines) models lead to rather different predictions for $d\sigma/dy$, in the bottomonium case both models 
give quite similar (almost identical) results (see Fig.~\ref{Fig-UPC1y}). This is in correspondence with our previous studies \cite{Cepila:2019skb} of quarkonium electroproduction off protons. 


%%%%%%%%%%%%%%%%%%%%%%%%%%%%%%%%%%%%%%%%%%%%%%%%%%%%%%%%%%%%%%%%%%%%%%%%%%%%%%%%%%%%%%%%
       %%%%%%%%%%%%%%%%%%%%%%%%%%%%% FIG. 3 %%%%%%%%%%%%%%%%%%%%%%%%%%%%%%%%
%%%%%%%%%%%%%%%%%%%%%%%%%%%%%%%%%%%%%%%%%%%%%%%%%%%%%%%%%%%%%%%%%%%%%%%%%%%%%%%%%%%%%%%%
\BF
%%%%%%%%%%%%%%%%
\PSfig{1.1}{110-2760-psi2S-coh.pdf}~~~~
\PSfig{1.1}{112-5020-psi2S-coh.pdf}~~~~\\
\vspace*{-0.30cm}
\Caption{
%----------------------
  \label{Fig-UPC1psi2S}
%----------------------
  The same as Fig.~\ref{Fig-UPC1psi} but for the coherent $\psip(2S)$ production in UPC
  at the collision energy $\sqrt{s_N} = 2.76\,\TeV$ (left panel) and $\sqrt{s_N} = 5.02\,\TeV$ (right panel).
  The experimental values are taken from the ALICE
  \cite{Adam:2015sia,ALICE:2021gpt} and LHCb \cite{LHCb:2022ahs} collaborations.
  }
\EF
%%%%%%%%%%%%%%%%%%%%%%%%%%%%%%%%%%%%%%%%%%%%%%%%%%%%%%%%%%%%%%%%%%%%%%%%%%%%%%%%%%%%%%%%

The experimental data on the production of radially excited heavy quarkonia in UPC are very scarcy. The ALICE collaboration \cite{Adam:2015sia} measured $d\sigma/dy$ for coherent production of $\psip(2S)$ at $\sqrt{s_N} = 2.76\,\TeV$ and $y=0$, as is depicted in the left panel of Fig.~\ref{Fig-UPC1psi2S} together with our results. Besides, a new experimental value of $d\sigma/dy$ has been obtained recently at higher energy $\sqrt{s_N} = 5.02\,\TeV$  in the ALICE experiment \cite{ALICE:2021gpt} at $y=0$, as well as several data points of $d\sigma/dy$ by the LHCb \cite{LHCb:2022ahs} experiment at various positive rapidities (see the right panel of Fig.~\ref{Fig-UPC1psi2S}).
%
One can see a reasonable agreement of our calculations with these data for both $c-\bar c$ interaction potentials, as well as for all models of the dipole cross section. Whereas at $\sqrt{s_N} = 2.76\,\TeV$ the better agreement with data is achieved with charmonium wave functions generated by the POW $c-\bar c$ potential, the higher collision energy  $\sqrt{s_N} = 5.02\,\TeV$ prefers excited charmonia described by the BT potential model. 

%%%%%%%%%%%%%%%%%%%%%%%%%%%%%%%%%%%%%%%%%%%%%%%%%%%%%%%%%%%%%%%%%%%%%%%%%%%%%%%%%%%%%%%%
       %%%%%%%%%%%%%%%%%%%%%%%%%%%%% FIG. 4 %%%%%%%%%%%%%%%%%%%%%%%%%%%%%%%%
%%%%%%%%%%%%%%%%%%%%%%%%%%%%%%%%%%%%%%%%%%%%%%%%%%%%%%%%%%%%%%%%%%%%%%%%%%%%%%%%%%%%%%%%
\BF
%%%%%%%%%%%%5%%%
\PSfig{1.1}{122-5020-upsi1S-coh.pdf}~~~~
\PSfig{1.1}{123-5020-upsi1S-inc.pdf}~~~~\\
\vspace{-0.30cm}
\Caption{
%------------------
  \label{Fig-UPC1y}
%------------------
  The same as Fig.~\ref{Fig-UPC1psi} but for the bottomonium production in UPC at the collision energy $\sqrt{s_N} = 5.02\,\TeV$.
  }
\EF
%%%%%%%%%%%%%%%%%%%%%%%%%%%%%%%%%%%%%%%%%%%%%%%%%%%%%%%%%%%%%%%%%%%%%%%%%%%%%%%%%%%%%%%%%

The next Fig.~\ref{Fig-UPC1y} represents analogous predictions as Fig.~\ref{Fig-UPC1psi} but for production of bottomonia in UPC. Here the GBW, KST and BGBK parametrizations give rather different results in correspondence with the previous analysis of the process $\gamma N\to\Y N$ in Ref.~\cite{Cepila:2019skb}. 
Differences in predictions using various models for $\sqq(r)$ can be treated as a measure of the theoretical uncertainty in our results.


%%%%%%%%%%%%%%%%%%%%%%%%%%%%%%%%%%%%%%%%%%%%%%%%%%%%%%%%%%%%%%%%%%%%%%%%%%%%%%%%%%%%%%
       %%%%%%%%%%%%%%%%%%%%%%%%%%%%% FIG. 5 %%%%%%%%%%%%%%%%%%%%%%%%%%%%%%%%
%%%%%%%%%%%%%%%%%%%%%%%%%%%%%%%%%%%%%%%%%%%%%%%%%%%%%%%%%%%%%%%%%%%%%%%%%%%%%%%%%%%%%%
\BF
%%%%%%%%%%%%%%%%
\PSfig{1.1}{140-5500-psi1S-coh-eff.pdf}~~~~
\PSfig{1.1}{144-5500-upsi1S-coh-eff.pdf}~~~~\\
\PSfig{1.1}{142-5500-psi2S-coh-eff.pdf}~~~~
\PSfig{1.1}{146-5500-upsi2S-coh-eff.pdf}~~~~\\
\vspace*{-0.3cm}
\Caption{
%----------------------
  \label{Fig-UPC2psi-y}
%-----------------------
  Manifestations of particular nuclear effects in
  coherent charmonium (left panels) and bottomonium (right panels)
  photo-production in UPC at the LHC collision energy $\sqrt{s_N}= 5.5\,\TeV$. Here top and bottom panels correspond to the production of 1S and 2S quarkonium states, respectively. The nuclear cross sections are calculated with charmonium wave functions generated by the BT potential adopting the KST model for the dipole cross section. 
  The dashed lines represent predictions in the high energy eikonal limit, Eq.~(\ref{coh-lcl}).
  The solid lines include additional corrections for a finite CL and the gluon shadowing.
  }
\EF
%%%%%%%%%%%%%%%%%%%%%%%%%%%%%%%%%%%%%%%%%%%%%%%%%%%%%%%%%%%%%%%%%%%%%%%%%%%%%%%%%%%%%%


Figure~\ref{Fig-UPC2psi-y} demonstrates importance of particular nuclear effects for the rapidity dependence  $d\sigma/dy$
at energy $\sqrt{s_N}= 5.5\,\TeV$ in coherent production of charmonia (left panels) and bottomonia (right panels) in UPC, whereas the top and bottom panels represent model predictions for 1S and 2S quarkonium states, respectively.
Here dashed lines correspond to our calculations in the standard high energy eikonal limit, $l_c\gg R_A$,
Eq.(\ref{coh-lcl}).
%
The solid lines incorporate additionally two effects,  
corrections to a finite CL and the leading twist GS effect. 
%
Our model predictions are based on the KST model for the dipole cross section and include quarkonium wave functions determined from the BT model for the $Q-\bar Q$ interaction potential.

One can see from Fig.~\ref{Fig-UPC2psi-y} that differences between solid and dashed curves at large forward (backward) rapidities represent the relevance of the finite-$l_c$ corrections which is more pronounced in coherent production of bottomonia compared to the charmonium case. Here effects of gluon shadowing are substantially diminished due to their small contribution to $d\sigma/dy$ at large photon energies given by a significantly reduced photon flux, Eq.~(\ref{flux}).
%
However, the latter effects dominate at the mid rapidities, where effects of CL reduction do not play any role since the ``frozen'' eikonal limit is acquired for the higher twist shadowing correction.  Both effects substantially suppress the cross sections $d\sigma/dy$. In comparison to charmonium production, a weaker onset of gluon shadowing at a fixed $\sqrt{s_N}$ in production of bottomonia is caused by larger values of the both the variable $x$ and the corresponding scale $\propto M_{\Y}^2\gg M_{\Jpsi}^2$ (see also Fig.1 in Ref.~\cite{Kopeliovich:2022jwe}).


%%%%%%%%%%%%%%%%%%%%%%%%%%%%%%%%%%%%%%%%%%%%%%%%%%%%%%%%%%%%%%%%%%%%%%%%%%%%%%%%%%%%
       %%%%%%%%%%%%%%%%%%%%%%%%%%%%% FIG. 6 %%%%%%%%%%%%%%%%%%%%%%%%%%%%%%%%
%%%%%%%%%%%%%%%%%%%%%%%%%%%%%%%%%%%%%%%%%%%%%%%%%%%%%%%%%%%%%%%%%%%%%%%%%%%%%%%%%%%%
\BF
%%%%%%%%%%%%%%%%
\PSfig{1.1}{141-5500-psi1S-inc-eff.pdf}~~~~
\PSfig{1.1}{145-5500-upsi1S-inc-eff.pdf}~~~~\\
\PSfig{1.1}{143-5500-psi2S-inc-eff.pdf}~~~~
\PSfig{1.1}{147-5500-upsi2S-inc-eff.pdf}~~~~\\
\vspace*{-0.3cm}
\Caption{
%---------------------
  \label{Fig-UPC3-inc}
%---------------------
  The same as Fig.~\ref{Fig-UPC2psi-y} but for incoherent production of the ground state $1S$ (top panels) and radially excited $2S$ (bottom panels) quarkonia in UPC at $\sqrt{s_N}=5.5\,\TeV$.
  }
\EF
%%%%%%%%%%%%%%%%%%%%%%%%%%%%%%%%%%%%%%%%%%%%%%%%%%%%%%%%%%%%%%%%%%%%%%%%%%%%%%%%%%%

The last Fig.~\ref{Fig-UPC3-inc} illustrates a manifestation of particular nuclear effects 
at energy $\sqrt{s_N}=5.5\,\TeV$
in the incoherent production of the $1S$ ground state (top panels) and $2S$ radially excited (bottom panels) quarkonia. 
The effects of reduced CL are visible only in the production of $\Jpsi$ at large positive and negative rapidities. For other quarkonium states, they can be neglected in accordance with values of $l_c$-correction factors which are close to unity (see Fig.\ref{Fig-lc-gf}).
At the mid rapidity, the
difference between solid and dashed lines shows the net effect of gluon shadowing with a maximal magnitude at $y=0$.





%%%%%%%%%%%%%%%%%%%%%%%%%%%%%%%%%%%%%%%%%%%%%%%%%%%%%%%%%%%%%%%%%%%%%
%%%%%%%%%%%%%%%%%%%%  S E C T I O N    IV     %%%%%%%%%%%%%%%%%%%%%%%
%%%%%%%%%%%%%%%%%%%%%%%%%%%%%%%%%%%%%%%%%%%%%%%%%%%%%%%%%%%%%%%%%%%%%
%
%
%
%==================================================
\section{Conclusions}
\label{conclusions}
%==================================================
%
%
%

In this paper we treat the heavy quarkonium [$\Jpsi(1S),\psip(2S),\Y(1S),\Yp(2S)$]
production in heavy-ion UPC 
within the light-front QCD dipole approach
in the energy range accessible by experiments at RHIC and the LHC. Here the main observations are the following:



\begin{itemize}

\item 
The quarkonium wave functions are well defined in the $Q\bar Q$ rest frame. They have been included in our calculations by solving the Schr\"odinger equation with several realistic $Q-\bar Q$ interaction potentials.
Consequently, the corresponding LF wave functions have been generated performing the boosting to the LF frame using the so-called Terent'ev prescription from Ref.~\cite{Terentev:1976jk}, which was found in Ref.~\cite{Kopeliovich:2015qna} to have a reasonable accuracy in comparison with exact solution. 

Here we ignore the model for the photon-like $V\to Q\bar Q$ transition, frequently used in the literature,  
avoiding so too large weight of the $D$-wave component in the $Q\bar Q$ rest frame, in consistence with solutions of the Schr\"odinger
equation.

\item 
The spin-dependent part of the wave function for $S$-wave quarkonia can be safely factorized from the radial component. 
Consequently, we
perform explicitly the transformation of two-dimensional heavy (anti)quark spinors from the rest to the LF frame known as the Melosh spin rotation. We derived new formulas for coherent and incoherent nuclear cross sections incorporating such a transformation
(see Eqs.~(\ref{sr-coh-inc})-(\ref{sigma-sr2})).

\item
UPC  at RHIC and the LHC at the mid rapidities, provide a sufficiently high energy and a long  coherence length for the lowest $|Q\bar Q\ra$ Fock state of the photon. As far as CL  considerably exceeds the nuclear 
size, one can rely on the high-energy eikonal approximation for nuclear effects (see Eqs.~(\ref{coh-lcl}) and (\ref{inc-lcl})).
The corresponding shadowing correction is small since diminishes with heavy quark mass as $1/m_Q^2$, so represents the higher twist effect. 

\item
At forward and/or backward rapidities, the eikonalization of $\sqq(r,s)$ cannot be applied anymore and 
we included corrections for a finite coherence length which have been calculated for the first time within a rigorous quantum-mechanical description, summing up all possible paths of the quarks.
A proper treatment of finite-$l_c$ corrections is absent in the most of the recent calculations of UPC.
%
These corrections lead to a significant modification
of rapidity distributions $d\sigma/dy$ at small photon energies when $l_c\lsim R_A$ (see Figs.~\ref{Fig-lc-gf}, \ref{Fig-UPC2psi-y} and \ref{Fig-UPC3-inc}).


\item
We also included the gluon shadowing corrections related to higher Fock components of the photon containing gluons. Those components have a coherence length much shorter than the lowest 
$|Q\bar Q\ra$ Fock state.
They represents the leading-twist effect since the transverse size of the $Q\bar Q-G$ dipole is much larger compared to the small-sized $Q\bar Q$ fluctuation and is almost independent of $m_Q$.
The dominant contribution to nuclear shadowing comes from the $|Q\bar QG\ra$ Fock state of the photon. 
In calculations of the corresponding shadowing factor one cannot use the standard eikonal approximation even at high energies since the $Q\bar Q-G$ size fluctuates during propagation through the medium. This is why we applied the Green function formalism.
The higher photon fluctuations with more gluons do not cause a significant shadowing effect (see also Sec. IV in Ref.~\cite{Kopeliovich:2022jwe}).

\item
We have also studied differences in our predictions employing KST, GBW and BGBK phenomenological parametrizations for the dipole cross section $\sqq(r,s)$ in order to estimate 
a corresponding measure of the theoretical uncertainty in our current analysis.
We  concluded that whereas in charmonium production in UPC, the main source of theoretical uncertainties is related predominantly to our choice of quarkonium wave functions, in the bottomonium case, the main role in the variability of predictions is played by various models for $\sqq(r,s)$.


\item
Our predictions for $d\sigma/dy$ are in a rather good accord with available data on coherent production of $\Jpsi(1S)$ and $\psip(2S)$ in UPC at the energies of RHIC and LHC (see Figs.~\ref{Fig-UPC1psi} and \ref{Fig-UPC1psi2S}). 
They can be tested not only by measurements at the LHC, but also in future experiments at the planned electron-ion colliders. 


\end{itemize}

%%%%%%%%%%%%%%%%%%%%%%%%%%%%%%%%%%%%%%%
\begin{acknowledgements}
This work was supported in part by grant ANID PIA/APOYO AFB220004.
%
The work of J.N. was partially supported by Grant
No. LTT18002 of the Ministry of Education, Youth and
Sports of the Czech Republic,
by the project of the
European Regional Development Fund No. CZ.02.1.01/0.0/0.0/16\_019/0000778
and by the Slovak Funding Agency, Grant No. 2/0020/22.
%
The work of M.K. was supported by the project of the International Mobility of Researchers - MSCA IF IV at CTU in Prague 
CZ.02.2.69/0.0/0.0/20\_079/0017983, Czech Republic.
\end{acknowledgements}
%%%%%%%%%%%%%%%%%%%%%%%%%%%%%%%%%%%%%%%%








% -----------------------------------------------------------
\begin{thebibliography}{99}
% -----------------------------------------------------------

\bibitem{Hufner:1997jg}
   J.~Hufner and B.~Z.~Kopeliovich,
   ``J / Psi N and Psi-prime N total cross-sections from photo-production data: Failure of vector dominance,''
   Phys. Lett. B \textbf{426}, 154-160 (1998).
   %doi:10.1016/S0370-2693(98)00257-3
   %[arXiv:hep-ph/9712297 [hep-ph]].
   %60 citations counted in INSPIRE as of 24 Jan 2023

\bibitem{Kopeliovich:1991pu} 
   B.Z.~Kopeliovich and B.G.~Zakharov;
   ``Quantum effects and color transparency in charmonium photoproduction on nuclei,''
   Phys. Rev. D \textbf{44}, 3466 (1991).

\bibitem{Ivanov:2007ms}
   Y.~Ivanov, B.~Kopeliovich and I.~Schmidt,
   ``Vector meson production in ultra-peripheral collisions at LHC,''
   arXiv:{0706.1532 [hep-ph]}.
   %23 citations counted in INSPIRE as of 06 May 2020

\bibitem{Kopeliovich:1993pw} 
   B.Z.~Kopeliovich, J.~Nemchik, N.N.~Nikolaev and B.G.~Zakharov;
   ``Decisive test of color transparency in exclusive electroproduction of vector mesons,''
   Phys. Lett. B \textbf{324}, 469 (1994).

\bibitem{Nemchik:1996pp}
   J.~Nemchik, N.~N.~Nikolaev, E.~Predazzi and B.~Zakharov,
   ``Color dipole systematics of diffractive photoproduction and electroproduction of vector mesons,''
   Phys. Lett. B \textbf{374}, 199 (1996).
   %doi:10.1016/0370-2693(96)00153-0
   %[arXiv:hep-ph/9604419 [hep-ph]].
   %71 citations counted in INSPIRE as of 06 Apr 2020

\bibitem{Nemchik:1996cw}
   J.~Nemchik, N.~N.~Nikolaev, E.~Predazzi and B.~Zakharov,
   ``Color dipole phenomenology of diffractive electroproduction of light vector mesons at HERA,''
   Z. Phys. C \textbf{75}, 71 (1997).
   %doi:10.1007/s002880050448
   %[arXiv:hep-ph/9605231 [hep-ph]].
   %190 citations counted in INSPIRE as of 06 Apr 2020

\bibitem{Kopeliovich:1993gk}
   B.~Kopeliovich, J.~Nemchik, N.~N.~Nikolaev and B.~Zakharov,
   ``Novel color transparency effect: Scanning the wave function of vector mesons,''
   Phys. Lett. B \textbf{309}, 179 (1993).
   %doi:10.1016/0370-2693(93)91523-P
   %[arXiv:hep-ph/9305225 [hep-ph]].
   %138 citations counted in INSPIRE as of 06 Apr 2020


\bibitem{Nemchik:1994fp}
   J.~Nemchik, N.~N.~Nikolaev and B.~Zakharov,
   ``Scanning the BFKL pomeron in elastic production of vector mesons at HERA,''
   Phys. Lett. B \textbf{341}, 228 (1994).
   %doi:10.1016/0370-2693(94)90314-X
   %[arXiv:hep-ph/9405355 [hep-ph]].
   %167 citations counted in INSPIRE as of 06 Apr 2020

\bibitem{Kopeliovich:2001xj}
   B.~Kopeliovich, J.~Nemchik, A.~Schafer and A.~Tarasov,
   ``Color transparency versus quantum coherence in electroproduction of vector mesons off nuclei,''
   Phys. Rev. C \textbf{65}, 035201 (2002).
   %doi:10.1103/PhysRevC.65.035201
   %[arXiv:hep-ph/0107227 [hep-ph]].
   %104 citations counted in INSPIRE as of 06 Apr 2020

\bibitem{Kopeliovich:2007wx}
   B.~Kopeliovich, J.~Nemchik and I.~Schmidt,
   ``Production of Polarized Vector Mesons off Nuclei,''
   Phys. Rev. C \textbf{76}, 025210 (2007).
   %doi:10.1103/PhysRevC.76.025210
   %[arXiv:hep-ph/0703118 [hep-ph]].
   %8 citations counted in INSPIRE as of 06 Apr 2020

%%%%%%%%%%%%%%%%%%%%%%%%%%%%%%%%%%%%%%%%% 10 %%%%%%%%%%%%%%%%%%%%%%%%%%%%%%%%%%%%%%%%%%%%%%

\bibitem{Nemchik:1994fq}
   J.~Nemchik, N.~N.~Nikolaev and B.~Zakharov,
   ``Anomalous a-dependence of diffractive leptoproduction of radial excitation rho-prime (2s),''
   Phys. Lett. B \textbf{339}, 194 (1994).
   %doi:10.1016/0370-2693(94)91154-1
   %[arXiv:nucl-th/9405025 [nucl-th]].
   %27 citations counted in INSPIRE as of 06 Apr 2020

\bibitem{Hufner:2000jb} 
   J.~Hufner, Y.P.~Ivanov, B.Z.~Kopeliovich and A.V.~Tarasov;
   ``Photoproduction of charmonia and total charmonium proton cross-sections,''
   Phys. Rev. D \textbf{62}, 094022 (2000).

\bibitem{Nemchik:2000de}
   J.~Nemchik,
   ``Wave function of 2S radially excited vector mesons from data for diffraction slope,''
   Phys. Rev. D \textbf{63}, 074007 (2001).
   %doi:10.1103/PhysRevD.63.074007
   %[arXiv:hep-ph/0003245 [hep-ph]].
   %12 citations counted in INSPIRE as of 19 Aug 2022

\bibitem{Nemchik:2000dd}
   J.~Nemchik,
   ``Anomalous t dependence in diffractive electroproduction of 2S radially excited light vector mesons at HERA,''
   Eur. Phys. J. C \textbf{18}, 711 (2001).
   %doi:10.1007/s100520100572
   %[arXiv:hep-ph/0003244 [hep-ph]].
   %10 citations counted in INSPIRE as of 19 Aug 2022

\bibitem{Kopeliovich:2000ra} 
   B.Z.~Kopeliovich, J.~Raufeisen and A.V.~Tarasov,
   ``Nuclear shadowing and coherence length for longitudinal and transverse photons,''
   Phys. Rev. C \textbf{62}, 035204 (2000).
   %doi:10.1103/PhysRevC.62.035204
   %[hep-ph/0003136].
   %%CITATION = doi:10.1103/PhysRevC.62.035204;%%
   %87 citations counted in INSPIRE as of 31 Jan 2020

\bibitem{Lappi:2020ufv}
   T.~Lappi, H.~M\"antysaari and J.~Penttala,
   ``Relativistic corrections to the vector meson light front wave function,''
   Phys. Rev. D \textbf{102}, 054020 (2020).

\bibitem{Mantysaari:2021ryb}
   H.~M\"antysaari and J.~Penttala,
   ``Exclusive heavy vector meson production at next-to-leading order in the dipole picture,''
   Phys. Lett. B \textbf{823}, 136723 (2021).
   %doi:10.1016/j.physletb.2021.136723
   %[arXiv:2104.02349 [hep-ph]].
   %34 citations counted in INSPIRE as of 13 Feb 2023

\bibitem{Sambasivam:2019gdd}
   B.~Sambasivam, T.~Toll and T.~Ullrich,
   ``Investigating saturation effects in ultraperipheral collisions at the LHC with the color dipole model,''
   Phys. Lett. B \textbf{803}, 135277 (2020).
   %doi:10.1016/j.physletb.2020.135277
   %[arXiv:1910.02899 [hep-ph]].
   %25 citations counted in INSPIRE as of 13 Feb 2023

\bibitem{Goncalves:2020vdp}
   V.~P.~Gon\c{c}alves, D.~E.~Martins and C.~R.~Sena,
   ``Coherent and incoherent $J/\Psi $ photoproduction in $Pb - Pb$ collisions at the LHC, HE-LHC and FCC,''
   Eur. Phys. J. A \textbf{57}, 82 (2021).
   %doi:10.1140/epja/s10050-021-00404-z
   %[arXiv:2007.13625 [hep-ph]].
   %5 citations counted in INSPIRE as of 13 Feb 2023

\bibitem{Kopeliovich:2022jwe}
   B.~Z.~Kopeliovich, M.~Krelina, J.~Nemchik and I.~K.~Potashnikova,
   ``Coherent photoproduction of heavy quarkonia on nuclei,''
   Phys. Rev. D \textbf{105}, 054023 (2022).
   %doi:10.1103/PhysRevD.105.054023
   %[arXiv:2201.13021 [hep-ph]].
   %1 citations counted in INSPIRE as of 01 Dec 2022

%%%%%%%%%%%%%%%%%%%%%%%%%%%%%%%%%%%%%%% 20 %%%%%%%%%%%%%%%%%%%%%%%%%%%%%%%%%%%%%%%%%%%%%%%

\bibitem{Ivanov:2002kc}
   Y.~Ivanov, B.~Kopeliovich, A.~Tarasov and J.~Hufner,
   ``Electroproduction of charmonia off nuclei,''
   Phys. Rev. C \textbf{66}, 024903 (2002).
   %doi:10.1103/PhysRevC.66.024903
   %[arXiv:hep-ph/0202216 [hep-ph]].
   %29 citations counted in INSPIRE as of 06 Apr 2020

\bibitem{Kopeliovich:1999am} 
   B.Z.~Kopeliovich, A.~Schafer and A.V.~Tarasov,
   ``Nonperturbative effects in gluon radiation and photoproduction of quark pairs,''
   Phys. Rev. D \textbf{62}, 054022 (2000).

\bibitem{Krelina:2018hmt}
   M.~Krelina, J.~Nemchik, R.~Pasechnik and J.~Cepila,
   ``Spin rotation effects in diffractive electroproduction of heavy quarkonia,''
   Eur. Phys. J. C \textbf{79}, 154 (2019).
   %doi:10.1140/epjc/s10052-019-6666-y
   %[arXiv:1812.03001 [hep-ph]].
   %5 citations counted in INSPIRE as of 06 Apr 2020

\bibitem{Cepila:2019skb}
   J.~Cepila, J.~Nemchik, M.~Krelina and R.~Pasechnik,
   ``Theoretical uncertainties in exclusive electroproduction of S-wave heavy quarkonia,''
   Eur. Phys. J. C \textbf{79}, 495 (2019).
   %doi:10.1140/epjc/s10052-019-7016-9
   %[arXiv:1901.02664 [hep-ph]].
   %10 citations counted in INSPIRE as of 06 Apr 2020

\bibitem{Krelina:2019egg}
   M.~Krelina, J.~Nemchik and R.~Pasechnik,
   ``$D$-wave effects in diffractive electroproduction of heavy quarkonia from the photon-like $V\rightarrow Q\bar Q$ transition,''
   Eur. Phys. J. C \textbf{80}, 92 (2020).
   %doi:10.1140/epjc/s10052-020-7678-3
   %[arXiv:1909.12770 [hep-ph]].
   %2 citations counted in INSPIRE as of 06 Apr 2020

\bibitem{Krelina:2020bxt}
   M.~Krelina and J.~Nemchik,
   ``D-wave effects in heavy quarkonium production in ultraperipheral nuclear collisions,''
   Phys. Rev. D \textbf{102}, 114033 (2020).
   %doi:10.1103/PhysRevD.102.114033
   %[arXiv:2010.00329 [hep-ph]].
   %1 citations counted in INSPIRE as of 03 Feb 2021

\bibitem{Bertulani:2005ru}
   C.~A.~Bertulani, S.~R.~Klein and J.~Nystrand,
   ``Physics of ultra-peripheral nuclear collisions,''
   Ann. Rev. Nucl. Part. Sci. \textbf{55}, 271-310 (2005).
   %doi:10.1146/annurev.nucl.55.090704.151526
   %[arXiv:nucl-ex/0502005 [nucl-ex]].
   %428 citations counted in INSPIRE as of 24 Jan 2023

\bibitem{Nemchik:2002ug}
   J.~Nemchik,
   ``Incoherent production of charmonia off nuclei as a good tool for study of color transparency,''
   Phys. Rev. C \textbf{66}, 045204 (2002).
   %doi:10.1103/PhysRevC.66.045204
   %[arXiv:hep-ph/0205276 [hep-ph]].
   %6 citations counted in INSPIRE as of 06 Apr 2020

\bibitem{DeJager:1987qc}
   H.De Vries, C.W.De Jager and C.De Vries,
   ``Nuclear charge and magnetization density distribution parameters from elastic electron scattering,''
   Atom. Data Nucl. Data Tabl. \textbf{36}, 495 (1987).
   %  doi:10.1016/0092-640X(87)90013-1

\bibitem{Bauer:1977iq}
   T.~Bauer, R.~Spital, D.~Yennie and F.~Pipkin,
   ``The Hadronic Properties of the Photon in High-Energy Interactions,''
   Rev. Mod. Phys. \textbf{50}, 261 (1978).
   %doi:10.1103/RevModPhys.50.261
   %937 citations counted in INSPIRE as of 19 May 2020

%%%%%%%%%%%%%%%%%%%%%%%%%%%%%%%%%%%%%%%%% 30 %%%%%%%%%%%%%%%%%%%%%%%%%%%%%%%%%%%%%%%%%%%%%%%

\bibitem{Kopeliovich:1981pz}
   B.~Z.~Kopeliovich, L.~I.~Lapidus and A.~B.~Zamolodchikov,
   ``Dynamics of Color in Hadron Diffraction on Nuclei,''
   JETP Lett. \textbf{33}, 595 (1981).
   %JINR-E2-81-251.
   %243 citations counted in INSPIRE as of 19 Aug 2022

\bibitem{Gribov:1968jf}
   V.~N.~Gribov,
   ``Glauber corrections and the interaction between high-energy hadrons and nuclei,''
   Sov.\ Phys.\ JETP {\bf 29}, 483 (1969)
   [Zh.\ Eksp.\ Teor.\ Fiz.\  {\bf 56}, 892 (1969)].
   %%CITATION = SPHJA,29,483;%%
   %532 citations counted in INSPIRE as of 06 Apr 2020

\bibitem{Kopeliovich:2016jjx}
   B.~Z.~Kopeliovich,
   ``Gribov inelastic shadowing in the dipole representation,''
   Int. J. Mod. Phys. A \textbf{31}, 1645021 (2016).
   %doi:10.1142/S0217751X16450214
   %[arXiv:1602.00298 [hep-ph]].
   %6 citations counted in INSPIRE as of 04 Aug 2020

\bibitem{Hufner:1996dr}
   J.~Hufner, B.~Kopeliovich and J.~Nemchik,
   ``Glauber multiple scattering theory for the photoproduction of vector mesons off nuclei and the role of the coherence length,''
   Phys. Lett. B \textbf{383}, 362 (1996).
   %doi:10.1016/0370-2693(96)00762-9
   %[arXiv:nucl-th/9605007 [nucl-th]].
   %57 citations counted in INSPIRE as of 04 Aug 2020

\bibitem{Kopeliovich:2005us}
   B.~Z.~Kopeliovich, I.~K.~Potashnikova and I.~Schmidt,
   ``Large rapidity gap processes in proton-nucleus collisions,''
   Phys. Rev. C \textbf{73}, 034901 (2006).
   %doi:10.1103/PhysRevC.73.034901
   %[arXiv:hep-ph/0508277 [hep-ph]].
   %47 citations counted in INSPIRE as of 19 Aug 2022

\bibitem{Bronzan:1974jh}
   J.~B.~Bronzan, G.~L.~Kane and U.~P.~Sukhatme,
   ``Obtaining Real Parts of Scattering Amplitudes Directly from Cross-Section Data Using Derivative Analyticity Relations,''
   Phys. Lett. B \textbf{49}, 272 (1974).
   %doi:10.1016/0370-2693(74)90432-8
   %197 citations counted in INSPIRE as of 23 Sep 2021

\bibitem{Forshaw:2003ki}
   J.~R.~Forshaw, R.~Sandapen and G.~Shaw,
   ``Color dipoles and rho, phi electroproduction,''
   Phys. Rev. D \textbf{69}, 094013 (2004).
   %doi:10.1103/PhysRevD.69.094013
   %[arXiv:hep-ph/0312172 [hep-ph]].
   %145 citations counted in INSPIRE as of 19 Aug 2022

\bibitem{Martin:1980jx}
   A.~Martin;
   ``A FIT of Upsilon and Charmonium Spectra,''
   Phys. Lett. B \textbf{93}, 338 (1980).

\bibitem{Barik:1980ai}
   N.~Barik and S.N.~Jena;
   ``Fine - Hyperfine Splittings Of Quarkonium Levels In An Effective Power Law Potential,''
   Phys. Lett. B \textbf{97}, 265 (1980). 

\bibitem{Buchmuller:1980su} 
   W.~Buchmuller and S.H.H.~Tye;
   ``Quarkonia and Quantum Chromodynamics,''
   Phys. Rev. D \textbf{24}, 132 (1981).  

%%%%%%%%%%%%%%%%%%%%%%%%%%%%%%%%%%%%%%%%%% 40 %%%%%%%%%%%%%%%%%%%%%%%%%%%%%%%%%%%%%%%%%%%%%%%%

\bibitem{GolecBiernat:1998js} 
   K.J.~Golec-Biernat and M.~Wusthoff,
   ``Saturation effects in deep inelastic scattering at low Q**2 and its implications on diffraction,''
   Phys. Rev. D \textbf{59}, 014017 (1998).
  
\bibitem{GolecBiernat:1999qd} 
   K.J.~Golec-Biernat and M.~Wusthoff,
   ``Saturation in diffractive deep inelastic scattering,''
   Phys. Rev. D \textbf{60}, 114023 (1999).

\bibitem{Bartels:2002cj}
   J.~Bartels, K.~J.~Golec-Biernat and H.~Kowalski,
   ``A modification of the saturation model: DGLAP evolution,''
   Phys. Rev. D \textbf{66}, 014001 (2002).
   %doi:10.1103/PhysRevD.66.014001
   %[arXiv:hep-ph/0203258 [hep-ph]].
   %390 citations counted in INSPIRE as of 23 Oct 2020

\bibitem{Ivanov:2002eq}
   Y.~P.~Ivanov, B.~Kopeliovich, A.~Tarasov and J.~Hufner,
   ``Electroproduction of charmonia off protons and nuclei,''
   AIP Conf. Proc. \textbf{660}, 283 (2003).
   %AIP Conf. Proc. \textbf{660}, no.1, 283 (2003).
   %doi:10.1063/1.1570580
   %[arXiv:hep-ph/0212322 [hep-ph]].
   %5 citations counted in INSPIRE as of 06 May 2020

\bibitem{Melosh:1974cu} 
   H.J.~Melosh,
   ``Quarks: Currents and constituents,''
   Phys. Rev. D \textbf{9}, 1095 (1974).

\bibitem{Terentev:1976jk} 
   M.V.~Terentev,
   ``On the Structure of Wave Functions of Mesons as Bound States of Relativistic Quarks,''
   Sov.\ J.\ Nucl.\ Phys.\ \textbf{24}, 106 (1976)
   [Yad.\ Fiz.\ \textbf{24}, 207 (1976)].

\bibitem{Kopeliovich:2015qna} 
   B.Z.~Kopeliovich, E.~Levin, I.~Schmidt and M.~Siddikov,
   ``Lorentz-boosted description of a heavy quarkonium,''
   Phys. Rev. D \textbf{92}, 034023 (2015).

\bibitem{Hufner:1996jw} 
   J.~Hufner, B.~Kopeliovich and A.~B.~Zamolodchikov,
   ``Inelastic J / psi photoproduction off nuclei: Gluon enhancement or double color exchange?,''
   Z. Phys. A \textbf{357}, 113 (1997).
   %doi:10.1007/s002180050222
   %[nucl-th/9607033].
   %doi:10.1007/s002180050222;%%
   %35 citations counted in INSPIRE as of 13 Apr 2020

\bibitem{fg} 
   R.P.~Feynman and A.R.~Gibbs, 
   ``Quantum Mechanics and Path Integrals,'' 
   McGraw-Hill Book Company, NY 1965.

\bibitem{Kopeliovich:2001ee}
   B.~Kopeliovich, A.~Tarasov and J.~Hufner,
   ``Coherence phenomena in charmonium production off nuclei at the energies of RHIC and LHC,''
   Nucl. Phys. A \textbf{696}, 669 (2001).
   %doi:10.1016/S0375-9474(01)01220-9
   %[arXiv:hep-ph/0104256 [hep-ph]].
   %116 citations counted in INSPIRE as of 19 Aug 2022

%%%%%%%%%%%%%%%%%%%%%%%%%%%%%%%%%%%%%%%% 50 %%%%%%%%%%%%%%%%%%%%%%%%%%%%%%%%%%%%%%%%%%%%%%

\bibitem{Kopeliovich:2008ek}
   B.~Kopeliovich, J.~Nemchik, I.~Potashnikova and I.~Schmidt,
   ``Gluon Shadowing in DIS off Nuclei,''
   J. Phys. G \textbf{35}, 115010 (2008).
   %doi:10.1088/0954-3899/35/11/115010
   %[arXiv:0805.4613 [hep-ph]].
   %9 citations counted in INSPIRE as of 09 Apr 2020

\bibitem{Krelina:2020ipn}
   M.~Krelina and J.~Nemchik,
   ``Nuclear shadowing in DIS at electron-ion colliders,''
   Eur. Phys. J. Plus \textbf{135}, 444 (2020).
   %doi:10.1140/epjp/s13360-020-00498-2
   %[arXiv:2003.04156 [hep-ph]].
   %2 citations counted in INSPIRE as of 17 Dec 2020

\bibitem{guzey-1}
   V.~Guzey and M.~Zhalov,
   ``Exclusive $J/{\psi}$ production in ultraperipheral collisions at the LHC: constrains on the gluon distributions in the proton and nuclei,''
   JHEP \textbf{10}, 207 (2013).
   %doi:10.1007/JHEP10(2013)207
   %[arXiv:1307.4526 [hep-ph]].

\bibitem{guzey-2}
   V.~Guzey, E.~Kryshen, M.~Strikman and M.~Zhalov,
   ``Evidence for nuclear gluon shadowing from the ALICE measurements of PbPb ultraperipheral exclusive $J/{\psi}$ production,''
   Phys. Lett. B \textbf{726}, 290 (2013).
   %doi:10.1016/j.physletb.2013.08.043
   %[arXiv:1305.1724 [hep-ph]].

\bibitem{Henkels:2020kju}
   C.~Henkels, E.~G.~de Oliveira, R.~Pasechnik and H.~Trebien,
   ``Exclusive photoproduction of excited quarkonia in ultraperipheral collisions,''
   Phys. Rev. D \textbf{102}, 014024 (2020).
   %doi:10.1103/PhysRevD.102.014024
   %[arXiv:2004.00607 [hep-ph]].
   %0 citations counted in INSPIRE as of 20 Jul 2020

\bibitem{Nemchik:1997xb}
   J.~Nemchik, N.~N.~Nikolaev, E.~Predazzi, B.~G.~Zakharov and V.~R.~Zoller,
   ``The Diffraction cone for exclusive vector meson production in deep inelastic scattering,''
   J. Exp. Theor. Phys. \textbf{86}, 1054 (1998).
   %doi:10.1134/1.558573
   %[arXiv:hep-ph/9712469 [hep-ph]].
   %72 citations counted in INSPIRE as of 19 Aug 2022

%
% ...... Data on charmonium production in UPCs ...................................
%

% ...... coherent J/\Psi .............. 200 GeV

\bibitem{Afanasiev:2009hy}
   S.~Afanasiev \textit{et al.} [PHENIX],
   ``Photoproduction of J/psi and of high mass e+e- in ultra-peripheral Au+Au collisions at s**(1/2) = 200-GeV,''
   Phys. Lett. B \textbf{679}, 321 (2009).
   %doi:10.1016/j.physletb.2009.07.061
   %[arXiv:0903.2041 [nucl-ex]].
   %108 citations counted in INSPIRE as of 06 May 2020

% ...... coherent J/\Psi .............. 2.76 TeV

\bibitem{Khachatryan:2016qhq}
   V.~Khachatryan \textit{et al.} [CMS],
   ``Coherent $J/\psi$ photoproduction in ultra-peripheral PbPb collisions at $\sqrt {s_{NN}} =$ 2.76 TeV with the CMS experiment,''
   Phys. Lett. B \textbf{772}, 489 (2017).
   %doi:10.1016/j.physletb.2017.07.001
   %[arXiv:1605.06966 [nucl-ex]].
   %69 citations counted in INSPIRE as of 17 Apr 2020
  
\bibitem{Abelev:2012ba}
   B.~Abelev \textit{et al.} [ALICE],
   ``Coherent $J/\psi$ photoproduction in ultra-peripheral Pb-Pb collisions at $\sqrt{s_{NN}} = 2.76$ TeV,''
   Phys. Lett. B \textbf{718}, 1273 (2013).
   %doi:10.1016/j.physletb.2012.11.059
   %[arXiv:1209.3715 [nucl-ex]].
   %220 citations counted in INSPIRE as of 17 Apr 2020
  
% ...... coherent + incoherent J/\Psi .............. 2.76 TeV

\bibitem{Abbas:2013oua}
   E.~Abbas \textit{et al.} [ALICE],
   ``Charmonium and $e^+e^-$ pair photoproduction at mid-rapidity in ultra-peripheral Pb-Pb collisions at $\sqrt{s_{\rm NN}}$=2.76 TeV,''
   Eur. Phys. J. C \textbf{73}, 2617 (2013).
   %doi:10.1140/epjc/s10052-013-2617-1
   %[arXiv:1305.1467 [nucl-ex]].
   %209 citations counted in INSPIRE as of 17 Apr 2020
  
%%%%%%%%%%%%%%%%%%%%%%%%%%%%%%%%%%%%%%%% 60 %%%%%%%%%%%%%%%%%%%%%%%%%%%%%%%%%%%%%%%%%%%%%%

% ...... coherent \Psi' .............. 2.76 TeV

\bibitem{Adam:2015sia}
   J.~Adam \textit{et al.} [ALICE],
   ``Coherent $\psi$(2S) photo-production in ultra-peripheral Pb Pb collisions at $\sqrt{s}_{\rm NN}$ = 2.76 TeV,''
   Phys. Lett. B \textbf{751}, 358 (2015).
   %doi:10.1016/j.physletb.2015.10.040
   %[arXiv:1508.05076 [nucl-ex]].
   %38 citations counted in INSPIRE as of 17 Apr 2020

% ...... coherent J/\Psi .............. 5.02 TeV

\bibitem{Acharya:2019vlb}
   S.~Acharya \textit{et al.} [ALICE],
   ``Coherent J/$\psi$ photoproduction at forward rapidity in ultra-peripheral Pb-Pb collisions at $\sqrt{s_{\rm{NN}}}=5.02$ TeV,''
   Phys. Lett. B \textbf{798}, 134926 (2019).
   %doi:10.1016/j.physletb.2019.134926
   %[arXiv:1904.06272 [nucl-ex]].
   %11 citations counted in INSPIRE as of 17 Apr 2020

\bibitem{ALICE:2021gpt}
   S.~Acharya \textit{et al.} [ALICE],
   ``Coherent $\rm{J/\psi}$ and $\rm{\psi'}$ photoproduction at midrapidity in ultra-peripheral Pb-Pb collisions at $\sqrt{s_{\mathrm{NN}}}~=~5.02$ TeV,''
   Eur. Phys. J. C \textbf{81}, 712 (2021).
   %doi:10.1140/epjc/s10052-021-09437-6
   %[arXiv:2101.04577 [nucl-ex]].
   %24 citations counted in INSPIRE as of 21 Jul 2022

% ...... coherent J/\Psi .............. 5.02 TeV

\bibitem{LHCb:2018ofh}
   A.~Bursche [LHCb],
   ``Study of coherent $J/\psi$ production in lead-lead collisions at $\sqrt{s_{\rm NN}} =5\ \rm{TeV}$ with the LHCb experiment,''
   Nucl. Phys. A \textbf{982}, 247 (2019).
   %doi:10.1016/j.nuclphysa.2018.10.069
   %19 citations counted in INSPIRE as of 17 Apr 2020

% ...... coherent J/\Psi plus \Psi' .............. 5.02 TeV

\bibitem{LHCb:2022ahs}
    R.~Aaij \textit{et al.} [LHCb],
    ``Study of coherent charmonium production in ultra-peripheral lead-lead collisions,''
    [arXiv:2206.08221 [hep-ex]].
    %1 citations counted in INSPIRE as of 28 Jul 2022











\end{thebibliography}
\end{document}
%%%%%%%%%%%%%%%%%%%%%%%%%%%%%%%%%%%%%%%%%%%%%%%%%%%%%%%%%%%%%%%%%%%%%%%%%%%%%%%%%%%%%%%%%%%
%%%%%%%%%%%%%%%%%%%%%%%%%%%%%%%%%%%%%%%%%%%%%%%%%%%%%%%%%%%%%%%%%%%%%%%%%%%%%%%%%%%%%%%%%%%