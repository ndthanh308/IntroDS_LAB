\subsubsection*{Ground state sector Hilbert space dimension}
\label{HilbertSpaceDim}
We now want to study the dimension of the ground state sector Hilbert space from the bulk perspective. In particular, we want to verify that the gravitational partition function computed using the Gibbons-Hawking prescription behaves like a partition function of some dual quantum system. The specific aspect we want to see explicitly is the relation $Z_{\text{grav}}(\beta = \infty) = \dim \mathcal{H}_{\text{BPS}}$. This means that we should be able to construct a set of orthogonal states of size $Z_{\text{grav}}(\infty)$ that will span the full Hilbert space of the ground state sector. We view this as a check of the holographic aspect of Gibbons-Hawking prescription \JB{or GKPW?}: while the Euclidean gravitational path integral computes an object called $Z_{\text{grav}}(\beta)$ by definition, it is not apriori obvious that it behaves like $\tr(e^{-\beta H})$ of some quantum system. This is something that is usually provided to us by the holographic principle. Here we will see how by just using the Gibbons-Hawking prescription we can verify some quantum aspects of Euclidean gravitational path integral without the need to refer to a dual boundary quantum system.
% Naively one could think it could be enough to compute the disk partition function at zero temperature, since then $Z_{\text{disk}} = \braket{\text{TFD}(\infty)} = \text{number of BPS states}$. 
% This, however, requires using the appropriate normalization of the thermofield double state, which is only clear from the boundary side. From only the bulk side we do not know how to apriori normalize the Hartle-Hawking wavefunctions in such a way that they reproduce the properly normalized disk partition function, which is why we need to look for a different way of finding the Hilbert space dimension. 
\\ 
In other words, we want to ask what is the minimal size $K$ of the set of states $\{ q_i \}_{i=1,\dots,K}$ created by some local operators with scaling dimension $\Delta$, such that for any state $\ket{\psi}$ we have
\be  
\ket{\psi} = \sum_{i=1}^K f_i \ket{q_i} \equiv \ket{q_f} ,
\ee
for an appropriate choice of coefficients $f_i$. In order to do that, let us consider the density matrix 
\be 
\rho_\text{mixed BPS} = \sum_{i} \ket{q_i} \bra{q_i}
\ee
The rank of this matrix captures the size of the BPS Hilbert space
\be
\mathrm{rank}(\rho_\text{mixed BPS})= :\lim_{n \rightarrow 0}\tr \left(\rho_\text{mixed BPS}\right)^n:
\ee
where $:$ denotes at which stage we take disorder average if the boundary system in question would correspond to something like SYK. But notice that 
\be 
\tr \left(\rho_\text{mixed BPS}\right)^n = \sum_{i_n=1}^K \bra{q_{i_n}}\ket{q_{i_1}}\bra{q_{i_1}}\ket{q_{i_2}}\cdots\bra{q_{i_{n-1}}}\ket{q_{i_n}}
\ee
This can be calculated using the gravity analog of resolvent technique, first used in this context in \cite{Penington:2019kki}. Let $M_{ij} \equiv \braket{q_i}{q_j}$ denote the matrix of overlaps of the basis vectors $\ket{q_i}$. We introduce the resolvent matrix 
\be  
\mathbf{R}_{ij}(\lambda) = \left( 
\frac{1}{\lambda  - M}
\right)_{ij} 
= \frac{\delta_{ij}}{\lambda} + \frac{1}{\lambda} \sum_{n=1}^\infty \frac{(M^n)_{ij}}{\lambda^n}
,
\ee
The Schwinger-Dyson equation is given by
\be
\mathbf{R}_{ij}(\lambda) = \frac{\delta_{ij}}{\lambda} + \frac{1}{\lambda} \sum_{n=1}^\infty Z_n^\mathcal{O} R^{n-1} \mathbf{R}_{ij}(\lambda)\label{SD}
\ee
where we denote $R\equiv \Tr \mathbf{R}$. If we take the trace of the above equation, we get
\be  
R(\lambda) = \frac{K}{\lambda} + \frac{1}{\lambda} \sum_{n=1}^\infty Z_n^\mathcal{O} R^n \label{traceSD}
\ee
Now recall that  $Z_n^\mathcal{O}$ is the pinwheel geometry with $n$ boundaries with two operator insertions on each boundary. It has a simple form 
\be  
Z_n^\mathcal{O} =
e^{2\mathbf{S}_j} \, y^n ,
\qquad 
y \equiv e^{-S_0}
\frac{\Delta \Gamma(\Delta)^2 \,\Gamma\left(\Delta+\frac{1}{2}\pm j\right)}{2\pi \Gamma(2\Delta)} ,
\qquad
e^{\mathbf{S}_j} \equiv e^{S_0} \cos(\pi j ) .
\ee
so (\ref{traceSD}) can be simplified to
\be
R(\lambda)=  \frac{K}{\lambda} + \frac{e^{2\mathbf{S}_j}}{\lambda} \frac{R y}{1-R y}
\ee
Solving for $R(\lambda)$ in the above equation and choose that solution that matches the asymptotic $1/\lambda$ behavior, we get
\be  
R(\lambda) = \frac{1}{2y} + \frac{K - e^{2\mathbf{S}_j}}{2\lambda} - \frac{\sqrt{(\lambda-\lambda_+)(\lambda-\lambda_-)}}{2 \lambda y} , \qquad \lambda_\pm = y (\sqrt{K}\pm e^{\mathbf{S}_j})^2 ,
\ee
as well as the eigenvalue density $D(\lambda) = \frac{1}{2\pi i}(R(\lambda - i \epsilon)-R(\lambda + i \epsilon))$
\be  
D(\lambda) = \frac{\sqrt{(\lambda_+ - \lambda)(\lambda - \lambda_-)}}{2\pi \lambda y}  + 
\delta(\lambda) (K - e^{2\mathbf{S}_j}) \theta(K-e^{2\mathbf{S}_j}) 
.
\ee
But now we know that
\be 
:\tr \left(\rho_\text{mixed BPS}\right)^n: = \frac{1}{2\pi i}\oint d \lambda\, \lambda^{n} R(\lambda)
\ee
so the dimension of Hilbert space of given by
\be
\mathrm{rank}(\rho_\text{mixed BPS})= \frac{1}{2\pi i} 
 \oint d\lambda \,  R = \int_{\lambda_-}^{\lambda_+}d\lambda\, D(\lambda)=\begin{cases}K&K<e^{2\mathbf{S}_j}\\e^{2\mathbf{S}_j}&K>e^{2\mathbf{S}_j}\end{cases}
\ee
and in general
\be 
:\tr \left(\rho_\text{mixed BPS}\right)^n: = \int_{\lambda_-}^{\lambda_+} d\lambda \lambda^{n-1} \frac{\sqrt{(\lambda_+ - \lambda)(\lambda - \lambda_-)}}{2\pi y} = y^{n}K^{n+1} \int_{\tilde \lambda_-}^{\tilde \lambda_+} d\tilde \lambda \tilde \lambda^{n-1} \frac{\sqrt{(\tilde \lambda_+ - \tilde \lambda)(\tilde \lambda - \tilde \lambda_-)}}{2\pi }
\ee
where 
\be 
\tilde \lambda_\pm = \left(1\pm \frac{e^{\mathbf{S}_j}}{\sqrt{K}}\right)^2
\ee

\subsubsection*{Standard deviation of reconstruction}
The calculation of section \ref{HilbertSpaceDim} can have error from wormholes. In order to quantify this, let us calculate the standard deviation of the dimension of Hilberspace, i.e.
\be
\lim_{n,m\rightarrow0}:\mathrm{tr}(\rho_\text{mixed BPS})^n\mathrm{tr}(\rho_\text{mixed BPS})^m:
\ee
To illustrate the possible connected contributions, let us calculate an example 
\be
: \mathrm{tr}(\rho_\text{mixed BPS})^2\mathrm{tr}(\rho_\text{mixed BPS})^2:=:\bra{q_i}\ket{q_j}\bra{q_j}\ket{q_i}\bra{q_k}\ket{q_l}\bra{q_l}\ket{q_k}:
\ee
\begin{align}
:% Figure removed:&=% Figure removed \quad K^4e^0\nonumber\\
&+% Figure removed \quad K^2e^0\nonumber\\
&+% Figure removed \quad\quad\quad\quad\quad K^3e^{-2S_0}
\end{align}
From this, we perhaps conclude that 
\be 
\sqrt{:\tr \left(\rho_\text{mixed BPS}\right)^n \,\tr \left(\rho_\text{mixed BPS}\right)^n: - :\tr \left(\rho_\text{mixed BPS}\right)^n:\,:\tr \left(\rho_\text{mixed BPS}\right)^n:}\, \sim\, y^{n} K^{n}
\ee
Generically, subleading compared to the leading piece by $e^{-4S_0}$
\subsubsection*{non-planar}
\subsubsection*{Notes on sources of errors}
Lets discuss now in more detail the approximations used in the above derivation.
\begin{itemize}
    \item \textbf{Non-planar geometries and nontrivial contractions:} In the resolvent method we have summed only over planar geometries, in the sense that the cylinders never had to cross "under" each other to connect to other boundaries. In principle, we should also include non-planar geometries. These however will be suppressed in $1/K$ compared to the leading contribution with the same number of boundaries, because such geometries lead to a structure of index contractions which contain smaller number of index loops. Similarly, we can imagine fully connected configurations in which the bulk geodesics connecting the operators form a non-planar configuration. An example would be pinwheel with four boundaries, where two geodesics connect at the front, and two other connect in the back. These configurations also contain less index loops and therefore will also be suppressed in $1/K$. Including these geometries, would lead to $O(1)$ corrections to the Hilbert space dimension
    \be  
\dim \mathcal{H}^j_{\text{BPS}} = e^{2S_0} \cos^2(\pi j) (1+ O(e^{-2S_0})) .
    \ee

    \item \textbf{Higher genus:} no trumpet 

    \item \textbf{Average of ratio and square root:}


\end{itemize}



\subsubsection*{trumpet is zero and remains zero}
In non-supersymmetric case, trumpet partition function is given by
\be
Z_{\text{Trumpet}}(\beta,b)=\frac{e^{-\frac{b^2}{2\beta}}}{\sqrt{2\pi\beta}}=\int_0^\infty dE\,\frac{\cos(b\sqrt{2E})}{\pi\sqrt{2E}}e^{-\beta E}
\ee
Density of states has support when $E\rightarrow0$. When the trumpet is glued to other polygons 
\be
\int db\,Z_{\text{Trumpet}}(\beta,b)P_k(b)\sim\beta^{k/2}
\ee
where $P_k(b)$ is a polynomial in terms of $b$ with degree $k$. Thus in this case, in the zero temperature limit (i.e. $\beta\rightarrow\infty$) even if $Z_{\text{Trumpet}}=0$, after glued to a polygon it would not remain zero.

This would work however, if the density of states do not have support all the way up to $E\rightarrow0$. In microcanonical ensemble, write
\be
\rho(E,b)=\Theta(E-E_\text{gap})f(E,b)
\ee
then in this case if we glue some polygons we get the integral
\be
\int db\,\Theta(E-E_\text{gap})f(E,b)P_k(b)=F(E)\Theta(E-E_\text{gap})
\ee
Then in canonical ensemble
\be
\int dE\,e^{-\beta E}F(E)\Theta(E-E_\text{gap})\sim e^{-\beta E_\text{gap}}f(\beta)
\ee
so as $\beta\rightarrow\infty$ the above gives zero.


\subsubsection*{Corrections to the degeneracy} 
\JB{To explain better. We allow for multiple defects if there is a weighted geodesic in between them.}
Because the contributions from multiple defects don't preserve supersymmetry, and therefore do not contribute to the ground state sector, we can very simply incorporate the SUSY conical defects correction into our resolvent analysis. A single SUSY defect of weight $w_i$ will modify the pinwheel result as 
\be  
Z_{j,n}^{\mathcal{O},\text{def}} = w_i \, e^{2\textbf{S}_j} y^n .
\ee
This follows from the fact that we're gluing together two different polygons, one with a defect and one without a defect. It is easy to see that the defect polygon that properly reproduces SUSY defect partition function is simply a disk polygon with an additional factor of $w_i$.
The total pinwheel result will now be given by a sum over SUSY conical defects of different flavors 
\be  
Z_{j,n}^{\mathcal{O}} \rightarrow 
Z_{j,n}^{\mathcal{O},\text{total}} =  
Z_{j,n}^{\mathcal{O}} \left(1+\sum_{\text{defects}} w_i \right)^2.
\ee
This modifies the total resolvent as ($W\equiv \sum w_i$)
\begin{align} 
R_{\text{total}}(\lambda) &= \frac{1}{2y} + \frac{K - e^{2\mathbf{S}_j}\left(1+W \right)^2}{2\lambda} - \frac{\sqrt{(\lambda-\lambda_+)(\lambda-\lambda_-)}}{2 \lambda y} , 
\\ 
\lambda_\pm &= y \left[\sqrt{K}\pm e^{\mathbf{S}_j}(1+W) \right]^2 
,
\end{align}
leading to the total rank of matrix $M$ 
\be
\mathrm{rank}(\rho_\text{mixed BPS})= \frac{1}{2\pi i} 
 \oint d\lambda \,  R = \int_{\lambda_-}^{\lambda_+}d\lambda\, D(\lambda)=
\begin{cases}
K&K<e^{2\mathbf{S}_j}(1+W)^2
\\
e^{2\mathbf{S}_j}(1+W)^2&K>e^{2\mathbf{S}_j}(1+W)^2
\end{cases} 
.
\ee

\subsection{reconstructing states}
This question has been previously studied in nonsupersymmetric case in \cite{Hsin:2020mfa}. Here, we closely follow their procedure. The question of how much of the state we can optimally reconstruct with given basis of size $K$ can be rephrased in terms of a maximization problem of the normalized overlap 
\be  
: \text{max}_{f_i} \frac{\braket{q_f}{\psi}}{\sqrt{\braket{\psi}\braket{q_f}}} :,
\ee
where $:$ denotes at which stage we take disorder average if the boundary system in question would correspond to something like SYK. After maximizing with respect to coefficients $f_i$ subject to the normalization constraint $\braket{q_f}=1$ \cite{Hsin:2020mfa} found that the optimal overlap reduces to 
\be  
: \text{max}_{f_i} \frac{\braket{q_f}{\psi}}{\sqrt{\braket{\psi}\braket{q_f}}} : 
=
\frac{:\sqrt{VM^{-1}V^\dag }:}{\sqrt{\braket{\psi}}} ,
\ee
where $M_{ij} \equiv \braket{q_i}{q_j}$ denotes the matrix of overlaps of the basis vectors $\ket{q_i}$, and $V_i \equiv \braket{\psi}{q_i}$. The square root in the expression above might in general be problematic (since the averaging is taken outside of it), however in our case we are interested in studying the above overlap close to 1. One can therefore argue that it's enough to study for what $K$ the quantity under square root is equal to one, which will then imply that the overlap itself is also equal to 1. We are then interested in the following question:
\be  
\text{For what minimal $K$ is    } \frac{:VM^{-1}V^\dag :}{\braket{\psi}} \text{    equal to 1?} 
\ee
We will study this quantity using the gravity analog of resolvent technique, first used in this context in \cite{Penington:2019kki}. We introduce the resolvent matrix 
\be  
\mathbf{R}_{ij} = \left( 
\frac{1}{\lambda  - M}
\right)_{ij} 
= \frac{\delta_{ij}}{\lambda} + \frac{1}{\lambda} \sum_{n=1}^\infty \frac{(M^n)_{ij}}{\lambda^n}
,
\ee
in terms of which we can express 
\be  
: V M^n V^\dag: = \frac{1}{2\pi i} \oint d\lambda \, \lambda^n \, :V \mathbf{R}_{ij} V^\dag_j: \, \,  .
\ee
To compute the expression under the integral we first organize the infinite expansion of the resolvent in terms of the number of boundaries the $:\braket{\psi}{q_1}\dots \braket{q_n}{\psi}:$ cylinder connects to. Using that we choose all the operators in the basis to have the same scaling dimension $\Delta$, we can express the resulting Schwinger-Dyson equation in terms of the trace of the resolvent $R\equiv \Tr \mathbf{R}$ as 
\be  
 :V \mathbf{R}_{ij} V^\dag_j: = \sum_{n=0}^\infty R^{n+1} Z_{n+2}^{\psi} ,
\ee
where  $Z_{n+2}^\psi$ is the pinwheel geometry with $n+2$ boundaries with two operator insertions on each boundary; note that the first and the last boundary contain a single $\mathcal{O}_\psi$ operator insertion of the state we want to reconstruct, while the rest of the operators are the basis ones with which we want to reconstruct $\ket{\psi}$. Working with the case $\Delta_\psi = \Delta$, the pinwheel has a simple form 
\be  
Z_n^\psi =Z_n^\mathcal{O} =
e^{2\mathbf{S}_j} \, y^n ,
\qquad 
y \equiv e^{-S_0}
\frac{\Delta \Gamma(\Delta)^2 \,\Gamma\left(\Delta+\frac{1}{2}\pm j\right)}{2\pi \Gamma(2\Delta)} ,
\qquad
e^{\mathbf{S}_j} \equiv e^{S_0} \cos(\pi j ) ,
\ee
where we also denoted the pinwheel with only $\mathcal{O}_{q_i}$ operator insertions as $Z_n^\mathcal{O}$. With this simple expression we can resum the series to get 
\be  
 :V \mathbf{R}_{ij} V^\dag_j: = 
 e^{2 \mathbf{S}_j} \frac{R y^2}{1-R y},
\ee
so the problem of finding the resolvent reduced to finding its trace $R(\lambda)$. To determine the trace, we can now write down another Schwinger-Dyson equation for the trace
\be  
R(\lambda) = \frac{K}{\lambda} + \frac{1}{\lambda} \sum_{n=1}^\infty Z_n^\mathcal{O} R^n 
=  \frac{K}{\lambda} + \frac{e^{2\mathbf{S}_j}}{\lambda} \frac{R y}{1-R y} ,
\ee
which determines the $R(\lambda)$ as (we choose the sign to match the asymptotic $1/\lambda$ behavior) 
\be  
R(\lambda) = \frac{1}{2y} + \frac{K - e^{2\mathbf{S}_j}}{2\lambda} - \frac{\sqrt{(\lambda-\lambda_+)(\lambda-\lambda_-)}}{2 \lambda y} , \qquad \lambda_\pm = y (\sqrt{K}\pm e^{\mathbf{S}_j})^2 ,
\ee
as well as the eigenvalue density $D(\lambda) = \frac{1}{2\pi i}(R(\lambda - i \epsilon)-R(\lambda + i \epsilon))$
\be  
D(\lambda) = \frac{\sqrt{(\lambda_+ - \lambda)(\lambda - \lambda_-)}}{2\pi \lambda y}  + 
\delta(\lambda) (K - e^{2\mathbf{S}_j}) \theta(K-e^{2\mathbf{S}_j}) 
.
\ee
The final expression can now be written as
\be  
: V M^n V^\dag: = \frac{1}{2\pi i} \oint d\lambda \, \lambda^n \, y \, (\lambda R- K) .
\ee
% To evaluate it we first deform the contour to avoid the branch cut that develops as we analytically continue to $n = -1$. 
To evaluate this we deform the integral away from zero towards the branch cut between $\lambda_-$ and $\lambda_+$.
This leads to the reconstruction rate (recall that $\langle \text{2pt} \rangle_{\text{disk}} = e^{2\mathbf{S}_j} y$)
\be  
\frac{: V M^{-1} V^\dag:}{\braket{\psi}} = \frac{y}{e^{2\mathbf{S}_j} y} \int_{\lambda_-}^{\lambda_+} d\lambda \, D(\lambda) = 
\begin{cases}
\, \, K e^{-2 S_0}/[\cos^2(\pi j)] \, ,  \qquad K<e^{2\mathbf{S}_j} ,
\\ 
\, \, 1   \,  \qquad \qquad \qquad \qquad    K>e^{2\mathbf{S}_j}.
\end{cases}
\ee
We therefore see that for a basis of size $K \geq e^{2S_0} \cos^2(\pi j)$ we can always reconstruct any other excitation in the bulk, after appropriately choosing coefficients $f_i$.
The computation was exactly analogous to the fixed energy case in non-supersymmetric JT \cite{Hsin:2020mfa}. This makes sense because here by taking an infinite Euclidean time evolution the operators are projected into the fixed energy ground state sector.

\subsubsection*{Guessing the polygon}
We write the Hartle-Hawking wavefunction as 
\be  
\Psi_{12}^j(\ell_{12}, a_{12}) = e^{\ii j a_{12}} \frac{2 \cos (\pi j)}{\pi} e^{-\ell_{12}/2} 
\left[ 
\xi_{12} \, e^{- \frac{\ii a_{12}}{2}} K_{\frac{1}{2}+j} (2 e^{-\ell_{12}/2})
+
\eta_{12} \, e^{\frac{\ii a_{12}}{2}} K_{\frac{1}{2}-j} (2 e^{-\ell_{12}/2}) 
\right]
,
\ee
where $\xi_{12}$ and $\eta_{12}$ are related to the fermionic superpartners of the length variable $\ell_{12}$ as 
\be 
\eta_{12} = \frac{-\ii}{\sqrt{2}} \Bar{\psi}_r , \qquad \xi_{12} = -\frac{1}{\sqrt{2}} \Bar{\psi}_l 
.
\ee
For the integration measure we'll choose for now 
\be  
\int d \mu =
\frac{1}{2 \hat{q}} \int_{-\infty}^{\infty} d\ell \, \int_{0}^{2\pi \hat{q}} da \, \int d\eta \, \int  d\xi .
\ee
We also need the Hartle-Hawking wavefunction in the opposite direction, which is given by 
\be  
\Psi_{21}^j(\ell_{12}, a_{12}) =  e^{-\ii j a_{12}} \frac{2 \cos (\pi j)}{\pi} e^{-\ell_{12}/2} 
\left[ 
\eta_{12} \, e^{ \frac{\ii a_{12}}{2}} K_{\frac{1}{2}+j} (2 e^{-\ell_{12}/2})
-
\xi_{12} \, e^{\frac{-\ii a_{12}}{2}} K_{\frac{1}{2}-j} (2 e^{-\ell_{12}/2}) 
\right]
,
\ee
and we work with $R$-charge $j$ quantized in units $1/\hat{q}$.
\\
\\
With these definitions, the partition function at zero temperature can be computed as 
\begin{align}
Z_j &=  \frac{e^{S_0}}{2}   \int d\ell_{12} \, da_{12} \, d\eta_{12} \, d\xi_{12} \, \Psi^j_{12}   \Psi^j_{21} 
\\
&= e^{S_0} \frac{4 \cos^2(\pi j)}{\pi} 
 \int d\ell_{12} \, 
e^{-\ell_{12}} \left[ 
K_{\frac{1}{2}+j} (2 e^{-\ell_{12}/2})^2 + 
K_{\frac{1}{2}-j} (2 e^{-\ell_{12}/2})^2
\right] 
\\
&= e^{S_0} \cos (\pi j) ,
\end{align}
where in the last line we have used 
\be
\frac{4}{\pi}\int d\ell \, e^{-\ell} K_{\frac{1}{2}-j} (2e^{-\ell})^2 
=  \frac{\frac{1}{2} -j}{\cos (\pi j)} , 
\qquad 
\frac{4}{\pi}\int d\ell \, e^{-\ell} K_{\frac{1}{2}+j} (2e^{-\ell})^2 
= \frac{\frac{1}{2} +j}{\cos (\pi j)} 
,
\ee
which follows from 
\be 
\int  d\ell \, e^{-\Delta \ell} K_{2\ii s_1} (2 e^{-\ell/2}) K_{2\ii s_2} (2 e^{-\ell/2}) = 
\frac{\Gamma(\Delta \pm \ii s_1 \pm \ii s_2)}{4 \, \Gamma(2\Delta)} 
.
\ee
Analogously we can compute the 2pt function on a disk
\be  
\langle 2\text{pt} \rangle_{j, \, \text{disk}} = 
e^{S_0} \int d\mu_{12} \Psi_{12}^j \Psi^j_{21} e^{-\Delta \ell_{12}}
.
\ee
The integral can be evaluated explicitly as 
\begin{align} 
\int d\mu_{12}  \,  \Psi^j_{12}  \Psi^j_{21}  e^{-\Delta \ell_{12}} 
&= \frac{4 \cos^2(\pi j)}{\pi} 
 \int d\ell_{12} \, 
e^{-\ell_{12}} \left[ 
K_{\frac{1}{2}+j} (2 e^{-\ell_{12}/2})^2 + 
K_{\frac{1}{2}-j} (2 e^{-\ell_{12}/2})^2
\right] e^{-\Delta \ell_{12}}  
\\
&= \frac{\cos^2(\pi j)}{2\pi} \frac{\Delta \Gamma(\Delta)^2}{\Gamma(2\Delta)} \Gamma\left(\Delta+\frac{1}{2}\pm j\right) ,
\end{align}
which leads to 
\be  
\langle 2\text{pt} \rangle_{j, \, \text{disk}} = 
\frac{e^{S_0} \cos^2(\pi j)}{2\pi} \frac{\Delta \Gamma(\Delta)^2}{\Gamma(2\Delta)} \Gamma\left(\Delta+\frac{1}{2}\pm j\right)
.
\ee
\\
\\
To define the polygon, we'll also define a rescaled HH wavefunction 
\be  
\Tilde{\Psi}^j_{12} = 
\frac{\Psi^j_{12}}{\cos(\pi j)} 
,
\ee
which satisfies 
\be  
\int d\mu_{12} \, \Psi^j_{12} \Tilde{\Psi}^{j'}_{21} = \delta_{j\, j'} .
\ee
The triangle can be now simply defined as (note the opposite orientation to HH)
\be 
I(3,2,1)  = \sum_{j'} \cos(\pi {j'})  \Tilde{\Psi}^{j'}_{32} \Tilde{\Psi}^{j'}_{21} 
\Tilde{\Psi}^{j'}_{13}.
\ee
We can easily verify that it properly reproduces the partition function as 
\be  
Z_j = e^{S_0} \int d\mu_{12} \int d\mu_{23} \int d\mu_{31} \,
I(3,2,1) \Psi^j_{12} \Psi^j_{23} \Psi^j_{31} .
\ee
It also satisfies the composition property as
\begin{align}
\int d\mu_{23} \, I(3,2,1) I(2,3,4) &= \sum_{j, \, j'} \cos(\pi j) \cos(\pi j') 
\Tilde{\Psi}^j_{21} \Tilde{\Psi}^{j}_{13}
\Tilde{\Psi}^{j'}_{34} \Tilde{\Psi}^{j'}_{42} 
\int d\mu_{23} \, \Tilde{\Psi}^{j'}_{23} 
\Tilde{\Psi}^{j}_{32} , 
\\
&= \sum_j \cos(\pi j) \Tilde{\Psi}^{j}_{21} \Tilde{\Psi}^{j}_{13}
\Tilde{\Psi}^{j}_{34} \Tilde{\Psi}^{j}_{42} 
\equiv I(2,1,3,4) .
\end{align}
We can therefore define the general polygon as 
\be  
I (i_1,i_2,\dots ,i_n) = \sum_{j'} \cos(\pi {j'}) \,
\Tilde{\Psi}^{j'}_{i_1 \, i_2} \Tilde{\Psi}^{j'}_{i_2 \, i_3} 
\dots \Tilde{\Psi}^{j'}_{i_n \, i_1} ,
\ee
which we'll also denote as 
\be  
I (\mu_{i_1 \, i_2} ,\mu_{i_2 \, i_3} , \dots , \mu_{i_n \, i_1} ) = \sum_{j'} \cos(\pi {j'}) \,
\Tilde{\Psi}^{j'}_{i_1 \, i_2} \Tilde{\Psi}^{j'}_{i_2 \, i_3} 
\dots \Tilde{\Psi}^{j'}_{i_n \, i_1} .
\ee

\subsubsection*{Ground state wormhole}
% We can try to compute a two boundary contribution using the above polygon. This would be given by 
% \begin{align}
% Z_{j, \, 2} &=  \int d\mu_{41} \, \Psi^j_{41} \int d\mu_{23}\,  \Psi^j_{23} \int d\mu_{43}  \, I(\mu_{14},\mu_{43},\mu_{32} ,\mu_{34} ) 
% \\
% &=  \sum_{j'} \cos (\pi j') \int d\mu_{41} \, \Psi^j_{41} \int d\mu_{23}\,  \Psi^j_{23} \int d\mu_{43}  \, \tPsi^{j'}_{14} \tPsi^{j'}_{43} \tPsi^{j'}_{32} \tPsi^{j'}_{34}  
% \\
% &= 
%  \int d\mu_{41} \, \Psi^j_{41} \tPsi^j_{14} \int d\mu_{23}\,  \Psi^j_{23}  \tPsi^j_{32} =1
% % \\
% % &= 1
% ,
% \end{align}
% which seems to match the $\Delta \rightarrow 0$ limit of a cylinder two-point function in equation (59) of \cite{LongPaper}. \JB{As pointed out on p.17 in that paper though, this ignores different windings of the geodesic around the wormhole and so we should probably compute a two point function with $\Delta \gg 1$.}
% \\
% \\
We can also use the above expressions to construct a wormhole geometry. In general we would run into a mapping class group issue, however, for large enough operator scaling dimensions $\Delta$ we can approximate the result by the naive computation.
For the two point function on a wormhole geometry the expression would be 
\begin{align}
\langle 2 \text{pt} \rangle_{j, \text{wh}} &= 
\int d\mu_{41} \, \Psi^j_{41} \int d\mu_{23}\,  \Psi^j_{23} \int d\mu_{43}  \, I(\mu_{14},\mu_{43},\mu_{32} ,\mu_{34} ) e^{-\Delta \ell_{34}} 
\\ 
&= 
\sum_{j'} \cos (\pi j') \int d\mu_{41} \tPsi^{j'}_{14} \, \Psi^j_{41} \int d\mu_{23}\,  \Psi^j_{23} \tPsi^{j'}_{32} \int d\mu_{43}  \,  \tPsi^{j'}_{43}  \tPsi^{j'}_{34}  e^{-\Delta \ell_{34}} 
\\ 
&= 
 \cos (\pi j) \int d\mu_{43}  \,  \tPsi^j_{43}  \tPsi^j_{34}  e^{-\Delta \ell_{34}} 
 \\
 &= \frac{e^{-S_0}}{\cos(\pi j)} \langle 2\text{pt} \rangle_{j, \, \text{disk}}
,
\end{align}
and we get
\be 
\langle 2 \text{pt} \rangle_{j, \text{wh}} = 
\frac{\cos(\pi j)}{2\pi} \frac{\Delta \Gamma(\Delta)^2 \Gamma\left(\Delta+\frac{1}{2}\pm j\right)}{\Gamma(2\Delta)}  ,
\ee
which precisely matches the equation (72) of \cite{LongPaper}.
% The integral can be explicitly evaluated as 
% \begin{align} 
% \int d\mu_{43}  \,  \tPsi^j_{43}  \tPsi^j_{34}  e^{-\Delta \ell_{34}} 
% &= \frac{4}{\pi} 
%  \int d\ell_{12} \, 
% e^{-\ell_{12}} \left[ 
% K_{\frac{1}{2}+j} (2 e^{-\ell_{12}/2})^2 + 
% K_{\frac{1}{2}-j} (2 e^{-\ell_{12}/2})^2
% \right] e^{-\Delta \ell_{34}}  
% \\
% &= \frac{1}{2\pi} \frac{\Delta \Gamma(\Delta)^2}{\Gamma(2\Delta)} \Gamma\left(\Delta+\frac{1}{2}\pm j\right) ,
% \end{align}


\subsubsection*{Pinwheel geometry} 
To compute the pinwheel with two operator insertions on each boundary we first glue together two polygons of opposite orientation of size $2n$ along every second geodesic. Every geodesic we glue over will also a factor of $e^{-\Delta \ell_{i\, i+1}}$. Each from the rest of $2n$ geodesics are gonna be glued to the boundary HH wavefunctions $\Psi^j_{i-1 \,i}$, which will give two Kronecker deltas $\delta_{j\,j'} \delta_{j\,j"}$. We will thus be left with $n$ integrals over $\tPsi^{j}_{i \, i+1} \tPsi^{j}_{i+1 \, i} e^{-\Delta \ell_{i\, i+1}} $ times a factor of $\cos^2(\pi j)$ coming from polygon's definition. Including the topological weighting for $n$ boundaries, we are thus left with  
\begin{align} 
Z_{j, \, n}^{\mathcal{O}} &= e^{S_0 (2-n)} \cos^2(\pi j) \left( 
\int d\mu_{i \, i+1} \, \tPsi^{j}_{i \, i+1} \tPsi^{j}_{i+1 \, i} \, e^{-\Delta \ell_{i\, i+1}} 
\right)^n 
\\
&= e^{S_0 (2-n)} \cos^2(\pi j) 
\left(\frac{\langle 2\text{pt} \rangle_{j, \text{wh}}}{\cos(\pi j)} \right)^n 
\\
&= e^{S_0 (2-n)} \cos^2(\pi j) 
\left(
\frac{\Delta \Gamma(\Delta)^2 \,\Gamma\left(\Delta+\frac{1}{2}\pm j\right)}{2\pi \Gamma(2\Delta)}
\right)^n 
.
\end{align}
We can also try to extend the above computation to the case of $p$ operator insertions on each of the boundaries. This can be thought of as constructing the Hilbert space basis in terms of multi-local boundary operators. For simplicity we'll assume that $p$ is of the form $p=4k$. To construct the pinwheel geometry with $p$ operator insertions of each of the $n$ boundaries we start with a single $n$-polygon. On each side of this polygon we'll now glue to it a polygon with four sides, glued along one of the sides with weighting $e^{-\Delta \ell}$. The resulting weighted polygon will now contain $3n$ boundaries
\be  
I_{\Delta} (1,\dots ,3n) = \sum_j \cos(\pi j) \Tilde{\Psi}_{1}^j  \dots \Tilde{\Psi}_{3n}^j \,  \langle \text{2pt} \rangle_{j,\text{wh}}^n  . 
\ee
$2n$ of these boundaries we'll now glue to the boundary HH wavefunctions (of opposite orientation which we'll leave implicit). This will leave us with $n$ boundaries and project us onto the particular $R$-charge sector $j$ 
\be  
\int d\mu_{n+1} \dots d\mu_{3n}  \Psi_{n+1}^j \dots \Psi_{3n}^j \, 
I_{\Delta} (1,\dots ,3n) = \cos(\pi j) \Tilde{\Psi}_{1}^j \dots  \Tilde{\Psi}_{n}^j \,  \langle \text{2pt} \rangle_{j,\text{wh}}^n  . 
\ee
If we now glue the above object to itself along the last three geodesics with weighting $e^{-\Delta \ell}$ we would obtain an $n$-boundary pinwheel with four operator insertions on each boundary. If we want to instead obtain an $n$-boundary pinwheel with $p$ operator insertions on each boundary, we repeat the procedure of gluing $n$ square polygons to each of the $n$-boundaries with $e^{-\Delta \ell}$. Two other boundaries of each of the squares are now glued again to boundary    HH wavefunctions. This whole procedure essentially adds a factor of $\langle \text{2pt} \rangle_{j,\text{wh}}^n$ to our expression, and leaves us with a pinwheel geometry containing 8 operator insertions on each boundary. Repeating this procedure many times will lead us to a general expression 
\be  
\cos(\pi j) \Tilde{\Psi}_{1}^j \dots  \Tilde{\Psi}_{n}^j \,  \langle \text{2pt} \rangle_{j,\text{wh}}^{n k} 
,
\ee
where $k$ is related to the number of boundary operator insertions as $k=\frac{p}{4}$.
Gluing now two copies of these final polygons finally gives us a pinwheel geometry with $4k$ operator insertions on each boundary 
\be  
Z_{j, \, n}^{\mathcal{O},p=4k} = e^{S_0 (2-n)} \cos^2(\pi j) 
\langle 2\text{pt} \rangle_{j, \text{wh}}^{np/2}
% 
\left(\frac{\langle 2\text{pt} \rangle_{j, \text{wh}}}{\cos(\pi j)} \right)^n 
.
\ee
% \\
% \\
% $1/\cos(\pi j)$. We have $n$ such integrals, times a factor of $\cos(\pi j)^2$ coming from the definition of polygons. We therefore find the expression for the $n$-pinwheel geometry as \JB{wrong:}
% \be  
% \mathcal{I}_{j,n} (\mu_{12},\mu_{34} , \dots ,\mu_{2n-1 \, 2n} ) = [\cos(\pi j)]^{2-n} \, \tPsi_{12} \tPsi_{34} \dots \tPsi_{2n-1 \, 2n} .
% \ee
% Since the integrals over $\int d\mu \Psi \tPsi =1$, gluing the pinwheel together we find that the $n$-boundary partition function in a fixed $j$ sector is 
% \be  
% Z_{j, \, n} = [\cos(\pi j)]^{2-n}
% .
% \ee
% \\ 
% With operator insertions
% \be  
% Z_{j,n}^{\mathcal{O}}= [\cos(\pi j)]^{2} 
% \left(\frac{\langle 2\text{pt} \rangle_{j, \text{wh}}}{\cos(\pi j)} \right)^n
%  .
% \ee

\subsection*{Statistics of $f_i$}

Recall that 
\be  
\ket{\psi} = \sum_{i=1}^K f_i \ket{q_i} 
\ee
We want to find the statistics of $f_i$. A similar problem has been previously studied in nonsupersymmetric case in \cite{Hsin:2020mfa}. Here, we closely follow their procedure. The question of how much of the state we can optimally reconstruct with given basis of size $K$ can be rephrased in terms of a maximization problem of the normalized overlap 
\be  
: \text{max}_{f_i} \frac{\braket{q_f}{\psi}}{\sqrt{\braket{\psi}\braket{q_f}}} :,
\ee
After maximizing with respect to coefficients $f_i$ subject to the normalization constraint $\braket{q_f}=1$ \cite{Hsin:2020mfa} found that the optimal overlap happens when
\be  
f=\frac{M^{-1}V}{\sqrt{V^\dagger M^{-1}V}}
\ee
where $M_{ij}=\braket{q_i}{q_j}$ and $V_i=\braket{q_i}{\psi}$. Thus
\be
f_i=\frac{(\braket{q_i}{q_j})^{-1}\braket{q_j}{\psi}}{\sqrt{V^\dagger M^{-1}V}}
\ee
To study its statistics, one thing to look at would be 
\be
:(\braket{q_i}{q_j})^{-1}\braket{q_j}{\psi}:
\ee
or more generally
\be
:(\braket{q_i}{q_j})^n\braket{q_j}{\psi}:
\ee
In terms of the resolvent, we can write this as
\be  
: M^n V: = \frac{1}{2\pi i} \oint d\lambda \, \lambda^n \, :\mathbf{R}_{ij} V_j: \, \,  .
\ee
To compute the expression under the integral we first organize the infinite expansion of the resolvent in terms of the number of boundaries the $:\braket{q_1}{q_2}\dots \braket{q_n}{\psi}:$ cylinder connects to. Using that we choose all the operators in the basis to have the same scaling dimension $\Delta$, we multiply the Schwinger-Dyson equation (\ref{SD}) by $V_j$ on the right to get
\be
\mathbf{R}_{ij}(\lambda)V_j = \frac{V_i}{\lambda} + \frac{1}{\lambda} \sum_{n=1}^\infty Z_n^\mathcal{O} R^{n-1} \mathbf{R}_{ij}(\lambda)V_j
\ee
\CY{not sure how to get something like this here}
\be  
 :\mathbf{R}_{ij} V_j: = \sum_{n=0}^\infty R^{n+1} Z_{n+1}^{\psi} ,
\ee
where  $Z_{n+1}^\psi$ is the pinwheel geometry with $n+1$ boundaries with two operator insertions on each boundary; note that the last boundary contain a single $\mathcal{O}_\psi$ operator insertion of the state we want to reconstruct, while the rest of the operators are the basis ones with which we want to reconstruct $\ket{\psi}$.





 






We also want to compute the following quantity 
\be  
\frac{: (V M^{-1} V^\dag)(V M^{-1} V^\dag):}{(\braket{\psi})^2} = \lim_{n,n' \to -1} \frac{: (V M^{n} V^\dag)(V M^{n'} V^\dag):}{(\braket{\psi})^2}
\ee
Recall that 
in terms of resolvent, this is given by
\be  
: (V M^{-1} V^\dag)(V M^{-1} V^\dag): = \frac{1}{(2\pi i)^2} \oint d\lambda \,d\lambda' \, \lambda^n {\lambda'}^{n'}\, :V \mathbf{R}(\lambda) V^\dag V \mathbf{R}(\lambda') V^\dag:
\ee

%\begin{align}
%:V \mathbf{R}(\lambda) V^\dag V \mathbf{R}(\lambda') V^\dag:&=\sum_{n,n'=0}^\infty R(\lambda)^{n+1}R(\lambda')^{n'+1}Z_{n+n'+4}\\
%&=e^{2\mathbf{S}}\sum_{n,n'=0}^\infty R(\lambda)^{n+1}R(\lambda')^{n'+1}y^{n+n'+4}\\
%&=e^{2\mathbf{S}}\frac{R(\lambda)R(\lambda')y^4}{(1-R(\lambda)y)(1-R(\lambda')y)}
%\end{align}
%Using the Schwinger-Dyson equation
%\be  
%: (V M^{-1} V^\dag)(V M^{-1} V^\dag): = \frac{1}{(2\pi i)^2} \oint d\lambda \,d\lambda' \, \lambda^n {\lambda'}^{n'}\, y^2(\lambda R(\lambda)-K)(\lambda' R(\lambda')-K)
%\ee
Note that the disconnected part is given by
\be
\frac{: V M^{-1} V^\dag ::V M^{-1} V^\dag:}{|\braket{\psi}|^2}=\begin{cases}K^2e^{-4S_0}/\cos^2(\pi j)&K<e^{2\mathbf{S}}\\1&K>e^{2\mathbf{S}}\end{cases}
\ee


\newpage
\section*{List of things to do}

\begin{itemize}
    \item Compute entropy using the replica trick to see if we can solve the problem of entropy becoming negative? \LVI{Maybe this is too hard. Let's focus on the goals below. }
    \item orbifold geometries?
    \item can we say something new about the random matrix universality of the projected operators?
    \item Classify sources of error: non-planar geometries, standard deviation of the result itself, going from average of square root to square root of average, going from average of ratio to ratio of averages
\item Statistics of coefficients $\alpha_i$ when writing $\ket{\psi} = \sum_{i=1}^K \alpha_i \ket{q_i}$.
\item Matter backreaction from one-loop determinant 

\item trumpet and its gluing discussion (why higher genus doesn't contribute) and draw some trumpets

\item Laplace transforms 

\item more general answers for the index contributions (with $\cos (\pi j)$) 

\item conical Hartle Hawking, polygon and planar resummation for the conical HH

    

\end{itemize}


\begin{align}
    \ket{\tilde{q}_1}&=\ket{q_1}\\
    \ket{\tilde{q}_2}&=\ket{q_2}-\frac{\bra{q_2}\ket{\tilde{q}_1}}{\bra{\tilde{q}_1}\ket{\tilde{q}_1}}\ket{\tilde{q}_1}=\ket{q_2}-\frac{\bra{q_2}\ket{\tilde{q}_1}}{\bra{\tilde{q}_1}\ket{\tilde{q}_1}}\ket{q_1}\\
    \ket{\tilde{q}_3}&=\ket{q_3}-\frac{\bra{q_3}\ket{\tilde{q}_1}}{\bra{\tilde{q}_1}\ket{\tilde{q}_1}}\ket{\tilde{q}_1}-\frac{\bra{q_3}\ket{\tilde{q}_2}}{\bra{\tilde{q}_2}\ket{\tilde{q}_2}}\ket{\tilde{q}_2}=\ket{q_3}-\frac{\bra{q_3}\ket{\tilde{q}_2}}{\bra{\tilde{q}_2}\ket{\tilde{q}_2}}\ket{q_2}-\left(\frac{\bra{q_3}\ket{\tilde{q}_1}}{\bra{\tilde{q}_1}\ket{\tilde{q}_1}}-\frac{\bra{q_3}\ket{\tilde{q}_2}\bra{q_2}\ket{\tilde{q}_1}}{\bra{\tilde{q}_2}\ket{\tilde{q}_2}\bra{\tilde{q}_1}\ket{\tilde{q}_1}}\right)\ket{q_1}\\
    \vdots\nonumber
\end{align}
then for $\ket{\tilde{q}_j}$ expansion, the coefficient in front of $\ket{q_i}$ is
\be
-\frac{\bra{q_j}\ket{\tilde{q}_i}}{\bra{\tilde{q}_i}\ket{\tilde{q}_i}}+\frac{\bra{q_j}\ket{\tilde{q}_{i+1}}\bra{q_{i+1}}\ket{\tilde{q}_i}}{\bra{\tilde{q}_{i+1}}\ket{\tilde{q}_{i+1}}\bra{\tilde{q}_i}\ket{\tilde{q}_i}}-\cdots+(-1)^{j-i}\frac{\bra{q_j}\ket{\tilde{q}_{j-1}}\cdots\bra{q_{i+1}}\ket{\tilde{q}_i}}{\bra{\tilde{q}_{j-1}}\ket{\tilde{q}_{j-1}}\cdots\bra{\tilde{q}_i}\ket{\tilde{q}_i}}
\ee


We now want to study the dimension of the ground state sector Hilbert space from the bulk perspective. In particular, we want to verify that the gravitational partition function computed using the Gibbons-Hawking prescription behaves like a partition function of some dual quantum system. The specific aspect we want to see explicitly is the relation $Z_{\text{grav}}(\beta = \infty) = \dim \mathcal{H}_{\text{BPS}}$. This means that we should be able to construct a set of orthogonal states of size $Z_{\text{grav}}(\infty)$ that will span the full Hilbert space of the ground state sector. We view this as a check of the holographic aspect of Gibbons-Hawking prescription \JB{or GKPW?}: while the Euclidean gravitational path integral computes an object called $Z_{\text{grav}}(\beta)$ by definition, it is not apriori obvious that it behaves like $\tr(e^{-\beta H})$ of some quantum system. This is something that is usually provided to us by the holographic principle. Here we will see how by just using the Gibbons-Hawking prescription we can verify some quantum aspects of Euclidean gravitational path integral without the need to refer to a dual boundary quantum system.
% Naively one could think it could be enough to compute the disk partition function at zero temperature, since then $Z_{\text{disk}} = \braket{\text{TFD}(\infty)} = \text{number of BPS states}$. 
% This, however, requires using the appropriate normalization of the thermofield double state, which is only clear from the boundary side. From only the bulk side we do not know how to apriori normalize the Hartle-Hawking wavefunctions in such a way that they reproduce the properly normalized disk partition function, which is why we need to look for a different way of finding the Hilbert space dimension. 
