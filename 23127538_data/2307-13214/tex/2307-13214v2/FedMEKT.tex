%% 
%% Copyright 2007-2020 Elsevier Ltd
%% 
%% This file is part of the 'Elsarticle Bundle'.
%% ---------------------------------------------
%% 
%% It may be distributed under the conditions of the LaTeX Project Public
%% License, either version 1.2 of this license or (at your option) any
%% later version.  The latest version of this license is in
%%    http://www.latex-project.org/lppl.txt
%% and version 1.2 or later is part of all distributions of LaTeX
%% version 1999/12/01 or later.
%% 
%% The list of all files belonging to the 'Elsarticle Bundle' is
%% given in the file `manifest.txt'.
%% 
%% Template article for Elsevier's document class `elsarticle'
%% with harvard style bibliographic references

% \documentclass[preprint,12pt]{elsarticle}
\documentclass[preprint,5p,times,twocolumn,authoryear]{elsarticle}


%% Use the option review to obtain double line spacing
%% \documentclass[preprint,review,12pt]{elsarticle}

%% Use the options 1p,twocolumn; 3p; 3p,twocolumn; 5p; or 5p,twocolumn
%% for a journal layout:
%% \documentclass[final,1p,times]{elsarticle}
%% \documentclass[final,1p,times,twocolumn]{elsarticle}
%% \documentclass[final,3p,times]{elsarticle}
%% \documentclass[final,3p,times,twocolumn]{elsarticle}
%% \documentclass[final,5p,times]{elsarticle}
% \documentclass[final,5p,times,twocolumn]{elsarticle}

%% For including figures, graphicx.sty has been loaded in
%% elsarticle.cls. If you prefer to use the old commands
%% please give \usepackage{epsfig}

%% The amssymb package provides various useful mathematical symbols
\usepackage{amssymb}
\usepackage{amsmath,amsfonts}
\usepackage{algorithm,tabularx} 
\usepackage{algorithmicx}
\usepackage{times}
\usepackage{array}
\usepackage{amsmath}
\usepackage{amssymb}
\usepackage[caption=false,font=normalsize,labelfont=sf,textfont=sf]{subfig}
\usepackage{textcomp}
\usepackage{varwidth}% http://ctan.org/pkg/varwidth
\usepackage{stfloats}
\usepackage{url}
\usepackage{verbatim}
\usepackage{graphicx}
\usepackage{cite}
\usepackage{multirow} %tabllular
\usepackage{booktabs}
\usepackage{epsfig}
\usepackage{xcolor}
\usepackage[noend]{algpseudocode}
%% The amsthm package provides extended theorem environments
%% \usepackage{amsthm}

%% The lineno packages adds line numbers. Start line numbering with
%% \begin{linenumbers}, end it with \end{linenumbers}. Or switch it on
%% for the whole article with \linenumbers.
%% \usepackage{lineno}

% \journal{Nuclear Physics B}

\begin{document}

\begin{frontmatter}

%% Title, authors and addresses

%% use the tnoteref command within \title for footnotes;
%% use the tnotetext command for theassociated footnote;
%% use the fnref command within \author or \address for footnotes;
%% use the fntext command for theassociated footnote;
%% use the corref command within \author for corresponding author footnotes;
%% use the cortext command for theassociated footnote;
%% use the ead command for the email address,
%% and the form \ead[url] for the home page:
%% \title{Title\tnoteref{label1}}
%% \tnotetext[label1]{}
%% \author{Name\corref{cor1}\fnref{label2}}
%% \ead{email address}
%% \ead[url]{home page}
%% \fntext[label2]{}
%% \cortext[cor1]{}
%% \affiliation{organization={},
%%             addressline={},
%%             city={},
%%             postcode={},
%%             state={},
%%             country={}}
%% \fntext[label3]{}
\cortext[cor1]{Corresponding author.}
\title{FedMEKT: Distillation-based Embedding Knowledge Transfer for Multimodal Federated Learning}

%% use optional labels to link authors explicitly to addresses:
%% \author[label1,label2]{}
%% \affiliation[label1]{organization={},
%%             addressline={},
%%             city={},
%%             postcode={},
%%             state={},
%%             country={}}
%%
%% \affiliation[label2]{organization={},
%%             addressline={},
%%             city={},
%%             postcode={},
%%             state={},
%%             country={}}

% \author{}
\author[label1]{Huy Q. Le }
\ead{quanghuy69@khu.ac.kr}
\author[label3]{ Minh N. H. Nguyen }
\ead{nhnminh@vku.udn.vn}
\author[label1]{Chu Myaet Thwal}
\ead{chumyaet@khu.ac.kr}
\author[label2]{Yu Qiao}
\ead{qiaoyu@khu.ac.kr}
\author[label2]{Chaoning Zhang}
\ead{chaoningzhang1990@gmail.com}
\author[label1]{Choong Seon Hong \corref{cor1}}
\ead{cshong@khu.ac.kr}

% \affiliation{organization={},%Department and Organization
%             addressline={}, 
%             city={},
%             postcode={}, 
%             state={},
%             country={}}
\affiliation[label1]{organization={Department of Computer Science and Engineering, Kyung Hee University},%Department and Organization
            % addressline={}, 
            city={Yongin-si},
            postcode={17104}, 
            % state={},
            country={Republic of Korea}}
\affiliation[label2]{organization={Department of Artificial Intelligence, Kyung Hee University},%Department and Organization
            % addressline={}, 
            city={Yongin-si},
            postcode={17104}, 
            % state={},
            country={Republic of Korea}}
\affiliation[label3]{organization={Digital Science and Technology Institute, The University of Danang—Vietnam-Korea University of Information and Communication Technology},%Department and Organization
            % addressline={}, 
            city={Da Nang},
            postcode={550000}, 
            % state={},
            country={Vietnam}}
\begin{abstract}
%% Text of abstract
Federated learning (FL) enables a decentralized machine learning paradigm for multiple clients to collaboratively train a generalized global model without sharing their private data. Most existing works have focused on designing FL systems for unimodal data, limiting their potential to exploit valuable multimodal data for future personalized applications. Moreover, the majority of FL approaches still rely on labeled data at the client side, which is often constrained by the inability of users to self-annotate their data in real-world applications. In light of these limitations, we propose a novel multimodal FL framework based on a semi-supervised learning approach to leverage representations from different modalities. To address the challenges of modality discrepancy and labeled data constraints in existing FL systems,  our proposed FedMEKT comprises local multimodal autoencoder learning, generalized multimodal autoencoder construction, and generalized classifier learning. %Specifically, we develop the multimodal embedding knowledge transfer mechanism in federated learning, namely, FedMEKT, which enables the fused multimodal knowledge exchange between server and clients using a small multimodal proxy dataset.
Bringing this concept into a system, we develop a distillation-based multimodal embedding knowledge transfer mechanism, namely FedMEKT, which allows the server and clients to exchange joint knowledge extracted from a multimodal proxy dataset. Specifically, our FedMEKT iteratively updates the generalized global encoders with joint embedding knowledge from participating clients through upstream and downstream multimodal embedding knowledge transfer. Through extensive experiments on three multimodal human activity recognition datasets,we demonstrate that FedMEKT not only achieves superior global encoder performance in linear evaluation but also guarantees user privacy for personal data and model parameters while demanding less communication cost than other baselines.
\end{abstract}

%%Graphical abstract
% \begin{graphicalabstract}
% %% Figure removed
% \end{graphicalabstract}

%%Research highlights
% \begin{highlights}
% \item Research highlight 1
% \item Research highlight 2
% \end{highlights}

\begin{keyword}
%% keywords here, in the form: keyword \sep keyword

%% PACS codes here, in the form: \PACS code \sep code

%% MSC codes here, in the form: \MSC code \sep code
%% or \MSC[2008] code \sep code (2000 is the default)

\end{keyword}

\end{frontmatter}

%% \linenumbers

%% main text
\section{}
\label{}

%% The Appendices part is started with the command \appendix;
%% appendix sections are then done as normal sections
%% \appendix

%% \section{}
%% \label{}

%% For citations use: 
%%       \citet{<label>} ==> Jones et al. [21]
%%       \citep{<label>} ==> [21]
%%

%% If you have bibdatabase file and want bibtex to generate the
%% bibitems, please use
%%
 \bibliographystyle{elsarticle-num-names} 
 \bibliography{<your bibdatabase>}

%% else use the following coding to input the bibitems directly in the
%% TeX file.

% \begin{thebibliography}{00}

% %% \bibitem[Author(year)]{label}
% %% Text of bibliographic item

% \bibitem[ ()]{}

% \end{thebibliography}
\end{document}

\endinput
%%
%% End of file `elsarticle-template-num-names.tex'.
