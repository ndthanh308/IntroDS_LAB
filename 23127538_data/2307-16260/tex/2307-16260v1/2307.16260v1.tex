\documentclass[12pt]{amsart}
\usepackage{amssymb, eucal, amsfonts, amsmath, xy,latexsym, bbm}
\usepackage{mathrsfs}

%\textwidth = 6.2in \textheight = 9in %\pagestyle{plain}
%\oddsidemargin=0cm \evensidemargin=0cm %\topmargin=0cm
\usepackage[margin=2cm]{geometry}
\usepackage[pdftex,colorlinks,hypertexnames=false,linkcolor=black,urlcolor=black,citecolor=black]{hyperref}

\numberwithin{equation}{section}


\newtheorem{Theorem}{Theorem}[section]
\newtheorem*{Theorem*}{Theorem}
\newtheorem*{Corollary*}{Corollary}
\newtheorem*{Problem}{Problem}
\newtheorem{Lemma}[Theorem]{Lemma}
\newtheorem{Proposition}[Theorem]{Proposition}
\newtheorem{Corollary}[Theorem]{Corollary}
%\newtheorem*{MainTheorem}{First Theorem}
%\newtheorem*{SecondTheorem}{Second Theorem}
%\newtheorem*{ThirdTheorem}{Third Theorem}
\newtheorem*{Conjecture}{Conjecture}

\theoremstyle{definition}
\newtheorem{Definition}[Theorem]{Definition}
\newtheorem{Question}[Theorem]{Question}


\theoremstyle{remark}
\newtheorem{Remark}[Theorem]{Remark}
\newtheorem*{Remark*}{Remark}
\newtheorem{Example}[Theorem]{Example}



% bb letters
\newcommand{\C}{\mathbb{C}}
%\newcommand{\N}{\mathbb{N}}
\newcommand{\Z}{\mathbb{Z}}
\newcommand{\Q}{\mathbb{Q}}
\renewcommand{\k}{\mathbbm{k}}
\renewcommand{\Z}{\mathbb{Z}}
\newcommand{\A}{\mathbb{A}}
\newcommand{\Sb}{\mathbb{S}}




% frak letters
\newcommand{\g}{\mathfrak{g}}
\newcommand{\p}{\mathfrak{p}}
%\renewcommand{\k}{\mathfrak{k}}
\renewcommand{\v}{\mathfrak{v}}
\newcommand{\gl}{\mathfrak{gl}}
\newcommand{\h}{\mathfrak{h}}
\renewcommand{\c}{\mathfrak{c}}
\renewcommand{\t}{\mathfrak{t}}
\renewcommand{\l}{\mathfrak{l}}
\renewcommand{\sl}{\mathfrak{sl}}
\newcommand{\so}{\mathfrak{so}}
\renewcommand{\sp}{\mathfrak{sp}}
\newcommand{\psl}{\mathfrak{psl}}
\newcommand{\z}{\mathfrak{z}}
\newcommand{\m}{\mathfrak{m}}
\newcommand{\n}{\mathfrak{n}}
\newcommand{\Sf}{\mathfrak{S}}
\newcommand{\q}{\mathfrak{q}}
\renewcommand{\b}{\mathfrak{b}}
\newcommand{\HH}{\mathfrak{H}}
\newcommand{\NN}{\mathfrak{N}}


% cal letters
\newcommand{\Nc}{\mathcal{N}}
%\renewcommand{\S}{\mathcal{S}}
\newcommand{\D}{\mathcal{D}}
\renewcommand{\O}{\mathcal{O}}
\newcommand{\W}{W}
\newcommand{\R}{\mathcal{R}}
\newcommand{\B}{\mathcal{B}}
\renewcommand{\SS}{\mathcal{S}}
\newcommand{\Zc}{\mathcal{Z}}
\newcommand{\I}{\mathcal{I}}
\newcommand{\Ac}{\mathcal{A}}
\newcommand{\V}{\mathcal{V}}
\newcommand{\K}{\mathcal{K}}
\newcommand{\M}{{\mathcal{M}}}
\newcommand{\Co}{\mathcal{C}}
\renewcommand{\P}{{\mathcal{P}}}
\renewcommand{\Ac}{\mathcal{A}}
\renewcommand{\H}{\mathcal{H}}
\newcommand{\F}{\mathcal{F}}
\newcommand{\E}{\mathcal{E}}
\renewcommand{\W}{\mathcal{W}}
\newcommand{\LL}{\mathcal{L}}
\newcommand{\N}{{\mathcal{N}}}
% scr letters
\newcommand{\Ss}{\mathscr{S}}
%\newcommand{\Tscr}{\mathscr{T}}
%\newcommand{\SL}{\mathscr{L}}


% operators
\newcommand{\ad}{\operatorname{ad}}
\newcommand{\Ad}{\operatorname{Ad}}
\newcommand{\red}{{\operatorname{red}}}
\newcommand{\ab}{{\operatorname{ab}}}
\newcommand{\Ht}{\operatorname{ht}}
\newcommand{\Lie}{\operatorname{Lie}}
\newcommand{\Ker}{\operatorname{Ker}}
\newcommand{\Id}{\operatorname{Id}}
\renewcommand{\span}{\operatorname{span}}
\newcommand{\im}{\operatorname{Im}}
\newcommand{\Hom}{\operatorname{Hom}}
\newcommand{\SL}{\operatorname{SL}}
\newcommand{\GL}{\operatorname{GL}}
\newcommand{\Spin}{\operatorname{Spin}}
\newcommand{\Sp}{\operatorname{Sp}}
\newcommand{\gr}{\operatorname{gr}}
\renewcommand{\th}{{\operatorname{th}}}
\newcommand{\Spec}{\operatorname{Spec}}
\newcommand{\Specm}{\operatorname{Specm}}
\newcommand{\res}{{\operatorname{res}}}
\newcommand{\Inc}{{\operatorname{Inc}}}
\newcommand{\tor}{{\operatorname{tor}}}
\newcommand{\rank}{{\operatorname{rank}}}
\newcommand{\Out}{\operatorname{Out}}
\newcommand{\Aut}{\operatorname{Aut}}
\newcommand{\Inn}{\operatorname{Inn}}
\newcommand{\End}{\operatorname{End}}
\newcommand{\Rees}{\operatorname{Rees}}
\newcommand{\Fun}{\operatorname{Fun}}
\newcommand{\Mat}{\operatorname{Mat}}
\newcommand{\chr}{\operatorname{char}}
\newcommand{\Cas}{\operatorname{Cas}}
\newcommand{\Ind}{\operatorname{Ind}}
\newcommand{\trdeg}{\operatorname{trdeg}}
\newcommand{\Sing}{\operatorname{Sing}}
\newcommand{\Comp}{\operatorname{Comp}}
\newcommand{\Der}{\operatorname{Der\,}}
%\renewcommand{\Der}{\operatorname{End}}
\newcommand{\Max}{\operatorname{Max}}
\newcommand{\Char}{\operatorname{char}}
\renewcommand{\mod}{\operatorname{-mod}}
\newcommand{\Mod}{\operatorname{-Mod}}
\newcommand{\Ann}{\operatorname{Ann}}
\newcommand{\Rad}{\operatorname{Rad}}
\newcommand{\Frac}{\operatorname{Frac}}
\newcommand{\Ham}{\operatorname{Ham}}
\newcommand{\Prim}{\operatorname{Prim}}
\newcommand{\Ass}{\operatorname{Ass}}
\newcommand{\op}{{\operatorname{op}}}
\newcommand{\reg}{{\operatorname{reg}}}
\newcommand{\Wh}{\mathscr{W}}
\newcommand{\bF}{\mathrm{F}}
\newcommand{\col}{\operatorname{col}}
\newcommand{\row}{\operatorname{row}}
\newcommand{\refl}{\operatorname{r}}
\newcommand{\codim}{\operatorname{codim}}



% abbreviations
\newcommand{\ve}{\varepsilon}
\newcommand{\la}{\langle}
\newcommand{\ra}{\rangle}

% underlines
\newcommand{\ug}{\underline{\g}}
\newcommand{\ue}{\underline{e}}
\newcommand{\uchi}{\underline{\chi}}
\newcommand{\uG}{\underline{G}}
\newcommand{\uO}{\underline{\O}}


% tildes
\newcommand{\te}{{\tilde{e}}}
\newcommand{\tX}{\widetilde{X}}

%hats
\newcommand{\hZ}{\widehat{Z}}


% misc
\renewcommand{\i}{{\bf i}}
\renewcommand{\j}{{\bf j}}
%\newcommand{\s}{{\bf s}}
\newcommand{\PeN}{\P_\epsilon(N)}






%\input {cyracc.def}
%\font\ququ=cmr10 scaled \magstep1
%\font\tencyrb=wncyr10 scaled \magstep1
%\font\tencyr=wncyr10 %scaled \magstephalf
%\font\tencyi=wncyi10 %scaled \magstephalf
%\font\tencysc=wncysc10 %scaled \magstephalf
%\def\rusb{\tencyrb\cyracc}
%\def\rus{\tencyr\cyracc}
%\def\rusi{\tencyi\cyracc}
%\def\rusc{\tencysc\cyracc}
\begin{document}
	


\newenvironment{changemargin}[1]{%
  \begin{list}{}{%
    \setlength{\topsep}{0pt}%
    \setlength{\topmargin}{#1}%
    \setlength{\listparindent}{\parindent}%
    \setlength{\itemindent}{\parindent}%
    \setlength{\parsep}{\parskip}%
  }%
  \item[]}{\end{list}}

\parindent=0pt
\addtolength{\parskip}{0.5\baselineskip}

\subjclass[2010]{17B45}
\title{Sandwich elements
and the Richardson property}

\author{Alexander Premet}
\address{Department of Mathematics, The University of Manchester, Oxford Road, M13 9PL, UK}
\email{alexander.premet@manchester.ac.uk}
\pagestyle{plain}
\begin{abstract}
A restricted Lie algebra $\LL$ over an algebraically closed field $\k$ of characteristic $p>0$ is said to possess the Richardson property if there exists a finite dimensional faithful restricted $\LL$-module $V$ with associated representation $\rho\colon \LL\to \gl(V)$ such that $\gl(V)=\rho(\LL)\oplus R$ where $R$ is a subspace of $\gl(V)$ such that  $[\rho(\LL),R]\subseteq R$. In \cite{P99}, the author conjectured that if $\LL$ has the Richardson property with respect to an irreducible restricted  $\LL$-module $V$ then there exists a reductive algebraic group $G$ over $\k$ such that $\LL\cong\Lie(G)$
as restricted Lie algebras.
In this note we confirm this conjecture under the assumption that $p>3$. This assumption is needed since our proof relies in a crucial way
on the classification of Lie algebras without strong degeneration obtained in \cite{P87a} for $p>5$ and in \cite{P86} for $p=5$.
\end{abstract}
\maketitle
\begin{center}
{\it In memory of Georgia Benkart}
\end{center}


%\medskip


%{\small \tableofcontents}
\section{Preliminaries and the statement of the main result}\label{intro}
\subsection{Restricted Lie algebras}\label{restricted}
Let $\LL$ be a finite dimensional restricted Lie algebra over an algebraically closed field $\k$ of characteristic $p>0$. We denote by $\LL\ni x\mapsto x^{[p]}\in \LL$ the $p$-operation on $\LL$. An element $x\in\LL$ is said to be {\it nilpotent} if $x^{[p]^N}=0$ for some $N\gg 0$. It follows from Jacobson's formula (expressing $(x+y)^{[p]}$ as a sum of Lie monomials in $x,y\in\LL$) that 
for any $d\in\Z_{\ge 0}$ the map $\pi_d\colon\,\LL\to\LL$ such that $\pi_d(x)=x^{[p]^d}$ for all $x\in\LL$ is a morphism of affine varieties given by a collection of homogeneous polynomial functions of degree $p^d$ on $\LL$. From this it follows that
the set $\N(\LL)$ of all nilpotent elements of $\LL$ is a Zariski closed conical subset of $\LL$.
Given $x\in\LL$ we write $\la x\ra_p$ for the $\k$-linear span of all $x^{[p]^i}$ with $i\ge 0$.
We say that $x\in\LL$ is {\it semisimple}
if it belongs to the $\k$-span of all  $x^{[p]^i}$ with $i\ge 1$. 
If $N\in\Z_{\ge 0}$ is sufficiently large, then the set $\LL_{ss}$ of all semisimple elements of $\LL$ coincides with the image of the morphism
$\pi_N$.

We say that an element $x\in\LL$ is {\it toral} if $x^{[p]}=t$. A Lie subalgebra $\t$ of $\LL$ is called {\it toral} (or a {\it torus})
if it consists of semisimple elements of $\LL$. 
Toral subalgebras of $\LL$ are abelian and their maximal dimension, denoted ${\rm MT}(\LL)$, is an important invariant of $\LL$. Furthermore, it is well--known (and easy to see) that any toral subalgebra $\t$ of $\LL$ has a $\k$-basis consisting of toral elemnts of $\LL$. In other words, the set $\t^{\rm tor}\,:=\,\{t\in \t\,|\,\,t^{[p]}=t\}$ is an $\mathbb{F}_p$-form of $\t$.


We denote by $e=e(\LL)$ the smallest nonnegative integer for which there is a nonempty Zariski open subset $U=U(e)$ of $\LL$ such that $\pi_e(U)\subseteq \LL_{ss}$.  It is proved in \cite{P87b}  that if $\t\subseteq \LL$ is a torus of dimension ${\rm MT}(\LL)$ then the centraliser $\c_\LL(\t)$ is a Cartan subalgebra of minimal dimension in $\LL$ and the equality 
$$\dim\LL-\dim\overline{U(e)}\,=\,\dim\c_{\LL}(\t)-\dim\t$$ holds. From this it is immediate that
$e(\LL)=0$ if and only if $\LL$ contains a toral Cartan subalgebra.
  
 
According to \cite{P90}, if ${\rm MT}(\LL)=s$ then there exist
homogeneous polynomial functions $\psi_0,\ldots,\psi_{s-1}\in\k[\LL]$ such that $\deg \psi_i=p^{s+e}-p^{i+e}$ for $0\le i\le s-1$ and
$$x^{[p^{s+e}]}\,=\,\,\sum_{i=0}^{s-1}\,\psi_i(x)\cdot x^{[p]^{i+e}}\ \qquad\ (\forall\,x\in \LL).$$
 By \cite{P90, P03b}, each $\psi_i$ has that property that $\psi_i(x^{[p]})=\psi_i(x)^p$ for all $x\in\LL$ and is invariant under the natural action of the restricted automorphism group
${\rm Aut}(\LL,[p])$ on $\k[\LL]$. This implies that the variety $\N(\LL)$ coincides with the zero locus of $\psi_0,\ldots, \psi_{s-1}$.
Since $\t\cap\N(\LL)=\{0\}$ for any $s$-dimensional torus $\t$ of $\LL$, the Affine Dimension Theorem yields
that all irreducible components of the nilpotent cone $\N(\LL)$  have dimension equal to
$\dim\LL-{\rm MT}(\LL)$. As a consequence, the homogeneous polynomial functions $\psi_0,\ldots,\psi_{s-1}$ form a regular sequence in $\k[\LL]$ and, in particular,  are algebraically independent. In \cite{P90}, the author conjectured that the variety $\N(\LL)$ is irreducible for any finite dimensional restricted Lie algebra $\LL$. This conjecture is verified in many cases, but is still open, in general.

Let $\t$ be a maximal toral subalgebra of $\LL$. The adjoint action of $\t$ on $\LL$ gives rise to a Cartan decomposition
$$\LL\,=\,\c_{\LL}(\t)\oplus\,\sum_{\gamma\in \Gamma}\,\LL_\gamma,\qquad\quad\LL_\gamma\,=\,\{x\in\LL\,|\,\,[t,x]=\gamma(t)\cdot x\ \ \,(\forall\,t\in\t)\}.$$ Here  
$\Gamma=\Gamma(\LL,\t)$ is the set roots of $\LL$ with respect to $\t$, a finite subset of 
$\t^*$ consisting of {\it nonzero} ${\mathbb F}_p$-valued linear functions on $\t^{\rm tor}$, and $\LL_\gamma\ne\{0\}$ is the root space of $\LL$ corresponding to $\gamma\in\Gamma$. 
Since $\t$ is a maximal torus of $\LL$ its centraliser $\c_{\LL}(\t)$ is
a Cartan subalgebra of $\LL$. Given $\mathbb{F}_p$-linearly independent roots $\gamma_1,\ldots,\gamma_d\in\Gamma\subset (\t^{\rm tor})^*$ we denote by $\Gamma(\gamma_1,\ldots,\gamma_d)$ the set of all $\gamma=a_1\gamma_1+\cdots a_d\gamma_d\in\Gamma$ with $a_i\in{\mathbb F}_p$, and set $$\LL(\gamma_1,\ldots,\gamma_d)\,:=\,\c_{\LL}(\t)\oplus\textstyle{\sum}_{\gamma\in
\Gamma(\gamma_1,\ldots,\gamma_d)}\,\LL_\gamma.$$ 
The subspace $\t(\gamma_1,\ldots,\gamma_d)\,:=\,\bigcap_{i=1}^d\ker\gamma_i$ is a subtorus of codimension $d$ in $\t$. Using the perfect pairing between 
$\t^{\rm tor}$ and $(\t^{\rm tor})^*$ it is easy to check that
$\LL(\gamma_1,\ldots,\gamma_d)$ coincides with the centraliser of 
$\t(\gamma_1,\ldots,\gamma_d)$ in $\LL$. In particular, $\LL(\gamma_1\,\ldots,\gamma_d)$ is a restricted Lie subalgebra of $\LL$. Conversely, if $\t_0$ is a subtorus of $\t$, then $\c_{\LL}(\t_0)=\LL(\gamma_1,\ldots,\gamma_d)$ for some $\mathbb{F}_p$-linearly independent roots $\gamma_1,\ldots,\gamma_d\in \Gamma$ such that
$\t_0\subseteq \t(\gamma_1,\ldots,\gamma_d)$.  The restricted Lie algebra $$\LL[\gamma_1,\ldots,\gamma_d]\,:=\,
\LL(\gamma_1,\ldots,\gamma_d)/{\rm rad}\big(\LL(\gamma_1,\ldots,\gamma_d)\big)$$ is referred to as a $d$-{\it section} of $\LL$. It is straightforward to see that ${\rm MT}\big(\LL[\gamma_1,\ldots,\gamma_d]\big) \le d$.

\subsection{The Richardson property and support varieties}\label{supp}
Given $\chi\in \LL^*$ we denote by $I_\chi$ the two-sided ideal of the universal enveloping algebra $U(\LL)$ generated by all elements
$x^p-x^{[p]}-\chi(x)^p\cdot 1$ with $x\in \LL$ (these elements are central in $U(\LL)$). The 
factor-algebra 
$U_\chi(\LL):=U(\LL)/I_\chi$ is called the {\it reduced enveloping algebra} associated with $\chi$. It is well--known that for any irreducible $\LL$-module $V$ there exists a unique linear function $\chi=\chi_V\in\LL^*$ such that $(x^p-x^{[p]})_V=\chi(x)^p\cdot {\rm Id}_V$ for all $x\in\LL$. In other words, the action of $\LL$ on $V$ extends to that of the associative algebra $U_\chi(\LL)$.


 We denote by $\mathcal{I}_\chi$ the ideal of the symmetric algebra $S(\LL)$ generated by generated by all elements $(x-\chi(x))^p$ with $x\in \LL$. Since $\mathcal{I}_\chi$ is a Poisson ideal of the Lie--Poisson algebra $S(\LL)$, the factor-algebra $S_\chi(\LL):=S(\LL)/\mathcal{I}_\chi$ caries a Poisson algebra structure induced by that of $S(\LL)$. We call
$S_\chi(\LL)$ the {\it reduced symmetric algebra} associted with $\chi$. It is well--known that both $U_\chi(\LL)$ and $S_\chi(\LL)$ have dimension $p^{\dim\LL}$ over $\k$. The Lie algebra $\LL$ acts on $U_\chi(\LL)$ and $S_\chi(\LL)$ by derivations turning both algebras into {\it restricted} $\LL$-modules. The module structures thus obtained are induced by the adjoint action of $\LL$ on itself, and we shall use the subscript ``ad'' to indicate the Lie algebra actions.
In general, it is unknown for which restricted Lie algebras $\LL$ the $\LL$-modules $U_\chi(\LL)_{\rm ad}$ and $S_\chi(\LL)_{\rm ad}$ are isomorphic.
Answering this question would be very important from the point of view of the theory of support varieties developed by Friedlander and Parshall \cite{FP1, FP2} in the context of $U_\chi(\LL)$-modules.

Let $\N_p(\LL)=\{x\in\LL\,|\,\,x^{[p]}=0\}$, the  {\it restricted nullcone} of $\LL$. The $\k$-subalgebra
of $U_\chi(\LL)$ generated by a nonzero $x\in\LL$ will be denoted by $u_\chi(x)$. It follows from the PBW theorem that for $x\in\N_p(\LL)\setminus\{0\}$ the map $ X\mapsto x-\chi(x)$ gives rise to an algebra isomorphism $u_\chi(x)\cong \k[X](X^p)$. Given a finite dimensional $U_\chi(\LL)$-module 
$M$ we write $\mathcal{V}_{\LL}(M)^\times$ for the set all {\it nonzero} $x\in \N_p(\LL)$ such that $M$ is not a free $u_\chi(x)$-module, and put
$\mathcal{V}_{\LL}(M)\,:=\,\mathcal{V}_{\LL}(M)^\times\cup\{0\}$. It is easy to see that
$\mathcal{V}_{\LL}(M)$ is a Zariski closed, conical subset of $\N_p(\LL)$. It is called the {\it rank variety} (or the {\it support variety}) of $M$. It is proved in \cite{FP2} that $M$ is a projective
$U_\chi(\LL)$-module if and only if $\mathcal{V}_{\LL}(M)=\{0\}$.

Let $E_1,\ldots, E_s$ be representatives of the equivalence classes of all irreducible $U_\chi(\LL)$-module, and set $\mathcal{V}_{\LL}(\chi)\,:=\,\mathcal{V}_{\LL}(E_1)\cup\cdots\cup
\mathcal{V}_{\LL}(E_s).$ This Zariski closed, conical subset of $\N_p(\LL)$ contains the rank varieties of all finite dimensional $U_\chi(\LL)$-modules. By an important result of Jantzen \cite{Jan86}, we have that $\mathcal{V}_{\LL}(0)=\N_p(\LL)$, and it is proved in \cite{P99} that
$\mathcal{V}_{\LL}(\chi)=\mathcal{V}_{\LL}\big(U_\chi(\LL)_{\rm ad}\big)$ for any $\chi\in\LL^*$. 

From the representation-theoretic viewpoint it would be very useful to have a more explicit description of the variety
$\mathcal{V}_{\LL}(\chi)$. Let $\LL_\chi=\{x\in\LL\,|\,\,\chi([x,\LL])=0\} $, the stabiliser of $\chi$ in $\LL$. This is a restricted Lie subalgebra of $\LL$, and it is proved in \cite[Theorem~6.1]{PSk} that $\mathcal{V}_{\LL}\big(S_\chi(\LL)_{\rm ad}\big)=\,\N_p(\LL_\chi)$.
Therefore, if we know that $U_\chi(\LL)_{\rm ad}\cong S_\chi(\LL)_{\ad}$ as $\LL$-modules, then
$\mathcal{V}_{\LL}(\chi)\,=\,\N_p(\LL_\chi)$. According to \cite[\S 3]{P99}, there is a very large class of restricted Lie algebra $\LL$ for which the 
$\LL$-modules $U_\chi(\LL)_{\rm ad}$ and $S_\chi(\LL)_{\ad}$ are isomorphic.
\subsection{The statement of the main result} A restricted Lie algebra is said to possess the {\it Richardson property} if there exists
a finite dimensional faithful restricted $\LL$-module $V$ with associated representation $\rho\colon\,\LL\to\gl(V)$ such that $\gl(V)=\rho(\LL)\oplus R$ where $R$ is a subspace of $\gl(V)$ such that $[\rho(\LL),R]\subseteq R$. According to  
\cite[Prop.~2.3]{P99}, if $\LL$ has the Richardson property then for any $\chi\in\LL^*$ the  $\LL$-modules $S_\chi(\LL)_{\rm ad}$ and $U_\chi(\LL)_{\rm ad}$ are isomorphic. 


In \cite[3.6]{P99}, the author speculated that if $\LL$ possesses the Richardson property with respect to an irreducible faithful restricted $\LL$-module, then there exists a reductive algebraic group $G$ over $\k$ such that $\LL\cong\Lie(G)$ as restricted Lie algebras. This conjecture  was recently mentioned in the survey article \cite{BF} by Benkart--Feldvoss as one of the interesting open problems in the theory of modular Lie algebras; see Problem~7(c) in {\it loc.\,cit.} 

Let $G$ be a connected reductive $\k$-group. Recall that $p={\rm char}(\k)$ is a good prime for $G$ if all positive roots of $G$ with respect to a basis of simple roots have coefficients less than $p$. If $p$ is not good for $G$ then we say that $p$ {\it bad}. The bad primes of connected reductive groups lie in the set $\{2,3,5\}$ and $p=5$ is bad for $G$ if and only if $G$ has a component of type ${\rm E}_8$.  A good prime $p$ is called {\it very good} if $G$ has no components of type ${\rm A}_{kp-1}$ with $k\ge 1$.
We say that $G$ is {\it standard} if $p$ is a good prime for $G$, the derived subgroup of $G$ is simply connected, and the Lie algebra $\g={\rm Lie}(G)$ admits a non-degenerate $(\Ad G)$-invariant symmetric bilinear form. A rational representation $\varphi\colon\, G\to \GL(V)$
is called {\it infinitesimally irreducible} if the differential ${\rm d}_e\varphi\colon\g\to\gl(V)$
is an irreducible representation of the Lie algebra $\g$.

The main goal of this note is to confirm the above-mentioned conjecture under the assumption that $p>3$. More precisely, we prove the following:
\begin{Theorem}\label{main-thm} Suppose ${\rm char}(\k)=p>3$ and
let $\LL$ be a restricted Lie algebra over $\k$ possessing the Richardson property with respect to a finite dimensional faithful irreducible restricted representation $\rho\colon\, \LL\to \gl(V)$. 
Then there exists a standard reductive algebraic $\k$-group $G$ with Lie algebra $\g$ and an infinitesimally irreducible rational representation 
$\varphi\colon\,G\to\GL(V)$ such that
$({\rm d}_e\varphi)(\g)=\rho(\LL)$. Moreover,
$\LL$ and $\g$ are isomorphic as restricted Lie algebras.  
\end{Theorem} By Schur's lemma, if $\LL$ admits a faithful irreducible representation then the centre of $\LL$ has dimension $\le 1$. 
In the final part of the paper we  
show, that if $p$ is a very good prime for a standard algebraic group $G$ whose centre has dimension $\le 1$, then the restricted Lie algebra $\g=\Lie(G)$ possesses the Richardson property with respect to a faithful irreducible representation $\rho\colon\,\g\to\gl(V)$ such that $p\nmid \dim V$. Furthermore, $\rho={\rm d}_e\varphi$, where $\varphi\colon\,G\to\GL(V)$
is an infinitesimally irreducible rational representation, and the 
$(\Ad G)$-invariant
trace form $(X,Y)\mapsto {\rm tr}\big(\rho(X)\circ \rho(Y)\big)$ on $\g$ is non-degenerate. 

%if and only if it is isomorphic as a restricted Lie algebra to a direct sum of restricted ideals of %the form $\Lie(H_i)$ where $H_i$ is simple algebraic group of type other than ${\rm A}_{kp-1}$ and, %possibly, a single copy of $\k$ or $\gl_{rp}$ with $r\ge 1$. It should be mentioned that the %possibility for $\LL$ to contain a restricted ideal isomorphic to $\gl_{rp}$ was omitted in the %conjecture made in \cite[3.6]{P99}.

In characteristics $2$ and $3$, we show that if $\LL$ has the Richardson property with respect to a finite dimensional faithful restricted $\LL$-module $V$, then for any nonzero $e\in\N(\LL)$ there exists an element $h\in \LL_{ss}$ such that $[h,e]=e$. In conjunction with some results of \cite{P87b} and \cite{P90} mentioned in Subsection~\ref{restricted} this implies that all Cartan subalgebras of $\LL$ are toral and have the same dimension equal to  ${\rm MT}(\LL)$.
In characteristics $3$, we show that any $e\in\N(\LL)$ lies in the subspace $[e,[e,\LL]]$. This enables us to deduce that the solvable radical of any subalgebra $\LL(\gamma_1,\ldots, \gamma_d)$ coincides with the centre of  $\LL(\gamma_1,\ldots, \gamma_d)$. We then use Skryabin's classification 
\cite{Sk} of simple $3$-modular Lie algebras of toral rank $1$ to show that for every $\gamma\in\Gamma(\LL,\t)$ the radical of $\LL(\gamma)$ coincides with $\ker\gamma\,=\,\{t\in\t\,|\,\, \gamma(t)=0\}$ and $\LL[\gamma]=\LL(\gamma)/\ker\gamma$ is one of $\sl_2$ or $\psl_3$.
It would be interesting to classify the finite dimensional restricted Lie algebras over fields of characteristic $3$ having these properties.
\subsection{Lie algebras without strong degeneration}\label{strong}
The main idea of the proof of Theorem~\ref{main-thm} is to show that if 
$\LL$ has the Richardson property then the factor-algebra $\LL/\z(\LL)$ does not contain nonzero elements $c$ with $(\ad\,c)^2=0$ and then apply the classification of such Lie algebras obtained in \cite{P86, P87a}. This is the main reason for us to impose the assumption that $p>3$.

An element $c\in\LL$ is called a {\it sandwich element} if $(\ad c)^2=0$. This term, coined by A.I.~Kostrikin, has to do the fact that in characteristic $p>2$ any $c\in\LL$ with $(\ad c)^2=0$ satisfies the {\it sandwich identity} $$(\ad c)\circ (\ad x)\circ (\ad c)=0\qquad\quad(\forall\,x\in\LL).$$ It is immediate from this identity the set  $\mathfrak{C}(\LL)$ of all sandwich elements of $\LL$ is closed under taking Lie brackets.
In conjunction with the Engel--Jacobson theorem on weakly closed sets this shows that $\mathfrak{C}(\LL)$ generates a nilpotent Lie subalgebra of $\LL$ invariant under the automorphism group of $\LL$.
We say that $\LL$ is {\it strongly degenerate} if
$\mathfrak{C}(\LL)\ne \{0\}$.

Sandwich elements play an important role in the study of Engel Lie algebras and in the classification theory of finite dimensional simple Lie algebras over algebraically closed fields of 
characteristic $p>3$; see \cite{K90}, \cite{KZ90}, \cite{P94}, \cite{PS1}, \cite{Str17}, \cite{Z91}.
 In his talks at the ICM's in Stockholm (1962) and Nice (1971), Kostrikin conjectured that
 a finite dimensional simple Lie algebra $L$ over an algebraically closed field of characteristic $p>3$ is either strongly degenerate or classical in the sense of Seligman, that is, has the form $L=[\Lie(G),\Lie(G)]$ for some simple algebraic $\k$-group $G$ of adjoint type. Kostrikin's conjecture attracted a lot of attention 
 in 1960's and 1970's and was first confirmed under various additional assumptions on $L$; see \cite{Ko67}, \cite{Ja71}, \cite{St73} and \cite{Be77}\footnote{This work of Georgia Benkart is based on her Yale PhD thesis written under the supervision of Nathan Jacobson.}. In full generality, the conjecture was proved in \cite{P87a} for $p>5$ and in \cite{P86} for $p=5$. 

In fact, a slightly more general result is proved in \cite{P87a, P86}.
A finite dimensional Lie algebra $L$ over an algebraically closed field of characteristic $p>3$ is
called {\it almost classical} if there exists
a semisimple  algebraic $\k$-group $G$ of adjoint type such that $L$ is isomorphic to a subalgebra of $\Lie(G)$ containing  $[\Lie(G),\Lie(G)]$. Note that under our assumptions on $p$  the Lie algebra $\Lie(G)$ identifies with the derivation algebra of the restricted Lie algebra
$[\Lie(G),\Lie(G)]$. The latter decomposes into a direct sum of classical simple Lie algebras which may include components isomorphic to $\psl_{rp}(\k)$ with $r>0$.
The main result of \cite{P86, P87a} states that for $p>3$ a finite dimensional Lie algebra $L$ over $\k$ is almost classical if and only if $\mathfrak{C}(L)=\{0\}$. 

{\bf Acknowledgement.} This work was started during the author's stay at MSRI (Berkeley) 
in February--April 2018. I would like to thank the Mathematical Sciences Research Institute for its hospitality and creative atmosphere during the programme ``Group Representation Theory and Applications''.
\section{The case of arbitrary Richardson $\LL$-modules}\label{Prelim}
\subsection{}\label{intro1}
From now on all $\LL$-modules are assumed to be finite dimensional and restricted. We say that a faithful $\LL$-module $V$ is {\it Richardson} if there exists a subspace $R$ of $\gl(V)$
such that $[\LL,R]\subseteq R$ and $\gl(V)=\LL\oplus R$, where we identify $\LL$ with its image in $\gl(V)$. 
Given a module $M$ for a Lie algebra $\g$ and $v\in M$ we denote by $\g_v$ the stabiliser of $v$ in $\g$. We write $\z(\g)$ for the centre of $\g$ and ${\rm rad}\, \g$ for the {\it radical} of $\g$, the largest solvable ideal of $\g$.

The following
lemma is inspired by a very old observation of Jacobson.
\begin{Lemma}\label{L1}
	Let $V$ be a Richardson module for $\LL$ (not necessarily irreducible).
	\begin{itemize}
		\item[(1)]\, If $p>2$ then any $e\in\N(\LL)$ lies in the subspace $[e,[e,\LL]]$.
		
		\smallskip
		\item[(2)]\, If $p>0$ then for any $e\in\N(\LL)$ there is an element $h\in \LL_{ss}$ such that $[h,e]=e$. 
		\end{itemize}
	\end{Lemma}
\begin{proof}
	Thanks to our conventions we may assume that 
	$e$ is a nonzero nilpotent element of $\gl(V)$. We first look for $x,y\in \gl(V)$ such that 
	$e=[e,[e,x]]$ and $e=[y,e]$. For $p>2$ one can argue as in \cite[Lemma~2]{Jac} to observe that $e$ can be included into an $\sl_2$-triple $\{e,h,f\}\subset\gl(V)$. For $p>0$, the Lie algebra $\gl(V)$ admits
	a $\Z$-grading $\gl(V)\,=\,\bigoplus_{i\in\Z}\,\gl(V)_i$ such that
	 $e\in \gl(V)_2$ and $[e,\gl(V)_0]=\gl(V)_2$; see \cite[Theorem~A(ii)]{P03a}. This result is applicable in arbitrary characteristic since $\gl(V)$ admits a non-degenerate $(\Ad \GL(V))$-invariant trace form. 
	 
	 As a consequence, there is $y\in \gl(V)$ such that $[y,e]=e$. Write $y=y'+y''$ with
	 $y'\in \LL$ and $y''\in R$. As $e\in\LL$ and $[\LL,R]\subseteq R$ it must be that $[y'',e]=0$ and $[y',e]=e$. As $\LL$ is a restricted Lie subalgebra of $\gl(V)$ we can replace $y'$
	 by its semisimple part  to find a semisimple element $h\in \LL$ such that $[h,e]=e$. This proves (2).
	 
	 If $p>2$ then $[e,[e,f]]=-2e\in\k^\times e$. Write $f=f'+f''$ with $f'\in\LL$ and $f''\in R$.
	 Since $(\ad e)^2$ preserves both $\LL$ and $R$ it must be that $[e,[e,f']]\in\k^\times e$. This proves (1). 
	 \end{proof}
 \subsection{}\label{intro1}
	 There are examples where a restricted Lie algebra admits infinitely families of indecomposable Richardson modules.
	 Indeed, suppose $\LL=\sl_2$ and $p\ge 3$. Let $V(m)$ be the Weyl module for the $\k$-group $\SL(2)$ with highest weight $m\in\Z_{\ge 0}$. Differentiating the rational action of $\SL(2)$ endows $V(m)$ with a natural restricted $\LL$-module structure. 
	 According to \cite{P91} the restricted $\LL$-module $V(m)$ is indecomposable if $p\nmid (m+1)$. It is well--known that the $\LL$-module $V(m)$ is a reducible  if $m\ge p$. 
	 
	 Suppose $m=kp+l$ where $k\in \Z_{\ge 0}$ and $1\le l \le p-2$. Then it follows from 
	 \cite[Lemma~2.6]{P91} that $V(m)\otimes V(m)^*\cong\,P\oplus \sum_{i=0}^{l}V(2i)$ if $l\le (p-1)/2$ and $V(m)\otimes V(m)^*\cong\,P'\oplus\sum_{i=0}^{p-l-2}V(2i)$ if $(p-1)/2<l<p-2$, where $P$ and $P'$ are projective modules over the restricted enveloping algebra $U_0(\LL)$. 
	 Since
	 $V(m)\otimes V(m)^*\cong \,\gl(V(m))$ and $V(2)\cong \LL$ as $\LL$-modules, we see that any $V(m)$ with $1\le l\le p-2$ and $k\in\Z_{\ge 0}$ is an indecomposable Richardson module for $\LL$.  
	 
	 If $\LL=\sl_2$ and $p=3$ then $\LL\cong V(p-1)$ is a projective module over $U_0(\LL)$. This means that {\it any} non-trivial finite-dimensional restricted $\LL$-mudule is Richardson for $\LL$.
	 Such instances are, of course, extremely rare.
\subsection{} Suppose $\LL$ admits a Richardson module $V$ (not necessarily irreducible). Our next result provides some insight into the structure of $\LL$.
	 \begin{Theorem}\label{T1}
	 	The following are true:
	 	\begin{itemize}
	 	\item[(1)] If $p\ge 2$ the all Cartan subalgebras of $\LL$ are toral of dimension equal to ${\rm MT}(\LL)$.
	 	
	 	\smallskip
	 	
	 	\item[(2)] Suppose $p>2$ and let $\t$ be any toral subalgebra of $\LL$. Then ${\rm rad}\,\c_{\LL}(\t)\,=\,\z(\c_{\LL}(\t))$ is a toral subalgebra of $\LL$ and the factor-algebra
	 	$\c_{\LL}(\t)/\z(\c_{\LL}(\t))$ has no nonzero sandwich elements.
	 	
	 	\smallskip
	 	
	 	\item[(3)] If $p>3$ then the factor-algebra $\LL/\z(\LL)$ is almost classical.	
	 		\end{itemize}
	 	\end{Theorem}
	\begin{proof}
		Let $\h$ be a Cartan subalgebra of $\LL$. It is well--known (and easy to see) that 
		$\h=\c_{\LL}(\t')$ where $\t'=\h\cap\LL_{ss}$, a maximal toral subalgebra of $\LL$. Let $\Gamma$ be the set
		of roots of $\LL$ with respect to $\t'$ and 
		$\widetilde{R}=R\oplus \sum_{\gamma\in\Gamma}\,\LL_\gamma$. Then $\gl(V)=\h\oplus \widetilde{R}$ and $[\h,\widetilde{R}]\subseteq \widetilde{R}$, meaning that $V$ is a Richardson module for $\h$. If $\h$ contains a nonzero nilpotent element, $e$ say, then  Lemma~\ref{L1}(2) yields that there is an  $h\in \h$ such that $[h,e]=e$. But then $(\ad h)^n(e)=e\ne 0$ for all $n\in\Z_{>0}$. Since $\h$ is nilpotent this is impossible. Since $\h$ is a restricted subalgebra of $\LL$, the semisimple and nilpotent parts of any element of $\h$ lie in $\h$. Consequently,  $\h=\h_{ss}=\t'$. 
		
		Now let $\t$ be any torus of maximal dimension in $\LL$. Then
		$\dim \t\ge \dim \t'$ and $\dim \c_{\LL}(\t)\le\dim \h$ by the main result of \cite{P87b} 
		(see Subsection~\ref{restricted} for a related discussion). Since $\h=\t'$ we now deduce that
		$\dim\h=\dim \t'={\rm MT}(\LL)$, proving (1).
		
		Suppose $p>2$ and let $\t$ be a toral subalgebra of $\LL$ (possibly zero). Let $\LL_0=\c_{\LL}(\t)$ and write $\Gamma(\LL,\t)$ for the set of roots of $\LL$ with respect to $\t$ (possibly empty). Now set $\widetilde{R}:=R\oplus \sum_{\gamma\in\Gamma(\LL,\t)}\,L_\gamma$. Then $\gl(V)=\LL_0\oplus\widetilde{R}$ and $[\LL_0, \widetilde{R}]\subseteq \widetilde{R}$, showing that $V$ is a Richardson module for $\LL_0$. Let $\z$ be the centre of $\LL_0$, a restricted ideal of $\LL_0$. If $\z$ contains a nonzero nilpotent element, $e$ say, then
		$e\in [e,[e,\LL_0]]$ by Lemma~\ref{L1}(1). But then $0\ne e\in [e,\z]$, a contradiction. 
		Hence $\z$ is toral. 
		
		Suppose $\LL_0/\z$ contains a nonzero sandwich element. Then there is $c\in \LL_0\setminus\z$ such that $[c,[c,\LL_0]]\subseteq\z$. Since $p\ge 3$ it must be that  $c^{[p]}\in\z$. As $\z$ is toral, we may replace $c$ by a suitable element of the form $c+z$ with $z\in\z$ to assume further that $c$ is nilpotent. By Lemma~\ref{L1}(1), we then have $c\in[c,[c,\LL_0]]\subseteq \z$, a contradiction. Hence $\LL_0/\z$ has no nonzero
		sandwich elements and. In particular, $\LL_0/\z$ has no nonzero abelian ideals.
		As a consequence, the radical of $\LL_0/\z$ is trivial.
		
		Let $\mathfrak{r}={\rm rad}\,\LL_0$. Then $\z\subseteq \mathfrak{r}$ and $\mathfrak{r}/\z$ is a solvable ideal of $\LL_0/\z$. The preceding remark shows that 
		$\mathfrak{r}=\z$, proving (2). Part (3) now follows from (2) and the main results of \cite{P87a} and \cite{P86} (see our discussion in Subsection~\ref{strong} for more detail).
	\end{proof}
In characteristic $3$, one can use Skryabin's description of simple Lie lagebras of rank $1$ to get more information on the $1$-sections of $\LL$.
\begin{Corollary}
Suppose $p=3$ and let $\alpha\in\Gamma(\LL,\t)$ where $\t$ is a maximal torus of $\LL$. Then
${\rm rad}\,\LL(\alpha)\,=\,\t(\alpha)\,=\,\ker\alpha$ and $\LL[\alpha]=\LL(\alpha)/\t(\alpha)$ is one of $\sl_2$ or
$\psl_3$.
\end{Corollary}
\begin{proof} Since $p=3$, Theorem~\ref{T1}(1) yields that $\LL(\alpha)\,=\,\LL_{-\alpha}\oplus \t\oplus\LL_\alpha$. 
Since $\t$ is a maximal torus and ${\rm rad}\,\LL(\alpha)\,=\,\z(\LL(\alpha))$ is toral by Theorem~\ref{T1}(2), it must be that ${\rm rad}\,\LL(\alpha)\subseteq \t$. Since $\LL_\alpha$ is nonzero, $\t(\alpha)\subseteq \z(\LL(\alpha))$, and $\LL[\alpha]$ is semisimple by Theorem~\ref{T1}(2), we have that $\t(\alpha)\,=\,{\rm rad}\,\LL(\alpha)$ and
${\rm MT}(\LL[\alpha])=1$. Furthermore, $\LL_{-\alpha}\ne \{0\}$ as otherwise $\LL(\alpha)$ would be solvable. 

Since ${\rm MT}(\LL[\alpha])=1$ the Lie algebra $\LL[\alpha]$ has a unique minimal ideal, say $S$.
Since $\LL[\alpha]$ has no nonzero sandwich elements by Theorem~\ref{T1}(2), repeating verbatim the argument used in the proof of \cite[Lemma~4]{P87a} one observes that the ideal $S$ is simple and $\LL[\alpha]$ is sandwiched between $\ad S$ and ${\rm Der}\,S$. Since ${\rm MT}(\LL[\alpha])=1$, the simple Lie algebra $S$ has absolute toral rank $1$. Applying \cite[Theorem~6.5]{Sk} one obtains that either $S\cong\sl_2$ or $S\cong \psl_3$. 

In any event, $S$ is a restricted ideal of $\LL[\alpha]$. Let $\widetilde{S}$ be the inverse image of $S$ in $\LL(\alpha)$, a restricted ideal of $\LL(\alpha)$, and let $\t'$ be a maximal torus of 
$\widetilde{S}$. Since $\t'$ contains $\t(\alpha)$ it is straightforward to see that $\dim \t'=\dim\t$. Applying Theorem~\ref{T1}(1) once again yields that $\t'$ is a self-centralising maximal torus of $\LL(\alpha)$. As $[\t',\LL(\alpha)]\subseteq \widetilde{S}$, all root spaces of $L(\alpha)$ with respect to $\t'$ are contained in $\widetilde{S}$. But then $\LL(\alpha)\subseteq  \t'+\widetilde{S}$ forcing $\LL(\alpha)=\widetilde{S}$. This completes the proof.
\end{proof}
\section{The case of irreducible Richardson modules}
\subsection{}\label{3.1} From now on we assume that $p>3$ and our Richardson $\LL$-module $V$ is irreducible. In particular, this means that $\c_{\gl(V)}(\LL)$ consists of scalar endomorphisms of $V$ (by Schur's lemma). Therefore, either $\c_{\gl(V)}(\LL)\,=\,\z(\LL)$ or $\LL$ is centreless and $\c_{\gl(V)}(\LL)\subset R$. By Theorem~\ref{T1}(3) (and our discussion in Subsection~\ref{strong}) there exists a semisimple algebraic $\k$ group $G$ of adjoint type such that
$\LL/\z(\LL)$ identifies with a Lie subalgebra of $\g:=\Lie(G)$ containing $[\g,\g]$.  Since $G$ is a group of adjoint type and $p>3$, it follows from \cite[Lemma~2.7]{BGP}, for example, that the restricted Lie algebra $\g$ is isomorphic to $\Der [\g,\g]$. 
Therefore, we may identify $\bar{\LL}:=\LL/\z(\LL)$ with a restricted ideal of $\g$ containing $[\g,\g]$.  

Let $T$ be a maximal torus of $G$ and denote by $\Phi$ the root system of $G$ with respect to $T$.
Let $\Pi$ be a set of simple roots in $\Phi$ and write $\Phi_+=\Phi_+(\Pi)$ for the set of positive roots with respect to $\Pi$.
All root spaces $\g_\alpha$ with $\alpha\in\Phi$ are contained in $[\g,\g]$ and
$\bar{\LL}\,=\,\bar{\t}\oplus \sum_{\alpha\in\Phi}\,\g_\alpha$ where $\bar{\t}:=\LL\cap \Lie(T)$.
Since $[\g_\alpha,\g_{-\alpha}]\subset \bar{\t}$ for all $\alpha\in\Phi$ and $p>3$, it follows from Seligman's results that $\bar{\t}$ is a self-centralising maximal torus of $\bar{\LL}$ and for every $\gamma\in \Gamma(\bar{\LL},\bar{\t})$ there is a unique $\widehat{\gamma}\in\Phi$ such that $\gamma=({\rm d}_e\widehat{\gamma})\vert_{\,\bar{\t}}$; see \cite[Ch.~II, \S 3]{Sel}. 

Let $\t$ be the inverse image of $\bar{\t}$ in $\LL$, By Theorem~\ref{T1}(1), this is a toral Cartan subalgebra of $\LL$. The above discussion enables us to identify $\Phi$ with  the set of roots $\Gamma(\LL,\t)$, i.e. given $x\in\t$ we write $\alpha(x)$ instead of $({\rm d}_e\alpha)(\bar{x})$ with $\bar{x}=x+\z(\LL)$. We have $\LL\,=\,\t\oplus \sum_{\alpha\in\Phi}\,\LL_\alpha$ and each root space $\LL_\alpha=\k e_\alpha$ with respect to $\t$ is $1$-dimensional. We choose root vectors $e_\alpha$ in such a way that $[[e_\alpha,e_{-\alpha}],e_\alpha]=2e_\alpha$ for all $\alpha\in\Phi_+$
and embed the torus $T$ of $G$ into ${\rm Aut}(\LL)$ by setting $t(x)=x$ and $t(e_\alpha)=\alpha(t)e_\alpha$ for all $t\in T$, $x\in\t$ and $\alpha\in\Phi$. 
 \subsection{}\label{3.2}  
 Given $\alpha\in\Phi_+$ we put $h_\alpha:=[e_\alpha,e_{-\alpha}]$. Obviously, $\{e_\alpha,h_\alpha,e_{-\alpha}\}$ is an $\sl_2$-triple in $\LL$ and we write $\mathfrak{s}_\alpha$ for the $\k$-span of $e_{\pm \alpha}$ and $h_\alpha$. It is straightforward to see that $\LL$ admits a natural restricted Lie algebra structure $x\mapsto 
 x^{[p']}$ with respect to which all $h_\alpha$ are toral and all $e_\alpha$ have the property that $e_\alpha^{[p']}=0$. Unfortunately, if $\dim\z(\LL)=1$, the restricted Lie algebras $(\LL,[p])$ and
 $(\LL,[p'])$ may be very different. In order to proceed further we have to investigate this problem. 
 
 Suppose that $\z(\LL)$ is spanned by $z={\rm Id}_V$.
 Since $\ad(x^{[p']})=\ad(x^{[p]})=(\ad x)^p$, there exists a linear function $\chi\in\LL^*$ such that 
 $x^{[p']}=x^{[p]}+\chi(x)^p\cdot z$ for all $x\in\LL$. Since $V$ is a restricted module for $(\LL,[p])$, it has $p$-character $\chi$ when regarded as a module over $(\LL,[p'])$.
   \begin{Lemma}\label{L2}
   Suppose $\dim\z(\LL) =1$. Then the linear function $\chi$ vanishes on $[\LL,\LL]$.	
   \end{Lemma}
   \begin{proof}
   	Suppose for a contradiction that $\chi$ does not vanish on $\mathfrak{s}_\beta$ for some $\beta\in\Phi_+$. The above discussion enables us to view $V$ as a $U_\chi(\mathfrak{s}_\beta)$-module. (To ease notation we identify $\chi$ with its restriction to $\mathfrak{s}_\beta$.) By the choice of $\beta$, the stabiliser of $\chi$ in $\mathfrak{s}_\beta$ is a {\it proper} Lie subalgebra of $\mathfrak{s}_\beta$, hence $1$-dimensional. Since the restricted Lie algebra $\sl_2$ has the Richardson property,
   	our discussion in Subsection~\ref{supp} shows that $\mathcal{V}_{\mathfrak{s}_\beta}(\chi)\,=\,\N_p((\mathfrak{s_\beta})_\chi)$ is either zero or a single line depending on $\chi$. 
   	
   	We may thus assume without loss of generality that $e_\beta\not\in  \mathcal{V}_{\mathfrak{s}_\beta}(\chi)$. Then all composition factors of the $u_\chi(\mathfrak{s}_\beta)$-module $V$ are free $u_\chi(e_\beta)$-modules, implying that all Jordan blocks of $e_\beta$ on $\gl(V)\cong V\otimes V^*$ have size $p$. Since $\LL$ is a direct summand of $\gl(V)$, all Jordan blocks of $\ad e_\beta\in\gl(\LL)$ must have size $p$ as well. In particular, $(\ad e_\beta)^{p-1}\ne 0$. 
   	
   	Now, if $\gamma\in\Phi\setminus\{\pm \alpha\}$ then $(\ad e_\beta)^{p-1}(e_\gamma)=0$ by the Mills--Seligman axioms 
   	(see \cite[Theorem~2.5]{BGP}, for example). Since $(\ad e_\beta)^3(\mathfrak{s}_\beta)=0=(\ad e_\beta)^2(\t)$ and $p>3$ we get $(\ad e_\beta)^{p-1}=0$. This contradiction shows that $\chi$ vanishes on all subalgebras $\mathfrak{s}_\beta$ with $\beta\in \Phi_+$. As such subalgebras span $[\LL,\LL]$, the result follows.
   	\end{proof}
   \subsection{}\label{3.3} It follows from Lemma~\ref{L2} that $e_{\pm \alpha}^{[p]}=0$ and $h_\alpha^{[p]}=h_\alpha$ for all $\alpha\in\Phi_+$. This must hold in the case where 
   $\LL$ is centreless as well, since $\LL$ then admits a unique $[p]$-structure. Now one can check 
   directly the action of the maximal torus $T$ of $G$ on $\LL$ described in Subsection~\ref{3.1} respects the $[p]$-structure of $\LL$, i.e. has the property that $t(x^{[p]})=t(x)^{[p]}$ for all $t\in T$ and $x\in \LL$ (one has to keep in mind here that $\t$ is restricted and $T$ acts trivially on $\t$). As a consequence, $T$ embeds into the automorphism group of the restricted enveloping algebra $U_0(\LL)$.
   \begin{Lemma}\label{L3}
 The module $V$ admits a rational $T$-action compatible with the action of $T$ on $\LL$.  
\end{Lemma}
  \begin{proof}
  By \cite[Prop.~3.5]{GG82} and \cite[Corollary~2]{Jan00}, the simple $U_0(\LL)$-module $V$ is gradable, i.e. decomposes into a direct sum of $T$-weight spaces $V=\bigoplus _{\lambda\in X(T)} V_\mu$ in such a way that each weight space $V_\mu$ is $\t$-stable and $t(e_\alpha . v)= (\mu+\alpha)(t)v$ for all $t\in T$,  $\alpha\in\Phi$ and $v\in V_\mu$.
  \end{proof}
Since $G$ is a group of adjoint type the torus $T$ acts faithfully on $[\g,\g]$. As $\LL$ embeds into $\gl(V)$, this implies that $T$ acts faithfully on $V$. We may thus identify $T$ with a connected subgroup of $\GL(V)$. Since $T\subset N_{\GL(V)}(\LL)$ the Lie algebra $\Lie(T)$ normalises $\LL$. 

Recall from Subsection~\ref{3.1} that $\bar{\LL}=\LL/\z(\LL)$ is sandwiched between $\g$ and $[\g,\g]$.
\begin{Proposition}\label{P1}
We have the equality $\bar{\LL}=\g$.	
	\end{Proposition}
\begin{proof}
	Let $\n(\LL)$ denote the normaliser of $\LL$ in $\gl(V)$. Given $x\in \n(\LL)$ we write $x=x'+x''$ with
	$x'\in \LL$ and $x''\in R$. As $[x,\LL]\subseteq \LL$ it must be that
	$[x'',\LL]=0$. Hence $\n(\LL)\,=\,\LL\oplus (R\cap\c_{\gl(V)}(\LL))$. Since $\Lie(T)\subset 
	\n(\LL)$ acts faithfully $[\g,\g]$ (and hence on $\LL$) the restriction of the first projection 
	$\n(\LL)\to \LL$ to $\Lie(T)$ is injective. Since $\g=\Lie(T)+\sum_{\alpha\in\Phi}\,\bar{\LL}_\alpha$ this implies that $\dim\,\g=\dim\,\bar{\LL}$. As $\bar{\LL}\subseteq \g$, the claim follows.
\end{proof}
\subsection{}\label{3.4} Suppose $H$ is a simple algebraic $\k$-group of adjoint type and $p>3$. Then $\Lie(H)$ is a perfect Lie algebra whenever $H$ has type other than ${\rm A}_{kp-1}$, and  $[\Lie(H),\Lie(H)]$ has codimension $1$ in $\Lie(H)$ if $H\cong{\rm PGL}_{kp}$. Indeed, in the latter case $\Lie(H)\,\cong\,\mathfrak{pgl}_{kp}$ and $[\Lie(H),\Lie(H)]\,\cong\,\mathfrak{psl}_{kp}$. Since our algebraic group $G$ is isomorphic to a a direct product of simple algebraic groups of adjoint type, $\dim(\g/[\g,\g])$ equals the number of simple components of $G$ having types ${\rm A}_{kp-1}$ with $k\in\Z_{>0}$.
\begin{Lemma}\label{L4}
The derived subalgebra of $\g$ has codimension $\le 1$ in $\g$.	
\end{Lemma}
\begin{proof}
	In view of Proposition~\ref{P1}, in order to prove the lemma it suffices to show that $\LL$ has codimension $\le 1$ in $\LL$.
	Since $\LL$ is a direct summand of $\gl(V)$, a self-dual $\LL$-module, the coadjoint $\LL$-module $\LL^*$ must be a direct summand of $\gl(V)$ as well. Since the trivial $\LL$-submodule $(\LL^*)^\LL$ of $\LL^*$ identifies canonically with the dual space $(\LL/[\LL,\LL])^*$, we have that  
	$$\dim(\LL/[\LL.\LL])=\dim(\LL^*)^\LL\le \dim\c_{\gl(V)}(\LL)=1.$$
	If follows that either $\g$ is perfect or $[\g,\g]$ has codimension $1$ in $\g$. 
\end{proof}
	Lemma~\ref{L4} shows that the group $G$ cannot have more that one component of type ${\rm A}_{kp-1}$, and its proof implies that $[\LL,\LL]$ has codimension $\le 1$ in $\LL$.
	\begin{Lemma}\label{L5}
		If $G$ has a component of type ${\rm A}_{kp-1}$, then $\z(\LL)$ is a $1$-dimensional subalgebra of $[\LL,\LL]$. In other words, $[\LL,\LL]$ is a non-split central extension of  $[\g,\g]$.
		\end{Lemma}
	\begin{proof} Since $G$ has a component of type ${\rm A}_{kp-1}$ it follows from Lemma~\ref{L4} that $G\cong G_1\times G_2$ where $G_1$ has no components of type ${\rm A}_{kp-1}$ and $G_2\cong{\rm PGL}_{kp}$. Setting $\g_i:=\Lie(G_i)$ for $i=1,2$ we get two commuting restricted ideals of $\g$ such that $\g=\g_1\oplus\g_2$. Furthermore, $[\g_1,\g_1]=\g_1$ is centreless and $\g_2\cong\mathfrak{pgl}_{kp}$.
		If $\z(\LL)\cap [\LL,\LL]=\{0\}$ then $\LL=\z(\LL)\oplus [\LL,\LL]$. In this case, Proposition~\ref{P1} entails that $$\g=\bar{\LL}\,\cong\,[\LL,\LL]\,\cong\,\g_1\oplus[\g_2,\g_2]\,\cong\,\g_1\oplus\psl_{kp}$$
		is a completely reducible $\ad \g$-module. Since $\mathfrak{pgl}_{kp}$ is a direct summand
		of $\g$, this is impossible. By contradiction, the result follows.
		\end{proof}
	\subsection{}\label{3.5}  If $G\cong G_1\times G_2$, where the $G_i$'s are as in the proof of Lemma~\ref{L5}, we let $\widetilde{G} = \widetilde{G}_1\times \widetilde{G}_2$ be the reductive $\k$-group such that $\widetilde{G}_1$ is a simple, simply connected cover of $G_1$ and $\widetilde{G}_2=\GL_{kp}$.
	If $G$ has no components of type ${\rm A}_{kp-1}$ and $\LL$ is centreless, we set $\widetilde{G}:=\widetilde{G}_1$. Finally, if $G$ has no components of type ${\rm A}_{kp-1}$ and $\dim\z(\LL)=1$ , we set $\widetilde{G}:=T_0\times \widetilde{G}_1$ where $T_0$ is a $1$-dimensional central torus of $\widetilde{G}$. In all cases, the derived subgroup of $\widetilde{G}$ is semisimple and simply connected. Since $p>3$, the Lie algebra 
	$\g_1\cong \Lie(\widetilde{G}_1)$ is a direct sum of simple ideals. By \cite[Lemma~2.5]{Pr97}, the Lie algebra $\widetilde{\g}:=\Lie(\widetilde{G})$ admits a non-degenerate symmetric $(\Ad \widetilde{G})$-invariant bilinear form, 
	say $b\colon\, \widetilde{\g}\times \widetilde{\g}\to \k$, and it follows from
	\cite[Theorem~A]{Gar} that $b$ can be chosen to be a trace form (associated with a rational representation of $\widetilde{G}$) provided that $p$ is a good prime for
	$\widetilde{G}_1$.
	\begin{Proposition}\label{P2} 
	In all cases, the restricted Lie algebra $\LL$ is isomorphic to $\widetilde{\g}$.
	\end{Proposition}
\begin{proof}
	By construction, $\z(\widetilde{\g})$ lies in $[\widetilde{\g},\widetilde{\g}]$ and coincides with the radical of the restriction of
	$b$ to $[\widetilde{\g},\widetilde{\g}]$. Therefore, $b$ gives rise to a non-degenerate $\g$-invariant symmetric bilinear form on $[\g,\g]\,\cong\, [\widetilde{\g},\widetilde{\g}]/\z(\widetilde{\g})$; we call it $\bar{b}$. Since $\Der [\g,\g]\,\cong \g$ and $\g\cong \widetilde{\g}/\z(\widetilde{\g})$ as Lie algebras, the bilinear form $\bar{b}$ is invariant under derivations of $[\g,\g]$. It follows that for any $2$-cocycle
	$\varphi$ on $[\g,\g]$ with values in $\k$ there is an $h\in\g$ such that 
	$\varphi(x,y)=\bar{b}([h,x], y)$ for all $x,y\in[\g,\g]$. Furthermore, the central extension of $[\g,\g]$ associated with $\varphi$ is trivial if and only if $h\in [\g,\g]$.
	
	If $G=G_1$ then $\g$ is perfect, whilst if $G$ has a component of type ${\rm A}_{kp-1}$ then $[\g,\g]$ has codimension $1$ in $\g$. In case $G=G_1$, this shows that all central extentions of $\g=[\g,\g]$ are trivial. Hence $\LL\cong\g$ if 
	$\LL$ is centreless and $\LL\,\cong\,\k\oplus\g\,\cong\, \Lie(T_0\times \widetilde{G}_1)$
	if $\dim \z(\LL)=1$ (we view $\k$ is a $1$-dimensional toral Lie algebra). As $\z(\LL)$ is a torus, this is an isomorphism of restricted Lie algebras.
	
	If $G$ has a component of type ${\rm A}_{kp-1}$ then 
	$[\widetilde{\g},\widetilde{\g}]\,\cong\, \g_1\oplus \sl_{kp}$ is a non-split central extension of $[\g,\g]$. Since the above discussion implies that
	$[\g,\g]$ admits a unique non-trivial central extension (up to equivalence), it follows from Lemma~\ref{L5} and Lemma~\ref{L2} that
	$[\LL,\LL]\cong\g_1\oplus \sl_{kp}$ as restricted Lie algebras. 
	
	Let $\{\alpha_1,\ldots\alpha_{kp-1}\}\subseteq \Pi$ be the simple roots of $G_2$ with respect to $T$ numbered as in \cite[Planche~I]{Bour}.
	Since 
	$\g_2\cong\mathfrak{pgl}_{kp}$ there exists $h\in \Lie(T)$ such that $[h,\g_1]=0$,  
	$\alpha_1(h)=1$, and $\alpha_i(h)=0$ for $i>0$ Since $h\not\in[\g,\g]$ we have that $\g=\k h\oplus [\g,\g]$. Since $\z(\LL)$ is a torus, it is immediate from Proposition~\ref{P1} 
	that there exists a toral element $\hat{h}\in \LL$ which maps onto $h$ under the canonical 
	homomorphism $\LL\twoheadrightarrow \bar{\LL}=\g$. Since $\LL=\k\hat{h}\oplus [\LL,\LL]$ and $[\LL,\LL]\cong\g_1\oplus\sl_{kp}$, applying Lemma~\ref{L2} once again we deduce that $\LL\cong\g_1\oplus\gl_{kp}\cong\Lie(\widetilde{G})$ as restricted Lie algebras. This completes the proof.
	\end{proof}
\subsection{}\label{3.6}  Proposition~\ref{P2} enables us to identify the restricted Lie algebras $\LL$ and $\widetilde{\g}$ and regard $V$ as an irreducible restricted $\widetilde{\g}$-module. We may also assume that $T$ is a maximal torus of the reductive group $\widetilde{G}$. Let $\n_+$ and $\n_-$ denote the $\k$-spans of all $e_\alpha$ and all $e_{-\alpha}$ with $\alpha\in\Phi_+$, respectively.
By Lemma~\ref{L3}, there is a rational action of $T$ on $V$ compatible with that of $\widetilde{\g}$. Since $V$ is irreducible, the fixed-point space $V^{\n_+}=\{v\in V\,|\,\, \n_+.\, v=0\}$ has dimension $1$ and is spanned by a highest weight vector $v_\lambda$ for $T$, where $\lambda\in X(T)$. Also, $V=U_0(\n_-)\,.\,v_\lambda$. Since the derived subgroup of $\widetilde{G}$ is simply connected, \cite[Theorem~2]{Jan00} shows that $\lambda\in X(T)$ can be chosen to be dominant and $p$-restricted, that is  $0\le \la\lambda,\alpha^\vee\ra\le p-1$ for all $\alpha\in\Pi$.
We denote by $L(\lambda)$ the irreducible rational $\widetilde{G}$-module with highest weight 
$\lambda$ and write $\rho$ for the corresponding representation of $\widetilde{G}$ in 
$\GL(L(\lambda))$. By the general theory of linear algebraic groups, 
the differential ${\rm d}_e\rho\colon\,\widetilde{\g}\to \gl(L(\lambda))$ is a restricted representation of $\widetilde{\g}$.
\begin{Lemma}\label{L6}
	We have that $V\cong L(\lambda)$ as $\widetilde{\g}$-modules. 
\end{Lemma}
\begin{proof}
Since $\lambda\in X(T)$ is a $p$-restricted dominant weight, \cite[Part~II, Prop.~9.24(b)]{Jan03} 
yields that
the $\widetilde{\g}$-module $L(\lambda)$ is irreducible. On the other hand, it is well--known that the irreducible restricted $\widetilde{\g}$-modules are determined up to isomorphism by their highest weights; see \cite[Part~II, Prop.~3.10]{Jan03}, for example. This shows that the restricted $\widetilde{\g}$-modules $V$ and $L(\lambda)$ are isomorphic.
\end{proof}
\subsection{} By our discussion in Subsection~\ref{3.5}, the derived subgroup of $\widetilde{G}$ is simply connected and the Lie algebra $\Lie(\widetilde{G})$ admits a non-degenerate 
$(\Ad \widetilde{G})$-invariant symmetric bilinear form.  In view of Lemma~\ref{L6},  in order to finish the proof of Theorem~\ref{main-thm} it remains show that $p$ is a good prime for $\widetilde{G}$. As $p>3$ by our general assumption, we just need to sort out the case where $p=5$ and $\widetilde{G}$ has components of type ${\rm E}_8$.
\begin{Lemma}\label{L7}
	If $\widetilde{G}$ has a component of type ${\rm E}_8$ then $p>5$.
	\end{Lemma}
\begin{proof}
Let $H$ be a simple component of type ${\rm E}_8$ in $\widetilde{G}$. It is immediate from our description of $\widetilde{G}$ that it contains 
a connected normal subgroup $H'$ such that $\widetilde{G}\cong H\times H'$. Let $\h=\Lie(H)$ and $\h'=\Lie(H')$. The ideals $\h$ and $\h'$ of $\widetilde{\g}$ commute and $\widetilde{\g}=\h\oplus \h'$.  To ease notation we identify $\widetilde{\g}$ with $\LL$. Since $\h$ is a direct summand of  $\widetilde{\g}$, the $\widetilde{\g}$-module $V$ is Richardson for $\h$. 
Let $\widetilde{R}$ be a subspace of $\gl(V)$ such that $[\h,\widetilde{R}]\subseteq \widetilde{R}$ and  $\gl(V)=\h\oplus \widetilde{R}$. 

It is well--known that the nilpotent variety $\N(\h)$ is irreducible and contains a unique open $(\Ad H)$-orbit $\O_{\rm reg}$. Pick $e\in\O_{\rm reg}$ and regard it as a nilpotent  element of $\gl(V)$.
Since the closed subgroup $\rho(H)$ of $\GL(V)$ normalises $\h=({\rm d}_e\rho)(\h)$, the tangent space $T_e(\Ad \rho(H)\,.\,e)$ is contained in $\h$. On the other hand,  since all adjoint  $\GL(V)$-orbits are smooth,
$$T_e(\Ad \rho(H)\,.\,e)\subseteq\,\h\cap T_e((\Ad \GL(V)\,.\,e)=\,\h\cap [\gl(V),e]\,=\,\h\cap\big([\h,e]\oplus [\widetilde{R},e]\big)=\,[\h,e],$$ for $[\h,e]\subseteq \h$ and $\h\cap [\widetilde{R},e]\subseteq \h\cap\widetilde{R}=\{0\}$. Therefore,
$\dim\,(\Ad H)\,.\,e\le \dim\, [\h,e]=\dim\h-\dim \h_e,$ forcing $\dim C_H(e)\ge \dim\h_e$. As $\Lie(C_H(e))\subseteq \h_e$, we deduce the adjoint $H$-orbit of $e\in\O_{\rm reg}$ is smooth. Thanks to a well-known result of Springer, this entails that $p$ is a good prime for $H$; see \cite[Theorem~5.9]{Spr66}. 
\end{proof}
This completes the proof of Lemma~\ref{L7}, and Theorem~\ref{main-thm} follows.
\subsection{} In this subsection we assume that $G$ is simple simply connected algebraic $\k$-group and $p$ is a very good prime for $G$.  All standard results on representations of reductive groups used in what follows can be found in \cite{Jan03}.
Given a dominant weight $\lambda\in X(T)$ we denote by
$V(\lambda)$ the Weyl module of highest weight $\lambda$. The quotient 
$L(\lambda):=V(\lambda)/{\rm rad}\,V(\lambda)$ is a simple rational $G$-module of highest weight $\lambda$ and
$\dim V(\lambda)$ can be computed by using the Weyl dimension formula. The Lie algebra $\g=\Lie(G)$ acts on $L(\lambda)$ via the differential at $e\in G$ of the irreducible representation $\rho_\lambda\colon\,G\to GL(L(\lambda))$, and ${\rm d}_e\rho_\lambda\colon\,\g\to\gl(L(\lambda))$  is irreducible if and only if $\lambda$ is $p$-restricted (as defined in Subsection~\ref{3.6}). 
As before, we adopt Bourbaki's numbering of simple roots in $\Pi$ and let $\theta$ be the highest root of $\Phi$ with respect to $\Pi$. Since $\rho$ is already engaged we write $\delta$ for the half-sum of the roots in $\Phi_+$.


Given a finite dimensional representation $\rho\colon\,\g\to  \gl(M)$ we denote by 
${\rm tr}_M(h_\theta^2)$ the trace of the endomorphism $\rho(h_\theta)^2$. 
It is not hard to see for any two finite-dimensional restricted $\g$-modules $M$ and $N$ one has
\begin{equation}\label{E1}
{\rm tr}_{M\otimes N}\,(h_\theta)^2\,=\,(\dim\,M)\cdot	{\rm tr}_N(h_\theta^2)+(\dim\,N)\cdot	{\rm tr}_M(h_\theta^2)
\end{equation}
(one should keep in mind that $M$ and $N$ decompose into a direct sum of weight spaces with restpect to $\t$ and $h_\theta$ has zero trace on both $M$ and $N$).

Our goal in this subsection is to find a $p$-restricted dominant $\lambda\in X(T)$ such that $p\nmid\dim L(\lambda)$ and 
${\rm tr}_{L(\lambda)}(h_\theta)^2\ne 0$. In the characteristic zero case closely related quantities are often
computed by using the notion of {\it Dynkin index} defined in \cite[Ch.~I, \S 2]{Dyn}. Therefore, it will be convenient for us to regard $G$ as the group of $\k$-points of  a simply connected Chevalley group scheme $G_\Z$ with the same root datum as $G$. 
We shall also assume that $T$ is obtained by base-changing a maximal split torus of $T_\Z$ of $G_\Z$, so that $X(T)=X(T_\Z)$. Then $h_\theta=H_\theta\otimes_\Z 1$ for the semisimple root vector $H_\theta={\rm d}_e(\theta^\vee)\in\Lie(T_\Z)$. 


Let $\g_\Z$ be Lie algebra of $G_\Z$ and denote by $V_\Z(\lambda)$ the Weyl 
module for $G_\Z$ with highest weight $\lambda\in X(T)$. Then $\g=\g_\Z\otimes_\Z \k$ and $V(\lambda)=V_\Z\otimes_\Z\k$. We denote by $X(\lambda)$ the set of $T_\Z$-weights of $V_\Z(\lambda)$ and write $n_\mu$ for the multiplicity of $\mu\in X(\lambda)$ in $V_\Z(\lambda)$. The Dynkin index $d(\lambda)$ of $V_\Z(\lambda)$ is defined as
$$d(\lambda)\,:=\,\frac{1}{2}\sum_{\mu\in X(\lambda)}n_\mu\, \mu(H_\theta)^2\,=\,\frac{1}{2}\,{\rm tr}_{V_\Z(\lambda)}\,(H_\theta)^2.$$
It is well-known that $d(\lambda)$ is an integer; see \cite[2.3]{LaSo}, for example. For all fundamental dominant weights $\varpi_i$ with $1\le i\le {\rm rk}(G_\Z)$, the integers
$d_(\varpi_i)$ are computed in \cite[Table~5]{Dyn}, and the three 
misprints in type ${\rm E}_8$ are corrected in \cite[p.~504]{LaSo}. 

Let $(\,\cdot\,,\,\cdot\,)$ be the scalar product on the $X(T_\Z)\otimes_\Z \mathbb{R}$ invariant under the action of the Weyl group $W(\Phi)$ and such that $(\theta,\theta)=2$. Dynkin proved in
\cite{Dyn} that
\begin{equation}\label{E2}
	d(\lambda)\,=\,\frac{\dim\,V_\mathbb{Q}(\lambda)}{\dim\,\g_\mathbb{Q}}\cdot (\lambda+2\delta,\lambda).
\end{equation}	 
Here $V_\mathbb{Q}(\lambda)=V_\Z(\lambda)\otimes_\Z\mathbb{Q}$ and $\g_\mathbb{Q}=\g_\Z\otimes_\Z\mathbb{Q}$. 
Dynkin's original argument was based on his classification results, but a shorter proof was later found in \cite[Ch.~I, \S 3.10]{Oni}. We refer to \cite[\S 1]{Pan} for more detail on the history of this formula.
\begin{Proposition}\label{P3}
If $p$ is a very good prime for $G$, then there exists a $p$-restricted dominant weight $\lambda\in X(T)$ such that $p\nmid\dim\,L(\lambda)$ and 
${\rm tr}_{L(\lambda)}\,(h_\theta)^2\ne 0$.
\end{Proposition}		
	\begin{proof}
		If $G$ is $\SL_n$ or $\Sp_{2n}$, where $p\nmid n$, then the natural $G$-module $L(\varpi_1)$ satisfies all our requirements since $p\nmid \dim L(\varpi_1)$ and
		 ${\rm tr}_{L(\varpi_1)}\,(h_\theta)^2=2$. If $G=\Sp_{2n}$ and $p\mid n$ we can take  
		 $L(\varpi_2)$ instead. Indeed, \cite[Cor.~2]{PSup} shows that in this case   
		 ${\rm rad}\,V(\varpi_2)\cong L(0)$, implying that 
		 $\dim\,L(\varpi_2)={n\choose 2}-2$. Since $d(\varpi_2)=2n-2$ by \cite[p.~504]{LaSo}, we have that $${\rm tr}_{L(\varpi_2)}={\rm tr}_{V(\varpi_2)}=2d(\varpi_2)\bmod p\in{\mathbb F}_p^\times.$$
		 If $G$ is of type ${\rm B}_n$ or ${\rm D}_n$, then \cite[p.~504]{LaSo} shows that we can take for $\lambda$ the minuscule weight $\varpi_n$ as both $d(\varpi_n)$ and 
		 $\dim L(\varpi_n)=\dim V(\varpi_n)$ are powers of $2$ (and $p>2$).
		 
		 Now suppose $G$ is an exceptional group. Since $p$ is a good prime for $G$, the Killing form of $\g$ is non-degenerate. This is well--known and follows, for example, from 
		 the fact that $d(\g_\mathbb{Q})=2h^\vee$, where $h^\vee$ is the dual Coxeter number of $\Phi$; see \cite[\S 1]{Gar} for more detail.
		 If $p\nmid \dim\,\g$, we can take for $L(\lambda)$ the adjoint $G$-module $\g\cong L(\theta)$. Indeed, our assumptions on $p$ and $G$ imply that 
		 $\g$ is a simple Lie algebra and $p\nmid h^\vee$.
		 
		 From now on we may assume that $G$ is exceptional and $p\mid\dim\,\g$. If $G$ is of type ${\rm E}_6$
		 we can take for $\lambda$ the minuscule weight $\varpi_1$. In this case $V(\varpi_1)\cong L(\varpi_1)$ has dimension $27$ and $d(\varpi_1)=6$ by 
		 \cite[p.~504]{LaSo}. 
		 If $G$ is of type ${\rm E}_7$ and $p=7$ we can take for $\lambda$ the fundamental weight $\varpi_6$. Then $d(\varpi_6)=648$ by \cite[p.~504]{LaSo}, whilst a quick look at \cite[6.52]{Lue} reveals that ${\rm rad}\,V(\varpi_6)\cong L(0)$ has dimension $1$ and $\dim L(\varpi_6)=1538$. Since both $648$ and $1538$ are invertible 
		 in ${\mathbb F}_7$ and 
		 ${\rm tr}_{L(\varpi_6)}={\rm tr}_{V(\varpi_6)}=2d(\varpi_6)\bmod 7,$ this choice of $\lambda$ satisfies all our requirements. If $G$ is of type ${\rm E}_7$ and $p=19$
		 we take for $\lambda$ the minuscule weight $\varpi_1$. Then $V(\varpi_1)\cong L(\varpi_1)$ has dimension $56$ and $d(\varpi_1)=36$ by 
		 \cite[p.~504]{LaSo}. Since both $56$ and $36$ are invertible in ${\mathbb F}_{19}$ this is a good choice for us.
		 
		 If $G$ is of type ${\rm G}_2$, ${\rm F}_4$ or ${\rm E}_8$, then the good primes dividing $\dim\, \g$ are $7$, $13$ and $31$, respectively. The tables in \cite{Lue} indicate that we cannot use the fundamental highest weights to construct a suitable $L(\lambda)$.
		 
		 Suppose $G$ is of type ${\rm E}_8$ and $p=31$. From \cite[6.53]{Lue} we get $\dim V_\mathbb{Q}(2\varpi_8)=27000$ and $\dim L(2\varpi_8)=26999$. Hence ${\rm rad}\,V(\lambda)\cong L(0)$, implying that 
		 ${\rm tr}_{L(2\varpi_8)}\,(h_\theta)^2=2d(2\varpi_8)\bmod 31$. 
		 Since all roots in $\Phi$ have the same length we can take for $(\,\cdot\,,\,\cdot\,)$ the dot  product from \cite[Planche~VII]{Bour}. Then
		 $$(2\varpi_8+2\delta\mid 2\varpi_8)\,=\,4(\varepsilon_7+\varpi_8+\delta\mid\varepsilon_7+\varpi_8)\,=\,
		 4(\varepsilon_2+2\varepsilon_3+3\varepsilon_4+4\varepsilon_5+5\varepsilon_6+7\varepsilon_7+24\varepsilon_8
		 \mid\varepsilon_7+\varepsilon_8)\,=\,124.$$
		 Using (\ref{E2}) we obtain $2d(2\varpi_8)=\frac{27000}{248}\cdot 248=27000.$ Since both $26999$ and $27000$ are invertible in $\mathbb{F}_{31}$, the module $L(2\varpi_8)$ satisfies all our requirements.
		 
		 Suppose $G$ is of type ${\rm F}_4$ and $p=13$. By \cite[Planche~VIII]{Bour}, we can  choose for $(\,\cdot\,,\,\cdot\,)$ the scalar product $(\,\cdot\mid\cdot\,)$. Setting $\lambda:=2\varpi_4$ we get	$$(\lambda+2\delta\mid\lambda)\,=\,
		 (2\varepsilon_1+11\varepsilon_1+5\varepsilon_2+3\varepsilon_3+\varepsilon_1\mid 2\varepsilon_1)=26.$$ By \cite[6.50]{Lue}, we have that 
		 $\dim V_\mathbb{Q}(\lambda)=324$ and $\dim L(\lambda)=323$, which implies that ${\rm rad}\,V(\lambda)\cong L(0)$ and ${\rm tr}_{L(\lambda)}\,(h_\theta)^2=2d(\lambda)\bmod 13$. 
		 Since $2d(\lambda)=\frac{324}{52}\cdot 52=324$ by (\ref{E2}) and both $323$ and $324$ are invertible in $\mathbb{F}_{13}$, this choice of $\lambda$ is good for us.
		 
		 Suppose $G$ is of type ${\rm G}_2$ and $p=7$. By \cite[Planche~IX]{Bour}, we can choose for $(\,\cdot\,,\,\cdot\,)$ the dot product $\frac{1}{3}(\,\cdot\mid\cdot\,)$.
		 Set $\lambda:=2\varpi_1$. Keeping in mind the numbering of simple roots in  \cite[6.49]{Lue} we get $\dim V(\lambda)=27$ and $\dim L(\lambda)=26$.  So ${\rm rad}\,V(\lambda)\cong L(0)$, forcing ${\rm tr}_{L(\lambda)}\,(h_\theta)^2=2d(\lambda)\bmod 7$. Note that $\lambda=2\theta_0$, where  $\theta_0$ is the longest short root of $\Phi_+$, and $\delta=\theta_0+\theta$. Then 
		 $$(\lambda+2\delta\mid\lambda)\,=\,\frac{4}{3}(\theta_0+\delta\mid\theta_0)\,=\,
		 \frac{4}{3}(2\theta_0+\theta\mid\theta_0)\,=\,\frac{4}{3}(4+3)\,=\,\frac{28}{3}.$$
		 Then (\ref{E2}) gives  $2d(\lambda)=2\cdot\frac{27}{14}\cdot \frac{28}{3}=36$. As both $26$ and $36$ are invertible in $\mathbb{F}_{7}$, this choice of $\lambda$ is good for us. The proof of the proposition is now complete.	
	\end{proof}
	\begin{Corollary}\label{C2}
	Suppose $G$ is a standard reductive $\k$-group such that $p$ is a very good prime for $G$ and $\dim Z(G)\le 1$. Then there exists an infinitesimally irreducible rational representation $\rho\colon\, G\to\GL(V)$ such that ${\rm d}_e\rho$ is a 
	faithful irreducible representation of $\g=\Lie(G)$ and the trace form 
	$(X,Y)\mapsto {\rm tr}\big(({\rm d}_e\rho)(X)\circ({\rm d}_e\rho)(Y)\big)$ is non-degenerate on $\g$.
	\end{Corollary}
\begin{proof}
	Since the derived subgroup $\D G$ of $G$ is simply connected we have that $G=Z(G)\cdot\D G$ and  
	$\D G\cong  G_1\times\cdots\times G_s$ where $G_1, \ldots, G_s$ are the simple components of $G$.
	For each $i\le s$ we choose a $p$-restricted dominant weight $\lambda_i$ of $G_i$ 
	satisfying the requirements of Proposition~\ref{P3} and consider the irreducible $\D G$-module
	$V:=L(\lambda_1)\otimes\cdots \otimes L(\lambda_s)$. We allow $Z(G)$ to act on $V$ by scalar operators in such a way that the differential of the action is nonzero, and denote by $\rho$ the infinitesimally irreducible representation of $G$ obtained this way. The Lie algebra 
	$\g=\Lie(Z(G))\oplus \sum_{i=1}^s\,\Lie(G_i)$ then acts faithfully  
	on $V$. Since $p\nmid \dim L(\lambda_i)$ for all $i$, it is immediate from (\ref{E1}) that
	the restriction of the symmetric bilinear form $(X,Y)\mapsto {\rm tr}\big(({\rm d}_e\rho)(X)\circ ({\rm d}_e\rho)(Y)\big)$ on $\g$ to each $\Lie(G_i)$ and to $\Lie(Z(G))$is nonzero. Since each ideal $\Lie(G_i)$ is a simple Lie algebra, the result follows. 
	\end{proof}
\begin{Remark}
Let $R$ to be the orthogonal complement of $({\rm d}_e\rho)(\g)$ with respect to the non-degenerate trace form associated with $V$. Then $\gl(V)=({\rm d}_e\rho)(\g)\oplus R$ and $[({\rm d}_e\rho)(\g)\, R]\subseteq R$. Since ${\rm d}_e\rho$ is faithful, we see that $V$ is a Richardson module for $\g$. Furthermore, the complement $R$ is invariant under the conjugation action of $\rho(G)$ on $\gl(V)$. 
\end{Remark}
\begin{Remark}
Suppose $G$ is as in Corollary~\ref{C2} (with $s\ge 1$) and consider 
$\widetilde{G}=\D G\times GL_{kp}$ where $k\ge 1$.
It is well--known (and straightforward to see) that there exist $u,v\in\N_p(\gl_{kp})$ such that
$[u,v]=1_{kp}$. Let $\mathfrak{a}$ denote the $3$-dimensional restricted Lie subalgebra of $\gl_{kp}$ spanned by $u$, $v$ and $1_{kp}$. Representation theory of nilpotent Lie algebras
shows that all faithful irreducible $\mathfrak{a}$-modules have the same dimension equal to $p$.
Consequently, any faithful irreducible  $\gl_{kp}$-module has dimension divisible by $p$.  
Using (\ref{E1}) one observes that
 in contrast with Corollary~\ref{C2} the trace forms associated with the irreducible faithful representations of $\Lie(\widetilde{G})$ are always degenerate. 
 \end{Remark}
\bibliographystyle{amsalpha}
%\bibliography{bib}
\begin{thebibliography} 
	{BMRT10}
	\bibitem[Ben77]{Be77}
	G.M.~Benkart.
	\newblock On inner ideals and $\ad$-nilpotent elements of Lie algebras,
	\newblock
	{\em Trans. Amer. Math. Soc.}, 232:61--81, 1977.
	
	\bibitem[BGP09]{BGP}
	G.~Benkart, T.~Gregory and A.~Premet.
	\newblock The Recognition Theorem for Graded Lie Algebras in Prime Characteristic.
	\newblock {\em Mem. Amer. Math. Soc.}, 197(920):xii+145~pp., 2009.
	
	\bibitem[BF15]{BF}
	G.M.~Benkart and J.~Feldvoss.
	\newblock Some problems in the representation theory of simple modular Lie algebras. In: {\em Lie Algebras and Related Topics}, 207–-228, 
	\newblock Contemp. Math., Vol.~652, AMS, Providence, RI, 2015.
	
	%\bibitem[BIO79]{BIO}
	%G.M.~Benkart, I.M.~Isaacs and J.M.~Osborn.
	%\newblock Albert--Zassenhaus Lie algebras and isomorphisms,
	%\newblock
	%{\em J.~Algebra}, 57:310--338, 1979.
	
		%\bibitem[Bl62]{Bl}
	%R.E.~Block.
	%\newblock Trace forms on Lie algebras,
	%\newblock
	%{\em Canad. J. Math.}, 14:553--564, 1962.
	
	
	%\bibitem[BMRT10]{BMRT}
	%M.~Bate, B.~Martin, G.~R\"ohrle and R.~Tange.
	%\newblock Complete reducibility and separability,
	%\newblock
	%{\em Trans. Amer. Math. Soc.}, 362:4283--4311, 2010.
	
	%\bibitem[BW82]{BW}
	%R.E.~Block and R.L.~Wilson.
	%\newblock The simple Lie $p$-algebras of rank $2$,
	%\newblock
	%{\em Ann. of Math.}, 115:93--168, 1982.
	
    %\bibitem[Bl62]{Bl}
	%R.E.~Block.
	%\newblock Trace forms on Lie algebras,
	%\newblock
	%{\em Canad. J. Math.}, 14:553--564, 1962.
	
	%\bibitem[Bor91]{Borel}
	%A. Borel.
	%\newblock {\em Linear Algebraic Groups}.
	%\newblock Graduate Texts in
	%Mathematics, Vol.~126.
	%\newblock Springer-Verlag, New York, second edition, 1991.
	

	
	\bibitem[Bou68]{Bour}
	N.~Bourbaki.
	\newblock {\em Groupes et Alg{\`e}bres de Lie}, IV, V, VI.
	\newblock Hermann, Paris, 1968.
	
	%\bibitem[Bou75]{Bour1}
	%N.~Bourbaki.
	%\newblock {\em Groupes et Alg{\`e}bres de Lie}, VII, VIII.
	%\newblock Hermann, Paris, 1975.
	
	%\bibitem[CLNP]{CLNP}
	%J.F.~Carlson, Z.~Lin, D.K.~Nakano, B.J.~Parshall.
	%\newblock The restricted nullcone,
	%\newblock
	%{\em Contemp. Math.}, 325:51--75, 2003.
	
	%\bibitem[Car93]{Car}
	%R.W.~Carter.
	%\newblock {\em Finite Groups of Lie Type: Conjugacy Classes and Complex Characters}.
	%\newblock Wiley Classics Library. John Wiley \& Sons Ltd., Chichester, 1993.
	%\newblock Reprint of the 1985
	%Wiley-Interscience publication.
	
	%\bibitem[CE16]{CE}
	%A.~Castillo-Ramirez and A.~Elduque.
	%\newblock Some special features of Cayley algebras, and $G_2$, in low characteristic.
	%\newblock{\em J. Pure Appl. Algebra}, 220:1188-1205, 2016.
	
	%\bibitem[CP13]{CP}
	%M.C.~Clarke and A.~Premet.
	%\newblock The Hesselink stratification of nullcones and base change.
	%\newblock {\em Invent. Math.}, 191:631--669, 2013.
	
		
	%\bibitem[dGE09]{dG-E}
	%W.A.~de Graaf and A.~Elashvili.
	%\newblock Induced nilpotent orbits of the simple {L}ie algebras of %exceptional
	%type.
	%\newblock {\em Georgian Math. J.}, 16:257--278, 2009.
	
	%\bibitem[Do93]{Do}
	%S.~Donkin.
	%\newblock On tilting modules for algebraic groups,
	%\newblock {\em Math. Z.}, 212:39–60, 1993.
	
		\bibitem[Dyn57]{Dyn}
	E.B.~Dynkin,
	\newblock Semisimple subalgebras of semisimple Lie algebras, 
	\newblock{\em AMS Transl. Ser. II},  6: 111--244, 1957. 
	
	
%\bibitem[FSW15]{FSW}
	%J.~Feldvoss, S.~Siciliano and Th.~Weigel.
	%\newblock Restricted Lie algebras with $0$-PIM.
	%\newblock{Transform. Groups}, 21(2):377--398, 2016.
	
	
	\bibitem[Gar09]{Gar}
	S.~Garibaldi.
	\newblock Vanishing of trace forms in low characteristics.
	\newblock {\em Algebra Number Theory}, 3(5):543--566, 2009.
	\newblock With an appendix by Alexander Premet.
	
	\bibitem[GG82]{GG82}
	R.~Gordon and E.L.~Green
	\newblock Graded Artin algebras.
	\newblock {\em J.~Algebra}, 76:111--137, 1982.
	
\bibitem[FP86]{FP1}
	E.M.~Friedlander and B.J.~ Parhall.
	\newblock Support varieties for restricted Lie algebras,
	\newblock {\em Invent. Math.}, 86:553--562, 1986.
	
	
	\bibitem[FP88]{FP2}
	E.M.~Friedlander and B.J.~ Parhall.
	\newblock Modular representation theory of Lie algebras
	\newblock {\em Amer. J. Math.}, 110:1055--1094, 1988.
	
	
	%\bibitem[HS15]{HSMax}
	%S.~Herpel and D.~Stewart.
	%\newblock Maximal subalgebras of Catran type in exceptional Lie algebras,
	%\newblock {\em Selecta Math. (N.S.)},
	%22(2):765--799, 2016.
	
	%\bibitem[Hum67]{Hum}
	%J.E.~Humphreys.
	%\newblock Algebraic Groups and Modular Lie Algebras.
	%\newblock {\em Mem. Amer. Math. Soc.}, no.~71, 76~pp., 1967.
	
	
	
	\bibitem[Jac51]{Jac}
	N.~Jacobson.
 \newblock{Completely reducible Lie algebras of linear transformations}
 \newblock {\em Proc. Amer. Math. Soc.}, 2:105--113, 1951.
	
\bibitem[Ja71]{Ja71}
J.B.~Jacobs.
\newblock{On classifying simple Lie algebras
	of prime characteristic by nilpotent elements}
\newblock {\em J. Algebra}, 19:30--50, 1971.	
	
	\bibitem[Jan86]{Jan86}
J.C.~Jantzen.
\newblock{Kohomologie von $p$-Lie-algebren und nilpotente elementen}
\newblock {\em Abh. Math. Sem. Univ. Hamburg}, 56:191--219, 1986.
	
	\bibitem[Jan00]{Jan00}
	J.C.~Jantzen.
	\newblock{Modular representations of reductive Lie algebras}
	\newblock {J. Pure Appl. Algebra}, 152:133--185, 2000.
	
	\bibitem[Jan03]{Jan03}
J.~C. Jantzen, \emph{Representations of Algebraic Groups}, second ed.,
  Mathematical Surveys and Monographs, vol. 107, American Mathematical Society,
  Providence, RI, 2003.

\bibitem[Kos67]{Ko67}
A.I.~Kostrikin.
\newblock Squares of adjoiunt endomorphisms in simple Lie $p$-algebras.
\newblock {\em Math. USSR-Izv.}, 1:435--473, 1967.

\bibitem[Kos90]{K90}
A.I.~Kostrikin, \emph{Around Burnside}. Translated from the Russian and with a preface by James Wiegold. Ergebnisse der Mathematik und ihrer Grenzgebiete, vol.~20, Springer-Verlag, Berlin, 1990. 

\bibitem[KZ90]{KZ90}
A.I.~Kostrikin and E.I.~Zelmanov.
\newblock A theorem on sandwich algebras.
\newblock {\em Trudy Mat. Inst. Steklov}, 183:106--111, 1990.

    %\bibitem[J1]{J04}
	%J.C.~Jantzen.
	%\newblock Nilpotent orbits in representation theory.
	%\newblock In {\em Lie Theory}, {\em Progress in Mathematics}, Vol.~228,
	%\newblock Birkh{\"a}user Boston, Boston, MA, 2004,  pp.~1--211.
	
	
	
	%\bibitem[Ku91]{Ku}
	%M.I.~Kuznetsov.
	%\newblock Melikyan algebras as Lie algebras of type $G_2$.
	%\newblock {\em Comm. Algebra}, 19:1281--1312, 1991.
	
	%\bibitem[Law95]{Law}
	%R.~Lawther.
	%\newblock Jordan block sizes of unipotent elements in exceptional algebraic
	%groups.
	%\newblock {\em Comm. Algebra}, 23:4125--4156, 1995.
	
	\bibitem[LaSo97]{LaSo}
	I.~Laszlo and C.~Sorger.
	\newblock The line bundles of the moduli of parabolic $G$-bundles over curves and their sections.
	\newblock {\em Ann. Sci. {\'E}cole Norm. Sup.}, 30:499--525, 1997.
	
	%\bibitem[LT07]{LT07}
	%R.~Lawther and D.M.~Testerman.
	%\newblock Centres of centralizers of unipotent elements in %simple algebraic
	% groups,
	%preprint 2007, 301~pp.
	
	%\bibitem[LS04]{LS04}
	%M.W.~Liebeck and G.M.~Seitz.
	%\newblock The Maximal Subgroups of Positive Dimension in Exceptional Algebraic Groups.
	%\newblock {\em Mem. Amer. Math. Soc.}, 169(802):vi+227~pp., 2004.
	
	
	%\bibitem[LT11]{LT11}
	%R.~Lawther and D.M.~Testerman.
	%\newblock Centres of Centralizers of Unipotent Elements in Simple Algebraic
	%Groups.
	%\newblock {\em Mem. Amer. Math. Soc.}, 210(988):vi+188~pp., 2011.
	
	
	%\bibitem[LMT09]{LMT}
	%P.~Levy, G.J.~McNinch and D.M.~Testerman.
	%\newblock Nilpotent subalgebras of semisimple Lie algebras.
	%\newblock {\em C.R. Math. Acad. Sci. Paris}, 347:477--482, 2009.
	
	\bibitem[L\"ub01]{Lue}
	F.~L\"ubeck.
	\newblock Small degree representations of finite Chevalley groups in defining characteristic,
	\newblock
	{\em LMS J. Comput. Math.}, 4:135--169, 2001.
	
	%\bibitem[Ma45]{Mal}
	%A.I.~Mal'cev.
	%\newblock Commutative subalgebras of semi-simple Lie algebras (Russian).
	%\newblock {\em Bull. Acad. Sci. USSR S{\'e}r. Math.} [Izvestia Acad. Nauk SSSR], 9:291--300, 1945.
	
	%\bibitem[McN03]{McN03}
	%G.J.~McNinch.
	%\newblock Sub-principal homomorphisms in positive characteristic.
	%\newblock {\em Math. Z.}, 244:433--455, 2003.
	
	
	%\bibitem[McN05]{McN04}
	%G.J.~McNinch.
	%\newblock Optimal $\SL(2)$-homomorphisms,
	%\newblock {\em Comment. Math. Helv}, 80:391--426, 2005.
	
	%\bibitem[Mor56]{Mor}
	%V.V.~Morozov.
	%\newblock Proof of the theorem of regularity (Russian).
	%\newblock {\em Uspehi Mat. Nauk}, 11:191--194, 1956.
	
	\bibitem[Oni94]{Oni}
	A.L.~Onishchik.
	\newblock {\em Topology of Transitive Transformation Groups.}
	\newblock J.~Barth, Leipzig, 1994.
	
	\bibitem[Pan09]{Pan}
	D.I.~Panyushev. 
	\newblock On the Dynkin index of a principal $\sl_2$-subalgebra.
	\newblock {\em Adv. Math.}, 221:1115--1121, 2009.
	
	%\bibitem[PV10]{PV10}
	%D.I.~Panyushev and E.B.~Vinberg.
	%\newblock The work of Vladimir Morozov on Lie algebras.
	%\newblock {\em Tranform. Groups}, 15:1001--1013, 2010.
	%\newblock Special issue dedicated to V.V.~Morozov.
	
	%\bibitem[PeS15]{PeS}
	%J.~Pevtsova and J.~Stark.
	%\newblock Varieties of elementary subalgebras of maximal dimension for modular Lie algebras.
	%\newblock  arXiv preprint {\tt arXiv:1503.01043v1} [math.RT],
	%\newblock  2015.
	
	\bibitem[Pre86]{P86}
	A.~Premet.
	\newblock Inner ideals in modular Lie algebras,
	\newblock {\em Vestsi Acad. Navuk BSSR Ser. Fiz.-Mat. Navuk}, no.~5, pp.~11-15, 1986 
	[Russian].
	
	
	\bibitem[Pre87a]{P87a}
	A.~Premet.
	\newblock Lie algebras without strong degeneration,
	\newblock {\em Math. USSR-Sb.}, 57:151--164, 1987.
	
	\bibitem[Pre87b]{P87b}
	A.~Premet.
	\newblock On Cartan subalgebras of Lie $p$-algebras.
	\newblock {\em Math. USSR-Izv.}, 29:145--157, 1987.
	
	
	%\bibitem[Pre85]{P85}
    %A.~Premet.
	%\newblock Algebraic groups associated with Cartan Lie $p$-algebras,
	%\newblock {\em Math. USSR-Sb.}, 50:85--97, 1985.
	
	%\bibitem[Pre88]{P88}
	%A.~Premet.
	%\newblock Weights of infinitesimally irreducible representations of Chevalley groups over a %field of prime characteristic,
	%\newblock {\em Math. USSR-Sb.}, 61:167--183, 1988.
	
	%\bibitem[Pre89]{P89}
	%A.~Premet.
	%\newblock Weights of infinitesimally irreducible representations of Chevalley groups over a %field of prime characteristic,
	%\newblock {\em Math. USSR-Sb.}, 61:167--183, 1988.
	
	
	\bibitem[Pre90]{P90}
	A.~Premet.
	\newblock Regular Cartan subalgebras and nilpotent elements in restricted Lie algebras,
	\newblock {\em Math. USSR-Sb.}, 66:555--570, 1990.
 
	
	\bibitem[Pre91]{P91}
	A.~Premet.
	\newblock The Green ring of a simple three-dimensional Lie $p$-algebra,
	\newblock {\em Soviet Math. (Iz. VUZ)}, 35(10):51--60, 1991.
	
    %\bibitem[Pre92]{P92}
	%A.~Premet.
	%\newblock A theorem on the restriction of invariants and nilpotent elements in $W_n$,
	%\newblock {\em Math. USSR-Sb.}, 73(1):135--159, 1992.
	
	\bibitem[Pre94]{P94}
	A.~Premet.
	\newblock A generatlisation of Wilson's theorem on Cartan subalgebras of simple Lie algebras.
	\newblock {\em J. Algebra}, 167:641--703, 1994.
	
	%\bibitem[Pre95]{P95}
	%A.~Premet.
	%\newblock An analogue of the Jacobson--Morozov theorem
	%for Lie algebras of reductive groups of good characteristics,
	%\newblock {\em Trans. Amer. Math. Soc.}, 347:2961--2988, 1995.
	
\bibitem[Pre97]{Pr97}
A.~Premet.
\newblock Support varieties of non-restricted modules over Lie algebras of reductive groups,
\newblock {\em J. London Math. Soc. (2)}, 55:236--2250, 1997.	
	
	\bibitem[Pre99]{P99}
	A.~Premet.
	\newblock Complexity of Lie algebra representations and nilpotent elements of the stabilizers of linear forms,
	\newblock {\em Math. Z.}, 228:255--282, 1999.
	
	\bibitem[Pre03a]{P03a}
	A.~Premet.
	\newblock Nilpotent orbits in good characteristic and the Kempf--Rousseau
	theory,
	\newblock {\em J. Algebra}, 260:338--366, 2003.
	\newblock Special issue celebrating the 80th birthday of Robert Steinberg.
	
    \bibitem[Pre03b]{P03b}
    A.~Premet.
    \newblock Nilpotent commuting varieties of reductive Lie algebras,
    \newblock {\em Invent. Math.}, 154:653--683, 2003.
    
    
    %\bibitem[Pre17]{P17}
	%A.~Premet.
	%\newblock A modular analogue of Morozov's theorem on maximal subalgebras of simple Lie algebras,
	%\newblock {\em Adv. Math.}, 311:833--884, 2017.
	
	
	\bibitem[PSk99]{PSk}
	A.~Premet and S.~Skryabin.
	\newblock Representations of restricted {L}ie algebras and families of
	associative {$\mathcal L$}-algebras.
	\newblock {\em J.~Reine Angew. Math.}, 507:189--218, 1999.
	
	
	%\bibitem[PSt16]{PSt}
	%A.~Premet and D.~Stewart.
	%\newblock Rigid orbits and sheets in reductive Lie algebras over fields of prime characteristic,
	%\newblock {\em J. Inst. Math. Jussieu},  doi:10.1017/S1474748016000086, to appear.
	
	\bibitem[PS97]{PS1}
	A.~Premet and H.~Strade.
	\newblock Simple Lie algebras of small characteristic: I. Sandwich elements.
	\newblock {\em J.~Algebra}, 189:419--480, 1997.
	
	
	%\bibitem[PS99]{PS2}
	%A.~Premet and H.~Strade.
	%\newblock Simple Lie algebras of small characteristic: II. Exceptional roots.
	%\newblock {\em J.~Algebra}, 216:190--301, 1999.
	
	
	%\bibitem[PS01]{PS3}
	%A.~Premet and H.~Strade.
	%\newblock Simple Lie algebras of small characteristic: III. The toral rank $2$ case.
	%\newblock {\em J.~Algebra}, 242:236--337, 2001.
	
	%\bibitem[PS08]{PS6}
	%A.~Premet and H.~Strade.
	%\newblock Simple Lie algebras of small characteristic: VI. Completion of the classification.
	%\newblock {\em J.~Algebra}, 320:3559--3604, 2008.
	
	%\bibitem[PSu83]{PSup}
	%A.~Premet and I.D.~Suprunenko.
	%\newblock Quadratic modules for Chevalley groups over fields of odd characteristic,
	%\newblock {\em Math. Nachr.}, 110:65--96, 1983.
	
	\bibitem[PSu83]{PSup}
	A.~Premet and I.D.~Suprunenko.
	\newblock The Weyl modules and the irreducible representations of the symplectic group with the fundamental highest weights,
	\newblock {\em Commun. Algebra}, 11:1309--1342, 1983.
	
	%\bibitem[Pur16]{Tom}
	%Th.~Purslow.
	%\newblock The restricted Ermolaev algebra and ${\rm F}_4$,
	%\newblock  arXiv preprint {\tt arXiv:1608.05124v2} [math.RT], 2016;
	%\newblock
	%to appear in \newblock {\em Emperiment. Math.}
	
%	\bibitem[Sei87]{Sei87}
%G.~M. Seitz, \emph{The maximal subgroups of classical algebraic groups}, Mem.
%  Amer. Math. Soc. \textbf{67} (1987), no.~365, iv+286.

%\bibitem[Sei91]{Sei91}
%G.~Seitz.
%\newblock Maximal Subgroups of Exceptional Algebraic Groups.
%\newblock {\em Mem. Amer. Math. Soc.}, 90(441): iv+197~pp., 1991.
	
	
%\bibitem[Sei00]{Sei}
%G.~Seitz.
%\newblock Unipotent elements,tilting modules, and saturation,
%\newblock
%{\em Invent. Math.}, 141:467--502, 2000.
	
	
	\bibitem[Sel67]{Sel}
	G.~Seligman.
	\newblock {\em Modular Lie Algebras}.
	\newblock
	Ergebnisse der Mathematik und ihrer Grenzgebiete, Band~40,
	\newblock Springer-Verlag, New York, 1967.
	
	
	\bibitem[Sk98]{Sk}
	S.~Skryabin.
	\newblock Toral rank $1$ simple Lie algebras ot low characteristics,
	\newblock
	{\em J. Algebra}, 200:650--700, 1998.
	
	%\bibitem[Ser06]{Serre}
	%J.-P.~Serre.
	%\newblock Coordonn{\'e}es de Kac.
	%\newblock {\em Oberwolfach Reports}, 3:1787--1790, 2006.
	
	%\bibitem[Spa84]{Spa}
	%N.~Spaltenstein.
	%\newblock Existence of good transverse slices to nilpotent orbits in good
	%characteristic,
	%\newblock
	%{\em J. Fac. Univ. Tokyo Sect. IA, Math.}, 31:283--2286, 1984.
	
  \bibitem[Spr66]{Spr66}
	T.A..~Springer.
	\newblock Some arithmetic results on semi-simple Lie algebras,
	\newblock
	{\em  Inst. Hautes {\'E}tudes Sci. Publ. Math.}, 30:115--141, 1966.
	
	
	%\bibitem[Sob15]{Sob}
	%P.~Sobaje.
	%\newblock Springer isomorphisms in characteristic $p$,
	%\newblock
	%{\em Transform. Groups}, 20:1141--1153, 2015.
	
	%\bibitem[Ste17]{Ste}
	%D.I.~Stewart.
	%\newblock On the minimal modules for exceptional Lie algebras:
	%Jordan blocks and stabilisers,
	%\newblock
	%{\em LMS J. Comput. Math.}, 19:235--258, 2016.
	
	%\bibitem[ST16]{ST}
	%D.I.~Stewart and A.~Thomas.
	%\newblock The Jacobson--Morozov theorem and complete reducibility of Lie subalgebras,
	%\newblock
	%{\em Proc. London Math. Soc.},  doi:10.1112/plms.12067, to appear.
	
	
	%\bibitem[Skr91]{Skr91}
	%S.M.~Skryabin.
	%\newblock Modular Lie algebras of Cartan type over algebraically %non-closed fields, I.
	%\newblock {\em Comm. Algebra}, 19:1629--1741, 1991.
	
	%\bibitem[Skr98]{Skr98}
	%S.M.~Skryabin.
	%\newblock Toral rank one simple Lie algebras of low characteristic.
	%\newblock {\em J.~Algebra}, 200:650--700, 1998.
	
	%\bibitem[Spr66]{Spr}
	%T.A.~Springer.
	%\newblock Some arithmetic results on semi-simple Lie algebras.
	%\newblock {\em Inst. Hautes {\'E}tudes Sci. Publ. Math.}, 30:115--141, 1966.
	
	%\bibitem[St61]{St}
	%R.~Steinberg.
	%\newblock Automorphisms of classical Lie algebras.
	%\newblock {\em Pacific J. Math.}, 11:1119--1129, 1961.
	
	\bibitem[Str73]{St73}
	H.~Strade.
	\newblock Nonclassical simple Lie algebras and strong degeneration.
	\newblock {\em Arch. Math. (Basel)}, 24:482--485, 1973.
	
	%\bibitem[Str04]{Str04}
	%H.~Strade.
	%\newblock {\em Simple Lie Algebras over Fields of Positive Characteristic. I. Structure Theory}.
	%\newblock
	%de Gruyter Expositions in Mathematics, Vol.~38,
	%\newblock Walter de Guyter \& Co., Berlin, 2004.
	
	\bibitem[Str17]{Str17}
	H.~Strade.
	\newblock {\em Simple Lie Algebras over Fields of Positive Characteristic. Vol.~II. Classifying the Absolute Toral Rank Two Case}. Second edition.
	\newblock
	De Gruyter Expositions in Mathematics, Vol.~42,
	\newblock Walter de Guyter \& Co., Berlin, 2017.

	%\bibitem[Tes88]{Tes88}
	%D.M.~Testerman.
	%\newblock Irreducible Subgroups of Exceptional Algebraic Groups.
	%\newblock {\em Mem. Amer. Math. Soc.}, 75(390): iv+90, 1988.
	

		%\bibitem[Tes89]{Tes89}
    %D.M.~Testerman. A construction of certain maximal subgroups of the algebraic
    %groups {$E_6$} and {$F_4$}, J. Algebra \textbf{122} no.~2, 299--322, 1989.
		
	%\bibitem[Tes95]{Tes}
	%D.M.~Testerman.
	%\newblock ${\rm A}_1$-type overgroups of elements of order $p$ in semisimple algebraic groups and the associated finite groups,
	%\newblock {\em J.~Algebra}, 177:34--76, 1995.

	%\bibitem[VAG04]{UGA04}
	%University of~Georgia VIGRE Algebra~Group.
	%\newblock Varieties of nilpotent elements for simple %{L}ie algebras~{I}: Good primes.
	%\newblock {\em J. Algebra}, 280:719--737, 2004.
	%\newblock The University of Georgia VIGRE Algebra Group: %D.J.~Benson,
	% P.~Bergonio, B.D.~Boe, L.~Chastkofsky, D.~Cooper, %J.~Jungster,
	% G.M.~Guy, J.J.~Hyun,
	% G.~Matthews, N.~Mazza, D.K.~Nakano, K.J.~Platt.
	
	%\bibitem[VAG05]{UGA05}
	%University of~Georgia VIGRE Algebra~Group.
	%\newblock Varieties of nilpotent elements for simple {L}ie algebras:~{II}.
	%{B}ad primes.
	%\newblock {\em J. Algebra}, 292:65--99, 2005.
	%\newblock The University of Georgia VIGRE Algebra Group: D.J.~Benson,
	%P.~Bergonio, B.D.~Boe, L.~Chastkofsky, B.~Cooper,
	%G.M.~Guy, J.~Hower, M.~Hunziker, J.J.~Hyun,
	%J.~Kujawa,
	%G.~Matthews, N.~Mazza, D.K.~Nakano, K.J.~Platt and C.~Wright.
	
	%\bibitem[Weis66]{W1}
	%B.Ju.~Weisfeiler.
	%\newblock A class of unipotent subgroups of semisimple algebraic groups (Russian).
	%\newblock {\em Uspehi Mat. Nauk}, 21:222--223, 1966.
	%\newblock {\tt arXiv:math/0005149v1} [math.AG] (English translation).
	%\bibitem[Weis78]{W2}
	%B.Ju.~Weisfeiler.
	%\newblock On the structure of the minimal ideal of some graded Lie algebras of characteristic $p>0$.
	%\newblock {\em J.~Algebra.}, 53:344--361, 1978.

\bibitem[Zel91]{Z91}
E.I.~Zelmanov.
\newblock Solution of the restricted Burnside problem for groups of odd exponent.
\newblock {\em Math. USSR-Izv.}, 36:41--60, 1991.	
\end{thebibliography}

\end{document}

