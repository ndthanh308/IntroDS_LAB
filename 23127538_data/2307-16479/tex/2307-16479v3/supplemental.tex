% \documentclass[]{article}
\documentclass[secnumarabic,amssymb, nobibnotes, aps, prd]{revtex4-2}

\RequirePackage[english]{babel}
\RequirePackage[utf8]{inputenc}
\RequirePackage{hyperref}
\hypersetup{colorlinks=true, citecolor=blue, urlcolor=blue, linkcolor=blue}

\RequirePackage[]{amsmath,amssymb}
\RequirePackage{bbm}
\RequirePackage[]{bm}

\RequirePackage[]{graphicx}

\RequirePackage{physics}
\RequirePackage{siunitx}
\sisetup{%
	exponent-product = \times,
	output-decimal-marker = {.},
	range-units = single,
	list-units = single,
	group-minimum-digits = 4,
	group-digits = integer,
	group-separator = {\,},
	per-mode = symbol,
	separate-uncertainty = true,
	detect-all = true}
% \DeclareSIUnit\Erec{\ensuremath{\textit{E}_\text{rec}}}

\RequirePackage{cleveref}

\usepackage[dvipsnames]{xcolor}
\newcommand{\red}[1]{\textcolor{red}{\textbf{#1}}}
\newcommand{\green}[1]{\textcolor{ForestGreen}{\textbf{#1}}}
\renewcommand{\thefigure}{S\arabic{figure}}

\begin{document}

\title{Supplemental material for \\ \textit{Interferometric phase estimation with two mode quantum states}}
%\date{\today}

\author{Quentin Marolleau}
% \email[contact: ]{\texttt{name}.\texttt{surname}@institutoptique.fr}
\author{Charlie Leprince}
\author{Victor Gondret}%
\author{Denis Boiron}%
\author{Christoph I. Westbrook}
\affiliation{Université Paris-Saclay, Institut d’Optique Graduate School, CNRS, Laboratoire Charles Fabry, 91127, Palaiseau, France}
\date{\today}%

\maketitle

\tableofcontents
\pagebreak

\section{Parametrization of the problem}

\subsection{Operator definitions}

% Figure environment removed

Following the notation of \cref{fig:schematic_MZI}, we denote $N$ the average value of the total number of atoms in the
interferometer:
\begin{equation}
	N \triangleq \ev{\hat{N}_{a_1} + \hat{N}_{a_2}} = \ev{\hat{N}_{b_1} + \hat{N}_{b_2}}
\end{equation}
We have
\begin{equation}
	\hat{S} = \frac{1}{\sqrt{2}} \begin{pmatrix}
		1  & 1 \\
		-1 & 1
	\end{pmatrix}
\end{equation}
corresponding to the special case of a beam splitter that does not apply any phase shift.
\begin{equation}
	\hat{\Phi} = \begin{pmatrix}
		e^{i \phi} & 0 \\
		0          & 1
	\end{pmatrix}
\end{equation}
\begin{equation}
	\hat{U} = \hat{S} \hat{\Phi} \hat{S} = e^{i \frac{\phi}{2}} \begin{pmatrix}
		i \sin(\frac{\phi}{2}) & \cos(\frac{\phi}{2})    \\
		-\cos(\frac{\phi}{2})  & -i \sin(\frac{\phi}{2})
	\end{pmatrix}
\end{equation}

\begin{equation}
	\binom{\hat{b}_1}{\hat{b}_2} = \hat{U} \binom{\hat{a}_1}{\hat{a}_2}
\end{equation}
We also have the beam splitters modelling the losses:
\begin{equation}
	\hat{S}_\eta =
	\begin{pmatrix}
		\sqrt{\eta}     & \sqrt{1-\eta} \\
		- \sqrt{1-\eta} & \sqrt{\eta}
	\end{pmatrix}
\end{equation}
such that
\begin{equation}
	\hat{c}_i = \sqrt{\eta} \, \hat{b}_i + \sqrt{1-\eta} \, \hat{v}_i \quad / \quad i \in \{ 1,2 \}
\end{equation}
We finally introduce notations for the number operators:
\begin{equation}
	\hat{N}_{\alpha_i} = \hat{\alpha}^\dagger_i \hat{\alpha}_i \quad / \quad  \alpha \in \{a,b,c \} \ , \ i \in\{ 1,2 \}
\end{equation}
input spin operators:
\begin{equation}
	\begin{cases}
		\hat{J}_x =\frac 12 \left( \hat{a}^\dagger_1 \hat{a}_2  + \hat{a}_2^\dagger \hat{a}_1\right)    \\
		\hat{J}_y =\frac 1{2i} \left( \hat{a}^\dagger_1 \hat{a}_2  - \hat{a}_2^\dagger \hat{a}_1\right) \\
		\hat{J}_z =\frac 12 \left( \hat{N}_{a_1} - \hat{N}_{a_2} \right)
	\end{cases}
\end{equation}
and the observables of interest:
\begin{equation}
	\left\{
	\begin{aligned}
		\hat{D}_\eta & = \frac{1}{2} \left( \hat{N}_{c_2} - \hat{N}_{c_1} \right)                      \\
		\hat{D}      & = \frac{1}{2} \left( \hat{N}_{b_2} - \hat{N}_{b_1} \right) = \hat{D}_{\eta = 1}
	\end{aligned}
	\right.
\end{equation}

\subsection{Two-mode squeezed vacuum state and preliminary results}

We recall the definition of a two-mode squeezed vacuum (TMS) state, with average total population $N$:
\begin{equation}
	\ket{\textrm{TMS}} \triangleq \sqrt{ \frac{2}{2+N}} \sum\limits_{n=0}^{\infty} \left( \frac{N}{2+N} \right)^{\frac n2} \ket{n,n}
\end{equation}
in order to keep compact notations during the calculations, we will often use the \emph{thermal weight}:
\begin{equation}
	P_{th}(n) = \frac{2}{2+N}\left( \frac{N}{2+N} \right)^n
\end{equation}
corresponding to the probability to measure $n$ particles in a given mode of the TMS state.

\paragraph*{} We also highlight the fact that
\begin{equation}
	\hat{J}_z \ket{n,n} = \hat{J}_z \ket{\mathrm{TMS}} = 0
	\label{eq:jz_gives_zero}
\end{equation}
and finally:
\begin{equation}
	\ev{\hat{J}_x} = \ev{\hat{J}_y} = 0
	\label{eq:ev_jx_jy_is_zero}
\end{equation}
for both twin Fock and two-mode squeezed vacuum states.

\section{Expansion of $\hat{D}$, $\hat{D}^2$, $\hat{D}_\eta$ and $\hat{D}^2_\eta$}

\begin{equation}
	\hat{D} = \cos (\phi) \hat{J}_z - \sin (\phi) \hat{J}_y
\end{equation}

\begin{equation}
	\hat{D}^2 = \cos^2 (\phi) \hat{J}_z^2 + \sin^2 (\phi) \hat{J}_y^2 - 2 \sin (\phi) \cos (\phi) \hat{J}_y \hat{J}_z
	+ i \sin (\phi) \cos (\phi) \hat{J}_x
	\label{eq:expansion_of_D2}
\end{equation}
Since there is no particle in the vacuum channels for the input state, we always have $\hat{v}_i \ket{\psi}_{\textrm{input}} = 0$.
For the sake of simplicity, we reduce the writing of $\hat{D}_\eta$ and $\hat{D}^2_\eta$ to the terms giving  a non zero
contribution. This means that (for either $\hat{D}_\eta$ and $\hat{D}_\eta^2$) we drop all the terms containing $\hat{v}_{i \in \{ 1,2 \}}$ annihilation operators on their
rightmost side:
\begin{equation}
	\hat{D}_\eta = \eta \hat{D} + \frac 12 \sqrt{\eta (1-\eta)} \left( \hat{v}^\dagger_2 \, \hat{b}_2 - \hat{v}^\dagger_1 \, \hat{b}_1 \right)
\end{equation}

\begin{equation}
	\begin{split}
		\hat{D}_\eta^2 = \eta^2 \hat{D}^2 + \frac {\eta(1-\eta)}{4} \left[ (\hat{v}^\dagger_1)^2 \, \hat{b}_1^2
			+ (\hat{v}^\dagger_2)^2 \, \hat{b}_2^2 + \hat{N}_{b_1}+\hat{N}_{b_2} - 2 \, \hat{b}_1 \hat{b}_2 \, \hat{v}^\dagger_1 \hat{v}^\dagger_2 \right] \\
		+ \frac{\eta \sqrt{\eta (1-\eta)}}{2} \left[ 2 \hat{D} \, \hat{b}_2 \hat{v}_2^\dagger + \frac 12 \hat{b}_2 \hat{v}_2^\dagger
			- 2 \hat{D} \, \hat{b}_1 \hat{v}_1^\dagger + \frac 12 \hat{b}_1 \hat{v}_1^\dagger \right] \\
		+ \frac{(1-\eta) \sqrt{\eta (1-\eta)}}{4} \left[ \left( 2 \hat{N}_{v_2} - \mathbbm{1} \right) \hat{b}_2 \hat{v}^\dagger_2
			+ \left( 2 \hat{N}_{v_1} - \mathbbm{1} \right) \hat{b}_1 \hat{v}^\dagger_1 \right]
	\end{split}
	\label{eq:D_eta_2}
\end{equation}

\section{Expectation values of $\hat{D}^2$ and $\hat{D}^2_\eta$}

\subsection{Lossless case}

\subsubsection*{With twin Fock states}

\begin{equation}
	\hat{J}_y^2 = \frac 14 \left[ \hat{N}_{a_1} \left( \mathbbm{1} + \hat{N}_{a_2} \right)
		+ \hat{N}_{a_2} \left( \mathbbm{1} + \hat{N}_{a_1} \right) - ( \hat{a}^\dagger_1 )^2 \, \hat{a}_2^2
		- ( \hat{a}^\dagger_2 )^2 \, \hat{a}_1^2 \right]
\end{equation}
thus with \eqref{eq:jz_gives_zero} \eqref{eq:ev_jx_jy_is_zero} and \eqref{eq:expansion_of_D2},
\begin{equation}
	\ev{\hat{D}^2}_{\mathrm{tf}} = \ev{\hat{J}_y^2}_{\mathrm{tf}} \sin^2 (\phi) = \frac{N}{4} \left( 1 + \frac{N}{2} \right) \sin^2 (\phi)
	\label{eq:d2_perfect_tf}
\end{equation}

\subsubsection*{With two-mode squeezed vacuum states}

We can check that
\begin{equation}
	m \neq n \Rightarrow \mel**{n,n}{\hat{D}^2}{m,m} = 0
\end{equation}
therefore assuring the simple relation:
\begin{equation}
	\ev{\hat{D}^2}_{\mathrm{tms}} = \sum\limits_{n=0}^\infty P_{th}(n) \ev{\hat{D}^2}_{\mathrm{tf}}
\end{equation}
leading to
\begin{equation}
	\ev{\hat{D}^2}_{\mathrm{tms}} = \frac N2 \left( 1 + \frac N2 \right) \sin^2 (\phi) = 2 \ev{\hat{D}^2}_{\mathrm{tf}}
	\label{eq:d2_perfect_tms}
\end{equation}

\subsection{Lossy case (i.e. eq. (8) in the main paper)}

\subsubsection*{With twin Fock states}

\vspace*{1ex}
Whatever the output state of the interferometer, we actually always have:
\begin{equation}
	\mathrm{Var}\left[ \hat{N}_{c_2} - \hat{N}_{c_1} \right] =
	\eta^2 \, \mathrm{Var}\left[ \hat{N}_{b_2} - \hat{N}_{b_1} \right]
	+ \eta (1-\eta) \, \ev{\hat{N}_{b_2} + \hat{N}_{b_1}}
\end{equation}
which in our case means:
\begin{equation}
	\ev{\hat{D}_\eta^2}_{\mathrm{tf}} = \eta^2 \ev{\hat{D}^2}_{\mathrm{tf}} + \frac{\eta (1- \eta)}{4} N
	\label{eq:d2_eta_tf}
\end{equation}
and therefore
\begin{equation}
	\ev{\hat{D}_\eta^2}_{\mathrm{tf}} = \eta^2 \frac N4 \left( 1 + \frac N2 \right) \sin^2 (\phi) + \frac{\eta (1- \eta)}{4} N
\end{equation}

\subsubsection*{With two-mode squeezed vacuum states}

Again we can check on \cref{eq:D_eta_2} that
\begin{equation}
	m \neq n \Rightarrow \mel**{n,n}{\hat{D}_\eta^2}{m,m} = 0
\end{equation}
thus we still have
\begin{equation}
	\ev{\hat{D}_\eta^2}_{\mathrm{tms}} = \sum\limits_{n=0}^\infty P_{th}(n) \ev{\hat{D}_\eta^2}_{\mathrm{tf}}
\end{equation}
and finally:
\begin{equation}
	\ev{\hat{D}_\eta^2}_{\mathrm{tms}} = \eta^2 \frac N2 \left( 1 + \frac N2 \right) \sin^2 (\phi) + \frac{\eta (1- \eta)}{4} N
	\label{eq:d2_eta_tms}
\end{equation}

\section{Expectation values of $\hat{D}^4$ and $\hat{D}^4_\eta$}

\subsection{Lossless case}

\subsubsection*{With twin Fock states}

We compute  $\ev{\hat{D}^4}_{\mathrm{tf}} = \norm{\hat{D}^2 \ket{\frac N2, \frac N2}}^2$. The only
non-vanishing term of $\hat{D}^2 \ket{n,n}$ are:
\begin{equation}
	\begin{cases}
		\displaystyle \hat{J}_y^2 \ket{n,n} & \displaystyle = \frac 14 \left( 2n \left( 1 + n\right) \ket{n,n}

		
		
		
		
		
		
		
		
		
		
		
		
		
		
		
		- \sqrt{\left( n - 1 \right) n \left( n + 1 \right) \left( n + 2 \right)} \big[ \ket{n+2,n-2} + \ket{n-2,n+2} \big]\right) \\
		\\
		\displaystyle \hat{J}_x \ket{n,n}   & = \displaystyle \frac 12 \sqrt{n(n+1)} \big( \ket{n+1,n-1} + \ket{n-1,n+1} \big)
	\end{cases}
	\label{eq:Jy2_Jx_system}
\end{equation}
all these vectors are mutually orthogonal, then:
\begin{equation}
	\norm{\hat{D}^2 \ket{\frac N2, \frac N2}}^2 = \frac{N}{4} \left( 1 + \frac N2 \right) \sin^2 (\phi) \left[ 1+ \frac{3}{2}
		\left( -1 + \frac N4 + \frac{N^2}{8} \right) \sin^2 (\phi) \right]
	\label{eq:d4_perfect_tf}
\end{equation}

\subsubsection*{With two-mode squeezed vacuum states}

Looking at \cref{eq:Jy2_Jx_system}, we can convince ourselves that the decomposition in the two-mode Fock basis of
$\hat{J}_y^2 \ket{n,n}$ and $\hat{J}_x \ket{n,n}$ with $n \in \mathbb{N}$ generate vectors that are all mutually orthogonal.

Therefore
\begin{equation}
	\norm{\hat{D}^2 \ket{\mathrm{TMS}}}^2 = \sum\limits_{n=0}^\infty P_{th}(n) \norm{\hat{D}^2 \ket{n, n}}^2
\end{equation}
and thus,
\begin{equation}
	\norm{\hat{D}^2 \ket{\mathrm{TMS}}}^2 = \frac{N}{2} \left( 1 + \frac N2 \right) \sin^2 (\phi) \left[ 1+ \frac{9N}{2}
		\left( 1 + \frac N2 \right) \sin^2 (\phi) \right]
	\label{eq:d4_perfect_tms}
\end{equation}

\subsection{Lossy case}

We follow the same procedure as before, but here the number of non-vanishing terms is much larger. We will only write
the final results.

\subsubsection*{With twin Fock states}

With
\begin{equation}
	\begin{cases}
		P^{\mathrm{tf}}_0(N,\eta) & = 64 - 320 \, \eta + 256 \, \eta N + 384 \, \eta^2 - 384 \, \eta^2 N + 96 \, \eta^2 N^2 - 144 \, \eta^3 + 156 \, \eta^3 N - 60 \, \eta^3 N^2 + 9 \, \eta^3 N^3 \\
		P^{\mathrm{tf}}_1(N,\eta) & = - 4 \, (N+2) \, \eta \left( 16 + 24 \, (N-2) \, \eta + 3 \left( 8 - 6N + N^2 \right) \eta^2 \right)                                                          \\
		P^{\mathrm{tf}}_2(N,\eta) & = 3 \left( -16 - 4N + 4N^2 + N^3 \right) \eta^3
	\end{cases}
\end{equation}
we have:
\begin{equation}
	\norm{\hat{D}_\eta^2 \ket{\frac N2, \frac N2}}^2 = \frac{\eta N}{1024}
	\Big[ P^{\mathrm{tf}}_0(N,\eta) + P^{\mathrm{tf}}_1(N,\eta) \cos (2\phi) + P^{\mathrm{tf}}_2(N,\eta) \cos(4\phi) \Big]
	\label{eq:d4_eta_tf}
\end{equation}

\subsubsection*{With two-mode squeezed vacuum states}

With
\begin{equation}
	\begin{cases}
		P^{\mathrm{tms}}_0(N,\eta) & = 8 + 24 \, \eta + 64 \, \eta N + 48 \, \eta^2 N + 72 \, \eta^2 N^2 + 12 \, \eta^3 N + 36 \, \eta^3 N^2 + 27 \, \eta^3 N^3 \\
		P^{\mathrm{tms}}_1(N,\eta) & = - 4 \, (N+2) \, \eta \left( 4 + 18 \, \eta N + 9 \, \eta^2 N^2 \right)                                                   \\
		P^{\mathrm{tms}}_2(N,\eta) & = 9N \left( N+2 \right)^2 \eta^3
	\end{cases}
\end{equation}
we have:
\begin{equation}
	\norm{\hat{D}_\eta^2 \ket{\mathrm{TMS}}}^2 = \frac{\eta N}{128}
	\Big[ P^{\mathrm{tms}}_0(N,\eta) + P^{\mathrm{tms}}_1(N,\eta) \cos (2\phi) + P^{\mathrm{tms}}_2(N,\eta) \cos(4\phi) \Big]
	\label{eq:d4_eta_tms}
\end{equation}

\section{Phase uncertainty $\Delta \phi$}

Phase uncertainties are computing using
\begin{equation}
	\Delta \phi =\frac{\sqrt{\mathrm{Var}\left[\hat{D}_\eta^2\right]}}{\left|\dfrac{\partial}{\partial \phi}\left[ \ev{\hat{D}_\eta^2} \right] \right|}.
	\label{eq:def_delta_phi}
\end{equation}

\subsection{Lossless case (i.e. eq. (10) in the main paper)}

\subsubsection*{With twin Fock states}

Injecting \eqref{eq:d2_perfect_tf} and \eqref{eq:d4_perfect_tf} into \cref{eq:def_delta_phi}, we get:
\begin{equation}
	\Delta \phi_{\mathrm{tf}} = \dfrac{1}{\cos (\phi ) \sqrt{N (N+2)}} \sqrt{2 + \left(-3 + \frac N4 + \frac{N^2}{8}\right) \sin ^2(\phi )}
\end{equation}

\subsubsection*{With two-mode squeezed vacuum states}

Injecting \eqref{eq:d2_perfect_tms} and \eqref{eq:d4_perfect_tms} into \cref{eq:def_delta_phi}, we get:
\begin{equation}
	\Delta \phi_{\mathrm{tms}} = \dfrac{1}{\cos (\phi ) \sqrt{N (N+2)}} \sqrt{1 + 2 N (N+2) \sin ^2(\phi )}
\end{equation}

\subsection{Lossy case}

\subsubsection*{With twin Fock states}

% \begin{equation}
% 	\begin{array}{l}
% 		\theta = \scriptsize \arcsec \left[ \sqrt{\frac{\sqrt{(\eta -1) (2 (\eta -1) \eta  (n-3)-1) \left(\eta  \left(\frac{\eta ^2 n}{8} \left(n^2-12 n+12\right)+2 \eta  (n-3) n+4 n-3\right)+1\right)}+\eta  \left(-6 (\eta -2) \eta +2n (\eta -1)^2 -7\right)+1}{(\eta -1) (2 (\eta -1) \eta  (n-3)-1)}} \right]
% 	\end{array}
% \end{equation}

% \begin{equation}
% 	\begin{split}
% 		\Delta \phi =& \frac{\csc (2 \theta)}{\sqrt{2} \eta ^2 n (n+2)}                                                                                                                                               \\
% 		&\Bigg[\eta \, n \bigg(\frac{\eta}{2}  (n+2) \textcolor{blue}{\bigg(}\frac{\eta ^2}{4} \left(n^2+2 n-24\right) \cos (4 \theta)+4 \textcolor{red}{\Big(} 6 (\eta -2) \eta +\frac{\eta \, n}{4} (\eta  (n-14)+16)+4 \textcolor{red}{\Big)} \cos (2 \theta) \textcolor{blue}{\bigg)} \\
% 			&+\frac{\eta ^3}{8} \left(3 n^3-52 n^2+132 n-144\right)+8 \eta ^2 (n-3) (n-2)+8 \eta \, (3 n-5)+8\bigg) \Bigg]^{1/2}
% 	\end{split}
% \end{equation}
Injecting \eqref{eq:d2_eta_tf} and \eqref{eq:d4_eta_tf} into \cref{eq:def_delta_phi}, we get:

\begin{equation}
	\begin{cases}
		Q^{\mathrm{tf}}_0(N,\eta) & = -144 \eta ^3+384 \eta ^2-320 \eta +3 \eta ^3 N^3-52 \eta ^3 N^2+64 \eta ^2 N^2+132 \eta ^3 N-320 \eta ^2 N+192 \eta  N+64 \\
		Q^{\mathrm{tf}}_1(N,\eta) & = -4 \eta \, (N+2) \left(\eta ^2 \left(N^2-14 N+24\right)+16 \eta \, (N-3)+16\right)                                        \\
		Q^{\mathrm{tf}}_2(N,\eta) & = \eta ^3 \left(N^3+4 N^2-20 N-48\right)
	\end{cases}
\end{equation}
\begin{equation}
	\begin{split}
		\Delta \phi = \frac{1}{4N(N+2) \, \eta^2 \, \left|\sin(2 \phi)\right|} \sqrt{\eta \, N \left[Q^{\mathrm{tf}}_0(N,\eta) + Q^{\mathrm{tf}}_1(N,\eta) \cos(2\phi) + Q^{\mathrm{tf}}_2(N,\eta) \cos(4\phi)\right]}
	\end{split}
\end{equation}

\subsubsection*{With two-mode squeezed vacuum states}

Injecting \eqref{eq:d2_eta_tms} and \eqref{eq:d4_eta_tms} into \cref{eq:def_delta_phi}, we get:
\begin{equation}
	\begin{cases}
		Q^{\mathrm{tms}}_0(N,\eta) & = 3 \eta +3 \eta ^3 N^3+4 \eta ^3 N^2+8 \eta ^2 N^2+\eta ^3 N+6 \eta ^2 N+7 \eta  N+1 \\
		Q^{\mathrm{tms}}_1(N,\eta) & = -2 \eta \, (N+2) \left(2 \eta ^2 N^2+4 \eta  N+1\right)                             \\
		Q^{\mathrm{tms}}_2(N,\eta) & = \eta ^3 N (N+2)^2
	\end{cases}
\end{equation}
\begin{equation}
	\begin{split}
		\Delta \phi = \frac{1}{N(N+2) \, \eta^2 \, \left|\sin(2 \phi)\right|} \sqrt{\eta \, N \left[Q^{\mathrm{tms}}_0(N,\eta) + Q^{\mathrm{tms}}_1(N,\eta) \cos(2\phi) + Q^{\mathrm{tms}}_2(N,\eta) \cos(4\phi)\right]}
	\end{split}
\end{equation}

\section{Optimal phase $\phi_0$}

When considering non-unit quantum efficiency, the phase uncertainty exhibits a minimum in $\phi_0 > 0$. In that case,
the study of the derivative of $\Delta \phi$ as a function of $\phi$ gives the analytic expression of $\phi_0$.

\subsubsection*{With twin Fock states}

\begin{equation}
	\begin{split}
		\phi_0 = \arccsc \Bigg[ \bigg(\eta ^3 N (N^2-12 N+12)-2 \sqrt{2} \Big(\\
			&(\eta -1) (-16 \eta ^2 (5 N^2-19 N+12) \\
			&+2 \eta ^5 N (N^3-15 N^2+48 N-36) \\
			&-2 \eta ^4 N (N^3-31 N^2+144 N-180) \\
			&+\eta ^3 (-33 N^3+268 N^2-540 N+144) \\
			&+\eta  (72-48 N)-8) \\
			&\Big)^{1/2}\\
			+16 \eta ^2 (N-3) N+8 \eta  (4 N-3)+8\bigg)^{1/2} \Bigg]
	\end{split}
\end{equation}

\section{Comparison of TF and TMS state in the presence of losses}

In the main paper, we noticed that when considering unit quantum efficiency, the TMS
outperforms the TF by a factor of $\sqrt{2}$, in the neighbourhood of the optimal phase $\phi_0 =0$
(see eq. (10) of the main paper).

However, it is questionable to conclude that this implies that in such an idealized context
(where the losses are zero) the TMS state is superior to the TF state for performing a quantum interferometry experiment.
Indeed, examination of Fig. 2 reveals that the phase neighbourhood around which the TMS exhibits a better behaviour is
very narrow. Overall, even in the absence of losses, the phase domain where sub-shot-noise interferometry can be observed is much
larger using TF states than using TMS states. We posit that this $\sqrt{2}$ factor advantage of the TMS state at $\phi_0=0$
is more likely a result of mathematical accident than a predictable outcome based on physical argument.

When losses are introduced, the optimal phase resolution continuously evolves towards a situation where TF states perform better,
regardless of the phase. This implies that, when focusing on the optimal phase resolution as a function of the number of particles,
one can find a crossing point between TF and TMS: numerically this crossing point exists for ${0.946 \lesssim \eta < 1}$
as it is shown in \cref{fig:delta_phi_phi0} below.

% Figure environment removed


\end{document}


\subsubsection*{With twin Fock states}

% {\eta  (N+2) \left(\eta ^2 \left(N^2-14 N+24\right)+16 \eta  (N-3)+16\right)}
%{(\eta -1) \left(\eta ^2 (4 n-6)+\eta  (6-4 n)-1\right)}

% \csc ^{-1}\left(\sqrt{\frac{\sqrt{(\eta -1) \left(-4 \eta ^2 \left(10 n^2-19 n+6\right)+2 \eta ^5 n \left(2 n^3-15 n^2+24 n-9\right)+2 \eta ^4 n \left(-2 n^3+31 n^2-72 n+45\right)+\eta ^3 \left(-33 n^3+134 n^2-135 n+18\right)+\eta  (9-12 n)-1\right)}+\eta ^3 (4 n-6)+\eta ^2 (12-8 n)+\eta  (4 n-7)+1}{(\eta -1) \left(\eta ^2 (4 n-6)+\eta  (6-4 n)-1\right)}}\right)