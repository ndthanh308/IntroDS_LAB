%% ****** Start of file apsguide4-2.tex ****** %
%%
%%   This file is part of the APS files in the REVTeX 4.2 distribution.
%%   Version 4.2b of REVTeX, December 2018.
%%
%%   Copyright (c) 2019 The American Physical Society.
%%
%%   See the REVTeX 4.2 README file for restrictions and more information.
%%
\documentclass[twocolumn,secnumarabic,amssymb, nobibnotes, aps, prd]{revtex4-2}
\RequirePackage[english]{babel}
\RequirePackage[utf8]{inputenc}
\RequirePackage{hyperref}
\hypersetup{colorlinks=true, citecolor=blue, urlcolor=blue, linkcolor=blue}

\RequirePackage[]{amsmath,amssymb}

\newcommand{\revtex}{REV\TeX\ }
\newcommand{\classoption}[1]{\texttt{#1}}
\newcommand{\macro}[1]{\texttt{\textbackslash#1}}
\newcommand{\m}[1]{\macro{#1}}
\newcommand{\env}[1]{\texttt{#1}}
\setlength{\textheight}{9.5in}

\RequirePackage[]{graphicx}

\RequirePackage{physics}
\RequirePackage{siunitx}
\sisetup{%
	exponent-product = \times,
	output-decimal-marker = {.},
	range-units = single,
	list-units = single,
	group-minimum-digits = 4,
	group-digits = integer,
	group-separator = {\,},
	per-mode = symbol,
	separate-uncertainty = true,
	detect-all = true}
% \DeclareSIUnit\Erec{\ensuremath{\textit{E}_\text{rec}}}

\RequirePackage{cleveref}

\usepackage[dvipsnames]{xcolor}
\newcommand{\red}[1]{\textcolor{red}{\textbf{#1}}}
\newcommand{\green}[1]{\textcolor{ForestGreen}{\textbf{#1}}}

%added by arXiv:
\usepackage{lmodern}

\begin{document}

\title{Sub-shot-noise interferometry with two mode quantum states}
% Sub
%\date{\today}

\author{Quentin Marolleau}
% \email[contact: ]{\texttt{name}.\texttt{surname}@institutoptique.fr}
\author{Charlie Leprince}
\author{Victor Gondret}%
\author{Denis Boiron}%
\author{Christoph I. Westbrook}
\affiliation{Université Paris-Saclay, Institut d’Optique Graduate School, CNRS, Laboratoire Charles Fabry, 91127, Palaiseau, France}
\date{\today}%

\begin{abstract}
    We study the feasibility of sub-shot-noise interferometry with imperfect detectors, starting from twin-Fock
    states and two mode squeezed vacuum states. We derive analytical expressions for the corresponding phase uncertainty.
    We find that one can achieve phase shift measurements below the standard quantum limit, as long as the losses are
    smaller than a given threshold, and that the measured phase is close enough to an optimal value. We provide our
    analytical formulae in a Python package, accessible online.
\end{abstract}

\maketitle

\section{Introduction}

The ability to map many physical quantities onto a phase shift makes interferometry both a crucial and generic technique
in metrology. It is widely known that because of entanglement, some non-classical states can lead to improved phase
resolution compared to their classical counterparts \cite{giovannettiQuantumEnhancedMeasurementsBeating2004,giovannettiQuantumMetrology2006,pezzeEntanglementNonlinearDynamics2009,pezzeQuantumMetrologyNonclassical2018}.
Given an experimental resource of $N$ identical bosons, an attractive choice is to use NOON states
$\frac{1}{\sqrt{2}} \left( \ket{N,0} + \ket{0,N} \right)$.
% which are two-mode maximally entangled states of $N$ particles.
Indeed  NOON states lead to a ``Heisenberg limited" phase uncertainty {$\Delta \phi = \mathcal{O}(N^{-1})$}
\cite{bollingerOptimalFrequencyMeasurements1996a,dowlingCorrelatedInputportMatterwave1998,giovannettiAdvancesQuantumMetrology2011},
known to be optimal \cite{heitlerQuantumTheoryRadiation1954,ouFundamentalQuantumLimit1997}.
This is a much more advantageous scaling than the best phase sensitivity reachable with classical systems ($\Delta \phi = 1/\sqrt{N}$),
provided by a coherent state, and usually called the \emph{standard quantum limit} (SQL) or \emph{shot noise}.
Other authors have proposed the use of ``twin Fock" (TF) states $\ket{\mathrm{TF}}=\ket{N/2,N/2}$ and have shown that
they also can achieve $1/N$ scaling in phase sensitivity \cite{hollandInterferometricDetectionOptical1993,bouyerHeisenberglimitedSpectroscopyDegenerate1997}.

Unfortunately, NOON states are extremely fragile and behave even worse than classical states when losses are present \cite{dunninghamInterferometryStandardQuantum2002,dornerOptimalQuantumPhase2009}.
In addition, they are very challenging to prepare, and
their realization with $N$ larger than a few units has not been achieved \cite{nagataBeatingStandardQuantum2007,afekHighNOONStatesMixing2010}.

The effect of loss in quantum enhanced interferometers has been studied more generally, and states minimizing the phase uncertainty in the presence of loss have been found
\cite{demkowicz-dobrzanskiQuantumPhaseEstimation2009,kolodynskiPhaseEstimationPriori2010,knyshScalingLawsPrecision2011}.
These states can be expressed as superpositions of  states of the form:
\begin{equation}
    \ket{N :: m}_\pm = \frac{1}{\sqrt{2}} \left( \ket{N-m,m} \pm \ket{m,N-m} \right).
\end{equation}
Like NOON states, these states involve superpositions of strong population imbalances between the two modes (a NOON state is in fact the special case $m=0$).
This imbalance is responsible for the enhanced sensitivity, but these states can retain their coherence despite a loss of order $m$ particles, and thus are more robust \cite{huverEntangledFockStates2008}.
In the presence of losses however, even these states can only surpass the standard quantum limit by a numerical factor, meaning that
$\Delta \phi = \mathcal{O}(N^{-1/2})$ is the best scaling possible \cite{kolodynskiPhaseEstimationPriori2010,escherGeneralFrameworkEstimating2011}.
Here again, although the optimal states are conceptually interesting, their experimental realization is not presently realistic.

On the other hand, it is well known that the mixing on a beam splitter of the twin-Fock states mentioned above gives
rise to a superposition of $\ket{2n::2k}_\pm$ states \cite{camposQuantummechanicalLosslessBeam1989,yuspasibkoInterferenceMacroscopicBeams2014},
and one might wonder about the robustness of such a superposition in the presence of loss.
A related state is the two-mode squeezed state (TMS) \cite{anisimovQuantumMetrologyTwoMode2010}, which is a
superposition of twin Fock states with different particle numbers. Both of these states are widely used and can be
produced with a large number of particles \cite{luoDeterministicEntanglementGeneration2017,deng2023heisenberglimited,andersMomentumEntanglementAtom2021,bookjansStrongQuantumSpin2011,shiskhakovMacroscopicHongOu2013,harderSingleModeParametricDownConversionStates2016}.
These states are different from another type of experimentally realizable states, the ``spin squeezed'' states, see
fig. 5 of~\cite{pezzeQuantumMetrologyNonclassical2018}.

Here we will analyse the performance of twin Fock and two-mode squeezed states in the presence of loss.
Unlike for spin squeezed states, the relevant observable is not simply the population difference and in fact several
choices are a priori possible. We will follow other authors in using the variance of the population difference as the
interferometric observable \cite{hollandInterferometricDetectionOptical1993,bouyerHeisenberglimitedSpectroscopyDegenerate1997}.
We find that the sensitivity in this case only differs from that of the optimal state by a numerical factor and that one
can surpass the standard quantum limit if the losses are low enough.


\section{Our model}

We will consider the interferometer configuration represented in \cref{fig:schematic_MZI},
and for the sake of clarity
we will distinguish the \emph{input} state (before the first beam splitter) that one must prepare, from the
\emph{probe} state (after the first beam splitter) that exhibits some phase sensitivity.
% Without loss of generality, we can assume that the phase shifting is only carried out by the $\hat{\Phi}$ operator.
We assume that the losses are only caused by the detectors, having the same quantum efficiency $\eta$, and we will consider ${\Delta \phi = 1 / \sqrt{\eta N}}$
to be the SQL, against which we should compare our results.
Our input state is either a twin Fock state or a two-mode squeezed state. Note that without an initial beam splitter, these states produce interference patterns that are independent of the phase  \cite{hongMeasurementSubpicosecondTime1987,
    camposQuantummechanicalLosslessBeam1989,yuspasibkoInterferenceMacroscopicBeams2014}.

% Figure environment removed

% these Therefore we add an initial beam splitter to create a state (the probe) which is sensitive to phase shifts.
%even though starting from TF or TMS state which are known to produce interference patterns that are independent on the phase
%\cite{hongMeasurementSubpicosecondTime1987,camposQuantummechanicalLosslessBeam1989,yuspasibkoInterferenceMacroscopicBeams2014}.

%After interfering at the first beam splitter, a $N$ particles TF state generates the following probe state \cite{camposQuantummechanicalLosslessBeam1989}:
%\begin{equation}
%    \ket{\psi_p} =\sum\limits_{k=0}^{N/2} \frac{(-1)^{\frac N2 -k}}{2^{N/2}} \sqrt{{2k \choose k}{N-2k \choose \frac N2-k}} \ket{2k,N-2k}
%\end{equation}
%It is a coherent superposition of ${\ket{2k::N-2k}_\pm = \frac{1}{\sqrt{2}} \left( \ket{2k,N-2k} \pm \ket{N-2k,2k} \right)}$
%weighed with the discrete arcsine law. An interesting point is the fact that $\ket{2k::N-2k}$ states have a better behaviour
%to losses than pure NOON states \cite{huverEntangledFockStates2008}, at the cost of a reduced phase sensitivity for a fixed
%number of particles $N$. Nonetheless, we can expect that this robustness to losses is transmitted to $\ket{\psi_p}$.

Two-mode squeezed states are well known in quantum optics, as they are spontaneously generated from vacuum with a
quadratic interaction hamiltonian.
By denoting ${\xi = r e^{i \theta}}$ the squeezing parameter, whose norm $r$ is
proportional to the interaction time, such a state reads
\begin{equation}
    \ket{\mathrm{TMS}} = \frac{1}{\cosh(r)} \sum\limits_{n=0}^\infty e^{i n \theta} \tanh^n(r) \ket{n,n}
\end{equation}
in the Fock basis relative to the modes $\hat{a}_1$ and $\hat{a}_2$ (see \cref{fig:schematic_MZI}).
%Such a state consists of a coherent superposition of twin Fock states  states.
%The preparation of pure TF states is less usual than TMS ones, but nevertheless has been achieved in the context of
%atom interferometry \cite{luckeTwinMatterWaves2011}.

The action of the interferometer on the input state is described by the unitary operator $\hat{U}$:
\begin{equation}
    \hat{U} = \hat{S} \begin{pmatrix}
        e^{i\phi} & 0 \\ 0 & 1
    \end{pmatrix}     \hat{S} = e^{i \frac{\phi}{2}}
    \begin{bmatrix}
        i \sin(\frac{\phi}{2}) & \cos(\frac{\phi}{2})    \\
        -\cos(\frac{\phi}{2})  & -i \sin(\frac{\phi}{2})
    \end{bmatrix}
\end{equation}
The losses are modelled by additional beam splitters $\hat{S}_\eta$ placed at the output ports:
\begin{equation}
    \hat{S}_\eta =
    \begin{bmatrix}
        \sqrt{\eta}     & \sqrt{1-\eta} \\
        - \sqrt{1-\eta} & \sqrt{\eta}
    \end{bmatrix}
\end{equation}
With the input and output annihilation operators defined in \cref{fig:schematic_MZI}, we introduce the additional
notations for the number operators:
\begin{equation}
    \hat{N}_{\alpha_i} = \hat{\alpha}^\dagger_i \hat{\alpha}_i
    \quad / \quad \alpha \in \{a,b,c \} \ , \ i \in\{ 1,2 \}
\end{equation}
% the input spin operators \cite{camposQuantummechanicalLosslessBeam1989}:
% \begin{equation}
%     \begin{cases}
%         \hat{J}_x =\frac 12 \left( \hat{a}^\dagger_1 \hat{a}_2  + \hat{a}_2^\dagger \hat{a}_1\right)    \\
%         \hat{J}_y =\frac 1{2i} \left( \hat{a}^\dagger_1 \hat{a}_2  - \hat{a}_2^\dagger \hat{a}_1\right) \\
%         \hat{J}_z =\frac 12 \left( \hat{N}_{a_1} - \hat{N}_{a_2} \right)
%     \end{cases}
% \end{equation}
and the detected particle number difference at the output, with and without losses:
\begin{equation}
    \left\{
    \begin{aligned}
        \hat{D}_\eta & = \frac{1}{2} \left( \hat{N}_{c_2} - \hat{N}_{c_1} \right)                      \\
        \hat{D}      & = \frac{1}{2} \left( \hat{N}_{b_2} - \hat{N}_{b_1} \right) = \hat{D}_{\eta = 1}
    \end{aligned}
    \right.
\end{equation}
% and its equivalent after application of the losses:
% \begin{equation}
% \end{equation}
We also denote ${N=\ev{\hat{N}_{a_1}+\hat{N}_{a_2}}=\ev{\hat{N}_{b_1}+\hat{N}_{b_2}}}$ the average number of particles in the initial state.
In the case of a twin Fock state $\ket{n,n}$, we simply have $N=2n$, whereas for a two-mode squeezed state $N = 2 \sinh^2(r)$,
i.e. twice the average number of particles per mode.
The mean number of detected particles therefore is $\eta N$.

The operator $\hat{U}$ provides the expansion of $\hat{D}$ in terms of the input modes:
\begin{equation}
    \hat{D} =\frac 12 \left[\cos(\phi) \, \left( \hat{N}_{a_1} - \hat{N}_{a_2} \right) +i \sin(\phi) \, \left( \hat{a}^\dagger_1 \hat{a}_2  - \hat{a}_2^\dagger \hat{a}_1\right) \right]
\end{equation}
Therefore, whatever the phase $\phi$ and the quantum efficiency $\eta$, a twin Fock state placed at the input of the
interferometer yields a vanishing expectation value for $\hat{D}$.
%\begin{equation}
%    \ev{\hat{D}} = 0
%\end{equation}
Due to linearity, the same is true for two-mode squeezed states.
This means that $\hat{D}$ itself is not a suitable
observable to recover information about the phase $\phi$ with those states.
However, following previous authors \cite{hollandInterferometricDetectionOptical1993,bouyerHeisenberglimitedSpectroscopyDegenerate1997}, one can study  $\hat{D}^2$ which characterizes the width of these distributions.
%the \emph{variance} of $\hat{D}$. 
Indeed, one can derive \cite{supplemental}:
%
\begin{equation}
    \left\{
    \begin{aligned}
        \ev{\hat{D}_\eta^2}_{\mathrm{tf}}  & = \eta^2 \frac N4 \left(1+\frac N2\right)\sin^2(\phi) + \eta (1-\eta) \frac N4 \\
        \ev{\hat{D}_\eta^2}_{\mathrm{tms}} & = \eta^2 \frac N2 \left(1+\frac N2\right)\sin^2(\phi) + \eta (1-\eta) \frac N4
    \end{aligned}
    \right.
\end{equation}
%
making explicit the phase dependence.
\section{Results}
The phase uncertainty can be computed analytically using
\begin{equation}
    \Delta \phi =\frac{\sqrt{\mathrm{Var}\left[\hat{D}_\eta^2\right]}}{\left|\dfrac{\partial}{\partial \phi}\left[ \ev{\hat{D}_\eta^2} \right] \right|}.
    \label{eq:phase_uncertainty}
\end{equation}
If the detectors are lossless ($\eta =1$), the phase uncertainties are given by:
\begin{equation}
    \begin{cases}
        \Delta \phi_{\mathrm{tf}} = \dfrac{1}{\cos (\phi ) \sqrt{N (N+2)}} \sqrt{2 + \left(-3 + \frac N4 + \frac{N^2}{8}\right) \sin ^2(\phi )} \\
        \Delta \phi_{\mathrm{tms}} = \dfrac{1}{\cos (\phi ) \sqrt{N (N+2)}} \sqrt{1 + 2 N (N+2) \sin ^2(\phi )}
    \end{cases}
\end{equation}
In the neighbourhood of ${\phi = 0}$, we find Heisenberg limited scaling ${\Delta \phi = \mathcal{O}(N^{-1})}$.
% Figure environment removed
% Figure environment removed

When we include losses, the analogous expressions become rather long and we leave them to the supplemental materials \cite{supplemental}.
%Nonetheless, the corresponding expressions exhibits a scaling ${\Delta \phi = \mathcal{O}(N^{-1/2})}$, which is expected as long as losses are present \cite{kolodynskiPhaseEstimationPriori2010}.
As an example, in \cref{fig:phase_uncertainty} we show that the phase uncertainty $\Delta \phi$ can be smaller than the
standard quantum limit. In addition, the phase uncertainty has a minimum at a non-zero phase $\phi_0$ which depends on the detection
efficiency, number of particles and the input state (see \cref{fig:optimal_phi}). The optimal phase is shifted due to a
divergence at zero phase in \cref{eq:phase_uncertainty}: ${\Delta \phi \underset{\phi = 0}{=} \mathcal{O}(\phi^{-1})}$.
% When $\eta$ is not too small, we find that $\Delta \phi$ can be smaller than the standard quantum limit.
This type of profile has been observed experimentally \cite{luckeTwinMatterWaves2011}.
From the study of the variations of $\Delta \phi$ as a function of $\phi$, one can compute the optimal phase
$\phi_0$ around which an experiment should operate to perform precision measurements.
This means that during an experiment, one must be able to tune a phase offset, for instance in optics by adding a tiltable glass plate.

In \cref{fig:optimal_phi} we show color maps of the values of $\phi_0$ as a function of the number of particles $N$ and
the quantum efficiency $\eta$. Regions where sub-shot-noise measurements are possible correspond to non-hashed regions.
It appears in these maps that this question is mostly related to the quantum efficiency of the detectors: depending on
whether one is dealing with twin Fock or two-mode squeezed states, a threshold of $\eta \approx 0.7$ or respectively
$\eta \approx 0.9$ must be achieved to surpass the SQL.

% Figure environment removed


We have computed $\Delta\phi_0$ \cite{supplemental}, the phase uncertainty when the measurement is performed at
the optimal phase $\phi_0$.
When $N$ is small, $\Delta\phi_0$ varies similarly to a
power law ${\Delta\phi_0 \approx 1 / N^\alpha}$ with ${0.5 < \alpha < 1}$, depending on the value of $\eta$ (an example is plotted in \cref{fig:delta_phi_phi0}). Experimentally,
in this region one obtains significant gains with respect to the standard quantum limit by increasing the number of particles.
In the asymptotic
region, where $N$ goes to infinity, we recover the ${\Delta\phi_0 = \mathcal{O}(N^{-1/2})}$ scaling \cite{kolodynskiPhaseEstimationPriori2010,escherGeneralFrameworkEstimating2011}.

We also computed
\begin{equation}
    \gamma(\eta) = \lim\limits_{N\rightarrow \infty}\sqrt{\eta N} \, \Delta\phi_0
    \label{eq:def_gamma}
\end{equation}
which is the ratio between $\Delta \phi_0$ and the standard quantum limit, in the asymptotic limit.
This quantity tells what value of $\eta$ must be reached to go below the SQL. It has been proven \cite{kolodynskiPhaseEstimationPriori2010}
that
\begin{equation}
    \sqrt{\eta N} \, \Delta\phi \geq \sqrt{1-\eta}
    \label{eq:lower_bound_gamma}
\end{equation}
but this bound is tight only when using optimal input states, as well as an observable which is not explicitly known.
In our case, the function $\gamma$ is actually a simple dilation of the lower bound (\ref{eq:lower_bound_gamma}) (see \cref{fig:res_limit}):
\begin{equation}
    \left\{
    \begin{aligned}
        \gamma^{\mathrm{tf}}(\eta)  & = \sqrt{3} \, \sqrt{1-\eta}                                                                                                                                           \\
        \gamma^{\mathrm{tms}}(\eta) & = \vphantom{\left(\frac{2}{5}\right)^{1/4}} \smash{\underbrace{\left(\frac{2}{5}\right)^{1/4} \sqrt{5 + 2 \sqrt{10}}}_{\displaystyle \approx 2.676}} \, \sqrt{1-\eta}
    \end{aligned}
    \right.
    \vspace*{3ex}
\end{equation}
Our simple measurement protocol is therefore similar to an optimal situation where ideal states are used.
% Figure environment removed

\section{Conclusion}

The chief conclusion of this work is that when accounting for non-unit quantum efficiency, there exist experimentally
accessible states which can achieve phase sensitivity close to the theoretical limit.  As in other schemes to surpass
the standard quantum limit, the quantum efficiency of the detectors is critical.
With TF states, a 95\% quantum efficiency results in a \SI{8}{\decibel} improvement compared to the SQL, which is not very
far from the theoretical bound of \SI{13}{\decibel} given by \cref{eq:lower_bound_gamma}. For a TMS the gain is only
\SI{4.4}{\decibel}.
Still, we expect that such improvement factors could be useful in some interferometers where
increasing the number of particles to reduce the shot noise is not practical.  Whether twin Fock or two-mode squeezed states
constitute a real advantage compared to spin squeezing will require more work in the future using comparisons
in realistic experimental situations \cite{pezzeQuantumMetrologyNonclassical2018}.
The fact that these relatively accessible states are not far from the optimized ones is an encouraging sign.



% We have shown in this paper that twin-Fock states and two-mode squeezed vacuum states may offer a phase sensitivity
% with optimal scaling in an interferometer, even when detection losses are taken into account.
% We examined in particular
% the expected phase uncertainty when considering the observable corresponding to the variance of the output number difference
% $\hat{D}_\eta^2 = \frac{1}{4}\left[\hat{N}_{c_2}-\hat{N}_{c_1}\right]^2$. The measurement of this quantity provides a phase
% uncertainty below the SQL, as long as the losses are not to large, and that phase difference $\phi$ to be measured is
% close enough to an optimal value $\phi_0$ (that we have computed).

% The main advantage of this technique is to offer a fairly simple experimental procedure to achieve sub-shot-noise interferometry
% in an actual experiment.

Our analytical formulae are provided in the supplementary materials and are implemented in a Python package, accessible online
at \url{https://github.com/quentinmarolleau/qsipy}.


\begin{acknowledgments}
    The research leading to these results has received funding from QuantERA Grant No. ANR-22-QUA2-0008-01 (MENTA) and
    ANR Grant No. 20-CE-47-0001-01 (COSQUA), the LabEx PALM, Région Ile-de-France in the framework of the DIM SIRTEQ program
    and Quantum-Saclay.
\end{acknowledgments}

\bibliography{bibliography}

\end{document}

