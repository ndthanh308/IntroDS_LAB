\chapter*{Supplementary Material: Attribute Regularized Soft Introspective VAE: Towards Cardiac Attributes Regularization Through MRI Domains}

\section{Implementation details}
\label{app:training}

We developed our deep learning framework using Python 3.8. All the methods were implemented in PyTorch. We ran the experiments on NVIDIA RTX A6000. 

$\beta$-\textbf{VAE} \cite{higgins:2017}: We followed the following public implementation\footnote{\url{https://github.com/1Konny/Beta-VAE/}}. We used 5 layers (32,32,32,32,256) with a bottleneck of 32 dimensions. We trained for up to 3000 epochs (early stopping after 500 epochs) and used ADAM optimizer with learning rates of $5\mathrm{e}{-5}$ and batch size of 16 for all experiments. The hyperparameters were for $\beta-VAE$: $\beta = 2$ and for Attri-VAE: $\beta$ = 2, $\gamma$ = 10 and $\delta = 10$ for ACDC and M$\&$Ms.

\textbf{\gls{sivae}} \cite{Daniel:2021}: We followed the official public implementation\footnote{\url{https://taldatech.github.io/soft-intro-vae-web/}}. We used 5 layers (64,128,256,512,512) with a bottleneck of 256 filters. We trained for up to 750 epochs (early stopping after 300 epochs) and we used ADAM optimizer with learning rates of $2\mathrm{e}{-4}$ and batch size of 16 for all of the experiments. We used a combination of the MSE loss and the perceptual loss weighted by the parameters $\delta$ as reconstruction loss. The hyperparameters were chosen for the different datasets are reported in Table. \ref{table:parameters}

\begin{table}[h]
    \centering
    
\resizebox{0.8\textwidth}{!}{\begin{tabular}{ c | c | c | c}
\hline
\textbf{ACDC} &  SIVAE &  \makecell{$\beta_{KL}$ = 1 \\ $\beta_{rec}$ = 0.8} &
                   \makecell{$\delta$ = 50 \\ $\gamma_{reg}$ =  0} \\
\cline{2-2}
\cline{4-4}
&  Attri-SIVAE & $\beta_{neg}= 1024$ & \makecell{$\delta$ = 100 \\ $\gamma_{reg}$ =  0.1}\\ 
\hline
\textbf{M$\&$Ms} &  SIVAE &  $\delta$ = 100, $\beta_{KL}$ = 1 &   $\gamma_{reg}$ =  0 \\
\cline{2-2}
\cline{4-4}
&  Attri-SIVAE & $\beta_{rec}$ = 0.8, $\beta_{neg}= 512$ & $\gamma_{reg}$ =  0.1 \\
\hline

\end{tabular}}
\vspace{2mm}


    \caption{List of hyperparameters chosen for methods based on SIVAE}
    \label{table:parameters}
\end{table}

\section{Datasets}
\label{app:dataset}

We processed two short-axis cardiac MRI public datasets. The first one, ACDC dataset \cite{Bernard:2018}, contains 150 MRI from the same scanner with annotations at end-diastole and end-systole. We keep the splitting of the dataset as described in the challenge, i.e 100 cases for training and validation and 50 other cases for the test set. The second one is the dataset from the M\&Ms challenge \cite{campello:2021} where the images come from different centers and different vendors of MRI. In total, the dataset is composed of 345 cases in total (the data from the Canadian center was not considered\footnote{\url{https://www.ub.edu/mnms/}}) and we keep the same split described in the paper: 175 cases for training and 34 cases for validation from three different centers, called A,B and C, and 136 cases for testing from the three same centers and one unknown center, named D. Please refer to the corresponding papers for the acquisition protocol. As a preprocessing step, we center the barycenter of the \gls{lv} and the myocardium of the selected slice and align the barycenter of the \gls{rv} along the horizontal axis. We also put to zeros the regions which did not belong to the ground-truth mask. We crop each image of dimension $128\times128$ and rescale the intensity between 0 and 1.

\section{External dataset attributes dimension:}
\label{app:dimdim}


% Figure environment removed
