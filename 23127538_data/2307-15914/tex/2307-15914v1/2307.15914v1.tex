\documentclass[notitlepage, 11pt]{article}
\usepackage{hyperref, amsmath, amsfonts, enumitem, amsthm, xcolor, parskip}
\usepackage{amssymb} %% <- for \square and \blacksuare
\usepackage[nottoc]{tocbibind}
\usepackage{tikz-cd}
\setlength{\parindent}{3pt}
\usepackage{subfiles}

\hypersetup{             
    colorlinks=true,                
    breaklinks=true,                
    urlcolor= black,                
    linkcolor= blue,                     
    bookmarksopen=false,
    filecolor=black,
    citecolor=blue,
    linkbordercolor=blue
}
\setlist[itemize]{itemsep=0pt, parsep=0pt}
\setlist[enumerate]{itemsep=0pt, parsep=0pt}
\theoremstyle{definition}
\newtheorem{defn}{Definition}
\newtheorem{eg}{Example}[section]
\theoremstyle{plain}
\newtheorem{thm}[defn]{Theorem}
\newtheorem{q}{Question}[section]
\newtheorem{lem}{Lemma}[defn]
\newtheorem{prop}[defn]{Proposition}
\newtheorem{cor}{Corollary}[defn]
\theoremstyle{remark}
\newtheorem{notat}[defn]{{Notation}}
\newtheorem{rem}{Remark}[section]
\newcommand{\R}{\mathbb{R}}
\newcommand{\Q}{\mathbb{Q}}
\newcommand{\C}{\mathbb{C}}
\newcommand{\bH}{\mathbb{H}}
\newcommand{\N}{\mathbb{N}}
\newcommand{\Z}{\mathbb{Z}}
\newcommand{\F}{\mathbb{F}}
\newcommand{\bP}{\mathbb{P}}
\newcommand{\tr}{\mathrm{tr}}
\newcommand{\ip}{\mathfrak{p}}
\newcommand{\im}{\mathfrak{m}}
\newcommand{\ig}{\mathfrak{g}}
\newcommand{\ia}{\mathfrak{a}}
\newcommand{\ib}{\mathfrak{b}}
\newcommand{\iq}{\mathfrak{q}}
\newcommand{\iP}{\mathfrak{P}}
\newcommand{\disc}{\mathfrak{d}}
\newcommand{\diff}{\mathfrak{D}}
\newcommand{\iI}{\mathfrak{I}}
\newcommand{\cC}{\mathcal{C}}
\newcommand{\cD}{\mathcal{D}}
\newcommand{\cO}{\mathcal{O}}
\newcommand{\dd}[1][t]{\frac{d}{d{#1}}}
\newcommand{\pard}[1][z]{\frac{\partial}{\partial {#1}}}
\newcommand{\mf}[1]{\mathfrak{#1}}
\newcommand{\lieg}{\mf{g}}
\newcommand{\lh}{\mf{h}}
\newcommand{\lk}{\mf{k}}
\newcommand{\hm}[2]{\langle#1,#2\rangle}
\newcommand{\symp}[2]{\omega(#1,#2)}


\title{Fields With Finitely Many Non-Commutative Division Algebras Over It.}
\author{Snehinh Sen}
\setlist[enumerate, 1]{label={(\roman*)}}

\begin{document}

\maketitle

\begin{abstract}
We classify fields having finitely many finite non-commutative (not necessarily central) division algebras over it. In the process, we introduce the notion of \textit{anti-closure} of a field and make comments on fields having a linear lattice of finite field extensions over it. 
\end{abstract}

\section{Introduction}\label{s1}

Frobenius showed in \cite{fro78} that the only finite division algebras over $\R$, upto isomorphism, are $\R$, $\C$ and $\bH$, the Hamiltonian Quaternions. In fact, as we shall show later on, his result is easily generalised to arbitrary real closed fields. As a consequence, one realises that the only fields having finitely many isomorphism classes of finite division algebras are fields which are real closed or algebraically closed. However, this excludes pathological fields, like finite fields, over which any finite division algebra is actually a field extension. 

The main goal of this article is to classify all fields which have only finitely many finite non-commutative division algebras upto isomorphism over it. It turns our, interestingly, that real closed fields are the only non-pathological examples yet again. Here is the precise statement. 

\textbf{Theorem.}\emph{\label{t02} 
A field $K$ with only finitely and positively many noncommutative nonisomorphic finite division algebras over it is real closed. }

As for the structure of this article, Section \ref{s2} is a recap of the basic results from the theory of Brauer groups and division algebras. Section \ref{s3} is about the aforementioned generalisation of Frobenius Theorem. Sections \ref{s4} and \ref{s5} deal with the notion of \textit{anti-closure}, its properties and its connection to fields having a linear lattice of finite extensions over it. Sections \ref{s6} and \ref{s7} is devoted to the development of enough machinery and settling the claim made above. 


\section{Basic Properties of Division Algebras and Brauer Groups} \label{s2}

We briefly recall some basic properties of division algebras and Brauer groups. Suitable references for these are \cite{gui18}. \cite{ser79}, \cite{ser97} and \cite{stack23}.

\begin{defn}\label{d:Divalg}
Let $K$ be a field and $D$ be a finite division algebra over $K$.
\begin{enumerate}
\item $r_K(D)$ denotes the \textit{rank} of $D$ over $K$, that is $dim_K(D)$.
\item $Z(D)$ denotes the center of $D$.
\item It is known (see, for eg., \cite{stack23}) that $r_{Z(D)}(D)$ is a square of an integer. Its positive square root is called the \textit{degree} of $D$ and is denoted by $\deg(D)$.
\end{enumerate}
\end{defn}

\begin{defn}\label{d:Brauer} 
Let $K$ be a field and $L/K$ be a Galois field extension.  
\begin{enumerate}
\item $B(L/K)$ denotes the \textit{Relative Brauer Group} or, equivalently, the second cohomology group $H^2(G(L/K), L^*)$.
\item $B(K) = H^2(/K)$ denotes the \textit{Brauer Group} of the field $K$. 
\item $S(K)$ denotes the classes of all finite non-commutative division algebras over $K$ upto isomorphism.
\end{enumerate}
\end{defn}

Now we state a few standard results on division algebras and Brauer groups which will be needed in the sequel. For a proof, one may refer to \cite{ser79}.

\begin{prop}\label{p1} 
Let $K$ be a field, $L/K$ a finite Galois extension, $D$ a finite division algebra over $K$ and $\overline{K}$ an algebraic closure or $K$ containing $L$.
\begin{enumerate}
\item We have the inflation-restriction exact sequence $$0 \to B(L/K) \to B(M/K) \to B(M/L)$$ where $M/K$ is any Galois extension containing $L$. 
\item We have an exact sequence $$0 \to B(L/K) \to B(K) \to B(L)$$ where the first map is essentially inclusion and the second map is given by $A\mapsto A\otimes_K L$ upto Morita equivalence. 
\item Suppose $[D]\in B(K)$ maps to $0$ in the above sequence (in other words, $L$ is a \emph{splitting field }of $D$), then $\deg(D)|[L:K]$.
\item If $g = |G(L/K)|$, then $g.B(L/K)=0$.
\item $B(K)$ is a torsion group which is a union of all $B(L/K)$ as $L/K$ varies over all finite Galois extensions of $K$.
\item As sets, there is a natural bijection between $B(K)$ and the set of isomorphism classes of finite division algebras $D$ over $K$ such that $Z(D)=K$.
\item As sets there is a natural bijection between $S(K)$ and the disjoint union 
\[ \coprod_{L} \left(B(L)\setminus \{[0]\}\right)\]
\end{enumerate}
where $L$ runs over finite extensions of $K$ inside $\bar{K}$ which are not conjugates. 
\end{prop}

Consequently, we get the following results.

\begin{cor}\label{p1.1}
Let $K$ be a field. Then $K$ has only finitely many finite noncommutative division algebras upto isomorphism over it if and only if 
\begin{enumerate}
\item For each $L/K$ finite, $B(L)$ is finite, and
\item For all but finitely many such extensions, $B(L)=0$. 
\end{enumerate}
\end{cor}

\begin{proof}
This is immediate from parts (vi) and (vii) of Proposition \ref{p1}. 
\end{proof}

\begin{cor}\label{p3.1}
Let $K$ be a separably closed field. Then $B(K)=0$.
\end{cor}

\begin{proof}
As $K^{sep}=K$, $B(K)=B(K/K) = 0$.
\end{proof}

Now we recall some results from Galois cohomology which will be essential in our proofs. Once again, a suitable reference would be \cite{ser79} or \cite{ser97}.



\begin{prop}\label{p2} 
Let $K$ be a field and $L/K$ be a cyclic extension. Then $B(L/K) = K^*/N(L^*)$ where $N = N_{L/K}$ is the norm map.
\end{prop}

\begin{cor}\label{cp2.1}
Let $K$ be a perfect field with characteristic $p$ and let $L/K$ be a degree $p$ extension. Then the norm map $N_{L/K}$ is surjective. Consequently, if $M/K$ is cyclic and has degree a power of $p$, then $B(M/K)=0$. 
\end{cor}

\begin{proof}
The first part follows as $K^* = (K^*)^p \subseteq N_{L/K}(L^*)$. For the Galois case, let $K=K_0\subseteq K_1\ldots K_n = M$ denote a tower of extensions such that each $[K_i:K_{i-1}] = p$. Then each $N_{K_i/K_{i-1}}$ is surjective, proving the surjectivity of $N_{M/K}$. Rest follows from Proposition \ref{p2}.  
\end{proof}

\begin{cor}\label{cp2.2}
Let $L/K$ be a cyclic extension such that $N_{L/K}$ is surjective and $B(L)=0$. Then $B(K)=0$.
\end{cor}

\begin{proof}
The exact sequence of Propsotion \ref{p1}, part (ii) and Proposition \ref{p2}, yield that $0\to B(K)\to 0$ is exact, showing that $B(K)=0$.
\end{proof}



\section{Frobenius Theorem for Real Closed Fields} \label{s3}

We shall now prove a version of Frobenius Theorem for real closed fields and state a corollary which is akin to our main problem. For psychological reasons, let $R$ denote a real closed field. Let $C$ denote an algebraic closure of $R$. It is known that $C = R(i)$ where $i^2=-1$. Likewise, let $H$ denote the rank $4$ $R$-central division algebra defined as follows.

\begin{enumerate}
\item $H$ has an $R-$ basis denoted by $1,i,j,k$.
\item $R\to H$ given by $r\mapsto r1$ is the central embedding of $R$ in $H$.
\item $i^2=j^2=k^2=ijk = - ikj = -1$.
\end{enumerate}

Similar to the real case, one may verify that $H$ is a division algebra. In fact, the inverse of a non-zero $\alpha = a+bi+cj+dk$ is given by $\frac{1}{N(\alpha)}(a-bi-cj-dk)$ where $N(\alpha) = a^2+b^2+c^2+d^2>0$ as $R$ is real closed. 

\begin{thm}[Frobenius Theorem for Real Closed Fields]\label{t1} 
Suppose $R$ is a real closed field. Then the only finite division algebras, upto isomorphism, over $R$ are $R, C$ and $H$ constructed above. 
\end{thm}

\begin{proof}
Note that $B(R) = B(C/R) = R^*/N(C^*) \cong R^*/(R^{*})^2\cong C_2$. So the only two $R-$central division algebras upto isomorphism are $R$ and $H$. However, any other finite $R-$division algebra must be $C-$central. By Corollary \ref{p3.1}, the only possibility here is $C$ itself. So we are done. 
\end{proof}

\begin{cor}\label{ct1.1}
A field $K$ has only finitely many non-isomorphic finite division algebras over it if and only if $K$ is real closed or algebraically closed. 
\end{cor}

\begin{proof}
Firstly, as every finite field extension of $K$ is a division algebra over $K$, we should have that any such $K$ has only finitely many finite non-conjugate field extensions. By Artin-Schreier Theorem, this is possible only if $K$ is real closed or algebraically closed. Converse is Theorem \ref{t1} and Corollary \ref{p3.1}.
\end{proof}

This corollary is the prime motivation for our consideration - what happens if we just restrict our attention to non-commutative division algebras (ommutative division algebras, being field extensions, are somewhat pathological)? 

\section{Anti-Closure and M-Groups} \label{s4}

The notion of anti-closure of a field, though not standard, will pop up in our analysis. Firstly, we define what we mean by the anti-closure.

\begin{defn}\label{d:anticlosure}
Let $K$ be a field and $\overline{K}$ be an algebraic closure of $K$. The \textit{anti-closure} of $K$ in $\overline{K}$ is defined as 
\[ K' := \bigcap_{L} L\]where $L/K$ runs over all finite non-trivial extensions of $K$ in $\bar{K}$.
\end{defn}

The reason for this nomenclature is that the algebraic closure satisfies 
\[ \overline{K} = \bigcup_L L\]

where the union is over an almost similar collection. If $K$ is algebraically closed, we define $K'$ to be $K$. It must be noted that $K'/K$ is a non-trivial field extension if and only if $K$ has a unique minimal extension over it. We now state a few properties of fields such that $K'\neq K$. Henceforth, we fix the algebraic closure and consider finite subextensions only.

\begin{prop}\label{p4:ac}
    Let $K$ be a field such that $K'\neq K$. Then we have the following properties. 
    \begin{enumerate}
    \item $K'/K$ is finite and normal.
    \item If $L/K$ is a non-trivial extension of $K$, then $K'\subseteq L$.
    \item $K$ is either perfect or separably (but not algebraically) closed.
    \item $[K':K]$ is a prime number $p$.
    \item If $K$ is not perfect, then $char(K)=[K':K]$
    \item Let $p$ be the prime above. Then every extension of $K$ is a $p-$extension.
\end{enumerate}
\end{prop}

\begin{proof}
\begin{enumerate}
\item By definition, $K'/K$ is finite. Also, any Galois conjugate of $K'$ in $\overline{K}$ is, by definition, a minimal extension. By the definition of $K'$, we get $K'$ is contained in all of its Galois conjugates, hence is normal.
\item This follows from the definition.
\item Suppose $K$ is not perfect. Then $K$ has a non-trivial purely inseparable extension, say $L/K$. As $K'\subseteq L$, we get that $K'/K$ is purely inseparable. Hence, every non-trivial extension of $K$, by the virtue of it containing $K'$, is inseparable, showing that $K$ is separably closed. 
\item If $K$ is perfect, then $K'/K$ is Galois and $G(K'/K)$ is a finite group with no proper subgroups. Hence, $G(K'/K)\cong C_p$ for some prime $p$, proving our claim. If $K$ is not perfect, then $K$ has an extension $L=K(\alpha^{1/p})$ with $\alpha\in K\setminus K^p$, where $p=char(K)$. As $1<[K':K] | [L:K| = p$, we get $[K':K]=p$, as desired.
\item Immediate from the previous part.
\item If $K$ is separably closed, this is obvious. Otherwise, $K$ is perfect. Let $L/K$ be a finite non-trivial extension of $K$ and $N/K$ be its normal closure in $\overline{K}$. Let $p=[K':K]$ and $G=G(N/K)$. Let $G_p$ be a Sylow p-group of $G$ and $M$ its fixed field. Then $p \not | [M:K]$. Hence $M$ does not contain $K'$, implying that $M=K$ and $L/K$ is a $p$-extension. 
\end{enumerate}
\end{proof}

Perfect fields with $K'\neq K$ are very rare. Here are a few examples.

\begin{eg}\label{e1}
\begin{enumerate}
\item Let $K$ be a real closed field. Then $K'$ is simply an algebraic closure of $K$. Conversely, if $K'$ is algebraically closed, then $K$ is real closed or algebraically closed (by Artin-Schreier Theorem).
\item Let $K$ be a perfect field with absolute Galois group isomorphic to $\Z_p$ for some $p$ prime. For example, let $K$ denote the fixed field of the closed subgroup $\Z_p\leq \hat{\Z}$ for the Galois extension $\overline{\F}_q/\F_q$ where $q$ is any prime or $K=\C((T^{1/n} : (p,n)=1))$. Then $K'$ is the fixed field of $p\Z_p$ and is not equal to $K$.
\end{enumerate}
\end{eg}

As we shall see soon, these are in fact the only examples. Now fields with $K'\neq K$ are very much similar to real closed fields as we shall show next. 

\begin{prop}\label{p5:acrc}
    Let $K$ be a field such that $K'\neq K$. Let $p=[K':K]$
    \begin{enumerate}
        \item The irreducible polynomials in $K[X]$ have degree equal to a power of $p$.
        \item Every polynomial in $K[X]$ having degree coprime to $p$ has a root in $K$.
        \item $K$ contains all the $p-$th roots of unity.
        \item Every element in $K$ is a $p^{th}$ power of an element in $K'$ and $K'$ contains all the $p^{th}$ roots of elements of $K$.
        \item If $char(K)\neq p$ or $K$ is separably closed, then there is an element $\alpha \in K'$ such that $\alpha^p \in K$ and $K'=K(\alpha)$. So $K$ is not closed under taking $p^{th}$ roots in this case. 
    \end{enumerate}
\end{prop}

\begin{proof}
\begin{enumerate}
\item Let $f(X)$ be an irreducible polynomial and $L/K$ be a splitting field. Then $\deg(f)$ divides the order of $[L:K]$, which is a power of $p$ by Proposition \ref{p4:ac}. Hence $\deg(f)$ is a power of $p$.
\item As the degree is coprime to $p$, there is an irreducible factor $\mu$ with degree coprime to $p$. The first part would then imply that $\mu$ is linear.
\item The polynomial $\Phi_p(X) = X^{p-1}+\ldots 1$, having degree lesser than $p$, should split completely in $K$ by the first . Hence, $K$ contains all $p^{th}$ roots of unity. 
\item The splitting field of $X^p-a$ for each $a\in K$ is either $K$ or $K'$.
\item This follows from Kummer theory and field theory (cf. \cite{gui18}).
\end{enumerate}
\end{proof}

In analogy to this analysis, we define a concept in group theory which shall help us study such field extensions better. Such groups have been studied anonymously for a long time. We give them a name for an easier reference.

\begin{defn}\label{d:Mgroups}
An \textit{M-group} is a Hausdorff topological group $G$ with a unique closed maximal subgroup.
\end{defn}

Note that if $K$ is perfect and $G=G(/K)$, then $K'\neq K$ if and only if $G$ is an $M-$group. Furthermore, we have the following proposition.

\begin{prop}\label{p6} 
Every $M$-group $G$ is monothetic. 
\end{prop} 

\begin{proof}
Let $x$ be an element of $G$ which is not in its maximal subgroup. Then $\overline{\langle x\rangle}$ is a closed subgroup of $G$ which is not contained in the unique maximal subgroup. Hence, it must be the whole group $G$.
\end{proof}

\begin{cor}\label{cp6.1}
Every finite (resp. profinite) $M$-group is cyclic (procyclic). 
\end{cor}

The following result satisfactorily classifies the absolute Galois group of all perfect fields such that $K'\neq K$. 

\begin{thm}\label{t3} 
Let $K$ be perfect with $K'\neq K$ and $G=G(/K)$. Then $G$ is either isomorphic to $C_2$ or the additive group $\Z_p$ where $p=[K':K]$.
\end{thm}

\begin{proof}
Note that $G$ is finite if and only if $K$ is real closed and $G\cong C_2$. So assume $G$ is infinite. By Proposition \ref{p4:ac} and Corollart \ref{cp6.1}, $G$ is a pro-$p$ procyclic group. Let $S = \{\log_p([G:N]) : N\text{ is normal and of finite index in }G\}$. As $G$ is an infinite group, $S$ is cofinal in $\N$ and hence $G \cong  \hat{G} = \varprojlim G/N = \Z_p$. 
\end{proof}

\section{Fields with Linear Lattices} \label{s5}

In this section, we shall classify the structure and behaviour of fields satisfying a peculiar condition with respect to their lattice of extensions.

\begin{defn}\label{d:ll}
Let $K$ be a field. We say that $K$ has a \textit{linear lattice over it} or satisfies the linear lattice condition if there are field extension $K^{(n)}$, $n\geq 0$ (possibly up to a finite index) over $K$ such that
\begin{enumerate}
\item $K= K^{(0)}$.
\item $K^{(i)}\subseteq K^{(i+1)}$ for each $i\geq 0$.
\item For every $L/K$ finite, there is an $i\geq 0$ such that $L=K^{(i)}$.
\end{enumerate}
We would say that this is a \textit{linear lattice of distinct field extensions} over $K$, if $K^{(i)}\neq K^{(i+1)}$ for each $i$. 
\end{defn}

An independent theory of such fields can be developed. However, we would use the results of the previous section to classify such fields.

\begin{thm}\label{t4} 
Let $K$ be a perfect field which is not algebraically closed. Then the following are equivalent.
\begin{enumerate}
\item $K$ has a linear lattice over it.
\item $K'\neq K$.
\item $G(/K)$ is an $M$-group.
\item $G(/K)$ is isomorphic to either $C_2$ or $\Z_p$ for some prime $p$.
\end{enumerate}
\end{thm}

\begin{proof}
$\text{(ii)}\iff \text{(iii)}\iff \text{(iv)}$ was already established. Suppose $\text{(i)}$ is assumed. Then, as $K$ is not algebraically closed, there must be a least $i\geq 1$ such that $K^{(i)}\neq K$. We realise that $K'=K^{(i)}\neq K$, proving $\text{(ii)}$. Now, as the closed subgroups of $C_2$ and $\Z_p$ form a linear lattice, we immediately get $\text{(i)}$ by assuming $\text{(iv)}$ (Galois correspondence) thus proving our claim.
\end{proof}

We are prepared to make structural comments on fields with such lattices. The analysis would be split into two cases. Let us say that $[K':K]=p>1$.

\subsection{Characteristic of \texorpdfstring{$K$}{K} is \texorpdfstring{$p$}{p}}

We shall use Artin-Schreier Theory to derive the following claim.

\begin{thm}\label{t5}
    Let $K$ be a perfect field such that $[K':K]=char(K)=p$. Let $\mathcal{P}(T)=T^p-T$. There is an element $\alpha_0 \in K$ such that if we inductively define $\alpha_{i+1}$ to be a root of $\mathcal{P}(T)-\alpha_i$ for each $i$, then $K^{(i)}=K(\alpha_i)$ gives the linear lattice of distinct field extensions over $K$. In fact, $\alpha_0$ can be chosen to be any element of $K$ such that $\mathcal{P}(T)-\alpha_0$ is irreducible.
\end{thm}

\begin{proof}
    Existence of such an $\alpha_0$ is asserted by Artin-Schreier Theory. Suppose, for the sake of a contradiction, $i\geq 1$ is smallest such that $\beta = \alpha_{i+1} \in K(\alpha_i)=L$. Then let $\gamma =\alpha_i$. Now let $x \in L$. Let $F = K(\alpha_{i-1})$. By Artin-Schreier Theory, $i>1$. So $x = a_0+a_1\gamma+\ldots+a_{p-1}\gamma^{i-1}$ for some $a_i \in F$. Now, as $K$ is perfect, there are elements $b_i \in F$ such that $a_i = b_i^p$. Thus $y = b_0+b_1\beta+\ldots+b_{p-1}\beta^{i-1}$ satisfies $\mathcal{P}(T)-x$. Hence, by Artin-Schreier Theory, $L$ has no degree $p$ extension. As $G(/L)$ is pro-$p$, we get $L$ is algebraically closed, contradicting the Artin-Schreier Theorem. 
\end{proof}

\textbf{Remark.} As $K$ is perfect with a positive characteristic, the norm map of any extension of $K$ is surjective. By cyclicity, $B(K^{(i)}/K)=0$ for each $i\geq 0$ by Corollary \ref{cp2.1}. Taking limit on $i$, we obtain that $B(K)=0$.

\subsection{Characteristic of \texorpdfstring{$K$}{K} is not \texorpdfstring{$p$}{p}}

Here Kummer Theory would be applicable instead of Artin-Schreier Theory. First, we will deal with the norm. 

\begin{prop}\label{p7}
    Let $K$ be a field such that characteristic of $K$ is not equal to $p=[K':K]$. Then the norm map $N_{K'/K}$ is surjective with the only exception being the case when $K$ is real closed. In this case, $coker(N_{K'/K})\cong C_2$.
\end{prop}

\begin{proof}
    Such a $K$ is perfect by Propositon \ref{p4:ac}. For each $a\in K$, either $X^p - a$ is irreducible or it has a root in $K$. In either case, the root lies in $K'$ by the definition of $K'$. In the first case, the minimal and characteristic polynomial is the same as $X^p-a$. So the norm is $(-1)^{p-1}a$. In the latter case, the characteristic polynomial is $(X-b)^p$, where $b^p = a$. So the norm is $a$. If $p\neq 2$, then this would immediately imply that the norm is surjective. 
    
    Now let $p=2$. Firstly, suppose $-1$ is not a square in $K$. Here $K'=K(i)$ where $i^2 = -1$. For each $a \in K$, $\sqrt{a}=c+di$ for some $c,d\in K$. Thus $a = (c+di)^2$, which would imply that $a=c^2$ or $a=-d^2$. Hence, every element of $K$ is either a square or the negative of a square. If $(-1)$ is in the image of the norm, then we would get that the norm map is surjective. 
    
    Otherwise, this shows that $N(L^*)=(K^*)^2$. So $K^*/N(L^*) \cong C_2$. If $x=s^2, y=t^2$, where $s,t\in K$, then $s+it \in K(i)$ will have norm $s^2+t^2$. So $K$ is pythagorean. For $c+di\in K(i)$, $c,d\in K$, if we set $\alpha = \sqrt{\frac{c+\sqrt{c^2+d^2}}{2}} + i\sqrt{\frac{-c+\sqrt{c^2+d^2}}{2}} \in K(i)$, we get $\alpha^2 = c+di$. Hence, $K'$ has no quadratic extensions, proving that $K'$ is algebraically closed and thus $K$ is real closed.
    
    Finally, suppose $(-1)$ is a square in $K$. Then, $X^2-a$ is irreducible if and only if $X^2+a$ is irreducible. Thus, every element in $K$ would appear as a norm from the first paragraph.
\end{proof}

Similar to the previous case, we get $B(K) = 0$ with the only exception being $K$ real closed. We now comment on the lattice structure above $K$.

\begin{thm}\label{t6}
    Let $K$ be a perfect field with $char K \neq p = [K':K]$. Suppose $(-1)$ is a $p^{th}$ power in $K$. There is an element $\alpha \in K=K^{(0)}$ such that $K^{(i)} = K(\alpha^{p^{-i}})$ gives the linear lattice of distinct field extensions over $K$. In fact $\alpha$ can be chosen to be any element which is not a $p^{th}$ power in $K$.
\end{thm}

\begin{proof}
    Existence of such an $\alpha$ is due to Kummer Theory. Suppose, for the sake of contradiction, $i$ is the least index for which $\alpha$ chosen as above satisfies $\alpha^{p^{-i}} \in K^{(i-1)} = L $. Clearly $i> 1$. Set $F = K^{(i-2)}$, $\beta = \alpha^{p^{-(i-1)}}$ and $\gamma = \alpha^{p^{-i}}$. Then $L = F(\beta)= F(\gamma)$. Now let $N$ denote the norm map from $L$ to $F$. Then $N(\gamma) = a$ implies that $N(\beta) = a^p$. But the characteristic polynomial of $\beta$ is of the form $X^p - c$, where $c$ is not a $p^{th}$ power in $F$. But the norm is $a^p = (-1)^{p-1}c$, showing that $c$ is a $p^{th}$ power. This is a contradiction.
\end{proof}

This settles the case when $p$ is odd or when $p=2$ and $-1$ is a square in $K$. If $-1$ is not a square in $K$, there are two possibilities, as can be seen in the proof of Proposition \ref{p7}. Namely, $K$ is either real closed or $(-1)$ is in the image of the norm map $N_{K'/K}$. The first case has a simple lattice structure, with $K^{(1)}=K'$ algebraically closed. Conversely, according to Satz 5 of \cite{art27}, every algebraically closed field of characteristic $0$ contains a real closed subfield. For the second case, we look at the following theorem. 

\begin{thm}\label{t7} 
    Let $K$ be a perfect field with $[K':K]=2$. Suppose $K$ is not real closed and $(-1)$ is not a square in $K$. Then every element of $K$ is a fourth power in $K'$. The linear lattice of distinct field extensions over $K$ is given by $K^{(0)}=K$ and $K^{(n+1)} = K(\alpha_0^{2^{-n}})$ for $n\geq 0$, where $\alpha_0$ is any element of $K'$ such that $N_{K'/K}(\alpha_0)$ is not a square in $K$.
\end{thm}

\begin{proof}
The proof of Proposition \ref{p7} shows that every element of $K$ is either a square or its negative is a square in $K$. Let $i\in K'$ satisfy $i^2=-1$. Now $\beta = u(1+i)$, where $u = \sqrt{\frac{1}{2}}$, satisfies $\beta^2 = i$. As $\sqrt{a}\in K'$ for each $a\in K$, we get that every element of $K$ is a fourth power in $K'$.

For the claim on lattices, we let $L=K'$. We realise that $L$ also has a linear lattice over it and also $(-1)$ is a square in $L$. So, by Theorem \ref{t6}, the linear lattice of distinct field extensions can be described by $L^{(i)} = L(\alpha^{2^{-i}})$ where $\alpha$ is an element of $L$ which is not a square in $L$. We claim that $\alpha = \alpha_0$ as above does the job. Indeed, suppose there is a $\gamma \in L$ such that $\gamma^2 = \alpha_0$. Then $N_{L/K}(\gamma)^2 = N_{L/K}(\alpha_0)$, contradicting our choice of $\alpha_0$.
\end{proof}

This classifies the structure of lattices over such fields to a satisfactory extent. We also have the following corollary.

\begin{cor}\label{ct7.1}
Let $K$ be a field with a linear lattice over it. Then either $K$ is real closed or any finite division algebra over it is a field extension. 
\end{cor}

\section{Quaternion Algebras and Merkurjev's Theorem} \label{s6}

In this section, we state some results about Brauer groups and quaternion algebras which play a crucial role in our proof of the targeted result.
\begin{defn}\label{d:qalg}
Let $K$ be a field of characteristic unequal to $2$. We say $D/K$ is a \textit{Quaternion Algebra} if there are $a,b\in K$ and $i, j, k$ in $D$ such that
\begin{enumerate}
    \item $D$ has a $K-$basis $1,i,j,k$.
    \item $i^2=a, j^2 = b, ij = -k, ji = k$.
\end{enumerate}

For a quaternion algebra, we have the \textit{norm map} $N(t+xi+yj+zk)=t^2 - ax^2 - by^2+abz^2$. Denote the above quaternion algebra as $Q(a,b)$.
\end{defn}

We state a few properties of such algebras. Readers are referred to \cite{voi21}.


\begin{prop}\label{p654} 
let $K$ be a field with characteristic unequal to $2$.
\begin{enumerate}
\item Every quaternion algebra over $K$ is isomorphic to either a division algebra with centre $K$ or $M_2(K)$.
\item The former alternative holds if and only if the norm map is isotropic as a quadratic form over $K$.
\item Every degree $2$ $K-$central division algebra is a quaternion algebra. 
\end{enumerate}
\end{prop}


As a consequence, we derive the following proposition. 

\begin{prop}\label{p8}
    Let $K$ be a perfect field such that
    \begin{enumerate}
        \item $(-1)$ is a square in $K$.
        \item $G = G(/K)$ is pro-2 and non-trivial.
        \item For every non-trivial finite extension $L/K$, $B(L) = 0$.
    \end{enumerate}
    Then $B(K) = 0$.
\end{prop}

\begin{proof}
	If characteristic is $2$, then every element of $K$ is a square in $K$. Let $L/K$ be a cyclic extension of degree $2$. Then by Corollary \ref{cp2.2}, $B(L)=0$ would imply that $B(K)$ is $0$. So we may assume characteristic of $K$ is not $2$.

    Suppose $B(K) \neq 0$. Let $L/K$ be a minimal proper extension. If $L=K'$, then we would immediately lead to a contradiction by Theorem \ref{t6}. So let us assume that there is another minimal proper extension $L_1/K$.

    Now, let $D$ be a non-commutative finite division algebra with center $K$. This will exist as $B(K)$ is assumed to be nonzero. As $L/K$ is a splitting extension of $D$, we get that $D$ has degree $2$ over $K$. So $D$ is a quaternion algebra, say $D\cong Q(a,b)$ with $a,b\in K^*$. Let $K_1  = K(\sqrt{a})$ and $K_2=K(\sqrt{b})$. If $K_1=K_2$, then $a=m^2b$ for some $m\in K^*$. In that case, the norm function for $Q(a,b)$ becomes $N(t+xi+yj+zk)=t^2 - m^2bx^2-by^2+m^2b^2z^2$. As $(-1)$ is a square, we realise that $(\sqrt{-1}mb + k)$ is a non-zero element with a zero norm, hence $Q(a,b)$ is not a division algebra. 

    So we must have that $K_1,K_2$ are linearly disjoint over $K$. Thus $N_{K_1K_2/K_1}$ is surjective as $B(K_1)=0$. In other words, there are $t,x,y,z\in K$ such that 
    $\sqrt{a} = (t+x\sqrt{a})^2 - b (y+z\sqrt{a})^2$. By using the linear independence of $1$ and $\sqrt{a}$ over $K$, we get that $t^2+x^2a - by^2 - abz^2 = 0$ and $1 = 2tx-2byz$. Hence, $\alpha = t + \sqrt{-1}xi +bj+\sqrt{-1}z$ is a non-zero element with a zero norm, showing once again that $Q(a,b)$ is not a division algebra.

   This contradicts that $B(K)\neq 0$. Hence $B(K)=0$, as desired. 
\end{proof}

Finally, we consider a theorem of Merkurjev \cite{mer83} which gives a partial resolution to the following conjecture of Brumer and Rosen \cite{bru68}.

\textit{\textbf{Conjecture.} 
    Let $K$ be a field and $p$ be a prime unequal to the characteristic of $K$. Then $B_p$, the $p-$part of $B=B(L/K)$ satisfies one of the following.
    \begin{enumerate}
        \item It is divisible (possibly trivial).
        \item $p=2$ and it is an elementary abelian $2-$group. 
    \end{enumerate}
}

Merkurjev proves the following theorem. 

\begin{thm}[Merkurjev]\label{t8} 
Suppose $K$ is a field and $p\neq char(K)$ be a prime. Let $\mu_p$ be the set of all $p^{th}$ roots of unity in some algebraic closure of $K$. If $[K(\mu_p):K]\leq 3$, then Brumer-Rosen conjecture holds. 
\end{thm}

We shall need a weaker form of this theorem, in which $\mu_p\subseteq K$. Here is the exact form in which we will need this result. 

\begin{prop}\label{p9}
    Let $K$ be a perfect field such that $B(K)$ is a finite non-trivial group and $B(L) = 0$ for every nontrivial finite extension $L/K$. Then $G(/K)$ is a pro-2 group and $(-1)$ is not a square in $K$.
\end{prop}

\begin{proof}
    By Merkurjev's theorem, if $B=B(K)$, we must have $B_p=0$ for every odd prime $p$ (otherwise, it contains a divisible, hence infinite, group but $B$ is finite). Thus $B=B_2$ and is an elementary abelian $2-$group. Now let $L/K$ be a proper extension. Clearly, $B(K)=B(L/K)$ as $B(L)=0$. So $B(K)$ is $[L:K]$ torsion and also $2-$torsion. As $B[K]\neq 0$, this implies that $2$ divides $[L:K]$. So, by the proof of Proposition \ref{p4:ac}, we get that $G(/K)$ is pro-$2$. Also, Proposition \ref{p8} would show that $(-1)$ is not a square in $K$.
\end{proof}

So we are left to analyse the case when $G(/K)$ is pro-2, $B(K)$ is non-zero and $-1$ is not a square in $K$. This is done by the following theorem. 

\begin{thm}\label{t9} 
Let $K$ be a perfect field such that $(-1)$ is not a square in $K$, $G=G(/K)$ is pro-2 and $B(K)\neq 0$. Then $K$ is either real closed or has a quadratic extension $M/K$ such that $B(M)\neq 0$.
\end{thm}

\begin{proof}
Throughout this proof, we fix an algebraic closure of $K$ and denote by $i$ a square root of $(-1)$ in it. Suppose $K$ is not real closed. If $L=K(i)$ has a nonzero Brauer group, we are done. Otherwise, we must have that $B(L/K)=B(K)\neq 0$. Hence, the norm map $N_{K(i)/K}$ is not surjective 

Suppose there is a $T$ in $K$ such that $T$ is not in the image of the norm and $K(\sqrt{T})\neq K(i)$. Then we claim that $\sqrt{-T}$ is not in the image of $N_{M(i)/M}$ where $M= K(\sqrt{-T})$. Indeed, if it was, then there will be $a,b,c,d\in K$ such that $(a+b\sqrt{-T})^2+(c+d\sqrt{-T})^2 = \sqrt{-T}$. That would imply, by the linear independence of $1,\sqrt{-T}$ over $K$, that $a^2+c^2 = (b^2+d^2)T$ and $2(ab+cd) = 1$. If $b^2+d^2 = 0$, then $a^2+c^2 = 0$. So for $s\in K$, we get $s = (a+bs)^2+(c+ds)^2$, which implies that norm $N_{K(i)/K}$ is surjective - a contradiction.

Otherwise, we will get $T = \frac{(a^2+c^2)(b^2+d^2)}{(b^2+d^2)^2} = \frac{(ab-cd)^2}{(b^2+d^2)^2} + \frac{(ad+bc)^2}{(b^2+d^2)^2}$, whence we get a contradiction to the choice of $T$. So, $\sqrt{-T}$ is not in the image of the norm $N_{M(i)/M}$. This would imply that $B(M(i)/M)\neq 0$, which, in turn, would show that $B(M)\neq 0$.

Finally, if there is no such $T$, then for each $u$ not in the image of the norm $N_{L/K}$, we get that $u=-a^2$ for some $a\in K$. So every element of $K$ is either a square or the negative of a square. This would show that $L=K'$. Now, as $N_{L/K}$ is not surjective, Proposition \ref{p7} would imply that $K$ is real closed. 
\end{proof}


\section{The Main Result} \label{s7}

Finally, we are in a position to state and prove the main result of this article.

\begin{thm}\label{t10} 
A field $K$ with only finitely and positively many noncommutative nonisomorphic finite division algebras over it is real closed.
\end{thm}

\begin{proof}
By Corollary \ref{p1.1}, we realise that if $K$ has finitely many noncommutative division algebras over it, then either $B(L)=0$ for each $L/K$ finite or there is a maximal finite extension $M/K$ such that $B(M)$ is non-zero, finite and for each $L/M$ non-trivial, $B(L)=0$. By Proposition \ref{p9}, $G=G(/M)$ is pro-$2$ and $M$ does not contain a square root of $(-1)$. Now by Theorem \ref{t9}, $M$ is either real closed or it has a quadratic extension with non-trivial Brauer group. The latter alternative violates the choice of $M$. Thus, $M$ must be real closed. But a real closed field has no finite sub-extension by the Artin-Schreier Theorem. This shows that $L=K$, settling our claim.
\end{proof}

In other words, for any field $K$, the cardinality of $S(K)$ is either $0$, $1$ or infinite, with $1$ occurring if and only if $K$ is real closed.


\section*{Acknowledgements}

I would like to thank Professor James Borger of the Australian National University for initiating the discussions leading to this article, reviewing my work and also suggesting to me possible changes, approaches as well as generalisations. I would like to thank the Australian National University and the organisers of the Future Research Talent (FRT) program as most the work presented was done during my stay over there. This work has been supported in part by the FRT award. I would also like to thank my professors and colleagues at my home institute, Indian Statistical Institute, Kolkata for their guidance, support and for nominating me for the aforementioned program.

\begin{thebibliography}{1}

\bibitem{art27} Artin, E., Schreier, O., (1927) Algebraische Konstruktion reeller Körper. \textit{Abh.Math.Semin.Univ.Hambg.} 5, 85–99 .

\bibitem{bru68} Brumer, A., Rosen, M., (1968) On the size of the Brauer Group, \textit{Proc. Amer. Math. Soc.} 19, 707-711.

\bibitem{conart} Conrad, K., \textit{The Artin Schreier Theorem}. \href{https://kconrad.math.uconn.edu/blurbs/galoistheory/artinschreier.pdf}.

\bibitem{fro78}  Frobenius, F. G. (1878) Über lineare Substitutionen und bilineare Formen, \textit{J. fur Reine Angew.} 84:1–63

\bibitem{gui18}  Guillot, P., (2018) \textit{A Gentle Course in Local Class Field Theory}, Cambridge University Press. 

\bibitem{mer83} Merkurjev, A., (1983) Brauer Group of a Field, \textit{Comm.Alg.} 22, 2611-2624.

\bibitem{ser79}  Serre, J.P., (1979) \textit{Local Fields}, Springer. 

\bibitem{ser97}  Serre, J.P., (1997) \textit{Galois Cohomology}, Springer. 

\bibitem{stack23} The {Stacks project authors}, (2023) Chapter 73W, \textit{Brauer Groups} \href{https://stacks.math.columbia.edu/tag/073W}.

\bibitem{voi21}  Voight, J., (2021) \textit{Quaternion Algebra}, Springer. 

\end{thebibliography}



\end{document}
