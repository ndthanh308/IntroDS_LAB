\usepackage{latexsym}
\usepackage{amsmath}
\usepackage{mathrsfs}
\usepackage{amssymb}
%\usepackage{amsthm}
\usepackage{epsfig}
%\usepackage{psfig}
\usepackage{color}
\usepackage{todonotes}
\usepackage[ruled,vlined,linesnumbered]{algorithm2e}
%\usepackage[ruled]{algorithm}
%\usepackage{algorithmic}
\usepackage{algpseudocode}
%\usepackage{algorithmicx}
\usepackage{xspace}
\usepackage{float}
\usepackage{mathtools}
\usepackage{tabularx}
\usepackage{relsize}
\usepackage[pagewise]{lineno}
\usepackage{enumitem}
\usepackage{cite}
%\usepackage{thmtools} 
%\usepackage{thm-restate}
%\declaretheorem[name=Theorem,numberwithin=section]{thm}
%\newtheorem{thm}{\bf Theorem}%Vibha[section]

%\usepackage{amsthm}
%\usepackage{lipsum}
%\newtheorem{theorem}{Theorem}
%\newtheorem{corollary}[theorem]{Corollary}
%\newtheorem{lemma}[theorem]{Lemma}

%\usepackage[linesnumbered, ruled, lined]{algorithm2e}
%\newtheorem{assume}{Assumption}
%\newtheorem{proposition}{Proposition}
%\newtheorem[theorem]{Claim}
%\newtheorem{reducerule}{Reduction}
%\theoremstyle{stylename}
%\usepackage{amsmath,amssymb,amsthm,amsopn}
%\newtheorem{proposition}{Proposition}
%%%%%%%%\newtheorem{duplicate}{Claim}
\newtheorem{mytheorem}{Theorem}[section]
\numberwithin{mytheorem}{section}
\newtheorem{mylemma}{Lemma}[section]
\numberwithin{mylemma}{section}
%%%%%%5\newtheorem{clm}{Claim}[mytheorem]
%%%%%%%\numberwithin{clm}{mytheorem}

\newtheorem{cor}[theorem]{Corollary}
\newtheorem{defn}{Definition}
%%%%%%\newtheorem{obs}{Observation}[mytheorem]
%%%%%%%%\numberwithin{obs}{mytheorem}


%\theoremstyle{definition}
\newtheorem{obs}{Observation}[section]
\numberwithin{obs}{section}
\newtheorem{clm}{Claim}[section]
\numberwithin{clm}{mylemma}



\let\oldnl\nl% Store \nl in \oldnl
\newcommand{\nonl}{\renewcommand{\nl}{\let\nl\oldnl}}% Remove line number


\newtheorem{fact}{Fact}
\newtheorem{assumption}{Assumption}
\newtheorem{subred}{Reduction Rule}
\newtheorem{step}{Step}
\newtheorem{construction}{Construction}

%\newcommand{\sbvdfull}{\textsc{Split to Block Vertex Deletion}\xspace}
%\newcommand{\sbvd}{\textsc{SBVD}\xspace}
%\newcommand{\stvdfull}{\textsc{Split to Threshold Vertex Deletion}\xspace}
%\newcommand{\stvd}{\textsc{STVD}\xspace}
%\newcommand{\hs}{\textsc{4-Hitting Set}\xspace}



\newcommand{\OO}{{\mathcal O}}
\newcommand{\npc}{\textsf{NP}-complete\xspace}
\newcommand{\nph}{\textsf{NP}-hard\xspace}
\newcommand{\no}{{\emph no}\xspace}
\newcommand{\yes}{{\emph yes}\xspace}
\newcommand{\opt}{{\sf OPT}\xspace}
\newcommand{\fpt}{{\sf FPT}\xspace}


%\newcommand{\s}{{\mathcal S}}
%\newcommand{\ccomma}{\mathbin{\raisebox{0.5ex}{,}}}


\newtheorem{innercustomgeneric}{\customgenericname}
\providecommand{\customgenericname}{}
\newcommand{\newcustomtheorem}[2]{%
  \newenvironment{#1}[1]
  {%
   \renewcommand\customgenericname{#2}%
   \renewcommand\theinnercustomgeneric{##1}%
   \innercustomgeneric
  }
  {\endinnercustomgeneric}
}
%\newcustomtheorem{subred}{Reduction Rule}
%\newcustomtheorem{step}{Step}

%\topmargin 0pt
%\advance \topmargin by -\headheight
%\advance \topmargin by -\headsep
%\textheight 8.9in
%\oddsidemargin 0pt
%\evensidemargin \oddsidemargin
%\marginparwidth 0.5in
%\textwidth 6.5in
%
%\parindent 0in
%\parskip 1.5ex
%\renewcommand{\baselinestretch}{1.25}

%\newcommand{\defparprob}[4]{
%\vspace{1mm}
%\noindent\fbox{
%\begin{minipage}{0.96\textwidth}
%#1 \\
%{\bf{Input:}} #2 \\
%{\bf{Parameter:}} #3 \\
%{\bf{Question:}} #4
%\end{minipage}
%}
%\vspace{1mm}
%}

\newcommand{\defparprob}[4]{
  \vspace{1mm}
\noindent\fbox{
  \begin{minipage}{0.96\textwidth}
  \begin{tabular*}{\textwidth}{@{\extracolsep{\fill}}lr} #1  & {\bf{Parameter:}} #3 \\ \end{tabular*}
  {\bf{Input:}} #2  \\
  {\bf{Question:}} #4
  \end{minipage}
  }
  \vspace{1mm}
}




\usepackage{todonotes}
%\newcommand{\pidel}{\textsc{$\Pi$-Deletion}}
\usepackage{complexity}
%\newcommand{\oo}{\mathcal{O}}
%\newcommand{\ostar}{\mathcal{O}^\star}
%\newcommand{\splittoblock}{{\sf SBVD}}
%\newcommand{\minsplittoblock}{{\sf MinSBVD}}
%\newcommand{\splittoth}{{\sf STVD}}
%\newcommand{\minsplittoth}{{\sf MinSTVD}}

\newtheorem{preprocessing rule}{Preprocessing Rule}
\newtheorem{reduction rule}{Reduction Rule}
\newtheorem{branching rule}{Branching Rule}


\newcommand{\parprob}[4]{
%  \vspace{5mm}
\noindent\fbox{
  \begin{minipage}{0.96\textwidth}
  \begin{tabular*}{\textwidth}{@{\extracolsep{\fill}}lr} \textsc{#1}  & {\bf{Parameter(s):}} #4
\\ \end{tabular*}
  {\bf{Input:}} #2  \\
  {\bf{Question:}} #3
  \end{minipage}
  }
  \vspace{5mm}
}

\newcommand{\prob}[3]{
  \vspace{-2mm}
\noindent\fbox{
  \begin{minipage}{0.96\textwidth}
  \begin{tabular*}{\textwidth}{@{\extracolsep{\fill}}lr} \textsc{#1}  
\\ \end{tabular*}
  {\bf{Input:}} #2  \\
  {\bf{Question:}} #3
  \end{minipage}
  }
  \vspace{5mm}
}

\newcommand{\lpd}[4]{
  \vspace{-2mm}
\noindent\fbox{
\begin{minipage}{0.96\textwidth}
#1 \\
{\em{Instance:}} #2 \\
{\em{Feasible Solution:}} #3 \\
{\em{Problem:}} #4\\
\end{minipage}
}
\vspace{1mm}
}

