\documentclass[aps,prl,twocolumn]{revtex4-2}
\usepackage{mathtools}
\usepackage{xcolor}
\usepackage{graphicx}
\usepackage{amsmath}
\usepackage{amssymb}

\newcommand{\ii}{\mathrm{i}}
\newcommand{\av}[1]{\left\langle #1 \right\rangle}              % redefines \begin{equation}
\newcommand{\e}{\varepsilon}              % 
\renewcommand{\t}{\theta}              % 
\renewcommand{\d}{\delta}              % 
\newcommand{\dt}{{\delta t}}              % 
\newcommand{\Dt}{{\Delta t}}              % 
\newcommand{\vp}{{\varphi}}              % 
\newcommand{\w}{{\omega}}              % 
\newcommand{\W}{{\Omega}}              % 


\begin{document}

\title{Dynamics of oscillator populations globally coupled with distributed phase shifts}

\author{Lev A. Smirnov}
\affiliation{Department of Control Theory, Research and Education Mathematical Center ``Mathematics for Future Technologies'',
Nizhny Novgorod State University, Gagarin Av. 23, 603022, Nizhny Novgorod, Russia}
%\affiliation{Institute of Applied Physics of the Russian Academy of Sciences, Ul’yanov Str. 46, 603950, 
%Nizhny Novgorod, Russia}
%\email{}
\author{Arkady Pikovsky}
\affiliation{Department of Physics and Astronomy, University of Potsdam, Karl-Liebknecht-Str. 24/25, 14476, Potsdam-Golm, Germany}
%\email{}

\date{\today}

\begin{abstract}
We consider a population of globally coupled oscillators in which phase shifts with respect to the global force are different. 
Such a setup appears for spatially distributed oscillators with different propagation times of the global forcing signal. In the presence of independent noises and in the thermodynamic limit, we show that the dynamics can be reduced, for arbitrary coupling function, to an effective ensemble of units with identical phase shifts but with a proper renormalization of the order parameters. The same reduction is shown to be valid, by virtue of an analysis of Ott-Antonsen equations, 
for oscillators with a Cauchy distribution of natural frequencies and with the first harmonics coupling. However, the reduction to an effective ensemble may fail if the coupling function is complex enough to ensure the multistability of locked states.
\end{abstract}

\maketitle

%\section{Introduction}


%\section{Motivation for distributed phase shifts}
Globally coupled populations of oscillators serve, since works of Winfree and Kuramoto~\cite{Winfree-67,Kuramoto-75}, as paradigmatic models
for collective synchronization. They describe, in particular,  arrays of Josephson junctions~\cite{Wiesenfeld-Colet-Strogatz-98}, coupled
spin-torque, micromechanical and electrochemical oscillators~\cite{Kiss-Zhai-Hudson-02a,*Tiberkevich_etal-09,*Heinrich_etal-11}, Belousov-Zhabotinsky 
chemical oscillators in droplets~\cite{toiya2010synchronization}.
In many cases, the global nature of coupling is determined by the setup: a common load for Josephson junction, spin-torque 
or electrochemical 
oscillators naturally ensures that a common oscillatory current flows through these units. Also, mechanical oscillators (metronomes or pedestrians)
on a common platform experience the same field. However, the oscillators may have different properties and possess intrinsic noise so 
that despite a global force, the equations for them are not identical.

The most common source of diversity is a spread of natural frequencies of oscillators; this feature has been
incorporated already in the original Kuramoto model~\cite{Kuramoto-75}. Later, one generalized this for a disorder in other
parameters of the oscillators~\cite{Pazo-Montbrio-11,*Iatsenko_etal-13,*Vlasov-Macau-Pikovsky-14,lee2009large}. In this letter, we focus 
on the effect of a disorder 
in the phase shifts in the coupling. The phase shift was absent in the original Kuramoto setup~\cite{Kuramoto-75},
but was introduced in a subsequent paper by Sakaguchi and Kuramoto~\cite{Sakaguchi-Kuramoto-86}. A constant (the same for all units) 
phase shift in the 
coupling, taken into account by Sakaguchi and Kuramoto, naturally appears when collective modes responsible 
for the interaction (e.g., macroscopic oscillations of a common load
for the Josephson junction~\cite{Wiesenfeld-Colet-Strogatz-98} or of a platform 
for the metronomes~\cite{kapitaniak2012synchronization}) are phase-shifted with respect to the forcing
of the oscillators. 

We consider a situation where the phase shifts $\alpha$ for the oscillators in the population are different.
We will describe these shifts in the thermodynamic limit with a distribution density and its circular moments
\begin{equation}
g(\alpha)=(2\pi)^{-1}\sum_m \eta_me^{im\alpha},\; 
\eta_m=\int_0^{2\pi}d\alpha\,e^{-im\alpha}g(\alpha)\,.
\label{eq:ga}
\end{equation}
Several natural circumstances can be behind this property. One example from the literature is a 
population of oscillators divided into two groups: conformists and contrarians~\cite{Hong-Strogatz-11}. The former synchronize
in phase to the global field acting on them, while the latter synchronize in anti-phase. Thus, one can describe
such a population as possessing a two-hump distribution of the phase shifts, in an ideal case, a distribution having two
delta-functions: $g(\alpha)=p \delta(\alpha)+(1-p)\delta(\alpha-\pi)$ (zero shift for conformists and shift $\pi$ for contrarians). 

Another situation where the phase shifts
are spread is where the global force has to propagate to reach 
spatially distributed oscillators~\cite{Vlasov-Pikovsky-Macau-15}. In this
case, the global force acting on different oscillators is subject to different time delays~\cite{lee2009large}, resulting in 
a spread of the phase shifts. Next, we sketch a derivation of the phase equations for this situation. We assume that the oscillators
are otherwise identical and are described by dynamical equations $\dot{\mathbf{x}}_k=\mathbf{f}(\mathbf{x}_k)$.
The force acting on oscillator $k$ is the global force, taken with $k$-dependent time shift $\tau_k$: 
$N^{-1}\sum_j \mathbf{\tilde P}(\mathbf{x}_j(t-\tau_k))$. We suppose that each oscillator possesses a limit
cycle with frequency $\omega$; thus, the phase $\vp=\Phi(\mathbf{x})$ can be introduced, which in the absence of coupling
rotates $\dot\vp=\omega$. Substituting the equation for the phase into the globally coupled system 
$\dot{\mathbf{x}}_k=\mathbf{f}(\mathbf{x}_k)+\e N^{-1}\sum_j \mathbf{\tilde P}(\mathbf{x}_j(t-\tau_k))$, we get
$\dot\vp_k=\w+\e \nabla\Phi(\mathbf{x}_k) N^{-1}\sum_j \mathbf{\tilde P}(\mathbf{x}_j(t-\tau_k))$. Now we use that $\e$ 
is a small parameter, and in the first order in $\e$ express $\mathbf{x}_k$ as a function of $\vp_k$ on the limit cycle.
Then $\nabla\Phi(\mathbf{x}_k)=\mathbf{S}(\vp_k)$ (phase sensitivity function), while the force is a function
of the delayed phase $\vp_j$: $\mathbf{\tilde P}(\mathbf{x}_j(t-\tau_k))=\mathbf{P}(\vp_j-\w \tau_k)$ (here we also
utilized that in the zero order in $\e$ $\dot\vp=\w$). This yields $\dot\vp_k=\w+\e\frac{1}{N}\sum_j \nabla\Phi(\vp)\mathbf{P}(\vp_j-\w\tau_k) $.
The next step is to transform to a slowly varying phase $\psi_k=\vp_k-\w t$ and to time-average the r.h.s.
This standard  procedure begets, after back-transformation to the original phase, the globally coupled ensemble
in the standard Daido-Kuramoto form~\cite{Daido-96}
\begin{equation}
\dot\vp_k=\w+\e\frac{1}{N}\sum_j \Gamma(\vp_j-\vp_k-\alpha_k),\quad \alpha_k=\w \tau_k\;,
\label{eq:dk}
\end{equation}
where $\Gamma(x)=\frac{1}{2\pi}\int_0^{2\pi} dy\;\mathbf{S}(y)\mathbf{P}(y+x)$ is a coupling function, $2\pi$-periodic
in its argument. One can see that a distribution of the delay (propagation) times $\tau_k$ results in the corresponding
distribution of the phase shifts $\alpha_k$.

Distribution of the phase shifts in coupling can also result from a distribution of the local
properties of oscillators. For example, a synaptic coupling of neurons includes a ``low-pass filter'' of an incoming globally
averaged synaptic field $S(t)$: $\tau_k \dot{s}_k+s_k=S$. Thus, if the mean field $S$ oscillates with frequency $\w$, the local 
synaptic current has the phase shift $\approx \w \tau_k$ with respect to $S(t)$, and a disorder in relaxation constants results in a distribution
of the phase shifts in the effective phase coupling model.
%(see a derivation leading to equation of type \eqref{eq:dk} in 
%Supplementary Material~\cite{}.)
%\section{Derivation of the main result}

To proceed, we
generalize equations \eqref{eq:dk}, by including independent white Gaussian noise terms 
$~\sigma \xi_k(t)$, $\av{\xi_k(t)\xi_m(t')=2\delta_{km}\delta(t-t')}$, and allowing for an explicit time dependence in the
parameter $\w$ and the coupling function $\Gamma$. Because the latter is $2\pi$-periodic, we represent it as 
$\Gamma(x;t)=\sum_m f_m(t)e^{im x}$. This allows for rewriting the model \eqref{eq:dk} as a set of Langevin equations 
\begin{equation}
\dot\vp_k=\omega(t)+\sum_m f_m(t) Z_me^{-im(\vp_k+\alpha_k)}+\sigma\xi_k(t)\;,
\label{eq:gkm}
\end{equation}
where we introduced Daido-Kuramoto order parameters $Z_m=N^{-1}\sum_j e^{i\vp_j}=\av{e^{i\vp}}$. In the thermodynamic limit,
we describe the evolution of the probability density $P(\vp,t|\alpha)$ of the phases $\vp$, conditioned by the phase shifts $\alpha$, with 
the Fokker-Planck equation
\begin{equation}
\partial_t P(\vp,t|\alpha)+\partial_\vp [(\w+\sum_m f_m Z_m e^{-im(\vp+\alpha)})P]=\sigma^2\partial_{\vp\vp}P
\label{eq:fpe}
\end{equation}
The order parameters are represented as $Z_m(t)=\int_0^{2\pi} d\vp\int_0^{2\pi}d\alpha\, e^{im\vp} P(\vp,t|\alpha)$.
We now change the variable $\t=\vp+\alpha$ and obtain
\begin{equation}
\partial_t P(\t,t|\alpha)+\partial_\t [(\w+\sum_m f_m Z_m e^{-im \t})P]=\sigma^2\partial_{\t\t}P\;.
\label{eq:th}
\end{equation}
The crucial observation is that although the density $P(\t,t|\alpha)$ generally depends on the phase shifts, the equation \eqref{eq:th}
does not. One can argue that any initial distribution dependence on the parameter $\alpha$ will eventually disappear,
so that asymptotically at large times  $P(\t,t|\alpha)\to P(\t,t)$. This property is well-established for a 
time-independent Fokker-Planck equation~\cite{Gardiner-96}. In terms of the theory of 
partial differential equations, this property means global asymptotic stability (GAS) of solutions of the parabolic 
Fokker-Planck equation \eqref{eq:th}, cf.~\cite{Calogero-12}. In the context of general Markov processes, this property
is nothing else as ergodicity, often formulated as ``loss of memory''~\cite{Kulik-18}. Although GAS is expected to be valid for 
\eqref{eq:th}~\cite{KK}, we have not found a proof in the mathematical literature. A proof of GAS for the time-dependent master equation, which is a closest finite-dimensional
analog of the Fokker-Planck equation has been 
given in Ref.~\cite{Earnshaw-Keener-10}. From the physical viewpoint, GAS of Eq.~\eqref{eq:th}
appears rather evident, particularly because it is defined on a finite domain $0\leq \t <2\pi$, and one does not face difficulties
in defining a convergence of probability distributions on an infinite domain.

With the GAS property, for large times, the order parameters $Z_m$ can be expressed via the density $P(\t,t)$ as
\begin{equation}
\begin{gathered}
Z_m(t)=%\av{e^{im(\t-\alpha)}}=
z_m(t) \eta_m,\quad 
%\eta_m=\int_0^{2\pi} d\alpha \,e^{-im\alpha} g(\alpha),\quad 
z_m(t)=\int_0^{2\pi} d\t \,e^{im\t} P(\t,t)\;.
\end{gathered}
\label{eq:orp}
\end{equation}
This closes the description of the population. The resulting equations \eqref{eq:th},\eqref{eq:orp} can be 
in words formulated as follows.
One considers an auxiliary ensemble $\t$ 
of identical oscillators without any phase shift. Only in the calculation of the
order parameters, one takes the order parameters $z_m$ of this auxiliary population and multiplies them, according to \eqref{eq:orp}, 
with
the circular moments $\eta_m$ of the distribution of the phase shifts $g(\alpha)$. 

One can go one step further to obtain the microscopic equations for the auxiliary ensemble $\t$. As it follows from
Eq.~\eqref{eq:th}, because of the relation \eqref{eq:orp}, the coupling term has the form $\sim\sum_m f_m\eta_m z_m e^{-im\t}$. 
This allows for introducing the effective coupling in terms of the Fourier modes as $\tilde{f}_m=f_m\eta_m$. Using the inverse 
Fourier transform, we obtain an effective coupling function for the auxiliary ensemble as
\begin{equation}
\tilde{\Gamma}(x)=(2\pi)^{-1}\sum_m \tilde{f}_me^{imx}=\int_0^{2\pi} \Gamma(y)g(x-y)\, dy\;.
\label{eq:con}
\end{equation}
Thus, the auxiliary ensemble is described by equations
\begin{equation}
\dot\theta_k=\omega+\varepsilon\frac{1}{N}\sum_j \tilde{\Gamma}(\theta_j-\theta_k)+\sigma\xi_k(t)\;.
\label{eq:equiv}
\end{equation}
For example, the original coupling function $\Gamma(x)$ may have quite a complex waveform, but if the distribution
of the phase shifts has just one harmonics 
\begin{equation}
g(\alpha)=(2\pi)^{-1}(1+\eta_1 \cos(\alpha-\alpha_0))\;,
\label{eq:g1h}
\end{equation}
then the effective
coupling function $\tilde{\Gamma}(x)$ will possess only one harmonics as well (we illustrate such a situation below).


We stress here that while the full problem \eqref{eq:fpe} as a \textit{nonlinear} Fokker-Planck equation can demonstrate
multistability and hysteresis, this does not contradict GAS because, for the latter property, one considers
Eq.~\eqref{eq:th} with predetermined values of the moments $Z_m(t)$ (in other words, GAS corresponds to a 
transversal stability of a solution for a particular time course of the global force).

The property of a unique asymptotic solution of the auxiliary problem heavily
relies on the loss of memory (dissipativity due to noise-induced diffusion) of the kinetic equation \eqref{eq:orp}. We next 
demonstrate that this dissipativity also may occur in a noise-free setup due to a distribution
of natural frequencies in the population. For this, we have to assume additionally that a distribution
of natural frequencies $w(\omega)$ is independent of the distribution of phase shifts $g(\alpha)$. Furthermore,
to be able to apply the Ott-Antonsen approach~\cite{Ott-Antonsen-08}, we assume that the coupling is due to the first harmonics only,
i.e. $f_m=0$ for $|m|>1$. Under these assumptions, the kinetic equation for the density $P(\vp,t|\alpha,\w)$ 
(which is the same as \eqref{eq:fpe} but with $\sigma=0$) can be represented as an infinite set of equations for the $\alpha,\w$-dependent
order parameters $S_m(t|\alpha,\w)=\int_0^{2\pi}d\vp\,e^{im\vp}P(\vp,t|\alpha,\w)$:
\begin{equation}
\begin{gathered}
m^{-1}\dot S_m(t|\alpha,\w)=i\w S_m+H e^{-i\alpha} S_{m-1}-H^* e^{i\alpha}S_{m+1}\,,\\
Z_m(t)=\av{S_m(t|\alpha,\w)}_{\alpha,\w}\,.
%=\\=\int_0^{2\pi} d\alpha\, g(\alpha)\int_0^{2\pi} d\w\,w(\w) S_m(t|\alpha,\w)
\end{gathered}
\label{eq:omd}
\end{equation}
Here we write a general first-harmonics forcing $H$ which can have any dependence on the order parameters $Z_m$ and on time, 
in the particular case of the Kuramoto model \eqref{eq:gkm} one has $H=f_1 Z_1$. Next, following 
the seminal Ott-Antonsen ansatz~\cite{Ott-Antonsen-08}, we take 
a Cauchy distribution of natural frequencies $w_c(\w)=\gamma[\pi((\w-\w_0)^2+\gamma^2)]$. Assuming that
the modes $S_m$ are analytic functions of the frequency $\w$ in the upper half-space, one can perform the
integration over $\w$ in \eqref{eq:omd} by the residue method, what yields 
$z_m(t;\alpha)=\av{S_m}_\w=S_m(t|\alpha,\w_0+i\gamma)$. This allows to rewrite \eqref{eq:omd} as 
\begin{equation}
\begin{gathered}
m^{-1}\dot z_m(t|\alpha)=(i\w_0-\gamma) z_m+H e^{-i\alpha} z_{m-1}-H^* e^{i\alpha}z_{m+1}\;,\\
Z_m(t)=\av{z_m(t|\alpha)}_\alpha
%=\int_0^{2\pi} d\alpha\, g(\alpha) z_m(t|\alpha)
\;.
\end{gathered}
\label{eq:omd2}
\end{equation}
Let us now perform a change of variables $q_m(t|\alpha)=e^{im\alpha} z_m(t|\alpha)$. This reduces \eqref{eq:omd2}
to
\begin{equation}
\begin{gathered}
m^{-1}\dot q_m(t|\alpha)=(i\w_0-\gamma) z_m+H  q_{m-1}-H^* q_{m+1}\;,\\
Z_m(t)=\av{q_m(t|\alpha)e^{-im \alpha}}_\alpha
%=\int_0^{2\pi} d\alpha\, g(\alpha) e^{-im\alpha} q_m(t|\alpha)
\;.
\end{gathered}
\label{eq:omd3}
\end{equation}
The final step is to assume that for each $\alpha$, circular moments are the powers of the first moment 
$q_m(t|\alpha)=Q^m(t|\alpha)$, $m\geq 0$ (this corresponds to the assumption that the distribution of the phases
is a wrapped Cauchy distribution)~\cite{Ott-Antonsen-08}. The whole hierarchy of equations reduces just to one equation 
(where $\alpha$ is a parameter)
\begin{gather}
\dot Q(t|\alpha)=(i\w_0-\gamma) Q(t|\alpha)+H-H^* Q^2 (t|\alpha)\;, \label{eq:eqQ}\\
Z_m(t)=\int_0^{2\pi} d\alpha Q^m(t|\alpha) e^{-im\alpha} g(\alpha)\;.\label{eq:eqZ}
\end{gather}

Note that in Eq.~\eqref{eq:eqQ} only the order parameter $Q$ depends on the phase shifts $\alpha$
through initial conditions,
but its dynamics is not. The GAS property above means that in the course of the dynamics, the memory about the initial state
gets lost and $Q(t|\alpha)\to Q(t)$. We derive this stability in the linear approximation. 
Let us assume that $Q(t|\alpha)=\bar{Q}(t)+\beta(t|\alpha)$, where $\beta$ is small.
Then the equation for $\bar{Q}$ is \eqref{eq:eqQ}, and the equation for $\beta$ reads
$\dot \beta=(i\omega_0-\gamma) \beta-2H^* \bar{Q}\beta$.
Let us introduce still another variable $Y$ according to $\beta=(1-|\bar{Q}|^2)Y$.
Then the equation for $Y$ reads
$\dot Y=(i\omega+H\bar{Q}^*-H^*\bar{Q}-\gamma (1+|\bar{Q}|^2)(1-|\bar{Q}|^2)^{-1})Y$.
Because $\text{Re}(H\bar{Q}^*-H^*\bar Q)=0$, for the absolute value of $Y$ we obtain
$
\frac{d}{dt}|Y|=-\gamma \frac{1+|\bar Q|^2}{1-|\bar Q|^2}|Y|.
$
Because according to \eqref{eq:eqQ} $0\leq |\bar{Q}|^2<1$, variable
$|Y|$ decays to zero exponentially, and thus $Y\to 0$. This means that also $\beta\to 0$, which proves
linear asymptotic stability. One can see that a spread of natural frequency is essential;
for $\gamma=0$, we have $|Y|=const$, and the memory about initial conditions is not lost (this
is another manifestation of the Watanabe-Strogatz integrability of a population of 
identical oscillators~\cite{Watanabe-Strogatz-93,*Watanabe-Strogatz-94}).
 
After the convergence  $Q(t|\alpha)\to Q(t)$, the dynamics of the population with a distribution of phase shifts
reduces to a single dynamical equation for the auxiliary order parameter $Q(t)$, the original
order parameters are related to it via circular moments of the distribution of the phase shifts:
\begin{gather}
\dot Q(t)=(i\w_0-\gamma) Q(t)+H-H^* Q^2 (t)\;, \label{eq:eqQ2}\\
Z_m(t)=\eta_m Q^m(t)
%,\quad \eta_m=\int_0^{2\pi} d\alpha e^{-im\alpha} g(\alpha)
\;.\label{eq:eqZ2}
\end{gather}
This completes a closed description for a population of oscillators with a first-harmonics-coupling, Cauchy distribution
of the natural frequencies and arbitrary distribution of the phase shifts. 

%\section{General remarks about GAS property}
Above, we have argued for the GAS property for noisy oscillators with arbitrary coupling and derived linear asymptotic stability for 
oscillators fulfilling the Ott-Antonsen ansatz (i.e., with the first harmonics coupling and the Cauchy
distribution of natural frequencies). In both cases, the dissipativeness leading to GAS is due to disorder, either in the form of noise or in the form 
of a continuous distribution of natural frequencies. The necessity of disorder is rather obvious because, in particular,
for a first-harmonics coupling without disorder (i.e., for a population of identical oscillators), the 
Watanabe-Strogatz integrability holds, which prevents asymptotic stability. However, the presence of disorder
in the form of a continuous distribution of natural frequencies is generally
not sufficient for GAS. On the one hand, a continuous distribution of natural frequencies leads to an effective
(non-chaotic) mixing in a population due to the mechanism of Landau damping~\cite{Strogatz-00,*fernandez2016landau,*Dietert-16}
(for explicit finite-dimensional dissipative reductions for distributions with nice analytic properties see~\cite{Klinshov_etal-21,*campa2022study,*Pyragas-Pyragas-22}). On the other hand, this damping
occurs only for differentially rotating oscillators and not for the locked ones. The latter do not mix but 
form a coherent cluster. One can expect GAS for the whole system if there can be just one cluster, i.e., where the coupling function $\Gamma(x)$ in \eqref{eq:dk} allows for one stable synchronous state for coupled units. This is the case
if this coupling function is just $\sin(\cdot)$, like in the case where the OA ansatz is applicable. Thus we expect that GAS will be valid
for coupling functions possessing only one stable synchronous state but will be violated for multi-stable 
couplings~\cite{Daido-95,*Komarov-Pikovsky-13a,*Komarov-Pikovsky-14}.
Note that with an unbounded noise (e.g., with a Gaussian white noise assumed in the discussion of the Fokker-Planck equation above)
the multi-stability disappears because now transitions between different stable clusters become possible. But for a bounded
noise, multistability can survive. 

There is still an interesting possibility for GAS to be valid, at least for some initial states, for continuous
distributions of natural frequencies even if the coupling $\Gamma(x)$ is so complex to beget multistability. 
This relies on the observation above that in the presence of GAS, the effective coupling function $\tilde\Gamma(x)$ appears, which is a convolution of
the original coupling function and the distribution of the phase shifts. Suppose that the distribution density 
of the phase shifts has the form \eqref{eq:g1h}, i.e., it has only one Fourier harmonics. Then the
effective coupling function $\tilde\Gamma(x)$ will also have only one Fourier harmonics, fulfilling 
conditions for the OA ansatz. Thus, this system will possess the state described by Eqs.~\eqref{eq:eqQ2},\eqref{eq:eqZ2}
above (provided the distribution of frequencies is a Cauchy one). This regime is expected to be robust,
so it has a finite basin of attraction even if the initial conditional distribution of the auxiliary phases $\theta=\varphi+\alpha$
does depend on $\alpha$. However, this state may coexist with other states, not fulfilling GAS. 
We illustrate this by considering a population of oscillators with a Cauchy distribution 
of natural frequencies, one-harmonic distribution of the phase shifts \eqref{eq:g1h}, and a rather complex original 
coupling function $\Gamma(x)$ (see inset in Fig.~\ref{fig:fig1}(a)).
%=-0.2 e^{ix}+(0.3+i0.3)e^{i2x}+(0.2-i0.2)e^{i3x}+c.c.$. 
We show two runs from different initial phase distributions; in one (panels (b,d)), the 
system converges to a state where the distribution of $\theta$ does not depend on $\alpha$, while in another run (panels (a,c))
this dependence does not disappear. In the former state, the theory developed above can be applied. Because the effective
coupling function $\tilde\Gamma(\cdot)$ possesses only one harmonics, the final state lies on the OA manifold.
To demonstrate this, we together with circular moments $z_m$, $m=1,2,3$, show the two circular cumulants
$\kappa_2=z_2-z_1^2$ and $\kappa_3=\frac{1}{2}(z_3-3z_2z_1+2z_1^3)$, which characterize deviations 
from the OA manifold~\cite{Tyulkina_etal-18}. One can see that over time, the magnitudes of these cumulants decrease (and saturate at the
level of the finite-size fluctuations). In contradistinction, for other initial conditions, a regime with a clear dependence
on the distribution on $\alpha$ establishes (see panel (c)), 
in which the effective coupling function cannot be introduced, and the state is far from the OA manifold.


% Figure environment removed

%\section{Conclusion}
Summarizing, we have demonstrated that in many situations, the dynamics of an ensemble of globally coupled
oscillators with distributed phase shifts can be reduced to an effective ensemble without phase shifts, with a 
proper renormalization of the order parameters. This reduction heavily relies on the ensemble's global asymptotic stability (GAS) property, ensuring convergence to the same asymptotic state (potentially non-stationary) from 
arbitrary initial distributions, provided that the time evolution of driving forces on the oscillators is fixed. This property
of independence of the final state on the initial distribution can be recast as ergodicity. The GAS property can be 
naturally expected for ensembles with unbounded independent noises (e.g., Gaussian or Cauchy noise) and \textit{any} coupling
function. The theory establishes that, in this case, the dynamics is described by an \textit{effective} coupling function 
(the convolution of the original one and the distribution of the phase shifts). 

The situation is more subtle if there is no noise,
but a disorder is due to a continuous distribution of natural frequencies. Here one can expect GAS only
for simple enough coupling functions, possessing no multistability. For example, for a single-harmonic coupling function and a Cauchy
distribution of natural frequencies, we have explicitly demonstrated asymptotic stability by virtue of the OA approach. Furthermore, we showed numerically
that the GAS property might be violated for cases where the coupling function possesses multistability. 

In our setup, we assumed that the phase shifts are random parameters of oscillators, independent of the natural frequencies and other
parameters that can potentially be distributed (e.g., noise strengths). Another idealization adopted in the present study is that the 
magnitude of the coupling is the same for all units. The latter assumption might be violated if the propagating global 
field decays on the way to remote units. Our theory is only valid if this effect is much smaller than the spread of the phase shifts. 
This happens if
the wavelength of the signal-transmitting wave is small enough.


\acknowledgments
The authors thank K. Khanin, E. Knobloch, and J. Winkler for valuable discussions. AP was supported by DFG (grant No. PI 220/21-1).
LAS was supported by the Russian Science Foundation (grant no. 22-12-00348).
%\bibliographystyle{apsrev4-1}
%\bibliography{bibliography}
%apsrev4-2.bst 2019-01-14 (MD) hand-edited version of apsrev4-1.bst
%Control: key (0)
%Control: author (8) initials jnrlst
%Control: editor formatted (1) identically to author
%Control: production of article title (0) allowed
%Control: page (0) single
%Control: year (1) truncated
%Control: production of eprint (0) enabled
\begin{thebibliography}{34}%
\makeatletter
\providecommand \@ifxundefined [1]{%
 \@ifx{#1\undefined}
}%
\providecommand \@ifnum [1]{%
 \ifnum #1\expandafter \@firstoftwo
 \else \expandafter \@secondoftwo
 \fi
}%
\providecommand \@ifx [1]{%
 \ifx #1\expandafter \@firstoftwo
 \else \expandafter \@secondoftwo
 \fi
}%
\providecommand \natexlab [1]{#1}%
\providecommand \enquote  [1]{``#1''}%
\providecommand \bibnamefont  [1]{#1}%
\providecommand \bibfnamefont [1]{#1}%
\providecommand \citenamefont [1]{#1}%
\providecommand \href@noop [0]{\@secondoftwo}%
\providecommand \href [0]{\begingroup \@sanitize@url \@href}%
\providecommand \@href[1]{\@@startlink{#1}\@@href}%
\providecommand \@@href[1]{\endgroup#1\@@endlink}%
\providecommand \@sanitize@url [0]{\catcode `\\12\catcode `\$12\catcode
  `\&12\catcode `\#12\catcode `\^12\catcode `\_12\catcode `\%12\relax}%
\providecommand \@@startlink[1]{}%
\providecommand \@@endlink[0]{}%
\providecommand \url  [0]{\begingroup\@sanitize@url \@url }%
\providecommand \@url [1]{\endgroup\@href {#1}{\urlprefix }}%
\providecommand \urlprefix  [0]{URL }%
\providecommand \Eprint [0]{\href }%
\providecommand \doibase [0]{https://doi.org/}%
\providecommand \selectlanguage [0]{\@gobble}%
\providecommand \bibinfo  [0]{\@secondoftwo}%
\providecommand \bibfield  [0]{\@secondoftwo}%
\providecommand \translation [1]{[#1]}%
\providecommand \BibitemOpen [0]{}%
\providecommand \bibitemStop [0]{}%
\providecommand \bibitemNoStop [0]{.\EOS\space}%
\providecommand \EOS [0]{\spacefactor3000\relax}%
\providecommand \BibitemShut  [1]{\csname bibitem#1\endcsname}%
\let\auto@bib@innerbib\@empty
%</preamble>
\bibitem [{\citenamefont {Winfree}(1967)}]{Winfree-67}%
  \BibitemOpen
  \bibfield  {author} {\bibinfo {author} {\bibfnamefont {A.~T.}\ \bibnamefont
  {Winfree}},\ }\bibfield  {title} {\bibinfo {title} {Biological rhythms and
  the behavior of populations of coupled oscillators},\ }\href@noop {}
  {\bibfield  {journal} {\bibinfo  {journal} {J. Theor. Biol.}\ }\textbf
  {\bibinfo {volume} {16}},\ \bibinfo {pages} {15} (\bibinfo {year}
  {1967})}\BibitemShut {NoStop}%
\bibitem [{\citenamefont {Kuramoto}(1975)}]{Kuramoto-75}%
  \BibitemOpen
  \bibfield  {author} {\bibinfo {author} {\bibfnamefont {Y.}~\bibnamefont
  {Kuramoto}},\ }\bibfield  {title} {\bibinfo {title} {Self-entrainment of a
  population of coupled nonlinear oscillators},\ }in\ \href@noop {} {\emph
  {\bibinfo {booktitle} {International Symposium on Mathematical Problems in
  Theoretical Physics}}},\ \bibinfo {editor} {edited by\ \bibinfo {editor}
  {\bibfnamefont {H.}~\bibnamefont {Araki}}}\ (\bibinfo  {publisher} {Springer
  Lecture Notes Phys., v. 39},\ \bibinfo {address} {New York},\ \bibinfo {year}
  {1975})\ p.\ \bibinfo {pages} {420}\BibitemShut {NoStop}%
\bibitem [{\citenamefont {Wiesenfeld}\ \emph {et~al.}(1998)\citenamefont
  {Wiesenfeld}, \citenamefont {Colet},\ and\ \citenamefont
  {Strogatz}}]{Wiesenfeld-Colet-Strogatz-98}%
  \BibitemOpen
  \bibfield  {author} {\bibinfo {author} {\bibfnamefont {K.}~\bibnamefont
  {Wiesenfeld}}, \bibinfo {author} {\bibfnamefont {P.}~\bibnamefont {Colet}},\
  and\ \bibinfo {author} {\bibfnamefont {S.}~\bibnamefont {Strogatz}},\
  }\bibfield  {title} {\bibinfo {title} {{Frequency locking in Josephson
  arrays: Connection with the Kuramoto model}},\ }\href@noop {} {\bibfield
  {journal} {\bibinfo  {journal} {Physical Review E}\ }\textbf {\bibinfo
  {volume} {57}},\ \bibinfo {pages} {1563} (\bibinfo {year}
  {1998})}\BibitemShut {NoStop}%
\bibitem [{\citenamefont {Kiss}\ \emph {et~al.}(2002)\citenamefont {Kiss},
  \citenamefont {Zhai},\ and\ \citenamefont {Hudson}}]{Kiss-Zhai-Hudson-02a}%
  \BibitemOpen
  \bibfield  {author} {\bibinfo {author} {\bibfnamefont {I.}~\bibnamefont
  {Kiss}}, \bibinfo {author} {\bibfnamefont {Y.}~\bibnamefont {Zhai}},\ and\
  \bibinfo {author} {\bibfnamefont {J.}~\bibnamefont {Hudson}},\ }\bibfield
  {title} {\bibinfo {title} {Emerging coherence in a population of chemical
  oscillators},\ }\href@noop {} {\bibfield  {journal} {\bibinfo  {journal}
  {Science}\ }\textbf {\bibinfo {volume} {296}},\ \bibinfo {pages} {1676}
  (\bibinfo {year} {2002})}\BibitemShut {NoStop}%
\bibitem [{\citenamefont {Tiberkevich}\ \emph {et~al.}(2009)\citenamefont
  {Tiberkevich}, \citenamefont {Slavin}, \citenamefont {Bankowski},\ and\
  \citenamefont {Gerhart}}]{Tiberkevich_etal-09}%
  \BibitemOpen
  \bibfield  {author} {\bibinfo {author} {\bibfnamefont {V.}~\bibnamefont
  {Tiberkevich}}, \bibinfo {author} {\bibfnamefont {A.}~\bibnamefont {Slavin}},
  \bibinfo {author} {\bibfnamefont {E.}~\bibnamefont {Bankowski}},\ and\
  \bibinfo {author} {\bibfnamefont {G.}~\bibnamefont {Gerhart}},\ }\bibfield
  {title} {\bibinfo {title} {Phase-locking and frustration in an array of
  nonlinear spin-torque nano-oscillators},\ }\href@noop {} {\bibfield
  {journal} {\bibinfo  {journal} {Appl. Phys. Lett.}\ }\textbf {\bibinfo
  {volume} {95}},\ \bibinfo {pages} {262505} (\bibinfo {year}
  {2009})}\BibitemShut {NoStop}%
\bibitem [{\citenamefont {Heinrich}\ \emph {et~al.}(2011)\citenamefont
  {Heinrich}, \citenamefont {Ludwig}, \citenamefont {Qian}, \citenamefont
  {Kubala},\ and\ \citenamefont {Marquardt}}]{Heinrich_etal-11}%
  \BibitemOpen
  \bibfield  {author} {\bibinfo {author} {\bibfnamefont {G.}~\bibnamefont
  {Heinrich}}, \bibinfo {author} {\bibfnamefont {M.}~\bibnamefont {Ludwig}},
  \bibinfo {author} {\bibfnamefont {J.}~\bibnamefont {Qian}}, \bibinfo {author}
  {\bibfnamefont {B.}~\bibnamefont {Kubala}},\ and\ \bibinfo {author}
  {\bibfnamefont {F.}~\bibnamefont {Marquardt}},\ }\bibfield  {title} {\bibinfo
  {title} {Collective dynamics in optomechanical arrays},\ }\href@noop {}
  {\bibfield  {journal} {\bibinfo  {journal} {Phys. Rev. Lett.}\ }\textbf
  {\bibinfo {volume} {107}},\ \bibinfo {pages} {043603} (\bibinfo {year}
  {2011})}\BibitemShut {NoStop}%
\bibitem [{\citenamefont {Toiya}\ \emph {et~al.}(2010)\citenamefont {Toiya},
  \citenamefont {Gonz{\'a}lez-Ochoa}, \citenamefont {Vanag}, \citenamefont
  {Fraden},\ and\ \citenamefont {Epstein}}]{toiya2010synchronization}%
  \BibitemOpen
  \bibfield  {author} {\bibinfo {author} {\bibfnamefont {M.}~\bibnamefont
  {Toiya}}, \bibinfo {author} {\bibfnamefont {H.~O.}\ \bibnamefont
  {Gonz{\'a}lez-Ochoa}}, \bibinfo {author} {\bibfnamefont {V.~K.}\ \bibnamefont
  {Vanag}}, \bibinfo {author} {\bibfnamefont {S.}~\bibnamefont {Fraden}},\ and\
  \bibinfo {author} {\bibfnamefont {I.~R.}\ \bibnamefont {Epstein}},\
  }\bibfield  {title} {\bibinfo {title} {Synchronization of chemical
  micro-oscillators},\ }\href@noop {} {\bibfield  {journal} {\bibinfo
  {journal} {The Journal of Physical Chemistry Letters}\ }\textbf {\bibinfo
  {volume} {1}},\ \bibinfo {pages} {1241} (\bibinfo {year} {2010})}\BibitemShut
  {NoStop}%
\bibitem [{\citenamefont {Paz\'o}\ and\ \citenamefont
  {Montbri\'o}(2011)}]{Pazo-Montbrio-11}%
  \BibitemOpen
  \bibfield  {author} {\bibinfo {author} {\bibfnamefont {D.}~\bibnamefont
  {Paz\'o}}\ and\ \bibinfo {author} {\bibfnamefont {E.}~\bibnamefont
  {Montbri\'o}},\ }\bibfield  {title} {\bibinfo {title} {The {K}uramoto model
  with distributed shear},\ }\href@noop {} {\bibfield  {journal} {\bibinfo
  {journal} {EPL}\ }\textbf {\bibinfo {volume} {95}},\ \bibinfo {pages} {60007}
  (\bibinfo {year} {2011})}\BibitemShut {NoStop}%
\bibitem [{\citenamefont {Iatsenko}\ \emph {et~al.}(2013)\citenamefont
  {Iatsenko}, \citenamefont {Petkoski}, \citenamefont {McClintock},\ and\
  \citenamefont {Stefanovska}}]{Iatsenko_etal-13}%
  \BibitemOpen
  \bibfield  {author} {\bibinfo {author} {\bibfnamefont {D.}~\bibnamefont
  {Iatsenko}}, \bibinfo {author} {\bibfnamefont {S.}~\bibnamefont {Petkoski}},
  \bibinfo {author} {\bibfnamefont {P.~V.~E.}\ \bibnamefont {McClintock}},\
  and\ \bibinfo {author} {\bibfnamefont {A.}~\bibnamefont {Stefanovska}},\
  }\bibfield  {title} {\bibinfo {title} {Stationary and traveling wave states
  of the {K}uramoto model with an arbitrary distribution of frequencies and
  coupling strengths},\ }\href@noop {} {\bibfield  {journal} {\bibinfo
  {journal} {Phys. Rev. Lett.}\ }\textbf {\bibinfo {volume} {110}},\ \bibinfo
  {pages} {064101} (\bibinfo {year} {2013})}\BibitemShut {NoStop}%
\bibitem [{\citenamefont {Vlasov}\ \emph {et~al.}(2014)\citenamefont {Vlasov},
  \citenamefont {Macau},\ and\ \citenamefont
  {Pikovsky}}]{Vlasov-Macau-Pikovsky-14}%
  \BibitemOpen
  \bibfield  {author} {\bibinfo {author} {\bibfnamefont {V.}~\bibnamefont
  {Vlasov}}, \bibinfo {author} {\bibfnamefont {E.~E.~N.}\ \bibnamefont
  {Macau}},\ and\ \bibinfo {author} {\bibfnamefont {A.}~\bibnamefont
  {Pikovsky}},\ }\bibfield  {title} {\bibinfo {title} {Synchronization of
  oscillators in a {K}uramoto-type model with generic coupling},\ }\href
  {https://doi.org/http://dx.doi.org/10.1063/1.4880835} {\bibfield  {journal}
  {\bibinfo  {journal} {Chaos}\ }\textbf {\bibinfo {volume} {24}},\ \bibinfo
  {eid} {023120} (\bibinfo {year} {2014})}\BibitemShut {NoStop}%
\bibitem [{\citenamefont {Lee}\ \emph {et~al.}(2009)\citenamefont {Lee},
  \citenamefont {Ott},\ and\ \citenamefont {Antonsen}}]{lee2009large}%
  \BibitemOpen
  \bibfield  {author} {\bibinfo {author} {\bibfnamefont {W.~S.}\ \bibnamefont
  {Lee}}, \bibinfo {author} {\bibfnamefont {E.}~\bibnamefont {Ott}},\ and\
  \bibinfo {author} {\bibfnamefont {T.~M.}\ \bibnamefont {Antonsen}},\
  }\bibfield  {title} {\bibinfo {title} {Large coupled oscillator systems with
  heterogeneous interaction delays},\ }\href@noop {} {\bibfield  {journal}
  {\bibinfo  {journal} {Physical review letters}\ }\textbf {\bibinfo {volume}
  {103}},\ \bibinfo {pages} {044101} (\bibinfo {year} {2009})}\BibitemShut
  {NoStop}%
\bibitem [{\citenamefont {Sakaguchi}\ and\ \citenamefont
  {Kuramoto}(1986)}]{Sakaguchi-Kuramoto-86}%
  \BibitemOpen
  \bibfield  {author} {\bibinfo {author} {\bibfnamefont {H.}~\bibnamefont
  {Sakaguchi}}\ and\ \bibinfo {author} {\bibfnamefont {Y.}~\bibnamefont
  {Kuramoto}},\ }\bibfield  {title} {\bibinfo {title} {A soluble active rotator
  model showing phase transition via mutual entrainment},\ }\href@noop {}
  {\bibfield  {journal} {\bibinfo  {journal} {Prog. Theor. Phys.}\ }\textbf
  {\bibinfo {volume} {76}},\ \bibinfo {pages} {576} (\bibinfo {year}
  {1986})}\BibitemShut {NoStop}%
\bibitem [{\citenamefont {Kapitaniak}\ \emph {et~al.}(2012)\citenamefont
  {Kapitaniak}, \citenamefont {Czolczynski}, \citenamefont {Perlikowski},
  \citenamefont {Stefanski},\ and\ \citenamefont
  {Kapitaniak}}]{kapitaniak2012synchronization}%
  \BibitemOpen
  \bibfield  {author} {\bibinfo {author} {\bibfnamefont {M.}~\bibnamefont
  {Kapitaniak}}, \bibinfo {author} {\bibfnamefont {K.}~\bibnamefont
  {Czolczynski}}, \bibinfo {author} {\bibfnamefont {P.}~\bibnamefont
  {Perlikowski}}, \bibinfo {author} {\bibfnamefont {A.}~\bibnamefont
  {Stefanski}},\ and\ \bibinfo {author} {\bibfnamefont {T.}~\bibnamefont
  {Kapitaniak}},\ }\bibfield  {title} {\bibinfo {title} {Synchronization of
  clocks},\ }\href@noop {} {\bibfield  {journal} {\bibinfo  {journal} {Physics
  Reports}\ }\textbf {\bibinfo {volume} {517}},\ \bibinfo {pages} {1} (\bibinfo
  {year} {2012})}\BibitemShut {NoStop}%
\bibitem [{\citenamefont {Hong}\ and\ \citenamefont
  {Strogatz}(2011)}]{Hong-Strogatz-11}%
  \BibitemOpen
  \bibfield  {author} {\bibinfo {author} {\bibfnamefont {H.}~\bibnamefont
  {Hong}}\ and\ \bibinfo {author} {\bibfnamefont {S.~H.}\ \bibnamefont
  {Strogatz}},\ }\bibfield  {title} {\bibinfo {title} {Kuramoto model of
  coupled oscillators with positive and negative coupling parameters: An
  example of conformist and contrarian oscillators},\ }\href
  {https://doi.org/10.1103/PhysRevLett.106.054102} {\bibfield  {journal}
  {\bibinfo  {journal} {Phys. Rev. Lett.}\ }\textbf {\bibinfo {volume} {106}},\
  \bibinfo {pages} {054102} (\bibinfo {year} {2011})}\BibitemShut {NoStop}%
\bibitem [{\citenamefont {Vlasov}\ \emph {et~al.}(2015)\citenamefont {Vlasov},
  \citenamefont {Pikovsky},\ and\ \citenamefont
  {Macau}}]{Vlasov-Pikovsky-Macau-15}%
  \BibitemOpen
  \bibfield  {author} {\bibinfo {author} {\bibfnamefont {V.}~\bibnamefont
  {Vlasov}}, \bibinfo {author} {\bibfnamefont {A.}~\bibnamefont {Pikovsky}},\
  and\ \bibinfo {author} {\bibfnamefont {E.~E.~N.}\ \bibnamefont {Macau}},\
  }\bibfield  {title} {\bibinfo {title} {Star-type oscillatory networks with
  generic kuramoto-type coupling: A model for {``Japanese drums synchrony''}},\
  }\href {https://doi.org/http://dx.doi.org/10.1063/1.4938400} {\bibfield
  {journal} {\bibinfo  {journal} {Chaos}\ }\textbf {\bibinfo {volume} {25}},\
  \bibinfo {eid} {123120} (\bibinfo {year} {2015})}\BibitemShut {NoStop}%
\bibitem [{\citenamefont {Daido}(1996)}]{Daido-96}%
  \BibitemOpen
  \bibfield  {author} {\bibinfo {author} {\bibfnamefont {H.}~\bibnamefont
  {Daido}},\ }\bibfield  {title} {\bibinfo {title} {Onset of cooperative
  entrainment in limit-cycle oscillators with uniform all-to-all interactions:
  {B}ifurcation of the order function},\ }\href@noop {} {\bibfield  {journal}
  {\bibinfo  {journal} {Physica D}\ }\textbf {\bibinfo {volume} {91}},\
  \bibinfo {pages} {24} (\bibinfo {year} {1996})}\BibitemShut {NoStop}%
\bibitem [{\citenamefont {Gardiner}(1996)}]{Gardiner-96}%
  \BibitemOpen
  \bibfield  {author} {\bibinfo {author} {\bibfnamefont {C.~W.}\ \bibnamefont
  {Gardiner}},\ }\href@noop {} {\emph {\bibinfo {title} {Handbook of Stochastic
  Methods}}}\ (\bibinfo  {publisher} {Springer},\ \bibinfo {address} {Berlin},\
  \bibinfo {year} {1996})\BibitemShut {NoStop}%
\bibitem [{\citenamefont {Calogero}(2012)}]{Calogero-12}%
  \BibitemOpen
  \bibfield  {author} {\bibinfo {author} {\bibfnamefont {S.}~\bibnamefont
  {Calogero}},\ }\bibfield  {title} {\bibinfo {title} {Exponential convergence
  to equilibrium for kinetic {F}okker-{P}lanck equations},\ }\href@noop {}
  {\bibfield  {journal} {\bibinfo  {journal} {Comm. Part. Diff. Eqs.}\ }\textbf
  {\bibinfo {volume} {37}},\ \bibinfo {pages} {1357} (\bibinfo {year}
  {2012})}\BibitemShut {NoStop}%
\bibitem [{\citenamefont {Kulik}(2018)}]{Kulik-18}%
  \BibitemOpen
  \bibfield  {author} {\bibinfo {author} {\bibfnamefont {A.}~\bibnamefont
  {Kulik}},\ }\href@noop {} {\emph {\bibinfo {title} {Ergodic behavior of
  {M}arkov processes}}}\ (\bibinfo  {publisher} {de Gruyter},\ \bibinfo
  {address} {Berlin/Boston},\ \bibinfo {year} {2018})\BibitemShut {NoStop}%
\bibitem [{KK()}]{KK}%
  \BibitemOpen
  \href@noop {} {}\bibinfo {note} {{K}. Khanin (private communication,
  2023)}\BibitemShut {NoStop}%
\bibitem [{\citenamefont {Earnshaw}\ and\ \citenamefont
  {Keener}(2010)}]{Earnshaw-Keener-10}%
  \BibitemOpen
  \bibfield  {author} {\bibinfo {author} {\bibfnamefont {B.~A.}\ \bibnamefont
  {Earnshaw}}\ and\ \bibinfo {author} {\bibfnamefont {J.~P.}\ \bibnamefont
  {Keener}},\ }\bibfield  {title} {\bibinfo {title} {Global asymptotic
  stability of solutions of nonautonomous master equations},\ }\href@noop {}
  {\bibfield  {journal} {\bibinfo  {journal} {SIAM Journal on Applied Dynamical
  Systems}\ }\textbf {\bibinfo {volume} {9}},\ \bibinfo {pages} {220} (\bibinfo
  {year} {2010})}\BibitemShut {NoStop}%
\bibitem [{\citenamefont {Ott}\ and\ \citenamefont
  {Antonsen}(2008)}]{Ott-Antonsen-08}%
  \BibitemOpen
  \bibfield  {author} {\bibinfo {author} {\bibfnamefont {E.}~\bibnamefont
  {Ott}}\ and\ \bibinfo {author} {\bibfnamefont {T.~M.}\ \bibnamefont
  {Antonsen}},\ }\bibfield  {title} {\bibinfo {title} {Low dimensional behavior
  of large systems of globally coupled oscillators},\ }\href@noop {} {\bibfield
   {journal} {\bibinfo  {journal} {CHAOS}\ }\textbf {\bibinfo {volume} {18}},\
  \bibinfo {pages} {037113} (\bibinfo {year} {2008})}\BibitemShut {NoStop}%
\bibitem [{\citenamefont {Watanabe}\ and\ \citenamefont
  {Strogatz}(1993)}]{Watanabe-Strogatz-93}%
  \BibitemOpen
  \bibfield  {author} {\bibinfo {author} {\bibfnamefont {S.}~\bibnamefont
  {Watanabe}}\ and\ \bibinfo {author} {\bibfnamefont {S.~H.}\ \bibnamefont
  {Strogatz}},\ }\bibfield  {title} {\bibinfo {title} {Integrability of a
  globally coupled oscillator array},\ }\href@noop {} {\bibfield  {journal}
  {\bibinfo  {journal} {Phys. Rev. Lett.}\ }\textbf {\bibinfo {volume} {70}},\
  \bibinfo {pages} {2391} (\bibinfo {year} {1993})}\BibitemShut {NoStop}%
\bibitem [{\citenamefont {Watanabe}\ and\ \citenamefont
  {Strogatz}(1994)}]{Watanabe-Strogatz-94}%
  \BibitemOpen
  \bibfield  {author} {\bibinfo {author} {\bibfnamefont {S.}~\bibnamefont
  {Watanabe}}\ and\ \bibinfo {author} {\bibfnamefont {S.~H.}\ \bibnamefont
  {Strogatz}},\ }\bibfield  {title} {\bibinfo {title} {Constants of motion for
  superconducting {J}osephson arrays},\ }\href@noop {} {\bibfield  {journal}
  {\bibinfo  {journal} {Physica D}\ }\textbf {\bibinfo {volume} {74}},\
  \bibinfo {pages} {197} (\bibinfo {year} {1994})}\BibitemShut {NoStop}%
\bibitem [{\citenamefont {Strogatz}(2000)}]{Strogatz-00}%
  \BibitemOpen
  \bibfield  {author} {\bibinfo {author} {\bibfnamefont {S.~H.}\ \bibnamefont
  {Strogatz}},\ }\bibfield  {title} {\bibinfo {title} {From {K}uramoto to
  {C}rawford: {E}xploring the onset of synchronization in populations of
  coupled oscillators},\ }\href@noop {} {\bibfield  {journal} {\bibinfo
  {journal} {Physica D}\ }\textbf {\bibinfo {volume} {143}},\ \bibinfo {pages}
  {1} (\bibinfo {year} {2000})}\BibitemShut {NoStop}%
\bibitem [{\citenamefont {Fernandez}\ \emph {et~al.}(2016)\citenamefont
  {Fernandez}, \citenamefont {G{\'e}rard-Varet},\ and\ \citenamefont
  {Giacomin}}]{fernandez2016landau}%
  \BibitemOpen
  \bibfield  {author} {\bibinfo {author} {\bibfnamefont {B.}~\bibnamefont
  {Fernandez}}, \bibinfo {author} {\bibfnamefont {D.}~\bibnamefont
  {G{\'e}rard-Varet}},\ and\ \bibinfo {author} {\bibfnamefont {G.}~\bibnamefont
  {Giacomin}},\ }\bibfield  {title} {\bibinfo {title} {Landau damping in the
  {K}uramoto model},\ }in\ \href@noop {} {\emph {\bibinfo {booktitle} {Annales
  Henri Poincar{\'e}}}},\ Vol.~\bibinfo {volume} {17}\ (\bibinfo {organization}
  {Springer},\ \bibinfo {year} {2016})\ pp.\ \bibinfo {pages}
  {1793--1823}\BibitemShut {NoStop}%
\bibitem [{\citenamefont {Dietert}(2016)}]{Dietert-16}%
  \BibitemOpen
  \bibfield  {author} {\bibinfo {author} {\bibfnamefont {H.}~\bibnamefont
  {Dietert}},\ }\bibfield  {title} {\bibinfo {title} {Stability and bifurcation
  for the kuramoto model},\ }\href@noop {} {\bibfield  {journal} {\bibinfo
  {journal} {Journal de Mathématiques Pures et Appliquées}\ }\textbf
  {\bibinfo {volume} {105}},\ \bibinfo {pages} {451 } (\bibinfo {year}
  {2016})}\BibitemShut {NoStop}%
\bibitem [{\citenamefont {Klinshov}\ \emph {et~al.}(2021)\citenamefont
  {Klinshov}, \citenamefont {Kirillov},\ and\ \citenamefont
  {Nekorkin}}]{Klinshov_etal-21}%
  \BibitemOpen
  \bibfield  {author} {\bibinfo {author} {\bibfnamefont {V.}~\bibnamefont
  {Klinshov}}, \bibinfo {author} {\bibfnamefont {S.}~\bibnamefont {Kirillov}},\
  and\ \bibinfo {author} {\bibfnamefont {V.}~\bibnamefont {Nekorkin}},\
  }\bibfield  {title} {\bibinfo {title} {Reduction of the collective dynamics
  for neural populations with realistic forms of heterogeneity},\ }\href@noop
  {} {\bibfield  {journal} {\bibinfo  {journal} {Phys. Rev. E}\ }\textbf
  {\bibinfo {volume} {103}},\ \bibinfo {pages} {L040302} (\bibinfo {year}
  {2021})}\BibitemShut {NoStop}%
\bibitem [{\citenamefont {Campa}(2022)}]{campa2022study}%
  \BibitemOpen
  \bibfield  {author} {\bibinfo {author} {\bibfnamefont {A.}~\bibnamefont
  {Campa}},\ }\bibfield  {title} {\bibinfo {title} {The study of the dynamics
  of the order parameter of coupled oscillators in the ott--antonsen scheme for
  generic frequency distributions},\ }\href@noop {} {\bibfield  {journal}
  {\bibinfo  {journal} {Chaos: An Interdisciplinary Journal of Nonlinear
  Science}\ }\textbf {\bibinfo {volume} {32}} (\bibinfo {year}
  {2022})}\BibitemShut {NoStop}%
\bibitem [{\citenamefont {Pyragas}\ and\ \citenamefont
  {Pyragas}(2022)}]{Pyragas-Pyragas-22}%
  \BibitemOpen
  \bibfield  {author} {\bibinfo {author} {\bibfnamefont {V.}~\bibnamefont
  {Pyragas}}\ and\ \bibinfo {author} {\bibfnamefont {K.}~\bibnamefont
  {Pyragas}},\ }\bibfield  {title} {\bibinfo {title} {Mean-field equations for
  neural populations with q-{G}aussian heterogeneities},\ }\href@noop {}
  {\bibfield  {journal} {\bibinfo  {journal} {Phys. Rev. E}\ }\textbf {\bibinfo
  {volume} {105}},\ \bibinfo {pages} {044402} (\bibinfo {year}
  {2022})}\BibitemShut {NoStop}%
\bibitem [{\citenamefont {Daido}(1995)}]{Daido-95}%
  \BibitemOpen
  \bibfield  {author} {\bibinfo {author} {\bibfnamefont {H.}~\bibnamefont
  {Daido}},\ }\bibfield  {title} {\bibinfo {title} {Multi-branch entrainment
  and multi-peaked order-functions in a phase model of limit-cycle oscillators
  with uniform all-to-all coupling},\ }\href@noop {} {\bibfield  {journal}
  {\bibinfo  {journal} {J. Phys. A: Math. Gen.}\ }\textbf {\bibinfo {volume}
  {28}},\ \bibinfo {pages} {L151} (\bibinfo {year} {1995})}\BibitemShut
  {NoStop}%
\bibitem [{\citenamefont {Komarov}\ and\ \citenamefont
  {Pikovsky}(2013)}]{Komarov-Pikovsky-13a}%
  \BibitemOpen
  \bibfield  {author} {\bibinfo {author} {\bibfnamefont {M.}~\bibnamefont
  {Komarov}}\ and\ \bibinfo {author} {\bibfnamefont {A.}~\bibnamefont
  {Pikovsky}},\ }\bibfield  {title} {\bibinfo {title} {Multiplicity of singular
  synchronous states in the {K}uramoto model of coupled oscillators},\
  }\href@noop {} {\bibfield  {journal} {\bibinfo  {journal} {Phys. Rev. Lett.}\
  }\textbf {\bibinfo {volume} {111}},\ \bibinfo {pages} {204101} (\bibinfo
  {year} {2013})}\BibitemShut {NoStop}%
\bibitem [{\citenamefont {Komarov}\ and\ \citenamefont
  {Pikovsky}(2014)}]{Komarov-Pikovsky-14}%
  \BibitemOpen
  \bibfield  {author} {\bibinfo {author} {\bibfnamefont {M.}~\bibnamefont
  {Komarov}}\ and\ \bibinfo {author} {\bibfnamefont {A.}~\bibnamefont
  {Pikovsky}},\ }\bibfield  {title} {\bibinfo {title} {The {K}uramoto model of
  coupled oscillators with a bi-harmonic coupling function},\ }\href@noop {}
  {\bibfield  {journal} {\bibinfo  {journal} {Physica D}\ }\textbf {\bibinfo
  {volume} {289}},\ \bibinfo {pages} {18} (\bibinfo {year} {2014})}\BibitemShut
  {NoStop}%
\bibitem [{\citenamefont {Tyulkina}\ \emph {et~al.}(2018)\citenamefont
  {Tyulkina}, \citenamefont {Goldobin}, \citenamefont {Klimenko},\ and\
  \citenamefont {Pikovsky}}]{Tyulkina_etal-18}%
  \BibitemOpen
  \bibfield  {author} {\bibinfo {author} {\bibfnamefont {I.~V.}\ \bibnamefont
  {Tyulkina}}, \bibinfo {author} {\bibfnamefont {D.~S.}\ \bibnamefont
  {Goldobin}}, \bibinfo {author} {\bibfnamefont {L.~S.}\ \bibnamefont
  {Klimenko}},\ and\ \bibinfo {author} {\bibfnamefont {A.}~\bibnamefont
  {Pikovsky}},\ }\bibfield  {title} {\bibinfo {title} {Dynamics of noisy
  oscillator populations beyond the {O}tt-{A}ntonsen ansatz},\ }\href@noop {}
  {\bibfield  {journal} {\bibinfo  {journal} {Phys. Rev. Lett.}\ }\textbf
  {\bibinfo {volume} {120}},\ \bibinfo {pages} {264101} (\bibinfo {year}
  {2018})}\BibitemShut {NoStop}%
\end{thebibliography}%


\end{document}
