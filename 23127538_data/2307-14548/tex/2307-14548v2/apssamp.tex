% ****** Start of file apssamp.tex ******
%
%   This file is part of the APS files in the REVTeX 4.2 distribution.
%   Version 4.2a of REVTeX, December 2014
%
%   Copyright (c) 2014 The American Physical Society.
%
%   See the REVTeX 4 README file for restrictions and more information.
%
% TeX'ing this file requires that you have AMS-LaTeX 2.0 installed
% as well as the rest of the prerequisites for REVTeX 4.2
%
% See the REVTeX 4 README file
% It also requires running BibTeX. The commands are as follows:
%
%  1)  latex apssamp.tex
%  2)  bibtex apssamp
%  3)  latex apssamp.tex
%  4)  latex apssamp.tex
%
\documentclass[%
 reprint,
%superscriptaddress,
%groupedaddress,
%unsortedaddress,
%runinaddress,
%frontmatterverbose, 
% preprint,
%preprintnumbers,
%nofootinbib,
%nobibnotes,
%bibnotes,
showkeys,showpacs,
 amsmath,amssymb,
 aps,
%pra,
%prb,
%rmp,
%prstab,
%prstper,
%floatfix,
]{revtex4-2}

\usepackage{graphicx}% Include figure files
\usepackage{dcolumn}% Align table columns on decimal point
\usepackage{bm}% bold math
\usepackage{mathtools}
\usepackage[dvipsnames]{xcolor}



\DeclareMathOperator{\erf}{erf}
\begin{document}

\preprint{APS/123-QED}

\title{Effect of social behaviors in the opinion dynamics $q$-voter model}

\author{Roni Muslim}
\email{roni.muslim@brin.go.id (Corresponding author)}
\affiliation{Research Center for Quantum Physics, National Research and Innovation Agency (BRIN), South Tangerang, 15314, Indonesia}%

\author{Henokh Lugo H.}%
\email{henokh.lugo@lecturer.itk.ac.id}
\affiliation{Information System, Institut Teknologi Kalimantan, Balikpapan, 76127, Indonesia
}%


\date{\today}

\begin{abstract}
Order-disorder phase transition is one of the interesting features in the opinion dynamics model of sociophysics. This phenomenon is caused by noise parameters, which can be treated as social responses such as anticonformity and independence in a social context. This paper explores the effect of anticonformity and independence in the occurrence of the order-disorder phase transition in the model. A probability $p$ represents anticonformity and independence, and the model is defined on the complete graph and two-dimensional (2D) square lattice. We also introduce a skepticism parameter represented by $s =\{0,1\}$, which describes a voter's propensity to behave independently and anticonformist. Our results analytically and numerically show that the model undergoes a continuous and discontinuous phase for all values of non-zero $s$ and at specific $q$-sized agent. No phase transition is observed in the model with independent voters on the 2D square lattice, and the model with anticonformist voters undergoes a continuous phase transition. We also analyze the effect of the probability $p$ and skepticism level $s$ on the outcome of the system's opinion. 
\end{abstract}
\keywords{Opinion dynamics, phase transition, scaling, universality class.}
% \pacs{42.50.Ex, 32.80.Wr, 32.80., 32.10.Fn}

\maketitle

%\tableofcontents


\section{\label{sec:Sec1}Introduction}
%-------------------------------------------------------------------------------
Over the past few decades, scholars have tried applying statistical physics's general concepts and rules to interdisciplinary fields such as socio-political, economics, biology, medicine, computer science, and many more. Particularly in its application in understanding socio-political phenomena, scientists have proposed several models of opinion dynamics that explain the rules of interaction between individuals that influence each other to make an agreement or vice versa. The basis for developing an opinion dynamics model does not have standard rules; it can be in direct observation of real social phenomena or can be the form of reasonable imagination. To read further how the views of scientists regarding the application of statistical physics to social phenomena can refer to Refs.~\cite{galam2012socio,castellano2009statistical,sen2014sociophysics,schweitzer2018sociophysics}.


In addition to their analysis of real socio-political phenomena, scientists have tried to understand further the physical phenomena that arise in the dynamics of opinion, such as the order-disorder phase transition in the Ising model. The order-disorder phase transition in the opinion dynamics model is often called the transition of the consensus-status quo at a critical point of a social parameter. The order-disorder phase transition is one of the interesting physical phenomena to study in opinion dynamics models. This phenomenon can explain the macroscopic changes in a system due to a complex interaction at the microscopic level. This phenomenon is usually caused by the presence of small changes in the fluctuation parameters of the system, such as the ferromagnetic-antiferromagnetic phenomenon due to temperature changes in the magnetic spin system, even in the opinion dynamics models of sociophysics \cite{holyst2000phase,li2012phase,mukherjee2016disorder,velasquez2017interacting,muslim2020phase,muslim2021phase,schawe2022higher,MUSLIM2022133379,MUSLIM2022128307}. In addition to phase changes due to small changes in noise parameters in the system; scientists have also observed scaling behavior and universality in opinion dynamics. Scaling can be interpreted as the collapse of data on a particular parameter, which we call parameter scaling. Likewise, the model's universality is defined by several critical parameters that make the macroscopic parameters of the model collapse. The universality class for the mean-field model corresponds to the model defined on the complete graph (fully-connected network) \cite{muslim2020phase, muslim2021phase, MUSLIM2022133379, MUSLIM2022128307}, as well as the opinion dynamics model which is defined on the two and three-dimensional \cite{calvelli2019phase}, according to the universality class of the two and three-dimensional Ising models \cite{stanley1971phase, campostrini2002critical}.

In the context of a social system, several social parameters such as conformity, anticonformity, and independence are tried to be applied to observe social phenomena that arise and are interpreted based on the principles of physics. In the social literature, these three social parameters are generally grouped into conformity and nonconformity \cite{willis1963two, willis1965conformity, hofstede2010}, extensively studied in the opinion dynamics model in recent years \cite{muslim2020phase, muslim2021phase, MUSLIM2022133379, MUSLIM2022128307,grabisch2020survey, baron2021consensus, lipiecki2022polarization}. As mentioned, these social parameters make the dynamic models richer in statistical physics features produced and more related to social phenomena, as mentioned in the previous study \cite{civitarese2021external}, how the external field (related to the independence behavior) and the behavior of voters' scepticism affect the opinion outcome.

This paper studies the opinion dynamics model discussed in the previous work \cite{civitarese2021external}, by adding a social anticonformist behavior represented by a probability $p$. Several studies consider a similar scenario as in Ref.~\cite{civitarese2021external}, namely external field as the influence of the mass media, which can affect the macroscopic state of the system, such as changes in consensus behavior \cite{candia2008mass, martins2010mass, crokidakis2012effects, azhari2022mass, muslim2023mass}. The concept of this external field is an analogy of the effect of the external field on the Ising model, where it can affect the critical point that makes the system undergoes an order-disorder phase transition \cite{ muslim2020phase, muslim2021phase,  MUSLIM2022133379, MUSLIM2022128307, nyczka2012phase, vieira2016phase, calvelli2019phase}. As mentioned by Civitarese~\cite{civitarese2021external}, the influence of this external field uniformly affects all agents, meaning that all agents will change their opinion when exposed to the effects of the external field, for example, changing from ``up'' to ``down" opinion (state) or vice versa. 

We follow the scenario conducted by Civitarese \cite{civitarese2021external}. However, as mentioned, this paper considers another social response called anticonformity and defines the model in a complete graph and two-dimensional square lattices to analyze the order-disorder phase transition. This parameter is not discussed in the previous study. We also analyze the effect of the probability anticonformity $p$ and the skepticism level $s$ on the outcome of the system opinion. In addition, we analyze other statistical properties such as the exit probability, scaling parameters, and model universality class, which are lacking in Ref.~\cite{civitarese2021external}. Finally, the structure of this paper is as follows: Section~\ref{sec:Sec1}: Introduction, Section~\ref{sec:Sec2}: Model and method, Section \ref{sec:Sec3}: Result and discussion, and Section \ref{sec:Sec4}: Final remark.


\section{\label{sec:Sec2} Model}
The nonlinear $q$-voter model was introduced by Castellano et al.~\cite{castellano2009nonlinear}, is an extended model from the original voter \cite{clifford1973model}. The nonlinear $q$-voter model based on the nonlinear form on the flip function $f(r,q) = r^q+\varepsilon \left[1-r^q-\left(1-r\right)^q\right]$  where $r$ is the fraction of disagreeing neighbors or the opposite of the voter's opinion, and $\varepsilon$ is the probability that the agent changes the opinion whenever the neighbor $q$ is not in a homogeneous state. We can easily check that either for $q = 1$ (any value of $\varepsilon$) or $q = 2, \varepsilon = 1/2$, the function $f(r,q)$ reduced to linear form, $f(r) = r$, as the original model. However, in fact, the nonlinear condition of the model can be obtained for $q \geq 2$ and $\varepsilon \neq 1/2$. In this paper, we extend the model by introducing an external field effect $p$ and $s$ and analyzing the macroscopic properties of the model.


The parameter $\varepsilon$ in the flip function $f(r,q)$ acts like noise and destroys the ordered state of the model, and it can be considered in different ways, for example, by relating them to social behaviors. In the social context, we borrow psychological terms called independence or anticonformity (contrarian), represented by the probability of the external field $p$. Anticonformity is a social behavior that attempts to contradict the opinion of a majority group. Independence can be defined as a social behavior resistant to influence, acting according to one's own will \cite{willis1963two, willis1965conformity}. Both of these behaviors belong to nonconformity behavior. Previous studies showed that the nonconformity behavior relates to the individualism index \cite{hofstede2001}. Although this nonconformity behavior is not directly related to the individualism index, in countries with a high individualism distance index, these two behaviors are commonly found \cite{hofstede2010, solomon2010consumer}. Further individualism index can be seen in \cite{hofstedeIDV}. 

As we stated previously, in this model, the independence and anticonformity behaviors are presented by the probability $p \in [0,1]$ and the parameter $s = \{0,1\}$ can be considered as a skepticism level, how likely is the agent to behave anticonformist or independent. In other words, skepticism means that voters find it difficult to believe the persuasion of others, with $s = 1$ representing the easiest the voters to behave nonconformist.

In the case of the two behaviors above, the model can be described as follows: (1) We generate the initial random state, where the density of fraction opinion and spin-down are equal. (2) In that random state, we randomly chose a group of voters who formed a committee ($q$-sized agent) and another voter who will follow the $q$-sized agent. We generate a random number $r$, and if $r$ is less than the probability independence $p$, the voter acts independently. With probability $s$, the voter is affected by the external field, and with probability $1/2$, the voters change their opinion. Therefore, the probability that the voters change their opinion from up to down or down to up is $ps/2$. Otherwise, the voter follows the standard rule of the $q$-voter model, i.e., with probability $(1-p)$, the voter adopts the $q$-sized agent whenever the $q$-sized agent is in a homogeneous state (conformity behavior). The illustration of the model is illustrated in \eqref{eq:illust_indep}.
\begin{equation}\label{eq:illust_indep}
    \begin{aligned}
        & \text{independence} & \cdots \quad  &\Uparrow \quad \rightarrow \quad \cdots \quad & \Downarrow \\
        & \text{conformity} &\, \uparrow \uparrow \uparrow \quad &\Downarrow \quad \rightarrow \quad \uparrow \uparrow \uparrow \quad & \Uparrow \\
    \end{aligned}
\end{equation}
where $\uparrow (\downarrow)$ stands for the state up (down) of the $q$-sized agent, while $\Uparrow (\Downarrow) $ stands for the state up (down) of the voter. The conformity model in \ref{eq:illust_indep} is the case for $q = 3$. 

In the anticonformity model, the voters change their opinion oppositely to the $q$-sized agent whenever the voter and the $q$-sized agent are homogeneous. In this case, the probability of a voter changing its state from up to down (down to up) is $psk^q\left[ps\left(1-k\right)^q\right]$. Otherwise, the voter follows the standard rule of the $q$-voter model (conformity behavior). The illustration of the model is exhibited in \eqref{eq:illust_anti}.
\begin{equation}\label{eq:illust_anti}
    \begin{aligned}
        & \text{anticonformity} &\, \uparrow \uparrow \uparrow  \quad  \Uparrow \quad \rightarrow \quad \uparrow \uparrow \uparrow \quad \Downarrow \\
        & \text{conformity} &\, \uparrow \uparrow \uparrow \quad \Downarrow \quad \rightarrow \quad \uparrow \uparrow \uparrow \quad \Uparrow \\
    \end{aligned}
\end{equation}
In the two-dimensional square lattices, the voter has four neighbor agents ($q=4$), similar to the Ising model in the two-dimensional square lattices lattice \cite{mccoy1973two}.

The nonlinear $q$-voter model with external field effect is considered in the previous work \cite{civitarese2021external} and is similar to the independence type in this paper. This paper explores the model analytically and numerically for two types of social interaction, namely, independence and anticonformity, represented by probability $p$. We study the model on the complete graph (mean-field approximation) and two-dimensional lattices, especially to analyze the critical point and the critical exponent of the model using finite-size analysis, which is defined as $m = \phi_{m}(x) N^{-\beta/\nu}, \chi= \phi_{\chi}(x) N^{\gamma/\nu}, p= p_{c}+c\,N^{-1/\nu},$ and $U = \phi_{U}(x)$ \cite{cardy1996scaling}, where $U = 1 - \langle m^4 \rangle/3\, \langle m^2 \rangle^2$ and $\chi = N [\langle m^2 \rangle - \langle m \rangle^2] $ are the fourth order Binder cumulant and susceptibility parameters, respectively \cite{binder2012monte}. The parameter $\phi_i(x)$ is the dimensionless scaling function that controls the finite-size scaling effect, where $x = (p-p_c) N^{1/\nu}$ and $i = \{m,\chi,U \}$. In addition, the order parameter $m$ (magnetization) can be computed using
\begin{equation}\label{eq:Eq3}
    m =  \dfrac{1}{N} \sum \sigma_i = \dfrac{N_{\uparrow}-N_{\downarrow}}{N_{\uparrow}+N_{\downarrow}},
\end{equation}
since each agent has two possible opinions $\sigma_i = \pm 1$, represented by Ising spin $\uparrow = +1$ and down $\downarrow = -1$.

In this model, we also analyze the effect of the skepticism $s$ on the outcome of probability anticonformity $p$, which is not discussed in the previous study~\cite{civitarese2021external}.

\section{\label{sec:Sec3} Result and Discussion}
\subsection{Mean-field approximation}
%%%%opinions%%%%%%%%%%%%%%%%%%%%%%%%%%%%%%%%%%%%%%%%%%%%%%%%%%%%%%%%%%%%%%%%%%%%%
The mean-field approximation postulates that the system is homogeneous and isotropic. In other words, all fluctuation in the system is neglected, implying that local and global concentration can be considered the same \cite{binney1992theory, amit2005field}. In addition, the system does not depend on the dimensions of space. At this point, we can describe the system's state using one parameter of the system, such as the fraction opinion $k$ in this model. Regarding graph structure, the graph's complexity is reduced in the mean-field regime, allowing us to analyze the system analytically more easily. Even though the complexity is reduced, we can observe statistical physics features such as continuous and discontinuous phase transitions, scaling, universality, relaxation time, and exit probability, which are parameters that have been studied frequently in recent years \cite{ MUSLIM2022133379,  MUSLIM2022128307, mili2022simple, rocha2021novel, li2020some, muslim2023mass}. 


\textit{\textbf{Model with independent agent}}- An independent agent acts independently to change their opinion in the population; the agents do not follow the neighboring agents ($q$-sized agent). We define a probability $p$, a parameter that controls the independent behavior of the agents, and another parameter $s$, which represents skepticism of the agents. In this model,  with probability $p$, agents act independently of other agents whenever agents influence the voter-agent, while with probability $s$, they act skeptically to the agents' opinions. With probability $1/2$ the voter-agent change its opinion $\pm S_i (t) = \mp S_i(t+1)$. Therefore,  the total probability that makes the agent-voter change its opinion is  $ps/2$. To describe the dynamics of the model, we follow the procedure in Ref.~\cite{civitarese2021external}, in defining the probability of fraction opinion increases $\varphi_{+}$ and decreases $\varphi_{+}$ and decreases $\varphi_{-}$ for $N_{\uparrow}/N \leq 1/2$ in finite system as:
\begin{align}
\varphi_{+} & = N_{\downarrow}\left[ \dfrac{\left(1-p\right) \prod_{j =1}^{q} \left(N_{\uparrow}-j+1\right)}{\prod_{j=1}^{q+1}\left(N-j+1\right)} +\dfrac{p\,s}{2\,N} \right], \label{eq:Eq4}\\
\varphi_{-} & = N_{\uparrow}\left[ \dfrac{\left(1-p\right) \prod_{j =1}^{q} \left(N_{\downarrow}-j+1\right)}{\prod_{j=1}^{q+1}\left(N-j+1\right)} +\dfrac{p}{N} \left(1-\dfrac{s}{2}\right) \right], \label{eq:Eq5}
\end{align}
and for $N_{\uparrow}/N > 1/2$:
\begin{align}
\varphi_{+} & = N_{\downarrow}\left[ \dfrac{\left(1-p\right) \prod_{j =1}^{q} \left(N_{\uparrow}-j+1\right)}{\prod_{j=1}^{q+1}\left(N-j+1\right)}+ \dfrac{p}{N} \left(1-\dfrac{s}{2}\right)\right], \label{eq:Eq6}\\
\varphi_{-} & = N_{\uparrow}\left[ \dfrac{\left(1-p\right) \prod_{j =1}^{q} \left(N_{\downarrow}-j+1\right)}{\prod_{j=1}^{q+1}\left(N-j+1\right)} +\dfrac{p\,s}{2\,N} \right] \label{eq:Eq7}.
\end{align}
Equations \eqref{eq:Eq4}-\eqref{eq:Eq7} describe the increasing (decreasing) fraction opinion density by $+1/N (-1/N)$ or remain constant during the dynamics process with probability $(1- \varphi_{+}-\varphi_{-})$ for a specific range of fraction opinion density $N_{\uparrow}/N$. To test the accuracy of the analytical result with the M-C result, we can expand Eqs.~\eqref{eq:Eq4}-\eqref{eq:Eq7} for any $N$. However, the mean-field results are more suitable for large $N \gg 1$. Therefore, we are more interested in examining Eqs.~\eqref{eq:Eq4}-\eqref{eq:Eq7} as:
\begin{align}
   &\varphi_+  =  
  \begin{dcases} \label{eq:Eq8}
    \left(1-k\right)\left[\left(1-p\right)k^q + \frac{ps}{2} \right] & \text{for} \,k \leq \dfrac{1}{2}, \\
  \left(1-k\right)\left[\left(1-p\right)k^q + p \left(1-\frac{s}{2}\right)\right] & \text{for} \,k > \dfrac{1}{2},
  \end{dcases} \\
   &\varphi_{-} =  
  \begin{dcases} \label{eq:Eq9}
    k\left[\left(1-p\right)\left(1-k\right)^q+ p \left(1-\frac{s}{2}\right)\right] & \text{for} \,k \leq \dfrac{1}{2}, \\
   k\left[\left(1-p\right)\left(1-k\right)^q+ \frac{ps}{2}\right] & \text{for} \,k > \dfrac{1}{2},
  \end{dcases} 
\end{align}
where $k =N_{\uparrow}/N$. It should be emphasized that because we are analyzing the model with the mean-field approach, it is more suitable when considering the system for a large population $N$ such in Eqs.~\eqref{eq:Eq8}-\eqref{eq:Eq9} instead of Eqs.\eqref{eq:Eq4}-\eqref{eq:Eq7}.



\textit{\textbf{Model with anticonformist agent}}- In this model,  with probability $p$, the agent-voter act as anticonformist, while probability $s$ acts skeptically to the opinion'  $q$-sized agent. When the $q$-sized agent shares the same opinion, the voter adopts the opposite opinion to the opinion'  $q$-sized agent $\pm, S_i(t) = \mp S_i(t+1)$. Similar to the model with independent voters, the probability of the fraction opinion up increases $ \varphi_{+} $ and $\varphi_{-} $ decreases (in the finite system for $N_{\uparrow}/N \leq 1/2$), respectively are:
\begin{align}
 \varphi_{+} & = \dfrac{\left(1-p \right) N_{\downarrow} \prod_{j = 1}^{q} \left(N_{\uparrow}-j+1 \right)}{\prod_{j = 1}^{q+1} \left(N-j+1 \right)} + \dfrac{p\,s \prod_{j = 1}^{q+1} \left(N_{\downarrow}-j+1 \right)}{\prod_{j = 1}^{q+1} \left(N-j+1 \right)}, \label{eq:Eq10}\\
 \varphi_{-} & = \dfrac{\left(1-p \right) N_{\uparrow} \prod_{j = 1}^{q} \left(N_{\downarrow}-j+1 \right)}{\prod_{j = 1}^{q+1} \left(N-j+1 \right)} + \dfrac{p\prod_{j = 1}^{q+1} \left(N_{\uparrow}-j+1 \right)}{\prod_{j = 1}^{q+1} \left(N-j+1 \right)},\label{eq:Eq11}
\end{align}
and for $N_{\uparrow}/N > 1/2$:
\begin{align}
 \varphi_{+} & = \dfrac{\left(1-p \right) N_{\downarrow} \prod_{j = 1}^{q} \left(N_{\uparrow}-j+1 \right)}{\prod_{j = 1}^{q+1} \left(N-j+1 \right)} + \dfrac{p\prod_{j = 1}^{q+1} \left(N_{\downarrow}-j+1 \right)}{\prod_{j = 1}^{q+1} \left(N-j+1 \right)}, \label{eq:Eq12} \\
 \varphi_{-} & = \dfrac{\left(1-p \right) N_{\uparrow} \prod_{j = 1}^{q} \left(N_{\downarrow}-j+1 \right)}{\prod_{j = 1}^{q+1} \left(N-j+1 \right)} + \dfrac{p\,s \prod_{j = 1}^{q+1}, \left(N_{\uparrow}-j+1 \right)}{\prod_{j = 1}^{q+1} \left(N-j+1 \right)}\label{eq:Eq13}.
\end{align}
Again for comparison to the numerical simulation, defined on the complete graph, we consider Eqs.~\eqref{eq:Eq10}-\eqref{eq:Eq13} for a large population size $N$. Therefore, Eqs.~\eqref{eq:Eq10}-\eqref{eq:Eq13} reduced to more simple forms as:
\begin{align}
   &\varphi_+  =  
  \begin{dcases} \label{eq:Eq14}
    \left(1-k\right)\left[\left(1-p\right)k^q + ps(1-k)^{q}\right] & \text{for} \, k \leq \dfrac{1}{2}, \\
      \left(1-k\right)\left[\left(1-p\right)k^q + p\left(1-k\right)^{q} \right] & \text{for} \, k > \dfrac{1}{2},
  \end{dcases} \\
   &\varphi_{-} =  
  \begin{dcases} \label{eq:Eq15}
    k\left[\left(1-p\right)\left(1-k\right)^q+ p\,k^{q}\right] & \text{for} \, k \leq \frac{1}{2}, \\
   k\left[\left(1-p\right)\left(1-k\right)^q+ p\,s\,k^{q}\right] & \text{for} \, k > \frac{1}{2}.
  \end{dcases} 
\end{align}
One can see in Eqs.~\eqref{eq:Eq8}-\eqref{eq:Eq9} and Eq.~\eqref{eq:Eq14}-\eqref{eq:Eq15}, the model reduces to the original $q$-voter model for $p = 0$ \cite{castellano2009nonlinear}. For the stationary condition of the fraction opinion, $k$ will be discussed in the next subsection.

%%%%%%%%%%%%%%%%%%%%%%%%% subsection %%%%%%%%%%%%%%%%%%%%%%%%%%%%%%%%%
\subsection{\label{subsec:Subsec.3.2}Time evolution and steady state}
The recursive formula of the fraction opinion $k$ at time $t$ can be driven from the discrete-time master equation, which can be written as:
\begin{equation}\label{eq:Eq16}
    \mathrm{P}(k,t+1) = \sum_{k'} \varphi(k' \to k) \mathrm{P}(k',t),
\end{equation}
where $\varphi(k'\to k) $ is a transition probability of fraction opinion from state $k'$ to $k$ at a single time $t$. By combining the equation of the  probability of finding fraction opinion $k$ at time $t$, $k(t) = \sum_{k}k\,\mathrm{P}(k,t)$ to Eq.~\eqref{eq:Eq16}, the recursive formula of the fraction opinion $k$ at a time step $t$ can be written as:
\begin{align} \label{eq:Eq17}
    k(t') = k(t) + \dfrac{1}{N} \left[\varphi_{+}(k)-\varphi_{-}(k) \right],
\end{align}
The functions $\varphi_{+}(k')$ and $\varphi_{-}(k')$ are the probability density of fraction opinion density increases and decreases which defined in Eqs.~\eqref{eq:Eq8}-\eqref{eq:Eq9} and Eq.~\eqref{eq:Eq14}-\eqref{eq:Eq15}. Note that the time step $\Delta t = 1/N$ since the total time for $N$ step is $1$. Therefore we can rewrite Eq.~\eqref{eq:Eq17} in continue version by taking limit $N \to \infty$, or $\Delta t \to 0$ as:
\begin{equation}\label{eq:Eq18}
    \dfrac{\mathrm{d}k(t)}{\mathrm{d}t} = \varphi_{+}(k)-\varphi_{-}(k).
\end{equation}
Equation Eq.~\eqref{eq:Eq17} is also called the rate equation of fraction opinion $k$, which describes the  \cite{hinrichsen2000non,krapivsky2010kinetic}.

The macroscopic behavior of the system can be analyzed easily by considering the equilibrium condition of Eq.~\eqref{eq:Eq18} $\mathrm{d}k(t)/\mathrm{d}t$ = 0, implies $\varphi_+(k) = \varphi_-(k)$. Therefore, from Eqs.~\eqref{eq:Eq8}-\eqref{eq:Eq9} and \eqref{eq:Eq18} the probability of independence $p$ at a stationary state is:
\begin{align}\label{eq:Eq19}
   p  =  
  \begin{dcases} 
    \dfrac{k\left(1-k\right)^q-k^q\left(1-k\right)}{k\left(1-k\right)^q-k^q\left(1-k\right)-k+s/2} & \text{for} \, k \leq \dfrac{1}{2}, \\
  \dfrac{k\left(1-k\right)^q-k^q\left(1-k\right)}{k\left(1-k\right)^q-k^q\left(1-k\right)-k+1-s/2} & \text{for} \, k > \dfrac{1}{2}.
  \end{dcases}
\end{align}
Eq.~\eqref{eq:Eq19} describes the relation of the probability $p$ and skepticism $s$ for stationary state for specific ranges of $k > 1/2$ and $k \leq 1/2$ and to be the same for $s = 1$. The transition of skepticism $s$ decreases non-smoothly at low $p$ and a large $q$-sized agent and decreases smoothly for small $q$ and all values of $q$-sized agent, with $k$ increases as $s$ and $p$ increase ~\cite{civitarese2021external}.

We can also observe the order-disorder phase transition of the model from the equilibrium condition of Eqs.~\eqref{eq:Eq8}-\eqref{eq:Eq9}. Thus, the independence probability $p$ at the equilibrium state can be written as:
\begin{equation} \label{eq:Eq20}
    p = \left[\dfrac{1}{2}\left(s\left(1-2\,k\right)\left[k\left(1-k\right)^q-k^q\left(1-k\right)\right]\right)+1\right]^{-1}
\end{equation}
Note that Eqs.~\eqref{eq:Eq19} and \eqref{eq:Eq20} are slightly different, where Eq.~\eqref{eq:Eq19} describes the order parameter for a specific range of $k$, while Eq.~\eqref{eq:Eq20} describes the order parameter for all value of $k = \{0,1\}$. The probability $p_c$ in Eq.~\eqref{eq:Eq20} is the critical independence that makes the system undergoes an order-disorder phase transition. In Ref.~\cite{nyczka2012phase}, authors have shown that for $s=1$, the model undergoes a discontinuous phase transition for $q > 5$ and continuous phase transition for $q \leq 5$. In this model, we also show that model undergoes a discontinuous phase transition for $q > 5$ and a continuous phase transition for $q \leq 5$ for all values of $s \neq 0$. Although responsible for the occurrence of the phase transition, the parameter $s$ has no impact on changing the phase-transition kind; namely cannot change from continuous to discontinuous phase transitions, and vice versa, but only shifts the critical point of the system.
Panel (a) in Fig.~\ref{fig:order_indep1} shows the plot of Eq.~\eqref{eq:Eq20} for several values of $q$-sized agent and $s = 1/2$. One can see the model undergoes a continuous phase transition for $q \leq 5$ and discontinuous for $q > 5$. Based on phenomena, in the social context, one can say that at $p < p_c$, the system is in the consensus (there is an agreement), the complete consensus occurred when there is no independent agent $p = 0$, and the consensus decreases as $p$ increases until at the critical point $p_c$, and at $p \geq p_c$ the system is in a status-quo or stalemate situation. As we can see that the skepticism $s$ also controls the critical point $p_c$ since for $s = 1$, the model is reduced to the model introduced in the previous study \cite{nyczka2012phase}.

% Figure environment removed

Similar to the model with independent voters, we are interested in analyzing the effect of the probability $p$ on the transition of the skepticism level $s$. Therefore by considering the steady state, $\varphi_{+}(k) = \varphi_{-}(k)$, the anticonformity probability at equilibrium state $p$ for specific range of $k$ can be written as:
\begin{align}\label{eq:Eq21}
   p  =  
  \begin{dcases} 
    \dfrac{k^q\left(1-k\right)-k\left(1-k\right)^q}{k^q-\left(1-k\right)^q\left[s\left(1-k\right)+k\right]} & \text{for} \, k \leq \dfrac{1}{2}, \\
  \dfrac{k\left(1-k\right)^q-k^q\left(1-k\right)}{k\left(1-k\right)^q+\left(1-k\right) \left(1-k^q\right)-s/2} & \text{for} \, k > \dfrac{1}{2}.
  \end{dcases}
\end{align}
Based on Eq.~\eqref{eq:Eq21}, the stationary concentration $k$ is affected by the probability $p$, skepticism level $s$, $q$-sized agent, and the initial concentration $k_0$. Similar to the for $k \leq 1/2$, the transition of skepticism $s$ is smooth both for $q = 3$ and $q = 9$ for all range $p$ (see Fig.~\ref{fig:order_map}), which indicates the existence of the continuous phase transition in this model. The effect of probability $p$ to the skepticism $s$ differs from the model with independent voters, showing the non-smoothness (discontinuous) transition of $s$ for $q = 9$ at low $p$ \cite{civitarese2021external}.
% Figure environment removed

In the same way as the model with independent voters, by considering the stationary condition of Eq.~\eqref{eq:Eq18}, the anticonformity probability $p$ at the equilibrium state, which makes the system undergoes an order-disorder phase transition when $k = 1/2$ can be written as:
\begin{equation} \label{eq:Eq22}
p = \left[s\left(\dfrac{k^{q+1}-\left(1-k\right)^{q+1}}{k^q\left(1-k\right)-k\left(1-k\right)^q}\right)+1\right]^{-1}
\end{equation}
Plot of Eq.~\eqref{eq:Eq22} for several values of $q$-sized agents and $s = 1/2$ is exhibited in Fig.~\ref{fig:order_indep1}, panel (b). One can see that the model undergoes a continuous phase transition for all values $q$-sized agents. As shown in Eq.~\eqref{eq:Eq22}, besides depending on the $q$-sized agent, the critical point $p_c$ also depends on the skepticism $s$, i.e., it decreases as $s$ increases. For a specific case $s = 1$, the model is reduced to the one introduced in Ref.~\cite{nyczka2012phase}. 

In general, Eqs.~\eqref{eq:Eq20} and \eqref{eq:Eq22} can be written as $m\sim [(p-p_c)/p_c]^{\beta}$ (after re-scaling $k \to (m+1)/2$), where $\beta = 1/2$ for all values of $q \neq 1$. For example, for the model with independent voters, for $q = 2$, we have $m (p,s) = \pm [((1+2s)p-1)/(1-p)]^{1/2}$, where the critical point is $p_c = 1/(1+2s) $, that is when $m = 0$. The $\beta =1/2$ is typical of the critical exponent of the mean-field Ising model for the magnetization that makes the best data collapse near the critical point $p_c$~\cite{gitterman2013phase}. We also find the same $\beta$ for the model with anticonformist voters based on Eq.~\eqref{eq:Eq22}. Other critical exponents corresponding to susceptibility $\chi$ and Binder cumulant $U$ will be obtained using M-C simulation in section the next section.

The time evolution of the fraction opinion $k$  can be analyzed from Eq.~\eqref{eq:Eq17} or Eq.~\eqref{eq:Eq18} for a large population version. One can see that, obtaining the exact solution of Eq.~\eqref{eq:Eq18} for $k$ in time step $t$, namely $\int \mathrm{d}k/[\varphi_{+}(k)-\varphi_{-}(k)] = \int \mathrm{d}t$, for any values of $q$-sized agent and $s$ will much need work. However, we can solve \eqref{eq:Eq17} numerically using for example, Runge-Kutta 4th order (R-K) \cite{pinder2018numerical}
. The comparison of the M-C simulation (data point) to the R-K method (dashed line) for the model with independent voters for typical values of $q$ and $p < p_c$ is exhibited in Fig.~\ref{fig:tim_evol_indep}. One can see that the M-C result agrees very well with the R-K method. Based on Fig.~\ref{fig:tim_evol_indep}, for $q = 3$ and $q =5$ the fraction opinion $k$ evolves to two stable states, while for $q = 7$ and $q = 9$ the fraction opinion $k$ evolves to three stable states, indicating the model undergoes a continuous phase transition for $q = 3, 5$ and discontinuous phase transition for $q = 7, 5$ as discussed on previous section [see Fig.~\ref{fig:order_indep1} panel (a)]. Fig.~\ref{fig:Fig4} shows the time evolution of the fraction opinion $k$ for the model with anticonformist voter. In this model, the fraction opinion $k$ evolves to two stable states for all $q$, indicating the model undergoes a continuous phase transition as discussed in the previous section [see Fig.~\eqref{fig:order_indep1} panel (b)].  The case of this stability can be analyzed from the effective potential and the probability of the density function of the system, which is discussed in the section.

% Figure environment removed

% Figure environment removed


\subsection{\label{sec:scaling_b} Scaling behavior}

Statistically, we can formulate the scaling parameter of the model that makes data collapse. We can follow the procedure introduced in Ref.~\cite{sznajd2011phase} to find the scaling parameter for several macroscopic parameters of the model, such as the order parameter $m$, susceptibility $\chi$, and Binder cumulant $U$. Firstly, we prepare the system's initial state in a disordered state, i.e., the fraction opinion $k = 1/2$. We randomly select a group of a $q$-sized agent and another agent (voter) interacting with each other based on the model. In this case, there are three rules followed by the voter. Rule (1): With probability $p$, the voter acts as a nonconformist (anticonformist or independent). Rule (2): With probability $1-p$, the voter acts as a conformist, and with probability $s$, the voters act skeptical. For the model with anticonformist voters, the voters will change their opinion whenever they have the same opinion as the $q$-sized agent (homogeneous state). For the model with an independent voter, with probability $1/2$, the voter changes their opinion. Thus the total probability of the voters changing their opinion is $ps$ for the anticonformist voter and $ps/2$ for the independent voter. Rule (3) with probability $p\,(1-s)$ and $p\,(1-s/2)$, the voters do not change their opinion, respectively. Therefore, rule (3) does not change the system's state. Then, based on these rules, the scaling parameter depends on the probability of nonconformist $p$ and skepticism $s$.

During the dynamic process, the voters follow one of the three rules above and make the sequence states where the state depends on the initial condition and the probability rules (1) and (2) with ratio $r = \text{probability rule (2)}/\text{probability rule (1)}$, where probability rule (1) and $(2)$ generally depends on the model. We can say that if we have two systems with the same ratio $r$, they are statistically the same. To compare the two systems, we need to obtain the average time during the dynamic process as a function of probability rules (1) and (2). The average time $\tau$ of two systems is inversely proportional to the total probability that makes the voters change their opinion \cite{sznajd2011phase}.
If two systems have the same ratio probability $r$ and state at an instant time $t/\tau$, then the macroscopic parameters of the systems will be the same, namely $m_1= m_2, \chi_1 = \chi_2,$ and $U_1 = U_2$. Therefore, by plotting
\begin{align}\label{eq:Eq23}
    m, \chi, U \quad \text{versus} \quad \frac{\text{prob. rule (2)}}{\text{total prob.}}
\end{align}
the data will be collapsed. Theoretically, these scaling parameters work for any system which statistically follows the rules (1), (2), and (3). This paper will test the model on the complete graph and two-dimensional square lattice discussed in the next section.

\subsection{\label{subsec:pot} Landau potential}
The Landau theory of free energy can be used to analyze an equilibrium phase transition of a thermodynamics system \cite{landau1937theory, plischke1994equilibrium}. In his hypothesis, Landau argued that the free energy could be expanded in power series near the critical point in terms of the order parameter. Landau potential also can be applied to analyze a nonequilibrium system such as the Langevin equation for two absorbing states using mean-field approximation \cite{al2005langevin,vazquez2008systems}. Generally, the Landau potential is not only described by thermodynamic parameters such as temperature, pressure, and volume but can also depend on the system's order parameters as in Eq.~\eqref{eq:Eq3}. Here, we use the Landau potential to analyze the order-disorder phase transition of the model. Based on the Landau theory, a potential $V$ can be written as $V (m) = \sum_{i=0}^{n}V_im^{i}$, where $m$ is defined as in Eq.~\eqref{eq:Eq3}, and $V_i$  in general as a function of the macroscopic parameters of the system such as the probability $p$, and the skepticism $s$. The potential $V(m)$ is symmetric under inversion $m \to -m$; therefore, the odd terms will vanish.

To analyze the order-disorder phase transition of the model, we only need to consider at least two  even terms of the potential $V(m)$ as:
\begin{equation}\label{eq:Eq24}
    V(m) = V_1 m^2 + V_2 m^4,
\end{equation}
where $m$ is defined in Eq.~\eqref{eq:Eq3}, and $V_1, V_2$ depend of the model. The maximum-minimum conditions that describe the order-disorder phase transitions can be analyzed from the extremum conditions of Eq.~\eqref{eq:Eq24}.
In addition,  the continuous and discontinuous phase conditions can be analyzed by selecting the appropriate values of $V_1$ and $V_2$. For $V_1 = 0$, the phase transition conditions are continuous and discontinuous. At the critical point, the continuous and discontinuous phase occurs for $V_2 \geq 0$ and $V_2 < 0$, respectively.


To obtain the effective potential of the system, we will use the probability of fraction opinion increases and decreases as shown in Eqs.~\eqref{eq:Eq8} and \eqref{eq:Eq9} for models with the independent agent and Eqs.~\eqref{eq:Eq14} and \eqref{eq:Eq15} for models with anticonformist agent. In general, the effective potential of the system can be obtained as $V(k) = -\int f(k) \, \mathrm{d}k$, where $f(k) = \varphi_+(k) - \varphi_-(k)$ is called an effective force that drives the spins to flip \cite{nyczka2012opinion}.

The general solution of the effective potential $V$ for the model with independent voters is:
\begin{align}\label{eq:Eq25}
    V(k,q,p,s) = & \left(1-p\right)\Big[\dfrac{k^{q+2}}{q+2}-\dfrac{k^{q+1}}{q+1}\nonumber \\
    &-\left(1-k\right)^{q+1}\dfrac{\left(k\,q+k+1\right)}{\left(q+1\right)\left(q+2\right)}\Big] \nonumber \\
    & -k\left(1-k\right)\dfrac{p\,s}{2}.
\end{align}
The plot of Eq.~\eqref{eq:Eq25} for typical values of $q$ and probability $p$ is exhibited in Fig.~\ref{fig:pot_indep}. As seen for $q = 3$ and $ q = 5$, the model undergoes a continuous phase transition characterized by the bistable potential for $ p < p_c (q)$ (dashed line), and the potential is monostable for $p > p_c$. However, three stable states for $q  = 7$ and $ q = 9 $ indicate that the model undergoes a discontinuous phase transition. These potentials confirm the time evolution of the fraction opinion $k$ in Fig.~\ref{fig:tim_evol_indep}; that is, the fraction opinion $k$ evolves to two stable states for $q = 3$ and $q = 5$ and three stable states for $q = 7$ and $q = 9$.


% Figure environment removed

For the model with independent voters, the general solution of potential $V$ is:
\begin{align}\label{eq:Eq26}
     V(k,q,p,s) = & \dfrac{\left(1-p+ps\right)}{\left(q+2\right)}\left[k^{q+2}-\left(1-k\right)^{q+1} \right]\nonumber \\
    & -\dfrac{\left(1-p\right)}{\left(q^2+3q+2\right)}\left[k^{q+1}-\left(1-k\right)^{q+1}\right] \nonumber \\
    & +\dfrac{ps\left(1-k\right)}{\left(q+2\right)}.
\end{align}
The plot of Eq.~\eqref{eq:Eq26} can be seen in Fig.~\ref{fig:Fig6_pot_anti}. For all values of $q$, the potential $V$ has two stable states for $p < p_c$ and monostable for $p > p_c$, indicating the model undergoes a continuous phase transition. This potential also confirms the time evolution of the fraction opinion $k$ in the previous section (see Fig.~\ref{fig:Fig4}).

The effective potential in Eqs.~\eqref{eq:Eq25} and \eqref{eq:Eq26} maximum at $k = 1/2$, that is $\partial^2V(k,q,p,s)/\partial k^2|_{k = 1/2} < 0 $, with two bistable states, minimum at $k = 1/2$, that is $\partial^2V(k,q,p,s)/\partial k^2|_{k = 1/2} > 0 $, with one stable state, and the transition maximum-minimum at $k = 1/2$, that is $\partial^2V(k,q,p,s)/\partial k^2|_{k = 1/2} = 0 $. Therefore we can obtain the critical point of the system as follows:
\begin{equation}\label{eq:Eq27_pc}
    p_c = \dfrac{\left(q - 1\right)2^{1-q}}{\left(q - 1\right)2^{1-q} + s},
\end{equation}
for the model with independent voters, and
\begin{equation}\label{eq:Eq28_pc}
    p_c = \dfrac{q - 1}{q - 1 + \left(q+1\right)s},
\end{equation}
for the mode with anticonformist voters.

The potential $V$ in Eqs.~\eqref{eq:Eq25} and \eqref{eq:Eq26} are functions of the fraction opinion $k$. However, we can easily convert $V$ as a function of the order parameter $m$ by substituting $k = (m+1)/2$ into equations \eqref{eq:Eq25} and \eqref{eq:Eq26} to get a similar form as the Landau potential in Eq.~\eqref{eq:Eq24}. Therefore,  for the model with the anticonformist agents, we obtain
\begin{align}
    V_1(q,p,s) = & \dfrac{ps}{4}-\dfrac{\left(q-1\right)\left(1-p\right)}{2^{1+q}}\quad \Rightarrow \nonumber \\
    p_c = & \dfrac{\left(q - 1\right)2^{1-q}}{\left(q - 1\right)2^{1-q} + s} \quad (\text{for} \,V_1(q,p,s) = 0), \label{eq:Eq29_pot} \\
    V_2(q,p,s) = & -\dfrac{\left(1-p\right)q\left(q-1\right)\left(q-5\right)}{2^{q+1}} \quad \Rightarrow \nonumber \\
    V_2(q,s) = & -\frac{q\left(q -1\right) \left(q -5\right) s}{4 q -4+2^{q +1} s} \quad (\text{at $p = p_c$}) \label{eq:Eq30_pot}.
\end{align}
We have the same result as Eq.~\eqref{eq:Eq27_pc} for the critical point $p_c$ by setting $V_1(q,p,s) = 0$. The parameter $V_2 (q,s)$ is always positive for $q \leq 5$ (typically a continuous phase transition) and negative for $q > 5$ (typically a discontinuous phase transition) for all values of $s \neq 0$.
\begin{align}
    V_1 (q,p,s) = & \dfrac{\left(q+1\right)ps -\left(q-1\right)\left(1-p\right)}{2^{q+1}}\quad \Rightarrow \quad  \nonumber \\
    & \quad p_c = \dfrac{q - 1}{q - 1 + \left(q+1\right)s} \quad (V_1 = 0) \label{eq:Eq31_pc}\\
    V_2 (q,p,s) = & \dfrac{\left(q-1\right)q}{2^{1+q}} \left[\left(q+1\right)ps -\left(1-p\right) \left(q-5\right)\right] \quad \Rightarrow \nonumber \\
    V_2(q,s) = & \dfrac{qs \left(q^2-1\right) 2^{1-q}}{q-1+\left(q+1\right) s} > 0\quad \nonumber \\
    & \text{at $p =  p_c$ and for all $s \neq 0$ and $q \neq 1$.}
\end{align}
As seen in Eq.~\eqref{eq:Eq31_pc}, we also find the same for as $p_c$ in Eq.~\eqref{eq:Eq28_pc}. The parameter $V_2 (q,s)$ is always positive for all values of $s \neq 0 $ and $q \neq 1$, indicating the typical of the continuous phase transition for all  $s \neq 0 $ and $q \neq 1$.

% Figure environment removed

\subsection{Fokker-Planck description}
In the previous section, we analyzed the model's order-disorder phase transition by considering the equilibrium condition of $k$ and the effective potential $V(k)$. Here, we analyze the order-disorder phase transition of the model by analyzing the stationary of the probability density function, which can be obtained from the Fokker-Planck (F-P) equation below~\cite{frank2005nonlinear}:
\begin{equation}\label{eq:Eq27}
    \dfrac{\partial P(k,t)}{\partial t} =-\dfrac{\partial}{\partial k} \left[\xi_1(k)P(k,t) \right]+\dfrac{1}{2} \dfrac{\partial^2}{\partial k^2} \left[ \xi_2(k) P(k,t)\right],
\end{equation}
where $\xi_1 = \left(\varphi_+ -\varphi_- \right)$ and $\xi_2 = \left(\varphi_+ + \varphi_- \right)/N$ are the diffusion-like and drift-like coefficients which depend on the model. As we can see, Eq.~\eqref{eq:Eq27} is just a one-dimensional of the F-P equation since the model is defined on the complete graph. The diffusion-like $\xi_1$ and drift-like $\xi_2$ coefficients for the model with independent voters, respectively, are:
 \begin{align}
    \xi_1 & = \left(1-p\right) \left[\left(1-k\right)k^q-k\left(1-k\right)^q\right]+\frac{ps}{2} \left(1-2\,k \right) \label{eq:Eq28}\\
    \xi_2 & = \dfrac{\left(1-p\right)}{N} \left[\left(1-k\right)k^q+k\left(1-k\right)^q\right]+\frac{ps}{2N}\label{eq:Eq29},
\end{align}   
and for the model with anticonformist voters, the diffusion-like $\xi_1$ and drift-like $\xi_2$ coefficients, respectively, are:
 \begin{align}
    \xi_1 = &  \left(1 - k\right)\left[\left(1 - p\right)k^q + ps\left(1 - k\right)^q\right] \nonumber \\
    & - k\left[\left(1 - p\right)\left(1 - k\right)^q + psk^q\right] \label{eq:Eq30} \\
    \xi_2  = &   \dfrac{\left(1-p\right)}{N} \left[\left(1-k\right)k^q+k\left(1-k\right)^q\right]+\left[ k^{q+1}+\left(1-k\right)^{q+1}\right] \nonumber \\
    & \times \dfrac{\,p\,s}{N}. \label{eq:Eq31}
\end{align}

We only need the stationary condition of Eq.~\eqref{eq:Eq27} to analyze the order-disorder phase transition. By a little mathematical modification, the general solution of Eq.~\eqref{eq:Eq27} for the stationary condition can be written as:
\begin{equation}\label{eq:Eq32}
 P(k)_{st} = \dfrac{C}{\xi_2} \exp\left[\int 2\dfrac{\xi_1}{\xi_2}\mathrm{d}k \right],   
\end{equation}
where $C$ is a constant that satisfied the normalized condition of $P(k)$. One can see that obtaining the exact solution of Eq.~\eqref{eq:Eq32} for any values of the $q$-sized agent, probability $p$, and $s$ are difficult to carry out. However, we can solve numerically Eq.~\eqref{eq:Eq32} for typical values of $q,p$, and $s$. The plot of Eq.~\eqref{eq:Eq32} for the model with independent voters is exhibited in Fig.~\ref{fig:prob_indep}. One can see the typical of the discontinuous phase transition, namely $P(k)_{st}$ has three maxima corresponding to three stable states on the effective potential in Eq.~\eqref{eq:Eq25}. For low $p$, $P(k)_{st}$ has two maxima, and the maxima appearing in $k = 1/2$ is getting higher as $p$ increases. For high $p$, the probability density function $P(k)_{st}$ has only one maximum at $k$, indicating the system is stable at $k = 1/2$.


% Figure environment removed

The plot of Eq.~\eqref{eq:Eq32} for the model with anticonformist voters is exhibited in Fig.~\ref{fig:prob_anti}. In this model, the probability density function $P(k)_{st}$ has two maxima at low $p < p_c$ corresponding to two stable states in the effective potential in Eq.~\eqref{eq:Eq26}. The `peaks' of $P(k)_{st}$ are getting lower as $p$ increases and only one maximum for $p > p_c$ at $k =1/2$. This  $P(k)_{st}$ corresponds to the occurrence of a continuous phase transition.

% Figure environment removed

\subsection{Exit probability}
Exit probability is of the parameters that compute in social dynamics. Exit probability can be defined as the probability that the system is in a homogeneous state with one of two homogeneous states. In this paper, we consider the model to reach a homogeneous state with $m = 1$, namely the probability of all agents having the same state up in the end. The differential equation of the exit probability can be derived from the F-P equation as \cite{gardiner1985handbook}:
\begin{equation} \label{eq:Eq34}
    \xi_1 \dfrac{\partial E(k)}{\partial k} + \dfrac{\xi_2}{2} \dfrac{\partial^2 E(k)}{\partial k^2}=0,
\end{equation}
with boundary conditions $ E(0) = 0$ and $E(1) = 1$, since the probability of the system in opinion up is zero when $k = 0$ and one when $k = 1$. 

Theoretically, we can solve Eq.~\eqref{eq:Eq34} exactly but obtaining $E(k)$ in any values of $q, p,$ and $s$ can need much work. However, in this paper, we solve Eq.~\eqref{eq:Eq34} numerically using finite-different method \cite{pinder2018numerical}. As an illustration of the exact solution of Eq.~\eqref{eq:Eq34}, we can solve it easily for a simple case, for example, $q = 2$ and without anticonformist voter ($p = 0$), $E(k) = 
1/2\left[1+\mathrm{erf}\left(\sqrt{N/2}\, \left(2\,k-1\right)\right)/\mathrm{erf}\left(\sqrt{N/2}\right)\right]$.
One can check that the exit probability $E(k)$ will coincide in $k = 1/2$ for all values of $N$, indicating the existence of the separator point $k_c$, which separates two ordered states $m = 1 (k = 1)$ and $m = -1 (k = 0)$, with the same probability as shown in panel (a) of Figs.~\ref{fig:exit_indep}. It is easy to understand that for the case without the nonconformist voter ($p = 0$), all agents reach a consensus state depending on the state of the initial number of fractions opinion $k$. If the fraction opinion $k > 0\,( k < 0)$, then the system will reach a full consensus state with all agents having the same state or opinion up (down) \cite{muslim2023mass}.

% Figure environment removed

The plot of Eq.~\eqref{eq:Eq34} for typical values of $q, p \leq 0$ and $s = 1/2$ for the model with independent voters is shown in Figs.~\ref{fig:exit_indep} [panel (b)]. We also obtain that the exit probability of the model coincides at $k =  1/2$, indicating the separation of two consensus states with the same probability at $k_c = 1/2$. One can see that the effect of the probability $p$ does not affect the `symmetry form' of the exit probability $E(k)$ to the fraction opinion $k$ because the probability $p$ `acts' homogeneous in reaching the order-disorder phase transition of the model. Therefore the probability of the system reaching two consensus states are the same. The same result is obtained for the model with anticonformist voters (not shown). It can also be seen in Fig.~\ref{fig:order_indep1} that the fraction opinion $k$ is symmetrical to either the independence or anticonformity probability $p$.

We can also obtain the scaling parameters between the exit probability $E(k)$ versus fraction opinion $k$ using the standard finite-size scaling $E(k) = N^{-\delta_1}  \phi((k-k_c)\,N^{\delta_2})$, and $ k- k_c =  \delta_1 N^{-\delta_2}$ \cite{sousa2008effects}. As seen in the main panels of Fig.~\ref{fig:exit_indep}, the data are collapsed for $\delta_1 = 0 $ and $\delta_2 = 1/2$. These $\delta_1 = 0 $ and $\delta_2 = 1/2$ are universal; we obtain the same value for all values of N. These scaling parameters are the same as the majority and the $q$-voter models with mass media effect, defined on the complete graph \cite{muslim2023mass, azhari2022mass}.



\subsection{\label{sec:numeric} Scaling and critical exponents}
\subsubsection{The model on the complete graph}
 \textbf{\textit{Model with independent voters.}} We perform numerical simulation for the model with several population sizes $N$, a typical value of $q$-sized agent, and scepticism level $s = 0.5$. We variate population size $N$ to analyze the order parameter $m$, susceptibility $\chi$, and Binder cumulant $U$ and obtain the critical exponents $\beta, \nu, \gamma$ using the finite-size relations to define the universality class of the model. The result for $q = 2$ is exhibited in Fig.~\ref{fig:critical_mf_indep}. The model undergoes a continuous phase transition with the critical point $p_c$ obtained from the crossing lines of Binder cumulant $U$ versus probability $p$, which occurred at $p_c \approx 0.5$ [see panel (c)]. This result agrees to the analytical treatment in Eq.~\eqref{eq:Eq20}, which is $p_c(k,s,q) = 1/2$, where $k = 1/2, s = 1/2, q=2$. The inset graphs show the scaling plot near the critical point $p_c$, and all the data are collapsed for $\beta \approx 0.5, \nu \approx  2.0$ and $\gamma \approx 1.0$. These scaling parameters indicate the model's universality class is the same as the Sznajd model \cite{muslim2020phase, MUSLIM2022133379}, the majority rule \cite{MUSLIM2022128307}, the kinetic exchange model \cite{crokidakis2014phase,krapivsky2010kinetic}, as well as the mean-field Ising universality class. These scaling parameters work for all values of scepticism level $s \neq 0$ and $q$-sized agent $1 < q \leq 5$.

% Figure environment removed


 \textit{\textbf{Model with anticonformist voters}}. The simulation result of this model is exhibited in Fig.~\ref{fig:critical_mf_anti}. One can see that the critical point that makes the model undergo a continuous phase transition is $p_c \approx 0.40$ [see panel (c)]. Using the finite-size scaling relations, we also obtain the same $\gamma \approx 1.0, \beta \approx 0.5,$ and $\nu \approx 2.0$ as the model with independent voters. These critical exponents indicating the model is identical to the model with independent voters and belongs to the mean-field Ising universality class.

% Figure environment removed

Fig.~\ref{fig:variate_s_anti} shows the comparison between numerical simulation versus analytical treatment in Eq.~\eqref{eq:Eq22} for $q =  5$, several values of skepticism level $s$ and population size $N = 10^5$. It can be seen that the numerical simulation is in well-agreement to the analytical treatment in Eq.~\eqref{eq:Eq20} after re-scaling $k \to m=2k-1$ [see the biggest panel]. Three small panels show the scaling plot of the order parameter $m$, susceptibility $\chi$, and Binder cumulant $U$ based on Eq.~\eqref{eq:Eq23} and show the data collapse for all $s$.

% Figure environment removed


\subsubsection{The model on the 2D square lattice}
\textbf{\textit{Model with independent voters}}. In this model, we consider the voter $S_{i,j}$ has four neighbors and follows the microscopic rules as the model, namely the voter $S_{i,j}$ adopts the opinion of the four neighbors whenever they have the same opinion. Otherwise, the voter acts independently. We consider several population sizes $N = L \times L$, where $L$ is the width size lattice with periodic boundary conditions. The result for order parameter $m$ [panel (a)], susceptibility $\chi$ [panel (b)], and Binder cumulant $U$ is exhibited in Fig.~\eqref{fig:2D_indep}. One can see no crossing lines between Binder cumulant $U$ versus probability $p$ indicating the order-disorder phase transition is absent in this model.

% Figure environment removed

\textbf{\textit{Model with anticonformist voters}}. The simulation result of this model is exhibited in Fig.~\ref{fig:2D_anti}. It can be seen that the critical point that makes the model undergo a continuous phase transition is $p_c \approx 0.663$. Using the finite-size scaling analysis, the critical exponents that make the best collapse of all data $N$ are $\nu \approx 4.0, \beta \approx 0.7, \gamma \approx 2.7$ (see the inset graphs). As we can see, the critical exponents of the model are not the same as the critical exponents of the 2D Ising model. Therefore the model is not identical to the 2D Ising model. Although the model is defined on the 2D square lattice, it does not necessarily has the same universality class as the 2D Ising model.

% Figure environment removed

As mentioned in the previous section, Eq.~\eqref{eq:Eq23} works for all models that statistically follow the rules. Here, we perform numerical simulation for the model with the anticonformist agent for several values of skepticism level $s$ as shown in Fig.~\ref{fig:var_s}. The left panels are the normal plot, and the right panels are the scaling plot of the data. As we can see in the right panels, all data collapse and works for all values of $q \neq 1$.

% Figure environment removed


The effect of the skepticism decreases the critical point $p_c$ as $s$ increases through the relation $p_c = 1/(1+3s)$, which is obtained from Eq.~\eqref{eq:Eq20} for $k = 1/2, q = 2$. In general, the relation of the critical point $p_c$ to the skepticism parameter $s$ can be written as:
\begin{equation}\label{eq:Eq35}
    p_c = \dfrac{1}{1+c\,s},
\end{equation}
where $c$ is a constant that depends on the $q$-sized agent, namely, for the model with independent agents, we have $c = 2^{q-1}/(q-1)$ and for the model with anticonformist agents, we have $c = (q+1)/(q-1)$.

Fig.~\ref{fig:pc_vs_q} shows the behavior of the critical point over $q$-sized agent for several values of skepticism level $s$ based on Eqs.~\eqref{eq:Eq27_pc} and \eqref{eq:Eq28_pc} for the model with independent and anticonformist voter, respectively. For the model with independent voters, the point $p_c$ decreases as $q$ increases [panel (a)], while for the model with anticonformity [panel (b)], the critical point $p_c$ increases as $q$ increases. Both models are the same for $s = 0$; there is no order-disorder phase transition.

% Figure environment removed

\section{\label{sec:Sec4} Final Remark}
This paper considers the nonlinear $q$-voter model defined on a complete graph and a two-dimensional square lattice. We consider two social behaviors, namely anticonformity and independence and analyze the effect of both social behaviors on the macroscopic parameters of the model, such as order-disorder phase transition, universality class, scaling behavior, and exit probability. We also introduce a parameter $s = \{0,1\}$, which represents the skepticism level of the voter to be either independent or anticonformist. The social behaviors are represented by a probability $p$ that acts like a noise that makes the model undergoes an order-disorder phase transition. 

Based on our results for the model on the complete graph in both analytical and numerical simulation, the model with anticonformist voters undergoes a continuous phase transition for all values of $q \neq 1$ and skepticism level $s \neq 0$, while the model with independent voters undergoes a continuous phase transition for $2 \leq q < 6$ and a discontinuous phase transition for $q \geq 6$ for all values of $\neq 0$. The order-disorder phase transition is analyzed from the effective potential and the stationary probability density function of the fraction opinion $k$. We observe that the fraction opinion $k$ outcome increases as the probability of anticonformist $p$ and skepticism level $s$ increase. We also can see that the skepticism $s$ transition is smooth for all ranges of anticonformist probability $p$, in contrast to what has been reported for the model with independence in the previous study \cite{civitarese2021external}.

In the 2D model, we observe that the order-disorder phase transition is absent in the model with independent voters and undergoes a continuous phase transition for the model with anticonformist voters with the critical point at $p_c \approx 0.663$. Based on finite-size scaling analysis, we obtain the critical exponents that make the best collapse of all data, namely $\nu \approx 4.0, \beta \approx 0.7,$ and $\gamma \approx 2.7$.

For the model on the complete graph, we analyze the exit probability, the probability of the model is in the consensus state or equilibrium state. The separator point of the model is $k_s = 1/2$, a point that separates two equilibrium states at $ k < 1/2$ and $k > 1/2$. We obtain the scaling parameter $\delta_1 = 0$ and $\delta_2 = 1/2$ based on the standard finite-size scaling that makes the best collapse of all data.

\vspace{5pt}
% \section*{Author contribution}
%-------------------------------------------------------------------------------
\noindent \textbf{Author Contribution}. \textbf{R. Muslim:} Main contributor, Conceptualization, Methodology,  Investigation, Formal analysis, Software, Data Curation, Visualization, Writing -- original draft  
\textbf{H. Lugo H.:} Software, Formal analysis, Visualization, Validation, Writing -- review \& editing.

\vspace{5pt}

% \section*{Acknowlegment}
\noindent \textbf{Acknowlegment.} \textbf{R.~Muslim} thanks the Quantum Matter Theory Group of the BRIN Research Center for Quantum Physics for its mini HPC so that the numerical data in this paper can be obtained.


\vspace{5pt}
% \section*{Declaration of Interests}
%-------------------------------------------------------------------------------
\noindent \textbf{Declaration of Interests}. The authors declare that they have no known competing financial interests or personal relationships that could have appeared to influence the work reported in this paper.  



\bibliography{apssamp}% Produces the bibliography via BibTeX.

\end{document}
%
% ****** End of file apssamp.tex ******
