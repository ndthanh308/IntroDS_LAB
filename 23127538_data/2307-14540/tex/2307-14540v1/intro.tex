\nsection{Introduction} \label{sec:intro}

Recently, high-level Autonomous Driving (AD) vehicles~\cite{sae2021}, e.g., Level-4 ones, are gradually becoming part of the transportation system by providing commercial services such as self-driving taxis~\cite{baidu_driverless_robotaxi, waymo-one}, buses~\cite{baidu_bus_service, uk_bus}, and trucks~\cite{tusimple-truck, aurora_truck}. In particular, AD companies such as Waymo and Baidu are already offering commercial RoboTaxi services without safety drivers~\cite{baidu_driverless_robotaxi, waymo_driverless}, and more others are performing tests on public roads~\cite{cruise_driverless_testing, pony_driverless_testing}. To achieve high driving automation, the \textit{high-level AD system} (the ``\textit{brain}'') in such a vehicle needs to localize itself with \textit{centimeter-level} accuracy on the map~\cite{levinson2007map, reid2019localization, ega_requirement_report} to ensure safe and correct driving. Thus, today's industry-grade high-level AD systems predominantly adopt a Multi-Sensor Fusion (MSF) based localization design, which combines sensor inputs, typically GPS, LiDAR, and IMU, for overall higher accuracy and robustness in practice~\cite{wan2018robust, gao2015ins, soloviev2008tight, udacity_av_apollo, udacity_av_nd, coursera_av}. 

Due to the reliance on sensor inputs, AD localization is inherently vulnerable to sensor spoofing attacks, in particular GPS spoofing~\cite{fusionripper, spoof_tesla}, a long-existing security problem that is fundamentally difficult in both prevention and detection in practice~\cite{psiaki2016gnss, fusionripper}. Although the MSF-based design is generally more robust against such single-source sensor attacks, recent work~\cite{fusionripper} find that state-of-the-art MSF algorithms are still vulnerable to strategic GPS spoofing attacks due to non-deterministic and practical factors such as sensor noises and algorithm inaccuracies. To leverage such non-deterministic vulnerabilities, the authors devise a lateral-direction localization attack named \fr{} to \textit{opportunistically} inject lateral deviations in the MSF localization outputs, which will be translated into lateral deviations in the physical world by the AD control. Such lateral-direction localization attack is especially safety-critical in the AD context due to the potential consequences of road departure~\cite{road_depart_safety}. 




% Specifically, the attacker would keep tailgating a victim AD vehicle until a vulnerable period occur
% and waits for a vulnerable period to inject large lateral deviations to cause the victim to drive off road or onto a wrong way.

% that as long as the attacker can tailgate a victim AD vehicle for certain duration (e.g., 2 min), GPS spoofing alone can take-over the fusion process and cause the victim to drive off road or onto a wrong way with high attack success rates.


%a wide range of road hazards such as driving off road or onto the wrong way even if the perception module is functioning perfectly~\cite{fusionripper}.

%\alfred{why emphasize ``high-level'' here?} \junjie{I want to distinguish them with low-level ones since they don't need centimeter level accuracy.}

%Although MSF is generally believed to have the potential to defeat sensor attacks such as GPS spoofing~\cite{davidson2016controlling, lee2017attack, zeng2018all, cardenas2019cyber}, recent work~\cite{fusionripper} discovers a fundamental design-level vulnerability in the representative MSF localization algorithms in academia and industry that allows attacker to use GPS spoofing alone to cause large \textit{lateral deviations} in MSF outputs, i.e., deviating to left or right. Leveraging this, they design an attack named \fr{}, which can deviate the AD vehicle to drive off road or onto a wrong way with high attack success rates.

%may be potentially applicable to high-level AD systems

So far, no software-based defenses have been proposed to defend against such latest lateral-direction localization attack in high-level AD systems. The closed-related ones are the recent \textit{physical-invariant} based defenses for small robotic vehicles such as drones and rovers~\cite{savior, ci}. Such defenses estimate system states (i.e., vehicle positions) based on control commands and use them to validate GPS signals.
% that uses physical-invariants based defenses~\cite{ci, savior}, where they estimate system states (i.e., vehicle positions) based on the control commands and use them to validate the GPS signals.
% To defend against such attack, one possibility is a recently-proposed defense direction that applies physical-invariants to estimate system states (e.g., vehicle positions) and use them to cross-check the sensor inputs (e.g., GPS position)~\cite{ci, savior}. 
While these works show high effectiveness for small robotic vehicles, we find that they have limited effectiveness in the AD context (evaluated as a baseline in~\S\ref{sec:eval_detection}), because (1) vehicle driving motions in the real-world are more diverse and complex (e.g., commonly have high-speed or curvy-road driving), and thus harder to model accurately, and (2) attack deviation goals can be much smaller while still being safety-critical, e.g., even less than 0.5-meter lateral deviations can cause lane departure. 
%with obvious attack deviation goals (e.g., $\sim$50 meters~\cite{savior})
%However, in this work, we find that the existing physical-invariants used in these works, e.g., bicycle model~\cite{kong2015kinematic}, are not accurate enough to detect stealthy attacks such as \fr{} (\S\ref{sec:motivation_savior}).
Moreover, these works did not consider the \textit{attack response} step after detection, which is especially critical for high-level AD systems due to the complex driving environment and the absence of onboard human drivers. A few recent works considered such attack response designs, e.g., attack recovery upon attack detection~\cite{choi2020software, zhang2020real}. However, they rely on similar state estimation models as above to replace the attacked sensors during the recovery period, which thus suffer from the same motion model accuracy limitations in AD context, and also counted on human operators to take over as soon as possible since such state estimations cannot replace physical sensors for a prolonged duration due to drifting~\cite{choi2020software}. Last but not least, they assume an effective attack detection in place, which does not yet exist in for high-level AD localization.

%the assumption that a human operator can exist to be ready to take over upon attack detection. 
%emergency operators. 
%Moreover, existing attack recovery works~\cite{choi2020software, zhang2020real} also suffer from the same problem as they also rely on similar state estimation models as replacement for the attacked sensors during the recovery period (typically 3--7 sec~\cite{eriksson2017takeover, gold2013take}) before taken over by the emergency operator, which high-level AD vehicles do not possess.


Compared to small robotic vehicles, the AD context may also have its unique defense opportunity for lateral-direction localization attacks, for example \textit{Lane Detection (LD)}~\cite{hillel2014recent, pan2018spatial}, which is directly related to the attack goals since it can measure the vehicle's physical lateral deviation in the ego lane in the real time. Today, LD is already widely used in low-level AD localization (e.g., for automated lane centering). However, due to its fundamental limit in achieving effective global localization (\S\ref{sec:opportunity}), it is currently \textit{not} used for high-level AD localization purposes. While less suitable for accuracy purposes, so far no prior works have explored its potential for \textit{defense purposes} in high-level AD localization.

%\textit{However, no prior works have explored LD for defense purposes beyond its original usages, e.g., automated lane centering in low-level AD systems.}

In this work, we thus perform the first concrete exploration of LD as a domain-specific defense opportunity for lateral-direction localization attacks in high-level AD systems. We start by systematically analyzing its high-level defense potential,
%qualitatively and quantitatively, 
and find that such an LD-based defense strategy has various design-level benefits such as generality to lateral-direction attacks, technology maturity, direct deployability, and independence to existing attacks. One potential downside is the lack of defense capability when lane line markings are not available (e.g., in intersections), but since latest lateral-direction attack design is fundamentally opportunistic, the attacker \textit{cannot} deterministically control the triggering of a desired deviation only at road regions without lane line markings. In fact, we find that a defense coverage of the road regions with lane line marking can already provide protection for \textit{99.2\%} of such opportunistic attack attempts (\S\ref{sec:opportunity}).

%although LD requires the existence of lane line markings, which are not always available on local roads, we quantitatively analyze the effectiveness of existing lateral-direction localization attack on regions without lane line markings and discover that LD-based defense can provide defense coverage for majority of the attack attempts.


%First, as a mature technology widely adopted in low-level AD systems, today's LD algorithms and camera features greatly improves the robustness of LD at challenging environmental conditions. Second, LD is highly deployable in high-level AD systems benefit from the abundance of open-source models and the wide-availability of camera sensors. Third, although LD requires the existence of lane line markings, which are not always available on local roads, we quantitatively analyze the effectiveness of existing lateral-direction localization attack on regions without lane line markings and discover that LD-based defense can provide defense coverage for majority of the attack attempts. Fourth, we analyze the dependency between MSF localization and LD, and find that they are largely independent, which makes LD a desired defense source for MSF localization attacks.


% In this paper, we are the first to design an effective real-time defense method against lateral-direction localization attacks on high-level AD systems. To overcome the above limitations of existing methods in our problem settings, we need to address 3 unique design challenges: (C1) Need to identify \textit{defense-suitable} information sources to assist attack detection and response. High-level AD vehicles are equipped with abundant sensors, which can potentially provide information sources that are beneficial for defense purpose. Among them, we need to find the ones that are independent of the already-used ones (to avoid being attacked simultaneously) and also reliable when applicable. (C2) Need systematic \textit{attack response} designs that consider the domain-specific challenges for high-level AD systems. For example, a proper emergency trajectory should be executed in the attack response period as high-level AD systems do not have human drivers to hand over to by design~\cite{baidu_driverless_robotaxi, waymo_driverless}. (C3) Hard to establish trust when information sources are conflicting during attack response. Due to the limitations of existing attack detection methods, it may not be able to tell which sensor is under attack but instead only knows whether the information sources are conflicting. Thus, a systematic trust assignment scheme is necessary to limit the impacts from the attacked sources as much as possible during the attack response period.

% To address C1, we identify \textit{lane boundaries} as a defense-suitable information source (\S\ref{sec:discovery}) for lateral-direction localization attacks. This information can provide lateral deviations with respect to lane departure, which is exactly the attack goal. Meanwhile, it is generally available during driving and can be reliably obtained via mature technologies such as camera-based lane detection~\cite{hillel2014recent, openpilot, pan2018spatial}. We further experimentally find that it is independent to existing sources/features used in representative AD localization algorithms today, as the goal of these algorithms is \textit{global} positioning on the map in world coordinates and thus the lane boundary information is almost useless as they are not generally distinctive across different roads and regions on the map. Although not quite useful from the functionality perspective, in this paper we show that it can qualify for defense purposes for lateral-direction attacks.



% Figure environment removed 


Motivated by such multi-dimensional defense potentials, we design the first domain-specific LD-based defense approach called \ld{} (\underline{L}ane \underline{D}etection based \underline{L}ateral-\underline{D}irection \underline{L}ocalization attack \underline{D}efense), which is capable of both real-time attack \textit{detection} and \textit{response}. To use LD for attack detection, a tradeoff is that at which information level (i.e., GPS or MSF output) should we perform the detection. Recognizing that GPS outputs naturally have large noises and existing attack cannot deterministically predict when and where will large deviations occur in MSF, we decide to detect at the MSF output level to take advantage of such attack non-determinism.
%Since LD and MSF measure the vehicle's position in different coordinate systems (i.e., local vs global), we first convert the MSF outputs to lateral deviations to the lane centerline to make it comparable to the LD measurements. Specifically, a CUmulative SUM (CUSUM) anomaly detector is used to check the agreement between MSF and LD for attack detection (\S\ref{sec:design_detection}).
In the attack response (AR) stage, we choose to safely stop the vehicle in the ego lane, since this can minimize the attackable duration after detection and thus fundamentally bounds the attack-achievable deviation in the AR period. To account for the inherent LD-side adaptive attack surface introduced by \ld{}, we further design a novel safety-driven fusion between LD and MSF that systematically penalizes the source that is more aggressive in causing lateral deviations, which can fundamentally reduce the attacker's capability in causing safety damages in AR period even in adaptive settings. 

%. As AD vehicles may operate at high speeds, a reliable localization is necessary to steer the vehicle during the safe stopping, which may last for $\sim$7 seconds when on the highway (\S\ref{sec:eval_ar}). Since either MSF or LD could be attacked, we design a \textit{safety-driven fusion} to calculate a fused localization for the response. In particular, the fusion design penalizes the more aggressive source among MSF and LD in the AD driving context, to prevent the fused localization from being biased towards the attacked one (\S\ref{sec:design_ar}). With the fused localization, the AD planning and control then enforce an attack response trajectory, which follows the lane centerline, to safely stop the vehicle within the current lane.


We evaluate our defense against the latest lateral-direction localization attack on a diverse set of real-world sensor traces with various environmental conditions. 
% from the KAIST dataset~\cite{jeong2019complex} and also from driving our own vehicle with ADAS localization capabilities (global positioning and LD). 
Our results show that \ld{} is much more effective at detecting the attack compared to direct adaptation of physical-invariant based detection for small robotic vehicles~\cite{savior}. Specifically, \ld{} can achieve effective detection with 100\% true positive rates and 0\% false positive rates on the sensor traces and the detections are timely when the lateral deviations are not yet large enough to touch the lane boundaries. Moreover, \ld{} is also effective at keeping the AD vehicle within the lane during the attack response periods, where the vehicle's final stopping deviations are \textit{always} smaller than the lane straddling deviation. We also collect a night-time driving trace and find that \ld{} also has high defense robustness in low visibility conditions. 

%Our results show that \ld{} can achieve similar results as in the other traces, indicating a high defense robustness.

To further evaluate the defense in end-to-end driving with closed-loop control, we implement \ld{} on two open-source high-level AD systems, Baidu Apollo~\cite{apollo} and Autoware~\cite{autoware}, and evaluate in an industrial-grade AD simulator~\cite{lgsvl} and the physical world with a real vehicle-sized AD chassis. 
Our results show consistent results in end-to-end drivings as in the trace-based evaluations. Fig.~\ref{fig:pixkit_stop_positions} shows the vehicle driving trajectories and stopping positions in the physical world experiments. As shown, \ld{} can promptly detect the attack and safely stop the vehicle at the center of the lane, while without \ld{}, the vehicle drives out of lane boundary, and we have to manually stop the vehicle to prevent the collision.
The demo videos of the simulation and physical world experiments are available at
\textbf{\url{https://sites.google.com/view/cav-sec/LD3}}.

Lastly, we explore two potential adaptive attacks against \ld{}: (1) an ideal stealthy attack with the full knowledge of the defense aiming to evade the detection, and (2) the latest LD-side attack~\cite{sato2021dirty} against production AD systems. Results show that \ld{} can effectively bound the deviations of the stealthy attack from reaching the attack goals and can safely stop vehicle under the LD-side attack.

%\alfred{talk about stealthy attack?} \junjie{added, marked in red above.}

% \junjie{possible arguement: GPS is also not accurate, then why MSF use it? --> GPS has the benefit of not requiring map information, and not subject to map inaccuracy or outdateness --> good availability. But LiDAR and camera both requires maps; LiDAR is much more accurate and less environmentally restricted than camera. Thus MSF uses GPS \& LiDAR at this point.}

%Since prior works~\cite{kang2020lane, evlampev2020map} have explored global positioning methods using lane boundaries information, thus an alternative design is to use the LD-based global position as an additional fusion input for MSF localization to improve its robustness against lateral-direction attack. However, it is likely that adding such fusion input would degrade the MSF accuracy since their methods can only achieve 0.5m-level accuracy. In addition, such design cannot fundamentally prevent \fr{} since the vulnerability is due to sensor noises and algorithm inaccuracies~\cite{fusionripper}. In contrast, \ld{} is able to achieve perfect attack detection and can also steer the vehicle to safely stop within the ego lane after detection.

In summary, this work makes the following contributions:
\vspace{-\topsep}
\begin{itemize}
\setlength{\itemsep}{0pt}
\setlength{\parskip}{0pt}
    \item We perform the first systematic exploration of using LD to defend against lateral-direction localization attacks on high-level AD systems. We quantitatively show that LD can provide comprehensive defense coverage for existing attacks despite the reliance on lane line markings, and is independent of the AD localization inputs.
    \item We design \ld{}, a real-time defense solution including both attack detection and response stages. Evaluation on real-world sensor traces shows that \ld{} can achieve effective and timely attack detection, and can effectively stop the vehicle safely within the current lane. We also validate the robustness of \ld{} under low visibility conditions on a night-time driving trace.
    \item We implement \ld{} on two popular open-source AD systems, Baidu Apollo and Autoware, and evaluate the defense in end-to-end drivings in both simulation and the physical world.
    \item We evaluate \ld{} against two adaptive attacks and show that it is effective at bounding the deviations in the stealthy attack from reaching the attack goals and is robust to recent LD-side attack.
    % evaluate \ld{} on real-world sensor traces and end-to-end simulation environments. Results show that \ld{} can achieve perfect and timely attack detection, and can effectively steer the vehicle to stop within the ego lane. 
    % We also find that the timing overhead of \ld{} is negligible via evaluations on an embedded ADAS device.
\vspace{-\topsep}
\end{itemize}

% \item We evaluate the recent physical-invariant based defenses and find that they have limited effectiveness in the AD context due to inaccurate state estimation models.