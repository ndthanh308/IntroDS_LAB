% \vspace{-0.05in}
\nsection{Background and Threat Model} \label{sec:background}
% \vspace{0.1in}

\nsubsection{High-level AD Localization and MSF} \label{sec:background_msf}
\vspace{0.05in}

Today's high-level (e.g., Level-4~\cite{sae2021}) AD systems widely adopt a modular design with functional components such as localization, perception, prediction, planning, and control~\cite{apollo, autoware, udacity_av_apollo, udacity_av_nd, coursera_av}. Among them, localization is one of the most important modules that provides global positioning on the map for other modules such as planning and control to make safety-critical driving decisions. Since high-level AD systems need to navigate on the roads complete autonomously without any drivers, a localization with \textit{centimeter-level} accuracy is required to localize the AD vehicle on the traffic lane~\cite{levinson2007map, reid2019localization, ega_requirement_report}. 
High-level AD systems are typically equipped with various positioning sensors with diverse properties. For example, GPS provides global positioning with high availability, however, it often contains large positioning noises due to satellite signal transmission interferences and multi-path effect~\cite{gps_error_sources}; on the other hand, LiDAR localization algorithms (LiDAR locators) are able to accurately position the vehicle on a prebuilt LiDAR reflectance map using point cloud matching~\cite{wan2018robust, ndt, gao2015ins, levinson2010robust}. However, LiDAR locator performance can be severely degraded under adverse weather conditions or with an outdated LiDAR map. Thus, to achieve both high accuracy and robustness, high-level AD systems predominantly adopt a \textit{Multi-Sensor Fusion} (MSF) based localization design to \textit{leverage the strengths and compensate the weaknesses of different sensors} such as GPS, LiDAR, and IMU~\cite{wan2018robust, gao2015ins, soloviev2008tight, udacity_av_apollo, udacity_av_nd, coursera_av}.
%\vspace{-0.18cm}
\nsubsection{Lateral-Direction Localization Attack} \label{sec:background_msf_attack}
%\vspace{-0.1cm}
For AD localization, a direct threat is the attacks targeting the localization sensors such as GPS spoofing~\cite{zeng2018all, narain2018security, spoof_tesla, utaustin_russia_report, utaustinspoofer, kerns2014unmanned, popperccs11}, in which the attacker transmits fake satellite signals to the victim GPS receiver and thus cause it to resolve positions manipulated by the attacker. However, due to the high robustness provided by sensor fusion, MSF is often considered as a promising defense strategy for GPS spoofing~\cite{davidson2016controlling, lee2017attack, zeng2018all, cardenas2019cyber}. Contrary to the common belief, prior work~\cite{fusionripper} proposes an \textit{opportunistic} lateral-direction localization attack method, called \fr{}, which can use GPS spoofing alone to inject \textit{lateral deviations} in the MSF localization outputs and thus cause the AD vehicle to drive off-road or onto the wrong way. \fr{} is consist of two attack stages: \textit{vulnerability profiling} and \textit{aggressive spoofing}. In the vulnerability profiling stage, it spoofs the GPS inputs of MSF localization with a small constant distance $d$ (e.g., 0.5 m) in the lateral direction, waiting to discover a vulnerable attack window. Whenever the AD vehicle's physical deviation is larger than certain threshold (e.g., 0.3 m), \fr{} launches the aggressive spoofing stage, where a scaling factor $f$ (e.g., 1.2) will be continuously applied to the spoofing distance in each second to quickly introduce large lateral deviations in the MSF localization outputs. \fr{} has shown high attack effectiveness on the representative MSF algorithms, including the one in the industry-grade Baidu Apollo AD system~\cite{apollo}. To best of our knowledge, \fr{}~\cite{fusionripper} is the only localization attack that is able to defeat the MSF based localization algorithm in high-level AD systems.


\nsubsection{Threat Model} \label{sec:background_threat_model}

\textbf{Attacker's capability.}
In this work, we assume the attacker can launch practical lateral-direction localization attacks through external means such as GPS spoofing, which can cause lateral deviations in the localization outputs. Specifically, we focus on the lateral-direction attacks since such attacks (1) can cause the AD vehicle to violate the traffic norm that a vehicle should be driving within its designated lane boundaries and should not have unexpected lane straddling behaviors, and (2) pose a direct threat to the AD vehicle and road safety, e.g., it can cause the AD vehicle to drive off highway cliff or onto the wrong way and being hit by other vehicles that failed to yield in time. 

In particular, we do not consider simultaneous attacks that target both AD localization and lane detection at the same time, since such simultaneous attack neither already exists, nor can be easily achieved today (detailed discussions in \S\ref{sec:discuss}).
% due to the difficulty of attack synchronization and the non-deterministic nature of existing lateral-direction localization attack~\cite{fusionripper}. 


% We consider as out-of-scope for the attacks that can manipulate both the AD localization and perception outputs.

\textbf{AD control assumption.}
Same as \fr{}~\cite{fusionripper} and also as a common design in academia~\cite{paden2016survey} and industry~\cite{apollo, autoware}, we assume the AD systems are designed to drive at the center of traffic lane and constantly correct the deviations to the center. Since AD controllers constantly correct such deviations at a high frequency, e.g., 100 Hz~\cite{apollo}, the lateral deviations in the AD localization will thus be directly reflected as physical world deviations, but to the opposite direction.