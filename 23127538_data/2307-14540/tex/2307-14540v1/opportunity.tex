% \vspace{-0.05in}
\nsection{Lane Detection for High-Level AD Localization Defense: Opportunity Analysis} \label{sec:opportunity}

\textbf{Motivation and novelty.} Currently, no software-based defense solutions have been proposed to address the latest GPS spoofing-based lateral-direction localization attack in high-level AD systems (\S\ref{sec:background_msf_attack}). The closest ones are the recent physical-invariants based detectors proposed for small robotic vehicles such as drones and rovers, e.g., SAVIOR~\cite{savior} and CI~\cite{ci}, which estimate the physical dynamics of drones and rovers to validate the GPS signal. 
Although they show high effectiveness for such small robotic vehicles under large attack deviation goals, their effectiveness in AD vehicle context is fundamentally more limited since (1) existing vehicle dynamics models have difficulties in modelling high-speed and curvy-road settings~\cite{kong2015kinematic, polack2017kinematic}; and (2) in the AD context, the attack deviation goals can be much smaller (thus harder to detect) while still being safety-critical. As we concretely evaluate later in \S\ref{sec:eval_detection}, direct adaptation of such existing physical-invariant based approach to the AD context suffers from very high false positives and is actually close to random guessing.


In comparison to small robotic vehicles, the AD context may also have its unique defense opportunities for such lateral-direction localization attacks. \textit{Lane Detection (LD)}~\cite{hillel2014recent, pan2018spatial}, a technology commonly used in low-level AD systems for lane centering~\cite{openpilot, autopilot}, is such an example that can be used to measure the vehicle's lateral position within the current lane in real time, which is directly related to the lateral-direction attack goal (lane departure). Although effective in low-level AD systems (e.g., Level-2 ones such as Tesla Autopilot~\cite{autopilot} that still count on human drivers to take over anytime), \textit{LD is currently not used for high-level AD localization purpose (e.g., Level-4 ones such as Waymo that do not assume onboard human drivers)}. This is because what LD can provide is by nature only \textit{local} positioning (i.e., relative positioning within ego lane), while high-level AD requires \textit{global} positioning (i.e., in world coordinates on a map) for safe and correct driving decision-making without human drivers. Although there exist camera-based global localization methods using lane markings~\cite{kang2020lane, evlampev2020map}, they are not generally adopted in state-of-the-art high-level AD localization~\cite{wan2018robust, gao2015ins, soloviev2008tight, udacity_av_apollo, udacity_av_nd, coursera_av} as they are far from reaching the required centimeter-level accuracy~\cite{levinson2007map, reid2019localization, ega_requirement_report}.
%not only require extra efforts for lane-map generation~\cite{}, but also 
% for high-level AD~\cite{levinson2007map, reid2019localization, ega_requirement_report}.
% To the best of our knowledge, \textit{no prior works have used LD to defend against localization attacks;
% and state-of-the-art high-level AD systems do not use LD for localization~\cite{wan2018robust, autoware, gao2015ins, soloviev2008tight, udacity_av_apollo, udacity_av_nd, coursera_av}. 

%nd (2) high-level AD localization requires centimeter-level localization accuracy~\cite{levinson2007map, reid2019localization, ega_requirement_report} for safe and correct driving without human drivers, while LD can only provide less-accurate lateral positioning~\cite{} and is incapable of longitudinal localization. 

While less suitable for global localization accuracy purposes in high-level AD,
in this paper we propose to be the first to explore novel use of LD for \textit{defense purposes} in high-level AD localization. To concretely understand the potential of such a domain-specific defense opportunity, we analyze LD's defense properties in the following 5 general aspects.

%\alfred{include argument on LD is not used for high-level Ad localization and why? I remember it is a common reviewer question.} \junjie{added.}



% Although seemingly promising, LD also has limitations that may hinder its defense practicality. 

\newparts{

\textbf{1) General to lateral-direction localization attack.}
As mentioned above, LD can provide real-time information directly related to the \textit{attack goal} of lateral-direction localization attacks. Thus, LD by nature has the potential to provide general defense capabilities to not only the existing attack designs such as those in~\S\ref{sec:background_msf_attack}, but also their potential adaptive versions or other new attack designs in the future, as long as the attack goal is to cause lateral deviations. 

%As mentioned above, LD is directly related to the attack goal of lateral-direction localization attacks. Therefore, LD-based defenses are general to any GPS spoofing methods that aim to cause lane departure, among which \fr{}~\cite{fusionripper} is the state-of-the-art and is so far the only effective one for MSF-based localization in high-level AD systems.
%\alfred{talk about the generality over fusionripper} \junjie{added.}

\textbf{2) Technology maturity.}
Benefit from the growing prosperity of Deep Neural Networks (DNNs), LD is already a mature technology that has been used for lane centering in low-level AD systems and vehicles, e.g., OpenPilot~\cite{openpilot}, Tesla Autopilot~\cite{autopilot}, GM Cadillac, Honda
Accord, Toyota RAV4, Volvo XC90, etc.
In fact, the existing camera-based LD solutions are quite robust to the dynamic environmental conditions. For example, Tesla Autopilot can effectively recognize lane lines even during a night storm~\cite{autopilot_night_rain}. 
Apart from DNN advancement, the camera auto-exposure and vehicle headlights also improve the usability of LD. Later in \S\ref{sec:eval}, we also evaluate our defense on datasets with various environmental conditions and show that it is robust to low visibility conditions.
}

\newparts{
\textbf{3) Defense deployability.}
Since today's high-level AD vehicles are all equipped with cameras for road object detection, using them for an LD-based defense solution is thus readily deployable without the need to install any new hardware. Moreover, many state-of-the-art LD models are publicly available~\cite{pan2018spatial, neven2018towards}, including those used in industry-grade lane centering systems~\cite{openpilot}; some high-level AD systems are also using LD for camera calibrations~\cite{apollo}.}
% Therefore, deploying LD to high-level AD systems typically will not impose technical challenges.


% Information is generally available in AD context

\newparts{
\textbf{4) Defense coverage.}
For LD to be effective, lane line markings are required, which may not be available in local road segments such as intersections. Interestingly, due to real-world sensor noises and algorithm inaccuracies, the attacks to MSF localization are \textit{fundamentally opportunistic}. For example, despite having a high overall attack success rate, latest lateral-direction localization attack cannot predict when and where a large deviation can be injected to the MSF outputs~\cite{fusionripper}.
Due to such opportunistic property, the attacker \textit{cannot deterministically cause a desired lateral deviation to only appear in regions without lane line markings}. Such an attack property is fundamental to the MSF localization designs popularly used in high-level AD systems, since with this design the attack effectiveness is fundamentally dependent on sensor noises and algorithm inaccuracies of other sources, which are neither observable nor controllable by a tailgating attacker~\cite{fusionripper}.

Motivated by this insight, we analyze all attack traces evaluated in the \fr{} paper~\cite{fusionripper} and our own evaluation later (\S\ref{sec:eval_method}), and find that LD can indeed provide a decent practical defense coverage: among all attack starting points in the traces, only \textit{0.8\% (15/1813)} achieved the attack goal in road regions without lane line markings. Thus, an LD-based defense, if effective, can already provide protection for the 99.2\% of the possible attack attempts. In addition, autonomous trucks, which are an important high-level AD application, are generally not subject to such limitation since they mainly operate on the ``middle mile'' (i.e., highways)~\cite{tusimple_ad_truck_middle_mile, walmart_ad_truck_middle_mile}, where lane line markings are generally always available. 
}

% FusionRipper attack success rate in areas without lane line markings: 
% 0.83\% = 15 / 1813 (among all attack traces)
% 1.41\% = 15 / 1062 (among all local attack traces)

\newparts{
\textbf{5) Independence to existing localization attack.}
To defend against existing attacks, a desired defense property is that the lane line markings perceived by LD are not already used in MSF localization. This is because if such information is already used, existing attacks might have already exploited their vulnerable periods (e.g., natural detection inaccuracies), making the additional use of such information for defense less likely to be effective. 
%make the defense robust and therefore practically usable.
In representative MSF localization designs, LiDAR locator is the only one among MSF inputs (\S\ref{sec:background_msf_attack}) that is possible to utilize lane line markings as features. Thus, we perform an experimental analysis to understand the dependency between state-of-the-art LiDAR locators~\cite{wan2018robust, autoware} and lane line markings in Appendix~\ref{app:lidar_lane_line_dependency}. Our results show that today's LiDAR localization algorithms have a \textit{statistically-strong independence} of the lane line markings, very likely because lane markings is much less useful for global localization on a map compared to more unique road features such as buildings, roadside layouts, and traffic signs. This thus suggests that LD can indeed provide independent defense information to existing attacks. However, such independence property will disappear in adaptive attack settings (i.e., consider attacking LD after the defense is deployed). Thus, we require our defense design to be fully-aware of such adaptive attack surface (\S\ref{sec:design_overview}), and also evaluate it later (\S\ref{sec:adaptive_attacks}).

%Admittedly, even with perfect independence between LD and MSF, attacks that target both can potentially bypass the defense. However, as will be discussed in \S\ref{sec:discuss}, such a simultaneous attack neither already exists, nor can be easily achieved.
% due to the difficulty of attack synchronization and non-determinism of the existing lateral-direction localization attack~\cite{fusionripper}.
%\alfred{is there a separate more detailed discussion on this? If so please cross-ref (again, no repetitive logic). In such detailed discussion, we should emphasize how fundamental such a property is. } \junjie{now refering to the limitation discussion section on simultaneous attack to MSF and LD.}
}

% After doing this, a new challenge is if an attacker is aware of this source, whether they can bypass it? We have evaluation referring to stealthy attack evaluation

% one adaptive attack is FusionRipper and LD attack together
% but fusion ripper is non-determinism, hard to do it together


