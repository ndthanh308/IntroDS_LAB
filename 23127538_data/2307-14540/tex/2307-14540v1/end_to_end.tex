% \vspace{-0.05in}
\nsection{End-to-End Evaluations} \label{sec:end_to_end_eval} 
% \vspace{-0.05in}

% \newparts{
% In this section, we implement \ld{} on 2 open-source full-stack AD systems, Baidu Apollo~\cite{apollo} and Autoware~\cite{autoware}, and evaluate its defense capability in end-to-end driving with closed-loop control in an industry-grade high-level AD simulator and on a real vehicle-sized AD development chassis. The demo videos are available on our project website at 
% \textbf{\url{https://sites.google.com/view/ld3-defense}}.}


\newparts{
In this section, we implement \ld{} on 2 open-source full-stack AD systems, Baidu Apollo~\cite{apollo} and Autoware~\cite{autoware}, and evaluate \ld{} under end-to-end drivings in both simulation and the physical world. The demo videos are available on our project website at 
\textbf{\url{https://sites.google.com/view/cav-sec/LD3}}.}

% integrate \ld{} in an industry-grade high-level AD system and show the defense effectiveness in end-to-end simulation environments with the presence of AD control.

\vspace{-0.05in}
\nsubsection{Evaluation in AD Simulator}  \label{sec:simulation}

\cut{
\textbf{Experimental setup.}
We implement \ld{} in Baidu Apollo v5.0.0~\cite{apollo} and evaluate under 4 driving scenarios with different driving speeds (local road and highway speeds) and road geometries (straight and curvy roads) in the LGSVL simulator~\cite{lgsvl}.
% following the design in Fig.~\ref{fig:design_overview}. We run the complete Baidu Apollo AD system with all functional modules enabled in an industry-grade AD simulator, LGSVL~\cite{lgsvl}. 
% We evaluate the benign and attacked drivings with \ld{} in 4 driving scenarios with different driving speeds (local road and highway speeds) and road geometries (straight and curvy roads).
In our evaluation, we include the \ld{} variant with naive AR design (\S\ref{sec:eval_ar}) and one without defense.
We repeat the simulation for 10 times with different attack starting times for each combination of simulation scenarios and defense settings.
The detailed simulation setup is in Appendix~\ref{app:simulation_setup}.}


\textbf{Experimental setup.}
We implement \ld{} in Baidu Apollo v5.0.0~\cite{apollo} following the design in Fig.~\ref{fig:design_overview}. Specifically, we reuse the SCNN model~\cite{pan2018spatial} for LD, which is currently used only for camera calibration in Baidu Apollo. We run the complete Baidu Apollo AD system with all functional modules enabled in a production-grade AD simulator, LGSVL~\cite{lgsvl}. Since LGSVL does not provide LiDAR locator maps required for MSF, we instead run Baidu Apollo localization in the Real-Time Kinematic mode, which directly takes the ground truth positions from LGSVL. To simulate the \fr{} attack effect, we add the lateral deviations from the same attack trace used in \S\ref{sec:eval_visibility} to the localization outputs.

We evaluate the benign and attacked drivings with \ld{} in 4 driving scenarios on two LGSVL maps: Single Lane Road (SLR) and San Francisco (SF). Specifically, the SLR map is a long straight road, and we create a low-speed (SLR-Low) and high-speed (SLR-High) driving scenario on it by adjusting the maximum cruising speed in Apollo planning. The SF map is a 1:1 re-creation of a portion of the San Francisco city, from which we select a straight (SF-Straight) and a curvy road (SF-Curvy). In our evaluation, we also include the \ld{} variant with the naive AR design (\S\ref{sec:eval_ar}) and a setting without any defenses. We repeat the simulation for 10 times with different attack starting times for each combination of simulation scenarios and defense settings.


\textbf{Results and demos.}
Our simulation results show that the attack detection rates for both \ld{} and \ld{}-NaiveAR are all 100\% in the 10 runs, and none of the benign drivings are falsely detected as under attack. 
Table~\ref{tbl:sim_results} shows the maximum lateral deviation achieved in the whole simulation (including both attack detection and response periods) in each scenario/defense setting and the corresponding vehicle stopping location. As shown, with \ld{}, the average maximum deviations are smaller than lane straddling deviation in all 4 scenarios and the vehicle can always safely stop in the lane. In comparison, due to the blind trust of the localization outputs in the AR period, \ld{}-NaiveAR has much higher maximum deviations than \ld{} and the vehicle's stopping locations are either lane straddling or already crashing into the road curb/barrier. Nevertheless, the No Defense setting is even worse than \ld{}-NaiveAR, where the vehicle is simply deviated to fall off the road in SLR-Low and SLR-High.
Snapshots of the vehicle stopping locations in SF-Straight are shown in Fig.~\ref{fig:sim_snapshot}. 
The demos of the 4 simulation scenarios and 3 defense settings are available on our project website.

% \textbf{\url{https://sites.google.com/view/ld3-defense}}


\begin{table*}[tbp]
\footnotesize
\centering
\caption{Maximum deviations to lane center and attack consequences under different defense settings in the 4 simulation scenarios in \S\ref{sec:simulation}. Each setting was run for 10 times with randomized attack starting times. Benign driving with \ld{} is also presented and was run for 10 times. The maximum deviations are represented as (mean, std) in meters.}
\label{tbl:sim_results}
% \vspace{-0.1in}
\setlength{\tabcolsep}{4.5pt}
\begin{tabular}{@{}c|c|cccccc|cc@{}}
\toprule
\multirow{3}{*}{\begin{tabular}[c]{@{}c@{}}Simulation\\ scenario\end{tabular}} & \multirow{3}{*}{\begin{tabular}[c]{@{}c@{}}Lane\\ straddle\\ dev\end{tabular}} & \multicolumn{6}{c}{Attacked} & \multicolumn{2}{|c}{Benign} \\ \cmidrule(l){3-10} 
 &  & \multicolumn{2}{c}{\ld} & \multicolumn{2}{c}{\ld-NaiveAR} & \multicolumn{2}{c}{No Defense} & \multicolumn{2}{|c}{\ld} \\ \cmidrule(l){3-10} 
 &  & Max dev & Consequence & Max dev & Consequence & Max dev & Consequence & Max dev & Consequence \\ \midrule
SLR-Low & 0.83 & 0.47, 0.08 & Stop in lane & 1.69, 0.06 & Stop w/ lane straddle & 7.94, 0.05 & Fall off road & 0.07, 5e-5 & Reach destination \\
SLR-High & 0.83 & 0.69, 0.06 & Stop in lane & 1.64, 0.16 & Stop w/ lane straddle & 7.93, 0.04 & Fall off road & 0.07, 5e-5 & Reach destination \\
SF-Straight & 1.00 & 0.67, 0.23 & Stop in lane & 1.02, 0.01 & Hit curb & 1.84, 0.16 & Hit tree or barrier & 0.14, 7e-4 & Reach destination \\
SF-Curvy & 0.75 & 0.43, 0.14 & Stop in lane & 0.90, 0.12 & Hit lane divider & 0.97, 0.14 & Hit lane divider & 0.31, 0.01 & Reach destination \\ \bottomrule
\end{tabular}
% \vspace{-0.1in}
\end{table*}



% Figure environment removed

% \vspace{0.05in}
\nsubsection{\newparts{Evaluation on AD Development Chassis of Real Vehicle Size and Closed-loop Control}} \label{sec:pixkit_eval}
\vspace{0.03in}

% Although existing AD simulation technologies are already widely used in AD development for safety testing, it is still unclear whether the physics modeling is accurate enough such that the defense capability can indeed be faithfully translated to real-world driving. Therefore, we further perform another end-to-end evaluate of \ld{} on a real vehicle-sized AD development chassis with closed-loop control.
% \newparts{In this section, we evaluate \ld{} on a real vehicle-sized AD development chassis with closed-loop control to understand the defense capabilities in the real world.}

% Figure environment removed


\textbf{Experimental setup.} We experiment on an AD chassis as shown in Fig.~\ref{fig:pixkit_side_by_side}, which is specifically designed for Level-4 AD system prototyping and testing. The chassis is of a real vehicle size, capable of closed-loop control, and fully equipped with Level-4 AD sensors including LiDAR, GPS, IMU, cameras, RADARs, and ultrasonic sensors.
Since AD vehicle testing is not allowed to be on public roads by default, we reserve a parking lot in our institute for the experiments. Specifically, we mark a straight traffic lane with 3.5 m width (the most common lane width in KAIST dataset and our night-time driving trace) in the parking lot and create the corresponding semantic map for Autoware.


We ported \ld{} to the Autoware AD system~\cite{autoware}, which is currently supported by the AD chassis. To facilitate the attack, we apply the same \fr{} attack trace used in \S\ref{sec:eval_visibility} and \S\ref{sec:simulation} to the localization outputs in Autoware.
Unlike OpenPilot and Baidu Apollo, the lane detector in Autoware can only detect lane lines in pixels rather than in the world coordinates. Therefore, we directly obtain the ground truth lane line information from the map using the unmodified localization outputs, since LD is already a mature technology (\S\ref{sec:opportunity}) and has been shown to be quite accurate in \S\ref{sec:eval} and \S\ref{sec:simulation}.
We enable the relevant components in Autoware including localization, global/local plannings, and control. During the experiments, the AD chassis is completely driven by Autoware unless taken over by us from a remote controller in emergency situations. We evaluate three defense settings: (1) \textit{w/ \ldi{} w/ attack}, (2) \textit{w/o \ldi{} w/ attack}, and (3) \textit{w/ \ldi{} w/o attack}.
For each, we experiment in driving speeds of 2 m/s (4.5 mph) and 4 m/s (9 mph) for safety concerns. 
We prolong the AR stage by using deceleration $<$3 $m/s^2$ in both cases to better showcase the driving behaviors during AR.
Specifically, we repeat the experiments for 3 times for \textit{w/ \ldi{} w/ attack}. Since the other two are always quite stable, we thus do not record more iterations for those experiments.


\begin{table*}[tbp]
\footnotesize
\begin{minipage}{0.55\linewidth}
    \footnotesize
    \centering
    \caption{The detection, maximum, and stopping deviations in the three settings at two different driving speeds. We repeat the experiments for \textit{w/ \ld{} w/ attack} for 3 times and report the (mean, std) deviations. We do not repeat the other two settings as they are quite stable.}
    \label{tbl:pixkit_devs}
    % \vspace{-0.1in}
    \setlength{\tabcolsep}{1.2pt}
    \begin{tabular}{@{}c|cccc|cc@{}}
    \toprule
    \multirow{3}{*}{Speed} & \multicolumn{4}{c|}{w/ attack} & \multicolumn{2}{c}{w/o attack} \\ \cmidrule(l){2-7} 
     & \multicolumn{3}{c|}{w/ \ld{}} & w/o \ld{} & \multicolumn{2}{c}{w/ \ld{}} \\ \cmidrule(l){2-7} 
     & Det dev & Max dev & \multicolumn{1}{c|}{Stop dev} & Max/Stop dev & Max dev & Stop dev \\ \midrule
    4 m/s & 0.07m, 0.01m & 0.36m, 3e-3m & \multicolumn{1}{c|}{0.05m, 0.05m} & 2.59m & 0.13m & 8e-3m \\
    2 m/s & 0.02m, 2e-3m & 0.27m, 0.04m & \multicolumn{1}{c|}{0.01m, 1e-3m} & 2.23m & 0.11m & 7e-3m \\ \bottomrule
    \end{tabular}
\end{minipage}\hfill
\begin{minipage}{0.43\linewidth}
    \footnotesize
    \centering
    \caption{Maximum physical deviations can be achieved without being detected under various LD fluctuation assumptions. The percentages indicate the probabilities of such fluctuations.}
    \vspace{-0.05in}
    \label{tbl:stealthy_detection}
    % \setlength{\tabcolsep}{5pt}
    \setlength{\tabcolsep}{3pt}
    \begin{tabular}{@{}ccccc@{}}
    \toprule
    \multirow{2}{*}{Trace} & \multirow{2}{*}{\begin{tabular}[c]{@{}c@{}}LD fluctuation\\ ($\mu, \sigma$)\end{tabular}} & \multicolumn{3}{c}{Max physical world deviation} \\ \cmidrule(l){3-5} 
     &  & 0 (100\%) & $\mu$ (50\%) & $\mu+3\sigma$ (0.3\%) \\ \midrule
    \textit{ka-local31} & 0.12m, 0.08m & 0.7m & 0.82m & 1.06m \\
    \textit{ka-local33} & 0.14m, 0.10m & 0.7m & 0.84m & 1.14m \\
    \textit{ka-highway36} & 0.29m, 0.10m & 0.7m & 0.99m & 1.29m \\
    \textit{ka-highway18} & 0.20m, 0.11m & 0.7m & 0.90m & 1.23m \\ \bottomrule
    \end{tabular}
\end{minipage}
% \vspace{-0.22in}
\end{table*}

% 4 m/s:
% Detection dev (mean, std): 0.06493076049897435 0.012452686654347397
% Maximum dev (mean, std): 0.36426716650795504 0.0032264096321182583
% Stopping dev (mean, std): 0.04520309618108121 0.04720487829962962
% NoDefense Maximum dev: 2.5913297119226946
% NoDefense Stopping dev: 2.584318858073847
% Benign Maximum dev: 0.13370482650200852
% Benign Stopping dev: 0.00814336299953038

% 2 m/s:
% Detection dev (mean, std): 0.020136245340817385 0.0022321999365849127
% Maximum dev (mean, std): 0.2731832984655125 0.04088672062665852
% Stopping dev (mean, std): 0.0058209895419398605 0.0017981851847734464
% NoDefense Maximum dev: 2.2288639764466422
% NoDefense Stopping dev: 2.2078183052640297
% Benign Maximum dev: 0.10658935648516672
% Benign Stopping dev: 0.00735364768826751


\textbf{Results and demos.} 
Table~\ref{tbl:pixkit_devs} shows the detection, maximum, and stopping deviations under the three settings. As shown, \ld{} on average can detect the attack when the vehicle's physical deviation is still small and start the AR stage. Within the AR period, the average maximum deviations are 0.36 m and 0.27 m at speeds of 4 m/s and 2 m/s, respectively, and the final stopping deviations are always within 0.1 m. In comparison, without \ld{}, the vehicle keeps deviating and we have to manually press the emergency button on the remote to prevent it from crashing into the curb. 
Such a distinctive driving behaviors with and without \ld{} are consistent with our trace-based (\S\ref{sec:eval}) and simulation results (\S\ref{sec:simulation}).
Without the attack, the vehicle's trajectories well align with the road centerline (i.e., the reference trajectory Autoware plans to enforce) and eventually complete the route and stop at the center of the lane.
We also record demo videos of the vehicle driving behaviors under the three settings (videos are available on our website). As an illustration, Fig.~\ref{fig:pixkit_stop_positions} visualizes the driving trajectories in the bird's eye view and shows the snapshots of final stopping positions at driving speed of 4 m/s.

% \todo{put the PIXKIT evaluation here as a subsection.
% 2 m/s: maximum deviation in AR: 0.23 meters, final stop deviation: 0.09 meters
% 4 m/s: maximum deviation in AR: 0.37 meters, final stop deviation: 0.10 meters}

