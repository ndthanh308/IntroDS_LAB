% \vspace{-0.05in}
\nsection{Limitations Discussion} \label{sec:discuss}
% \vspace{0.05in}

\textbf{Defense coverage of lane detection.}
% As the first work to discover and leverage information sources available in high-level AD systems, we choose to use the lane boundary as a defense-suitable one for attack detection and response against lateral-direction localization attacks. 
In this work, we are the first to explore the novel usage of LD for defense. However, as a defense relying on LD, a potential limitation is the lane line marking coverage. However, as we analyzed in \S\ref{sec:design_overview}, the non-deterministic nature of attacks to MSF localization greatly alleviate such a limitation, where an LD-based defense has the potential to defend against the majority (99.2\%) of the attack attempts. In addition, for important AD applications such as autonomous trucks, they are naturally not subject to such limitation as they mostly operate on highways~\cite{tusimple_ad_truck_middle_mile, walmart_ad_truck_middle_mile}.
% However, the applicability of LD are limited in areas with lane line markings. 
At design level, since high-level AD systems come with semantic maps with accurate road geometry information, \ld{} knows exactly where are the regions without lane line markings and can temporarily disable the defense in such regions (\S\ref{sec:design_detection}).
To address this limitation, a potential future improvement is to also consider other road markings available in such regions, e.g., stop lines~\cite{lin2017stop} and crosswalk markings~\cite{bailo2017robust} in intersections, to help localize the vehicle and to detect MSF deviations. Nevertheless, it is unclear how prevalent such road markings are and how mature and robust the existing perception algorithms are to recognize such road markings.
%\alfred{talk about future improvement direction? if it is a pure argument, it is not necessary to repeat it here.} \junjie{added.}
% a good thing is that as a defense integrated in the AD system, \ld{} knows exactly where it is not applicable and can decide to temporally disable the attack detection (\S\ref{sec:design_detection}).
% In addition, \ld{} can already cover challenging scenarios such as highways, where the lane line markings are generally available, and the attack consequences can be more severe due to the high speeds.
% Nevertheless, a promising extension of this work is to also include information sources that are available in the intersections, e.g., stop lines, to help localize the vehicle and cross-check the MSF deviations.

% \todo{MSF is opportunistic; highway doesn’t have intersections and generally always have lane boundary markings.}

% \todo{mention the FusionRipper intersection attack success rate here: attack success rate:
% 0.83\% = 15 / 1813 (among all attack traces);
% 1.41\% = 15 / 1062 (among all local attack traces)}

% \todo{When talking about the number here, talk from victim’s point of view: Percentage of my driving is protected: 99.17\%.}

% Road markings as an independent source

\textbf{Simultaneous attacks to MSF and LD.}
Since \ld{} leverages LD to detection lateral-direction attack on MSF,
attacks that simultaneously target MSF and LD can thus potentially bypass our detection. 
In fact, such a vulnerability is a general limitation for CPS security research that uses sensor cross-checking/fusion for defense purposes~\cite{feng2018efficient, feng2017efficient, aguilar2017developing, tanil2017detecting, khanafseh2014gps, lee2015gps}. However, in practice, the defense value of \ld{} highly depends on \textit{whether such a simultaneous attack already exists or can be easily achieved}. For MSF and LD, neither of them holds today, since (1) although individual attacks on MSF or LD exist, no existing work shows that they can be effectively \textit{coordinated and synchronized} to achieve simultaneous attack effect control, and (2) it is far from trivial to achieve this with existing individual attack vectors. Specifically, among the attack vectors on camera~\cite{petit2015remote, yan2016can, nassi2020phantom, sayles2021invisible, kohler2021they, sato2021dirty, kang2020lane, jing2021too}, only three works~\cite{nassi2020phantom, sato2021dirty, jing2021too} actually evaluated and shown attack effectiveness on LD in realistic AD settings. All these three works consider adding malicious patterns to the ground (e.g., via road patch or stickers) as the attack vector. However, considering the non-deterministic nature of the existing high-level localization attacks (\S\ref{sec:opportunity}), it would be hard, if not impossible, for the attacker to figure out where to place the attack pattern beforehand, not to mention how to carefully synchronize the malicious pattern with the localization-side attack to effectively bypass \ld{}. Therefore, we consider such simultaneous attack design neither already exists nor can be easily achieved, and leave the systematic exploration of its feasibility as a future direction.

\textbf{Delay between attack and detection.} Another limitation is that our detection and response happen after the attack has occurred to some extent (i.e., some deviations have already been caused by the attack). Even though our system can greatly reduce the safety consequences and transition the vehicle into a minimal-risk condition, it is still better if we can detect the attack immediately after the first injection is sent to the system. We thus consider this as another future direction.

%it is quite non-trivial to simultaneously launch DRP with MSF attack~\cite{fusionripper} due to its non-deterministic nature (\S\ref{sec:opportunity})--it would be hard, if not impossible, for the attacker to figure out where to place the attack patch beforehand, not to mention how to carefully synchronize the dirty pattern with the MSF attack to effectively bypass \ld{}.




%Among them, only DRP attack~\cite{sato2021dirty} is applicable to us; the other two~\cite{nassi2020phantom, jing2021too} only show effectiveness on road regions without lane line markings (Jing et al.~\cite{jing2021too} is specifically designed for such regions), thus they will not affect our design as in these regions \ld{} will be disabled. However, it is quite non-trivial to simultaneously launch DRP with MSF attack~\cite{fusionripper} due to its non-deterministic nature (\S\ref{sec:opportunity})--it would be hard, if not impossible, for the attacker to figure out where to place the attack patch beforehand, not to mention how to carefully synchronize the dirty pattern with the MSF attack to effectively bypass \ld{}.
% Since \ld{} leverages LD to detection lateral-direction attack on MSF, attacks that can synchronously manipulate MSF and LD outputs would be able to evade our detection. Prior work has designed an attack to break LiDAR and camera perceptions at the same time~\cite{msf_adv}, however, no work has been proposed so far to achieve simultaneous manipulations on localization (i.e., MSF) and perception (i.e., LD). Moreover, since the existing lateral-direction localization attack is opportunistic~\cite{fusionripper}, it is thus difficult, if not impossible, for the attacker to predict when and where will MSF have large lateral deviations and apply physical-world lane line marking perturbations to the road~\cite{sato2021dirty} at the same location prior to attack.


% Therefore, we consider the chance of launching such simultaneous attacks to be minimal and leave them as future work.

% \todo{clarify: No existing simultaneous attacks; MSF attack is opportunistic; acknowledge limitation}

\cut{ % revision: not a limitation any more
\textbf{Lack of closed-loop in real vehicle experiments.} 
Although we include closed-loop evaluations with the AD control and simulation world feedback in the end-to-end simulations (\S\ref{sec:simulation}), the evaluations using the real vehicle (\S\ref{sec:eval_real_car}) still adopt a trace-based evaluation where we model the control effect by assuming the MSF deviations will result into a same amount of physical world deviation in the opposite direction. Ideally, a similar closed-loop evaluation should also be conducted on real AD vehicles to more realistically validate the defense capabilities. However, this is way beyond the affordability of normal academia research groups, and actually even companies such as Waymo and Uber also heavily rely on trace-based and simulation-based evaluations when developing and testing their AD systems for safety and budget considerations~\cite{bansal2018chauffeurnet, frossard2018end}. We thus leave this to future work, e.g., we have an ongoing effort to develop a chassis capable of closed-loop control with AD sensor setup, which might be used for this purpose once built and tested.
}


%However, we currently do not have the necessary hardware and compatible vehicle to run the complete AD system pipeline. 
%We are in the process of developing a real-vehicle sized chassis capable of closed-loop control with AD sensor setup, and plan to use it for validating \ld{} in the future.


%have plans to purchase a AD development chassis with the complete AD sensor set, including GPS, LiDAR, IMU, Camera, etc. We plan to validate \ld{} using this vehicle in a real-world testing facility in the future.

\cut{
\nsubsection{Alternative Design} \label{sec:discuss_alternative_designs}

\textbf{Apply LD as a fusion input in MSF.}
In \ld{}, we use LD outputs in a post-processing step in AD localization to cross validate the MSF lateral deviations. An alternative design might be to directly include LD \textit{as one of the fusion inputs in MSF localization} to mitigate lateral-direction attacks. However, such a design faces several challenges:
(1) it is unclear how to practically fuse relative positioning sources such as LD with global positioning sources such as GPS and LiDAR locator without degrading MSF accuracy. Existing works are able to estimate an LD-based global localizations using semantic maps~\cite{kang2020lane, evlampev2020map}, however, their positioning accurate are at 0.5m level, which might severely degrade the resulting MSF accuracy (at centimeter-level~\cite{wan2018robust}) if got fused together;
(2) additional fusion inputs may be able to increase the robustness against \fr{} to certain degree, but it cannot fundamentally prevent \fr{}~\cite{fusionripper}. On the other hand, \ld{} is able to achieve perfect detection performance against \fr{} and can safely steer the AD vehicle to stop in the ego lane after detection.
}


% \textbf{Attack response after stopping.}
% Call support and police, reboot system, etc. 
% In our emergency trajectory generation (\S\ref{sec:design_ar}), we focus on the most basic requirements, i.e., slowing down and steering. However, real world drivings are more complicated in which the emergency trajectory may stop in the middle of an intersection. (This is still part of the applicable domain issue for lane boundaries). To address, a potential solution is to design a speed profile that consider that can allow the vehicle to pass the intersection before stops.

% should also consider the front obstacles. Generally it is not a concern since the AD planning maintains a distance larger than the stopping distance with the front vehicles in normal driving. However, in the cases such as a vehicle suddenly cuts in and stops, we can do nothing.. To address this, a straightforward solution is to reuse the planning logic in high-level AD systems for 

\cut{
\textbf{Alternative attack response goals.}
In the current \ld{} design, we set the AR goal as to stop with in lane boundaries as soon as possible. Although such an AR goal can prevent more serious consequences that could occur if the vehicle stops out of lane boundaries, e.g., being hit by another vehicle that failed to yield in time, it may be more desirable to pull over to the roadside or even keep driving in the ego lane if possible. However, such AR goals requires more sophisticated AR designs, and thus we leave them as future work.
}

% Instead of stopping in lane. Use the fused EH pose to navigate to pull over to the road side. 