%%%%%%%%%%%%%%%%%%%%%%%%%%%%%%%%%%%%%%%%%%%%%%%%%%%%%%%%%%%%%%%%%%%%%%%%%%%%%%%%
% Template for USENIX papers.
%
% History:
%
% - TEMPLATE for Usenix papers, specifically to meet requirements of
%   USENIX '05. originally a template for producing IEEE-format
%   articles using LaTeX. written by Matthew Ward, CS Department,
%   Worcester Polytechnic Institute. adapted by David Beazley for his
%   excellent SWIG paper in Proceedings, Tcl 96. turned into a
%   smartass generic template by De Clarke, with thanks to both the
%   above pioneers. Use at your own risk. Complaints to /dev/null.
%   Make it two column with no page numbering, default is 10 point.
%
% - Munged by Fred Douglis <douglis@research.att.com> 10/97 to
%   separate the .sty file from the LaTeX source template, so that
%   people can more easily include the .sty file into an existing
%   document. Also changed to more closely follow the style guidelines
%   as represented by the Word sample file.
%
% - Note that since 2010, USENIX does not require endnotes. If you
%   want foot of page notes, don't include the endnotes package in the
%   usepackage command, below.
% - This version uses the latex2e styles, not the very ancient 2.09
%   stuff.
%
% - Updated July 2018: Text block size changed from 6.5" to 7"
%
% - Updated Dec 2018 for ATC'19:
%
%   * Revised text to pass HotCRP's auto-formatting check, with
%     hotcrp.settings.submission_form.body_font_size=10pt, and
%     hotcrp.settings.submission_form.line_height=12pt
%
%   * Switched from \endnote-s to \footnote-s to match Usenix's policy.
%
%   * \section* => \begin{abstract} ... \end{abstract}
%
%   * Make template self-contained in terms of bibtex entires, to allow
%     this file to be compiled. (And changing refs style to 'plain'.)
%
%   * Make template self-contained in terms of figures, to
%     allow this file to be compiled. 
%
%   * Added packages for hyperref, embedding fonts, and improving
%     appearance.
%   
%   * Removed outdated text.
%
%%%%%%%%%%%%%%%%%%%%%%%%%%%%%%%%%%%%%%%%%%%%%%%%%%%%%%%%%%%%%%%%%%%%%%%%%%%%%%%%
\documentclass[conference]{IEEEtran}
\pagestyle{plain}
% \documentclass[letterpaper,twocolumn,10pt]{article}
% \usepackage{usenix-2020-09}

% to be able to draw some self-contained figs
\usepackage{tikz}
\usepackage{url}
\usepackage{cite}
\usepackage{amsmath,amssymb,amsfonts}
\usepackage{graphicx}
\usepackage{textcomp}
\usepackage{xcolor}
\usepackage{subcaption}
\usepackage{multirow}
\usepackage{pifont}
\usepackage{booktabs}
% For better looking scientific notation
\usepackage{siunitx}

% for strikeout text
\usepackage{soul}

\usepackage{breakurl}
\def\UrlBreaks{\do\/\do-}
\usepackage[colorlinks]{hyperref}
\setstcolor{red}

\usepackage{algorithm, algorithmicx}
\usepackage{algpseudocode}
\renewcommand{\algorithmicrequire}{\textbf{Input:}}
\renewcommand{\algorithmicensure}{\textbf{Output:}}
\let\oldReturn\Return
\renewcommand{\Return}{\State\oldReturn}
\algnewcommand\algorithmicforeach{\textbf{for each}}
\algdef{S}[FOR]{ForEach}[1]{\algorithmicforeach\ #1\ \algorithmicdo}
\newcommand{\spooftaxi}{\text{S\textsc{poof}}\text{T\textsc{axi}}}
\newcommand{\attacktrials}{\text{A\textsc{ttack}}\text{T\textsc{rials}}}


% Use regular font for links to save space
% \urlstyle{rm}

\newlength{\bibitemsep}\setlength{\bibitemsep}{.24\baselineskip plus .05\baselineskip minus .05\baselineskip}
\newlength{\bibparskip}\setlength{\bibparskip}{0pt}
\let\oldthebibliography\thebibliography
\renewcommand\thebibliography[1]{%
  \oldthebibliography{#1}%
  \setlength{\parskip}{\bibitemsep}%
  \setlength{\itemsep}{\bibparskip}%
}

% \iffalse
\iftrue

\newcommand{\cut}[1]{}
\newcommand{\alfred}[1]{}
\newcommand{\junjie}[1]{}
\newcommand{\ziwen}[1]{}
\newcommand{\yunpeng}[1]{}
\newcommand{\toreview}[1]{#1}
\newcommand{\todo}[1]{}
\newcommand{\op}{OpenPilot}
\newcommand{\ld}{$\mathrm{LD^3}$}
\newcommand{\ldi}{$LD^3$}
\newcommand{\fr}{\textit{FusionRipper}}
\newcommand{\red}[1]{#1}
\newcommand{\blue}[1]{#1}
\newcommand{\newparts}[1]{#1}

\else

\newcommand{\cut}[1]{}
\newcommand{\alfred}[1]{\textcolor{red}{[Alfred: #1]}}
\newcommand{\junjie}[1]{\textcolor{orange}{[Junjie: #1]}}
\newcommand{\ziwen}[1]{\textcolor{red}{[Ziwen: #1]}}
\newcommand{\yunpeng}[1]{\textcolor{red}{[Yunpeng: #1]}}
\newcommand{\toreview}[1]{\textcolor{red}{#1}}
\newcommand{\todo}[1]{\textcolor{red}{[TODO: #1]}}
\newcommand{\op}{OpenPilot}
\newcommand{\ld}{$\mathrm{LD^3}$}
\newcommand{\ldi}{$LD^3$}
\newcommand{\fr}{\textit{FusionRipper}}
\newcommand{\red}[1]{{\color{red}#1}}
\newcommand{\blue}[1]{{\color{blue}#1}}
\newcommand{\newparts}[1]{{\color{blue}#1}}

\fi


\newcommand{\circled}[1]{\raisebox{.5pt}{\textcircled{\raisebox{-.9pt} {{\small #1}}}}}

\newcommand{\ready}{\textcolor{blue}{[Ready for review]}}

%%% For paper with diff
%\usepackage{ulem}
\newcommand{\diff}[1]{{\color{blue}#1}}
%\newcommand{\diffst}[1]{{\color{blue}\st{#1}}}

%%% For submission
%\newcommand{\diff}[1]{{#1}}
\newcommand{\diffst}[1]{}


%%% subcaption reference format
\captionsetup[sub]{labelformat=simple}
\makeatletter
\renewcommand\p@subfigure{\thefigure\,}
\renewcommand\thesubfigure{(\alph{subfigure})}
\makeatother

% For the rebuttal boxes
\usepackage{mdframed}
\mdfdefinestyle{rebuttalstyle}{innerleftmargin=3pt, innerrightmargin=3pt, innertopmargin=3pt, innerbottommargin=3pt}

\newcounter{response}[section]
\newenvironment{response}{\refstepcounter{response}
\vspace{-1mm}
\begin{mdframed}[style=rebuttalstyle]
\noindent \textbf{Response~\theresponse:}\normalfont
}
{
\end{mdframed}
\vspace{-1mm}
}

\newcounter{revision}[section]
\newenvironment{revision}{\refstepcounter{revision}
\vspace{-1mm}
\begin{mdframed}[style=rebuttalstyle]
\noindent \textbf{Revision~\therevision:}\normalfont
}
{
\end{mdframed}
\vspace{-1mm}
}

\newcounter{comment}[section]
\newenvironment{comment}{\refstepcounter{comment}
\vspace{-1mm}
\begin{mdframed}[style=rebuttalstyle]
\noindent 
}
{
\end{mdframed}
\vspace{-1mm}
}

\newenvironment{smashedalign}
 {\par$\!\aligned}
 {\endaligned$\par}

% \iffalse
\iftrue

\newcommand{\ncaption}[1]{\caption{#1}}
\newcommand{\nsection}[1]{\section{#1}}
\newcommand{\nsubsection}[1]{\subsection{#1}}
\newcommand{\nsubsubsection}[1]{\subsubsection{#1}}

\else

%\newcommand{\ncaption}[1]{\vspace{-0.27cm}\caption{#1}{\vspace{-0.37cm}}}
\newcommand{\ncaption}[1]{\caption{#1}}
\newcommand{\nsection}[1]{\vspace{-0.32cm}\section{#1}\vspace{-0.25cm}}
\newcommand{\nsubsection}[1]{\vspace{-0.32cm}\subsection{#1}\vspace{-0.25cm}}
\newcommand{\nsubsubsection}[1]{\vspace{-0.2cm}\subsubsection{#1}\vspace{-0.15cm}}

\fi



% \IEEEoverridecommandlockouts
% \makeatletter\def\@IEEEpubidpullup{6.5\baselineskip}\makeatother
% \IEEEpubid{\parbox{\columnwidth}{
%     Network and Distributed System Security (NDSS) Symposium 2023\\
%     28 February - 4 March 2023, San Diego, CA, USA\\
%     ISBN 1-891562-83-5\\
%     https://dx.doi.org/10.14722/ndss.2023.23xxx\\
%     www.ndss-symposium.org
% }
% \hspace{\columnsep}\makebox[\columnwidth]{}}

%-------------------------------------------------------------------------------
\begin{document}
%-------------------------------------------------------------------------------


\title{\Large \bf Lateral-Direction Localization Attack in High-Level Autonomous Driving:\\Domain-Specific Defense Opportunity via Lane Detection}


%\author{\IEEEauthorblockN{{\rm Junjie Shen}$^{}$\quad {\rm Yunpeng Luo}$^{}$ \quad {\rm Ziwen Wan}$^{}$ \quad {\rm Qi Alfred Chen}}
%\IEEEauthorblockA{$^{}$University of California, Irvine}
%\IEEEauthorblockA{
%$^{}$\{junjies1, yunpel3, ziwenw8, alfchen\}@uci.edu}}


\author{{\rm Junjie Shen}$^{}$\quad {\rm Yunpeng Luo}$^{}$ \quad {\rm Ziwen Wan}$^{}$ \quad {\rm Qi Alfred Chen}\\% <-this % stops a space
University of California, Irvine \ \ \ \{junjies1, yunpel3, ziwenw8, alfchen\}@uci.edu}

\maketitle

\begin{abstract}

The Fast Reciprocal Square Root Algorithm is a well-established approximation technique consisting of two stages: first, a coarse approximation is obtained by manipulating the bit pattern of the floating point argument using integer instructions, and second, the coarse result is refined through one or more steps, traditionally using Newtonian iteration but alternatively using improved expressions with carefully chosen numerical constants found by other authors. The algorithm was widely used before microprocessors carried built-in hardware support for computing reciprocal square roots. At the time of writing, however, there is in general no hardware acceleration for computing other fixed fractional powers. This paper generalises the algorithm to cater to all rational powers, and to support any polynomial degree(s) in the refinement step(s), and under the assumption of unlimited floating point precision provides a procedure which automatically constructs provably optimal constants in all of these cases. It is also shown that, under certain assumptions, the use of monic refinement polynomials yields results which are much better placed with respect to the cost/accuracy tradeoff than those obtained using general polynomials. Further extensions are also analysed, and several new best approximations are given.

\end{abstract}


% 06/01/21: adjust the vspace after all figure/table/alg
% \setlength{\textfloatsep}{10pt}
% \setlength{\floatsep}{10pt}
\setlength{\textfloatsep}{6pt}
\setlength{\floatsep}{6pt}

% Figure environment removed

\section{Introduction}
Automatic 3D reconstruction of clothed humans using image inputs has gained increasing significance due to its potential applications in a wide array of AR/VR scenarios. High-fidelity reconstructions typically depend on sophisticated capture systems, which are developed with dense camera arrays~\cite{collet2015high,joo2015panoptic,joo2018total}, programmable light-stages~\cite{Vlasic2009, guo2019relightables}, and depth sensors~\cite{newcombe2011kinectfusion,DoubleFusion,BodyFusion,dou2016fusion4d,newcombe2015dynamicfusion}. However, stringent capture environments equipped with complex hardware pose significant challenges for consumer-level applications.


In this context, considerable research effort has been dedicated to developing methods that allow for more flexible capture configurations, such as utilizing a few RGB inputs. Among these works, learning implicit functions \cite{iccv2020PIFu, saito2020pifuhd, hong2021stereopifu} has proven effective in achieving highly detailed reconstructions by integrating the advancements of deep neural networks. These methods employ large multi-layer perceptrons (MLPs) to predict the occupancy probability or truncated signed distance function (TSDF) value of every queried 3D point based on its associated local feature, which is extracted from images. They can recover a continuous surface at arbitrary resolutions without topology restrictions.


However, in typical MLP-based implicit networks, the occupancy or TSDF value at each location is solved independently with planar image features, rendering them less capable of addressing challenging cases such as occlusions. Consequently, these methods suffer from generalization and robustness issues, particularly when tackling strong occlusions caused by large motion or multiple interacting humans. 
Some follow-up studies  \cite{zheng2021deepmulticap,zheng2021pamir,huang2020arch} utilize an extra geometric model, SMPL~\cite{Loper2015}, to improve robustness by introducing strong shape priors. 
Their success typically relies on the assumption of geometrical similarity \cite{huang2020arch} between the shape prior and target reconstruction, making them intractable for handling complex cases with loose clothes and sensitive to errors in SMPL model fitting.



%\ping{this paragraph sounds like `TSDF is better than MLP/SMPL, and we use TSDF to solve the problem'. But in Sec 3, we are telling a different story, saying `MLP needs a 3D convolutional encoder'. We need to make these two sections consistent.}\sicong{I think in this paragraph we claim that the TSDF}


%We opt for Trucated Signed Distance Funtion (TSDF) volumetric representations as they are naturally suitable for convolution operations, which have shown remarkable performance for learning hierarchical features on 2D visual perception tasks \cite{SunXLW19}. 
%Meanwhile, TSDF also describes the gradual geometry change around shape surface, which is not reflected by occupancy volume. 

We instead revisit the 3D volumetric representation and resort to 3D convolutional neural networks (CNNs) for feature learning, due to their impressive performance in feature learning and the ability to incorporate spatial context. However, volumetric methods and 3D convolution involve discretization, which might raise concerns regarding whether a discretized volume can preserve subtle geometric details as continuous representations learned in implicit functions. We investigate the relationship between volume resolution and quantization error on synthetic data by converting target mesh objects to TSDF volumes, as shown in Figure~\ref{fig:quantization_error}. We observe that the quantization errors are significantly reduced by increasing volume resolution and become nearly negligible when reaching a relatively high resolution (e.g., 512 or higher). In other words, achieving fine-detailed reconstruction is not supposed to be restricted by the use of volume representations as long as a proper volume resolution is utilized. Therefore, we present a method with high-resolution feature volumes, e.g., 256 and 512, while traditional volumetric methods \cite{varol18_bodynet,gilbert2018volumetric} are often limited to much lower resolutions, such as 32 or 128.



On the other hand, an increase in volume resolution may lead to a cubic growth of memory overhead \cite{8100085}. Reducing memory costs while guaranteeing the granularity of volumetric representations is necessary for pursuing high-quality reconstruction. Thus, we adopt a coarse-to-fine approach and cull away irrelevant voxels to build a sparse high-resolution feature volume. At the coarse level, the network computes an initial TSDF by applying a U-Net with sparse 3D CNN \cite{3DSemanticSegmentationWithSubmanifoldSparseConvNet} on the sparse feature volume, which is carved by a visual hull. Through our experiments, it turns out that more than 95\% of the volume grids are discarded by the visual hull culling, making the sparse 3D CNN efficient. At the fine level, the network focuses on a narrow band near the zero-level set of the initial TSDF and discretizes the narrow band with smaller voxels. By employing this narrow-band culling, we further shrink the sampling space, resulting in a relatively small range of grid numbers (usually 300K--500K in our experiments) even with a high volume resolution of 512. The remaining voxels in the narrow band are associated with features that fuse high-frequency information from the computed normal maps upon the low-frequency shape from the coarse level to compute the TSDF at high resolution. The final mesh is then extracted from the TSDF using the Marching-Cube algorithm ~\cite{Lorensen87marchingcubes}.
% Different from the u-net sturcture to preserve global topology context, we then apply a shallow 3dcnn to compute the final TSDF $D_{final}$ which contain more local geometry detail.




% \ping{this paragraph can be expanded. It is an important contribution and often ignored by other works. stress on the novel idea of regressing blending weights instead of colors}

In addition to geometry, high-quality mesh texture is also a crucial factor contributing to visual appearance. Directly computing a color field in 3D space, as in \cite{iccv2020PIFu}, struggles to capture high-frequency texture details, while the neural radiance field (NeRF) \cite{yu2020pixelnerf} or the DoubleField~\cite{shao2022doublefield} require expensive per-instance optimization and are often unstable for sparse input images. In contrast, we adopt an image-based rendering approach to compute a texture atlas map, which is efficient and widely supported in existing computer graphics tools. 
Specifically, we compute a blending weight at each 3D point on the mesh surface to determine its color as a weighted average of the colors at its image projections. The blending weights can be computed at a relatively coarse resolution, e.g., 512 volume resolution in our case, and leave texture details to the high-resolution images, such as 1K or 2K. Unlike previous methods that generate blurry texturing results under sparse input, our method generalizes well on both synthetic and real data with just a few input views. 
Figure~\ref{fig:teaser} shows two examples reconstructed by our method. Despite the challenging garment, pose, and occlusion, our method recovers faithful shape, normal, and texture on the right.

%with a wide variety of poses and clothing styles, and it is also adaptive to handle input image with arbitrary resolutions.
%\sicong{For this concern we claim that when the resolution of dicretized volume meets certain threshold (which is 256 in our experiment), the quantization error can be neglected.} 



In summary, the main contributions of this paper are as follows:
\begin{itemize}
\vspace{-0.1in}
  \item 
  We revisit the 3D volumetric representation and demonstrate that it can support clothed human reconstruction with equal or even better performance compared to implicit representation. 
  \item 
  We develop a memory and computation-efficient method for high-resolution volumetric reconstruction using sophisticated sparse 3D CNN, coarse-to-fine estimation, and voxel culling by visual hull and narrow bands. 
  \item 
  We introduce a novel method to compute a texture atlas map, which captures rich appearance details from high-resolution input images.
  \item 
  We achieve impressive results on standard benchmark datasets Twindom and MultiHuman, significantly reducing the point-2-surface (P2S) precision to approximately 0.2cm from just six input views, with more than $50\%$ error reduction compared to the state-of-the-art methods, including DoubleField~\cite{shao2022doublefield} and PIFuHD~\cite{saito2020pifuhd}.
\end{itemize}
\vspacebeforesection
\section{Background}
\label{sec:background}

In this section, we provide the necessary background information to ensure a comprehensive understanding of the attack described in this paper. We start with a description of the Distributed Hash Table (DHT) used by IPFS, followed by its content resolution mechanisms. We also detail techniques for network size estimation, necessary for our attack detection and mitigation mechanisms.

\vspacebeforesection
\subsection{IPFS DHT}
\label{sec:kad_dht}

We review the features of the Kademlia DHT~\cite{maymounkov2002kademlia} and its \texttt{libp2p} implementation~\cite{libp2p_github} that are the most relevant to our attack.
To participate in the DHT, each peer generates a public/private key pair and derives an identity $\peerid \in \{0,1\}^{256}$ as the hash of its public key.
Ideally, each peer generates a random key pair and, therefore, peer IDs are distributed uniformly and independently over the space $\{0,1\}^{256}$.
While honest nodes follow this rule, malicious nodes may generate and choose from an arbitrary number of key pairs.
Each peer maintains a routing table consisting of $m=256$ buckets.
The $i$-th bucket contains the addresses of up to $k=20$ peers whose peer IDs share a common prefix of exactly $i$ bits with the peer's own peer ID. 

%
A new participant node joins the IPFS network by contacting one of the hardcoded bootstrap nodes. This bootstrap node provides the new node with some initial peers allowing it to join the DHT. The new node uses this information to perform a walk through the DHT towards its own peer ID.
The walk allows to: \textit{(i)}~make sure that there is no other node in the network with the same ID; \textit{(ii)}~discover new peers and fill the newcomer's DHT routing table. At the same time, the newcomer establishes \bitswap~\cite{de2021accelerating} connections to a subset of encountered peers (usually around 300 of them). The core role of the \bitswap protocol is to enable bilateral content transfer and to play the role of a cache for recently-accessed content.

The main DHT operation $\Call{GetClosestPeers}{\key}$ returns the $k=20$ closest peers to $\key$. 
%
In Kademlia, the distance between two keys $x$ and $y$ in the key space is given by $x \oplus y \in \{0,...,2^{256}-1\}$, where $\oplus$ denotes the bitwise XOR operation on the keys; the resulting binary string is interpreted as an integer.
%
When a client wants to find the peers with IDs closest to $\key$, it sends a request to the $\alpha=3$ peers in its routing table whose peer IDs are closest to $\key$. Each of these peers returns the $k$ closest peers to $\key$ in its own routing table and the addresses of these peers. 
%
The client again sends a request to the $\alpha$ peers closest to $\key$, among peers in its routing table and those whose addresses it just received. This process repeats until the client does not find any more peers closer to $\key$.
Due to network churn and imperfect routing tables, we observed in our experiments that successive calls to $\Call{GetClosestPeers}{\key}$ do not always return the same set of $k=20$ peers (we provide more details in \Cref{sec:evaluation}, \Cref{fig:20closest}). This is an important limitation affecting our attack.

\vspacebeforesection
\subsection{Content Resolution in IPFS}
\label{sec:ipfs}

IPFS is a content-centric network.
It allows its participant to request files without specifying their location. 
%
Content is indexed by content IDs $\cid \in \{0,1\}^{256}$ that are derived from a hash of that content.
Both peer IDs and CIDs are used as keys in the DHT.
Each node can play the role of a \provider, \downloader, or \resolver. 
The process of content advertisement and resolution is illustrated in \Cref{fig:add_get_provider}.

%
When a \provider wishes to publish content with a given $\cid$ on IPFS, it creates a \emph{provider record} that contains $cid$ and the \provider's address.
During a $\Call{Provide}{\cid}$ operation, the \provider first uses $\Call{GetClosestPeers}{\cid}$ to locate the $k=20$ peers with their peer IDs closest to $\cid$, 
%
and then sends them a $\mathsf{PutProvider}$ message including the provider record (\Cref{fig:add_get_provider}(a)).
We call the peers that hold provider records for $\cid$ the \emph{resolvers} for $\cid$.

Each CID can have several \providers. In fact, by default, each IPFS client becomes a provider for each piece of content it downloads for a fixed amount of time (12h, 24h, or 48h depending on the client version or custom configuration). As a result, the system provides an auto-scaling feature with supply automatically rising with demand.

%
When a \downloader wishes to fetch a piece of content, it first sends a request to all its \bitswap peers. If none of them has the content, the \downloader uses the DHT-based resolution system. We stress that the \bitswap protocol plays the supporting role of a cache in the dissemination of popular files. However, the mechanism does not provide reliable content resolution, in particular for new or less popular content. %

When \bitswap unstructured search fails, the \downloader resolves $\cid$ using $\Call{FindProviders}{\cid}$. This operation uses a DHT walk identical to that of $\Call{GetClosestPeers}{\cid}$ to find $k$ \resolvers but also queries encountered nodes for a provider record for $\cid$ (\Cref{fig:add_get_provider}(b)). The process terminates when either 20 \providers have been found, or all \resolvers have been asked. Querying all encountered nodes (\ie, not only the designated \resolvers) is useful because some of the encountered nodes may have a provider record in their cache.
%

Upon receiving a provider record, the client connects to the address specified in the provider record to retrieve the actual content (\Cref{fig:add_get_provider}(c)).
Provider records are not authenticated, and therefore malicious \providers may respond with incorrect provider records (or may not respond at all). However, the integrity of the content is preserved because the hash of the retrieved content can be verified against its $\cid$.
%


%

\input{img/add_get_provider.tex}

\vspacebeforesection
\subsection{Network Size Estimator}
\label{sec:netsize}

The number of nodes in a decentralized system is generally unknown due to the avoidance of centralized membership management.
This number is nonetheless useful for optimizations, deciding on individual node configurations, or security mechanisms.
Various methods were proposed for the decentralized estimation of unstructured and structured networks~\cite{eli-sohl-dht-size-estimation,kostoulas2005decentralized, manku2003symphony}.
We use in this work a mechanism developed initially by Protocol Labs as part of a mechanism for decreasing the latency of publishing content in IPFS~\cite{network-size-estimation-notion,network-size-estimation-github-pr}.

%
%
%
%
%
%
%
%
%
%

Each node in the DHT refreshes its routing table periodically (every $10$ minutes in \texttt{libp2p}). 
For this, the node samples $m$ random keys (one for each bucket of its routing table)
%
and queries the DHT to obtain the $k=20$ closest peer IDs to each key.
Using these, the node then computes the average distance between each one of these keys $\key_j$ for $j=1,\dots,m$ and their $i$-th closest peer ID for $i=1,...,k$ (with $m=256$ and $k=20$).
\begin{equation}
    \label{equ:avg-dist}
    \overline{D}_i = \frac{1}{m} \sum_{j=1}^m \operatorname{dist}(\key_j, \peerid_{j}^{(i)})
\end{equation}
where $\peerid_{j}^{(i)}$ is the $i$-th closest peer ID to $\key_j$.
With $N$ peers in the DHT and peer IDs uniformly distributed in the hash space, the expected distance between a $\key$ and its $i$-th closest peer ID is $\frac{2^{256}i}{N+1}$. The node then runs a least square regression to compute the value of $N$ for which the expected distances best fit the empirical average distances, \ie,
\begin{equation}
    \label{equ:netsize-least-squares}
    \hat{N} = \arg\min_{N} \sum_{i=1}^k \left(\overline{D}_i - \frac{2^{256}i}{N+1}\right)^2.
\end{equation}
The resulting estimate $\hat{N}$ can be computed in closed form.
%

When a node starts running, it must perform DHT queries for a few random keys to initialize its network size estimate. 
Since a larger number of queries will result in higher accuracy, making more queries than what is needed to initialize one's routing table is recommended.
Thereafter, keeping the estimate up-to-date does not require any excess DHT queries beyond what is already used for refreshing the routing table as this is done frequently (every 10 minutes).

While the network size estimate has a stochastic variance resulting from the probability distribution of the honest peer IDs, it is hard for an attacker to bias the estimate significantly. Since the estimator uses the density of peer IDs around keys chosen uniformly at random, the adversary would require numerous Sybil nodes (on the order of the whole network size) to significantly affect the peer ID density around those keys.

% \vspace{-0.05in}
\nsection{Lane Detection for High-Level AD Localization Defense: Opportunity Analysis} \label{sec:opportunity}

\textbf{Motivation and novelty.} Currently, no software-based defense solutions have been proposed to address the latest GPS spoofing-based lateral-direction localization attack in high-level AD systems (\S\ref{sec:background_msf_attack}). The closest ones are the recent physical-invariants based detectors proposed for small robotic vehicles such as drones and rovers, e.g., SAVIOR~\cite{savior} and CI~\cite{ci}, which estimate the physical dynamics of drones and rovers to validate the GPS signal. 
Although they show high effectiveness for such small robotic vehicles under large attack deviation goals, their effectiveness in AD vehicle context is fundamentally more limited since (1) existing vehicle dynamics models have difficulties in modelling high-speed and curvy-road settings~\cite{kong2015kinematic, polack2017kinematic}; and (2) in the AD context, the attack deviation goals can be much smaller (thus harder to detect) while still being safety-critical. As we concretely evaluate later in \S\ref{sec:eval_detection}, direct adaptation of such existing physical-invariant based approach to the AD context suffers from very high false positives and is actually close to random guessing.


In comparison to small robotic vehicles, the AD context may also have its unique defense opportunities for such lateral-direction localization attacks. \textit{Lane Detection (LD)}~\cite{hillel2014recent, pan2018spatial}, a technology commonly used in low-level AD systems for lane centering~\cite{openpilot, autopilot}, is such an example that can be used to measure the vehicle's lateral position within the current lane in real time, which is directly related to the lateral-direction attack goal (lane departure). Although effective in low-level AD systems (e.g., Level-2 ones such as Tesla Autopilot~\cite{autopilot} that still count on human drivers to take over anytime), \textit{LD is currently not used for high-level AD localization purpose (e.g., Level-4 ones such as Waymo that do not assume onboard human drivers)}. This is because what LD can provide is by nature only \textit{local} positioning (i.e., relative positioning within ego lane), while high-level AD requires \textit{global} positioning (i.e., in world coordinates on a map) for safe and correct driving decision-making without human drivers. Although there exist camera-based global localization methods using lane markings~\cite{kang2020lane, evlampev2020map}, they are not generally adopted in state-of-the-art high-level AD localization~\cite{wan2018robust, gao2015ins, soloviev2008tight, udacity_av_apollo, udacity_av_nd, coursera_av} as they are far from reaching the required centimeter-level accuracy~\cite{levinson2007map, reid2019localization, ega_requirement_report}.
%not only require extra efforts for lane-map generation~\cite{}, but also 
% for high-level AD~\cite{levinson2007map, reid2019localization, ega_requirement_report}.
% To the best of our knowledge, \textit{no prior works have used LD to defend against localization attacks;
% and state-of-the-art high-level AD systems do not use LD for localization~\cite{wan2018robust, autoware, gao2015ins, soloviev2008tight, udacity_av_apollo, udacity_av_nd, coursera_av}. 

%nd (2) high-level AD localization requires centimeter-level localization accuracy~\cite{levinson2007map, reid2019localization, ega_requirement_report} for safe and correct driving without human drivers, while LD can only provide less-accurate lateral positioning~\cite{} and is incapable of longitudinal localization. 

While less suitable for global localization accuracy purposes in high-level AD,
in this paper we propose to be the first to explore novel use of LD for \textit{defense purposes} in high-level AD localization. To concretely understand the potential of such a domain-specific defense opportunity, we analyze LD's defense properties in the following 5 general aspects.

%\alfred{include argument on LD is not used for high-level Ad localization and why? I remember it is a common reviewer question.} \junjie{added.}



% Although seemingly promising, LD also has limitations that may hinder its defense practicality. 

\newparts{

\textbf{1) General to lateral-direction localization attack.}
As mentioned above, LD can provide real-time information directly related to the \textit{attack goal} of lateral-direction localization attacks. Thus, LD by nature has the potential to provide general defense capabilities to not only the existing attack designs such as those in~\S\ref{sec:background_msf_attack}, but also their potential adaptive versions or other new attack designs in the future, as long as the attack goal is to cause lateral deviations. 

%As mentioned above, LD is directly related to the attack goal of lateral-direction localization attacks. Therefore, LD-based defenses are general to any GPS spoofing methods that aim to cause lane departure, among which \fr{}~\cite{fusionripper} is the state-of-the-art and is so far the only effective one for MSF-based localization in high-level AD systems.
%\alfred{talk about the generality over fusionripper} \junjie{added.}

\textbf{2) Technology maturity.}
Benefit from the growing prosperity of Deep Neural Networks (DNNs), LD is already a mature technology that has been used for lane centering in low-level AD systems and vehicles, e.g., OpenPilot~\cite{openpilot}, Tesla Autopilot~\cite{autopilot}, GM Cadillac, Honda
Accord, Toyota RAV4, Volvo XC90, etc.
In fact, the existing camera-based LD solutions are quite robust to the dynamic environmental conditions. For example, Tesla Autopilot can effectively recognize lane lines even during a night storm~\cite{autopilot_night_rain}. 
Apart from DNN advancement, the camera auto-exposure and vehicle headlights also improve the usability of LD. Later in \S\ref{sec:eval}, we also evaluate our defense on datasets with various environmental conditions and show that it is robust to low visibility conditions.
}

\newparts{
\textbf{3) Defense deployability.}
Since today's high-level AD vehicles are all equipped with cameras for road object detection, using them for an LD-based defense solution is thus readily deployable without the need to install any new hardware. Moreover, many state-of-the-art LD models are publicly available~\cite{pan2018spatial, neven2018towards}, including those used in industry-grade lane centering systems~\cite{openpilot}; some high-level AD systems are also using LD for camera calibrations~\cite{apollo}.}
% Therefore, deploying LD to high-level AD systems typically will not impose technical challenges.


% Information is generally available in AD context

\newparts{
\textbf{4) Defense coverage.}
For LD to be effective, lane line markings are required, which may not be available in local road segments such as intersections. Interestingly, due to real-world sensor noises and algorithm inaccuracies, the attacks to MSF localization are \textit{fundamentally opportunistic}. For example, despite having a high overall attack success rate, latest lateral-direction localization attack cannot predict when and where a large deviation can be injected to the MSF outputs~\cite{fusionripper}.
Due to such opportunistic property, the attacker \textit{cannot deterministically cause a desired lateral deviation to only appear in regions without lane line markings}. Such an attack property is fundamental to the MSF localization designs popularly used in high-level AD systems, since with this design the attack effectiveness is fundamentally dependent on sensor noises and algorithm inaccuracies of other sources, which are neither observable nor controllable by a tailgating attacker~\cite{fusionripper}.

Motivated by this insight, we analyze all attack traces evaluated in the \fr{} paper~\cite{fusionripper} and our own evaluation later (\S\ref{sec:eval_method}), and find that LD can indeed provide a decent practical defense coverage: among all attack starting points in the traces, only \textit{0.8\% (15/1813)} achieved the attack goal in road regions without lane line markings. Thus, an LD-based defense, if effective, can already provide protection for the 99.2\% of the possible attack attempts. In addition, autonomous trucks, which are an important high-level AD application, are generally not subject to such limitation since they mainly operate on the ``middle mile'' (i.e., highways)~\cite{tusimple_ad_truck_middle_mile, walmart_ad_truck_middle_mile}, where lane line markings are generally always available. 
}

% FusionRipper attack success rate in areas without lane line markings: 
% 0.83\% = 15 / 1813 (among all attack traces)
% 1.41\% = 15 / 1062 (among all local attack traces)

\newparts{
\textbf{5) Independence to existing localization attack.}
To defend against existing attacks, a desired defense property is that the lane line markings perceived by LD are not already used in MSF localization. This is because if such information is already used, existing attacks might have already exploited their vulnerable periods (e.g., natural detection inaccuracies), making the additional use of such information for defense less likely to be effective. 
%make the defense robust and therefore practically usable.
In representative MSF localization designs, LiDAR locator is the only one among MSF inputs (\S\ref{sec:background_msf_attack}) that is possible to utilize lane line markings as features. Thus, we perform an experimental analysis to understand the dependency between state-of-the-art LiDAR locators~\cite{wan2018robust, autoware} and lane line markings in Appendix~\ref{app:lidar_lane_line_dependency}. Our results show that today's LiDAR localization algorithms have a \textit{statistically-strong independence} of the lane line markings, very likely because lane markings is much less useful for global localization on a map compared to more unique road features such as buildings, roadside layouts, and traffic signs. This thus suggests that LD can indeed provide independent defense information to existing attacks. However, such independence property will disappear in adaptive attack settings (i.e., consider attacking LD after the defense is deployed). Thus, we require our defense design to be fully-aware of such adaptive attack surface (\S\ref{sec:design_overview}), and also evaluate it later (\S\ref{sec:adaptive_attacks}).

%Admittedly, even with perfect independence between LD and MSF, attacks that target both can potentially bypass the defense. However, as will be discussed in \S\ref{sec:discuss}, such a simultaneous attack neither already exists, nor can be easily achieved.
% due to the difficulty of attack synchronization and non-determinism of the existing lateral-direction localization attack~\cite{fusionripper}.
%\alfred{is there a separate more detailed discussion on this? If so please cross-ref (again, no repetitive logic). In such detailed discussion, we should emphasize how fundamental such a property is. } \junjie{now refering to the limitation discussion section on simultaneous attack to MSF and LD.}
}

% After doing this, a new challenge is if an attacker is aware of this source, whether they can bypass it? We have evaluation referring to stealthy attack evaluation

% one adaptive attack is FusionRipper and LD attack together
% but fusion ripper is non-determinism, hard to do it together



\section{Design}
The design philosophy of causal-learn is centered around building an open-source, modular, easily extensible and embeddable Python platform for learning causality from data and making use of causality for various purposes. Due to the different goals, assumptions, and techniques between causal learning and traditional learning tasks, newcomers to the field often find it hard to get a clear picture of the developments in modern causality research. Thus, we briefly introduce the algorithms and functionalities in causal-learn with a special focus on their use cases and suitable application scenarios.

\subsection{Search methods}
Causal-learn covers representative causal discovery methods across all major categories with official implementation of most algorithms. We briefly introduce the methods as follows. It is worth noting that we are actively updating the library to incorporate latest algorithms.
\begin{itemize}
    \item \textbf{Constraint-based causal discovery methods.} Current algorithms under that category are PC \citep{spirtes2000causation}, FCI \citep{spirtes1995causal}, and CD-NOD \citep{huang2020causal}. PC is a classical and widely-used algorithm with consistency guarantee under independent and identically distributed (i.i.d.) sampling assuming no latent confounders, the faithfulness assumption, and the causal Markov condition, which has been extensively applied in many fields. By continuously applying (conditional) independence tests on subsets of variables of increasing size in a smart way, its search procedure returns a Markov Equivalence Class (MEC), of which the graphical object consists of a mixture of directed and undirected edges, known as a Completed Partially Directed Acyclic Graph (CPDAG). PC is highly adaptable to various use cases, facilitated by the selection of an appropriate independence test; it can handle data with different assumptions, such as Fisher-Z test \citep{fisher1921014} for linear Gaussian data, Chi/G-squared test \citep{tsamardinos2006max} for discrete data, and Kernel-based Conditional Independence (KCI) test \citep{zhang2011kernel} for the nonparametric case. Moreover, causal-learn provides an extension, Missing-Value PC (MV-PC) \citep{tu2019causal}, to address issues of missing data. Furthermore, we have implemented FCI for causal structures that include hidden confounders (it indicates the possible existence of hidden confounders whenever the possibility cannot be excluded, but it cannot help determine possible relations among them), and causal discovery from nonstationary/heterogeneous data (CD-NOD). These constraint-based methods offer wide applicability as they can accommodate various types of data distributions and causal relations, provided that appropriate conditional independence testing methods are utilized. However, genenerally speaking, they may not be able to determine the complete causal graph uniquely and, accordingly, there usually exist some undirected edges in the returned CPDAGs.

    \item \textbf{Score-based causal discovery methods.} Different from the search style of constraint-bed methods, score-based methods find the causal structure by optimizing a properly defined score function. Greedy Equivalence Search (GES) \citep{chickering2002optimal} is a well-known two-phase procedure that directly searches over the space of equivalence classes. Similarly, exact search (e.g., A* \citep{yuan2013learning}, Dynamic Programming \citep{silander2006simple}), and permutation-based search (e.g., GRaSP \citep{lam2022greedy}) apply different search strategies to return a set of the sparsest Directed Acyclic Graphs (DAGs) that contains the true model under assumptions strictly weaker than faithfulness. These score-based methods are versatile, able to accommodate a wide array of data and causal relations by choosing suitable score functions, such as BIC \citep{schwarz1978estimating} for linear Guassian data, BDeu \citep{buntine1991theory} for discrete data, and Generalized Score \citep{huang2018generalized} for the nonparametric case. The choice of score function can be conveniently adjusted as a hyperparameter.

    \item \looseness=-1 \textbf{Causal discovery methods based on constrained functional causal models.} While constraint-based and score-based methods offer flexibility through the selection of an appropriate independence test or score function, they are limited to returning equivalence classes, yielding non-unique solutions where the causal direction between certain variable pairs remains indeterminate. In contrast, assuming specific Functional Causal Models (FCMs)--that is, functions in a particular functional class to specify how the effect is generated from its direct causes and noise--allows for the full determination of the causal structure, albeit at the cost of certain trade-offs. Causal-learn incorporates algorithms based on several FCM variants, capable of producing unique causal directions. Examples include the linear non-Gaussian acyclic model (LiNGAM) \citep{shimizu2006linear} and its variant, i.e., DirectLiNGAM \citep{shimizu2011directlingam}, which have been extensively applied for non-Gaussian noises with linear relations. VAR-LiNGAM \citep{hyvarinen2010estimation}, which combines LiNGAM with vector autoregressive models (VAR), to estimate both time-delayed and instantaneous causal relations from time series. RCD \citep{maeda2020rcd}, an extension of LiNGAM, allows for hidden confounders, while CAM-UV \citep{maeda2021causal} further extends this to the nonlinear additive noise case. In addition, the additive noise model (ANM) \citep{hoyer2008nonlinear} has been proven to be identifiable in the presence of nonlinearity and additive noises. Furthermore, we have also incorporated the post-nonlinear (PNL) causal model \citep{zhang2009identifiability}, a highly general form (with LiNGAM and ANM as special cases) that has been demonstrated to be identifiable in the generic case, barring five specific situations described in \citep{zhang2009identifiability}.

    \item \textbf{Causal representation learning: Finding causally related hidden variables.} Latent variables play an instrumental role in a multitude of real-world scenarios, often acting as hidden confounders that influence observed variables. Unfortunately, most existing methods may fail to produce convincing results in cases with latent variables (confounders). In causal-learn, we implement the Generalized Independent Noise (GIN) condition \citep{xie2020generalized} for estimating linear non-Gaussian latent variable causal model, which allows causal relationships between latent variables and multiple latent variables behind any two observed variables. This promises to improve the detection and understanding of the complex, often hidden, causal structures that govern real-world phenomena.

\end{itemize}

Besides, causal-learn also has Granger causality \citep{granger1969investigating, granger1980testing} implemented for statistical but not causal\footnote{As mentioned by Granger, Granger causality is not necessarily true causality. In fact, If one assumes 1) that there is no latent confounding process, 2) that the data are sampled at the right causal frequency, and 3) that there are no instantaneous causal influences, then Granger causality defined by Granger \citep{granger1980testing} can be seen as causal relations that can be discovered from stochastic processes with constraints-based methods such as PC. Of course, if those assumptions are violated, one may still apply Granger causal analysis, but the estimated relations may not be true causal influences.} time series analysis. Through the collective efforts of various teams and the contributions of the open-source community, causal-learn is always under active development to incorporate the most recent advancements in causal discovery and make them available to both practitioners and researchers.

\subsection{(Conditional) independence tests}

In addition to its comprehensive search methods, causal-learn also provides a variety of (conditional) independence tests as independent modules. Besides being an essential parts of several search methods, these tests can also be independently utilized and seamlessly integrated into existing statistical analysis pipelines. Currently,the library features a diverse array of such tests including Fisher-z test \citep{fisher1921014}, Missing-value Fisher-z test, Chi-Square test, Kernel-based conditional independence (KCI) test and independence test \citep{zhang2011kernel}, and G-Square test \citep{tsamardinos2006max}, each with distinct capabilities and benefits. The Fisher-z test is ideally suited for linear-Gaussian data, while the Missing-value Fisher-z test addresses the challenges of missing values by implementing a testwise-deletion approach. For categorical variables, the Chi-Square and G-Square tests are most effective. For users interested in a nonparametric test or the case with mixed categorical and continuous data types, the KCI test is an option. Overall, the range of tests offered by causal-learn underscores its versatility in handling diverse data types.

\subsection{Score functions}
\looseness=-1
Moreover, a diverse range of score functions is available in \textit{causal-learn}. These score functions quantify the goodness of fit of a model to the data, a crucial measure in score-based causal discovery methods, and can also be utilized independently for model selection in a broader range. Among these, the Bayesian Information Criterion (BIC) score \citep{schwarz1978estimating} is used extensively, offering a balance between model complexity and fit to the data. Another important score function is the Bayesian Dirichlet equivalent uniform (BDeu) score \citep{buntine1991theory}. This score function, especially beneficial for discrete data, incorporates a uniform prior over the set of Bayesian networks. Additionally, the Generalized Score \citep{huang2018generalized} is also available in causal-learn, which offers the flexibility to accommodate more complex scenarios and is beneficial for nonparametric cases where the true data-generating process does not align with the assumptions of BIC (linear Gaussian) or BDeu (discrete).


\subsection{Utilities}

Causal-learn further offers a suite of utilities designed to streamline the assembly of causal analysis pipelines. The package features a comprehensive range of graph operations encompassing transformations among various graphical objects integral to causal discovery. These include Directed Acyclic Graphs (DAGs), Completed Partially Directed Acyclic Graphs (CPDAGs), Partially Directed Acyclic Graphs (PDAGs), and Partially Ancestral Graphs (PAGs). Additionally, to enhance the convenience of experimental processes, \textit{causal-learn} features a set of commonly used evaluation metrics to appraise the quality of the causal graphs discovered. These metrics include precision and recall for arrow directions or adjacency matrices, along with the Structural Hamming Distance \citep{acid2003searching}.

\subsection{Demos, documentation, and benchmark datasets}

The \textit{causal-learn} package also contains extensive usage examples of all search methods, (conditional) independence tests, score functions, and utilities at 
\\ \centerline{ \url{https://github.com/py-why/causal-learn/tree/main/tests}.} 
\\
Furthermore, detailed documentation is available at \\
\centerline{\url{https://causal-learn.readthedocs.io/en/latest/}.} \\
It is worth noting that it also includes a collection of well-tested benchmark datasets--since ground-truth causal relations are often unknown for real data, evaluation of causal discovery methods has been notoriously known to be hard, and we hope the availability of such benchmark datasets can help alleviate this issue and inspire the collection of more real-world datasets with (at least partially) known causal relations. 



\section{Evaluation} \label{sec:evaluation}

\begin{table*}[tbp]
\centering
\small
\begin{tabular}{cccccccccc}
\toprule
& \multicolumn{3}{c}{\msr} & \multicolumn{3}{c}{\negc} & \multicolumn{3}{c}{\wsj} \\
& Acc. & F1 & wF1 & Acc. & F1 & wF1 & Acc. & F1 & wF1 \\ \cmidrule(lr){2-4} \cmidrule(lr){5-7} \cmidrule(lr){8-10} 
\udel & 66.86 & 56.76 & 64.3 & \textbf{80.80} & 55.45 & 77.9 & 63.74 & 64.23 & 63.2 \\
\icsi & \underline{71.19} & 64.73 & 70.4 & 80.36 & 64.53 & \underline{78.6} & 64.62 & 64.15 & 63.4 \\
\cnts & 68.59 & 61.39 & 67.2 & 78.68 & 61.62 & 76.8 & 64.31 & 64.59 & 64.4 \\
\osu & 68.02 & 60.28 & 66.6 & 79.24 & 57.04 & 76.5 & 69.20 & 69.63 & 68.9 \\
\isg & 67.05 & 58.83 & 65.3 & 77.34 & 59.52 & 75.6 & 69.15 & 69.35 & 69.2 \\ \midrule
\bert & \textbf{71.68} & \underline{66.70} & \textbf{71.4} & 77.79 & \underline{72.87} & 77.7 & \underline{80.95} & \underline{80.93} & \underline{80.9} \\
\roberta & 70.91 & \textbf{67.53} & \underline{70.7} & \textbf{80.80} & \textbf{77.29} & \textbf{80.7} & \textbf{82.61} & \textbf{82.70} & \textbf{82.6} \\ \midrule
Average & 69.19 & 62.32 & 67.99 & 79.29 & 64.05 & 77.69 & 70.65 & 70.80 & 70.37 \\
\bottomrule
\end{tabular}
\caption{\label{tab:performance} Overall accuracy (Acc.), macro-averaged F1 (F1), and weighted-macro F1 (wF1) scores of the algorithms depicted in Section~\ref{sec:algorithm}. For instance, \msr-\udel refers to a C5.0 classifier trained on the \msr~corpus, using the feature set mentioned in \citet{greenbacker-mccoy-2009-udel}.}
%Its Acc., F1 and wF1 of this model are 66.86, 56.76, and 64.3, respectively.}
\end{table*}


In this section, we introduce the evaluation protocol and report the performance of the models.

\subsection{Implementation Details} \label{sec:implementation}

For \bert and \roberta, we used \textit{bert-base-cased} and \textit{roberta-base}, both from Hugging Face. For fine-tuning, we set the batch size to 16, the learning rate to 1e-3, the dropout rate to 0.5, and the size of the output layer to 256. We ran each model for 20 epochs and used the one that achieved the highest F1 score on the development set. The implementation details of the classic ML-based models can be found in Appendix~\ref{sec:appendixML}.

\subsection{Evaluation Protocol} \label{sec:protocol}

The main evaluation metric in the GREC-MSR shared tasks was accuracy. 
In addition to accuracy, we also report macro-F1 and weighted-macro F1. We argue that different metrics evaluate algorithms from different perspectives and provide us with different meaningful insights. 
For pragmatic tasks like REG, it makes sense to ask how well an algorithm performs on naturally distributed data which is often imbalanced. For these cases, reporting accuracy and weighted F1 are logical. 
Furthermore, analogous to other classification tasks, minority categories should not be overlooked. Take as an example the class \emph{description} in the \negc corpus, which occurs only 4\%. If a model fails to produce this class, the produced document might sound unnatural. Therefore, it is important to ensure that an algorithm is not over- or under-generating certain classes. Looking into accuracy and macro-F1 together provides insights into such cases.

\subsection{Performance of the Models}\label{subsec:overallacc}

The overall accuracy of the models, their macro F1, and their weighted-macro F1 are presented in Table \ref{tab:performance}. 
We also present the ranking of the models based on these scores in Appendix~\ref{sec:app_rank}. 


\paragraph{PLM-based Models.} The best-performing models across all corpora and metrics are PLM-based models.  In six out of nine rankings, \bert and \roberta are ranked as the top two models. The sole exception is \negc, where \bert is the second worst model. The benefit of using PLMs is the largest on the \wsj corpus. For example, \roberta improves the macro F1 score from 69.63 (i.e., the performance of the best ML-based model) to 82.70.


\paragraph{ML-based Models.} In contrast to the robust performance of the PLM models, the performance of the classic ML models is more corpus-dependent. In the case of \msr and \negc, \icsi is the best-performing model, while in the case of \wsj, it is at the bottom section of the rankings. Another interesting observation is the performance of the \udel models. In terms of accuracy, \udel has the highest performance in \negc, while it has the lowest performance in both \msr and \wsj. In terms of macro-F1 rankings, the \negc \udel model dropped from first to last place, whereas \bert improved from penultimate place to second place. In general, our ML models yielded lower scores than the original models used in the GREC study \citep{belz2009generating}. This could be attributed to a variety of factors, including differences in feature engineering and model parameters.

\paragraph{Comparing Different Metrics.} 

Upon comparing average scores across the three metrics, we observe that for \msr and \negc, PLMs are clear winners only when macro-F1 is the metric in question. However, for \wsj, PLMs are winners on all three metrics. This may be because the distribution of categories in \wsj is much more balanced than in the other two corpora.
% \vspace{-0.05in}
\nsection{End-to-End Evaluations} \label{sec:end_to_end_eval} 
% \vspace{-0.05in}

% \newparts{
% In this section, we implement \ld{} on 2 open-source full-stack AD systems, Baidu Apollo~\cite{apollo} and Autoware~\cite{autoware}, and evaluate its defense capability in end-to-end driving with closed-loop control in an industry-grade high-level AD simulator and on a real vehicle-sized AD development chassis. The demo videos are available on our project website at 
% \textbf{\url{https://sites.google.com/view/ld3-defense}}.}


\newparts{
In this section, we implement \ld{} on 2 open-source full-stack AD systems, Baidu Apollo~\cite{apollo} and Autoware~\cite{autoware}, and evaluate \ld{} under end-to-end drivings in both simulation and the physical world. The demo videos are available on our project website at 
\textbf{\url{https://sites.google.com/view/cav-sec/LD3}}.}

% integrate \ld{} in an industry-grade high-level AD system and show the defense effectiveness in end-to-end simulation environments with the presence of AD control.

\vspace{-0.05in}
\nsubsection{Evaluation in AD Simulator}  \label{sec:simulation}

\cut{
\textbf{Experimental setup.}
We implement \ld{} in Baidu Apollo v5.0.0~\cite{apollo} and evaluate under 4 driving scenarios with different driving speeds (local road and highway speeds) and road geometries (straight and curvy roads) in the LGSVL simulator~\cite{lgsvl}.
% following the design in Fig.~\ref{fig:design_overview}. We run the complete Baidu Apollo AD system with all functional modules enabled in an industry-grade AD simulator, LGSVL~\cite{lgsvl}. 
% We evaluate the benign and attacked drivings with \ld{} in 4 driving scenarios with different driving speeds (local road and highway speeds) and road geometries (straight and curvy roads).
In our evaluation, we include the \ld{} variant with naive AR design (\S\ref{sec:eval_ar}) and one without defense.
We repeat the simulation for 10 times with different attack starting times for each combination of simulation scenarios and defense settings.
The detailed simulation setup is in Appendix~\ref{app:simulation_setup}.}


\textbf{Experimental setup.}
We implement \ld{} in Baidu Apollo v5.0.0~\cite{apollo} following the design in Fig.~\ref{fig:design_overview}. Specifically, we reuse the SCNN model~\cite{pan2018spatial} for LD, which is currently used only for camera calibration in Baidu Apollo. We run the complete Baidu Apollo AD system with all functional modules enabled in a production-grade AD simulator, LGSVL~\cite{lgsvl}. Since LGSVL does not provide LiDAR locator maps required for MSF, we instead run Baidu Apollo localization in the Real-Time Kinematic mode, which directly takes the ground truth positions from LGSVL. To simulate the \fr{} attack effect, we add the lateral deviations from the same attack trace used in \S\ref{sec:eval_visibility} to the localization outputs.

We evaluate the benign and attacked drivings with \ld{} in 4 driving scenarios on two LGSVL maps: Single Lane Road (SLR) and San Francisco (SF). Specifically, the SLR map is a long straight road, and we create a low-speed (SLR-Low) and high-speed (SLR-High) driving scenario on it by adjusting the maximum cruising speed in Apollo planning. The SF map is a 1:1 re-creation of a portion of the San Francisco city, from which we select a straight (SF-Straight) and a curvy road (SF-Curvy). In our evaluation, we also include the \ld{} variant with the naive AR design (\S\ref{sec:eval_ar}) and a setting without any defenses. We repeat the simulation for 10 times with different attack starting times for each combination of simulation scenarios and defense settings.


\textbf{Results and demos.}
Our simulation results show that the attack detection rates for both \ld{} and \ld{}-NaiveAR are all 100\% in the 10 runs, and none of the benign drivings are falsely detected as under attack. 
Table~\ref{tbl:sim_results} shows the maximum lateral deviation achieved in the whole simulation (including both attack detection and response periods) in each scenario/defense setting and the corresponding vehicle stopping location. As shown, with \ld{}, the average maximum deviations are smaller than lane straddling deviation in all 4 scenarios and the vehicle can always safely stop in the lane. In comparison, due to the blind trust of the localization outputs in the AR period, \ld{}-NaiveAR has much higher maximum deviations than \ld{} and the vehicle's stopping locations are either lane straddling or already crashing into the road curb/barrier. Nevertheless, the No Defense setting is even worse than \ld{}-NaiveAR, where the vehicle is simply deviated to fall off the road in SLR-Low and SLR-High.
Snapshots of the vehicle stopping locations in SF-Straight are shown in Fig.~\ref{fig:sim_snapshot}. 
The demos of the 4 simulation scenarios and 3 defense settings are available on our project website.

% \textbf{\url{https://sites.google.com/view/ld3-defense}}


\begin{table*}[tbp]
\footnotesize
\centering
\caption{Maximum deviations to lane center and attack consequences under different defense settings in the 4 simulation scenarios in \S\ref{sec:simulation}. Each setting was run for 10 times with randomized attack starting times. Benign driving with \ld{} is also presented and was run for 10 times. The maximum deviations are represented as (mean, std) in meters.}
\label{tbl:sim_results}
% \vspace{-0.1in}
\setlength{\tabcolsep}{4.5pt}
\begin{tabular}{@{}c|c|cccccc|cc@{}}
\toprule
\multirow{3}{*}{\begin{tabular}[c]{@{}c@{}}Simulation\\ scenario\end{tabular}} & \multirow{3}{*}{\begin{tabular}[c]{@{}c@{}}Lane\\ straddle\\ dev\end{tabular}} & \multicolumn{6}{c}{Attacked} & \multicolumn{2}{|c}{Benign} \\ \cmidrule(l){3-10} 
 &  & \multicolumn{2}{c}{\ld} & \multicolumn{2}{c}{\ld-NaiveAR} & \multicolumn{2}{c}{No Defense} & \multicolumn{2}{|c}{\ld} \\ \cmidrule(l){3-10} 
 &  & Max dev & Consequence & Max dev & Consequence & Max dev & Consequence & Max dev & Consequence \\ \midrule
SLR-Low & 0.83 & 0.47, 0.08 & Stop in lane & 1.69, 0.06 & Stop w/ lane straddle & 7.94, 0.05 & Fall off road & 0.07, 5e-5 & Reach destination \\
SLR-High & 0.83 & 0.69, 0.06 & Stop in lane & 1.64, 0.16 & Stop w/ lane straddle & 7.93, 0.04 & Fall off road & 0.07, 5e-5 & Reach destination \\
SF-Straight & 1.00 & 0.67, 0.23 & Stop in lane & 1.02, 0.01 & Hit curb & 1.84, 0.16 & Hit tree or barrier & 0.14, 7e-4 & Reach destination \\
SF-Curvy & 0.75 & 0.43, 0.14 & Stop in lane & 0.90, 0.12 & Hit lane divider & 0.97, 0.14 & Hit lane divider & 0.31, 0.01 & Reach destination \\ \bottomrule
\end{tabular}
% \vspace{-0.1in}
\end{table*}



% Figure environment removed

% \vspace{0.05in}
\nsubsection{\newparts{Evaluation on AD Development Chassis of Real Vehicle Size and Closed-loop Control}} \label{sec:pixkit_eval}
\vspace{0.03in}

% Although existing AD simulation technologies are already widely used in AD development for safety testing, it is still unclear whether the physics modeling is accurate enough such that the defense capability can indeed be faithfully translated to real-world driving. Therefore, we further perform another end-to-end evaluate of \ld{} on a real vehicle-sized AD development chassis with closed-loop control.
% \newparts{In this section, we evaluate \ld{} on a real vehicle-sized AD development chassis with closed-loop control to understand the defense capabilities in the real world.}

% Figure environment removed


\textbf{Experimental setup.} We experiment on an AD chassis as shown in Fig.~\ref{fig:pixkit_side_by_side}, which is specifically designed for Level-4 AD system prototyping and testing. The chassis is of a real vehicle size, capable of closed-loop control, and fully equipped with Level-4 AD sensors including LiDAR, GPS, IMU, cameras, RADARs, and ultrasonic sensors.
Since AD vehicle testing is not allowed to be on public roads by default, we reserve a parking lot in our institute for the experiments. Specifically, we mark a straight traffic lane with 3.5 m width (the most common lane width in KAIST dataset and our night-time driving trace) in the parking lot and create the corresponding semantic map for Autoware.


We ported \ld{} to the Autoware AD system~\cite{autoware}, which is currently supported by the AD chassis. To facilitate the attack, we apply the same \fr{} attack trace used in \S\ref{sec:eval_visibility} and \S\ref{sec:simulation} to the localization outputs in Autoware.
Unlike OpenPilot and Baidu Apollo, the lane detector in Autoware can only detect lane lines in pixels rather than in the world coordinates. Therefore, we directly obtain the ground truth lane line information from the map using the unmodified localization outputs, since LD is already a mature technology (\S\ref{sec:opportunity}) and has been shown to be quite accurate in \S\ref{sec:eval} and \S\ref{sec:simulation}.
We enable the relevant components in Autoware including localization, global/local plannings, and control. During the experiments, the AD chassis is completely driven by Autoware unless taken over by us from a remote controller in emergency situations. We evaluate three defense settings: (1) \textit{w/ \ldi{} w/ attack}, (2) \textit{w/o \ldi{} w/ attack}, and (3) \textit{w/ \ldi{} w/o attack}.
For each, we experiment in driving speeds of 2 m/s (4.5 mph) and 4 m/s (9 mph) for safety concerns. 
We prolong the AR stage by using deceleration $<$3 $m/s^2$ in both cases to better showcase the driving behaviors during AR.
Specifically, we repeat the experiments for 3 times for \textit{w/ \ldi{} w/ attack}. Since the other two are always quite stable, we thus do not record more iterations for those experiments.


\begin{table*}[tbp]
\footnotesize
\begin{minipage}{0.55\linewidth}
    \footnotesize
    \centering
    \caption{The detection, maximum, and stopping deviations in the three settings at two different driving speeds. We repeat the experiments for \textit{w/ \ld{} w/ attack} for 3 times and report the (mean, std) deviations. We do not repeat the other two settings as they are quite stable.}
    \label{tbl:pixkit_devs}
    % \vspace{-0.1in}
    \setlength{\tabcolsep}{1.2pt}
    \begin{tabular}{@{}c|cccc|cc@{}}
    \toprule
    \multirow{3}{*}{Speed} & \multicolumn{4}{c|}{w/ attack} & \multicolumn{2}{c}{w/o attack} \\ \cmidrule(l){2-7} 
     & \multicolumn{3}{c|}{w/ \ld{}} & w/o \ld{} & \multicolumn{2}{c}{w/ \ld{}} \\ \cmidrule(l){2-7} 
     & Det dev & Max dev & \multicolumn{1}{c|}{Stop dev} & Max/Stop dev & Max dev & Stop dev \\ \midrule
    4 m/s & 0.07m, 0.01m & 0.36m, 3e-3m & \multicolumn{1}{c|}{0.05m, 0.05m} & 2.59m & 0.13m & 8e-3m \\
    2 m/s & 0.02m, 2e-3m & 0.27m, 0.04m & \multicolumn{1}{c|}{0.01m, 1e-3m} & 2.23m & 0.11m & 7e-3m \\ \bottomrule
    \end{tabular}
\end{minipage}\hfill
\begin{minipage}{0.43\linewidth}
    \footnotesize
    \centering
    \caption{Maximum physical deviations can be achieved without being detected under various LD fluctuation assumptions. The percentages indicate the probabilities of such fluctuations.}
    \vspace{-0.05in}
    \label{tbl:stealthy_detection}
    % \setlength{\tabcolsep}{5pt}
    \setlength{\tabcolsep}{3pt}
    \begin{tabular}{@{}ccccc@{}}
    \toprule
    \multirow{2}{*}{Trace} & \multirow{2}{*}{\begin{tabular}[c]{@{}c@{}}LD fluctuation\\ ($\mu, \sigma$)\end{tabular}} & \multicolumn{3}{c}{Max physical world deviation} \\ \cmidrule(l){3-5} 
     &  & 0 (100\%) & $\mu$ (50\%) & $\mu+3\sigma$ (0.3\%) \\ \midrule
    \textit{ka-local31} & 0.12m, 0.08m & 0.7m & 0.82m & 1.06m \\
    \textit{ka-local33} & 0.14m, 0.10m & 0.7m & 0.84m & 1.14m \\
    \textit{ka-highway36} & 0.29m, 0.10m & 0.7m & 0.99m & 1.29m \\
    \textit{ka-highway18} & 0.20m, 0.11m & 0.7m & 0.90m & 1.23m \\ \bottomrule
    \end{tabular}
\end{minipage}
% \vspace{-0.22in}
\end{table*}

% 4 m/s:
% Detection dev (mean, std): 0.06493076049897435 0.012452686654347397
% Maximum dev (mean, std): 0.36426716650795504 0.0032264096321182583
% Stopping dev (mean, std): 0.04520309618108121 0.04720487829962962
% NoDefense Maximum dev: 2.5913297119226946
% NoDefense Stopping dev: 2.584318858073847
% Benign Maximum dev: 0.13370482650200852
% Benign Stopping dev: 0.00814336299953038

% 2 m/s:
% Detection dev (mean, std): 0.020136245340817385 0.0022321999365849127
% Maximum dev (mean, std): 0.2731832984655125 0.04088672062665852
% Stopping dev (mean, std): 0.0058209895419398605 0.0017981851847734464
% NoDefense Maximum dev: 2.2288639764466422
% NoDefense Stopping dev: 2.2078183052640297
% Benign Maximum dev: 0.10658935648516672
% Benign Stopping dev: 0.00735364768826751


\textbf{Results and demos.} 
Table~\ref{tbl:pixkit_devs} shows the detection, maximum, and stopping deviations under the three settings. As shown, \ld{} on average can detect the attack when the vehicle's physical deviation is still small and start the AR stage. Within the AR period, the average maximum deviations are 0.36 m and 0.27 m at speeds of 4 m/s and 2 m/s, respectively, and the final stopping deviations are always within 0.1 m. In comparison, without \ld{}, the vehicle keeps deviating and we have to manually press the emergency button on the remote to prevent it from crashing into the curb. 
Such a distinctive driving behaviors with and without \ld{} are consistent with our trace-based (\S\ref{sec:eval}) and simulation results (\S\ref{sec:simulation}).
Without the attack, the vehicle's trajectories well align with the road centerline (i.e., the reference trajectory Autoware plans to enforce) and eventually complete the route and stop at the center of the lane.
We also record demo videos of the vehicle driving behaviors under the three settings (videos are available on our website). As an illustration, Fig.~\ref{fig:pixkit_stop_positions} visualizes the driving trajectories in the bird's eye view and shows the snapshots of final stopping positions at driving speed of 4 m/s.

% \todo{put the PIXKIT evaluation here as a subsection.
% 2 m/s: maximum deviation in AR: 0.23 meters, final stop deviation: 0.09 meters
% 4 m/s: maximum deviation in AR: 0.37 meters, final stop deviation: 0.10 meters}


% \vspace{-0.05in}
\nsection{Evaluation against Adaptive Attacks} \label{sec:adaptive_attacks}

% In \S\ref{sec:eval} and \S\ref{sec:end_to_end_eval}, we demonstrate the effectiveness of \ld{} at detecting and responding to existing state-of-the-art lateral-direction localization attack. 
In this section, we take a step further to examine \ld{}'s capability under potential adaptive attacks, including (1) an idealized stealthy attack that can evade the detection, and (2) the latest LD-side attack, which is the inherent new attack surface introduced by \ld{} approach (\S\ref{sec:design_challenges}).

% \vspace{-0.05in}
\nsubsection{Stealthy Attack Evaluation} \label{sec:stealthy_attack}
\vspace{0.05in}

% Since the evaluation in the above sections are all based on \fr{}, which does not assume the existence of defenses such as \ld{}. However, \fr{} is a realistic evaluation target since it is by far the only attack that can break the MSF localization on high-level AD systems. Nevertheless, 
In this evaluation, we analyze the maximum lateral deviations that a hypothetical stealthy attack can achieve by assuming stronger and unrealistic attack capabilities.

% assume the attacker can have \textit{precise} and \textit{instant} control over the lateral deviations in the MSF localization outputs, and evaluate the maximum lateral deviations a hypothetical stealthy attack can cause without being detected.

\textbf{Evaluation methodology.}
Based on the CUSUM anomaly detection formulation (\S\ref{sec:design_detection}), the attack should satisfy $S_{i-1} + \lvert D_i^{\text{MSF}} - D_i^{\text{LD}} \rvert - b < \tau$ in order to prevent detection. Assuming the last CUSUM statistic $S_{i-1} = 0$, the maximum MSF lateral deviation without being detected is thus $D_{i, max}^{\text{MSF}} = D_i^{\text{LD}} + \tau + b$, which is also the maximum physical world deviation given the control assumption (\S\ref{sec:background_threat_model}).
% As shown, the deviation that a stealthy attack can cause depends on the CUSUM parameters and the LD lateral deviation.
Since $\tau$ and $b$ are fixed in the defense, the attacker can carefully select a timing where the LD has a large lateral deviation fluctuation to the actual vehicle location due to detection noises, and apply the MSF lateral deviation to the same direction as the LD's fluctuation direction to achieve a large physical world deviation. Therefore, the attacker's capability on capturing a particular LD fluctuation window determines the maximum physical world deviations she can achieve without being detected. Thus, we evaluate the maximum physical world deviations by assuming various levels of LD fluctuations that the attacker can capture.

% that the attacker can capture particular levels of LD fluctuations. 

\textbf{Assumptions on attack capabilities.}
In this evaluation, we assume the attacker has very unrealistic attack capabilities in order to achieve such a stealthy attack. In particular, the attacker should have a \textit{white-box knowledge} on (1) where exactly on the road that the LD will have a large fluctuation and how much it is, and (2) the attack detection method and parameters used in the target AD system. Moreover, the attacker should also have \textit{precise} and \textit{instantaneous} control over the lateral deviations in the MSF localization outputs in order to execute such attack when large fluctuations appear.

\cut{
\begin{table}[tbp]
\centering
\footnotesize
\caption{Maximum physical deviations can be achieved without being detected under various LD fluctuation assumptions. The percentages indicate the probabilities of such fluctuations.}
\vspace{-0.1in}
\label{tbl:stealthy_detection}
% \setlength{\tabcolsep}{5pt}
\begin{tabular}{@{}ccccc@{}}
\toprule
\multirow{2}{*}{Trace} & \multirow{2}{*}{\begin{tabular}[c]{@{}c@{}}LD fluctuation\\ ($\mu, \sigma$)\end{tabular}} & \multicolumn{3}{c}{Max physical world deviation} \\ \cmidrule(l){3-5} 
 &  & 0 (100\%) & $\mu$ (50\%) & $\mu+3\sigma$ (0.3\%) \\ \midrule
\textit{ka-local31} & 0.12m, 0.08m & 0.7m & 0.82m & 1.06m \\
\textit{ka-local33} & 0.14m, 0.10m & 0.7m & 0.84m & 1.14m \\
\textit{ka-highway36} & 0.29m, 0.10m & 0.7m & 0.99m & 1.29m \\
\textit{ka-highway18} & 0.20m, 0.11m & 0.7m & 0.90m & 1.23m \\ \bottomrule
\end{tabular}
\vspace{-0.05in}
\end{table}
}

\textbf{Results.}
Table~\ref{tbl:stealthy_detection} shows the maximum physical world deviations that the stealthy attack can achieve under different LD fluctuation assumptions. Specifically, we calculate LD fluctuation distributions in each trace and assume that the attacker knows where a certain level of fluctuation happens. Without any such assumptions, the attacker can at most inject $\tau+b=0.7$ m lateral deviation, which is just about to touch the lane boundaries. On the other hand, the attacker can \textit{at most} cause 0.99 m and 1.29 m lateral deviations on the 4 traces if she can capture an average and a 3-$\sigma$ LD fluctuation, respectively. Note that the probabilities of such fluctuations to appear are 50\% and 0.3\% according to the normal distribution. In conclusion, even under very unrealistic attack assumptions, the maximum lateral deviations are still less than the local road attack goal (1.3 m) for \fr{}, which shows that \ld{} is quite effective at bounding the lateral deviations. Moreover, it also highlights that LD is indeed a mature technology (\S\ref{sec:opportunity}) suitable for defense given its high stability.
% means that LD is a reliable defense source due to its stability. 

% \begin{table}[tbp]
% \centering
% \footnotesize
% \caption{Maximum deviations that can be achieved without being detected by \ld.}
% \label{tbl:stealthy_detection}
% \setlength{\tabcolsep}{2pt}
% \begin{tabular}{@{}cccc@{}}
% \toprule
% Trace & \begin{tabular}[c]{@{}c@{}}Max Dev w/o\\ Detection\end{tabular} & \begin{tabular}[c]{@{}c@{}}Lane Straddle\\ Dev\end{tabular} & \begin{tabular}[c]{@{}c@{}}Attack Goal\\ Dev\end{tabular} \\ \midrule
% ka-local31 & 1.39 & 0.7 & 1.3 \\
% ka-local33 & 1.06 & 0.7 & 1.3 \\
% ka-highway36 & 1.18 & 0.7 & 1.9 \\
% ka-highway18 & 1.13 & 0.7 & 1.9 \\ \bottomrule
% \end{tabular}
% \end{table}


% risk to build upon consecutive frames, clearly say we assumed the max attack capability, instant influence on physical world deviation, can precisely control timing and deviationr


% To detection: prevent detection, achieve large deviation Table~\ref{tbl:stealthy_detection}.

% highway18 max adaptive deviation:
% 1.13 m = 0.43 (LD max deviation in benign case) + 0.6 (cusum bias) + 0.1 (cusum threshold)
% highway36
% 1.18 m = 0.48 (LD max deviation in benign case) + 0.6 (cusum bias) + 0.1 (cusum threshold)
% local31
% 1.39 m = 0.59 + 0.7 + 0.1
% local33
% 1.06 m = 0.36 + 0.6 + 0.1

% \junjie{Future work: need an evaluation where we attack from LD side to trigger detection, but no adaptive design in AR.}

% Figure environment removed

% \vspace{-0.05in}
\nsubsection{LD-side Adaptive Attack Evaluation} \label{sec:ld_attack}
\vspace{0.05in}

% \newparts{
% Since \ld{} uses lane detection as a defense source, a direct adaptive attack direction is thus LD-side attacks. Therefore, 
% In this section, we evaluate \ld{} against LD-side attacks.}
% , which are direct adaptive attack direction to \ld{}.}

\newparts{
\textbf{Evaluation methodology.}
We explore the defense capability of \ld{} against the latest LD attack in production low-level AD systems, 
named Dirty Road Patch (DRP) attack~\cite{sato2021dirty}, which is designed to affect the detected lane line shapes to mislead the automated lane centering system to drive the vehicle out of the lane boundaries. 
In LD, the lane line shapes are represented as polynomial functions, which are used in \ld{} to calculate the vehicle's lateral deviations  (Appendix~\ref{app:design_impl}). 
% Therefore, in our evaluation, we focus on the attacked lane line polynomials influenced by DRP attack. 
From the 40 attack traces used in the original DRP attack paper, we extract the attacked lane line polynomials in each frame and calculate an averaged LD deviation trace.
In \ld{} design, LD attacks cannot disrupt the driving behaviors before the attack is detected since only MSF outputs are used for navigation at this moment.
To cause vehicle deviations, the LD attack has to trigger the detection in the first place and affect the \textit{fused} localization in the AR period (\S\ref{sec:design_ar}) in order to affect the vehicle control. Therefore, we focus on the AR period in our evaluation. 
To model the DRP attack effect, we apply the deviation trace (start from the detection deviation 0.7 m) to the LD side in the KAIST traces.
Since the MSF side is benign and should generally well-align with the physical positions of the vehicle, we set the MSF outputs in the AR period with the same deviation as the fused localization, but to the opposite direction based on the control assumption (\S\ref{sec:background_threat_model}).}

\newparts{
\textbf{Results.} Fig.~\ref{fig:ld_attack_devs} shows the maximum and stopping deviations in KAIST traces. As shown, none of them is able to even cause lane straddling. On average, the maximum and stopping deviations in the AR period are only 0.08 m ($\delta=0.08$ m) and 0.02 m ($\delta=0.03$ m), respectively. Such a result indicate that \ld{} is quite robust to adaptive attack to the LD side as well. This is because the safety-driven fusion (\S\ref{sec:design_ar}) in \ld{} can effectively penalize the more aggressive source in the driving context, which in this case is the attacked LD outputs, and prevent the fused localization from being influenced by it.}



% \vspace{-0.05in}
\nsection{Limitations Discussion} \label{sec:discuss}
% \vspace{0.05in}

\textbf{Defense coverage of lane detection.}
% As the first work to discover and leverage information sources available in high-level AD systems, we choose to use the lane boundary as a defense-suitable one for attack detection and response against lateral-direction localization attacks. 
In this work, we are the first to explore the novel usage of LD for defense. However, as a defense relying on LD, a potential limitation is the lane line marking coverage. However, as we analyzed in \S\ref{sec:design_overview}, the non-deterministic nature of attacks to MSF localization greatly alleviate such a limitation, where an LD-based defense has the potential to defend against the majority (99.2\%) of the attack attempts. In addition, for important AD applications such as autonomous trucks, they are naturally not subject to such limitation as they mostly operate on highways~\cite{tusimple_ad_truck_middle_mile, walmart_ad_truck_middle_mile}.
% However, the applicability of LD are limited in areas with lane line markings. 
At design level, since high-level AD systems come with semantic maps with accurate road geometry information, \ld{} knows exactly where are the regions without lane line markings and can temporarily disable the defense in such regions (\S\ref{sec:design_detection}).
To address this limitation, a potential future improvement is to also consider other road markings available in such regions, e.g., stop lines~\cite{lin2017stop} and crosswalk markings~\cite{bailo2017robust} in intersections, to help localize the vehicle and to detect MSF deviations. Nevertheless, it is unclear how prevalent such road markings are and how mature and robust the existing perception algorithms are to recognize such road markings.
%\alfred{talk about future improvement direction? if it is a pure argument, it is not necessary to repeat it here.} \junjie{added.}
% a good thing is that as a defense integrated in the AD system, \ld{} knows exactly where it is not applicable and can decide to temporally disable the attack detection (\S\ref{sec:design_detection}).
% In addition, \ld{} can already cover challenging scenarios such as highways, where the lane line markings are generally available, and the attack consequences can be more severe due to the high speeds.
% Nevertheless, a promising extension of this work is to also include information sources that are available in the intersections, e.g., stop lines, to help localize the vehicle and cross-check the MSF deviations.

% \todo{MSF is opportunistic; highway doesn’t have intersections and generally always have lane boundary markings.}

% \todo{mention the FusionRipper intersection attack success rate here: attack success rate:
% 0.83\% = 15 / 1813 (among all attack traces);
% 1.41\% = 15 / 1062 (among all local attack traces)}

% \todo{When talking about the number here, talk from victim’s point of view: Percentage of my driving is protected: 99.17\%.}

% Road markings as an independent source

\textbf{Simultaneous attacks to MSF and LD.}
Since \ld{} leverages LD to detection lateral-direction attack on MSF,
attacks that simultaneously target MSF and LD can thus potentially bypass our detection. 
In fact, such a vulnerability is a general limitation for CPS security research that uses sensor cross-checking/fusion for defense purposes~\cite{feng2018efficient, feng2017efficient, aguilar2017developing, tanil2017detecting, khanafseh2014gps, lee2015gps}. However, in practice, the defense value of \ld{} highly depends on \textit{whether such a simultaneous attack already exists or can be easily achieved}. For MSF and LD, neither of them holds today, since (1) although individual attacks on MSF or LD exist, no existing work shows that they can be effectively \textit{coordinated and synchronized} to achieve simultaneous attack effect control, and (2) it is far from trivial to achieve this with existing individual attack vectors. Specifically, among the attack vectors on camera~\cite{petit2015remote, yan2016can, nassi2020phantom, sayles2021invisible, kohler2021they, sato2021dirty, kang2020lane, jing2021too}, only three works~\cite{nassi2020phantom, sato2021dirty, jing2021too} actually evaluated and shown attack effectiveness on LD in realistic AD settings. All these three works consider adding malicious patterns to the ground (e.g., via road patch or stickers) as the attack vector. However, considering the non-deterministic nature of the existing high-level localization attacks (\S\ref{sec:opportunity}), it would be hard, if not impossible, for the attacker to figure out where to place the attack pattern beforehand, not to mention how to carefully synchronize the malicious pattern with the localization-side attack to effectively bypass \ld{}. Therefore, we consider such simultaneous attack design neither already exists nor can be easily achieved, and leave the systematic exploration of its feasibility as a future direction.

\textbf{Delay between attack and detection.} Another limitation is that our detection and response happen after the attack has occurred to some extent (i.e., some deviations have already been caused by the attack). Even though our system can greatly reduce the safety consequences and transition the vehicle into a minimal-risk condition, it is still better if we can detect the attack immediately after the first injection is sent to the system. We thus consider this as another future direction.

%it is quite non-trivial to simultaneously launch DRP with MSF attack~\cite{fusionripper} due to its non-deterministic nature (\S\ref{sec:opportunity})--it would be hard, if not impossible, for the attacker to figure out where to place the attack patch beforehand, not to mention how to carefully synchronize the dirty pattern with the MSF attack to effectively bypass \ld{}.




%Among them, only DRP attack~\cite{sato2021dirty} is applicable to us; the other two~\cite{nassi2020phantom, jing2021too} only show effectiveness on road regions without lane line markings (Jing et al.~\cite{jing2021too} is specifically designed for such regions), thus they will not affect our design as in these regions \ld{} will be disabled. However, it is quite non-trivial to simultaneously launch DRP with MSF attack~\cite{fusionripper} due to its non-deterministic nature (\S\ref{sec:opportunity})--it would be hard, if not impossible, for the attacker to figure out where to place the attack patch beforehand, not to mention how to carefully synchronize the dirty pattern with the MSF attack to effectively bypass \ld{}.
% Since \ld{} leverages LD to detection lateral-direction attack on MSF, attacks that can synchronously manipulate MSF and LD outputs would be able to evade our detection. Prior work has designed an attack to break LiDAR and camera perceptions at the same time~\cite{msf_adv}, however, no work has been proposed so far to achieve simultaneous manipulations on localization (i.e., MSF) and perception (i.e., LD). Moreover, since the existing lateral-direction localization attack is opportunistic~\cite{fusionripper}, it is thus difficult, if not impossible, for the attacker to predict when and where will MSF have large lateral deviations and apply physical-world lane line marking perturbations to the road~\cite{sato2021dirty} at the same location prior to attack.


% Therefore, we consider the chance of launching such simultaneous attacks to be minimal and leave them as future work.

% \todo{clarify: No existing simultaneous attacks; MSF attack is opportunistic; acknowledge limitation}

\cut{ % revision: not a limitation any more
\textbf{Lack of closed-loop in real vehicle experiments.} 
Although we include closed-loop evaluations with the AD control and simulation world feedback in the end-to-end simulations (\S\ref{sec:simulation}), the evaluations using the real vehicle (\S\ref{sec:eval_real_car}) still adopt a trace-based evaluation where we model the control effect by assuming the MSF deviations will result into a same amount of physical world deviation in the opposite direction. Ideally, a similar closed-loop evaluation should also be conducted on real AD vehicles to more realistically validate the defense capabilities. However, this is way beyond the affordability of normal academia research groups, and actually even companies such as Waymo and Uber also heavily rely on trace-based and simulation-based evaluations when developing and testing their AD systems for safety and budget considerations~\cite{bansal2018chauffeurnet, frossard2018end}. We thus leave this to future work, e.g., we have an ongoing effort to develop a chassis capable of closed-loop control with AD sensor setup, which might be used for this purpose once built and tested.
}


%However, we currently do not have the necessary hardware and compatible vehicle to run the complete AD system pipeline. 
%We are in the process of developing a real-vehicle sized chassis capable of closed-loop control with AD sensor setup, and plan to use it for validating \ld{} in the future.


%have plans to purchase a AD development chassis with the complete AD sensor set, including GPS, LiDAR, IMU, Camera, etc. We plan to validate \ld{} using this vehicle in a real-world testing facility in the future.

\cut{
\nsubsection{Alternative Design} \label{sec:discuss_alternative_designs}

\textbf{Apply LD as a fusion input in MSF.}
In \ld{}, we use LD outputs in a post-processing step in AD localization to cross validate the MSF lateral deviations. An alternative design might be to directly include LD \textit{as one of the fusion inputs in MSF localization} to mitigate lateral-direction attacks. However, such a design faces several challenges:
(1) it is unclear how to practically fuse relative positioning sources such as LD with global positioning sources such as GPS and LiDAR locator without degrading MSF accuracy. Existing works are able to estimate an LD-based global localizations using semantic maps~\cite{kang2020lane, evlampev2020map}, however, their positioning accurate are at 0.5m level, which might severely degrade the resulting MSF accuracy (at centimeter-level~\cite{wan2018robust}) if got fused together;
(2) additional fusion inputs may be able to increase the robustness against \fr{} to certain degree, but it cannot fundamentally prevent \fr{}~\cite{fusionripper}. On the other hand, \ld{} is able to achieve perfect detection performance against \fr{} and can safely steer the AD vehicle to stop in the ego lane after detection.
}


% \textbf{Attack response after stopping.}
% Call support and police, reboot system, etc. 
% In our emergency trajectory generation (\S\ref{sec:design_ar}), we focus on the most basic requirements, i.e., slowing down and steering. However, real world drivings are more complicated in which the emergency trajectory may stop in the middle of an intersection. (This is still part of the applicable domain issue for lane boundaries). To address, a potential solution is to design a speed profile that consider that can allow the vehicle to pass the intersection before stops.

% should also consider the front obstacles. Generally it is not a concern since the AD planning maintains a distance larger than the stopping distance with the front vehicles in normal driving. However, in the cases such as a vehicle suddenly cuts in and stops, we can do nothing.. To address this, a straightforward solution is to reuse the planning logic in high-level AD systems for 

\cut{
\textbf{Alternative attack response goals.}
In the current \ld{} design, we set the AR goal as to stop with in lane boundaries as soon as possible. Although such an AR goal can prevent more serious consequences that could occur if the vehicle stops out of lane boundaries, e.g., being hit by another vehicle that failed to yield in time, it may be more desirable to pull over to the roadside or even keep driving in the ego lane if possible. However, such AR goals requires more sophisticated AR designs, and thus we leave them as future work.
}

% Instead of stopping in lane. Use the fused EH pose to navigate to pull over to the road side. 
\section{Related Work}
%\subsection{Cost Volume based Deep Stereo Matching}
%Stereo matching is a typical problem that has been studied for decades and a well-known four-step pipeline \cite{scharstein2002taxonomy} has been established, where cost volume construction is an indispensable step. Current state-of-the-art stereo matching methods are all cost volume based methods and they can be categorized into two types. Typically, a cost volume is a 4D tensor of height, width, disparity, and features. The first category just uses a full correlation to generate a single-feature cost volume. Such methods are usually efficient but lose much information because of the decimation of feature channels. Many previous work, including Dispnet \cite{dispnet}, MADNet \cite{madnet}, IResNet \cite{iresnet} and AANet \cite{aanet}, belong to this category. The second category usually uses concatenation \cite{gcnet} or group-wise correlation \cite{gwcnet} to generate a multi-feature 4D cost volume. Such a method can achieve better performance while requiring higher computational complexity and memory consumption. Actually, a majority of the top-performing networks in public leaderboards belong to this category, such as GANet \cite{ganet}, CSPN \cite{cspn} and ACFNet \cite{acfnet}. These methods generally employ multiple 3D convolution layers to constantly regularize the 4D cost volume and then apply softmax over the disparity dimension to produce a discrete disparity probability distribution. The final predicted disparity is obtained by softly weighting indices according to their probability, which is also called soft argmin in GCNet \cite{gcnet}. However, soft argmin leaves the output susceptible to multi-modal disparity probability distributions. ACFNet \cite{acfnet} observes this problem and proposes to directly supervise the cost volume with unimodal ground truth distributions. In contrast, we define an uncertainty estimation to quantify the degree to which the cost volume tends to be multi-modal distribution, higher implies the higher possibility of estimation error.

\subsection{Multi-scale Cost Volume based Stereo Matching}
Cost volume construction is an indispensable step in the well-known four-step pipeline for stereo matching \cite{scharstein2002taxonomy, pamisurvey1, pamisurvey2}. Typically, current state-of-the-art stereo matching methods can be categorized into two types of cost volume-based methods, where the cost volume is a 4D tensor of height, width, disparity, and features. The first category usually uses the single-feature 3D cost volume generated by full correlation, which is efficient while losing much information due to the decimation of feature channels. Many real-time methods, such as Dispnet \cite{dispnet}, MADNet \cite{madnet, madnet_pami} and AANet \cite{aanet}, belongs to the category. Moreover, two-stage refinement \cite{mcvmfc} and pyramidal towers \cite{madnet} are commonly applied in the single-feature cost volume based network to construct multi-scale cost volume. The second category usually uses the multi-feature 4D cost volume generated by concatenation \cite{gcnet} or group-wise correlation \cite{gwcnet}, which can achieve better performance with higher computational complexity and memory consumption. Most top-performing networks, including GANet \cite{ganet}, CSPN \cite{cspn} and ACFNet \cite{acfnet} belong to this category. 
% In these methods, the 4D cost volume is constantly regularized by multiple 3D convolution layers and then a discrete disparity probability distribution can be produced by softmax. Next, the final predicted disparity can be obtained by softly weighting indices according to their probability \cite{gcnet}. However, such output is susceptible to multimodal disparity probability distributions and ACFNet \cite{acfnet} gives a solution by directly supervising the cost volume with unimodal ground truth distributions to alleviate this problem. 
Recently, to alleviate the high computational complexity and memory consumption when employing multi-feature 4D cost volumes, \cite{cvpmvsnet, cascade, uscnet} propose to use cascade cost volume representation in multi-view stereo. These methods usually first predict an initial disparity at the coarsest resolution of the image and then gradually refine the disparity by narrowing down the disparity search space. More closely related to our approach is Casstereo \cite{cascade}, which first extended such representation to stereo matching. It selected to uniform sample a pre-defined range to generate the next stage’s disparity search range. Instead, we employ pixel-level uncertainty estimation to adaptively adjust the next stage disparity searching range and generate pseudo-labels for subsequent domain adaptation. Our method also shares similarities with UCSNet \cite{uscnet}, which constructs uncertainty-aware cost volume in multi-view stereo while it doesn’t employ uncertainty estimation to generate pseudo-labels.

%\subsection{Multi-scale Cost Volume based Deep Stereo Matching} 
% \subsection{Multi-scale Cost Volume based Stereo Matching} 
%Multi-scale cost volume firstly was applied in the single-feature cost volume based network with the form of two-stage refinement \cite{mcvmfc} and pyramidal towers \cite{madnet}. Recently, cascade cost volume representation \cite{cvpmvsnet, cascade, uscnet} was proposed in multi-view stereo to alleviate the high computational complexity and memory consumption when employing multi-feature 4D cost volumes. These methods generally predict an initial disparity at the coarsest resolution of the image. Then, they will narrow down the disparity search space and gradually refine the disparity. More closely related to our approach is Casstereo \cite{cascade}, which first extended such representation to stereo matching. It selected to uniform sample a pre-defined range to generate the next stage’s disparity search range. Instead, we employ uncertainty estimation to adaptively adjust the next stage pixel-level disparity searching range and push the next stage's cost volume to be predominantly unimodal.

% The single-feature cost volume based network with the form of two-stage refinement \cite{mcvmfc} and pyramidal towers \cite{madnet} first employ multi-scale cost volume for stereo matching. Recently, to alleviate the high computational complexity and memory consumption when employing multi-feature 4D cost volumes, \cite{cvpmvsnet, cascade, uscnet} propose to use cascade cost volume representation in multi-view stereo, which generally predict an initial disparity at the coarsest resolution of the image. Then, the disparity search space is narrowed down and the disparity is gradually refined. More closely related to our approach is Casstereo \cite{cascade}, which first extended such representation to stereo matching. It selected to uniform sample a pre-defined range to generate the next stage’s disparity search range. Instead, we employ uncertainty estimation to adaptively adjust the next stage pixel-level disparity searching range and push the next stage's cost volume to be predominantly unimodal.

% Figure environment removed

\subsection{Robust Stereo Matching} 
There exist three categories of generalization definitions for robust stereo matching. 1) Cross-domain Generalization: the network’s ability to perform well on unseen scenes (cannot see the image pairs of the target domain in advance). Towards this end, Jia et al \cite{sungeneralizaiton} propose to incorporate scene geometry priors into an end-to-end network. Zhang et al \cite{dsmnet} introduce a domain normalization and a trainable non-local graph-based filter to construct a domain-invariant stereo matching network. 2) Adapt Generalization: the network’s ability to adapt pre-trained models to the new domain with unlabeled target data. Previous work usually pre-trains the models on synthetic data and then adapts it to new target domains with Graph Laplacian regularization \cite{zoom}, non-adversarial progressive color transfer \cite{adastereo}, and Knowledge Reverse Distillation \cite{aohnet}. More closely related to our approach are \cite{aohnet, unsuperviseddomainadaptation} in stereo matching and Monoresmatch \cite{monoresmatch} in monocular depth estimation, which also proposes to generate a pseudo-label for domain adaptation. However, these methods all select to employ classical stereo matching methods \cite{sgm} alongside with confidence estimators, e.g., left-right consistency check to generate pseudo-labels. That is all these methods need an independent method to generate corresponding pseudo-labels. Instead, the proposed method is an end-to-end network that can generate the predicted disparity map, corresponding uncertainty map and pseudo-labels jointly, which is a more simple, yet efficient way. 
% Instead, our proposed method can employ pixel-level and area-level uncertainty estimation to self-distill the predicted disparity maps of our pre-training model and generate sparse while reliable pseudo-labels to align the domain gap, which is a more simple, yet efficient way. 
3) Joint Generalization: the network’s ability to perform well on a variety of datasets with the same model parameters. MCV-MFC \cite{mcvmfc} introduces a two-stage finetuning scheme to achieve a good trade-off between generalization and fitting capability on multiple datasets. However, it doesn’t touch the inner difference between diverse datasets, e.g, the unbalanced disparity distribution. To further address this problem, we propose a cascade cost volume to adaptively the next stage disparity searching space, where the pixel-level uncertainty estimation is at the core.

% \subsection{Monocular Depth Estimation}
% Monocular depth estimation aims to estimate depth values from a single image, instead of stereo images or multiple frames in a video. This problem is ill-posed because of the ambiguity of object sizes. However, humans could estimate the depth from a single image with prior knowledge of the scenes. Recently, learning based methods were explored to learn depth values by supervised or unsupervised learning. Eigen et al. first employed Convolutional Neural Networks (CNN) to predict depth in a coarse-to-fine manner and further improved its performance by multi-task learning. Liu et al. presented deep convolutional neural fields model by combining deep model with continuous CRF. Li et al. [22] refined deep CNN outputs with a hierarchical CRF. Multi-scale continuous CRF was formulated into a deep sequential network by Xu et al. [45] to refine depth estimation. Unsupervised methods tried to train monocular depth estimation with stereo
% image pairs or image sequences and test on single images. Garg et al. [9] used novel image view synthesis loss to train a depth estimation network in an unsupervised way. Godard et al. [11] introduced left-right consistency regularization to improve the performance of view synthesis loss. Recently, some work also propose to use the stereo matching network as a proxy to learn depth from synthetic data or directly employ traditional stereo matching methods to distill proxies labels from the target domain, which proves the feasibility of distilling stereo matching networks to learn monocular depth estimation.



\section{Conclusion and Future Work}
In this work, I design corruption-robust algorithms for the Lipschitz contextual search problem. I present the \emph{agnostic checking} technique and demonstrate its effectiveness in designing corruption-robust algorithms. There are several open problems for future research. First, in the algorithm I propose for pricing loss, the schedule for agnostic checks is fixed upfront. Can the learner design an adaptive checking schedule for the pricing loss? Second, this work assumes the learner has knowledge of the Lipschitz constant $L$. Can the learner design efficient no-regret algorithms without knowledge of $L$? 
\subsubsection*{Acknowledgments}
This publication was made possible by an ETH AI Center doctoral fellowship to Manish Prajapat. We would like to thank Mohammad Reza Karimi, Pragnya Alatur and Riccardo De Santi for the insightful discussions. We thank Bhavya Sukhija and Alizée Pace for reviewing the manuscript.

The project has received funding from the European Research Council (ERC) under the European Union’s Horizon 2020 research and innovation program grant agreement No 815943 and the Swiss National Science Foundation under NCCR Automation grant agreement 51NF40 180545.

%\newpage

\bibliographystyle{ieeetr}
{
\footnotesize
\bibliography{ref_ld3, ref_fusionripper, ref_msf_survey, ref_reviews}
}

\appendix
\begin{comment}
\section{System Architecture}
\label{appendix:architecture}
\system has a novel modularized system architecture with three key components: 
\emph{StreamManager}, 
\emph{TxnManager} and \emph{TxnScheduler}. 
These components are instantiated in each thread locally.
The execution outline of \system is presented in Algorithm~\ref{alg:algo}.
Transactional stream processing is continuous and potentially never ends (Line 1$\sim$8).
The dependency resolution and execution of state transactions are separated into two non-overlapping phases by punctuations~\cite{Tucker:2003:EPS:776752.776780} (Line 2 and 5), which guarantees that no subsequent input event will have a smaller timestamp. 
Effectively, a batch of state transactions is collected during the first phase, and processed during the second phase.

In the first phase (i.e., stream processing phase), 
the \emph{StreamManager} conducts preprocessing for every input event ($e$). Similar to some prior works~\cite{tstream}, state transactions may be issued but not immediately processed during preprocessing (Line 3).
The \emph{pre\_processing} and \emph{post\_processing} functions are exposed as APIs to users.
The \emph{TxnManager} handles dependency resolution (Line 4) among state transactions and insert decomposed operations to construct a \tpg. We discuss the detailed two-phase \tpg construction process in Section~\ref{subsec:construction}.

In the second phase  (i.e., transaction processing phase), 
the \emph{TxnManager} is first involved again to refine (Line 6) the constructed \tpg with further dependency resolution.
The \emph{TxnScheduler} 
schedules operations for concurrent execution based on the constructed \tpg according to the three dimensions of scheduling decisions (Line 7). 
In particular, a scheduling decision model $M$ is instantiated based on the constructed \tpg (Line 14).
\textbf{\circled{1}} Guided by $M$, execution threads adopt an exploration strategy (Section~\ref{subsec:explore}) to explore the constructed \tpg for operations available to be scheduled constrained by dependencies. 
\textbf{\circled{2}} 
During exploration, one or multiple operations may be treated as the 
% basic 
unit of scheduling (Section~\ref{subsec:granularity}). 
Subsequently, \textbf{\circled{3}} every thread executes operation(s) in the unit of scheduling with various abort handling mechanisms (Section~\ref{subsec:abort_handling}).
Only when state transactions are processed (i.e., committed or aborted) can the associated input events be postprocessed (Line 8) by the \emph{StreamManager} based on transaction processing results.
\end{comment}

\begin{comment}
\begin{algorithm}
\footnotesize
    \KwData{$e$ \tcp{Input event}}
    \KwData{$txn_{ts}$ \tcp{State transaction}}
    \KwData{$G$ \tcp{The currently constructed TPG}}
    \While{!finish processing of input streams}{
        \eIf(\tcp*[h]{Phase 1}){\text{$e$ is not a $punctuation$}}{
                $txn_{ts}$ $\gets$ PRE\_Processing($e$)\;
                \textbf{TPG\_Construction}($G$, $txn_{ts}$)\; 
          }(\tcp*[h]{Phase 2}){
                \textbf{TPG\_Refinement}($G$)\; 
                \textbf{TXN\_Scheduling}($G$)\; 
                POST\_Processing()\;
          }
    }
    
    \SetKwFunction{FMain}{TPG\_Construction}
    \SetKwProg{Fn}{Function}{:}{}
    \Fn{\FMain{$G$, $txn_{ts}$}}{
        $O_{1..k}$ $\gets$ \textbf{Partition} $txn_{ts}$\;
        \ForEach{\text{operation $O_{i}$ $\in$ $O_{1..k}$}}{
            \textbf{Identify} its \ld\;
            $G$ $\gets$ $G$ + $O_{i}$ \;
        }
    }
    \SetKwFunction{FMain}{TPG\_Refinement}
    \SetKwProg{Fn}{Function}{:}{}
    \Fn{\FMain{$G$}}{
        \ForEach{\text{vertex $e_{i}$ $\in$ $G$}}{
            \textbf{Identify} its \td, \pd\;
        }
    }
    
    \SetKwFunction{FMain}{TXN\_Scheduling}
    \SetKwProg{Fn}{Function}{:}{}
    \Fn{\FMain{$G$}}{
        $M$ $\gets$ Instantiated with $G$;\tcp{A decision model}
        \While{!finish scheduling of $G$
        }{
          \textbf{\circled{2}} $Scheduling Unit$ $\gets$ \textbf{\circled{1}} \emph{Explore}($G$, $M$)\; 
            \textbf{\circled{3}} \emph{Execute with Abort Handling} ($Scheduling Unit$)\; 
        }
    }
  \caption{Execution Outline of \system}
  \label{alg:algo}
\end{algorithm}
\end{comment}


%\newpage
%%\iffalse
\documentclass[11pt]{article}
\usepackage[margin=1in, top = 0.85in, letterpaper]{geometry}
\usepackage{lineno}
\usepackage{amsthm}
\newtheorem{theorem}{Theorem}
\newcommand{\cS}{\mathcal{S}}
\newcommand{\cD}{\mathcal{D}}
\newcommand{\cE}{\mathcal{E}}
\linenumbers


\makeatletter
\def\thm@space@setup{\thm@preskip=0pt
\thm@postskip=0pt}
\makeatother
\newtheoremstyle{newstyle}      
{} %Aboveskip 
{} %Below skip
{\mdseries} %Body font e.g.\mdseries,\bfseries,\scshape,\itshape
{} %Indent
{\bfseries} %Head font e.g.\bfseries,\scshape,\itshape
{.} %Punctuation afer theorem header
{ } %Space after theorem header
{} %Heading

\theoremstyle{newstyle}
\newtheorem{thm}{Theorem}[]
\newtheorem{prop}[thm]{Proposition}
\newtheorem{lem}{Lemma}
\newtheorem{cor}{Corollary}

\makeatletter
\newenvironment{pf}[1][\proofname]{\par
  \pushQED{\qed}%
  \normalfont \topsep0\p@\relax
  \trivlist
  \item[\hskip\labelsep\itshape
  #1\@addpunct{.}]\ignorespaces
}{%
  \popQED\endtrivlist\@endpefalse
}
\makeatother

\begin{document}

\noindent I thank all reviewers for the very insightful comments! I first give two lower bounds for pricing loss, showing the upper bounds (Thm 13, Cor 15) are essentially tight. Then I address more specific questions from reviewers. %Also, all reviewers noticed that in Alg 1, the else block containing MidPointQuery in fact should have one less indent, so this block is executed when both endpoints have been queried. %All reviewer's comments will be considered when updating the paper. 

%Due to space constraints, I cannot respond to all, but I believe I responded to the most important comments that will (positively) change your opinion. 

\noindent\textbf{Lower Bound}
In the following assume $L=1$. The first lower bound essentially shows it is impossible to achieve a bound of the form $O(T^{d/(d+1)}) + C\cdot o(T^{1/(d+1)})$ when $C$ is unknown. 

\vspace{0.02in}
\begin{thm}
Let $A$ be any algorithm to which the corruption budget $C$ is unknown. Suppose $A$ achieves a cumulative pricing loss $R(T) = o(T)$ when $C = 0$. Then, there exists some corruption strategy with $C = 2R(T)$, such that the algorithm suffers $\Omega(T)$ regret. 
\end{thm}
\begin{pf}
Let us consider two environments. In the first environment, the adversary chooses $f(x)\equiv 0.5$ and never corrupts the signal. In the second environment, the adversary chooses $f(x)\equiv1$ and corrupts the signal whenever the query is above 0.5. Now since the algorithm achieves regret $R(T)$ when $C=0$, then, when $f(x) = 0.5$, the seller can only query values above $0.5$ for at most $2R(T)$ times. However, by choosing $C = 2R(T)$, the adversary can make the two environments indistinguishable from the seller. Hence the seller necessarily incurs $\Omega(T)$ regret. 
\end{pf}

%effectively manipulating the seller into believing the true price is 0.5. 

Next, we will prove a lower bound against randomized adversary when corruption budget $C$ is known. Define an environment as the tuple $(f, C, \cS)$, representing the function, corruption budget, and corruption strategy respectively. The adversary draws an environment from some probability distribution $\cD$, and the corruption budget is defined as the expected value of $C$ under distribution $\cD$. Note that the upper bounds (Thm 13, Cor 15) also hold for randomized adversaries. 
\vspace{0.02in}
\begin{thm}
There exists some $\cD$ where any algorithm incurs expected regret $\Omega(\sqrt{CT})$. 
\end{thm}
\begin{pf}
The adversary uses two environments $\cE_{\{1,2\}}$. In $\cE_1$, the function $f(x) \equiv 0.5$ and the corruption budget is 0. In $\cE_2$, the function $f(x) \equiv 1$, the corruption budget is $C_0$, and the adversary corrupts any query above 0.5 until the budget is depleted. The adversary chooses the first environment with probability $1-p$ and the second environment with probability $p$. If the learner's algorithm queried more than $C_0$ rounds above $0.5$, then the learner incurs regret $0.5C_0$ in $\cE_1$, giving expected regret $\Omega((1-p) C_0)$. If the algorithm did not query more than $C_0$ rounds above $0.5$, then the algorithm incurs regret $0.5(T - C_0)$ in $\cE_2$, giving expected regret $\Omega((T-C_0)p)$. Choosing $p = \sqrt{C/T}, C_0 = \sqrt{CT}$ completes the proof. 
% Consider the following corruption strategy. The space is discretized into $\sqrt{T/C}$ hypercubes of equal length. The context that the adversary selects will be the center of each hypercube, and the value at each context will be 0.5 except for one context which has value $0.5 + \eps$. Each hypercube will be selected by the adversary for $\sqrt{CT}$ rounds. The adversary will corrupt the first $C$ queries in the hypercube which has value $0.5+\eps$. At a high level, the learner can only detect the hypercube with value $0.5 + \eps$ if he performs checking rounds in every hypercube for at least $C$ rounds. Alternatively, if the learner does not perform checking query, he will incur regret $\eps \sqrt{CT}$. 
% Can we choose there to be $1/\eps = \sqrt{T/C}^{1/d}$ hypercubes that has value $0.5+\eps$? 
\end{pf}


\noindent\textbf{Reviewer 1.} The idea is indeed natural and simple, but the analysis is non-trivial and quite interesting. In the proof for absolute loss in App A, Lemma 17,18 (on MidpointQuery), Lemma 25 and part of proof in Thm 1 (on regret of safe intervals) are borrowed from Mao et. al. The introduction of the three types of intervals and Lemma 19-24 (on various properties of the three types) are entirely new. The analysis for pricing loss is also entirely new (Sec 4.2, App B). Sec 4.1 is adapted from Mao et. al. but leaves out the bisection part as it starts with a uniform discretization. I will add more careful remarks pointing out which results are from previous work. Krishnamurthy et. al. and Leme et. al. study corruption-robust linear contextual search. Their methods are respectively: maintain a knowledge set that contains the true unknown parameter $\theta$; maintain a density over the parameter space. By contrast, our setting is non-parametric and our method is fundamentally different. The different initialization for $d> 1$ does not change the final regret bound, it just makes writing some steps in the proof slightly easier (last sentence in App A.2). For Thm 27, this is because a hypercube at depth $h$ incur loss $2^{-h}$. 

\noindent\textbf{Reviewer 2.} I believe your main concerns are addressed in the lower bound above. For Alg 1, the final else block indeed should have one less indent. 

\noindent\textbf{Reviewer 3.} Indeed, the best the learner can say is that safe intervals are not marked \emph{unsafe}, but not all corrupted or correcting intervals will be marked \emph{unsafe}. Maybe \emph{surprising} is a good choice (also consistent with Sec 4 for pricing). The else block containing MidPointQuery in fact should have one less indent, so this block is executed when both endpoints have been queried. For your final observation, I believe a form of soft-checking with a weight-update may be a suitable solution. 



%Specifically, this means when the first context arrives in some interval, the learner queries an endpoint of $S_j$ (say the higher endpoint); and later in a subsequent round when a context arrives in this interval, the learner queries the other endpoint. 

\end{document}

%\fi
%%%%%%%%%%%%%%%%%%%%%%%%%%%%%%%%%%%%%%%%%%%%%%%%%%%%%%%%%%%%%%%%%%%%%%%%%%%%%%%%
\end{document}
%%%%%%%%%%%%%%%%%%%%%%%%%%%%%%%%%%%%%%%%%%%%%%%%%%%%%%%%%%%%%%%%%%%%%%%%%%%%%%%%

%%  LocalWords:  endnotes includegraphics fread ptr nobj noindent
%%  LocalWords:  pdflatex acks

