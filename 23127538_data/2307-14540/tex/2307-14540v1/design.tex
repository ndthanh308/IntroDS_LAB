% \vspace{-0.05in}
\nsection{Novel LD-based Defense Design: \ld{}} \label{sec:design}

% To leverage the defense-suitable information source and address challenges \textbf{C2} and \textbf{C3}, we propose 

Considering the multi-dimensional defense opportunities above, in this paper we are motivated to design the first domain-specific lane detection-based defense approach against lateral-direction AD localization attack, named \textit{\textbf{\ldi{}} (\underline{L}ane \underline{D}etection based \underline{L}ateral-\underline{D}irection \underline{L}ocalization attack \underline{D}efense)}. In this section, we first describe the associated design challenges and then present the design details.



%In this section, we present our defense design, \textbf{\ld{}}, which stands for \underline{L}ane \underline{D}etection based \underline{L}ateral-\underline{D}irection \underline{L}ocalization attack \underline{D}efense.
% Specifically, our design leverages the LD to perform both \textit{detection} and \textit{response} to the lateral-direction localization attacks targeting high-level AD systems.

\nsubsection{Design Challenges} \label{sec:design_challenges}

Although LD comes with various defense opportunities, systematically leveraging it for AD localization defense purposes still needs to address the following design challenges:

%can readily provide information directly related to lateral-direction attacks (i.e., lateral deviation to lane departure)

\textbf{C1: Non-trivial design details for attack detection.} Although at the high level LD can provide information directly related to lateral-direction attacks (i.e., lateral deviation to lane departure), at the detailed defense design level there are still many technical challenges we need to address, for example (1) incompatibility of the coordinate systems, i.e., LD is by default in \textit{local} positioning coordinate system (i.e., within the ego lane), while the attack is in \textit{global} coordinate system (i.e., the world coordinates); (2) choice of the attack-influenced information level for attack detection, e.g., directly at the spoofed GPS signal level or at the attack-influenced MSF output level; and (3) sufficient robustness to natural LD inaccuracies in practice, e.g., missing or incorrect detection, for minimizing possible false positives in attack detection.


%Since LD measures AD vehicle's relative position in the ego lane, at the high level it seems to be straightforward to use it to detect any unexpected lateral deviations in localization. However, we find there are still quite some non-trivial detailed design questions that are critical to the detection performance in practice, for example at which attack-influenced information level should the detection be performed, e.g., at the GPS signal or the MSF output level; and how to improve the robustness to natural LD inaccuracies, e.g., missing or incorrect detection.

%how to handle the incompatibility of the coordinate systems between LD and AD localization; 



%since LD is not readily used in high-level AD localization (\S\ref{sec:opportunity}), there are still many critical design  that can directly impact the detection effectiveness in practice, for example (1) incompatibility of the coordinate systems, for which LD is by default in \textit{local} positioning coordinates (i.e., within the ego lane), while the direct target of existing attacks is in \textit{global} coordinates (i.e., the world coordinates); (2) choices of the attack-influenced information level for attack detection, e.g., directly at the spoofed GPS signal level or the attack-influenced MSF output level; and (3) robustness to natural LD inaccuracies, e.g., missing or incorrect detection.


\textbf{C2: Need for AD-specific attack response design.} Since high-level AD vehicles are travelling at high speed and by design cannot assume on-board human driver ready for take-over at any time (already the case in some commercial AD services~\cite{waymo_driverless, baidu_driverless_robotaxi}), it is necessary to further design an attack response step that can (1) minimize the safety risks during response, and (2) assume no dependence on human assistance. For small robotic vehicles such as drones and rovers, prior works have considered using state estimation models to replace the attacked physical sensor after attack detection~\cite{choi2020software, zhang2020real}. However, such methods still count on human operators to take over as soon as possible since such state estimations cannot replace physical sensors for a prolonged duration due to drifting~\cite{choi2020software}, not to mention that such models are suffering from much more severe motion model accuracy limitations when applied to the AD context (\S\ref{sec:eval_detection}). Thus, a new design is needed to achieve our AD-specific response goal above. 


%For small robotic vehicles such as drones and rovers, various prior works propose attack detection-only defense solutions, which are indeed already practically useful since the attack can be directly handled by emergency stop and/or human operators. However, in high-level AD settings, we find attack detection alone is not enough in practice, since AD vehicles are travelling in much higher speed and by default there does not exist on-board human driver ready for taking over the driving all the time. Thus, in AD context it is \textit{necessary} to design an automatic and safe attack response method following up the attack detection.


\textbf{C3: Adaptive attack from LD side.} While LD is currently independent to existing high-level AD localization attacks due to the lack of use (\S\ref{sec:opportunity}), our defense-purpose use of it in \ld{} is inherently introducing a new attack surface from the LD side. In fact, recent works have already discovered concrete lateral-direction attacks against LD in production AD context~\cite{sato2021dirty}. To systematically account for such inherent adaptive attack surface, our defense design thus needs to consider the more challenging setup where both the attack detection and response designs cannot simply assume the LD side is trustworthy (and use it as the benign reference accordingly) when its outputs are inconsistent with the AD localization side.

%; instead, we need to consider the more challenging setup that both sides 

\cut{

thus needs to further avoid simply taking the LD-side inputs as the benign value when they are in conflict 


faces the challenge that we cannot simply trust the LD-side input in both attack detection and response 

when it is in conflict with the localization




The usage of LD in the defense unavoidably exposes it to potential LD-side attacks, e.g., DRP attack~\cite{sato2021dirty}. Therefore, it is necessary to design the defense such that it is also resilient to the new threats.



\ld{} leverages the camera-based LD to facilitate both attack \textit{detection} and \textit{response}. 
Since LD measures AD vehicle's relative position in the traffic lane, it is thus straightforward to use it to detect any unexpected lateral deviations in localization.
However, to utilize LD for defense purposes, we recognize several design \textit{challenges}, for which we need to weigh between potential solutions to determine a practical defense design.
\vspace{-\topsep}
\begin{itemize}
\setlength{\itemsep}{0pt}
\setlength{\parskip}{0pt}
    \item \textit{C1: At which attack-influenced information level should we perform the detection?} Since existing lateral-direction localization attack leverages GPS spoofing as an attack vector to affect the MSF outputs, one has to determine whether to perform the detection at the GPS or at the MSF output level.
    \item \textit{C2: Besides attack detection, an AD-specific attack response design is required.} Although there exists prior attack recovery methods~\cite{choi2020software, zhang2020real} for drones and rovers, which use state estimation models as sensor replacement during recovery, their methods suffer from the same model accuracy issue as in SAVIOR~\cite{savior} and CI~\cite{ci} (more details in \S\ref{sec:eval_detection}). In addition, they often intend to maintain a normal system operation during attack recovery and assume a human driver would take over the control afterwards. However, this is not the case for high-level AD vehicles, where typically no safety drivers will be onboard once commercially deployed~\cite{baidu_driverless_robotaxi, waymo_driverless}. Lastly, comparing to drones and rovers, the more complex driving environment and higher driving speeds demand an AD-specific response design that reflects the safety-first principle~\cite{ad_safety_first}. 
    % Thus, a systematic attack response design is especially necessary for high-level AD systems.
    \item \textit{C3: How to make the defense resilient to adaptive attacks that target LD?} The usage of LD in the defense unavoidably exposes it to potential LD-side attacks, e.g., DRP attack~\cite{sato2021dirty}. Therefore, it is necessary to design the defense such that it is also resilient to the new threats.
    % Therefore, it is important to design the defense
    \vspace{-\topsep}
\end{itemize}
}


\vspace{-0.05in}
\nsubsection{Design Overview} \label{sec:design_overview}
% \vspace{0.05in}


In this section, we explain each design component in \ld{} and how they address above design challenges. Fig.~\ref{fig:design_overview} shows an overview of \ld{} fitted in a typical high-level AD system.



\begin{table*}[tbp]
\footnotesize
\begin{minipage}{0.58\linewidth}
	\centering
    % Figure removed
    % \vspace{-0.25in}
    \captionof{figure}{Overview of \ld{} design integrated in a typical high-level AD system. New components are highlighted in yellow.}
    \label{fig:design_overview}
\end{minipage}\hfill
\begin{minipage}{0.39\linewidth}
	\centering
    % Figure removed
    % \vspace{-0.25in}
    \captionof{figure}{Illustration of safety-driven fusion in the attack response (\S\ref{sec:design_ar}).}
    \label{fig:safety_driven_fusion}
\end{minipage}
\vspace{-0.2in}
\end{table*}


\cut{
% Figure environment removed
}


% \textbf{Attack detection via MSF/LD cross-checking.}
\textbf{Attack detection at MSF output level.} As shown in Fig.~\ref{fig:design_overview}, the attack detection step is performed in the localization module to constantly check the consistency between the LD outputs and original localization output and raise anomalies use popular anomaly detectors such as CUmulative SUM (CUSUM). To address the incompatibility of their coordinate systems mentioned in \textit{C1}, we convert both into a unified lateral deviation representation w.r.t. the \textit{lane centerline} since that's directly related to the lateral-direction attack goal. Regarding the choice of the attack-influenced information level for attack detection, we choose to detect at the MSF output level rather than at the GPS output level since (1) in normal conditions, GPS positions can naturally have large noises while MSF outputs are at centimeter-level accuracy~\cite{wan2018robust}. Thus, performing the detection at the MSF level can better reduce false positives; and (2) detecting at the MSF output level also allows \ld{} taking advantage of the \textit{opportunistic} property of \fr{}, for which the attacker cannot predict where and when MSF will exhibit large deviations. This thus can make it much more difficult for the attacker to easily bypass the detection by targeting locations without lane line markings. We also have designs for addressing false positives from common lane detection inaccuracies.


%After that, we feed their residuals to an anomaly detector, e.g., CUmulative SUM (CUSUM), to check if any large discrepancy exists and raise anomaly if so.



%post-processing step in the localization module to constantly check the consistency between the MSF and LD outputs. 


%leverage the lateral deviations perceived by LD to perform the attack detection, which is designed as a post-processing step in the localization module to constantly check the consistency between the MSF and LD outputs. 


%The attack detection is conducted by checking the consistency of lateral deviations in the MSF and LD outputs. This is because the lateral deviation in the attacked MSF will be translated into physical world deviations \textit{in the opposite direction} due to AD control~\cite{fusionripper}, and LD measures the relative positions of the lane lines to the vehicle in the physical world.
%While in the benign cases, the lateral deviation in the MSF is a natural outcome of the vehicle's physical world lateral movement, and thus the lateral deviations from LD and MSF will be consistent.
%However, as LD and MSF localization outputs are in different coordinate frames (i.e., \textit{local} versus \textit{global} as described in \S\ref{sec:design_challenges}), we need to first convert them to a unified local frame w.r.t. the \textit{lane centerline}.
% using the \textit{semantic map}, which is a pre-built map on high-level AD systems storing the road geometry information. 



%\newparts{In particular, for \textit{C1}, we opt to detect the attack at the MSF output level rather than at the GPS output level for two reasons. (1) Since in normal conditions, GPS positions can naturally have large noises while MSF outputs are at centimeter-level accuracy~\cite{wan2018robust}, perform the detection at the MSF level will incur fewer false positives. (2) Detecting at the MSF output level also allows \ld{} taking advantage of the \textit{opportunistic} property of \fr{}, for which the attacker cannot predict where and when will MSF exhibit large deviations. This practically makes it much more difficult for the attacker to target locations without lane line markings.}

\textbf{Attack response via safe in-lane stopping.}
As discussed in \textit{C2}, we need a new AD-specific design for the Attack Response (AR) step. There are several common choices in human driving if the vehicle navigation is malfunctioning, for example maintaining driving in the current lane waiting for the system to recover, or pulling over to the road side. However, these cannot apply to the context of AD localization attacks, since without knowing the accurate real-time location, we cannot even know how to safely and correctly drive in the current lane or to road side. We also cannot blindly count on the LD outputs to drive due to the need to account for the adaptive attack surface on the LD side (\textit{C3}). Thus, we consider the safest AR choice is to try to safely stop in the current lane, which has the minimal reliance on the attack-time localization accuracy for maximizing safety in the AR period. More importantly, on the attack side, since this minimizes the attackable duration after detection, it can fundamentally \textit{bound} the attack-achievable deviation in AR period. Even though stopping in the ego lane is not ideal, it is commonly recognized~\cite{nhtsa_standard} as one of the fallback strategies to transition to a minimal risk condition when the AD vehicle cannot operate safely. In most driving scenarios, stopping in the ego lane shall not cause a collision as long as the tailgating vehicle is driving with safe following distance and speed, which is much safe than driving out of the ego lane.

%If the detection can be sufficiently early when the attack-introduced deviation is small, such bounding can be more effective since the attack deviation is exponentially growing.



%if the attack can be detected sufficiently early when the attack-introduced deviation is not that large, the achievable deviation



%AR for localization attacks is uniquely challenging in that since we are unsure about the correct location, we do not even know where the roadside is. Thus, we need to minimize the 

%To address \textit{C2}, we propose an Attack Response (AR) design to control the vehicle to immediately stop in the current lane. 
% In this work, we focus on this simplest AR goal--stopping in the current lane.
%\newparts{Generally, AR can be designed with goals of various design complexities, e.g., pulling over to the roadside, maintaining the same speed in the current lane, etc. Since more complex AR goals require more advanced AR designs and may impose higher safety risks, in this paper, we focus on achieving the simplest AR goal, i.e., stopping in the current lane.}


% For the longitudinal control, it is straightforward to apply a safe deceleration to reduce the speed to zero. The lateral control is more challenging as the localization outputs are required to correct the vehicle back to the planned trajectory. Since either MSF or LD can be under attack (\S\ref{sec:motivation_challenges}), we apply a fusion-based design to combine MSF/LD lateral to achieve a middle ground between the attacked and the benign one to steer the vehicle in the lane boundaries before it fully stops.

%\alfred{we should introduce attack-resilient sensor fusion? This is essentially what you are doing, and you are not the first to consider fusion design under adaptive attack scenario. But need to check the definition of attack-resilient sensor fusion first to see if your design fits that definition; if so, then you are a safety-driving attack-resilient sensor fusion.}
\textbf{Safety-driven fusion for adaptive LD attack.} Although the in-lane stopping AR strategy can already bound the attack-achievable deviation, it is still highly desired if we can minimize the attacker's impact on the localization accuracy during the AR period, since to safely stop, there is still a long stopping distance that the ego vehicle has to travel, especially when the speed is high (e.g., over 50 meters at 60 mph~\cite{stopping_distance}). To account for the adaptive LD attack surface (\textit{C3}), the key challenge is how to decide which side (LD or MSF) to trust when they are conflicting with each other in AR. Motivated by the safety-first principle in production AD design~\cite{ad_safety_first}, we propose a novel \textit{safety-driven fusion} design, which systematically decides the contributions from different fusion inputs based on their tendencies to cause unsafe driving; the higher such tendency is, the smaller their contributions will be to the final fusion output. In our problem context, such a tendency is judged by the deviation aggressiveness to cause lane departure, which will thus by default penalize the attacked side no matter it is from LD or MSF, leading to less attack-introduced deviation. To bypass this penalty and more effectively influence the fused results, the attacked side has to be less aggressive in lateral deviations. However, given the limit on the attackable duration imposed by the in-lane stopping AR strategy, the attack-achievable deviation during AR will still be reduced. Thus, under our AR design that bounds the attackable duration, such safety-driven fusion design can further fundamentally reduce the attacker's capability in causing safety damages during AR even in adaptive settings. 




%With such a design, an attacked source needs to be less aggressive in order to gain higher weights in the fusion. Yet, less aggressive means that the lateral deviation it can cause in the AR period will also be limited.

%\newparts{
%No matter which AR goal we enforce, the AD system has to maintain correct driving for a certain period before stopping, especially when speed is high. Thus, at least one positioning source (local or global) is required for navigation. As mentioned in \textit{C3}, since the detection may also be triggered by LD-side attacks, we cannot simply throw away MSF outputs and purely rely on LD for lane keeping. 
%Therefore, the dilemma is to decide which side to trust when they are conflicting with each other in AR. To address that, a common design in robotics is to fuse all sources but assign uncertainties to indicate the noisy levels of the sources~\cite{thrun2005probabilistic, friedland2012control}.
%Although both MSF and LD algorithms outputs uncertainty measurements in some forms, such measurements may as well be influenced by the attack. Therefore, motivated by the safety-first principle in AD design~\cite{ad_safety_first}, we propose a novel a \textit{safety-driven fusion} design, which estimates the trustworthiness of both sources by their safety implication in the driving context to prevent the fusion results being biased towards the potentially unsafe source.}

% a systematic trust/uncertainty measurement method should be considered in the attack response design to prevent the fusion results being biased towards the attacked source.

% As mentioned in \S\ref{sec:motivation_challenges}, since the fusion sources, i.e., MSF and LD, are likely conflicting with each other due to the attack, our fusion design thus needs to assign different trust or uncertainty levels to the sources such that the fusion results will not be biased towards the attacked one. 

%To address \textit{C3}, we incorporate a \textit{Kalman Filter} (KF) based fusion design, which determines the contributions of the MSF and LD using \textit{uncertainties}. However, we cannot directly use the uncertainties output from the MSF or LD algorithms as they can be potentially manipulated by the attacker. Thus, we design a customized uncertainty calculation based on the cumulative lateral deviations calculated from both sources to penalize the \textit{more aggressive} one during the fusion. With such a design, an attacked source needs to be less aggressive in order to gain higher weights in the fusion. Yet, less aggressive means that the lateral deviation it can cause in the AR period will also be limited.

\vspace{-0.05in}
\nsubsection{Attack Detection Design} \label{sec:design_detection}
% \vspace{0.05in}

\begin{algorithm}[tbp]
\footnotesize
\caption{Attack detection by checking the consistency between MSF and LD}
\label{alg:attack_detection}
\textbf{Notations:} $MSF$: MSF position output; $LD$: lane detection output; $S$, $b$, $\tau$: CUSUM statistic, weight, anomaly threshold; $D$: deviation to lane centerline; $lw_{\text{map}}$: lane width from semantic map\\
\textbf{Initialize:} $S_0\gets 0$
\begin{algorithmic}[1]
\ForEach{new lane detection output $LD_i$} \Comment{e.g., runs at 20Hz} 
    \State $MSF_i \gets \text{latest MSF position}$ \Comment{MSF is often more frequent than LD}
    \State $D_i^{\text{MSF}}$ $\gets$ \text{M\textsc{ap}}\text{L\textsc{ane}}\text{D\textsc{ev}}($MSF_i$) \Comment{MSF dev. (Appendix~\ref{app:map_apis})}
    \State $lw_{\text{map}}$ $\gets$ \text{M\textsc{ap}}\text{L\textsc{ane}}\text{W\textsc{idth}}($MSF_i$) \Comment{lane width (Appendix~\ref{app:map_apis})}
    \State $D_i^{\text{LD}}$ $\gets$ \text{L\textsc{d}}\text{D\textsc{ev}}($LD_i$, $lw_{\text{map}}$) \Comment{LD dev. to centerline (Alg.~\ref{alg:ld_dev_calc})}
    \State $S_i \gets$ max(0, $S_{i-1} + \lvert D_i^{\text{MSF}} - D_i^{\text{LD}} \rvert - b$) \Comment{calc. CUSUM statistic}
    \If{$S_i > \tau$}
        \State \text{attacked} $\gets \text{true}$ \Comment{report under attack if over threshold}
        \State \textbf{break}
    \EndIf
\EndFor
\State $\Rightarrow$ \textit{switch to attack response}
\end{algorithmic}
\end{algorithm}

As described above, we choose to perform the attack detection at the MSF output level, which is thus designed as a post-processing step in the localization module as shown in Fig.~\ref{fig:design_overview}.
%we leverage the lateral deviations perceived by LD to perform the attack detection, which
% the attack detection steps will introduce two new components to the high-level AD systems: \textit{lane detection (LD)} and \textit{attack detection}.
% \textbf{Attack detection.}
%the attack detection is designed as a post-processing step in the localization module to constantly check the consistency between the MSF and LD outputs. 
% Since MSF and LD usually operates at different frequencies, e.g., Baidu Apollo runs MSF localization at 100 Hz and LD at 20 Hz, we select the less frequent one to trigger the attack detection.
The detection algorithm is shown in 
Alg.~\ref{alg:attack_detection}.
% Since MSF outputs positions in the global coordinate system and LD measures the relative distance of the vehicle to the lane lines in the local frame, their results are not directly comparable. 
As mentioned in \textit{C1}, the MSF and LD outputs are in different coordinate systems. 
Therefore, we first need to convert them to a unified coordinate system such that they are comparable. 
% the MSF outputs to the same local frame as the LD using the semantic map in high-level AD systems. 
For MSF outputs, we obtain an \textit{MSF-based lateral deviation to the lane centerline} ($D_i^{MSF}$ in Alg.~\ref{alg:attack_detection}) by querying the MSF position in the semantic map~\cite{lyft_semantic_map}, which is a standard utility on high-level AD systems storing the road geometry information of the area that the AD vehicle is allowed to drive.
% Appendix~\ref{app:map_apis} lists the semantic map APIs required in \ld{} that is generally available in high-level AD systems. 
For the LD outputs, we can calculate the lateral deviation to the centerline based on the left and right lane line polynomial functions (detailed in Appendix~\ref{app:design_impl}). 
However, real-world lane markings can be complicated and confusion sometimes. For example, it is common to find that one of the lane lines missing or incorrectly detected in regions with lane splitting and merging. Therefore, we design two optimizations to calculate a more robust lateral deviation from the LD outputs leveraging the lane width from the semantic map (detailed in Appendix~\ref{app:design_impl}), which is a problem-specific improvement opportunity since in the main LD usage domain, low-level AD systems, such semantic maps are not generally available.
Since \ld{} relies on the existence of lane line markings, we disable the attack detection prior to entering these regions based on the information from the semantic map.

% Specifically, to handle cases when one of the lane lines is missing or incorrectly detected, we apply two optimizations to calculate a more robust lateral deviation implementation details in \S\ref{sec:design_impl}.


% After obtaining the MSF- and LD-based lateral deviations, we can then use their deviation consistency to determine if the MSF localization is under attack. This is because the attack-introduced lateral deviation in the MSF results in a physical world deviation to the opposite direction, which will be observed by the LD. While in the benign cases, the lateral deviation in the MSF is a natural outcome of the vehicle's physical world lateral movement, and thus the lateral deviations from LD and MSF will be consistent.

After obtaining the MSF- and LD-based lateral deviations, we can then use their deviation consistency to determine if MSF localization is under attack. To do so,
% To check the consistency between the MSF and LD lateral deviations, 
we apply the widely-used CUSUM anomaly detector (line 6--10 in Alg.~\ref{alg:attack_detection}),
which has shown high detection effectiveness in prior works~\cite{urbina2016limiting, savior}.
The CUSUM detector calculates a statistic $S_{i} = max(0, S_{i-1} + \lvert r_i \rvert - b); S_0 = 0$, where $r_i = D_i^{\text{MSF}} - D_i^{\text{LD}}$ is the residual between the MSF and LD lateral deviations, $b$ is a weight to prevent the CUSUM statistic from monotonically increasing in the benign scenarios. We consider as under attack if $S_i$ is over a certain threshold $\tau$.
Once an attack is detected, we then switch to the Attack Response stage.

\vspace{-0.05in}
\nsubsection{Attack Response Design} \label{sec:design_ar}
% \vspace{0.05in}

% The Attack Response (AR) stage is separated into two components, with one in the localization module and another in the planning module as shown in Fig.~\ref{fig:design_overview}. In the current AR design, we focus on achieving the most basic AR goal to stop within the current lane boundaries.

As described in~\S\ref{fig:design_overview}, we consider safe in-lane stopping as the safest AR choice. 
%Besides basic emergency operations such as turning on hazard warning lights to signal surrounding vehicles, the AR stage is designed to properly handle our attack response goal: safely stop in the current lane (\S\ref{sec:design_overview}). 
As shown in Fig.~\ref{fig:design_overview}, the AR is composed of two components to safely drive the vehicle before stop: (1) AR trajectory generation on the planning side, and (2) safety-driven fusion of MSF and LD on localization side.

% also need a systematic AR design to control the vehicle to stay in the current lane. In high-level AD systems, the longitudinal and lateral controllers calculate throttling and steering commands based on the current position and a planned trajectory. However, during AR, we can no longer purely rely on the MSF localization for positioning and the planned trajectory also needs to be updated to reflect the intended AR trajectory. Thus, we design a \textit{safety-driven fusion} component in the localization module and an \textit{emergency trajectory generation} component in the planning module as shown in Fig.~\ref{fig:design_overview} to achieve our AR goal: stopping within the current lane boundaries.


\textbf{Planning-side AR: AR trajectory generation.} 
%To realize the AR strategy, we need to design an AR planning trajectory such that the lateral and longitudinal controllers can use as reference to enforce the AR goal.
%Since our AR goal is to stop in the ego lane, we design the AR trajectory to be aligned with the lane centerline. To reduce the speed, we then set a slowing-down speed profile on the AR trajectory based on a safe deceleration value used in high-level AD systems.
The planning module in the high-level AD system periodically generates planned trajectories, which the controllers take as speed and lateral position references to produce throttling and steering commands. Thus, to enforce the AR goal, the planning module needs to generate an AR trajectory with a stopping motion. 
Since our AR goal is to stop in the ego lane, we designed the AR trajectory to be aligned with the lane centerline. To reduce the speed, we then set a slowing-down speed profile on the AR trajectory based on a safe deceleration value used in high-level AD systems. Generally, a deceleration $<$4.6 $\mathrm{m/s^2}$ is considered as safe for maintaining steady control~\cite{deceleration}. Thus, to calculate the speed profile of the AR trajectory, we apply 4 $\mathrm{m/s^2}$ as deceleration, which is also defined in Baidu Apollo as the maximum allowed deceleration to ensure safety~\cite{apollo}. 
%To do that, we generate a speed profile of the AR trajectory using the maximum safe deceleration for vehicles (Appendix~\ref{app:design_impl}). 
% Generally, a deceleration $<$4.6 $\mathrm{m/s^2}$ is considered as safe for maintaining steady control~\cite{deceleration}. Thus, we apply 4 $\mathrm{m/s^2}$ as the deceleration, which is also defined in Baidu Apollo as the maximum allowed deceleration to ensure safety~\cite{apollo}. 
Note that since the original planning algorithms are typically designed under the assumption that the localization accuracy is high (i.e., cm-level~\cite{reid2019localization}), we find directly re-using such algorithms in AR will result in unstable control since the planned trajectories are too sensitive to the larger localization errors and uncertainties after fusing the LD and MSF sides when one side is under attack. Thus, we directly set the planned trajectory as the centerline of the ego lane to achieve more stable control.

%to quickly reflect the deviation correction direction we intend the AD control to enforce.
%the planned trajectory will by default always start from the current position for trajectory smoothness. 
%Such a design works well in normal driving when localization accuracy is high (i.e., at centimeter-level~\cite{reid2019localization}).
%However, since the fused position from safety-driven fusion is a result of contention between MSF and LD rather than from actual measurement, it is by nature only an approximation to the vehicle's actual position and thus is at a coarse granularity. Therefore, reusing the original planning logic, which by default assumes centimeter-level localization accuracy~\cite{reid2019localization}, in AR will result in nonstable driving.
% at coarse granularity rather than an accurate measurement of the actual position, reusing the original planning logic in AR will result in nonstable driving. 
% However, since the positions from the safety-driven fusion are simply instantiations of the fused lateral deviations, there is no guarantee that these positions will be smooth in consecutive frames. 
%Nevertheless, since the fused position is closer to the benign side between MSF and LD and our AR goal is to stop in the current lane, we therefore design the AD planning to generate an AR trajectory at the centerline of the current lane to quickly reflect the deviation correction direction we intend the AD control to enforce.
% position error between the fused position and our intended trajectory in the lateral control objective~\cite{friedland2012control}.
% Therefore, we set the AR trajectory to be exactly aligned with the lane centerline. This way, the fused lateral deviations will be instantly reflected as the position error in the lateral control objective~\cite{friedland2012control} to be corrected in the next control iterations.
%\alfred{polish the logic} \junjie{tried, pls see above.}
% We design the emergency trajectory generation as an independent planning scenario which overtakes the original planning logic after the attack is detected as shown in Fig.~\ref{fig:design_overview}. 


% (mention that it general design, not only for lateral direction attacks)




\begin{algorithm}[tbp]
\footnotesize
\caption{Safety-driven fusion for attack response}
\label{alg:ar_fusion}
\textbf{Notations:} $D$: deviation to lane centerline; $P$: uncertainty from MSF or LD outputs; $MSF$: MSF position output; $kf$: 1-dimensional Kalman Filter;  $R$: uncertainty for KF update
\begin{algorithmic}[1]
\Function{\text{F\textsc{used}}\text{P\textsc{ose}}}{$D^{\text{MSF}}$, $D^{\text{LD}}$, $P^{\text{MSF}}$, $P^{\text{LD}}$, $MSF$}
    \State $R^{\text{MSF}}, R^{\text{LD}} \gets \text{U\textsc{ncertainty}}(D^{\text{MSF}}$, $D^{\text{LD}}$, $P^{\text{MSF}}$, $P^{\text{LD}})$
    \State $kf.update(D^{\text{MSF}}, R^{\text{MSF}})$; $d \gets kf.predict()$
    \State $kf.update(D^{\text{LD}}, R^{\text{LD}})$; $d \gets kf.predict()$
    \State $pose_{\text{center}}$, $heading_{\text{center}}$ $\gets \text{M\textsc{ap}}\text{L\textsc{ane}}\text{P\textsc{oint}}(MSF)$ \Comment{Appendix~\ref{app:map_apis}}
    \State $pose_{\text{fusion}}$ $\gets \text{A\textsc{dd}}\text{D\textsc{ev}}\text{T\textsc{o}}\text{P\textsc{oint}}(pose_{\text{center}}$, $heading_{\text{center}}, d)$
    \State \textbf{return} $pose_{\text{fusion}}$
\EndFunction
\end{algorithmic}
\end{algorithm}


\textbf{Localization-side AR: safety-driven fusion.}
% Since the attack can happen on either MSF or LD side, we cannot simply fall back to an LD-based Automated Lane Centering lateral control design as in lower-level AD systems~\cite{openpilot}. Instead, we address this by applying a 
% As mentioned in \S\ref{sec:design_overview}, we design a \textit{safety-driven fusion} on the MSF/LD outputs to \textit{leverage the safer source} and \textit{penalize the more aggressive one} in the driving context as shown in Alg.~\ref{alg:ar_fusion}. Specifically, 
As described in \S\ref{sec:design_overview}, we need to design a safety-driven fusion algorithm on the localization side that can systematically fuse LD and MSF outputs while taking less contributions from the side that is more aggressive in causing lateral deviations. To achieve this, we leverage a classic fusion algorithm design, \textit{Kalman Filter} (KF) based fusion, which can systematically determine the contributions of each fusion source using \textit{uncertainties}~\cite{thrun2005probabilistic, friedland2012control}. In the original design, the uncertainty score calculation are based on the noise-level measurements reported by the sources themselves, which thus are not suitable in attack settings since such measurements are also fundamentally under the attacker's control.

To systematically realize our safety-driven fusion design between LD and MSF, we thus still leverage such uncertainties-based fusion framework but design novel uncertainty score calculation based on their tendencies to cause lane departure. 
%As briefly mentioned in \S\ref{sec:design_overview}, we use a 1-dimensional Kalman Filter (KF) to fuse the two sources (MSF and LD), and assign different uncertainty scores based on their safety implication in the driving context to indicate their trustworthiness.
% Due to the consistency-checking based detection design (\S\ref{sec:design_detection}), it is not straightforward to assign the uncertainties since we cannot tell which source is the malicious one.
% Since we cannot purely trust uncertainty output from the MSF/LD algorithms since they could be influenced by the attack, we thus design a novel uncertainty calculation, which uses the \textit{cumulative lateral deviations} on each source to penalize the more aggressive one during the fusion. 
% The intuition is that with such a design, the malicious source needs to be less aggressive in order to gain higher trust in the fusion. Yet, less aggressive means that the lateral deviation it can cause in the AR period will also be limited. 
Alg.~\ref{alg:ar_uncertainty} lists the pseudocode for the uncertainty calculation. As shown (lines 2 and 7), we store the historical lateral deviations from MSF and LD in two fixed-size windows. To obtain the uncertainties, we first calculate the cumulative deviations in these two windows, and then calculate their proportions to the geometric mean of them (lines 8 and 11). We choose geometric mean over arithmetic mean since it can better penalize the source with a larger cumulative deviation. To increase the design flexibility, we include both our cumulative lateral deviation based uncertainty and the uncertainty from MSF/LD algorithms in the final uncertainty and use a weight $\lambda$ to adjust their fractions (lines 12 and 13).

With the uncertainties, we apply standard KF update/predict operations to fuse the MSF and LD lateral deviations (lines 3 and 4 in Alg.~\ref{alg:ar_fusion}). 
% To create the global position required by the AD planning and control, 
We then add the fused lateral deviation to the closest centerline point along the lateral direction based on the lane heading to instantiate a fused localization in the global coordinate system (line 5--6 in Alg.~\ref{alg:ar_fusion}). Fig.~\ref{fig:safety_driven_fusion} illustrates an example of the safety-driven fusion process.



\begin{algorithm}[tbp]
\footnotesize
\caption{Cumulative lateral deviations based uncertainties calculation}
\label{alg:ar_uncertainty}
\textbf{Notations:} $D$: deviation to lane centerline; $P$: uncertainty from MSF or LD outputs; $DS$: deviation history; $w$: deviation history window size; $\lambda$: weight of the deviation history based uncertainty\\
\textbf{Initialize:} $DS^{\text{MSF}} \gets \{ \}$; $DS^{\text{LD}} \gets \{ \}$
\begin{algorithmic}[1]
\Function{\text{U\textsc{ncertainty}}}{$D^{\text{MSF}}$, $D^{\text{LD}}$, $P^{\text{MSF}}$, $P^{\text{LD}}$}
    \State $DS^{\text{MSF}}$ $\gets$ $DS^{\text{MSF}}$ $||$ $\lvert D^{\text{MSF}} \rvert$ \Comment{append MSF dev. to the history}
    \State $DS^{\text{LD}}$ $\gets$ $DS^{\text{LD}}$ $||$ $\lvert D^{\text{LD}} \rvert$ \Comment{append LD dev. to the history}
    \If{size of $DS^{\text{MSF}} > w$} \Comment{remove first element if full}
        \State $DS^{\text{MSF}} \gets DS^{\text{MSF}} \setminus DS^{\text{MSF}}[0]$
        \State $DS^{\text{LD}} \gets DS^{\text{LD}} \setminus DS^{\text{LD}}[0]$
    \EndIf
    \State $s^{\text{MSF}} \gets \sum_{n=1}^{w} DS^{\text{MSF}}$; $s^{\text{LD}} \gets \sum_{n=1}^{w} DS^{\text{LD}}$ \Comment{dev.  sums}
    \State $s^{GeoMean} \gets \sqrt{s^{\text{MSF}} \cdot s^{\text{LD}}}$ \Comment{geometric mean of dev. sums}
    \State $f^{\text{MSF}} \gets s^{\text{MSF}} / s^{GeoMean}$ \Comment{dev. history based uncertainty for MSF}
    \State $f^{\text{LD}} \gets s^{\text{LD}} / s^{GeoMean}$ \Comment{dev. history based uncertainty for LD}
    \State $R^{\text{MSF}} \gets \lambda f^{\text{MSF}} + (1 - \lambda) P^{\text{MSF}}$ \Comment{MSF uncertainty}
    \State $R^{\text{LD}} \gets \lambda f^{\text{LD}} + (1 - \lambda) P^{\text{LD}}$ \Comment{LD uncertainty}
    \State \textbf{return} $R^{\text{MSF}}, R^{\text{LD}}$
\EndFunction
\end{algorithmic}
\end{algorithm}


