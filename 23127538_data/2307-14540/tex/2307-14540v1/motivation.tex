\nsection{Defense Motivation and Challenges} \label{sec:motivation}

In this section, we first evaluate whether the recent physical-invariant based defenses can effectively detect the existing AD localization attack. Motivated by the limitations of these defenses, we then identify the unique challenges for effective defenses in the high-level AD context.

% \subsection{Detection Effectiveness of Physical-Invariant based Defenses against AD localization Attack}
\nsubsection{Effectiveness of Existing Defenses} \label{sec:motivation_savior}

\textbf{Evaluation methodology.} 
We evaluate the detection effectiveness of physical-invariant based defenses against \fr{}~\cite{fusionripper}, which attacks MSF based AD localization using GPS spoofing. In our evaluation, we focus on SAVIOR~\cite{savior} rather than CI~\cite{ci} since SAVIOR adopts more complex nonlinear state estimation models and has shown superior detection performance than CI~\cite{savior}. Following the similar methodology as in SAVIOR's ground rover evaluation, we use the kinematic bicycle model~\cite{kong2015kinematic} and an Extended Kalman Filter (EKF) to predict the system state (i.e., position in x, y coordinates) given the vehicle control commands (i.e., steering and acceleration). Although the vanilla bicycle model does not have tunable parameters, we follow a similar implementation as SAVIOR by adding coefficients to the bicycle model equations~\cite{savior_code}. Same as SAVIOR, we use the \textit{nlgreyest} system identification tool from Matlab~\cite{matlab_si} to find the coefficients that can best fit the sensor and control trace.

During the evaluation, we continuously calculate the residuals between the GPS measurements and the predicted positions from the EKF, and feed the residuals to a CUSUM anomaly detector for attack detection. An execution that triggers the CUSUM detector will be considered as under attack. Specifically, we calculate the true positive rate (TPR) and false positive rate (FPR) by evaluating SAVIOR on the spoofed GPS traces and benign GPS traces and their corresponding control outputs, respectively. To systematically evaluate the detection performance, we vary the CUSUM thresholds to obtain the Receiver Operating Characteristic (ROC) curve.

\textbf{Experimental setup.}
For illustration purpose, we present the evaluation results on \textit{ka-local31}, which is a local road KAIST trace~\cite{jeong2019complex} (More SAVIOR evaluations will be presented in \S\ref{sec:eval_detection}). Since the KAIST traces do not contain control outputs, we replay the trace in Baidu Apollo to record the vehicle control commands (more details in Appendix~\ref{app:savior_collect_control}). In our evaluation, we use the best GPS spoofing parameters ($d$=0.5 m, $f$=1.2), which can achieve over 99\% success rate for the \textit{off-road} attack on Baidu Apollo MSF~\cite{fusionripper}.

\textbf{SAVIOR detection effectiveness and analysis.}
Fig.~\ref{fig:motivation_savior_roc} shows the ROC curve of SAVIOR on \textit{ka-local31}. As shown, SAVIOR's detection performance is only slightly better than random guessing and far from a perfect detector. For example, when the TPR of SAVIOR is 100\%, the lowest FPR it can achieve is still at 79\%. Such a detection performance would render the detection method unpractical in real world as it would introduce lots of false positives in normal driving.

% The root cause behind the poor detection performance is that \textit{the physical invariants used in the existing defenses, e.g., bicycle model, are not accurate enough to detect stealthy attacks targeting high-level AD systems}, which generally have much more sophisticate system designs and thus have more stringent requirement on attack stealthiness. Due to the inaccuracy of the predicted system state from the physical invariants, the existing defenses have to set a relatively large threshold to tolerate the deviations between the predicted state and the sensor measurement in the benign scenarios. While this might be acceptable for robotics vehicles such as drones and ground rovers, which have less requirement on attack stealthiness, attacks targeting high-level AD systems has to use small attack parameters to prevent being recognized as outliers, e.g., \fr{} on \textit{ka-local31} uses a sub-meter spoofing distance (0.5 m) during its first attack stage. In comparison, the bicycle model used in SAVIOR is reported to have an average position error of 0.33 m within 0.8 sec under low-speed settings (e.g., 13.8 m/s) in~\cite{kong2015kinematic} and its error keeps accumulating as time progresses; comparably, the same bicycle model incurs an average error of 1.076 m on \textit{ka-local31} within 1 sec.

The reason behind the poor detection performance in the AD context is twofold. First, the physical dynamics of the vehicle are much harder to model due to the complex physical moving characteristics, e.g., tire-road frictions, aerodynamic forces, road bank angles, etc.~\cite{rajamani2011vehicle}. For example, prior study~\cite{polack2017kinematic} finds that the error of kinematic bicycle model increases very fast at high speeds (e.g., 25 $\mathrm{m/s}$) or on curvy roads (e.g., steering angle at 4$^\circ$). In comparison, the bicycle model used in SAVIOR is reported having an average position error of 0.33 m \textit{within 0.8 sec} under low-speed settings (e.g., 13.8 m/s) in~\cite{kong2015kinematic} and its error keeps accumulating as time progresses; comparably, the same bicycle model incurs an average error of 1.076 m on \textit{ka-local31} within 1 sec, where the trace contains many turns and curvy roads.

Second, the attack deviation goals in the AD context can be much smaller but still being safety-critical. 
While SAVIOR is effective at detecting attacks on small robotics vehicles such as drones with large deviation goals (e.g., $\sim$50 m~\cite{savior}), attacks targeting high-level AD systems requires much smaller deviation and thus harder to detect. For example, even less than 0.5 m lateral deviations are enough to cause lane departure on narrow urban roads (e.g., 2.7 m wide~\cite{road_shoulder_width}).

\junjie{position against SAVIOR: current savior design is limited, cite bicycle model problem. Then say we explore a different defense direction from SAVIOR. It is not to dismiss the line of work like SAVIOR and CI is not usable.}

% \junjie{I deleted the natural GPS deviation arguement since that is not so fundamental.}

% Due to the inaccuracy of the predicted system state from the physical invariants, the existing defenses have to set a relatively large threshold to tolerate the deviations between the predicted state and the sensor measurement in the benign scenarios. While this might be acceptable for robotics vehicles such as drones and ground rovers, which have less requirement on attack stealthiness, attacks targeting high-level AD systems has to use small attack parameters to prevent being recognized as outliers, e.g., \fr{} on \textit{ka-local31} uses a sub-meter spoofing distance (0.5 m) during its first attack stage. 

\cut{
Another reason for the high FPR in SAVIOR is that \textit{GPS measurements can naturally have large fluctuations} because of the satellite signal inferences during the transmission~\cite{gps_error_sources}, which may trigger false detection if a benign GPS measurement has a large deviation to the ground truth position of the vehicle. To verify this, we calculate the deviations between the benign GPS and the ground truth positions in the KAIST traces. As shown in Fig.~\ref{fig:motivation_gps_devs}, the natural GPS deviations have a long-tail distribution, where majority of the GPS measurements has less than 0.4 m deviation. But still, there exists many measurements with deviations $>$6.5 m, and note that the KAIST dataset uses a high-end GPS receiver (SOKKIA GRX2), which costs around \$8000~\cite{kaist_gps_price}.}

% Figure environment removed

% % Figure environment removed

% \begin{table}[tbp]
% \footnotesize
% \begin{minipage}{0.485\linewidth}
% 	\centering
%     % Figure removed
%     \vspace{-0.27in}
%     \captionof{figure}{Detection ROC curve of SAVIOR on a \textit{ka-local31} against the \fr{} attack.}
%     \label{fig:motivation_savior_roc}
% \end{minipage}\hfill\hspace{0.015\linewidth}
% \begin{minipage}{0.485\linewidth}
% 	\centering
%     % Figure removed
%     \vspace{-0.25in}
%     \captionof{figure}{Long-tail distribution of natural GPS deviations to the ground truth positions in KAIST traces.}
%     \label{fig:motivation_gps_devs}
% \end{minipage}
% \vspace{-0.1in}
% \end{table}

