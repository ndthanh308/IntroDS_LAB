\nsection{Lane Boundary: Defense-Suitable Information Source for High-Level AD Localization} \label{sec:discovery}
% \nsection{Lane Detection: Defense-Suitable Information Source for High-Level AD Localization} \label{sec:discovery}

% Since defenses for high-level AD systems have the freedom to leverage other information sources, we aim to address \textbf{C1} by identifying the ones that are suitable for defending against lateral direction localization attacks.
% As mentioned in~\S\ref{sec:motivation_challenges}, such information sources may never provide more accurate positioning than the existing ones used in AD localization in common circumstances, however, they may be useful for defense purposes when the AD localization is substantially wrong due to attacks.
% More concretely, we need to find the ones that are

% Specifically, such sources should be ``\textit{defense-suitable}'' for high-level AD localization, i.e., they should be (1) \textit{independent of the ones that are already used for AD localization}, and (2) \textit{reliable when the sources are applicable}. Otherwise, attacks targeting these information sources may break the defense and AD localization at the same time and lead to false negatives, or the defense may have high false positives in benign drivings.

% Moreover, it is also challenging to leverage the information source for detection purpose if it does not directly measure the vehicle's global position.
% such that the defense does not introduce false positives when the information source is applicable.
% reasonable in security since the attacker wants to make the systems substantially wrong
% Limited scope: Longitudinal direction is not suitable for localization accuracy
% Not considered for AD developers since they emphasize on the accuracy
% In the cases when the data is applicable, the information is accurate
% Lateral direction
% To address this, leverage other sources, requirements: independent, reliable (not much FP) (define what is defense-suitable, e.g., what are the requirements); but challenging: if a sensor is useful for localization, should already be used → must be some limitations with those sources → defense is more important than accuracy → could be used for cross checking; repurpose the sensor for our defense purpose

To address \textbf{C1}, a natural information source directly related to lateral-direction attacks is the relative distance to the current lane boundaries, which is generally available during driving and also can be obtained via mature technologies such as camera-based lane detection~\cite{hillel2014recent, openpilot, pan2018spatial}. However, we find that state-of-the-art AD localization algorithms actually do not use such features~\cite{wan2018robust, autoware, gao2015ins, soloviev2008tight, udacity_av_apollo, udacity_av_nd, coursera_av}, as they can only tell \textit{local} lateral positions specific to the current lane; for the purpose of \textit{global} localization on the map in the world coordinate system needed by high-level AD systems, they are almost useless as they are not generally distinctive across different roads and regions.
% on the map.

% Although not of much use from the functionality perspective, we find that it does qualify as a defense-suitable information source for AD localization, e.g., independence to existing sources/features used in state-of-the-art algorithms and reliability when applicable.

As mentioned in \S\ref{sec:motivation_challenges}, defense-suitable sources need to satisfy the independence and reliability requirements. 
Since lane detection techniques are already very mature~\cite{hillel2014recent}, thus they are presumably reliable in practice, which will also be empirically proven in our evaluations later (\S\ref{sec:eval}).
However, it is unclear if lane boundaries are independent of the ones already used in AD localization as LiDAR can also sense the lane line markings. Since LiDAR locator is the only one among the MSF inputs (\S\ref{sec:background_msf_attack}) that is possible to utilize lane boundaries, 
% we thus quantify the relationship between LiDAR locator and the lane boundaries, 
we thus rigorous validate its independence to LiDAR locators.
% via experiments.

% To prove the independence, we aim to find that whether the existing information sources in AD localization use lane boundary information for localization. Among the inputs in MSF localization (\S\ref{sec:background_msf_attack}), LiDAR locator is the only one that is possible to utilize lane boundaries. Intuitively, lane boundaries are ``boring'' features for global localization since they are prevalent on roads and are visually similar when the vehicle is driving. Nevertheless, we still aim to quantify the relationship between LiDAR locator and the lane boundaries.

% , if already used, not satisfy independence requirement. thus, this section, rigorously validate its independence experimentally

% Although not of much use from the functionality perspective, we find that it does qualify as a defense-suitable information source for AD localization, e.g., independence to existing sources/features used in state-of-the-art algorithms and reliability when applicable.


% they are not used for localization purposes in today's high-level AD systems~\cite{apollo, autoware}. However, they appear to be a promising information source for defense purpose since AD localization attacks need to at least achieve lane-straddling lateral deviations to cause real world consequences.


% One such information source available in high-level AD systems is the \textit{lane boundaries}, which can help determine the vehicle the relative position within the current driving lane. They are widely-used for Automated Lane Centering in \textit{lower-level} AD systems, such as openpilot~\cite{openpilot} and Tesla Autopilot~\cite{autopilot}, where \textit{camera-based lane detection} algorithms are applied to identify lane line positions and shapes to help steer the vehicle at the lane center. Since lane boundaries cannot provide accurate global positioning~\cite{kang2020lane, evlampev2020map}, they are not used for localization purposes in today's high-level AD systems~\cite{apollo, autoware}. However, they appear to be a promising information source for defense purpose since AD localization attacks need to at least achieve lane-straddling lateral deviations to cause real world consequences. Thus, we first examine if the lane boundaries information satisfies the independence and reliability requirements as mentioned above. 

% In this section, we focus on evaluating the independence requirement (\S\ref{sec:motivation_challenges}) since lane detection techniques are already very mature~\cite{hillel2014recent} and thus are presumably reliable in practice, which will also be empirically proven in our evaluation later (\S\ref{sec:eval}). To prove the independence, we aim to find that whether the existing information sources in AD localization use lane boundary information for localization. Among the inputs in MSF localization (\S\ref{sec:background_msf_attack}), LiDAR locator is the only one that is possible to utilize lane boundaries. Intuitively, lane boundaries are ``boring'' features for global localization since they are prevalent on roads and are visually similar when the vehicle is driving. Nevertheless, we still aim to quantify the relationship between LiDAR locator and the lane boundaries.


\textbf{Evaluation methodology.} 
To do that, we compare the executions of LiDAR locator with the complete lane boundary information and the one without such information. \textit{If a LiDAR locator does not rely on the lane boundary information, we should observe a high similarity between the two executions.} Specifically, the LiDAR operates by scanning the surrounding environment and outputting point clouds, which store the 3D positions and intensities of the reflected points from the lasers. Lane boundaries, or lane line markings, will exhibit distinctive intensities from the other road surface due to their color differences. Thus, we created the lane line marking removed point cloud data (PCD) by keeping their intensities the same as other road surface area. More technical details on removing the lane line markings is in Appendix~\ref{app:laneline_removal}.

\todo{put correlation analysis using incorrect lane marking information here.}

\textbf{Experimental setup.} 
We evaluate on 2 LiDAR locators, one from Baidu Apollo (BA-LiDAR locator)~\cite{wan2018robust} and another from Autoware (AW-LiDAR locator)~\cite{autoware}. Details of the LiDAR locators can be found in Appendix~\ref{app:lidar_locators}. Since MSF localization takes not only position measurements but also position uncertainties from LiDAR locator as inputs, we calculate both the \textit{position accuracies} and \textit{uncertainty correlation} with and without lane line markings removal to show the similarity.
We evaluated on the same 5 local road and highway traces in \fr{}~\cite{fusionripper} from two datasets. For each trace, we exclude intersections since they do not have lane line markings and thus are out of the applicable domain for lane boundaries. Among them, since \textit{ba-local} does not provide ground truth positions, we calculate the position accuracy based on the LiDAR locator with the complete lane line markings.

% Among them, since \textit{ba-local} does not provide ground truth positions, we calculate the position accuracy based on the LiDAR locator with the complete lane line markings instead of the ground truth positions.

% At design level, Baidu Apollo LiDAR locator (BA-LiDAR locator)~\cite{wan2018robust} considers point cloud intensities in its position calculation. Thus, the removal of lane line intensities do have the potential to affect the BA-LiDAR locator performance. On the other hand, Autoware LiDAR locator (AW-LiDAR locator)~\cite{autoware} only uses the position data in the PCD and completely ignores the intensities. This means that AW-LiDAR locator does not consider lane line markings at the design level. Nevertheless, we still include AW-LiDAR locator in our evaluation. When used in AD localization, the MSF takes not only the position measurement but also the position uncertainty from LiDAR locator as inputs. Thus, we report the \textit{position accuracies} and \textit{uncertainty correlation} with and without lane line markings removal to show the similarity. Specifically, AW-LiDAR locator implements the Normal Distributions Transform (NDT) algorithm~\cite{ndt}, which does not output position uncertainty by default. Thus, we follow a common adaptation for NDT to use the point cloud matching fitness score as the uncertainty~\cite{merten2008three}. We evaluated on the same 5 local road and highway traces in \fr{}~\cite{fusionripper} from two datasets. For each trace, we exclude the road segments that do not have lane line markings, e.g., intersections. Among them, since \textit{ba-local} does not provide ground truth positions, we calculate the position accuracy based on the LiDAR locator with the complete lane line markings instead of the ground truth positions.


\textbf{Results.}
Table~\ref{tbl:correlation} in the Appendix shows the experiment results. For the position accuracy, we report the Root Mean Squared Error (RMSE) between the LiDAR locator positions and the ground truth positions or the ones without lane line marking removal. For the correlation, we use the commonly-used Pearson's correlation, and a correlation coefficient $>$0.5 is considered strongly correlated~\cite{cohen2013statistical}. As shown, the uncertainty correlation coefficients with and without lane line marking removal are all well above the threshold for strong correlation, and their position accuracies are also all at centimeter-level. Particularly, since AW-LiDAR locator does not use lane line markings at the design level (Appendix~\ref{app:lidar_locators}), the traces consequently show perfect correlations and identical position accuracies with and without removing the lane line markings. Such a result suggests that the existing LiDAR locators used in high-level AD systems largely ignore the lane boundary information when localizing the vehicle on the map, which might because global localization focuses more on the unique features on the road, such as buildings, roadside layouts, and traffic signs. As a result, this indicates that lane boundaries are indeed largely independent of the ones that are already used in high-level AD localization and thus pose great potential as a defense-suitable information source.



% \begin{table*}[tbp]
% \footnotesize
% 	\begin{minipage}{0.32\linewidth}
% 		\centering
%         % Figure removed
%         \vspace{-0.1in}
%         \captionof{figure}{Demonstration of lane line markings removal. Top figure shows the original PCD; bottom figure shows the one with lane line markings removed.}
%         \label{fig:laneline_removal_example}
% 	\end{minipage}\hfill
% 	\begin{minipage}{0.665\linewidth}
%         \centering
%         \caption{Position uncertainty correlation and position accuracy of LiDAR locators using the original PCDs and lane line markings removed PCDs. For the correlation, we report the Pearson's correlation coefficients. We ignore the $p$-value since they are all statistical significant; the position accuracies are in meters.}
%         \label{tbl:correlation}
%         \setlength{\tabcolsep}{2pt}
%         \begin{tabular}{@{}cccccccc@{}}
%         \toprule
%         \multirow{3}{*}{Trace} & \multicolumn{4}{c}{BA-LiDAR Locator} & \multicolumn{3}{c}{AW-LiDAR Locator} \\ \cmidrule(l){2-8} 
%          & \multicolumn{2}{c}{\begin{tabular}[c]{@{}c@{}}Uncertainty Correlation\\ (original vs no-marking)\end{tabular}} & \multicolumn{2}{c}{Position Accuracy} & \multirow{2}{*}{\begin{tabular}[c]{@{}c@{}}Fitness\\ Correlation\\ (original vs\\ no-marking)\end{tabular}} & \multicolumn{2}{c}{Position Accuracy} \\ \cmidrule(lr){2-5} \cmidrule(l){7-8} 
%          & Easting & Northing & \begin{tabular}[c]{@{}c@{}}RMSE\\ original\end{tabular} & \begin{tabular}[c]{@{}c@{}}RMSE\\ no-marking\end{tabular} &  & \begin{tabular}[c]{@{}c@{}}RMSE\\ original\end{tabular} & \begin{tabular}[c]{@{}c@{}}RMSE\\ no-marking\end{tabular} \\ \midrule
%         ba-local & 0.79 (4e-230) & 0.71 (3e-164) & - & 0.067 & 1.0 (0.0) & - & 0.0 \\
%         ka-local08 & 0.97 (0.0) & 0.98 (0.0) & 0.061 & 0.062 & 1.0 (0.0) & 0.074 & 0.074 \\
%         ka-local31 & 0.98 (0.0) & 0.96 (0.0) & 0.057 & 0.057 & 1.0 (0.0) & 0.077 & 0.077 \\
%         ka-local07 & 0.91 (0.0) & 0.90 (0.0) & 0.077 & 0.080 & 1.0 (0.0) & 0.076 & 0.076 \\
%         ka-highway17 & 0.99 (0.0) & 0.94 (0.0) & 0.059 & 0.060 & 1.0 (0.0) & 0.077 & 0.077 \\
%         ka-highway06 & 0.78 (0.0) & 0.88 (0.0) & 0.064 & 0.064 & 1.0 (0.0) & 0.077 & 0.077 \\ \bottomrule
%         \end{tabular}
% 	\end{minipage}
% \end{table*}

% are represented as $r$ ($p$-value)
% Specifically, since ba-local does not provide ground truth positions, we directly calculate the RMSE between the original and no-marking LiDAR localization results

