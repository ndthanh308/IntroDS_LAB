% \vspace{-0.05in}
\nsection{Related Work} \label{sec:related_work}
% \vspace{-0.02in}

\textbf{AD system security.} Prior works have studied attacks and defenses of AD system components for environment sensing and decision-making, such as object detection, tracking, localization, lane detection, and planning~\cite{sp:2021:ningfei:msf-adv, ndss:2022:ziwen:planfuzz, sato2021dirty, arxiv:2022:shen:sok, sato2021wip, sato2020hold, dipalma2021security, ma2023wip, wang2022poster, huai2023doppelganger, jia2020fooling, cao2019adversarial, shin2017illusion, shen2023detecting}. Specifically for attacking localization, sensor spoofing/jamming attacks targeting GPS, LiDAR, IMU, camera, RADAR~\cite{fusionripper, zeng2018all, popperccs11, utaustinspoofer, narain2018security, kerns2014unmanned, cao2019adversarial, petit2015remote, son2015rocking, yan2016can} have been proposed. Only \fr{}~\cite{fusionripper} is able to break the MSF localization on high-level AD systems and cause lateral deviations in MSF outputs. Thus, we target \fr{} and show that \ld{} can effectively detect \fr{} and steer the vehicle to safely stop in the ego lane.

%\textbf{AD localization attacks.}
%Since AD systems rely on sensors for localization, prior works proposed sensor spoofing/jamming attacks targeting GPS, LiDAR, IMU, camera, RADAR~\cite{fusionripper, zeng2018all, popperccs11, utaustinspoofer, narain2018security, kerns2014unmanned, cao2019adversarial, petit2015remote, son2015rocking, yan2016can}. Among them, only \fr{}~\cite{fusionripper} has shown to be able to break the MSF localization on high-level AD systems and cause lateral deviations in MSF outputs. Thus, we target \fr{} and show that \ld{} can effectively detect \fr{} and steer the vehicle to safely stop in ego lane.

% focus on defending against such lateral-direction localization attack.


\textbf{Physical-invariant based defenses.}
Recently, researchers propose physical-invariant based defenses, CI~\cite{ci} and SAVIOR~\cite{ savior}, to detect sensor attacks such as GPS spoofing by cross-checking sensor measurements with system state estimations based on the physical invariants, i.e., the relationships between system states and control inputs. However, as shown in \S\ref{sec:eval_detection}, 
the direct adaptation of existing physical-invariant based approach 
% to the AD context suffers from very high false positives and is actually close to random guessing.
% the detection effectiveness of physical-invariant based detection methods are 
is largely limited because of the complexity of physical dynamics and much smaller attack deviation goals in the AD context.
% mostly because the existing state estimation models, e.g., bicycle model~\cite{kong2015kinematic}, are not accurate enough to detect stealthy attacks such as \fr{}. 
In addition, 
% both works evaluate on robotics systems such as drones and ground rovers, and 
none of them has proposed attack response designs, which is especially important for AD systems (\S\ref{sec:design_overview}). Nevertheless, such physical-invariant based attack detection methods are complimentary to \ld{} and can be incorporated into our design for attack detection if the accuracy of state-estimation model can be further improved.


\textbf{Attack response/recovery.}
According to a survey on the broader Cyber-Physical Systems security, existing defenses mostly focus on attack detection and very few works studied attack responses~\cite{giraldo2017security}. Particularly, Choi et al.~\cite{choi2020software} and Zhang et al.~\cite{zhang2020real} recently propose \textit{attack recovery} methods, which apply similar state estimations as above to replace attacked sensors in the attack recovery period. Thus, they suffer from the same model accuracy limitations in the AD context.
Moreover, they intend to maintain normal operations of the system for a short duration until the system is taken-over by the human driver, which does not exist on high-level AD vehicles when deployed commercially~\cite{baidu_driverless_robotaxi, waymo_driverless}.
Additionally, attack responses in high-level AD systems require more careful design on AR trajectories (\S\ref{sec:design_ar}) to safely navigate the vehicle.
% will not unexpectedly drive out of lane boundaries or can safety navigate to the roadside.

% A common assumption in these works is that the sensor under attack has been correctly identified. However, this assumption may not be true for \fr{} since existing detection methods fail to differentiate the attacked and benign GPS inputs (\S\ref{sec:motivation_savior}). In addition, the same limitation on the state estimation accuracy for the physical-invariant based defenses also applies to these two works, since inaccurate state estimation in attack recovery can potentially result into even larger lateral deviations. Moreover, their attack recovery designs intend to maintain a normal operation of the system for a certain duration until the attack is finished, or an emergency operation has been performed, e.g., taken-over by the drone operator. However, attack responses in high-level AD systems require more careful design on the emergency trajectories (\S\ref{sec:design_ar}) such that the AD vehicle will not unexpectedly drive out of lane boundaries or can safety navigate to the roadside.


% \junjie{Arguments for Software Sensor Attack Response:
% 1) they need to identify which sensor is under attack. However, current AD localization attack is quite stealthy such that the existing sensor attack detection methods fail to detect the attack.
% 2) the vehicle models they rely on naturally will accumulate errors. E.g., bicycle model introduces > 1 meter error on the KAIST traces in 1 sec.
% 3) cannot handle attacks that targets LiDAR localization.
% 4) we change the planning decision, they keep the same planning.
% }




%\textbf{AD system security.} 
%Since AD systems heavily rely on sensors, prior works have studied sensor attacks/defenses in AD context~\cite{cao2019adversarial, shin2017illusion, shen2023detecting}. Besides sensor-level attacks, prior works
%also studied attacks and defenses of AD system components
%related to environmental sensing, such as object detection and
%tracking, localization, lane detection, and planning~\cite{sp:2021:ningfei:msf-adv, ndss:2022:ziwen:planfuzz, sato2021dirty, arxiv:2022:shen:sok, sato2021wip, sato2020hold, dipalma2021security, ma2023wip, wang2022poster, huai2023doppelganger, jia2020fooling}. In this paper, we show that \ld{} can effectively detect \fr{} (localization attack) and steer the vehicle to safely stop in ego lane.


