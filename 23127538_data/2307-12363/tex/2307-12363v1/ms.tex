\documentclass[preprint,prd,amsmath,amssymb,aps,nofootinbib]{revtex4-1}

\usepackage{graphicx}
\usepackage{siunitx}
\usepackage{bm}
\usepackage{natbib}
\bibliographystyle{apsrev4-1}

%The control sequence for ads
\def\nat{Nature\ }
\def\aap{Astron.\ Astrophys.\ }
\def\apj{Astrophys.\ J.\ }
\def\apjl{Astrophys.\ J.\ Lett.\ }
\def\apjs{Astrophys.\ J.\ Supp.\ }
\def\aj{Astron.\ J.\ }
\def\mnras{Mon.\ Not.\ R.\ Astron.\ Soc.\ }
\def\physrep{Phys.\ Rept.\ }
\def\prd{Phys.\ Rev.\ D\ }
\def\prl{Phys.\ Rev.\ Lett.\ }
\def\apss{Astrophys.\ Space\ Sci.\ }
\def\araa{Annu.\ Rev.\ Astron.\ Astrophys.\ }
\def\jcap{J.\ Cosmol.\ Astropart.\ Phys.\ }
\def\ssr{Space\ Sci.\ Rev.\ }

\begin{document}
\title{On the Origin of Galactic Diffuse TeV-PeV Emission: Insight from LHAASO and IceCube}
\author{Kai Yan$^{1,2}$}
\author{Ruo-Yu Liu$^{1,2}$}\email{Correspondence: ryliu@nju.edu.cn}
\author{Rui Zhang$^{3,4}$}
\author{Chao-Ming Li$^{1,2}$}
\author{Qiang Yuan$^{3,4}$}
\author{Xiang-Yu Wang$^{1,2}$}

\affiliation{
$^1$School of Astronomy and Space Science, Nanjing University, Nanjing 210023, China \\
$^2$Key laboratory of Modern Astronomy and Astrophysics (Nanjing University), Ministry of Education, Nanjing 210023, People's Republic of China\\
$^3$Key Laboratory of Dark Matter and Space Astronomy, Purple Mountain Observatory, Chinese Academy of Sciences, Nanjing 210023, China\\
$^4$School of Astronomy and Space Science, University of Science and Technology of China, Hefei 230026, China
}

\date{\today}

\begin{abstract}
The high-energy diffuse gamma-ray emission and neutrino emission are expected from the Galactic plane, generated by hadronuclear interactions between cosmic rays (CR) and interstellar medium (ISM). Therefore, measurements of these diffuse emissions will provide important clues on the origin and nature of Galactic CRs. Comparing the latest observations of LHAASO and IceCube on the diffuse Galactic gamma-ray and neutrino emissions respectively, we suggest that the diffuse gamma-ray emission at multi-TeV energies contains a considerable contribution of a leptonic component. By modelling the gamma-ray halos powered by middle-aged pulsars in our Galaxy with taking into account the magnetic field configuration and the interstellar radiation field in the Galaxy, we demonstrate that the collective contribution of pulsar halos can account for the excess in the measured diffuse gamma-ray emission with respect to the predicted flux from CR-ISM interactions. The resulting one-dimensional profile along the Galactic longitude is also consistent with the observation.  
\end{abstract}

\maketitle

\section{Introduction}
The diffuse gamma-ray emission of the Galactic Plane (DGE) is the most prominent structure in the gamma-ray sky, as firstly revealed by SAS-2 \citep{Fichtel75} and COS-B \citep{Bignami75} about half a century ago, and later by EGRET, Fermi-LAT at GeV band. Ground-based TeV gamma-ray instruments can also measure part of the Galactic plane and the diffuse emission from Galactic plane has been discovered at energies up to TeV-PeV band \citep{MILAGRO05, ARGO15_diffuse, Asgamma21_diffuse} in recent decades. It is believed that the DGE mainly originates from hadronuclear interactions between Galactic cosmic rays (CRs) and interstellar medium (ISM) \citep{Stecker73, Strong10, Fermi12_DGE, Lipari18}, with possible contribution from unresolved, faint sources of either leptonic or hadronic emission. Therefore, the DGE contains critical information of the origin of Galactic CRs and extreme particle accelerators in our Galaxy. Very recently, the Large High-Altitude Air Shower Observatory (LHAASO) published the most complete TeV-PeV gamma-ray source catalog up to date in the declination between $-20^\circ$ and 80$^\circ$ \citep{LHAASO_catalog}, and the measurement of DGE in $10-1000$\,TeV after subtracting the contribution of these sources \citep{LHAASO_diffuse}. The measured flux, interestingly, is about 3 times higher than the prediction of CR-ISM interactions in the inner Galactic plane and 2 times higher than that predicted in the outer Galactic plane in the range of $\sim 10-60\,$TeV. The excess could be either caused by CR overdensity with respect to the conventional model prediction due to complicated propagation effect in reality, or by contribution of unresolved sources. The latter scenario has been suggested to account for excesses in DGE measured by some other instruments \citep{Linden18, Liu21, ZhangPP22, Vecchiotti22, Yan23, Sun:2023ibg}, although the dominating source population has not been unambiguously identified. Either leptonic sources such as pulsar halos, or hadronic sources such as young stellar clusters may be responsible for the excess. 

On the other hand, the IceCube neutrino telescope has recently reported discovery of high-energy neutrinos from our Galaxy at a significance over $4\sigma$ level with 10-year data of cascade events \citep{IceCube23}. The inferred Galactic neutrino flux contributes approximately $7-12\,\%$ of the all-sky astrophysical flux at 30\,TeV. The discovery unambiguously serves as a smoking gun for the hadronuclear interactions between high-energy CRs and gas in the Milky Way. Although  no individual neutrino source has been identified due to the poor angular resolution and insufficient statistics, the extracted neutrino flux based on certain all-sky spatial templates can nevertheless provide insights into the hadronic component of the DGE. In this article,  we will attempt to understand the origin of the DGE based on the latest measurements of LHAASO and IceCube.

\section{Insight of Galactic Diffuse Gamma-ray Emission from Icecube and LHAASO}
The neutrino analysis is based on theoretical all-sky spatial templates, while the DGE flux given by LHAASO is extracted from a portion of the Galactic plane ($|b|<5^\circ$ and $15^\circ<l<235^\circ$), after masking the contribution of all known sources. Furthermore, the neutrino emission measured by IceCube likely contains a considerable contribution also from neutrino sources in the Galaxy \citep{IceCube23}, in addition to those truly diffusive emission from CR-ISM interactions. Hence, the results of the two measurements cannot be compared directly. 

As the first step of the analysis, we compare the Galactic neutrino flux to the total flux of known gamma-ray sources powered by hadronic processes, in order to estimate the fraction of the neutrino flux related to CR-ISM interactions. As a ground-based gamma-ray detector, LHAASO cannot detect 
most sources in the southern hemisphere. We therefore combine sources reported in the first LHAASO catalog with $|b|<5^\circ$ and sources recorded in the Galactic Plane survey of High Energy Spectroscopic System (HESS) with declination less than $-20^\circ$ to get a complete source list in the entire Galactic plane. To exclude gamma-ray sources of leptonic origin, we do not count in sources with spatial association of pulsars, despite of suggestions that pulsars or pulsar wind nebulae may also be high-energy proton accelerators \citep{Atoyan96, Amato03, Liu21_crab}. On the other hand, we find that the neutrino flux from Galactic plane with $|b|<5^\circ$ is about 50\% of the all-sky flux according to the Fermi-LAT $\pi^0$ template \citep{Fermi16_DGE_template}, and about 70\% of the all sky flux according to the KRA$_\gamma^5$ template \citep{Gaggero15_DGE}, which are employed in IceCube's data analysis. The gamma-ray flux and the single-flavor neutrino flux generated in the same hadronuclear interactions are approximately related by
$E_\gamma^2\frac{dN_\gamma}{dE_\gamma}=2E_\nu^2\frac{dN_\nu}{dE_\nu}$, 
where $E_\gamma=2E_\nu$. We then can convert the neutrino flux in the Galactic plane to corresponding gamma-ray flux and compare with the total source flux. As shown in Fig~\ref{fig:multi_flux}, the sum of the source contribution reaches about 2/3 of the total flux converted from the neutrino measurement with the $\pi^0$ template, from a few TeV up to several tens of TeV. Although the measured neutrino flux is based on the template only for CR-ISM interactions, the contribution from sources would increase the normalization of the inferred neutrino spectrum significantly. Beyond $\sim 100$\,TeV, on the other hand, we can see the source contribution is negligible, implying that Galactic neutrinos is dominated by CR-ISM interactions above this energy. The neutrino flux in the Galactic plane obtained with the KRA$_\gamma^5$ template is even slightly lower than the source contribution, which is not physical. Actually, the KRA$_\gamma^5$ template gives rise to a lower significance of neutrino detection than the $\pi^0$ template, which may not reflect the neutrino flux level in the Galaxy accurately. Nevertheless, it also indicates a significant contribution of sources to the neutrino emission in the Galactic plane.


% Figure environment removed


Next, to estimate the hadronic gamma-ray component in the DGE measured by LHAASO, we need to find out the neutrino flux from the same region of interest (ROI) as LHAASO's DGE analysis. LHAASO's DGE analysis masked each of the known sources and the surrounding ISM in the sky map to reduce the contamination of the source to the measured DGE. Therefore, we also need to mask the same region in the all-sky template for the neutrino analysis. We find that the flux ratio of the remaining region to the all-sky map is 12.5\% for the Fermi-LAT $\pi^0$ template and 8.1\% for the KRA$_\gamma^5$ template  (see Appendix for details). Dividing the rescaled neutrino flux by the corresponding solid angle of the remaining region, which is 0.206\,sr for the inner Galactic plane ($|b|<5^\circ, 15^\circ<l<125^\circ$) and 0.268\,sr for the outer Galactic plane ($|b|<5^\circ, 125^\circ<l<235^\circ$), we can obtain the average neutrino intensity, which is supposed to be related with CR-ISM interactions, from the same ROI of LHAAOS's DGE measurement. After converting the neutrino intensity into the corresponding gamma-ray intensity, we can compare it to LHAASO's DGE measurement.

We note that some uncertainties may exist in the above analysis. First, we may expect the existence of some unresolved hadronic TeV gamma-ray sources in the Galaxy, such as star-forming regions, historical hypernova remnants, interactions of freshly injected CRs and medium surrounding their accelerators \citep{Liu21, Ahlers14, ZhangPP22}. Considering contributions of these sources would further reduce the gamma-ray flux related to CR-ISM interactions. On the other hand, some of known gamma-ray sources without pulsar association could be of the leptonic origin actually. If so, it would reduce the fraction of the hadronic gamma-ray flux from sources, and increase the gamma-ray flux related to CR-ISM interactions. The uncertainties from these two aspects operate in the opposite direction, and may somehow cancel each other to certain extent. An accurate evaluation of the uncertainty relies on further studies of these sources, via multiwavelength and multimessenger observations. 

%Note that the pionic gamma-ray flux and the single-flavor neutrino flux generated in the same hadronuclear interactions are approximately related by $E_\gamma^2\frac{dN_\gamma}{dE_\gamma}=2E_\nu^2\frac{dN_\nu}{dE_\nu}$, where $E_\gamma=2E_\nu$. According to this relation, we then may derive the corresponding gamma-ray intensity of the hadronic origin in the same ROI of LHAASO's DGE measurement (see Appendix for more details).

% Figure environment removed

Finally, we present the comparison between the gamma-ray flux derived from the neutrino measurement and the DGE measured by LHAASO in Fig.~\ref{fig:ic_flux}. We see that the derived gamma-ray intensity related to CR-ISM interactions is lower than the LHAASO measured DGE flux below $\sim 100\,$TeV for both the inner Galactic region and the outer Galactic region. The predicted diffuse gamma-ray intensity from CR-ISM interactions, following the calculation by Zhang et al. \citep{ZhangR23}, is also shown for reference. This calculation employs GALPROP \citep{GALPROP11} and is based on up-to-date measurements of
the local CR spectra from various instruments for CR protons, helium nuclei and heavier isotopes such as Li, Be, B, C, O, as well as electrons and positrons \cite{ZhangR23, Yuan20}. We see that an excess in the DGE measured by LHAASO is obvious below $\sim 60$\,TeV with respect to the model prediction. Interestingly, the gamma-ray intensity derived from the neutrino measurement is well consistent with the model prediction, corroborating the existence of the excess in DGE. 
The excess indicates an additional component contributing to the DGE, and must be of the leptonic origin. A plausible candidate population for the leptonic DGE excess is pulsar halos, which have been discussed as potential contributors to DGE \citep{Linden18, Vecchiotti22, Martin22, Yan23, Dekker23} with simple modelings of the contribution of the pulsar halo population. In what follows, we will explore in detail the pulsar halo interpretation for the DGE excess measured by LHAASO.

\section{Pulsar Halos as Dominating Sources of DGE at Multi-TeV}

Pulsar halos are produced by the IC scattering of energetic electron/positron pairs, which have escaped from their pulsar wind nebulae (PWNe), off the background radiation field in the ISM \citep{Liu22, Lopez-Coto22, Fangkun22}. From the perspective of energy budget, pulsar halos are powered by the rotational energy of pulsars, with a fraction $\eta_e$ of the pulsar's spindown power converted into energetic pairs and subsequently into radiations. Pulsar halos are apt to form around pulsars with age beyond several tens of thousand years \citep{Giacinti20}, as the confinements of pairs are relatively weak inside these middle-aged PWNe while the spindown powers of the pulsars are not very low. Although the transport mechanism of injected pairs in the surrounding ISM is still under debate, a consensus has been reached that pulsar halos are extended TeV gamma-ray sources without sharp boundaries. It is therefore natural to expect that many pulsar halos are faint, diffusive gamma-ray sources and most of them are not resolved by instruments. 

% Figure environment removed

To estimate their contribution to the DGE, we single out pulsars with age between 50\,kyr and 10\,Myr from the ATNF pulsar catalog \citep{Manchester05}. Here we consider two propagation models for escaping pairs. The first one is the two-zone isotropic diffusion model (2ID model) with a suppressed diffusion coefficient $D_0(E)$ within a radius of $r_b$ from the central pulsar and the typical diffusion coefficient of the ISM $D_{\rm ISM}$ as inferred from the secondary-to-primary CR ratio for the region beyond $r_b$ \citep{Fang18, Profumo18}. In the second model, we consider the anisotropic diffusion model (AD model) \citep{Liu19_prl} in which particle diffuse more rapidly along the mean magnetic field direction than they diffuse perpendicular to the mean field direction. In both models, we generate the interstellar magnetic field strength $B_0$ in the vicinity of each pulsar based on their positions in the Galaxy and the Galactic magnetic field (GMF) model suggested by Jansson \& Farrar \citep{JF12_regular, JF12_random}, and consider the interstellar radiation field density following the model proposed by Popescu et al. \citep{Popescu17}. As such, we can calculate cooling and radiation of pairs around each pulsars. For the AD model, we also generate the mean direction of the magnetic field because it determines the preferential diffusion direction of injected pairs. The turbulent level of the magnetic field, indicated by the Alfv{\'e}nic Mach number $M_A=\Delta B/B_0$, is then randomly assigned in the range of $(0.1, 1)$ which is typical for ISM. A detailed description of our model can be found in Appendix.


% Figure environment removed


We then can simulate the pulsar halos powered by each of selected pulsars based on the model. To compare with LHAASO's measurement, we make a two-dimensional intensity map of projection of each pulsar halo onto the Galactic plane according to the Galactic coordinate and distance of each pulsar, and then convolve the intensity map with LHAASO's point spread function \citep{LHAASO_whitepaper} to mimic the sky viewed by LHAASO. In Fig.~\ref{fig:hotmap0} and Fig.~\ref{fig:hotmap1}, we present the projected intensity map of 1179 selected pulsars in the region where the DGE is extracted under the 2ID model and the AD model respectively. Then we mask the same region as used in the LHAASO data analysis, and calculate the average gamma-ray intensity in the remaining Galactic plane for $15^\circ<l<125^\circ$ and $125^\circ<l<235^\circ$ respectively. It is worth noting the existence of pulsars that are invisible to us, if their lighthouse-like radiation beam do not sweep Earth as they spin. These invisible off-beamed pulsars, however, may also generate pulsar halos around them and emissions of these pulsar halos would contribute to DGE \citep{Yan23}. Therefore, the average gamma-ray intensity obtained above need be multiplied by a beam-correction factor following Ref.~\citep{Yan23,Tauris98} to account for the contribution from halos of those off-beamed pulsars. 

% Figure environment removed

In Fig.~\ref{fig:sed1}, we show that the DGE excess measured by LHAASO can be well explained by the pulsar halo population with a reasonable set of model parameters. Furthermore, an excess of DGE in $10-500\,$GeV is also reported in the same region of LHAASO's analysis based on Fermi-LAT's data \citep{ZhangR23}. Although the Fermi-LAT excess is not necessarily related to pulsar halos since there may be more potential GeV gamma-ray emitters in the Galaxy, we find that the Fermi-LAT excess can be accounted for by pulsar halos as well if some specific (but reasonable) model parameters are chosen. Beside the spectrum, we also compare the one-dimensional Galactic longitude profile between our simulation and LHAASO's observation. Because halos of those undetectable, off-beamed pulsars cannot be included in the modeled profile and because of the possible variation in the pair injection spectrum from individual pulsar halos, we do not expect a quantitative match between the measured profile and the modeled one. Nevertheless, we find the tendency of the longitudinal profile of the DGE excess and that predicted by our model are generally in good consistency, supporting the pulsar halo interpretation. 

\section{Discussion}
Our results suggest that pulsar halos make a considerable, if not dominant, contribution to the DGE emission measured by LHAASO below $\sim 60$\,TeV, extending probably down to $\sim 10\,$GeV. This conclusion does not depend on the model of pulsar halos and it only requires that approximately $10\%$ of a pulsar's spindown power has converted to electron/positron pairs escaping to ISM with an injection spectral index $s\gtrsim 2$. The main uncertainty lies in the halos powered by those off-beamed pulsars, which are now dealt with multiplying a beam-correction factor to contribution of halos formed around detected pulsars. If these invisible pulsars have similar properties to those detected ones in a statistical way (such as following the same luminosity function and spatial distribution), the total flux from the pulsar halo population predicted by the present model would be basically correct.

The collective emission of pulsar halos constitutes a non-negligible background for other gamma-ray sources in the Milky Way. Therefore, it is important to understand the nature of these pulsar halos so as to better eliminate their influence when studying other sources. With increasing exposure time, sensitive TeV gamma-ray instruments such as LHAASO may resolve more and more pulsar halos, and hence reduces the background of this leptonic component. However, it would not be straightforward to recognize halos of those off-beamed pulsars. Accurate measurements of the intensity profile of gamma-ray sources by imaging air Cherenkov telescopes of high angular resolution such as the Cherenkov Telescope Array \citep{CTA11}, as well as a mulitwavelength study, would help the identification. On the other hand, the neutrino measurement is crucial to distinguish the hadronic component from the leptonic component. The next-generation neutrino telescopes with improved angular resolution and effective area \citep{IC-gen2,Trident23} would be capable of identifying bright neutrino sources from the Galaxy. The advanced measurement of high-energy neutrinos, along with the high-sensitivity gamma-ray observations, will hopefully unravel the composition of the DGE, uncover the nature of Galactic leptonic and hadronic gamma-ray emitters, and finally solve the century-old puzzle of the cosmic-ray origin.

% Figure environment removed

%It is interesting to mention that the neutrino flux obtained with the $pi^0$ template is consistent with the DGE measured by Tibet AS$\gamma$. The latter is $2-3$ times higher than the DGE measured by LHAASO. It has been proposed that the DGE measured by AS$\gamma$ contains the contribution from extended sources given the insufficient masking region of known TeV gamma-ray sources. 
\section*{Acknowledgement}
We are grateful to Lu Lu for helpful discussions on IceCube's results. 
This work is supported by the National Natural Science Foundation of China (No. 2031105, 12220101003) and the Project for Young Scientists in Basic Research of Chinese 
Academy of Sciences (No. YSBR-061).

\clearpage

\section*{Appendix}
\subsection{Compare Measurements between LHAASO and IceCube}
The neutrino emission measured by IceCube contains a considerable contribution from Galactic neutrino emission sources. To estimate the source component in the measured neutrino flux and the converted hadronic gamma-ray flux, we add up the gamma-ray  sources of potential hadronic origin observed by HESS and LHAASO. Among the 90 sources listed in the 1st LHAASO catalog, 85 of them are Galactic sources. Among the 85 Galactic sources, we exclude 35 sources with pulsar or PWN associations which are like of the leptonic origin, and focus on those sources located within Galactic latitude $|b| \le 5^{\circ}$. There are 45 remaining gamma-ray sources of potential hadronic origin in the Galactic plane, measured in the range of 1-25$\rm \,TeV$ by LHAASO-WCDA and $\rm \ge 25\,TeV$ by LHAASO-KM2A. Note some sources only have detection by only one of the two detectors of LHAASO. The best-fit spectral parameters and the uncertainties are given in Ref.\citep{LHAASO_catalog}. On the other hand, HESS Galactic Plane Survey (HGPS) \cite{HGPS_2018} announced 78 sources from 300\,GeV to 30\,TeV. Among them, 14 are confirmed PWNe \cite{HESS_2018} and others are regarded as potential hadronic sources. Part of them are the same sources recorded in the LHAASO catalog so we only consider 43 sources in HGPS with declination smaller than $-20^\circ$, where is outside LHAASO's field of view. 

Both catalogs provide the best-fit spectra of these sources either described with a power-law function
\begin{equation}
    F(E) = N_0 (\frac{E}{E_0})^{-\Gamma},    
\end{equation}
or exponential cutoff power-law function from 
\begin{equation}
    F(E) = N_0 (\frac{E}{E_0})^{-\Gamma} \exp(-\frac{E}{E_{\rm c}}), 
\end{equation}
as well as the uncertainties of these parameters, i.e., $\Delta N_0$, $\Delta \Gamma$, and $\Delta E_{\rm c}$. The flux uncertainty is evaluated via
\begin{equation}
    \Delta F(E) = F(E)\sqrt{\frac{\Delta N_0^2}{{N_0}^2} + \log^2(\frac{E}{E_0})\Delta \Gamma^2 + 
    \frac{E^2 \Delta E_{\rm c}^2}{{E_{\rm c}}^4}
    }
\end{equation}
noting that the last term in the square root should be dropped out if the spectrum is fitted with simply a power-law function without cutoff. The total flux is added up via $F_{\rm tot}(E)=\sum\limits_i F_i(E)$ where $F_i$ represents the spectrum of the $i$th source in the list, and the uncertainty is calculated by $\Delta F_{\rm tot}(E)=\sqrt{\sum\limits_i \Delta F_i(E)^2}$. 

On the other hand, the neutrino flux is obtained based on certain all-sky template for hadronic gamma-ray emissions in IceCube's analysis \citep{IceCube23}. To compare with the hadronic emission from Galactic sources as discussed before, we exclude the region outside $|b| \le 5^{\circ}$. The integrated flux of the remaining region is 50.2\% and 71.9\% of the all-sky integrated flux in $10\,$TeV for the Fermi-LAT $\pi^0$ template and the KRA$_\gamma^5$ template respectively. The inferred gamma-ray fluxes in the ROI of $|b| \le 5^{\circ}$ derived from the neutrino measurement and that from hadronic sources are compared as shown in Fig.~\ref{fig:multi_flux}.

To obtain the DGE of the hadronic origin from the same region of LHAASO's measurement, we need to further exclude the region outside $15^\circ<l<235^\circ$ and $|b|<5^\circ$ in the $\pi^0$ template and KRA$_\gamma^5$ template presented in the left panels of Fig.~\ref{fig:ic_map}. We also need to mask the same region as that in LHAASO's DGE analysis \citep{LHAASO_diffuse}. The remaining region is shown in Fig.~\ref{fig:ic_map}. For the inner (outer) Galactic region $15^\circ<l<125^\circ$ ($125^\circ<l<235^\circ$), the integrated flux of the remaining region is 6.87\% (5.69\%) and 5.74\% (2.49\%) of the all-sky integrated flux in $10\,$TeV for the Fermi-LAT $\pi^0$ template and the KRA$_\gamma^5$ template respectively. Note that the KRA$_\gamma^5$ template is mildly energy dependent, and consequently the fraction of the remaining flux also changes with energy slowly for this template. This effect is considered in our analysis and the dependence of the fraction as a function of energy is shown in Fig.~\ref{fig:ratio_kragamma}. The corresponding solid angle of the remaining region is 0.206\,sr for the inner Galactic plane and 0.268\,sr for the outer Galactic plane. The neutrino flux of single flavor (assuming a flavor ratio of $1:1:1$ after oscillation) is related to the co-produced gamma-ray flux of the hadronuclear origin by $E_\gamma^2\frac{dN_\gamma}{dE_\gamma}=2E_\nu^2\frac{dN_\nu}{dE_\nu}$, 
where $E_\gamma=2E_\nu$. 
The inferred DGE of the hadronic origin can be obtained via dividing the flux by the solid angle.



% Figure environment removed

% Figure environment removed


\subsection{Injection of Electron/Positron Pairs}
We assume that pairs are injected in a power-law spectrum with a high-energy cutoff at a rate of
\begin{equation}
Q\left(E_{\rm e}, t\right)=Q_{\rm 0}\left(t \right) E_{\rm e}^{-s} e^{-E_{\rm e} / E_{\rm \max }}
\label{eq:Qe}
\end{equation}
where $Q_0$ is the normalization which is related to the spindown power of the pulsar $L_{\rm s}$ by $\int E_{\rm e}Q(E_{\rm e},t)dE_e=\eta_e L_{\rm s}(t)$. The temporal evolution of the spindown power can be given by $L_{\rm s}(t)=\eta_e L_{\rm s, 0}/(1+t/\tau_0)^2$ \citep{Pacini73}, simply assuming pulsars as a rotational magnetic dipole. The present spindown power can be obtained by $L_{\rm s}=-4\pi^2I\dot{P}P^{-3}$ once the rotational period $P$ and the first derivative of the period $\dot{P}$ of the pulsar is measured, with $I$ being the pulsar's moment of inertia which is taken to be a typical value of $10^{45} \, \rm g/cm^2$. Here $L_{\rm s, 0}$ is the initial spindown power of the pulsar and $\tau_0$ is the initial spindown timescale which can be derived given $P$, $\dot{P}$ and the initial rotational period $P_0$. The initial rotational period of a pulsar is an unknown parameter, we here assume it to be
\begin{equation}
    P_0=\left\{
\begin{aligned}
& 0.1P, \  P > 300\,\rm ms ,\\
& 30\, {\rm ms},  \ P \le 300\,\rm ms.
\end{aligned}
\right.
\end{equation}
The cutoff energy $E_{\rm max}$ in the injection spectrum describes the acceleration limit of the PWN and can be given by $E_{\rm max}=(2\eta_{\rm B})^{1/2}e(L_{\rm s}/c)^{1/2}$ \citep{deJager92, Amato21}. $\eta_{\rm B}$ is the fractional energy converted from kinetic energy of the pulsar wind to the magnetic energy at the pulsar wind termination shock, which is suggested to be $\gtrsim 0.1$ for middle-aged pulsar \citep{Yan23}. For simplicity, we take $\eta_{\rm B}=0.1$ here for all pulsars. The beam-correction factor $f_{\rm beam}$ depends on the age or rotational period of the pulsar, which reads \citep{Tauris98}
$f_{\rm beam} = 0.011\left[\rm log\left( \tau_{\rm age}/100{\rm Myr}\right) \right]^{2}+0.15$ where $\tau_{\rm age}$ is the age of the pulsar, or $f_{\rm beam}=0.09[\log(P/10{\rm s})]^2+0.03$. Both relations yield a similar beam-correction factor. The flux of each halo is divided by the corresponding beam-correction factor to account for the contribution of those off-beamed pulsar with similar properties. 

To fit the DGE excess from $\sim 10\,{\rm GeV} - 60\,$TeV, we employ $s=2.2$ for both the 2ID model and the AD model with $\eta_e=0.15$ and 0.1 respectively, as shown in Fig.~\ref{fig:sed1}. If we only focus on the excess from $10\,{\rm TeV} - 60\,$TeV as measured by LHAASO, a broader range of $s$ and $\eta_e$ are available, e.g., a harder injection spectrum (i.e., a smaller $s$) and a smaller pair conversion efficiency (i.e., a smaller $\eta_e$) with a lower spectral cutoff energy (i.e., a smaller $\eta_{\rm B}$).


\subsection{Models for Pulsar Halos}
\noindent {\bf Two-zone Isotropic Diffusion (2ID) model}\\
Under this model , the diffusion coefficient is suppressed in the inner zone to a radius of $r_{\rm b}$ (typically a few tens of parsecs), and beyond the suppressed diffusion zone, a typical diffusion coefficient is assumed, i.e.,
\begin{equation}
D(E,r)=\left\{
\begin{aligned}
& D_0(E/100 \, {\rm TeV})^{1/3}, \  r < r_b ,\\
& D_{\rm ISM}(E/100 \, {\rm TeV})^{1/3},  \ r \ge r_b.
\end{aligned}
\right.
\end{equation}
For simplicity we fix $r_b=20 \, \rm pc$, $D_0=4.5\times 10^{27} \, \rm cm^2 s^{-1}$, $D_{\rm ISM}=1.8\times 10^{30} \, \rm cm^2 s^{-1}$ \citep{Trotta11}. The analytical solution of particle distribution in both momentum space and real space in the 2ID model is given by Ref.\citep{Tang19, Dimauro19}. For cooling of pairs, we consider synchrotron cooling in the Galactic magnetic field, and the IC cooling in the background radiation field including CMB and ISRF from the far infrared band to the ultraviolet band. The interstellar magnetic field strength around each pulsar is based on the model proposed by Jansson \& Farrar \citep{JF12_regular, JF12_random}, as shown in Fig.~\ref{fig:Btot}. The Monte Carlo method is used to generate the magnetic field strength. The ISRF is based on the model proposed by Popescu et al. \citep{Popescu17}.

\vspace{5pt}
\noindent {\bf Anisotropic Diffusion (AD) model}\\
Under the AD model, the perpendicular diffusion coefficient $D_{\perp}$ is suppressed by a factor of $M_{\rm A}^4$ with respect to the parallel diffusion coefficient $D_{\parallel}$, i.e., $D_\perp=M_A^4D_\parallel$. We assume $D_\parallel$ to be the typical diffusion coefficient in the ISM, i.e., same as $D_{\rm ISM}$ in the 2ID model. We follow the analytical solution provided by Refs.~\citep{Fang23, Delahaye10} to calculate the distribution of pairs in both real space and energy space. The same GMF and ISRF are considered in this model to calculate the cooling of pairs. Note that injected pairs diffuse preferentially along the mean magnetic field direction in the AD model, and hence the magnetic field direction matters. Therefore, in addition to the magnetic field strength, we also need to consider the magnetic field direction in the AD model. The GMF in the JF12 model contains both a regular component and a random component. For the regular component, we can simply obtain the direction of the regular field around each pulsar based on the positions of the pulsar in the Milky Way. For the random striated component, the so called ``ordered random fields'', is randomly assigned either a direction aligned with the regular component or anti-aligned with the regular component \citep{JF12_regular}. For those ``true random fields'', which may represent the influence of local astrophysical objects, we assign a random direction with an equal possibility in the $4\pi$ solid angle. The final direction (as well as the strength) of the magnetic field is determined by the vector sum of the regular component and the random component. 

% Figure environment removed

% Figure environment removed


\vspace{5pt}
\noindent {\bf 2D Gamma-ray Intensity Map}\\
After obtaining the pair distribution around each pulsar, we can calculate the gamma-ray emissivity of pairs through the IC process. By integrating the emission over the line of sight (LOS) toward a certain direction in the Galactic plane, we can project each gamma-ray halo onto the plane of the sky, and obtain the 2D intensity map composed of emissions from pulsar halos as shown in Fig.~\ref{fig:hotmap0} and Fig.~\ref{fig:hotmap1}. For the 2ID model, we follow the projection method shown in Ref.~\citep{Liu19} to overlaid each pulsar halo in the Galactic plane. For the AD model, the projection method is detailed Ref.~\citep{Liu19_prl} but additional procedure is carried out to minimize the possible bias arising from the randomly generated magnetic field direction, as the latter largely determines the morphology of pulsar halos in the AD model. To do this, we generate 10,000 realizations of the GMF, and obtain intensity maps in the Galactic plane composed of pulsar halos based on each of the realization of the GMF. 

For reference, we show the 2D probability distribution of  the magnetic field direction with respect to the cosine of the angle between the generated field and of the LOS ($\cos \phi$), and the azimuthal angle of projected field on the celestial plane ($\zeta$) in Fig.~\ref{fig:B2}. In left panels, we show a pulsar located in the magnetic field dominated by the random component. In this case, the probability distribution of $\cos\phi$ and $\zeta$ is almost homogeneous (the top-left panel) and the average expectation of the halo morphology is nearly isotropic (the bottom-left panel). On the contrary, if a pulsar located in the magnetic field dominated by the regular component, as shown in the right panels, the average magnetic field would have a preferential direction. This can be seen from the enhancement of the probability in the lower-middle part of the $\cos\phi - \zeta$ plane. The resulting halo is correspondingly elongated along the projected direction of the magnetic field in the celestial plane. %Note that the double peaks mainly reflect the influence of the ``ordered random field'' -- the striated random component, which basically determine the direction of the total magentic field in this case.  

Finally, we average over all the generated gamma-ray intensity maps of the Galactic plane, and get the average expectation of the DGE contributed by pulsar halos under the AD model. Particularly, for the observed two pulsar halos around Geminga and Monogem, we restrict the inclination angle between the magnetic field and the LOS to be less than $5^{\circ}$, in order to be consistent with the observed morphology of these two halos \citep{Liu19_prl}.

\subsection{Model for CR-ISM interactions}
We model the CR-ISM interactions following the same method suggested in Refs.~\cite{ZhangR23, Yuan20}. In addition to the generated pionic gamma-ray spectrum as already shown in Ref.~\cite{ZhangR23}, we also present the all-sky template (the top-left panel of Fig.~\ref{fig:crgam_profile}) and the one after the mask (the top-right panel of Fig.~\ref{fig:crgam_profile}). We then obtain the 1D Galactic longitudinal profile in the energy ranges of $10-63\,$TeV and $63-1000\,$TeV, extracted from the same ROI of LHAASO's DGE analysis. The results are compared with LHAASO's measurement in the bottom panels of Fig.~\ref{fig:crgam_profile}, based on which the flux ratio presented in Fig.~\ref{fig:profile} is obtained.

% Figure environment removed

% % Figure environment removed


\bibliographystyle{apsrev}
\bibliography{ms}

\end{document}
