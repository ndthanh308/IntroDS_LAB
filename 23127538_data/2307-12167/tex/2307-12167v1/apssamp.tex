% ****** Start of file apssamp.tex ******
%
%   This file is part of the APS files in the REVTeX 4.2 distribution.
%   Version 4.2a of REVTeX, December 2014
%
%   Copyright (c) 2014 The American Physical Society.
%
%   See the REVTeX 4 README file for restrictions and more information.
%
% TeX'ing this file requires that you have AMS-LaTeX 2.0 installed
% as well as the rest of the prerequisites for REVTeX 4.2
%
% See the REVTeX 4 README file
% It also requires running BibTeX. The commands are as follows:
%
%  1)  latex apssamp.tex
%  2)  bibtex apssamp
%  3)  latex apssamp.tex
%  4)  latex apssamp.tex
%
\documentclass[%
 reprint,
%superscriptaddress,
%groupedaddress,
%unsortedaddress,
%runinaddress,
%frontmatterverbose, 
%preprint,
%preprintnumbers,
%nofootinbib,
%nobibnotes,
%bibnotes,
 amsmath,amssymb,
 aps,
%pra,
%prb,
%rmp,
%prstab,
%prstper,
%floatfix,
]{revtex4-2}

\usepackage{graphicx}% Include figure files
\usepackage{dcolumn}% Align table columns on decimal point
\usepackage{bm}% bold math
\usepackage{braket}
\usepackage{siunitx}
%\usepackage{hyperref}% add hypertext capabilities
%\usepackage[mathlines]{lineno}% Enable numbering of text and display math
%\linenumbers\relax % Commence numbering lines

%\usepackage[showframe,%Uncomment any one of the following lines to test 
%%scale=0.7, marginratio={1:1, 2:3}, ignoreall,% default settings
%%text={7in,10in},centering,
%%margin=1.5in,
%%total={6.5in,8.75in}, top=1.2in, left=0.9in, includefoot,
%%height=10in,a5paper,hmargin={3cm,0.8in},
%]{geometry}

\begin{document}

\preprint{APS/123-QED}

\title{Nonlinear Multi-Resonant Cavity~Quantum Photonics~Gyroscopes\\ for
Quantum Light Navigation}% Force line breaks with \\

\author{Mengdi Sun}
\thanks{mengdis@vt.edu}%
%\altaffiliation[]{Bradley Department of Electrical and Computer Engineering, Virginia Tech, Arlington, VA, USA}%Lines break automatically or can be forced with \\
% \email{zinlin@vt.edu}
\author{Vassilios Kovanis}%
\affiliation{Bradley Department of Electrical and Computer Engineering, Virginia Tech, Arlington, VA, USA}%

%\collaboration{MUSO Collaboration}%\noaffiliation

\author{Marko Lon\v{c}ar}
% \homepage{http://www.Second.institution.edu/~Charlie.Author}
\affiliation{
John A. Paulson School of Engineering and Applied Sciences, Harvard University, Cambridge, MA, USA}%

\author{Zin Lin}%
\affiliation{Bradley Department of Electrical and Computer Engineering, Virginia Tech, Blacksburg, VA, USA}%

% \author{Delta Author}
% \affiliation{%
%  Authors' institution and/or address\\
%  This line break forced with \textbackslash\textbackslash
% }%

%\collaboration{CLEO Collaboration}%\noaffiliation

\date{\today}% It is always \today, today,
             %  but any date may be explicitly specified

\begin{abstract}
We propose an on-chip all-optical gyroscope based on nonlinear multi-resonant cavity quantum photonics in thin film $\chi^{(2)}$ resonators---Quantum-Optic Nonlinear Gyro or QONG in short. The key feature of our gyroscope is {\it co-arisal and co-accumulation} of quantum correlations, nonlinear wave mixing and non-inertial signals, all inside the same sensor-resonator. We theoretically analyze the Fisher Information of our QONGs under fundamental quantum noise conditions. Using Bayesian optimization, we maximize the Fisher Information and show that $\sim 900\times$ improvement is possible over the shot-noise limited linear gyroscope with the same footprint, intrinsic quality factors and power budget.
\end{abstract}

%\keywords{Suggested keywords}%Use showkeys class option if keyword
                              %display desired
\maketitle

%\tableofcontents

\section{Introduction}
Gyroscopes are critical components of an inertial navigation system for augmenting the GPS guidance or salvaging GPS-denied operational environments \citep{doddoc}. In an optical gyroscope, the rotation rate is measured through the phase shift between two counter-propagating beams in an optical loop. This approach was first proposed by Sagnac in 1913 \citep{Sagnac,post1967sagnac} and soon different types of optical gyroscopes were developed \citep{Arditty,Chow}. Careful studies have been performed on the sensitivity and the quantum noise of these gyroscopes \citep{Cresser,MATSKO20182289}, and remarkable levels of sensitivity ($<0.001^\circ$/h) have been achieved in state-of-the-art discrete component optical gyroscopes, including fiber optic gyroscopes (FOG) \citep{Wang:14,Ma,Sanders}, ring laser gyroscopes (RLG) \citep{Korth}, atom-laser gyroscope \citep{Dowling} and optical cavity gyroscopes \citep{Liang:17,Zhang:17,Khial2018}. However, bulky components and relatively high power consumption remain major roadblocks to further exploiting discrete component optical gyroscopes. On the other hand, on-chip optical gyroscopes \citep{vahala,Ge:15,Scheuer:07,Cimi,MatthewGrant} exhibit great potential for fully integrated inertial navigation platforms (free of fragile moving parts) and can outperform their discrete component counterparts in size, weight, power consumption, maneuverability, manufacturing scalability, robustness and the ability to operate in harsh environments. However, on-chip gyros are yet to reach sensitivity levels smaller than $1^\circ$/h. This is due to fundamentally limited optical path lengths even in ultra-high quality factor resonantors \citep{vahala}, leaving dubious prospects for further improvements via increasing resonator size or quality factors. To address the challenge of this seemingly intrinsic trade-off between sensitivity and compactness, novel physics and designs have been investigated, including exceptional point sensing \citep{Ren:17}, slow light \citep{Yariv,Anping}, dispersive enhancements \citep{Smith08,Shahriar}, dynamic thermal drift cancellation \citep{Khial2018,Digonnet} and nuclear magnetic resonance \citep{Larsen}.

Meanwhile, driven by the emerging trend of quantum technologies \citep{Toth_2014,Cappellaro,CRC}, quantum light sensors have been identified as a \textit{promising} option that can extend the fundamental sensitivity limits beyond the shot noise regime \citep{Caves,Saikat,Fink_2019}. These ideas were reinforced by decades of development and analysis that lead to the construction of very large laser interferometers with extreme sensitivity that is capable of detecting gravitational waves from remote cosmological events. Recently, squeezed light was used in the LIGO in the US and the VIRGO in Italy to substantially improve the sensitivity of the observing runs that happened late in April 2019 \citep{PRL2019,PRL2}. On the other hand, recent advances in nanofabrication, integration and packaging of ultra-coherent laser sources \citep{Jin2021}, low-loss photonic circuits \citep{Krasnokutska:18,Liu:22} and highly efficient photo-detectors \citep{Li:18} have opened up exciting opportunities for realizing fully on-chip quantum devices. Along this trend, we identify on-chip quantum light gyroscopes, which combine high sensitivities, low power consumption, and small form factors, as promising candidates for next-generation rotation sensing.

In this paper, we theoretically introduce a new type of on-chip quantum light gyroscope that exploits nonlinear multi-resonant cavity quantum photonics in integrated thin film resonators with strong quadratic $\chi^{(2)}$ nonlinearities. We call our gyroscope Quantum-Optic Nonlinear Gyro or QONG in short. Instead of externally injecting quantum states of light into the gyroscope \citep{Saikat}, one of the distinguishing features of our gyroscope is that it \textit{fuses} quantum-coherent nonlinear interactions, quantum light generation and non-inertial signal accumulation inside the same sensor-resonator, enabling $\gtrsim 900\times$ improvements in gyroscopic sensitivity over the linear shot noise limit. In our scheme, classical laser light (coherent state) is injected into a doubly-resonant $\chi^{(2)}$ cavity, and output light is measured at the fundamental ($\omega_1$) and second-harmonic frequencies ($\omega_2=2\omega_1$). The sensitivity of the gyroscope is evaluated by Fisher information (FI) \citep{Zhang2019,Anderson2023}, and the latter is maximized by Bayesian optimization \citep{Shahriari}. Various parameter regimes associated with both fundamental and second harmonic injection schemes were investigated, which reveal correlated noise suppression and sensitivity enhancements via parametric oscillations and critically sensitive three wave mixing dynamics. We predict that, under quantum noise conditions, a minimum detectable rotation rate (MDR) of $<0.01$ $^\circ$/h can be achieved using a thin film lithium niobate (TFLN) ring resonator with a diameter of 20 mm, intrinsic quality factors $Q_{i2}=10^6$ at the second harmonic wavelength (795 nm), $Q_{i1}=10^7$ at the fundamental wavelength (1590 nm). We discuss the scope, validity and implications of our approach and results, while the key sensitivity enhancement factors are summarized in Table~\ref{tab:1} of Section~\ref{sec:dtab}.
\section{Gyroscopic model}
\subsection{Linear resonant gyroscope as a baseline}
\label{sec:Linearmodel}
We first review a basic interferometric scheme probing the gyroscopic shift of a linear resonant cavity, as outlined in Fig.~\ref{fig:1}. We perform a quantum noise analysis similar to Ref.~\citep{zhang2019quantum} or Section 4 of Ref.~\citep{MATSKO20182289}. Two identical counter-propagating probes (seeded from the same on-chip laser) are injected into the clockwise (CW) and the counterclockwise (CCW) modes of a ring resonator; at the exit, the two probe fields are set to interfere via balanced homodyne detection \citep{Stefszky_2012}. In the absence of rotation, the CW and the CCW modes are degenerate and the exiting fields register a vanishing {\it differential} photocurrent signal at the detection setup \citep{MATSKO20182289}. Rotational motion induces a frequency splitting proportional to the rotation rate $\Omega$, which in turn induces a phase difference between the outgoing CW and CCW probes. Subsequently, interference of the two probe fields gives rise to a non-zero differential signal and the underlying $\Omega$ can be measured. For conceptual simplicity, we assume that the frequency of the probe laser is always locked to the degenerate frequency of the unperturbed gyro \citep{MATSKO20182289}. In principle, this can be achieved by self-injection locking the laser to an independent rotation-insensitive cavity (such as a high-Q spiral resonator \citep{Ciminelli}) having the exact same frequency as the unperturbed gyro ring. It has been demonstrated \citep{Kondratiev2023} that self-injection locking to a high-Q cavity can produce an ultra-coherent \textit{integrated} laser with a sub-Hertz linewidth; therefore, we can readily approximate the laser state as a quantum-mechanical coherent state. In the Heisenberg picture, the ring resonator gyro obeys the Heisenberg-Langevin equations \citep{Drummond2004}:

\begin{align}
    \frac{d \hat{a}_\text{cw}}{dt} &= \left(-\frac{\kappa}{2} - \frac{\gamma}{2} + i\delta \right)\hat{a}_\text{cw} + i\beta \hat{a}_\text{ccw} + \sqrt{\kappa}\hat{b}^\text{in}_\text{cw} + \sqrt{\gamma}\hat{c}^\text{in}_\text{cw} \label{eq:a1}\\
    \frac{d \hat{a}_\text{ccw}}{dt} &= \left(-\frac{\kappa}{2} - \frac{\gamma}{2} - i\delta \right)\hat{a}_\text{ccw} + i\beta \hat{a}_\text{cw} + \sqrt{\kappa}\hat{b}^\text{in}_\text{ccw} + \sqrt{\gamma}\hat{c}^\text{in}_\text{ccw} \label{eq:a2}
\end{align}

where $\hat{a}_\text{cw}$ and $\hat{a}_\text{ccw}$ are the annihilation operators for the cavity CW and CCW modes excited by the injections $\hat{b}^\text{in}_\text{cw}$ and $\hat{b}^\text{in}_\text{ccw}$ respectively. $\hat{c}^\text{in}_\text{cw}$ and $\hat{c}^\text{in}_\text{ccw}$ represent intrinsic loss channels (such as radiative losses). $\kappa$ and $\gamma$ are the decay rates for the coupling and the intrinsic losses. We approximate Rayleigh-type back-scattering as a linear (conservative) coupling $\beta$ between CW and CCW modes inside the cavity \citep{vahala}. $\delta$ is the rotation-induced resonant frequency shift due to the Sagnac effect. Here we have assumed single-photon normalization for each eigenmode so that $\hat{a}^\dagger \hat{a}$, for example, represents the photon number operator inside the cavity.

We denote $\langle \hat{A} \rangle = \bra{\psi} \hat{A} \ket{\psi} $ as the usual notation for computing the expectation value of a physical observable $\hat{A}$ with respect to the quantum state $\ket{\psi}$. In the linear problem, we will consider coherent states of the same amplitude $b^\text{in}$ in the input waveguides and vacuum states in the intrinsic loss channels for both CW and CCW light \citep{MATSKO20182289}. The input quantum state of the gyro is then given by $\ket{\psi} = \ket{b^\text{in}}_\text{cw} \ket{b^\text{in}}_\text{ccw }\ket{0}_\text{cw} \ket{0}_\text{ccw}$. The classical counterpart of the input operator $\hat{b}^\text{in}$ is the input amplitude of a coherent state in the feeder waveguide, and can be related to the input power $P$ by the formula:

\begin{align}
    |b^\text{in}|^2 = \langle \hat{b}^{\text{in}\dagger} \hat{b}^\text{in} \rangle = \frac{P}{\hbar\omega} \label{eq:a3}
\end{align}

The output operators in the waveguides are given by \citep{MATSKO20182289}:

\begin{align}
    \hat{b}^\text{out}_\text{cw} &=\hat{b}^\text{in}_\text{cw}-\sqrt{\kappa_1}\hat{a}_\text{cw} \label{eq:a4}\\
    \hat{b}^\text{out}_\text{ccw} &=\hat{b}^\text{in}_\text{ccw}-\sqrt{\kappa_1}\hat{a}_\text{ccw} \label{eq:a5}
\end{align}

The clockwise and counterclockwise signals are set to interfere through a directional coupler/beam splitter with a controllable phase shift $\phi$, followed by photodetection. The signal incident on the photodetectors and the photocurrent operators are then given by:

\begin{align}
    \hat{b}_+ &=\left(\hat{b}^\text{out}_\text{cw} e^{i\phi/2}+i\hat{b}^\text{out}_\text{ccw} e^{-i\phi/2}\right)/\sqrt{2} \label{eq:a6} \\
    \hat{b}_- &= \left(i \hat{b}^\text{out}_\text{cw} e^{i\phi/2}+\hat{b}^\text{out}_\text{ccw} e^{-i\phi/2}\right)/\sqrt{2} \label{eq:a7} \\
    \hat{i}_+ &= \hat{b}_+^\dagger \hat{b}_+ \label{eq:a8} \\
    \hat{i}_- &= \hat{b}_-^\dagger \hat{b}_- \label{eq:a9}
\end{align}

We measure the differential current signal:

\begin{align}
    \hat{i} = \hat{i}_+ - \hat{i}_- \label{eq:a10}
\end{align}

As a figure of merit, we will investigate the minimum detectable frequency shift by calculating the ratio between the standard variation of the measured differential current and the derivative of the mean value of the current over the rotation-induced frequency shift, as reported by Dowling in 1998 \citep{Dowling}:

\begin{align}
    \delta_\text{min} = \frac{\sqrt{\langle \hat{i}^2 \rangle - \langle \hat{i} \rangle^2}}{\left\lvert \frac{\partial \langle \hat{i} \rangle}{\partial \delta} \right\rvert}\Bigg|_{\delta=0} \label{eq:a12}
\end{align}

Since the resonant frequency shift due to the Sagnac effect is given by $\delta = \frac{2\pi r \Omega}{\lambda n_0}$ \citep{MATSKO20182289}, the minimum detectable rotation rate (MDR) is given by:

\begin{align}
    \Omega_\text{min} = \frac{\lambda n_0}{2\pi R} \frac{\sqrt{\langle \hat{i}^2 \rangle - \langle \hat{i} \rangle^2}}{\left\lvert \frac{\partial \langle \hat{i} \rangle}{\partial \delta} \right\rvert}\Bigg|_{\delta=0} \label{eq:a13}
\end{align}

where R and $n_0$ are the radius and the refractive index of the micro-ring. $\lambda$ is the wavelength of the input light. We emphasize that $\Omega_\text{min}$ is a holistic measure that considers the \textit{deterministic} sensitivity of the noise-averaged photocurrent with respect to $\Omega$ as well as the variance of the measured current signals \textit{due to noise} (Both are critical to correctly characterizing the overall sensitivity of the gyro; it has been pointed out~\citep{Wang2020} that an analysis only of the deterministic sensitivity could often lead to misleading conclusions).

If we ignore Rayleigh back-scattering $\beta=0$, we can derive a simple closed-form expression for MDR in the linear gyroscope (LG):

\begin{align}
    \Omega_\text{min}^\text{LG} = \frac{\sqrt{2} \lambda n_0 (\kappa + \gamma)^2}{32 \pi R \kappa \sqrt{N}} \label{eq:a14}
\end{align}

where $N = \frac{P}{\hbar \omega}$ is the incident number of photons per unit time. $\Omega_\text{min}^\text{LG}$ is minimized at $\kappa = \gamma$, yielding $\text{MDR}^\text{LG}_\text{min} = \frac{\sqrt{2} c n_0}{4 R \sqrt{N} Q_i}$, where the intrinsic quality factor is defined by $Q_i = \frac{\omega}{\gamma}$. Note that this is only an example to illustrate the sensitivity dependence of the simplest linear gyroscope without considering Rayleigh back-scattering, which will be taken into account in the following discussions. We note that, in our analysis, we only consider fundamental quantum noise: without loss of generality, we have assumed perfect beam splitters and detectors external to the resonator, while we do consider realistic losses inside the resonator. This linear quantum result will serve as a baseline comparison for our later analysis of a new mode of gyroscope that relies on nonlinear quantum optical effects. Note that the scaling $\text{MDR}^\text{LG}_\text{min} \sim \frac{1}{\sqrt{N} Q_i}$ recovers the familiar shot noise limit or the standard quantum limit \citep{Dowling}. In addition, the $\frac{1}{R}$ dependence in Equation \eqref{eq:a14} indicates that a larger ring radius $R$ offers better sensitivity, which is one of the common control knobs of classical linear optical gyroscope.

% Figure environment removed

\subsection{Nonlinear Multi-Resonant Cavity Quantum Photonics Gyro} \label{sec:Noninearmodel}
We now consider the gyroscopic operation of a doubly resonant ring resonator with quadratic $\chi^{(2)}$ nonlinearities (Fig. \ref{fig:2}). Quadratic nonlinearities are well-known generators of quantum-coherent correlations such as squeezing and entanglement \citep{GaetaandLipson,Masada2015}. Our Nonlinear Multi-resonant Cavity Quantum Photonics Gyro, or Quantum-Optic Nonlinear Gyro (QONG) in short, \textit{fuses} nonlinear dynamics, quantum correlations, and non-inertial Sagnac effects in the same sensor-resonator, to maximally leverage any possible nonlinear quantum-optical effects for gyroscopic sensitivity. Specifically, we investigate the following Heisenberg-Langevin equations:

\begin{align}
\begin{split}
    \frac{d \hat{a}_{1,\text{cw}}}{dt} &= -\left(\frac{\kappa_1}{2} + \frac{\gamma_1}{2} - i\delta_1\right)\hat{a}_{1,\text{cw}} + i\beta_1 \hat{a}_{1,\text{ccw}} \\
    & + \chi \hat{a}_{1,\text{cw}}^\dagger \hat{a}_{2,\text{cw}} + \sqrt{\kappa_1}\hat{b}^\text{in}_{1,\text{cw}} + \sqrt{\gamma_1}\hat{c}^\text{in}_{1,\text{cw}}
\end{split}\label{eq:a15} \\
\begin{split}
    \frac{d \hat{a}_{1,\text{ccw}}}{dt} &= -\left(\frac{\kappa_1}{2} + \frac{\gamma_1}{2} + i\delta_1\right)\hat{a}_{1,\text{ccw}} + i\beta_1 \hat{a}_{1,\text{cw}} \\
    & + \chi \hat{a}_{1,\text{ccw}}^\dagger \hat{a}_{2,\text{ccw}} + \sqrt{\kappa_1}\hat{b}^\text{in}_{1,\text{ccw}} + \sqrt{\gamma_1}\hat{c}^\text{in}_{1,\text{ccw}}
\end{split}\label{eq:a16} \\
\begin{split}
    \frac{d \hat{a}_{2,\text{cw}}}{dt} &= -\left(\frac{\kappa_2}{2} + \frac{\gamma_2}{2} - i\delta_2\right)\hat{a}_{2,\text{cw}} + i\beta_2 \hat{a}_{2,\text{ccw}} \\
    & - \frac{1}{2} \chi \hat{a}_{1,\text{cw}}^2 + \sqrt{\kappa_2}\hat{b}^\text{in}_{2,\text{cw}} + \sqrt{\gamma_2}\hat{c}^\text{in}_{2,\text{cw}}
\end{split}\label{eq:a17} \\
\begin{split}
    \frac{d \hat{a}_{2,\text{ccw}}}{dt} &= -\left(\frac{\kappa_2}{2} + \frac{\gamma_2}{2} + i\delta_2\right)\hat{a}_{2,\text{ccw}} + i\beta_2 \hat{a}_{2,\text{cw}} \\
    & - \frac{1}{2} \chi \hat{a}_{1,\text{ccw}}^2 + \sqrt{\kappa_2}\hat{b}^\text{in}_{2,\text{ccw}} + \sqrt{\gamma_2}\hat{c}^\text{in}_{2,\text{ccw}}
\end{split}\label{eq:a18}
\end{align}

In these equations, the index $j=1,2$ in the field operators ($\hat{a}_j$, $\hat{b}_j$, $\hat{c}_j$) stands for the fundamental $\omega_1$ and second harmonic $\omega_2 = 2\omega_1$ resonances. $\kappa_j$ and $\gamma_j$ are the decay rates of the coupling and the intrinsic loss channels. Rayleigh scattering rate between CW and CCW modes at each resonance $j$ is again characterized by $\beta_j$. The nonlinear coupling terms $\chi \hat{a}_\text{1}^\dagger \hat{a}_\text{2}$ indicate a multi-photon process in which one incident photon with $\omega_2$ breaks down into two photons of half the frequency $\omega_1 = \omega_2/2$ (parametric down conversion \citep{Couteau}), or its reverse, ${1 \over 2} \chi \hat{a}_\text{1}^2$, indicating that two photons with $\omega_1$ combine into one photon with the double frequency $\omega_2 = 2\omega_1$ (second harmonic generation \citep{Kleinman}). These are quantum-coherent, energy-conserving, three-wave mixing processes, which preserve the fundamental commutation relations \citep{Breunig}. The rotation-induced frequency shifts ($\delta_1$, $\delta_2$) are different for each resonance, have opposite polarity between CW and CCW modes, and can be approximated by $\delta_2 = 2\delta_1$ (since $\delta = {2\pi r \Omega \over \lambda n_0}$ \citep{MATSKO20182289}). Note that here the material dispersion of the lithium niobate is neglected due to relatively small index difference (n=2.21 at 1590 nm and n=2.25 at 795 nm). In order to improve the accuracy of the model, however, the dispersion effect should be taken into account in future exploration. The key parameter in this model is the nonlinear modal coupling strength \citep{Drummond2004,Lu:19}:

\begin{align}
    \chi = {\epsilon_0 \over \hbar} \iiint {3 \chi^{\left( 2 \right)} (r) \over 4\sqrt{2}} u_{1}^*(z,r,\theta)^2 u_{2}(z,r,\theta) r dr d\theta dz \label{eq:a19}
\end{align}

where $u_{1}^*(z,r,\theta)$ and $u_{2}(z,r,\theta)$ are the electric field profiles (in polar coordinates) of the fundamental and the second harmonic eigenmodes of the gyroscopic resonator. Given that our sensor is a ring resonator of radius $R$, it is instructive to decompose $\chi$ into cross-sectional modal overlap $\zeta$ and the remaining contributions. Following \citep{Guo:16}, we approximate:

\begin{align}
    \chi &\approx \sqrt{\hbar \omega_1^2 \omega_2 \over \epsilon_0 2\pi R} {\zeta \over \epsilon_1 \sqrt{\epsilon_2}} {3 \chi^{(2)} \over 4 \sqrt{2}} \label{eq:a20}\\
    \zeta &= {\iint u_{1}^*(z,r)^2 u_{2}(z,r) dr dz \over \iint |u_{1}^*(z,r)|^2 dr dz \sqrt{\iint |u_{2}(z,r)|^2 dr dz}} \label{eq:a21}
\end{align}

It is important to realize that the nonlinear Langevin equations \citep{Drummond2004,Drummond} encode the time evolution of \textit{four coupled infinite-dimensional} quantum operators; as such, it is very challenging to obtain an exact solution either analytically or numerically (we note that straightforward numerical methods using a truncated Fock basis \citep{johansson2012qutip} are not feasible because our system typically involves milli-watts of optical power amounting to $\sim 10^{16}$ photons). However, at milli-watt injection powers, quantum fluctuations can be considered ``small signals'' compared to much stronger average field intensities at steady state, so that each operator can be decomposed into a classical scalar-valued amplitude and a quantum fluctuation operator, e.g. $\hat{a} = \alpha + \hat{\delta a}$. The details of calculating steady state solutions are included in Appendix \ref{sec:appendix1}. The classical amplitude represents a steady-state solution to the mean-field averaged Langevin equations at the classical (large photon number) limit while the ``small-signal'' fluctuation operator approximately obeys the linearized Langevin equations in the vicinity of the steady-state mean-field solution. Linearizing a nonlinear steady state to study the fluctuations in its vicinity is commonly known as small-signal modeling in electronics engineering \citep{Tang}. In a similar spirit, the small-signal treatment of quantum fluctuation operators in the Heisenberg-Langevin picture is a simple but effective approach widely accepted for steady-state noise analysis in laser and nonlinear quantum optics literature with experimental support \citep{MATSKO20182289,Gillner,Yamamoto1,chembo2016quantum,pontula2022strong}. Theoretically, it is important to note that such an approach is justified as long as the steady state we consider is a hyperbolic fixed point whose neighborhood is a topologically stable manifold that ensures small fluctuations (Hartman-Grobman theorem \citep{Costa2021}). On the other hand, a more sophisticated phase-space formalism, which employs quasi-probability distributions, Fokker-Planck equations and stochastic calculus, can be used to study more complicated dynamics such as large fluctuations at non-hyperbolic critical points and self-pulsing (limit-cycle) solutions \citep{Drummond_1980}.
Using the small-signal approximation, we can compute the differential photocurrent signals at both the fundamental and the second harmonic resonances (see also Fig. \ref{fig:2}):

\begin{align}
    \hat{i}_1 = i A_1 \left( \hat{b}^{\text{out} \dagger}_\text{1,cw} \hat{b}^\text{out}_\text{1,ccw} e^{-i\phi_1}-\hat{b}^{\text{out} \dagger}_\text{1,ccw} \hat{b}^\text{out}_\text{1,cw} e^{i\phi_1}\right) \label{eq:a22}\\
    \hat{i}_2 = i A_2 \left( \hat{b}^{\text{out} \dagger}_\text{2,cw} \hat{b}^\text{out}_\text{2,ccw} e^{-i\phi_2}-\hat{b}^{\text{out} \dagger}_\text{2,ccw} \hat{b}^\text{out}_\text{2,cw} e^{i\phi_2}\right) \label{eq:a23}
\end{align}

Here $A_1$ and $A_2$ are constant factors determined by the frequencies of the light and the responsivity of the photodetectors. $e^{i\phi_1}$ and $e^{i\phi_2}$ are the propagation phase shifts that each output light experiences, which can be set to zero here. Here we measure both $\hat{i}_1$ and $\hat{i}_2$ to extract maximal information out of the nonlinear wave-mixing gyro.

The output of our quantum-optic nonlinear gyro (QONG) is now characterized by a mean vector $\langle \mathbf{i} \rangle$ and a covariance matrix $\langle \Delta \mathbf{i}^2 \rangle$:

\begin{align}
    {\langle \mathbf{i} \rangle} &= \begin{pmatrix}
    \langle \hat{i}_1 \rangle \\
    \langle \hat{i}_2 \rangle
    \end{pmatrix} \label{eq:a24}\\
    {\langle \Delta \mathbf{i}^2 \rangle} &= \begin{pmatrix}
    \langle \hat{i}_1^2 \rangle - {\langle \hat{i}_1 \rangle}^2 & {\langle \hat{i}_1 \hat{i}_2 \rangle + \langle \hat{i}_2 \hat{i}_1 \rangle \over 2} - \langle \hat{i}_1 \rangle \langle \hat{i}_2 \rangle \\
   {\langle \hat{i}_1 \hat{i}_2 \rangle + \langle \hat{i}_2 \hat{i}_1 \rangle \over 2} - \langle \hat{i}_2 \rangle \langle \hat{i}_1 \rangle & \langle \hat{i}_2^2 \rangle - {\langle \hat{i}_2 \rangle}^2
    \end{pmatrix} \label{eq:a25}
\end{align}

Assuming that the joint probability distribution of the measured photocurrents follow a bi-variate Gaussian, we can express the Fisher information~\citep{Anderson2023} of our QONG:

\begin{align}
    I(\delta) = ({{d \langle \mathbf{i} \rangle} \over {d \delta}})^T {\langle \Delta \mathbf{i}^2 \rangle}^{-1} ({{d \langle \mathbf{i} \rangle} \over {d \delta}}) \label{eq:a26}
\end{align}

The details of the statistial analysis of the differential currents are included in Appendix \ref{sec:appendix2}. Then the sensitivity is determined by Cramer-Rao bound \citep{Braunstein}:

\begin{align}
    \delta_\text{min} = {1 \over {\sqrt{I(\delta)}}} \label{eq:a27}
\end{align}
Similar to the linear gyroscope, here MDR is given by:
\begin{align}
    \Omega_\text{min} = {\lambda n_0 \over 2 \pi R} \delta_\text{min} \label{eq:a28}
\end{align}

Before we provide further estimation for particular QONG implementation, we want to offer a few remarks of our modeling approach:

\begin{itemize}
    \item In this paper, we have stuck to a Langevin description of our quantum gyro, which takes into account quantum noise through the Langevin fluctuation operator $\hat{b}$ or $\hat{c}$ in each coupling or dissipation channel (with the rates determined by the fluctuation-dissipation theorem), preserving the fundamental commutation relations \citep{Drummond2004}. An equivalent formulation can also be considered in terms of a density operator $\hat{\rho}$ \citep{Drummond1}, leading to a Lindblad Master equation of the form (under vacuum noise conditions):
    
\begin{align}
    {d \hat{\rho} \over dt} &= -{i \over \hbar} [H,\hat{\rho}] 
    + \sum \gamma_i [[\hat{a}_{i,j},\hat{\rho}],\hat{a}_{i,j}^\dagger]\\
    H &= \sum \hbar\omega_{i,j} \hat{a}_{i,j}^\dagger \hat{a}_{i,j} 
    +\sum i \hbar {\kappa_i \over 2} (\hat{b}_{i,j}^\dagger \hat{a}_{i,j}  + 
    \hat{a}_{i,j}^\dagger \hat{b}_{i,j} ) \notag \\
    &+ \sum i \hbar {\chi \over 2} ({\hat{a}_{1,j}^\dagger}{^2} \hat{a}_{2,j} - \hat{a}_{1,j}^2 \hat{a}_{2,j}^\dagger) \notag \\
    &+ \sum \hbar \beta_i \left( \hat{a}^\dagger_{i,\text{cw}} \hat{a}_{i,\text{ccw}} + \hat{a}^\dagger_{i,\text{ccw}} \hat{a}_{i,\text{cw}} \right) \label{eq:a31}
\end{align}

We note that while the Langevin form is widely utilized in many experimental situations \citep{Guo:16}, more sophisticated theoretical analysis, delineating the open-system quantum dynamics \citep{Kapral_2015}, can be performed using the density operator formalism and the Master equation, which will be the subject of future investigations. In particular, our simple perturbative approach restricts our solution to examine the quantum fluctuations around a stable hyperbolic fixed point. On the other hand, non-hyperbolic fixed points and non-steady state attractors (such as limit cycles) require more sophisticated non-perturbative treatment (while their implications for quantum correlations and sensing remain unexplored). One such treatment involves expanding the density operator in a non-diagonal coherent state basis (so-called positive P representation), deriving a Fokker-Planck equivalent of the Lindblad Master equation and simulating the associated stochastic dynamics \citep{Drummond2004}. However, to the best of our knowledge, Fokker-Planck equations corresponding to more than two bosonic operators \citep{Drummond_1980,Drummond1} have not been well studied; our nonlinear multi-resonant cavity quantum photonic gyro is described by 4 coupled Langevin equations and will lead to an 8+1 dimensional Fokker Planck equation, which requires substantial computational resources and will be the subject of future investigations.

    \item In our approach, we have assumed idealized sources and detectors in order to simplify our gyroscopic model to physically most crucial components, and thereby to unveil the fundamental information-theoretic limits (in the same spirit as the analysis presented in Ref.~\citep{zhang2019quantum} or the Section 4 of Ref.~\citep{MATSKO20182289}). Future works will develop more detailed models that can compute commonly accepted experimental metrics such as the integration-time dependent Allan deviation curve \citep{Giglio}, for example, by incorporating the quantum theory of photodetection \citep{Shapiro,Carmichael1999}, which can explicitly take into account photo-electron generation rates and detector integration times.
    
    \item Last but not least, we note that our present model focuses on $\chi^{(2)}$ processes to delineate their effects on the gyroscopic sensitivity. A more thorough gyro model may also consider $\chi^{(3)}$ (Kerr-type self modulation) nonlinearities, which may come into effect at ultra-high quality factors and are found to limit the sensitivity of the (otherwise) \textit{linear} gyroscope~\cite{MATSKO20182289,vahala}. While we shall take into account $\chi^{(3)}$ processes in detailed comprehensive models in the future (see Section \ref{sec:s&o}), we note that Eqs.~\ref{eq:a15}-\ref{eq:a18} are fully applicable to material platforms, such as thin film lithium niobate \citep{Zhu:21,Lu:19}, which possess prominent $\chi^{(2)}$. Furthermore, unlike their linear counterparts, resonators with $\chi^{(2)}$ can be engineered to exhibit negative Kerr shifts via cascaded second-order effects \citep{Cui2022}, which can mitigate the intrinsic positive Kerr shift; we shall investigate such cancellation schemes in our future works. On the other hand, we would like to emphasize that $\chi^{(3)}$ processes, including even the Kerr shift, need not be treated as a nuisance, but as extra complexities and additional degrees of freedom that can be optimized to our advantage (see Section \ref{sec:s&o}). For example, it has been recently reported that the bistability effects associated with the Kerr-shift self-modulation can even enhance sensitivities under appropriate sensing schemes~\cite{silver2021nonlinear,peters2022exceptional}.
\end{itemize}

% Figure environment removed

\subsection{Thin film lithium niobate as an implementation platform}
Our nonlinear multi-resonant cavity quantum photonic gyro (or quantum-optic nonlinear gyro QONG in short) can be implemented in any thin film material platform, including LiNbO3 \citep{Zhu:21}, AlN \citep{ALN}, SiC \citep{Sic}, GaAs \citep{GaAs}, etc, which has prominent $\chi^{(2)}$. In this work we consider thin film lithium niobate (TFLN) as a particularly promising platform, as it has gained widespread popularity for realizing quantum-grade ultra-low loss photonic integrated circuits \citep{Alireza,McCutcheon:09}. Indeed, lithium niobate has been traditionally employed in quantum optics applications as a nonlinear medium for generating squeezed light and entangled photon states \citep{Chen:22,Zhao}. However, traditional LN crystals are bulky and suffer from relatively limited strength of light-matter interactions (leading to very weak nonlinear coupling $\chi \sim 10^3~\mathrm{Hz}$ in Eqs.~\ref{eq:a20}). Only recently, high quality wafer-scale TFLN becomes widely available for realizing integrated photonic circuits with nonlinear and electro-optic functionalities~\citep{Zhu:21}. Associated with large $\chi^{(2)}$, low optical loss and strong nanophotonic confinement \citep{MianZhang}, TFLN devices offer orders of magnitude enhancements in nonlinear coupling $\chi \sim 10^6 \mathrm{Hz}$ \citep{Lu:19}.

Fig.~\ref{fig:3} shows the design of a TFLN ring resonator which can used as an QONG. Note that feeder waveguides of different dimensions, frequency cutoffs, and dispersion characteristics, can be designed to selectively couple to the fundamental (1590nm) and the second harmonic (795nm) modes \citep{Bi:12}, and their coupling rates can be further tuned by TFLN electro-optics~\citep{Shams-Ansari2022}. To realize strong nonlinear coupling $\chi$ between the two resonances, two zeroth-order transverse electric eigenmodes (TE00) can be (quasi-)phase-matched \citep{Lu:19} via periodic poling \citep{Fejer} that achieves crystal domain inversion, leading to periodically varying nonlinear susceptibility $\chi^{(2)}$ which compensates wave vector mismatch between the fundamental and the second harmonic modes $k_\chi = k(\omega_2) - 2 k(\omega_1)$ \citep{Lu:19}. Apart from the phase-matched resonator itself, a fully integrated QONG can be implemented in TFLN, incorporating flip-chip bonded semiconductor lasers \citep{Okhotnikov} and heterogeneously integrated uni-travelling carrier photodetectors \citep{Guo:22}. Furthermore, we note that TFLN comes with unique electro-optic control capabilities \citep{Zhu:21} which can be used for tuning resonator parameters such as the coupling rates to the waveguides \citep{Guarino2007}, managing long term temperature stability, cancelling thermal drifts and electronic noise \citep{Khial2018,Wang:22}, and performing signal processing \citep{8979176}.

% Figure environment removed

In our gyroscopic model, intrinsic losses $\gamma$ and back-scattering rates $\beta$ should be treated as ``fixed'' parameters which depend on the experimentally feasible characteristics of a particular implementation platform, such as residual material losses and surface roughness due to fabrication imperfections. In thin film lithium niobate, intrinsic quality factors reaching $\sim 10^8$ have been demonstrated \citep{Tobias}, and we expect proportionate back-scattering rates with the same order of magnitude. It is important to realize that, apart from $\gamma$ and $\beta$, almost all other parameters can be designed, engineered and optimized, including injection powers, coupling rates, resonator radius, resonator waveguide cross section and dispersion as well as quasi-phase matching processes. We will utilize these parameters as degrees of freedom (DoF) in optimizing the Fisher Information and hence the minimum detectable rotation (MDR) of our QONG (as compared to the linear gyro). While any optimization algorithm can be employed, gradient-free global optimization methods are most suitable for a relatively low-dimensional problem like our two-resonance gyro (where about 5--10 DoFs can be optimized). We will use Bayesian optimization, a simple but powerful machine-learning-based optimization algorithm which requires relatively few function evaluations (as compared to other heuristic methods such as simulated annealing and evolutionary algorithms \citep{SimulatedAnnealing,Eiben}) and has been observed to be particularly effective for optimizing $\sim 20$ DoFs \citep{Shahriari}.

\section{Results}
\label{sec:r&d}
As noted above, the Fisher Information and the Minimum Detectable Rotation (MDR) of the gyroscope is determined by various parameters and can be optimized by judiciously adjusting their values. In our design, the operational wavelengths (of the input/output light) are fixed at $\lambda_1$ = 1590 nm and at $\lambda_2$ =795 nm. The material property of TFLN is taken from the literature \citep{MianZhang}: the refractive index is $n=2.2$ while the second order nonlinear susceptibility is $\chi^{(2)}$ = 30 pm/V. The intrinsic quality factors are fixed at $Q_{i1} = 10^7$ for the 1590 nm and $Q_{i2} = 10^6$ for the 795 nm. The Rayleigh scattering rates for both cavity modes $\beta_1=5.4 \times 10^4$ Hz and $\beta_2=5.4 \times 10^5$ Hz are inferred by adjusting the literature-reported values~\citep{vahala} to the quality factors $Q_i$ of our TFLN platform. We have also fixed the radius of the resonator at $R =20$ mm as well as the cross-sectional dimensions of the resonator waveguide ($1.2~\si{\micro\meter}$ width $\times 0.6~\si{\micro\meter}$ thickness) and the fabrication side wall angle of $75 \si{\degree}$, leading to a cross-sectional area of $0.8~\si{\micro\meter}^2$. The two TE00 modes have phase mismatch of $1.354~\si{\micro\meter}^{-1}$, which can be compensated by poling with a period of $4.64~\si{\micro\meter}$. Based on Eqs.~\ref{eq:a20}, the quasi-phase matched $\chi^{(2)}$ \citep{Lu:19}, and the numerically simulated modal overlapping factor $\zeta = 1.18~/\si{\micro\meter}$, the nonlinear coupling strength is $\chi = 1.26 \times 10^6$ Hz, which is independent of the injection schemes. The rest of the parameters remain to be determined, including the injection power at the fundamental and the second harmonics $P_1$ and $P_2$, the quality factors due to coupling to the waveguides for each cavity mode: $Q_{c1}$ and $Q_{c2}$. These four parameters will be determined by Bayesian optimization. We will investigate gyroscopic performance under different injection schemes including (1) coherent state input at the second harmonic ($\lambda_2$), (2) coherent state input at the sub-harmonic ($\lambda_1$), and (3) coherent state inputs at both second and sub-harmonics ($\lambda_1$ and $\lambda_2$).

\subsection{Optical parametric oscillator gyro (coherent injection at second harmonic)}\label{Sec:para}
First, we study the performance of an optical parametric oscillator gyroscope under the coherent injection at the second harmonic frequency. As shown in Fig. \ref{fig:4}(a), classical laser light with a wavelength $\lambda_2$ = 795 nm is injected from opposite directions from the waveguide ports $b^{\text{in}}_\text{2,cw}$ and $b^{\text{in}}_\text{2,ccw}$, while no light is injected from the waveguide port $b^{\text{in}}_\text{1,cw}$ or $b^{\text{in}}_\text{1,cw}$, i.e., $P_1=0$. The carefully phase-matched fundamental and second harmonic modes facilitate parametric down conversion, in which one photon with higher frequency (shorter wavelength $\lambda_2$ breaks down into two photons with half the frequency (longer wavelength $\lambda_1$), generating phase-squeezed signals~\citep{Drummond2004}. First we investigate the mininum detectable rotation (MDR) as a function of the input power $P_2$ and the coupling factor $Q_{c2}$. Using Bayesian optimization, we identify a high sensitivity regime, that is, low MDR, as shown in the 2D density plot of MDR in terms of $P_2$ and $Q_{c2}$ (Fig.~\ref{fig:4}b). We mapped over the parameter space where $Q_{c2}$ ranges from $5 \times 10^5$ to $6 \times 10^5$ and $P_2$ ranges from 20 mW to 30 mW. At the second harmonic injection, the selection of $P_2$ is determined by the critical power $P_c$, below which the steady-state solutions of the system become unstable. This phenomenon has been discussed in details by Drummond \citep{Drummond}. Here $P_c$ = 14.05 mW such that $P_2$  should be larger than this value. Within this region, small MDR (0 - 1.7 $^\circ$/h) is observed (indicated by rainbow colors), showing that high sensitivity is achieved for a sizable parameter range such that the enhanced sensitivity is not an isolated singularity. The lowest MDR ($\Omega_\text{min} < 0.25$ $^\circ$/h) appears in the narrow purple band. Outside this region, MDR gradually increases as the color becomes more greenish and reddish, indicating reduced sensitivity. In principle, as the rotation rate increases, the difference between the CW and CCW mode becomes more significant. To benchmark the gyroscope performance, we compared the sensitivity of our optical parametric oscillator (OPO) gyro under the second harmonic injection (solid blue line) with the sensitivity of a standard linear gyroscope (solid red line) in Fig. \ref{fig:4}(c). The optimal coupling factors for the OPO gyro are $Q_{c1}=1.018 \times 10^5$ and $Q_{c2} = 5.462 \times 10^5$, both discovered by Bayesian optimization. Meanwhile, the highest sensitivity of the linear gyroscope is found at $Q_{c1} = 9.58 \times 10^6 \approx 10^7$. Note that, in the limit of vanishing $\beta_1$, the lowest linear gyro MDR is achieved when $\kappa_1 = \gamma_1$ (see Eq.~\ref{eq:a14}). Since $Q_{i1}$ is fixed at $10^7$, the optimal $Q_{c1}$ for the linear gyro is expected to be close but not exactly equal to this value, considering the influence of the small but non-zero back-scattering.
In Fig. \ref{fig:4}(d), as the input power is increased from 20 mW to 30 mW, the MDR of the OPO gyro drops from $\sim$ 0.42 $^\circ$/h down to near zero and rises back to $\sim$ 1.21 $^\circ$/h with a local minimum at 23.507 mW, corresponding to the optimal sensitivity point ($\Omega_\text{min} = 4.465 \times 10^{-3}$ $^\circ$/h). To better visualize this local sensitivity minimum, a zoomed-in curve is shown in the inset of Fig. \ref{fig:4}(d). We also found that the optimal-sensitivity point is associated with 9.9 dB phase squeezing at an experimentally feasible value on a TFLN platform \citep{Alireza}. Meanwhile, the MDR of the linear gyroscope remains $> 0.49~^\circ$/h, which demonstrates that the OPO gyro is $\sim$ 124.4 $\times$ more sensitive than the linear gyro under the same injection power, resonator size and intrinsic quality factors. In order to visualize this effect, we investigated the mean current values as a function of the rotation rate $\Omega$ at different frequencies in Fig. \ref{fig:4}(d). The mean values of the output differential currents at the subharomonic ($<i_1>$) and the second harmonic ($<i_2>$) are expressed by the solid blue and red line respectively, while that of the linear gyroscope at the same power consumption is expressed by the solid green line for reference. It is shown that as $\Omega$ increases from 0 - 100 $^\circ$/h, $<i_1>$ increases from 0 - 0.43 nA, $<i_2>$ increases from 0 - 0.4 nA while the output current of the linear gyroscope increases from 0 - 0.17 nA. This result shows that the nonlinear eigenmode dispersion, as discussed in Section \ref{sec:Noninearmodel}, produces stronger output signals (differential currents) at both wavelengths compared to the linear gyroscope.

% Figure environment removed

\subsection{Coherent injection at the fundamental frequency}
Aside from injecting light at the second harmonic, we also investigated the scheme of the fundamental frequency (subharmonic) injection. As shown in Fig. \ref{fig:6}(a), the input light at the wavelength $\lambda_1$ = 1590 nm is injected from the waveguide ports $b^{\text{in}}_\text{1,cw}$ and $b^{\text{in}}_\text{1,cw}$ while $P_2=0$, stimulating intra-cavity up-conversion (two photons of lower energy are combined to one photon of higher energy). We study the sensitivity (MDR) in terms of $P_1$ and $Q_{c1}$, as indicated by the 2D density plot Fig.~\ref{fig:6}(b), in which variations in MDR again show up in rainbow colors.  In Fig. \ref{fig:6}(b) we mapped over the parameter space where $Q_{c1}$ ranges from $6 \times 10^6$ to $7 \times 10^6$ and $P_1$ ranges from $0.9~\mu W$ to $1~\mu W$. Note that much lower power injection is required at this injection scheme compared to coherent injection at the second harmonic, related to the fact that the steady-state solutions of the cavity modes are stable only when $P_1$ is below the critical power when the second harmonic injection is absent \citep{Drummond}. Here the critical power is $P_c = 3.24$ mW. Only solutions of $P_1$ smaller than this value are stable, resulting in orders of magnitude lower power consumption. In Fig. \ref{fig:6}(b), MDR varies from 0 - 5 $^\circ$/h across the entire parameter space. The lowest MDR is found in a narrow band-like region. The mean values of the output currents are also studied here. To better evaluate the gyroscope performance, we compared it with the linear gyroscope at the same power consumption. Figure \ref{fig:6}(c) shows the sensitivity of the fundamental frequency injection (solid blue line) and the linear gyroscope (solid red line) in terms of the injection power at fixed optimal Q factors ($Q_{c1}=6.747 \times 10^6, Q_{c2} = 6.675 \times 10^7$), which are determined by Bayesian optimization, and optimal $Q_c = 9.58 \times 10^6$ in the linear gyro. As $P_1$ increases from $0.9~\mu W$ to $1~\mu W$, MDR drops from $\sim$ 0.46 $^\circ$/h down to near zero at $0.945~\mu W$, then rises back to $\sim$ 0.49 $^\circ$/h. Meanwhile, the sensitivity of the linear gyroscope monotonically but slowly decreases from $\sim$ 89.78 $^\circ$/h to $\sim$ 85.17 $^\circ$/h. The lowest MDR is found in a narrow band-like region. The optimal sensitivity $\Omega_\text{min} = 0.093$ $^\circ$/h is found at $P_1=0.945$ $\mu W$. At the optimal point, a surprisingly high sensitivity improvement of $\sim$ 942.5 $\times$ is observed (0.093 $^\circ$/h vs 87.62 $^\circ$/h). To better visualize the sensitivity of the fundamental frequency injection scheme, a magnified plot of the optimal sensitivity region is shown in the inset of Fig. \ref{fig:6}(d), clearly exhibiting the local sensitivity minimum at $0.945~\mu W$. We note that no phase squeezing is observed in this scheme (although a 6.03 dB amplitude squeezing was observed at the second harmonic (795 nm)); rather than squeezing, we posit that the sensitivity improvement is largely due to nonlinear eigenmode dispersion in the presence of critically sensitive $\chi^{(2)}$-mediated wave mixing processes, not disimilar to enlarged Sagnac shifts reported in dispersive materials \citep{Dispersivegyro}. Most importantly, the fundamental frequency injection scheme merges high sensitivity and low power consumption together in a compact form, which shows great potential in practical applications. Similar to the analysis in Section \ref{Sec:para}, the mean differential currents ($<i_1>$, $<i_2>$ and linear) as a function of the rotation rate $\Omega$ are shown in Fig. \ref{fig:6}(d) in solid blue, red and green lines. As $\Omega$ increases from 0 - 100 $^\circ$/h, $<i_1>$ increases from  0 - 0.05 pA, $<i_2>$ increases from 0 - 0.0062 pA and the output differential current of the linear gyroscope increases from 0 - 0.0069 pA. The differential currents are in the pA range, not the nA range, again because of much weaker input power. It is shown that the sensitivity of $<i_1>$ is almost $10 \times$ larger than that of $<i_2>$ and that of the output current in the linear gyroscope, suggesting that, even though very little second harmonic power is ultimately extracted, the presence of a non-linearly interacting second harmonic mode critically enhances the sensitivity of the fundamental mode.

% Figure environment removed

\subsection{Dual frequency injection}
For the sake of completeness, we also studied the dual injection scheme (Fig.~\ref{fig:7}a) where coherent light is injected at both fundamental and second harmonics ($\lambda_1$ = 1590 nm and $\lambda_2$ = 795 nm). Here, we optimize four independent parameters: $P_1$, $P_2$, $Q_{c1}$, and $Q_{c2}$. The 2D density plot of the sensitivity (MDR) in terms of $P_1$ and $P_2$ is shown in Fig. \ref{fig:7}(b). Here $Q_{c1}$ and $Q_{c2}$ are fixed at the optimal values of $4.353 \times 10^5$ and $8.769 \times 10^6$, again discovered by Bayesian optimization. In Fig. \ref{fig:7}(b), $P_1$ ranges from 1 mW to 2 mW and $P_2$ ranges from 1 mW to 2 mW. Similar to the scheme of fundamental frequency injection, we found that low input power is also necessary for high sensitivity, which is beneficial for integrated optical gyroscopes. As shown in the figure, MDR ranges from 0 - 1.4 $^\circ$/h across the entire parameter space. The region of the lowest MDR is expressed by a narrow purple band in the figure while outside this region MDR gradually increases. We also compare the MDR of the dual injection scheme (solid blue line) with the linear gyroscope (solid red line), as shown in Fig. \ref{fig:7}(c). Note that in the figure the x-axis denotes the total input power $P$, which equals to $P_1+P_2$ for the dual frequency injection and $P_1$ for the linear gyroscope. Here $Q_{c1}$, $Q_{c2}$ and $P_2$ are fixed at $4.353 \times 10^5$, $8.769 \times 10^6$ and 1.873 mW, which are the optimal values discovered by Bayesian optimization. It is shown that the sensitivity of dual frequency injection drops from $\sim 0.88 ^\circ$/h down to near zero as $P$ increases from 3 mW to 3.373 mW, then rises back to $\sim$ 0.762 $ ^\circ$/h as $P$ increases to 4 mW while sensitivity of the linear gyroscope drops from 1.555 $^\circ$/h to 1.347 $^\circ$/h within the same range of injection power. At the optimal power (3.373 mW), a high sensitivity of 0.013 $^\circ$/h is observed, leading to a substantial sensitivity improvement of $\sim$ 113.1 $\times$ over the linear gyroscope (0.013 $^\circ$/h vs 1.47 $^\circ$/h). To better visualize this optimal point, a magnified plot of the optimal sensitivity region is shown in the inset of Fig. \ref{fig:7}(c). To better understand the influence of rotation, the mean output currents are shown in Fig. \ref{fig:7}(d). As the rotation rate $\Omega$ increases from 0 - 100 $ ^\circ$/h, $<i_1>$, $<i_2>$ and the mean differential current of the linear gyroscope monotonically increases from 0 to 5.3 pA, 11.1 pA and 24.7 pA respectively. Note that here neither the sub nor the second harmonic mode shows stronger output current compared to the linear gyroscope. This is because the injection power of the linear gyroscope equals to the summation of both the sub and the second harmonic injection. Nonetheless, significant sensitivity improvement is still observed in this case due to the combination of the nonlinear coupling and the generation of the phase-squeezed photons.

% Figure environment removed

\subsection{Discussion}
\label{sec:dtab}
Table \ref{tab:1} summarizes the maximal sensitivity enhancement factors (over the linear baseline) that can obtained in multiple operational regimes over a wide range of critical power requirements. Under the optimal second-harmonic injection at $\approx 23.5$ mW, a 9.9 dB quadrature squeezing is predicted, where our nonlinear multi-resonant cavity quantum photonics gyro (or quantum-optic nonlinear gyro QONG in short) can be nearly $124.4 \times$ more sensitive than an optimized linear gyro with the same radius, intrinsic quality factor and power budget, allowing for a minimum detectable rotation (MDR) as small as 0.0044 $^\circ$/h. Alternatively, even larger enhancement factors can be obtained at smaller powers under the fundamental and the dual frequency injection schemes. The dual frequency injection scheme achieves near 7 $\times$ sensitivity improvement over the fundamental frequency injection scheme (0.013 $^\circ$/h vs 0.093 $^\circ$/h), which could be the result of the extra squeezed photons generated by the process of parametric down conversion. In fact, these two latter scenarios do not necessarily promote squeezing but rely on nonlinear eigenmode dispersion due to critically-sensitive three-wave mixing dynamics between the two resonances---an effect that resembles the enlarged Sagnac shifts observed in suitably dispersive materials \citep{Dispersivegyro}.
In either of our fundamental or second-harmonic injection scheme, we measured the output signals at both the fundamental and the second harmonic frequencies in order to fully utilize the input pump power (which gets converted into both harmonics), setting up a fair comparison to a linear gyro under the same pump power. For a more conservative comparison, one may argue for using dual inputs and outputs in the linear case. Aside from the fact that having to use two different frequency lasers can be disadvantageous, a simple calculation readily shows that measuring two \textit{non-interacting} resonances in a linear gyro can offer only up to $\sqrt{2}\times$ improvement (under the same power budgets)---in fact, much less than $\sqrt{2}$ due to the smaller $Q_{i2}$---highlighting that nonlinear effects are indeed indispensable for significant sensitivity enhancements. Most importantly, the crucial insight we have drawn from our investigations is to realize that multiple resonances in a nonlinear resonator can be engineered to reinforce each other through nonlinear wave mixing, and can be used as powerful degrees of freedom to optimize sensitivities. This critical realization suggests an exciting future direction: to generalize our QONG from just two resonances to many more nonlinearly interacting resonances (see also Section \ref{sec:s&o}), which may lead to even better sensitivities and functionalities (approaching the ultimate Heisenberg limit).
\begin{table}
  \centering
  % Figure removed
  \caption{Optimal sensitivities of various injection schemes}
  \label{tab:1}
\end{table}
\section{Summary and Outlook}
\label{sec:s&o}
We have introduced a new type of quantum light gyroscopes based on nonlinear multi-resonant cavity quantum photonics (quantum-optic nonlinear gyro or QONG in short). Specifically, we analyzed and optimized the quantum-enhanced gyroscopic sensitivity of a doubly resonant $\chi^{(2)}$ cavity, revealing that, under quantum noise conditions, $\gtrsim 900\times$ enhancement is possible over the classical shot noise limit. We highlight that our current design, which uses two resonances, represents only an elementary step and a relatively simple example of an QONG. In future works, we will develop a \textit{comprehensive} QONG inertial sensing paradigm, where a synergistic amalgamation of both quadratic $\chi^{(2)}$ and cubic $\chi^{(3)}$ nonlinearities, along with multiple intermixing resonances, mutually reinforced Sagnac shifts, co-arising quantum correlations, electro-optics dynamical control and geometry-induced anomalous dispersion effects, can unleash extraordinary complexities and freedoms, which can be fully exploited by state-of-the-art optimization techniques \citep{SaadHikmat,floudas2013deterministic,baydin2018automatic,Shakir} in order to identify unprecedented regimes for gyroscopic operation and sensitivities. A full incarnation of an QONG can be described by a Heisenberg-Langevin system of the form (or an equivalent density-operator Master equation \citep{Drummond_1980}):

\begin{align}
{d\hat{a}^\mu_j \over dt} &= \left(i\omega_j + i \delta_j^\mu(\Omega) - {\kappa_j \over 2}- {\gamma_j \over 2}\right) \hat{a}^\mu_j 
+ i \beta_j \hat{a}^\nu_j \notag \\
&+ \sum_{kl\alpha\beta} f^{(2)}_{kl\alpha\beta}\left( \hat{a}^\alpha_k, \hat{a}^\beta_l, (\hat{a}^\alpha_k)^\dagger, (\hat{a}^\beta_l)^\dagger \right ) \notag \\
&+ \sum_{klm\alpha\beta\theta} f^{(3)}_{klm\alpha\beta\theta}\left( \hat{a}^\alpha_k, \hat{a}^\beta_l, \hat{a}^\theta_m, 
(\hat{a}^\alpha_k)^\dagger, (\hat{a}^\beta_l)^\dagger, (\hat{a}^\theta_m)^\dagger \right ) \notag \\
&+ \sum_k \sqrt{\kappa^c_{jk}} \hat{a}_{jk,in}^\mu + \sum_k \sqrt{\gamma^r_{jk}} \hat{\eta}_{jk}^\mu \label{eq:a32}
\end{align}

for a selected set of carefully phase-matched and dispersion-engineered resonances $\{ \omega_j, j=1,...,N \}$. Here, $\delta^\mu (\Omega), \mu \in \{\text{cw,ccw}\}$, is the rotation-dependent Sagnac shift in the CW or CCW mode at each resonance. The functions $f^{(2)}$ and $f^{(3)}$ are polynomials of the annihilation and creation operators, representing all possible quantum-coherent three-wave mixing and four-wave mixing interactions between the selected resonances; these processes include sum and difference frequency generations of different orders and combinations as well as Kerr-variety self-phase and cross-phase modulation, and even cascaded processes \citep{Misoguti}. It is important to note that the strengths of different $f^{(2)}$ and $f^{(3)}$ terms are determined by nonlinear coupling factors \citep{Guo:16} which characterize the field concentration and nonlinear overlaps of the modes of the photonic resonator and can be computed from nanophotonic simulations. Therefore, on-chip structural parameters, ranging from a few simple shape parameters to entire permittivity distributions, can serve as design degrees of freedom \citep{Molesky2018}, by which we can engineer and optimize the different nonlinear processes (e.g. their relative contributions). The outputs of this multi-resonance system are collected by multiple waveguide ports and are set to passively interfere with each other and/or go through active electro-optics pulse processing (readily achievable on a TFLN platform \citep{Zhu:21}) before arriving at multiple photodetectors to yield multiple photocurrent signals $\mathbf{i} = \{\hat{i}_1,...,\hat{i}_M\}$. From these multi-variable (vector-valued) measurements, one can perform deep inferential analysis (such as advanced Bayesian computing \citep{csillery2010approximate}) to deduce the underlying non-inertial motion; the sensitivity of the entire process can be characterized by an end-to-end computation of Fisher Information, which will serve as an optimization figure of merit. We recognize tremendous opportunity in analyzing and optimizing such a system with increasing levels of mathematical and computational vigor, starting from steady-state analysis, small-signal modeling, classical stochastic simulations, to the non-perturbative quantum phase-space apparatus involving positive P-representations, Fokker-Planck equations and stochastic calculus \citep{Drummond2004,Drummond_1980}, from few-parameter deterministic global optimization \citep{horst2013global,floudas2013deterministic}, multi-parameter Bayesian optimization \citep{Shahriari} and evolutionary algorithms \citep{Back}, machine-learning assisted hybrid optimization \citep{Qi}, and Monte Carlo gradient computations \citep{Shakir} to billion-voxel topology optimization \citep{aage2017giga} and full end-to-end inverse design \citep{Zin2} of the entire workflow from the underlying resonator geometry to multi-variable inferential processes. Experimentally, thin film lithium niobate (TFLN) continues to offer the most suitable platform which features state-of-the-art on-chip frequency combs, pulse shaping, frequency shifting and ultra-fast signal processing capabilities \citep{McCutcheon:09,Wang2019,Yu2022,Lu2023}.

\begin{acknowledgments}
We thank Charles Roques-Carmes, Steven G. Johnson, Kiyoul Yang and Michael Larsen for informative discussions. Financial support was provided by generous gifts from the Virginia Tech Foundation. VK thanks the late Stavros Katsanevas for pointing out the new runs of VIRGO based on quantum mechanical states in November of 2018.
\end{acknowledgments}

\appendix

\section{Steady state solutions}\label{sec:appendix1}

As discussed in Section \ref{sec:Noninearmodel}, the nonlinear coupled equations are solved by linearization. The classical scalar valued amplitudes $\alpha$ are obtained from the steady state analysis of Eqs.~\ref{eq:a15}-\ref{eq:a18}. To evaluate the steady state solutions,  the noise terms containing the quantum operators are omitted. The equations are thus simplified as follows:
\begin{align}
\begin{split}
    f_1 &= \left({\kappa_1 \over 2} + {\gamma_1 \over 2}-i\delta_1\right)a_\text{1,cw}-i\beta_1 a_\text{1,ccw} \\
    & -\chi a_\text{1,cw}^* a_\text{2,cw}-\sqrt{\kappa_1}b^\text{in}_\text{1,cw} 
\end{split}\label{eq:a33}\\
\begin{split}
    f_2 &= \left({\kappa_1 \over 2} + {\gamma_1 \over 2}+i\delta_1\right) a_\text{1,ccw}-i\beta_1 a_\text{1,cw} \\
    & -\chi a_\text{1,ccw}^* a_\text{2,ccw}-\sqrt{\kappa_1}b^\text{in}_\text{1,ccw} 
\end{split}\label{eq:a34}\\
\begin{split}
    f_3 &= \left({\kappa_2 \over 2} + {\gamma_2 \over 2}-i\delta_2\right)a_\text{2,cw}-i\beta_2 a_\text{2,ccw} \\
    & +{1 \over 2} \chi a_\text{1,cw}^2-\sqrt{\kappa_2}b^\text{in}_\text{2,cw}
\end{split}\label{eq:a35}\\
\begin{split}
    f_4 &= \left({\kappa_2 \over 2} + {\gamma_2 \over 2}+i\delta_2\right) a_\text{2,ccw}-i\beta_2 a_\text{2,cw} \\
    & +{1 \over 2} \chi a_\text{1,ccw}^2-\sqrt{\kappa_2}b^\text{in}_\text{2,ccw}
\end{split}\label{eq:a36}
\end{align}

Next, the steady state solutions are hence obtained by solving the equations $F(f_1,f_2,f_3,f_4) = 0$. Aside from the cavity modes $a_{n,cw/ccw}$, the input fields are also expressed as the steady states $b_{n,cw/ccw}$. These values are determined by the input power $P_n$ at both waveguide ports as discussed in Equation~\ref{eq:a3}. Here we fix $b_n$ as real values, which reduces the phase noise as reported by Dowling \citep{Dowling}. At different injection schemes, different arrangements of $b_n$ are employed. For example, $b_2=0$ at when the input light is injected at the fundamental frequency and $b_1=0$ at the second harmonic injection. Though the steady state solutions of a similar system has been studied by Drummond \citep{Drummond}, in which the analytical solutions are given at each injection scheme. In our system, however, the rotation-induced frequency shift $\delta_n$ and the Rayleigh back-scattering $\beta_n$ which introduces cross-coupling between the CW and the CCW modes make it impossible to calculate the analytical solution, hence we calculated the numerical solutions instead. Since these equations are nonlinear equations, multiple solutions are expected at each set of parameters. In order to discover the steady state solutions, linear stability analysis is performed \citep{erneux_glorieux_2010,GAVRIELIDES1997253}. By checking the eigen values of the Jacobian matrix \citep{2022Linear} associated with each set of solution we can determine the stability of these solutions. The Jacobian matrix J of Eqs~\ref{eq:a33}-\ref{eq:a36} is obtained by taking the gradient over a vector of the unknown variables $a_\text{n,cw/ccw}$:
\begin{align}
    J &= \nabla F |_{a_\text{n,cw/ccw}} \label{eq:a37}
\end{align}
Note that in Equation \ref{eq:a37} the function system $F$ and the variables $a_\text{n,cw/ccw}$ are both vectors, hence the gradient operator $\nabla$ generates a matrix $J$ instead of a vector. When the real part of the eigen values of the matrix is negative, the solution is stable and can be used for further calculation. 

\section{The algebra of the quantum operators}\label{sec:appendix2}

The system is assumed to be quantum-limited, meaning that the shot noise is considered as the main source of noise. To this end, it is necessary to investigate the statistical properties of the output light. As discussed in Section \ref{sec:Noninearmodel}, the output light are expressed by the operators $\hat{b}_\text{n,cw/ccw}^\text{out}$. As shown in Fig. \ref{fig:2}, the output light is measured by homodyne detection that $\hat{b}_\text{n,cw}^\text{out}$ and $\hat{b}_\text{n,ccw}^\text{out}$ are coupled with each other before being detected by two independent photodetectors:
\begin{align}
    \hat{b}_{1+} = \left( \hat{b}^\text{out}_\text{1,cw} e^{-i\phi_1}+i \hat{b}^\text{out}_\text{1,ccw} e^{i\phi_1}\right)/\sqrt{2} \label{eq:a38}\\
    \hat{b}_{1-} = \left( i \hat{b}^\text{out}_\text{1,cw} e^{-i\phi_1}+ \hat{b}^\text{out}_\text{1,ccw} e^{i\phi_1}\right)/\sqrt{2} \label{eq:a39}\\
    \hat{b}_{2+} = \left( \hat{b}^\text{out}_\text{2,cw} e^{-i\phi_2}+i \hat{b}^\text{out}_\text{2,ccw} e^{i\phi_2}\right)/\sqrt{2} \label{eq:a40}\\
    \hat{b}_{2-} = \left( i \hat{b}^\text{out}_\text{2,cw} e^{-i\phi_2}+ \hat{b}^\text{out}_\text{2,ccw} e^{i\phi_2}\right)/\sqrt{2} \label{eq:a41}
\end{align}
Here $\phi_1$ and $\phi_2$ are the propagation phase shifts of the output light at the sub and the second harmonics, which can be arbitrarily selected. Here we set them to zero. The resultant differential current current is given by:
\begin{align}
    \hat{i}_{1} = R \hbar \omega_1 \left( \hat{b}_{1+}^\dagger \hat{b}_{1+} - \hat{b}_{1-}^\dagger \hat{b}_{1-}\right) \label{eq:a42}\\
    \hat{i}_{2} = R \hbar \omega_2 \left( \hat{b}_{2+}^\dagger \hat{b}_{2+} - \hat{b}_{2-}^\dagger \hat{b}_{2-}\right) \label{eq:a43}
\end{align}
Here R is the responsivity of the photodetectors, set as $0.58~A/W$ in our analysis. Setting $A_1=R \hbar \omega_1$ and $A_2=R \hbar \omega_2$, Eqs.~\ref{eq:a42}-\ref{eq:a43} can be further simplified as:
\begin{align}
    \hat{i}_1 = i A_1 \left( \hat{b}^{\text{out} \dagger}_\text{1,cw} \hat{b}^\text{out}_\text{1,ccw} -\hat{b}^{\text{out} \dagger}_\text{1,ccw} \hat{b}^\text{out}_\text{1,cw} \right) \label{eq:a44}\\
    \hat{i}_2 = i A_2 \left( \hat{b}^{\text{out} \dagger}_\text{2,cw} \hat{b}^\text{out}_\text{2,ccw} -\hat{b}^{\text{out} \dagger}_\text{2,ccw} \hat{b}^\text{out}_\text{2,cw} \right) \label{eq:a45}
\end{align}
Following Maleki's approach \citep{MATSKO20182289}, the output operators are linearized as $\hat{b}_\text{n,cw/ccw}^\text{out} = b_\text{n,cw/ccw}^\text{out} + \hat{\delta b}_\text{n,cw/ccw}^\text{out}$. Here we analyze the perturbation terms, such that Eqs.~\ref{eq:a44}- \ref{eq:a45} are simplified as:
\begin{align}
\begin{split}
    \hat{\delta i}_1 = i A_1 \bigl(& b^{\text{out}}_\text{1,ccw} \hat{\delta b}^{\text{out} \dagger}_\text{1,cw} + b^{\text{out} *}_\text{1,cw} \hat{\delta b}^\text{out}_\text{1,ccw} \\
    &- b^{\text{out}^*}_\text{1,ccw} \hat{\delta b}^\text{out}_\text{1,cw} - b^{\text{out}}_\text{1,cw} \hat{\delta b}^{\text{out} \dagger}_\text{1,ccw} \bigr) 
\end{split}\label{eq:a46}\\
\begin{split}
    \hat{\delta i}_2 = i A_2 \bigl(& b^{\text{out}}_\text{2,ccw} \hat{\delta b}^{\text{out} \dagger}_\text{2,cw} + b^{\text{out} *}_\text{2,cw} \hat{\delta b}^\text{out}_\text{2,ccw} \\
    &- b^{\text{out}^*}_\text{2,ccw} \hat{\delta b}^\text{out}_\text{2,cw} - b^{\text{out}}_\text{2,cw} \hat{\delta b}^{\text{out} \dagger}_\text{2,ccw} \bigr)
\end{split}\label{eq:a47}
\end{align}
Then quadrature basis expansion is performed to separate the real and the imaginary parts ($X$ and $Y$) of the operators:
\begin{align}
    b_\text{n,cw/ccw}^\text{out} = X_\text{n,cw/ccw}^\text{out} + i Y_\text{n,cw/ccw}^\text{out} \label{eq:a48}\\
    b_\text{n,cw/ccw}^{\text{out} *} = X_\text{n,cw/ccw}^\text{out} - i Y_\text{n,cw/ccw}^\text{out} \label{eq:a49}\\
    \hat{\delta b}^\text{out}_\text{n,cw/ccw} = \hat{\delta X}_\text{n,cw/ccw}^\text{out} + i \hat{\delta Y}_\text{n,cw/ccw}^\text{out} \label{eq:a50}\\
    \hat{\delta b}^{\text{out} \dagger}_\text{n,cw/ccw} = \hat{\delta X}_\text{n,cw/ccw}^\text{out} - i \hat{\delta Y}_\text{n,cw/ccw}^\text{out} \label{eq:a51}
\end{align}
Hence, Eqs.~\ref{eq:a46}-\ref{eq:a47} are converted to:
\begin{align}
\begin{split}
    \hat{\delta i}_1 = 2 A_1 \bigl(& X^\text{out}_\text{1,ccw} \hat{\delta Y}^\text{out}_\text{1,cw} - Y^\text{out}_\text{1,ccw} \hat{\delta X}^\text{out}_\text{1,cw} \\
    &- X^\text{out}_\text{1,cw} \hat{\delta Y}^\text{out}_\text{1,ccw} + Y^\text{out}_\text{1,cw} \hat{\delta X}^\text{out}_\text{1,ccw} \bigr)
\end{split}\label{eq:a52}\\
\begin{split}
    \hat{\delta i}_2 = 2 A_2 \bigl(& X^\text{out}_\text{2,ccw} \hat{\delta Y}^\text{out}_\text{2,cw} - Y^\text{out}_\text{2,ccw} \hat{\delta X}^\text{out}_\text{2,cw} \\
    &- X^\text{out}_\text{2,cw} \hat{\delta Y}^\text{out}_\text{2,ccw} + Y^\text{out}_\text{2,cw} \hat{\delta X}^\text{out}_\text{2,ccw} \bigr)
\end{split}\label{eq:a53}
\end{align}
Next we need to determine the statistical properties of $\hat{\delta i}_1$ and $\hat{\delta i}_2$. Nevertheless, the output light is in complex quantum states (squeezed vacuum/squeezed coherent) which are difficult to calculate. On the other hand, these quadrature operators are nothing but linear combinations of the input light which is in relatively simple quantum states (vacuum/coherent). To this end, we calculate the mean values and the variances from the input states. 
Rewrite Eqs.~\ref{eq:a52} and \ref{eq:a53} in the form below:
\begin{align}
    {\hat{\delta i}_1} &= \sum_{n=1}^2 (b_\text{x,n,cw/ccw}^{(1)} \hat{b}_\text{X,n,cw/ccw}^\text{in} + b_\text{y,n,cw/ccw}^{(1)} \hat{b}_\text{Y,n,cw/ccw}^\text{in} \notag \\
&+ c_\text{x,n,cw/ccw}^{(1)} \hat{c}_\text{X,n,cw/ccw}^\text{in} + c_\text{y,n,cw/ccw}^{(1)} \hat{c}_\text{Y,n,cw/ccw}^\text{in} ) \label{eq:a54} \\
    {\hat{\delta i}_2} &= \sum_{n=1}^2 (b_\text{x,n,cw/ccw}^{(2)} \hat{b}_\text{X,n,cw/ccw}^\text{in} + b_\text{y,n,cw/ccw}^{(2)} \hat{b}_\text{Y,n,cw/ccw}^\text{in} \notag \\
&+ c_\text{x,n,cw/ccw}^{(2)} \hat{c}_\text{X,n,cw/ccw}^\text{in} + c_\text{y,n,cw/ccw}^{(2)} \hat{c}_\text{Y,n,cw/ccw}^\text{in} ) \label{eq:a55}
\end{align}
In Eqs.~\ref{eq:a54}-\ref{eq:a55}, $\hat{b}_\text{X,n,cw/ccw}^\text{in}$, $\hat{b}_\text{Y,n,cw/ccw}^\text{in}$, $\hat{c}_\text{X,n,cw/ccw}^\text{in}$ and $\hat{c}_\text{Y,n,cw/ccw}^\text{in}$ are the quadrature operators (real and imaginary parts) of the injection light light $\hat{b}_\text{n,cw/ccw}^\text{in}$ and the intrinsic loss channels $\hat{c}_\text{n,cw/ccw}^\text{in}$ for sub/second (n=1,2) harmonic light, where $\hat{b}_\text{X,n,cw/ccw}^\text{in}$, $\hat{b}_\text{Y,n,cw/ccw}^\text{in}$ are in coherent states and   $\hat{c}_\text{X,n,cw/ccw}^\text{in}$ and $\hat{c}_\text{Y,n,cw/ccw}^\text{in}$ are in vacuum states.  $b_\text{x,n,cw/ccw}^{(1,2)}$, $b_\text{y,n,cw/ccw}^{(1,2)}$, $c_\text{x,n,cw/ccw}^{(1,2)}$ and $c_\text{y,n,cw/ccw}^{(1,2)}$ are the corresponding coefficients of these operators at either sub or second harmonic. Assume $\psi_1$/$\psi_2$ and $N_1$/$N_2$ are the initial phases and the numbers of the injected photons of the input light $\hat{b}_\text{1,cw/ccw}$ and $\hat{b}_\text{2,cw/ccw}$, where $N_1 = |b_1^\text{in}|^2$ and $N_2 = |b_2^\text{in}|^2$.
With all these ingredients, now we can calculate mean values and the variances of both differential currents. For a coherent state $|\alpha>$, the mean values of both quadrature operators and their squares are given by~\citep{scully_zubairy_1997}:
\begin{align}
    <\alpha|\hat{X}|\alpha> = \sqrt{N} \cos{\psi} \label{eq:a56} \\
    <\alpha|\hat{Y}|\alpha> = \sqrt{N} \sin{\psi} \label{eq:a57} \\
    <\alpha|\hat{X}^2|\alpha> = {{4 N (\cos{\psi})^2 +1} \over 4} \label{eq:a58} \\
    <\alpha|\hat{Y}^2|\alpha> = {{4 N (\sin{\psi})^2 +1} \over 4} \label{eq:a59}
\end{align}
and the variances are defined as the mean values of the square minus the square of the mean values of the quadrature operators:
\begin{align}
    <\alpha|\Delta \hat{X}^2|\alpha> = <\alpha|\hat{X}^2|\alpha> - (<\alpha|\hat{X}|\alpha>)^2
    = {1 \over 4} \label{eq:a60} \\
    <\alpha|\Delta \hat{Y}^2|\alpha> = <\alpha|\hat{Y}^2|\alpha> - (<\alpha|\hat{Y}|\alpha>)^2
    = {1 \over 4} \label{eq:a61}
\end{align}
Note that when two quadrature operators do not share the same eigen vectors, the definitions are different. For example, the inner product of a real quadrature operator at the second harmonic and an imaginary quadrature operator at the fundamental frequency injection is given by:
\begin{align}
    <\alpha|\hat{X_2} \hat{Y_1}|\alpha> = \sqrt{N_1} \sqrt{N_2} \cos{\psi_2} \sin{\psi_2} \label{eq:a62} \\
    <\alpha|\hat{X_2} \hat{Y_1}|\alpha> - <\alpha|\hat{X_1} |\alpha> <\alpha|\hat{Y_1} |\alpha> = 0\label{eq:a63}
\end{align}
Now we can obtain the mean values and the variances of $\hat{\delta i}_1$ and $\hat{\delta i}_2$:
\begin{align}
\begin{split}
    <\hat{\delta i}_1> = \sum_{n=1}^2 \bigl(&b_\text{x,n,cw/ccw}^{(1)} \sqrt{N_n} \cos{\psi_n} \\
    &+ b_\text{y,n,cw/ccw}^{(1)} \sqrt{N_n} \sin{\psi_n} \bigr)
\end{split}\label{eq:a64} \\
\begin{split}
    <\hat{\delta i}_2> = \sum_{n=1}^2 \bigl(&b_\text{x,n,cw/ccw}^{(2)} \sqrt{N_n} \cos{\psi_n} \\
    &+ b_\text{y,n,cw/ccw}^{(2)} \sqrt{N_n} \sin{\psi_n} \bigr)
\end{split}\label{eq:a65} \\
\begin{split}
    {<\Delta \hat{\delta i}_1^2>} = {1 \over 4} \sum_{n=1}^2 \bigl([&(b_\text{x,n,cw/ccw}^{(1)})^2 + (b_\text{y,n,cw/ccw}^{(1)})^2 \\
    &+ (c_\text{x,n,cw/ccw}^{(1)})^2+(c_\text{y,n,cw/ccw}^{(1)})^2 ]\bigr)
\end{split}\label{eq:a66} \\
\begin{split}
    {<\Delta \hat{\delta i}_2^2>} = {1 \over 4} \sum_{n=1}^2 \bigl([&(b_\text{x,n,cw/ccw}^{(2)})^2 + (b_\text{y,n,cw/ccw}^{(2)})^2 \\
    &+ (c_\text{x,n,cw/ccw}^{(2)})^2+(c_\text{y,n,cw/ccw}^{(2)})^2 ]\bigr)
\end{split}\label{eq:a67} 
\end{align}
As discussed in Section \ref{sec:Noninearmodel}, in order to determine the covariance matrix, it is also necessary to calculate the correlation between $\hat{\delta i}_1$ and $\hat{\delta i}_2$ following the rule of operator calculation defined above:
\begin{align}
\begin{split}
    {< \hat{\delta i}_1 \hat{\delta i}_2> - <\hat{\delta i}_1> <\hat{\delta i}_2>} &= \sum_{n=1}^2 \bigl[b_\text{x,n,cw/ccw}^{(1)} b_\text{x,n,cw/ccw}^{(2)} \\
    &+ b_\text{y,n,cw/ccw}^{(1)} b_\text{y,n,cw/ccw}^{(2)} \\
    &+ c_\text{x,n,cw/ccw}^{(1)} c_\text{x,n,cw/ccw}^{(2)} \\
    &+c_\text{y,n,cw/ccw}^{(1)} c_\text{y,n,cw/ccw}^{(2)}\bigr]
\end{split}\label{eq:a68} \\
    {< \hat{\delta i}_2 \hat{\delta i}_1> - <\hat{\delta i}_2> <\hat{\delta i}_1>} &= {< \hat{\delta i}_1 \hat{\delta i}_2> - <\hat{\delta i}_1> <\hat{\delta i}_2>} \label{eq:a69}
\end{align}
With everything discussed in this section, particularly Eqs.~\ref{eq:a64}-\ref{eq:a69}, now we can calculate Eqs.~\ref{eq:a24}-\ref{eq:a26} to determine the Fisher information and the corresponding sensitivity of the system.

% The \nocite command causes all entries in a bibliography to be printed out
% whether or not they are actually referenced in the text. This is appropriate
% for the sample file to show the different styles of references, but authors
% most likely will not want to use it.
\nocite{*}

\bibliography{apssamp}% Produces the bibliography via BibTeX.

\end{document}
%
% ****** End of file apssamp.tex ******
