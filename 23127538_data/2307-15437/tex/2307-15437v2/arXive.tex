\documentclass[aps,prl,10pt,superscriptaddress,twocolumn,longbibliography]{revtex4-2}
% \renewcommand{\baselinestretch}{2.0}

\usepackage{tikz}
\usepackage[cmex10]{amsmath}
\usepackage{amssymb}
\usepackage{physics}
\usepackage{upgreek}
\usepackage{titlesec}
\usepackage{siunitx}
\usepackage{here}
% \usepackage{color}
% \usepackage{comment}
\usepackage{pdfpages}
\usepackage{pgffor} % for loops

% Fix for a pdfpages rotation bug with revtex
\makeatletter
\AtBeginDocument{\let\LS@rot\@undefined}
\makeatother

% the name of the supplement PDF file
\def\supplementfilename{Supplementary.pdf}

% Determine the number of pages 
% in the supplement file and store
\pdfximage{\supplementfilename}
\def\numbersupplementpages{\the\pdflastximagepages}

% Are we submitting to the arXiv? 
% Un-comment the appropriate line
\newif\ifarXiv
\arXivtrue 
% \arXivfalse

\newcommand{\qflux}{\frac{\Phi_0}{2\pi}}
\newcommand{\half}{\frac{1}{2}}
\newcommand{\vp}{\varphi}
\newcommand{\ej}{E_\mathrm{J}}
\newcommand{\ec}{E_\mathrm{c}}
\newcommand{\cj}{C_\mathrm{J}}
\newcommand{\ejec}{E_\mathrm{J}/E_\mathrm{c}}

\newcommand{\bt}[1]{{\color{blue}#1}}

% \usepackage{xcolor}
% \usepackage{ulem}
% \DeclareRobustCommand{\erasesec}{\bgroup\markoverwith{\textcolor{red}{\rule[.5ex]{2pt}{0.4pt}}}\ULon}
% \newcommand{\addsec}[1]{\textcolor{red}{#1}}
% \newcommand{\addsec}[1]{\textcolor{black}{#1}}
% \newcommand{\erasesec}[1]{}
% \newcommand{\add}[1]{\textcolor{black}{#1}}
% \newcommand{\erase}[1]{}

\usepackage{hyperref}
\hypersetup{
    colorlinks=true,
    linktocpage=true,
    linkcolor=blue,
    filecolor=magenta,      
    urlcolor=cyan,
    bookmarks=true,}
    
\graphicspath{{images/}}



\begin{document}
\title{One photon simultaneously excites two atoms \\ in a ultrastrongly coupled light-matter system}

\author{A. Tomonaga}\email{akiyoshi.tomonaga@aist.go.jp}\thanks{Current affiliation is National Institute of Advanced Industrial Science and Technology (AIST), Tsukuba, Ibaraki 305-8563, Japan}
\affiliation{Department of Physics, Tokyo University of Science, Shinjuku, Tokyo 162--0825, Japan}
\affiliation{Center for Quantum Computing, RIKEN, Wakoshi, Saitama 351--0198, Japan}
\author{R. Stassi}
\affiliation{Dipartimento di Scienze Matematiche e Informatiche, Scienze Fisiche e Scienze della Terra, Universit\`a di Messina, I-98166 Messina, Italy}
\affiliation{Theoretical Quantum Physics Laboratory, Cluster for Pioneering Research, RIKEN, Wakoshi, Saitama 351-0198, Japan}
\author{H. Mukai}
\affiliation{Department of Physics, Tokyo University of Science, Shinjuku, Tokyo 162--0825, Japan}
\affiliation{Center for Quantum Computing, RIKEN, Wakoshi, Saitama 351--0198, Japan}
\author{F. Nori}
\affiliation{Center for Quantum Computing, RIKEN, Wakoshi, Saitama 351--0198, Japan}
\affiliation{Theoretical Quantum Physics Laboratory, Cluster for Pioneering Research, RIKEN, Wakoshi, Saitama 351-0198, Japan}
\affiliation{Physics Department, The University of Michigan, Ann Arbor, Michigan 48109-1040, USA.}
\author{F. Yoshihara}
\affiliation{Department of Physics, Tokyo University of Science, Shinjuku, Tokyo 162--0825, Japan}
\affiliation{Advanced ICT Research Institute, National Institute of Information and Communications Technology, Koganei, Tokyo 184-8795, Japan}
\author{J. S. Tsai}\email{tsai@riken.jp}
\affiliation{Department of Physics, Tokyo University of Science, Shinjuku, Tokyo 162--0825, Japan}
\affiliation{Center for Quantum Computing, RIKEN, Wakoshi, Saitama 351--0198, Japan}


\begin{abstract}
We experimentally investigate a superconducting circuit composed of two flux qubits ultrastrongly coupled to a common $LC$ resonator. Owing to the large anharmonicity of the flux qubits, the system can be correctly described by a generalized Dicke Hamiltonian containing spin-spin interaction terms. 
In the experimentally measured spectrum, an avoided level crossing provides evidence of the exotic interaction that allows the \textit{simultaneous} excitation of \textit{two} artificial atoms by absorbing \textit{one} photon from the resonator. This multi-atom ultrastrongly coupled system opens the door to studying nonlinear optics where the number of excitations is not conserved. This enables novel processes  for quantum-information processing tasks on a chip.
\end{abstract}

\maketitle

%%%%%%%%%%%%%%%%%%%%%%%%%%%%%%%%%%%%%%%%%%%

%\section*{Introduction}
{\it Introduction.}---Superconducting circuits provide a versatile and flexible platform for modeling various quantum systems~\cite{nakamura_coherent_1999,krantz_quantum_2019,blais_circuit_2021}. 
In this platform, artificial atoms can be designed to have tailored energy transitions and controllable interactions with microwave photons~\cite{gu_microwave_2017}. Moreover, superconducting circuits also became one of the main platforms for scalable quantum information processing and quantum simulation~\cite{kjaergaard_superconducting_2020,kwon_gate-based_2021}.

Taking advantage of the high electromagnetic field in a one-dimensional resonator and the huge dipole moment of artificial atoms, these systems achieve a larger light-matter interaction than the bare atomic or resonator frequencies~\cite{niemczyk_circuit_2010, kockum_ultrastrong_2019, forn-diaz_ultrastrong_2019, yoshihara_superconducting_2017,forn-diaz_ultrastrong_2017}. This ultrastrong (deep-strong) interaction might lead to promising applications, such as high-speed and high-efficiency quantum information processing devices~\cite{romero_ultrafast_2012,wang_ultrafast_2017,stassi_scalable_2020,chen_fast_2021}, and the observation of unique physical phenomena, such as quantum vacuum radiation and entanglement in the ground state~\cite{stassi_spontaneous_2013,PhysRevA.81.042311}.
One of the most fascinating theoretical predictions in the ultrastrong coupling regime is that one photon can simultaneously excite two atoms~\cite{garziano_one_2016} when the parity symmetry of the system is broken. Similarly to Rabi oscillations, this process, which is mediated by virtual excitations, is a coherent and unitary process and the atoms can jointly emit one photon.
The reverse phenomenon, i.e., two photon excitation of an atom or molecule, is currently adopted for specific spectroscopic machines~\cite{doi:10.1146/annurev.bioeng.2.1.399,denk_two-photon_1990}. Likewise, we believe that the two-atom excitation process can open the door to new applications. 

We experimentally investigated a circuit composed of two flux qubits ultrastrongly coupled to a common $LC$ resonator. Flux qubits,  which form the artificial atoms, share the same inductor with the $LC$ resonator; as a consequence, they interact with each other. This system is described by the Dicke Hamiltonian generalized to include atomic longitudinal couplings and the spin-spin interaction term.

Away from the flux qubit optimal point, where the parity symmetry of the system is broken, in the experimentally measured spectrum, we observe an energy-level anti-crossing, which indicates hybridization between the bare states $\ket{gg1}$ and $\ket{ee0}$, where $g$($e$) and $0$ respectively indicate the atomic ground (excited) and zero photon states. This is the fingerprint of the interaction that allows \textit{one} photon to \textit{simultaneously} excite \textit{two} atoms and the reverse process. When the system is set up in the one-photon state, the artificial atoms and the resonator can exchange excitation in a Rabi-like oscillation.

Since the atom--light and atom--atom interactions are very strong, the atomic states should be strongly hybridized with each other, and should not be possible to clearly observe the effect of ``one photon exciting two atom'' (OPETA) effect. However, the direct atom-atom interaction partially suppress the atom--light interaction. Moreover, depending on the phase of the longitudinal interaction, the light and matter decouple. Therefore, the spectrum is asymmetric with respect to the sign of the flux bias, and the OPETA effect is clearly observable.
%%%%%%%%%%
% Figure environment removed

%%%%%%%%%%%%%%%%%%%%%%
%\section*{Results}
%\section*{Device description}
{\it Device description.}---Figure~\ref{Fig1}(a) shows an optical microscope image of the artificial-atom--resonator circuit. The $LC$ resonator is composed of an interdigital capacitor and line inductance made of a superconducting thin film~\cite{miyanaga_ultrastrong_2021,zotova_compact_2023}.
The two flux qubits are inductively coupled to the $LC$ resonator via a Josephson junction [Fig.~\ref{Fig1}(b)], which increases the couplings to the ultrastrong regime.
The energies of the flux qubits~\cite{chiorescu_coherent_2004} can be changed applying an external magnetic flux to the loop from a global coil and using an on-chip bias line.
Figure~\ref{Fig1}(c) shows the equivalent circuit with lumped elements and Josephson junctions.

The Hamiltonian of the entire system is~\cite{tomonaga_quasiparticle_2021,billangeon_circuit-qed-based_2015,PhysRevB.73.174526}
\begin{align}
\hat{\mathcal{H}}_{\mathrm{tot}}=\hat{\mathcal{H}}_\mathrm{q1}+\hat{\mathcal{H}}_\mathrm{q2}+\hat{\mathcal{H}}_\mathrm{r}+\hat{\mathcal{H}}_{\mathrm{int}} 
\,,
\label{eq:TotalHami}
\end{align}
where $\hat{\mathcal{H}}_{\mathrm{q}k}$ ($k=1,2$), $\hat{\mathcal{H}}_\mathrm{r}$, and $\hat{\mathcal{H}}_{\mathrm{int}}$ represent the qubits, resonator, and atom-resonator plus atom-atom couplings, respectively.
The Hamiltonian of the resonator is $\hat{\mathcal{H}}_\mathrm{r}=\hbar\omega_\mathrm{r}\qty(\hat{a}^\dagger \hat{a}+1/2)$,
where $\omega_\mathrm{r}\equiv 1/\sqrt{L_\mathrm{r}C_\mathrm{r}}$ is the resonance frequency, $\hat{a}\equiv(\hat{\phi}_\mathrm{r}-iZ_\mathrm{r} \hat{q}_\mathrm{r})/\sqrt{2\hbar Z_\mathrm{r}}$ is the annihilation operator, $\hat{a}^\dagger\equiv(\hat{\phi}_\mathrm{r}+iZ_\mathrm{r} \hat{q}_\mathrm{r})/\sqrt{2\hbar Z_\mathrm{r}}$ is the creation operator, $Z_\mathrm{r}=\sqrt{L_\mathrm{r}/C_\mathrm{r}}$ is the characteristic impedance of the $LC$ resonator, and $\hat{q}_\mathrm{r}$ is the conjugate variable of $\hat{\phi}_\mathrm{r}=\Phi_0 \hat{\varphi}_\mathrm{r}$.
The Hamiltonian of the $k$-th artificial atom is defined as
\begin{align}
\hat{\mathcal{H}}_{{\mathrm{q}k}}\equiv4E_\mathrm{c}\hat{\vb{q}}_k^\mathrm{T} \vb{M}_k^{-1}{\hat{\vb{q}}_k}
+E_\mathrm{Lr}\hat{\varphi}_{\beta k}^2
+\hat{\mathcal{U}}_{\mathrm{J}k}\,,
\label{eq:Qhami}
\end{align}
where $E_c$ is the charging energy of the Josephson junction, $\vb{M}_k$ is the normalized mass matrix, $E_\mathrm{Lr}=\Phi_0^2/(2L_\mathrm{r})$, and $\hat{\mathcal{U}}_{\mathrm{J}k}$ is the qubit potential energy of Josephson junctions:
\begin{align}
        \!\!\! \hat{\mathcal{U}}_{\mathrm{J}k}(\hat{\varphi}_{\mathrm{e}k}) =& -\ej
        \left[\,
            \beta_k\cos{(\hat{\varphi}_{\beta k})}
            +\cos{(\hat{\varphi}_{ak})}+\cos{(\hat{\varphi}_{bk})}
        \right. \notag \\ 
        &\,\,\,\,\left.
        +\alpha_k\cos{(\varphi_{\mathrm{e}k}-\hat{\varphi}_{ak}-\hat{\varphi}_{bk}-\hat{\varphi}_{\beta k})}
        \right] \,.
    \label{eq:uj}
\end{align}
Here, $E_\mathrm{J}$ is the current energy of the Josephson junction, and $\varphi_{\mathrm{e}k}$ represents the external flux for the loop of each atom. The interaction Hamiltonian
\begin{align}
\hat{\mathcal{H}}_{\mathrm{int}}=-E_\mathrm{Lr}(\hat{\varphi}_{\beta1}\hat{\varphi}_\mathrm{r}-\hat{\varphi}_{\beta2}\hat{\varphi}_\mathrm{r}+\hat{\varphi}_{\beta1}\hat{\varphi}_{\beta2})
    \label{eq:Hint}
\end{align}
is obtained from the boundary condition (Kirchhoff's voltage law) of the loop forming the resonator with elements $L_\mathrm{r}$ and $C_\mathrm{r}$.

By approximating each atom as a two-level system~\cite{yoshihara_hamiltonian_2022}, on the basis of persistent currents of the superconducting loop, we obtain the total Hamiltonian in Eq.~\eqref{eq:TotalHami} as
%using the eigenvectors $\ket{i}_k$ (${i}\in \mathbb{N}$) of the atom Hamiltonians $\mathcal{H}_{\mathrm{q}k}$, and 
% \begin{align}
% \hat{\mathcal{H}}_{\mathrm{tot}}=&\hbar\sum_{i}{(\Omega_{i}^{(1)}\ket{i}_1\bra{i}_1+\Omega_{i}^{(2)}\ket{i}_2\bra{i}_2)}
% +\hat{\mathcal{H}}_\mathrm{r} \notag \\ \notag
% &-\hbar\sum_{i,j}{(g_{ij}^{(1)}\ket{i}_1\bra{j}_1+g_{ij}^{(2)}\ket{i}_2\bra{j}_2)(\hat{a}^\dagger+\hat{a})}    \\
% &-E_\mathrm{L}\sum_{i,j}{g_{ij}^{(1)}g_{ij}^{(2)}\ket{i}_1\ket{i}_2\bra{j}_1\bra{j}_2}\,,
% \label{Hket}
% \end{align}
% where $\hbar\Omega_{i}^{(k)}$ is $i$-th eigenenergy of atom $k$ and $\hbar g_{ij}^{(k)}=I_\mathrm{zpf}\Phi_0\mel{i}{\hat{\varphi}_{\beta k}}{j}$ is the coupling matrix element ($I_{\mathrm{zpf}}=\sqrt{\hbar\omega_\mathrm{r}/2L_\mathrm{r}}$)
\begin{align}
\hat{\mathcal{H}}_\mathrm{tot}/\hbar&\simeq\omega_\mathrm{r}\hat{a}^\dagger \hat{a}+\frac{\varepsilon_1}{2}\hat{\sigma}_{z1}+\frac{\Delta_1}{2}\hat{\sigma}_{x1}+\frac{\varepsilon_2}{2}\hat{\sigma}_{z2}+\frac{\Delta_2}{2}\hat{\sigma}_{x2} \notag
\\&-(g_1\hat{\sigma}_{z1}-g_2\hat{\sigma}_{z2})(\hat{a}^\dagger+\hat{a}) 
-\frac{2g_1 g_2}{\omega_\mathrm{r}}\hat{\sigma}_{z1}\hat{\sigma}_{z2}\,,
\label{HRabi_z}
\end{align}
where $\varepsilon_k$ is the persistent current energy of each qubit, $\Delta_k$ is the qubit energy gap when $\varepsilon_k=0$, while $\hat{\sigma}_{zk}$ and $\hat{\sigma}_{xk}$ are the Pauli matrices for the $k$-th qubit. We define $\varepsilon_k>0$ when the qubit current flows anticlockwise and vice versa.

After a unitary transformation that diagonalizes the atomic Hamiltonians $\hat{\mathcal{H}}_{\mathrm{qk}}$, we obtain a generalized Dick Hamiltonian~\cite{PhysRevA.86.014303} with the spin-spin interaction:
\begin{align}
\hat{\mathcal{H}}_\mathrm{tot}/\hbar\simeq\omega_\mathrm{r}\hat{a}^\dagger \hat{a}+\frac{\omega_\mathrm{q1}}{2}\hat{\sigma}_{z1}+\frac{\omega_\mathrm{q2}}{2}&\hat{\sigma}_{z2}\notag \\
-(g_1\hat{\Lambda}_1-g_2\hat{\Lambda}_2)(\hat{a}^\dagger+\hat{a}) 
&-\frac{2g_1 g_2}{\omega_\mathrm{r}}\hat{\Lambda}_1\hat{\Lambda}_2\,,
\label{HRabi}
\end{align}
where $\omega_{\mathrm{q}k}={\rm sgn}(\varepsilon_k)(\varepsilon_k^2+\Delta_k^2)^{1/2}$ is the qubit frequency and $\hat{\Lambda}_k=(\cos{\theta_k}\,\hat{\sigma}_{xk}+\sin{\theta_k}\,\hat{\sigma}_{zk})$ gives the direction of the interaction, with $\theta_k\simeq-\arctan(\Delta_k/\varepsilon_k)$. 
%The sign of $\omega_{\mathrm{q}k}$ corresponds to that of $\varepsilon_k$.
For $\theta_k=0$ $(\varepsilon_k=0)$ the interaction is transverse. When $\theta_k\neq0$, the interaction has a longitudinal component and the OPETA effect is allowed.

%$\theta_k=\arctan(-\abs{g_{01}^{(k)}}/g_{00}^{(k)})\simeq\arctan(\abs{\varepsilon_k}/\Delta_k)$ 
%The spin-spin interaction reduces the current flowing in the resonator loop, which in this system is the ferromagnetic coupling.
%The relationship between Eq.~\eqref{HRabi} and Eq.~\eqref{HRabi_z} is unitary rotation transformation with $\theta_k$.
%In two level approximation, $\omega_{\mathrm{q}k}$, $\varepsilon_k$, and $\Delta_k$ are derived as absolute value, but we need to consider the current direction (signs of $\varepsilon_k$ and/or $\omega_{\mathrm{q}k}$) because two qubit are coupled to the same resonator. The boundary condition of the resonator loop (relative phase of $\varepsilon_{\beta1}$ and $\varepsilon_{\beta2}$) make shift the lowest energy of cosine potential Eq.~\eqref{eq:uj} and the asymmetry property for energy spectrum. 
%By define the sign of $\varepsilon_k$ and/or $\omega_{\mathrm{q}k}$, we can describe this asymmetry property even though we use the two level Hamiltonian.  
%We can also define $\omega_{\mathrm{q}k}\equiv\mathrm{sign}(\varepsilon_k)(\varepsilon_k^2+\Delta_k^2)^{1/2}$ when $\varepsilon_k\neq0$ and $\omega_{\mathrm{q}k}\equiv\Delta_k$ when $\varepsilon_k=0$.

% Figure environment removed

%%%%%%%%%%%%%%%%%%%%%%
%\section*{Energy spectrum}
{\it Energy spectrum.}---Figure~\ref{Fig2}(a) shows the raw data of the measured spectrum as a function of the persistent current energy $\varepsilon_1$ of qubit 1, after fixing the value of $\varepsilon_2/2\pi$ at $-3.21$~GHz when $\varepsilon_1=0$.
In Fig.~\ref{Fig2}(b) the spectrum is fitted with the numerically calculated transition frequencies $\omega_{ij}$ between the $i$-th and $j$-th eigenstates of the total Hamiltonian $\hat{\mathcal{H}}_\mathrm{tot}$.
The persistent current energy for qubit 2 and the resonator frequency are affected by the external magnetic flux applied to qubit 1~\cite{yoshihara_superconducting_2017}.
Thus, to derive the transitions frequencies $\omega_{ij}$, in Eq.~\eqref{HRabi_z}, we substitute $\varepsilon_2\xrightarrow{}\varepsilon_{2}+A\varepsilon_1$ and $\omega_\mathrm{r}\xrightarrow{}\omega_\mathrm{r}(1+B_\pm\varepsilon_1)$, where $A$ and $B_\pm$ are fitting parameters. Because the spectrum is asymmetric with respect to the sign of $\varepsilon_1$, we use two different values for $B_\pm$, where $B_+$ is used when $\varepsilon_1\ge0$ and vice versa.
From the fitting, we obtain $A=-9.10\times10^{-3}$, $B_-=1.13\times10^{-3}$, and $B_+=0.79\times10^{-3}$.

Flux qubits 1 and 2 are almost identical except for the loop size; consequently, they have similar fitted parameters, i.e., $\Delta_\mathrm{q1}=0.26\,\omega_\mathrm{r}$ and $\Delta_\mathrm{q2}=0.25\,\omega_\mathrm{r}$.
We find atom-resonator couplings rates of $g_{1}/\omega_\mathrm{r}=0.64$ and $g_{2}/\omega_\mathrm{r}=0.66$, indicating that the artificial atoms are ultrastrongly coupled with the resonator.
% In total, we use 11 parameters to fit the spectrum, this includes the offset value when $\varepsilon_1=0$ and the persistent current $I_\mathrm{p1}$ in qubit 1 to derive $\varepsilon_1=I_\mathrm{p1}\Phi_0(\varphi_\mathrm{e1}-0.5)$, where $\Phi_0$ is the flux quantum.
% field, because $\varphi_{\beta 1}$ changes the shared Josephson inductance as a function of $L_{\mathrm{J}\beta 1}/\cos{\varphi_{\beta 1}}$.The effective resonator frequency represented by $\omega_\mathrm{r}'(\varphi_{\beta1})=[C_\mathrm{r} (L_\mathrm{r}+L_{\mathrm{J}\beta 1}/\cos{\varphi_{\beta 1})}]^{-1/2}$.At $\varphi_{\mathrm{e}1}/2\pi\simeq0.5$, $\varphi_{\beta1}\simeq0$ and $\varepsilon_{1}$ is proportional to $\varphi_{\beta 1}$. 

% In the spectrum at $\varepsilon_1\simeq0$, $\omega_{41}$ and $\omega_{51}$ vanish because the transition matrix element $\mel{\psi_i}{\hat{\sigma}_{z1}\hat{\sigma}_{z2}+a^\dagger+a}{\psi_j}$ becames zero, where $\ket{\psi_i}$ is the $i$-th eigenvector of the total Hamiltonian~\eqref{HRabi}.

Observing $\omega_{30}$ (blue curve) and $\omega_{40}$ (red curve) in Figs.~\ref{Fig2}(b) and especially \ref{Fig3}(a), it is possible to notice that the spectrum is asymmetric with respect to the sign of $\varepsilon_1$. This occurs due to the presence of atom-light longitudinal interactions when two or more qubits are coupled to the same cavity mode \cite{jaako2016ultrastrong}. Assuming that there are only longitudinal couplings, the atomic states are associated to photonic coherent states. However, if $\varepsilon_1>0$, the atomic and photonic states are decoupled if $M=m_1-m_2=0$, where $m_k=\pm 1$ is the eigenstate of $\sigma_{zk}$ (details in Supplemental Material~\cite{noauthor_see_nodate}). If $\varepsilon_1<0$, the atomic and photonic states are decoupled if $M=-m_1-m_2=0$. In our experimental setup we have both longitudinal and transverse couplings; the presence of the longitudinal components justifies the asymmetry in Fig.~\ref{Fig2}.

%%%%%%%%%%
% Figure environment removed
%%%%%%%%%%%%%%%%%%%%%%

%\section*{One photon simultaneously excites two atoms}
{\it One photon simultaneously excites two atoms.}---
We indicate with $\ket{\psi_i}$ the eigenstate of the system Hamiltonian $\hat{\mathcal{H}}_\mathrm{tot}$ with eigenenergies $\hbar \omega_{i0}$.
The terms $\omega_{\mathrm{q}i}\hat\sigma_{zi}/2$ in Eq.~(\ref{HRabi}) define the ground $\ket{g}$ and excited $\ket{e}$ atomic bare states.

In Fig.~\ref{Fig3}(a), which is an enlarged view of the red dashed rectangle in Fig.~\ref{Fig2}(b), the white arrow indicates the anticrossing between the eigenstates $\ket{\psi_3}$ and $\ket{\psi_4}$, with eigenfrequencies $\omega_{30}$ and $\omega_{40}$. 
In correspondence of the anticrossing [see Fig.~\ref{Fig3}(b)], we numerically calculate the projection $P_{j}^{(i)}\equiv\abs{\braket{\psi_i}{j}}^2$ of the third and fourth eigenstates $\ket{\psi_i}$ $(i=3,4)$ on the bare states $\ket{j}=\{\ket{gg1},\ket{ee0}\}$ as a function of $\varepsilon_1$. 
Figure~\ref{Fig3}(c) shows that, at the anticrossing, the third and fourth eigenstates are approximate symmetric and asymmetric superpositions of $\ket{gg1}$ and $\ket{ee0}$.
Considering also that the sum of the bare qubit frequencies are nearly equal to the bare resonator frequency, $\omega_\mathrm{q1}+\omega_\mathrm{q2}\simeq\omega_\mathrm{r}$, the anticrossing is the signature of the OPETA effect.
Half of the minimum split between $\omega_{30}$ and $\omega_{40}$ in the spectrum gives the effective coupling between $\ket{gg1}$ and $\ket{ee0}$, that is 22~MHz.

With respect to the theoretical prediction in Ref.~\onlinecite{garziano_one_2016}, our system has a much larger coupling. This implies that the system eigenstates should have a strong dressing, and in principle we could not observe a clean OPETA effect. However, the spin-spin interaction, that is not considered in Ref.~\onlinecite{garziano_one_2016}, reduces the dressing. When $\varepsilon_1$ and $\varepsilon_2$ are both negative, the system states are decoupled with respect to the longitudinal interaction if $M=m_1-m_2=0$.
This occurs when $m_1=m_2=\pm 1$, so when the atoms are both either in the ground $\ket{gg}$ or in the excited $\ket{ee}$ states. However, the transverse interactions still affect our system generating a small dressing that reduces the projection $P_{gg1}^{(4)}$ and $P_{ee0}^{(3)}$ to almost $0.8$ at $\varepsilon_1/2\pi=-2.4$~GHz.



%According to the theoretical prediction in Ref.~\onlinecite{garziano_one_2016}, $g/\omega_\mathrm{r}\simeq0.1$--$0.2$ is the coupling constant that maximizes the expectation values of $\ket{gg1}$ and $\ket{ee0}$ around the anticrossing, whereas the circuit we measured has a much larger coupling constant.
%The theoretical work (Ref.~\onlinecite{garziano_one_2016}) use the ideal Hamiltonian Eq.~\eqref{HRabi} with no spin-spin interaction, and our circuit using superconducting flux qubit cannot ignore the spin-spin interaction.
%In our system with Eq.~\eqref{HRabi_z}, the states $\ket{\psi_{3,4}}$ more correspond to $\ket{gg1}$ and $\ket{ee0}$ with increasing $g/\omega_\mathrm{r}$ (see supplemental material).
%Also, the anticross between $\omega_{30}$ and $\omega_{40}$ (effective coupling between $\ket{gg1}$ and $\ket{ee0}$) has maximum value at around $g/\omega_\mathrm{r}\simeq0.7$, it is really closed to our measured system (see supplemental material~\cite{noauthor_see_nodate}).

% An another theoretical approach to see two qubits excitation in Ref.~\onlinecite{wang_observing_2017}, which consider a symmetric loop flux qubit, mention that the longitudinal coupling term $\sigma_{x1}\sigma_{x2}$ assists exchange the energy between the states $\ket{gg1}$ and $\ket{ee0}$, but our system with single loop flux qubits has only transverse coupling which surpress the effect of qubit-resonator coupling. 




% On the other hand, as examined in the literature (capacities), the fraction of coupling terms in the Hamiltonian of the whole system characterizes the non-classical nature of the system. In the present system, the spin-spin interaction term also accounts for a large fraction of the total Hamiltonian, confirming a phenomenon similar to that predicted in Ref.

%\section*{Discussion}
{\it Discussion.}---We measured the spectrum of a circuit composed of two artificial atoms ultrastrongly coupled to a $LC$ resonator. The generalized Dick Hamiltonian with spin-spin interaction describes well the measured spectrum.
At the energy where the sum of the atomic energies almost matches the one of the resonator, we observed one anticrossing between the states $\ket{gg1}$ and $\ket{ee0}$.
This experimentally confirms the recent theoretical prediction that one-photon can simultaneously excites two atoms ~\cite{garziano_one_2016}, opening a new chapter in quantum nonlinear optics.

Future work would involve reading out the qubit and photon states~\cite{felicetti_parity-dependent_2015} as well as observing the OPETA dynamics. 
These studies could also be extended to explore, for example, photon down- and up-conversion~\cite{PhysRevA.95.063849} and ultrafast two qubit gates~\cite{romero_ultrafast_2012,wang_ultrafast_2017}.


\bibliography{2QUS.bib}
%%%%%%%%%%%%%%%%%%%%%%%%%%%%%%%%%%%%%%%%%%%

\section*{Acknowledgement}
We thank Y. Zhou, R. Wang, and S. Kwon for their thoughtful comments on this research.
This paper was based on results obtained from JSPS KAKENHI (Grant Number JP 22K21294) and a project, JPNP16007, commissioned by the New Energy and Industrial Technology Development Organization (NEDO), Japan.
Supporting from JST CREST (Grant No. JPMJCR1676) and Moonshot R\&D (Grant No. JPMJMS2067) is also appreciated.
F.N. is supported in part by: Nippon Telegraph and Telephone Corporation (NTT) Research, the Japan Science and Technology Agency (JST) [via the Quantum Leap Flagship Program (Q-LEAP), and the Moonshot R\&D Grant Number JPMJMS2061], the Asian Office of Aerospace Research and Development (AOARD) (via Grant No. FA2386-20-1-4069), and the Foundational Questions Institute Fund (FQXi) via Grant No. FQXi-IAF19-06.


\section*{Author contributions}
A.T. designed the device, carried out the experiment and analyzed the data.
A.T. and R.S. performed theoretical and numerical calculations.
H.M. carried out part of the experiment.
All the authors participated in the discussions and wrote and contributed to editing the manuscript.
% All the authors wrote the manuscript, participated in the discussions, and contributed to editing the manuscript.


\ifarXiv
    \foreach \x in {1,...,\numbersupplementpages}
    {
        \clearpage
        \includepdf[pages={\x,{}}]{\supplementfilename}
    }
\fi



\end{document}
