% ****** Start of file apssamp.tex ******
%
%   This file is part of the APS files in the REVTeX 4.2 distribution.
%   Version 4.2a of REVTeX, December 2014
%
%   Copyright (c) 2014 The American Physical Society.
%
%   See the REVTeX 4 README file for restrictions and more information.
%
% TeX'ing this file requires that you have AMS-LaTeX 2.0 installed
% as well as the rest of the prerequisites for REVTeX 4.2
%
% See the REVTeX 4 README file
% It also requires running BibTeX. The commands are as follows:
%
%  1)  latex apssamp.tex
%  2)  bibtex apssamp
%  3)  latex apssamp.tex
%  4)  latex apssamp.tex
%
\documentclass[%
 reprint,
superscriptaddress,
%groupedaddress,
%unsortedaddress,
%runinaddress,
%frontmatterverbose, 
%preprint,
%preprintnumbers,
%nofootinbib,
%nobibnotes,
%bibnotes,
 amsmath,amssymb,
 aps,
%pra,
%prb,
%rmp,
%prstab,
%prstper,
%floatfix,
]{revtex4-2}

\usepackage{graphicx}% Include figure files
\usepackage{dcolumn}% Align table columns on decimal point
\usepackage{multirow}
\usepackage{bm}% bold math
%\usepackage{hyperref}% add hypertext capabilities
%\usepackage[mathlines]{lineno}% Enable numbering of text and display math
%\linenumbers\relax % Commence numbering lines

%\usepackage[showframe,%Uncomment any one of the following lines to test 
%%scale=0.7, marginratio={1:1, 2:3}, ignoreall,% default settings
%%text={7in,10in},centering,
%%margin=1.5in,
%%total={6.5in,8.75in}, top=1.2in, left=0.9in, includefoot,
%%height=10in,a5paper,hmargin={3cm,0.8in},
%]{geometry}

\begin{document}

\preprint{APS/123-QED}

\title{Unusual spin effect in alkali vapor induced by two\\orthogonal multiple harmonics of magnetic field}% Force line breaks with \\
%\thanks{A footnote to the article title}%

\author{E. N. Popov}
 \email{enp-tion@yandex.ru}
 \affiliation{
 Laboratory of Quantum Processes and Measurements, ITMO University,\\199034, 3b Kadetskaya Line, Saint-Petersburg, Russia
}
\author{A. A. Gaidash}
\author{A. V. Kozubov}
\affiliation{
 Laboratory of Quantum Processes and Measurements, ITMO University,\\199034, 3b Kadetskaya Line, Saint-Petersburg, Russia
}
\affiliation{Department of Mathematical Methods for Quantum
  Technologies, Steklov Mathematical Institute of Russian Academy of
  Sciences, 119991, 8 Gubkina St, Moscow, Russia} 

%\collaboration{MUSO Collaboration}%\noaffiliation

\author{S. P. Voskoboynikov}
%\homepage{http://www.Second.institution.edu/~Charlie.Author}
\affiliation{
 Higher School of Software Engineering, Peter the Great St.Petersburg Polytechnic University,\\195251, 29 Polytechnicheskaya, Saint-Petersburg, Russia% with \\
}%



\date{\today}% It is always \today, today,
             %  but any date may be explicitly specified

\begin{abstract}
In this paper, we describe the unusual low-frequency magnetic resonances in alkali vapor with oriented atomic spins regarding the framework of density matrix formalism. The feature of the resonance is the absence of a constant component in the external magnetic field. To explain steep increase of the spin orientation at certain frequencies, we define special closed atomic spin trajectories governed by periodic magnetic perturbation. Any closed trajectory is characterized by the frequency of spin motion. The resonance effect was numerically verified in the paper. For instance, these trajectories can be observed in an alkali vapor via optical excitation. Surprisingly, the width of the resonance line is found to be narrower, as one may expect. 
%\begin{description}
%\item[Usage]
%Secondary publications and information retrieval purposes.
%\item[Structure]
%You may use the \texttt{description} environment to structure your abstract;
%use the optional argument of the \verb+\item+ command to give the category of each item. 
%\end{description}
\end{abstract}

%\keywords{Suggested keywords}%Use showkeys class option if keyword
                              %display desired
\maketitle

%\tableofcontents

\section{\label{sec:level1}INTRODUCTION}

The phenomenon of the electron spin resonance (ESR) was firstly observed in Kazan State University by Evgenii Zavoisky in 1944 \cite{zavoisky}. Since that, ESR was widely studied, and now it is one of the most important methods of spectroscopy and metrology \cite{epr:book1,epr:book2}. Here we discuss ESR in a gas cell with polarized vapor of alkali, which can be described as a hot atomic ensemble. Since the mean path of alkali atoms without depolarization is long enough, the linewidth of spin and optical resonances in the vapor are narrow in compare to liquid medium or crystals. Magnetization in a gas cell is induced by optical excitation of alkali vapor \cite{orientation,Kastler:63,nobel}. Alkali spins can be oriented and then detected by the light, which excites an atomic resonance transition. Based on the optical scheme of excitation, ESR in a sodium vapor was proposed in 1957 by Dehmelt \cite{first:res} and later was experimentally performed by Bell and Bloom \cite{bell:bloom}. Regarding vapor of alkali, the optical excitation was implemented by spectral lamps in early ESR experiments. Later, the spectral lamps were superseded by lasers as a more efficient and compact sources \cite{laser:spectroscopy}.

Explicit distinction of observed magnetic resonance in a gas cell have provided numerous advantages for practical applications. Furthermore, extensive development of laser technologies \cite{happer:article,happer:book,kubo,aleksandrov,happer:edu,suter,skolnik,kozlov,talker} have led to the progress in the optical control of an atomic state in a gas cell. An essential aspect of the study is the depolarization induced by collisions of alkali atoms in the gas cell. The effect determines the form of the magnetic resonance curve. The cornerstone of the depolarization is the population mixing among the Zeeman sub-levels \cite{cesium:transitions,sodium:relaxation,rubidium:relaxation,rubidium:kr,bouchiat,quad:relaxation,franz:franz,coating:relaxation,thesis:breault}. Recent progress in coating technology allows increasing life-time of atomic spin orientation in a gas cell up to several minutes \cite{minute:decay,balabas:10,advances:decay}. Slow relaxation (in order of minutes) is a reason of extreme ESR line narrowing, that allows utilizing a gas cell as a sensitive element for precision magnetometers. Study of a magnetic resonance is topical, especially for the magnetometry \cite{pomerantsev,farr,vershovsky,nature:magnetometer,bison,budker:kimball,alexandrov:vershovsky,zhivun,konrack}. It should be noted, that the resonance scheme can also be applied to nuclear spins. A spin orientation can be transferred from the alkali electrons to the noble gas nuclei in the cell via long collisions. This effect is known as \textquotedblleft spin-exchange\textquotedblright \cite{theory:spexch,experiment:spexch,general:spexch,estimate:spexch,review:spexch,full:spexch,nature:spexch}.

In the paper, we theoretically describe an unusual spin effect occurred in alkali vapor with applied magnetic field without a constant component. Mathematically, the temporal mean of the magnetic field vector is equal to zero:
\begin{equation}\label{basic:mag}
\mathbf{B}(t+T)=\mathbf{B}(t)
\end{equation}
\begin{equation}\label{basic:absent}
\frac{\Omega}{2\pi}\int\limits_{T}\mathbf{B}(t)\mathrm{d}t=0,\qquad\Omega=\frac{2\pi}{T},
\end{equation}
\noindent where $T$ is a period and $\Omega$ is a repetition frequency. The alternating magnetic field $\mathbf{B}(t)$ governs the dynamics of alkali spins, producing periodic perturbation.

The observed phenomenon entails a strong response of alkali spins to a specific temporal profile of the alternating magnetic field (\ref{basic:mag}). Moreover, the spin orientation steeply increases when frequency $\Omega$ is equal to a certain peculiar value. The latter property is the most interesting one, since the general precession of alkali spins with own Larmor frequency is absent. We call the effect {\it a resonance}, due to the existence of some peaks in the frequency dependence. However, it is not a classical ESR case, where the resonance frequency is defined by the magnitude of DC-field. Instead, every resonance frequency is defined by the amplitude of an alternating magnetic field and its periodic temporal profile.

Alkali spins in the presence of external field (\ref{basic:mag}) can be defined as a linear dynamic system with unpredictable behavior, which can not be described via simple analytical laws. Withal, numerical calculations allow finding the resonance frequencies, but do not explain its nature. Mathematically, we could describe the effect as a parametric resonance \cite{parres}. However, there is a problem: inner system's parameters, that determine the own frequency, are absent. Moreover, the resonance depends on a temporal profile of the periodic perturbation even for a fixed frequency of repetition. Therefore, we consider the  resonance as a topological effect and explain it by the closed trajectories in the Phase Space of spin orientation components.

Similar resonances in vapor of alkali are considered in a very few works. The closest research is an experiment performed in \cite{sensor}. The authors demonstrate a magnetometer, where external perturbation is formed by three orthogonal radio-fields with different frequencies. Indeed, the parametric resonance was induced in the experiment. In comparison to the quasi-adiabatic classical theory provided in \cite{sensor}, we present a quantum theory of the oriented spins dynamics. Furthermore, we provide the detailed explanation of the resonance effect. According to the proposed approach, we have found some curious narrow resonance lines in frequency dependence of the spin orientation.

\section{OPTICAL SCHEME FOR EXCITATION AND DETECTION OF THE ORIENTED SPINS}

In the section, we suggest an optical scheme for observation of the unusual spin resonance, which is shown in the figure~\ref{picture_scheme}. A gas cell is filled with a vapor of non-zero spin atoms, which are sensitive to an external field. Inside the cell, spins are collectively oriented and scanned by two lasers. The resonance dynamics of the oriented spins is observed under an alternating homogeneous magnetic field, which is produced by Helmholtz coils around the cell. Three key parts regarding the optical scheme are discussed below.

% Figure environment removed

\subsection{Gas cell}

A gas cell contains a vapor of $\mathrm{^{87}Rb}$ and a mix of inert gases. A gas mix composes of diatomic nitrogen in order to reduce alkali fluorescence and some noble monatomic gases as a buffer. Further, we call the vapor of $\mathrm{^{87}Rb}$ just {\it alkali} and the mix of inert gases just {\it buffer} for the sake of simplicity. It is important that non-excited alkali atoms can populate two different hyperfine levels. Therefore, to define an alkali atom, which is at the ground hyperfine level with total angular momentum $F$, we use a short term: {\it $L_F$-atom}.

Temperature within the cell is about $80^\circ\,\mathrm{C}$. Under the conditions, the concentration of the alkali is $5–6$ orders of magnitude less than the concentration of the buffer. Since an electron cloud of alkali has an enlarged radius and anisotropic form, multiple collisions between an excited atom $\mathrm{^{87}Rb}$ and molecules $\mathrm{N}_2$ lead to rapid non-radiative transitions from the upper to one of the two ground levels in the $\mathrm{d1}$-line (Fig.~\ref{picture_scheme}). Moreover, since buffer environment freezes free motion of alkali, a surface of the cell influences weakly enough for essential increase of spin orientation life-time. The latter properties of the gas cell are the reason for the principal feasibility of a complicated collective spin motion in an external magnetic field.

\subsection{Pumping of alkali spin orientation}

A circularly polarized light induces a spin orientation, which is directed along the path of the light propagation. As shown in the figure~\ref{picture_scheme}, the {\it pump light} propagates along the Z-axis and causes the resonance transition between the levels with a total angular momentum $F=1$ and $F'=2$. Full data about the $\mathrm{d-1}$ line of $\mathrm{^{87}Rb}$ can be found in \cite{steck}. In presence of optical excitation, alkali atoms are pumped and populate the upper levels with one-sided change of the angular momentum, which is projected along the path of the light propagation. Therefore, alkali atoms accumulate non-zero angular momentum directed along the Z-axis.

In the optical scheme, orientations of $\mathrm{L_1}$- and $\mathrm{L_2}$-atoms' spins are achieved through {\it depopulation} and {\it repopulation}, respectively. Both processes create the non-equilibrium population among Zeeman sub-levels belonging to the ground levels of the $\mathrm{d-1}$ line. The nature of depopulation is a selective depletion of certain sub-levels by a circularly polarized light. It should be noted, that repopulation process is more efficient than depopulation. As presented in \cite{scheme:re}, repopulation is based on the feature of collisional decay, that is a nucleus spin state is not destroyed during transition of an excited alkali atom from upper to a ground level. Preserved nucleus spin produces a spin orientation of $\mathrm{L_2}$-atoms.

\subsection{Scanning of alkali spin orientation}

An atomic ensemble with oriented spins is circular birefringent due to induced optical anisotropy. The magnitude of the circular birefringence is determined by a spin orientation projection along the path of light propagation. Since refraction indices for the orthogonal circular components of a passing light are different, the linear polarization plane rotates around the path of light propagation. To observe the latter, a detection scheme with a polarization beam splitter can be implemented (see PBS in the Fig.~\ref{picture_scheme}).

Mathematically, the angle of the optical rotation is proportional to the expectation value of the full angular momentum:
\begin{equation}\label{angular}
\Delta\psi(t)\propto\langle\hat{F}_x\rangle_{Rb},
\end{equation}
\noindent where $\Delta\psi(t)$ is the measurable angular rotation of the polarization plane, $\hat{F}_x$ is an X-component of the total angular momentum operator. The X-axis is directed along the {\it scanning light} in the Figure~\ref{picture_scheme}. Therefore, by selecting the scanning light direction, we can measure any spin orientation component of the alkali vapor in the gas cell.

The spectral linewidth of the scanning laser should be narrow enough to provide the hyperfine structure resolution. The condition above is necessary for selective measurement of the spin orientation, which is produced by $\mathrm{L_1}$- or $\mathrm{L_2}$-atoms. As shown in the Figure~\ref{picture_scheme}, the scanning light frequency  should be close to a transition between hyperfine levels with total angular momentum $F=2$ and $F'=2$. In the case, $\mathrm{L_2}$-atoms affect the polarization of the scanning light much stronger than $\mathrm{L_1}$-atoms. Withal, laser frequency should be detuned from the optical resonance for excluding a redundant depletion of ground level $F=2$. Since the $F=1$ level is broadened by a pump light, the spin orientation scanning of $\mathrm{L_2}$-atoms is more effective than of $\mathrm{L_1}$-atoms.

\section{MAGNETIC RESONANCE AND CLOSED SPIN LOOPS}

In the following section, we discuss the nature of observed resonance phenomena. To excite a magnetic resonance in the gas cell, an ensemble of alkali spins is perturbed by two harmonic magnetic fields with different frequencies:
\begin{equation}\label{mag:field}
\mathbf{B}(t)=B_0 \mathbf{l_z}\sin\left( a\Omega t\right)-B_0 \mathbf{l_x} \cos\left(b\Omega t\right),
\end{equation}
\noindent where $B_0$ is the amplitude of a magnetic field and $\Omega$ is the repetition frequency, $a$ and $b$ are arbitrary coefficients in a general case. In the paper we consider $a=1$ and $b=2$, generalization is out of the scope. To avoid a spin dephasing, which occurs due to alkali diffusion, a magnetic field should be homogeneous. For instance, a magnetic field can be generated by Helmholtz coils, see the Fig.~\ref{picture_scheme}. The geomagnetic field can be removed by a compensation current or a magnetic shield.

According to the equation (\ref{mag:field}), the constant component and the temporal mean of the magnetic field are equal to zero. Therefore, alkali spins are not subject to a continuous precession with the non-zero averaged angular velocity. %Instead, the spin orientation move along a complicate trajectory in the Phase State.

In our work, we utilize two orthogonal components of an external magnetic field instead of three as in \cite{sensor}. Another crucial difference from \cite{sensor} is that we consider non-adiabatic dynamics. To achieve the latter, the following inequalities should be satisfied:
\begin{equation}\label{mag:cond}
\gamma B_0 \gg \tau^{-1},\qquad \tau \gg \Omega^{-1},
\end{equation}
where $\gamma$ is the gyromagnetic ratio and $\tau$ is the life-time of the spin orientation. The feature of the non-adiabatic case are frequent flips of the alkali spin orientation due to a strong magnetic field. Since life-time $\tau$ is longer than perturbation period, alkali spins are able to repeat a similar periodic motion until the total dephasing. The latter property of the non-adiabatic case is beneficial; it allows observing a closed trajectory of the spin motion in an alkali vapor and propose an explanation for the existence of the resonance.

\subsection{Spin loops}

% Figure environment removed

To describe the spin dynamics in the simplest form, we consider the Pauli equation with magnetic field defined by the formula (\ref{mag:field}):
\begin{equation}\label{spinor:dyn}
i\hbar\dot{\varphi}=\gamma \left(\hat{\sigma}\cdot\mathbf{B}(t)\right)\varphi,
\end{equation}
\begin{equation}
\hat{\sigma}=\frac{\hbar}{2}\left(\hat{\sigma}_x \mathbf{x}+\hat{\sigma}_y \mathbf{y}+\hat{\sigma}_z \mathbf{z}\right),
\end{equation}
\noindent where $\varphi$ is the spinor, $\gamma$ is the gyromagnetic ratio of a system, $\hat{\sigma}_\alpha$ are Pauli matrices.

The solution of the equation (\ref{spinor:dyn}) is a trajectory in Hilbert space of spinor states. We investigate only three components $S_\alpha$ derived from the solution, which can be measured by the scanning light as described above:
\begin{equation}\label{s:classcomp}
S_\alpha(t)=\left(\varphi^\dagger \hat{\sigma}_\alpha\varphi\right),\qquad \alpha\in\left\{x,y,z\right\},
\end{equation}
\begin{equation}\label{s:classvec}
\mathbf{S}(t)=S_x(t)\,\mathbf{x}+S_y(t)\,\mathbf{y}+S_z(t)\,\mathbf{z}.
\end{equation}
\noindent The vector $\mathbf{S}(t)$ describes the spin orientation of a dynamical system. Moreover, a point with coordinates $S_\alpha(t)$ draws the path on 3D sphere, we call it {\it spin orientation trajectory}.

In general, spin orientation trajectories are not closed, i.e. the following condition is not satisfied:
\begin{equation}\label{closed:tr}
\mathbf{S}_{loop}(t+T)=\mathbf{S}_{loop}(t),\qquad T=\frac{2\pi}{\Omega}.
\end{equation}
However, it turns out that certain initial parameters lead to unique closed ones. Here and further, we call them {\it spin loops}. The defined spin loops can be self-intersected.

It should be noted, the Pauli equation (\ref{spinor:dyn}) does not take into account a relaxation and an excitation. If we introduce these processes here, the equilibrium state would be set to Z-axis close to the origin in the phase space of spin orientation components. Therefore, spin orientation trajectory would be relocated from the sphere to the inner area.

\subsection{Resonance hypothesis}

We suppose that alkali spins are oriented as efficiently as possible, if the spin orientation trajectory is equivalent to a spin loop and the temporal mean of $\mathbf{S}$ is directed along the path of pump light propagation as follows:
\begin{equation}\label{loop:mean}
\mathcal{S}=\frac{1}{T}\oint\limits_{T}\mathbf{S}_{loop}(t)\mathrm{d}t, \qquad \mathcal{S}\upuparrows \mathbf{z}.
\end{equation}
\noindent Since the latter condition is satisfied under certain discrete frequencies $\Omega$ of the alternating magnetic field (\ref{mag:field}), we call the effect {\it the resonance}. The hypothesis is based on the idea that a circularly polarized light can produce an alkali quantum state with an angular momentum directed only along the path of the light propagation. Furthermore, to hold the expectation value of the angular momentum enough to overcome dephasing, alkali spins should motion along the closed trajectory (\ref{closed:tr}). Exploring the condition (\ref{loop:mean}) we found some spin loops. However, since adiabatic condition is satisfied, the most of the spin loops are not relevant. Therefore, we consider only four spin loops (see Table~\ref{table1} for the details), which arise under the highest frequency $\Omega$.

\begin{table}[h]
\caption{\centering \it \small Four spin loops at the highest frequencies and with temporal mean directed along Z-axis.}\label{table1}
\begin{ruledtabular}
\begin{tabular}{cccc}
		Name of&
		Initial spinor& 
        \multirow{2}{*}{$\Omega/\gamma B_0$} & 
		\multirow{2}{*}{$\left(\mathcal{S}\cdot \mathbf{z}\right)$}
				\\
        resonance&
        $\left(\varphi_+,\varphi_-\right)$& &\\
				\colrule
		& & &
				\\
				Ball & $\left(1,0\right)$ & $0.099$ & $0.125$ \\
				Flower & $\left(0,1\right)$ & $0.126$ & $0.162$ \\
			  Ring & $\left(0,1\right)$ & $0.175$ & $0.394$ \\
			  Knot & $\left(0,1\right)$ & $0.259$ & $0.117$ \\
            & & &
\end{tabular}
\end{ruledtabular}
\end{table}

Despite simplicity, the Pauli equation allows claiming a correspondence between some spin loops and resonance peaks, which are obtained further in the Section~\ref{sec:calc and disc}. The spin loops from the Table~\ref{table1} are drawn in the Figure~\ref{pic2}, where the spin orientation of alkali moves due to periodic perturbation by the field (\ref{mag:field}).

\section{QUANTUM MODEL OF ALKALI DYNAMICS}

To verify the hypothesis about interconnection between defined spin loops and a magnetic resonance, we propose a quantum model for the optical scheme in the Fig.~\ref{picture_scheme}. The dynamics of the alkali spin orientation can be predicted with any frequency $\Omega$ of the magnetic field (\ref{mag:field}). The model takes into account several processes, which are listed in the Table~\ref{table2}.

\begin{table}[h]
\caption{\centering \it \small Processes in the gas cell and their origin. An inverse life-time or a process rate is denoted by symbol $\mathcal{V}$.}\label{table2}
\begin{ruledtabular}
\begin{tabular}{ccc}
         № & 
         $\mathcal{V},\ c^{-1}$ &
		Process / Origin \\ 
        \colrule
        \multirow{3}{*}{I} &
        \multirow{3}{*}{$10^9-10^{10}$} &
        Decoherence of electric dipole \\
        & & oscillations on optical transitions /\\
        & & alkali-buffer elastic collisions\\
        \colrule
        \multirow{3}{*}{II} &
        \multirow{3}{*}{$10^8-10^9$} &
        Decay without fluorescence / \\
        & & non-elastic collisions between\\
        & & excited alkali and buffer atoms\\
        \colrule
        \multirow{2}{*}{III} &
        \multirow{2}{*}{$10^8-10^9$} &
        Inhomogeneous broadening\\
        & & of d-1 line / Doppler effect \\
        \colrule
        \multirow{3}{*}{IV} &
        \multirow{3}{*}{$10^4-10^5$} &
        Excitation of alkali atoms / \\
        & & absorption of circularly polarized\\
        & & pump light (laser)\\
        \colrule
        \multirow{3}{*}{V} &
        \multirow{3}{*}{$10^4-10^5$} &
        Motion of alkali spins / \\
        & & fast precession under external\\
        & & alternating magnetic field\\
        \colrule
        \multirow{3}{*}{VI} &
        \multirow{3}{*}{$10^2-10^3$} &
        Mixing of population of ground \\
        & & hyperfine levels / spin exchange \\
        & & and collision with walls of the cell \\
\end{tabular}
\end{ruledtabular}
\end{table}

Rates of the relaxation processes (I, II, III, VI) are estimated by concentration of the buffer, temperature, size, and coating of the gas cell. Magnitudes of the perturbation (IV and V) are determined by an amplitude of the pump light and an amplitude of the external magnetic field. An inverse life-time determines a number of atomic state changes per second for each described process.

The ensemble of alkali atoms is described by a density matrix, which rank is equal to the number of Zeeman sub-levels. There are $32$ sub-levels in $d1$-line of $\mathrm{^{87}Rb}$. Then the master-equation for density matrix $\hat{\rho}$ is as follows:

\begin{equation}\label{master:equation}
\begin{array}{l}
\displaystyle i\hbar\frac{\mathrm{d}\hat{\rho}}{\mathrm{d}t}=\left[\hat{H},\hat{\rho}\right]-
\mathcal{V}_{dcy}\left(\hat{\rho}-\mathcal{D}\left\{\hat{\rho}\right\}\right) \vspace{0.2 cm} - \\ \displaystyle -\mathcal{V}_{dec}\left(\hat{\rho}-\mathcal{R}\left\{\hat{\rho}\right\}\right) - \mathcal{V}_{mix}\left(\hat{\rho}-\hat{\rho}_0\right),
\end{array}
\end{equation}

\noindent where $\mathcal{D}$ and $\mathcal{R}$ are two superoperators of relaxation. The former describes the process I from the Table~\ref{table2}; it makes non-diagonal elements with an optical frequency of phase rotation equal to zero. The latter describes the process II from the Table~\ref{table2}; it decomposes the full density matrix $\hat{\rho}$ to a tensor product of an electron and nuclear density matrices. Notice, the electron density matrix is reduced to the equilibrium ground state, and spin orientation of the nuclear density matrix is preserved \cite{franz:franz, scheme:re}. The last term in expression (\ref{master:equation}) describes the process VI from the Table~\ref{table2}. A matrix $\hat{\rho}_0$ corresponds to the state of thermodynamic equilibrium with mixed population among the Zeeman sub-levels. The symbol $\mathcal{V}_{dec}$ denotes the frequency of elastic collisions between an alkali atom and buffer atoms, $\mathcal{V}_{dcy}$ is the frequency of strong collisions between an excited alkali atom and buffer atoms results to decay to ground levels, $\mathcal{V}_{mix}$ is the relaxation rate of an alkali spin orientation.

The operator $\hat{H}$ comprises the unperturbed Hamiltonian $\hat{H}_0$ and 
an operator of interaction $\hat{V}$:

\begin{equation}
\hat{H}=\hat{H}_0+\hat{V}, \qquad \hat{V}=\hat{V}_E+\hat{V}_B,
\end{equation}
\begin{equation}\label{V:eq}
\hat{V}_E=-\left(\mathbf{\hat{d}\cdot E}\right), \qquad 
\hat{V}_B=\sum\limits_{n=1}^2 g_n\gamma_e \left(\mathbf{\hat{F}_n\cdot B}\right),
\end{equation}
\begin{equation}
\mathbf{E}=\frac{\mathcal{E}}{2}\,\mathbf{l}_+ e^{i(kz-\omega t)}+c.c.,\qquad k=\frac{\omega}{c},
\end{equation}

\noindent where $\mathcal{E}$ is the constant amplitude of the pump light, $\mathbf{l}_+$ is the unit vector of circular polarization, $\omega$ is the frequency of the pump light, the dipole operator $\mathbf{\hat{d}}$ describes all optical transitions between Zeeman sub-levels in $d1$-line of $\mathrm{^{87}Rb}$ \cite{steck}, $\gamma_e$ is the electron gyromagnetic ratio, $g_n$ is the g-factor of the ground hyperfine levels of $\mathrm{^{87}Rb}$, $\mathbf{\hat{F}}_1$ and $\mathbf{\hat{F}}_2$ are total angular momentum operators of the $\mathrm{L_1}$-atom and $\mathrm{L_2}$-atom respectively, magnetic field $\mathbf{B}$ is defined by (\ref{mag:field}). Interactions $\hat{V}_E$ and $\hat{V}_B$ describe the process IV and the process V from the Table~\ref{table2} correspondingly.

\section{CALCULATION AND DISCUSSION}\label{sec:calc and disc}

% Figure environment removed

Now we can implement the quantum model and study a behavior of an alkali vapor after transition to the steady dynamics. According to the resonance hypothesis, we should discover the strongest dissimilarity from the equilibrium state, when varied frequency $\Omega$ of the magnetic field (\ref{mag:field}) approaches the values from the Table~\ref{table1}.

Mathematically, the formula (\ref{master:equation}) is a linear system of non-homogeneous differential equations with variable coefficients. To observe the resonance, let us consider a spin orientation dependence on the frequency $\Omega$. Since spin motion repeats in time, the results should be obtained from the last period of the steady dynamics.

Under optical pumping, alkali atoms populate the ground level $F=1$ and the ground level $F=2$. As noted in the first section, we scan a spin orientation from $\mathrm{L_2}$-atoms. In the quantum model, this data is contained in a cropped density matrix related to a subspace of $\mathrm{L_2}$-atoms with rank $2F+1$ defined by the number of Zeeman sub-levels:
\begin{equation}
\hat{\rho}_{[2]}=\hat{P}_2\hat{\rho}\hat{P}_2,
\end{equation}
\noindent where $\hat{P}_2$ is the operator of projection to the ground hyperfine level with a total angular momentum $F=2$.  Diagonal and non-diagonal elements describe the population of $\mathrm{L_2}$-atoms' Zeeman sub-levels and low-frequency coherence between them, respectively.

We define components of the spin orientation, which are parallel and orthogonal to the path of pump light propagation, as {\it longitudinal} and {\it transverse} respectively. A longitudinal component is associated with inhomogeneous distribution of the diagonal elements. The magnitude and the direction of the transverse component is determined by the absolute values and phases of a complex non-diagonal elements of $\hat{\rho}_n$. Similar to the classical approach as in (\ref{spinor:dyn}--\ref{s:classvec}), the spin orientation of $\mathrm{F_2}$-atoms can be described by the following vector in the quantum model:

\begin{equation}\label{spin:comp}
S_{[2],\alpha}=\mathrm{Tr}\left\{\,\hat{\rho}_{[2]}\hat\Sigma_{\alpha} \,\right\},\qquad\alpha\in\{x,y,z\},
\end{equation}
\begin{equation}
\mathbf{S_{[2]}} =
S_{[2],x}\,\mathbf{x}+
S_{[2],y}\,\mathbf{y}+
S_{[2],z}\,\mathbf{z},
\end{equation}
\noindent where $\hat\Sigma_{\alpha}$ is the Pauli matrices, equivalent for a Spin-2 particle. It is important, that an external alternating magnetic field totally determines the dynamics of the vectors $\mathbf{S_{[2]}}$ as well as a behavior of gyro precession.

Unlike the formulas (\ref{s:classcomp}) and (\ref{s:classvec}), spin orientation behavior related to the vector $\mathbf{S_{[2]}}$ has a qualitative distinction: it moves along a closed trajectory at any frequency $\Omega$ due to the existence of the steady dynamics determined by equilibrium state.% These closed trajectories have a complicated globular form.

As the next step, we define two frequency dependencies: the first is the range of the transverse component, the second is the temporal mean of the longitudinal component:
\begin{equation}\label{c1}
C_{1}(\Omega) = \mathrm{Range}\left[S_{[2],x}(t)\right],
\end{equation}
\begin{equation}\label{c2}
C_{2}(\Omega) = \frac{1}{T}\int\limits_{T}S_{[2],z}(t)\,\mathrm{d}t.
\end{equation}

\noindent The convolution $C_1$ describes the diameter of the globular trajectory in the phase space of spin orientation components, and the convolutions $C_2$ describes the $Z$-offset of the one. Notice: $C_1$ and $C_2$ are measurable with the scanning light. Moreover, $C_2$ can be measured by absorption of the circularly polarized pump light.

According to the hypothesis about interconnection between the spin loops' existence and the emergence of the resonance, one may observe several peaks in the Figure~\ref{pic3}. From the numerical estimations, two observations follow immediately:
\begin{itemize}
\item
The ratios $\Omega_{res}/\left(g_{2}\gamma_e B_0\right)$ are very close to the corresponding values from the Table~\ref{table1}, where $\Omega_{res}$ is the local maximum frequencies of the curves.
\item
The peak magnitudes' dependency on the parameter $\left(\mathcal{S}\cdot 
\mathbf{z}\right)$ from the Table~\ref{table2} is monotonic, but not linear. Also, the highest peak and the maximum value of $\left(\mathcal{S}\cdot \mathbf{z}\right)$ relate to the same resonance frequency.
\end{itemize}

The results confirm the hypothesis that the resonance is explained by the existence of spin loops (\ref{closed:tr}). When conditions of the resonance are met, the trajectory of $\mathbf{S_{[2]}}$ in the phase space converges to one of the spin loops from the Figure~\ref{pic2}. If a frequency $\Omega$ is not equal to resonance frequency $\Omega_{res}$, the closed trajectories of $\mathbf{S_{[2]}}$ look as an asymmetric tangle with low diameter. However, their shapes resemble neighboring spin loops.

At the end of the day, we present a profitable property of the \textquotedblleft ring\textquotedblright resonance:
\begin{equation}\label{inequality}
\mathrm{W} \approx 0.5\ kHz \quad < \quad 
\mathcal{V}_{mix}=1\ kHz,
\end{equation}
\noindent where $\mathrm{W}$ is the estimated full width at half maximum of the convolution $C_2$, and $\mathcal{V}_{mix}$ is the rate of spin relaxation introduced in the quantum model (\ref{master:equation}). However, full width at half maximum for a classical ESR in alkali vapor must be about four times the relaxation rate. It is curious that the studied resonance is narrower than classical ESR. We consider it as a potential advantage for applications based on the observed spin effect. We hypothesized, the inequality (\ref{inequality}) may be explained by the topological nature of the resonance, which is not based on matching between frequency of an alternating magnetic field and Larmor frequency.

\section*{Funding}
This work was financially supported by Russian Ministry of Education (Grant No. 2019-0903)

\begin{acknowledgments}
We thank our colleagues A. Kiselev from Laboratory of Quantum Processes and Measurements ITMO, and G. Miroshnichenko from Institute "High School of Engineering" ITMO for fruitful discussions during the research.
\end{acknowledgments}


\nocite{*}

\bibliography{apssamp}% Produces the bibliography via BibTeX.

\end{document}
%
% ****** End of file apssamp.tex ******
