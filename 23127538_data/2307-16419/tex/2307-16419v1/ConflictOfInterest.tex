\documentclass[journal, a4paper, draftclsnofoot, onecolumn, twoside, 12pt]{article}
\usepackage{amssymb}
\usepackage{amsmath}
\usepackage[left=2.8cm,right=2.8cm,top=2.1cm,bottom=2.1cm]{geometry}
\usepackage{graphicx}
\usepackage{pst-grad}
\usepackage{pst-plot}
\usepackage{pst-text}
\usepackage{subfigure}
\usepackage{color}
\usepackage{indentfirst}
\usepackage{ulem}
\usepackage{subeqnarray}
\usepackage{cases}
\usepackage{float}
\usepackage{algorithm}
\usepackage{algorithmicx}
\usepackage{algpseudocode}
\usepackage{graphicx}
\usepackage{epstopdf}
\usepackage{amsmath}
\usepackage{CJK}
\usepackage{amsmath}
\usepackage{multirow}

\usepackage{xcolor,colortbl}
\definecolor{Gray}{gray}{0.9}

\DeclareMathOperator*{\argmax}{argmax}
\DeclareMathOperator*{\argmin}{argmin}
\renewcommand{\algorithmicrequire}{\textbf{Input:}}  %
\renewcommand{\algorithmicensure}{\textbf{Output:}} %
\newtheorem{definition}{Definition}
\newtheorem{theorem}{Theorem}
\newtheorem{lemma}{Lemma}
\newtheorem{proof}{Proof}
\newtheorem{proposition}{Proposition}
\newtheorem{collary}{Collary}
\newtheorem{example}{Example}
\newtheorem{problem}{Problem}
\newtheorem{assumption}{Assumption}
\newtheorem{property}{Property}
\newtheorem{remark}{Remark}


\def\eg{\emph{e.g}\onedot} 
\def\Eg{\emph{E.g}\onedot}
\def\ie{\emph{i.e}\onedot} 
\def\Ie{\emph{I.e}\onedot}
\def\cf{\emph{c.f}\onedot} 
\def\Cf{\emph{C.f}\onedot}
\def\etc{\emph{etc}\onedot} 
\def\vs{\emph{vs}\onedot}
\def\wrt{w.r.t\onedot} 
\def\dof{d.o.f\onedot}
\def\etal{\emph{et al}\onedot}
\makeatother

\def\Vec#1{{\boldsymbol{#1}}}
\def\Mat#1{{\boldsymbol{#1}}}
\newcommand{\st}{{\rm s.t.}\xspace}
\def\SPD#1{\mathcal{S}_{++}^{#1}}
\def\SYM#1{\operatorname{Sym}({#1})}
\def\GRASS#1#2{\mathcal{G}({#2},{#1})}
\def\ST#1#2{\mathrm{St}({#2},{#1})}
\newcommand{\DIAG}{\mbox{Diag\@\xspace}}


\def\TODO#1{{\color{red}{\bf [TODO:} {\it{#1}}{\bf ]}}}
\def\NOTE#1{{\bf [NOTE:} {\it\color{blue}{#1}}{\bf ]}.}
\def\CHK#1{{\bf [CHECK:} {\it\color{red} {#1}}{\bf ]}.}
\def\REFINE#1{{\color{violet}{\bf [REFINE:} {\it{#1}}{\bf ]}}}


\def\PF#1{{\color{magenta}{\bf [Pengfei:} {\it{#1}}{\bf ]}}}
\def\MH#1{{\color{purple}{\bf [Mehrtash:} {\it{#1}}{\bf ]}}}
\def\LP#1{{\color{blue}{\bf [Lars:} {\it{#1}}{\bf ]}}}
\def\SR#1{{\color{blue}{\bf [SR:} {\it{#1}}{\bf ]}}}

\title{\textbf{Conflict-of-interest Statement}}
\begin{document}

\maketitle


\noindent 
\textbf{Manuscript Title:} Subspace Distillation for Continual Learning\\

\noindent The authors whose names are listed immediately below certify that they have NO affiliations with or involvement in any
organization or entity with any financial interest (such as honoraria; educational grants; participation in speakers’ bureaus;
membership, employment, consultancies, stock ownership, or other equity interest; and expert testimony or patent-licensing
arrangements), or non- financial interest (such as personal or professional relationships, affiliations, knowledge or beliefs) in
the subject matter or materials discussed in this manuscript.
\\
\begin{table}[h]
\centering
\begin{tabular}{llllllll}
\hline
\multicolumn{1}{l|}{Author Name} & \multicolumn{6}{l|}{Signature} & Date \\ \hline
    Kaushik Roy         &  \multicolumn{6}{l}{}         &  May 30, 2022\\
    Christian Simon     &  \multicolumn{6}{l}{}         &  May 30, 2022\\
    Peyman Moghadam     &  \multicolumn{6}{l}{}         &  May 30, 2022\\
    Mehrtash Harandi    &  \multicolumn{6}{l}{}         &  May 30, 2022\\
\end{tabular}
\end{table}
 
\end{document}
