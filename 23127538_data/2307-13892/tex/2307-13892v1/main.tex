\documentclass{article}


% if you need to pass options to natbib, use, e.g.:
\PassOptionsToPackage{numbers, compress}{natbib}
% before loading neurips_2023


% ready for submission
\usepackage[final]{neurips_2023}


% to compile a preprint version, e.g., for submission to arXiv, add add the
% [preprint] option:
%     \usepackage[preprint]{neurips_2023}


% to compile a camera-ready version, add the [final] option, e.g.:
%     \usepackage[final]{neurips_2023}


% to avoid loading the natbib package, add option nonatbib:
%\usepackage[nonatbib]{neurips_2023}

\usepackage[utf8]{inputenc} % allow utf-8 input
\usepackage[T1]{fontenc}    % use 8-bit T1 fonts
\usepackage{hyperref}       % hyperlinks
\usepackage{url}            % simple URL typesetting
\usepackage{booktabs}       % professional-quality tables
\usepackage{amsfonts}       % blackboard math symbols
\usepackage{nicefrac}       % compact symbols for 1/2, etc.
\usepackage{microtype}      % microtypography
\usepackage{xcolor}         % colors
\usepackage{graphicx}
\usepackage{textcomp}
\usepackage{subcaption}

% add algorithm reference to autoref package
\renewcommand{\sectionautorefname}{Section}

% let autoref use section instead of (sub)subsection
\let\subsectionautorefname\sectionautorefname{}
\let\subsubsectionautorefname\sectionautorefname{}


\title{AI4GCC - Team: Below Sea Level \\ Score and Real World Relevance}

% \author{Bram Renting} 
% \author{Phillip Wozny*}

% \author{Bram Renting*, Phillip Wozny*,\\Robert Loftin, Claudia Wieners, Erman Acar \\ (Team Name: Below Sea Level)}


% The \author macro works with any number of authors. There are two commands
% used to separate the names and addresses of multiple authors: \And and \AND.
%
% Using \And between authors leaves it to LaTeX to determine where to break the
% lines. Using \AND forces a line break at that point. So, if LaTeX puts 3 of 4
% authors names on the first line, and the last on the second line, try using
% \AND instead of \And before the third author name.


\author{%
Phillip Wozny*
\\
Tilburg University \\ Vrije Universiteit Amsterdam 
\And
Bram Renting*
\\
Leiden University \\ Delft University of Technology
\AND 
Robert Loftin \\
Delft University of Technology
\And 
Claudia Wieners\\
Utrecht University
\And
Erman Acar \\
University of Amsterdam 
  % examples of more authors
  % \And
  % Coauthor \\
  % Affiliation \\
  % Address \\
  % \texttt{email} \\
  % \AND
  % Coauthor \\
  % Affiliation \\
  % Address \\
  % \texttt{email} \\
  % \And
  % Coauthor \\
  % Affiliation \\
  % Address \\
  % \texttt{email} \\
  % \And
  % Coauthor \\
  % Affiliation \\
  % Address \\
  % \texttt{email} \\
}


\begin{document}


\maketitle


\begin{abstract}
As our submission for track three of the AI for Global Climate Cooperation (AI4GCC) competition, we propose a negotiation protocol for use in the RICE-N climate-economic simulation. Our proposal seeks to address the challenges of carbon leakage through methods inspired by the Carbon Border Adjustment Mechanism (CBAM) and Climate Clubs (CC). We demonstrate the effectiveness of our approach by comparing simulated outcomes to representative concentration pathways (RCP) and shared socioeconomic pathways (SSP). Our protocol results in a temperature rise comparable to RCP 3.4/4.5 and SSP 2. Furthermore, we provide an analysis of our protocol's World Trade Organization compliance, administrative and political feasibility, and ethical concerns. We recognize that our proposal risks hurting the least developing countries, and we suggest specific corrective measures to avoid exacerbating existing inequalities, such as technology sharing and wealth redistribution. Future research should improve the RICE-N tariff mechanism and implement actions allowing for the aforementioned corrective measures. 
\end{abstract}

\section{Introduction}

The 2022 UN Emissions Gap Report \cite{unep} indicates that states are not on track to keep global average surface temperature rise under the Paris Agreement target of 1.5\textdegree C since preindustrial levels \cite{pohl2023enjoying}. The frequency of extreme weather events increases exponentially with temperature rise. For instance, the frequency of unseasonably hot days increases six times under 1\textdegree C temperature rise and 20 times under 2\textdegree C temperature rise. As such, failing to meet the 1.5\textdegree C target would have disastrous consequences \cite{knutti2016scientific, fischer2015anthropogenic}. 


The difficulties of addressing climate change and potential avenues for building international cooperation become apparent after reframing the problem as a public good social dilemma~\cite{rashidi2022strategic}. The climate is a global public good as all states benefit from its well-being even if they are not directly responsible for its maintenance through emission reduction policies. As such, states face the dilemma of whether to rationally pursue their self-interest or to altruistically cooperate to maintain the climate~\cite{kaul1999defining,kaul1999global}. In this context, free-riding implies benefiting from the emissions reductions of others while domestically failing to follow suit. Free-riding can also occur temporally, as the cost of climate damage is borne by subsequent generations. Policymakers are tasked with the design of mechanisms that alter rewards such that cooperation becomes the rational choice~\cite{nordhaus2015climate}.

Multi-agent systems (MAS) is a framework suited for modeling social dilemmas~\cite{de2006learning, leibo2017multi}. MAS aids policymakers by shedding light on the decision-making processes of simulated agents~\cite{zhang2022ai}.

In the present policy brief, we will leverage MAS as a means of discovering climate negotiation protocols. We will identify which features are optimal with respect to both climate and economic health and devise a list of policy recommendations based on the optimal protocol. Finally, we will characterize the feasibility and ethical considerations of implementing the suggested policy.

\section{The shortcomings of previous policies}
The 1997 Kyoto Protocol aimed to reduce greenhouse gas (GHG) emissions in developed countries to 5\% less than pre-1990 levels~\cite{gardiner2004global}. Relying on voluntary action and without sanction mechanisms, the Kyoto Protocol failed to meet its goal~\cite{nordhaus2020climate}. Except for eight of the 15 EU signatories, developed countries failed to meet their targets. Relative to pre-1990 levels, Canada increased GHG emissions by 25\%, Japan by 14\%, and the United States by 8.4\%~\cite{jakob2021carbon}.

Subsequent attempts to reduce GHG emissions enabled free-riding due to the following policy conditions. In 2004, the European Union (EU) established its Emissions Trading System (ETS), in which businesses can purchase emission offset certificates, thereby increasing the cost of carbon in participating countries~\cite{schippers2022proposal}. The 2015 Paris Climate agreement allowed states to set their own cost of carbon, resulting in carbon leakage~\cite{overland2022climate, branger2016carbon, monjon2011border}. The aforementioned occurs when emissions increase in GHG emissions in one country as a result of emission decreases in another country. Carbon leakages have two principal causes, demand for fossil fuels by non-mitigating countries and competitive relocation of carbon-intensive industries to save costs. Carbon-leaking countries are free-riders as they do not pay the cost of emissions reduction; thereby, lowering their prices and increasing competitiveness~\cite{jakob2021carbon}. 

Multiple international agreements aim to address carbon leakage. The Carbon Border Adjustment Mechanism (CBAM), a feature of the European Green Deal taking effect in 2026, imposes a tariff on carbon-intensive goods whose value depends on the difference between the EU and the exporting countries' carbon taxes. That is, the import price of goods from carbon-leaking states will increase to levels comparable to that of ETS complaint states. Carbon-leaking countries can only lower their tariff by taxing carbon domestically, thereby reducing their emissions~\cite{pirlot2022carbon, vidigal2022false, overland2022climate}. However, by narrowly focusing on exports, CBAM leaves large swaths of carbon-intensive economic activity untouched~\cite{tarr2022carbon}. Initially proposed by William Nordhaus and codified by Article 6 of the Paris Agreement, a carbon club (CC) is a similarly constructed mechanism centered around a coalition of states that agree to a common emissions reduction target and impose a uniform tariff on all goods whose value depends on the domestic carbon tax of an exporting country. Therefore countries that export non-carbon-intensive goods but still fail to reduce GHG domestically will face higher tariffs than club members. Said exporter can always join the club by taxing carbon domestically and reducing emissions~\cite{nordhaus2015climate, nordhaus2021dynamic}.

\section{Proposed solution and recommendations}

Track two of the AI4GCC competition invites participants to develop and evaluate novel mechanisms within the RICE-N climate and economic model~\cite{zhang2022ai}. Said mechanisms take the form of negotiation protocols, and sets of actions independent of the climate-economic simulation step. Agreements between states made during the negotiation steps constrain the space of actions possible during the climate-economic simulation step. 

We developed a base negotiation protocol that borrows the CBAM and CC tariff mechanism called \emph{Basic Club} (BC) which works as follows. 
\begin{itemize}
    \item Once every five years, states gather to propose a target mitigation level.
    \item States accept or reject the proposed mitigation levels and commit to the highest accepted mitigation level, thereby always mitigating at least the given level. A club is formed as the subset of states committed to a given mitigation level.
     \item Club members set a minimum tariff on goods from states with mitigation rates lower than the club's. The tariff minimum inversely depends on the mitigation rate of the exporting country. For example, a club of mitigation 9 will, to a member of club 7, tariff at least 3. 
    \item Club members set a maximum tariff limit for states with mitigation rates greater than or equal to theirs. The tariff ceiling depends inversely on the mitigation rate of the exporting country. For example, a club of mitigation 6 will, to a member of club 8, tariff at most 2. 
    \item The climate and economy and simulated for a five-year period and the process repeats. 
\end{itemize}


\begin{table}
\caption{The design elements employed to modify the base protocol. Discrete Defect (DD) resulted in the best balance of economic and climate performance.}\label{tb:de}
\resizebox{\textwidth}{!}{
\begin{tabular}{p{0.35\linewidth} p{0.65\linewidth}}
\toprule
\textbf{Name}                      & \textbf{Description}                                                                                                                           \\ \midrule
Discrete Defection (DD)            & Discrete Defection adds an action step called Defect. Defecting states are no longer obliged to mitigate at the level of their club. \\ \midrule
Free Trade (FT)                    & Club members and states with mitigation rates that are higher than a given club have no tariffs. This is intended to incentivize the club membership further.  \\ \midrule
Max Punishment (MP)                & If a state defects once, then it receives a maximum tariff for the rest of the simulation and loses access to free trade zones.         \\ \midrule
Hard Defect (HD)                   & Defecting results in zero-level mitigation for a single economic step.                                                                
\\ \bottomrule
\end{tabular}}
\end{table}

We modified the foundational BC protocol with specific \emph{design elements} (see \autoref{tb:de}) intended to capture specific phenomena, such as states which are nominally committed to emissions reduction but in practice fail to follow through. By evaluating modified negotiation protocols along economic or climate dimensions, measured as \emph{gross output} and \emph{temperature rise}, respectively, we can explore the Pareto frontier of optimal design elements. Furthermore, we can illuminate how specific design elements work towards either climate or economic goals. 


% Figure environment removed

Comparing different combinations of the design elements, \autoref{fig:pareto} suggests that the optimal ones are Discrete Defect (BCDD) and BC without any modifications. BCDD and BC differ in how they handle defection. With BCDD, states can propose ambitious mitigation targets, then decide not to follow through as a discrete action. Under BC, on the other hand, states can only defect by proposing or accepting low mitigation rates. BCDD is the negotiation protocol that best balances economic and climate metrics. However, BC results in better climate outcomes. We believe that the difference in economic index between BC and BCDD is not an accurate representation of the real-world damage that would result from approximately 1\textdegree C temperature rise (see submission 3 for our critique of the damage function). As such, we base our recommendations on BC. A formal description of the protocol can be found in Appendix B. 

% ~\ref{appendix:formalization}. 

\paragraph{Policy recommendations}

Our solution can be implemented through the following policy recommendations:

\begin{itemize}
    \item Allow states with common mitigation ambitions to establish CCs. 
    \item Let each CC impose a uniform tariff on all goods from non-club members whose value depends on the difference between the exporting country and the CC's emission costs~\cite{nordhaus2015climate}.
    \item Create free trade zones within CCs and between CCs of higher mitigation levels~\cite{hovi2016climate, sabel2017governing}.
    \item Redistribute tariff revenue to support developing countries negatively impacted by the CC and finance sustainable infrastructure~\cite{nordhaus2021dynamic, perdana2022making}.
\end{itemize}

\section{Effectiveness} 
% ~\ref{appendix:training}

After training our model according to the configuration described in Appendix A, we compared our proposed protocol to No Protocol, which consists of disabling all negotiation functionality in the RICE-N model. \autoref{table:rcp} compares the evaluated protocols to their corresponding representative concentration pathways (RCP) and shared socioeconomic pathways (SSP) based on estimated temperature rise. Both pathways represent possible emissions scenarios based on a given degree of global mitigation~\cite{meinshausen2020shared, masson2021climate}.

\subsection{Comparison to RCP and SSP pathways}

By comparing temperature rise from a protocol to a pathway, we can make inferences about the resulting climate, demographic and economic consequences of a given protocol. As visible in \autoref{fig:gt}, our proposal results in a temperature rise of 2.09\textdegree C, compared to a 4.43\textdegree C rise of no protocol benchmark. As such, its resulting climate and economic conditions are comparable to RCP 3.4/4.5 and SSP 2. 


\begin{table}
\centering
\caption{Our proposed protocol aligns with RCP 3.4 and SSP 2 in approximate temperature rise over a 100-year time period. Both scenarios are considered compromises between extremely stringent and overly lenient approaches.}\label{table:rcp}
\resizebox{\textwidth}{!}{
\begin{tabular}{llll}
\toprule
Protocol Name     & Temperature Rise & Corresponding RCP  & Corresponding SSP \\ \midrule
No Protocol       & 4.43             & RCP 7.5/8.5      & SSP 7         \\
BCDD & 3.20 &  RCP 6.0 & SSP 2/4.5 \\
BC & 2.09             & RCP 3.4/4.5      & SSP 2        \\
\bottomrule
\end{tabular}}
\end{table}

RCP and SSP pathways are evaluated by the degree to which they change the frequency of weather events that previously occurred once in 10 years in the period between 1850-1900. For example, with SSP 2, once in 10 years extreme heat events will likely occur 5.6 times under our protocol compared to 9.4 times under the benchmark. Heavy precipitation one in 10-year events will occur 1.7 times under our protocol compared to 2.7 times without a protocol. One in 10-year agricultural and ecological droughts will increase frequency from once every 10 years to 2.4 and 4.1 occurrences per 10 years under our protocol and the benchmark, respectively~\cite{masson2021climate}.

% Figure environment removed

By 2100, sea levels would rise between .66 and 1.33 meters under our protocol compared to the benchmark's rise between .98 and 1.88 meters. Moreover, drought-induced migration is estimated to increase by an average of 201\% under the proposed protocol compared to 477.4\% globally\cite{smirnov2023climate}. 

As evident in the comparison to aforementioned pathways, our protocol is preferable to complete inaction; however, the resulting climate conditions would still result in drastic changes in the frequency of extreme weather events, sea level rise, and mass migration ~\cite{masson2021climate, smirnov2023climate, meinshausen2020shared}. This underscores the necessity of a swift and coordinated response to climate change. 

% % Figure environment removed

% Figure environment removed

The average mitigation rates and associated abatement costs are illustrated in \autoref{fig:mmr} and \autoref{fig:mac}. Assuming that our protocol corresponds to SSP~2, then our protocol would likely result in a disproportionate financial burden on developing countries and fossil fuel-dependent regions~\cite{leimbach2019burden}. Though the RICE-N simulated regions do not directly map onto real-world correlates, we can evaluate the correlation of total abatement costs with features indicative of developing countries; namely, capital, production factor, carbon intensity, and gross output. For this purpose, we evaluated BC:DD 40 times to generate sufficient output data to measure the aforementioned correlations. 

% Figure environment removed

As is shown in \autoref{fig:cor}, abatement is slightly negatively correlated with both capital and output ($r=-0.26$, $r=-0.25$, respectively). To a small but significant degree ($p<.05$), low-capital states pay more for climate mitigation. Production factor, a measure of the level of technology of a given state, is moderately negatively correlated with abatement costs ($r=-.4$, $p<.05$); that is, high-technology states tend to pay less for climate mitigation. Furthermore, carbon intensity is very strongly positively correlated with abatement costs ($r=0.94$, $p<.05$). This intuitive result indicates that states whose economic activity depends on carbon pay significantly more in mitigation costs. 

The imbalance of abatement burden sharing underscores the necessity of technology transfer and wealth redistribution policies as an accompaniment to our proposed protocol. However, in its current state, such measures are outside the scope of the current RICE-N implementation. 


\section{Feasibility}

\paragraph{Legal concerns}

The chief concern regarding Carbon Border Adjustment Mechanism (CBAM) adjacent legal instruments is their compliance with the World Trade Organization (WTO) General Agreement on Tariff And Trade's (GATT) ``most favored nation'' clause, stating that tariffs must be non-discriminatory. That is, a tariff applied to a single good must be the same for all states exporting it. While CBAM would seem to violate that clause, exceptions are made in the following conditions~\cite{vidigal2022false}. 

\begin{itemize}
    \item The agreement promotes one of the GATT article XX (g) objectives; namely, ``relating to the conservation of exhaustible natural resources.''
    \item The agreement should contribute to the objective.
    \item The agreement should not discriminate between countries. If it appears to, then its discrimination must be on the grounds justifies the rationale. 
\end{itemize}

This legal framework has precedent since the 1998 WTO Appellate Body Report ``United States - Import Prohibition of Certain Shrimp and Shrimp Productions''~\cite{shaffer1999united}. Our proposed protocol could then inherit CBAM's WTO compliance with respect to non-discrimination. 

Uniform tariffs are the other key component of our solution. The case for WTO compliance with uniform tariffs is grounded in two ways. First, uniform tariffs can correct trade imbalances caused by carbon-leaking states who have an unfair advantage. Second uniform tariffs can be considered ``sanctions against misconduct'' as is standard practice in foreign relations~\cite{pihl2020climate, mavroidis201516}. 

\paragraph{Administrative concerns}

The advantage of a CC over CBAM is the administrative simplicity of a uniform tariff. The CBAM only applies to a subset of goods. Determining the carbon content of goods is costly, opaque, and vulnerable to fraud. Given the impact of tariffs on affected industries, carbon monitoring invites extensive corporate lobbying which further distorts the accuracy of an audit~\cite{bierbrauer2021co2}. In contrast, a uniform tariff does not require carbon content determination and does not seek to punish any specific industry. Rather, uniform tariffs promote CC participation by affecting all exports~\cite{nordhaus2015climate}. Therefore, CCs are administratively more feasible than CBAM and more robust against manipulation. However, both CCs and CBAM require states to calculate carbon pricing. Some developing states lack the administrative apparatus necessary to calculate and tax carbon emissions domestically \cite{dadush2021eu}. Tariff revenue generated will need to be reinvested in carbon pricing administration of developing countries. 

\paragraph{Political Obstacles}

Whether CBAM and similar policies are successful depends partially on the involvement of large states, such as the US. In the US, climate ambitions are highly politicized, as evident in former president Trump's withdrawal from the Paris Climate Agreement \cite{martin2019multi}. Moreover, the US does not have a national ETS \cite{dadush2021eu}. Despite wavering commitments at the federal level, sub-national coalitions of states remain committed to international climate agreements through regional carbon markets. Furthermore, states politically opposed to climate mitigation may still be convinced to join the CBAM due to trade dependencies with the EU \cite{martin2019multi}. 


\section{Ethical considerations \& risks}

Structurally similar to CBAM, our proposal inherits its ethical considerations. Specifically, our  protocol risks reproducing existing developmental inequalities unless specific corrective measures are taken into account \cite{perdana2022making}. The least developing countries (LDC) that trade with the EU yet lack green industry will be disproportionately hurt by CBAM and thereby by our protocol as well. Given that offering exemptions to CBAM invites further carbon leakage, LDCs will require both revenue redistribution and technology sharing \cite{perdana2022making, nordhaus2021dynamic}. The former can weaken the impact of tariffs temporarily while the latter aids the LDC in increasing domestic mitigation efforts. Financing technology transfer and revenue redistribution are possible if tariff revenue is reallocated to these ends. Revenue raised by tariffs on LDC imports can be reinvested back in the LDCs themselves \cite{eicke2021pulling}. Failing to reinvest tariff revenue into technology transfer and revenue redistribution risks turning CBAM into a simple tax on LDCs \cite{goldthau2022open}. 

There is also the ethical risk of a literal interpretation of the model outputs. There is a simulation to reality gap that must be acknowledged by policymakers who intend to leverage its insights. Furthermore, there are concerns that CBAM and related policies can introduce further trade distortions \cite{lim2021pitfalls}. However, our proposal intends to correct trade distortions caused by carbon leakage ~\cite{bierbrauer2021co2}.

\section{Conclusion}

Carbon leakage undermines previous policy attempts to curtail GHG emissions globally \cite{overland2022climate}. Drawing inspiration from the extensive literature addressing carbon leakage, we leverage reinforcement learning in a multi-agent system to simulate novel negotiation protocols which cultivate international cooperation. Our proposed policy, results in climate and economic conditions comparable to RCP 3.4/4.5 and SSP 2. While these are by no means the worst socioeconomic pathways possible, they still imply dramatic increases in the frequency of extreme weather events and come at great economic cost \cite{masson2021climate, smirnov2023climate}. Furthermore, our proposal runs the risk of punitively taxing developing countries lacking in green infrastructure. As such, it is critical that technology is shared and wealth is redistributed to avoid exacerbating existing inequalities \cite{goldthau2022open}. 

One shortcoming of our present submission is the lack of robustness testing. This can be performed by measuring the resilience of clubs to defection by principal members, states with the highest output, and their trading partners. However, as evident in submission 3, the sanction mechanism of RICE-N does not impact reward as expected. Correcting this is a necessary precondition to proper robustness testing. See submission 3 for details. 

Future research on RICE-N for modeling cooperation should correct the tariff component, allow for technology sharing and wealth redistribution, and perform thorough robustness testing. Furthermore, RICE-N can be directly calibrated to real world nation-states for more realistic scenario analysis, such as resilience of climate clubs in the face of defection by large states.  




\begin{ack}
We would like to thank Maikel van der Knaap, Cale Davis, Albert Bomer, Catholijn Jonker, and Holger Hoos for their time spent discussing various topics of this competition.

This research was (partly) funded by the \href{https://hybrid-intelligence-centre.nl}{Hybrid Intelligence Center}, a 10-year programme funded by the Dutch Ministry of Education, Culture and Science through the Netherlands Organisation for Scientific Research, grant number 024.004.022.
\end{ack}



% \section{Submission of papers to NeurIPS 2023}


% Please read the instructions below carefully and follow them faithfully. \textbf{Important:} This year the checklist will be submitted separately from the main paper in OpenReview, please review it well ahead of the submission deadline: \url{https://neurips.cc/public/guides/PaperChecklist}.




% \subsection{Style}


% Papers to be submitted to NeurIPS 2023 must be prepared according to the
% instructions presented here. Papers may only be up to {\bf nine} pages long,
% including figures. Additional pages \emph{containing only acknowledgments and
% references} are allowed. Papers that exceed the page limit will not be
% reviewed, or in any other way considered for presentation at the conference.


% The margins in 2023 are the same as those in previous years.


% Authors are required to use the NeurIPS \LaTeX{} style files obtainable at the
% NeurIPS website as indicated below. Please make sure you use the current files
% and not previous versions. Tweaking the style files may be grounds for
% rejection.


% \subsection{Retrieval of style files}


% The style files for NeurIPS and other conference information are available on
% the website at
% \begin{center}
%   \url{http://www.neurips.cc/}
% \end{center}
% The file \verb+neurips_2023.pdf+ contains these instructions and illustrates the
% various formatting requirements your NeurIPS paper must satisfy.


% The only supported style file for NeurIPS 2023 is \verb+neurips_2023.sty+,
% rewritten for \LaTeXe{}.  \textbf{Previous style files for \LaTeX{} 2.09,
%   Microsoft Word, and RTF are no longer supported!}


% The \LaTeX{} style file contains three optional arguments: \verb+final+, which
% creates a camera-ready copy, \verb+preprint+, which creates a preprint for
% submission to, e.g., arXiv, and \verb+nonatbib+, which will not load the
% \verb+natbib+ package for you in case of package clash.


% \paragraph{Preprint option}
% If you wish to post a preprint of your work online, e.g., on arXiv, using the
% NeurIPS style, please use the \verb+preprint+ option. This will create a
% nonanonymized version of your work with the text ``Preprint. Work in progress.''
% in the footer. This version may be distributed as you see fit, as long as you do not say which conference it was submitted to. Please \textbf{do
%   not} use the \verb+final+ option, which should \textbf{only} be used for
% papers accepted to NeurIPS. 


% At submission time, please omit the \verb+final+ and \verb+preprint+
% options. This will anonymize your submission and add line numbers to aid
% review. Please do \emph{not} refer to these line numbers in your paper as they
% will be removed during generation of camera-ready copies.


% The file \verb+neurips_2023.tex+ may be used as a ``shell'' for writing your
% paper. All you have to do is replace the author, title, abstract, and text of
% the paper with your own.


% The formatting instructions contained in these style files are summarized in
% Sections \ref{gen_inst}, \ref{headings}, and \ref{others} below.


% \section{General formatting instructions}
% \label{gen_inst}


% The text must be confined within a rectangle 5.5~inches (33~picas) wide and
% 9~inches (54~picas) long. The left margin is 1.5~inch (9~picas).  Use 10~point
% type with a vertical spacing (leading) of 11~points.  Times New Roman is the
% preferred typeface throughout, and will be selected for you by default.
% Paragraphs are separated by \nicefrac{1}{2}~line space (5.5 points), with no
% indentation.


% The paper title should be 17~point, initial caps/lower case, bold, centered
% between two horizontal rules. The top rule should be 4~points thick and the
% bottom rule should be 1~point thick. Allow \nicefrac{1}{4}~inch space above and
% below the title to rules. All pages should start at 1~inch (6~picas) from the
% top of the page.


% For the final version, authors' names are set in boldface, and each name is
% centered above the corresponding address. The lead author's name is to be listed
% first (left-most), and the co-authors' names (if different address) are set to
% follow. If there is only one co-author, list both author and co-author side by
% side.


% Please pay special attention to the instructions in Section \ref{others}
% regarding figures, tables, acknowledgments, and references.


% \section{Headings: first level}
% \label{headings}


% All headings should be lower case (except for first word and proper nouns),
% flush left, and bold.


% First-level headings should be in 12-point type.


% \subsection{Headings: second level}


% Second-level headings should be in 10-point type.


% \subsubsection{Headings: third level}


% Third-level headings should be in 10-point type.


% \paragraph{Paragraphs}


% There is also a \verb+\paragraph+ command available, which sets the heading in
% bold, flush left, and inline with the text, with the heading followed by 1\,em
% of space.


% \section{Citations, figures, tables, references}
% \label{others}


% These instructions apply to everyone.


% \subsection{Citations within the text}


% The \verb+natbib+ package will be loaded for you by default.  Citations may be
% author/year or numeric, as long as you maintain internal consistency.  As to the
% format of the references themselves, any style is acceptable as long as it is
% used consistently.


% The documentation for \verb+natbib+ may be found at
% \begin{center}
%   \url{http://mirrors.ctan.org/macros/latex/contrib/natbib/natnotes.pdf}
% \end{center}
% Of note is the command \verb+\citet+, which produces citations appropriate for
% use in inline text.  For example,
% \begin{verbatim}
%    \citet{hasselmo} investigated\dots
% \end{verbatim}
% produces
% \begin{quote}
%   Hasselmo, et al.\ (1995) investigated\dots
% \end{quote}


% If you wish to load the \verb+natbib+ package with options, you may add the
% following before loading the \verb+neurips_2023+ package:
% \begin{verbatim}
%    \PassOptionsToPackage{options}{natbib}
% \end{verbatim}


% If \verb+natbib+ clashes with another package you load, you can add the optional
% argument \verb+nonatbib+ when loading the style file:
% \begin{verbatim}
%    \usepackage[nonatbib]{neurips_2023}
% \end{verbatim}


% As submission is double blind, refer to your own published work in the third
% person. That is, use ``In the previous work of Jones et al.\ [4],'' not ``In our
% previous work [4].'' If you cite your other papers that are not widely available
% (e.g., a journal paper under review), use anonymous author names in the
% citation, e.g., an author of the form ``A.\ Anonymous'' and include a copy of the anonymized paper in the supplementary material.


% \subsection{Footnotes}


% Footnotes should be used sparingly.  If you do require a footnote, indicate
% footnotes with a number\footnote{Sample of the first footnote.} in the
% text. Place the footnotes at the bottom of the page on which they appear.
% Precede the footnote with a horizontal rule of 2~inches (12~picas).


% Note that footnotes are properly typeset \emph{after} punctuation
% marks.\footnote{As in this example.}


% \subsection{Figures}


% % Figure environment removed


% All artwork must be neat, clean, and legible. Lines should be dark enough for
% purposes of reproduction. The figure number and caption always appear after the
% figure. Place one line space before the figure caption and one line space after
% the figure. The figure caption should be lower case (except for first word and
% proper nouns); figures are numbered consecutively.


% You may use color figures.  However, it is best for the figure captions and the
% paper body to be legible if the paper is printed in either black/white or in
% color.


% \subsection{Tables}


% All tables must be centered, neat, clean and legible.  The table number and
% title always appear before the table.  See Table~\ref{sample-table}.


% Place one line space before the table title, one line space after the
% table title, and one line space after the table. The table title must
% be lower case (except for first word and proper nouns); tables are
% numbered consecutively.


% Note that publication-quality tables \emph{do not contain vertical rules.} We
% strongly suggest the use of the \verb+booktabs+ package, which allows for
% typesetting high-quality, professional tables:
% \begin{center}
%   \url{https://www.ctan.org/pkg/booktabs}
% \end{center}
% This package was used to typeset Table~\ref{sample-table}.


% \begin{table}
%   \caption{Sample table title}
%   \label{sample-table}
%   \centering
%   \begin{tabular}{lll}
%     \toprule
%     \multicolumn{2}{c}{Part}                   \\
%     \cmidrule(r){1-2}
%     Name     & Description     & Size ($\mu$m) \\
%     \midrule
%     Dendrite & Input terminal  & $\sim$100     \\
%     Axon     & Output terminal & $\sim$10      \\
%     Soma     & Cell body       & up to $10^6$  \\
%     \bottomrule
%   \end{tabular}
% \end{table}

% \subsection{Math}
% Note that display math in bare TeX commands will not create correct line numbers for submission. Please use LaTeX (or AMSTeX) commands for unnumbered display math. (You really shouldn't be using \$\$ anyway; see \url{https://tex.stackexchange.com/questions/503/why-is-preferable-to} and \url{https://tex.stackexchange.com/questions/40492/what-are-the-differences-between-align-equation-and-displaymath} for more information.)

% \subsection{Final instructions}

% Do not change any aspects of the formatting parameters in the style files.  In
% particular, do not modify the width or length of the rectangle the text should
% fit into, and do not change font sizes (except perhaps in the
% \textbf{References} section; see below). Please note that pages should be
% numbered.


% \section{Preparing PDF files}


% Please prepare submission files with paper size ``US Letter,'' and not, for
% example, ``A4.''


% Fonts were the main cause of problems in the past years. Your PDF file must only
% contain Type 1 or Embedded TrueType fonts. Here are a few instructions to
% achieve this.


% \begin{itemize}


% \item You should directly generate PDF files using \verb+pdflatex+.


% \item You can check which fonts a PDF files uses.  In Acrobat Reader, select the
%   menu Files$>$Document Properties$>$Fonts and select Show All Fonts. You can
%   also use the program \verb+pdffonts+ which comes with \verb+xpdf+ and is
%   available out-of-the-box on most Linux machines.


% \item \verb+xfig+ "patterned" shapes are implemented with bitmap fonts.  Use
%   "solid" shapes instead.


% \item The \verb+\bbold+ package almost always uses bitmap fonts.  You should use
%   the equivalent AMS Fonts:
% \begin{verbatim}
%    \usepackage{amsfonts}
% \end{verbatim}
% followed by, e.g., \verb+\mathbb{R}+, \verb+\mathbb{N}+, or \verb+\mathbb{C}+
% for $\mathbb{R}$, $\mathbb{N}$ or $\mathbb{C}$.  You can also use the following
% workaround for reals, natural and complex:
% \begin{verbatim}
%    \newcommand{\RR}{I\!\!R} %real numbers
%    \newcommand{\Nat}{I\!\!N} %natural numbers
%    \newcommand{\CC}{I\!\!\!\!C} %complex numbers
% \end{verbatim}
% Note that \verb+amsfonts+ is automatically loaded by the \verb+amssymb+ package.


% \end{itemize}


% If your file contains type 3 fonts or non embedded TrueType fonts, we will ask
% you to fix it.


% \subsection{Margins in \LaTeX{}}


% Most of the margin problems come from figures positioned by hand using
% \verb+\special+ or other commands. We suggest using the command
% \verb+\includegraphics+ from the \verb+graphicx+ package. Always specify the
% figure width as a multiple of the line width as in the example below:
% \begin{verbatim}
%    \usepackage[pdftex]{graphicx} ...
%    % Figure removed
% \end{verbatim}
% See Section 4.4 in the graphics bundle documentation
% (\url{http://mirrors.ctan.org/macros/latex/required/graphics/grfguide.pdf})


% A number of width problems arise when \LaTeX{} cannot properly hyphenate a
% line. Please give LaTeX hyphenation hints using the \verb+\-+ command when
% necessary.


% \begin{ack}
% Use unnumbered first level headings for the acknowledgments. All acknowledgments
% go at the end of the paper before the list of references. Moreover, you are required to declare
% funding (financial activities supporting the submitted work) and competing interests (related financial activities outside the submitted work).
% More information about this disclosure can be found at: \url{https://neurips.cc/Conferences/2023/PaperInformation/FundingDisclosure}.


% Do {\bf not} include this section in the anonymized submission, only in the final paper. You can use the \texttt{ack} environment provided in the style file to autmoatically hide this section in the anonymized submission.
% \end{ack}



% \section{Supplementary Material}

% Authors may wish to optionally include extra information (complete proofs, additional experiments and plots) in the appendix. All such materials should be part of the supplemental material (submitted separately) and should NOT be included in the main submission.


% \section*{References}

\bibliographystyle{IEEEtranN}

\bibliography{ref}




\end{document}