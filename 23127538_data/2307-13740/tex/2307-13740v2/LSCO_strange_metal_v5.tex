
\documentclass[aps,prb,reprint,superscriptaddress,floatfix]{revtex4-2}

%\usepackage{amsmath}    % need for subequations
\usepackage{amssymb}
\usepackage{graphicx}   % need for figures
%\usepackage[lofdepth,lotdepth, caption=false]{subfig}
%\usepackage{verbatim}   % useful for program listings
\usepackage{color}      % use if color is used in text
\usepackage{hyperref}   % use for hypertext links, including those to external documents and URLs
%\usepackage{dcolumn}

\def\lno{La$_2$NiO$_4$}
\def\lnod{La$_2$NiO$_{4+\delta}$}
\def\lsno{La$_{2-x}$Sr$_x$NiO$_4$}
\def\lco{La$_2$CuO$_4$}
\def\lsco{La$_{2-x}$Sr$_x$CuO$_4$}
\def\lbco{La$_{2-x}$Ba$_x$CuO$_4$}
\def\lbcoate{La$_{1.875}$Ba$_{0.125}$CuO$_4$}
\def\lnsco{La$_{1.6-x}$Nd$_{0.4}$Sr$_x$CuO$_4$}
\def\lesco{La$_{1.8-x}$Eu$_{0.2}$Sr$_x$CuO$_4$}
\def\ybco{YBa$_2$Cu$_3$O$_{6+x}$}
\def\lcod{La$_2$CuO$_{4+\delta}$}
\def\bscco{Bi$_2$Sr$_2$CaCu$_2$O$_{8+\delta}$}
\def\hbco{HgBa$_2$CuO$_{4+\delta}$}
\def\ncco{Nd$_{2-x}$Ce$_x$CuO$_4$}

\newcommand{\LSNO}{La$_{2-x}$Sr$_{x}$NiO$_{4}$}
\newcommand{\LSNOn}{La$_{1.75}$Sr$_{0.25}$NiO$_{4}$}
%
\def\newr{\color{red}}
%\def\newr{\color{black}}
\def\newb{\color{blue}}
%\def\newb{\color{black}}

\begin{document}

\title{From non-metal to strange metal at the stripe-percolation transition in \lsco}

\author{J. M. Tranquada}
\email{jtran@bnl.gov}
\affiliation{Condensed Matter Physics and Materials Science Division, Brookhaven National Laboratory, Upton, New York 11973-5000, USA}
\author{P. M. Lozano}
\altaffiliation{Current address: Advanced Photon Source, Argonne National Laboratory, Argonne, Illinois 60439, USA}
\affiliation{Condensed Matter Physics and Materials Science Division, Brookhaven National Laboratory, Upton, New York 11973-5000, USA}
\affiliation{Department of Physics and Astronomy, Stony Brook University, Stony Brook, NY 11794-3800, USA}
\author{Juntao Yao}
\affiliation{Condensed Matter Physics and Materials Science Division, Brookhaven National Laboratory, Upton, New York 11973-5000, USA}
\affiliation{Department of Material Science \&\ Chemical Engineering, Stony Brook University, Stony Brook, NY 11794-3800, USA}
\author{G. D. Gu}
\affiliation{Condensed Matter Physics and Materials Science Division, Brookhaven National Laboratory, Upton, New York 11973-5000, USA}
\author{Qiang Li}
\email{liqiang@bnl.gov}
\affiliation{Condensed Matter Physics and Materials Science Division, Brookhaven National Laboratory, Upton, New York 11973-5000, USA}
\affiliation{Department of Physics and Astronomy, Stony Brook University, Stony Brook, NY 11794-3800, USA}


\date{\today} 

\begin{abstract}
The nature of the normal state of cuprate superconductors continues to stimulate considerable speculation.  Of particular interest has been the linear temperature dependence of the in-plane resistivity in the low-temperature limit, which violates the prediction for a Fermi liquid.  We present measurements of anisotropic resistivity in \lsco\ that confirm the strange-metal behavior for crystals with $x > p^\ast \sim 0.19$ and contrast with the non-metallic behavior for $x<p^\ast$.  We propose that the changes at $p^\ast$ are associated with a first-order transition from doped Mott insulator to conventional metal; the transition appears as a crossover due to intrinsic dopant disorder.  We consider results from the literature that support this picture; in particular, we present a simulation of the impact of the disorder on the first-order transition and the doping dependence of stripe correlations.  Below $p^\ast$, the strong electronic interactions result in charge and spin stripe correlations that percolate across the CuO$_2$ planes; above $p^\ast$, residual stripe correlations are restricted to isolated puddles.  We suggest that the $T$-linear resistivity results from scattering of quasiparticles from antiferromagnetic spin fluctuations within the correlated puddles.  This is a modest effect compared to the case at $x<p^\ast$, where there data suggest that there are no coherent quasiparticles in the normal state.
\end{abstract}

\maketitle

\section{Introduction}

There are a variety of normal-state properties of hole-doped cuprate superconductor compounds that deviate from expectations of Fermi liquid theory.  For example, there are the anomalous temperature dependences of properties such as magnetic susceptibility, in-plane resistivity $\rho_{ab}$, the Hall effect, and various spectroscopic features observed especially in underdoped cuprates that have been discussed commonly in terms of pseudogap phenomena \cite{batl96,timu99,tall01,norm05,lee06,hufn08,tail10,keim15,kord15}.  The crossover temperature $T^\ast$ associated with the pseudogap behaviors decreases as the doped-hole concentration $p$ increases, extrapolating to zero at $p^\ast\sim0.19$ \cite{tall01}.  In a cuprate with a lower superconducting transition temperature ($T_c$), such as \lsco\ (LSCO), where accessible magnetic field strengths can suppress the superconducting order, another anomaly appears for $p\sim p^\ast$ \cite{daou09b,coop09}; this is strange-metal behavior, in which $\rho_{ab}$ varies in linear proportion to $T$ in the low-temperature limit, in violation of the Fermi-liquid prediction of $T^2$ variation \cite{varm89,ande92}.  This behavior extends over the range $p^\ast\lesssim p < p_c$ \cite{huss11,legr19,ayre21}, where $p_c\sim0.3$ is the critical point at which superconducting order disappears \cite{spiv08}.

The cause of the strange-metal behavior has continued to be a major topic of discussion \cite{phil22,varm20b,hart22,zaan19,chow22,pate23,seib21,capr22,bagg23,bane21}.  To the extent that strange metal behavior is associated with $p^\ast$, it begs the question of the underlying physics of the cuprate phase diagram.  A crucial unresolved issue concerns whether there should be a first-order transition as a function of doping as one moves from the antiferromagnetic insulator at $p=0$ to a more or less conventional metal at $p>p_c$.  Anderson argued from the beginning that the parent cuprates, such as \lco, are Mott insulators \cite{ande87} and later made the case that the superexchange mechanism underlying the antiferromagnetism is incompatible with Fermi-liquid theory \cite{ande97a}.  In response, Laughlin argued \cite{laug98} that adiabatic continuity should apply from the overdoped metallic phase through the full doping range where superconducting order exists, so that if there is a Mott insulator phase at $p=0$, there should be a first order transition at a $p$ between the antiferromagnetic state and the onset of superconductivity.  From his perspective, the pseudogap behavior could be attributed to a competing order that is compatible with conventional band theory ($d$-density-wave order in his case), with $p^\ast$ corresponding to a quantum critical point (QCP) and strange-metal behavior resulting from critical scattering \cite{laug14}; others have considered a QCP due to spin-density-wave (SDW) \cite{tail10,teix23}, charge-density-wave (CDW) \cite{cast97,tsve14,seib21,capr22,bagg23}, or pair-density-wave \cite{bane21} orders.  

Laughlin proposed that the only way to distinguish a state incompatible with Fermi-liquid theory would be to look at the behavior of $\rho_{ab}$ in the low-temperature limit of the normal state, which is obscured by superconductivity in the absence of a magnetic field.  At that time, Ando and Boebinger \cite{ando95,boeb96} had discovered that on suppressing the superconductivity in LSCO with a $c$-axis magnetic field of 60~T, both $\rho_{ab}$ and $\rho_c$ grew at low temperature as $\ln(1/T)$ for $x\lesssim0.17$; they characterized this as an insulating state \cite{boeb96}.  The results for $\rho_{ab}$ have now been reproduced by Caprara {\it et al.}\ \cite{capr20} who applied fields of $\gtrsim45$~T to LSCO thin films.   In Fig.~\ref{fg:rsh}, we show their results, measured in high field, in the form of sheet resistance, $R_{\rm s}$, at $T=4$~K and at 70~K as a function of $x$; in field, there is a rise in $\rho_{ab}$ at low temperature for $x\lesssim p^\ast$.  Whether or not the resistivity truly diverges as $T\rightarrow 0$, this behavior is inconsistent with a conventional metallic state.  

For a two-dimensional (2D) metal treated by Boltzmann theory, the Mott-Ioffe-Regel limit on resistivity (electronic mean free path equals the lattice spacing \cite{ioff60}) corresponds to $h/e^2$ \cite{werm17}.  The data shown in Fig.~\ref{fg:rsh} are all below, but a significant fraction of, this limit.
%The data shown are all below the Mott-Ioffe-Regel limit on resistivity for a metal in two dimensions (2D) \cite{ioff60,werm17}, which is $h/e^2$.  
For conventional metals with large electron scattering, the resistivity is empirically observed to saturate at a value of $200\pm100$~$\mu\Omega$-cm \cite{mooi73,gunn03,huss04}, which corresponds to a sheet resistance for LSCO of $R_{\rm sat} \sim 0.12h/e^2$.   
%In contrast, these low-temperature values for $x<p^\ast$ are {\it greater} than the high-temperature {\it saturation} limit observed in conventional metals \cite{mooi73,gunn03,huss04}: this limit of $\sim200$~$\mu\Omega$-cm corresponds to a sheet resistance for LSCO of $R_{\rm sat} \sim 0.12h/e^2$.  
While $R_{\rm sat}$ should have some dependence on carrier density \cite{werm17}, we note the striking coincidence that the crossover from non-metallic to metallic temperature dependence at $p^\ast$ occurs where $R_s$ crosses through the typical $R_{\rm sat}$.
%the empirical crossover at $p^\ast$ shown in Fig.~\ref{fg:rsh} is intriguing.

%  Another feature to note is that the magnitude of $R_{\rm s}$ is comparable to the quantum of resistance, $h/e^2$.  
%At 4~K and $x\leq0.125$, $R_{\rm s} > h/(4e^2)$, the quantum of resistance for pairs, consistent with a previous result on a related compound \cite{li19a}.  Within this context, the strange-metal behavior at higher doping is a modest deviation from conventional behavior.
%\cite{gunn03}

% Figure environment removed

Such anomalous behavior provides justification for considering underdoped cuprates in terms of a model of a hole-doped Mott insulator.  Based on early computational evidence for phase separation of doped holes and antiferromagnetic spins in a doped $t$-$J$ model \cite{emer90},  Emery and Kivelson \cite{emer93} proposed that inclusion of long-range Coulomb repulsion would yield frustrated phase separation, which can take the form of charge and spin stripes \cite{low94}.  Experimental studies on LSCO and related compounds provide evidence that charge and spin stripes develop together \cite{tran95a,hunt99,savi02,birg06,crof14}. It is important to note here that, while these correlations can be described as intertwined spin density wave (SDW) and charge density wave (CDW) orders, they are a consequence of competition between antiferromagnetism driven by superexchange \cite{ande87,ande97a} and the frustrated kinetic energy of the doped holes \cite{frad15}.  They are distinct from the individual SDW or CDW orders that can develop in conventional metals and that have been considered by some \cite{tail10,cast97,seib21,capr22,bagg23}.

Given that cuprates with $p\gtrsim p_c$ appear to have Fermi-liquid character \cite{coop09,plat05,vign08,kram19}, reconciling this with the strongly-correlated behavior of underdoped samples appears to imply that there is a first-order transition from doped Mott insulator to conventional metal somewhere under the superconducting dome.  Complicating the identification of such a first-order transition is the presence of significant charge disorder.  The role of intrinsic disorder on the anticipated first-order transition in cuprates has been considered previously \cite{burg01}, and it has been noted that, even in a half-filled correlated system, disorder with percolative character can lead to a metallic response \cite{szab20}.  The picture presented here is less extreme than the latter, but anticipates the first-order transition at higher-doping than the former.

%Even in a half-filled correlated system, can get metallic character from disorder with a percolative character \cite{szab20}.  (Dagotto also considered disorder effects \cite{burg01}.)

Working with others, we have recently reported a study of overdoped LSCO \cite{li22} in which we presented evidence for inhomogeneous superconductivity and made the case that the heterogeneity was a natural consequence of the random substitution of Sr$^{2+}$ ions for La$^{3+}$ together with the absence of effective screening.  In the present article, we present further measurements on our LSCO crystals.  We begin by presenting ac susceptibility measurements of the superconducting transition to demonstrate that, for each $x$, there is a single sharp bulk superconducting transition, consistent with previous reports \cite{bozo16,he16}.  Then we consider $\rho_c(T)$ and $\rho_{ab}(T)$ for compositions spanning $p^\ast$.  In particular, we demonstrate that $\rho_c(T)$ changes from insulating to metallic character as $x$ increases through $p^\ast$.  In a corresponding fashion, $\rho_{ab}(T)$ changes from flattening out near $T_c$ for $x < p^\ast$ to exhibiting strange-metal behavior for $x > p^\ast$, consistent with previous work \cite{coop09}.

We then review experimental results in the literature that support the interpretation that $p^\ast$ corresponds to a percolative transition 
%resulting from the first-order transition, between the doped-Mott insulator, 
between the stripe-correlated phase and a conventional metallic phase.
%, that is masked by disorder.   
Superconductive pairing is optimized where the stripe phase is strong \cite{tran21a}, and bulk superconductivity weakens as regions of stripe phase are limited to non-percolating puddles with overdoping \cite{li22,spiv08}.  Within this picture, it appears that the normal-state resistivity for $p>p^\ast$ should be dominated by the scattering of quasiparticles from isolated patches of strongly-correlated electrons.  This is consistent with the two charge-carrier components inferred by Ayres {\it et al.} for a variety of cuprates \cite{ayre21} and with aspects of some recent models \cite{pate23,seib21,capr22,bagg23,rice17,lee21c,ma23}, especially including recent numerical studies of the Hubbard model \cite{huan19,huan23,wu22}.

%We confirm the strange-metal behavior in LSCO for $p\gtrsim p_\ast$ and also show that there is a transition along the $c$ axis from insulating to metallic behavior at $p^\ast$.

The rest of this paper is organized as follows: In Sec.~\ref{sc_exp}, we describe our experimental methods, with the characterizations (ac susceptibility and anisotropic resistivity) of our LSCO crystals presented in Sec.~\ref{sc_res}.  The interpretation of the results, and especially the nature of $p^\ast$, are described in Sec.~\ref{sc_int}.  The relevance of our results to other studies are considered in Sec.~\ref{sc_dis}, which is followed by our conclusions in Sec.~\ref{sc_con}.

%%%%%%%%%%%%%%
%Given that the standard model of superconductivity is based on Fermi liquid theory \cite{bard57}, deviations from the latter theory in high-temperature superconductors, such as hole-doped cuprates, raise interesting questions.  One prediction of the Fermi-liquid model is that the resistivity should vary as the square of the temperature $T$ in the low-$T$ limit. In contrast, it was observed early on that the in-plane resistivity, $\rho_{ab}$, varies linearly with $T$ for cuprates with doped-hole concentrations $p$ slightly beyond the value at which the superconducting transition $T_c$ is maximized \cite{gurv87,mart90}.  Since the significance of this strange metal behavior was first recognized \cite{varm89,ande92}, it has continued to receive considerable attention, especially in recent years \cite{phil22,varm20b,hart22,zaan19}.  

%Of course, the cuprates exhibit many other deviations from simple Fermi-liquid behavior, especially at smaller values of $p$.  For example, properties such as the temperature dependence of the spin susceptibility \cite{allo89}, resistivity \cite{ito93}, and Hall effect \cite{hwan94,ando04} in underdoped cuprates led to the concept of the pseudogap \cite{timu99,tall01,norm05,hufn08}.  From a conventional perspective, the pseudogap has been interpreted as a consequence of some sort of competing order that gaps some states near the Fermi level, and various competing orders have been proposed \cite{cast97,sach00,chak01a,varm06}.  Experimental studies indicate that the onset temperature $T^*$ of pseudogap effects decreases with increasing $p$  \cite{keim15}, tending to zero at $p^*\sim0.19$ \cite{tall01}.  Experiments initially associated the $T$-linear resistivity with proximity of doping to $p^*$ \cite{daou09b,coop09}, which has often been identified as a quantum critical point \cite{sach03,norm05,varm20b}.  While it has been tempting to associate it with quantum critical fluctuations \cite{tail10,doir12,varm20b}, such a perspective is constrained by the difficulty in identifying a universal order parameter for the cuprates that disappears at $p^*$ due to quantum fluctuations.  

%Recent experiments have made clear that the strange-metal behavior is present across the regime of overdoped superconductivity, corresponding to $p^*< p<p_c$ \cite{huss11,legr19,ayre21}, where $p_c\sim0.3$ is the quantum critical point corresponding to the superconductor-to-metal transition \cite{spiv08}.  These results have motivated phenomenological models that consider a distribution of random interactions \cite{pate19,chow22} or electronic scattering from fluctuating charge-density-wave (CDW) \cite{seib21,capr22,bagg23}, or pair-density-wave \cite{bane21} correlations.

%An alternative approach is to start with the fact that a cuprate such as \lsco\ (LSCO) is a hole-doped antiferromagnet.  The tendency of the holes to minimize their kinetic energy by delocalizing competes with the strong superexchange between local Cu moments.  Their mutual frustration is reduced by the cooperative formation of intertwined spin and charge stripes \cite{emer93,whit98a,birg06,fuji12a,frad15,tran21a}.  How the stripe correlations evolve with $p$ across $p^*$ is distinct from the picture of a quantum critical point.  An important confounding factor is the intrinsic disorder associated with the random distribution of charged dopant ions, which must be taken into account \cite{sing02a,sing05,li22}.

%In this paper, we present the case that the phenomena associated with the identification of $p^*$ in LSCO can be interpreted in terms of an effective percolation transition for spin and charge stripe correlations.  Besides showing how a variety of experimental results in the literature support this picture, we also report measurements of resistivity both along the $c$ axis, $\rho_c$, and in-plane for a number of LSCO single crystals.  In particular, the normal-state temperature dependence of $\rho_c$ crosses over from insulating to metallic at $p^*$, reflecting the change in electronic correlations within the CuO$_2$ planes.  Measurements of ac magnetic susceptibility provide further evidence of the granular character of the samples.  We then associate the strange-metal behavior of $\rho_{ab}$ at $p>p^*$ with quasiparticle scattering from residual patches of dynamic spin stripe correlations, compatible with theoretical analyses of the strong impact of antiferromagnetic correlations on electron transport \cite{rice17,lee21c,ma23}.

%In cuprates where the dopant-induced hole density $p$ is low, the normal-state $\rho_{ab}$ is limited by pseudogap behavior below a temperature $T^*$ \cite{keim15}, but this scale tends to zero at $p^*\sim0.19$ \cite{tall01}.  

%Debate over interpretation of anomalous normal-state behavior in cuprates (related to pseudogap).  Anderson argued \cite{ande97a} that the antiferromagnetic correlations seen in parent cuprates and underdoped compositions is associated with a parent Mott insulator, which cannot be reconciled with Fermi-liquid theory.  In a critique, Laughlin argued \cite{laug98} that adiabatic continuity should apply, so that any phase transitions that occur with doping should be second order, with a possible quantum critical point.

%Early calculations on the $t$-$J$ model gave indications that holes doped into the antiferromagnet would lead to phase separation \cite{emer90}.  Taking account of long-range Coulomb interactions, the idea of frustrated phase separation was developed by Emery and Kivelson \cite{emer93}.  As this behavior occurs with the coexistence of the local AFM correlations, it needs to occur below the first-order transition to a more conventional metallic phase.

%adiabatic continuity - Laughlin argued for this, especially with his Hartree-Fock analysis \cite{laug14}.  Problem: AF correlations then require an electronic gap.  Similarly, when AF order goes away, gap should close and system become metallic.  This is inconsistent with experiment.

%To the extent that one reaches a more conventional metallic state in very over-doped cuprates, then, there must be a first-order transition as a function of doping.  This gets masked by disorder.  Superconductivity seems to prefer the strongly-correlated state, and decays in the over-doped mixed state.

%Laughlin emphasized identifying states of matter in the zero-temperature limit.  This is problematic for undoped curprates.  You have to go to high temperature and look at the disordered state to appreciate the difference.

%PWA emphasized the difference between FM and AFM.

%DMFT has been able to capture the proper insulating paramagnetic phase of \lco\ \cite{choi16}.  Tremblay has evaluated the first-order transition from Mott insulator to a metallic state.  Picture can be a bit more complicated with stripe order, but this should probably still have a first-order transition.

%Disorder complicates things.  Increasing disorder can even lead to a transition from a half-filled Mott insulator to a percolating metallic phase \cite{szab20}.

%Experimental measurements on hole-doped cuprates have revealed a number of deviations from that theo

%The hole-doped cuprate superconductors have a variety of intriguing properties that deviate from conventional behavior.  Given that the standard model of superconductivity is based on Fermi liquid theory \cite{bard57}, it is understandable that many experimental properties are compared with expectations for the Fermi-liquid model.   Applying such a perspective to properties such as the temperature dependence of the spin susceptibility \cite{allo89}, resistivity \cite{ito93}, and Hall effect \cite{hwan94,ando04} led to the concept of the pseudogap \cite{timu99,tall01,norm05,hufn08}.  An attempted explanation is some sort of competing order that Induces a gap in some states near the Fermi level, and various competing orders have been proposed \cite{cast97,sach00,chak01a,varm06}.  Experimental studies indicate that the onset temperature $T^*$ of pseudogap effects decreases with increasing doped-hole concentration $p$  \cite{keim15}, tending to zero at a doped-hole concentration $p^*\sim0.19$ \cite{tall01}.

%competing order and QCP

%deviations from FL theory: linear resistivity

%disorder

%%%%%%%%%%%%%%
%strange-metal behavior is one of a number of anomalous behaviors observed in hole-doped cuprates

%To understand it, it is necessary to take proper account of the character of cuprates

%pseudogap picture with deviations from FL theory

%competing order that terminates at a QCP.  Quantum fluctuations have the frequency and temperature dependence relevant to explaining strange-metal behavior.

%But no competing order ending in a QCP has been identified.  And strange metal behavior extends well above $p^*$.

%Another approach is to take seriously the fact that the cuprates are doped antiferromagnets, with the correlations of the local Cu moments evolving continuously from the charge-transfer Mott insulator state as holes are introduced.  In LSCO, this involves the development of intertwined charge and spin stripes.  Not charge or spin, but charge and spin, spatially segregated.  But challenge of pinning in a square lattice (which is actually good for the bulk SC).

%One must also take account of the intrinsic disorder

%In the case of LSCO, it has been argued that one has a crossover from stripe correlations dominating the planes to metallic quasiparticles dominating \cite{tran21a}.  But how does this crossover take place?  We present the case that this happens effectively through a percolation threshold.  We collect a number of different results consistent with this picture.

%We then use this picture to argue that the strange-metal behavior at $p>p^*$ is a consequence of the residual stripe correlations, which are detected through spin fluctuations.
%%%%%%%%%%%%%%%%%%%%%

%For example, metals described by the Fermi-liquid model exhibit a resistivity that varies as the square of the temperature $T$ in the low-$T$ limit. In contrast, it was observed early on that the in-plane resistivity, $\rho_{ab}$, for cuprates at particular compositions varies as $T$ \cite{gurv87,mart90}.  Since the significance of this strange metal behavior was first recognized \cite{varm89,ande92}, it has continued to receive considerable attention \cite{phil22,varm20b,hart22,zaan19}.  In cuprates where the dopant-induced hole density $p$ is low, the normal-state $\rho_{ab}$ is limited by pseudogap behavior below a temperature $T^*$ \cite{keim15}, but this scale tends to zero at $p^*\sim0.19$ \cite{tall01}.  Experiments initially associated the $T$-linear resistivity with proximity of doping to $p^*$ \cite{daou09b,coop09}, which has often been identified as a quantum critical point \cite{sach03,norm05,varm20b}.  While it has been tempting to associate it with quantum critical fluctuations \cite{tail10,doir12,varm20b}, such a perspective is constrained by the difficulty in identifying a universal order parameter for the cuprates that disappears at $p^*$ due to quantum fluctuations.  

%Recent experiments have made clear that the strange-metal behavior is present across the regime of overdoped superconductivity, corresponding to $p^*< p<p_c$ \cite{huss11,legr19,ayre21}, where $p_c\sim0.3$ is the quantum critical point corresponding to the superconductor-to-metal transition \cite{spiv08}.  These results have motivated phenomenological models that consider a distribution of random interactions \cite{pate19,chow22} or electronic scattering from fluctuating charge-density-wave (CDW) \cite{seib21,capr22,bagg23}, antiferromagnetic (AFM) \cite{emer93,rice17,lee21c,ma23}, or pair-density-wave \cite{bane21} correlations.

%In this paper, we use the system \lsco\ (LSCO, where $p=x$) as an example and present the case that $p^*$ does not correspond to a critical point for fluctuation disorder of a spatially-uniform order parameter, but rather the percolation limit for spin and charge stripe correlations.  The strong correlations that lead to stripe modulations also control the evolution of the mobile hole density and the pairing interaction.  Of course, these correlations evolve within a spatially inhomogeneous environment due to the random distribution of poorly-screened dopant ions \cite{tran21a,li22,spiv08}, which means that there can be a mixture of stripe-correlated patches and uniformly-doped regions.  For $p<p^*$, where stripe correlations percolate across the sample, the conductivity of quasiparticles in the uniformly-doped regions is constrained by scattering from antiferromagnetically-correlated Cu moments within spin stripes.  This constraint is removed for $p>p^*$; however, stripe-correlated patches survive and continue to impact the normal-state transport, which is reflected in both the $p$ and $T$ dependence of the $c$-axis resistivity, $\rho_c$, as we will demonstrate.  In this picture, the strange-metal behavior occurs in the presence of domains of fluctuating CDW and AFM correlations, consistent with the two charge-carrier components inferred by Ayres {\it et al.} for a variety of cuprates \cite{ayre22} and  with recent models \cite{seib21,capr22,bagg23,rice17,lee21c,ma23}, especially including recent numerical studies of the Hubbard model \cite{huan19,huan23,wu22}.

%%%%%%%

%The rest of this paper is organized as follows: In Sec.~II, we describe our experimental methods, with the characterizations of our LSCO crystals presented in Sec.~III.  We 

\section{Experimental Methods}
\label{sc_exp}

The \lsco\ crystals used here are the same ones previously studied in Refs.~\onlinecite{li18,miao21,li22}.  They were grown at Brookhaven by the travelling-solvent floating-zone method.  Post-growth annealing in O$_2$ is described in Ref.~\onlinecite{li22}.  

The ac magnetic susceptibility was measured in a 7-T Quantum Design Magnetic Properties Measurement System with a SQUID (superconducting quantum interference device) magnetometer. The measurements were conducted under a zero dc field, with an applied ac magnetic drive amplitude of 2~Oe at a frequency of 100~Hz.  Separate measurements were performed with the ac magnetic field either parallel or perpendicular to the $c$-axis.

The resistivity measurements were performed on rectangular-plate-shaped single crystals with four different doping values: $x=0.17$, 0.21, 0.25, and 0.29. Samples were oriented and cut for both in-plane and out-of-plane resistivity measurements for each doping level. The electrical contacts were done in a four-point in-line configuration, and the measurements were performed in a Quantum Design Physical Property Measurement System equipped with a 14-T superconducting magnet.


\section{Results} 
\label{sc_res}

\subsection{ac susceptibility}

Our results for the magnetic susceptibility are shown in Fig.~\ref{fg:ac}.  The temperature at the peak of $\chi''$ provides a measure of $T_c$.  As one can see, there is a single narrow peak in $\chi''$ for $x=0.17$ with ${\bf H} || {\bf c}$, which measures the superconducting transition within the CuO$_2$ planes.  The fact that the peak for ${\bf H} \perp {\bf c}$ occurs at a slightly lower temperature is consistent with the fact that the superconducting response between the planes is dependent on Josephson coupling between the planes; note that the temperature at which $\chi''$ begins to rise on cooling for both field orientations is essentially the same.  A similar anisotropy has been reported in measurements of superconducting stiffness on ring shaped crystals at the same composition \cite{sama24}.

% Figure environment removed

For $x=0.25$, we see a single peak $\chi''$ at $\approx17$~K; however, $\chi''$ begins to rise from $T>30$~K.  For $x=0.29$, the peak has moved below 5~K, but the initial rise is still above 30~K.  It is of interest to compare with work on high-quality LSCO thin films across the overdoped regime \cite{bozo16}, where the superconducting transition was measured by mutual inductance at frequencies of 20--90 kHz with ${\bf H} || {\bf c}$ \cite{he16}.  In the imaginary part of the mutual inductance, there is a single narrow peak for each sample.  Similar measurements on LSCO films in an earlier study by a different group \cite{lemb11} show peaks with a bit more width and structure, closer to our results.  In any case, we argue that there is a strong qualitative similarity between our results and those for thin films \cite{he16,lemb11}.  The tails to high temperature in our results are somewhat subtle; nevertheless, they are consistent with the results from dc magnetization reported in \cite{li22}, where we made the case for an onset of granular, non-percolating superconductivity above the bulk transition for the $x=0.25$ and 0.29 crystals.

\subsection{Anisotropic Resistivity}

% Figure environment removed

%Those results suggest that we take a closer look at the connections between $\rho_c$ and $\rho_{ab}$. 
 In Fig.~\ref{fg:rho}, we compare the temperature dependence of  $\rho_c$ (top) and $\rho_{ab}$ (bottom) for four compositions of LSCO.  Let us first consider the behavior of $\rho_c$.  For $x=0.17$, $\rho_c$ is large in magnitude and there is a kink below 250~K corresponding with the structural transition from tetragonal to orthorhombic symmetry; this is followed by an upturn on cooling below 80~K, suggestive of insulating behavior (consistent with the results reported in high magnetic fields \cite{boeb96,capr20}), before reaching the onset of superconductivity.    In comparison, the results for $x=0.21$, 0.25, and 0.29 show a reduced (though still large) magnitude and a metallic temperature dependence in the normal state. 
 
%Besides the large change in magnitude, there is a substantial change in the temperature dependence of $\rho_c$ between $x=0.17$ and 0.21 that correlates with the change in $\rho_{ab}$.  Looking at $\rho_c$ for $x=0.17$ on cooling, there is a kink below 250~K corresponding with the structural transition from tetragonal to orthorhombic symmetry, and then there is a rise below 80~K of insulating character.  
Next we consider $\rho_{ab}$.  For $x=0.17$, we see a metallic temperature dependence; %, with upward curvature on cooling that roughly correlates with the changes in $\rho_c$; furthermore, 
on depressing the superconducting transition by applying a 14~T magnetic field along the $c$ axis,  there is a flattening of $\rho_{ab}$ in the extended normal-state region, which has been associated with the slowing of magnetic fluctuations \cite{bour19} as expected for $x<p^*$.  
%Moreover, the magnitude of $\rho_{ab}$ in this regime is {\it larger} than the {\it saturation} resistivity observed in conventional metals \cite{mooi73,huss04}. 
This behavior contrasts with that of the crystals with $x>p^*$, where we see that $\rho_{ab}$ has reduced magnitude and $T$-dependence.  The insets indicate that depressing $T_c$ with a magnetic field results in a $T$-linear extension of the normal-state behavior to lower temperature.  

To emphasize the strange-metal character of $\rho_{ab}$ for $x>p^\ast$, we plot $d\rho_{ab}/dT$ in Fig.~\ref{fg:drho} .  If $\rho_{ab}$ varies linearly with $T$, then $d\rho_{ab}/dT$ should be a constant, which is approximately the case for $x=0.21$.  There is some $T$-dependent slope that appears for $x=0.25$ and 0.29, but this is small compared to the case of $x=0.17$.  Our results are consistent with a large deviation from Fermi-liquid  behavior in LSCO for $p^*<x<p_c$, as previously identified \cite{coop09,ayre21}.  

% Figure environment removed

Altogether, our transport results support the picture of a transition from non-metal to strange metal in LSCO at $x\sim p^\ast$.  There is a clear crossover in the behavior of $\rho_c$ from low-temperature insulator-like to metallic character.  In $\rho_{ab}$, there is a large jump in the residual resistivity across $p^\ast$; for $x=0.17$, the residual resistivity appears to be $> 250$~$\mu\Omega$-cm, which is above the level at which saturation typically occurs in metals with strong electron-phonon interactions \cite{mooi73,gunn03,huss04}, as discussed in the Introduction.  In particular, our results are consistent with the thin-film data \cite{capr20} shown in Fig.~\ref{fg:rsh}.

\section{Interpretation}
\label{sc_int}

We have outlined our interpretation of the doping-dependent behavior of LSCO in the introduction.  In this section, we present more supporting details, reviewing both experimental results and theoretical analyses.

\subsection{Spin and charge stripe correlations in LSCO}

The spin stripe correlations in LSCO evolve with doping directly from the antiferromagnetic parent phase at $x=0$ \cite{yama98a,birg06}.  For $x>0.05$, the spin stripes are approximately parallel to the Cu-O bonds, and they are detected by neutron scattering at peak positions separated from the antiferromagnetic wave vector in a direction parallel to a Cu-O bond by an amount $\delta a^*$, where $a^*=2\pi/a$ and $a \sim 3.8$~\AA\ is the lattice parameter of a square CuO$_2$ plane.  The doping dependence of $\delta$ \cite{yama98a,zhu23} is indicated by green circles in Fig.~\ref{fg:model}(b).

Most of the magnetic spectral weight is in the inelastic response.  By integrating the magnetic dynamical structure factor over excitation energy and momentum, one obtains a quantity proportional to $\langle m^2\rangle$, where $m$ is the magnetic moment per Cu site.  Several studies \cite{hayd96a,fuji12a,zhu23,waki07b} have provided results that collectively demonstrate the decrease of $\langle m^2\rangle_x$ for doping $x$ relative to the result at $x=0$, as shown by the open green circles in Fig.~\ref{fg:model}(a).  (For samples at larger $x$, we have used available data with $\hbar\omega<100$~meV that have been compared to lower dopings \cite{zhu23,waki07b}.)  We are not aware of a comparable measure of the charge-stripe response vs.\ doping.

% Figure environment removed

Charge stripe order in LSCO was first detected near $x=0.12$ by hard x-ray scattering \cite{crof14}.  Recent resonant soft x-ray scattering studies \cite{wen19,lin20,miao21} have extend the range over which charge-stripe correlations have been observed.  The associated scattering peaks appear at wave vectors of $2\delta a^*$ about reciprocal lattice vectors; the experimental values are presented as blue diamonds in Fig.~\ref{fg:model}(b).  Within the experimental uncertainties (not shown), these charge and spin modulations (measured at low temperature) are mutually commensurate.
The concept that spin and charge stripes are part of the same intertwined state, representative of strong correlations, is supported by numerical studies of the Hubbard and $t$-$J$ models using parameters relevant to hole-doped cuprates \cite{whit98a,zhen17,jian21,jian21c,pons23}.  Calculations yield dynamic \cite{mai22}, as well as static, stripes.

The spin stripe correlations are purely dynamic for $x\gtrsim0.13$ \cite{khay05,chan08}.  Similarly, the charge stripe correlations are also likely to be fluctuating in that range \cite{vona23,bado16}.  Even where stripe order is detected, it can be inhomogeneous.  Measurements with muon spin rotation ($\mu$SR) spectroscopy \cite{savi02,savi05} and $^{63}$Cu \cite{imai17} and $^{139}$La \cite{arse20} nuclear magnetic resonance (NMR) studies indicate that the charge and spin orders occur in only a fraction of the sample volume for $x\sim0.12$.   The key issue is not whether the stripes are static, but the relative area within a CuO$_2$ plane that they occupy.  

In the Appendix, we comment on some experimental results for LSCO thin films with $x>0.3$ that have created some confusion in the field.


%1) spin stripes evolve with doping from the AF state; mention Fig. 5(b)

%2) most spin weight is dynamic and it decreases with doping; mention Fig. 5(a)

%3) charge order first studied, with spin stripe order in LBCO, LNSCO, and LESCO.  charge stripes seen over a broad range, but more difficult to detect and not quite as wide as spins

%4) where there is a significant correlation length, such as at LT, charge and spin stripes are mutually commensurate

%5) dynamic correlations observed at elevated T, and for x > 0.13

%6) numerical studies of Hubbard and $t$-$J$ models support intertwined charge and spin stripe correlations

%7) some differences in character of charge order in some other cuprates, such as YBCO, but IC spin correlations evolve in the same way.  

%8) inhomogeneity: muSR can see similar local field strength but finite volume fraction; NMR distribution of environments from La?

%NS shows that weight of spin fluctuations decays with doping, especially beyond $p^\ast$.  Scattering from dynamic charge stripes has not yet been detected beyond $p=0.21$; however, we would expect the signal to be quite weak.

%onset of slow spin fluctuations together with charge order indicated by Cu NQR

%%%%%

%There have been differing perspectives on the nature of charge and spin correlations in cuprates.  In the case of LSCO, we believe that there is a solid experimental case for intertwined charge and spin stripe orders \cite{tran13a,frad15}, and we will discuss some of that evidence here.  Much of the case comes from parallels with the virtually isostructural compounds \lbco\ (LBCO) \cite{fuji04,duns08,huck11}, \lnsco\ (LNSCO) \cite{tran97a,ichi00,ma21,ma22a,ma22b}, and \lesco\ (LESCO) \cite{fink11,lee22}, in which the low-temperature structural phase has only 2-fold rotational symmetry within the CuO$_2$ planes, thus allowing pinning of charge and spin stripe correlations that appear in dynamic form at higher temperatures \cite{ma22a,lee22,fuji04,miao17}.

%The evidence for spin-stripe order comes from magnetic superlattice peaks detected by single-crystal neutron diffraction \cite{huck11,fuji04,tran97a,ma21,lee22}.  The peak positions are separated from the antiferromagnetic wave vector in a direction parallel to a Cu-O bond by an amount $\delta a^*$, where $a^*=2\pi/a$ and $a \sim 3.8$~\AA\ is the lattice parameter of a square CuO$_2$ plane.  The charge order peaks are split around fundamental Bragg peaks by $2\delta a^*$ and have been detected by neutron diffraction \cite{fuji04,tran97a,duns08}, hard x-ray diffraction \cite{vonz98,huck11}, and by resonant soft x-ray scattering \cite{abba05,lee22,gupt21,miao19,fink11}.  For a given compound and composition exhibiting static stripe order, the values of $\delta$ from charge and spin order peaks are the same within experimental uncertainties, which indicates that the orders are mutually commensurate.  The charge order generally sets in at a higher-temperature than the spin order; however, the spatial modulation of the antiferromagnetic spin correlations is already established, and it is just the pinning of the spin direction that requires further entropy reduction \cite{tran08}.

%The bond-parallel stripe correlations appear for $x\gtrsim0.05$ \cite{birg06,duns08b}; the incommensurability $\delta$, proportional to the stripe density, initially grows linearly with $x$, but the growth slows for $x \gtrsim 1/8$ and tends to reach a limit for $x$ above 0.15 \cite{birg06,huck11,lee22}.  A very similar evolution of $\delta$ with $x$ is observed in LSCO \cite{yama98a}, as we will illustrate in Sec.~\ref{sc-model}, even though the occurrence of stripe order is more limited and much weaker.  Even at $x=0.12$, where the static order is strongest \cite{kimu00b,chan08,crof14}, that volume fraction is limited \cite{savi05}.

%The weakness of the stripe order in LSCO is understandable in terms of subtle differences in crystal structure relative to the LBCO, LNSCO, and LESCO, as it does not have the same degree of bond anisotropy.  That anisotropy is determined by the pattern of tilting of CuO$_6$ octahedra.  The tilts in the low-temperature phase of LBCO are around one of the Cu-O bond directions, whereas the dominant tilt in LSCO is at 45$^{\circ}$ to the bonds \cite{axe94}.  Recent diffraction studies on single crystals of LSCO with $x=0$ \cite{reeh06} and $x=0.12$ \cite{fris22} have demonstrated that a variety of weak diffraction peaks that are forbidden in the previously assumed symmetry.  It has been noted in studies of $x=0.07$ \cite{jaco15} and $x=0$ \cite{sapk21} that, among the new peaks, there is evidence for a rotation component around the Cu-O bonds, such that the net rotation axis is shifted from the 45$^circ$ direction, introducing some bond anisotropy within the CuO$_2$ planes.  Consistent with this, it has been observed In LSCO $x=0.12$ that the stripes are rotated slightly with respect to the lattice.  This is seen both for spin \cite{kimu00b} and charge order \cite{tham14}.

%There is competition between stripe order and bulk superconductivity \cite{komi04,ichi00,huck11,fink11}, with the depression of $T_c$ increasing in the sequence LSCO : LBCO : LNSCO : LESCO.  In LSCO, spin-stripe order is observed only for $p\lesssim0.13$ \cite{khay05,chan08}.  For $p>0.13$, there is a gap in the low-energy spin fluctuations that can be closed, along with suppression of the superconductivity, by application of a sufficiently strong $c$-axis magnetic field \cite{khay05,chan08,lake01,lake02,tran04b}.   Substitution of a small amount of Zn for Cu has a similar effect as a magnetic field, depressing $T_c$ while enhancing the volume fraction of stripe order \cite{hiro01,gugu17}.

%The  magnitude of antiferromagnetic spin-stripe correlations (at least for energies below the scale of the effective superexchange energy, where neutron scattering measurements have been done) decreases with doping and trends toward zero at $p\sim0.30$ \cite{waki04,waki07b,zhu23}.  Charge order also seems to turn to fluctuations, which are a bit harder to follow than the spin correlations; they have been identified at $p=0.21$ \cite{miao21}, but not beyond.  

%Superconductivity clearly occurs in compositions where stripe correlations are strong.  Indications of fluctuating pair correlations in the normal state from measurements of the Nernst effect \cite{wang06} and from a study of shot noise \cite{zhou19} extend to the highest temperature near $x\sim0.12$, where stripe correlations are optimal.

%The decrease of $T_c$ with overdoping also coincides with the decrease in strength of stripe correlations.  In a recent paper with coworkers \cite{li22}, as mentioned above, we provided evidence for inhomogeneous superconductivity in the overdoped regime, with isolated grains having lower $p$ driving the onset of superconductivity and the bulk transition resulting from proximity effect in the surrounding higher $p$ regions.  The decrease in area of lower $p$ grains, which we expect to exhibit stripe correlations, can explain the decrease in magnitude of the antiferromagnetic spin correlations.  (In the Appendix, we explain why certain published results \cite{dean13,li23} do not contradict this picture.)   Direct imaging of inhomogeneous superconductivity that develops only with overdoping in the related system (Pb,Bi)$_2$Sr$_2$CuO$_{6+\delta}$ have been provided by a recent scanning tunneling microscopy study \cite{trom23,ye23}.  There are indications of related behavior in LSCO \cite{kato08}.

%Evidence for heterogeneity is also provided by muon spin rotation ($\mu$SR) studies.  Each muon probes a local environment and are sensitive to intrinsic magnetic fields or the response to an applied field.  Assuming that the muons sample all regions uniformly, it is possible to obtain a measure of the volume fraction of distinct environments.   For example, $\mu$SR measurements on LBCO samples in zero field measure the volume fraction with static magnetic order, revealing that for samples with $x$ between 0.11 and 0.15 the magnetic volume fraction at low temperature is greater than 80\%, while it falls to $\sim50$\%\ and 40\%\ for $x=0.155$ and 0.17, respectively \cite{gugu16}.  In contrast, the low-temperature magnetic volume fraction in LSCO with $x=0.12$ is less than 50\%\ \cite{savi02,savi05}.   Direct evidence that the enhanced magnetic volume fraction in LBCO is associated with lattice symmetry is provided by measurements in which the lattice symmetry is modified by application of uniaxial stress.  Squeezing crystals of LBCO with $x=0.115$ and 0.135 along a diagonal of the square lattice causes a significant reduction of the magnetic volume fraction and raises the bulk $T_c$ to values quite similar to LSCO at comparable doping \cite{gugu24}.

%At higher temperatures, in the absence of static magnetic order, the presence of slowly fluctuating magnetic moments can be detected through the temperature dependence of the dynamic depolarization rate of muons in zero field.  Measurements on LSCO with $x$ in the range of 0.024 to 0.15 \cite{wata08} show that the onset temperature for muon depolarization is quantitatively consistent with that at which $\rho_{ab}$ reaches a minimum, before rising at low temperature in an insulating fashion when the superconductivity is suppressed by a large $c$-axis magnetic field \cite{boeb96,capr20}.   In the overdoped regime of LSCO, a $\mu$SR study demonstrated a heterogeneous response to an applied magnetic field in the normal state \cite{macd10}, consistent with intrinsic charge inhomogeneity.

%(Fig.~4) plot of magnetic volume fraction vs. x in LBCO from $\mu$SR \cite{gugu16}.  Demonstrates heterogeneity.

%Also: $\mu$SR measurements (in a transverse field in the normal state) of heterogeneous response to an applied field in seen at all dopings but especially overdoped LSCO with $x=0.25$ and 0.3 \cite{macd10}.

%Evidence for the connection between the upturn in $\rho_{ab}$ seen in high magnetic field \cite{boeb96,capr20} and stripe correlations is provided by a $\mu$SR study \cite{wata08}, in which the temperature at which the dynamic depolarization rate of muon spins begins to increase on cooling.  The doping dependence of the temperature at which $\rho_{ab}$ hits a minimum (when superconductivity is suppressed by field) and muon depolarization starts to grow are in good quantitative agreement \cite{wata08}.

%muSR in overdoped samples?

%inhomogeneity

%Mention lack of evidence for a QCP.

%Zn-doping

%In LSCO, the same low-temperature structure does not occur; however, there is evidence for a weaker version of the broken rotational symmetry.  This was identified in LSCO $x=0.07$ \cite{jaco15} and later confirmed in \lco\ \cite{sapk21}.  Because the anisotropy is weaker, the spin and charge stripe orders are also weaker, and the competition with bulk superconductivity is reduced.



%Connection between spin and charges stripes is clearly established for \lbco\ \cite{huck11}, \lnsco\ \cite{tran97a,ma21,ma22a,ma22b}, and \lesco\ \cite{lee22}.

%Periods of spin and charge order are commensurate with one another, with the spin period equal to twice the charge period.  The spin wave vector (inversely proportional to the period in real space) grows with doping $p$ until it saturates at $\sim\frac18$ for $p\gtrsim\frac18$ and above.  Stripe order occurs only in a low-temperature crystalline phase in which the Cu-O bonds in orthogonal directions within a plane become inequivalent.$

%Theoretical support for spin and charge stripe correlations \cite{frad15}.

\subsection{Model of stripe-to-metal transition in the presence of disorder}
\label{sc-model}

%The correlation length for stripe order is always quite finite.  This can be understood in terms of the intrinsic distribution of local hole densities as demonstrated by the nuclear magnetic resonance studies of Imai and coworkers  \cite{sing02a,sing05}.  Variations in hole density lead to variations in stripe spacing, which in turn limit the spin-spin or charge-charge correlation length measured by diffraction \cite{tran99a}.  In a recent study of overdoped LSCO crystals \cite{li22}, the intrinsic disorder associated with the random distribution of Sr dopants was demonstrated by scanning transmission electron microscopy.  

The stripe correlations in LSCO appear to evolve smoothly with doping, with a continuous decrease in $\langle m^2\rangle$.  To model the impact of a first-order transition from a stripe non-metal phase to a metallic phase, we consider the role of disorder.  In the absence of disorder, we assume that stripe correlations with $\delta=x$ are present throughout the CuO$_2$ planes for $x\le x_0$ and are replaced by conventional metal for $x>x_0$, as indicated by the lines in Fig.~\ref{fg:model}(b) for $x_0=0.175$. % To obtain behavior closer to experiment, we have to take account of disorder.  
%The presence of disorder in stripe correlations is evident even in the case of strongest ordering, such as \lbco\ with $x=1/8$, where diffraction measurements show that both spin and charge order diffraction peaks are broader than resolution \cite{huck11}. 

One of the earliest analyses of the role of intrinsic disorder was performed by Imai and coworkers \cite{sing02a} to understand the distribution of $^{63}$Cu nuclear spin-lattice relaxation rates measured in LSCO by nuclear quadrupole resonance.  The impact of the random distribution of Sr on the La site depends on the volume over which it is averaged.  They obtained a good fit to their data by averaging over a circular area with a radius of 3~nm.  In a later analysis of $^{17}$O nuclear magnetic resonance measurements on LSCO \cite{sing05}, they found a patch radius of 2--5~nm gave a consistent description of the data.

The relevant area (or volume) to consider is that over which interactions (including long-range Coulomb) determine the local electronic response.
A plausible choice is the coherence volume of the superconducting state. In Ref.~\onlinecite{li22}, we estimated that the coherence volume near optimal doping (with coherence length $\xi_{\rm SC}\sim 1$~nm \cite{hoff02})  corresponds to $N=40$ formula units, involving a circular area within a CuO$_2$ plane and extending over one unit cell along the $c$-axis.  If there are $k$ Sr atoms within a particular local volume, this corresponds to a local concentration $x_{\rm loc} = k/N$; the average number of Sr atoms in a unit volume is then $\lambda=\langle x_{\rm loc}\rangle N$.  Given an integer choice of $\lambda$, we can calculate the probability $P(k)$ that there are $k$ Sr atoms in a volume using the Poisson distribution \cite{pois24}:
\begin{equation}
  P(k) = {\lambda^k e^{-\lambda} \over k!}.
\end{equation}
With $N=40$, our choice of $x_0=0.175$ corresponds to $k_0=7$.  For each value of $x=\lambda/N$, we can calculate
\begin{equation}
  \langle\delta\rangle = \sum_{k=1}^{k_0}P(k){k\over N},
\end{equation}
and
\begin{equation}
  A = \sum_{k=1}^{k_0}P(k),
\end{equation}
where $A$ is the relative stripe area (or volume) for the sample.

For our choice of $N=40$, the relative stripe area and $\langle\delta\rangle$ are plotted as orange squares in Fig.~\ref{fg:model}(a) and (b), respectively. 
% For comparison, we have over-plotted the experimental results for $\delta$ in LSCO obtained from inelastic neutron scattering measurements of low-energy incommensurate spin fluctuations \cite{yama98a,zhu23} and from resonant soft x-ray scattering measurements of charge order and correlations \cite{crof14,wen19,lin20}.   
The model provides qualitative agreement with the decrease in $\langle m^2\rangle_x$ with doping, and it does a good job of describing the smooth saturation of $\delta$ vs.~$x$ at larger $x$. 
%increases through 1/8.  For the stripe area, we compare with the integrated magnetic scattering rate, which is proportional to the time and position average $\langle m^\rangle$ for local moment $m$, normalized to the case of $x=0$, as determined by neutron scattering studies \cite{hayd96a,fuji12a,zhu23,waki07b}.  We expect the Cu magnetic moments to be significant only within the area occupied by stripes, and the presence of charge stripes may reduce the amplitude relative to the stripe area.  We see in Fig.~\ref{fg:model}(a) that, though the data have large error bars, the trend is compatible with our assumptions.
We note that the relative stripe area reaches a value of $0.5$ for $x\approx p^\ast$, and the doping corresponding to maximum $T_c$ is associated with a stripe area of $\sim 2/3$.  

For striped regions to determine the normal-state transport properties, they should percolate across the CuO$_2$ planes.  Given the quasi-2D of the electronic correlations, we expect that the stripe percolation threshold should correspond to $A \approx 0.5$.  Within our picture, it follows that $p^\ast$ should correspond to the stripe percolation threshold. 

Our model assumes the presence of inhomogeneity.  We already noted that NMR studies \cite{sing02a,sing05} provided evidence of electronic heterogeneity in LSCO.  Evidence has also been provided by observations of granular superconductivity in overdoped LSCO \cite{li22}.  In that regime, metallic regions far from a stripe grain would not be superconducting.  Consistent with that, specific heat measurements show that the density of residual, unpaired electronic states grows rapidly for $x\gtrsim0.21$ \cite{wang07}.

For evidence of a mixture of striped and metallic regions, we have to consider $x<p^\ast$ and perturb the system to pin the stripe regions.  One way to do this is to change the symmetry of the lattice structure, as occurs when changing the dopants from Sr to Ba \cite{huck11}.  Zero-field $\mu$SR measurements \cite{gugu16} on \lbco\ with $0.11\le x\le0.17$ found that the magnetic volume fraction (associated with spin stripe order) was $>80$\%\ for $x\le0.15$ but rapidly fell to 50\%\ for $x=0.155$ and below that for 0.17; at the same time, there was no significant change in the local hyperfine field, which probes the spin-stripe order parameter within the ordered volume fraction.

\subsection{Evidence for the stripe percolation threshold}

%As discussed above, disorder turns a first-order transition from strongly-correlated stripes to normal metal into a crossover, in which the relative mixture of these two components changes with average doping $x$.  For properties sensitive to the spin and charge stripe correlations, we expect to see a percolation transition.  The dimensionality associated with this transition is an interesting question.  Certainly the spin correlations are quasi-2D, but the conductivity crosses over to 3D  above $p^\ast$.  We estimate that the transition should occur when the relative stripe area reaches 50\%, which, for the model results shown in Fig.~\ref{fg:model}(a), would occur for $x \sim p^\ast$.   

%(Here we emphasize percolation at $T=0$ rather than $T\sim T_c$, as considered elsewhere \cite{miha02,pelc18}.)
The role of percolation has been considered previously in terms of the intrinsic charge inhomogeneity, the associated variation in local pairing gap, and the development of superconducting phase order as a function of temperature \cite{miha02,pelc18}.  Our invocation of percolation in the present case is slighlty different, as we are concerned with the low-temperature crossover as a function of doping between local grains with stripe correlations that are associated with strong pairing and those of conventional metal in which intrinsic pairing is weak.

In the study of the percolation of antiferromagnetic order in an insulating compound where the magnetic ions are diluted by substitution of nonmagnetic ions, it is possible to determine the percolation threshold through measurements of the ordered magnetic moment and the spin-spin correlation length.  A good example is the study of the impact of the substitution of nonmagnetic Zn and Mg for Cu in \lco\ \cite{vajk02}.  In the present case, this is not practical, as the charge and spin stripe correlations are generally dynamic at low temperature and in the presence of bulk superconducting order.  The spin-spin and charge-charge correlation lengths of dynamic stripes do not provide a useful measure of the evolution of the volume fraction of stripe correlations with doping.

%We have noted that 
Application of a $c$-axis magnetic field can depress the superconductivity and enhance the strength of spin \cite{khay05,chan08,lake01,lake02,wen08b} and charge \cite{huck13,wen23} stripe orders.  %Based on measurements of the Seebeck coefficient and its temperature dependence in magnetic fields up to 34~T, Badoux {\it et al.} \cite{bado16} concluded that charge stripe ordering ended at $x = 0.15$, which is significantly below $p^\ast$.  An assumption in that work was that $p^\ast$ represents a quantum critical point, and the conclusion was that the quantum critical point cannot be due to CDW order.  We do not dispute the latter point, because we interpret $p^\ast$ as being associated with a first-order, and not a second-order, transition; however, we do point out that x-ray scattering studies in recent years have demonstrated evidence for charge-stripe correlations that extend up to an beyond $p^\ast$ \cite{wen19,miao21,vona23}.  The correlation lengths are limited for $x$ in the vicinity of $p^\ast$, so the correlations may be dynamic, but that is not a concern for our interpretation, given that we expect the volume fraction for stripe correlations to be $\sim50$\%\ near $p^\ast$.
%
There is a question, however, of how large a magnetic field is needed to suppress all superconducting correlations in LSCO and to induce stripe order.  The onset of the freezing of Cu spins can be detected by  $^{139}$La nuclear magnetic resonance, which can be performed in high fields.  Such measurements, together with high-field ultrasound measurements, indicate that the magnetic field at which the onset of spin freezing is detected rises linearly with $x$ \cite{frac20,vino22}, consistent with an extrapolation of the onset of static spin-stripe order that was detected by neutron scattering \cite{khay05,chan08}.  For $x=0.17$, the onset field for spin freezing is approximately 30~T, where the upper critical field for superconductivity is estimated to be $\sim50$~T \cite{frac20}.  The signature of the spin freezing becomes weak for $x\gtrsim0.17$; ultrasound indicates a weak onset for $x=0.188$, but the signature is absent for samples with $x\ge0.21$.  %We argue that these results are consistent with our picture of a stripe percolation transition.  
These results are consistent with our model.  Once the volume of the stripe correlations falls below the percolation value, it should not be possible to induce order even by suppressing competing orders.

%need to compare with \cite{bado16}, who argued that CDW order disappears well below $p^\ast$; inferred from transport.  But evidence for charge correlations to higher doping provided by x-ray scattering measurements \cite{wen19,miao21,vona23}.

%suppress SC in high field, NMR probes slowing of spin fluctuations associated with spin stripes \cite{frac20,vino22}.  onset field is an extrapolation of that at which the onset of static spin-stripe order was detected by neutron scattering \cite{khay05,chan08}.

%Because of the disorder, measurements of stripe correlation length by diffraction do not provide a useful measure of the spatial extent of stripe correlations in any one composition, let alone as a function of $p$.  At larger $p$, such an approach is also confounded by the absence of static stripe order.  Hence, we have to rely on less direct approaches to analyze percolation behavior.  (Perhaps note the study of percolation of AFM order in Zn- and Mg-doped \lco\ by Greven and coworkers \cite{vajk02}.)

% Figure environment removed

%To appreciate what underlies the significance of $p^*$, we note that neutron scattering experiments on LSCO and related cuprates have established that the wave vector characterizing low-energy spin-stripe correlations initially grows in proportion to doping, but saturates at $p\sim\frac18$, where the charge-stripe period is $4a$, with $a\approx3.8$~\AA\ being the Cu-Cu lattice spacing \cite{yama98a,birg06,enok13}.  The strength of the dynamic spin-stripe scattering decreases with continued doping \cite{waki04,li22}, consistent with the idea that holes added beyond $p\sim\frac18$ tend to go into uniformly-doped regions.  As already mentioned, this process happens in an inhomogeneous environment of randomly positioned dopant ions that are poorly screened \cite{tran21a,li22,spiv08}, so that at any particular average doping value, there is a distribution of local doped-hole concentrations.  Within this evolving environment, there must be a percolation threshold, below which stripe correlations are continuously connected across a CuO$_2$ plane, whereas, at large $p$, stripe correlations are restricted to finite pockets and uniformly-doped regions percolate across each plane.  (Here we emphasize percolation at $T=0$ rather than $T\sim T_c$, as considered elsewhere \cite{miha02,pelc18}.)

It is not practical to perform all measurements at such high magnetic fields.  Instead, 
%To demonstrate that the stripe percolation limit corresponds to $p^*$, 
we consider practical low-field measures of the doping dependence of spin correlations.  One is provided by the magnetic susceptibility.  For isolated spins, the susceptibility follows the Curie behavior, growing with cooling as $1/T$; however, for a 2D network of antiferromagnetically-coupled spins, the susceptibility decreases as correlations grow beyond nearest neighbors \cite{huck08}.  The magnetic susceptibility data measured on polycrystalline samples of LSCO show, in the underdoped regime, a peak at a temperature $T_{\chi}$, with a decrease at $T<T_{\chi}$ \cite{naka94}, as shown in Fig.~\ref{fg:comp}(a).  It is observed that $T_{\chi}$ drops as $p\rightarrow p^*$, consistent with the approach of an antiferromagnetic system to a percolation limit. Beyond that point, a Curie-like component develops \cite{naka94,kais12}, as one would expect in a system in which antiferromagnetic correlations are limited to finite grains \cite{bree73}.

Another probe of the spin-stripe correlations is provided by substitution of a small fraction (1\%) of Zn for Cu \footnote{Here, we are interested in a small perturbation of the planes that does not disrupt the intrinsic correlations; we explain in the Appendix why we believe that larger Zn concentrations, as studied in Ref.~\cite{risd08,adac08}, which induce magnetic order for $x\gg p^\ast$ are not relevant to this point.}, where it has been shown that a small concentration ($\lesssim1$\%) of Zn  has a similar effect on the superconductivity as a $c$-axis magnetic field \cite{nach96,loza21a}.  Neutron scattering measurements have demonstrated that Zn-doping enhances spin-stripe order \cite{kimu99,wen12a}.  The temperature $T_{\rm Zn}$ at which the spin correlations begin to freeze has been measured as a function of doping by $\mu$SR \cite{pana02}; as one can see in Fig.~\ref{fg:comp}(a), $T_{\rm Zn}$ drops toward zero at $p^*$.  (Related results have been reported for Zn-doped \ybco\ \cite{mend99,tall01}.)  This is consistent with a picture in which dynamic correlations that percolate across the sample can be pinned by Zn defects, but the pinning is no longer effective beyond the percolation threshold.

%Without Zn, a gap develops in the spin-stripe excitations at $T<T_c$ for $x \gtrsim 0.13$ \cite{chan08}.  Applying a magnetic field along the $c$ axis depresses the superconductivity and can decrease the spin gap.  Recent nuclear magnetic resonance studies using very high magnetic fields have shown that the ability to induce a quasistatic magnetic order disappears at $p\sim p^*$ \cite{frac20,vino22}, again consistent with the percolation scenario.

Local antiferromagnetic spin correlations associated with spin stripes strongly scatter the electronic excitations with wave vectors near $(\pi/a,0)$ and $(0,\pi/a)$ \cite{wu22,krie22b} that correspond to the ``antinodal'' (AN) region, where the superconducting $d$-wave gap should have its maximum.  The resulting damping causes a shift in the effective peak energy, $E_{\rm AN}$, detected by photoemission in this region \cite{ino98,sato99,yosh07}; $E_{\rm AN}$ decreases with doping, as shown in Fig.~\ref{fg:comp}(b), in a fashion that correlates with the approach to the stripe percolation limit.  The appearance of coherent AN states for $p>p^*$ has been observed in \bscco\ by scanning tunneling microscopy (STM) \cite{fuji14a} and angle-resolved photoemission spectroscopy (ARPES) \cite{droz18,chen19a}.  In LSCO, electronic dispersion for wave vectors along the $c$ axis is observed only for $p>p^*$ and only with an AN wave vector component \cite{hori18}. 

The electronic states in the AN region have a relatively flat dispersion, which, in combination with being near the Brillouin zone boundary, means that they dominate the density of states near the Fermi level \cite{zhon22}.  This also makes the AN states and their degree of coherence important for conductivity between planes, along the $c$ axis.  We have already seen in Fig.~\ref{fg:rho} that the temperature dependence of $\rho_c$ crosses from insulator-like to metallic as $x$ crosses $p^\ast$.  As a measure of the $c$-axis conductivity, we plot $1/\rho_c$, measured in the normal state at 50~K, in Fig.~1(b) \cite{li22}.  %As one can see, LSCO is essentially insulating along $c$ for $p<p^*$, and it becomes increasingly metallic as $p$ grows beyond $p^*$.    
The gradual rise of the metallic conductivity in the overdoped region is distinct from the variation of the density of states at the Fermi energy, which has a maximum at $p\sim0.21$ due to a Lifshitz transition \cite{zhon22}.%; the growth in coherent AN states occurs more gradually, and the Hall coefficient does not change sign until $p>p_c$ \cite{taka89b}.

To provide context, we summarize in Fig.~\ref{fg:comp} experimental results for several different doping-dependent properties measured on LSCO.  Figure~\ref{fg:comp}(c,d) show that a measure of the superfluid density peaks at $x\approx p^*$, which is higher than the doping at which the superconducting transition temperature, $T_c$, peaks.  The quantity plotted in Fig.~\ref{fg:comp}(c) is the inverse square of the magnetic penetration depth (at $T\ll T_c$) measured by mutual inductance on LSCO films \cite{lemb11}.  The data shown are consistent with a collection of related measures of superfluid density in LSCO summarized in Ref.~\cite{rour11}, as well as with results from $\mu$SR on other cuprates \cite{bern01}.
(The only exception involves the encapsulated LSCO thin films studied by Bozovic {\it et al.} \cite{bozo16}, where a continuous decrease of superfluid density was observed for $p>0.16$.)


\section{Discussion}
\label{sc_dis}

\subsection{First-order transition}

%We suggest that the strange-metal behavior is a consequence of inelastic scattering of quasiparticles from finite patches of stripe correlations.  
We have presented the case for a first-order transition from correlated stripes to conventional metal that is masked by disorder.
Such a picture is compatible with a recent study of in-plane optical conductivity in overdoped LSCO \cite{mich21}, where a coherent Drude component was found to grow monotonically with $p$, while the incoherent mid-infrared (MIR) component decreases in the overdoped regime, correlates with $T_c$, and is absent for $p\gtrsim p_c$.  We associate the MIR component with the regions of surviving stripe correlations; it has previously been shown that the energy scale of the MIR conductivity in cuprates matches well with that observed in two-magnon Raman scattering \cite{tran21a,suga03}, which evolves continuously from the parent antiferromagnetic insulator. 

Angle-resolved photoemission measurements on LSCO \cite{hori18,zhon22} suggest that the nominal Fermi surface undergoes a Lifshitz transition, from a hole-like to electron-like pocket, at $x\sim 0.21$.  If the system were electronically homogeneous, one would expect a corresponding change in sign of the Hall coefficient from positive to negative; however, the sign change does not occur until $p>0.3$ \cite{taka89b}.  We suggest that the intrinsic disorder we have discussed can explain this effect.  %Similar behavior occurs in LNSCO, where the Hall effect is positive to at least $p=0.24$ \cite{coll17}, while photoemission suggests that the Lifshitz transition is between $p=0.20$ and $p=0.24$ \cite{matt15}.

The model commonly used to describe cuprates is the Hubbard model \cite{arov22,qin22}.  It typically excludes the extended Coulomb interaction and limits the number of atomic orbitals per unit cell, most often to one; nevertheless, even the simplified model is challenging to evaluate, especially at low temperature. There is no consensus on a phase diagram as a function of doping for the case of intermediate coupling, relevant to cuprates, where the bandwidth is comparable to the onsite Coulomb repulsion, but there are notable results.    %There is evidence for charge and spin stripe phases that compete with various possible other states, including spatially-uniform $d$-wave superconductivity.

The temperature-dependent crossover behavior in properties such as magnetic susceptibility and $\rho_c$ associated with the onset of the pseudogap phase has been associated with the Widom line \cite{xu05w}, which terminates at zero temperature in a first-order transition \cite{sord13}.   The value of $p$ at the transition is sensitive to model parameters \cite{wu22,wals23}, but is roughly consistent with the experimental $p^\ast$.  It is recognized that the antinodal pseudogap is caused by antiferromagnetic spin fluctuations, as calculated with the dynamical cluster approximation (DCA) to dynamical mean-field theory \cite{gunn15,wu22}; in terms of spatial correlations, the spin fluctuations are associated with stripe correlations obtained in other calculations such as density-matrix renormalization group \cite{whit98a} and related approaches \cite{zhen17}, including tensor-network calculations \cite{pons23}, as well as with DCA usig larger cluster sizes \cite{mai22}.   These theoretical results are compatible with our empirical picture.

%Also, above the Lifshitz transition, if system is homogeneous, Hall coefficient should change sign, but it doesn't until $p>0.3$ \cite{taka89b}; already mentioned in discussion of variation of $c$-axis resistivity.  Demonstrates that the picture from ARPES is not associated with the net transport response, which can be understood in terms of the inhomogeneity.

%Hall effect is also positive in LNSCO to at least $p=0.24$ \cite{coll17}, but ARPES suggests Lifshitz transition is between $p=0.20$ and $p=0.24$ \cite{matt15}.


% Evidence for inhomogeneous superconductivity in the $x=0.25$ and 0.29 samples was provided recently \cite{li22}, consistent with direct evidence of such behavior in overdoped (Pb,Bi)$_2$Sr$_2$CuO$_{6+\delta}$ in STM studies \cite{trom23,ye23}.

%discuss theory: Hubbard model for Mott transition; Hubbard model alone may be a crude approximation for real cuprates.


%Where does the first-order Mott transition occur?  Sordi {\it et al.} \cite{sord13} evaluate in terms of $\rho_c(T)$.   Where the low-T upturn goes away, they identify as the transition, which they put at $p\sim 0.05$.  They call it the pseudogap phase, which they seem to distinguish from the Mott insulator.

\subsection{Strange-metal behavior in LSCO}

A common approach has been to associate strange-metal behavior with quantum critical fluctuations of some order parameter.  We have presented empirical evidence, both our own and from the literature, that the change at $p^\ast$ is associated with a first-order transition that is masked by disorder.  If we are correct, then the quantum-critical analyses are not directly relevant.   Nevertheless, models that consider antiferromagnetic fluctuations \cite{teix23} may have partial relevance.  Low-energy incommensurate spin fluctuations are certainly present at $p>p^\ast$ \cite{zhu23}, and the weight at low frequency grows on cooling toward $T_c$ \cite{waki04}; the difference is that they cannot diverge at $\hbar\omega=0$ as $T\rightarrow 0$ (when superconductivity is suppressed), because, we argue, they are confined to finite bubbles.

Patel {\it et al.} \cite{pate23} have evaluated a model with several possible contributions to electronic scattering; a $T$-linear scattering rate is given by a component describing spatially-random interactions.  The nature of the interaction is not specified, but the disorder could be consistent with our evidence for heterogeneity.

Recent calculations within the Hubbard model using determinantal quantum Monte Carlo calculations show promising results \cite{huan19}, with $T$-linear resistivity for $p\gtrsim0.2$; however, they are limited to temperatures far above room temperature.  Perhaps more significant are calculations using the dynamical cluster approximation that show, just beyond the first-order termination of the strongly-correlated pseudogap phase, a non-Fermi-liquid phase with a scattering rate that varies linearly with $T$ \cite{wu22}; here, the calculations extend down to $T$ of order 100~K.  Importantly, it is found that, in this phase, the main contribution to the electronic self-energy is from antiferromagnetic spin fluctuations \cite{wu22}.  These calculations do not explicitly include disorder, which would seem to be different from our picture.  On the other hand, an effective model of electrons scattering from fluctuating impurity magnetic moments yields the desired behavior \cite{ciuc23}.  We would view the impurity moments as representative of stripe patches.

%Calculations within the Hubbard model that give $T$-linear resistivity: \cite{huan19}. NFL beyond $p^\ast$, before FL, which identifies AFM spin fluctuations as the dominant factor \cite{wu22}
%Consistent with our picture, a recent theoretical study has found strange-metal behavior in a model of electrons scattering from fluctuating impurity magnetic moments \cite{ciuc23}, where we would associate the magnetic moments with our stripe patches. 

%Some models - not clear how to compare with experiment

%not QCP

%but low-energy spin fluctuations should be important
%(similarity to AFM QCP)

%but should be heterogeneous. (could DCA be like this?)

Some theorists have emphasized the role of defect scattering and have proposed that it is responsible for the decrease in $T_c$ with overdoping \cite{leeh17,ozde22,pal23}.  The impact of defect scattering on quasiparticles shows up in the residual resistivity, $\rho_0$.  We agree that there is substantial disorder from dopants, and we have emphasized its impact on the evolution of correlations with doping in LSCO.  We also acknowledge that there is a significant residual resistivity in overdoped samples.  A linear extrapolation of $\rho_{ab}(T)$ for our $x=0.21$ crystal yields $\rho_0\approx15$~$\mu\Omega\,$cm, compatible with high-field studies \cite{coop09,boeb96}.  Dividing by the interlayer spacing (6.6~\AA) to convert to sheet resistance, we find that $R_0\approx 10^{-2}\, h/e^2$.  For comparison, $R_0$ for an excellent 2D conductor, PdCoO$_2$, is $5\times 10^{-6}\, h/e^2$ \cite{hick12}.  

Where we differ is on the role of defect scattering as the controlling factor in limiting $T_c$ in overdoped samples.  The analysis of dirty $d$-wave superconductivity is based on weak-coupling BCS theory.  We argue that weak-coupling BCS theory is not relevant to describing superconductivity in the cuprates.  As should be clear from Fig.~\ref{fg:comp}, $T_c$ is largest for $x<p^\ast$, where, as we have discussed, the stripe correlations dominate.  In this regime, the electronic state is non-metallic as indicated by the high-field resistivity results in Fig.~\ref{fg:rsh}; the issue of residual resistivity has little relevance relative to the effects of antiferromagnetic scattering that localize the charge carriers.  Nevertheless, superconductivity develops in this environment.  We have argued that it is the loss of the strongly-correlated environment that leads to the decay of superconductivity in the very overdoped regime \cite{li22}.

%As $p$ is reduced below $p^\ast$, $R_0$ rises towards $h/e^2$ \cite{bour19,boeb96,fuku96,shi13,capr20},  consistent with the role of magnetic moments in limiting charge transport.

%issue of defect scattering (small compared to magnetic scattering: Tc highest where charge localizes in high field; what do defects matter?)
%role of defect scattering and loss of SC: what about underdoped where resistivity is insulating in the normal state?  Defect exist, but at all dopings, and correlation effects are much larger below $p^\ast$

%theoretical models for strange-metal behavior

%[strange metal: non-FL phase in \cite{wu22} from cluster DMFT - dynamic cluster approximation, DCA; does this simulate our inhomogeneous phase? write to Tremblay?]. scattering is due to AFM fluctuations.

%Hayden emphasizes spin fluctuations; we agree, but in a specific way

% A related consequence is a residual resistivity, $\rho_0$, due to orientation fluctuations of the magnetic moments.  


\subsection{Other cuprates}

Suppose our description of LSCO is approximately correct; does it have any relevance for other cuprate families?

First consider our interpretation of a first-order transition converted to a crossover at $p^\ast$.  An immediate challenge is the observation of quantum oscillations in \ybco\ \cite{doir07,seba08,seba15}.  The quantum oscillation signals, observed as a function of magnetic field at low temperature in the absence of superconducting order, are interpreted in terms of the Fermi surface of a conventional metal \cite{chak11,seba12}.  Experiments indicate that quantum oscillations are observable in the doping range $0.08\lesssim p \lesssim0.18$ \cite{seba12,rams15}, where the upper limit occurs close to the estimated $p^\ast$ \cite{tall01}.

Recent measurements indicate that the quantum oscillations come uniquely from very small pockets, presumably associated with nodal state, with no contribution coming from antinodal states \cite{hart20}.  This is consistent with measurements of $\rho_c$, which exhibit a low-temperature upturn for $p\lesssim p^\ast$ \cite{ito91,take94,coop00}, demonstrating non-metallic character.  For $p\lesssim 0.10$, insulating character within the CuO$_2$ planes was inferred from thermal conductivity measurements \cite{sun04}.  The CDW features seen in zero field show differences from LSCO \cite{hayd24}; however, antiferromagnetic spin excitations are observed across the full range \cite{dai01,hink10,suga03}, and the incommensurate wave vector of the lowest-energy spin fluctuations evolves with doping in a fashion very similar to that in LSCO \cite{dai01,enok13}.   While the normal-state character of YBCO for $p<p^\ast$ is complicated, it is clear that it is far from a conventional metal.

%first QO experiments on YBCO \cite{doir07,seba08}
%Quantum oscillations vs.\ $p$ in YBCO \cite{rams15}; extrapolated to exist for $0.08\le p \le 0.18$.  also \cite{seba12}.
%No coherent, conventional antinodal pockets \cite{hart20}

%What about other cuprates?  could our interpretation of a first-order transition converted by disorder to a crossover at $p^\ast$ apply to all hole-doped cuprates?

With respect to the role of inhomogeneity in the strange-metal behavior, we have already noted the heterogeneous superconducting gap seen in STM studies of (Pb,Bi)$_2$Sr$_2$CuO$_{6+\delta}$ \cite{trom23,ye23}.  Linear resistivity in overdoped Bi$_2$Sr$_2$CuO$_{6+\delta}$ has recently been reconfirmed \cite{zang23}.  ``Planckian'' dissipation has also been observed in overdoped \bscco\ \cite{legr19}, where angle-resolved photoemission finds a rapid decay in the antinodal superconducting gap size \cite{vall20} despite the enhanced quasiparticle coherence relative to $p<p^\ast$ \cite{chen19a}; STM shows substantial heterogeneity of local superconducting gap sizes in this system \cite{mcel05a,fang06}.

Tl$_2$Ba$_2$CuO$_{6+\delta}$ (Tl2201) is another system in which a $T$-linear contribution to $\rho_{ab}$ has been reported for overdoped samples \cite{huss13,ayre21}. %, although, when normalized by layer spacing, the associated coefficient is 4 times smaller than in LSCO \cite{ayre21}.  
Measurements suggest that it is a relatively clean system \footnote{Note that substitutional defects of Cu on the Tl site can occur at the level of 7\%\ \cite{mack93}.}: a full hole-like Fermi surface has been detected by photoemission \cite{plat05} on a crystal with $p\approx0.27$, where the Hall number corresponds to $1+p$ \cite{putz21}.  Quantum oscillations have also been observed for crystals with $p=0.27$ and 0.30 \cite{vign08,bang10}.
In contrast, $\mu$SR measurements \cite{uemu93,nied93} indicated that the ratio of the superfluid density to the effective mass decreases with $T_c$ on the overdoped side, which, assuming that the effective mass does not change, is inconsistent with weak-coupling BCS theory in the clean limit.  While scattering from defects might explain the effect, that would appear to be incompatible with the photoemission and quantum oscillation results.  This led Uemura \cite{uemu01} to argue for phase separation in overdoped cuprates between superconducting and normal-metal phases, which is compatible with our disorder model. 

Nuclear quadrupole resonance measurements in Tl2201 using $^{63}$Cu nuclei provide evidence for antiferromagnetic spin fluctuations  in the normal state that weaken with overdoping, similar to other cuprate families \cite{fuji91,kamb93};  however, the coefficient of the $T$-linear contribution to $\rho_{ab}$, when expressed in terms of sheet resistance, is only a quarter of that in LSCO for the same doping \cite{ayre21}.  Hence, the impact of spin fluctuations is probably weaker than in LSCO, enabling more coherent quasiparticles.

%in Tl2201, superfluid decreases in overdoped regime, while carrier density appears to increase (Hall effect)\cite{putz21}.  QO seen in very overdoped (any Tc?), so argued to be clean; that means reduction in superfluid density is unlikely to be explainable by defect scattering.  Also, rho0 is very small.  T-linear component in resistivity, but 4 times smaller than in LSCO.  Compatible with reduced level of spin fluctuations detected by Cu NQR?

%behavior in overdoped (Pb,Bi)$_2$Sr$_2$CuO$_{6+\delta}$ in STM studies \cite{trom23,ye23}.
%Linear resistivity in overdoped Bi2201 \cite{zang23}

%Uemura and coworkers used $\mu$SR to measure the superfluid density in overdoped Tl2201 and found that both $T_c$ and $\rho_s$ decrease \cite{uemu93}, and he later argued for phase separation, between superconducting and normal metal phases, in the overdoped regime \cite{uemu01}.  This is similar to Bozovic's result for LSCO \cite{bozo16}.

%Tl2201 Hall effect; never mind, there is no Lifshitz transition.  But data are in: \cite{putz21}.  Haase might help on NMR comparison.
%Is there c-axis resistivity data?  ADMR only for very overdoped samples (low $T_c$).

%Argument for defects destroying superfluid in overdoped regime.  Hard to see how this can explain the overdoped Tll2201 data.

%For YBCO, Ando's group reported insulating character for $x\lesssim0.55$ ($p~\sim10$\%) based on in-plane thermal conductivity \cite{sun04}.  $\rho_c$ shows a slight upturn at low temperature even for $6.93$ ($p\approx0.17$) \cite{ito91,take94,coop00}, suggesting insulating character along the $c$ axis up to $p^\ast$, similar to LSCO.


\section{Conclusions}
\label{sc_con}

The in-plane resistivity in LSCO is observed to vary approximately linear with temperature down to low temperature for hole concentrations greater than $p^\ast\sim0.19$ and up to $p_c$, the superconductor-to-metal transition.  The cause appears to be residual, spatially-inhomogeneous antiferromagnetic spin fluctuations associated with residual stripe correlations.  This behavior is different from the $T^2$ dependence expected for a Fermi liquid but modest compared to the non-metallic behavior observed for $x<p^\ast$.

Below $p^\ast$, many properties exhibit the effects of strong correlations, and in LSCO charge and spin stripes are one consequence of these.  A consistent interpretation of a range of experimental results is achieved with the assumption that there is a first-order transition between strongly-correlated and more conventional phases.  The transition appears as a crossover at $p^\ast$ due to the presence of intrinsic disorder of the dopant ions.  The stripe correlations percolate across the CuO$_2$ planes for $x < p^\ast$, while they are limited to finite puddles in the overdoped region.  Without the disorder, we expect that there would be no superconductivity or strange-metal behavior for $x>p^\ast$.

%To summarize, we have discussed evidence that the strange-metal behavior in LSCO occurs at doped-hole concentrations beyond the percolation limit for stripe correlations.  In this overdoped regime, quasiparticles can scatter from finite domains of fluctuating spin and charge stripe correlations; these patches disappear along with the superconductivity at a critical doped-hole concentration.  There are indications that this picture should apply to all hole-doped cuprates.


\begin{acknowledgments}
We thank S. A. Kivelson and A. Tsvelik for valuable comments.
Work at Brookhaven is supported by the Office of Basic Energy Sciences, Materials Sciences and Engineering Division, U.S. Department of Energy (DOE) under Contract No.\ DE-SC0012704.   
\end{acknowledgments}

\appendix*
%\null\ \ 
\section{Discussion of other work}

\subsection{Studies of samples at $p > p_c$}

There has been some confusion associated with an early resonant inelastic x-ray scattering study of magnetic excitations in LSCO thin films with $x$ as high as 0.4 \cite{dean13}.  These have been interpreted by some as evidence that antiferromagnetic spin fluctuations are still present beyond the superconducting dome.  There are a number of problems with this interpretation.  An important one is that the measurements do not come sample wave vectors close to the antiferromagnetic wave vector. This is because measurements with x-rays at the energy of the Cu $L_3$ edge cannot provide such a momentum transfer; one can only reach the antiferromagnetic zone boundary.  That means that the measurements are essentially within a ferromagnetic Brillouin zone.  When there is antiferromagnetic order, then the spin-wave dispersion is the same in both zones, though the intensity is very weak in the ferromagnetic zone.  In the absence of order, there is no such symmetry constraint.

The measurements were also performed with a coarse energy resolution.  More recent measurements \cite{meye17,roba19} with better energy resolution have shown that there is a large change in energy width of the excitations with doping and some reduction of intensity near the zone boundary.  Furthermore, theoretical calculations suggest that ferromagnetic excitations may develop at high doping \cite{jia14}, while truly antiferromagnetic fluctuations decrease substantially \cite{huan17}.  

The key point is that measurements by inelastic neutron scattering on an LSCO crystal with $x=0.30$ found that spin fluctuations near the antiferromagnetic wave vector at energies below 100 meV have negligible intensity \cite{waki04,waki07b}.  The RIXS studies do not contradict this result.

A new publication might create confusion in a different direction.  In films with $x=0.35$, 0.45, and 0.6, a diffraction peak at $(\delta,0,L)$ with $\delta\approx1/6$ has been been seen in RIXS measurements, with little intensity change up to 300~K \cite{li23}.  This peak has been interpreted as evidence of charge order.  We feel that such an interpretation is premature and questionable.  In particular, one needs further characterization of these samples and the level of apical-oxygen vacancies, which are known to be an issue for $x\gtrsim0.3$ \cite{rada94}.  A related issue has occurred in studies of NdNiO$_2$ films where incomplete oxygen reduction can result in periodic partial occupancy of apical O sites that has been detected both by electron diffraction and RIXS \cite{raji23,parz24}; the peak in RIXS can be misinterpreted as evidence of charge order.

We have sensitivity to this issue, as one of us (JMT) has his own experience with misinterpreting a set of unexpected superlattice peaks (measured by neutron diffraction in La$_2$NiO$_{4+\delta}$) in terms of CDW order \cite{tran93}.  Further investigation revealed that they were a consequence of the ordering of interstitial oxygens \cite{tran94b}.

\subsection{Impact of larger concentrations of Zn dopants}

In considering the impact of Zn doping on spin freezing, as in the data of Ref.~\onlinecite{pana02} shown in Fig.~\ref{fg:comp}(a), we are interested in considering the Zn as a minimal perturbation that depresses the superconductivity without significantly modifying the intrinsic stripe correlations.  In fact, Zn locally acts as a strong perturbation, preventing the motion of holes through the Zn site, which can limit the flipping of spins on neighboring Cu sites.  From the $\mu$SR study of Nachumi {\it et al.} \cite{nach96}, where the impact of Zn was studied in steps of 0.25\%\ in impurity concentration, 1\%\ Zn is already a significant perturbation.  A study combining $\mu$SR and neutron diffraction showed that larger Zn concentrations can depress spin-stripe and superconducting transitions in a similar fashion \cite{gugu17}. 

Studies by Adachi {\it et al.} \cite{adac08} and Risdiana {\it et al.} \cite{risd08} showed that one can enhance the magnetic volume fraction measured by $\mu$SR at $T<1$~K by adding more Zn.  For example, with 3\%\ Zn in LSCO, it was possible to detect some local magnetism for doping up to $x=0.27$ \cite{risd08}.   We suggest that 3\%\ Zn is a very large perturbation that significantly modifies the intrinsic correlations, and hence those results do not provide a useful measure of the unperturbed volume fraction of stripe correlaions.

%discuss \cite{adac08}; compare with $\mu$SR study \cite{nach96}, which shows that 1\%\ Zn is already showing more than a small perturbation.  The issue is not whether we can induce magnetism, but whether we can detect the intrinsic correlations with a small perturbation.

%%%%%%%%%%%%%%%%%%%%%%%%%
%Notes:

%evidence for patchy-ness of stripes above $p^\ast$ and coexistence with uniform carriers (Ref A):

%Imai's NMR work provides evidence for the spread in local  magnetic response and hole concentration \cite{sing02a,sing05}; mentions overdoped results, but not plotted.  (Does Haase have anything relevant on this?)


%Need to cite Stephen Hayden's recent LSCO paper \cite{zhu23}.

%Hayden argues that low-energy spin fluctuations are important for strange metal behavior ($x=0.22$).  We agree, but argue that they are not homogeneous.  He compares with Sr$_3$Ru$_2$O$_7$ in $B\sim 8$~T.   

%Should we mention heat capacity (Girod, 2021)?  and Hayden's discussion?

\bibliography{LNO,theory}

\end{document}

ADMR and Planckian scattering rate in LNSCO $p=0.24$, with strongest in-plane scattering in AN direction but Planckian part being isotropic in-plane \cite{gris21} $p=0.21$ \cite{fang22}
%Planckian scattering in LNSCO and LSCO \cite{atae22}


