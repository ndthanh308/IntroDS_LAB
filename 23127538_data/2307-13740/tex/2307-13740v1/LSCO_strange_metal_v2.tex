
\documentclass[aps,prl,reprint,superscriptaddress,floatfix]{revtex4-2}

%\usepackage{amsmath}    % need for subequations
\usepackage{amssymb}
\usepackage{graphicx}   % need for figures
%\usepackage[lofdepth,lotdepth, caption=false]{subfig}
%\usepackage{verbatim}   % useful for program listings
\usepackage{color}      % use if color is used in text
\usepackage{hyperref}   % use for hypertext links, including those to external documents and URLs
%\usepackage{dcolumn}

\def\lno{La$_2$NiO$_4$}
\def\lnod{La$_2$NiO$_{4+\delta}$}
\def\lsno{La$_{2-x}$Sr$_x$NiO$_4$}
\def\lsco{La$_{2-x}$Sr$_x$CuO$_4$}
\def\lbco{La$_{2-x}$Ba$_x$CuO$_4$}
\def\lbcoate{La$_{1.875}$Ba$_{0.125}$CuO$_4$}
\def\lnsco{La$_{1.6-x}$Nd$_{0.4}$Sr$_x$CuO$_4$}
\def\lesco{La$_{1.8-x}$Eu$_{0.2}$Sr$_x$CuO$_4$}
\def\ybco{YBa$_2$Cu$_3$O$_{6+x}$}
\def\lcod{La$_2$CuO$_{4+\delta}$}
\def\bscco{Bi$_2$Sr$_2$CaCu$_2$O$_{8+\delta}$}
\def\hbco{HgBa$_2$CuO$_{4+\delta}$}
\def\ncco{Nd$_{2-x}$Ce$_x$CuO$_4$}

\newcommand{\LSNO}{La$_{2-x}$Sr$_{x}$NiO$_{4}$}
\newcommand{\LSNOn}{La$_{1.75}$Sr$_{0.25}$NiO$_{4}$}
%
\def\newr{\color{red}}
%\def\newr{\color{black}}
\def\newb{\color{blue}}
%\def\newb{\color{black}}

\begin{document}

\title{Strange-metal behavior in \lsco\ beyond the stripe-percolation transition}

\author{P. M. Lozano}
\altaffiliation{Current address: Advanced Photon Source, Argonne National Laboratory, Argonne, Illinois 60439, USA}
\affiliation{Condensed Matter Physics and Materials Science Division, Brookhaven National Laboratory, Upton, New York 11973-5000, USA}
\affiliation{Department of Physics and Astronomy, Stony Brook University, Stony Brook, NY 11794-3800, USA}
\author{G. D. Gu}
\affiliation{Condensed Matter Physics and Materials Science Division, Brookhaven National Laboratory, Upton, New York 11973-5000, USA}
\author{Qiang Li}
\email{liqiang@bnl.gov}
\affiliation{Condensed Matter Physics and Materials Science Division, Brookhaven National Laboratory, Upton, New York 11973-5000, USA}
\affiliation{Department of Physics and Astronomy, Stony Brook University, Stony Brook, NY 11794-3800, USA}
\author{J. M. Tranquada}
\email{jtran@bnl.gov}
\affiliation{Condensed Matter Physics and Materials Science Division, Brookhaven National Laboratory, Upton, New York 11973-5000, USA}


\date{\today} 

\begin{abstract}
The nature of the normal state of cuprate superconductors continues to stimulate considerable speculation.  Of particular interest has been the linear temperature dependence of the in-plane resistivity in the low-temperature limit, which violates the prediction for a Fermi liquid.  We present evidence that the behavior is a consequence of an inhomogeneous mixture of quasiparticles and strongly-correlated patches of spin and charge stripes.  In the case of \lsco, the strange-metal behavior is observed only for doped-hole concentrations that exceed the percolation threshold for stripe correlations,  where we observe that the resistivity measured perpendicular to the planes develops a metallic temperature dependence.
\end{abstract}

\maketitle


The hole-doped cuprate superconductors have a variety of intriguing properties that deviate from conventional behavior.  For example, metals described by the Fermi-liquid model exhibit a resistivity that varies as the square of the temperature $T$ in the low-$T$ limit. In contrast, it was observed early on that the in-plane resistivity, $\rho_{ab}$, for cuprates at particular compositions varies as $T$ \cite{gurv87,mart90}.  Since the significance of this strange metal behavior was first recognized \cite{varm89,ande92}, it has continued to receive considerable attention \cite{phil22,varm20b,hart22,zaan19}.  In cuprates where the dopant-induced hole density $p$ is low, the normal-state $\rho_{ab}$ is limited by pseudogap behavior below a temperature $T^*$ \cite{keim15}, but this scale tends to zero at $p^*\sim0.19$ \cite{tall01}.  Experiments initially associated the $T$-linear resistivity with proximity of doping to $p^*$ \cite{daou09b,coop09}, which has often been identified as a quantum critical point \cite{sach03,norm05,varm20b}.  While it has been tempting to associate it with quantum critical fluctuations \cite{tail10,doir12,varm20b}, such a perspective is constrained by the difficulty in identifying a universal order parameter for the cuprates that disappears at $p^*$ due to quantum fluctuations.  

Recent experiments have made clear that the strange-metal behavior is present across the regime of overdoped superconductivity, corresponding to $p^*< p<p_c$ \cite{huss11,legr19,ayre21}, where $p_c\sim0.3$ is the quantum critical point corresponding to the superconductor-to-metal transition \cite{spiv08}.  These results have motivated phenomenological models that consider a distribution of random interactions \cite{pate19,chow22} or electronic scattering from fluctuating charge-density-wave (CDW) \cite{seib21,capr22,bagg23}, antiferromagnetic (AFM) \cite{emer93,rice17,lee21c,ma23}, or pair-density-wave \cite{bane21} correlations.

In this Letter, we use the system \lsco\ (LSCO, where $p=x$) as an example and present the case that $p^*$ does not correspond to a critical point for fluctuation disorder of a spatially-uniform order parameter, but rather the percolation limit for spin and charge stripe correlations.  The strong correlations that lead to stripe modulations also control the evolution of the mobile hole density and the pairing interaction.  Of course, these correlations evolve within a spatially inhomogeneous environment due to the random distribution of poorly-screened dopant ions \cite{tran21a,li22,spiv08}, which means that there can be a mixture of stripe-correlated patches and uniformly-doped regions.  For $p<p^*$, where stripe correlations percolate across the sample, the conductivity of quasiparticles in the uniformly-doped regions is constrained by scattering from antiferromagnetically-correlated Cu moments within spin stripes.  This constraint is removed for $p>p^*$; however, stripe-correlated patches survive and continue to impact the normal-state transport, which is reflected in both the $p$ and $T$ dependence of the $c$-axis resistivity, $\rho_c$, as we will demonstrate.  In this picture, the strange-metal behavior occurs in the presence of domains of fluctuating CDW and AFM correlations, consistent with the two charge-carrier components inferred by Ayres {\it et al.} for a variety of cuprates \cite{ayre22} and  with recent models \cite{seib21,capr22,bagg23,rice17,lee21c,ma23}, especially including recent numerical studies of the Hubbard model \cite{huan19,huan23,wu22}.

To provide context, we summarize in Fig.~\ref{fg:comp} experimental results for several different doping-dependent properties measured on LSCO.  Figure~\ref{fg:comp}(c,d) show that a measure of the superfluid density peaks at $x\approx p^*$, which is higher than the doping at which the superconducting transition temperature, $T_c$, peaks.  The quantity plotted in Fig.~\ref{fg:comp}(c) is the inverse square of the magnetic penetration depth (at $T\ll T_c$) measured by mutual inductance on LSCO films \cite{lemb11}.  The data shown are consistent with a collection of related measures of superfluid density in LSCO summarized in Ref.~\cite{rour11}, as well as with results from muon spin rotation spectroscopy on other cuprates \cite{bern01}.
(The only exception involves the encapsulated LSCO thin films studied by Bozovic {\it et al.} \cite{bozo16}, where a continuous decrease of superfluid density was observed for $p>0.16$.)

% Figure environment removed

To appreciate what underlies the significance of $p^*$, we note that neutron scattering experiments on LSCO and related cuprates have established that the wave vector characterizing low-energy spin-stripe correlations initially grows in proportion to doping, but saturates at $p\sim\frac18$, where the charge-stripe period is $4a$, with $a\approx3.8$~\AA\ being the Cu-Cu lattice spacing \cite{yama98a,birg06,enok13}.  The strength of the dynamic spin-stripe scattering decreases with continued doping \cite{waki04,li22}, consistent with the idea that holes added beyond $p\sim\frac18$ tend to go into uniformly-doped regions.  As already mentioned, this process happens in an inhomogeneous environment of randomly positioned dopant ions that are poorly screened \cite{tran21a,li22,spiv08}, so that at any particular average doping value, there is a distribution of local doped-hole concentrations.  Within this evolving environment, there must be a percolation threshold, below which stripe correlations are continuously connected across a CuO$_2$ plane, whereas, at large $p$, stripe correlations are restricted to finite pockets and uniformly-doped regions percolate across each plane.  (Here we emphasize percolation at $T=0$ rather than $T\sim T_c$, as considered elsewhere \cite{miha02,pelc18}.)

To demonstrate that the stripe percolation limit corresponds to $p^*$, we consider practical measures of the doping dependence of spin correlations.  One is provided by the magnetic susceptibility.  For isolated spins, the susceptibility follows the Curie behavior, growing with cooling as $1/T$; however, for a network of antiferromagnetically-coupled spins, the susceptibility decreases as correlations grow beyond nearest neighbors \cite{huck08}.  The magnetic susceptibility data measured on polycrystalline samples of LSCO show, in the underdoped regime, a peak at a temperature $T_{\chi}$, with a decrease at $T<T_{\chi}$ \cite{naka94}, as shown in Fig.~\ref{fg:comp}(a).  It is observed that $T_{\chi}$ drops as $p\rightarrow p^*$, consistent with the approach of an antiferromagnetic system to a percolation limit. Beyond that point, a Curie-like component develops \cite{naka94,kais12}, as one would expect in a system in which antiferromagnetic correlations are limited to finite grains \cite{bree73}.

Another probe of the spin-stripe correlations is provided by substitution of a small fraction (1\%) of Zn for Cu.  Neutron scattering measurements have demonstrated that Zn-doping enhances spin-stripe order \cite{kimu99,wen12a}.  The temperature $T_{\rm Zn}$ at which the spin correlations begin to freeze has been measured as a function of doping by muon spin-rotation spectroscopy \cite{pana02}; as one can see in Fig.~\ref{fg:comp}(a), $T_{\rm Zn}$ drops toward zero at $p^*$.  (Related results have been reported for Zn-doped \ybco\ \cite{mend99,tall01}.)  This is consistent with a picture in which dynamic correlations that percolate across the sample can be pinned by Zn defects, but the pinning is no longer effective beyond the percolation threshold.

Without Zn, a gap develops in the spin-stripe excitations at $T<T_c$ for $x \gtrsim 0.13$ \cite{chan08}.  Applying a magnetic field along the $c$ axis depresses the superconductivity and can decrease the spin gap.  Recent nuclear magnetic resonance studies using very high magnetic fields have shown that the ability to induce a quasistatic magnetic order disappears at $p\sim p^*$ \cite{frac20,vino22}, again consistent with the percolation scenario.

% Figure environment removed

Local antiferromagnetic spin correlations associated with spin stripes strongly scatter the electronic excitations with wave vectors near $(\pi/a,0)$ and $(0,\pi/a)$ \cite{wu22,krie22b} that correspond to the ``antinodal'' (AN) region, where the superconducting $d$-wave gap (in principle) has its maximum.  The resulting damping causes a shift in the effective peak energy, $E_{\rm AN}$, detected by photoemission in this region \cite{ino98,sato99,yosh07}; $E_{\rm AN}$ decreases with doping, as shown in Fig.~\ref{fg:comp}(b), in a fashion that correlates with the approach to the stripe percolation limit.  The appearance of coherent AN states for $p>p^*$ has been observed in \bscco\ by scanning tunneling microscopy (STM) \cite{fuji14a} and angle-resolved photoemission spectroscopy (ARPES) \cite{droz18,chen19a}.  In LSCO, electronic dispersion for wave vectors along the $c$ axis is observed only for $p>p^*$ and only with an AN wave vector component \cite{hori18}. 

The electronic states in the AN region have a relatively flat dispersion, which, in combination with being near the Brillouin zone boundary, means that they dominate the density of states near the Fermi level \cite{zhon22}.  This also makes the AN states and their degree of coherence important for conductivity between planes, along the $c$ axis.  To explore this, resistivity measurements were performed on square-plate-shaped bulk single crystals grown by the floating-zone technique with four different doping values: $x=0.17$, 0.21, 0.25, and 0.29. Previous characterizations and descriptions of the crystal growth have been presented in Refs.~\cite{li18,li22}. Samples were oriented and cut for both in-plane and out-of-plane resistivity measurements for each doping level. The electrical contacts were done in a four-point in-line configuration, and the measurements were performed in a Quantum Design Physical Property Measurement System equipped with a 14 T superconducting magnet.

As a measure of the $c$-axis conductivity, we plot $1/\rho_c$, measured in the normal state at 50~K, in Fig.~1(b) \cite{li22}.  As one can see, LSCO is essentially insulating along $c$ for $p<p^*$, and it becomes increasingly metallic as $p$ grows beyond $p^*$.    This gradual change in the conductivity is distinct from the variation of the density of states at the Fermi energy, which has a maximum at $p\sim0.21$ due to a Lifshitz transition \cite{zhon22}; the growth in coherent AN states occurs more gradually, and the Hall coefficient does not change sign until $p>p_c$ \cite{taka89b}.

Those results suggest that we take a closer look at the connections between $\rho_c$ and $\rho_{ab}$.  In Fig.~\ref{fg:rho}, we compare the temperature dependence of these quantities for four compositions of LSCO.  
Besides the large change in magnitude, there is a substantial change in the temperature dependence of $\rho_c$ between $x=0.17$ and 0.21 that correlates with the change in $\rho_{ab}$.  Looking at $\rho_c$ for $x=0.17$ on cooling, there is a kink below 250~K corresponding with the structural transition from tetragonal to orthorhombic symmetry, and then there is a rise below 80~K of insulating character.  While $\rho_{ab}(x=0.17)$ has a metallic temperature dependence, the upward curvature on cooling roughly correlates with the changes in $\rho_c$; furthermore, depressing the superconducting transition by applying a 14~T magnetic field along the $c$ axis causes a flattening of $\rho_{ab}$ in the extended normal-state region, which has been associated with the slowing of magnetic fluctuations \cite{bour19} as expected for $x<p^*$.

% Figure environment removed

These behaviors contrast with those of the crystals with $x>p^*$, where we see that $\rho_c$ exhibits a metallic $T$ dependence in each case, with a magnitude that continuously decreases with $x$.  For $\rho_{ab}$, the insets indicate that depressing $T_c$ with a magnetic field results in a $T$-linear extension of the normal-state behavior to lower temperature.  To emphasize the strange-metal character of $\rho_{ab}$ for these compositions, we plot $d\rho_{ab}/dT$ in Fig.~\ref{fg:drho} .  If $\rho_{ab}$ varies linearly with $T$, then $d\rho_{ab}/dT$ should be a constant, which is approximately the case for $x=0.21$.  There is some $T$-dependent slope that appears for $x=0.25$ and 0.29, but this is small compared to the case of $x=0.17$.  Our results are consistent with a large deviation from Fermi-liquid  behavior in LSCO for $p^*<x<p_c$, as previously identified \cite{coop09,ayre21}.  

We suggest that the strange-metal behavior is a consequence of inelastic scattering of quasiparticles from finite patches of stripe correlations.  Such a picture is compatible with a recent study of in-plane optical conductivity in overdoped LSCO \cite{mich21}, where a coherent Drude component was found to grow monotonically with $p$, while the incoherent mid-infrared (MIR) component decreases in the overdoped regime, correlates with $T_c$, and is absent for $p\gtrsim p_c$.  We associate the MIR component with the regions of surviving stripe correlations; it has previously been shown that the energy scale of the MIR conductivity in cuprates matches well with that observed in two-magnon Raman scattering \cite{tran21a,suga03}.  Evidence for inhomogeneous superconductivity in the $x=0.25$ and 0.29 samples was provided recently \cite{li22}, consistent with direct evidence of such behavior in overdoped (Pb,Bi)$_2$Sr$_2$CuO$_{6+\delta}$ in STM studies \cite{trom23,ye23}.

Consistent with our picture, a recent theoretical study has found strange-metal behavior in a model of electrons scattering from fluctuating magnetic moments \cite{ciuc23}, where we would associate the magnetic moments with our stripe patches.  A related consequence is a residual resistivity, $\rho_0$, due to orientation fluctuations of the magnetic moments.  A linear extrapolation of $\rho_{ab}(T)$ for our $x=0.21$ crystal yields $\rho_0\approx15$~$\mu\Omega\,$cm, compatible with high-field studies \cite{coop09,boeb96}.  Dividing by the interlayer spacing (6.6~\AA) to convert to sheet resistance, we find that $R_0\approx 10^{-2}\, h/e^2$, where $h/e^2$ is the Mott-Ioffe-Regel quantum of resistance.  For comparison, $R_0$ for an excellent 2D conductor, PdCoO$_2$, is $5\times 10^{-6}\, h/e^2$ \cite{hick12}.  As $p$ is reduced below $p^\ast$, $R_0$ rises towards $h/e^2$ \cite{bour19,boeb96,fuku96,shi13,capr20},  consistent with the role of magnetic moments in limiting charge transport.

To summarize, we have discussed evidence that the strange-metal behavior in LSCO occurs at doped-hole concentrations beyond the percolation limit for stripe correlations.  In this overdoped regime, quasiparticles can scatter from finite domains of fluctuating spin and charge stripe correlations; these patches disappear along with the superconductivity at a critical doped-hole concentration.  There are indications that this picture should apply to all hole-doped cuprates.


{\it Acknowledgments}.
We thank S. A. Kivelson and A. Tsvelik for valuable comments.
Work at Brookhaven is supported by the Office of Basic Energy Sciences, Materials Sciences and Engineering Division, U.S. Department of Energy (DOE) under Contract No.\ DE-SC0012704.   

\bibliography{LNO,theory}

\end{document}

ADMR and Planckian scattering rate in LNSCO $p=0.24$, with strongest in-plane scattering in AN direction but Planckian part being isotropic in-plane \cite{gris21} $p=0.21$ \cite{fang22}
Planckian scattering in LNSCO and LSCO \cite{atae22}
Planckian dissipation in overdoped cuprates \cite{legr19}
Linear resistivity in overdoped Bi2201 \cite{zang23}

