\section{Discussion}
\label{sec:discussion}

\input{figures/figure_complexity.tex}
\subsection{Effectiveness of Considering the Interaction Representation}


The Graph convolution Siamese network~\cite{liang2017similarity} provides a basic framework to deal with pairwise graph inputs by predicting the interaction from individually embedded inputs (i.e., $I(G_\mathcal{A})$ and $I(G_\mathcal{B})$).
However, as the interaction is only minimally reflected in the final prediction layer, previous works have extended the scope to leverage interaction representations (i.e., $I_n(G_\mathcal{A},G_\mathcal{B})$) during the embedding stage.
\textit{MHCADDI} exchanges node representations between two graphs in the message-passing step, while \textit{SimGNN} compares all node pairs of two graphs after the node embedding layers.
Since \textit{MHCADDI} and \textit{SimGNN} show a substantial gain in predictive power compared to simple Siamese networks (see Table \ref{tab:AUROC} and Table \ref{tab:GED_results}), it can be concluded that utilizing $I(G_\mathcal{A},G_\mathcal{A})$ in the embedding stage is crucial when dealing with pairwise graphs.
However, as these methods utilize all node pairs, unnecessary complexity and redundant information are produced.
This results in suboptimal performance and efficiency, thereby compromising the final prediction.
Our CAGPool focuses on subgraphs that are extracted by the interaction representation, mitigating the issue of noisy representations.
The experimental results in Table \ref{tab:AUROC} and Table \ref{tab:GED_results} show that CAGPool successfully utilizes the interaction representations at the graph-level $I_g(G_\mathcal{A},G_\mathcal{A})$, achieving state-of-the-art performance on both tasks.


\subsection{The Efficiency of Co-attention Graph Pooling}
\label{sec:complexity}
Figure \ref{fig:fig_complexity} illustrates the complexity of each method that treats pairwise graph input.
The Graph convolution Siamese network, being the most basic form, has an overall complexity bounded by the complexity of embedding each graph with GNNs: $O(|E_\mathcal{A}| + |E_\mathcal{B}|)$.
Recent methods, such as \textit{MHCADDI} and \textit{SimGNN}, construct interaction representations at the node-level and require pairwise calculations for every node pair between $V_\mathcal{A}$ and $V_\mathcal{B}$.
Therefore, the computational complexity is bounded by $O(|V_\mathcal{A}||V_\mathcal{B}|)$.
Strictly speaking, \textit{SimGNN} has a complexity of $O(\mbox{max}(|V_\mathcal{A}|,|V_\mathcal{B}|)^2)$.
On the other hand, our approach requires additional computation with a complexity of $O(|V_\mathcal{A}| + |V_\mathcal{B}|)$ for constructing the interaction representation at the graph-level and selecting the nodes.
Therefore, CAGPool maintains the $O(|E_\mathcal{A}| + |E_\mathcal{B}|)$ complexity of the simple graph convolution Siamese network without increasing the complexity upper bound.
Additionally, we only need $W_\alpha \in \mathbb{R}^{2nF' \times 2nF'}$ and $b_\alpha \in \mathbb{R}^{2nF'}$ (see Equation (\ref{eq:alpha})) as additional trainable parameters, where $n$ is the number of GCN layers and $F'$ is the hidden dimension.
Given inputs $X_A$ and $X_B$, our module produces $X_A’$ and $X_B’$, demonstrating a 31.2 $\sim$ 64.7\% faster running time than the node-level interaction module when we set the number of nodes from 50 to 200.
We describe the details in the supplementary material.

\subsection{Ablation Study}
\section{Ablation study on YCBV}
\label{sec:ablation_ycbv}

In Tab.~\ref{tab:ablation_ycbv} we report the results of our ablation study on YCBV~\cite{ycbv}.
We choose the Large Marker object and train a single model on it for each modification we applied.
Each model is trained for 20 epochs on the standard training set.
For the computation of the Feature Matching Recall (FMR), we set the distance threshold $\tau_1=10$ voxels and the inlier ratio threshold $\tau_2=5$\%, to account for the different density of the scene point cloud in YCBV.
All the other settings and parameters are the same as those in our ablation study on LMO~\cite{lmo} in the main paper.

We can observe that some changes do not increment performance, but instead cause a slight drop, in particular when adapting the safety threshold to the object dimension (third row, $-0.4$) and when colour augmentation is applied (sixth row, $-$0.3).
These additions do not benefit this particular object, but are instead advantageous when averaging all the object in the dataset.

We can note that, as in the ablation study on the LMO dataset in the main paper, the most significant improvements in ADD-S AUC result from applying the safety threshold ($+$1.5), adding RGB information ($+$5.5), and using the Adam optimiser ($+$12.3).
\renewcommand{\arraystretch}{0.9}
\begin{table}%[t!]
\centering
\tabcolsep 3pt
\caption{
Ablation study on the Large Marker object of YCBV.
Performance are compared in terms of RRE [radiants] and RTE [cm] errors (the lower the better), and FMR and ADD-S AUC (shortened to ADD) scores (the higher the better).
$\Delta$ shows the improvement of each contribution in terms of ADD-S AUC with respect to the previous row.
}
\vspace{-3mm}
\resizebox{\columnwidth}{!}{%
\begin{tabular}{clrrrrr}
\toprule
& Improvements &
RRE{\color{black!50}{$\,\downarrow$}} &
RTE{\color{black!50}{$\,\downarrow$}} & 
FMR{\color{black!50}{$\,\uparrow$}} & 
ADD{\color{black!50}{$\,\uparrow$}} & 
$\Delta$ \\ 
\toprule
& Baseline & 2.0 & 4.6 & 0.00 & 77.2 & -- \\
\midrule
\multirow{2}{*}{\rotatebox{90}{Loss}} & $+$ $\tau_{NS} = 0.1 D_S$ & 2.0 & 4.2 & 0.00 & 78.7 & $+$1.5 \\
& $+$ $\tau_{NS} = 0.1 D_O$ & 2.0 & 4.3 & 0.00 & 78.3 & $-$0.4 \\
\midrule
\multirow{2}{*}{\rotatebox{90}{Arch.}} & $+$ Independent weights & 2.0 & 4.1 & 0.00 & 79.4 & $+$1.1 \\
& $+$ Add RGB information & 1.2 & 3.2 & 49.1 & 84.9 & $+$5.5 \\
\midrule
\multirow{2}{*}{\rotatebox{90}{Aug.}} & $+$ Color augmentation & 1.2 & 3.3 & 50.0 & 84.6 & $-$0.3 \\
& $+$ Random erasing & 1.2 & 3.1 & 53.4 & 85.2 & $+$0.6 \\
\midrule
\multirow{2}{*}{\rotatebox{90}{Optim.}} & $+$ SGD $\to$ Adam & 0.0 & 0.4 & 100 & 97.5 & $+$12.3 \\
& $+$ Adam $\to$ AdamW  & 0.0 & 0.4 & 100 & 97.5 & 0 \\
& $+$ Exp $\to$ Cosine & 0.0 & 0.4 & 100 & 97.4 & $-$0.1 \\\bottomrule
\end{tabular}}
\label{tab:ablation_ycbv}
\end{table}
\renewcommand{\arraystretch}{1}

\section{Additional ablation study on LMO}

We include an ablation study on the $t_\text{scale}$ hyperparameter, which is used to set the radius of the ball volume in which negative mining around a certain point is not allowed. We train on the Can object of LMO using the standard setting, and varying only $t_\text{scale}$. The results are shown in Tab.~\ref{tab:ablation_ycbv}.
We can observe that our choice of $t_\text{scale} = 0.1$ leads to the best result. When $t_\text{scale}$ is increased, many candidate points are forbidden to be used as negatives, therefore decreasing the final performance. On the other hand, a lower $t_\text{scale}$ implies negative pairs composed by points which are near in the 3D space. This reduces the performance, as similar points are forced to have different descriptors. Notably, the worst results is obtained when $t_\text{scale} = 0.1$, i.e. when no negative candidates are excluded.

\begin{table}
\tabcolsep 3pt
\caption{
Ablation study on the Can object of LMO. Performance is shown in terms of ADD-0.1 (the higher the better) in function of the hyperparameter $t_\text{scale}$.}
\centering
\resizebox{.9\columnwidth}{!}{
\begin{tabular}{c|ccccc}
    \toprule
    $t_\text{scale}$ & 0.0 & 0.01 & 0.05 & \textbf{0.1} & 0.5 \\
    ADD-0.1d & 66.55 & 91.80 & 93.79 & \textbf{93.95} & 81.28 \\
    \bottomrule
\end{tabular}
\label{tab:tscale}
}
\end{table}

We conduct an ablation study to validate (1) whether co-attention serves as an effective component when treating pairwise graphs, and (2) whether CAGPool demonstrates a meaningful improvement compared to individual pooling without considering interaction representation.

For (1), it can be concluded that leveraging co-attention is effective by comparing \textit{MHCADDI-ML} and CAGPool with \textit{MPNN-Concat} (the vanilla Siamese Network architecture in Figure~\ref{fig:fig_complexity}).
Moreover, CAGPool shows a significant improvement in performance even when compared to \textit{MHCADDI-ML}, implying that CAGPool is a more effective way to utilize the co-attention mechanism.

For (2), we only consider hierarchical pooling methods with node selection to match our experimental settings.
For a fair comparison with other hierarchical node selection-based pooling methods, we implemented TopKPool~\cite{gao2019graph} and SAGPool~\cite{pmlr-v97-lee19c} and kept the pooling ratio at 50\%.
Although all pooling methods show a gain in performance, our co-attention-based approach serves as the most effective pooling method for pairwise graph prediction.

\subsection{Future Works}

\textbf{Studies on the extracted subgraphs.}
The main focus of our work is the proposal of a novel pooling method, CAGPool, for extracting subgraphs from graph pairs, which we refer to as $G'_\mathcal{A} = (V'_\mathcal{A}, E'_\mathcal{A})$ and $G'_\mathcal{B} = (V'_\mathcal{B}, E'_\mathcal{B})$, that can help predict labels such as drug-drug interactions.
Even if some nodes, $v \in V'$, are isolated, they contain important information due to the use of GCN layers.
In the context of predicting drug-drug interactions, it can be extremely difficult to disambiguate the functional groups (i.e., subgraphs) related to a specific side effect, even for experts in the biomedical field.
This is because these subgraphs do not directly interact with each other, but rather affect each other through complex biological pathways within the human body.
While our research did not specifically investigate this analysis, we expect that our method can facilitate future research in this area by identifying subgraphs that are likely to be related to the functional groups responsible for the side effects between drugs.


