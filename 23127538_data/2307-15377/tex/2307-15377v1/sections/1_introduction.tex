\section{Introduction}
\label{sec:introduction}
\PARstart{R}{ecent} advancements in both aggregation ~\cite{kipf2017semi,velickovic2017graph,gcnii_icml20,pna_neurips20,magna_ijcai21} and pooling operations~\cite{ying2018hierarchical,pmlr-v97-lee19c,Yuan2020StructPool,baek2021accurate,muchpool_ijcai21,sep_icml22,https://doi.org/10.48550/arxiv.2209.02939} have significantly improved the capabilities of Graph Neural Networks (GNNs), enabling for more robust learning of complex graph representations and enhancing performance in downstream tasks like graph classification, node classification, and link prediction.
However, the scope of these approaches are limited on a single graph input while many real-world tasks (e.g., scene graph matching, code search, and drug-drug interaction prediction) require pair-wise analysis of graph-structures.
Therefore, recent studies in GNNs have shifted their focus to representation learning over pairs of input graphs.

One of the earliest approaches for paired graph representation learning using GNNs is the graph convolutional Siamese network~\cite{ktena2017distance}.
In Graph Convolutional Siamese network, the input pairs must share identical graph topology, and the paired training is done by simply concatenating the individual graph representations.
However, since it does not take the interaction between the graphs during the embedding process into account, each graph is embedded into a single \textit{static} representation regardless of its pair.
This static representation can limit the expressiveness of the many-to-many relationships between pairwise graphs~\cite{deac2020empowering}.
For instance, when predicting interactions between chemical compounds, each molecular graph can have multiple functional groups, which are important sub-graphs for the task. 
Since the contribution of each functional group to the interaction depends on its pair, representing molecules with a single static representation may be limited in terms of expressiveness. 
To overcome this limitation, it is necessary to consider the interaction between the input pair of graphs.

Subsequent works~\cite{li2019graph,bai2019simgnn,deac2020empowering} proposed architectures considering the interaction between the input pair using the co-attention mechanism, which is an intuitive way to contemplate pairwise interactions.
While co-attention has improved the predictive power of these methods, they are still limited to obtaining the interaction representation at the \textit{node-level}. 
This not only leads to increased complexity but also generates redundant output as every node pair of the two input graphs must be considered.

\input{figures/figure_intro.tex}

In this paper, we propose an efficient method for considering interactions between graphs at the \textit{graph-level} by applying co-attention to graph pooling. 
Our Co-attention Graph Pooling (CAGPool) dynamically represents each input graph based on its interaction with the opposite graph in the pair (as illustrated in Figure \ref{fig:fig_illustration}), while adding minimal computation complexity. 
Our model outperforms baselines even without using the additional information commonly used in baseline methods on real-world public benchmark datasets for both classification and regression tasks in paired graph representation learning: drug-drug interaction classification and graph similarity regression.
The implementation is publicly available\footnote{https://github.com/LeeJunHyun/CoAttentionGraphPooling}.