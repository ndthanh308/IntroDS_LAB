Recent work has proposed and explored using coreset techniques for quantum algorithms that operate on classical data sets to accelerate the applicability of these algorithms on near-term quantum devices. We apply these ideas to Quantum Boltzmann Machines (QBM) where gradient-based steps which require Gibbs state sampling are the main computational bottleneck during training. By using a coreset in place of the full data set, we try to minimize the number of steps needed and accelerate the overall training time. In a regime where computational time on quantum computers is a precious resource, we propose this might lead to substantial practical savings. We evaluate this approach on 6x6 binary images from an augmented bars and stripes data set using a QBM with 36 visible units and 8 hidden units. Using an Inception score inspired metric, we compare QBM training times with and without using coresets.