\documentclass[
 reprint,
 amsmath,amssymb,
 aps,
 onecolumn,
 bibnotes,
 superscriptaddress
]{revtex4-2}


\usepackage{placeins}
\usepackage{lipsum}
\usepackage{graphicx}
\usepackage{epstopdf}

\usepackage{dcolumn}
\usepackage{bm}
\usepackage[colorlinks=true, citecolor={blue!80!black}, urlcolor={blue!50!black}, linkcolor = {blue!80!black}]{hyperref}
\usepackage[colorinlistoftodos]{todonotes}
\usepackage[capitalize]{cleveref}
\usepackage[english]{babel}
\usepackage{braket}
\usepackage{afterpage} % to get full sized figures to be where we want them to be

\usepackage{upgreek}

\usepackage{siunitx}
\DeclareSIUnit{\belmilliwatt}{Bm}
\DeclareSIUnit{\dBm}{\deci\belmilliwatt}
\renewcommand{\thefigure}{S\arabic{figure}}
\renewcommand{\theequation}{S\arabic{equation}}
\renewcommand{\thetable}{S\arabic{table}}

\newcommand{\fr}{$f_\mathrm{r}$}
\newcommand{\fnull}{$f_0$}
\newcommand{\fd}{$f_\mathrm{d}$}


\newcommand{\vf}{$v_\mathrm{F}$}

% We are using \Vla, \Vpa, \Vra and so on instead of replacing the a by a 1 because otherwise latex complains
\newcommand{\Vc}{$V_\mathrm{C}$}
\newcommand{\Vra}{$V_\mathrm{R1}$}
\newcommand{\Vlpa}{$V_\mathrm{LP1}$}
\newcommand{\Vpb}{$V_\mathrm{P2}$}
\newcommand{\Vrb}{$V_\mathrm{R2}$}
\newcommand{\Vlb}{$V_\mathrm{L2}$}



\begin{document}

\preprint{APS/123-QED}

\title{Supplementary information: Strong tunable coupling between two distant superconducting spin qubits}

\author{Marta Pita-Vidal}
\thanks{These two authors contributed equally.}
\affiliation{QuTech and Kavli Institute of Nanoscience, Delft University of Technology, 2600 GA Delft, The Netherlands}

\author{Jaap J. Wesdorp}
\thanks{These two authors contributed equally.}
\affiliation{QuTech and Kavli Institute of Nanoscience, Delft University of Technology, 2600 GA Delft, The Netherlands}

\author{Lukas J. Splitthoff}
\affiliation{QuTech and Kavli Institute of Nanoscience, Delft University of Technology, 2600 GA Delft, The Netherlands}

\author{Arno Bargerbos}
\affiliation{QuTech and Kavli Institute of Nanoscience, Delft University of Technology, 2600 GA Delft, The Netherlands}

\author{Yu Liu}
\affiliation{Center for Quantum Devices, Niels Bohr Institute, University of Copenhagen, 2100 Copenhagen, Denmark}

\author{Leo P. Kouwenhoven}
\affiliation{QuTech and Kavli Institute of Nanoscience, Delft University of Technology, 2600 GA Delft, The Netherlands}

\author{Christian Kraglund Andersen}%\email{C.K.Andersen@tudelft.nl}
\affiliation{QuTech and Kavli Institute of Nanoscience, Delft University of Technology, 2600 GA Delft, The Netherlands}



\date{July 28, 2023}

\maketitle

\tableofcontents




\newpage
 \section{\label{sec:theory} Theoretical description of longitudinal ASQ-ASQ coupling}


\subsection{General description of the estimation of $J$ used in the main text}

We derive a general expression for the coupling strength $J$ in terms of Andreev current operators. The derived expression facilitates the data analysis presented in the main text, where we use the experimentally obtained current-phase relationship, which differs from that expected from the ideal quantum dot junction theory \cite{Padurariu2010, Bargerbos2022b}. The current operator for each individual ASQ can be expressed as $\hat{I_i} = -\frac{2\pi}{\Phi_0}\frac{\partial H_i}{\partial \phi_i}$, where $i=1,2$. Here,  $\Phi_0=h/2e$ denotes the magnetic flux quantum,~$H_i=-\frac{\hbar \omega_i(\phi_i)}{2} \sigma^z_i$ in the subspace of the two spinful doublet states, $\phi_i$ is the phase drop across ASQ$i$, $\sigma_i^z$ is the $z$ Pauli matrix for ASQ$i$ and the $z$ axis is chosen along the spin-polarization direction for each qubit. As a result, the current operator can be related to the qubit frequency by
\begin{equation}\label{eq:current_operator_from_derivative}
\hat{I_i}= \frac{\pi h}{\Phi_0} \frac{\partial f_i(\phi_i)}{\partial \phi_i}\sigma^z_{i} = \frac{I_i}{2}\sigma^z_i,
\end{equation}
where we have defined the amplitude of the spin-dependent current $I_i =\frac{2\pi h}{\Phi_0} \frac{\partial f_i(\phi_i)}{\partial \phi_i} \approx h\frac{\partial f_i(\Phi_i)}{\partial \Phi_i} $ as in the main text, where the last approximation holds in the limit of $L_{\rm J,C} \ll L_{{\rm J},i}^I, L_{{\rm J},i}^\sigma  \forall i$, such that the phase drop can be directly related to the external flux applied through the loop: $\phi_i = \frac{2\pi}{\Phi_0}\Phi_i$. In the subspace of the doublet states for each ASQ we can expand the two-qubit Hamiltonian to first order around the phase bias $\phi_1$, given by the perturbation of the current through ASQ2, $\delta\phi_1 = \frac{2\pi}{\Phi_0} M\hat{I_2}$. Here, $M$ denotes an effective mutual inductance that determines how much phase drops over ASQ1 due to a current in ASQ2. We obtain
\begin{equation} \label{eq:derivation}
\begin{split}
H & = H_1(\phi_1 + \delta\phi_1) + H_2(\phi_2) \\
& = H_1(\phi_1 + \frac{2\pi}{\Phi_0} M\hat{I_2}) + H_2(\phi_2) \\
& \approx H_1(\phi_1) +  \frac{2\pi}{\Phi_0}\frac{\partial H_1(\phi_1)} {\partial\phi_1} M \hat{I_2} + H_2(\phi_2) \\
& =  H_1(\phi_1) + H_2(\phi_2)- M\hat{I_1}\hat{I_2}\\
& =  -\frac{\hbar\omega_1}{2}\sigma_{1}^z  -\frac{\hbar\omega_2}{2}\sigma_{2}^z - \frac{1}{4}M{I_1}{I_2}\sigma^z_{1}\sigma^z_{2}.
\end{split}
\end{equation}
In the limit of $L_{\rm J,C} \ll L_{{\rm J},i}^\sigma \forall i$, where $L_{{\rm J},i}^\sigma$ is the spin-dependent Josephson inductance of ASQ$i$, $M$ is given by the parallel combination of the spin-independent inductances of the three SQUID branches, $M=\frac{L_{\rm J, C} L_{\rm ASQ}}{L_{\rm J, C}+L_{\rm ASQ}}$. Here, $L_{\rm ASQ}(\phi_1, \phi_2)$ is the parallel combination of the spin-independent Josephson inductances of the ASQs:
$$ 
\frac{1}{L_{\rm ASQ}(\phi_1, \phi_2)} = \frac{\cos(\phi_1)}{L_{{\rm J},1}^I} + \frac{\cos(\phi_2)}{L_{{\rm J},2}^I}.
$$
By comparison to Eq.~(1) in the main text, we thus find
\begin{equation}\label{eq:J_current_derivatives}
J = \frac{M}{2h}{I_1}{I_2}= \frac{1}{2h}\frac{L_{\rm J, C} L_{\rm ASQ}}{L_{\rm J, C}+L_{\rm ASQ}} {I_1}{I_2}.
\end{equation}

\subsection{Analytical and numerical calculation of $J$ assuming a sinusoidal current-phase relation}
A simple model of the Hamiltonian for each ASQ is given by~\cite{Padurariu2010, Bargerbos2022b}
\begin{equation}
H_i(\phi_i) = -E^I_{{\rm J}, i} \cos{\phi_i} +  E^\sigma_{{\rm J}, i} \sigma_i^z \sin{\phi_i}, 
\label{Eq:ASQ-Hamiltonian}
\end{equation}
where $E^I_{{\rm J}, i} = \Phi_0^2/(4\pi^2 L^I_{{\rm J}, i})$ and  $E^\sigma_{{\rm J}, i} = \Phi_0^2/(4\pi^2 L^\sigma_{{\rm J}, i})$ denote the spin-independent and spin-dependent Josephson energies, respectively.
The total Hamiltonian of the coupled system of Fig.~1(a) in the main text is thus
\begin{align}
H(\phi) &\,= H_1(\varphi_1-\phi) +  H_2(\varphi_2-\phi) + E_{\rm J, C} \cos{(\phi)} \\
&\,=  -E^I_{{\rm J},1} \cos{(\varphi_1-\phi)} +  E^\sigma_{{\rm J}, 1} \sigma_1^z \sin{(\phi-\varphi_1)}  -E^I_{{\rm J},2} \cos{(\varphi_2-\phi)} +  E^\sigma_{{\rm J}, 2} \sigma_2^z \sin{(\varphi_2-\phi)} + -E_{\rm J, C} \cos{\phi}, 
\label{Eq:total-Hamiltonian}
\end{align}
where $E_{\rm J, C} = \Phi_0^2/(4\pi^2 L_{\rm J, C})$ and the reduced flux, $\varphi_i = 2 \pi \Phi_i /\Phi_0$, is the magnetic flux through the loop containing ASQ$i$ expressed in units of phase.


\subsubsection{Analytical solution}
Following Ref.~\cite{Padurariu2010}, assuming the energy-phase relation in Eq.~\eqref{Eq:ASQ-Hamiltonian}, the lowest order in $E^\sigma_{{\rm J}, 1}/E_{\rm J, C}$ and $E^\sigma_{{\rm J}, 2}/E_{\rm J, C}$ yields the coupling energy in the form
\begin{equation}
J = - 2\frac{E^\sigma_{{\rm J}, 1}E^\sigma_{{\rm J}, 2}}{|\tilde{E}|}   \cos{(\varphi_{\tilde{E}}-\varphi_1)}  \cos{(\varphi_{\tilde{E}}-\varphi_2)}, 
\label{Eq:coupling-Yuli}
\end{equation}
where
\begin{equation}
\tilde{E} = E^I_{\rm J,1}e^{i\varphi_1} + E^I_{\rm J,2}e^{i\varphi_2} + E_{\rm J, C}.
\end{equation}


\subsubsection{Numerical diagonalization}
To go beyond the limit of~\cref{Eq:coupling-Yuli}, i.e. for strong coupling, where the phase-drop on each ASQ is no longer linearly related to the applied flux, we solve the eigenenergies of the system numerically. For a given set of parameters, the energies of the four possible states of the qubit-qubit system ($E_{\uparrow \uparrow}$,  $E_{\downarrow \uparrow}$,  $E_{\uparrow \downarrow}$ and  $E_{\downarrow \downarrow}$) are obtained as the minima in $\phi$ of the four eigenvalues of $H(\phi)$. From these four energies, we calculate the coupling strength $J$ given the longitudinal (or Ising) type coupling Hamiltonian presented in Eq.~(1) in the main text. In this situation, the four eigenenergies of the coupled system are
\begin{align}
E_{\uparrow \uparrow} &\,= \frac{\hbar\omega_1}{2} + \frac{\hbar\omega_2}{2} + \frac{hJ}{2}, \\
E_{\downarrow \uparrow} &\,=  -\frac{\hbar\omega_1}{2} + \frac{\hbar\omega_2}{2} -\frac{hJ}{2},  \\
E_{\uparrow \downarrow} &\,=  \frac{\hbar\omega_1}{2} - \frac{\hbar\omega_2}{2} -\frac{hJ}{2}, \\
E_{\downarrow \downarrow} &\,=  -\frac{\hbar\omega_1}{2}  -\frac{\hbar\omega_2}{2} + \frac{hJ}{2}.
\label{Eq:eigenenergies}
\end{align}
Thus, from the numerically solved eigenenergies, we can find $J$ as 
\begin{equation}
J =  \frac{1}{2h} ( E_{\uparrow \uparrow} - E_{\uparrow \downarrow} - E_{\downarrow \uparrow} + E_{\downarrow \downarrow}).
\label{Eq:numerical_J}
\end{equation}

\subsubsection{Numerics including the transmon degree of freedom}

To fit the transmon spectroscopy data presented in Sec.~\ref{Sss:setpoint} and \ref{s:new-dataset}, we add a charging energy term to~\cref{Eq:total-Hamiltonian} corresponding to the transmon island and numerically diagonalize the resulting Hamiltonian in the phase basis~\cite{Bargerbos2020, Kringhoj2020b}
\begin{equation}
H_\mathrm{Transmon} =  -4E_{\rm c}\partial_\phi^2 + H(\phi)
\label{Eq:total-transmon-Hamiltonian}
\end{equation}
where $E_{\rm c}$ denotes the charging energy of the transmon island and $H(\phi)$ is defined in~\cref{Eq:total-Hamiltonian}.


\subsection{Method comparison}

Given the different approaches to calculate $J$, we now compare the different methods assuming the sinusoidal energy-phase  of~\cref{Eq:ASQ-Hamiltonian}, see Figs.~\ref{fig:supplement-theory} and \ref{fig:supplement-theory-realparameters}.
The analytical expression of~\cref{Eq:coupling-Yuli} is indicated with dashed lines.
The continuous lines are obtained numerically from exact diagonalization of the total Hamiltonian in~\cref{Eq:total-Hamiltonian} using~\cref{Eq:numerical_J}.  
The numerical diagonalization and the analytical expression of~\cref{Eq:coupling-Yuli} show near perfect agreement. 
Only when $E_{J,i}^\sigma \sim E_{J,C}$ (~\cref{fig:supplement-theory}c) a slight deviation is visible since~\cref{Eq:coupling-Yuli} is only valid in the limit $ E_{J,i}^\sigma \ll E_{J,C}$. 
We then test the estimate of $J$ on the sinusoidal energy-phase relation of~\cref{Eq:ASQ-Hamiltonian} using~\cref{eq:J_current_derivatives}, which is also used in the main text for the experimentally obtained energy-phase relation. This is shown with dotted lines, for different sets of parameters. 
In Fig.~\ref{fig:supplement-theory} we use parameters corresponding to the limit $L_{\rm J,C} \ll L_{{\rm J},i}^\sigma, L_{{\rm J},i}^I\forall i$ and, given the agreement between the different methods, we note that the approximations made in Sec.~\ref{sec:theory} are valid. Thus, the general estimate from Eq.~\eqref{eq:J_current_derivatives} (dotted line) agrees well with the exact value of $J$ found by numerical diagonalization of the full Hamiltonian, as expected. 
To illustrate the estimates obtained from the different methods outside of this limit, we use values of $E^I_{{\rm J}, i}$ in Fig.~\ref{fig:supplement-theory-realparameters}
that instead deviate from the limit $L_{\rm J,C} \ll  L_{{\rm J},i}^I\forall i$. In this case, we see that the estimate from Eq.~\eqref{eq:J_current_derivatives} deviates strongly from the exact numerical calculation due to the non-linear flux-phase relation.

% Figure environment removed


% Figure environment removed
\FloatBarrier

\subsection{Master equation approach to longitudinal coupling experiment}
We now present a simple master equation simulation to investigate the effect of the drives on the coupled two-qubit system in presence of decay. We solve the Lindblad master equation for the time evolution of the system density matrix, $\rho$, of the following form 
\begin{equation}\label{eq:lindblad_me}
\dot{\rho} = \left[\rho, H'\right] - \sum_n\frac{1}{2}\left[2C_n\rho C_n^\dagger - \rho C_n^\dagger C_n - C_n^\dagger C_n \rho\right],
\end{equation}
where $H'$ describes the two-qubit system in the rotating frame of the two drives, which have certain detuning $\Delta_i$ from qubit $i$.
This results in the following Hamiltonian
\begin{equation}
H'/\hbar = \frac{\Delta_1}{2}\sigma^z_{2} + \frac{\Delta_2}{2} \sigma^z_{2} + \frac{\Omega_{p1}}{2} \sigma^x_{1} + \frac{\Omega_{p2}}{2}\sigma^x_{2} + 2\pi \frac{J}{2}\sigma^z_{1}\sigma^z_{2},
\end{equation}
where $\Omega_{pi}$ denotes the drive amplitude of the tone near qubit $i$ and $\Delta_i = \omega_i - \omega_{pi}$ is the detuning of that drive frequency with the qubit frequency. 
Additionally, we apply the collapse operators $C_n$ on the individual qubits to simulate the effect of finite $T_1$ and $T_2$: $C_n \in \{\sqrt{\gamma_{1,1}}\sigma_1^-, \sqrt{\gamma_{\phi_1}/2} \sigma^z_{1}\, \sqrt{\gamma_{1,2}}\sigma_2^-, \sqrt{\gamma_{\phi_2}/2} \sigma^z_{2}\}$, where $\gamma_{1, i}=1/T_1^{{\rm ASQ}i}$,  $\gamma_{\phi_i}=1/T_2^{{\rm ASQ}i}$, $\sigma_i^+ = \ket{\uparrow_i}\bra{\downarrow_i}$ and $\sigma_i^- = \ket{\downarrow_i}\bra{\uparrow_i}$. We then solve~\cref{eq:lindblad_me} for the steady state solution using Qutip~\cite{qutip2013}. 
From the above evolution of the master equation under a certain drive amplitude, we obtain the populations of the states $\left\{\ket{\uparrow_1\uparrow_2}, \ket{\uparrow_1\downarrow_2},\ket{\downarrow_1,\uparrow_2},\ket{\downarrow_1\downarrow_2}\right\}$. Then, assuming a dispersive shift for each state and a linewidth of the resonator mode, we calculate the signal as the sum of populations times the displaced Lorentzians corresponding to each state and subtract the median for each linecut as is done with the experimental data.   
We compare the data measured in Fig.~3 of the main text to the master equation simulation with realistic parameters, as shown in~\cref{fig:supplement-qutip-fig3}. The simulations reproduce the main features seen in the data. 

% Figure environment removed

The peak height difference between the drive being on and off (black and red linecuts in~\cref{fig:supplement-qutip-fig3}) depends on the difference between the initial and final populations $P_{\downarrow\downarrow}$ of $\ket{\downarrow_1\downarrow_2}$ in the limit of large dispersive shift, which we consider here for simplicity. Consider the case where we apply a spectroscopy tone at $f_2-J$ on ASQ2, in the absence of a pump tone. In steady state, we get $P_{\downarrow\downarrow} =P_{\downarrow\uparrow}=0.5$ and $P_{\uparrow\uparrow} = P_{\uparrow\downarrow}=0$ 
due to the spectroscopy saturating ASQ2. Now, if we set a separate pump tone driving ASQ1 at $f_{\rm p}=f_1-J$ to a sufficiently high amplitude $\Omega_{p1}$, we obtain $P_{\downarrow\downarrow}= P_{\downarrow\uparrow} = P_{\uparrow\downarrow} = 0.33$. 
Thus the height of the driven peak at $f_2+J$ (red right peak) should be the height of the undriven (black) peak divided by a factor of $0.5/(0.5-0.33)\sim 2.94$ (as opposed to a factor of 2 which one might naively expect). The residual lowering of the peak observed in the experiment, we attribute to additional losses in the resonator mode under a strong drive. The height of the peak at $f_2-J$, on the other hand, is expected to have a similar height if no $T_1$ decay is present. 
In presence of finite and similar $T_1$ for ASQ1 and ASQ2, however, the final populations end up becoming $P_{\downarrow\downarrow}= P_{\downarrow\uparrow} = P_{\uparrow\downarrow} = P_{\uparrow\uparrow} = 0.25$, thus increasing the signal at $f_2-J$ and leading to a higher peak reaching half the height of the undriven peak at $f_2+J$ (as also seen in in~\cref{fig:supplement-qutip-fig3}(d)). However, beyond these limiting cases, depending on the exact ratio of the $T_1$ lifetimes of ASQ1 and ASQ2 the steady-state populations will vary. 


\newpage
\section{\label{Ss:device} Methods}

\subsection{Device overview}

The physical implementation of the device investigated is shown in Fig.~\ref{fig:device}. 
The chip, 6~mm long and 6~mm wide, consists of two devices coupled to a single transmission line with an input capacitor to increase the directionality of the outgoing signal (Fig.~\ref{fig:device}h).
For the experiments performed here, only the device discussed in the main text, highlighted in Fig.~\ref{fig:device}g, was measured. The resonator of the second device (uncolored device in Fig.~\ref{fig:device}g) was not functional and thus was not investigated.

For each device, a lumped element readout resonator is capacitively coupled to the feedline (Fig.~\ref{fig:device}e). The resonator is additionally capacitively coupled to the transmon island, which is connected to ground via  three Josephson junctions in parallel (the coupling junction, ASQ1 and ASQ2) defining two loops (Fig.~\ref{fig:device}b). The three junctions are implemented on two separate Al/InAs nanowires. 
The junctions are defined by etching the aluminum shell of the nanowire in a \SI{95}{nm}-long section for the coupling junction and \SI{215}{nm}-long sections for each of the ASQ junctions.  
The coupling junction is controlled by a single \SI{200}{nm}-wide electrostatic gate centered at the middle of the junction, controlled with a DC voltage \Vc. 
Each of the quantum dot junctions is defined by three gates consisting of two \SI{50}{nm} wide tunnel gates (L, R) surrounding a \SI{60}{nm} wide plunger gate (P), separated from each other by \SI{45}{nm} (Fig.~\ref{fig:device}c, d).
We define the DC voltages used for the left and right tunnel gates as \Vlpa~and \Vra~for ASQ1 or \Vlb~and \Vrb~for ASQ2. The plunger gate of ASQ1 is also set to \Vlpa because it was shorted to the left tunnel gate due to a fabrication imperfection. 
All gate lines except for the plunger lines incorporate a fourth-order Chebyshev LC-LC filter with a cut-off frequency at \SI{2}{GHz}. The first and second inductive elements, of \SI{5.2}{nH} and \SI{6.1}{nH} respectively, are implemented using thin strips of NbTiN with widths of \SI{3.5}{\micro m} and \SI{300}{nm}, respectively. The first and second capacitive elements, of \SI{2.45}{pF} and \SI{2.08}{pF} respectively, are implemented with parallel plate capacitors.
The plunger gate of ASQ2  is connected to a bias-tee on the printed circuit board formed by a \SI{100}{\kilo\ohm} resistor and a \SI{100}{pF} capacitor. This permits the simultaneous application of a DC signal, \Vpb, to control the level of the quantum dot junction, and microwave tones, $f_{\rm d}$ and $f_{\rm p}$, to drive either of the spin-flip transitions or the transmon. We also drive ASQ1 using the same gate line, because the bias-tee at the plunger gate of ASQ1 was not functional. The flux through the loop containing ASQ1 is controlled using a flux line (shown in amber). Its design in the area of the loops was inspired by Ref.~\cite{Rot2022}. We furthermore incorporate a  \SI{25}{pF}  parallel plate capacitor near the end of the flux line which, together with the \SI{1}{nH} inductance of the rest of the flux line, implements an LC low-pass filter with a cut-off at \SI{1}{GHz}. 



% Figure environment removed



% Figure environment removed


\subsection{Summary of device parameters}\label{Sss:parameters_table}

\begin{table*}[h!]
\begin{ruledtabular}
\begin{tabular}{rcrc}
Bare resonator frequency, $f_{\rm r, 0}$ & \SI{4.229}{GHz} & Resonator $Q_{\rm c}$ & 1.3k \\
Resonator $Q_{\rm i}$ & $\sim$~35k & Transmon decay time, $T_{1}^{\rm t}$ & \SI{53.6}{ns}  \\
Resonator-transmon coupling, $g/h$ & $\sim$~\SI{287}{MHz}   & Transmon Ramsey time, $T_{\rm 2R}^{\rm t}$ & \SI{80.0}{ns}   \\
Transmon charging energy, $E_{\rm c}/h$ &   \SI{200}{MHz} &
\\
\end{tabular}
\end{ruledtabular}
\caption{
{\bf Values of relevant device parameters.} The resonator bare frequency and quality factors are measured when all electrostatic gates are at \SI{-1000}{mV} and thus all three junctions are pinched off (see Fig.~\ref{fig:resonator}). The transmon charging energy is extracted from the transmon anharmonicity in two-tone spectroscopy. The resonator-transmon coupling is extracted from a single-tone spectroscopy measurement at their anti-crossing (see Fig.~\ref{fig:supplement-pinchoff}). The transmon coherence values were measured with both ASQs in pinch off and at $V_{\rm C}$~=~\SI{1500}{mV}, which sets the transmon frequency to $f_{\rm t}$~=~\SI{5.45}{GHz}. 
}
\label{tab:circuit_parameters}
\end{table*}




\subsection{Nanofabrication details}

The device fabrication occurs in several steps identical to that described in~\cite{Bargerbos2022}, and repeated here for the sake of completeness. The substrate consists of 525~$\upmu$m-thick high-resistivity silicon, covered in \SI{100}{nm} of low-pressure chemical vapor deposited $\rm{Si_3N_4}$. In the first step, a 4-inch wafer of such substrate is cleaned by submerging it for \SI{5}{min} in HNO$_3$ while ultasonicating, followed by two short H$_2$O immersions to rinse the HNO$_3$ residues. Afterwards, a \SI{20}{nm}-thick NbTiN film is sputtered on top of the substrate using an \textit{AJA International ATC 1800} sputtering system. Subsequently, Ti/Pd e-beam alignment markers are patterned on the wafer, which is thereafter diced into smaller individual dies of approximately \SI{12}{mm}$\times$\SI{12}{mm}. In the next step, the gate electrodes and the rest of the NbTiN circuit elements are patterned on one die covered by \SI{110}{nm}-thick \textit{AR-P 6200} (positive) e-beam resist using electron-beam lithography. The structures are then etched using $\rm{SF_6}$/$\rm{O_2}$ reactive ion etching for \SI{47}{s}. Subsequently, \SI{28}{nm} of $\rm{Si_3N_4}$ dielectric are deposited on top of the gate electrodes using plasma-enhanced chemical vapor deposition and etched in patterns with a buffered oxide etchant (for \SI{3}{min}). This dielectric is used as a gate dielectric, as well as as the dielectric for the crossovers at the DC gate lines and flux line and for the crossover that generates the twist in the loop containing ASQ1. 

The nanowires are deterministically placed on top of the dielectric using a nanomanipulator and an optical microscope.  These nanowires are $\sim$\SI{10}{\micro m}-long epitaxial superconductor-semiconductor nanowires with a \SI{110}{nm}-wide hexagonal InAs core and a \SI{6}{nm}-thick Al shell covering two of their facets, in turn covered by a thin layer of aluminium oxide. The growth conditions were almost identical to those detailed in Ref.~\cite{Krogstrup2015}, with the only two differences being that this time the As/In ratio was 12, smaller than in Ref.~\cite{Krogstrup2015}, and that the oxidation of the Al shell was now performed in-situ, for better control, reproducibility and homogeneity of the oxide layer covering the shell. Inspection of the nanowire batch, performed under a scanning electron microscope directly after growth, indicated an average wire length of $9.93 \pm$~\SI{0.92}{\micro m} and an average wire diameter of $111 \pm 5$~nm.

After nanowire placement, three sections of the aluminium shells are selectively removed by wet etching for \SI{55}{s} with \textit{MF-321} developer. 
These sections form the two quantum dot junctions and the coupling junction, with lengths \SI{215}{nm} and \SI{95}{nm}, respectively. After the junctions etch, the nanowires are contacted to the transmon island and to ground by a \SI{110}{s} argon milling step followed by the deposition of \SI{150}{nm}-thick sputtered NbTiN. Finally, the chip is diced into 6 by 6 millimeters, glued onto a solid gold-plated copper block with silver epoxy, and connected to a custom-made printed circuit board using aluminium wire-bonds (Fig.~\ref{fig:device}g). 



\subsection{Cryogenic and room temperature measurement setup}


% Figure environment removed

The device was measured in an \textit{Oxford instruments Triton} dilution refrigerator with a base temperature of approximately \SI{20}{mK}. Details of the wiring at room and cryogenic temperatures are shown in Fig.~\ref{fig:cryogenic_setup}. The setup contains an input radio-frequency (RF) line, an output RF line, an extra RF line for the drive tones, a flux-bias line and multiple direct current (DC) lines used to tune the electrostatic gate voltages. The DC gate lines are filtered at base temperature with multiple low-pass filters connected in series. 
The input, flux and drive RF lines contain attenuators and low-pass filters at different temperature stages, as indicated. In turn, the output RF line contains amplifiers at different temperature stages: a traveling wave parametric amplifier (TWPA) at the mixing chamber plate ($\approx$~\SI{20}{mK}),  a high-electron-mobility transistor (HEMT) amplifier at the \SI{4}{K} stage, and an additional amplifier at room temperature. 
A three-axis vector magnet, for which the $y$ and $z$ coils are illustrated by yellow rectangles in Fig.~\ref{fig:cryogenic_setup} ($x$-axis not shown), is thermally anchored to the \SI{4}{K} temperature stage, with the device under study mounted at its center. The three magnet coils are controlled with \textit{Yokogawa GS610} current sources. The current through the flux line, $I$, is controlled with a \textit{Yokogawa GS200} current source.
At room temperature, a vector network analyzer (VNA) is connected to the input and output RF lines for spectroscopy at frequency $f_{\rm r}$. On the input line, this signal is combined with a separate IQ-modulated tone also at $f_{\rm r}$, only used for time-domain measurements. The IQ-modulated drive tone at frequency $f_{\rm d}$ and the pump tone at frequency $f_{\rm p}$ are both sent through the drive line. For time-domain measurements, the output signal is additionally split off into a separate branch and down-converted to be measured with a \textit{Quantum Machines OPX}.


\subsection{\label{Ss:methods} Data processing}

\subsubsection{Background subtraction for single-tone and two-tone spectroscopy measurements}
For all single-tone spectroscopy measurements shown in the main text and Supplementary Information, we plot the amplitude of the transmitted signal, $|S_{21}|$, with the frequency-dependent background, $|{S_{21, {\rm b}}(f_{\rm r})}|$, divided out, in dB:  $10\log_{10}(|S_{21}|/|{S_{21, {\rm b}}}|)$. The background is extracted from an independent measurement of the transmission through the feedline as a function of \Vc. To determine the background for each $f_{\rm r}$ we do not consider transmission data for which the resonator frequency is more than \SI{20}{MHz} close to $f_{\rm r}$, so that the presence of the resonator does not impact the extracted background.

For two-tone spectroscopy measurements, we instead plot the transmitted signal, $|S_{21}|$, with the frequency-independent background, $|{S_{21, {\rm m}}}|$, subtracted: $|S_{21}|-|{S_{21, {\rm m}}}|$. In this case, the background is defined as the median of $|S_{21}|$ of each frequency trace.

  
\subsubsection{Gaussian fits to extract $J$ from spectroscopy measurements}
To extract the value of the coupling strength from the peak splitting observed in spectroscopy measurements, we follow the following procedure:
\begin{enumerate}
    \item We first fit the two-tone spectroscopy signal in the absence of a pump tone with a single Gaussian function of the form
    \begin{equation}
        \label{eq:single_Gaussian}
        \frac{A}{\sqrt{2\pi\sigma^2}}\exp\left(\frac{-(f_{\rm d}-f_a)^2}{2\sigma^2}\right)+Bf_{\rm d}+C,
    \end{equation}
    from which we extract the position of the first peak, $f_a$, and its width, $\sigma$.
    \item Next, we fit the two-tone spectroscopy signal in the presence of a pump tone with a double Gaussian function of the form
    \begin{equation}
        \label{eq:double_Gaussian}
        \frac{A_a}{\sqrt{2\pi\sigma^2}}\exp\left(\frac{-(f_{\rm d}-f_a)^2}{2\sigma^2}\right)+\frac{A_b}{\sqrt{2\pi\sigma^2}}\exp\left(\frac{-(f_{\rm d}-f_b)^2}{2\sigma^2}\right)+Bf_{\rm d}+C,
    \end{equation}
    for which the peak widths $\sigma$, as well as the position of the first peak, $f_a$, are fixed to their values extracted from the previous fit. From this fit, we extract the position of the second peak, $f_b$, as well as the chi-square of the fit, $\chi^2_{\rm double}$.
    \item Next, we repeat a fit to the two-tone spectroscopy signal in the presence of a pump tone with a single Gaussian function (Eq.~\eqref{eq:single_Gaussian}) and extract the chi-square of the fit, $\chi^2_{\rm single}$.
    \item To determine whether a double Gaussian fits better than a single Gaussian, we compare the goodness of fit of a double and single Gaussian fit. If $(\chi^2_{\rm single}-\chi^2_{\rm double})/\chi^2_{\rm double} \geq 0.1$, we conclude that the data shows two peaks and extract $J$ as $J=(f_a-f_b)/2$ and the error of $J$ as the error of $f_b$ extracted from the double Gaussian fit (its one-sigma confidence interval).
    \item If, else, $(\chi^2_{\rm single}-\chi^2_{\rm double})/\chi^2_{\rm double} < 0.1$, we conclude that the data shows a single peak and thus $J=0$.
\end{enumerate}
    
\subsubsection{Determination of the flux axis}\label{sss:fits_spin_flip}
To determine the flux axis for data that we display in the main text as a function of $\Phi_i$, we map the corresponding flux control parameter ($I$ for the loop containing ASQ1 and $B_y$ for the loop containing ASQ2) to the fluxes $\Phi_1$ and $\Phi_2$. To do so, we need to determine the value of the control parameter corresponding to $\Phi_i=0$ (denoted as $I_{\Phi_1=0}$ and $B_{y, \Phi_2=0}$, respectively) as well as the one flux quanta (denoted as $I_{\Phi_0}$ and $B_{y, \Phi_0}$, respectively). The former is independently determined for each separate measurement, from fits of the data to the expected transitions.

The values of the flux quanta ($I_{\Phi_0}=$~\SI{9.62}{mA} for $\Phi_1$ and $B_{y, \Phi_0}=$~\SI{3.16}{mT} for $\Phi_2$) are fixed throughout all main text and supplementary figures and is extracted from fits to the data in Fig.~2a and b in the main text. The data in Fig.~2b is fitted with a sinusoidal dependence of the form
\begin{equation}
    \label{eq:sin}
    2E^\sigma_{\rm J, 2}\sin\left(2\pi\frac{B_y-B_{y, \Phi_2=0}}{B_{y, \Phi_0}}\right)+C,
\end{equation}
as expected for a quantum dot Josephson junction \cite{Padurariu2010, Bargerbos2022b}. The data in Fig.~2a is instead fitted with a phenomenological skewed sinusoidal dependence of the form
\begin{equation}
    \label{eq:sin_sk}
    2E^\sigma_{\rm J, 1}\sin\left(2\pi\frac{I-I_{\Phi_1=0}}{I_{\Phi_0}} + S\sin\left(2\pi\frac{I-I_{\Phi_1=0}}{I_{\Phi_0}}\right) \right)+C,
\end{equation}
where $-1< S <1$ is the skewness parameter. For the data in Fig.~1a, we extract $S=-0.39$. The observed skewness of the spin-flip is, to our knowledge, not predicted by existing models~\cite{Padurariu2010, Bargerbos2022b}, so further investigation is needed to explain its origin.

\newpage


\section{\label{Ss:tuneup} Basic characterization and tuneup}

\subsection{Readout resonator characterization} \label{Sss:resonator}

In this section, we perform a fit to a bare resonator spectroscopy trace and extract the resonator parameters shown in Tab.~\ref{tab:circuit_parameters}.  The result of a single-tone resonator trace, performed with all Josephson junctions pinched off, is shown with black markers in Fig.~\ref{fig:resonator}. The grey lines show the best fit of the complex transmission to the expected dependence \cite{Khalil2012, ResonatorRepo}
\begin{equation}
S_{21}(f_{\rm r}) = 1 - \frac{1 +  i \alpha} { 1 + \frac{Q_{\rm c}}{Q_{\rm i}} + 2  Q_{\rm c} i \frac{f_{\rm r}-f_{\rm r,0}}{f_{\rm r,0}}}, 
\label{Eq:resonator_fit}
\end{equation}
where $f_{\rm r,0}$ is the bare resonator resonance frequency, $Q_{\rm c}$ and $Q_{\rm i}$ are the coupling and internal quality factors, respectively, and $\alpha$ is a real number to account for the resonator asymmetry. 

% Figure environment removed


\subsection{Gate and flux characterization} \label{Sss:parameter}

Throughout this manuscript, we use $B_y$, which affects both $\Phi_1$ and $\Phi_2$, to tune $\Phi_2$ and the current through the flux line, $I$, to tune $\Phi_1$. Fig.~\ref{fig:flux}a shows the $B_y$ tunability of $\Phi_2$, for which the period corresponds to \SI{3.25}{mT}. Note that this value is slightly larger than the actual flux quantum due to the small flux jumps present in the signal. Fig.~\ref{fig:flux}c and b show how the current through the flux line, $I$, controls $\Phi_1$, for which a flux quantum corresponds to \SI{9.61}{mA}, while leaving $\Phi_2$ unaffected.

% Figure environment removed

We now investigate the performance of all electrostatic gates. 
Fig.~\ref{fig:supplement-pinchoff} shows the resonator frequency, measured by single-tone spectroscopy, while various combinations of gates are varied. In all cases, only at most one of the three junctions is open, thus not defining any loops. All junctions can be fully pinched off using any of the gates that control them while leaving the rest of the gates open, which confirms the proper functionality of all gates.


% Figure environment removed

Panels a-c show the effect of varying the gates of ASQ1 either simultaneously (a) or separately (b, c). Note that the left and plunger gates of ASQ1 are connected to each other on-chip and thus are always set at the same voltage,~\Vlpa, while the right gate of ASQ1 is set at voltage~\Vra. 
The effect of the left, plunger and right gates of ASQ2, respectively set at voltages~\Vlb, \Vpb~and~\Vrb, is shown in panels e-h. Although the pinch-off voltages for ASQ2 are slightly lower, this junction displays a behavior similar to that of ASQ1.
Panel d shows the effect of~\Vc, the voltage of the coupling Josephson junction gate, which tunes $E_{\rm J,C}$. By varying~\Vc, the transmon frequency can be tuned to values above the bare resonator frequency, thus resulting in an avoided crossing between the resonator and transmon frequencies at around~\Vc~=~\SI{500}{mV}.
We find a transmon-resonator coupling strength $g/h\sim$~\SI{287}{MHz} as half of the distance between the two resonances observed at the avoided crossing.

Next, we measure the~\Vc-dependence of the transmon frequency $f_{\rm t}$ with both ASQs pinched off, from which we calibrate the \Vc~to~$E_{\rm J, C}$ map which is used for the data processing behind Fig.~4 in the main text. 

The black markers in Fig.~\ref{fig:supplement-transmon-EJ}a show the transmon frequency as a function of \Vc, measured directly after the data shown in Fig.~4 of the main text and at the same magnetic field conditions ($B_r=$~\SI{35}{mT} applied in the chip plane and six degrees away from the $z$ direction). For comparison, we also show the transmon frequencies measured while taking the data in Fig.~4e (teal markers), with both ASQs open. In this case, the measured frequencies deviate from the black markers, since they instead result from a parallel combination of $E_{\rm J, C}$ and the Josephson energies of both qubits. The black markers in Fig.~\ref{fig:supplement-transmon-EJ}a are used to determine the \Vc-dependence of $E_{\rm J, C}$ shown in Fig.~\ref{fig:supplement-transmon-EJ}b, given the value of $E_{\rm c}$ independently determined from a measurement of the transmon anharmonicity (see Tab.~\ref{tab:circuit_parameters}). These data are used to determine the x-axis of Fig.~4e in the main text.



% Figure environment removed



\subsection{ASQ gate setpoints} \label{Sss:setpoint}

In this section, we discuss the tune-up of each individual ASQs, which results in the chosen gate setpoints specified in Tab.~\ref{tab:setpoints}. 

The tune-up of ASQ1 is presented in Fig.~\ref{fig:ASQ1_2Dmap}. We first set the junction containing ASQ2 to pinch-off by setting its three gates to \SI{-1000}{mV} and set the ASQ1 gates to a region where we detect a sizable spin-splitting energy in a low-resolution measurement.  From the transmon frequency at $\Phi_1=0,\Phi_0/2$ we estimate the spin-independent Josephson energy $E^I_{\rm J, 1}$ and map it out over a region in gate space using the two gate voltages of ASQ1 (Fig.~\ref{fig:ASQ1_2Dmap}a). Then, we proceed to investigating the value of the spin-dependent Josephson energy, $E^\sigma_{\rm J, 1}$. One way of doing so would be directly mapping out the spin-flip frequency $f_1$ in gate space. However, the visibility of the transition is significantly reduced at $B=0$ due to the thermal population of the ASQ as well as to the smaller matrix element from driving the spin transition. We instead perform $\Phi_1$-dependent transmon spectroscopy at a few selected gate points indicated with markers in Fig.~\ref{fig:ASQ1_2Dmap}a (Fig.~\ref{fig:ASQ1_2Dmap}b-g). 
For each gate setpoint we estimate the values of $E^\sigma_{\rm J, 1}/h$ by matching the distance between transmon frequencies at $\Phi_1=\Phi_0/4$ to its theoretically expected value extracted from numerical diagonalization of Eq.~\eqref{Eq:total-transmon-Hamiltonian} in the phase basis. Similarly, $E^I_{\rm J, 1}$ is estimated by fitting the measured transmon frequencies at $\Phi_1=0, \Phi_0/2$ to their theoretically expected values. The resulting quantities are indicated as labels on each panel. We choose the gate setpoint used for ASQ1 in the main text by maximizing $E^\sigma_{\rm J, 1}$ while keeping the value of $E^I_{\rm J, 1}$ low, since a high value negatively impacts the maximal coupling strength $J$. The chosen ASQ1 gate setpoint (see Tab.~\ref{tab:setpoints}) is indicated in Fig.~\ref{fig:ASQ1_2Dmap}a with a purple marker.



% Figure environment removed

% Figure environment removed


% Figure environment removed

Next, we pinch off the junction containing ASQ1 to tune-up the gate configuration of ASQ2. We perform an investigation analogous to the one detailed above, as shown in Fig.~\ref{fig:ASQ2_2Dmap}.  Fig.~\ref{fig:ASQ2_2Dmap}a is measured in the same way as  Fig.~\ref{fig:ASQ1_2Dmap}a and displays the evolution of $E^I_{\rm J, 2}$ with the tunnel gates, \Vlb~and \Vrb, while \Vpb~is kept at \SI{0}{mV}. Fig.~\ref{fig:ASQ2_2Dmap}b shows the tunnel gate dependence of $E^\sigma_{\rm J, 2}$, determined from direct spin-flip spectroscopy of ASQ2 at $B=0$: $E^\sigma_{\rm J, 2}/h = f_2(B=0, \Phi_2=\Phi_0/4)/2$. Similarly to the strategy for ASQ1, we choose a gate setpoint for ASQ2 by maximizing $E^\sigma_{\rm J, 2}$ while keeping $E^I_{\rm J, 2}$ as low as possible. However, for some gate points in this region of gate space, a singlet state is also slightly visible in transmon spectroscopy (as can be seen in Fig.~\ref{fig:ASQ2_2Dmap}c, around $\Phi_2=0$). The presence of the singlet state indicates that the singlet phase of the system is only separated by an energy gap comparable to the thermal energy of the system. Consequently, while choosing the ASQ2 setpoint we also minimize the visibility of the singlet state. The chosen setpoint (see Tab.~\ref{tab:setpoints} is indicated in Fig.~\ref{fig:ASQ2_2Dmap}a and b with a maroon marker).

Finally, we perform in-field spin-flip spectroscopy of both ASQs, as well as transmon spectroscopy at zero field, to more accurately determine their Josephson energies detailed in Tab.~\ref{tab:setpoints}. 

\begin{table*}[h!]
\begin{ruledtabular}
\begin{tabular}{cccccc}
\textrm{}& $V_{{\rm L} i}$ (mV)  & $V_{{\rm P} i}$ (mV)  & $V_{{\rm R} i}$ (mV) & $E^I_{{\rm J}, i}/h$ (GHz) & $E^\sigma_{{\rm J}, i}/h$ (GHz) \\
\colrule
ASQ1 & 62.0 & 62.0 & 350.0 & 2.29 & 0.82  \\
ASQ2 & 206.50 & 0.0 & -624.0 & 0.45 & 0.63  \\
\end{tabular}
\end{ruledtabular}
\caption{
ASQ1 and ASQ2 gate voltage set points and extracted model parameters from the measurements in Fig.~\ref{fig:parameter-extraction}.
}
\label{tab:setpoints}
\end{table*}


The spin-flip spectroscopy shown in Fig.~\ref{fig:parameter-extraction}a and b is performed under the same magnetic field conditions as the ones where we measured coupling in the main text. We extract $E^\sigma_{{\rm J}, 1}/h=$~\SI{0.82}{GHz} and $E^\sigma_{{\rm J}, 2}/h=$~\SI{0.63}{GHz} from fits of a skewed and non-skewed sine, respectively, to the measured data (see Sec.~\ref{sss:fits_spin_flip}). The $E^\sigma_{{\rm J}, i}/h=$ values are determined as one fourth of the flux dispersion of the fit result. $E^I_{{\rm J}, 1}/h=$~\SI{2.29}{GHz} and $E^I_{{\rm J}, 2}/h=$~\SI{0.45}{GHz} are determined similarly to how it was done for Figs.~\ref{fig:ASQ1_2Dmap} and \ref{fig:ASQ2_2Dmap}, by fitting the $\Phi_i$-dependent data to the expected transmon frequencies obtained by numerical diagonalization of the Hamiltonian in Eq.~\eqref{Eq:total-transmon-Hamiltonian} at $\Phi_i$ being integer multiples of $\Phi_0/2$. In both cases, we fix the spin-dependent part of the transmon potential to that extracted from the fits in Fig.~\ref{fig:parameter-extraction}a and b.





\subsection{Andreev spin qubit readout} \label{Sss:readout}

In the main text (Fig.~1d) we discussed how, when both loops are open, we observe four possible resonator frequencies, depending on the four possible spin states of the ASQ1-ASQ2 system, $\left\{\ket{\uparrow_1\uparrow_2}, \ket{\uparrow_1\downarrow_2},\ket{\downarrow_1,\uparrow_2},\ket{\downarrow_1\downarrow_2}\right\}$. This allows us to perform two-tone spectroscopy of either one of the two qubit transitions, $f_1$ and $f_2$, which are present when both ASQ junctions are open. Here, we show the analogous situation when only {\it one} out of the two ASQs is open, while the junction containing the other one is fully pinched off (Fig.~\ref{fig:readout}). 


Fig.~\ref{fig:readout}a shows the $\Phi_1$-dependence of resonator spectroscopy, at zero magnetic field and when only ASQ1 is open. In this case, we observe two branches of the resonator frequency, corresponding to the two possible states of ASQ1: $\ket{\uparrow_1}$ or $\ket{\downarrow_1}$). The different visibility of each of the branches is a consequence of the different thermal populations of the two states at $B_r=0$. This is expected, since the spin-splitting of ASQ1 varies with flux reaching up to $2E^\sigma_{{\rm J}, 1}/h=$~\SI{1.64}{GHz}, comparable, when transformed into an effective temperature, to typical electron temperatures found in other experiments \cite{Jin2015, PitaVidal2023, Uilhoorn2021}. 
Fig.~\ref{fig:readout}b shows the analogous situation but now for ASQ2. In this case, the resonator also displays two separate frequencies. After fixing $B_y$ so that $\Phi_2\sim - \Phi_0/4$ and so that the separation between the resonator frequencies corresponding to $\ket{\uparrow_2}$ and $\ket{\downarrow_2}$ is sizable, we open ASQ1 to its setpoint. In such situation, when performing resonator spectroscopy versus $\Phi_1$, we observe four different transitions, labeled with their corresponding states in Fig.~\ref{fig:readout}c. Note that, in this case, the difference in visibility becomes more perceptible due to the larger energy separation between the different states $\left\{\ket{\uparrow_1\uparrow_2}, \ket{\uparrow_1\downarrow_2},\ket{\downarrow_1,\uparrow_2},\ket{\downarrow_1\downarrow_2}\right\}$.

% Figure environment removed





\subsection{Magnetic field angle dependence and determination of the spin-orbit direction} \label{Sss:angle}


In this section, we specify the measurements performed to determine the zero-field spin-polarization direction for each Andreev spin qubit. For each qubit, we perform spin-flip spectroscopy measurements, like those shown in Fig.~\ref{fig:parameter-extraction}a and b, for different magnetic field directions. As reported previously in Ref.~\cite{Bargerbos2022b}, we observe that both the flux dispersion of the spin-flip transition, $df$, as well as the $g$-factor, depend strongly on the direction of the applied magnetic field. To determine these quantities, the maxima, $f_i^{\rm max}$, and minima, $f_i^{\rm min}$, of the spin-flip frequencies are first extracted by hand  from two-tone spectroscopy measurements of the spin-flip transition,  analogous to those in Fig.2a-c of the main text. The $g$-factors are calculated from the average of these maximum and minimum frequencies, as $g = (f_i^{\rm max}+f_i^{\rm min})/(2\mu_{\rm B} B_r)$, where $\mu_{\rm B}$ is the Bohr magnetron and $B_r$ is the magnitude of the applied magnetic field. The frequency dispersion is determined as $df=(f_i^{\rm max}-f_i^{\rm min})/2$. The dependence of $g$ and $df$ on the magnetic field direction is shown in  Fig.~\ref{fig:supplement-angle-dep} with purple and maroon markers for ASQ1 and ASQ2, respectively. 

First, we investigate the dependence on the angle within the chip plane and away from the nanowires axis, $\theta_{\phi=90}$.  $\theta_{\phi=90}=0$ indicates that the field is applied approximately along the nanowires axis, while 
$\theta_{\phi=90}=90$ degrees indicates that the field is applied in-plane but approximately perpendicular to the nanowire axis. We find that the $g$-factor of ASQ1 depends strongly on $\theta_{\phi=90}$, while that of ASQ2 stays almost constant, fluctuating only between 5.5 and 6.5 (Fig.~\ref{fig:supplement-angle-dep}a). Within this plane, the $g$-factor of ASQ1 is found to be maximal when the magnetic field $B_r$ is applied approximately along the nanowires axis, while for ASQ2 it is maximized for $\theta_{\phi=90}\sim 31$ degrees away from the nanowire axis. Performing the same experiment while varying the field direction in the $x$-$z$ plane, the plane perpendicular to the chip and containing the nanowires axis, we observe a similar dependence (Fig.~\ref{fig:supplement-angle-dep}c). This time, the ASQ1 $g$-factor is again maximized along the nanowires axis, while that of ASQ2 becomes maximal when $B_r$ is applied $\theta_{\phi=0}\sim 60$~degrees away from the nanowires axis. This variability of the $g$-factor dependence for different configurations is consistent with previous observations of quantum dots implemented in InAs nanowires and is thought to be due to mesoscopic fluctuations of the electrostatic environment at the quantum dot \cite{Han2023, Bargerbos2022b}.


% Figure environment removed

To learn about the zero-field spin-polarization direction of each qubit, we now focus on the field-direction dependence of the flux dispersion of the spin-flip transition. We denote by $df$ the difference in frequency between the maximum and minimum of the spin-flip frequency versus flux. When the field is applied along the zero-field spin-polarization direction, we expect that $df=4 E^\sigma_{{\rm J}, i}$ (see Sec.~\ref{sec:theory}). However, when a component of the applied magnetic field is perpendicular to the zero-field spin-polarization direction, $df$ is reduced due to the hybridization of the two spin states \cite{Bargerbos2022b}. Fig.~\ref{fig:supplement-angle-dep}b shows the $\theta_{\phi=90}$ dependence of $df$ for both ASQs. We find that, in both cases, $df$ becomes minimal at a direction approximately perpendicular to the nanowire axis.
We also perform a similar experiment in the $x$-$z$ plane (see Fig.~\ref{fig:supplement-angle-dep}d). 
The extracted angles
constitute two of the directions perpendicular to the spin-polarization axis. Their cross-product thus returns the zero-field spin polarization directions for each qubit, which are indicated in Tab.~\ref{tab:directions} in spherical coordinates, where $\theta_{||}$ denotes the polar angle away from the nanowire axis and  $\phi_{||}$ denotes the azimuthal angle measured away from the $x$-axis (see Fig.~\ref{fig:supplement-angle-dep}). The angle between this direction and the direction of the applied magnetic field in Figs.~3 and 4 of the main text ($\theta=185, \phi=90$) is indicated in the last column of Tab.~\ref{tab:directions}.

The angle used for all measurements in the main text, except for Fig.~2c,d,  is indicated with vertical dotted lines in Fig.~\ref{fig:supplement-angle-dep}a and b. We chose this angle following various considerations. First, we wanted to minimize the field component perpendicular to the spin directions of each of the ASQs. The reason for this is that we expect transverse qubit-qubit coupling terms to emerge under the presence of a large perpendicular Zeeman energy compared to $E_{\rm J, i}^\sigma$, at the cost of the longitudinal component. Second, we wanted to maximize the difference in $g$-factors to avoid crossings between the qubit frequencies versus flux. This enables the possibility of spectroscopically measuring the coupling strength at every flux point. Finally, we chose the total field magnitude $B_r=$~\SI{35}{mT}  to set the ASQ frequencies at setpoints that did not cross neither the resonator nor any transmon transition frequency for any value of flux. 



\begin{table}[h!]
\begin{ruledtabular}
\begin{tabular}{cccc}
\textrm{}&  $\theta_{||}$  & $\phi_{||}$ & $\alpha$ \\
\colrule
ASQ1 & 8.54 & 54.76 & 5.1 \\
ASQ2 & 22.73 & 157.15 & 21.5 \\
\end{tabular}
\end{ruledtabular}
\caption{ {\bf Zero-field spin-polarization directions for ASQ1 and ASQ2 in degrees.} 
 The zero-field spin-polarization direction ($\theta_{||}, \phi_{||}$) is calculated as the vector product of the two perpendicular directions 
 indicated with colored lines in the x-axis of Fig.~\ref{fig:supplement-angle-dep}b,d. 
 $\alpha$ denotes the angle between the field applied in the main text measurements and the spin-polarization direction for each qubit.
}\label{tab:directions}
\end{table}



\FloatBarrier
\newpage
\newpage
\section{\label{coherence-data} Supplementary coherence data}
We now describe the functions used for extracting the coherence times quoted in the main text. To determine  $T_1$  we fit an exponential decay 
\begin{equation}\label{eq:coh-T_1_decay}
    y(t) = a\cdot \mathrm{exp}[{t/T_1}] + c
\end{equation}
where $a$, $T_1$ and $c$ are free fit parameters.
For Ramsey and Hahn echo (see~\cref{fig:supplement-T2echo}) experiments we fit a sinusoide with a exponential decay envelope and sloped background 
\begin{equation}\label{eq:coh-T_2_decay}
 y(t) = a \cdot \cos\left(\frac{2\pi}{ p} t - \phi\right)\cdot\mathrm{exp}\left[(-t/T_2)^{d+1}\right] + c + et 
 \end{equation}
 where $a$, $T_2$, $\phi$, $c$, $e$ and $p$ are free fit parameters and $d$ was fixed to 1, resulting in a Gaussian decay envelope. We found that $d=1$ gave the least $\chi$-squared in the fit compared to $d=0,2$. The tilted background was included to compensate for a slightly non-linear Rabi response. 

\subsection{Hahn echo decay time measurements of ASQ1 and ASQ2}\label{sec:hahn_decay}
To verify the slow nature of the noise causing dephasing, we performed Hahn-echo experiments with artificial detuning, shown in~\cref{fig:supplement-T2echo}. The resulting data was fit using~\cref{eq:coh-T_2_decay}. Note that for these measurement we found that the data was not always within the range of the identity and $\pi$-pulse calibration points. We suspect this is due to the additional echo pulse inducing leakage to other states outside the spin-subspace. We therefore normalized the data setting 0 and 1 to the minimum and maximum of the fit envelope at $\tau=0$ instead of using the calibration points as was done in the main text.  

% Figure environment removed

\subsection{Coherence properties of the transmon}
Although the transmon was used in this work to facilitate spin readout, we now demonstrate its coherence properties. \cref{fig:supplement-tmon-coherence} shows measurements of the transmon's Rabi oscillations, Ramsey coherence time and energy decay time. 

The transmon $T_1$ was found to be lower than that of previous implementations of a transmon using gate-tunable nanowires \cite{Larsen2015, Casparis2016, Luthi2018, Kringhoj2021, Bargerbos2023, PitaVidal2023}, which we suspect may be due to it being too strongly coupled to the flux-bias line or drive lines. This in turn limits the ASQs $T_1$ due to Purcell decay via the transmon.  Given a transmon $T_1$ of \SI{53.6}{ns}, we can estimate the limit set by Purcell decay for each ASQ. At their setpoints in Fig.~2 in the main text, the detunings from the transmon were \SI{1.7}{GHz} and \SI{2.2}{GHz} for ASQ1 and ASQ2, respectively. From measurements of the avoided crossing between the ASQ spin-flip and transmon transitions under similar conditions, we extract that the coupling strengths between transmon and ASQ are in the range \SIrange{50}{100}{MHz} for both qubits. These quantities allow us to estimate the Purcell limit of $T_1$ for both qubits to be 14-\SI{56}{\micro s} for ASQ2 and  23-\SI{92}{\micro s} for ASQ1. The higher harmonics of the transmon can reduce these lifetimes even more, especially for ASQ1 which resides close to the first higher harmonic. However, we did not measure the lifetime of that transition. 


% Figure environment removed


\FloatBarrier 


\subsection{Scaling of extracted T2* with pulse length}
Due to the short dephasing time of ASQs with respect to the pulse length, the pulse length influences the observed life time of the ASQs when the pulses are (partly) overlapping (see ~\cref{fig:supplement-pulse-length-dep}). This is the case because, during the part of the decay time $\tau$, the ASQ is being driven.  Therefore, care should be taken when pulses of length comparable to $T_2^*$ are used. In the main text we report values obtained using short \SI{4}{\nano\second} FWHM pulses. 

% Figure environment removed


\subsection{Single-shot readout contrast of individual ASQs}
In~\cref{fig:supplement-Readout-fidelity} we show examples of single-shot readout outcomes. These are measured at the setpoints used in the main text and at magnetic field settings of the main text for ASQ1 and for ASQ2 we go to a higher magnetic field in order to reduce thermal population of the excited spin state. We obtain an average signal-to-noise ratio for distinguishing spin-up and spin-down, based on double Gaussian fits to the up and down initializations $\textrm{SNR}=|\mu_\uparrow-\mu_\downarrow|/(2\sigma)$ where $\mu_i, \sigma$ are the mean and width of the fitted Gaussian corresponding to state $i$, of 1.5 and 1.3 in a integration time of $\SI{1}{\micro\second}$, $\SI{1.5}{\micro\second}$ for ASQ1 and ASQ2 respectively. Note that we use the fit parameters of the initialization without a $\pi$-pulse here as the pulse can cause excitations of other states, which we suspect to be the transmon excited states, visible as an extra tail in the Gaussian corresponding to the excited spin state in~\cref{fig:supplement-Readout-fidelity}b, c. Additionally, these values are strongly flux and magnetic field dependent and thus could be optimized further in future works, as we did not perform an exhaustive study here.  

The SNR is a  measurement of the pure readout contrast, rather than the more standard readout fidelity $F=1-P(\uparrow|\downarrow)/2 - P(\downarrow|\uparrow)/2$, where $P(a|b)$ denotes the probability of obtaining state $a$ when preparing state $b$. This is because $F$ includes the effects of thermal population and imperfect $\pi$-pulse, due to dephasing during the $\pi$-pulse and imperfect calibrations. At the main text gate setpoint, and the magnetic field settings mentioned above using the indicated threshold (black dashed line in~\cref{fig:supplement-Readout-fidelity}) we obtain $F=0.75$ and $F=0.67$ for ASQ1 and ASQ2 respectively.

% Figure environment removed
\FloatBarrier



\newpage
\newpage
\section{Supporting data for the longitudinal coupling measurements}\label{Sss:consistency-checks}

Fig.~\ref{fig:consistency_check_fp} shows data taken in the same way as in Fig.~3 of the main text and under the same field, gate and flux conditions, but for varying frequency of the pump tone $f_{\rm p}$. We find that, when $f_{\rm p}$ matches the transition frequency of one of the qubits, and thus continuously drives it to its excited state, the frequency of the other qubit splits in two, as discussed in the main text (red lines in Fig.~\ref{fig:consistency_check_fp}c and d). When the pump tone frequency instead does not match the transition frequency of the first qubit, the frequency of the second qubit does not split, as expected (black lines in Fig.~\ref{fig:consistency_check_fp}c and d). This confirms that the frequency splitting observed in the main text is indeed the result of both states of the other ASQ being populated.

In Fig.~\ref{fig:consistency_check_frequencies} we perform a similar experiment for two fixed pump frequencies away from the spin-flip transitions and as a function of the  pump tone power. For Fig.~\ref{fig:consistency_check_frequencies}a and b, the pump tone drives the transmon transition and for  Fig.~\ref{fig:consistency_check_frequencies}c and d its frequency is set to a value $f_{\rm p}=\SI{5.8}{\giga\hertz}$ where no transition is driven. In neither of the two cases do we observe a splitting of any of the two ASQ transitions at any power, as expected (the disappearance of the signal at high drive powers was generally seen throughout the work independent of drive frequency and corresponds to disappearance of the readout resonator resonance). Note that the instability that can be observed in the measured transition frequencies was also observed versus time and is thus unrelated to the presence of the pump tone.


% Figure environment removed


% Figure environment removed






\section{Longitudinal coupling at different gate sepoint} \label{s:new-dataset}

In this section, we present longitudinal coupling measurements similar to those in the main text, but now taken at a different gate configuration for both Andreev spin qubits. The new gate setpoints, at which the two qubits are set for all results discussed in this section, are indicated in Tab.~\ref{tab:setpoints_new}. 

\begin{table*}[h!]
\begin{ruledtabular}
\begin{tabular}{cccccc}
\textrm{}& $V_{{\rm L} i}$ (mV)  & $V_{{\rm P} i}$ (mV)  & $V_{{\rm R} i}$ (mV) & $E^I_{{\rm J}, i}/h$ (GHz) & $E^\sigma_{{\rm J}, i}/h$ (GHz) \\
\colrule
ASQ1 & 61.0 & 61.0 & 376.0 & 1.79 & 0.66  \\
ASQ2 & 53.0 & 0.0 & -700.0 & 0.53 & 1.3  \\
\end{tabular}
\end{ruledtabular}
\caption{
ASQ1 and ASQ2 gate voltage set points and extracted model parameters from the measurements in Fig.~\ref{fig:parameter-extraction_new} .
}
\label{tab:setpoints_new}
\end{table*}

We start by performing basic characterization measurements. The values of the spin-independent, $E^I_{{\rm J}, i}$, and spin-dependent, $E^\sigma_{{\rm J}, i}$, Josephson energies for both qubits are extracted from spin-flip and transmon spectroscopy measurements (see Fig.~\ref{fig:parameter-extraction_new}). Fig.~\ref{fig:parameter-extraction_new}a and b show in-field spectroscopy of the ASQ1 and ASQ2 spin-flip transitions, respectively. The values of $E^\sigma_{\rm J, 1}$ and $E^\sigma_{\rm J, 2}$ are extracted from fits to a skewed (Eq.~\ref{eq:sin_sk}) and a non-skewed (Eq.~\ref{eq:sin}) sinusoidal relation, respectively. Fig.~\ref{fig:parameter-extraction_new}c and d show transmon spectroscopy measurements performed at zero magnetic field, each with only one of the two ASQs open (ASQ1 in panel c and ASQ2 in panel d). 
The values of $E^I_{{\rm J}, i}$ are estimated by fitting the measured transmon transitions with Eq.~\ref{Eq:total-transmon-Hamiltonian} at $\Phi_i$ being integer multiples of $\Phi_0/2$.
 
% Figure environment removed


Before investigating the longitudinal coupling strength at the new gate setpoints, we characterize their magnetic field dependence. 
The characterization is done analogously to that for the previous gate setpoint (discussed around Fig.~\ref{fig:supplement-angle-dep}) and can be found in the data repository. The relevant extracted parameters are summarized in Tab.~\ref{tab:directions_new}. By performing spin-flip spectroscopy of each of the two qubits at different field directions, we extract their $g$-factors on the chip plane, which range between 6 and 15 for ASQ1 and between 9 and 15 for ASQ2. The values along the $B_z$ and $B_y$ directions are reported in Tab.~\ref{tab:directions_new}.
As before, we determine the spin-flip polarization direction for ASQ1 by finding two perpendicular directions in the $y-z$ and $x-z$ planes. The resulting spin-polarization direction is reported in Tab.~\ref{tab:directions_new} and is this time found to be approximately 1.85 degrees away from the nanowire axis. For ASQ2 we did not determine the full spin-orbit direction as we only data measured in the $y-z$ plane. 



\begin{table*}[h!]
\begin{ruledtabular}
\begin{tabular}{cccccccc}
\textrm{}& $g_z^{{\rm ASQ}i}$ & $g_y^{{\rm ASQ}i}$  &  $\theta_{||}$  & $\phi_{||}$ & $\alpha$ \\
\colrule
ASQ1 & 14.9 & 6.7 & 1.85 & 64.4 & 1.36 \\
ASQ2 & 14.1 & - & - & - & - \\
\end{tabular}
\end{ruledtabular}
\caption{ {\bf Summary of $g$-factors and relevant angles for ASQ1 and ASQ2 at their alternative setpoint.}  
 The zero-field spin-polarization direction ($\theta_{||}, \phi_{||}$ in spherical coordinates) is calculated as the vector product of the two perpendicular directions $\theta_{yz, \perp}$  and  $\theta_{xz, \perp}$.
 $\alpha$ denotes the angle between the field applied for the longitudinal coupling measurements of Fig.~\ref{fig:fig_longitudinal_combined}
 and the spin-polarization direction for each qubit.
}
\label{tab:directions_new}
\end{table*}

% Figure environment removed


Next, we measure the longitudinal coupling energy at the selected gate setpoints in the same way as for Fig.~3 in the main text. Fig.~\ref{fig:fig_longitudinal_combined}a-d show a longitudinal coupling measurement for fixed control parameters $B_z=$~\SI{25}{mT}, $\Phi_1=0.1\Phi_0$,  $\Phi_2=0.48\Phi_0$ and \Vc~=~\SI{180}{mV}. These parameters set $f_1=$~\SI{4.7}{GHz}, $f_2=$~\SI{5.3}{GHz} and $L_{\rm J, C} = $~\SI{22.3}{nH}. Similarly to Fig.~3 in the main text, we find that the frequency of each of the qubits splits by $2J$ when the other qubit is driven with a pump tone $f_{\rm p}$. From spectroscopy of ASQ2 while ASQ1 is driven (Fig.~\ref{fig:fig_longitudinal_combined}a, b), we find  $J=-110.0\pm3.2$~MHz, while from spectroscopy of ASQ1 while ASQ2 is driven (Fig.~\ref{fig:fig_longitudinal_combined}c, d), we find $-107.0\pm3.1$~MHz. These two values are equal within their one-sigma confidence intervals, consistent with the theory prediction.



Finally, we investigate the flux and $L_{\rm J, C}$ dependence of the coupling strength, similarly to how it is done in Fig.~4 of the main text. Fig.~\ref{fig:fig_longitudinal_combined}e, f show the tunability of $J$  at the setpoint of Tab.~\ref{tab:setpoints_new}. These measurements are taken at the same $B_z$ and $\Phi_2$ conditions as in Fig.~\ref{fig:fig_longitudinal_combined}a-d, which result on a fixed supercurrent difference through ASQ2, $I_{\rm 2}=$~\SI{-5.6}{nA}. Panel e shows the $\Phi_1$ dependence of $J$ at \Vc~=~\SI{180}{mV}, which fixes $L_{\rm J, C} = $~\SI{22.3}{nH}. Similarly to what was found in the main text, we observe that $J$ can take both positive and negative values 
and that it follows a similar shape as that predicted by Eq.~\ref{eq:J_current_derivatives}.
We however note that, especially around $\Phi_1=\Phi_0$,  the data deviates from the behavior predicted by Eq.~\ref{eq:J_current_derivatives}. This is due to the fact that this data is not taken for parameters consistent with the limit $L_{\rm J,C} \ll L_{{\rm J},i}^\sigma, L_{{\rm J},i}^I \forall i$. As shown in Fig.~\ref{fig:supplement-theory-realparameters}, when this limit is not met Eq.~\ref{eq:J_current_derivatives} overestimates the value of $J$ in the region of $\Phi_1 \sim \Phi_0$.

Finally, Fig.~\ref{fig:fig_longitudinal_combined}f shows the evolution of $J$ with  $L_{\rm J, C}$ at a fixed $\Phi_1=1.1 \Phi_0$ setpoint, indicated with a yellow marker in Fig.~\ref{fig:fig_longitudinal_combined}e, which sets $I_{\rm 1}=$~\SI{1.7}{nA}. As expected, we find an increase of the magnitude of $J$ with $L_{\rm J, C}$, which is proportional, with a scaling factor $A = 0.79 \pm 0.02$, to  Eq.~2 from the main text given $L_{\rm ASQ} = \frac{\Phi_0^2}{4\pi^2} / (E^I_{\rm J, 1}\cos(\frac{2\pi}{\Phi_0}\Phi_1)+E^I_{\rm J, 2}\cos(\frac{2\pi}{\Phi_0}\Phi_2))=$~\SI{176.9}{nH}.

\bibliography{bibliography_sup.bib}

\end{document}