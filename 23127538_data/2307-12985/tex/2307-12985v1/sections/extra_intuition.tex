%\section{Additional intuition for beta-binomial thinning when $b$ and $\epsilon b$ are both integers.}
%\label{appendix_integer_intuition}

%Recall that if $\by \sim \NB\left( \mu, b \right)$ and $b$ is an integer, then we can interpret $\by$ as the number of successes that occur in a sequence of i.i.d. $\mathrm{Bernoulli}\left(\frac{\mu}{b+\mu}\right)$ trials before $b$ failures occur. This means that $\by$ is the sum of $b$ independent $\text{Geom}\left(\frac{\mu}{b+\mu}\right)$ random variables. 


%If $\epsilon b$ and $(1-\epsilon) b$ are both integers, then we know that the 


%we can view the thinning procedure from Theorem~\ref{theorem_nbthin} as decomposing $\by$ into the $\bytr$ successes that occurred before the first $\epsilon b$ failures, and then the remaining $\byte$ successes. This interpretation lends itself to a derivation of the beta-binomial thinning procedure.





%Since $\bx=X$ is fixed, we know for sure 

%We imagine an urn with $b+X-1$ total elements in it, of which $X$ are labeled ``successes" and $b-1$ are labeled as failures. \textcolor{red}{do I need to explain why we only want $b-1$ and not $b$ in the urn? It's clear to me if I draw a more }


%Since $\bx=X$ is now fixed, to simulate which successes occurred before the first $b_1$ failures we should draw WITHOUT REPLACEMENT from this urn until we draw 

%When $b, \epsilon b$, and $(1-\epsilon)b$ are all integers, the distribution $\bb(X,\epsilon b, (1-\epsilon)b)$ is the \emph{negative hypergeometric} distribution with parameters $b+X-1$, $X$, and $\epsilon b$. \textcolor{red}{do I need a source that is not wikipedia + my pen and pencil???} The interpretation of drawing $\bxtr \mid bx = X \sim \mathrm{NegHyp(b+X-1, X, \epsilon b})$ is clear. 
