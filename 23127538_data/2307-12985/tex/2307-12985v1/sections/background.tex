\section{Background}
\label{section_background_nb}

\subsection{A review of data thinning}
\label{subsec_csreview}

Our goal is to decompose an scRNA-seq dataset $X \in \mathbb{Z}_{\geq 0}^{n \times p}$ into independent datasets  $\xtr \in \mathbb{Z}_{\geq 0}^{n \times p}$ and $\xte \in \mathbb{Z}_{\geq 0}^{n \times p}$ drawn from the same model as $X$, up to a parameter scaling. 
We first review \emph{Poisson count splitting}, which accomplishes this goal if $\bold{X}_{ij} \overset{\mathrm{ind.}}{\sim} \mathrm{Poisson}(\Lambda_{ij})$. 

\begin{algorithm}[Poisson count splitting]
\label{alg_cs}
Let $X \in \mathbb{Z}_{\geq 0}^{n \times p}$. For a chosen $M \in \mathbb{Z}^{+}$ and $\epsilon_1,\ldots,\epsilon_M \in (0,1)$ such that $\sum_{m=1}^M \epsilon_m = 1$, draw $\left(\bold{X}_{ij}^{(1)},\bold{X}_{ij}^{(2)}, \ldots, \bold{X}_{ij}^{(M)}\right) \mid \bold{X}_{ij} = X_{ij} \sim \mathrm{Multinomial}\left(X_{ij}, \epsilon_1,\ldots, \epsilon_M\right)$.
\end{algorithm}

\begin{theorem}
\label{theorem_poithin}	
Let $X \in \mathbb{Z}_{\geq 0}^{n \times p}$ be a dataset with entries $X_{ij}$ drawn from  $\bold{X}_{ij} \overset{\mathrm{ind.}}{\sim} \poi(\mu_{ij})$. If we apply Algorithm~\ref{alg_cs} to each element $X_{ij}$ of this data, then
(1) $\bxm_{ij} \overset{\mathrm{ind.}}{\sim} \mathrm{Poisson}(\epsilon_m \mu_{ij})$ for $i=1,\ldots,n$, $j=1,\ldots,p$, and (2) the folds $\bxo, \ldots, \bxM$ are mutually independent.
\label{theorem_poithin}
\end{theorem}

Theorem~\ref{theorem_poithin} follows from the well-known binomial thinning property of the Poisson distribution (see \citealt{durrett2019probability}, Section 3.7.2). For any fold $m \in \{ 1,\ldots,M\}$, we define $\bxmm = \bx - \bxm$. The mutual independence between folds given in Theorem~\ref{theorem_poithin} ensures that, for all $m \in \{ 1,\ldots,M\}$, $\bxm$ is independent of $\bxmm$. Thus, by treating $\bxmm$ as a training set and $\bxm$ as a test set, we arrive at an alternative to cross-validation that uses Poisson count splitting, rather than sample splitting, to create training and test sets. 

\cite{sarkar2021separating} and \cite{neufeld2022inference} apply Poisson count splitting to overcome the challenges arising in Examples 1 and 2 of Section~\ref{section_intro_nb}, under the assumption that the scRNA-seq data follows a Poisson distribution. Unfortunately, the 
Poisson assumption in Theorem~\ref{theorem_poithin} is necessary to achieve independence between the folds $\bxo, \ldots, \bxM$.  If the data instead follow a negative binomial distribution, then $\bxo, \ldots, \bxM$ are correlated, and Poisson count splitting will fail to provide a valid approach for model evaluation or inference. %(see Theorem~\ref{theorem_nb_binom_thin}). 

\cite{neufeld2023data} and \cite{dharamshi2023generalized} expand upon Poisson count splitting to describe a general recipe for decomposing a single random variable $\bold{Y} \sim F_{\theta}$, for some distribution $F_\theta$ indexed by an unknown parameter $\theta$, into independent pieces $\bold{Y}^{(m)} \overset{\mathrm{ind.}}{\sim} Q^{(m)}_{\theta}$, where $Q^{(m)}_\theta$ is a (possibly different) distribution indexed by the same parameter $\theta$. They refer to this general framework as \emph{data thinning}. In particular, the framework developed by \cite{neufeld2023data} enables us to decompose a negative binomial random variable into $M \geq 2$ independent negative binomial random variables, without knowledge of the mean parameter. While the property of the Poisson distribution that allows for Poisson count splitting is well-known, its negative binomial counterpart is less well-known. In fact, prior to \cite{neufeld2023data}, thinning the negative binomial distribution appears to be unexplored outside of the time series literature of the 1990s \citep{joe1996time}. 

%\vspace{-50mm}

\subsection{Negative binomial models for scRNA-seq data}
\label{subsec_nb}

Throughout this paper, we let $\NB(\mu, b)$ denote the negative binomial distribution with mean $\mu$ and variance $\mu + \frac{\mu^2}{b}$, for $\mu > 0$ and $b > 0$. This parameterization is commonly used when modeling scRNA-seq data. Note that if $\by \mid \tau \sim \mathrm{Poisson}(\mu \tau)$ and $\tau \sim \mathrm{Gamma}(b,b)$, then $\by \sim \NB(\mu, b)$.  Because the variance $\mu + \frac{\mu^2}{b}$ is always strictly larger than the mean $\mu$, the negative binomial model is overdispersed relative to the Poisson model. This motivates its use in the analysis of RNA sequencing data, where the data are non-negative integers with excess variance relative to the Poisson distribution \citep{choudhary2022comparison, hafemeister2019normalization, sarkar2021separating, lopez2018deep}. We refer to the parameter $b$ as the overdispersion parameter. As $b \rightarrow \infty$, the negative binomial distribution approaches the Poisson distribution. In the scRNA-seq literature, it is common to assume that each gene, but not each cell, has its own overdispersion parameter \citep{hafemeister2019normalization, love2014moderated, lopez2018deep}. Thus, in what follows, we will assume that $\bold{X}_{ij} \overset{\mathrm{ind.}}{\sim} \mathrm{NB}\left( \mu_{ij}, b_j\right)$. 


