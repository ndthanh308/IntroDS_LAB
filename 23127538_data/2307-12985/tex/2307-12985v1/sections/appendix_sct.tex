\section{Implementation details for Section~\ref{section_simulation_nb}}
\label{appendix_sct}

We use the \texttt{R} package \texttt{sctransform} in the following manner to estimate gene-specific overdispersion parameters in the simulations for Section~\ref{section_simulation_nb}. 

Briefly, for each gene $j=1,\ldots,p$, \texttt{sctransform} begins by fitting a negative binomial GLM with $\bold{X}_j$ as the response, and the logged total number of unique molecular identifiers (UMIs) for the $n$ cells as the covariate. (In our simulations, the total number of UMIs for a cell is the row sum for that cell.) This yields a maximum likelihood estimate $\hat{b}_j^\mathrm{MLE}$ for each gene $j=1,\ldots,p$. These maximum likelihood estimates are known to be quite noisy for sparse negative binomial data.  Furthermore, the ``null model" that only includes the total number of UMIs as a covariate may be the correct model for 
the majority of the genes, but will be incorrect for any genes that exhibit true differential expression across unknown latent variables. Thus, as a second step,  \texttt{sctransform} fits a smooth kernel regression to estimate a relationship between the average expression of each gene and the gene-specific overdispersion. These smoothed estimates are used as the gene-specific overdispersion parameters. 

We run the \texttt{vst()} function from the \texttt{sctransform} package in \texttt{R} with its default settings in Section~\ref{section_simulation_nb}. We note that we simulated data in which the two main assumptions of \texttt{sctransform} are met: most genes are not differentially expressed, and there is a smooth relationship between the average expression of a gene and its parameter $b_j$. We use the same strategy in Section~\ref{section_realData_nb}, although in the real data setting we do not know for sure that the modeling assumptions of \texttt{sctransform} are met. 
