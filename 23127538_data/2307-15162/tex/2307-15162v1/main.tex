% ****** Start of file apssamp.tex ******
%
%   This file is part of the APS files in the REVTeX 4.2 distribution˜.
%   Version 4.2a of REVTeX, December 2014
%
%   Copyright (c) 2014 The American Physical Society.
%
%   See the REVTeX 4 README file for restrictions and more information.https://www.overleaf.com/project/5bffa8faee1fda2bffb5043d
%
% TeX'ing this file requires that you have AMS-LaTeX 2.0 installed
% as well as the rest of the prerequisites for REVTeX 4.2
%
% See the REVTeX 4 README file
% It also requires running BibTeX. The commands are as follows:
%
%  1)  latex apssamp.tex
%  2)  bibtex apssamp
%  3)  latex apssamp.tex
%  4)  latex apssamp.tex
%
\documentclass[%
reprint,
superscriptaddress,
%groupedaddress,
%unsortedaddress,
%runinaddress,
%frontmatterverbose, 
%preprint,
%preprintnumbers,
%nofootinbib,
%nobibnotes,
%bibnotes,
 amsmath,amssymb,
 aps,
%pra,
%prb,
%rmp,
%prstab,
%prstper,
floatfix,
]{revtex4-2}
\usepackage[english]{babel}
\usepackage{comment}
\usepackage{amsmath}
\usepackage{natbib}
\usepackage{ulem}
\usepackage{graphicx}% Include figure files
\usepackage{dcolumn}% Align table columns on decimal point
\usepackage{bm}% bold math
\usepackage{amsmath}
\usepackage{siunitx}
\usepackage{listings}
\usepackage{mathtools}
\usepackage{float}
\usepackage{dsfont}
%\usepackage{hyperref}% add hypertext capabilities
%\usepackage[mathlines]{lineno}% Enable numbering of text and display math
%\linenumbers\relax % Commence numbering lines

%\usepackage[showframe,%Uncomment any one of the following lines to test 
%%scale=0.7, marginratio={1:1, 2:3}, ignoreall,% default settings
%%text={7in,10in},centering,
%%margin=1.5in,
%%total={6.5in,8.75in}, top=1.2in, left=0.9in, includefoot,
%%height=10in,a5paper,hmargin={3cm,0.8in},
%]{geometry}

\usepackage{xcolor}
\usepackage{url}
%\usepackage[colorlinks=true,linkcolor=blue]{hyperref}
\usepackage{hyperref}
\hypersetup{
    colorlinks=true,allcolors=blue,}

\begin{document}
\preprint{APS/123-QED}

\title{Majorana bound states in $d$-wave superconductor planar Josephson junction}
% AUTHORS % % % % % % % % % % % %
\author{Hamed Vakili}
\affiliation{Department of Physics and Astronomy and Nebraska Center for Materials and Nanoscience, University of Nebraska, Lincoln, Nebraska 68588, USA}
\author{Moaz Ali}
\affiliation{Department of Physics and Astronomy and Nebraska Center for Materials and Nanoscience, University of Nebraska, Lincoln, Nebraska 68588, USA}
\author{Mohamed Elekhtiar}
\affiliation{Department of Physics and Astronomy and Nebraska Center for Materials and Nanoscience, University of Nebraska, Lincoln, Nebraska 68588, USA}
\author{Alexey A. Kovalev}
\affiliation{Department of Physics and Astronomy and Nebraska Center for Materials and Nanoscience, University of Nebraska, Lincoln, Nebraska 68588, USA}
\date{\today}% It is always \today, today,
             %  but any date may be explicitly specified

\begin{abstract}
We study phase-controlled planar Josephson junction comprising a two-dimensional electron gas with strong spin-orbit coupling and $d$-wave superconductors, which have an advantage of high critical temperature. We show that a region between the two superconductors can be tuned into topological state by the in-plane Zeeman field, and can host Majorana bound states. The phase diagram as a function of the Zeeman field, chemical potential, and the phase difference between superconductors exhibits the appearance of robust Majorana bound states for a wide range of parameters. We further investigate the behavior of the topological gap and its dependence on the type of $d$-wave pairing, i.e., $d$, $d+is$, or $d+id^\prime$, and note the difficulties that can arise due to the presence of gapless excitations in pure $d$-wave superconductors. On the other hand, the planar Josephson junctions based on superconductors with $d+is$ and $d+id^\prime$ pairings can potentially lead to realizations of Majorana bound states. Our proposal can be realized in twisted bilayer $d$-wave superconductors realizable in mechanically exfoliated van der Waals copper oxide heterostructures. 
\end{abstract}
\pacs{Valid PACS appear here}
\maketitle 

Majorana bound states (MBS) are non-abelian anyons that can be used in realizations of quantum computers relying on topological protection~\cite{Nayak2008,Kitaev2003}. Anyons are quasiparticles that are described by the statistics of neither fermions nor bosons and have exotic properties such as fractional charge~\cite{Nayak2008,Kitaev2003}. Topological protection associated with realizations of non-abelian anyons in condensed matter systems can be used to encode quantum information in a way that is robust against decoherence~\cite{Alicea2012,Beenakker2013,Elliott2015}. The current common platforms to realize MBS are relying on proximity effect~\cite{FuKane,Eschrig2019,Zutic2019,Gngrd2022} with $s$-wave superconductors which have low critical temperatures~\cite{Alicea2012,Beenakker2013,Elliott2015}. The topological gap in such realizations is relatively small, making the system more sensitive to disorder~\cite{Takei2013,Zhang2019} and various imperfections~\cite{Rainis2012,Stanescu2014}, and requiring operation at very low temperatures~\cite{pikulin_protocol_2021,aghaee_inas-hybrid_2022,Huo_2023}.

The $d$-wave superconductivity is a very common type of superconductivity in strongly correlated systems such as cuprates~\cite{Yanase2003}. Such superconductors are associated with gapless excitations~\cite{Sato2017} and higher critical temperatures compared to $s$-wave superconductors. Unfortunately, the presence of gapless excitations is not compatible with topological superconductivity. Furthermore, in $d$-wave superconductors, gapless excitations lead to the appearance of Andreev bound states (ABS)~\cite{Adagideli1999,Chen2001,Sengupta2001,Tanaka2012}. 
To resolve issues associated with gapless excitations, one can use $d$-wave superconductors with inversion asymmetry and applied magnetic field~\cite{MZMTI,dwave2deg,Liu2018,magnetchain,Yanase2022}, or use a twisted bilayer of $d$-wave superconductors that behaves as an effective $d+id'$ superconductor~\cite{Liu2018,Crawford2020,Mercado2022,Margalit2022,Can2021}. As has been demonstrated in Ref.~\cite{Can2021}, a twisted bilayer of $d$-wave superconductors offers high degree of tunability via variations in twist angle and offers realization of $d+is$ or $d+id^\prime$ pairings within the same system. The planar Josephson junction (JJ) platform has been studied for $s$-wave superconductors, and the appearance of robust tunable MBS has been predicted~\cite{JJ2017,Halperin2017,Zhou2022}. Recently, there has been substantial interest in various realizations of planar JJ based on $s$-wave superconductors in the context of superconducting diode effect~\cite{Ando2020,Mayer2020,Dartiailh2021,Baumgartner2021,Amundsen2022}. It has been demonstrated that the superconducting diode effect also appears in a JJ based on $d$-wave superconductors and a topological insulator~\cite{Tanaka2022}. 

In this work, we demonstrate that a planar JJ comprising a two-dimensional electron gas (2DEG) with strong spin-orbit coupling and $d$-wave superconductors can host MBS. Such systems can be realized using
exfoliated van der Waals copper oxide heterostructures combined with layered semiconductors, such as Bi$_2$O$_2$Se~\cite{Wu2017,Margalit2022}. In the case of pure $d$-wave pairing, we observe that MBS coexist with gappless bulk states which can be detrimental for storing quantum information. However, for the $d+is$ and $d+id^\prime$ pairings there exists a bulk gap and MBS arise in the planar JJ geometry. The $d+is$ pairing leads to realizations of robust MBS in analogy to JJ based on $s$-wave superconductor~\cite{Halperin2017}. For $d+id^\prime$ pairing in addition to MBS, the chiral edge modes exist at zero energy in the proximized 2DEG due to nontrivial bulk topology; however, they can be gapped out inside JJ by properly tuning the Zeeman field, the relative phase of superconductors, and the chemical potential. As a result for $d+id^\prime$ pairing, one can physically separate the MBS and the chiral edge modes, as we show in our study.
 
%% Figure environment removed

% Figure environment removed

\textit{Model and symmetry analysis.} -- We consider the planar JJ geometry with an in-plane Zeeman field (due to magnetic field or induced by proximity with a ferromagnet), and mostly assume the limit of large superconductors. When necessary, we use periodic boundaries along the 
$y$ axis in Fig.~\ref{fig:MBS}a). The phase difference of the left and right $d$-wave superconductors can be controlled by applying current or magnetic flux.
The Bogoliubov–de Gennes (BdG) Hamiltonian of the system, written in the Nambu basis ($\psi_\uparrow,\psi_\downarrow,\psi^\dagger_\downarrow,-\psi^\dagger_\uparrow$) reads 
\begin{align}
H &=\left(\frac{\boldsymbol{p}^2}{2m^*} -\mu+\frac{\alpha_{SO}}{\hbar}\left(\boldsymbol{\sigma} \times \boldsymbol{p}\right)_z +\frac{m^*}{2}\alpha_{SO}^2\right) \tau_{z}\nonumber\\  
&+ h(x) \sigma_y+\tau_x \text{Re}\Delta(\boldsymbol k,x) -\tau_y\text{Im}\Delta(\boldsymbol k,x).\label{eq:Hamiltonian}
\end{align}
Here, $\alpha$ is the Rashba spin-orbit coupling strength, $h(x)$ describes the Zeeman energy, e.g., for an external magnetic field $h(x)=g(x)\mu_B B/2$ where $g(x)$ is the position dependent $g$-factor. We denote the Zeeman energy in the superconducting regions by $h_{SC}$ and in the junction by $h_J$, $m^*$ is the effective electron mass, and $\Delta(\boldsymbol k,x)$ is the proximity-induced pairing in 2DEG, which reads
\begin{align}
    \Delta(\boldsymbol k,x) &=  2e^{i\phi(x)} [\Tilde{\Delta}(x)\left(\cos{k_y}- e^{i\beta}\cos{k_x}\right)\nonumber \\ &+\Tilde{\Delta}^\prime(x) 2i \sin{k_x} \sin{k_y}].
\end{align}
Here $\beta$ determines the pairing type, e.g., $\beta = 0$ corresponds to $d_{x^2-y^2}$ pairing and $\beta =\pi/4$  corresponds to $d+is$ pairing~\cite{Can2021}. The term $\Tilde{\Delta}^\prime$ is necessary to realize $d+id^\prime$ pairing. The terms $\Tilde{\Delta}(x)$ and $\Tilde{\Delta}^\prime(x)$ are only nonzero in the regions covered by superconductors where $\Tilde{\Delta}(x)=\Delta_0$, $\Tilde{\Delta}^\prime(x)=\Delta_0^\prime$, $\phi(x)=\phi_L$ for the region covered by the left superconductor, and $\phi(x)=\phi_R$ for the region covered by the right superconductor. We apply the Zeeman field in the $y$-direction. 

The symmetries, such as the time reversal symmetry, the particle-hole symmetry, and the chiral symmetry determine the type of topological superconductivity realizable in our system.
The Hamiltonian~\eqref{eq:Hamiltonian} breaks time-reversal symmetry when the Zeeman field is present. For the case of $d$ pairing (or $d+id'$ pairing when $\Delta^\prime \neq 0$) in the presence of the Zeeman field, we can still define an effective time reversal symmetry as $\Tilde{T}=M_x T$ where $T=i\sigma_y K$ is the usual time reversal symmetry and $M_x$ is the mirror with respect to the $y\mbox{-}z$ plane. For $\Tilde{T}$, we have a relation $\Tilde{T}^2=1$, which in combination with the particle-hole symmetry, $P=\sigma_y\tau_y K$, and the chiral symmetry (the $P\Tilde{T}$ symmetry operator), $\Tilde{C}=M_x \tau_y$, places such system in the BDI symmetry class with a $\mathbb{Z}$ topological invariant. In our analysis, we also calculate $\mathbb{Z}_2$ Pfaffian invariant which is determined by the parity of $\mathbb{Z}$ and characterizes systems in the D symmetry class. The D symmetry class is realized in Fig.~\ref{fig:MBS} when the mirror symmetry with respect to the $y\mbox{-}z$ plane is broken or for $d+is$ pairing.
 
\textit{Topological superconductivity and MBS.} -- Our symmetry analysis suggests a possibility for topological superconductivity in the quasi-one-dimensional region between superconductors in Fig.~\ref{fig:MBS}a). A sufficiently large but finite system along the $y$-axis can be used for realizing MBS. We use the tight-binding version of the BdG Hamiltonian \eqref{eq:Hamiltonian} on a two-dimensional square lattice given by:
\begin{align}
    H_{TB} = &\sum_{<ij,mn>}\left[-t ~c^\dagger_{i,j} c_{m,n} + it_{SO}~c^\dagger_{i,j}\left( \mathbf{\sigma} \times \mathbf{r}_{ij,mn}\right)c_{m,n}\right]\nonumber\\
    &+ \sum_{i,j}\left[(4t-\mu+t_{SO}^2) ~c^\dagger_{i,j} c_{i,j}+ h~c^\dagger_{i,j} \sigma_y~c_{i,j}   \right]\nonumber\\
    &+ \sum_{i,j}e^{i\phi/2}\Tilde{\Delta}\left[~c^\dagger_{i,j\pm 1,\uparrow} c^\dagger_{i,j,\downarrow}-~e^{i\beta}c^\dagger_{i\pm 1,j,\uparrow} c^\dagger_{i,j,\downarrow}\right]+\nonumber\\
    &\sum_{i,j}e^{i\phi/2}\Tilde{\Delta}^\prime\left[~c^\dagger_{i\pm1,j\pm1,\uparrow} c^\dagger_{i,j,\downarrow}-~c^\dagger_{i\pm1,j\mp 1,\uparrow} c^\dagger_{i,j,\downarrow}\right]\nonumber\\
    & + H.c., \label{eq:TBHamiltonian}
\end{align}
where in our calculations we use the lattice spacing $a_c = 10$~nm, $t_{SO}=0.3t$, $t = \frac{\hbar^2}{2m^* a_c^2}$, and $m^*$= $0.05 m_e$, with $m_e$ being the electron rest mass. We study the phase diagram as a function of the Zeeman field, chemical potential, and the phases of superconductors. To describe a system with periodic boundaries along the $y$ axis, we use the momentum representation along the $y$-axis of the tight-binding Hamiltonian, $H_\text{TB}(k_y)$.
To calculate the BDI class  $\mathbb{Z}$ topological invariant, we employ the eigen basis of the chiral symmetry in which $\Tilde{C}$ is diagonal~\cite{Sau2012,Halperin2017} with $\mathds{1}$ in the upper left block and $-\mathds{1}$ in the lower right block. In this basis,
\begin{equation}
\Tilde{H}_\text{TB}(k_y)=
\begin{pmatrix}
0 & A(k_y)\\
A^\dagger(k_y) & 0
\end{pmatrix},  
\end{equation}
where for a gapped Hamiltonin $A(k_y)$ defines a complex function $z(k_y)=Det(A(k_y))/|Det(A(k_y))|$ and a winding number $W=(-i/2\pi)\int_{k_y=0}^{k_y=2\pi}dz(k_y)/z(k_y)$. To calculate the D class $\mathbb{Z}_2$ topological invariant, we use the expression $Q = \text{sign}(\text{Pf}(H_{k_y=\pi}\sigma_y\tau_y)/\text{Pf}(H_{k_y=0}\sigma_y\tau_y))$.
The two topological numbers are related by equation $(-1)^W = Q$~\cite{Sau2012}.

\textit{Planar JJ with $d$-wave pairing.} --
We first study the planar JJ in Fig.~\ref{fig:MBS}a) with pure $d$-wave pairing by varying the chemical potential, the phase difference ($\Delta\phi = \phi_R - \phi_L$), and the Zeeman field. We use parameters: $W_J$ = 5$a_c$, $W_L=W_R$ = 20$a_c$, $\Delta_0 = 0.3$t, $h_{SC}=0$, $\beta=0$, and $\Delta^\prime_0 = 0.0$.
The nontrivial topological regions in parameter space, as determined by $Q$, are reminiscent of the planar JJ based on the $s$-wave superconductor, which suggests the same mechanism for the formation of MBS and the relevance of the Andreev reflection~\cite{Halperin2017}.  
In Fig.~\ref{fig:MBS}b), we show the energy gap of the system as a function of $\mu$ and $h_J$. The gap shows rapid changes, exhibiting numerous lines with zero gap, which suggests that this is a finite size effect associated with the gapless excitations in the bulk of 2DEG. According to Fig.~\ref{fig:MBS}c), the gap closes along the lines that seem to be associated with changes in the two topological invariants, $W$ or $Q$, which are calculated from $H_\text{TB}(k_y)$. From decrease of the gap for larger system sizes, we conclude that in the region with $Q = -1$ MBS will coexist with gapless excitations in the planar JJ based on the $d$-wave pairing. Even though the topological invariant $Q$ shows large continuous region with $Q=-1$ and well localized MBS can be realized as shown in Fig.~\ref{fig:MBS}b), MBS may still hybridize with the gapless excitations. This behavior can be detrimental for quantum coherence associated with MBS. Nevertheless, we expect that the signature of $Q = -1$ region can be seen in studies of the Josephson current and the superconducting diode effect.  

% Figure environment removed
 
\textit{Planar JJ with $d+is$ pairing.} --
Since $d$-wave superconductors have directional pairing resulting in $d_{x^2-y^2}$ and $d_{xy}$ components in a general reference frame, the orientation of the superconductor lattices can be used to control the pairing potential  in twisted bilayer $d$-wave superconductors realizable in mechanically exfoliated van der Waals copper oxide heterostructures. It has been shown~\cite{Can2021} that by having a twisted bilayer of $d$-wave superconductors, $d+is$, and $d+id^\prime$ pairings can be realized. In Fig.~\ref{fig:Z2}, we consider a system with $d+is$ pairing and use parameters: $h_{SC} = 0$, $\Delta_0 = 0.3t$, $\Delta^\prime_0 = 0.0$, $W_J$ = 5$a_c$, $W_L=W_R$ = 40$a_c$. We take $\beta =\pi/4$ predicted by the ab-initio calculations~\cite{Can2021}. In Fig.~\ref{fig:Z2}a), we calculate the phase diagram for $Q$ as a function of $h_J$ and $\phi$. The large diamond-shaped region, also typical for the planar JJ based on the $s$-wave superconductor~\cite{Halperin2017}, defines parameters for which MBS can be realized. In Fig.~\ref{fig:Z2}b), we plot the gap. The gap closes at a line of quantum phase transition between the $Q=-1$ and $Q=1$ regions. In Figs.~\ref{fig:Z2}c) and d), we perform similar calculations as a function of $\mu$ and $h_J$ with fixed $\phi = \pi$. Results in Fig.~\ref{fig:Z2} predict robust MBS in the planar JJ with $d+is$ pairing for a wide range of parameters. We also observe in Fig.~\ref{fig:coverB} that the region with $Q=-1$ persists even when the Zeeman energy is present in the superconducting regions. Figure~\ref{fig:coverB}a) shows the phase diagram for $Q$ as a function of $\mu$ and $h_J = h_{SC} = h$. A relatively large region corresponding to $Q=-1$ in Fig.~\ref{fig:coverB}a) can lead to realizations of robust MBS due to a sizable topological gap shown in Fig.~\ref{fig:coverB}b). In Figs.~\ref{fig:coverB}c) and d), 
 we show that increasing the ratio $h_{SC}/h_{J}$ expands the range of $\phi$ where MBS can be realized for certain parameters. 

% Figure environment removed

\textit{Planar JJ with $d+id^\prime$ pairing.} -- 
We consider $d+id^\prime$ pairing in Fig.~\ref{fig:d+id}. The phase diagram for $Q$ and $W$ as a function of $\mu$ and $h_J$ in Fig.~\ref{fig:d+id}a) looks somewhat similar to the case of $d$-wave pairing in Fig.~\ref{fig:MBS}d). The $id^\prime$ component of the pairing potential breaks the time reversal symmetry and removes the gapless states in the bulk of 2DEG. At the same time, the topology of the bulk of the proximized 2DEG corresponds to the even Chern number and results in the chiral edge modes~\cite{Sato2010} in the system in Fig.~\ref{fig:MBS}a). A finite size effect associated with the gapless excitations of the chiral edge modes of 2DEG results in gap closings and  changes in $W$ shown in Fig.~\ref{fig:d+id}a), where periodic boundary along the $y$ axis in Fig.~\ref{fig:MBS}a) has been used. In Fig.~\ref{fig:d+id}c), we show the phase diagram for $W$ as a function of $\phi$ and $h_J$, where one can identify a diamond-shaped region with $Q=-1$. Results in Figs.~\ref{fig:d+id}a) and c) suggest that MBS will coexist with the chiral edge modes in system in Fig.~\ref{fig:MBS}a). If not separated spatially, the chiral edge modes can hybridize with MBS. In Figs.~\ref{fig:d+id}b) and d), we study the gap inside of the planar JJ in Fig.~\ref{fig:MBS}a) with the periodic boundary along the $y$ axis where the outer edge modes have been removed. Figures~\ref{fig:d+id}b) and d) show that the planar JJ with the Zeeman term can gap out the edge modes running along the JJ for a large range of parameters.  Using Figs.~\ref{fig:d+id}b) and d), we suggest a setup in Fig.~\ref{fig:d+id}e) with an angled JJ that allows to separate the edge modes from MBS by using four superconducting regions. Using parameters $h_J=0.4t$, $h_{SC} = 0$, $\mu = 2t$, $\Delta_0 = 0.3t$, $\Delta^\prime_0 = 0.06t$, and phases of four superconducting regions $\phi_1=\phi_2=\pi/2$, $\phi_3=\phi_4 = -\pi/2$ for setup in Fig.~\ref{fig:d+id}e), we are able to realize robust MBS, as shown in Fig.~\ref{fig:d+id}f) by plotting the $|\psi|^2$ of the lowest eigenvalue of the system. 


% Figure environment removed

\textit{Conclusions.} -- 
We have studied the planar JJ comprising a 2DEG with strong spin-orbit coupling and $d$-wave superconductors. The proximity effect can induce the superconducting pairing potential in 2DEG with the same symmetry as the host superconductor. Here, we have considered different types of $d$-wave superconductors with high critical temperature and large intrinsic gap. Apart from superconductors with pure $d$-wave pairing, we have also considered a twisted bilayer $d+is$, and $d+id^\prime$ superconductors realizable in mechanically exfoliated van der Waals copper oxide heterostructures~\cite{Can2021}. 
In case of $d$-wave pairing, we have demonstrated that the planar JJ can lead to MBS for a wide range of parameters, in analogy to realizations based on $s$-wave superconductors; however, the presence of gapless excitations in the bulk of 2DEG may hinder the quantum coherence. Nevertheless, we expect interesting manifestations in the superconducting diode effect. In case of $d+is$ pairing, we have demonstrated realizations of robust MBS for a wide range of parameters. 
In the case of $d+id^\prime$ pairing there are no gapless states in the bulk of 2DEG; however, the even Chern number associated with the bulk leads to appearance of gapless chiral edge modes, which can hybridize with MBS. To realize MBS with $d+id^\prime$ pairing, we have proposed a modified JJ in which the chiral edge modes are gapped and do not hybridize with MBS. It would be interesting to consider generalizations of our ideas to 2DEG with cubic Rashba interactions~\cite{Alidoust2021,Luethi2023}.

The cuprate-based superconductors have shown critical temperatures of up to 133 K~\cite{Keimer2015}. 
One direct advantage of using cuprate-based superconductors is that MBS can exist at higher temperatures compared to realizations based on pure $s$-wave superconductors. Another advantage stems from much larger intrinsic gap which should result in a larger proximity-induced topological gap and better protection against disorder and thermal excitations~\cite{Court_2008,Wang2013,Yu2019}. Our results should help in identifying new platforms for realizations of robust and easily tunable MBS with better protection against decoherence.

\textit{Acknowledgments.} --
This work was supported by the U.S. Department of Energy, Office of Science, Basic Energy Sciences, under
Award No. DE-SC0021019.


\bibliography{main}
\bibliographystyle{apsrev4-1}
\end{document}

