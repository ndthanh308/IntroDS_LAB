%% bare_conf.tex
%% V1.4b
%% 2015/08/26
%% by Michael Shell
%% See:
%% http://www.michaelshell.org/
%% for current contact information.
%%
%% This is a skeleton file demonstrating the use of IEEEtran.cls
%% (requires IEEEtran.cls version 1.8b or later) with an IEEE
%% conference paper.
%%
%% Support sites:
%% http://www.michaelshell.org/tex/ieeetran/
%% http://www.ctan.org/pkg/ieeetran
%% and
%% http://www.ieee.org/

%%*************************************************************************
%% Legal Notice:
%% This code is offered as-is without any warranty either expressed or
%% implied; without even the implied warranty of MERCHANTABILITY or
%% FITNESS FOR A PARTICULAR PURPOSE! 
%% User assumes all risk.
%% In no event shall the IEEE or any contributor to this code be liable for
%% any damages or losses, including, but not limited to, incidental,
%% consequential, or any other damages, resulting from the use or misuse
%% of any information contained here.
%%
%% All comments are the opinions of their respective authors and are not
%% necessarily endorsed by the IEEE.
%%
%% This work is distributed under the LaTeX Project Public License (LPPL)
%% ( http://www.latex-project.org/ ) version 1.3, and may be freely used,
%% distributed and modified. A copy of the LPPL, version 1.3, is included
%% in the base LaTeX documentation of all distributions of LaTeX released
%% 2003/12/01 or later.
%% Retain all contribution notices and credits.
%% ** Modified files should be clearly indicated as such, including  **
%% ** renaming them and changing author support contact information. **
%%*************************************************************************


% *** Authors should verify (and, if needed, correct) their LaTeX system  ***
% *** with the testflow diagnostic prior to trusting their LaTeX platform ***
% *** with production work. The IEEE's font choices and paper sizes can   ***
% *** trigger bugs that do not appear when using other class files.       ***                          ***
% The testflow support page is at:
% http://www.michaelshell.org/tex/testflow/



\documentclass[conference]{IEEEtran}

% Some Computer Society conferences also require the compsoc mode option,
% but others use the standard conference format.
%
% If IEEEtran.cls has not been installed into the LaTeX system files,
% manually specify the path to it like:
% \documentclass[conference]{../sty/IEEEtran}





% Some very useful LaTeX packages include:
% (uncomment the ones you want to load)


% *** MISC UTILITY PACKAGES ***
%
%\usepackage{ifpdf}
% Heiko Oberdiek's ifpdf.sty is very useful if you need conditional
% compilation based on whether the output is pdf or dvi.
% usage:
% \ifpdf
%   % pdf code
% \else
%   % dvi code
% \fi
% The latest version of ifpdf.sty can be obtained from:
% http://www.ctan.org/pkg/ifpdf
% Also, note that IEEEtran.cls V1.7 and later provides a builtin
% \ifCLASSINFOpdf conditional that works the same way.
% When switching from latex to pdflatex and vice-versa, the compiler may
% have to be run twice to clear warning/error messages.






% *** CITATION PACKAGES ***
%
%\usepackage{cite}
% cite.sty was written by Donald Arseneau
% V1.6 and later of IEEEtran pre-defines the format of the cite.sty package
% \cite{} output to follow that of the IEEE. Loading the cite package will
% result in citation numbers being automatically sorted and properly
% "compressed/ranged". e.g., [1], [9], [2], [7], [5], [6] without using
% cite.sty will become [1], [2], [5]--[7], [9] using cite.sty. cite.sty's
% \cite will automatically add leading space, if needed. Use cite.sty's
% noadjust option (cite.sty V3.8 and later) if you want to turn this off
% such as if a citation ever needs to be enclosed in parenthesis.
% cite.sty is already installed on most LaTeX systems. Be sure and use
% version 5.0 (2009-03-20) and later if using hyperref.sty.
% The latest version can be obtained at:
% http://www.ctan.org/pkg/cite
% The documentation is contained in the cite.sty file itself.






% *** GRAPHICS RELATED PACKAGES ***
%
\ifCLASSINFOpdf
  % \usepackage[pdftex]{graphicx}
  % declare the path(s) where your graphic files are
  % \graphicspath{{../pdf/}{../jpeg/}}
  % and their extensions so you won't have to specify these with
  % every instance of \includegraphics
  % \DeclareGraphicsExtensions{.pdf,.jpeg,.png}
\else
  % or other class option (dvipsone, dvipdf, if not using dvips). graphicx
  % will default to the driver specified in the system graphics.cfg if no
  % driver is specified.
  % \usepackage[dvips]{graphicx}
  % declare the path(s) where your graphic files are
  % \graphicspath{{../eps/}}
  % and their extensions so you won't have to specify these with
  % every instance of \includegraphics
  % \DeclareGraphicsExtensions{.eps}
\fi
% graphicx was written by David Carlisle and Sebastian Rahtz. It is
% required if you want graphics, photos, etc. graphicx.sty is already
% installed on most LaTeX systems. The latest version and documentation
% can be obtained at: 
% http://www.ctan.org/pkg/graphicx
% Another good source of documentation is "Using Imported Graphics in
% LaTeX2e" by Keith Reckdahl which can be found at:
% http://www.ctan.org/pkg/epslatex
%
% latex, and pdflatex in dvi mode, support graphics in encapsulated
% postscript (.eps) format. pdflatex in pdf mode supports graphics
% in .pdf, .jpeg, .png and .mps (metapost) formats. Users should ensure
% that all non-photo figures use a vector format (.eps, .pdf, .mps) and
% not a bitmapped formats (.jpeg, .png). The IEEE frowns on bitmapped formats
% which can result in "jaggedy"/blurry rendering of lines and letters as
% well as large increases in file sizes.
%
% You can find documentation about the pdfTeX application at:
% http://www.tug.org/applications/pdftex





% *** MATH PACKAGES ***
%
%\usepackage{amsmath}
% A popular package from the American Mathematical Society that provides
% many useful and powerful commands for dealing with mathematics.
%
% Note that the amsmath package sets \interdisplaylinepenalty to 10000
% thus preventing page breaks from occurring within multiline equations. Use:
%\interdisplaylinepenalty=2500
% after loading amsmath to restore such page breaks as IEEEtran.cls normally
% does. amsmath.sty is already installed on most LaTeX systems. The latest
% version and documentation can be obtained at:
% http://www.ctan.org/pkg/amsmath





% *** SPECIALIZED LIST PACKAGES ***
%
%\usepackage{algorithmic}
% algorithmic.sty was written by Peter Williams and Rogerio Brito.
% This package provides an algorithmic environment fo describing algorithms.
% You can use the algorithmic environment in-text or within a figure
% environment to provide for a floating algorithm. Do NOT use the algorithm
% floating environment provided by algorithm.sty (by the same authors) or
% algorithm2e.sty (by Christophe Fiorio) as the IEEE does not use dedicated
% algorithm float types and packages that provide these will not provide
% correct IEEE style captions. The latest version and documentation of
% algorithmic.sty can be obtained at:
% http://www.ctan.org/pkg/algorithms
% Also of interest may be the (relatively newer and more customizable)
% algorithmicx.sty package by Szasz Janos:
% http://www.ctan.org/pkg/algorithmicx




% *** ALIGNMENT PACKAGES ***
%
%\usepackage{array}
% Frank Mittelbach's and David Carlisle's array.sty patches and improves
% the standard LaTeX2e array and tabular environments to provide better
% appearance and additional user controls. As the default LaTeX2e table
% generation code is lacking to the point of almost being broken with
% respect to the quality of the end results, all users are strongly
% advised to use an enhanced (at the very least that provided by array.sty)
% set of table tools. array.sty is already installed on most systems. The
% latest version and documentation can be obtained at:
% http://www.ctan.org/pkg/array


% IEEEtran contains the IEEEeqnarray family of commands that can be used to
% generate multiline equations as well as matrices, tables, etc., of high
% quality.




% *** SUBFIGURE PACKAGES ***
%\ifCLASSOPTIONcompsoc
%  \usepackage[caption=false,font=normalsize,labelfont=sf,textfont=sf]{subfig}
%\else
%  \usepackage[caption=false,font=footnotesize]{subfig}
%\fi
% subfig.sty, written by Steven Douglas Cochran, is the modern replacement
% for subfigure.sty, the latter of which is no longer maintained and is
% incompatible with some LaTeX packages including fixltx2e. However,
% subfig.sty requires and automatically loads Axel Sommerfeldt's caption.sty
% which will override IEEEtran.cls' handling of captions and this will result
% in non-IEEE style figure/table captions. To prevent this problem, be sure
% and invoke subfig.sty's "caption=false" package option (available since
% subfig.sty version 1.3, 2005/06/28) as this is will preserve IEEEtran.cls
% handling of captions.
% Note that the Computer Society format requires a larger sans serif font
% than the serif footnote size font used in traditional IEEE formatting
% and thus the need to invoke different subfig.sty package options depending
% on whether compsoc mode has been enabled.
%
% The latest version and documentation of subfig.sty can be obtained at:
% http://www.ctan.org/pkg/subfig




% *** FLOAT PACKAGES ***
%
%\usepackage{fixltx2e}
% fixltx2e, the successor to the earlier fix2col.sty, was written by
% Frank Mittelbach and David Carlisle. This package corrects a few problems
% in the LaTeX2e kernel, the most notable of which is that in current
% LaTeX2e releases, the ordering of single and double column floats is not
% guaranteed to be preserved. Thus, an unpatched LaTeX2e can allow a
% single column figure to be placed prior to an earlier double column
% figure.
% Be aware that LaTeX2e kernels dated 2015 and later have fixltx2e.sty's
% corrections already built into the system in which case a warning will
% be issued if an attempt is made to load fixltx2e.sty as it is no longer
% needed.
% The latest version and documentation can be found at:
% http://www.ctan.org/pkg/fixltx2e


%\usepackage{stfloats}
% stfloats.sty was written by Sigitas Tolusis. This package gives LaTeX2e
% the ability to do double column floats at the bottom of the page as well
% as the top. (e.g., "% Figure environment removed


\subsection{Neural Networks}

ConvLSTM-based \cite{10.5555/2969239.2969329} network was implemented as it is known to learn spatio-temporal relationship better than the conventional LSTM and CNN-LSTM. As many of the 5G parameters are related to each other and also the parameter may change over time due to varying cell load and mobility of the UE, using ConvLSTM will effectively exploit these relationships. For comparison, the conventional LSTM, modified from \cite{10.1145/3386901.3388911}, and CNN-LSTM are also implemented. Aside from the LSTM-based network, one of the recent works \cite{10147378} has implemented a Transformer-based (Self-Attention) \cite{attention} neural network and shown good performance on South Korea's commercial 5G Non-Standalone (NSA) networks. Therefore, a similar Transformer-based network was also implemented for comparison. To allow for a fair comparison, the model and feed-forward dimension had been increased to 256 and 512, respectively. The structure of LSTM, CNN-LSTM, Transformer, as well as our proposed ConvLSTM network can be seen in Fig. \ref{fig:LSTM}, \ref{fig:CNNLSTM}, \ref{fig:Transformer}, and \ref{fig:ConvLSTM}, respectively.



% Figure environment removed

As for the prediction time window, a preliminary study was conducted. With the data sample interval of one point per second, it was found that a time window of five seconds yield the best result. This is likely due to shorter time window doesn't give the neural network enough context to accurately predicted the throughput, while the longer time window may cause the network to bias toward an incorrect trend caused by fluctuating input data. The optimal number of epochs was found for each type of model, so ConvLSTM, LSTM, CNN-LSTM, and transformer were trained for 10, 125, 100, and 150 epochs, respectively, and a batch size of 32 was used. All fully connected layers in every LSTM network utilized the ReLU activation function, whereas GELU was used in the Transformer network to match the implementation in the earlier work. Finally, the Adam optimizer was used for training of all networks.



\subsection{Input Parameters}

While data of all RF parameters with high sampling intervals from the modem chipset are available when using the professional network drive test tools on a modified smartphone, application developers rely on Android API to get information about the RF conditions, which only outputs limited data with very limited update intervals. Even though the correlation has been found between low-level RF parameters such as Tx Power, Resource Block (RB), and uplink throughput \cite{10118777}\cite{10.1145/3386901.3388911}\cite{10147378}\cite{9495144}, these information are not available to typical users. Since the real-world implementation is highly desirable, the data was collected with the sample interval of one second per data point to match the update interval of signal parameters via Android API, and only the parameters that are available via the API; CSI-RSRP, CSI-RSRQ, CSI-SINR, SSB ARFCN (Frequency), and past throughput were used. Despite lacking information about the cell load information, which can easily be derived from RB allocation, the study shows that RSRQ contains the information about the cell load and may be used to predict such information \cite{8506494}. To keep the data collection simple, the data was collected using NSG, and not directly from Android API, but since this set of parameters matches the output information of the API, it will now be referred to as \textit{Android API Data}.

To evaluate how much each model's performance degrades compared to the case when all of the parameters are available, each model will also be evaluated with full RF parameters including; CSI-RSRP, CSI-RSRQ, CSI-SINR, Resource Block Allocation (RB), Schedule Count, SSB-ARFCN (Frequency), PUCCH Tx Power, Channel Bandwidth, and past throughput. This set of parameters will now be referred to as \textit{Full Data}.

Lastly, to match the parameters used in \cite{10147378}, the model will also be tested using the same parameters in the study with two changes. First, there is no LTE information when UE is connected directly to the 5G SA network, so no LTE-related information will be provided to the network. Secondly, transport block size (TBS) was replaced with the throughput directly to match the other test cases. This resulted in four parameters being utilized; CSI-RSRP, Resource Block Allocation (RB), PUCCH Tx Power, and past throughput; and will now be referred to as \textit{SURE Data}.

\subsection{Evaluation Metrics}

Two metrics, RMSE and MAPE, will be used to evaluate the prediction accuracy of the models. While Root Mean Square Error (RMSE) can be used to paint the picture of instantaneous throughput prediction accuracy directly, applying mean absolute percentage error (MAPE) to the predicted throughput data directly will be problematic as 5G uplink throughput often hits zero during handover in the area with weak signal due to failure of RACH procedure. Therefore, MAPE is applied to the sum of the data being transferred instead. Finally, all of the prediction output data points with negative values will be normalized to zero as it's impossible to have a negative throughput.

\subsection{Evaluation Data}

As seen in Table \ref{tab:TestingData}, 11 traces will be used for model performance evaluation. Even though five of the traces are from commuter trains, each of them has a different purpose. The data from Keisei SkyAccess Line is from the same route as one of the training data, while another two traces on JR Chuo Line (Rapid) are from different sections of the line compared to the training data. By comparing these data, the performance of the model on seen and unseen data can be observed. Furthermore, two traces on JR Musashino Line are collected with the UE locked to a specific frequency band to simulate low-end UE or UE belonging to visiting tourists, which may not support all of the 5G frequency bands. Due to 5G SA being unavailable in the Japanese subway, the data from an elevated metro line was used for the evaluation. To keep the evaluation simple, the weighted average will be applied to some of the testing data with similar characteristics.

\section{Results and Analysis}

% Figure environment removed
% Figure environment removed
\subsection{Instantaneous Throughput Prediction Accuracy}
% Figure environment removed

When comparing our proposed model to the others, our proposed method delivers superior performance when performing throughput prediction using just only the Android API data. The predicted throughput trace in Fig. \ref{fig:GraphPred} shows that our model is resilient to extreme fluctuation of throughput and can predict the ground truth the most accurately. Even though the model is trained using the data commuter train, it performed well across the board, achieving the RMSE of 1.65 Mbps on an unseen train line and the RMSE of between 1.55 Mbps and 2.23 Mbps with an average of 1.80 Mbps when considering all of the unseen cases (see Fig. \ref{fig:CaseMSE}). Our model significantly outperformed the conventional LSTM, CNN-LSTM model, and Transformer model, which achieve the average RSME of 4.59 Mbps, 4.00 Mbps, and 3.94 Mbps, respectively. When simulating the band incompatibility by forcing the UE to only connect to a specific band, while other model yields some degradation in performance, especially when connecting to frequency band n28 (700 MHz), our model maintains nearly the same performance with RMSE of 1.23 Mbps and 1.73 Mbps when operating on frequency band n3 (1.8 GHz) and n28 (700 MHz), respectively (see Fig. \ref{fig:FreqMSE}). When considering different input parameter sets, it has been found that our model performs better when the parameter is limited to Android API data (see Fig. \ref{fig:DataMSE}), outperformed other models in all cases including when the other models are given SURE data or full data. Finally, as seen in Fig. \ref{fig:Unseen}, our proposed model performed well in both seen and unseen cases. However, the performance in the seen case is significantly better with the RMSE of 1.14 Mbps when predicting the uplink throughput on an Airport Express train on Keisei SkyAccess Line from Aoto Station to the Narita Airport, compared to RMSE of 1.66 Mbps in the unseen train route. Therefore, training this model for a specific use case can be considered for high-accuracy applications.

\subsection{Total Data Transfer Prediction Accuracy}
When looking at the prediction of the total data being transferred, our proposed model yields the maximum MAPE of 1.92\% in unseen cases (see Fig. \ref{fig:CaseMAPE}), delivering more than 98\% prediction accuracy. When considering UE incompatibility, our model shows a slight degradation with the maximum MAPE of 2.54\% on frequency band n3 (1.8 GHz), however, the performance is still better than other implementations (see Fig. \ref{fig:FreqMAPE}). Similar to the instantaneous throughput prediction accuracy, our proposed model works better when Android API data is being given compared to the full data, hitting the average MAPE of 1.10\% compared to 6.92\% (see Fig. \ref{fig:DataMAPE}). Lastly, the model performed significantly better on the seen data, yielding the MAPE of just 0.21\% and delivers more than 99.5\% accuracy, compared to MAPE of 1.07\% in the unseen case.
%% Figure environment removed


\section{Conclusions and Future Work}

In this paper, we propose a ConvLSTM-based neural network to predict the throughput using only the RF parameters accessible via Android API to ensure that the implementation is feasible for real-world applications. More than 12 hours of real-world RF parameters on a commercial 5G Standalone (SA) network were obtained using a network drive test tool on commuter trains, then used to train the model. The model was then evaluated using another set of real-world data on various kinds of transportation such as trains, trams, metros, cars, and walking against various different types of models used in earlier literature including conventional LSTM, CNN-LSTM, and Transformer (Self-Attention).

While transformer model outperformed our model when both models are given the full data or parameters similar to \cite{10147378}, the results show that our model can accurately predict the uplink throughput on commercial 5G SA network with extreme throughput fluctuation, dropout, and blind spots, when the input data is limited to what offers by Android API, achieving an average RMSE of 1.80 Mbps when considering the instantaneous throughput and 98.9\% accuracy when considering the total amount of data transferred, outperforming all other models. It can be said that with an adequate model, low-level parameters might not be necessary for accurate prediction as our model achieves higher accuracy when using Android API data as the input and even slightly outperformed the transformer model with full data. Since our model offers high accuracy for both instantaneous and total data prediction, while only using the accessible RF parameters via Android API, it's highly suitable for implementation in smartphone applications for a variety of use cases on the 5G mobile network including real-time video transmission, self-driving vehicle, and large file transfer.

As for future work, the model may be improved to incorporate supports for throughput prediction with newer 5G uplink throughput enhancement features, such as Uplink MIMO and Uplink Carrier-Aggregation, enabled and active on both the network and the UE side as well adding the support of the upcoming 5G Frequency Range 2 (mmWave) SA network, which has been actively tested by MNOs and RAN manufacturers around the world, recently.

\section*{Acknowledgement}
This work was supported by NICT (Grant No. 03801), Japan. Additionally, the authors would like to express their gratitude to \textbf{PEI Xiaohong} of Qtrun Technologies for providing Network Signal Guru (NSG) and AirScreen, the cellular network drive test software used for result collection and analysis in this research. The authors would like to thank \textbf{ElimZ} and \textbf{Lunaleaf} for inspiring this research. Additionally, we would like to thank \textbf{Shota Hirose} for providing useful suggestions on the neural network architecture, and \textbf{Fangzheng Lin} for reviewing this paper. Finally, the first author thanks \textbf{oElimZo Crew} and the members of \textbf{Saintsitive's Discord}, who provide emotional support and make this research much more bearable.



% use section* for acknowledgment





% trigger a \newpage just before the given reference
% number - used to balance the columns on the last page
% adjust value as needed - may need to be readjusted if
% the document is modified later
%\IEEEtriggeratref{8}
% The "triggered" command can be changed if desired:
%\IEEEtriggercmd{\enlargethispage{-5in}}

% references section

% can use a bibliography generated by BibTeX as a .bbl file
% BibTeX documentation can be easily obtained at:
% http://mirror.ctan.org/biblio/bibtex/contrib/doc/
% The IEEEtran BibTeX style support page is at:
% http://www.michaelshell.org/tex/ieeetran/bibtex/
%\bibliographystyle{IEEEtran}
% argument is your BibTeX string definitions and bibliography database(s)
%\bibliography{IEEEabrv,../bib/paper}
%
% <OR> manually copy in the resultant .bbl file
% set second argument of \begin to the number of references
% (used to reserve space for the reference number labels box)
\Urlmuskip=0mu plus 1mu\relax
\bibliographystyle{IEEEtran}
\bibliography{IEEEabrv,b_reference}





% that's all folks
\end{document}


