
\section{Conclusions and Future Work} \label{sec:conclusion}

 \begin{wrapfigure}{r}{0.5\textwidth}
\vspace{-15pt}
\begin{center}
% Figure removed
\caption{
Synthetic signal (top), its frequency modes (middle), and the error due to half precision, as a percentage of the amplitude (bottom).
The percentage error increases for higher frequencies.
}
\label{fig:synthetic}
\end{center}
\vspace{-20pt}
\end{wrapfigure}

In this work, we studied the numerical stability of half-precision training for FNO, and we devised a new training routine which results in a significant improvement in runtime and memory usage.
Specifically, we showed that using \texttt{tanh} pre-activation before the Fourier transform mitigates numerical instability.
We also showed that the range of half precision is too small to learn high frequency modes, and therefore, reducing the learnable frequency modes also helps performance.
We show that with these modifications, running FNO in half precision results in up to a 50\% reduction in memory, with little to no decrease in accuracy, on the Navier Stokes and Darcy flow equations.
%
Overall,  half-precision FNO makes it possible to train on significantly larger datapoints with the same batch size. 
Going forward, we plan to apply this on real-world applications that require super resolution to enable larger scale training.


\begin{table}[t]
\centering
\caption{\textbf{Zero-shot super resolution}. 
FNO is resolution invariant. We test zero-shot super-resolution by training each model on $128\times 128$ resolution for 19 hours.
We find that half-precision has a small decrease in accuracy compared to full precision, and using a precision schedule achieves significantly better accuracy with the same training time.
}
\label{tab:super-res}
\resizebox{\textwidth}{!}{
\begin{tabular}{lrr rr rr rr}
\toprule
{} & \multicolumn{2}{c}{128x128} & \multicolumn{2}{c}{256x256} & \multicolumn{2}{c}{512x512} & \multicolumn{2}{c}{1024x1024} \\
\cmidrule{2-9}
{} & $H^1$ & $L^2$ & $H^1$ & $L^2$ & $H^1$ & $L^2$ & $H^1$ & $L^2$ \\
\midrule
Full precision & 0.00557 & 0.00213 & 0.00597 & 0.00213 & 0.00610 & 0.00213 & 0.00616 & 0.00213 \\
Half precision & 0.00624 & 0.00236 & 0.00672 & 0.00228 & 0.00688 & 0.00226 & 0.00693 & 0.00226 \\
Precision schedule & \textbf{0.00503} & \textbf{0.00170} & \textbf{0.00542} & \textbf{0.00170} & \textbf{0.00555} & \textbf{0.00170} & \textbf{0.00558} & \textbf{0.00170} \\
\bottomrule
\end{tabular}
}
\end{table}
