
\section{Additional Details from \cref{sec:theory}} \label{app:theory}

In this section, we give the full proofs, details, and discussions from \cref{sec:theory}.

First, for convenience, we restate the definition of the discretization error.
Let $D$ denote the closed unit hypercube $[0,1]^d$ for dimension $d\in\N$.
Let $Q_1,\dots,Q_n$ denote the unique (up to $d$) partitioning of $D$, such that each $Q_j$ is a hypercube with sidelength $\nicefrac{1}{m}$.
For each $1\leq j\leq n$, let $\xi_j\in Q_j$ denote the vertex that is closest to the origin; formally, we have $Q_j=\prod_{k=1}^d [s_{j,k},t_{j,k}]$, and we define $\xi_j=(s_{j,1},\dots,s_{j,d})$.

Let $v:D\rightarrow \R$ denote an intermediate function within the FNO.
Recall from the previous section that the primary operation in FNO is the Fourier convolution operator, $(\K v_t) (x)$.
We say the \emph{discretization error} of $\F(v)$ is the absolute difference between the Fourier transform of $v$, and the discrete Fourier transform of $v$ via discretization of $v$ on $\Q_d=(\{Q_1,\dots,Q_n\},\{\xi_1,\dots,\xi_n\})$.
Formally, given Fourier basis function $\varphi_\omega(x)=e^{2\pi i  \langle \omega, x \rangle}$,
\begin{equation*}
\texttt{Disc}(v,\Q_d,\omega)=\Big|\int_D v(x)\varphi_\omega(x)dx - \sum_{j=1}^n v(\xi_j)\varphi_\omega(\xi_j)|Q_j|\Big|.
\end{equation*}




\subsection{Details for Resolution and Precision Bounds} \label{app:proofs}

In this section, we give the full details for \cref{thm:discretization_error} and \cref{thm:precision_error}.


\noindent\textbf{\cref{thm:discretization_error} (restated).}
\emph{
For any $D=[0,1]^d$, $M> 0$, and $L \geq 1$, let $\K \subset C(D)$ be the set of L-Lipschitz functions, bounded by $||v||_\infty\leq M$. Then there exists constants $c_1, c_2 > 0$ such that for all $n, d, \omega$, we have
\begin{equation*}
c_1\sqrt{d}\cdot Mn^{-\nicefrac{2}{d}}\leq 
\sup_{v\in\K}\left(\texttt{Disc}(v,\Q_d,1)\right) \text{ and }
\sup_{v\in\K}\left(\texttt{Disc}(v,\Q_d,\omega)\right)\leq c_2\sqrt{d}(|\omega|+L)M n^{-\nicefrac{1}{d}}.
\end{equation*}
}

\begin{proof}
Let $\varphi_\omega^r(x)=\sin(2\pi \langle \omega, x \rangle)$ denote the real part of the Fourier base of $v$ at frequency $\omega$.
Define
\begin{equation}
\texttt{Disc}^r(v,Q_d,\omega)=\left|\int_D v(x)\varphi_\omega^r(x)dx - \sum_{j=1}^n v(\xi_j)\varphi_\omega^r(\xi_j)|Q_j|\right|.
\end{equation}


Recall that as defined above, $\Q_d=(\{Q_1,\dots,Q_n\},\{\xi_1,\dots,\xi_n\})$, where $\{Q_1,\dots,Q_n\}$ is the unique (up to $d$) partitioning of $D$, and for each $1\leq j\leq n$, $\xi_j\in Q_j$ denotes the vertex that is closest to the origin; formally, we have $Q_j=\prod_{k=1}^d [s_{j,k},t_{j,k}]$, and we define $\xi_j=(s_{j,1},\dots,s_{j,d})$.
First we prove the upper bound. We have:

\begin{align}
\texttt{Disc}^r(v,Q_d,\omega)&=\left|\int_D v(x)\varphi_\omega^r(x)dx - \sum_{j=1}^n v(\xi_j)\varphi_\omega^r(\xi_j)|Q_j|\right|\\
&=\left|\int_D v(x)\varphi_\omega^r(x)dx - \sum_{j=1}^n\int_{Q_j} v(\xi_j)\varphi_\omega^r(\xi_j)dx\right|\\
&=\left|\sum_{j=1}^n \int_{Q_j}\left(v(x)\varphi_\omega^r(x)-v(\xi_j)\varphi_\omega^r(\xi_j)\right)dx\right|\\
&=\left|\sum_{j=1}^n\int_{Q_j}\left(v(x)(\varphi_\omega^r(x)-\varphi_\omega^r(\xi_j))
+(v(x)-v(\xi_j))\varphi_\omega^r(\xi_j)\right)dx\right|\\
&\leq\sum_{j=1}^n\int_{Q_j}\left(|v(x)|\cdot|\varphi_\omega^r(x)-\varphi_\omega^r(\xi_j)|
+|v(x)-v(\xi_j)|\cdot|\varphi_\omega^r(\xi_j)|\right)dx\\
&\leq\sum_{j=1}^n\int_{Q_j}\left(M\cdot\left(|\omega| \frac{\sqrt{d}}{m}\right)
+\left(L\cdot\frac{\sqrt{d}}{m}\right)\cdot 1\right)dx\\
&=\sum_{j=1}^n\int_{Q_j}\left(\frac{\sqrt{d}}{m}\left(M |\omega| +L\right)\right)dx\\
&=n\cdot \left(\frac{1}{m^d}\right)\sqrt{d} \left(M |\omega| +L\right)n^{-\nicefrac{1}{d}}\\
&=\sqrt{d}\left(M |\omega|+L\right)n^{-\nicefrac{1}{d}}.
\end{align}
Bounding $\texttt{Disc}^c(v,Q_d,\omega)$, the complex part of the Fourier base, follows an identical argument.
Setting $c_2=2$ concludes the proof of the upper bound, which is true for all $\omega$.
Also note that we only needed the fact that $\xi_j\in Q_j$ for all $1\leq j\leq n$.

% lower bound

Now we move to the lower bound.
We set $v(x)=x_1\cdots x_d$ and $\omega=1$, and we will show that
\begin{align*}
\left|\int_D v(x) \sin(2\pi x)dx - \sum_{j=1}^n v(\xi_j)\sin(2\pi \xi_j)\right|=\frac{d}{3\cdot 2^d \pi^{d-2}} \cdot n^{\nicefrac{1}{2d}}.
\end{align*}

First, in one dimension, $\int_0^1 x_1\sin(2\pi x_1)dx_1=\frac{1}{2\pi}.$
Therefore, for $d$ dimensions, we have

\begin{equation*}    
\int_D v(x)\sin(2\pi x)dx=\left(\int_0^1 x_1\sin(2\pi x_1)dx_1\right)^d=(2\pi)^{-d}.
\end{equation*}


%%%
Next, we will show a lower bound on $\sum_{j=1}^n v(\xi_j)\sin(2\pi \xi_j)$.
Here, we will need the explicit definition of the $\xi_j$'s defined at the start of this section. Specifically, for each $1\leq j\leq n$, we have $Q_j=\prod_{k=1}^d [s_{j,k},t_{j,k}]$, and we define $\xi_j=(s_{j,1},\dots,s_{j,d})$.
Now we note that by construction, since the unit hypercube $D$ is partitioned uniformly across each dimension in segments of $\nicefrac{1}{m}$, we can parameterize each $Q_j$ by a unique $d$-tuple $(i_1,\dots i_d)$, where each $i_k$ is an integer from $0$ to $m-1$, and the corresponding $\xi_j = (s_{j,1},\dots,s_{j,d}) = (\nicefrac{i_1}{m},\dots\nicefrac{i_d}{m})$.
Also, each $d$-tuple of integers from $0$ to $m-1$ defines a unique $Q_j$.
In other words, we are defining a different parameterization of the $Q_j$'s by using the fact that the $\xi_j$'s form a uniform lattice across the unit hypercube. This parameterization is convenient for the next part of the proof.

Recall that $v(x)=x_1\cdots x_d$.
We have

\begin{align}
\sum_{j=1}^n \left(v(\xi_j)\sin(2\pi \xi_j)|Q_j|\right)
&=\frac{1}{m^d}\left(\sum_{j=1}^n \left(
\prod_{k=1}^d s_{j,k}\sin(2\pi s_{j,k})\right)\right)\\
&=\frac{1}{m^d}\left(\sum_{i_1=1}^m \cdots \sum_{i_d=1}^m 
\left(\prod_{k=1}^d \left(\frac{i_k}{m}\cdot\sin\left(2\pi \frac{i_k}{m}\right)\right)\right)\right)\\
&=\frac{1}{m^d}\left(\sum_{i_1=1}^m \frac{i_1}{m}\sin\left(2\pi \frac{i_1}{m}\right) \right) \cdots 
\left(\sum_{i_d=1}^m \frac{i_d}{m} \sin\left(2\pi \frac{i_d}{m}\right) \right)\\
&=\frac{1}{m^d}\left(\sum_{j=1}^m \left(\frac{j}{m}\right)
\cdot\sin\left( 2\pi\frac{j}{m}\right)\right)^d\\
&=m^{-2d}\left(\sum_{j=1}^m \left(j\cdot\sin\left(2\pi \frac{j}{m}\right)\right)\right)^d\\
%here
&=m^{-2d}\left(-\frac{m}{2}\text{cot}\left(\frac{\pi}{m}\right)\right)^d\\
&\geq 2^{-d}\cdot m^{-d}\left(-\frac{m}{\pi}+\frac{1}{3}\cdot \frac{\pi}{m}\right)^d\\
&= (2\pi)^{-d}\cdot\frac{(m+\frac{\pi^2}{3m})^d}{m^{-d}}\\
&\geq (2\pi)^{-d}\left(1+d\cdot\frac{\pi^2}{3}\cdot m^{-2}\right).\\
\end{align}

Finally, since $n=m^d$, we have
\begin{equation*}
\left|(2\pi)^{-d} - (2\pi)^{-d}\left(1+d\cdot\frac{\pi^2}{3}\cdot m^{-2}\right)\right|
=\frac{d}{3\cdot 2^d \pi^{d-2}} \cdot n^{\nicefrac{1}{2d}},
\end{equation*}
which concludes the proof.
\end{proof}


%%%%%%%%%%%%%%%%%%%%%%%%%%%%%%%%%%%%%%%%%%%%%%%%%%%%%%%%%%%%

Note that the upper bound is true for all $\omega$, while the lower bound is shown for $\omega=1$. It is an interesting question for future work to show the lower bound for $\omega>1$.

Now we bound the precision error of $\F(v)$.
Recall from \cref{sec:theory} that the \emph{precision error} of $\F(v)$ is the absolute difference between $\F(v)$ and $\overline{\F(v)}$, computing the discrete Fourier transform in half precision. Specifically, we define an $(a_0, \epsilon,T)$-\emph{precision system} as a mapping $q:\R\rightarrow S$ for the set $\{0\}\cup\{a_0(1+\epsilon)^j\}_{j=0}^T\cup\{-a_j(1+\epsilon)^j\}_{j=0}^T$, such that for all $x\in\R$, $q(x)=\texttt{argmin}_{y\in S}|x-y|$.

This represents a simplified version of the true mapping used by Python from $\R$ to \texttt{float32} or \texttt{float16}.
Recall that these datatypes allocate some number of bits for the mantissa and for the exponent. Then, given a real number $x\cdot 2^y$, its value in \texttt{float32} or \texttt{float16} would be 
$(x+\epsilon_1)\cdot 2^{(y+\epsilon_2)}=x\cdot 2^y+\epsilon_2 x\cdot 2^y+\epsilon_1\cdot 2^y+\epsilon_1\epsilon_2 2^y=(1+\nicefrac{\epsilon_1}{x}+\epsilon_2)x\cdot 2^y$. 
%Furthermore, the mantissa typically ranges between 0 and 2.
Therefore, our definition above is a simplified but reasonable approximation of floating-point arithmetic.

Now, we define 
\begin{equation*}
\texttt{Prec}(v,\Q_d,q,\omega)=\Big|\sum_{j=1}^n v(\xi_j)\varphi_\omega(\xi_j)|Q_j|
-\sum_j q(v(\xi_j))q(\varphi_\omega(\xi_j))|Q_j|\Big|.
\end{equation*}

Now we bound the precision error of $\F(v)$.

\noindent\textbf{\cref{thm:precision_error} (restated).}
\emph{
For any $D=[0,1]^d$, $M> 0$, and $L \geq 1$, let $\K \subset C(D)$ be the set of L-Lipschitz functions, bounded by $||v||_\infty\leq M$. Furthermore let $q$ be an $(a_0, \epsilon,T)$-precision system.
There exists $c>0$ such that for all $n,d,\omega$,
\begin{equation*}
\sup_{v\in\K}\left(\texttt{Prec}(v,\Q_d,q,\omega)\right)\leq c\cdot \epsilon M.
\end{equation*}
}


\begin{proof}
Let $\varphi_\omega^r(x)= \sin(2\pi \langle \omega, x \rangle)$ denote the real part of the Fourier base of $v$ at frequency $\omega$.
Define
\begin{equation}
\texttt{Prec}^r(v,\Q_d,q,\omega)=
\left|\sum_{j=1}^n v(\xi_j)\varphi_\omega^r(\xi_j)|Q_j| -
\sum_{j=1}^n q(v(\xi_j))q(\varphi_\omega^r(\xi_j))|Q_j|\right|
\end{equation}

We prove the upper bound as follows. We have
\begin{align*}
\texttt{Prec}^r(v,\Q_d,q,\omega)&=\left|\sum_{j=1}^n v(\xi_j)\varphi_\omega^r(\xi_j)|Q_j| -
\sum_{j=1}^n q(v(\xi_j))q(\varphi_\omega^r(\xi_j))|Q_j|\right|\\
&=\left| \frac{1}{n}\sum_{j=1}^n \left(v(\xi_j)\varphi_\omega^r(\xi_j)
- q(v(\xi_j))q(\varphi_\omega^r(\xi_j))\right)\right|\\
&\leq\left| \frac{1}{n}\sum_{j=1}^n\left(v(\xi_j)(\varphi_\omega^r(\xi_j)-q(\varphi_\omega^r(\xi_j))) + (v(\xi_j)-q(v(\xi_j)))q(\varphi_\omega^r(\xi_j))\right)\right|\\
&\leq\frac{1}{n}\sum_{j=1}^n \left(|v(\xi_j)|\cdot |\varphi_\omega^r(\xi_j)-q(\varphi_\omega^r(\xi_j))| + |v(\xi_j)-q(v(\xi_j))|\cdot |q(\varphi_\omega^r(\xi_j))|\right)\\
&\leq\frac{1}{n}\sum_{j=1}^n \left(M\cdot (\epsilon\cdot 1) + (\epsilon\cdot M) \cdot 1\right)\\
&=\frac{1}{n}\cdot n \left( 2\epsilon M\right)\\
&= 2\epsilon M.
\end{align*}
Bounding $\texttt{Prec}^c(v,\Q_d,q,\omega)$, the complex part of the Fourier base, follows an identical argument.
Setting $c=4$ concludes the proof.
\end{proof}



%%%%%%%%%%%%%%%%%%%%%%%%%%%%%%%%%%%%%%%%%%%%%%%%%%%%%%%%%%
%%%%%%%%%%%%%%%%%%%%%%%%%%%%%%%%%%%%%%%%%%%%%%%%%%%%%%%%%%


\subsection{General Bounds} \label{app:general_bounds}

Now, we give results similar to \cref{thm:discretization_error} and \cref{thm:precision_error}, but with a general function $f$, rather than for $\F(v)$, a function $v$ times the Fourier basis function. 
%$\varphi_\omega(x)=e^{2\pi j\langle \omega, x \rangle}$ for a given $j$.


\begin{theorem} \label{thm:disc_general}
For any $D=[0,1]^d$, $M> 0$, and $L \geq 1$, let $\K \subset C(D)$ be the set of L-Lipschitz functions, bounded by $||f||_\infty\leq M$. Then for all $n,d,\omega$, we have
\begin{equation*}
2^{-d+1}\cdot d\cdot n^{-\nicefrac{1}{d}}\leq \sup_{f\in\K}\left(\texttt{Disc}(f,\Q,\omega)\right)\leq L\sqrt{d} \cdot n^{-\nicefrac{1}{d}}.
\end{equation*}
\end{theorem}

% todo: make Disc_g, Prec_g to denote a general function 
\begin{proof}

First, we define
\begin{equation}
\texttt{Disc}^r(f,\Q_d,\omega)=\left|\int_D f(x)dx - \sum_{j=1}^n f(\xi_j)|Q_j|\right|.
\end{equation}

Recall that $\Q_d=(\{Q_1,\dots,Q_n\},\{\xi_1,\dots,\xi_n\})$, where the $n=m^d$ hypercubes of side length $\nicefrac{1}{m}$ subdivide $D$.
Now we prove the upper bound. We have:

\begin{align}
\texttt{Disc}^r(f,\Q_d,\omega)&=\left|\int_D f(x)dx - \sum_{j=1}^n f(\xi_j)|Q_j|\right|\\
&=\left|\int_D f(x)dx - \sum_{j=1}^n\int_{Q_j} f(\xi_j)dx\right|\\
&=\left|\sum_{j=1}^n \int_{Q_j}\left(f(x)-f(\xi_j)\right)dx\right|\\
&\leq\sum_{j=1}^n\int_{Q_j}\left|f(x)-f(\xi_j\right|dx\\
&\leq\sum_{j=1}^n\int_{Q_j}\left(L\cdot\frac{\sqrt{d}}{m}\right)dx\\
&=n\cdot \frac{1}{m^d}\left(L\cdot\frac{\sqrt{d}}{m}\right)\\
&=L\sqrt{d}\cdot n^{-\nicefrac{1}{d}}
\end{align}
This concludes the proof of the upper bound.

For the lower bound, first, in one dimension, $\int_0^1 x_1 dx_1=\frac{1}{2}.$
Therefore, for $d$ dimensions, we have

\begin{equation*}    
\int_D f(x)dx=\left(\int_0^1 x_1 dx_1\right)^d=(2)^{-d}.
\end{equation*}

Next, we will use the same explicit definition and reparameterization of the $\xi_j$ as we used in the lower bound of \cref{thm:discretization_error}:
for each $1\leq j\leq n$, we have $Q_j=\prod_{k=1}^d [s_{j,k},t_{j,k}]$, and we define $\xi_j=(s_{j,1},\dots,s_{j,d})$.
Now we note that by construction, since the unit hypercube $D$ is partitioned uniformly across each dimension in segments of $\nicefrac{1}{m}$, we can parameterize each $Q_j$ by a unique $d$-tuple $(i_1,\dots i_d)$, where each $i_k$ is an integer from $0$ to $m-1$, and the corresponding $\xi_j = (s_{j,1},\dots,s_{j,d}) = (\nicefrac{i_1}{m},\dots\nicefrac{i_d}{m})$.
Also, each $d$-tuple of integers from $0$ to $m-1$ defines a unique $Q_j$.
In other words, we are defining a different parameterization of the $Q_j$'s by using the fact that the $\xi_j$'s form a uniform lattice across the unit hypercube. This parameterization is convenient for the next part of the proof.
Recall that $f(x)=x_1\cdots x_d$.
We have
\begin{align}
\sum_{j=1}^n \left(f(\xi_j)|Q_j|\right)
&=\frac{1}{m^d}\left(\sum_{j=1}^n \left(
\prod_{k=1}^d f(s_{j,k})\right)\right)\\
&=\frac{1}{m^d}\left( \sum_{i_1=1}^m\cdots \sum_{i_d=1}^m \left(
\prod_{k=1}^d f\left(\frac{i_k}{m}\right)\right)\right)\\
&=\frac{1}{m^d}\left( \sum_{i_1=1}^m \frac{i_1}{m}\right) \cdots
\left( \sum_{i_d=1}^m \frac{i_d}{m}\right)\\
&=\frac{1}{m^d}\left( \sum_{j=1}^m \frac{j}{m}\right)^d\\
% here
&=m^{-2d}\left(\frac{m(m+1)}{2}\right)^d\\
&=2^{-d}\left(1+\frac{1}{m}\right)^d\\
&\geq 2^{-d}\left(1+2dm^{-1}\right)^d\\
\end{align}

Finally, since $n=m^d$, we have
\begin{equation*}
\left|2^{-d} - 2^{-d}\left(1+2dm^{-1}\right)\right|
=2^{-d+1}\cdot d\cdot n^{\nicefrac{1}{d}},
\end{equation*}
which concludes the proof.
\end{proof}


Now we similarly bound the precision error of $f$.

\begin{theorem} \label{thm:prec_general}
For any $D=[0,1]^d$, $M> 0$, and $L \geq 1$, let $\K \subset C(D)$ be the set of L-Lipschitz functions, bounded by $||f||_\infty\leq M$. Furthermore let $q$ be an $(a_0, \epsilon,T)$-precision system.
Then for all $n,d,\omega$, there exists $c>0$ such that
\begin{equation*}
\nicefrac{1}{4}\cdot\epsilon M \leq \sup_{f\in\K}\left(\texttt{Prec}(f,\Q_d,q,\omega)\right)\leq \epsilon M.
\end{equation*}
\end{theorem}
% todo: still need to change disc, prec notation


\begin{proof}
Define
\begin{equation}
\texttt{Prec}^r(f,\Q_d,q,\omega)=
\left|\sum_{j=1}^n f(\xi_j) |Q_j| -
\sum_{j=1}^n q(f(\xi_j))|Q_j|\right|
\end{equation}

We prove the upper bound as follows. We have
\begin{align*}
\texttt{Prec}^r(f,\Q_d,q,\omega)&=\left|\sum_{j=1}^n f(\xi_j)|Q_j| -
\sum_{j=1}^n q(f(\xi_j))|Q_j|\right|\\
&=\left| \frac{1}{n}\sum_{j=1}^n \left(f(\xi_j)
- q(f(\xi_j))\right)\right|\\
&\leq \frac{1}{n}\sum_{j=1}^n \left|f(\xi_j)
- q(f(\xi_j))\right|\\
&\leq \frac{1}{n}\sum_{j=1}^n \left(\epsilon M\right)\\
&=\epsilon M
\end{align*}
This concludes the proof of the upper bound.


For the lower bound, given the definition of an $(a_0, \epsilon,T)$-precision system, it follows that there exists $y$ such that $\nicefrac{M}{2}<y<M$ and $\nicefrac{1}{2}\cdot\epsilon y<|y-q(y)|.$
Then, for the lower bound we set $f(x)=y$, and we have
\begin{align*}
\texttt{Prec}^r(f,\Q_d,q,\omega)&=\left|\sum_{j=1}^n f(\xi_j)|Q_j| -
\sum_{j=1}^n q(f(\xi_j))|Q_j|\right|\\
&=\left| \frac{1}{n}\sum_{j=1}^n \left(f(\xi_j)
- q(f(\xi_j))\right)\right|\\
&\geq\left| \frac{1}{n}\sum_{j=1}^n \left(\frac{1}{2}\cdot\epsilon y\right)\right|\\
&=\frac{1}{2}\cdot\epsilon y\\
&\geq\frac{1}{4}\cdot\epsilon M\\
\end{align*}
This concludes the proof.
\end{proof}

\subsection{Plotting Theoretical Bounds} \label{app:simulation}

In this section, we run additional experiments by plotting our theoretical bounds alongside the true empirical discretization and resolution errors of the Darcy flow dataset.
Specifically, we plot both our bounds assuming a Fourier basis (\cref{thm:discretization_error}, \cref{thm:precision_error}) and our bounds for general functions (\cref{thm:disc_general}, \cref{thm:prec_general}) compared to the true Darcy flow dataset, measured after 10 epochs, at the start of the FNO block (just before the forward FFT).
The true errors are calculated using the definitions of precision and discretization error in the previous section, and we use the true difference in precision between \texttt{float32} and \texttt{float16} when computing the precision error.
See \cref{fig:theory_bounds}.
We find that the discretization error is higher than the precision error, as expected by our theory. 
Furthermore, the Darcy flow discretization and precision errors are always lower than their respective upper and lower bounds.
Note that the lower bounds in our theorems are \emph{worst-case} lower bounds (we showed there exists a function from the class of bounded $L$-Lipschitz functions which achieves at least the desired error), which means that the true errors can be lower than the bounds, which is the case in \cref{fig:theory_bounds}.


% Figure environment removed