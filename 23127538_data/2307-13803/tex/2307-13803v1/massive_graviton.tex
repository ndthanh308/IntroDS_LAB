\documentclass[11pt,a4paper,twoside,groupcitations]{article}
\usepackage[]{latexsym,amsmath,amssymb}
\usepackage[]{epsfig}
\usepackage{color}
\usepackage{braket}
%\usepackage{multicol}
\usepackage{graphicx}
\pagestyle{plain}
\usepackage{cite}
\usepackage{stackrel}
\usepackage{amsfonts, bbold, mathtools, physics, cancel, scalerel, stackengine, wasysym, subcaption}
%
\flushbottom
\setlength{\oddsidemargin}    {0.0 cm}
\setlength{\evensidemargin}   {0.0 cm}
\setlength{\topmargin}        {0.0 cm}
\setlength{\textwidth}        {16.5cm}
\setlength{\textheight}       {22.0cm}
%
%
\newcommand{\be}{\begin{eqnarray}}
\newcommand{\ee}{\end{eqnarray}}
\def\beq{\begin{equation}}
\def\eeq{\end{equation}}
%\newcommand{\bra}[1]{\mbox{$\langle\, #1 \mid$}}
\newcommand{\bbra}[1]{\mbox{$\left\langle\, #1 \right\mid$}}
%\newcommand{\ket}[1]{\mbox{$\mid #1\,\rangle$}}
\newcommand{\bket}[1]{\mbox{$\left\mid #1\,\right\rangle$}}
\newcommand{\pro}[2]{\mbox{$\langle\, #1 \mid #2\,\rangle$}}
\newcommand{\expec}[1]{\mbox{$\langle\, #1\,\rangle$}}
\newcommand{\expecl}[1]{\mbox{$\left\langle\,\strut\displaystyle{#1}\,\right\rangle$}}
%\newcommand{\real}{\mbox{{\rm I\hspace{-2truemm} R}}}
\renewcommand{\natural}{\mbox{{\rm I\hspace{-2truemm} N}}}
\newcommand{\re}{\real{\rm e}}
\renewcommand{\a}{\hat a}
\newcommand{\ac}{\hat a^{\dagger}}
\renewcommand{\b}{\hat b}
\newcommand{\bc}{\hat b^\dagger}
\newcommand{\ve}{\varepsilon}
\newcommand{\lp}{\ell_{\rm p}}
\newcommand{\mpl}{m_{\rm p}}
\newcommand{\gn}{G_{\rm N}}
\newcommand{\rh}{r_{\rm H}}
\newcommand{\Rh}{R_{\rm H}}
\renewcommand{\d}{\mbox{${\rm d}$}}
\newcommand{\ep}{\mathcal{E}_{\rm p}}
\newcommand{\Det}{\text{Det}\,}
%\DeclareMathOperator{\Tr}{Tr}
%\newcommand{\tr}{{\rm tr}}
%
%
\setcounter{equation}{0}
\def\theequation{\thesection.\arabic{equation}}
%
%
%
\title{\bf Massive graviton from diffeomorphism invariance}
%\title{\bf Diffeomorphism invariance in massive quantum gravity}
%
%
\author{
João~M.~L.~de~Freitas$^{a}$\thanks{E-mail: matheus.leal@ufpr.br},
$\ $
Iber\^e Kuntz$^{a}$\thanks{E-mail: kuntz@fisica.ufpr.br}
\\
\\
$^a${\em Departamento de F\'isica, Universidade Federal do Paran\'a}
\\
{\em PO Box 19044, Curitiba -- PR, 81531-980, Brazil}
\\
\\
}
%
%
\date{}
%
\begin{document}
%
\maketitle
%
%
\begin{abstract}
We describe a mechanism in which the graviton acquires a mass from the functional measure without violating the diffeomorphism symmetry nor including St\"uckelberg fields. Since gauge invariance is not violated, the number of degrees of freedom goes as in general relativity. For the same reason, Boulware-Deser ghosts and the vDVZ disconinuity do not show up. The graviton thus becomes massive at the quantum level while avoiding the usual issues of massive gravity. 
\end{abstract}
%
%\textit{PACS 04.60 - Quantum theory of gravitation.}
%\end{abstract}
%
%
%\pacs{}
%
%
%%%%%%%%%%%%%%%%%%%%%%%%%%%%%%%%%%%%%%%%%%%%%%%%%%%%%%%%%%%%%%%%
%%%
%%%                     INTRODUCTION
%%%
%%%%%%%%%%%%%%%%%%%%%%%%%%%%%%%%%%%%%%%%%%%%%%%%%%%%%%%%%%%%%%%%
%
\newpage
%
%
\section{Introduction}
%
\label{S:intro}
%
\setcounter{equation}{0}

General relativity and the standard model have both been extremely successful in describing fundamental phenomena. Nonetheless, many problems in the interface of gravity and high-energy physics, such as the dark sector and singularities, cannot be explained by either of these theories. This has led to a plethora of attempts to modify these models.

One interesting modification of general relativity regards massive gravity. Attempts of giving the graviton a mass dates back to the 30s \cite{Fierz:1939ix}, but it was only recently that consistent theories of massive gravity have been found \cite{Dvali:2000xg,Dvali:2000hr,Dvali:2000rv,Bergshoeff:2009hq,deRham:2010kj} (see also Ref.~\cite{deRham:2014zqa} for an in-depth review). A massive graviton indeed introduces many issues, such as the violation of diffeomorphism (gauge) invariance, the van Dam-Veltman-Zakharov (vDVZ) discontinuity and the presence of Boulware-Deser ghosts \cite{Boulware:1974sr}. Gauge invariance can be reinstated via St\"uckelberg fields, which is legitimate but does require new degrees of freedom. The vDVZ discontinuity \cite{vanDam:1970vg,Zakharov:1970cc}, namely the disagreement with general relativity in the massless limit, is usually conjectured to be solved at the non-linear regime by the Vainshtein screening mechanism \cite{Vainshtein:1972sx,Babichev:2013usa}. Only a few models have successfully implemented the Vainshtein mechanism \cite{Babichev:2013usa,Babichev:2009jt,Deffayet:2001uk,Nicolis:2008in,deRham:2016plk}, which can also bring along the possibility of superluminal velocities \cite{Dvali:2000hr,Nicolis:2008in,Luty:2003vm,Nicolis:2004qq,Adams:2006sv,deFromont:2013iwa}.

In this paper, we propose a novel procedure to give the graviton a mass without introducing any of the aforementioned issues and without modifying classical general relativity. Our finding is based on a non-trivial path-integral measure, which is required for obtaining gauge invariant correlation functions. The functional measure introduces non-linear loop corrections, which act as a gravitational potential and results in a (quantum) mass for the graviton in the linear regime. The most important point is the preservation of the diffeomorphism invariance, which is responsible for keeping the theory free of ghosts and of the zDVZ discontinuity. The so obtained mass is, however, pure imaginary, thus precluding the existence of gravitons as asymptotic states.

This paper is organized as follows. In Sec.~\ref{measure1}, we review some aspects of the functional measure in quantum field theory. The one-loop correction induced by the functional measure is then studied in Sec.~\ref{massg}, where the graviton mass is calculated. Comparison with experimental data yields stringent bounds on the model. In Sec.~\ref{newton}, we obtain Newton's potential and discuss its consequences. Finally, we draw our conclusions in Sec.~\ref{conc}.

\section{Functional measure in quantum field theory}
\label{measure1}

A central object in quantum field theory is the generating functional:
\begin{equation}
	Z[J]
	=
	\int\mathrm{d}\mu[\varphi] e^{i \left( S[\varphi^i] + J_i \varphi^i \right)},
	\label{Z}
\end{equation}
where $S[\varphi]$ is the classical action for some generic set of fields $\varphi^i = (\phi(x), A_\mu(x), g_{\mu\nu}(x), \ldots)$. All information regarding any physical system is contained in its corresponding generating functional $Z[J]$. In particular, correlation functions are obtained by simple functional differentiations, from which scattering amplitudes can be obtained using the LSZ (Lehmann--Symanzik--Zimmermann) formula. Despite its importance in field theory, a rigorous mathematical foundation remains unknown, particularly with respect to the functional measure $\mathrm{d}\mu[\varphi]$. Operationally, one can however define the aforementioned object as \cite{Mottola:1995sj,DeWitt:2003pm,Toms:1986sh,Casadio:2022ozp}
\begin{equation}
    \mathrm{d}\mu[\varphi] = \mathcal{D}\varphi^i \sqrt{\Det G_{ij}}
    \ ,
    \label{measure}
\end{equation}
where $\mathcal{D}\varphi^i = \prod_i \mathrm{d}\varphi^i$ and $\Det G_{ij}$ denotes the functional determinant of the configuration-space metric $G_{ij}$. The factor $\sqrt{\Det G_{ij}}$ is required to account for a non-trivial configuration space, whose typical example is that of a non-linear sigma model.

The presence of a non-trivial functional measure is usually sidestepped by writing $\delta^{(4)}(0) = 0$ in dimensional regularization, in which case one finds
\begin{align}
    \Det G_{ij}
    &=
    e^{\delta^{(4)}(0) \int\mathrm{d}^4x \sqrt{-g} \, \tr\log G_{ij}}
    \nonumber
    \\
    &\stackrel{dim.\, reg.}{=} \ 1
    \ .\label{dimr}
\end{align}
One should, however, be extremelly careful when setting infinities to zero. In fact, a Gaussian regularization
\begin{equation}
    \delta^{(4)}(x)
    =
    \frac{\Lambda^4}{(2\pi)^{2}} e^{\frac{-x^2 \Lambda^2}{2}}
    \ ,
\end{equation}
for some (soft) cutoff $\Lambda$, tells a completely different story. For $\delta^{(4)}(0) = \tfrac{\Lambda^4}{(2\pi)^{2}}$, we find \cite{Kuntz:2022kcw}:
\begin{equation}
	Z[J]
	=
	\int \mathcal{D}\varphi^i e^{i \left( S_\text{eff}[\varphi^i] + J_i \varphi^i \right)},
\end{equation}
with the Wilsonian effective action
\begin{equation}
    S_\text{eff} = \int\mathrm{d}^4x \sqrt{-g}
    \left(
    \mathcal L
	-
    i \zeta \, \tr\log G_{IJ}
	\right)
	\ ,
	\label{genact}
\end{equation}
for some bare Lagrangian $\mathcal L$. Here $\zeta=\zeta(\Lambda)$ is a Wilsonian coefficient whose running is such that
\begin{equation}
    \Lambda \frac{dZ[J]}{d\Lambda} = 0
    \label{RGE}
    \ .
\end{equation}
Note that the Dirac delta divergence is polynomial. The coefficient $\zeta$ is thus expected to be UV sensitive.

One should note that the configuration-space metric $G_{ij}$ must be seen as part of the definition of the theory. Although it is typically identified with the bilinear form in the kinetic term \cite{Meetz:1969as,Slavnov:1971mz,Vilkovisky:1984st,Fradkin:1973wke,Fradkin:1976xa}, this identification is not based on physical reasonings. Nonetheless, the functional measure must be invariant under the underlying symmetries, which allows one to determine $G_{ij}$ in the same spirit as effective field theories. In particular, for pure gravity this procedure results in the well-known DeWitt metric \cite{DeWitt:1967yk}. To lowest order, the most general configuration-space metric for arbitrary fields yields \cite{Casadio:2021rwj,Kuntz:2022tat,Casadio:2020zmn}

\begin{equation}
    S_\text{eff} = \int\mathrm{d}^4x \sqrt{-g}
    \left(
    \mathcal L
	-
    i \gamma \, \tr\log |g_{\mu\nu}|
	\right)
	\ ,
	\label{eq:newac}
\end{equation}
where $\gamma$ is obtained from $\zeta$ via a finite renormalization.

In spite of the form of the correction, we stress that Eq.~\eqref{eq:newac} does not violate diffeomorphism invariance. The apparent violation results from the fact that $\sqrt{\Det G_{ij}}$ transforms as a (functional) scalar density, thus so does the last term in Eq.~\eqref{eq:newac}. However, the $\mathcal{D}\varphi^i$ also transforms as a scalar density in such a way that the full measure $\mathcal{D}\varphi^i \sqrt{\Det G_{ij}}$ is invariant~\footnote{Note that $S_\text{eff}$ is the Wilsonian action before performing the path integral. Because the functional measure and the classical action are invariant under diffeomorphisms, the 1PI effective action $\Gamma[g]$ naturally reflects such symmetry. At the one-loop level of $\Gamma[g]$, the configuration-space metric enters the usual correction $\log\Det\mathcal{H}_{ij}$, transforming the Hessian $\mathcal{H}_{ij}$, whose determinant is basis-dependent, into $\mathcal{H}^i_{\ j}$, whose determinant is invariant \cite{Toms:1986sh,Ellicott:1987ir}.}. Therefore, variations of the apparent symmetry-breaking term under spacetime diffeomorphisms are canceled by the functional Jacobian that shows up from $\mathcal{D}\varphi^i$, keeping the quantum theory and all observables invariant~\footnote{Strictly speaking, the invariance of off-shell quantities, such as the effective action, requires a connection in configuration space \cite{Vilkovisky:1984st,DeWitt:1988dq}. Because we are mainly interested in the phenomenology of the functional measure, we shall not dwell on this topic.}. This is, in fact, the reason one can generate a mass for the graviton without violating the gauge symmetry.

Finally, the above findings were obtained in the Lorentzian path integral. The imaginary factor in Eq.~\eqref{genact} (hence in Eq.~\eqref{eq:newac}) shows up when the $i=\sqrt{-1}$ in the argument of the exponential in the Lorentzian path integral is pulled out to write the measure as a correction to the classical action. Defining the functional measure in the Euclidean formalism
\begin{equation}
	Z_E[J]
	=
	\int\mathrm{d}\mu[\varphi] e^{- \left( S_\text{eff}^E[\varphi^i] + J_i \varphi^i \right)}
\end{equation}
yields a real one-loop correction:
\begin{equation}
	S_\text{eff}^E
	=
	\int\mathrm{d}^4x \sqrt{-g}
    \left(
    \mathcal L
	-
    \gamma \, \tr\log |g_{\mu\nu}|
	\right)
	\ .
	\label{acE}
\end{equation}
One thus faces the problem of whether the path-integral measure should be defined in the Euclidean space (and rotated back to real time) or straight in the Lorentzian space. The former is required for a better mathematical construction of the path integral, albeit still largely formal. On the other hand, as far as our current experiments are concerned, Nature is fundamentally Lorentzian. For this reason, we shall perform our calculations by defining the measure on the latter. Naturally, predictions shall be different in different schemes. At this level of formality, only time will tell which one, if any, is correct.

%Although this might seem an ambiguity, this difference makes perfect sense. Indeed, the effective action, hence all $n$-point functions, depends on the boundary conditions adopted. The Lorentzian formalism automatically reproduces Feynman's boundary conditions, which is useful for scattering problems. In this case, $n$-point functions are complex and do not evolve causally, but they serve as a stepping stone to obtaining scattering amplitudes. On the other hand, time evolutions require retarded boundary conditions to obtain real and causal correlation functions. The latter can be obtained by adopting the in-in (or Schwinger-Keldysh) formalism of path integrals \cite{Keldysh:1964ud,Jordan:1986ug}. In Ref.~\cite{Barvinsky:1987uw}, it was proven that both Feynman's and retarded boundary conditions can be obtained by analytically continuing the result from the Euclidean time, which amounts in replacing Green's functions by their corresponding retarded or Feynman's expressions.

%Different applications thus require different boundary conditions, where either Eq.~\eqref{eq:newac} or Eq.~\eqref{acE} is used. In particular, what we perceive as a (quantum) mass is fundamentally the result of a tower of second-order self-interactions. Indeed, the mass of a particle is measured via decay rates, which are inferred from scattering amplitudes using the corresponding Feynman's dressed propagator from Eq.~\eqref{eq:newac} (see Sec.~\ref{massg}). On the other hand, the retarded boundary condition is the appropriate condition for calculating the quantum generalization of Newton's potential (see Sec.~\ref{newton}), which follows from solving the effective equations of motion for the gravitational perturbation. In this case, we shall thus employ Eq.~\eqref{acE} and analitically continue the result back to real time by imposing the retarted Green's function.

\section{Massive graviton}
\label{massg}
When the bare Lagrangian $\mathcal L$ is the Einstein-Hilbert term, the functional measure changes the dynamics of space-time as follows~\footnote{We adopt the metric signature $(-+++)$.}:
\begin{equation}
    S_\text{eff} = \int\mathrm{d}^4x \sqrt{-g}
    \left(
    \frac{M_p^2}{2} R
	-
    i \gamma \, \tr\log |g_{\mu\nu}|
    + \mathcal{L}_m
	\right)
	\ ,
	\label{effac}
\end{equation}
where $M_p$ is the reduced Planck mass, $R$ is the Ricci scalar and $\mathcal{L}_m$ is the Lagrangian for matter fields.
The corresponding equations of motion read~\footnote{One should note that $T_{\mu\nu}$ is covariantly conserved in the sense of the Slavnov-Taylor identities, i.e. $\langle \nabla^\mu T_{\mu\nu} \rangle = 0$. In particular, such conservation holds at every loop order.}:
\begin{equation}
	G_{\mu\nu} + i \frac{2\gamma}{M_p^2} \left[1 + \frac12 \log(-g)\right] = \frac{1}{M_p^2} T_{\mu\nu}
	\ ,
	\label{eom}
\end{equation}
where $T_{\mu\nu}$ is the energy-momentum tensor for $\mathcal{L}_m$. One should note that general relativity is smoothly recovered in the limit $\gamma\to 0$. The parameter $\gamma$ is proportional to the graviton mass (see Eq.~\eqref{mass}), thus there is no tension with the experimental tests of general relativity. As we shall see in the next section, gauge invariance prevents the vDVZ discontinuity from appearing.
  
Perturbing the metric around Minkowski
\begin{equation}
	g_{\mu\nu}
	=
	\eta_{\mu\nu}
	+ \frac{2}{M_p} 
	h_{\mu\nu}
	\label{pert}
\end{equation}
leads to the Fierz-Pauli action
\begin{align}
	S_\text{eff} =
	\int \mathrm{d}^4x
	\bigg[
		&
		- \frac{1}{2} \partial_{\lambda} h_{\mu \nu} \partial^{\lambda} h^{\mu \nu}
		+ \frac{1}{2} \partial_{\lambda} h \partial^{\lambda} h
		- \partial_{\mu} h^{\mu \nu} \partial_{\nu} h
		+ \partial_{\mu} h_{\nu \lambda} \partial^{\nu} h^{\mu \lambda}
		\nonumber
		\\
		&
		- \frac{i}{2} m^2_g (h^2 - h_{\mu \nu} h^{\mu \nu})
		+ M_p^{-1} h_{\mu\nu} T^{\mu\nu}
	\bigg]
	\ .
	\label{eq:FP}
\end{align}
The graviton mass is thus given by
\begin{equation}
	m^2 \equiv i m^2_g = \frac{4 i \gamma}{M_p^2}
	\ .	
	\label{mass}
\end{equation}
Notice that the mass term in Eq.~\eqref{eq:FP} comes with the correct relative coefficient between $h^2$ and $h_{\mu\nu}h^{\mu\nu}$ for a ghost-free theory. We stress that such a coefficient is not finely tuned by hand, it follows directly from the quantum correction due to the functional measure.
Comparing the modulus of Eq.~\eqref{mass} to the bound found by LIGO \cite{LIGOScientific:2016lio}:
\begin{equation}
	m_g < 1.2 \times 10^{-22} \, \text{eV}
\end{equation}
translates into a bound on $\gamma$:
\begin{equation}
	\gamma < 2 \times 10^{-26} \, \text{GeV}^4 \ . 
\end{equation}

A few comments are in order. Because of \eqref{RGE}, the graviton mass \eqref{mass} runs with the energy scale.
%\begin{equation}
%	\Lambda\frac{\d m^2}{\d\Lambda}
%	=
%	- \frac{i\beta}{2\pi^2 M_p^2} \Lambda^4
%%	- \frac{i n \beta}{16\pi^{2} M_p \sqrt{\gamma(\Lambda)}} \Lambda^n
%	\ .
%\end{equation}
In particular, in the classical limit $\hbar \to 0$, the functional measure correction vanishes and so does the graviton mass $m$. We stress that the massless limit $m\to 0$ (or, equivalently, $\gamma\to 0$) is smooth as can be seen from the non-linear theory Eq.~\eqref{eom}.
%{\color{red} Therefore, no vDVZ discontinuity or strong-coupling issue arises for $m\to 0$. Vainshtein screening mechanism is thus not needed, which avoids issues with superluminal velocities.}
Therefore, all tests of general relativity are automatically satisfied. Secondly, since the gauge symmetry is not broken, the counting of degrees of freedom goes as usual for general relativity. Namely, a symmetric second rank tensor contains 10 degrees of freedom, 8 of which can be eliminated by gauge transformations, yielding only 2 propagating modes rather than 5. This follows because one has not started with massive gravity \textit{ab initio}. Indeed, since the functional measure does not contain derivatives, no new degrees of freedom show up and the spectrum continues to be determined from the classical Einstein-Hilbert action. The mass is generated only after quantization for the propagating modes that had already been present in the classical theory.
As a result, Boulware-Deser ghosts do not appear in the theory as the action Eq.~\eqref{effac} does not contain higher derivatives.

The resulting mass squared \eqref{mass} is, however, pure imaginary. The imaginary part of the mass corresponds to its width, which is a measure of its lifetime. When the width is much smaller than the real part of the mass, the particle is interpreted as a resonance, which can still be approximately treated as an asymptotic state. A pure imaginary mass, on the other hand, is highly unstable. They can only exist as virtual particles at intermediate stages of physical processes, mediating the gravitational interaction, but rapidly decaying into lighter by-products. Our result thus suggests that gravitons can never appear as asymptotic states.
Finally, we stress that our proposal has also the advantage of being a top-down approach. We indeed know the non-linear theory of massive gravity from the onset.

\section{Newtonian potential}
\label{newton}

Although the above findings suggest that gravitons cannot be measured at asymptotic states, virtual gravitons with complex mass do affect interactions.
An immediate consequence of such a massive graviton regards the modification on the Newtonian potential. As a result of the presence of mass, one usually expects a Yukawa potential. However, because the graviton mass is complex, the resulting potential shall develop an oscillating behavior modulated by the Yukawa decay, as we shall now see.

From Eq.~\eqref{eq:FP}, we obtain the effective equations of motion for the graviton:
\begin{equation}
    \Box h_{\mu \nu}
    - \Box h \eta_{\mu \nu}
    + \partial_{\mu} \partial_{\nu} h
    + \partial_{\alpha} \partial_{\beta} h^{\alpha \beta} \eta_{\mu \nu}
    - \partial^{\lambda} \partial_{\mu} h_{\lambda \nu} - \partial^{\lambda} \partial_{\nu} h_{\lambda \mu}
    =
    - M_p^{-1} T_{\mu \nu}
    - m^2 ( h_{\mu \nu} -  h \eta_{\mu \nu})
    \label{eomugly}
\end{equation}
The divergence and the trace of Eq.~\eqref{eomugly} read
\begin{align}
	\label{div}
    \partial^{\mu} h_{\mu \nu}
    &=
	\partial_\nu h
	\\
	\label{trace}
	h
	&=
	\frac{M_p^{-1}}{3 m^2} T
	\ .
\end{align}
Using Eqs.~\eqref{div}--\eqref{trace} in Eq.~\eqref{eomugly} gives
\begin{equation}
	\label{conservedgrav}
    (\Box + m^2) h_{\mu \nu}
    =
    - M_p^{-1} \left(
    	T_{\mu \nu}
    	- \frac{1}{2} T \eta_{\mu \nu}
    \right)
    + \left(
    	\partial_\mu\partial_\nu h
    	- \frac12 \eta_{\mu\nu} m^2 h
    \right)
    \ .
\end{equation}
The first term on the RHS of Eq.~\eqref{conservedgrav} is the usual general relativistic result (apart from the mass term on the LHS), which is then modified by the second term on the RHS, leading to the vDVZ discontinuity. In our case, because gauge invariance is not broken, the second term can be eliminated by a choice of gauge. In fact, under a diffeomorphism such a term transforms as
\begin{equation}
	\partial_\mu\partial_\nu h - \frac12 \eta_{\mu\nu} m^2 h
    \to
    \partial_\mu\partial_\nu h - \frac12 \eta_{\mu\nu} m^2 h
    + 2 \partial_\mu\partial_\nu\partial_\alpha \xi^\alpha
    - \eta_{\mu\nu} m^2 \partial_\alpha \xi^\alpha
    \ .
\end{equation}
Thus choosing $\partial_\alpha \xi^\alpha = -h/2$ results in:
\begin{equation}
	(\Box + m^2) h_{\mu \nu}
    =
    - M_p^{-1} \left(
    	T_{\mu \nu}
    	- \frac{1}{2} T \eta_{\mu \nu}
    \right)
    \ .
    \label{massiveh}
\end{equation}
One should notice the apperance of the factor of $1/2$ instead of the infamous $1/3$ of gauge-violating massive theories. Eq.~\eqref{massiveh} shows that the mass produced by the functional measure is perfectly consistent with all general relativistic tests as no vDVZ discontinuity takes place.

One can easily solve Eq.~\eqref{massiveh} in momentum space:
\begin{equation}
	\label{gensolcon}
    h_{\mu \nu}
    =
    M_p^{-1}
    \int \frac{\dd[4]{p}}{(2\pi)^4}
    \frac{e^{i p_{\alpha} x^{\alpha}}}{p^2 - m ^2}
    \left[
    	\tilde{T}_{\mu \nu}
    	- \frac{1}{2} \eta_{\mu \nu} \tilde{T}
    \right]
    \ .
\end{equation}
For a static point source of mass $M$ at the origin:
\begin{equation}
    T_{\mu \nu} = M \delta_{\mu 0} \delta_{0 \nu} \delta(\vec{x})
    \ ,
\end{equation}
one finds
\begin{equation}
    h_{0 0}
    =
    \frac12 \frac{M}{M_p} \frac{1}{4 \pi r} e^{ i m r}
    \ .
    \label{eq:newton}
\end{equation}
Despite Eq.~\eqref{eq:newton} having a Yukawa-like functional form, the complex exponential leads to novel predictions. Indeed, its real part provides the Newtonian potential~\footnote{We recall that $h_{00} = -2V$.}
\begin{align}
	V(r)
	=
	- \frac{M}{16 \pi M_p} \frac{e^{ - \tfrac{m_g}{\sqrt{2}}  r}}{r} \cos\left(\frac{m_g}{\sqrt{2}} \, r\right)
	\ ,
	\label{newton2}
\end{align}
whereas, by the optical theorem, its imaginary part relates to the total cross section~\footnote{Note that the imaginary part of Eq.~\eqref{eq:newton} remains finite at $r=0$ because $\sin(x)/x \to 1$.}. At small distances $r \ll \sqrt{2}/m_g$, our result recovers Newton's potential:
\begin{equation}
	V(r)
	=
	- \frac{M}{16 \pi M_p \, r}
	\left(1 - \frac{m_g}{\sqrt{2}} r + \frac{m_g^3}{6 \sqrt{2}} r^3 \right)
	+ \cdots
	\ .
\end{equation}
The leading correction is constant, thus does not affect the dynamics, so the first measurable new effect shows up only at next-to-leading order. 
We see that Newton's potential is modified at large distances $r \sim \sqrt{2}/m_g$.

The cosine function in Eq.~\eqref{newton2} can turn the potential's derivative positive, thus creating islands of bounded motion of decreasing depth. At each of these islands, gravity becomes repulsive. But because of the utterly small graviton mass, such an effect is only felt at very large distances:
\begin{equation}
	r > \frac{\pi\sqrt{2}}{2} m_g^{-1}
	\ .
\end{equation}
At the first and highest barrier $r_0 \sim 3.1 \, m_g^{-1}$, the potential height is given by (see Figure~\ref{fig:pot})
\begin{equation}
	V(r_0) \sim 4.2 \times 10^{-3} \, \frac{M m_g}{M_p}
	\ .
	\label{1well}
\end{equation}
% Figure environment removed
If there is enough energy to overcome this barrier, the system alternates between regions of attractive and repulsive gravity as the distance increases. The system could get trapped between two consecutive barriers, thus forming a gravitational bound state, should its energy be smaller than the potential well (see Figure~\ref{fig:pot2}). This effect, however, is rapidly weakened by the exponential suppression in Eq.~\eqref{newton2}. Even at the deepest well depicted in Figure~\ref{fig:pot2}, the existing energy from surrounding astrophysical events is likely enough to keep such bound states from forming.
On the other hand, should the energy be smaller than Eq.~\eqref{1well}, the system would not collapse as the distance decreases, thus resolving the singularity at $r=0$. Note that the ratio $m_g/M_p$ is utterly negligible in Eq.~\eqref{1well}, thus only very massive objects, such as black holes, could prevent such collapse from happening.

\section{Conclusions}
\label{conc}
%\setcounter{equation}{0}

Understanding the quantum nature of gravity requires, among other things, grasping the graviton kinematics and dynamics. Quantum corrections trigger new graviton self-interactions, which affect the graviton dynamics. As in any gauge theory, it is generally believed that the diffeomorphism invariance precludes the appearance of mass terms in the effective action of general relativity. However, this follows by side-stepping the non-trivial Jacobian that shows up in the path integral when diffeomorphisms (or, more generally, field redefinitions) are performed. A proper regularization of the Jacobian shows that it cannot be simply ignored.

A diffeomorphism-invariant functional measure thus requires the factor $\sqrt{\Det G_{ij}}$ to cancel out the functional Jacobian, in very much the same way that $\sqrt{-g}$ is needed for integrations in curved spacetimes. This additional factor contributes as a quantum effective potential for the graviton, giving it a mass in the linear regime. Therefore, general covariance is actually the reason for the existence, rather than for the absence as commonly thought, of a mass for the graviton. 

We stress that such a mass is not obtained by modifying general relativity. One is simply quantizing general relativity by properly considering the geometry of configuration space, which reflects in the definition of the functional measure. One can view different configuration-space metrics as different quantization approaches to the same classical theory. The naive approach, where the functional Jacobian is disregarded, corresponds to a flat configuration space.
Unfortunately, there is no known physical principle other than symmetry to help us determining the configuration-space metric. On symmetry grounds, the lowest-order configuration-space metric is curved and depends on the spacetime metric. This situation is dramatically different for Yang-Mills fields, where the lowest-order configuration-space metric is trivial and no gauge-invariant higher-order term exists. Therefore, gauge invariance manifests itself quite differently in gravity than in the other interactions, being the sole responsible for keeping Yang-Mills fields massless (in the unbroken vacuum) and the graviton massive.
%
%
%
%
 %
%
%
\section*{Acknowledgments}
%
%
%
IK is grateful to the National Council for Scientific and Technological Development -- CNPq (Grant No. 303283/2022-0) for partial financial support.

%%%%%%%%%%%%%%%%%%%%%%%%%%%%%%%%%%%%%%%%%%%%%%%%%%%%%%%%%%%%%%%%%
%%%
%%%                     BIBLIOGRAPHY
%%%
%%%%%%%%%%%%%%%%%%%%%%%%%%%%%%%%%%%%%%%%%%%%%%%%%%%%%%%%%%%%%%%%%
%
%
%
%
\begin{thebibliography}{99}
%\cite{Fierz:1939ix}
\bibitem{Fierz:1939ix}
M.~Fierz and W.~Pauli,
%``On relativistic wave equations for particles of arbitrary spin in an electromagnetic field,''
Proc. Roy. Soc. Lond. A \textbf{173} (1939), 211-232.
%doi:10.1098/rspa.1939.0140
%1715 citations counted in INSPIRE as of 02 Mar 2023

%\cite{Dvali:2000xg}
\bibitem{Dvali:2000xg}
G.~R.~Dvali and G.~Gabadadze,
%``Gravity on a brane in infinite volume extra space,''
Phys. Rev. D \textbf{63} (2001), 065007
%doi:10.1103/PhysRevD.63.065007
[arXiv:hep-th/0008054 [hep-th]].
%518 citations counted in INSPIRE as of 02 Mar 2023

%\cite{Dvali:2000hr}
\bibitem{Dvali:2000hr}
G.~R.~Dvali, G.~Gabadadze and M.~Porrati,
%``4-D gravity on a brane in 5-D Minkowski space,''
Phys. Lett. B \textbf{485} (2000), 208-214
%doi:10.1016/S0370-2693(00)00669-9
[arXiv:hep-th/0005016 [hep-th]].
%3244 citations counted in INSPIRE as of 02 Mar 2023

%\cite{Dvali:2000rv}
\bibitem{Dvali:2000rv}
G.~R.~Dvali, G.~Gabadadze and M.~Porrati,
%``Metastable gravitons and infinite volume extra dimensions,''
Phys. Lett. B \textbf{484} (2000), 112-118
%doi:10.1016/S0370-2693(00)00631-6
[arXiv:hep-th/0002190 [hep-th]].
%411 citations counted in INSPIRE as of 02 Mar 2023

%\cite{Bergshoeff:2009hq}
\bibitem{Bergshoeff:2009hq}
E.~A.~Bergshoeff, O.~Hohm and P.~K.~Townsend,
%``Massive Gravity in Three Dimensions,''
Phys. Rev. Lett. \textbf{102} (2009), 201301
%doi:10.1103/PhysRevLett.102.201301
[arXiv:0901.1766 [hep-th]].
%737 citations counted in INSPIRE as of 02 Mar 2023

%\cite{deRham:2010kj}
\bibitem{deRham:2010kj}
C.~de Rham, G.~Gabadadze and A.~J.~Tolley,
%``Resummation of Massive Gravity,''
Phys. Rev. Lett. \textbf{106} (2011), 231101
%doi:10.1103/PhysRevLett.106.231101
[arXiv:1011.1232 [hep-th]].
%1584 citations counted in INSPIRE as of 02 Mar 2023

%\cite{deRham:2014zqa}
\bibitem{deRham:2014zqa}
C.~de Rham,
%``Massive Gravity,''
Living Rev. Rel. \textbf{17} (2014), 7
%doi:10.12942/lrr-2014-7
[arXiv:1401.4173 [hep-th]].
%869 citations counted in INSPIRE as of 02 Mar 2023

%\cite{Boulware:1974sr}
\bibitem{Boulware:1974sr}
D.~G.~Boulware and S.~Deser,
%``Classical General Relativity Derived from Quantum Gravity,''
Annals Phys. \textbf{89} (1975), 193.
%doi:10.1016/0003-4916(75)90302-4
%224 citations counted in INSPIRE as of 02 Mar 2023

%\cite{vanDam:1970vg}
\bibitem{vanDam:1970vg}
H.~van Dam and M.~J.~G.~Veltman,
%``Massive and massless Yang-Mills and gravitational fields,''
Nucl. Phys. B \textbf{22} (1970), 397-411.
%doi:10.1016/0550-3213(70)90416-5
%1167 citations counted in INSPIRE as of 02 Mar 2023

%\cite{Zakharov:1970cc}
\bibitem{Zakharov:1970cc}
V.~I.~Zakharov,
%``Linearized gravitation theory and the graviton mass,''
JETP Lett. \textbf{12} (1970), 312.
%860 citations counted in INSPIRE as of 02 Mar 2023

%\cite{Vainshtein:1972sx}
\bibitem{Vainshtein:1972sx}
A.~I.~Vainshtein,
%``To the problem of nonvanishing gravitation mass,''
Phys. Lett. B \textbf{39} (1972), 393-394.
%doi:10.1016/0370-2693(72)90147-5
%1520 citations counted in INSPIRE as of 02 Mar 2023

%\cite{Babichev:2013usa}
\bibitem{Babichev:2013usa}
E.~Babichev and C.~Deffayet,
%``An introduction to the Vainshtein mechanism,''
Class. Quant. Grav. \textbf{30} (2013), 184001
%doi:10.1088/0264-9381/30/18/184001
[arXiv:1304.7240 [gr-qc]].
%353 citations counted in INSPIRE as of 02 Mar 2023

%\cite{Babichev:2009jt}
\bibitem{Babichev:2009jt}
E.~Babichev, C.~Deffayet and R.~Ziour,
%``Recovering General Relativity from massive gravity,''
Phys. Rev. Lett. \textbf{103} (2009), 201102
%doi:10.1103/PhysRevLett.103.201102
[arXiv:0907.4103 [gr-qc]].
%127 citations counted in INSPIRE as of 19 Jul 2023

%\cite{Deffayet:2001uk}
\bibitem{Deffayet:2001uk}
C.~Deffayet, G.~R.~Dvali, G.~Gabadadze and A.~I.~Vainshtein,
%``Nonperturbative continuity in graviton mass versus perturbative discontinuity,''
Phys. Rev. D \textbf{65} (2002), 044026
%doi:10.1103/PhysRevD.65.044026
[arXiv:hep-th/0106001 [hep-th]].
%491 citations counted in INSPIRE as of 19 Jul 2023

%\cite{Nicolis:2008in}
\bibitem{Nicolis:2008in}
A.~Nicolis, R.~Rattazzi and E.~Trincherini,
%``The Galileon as a local modification of gravity,''
Phys. Rev. D \textbf{79} (2009), 064036
%doi:10.1103/PhysRevD.79.064036
[arXiv:0811.2197 [hep-th]].
%1774 citations counted in INSPIRE as of 19 Jul 2023

%\cite{deRham:2016plk}
\bibitem{deRham:2016plk}
C.~de Rham, A.~J.~Tolley and S.~Y.~Zhou,
%``The $\Lambda_{2}$ limit of massive gravity,''
JHEP \textbf{04} (2016), 188
%doi:10.1007/JHEP04(2016)188
[arXiv:1602.03721 [hep-th]].
%39 citations counted in INSPIRE as of 19 Jul 2023

%\cite{Luty:2003vm}
\bibitem{Luty:2003vm}
M.~A.~Luty, M.~Porrati and R.~Rattazzi,
%``Strong interactions and stability in the DGP model,''
JHEP \textbf{09} (2003), 029
%doi:10.1088/1126-6708/2003/09/029
[arXiv:hep-th/0303116 [hep-th]].
%683 citations counted in INSPIRE as of 19 Jul 2023

%\cite{Nicolis:2004qq}
\bibitem{Nicolis:2004qq}
A.~Nicolis and R.~Rattazzi,
%``Classical and quantum consistency of the DGP model,''
JHEP \textbf{06} (2004), 059
%doi:10.1088/1126-6708/2004/06/059
[arXiv:hep-th/0404159 [hep-th]].
%549 citations counted in INSPIRE as of 19 Jul 2023

%\cite{Adams:2006sv}
\bibitem{Adams:2006sv}
A.~Adams, N.~Arkani-Hamed, S.~Dubovsky, A.~Nicolis and R.~Rattazzi,
%``Causality, analyticity and an IR obstruction to UV completion,''
JHEP \textbf{10} (2006), 014
%doi:10.1088/1126-6708/2006/10/014
[arXiv:hep-th/0602178 [hep-th]].
%879 citations counted in INSPIRE as of 19 Jul 2023

%\cite{deFromont:2013iwa}
\bibitem{deFromont:2013iwa}
P.~de Fromont, C.~de Rham, L.~Heisenberg and A.~Matas,
%``Superluminality in the Bi- and Multi- Galileon,''
JHEP \textbf{07} (2013), 067
%doi:10.1007/JHEP07(2013)067
[arXiv:1303.0274 [hep-th]].
%61 citations counted in INSPIRE as of 19 Jul 2023

%\cite{Mottola:1995sj}
\bibitem{Mottola:1995sj}
E.~Mottola,
%``Functional integration over geometries,''
J. Math. Phys. \textbf{36} (1995) 2470 
%doi:10.1063/1.531359
[arXiv:hep-th/9502109 [hep-th]].
%92 citations counted in INSPIRE as of 17 Nov 2022

%\cite{DeWitt:2003pm}
\bibitem{DeWitt:2003pm}
B.~S.~DeWitt,
``\emph{The global approach to quantum field theory}'', Oxford University Press, Oxford, 2003.
%Int. Ser. Monogr. Phys. \textbf{114} (2003), 1-1042
%95 citations counted in INSPIRE as of 14 Nov 2022

%\cite{Toms:1986sh}
\bibitem{Toms:1986sh}
D.~J.~Toms,
%``The Functional Measure for Quantum Field Theory in Curved Space-time,''
Phys. Rev. D \textbf{35} (1987) 3796.
%doi:10.1103/PhysRevD.35.3796
%46 citations counted in INSPIRE as of 17 Nov 2022

%\cite{Casadio:2022ozp}
\bibitem{Casadio:2022ozp}
R.~Casadio, A.~Kamenshchik and I.~Kuntz,
%``Background independence and field redefinitions in quantum gravity,''
Annals Phys. \textbf{449} (2023), 169203
%doi:10.1016/j.aop.2022.169203
[arXiv:2210.04368 [hep-th]].
%4 citations counted in INSPIRE as of 16 Apr 2023

%\cite{Kuntz:2022kcw}
\bibitem{Kuntz:2022kcw}
I.~Kuntz and R.~da Rocha,
%``Transport coefficients in AdS/CFT and quantum gravity corrections due to a functional measure,''
Nucl. Phys. B \textbf{993} (2023), 116258
%doi:10.1016/j.nuclphysb.2023.116258
[arXiv:2211.11913 [hep-th]].
%0 citations counted in INSPIRE as of 19 Jul 2023

%\cite{Vilkovisky:1984st}
\bibitem{Vilkovisky:1984st}
G.~A.~Vilkovisky,
%``The Unique Effective Action in Quantum Field Theory,''
Nucl. Phys. B \textbf{234} (1984) 125.
%doi:10.1016/0550-3213(84)90228-1
%359 citations counted in INSPIRE as of 17 Nov 2022

%\cite{Meetz:1969as}
\bibitem{Meetz:1969as}
K.~Meetz,
%``Realization of chiral symmetry in a curved isospin space,''
J. Math. Phys. \textbf{10} (1969) 589.
%doi:10.1063/1.1664881
%59 citations counted in INSPIRE as of 14 Nov 2022

%\cite{Slavnov:1971mz}
\bibitem{Slavnov:1971mz}
A.~A.~Slavnov and L.~D.~Faddeev,
%``Invariant perturbation theory for non-linear chiral lagrangian,''
Teor. Mat. Fiz. \textbf{8} (1971)  297.
%doi:10.1007/BF01029338
%27 citations counted in INSPIRE as of 14 Nov 2022

%\cite{Fradkin:1973wke}
\bibitem{Fradkin:1973wke}
E.~S.~Fradkin and G.~A.~Vilkovisky,
%``S matrix for gravitational field. ii. local measure, general relations, elements of renormalization theory,''
Phys. Rev. D \textbf{8} (1973) 4241. 
%doi:10.1103/PhysRevD.8.4241
%157 citations counted in INSPIRE as of 14 Nov 2022

%\cite{Fradkin:1976xa}
\bibitem{Fradkin:1976xa}
E.~S.~Fradkin and G.~A.~Vilkovisky,
%``On Renormalization of Quantum Field Theory in Curved Space-Time,''
Lett. Nuovo Cim. \textbf{19} (1977) 47. 
%doi:10.1007/BF02746592
%23 citations counted in INSPIRE as of 14 Nov 2022

%\cite{DeWitt:1967yk}
\bibitem{DeWitt:1967yk}
B.~S.~DeWitt,
%``Quantum Theory of Gravity. 1. The Canonical Theory,''
Phys. Rev. \textbf{160} (1967), 1113-1148.
%doi:10.1103/PhysRev.160.1113
%3014 citations counted in INSPIRE as of 12 Jul 2023

%\cite{Casadio:2021rwj}
\bibitem{Casadio:2021rwj}
R.~Casadio, A.~Kamenshchik and I.~Kuntz,
%``Covariant singularities in quantum field theory and quantum gravity,''
Nucl. Phys. B \textbf{971} (2021), 115496
%doi:10.1016/j.nuclphysb.2021.115496
[arXiv:2102.10688 [hep-th]].
%4 citations counted in INSPIRE as of 23 Nov 2022

%\cite{Kuntz:2022tat}
\bibitem{Kuntz:2022tat}
I.~Kuntz, R.~Casadio and A.~Kamenshchik,
%``Covariant singularities: A brief review,''
Mod. Phys. Lett. A \textbf{37} (2022) no.10, 2230007
%doi:10.1142/S0217732322300075
[arXiv:2203.11259 [hep-th]].
%1 citations counted in INSPIRE as of 23 Nov 2022

%\cite{Casadio:2020zmn}
\bibitem{Casadio:2020zmn}
R.~Casadio, A.~Kamenshchik and I.~Kuntz,
%``Absence of covariant singularities in pure gravity,''
Int. J. Mod. Phys. D \textbf{31} (2022) no.01, 2150130
%doi:10.1142/S0218271821501303
[arXiv:2008.09387 [gr-qc]].
%5 citations counted in INSPIRE as of 23 Nov 2022
%
%\cite{Ellicott:1987ir}
\bibitem{Ellicott:1987ir}
P.~Ellicott and D.~J.~Toms,
%``On the New Effective Action in Quantum Field Theory,''
Nucl. Phys. B \textbf{312} (1989), 700-714.
%doi:10.1016/0550-3213(89)90579-8
%31 citations counted in INSPIRE as of 12 Jul 2023
%\cite{DeWitt:1988dq}
\bibitem{DeWitt:1988dq}
B.~S.~DeWitt,
``The effective action'',
in Batalin, I.A. (Ed.) et al.: Quantum field theory and quantum statistics, Vol. 1, 191-222,
CRC Press (1987).
%3 citations counted in INSPIRE as of 27 Nov 2020

%\cite{LIGOScientific:2016lio}
\bibitem{LIGOScientific:2016lio}
B.~P.~Abbott \textit{et al.} [LIGO Scientific and Virgo],
%``Tests of general relativity with GW150914,''
Phys. Rev. Lett. \textbf{116} (2016) no.22, 221101
[erratum: Phys. Rev. Lett. \textbf{121} (2018) no.12, 129902]
%doi:10.1103/PhysRevLett.116.221101
[arXiv:1602.03841 [gr-qc]].
%1502 citations counted in INSPIRE as of 05 May 2023

\end{thebibliography}
%
\end{document}
