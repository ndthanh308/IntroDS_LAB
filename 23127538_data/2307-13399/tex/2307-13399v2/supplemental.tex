\cleardoublepage
\section*{Supplemental material}
Figure~\ref{fig:3} shows the mass distributions of the selected 
\Xibm and \Xibz signal candidates.
These distributions are fitted with an asymmetric Gaussian-like function
with power-law tails for the signal component, an exponential function for the combinatorial background and an empirical function for partially reconstructed backgrounds.
The fitted signal yields are 
$12\,020 \pm  140$ for $\Xibm\rightarrow \Xicz\pim$,
$27\,300 \pm  210$ for $\Xibz\rightarrow \Xicp\pim$,
$ 5\,680 \pm  100$ for $\Xibm\rightarrow \Xicz\pim\pip\pim$,
$11\,690 \pm  180$ for $\Xibz\rightarrow \Xicp\pim\pip\pim$ decays.
% Figure environment removed


Figures~\ref{fig:4} and~\ref{fig:5} show the mass-difference ($Q = m_{\Xib\pi\pi} - m_{\Xib} - 2 m_{\pi}$) distributions inside and outside the mass windows corresponding to the known $\Xib\pi$ intermediate resonances. Simultaneous fits are performed using the fit models presented in the Letter, with the resonance parameters fixed to their nominal values. Yields are allowed to vary in the fit. The observed suppression of decays not proceeding through the respective intermediate resonances confirms the decay pattern described in the text.
% Figure environment removed
% Figure environment removed




\clearpage
