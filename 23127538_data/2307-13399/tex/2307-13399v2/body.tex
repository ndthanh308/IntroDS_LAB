% introduction
The \Xibsigned baryons form an isospin doublet and are made of a $b$ quark,
an $s$ quark and a lighter $q$ ($u$ or $d$) quark.
Their ground states have angular momentum $L=0$ between the $b$ quark and the light diquark.
Three isospin doublets of such non-excited states are expected in the
quark model~\cite{Gell-Mann:1964ewy,Zweig:352337,Klempt:2009pi}
with different spin-parity $J^P$ and angular momentum of the $sq$ diquark $J_{sq}$.
The \Xibsigned, \Xibprime and \Xibstar states are characterized by  ($J_{sq}$, $J^P$) values of ($0,{\frac{1}{2}}^+$), ($1,{\frac{1}{2}}^+$) and ($1, {\frac{3}{2}}^+$), respectively.
Although five of these states have been observed
experimentally~\cite{PhysRevLett.99.052002,
CDF:2011ipk,
CMS:2012frl,
LHCb:2014nae},
the \Xibzprime baryon remains unobserved.
This may be because its mass lies below the threshold for the
$\Xibzprime\to\Xibm\pip$ decay~\cite{LHCb-PAPER-2020-016,PhysRevD.96.116016},
meaning that it only decays to either the $\Xibz\piz$ or $\Xibz\gamma$ final states making it experimentally challenging  to observe.
A number of excited states of higher mass is expected, with predictions for their properties available, \eg in Refs.~\cite{PhysRevD.98.031502,PhysRevD.99.094016,PhysRevD.96.116016,PhysRevD.104.054012,th2_2021,PhysRevD.103.094003,Chen:2014nyo,PhysRevD.84.014025,Roberts:2007ni}.
The \cms collaboration has reported the observation of the \XibmstarstarB resonance in the $\Xibm\pip\pim$ final state,
using \Xibm decays to final states containing \jpsi mesons~\cite{PhysRevLett.126.252003}.

In this Letter,
both the \Xibm\pip\pim and \Xibz\pip\pim final states and their intermediate \Xibm\pip and \Xibz\pim states
are investigated experimentally
(the inclusion of charge conjugate processes and the use of natural units are implicit throughout this Letter),
using $pp$ collision data collected by the \lhcb experiment
at centre-of-mass energies of $7,8,13$ \tev, corresponding to an integrated luminosity of 9\invfb.
The observation of three narrow states is reported and their properties are measured.





% Detector
The \lhcb detector~\cite{LHCb-DP-2008-001,LHCb-DP-2014-002} is a single-arm forward spectrometer
covering the \mbox{pseudorapidity} range $2<\eta <5$.
The detector includes a high-precision tracking system consisting of a silicon-strip vertex detector
surrounding the $pp$ interaction region~\cite{LHCb-DP-2014-001},
a large-area silicon-strip detector located upstream of a dipole magnet
with a bending power of about $4{\mathrm{\,Tm}}$, and three stations of silicon-strip detectors and straw
drift tubes~\cite{LHCb-DP-2013-003,LHCb-DP-2017-001} placed downstream of the magnet.
Different types of charged hadrons are distinguished using information
from two ring-imaging Cherenkov detectors~\cite{LHCb-DP-2012-003}.
Simulated data samples are produced with the software packages described in Refs.~\cite{Sjostrand:2007gs,LHCb:2011dpk,Lange:2001uf,Allison:2006ve,Clemencic_2011} and are used to model
the detector resolution and optimize the selection criteria.





% Selection
Samples of \Xibm (\Xibz) candidates are formed from $\Xicz\pim$ ($\Xicp\pim$) and
$\Xicz\pim\pip\pim$ ($\Xicp\pim\pip\pim$) combinations,
where the \Xicz (\Xicp) baryon is reconstructed in the $p \Km \Km \pip$ ($p \Km \pip$) final state.
To suppress background coming from promptly produced particles, all \Xibsigned decay products
are required to be displaced significantly from 
all primary $pp$ collision vertices (PVs) in the event.
The reconstructed \Xicz (\Xicp) candidates are required to have a mass within 20\mev (25\mev) of the
respective world-average mass values~\cite{PDG2022}.
Displaced pion tracks are combined with \Xic candidates to form \Xibsigned candidates, 
requiring good vertex-fit quality and significant displacement of the \Xibsigned decay point from any PV in the event.
All charged particles are required to have particle-identification (PID) information consistent with their respective mass hypotheses.
PID variables are based on neural network algorithms~\cite{DeCian:2255039} and their distributions
in simulation are calibrated 
using data~\cite{LHCb-DP-2018-001}.
If multiple candidates per collision event pass the selection requirements, all of them are preserved in the data sample.
The topological, kinematic and PID variables are used as inputs to a 
Boosted Decision Tree (BDT) classifier~\cite{ROE2005577} that discriminates  \Xibsigned signal candidates from random track combinations.
The classifier is trained using simulated \Xibsigned decays as a signal proxy 
and \Xibsigned data candidates in the sideband $5900 < m(\Xicsigned\pim,\Xicsigned\pim\pip\pim) < 6000\mev$ as a background proxy.
The mass distributions of the selected $\Xibsigned \rightarrow \Xicsigned \pim$ and $\Xibsigned \rightarrow \Xicsigned \pim\pip\pim$
candidates are shown in the supplemental material~\cite{supplemental}.
The decay mode $\Xibz \rightarrow \Xicp\pim\pip\pim$ is observed for the first time experimentally.
The selection requirement on the BDT classifier output is optimized for the observation of new states
and retains 96\% of the \Xibm and  92\% of the \Xibz signal candidates.
Additional vetoes  are imposed, as described in Ref.~\cite{LHCb-PAPER-2019-042}, to suppress other abundant processes with displaced vertices \eg those coming from \Dz, \Dp, \Ds and \Lc decays with a misidentified particle.

The \Xibsigned candidates within a $\pm 75$\mev window around the peak position are combined
with one charged pion (two pions) to investigate the
\Xibm\pip and \Xibz\pim (\Xibm\pip\pim and \Xibz\pim\pip)
mass spectra.
In order to improve the mass resolution, the
obtained candidates are refitted
with the masses of the \Xibsigned and \Xicsigned baryon candidates constrained to their known values~\cite{PDG2022}
and the \Xibsigned flight direction
to originate from the PV~\cite{Hulsbergen:2005pu}.
Additional requirements are applied to the $\Xib\pim\pip$ candidates, where 
the \Xibzstar, \Xibmprime and \Xibmstar intermediate states are selected
according to their observed widths and known mass values~\cite{PDG2022}.
Signal mass windows for the $\Xib\pi$ intermediate resonances are defined as $|m(\Xib \pi)-m_{\Xibzstar}|<3\mev$,
$|m(\Xib \pi)-m_{\Xibmprime}|<1.25\mev$
and $|m(\Xib \pi)-m_{\Xibmstar}|<5\mev$,
each corresponding to 2.5\,$\sigma$ of the observed experimental peak.





% Fit
The signal yields and lineshape parameters of the signal resonances are determined with an extended unbinned maximum-likelihood
fit to the $Q$-value distributions defined as
$m_{\Xib\pi} - m_\Xib - m_\pi$ and $m_{\Xib\pi\pi} - m_\Xib - 2m_\pi$
for $\Xib\pi$ and $\Xib\pi\pi$ decays, respectively.
The mass distributions of the
$\Xibm\pip$ and $\Xibz\pim$ ($\Xibm\pip\pim$ and $\Xibz\pim\pip$) samples are shown in Fig.~\ref{fig:1}~(Fig.~\ref{fig:2}), together
with the results of the fit.
All signal components are modeled using relativistic Breit--Wigner distributions~\cite{Jackson1964}
including Blatt--Weisskopf form factors~\cite{Blatt:1952ije} with a radius of $3\gev^{-1}$.
The orbital angular momentum between the \Xibsigned baryons and the pions is assumed according to the expected spin assignment.
The relativistic Breit--Wigner distributions are convolved with functions parameterizing
the detector resolution.
These resolution models are determined from simulation samples and are consistent with a resolution that scales as $\sqrt{Q}$.
Simulation shows that for each resonance, the resolution is comparable to or smaller than the measured natural widths of the peaks,
with the exception of the \Xibmprime baryon.
The background contribution is parameterized as $(Q - d)^n$, where $d$ and $n$
parameters vary freely in the fit.
This function, which is validated using wrong-charge $\Xibm\pim$ and $\Xibz\pip$ candidates, is sufficient to describe the smooth background coming from random track combinations.
Additional components are included in the fit to the \Xibz\pim spectrum  to describe 
partially reconstructed candidates coming from higher mass resonances.
These components are referred to as \emph{reflections} in the rest of this Letter.
The reflections of the newly observed states in the $\Xib\pi\pi$ mode to the $\Xib\pi$ spectrum
are studied with simulation
and dedicated components are included in the fit, modeled as Gaussian functions with power-law tails~\cite{Skwarnicki:1986xj}.
The means of the reflection components vary freely in the fits to data and their fitted values are consistent
with expectations from simulated backgrounds and cross-checks in data.
The fit confirms the presence of partially reconstructed
$\XibmstarstarB \rightarrow \Xibzstar(\Xibz\pi^0)\pim$ decays and shows hints of a contribution from the decay chain $\XibmstarstarA \rightarrow \Xibzprime(\Xibz\pi^0)\pim$, where 
neither the \XibmstarstarA state, the expected lighter resonance equivalent to that found in the neutral system, nor the \Xibzprime state has been observed experimentally to date (Fig.~\ref{fig:2}).
This component has been validated simulating different mass and width hypotheses for the two particles involved,
taking into account expected isospin splittings given the masses of their charged partners.
However, a precise estimation of the \XibmstarstarA and \Xibzprime state properties is not 
possible due to the limited yield and the presence of two unknown mass values.
The resolutions of the signal components are fixed to the values obtained from simulation.
The fit models are validated with \mbox{pseudoexperiments} and no significant bias is found on any of the parameters of interest.





% Results
The fitted yields in the $\Xib\pi$ mass spectra
are $2019 \pm 58$ for the \Xibzstar baryon,
$1750 \pm 50$ for the \Xibmprime baryon and $3380 \pm 110$ for the \Xibmstar baryon.
The \XibmstarstarB state~\cite{PhysRevLett.126.252003} is confirmed in the $\Xibzstar\pim$ mass distribution (Fig.~\ref{fig:2}a),
while two new peaks are observed in the $\Xibmprime\pip$  (Fig.~\ref{fig:2}b)
and $\Xibmstar\pip$  (Fig.~\ref{fig:2}c) mass distributions,
referred to as \XibzstarstarA and \XibzstarstarB in this Letter.
For the newly observed states, the signal yields are $136 \pm 17$ for
the \XibmstarstarB resonance, $147 \pm 19$ for the \XibzstarstarA resonance and $69 \pm 14$ for the \XibzstarstarB resonance.
The three peaks are observed with local significances of
18\,$\sigma$, 15\,$\sigma$ and 9\,$\sigma$, based on the differences in log-likelihood between a fit with the signal yield fixed to zero and the default fit.
These significances are reduced to  12\,$\sigma$, 10\,$\sigma$ and 8\,$\sigma$, once systematic uncertainties on the yields are taken into account.
% Figure environment removed
%
% Figure environment removed





% Systematic        
Different sources of systematic uncertainties are considered in the determination of the resonance parameters.
All these systematic uncertainties are summarized in Table~\ref{tab:sys}.
%
One of the most important contributions to the mass measurements comes from the knowledge of the momentum scale at \lhcb.
The associated systematic uncertainty is assigned as the larger of the changes in the measured parameters when the momentum scale is changed by its uncertainty, which is estimated to be $3 \times 10^{-4}$~\cite{LHCb-PAPER-2013-011}.
An additional uncertainty arises from the empirical description of the background shapes and is estimated by modeling them with alternative functions.
A third uncertainty is assigned by varying the description of the reflection components and their properties, obtained either from simulation or from data, where relevant.
%
Further sources of uncertainty on the measurement of the natural widths are included to describe the known differences in resolution between 
data and simulated events. Differences are expected to be within $5$\%, based on previous studies~\cite{LHCb-PAPER-2014-061,LHCb-PAPER-2017-002,LHCb-PAPER-2018-013,LHCb-PAPER-2018-032}, therefore uncertainties are estimated by varying the  resolution function width and the parametrization of the mass resolution function by that amount.
%
Possible uncertainties can arise from the  assumed relativistic Breit--Wigner distributions and their parameters. Lower mass states are assumed to decay with angular momentum $l=0$, while higher mass states with $l=1$.
For the newly observed states, the hypotheses assuming $l=0,2,3$ 
are tested and the largest shifts of the fitted parameters with respect to the default fit are assigned as systematic uncertainties.
%
A further uncertainty on the baryon mass $m_0$ is assigned due to the
limited knowledge of the \Xibm and \Xibz baryon masses~\cite{PDG2022}.
\begin{table}[t]
\caption{\small Systematic uncertainties (\mev) on the measured physical properties. The parameters
$Q_0$ and $\Gamma$ are the mean and the width of the Breit--Wigner distribution, respectively.}
\centering
\begin{tabular}{lccccccccc}
\hline
 & & \multicolumn{2}{c}{\Xibzstar}& & \multicolumn{2}{c}{\Xibmprime} & &\multicolumn{2}{c}{\Xibmstar} \\
 
Source         & & $Q_0$ &  $\Gamma$ & & $Q_0$ &  $\Gamma$ & & $Q_0$ &  $\Gamma$ \\
\hline
Momentum scale & & 0.006 &	0.001 & &  0.001 & 0.001& &	0.008&	0.001\\
Background	   & & 0.003 &	0.029 & &  0.000 & 0.006& & 0.004&	0.073\\
Reflections    & &       &        & &  0.000 & 0.000& & 0.002&  0.007\\
Resolution     & & 0.001 &	0.038 & &  0.002 & 0.027& &	0.000&	0.033\\
BW param.      & & 0.001 &	0.001 &	&  0.000 & 0.000& & 0.001&	0.002\\
%
Total		   & & 0.007 &  0.048 &	&  0.002 & 0.028& & 0.010&	0.081\\
\hline
\end{tabular}
%
\begin{tabular}{lccccccccc}
 & & \multicolumn{2}{c}{\XibmstarstarB}& & \multicolumn{2}{c}{\XibzstarstarA} & &\multicolumn{2}{c}{\XibzstarstarB} \\
Source         & & $Q_0$ &  $\Gamma$ & & $Q_0$ &  $\Gamma$ & & $Q_0$ &  $\Gamma$\\
\hline
Momentum scale & & 0.008 &	0.002 & & 0.007 &0.001 & & 0.009 & 0.006\\
Background	   & & 0.004 &	0.035 & & 0.022	&0.089 & & 0.023 & 0.025\\
Resolution	   & & 0.004 &	0.054 & & 0.001	&0.035 & & 0.006 & 0.073\\
BW param.      & & 0.016 &	0.050 & & 0.056	&0.007 & & 0.001 & 0.079\\
%																		
Total		   & & 0.019 &	0.081 & & 0.060	&0.096 & & 0.026 & 0.111\\
\hline
\end{tabular}
\label{tab:sys}
\end{table}
%





% Conclusions 
The numerical results are summarized in Table~\ref{tab:resultstot}. 
The properties of the \Xibzstar, \Xibmprime and \Xibmstar baryons are measured with world-leading precision.
For the narrow \Xibmprime state, its natural width is compatible with zero once the systematic uncertainties are considered, and an upper limit $< 0.05$\mev is estimated at 90\% confidence level.
\begin{table}[t]
\caption{\small Masses and widths of the states considered in this Letter.
The first uncertainty is statistical, the second systematic.
The third uncertainty on $m_0$ is due to limited 
knowledge of the \Xibm and \Xibz baryon masses~\cite{PDG2022}.}
\centering
\begin{tabular}{ll@{\,}  r @{$\,\pm\;$} c @{$\,\pm\;$} l  }
\hline
State & Observ. & \multicolumn{3}{c}{Value~$\mathrm{(MeV)}$}\\ 
%
\hline
\XibmstarstarB & $Q_0$   &    23.6 &  0.11  &  0.02  \\
               &$\Gamma$ &   0.94  &  0.30  &  0.08  \\
               &$m_0$    & 6099.74 & 0.11   &  0.02 {$\,\pm\;$}0.6 $(\Xibm)$\\
%\hline
%
\hline
\XibzstarstarA &$Q_0$     &  16.20 &  0.20  &  0.06  \\ 
               & $\Gamma$ &   2.43 &  0.51  &  0.10  \\
               & $m_0$    &6087.24 &  0.20  &  0.06 {$\,\pm\;$}0.5 $(\Xibz)$\\
\hline
%
\XibzstarstarB & $Q_0$    &  24.32  &  0.15  &  0.03  \\
               & $\Gamma$ &   0.50  &  0.33  &  0.11  \\
               & $m_0$    & 6095.36 &  0.15  &  0.03 {$\,\pm\;$}0.5 $(\Xibz)$\\
\hline
%
%
\Xibzstar      & $Q_0$    & 15.80   &  0.02  &  0.01  \\
               & $\Gamma$ & 0.87    &  0.06  &  0.05  \\
               & $m_0$    & 5952.37 &  0.02  &  0.01 {$\,\pm\;$}0.6 $(\Xibm)$\\
\hline
%
\Xibmprime     & $Q_0$    &  3.66   &  0.01  &  0.00  \\
               & $\Gamma$ &  0.03   &  0.01  &  0.03  \\
               & $m_0$    & 5935.13 &  0.01  &  0.00 {$\,\pm\;$}0.5 $(\Xibz)$\\
\hline
%
\Xibmstar      & $Q_0$    & 24.27  &  0.03  &  0.01  \\
               & $\Gamma$ &    1.43  &  0.08  &  0.08  \\
               & $m_0$    & 5955.74  &  0.03  &  0.01 {$\,\pm\;$}0.5 $(\Xibz)$\\
%
\hline
\end{tabular}
\label{tab:resultstot}
\end{table}
%

In summary, the first observation of two new baryons \XibzstarstarA and \XibzstarstarB, with quark content $bsu$, 
is reported in the \Xibz\pip\pim final state.
Additionally, this Letter confirms the observation of the \XibmstarstarB charged state by the \cms collaboration~\cite{PhysRevLett.126.252003},
with improved significance and sensitivity on its physical parameters. This measurement uses final states with up to nine tracks, most of which are pions, showing excellent performance of the LHCb tracking, reconstruction and PID systems.
Finally, the decay mode \mbox{$\Xibz \rightarrow \Xicp\pim\pip\pim$} is observed for the first time.
The properties of the \Xibzstar, \Xibmprime and \Xibmstar baryons are measured with high precision.
Determination of the spin and parity for the new baryons is not possible given the low signal yields. However, data indicate that the \XibmstarstarB baryon decays mainly through the \Xibzstar\pim state, the \XibzstarstarA baryon mainly through the \Xibmprime\pip state, and the \XibzstarstarB baryon mainly through the \Xibmstar\pip state,
with no significant contributions to the signals from events
outside their respective $m_{\Xib\pi}$ mass windows~\cite{supplemental}.
These patterns closely resemble those observed in the \Xicz and \Xicp baryon systems~\cite{PDG2022}.
An interpretation would be that the new states are $P$-wave states
($l=1$ between the $b$ quark and the $qs$ diquark)
coupling to the $b$ quark
to give a pair of states with $J^P = {\frac{1}{2}}^-$ and
${\frac{3}{2}}^-$.
One might expect the dominant decay mode of the lighter one
to be $\Xibprime\pi$ and for the heavier one $\Xibstar\pi$.
The lighter \XibmstarstarA state could not be observed as it would likely decay primarily through the intermediate $\Xibzprime$
resonance which is below threshold to decay to $\Xibm \pi^+$. However, hints of such 
$\XibmstarstarA \rightarrow \Xibzprime(\Xibz\piz) \pim$ decay
are observed in the $\Xibz\pi^-$ spectrum as a partially reconstructed feed-down component.

