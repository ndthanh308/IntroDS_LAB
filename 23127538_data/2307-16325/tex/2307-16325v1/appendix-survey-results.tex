\section{Survey Results}\label{app:survey-results}
The results of our survey helped us refine the semi-structure guide of questions for the interview.
We now describe the demographics as well as general results elicited from the survey responses.

\myparagraphnew{Demographics} Of the $39$ responses we received, $25$ worked in a full time employment, and $12$ worked as both freelancers and full-time employees.
Almost all of them ($85\%$) identified themselves as developers with $25$ of them having more than five years of professional experience and six with at least three years of experience.
$53\%$ participants helped release a new version of software or service at least on a monthly basis in the past two years, with $26\%$ on quarterly basis. All the participants ranked themselves as at least slightly knowledgeable in security, with five being extremely knowledgeable, eight very knowledgeable and $20$ moderately knowledgeable.
$50\%$ of the participants entered their location as Asia, with the rest distributed equally between North America, Europe, United Kingdom and Africa.

\myparagraphnew{Prioritizing Security by Organizations and Individuals}
Through the survey, we asked the participants to rate the importance of privacy, security against malicious attacks, ease of use, multi-platform compatibility, multitude of features and responsiveness with respect to applications they help develop from their individual perspective.
Furthermore, we asked the participants to rate how these are prioritized by their organizations based on their personal experience.

All participants individually expressed that securing against malicious attack is very important, with $83\%$ working in organizations expressing that it is of extreme importance.
However, from their organization's perspective, only $30/35$ participants shared that securing against malicious attacks is \textit{at least} very important, with two selecting slightly important and three moderately important. The remaining two participants chose not to answer.
In other words, the importance of security against malicious attacks might not be prioritized similarly by an organization and an individual of the same organization.
For similar questions about protecting privacy in software and or services, $25$ participants expressed that it is at least very important, with two selecting moderately important.
Similar to the trend observed for securing against malicious attacks, participants expressed that they think their organizations prioritizes privacy differently compared to themselves.


To summarize, {\em an organization and its practitioners can have significantly different priorities on security and privacy} for their software or services.


\myparagraphnew{Reliance on Automated and/or Manual Analysis Techniques} When asked how the participants relied on automated and manual techniques for finding security vulnerabilities,
seven participants expressed that they rely on automated techniques for reasons such as lack of security-related expertise, manual testing being time-consuming and for automatically preventing intruders from attacking their systems.

All the participants ($26/39$) who chose both automated analysis and manual analysis techniques expressed that they do it because of additional coverage, with the manual technique being used to cover corner cases, application specific logic, or out of scope issues.

Finally, the participants ($6/39$) who expressed that they rely only on manual analysis techniques shared that it is due to lack of effectiveness, or lack of resources, or due to simply being more comfortable with manual analysis techniques.

To summarize, {\em practitioners mostly rely on a combination of automated and manual techniques to increase coverage, with the only exceptions being an exclusive reliance on automated techniques due to lack of security expertise, and on only manual techniques due to expertise/comfort with the same.}

\myparagraphnew{Impact due to Unsound \sasts}
We asked participants how their software or service would get impacted in case there was a soundness issue  \sasts they use.
Interestingly, almost all practitioners expressed that \textit{even in the case of flaws of \sasts, their applications would be moderately impacted at most}, explaining that they do not entirely depend on these tools for ensuring security and instead \textit{rely on multiple tools and/or manual reviews}.



The few participants who shared that they would be significantly affected were either involved with tool development, or were entirely dependent on \sasts{}. In other words, {\em practitioners take the impact of flaws in security tools lightly as they use multiple tools and/or manual analysis techniques to overcome limitations}.

