\vspace{-0.5em}
\subsection{\interviewSectionSecAndOrg}\label{sec:interviewSectionSecAndOrg}
\vspace{-0.5em}

Unsurprisingly, all participants agreed that delivering secure software is important.
However, we were also interested to learn about the prioritization of security in organizations.
Therefore, we queried participants about the potential tension within their organizations related to prioritizing software security at the expense of features and vice versa.
Given that introducing and using \sasts in a development workflow requires nontrivial effort from individuals and potential financial investment from an organization, we expected most participants and organizations had a vested interest in prioritizing security.
We found that this expectation to generally hold true, with a few exceptions.



\myparagraphnew{Prioritizing Security vs Functionality Deadlines}
Most participants indicated a prioritization of security over deadlines, \eg as P08 states, \myquoteinline{Security gets the highest priority. Always.\ldots Even if we are not meeting the deadline, we cannot break this.}
We found that various factors can be responsible for necessitating this prioritization, \eg the need to be compliant with existing laws and standards:
\myquote{We serve the government \dots we need to have some certifications that we are complying \dots (If) we have a release tomorrow and the security team found a vulnerability today, we have to block that release, and we have to fix it. Then we will release that. \dots We cannot compromise that.}{P20~\textsubscript{Telematics}}
It can also be due to safety-critical and/or business-critical nature of the product being built, since a bug can be costly, both in terms of lives and financial measures:

\myquote{So our security and safety and usage of the static analysis tools is mostly to prevent bugs, which could be life-threatening, of course, but also they could cost us millions}{P04\textsubscript{Automobile Sensors}}
For open-source collaborations, the concept of deadlines may not be applicable at all.
As P02 described, security is always of priority, and there is \myquoteinline{No such thing as deadlines. It's ready when it's ready}.

\finding{Participants generally said that they err on the side of security, fixing any known vulnerabilities before releasing a feature, regardless of deadlines.}

However, some participants expressed that prioritizing security is not always possible, even when security is generally a high priority from the organization's perspective.
This can be due to the management prioritizing bug-fixing for the sake of users, as shared by P07,
\myquote{\dots Our user was facing a lot of issues. So, there was a deadline pressure on us to deliver the product very quickly}{P07~\textsubscript{B2B, SAAS}}
This is true even for a security-testing product, albeit rarely:
\myquote{So in most cases, we try to be really strict because it's a security testing product \dots Then it's kind of a business trade off. \dots a new feature, we can usually just delay it \dots If it's an existing feature that we now uncover the vulnerability, you can't usually switch off the feature because you have customers relying on it.\dots Eventually you get that fixed and then responsibly disclose it. \dots I think we had to do one of those in the five years I've been with the company.}{P01\textsubscript{Program Analysis for Security}}
The "overriding" of security to meet deadlines for existing features may incur heavy cost, however. P07 expressed the following after further conversation, \myquoteinline{That had a certain impact. We found \underline{.3 to .4 million <currency>} of fraudulent activity after that release}.

\finding{Select situations can lead to the de-prioritization of software security, including maintaining support for {\em existing} features, or fixing bugs that being experienced by prominent users.}

\add{This finding echoes similar observations in prior work, that in some cases security is forgone for functionality bugs or releasing other features~\cite{FCV+21,PTL+20,AC19,XWM14}.}

Further, contrary to our initial intuition, P06 shared that an organization may not prioritize security unless it is required by its clients. %
\myquote{Security is a great concern \dots So if the client is strict enough to focus on the security aspects, then we follow it. Other than that, actually our <previous org> do not care (about security) \ldots}{P06\textsubscript{Software Service}}
Unsurprisingly, several participants shared that an organization may not afford to miss functionality deadlines if it is still in startup or growth stage:
\myquote{When I worked on a startup environment, it is always expected that we ship the features to production as soon as possible...There is a little room to explore the security options \dots}{P10~\textsubscript{Healthcare}}
\finding{Participants expressed that in certain circumstances, organizations may entirely forego security considerations and prioritize releasing features as rapidly as possible, particularly when the client does not care, or when the organization is in its early growth stage.}



































