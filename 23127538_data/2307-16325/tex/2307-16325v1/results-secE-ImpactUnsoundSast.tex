\vspace{-1.5em}
\subsection{\interviewSectionImpactUnsoundSast}\label{sec:interviewSectionImpactUnsoundSast}
\vspace{-1em}
After learning about what participants expect from SASTs, we aimed to understand if and how participants were impacted by flaws in \sasts, \ie their inability to detect what they claim as ``in scope'', how participants generally addressed the flaws, and their experiences reporting the flaws to SASTs.

\myparagraphnew{Impact of Unsound \sast{}} All developers across survey and interviews, save for a few, shared that while they had experienced  false negatives, they had not experienced any adverse impact due to flaws/unsoundness in \sasts.
The practitioners explained that while false negatives are not observable since they are not reported by the SAST, they expect manual/code reviews to detect vulnerabilities missed by SASTs.
Therefore, as any false negatives resulting from even unknown flaws in SASTs are addressed by their manual reviews, unsound \sasts do not impact their software.


\finding{Practitioners are not overly concerned about the impact of unknown unsoundness issues in SASTs, as they expect subsequent manual reviews to find what the SAST missed.}




P18, who works with an internal static analysis team, offered an alternate explanation as to why developers may overlook false negatives of SASTs, or their impact, because the assumption is that \sasts \textit{just work}
\myquote{If the tools miss something, we can not detect that issue, and we just overlook the issues\dots because no one ever reports about false negatives, and we don't check if the tool ever miss the vulnerabilities}{P18~\textsubscript{Fortune 500 Global R\&D Center}}
\finding{Developers may use SASTs in a state of denial, i.e., assume that SASTs just work, and hence, simply overlook any evidence of false negatives, or flaws in the SASTs that lead to false negatives.}
Among the exceptions, P02's organization tried and stopped using \sast because of false negatives, thus effectively negating any potential impact, as previously described in~\ref{sec:interviewSectionLimitExpectSast}.
On the other hand, P01 shared that while their own \sast product unintentionally introduced a vulnerability, which could've impacted their clients, \myquoteinline{never public, no customer ever suffered}, as it was detected during development.


\myparagraphnew{Addressing/Reporting flaws to \sasts} Participants expressed that security is important, but shared challenges associated with reporting flaws to \sasts.

Generally, flaw reports consist of either a minimal code example that demonstrates the flaw, or actual code snippet from software.
However, P04 and P09 shared that going for either is problematic for two very different reasons. First, sharing actual code snippet may require going against company or client's  confidentiality policy. P04 circumvents this because of a pre-existing NDA between their organization and the \sast, \myquoteinline{we have an NDA signed, so in case I cannot get a small example, they can also check our source code}, whereas P09 is unable to do so.
\myquote{For certain external communications, it's a little bit difficult to do. What we can share with third party or other party is very strictly regulated by the state bank\dots. If we want something from <tool>, we have to justify why we are sharing this particular code snippet. In particular, I think if you don't share a large amount of code with them, they won't even be able to tell why this is problematic}{P09\textsubscript{Fintech}}
On the other hand, several participants stated that sometimes, developers are not willing to report flaws since it is "\textit{additional work}" (\ie reporting the flaw, following up): %
\myquote{We were asked to not do things on our own, because
they will maybe increase more pressure \dots  I would actually report it to my team lead, but I
don't think they would actually report it to back to them}{P05\textsubscript{Web Applications}}
\myquote{That might not happen as well because inherently developers are lazy. If you want to share this, you have to go through with certain things}{P11\textsubscript{Website Backend of Program Analysis for Security}}
Finally, some participants shared that while they have reported flaws to \sasts, the lack of response, or lack of addressing flaws discouraged them from reporting flaws later on.
P02 said, \myquoteinline{Nothing as far as I recall} when asked about whether anything happened after reporting false negatives to \sast, whereas P04 said that some \sast developers might be unwilling to accept a flaw as an issue.
\myquote{So, <SAST\textsubscript{A}>, we have a worse experience. They are mostly evasive, so they are not really progressing as <SAST\textsubscript{B}>. It takes a lot of time to convince them that they are bugs. Even though you have a small example, they still ask you to try different configurations and all that stuff, but we were aware of that before we came to this part, before we selected them. Because simply they (<SAST\textsubscript{A}>) are, I wouldn't say confident, but they are confident that their solution works.}{P04\textsubscript{Automobile Sensors}}

\finding{Participants may hesitate to report flaws/false negatives in SASTs for several reasons, ranging from prior negative experiences with SASTs (including inaction on reported flaws), or issues internal to the organization, such as the need to maintain product confidentiality (without an explicit NDA), red tape, and the lack of incentive to perform the additional effort.}

P01 shared some insight to decisions related to fixing flaws in \sast, sharing that while severity and likeliness ("correlates to presence in open source libraries") are motivating factors, so is what the business-competitors are detecting.
To understand this in-depth, we presented a hypothetical scenario to P01 where a class of vulnerability is ignored by the rest of the \sast building industry and asked how is it decided whether to address it in their \sast. P01's response was \myquoteinline{It depends on the effort and depends on how critical it is}.

\myparagraphnew{Exploiting Flaws and Evasive Developers} We adopted the concept of evasive developers from ~\cite{ACK+22, Wv08}, defined as a developer who actively attempts to bypass a \sast's checks.
The motives vary, such as malice, lack of stake (third-party contractor), and/or simply being lazy.
A majority of the participants stated that while they consider evasive developers realistic, such developers are unlikely to cause serious harm in their organizational context due to several factors, such as company policies \eg \myquoteinline{It is strictly prohibited, and it is communicated in that way that it is not acceptable to bypass those checks (P08)}, and manual code reviews.
\myquote{The process that we have is designed that, first, it needs to pass the review of the initial reviewer which allows it to get it on the main branch. So if we, put another hurdle here and we say that there are two friends which decide that this is okay, it still needs to come through the third guy who is gonna test, the test will kill. So that's already three guys that would need to accept the issue in the whole team.}{P04\textsubscript{Automobile Sensors}}
On the other hand, some participants shared that they have observed their colleagues being evasive, or they themselves attempted to be evasive due to stressed work environment.
\myquote{We had six people and one would actually do something like that.}{P05\textsubscript{Web Applications}}
\myquote{There was an extreme pressure because we \underline{needed to bypass the \sast tests}, otherwise we would not receive green flag from the security team. So it actually \underline{happened once}. We used to work late night to resolve all those conflicts and red flags.}{P06\textsubscript{Software Service}}
In contrast, P01 expressed that in an organization a developer being evasive is unlikely due to ownership at their organization, \myquoteinline{I want to believe that our developers are responsible \dots I don't believe anyone will try to game our system like that}.

\finding{The risk of evasive developers is real. That is, while some participants consider the scenario of ``evasive developers'' as adequately prevented by existing code reviews, this optimism is not universal: others have prior experience of evasive developers in their teams, or have evaded SASTs themselves.}


