\section{Survey Protocol}\label{sec:online-survey}
To understand how practitioners perceive security tools, and whether security is prioritized by individuals and organizations similarly, we prepared an online survey questionnaire (abridged questionnaire in Appendix~\ref{app:survey-questions}) and drafted a research protocol.
We piloted the initial survey with five participants. Three of the participants were graduate students, and the rest had doctoral degrees. All pilots were from computer science background, with additional experience in software engineering and/or security.
By incorporating their feedback, we improved the survey by modifications and additional descriptions as necessary.
Our final survey protocol received the approval of our Institutional Review Boards (IRBs).
The experimental protocol of both our survey and interviews included a consent form which emphasized that the data of the participants will remain confidential and de-identified.
Furthermore, a participant could optionally submit their email address to have the chance of winning one of two $\$50.00$ gift cards or the equivalent value in local currency vouchers. The winners would be chosen from qualified participants who completed the survey and provided valid responses in the survey.



\subsection{Survey Recruitment}
To diversify our recruitment approach in terms of experience, culture and industry contexts, we leveraged multiple recruitment channels.
We sent invitation emails describing the goal of the survey (\ie in order to learn about their professional experiences and opinions about \sasts) to the following groups of people:

\myparagraphnew{Professional Networks}: We relied on Snowball-sampling~\cite{SnowballSamplinggoodman1961a} to recruit from our professional networks across borders.
We sent survey invitation emails that similarly contained research goal. Furthermore, we requested them to forward the invitation to their relevant colleagues who might be potentially interested.
Through this approach, we received $21$ complete and valid survey responses.

\myparagraphnew{Open Source Software (OSS) Developers}: We emailed OSS developers who had interacted with \sasts via CI/CD actions, such as GitHub Workflows, in open-source repositories that $(a)$ had at least one star or watcher, $(b)$ were not a fork, and $(c)$ used one of the top ten programming languages in GitHub between Aug - Sept 2021. We collected publicly available email addresses only, and explicitly stated our recruitment procedure in our initial contact, which is common in other recent studies (\eg \cite{EBW22}).
\\\textit{\underline{Ethical Considerations:}}
We considered several \textit{potential trade-offs} that factored into this recruitment strategy, in addition to following the guidance provided by our IRB:
$(a)$ It is \textit{difficult} to recruit practitioners across borders who have the relevant experience, \ie configured and used automated security analysis tools,
$(b)$ we were collecting publicly available information and not amplifying the public visibility of the participants email address or other information, and
$(c)$ we carefully considered the Menlo Report's ethical guidelines on Information and Communication Technology Research from the Dept. of Homeland Security~\cite{KD12,DKB13}.
Specifically based on these guidelines, the only potential \textit{harm} to an invited person would be receiving one unsolicited email, whereas the benefit of this research is potentially helping create more secure software, for everyone, by understanding the needs and challenges of practitioners related to security analysis techniques.
In the end, we contacted $1,918$ potential participants using via email exactly once and received $18$ complete and valid survey responses.




\subsection{Online Survey and Data Analysis}
Our survey (Appendix~\ref{app:survey-questions}) consisted of Likert Scale based questions, with optional, open-ended response to clarify their selected choice(s).
Our analysis prioritized the text-based responses since these provided additional context for the selected choice(s) in Likert scale.
One of the authors open-coded the responses for analysis.
The responses of the survey guided our Interview protocol, which we describe next.
