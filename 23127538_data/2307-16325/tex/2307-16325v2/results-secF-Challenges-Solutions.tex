\subsection{\interviewSectionChallengeSolution}\label{sec:interviewSectionChallengeSolution}

Finally, we wanted to learn about the pain-points of participants related to \sasts.
Our approach was to present hypothetical (but ideal) scenarios, such as unlimited resources to fix or address just one issue of \sast, with the goal of getting the participants to focus on the most severe \sast-specific issues in their perspective. %


A few participants wanted to invest their resources on improving analysis techniques, both for reducing false negatives \eg \myquoteinline{I guess the first thing would be I would try to make it so that we're covering all of the most obvious} (P14), and providing meaningful alert messages \eg
\myquote{So the static analysis tool should be able to detect all the security issues within its scope and within possibilities. It should show meaningful messages \dots it should expose enough information about the issue so that the respective developer can address the issue easily}{P10~\textsubscript{Healthcare}}
Alternately, P02 (who was generally unimpressed by SASTs throughout the study) wanted unlimited human resources for manual analysis:
\myquote{If I've got unlimited time and resources, then some poor, unfortunate soul is going \dots going to have to go through all of the false positives in <SAST> and just confirm that they are actually false positives because there's just so many of them. \dots If those unlimited resources included some experienced security researchers, I get them doing some manual analysis. Because to be perfectly honest, the best vulnerability reports we get, which generally tends to be the more serious issues, they're not found by tools, they're found by people}{P02\textsubscript{OSS - Server}}
Several participants focused on \sast CI/CD integration issues explaining that often configuration is a major pain-point for them \eg \myquoteinline{I would definitely say integrations would be the top. \dots I think the best example would just be for all major CI/CDs to have an open source example of how to implement and integrate with various things.}. Other responses covered niches, such as better language-specific support, concurrency and abstraction support.

Finally, participants generally agreed that actionable reports that explain what can be done to address an issue, or provide more context, would be useful \eg \myquoteinline{if you write this code like this, this issue should be resolved} (P12), and \myquoteinline{An explanation of why the tool flagged that particular code is very helpful. It saves us having to second guess on why is the tool reporting that} (P02).

\finding{The key pain points for developers when it comes to SAST tools include: false negatives, lack of meaningful alert messages/reports, and configuration/integration into product CI/CD pipelines.}

