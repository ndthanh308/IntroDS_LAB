\subsection{\interviewSectionOrgOfSast}\label{sec:interviewSectionOrgOfSast}
\vspace{-0.5em}
With the understanding of how security is viewed from a participant's perspective in the context of their organization, we aimed to understand their organizational context of \sasts.
Particularly, we asked participants about reasons for using \sasts{}, the selection process of \sasts, and how they generally use those \sasts in their workflows.
Although participants named \sasts, \eg Coverity, Find-Sec-Bugs, SonarQube, Semmle/CodeQL, WhiteSource/Mend, CryptoGuard, Fortify, and VeraCode, we anonymize such details to reduce the chance of profiling developers and to avoid creating any impressions specific to any particular \sast.


\myparagraphnew{Selecting \sasts}
To start, we asked participants about the events that led to choosing a given \sast{} and to walk us through the selection process at their organizations.
Interestingly, we did not find any pattern in the selection processes that would hold true for a majority of the participants.

Particularly, only P04 shared that they performed a multi-stage evaluation, \ie they started with a preliminary list of $10-15$ \sasts, and filtered to four \sasts based on their own product-specific needs. Next, they evaluated those four \sasts using a {\em custom} benchmark, and settled on a \sast that was the most usable.


Further, six practitioners shared that they chose \sasts solely based on popularity, developer friendly documentation and/or ease of use.
\myquote{We didn't evaluate that many tools in terms of static analysis tools. We take what is the industry standard across different companies. Like <tool> is pretty popular, so that is our first choice.}{P08\textsubscript{Media, Web and Back-end servicess}}
P09 additionally mentioned that \textit{<SAST\textsubscript{A}>} was chosen due to regulatory reasons, \myquoteinline{I believe it was either PCI DSS requirement or a regulatory requirement}, admitting that they do not remember the exact standard.
On the other hand, four participants reported using previous experience or familiarity to select a \sast.
\myquote{Because I actually inherited some of that. The person who, actually set a lot of it up ... I think that it was what was available and what he was familiar with at the time.}{P03\textsubscript{OSS - Internet Anonymity Network}}
Several developers justified that they prefer freely available \sasts because it helps cut costs,
\eg P07 said \myquoteinline{...As of now, we are looking at a free solution. If we find benefits, then we'll go for the paid solution ...}

Corporate influence is an additional factor for selecting a particular \sast, particularly when it comes down to cost, \eg as P14 said, \myquoteinline{A lot of it comes down from management, \dots because they're the ones that are paying for it}.
In a similar vein,
P08 shared:
\myquote{We have different teams and different teams have different requirements. In my team, we use <SAST\textsubscript{A}>, and it is enforced by the team leader or the team owner to use <SAST\textsubscript{A}> as a code analysis tool.}{P08\textsubscript{Media, Web and Back-end Services}}
\finding{Participants generally recall selecting SASTs due to factors such as recommendations/reputation, ease of use/integration, corporate pressure, cost, or compliance requirements. Only {\em one} participant selected a SAST for their product via exhaustive testing of 10-15 tools using a (custom) benchmark.}
Furthermore, we asked participants about whether they considered using benchmarks, such as the OWASP benchmark while selecting \sast.
Most participants said that they were not familiar with any benchmarks, with the rest sharing that benchmarks are not representative of their specific application context, \eg \myquoteinline{The thing is, OWASP is something that only covers your basics. It doesn't go beyond} (P09).
Furthermore, P01 shared that while community-based benchmarks such as OWASP are usually neutral, many others are biased.
\myquote{%
Quite a few of these benchmarks are created by tool vendors where their tool finds some specific edge case. No one in the right mind would write an application like this, but their tool finds a specific edge case, so they put it in the benchmark.}{P01\textsubscript{Program Analysis for Security}}
\vspace{0.25em}
\finding{Participants who are aware of benchmarks generally do not trust them for evaluating/selecting SASTs, viewing benchmarks as either too basic to model real problems, or biased towards specific SASTs, given that vendors often contribute to their construction.}
\myparagraphnew{Preference between Manual Techniques and \sast{}}
As expected, participants who use \sasts stated that they found them useful, regardless of the selection process. %
Several participants shared that they use \sasts because they help focus manual analysis efforts on non-trivial issues by \textit{automatically} finding the trivial issues, \eg \myquoteinline{\dots helps find all the stupid stuff for you. Then you can concentrate on the actual logic (P01)} and makes it easier to analyze a large code base, \eg \myquoteinline{Is it possible to go through each of the code change by a human being? (P20)} and \myquoteinline{I think they're absolutely useful. It kind of reduces the number of mistakes you can make} (P09).
Furthermore, several shared that it is helpful for applying a rigorous quality control to the whole code base without being affected by subjective analysis, \eg
\myquote{Lot of reasons to be paranoid about it. None of us really, totally trust ourselves. And so, we need to have these tools to make the job of finding our own mistakes easier. If only one person is working on a thing, you're stuck with only that person's blind spots.
}{P03\textsubscript{OSS - Internet Anonymity Network}}
\finding{Participants consider SASTs highly useful for both reducing developer effort and helping to cover what subjective manual analysis may miss.}
\myparagraphnew{Reasons for not relying on \sast{}}
Finally, we had two participants in our study who do not rely on \sasts.
P13 stated that while their product needs to be secure, it is not public-facing, \ie \myquoteinline{Even if there is a problem in some projects, so one can access those deployed or the application from outside of our internet}.
Interestingly, P02 shared that while they have tried premium \sasts, they did not find them useful in their particular application niche, \ie web servers, stating that
\myquote{The primary issue with the <generic SAST> tools, every time we've looked at these tools, is it's all false positives and no genuine issues at all, which is somewhat demoralizing if you try to wade through large amounts of these reports.
}{P02\textsubscript{OSS - Java App Server}}
That is, as P02 further elaborated, since their product is a web server, it is required to handle "vulnerable" requests, such as "HTTP" headers, in code based on existing standards. These code components, however, trigger \sasts built to target web-applications, resulting in high false positives.

\finding{\update{The few participants who do not use \sasts{} cite the lack of a ``{\em fit}'' for their product: \ie{} as the product does not need extensive testing (echoing similar observations in prior work~\cite{WZW+15}), or because generic \sasts{} flag features (\eg handling standard-mandated vulnerable HTTP requests) as vulnerabilities.}
}
