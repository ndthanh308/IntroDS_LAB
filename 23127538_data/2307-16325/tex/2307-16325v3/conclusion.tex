\section{Conclusion}
\label{sec:conclusion}

This paper provides a comprehensive understanding of how practitioners with diverse business and security needs choose \sasts, and their perspectives and assumptions about limitations of \sasts.
By qualitatively analyzing the responses from $20$ in-depth interviews, we uncover $17$ key findings that demonstrate that contrary to existing literature, practitioners have a higher level of tolerance for false positives, and prioritize avoiding false negatives.
Moreover, we find that practitioners, regardless of their strong preference for security, rely on reputation to choose \sasts, as they {\em do not trust benchmarks} or find them reliable.
Finally, practitioners may be overconfident in assuming their ability to address a \sast's flaw with manual analysis, and are generally hesitant to report such flaws.
We conclude with research directions towards automated evaluation of \sasts, aligning \sasts with what developers desire, and creating dedicated protocols for reporting flaws in \sasts{}.
