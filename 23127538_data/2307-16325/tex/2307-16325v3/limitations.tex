\vspace{-0.5em}
\section{Threats to Validity}\label{sec:limitations}
\vspace{-0.5em}


\add{
This study seeks to understand the diverse perspectives of practitioners with different types of business and security needs, and is affected by the following threats to validity:

\myparagraphnew{Internal Validity} Practitioners with different experiences and roles at organizations may provide responses influenced by over/under-reporting, self-censorship, recall, and sampling bias.
We mitigated these factors by asking participants to share organization-specific incidents and experiences, with follow-up questions to understand their context, and reassuring that the responses would remain anonymous and untraceable (Section~\ref{sec:interview-guide}).
Moreover, some participants may have experienced loss of agency in selecting \sasts{} (\eg P08, \fnumber{4}).
However, all our participants have played key roles in selecting {\em or} using \sasts in their organizations (see Section~\ref{sec:interviewSectionOrgOfSast}), leading to useful experiences and observations that reveal meaningful patterns in SAST selection.


\myparagraphnew{External Validity} Due to the nature of interview-based qualitative research focusing on a specific experience (here: with \sast{}), \textit{generalizability} is considered an issue for recruitment through snowball/convenience sampling.
Findings from such studies are considered \textit{"softly generalisable"}~\cite{braun2021thematic}.
Prior research demonstrates that such studies are reliable for identifying salient trends~\cite{AB96,DLB05}; indeed, given the diverse organizational and product contexts of our participants, their responses provide key insight into how \sasts{} are used in practice in complex organizations.

In other words, given the number of our participants (n=20), and the recruitment process, we do not claim that the participants are representative of the broader developer population, or that the findings are \textit{generalizable}.
That said, this study captures and analyzes the experiences of participants from diverse organizational and security contexts and offers salient insights related to the use of \sasts{} in practice.

}
