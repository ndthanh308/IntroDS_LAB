\documentclass[12pt]{amsart}

\usepackage[indent=2em, skip=0.5em]{parskip}

\usepackage{amsmath,amssymb,amsthm}
\usepackage{microtype}
\usepackage[margin = 1in]{geometry}
\usepackage{graphicx}
\usepackage[usenames,dvipsnames]{xcolor}
\usepackage{hyperref}
\hypersetup{
    colorlinks=true,
    linkcolor=cyan!80!black,
    citecolor=MidnightBlue,
    urlcolor=magenta,
}
\usepackage{yhmath}
\usepackage{bm}
\usepackage{constants}

\usepackage{booktabs}
\usepackage{multirow}
\renewcommand{\arraystretch}{1.1}

\theoremstyle{plain}
\newtheorem{theorem}{Theorem}
\newtheorem{proposition}{Proposition}
\newtheorem{example}{Example}
\newtheorem{definition}{Definition}
\newtheorem{assumption}{Assumption}

\theoremstyle{remark}
\newtheorem{remark}{Remark}

\let\hat\widehat
\let\tilde\widetilde

\newcommand{\fx}{\mathfrak{x}}
\newcommand{\fb}{\mathfrak{b}}
\newcommand{\fa}{\mathfrak{a}}
\newcommand{\fs}{\mathfrak{s}}
\newcommand{\state}{\varphi}
\newcommand{\st}{\,:\,}
\newcommand{\bone}{\mathbf{1}}
\DeclareMathOperator{\cov}{Cov}
\DeclareMathOperator{\tr}{Tr}
\DeclareMathOperator{\aic}{AIC}
\DeclareMathOperator{\saic}{soft-AIC}
\DeclareMathOperator{\bic}{BIC}
\newcommand{\gaic}[1]{\aic^{(#1)}}
\DeclareMathOperator{\scree}{scree}
\newcommand{\ind}{\mathbb{I}}
\newcommand{\cN}{\mathcal{N}}
\newcommand{\bP}{\mathbb{P}}
\newcommand{\bE}{\mathbb{E}}
\newcommand{\bR}{\mathbb{R}}
\newcommand{\bI}{\mathbb{I}}
\newcommand{\convas}{\xrightarrow{\mathrm{a.s.}}}

\newcommand{\semicirc}{\mathrm{sc}}
\newcommand{\tw}{\mathrm{TW}}
\newcommand{\BBP}{\mathrm{BBP}}
\newcommand{\soft}{\mathrm{soft}}

\title[Model selection in the spiked Wigner model]{Consistent model selection in the spiked Wigner model via AIC-type criteria}
\author[S. S. Mukherjee]{Soumendu Sundar Mukherjee}
\address{Statistics and Mathematics Unit, Indian Statistical Institute, 203 B. T.~Road, Kolkata 700108, India}
\email{ssmukherjee@isical.ac.in}

\begin{document}

\begin{abstract}
Consider the spiked Wigner model
\[
    X = \sum_{i = 1}^k \lambda_i u_i u_i^\top + \sigma G,
\]
where $G$ is an $N \times N$ GOE random matrix, and the eigenvalues $\lambda_i$ are all spiked, i.e. above the Baik-Ben Arous-P\'ech\'e (BBP) threshold $\sigma$. We consider AIC-type model selection criteria of the form
\[
    -2 \, (\text{maximised log-likelihood}) + \gamma \, (\text{number of parameters})
\]
for estimating the number $k$ of spikes. For $\gamma > 2$, the above criterion is strongly consistent provided $\lambda_k > \lambda_{\gamma}$, where $\lambda_{\gamma}$ is a threshold strictly above the BBP threshold, whereas for $\gamma < 2$, it almost surely overestimates $k$. Although AIC (which corresponds to $\gamma = 2$) is not strongly consistent, we show that taking $\gamma = 2 + \delta_N$, where $\delta_N \to 0$ and $\delta_N \gg N^{-2/3}$, results in a weakly consistent estimator of $k$. We also show that a certain soft minimiser of AIC is strongly consistent.
\end{abstract}

\maketitle

%%%%%%%%%%%%%%%%%%%%%%%%%%%%%%%%%%%%%%%%%%%%%%%%%%%%%%%%%%%%%%%%%%%%%%%%%%%%%%%%
\section{Introduction}\label{sec:intro}
%%%%%%%%%%%%%%%%%%%%%%%%%%%%%%%%%%%%%%%%%%%%%%%%%%%%%%%%%%%%%%%%%%%%%%%%%%%%%%%%
    Model selection criteria such as the Akaike Information Criterion (AIC) \cite{akaike1998information} or the Bayesian Information Criterion (BIC) \cite{schwarz1978estimating} are staples of classical statistics. It is well-known that in classical fixed dimensional settings, AIC is not consistent in the sense that it tends to select models of higher complexity that the true one. BIC, which adds a more severe complexity penalty, is known to be consistent. See, for instance, \cite{bozdogan1987model, claeskens2008model}.

    There is a recent line of work that studies the consistency properties of various classical model selection criteria in high-dimensional settings. For instance, \cite{bai2018consistency} showed that in the \emph{spiked covariance model} \cite{johnstone2001distribution} of high-dimensional principal components analysis, where the population covariance matrix has a small number of so-called spiked eigenvalues separated from the rest, AIC is consistent but requires more separation for the spiked eigenvalues than what the so-called Baik-Ben Arous-P\'ech\'e (BBP) threshold demands (loosely speaking, this is a spectral threshold for the spiked eigenvalues below which the behaviour of the extreme eigenvalues of a spiked model resemble those of a non-spiked pure noise model, thereby rendering consistent model selection moot). Later, in \cite{chakraborty2020high} and \cite{hu2020detection}, it was shown that the per parameter penalty in AIC can be modified suitably using results from random matrix theory so that the resulting criterion becomes strongly consistent above any arbitrary spectral threshold above the BBP threshold. Further, it was also shown in \cite{chakraborty2020high} that an estimator may be obtained by tweaking the above-mentioned criterion, which becomes weakly consistent just above the BBP threshold. Note that, in contrast to the fixed dimensional situations, BIC becomes inconsistent. Recently, AIC, BIC, and several other model selection criteria have also been analysed in the context of high-dimensional linear regression \cite{bai2022asymptotics, bai2023koo}.

    In this article, we consider another natural high-dimensional statistical model, the so-called \emph{spiked Wigner model}, where one observes a low-rank signal matrix perturbed (additively) by a Wigner random matrix. There has been a lot of recent interest in the signal detection problem for this model, especially for the rank-one version (an incomplete list of recent works include \cite{perry2018optimality, chung2019weak, jung2020weak, jung2021detection, el2020fundamental, chung2022asymptotic, chung2022weak, jung2023detection, pak2023optimal}). Below the BBP threshold for this model, although consistent detection is not possible, one can still detect the presence of the signal with non-trivial probability \cite{el2020fundamental}. 
In this article, we consider the problem of consistently estimating the number of spiked eigenvalues (i.e. eigenvalues above the BBP threshold), which is essentially a problem of model selection. We ask if classical model selection criteria such as the AIC work in this setting. To that end, consider model selection criteria of AIC type:
\[
    -2 \, (\text{maximised log-likelihood}) + \gamma \, (\text{number of parameters}),
\]
with AIC corresponding to the special case $\gamma = 2$. We refer to the above criterion as $\gaic{\gamma}$.

    Contrary to the spiked covariance model, where AIC is strongly consistent under a certain amount of extra signal above the BBP threshold, we cannot say here that AIC is strongly consistent. However, with a per parameter penalty factor of $2 + \delta_N$, where $\delta_N \to 0$ and $\delta_N \gg N^{-2/3}$, we can show weak-consistency of the resulting criterion $\gaic{2 + \delta_N}$. We also show that a certain soft minimiser of AIC is strongly consistent. In general, for $\gamma > 2$, $\gaic{\gamma}$ is strongly consistent provided $\lambda_k > \lambda_{\gamma}$ where $\lambda_{\gamma}$ is a threshold strictly above the BBP threshold. For $\gamma < 2$, $\gaic{\gamma}$ almost surely overestimates $k$. Our empirical results suggest that these results are valid under more general noise profiles that shun the assumption of independent entries.

    The rest of the paper is organised as follows. In Section~\ref{sec:set-up}, we discuss the spiked Wigner model and associated random matrix theoretic results, derive the AIC-type criteria explicitly and state our main results. In Section~\ref{sec:simu}, we report empirical results comparing the various proposed estimators via several simulation experiments. Finally, Section~\ref{sec:proofs} collects all the proofs.

%%%%%%%%%%%%%%%%%%%%%%%%%%%%%%%%%%%%%%%%%%%%%%%%%%%%%%%%%%%%%%%%%%%%%%%%%%%%%%%%
\section{Set-up and main results}\label{sec:set-up}
%%%%%%%%%%%%%%%%%%%%%%%%%%%%%%%%%%%%%%%%%%%%%%%%%%%%%%%%%%%%%%%%%%%%%%%%%%%%%%%%
We first recall the definition AIC for a generic model selection problem. Suppose we have a collection of $q$ candidate (finite-dimensional) statistical models $M_1, \ldots, M_q$. Let $d_{M_j}$ denote the dimension of the parameter space under model $M_j$. This is intuitively a simple measure of model complexity. AIC attaches a score
\[
    \aic_j = - 2 \ell_{M_j} + 2 d_{M_j}
\]
to the model $M_j$. Here $\ell_{M_j}$ is the maximised log-likelihood under model $M_j$. In a decision-theoretic set-up, AIC is derived as an unbiased estimator of the risk of the MLE when one measures loss in terms of the Kullback-Liebler divergence. One selects a model $M_{j^*}$ such that
\[
    j^* \in \arg\min_{1 \le j \le q} \aic_j.
\]
In this article, we will consider a generalisation of the AIC scores where the per parameter penalty is changed from $2$ to some value $\gamma \ge 0$. We define
\[
    \gaic{\gamma}_j := - 2 \ell_{M_j} + \gamma d_{M_j}.
\]
Thus $\aic_j = \gaic{2}_j$.

\subsection{The spiked Wigner model}
Recall that a $N \times N$ Wigner matrix $W$ is a random symmetric matrix whose diagonal and above-diagonal entries are independent, the above-diagonal entries $W_{ij}, i > j$, having a symmetric law $\mu_1$ with variance $1$ and the diagonal entries $W_{ii}$ having a potentially different symmetric law $\mu_2$ with finite variance. We are interested in the \emph{spiked Wigner model}:
\begin{equation}\label{eq:spiked-Wigner}
    X = A + \frac{\sigma}{\sqrt{N}}W,
\end{equation}
where $A$ is a rank-$k$ positive semi-definite matrix with spectral decomposition \begin{equation}
    A = \sum_{i = 1}^k \lambda_i u_i u_i^\top,
\end{equation}
all whose non-zero eigenvalues are $> \sigma$, the BBP threshold for this model. Here we want to estimate the unknown rank $k$. Under model $M_j$, we have $k = j$.

To derive $\gaic{\gamma}$, we need to know the likelihood, which would be different for different distributions of the $W_{ij}$'s. To get around this issue, we will consider normally distributed entries and resort to \emph{universality phenomena} in random matrix theory due to which, under appropriate assumptions, the asymptotic behaviour of the spectrum in certain aspects becomes insensitive to the distribution of the entries. In fact, we will work with matrices from the so-called Gaussian Orthogonal Ensemble (GOE). An $N \times N$ GOE random matrix $G$ is a random symmetric matrix whose upper diagonal entries are i.i.d. $\cN(0, \frac{1}{N})$ and diagonal entries are i.i.d. $\cN(0, \frac{2}{N})$.  Thus we will consider the \emph{spiked GOE model}:
\begin{equation}\label{eq:spiked-GOE}
    X = A + \sigma G,
\end{equation}
under which we will derive $\gaic{\gamma}$ for both the cases $\sigma$ known and $\sigma$ unknown. We will then use the resulting criteria as \emph{a proxy for the actual AIC} for the more general spiked Wigner model \eqref{eq:spiked-Wigner}. The density of $X$ (with respect to the Lebesgue measure on $\bR^{N(N + 1)/2}$) under model \eqref{eq:spiked-GOE} is
\begin{equation}\label{eq:density-X}
    C_N \sigma^{-\frac{N(N + 1)}{2}}e^{-\frac{N}{4 \sigma^2} \tr(X - A)^2},
\end{equation}
where $C_N$ is a normalising constant. The simple form of the density above is the main reason for working with the spiked GOE model instead of some other Gaussian Wigner model.

\begin{assumption}\label{assmp:q-lambda1-fixed}
    We will assume throughout that $q$, the number of candidate models, and the eigenvalue of the signal matrix $A$, are all fixed (i.e. they do not change with $N$).
\end{assumption}

\subsection{Random matrix theoretic results}
We now recall some relevant random matrix theoretic results on the spiked Wigner model. There is substantial body of literature surrounding this model. Of particular relevance to us are the works \cite{capitaine2009largest, benaych2011eigenvalues, benaych2011fluctuations}.

The following assumption on the spiked eigenvectors is needed for some results on the spiked Wigner model.
\begin{assumption}\label{assmp:evecs}
Under Model $M_j$, The spiked eigenvectors $u_i$'s form a uniformly random $j$-frame (i.e. $j$ mutually orthogonal unit vectors).
\end{assumption}
Note that the uniform distribution on the set of all $j$-frames is orthogonally invariant. Due to the orthogonal invariance of GOE matrices, Assumption~\ref{assmp:evecs} is redundant for the spiked GOE model.

It is well-known that bulk empirical spectral measure $\frac{1}{N}\sum_{i = 1}^N \delta_{\ell_i}$ converges weakly almost surely to the semi-circle law whose density is given by
\[
    \varrho_{\semicirc}(x; \sigma^2) = \frac{1}{2\pi \sigma^2} \sqrt{4\sigma^2 - x^2} \, \bI(|x| \le 2\sigma).
\]

To discuss the behaviour of the extreme eigenvalues, we need the function
\[
    \psi_{\sigma}(x) = x + \frac{\sigma^2}{x}.
\]
A plot of this function for $\sigma = 1$ is given in Figure~\ref{fig:graphs}-(a). Note that $\psi_{\sigma}$ achieves its minimum value of $2\sigma$ at $x = \sigma$, to the right of which it is strictly increasing. We first recall the first order behaviour of the edge eigenvalues.
\begin{proposition}\label{prop:swm-extr}
    For the spiked GOE model, under model $M_j$, we have the following:
\begin{enumerate}
    \item[(a)] For $1 \le i \le j$, $\ell_i \convas \psi_{\sigma}(\lambda_i)$.
    \item[(b)] For $j < i \le q$, we have $\ell_i \convas \psi_{\sigma}(\sigma) = 2\sigma$.
\end{enumerate}
The same holds under the spiked Wigner model if $\mu_1, \mu_2$ satisfy a Poincar\'{e} inequality (cf. Theorem~2.1 of \cite{capitaine2009largest}) or if Assumption~\ref{assmp:evecs} holds (cf. Theorem~2.1 of \cite{benaych2011eigenvalues}).
\end{proposition}

We also recall the following fluctuation result on the non-spiked extreme eigenvalues.

\begin{proposition}\label{prop:swm-fluc}
For the spiked GOE model, under model $M_j$, we have that for any fixed $i > j$,
\[
    N^{2/3}(\ell_i - 2\sigma) \xrightarrow{d} \tw_{i - j},
\]
where $\tw_{p}$ denotes the GOE Tracy-Widom distribution of order $p$. The same holds under the spiked Wigner model if $\mu_1$ and $\mu_2$ have sub-exponential tails and Assumption~\ref{assmp:evecs} holds (cf. Proposition~5.3 of \cite{benaych2011fluctuations}).
\end{proposition}

We actually need something weaker than Proposition~\ref{prop:swm-fluc} --- only that $N^{2/3}(\ell_i - 2\sigma)$ is a tight sequence.

% Figure environment removed

\subsection{Derivation of \texorpdfstring{$\boldsymbol{\gaic{\gamma}}$}{}}
We now derive exact expressions for the model scores $\gaic{\gamma}_j$ under the spiked GOE model \eqref{eq:spiked-GOE}. Given these scores, we estimate $k$ by minimising $\gaic{\gamma}_j$ over $j \in \{0, 1, \ldots, q\}$:
\begin{equation}
    \hat{k}_{\gamma} := \min_{j \in \{0, 1, \ldots, q\}} \gaic{\gamma}_j.
\end{equation}

\subsubsection{Known \texorpdfstring{$\sigma$}{}} Consider first the case of known $\sigma$. The log-likelihood is
\[
    \ell(A) = \log C_N - \frac{N(N + 1)}{4} \log \sigma^2 - \frac{N}{4 \sigma^2} \|X - A\|_F^2.
\]
Let $X = \sum_{i = 1}^N \ell_i v_i v_i^\top$ be the spectral decomposition of $X$. Then the maximum likelihood estimate (MLE) of $A$ is the best rank-$j$ positive semi-definite approximation to $X$ in Frobenius norm, which is given by
\begin{equation}\label{eq:MLE-A}
    \hat{A}_j = \sum_{i = 1}^j \max\{\ell_i, 0\} v_i v_i^\top.
\end{equation}
This is a well-known result (see, for example, Lemma 19 of \cite{clarkson2017low}). Since we will select from a bounded number of candidate models (i.e. $q = O(1)$), for all $0 \le j \le q$,
\[
    \liminf_{N \to \infty} \ell_j \ge 2\sigma, \text{ a.s.}
\]
It follows that, almost surely, for $N$ large enough, we have
\begin{equation}\label{eq:MLE-A-simp}
    \hat{A}_j = \sum_{i = 1}^j \ell_i v_i v_i^\top.
\end{equation}
In the case $j = 0$, we take $\hat{A}_0 = 0$, the zero matrix. Hence
\begin{align*}
    \ell(\hat{A}_j) &= \log C_N - \frac{N(N + 1)}{4} \log \sigma^2 - \frac{N}{4\sigma^2} \|X - \hat{A}_j\|_F^2 \\
                    &= \log C_N - \frac{N(N + 1)}{4} \log \sigma^2 - \frac{N}{4\sigma^2} \sum_{i > j} \ell_i^2.
\end{align*}
Therefore, in the case of known $\sigma$, we have
\begin{align} \nonumber
    \gaic{\gamma}_j &= -2 \ell(\hat{A}_j) + \gamma \bigg(N j - \frac{j(j - 1)}{2}\bigg) \\
                    &= - 2\log C_N + \frac{N(N + 1)}{2} \log \sigma^2 + \frac{N}{2 \sigma^2} \sum_{i > j} \ell_i^2 + \gamma \bigg(N j - \frac{j(j - 1)}{2}\bigg). \label{eq:gaic-known-sigma}
\end{align}

\subsubsection{Unknown \texorpdfstring{$\sigma$}{}} The log-likelihood is
\[
    \ell(A, \sigma^2) = \log C_N - \frac{N(N + 1)}{4} \log \sigma^2 - \frac{N}{4 \sigma^2} \|X - A\|_F^2.
\]
As in the case of known $\sigma$, the maximum likelihood estimate (MLE) of $A$ under model $M_j$ is given by \eqref{eq:MLE-A-simp} (almost surely, for large enough $N$). The MLE of $\sigma^2$ under model $M_j$ is given by
\[
    \hat{\sigma^2_j} = \frac{1}{N + 1} \|X - \hat{A}_j\|_F^2 = \frac{1}{N + 1}\sum_{i > j} \ell_i^2.
\]
Thus
\begin{align*}
    \ell(\hat{A}_j, \hat{\sigma^2_j}) &= \log C_N - \frac{N(N + 1)}{4} \log \hat{\sigma^2_j} - \frac{N}{4\hat{\sigma^2_j}} \|X - \hat{A}_j\|_F^2 \\
                                    &= \log C_N - \frac{N(N + 1)}{4} \log \hat{\sigma^2_j} - \frac{N}{4\hat{\sigma^2_j}} \sum_{i > j} \ell_i^2,
\end{align*}
and
\begin{align}\nonumber
    \gaic{\gamma}_j &= -2 \ell(\hat{A}, \hat{\sigma^2_j}) + \gamma \bigg(1 + N j - \frac{j(j - 1)}{2}\bigg) \\
                    &= - 2\log C_N + \frac{N(N + 1)}{2} \log \hat{\sigma^2_j} + \frac{N}{2 \hat{\sigma^2_j}} \sum_{i > j} \ell_i^2 + \gamma \bigg(1 + N j - \frac{j(j - 1)}{2}\bigg). \label{eq:gaic-uknown-sigma}
\end{align}

\subsection{Results on \texorpdfstring{$\boldsymbol{\hat{k}_{\gamma}}$}{}}
We are now ready to state our main results on the selection properties of $\hat{k}_{\gamma}$. We will distinguish between two notions of consistency.
\begin{definition}[Consistency]\label{def:consistency}
    An estimator $\hat{k}$ of $k$ is called \emph{strongly consistent} if $\hat{k} \convas k$. It is called \emph{weakly consistent} if $\bP(\hat{k} = k) \to 1$.
\end{definition}
\begin{theorem}\label{thm:gamma-aic}
    Assume that the conclusions of Proposition~\ref{prop:swm-extr} hold. Regardless of whether $\sigma$ is known or unknown, we have the following under Assumption~\ref{assmp:q-lambda1-fixed}:
    \begin{enumerate}
    \item [(a)] If $\gamma \le 2$, then almost surely, $\liminf_{N \to \infty} \hat{k}_{\gamma} \ge k$.
    
    \item [(b)] If $\gamma > 2$, then almost surely, $\limsup_{N \to \infty} \hat{k}_{\gamma} \le k$.

    \item [(c)] Further, if $\lambda_k > \lambda_{\gamma} := \psi_{\sigma}^{-1}(\sqrt{2\gamma}\sigma)$, then for $\gamma > 2$, $\liminf_{N \to \infty} \hat{k}_{\gamma} \ge k$.
    \end{enumerate}
    As a consequence, if $\lambda_k > \lambda_{\gamma}$, then $\hat{k}_{\gamma}$ is strongly consistent for $k$.
\end{theorem}

In Figure~\ref{fig:graphs}-(b), we plot the threshold $\lambda_{\gamma}$ as a function of $\gamma$ for $\sigma = 1$. This is strictly bigger than the BBP threshold $\psi_{\sigma}^{-1}(2 \sigma) = \sigma$. Thus, as far as consistent selection is concerned, choosing a penalty factor $\gamma > 2$ is suboptimal.

Note also that Theorem~\ref{thm:gamma-aic} does not say anything about the consistency of AIC (i.e. the case $\gamma = 2$). In fact, a look at its proof (given in Section~\ref{sec:proofs}) reveals that the scores $\aic_j$, $k \le j \le q$ become asymptotically of the same order. As the true model $M_k$ is the least complex among the models $M_j, k \le j \le q$, there is hope that one may be able to identify the true model by selecting the least complex model close to the minimiser of $\aic_j$. Under additional assumptions, this is indeed possible (see Section~\ref{sec:soft-aic} below). 


Since $\aic$ is at the borderline of the dichotomy revealed in parts (a) and (b) of Theorem~\ref{thm:gamma-aic}, a natural question is if we can use a penalty factor $\gamma_N = 2 + \delta_N$, where $\delta_N$ shrinks to $0$ at an appropriate rate and achieve consistency. Using the $N^{-2/3}$ fluctuations of the non-spiked extreme eigenvalues (cf. Proposition~\ref{prop:swm-fluc}), we can establish the following weak consistency result.
 
\begin{theorem}\label{thm:almost-aic-weak-consistency}
    Assume that the conclusions of Propositions~\ref{prop:swm-extr} and \ref{prop:swm-fluc} hold. Let $\delta_N$ be a sequence such that $\delta_N \to 0$ and $\delta_N \gg N^{-2/3}$. Then, under Assumption~\ref{assmp:q-lambda1-fixed}, $\hat{k}_{2 + \delta_N}$ is weakly consistent for $k$, i.e. $\bP(\hat{k}_{2 + \delta_N} = k) \to 1$. This is true regardless of whether $\sigma$ is known or unknown.
\end{theorem}

\begin{remark}
    A natural question left unanswered here is if there is a choice of $\delta_N$ for which $\hat{k}_{2 + \delta_N}$ is strongly consistent.
\end{remark}

\subsection{A soft-minimisation approach}\label{sec:soft-aic}
Although from Theorem~\ref{thm:gamma-aic}, we cannot guarantee strong consistency of AIC, as discussed earlier, there is some hope of recovering the true model by selecting the least complex model close to the minimiser of AIC. We now make this precise.

For a threshold $\hat{\xi} > 0$, consider the following estimator of $k$ which we dub $\saic$:
\begin{equation}\label{eq:soft-aic}
    \hat{k}_{2, \,\soft} := \min\bigg\{j : |\aic_j - \min_{0 \le j' \le q} \aic_{j'}| < \frac{\hat{\xi}}{3}\bigg\}.
\end{equation}
The threshold $\hat{\xi}$ has to be chosen carefully so that in the minimisation above models $M_j$ with $j < k$ are automatically discarded. To that end, define for $j \le k$,
\[
    \xi_j := \frac{1}{2\sigma^2}(\psi_{\sigma}(\lambda_j) - 4\sigma^2).
\]
In effect, our threshold $\hat{\xi}$ should be smaller than $\xi_k$ in an appropriate sense.

\begin{theorem}\label{thm:soft-aic}
     Assume that the conclusions of Proposition~\ref{prop:swm-extr} hold. Suppose also that Assumption~\ref{assmp:q-lambda1-fixed} holds and we can construct a threshold $\hat{\xi}$ such that almost surely,
    \begin{equation}\label{eq:soft-aic-thres-prop}
        0 < \liminf_{N \to \infty} \hat{\xi} \le \limsup_{N \to \infty} \hat{\xi} \le \xi_k.
    \end{equation}
    Then $\hat{k}_{2, \,\soft} \convas k$.
\end{theorem}

\subsubsection{Construction of a suitable threshold \texorpdfstring{$\hat{\xi}$}{}}
How do we construct a threshold $\hat{\xi}$ such that \eqref{eq:soft-aic-thres-prop} holds? For a strongly consistent estimator $\hat{\sigma^2}$ of $\sigma^2$, set
\[
    \hat{\xi}_j := \frac{1}{2\hat{\sigma^2}}(\ell_j^2 - 4\hat{\sigma^2}) \convas \begin{cases}
        \xi_j & \text{ if } j \le k, \\
        0 & \text{ if } k < j \le q.
\end{cases}
\]
Since $\lambda_1$ is bounded, then so is $\frac{\xi_1}{\xi_k}$. Assume that we \emph{know} an a priori upper bound $B$ on $\frac{\xi_1}{\xi_k}$, i.e. $\xi_1 \le B \xi_k$. For instance, in the equal-spikes case (i.e. $\lambda_1 = \cdots = \lambda_k$), we can take $B = 1$. Then, since $q$ is bounded, we may take
\begin{equation}\label{eq:xi-hat}
    \hat{\xi} = \frac{1}{qB} \sum_{j = 1}^q \hat{\xi}_j \xrightarrow{a.s.} \xi:= \frac{1}{qB} \sum_{j = 1}^k \xi_j.
\end{equation}
Clearly,
\[
     0 < \xi \le \frac{kB\xi_k}{qB} = \frac{k}{q} \xi_k \le \xi_k,
\]
so that \eqref{eq:soft-aic-thres-prop} holds for $\hat{\xi}$.

\subsection{The scree-plot estimator}
In spiked models such as PCA, one typically first looks at a plot of the eigenvalues for a kink therein. Such plots are commonly referred to as scree plots. In the context of our spiked model, one can construct such an estimator as follows (we will refer to this as the scree-plot estimator):
\[
    \hat{k}_{\scree} = \sup\big\{0 \le j \le q : \frac{\ell_j}{2\hat{\sigma}} > 1\big\},
\]
where $\hat{\sigma}$ is a strongly consistent estimate of $\sigma$. This will serve as a benchmark estimator.

\begin{proposition}\label{prop:scree}
    Assume that the conclusions of Proposition~\ref{prop:swm-extr} holds. Then, under Assumption~\ref{assmp:q-lambda1-fixed}, $\hat{k}_{\scree}$ is strongly consistent for $k$.
\end{proposition}

\subsection{Estimating \texorpdfstring{$\boldsymbol{\sigma}$}{}.}
Instead of using the $\gaic{\gamma}$ scores \eqref{eq:gaic-uknown-sigma} for unknown $\sigma$, one may also use the scores \eqref{eq:gaic-known-sigma} for known $\sigma$ by plugging in some strongly consistent estimate of $\sigma$. We also need such an estimator for the proposed construction \eqref{eq:xi-hat} of the threshold $\hat{\xi}$ in $\saic$ and for the scree-plot estimator $\hat{k}_{\scree}$. Technically, we could use
\[
    \hat{\sigma^2_0} = \frac{1}{N + 1}\sum_{j = 1}^N \ell_j^2 = \frac{1}{N + 1} \|X\|_F^2,
\]
which converges almost surely to $\sigma^2$. However, if some of the top spiked eigenvalues are large, then for small to moderate $N$, $\hat{\sigma^2_0}$ tends to overestimate $\sigma^2$, leading to poor performance (see Section~\ref{sec:simu} for empirical demonstrations of this phenomenon). Instead, we can throw away some of the extreme eigenvalues and adjust the estimate accordingly. For $\alpha \in (0, 1)$, let $q_{\alpha}$ denote the top $\alpha$-th quantile of the standard semi-circle law, i.e.
\[
    \alpha = \int_{q_{\alpha}}^2 \varrho_{\semicirc}(x; 1) \, dx.
\]
Then we may take the following trimmed estimator of $\sigma^{2}$:
\[
    \hat{\sigma}^2_{\alpha} := \frac{\frac{1}{N}\sum_{j : \ell_{\lfloor \alpha N \rfloor} \le \ell_j \le \ell_{\lfloor (1 - \alpha) N \rfloor}} \ell_j^2} {\int_{-q_{\alpha}}^{q_{\alpha}} x^2 \varrho_{\semicirc}(x; 1) \, dx}.
\]
This also converges almost surely to $\sigma^2$.

%%%%%%%%%%%%%%%%%%%%%%%%%%%%%%%%%%%%%%%%%%%%%%%%%%%%%%%%%%%%%%%%%%%%%%%%%%%%%%%%
\section{Empirical results}\label{sec:simu}
%%%%%%%%%%%%%%%%%%%%%%%%%%%%%%%%%%%%%%%%%%%%%%%%%%%%%%%%%%%%%%%%%%%%%%%%%%%%%%%%
In this section\footnote{The results of this section can be replicated using the \textbf{R} code available at \protect{\url{https://soumendu041.gitlab.io/swm-aic}}.}, we report simulation experiments comparing $\aic$, $\gaic{2 + \delta_N}$, $\gaic{\gamma}$, $\saic$ and the scree-plot estimator. We report two performance metrics for each estimator $\hat{k}$ based on Monte Carlo replications: (i) average dimensionality of the selected model along with the corresponding standard error estimates and an estimate of the \emph{probability of correct selection (PCS)}, i.e. $\bP(\hat{k} = k)$.

For the AIC-type estimators, we use the scores \eqref{eq:gaic-known-sigma} for known $\sigma$ together with various estimates of $\sigma$, and also the scores \eqref{eq:gaic-uknown-sigma} for unknown $\sigma$. For the scree-plot estimator, we use estimates of $\sigma$. Below S-1 refers to the situation, where we use the \emph{oracle} value of $\sigma$ in the estimators. S-2 (resp. S-3) refers to the situation where we use $\hat{\sigma^2_0}$ (resp. $\hat{\sigma^2_{\alpha}}$) as an estimate of $\sigma$. Finally, S-4 refers to the situation (applicable to the AIC-type estimators only) where the scores corresponding to unknown $\sigma$ are used.

In the experiments below, we take $\sigma^2 = 1$, $N = 1000$, $\gamma = 2.15$. We use $B = 5$ for $\saic$, $\delta_N = \frac{0.1}{\sqrt{N}}$ and take $\alpha = 0.1$ in $\hat{\sigma^2_{\alpha}}$. We search over $q = 20$ candidate models. The performance metrics are all based on $100$ Monte Carlo replications.

We consider three different noise profiles: (i) GOE; (ii) Wigner with Rademacher entries (i.e. random signs); and (iii) Schur-Hadamard product of independent symmetric Toeplitz and Hankel random matrices each with i.i.d. $\cN(0, 1)$ entries, with the resulting matrix being scaled by $\sqrt{N}$. In Tables~\ref{table:1} and \ref{table:2}, these profiles are denoted by ``GOE'', ``Rad.'' and ``$T\odot H$'', respectively. Note that profile (iii) has dependent (but uncorrelated) entries. The empirical eigenvalue distribution of this profile was studied in \cite{bose2014bulk, mukherjee2022convergence} where it was shown to converge to the semi-circle law. We expect that the results on the behaviour of edge eigenvalues (stated under the spiked Wigner model) would also hold for this profile (as an instantiation of universality). Therefore it is natural to anticipate that the selection properties of the various estimators considered would be valid under this noise profile as well. This is indeed confirmed in our experiments.

In our first experiment reported in Table~\ref{table:1}, we have $k = 4$ spiked eigenvalues, with $\lambda_4 = 1.1$, just above the BBP threshold $1$. Since $\lambda_{\gamma} = 1.31$, $\gaic{\gamma}$ always underestimates $k$. $\aic$, $\gaic{2 + \delta_N}$, and $\saic$ all perform much better than the simple scree-plot estimator. Further, all of these three AIC-type estimators perform the best in S-4, where the scores for unknown $\sigma^2$ are used. Also, note that using $\hat{\sigma^2_0}$ as an estimator of $\sigma$ results in disastrous performance, likely due to the effect of the relatively large top eigenvalue $\lambda_1 = 5$. The trimmed estimator $\hat{\sigma^2_{\alpha}}$ fares much better in comparison. Somewhat surprisingly, under the $T\odot H$ noise profile, all the estimators suffer from degraded performance in S-1.

\begin{table}[!t]
    \centering
    \setlength{\tabcolsep}{0.3em}
    \caption{$k = 4$, $(\lambda_1, \lambda_2, \lambda_3, \lambda_4)^\top = (5, 1.5, 1.2, 1.1)^\top$, $\sigma^2 = 1$. $N = 1000$, $\delta_N = \frac{0.1}{\sqrt{N}}$, $\lambda_{2 + \delta_N} = 1.04$, $\gamma = 2.15$, $\lambda_{\gamma} = 1.31$. S-1: oracle $\sigma^2$; S-2: $\hat{\sigma^2_0}$; S-3: $\hat{\sigma^2_{\alpha}}$ with $\alpha = 0.1$; S-4: unknown $\sigma^2$.}
    \label{table:1}
    \tiny
    \begin{tabular}{l|l*{4}{c}|*{4}{c}|*{4}{c}|*{4}{c}|*{3}{c}}
        \toprule
        \multicolumn{2}{c}{} & \multicolumn{4}{c}{$\boldsymbol{\aic}$} & \multicolumn{4}{c}{$\boldsymbol{\gaic{2 + \delta_N}}$} & \multicolumn{4}{c}{$\boldsymbol{\saic}$} & \multicolumn{4}{c}{$\boldsymbol{\gaic{\gamma}}$} & \multicolumn{3}{c}{$\boldsymbol{\scree}$} \\
        \midrule
        \multicolumn{2}{c}{} & S-1 & S-2 & S-3 & S-4 & S-1 & S-2 & S-3 & S-4 & S-1 & S-2 & S-3 & S-4 & S-1 & S-2 & S-3 & S-4 & S-1 & S-2 & S-3 \\
        \midrule
        \multirow{3}{*}{\rotatebox{90}{\bfseries{GOE}}}
        & \bfseries{mean} & 3.44 & 2.68 & 3.23 & 3.78 & 3.40 & 2.67 & 3.21 & 3.67 & 3.24 & 2.61 & 3.07 & 3.45 & 2.11 & 1.99 & 2.04 & 2.13 & 3.36 & 2.66 & 3.18 \\
        & \bfseries{sd}   & 0.54 & 0.57 & 0.51 & 0.56 & 0.53 & 0.57 & 0.54 & 0.60 & 0.55 & 0.53 & 0.50 & 0.58 & 0.31 & 0.17 & 0.20 & 0.34 & 0.52 & 0.55 & 0.54 \\
        & \bfseries{PCS}  & 0.46 & 0.05 & 0.27 & 0.64 & 0.42 & 0.05 & 0.27 & 0.56 & 0.30 & 0.02 & 0.16 & 0.49 & 0.00 & 0.00 & 0.00 & 0.00 & 0.38 & 0.04 & 0.25 \\
        \midrule
        \multirow{3}{*}{\rotatebox{90}{\bfseries{Rad.}}}
        & \bfseries{mean} & 3.36 & 2.62 & 3.12 & 3.68 & 3.29 & 2.57 & 3.07 & 3.65 & 3.12 & 2.41 & 2.90 & 3.47 & 2.01 & 2    & 2.01 & 2.04 & 3.24 & 2.54 & 3.02 \\
        & \bfseries{sd}   & 0.50 & 0.49 & 0.46 & 0.63 & 0.52 & 0.50 & 0.43 & 0.61 & 0.43 & 0.49 & 0.46 & 0.58 & 0.10 & 0    & 0.10 & 0.20 & 0.49 & 0.50 & 0.43 \\
        & \bfseries{PCS}  & 0.37 & 0.00 & 0.17 & 0.53 & 0.32 & 0.00 & 0.13 & 0.51 & 0.16 & 0.00 & 0.06 & 0.42 & 0.00 & 0    & 0.00 & 0.00 & 0.27 & 0.00 & 0.10 \\
        \midrule
        \multirow{3}{*}{\rotatebox{90}{\boldsymbol{$T\odot H$}}}
        & \bfseries{mean} & 4.48 & 2.72 & 3.16 & 3.62 & 4.42 & 2.69 & 3.13 & 3.57 & 4.24 & 2.63 & 3.00 & 3.43 & 2.53 & 2.00 & 2.11 & 2.20 & 4.27 & 2.69 & 3.10 \\
        & \bfseries{sd}   & 3.12 & 0.59 & 0.63 & 0.74 & 3.05 & 0.58 & 0.63 & 0.70 & 2.98 & 0.58 & 0.64 & 0.64 & 1.22 & 0.28 & 0.37 & 0.47 & 2.78 & 0.58 & 0.63 \\
        & \bfseries{PCS}  & 0.15 & 0.07 & 0.29 & 0.53 & 0.14 & 0.06 & 0.27 & 0.54 & 0.14 & 0.05 & 0.20 & 0.48 & 0.06 & 0.00 & 0.00 & 0.01 & 0.13 & 0.06 & 0.25 \\
        \bottomrule
    \end{tabular}
\end{table}

Our second experiment (see Table~\ref{table:2}) is in a relatively easier setting: we have $k = 4$ with $\lambda_4 = 1.5$. The scree-plot estimator catches up with the AIC-type estimators. As $\lambda_4 > \lambda_{\gamma}$, $\gaic{\gamma}$ also performs quite well. Notably, both $\aic$ and $\gaic{2 + \delta_N}$ perform much worse in S-4 than S-1 or S-3, in contrast with the results in the first experiment. We again observe degraded performance for all the estimators in S-1 under the $T\odot H$ noise profile.

\begin{table}[!t]
    \centering
    \setlength{\tabcolsep}{0.3em}
    \caption{$k = 4$, $(\lambda_1, \lambda_2, \lambda_3, \lambda_4)^\top = (10, 3, 1.5, 1.5)^\top$, $\sigma^2 = 1$. $N = 1000$, $\delta_N = \frac{0.1}{\sqrt{N}}$, $\lambda_{2 + \delta_N} = 1.04$, $\gamma = 2.15$, $\lambda_{\gamma} = 1.31$. S-1: oracle $\sigma^2$; S-2: $\hat{\sigma^2_0}$; S-3: $\hat{\sigma^2_{\alpha}}$ with $\alpha = 0.1$; S-4: unknown $\sigma^2$.}
    \label{table:2}
    \tiny
    \begin{tabular}{l|l*{4}{c}|*{4}{c}|*{4}{c}|*{4}{c}|*{3}{c}}
        \toprule
        \multicolumn{2}{c}{} & \multicolumn{4}{c}{$\boldsymbol{\aic}$} & \multicolumn{4}{c}{$\boldsymbol{\gaic{2 + \delta_N}}$} & \multicolumn{4}{c}{$\boldsymbol{\saic}$} & \multicolumn{4}{c}{$\boldsymbol{\gaic{\gamma}}$} & \multicolumn{3}{c}{$\boldsymbol{\scree}$} \\
        \midrule
        \multicolumn{2}{c}{} & S-1 & S-2 & S-3 & S-4 & S-1 & S-2 & S-3 & S-4 & S-1 & S-2 & S-3 & S-4 & S-1 & S-2 & S-3 & S-4 & S-1 & S-2 & S-3 \\
        \midrule
        \multirow{3}{*}{\rotatebox{90}{\bfseries{GOE}}}
        & \bfseries{mean} & 4.06 & 3.79 & 4.03 & 4.32 & 4.06 & 3.79 & 4.03 & 4.27 & 4    & 3.36 & 4 & 4.01 & 3.98 & 2.68 & 3.98 & 4    & 4.04 & 3.79 & 4.02 \\
        & \bfseries{sd}   & 0.24 & 0.41 & 0.17 & 0.51 & 0.24 & 0.41 & 0.17 & 0.49 & 0    & 0.56 & 0 & 0.10 & 0.14 & 0.55 & 0.14 & 0    & 0.20 & 0.41 & 0.14 \\
        & \bfseries{PCS}  & 0.94 & 0.79 & 0.97 & 0.70 & 0.94 & 0.79 & 0.97 & 0.75 & 1    & 0.40 & 1 & 0.99 & 0.98 & 0.04 & 0.98 & 1    & 0.96 & 0.79 & 0.98 \\
        \midrule
        \multirow{3}{*}{\rotatebox{90}{\bfseries{Rad.}}}
        & \bfseries{mean} & 4.03 & 3.94 & 4    & 4.20 & 4.03 & 3.94 & 4    & 4.16 & 4    & 3.30 & 4 & 4    & 4    & 2.53 & 4    & 4    & 4.01 & 3.94 & 4    \\
        & \bfseries{sd}   & 0.17 & 0.24 & 0    & 0.43 & 0.17 & 0.24 & 0    & 0.37 & 0    & 0.46 & 0 & 0    & 0    & 0.50 & 0    & 0    & 0.10 & 0.24 & 0    \\
        & \bfseries{PCS}  & 0.97 & 0.94 & 1    & 0.81 & 0.97 & 0.94 & 1    & 0.84 & 1    & 0.30 & 1 & 1    & 1    & 0.00 & 1    & 1    & 0.99 & 0.94 & 1    \\
        \midrule
        \multirow{3}{*}{\rotatebox{90}{\boldsymbol{$T\odot H$}}}
        & \bfseries{mean} & 5.53 & 3.75 & 4    & 4.37 & 5.47 & 3.75 & 4    & 4.29 & 4.92 & 3.24 & 4 & 4    & 4.18 & 2.71 & 3.92 & 3.96 & 5.28 & 3.73 & 4    \\
        & \bfseries{sd}   & 2.90 & 0.52 & 0    & 0.54 & 2.82 & 0.52 & 0    & 0.50 & 2.25 & 0.73 & 0 & 0    & 1.00 & 0.69 & 0.31 & 0.24 & 2.52 & 0.53 & 0    \\
        & \bfseries{PCS}  & 0.61 & 0.79 & 1    & 0.66 & 0.62 & 0.79 & 1    & 0.73 & 0.74 & 0.41 & 1 & 1    & 0.84 & 0.13 & 0.93 & 0.97 & 0.64 & 0.77 & 1    \\
        \bottomrule
    \end{tabular}
\end{table}

\subsection{Comparing \texorpdfstring{$\boldsymbol{\aic}$}{} and \texorpdfstring{$\boldsymbol{\gaic{2 + \delta_N}}$}{}}
Finally, in Table~\ref{table:3}, we compare the performance of $\aic$ and $\gaic{2 + \delta_N}$ as $N$ varies. We use the same set-up as in our second experiment (except that we use a slightly higher value of $\delta_N = \frac{0.5}{\sqrt{N}}$) and report only the results for oracle $\sigma$, i.e. S-1. The results favour $\gaic{2 + \delta_N}$ in accordance with our theoretical results.

\begin{table}[!t]
    \centering
    \setlength{\tabcolsep}{0.3em}
    \caption{$k = 4$, $(\lambda_1, \lambda_2, \lambda_3, \lambda_4)^\top = (10, 3, 1.5, 1.5)^\top$, $\sigma^2 = 1$. $\delta_N = \frac{0.5}{\sqrt{N}}$.}
    \label{table:3}
    \tiny
\begin{tabular}{l|l*{5}{c}}
    \toprule
    \multicolumn{1}{c}{} & $N$ & 1000 & 2000 & 3000 & 4000 & 5000 \\
    \midrule
    \multirow{3}{*}{$\boldsymbol{\aic}$}
    & \bfseries{mean} & 4.06 & 4.12 & 4.05 & 4.07 & 4.15 \\
    & \bfseries{sd}   & 0.24 & 0.33 & 0.22 & 0.26 & 0.39 \\
    & \bfseries{PCS}  & 0.94 & 0.88 & 0.95 & 0.93 & 0.86 \\
    \midrule
    \multirow{3}{*}{$\boldsymbol{\gaic{2 + \delta_N}}$}
    & \bfseries{mean} & 4.03 & 4.03 & 4.01 & 4.04 & 4.02 \\
    & \bfseries{sd}   & 0.17 & 0.17 & 0.10 & 0.20 & 0.14 \\
    & \bfseries{PCS}  & 0.97 & 0.97 & 0.99 & 0.96 & 0.98 \\
    \bottomrule
\end{tabular}
\end{table}

%%%%%%%%%%%%%%%%%%%%%%%%%%%%%%%%%%%%%%%%%%%%%%%%%%%%%%%%%%%%%%%%%%%%%%%%%%%%%%%%
\section{Proofs}\label{sec:proofs}
%%%%%%%%%%%%%%%%%%%%%%%%%%%%%%%%%%%%%%%%%%%%%%%%%%%%%%%%%%%%%%%%%%%%%%%%%%%%%%%%
Note that as long as $j = o(N)$,
\begin{equation}\label{eq:sigma2hat-conv-sigma2}
    \hat{\sigma^2_j} \convas \int x^2 \varrho_{\mathrm{sc}}(x; \sigma^2) \, dx = \sigma^2.
\end{equation}
Because of this, and the fact that we have only one extra parameter in the case of unknown $\sigma$, the proofs of Theorems~\ref{thm:gamma-aic}, \ref{thm:almost-aic-weak-consistency} and \ref{thm:soft-aic} will essentially be the same regardless of whether $\sigma$ is known or unknown. For simplicity, we will only write down the details for the case of known $\sigma$.

\begin{proof}[Proof of Theorem~\ref{thm:gamma-aic}]
    For $j < k$, we have, using Proposition~\ref{prop:swm-extr}-(a), that
\begin{align*}
    \frac{1}{N}(\gaic{\gamma}_j - \gaic{\gamma}_k) &=  \frac{1}{2\sigma^2} \sum_{i = j + 1}^k \ell_i^2 - \theta (k - j) \bigg(1 - \frac{k + j - 1}{2N} \bigg) \\
              &\convas \frac{1}{2\sigma^2} \sum_{i = j + 1}^k (\psi_{\sigma}(\lambda_i))^2 - 2 (k - j) \\
              &\ge \frac{1}{2\sigma^2} (k - j) (\psi_{\sigma}(\lambda_k))^2 -  \gamma (k - j) \\
              &= \frac{(k - j)}{2\sigma^2} [(\psi_{\sigma}(\lambda_k))^2 -  2\gamma \sigma^2].
\end{align*}
Note that for $\gamma \le 2$,
\begin{align*}
    (\psi_{\sigma}(\lambda_k))^2 - 2\gamma\sigma^2 &> 4\sigma^2 - 2\gamma \sigma^2 \\
                                                   &= 2\sigma^2 (2 - \gamma) \ge 0,
\end{align*}
where we have used the fact that $\lambda_k > \sigma$ implies that $\psi_\sigma(\lambda_k) > 2 \sigma$. It follows that
\[
    \liminf_{N \to \infty}\hat{k}_{\gamma} \ge k \text{ a.s.}
\]
for any $\gamma \le 2$. This establishes (a).

On the other hand, for $j > k$, Proposition~\ref{prop:swm-extr}-(b) gives that
\begin{align*}
    \frac{1}{N}(\gaic{\gamma}_j - \gaic{\gamma}_k) &= -\frac{1}{2\sigma^2} \sum_{i = k + 1}^j \ell_i^2 + \gamma (j - k) \bigg(1 - \frac{k + j - 1}{2N} \bigg) \\
              &\convas -\frac{1}{2\sigma^2} \sum_{i = k + 1}^j 4 \sigma^2 + \gamma (j - k) \\
              &= (\gamma - 2) (j - k) > 0
\end{align*}
if $\gamma > 2$. It follows that
\[
    \limsup_{N \to \infty}\hat{k}_{\gamma} \le k \text{ a.s.}
\]
for any $\gamma > 2$. This establishes (b).

Finally, for $\gamma > 2$, $\hat{k}_{\gamma}$ will be strongly consistent provided
\[
    \psi_{\sigma}(\lambda_k) > \sqrt{2\gamma} \sigma > 2\sigma
\]
which is equivalent to
\[
    \lambda_k > \psi_{\sigma}^{-1}(\sqrt{2\gamma}\sigma) > \psi_{\sigma}^{-1}(2\sigma) = \sigma.
\]
This establishes (c).
\end{proof}

\begin{proof}[Proof of Theorem~\ref{thm:almost-aic-weak-consistency}]
    By Proposition~\ref{prop:swm-fluc}, $\tilde{\ell}_{j} = N^{2/3}(\ell_{j} - 2\sigma) = O_P(1)$ for $k < j \le q$. Therefore, for $k < j \le q$,
\begin{align*}
    \frac{1}{N}(\gaic{\gamma}_j - \gaic{\gamma}_k) &= -\frac{1}{2\sigma^2} \sum_{i = k + 1}^j \ell_i^2 + \gamma_N (j - k) \bigg(1 - \frac{k + j - 1}{2N} \bigg) \\
                                                   &= -\frac{1}{2\sigma^2} \sum_{i = k + 1}^j (2\sigma + N^{-2/3}\tilde{\ell_i})^2 + (2 + \delta_N) (j - k) \bigg(1 - \frac{k + j - 1}{2N} \bigg) \\
                                                   &= \delta_N (j - k) - \frac{1}{\sigma} \sum_{i = k + 1}^j N^{-2/3} \tilde{\ell}_i + O_P\bigg(\frac{1}{N}\bigg) \\
                                                   &\ge \delta_N (j - k) - \frac{1}{\sigma} (j - k) N^{-2/3} \tilde{\ell}_{k + 1} + O_P\bigg(\frac{1}{N}\bigg) \\
                                                   &= \delta_N(j - k) (1 + o_P(1)),
\end{align*}
provided $\delta_N \gg N^{-2/3}$. Thus $\bP(\gaic{\gamma}_j - \gaic{\gamma}_k \ge \epsilon N\delta_N(j - k)) \rightarrow 1$ for any $\epsilon \in (0, 1)$. It follows that $\bP(\hat{k}_\gamma \le k) \rightarrow 1$. Combining this with Theorem~\ref{thm:gamma-aic}-(a), we get the desired weak-consistency result.
\end{proof}

\begin{proof}[Proof of Theorem~\ref{thm:soft-aic}]
    Let $\xi_0 := \liminf_{N \to \infty} \hat{\xi} > 0$. There is a set $E$ with $\bP(E) = 0$, such that for all $\omega \in E^c$, we have the following:
\begin{enumerate}
    \item[(i)]  For $j < k$,
        \[
            \frac{1}{N}(\aic_j - \aic_k) \to \sum_{i = j + 1}^k \frac{1}{2\sigma^2}[\psi_{\sigma}(\lambda_i)^2 - 4\sigma^2] \ge (k - j) \xi_k \ge \xi_k.
        \]
        \item[(ii)] For $k < j \le q$,
        \[
            \frac{1}{N}(\aic_j - \aic_k) \to 0.
        \]
    \item[(iii)] For all sufficiently large $N$ (depending possibly on $\omega$), $\frac{\xi_0}{2} < \hat{\xi} < \frac{6}{5}\xi_k$.
\end{enumerate}
Thus, for any $\omega \in E^c$, we can choose $N_0(\omega)$ large enough such that for all $N \ge N_0(\omega)$, the following hold:
\begin{enumerate}
    \item[(a)] For all $j < k$, $\frac{1}{N}(\aic_j - \aic_k) \ge \frac{4\xi_k}{5} > \frac{2\hat{\xi}}{3}$.

    \item[(b)] For all $k < j \le q$, $\frac{1}{N}(\aic_j - \aic_k) < \frac{\xi_0}{6} < \frac{\hat{\xi}}{3}$.
\end{enumerate}
Together, (a) and (b) imply that
\[
    \frac{1}{N}|\aic_k - \min_{0 \le j' \le q} \aic_{j'}| < \frac{\hat{\xi}}{3},
\]
and for $j < k$,
\[
    \frac{1}{N}|\aic_j - \min_{0 \le j' \le q} \aic_{j'}| > \frac{1}{N}|\aic_j - \aic_k| - \frac{1}{N}|\aic_k - \min_{0 \le j' \le q} \aic_{j'}| > \frac{2\hat{\xi}}{3} - \frac{\hat{\xi}}{3} = \frac{\hat{\xi}}{3}.
\]
 Therefore $\hat{k}_{2, \,\soft} = k$ for all $N \ge N_0(\omega)$. This completes the proof.
\end{proof}

\begin{proof}[Proof of Proposition~\ref{prop:scree}]
    This is an immediate consequence of the following facts:
    \begin{enumerate}
        \item[(a)] For $k < j \le q$, $\limsup_{N \to \infty}\frac{\ell_j}{2\hat{\sigma}} \le 1$ almost surely.
        \item[(b)] For $j \le k$, we have $\ell_j \convas \psi_{\sigma}(\lambda_j) > 2 \sigma$ so that $\lim_{N \to \infty} \frac{\ell_j}{2\hat{\sigma}} > 1$. \qedhere
    \end{enumerate}
\end{proof}

\section*{Acknowledgements}
The research of the author is partially supported by an INSPIRE research grant (DST/INSPIRE/04/2018/002193) from the Department of Science and Technology, Government of India.

\bibliographystyle{alpha-abbrv}
\bibliography{swm-aic.bib}

\end{document}
