\documentclass[dvipsnames]{amsart}
\usepackage[utf8]{inputenc}
\usepackage{graphicx}
\usepackage{amssymb}
\usepackage{amsmath,tensor}
\usepackage{float}
\usepackage{xcolor}
\usepackage[
%backend=biber,
style=alphabetic,
%sorting=ynt
maxbibnames=9,
doi=false,isbn=false,url=false,
]{biblatex}
\usepackage{mathscinet}

\usepackage{hyperref}
\usepackage{cleveref}
\usepackage{comment}
\usepackage[all]{xy}

\addbibresource{sll.bib}

\newtheorem{thm}{Theorem}[section] 
\newtheorem{prop}[thm]{Proposition}
\newtheorem{lem}[thm]{Lemma}
\newtheorem{cor}[thm]{Corollary}
\newtheorem{conj}[thm]{Conjecture}
\newtheorem{assumption}[thm]{Assumption}

\theoremstyle{definition}
\newtheorem{defn}[thm]{Definition}
\newtheorem{notation}[thm]{Notation}
\newtheorem{ex}[thm]{Example}

\theoremstyle{remark}
\newtheorem{rem}[thm]{Remark}
\newtheorem{ques}[thm]{Question}


% FONTS

\newcommand{\uw}{\underline{w}}
\newcommand{\ux}{\underline{x}}
\newcommand{\uy}{\underline{y}}
\newcommand{\uz}{\underline{z}}
\newcommand{\un}{\underline{n}}
\newcommand{\um}{\underline{m}}
\newcommand{\eb}{\mathbf{e}}
\newcommand{\hg}{\mathfrak{h}}
\newcommand{\mg}{\mathfrak{m}}
\def\R{\mathbb{R}}
\def\Z{\mathbb{Z}}
\newcommand{\gdim}{\mathrm{gd}}

% GREEK

\newcommand{\al}{\alpha}
\newcommand{\De}{\Delta}

% GENERAL TEX SHORTCUTS

\def\hat{\widehat}
\newcommand{\ot}{\otimes}
\newcommand{\co}{\colon}
\newcommand{\pa}{\partial}
\DeclareMathOperator{\grank}{grk}
\DeclareMathOperator{\Hom}{Hom}
\DeclareMathOperator{\End}{End}
\DeclareMathOperator{\Image}{Image}
\DeclareMathOperator{\id}{id}
\newcommand{\namedto}[1]{\stackrel{#1}{\longrightarrow}}
\newcommand{\sumset}{\stackrel{\oplus}{\subset}}
\DeclareMathOperator{\Id}{Id}
\newcommand{\im}{\operatorname{Im}}
\newcommand{\inv}{^{-1}}

% PROJECT SPECIFIC NOTATION

\newcommand{\mi}{\underline}
\newcommand{\ma}{\overline}
\newcommand{\expr}{\leftrightharpoons}

\newcommand{\SCWebAlg}{\SC(S_\star)}
\newcommand{\SCWebDiag}{\SC_{\Webs}} 
\newcommand{\SSBim}{\mathcal{S}\mathbb{S}\textrm{Bim}}
\newcommand{\Dem}{\mathcal{D}}
\newcommand{\HB}{\mathbb{H}}
\newcommand{\HC}{\mathcal{H}}
\newcommand{\SHC}{\mathcal{SH}}
\newcommand{\HA}{\mathfrak{H}}
\newcommand{\SC}{\mathcal{SC}}
\newcommand{\mt}{\emptyset}
\newcommand{\Sym}{\operatorname{Sym}}
\newcommand{\Frob}{\mathbf{Frob}}
\newcommand{\SSDiag}{\mathbf{SSDiag}}
\newcommand{\SBSBim}{\mathbf{SBSBim}}
\newcommand{\DM}{\mathbb{D}}
\newcommand{\simto}{\stackrel{\sim}{\to}}
\newcommand{\LL}{\mathbb{LL}}
\newcommand{\calL}{\mathcal{L}}
\newcommand{\calM}{\mathcal{M}}


\DeclareMathOperator{\BS}{BS}
\DeclareMathOperator{\Fl}{Fl}
\DeclareMathOperator{\Cox}{Cox}
\DeclareMathOperator{\leftdes}{LD}
\DeclareMathOperator{\rightdes}{RD}
\DeclareMathOperator{\leftred}{LR}
\DeclareMathOperator{\rightred}{RR}
\DeclareMathOperator{\core}{core}
\DeclareMathOperator{\cube}{cube}
\DeclareMathOperator{\Webs}{Webs}
\DeclareMathOperator{\Std}{Std}
\newcommand{\evaluation}{\mathcal{F}}
\newcommand{\genha}{b}
\DeclareMathOperator{\habs}{bs}
\DeclareMathOperator{\defect}{def}
\DeclareMathOperator{\poly}{poly}
\DeclareMathOperator{\term}{term}
\newcommand{\specialrex}{Y}
\newcommand{\specialI}{I^{\mathrm{special}}}
\DeclareMathOperator{\quot}{quot}

\newcommand{\leftup}[1]{{}^{#1}\hspace{-0.03cm}}
\newcommand{\leftdown}[1]{\phantom{R}_{#1}}
\newcommand{\restrict}[3]{\tensor*[_{#1}]{{#2}}{_{#3}}}

\newcommand{\ig}[2]{\vcenter{\xy (0,0)*{% Figure removed} \endxy}}

%%%%%%%%%%%%%colored letters$$$$

\newcommand{\sberry}{\textcolor{WildStrawberry}{s}}
\newcommand{\teal}{\textcolor{teal}{t}}
\newcommand{\urial}{\textcolor{Brown}{u}}


\def\BE#1{\textcolor[rgb]{.75,0.00,0.00}{[BE: #1]}}
\def\HK#1{\textcolor[rgb]{.0,0,.75}{[HK: #1]}}
\def\hk#1{\textcolor[rgb]{.0,0,.75}{ #1}}
\def\NL#1{\textcolor[rgb]{.2, .5, .2}{[NL: #1]}}
\def\LP#1{\textcolor[rgb]{.25,0,.5}{[LP: #1]}}


\title{Subexpressions and the Bruhat order for double cosets}
\author{Ben Elias, Hankyung Ko, Nicolas Libedinsky, Leonardo Patimo}
\date{April 2021}
\begin{document}
\maketitle

\begin{abstract}

The Bruhat order on a Coxeter group is often described by examining subexpressions of a reduced expression. We prove that an analogous description applies to the Bruhat order on double cosets. This establishes the compatibility of the Bruhat order on double cosets with concatenation, leading to compatibility between the monoidal structure and the ideal of lower terms in the singular Hecke 2-category.


% This  reveals a compatibility between the monoidal structure and the ideal of lower terms  in the singular Hecke 2-category.  





\end{abstract}


\section{Introduction}

Let $(W,S)$ be a Coxeter system. There are many equivalent definitions of the Bruhat order on $W$. One of the most useful definitions is this: $x \le y$ if a reduced expression for $x$ appears as a subexpression of a (or equivalently any) reduced expression for $y$.

In \cite{WThesis}, Williamson defines the concept of a reduced expression for a double coset $p \in W_I \backslash W / W_J$. Here $I$ and $J$ are subsets of $S$, and $W_I$ and $W_J$ the parabolic subgroups they generate, which are assumed to be finite. The recent paper \cite{EKo} expands greatly on this concept, defining expressions (not necessarily reduced) and giving many equivalent and more practical definitions of reduced expressions. 
One important omission from \cite{EKo}, which we rectify in this paper, is a definition of the Bruhat order in terms of subexpressions of reduced expressions.

Expressions for double cosets have a different flavor than ordinary expressions. Ordinary expressions  are lists in $S$.  One can omit some of the elements of this list and their multiplication is still a well-defined element of $W$. Meanwhile,   expressions for double cosets are lists of subsets of $S$, where two subsets adjacent in the list  differ by one element of $S$. One can not omit some of the elements in such a list.
% Meanwhile, double cosets form a category, and expressions for double cosets are lists of generating morphisms in this category. One can multiply them one by one to obtain a sequence of cosets. One can not omit some of the elements in this list, or the morphisms are no longer composable.

The appropriate replacement for subexpressions is the notion of a \emph{path subordinate to an expression} (see Definition~\ref{defn:pathsubord}). 
A subordinate path is a sequence of double cosets obtained from an expression, but using a nondeterministic version of multiplication. This is analogous to the non-determinism in an ordinary subexpression, where instead of multiplying by a simple reflection $s$, we are permitted to multiply by either $1$ or $s$. The appropriate replacement for ``the element of $W$ associated to the subexpression''  is the \emph{terminus} of a path.

The Bruhat order on cosets is usually defined by $p \le q$ if and only if $\mi{p} \le \mi{q}$ in the ordinary Bruhat order. Here $\mi{p}$ and $\mi{q}$ are the minimal elements in these cosets. Our main result, Theorem~\ref{thm:bruhat}, states that $p \le q$ if and only if $p$ is the terminus of a path subordinate to a (or equivalently any) reduced expression for $q$.

One important consequence of the subexpression definition of Bruhat order is its compatibility with concatenation. Suppose that the product $xyz$ is reduced, in that $\ell(xyz) = \ell(x) + \ell(y) + \ell(z)$, and that $y' < y$. Then $xy' z < xyz$.

An even stronger statement is true. Note that $x y' z$ might not be reduced. Then any element obtained as a subexpression of the concatenation of reduced expressions of $x$, $y'$, and $z$ is also strictly less than $xyz$.

\begin{ex} 
Consider $x=z=s$ and $ t=y$ for $s$ and $t$ non-commuting simple reflections, and $y' = 1 \le  y$. The first statement says that $1 = ss < sts$. The stronger statement says that any subexpression of $(s,s)$ expresses an element smaller than $sts$. This implies that $s < sts$. \end{ex}

One can phrase the stronger result succinctly using the $*$-product (or Demazure product)  rather than the ordinary product in $W$. To whit,  $x * y' * z < x * y * z$. As a consequence of our main theorem, we prove  in Theorem~\ref{thm:bruhatconcatcompat} the corresponding result for double cosets.


Finally, one of the reasons that the compatibility of Bruhat order with concatenation is important is that it plays a role in the Hecke category (e.g. the category of Soergel bimodules), where concatenation is lifted to a monoidal structure. Meanwhile, expressions for double cosets yield  1-morphisms
in the singular Hecke 2-category (e.g. the 2-category of singular Soergel bimodules). By virtue of Theorem~\ref{thm:bruhatconcatcompat}, we establish in Proposition~\ref{prop:lowertermslocal} a compatibility between the monoidal structure and the ideal of lower terms in the singular Hecke 2-category.

{\bf Acknowledgments.} NL was partially supported by FONDECYT-ANID grant 1230247. BE was partially supported by NSF grant DMS-2201387.

\section{Bruhat order}\label{ss:bruhat}

\subsection{Recollections} 
 We recall some notations and results from \cite{EKo}, where the reader can find many more details and explanations.

 Let $(W,S)$ be a Coxeter system. For $I\subset S$, we denote by $W_I$ the subgroup of
$W$ generated by $I$.  When $W_I$ is finite, we say that $I$ is \textit{finitary}, and we write $w_I$ for the
longest element of $W_I$. We write $\ell(I) := \ell(w_I)$.

For $J, I\subset S$, a \emph{$(J,I)$-coset} is an element $p$ in 
 $W_{J}\backslash W/W_{I}$. When we write ``the coset $p$" we  mean the triple $(p,J,I)$. It might happen that  $(p,J,I)\neq (p',J',I')$, even though $p=p'$ as subsets of $W$, and we distinguish between $p$ and $p'$ in this case. If $p$ is a $(J,I)$-coset we denote by $\overline{p}\in W$ and $\underline{p}\in W$ the maximal and minimal elements in the Bruhat order in the set $p$.

 
 A (singular) \emph{multistep  expression} is a sequence of finitary subsets of $S$ of the form 
\begin{equation}\label{ex}
 L_{\bullet}=[[ I_0\subset K_1\supset I_1\subset K_2 \supset \cdots \subset K_m\supset I_m]].
 \end{equation}
 By convention, if we write a multistep expression as $[[K_1 \supset I_1 \subset \cdots]]$, this means that $I_0 = K_1$,
and similarly $[[\cdots \subset K_m]]$ means that $I_m = K_m$. 
For $L\subset S$ and $s\in S$ we use the notation $Ls:=L\cup \{s\}$ (here we assume $s\notin L$).


A (singular)  \emph{singlestep expression} $I_{\bullet}=[I_0,I_1,\ldots, I_d]$  is a sequence of finitary subsets of $S$ such that, for all $1\leq i\leq d$, either $I_i=I_{i-1}s$  or  $I_i=I_{i-1}/ s$ for some $s\in S.$ To each singlestep expression, one can associate a multistep  expression by remembering its local maxima and minima.

\begin{defn}\label{Coxmon}
    Let $(W,S)$ be a Coxeter system. We define the \textit{Coxeter monoid} $(W,*, S)$ by the following presentation. It has generators $s\in S$ and relations 
    \begin{itemize}
\item $s*s=s$ for $s\in S.$
\item $\underbrace{s*t*s*\cdots}_{m_{st}}=\underbrace{t*s*t*\cdots}_{m_{st}}$ for all $s,t\in S. $
    \end{itemize}
\end{defn}

Elements of the Coxeter monoid $(W, *, S)$ are naturally in bijection with the elements of $W$, and we implicitly use this bijection. We write $x.y = xy$ as shorthand for the statement $xy = x * y$, which is equivalent to $\ell(x) + \ell(y) = \ell(xy)$ (see \cite[Lemma 2.3]{EKo}). If $x.y = xy$ we also say that the product $xy$ is \emph{reduced}.

\begin{lem}\label{starprod}
If $a,b,c\in W$ and $a\leq b$, then $a*c\leq b*c.$
\end{lem}
\begin{proof}
If $\ell(c)=1$, then $c$ is a simple reflection and the result follows by \cite[(2.1)]{EKo} and \cite[Proposition 2.2.7]{BjornerBrenti}. The general case easily follows by induction on $\ell(c)$.
\end{proof}

We say that the multistep  expression $L_{\bullet}$ in  \eqref{ex}  \textit{expresses}  $p$ and we write $L_{\bullet}\expr p$ if $p$ is the unique $(I_0,I_m)$-coset with
$$
\overline{p}=w_{I_0}*w_{K_1}*w_{I_1} * (\cdots) *w_{I_m}.
$$
We say that $L_\bullet$ is \emph{reduced} if
 \begin{equation}\label{reduced}
\ma{p}=(w_{K_1}w_{I_1}^{-1}).(w_{K_2}w_{I_2}^{-1}).(\cdots).(w_{K_m}w_{I_m}^{-1})
  \end{equation}
is reduced.

A  singlestep expression $I_{\bullet}=[I_0,I_1,\ldots, I_m]$ expresses the $(I_0,I_m)$-coset of the associated multistep expression, which is also the unique coset $p$ such that
\begin{equation}\label{expressp}
\overline{p}=w_{I_0}*w_{I_1}*\cdots *w_{I_m}.
\end{equation}
We say that a singlestep expression is reduced if its associated multistep expression is reduced. 
 


\begin{defn} (see \cite[Definition 1.2.13]{WThesis})  Let $I, J \subset S$ be finitary, and $p, q \in W_J\setminus W/W_I$. Then $p \le q$ in the Bruhat order on double cosets if and only if $\mi{p} \le \mi{q}$ in the Bruhat order on $W$. \end{defn}

In \cite[\S 3.4]{EKo} a length function is defined for double cosets $\ell(p)$ and for expressions $\ell(I_{\bullet})$, such that the length of a double coset equals the length of any reduced expression. We note here an equivalent formulation of the length of a coset. For an $(I,J)$-coset $q$, we have
\begin{equation} \ell(q) = 2 \ell_W(\ma{q}) - \ell(I) - \ell(J), \end{equation}
where $\ell_W$ is the usual length function on $W$.
 Using \cite[Lemma 2.12 (3)]{EKo}, the reader can easily verify that this formula agrees with the definition of length given in \cite[Definition 3.14]{EKo}. 
By \cite[Chapter
2, Proposition 1.3.4]{WThesis}, for any coset $q$ there is a reduced expression $I_{\bullet}\expr q$. The  third bullet of \cite[Proposition 3.17]{EKo} tell us that  $\ell(I_{\bullet})=\ell(q)$. It is clear from the definition of $\ell(I_{\bullet})$ that the length of an expression will strictly increase with the addition of each index. Thus we can see that a coset $q$ has length zero if and only if $q\expr [I]$, or in other words, if $I=J$ and $q = q_{\id}$ is the coset containing the identity element.

We will need the following lemma on the classical Bruhat order.

\begin{lem} \label{liftinglemma} Let $v, w, x \in W$ and $v \le w$. If $vx = v.x$ (resp. $xv = x.v$) then there exists some $x' \le x$ such that $v.x \le w.x'$ (resp. $x.v \le x'.w$). \end{lem}

\begin{proof} First we consider the case when $x = s$ is a simple reflection. Let $w'$ be the larger of $\{w, ws\}$ in the Bruhat order; clearly $w' = w . x'$ for some $x' \le x$. By hypothesis we have that $vs > v$. We wish to show that $vs \le w'$. When $ws > w$ this follows quickly from the subexpression version of the Bruhat order. When $ws < w$ this follows from \cite[Proposition 2.2.7]{BjornerBrenti}.
For more general $x$, we apply the case of a simple reflection $\ell(x)$ times. \end{proof}


\subsection{Subordinate paths}

\begin{defn}\label{defn:pathsubord} (see \cite[Definition 2.17]{EKo}) Let $I_{\bullet}= [I_0, \ldots, I_d]$ be a singlestep expression. A \emph{path subordinate to $I_{\bullet}$} is a sequence $p_{\bullet} = [p_0, \ldots, p_d]$ where $p_i$ is a $(I_0, I_i)$-coset. The sequence satisfies:
\begin{itemize}
\item $p_0 = p_{\id}$, the $(I_0, I_0)$-coset containing the identity.
\item If $I_k \subset I_{k+1}$ then $p_{k+1}$ is the unique double coset containing $p_k$.
\item If $I_k \supset I_{k+1}$ then $p_{k+1}$ is one of the double cosets contained in $p_k$.
\end{itemize}
We write $p_{\bullet} \subset I_{\bullet}$. The final $(I_0, I_d)$-coset $p_d$ is called the \emph{terminus} of the path, and denoted $\term(p_{\bullet})$. 
A path is said to be \textit{forward} if every time that $I_k \supset I_{k+1}$ then $\ma{p_{k+1}}=\ma{p_k}$. Each expression has a unique forward path. One can prove that if $p_{\bullet}$ is the forward path of $I_{\bullet}$, then $I_{\bullet} \expr \term(p_{\bullet}).$
\end{defn}

We can also summarize the last two conditions 
above by saying that $p_k \cap p_{k+1}$ is nonempty. Paths subordinate to an expression $I_{\bullet}$ are the singular analogue of subexpressions, and the sequence $p_{\bullet}$ is analogous to the Bruhat stroll of \cite[Section 2.4]{Soergelcalculus}.

\begin{thm}\label{thm:bruhat}
Let $p,q$ be $(J,I)$-cosets, for fixed finitary subsets $J,I\subset S$. The following conditions are equivalent.
\begin{enumerate}
    \item\label{bruhatma} $\ma{p}\leq \ma{q}$ in $W$.
    \item\label{bruhatmi} $\mi{p}\leq \mi{q}$ in $W$.
    \item\label{bruhat1} There exists  a reduced expression $I_\bullet \expr q$ and a subordinate path $p_\bullet\subset I_\bullet$ such that $p=\term(p_\bullet)$.
    \item\label{bruhat2} For any reduced expression $I_\bullet \expr q$, there exists a subordinate path $p_\bullet\subset I_\bullet$ such that $p=\term(p_\bullet)$.
\end{enumerate}
\end{thm}


\begin{proof}  In \cite[Proposition 1.2.14]{WThesis}, Williamson proves that the projection map $\quot : W\to W_J\setminus W/W_I$ is a \emph{strict poset morphism}, a concept defined immediately before that proposition.

\eqref{bruhatma}$\implies$\eqref{bruhatmi}: Since $\quot$ is a poset morphism, $\ma{p} \le \ma{q}$ implies $p \le q$, which by definition means $\mi{p} \le \mi{q}$.

\eqref{bruhatmi}$\implies$\eqref{bruhatma}:
By definition, $\mi{p} \le \mi{q}$ implies $p \le q$. Since $\quot$ is a strict poset morphism and $\quot(\ma{p}) = p$, there exists some $x \in q$ such that $\ma{p} \le x$. But then $x \le \ma{q}$, so by transitivity $\ma{p} \le \ma{q}$.

% To be a poset morphism means that if $v \in p$, $w \in q$ and $v \le w$, then $\mi{p} \le \mi{q}$. In particular if $v=\ma{p}$ and $w=\ma{q}$ we obtain the implication $\implies.$
% The definition of the projection map  being strict is that if  $p\leq q$, $v\in p$ and $w\in q$, then $v\leq \ma{q}$ and  $ \mi{p}\leq w.$ We obtain the  implication $\impliedby$ by replacing $v=\ma{p}.$

%\eqref{bruhatmi}$\implies$\eqref{bruhatma}: Let $x \in W_J$ and $y \in W_I$ be such that $x . \mi{p} . y  = \ma{p}$. By Lemma~\ref{liftinglemma}, there exists $x' \le x$ and $y' \le y$ such that $\ma{p} \le x' \mi{q} y'$. Since $x' \mi{q} y' \in q$, it must satisfy $x' \mi{q} y' \le \ma{q}$, whence the desired result.

% \eqref{bruhatma}$\Leftrightarrow$\eqref{bruhatmi}:
%  Let $x\in W_J$ and $y\in W_I$ be such that $\mi{q}=x.\ma{q}.y$. Then by the lifting property we have $x'\ma{p}y'\leq \mi{q}$ where $x'\leq x$ and $y'\leq y$. In particular $x'\ma{p}y'\in p$ and thus $\mi{p}\leq x'\ma{p}y'\leq \mi{q}$. 
% The converse is proved in the same way.

    \eqref{bruhat1}$\implies$\eqref{bruhatma} or \eqref{bruhatmi}:  We prove the claim by induction on the length of $q$.     When $\ell(q)=0$, the only reduced expression for $q$ is $[I]$, and the only subordinate path has terminus $p = q$.

Let $\ell(q)>0$, let $I_\bullet=[I_0,\ldots, I_d]\expr q$ be a reduced expression, and let $p_\bullet\subset I_\bullet$ be such that $\term(p_\bullet) = p$.  
    Let
    $I_{[0,d-1]} := [I_0, \ldots, I_{d-1}]$, an expression of strictly smaller length, which is reduced by \cite[Proposition 3.12]{EKo}. Let  $q'\expr I_{[0,d-1]}$. 
    Then $[p_0,\ldots,p_{d-1}]$ is a path subordinate to $I_{[0,d-1]}$. By induction hypothesis, we have $\ma{p_{d-1}}\leq \ma{q'}$ and $\mi{p_{d-1}}\leq \mi{q'}$. There are now two cases.
    
    When $I_d = I_{d-1}s$, for some $s\in S$, we have 
    \[ \mi{p}\leq\mi{p_{d-1}}\leq \mi{q'} = \mi{q}. \] 
     The last equality is because $I_{\bullet}$ is reduced (see \cite[Definition 2.24]{EKo}).
     
     When $I_ds = I_{d-1}$, we have \[ \ma{p}\leq \ma{p_{d-1}}\leq \ma{q'}=\ma{q}.\]
    
    
    \eqref{bruhatma} and \eqref{bruhatmi}$\implies$\eqref{bruhat2}: 
    We prove the claim by induction on the length of $q$. 
    
    When $\ell(q)=0$, the only reduced expression for $q$ is $[I]$, and $\mi{q} = \id$. If $\mi{p} \le \mi{q}$ then $\mi{p}=\id$ and $p = q$, thus establishing the base case. 
%When $\ell(q)=0$, then $I=J$ is necessary and $\mi{q} = \id$. If $\mi{p} \le \mi{q}$ then $p = q$, establishing the base case.

    Suppose $\ell(q)>0$ and $\mi{p}\leq \mi{q}$. %, and $\ma{p}\leq \ma{q}$. 
    Let $I_\bullet=[I_0,\ldots, I_d]\expr q$ be a reduced expression.  We consider  the reduced expression $I_{[0,d-1]} := [I_0, \ldots, I_{d-1}] \expr q'$ of  strictly smaller length. We let $p'$ be the minimal $(J,I_{d-1})$-coset containing or contained in $p$, i.e., $p' = W_J \mi{p} W_{I_{d-1}}$.
    If we can prove that $\mi{p'} \le \mi{q'}$ or $\ma{p'} \le \ma{q'}$, then by induction, there is a subordinate path $p'_{\bullet} \subset I_{[0,d-1]}$ with $\term(p'_{\bullet}) = p'$. Then $p_{\bullet} := [p'_{\bullet}, p] \subset I_{\bullet}$ is the desired subordinate path with terminus $p$.
    
    As usual, $I_d = I_{d-1}s$ or $I_ds = I_{d-1}$ for some $s\in S$, and we treat the cases separately. Note that $q' = W_J \mi{q} W_{I_{d-1}}$ either way, because $I_{\bullet}$ is reduced. 

    If $I_d = I_{d-1}s$, then $\mi{p'}= \mi{p}\leq \mi{q} = \mi{q'}$, where the last equality is due to $I_\bullet$ being reduced.

    If $I_ds = I_{d-1}$ then $p \subset p'$ and $\ma{p'} = \ma{p} . y$ for some $y \in W_{I_{d-1}}$. Since $\ma{p} \le \ma{q}$, Lemma~\ref{liftinglemma} implies that there exists $y' \le y$ such that $\ma{p'} \le \ma{q}.y'$. But $\ma{q}.y' \in q'$, hence $\ma{q}.y' \le \ma{q'}$. 

    % If $Is = I_{d-1}$, then write $\mi{q}= x.\mi{q'}.y$ for $x\in W_J, y\in W_{Is}$. 
    % Then the lifting property \ben{I don't get it, will think later} says that $\mi{p}\leq \mi{q}$ implies $x'\mi{p}y' \leq \mi{q'}$ for some $x'\leq x \in W_I$ and $y'\leq y\in W_{Is}$. So $\mi{p'}\leq \mi{q'}$ and we apply the induction step to find $p_\bullet\subset I_{[0,d-1]}$ where $\term{p_\bullet} = p'$. Then $[p_\bullet,p]\subset I_\bullet$ is a desired subordinate path.

    The direction \eqref{bruhat2}$\implies$\eqref{bruhat1} is clear, and thus the proof is complete.
\end{proof}

Thus the Bruhat order on double cosets has an equivalent definition: $p \le q$ if $p$ is the terminus of a path subordinate to some (or any) reduced expression of $q$.

\subsection{Concatenation of paths}\label{concat}

If $\ux'$ is a subexpression of $\ux$ and $\uy'$ is a subexpression of $\uy$, it is clear that the concatenation $\underline{x'y'}$ is a subexpression of the concatenation $\underline{xy}$. It is less obvious how to ``concatenate'' two subordinate paths.
Recall from Definition~\ref{Coxmon} the star product. In \cite{EKo} a $*$-product was defined on double cosets. If $p$ is an $(I,J)$-coset and $q$ is a $(J,K)$-coset then $p * q$ is the $(I,K)$-coset satisfying $\ma{(p * q)} = \ma{p} * \ma{q}$ (see \cite[Lemma 2.7]{EKo} to see that this is well-defined). 


\begin{lem} \label{lem:inclusions} Let $p$ be an $(I,J)$-coset, $q$ be a $(J,K)$-coset, and $q'$ be a $(J,K')$-coset, with $K' \subset K$ and $q' \subset q$. Then $p * q' \subset p * q$. \end{lem}

\begin{proof} By \cite[Lemma 2.15]{ EKo} we know that $\ma{q} = \ma{q'} . y$ for some $y \in W_{K}$. Then,  by \cite[Lemmas 2.3 and 2.2]{EKo}, $\ma{p} * \ma{q} = \ma{p} * \ma{q'} * y = (\ma{p} * \ma{q'}) . y'$ for some $y' \le y$. In particular, $\ma{p} * \ma{q}$ and $\ma{p} * \ma{q'}$ are in the same right $W_{K}$-coset, implying $p * q' \subset p * q$. \end{proof}

\begin{defn} Let $P_{\bullet} = [P_0, P_1, \ldots, P_c]$ be an expression and $p_{\bullet}$ a subordinate path with terminus $p$, and similarly for $Q_{\bullet} = [Q_0, \ldots, Q_d]$, $q_{\bullet}$ and $q$. Suppose $P_c = Q_0$, so the composition $P_{\bullet} \circ Q_{\bullet}$ exists. Then define the concatenation $p_{\bullet} \circ q_{\bullet}$ as the sequence of cosets
\begin{equation} [p_0, p_1, \ldots, p_c = p = p * q_0, p * q_1, \ldots, p * q_d = p * q]. \end{equation}
\end{defn}

Note that $p * q_0 = p$ since $q_0 = q_{\id}$ is the identity $(P_c,P_c)$-coset, which acts as the identity under $*$-composition  (recall that for any $J\subset S,$ we have $w_J*w_J=w_J$). We remark that if $P_{\bullet}\expr p$ and $Q_{\bullet}\expr q$ then by definition of the star product on cosets and by Equation~\ref{expressp}, we have $P_{\bullet} \circ Q_{\bullet}\expr p*q$.

\begin{lem} The sequence $p_{\bullet} \circ q_{\bullet}$ is a path subordinate to $P_{\bullet} \circ Q_{\bullet}$. \end{lem}

\begin{proof} We need to verify that each term of $p_{\bullet} \circ q_{\bullet}$ is a coset of the appropriate kind, and that the intersection of two adjacent terms in the sequence is nonempty. The only interesting part is to prove that $(p * q_i) \cap (p * q_{i+1})$ is nonempty. Either $q_i \subset q_{i+1}$ or $q_{i+1} \subset q_i$, and the result follows from Lemma~\ref{lem:inclusions} either way. \end{proof}


Recall that for $x, y \in W$ we write $xy = x.y$ if $\ell(x) + \ell(y) = xy$,  and we call the composition \emph{reduced}. 

\begin{notation} \label{dotforcosets} %This notation is not found in \cite{EKo}. 
For an $(I,J)$-coset $p$, a $(J,K)$-coset $q$, and a $(I,K)$-coset $r$, let us write $p . q = r$ if we can find $I_{\bullet}$ and $K_{\bullet}$ such that
\[ I_{\bullet} \expr p, \quad K_{\bullet} \expr q, \quad I_{\bullet} \circ K_{\bullet} \expr r\]
are all reduced expressions. We say that $r$ is a \emph{reduced composition} of $p$ and $q$.  By \cite[Proposition 4.3]{EKo}, $p . q = r$ if and only if $\ma{r} = \ma{p}. (w_J^{-1} \ma{q}) = (\ma{p} w_J^{-1}) . \ma{q}$.
\end{notation}




\subsection{Bruhat order and concatenation}

% The following fact was claimed in the introduction.

% \begin{lem} Let $x, y, z \in W$, and pick reduced expressions $\ux$ for $x$ and $\uy$ for $y$. Then $z \le x * y$ if and only if we can find a reduced expression for $z$ inside the concatenation $\un{xy}$.  \end{lem}

% \begin{proof} Equivalently, we must prove that $x * y$ is the unique maximal element amongst all elements expressed by subexpressions of $\un{xy}$. By induction on $\ell_W(y)$ this can be reduced to the case when $y = s$ is a simple reflection, in which case it is a simple consequence of Lemma~\ref{liftinglemma}. We leave the details to the reader. \end{proof}

% Here is the property of the ordinary Bruhat order we wish to generalize to double cosets. We include the proof for pedagogical reasons.

% \begin{lem} Let $x, y, y', z \in W$ with $y' < y$, and suppose $x . y . z$ is a reduced composition. Then $x*y'*z < x.y.z$. \end{lem}

% \begin{proof} Pick reduced expressions $\ux$, $\uy$, and $\uz$ for $x$, $y$, and $z$ respectively. Let $\uy'$ be a proper subexpression of $\uy$ which is a reduced expression for $y'$. Some subexpression of $\un{xy'z}$ expresses $x*y'*z$, and any subexpression of $\un{xy'z}$ is a proper subexpression of $\un{xyz}$. Thus $x*y'*z < x.y.z$. \end{proof}
 
% In \cite{EKo} a $*$-product was defined on double cosets, satisfying $\ma{p * q} = \ma{p} * \ma{q}$. The main technicality in adapting the proof above to double cosets is again the difference between subexpressions and subordinate paths. One can concatenate subexpressions, but it is not immediately obvious how to concatenate subordinate paths.

\begin{thm}  \label{thm:bruhatconcatcompat} Suppose that $q$ and $q'$ are $(J,I)$-cosets with $q' < q$, and let $p$ be a $(K,J)$-coset and $r$ a $(I,L)$-coset. If $p . q . r$ is a reduced composition, then 
\begin{equation} p * q' * r < p . q . r. \end{equation}
\end{thm}

\begin{proof} Pick reduced expressions $P_{\bullet}$ and $Q_{\bullet}$ and $R_{\bullet}$ for $p$, $q$, and $r$ respectively. Then $P_{\bullet} \circ Q_{\bullet} \circ R_{\bullet}$ is a reduced expression for $p.q.r$.  Let 
$q'_{\bullet} $ be a path subordinate to $Q_{\bullet}$ with terminus $q'$. Let $p_{\bullet}$ and $r_{\bullet}$ the forward paths of $P_{\bullet}$ and $R_{\bullet}$.
The concatenation $p_{\bullet} \circ q'_{\bullet} \circ r_{\bullet}$ is a path subordinate to $P_{\bullet} \circ Q_{\bullet} \circ R_{\bullet}$ which expresses $p * q' * r$. Now the result follows from Theorem~\ref{thm:bruhat}. \end{proof}









% There is a classical result about Bruhat order and its compatibility with concatenation, whose double coset analogue can be easily proven with the above proposition. 
% %Remark~\ref{rem:lowertermsandconcat}.
% Let $w, x, y, z \in W$, and pick reduced expressions $\ux$ for $x$ and $\uy$ for $y$. We say that $z \le x * y$ if one can find a reduced expression for $z$ as a subexpression of the concatenation $\ux \uy$. We say that $x * y \le w$ if for all $z \le x*y$ we also have $z \le w$.

% \begin{rem} \label{rem:starproductdef} Indeed, defining $x * y$ as the maximal $z$ such that $z \le x*y$ will equip $W$ with an alternate product called the \emph{star product} or \emph{Demazure product} (it has other names as well). One can define similar operations on double cosets, and this was a major topic of study in \cite{EKo}. See \S\ref{ss:nilCoxeter} for more details.  \end{rem}

% The classical result we seek to imitate is this: if $y' < y$ and $x.y.z$ is a reduced composition then $x*y'*z < x.y.z$. The proof is immediate from the subexpression definition of Bruhat order. Since $\uy'$ (a reduced expression for $y'$) can be chosen to be a proper subexpression of $\uy$ (etcetera), any subexpression of $\ux \uy' \uz$ is a proper subexpression of $\ux \uy \uz$.

% One can make completely analogous definitions in the context of double cosets, and the proof is the same.

% \begin{cor} \label{cor:bruhatconcatcompat} Suppose that $p$ and $q$ are $(J,I)$-cosets with $p < q$, and let $r_1$ be a $(K,J)$-coset and $r_2$ a $(I,L)$-coset. If $r_1 . q . r_2$ is a reduced composition, then 
% \begin{equation} r_1 * p * r_2 < r_1 . q . r_2. \end{equation}
% \end{cor}

% \begin{rem} This corollary will be used in a sequel to justify that the ideal of lower terms (in the category of singular Soergel bimodules) is sent to the ideal of lower terms by certain concatenations with identity maps. \end{rem}

% % \begin{proof} Let $I_{\bullet}$ be a reduced expression for $q$. By Proposition~\ref{prop:bruhat}, there exists a subordinate path $p_{\bullet} \subset I_{\bullet}$ with $\term(p_{\bullet}) = p$. In EKo, it is proven that $r_1 * p_{\bullet}*r_2$ is a path subordinate to $r_1 \circ I_{\bullet} \circ r_2$, and hence $r_1 * p * r_2 \le r_1 . q . r_2$. \end{proof}


\section{Ideals of lower terms in the singular Hecke 2-category} \label{sec:lower}

The singular Hecke 2-category is a categorification of the Hecke algebroid (see \cite[Definition 2.2.1]{WThesis}). Like most categorifications it has several incarnations, all of which are isomorphic in non-degenerate characteristic zero situations, but which may differ in general. Currently, in the literature, there is a geometric incarnation (using perverse sheaves on partial flag varieties, equivariant under parabolic subgroups of a Lie group, see \cite[p8-10]{WThesis}), and an algebraic incarnation using singular Soergel bimodules (which are direct summands of singular Bott-Samelson bimodules, to be defined below). The latter is the topic of Williamson's thesis \cite{WThesis}, which also appears in a shortened article version \cite{SingSb}.

One can also expect a diagrammatic incarnation by generators and relations, following the rubrick set out in \cite{ESW}, see also \cite[Chapter 24]{GBM}. The diagrammatic version has not been fully developed. In the interest in developing it further, it is important to establish combinatorially some basic facts about morphisms in the 2-category. In this paper, we address the ideal of lower terms.

 We state our result as it applies to singular Bott-Samelson bimodules.
In Proposition~\ref{ass} we isolate some properties 
 of these bimodules. Eventually, these properties shall be proven separately for the diagrammatic category, at which point the same proof given here will suffice in that context.

We fix a Coxeter system $(W,S)$ and a realization thereof \cite[Section 3.1]{Soergelcalculus}. We consider the polynomial ring $R$ of this realization, and for each finitary subset $I \subset S$, the subring $R^I$ of $W_I$-invariants in $R$. The ring $R$ is graded, and all $R^I$-modules will be graded, but we will not keep track of grading shifts beyond the definition as they will play no significant role. The background on this material in \cite[Chapter 3.1]{KELP1} should be sufficient.

\subsection{Singular Bott-Samelson bimodules}


\begin{defn} If $I\subset J$ then we define the graded  $(R^I, R^J)$-bimodule 
$$\BS([I,J]) = R^I, $$
and the graded $(R^J, R^I)$-bimodule 
$$\BS([J,I]) = R^I(\ell(J) - \ell(I)).  $$ 
By convention, the grading shift by the positive integer $\ell(J) - \ell(I)$ places $1 \in R^I$ in negative degree $\ell(I) - \ell(J)$. \end{defn}

\begin{defn} We define the \emph{singular Bott--Samelson bimodule} $\BS(I_{\bullet})$ associated to a singular expression $I_{\bullet} = [I_0, \ldots, I_d]$ as the graded $(R^{I_0}, R^{I_d})$-bimodule
\begin{equation} \BS(I_{\bullet}) = \BS([I_0,I_1]) \ot_{R^{I_1}} \BS([I_1, I_2]) \ot_{R^{I_2}} \cdots \ot_{R^{I_{d-1}}} \BS([I_{d-1},I_d]). \end{equation}
\end{defn}

%\begin{defn} Let \[ I_{\bullet} = [[I= I_0 \subset K_1 \supset I_1 \subset \ldots K_m \supset I_m = J]]\] be a multistep $(I,J)$ expression. To this expression, we can associate a \emph{singular Bott-Samelson bimodule}, the $(R^I, R^J)$-bimodule\[ \BS(I_{\bullet}) := R^{I_0} \ot_{R^{K_1}} R^{I_1} \ot_{R^{K_2}} \cdots \ot_{R^{K_m}} R^{I_m}.\]
%This is an $(R^I, R^J)$-bimodule. \end{defn}

We work in the 2-category of bimodules, looking at $(R^I, R^J)$-bimodules for various finitary $I, J \subset S$. The word ``morphism'' refers to a bimodule map. Below, $\Hom$ and $\End$ denote morphism spaces in the category of bimodules. 

The assumptions about Bott-Samelson bimodules we need fit into the following black box. 


\begin{prop}\label{ass} 
For each $(I,J)$-coset $p$ there exists an indecomposable graded $(R^I, R^J)$-bimodule $B_p$, which is a direct summand with multiplicity one of $\BS(I_{\bullet})$ for any reduced expression $I_{\bullet} \expr p$, but not a direct summand of $\BS(I'_{\bullet})$ whenever $\ell(I'_{\bullet}) < \ell(p)$. Every indecomposable summand of a singular Bott-Samelson bimodule is isomorphic to some $B_p$, up to grading shift. Finally, if $I_{\bullet}$ is an expression (not necessarily reduced) for a coset $p$, then every summand of $\BS(I_{\bullet})$ is isomorphic to a grading shift of $B_q$ for some $q \le p$. \end{prop}

\begin{proof}
A proof of the first claim can be found in \cite[Theorem 5.4.2]{WThesis}. A proof of the second one can be found in the proof of the same theorem. 


We prove now the last claim. For $p$ an $(I,J)$-coset, let the reader recall from \cite[p25]{WThesis} the definitions of the standard basis ${}^I\hspace{-0.05cm}H_p^J$ and of the standard generators ${}^I\hspace{-0.05cm}H^J$ in the Hecke algebroid over the ring $A=\mathbb{Z}[v,v^{-1}].
 $
 By \cite[Theorem 5.4.2]{WThesis} we have  $$\mathrm{ch}(B_q)\in {}^I\hspace{-0.05cm}H_q^J+\sum_{r<q}A\,\cdot\, {}^I\hspace{-0.05cm}{H}_r^J. $$
%
%By \cite[Theorem 3]{WThesis} we have that $\mathrm{ch}(B_q)={}^I\hspace{-0.05cm}\underline{H}_q^J,$
Since the characters of the indecomposable bimodules $B_q$ form a basis for the Hecke algebroid, the decomposition of a Bott-Samelson bimodule is determined by its character. Thus, to prove the third claim it is enough to show that 
\begin{equation}\label{character}
    \mathrm{ch}(\BS(I_{\bullet}))\in \sum_{q\leq p}A\,\cdot\, {}^I\hspace{-0.05cm}{H}_q^J.
\end{equation}


We prove Equation \eqref{character}  by induction on the width of the expression (i.e. the number of parabolic subgroups appearing in the singlestep expression). For width zero, $I_{\bullet}=[I]$, and 
$\mathrm{ch}(\BS([I]))={}^I\hspace{-0.05cm}{H}^I$ , 
establishing the base case.  Let us suppose the claim true for the expression $I'_{\bullet}=[I_0, \ldots, I_d]\expr p'$, we will prove it for $I_{\bullet}=[I_0, \ldots,  I_{d+1}]\expr p.$ We have that $$\BS(I_{\bullet})=\BS(I'_{\bullet})\otimes_{R^{I_d}}\BS([I_d,I_{d+1}]).$$ 
By the induction hypothesis,  applying the character map and \cite[Theorem 2]{WThesis}, we obtain that $\mathrm{ch}(\BS(I_{\bullet}))$ is a linear combination over $A$ of elements of the form \begin{equation}\label{star}{}^{I_0}\hspace{-0.05cm}H_{q'}^{I_d}*_{I_d}{}^{I_d}\hspace{-0.05cm}H^{I_{d+1}}\hspace{0.4cm}\mathrm{for\ }q'\leq p'.
\end{equation}
By \cite[Proposition 2.2.4]{WThesis}, we see that the element in Equation \eqref{star}
 is 
 a linear combination over $A$ of ${}^{I_0}\hspace{-0.05cm}H_q^{I_{d+1}}, $ where $q\cap q'$ is non-empty. %and ${}^{I_0}\hspace{-0.05cm}H_p^{I_{d+1}} $ appears only when $q'=p'$ and in that case appears with coefficient $1$. 
 As $\overline{p'}*w_{I_{d+1}}=\overline{p},$
 and $q'\leq p',$ we  deduce $\overline{q}\leq \overline{q'}* w_{I_{d+1}}\leq \overline{p}$, where the last inequality is due to Lemma~\ref{starprod}.
\end{proof}


\begin{cor} \label{cor:factorthroughnonreduced} Let $I_{\bullet}$ be an expression, not necessarily reduced, for a coset $p$. Then every morphism which factors through $\BS(I_{\bullet})$ is a finite sum of morphisms factoring through $\BS(M_{\bullet})$, for various reduced expressions $M_{\bullet} \expr p'$, for various cosets $p' \le p$. \end{cor}

\begin{proof} A morphism factoring through an object can be written as a finite sum of morphisms factoring through its direct summands, and vice versa. Thus a morphism which factors through $\BS(I_{\bullet})$ can be factored instead through $B_{p'}$ for various $p' \le p$ (up to taking finite sums). Morphisms factoring through $B_{p'}$ can be factored instead through $\BS(M_{\bullet})$ for a reduced expression. \end{proof}



\subsection{Lower terms}

Inside any additive category, given a collection of objects, the set of finite sums of morphisms, each of which factors through one of those objects, forms a two-sided ideal. This is the same as the ideal generated by the identity maps of those objects.

\begin{defn} Let $p$ be an $(I,J)$-coset. Consider the set of reduced expressions $M_{\bullet}$ for any $(I,J)$-coset $q$ with $q < p$. Let $\Hom_{< p}$ denote the ideal generated by the identity maps of $\BS(M_{\bullet})$ for such expressions. This is a two-sided ideal in the category of $(R^I, R^J)$-bimodules, the \emph{ideal of lower terms} relative to $p$. \end{defn}

If $B$ and $B'$ are $(R^I, R^J)$-bimodules, then $\Hom(B,B')$ is itself an $(R^I, R^J)$-bimodule in the natural way. The left action of $R^I$ (resp. the right action of $R^J$) commutes with any morphism, and therefore it preserves the factorization of morphisms. Consequently, $\Hom_{< p}(B,B') \subset \Hom(B,B')$ is a sub-bimodule.

\begin{notation} For an expression $I_{\bullet}$ write $\End_{< p}(I_{\bullet}) := \Hom_{< p}(\BS(I_{\bullet}), \BS(I_{\bullet}))$, the sub-bimodule of $\End(\BS(I_{\bullet}))$ living in the ideal $\Hom_{< p}$. \end{notation}

\subsection{The locality of lower terms}

\begin{prop} \label{prop:lowertermslocal} Let $P_{\bullet} \expr p$ and $Q_{\bullet} \expr q$ and $R_{\bullet} \expr r$ be reduced expressions such that $P_{\bullet} \circ Q_{\bullet} \circ R_{\bullet} \expr p.q.r$ is reduced. Then
\begin{equation} \id_{P_{\bullet}} \ot \End_{< q}(Q_{\bullet}) \ot \id_{R_{\bullet}} \subset \End_{< p.q.r}(P_{\bullet} \circ Q_{\bullet} \circ R_{\bullet}). \end{equation} \end{prop}

The proposition above states that the concept of ``lower terms'' is preserved by tensor product with identity maps, but only under reduced compositions of reduced expressions. 

\begin{proof} Let $\phi \in \End_{<q}(Q_{\bullet})$. There exists some reduced expression $Q'_{\bullet} \expr q'$ with $q' < q$, such that $\phi$ factors through $Q'_{\bullet}$ (i.e. through $\BS(Q'_{\bullet})$). Then $\id_{P_{\bullet}} \ot \phi \ot \id_{R_{\bullet}}$ factors through $P_{\bullet} \circ Q'_{\bullet} \circ R_{\bullet}$. This need not be a reduced expression, but it is an expression for the double coset $p*q'*r$. By Theorem~\ref{thm:bruhatconcatcompat}, $p*q'*r < p.q.r$. By Corollary~\ref{cor:factorthroughnonreduced}, $\id_{P_{\bullet}} \ot \phi \ot \id_{R_{\bullet}}$ is a finite sum of morphisms factoring through  Bott-Samelsons expressing
cosets strictly smaller than $p.q.r$, whence it lives in $\End_{<p.q.r}(P_{\bullet} \circ Q_{\bullet} \circ R_{\bullet})$. \end{proof}
\printbibliography
\end{document}



