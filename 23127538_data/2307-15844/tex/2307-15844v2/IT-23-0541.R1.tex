
%% bare_jrnl.tex
%% V1.4b
%% 2015/08/26
%% by Michael Shell
%% see http://www.michaelshell.org/
%% for current contact information.
%%
%% This is a skeleton file demonstrating the use of IEEEtran.cls
%% (requires IEEEtran.cls version 1.8b or later) with an IEEE
%% journal paper.
%%
%% Support sites:
%% http://www.michaelshell.org/tex/ieeetran/
%% http://www.ctan.org/pkg/ieeetran
%% and
%% http://www.ieee.org/

%%*************************************************************************
%% Legal Notice:
%% This code is offered as-is without any warranty either expressed or
%% implied; without even the implied warranty of MERCHANTABILITY or
%% FITNESS FOR A PARTICULAR PURPOSE! 
%% User assumes all risk.
%% In no event shall the IEEE or any contributor to this code be liable for
%% any damages or losses, including, but not limited to, incidental,
%% consequential, or any other damages, resulting from the use or misuse
%% of any information contained here.
%%
%% All comments are the opinions of their respective authors and are not
%% necessarily endorsed by the IEEE.
%%
%% This work is distributed under the LaTeX Project Public License (LPPL)
%% ( http://www.latex-project.org/ ) version 1.3, and may be freely used,
%% distributed and modified. A copy of the LPPL, version 1.3, is included
%% in the base LaTeX documentation of all distributions of LaTeX released
%% 2003/12/01 or later.
%% Retain all contribution notices and credits.
%% ** Modified files should be clearly indicated as such, including  **
%% ** renaming them and changing author support contact information. **
%%*************************************************************************


% *** Authors should verify (and, if needed, correct) their LaTeX system  ***
% *** with the testflow diagnostic prior to trusting their LaTeX platform ***
% *** with production work. The IEEE's font choices and paper sizes can   ***
% *** trigger bugs that do not appear when using other class files.       ***                          ***
% The testflow support page is at:
% http://www.michaelshell.org/tex/testflow/



\documentclass[journal]{IEEEtran}
%
% If IEEEtran.cls has not been installed into the LaTeX system files,
% manually specify the path to it like:
% \documentclass[journal]{../sty/IEEEtran}





% Some very useful LaTeX packages include:
% (uncomment the ones you want to load)


% *** MISC UTILITY PACKAGES ***
%
%\usepackage{ifpdf}
% Heiko Oberdiek's ifpdf.sty is very useful if you need conditional
% compilation based on whether the output is pdf or dvi.
% usage:
% \ifpdf
%   % pdf code
% \else
%   % dvi code
% \fi
% The latest version of ifpdf.sty can be obtained from:
% http://www.ctan.org/pkg/ifpdf
% Also, note that IEEEtran.cls V1.7 and later provides a builtin
% \ifCLASSINFOpdf conditional that works the same way.
% When switching from latex to pdflatex and vice-versa, the compiler may
% have to be run twice to clear warning/error messages.






% *** CITATION PACKAGES ***
%
\usepackage{cite}
% cite.sty was written by Donald Arseneau
% V1.6 and later of IEEEtran pre-defines the format of the cite.sty package
% \cite{} output to follow that of the IEEE. Loading the cite package will
% result in citation numbers being automatically sorted and properly
% "compressed/ranged". e.g., [1], [9], [2], [7], [5], [6] without using
% cite.sty will become [1], [2], [5]--[7], [9] using cite.sty. cite.sty's
% \cite will automatically add leading space, if needed. Use cite.sty's
% noadjust option (cite.sty V3.8 and later) if you want to turn this off
% such as if a citation ever needs to be enclosed in parenthesis.
% cite.sty is already installed on most LaTeX systems. Be sure and use
% version 5.0 (2009-03-20) and later if using hyperref.sty.
% The latest version can be obtained at:
% http://www.ctan.org/pkg/cite
% The documentation is contained in the cite.sty file itself.



% *** GRAPHICS RELATED PACKAGES ***
%
\ifCLASSINFOpdf
  \usepackage[pdftex]{graphicx}
  % declare the path(s) where your graphic files are
  % \graphicspath{{../pdf/}{../jpeg/}}
  % and their extensions so you won't have to specify these with
  % every instance of \includegraphics
  % \DeclareGraphicsExtensions{.pdf,.jpeg,.png}
\else
  % or other class option (dvipsone, dvipdf, if not using dvips). graphicx
  % will default to the driver specified in the system graphics.cfg if no
  % driver is specified.
  % \usepackage[dvips]{graphicx}
  % declare the path(s) where your graphic files are
  % \graphicspath{{../eps/}}
  % and their extensions so you won't have to specify these with
  % every instance of \includegraphics
  % \DeclareGraphicsExtensions{.eps}
\fi
% graphicx was written by David Carlisle and Sebastian Rahtz. It is
% required if you want graphics, photos, etc. graphicx.sty is already
% installed on most LaTeX systems. The latest version and documentation
% can be obtained at: 
% http://www.ctan.org/pkg/graphicx
% Another good source of documentation is "Using Imported Graphics in
% LaTeX2e" by Keith Reckdahl which can be found at:
% http://www.ctan.org/pkg/epslatex
%
% latex, and pdflatex in dvi mode, support graphics in encapsulated
% postscript (.eps) format. pdflatex in pdf mode supports graphics
% in .pdf, .jpeg, .png and .mps (metapost) formats. Users should ensure
% that all non-photo figures use a vector format (.eps, .pdf, .mps) and
% not a bitmapped formats (.jpeg, .png). The IEEE frowns on bitmapped formats
% which can result in "jaggedy"/blurry rendering of lines and letters as
% well as large increases in file sizes.
%
% You can find documentation about the pdfTeX application at:
% http://www.tug.org/applications/pdftex





% *** MATH PACKAGES ***
%
\usepackage{amsmath,amsthm,amssymb}
\usepackage{mathtools}
% A popular package from the American Mathematical Society that provides
% many useful and powerful commands for dealing with mathematics.
%
% Note that the amsmath package sets \interdisplaylinepenalty to 10000
% thus preventing page breaks from occurring within multiline equations. Use:
\interdisplaylinepenalty=2500
% after loading amsmath to restore such page breaks as IEEEtran.cls normally
% does. amsmath.sty is already installed on most LaTeX systems. The latest
% version and documentation can be obtained at:
% http://www.ctan.org/pkg/amsmath

\usepackage{cleveref}

% cleveref does not refer to equations as `eqn.'
\crefname{equation}{}{} 

\newtheorem{theorem}{Theorem}
\newtheorem{lemma}[theorem]{Lemma}
\newtheorem{corollary}[theorem]{Corollary}
\newtheorem{claim}[theorem]{Claim}
\newtheorem{proposition}[theorem]{Proposition}
\newtheorem{definition}{Definition}


\theoremstyle{remark}
\newtheorem{remark}{Remark}

\newtheorem{example}{Example}


% Parentheses
\newcommand{\bc}[1]{\left\{{#1}\right\}}
\newcommand{\br}[1]{\left({#1}\right)}
\newcommand{\bs}[1]{\left[{#1}\right]}
\newcommand{\abs}[1]{\left| {#1} \right|}

% Expectations
\newcommand{\E}[1]{\mathbb{E}\bs{{#1}}}

% Miscellaneous
\newcommand{\cond}{\,|\,}
\newcommand{\bigcond}{\,\big|\,}

\DeclarePairedDelimiterX{\infdivx}[2]{(}{)}{  #1\;\delimsize\|\;#2}
\newcommand{\D}{D\infdivx}

% Conditional KL Divergence
\DeclarePairedDelimiterX{\cinfdivx}[3]{(}{)}{  #1\;\delimsize\|\;#2 \,\raisebox{-0.2ex}{\scalebox{1.5}{|}}\, #3}
\newcommand{\cDiv}{D\cinfdivx}

\newcommand{\mc}{-\!\!\circ\!\!-}

\DeclareMathOperator{\SI}{SI}
\DeclareMathOperator{\II}{I}
\DeclareMathOperator{\HH}{H}
\DeclareMathOperator{\WW}{W}
\DeclareMathOperator{\cII}{\mathcal{I}\,}

% \addbibresource{references.bib}



% *** SPECIALIZED LIST PACKAGES ***
%
%\usepackage{algorithmic}
% algorithmic.sty was written by Peter Williams and Rogerio Brito.
% This package provides an algorithmic environment fo describing algorithms.
% You can use the algorithmic environment in-text or within a figure
% environment to provide for a floating algorithm. Do NOT use the algorithm
% floating environment provided by algorithm.sty (by the same authors) or
% algorithm2e.sty (by Christophe Fiorio) as the IEEE does not use dedicated
% algorithm float types and packages that provide these will not provide
% correct IEEE style captions. The latest version and documentation of
% algorithmic.sty can be obtained at:
% http://www.ctan.org/pkg/algorithms
% Also of interest may be the (relatively newer and more customizable)
% algorithmicx.sty package by Szasz Janos:
% http://www.ctan.org/pkg/algorithmicx




% *** ALIGNMENT PACKAGES ***
%
%\usepackage{array}
% Frank Mittelbach's and David Carlisle's array.sty patches and improves
% the standard LaTeX2e array and tabular environments to provide better
% appearance and additional user controls. As the default LaTeX2e table
% generation code is lacking to the point of almost being broken with
% respect to the quality of the end results, all users are strongly
% advised to use an enhanced (at the very least that provided by array.sty)
% set of table tools. array.sty is already installed on most systems. The
% latest version and documentation can be obtained at:
% http://www.ctan.org/pkg/array


% IEEEtran contains the IEEEeqnarray family of commands that can be used to
% generate multiline equations as well as matrices, tables, etc., of high
% quality.




% *** SUBFIGURE PACKAGES ***
%\ifCLASSOPTIONcompsoc
%  \usepackage[caption=false,font=normalsize,labelfont=sf,textfont=sf]{subfig}
%\else
%  \usepackage[caption=false,font=footnotesize]{subfig}
%\fi
% subfig.sty, written by Steven Douglas Cochran, is the modern replacement
% for subfigure.sty, the latter of which is no longer maintained and is
% incompatible with some LaTeX packages including fixltx2e. However,
% subfig.sty requires and automatically loads Axel Sommerfeldt's caption.sty
% which will override IEEEtran.cls' handling of captions and this will result
% in non-IEEE style figure/table captions. To prevent this problem, be sure
% and invoke subfig.sty's "caption=false" package option (available since
% subfig.sty version 1.3, 2005/06/28) as this is will preserve IEEEtran.cls
% handling of captions.
% Note that the Computer Society format requires a larger sans serif font
% than the serif footnote size font used in traditional IEEE formatting
% and thus the need to invoke different subfig.sty package options depending
% on whether compsoc mode has been enabled.
%
% The latest version and documentation of subfig.sty can be obtained at:
% http://www.ctan.org/pkg/subfig




% *** FLOAT PACKAGES ***
%
%\usepackage{fixltx2e}
% fixltx2e, the successor to the earlier fix2col.sty, was written by
% Frank Mittelbach and David Carlisle. This package corrects a few problems
% in the LaTeX2e kernel, the most notable of which is that in current
% LaTeX2e releases, the ordering of single and double column floats is not
% guaranteed to be preserved. Thus, an unpatched LaTeX2e can allow a
% single column figure to be placed prior to an earlier double column
% figure.
% Be aware that LaTeX2e kernels dated 2015 and later have fixltx2e.sty's
% corrections already built into the system in which case a warning will
% be issued if an attempt is made to load fixltx2e.sty as it is no longer
% needed.
% The latest version and documentation can be found at:
% http://www.ctan.org/pkg/fixltx2e


%\usepackage{stfloats}
% stfloats.sty was written by Sigitas Tolusis. This package gives LaTeX2e
% the ability to do double column floats at the bottom of the page as well
% as the top. (e.g., "% Figure environment removed
%
% Note that often IEEE papers with subfigures do not employ subfigure
% captions (using the optional argument to \subfloat[]), but instead will
% reference/describe all of them (a), (b), etc., within the main caption.
% Be aware that for subfig.sty to generate the (a), (b), etc., subfigure
% labels, the optional argument to \subfloat must be present. If a
% subcaption is not desired, just leave its contents blank,
% e.g., \subfloat[].


% An example of a floating table. Note that, for IEEE style tables, the
% \caption command should come BEFORE the table and, given that table
% captions serve much like titles, are usually capitalized except for words
% such as a, an, and, as, at, but, by, for, in, nor, of, on, or, the, to
% and up, which are usually not capitalized unless they are the first or
% last word of the caption. Table text will default to \footnotesize as
% the IEEE normally uses this smaller font for tables.
% The \label must come after \caption as always.
%
%\begin{table}[!t]
%% increase table row spacing, adjust to taste
%\renewcommand{\arraystretch}{1.3}
% if using array.sty, it might be a good idea to tweak the value of
% \extrarowheight as needed to properly center the text within the cells
%\caption{An Example of a Table}
%\label{table_example}
%\centering
%% Some packages, such as MDW tools, offer better commands for making tables
%% than the plain LaTeX2e tabular which is used here.
%\begin{tabular}{|c||c|}
%\hline
%One & Two\\
%\hline
%Three & Four\\
%\hline
%\end{tabular}
%\end{table}


% Note that the IEEE does not put floats in the very first column
% - or typically anywhere on the first page for that matter. Also,
% in-text middle ("here") positioning is typically not used, but it
% is allowed and encouraged for Computer Society conferences (but
% not Computer Society journals). Most IEEE journals/conferences use
% top floats exclusively. 
% Note that, LaTeX2e, unlike IEEE journals/conferences, places
% footnotes above bottom floats. This can be corrected via the
% \fnbelowfloat command of the stfloats package.




% \section{Conclusion}
% The conclusion goes here.





% if have a single appendix:
%\appendix[Proof of the Zonklar Equations]
% or
%\appendix  % for no appendix heading
% do not use \section anymore after \appendix, only \section*
% is possibly needed

% use appendices with more than one appendix
% then use \section to start each appendix
% you must declare a \section before using any
% \subsection or using \label (\appendices by itself
% starts a section numbered zero.)
%


\appendices

\section{Proof of \Cref{lem:mrf-mi}}\label{app:1}
    \noindent We have
    \begin{align}
        &\II(X_{\mathcal{B}(i \leftarrow j)} \wedge X_{\mathcal{B}(j \leftarrow i)}) \nonumber \\
        &= \II(X_i \wedge X_j) + \II(X_i \wedge X_{\mathcal{B}(j \leftarrow i) \setminus \bc{j}} \cond X_j)\nonumber \\
        &\qquad\qquad+ \II(X_{\mathcal{B}(i \leftarrow j) \setminus \bc{i}} \wedge X_j \cond X_i)\nonumber\\
        &\qquad\qquad\qquad+ \II(X_{\mathcal{B}(i \leftarrow j) \setminus \bc{i}} \wedge X_{\mathcal{B}(j \leftarrow i) \setminus \bc{j}} \cond X_i, X_j) \nonumber \\ 
        &= \II(X_i \wedge X_j) + \II(X_{\mathcal{B}(i \leftarrow j) \setminus \bc{i}} \wedge X_{\mathcal{B}(j \leftarrow i) \setminus \bc{j}} \cond X_i, X_j)\nonumber\\
        &= \II(X_i \wedge X_j) + \HH(X_{\mathcal{B}(i \leftarrow j) \setminus \bc{i}} \cond X_i)\nonumber \\
        &\qquad\qquad\qquad\qquad- \HH(X_{\mathcal{B}(i \leftarrow j) \setminus \bc{i}} \cond X_i, X_{\mathcal{B}(j \leftarrow i)})\label{eq:mct-proof}
    \end{align}
    where the previous two inequalities are by \Cref{eq:mct-mc-def}.

    The claim of \Cref{lem:mrf-mi} would follow from \Cref{eq:mct-proof} upon showing that 
    \begin{align}
        \HH(X_{\mathcal{B}(i \leftarrow j) \setminus \bc{i}} \cond X_i, X_{\mathcal{B}(j \leftarrow i)}) = \HH(X_{\mathcal{B}(i \leftarrow j) \setminus \bc{i}} \cond X_i) \label{eq:mct-mi-proof-claim}. 
    \end{align}
    
    Without loss of generality, set $j$ to be the root of the tree; this defines a \emph{directed} tree whose leaves are from among the vertices (in $\mathcal{M}$) with no descendants. Denote the parent of $i'$ in the (directed) tree by $\parent(i')$. Note that $\parent(i) = j$ in \Cref{eq:mct-mi-proof-claim}. We shall use induction on the \emph{height} of $i'$, i.e., the maximum distance of $i'$ from a leaf of the directed tree, to show that 
    \begin{align}
        &\HH(X_{\mathcal{B}(i' \leftarrow \parent(i')) \setminus \bc{i'}} \cond X_{i'}, X_{\mathcal{B}(\parent(i') \leftarrow i')}) \nonumber\\
        &\qquad\qquad\qquad\qquad\qquad= \HH(X_{\mathcal{B}(i' \leftarrow \parent(i')) \setminus \{i'\}} \cond X_{i'}) \label{eq:mct-mi-proof-claim-2}, 
    \end{align}
    which proves \Cref{eq:mct-mi-proof-claim} upon setting $i'=i$ and $\parent(i') = \parent(i) = j$. 
    
    First, assume that $i'$ is a leaf. Then $\mathcal{B}(i' \leftarrow \parent(i')) \setminus \bc{i'} = \varnothing$ and \Cref{eq:mct-mi-proof-claim-2} holds trivially. 
    % Figure environment removed
    
    Next, assume the induction hypothesis that \Cref{eq:mct-mi-proof-claim-2} is true for all vertices at height $< h$, and consider a vertex $i'$ at height $h$. Let $i'$ have children $1, \ldots, t$; each of these vertices is the root of subtree $T_\tau = \mathcal{B}(\tau \leftarrow i')$, $1 \leq \tau \leq t$. See \Cref{fig:diag-proof-in-app-a}. Further, each vertex $\tau$, $1 \leq \tau \leq t$, has height $< h$. Then in \Cref{eq:mct-mi-proof-claim-2},
    \begin{align}
        &\HH(X_{\mathcal{B}(i' \leftarrow \parent(i')) \setminus \bc{i'}} \cond X_{i'}, X_{\mathcal{B}(\parent(i') \leftarrow i')})\nonumber\\
        &= \HH\br{(X_{T_\tau \setminus \bc{\tau}}, X_\tau)_{1 \leq \tau \leq t} \cond X_{i'}, X_{\mathcal{B}(\parent(i') \leftarrow i')}}\nonumber\\
        &= \sum_{\tau = 1}^t \left[\HH \br{X_\tau \cond \br{X_{T_\sigma}}_{1 \leq \sigma \leq \tau - 1}, X_{i'}, X_{\mathcal{B}(\parent(i') \leftarrow i')}}\right.\nonumber\\
        &\left. + \HH\br{X_{T_\tau \setminus \bc{\tau}} \cond X_\tau, \br{X_{T_\sigma}}_{1 \leq \sigma \leq \tau - 1}, X_{i'}, X_{\mathcal{B}(\parent(i') \leftarrow i')}}\right].\label{eq:3.3}
    \end{align}
    In \Cref{eq:3.3}, for each $\tau$, $1 \leq \tau \leq t$, the first term within $\bs{\cdot}$ is 
    \begin{align}
        &\HH \br{X_\tau \cond \br{X_{T_\sigma}}_{1 \leq \sigma \leq \tau - 1}, X_{i'}, X_{\mathcal{B}(\parent(i') \leftarrow i')}}\nonumber\\
        &\qquad = \HH(X_\tau \cond X_{i'}) = \HH \br{X_\tau \cond \br{X_{T_\sigma}}_{1 \leq \sigma \leq \tau - 1}, X_{i'}} \label{eq:app-1-proof-1}
    \end{align}
    by \Cref{eq:mct-mc-def} since
    \begin{align*}
        \br{\bigcup_{\sigma = 1}^{\tau - 1} T_{\sigma}, \mathcal{B}(\parent(i') \leftarrow i')} \subseteq \mathcal{B}(i' \leftarrow \tau)
    \end{align*}
    (see \Cref{fig:diag-proof-in-app-a}). In the second term in $\bs{\cdot}$, we apply the induction hypothesis to vertex $\tau$ which is at height $h-1$. Note that $\parent(\tau) = i'$. Since
    \begin{align*}
        &X_{T_\tau \setminus \bc{\tau}} = X_{\mathcal{B}(\tau \leftarrow \parent(\tau)) \setminus \bc{\tau}}\\
        &\quad\text{and} \br{\br{X_{T_\sigma}}_{1 \leq \sigma \leq \tau - 1}, X_{i'}, X_{\mathcal{B}(\parent(i') \leftarrow i')}} \subseteq \mathcal{B}(\parent(\tau) \leftarrow \tau),
    \end{align*}
    by the induction hypothesis at vertex $\tau$, we get
    \begin{align}
        &\HH\br{X_{T_\tau \setminus \bc{\tau}} \cond X_\tau, \br{X_{T_\sigma}}_{1 \leq \sigma \leq \tau - 1}, X_{i'}, X_{\mathcal{B}(\parent(i') \leftarrow i')}}\nonumber\\
        &= \HH\br{X_{T_\tau \setminus \bc{\tau}} \cond X_\tau} \nonumber\\
        &= \HH\br{X_{T_\tau \setminus \bc{\tau}} \cond X_\tau, \br{X_{T_\sigma}}_{1 \leq \sigma \leq \tau - 1}, X_{i'}} \label{eq:app-1-proof-2}
    \end{align}
    with the last equality being due to \Cref{eq:mct-mc-def}. Substituting \Cref{eq:app-1-proof-1}, \Cref{eq:app-1-proof-2} in \Cref{eq:3.3}, we obtain
    \begin{align*}
        &\HH(X_{\mathcal{B}(i' \leftarrow \parent(i')) \setminus \bc{i'}} \cond X_{i'}, X_{\mathcal{B}(\parent(i') \leftarrow i')})\\
        &= \sum_{\tau = 1}^t \left[\HH \br{X_\tau \cond \br{X_{T_\sigma}}_{1 \leq \sigma \leq \tau - 1}, X_{i'}}\right.\\
        &\left.\qquad\qquad\qquad + \HH\br{X_{T_\tau \setminus \bc{\tau}} \cond X_\tau, \br{X_{T_\sigma}}_{1 \leq \sigma \leq \tau - 1}, X_{i'}}\right]\\
        &= \sum_{\tau=1}^t \HH\br{X_{T_\tau} \cond \br{X_{T_\sigma}}_{1 \leq \sigma \leq \tau - 1}, X_{i'}}\\
        &= \HH(X_{\mathcal{B}(i' \leftarrow \parent(i')) \setminus \{i'\}} \cond X_{i'})
    \end{align*}
    (see \Cref{fig:diag-proof-in-app-a}) which is \Cref{eq:mct-mi-proof-claim-2}. \hfill \qed

\section{Proof of \Cref{lem:mct-local-prop} and \Cref{th:g-g}}\label{app:2}
\begin{proof}[Proof of \Cref{lem:mct-local-prop}]
    Considering first \Cref{eq:mct-local-prop-1}, suppose that vertex $i \in \mathcal{M}$ has $k$ neighbor, with $\mathcal{N}(i) = \bc{i_1, \ldots, i_k}$, $1 \leq k \leq m-1$. Then
    \begin{align*}
        \mathcal{M} \setminus (\bc{i} \cup \mathcal{N}(i)) = \bigcup_{l=1}^k \mathcal{B}(i_l \leftarrow i) \setminus \bc{i_l}.
    \end{align*}
    The claim of the lemma is 
    \begin{align}
        X_i \mc \br{X_{i_u}}_{1 \leq u \leq k} \mc \br{X_{\mathcal{B}(i_l \leftarrow i) \setminus \bc{i_l}}}_{1 \leq l \leq k}. \label{eq:mct-local-prop-proof-1}
    \end{align}
    We have 
    \begin{align}
        &\II\br{X_i \wedge \br{X_{\mathcal{B}(i_l \leftarrow i) \setminus \bc{i_l}}}_{1 \leq l \leq k} \bigcond \br{X_{i_u}}_{1 \leq u \leq k}} \nonumber\\
        &= \sum_{l=1}^k \II\left(X_i \wedge X_{\mathcal{B}(i_l \leftarrow i) \setminus \bc{i_l}} \bigcond \br{X_{\mathcal{B}(i_j \leftarrow i) \setminus \bc{i_j}}}_{1 \leq j \leq l-1},\right. \nonumber \\ 
        &\qquad\qquad\qquad\qquad\qquad\qquad\qquad\qquad\qquad\left.\br{X_{i_u}}_{1 \leq u \leq k}\right)\nonumber\\
        &\leq \sum_{l=1}^k \II\Big(\bs{X_i, \br{X_{\mathcal{B}(i_j \leftarrow i) \setminus \bc{i_j}}}_{1 \leq j \leq l-1}, \br{X_{i_u}}_{1 \leq u \neq l \leq k}} \nonumber \\ 
        &\qquad\qquad\qquad\qquad\qquad\qquad\wedge X_{\mathcal{B}(i_l \leftarrow i) \setminus \bc{i_l}} \bigcond X_{i_l}\Big). \label{eq:mct-local-prop-proof-2}
    \end{align}
    For each $l$, $1 \leq l \leq k$, the rvs within $\bs{\cdot}$ above have indices that lie in $\mathcal{B}(i \leftarrow i_l) \setminus \bc{i_l}$. Hence, by \Cref{lem:mrf-mi} (specifically \Cref{eq:mrf-mi-2b}), each term in the sum in \Cref{eq:mct-local-prop-proof-2} equals zero. This proves \Cref{eq:mct-local-prop-proof-1}. See \Cref{diag:local_prop_diag}.

    % Figure environment removed

    Turning to \Cref{eq:mct-local-prop-2}, we have
    \begin{align*}
        &\II\br{X_A \wedge X_{\mathcal{M} \setminus (A \cup \mathcal{N}(A))} \cond \mathcal{N}(A)} \\
        &= \II\br{(X_i, i \in A) \wedge X_{\mathcal{M} \setminus \bigcup_{u \in A} \br{\bc{u} \cup \mathcal{N}(u)}} \bigcond X_{\bigcup_{v \in A} \mathcal{N}(v)}}\\
        &\leq \sum_{i \in A} \II\left(X_i \wedge X_{\bigcup_{j \in A \setminus \bc{i}} (\bc{j} \cup \mathcal{N}(j))},\right.\\  
        &\qquad\qquad\qquad\qquad\qquad\qquad\left.X_{\mathcal{M} \setminus \bigcup_{u \in A} \br{\bc{u} \cup \mathcal{N}(u)}} \bigcond X_{\mathcal{N}(i)}\right)\\
        &= \sum_{i \in A} \II\left(X_i \wedge X_{\br{\bigcup_{j \in A \setminus \bc{i}} (\bc{j} \cup \mathcal{N}(j))}\setminus \mathcal{N}(i)},\right.\\ 
        &\qquad\qquad\qquad\qquad\qquad\qquad\left.X_{\mathcal{M} \setminus \bigcup_{u \in A} \br{\bc{u} \cup \mathcal{N}(u)}} \bigcond X_{\mathcal{N}(i)}\right)\\
        &= 0
    \end{align*}
    by \Cref{eq:mct-local-prop-1} since for each $i \in A$,
    \begin{align*}
        &\br{\br{\bigcup_{j \in A \setminus \bc{i}} (\bc{j} \cup \mathcal{N}(j))}\setminus \mathcal{N}(i)} \\
        &\qquad\cup \br{\mathcal{M} \setminus \bigcup_{u \in A} \br{\bc{u} \cup \mathcal{N}(u)}} \subseteq \mathcal{M} \setminus (\bc{i} \cup \mathcal{N}(i)). \qedhere
    \end{align*}
\end{proof}
\begin{proof}[Proof of \Cref{th:g-g}]
    The converse claim is immediately true upon choosing: for every $(i,j) \in \mathcal{E}$, $A = \mathcal{B}(i \leftarrow j) \setminus \bc{i}$, $S = {i}$, $B = \bc{j}$.

    Turning to the first claim, let 
    \begin{align*}
        A = \bigsqcup_{\alpha = 1}^a A_\alpha, \quad B = \bigsqcup_{\beta = 1}^b B_\beta, \quad S = \bigsqcup_{\sigma = 1}^s S_\sigma
    \end{align*}
    be representations in terms of maximally connected subsets of $A$, $B$ and $S$, respectively. With $N = \mathcal{M} \setminus (A \cup B \cup S)$, let $N = \sqcup_{\nu = 1}^n N_\nu$ be a decomposition into maximally connected subsets of $N$. Denote 
    \begin{align*}
        \mathcal{A} = \bc{A_\alpha, 1 \leq \alpha \leq a}, \quad \mathcal{B} = \bc{B_\beta, 1 \leq \beta \leq b},\\
        \quad \mathcal{S} = \bc{S_\sigma, 1 \leq \sigma \leq s}, \quad \mathcal{N} = \bc{N_\nu, 1 \leq \nu \leq n}.
    \end{align*}
    Referring to \Cref{def:agglomerated-tree} and recalling \Cref{lem:agglom-mct}, the tree $\mathcal{G}' = (\mathcal{M}', \mathcal{E}')$ with vertex set $\mathcal{M}' = \mathcal{A} \cup \mathcal{B} \cup \mathcal{S} \cup \mathcal{N}$ and edge set in the manner of \Cref{def:agglomerated-tree} constitutes an agglomerated MCT. 

    Next, we observe that since each $N_\nu \in \mathcal{N}$, $1 \leq \nu \leq n$, is maximally connected in N, the neighbors of $N_\nu$ in $\mathcal{G}'$ cannot be in $\mathcal{N}$. Therefore, neighbors of a given $N_\nu$ in $\mathcal{G}'$ that are not in $\mathcal{S}$ must be in $\mathcal{A}$ or $\mathcal{B}$. However, $N_\nu$ cannot have a non$\mathcal{S}$ neighbor in $\mathcal{A}$ and also one in $\mathcal{B}$, for then $A$ and $B$ would not be separated by $S$ in $\mathcal{G}$. Accordingly, for \emph{each} $N_\nu$ in $\mathcal{N}$, if its non$\mathcal{S}$ neighbors in $\mathcal{G}'$ are only in $\mathcal{A}$, add $N_\nu$ to $\mathcal{A}$; let $N'$ be the union of all such $N_\nu$s. Consider $A' = A \cup N'$ and write $A' = \sqcup_{\alpha=1}^{a'} A_\alpha'$ where the $A_\alpha'$s are maximally connected subsets of $A'$. Let $\mathcal{A}' = \bc{A_\alpha', 1\leq \alpha \leq a'}$.

    Now note that $\mathcal{A}'$ and $\mathcal{B}$ are separated in $\mathcal{G}'$ by $\mathcal{S}$. Thus, to establish \Cref{eq:g-g}, it suffices to show the (stronger) assertion 
    \begin{align}
        X_{\mathcal{A}'} \mc X_\mathcal{S} \mc X_\mathcal{B}. \label{eq:global-proof-1}
    \end{align}
    By the description of $\mathcal{A}'$, each of its components (maximal subsets of $A'$) has its neighborhood in $\mathcal{G}'$ that is contained \emph{fully} in $\mathcal{S}$. Let $\tilde{\mathcal{S}} \subseteq \mathcal{S}$ denote the union of all such neighborhoods. Then, by \Cref{lem:mct-local-prop} (\Cref{eq:mct-local-prop-2}) applied to the agglomerated tree $\mathcal{G}'$, since there is no edge in $\mathcal{G}'$ that connects any two elements of $\mathcal{A}'$,
    \begin{align*}
        X_{\mathcal{A}'} \mc X_{\tilde{\mathcal{S}}} \mc X_{\mathcal{M}' \setminus (\mathcal{A}' \cup \tilde{\mathcal{S}})}
    \end{align*}
    so that
    \begin{align}
        0 &= \II(X_{\mathcal{A}'} \wedge X_{\mathcal{M}' \setminus (\mathcal{A}' \cup \tilde{\mathcal{S}})} \cond X_{\tilde{\mathcal{S}}}) \nonumber\\
        &= \II(X_{\mathcal{A}'} \wedge X_{\mathcal{M}' \setminus (\mathcal{A}' \cup \tilde{\mathcal{S}})}, X_{\mathcal{S}\setminus\tilde{\mathcal{S}}} \cond X_{\tilde{\mathcal{S}}}) \nonumber\\
        &\geq \II(X_{\mathcal{A}'} \wedge X_{\mathcal{M}' \setminus (\mathcal{A}' \cup \mathcal{S})} \cond X_\mathcal{S}) \label{eq:global-proof-2}
    \end{align}
    since $\tilde{\mathcal{S}} \subseteq \mathcal{S}$. Finally, \Cref{eq:global-proof-2} implies \Cref{eq:global-proof-1} as $\mathcal{B} \subseteq \mathcal{M}'\setminus(\mathcal{A}' \cup \mathcal{S})$.
\end{proof}
\section{Proof of \Cref{lem:tech-1}} \label{app:3}
\begin{proof}
    For $1 \leq c \leq 1.2$, $x \geq c \ln^2 x$ holds unconditionally; so assume that $c \geq 1.2$. Consider the function $f(x) = x - c \ln^2 x$. Then, using \cite[Lemma A.1]{sss-ml-book}, $x \geq 4c \ln 2c$ implies $x \geq 2c \ln x$ which, in turn, implies $f'(x) \geq 0$. Therefore, for $x \geq 4c \ln c$, $f(x)$ is increasing in $x$. It is easy to check numerically that $f(16c \ln^2 c)$ is positive for $c \geq 1.2$. Thus, for all $x \geq \max\{4c \ln 2c, 16c \ln^2 c\}$, $f(x) \geq 0$ and so $x \geq c \ln^2 x$. 
\end{proof}
% you can choose not to have a title for an appendix
% if you want by leaving the argument blank
% \section{}
% Appendix two text goes here.


% use section* for acknowledgment
% \section*{Acknowledgment}


% The authors would like to thank...


% Can use something like this to put references on a page
% by themselves when using endfloat and the captionsoff option.
\ifCLASSOPTIONcaptionsoff
  \newpage
\fi



% trigger a \newpage just before the given reference
% number - used to balance the columns on the last page
% adjust value as needed - may need to be readjusted if
% the document is modified later
% \IEEEtriggeratref{42}
% The "triggered" command can be changed if desired:
%\IEEEtriggercmd{\enlargethispage{-5in}}

% references section

% can use a bibliography generated by BibTeX as a .bbl file
% BibTeX documentation can be easily obtained at:
% http://mirror.ctan.org/biblio/bibtex/contrib/doc/
% The IEEEtran BibTeX style support page is at:
% http://www.michaelshell.org/tex/ieeetran/bibtex/
% \bibliographystyle{IEEEtranS}
% argument is your BibTeX string definitions and bibliography database(s)
% \bibliography{IEEEabrv,references}
%
% <OR> manually copy in the resultant .bbl file
% set second argument of \begin to the number of references
% (used to reserve space for the reference number labels box)

% \printbibliography

% Generated by IEEEtranS.bst, version: 1.12 (2007/01/11)
\begin{thebibliography}{10}
  \providecommand{\url}[1]{#1}
  \csname url@samestyle\endcsname
  \providecommand{\newblock}{\relax}
  \providecommand{\bibinfo}[2]{#2}
  \providecommand{\BIBentrySTDinterwordspacing}{\spaceskip=0pt\relax}
  \providecommand{\BIBentryALTinterwordstretchfactor}{4}
  \providecommand{\BIBentryALTinterwordspacing}{\spaceskip=\fontdimen2\font plus
  \BIBentryALTinterwordstretchfactor\fontdimen3\font minus
    \fontdimen4\font\relax}
  \providecommand{\BIBforeignlanguage}[2]{{%
  \expandafter\ifx\csname l@#1\endcsname\relax
  \typeout{** WARNING: IEEEtranS.bst: No hyphenation pattern has been}%
  \typeout{** loaded for the language `#1'. Using the pattern for}%
  \typeout{** the default language instead.}%
  \else
  \language=\csname l@#1\endcsname
  \fi
  #2}}
  \providecommand{\BIBdecl}{\relax}
  \BIBdecl
  
  \bibitem{Abdallah2010AMO}
  S.~A. Abdallah and M.~D. Plumbley, ``A measure of statistical complexity based
    on predictive information,'' \emph{ArXiv}, vol. abs/1012.1890, 2010.
  
  \bibitem{kontoyiannis}
  \BIBentryALTinterwordspacing
  A.~Antos and I.~Kontoyiannis, ``{Convergence properties of functional estimates
    for discrete distributions},'' \emph{Random Structures \& Algorithms},
    vol.~19, no. 3‐4, pp. 163--193, 2001.
  \BIBentrySTDinterwordspacing
  
  \bibitem{sb-pn-si-mct}
  S.~Bhattacharya and P.~Narayan, ``Shared information for a {M}arkov chain on a
    tree,'' in \emph{2022 IEEE International Symposium on Information Theory},
    2022, pp. 3049--3054.
  
  \bibitem{sb-pn-isit-2023}
  ------, ``Shared information for the cliqueylon graph,'' in \emph{2023 IEEE
    International Symposium on Information Theory (ISIT)}.\hskip 1em plus 0.5em
    minus 0.4em\relax IEEE, Jun. 2023.
  
  \bibitem{boda-narayan-universal-sampling-rate-distortion}
  \BIBentryALTinterwordspacing
  V.~P. Boda and P.~Narayan, ``Universal sampling rate distortion,'' \emph{IEEE
    Transactions on Information Theory}, vol.~64, no.~12, pp. 7742--7758, Dec.
    2018.
  \BIBentrySTDinterwordspacing
  
  \bibitem{boda-prashanth-correlated-bandits}
  \BIBentryALTinterwordspacing
  V.~P. Boda and L.~A. Prashanth, ``Correlated bandits or: {H}ow to minimize
    mean-squared error online,'' in \emph{Proceedings of the 36th International
    Conference on Machine Learning}, vol.~97.\hskip 1em plus 0.5em minus
    0.4em\relax PMLR, 6 2019, pp. 686--694.
  \BIBentrySTDinterwordspacing
  
  \bibitem{cesa-bianchi_lugosi_2006}
  N.~Cesa-Bianchi and G.~Lugosi, \emph{Prediction, Learning, and Games}.\hskip
    1em plus 0.5em minus 0.4em\relax Cambridge University Press, 2006.
  
  \bibitem{chan-tightness}
  C.~Chan, ``On tightness of mutual dependence upperbound for secret-key capacity
    of multiple terminals,'' \emph{ArXiv}, vol. abs/0805.3200, 2008.
  
  \bibitem{chan-hidden-flow}
  ------, ``The hidden flow of information,'' \emph{2011 IEEE International
    Symposium on Information Theory Proceedings}, pp. 978--982, 2011.
  
  \bibitem{chan-linear-perfect}
  ------, ``Linear perfect secret key agreement,'' in \emph{2011 IEEE Information
    Theory Workshop}, 2011, pp. 723--726.
  
  \bibitem{chan-shared-information}
  C.~Chan, A.~Al-Bashabsheh, J.~B. Ebrahimi, T.~Kaced, and T.~Liu, ``Multivariate
    mutual information inspired by secret-key agreement,'' \emph{Proceedings of
    the IEEE}, vol. 103, no.~10, pp. 1883--1913, 2015.
  
  \bibitem{chan-clustering}
  C.~Chan, A.~Al-Bashabsheh, and Q.~Zhou, ``Agglomerative info-clustering:
    Maximizing normalized total correlation,'' \emph{IEEE Transactions on
    Information Theory}, vol.~67, no.~3, pp. 2001--2011, 2021.
  
  \bibitem{chan-successive}
  C.~Chan, A.~Al-Bashabsheh, Q.~Zhou, N.~Ding, T.~Liu, and A.~Sprintson,
    ``Successive omniscience,'' \emph{IEEE Transactions on Information Theory},
    vol.~62, no.~6, pp. 3270--3289, 2016.
  
  \bibitem{chan-info-clustering}
  C.~Chan, A.~Al-Bashabsheh, Q.~Zhou, T.~Kaced, and T.~Liu, ``Info-clustering: A
    mathematical theory for data clustering,'' \emph{IEEE Transactions on
    Molecular, Biological and Multi-Scale Communications}, pp. 64--91, 2016.
  
  \bibitem{chan-mutual-dependence}
  C.~Chan and L.~Zheng, ``Mutual dependence for secret key agreement,'' in
    \emph{2010 44th Annual Conference on Information Sciences and Systems
    (CISS)}, 2010, pp. 1--6.
  
  \bibitem{chow-liu-trees}
  C.~Chow and C.~Liu, ``Approximating discrete probability distributions with
    dependence trees,'' \emph{IEEE Transactions on Information Theory}, pp.
    462--467, 1968.
  
  \bibitem{chow-wagner}
  C.~Chow and T.~Wagner, ``Consistency of an estimate of tree-dependent
    probability distributions (corresp.),'' \emph{IEEE Transactions on
    Information Theory}, vol.~19, pp. 369--371, 1973.
  
  \bibitem{clrs}
  T.~H. Cormen, C.~E. Leiserson, R.~L. Rivest, and C.~Stein, \emph{Introduction
    to Algorithms, Third Edition}, 3rd~ed.\hskip 1em plus 0.5em minus 0.4em\relax
    The MIT Press, 2009.
  
  \bibitem{courtade-coded-cooperative}
  T.~A. Courtade and T.~R. Halford, ``Coded cooperative data exchange for a
    secret key,'' \emph{IEEE Transactions on Information Theory}, vol.~62, pp.
    3785--3795, 2016.
  
  \bibitem{csiszar2011information}
  \BIBentryALTinterwordspacing
  I.~Csisz{\'a}r and J.~K{\"o}rner, \emph{Information Theory: Coding Theorems for
    Discrete Memoryless Systems}.\hskip 1em plus 0.5em minus 0.4em\relax
    Cambridge University Press, 2011.
  \BIBentrySTDinterwordspacing
  
  \bibitem{csiszar-narayan-secrecy-capacities}
  \BIBentryALTinterwordspacing
  I.~Csisz{\'a}r and P.~Narayan, ``Secrecy capacities for multiple terminals,''
    \emph{IEEE Transactions on Information Theory}, vol.~50, no.~12, pp.
    3047--3061, Dec. 2004.
  \BIBentrySTDinterwordspacing
  
  \bibitem{feng-dynamic-sampling}
  W.~Feng, N.~K. Vishnoi, and Y.~Yin, ``Dynamic sampling from graphical models,''
    \emph{Proceedings of the 51st Annual ACM SIGACT Symposium on Theory of
    Computing}, 2019.
  
  \bibitem{gacs-common-information}
  P.~G\'acs and J.~K\"orner, ``Common information is far less than mutual
    information,'' \emph{Problems of Control and Information Theory}, vol.~2, 01
    1973.
  
  \bibitem{Georgii+2011}
  \BIBentryALTinterwordspacing
  H.-O. Georgii, \emph{Gibbs Measures and Phase Transitions}.\hskip 1em plus
    0.5em minus 0.4em\relax De Gruyter, 2011.
  \BIBentrySTDinterwordspacing
  
  \bibitem{goppa-mi}
  V.~D. Goppa, ``Nonprobabilistic mutual information without memory,''
    \emph{Problems of Control and Information Theory}, vol.~4, pp. 97--102, 1975.
  
  \bibitem{han-nonnegative}
  \BIBentryALTinterwordspacing
  T.~S. Han, ``Nonnegative entropy measures of multivariate symmetric
    correlations,'' \emph{Information and Control}, vol.~36, no.~2, pp. 133--156,
    1978.
  \BIBentrySTDinterwordspacing
  
  \bibitem{jiao-minimax}
  J.~Jiao, K.~Venkat, Y.~Han, and T.~Weissman, ``Minimax estimation of
    functionals of discrete distributions,'' \emph{IEEE Transactions on
    Information Theory}, vol.~61, no.~5, pp. 2835--2885, 2015.
  
  \bibitem{koller-friedman-pgm}
  D.~Koller and N.~Friedman, \emph{Probabilistic Graphical Models: Principles and
    Techniques - Adaptive Computation and Machine Learning}.\hskip 1em plus 0.5em
    minus 0.4em\relax The MIT Press, 2009.
  
  \bibitem{koralov2007theory}
  \BIBentryALTinterwordspacing
  L.~Koralov, Y.~Sinai, and {\^A}.~Sinaj, \emph{Theory of Probability and Random
    Processes}, ser. Universitext (Berlin. Print).\hskip 1em plus 0.5em minus
    0.4em\relax Springer, 2007.
  \BIBentrySTDinterwordspacing
  
  \bibitem{szepesvari-bandits}
  \BIBentryALTinterwordspacing
  T.~Lattimore and C.~Szepesvari, ``Bandit algorithms,'' 2017.
  \BIBentrySTDinterwordspacing
  
  \bibitem{lauritzen1996}
  S.~L. Lauritzen, \emph{Graphical Models}.\hskip 1em plus 0.5em minus
    0.4em\relax Oxford University Press, 1996.
  
  \bibitem{liu-common-information}
  W.~Liu, G.~Xu, and B.~Chen, ``The common information of n dependent random
    variables,'' \emph{2010 48th Annual Allerton Conference on Communication,
    Control, and Computing (Allerton)}, pp. 836--843, 2010.
  
  \bibitem{mackay-information-theory}
  D.~J.~C. MacKay, \emph{Information Theory, Inference \& Learning
    Algorithms}.\hskip 1em plus 0.5em minus 0.4em\relax USA: Cambridge University
    Press, 2002.
  
  \bibitem{narayan-isit}
  P.~Narayan, ``Omniscience and secrecy,'' 2012, plenary Talk, \emph{IEEE
    International Symposium on Information Theory}, Cambridge, MA.
  
  \bibitem{tyagi-narayan-now}
  \BIBentryALTinterwordspacing
  P.~Narayan and H.~Tyagi, ``Multiterminal secrecy by public discussion,''
    \emph{Foundations and Trends in Communications and Information Theory},
    vol.~13, no. 2-3, pp. 129--275, 2016.
  \BIBentrySTDinterwordspacing
  
  \bibitem{nitinawarat-perfect}
  S.~Nitinawarat and P.~Narayan, ``Perfect omniscience, perfect secrecy, and
    {S}teiner tree packing,'' \emph{IEEE Transactions on Information Theory},
    vol.~56, pp. 6490--6500, 2010.
  
  \bibitem{nitinawarat-secret-key}
  S.~Nitinawarat, C.~Ye, A.~Barg, P.~Narayan, and A.~Reznik, ``Secret key
    generation for a pairwise independent network model,'' \emph{IEEE
    Transactions on Information Theory}, vol.~56, no.~12, p. 6482–6489, 12
    2010.
  
  \bibitem{paninski}
  \BIBentryALTinterwordspacing
  L.~Paninski, ``Estimation of entropy and mutual information,'' \emph{Neural
    Computation}, vol.~15, no.~6, p. 1191–1253, 6 2003.
  \BIBentrySTDinterwordspacing
  
  \bibitem{pearl-bayes}
  J.~Pearl, ``Reverend {B}ayes on inference engines: A distributed hierarchical
    approach,'' in \emph{Proceedings of the Second AAAI Conference on Artificial
    Intelligence}, ser. AAAI'82.\hskip 1em plus 0.5em minus 0.4em\relax AAAI
    Press, 1982, p. 133–136.
  
  \bibitem{sss-ml-book}
  S.~Shalev-Shwartz and S.~Ben-David, \emph{Understanding Machine Learning - From
    Theory to Algorithms.}\hskip 1em plus 0.5em minus 0.4em\relax Cambridge
    University Press, 2014.
  
  \bibitem{tyagi-common-information}
  H.~Tyagi, ``Common information and secret key capacity,'' \emph{IEEE
    Transactions on Information Theory}, vol.~59, pp. 5627--5640, 2013.
  
  \bibitem{tyagi-how-many-queries}
  \BIBentryALTinterwordspacing
  H.~Tyagi and P.~Narayan, ``How many queries will resolve common randomness?''
    \emph{{IEEE} Transactions on Information Theory}, vol.~59, no.~9, pp.
    5363--5378, 2013.
  \BIBentrySTDinterwordspacing
  
  \bibitem{tyagi-converses}
  H.~Tyagi and S.~Watanabe, ``Converses for secret key agreement and secure
    computing,'' \emph{IEEE Transactions on Information Theory}, vol.~61, pp.
    4809--4827, 2015.
  
  \bibitem{vershynin_2018}
  R.~Vershynin, \emph{High-Dimensional Probability: An Introduction with
    Applications in Data Science}, ser. Cambridge Series in Statistical and
    Probabilistic Mathematics.\hskip 1em plus 0.5em minus 0.4em\relax Cambridge
    University Press, 2018.
  
  \bibitem{watanabe-tc}
  S.~Watanabe, ``Information theoretical analysis of multivariate correlation,''
    \emph{IBM Journal of Research and Development}, vol.~4, no.~1, pp. 66--82,
    1960.
  
  \bibitem{wyner-common-information}
  A.~Wyner, ``The common information of two dependent random variables,''
    \emph{IEEE Transactions on Information Theory}, vol.~21, no.~2, pp. 163--179,
    1975.
  
  \bibitem{audibert-best-arm-identification}
  J.~yves Audibert, S.~Bubeck, and R.~Munos, ``Best arm identification in
    multi-armed bandits,'' in \emph{Proceedings of the Twenty-Third Annual
    Conference on Learning Theory}, 2010, pp. 41--53.
  
  \end{thebibliography}
    

  

% biography section
% 
% If you have an EPS/PDF photo (graphicx package needed) extra braces are
% needed around the contents of the optional argument to biography to prevent
% the LaTeX parser from getting confused when it sees the complicated
% \includegraphics command within an optional argument. (You could create
% your own custom macro containing the \includegraphics command to make things
% simpler here.)
%\begin{IEEEbiography}[{% Figure removed}]{Michael Shell}
% or if you just want to reserve a space for a photo:

% \begin{IEEEbiography}{Sagnik Bhattacharya}
% Biography text here.
% \end{IEEEbiography}

% if you will not have a photo at all:
\begin{IEEEbiographynophoto}{Sagnik Bhattacharya}
received the Bachelor of Technology in Electrical Engineering from the Indian Institute of Technology Kanpur, India, in 2019. He is currently a PhD candidate in the Department of Electrical and Computer Engineering at the University of Maryland, College Park. His research interests are in information theory, statistical learning, and their practical applications.
\end{IEEEbiographynophoto}

% insert where needed to balance the two columns on the last page with
% biographies
% \newpage

\begin{IEEEbiographynophoto}{Prakash Narayan}
  received the Bachelor of Technology degree in Electrical Engineering from the
  Indian Institute of Technology, Madras in $1976$.
  He received the M.S. degree in Systems Science and Mathematics in $1978$ and
  the D.Sc. degree in Electrical Engineering in $1981$, both from Washington
  University, St. Louis, MO.
  
  He is Professor of Electrical and Computer Engineering at the University
  of Maryland, College Park, with a joint appointment at the Institute for
  Systems Research. His research interests are in network information theory,
  coding theory, communication theory, communication networks, statistical learning, 
  and cryptography.
\end{IEEEbiographynophoto}

% You can push biographies down or up by placing
% a \vfill before or after them. The appropriate
% use of \vfill depends on what kind of text is
% on the last page and whether or not the columns
% are being equalized.

% \vfill

% Can be used to pull up biographies so that the bottom of the last one
% is flush with the other column.
% \enlargethispage{-5in}



% that's all folks
\end{document}


