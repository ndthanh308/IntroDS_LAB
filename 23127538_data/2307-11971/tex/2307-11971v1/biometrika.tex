%%% save the original kernel definitions
\let\latexarabic\arabic
\let\latexdocument\document
\let\latexenddocument\enddocument

%%% fix for bad usage of ntheorem
\RequirePackage[thmmarks]{ntheorem}

\makeatletter
\renewtheoremstyle{plain} 
  {\item[\hskip\labelsep \theorem@headerfont ##1\ \textup{##2}\theorem@separator]} 
  {\item[\hskip\labelsep \theorem@headerfont ##1\ \textup{##2}\ (##3)\theorem@separator]}
\makeatother

%
%\documentclass[manuscript]{biometrika}
\documentclass[article,lineno]{biometrika}

%%% recover the original definitions
\let\document\latexdocument
\let\enddocument\latexenddocument
\AtEndDocument{\printhistory}
\let\arabic\latexarabic
\def\rm{}

\usepackage{amsfonts, amsmath, amssymb, bbm}

%% Please use the following statements for
%% managing the text and math fonts for your papers:
\usepackage{times}
%\usepackage[cmbold]{mathtime}
\usepackage{bm}
%\usepackage{natbib}
\usepackage{moreverb,url}
\usepackage{multirow}
\usepackage{float}
\usepackage{tikz}
\usetikzlibrary{arrows,shapes.arrows,shapes.geometric,shapes.multipart, decorations.pathmorphing,positioning,shapes.swigs}

\usepackage{mathtools}
\mathtoolsset{showonlyrefs}
\graphicspath{{./Figures/}}
\usepackage{color}
\definecolor{darkred}{RGB}{100,0,0}
\definecolor{darkgreen}{RGB}{0,100,0}
\definecolor{darkblue}{RGB}{0,0,150}

\definecolor{purple}{rgb}{0.4,.1,.9}
\newcommand{\lily}[1]{\textcolor{magenta}{(Lily: #1)}}
\newcommand{\lilyx}[1]{\textcolor{magenta}{#1}}
\newcommand{\lilyd}[1]{\textcolor{purple}{(delete #1)}}
\newcommand{\andrew}[1]{\textcolor{blue}{(Andrew: #1)}}
\newcommand{\andrewx}[1]{\textcolor{purple}{#1}}


% links
%\usepackage{xr-hyper}
\usepackage{hyperref}
\hypersetup{colorlinks=true, linkcolor=darkred, citecolor=darkgreen, urlcolor=darkblue}
%\usepackage{url}

\usepackage[plain,noend]{algorithm2e}

\makeatletter
%\renewcommand{\AlTitleSty}[1]{\textit{#1}\unskip}% default definition
\renewcommand{\AlCapSty}[1]{{{#1}}\unskip}% default definition
\renewcommand{\algorithmcfname}{\textit{Algorithm}}%
\renewcommand{\algocf@captiontext}[2]{\quad #1\algocf@typo. \AlCapFnt{}#2} % text of caption
\renewcommand{\AlTitleFnt}[1]{#1\unskip}% default definition
\def\@algocf@capt@plain{top}
\renewcommand{\algocf@makecaption}[2]{%
  \addtolength{\hsize}{\algomargin}%
  \sbox\@tempboxa{\algocf@captiontext{#1}{#2}}%
  \ifdim\wd\@tempboxa >\hsize%     % if caption is longer than a line
    \hskip .5\algomargin%
    \parbox[t]{\hsize}{\algocf@captiontext{#1}{#2}}% then caption is not centered
  \else%
    \global\@minipagefalse%
    \hbox to\hsize{\box\@tempboxa}% else caption is centered
  \fi%
  \addtolength{\hsize}{-\algomargin}%
}
\makeatother

%%% User-defined macros should be placed here, but keep them to a minimum.
\def\Bka{{\it Biometrika}}
\def\AIC{\textsc{aic}}
\def\T{{ \mathrm{\scriptscriptstyle T} }}
\def\v{{\varepsilon}}
\newcommand{\E}{\operatorname{\mathbb{E}}}
\renewcommand{\P}{\operatorname{\mathbb{P}}}
\def\bbM{\mathbb{M}}
\addtolength\topmargin{35pt}

\begin{document}

\jname{Biometrika}
%% The year, volume, and number are determined on publication
\jyear{2021}
\jvol{103}
\jnum{1}
%% The \doi{...} and \accessdate commands are used by the production team
%\doi{10.1093/biomet/asm023}
\accessdate{Advance Access publication on 31 August 2022}

%% These dates are usually set by the production team
\received{{\rm 2} January {\rm 2017}}
\revised{{\rm 8} June {\rm 2021}}

%\received{January 2017}
%\revised{April 2017}

\markboth{Andrew Ying, Ronghui Xu}{Miscellanea}

%% Here are the title, author names and addresses
\title{On Defense of the Hazard Ratio}

\author{Andrew Ying}
\affil{\email{aying9339@gmail.com}}

\author{Ronghui Xu}
\affil{Herbert Wertheim School of Public Health, Department of Mathematics, and Halicioglu Data Science Institute, University of California San Diego,\\ La Jolla, California 92093, U.S.A.
	\email{rxu@ucsd.edu }}



%\input notation.tex

\maketitle
\begin{abstract}
There has been debate on whether the hazard function should be used for causal inference in time-to-event studies. The main criticism is that there is selection bias because the risk sets beyond the first event time are comprised of subsets of survivors who are no longer balanced in the risk factors, even in the absence of unmeasured confounding, measurement error, and model misspecification. In this short communication we use the potential outcomes framework and the single-world intervention graph to show that there is indeed no selection bias when estimating the average treatment effect, and that the hazard ratio over time can provide a useful interpretation in practical settings. 
\end{abstract}
\begin{keywords}
Causal Inference; Cox Model; Hazard Function; Potential Outcomes; Single-World Intervention Graph (SWIG); Survival Analysis. 
\end{keywords} 

\section{Background}

%\lily{relation between hazard and Cox}
Modeling the hazard ratios through the Cox model is perhaps one of the most celebrated and best adopted approaches for analyzing time-to-event data, partly due to its flexible semiparametric modeling form  \citep{reid:94}. It has been commonly used to analyze randomized clinical trials with survival endpoints. 
%\andrew{Add how reporting Cox model has been common practice for randomized trials and citation maybe?}
On causal interpretation of the hazard function, \citet{hernan2010hazards} first ``blew the whistle.'' 
%claiming there is danger of interpreting  causally. 
He warned that a time-fixed hazard ratio can be misleading and a time-varying one has no causal interpretation, mainly because there is selection bias caused by differential survival distributions between the groups unless the null hypothesis of no treatment effect holds. He claimed that the selection bias is present even in the absence of unmeasured confounding, measurement error, and model misspecification. He illustrated the selection bias via an example from the Women’s Health Initiative \citep{anderson2004effects, prentice2005statistical}.
%\lily{add ref}\andrew{I picked two big names to cite}. 
Following that, \citet{aalen2015does} formalized this selection bias as a  collider bias when conditioning on a risk set in the presence of heterogeneity.
%\lily{did they mention heterogeneity?} \andrew{See their section 4 on frailty}
More recently \citet{martinussen2020subtleties} attempted 
%\lily{in my opinion their subtlety is all lost in the published version which just repeats what had been said before}\andrew{Well...You may change the tone if you want}
to provide more insight into the subtle interpretation of hazard contrasts,  
as well as constructed an alternative estimand called the ``causal hazard ratio'' in order to deliver a causal interpretation.

\citet{prentice2022intention} presented a defense of hazard rate %\lily{rate or ratio?} 
modeling, and Cox regression in particular, mainly from the applications perspective, for the intent-to-treat reporting of causal effects in randomized controlled trials. They elaborated with a comprehensive survival analysis of the massive Women’s Health Initiative's randomized, placebo-controlled hormone replacement therapy trials. 

In this paper, we formalize a mathematical defense for the hazard ratios. We prove that there is indeed no selection bias and it is safe to interpret hazard ratios causally. We refute the previous claims point by point below. 
%We conjecture that literature \citep{hernan2010hazards, aalen2015does} has this misunderstanding mainly due to the noncollapsibility of the Cox model \citep{martinussen2013collapsibility}.



%Quote from the abstract of \citet{aalen2015does}: ``Despite the fact that treatment assignment is randomized, the hazard ratio is not a quantity which admits a causal interpretation in the case of unmodelled heterogeneity. This problem arises because the risk sets beyond the first event time are comprised of the subset of individuals who have not previously failed. The balance in the distribution of potential confounders between treatment arms is lost by this implicit conditioning, whether or not censoring is present. ''
%\subsection{Notation}

For the rest of this section we introduce some notation. Define $T$ as the time to event of interest, $A$  a randomized binary treatment assignment, and $L$  baseline covariates that can be partially or completely unobserved. We assume no censoring here in order to keep the focus on the hazard function itself. 
%and censoring only brings complication. 
Denote $Y(t) = \mathbbm{1}(T \geq t)$ the at-risk process at time $t$, indicating whether a subject has survived at least to time $t$. The hazard function for $T$ is defined as 
\begin{equation}
    \lambda(t|\cdot) := \lim_{\Delta t \to 0+}\frac{1}{\Delta t}\P(t \leq T < t + \Delta t|T \geq t, \cdot) = \lim_{\Delta t \to 0+}\frac{1}{\Delta t}\P( Y(t + \Delta t) = 0|Y(t) = 1, \cdot),
\end{equation}
where $\P$ denotes the underlying probability measure. 
%The hazard function for the potential outcomes can be similarly defined. 


Analyses of time-to-event endpoints in randomised experiments are commonly based on the Cox proportional hazards model
\begin{equation}\label{eq:fixed}
    \lambda(t|A) = \lambda_0(t)\exp(\beta A),
\end{equation}
for a binary treatment $A$. Define also the following saturated generalized Cox model allowing time-varying treatment effect:
\begin{equation}\label{eq:varying}
    \lambda(t|A) = \lambda_0(t)\exp\{\beta(t) A\},
\end{equation}
so that 
\begin{equation}
    \exp\{\beta(t)\} = \frac{\lambda(t|A = 1)}{\lambda(t|A = 0)}. 
\end{equation}
We will discuss whether and when $\beta$ and $\beta(t)$ can be causallly interpreted. To that end, we introduce the potential outcomes \citep{rubin1974estimating, holland1986statistics}. Denote $T_a$ the potential time to event if $A$ were to be set to $a$, and correspondingly $Y_a(t) = \mathbbm{1}(T_a \geq t)$. Since $A$ is randomized, we have
\begin{equation}\label{eq:rand}
    A \perp (T_1, T_0, Y_1(t), Y_0(t),  L).
\end{equation}
We assume the standard causal consistency assumption
\begin{equation}
    T = T_A,
\end{equation}
and the positivity assumption that %the trial is well balanced,
\begin{equation}
    0 < \P(A = 1) < 1.
\end{equation}


\section{Literature review}\label{sec:review}

\subsection{\citet{hernan2010hazards} stated that a time-fixed hazard ratio is misleading
%\lily{double check if he used the word misleading}\andrew{He did, on page 3}
}

\citet{hernan2010hazards} claimed that although the hazard ratios
may change over time in studies, often in practice a single hazard ratio is averaged over the duration of the study’s follow-up. As a result, the conclusions from the study may critically depend on the duration of the follow-up.


\subsection{\citet{hernan2010hazards}, \citet{aalen2015does} and \citet{martinussen2020subtleties}
%\lily{cannot use cite like this}\andrew{how about this} 
claimed that a time-varying hazard ratio has no causal interpretation}\label{sec:timevarying}
Meanwhile by \eqref{eq:rand} and consistency,
\begin{equation}\label{eq:counterhr}
    \exp\{\beta(t)\} = \lim_{dt \to 0+}\frac{\P(t \leq T < t + dt|T \geq t, A = 1)}{\P(t \leq T < t + dt|T \geq t, A = 0)} = \lim_{dt \to 0+}\frac{\P(Y_1(t + dt) = 0|Y_1(t) = 1)}{\P(Y_0(t + dt) = 0|Y_0(t) = 1) }.
\end{equation}
\citet{hernan2010hazards}, \citet{aalen2015does} and \citet{martinussen2020subtleties}  concluded that \eqref{eq:counterhr} cannot be causally interpreted because the risk sets, i.e.~the conditioning sets comprised of the subsets of individuals who have not previously failed, differ beyond the first event time. 

In addition, \citet{hernan2010hazards} stated that ``differential selection of less susceptible women over time ... is the built-in selection bias of period-specific hazard ratios.'' 

%\citet{hernan2010hazards} gave an intuitive example for this built-in selection bias for WHI study.\andrew{add}

%Nonetheless, we argue that the hazard function can have causal interpretation under other causal frameworks. Therefore, whether the hazard ratio $\beta(t)$ is causal is truly a philosophical question. Indeed,



\subsection{\citet{aalen2015does} claimed that the selection bias results from a collider bias}

\citet{aalen2015does} provided a visualization to illustrate the selection bias, which we recreate here in Figure \ref{fig:dag}, as a causal directed acyclic graph (DAG) describing the data generating process of the observed data. 
% Figure environment removed
In Figure \ref{fig:dag}, both $A$ and $L$ point to the risk sets $Y(t)$ and $Y(t + dt)$ admitting that $A$ and $L$ affect $Y(t)$ and $Y(t + dt)$ possibly; there is no arrow between $A$ and $L$ because $A$ is randomized. %For any $t\geq 0$, 
%The second perspective to blame for using the hazard function is from graph theory. 
Since $A$ is randomized, a direct 
%Here we paraphrase \citet{aalen2015does} using the notation defined above: ``If we consider the 
comparison of $Y(t)$ for any $t$ (including $t + dt$) between the treatments groups 
%or $Y(t + dt)$ separately, then a random allocation will ensure
is a valid assessment of the causal effect. 
%For instance, when considering the effect of $A$ on $Y(t + dt)$, the noncausal path, via $L$, is closed as long as we do not condition on $Y(t)$.
%closes the path via $L$ as long as it is not conditioned upon. 
On the other hand, the node $Y(t) = \mathbbm{1}(T \geq t)$ is a collider and if one considers the probability of surviving up to time $t + dt$ conditional on survival up to time $t$, then 
%the situation is changed. This change arises because we condition on a collider $Y(t)$, which activates 
the non-causal path $A \to Y(t) \leftarrow L \to Y(t + dt)$ is activated. 
%If $L$ is (in whole or partially) unknown, 
This path cannot be closed unless $L$ is completely known, which implies that we generally have $A \not\perp L | Y(t) = 1$.  This can also been seen intuitively  if the treatment has any non-zero effect, since at any $t>0$ the two groups would have failed at different rates and are no longer exchangeable, despite the fact that  
%so the compositions of the groups of treated and non-treated survivors at time $t$ differ systematically, even if 
the treatment was randomly assigned at $t = 0$. %This is a problem if we wish to assign meaning to differences of the respective hazard rates at time $t$ since the hazards at time $t$ are sensitive to previous survival in the two groups.''

We note that the above DAG and argument are based on the observed variables instead of the potential variables. And the risk set, or equivalently, $Y(t)$, is also observed as opposed to being potential. 






\subsection{\citet{martinussen2020subtleties} proposed an alternative ``causal'' HR}

Given their conclusion that the hazard ratio does not deliver causal interpretation,  \citet{martinussen2020subtleties} proposed the so-called ``causal hazard ratio, ''
\begin{equation}\label{eq:causalhr}
    \frac{\lim_{dt \to 0+}\P(t \leq T_1 < t + dt|T_1 \geq t, T_0 \geq t)/dt}{\lim_{dt \to 0+}\P(t \leq T_0 < t + dt|T_1 \geq t, T_0 \geq t)/dt}.
\end{equation}
\citet{martinussen2020subtleties} believed that \eqref{eq:causalhr} delivers valid causal interpretation because it compares quantities conditioned on the same subpopulation.









\section{Defense of hazard ratio in a causal setting}
%We defend the hazard ratio point by point listed in Section \ref{sec:review}.


\subsection{Time-fixed hazard ratio delivers causal interpretation}

We first confirm that a time-fixed hazard ratio $\beta$ delivers causal interpretation. Note that under model \eqref{eq:fixed}, consistency, and positivity,
\begin{equation}
    \exp(\beta) = \frac{\log \P(T_1 > t)}{\log \P(T_0 > t)}.
\end{equation}
Indeed this is confirmed in \citet{martinussen2020subtleties}.

While it might be the case that the underlying true hazard ratio changes over time, any practitioner understand the need for 
%Moreover, whether oversimplifying the Cox model from $\beta(t)$ to $\beta$ leads to misleading interpretation is not the fault of the Cox model but rather how 
parsimony. % a practitioners needs. 
\citet{prentice2022intention} showed that among other things, a single averaged hazard ratio is more sensitive to early treatment effect than, for example, the restricted mean survival time. 

\subsection{Time-varying hazard ratio has a causal interpretation}\label{sec:frameworks}

We note that the definitions of causal interpretation vary across different frameworks of causality. We examine below three different 
%with Donald Rubin's, James Robins', and Judea Pearl's 
frameworks.
\begin{enumerate}
    \item[(a)] {\it Rubin's framework}:

The definition of causal interpretation in \citet{hernan2010hazards}, \citet{aalen2015does} and  \citet{martinussen2020subtleties} is best formalized from Donald Rubin's framework. According to definitions found in, for example, \cite{rubin1974estimating, rubin1978bayesian} and \citet{frangakis2002principal}, the above contrast \eqref{eq:counterhr} does not convey any causal interpretation. Given a random sample indexed by units $i$, according to \citet[Equation (2.1)]{frangakis2002principal}, in our notation: ``a causal effect of assignment on the outcome $T$ is defined to be a comparison between the potential outcomes on a common set of units, e.g., a comparison between $\{T_{1, i}:i \in \text{set}_1\}$ and $\{T_{0, i}: i \in \text{set}_0\}$ given the groups of units, $\text{set}_1$ and $\text{set}_0$, being compared are identical. '' 
Under this definition, $\beta(t)$ cannot be interpreted causally because the numerator and denominator in \eqref{eq:counterhr} are conditioned on $\text{set}_1 = \{i: Y_{1, i}(t) = 1\}$ and $\text{set}_0 = \{i: Y_{0, i}(t) = 1\}$, which cannot be identical except for under the sharp null that there is no individual treatment effect. 

Despite of the above, we can still rewrite
\begin{equation}
    \exp\{\beta(t)\} = \frac{\lim_{dt \to 0+}\log \P(T_1 > t + dt|T_1 > t)/dt}{\lim_{dt \to 0+}\log \P(T_0 > t + dt|T_0 > t)/dt}
    = \frac{ d \log \P(T_1 > t )/dt}{ d \log \P(T_0 > t )/dt},
\end{equation}
and in this way the time-varying hazard ratio seems to align with Rubin's framework.

\vskip .1in
    \item[(b)] {\it Robins' framework}:

\citet[Page 7, Technical Point 1.1]{hernan2020causal} defined: ``$\cdots$ population causal effect may also be defined as a contrast of, say, medians, variances, hazards, or cdfs of counterfactual outcomes. In general, a population causal effect can be defined as a contrast of any functional of the marginal distributions of counterfactual outcomes under different actions or treatment values. ''
Since the hazard function itself is a functional of the marginal distributions, and hazard ratio is a contrast, the hazard ratio should be causally interpretable in this sense.

\vskip .1in
\item[(c)] {\it Pearl's framework}:

\citet[Page 70, Definition 3.2.1]{pearl2009causality} defined: ``Given two disjoint sets of variables, $X$ and $Y$, the causal effect of $X$ on $Y$, denoted either as $\P(y|\hat x)$ or as $\P(y|\text{do}(x))$, is a function from $X$ to the space of probability distributions on $Y$.''
The latter translates to potential outcomes with our notation as: given %two disjoint sets of variables, 
$A$ and $T$, the causal effect of $A$ on $T$, denoted as $\P(T_a)$, is a function from $A$ to the space of probability distributions on $T$. Note that the knowledge of $\P(T_a)$ is equivalent to knowing the counterfactual hazard function $\lim_{dt \to 0+}\P\{Y_a(t + dt) = 0|Y_a(t) = 1\}/dt$ throughout time $t$. 
Therefore contrasting hazard functions between the treatment groups should yield valid causal interpretation under Judea Pearl's framework.
\end{enumerate}


%The first main criticism for using the hazard function is that the hazard function  

Another confusion point in \citet{hernan2010hazards}  is about `less susceptible women'. This has  been well understood in the literature as unobserved heterogeneity, i.e.~frailty, since at least as early as \cite{Lancaster}. It has also been shown in \cite{omor:john} that denoting $V$ the multiplicative frailty on the hazard function,  $E(V|T>t)$ is non-increasing in $t>0$, with no distributional assumption on $V$ required. In other words, the `differential selection ... over time' is due to the existence of unobserved heterogeneity, also referred to as over-dispersion, and not `period-specific hazard ratios' as speculated in \citet{hernan2010hazards}. 


\subsection{No collider bias}


%We also explain this by graph theory. To clarify the misunderstanding,
Here as opposed to \citet{aalen2015does} we draw a single-world intervention graph (SWIG) \citep{richardson2013single} in Figure \ref{fig:swig} which depicts the data generating process like the DAG in Figure \ref{fig:dag}, except on the potential outcomes. From a SWIG, one can better read dependency among variables in the counterfactual world by d-separation. To form a SWIG from a DAG, one first copies the DAG including all nodes and arrows. One then splits the treatment variable node $A$ into a random part $A$ and an intervention part $a$. All arrows pointing to $A$ in the original DAGs still point to $A$ in the SWIG. All arrows pointing to others from $A$ in the original DAGs now are pointing to others from the intervention node $a$ in the SWIG. Lastly, one turns all the post-treatment variables into potential variables. Following this way, the DAG in Figure \ref{fig:dag} is transformed into the SWIG in Figure \ref{fig:swig}. Since there is no arrow between $A$ and $Y_a(t)$, one can read that $A \perp L | Y_a(t) = 1$ and hence \emph{$Y_a(t)$ is not a collider}. It follows that the compositions of the groups of treated and untreated survivors at time $t$ are exchangeable in the counterfactual world, therefore the comparison in \eqref{eq:counterhr} still delivers a causal meaning.
% Figure environment removed
%The above can also be explained analytically. 
%In a randomized trial involving time to event outcome, the Cox model is typically adopted to model the hazard function with respect to the treatment assignment as

%We want to clarify that indeed there is nothing wrong interpreting $\beta(t)$ causally. 
The above shows that the hazard function is in fact conditional upon the counterfactual at-risk process $Y_a(t)$ under randomization, 
%\lily{not reflected in your definition of hazard. Are we thinking MSM here?} 
instead of the observed at-risk process $Y(t)$. Such an observation equips $\beta(t)$ with a causal interpretation because $A$ and $Y_a(t)$ are independent according to \eqref{eq:rand}. 








\subsection{Causal hazard ratio is not satisfactory}

According to Section \ref{sec:frameworks}, the causal hazard ratio as defined in \eqref{eq:causalhr} can be causally interpreted by Rubin's causal framework because the numerator and the denominators condition on the same subpopulation. However, it does not fit Robins' framework because it is not a contrast of any functional of the marginal distributions of counterfactual outcomes, but rather the joint distributions. It also does not fit Pearl's framework for the same reason. Furthermore, it can never be nonparametrically identified because it hinges upon the joint distribution of $(T_1, T_0)$, which cannot be simultaneously observed. \cite{axel:nevo} developed an approach for sensitivity analysis, which unfortunately cannot be used as primary analysis of any randomized clinical trials. Last but not least, the causal hazard ratio is a type of ``survivor average causal effect," of which
\citet[Section 7]{dawid2000causal} questioned the use and called it ``a fundamental use of fatalism.'' He stated ``... it is only under the unrealistic assumption of fatalism that this group has any meaningful identity, and thus only in this case could such inferences even begin to have any useful content.'' 

%As another example, 
In our own experience analyzing observational studies of exposure during  pregnancy, while live birth is a post treatment outcome and should be used to form survivor principal strata \citep{ying2020causal}, in practice it is nonetheless standard to stratify by the observed live birth outcome 
%in analyzing the primary outcome of major structural malformation 
\citep{chambers2001postmarketing, cham:etal:2010, cham:2013, cham:2016, cham:etal:2019, cham:plusone, cham:hydroxy}. This is how the primary outcome results are presented in the teratology and dysmorphology literature, as well as the United States Food and Drug Administration vaccine and medication labelling.  
%xu:cham, xu:luo:cham}.
%\lily{check year for cham:etal:2019, page number for PLUS ONE, cham:hydroxy; there is also a Chambers in early 2000 on design of such studies, should be cited in the 2010 paper}\andrew{Done}
%\andrew{These ones?}

    

\section{A useful interpretation of changing hazard ratio}

Here in addition to the Women's Health Initiative examples described in \citet{prentice2022intention}, we provide other examples to illustrate the desired interpretation of $\beta(t)$ in practice. 

In a two-group setting where the proportional hazards assumption is violated and the hazard functions  cross each other, a representative example is organ transplantation \citep{oq:pess}. In such a situation, the physician and their patient would in fact like to know, that compared to the same patient 
%\lily{is this the precise counterfactual language?}\andrew{I think so} 
without transplant, the patient who receives it is initially at a higher risk immediately following the transplant surgery but, once the patient survives the surgery and also does not develop the so-called graft-versus-host disease, in the long run the patient is much better off (i.e.~with lower hazard for death etc.) compared to the counterfactual scenario where they did not receive the transplant. 

A similar situation can be described for immunotherapy to treat cancer which has a delayed effect.

Finally, during the recent Covid-19 pandemic, vaccine effectiveness is measured as one minus the hazard ratio.  \cite{Lin:etal:2022, lin2022effects} 
%\lily{they have a 2nd NEJM on pediatric vaccine I think}\andrew{This one?} 
used time-varying hazard ratio, by placing a change point at every month, to estimate the current risk of Covid-19 after receiving the vaccine. Because the hazard pertains to the current risk, estimation of the hazard ratio from the time of the first dose does not bias the estimation of the hazard ratio since full vaccination, and enables evaluation of the vaccine effect during the ramp-up period. This last fact applies to general dynamic treatment regimes, where the treatment immediately impacts the hazard, or more generally transition intensity of multi-state systems.  This impact is then reflected in cumulative quantities such as the survival probabilities. 

%The above examples as well as the SWIG in Figure \ref{fig:swig}  point to a \emph{potential system}'s view following, as the simplest scenario, a binary point treatment at baseline, of the two potential worlds  and how they behave and evolve  over time, depending on whether there is a treatment effect. 

%\section{Conclusion}

%Researchers have defined alternatives as opposed to $\beta(t)$, which ``avoid selection bias'' and convey causal interpretation under Rubin's definition. These include causal hazard ratios defined by the controlled direct effect of treatment \citep{aalen2015does} or within principal strata \citep{martinussen2020subtleties}. However, both quantities require stringent and questionable assumptions pointed out by themselves.

%Time will tell?

%\lily{Also survival text sequential conditioning might be worked into some earlier part}


%Extension to observational and  time-varying treatment 

%MSM 



%\citet{aalen2015does} proposed an alternative quantity, a controlled as compared to \eqref{eq:counterhr}
\section*{Acknowledgement}

The authors would like to acknowledge the Joint Statistical Meetings 2022 round table discussion sponsored by the Statistics in Epidemiology Section, which motivated the writing of this short communication. The authors also acknowledge discussion at the Lifetime Data Science (LiDS) conference 2023 which furthered our thinking on the subject.

\bibliographystyle{biometrika}
\bibliography{ref}



\end{document}
