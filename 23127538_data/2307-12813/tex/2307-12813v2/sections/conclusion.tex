\section{Conclusion and Limitation}
\label{sec:conclusion}

% In this paper, we present a more flexible and realistic setting for object understanding, called Described Object Detection, which encompasses both object detection and referring expression comprehension tasks. For this, we propose a benchmark more challenging than existing benchmarks, with unrestricted language reference, complete annotation, and absence expressions. We evaluate several state-of-the-art methods from previous tasks on this dataset, and propose a new baseline outperforming them for future research.
% This work contributes to the development of object understanding and provides a new benchmark for evaluating models in this area.

In this paper, we bring the Described Object Detection (DOD) task to the foreground. 
For this task, we introduce a dataset called \ddd{}, which annotates described objects without omission and features flexible language expressions, whether long or short, complex or simple. 
Our evaluation of SOTA methods from REC or OVD on \ddd{} reveals challenges faced by REC, OVD, and bi-functional approaches.
Based on these observations, we propose a baseline that largely improves REC methods for DOD task.
We believe that the dataset and findings will contribute to advancing the understanding and development of DOD methods, facilitating future research in this area.

\noindent \textbf{Limitation and broader impact.}
% This work establishes the research foundation for the DOD task. Compared to traditional detection algorithms, DOD models have lower customization thresholds, enabling users to specify the detection target using language. This may lead to potential abuse, such as detecting people's privacy. Moreover, a significant amount of GPU computing was utilized in model development and algorithm evaluation, resulting in carbon emissions.
This work does have some limitations. Due to the significant annotation cost brought by our complete annotation process, we are unable to propose a huge dataset with millions or billions of images. Besides, the evaluation and findings in this work may be dependent on the choice of descriptions and the image sources. This work only serves as a starting point for DOD and we hope there will be other DOD datasets with larger scales.
In the broader community, compared to traditional detection algorithms, DOD models have a lower customization threshold, enabling users to specify the detection target using language. This may lead to potential abuse.

\noindent \textbf{Future work.}
During peer-review process, some new works with potential for DOD emerges, including Shikra~\cite{chen2023shikra}, Kosmos-2~\cite{peng2023kosmos} and Qwen-VL~\cite{bai2023qwenvl}. We will continue to investigate such methods for DOD and update them in \href{https://github.com/Charles-Xie/awesome-described-object-detection}{this list}.

\noindent \textbf{Acknowledgments.}
This work was supported in part by the National Natural Science Foundation of China under Grant 62076183, 61936014 and 61976159, in part by the Natural Science Foundation of Shanghai under Grant 20ZR1473500, in part by the Shanghai Science and Technology Innovation Action Project under Grant 20511100700 and 22511105300, in part by the Shanghai Municipal Science and Technology Major Project under Grant 2021SHZDZX0100, and in part by the Fundamental Research Funds for the Central Universities. The authors would also like to thank the anonymous reviewers for their careful work and valuable suggestions.
