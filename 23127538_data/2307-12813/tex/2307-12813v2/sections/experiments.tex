\section{Experimental Analyses}
\label{sec:experiments}

\subsection{Comparison of baselines on our metrics}
\label{subsec:experimental_comp}

\subsection{Comparison with state of the art}
\label{sec:comparison}

In this section we intend to answer RQ2 by comparing \dataset with the state-of-the-art while list of libraries generated by Li et al~\cite{10.1109/SANER.2016.52} (RQ2.a) and against LibRadar~\cite{10.1145/2889160.2889178}, a state-of-the-art tool to detect libraries in Android apps (RQ2.b).


\subsubsection{Comparison with state-of-the-art white list}

Since Li et al.'s dataset was generated prior to 2016, it would be unfair to compare both datasets as is, rather than the approaches themselves.
Hence, we applied our approach and only retain libraries that were available before 2016.
In this section, we call our dataset \dataset$_{2016}$

\noindent
\textbf{Comparison dataset:}
To compare with the work of Li et al., we retrieved the package names available in the "libraries" folder of the repository~\cite{commonLibrariesRepo} described in their paper~\cite{LI2019157}.
After removing any duplicates, the resulting list, which we refer to as \textbf{comparison\_dataset}, contains \num{5926} package names.

% Figure environment removed


\noindent
\textbf{Comparison:}
Upon a first examination, we found that the two lists have little in common, as \dataset$_{2016}$ contains \num{9551} package names while \textbf{comparison\_dataset} contains only \num{5926} package names, and the intersection of the two lists is only 66, as shown in Figure~\ref{fig:intersection}.
This suggests that our list is significantly larger and more comprehensive than the state-of-the-art list.
However, it should be noted that this comparison was made using strict one-to-one string matching, so some of our package names might be prefixes of their package names, and vice versa.

As a result, we conducted a follow-up comparison to determine if any of the libraries in \dataset$_{2016}$ might be prefixes of the package names in the \textbf{comparison\_dataset} and vice versa.
Results indicate that 280 of our library names are prefixes of 1636 of their library names.
Additionally, 101 of their library names are prefixes of 194 of our library names.
The total number of common libraries using prefix matching is \num{1722}, as shown in Figure~\ref{fig:intersection}.
These findings show some overlap between the two lists and that \dataset$_{2016}$ covers a larger part of the \textbf{comparison\_dataset}.
However, the overall intersection of the two lists is still relatively small, with only a small proportion of libraries being common to both lists.

To provide further assessment of the \textbf{comparison\_dataset}, a qualitative study was conducted. 
This study aimed to identify potential libraries in the \textbf{comparison\_dataset} that were not included in the package names found in both \dataset$_{2016}$ and the \textbf{comparison\_dataset} with prefix matching (in other words, we want to answer the following question: does the \textbf{comparison\_dataset} contain many libraries that \dataset$_{2016}$ does not?). 
A total of 50 package names were randomly selected. 
We manually searched the Internet to identify any mentions, repositories, or websites that would indicate that they are libraries. 
Results indicate that only 2 out of the 50 package names were found to be actual libraries.
When searching the Internet we found  that these two libraries had dedicated website which were not crawled by our approach.
The 50 package names are available in the project's repository.

We did not further manually check whether the libraries identified by \dataset$_{2016}$ but not by  \textbf{comparison\_dataset} are true libraries or not because, by construction, \dataset only contains "true" libraries.

\highlight{
\textbf{RQ2.a answer:}
Our approach, \dataset$_{2016}$, exhibits a larger coverage of libraries compared to the state-of-the-art dataset by Li Li et al.
Although there is some overlap between the two lists, the overall intersection remains relatively small. 
Our qualitative analysis shows that Li et al. dataset is not reliable, unlike  \dataset.
}

\subsubsection{Comparison with LibRadar}

LibRadar was introduced as a tool to detect third-party libraries in Android apps.
Although its primary purpose is not to generate a dataset of libraries, it can be used to create a list by running it on apps and retaining the libraries identified in the output.
To this end, we executed LibRadar on \num{100000} recent apps (i.e., collected after 2020) and compiled a list of libraries. 
LibRadar successfully completed the analysis for \num{62308} apps, producing a list of \num{12239} unique package names.
However, this list cannot be used as a whitelist since it contains over \num{3000} obfuscated package names (e.g., a.a.a). 
While these package names might represent libraries in apps, they cannot be employed as is for a whitelist, since a.a.a might not be a library in another app. 
Consequently, we cannot directly compare both lists. 

To further investigate which of LibRadar or \dataset would be more useful into filtering libraries in apps, we have randomly selected 100 apps from the \num{62308} previously analyzed apps.
For each of these apps we
\dcircle{1} applied LibRadar to extract the libraries it would detect; and 
\dcircle{2} extracted the package name of each class and used \dataset to check if they were libraries or not.
In total, LibRadar is able to filter 174 non-obfuscated libraries from these 100 apps, whereas \dataset can filter \num{2745}.
The median number of libraries found per app is 1 for LibRadar and 28 for \dataset.
This result shows that LibRadar misses a vast amount of libraries in Android apps that \dataset would not miss.


\highlight{
\textbf{RQ2.b answer:}
Our empirical analysis shows that \dataset covers more non-obfuscated libraries than LibRadar in apps.
}


We make comparisons on the baselines introduced in \cref{sec:baselines}, mainly with the intra-scenario setting.
Unless explicitly noted, this is the default setting, instead of the more difficult inter-scenario.

\noindent \textbf{Existing SOTAs are insufficient for DOD, and bi-functional models outperform others.}
As demonstrated in \cref{tab:comparison}, existing methods, while achieving SOTA performance on their original benchmarks, fall short in delivering strong performance on \ddd{}.
Among them, recent bi-functional methods~\cite{liu2023groundingdino,yan2023universal} are notably superior to others, and currently, OVD methods outperform REC.
The inferiority of REC methods is likely due to their impractical setting described in \cref{sec:introduction}, which involves predicting one and only one instance for each reference. We will delve into this further.

\noindent \textbf{Rejecting irrelevant references are difficult, which REC are naturally incapable of.}
In contrast to intra-scenario evaluation, the inter-scenario setting assesses all references in the dataset for each image. Since references from other scenarios are likely not semantically relevant to the images, this necessitates the ability to reject irrelevant references for an image. This aligns with the evaluation in standard detection tasks.
From \cref{tab:comparison}, it is evident that OFA, a REC method, almost completely fails in this setting. This is caused by its prediction of a region for every reference, resulting in a large number of false positives when there are numerous candidate references. This underscores the importance of empowering REC methods with the ability to reject false positives.
We find that none of the verified methods achieve good performance under the inter-scenario setting, indicating that existing methods are far from being capable of DOD. This highlights the challenge of \ddd{}

\noindent \textbf{The proposed baseline outperforms existing methods.}
The proposed baseline is based on OFA, but our improvements significantly enhance its performance. It outperforms all existing methods in the intra-scenario setting and surpasses them by a wider margin in the inter-scenario setting. This may suggest that the proposed baseline has a stronger ability to reject irrelevant references.
Nonetheless, the proposed method is far from perfect and can only serve as a baseline for future research.

\input{figures/score_distribution}

\subsection{Further analysis}

\noindent \textbf{Absence descriptions are more difficult for most methods.}
As shown in \cref{tab:comparison}, the performance of baseline methods on \textsl{PRES} (presence descriptions) is consistently superior to that on \textsl{ABS} (absence descriptions). This suggests that existing methods may not effectively differentiate between the presence and absence of attributes in a language description.


\noindent \textbf{REC methods fail to provide good confidence scores.}
We visualized the score distributions from baselines for TPs and FPs, to assess their capabilities in classification and confidence estimation.
As in \cref{fig:score_distribution}, the confidence scores from OFA do not exhibit a clear distinction between TP and FP cases. This can be attributed in part to the seq2seq framework in OFA, which does not directly yield confidence scores, and in part to the grounding formulation of REC, which identifies the image region most similar to the text description without distinguishing between positive and negative.

With a task decomposition step to enhance binary classification performance, our OFA-DOD demonstrates a significant disparity between TP and FP score distributions, yielding more reliable classification results. Note that this improvement does not necessitate modifications to the model framework or training datasets; rather, it is attributed to a more appropriate task formulation.

\begin{table}[t]
    \centering
    \caption{Evaluation regarding different number of instances in a image for each reference.}
    % \resizebox{\textwidth}{!}{
    \begin{tabular}{l | c | c | c c c c}
        \hline
        Method & No-instance & One-instance & \multicolumn{4}{c}{Multi-instance mAP(\%)} $\uparrow$ \\
        & FPPC (\%) $\downarrow$ & mAP (\%) $\uparrow$ & 2 & 3 & 4 & 4+ \\
        \hline
        OFA & 100.0 & 14.8 & 9.5 & 7.9 & 5.4 & 3.7 \\
        CORA & 17.3 & 9.7 & 8.4 & 9.5 & 9.0 & 8.5 \\
        OWL-ViT & 41.9 & 21.1 & 17.3 & 16.6 & 16.0 & 14.0 \\
        UNINEXT & 100.0 & 55.7 & 26.2 & 18.6 & 14.4 & 9.0 \\
        G-DINO & 100.0 & 63.7 & 28.3 & 19.7 & 15.9 & 10.1 \\
        OFA-DOD & 35.6 & 56.4 & 19.6 & 12.7 & 10.3 & 7.1 \\
        \hline
    \end{tabular}
    % }
    % \vspace{-10pt}
    \label{tab:analysis_box_num}
\end{table}


\noindent \textbf{Multi-instance detection is challenging for methods other than OVD.}
For each image, \ddd{} can have zero to multiple instances \textbf{for a single description}. To assess how current methods handle varying numbers of instances, we conducted evaluations under three different settings: \textbf{no-instance}, where for a reference, evaluations are limited to images without any referred instance; \textbf{one-instance}, for images with a single instance; and \textbf{multi-instance}, for images with multiple instances. As shown in \cref{tab:analysis_box_num}, OVD methods outperform others when multiple instances are referred by the description, although they may not be as competitive on the entire dataset or images with few instances. Notably, OWL-ViT maintains consistent performance even as the number of instances increases, which sets it apart from other methods. In contrast, REC and current bi-functional methods struggle in multi-instance scenarios. This highlights the strength of OVD methods in multi-target detection, while REC and current bi-functional approaches are less robust in such situations.

\noindent \textbf{REC and bi-functional methods lack the ability to reject negative instances.} In the \textbf{no-instance} column of \cref{tab:analysis_box_num}, we do not report mAP since there are no positive instances in GT for the corresponding reference, making AP inapplicable. Predictions on such images are FPs, so we measure the ratio of images where FPs are produced to the total number of no-instance images for a given reference, namely False Positives Per Category (FPPC). We report the average FPPC over all references.
We observe that most baselines are incapable of determining whether an image contains the referred target or not, yet they still produce predictions. This behavior is expected for REC methods. Bi-functional methods, trained and inferred with the REC task formulation, also exhibit this issue. Only the OVD method and our proposed baseline can effectively reject such negative image-text pairs.

\begin{table}[t]
    \centering
    \caption{Evaluation one references with various lengths.}
    % \resizebox{\textwidth}{!}{
    \begin{tabular}{l | c c c c}
        \hline
        Method & \textit{short} & \textit{middle} & \textit{long} & \textit{very long} \\
        \hline
        OFA & 4.9 & 5.4 & 3.0 & 2.1 \\
        % CORA & 3.3 & 5.0 & 5.8 & 4.4 \\
        OWL-ViT & 20.7 & 9.4 & 6.0 & 5.3 \\
        UNINEXT & 18.5 & 23.3 & 17.4 & 16.1 \\
        G-DINO & 22.6 & 22.5 & 18.9 & 16.5 \\
        OFA-DOD & 23.6 & 22.6 & 20.5 & 18.4 \\
        \hline
    \end{tabular}
    % }
        % \vspace{-10pt}
    \label{tab:analysis_ref_length}
\end{table}


\noindent \textbf{OVD methods suffer from long descriptions greatly while others do not.}
We partition the references according to their lengths and then evaluate on these partitions. The results are shown in \cref{tab:analysis_ref_length}, where \textit{short}, \textit{middle}, \textit{long} and \textit{very long} corresponding to references with 1\textasciitilde 3, 4\textasciitilde 6, 7\textasciitilde 9, and more than 9 words.
For \textit{short} descriptions, which is close to OVD setting, OVD and bi-functional methods obtain similar performance. However, as the length of references increases, the performance of OVD methods decrease fast, while REC and bi-functional methods suffer less from this. We can see that OVD methods are sensitive to long references, as expected, while other two types do not.

More experiments and additional \textbf{qualitative results} are available in \supp{}.

% \input{tables/recall}

% \noindent \textbf{Compared with REC, OVD methods tend to make predictions only when it's certain.}
% % \noindent \textbf{Analysis on recalls.}
% The recalls of different methods are shown in \cref{tab:recall}. OVD methods are obviously bad at recall, which means that it tends to produce a result when it is quite certain. Grounding-DINO, though performs not as good as the proposed baseline in terms of mAPs, obtains the best recall. This indicates that it tends to produce more detection results.

% \Todo{do we need to consider inference strategy (one ref at a time vs all ref at a time? No.}

% \section{Visualization and Analysis}

\subsection{Analysis on In-context Learning}
Previous quantitative results and analysis have answered our question that prior knowledge about temporal modeling from LLM greatly benefit bottom-up future prediction by finetuning LLM. In this part we want to further exploring the following two questions: Can LLM infers goal based on observed actions and how would predicting goal influence top-down future prediction? What will happen if explicitly knowing the goal and how would the given goal affect LLM's prediction. To answer these first questions, we designed three qualitative experiments and analysis the results. For the second question, we designed counterfactual prediction experiment.

% We wish to study qualitatively HOW in-context learning with optionally Chain-of-Thought would make LLM prediction the future action sequences. We demonstrate a few examples from our training data to the LLM and query for the action sequence of interest. We design three type of prompt corresponding to three questions we want to know about. Concretely, we wish to study the LLM's capability in 1)predicting the goal, 2)predicting the future actions, and 3)predicting the future actions given the predicted goal of the actor. Figure 2 is an illustrative examples of our in-context learning approach. 

\noindent\textbf {ICL: Goal Prediction}
For top-down future anticipation methods, we hope the model to "think about future" before making predictions. In order to achieve this, we design qualitative experiment to see if the model could output reasonable goal given previous actions. By manually checking the predicted goal, we find LLM is able to generate reasonable and related long-term goal according to the previous observations. Figure~\ref{fig:vis2} shows an example video. For this video, LLM succeed predicting the goal as 'painting a room' given the action recognition results even there are few obvious mistakes like recognizing 'paintbrush' to 'stick' and 'container' to 'string'. Although successfully predicting 'painting' as the goal, LLM wrongly predict the goal as a room while the video happens outdoor. This is probably due to our discrete representations lost the detailed visual context.


\noindent\textbf{ICL: Future Action Prediction}
After verifying that LLM could make reasonable predictions of potential goal given past actions, we design the second experiment to use ICL to predict future actions. Follwing the ICL prompt design, we provide LLM several examples and history actions. We demonstrate the prompts and answers from LLM in the yellow block in Figure~\ref{fig:vis2}. In this example, the prediction edit distance for verb and noun are 0.55 and 0.55. For the first action, the model predict 'clean brush' while the answer is 'dip paintbrush', even though it failed in both verb and noun during metric calculating, but 'brush' and 'paintbrush' are actually synonyms so that the noun prediction also makes sense. This reveals that LLM results have further improvement space by using better post-processing methods to deal with complex linguistic phenomenon and the ambiguity of the task itself.

\noindent\textbf{CoT: Goal and Future Action Prediction}
The previous two experiments show that LLM can infer goals based on previous actions and do bottom-up future prediction using ICL. We further design experiment to see if LLM can predict both goal and future in a top-down fashion. In the green block in Figure~\ref{fig:vis2}, we demonstrate the procedure of top-down prediction using CoT prompts. The model succeed to predict the goal and get a significant improvement on verb from 0.55 to 0.10. For the first action, the model predicts 'dip brush' which actually match the ground-truth 'dip paintbrush'. Compared with bottom-up prediction 'clean brush', it's also more reasonable since it's not common to wash the brush during painting. Besides, bottom-up prediction shows the pattern of simply repeating the action 'paint wall' while top-down model successfully predict to 'paint wall' and 'dip brush' alternatively, which is not only closer to ground-truth labels but also better match common sense in the scene of 'painting a wall'. From this example, we can see predicting the goal before anticipating future can help generate examples more congenial with reason and common sense.

\noindent\textbf{Counterfactual: What If Given an Alternative Goal?}
To better understand the how would providing goal explicitly during prediction, we designed counterfactual prediction experiments: We first use GPT-3.5-Turbo to generate goal for the given example as we did above and treat this as the 'oracle goal'. We then manually design an 'alternative goal' which also makes sense to human when given the previous actions but different from the oracle. We then perform ICL prediction based on two different goals and the same observations. Figure~\ref{fig:vis3} illustrates the overall flow and two examples. Taking the second sample as example, even though sharing the same observations, the goal of 'fix machine' generates future actions more highly related with fixing such as 'tighten nut' and 'turn screw' while the goal of 'examine machine' generates totally different but highly goal-related actions like 'test machine' and 'record data'.

The counterfactual prediction experiment shows that explicit goal hugely influence LLM's prediction and LLM can generate reasonable prediction (although not correct) when giving the alternative goal. This could also explain the performance drop after explicitly adding LLM-generated goal into prompts hurts the performance shown in Table~\ref{tab:input_types} in some way: explicitly adding suboptimal goal will distract LLM from effectively reasoning from history actions.

%To observe how does the latent goal impact the LLLM's prediction, we come up with this setting where we switch the correct latent goal output by the model earlier with a perturbed but relative latent goal designed by ourselves and observe the change in predictions. Concretely, we want to know if the LLM rely on the explicitly expressed latent goal in the context to make predictions which is crucial in understanding the effectiveness of our top-down approach. As illustrated in Figure 4, we observed that even small changes in the latent goal would make the language model output quite different predictions. We also found that the predictions made in the counterfactual setting are consistent with the new goal. For example, when we switch the scene descriptor from "fixing machine" to "examining machine", the language model outputs many things only related to "examining machine", such as "examine circuits" or "record data". This shows that explicit latent goals have a high impact on large language model's prediction.


% Figure environment removed

% \noindent \textbf{Qualitative analysis.}
% We visualize the results of different baseline methods on the proposed benchmark. This is shown in \cref{fig:qualitative}. We can see that the proposed method handles such images better.

\begin{table}[t]
    \centering
    \caption{Ablation on the proposed baseline for its improvement components and the training data.}
    \begin{subtable}[t]{0.45\textwidth}
        \centering
        \caption{Method components.}
        \begin{tabular}{c c c c | c}
        \hline
        OFA & GD & RD & TD & mAP(\%) \\
        \hline
        \cmark & \xmark & \xmark & \xmark & 3.4 \\
        \cmark & \cmark & \xmark & \xmark & 10.5 \\
        \cmark & \cmark & \cmark & \xmark & 17.2 \\
        \cmark & \cmark & \cmark & \cmark & 21.6 \\
        \hline
       \end{tabular}
       % \vspace{-10pt}
       \label{tab:ablation_components}
    \end{subtable}
    \hfill
    \begin{subtable}[t]{0.45\textwidth}
        \centering
        \caption{Training data.}
        \begin{tabular}{c c c c | c}
        \hline
        REC & OD & I2T & MLM & mAP(\%) \\
        \hline
        \cmark & \cmark & \cmark & \cmark & 21.6 \\
        \cmark & \xmark & \cmark & \cmark & 16.4  \\
        \cmark & \cmark & \xmark & \cmark & 14.2 \\
        \cmark & \cmark & \cmark & \xmark & 20.3 \\
        \hline
       \end{tabular}
       \label{tab:ablation_tasks}
    \end{subtable}
    % \vspace{-10pt}
    \label{tab:method_ablation}
\end{table}


\subsection{Ablation on the proposed baseline}

\noindent \textbf{Method components.}
In \cref{tab:method_ablation}, we perform ablation on the proposed improvements in our baseline, step-by-step from OFA to OFA-DOD, to see how they affect the performance.
Granularity decomposition (GD) makes the method more suitable for localization task.
It disentangle tasks of global or local granularity by handling them with 2 separated branch.
Reconstructed data (RD) uniforms REC and OD data into the same form, and prepares multi-instance samples with both short and long references.
Task decomposition (TD) is proposed to help rejecting FPs.
It breaks down the DOD task into a REC step followed by a VQA step.
All three of them improve the performance obviously.

\noindent \textbf{Training tasks.}
We also perform a drop-one-out ablation on the multi-modal multi-task training data, in \cref{tab:ablation_tasks}.
\textbf{Detection} data provides samples for localization, especially multi-instance situation.
It is instinctively important for learning to localize, and indeed matters for performance.
\textbf{I2T} (image-to-text, like image captioning and visual question answering) often helps the generalization and zero-shot performance of multi-modal methods. We find that it does affect the zero-shot performance on \ddd{} greatly.
\textbf{MLM} is theoretically important for language understanding and generalization. However, we find it actually is not.
Removing the MLM task has no significant effect on the performance.
We surmise that the generalization ability of OFA-DOD on \ddd{} mainly comes from I2T.
