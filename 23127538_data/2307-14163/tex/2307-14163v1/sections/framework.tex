% !TeX root = ../MVFD_arxiv.tex


\section{The framework}\label{sec2}
In this section we  present a formal mathematical setup for the local regularity  for bivariate   stochastic processes  (also called scalar random fields, or simply random fields) and the data observed for such processes.

\subsection{Data}\label{sec:data}
Consider $N$ independent realizations, also called sheets, $X^{(1)},\ldots,X^{(j)},\ldots X^{(N)}$   of a stochastic process $X $ defined on a continuous domain $\cT\in\mathbb R^2$. For simplicity, we here focus on domains $\mathcal T$ in the plane, the extension to higher dimensions would not raise different challenges. For the purpose of describing our methodology, we distinguish three observational scenarios of the $N$ realizations. 
First, the ideal, infeasible situation where the sheets $X^{(j)}$ are \emph{completely observed}, that is without error over the entire domain $\cT$.  
The second case is the one where the $X^{(j)}$ are observed (measured) at some \emph{discrete points} in the domain $\cT$, \emph{without noise}. The domain points can be fixed to be the same for all the $X^{(i)}$'s (common design), or can be randomly drawn for each sheets separately (independent design). Finally, the most realistic scenario is the one where in the second case we admit that the realizations of $X$  are observed at discrete domain points \emph{with noise}. 



 To formally describe the second and third scenarios, let  $M_1, \dotsc,M_N$ be an independent sample of an integer-valued random variable $M$ with expectation $\EE[M]=\Mmu$. 
% which increases with $N$. 
In the independent design case, for each $1\leq j \leq N$, and given  $M_j$, let $\Tnm\in\cT$,  $1\leq m \leq  M_j$, be a random sample of a random vector $\TT\in\cT$. The $\Tnm$'s represent the observation points for the realization $\Xp{j}$. We assume that the realizations of $X$, $M$ and $\TT$ are mutually independent. 
In the common design case, $M\equiv \mathfrak m$ and the $\Tnm$'s are the same for all $j$.  Let $\mathcal T_{obs}^{(j)} $ denote the set of observation times $ \Tnm $, $1\leq m \leq M_j$, on the sheet $\Xp{j}$. With common design,  $\mathcal T_{obs}^{(j)} $ does not depend on $j$, while with independent design the expected cardinal of $\mathcal T_{obs}^{(j)} $ can be random with mean $\mathfrak m$.   The following presentation  includes both independent design and common design cases. \color{black}  Finally, the data  consist of  the pairs  $(\Ynm , \Tnm ) \in\mathbb R \times \cT $ where $\Ynm$ is defined as
\begin{equation}\label{model_eq}
	\Ynm = \Xn (\Tnm) + \enm, \quad\text{with}  \quad  \enm = \sigma(\Tnm,\Xn(\Tnm)) \unm, 
	\quad 1\leq i \leq N,  \; 1\leq m \leq M_j.
\end{equation}
Here, the $\unm  \in\mathbb R $ are independent copies of a centered variable $e$ with unit variance, and $\sigma^2(\cdot,\cdot)\geq 0$ is some unknown, bounded conditional variance  function which account for possibly heteroscedastic measurement errors. The case  $\sigma(t,x)\equiv 0$ corresponds to our second scenario, while in the third scenario we have positive conditional variance. 

For each $1\leq j \leq N$,  let $\widetilde X^{(j)}$ denote an observable approximation of $X^{(j)}$. If   the sheets $X^{(j)}$ were  completely observed, as in our infeasible first scenario, $\widetilde X^{(j)} = X^{(j)}$. 
When $X^{(j)}$ are observed  only at  some discrete points $\Tnm$,  arbitrary  $\widetilde X^{(j)}(\Tt)$ can be obtained by simple interpolation or defined equal to the value of $\widetilde X^{(j)}$ at the nearest neighbor of $\Tt$.  Finally, with noisy, discretely observed sheets,   $\widetilde X^{(j)}$  is a pilot nonparametric estimator  of $X^{(j)}$, such as kernel smoothing, splines \emph{etc}.  


Let us next introduce a general class of stochastic processes (random fields)  $X$ with  irregular realizations $X^{(j)}$, for which the regularity can vary over the domain $\mathcal T$. 


\subsection{A class of multivariate processes}
Let $\mathcal{T}$ be an open, bounded bi-dimensional rectangle with the closure included in $(0,\infty)^2$. In the following, $H_1,H_2 : \mathcal T \to (0,1)$ are two continuously differentiable functions such that 
\begin{equation}\label{low_thres}
\min_{i=1,2} \inf_{t\in\mathcal T} H_i(\Tt) >0.
\end{equation} 	
Let $\overline{H} = \max\{H_1 ,H_2\}$.
We also consider the vector-valued function $\mathbf{L}=(L_1^{(1)},L_2^{(1)},L_1^{(2)},L_2^{(2)}),$  where the components  are  non-negative, Lipschitz  continuous functions defined on $\mathcal{T}$ such that 
\begin{equation}\label{id_L}
	L_i^{(1)}(\Tt) +L_i^{(2)}(\Tt) >0,\qquad \forall \Tt\in\mathcal T, \; i=1,2.
\end{equation}
%We denote $\mathbf{L}=(L_1^{(1)},L_2^{(1)},L_1^{(2)},L_2^{(2)}),$ and assume that a constant $C_{\mathbf L}$ exists such that 
%\begin{equation}\label{up_thres}
%C_{\mathbf{L}}=\max_{i,j=1,2} \sup_{t\in\mathcal T} L_j^{(i)}(\Tt) <\infty.
%\end{equation} 	

Let $X$ be a real-valued, second order stochastic process defined on $(0,\infty)^2$. Let $(e_1,e_2)$ be the canonical basis of $\mathbb R ^2,$ and, for sufficiently small scalars $\Delta$, let   
$$
\theta_{\Tt}^{(i)}(\Delta)=\EE\left[\left\{X\left(\Tt-\frac{\Delta}{2}e_i\right)-X\left(\Tt+\frac{\Delta}{2}e_i\right)\right\}^2\right],\quad i=1,2.
$$ 


\begin{definition}\label{def}
Let $H_1$, $H_2$ satisfy \eqref{low_thres}.	The class $\mathcal {H}^{H_1,H_2}(\mathbf{L},\mathcal{T})$ is the set of stochastic processes $X$ satisfying the following condition:  constants $  \Delta_0, C,\beta>0$ exist such that 
	for any $\Tt\in \mathcal T$ and $ 0<\Delta\leq\Delta_0$,  
	\begin{equation}\label{as_repr}
		\left|\theta_{\Tt}^{(i)}(\Delta)-L_1^{(i)}(\Tt)\Delta^{2H_1(\Tt)} -L_2^{(i)}(\Tt)\Delta^{2H_2(\Tt)}\right|\leq C\Delta^{2 \overline{H}(\Tt)+\beta}, \quad i=1,2.
	\end{equation}
	Let  
	$$
	\mathcal{H}^{H_1,H_2} %=\mathcal{H}^{H_1,H_2}(\mathcal{T})
	=\bigcup_{\mathbf{L}}\mathcal {H}^{H_1,H_2}(\mathbf{L},\mathcal{T}) ,
	$$
	where the union is taken over the set of four-dimensional functions $\mathbf{L}$ with non negative positives Lipschitz  continuous components satisfying \eqref{id_L} 
	%and \eqref{up_thres}. 
	The functions $H_1,H_2$ define the local regularity of the process, while $\mathbf{L}$ represent  the local Hölder constants.
\end{definition}

Definition \ref{def}  is general, and extends the local regularity notion considered by \cite{GKP} for processes defined on a compact interval on the real line.  A main example we have in mind is the  multi-fractional Brownian  sheet (MfBs) with a time-deformation. MfBs  is a generalization of the standard fractional Brownian sheet, where the Hurst parameter is allowed to vary along the  domain. The definition of this general class of  processes and some of their properties are provided in Section \ref{BfMs}.



