\documentclass[12pt]{article}
%
%
% Retirez le caractere "%" au debut de la ligne ci-dessous si votre
% editeur de texte utilise des caracteres accentues
% \usepackage[latin1]{inputenc}

%
% Retirez le caractere "%" au debut des lignes ci-dessous si vous
% utiisez les symboles et macros de l'AMS
 \usepackage{amsmath}
 \usepackage{amsfonts}
%
%


\usepackage{graphicx}
\usepackage{color}

\usepackage[left=2.5cm,right=2.5cm,top=2.5cm,bottom=2.5cm]{geometry}
\setlength{\parskip}{6pt}


\usepackage[utf8]{inputenc}
\usepackage[T1]{fontenc}
\usepackage{xcolor,graphicx}
\usepackage{hyperref}
\usepackage{subcaption}
\usepackage{tikz}


\hypersetup{%
colorlinks=true,
breaklinks=true,
linkcolor=red,
anchorcolor=black,
citecolor=brown,
filecolor=blue,
menucolor=red,
urlcolor=red,
}
\usepackage[round,longnamesfirst]{natbib}
\usepackage{mathtools, amsfonts, amssymb,amsthm,stmaryrd}
\mathtoolsset{showonlyrefs}


\usepackage{enumerate}
\usepackage{comment}
\usepackage{ifthen}
\usepackage{mathrsfs}
\newtheorem{assump}{Assumption}
\newtheorem{assumpA}{Assumption}
\newtheorem{definition}{Definition}
\renewcommand\theassump{H\arabic{assump}}

\newcommand{\EE}{\mathbb{E}}
\newcommand{\EEMT}{\mathbb{E}_{M,T}}
\newcommand{\PP}{\mathbb{P}}
\newcommand{\utheta}{\overline{\widehat \theta}}
\newcommand{\ltheta}{\underline{\widehat \theta}}
\newcommand{\htheta}{\widehat \theta}
\newcommand{\HH}{\widehat H}
\newcommand{\TX}{\widetilde X^{(i)}}
\newcommand{\TL}{\widetilde L}
\newcommand{\Ttmh}{\frac{T_m^{(i)} - t}{h}}
\newcommand{\Tsmh}{\frac{T_m^{(i)} - s}{h}}

%%%%%%%%%% Commandes Omar
\newcommand{\Xtemp}[1]{%
\ifthenelse{\equal{#1}{0}}
{X^{(\n0)}}
{
\ifthenelse{\equal{#1}{1}}
{X^{[\n1]}}
{X^{(#1)}}
}}
\newcommand{\hXtemp}[1]{%
\ifthenelse{\equal{#1}{0}}
{\widehat X^{(\n0)}}
{
\ifthenelse{\equal{#1}{1}}
{\widehat X^{[\n1]}}
{\widehat X^{(#1)}}
}}

\newcommand{\Zz}{\boldsymbol z}
\newcommand{\Yy}{\boldsymbol y}
\newcommand{\Tt}{\boldsymbol t}
\newcommand{\Xx}{\boldsymbol x}
\newcommand{\TT}{\boldsymbol T}
\newcommand{\Ss}{\boldsymbol s}
\newcommand{\Uu}{\boldsymbol u}
\newcommand{\Vv}{\boldsymbol v}
\newcommand{\mO}{\mathcal O}
\newcommand{\cT}{\mathcal T}
\newcommand{\cU}{\mathcal U}
\newcommand{\Ynm}{{Y^{(j)}_m}}
\newcommand{\Tnm}{{\boldsymbol t^{(j)}_m}}
\newcommand{\Tnlm}{{T^{[n_1]}_m}}
\newcommand{\enm}{{\varepsilon^{(j)}_m}}
\newcommand{\unm}{{e^{(j)}_m}}
\newcommand{\Xp}[1]{X^{(#1)}}
\newcommand{\Xtp}[1]{\widetilde{X}^{(#1)}}
\newcommand{\Xc}[1]{X^{[#1]}}
\newcommand{\X}[1]{{\Xtemp{#1}}}
\newcommand{\Xt}[1]{{\Xtemp{#1}_t}}
\newcommand{\XT}[1]{{\Xtemp{#1}_\T}}
%\newcommand{\Xn}{{\Xtemp{i}}}
\newcommand{\Xn}{{\Xtemp{j}}}
\newcommand{\Xtn}{{\Xtemp{j}_t}}
\newcommand{\XTn}{{\Xtemp{j}_\T}}  
\newcommand{\D}{\rm{d}}
%%%%%%%%%%%%% Fin commandes Omar


\newcommand{\Ee}{\boldsymbol e}
\newcommand{\Eeps}{\boldsymbol \varepsilon}
\newcommand{\Eeta}{\boldsymbol \eta}
\newcommand{\Hh}{\boldsymbol H }
\newcommand{\Rplus}{\mathbb R_+}


\newtheorem{theorem}{Theorem}
\newtheorem{corollary}{Corollary}
\newtheorem{proposition}{Proposition}
\newtheorem{lemma}{Lemma}
\newtheorem{remark}{Remark}

\newcounter{assumptionHt}
\newcounter{assumptionH}
\newcounter{assumptionE}
\newcounter{assumptionLP}

\newenvironment{assumptionHt}%
{%
\begin{enumerate}[({G}1)]%
    \setcounter{enumi}{\value{assumptionHt}}%
    }{%
    \setcounter{assumptionHt}{\value{enumi}}%
\end{enumerate}
}
%
\newenvironment{assumptionH}%
{%
\medskip
\noindent\textbf{Assumptions.}
  \begin{enumerate}[({H}1)]%
  \setcounter{enumi}{\value{assumptionH}}%
}{%
  \setcounter{assumptionH}{\value{enumi}}%
  \end{enumerate}
}

\newenvironment{assumptionHbis}[1][]%
{%
\medskip
\noindent \textbf{Assumptions.}
#1
\begin{enumerate}[({H}1)]%
  \setcounter{enumi}{\value{assumptionH}}%
  }{%
  \setcounter{assumptionH}{\value{enumi}}%
\end{enumerate}
}

\newenvironment{assumptionE}%
{%
\medskip
\noindent \textbf{Assumptions.}
\begin{enumerate}[({E}1)]%
  \setcounter{enumi}{\value{assumptionE}}%
  }{%
  \setcounter{assumptionE}{\value{enumi}}%
\end{enumerate}
}

\newenvironment{assumptionLP}%
{%
\medskip
\noindent \textbf{Assumptions.}
\begin{enumerate}[({LP}1)]%
  \setcounter{enumi}{\value{assumptionLP}}%
  }{%
  \setcounter{assumptionLP}{\value{enumi}}%
\end{enumerate}
}

\newcommand{\assrefHt}[1]{(\hyperref[#1]{G\ref{#1}})}
\newcommand{\assrefH}[1]{(\hyperref[#1]{H\ref{#1}})}
\newcommand{\assrefE}[1]{(\hyperref[#1]{E\ref{#1}})}
\newcommand{\assrefLP}[1]{(\hyperref[#1]{LP\ref{#1}})}

\newcommand{\NN}{\mathbb{N}}
\newcommand{\RR}{\mathbb{R}}
\newcommand{\ZZ}{\mathbb{Z}}
\newcommand{\LL}{\mathbb{L}}

\newcommand{\T}{{t_0}}
\newcommand{\HT}{H_\T}
\newcommand{\KT}{\degree}
\newcommand{\ST}{\varsigma_\T}
\newcommand{\hHT}{\widehat{H}_\T}
\newcommand{\hKT}{\hdegree}
\newcommand{\hST}{\widehat{\alpha}_\T}
\newcommand{\Mmu}{\mathfrak{m}}


\title{Learning the regularity of multivariate functional data}
\author{Omar Kassi\footnote{Ensai, CREST - UMR 9194, France; omar.kassi@ensai.fr} \qquad	
	Nicolas Klutchnikoff\footnote{Univ Rennes, IRMAR - UMR 6625, France; Nicolas.klutchnikoff@univ-rennes2.fr}	\qquad
	Valentin Patilea\footnote{Ensai, CREST - UMR 9194, France; valentin.patilea@ensai.fr}
}
\date{\today}


\mathtoolsset{showonlyrefs}
\usepackage{xr}
\externaldocument{MVFD_SM}


\begin{document}

\maketitle

\begin{abstract}
Combining information both within and between sample realizations, we propose a simple estimator for the local regularity of surfaces in the functional data framework. The independently generated surfaces are measured with errors at possibly random discrete times. Non-asymptotic exponential bounds for the concentration of the regularity estimators are derived. An indicator for anisotropy is proposed and an exponential bound of its risk is derived. Two applications are proposed. We first consider the class of multi-fractional, bi-dimensional, Brownian sheets with domain deformation, and study the nonparametric estimation of the deformation. As a second application, we build minimax optimal, bivariate kernel estimators for the reconstruction of the surfaces.

	
	%%%%
	%%%%short version, less 100 words
	%%%%
	
\medskip	
	
	\textbf{Key words:} Concentration of estimators, Hölder exponent, Kernel smoothing, Random fields
	
	\textbf{MSC2020: } 62R10; 62G07; 62M99; 60G22
	%60XXX
	
\end{abstract}

%--------------------------------------------------------------------------


\bigskip

%--------------------------------------------------------------------------
% Figure environment removed

\section{Introduction}
Automatic 3D reconstruction of clothed humans using image inputs has gained increasing significance due to its potential applications in a wide array of AR/VR scenarios. High-fidelity reconstructions typically depend on sophisticated capture systems, which are developed with dense camera arrays~\cite{collet2015high,joo2015panoptic,joo2018total}, programmable light-stages~\cite{Vlasic2009, guo2019relightables}, and depth sensors~\cite{newcombe2011kinectfusion,DoubleFusion,BodyFusion,dou2016fusion4d,newcombe2015dynamicfusion}. However, stringent capture environments equipped with complex hardware pose significant challenges for consumer-level applications.


In this context, considerable research effort has been dedicated to developing methods that allow for more flexible capture configurations, such as utilizing a few RGB inputs. Among these works, learning implicit functions \cite{iccv2020PIFu, saito2020pifuhd, hong2021stereopifu} has proven effective in achieving highly detailed reconstructions by integrating the advancements of deep neural networks. These methods employ large multi-layer perceptrons (MLPs) to predict the occupancy probability or truncated signed distance function (TSDF) value of every queried 3D point based on its associated local feature, which is extracted from images. They can recover a continuous surface at arbitrary resolutions without topology restrictions.


However, in typical MLP-based implicit networks, the occupancy or TSDF value at each location is solved independently with planar image features, rendering them less capable of addressing challenging cases such as occlusions. Consequently, these methods suffer from generalization and robustness issues, particularly when tackling strong occlusions caused by large motion or multiple interacting humans. 
Some follow-up studies  \cite{zheng2021deepmulticap,zheng2021pamir,huang2020arch} utilize an extra geometric model, SMPL~\cite{Loper2015}, to improve robustness by introducing strong shape priors. 
Their success typically relies on the assumption of geometrical similarity \cite{huang2020arch} between the shape prior and target reconstruction, making them intractable for handling complex cases with loose clothes and sensitive to errors in SMPL model fitting.



%\ping{this paragraph sounds like `TSDF is better than MLP/SMPL, and we use TSDF to solve the problem'. But in Sec 3, we are telling a different story, saying `MLP needs a 3D convolutional encoder'. We need to make these two sections consistent.}\sicong{I think in this paragraph we claim that the TSDF}


%We opt for Trucated Signed Distance Funtion (TSDF) volumetric representations as they are naturally suitable for convolution operations, which have shown remarkable performance for learning hierarchical features on 2D visual perception tasks \cite{SunXLW19}. 
%Meanwhile, TSDF also describes the gradual geometry change around shape surface, which is not reflected by occupancy volume. 

We instead revisit the 3D volumetric representation and resort to 3D convolutional neural networks (CNNs) for feature learning, due to their impressive performance in feature learning and the ability to incorporate spatial context. However, volumetric methods and 3D convolution involve discretization, which might raise concerns regarding whether a discretized volume can preserve subtle geometric details as continuous representations learned in implicit functions. We investigate the relationship between volume resolution and quantization error on synthetic data by converting target mesh objects to TSDF volumes, as shown in Figure~\ref{fig:quantization_error}. We observe that the quantization errors are significantly reduced by increasing volume resolution and become nearly negligible when reaching a relatively high resolution (e.g., 512 or higher). In other words, achieving fine-detailed reconstruction is not supposed to be restricted by the use of volume representations as long as a proper volume resolution is utilized. Therefore, we present a method with high-resolution feature volumes, e.g., 256 and 512, while traditional volumetric methods \cite{varol18_bodynet,gilbert2018volumetric} are often limited to much lower resolutions, such as 32 or 128.



On the other hand, an increase in volume resolution may lead to a cubic growth of memory overhead \cite{8100085}. Reducing memory costs while guaranteeing the granularity of volumetric representations is necessary for pursuing high-quality reconstruction. Thus, we adopt a coarse-to-fine approach and cull away irrelevant voxels to build a sparse high-resolution feature volume. At the coarse level, the network computes an initial TSDF by applying a U-Net with sparse 3D CNN \cite{3DSemanticSegmentationWithSubmanifoldSparseConvNet} on the sparse feature volume, which is carved by a visual hull. Through our experiments, it turns out that more than 95\% of the volume grids are discarded by the visual hull culling, making the sparse 3D CNN efficient. At the fine level, the network focuses on a narrow band near the zero-level set of the initial TSDF and discretizes the narrow band with smaller voxels. By employing this narrow-band culling, we further shrink the sampling space, resulting in a relatively small range of grid numbers (usually 300K--500K in our experiments) even with a high volume resolution of 512. The remaining voxels in the narrow band are associated with features that fuse high-frequency information from the computed normal maps upon the low-frequency shape from the coarse level to compute the TSDF at high resolution. The final mesh is then extracted from the TSDF using the Marching-Cube algorithm ~\cite{Lorensen87marchingcubes}.
% Different from the u-net sturcture to preserve global topology context, we then apply a shallow 3dcnn to compute the final TSDF $D_{final}$ which contain more local geometry detail.




% \ping{this paragraph can be expanded. It is an important contribution and often ignored by other works. stress on the novel idea of regressing blending weights instead of colors}

In addition to geometry, high-quality mesh texture is also a crucial factor contributing to visual appearance. Directly computing a color field in 3D space, as in \cite{iccv2020PIFu}, struggles to capture high-frequency texture details, while the neural radiance field (NeRF) \cite{yu2020pixelnerf} or the DoubleField~\cite{shao2022doublefield} require expensive per-instance optimization and are often unstable for sparse input images. In contrast, we adopt an image-based rendering approach to compute a texture atlas map, which is efficient and widely supported in existing computer graphics tools. 
Specifically, we compute a blending weight at each 3D point on the mesh surface to determine its color as a weighted average of the colors at its image projections. The blending weights can be computed at a relatively coarse resolution, e.g., 512 volume resolution in our case, and leave texture details to the high-resolution images, such as 1K or 2K. Unlike previous methods that generate blurry texturing results under sparse input, our method generalizes well on both synthetic and real data with just a few input views. 
Figure~\ref{fig:teaser} shows two examples reconstructed by our method. Despite the challenging garment, pose, and occlusion, our method recovers faithful shape, normal, and texture on the right.

%with a wide variety of poses and clothing styles, and it is also adaptive to handle input image with arbitrary resolutions.
%\sicong{For this concern we claim that when the resolution of dicretized volume meets certain threshold (which is 256 in our experiment), the quantization error can be neglected.} 



In summary, the main contributions of this paper are as follows:
\begin{itemize}
\vspace{-0.1in}
  \item 
  We revisit the 3D volumetric representation and demonstrate that it can support clothed human reconstruction with equal or even better performance compared to implicit representation. 
  \item 
  We develop a memory and computation-efficient method for high-resolution volumetric reconstruction using sophisticated sparse 3D CNN, coarse-to-fine estimation, and voxel culling by visual hull and narrow bands. 
  \item 
  We introduce a novel method to compute a texture atlas map, which captures rich appearance details from high-resolution input images.
  \item 
  We achieve impressive results on standard benchmark datasets Twindom and MultiHuman, significantly reducing the point-2-surface (P2S) precision to approximately 0.2cm from just six input views, with more than $50\%$ error reduction compared to the state-of-the-art methods, including DoubleField~\cite{shao2022doublefield} and PIFuHD~\cite{saito2020pifuhd}.
\end{itemize}
% !TeX root = ../MVFD_arxiv.tex


\section{The framework}\label{sec2}
In this section we  present a formal mathematical setup for the local regularity  for bivariate   stochastic processes  (also called scalar random fields, or simply random fields) and the data observed for such processes.

\subsection{Data}\label{sec:data}
Consider $N$ independent realizations, also called sheets, $X^{(1)},\ldots,X^{(j)},\ldots X^{(N)}$   of a stochastic process $X $ defined on a continuous domain $\cT\in\mathbb R^2$. For simplicity, we here focus on domains $\mathcal T$ in the plane, the extension to higher dimensions would not raise different challenges. For the purpose of describing our methodology, we distinguish three observational scenarios of the $N$ realizations. 
First, the ideal, infeasible situation where the sheets $X^{(j)}$ are \emph{completely observed}, that is without error over the entire domain $\cT$.  
The second case is the one where the $X^{(j)}$ are observed (measured) at some \emph{discrete points} in the domain $\cT$, \emph{without noise}. The domain points can be fixed to be the same for all the $X^{(i)}$'s (common design), or can be randomly drawn for each sheets separately (independent design). Finally, the most realistic scenario is the one where in the second case we admit that the realizations of $X$  are observed at discrete domain points \emph{with noise}. 



 To formally describe the second and third scenarios, let  $M_1, \dotsc,M_N$ be an independent sample of an integer-valued random variable $M$ with expectation $\EE[M]=\Mmu$. 
% which increases with $N$. 
In the independent design case, for each $1\leq j \leq N$, and given  $M_j$, let $\Tnm\in\cT$,  $1\leq m \leq  M_j$, be a random sample of a random vector $\TT\in\cT$. The $\Tnm$'s represent the observation points for the realization $\Xp{j}$. We assume that the realizations of $X$, $M$ and $\TT$ are mutually independent. 
In the common design case, $M\equiv \mathfrak m$ and the $\Tnm$'s are the same for all $j$.  Let $\mathcal T_{obs}^{(j)} $ denote the set of observation times $ \Tnm $, $1\leq m \leq M_j$, on the sheet $\Xp{j}$. With common design,  $\mathcal T_{obs}^{(j)} $ does not depend on $j$, while with independent design the expected cardinal of $\mathcal T_{obs}^{(j)} $ can be random with mean $\mathfrak m$.   The following presentation  includes both independent design and common design cases. \color{black}  Finally, the data  consist of  the pairs  $(\Ynm , \Tnm ) \in\mathbb R \times \cT $ where $\Ynm$ is defined as
\begin{equation}\label{model_eq}
	\Ynm = \Xn (\Tnm) + \enm, \quad\text{with}  \quad  \enm = \sigma(\Tnm,\Xn(\Tnm)) \unm, 
	\quad 1\leq i \leq N,  \; 1\leq m \leq M_j.
\end{equation}
Here, the $\unm  \in\mathbb R $ are independent copies of a centered variable $e$ with unit variance, and $\sigma^2(\cdot,\cdot)\geq 0$ is some unknown, bounded conditional variance  function which account for possibly heteroscedastic measurement errors. The case  $\sigma(t,x)\equiv 0$ corresponds to our second scenario, while in the third scenario we have positive conditional variance. 

For each $1\leq j \leq N$,  let $\widetilde X^{(j)}$ denote an observable approximation of $X^{(j)}$. If   the sheets $X^{(j)}$ were  completely observed, as in our infeasible first scenario, $\widetilde X^{(j)} = X^{(j)}$. 
When $X^{(j)}$ are observed  only at  some discrete points $\Tnm$,  arbitrary  $\widetilde X^{(j)}(\Tt)$ can be obtained by simple interpolation or defined equal to the value of $\widetilde X^{(j)}$ at the nearest neighbor of $\Tt$.  Finally, with noisy, discretely observed sheets,   $\widetilde X^{(j)}$  is a pilot nonparametric estimator  of $X^{(j)}$, such as kernel smoothing, splines \emph{etc}.  


Let us next introduce a general class of stochastic processes (random fields)  $X$ with  irregular realizations $X^{(j)}$, for which the regularity can vary over the domain $\mathcal T$. 


\subsection{A class of multivariate processes}
Let $\mathcal{T}$ be an open, bounded bi-dimensional rectangle with the closure included in $(0,\infty)^2$. In the following, $H_1,H_2 : \mathcal T \to (0,1)$ are two continuously differentiable functions such that 
\begin{equation}\label{low_thres}
\min_{i=1,2} \inf_{t\in\mathcal T} H_i(\Tt) >0.
\end{equation} 	
Let $\overline{H} = \max\{H_1 ,H_2\}$.
We also consider the vector-valued function $\mathbf{L}=(L_1^{(1)},L_2^{(1)},L_1^{(2)},L_2^{(2)}),$  where the components  are  non-negative, Lipschitz  continuous functions defined on $\mathcal{T}$ such that 
\begin{equation}\label{id_L}
	L_i^{(1)}(\Tt) +L_i^{(2)}(\Tt) >0,\qquad \forall \Tt\in\mathcal T, \; i=1,2.
\end{equation}
%We denote $\mathbf{L}=(L_1^{(1)},L_2^{(1)},L_1^{(2)},L_2^{(2)}),$ and assume that a constant $C_{\mathbf L}$ exists such that 
%\begin{equation}\label{up_thres}
%C_{\mathbf{L}}=\max_{i,j=1,2} \sup_{t\in\mathcal T} L_j^{(i)}(\Tt) <\infty.
%\end{equation} 	

Let $X$ be a real-valued, second order stochastic process defined on $(0,\infty)^2$. Let $(e_1,e_2)$ be the canonical basis of $\mathbb R ^2,$ and, for sufficiently small scalars $\Delta$, let   
$$
\theta_{\Tt}^{(i)}(\Delta)=\EE\left[\left\{X\left(\Tt-\frac{\Delta}{2}e_i\right)-X\left(\Tt+\frac{\Delta}{2}e_i\right)\right\}^2\right],\quad i=1,2.
$$ 


\begin{definition}\label{def}
Let $H_1$, $H_2$ satisfy \eqref{low_thres}.	The class $\mathcal {H}^{H_1,H_2}(\mathbf{L},\mathcal{T})$ is the set of stochastic processes $X$ satisfying the following condition:  constants $  \Delta_0, C,\beta>0$ exist such that 
	for any $\Tt\in \mathcal T$ and $ 0<\Delta\leq\Delta_0$,  
	\begin{equation}\label{as_repr}
		\left|\theta_{\Tt}^{(i)}(\Delta)-L_1^{(i)}(\Tt)\Delta^{2H_1(\Tt)} -L_2^{(i)}(\Tt)\Delta^{2H_2(\Tt)}\right|\leq C\Delta^{2 \overline{H}(\Tt)+\beta}, \quad i=1,2.
	\end{equation}
	Let  
	$$
	\mathcal{H}^{H_1,H_2} %=\mathcal{H}^{H_1,H_2}(\mathcal{T})
	=\bigcup_{\mathbf{L}}\mathcal {H}^{H_1,H_2}(\mathbf{L},\mathcal{T}) ,
	$$
	where the union is taken over the set of four-dimensional functions $\mathbf{L}$ with non negative positives Lipschitz  continuous components satisfying \eqref{id_L} 
	%and \eqref{up_thres}. 
	The functions $H_1,H_2$ define the local regularity of the process, while $\mathbf{L}$ represent  the local Hölder constants.
\end{definition}

Definition \ref{def}  is general, and extends the local regularity notion considered by \cite{GKP} for processes defined on a compact interval on the real line.  A main example we have in mind is the  multi-fractional Brownian  sheet (MfBs) with a time-deformation. MfBs  is a generalization of the standard fractional Brownian sheet, where the Hurst parameter is allowed to vary along the  domain. The definition of this general class of  processes and some of their properties are provided in Section \ref{BfMs}.




% !TeX root = ../MVFD_arxiv.tex


\section{Local regularity estimation approach}\label{sec_*}
The idea is to relate the functional parameters $H_1,H_2$ and $\mathbf L$ to quantities which are estimable from the data. In other words, we build estimating equations for each of the parameters we want to estimate. The parameter estimate is then obtained from the sample version of the estimating equation.  

Before proceeding with the local regularity estimation, let us discuss on the identification aspect. Definition \ref{def} is too general and does not allow to identify all the unknown parameters without further restrictions. To be more clear, let $H_1,H_2,\widetilde H_1$ and $\widetilde H_2$  be  continuously differentiable functions  taking values in $(0,1).$ Assume that  $X\in \mathcal {H}^{H_1,H_2}(\mathbf{L},\mathcal{T})$ and  $X\in \mathcal {H}^{\widetilde H_1,\widetilde H_2}(\widetilde{\mathbf L},\mathcal{T})$, for some $\mathbf{L}$ and $\widetilde{\mathbf L}$. Then necessarily  
$$
\min\{H_1(\Tt),H_2(\Tt)\} \! = \! \min\{\widetilde H_1(\Tt),\widetilde H_2(\Tt)\}\hspace{0.1cm}\text{ and } \hspace{0.1cm}\max\{H_1(\Tt),H_2(\Tt)\} \! = \! \max\{\widetilde H_1(\Tt),\widetilde H_2(\Tt)\},
$$ 
 for any $\Tt\in \cT$, and, modulo a permutation of the components,   $\mathbf{L}\equiv \widetilde{\mathbf L}$. In general, the permutation depends on the domain point $\Tt$. We deduce from these facts that, for instance, only  
$$\underline H (\Tt)=\min\{H_1(\Tt),H_2(\Tt)\}\quad \text{  and  } \quad \overline{H}(\Tt)=\max\{H_1(\Tt),H_2(\Tt)\},
$$ 
are expected to be identifiable in the general framework we consider. Concerning the components of  $\mathbf{L}$,  the identifiable quantities are provided below.   

\subsection{ Estimating equations for $\underline{H}(\Tt)$ and $\overline{H}(\Tt)$}
Let  $X\in \mathcal H^{H_1,H_2}$. 
Since $\Delta^{b}$ is negligible compared to $\Delta^{a}$ if $0< a <b$ and $\Delta$ is small, in view of Definition \ref{def} we first define  the estimation equation for $\underline{H}(\Tt)$, for some fixed $\Tt\in \mathcal T$. 

For $i=1,2$ and $\Delta$ sufficiently small, we have
$$\theta_{\Tt}^{(i)}(\Delta)= K_1^{(i)}(\Tt)\Delta^{2\underline H (\Tt)}+O(\Delta^{\widetilde{\beta}})=K_1^{(i)}(\Tt)\Delta^{2\underline H (\Tt)}+K_2^{(i)}(\Tt)\Delta^{2\overline H (\Tt)}+O(\Delta^{\overline H (\Tt)+\beta}),$$
where 
$$K_1^{(i)}(\Tt) =\left\lbrace\begin{array}{lll}
	\!\!  L_1^{(i)} (\Tt)\quad &\!\!\! \text{if }  H_1(\Tt)< H_2(\Tt)\\
	\!\! L_2^{(i)}(\Tt)\quad &\!\!\!\text{if }   H_2(\Tt)< H_1(\Tt)\\
	\!\! L_1^{(i)}(\Tt)+L_2^{(i)}(\Tt) \quad &\!\!\!\text{if }   H_1(\Tt)=H_2(\Tt) 
\end{array}\right. \!\!, \quad K_2^{(i)}(\Tt) =\left\lbrace\begin{array}{lll}
	\!\!  L_1^{(i)} (\Tt)\quad &\!\!\! \text{if }  H_1(\Tt)> H_2(\Tt)\\
	\!\! L_2^{(i)}(\Tt)\quad &\!\!\!\text{if }   H_2(\Tt)>H_1(\Tt)\\
	\!\! 0\quad &\!\!\!\text{if }   H_1(\Tt)=H_2(\Tt) 
\end{array}\right. \!\!,
$$
and 
$$
\widetilde \beta =\left\lbrace\begin{array}{lll}
	\!\! 2\overline{H}(\Tt)& \text{if}&  \underline H(\Tt)< \overline{H}(\Tt) \\
	\!\! 2\underline H(\Tt)+\beta & \text{if}&  \underline H(\Tt)=\overline{H}(\Tt) 
\end{array}\right. ,\;\; \text{  } \;\; 
$$
Related to the previous discussion on the identifiability, 
similarly to the role of $\underline H (\Tt)$ and $\overline H (\Tt)$ for $H_1(\Tt)$ and $H_2(\Tt)$, the functions $K_1^{(i)}(\Tt)$ and $K_2^{(i)}(\Tt)$, $i=1,2$
are the identifiable functionals of $\mathbf L$. More precisely, given  the order choice in the case $H_1(\Tt)\neq H_2(\Tt)$, the functions $K_1^{(i)}(\Tt)$ and $K_2^{(i)}(\Tt)$ represent  the identifiable components of $\mathbf L$. When $H_1(\Tt)=H_2(\Tt) $, only the $ L_1^{(i)}(\Tt)+L_2^{(i)}(\Tt) $ are identifiable. See also the discussion following Proposition \ref{conc_Lest}. 


Next, we define $$\gamma_{\Tt}(\Delta)=\theta^{(1)}_{\Tt}(\Delta)+\theta^{(2)}_{\Tt}(\Delta).$$
The reason for considering this quantity, instead of considering separately $\theta^{(1)}_{\Tt}(\Delta)$ and $\theta^{(2)}_{\Tt}(\Delta)$, 
is that the Hölder constant associated to $\underline{H} (\Tt)$ can vanish, and this would prevent from estimating the lower regularity exponent. On contrary,  $\gamma_{\Tt}(\Delta)$ can be written as 
\begin{multline}\label{eq:K1K2}
\gamma_{\Tt}(\Delta)=\left(K_1^{(1)}(\Tt)+K_1^{(2)}(\Tt)\right)\Delta^{2\underline H (\Tt)} +
\left(K_2^{(1)}(\Tt)+K_2^{(2)}(\Tt)\right)
\Delta^{2\overline H (\Tt)}+ O(\Delta^{ 2\overline H (\Tt) +\beta})\\=: K_1(\Tt)\Delta^{2\underline H (\Tt)}
+ K_2(\Tt)\Delta^{2\overline H (\Tt)}+O(\Delta^{2\overline H (\Tt) +\beta}),
\end{multline}
and  condition \eqref{id_L} guarantees  $K_1(\Tt), K_2(\Tt)>0$, and thus allows  to consistently estimate $\underline H (\Tt)$. We also consider 
\begin{equation}\label{eq:alpha}
	\alpha_{\Tt}(\Delta)=\left|\frac{\gamma_{\Tt}(2\Delta)}{(2\Delta)^{2\underline{H}(\Tt)}}-\frac{\gamma_{\Tt}(\Delta)}{\Delta^{2\underline{H}(\Tt)}}\right|.
\end{equation}

\quad 

\begin{proposition}\label{proprox}
Let $X$ belong to the class $ \mathcal {H}^{H_1,H_2}(\mathbf{L},\mathcal{T})$, introduced by Definition \ref{def}. Then, for any $\Tt\in\cT$, 
	\begin{equation}\label{proxy_H_low}
		\underline{H}(\Tt) = \frac{\log(\gamma_{\Tt}(2\Delta))-\log(\gamma_{\Tt}(\Delta))}{2\log(2)} + O(\Delta^{\widetilde \beta -2\underline H(\Tt)}),
	\end{equation}
	and
	\begin{equation*}
		\overline{H}(\Tt)-\underline{H}(\Tt) = \frac{\log(\alpha_{\Tt}(2\Delta))-\log(\alpha_{\Tt}(\Delta))}{2\log(2)} + O(\Delta^{ \beta }).
	\end{equation*}
\end{proposition}

\smallskip


To estimate $\underline H (\Tt)$ we thus use the dominating term on the right-hand side of the  representation \eqref{proxy_H_low} as a proxy, for which we compute an estimate. 
To build a sample counterpart of the proxy quantity, we can estimate 
$\theta_{\Tt}^{(i)}(\Delta)$ by 
\begin{equation} \label{eq_theta_hat}
\widehat{\theta}_{\Tt}^{(i)}(\Delta)= \frac{1}{N}\sum_{j=1}^{N}\left\{\widetilde{X}^{(j)}(\Tt-(\Delta/2) e_i)-\widetilde{X}^{(j)}(\Tt+(\Delta/2) e_i)\right\}^2, \quad i=1,2,
\end{equation}
where $\widetilde X^{(j)}$ is the observable approximation of $\Xn$, In the ideal, infeasible scenario where    the sheets $X^{(j)}$ are completely observed, 
%as in our infeasible first scenario, 
$\widetilde X^{(j)} = X^{(j)}$. When $X^{(j)}$ are observed  only at  some discrete points $\Tnm$,  the  $\widetilde X^{(j)}(\Tt)$'s can be obtained by simple interpolation   or using nearest neighbors. 
Finally, with noisy, discretely observed sheets,   $\widetilde X^{(j)}$  can be a pilot nonparametric estimator  of $X^{(j)}$, such as bivariate kernel smoothing, splines \emph{etc}. 


An estimator of $\gamma_{\Tt}(\Delta)$ is then given by $\widehat{\gamma}_{\Tt}(\Delta)=\widehat{\theta}_{\Tt}^{(1)}(\Delta)+\widehat{\theta}_{\Tt}^{(2)}(\Delta).$ By plugging this estimator of $\gamma_{\Tt}(\Delta)$ into \eqref{proxy_H_low}, we obtain an estimator of $\underline{H}(\Tt)$~:
\begin{equation}\label{est_under}
	\widehat{\underline{H}}(\Tt)=\left \lbrace 
	\begin{array}{cl}
		\frac{\log(\widehat{\gamma}_{\Tt}(2\Delta))-\log(\widehat{\gamma}_{\Tt}(\Delta))}{2\log(2)}\quad &\text{if}\quad \widehat{\gamma}_{\Tt}(2\Delta),\widehat{\gamma}_{\Tt}(\Delta)>0\\
		1&\text{otherwise}
	\end{array}
	\right..
\end{equation}
Moreover, replacing $\gamma_{\Tt}$ by  $\widehat{\gamma}_{\Tt}$  and $\underline{H}(\Tt)$ by  $\widehat{\underline{H}}(\Tt)$ in \eqref{eq:alpha}, we get an estimator of $\alpha_{\Tt}$~: 
\begin{equation}\label{eq:alpha_hat}
	\widehat{\alpha}_{\Tt}(\Delta)=\left \lbrace 
	\begin{array}{cl}
		\left|\frac{\widehat{\gamma}_{\Tt}(2\Delta)}{(2\Delta)^{2\widehat{\underline{H}}(\Tt)}}-\frac{\widehat{\gamma}_{\Tt}(\Delta)}{\Delta^{2\widehat{\underline{H}}(\Tt)}}\right|\quad &\text{if}\quad \frac{\widehat{\gamma}_{\Tt}(2\Delta)}{(2\Delta)^{2\widehat{\underline{H}}(\Tt)}}\ne\frac{\widehat{\gamma}_{\Tt}(\Delta)}{\Delta^{2\widehat{\underline{H}}(\Tt)}}\\
		1&\text{otherwise}.
	\end{array}
	\right..
\end{equation}
Finally,  using the second part of Proposition \ref{proprox}, we estimator an estimator of $\overline{H}(\Tt)-\underline{H}(\Tt)$ under the form 
$$
\widehat{(\overline{H}-\underline{H})}(\Tt)=\frac{\log(\widehat{\alpha}_{\Tt}(2\Delta))-\log(\widehat{\alpha}_{\Tt}(\Delta))}{2\log(2)}.
$$

%We deduce from above an issue of interest is to know whether

It will be shown below that, for the pointwise estimation of $\overline{H}$ and $\mathbf L$,  we have to distinguish between the isotropic and anisotropic cases. Here, the \emph{isotropic} and \emph{anisotropic} cases are defined locally, and correspond to  $\underline{H}(\Tt)=\overline{H}(\Tt)$ and $\underline{H}(\Tt)< \overline{H}(\Tt)$, respectively. We herein propose an estimator of $\overline{H}(\Tt)$ which  adapts to isotropy. Let us consider  the  event 
\begin{equation}\label{def_A_N}
	 A_N(\tau)=A_N(\tau; \Tt)=  \left\{ \widehat{(\overline{H}-\underline{H})}(\Tt)\geq \tau\right\},
\end{equation}
for some appropriate, small threshold $\tau>0$. 
%In the case where $\widehat{(\overline{H}-\underline{H})}(\Tt)<\tau$, we can postulate that $\underline{H}(\Tt)=\overline{H}(\Tt).$
We then define  the following estimator for $\overline{H}(\Tt)$~:
\begin{equation}\label{est_over}
	\widehat{\overline{H}}(\Tt)=\widehat{\underline{H}}(\Tt)+\widehat{(\overline{H}-\underline{H})}(\Tt)\mathbf{1}_{A_N(\tau)}.
\end{equation}
 Here, for a set $S$, $\mathbf 1_S$ denotes the indicator of $S$.   In Section \ref{info_tau}, we provide an exponential bound  for the probability that the anisotropy detection rule defined by $\mathbf{1}_{A_N(\tau)}$ fails. In particular, that indicates how small $\tau $ is allowed to be such that $\mathbf{1}_{A_N(\tau)}$ detects anisotropy with high probability. 



\subsection{Estimating equations for $\mathbf{L}(\Tt)$}

Assume for the moment that $\underline{H}(\Tt)< \overline{H}(\Tt)$. A sample-based diagnosis   procedure for detecting this situation  can be built using the results in Section \ref{info_tau} below. Without 
loss of generality, we consider  $\underline{H}(\Tt)=H_1(\Tt)$.
Let us recall that, for $i=1,2$,
$$
\theta_{\Tt}^{(i)}(\Delta)=L_1^{(i)}(\Tt)\Delta^{2H_1(\Tt)} +L_2^{(i)}(\Tt)\Delta^{2H_2(\Tt)}+O(\Delta^{2{H}_2+\beta}).
$$



\begin{proposition}\label{prop_Lest} Let $ X\in \mathcal {H}^{H_1,H_2}$.
	%(\mathbf{L},\mathcal{T})$.
	Denote $D(\Tt)= {H}_2(\Tt)-{H}_1(\Tt) >0$.
	For $i=1,2$,
	\begin{equation*}
		L_1^{(i)}(\Tt)= \frac{\theta_{\Tt}^{(i)}(\Delta)}{\Delta^{2 H_1 (\Tt)}}+O(\Delta^{2{D}(\Tt)}),
	\end{equation*}
and
	\begin{equation*}
		L_2^{(i)}(\Tt)= \frac{1}{(2^{2D(\Tt)}-1)\Delta^{2D(\Tt)}}\left|\frac{\theta_{\Tt}^{(i)}(2\Delta)}{(2\Delta)^{ 2H_1(\Tt)}}-\frac{\theta_{\Tt}^{(i)}(\Delta)}{\Delta^{ 2H_1(\Tt)}}\right|+O(\Delta^\beta).
	\end{equation*}
\end{proposition}
We denote the estimators of the local Hölder constants by
\begin{equation}\label{est_Lcomp}
\widehat{L_1^{(i)}}(\Tt), \quad \widehat{L_2^{(i)}}(\Tt), \qquad i=1,2.
\end{equation}
The estimators of $L_1^{(i)}(\Tt)$, $i=1,2$, are obtained by plugging into its dominating term derived in Proposition \ref{prop_Lest} the estimators  in \eqref{eq_theta_hat}, \eqref{est_under}. For the estimators of $L_2^{(i)}(\Tt)$, we first consider  
$$
\widehat D(\Tt) =  \widehat{(\overline{H}-\underline{H})}(\Tt) \;\;\; \text{ if } \; \widehat{(\overline{H}-\underline{H})}(\Tt)\neq 0,\quad \text{ and } \; \widehat D(\Tt) = 0 \;\text{ otherwise}.
$$
If $\widehat D(\Tt) \neq  0$,  the estimators of $L_2^{(i)}(\Tt)$, $i=1,2$, are obtained by plugging into its dominating term the estimated quantities, otherwise they are set equal to zero. 


% !TeX root = ../MVFD_arxiv.tex



\section{Non-asymptotic results }\label{sec4}
We now derive concentration inequalities for the pointwise estimators of the  parameters $(H_1,H_2)$ and $\mathbf{L}= (L_1^{(1)},L_2^{(1)},L_1^{(2)},L_2^{(2)})$. For this purpose, we need to measure the error between each realizations $X^{(j)}$ of $X$ and its observable approximation of $\Xtp{j}$, as  defined in Section \ref{sec2}. We consider the following $\mathbb{L}^p$-risk~: 
$$
R_p (\mathfrak m) =\sup_{\Tt\in\cT }\EE[|\xi^{(j)}(\Tt)|^p] ,\qquad \xi^{(j)}(\Tt)=\Xtp{j}(\Tt)-\Xp{j}(\Tt).
$$
In general, the $\mathbb{L}^p$-risk depends on the expected number $\mathfrak m$ of observed points $\Tnm$. In the ideal scenario where $\Xp{j}$ is observed everywhere without error, $R_p \equiv 0$.
We also consider the following assumptions. Below, $B(\Tt; r)$ denote the ball of radius $r$ centered at $\Tt$. 

 


\begin{assumptionH}
	\item\label{ass_D}  Let $X$ belong to the class $ \mathcal {H}^{H_1,H_2}$,
	%(\mathbf{L},\mathcal{T})$, 
	introduced by Definition \ref{def}, and let $X^{(j)}$, $1\leq j \leq N$,  be independent realizations of $X$.
	
%		\item\label{ass_M} A constant $C_{\mathfrak m}$ exists such that 
%		$$
%		C_{\mathfrak m}^{-1} \leq \frac{M_j}{\mathfrak m}\leq C_{\mathfrak m},\quad \forall 1\leq j\leq N
%		$$		
	\item\label{ass_H1}  Three positive constants $\mathfrak{a}$, $\mathfrak{A}$ and $r$ exist such that, for any $\Tt\in\mathcal T$,  %(\textcolor{red}{introduire $\rho$})
	$$
	\EE\left| 	X^{(j)}\left(\Tt\right)-
	 X^{(j)}\left(\Ss
	 \right)\right|^{2p}\leq  \frac{p!}{2}\mathfrak{a} \mathfrak{A}^{p-2} \|\Tt-\Ss\|^{2p\underline H (\Tt)}
	\qquad \forall \Ss\in B(\Tt; r) ,\; \forall p\geq 1.
	$$
	
	\item\label{ass_H2} Two positive constants $\mathfrak{c}$ and $\mathfrak{D}$, and a function $\rho(\mathfrak m)\leq 1$, exist such that 
	$$
	R_{2p} (\mathfrak m) \leq  \frac{p!}{2}\mathfrak{c} \mathfrak{D}^{p-2}\rho(\mathfrak m)^{2p}, \qquad \forall p\geq 1,\; \forall \mathfrak m>1.
	$$
	\item\label{ass_H3} Two positive constants  $\mathfrak L $ and $\nu$ exist such that 
	$$
	R_2(\mathfrak m) \leq \mathfrak L \mathfrak m ^{-\nu},\qquad \forall \mathfrak m >1.
	$$
\end{assumptionH}

\smallskip

The condition in (H\ref{ass_H1}) imposes  sub-Gaussian  local increments for $X$.  It is satisfied by the processes in the wide class of
multi-fractional Brownian  sheet (MfBs) with a  domain-deformation, as considered  in Section \ref{BfMs}. In the case of noisy, discretely observed realizations $X^{(j)}$, the observable approximation can be obtained from existing bivariate nonparametric smoothing approaches. Under mild conditions, the standard nonparametric smoothers satisfy Assumption \ref{ass_H2}, with $\rho(\mathfrak m) =1$, and Assumption \ref{ass_H3}. See \cite{fan2016multivariate} for the case of local polynomials, and \cite{BELLONI2015} for general series estimators. In the second scenario, where the $X^{(j)}$ are observed  without noise  at discrete points $\Tnm$ in the domain $\mathcal T$, we can simply define $ \widetilde{X}^{(j)}(\Tt)$ as the value of $X^{(j)}$ at the nearest observed point $\Tnm$  to $\Tt$. To provide a simple justification that this simple choice is valid, let us consider that 
a constant $C>0$ exists such that 
$$
C^{-1} \leq  M_j/\mathfrak m\leq C,\quad \forall 1\leq j\leq N.
$$	
Then, with probability exponentially close to 1, there exists at least one point $\Tnm$ in the ball  $B(\Tt; \widetilde r)$, provided $\widetilde r = \mathfrak m^{-\delta}$, for some $\delta \in(1/2,1)$. 
Assumption \ref{ass_H2} is then implied by \ref{ass_H1} with    $\rho(\mathfrak m)=\mathfrak m^{-\delta\underline \beta}$, and $\underline \beta $ from \eqref{low_thres}.   In particular, this also guarantees \ref{ass_H3} with   $\nu = 2\delta\underline \beta$. 

 




\subsection{Concentration bounds for the regularity estimates}
% With  non-noisy sample paths observed everywhere.}

We first derive  the exponential bound for the concentration of the local regularity exponents.  On the one hand, the concentration will depend on the non-stochastic approximation error between
the true parameter and their respective dominating terms. From Proposition  \ref{proprox} these approximation errors are
\begin{equation*}
	R(\underline H )(\Tt) = 	\underline{H}(\Tt) - \frac{\log(\gamma_{\Tt}(2\Delta))-\log(\gamma_{\Tt}(\Delta))}{2\log(2)} ,
\end{equation*}
and
\begin{equation*}
	R(\overline H - \underline H )(\Tt) = \{\overline H -\underline H \}(\Tt) - \frac{\log(\alpha_{\Tt}(2\Delta))-\log(\alpha_{\Tt}(\Delta))}{2\log(2)},
\end{equation*}
respectively. We have
\begin{equation}\label{rates_eRR}
	R(\underline H )(\Tt)  = O(\Delta^{\widetilde \beta -2\underline H(\Tt)}) \quad \text{and} \quad 
	R(\overline H - \underline H )(\Tt) = 	 O(\Delta^{ \beta }).
\end{equation}
On the other hand, the concentration of the  local regularity exponents estimators will also depend on the error between the realizations of $X$ and their observable approximations $\widetilde X^{(j)}$. Finally, since we use Bernstein's inequality, the concentration will also depend on the bound of the moments  in Assumption \ref{ass_H1}. To account for these, let 
$$
\varrho(\Delta,\mathfrak m) = \max\{ \Delta^{2\underline H (\Tt)},\rho^{2}(\mathfrak m)\}^{-1}.
$$ 
Note that $\varrho(\Delta,\mathfrak m) = \Delta^{-2\underline H (\Tt)}$ in the ideal case where $\widetilde X^{(j)} = X^{(j)}$ and thus $\rho (\mathfrak m)=0$. 

\begin{proposition}\label{propCH}
Assumptions (H\ref{ass_D}) to  (H\ref{ass_H3}) hold true. Let $\widehat{\underline{H}}(\Tt) $  and $ \widehat{\overline{H}}(\Tt)$ be the estimators defined in \eqref{est_under} and \eqref{est_over}, respectively. If $\Delta$ is sufficiently small and $\mathfrak m$ sufficiently large,  constants $C_1,\dots,C_5$ exist such that,  
\begin{equation}\label{eq:cdt_eps}
\forall \varepsilon,\tau  \in (0,1) \quad \text{satisfying} \quad \max \{ |\log(\Delta)| |R(\underline H )(\Tt)|, \;|R(\overline H - \underline H )(\Tt)  |\}\leq \varepsilon \leq 2\tau,
\end{equation}
then
\begin{equation}\label{eq:conc-Hhat-around-H_main}
	\mathbb{P}\left[
	|\underline{\widehat{H}}(\Tt)-\underline{H}(\Tt)|\geq 
	\varepsilon 
	\right]
	\leq  p_1,
\end{equation}
and 
\begin{equation}\label{eq:concentration-overlineH_main}
	\PP\left[\left|\widehat{\overline{H}}(\Tt)-\overline{H}(\Tt)\right|\geq \varepsilon\right] 
	\leq C_3\{p_1+p_2+p_3\},
\end{equation}
with
\begin{align}
	p_1&= C_1\exp \left(-C_2N \times \varepsilon^2 \times \Delta ^{4\underline{H}(\Tt)}\varrho(\Delta,\mathfrak m)\right),\\
	p_2 &=	
	\exp\left[ - C_4N\times \varepsilon^2\times \frac{\Delta^{4\overline{H}(\Tt)}\varrho(\Delta,\mathfrak m)}{\log^2(\Delta)}\Delta^{4D(\Tt)}
	\right]\mathbf1_{\{\underline H(\Tt)<\overline H (\Tt)\}},
	\\ p_3&=  \exp\left[
	- C_5N\times \tau^2 \times \frac{\Delta^{4\overline{H}(\Tt)}\varrho(\Delta,\mathfrak m)}{\log^2(\Delta)}\Delta^{4D(\Tt)}
	\right],
\end{align}
where 
$$
D(\Tt)= \overline{H}(\Tt)-\underline{H}(\Tt) \quad \text{ and }  \quad 
\varrho(\Delta,\mathfrak m) = \max\{ \Delta^{2\underline H (\Tt)},\rho(\mathfrak m)^{2}\}^{-1}.
$$ 
\end{proposition}

\medskip 

The term $p_2$ is specific to the anisotropic case, it disappears when $\underline H(\Tt)=\overline H (\Tt)$. We next derive  the bounds for the concentration of the local Hölder constants' estimators. 
In the case where $\underline{H}(\Tt) \neq \overline{H}(\Tt)$, without loss of generality, we set
$$
\underline{H}(\Tt)=H_1(\Tt) < H_2(\Tt)= \overline{H}(\Tt),
$$
such that $L_1^{(1)}(\Tt)$ and $L_1^{(2)}(\Tt)$ are the Hölder constants corresponding to $\underline{H}(\Tt)$.
et 
\begin{equation*}
	R(L_1^{(i)})(\Tt) = 	L_1^{(i)}(\Tt) -\frac{\theta_{\Tt}^{(i)}(\Delta)}{\Delta^{2 H_1 (\Tt)}}=O(\Delta^{2{D}(\Tt)}),
\end{equation*}
and
\begin{equation*}
	R(L_2^{(i)})(\Tt) = 	L_2^{(i)}(\Tt) - \frac{1}{(2^{2D(\Tt)}-1)\Delta^{2D(\Tt)}}\left|\frac{\theta_{\Tt}^{(i)}(2\Delta)}{(2\Delta)^{ 2H_1(\Tt)}}-\frac{\theta_{\Tt}^{(i)}(\Delta)}{\Delta^{ 2H_1(\Tt)}}\right| = O(\Delta^\beta), \quad i=1,2.
\end{equation*}

\medskip

\begin{proposition}\label{conc_Lest}
Assume that the conditions of Proposition \ref{propCH} hold true. Then, for the estimators in \eqref{est_Lcomp}, positive constants $\mathfrak C_1,...,\mathfrak C_4$ exists such that,  for $i=1,2$, and \color{black}	
\begin{equation}\label{eq:cdt_epsL}
	\forall \varepsilon   \in (0,1)  \text{ satisfying }  \max \left\{|R(L_1^{(i)})(\Tt)|,\; |\log(\Delta)| |R(\underline H )(\Tt)|, \; |R(\overline H \!- \!\underline H )(\Tt)|  \right\}\leq \varepsilon ,
\end{equation}
 we have 

\begin{equation}\label{eq:conc_L1_main}
\PP\left(\left|\widehat{L_1^{(i)}}(\Tt)-L_1^{(i)}(\Tt)\right| \geq \varepsilon \right) \leq 
\mathfrak C_1 \exp\left(
	- \mathfrak C_2 N \times \varepsilon^2\times \frac{\Delta^{4\underline H(\Tt)}\varrho(\Delta,\mathfrak m)}{\log^2(\Delta)}
	\right).
\end{equation} 
 Moreover,  if in addition $ |R(L_2^{(i)})(\Tt)|\leq \varepsilon$, $i=1,2$ then
\begin{multline}\label{eq:conc_L2_main}
	\PP\left(\left|\widehat{L_2^{(i)}}(\Tt)-L_2^{(i)}(\Tt)\right|
	\geq \varepsilon \right) \\ \leq\mathfrak  C_3\exp\left(
	-\mathfrak C_4N\times \varepsilon\Delta^{4D(\Tt)}\min\{\varepsilon,\Delta^{4D(\Tt)}\}
	\times \frac{\Delta^{4\overline H (\Tt)}\varrho(\Delta,\mathfrak m)}{\log^4(\Delta)}\times 
(2^{2D(\Tt)}-1)^2	\right).
\end{multline} 
\end{proposition}


\medskip

The second exponential bound in Proposition \ref{conc_Lest} becomes trivial when 
$D(\Tt)=0$, and this reveals that the case $H_1(\Tt)=H_2(\Tt)$  requires special attention. 
In  this case,  the estimator proposed for  $L_1^{(i)}(\Tt)$  becomes an estimator of $L_1^{(i)}(\Tt)+L_2^{(i)}(\Tt)$, $i=1,2$. 
 The indicator of the set defined in \eqref{def_A_N} provides a tool for detecting whether $H_1(\Tt)=H_2(\Tt)$ or not, given a data set. In the following, we investigate  the risk associated to this diagnosis tool.


\subsection{A risk bound for the anisotropy detection}\label{info_tau}
Assume without loss of generality that $H_1(\Tt)\leq H_2(\Tt)$. 
Equation \eqref{eq:K1K2} then becomes 
\begin{multline}
\gamma_{\Tt}(\Delta)=\theta_{\Tt}^{(1)}(\Delta)+\theta_{\Tt}^{(2)}(\Delta)\\
=\left(L_1^{(1)}(\Tt)+L_1^{(2)}(\Tt)\right)\Delta^{2{H}_1(\Tt)} +\left(L_2^{(1)}(\Tt)+L_2^{(2)}(\Tt)\right)\Delta^{2{H}_2(\Tt)}+O(\Delta^{2{H}_2(\Tt)+\beta})\\ 
=K_1(\Tt)\Delta^{2H_1(\Tt)}+K_2(\Tt)\Delta^{2H_2(\Tt)} +O(\Delta^{2H_2(\Tt) +\beta}).
\end{multline}
We can now write
$$
\frac{\log(\alpha_{\Tt}(2\Delta))-\log(\alpha_{\Tt}(\Delta))}{2\log2}= D(\Tt)+O\left(\Delta^\beta\right).
$$
Therefore, if $D(\Tt)=H_2(\Tt)- H_1(\Tt)=0$, we get 
$$
\frac{\log(\alpha_{\Tt}(2\Delta))-\log(\alpha_{\Tt}(\Delta))}{2\log2}= O\left(\Delta^\beta\right).
%= o(\Delta^{a}),
$$ 
%for any $0<a<\beta$. 
We deduce that,  for the event $A_N(\tau)$
% $$
% A_N(\tau)=\left\{ \widehat{(\overline{H}-\underline{H})}(\Tt)\geq \tau\right\}, 
% $$
introduced in \eqref{def_A_N},  we have to  choose $\tau$ such that $\Delta = o(\tau^{1/\beta})$. The following result proposes an exponential bound for the risk associated to the rule defined by the indicator $\mathbf{1}_{A_N(\tau)}$ in the definition \eqref{est_over}. 




\medskip


\begin{proposition}\label{prop5_simple} Assume that the conditions of Proposition \ref{propCH} hold true. Let 
	$$  \max \{ |\log(\Delta)| |R(\underline H )(\Tt)|, \;|R(\overline H - \underline H )(\Tt)  |\}\leq  2\tau  \leq \left\{ \overline H(\Tt)-\underline H(\Tt)\right\} + \mathbf{1}_{\{\underline H(\Tt)=\overline H(\Tt)\}} .
	$$
If $\Delta$ is sufficiently small and $\mathfrak m$ sifficiently large, for $	A_N(\tau)$ defined in \eqref{def_A_N}, we have 
$$
%\PP(A_N(\tau))\mathbf{1}_{\underline H(\Tt)=\overline H(\Tt)}+\PP\left(\Bar{A}_N(\tau)\right)\mathbf{1}_{\underline H(\Tt)\ne\overline H(\Tt)}
\PP\left( \mathbf{1}_{A_N(\tau)}\neq \mathbf{1}_{\{\underline H(\Tt)<\overline H(\Tt)\}}\right)
\leq  C_3\exp\left[
	- C_5N\times \tau^2 \times \frac{\Delta^{4\overline{H}(\Tt)}\varrho(\Delta,\mathfrak m)}{\log^2(\Delta)}  \Delta^{4D(\Tt)}
	\right],
$$
where $ C_3$ and $ C_5$ are the positive constants defined as in Proposition \ref{propCH}.
\end{proposition}

For a choice of $\Delta$,  Proposition \ref{prop5_simple} allows to determine the rate of decrease for $\tau$ such that  the indicator of $A_N(\tau)$  detects with high accuracy whether $H_1(\Tt)=H_2(\Tt)$ or not. The fastest rate  depends on the approximation errors \eqref{rates_eRR}, which are characteristics of the process $X$. 

% !TeX root = ../MVFD_arxiv.tex

\section{Examples}\label{sec:example}

We propose two applications where our estimation approach of the local regularity for multivariate functional data opens the door to new procedures and sharp results. 


\subsection{Estimating the characteristics of general Gaussian processes}\label{BfMs}

The multifractional Brownian motion (MfBm) is a generalization of the standard fractional Brownian motion, where the  Hurst parameter is allowed to vary along the path. There are several possible definitions of such a process. They lead to indistinguishable processes, up to a multiplication by a deterministic function. Here, the multi-parameter,  anisotropic multifractional Brownian sheet, which is a multivariate extension, is defined following~\citet{herbin_06}. This definition relies on the so-called harmonizable representation of the MfBm, see \cite{peltier:inria}, \cite{benassi97}, \cite{ayache2011}, \cite{lebo2018} among others. 
\begin{definition}
	Set $d\in \mathbb{N}^\star$ and let $\Eeta=(\eta_1,\dotsc,\eta_d) : [0,\infty)^d\rightarrow (0,1)^d$ be a deterministic map. The multifractional Brownian sheet $W = (W(\Uu) : \Uu \in (0,\infty)^d)$ with Hurst functional parameter $\Eeta$ is defined as follows~:
	$$ 
	W(\Uu)= \left(\prod_{k=1}^d\frac{1}{C(\eta_k(\Uu))}\right)\int_{\mathbb{R}^d}\displaystyle\prod_{k=1}^d\frac{e^{i t_k\zeta_k}-1}{ |\zeta_k|^{\eta_k(\Uu)+\frac{1}{2}}}\widehat{\boldsymbol B}(\D \boldsymbol\zeta), \qquad  \Uu\in (0, \infty)^d,
	$$
	where $\boldsymbol  \zeta =(\zeta_1,\dots,\zeta_d)$ and $\widehat{\boldsymbol B} $ is the Fourier Transform of the white noise in $\mathbb{R}^d$. Here, for any positive $x$,
	$$
	C(x) = \left[ \frac{2\pi}{\Gamma(2x+1)\sin(\pi x)}\right]^{1/2}.
	$$
\end{definition}

Notice that, when $d=1$, the measure $\widehat{\boldsymbol B}(\D \boldsymbol\zeta)$ is the unique complex-valued Gaussian measure which can be associated to a standard Gaussian measure on $\RR$ by a `stochastic Parseval identity', see~\citet{stoev_taquu_2006}, equation~(2.4). In particular, the construction of $\widehat{\boldsymbol B}(\D \boldsymbol\zeta)$ ensures that $W$ is real-valued. 

We focus on the case $d=2$, and redefine $W=(W(\Uu) : \Uu\in\cU)$ as the restriction to an open subset $\cU\subset (0, \infty)^2$ of the multifractional Brownian sheet with Hurst functional parameter $\Eeta=(\eta_1, \eta_2)$. 
Note that $W$ is a centered Gaussian process with covariance function
\begin{equation*}
	\EE[W(\Uu)W(\Vv)]
	\!=\! \prod_{ k =1,2}\!\!
	D(\eta_{k }(\Uu),\eta_{k }(\Vv))
	\left[u_{k  }^{\eta_{k  }(\Uu)+\eta_{ k}(\Vv)}\!+\!v_{k }^{\eta_{ k  }(\Uu)\!+\!\eta_{ k  }(\Vv)}\!-|u_{ k }-v_{k }|^{\eta_{ k }(\Uu)+\eta_i(\Vv)}\right],
\end{equation*}
$\Uu =(u_1,u_2),\Vv=(v_1,v_2)\in\cU$, where 
\begin{equation}\label{def_D_func}
	D(x,y) = C^{\,2}((x+y)/2)\cdot(2C(x)C(y))^{-1} \;\;\; \text{ and } \;\;\;  D(x,x)\equiv 1/2.
\end{equation}
In particular, the variance of $W$ is given by $\EE[W^2(\Uu)]=u_1^{2\eta_1(\Uu)}u_2^{2\eta_2(\Uu)}$.

Moreover, we consider a domain deformation $A$, that is a positive  and invertible application $A:\cT\to \cU$. Let  
$$
X= W\circ A \quad \text{ and } \quad \theta(\Tt,\Ss)= \EE\left[\{X(\Tt)-X(\Ss)\}^2\right], \quad \forall \Tt,\Ss\in \cT.
$$



\begin{proposition}\label{mprop}
	 If  $\Eeta:\cU\rightarrow (0,1)^2 $  and $A:\cT\to \cU$   are continuously differentiable,
	\begin{multline*}
		\theta(\Tt, \Ss)
		= |A_1(\Tt)|^{2H_1(\Tt)}|\partial_1A_2(\Tt)(t_1-s_1)+\partial_2A_2(\Tt)(t_2-s_2)|^{2H_2(\Tt)}\\
		+|A_2(\Tt)|^{2H_2(\Tt)}|\partial_1A_1(\Tt)(t_1-s_1)+\partial_2A_1(\Tt)(t_2-s_2)|^{2H_1(\Tt)}+O(\|\Tt-\Ss\|^2)\\
		+O(\|\Tt-\Ss\|^{2\underline{H}(\Tt)+1})+O\left(\|\Tt-\Ss\|^{2H_1(\Tt) +2H_2(\Tt)}\right),
		\qquad
		\Tt, \Ss \in\cT,
	\end{multline*}
	where $\partial_1, \partial_2$ denote the partial derivatives and 
	$$H_1=\eta_1\circ A\quad \text{and}\quad H_2=\eta_2\circ A. $$
\end{proposition}

The proof of the following corollary is immediate, and will thus be omitted.


\begin{corollary}\label{mycoro}
	Assume the conditions of Proposition\ref{mprop}, and that there exist $\rho\in(0,1)$ such that 
	$$
	%\exists \rho >0,\quad \quad 
	0\leq \overline{H}(\Tt)-\underline{H}(\Tt)\leq \frac{1-\rho}{2}.$$
	Then  $X=W\circ A \in \mathcal{H}^{H_1,H_2}$
	with  $\mathbf L$ given  by~: 
	$$
	L_1^{(1)}(\Tt)= |A_2(\Tt)|^{2H_2(\Tt)}|\partial_1 A_1(\Tt)|^{2H_1(\Tt)},\quad L_2^{(1)}(\Tt)=|A_1(\Tt)|^{2H_1(\Tt)}|\partial_1 A_2(\Tt)|^{2H_2(\Tt)},$$
	$$L_1^{(2)}(\Tt)=|A_2(\Tt)|^{2H_2(\Tt)}|\partial_2 A_1(\Tt)|^{2H_1(\Tt)},\quad L_2^{(2)}(\Tt)=|A_1(\Tt)|^{2H_1(\Tt)}|\partial_2 A_2(\Tt)|^{2H_2(\Tt)} .
	$$
\end{corollary}

 Let us note that without domain deformation, \emph{i.e.}, when $A$ is the identity, $\mathbf L=(1,0,0,1)$.  The estimation approach introduced in Section \ref{sec_*} allows to estimate $H_1$, $H_2$ and $\mathbf L$  in general. The  estimation of the domain deformation $A$ is  investigated in the following.


\subsubsection {Estimating equations for the domain  deformation}
When one realization of the process is observed on a dense, regular grid, the estimation of the Hurst function of a multifractional Brownian motion was considered by \cite{hsing2020}. See also \cite{hsing2016}.The use of deformation to model non-stationary processes was first introduced  to the spatial statistics literature by \cite{sampson92}. One dimensional deformations behave locally  as a change of scale. In two dimension, deformations can rotate,  as well as scale local coordinates. See \cite{anderes2008}, \cite{anderes2009consistent} and \cite{Clerc2003} for more details. The fact that the deformation can rotate is mainly  related to the identification problem discussed in Section \ref{sec_*}. 


As a consequence of our  new approach,  we can build a nonparametric estimator of the deformation $A$ under mild technical conditions. We consider that 
\begin{equation}\label{ini_cd}
	 \text{some   $(t_0,s_0)\in\cT$  is  given for which  $A_1(t_0,s_0)$ and $A_2(t_0,s_0)$ are known. }
\end{equation}
This  initial condition avoids identification issues arising in a fully non parametric setup. We also  assume that the time-deformation $A$ 
is such that 
\begin{equation}\label{simpl_A}
\inf_{\Tt\in\cT} A_k(\Tt) >0, \quad  \inf_{\Tt\in\cT} \partial_{i} A_k(\Tt) \geq 0 \quad \text{ and } \quad \inf_{\Tt\in\cT} \{\partial_{1} A_k(\Tt) +\partial_{2} A_k(\Tt) \} >0, \qquad    i,k=1,2. 
\end{equation}
%where $ \partial_{i}$ stands for the partial derivative.
Finally, we set $H_1(\Tt)<H_2(\Tt)$  and focus on the first coordinate $A_1$ of the deformation $A$.  By Corollary \ref{mycoro}, we have   
$$
L_1^{(1)}(\Tt)= A_2(\Tt)^{2H_2(\Tt)}\partial_1 A_1(\Tt)^{2H_1(\Tt)}.
$$
Since the variance of $X$ is given by
\begin{equation} \label{rel_v_A}
v(\Tt)=\EE[X(\Tt)^2]=A_2(\Tt)^{2H_2(\Tt)}A_1(\Tt)^{2H_1(\Tt)},
\end{equation}
it follows that 
$$ \left(\frac{L_1^{(1)}(\Tt)}{v(\Tt)}\right)^{\frac{1}{2H_1(\Tt)}}=\frac{\partial_1A_1(\Tt)}{A_1(\Tt)}.$$
Integrating both sides we obtain
$$
\log A_1(\Tt)=\int_{t_0}^{t_1} f_1(s,t_2){\D} s+h(t_2),\quad \text{ for} \quad \Tt=(t_1,t_2)\in \cT,
$$
where $h$ is a real-valued function of $t_2$ and 
$$
f_1(\Tt)= \left(\frac{L_1^{(1)}(\Tt)}{v(\Tt)}\right)^{\frac{1}{2H_1(\Tt)}}.
$$
The function   $h$ is determined by  
$$
\frac{h^\prime(t_2)}{h(t_2)}=g_1(t_0,t_2):=\left(\frac{L_1^{(2)}(t_0,t_2)}{v(t_0,t_2)}\right)^{\frac{1}{2H_1(t_0,t_2)}}.
$$
This leads us to the following estimating equation~: 
\begin{equation}\label{est_A1}
A_1(\Tt)=\lambda_1\exp\left(\int_{t_0}^{t_1} f_1(s,t_2){\D} s+\int_{s_0}^{t_2}g_1(t_0,s)\D s\right),
\quad \text{ where } \quad \lambda_1=A_1(t_0,s_0).
\end{equation}
An estimator $\widehat{A}_1$ of    the first component of the domain deformation is easily obtained  
by replacing $f_1$ and $g_1$ by their estimates in \eqref{est_A1}. Estimators of $f_1$ and $g_1$ are  naturally obtained by plugging into their expressions  the estimators of    $L_1^{(1)}$, $L_1^{(2)}$,  $H_1$ and  an estimator $\widehat v(\Tt)$ of the variance $v(\Tt)$.


To provide a theoretical result for $\widehat{A}_1$, for simplicity, in addition to \eqref{low_thres}, we assume that constants $\underline\beta$, $\overline\beta$ are known such that 
\begin{equation}\label{simpl_bet}
0< \underline\beta \leq \min_{k=1,2} \inf_{\Tt\in\mathcal T} H_k(\Tt) \qquad \text{ and } \qquad %C_{\mathbf L} 
 \max_{i=1,2}\sup_{\Tt \in \cT} L_1^{(i)}(\Tt) \leq \overline \beta . 
\end{equation}
We then truncate correspondingly  the estimators, \emph{i.e.}, we replace  $\widehat H_k(\Tt)$ and $\widehat{L^{(i)}_1}(\Tt)$  by 
$$
\max \{\widehat H_k(\Tt), \underline \beta \}  \qquad \text{ and  }\qquad  \min\left\{\widehat{L^{(i)}_1}(\Tt), \overline\beta\right\},\qquad \forall \Tt\in \cT,\; k,i=1,2, 
$$  
respectively. Given the relationship \eqref{rel_v_A} and  condition \eqref{simpl_A}, 
the variance $v(\Tt)$ is necessarily  bounded away from zero.
Finally,   for  the estimator of $v(\Tt)$, we assume that, a constant $C_v$ exists such that 
\begin{equation} \label{simpl_v}
\sup_{\Tt\in \cT} \mathbb E\left[ \{v(\Tt)/ \widehat v (\Tt)\}^p\right]	 < C_v^p, \qquad \forall p\geq 1 .
\end{equation}
This  condition   can be satisfied  if, for instance, a positive lower bound $\underline a$ for $A_1$ and $A_2$ is known in \eqref{simpl_A}. By \eqref{rel_v_A}, we then have 
$$
v(\Tt)> \underline v := \min (\underline a ^4,1).
$$
In this case,  $\widehat v (\Tt)$ can be simply defined as maximum between $\underline v$ and the empirical second order moment of the observable approximations $\widetilde X^{(j)}$.

Let
		$$
		F_1 :=\sup_{\Tt\in \cT}\EE\left[|\widehat f_1(\Tt)-f_1(\Tt)|\right]  ,\qquad  G_1 := \sup_{\Tt\in \cT}\EE\left[|\widehat g_1(\Tt)-g_1(\Tt)|\right]  , 
		$$ 
and $\operatorname{diam}(\mathcal T)= \sup_{\Ss^\prime,\Ss\in \cT}\|\Ss^\prime -\Ss\|$.  For the next result,  let  $\Delta = \mathfrak m^{-a}$ and $\rho(\mathfrak m) = \mathfrak m^{-b}$, with $a>0$, $b\geq 0$ and $\rho(\mathfrak m)$ introduced in Assumption \ref{ass_H2}. Moreover, let  
$$
  \chi (\Tt) = a\{2D(\Tt)+1\} - \min\{a\underline H (\Tt), b/2 \} >0. 
$$

\begin{proposition}\label{prop_def_A}
The assumptions of Propositions  \ref{propCH} and \ref{mprop}, and conditions \eqref{ini_cd}, \eqref{simpl_A}, \eqref{simpl_bet} and \eqref{simpl_v} hold true. Moreover, we assume that  constants $\mathfrak a_1$ and $\mathfrak A_1$ exist such that 
$$
\EE[X(\Tt)^{2p}]\leq \frac{p!}{2}\mathfrak a_1\mathfrak A_1^{p-2},\quad \forall p\in\{1,2\ldots\}.
$$ 
Then $F_1+G_1 <\infty$. 
Moreover, let 
\begin{equation}\label{cdt_ell}
\left[ 1/2-   \chi (\Tt)   \liminf_{\mathfrak m, N} \{\log(\mathfrak m)/\log(N)\}\right]_+<\ell < 1/2.
\end{equation}
Then, if $\mathfrak m$ ad $N$ are sufficiently large, positive constants $\mathfrak C_v$, $\tilde q_1$ and $\tilde q_2$ exist such that, 
\begin{multline}		\EE \left[\left|\widehat A_1(\Tt)-A_1(\Tt) \right| \right]\\  \leq \mathfrak C_v A_1(\Tt) \operatorname{diam}(\mathcal T)  \max\left\{1,F_1,G_1\right\}  \left\{  
 \frac{a\log(\mathfrak m) }{	\mathfrak m^{  2aD  (\Tt) - \chi (\Tt)   } } 
	N^{\ell-1/2}+\tilde q_1
	\exp(-\tilde q_2N^\ell) \right\}.
\end{multline}
\end{proposition}

A similar representation can be derived for $A_2$, that is 
\begin{equation}\label{est_A2}
	A_2(\Tt)=\lambda_2\exp\left(\int_{t_0}^{t_1} f_2(s,t_2){\D} s+\int_{s_0}^{t_2}g_2(t_0,s)\D s\right),
\end{equation}
where  
$$
f_2(\Tt)= \left(\frac{L_2^{(1)}(\Tt)}{v(\Tt)}\right)^{\frac{1}{2H_2(\Tt)}},\quad g_2(t_0,t_2)= \left(\frac{L_2^{(2)}(t_0,t_2)}{v(t_0,t_2)}\right)^{\frac{1}{2H_2(t_0,t_2)}}\quad \text{and}\quad \lambda_2 =A_2(t_0,s_0).
$$
Estimators of $f_2$ and $g_2$,  are  obtained by plugging into their expressions  the estimators of $L_2^{(i)}$,  $i=1,2$,  $H_2$, and an estimator of  $v(\Tt)$. Let $\widehat A_2$ be the estimator of $A_2$ obtained by plug-in using \eqref{est_A2}. Under the conditions of Proposition \ref{prop_def_A}, we can show that $F_2+G_2 $ is finite and derive a similar bound for the $\mathbb L^1-$risk of $\widehat A_2$. The arguments are similar and thus omitted. 

 
 
 
%\input{sections/exampleG_inv}
% !TeX root = ../MVFD_arxiv.tex


\subsection{Adaptive optimal bivariate smoothing }\label{sec6}

Let us consider the problem of nonparametric pointwise estimation of a 2-dimensional  aniso\-tropic regression function from a class of functions which are $\gamma_i-$Hölder continuous in the direction $e_i$, with $\gamma_i\in (0,1]$,  $i=1,2$. It is well-known that, under some conditions on the noise and given an iid sample of size $M_0$, 
the minimax rate of convergence for the estimation of a regression function $f$  over the class is
$$
M_0^{-\frac{\boldsymbol \gamma}{2\boldsymbol \gamma+1}},
$$
where the effective smoothness $\boldsymbol {\gamma}$ is defined by the formula 
$$
\frac{1}{\boldsymbol \gamma}=\frac{1}{\gamma_1}+\frac{1}{\gamma_2}.
$$
See \cite{lep_hoff}, \cite{opt_var}, \cite{GK}, \cite{dun_aniso}. 


In the context of multivariate functional data, a natural issue is the reconstruction of the realizations of $X$ using the data.
To match the standard nonparametric regression setup, we hereafter consider the case where the set $\mathcal{H}^{H_1,H_2}$ in Definition \ref{def} is built with the restriction $\boldsymbol L = (L_1,0,0,L_2)$. Fortunately,  the local regularity of a process $X\in \mathcal H^{H_1,H_2}$  is intrinsically linked to the regularity of the sample paths of the process.
Let $\Tt $ be some fixed point in the domain $\cT$, and assume that 
\begin{equation}\label{Yor_us}
	\max_{i=1,2}\sup_{0<\Delta \leq \Delta_0}  \frac{\EE\left[\left\{X\left(\Tt-\Delta e_i/2\right)-X\left(\Tt+\Delta e_i/2\right)\right\}^{2p}\right]}{\EE\left[\left\{X\left(\Tt-\Delta e_i/2\right)-X\left(\Tt+ \Delta e_i/2\right)\right\}^2\right]^p}<\infty, \qquad \forall  p\in \mathbb N.
\end{equation}
By \citet[Theorem 2.1, page 26]{Yor}, 
almost  any realization of $X$  is locally $\alpha-$Hölder continuous in the direction $e_i$, for any order $0\leq \alpha < H_i(\Tt)$. See also Lemma SM.\ref{reg_RY_SM} in the Supplementary Material. 




%This fact leads to the so-called \textit{adaptive procedures}, where the estimating procedure need to adapt to the regularity of the target function $f$ (see \textcolor{red}{*****}.....). 


Let us notice that, with the simplified structure of $\boldsymbol L$ in the definition of $ \mathcal{H}^{H_1,H_2}$,  the  identification problem mentioned in Section \ref{identification} no longer occurs, and we have
\begin{equation}\label{main_eq}
	\theta_{\Tt}^{(i)}(\Delta)= L_i(\Tt)\Delta^{2H_i(\Tt)}+O(\Delta^{2\overline{H}(\Tt)+\beta}),\quad i=1,2.
\end{equation}
Following the methodology introduced in Section \ref{sec_*},  we  consider  the  estimating equations for the local regularity exponents~:
\begin{equation*}
	H_i(\Tt) = \frac{\log(\theta^{(i)}_{\Tt}(2\Delta))-\log(\theta^{(i)}_{\Tt}(\Delta))}{2\log(2)} + O(\Delta^{ \beta }),\quad i=1,2.
\end{equation*}
Applying these equations with a learning set of realizations of $X$, we get the estimators  
$\widehat H_i(\Tt)$. 

Consider a new  realization
$X^{new}$ of $X$, from which we observe   $(Y^{new}_m,\Tt^{new}_m), 1\leq m\leq M_0$ with 
\begin{equation}
	Y^{new}_m=X^{new}(\Tt^{new}_m)+\varepsilon^{new}_m,\quad \quad 1\leq m\leq M_0.
\end{equation}
Here, $M_0$ is a realization of the variable $M$, while the $\Tt^{new}_m$ are independent realizations of the bi-dimensional vector $\boldsymbol T$, with $M$ and $\boldsymbol T$  introduced in Section \ref{sec:data}. We propose to use the Nadaraya-Watson estimator to estimate $X^{new}(\Tt)$, and we consider the simpler version with two bandwidths. Formally, let $K:\mathbb R ^2\to \Rplus$ be a density with the support in $[-1,1]\times [-1,1]$,  and $\mathbf{B}=\operatorname{diag}(1/h_1,1/h_2)$ a positive, $2\times 2$ bandwidth matrix. Considering the 2-dimensional vectors $\Tt$ and $\Tt^{new}_m$ as column matrices, the Nadaraya-Watson estimator is then given by 
$$
\widehat X^{new}(\Tt;\mathbf{B} )=\sum_{m=1}^{M_0}Y^{new}_m\frac{K\left(\mathbf{B}(\Tt^{new}_m-\Tt)\right)}{\sum_{m=1}^{M_0}K\left(\mathbf{B}(\Tt^{new}_m-\Tt)\right)}.
%:=\sum_{m=1}^{M_0}Y^{new}_mW_m(\Tt).
$$
To achieve the optimal rate of convergence, the bandwidths have to be selected  according to the regularity of the sheet. 


For deriving the properties of $\widehat X^{new}(\Tt)$, we impose the following mild assumptions.



\begin{assumptionLP}
	\item\label{LP1} Two positive constants $\kappa$ and $r$ exist such that  
	$$\kappa^{-1}\mathbf{1}_{B(0,r)}(\Tt)\leq K(\Tt)\leq\kappa ,\quad \forall \Tt\in \cT. $$
	$h_1,h_2\in\mathcal H$ with $\mathcal H$ a bandwidth range satisfying $\sqrt{\mathfrak m} \inf  \mathcal H \rightarrow \infty$ and $\sup  \mathcal H \rightarrow 0$. 
%	where $B(0,r)$ denote the ball of center 0 and radius $r$.
	
	\item\label{LP2} A constant $c$ exists such that $f_{\mathbf{T}}(\Tt)\geq c>0$,  $\forall \Tt\in \cT$, where $f_{\mathbf{T}}$ is the density function of the random vector $\boldsymbol{T}$ that generated the independent copies  $\Tt^{new}_m$, $1\leq m\leq M_0 $.
	
	\item\label{LP3e} 
	The error terms $\varepsilon^{new}_m$ are iid,  zero mean random variables with  constant variance $\sigma^2$. The variables  $M_0$, $X^{new}$, $\Tt^{new}_m$, and $\varepsilon^{new}_m$,  $1\leq m\leq M_0 $, are mutually independent. 	  A constant $c>0$ exists such that $c^{-1}\leq M_0/\mathfrak m \leq c$. 	
	
	
	\item \label{LP4} The estimators $\widehat H_i(\Tt)$ and $\widehat L_i(\Tt)$ are independent of the variables $M_0$, $X^{new}$, $\Tt^{new}_m$, and $\varepsilon^{new}_m$. Moreover,  
	$$
	\PP\left(\max\left\{|\widehat H_i(\Tt)-H_i(\Tt)|, |\widehat L_i(\Tt)-L_i(\Tt)|\right\}>\log^{-a}(\mathfrak m)\right)\leq \mathfrak k_1 \exp \left(-\mathfrak m\right), \qquad i=1,2,
	$$
	where $\mathfrak k_1$ is some positive constant and $a>1$.
\end{assumptionLP}

In view of our result from Section \ref{sec4},  condition LP\ref{LP4} holds true under mild conditions. 
Let us consider the pointwise, conditional mean square risk of $\widehat X^{new}$,  given the integer $M_0$, that is
$$
\mathcal R \left( \Tt;\textbf B, M_0\right)=\EE\left[\left\{\widehat X ^{new}(\Tt;\textbf B)-X^{new}(\Tt)\right\}^2\Big{|} M_0\right].
$$
We first derive a bound of this risk when $H_1$, $H_2$ and $L_1,L_2$ are given. 




\medskip

\begin{proposition}\label{prop_risk1}
	Assume that  (LP\ref{LP1}), (LP\ref{LP2}) and (LP\ref{LP3e})  hold true. Then 
	$$
	\mathcal{R}(\Tt; \textbf B,M_0)\leq \frac{\kappa^2}{c\pi}\frac{\sigma^2}{M_0 h_1h_2}+ 2L_1(\Tt)h_1^{2
		H_1(\Tt)}+ 2L_2(\Tt)h_2^{2H_2(\Tt)}+\text{ negligible terms.}
	$$
\end{proposition}



\medskip


Minimizing the dominating terms in the upper bound of the risk yields  optimal  bandwidths. This choice of the bandwidths, and the resulting risk rate,  will depend on the regularity of the process and the Hölder constants. These facts are gathered in the following result. Let
$$
\mathcal H (\Tt) = 2H_1(\Tt)H_2(\Tt)+H_1(\Tt)+H_2(\Tt).
$$


\medskip

\begin{corollary}\label{cor_ad}
	The minimum of the dominant terms in the risk bound in Proposition \ref{prop_risk1} is attained at $(h_1^*, h_2^*)$, with 
	$$
	h^*_1=
	\left(\frac{1}{M_0}\right)^
	{\frac{H_2(\Tt)}{\mathcal H (\Tt)}}
		%{2H_1(\Tt)H_2(\Tt)+H_1(\Tt)+H_2(\Tt)}}
		 \left(\frac{\Lambda_1(\Tt)^{2H_2(\Tt)+1}}{\Lambda_2(\Tt)}\right)^{\frac{1}{2\mathcal H (\Tt)}}
		 	%{4H_1(\Tt)H_2(\Tt)+2H_1(\Tt)+2H_2(\Tt)}},
	\quad 
	\text{ and } \quad 
	h^*_2=\left(\frac{1}{M_0}\right)^{\frac{H_1(\Tt)}{\mathcal H (\Tt)}}
		%{2H_1(\Tt)H_2(\Tt)+H_1(\Tt)+H_2(\Tt)}}
	\left(\frac{\Lambda_2(\Tt)^{2H_1(\Tt)+1}}{\Lambda_1(\Tt)}\right)^{\frac{1}{2\mathcal H (\Tt)}},
		%{4H_1(\Tt)H_2(\Tt)+2H_1(\Tt)+2H_2(\Tt)}},
	$$
	where $\Lambda_i(\Tt)=\kappa^2\sigma^2/\{4c\pi H_i(\Tt)L_i(\Tt)\},$ $i=1,2.$
	Then, up to negligible terms,
	\begin{equation}\label{risk_1}
		\mathcal{R}(\Tt;\textbf B^*, M_0)\leq M_0^{-\frac{2\omega(\Tt)}{2\omega(\Tt)+1}}\Gamma_1 (\Tt),
	\end{equation}
	where $\mathbf{B}^*=\operatorname{diag}(1/h^*_1,1/h^*_2)$,
	%$1/\omega(\Tt)=1/H_1(\Tt)+1/H_2(\Tt),$ and 
	$$
	\frac{1}{\omega(\Tt)} = \frac{1}{H_1(\Tt)}+ \frac{1}{H_2(\Tt)}\quad \text{and} \quad \Gamma_1(\Tt)=  \frac{\kappa^2}{\pi} \frac{\sigma^2 }{c}  \Lambda_1(\Tt)^{\frac{H_2(\Tt)}{\mathcal H (\Tt)} }\Lambda_2(\Tt)^{\frac{H_1(\Tt)}{\mathcal H (\Tt)}}
		%{4H_1(\Tt)H_2(\Tt)+2H_1(\Tt)+2H_2(\Tt)}}
	\left\{1+2H_1(\Tt)+2H_2(\Tt)\right\}.$$
\end{corollary}

\medskip

The rate of $\mathcal{R}(\Tt;\textbf B^*, M_0)$ derived in Corollary \ref{cor_ad} matches the minimax rate for bivariate regression, provided that condition \eqref{Yor_us} is also satisfied. 

Finally, we derive the bound of the pointwise, conditional mean square risk when the regularity parameters are estimated, following our methodology. Let $\widehat h_1^*$ and $\widehat h_2^*$ be the bandwidths obtained by replacing $H_i(\Tt)$ and $L_i(\Tt)$ by their estimates $\widehat H_i(\Tt)$ and $\widehat L_i (\Tt)$ in the expressions of $h_1^*$ and $h_2^*$, respectively. Let $\widehat {\textbf B}^*$ be the corresponding bandwidth matrix.



\medskip


\begin{proposition}\label{risk_2}
	Assume the conditions of 
	Proposition \ref{prop_risk1} and  (LP\ref{LP4}) hold true. Then  
	$$
	\mathcal R(\Tt;\widehat {\textbf B}^*,M_0)\leq \Gamma_2(\Tt) M_0^{-\frac{2\omega(\Tt)}{2\omega(\Tt)+1}+2\log^{-a}(\mathfrak m)}\times \{1+o(\log^{-a}(\mathfrak m))\},
	$$
	where 
	$$
	\Gamma_2(\Tt)= \frac{\kappa^2\sigma^2}{c\pi\Lambda_1^{\alpha_1(\Tt)}(\Tt)\Lambda_2^{\alpha_2(\Tt)}(\Tt)}+L_1(\Tt)\left(\frac{\Lambda_1(\Tt)^{2H_1(\Tt)+1}}{\Lambda_2(\Tt)}\right)^{\alpha_1(\Tt)}+L_2(\Tt)\left(\frac{\Lambda_2(\Tt)^{2H_2(\Tt)+1}}{\Lambda_1(\Tt)}\right)^{\alpha_2(\Tt)},
	$$
	and 
	$$
	\alpha_i(\Tt)= \frac{\omega(\Tt)}{H_i(\Tt)(2\omega(\Tt)+1)},\qquad i=1,2.
	$$
\end{proposition}

\medskip

Proposition \ref{risk_2} shows that, modulo some constant terms,  the price for the estimation of the local regularity is the factor
$M_0^{2 \log^{-a} (\mathfrak m)}$, for some $a>1$. Since 
$\mathfrak m^{\log^{-1} (\mathfrak m)}=e$ for any $\mathfrak m >0$, 
the factor is essentially equal to 1 under very mild condition. 

%With functional data, the adaptation to the regularity of the surfaces is granted  almost for free, \emph{i.e.,} without deterioration of the rate by a logarithmic factor as it happens for the adaptation in standard nonparametric statistics.  


%The reason is the replication feature of this type of data, that means several realizations of a same process are observed instead of only one 



\begin{comment}
\section{System Architecture}
\label{appendix:architecture}
\system has a novel modularized system architecture with three key components: 
\emph{StreamManager}, 
\emph{TxnManager} and \emph{TxnScheduler}. 
These components are instantiated in each thread locally.
The execution outline of \system is presented in Algorithm~\ref{alg:algo}.
Transactional stream processing is continuous and potentially never ends (Line 1$\sim$8).
The dependency resolution and execution of state transactions are separated into two non-overlapping phases by punctuations~\cite{Tucker:2003:EPS:776752.776780} (Line 2 and 5), which guarantees that no subsequent input event will have a smaller timestamp. 
Effectively, a batch of state transactions is collected during the first phase, and processed during the second phase.

In the first phase (i.e., stream processing phase), 
the \emph{StreamManager} conducts preprocessing for every input event ($e$). Similar to some prior works~\cite{tstream}, state transactions may be issued but not immediately processed during preprocessing (Line 3).
The \emph{pre\_processing} and \emph{post\_processing} functions are exposed as APIs to users.
The \emph{TxnManager} handles dependency resolution (Line 4) among state transactions and insert decomposed operations to construct a \tpg. We discuss the detailed two-phase \tpg construction process in Section~\ref{subsec:construction}.

In the second phase  (i.e., transaction processing phase), 
the \emph{TxnManager} is first involved again to refine (Line 6) the constructed \tpg with further dependency resolution.
The \emph{TxnScheduler} 
schedules operations for concurrent execution based on the constructed \tpg according to the three dimensions of scheduling decisions (Line 7). 
In particular, a scheduling decision model $M$ is instantiated based on the constructed \tpg (Line 14).
\textbf{\circled{1}} Guided by $M$, execution threads adopt an exploration strategy (Section~\ref{subsec:explore}) to explore the constructed \tpg for operations available to be scheduled constrained by dependencies. 
\textbf{\circled{2}} 
During exploration, one or multiple operations may be treated as the 
% basic 
unit of scheduling (Section~\ref{subsec:granularity}). 
Subsequently, \textbf{\circled{3}} every thread executes operation(s) in the unit of scheduling with various abort handling mechanisms (Section~\ref{subsec:abort_handling}).
Only when state transactions are processed (i.e., committed or aborted) can the associated input events be postprocessed (Line 8) by the \emph{StreamManager} based on transaction processing results.
\end{comment}

\begin{comment}
\begin{algorithm}
\footnotesize
    \KwData{$e$ \tcp{Input event}}
    \KwData{$txn_{ts}$ \tcp{State transaction}}
    \KwData{$G$ \tcp{The currently constructed TPG}}
    \While{!finish processing of input streams}{
        \eIf(\tcp*[h]{Phase 1}){\text{$e$ is not a $punctuation$}}{
                $txn_{ts}$ $\gets$ PRE\_Processing($e$)\;
                \textbf{TPG\_Construction}($G$, $txn_{ts}$)\; 
          }(\tcp*[h]{Phase 2}){
                \textbf{TPG\_Refinement}($G$)\; 
                \textbf{TXN\_Scheduling}($G$)\; 
                POST\_Processing()\;
          }
    }
    
    \SetKwFunction{FMain}{TPG\_Construction}
    \SetKwProg{Fn}{Function}{:}{}
    \Fn{\FMain{$G$, $txn_{ts}$}}{
        $O_{1..k}$ $\gets$ \textbf{Partition} $txn_{ts}$\;
        \ForEach{\text{operation $O_{i}$ $\in$ $O_{1..k}$}}{
            \textbf{Identify} its \ld\;
            $G$ $\gets$ $G$ + $O_{i}$ \;
        }
    }
    \SetKwFunction{FMain}{TPG\_Refinement}
    \SetKwProg{Fn}{Function}{:}{}
    \Fn{\FMain{$G$}}{
        \ForEach{\text{vertex $e_{i}$ $\in$ $G$}}{
            \textbf{Identify} its \td, \pd\;
        }
    }
    
    \SetKwFunction{FMain}{TXN\_Scheduling}
    \SetKwProg{Fn}{Function}{:}{}
    \Fn{\FMain{$G$}}{
        $M$ $\gets$ Instantiated with $G$;\tcp{A decision model}
        \While{!finish scheduling of $G$
        }{
          \textbf{\circled{2}} $Scheduling Unit$ $\gets$ \textbf{\circled{1}} \emph{Explore}($G$, $M$)\; 
            \textbf{\circled{3}} \emph{Execute with Abort Handling} ($Scheduling Unit$)\; 
        }
    }
  \caption{Execution Outline of \system}
  \label{alg:algo}
\end{algorithm}
\end{comment}

%--------------------------------------------------------------------------

\medskip

\noindent \textbf{\large Acknowledgements:}
V. Patilea acknowledges support from the grant of the Ministry of Research, Innovation and Digitization, CNCS/CCCDI-UEFISCDI,  number PN-III-P4-ID-PCE-2020-1112, within PNCDI III.

\medskip

\noindent\textbf{\large Supplementary Material:}
In the Supplement we provide complements for the proofs  of Propositions \ref{propCH}, \ref{conc_Lest}, \ref{mprop}, \ref{prop_risk1}, and we prove some technical lemmas.  
Moreover, the justification for the local  Hölder continuity of the  realizations of $X$,  stated in Section \ref{sec6} above, is provided. 


%\newpage

%\bibliographystyle{chicago}
\bibliographystyle{apalike}
%\bibliographystyle{plainnat}

\bibliography{biblio_final.bib}

\end{document}