% !TeX root = ../MVFD_arxiv.tex

%\newpage

\appendix

\section{Proofs}
Below $\sim$ means left side bounded above and below by constants times the right side.

\begin{proof}[Proof of Proposition \ref{proprox}]
	By definition, we have that 
	$$
	\gamma_{\Tt}(\Delta)=\left(K^{(1)}_1(\Tt)+K_1^{(2)}(\Tt)\right)\Delta^{2\underline H (\Tt)}
	%\Delta^{2\overline H (\Tt)}
	+O(\Delta^{\widetilde \beta})=: \underline K(\Tt)\Delta^{2\underline H (\Tt)}+O(\Delta^{\widetilde \beta}).
	$$
	Moreover,  $\underline K(\Tt) = K_1(\Tt)+K_2(\Tt)$ if $\underline H (\Tt)=\overline H (\Tt)$, and $\underline K(\Tt) = K_1(\Tt)$ otherwise, with 
	$K_1(\Tt)$ and $K_2(\Tt)$ defined \eqref{eq:K1K2}. 
	We deduce
	\begin{multline}\label{eq_gamma1}
		\!\!\!\!\frac{\log(\gamma_{\Tt}(2\Delta))\! -\log(\gamma_{\Tt}(\Delta))}{2\log(2)}= \frac{\log\!\left(\!\underline K(\Tt)(2\Delta)^{2\underline H (\Tt)}\!+O(\Delta^{\widetilde \beta})\!\right)\!-\log\!\left(\!\underline K(\Tt)\Delta^{2\underline H (\Tt)}\!+O(\Delta^{\widetilde \beta})\!\right)}{2\log(2)}    \\
		= \underline H (\Tt)+\frac{\log\left(1+O(\Delta^{\widetilde \beta-2\underline H (\Tt)})\right)-\log(1+O(\Delta^{\widetilde \beta -2\underline H (\Tt)}))}{2\log(2)} 
		= \underline H (\Tt)+O(\Delta^{\widetilde \beta-2\underline H (\Tt)}),
	\end{multline}
	which gives the first part of the statement. For the second part, by the expansion \eqref{eq:K1K2},
	$$
	\gamma_{\Tt}(\Delta)= K_1(\Tt)\Delta^{2\underline H (\Tt)}+K_2(\Tt)\Delta^{2\overline H (\Tt)}+ O(\Delta^{2\overline H (\Tt)+\beta}).
	$$
	Therefore, $\alpha_{\Tt}(\Delta)$ can be written as 
	\begin{align*}
		\alpha_{\Tt}(\Delta)&=  \left|\frac{\gamma_{\Tt}(2\Delta)}{(2\Delta)^{2\underline{H}(\Tt)}}-\frac{\gamma_{\Tt}(\Delta)}{\Delta^{2\underline{H}(\Tt)}}\right|\\
		&= \left| K_2(\Tt)\left(2^{2\overline H(\Tt)-2\underline H (\Tt)}-1\right)\Delta^{2\overline H(\Tt)-2\underline H (\Tt)} +O(\Delta^{2\overline H(\Tt)-2\underline H (\Tt)+\beta})\right|\\
		& =: \left| \overline K(\Tt)\right|\Delta^{2\overline H(\Tt)-2\underline H (\Tt)} +O(\Delta^{2\overline H(\Tt)-2\underline H (\Tt)+\beta}).
	\end{align*}
	Finally, replace $\gamma_{\Tt}$ by $\alpha_{\Tt}$ in
  \eqref{eq_gamma1}, and derive the representation for $\overline H(\Tt)-\underline H (\Tt)$. \end{proof}



\begin{proof}[Proof of Proposition \ref{prop_Lest}]
	Similar to that of Proposition \ref{proprox}. 
\end{proof}




\begin{proof}[Proof of Proposition \ref{propCH}]
 We next simply write $\varrho(\Delta)$ instead of $\varrho(\Delta,\mathfrak m)$. The proof is organized in  several steps. 
First, using Assumptions \ref{ass_H1}, \ref{ass_H2} and \ref{ass_H3}, 	combined with Bernstein's inequality, a constant $\mathfrak u >0$ exists such that, for any $i=1,2$, $\varepsilon \in(0,1)$, and  $0<\Delta\leq \Delta_0$~:  
    \begin{equation}\label{eq:assumption-theta-hat_main}
	\max\left\{
	\PP\left(\widehat{\theta}_{\Tt}^{(i)}(\Delta)-\theta_{\Tt}^{(i)}(\Delta)\geq \varepsilon\right),
	\PP\left(\widehat{\theta}_{\Tt}^{(i)}(\Delta)-\theta_{\Tt}^{(i)}(\Delta)\leq -\varepsilon\right)
	\right\}
	\leq \exp\left( -\mathfrak{u}N\varepsilon^2\varrho(\Delta)\right),
\end{equation}
with $\widehat{\theta}_{\Tt}^{(i)}(\Delta)$ defined in \eqref{eq_theta_hat}, and provided that 
$\mathfrak m$ is sufficiently large.  The proof of \eqref{eq:assumption-theta-hat_main} is provided in the Supplementary Material. 

\noindent
	\textbf{\textit{Step 1 : proof of equation \eqref{eq:conc-Hhat-around-H_main}.}}
	For  $ \max \{R(\underline H )(\Tt),R(\overline H - \underline H )(\Tt)  \}\leq \varepsilon \leq 2\tau$, we have 
	$$
	\mathbb{P}\left[|\underline{\widehat{H}}(\Tt)-\underline{H}(\Tt)|\geq 2\varepsilon \right] \leq 
		\mathbb{P}\left[|\underline{\widehat{H}}(\Tt)-\underline{H}(\Tt)+R(\underline H )(\Tt) |\geq \varepsilon \right] = :A_\varepsilon
	$$
	Using the definitions and elementary inequalities,
	%of  $\underline{\widehat{H}}(\Tt)$ and $\underline{H}(\Tt)$, 
	 we have
	\begin{align*}
		A_\varepsilon
		&=  \mathbb{P}\left[\left|\log\left(\frac{\widehat{\gamma}_{\Tt}(2\Delta)\gamma_{\Tt}(\Delta)}{\gamma_{\Tt}(2\Delta)\widehat{\gamma}_{\Tt}(\Delta)}\right)\right|\geq 2\varepsilon\log2 \right]\\
		&\leq \mathbb{P}\left[\frac{\widehat{\gamma}_{\Tt}(2\Delta)\gamma_{\Tt}(\Delta)}{\gamma_{\Tt}(2\Delta)\widehat{\gamma}_{\Tt}(\Delta)}\geq 2^{2\varepsilon}\right]+\mathbb{P}\left[\frac{\widehat{\gamma}_{\Tt}(2\Delta)\gamma_{\Tt}(\Delta)}{\gamma_{\Tt}(2\Delta)\widehat{\gamma}_{\Tt}(\Delta)}\leq 2^{-2\varepsilon}\right]\\
		&\leq  \mathbb{P}\left[\frac{\widehat{\gamma}_{\Tt}(2\Delta)}{\gamma_{\Tt}(2\Delta)}\geq 2^\varepsilon\right]+\mathbb{P}\left[\frac{\widehat{\gamma}_{\Tt}(2\Delta)}{\gamma_{\Tt}(2\Delta)}\leq 2^{-\varepsilon}\right] +\mathbb{P}\left[\frac{\widehat{\gamma}_{\Tt}(\Delta)}{\gamma_{\Tt}(\Delta)}\geq 2^\varepsilon\right]+\mathbb{P}\left[\frac{\widehat{\gamma}_{\Tt}(\Delta)}{{\gamma_{\Tt}}(\Delta)}\leq 2^{-\varepsilon}\right]\\
		&\leq 4\exp\left(-\mathfrak{u}N(2^\varepsilon-1)^2  \gamma_*(\Delta) \varrho(\Delta)\right) +4 \exp\left( - \mathfrak{u}N(1-2^{-\varepsilon})^2\gamma_*(\Delta)\varrho(\Delta) \right),
	\end{align*}
where $\gamma_*(\Delta) = \min\{\gamma^2_{\Tt}(2\Delta),\gamma^2_{\Tt}(\Delta)\}$, and the last line is a  direct consequence of~\eqref{eq:assumption-theta-hat_main}.   
	
By elementary algebra and the fact that, for small $\Delta$,  we have $\gamma_{\Tt}(\Delta) = K_1(\Tt)\Delta^{2\underline H (\Tt)}\{1+o(1)\}$,  we deduce that   positive constants  $C_1$ and $C_2$  exist such that 
	\begin{equation}\label{eq:concentration-Hhat-around-H}
		\mathbb{P}\left[
		|\underline{\widehat{H}}(\Tt)-\underline{H}(\Tt)|
		\geq \varepsilon 
		\right]
		\leq C_1\exp \left(-C_2N\varepsilon^2\Delta ^{4\underline{H}(\Tt)}\varrho(\Delta)\right).
	\end{equation}
	
	%\medskip
	
	%%%%%%%%%%%%%%%%%%%%%%
	
	
	\noindent
	\textbf{\textit{Step 2.}} This step consists in proving that 
	constants $\tilde L_5$, $\tilde L_6$ exist such that
	\begin{equation}\label{eq:bound-alpha_main}
		\PP(|\widehat{\alpha}_{\Tt}(\Delta)-\alpha_{\Tt}(\Delta)|\geq \varepsilon)
		\leq \tilde L_5
		\exp\left(
		-\tilde L_6N\varepsilon^2\frac{\Delta^{4\overline{H}(\Tt)}\varrho(\Delta)}{\log^2(\Delta)}
		\right),
	\end{equation} 
	provided $\Delta$ is sufficiently small.
	The proof is relegated to the Supplementary Material.
	 
	

	
	%%%%%%%%%%%%%%
	%%%%%%%%%%%%%%
	%%%%%%%%%%%%%%
	
	\noindent
	\textbf{\textit{Step 3.}} To prove equation~\eqref{eq:concentration-overlineH_main}, we recall that
	\begin{equation*}
		\overline H(\Tt) = \underline H(\Tt) + D(\Tt)
		\qquad\text{and}\qquad
		\widehat{\overline{H}}(\Tt)=\widehat{\underline{H}}(\Tt)+\widehat{D}(\Tt)\mathbf{1}_{A_N(\tau)},
	\end{equation*}
	where the event $A_N(\tau)$ is defined in~\eqref{def_A_N}, and $\widehat{D}(\Tt)$ is the estimator of the difference ${\overline{H}}(\Tt)-{\underline{H}}(\Tt)$. We simply write $A_N$ instead of $A_N(\tau)$ in the sequel. Two  cases can be distinguished~: the isotropic case, where $\underline{H}(\Tt)=\overline{H}(\Tt)$, and the anisotropic case, where $\underline{H}(\Tt)<\overline{H}(\Tt)$. In the anisotropic  situation, we use  \eqref{eq:concentration-Hhat-around-H}, with $\varepsilon $ as in \eqref{eq:cdt_eps}, to get 
	\begin{multline*}
		\PP\left[\left|\widehat{\overline{H}}(\Tt)-\overline{H}(\Tt)\right|\geq 2\varepsilon\right]
		\leq
		\PP\left[\left|\widehat{\underline{H}}(\Tt)-\underline H(\Tt)\right|\geq \varepsilon\right]
		+ \PP\left[\left|\widehat D(\Tt) - D(\Tt)\right|\geq \varepsilon\right]
		+ \PP[\bar A_N]\\
		\leq
		C_1\exp \left(-C_2N \varepsilon^2\Delta ^{4\underline{H}(\Tt)}\varrho(\Delta)\right) 
		+ \PP\left[\left|\widehat D(\Tt) - D(\Tt)\right|\geq \varepsilon\right] 
		+ \PP[\bar A_N].
	\end{multline*}
Here, for a set $A$, $\overline{A}$ denotes its complement. We now remark that,  for $\varepsilon $ as in \eqref{eq:cdt_eps},
	\begin{multline*}
\hspace{-.2cm} \PP\left[\left|\widehat D(\Tt)-D(\Tt)\right|\geq \varepsilon\right]
\leq \PP\left[\left|\log\left(\frac{\widehat{\alpha}_{\Tt}(2\Delta)\alpha_{\Tt}(\Delta)}{\widehat{\alpha}_{\Tt}(\Delta)\alpha_{\Tt}(2\Delta)}\right)\right|\geq \varepsilon\log(2)\right]\\
\leq \PP\!\left[\frac{\widehat{\alpha}_{\Tt}(2\Delta)}{\alpha_{\Tt}(2\Delta)}\geq2^{\varepsilon/2}\right]+\PP\!\left[\frac{\widehat{\alpha}_{\Tt}(\Delta)}{\alpha_{\Tt}(\Delta)}\leq2^{-\varepsilon/2}\right]+\PP\!\left[\frac{\widehat{\alpha}_{\Tt}(2\Delta)}{\alpha_{\Tt}(2\Delta)}\leq2^{-\varepsilon/2}\right]+\PP\!\left[\frac{\widehat{\alpha}_{\Tt}(\Delta)}{\alpha_{\Tt}(\Delta)}\geq 2^{\varepsilon/2}\right].
	\end{multline*}
 We focus on the first term, the other three can be bounded similarly.  By~\eqref{eq:bound-alpha_main}, for small $\Delta$,
	\begin{align*}
		\PP\left[\frac{\widehat{\alpha}_{\Tt}(2\Delta)}{\alpha_{\Tt}(2\Delta)}\geq2^{\varepsilon/2}\right]
		&=\PP\left[\widehat{\alpha}_{\Tt}(2\Delta)-\alpha_{\Tt}(2\Delta)\geq (2^{\varepsilon/2}-1)\alpha_{\Tt}(2\Delta)\right]\\
		&\leq \tilde L_5
		\exp\left[
		-\tilde L_6N\left((2^{\varepsilon/2}-1)\alpha_{\Tt}(2\Delta)\right)^2\frac{\Delta^{4\overline{H}(\Tt)}\varrho(\Delta)}{\log^2(\Delta)}
		\right].
	\end{align*}
	Since $(2^{\varepsilon/2}-1)^2 \geq \varepsilon^2\log^2(2)/4$, we obtain that
	\begin{equation*}
		\PP\left[\frac{\widehat{\alpha}_{\Tt}(2\Delta)}{\alpha_{\Tt}(2\Delta)}\geq2^{\varepsilon/2}\right]
		\leq \tilde L_5
		\exp\left[
		-\tilde L_7N\varepsilon^2\frac{\Delta^{4\overline{H}(\Tt)}\varrho(\Delta)}{\log^2(\Delta)}\Delta^{4D(\Tt)}
		\right],
	\end{equation*}
	for some positive constant $\tilde L_7$. The same inequality, with possibly different constants, remains valid for the three other terms. Therefore, a  constant $\tilde L_8$ exists such that,  for $\varepsilon $ as in \eqref{eq:cdt_eps},
	\begin{equation}\label{eq:concentration-Difference}
		\PP\left[\left|\widehat{D}(\Tt)-D(\Tt)\right|\geq \varepsilon\right]
		\leq \tilde L_8
		\exp\left[
		-\tilde L_7N\varepsilon^2\frac{\Delta^{4\overline{H}(\Tt)}\varrho(\Delta)}{\log^2(\Delta)}\Delta^{4D(\Tt)}
		\right].
	\end{equation}
	
	Finally, it remains to bound $\PP[\bar A_N]$. Since $\tau \leq D(\Tt)/2$, we obtain 
	\begin{equation*}
		\PP\left(\bar A_N\right)=\PP\left(\widehat D (\Tt)- D(\Tt) \leq \tau -D(\Tt)\right)\leq \PP\left(\widehat D (\Tt)- D(\Tt) \leq -\tau \right).
	\end{equation*}
	Using \eqref{eq:concentration-Difference} with $2\tau$ in place of $\varepsilon$ (which is allowed by the condition $\varepsilon \leq 2\tau$)   leads to 
	\begin{equation}\label{A_N_aniso}
		\PP[\bar A_N] \leq \tilde L_8
		\exp\left[
		-4\tilde L_7N\tau^2\frac{\Delta^{4\overline{H}(\Tt)}\varrho(\Delta)}{\log^2(\Delta)}\Delta^{4D(\Tt)}
		\right].
	\end{equation}
	This implies, for $\varepsilon $ as in \eqref{eq:cdt_eps},
	\begin{multline}\label{eq:anisotropic}
		\PP\left[\left|\widehat{\overline{H}}(\Tt)-\overline{H}(\Tt)\right|\geq \varepsilon\right]
		\leq
		  C_1\exp \left[-C_2N (\varepsilon^2/4)\Delta ^{4\underline{H}(\Tt)}\varrho(\Delta)\right] 
		\\ + \tilde L_8
		\exp\left[
		-\tilde L_7N(\varepsilon^2/4)\frac{\Delta^{4\overline{H}(\Tt)}\varrho(\Delta)}{\log^2(\Delta)}\Delta^{4D(\Tt)}
		\right]
		+ \tilde L_8
		\exp\left[
		-4\tilde L_7N\tau^2\frac{\Delta^{4\overline{H}(\Tt)}\varrho(\Delta)}{\log^2(\Delta)}\Delta^{4D(\Tt)}
		\right],
	\end{multline}
and this concludes the proof  of the anisotropic case. 

For the isotropic case, where $\underline H(\Tt)=\overline H(\Tt)$, we use \eqref{eq:concentration-Hhat-around-H} and decompose as follows~: 
	\begin{multline*}
		\PP\left[\left|\widehat{\overline{H}}(\Tt)-\overline{H}(\Tt)\right|\geq \varepsilon\right] \leq  \PP\left[\left|\widehat{\underline{H}}(\Tt)-\underline{H}(\Tt)\right|\geq \varepsilon\right]+ \PP[A_N]\\
		\leq C_1\exp \left[-C_2N\varepsilon^2\Delta ^{4\underline{H}(\Tt)}\varrho(\Delta)\right] +\PP[A_N].
	\end{multline*}
We now have to bound $\PP[A_N]$, instead of  $\PP[\Bar A_N]$. For this, we can simply write 
	\begin{equation*}
		\PP[A_N]= \PP[\widehat D (\Tt)\geq \tau]=\PP[\widehat D (\Tt)- D(\Tt)\geq \tau].
	\end{equation*}
	Using \eqref{eq:concentration-Difference}  with $2\tau$ in place of $\varepsilon$ we then obtain 
	\begin{equation}\label{A_N_iso}
		\PP[ A_N] \leq \tilde L_8
		\exp\left[
		-4\tilde L_7N\tau^2\frac{\Delta^{4\overline{H}(\Tt)}\varrho(\Delta)}{\log^2(\Delta)}\Delta^{4D(\Tt)}
		\right],
	\end{equation}
	which leads to
	\begin{multline}\label{eq:isotropic}
		\PP \left[\left|\widehat{\overline{H}}(\Tt) - \overline{H}(\Tt)\right| \geq \varepsilon\right] 
	 	\leq   C_1\exp  \left[
		-C_2N\varepsilon^2\Delta ^{ 4\underline{H}(\Tt)}\varrho(\Delta)
		\right] 
		\\ +  \tilde L_8   \exp \left[ 
		-4 \tilde L_7N\tau^2\frac{\Delta^{ 4\overline{H}(\Tt)}\varrho(\Delta)}{\log^2(\Delta)}\Delta^{ 4D(\Tt)}
		\right] .
	\end{multline}
	Combining  \eqref{eq:anisotropic} and \eqref{eq:isotropic}, three positive constants $L_3,L_4$ and $L_5$ exists such that  
	\begin{multline*}%\label{eq:concentration-overlineH}
		\PP\left[\left|\widehat{\overline{H}}(\Tt)-\overline{H}(\Tt)\right|\geq \varepsilon\right] 
		\leq L_3\Big\{\exp \left[
		-L_2N\varepsilon^2\Delta ^{4\underline{H}(\Tt)}\varrho(\Delta)
		\right] 
		\\ \left. +\exp\left[
		- L_4N\tau^2\Delta^{4\overline{H}(\Tt)}\varrho(\Delta)\log^{-2}(\Delta)\Delta^{4D(\Tt)}
		\right]+P\right\},
	\end{multline*}
	 for any $\varepsilon $ as in \eqref{eq:cdt_eps}, where 
	$$
	P= \exp\left[
	- L_5N\varepsilon^2\Delta^{4\overline{H}(\Tt)}\varrho(\Delta)\log^{-2}(\Delta)\Delta^{4D(\Tt)}
	\right]\mathbf1_{\{\underline H(\Tt)<\overline H (\Tt)\}}.
	$$
	Let us note that $P$ is a term which only occurs in the anisotropic case. %The proof is now complete. 
	\end{proof}



	
\begin{proof}[Proof of Proposition \ref{conc_Lest}] 

	\textbf{\textit{Proof of  \eqref{eq:conc_L1_main}}.} Here is  the anisotropic case,  and we consider $H_1(\Tt)=\underline H (\Tt)$. Set   $\varepsilon$ as in \eqref{eq:cdt_epsL}  and, for $i=1,2$,   define  
\begin{equation*}
\PP\left(\left|\widehat{L_1^{(i)}}(\Tt)-L_1^{(i)}(\Tt)\right|\geq 2\varepsilon  \right) \leq 
\PP\left(\left|\widehat{L_1^{(i)}}(\Tt)-L_1^{(i)}(\Tt)+ R(L_1^{(i)})(\Tt)\right|\geq \varepsilon  \right)=:\mathfrak A_{\varepsilon}^{(i)}.
\end{equation*}
Using the definition of $\widehat{L_1^{(i)}}(\Tt)$ we can decompose~:
\begin{equation}\label{eq:decom-holder-const}
	\mathfrak A_{\varepsilon}^{(i)} \leq 
			\PP\left(\frac{\widehat{\theta}_{\Tt}^{(i)}(\Delta)}{\theta_{\Tt}^{(i)}(\Delta)}\geq\left(1+ \varepsilon\frac{\Delta^{2\underline{H}(\Tt)}}{\theta_{\Tt}^{(i)}(\Delta)}\right)^{\!\frac{1}{2}}\right)
			+ \PP\left(\frac{\Delta^{2\widehat{\underline{H}}(\Tt)}}{\Delta^{2\underline{H}(\Tt)}}\leq\left(1+ \varepsilon\frac{\Delta^{2\underline{H}(\Tt)}}{\theta_{\Tt}^{(i)}(\Delta)}\right)^{\!-\frac{1}{2}}\right).
\end{equation}
To bound these two terms, we first notice that a constant $K$ exists such that,  $\forall \varepsilon$ as in \eqref{eq:cdt_epsL},
\begin{equation*}
\left(1+ \varepsilon\frac{\Delta^{2\underline{H}(\Tt)}}{\theta_{\Tt}^{(i)}(\Delta)}\right)^{\frac{1}{2}}\geq
1+\varepsilon \frac{1}{2\sqrt{ 1+ L_1^{(i)} (\Tt)}}+\varepsilon O(\Delta^{2\overline{H}(\Tt)-\underline H(\Tt)})
\geq 1+\varepsilon K,
\end{equation*}
  provided $\Delta$ is sufficiently small. Then, by  \eqref{eq:assumption-theta-hat_main},  for sufficiently large $\mathfrak m$,  
\begin{equation*}
    \PP\left(\frac{\widehat{\theta}_{\Tt}^{(i)}(\Delta)}{\theta_{\Tt}^{(i)}(\Delta)}\geq\left(1+ \varepsilon\frac{\Delta^{2\underline{H}(\Tt)}}{\theta_{\Tt}^{(i)}(\Delta)}\right)^{\frac{1}{2}}\right) 
    \leq \exp \left(-\mathfrak{u}KN\varepsilon^2\{ \theta_{\Tt}^{(i)}(\Delta)\}^2\varrho(\Delta)\right).
\end{equation*}
Since $\theta_{\Tt}^{(i)}(\Delta)\sim L_1^{(i)}(\Tt)\Delta^{2\underline H (\Tt)}$ we obtain~:
\begin{equation}\label{eq:first-term-L}
  \PP\left(
  	\frac{\widehat{\theta}_{\Tt}^{(i)}(\Delta)}{\theta_{\Tt}^{(i)}(\Delta)}\geq\left(1+ \varepsilon\frac{\Delta^{2\underline{H}(\Tt)}}{\theta_{\Tt}^{(i)}(\Delta)}\right)^{\frac{1}{2}}
	\right)
 \leq\exp \left(
 	-\mathcal L N\varepsilon^2\Delta^{4\underline H (\Tt)}\varrho(\Delta)
	\right),
\end{equation}
for some  constant $\mathcal L$.  For the second term, since $\Delta$ is  small, and $\log(x)\leq x-1$, $x>0$,  
\begin{equation*}
    \PP\!\left[\!\frac{\Delta^{2\widehat{\underline{H}}(\Tt)}}{\Delta^{2\underline{H}(\Tt)}}\leq\! \left\{\!1\!+ \varepsilon\frac{\Delta^{2\underline{H}(\Tt)}}{\theta_{\Tt}^{(i)}(\Delta)}\right\}^{\!\!-\frac{1}{2}}\right] \!\leq \PP\!\left[\!\frac{\Delta^{2\widehat{\underline{H}}(\Tt)}}{\Delta^{2\underline{H}(\Tt)}}\leq 1\! - \! K \varepsilon\right]
    \!\leq \PP\left[\!\widehat{\underline{H}}(\Tt)\!-\underline{H}(\Tt)\!\geq - \frac{K/2}{\log(\Delta)} \varepsilon\right]\!,
\end{equation*}
provided $\varepsilon$ satisfies \eqref{eq:cdt_epsL}.
Using \eqref{eq:concentration-Hhat-around-H}, we get
\begin{equation}\label{eq:second-term-L} 
\PP\left(
	 \frac{\Delta^{2\widehat{\underline{H}}(\Tt)}}{\Delta^{2\underline{H}(\Tt)}}\leq\left(1+ \varepsilon\frac{\Delta^{2\underline{H}(\Tt)}}{\theta_{\Tt}^{(i)}(\Delta)}\right)^{-\frac{1}{2}}
	  \right) 
\leq   C_1\exp\left(
	-\frac{C_2K^2}{4}N\varepsilon^2\frac{\Delta^{4\underline H(\Tt)}\varrho(\Delta)}{\log^2( \Delta)}
	\right). 
\end{equation}
Finally, combining \eqref{eq:decom-holder-const}, \eqref{eq:first-term-L}, \eqref{eq:second-term-L}, and considering, without loss of generality, $\Delta_0\leq e^{-1}$ in Definition \ref{def}, positive constants $ \mathfrak C _1$ and  $ \mathfrak C_2$ exist such that,    $\forall \varepsilon$ as in \eqref{eq:cdt_epsL}, we have~:
\begin{equation*}%\label{eq:concentration_L1}
\mathfrak A_{\varepsilon}^{(i)} \leq  \mathfrak C_1 
\exp\left(
	- \mathfrak C_2 N \varepsilon^2\frac{\Delta^{4\underline H(\Tt)}\varrho(\Delta)}{\log^2(\Delta)}
	\right),\qquad i=1,2.
\end{equation*} 

%%%%%%%%%%%%%%%
%\medskip
\noindent \textbf{\textit{Proof of \eqref{eq:conc_L2_main}:}} 
%Recall that 
%\begin{equation*}
% L_2^{(i)}= \frac{1}{(2^{2D(\Tt)}-1)\Delta^{2D(\Tt)}}\left|\frac{\theta_{\Tt}^{(i)}(2\Delta)}{(2\Delta)^{2\underline H(\Tt)}}-\frac{\theta_{\Tt}^{(i)}(\Delta)}{\Delta^{2\underline H(\Tt)}}\right|+O(\Delta^\beta),
%\end{equation*}
%with  $D(\Tt)=\overline H(\Tt) -\underline H (\Tt).$ 
For $\Tt\in \cT$ and $i=1,2$, let
\begin{equation*}
\alpha_{\Tt}^{(i)}(\Delta)= \left|\frac{\theta_{\Tt}^{(i)}(2\Delta)}{(2\Delta)^{2\underline H(\Tt)}}-\frac{\theta_{\Tt}^{(i)}(\Delta)}{\Delta^{2\underline H(\Tt)}}\right|
\quad \text{ and } \quad
\widehat{ \alpha}_{\Tt}^{(i)}(\Delta)= \left|\frac{\widehat{\theta}_{\Tt}^{(i)}(2\Delta)}{(2\Delta)^{2\widehat{\underline H}(\Tt)}}-\frac{\widehat{\theta}_{\Tt}^{(i)}(\Delta)}{\Delta^{2\widehat{\underline H}(\Tt)}}\right|.
\end{equation*}
For any $\varepsilon$ such that  $ |R(L_2^{(i)})(\Tt)|\leq \varepsilon$ (recall $|R(L_2^{(i)})(\Tt)|=O(\Delta^\beta)$)  and $i=1,2$, we decompose as follows~:
\begin{equation*}%\label{eq:decompose-L2}
 \PP\left(\widehat{L_2^{(i)}}(\Tt)-L_2^{(i)}(\Tt)\geq 2 \varepsilon,  \widehat D(\Tt) \neq 0 \right)\leq B^{(i)}_1+B^{(i)}_2+B^{(i)}_3+B^{(i)}_4,
\end{equation*}
 where 
 \begin{equation*}
 B^{(i)}_1=\PP\!\left(\!\widehat{\alpha}_{\Tt}^{(i)}(\Delta)-\alpha_{\Tt}^{(i)}(\Delta)\geq \sqrt{\varepsilon/3}\right),\quad 
 B^{(i)}_2=\PP\!\left(\frac{\widehat{\alpha}_{\Tt}^{(i)}(\Delta)-\alpha_{\Tt}^{(i)}(\Delta)}{\{4^{D(\Tt)}-1\}\Delta^{2D(\Tt)}}\geq \varepsilon/3\!\right),
 \end{equation*}
 \begin{align}\label{B3B4_main}
 B^{(i)}_3&=\PP\left(\alpha_{\Tt}^{(i)}(\Delta)\left(\frac{1}{\{4^{\widehat{D}(\Tt)}-1\}\Delta^{2\widehat{D}(\Tt)}} -\frac{1}{\{4^{D(\Tt)}-1\}\Delta^{2D(\Tt)}}\right)\geq \varepsilon/3 ,\widehat D(\Tt) \neq 0 \right),\\
 B^{(i)}_4&=\PP\left(\frac{1}{\{4^{\widehat{D}(\Tt)}-1\}\Delta^{2\widehat{D}(\Tt)}} -\frac{1}{\{4^{D(\Tt)}-1\}\Delta^{2D(\Tt)}}\geq \sqrt{ \varepsilon/3}, \widehat D(\Tt) \neq 0 \right).\notag
 \end{align}
 By the arguments used for \eqref{eq:bound-alpha_main}, $\mathfrak m$ sufficiently large,  constants $\tilde C_1$ and $\tilde C_2$ exists  such that, 
$\forall \varepsilon$ such that $|\log(\Delta)| |R(\underline H )(\Tt)| \leq \varepsilon$ (recall $|R(\underline H )(\Tt)|=O(\Delta^{2{D}(\Tt)})$ in the anisotropic case),   
\begin{multline}\label{eq:bound-B1}
B^{(i)}_1 \leq   \tilde C_1 \exp  \left[
	  -\tilde C_2   N\varepsilon\frac{\Delta^{4\overline H (\Tt)}\varrho(\Delta)}{\log^2(\Delta)}
	 \right]  \\
\text{ and }\qquad 
B_2^{(i)}  \leq   \tilde C_1 \exp \left[ 
	- \tilde C_2  N\varepsilon^2(4^{D(\Tt)}  -1)^2 \Delta^{4D(\Tt)}\frac{\Delta^{4\overline H (\Tt)}\varrho(\Delta)}{\log^2(\Delta)}
	  \right] .
\end{multline}
To bound $B^{(i)}_3$ and $B^{(i)}_4$ in \eqref{B3B4_main}, we  use the fact that $\alpha_{\Tt}^{(i)}(\Delta)\sim L_2^{(i)}(\Tt)(4^{D(\Tt)}-1)\Delta^{2D(\Tt)}$,  Lemma SM.\ref{lemma_L2_use} in the Supplement, and the fact that  $L_2^{(1)}$, $L_2^{(2)}$ are bounded functions. Moreover, we show  the probability of the event $\{\widehat D(\Tt)=0\}$ is negligible, see \eqref{bound_Dhat_negli}.  The details are given in the Supplementary. From that and  \eqref{eq:bound-B1}, the proof follows. \end{proof}

%%%%%%%%%%%%%
%\smallskip

\begin{proof}[Proof of Proposition \ref{prop5_simple}]
	A direct consequence of \eqref{A_N_aniso}  and \eqref{A_N_iso}.
\end{proof}

%\smallskip

%%%%%%%%%%%%%
%%%%%%%%%%%%%

\begin{proof}[Proof of Proposition \ref{mprop}]
	First, let us denote  $$B(\Tt,\Ss) =2D(H_1(\Tt),H_1(\Ss))D(H_2(\Tt),H_2(\Ss))\qquad \forall \Tt,\Ss\in \cT,$$ with $D(x,y)$ defined in \eqref{def_D_func}. By construction the function $B(\cdot,\cdot)$ is symmetric. Moreover, we show in the Supplementary Material that 
\begin{equation}\label{eq:B-diag}
	B(\Tt,\Ss)= \frac{1}{2}+O(\|\Tt-\Ss\|^2).
\end{equation}
	
Next,  using the covariance function structure of the process $X$, we can write~: 
	\begin{multline}
			\theta(\Tt,\Ss)= \EE[X^2(\Tt)]+\EE[X^2(\Ss)]- 2\EE[X(\Tt)X(\Ss)]\\
%			=&|A_1(\Tt)|^{2H_1(\Tt)}|A_2(\Tt)|^{2H_2(\Tt)}+|A_1(\Ss)|^{2H_1(\Ss)}|A_2(\Ss)|^{2H_2(\Ss)}\\
%			-&B(\Tt,\Ss) \times \left[|A_1(\Tt)|^{H_1(\Tt)+H_1(\Ss)}+|A_1(\Ss)|^{H_1(\Tt)+H_1(\Ss)}-|A_1(\Tt)-A_1(\Ss)|^{H_1(\Tt)+H_1(\Ss)}\right]\\
%			\times&\left[|A_2(\Tt)|^{H_2(\Tt)+H_2(\Ss)}+|A_2(\Ss)|^{H_2(\Tt)+H_2(\Ss)}-|A_2(\Tt)-A_2(\Ss)|^{H_2(\Tt)+H_2(\Ss)}\right]\\
			=  \mathfrak B_1(\Tt,\Ss) +\mathfrak B_1(\Ss,\Tt) +\mathfrak B_2(\Tt,\Ss)+\mathfrak B_3(\Tt,\Ss)-\mathfrak B_4(\Tt,\Ss), \quad \forall \Tt,\Ss\in \cT,
		\end{multline}
		where 
		\begin{align*}
			\mathfrak B_1 (\Tt,\Ss)&= |A_1(\Tt)|^{2H_1(\Tt)}|A_2(\Tt)|^{2H_2(\Tt)}-B(\Tt,\Ss)|A_1(\Tt)|^{H_1(\Tt)+H_1(\Ss)}\\
			&\times\left(|A_2(\Tt)|^{H_2(\Tt)+H_2(\Ss)}\!+|A_2(\Ss)|^{H_2(\Tt)+H_2(\Ss)}\right),\\
			\mathfrak B_2(\Tt,\Ss) &= B(\Tt,\Ss)\left(|A_1(\Tt)|^{H_1(\Tt)+H_1(\Ss)}\!+|A_1(\Ss)|^{H_1(\Tt)+H_1(\Ss)}\right)|A_2(\Tt)\!-\!A_2(\Ss)|^{H_2(\Tt)+H_2(\Ss)},\\
			\mathfrak B_3(\Tt,\Ss)&=B(\Tt,\Ss)\left(|A_2(\Tt)|^{H_2(\Tt)+H_2(\Ss)}\!+|A_2(\Ss)|^{H_2(\Tt)+H_2(\Ss)}\right)|A_1(\Tt)\!-\!A_1(\Ss)|^{H_1(\Tt)+H_1(\Ss)},\\
			\mathfrak B_4(\Tt,\Ss)&= B(\Tt,\Ss)|A_1(\Tt)\!-\!A_1(\Ss)|^{H_1(\Tt)+H_1(\Ss)}|A_2(\Tt)-A_2(\Ss)|^{H_2(\Tt)+H_2(\Ss)}.
		\end{align*}
		Let 
		\begin{equation}
			a(\Tt,\Ss)=\frac{|A_1(\Tt)|^{H_1(\Tt)-H_1(\Ss)}|A_2(\Tt)|^{H_2(\Tt)-H_2(\Ss)}- B(\Tt,\Ss)}{B(\Tt,\Ss)}.
		\end{equation}
		We then have~:
		\begin{multline}\label{B1+B1}
			\mathfrak B_1(\Tt,\Ss) +\mathfrak B_1(\Ss,\Tt)=B(\Tt,\Ss) \left(a(\Tt,\Ss)|A_2(\Tt)|^{H_2(\Tt)+H_2(\Ss)}-|A_2(\Ss)|^{H_2(\Tt)+H_2(\Ss)}\right)\\
			\times\left(|A_1(\Tt)|^{H_1(\Tt)+H_1(\Ss)}-a(\Ss,\Tt)|A_1(\Ss)|^{H_1(\Tt)+H_1(\Ss)}\right)\\
			+|A_1(\Ss)|^{H_1(\Tt)+H_1(\Ss)}|A_2(\Tt)|^{H_2(\Tt)+H_2(\Ss)}B(\Tt,\Ss) \left\{ a(\Tt,\Ss)a(\Ss,\Tt) -1\right\}.
		\end{multline}
Using \eqref{eq:B-diag}, we show in the Supplementary Material that
		\begin{equation}\label{eq:double-a}
	a(\Tt,\Ss)a(\Ss,\Tt)-1 =O(\|\Tt-\Ss\|^2).
\end{equation}	
We can next deduce that 
\begin{align*}
	B(\Tt,\Ss) &\left(a(\Tt,\Ss)|A_2(\Tt)|^{H_2(\Tt)+H_2(\Ss)}-|A_2(\Ss)|^{H_2(\Tt)+H_2(\Ss)}\right)\times\\
	&\left(|A_1(\Tt)|^{H_1(\Tt)+H_1(\Ss)}-a(\Ss,\Tt)|A_1(\Ss)|^{H_1(\Tt)+H_1(\Ss)}\right)=O(\|\Tt-\Ss\|^2).
\end{align*}
From this and \eqref{eq:double-a}, equation \eqref{B1+B1} becomes~:
\begin{equation*}%\label{eq:B1B1}
	\mathfrak B_1(\Tt,\Ss) +\mathfrak B_1(\Ss,\Tt)= O(\|\Tt-\Ss\|^2).
\end{equation*}
For the terms $B_3(\Tt,\Ss)$ and $B_4(\Tt,\Ss)$, we apply \eqref{eq:B-diag}. Finally,  we get 
\begin{multline*}
	\EE[(X(\Tt)-X(\Ss))^2] \\ =\mathfrak B_4(\Tt,\Ss)+ \frac{1}{2}\left(|A_1(\Tt)|^{H_1(\Tt)+H_1(\Ss)}+|A_1(\Ss)|^{H_1(\Tt)+H_1(\Ss)}\right)|A_2(\Tt)-A_2(\Ss)|^{H_2(\Tt)+H_2(\Ss)}\\
	+\frac{1}{2}\! \left(\!|A_2(\Tt)|^{H_1(\Tt)+H_1(\Ss)}+|A_2(\Ss)|^{H_1(\Tt)+H_1(\Ss)}\!\right)\!|A_1(\Tt)\!-A_1(\Ss)|^{H_1(\Tt)+H_1(\Ss)}
	+O(\|\Tt-\Ss\|^2).
\end{multline*}
The last expression and the Taylor expansion then imply
\begin{align*}
	\EE[(X(\Tt)-X(\Ss))^2]=&|A_1(\Tt)|^{2H_1(\Tt)}|\partial_1 A_2(\Tt)(t_1-s_1)+\partial_2 A_2(\Tt)(t_2-s_2)|^{2H_2(\Tt)}\\
	+&|A_2(\Tt)|^{2H_2(\Tt)}|\partial _1A_1(\Tt)(t_1-s_1)+\partial_2 A_1(\Tt)(t_2-s_2)|^{2H_1(\Tt)}\\
	+&O(\|\Tt-\Ss\|^2)+O(\|\Tt-\Ss\|^{2\underline{H}(\Tt)+1})+ O\left(\|\Tt-\Ss\|^{2\underline H(\Tt)+2\overline H (\Tt)}\right).
\end{align*}
\end{proof}



%%%%%%%%%%%%%
%%%%%%%%%%%%%
%%%%%%%%%%%%%

\begin{proof}[Proof of  Proposition \ref{prop_def_A}]
We  start by showing that 
\begin{equation}\label{eq_f1_g1}
\underset{\Tt\in \cT}{\sup}\EE\left[ |\widehat f_1(\Tt)-f_1(\Tt)|^2\right]<\infty \qquad \text{ and }\qquad \underset{\Tt\in \cT}{\sup}\EE\left[ |\widehat g_1(\Tt)-g_1(\Tt)|^2\right]<\infty.
\end{equation}
Since 
$
\EE|\widehat f_1(\Tt)\! -f_1(\Tt)|^2\leq 2f_1^2(\Tt)\!+2\EE|\widehat f_1(\Tt)|^2
$, it suffices to bound $f_1^2(\Tt)$ and $\EE|\widehat f_1(\Tt)|^2$.  
By \eqref{simpl_A} and \eqref{simpl_bet},
\begin{equation*}
f_1^2(\Tt)= \left(L_1^{(1)}(\Tt) \big/v(\Tt)\right)^{\frac{1}{H_1(\Tt)}}\leq \left(\overline \beta / \underline v\right)^{\frac{1}{\underline \beta}}.
\end{equation*}
Moreover,
\begin{equation*}
\widehat f_1^2(\Tt)=\left( \widehat L_1^{(1)}(\Tt)\big / \widehat v(\Tt) \right) ^{\frac{1}{\widehat H_1(\Tt)}} \leq \left( \overline \beta/\underline v\right)^{\frac{1}{\underline \beta}}\times \left\{v(\Tt)/\widehat v(\Tt)\right\}^{\frac{1}{\underline \beta}}.
\end{equation*}
Therefore, by  \eqref{simpl_v} we obtain 
\begin{equation*}
\EE |\widehat f_1(\Tt)|^2 \leq  \ \left( \overline \beta/\underline v\right)^{\frac{1}{\underline \beta}} \times\EE\left[ \left\{v(\Tt)/\widehat v(\Tt)\right\}^{\frac{1}{\underline \beta}}\right]<\infty.
\end{equation*}
The first part of \eqref{eq_f1_g1} follows, the second part can be obtained with similar arguments.  

Next, by the condition \eqref{simpl_v} and Fubini's Theorem, a constant $\mathfrak C_v$ such that 
	\begin{multline}\label{deux_int}
		\EE\left[|A_1(\Tt)-\widehat A_1(\Tt)|\right]\leq \mathfrak C_v A_1(\Tt)\left[  \int_{t_0}^{t_1} \EE|\widehat f_1(s,t_2)-f_1(s,t_2)|{\D} s  \right. \\ \left. + \int_{s_0}^{t_2}\EE|\widehat g_1(t_0,s)-g_1(t_0,s)|\D s \right].
	\end{multline}
A detailed justification of the last inequality is provided in the Supplementary Material. We next bound the two integrals in \eqref{deux_int}. For $\lambda \in (0,1)$, we define the set 
	$$
	\mathcal O(\lambda)=\left\{|\widehat f_1(s,t_2)-f_1(s,t_2)|\leq \lambda \right\}.
	$$
By Cauchy-Schwarz inequality,
	\begin{multline}\label{exp}
		\int_{t_0}^{t_1}\EE\left[ |\widehat f_1(s,t_2)-f_1(s,t_2)|\right]{\D} s\leq \int_{t_0}^{t_1}\left(\lambda+\EE\left[ |\widehat f_1(s,t_2)-f_1(s,t_2)|\mathbf{1}_{\overline{\mathcal O}(\lambda)}\right]\right){\D} s\\
		\leq \int_{t_0}^{t_1}\left(\lambda+\EE\left[ |\widehat f_1(s,t_2)-f_1(s,t_2)|^2\right]^{1/2}\PP^{1/2}(\overline{\mathcal O}(\lambda))\right){\D} s.
	\end{multline}
We now want to apply  Lemma SM.\ref{lem} in the Supplementary Material with
$$
a_N=\widehat{ L_1^{(1)}}(\Tt), \quad b_N=\widehat v(\Tt), \quad c_N= \frac{1}{\widehat H_1(\Tt)} \quad \text{and } \quad 
a=L_1^{(1)}(\Tt), \quad b= v(\Tt), \quad c= \frac{1}{ H_1(\Tt)}.
$$
and $\mathfrak C= \overline \beta/\underline v$.  We first note that, by Proposition \ref{conc_Lest} with $i=1$, for any $\varepsilon$ satisfying \eqref{eq:cdt_epsL},   
$$
\PP\left(\left|\widehat{L_1^{(1)}}(\Tt)-L_1^{(1)}(\Tt)\right|\geq \varepsilon  \right)\leq \mathfrak{C}_1\exp\left(-\mathfrak{C}_2N\varepsilon^2\frac{\Delta ^{4H_1(\Tt)}\varrho(\Delta,\mathfrak m)}{\log^2(\Delta)}\right).
$$
Moreover, by Lemma SM.\ref{concentration:variance}, and for sufficiently large $\mathfrak m$, there exists a constant $\mathfrak e$ such that 
$$
  \forall \eta \in (0,1),\qquad \PP(|\widehat v(\Tt)-v(\Tt)|\geq \eta)\leq 2\exp(- \mathfrak  e N\eta ^2). 
$$
It remains to derive a bound for the concentration of $c_N$, which follows from that on the concentration of $\widehat{H}_1(\Tt)$, and the fact that $H_1\leq 1$~: 
 for any $\varepsilon$ satisfying \eqref{eq:cdt_epsL},  and thus \eqref{eq:cdt_eps},  
\begin{multline*}
	\PP\left(\pm \{c_N-c \} \geq\varepsilon\right)
	%	\PP\left(\pm \{H_1(\Tt)-\widehat{H}_1(\Tt)\} \geq \varepsilon\widehat{H}_1(\Tt)H_1(\Tt) \right)
	=\PP\left( \pm \{H_1(\Tt)-\widehat{H}_1(\Tt)\}\{1\pm \varepsilon H_1(\Tt)\}\geq \varepsilon H_1^2(\Tt) \right)\\
	\leq \PP\left(\pm \{H_1(\Tt)-\widehat{H}_1(\Tt)\}\geq  H_1^2(\Tt)\varepsilon/\{1+ \varepsilon H_1(\Tt)\}\right),
\end{multline*}
and thus, by Proposition \ref{propCH}, 
$$
\PP\left(|c_N-c | \geq\varepsilon\right) \leq \widetilde C_1\exp\left(-\widetilde C_2N\left\{\underline \beta^4/4\right\}\varepsilon^2\Delta^{4H_1(\Tt)}\varrho(\Delta,\mathfrak m)\right) .
$$
Finally, by Lemma SM.\ref{lem} we obtain 
	$$
	\PP(\overline{\mathcal O}(\lambda))= \PP\!\left (|\widehat f_1(s,t_2)\!-f_1(s,t_2)|\geq \lambda\!\right)\!\leq q_1 \exp\left\{\! -q_2N\lambda^2\Delta^{2H_1(s,t_2)}\!\varrho(\Delta,\mathfrak m)\log^{-2}(\Delta) \right\}\!,
	$$
	for some  constants $q_1$, $q_2$, provided  \eqref{eq:cdt_epsL} is satisfied with $\varepsilon = \lambda$. By \eqref{exp}, we now  write 
	\begin{multline*}
		\int_{t_0}^{t_1}\EE\left[ |\widehat f_1(s,t_2)-f_1(s,t_2)|\right]{\D} s
%		\\		\leq  \int_{t_0}^{t_1}\left(\lambda+\EE\left[ |\hat f_1(s,t_2)-f_1(s,t_2)|^2\right]^\frac{1}{2}q_1\exp\left\{ -q_2N\lambda^2\frac{\Delta^{2H_1(s,t_2)}\varrho(\Delta,\mathfrak m)}{\log^2(\Delta)} \right\}\right){\D} s
		\\
	\hspace{-1cm} 	\leq  \int_{t_0}^{t_1}\left(\lambda+\underset{\Tt\in \cT}{\sup}\EE\left[ |\widehat f_1(\Tt)-f_1(\Tt)|^2\right]^{1/2} q_1\exp\left\{ -q_2N\lambda^2\frac{\Delta^{2H_1(s,t_2)}\varrho(\Delta,\mathfrak m)}{\log^2(\Delta)} \right\}\right){\D} s\\
		\leq\operatorname{ diam }(\cT)\max\left\{1,\underset{\Tt\in \cT}{\sup}\EE\left[ |\widehat f_1(\Tt)-f_1(\Tt)|^2\right]^{1/2}\right \}\!\left [\lambda+q_1\exp\left\{ -q_2N\lambda^2\frac{\Delta^{2}\varrho(\Delta,\mathfrak m)}{\log^2(\Delta)} \right\}\right].
	\end{multline*}
A simple choice of $\lambda$ can be defined as follows~: for some suitable   $\ell\in (0,1/2)$, let
	$$
	\lambda=\frac{|\log(\Delta)|}{\Delta \sqrt{\varrho(\Delta,\mathfrak m)}}N^{\ell-1/2},
	$$
	 which satisfies \eqref{eq:cdt_epsL}, provided that $\ell $ satisfies \eqref{cdt_ell}.
 We then obtain 
	$$
	\int_{t_0}^{t_1}\EE\left[ |\widehat f_1(s,t_2)-f_1(s,t_2)|\right]{\D} s\leq C \left( \frac{|\log(\Delta)|}{\Delta \sqrt{\varrho(\Delta,\mathfrak m)}}N^{\ell-1/2}+q_1e^{-q_2N^\ell}\right),
	$$
	for some constant $C$. Up to a change of the constants $C$, $q_1$, $q_2$, a similar bound holds true for the second integral on the RHS on \eqref{deux_int}.  It remains to replace   $\Delta $ and $\rho(\mathfrak m) $ by $\mathfrak m^{-a}$ and $\mathfrak m ^{-b}$, respectively.  
\end{proof}



%\medskip


\begin{proof}[Proof of Proposition \ref{prop_risk1}.]
	The risk $\mathcal R(\Tt; \boldsymbol B ,M_0)$ is the sum of the squared bias and the variance, for which we derive the 
 following bounds in Lemma SM.\ref{lemma_BV}  in the Supplement~: 
\begin{multline*}
 \EE\!\left[\!\left(\sum_{m=1}^{M_0}\{X^{new}(\Tt^{new}_m;\!\boldsymbol B)-X^{new}(\Tt)\} W^{new}_m(\Tt) \right)^2\! \Big{|}M_0\! \right]\\
 \leq 2\left\{\!L_1(\Tt) h_1^{2H_1(\Tt)}\!+L_2(\Tt) h_2^{2H_2(\Tt)}\!\right\}\!\{1+o(1)\},
\end{multline*}
and, for some $a_1>1$,
 $$
 \EE\left[\left(\sum_{m=1}^{M_0}\varepsilon^{new}_m  W^{new}_m(\Tt)  \right)^2 \Big{|}M_0\right]\leq \frac{\kappa^2\sigma^2}{c\pi}\frac{1}{M_0 h_1h_2}\left\{1+a_1 M_0^{-1/4}\right\}.
 $$
\end{proof}

\begin{proof}[Proof of Proposition \ref{risk_2}.]
Let $\widehat \omega (\Tt)$,  $\widehat \alpha_i(\Tt)$ and  $\widehat \Lambda_i (\Tt)$ be the estimators obtained by replacing $H_i(\Tt)$ and $L_i(\Tt)$ in the definitions of     $ \omega (\Tt)$,  $ \alpha_i(\Tt)$ and $ \Lambda_i(\Tt)$, respectively. We define the sets 
		$$
		\mathcal{F}=\bigcap_{i=1,2} \left\{ |\widehat \alpha _i(\Tt)-\alpha_i(\Tt)|\leq \log^{-a}(\mathfrak m)\right\}
		%\cap \left\{ |\widehat \alpha _2(\Tt)-\alpha_2(\Tt)|\leq \log^{-a}(\mathfrak m)\right\},
%		$$
\quad\text{		and } \quad 
%		$$
		\mathcal E= \bigcap_{i=1,2}\left\{ \left|  \widehat \Lambda_i(\Tt)/\Lambda_i(\Tt)-1\right| \leq \log^{-a}(\mathfrak m)\right\},
		%\cap\left\{ \left| \frac{\widehat \Lambda_2(\Tt)}{\Lambda_2(\Tt)}-1\right| \leq \log^{-a}(\mathfrak m)\right\},
		$$
with $a$ from assumption \ref{LP4}.	


		Since  $\widehat h^*_1$ and $\widehat h^*_2$ are independent of the new realization $X^{new}$,   by  \eqref{risk_1} we obtain 
		\begin{multline*}
\EE\left[\{\widehat X^{new}(\Tt;\widehat{\mathbf{B}}^* )-X^{new}(\Tt)\}^2 \big{|}M_0,\widehat h_1^*,\widehat h_2^*  \right]\leq  \frac{\kappa^2}{c\pi} \frac{\sigma^2}{M_0 \widehat h^*_1 \widehat h^*_2}\\+ 2L_1(\Tt)\{\widehat h_1^*\}^{2
				H_1(\Tt)}+ 2L_2(\Tt)\{\widehat h_2^*\}^{2H_2(\Tt)}.
		\end{multline*}
		Replacing the expressions of $\widehat h^*_1$ and $\widehat h^*_2$, we have 
		$$
					\frac{\kappa^2}{c\pi} \frac{\sigma^2}{M_0 \widehat h_1^* \widehat h_2^*}=  \frac{\kappa^2\sigma^2}{c\pi} M_0^{ \alpha_1(\Tt)+\alpha_2(\Tt)-1}\Lambda_1^{-\alpha_1(\Tt)}(\Tt)\Lambda_2^{-\alpha_2(\Tt)}(\Tt)\times \widehat {\Upsilon}(\Tt),
		$$
	where 	
	\begin{equation*}
\widehat {\Upsilon}(\Tt)=
	M_0^{ \widehat \alpha_1(\Tt)+\widehat\alpha_2(\Tt)-\alpha_1(\Tt)-\alpha_2(\Tt)}\frac{\widehat \Lambda_1^{-\widehat\alpha_1(\Tt)}(\Tt)\widehat \Lambda_2^{-\widehat\alpha_2(\Tt)}(\Tt)}{ \Lambda_1^{-\alpha_1(\Tt)}(\Tt)\Lambda_2^{-\alpha_2(\Tt)}(\Tt)}.
		\end{equation*}
	Let $\EE_{M_0}[\cdot] = \EE[\cdot \mid M_0]$.	Then, on the event $\mathcal F\cap \mathcal E$,  using Cauchy-Schwarz inequality, 
\begin{multline*}
\EE_{M_0\!\!}\left[ \frac{\kappa^2}{c\pi} \frac{\sigma^2}{M_0 \widehat h^*_1 \widehat h^*_2}  \mathbf{1}_{\mathcal{F}\cap\mathcal{E} }  \right]\leq \frac{\kappa^2\sigma^2}{c\pi} M_0^{ \alpha_1(\Tt)+\alpha_2(\Tt)-1}\Lambda_1^{-\alpha_1(\Tt)}(\Tt)\Lambda_2^{-\alpha_2(\Tt)}(\Tt) \EE_{M_0}\left[  \widehat {\Upsilon}(\Tt)\mathbf{1}_{\mathcal{F}\cap\mathcal{E} } \right]
\\ \leq\frac{\kappa^2\sigma^2}{c\pi} M_0^{ \alpha_1(\Tt)+\alpha_2(\Tt)-1+2\log^{-a}(\mathfrak m)}\Lambda_1^{\!-\alpha_1(\Tt)}(\Tt)\Lambda_2^{\!-\alpha_2(\Tt)}(\Tt)\EE_{M_0\!\!} \left[\frac{\widehat \Lambda_1^{-2\widehat\alpha_1(\Tt)}(\Tt)\widehat \Lambda_2^{-2\widehat\alpha_2(\Tt)}(\Tt)}{ \Lambda_1^{-2\alpha_1(\Tt)}(\Tt)\Lambda_2^{-2\alpha_2(\Tt)}(\Tt)}\mathbf{1}_{\mathcal{F}\cap\mathcal{E} }\! \right]\!.
\end{multline*}
	Next, on the event $\mathcal E$, for $i=1,2$, we have
		$$
		\frac{\widehat \Lambda_i^{-2\widehat\alpha_i(\Tt)}(\Tt)}{\Lambda_i^{-2\alpha_i(\Tt)}(\Tt)}= \frac{\widehat \Lambda_i^{-2\widehat\alpha_i(\Tt)}(\Tt)}{\Lambda_i^{-2\widehat\alpha_i(\Tt)}(\Tt)}\frac{\Lambda_i^{-2\widehat\alpha_i(\Tt)}(\Tt)}{\Lambda_i^{-2\alpha_i(\Tt)}(\Tt)}\leq \left(1+\log^{-a}(\mathfrak m)\right)^{-2\widehat \alpha_i(\Tt)}\Lambda_i(\Tt)^{2\log^{-a}(\mathfrak m)}.
		$$
	Note that, by definition, $2\alpha_i(\Tt)<1$, $i=1,2$. 	It follows that on the event $\mathcal F \cap \mathcal E$ we have  
		$$
		\frac{\widehat \Lambda_1^{-2\widehat\alpha_1(\Tt)}(\Tt)\widehat \Lambda_2^{-2\widehat\alpha_2(\Tt)}(\Tt)}{ \Lambda_1^{-2\alpha_1(\Tt)}(\Tt)\Lambda_2^{-2\alpha_2(\Tt)}(\Tt)}= 1+O_\PP(\log^{-a}(\mathfrak m)).
		$$
		Consequently, under $\mathcal F \cap \mathcal E$ we obtain 
		$$
		\EE_{M_0\!\!} \left[ \frac{\kappa^2}{c\pi} \frac{\sigma^2}{M_0 \widehat h^*_1 \widehat h^*_2} \mathbf{1}_{\mathcal{F}\cap\mathcal{E} } \right]\leq \frac{\kappa^2 \sigma^2}{c\pi\Lambda_1^{\alpha_1(\Tt)}(\Tt)\Lambda_2^{\alpha_2(\Tt)}(\Tt)} M_0^{ -\frac{\omega(\Tt)}{2\omega(\Tt)+1}+2\log^{-a}(\mathfrak m)}  \{1+O(\log^{-a}(\mathfrak m))\}.
		$$
		By similar argument, we can also show that on the event $\mathcal F \cap \mathcal E$ we have the bound 
\begin{multline*}
			\EE_{M_0\!\!}\left[ L_1(\Tt) \{\widehat h^*_1\}^{2H_1(\Tt)}  \mathbf{1}_{\mathcal{F}\cap\mathcal{E} }\right] \\ \leq L_1(\Tt)\left[\Lambda_1(\Tt)^{2H_1(\Tt)+1}\big/\Lambda_2(\Tt)\right]^{\!\alpha_1(\Tt)}M_0^{-\frac{\omega(\Tt)}{2\omega(\Tt)+1}+2\log^{-a}(\mathfrak m)}\!\times \{1+O(\log^{-a}(\mathfrak m))\},	 
\end{multline*}
		and symmetrically the bound for $ \EE_{M_0\!\!}\left[ L_2(\Tt) \{\widehat h^*_2\}^{2H_2(\Tt)} \mathbf{1}_{\mathcal{F}\cap\mathcal{E} }\big{|}M_0\right]$.
	Since
		\begin{multline*}
			\EE_{M_0\!\!} \left[ \{\widehat X^{new}(\Tt;\widehat{\mathbf{B}}^* )-X^{new}(\Tt)\}^2 \right]\leq \EE_{M_0\!\!}\left[ \{\widehat X^{new}(\Tt;\widehat{\mathbf{B}}^* )-X^{new}(\Tt)\}^2\mathbf{1}_{\mathcal F}\mathbf{1}_{\mathcal E} \right]\\
			+\EE_{M_0\!\!}\left[ \{\widehat X^{new}(\Tt;\widehat{\mathbf{B}}^* )-X^{new}(\Tt)\}^2\mathbf{1}_{\overline{\mathcal F}} \right]+\EE_{M_0\!\!}\left[ \{\widehat X^{new}(\Tt;\widehat{\mathbf{B}}^* )-X^{new}(\Tt)\}^2\mathbf{1}_{\overline{\mathcal E}} \right],
		\end{multline*}
	and given the facts above, it remains to investigate the last two expectations in the last diplay. 
		Using \ref{LP4}, \ref{ass_D} and Cauchy-Schwarz inequality,  we get 
		$$
		\EE_{M_0\!\!}\left[ \{\widehat X^{new}(\Tt;\widehat{\mathbf{B}}^* )\!-\!X^{new}(\Tt)\}^2\mathbf{1}_{\overline{\mathcal F}} \right]\!+\EE_{M_0\!\!}\left[ \{\widehat X^{new}(\Tt;\widehat{\mathbf{B}}^* )\! -\! X^{new}(\Tt)\}^2\mathbf{1}_{\overline{\mathcal E}} \right]\! \!= o(\log^{-a}(\mathfrak m)).
		$$ 
		We finally deduce 
		$$
		\EE_{M_0\!\!}\left[ \{\widehat X^{new}(\Tt;\widehat{\mathbf{B}}^* )-X^{new}(\Tt)\}^2 \right]\leq \Gamma_2(\Tt) M_0^{-\frac{\omega(\Tt)}{2\omega(\Tt)+1}+2\log^{-a}(\mathfrak m)}\times \{1+O(\log^{-a}(\mathfrak m))\},
		$$
		with $\Gamma_2(\Tt)$ defined in Proposition \ref{risk_2}.		
		\end{proof}
%\end{appendix}


%\bibliographystyle{chicago}
\bibliographystyle{apalike}
%\bibliographystyle{plainnat}

\bibliography{biblio_final.bib}
\end{document}
