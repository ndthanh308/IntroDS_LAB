% !TeX root = ../MVFD_arxiv.tex

\section{Examples}\label{sec:example}

We propose two applications where our estimation approach of the local regularity for multivariate functional data opens the door to new procedures and sharp results. 


\subsection{Estimating the characteristics of general Gaussian processes}\label{BfMs}

The multifractional Brownian motion (MfBm) is a generalization of the standard fractional Brownian motion, where the  Hurst parameter is allowed to vary along the path. There are several possible definitions of such a process. They lead to indistinguishable processes, up to a multiplication by a deterministic function. Here, the multi-parameter,  anisotropic multifractional Brownian sheet, which is a multivariate extension, is defined following~\citet{herbin_06}. This definition relies on the so-called harmonizable representation of the MfBm, see \cite{peltier:inria}, \cite{benassi97}, \cite{ayache2011}, \cite{lebo2018} among others. 
\begin{definition}
	Set $d\in \mathbb{N}^\star$ and let $\Eeta=(\eta_1,\dotsc,\eta_d) : [0,\infty)^d\rightarrow (0,1)^d$ be a deterministic map. The multifractional Brownian sheet $W = (W(\Uu) : \Uu \in (0,\infty)^d)$ with Hurst functional parameter $\Eeta$ is defined as follows~:
	$$ 
	W(\Uu)= \left(\prod_{k=1}^d\frac{1}{C(\eta_k(\Uu))}\right)\int_{\mathbb{R}^d}\displaystyle\prod_{k=1}^d\frac{e^{i t_k\zeta_k}-1}{ |\zeta_k|^{\eta_k(\Uu)+\frac{1}{2}}}\widehat{\boldsymbol B}(\D \boldsymbol\zeta), \qquad  \Uu\in (0, \infty)^d,
	$$
	where $\boldsymbol  \zeta =(\zeta_1,\dots,\zeta_d)$ and $\widehat{\boldsymbol B} $ is the Fourier Transform of the white noise in $\mathbb{R}^d$. Here, for any positive $x$,
	$$
	C(x) = \left[ \frac{2\pi}{\Gamma(2x+1)\sin(\pi x)}\right]^{1/2}.
	$$
\end{definition}

Notice that, when $d=1$, the measure $\widehat{\boldsymbol B}(\D \boldsymbol\zeta)$ is the unique complex-valued Gaussian measure which can be associated to a standard Gaussian measure on $\RR$ by a `stochastic Parseval identity', see~\citet{stoev_taquu_2006}, equation~(2.4). In particular, the construction of $\widehat{\boldsymbol B}(\D \boldsymbol\zeta)$ ensures that $W$ is real-valued. 

We focus on the case $d=2$, and redefine $W=(W(\Uu) : \Uu\in\cU)$ as the restriction to an open subset $\cU\subset (0, \infty)^2$ of the multifractional Brownian sheet with Hurst functional parameter $\Eeta=(\eta_1, \eta_2)$. 
Note that $W$ is a centered Gaussian process with covariance function
\begin{equation*}
	\EE[W(\Uu)W(\Vv)]
	\!=\! \prod_{ k =1,2}\!\!
	D(\eta_{k }(\Uu),\eta_{k }(\Vv))
	\left[u_{k  }^{\eta_{k  }(\Uu)+\eta_{ k}(\Vv)}\!+\!v_{k }^{\eta_{ k  }(\Uu)\!+\!\eta_{ k  }(\Vv)}\!-|u_{ k }-v_{k }|^{\eta_{ k }(\Uu)+\eta_i(\Vv)}\right],
\end{equation*}
$\Uu =(u_1,u_2),\Vv=(v_1,v_2)\in\cU$, where 
\begin{equation}\label{def_D_func}
	D(x,y) = C^{\,2}((x+y)/2)\cdot(2C(x)C(y))^{-1} \;\;\; \text{ and } \;\;\;  D(x,x)\equiv 1/2.
\end{equation}
In particular, the variance of $W$ is given by $\EE[W^2(\Uu)]=u_1^{2\eta_1(\Uu)}u_2^{2\eta_2(\Uu)}$.

Moreover, we consider a domain deformation $A$, that is a positive  and invertible application $A:\cT\to \cU$. Let  
$$
X= W\circ A \quad \text{ and } \quad \theta(\Tt,\Ss)= \EE\left[\{X(\Tt)-X(\Ss)\}^2\right], \quad \forall \Tt,\Ss\in \cT.
$$



\begin{proposition}\label{mprop}
	 If  $\Eeta:\cU\rightarrow (0,1)^2 $  and $A:\cT\to \cU$   are continuously differentiable,
	\begin{multline*}
		\theta(\Tt, \Ss)
		= |A_1(\Tt)|^{2H_1(\Tt)}|\partial_1A_2(\Tt)(t_1-s_1)+\partial_2A_2(\Tt)(t_2-s_2)|^{2H_2(\Tt)}\\
		+|A_2(\Tt)|^{2H_2(\Tt)}|\partial_1A_1(\Tt)(t_1-s_1)+\partial_2A_1(\Tt)(t_2-s_2)|^{2H_1(\Tt)}+O(\|\Tt-\Ss\|^2)\\
		+O(\|\Tt-\Ss\|^{2\underline{H}(\Tt)+1})+O\left(\|\Tt-\Ss\|^{2H_1(\Tt) +2H_2(\Tt)}\right),
		\qquad
		\Tt, \Ss \in\cT,
	\end{multline*}
	where $\partial_1, \partial_2$ denote the partial derivatives and 
	$$H_1=\eta_1\circ A\quad \text{and}\quad H_2=\eta_2\circ A. $$
\end{proposition}

The proof of the following corollary is immediate, and will thus be omitted.


\begin{corollary}\label{mycoro}
	Assume the conditions of Proposition\ref{mprop}, and that there exist $\rho\in(0,1)$ such that 
	$$
	%\exists \rho >0,\quad \quad 
	0\leq \overline{H}(\Tt)-\underline{H}(\Tt)\leq \frac{1-\rho}{2}.$$
	Then  $X=W\circ A \in \mathcal{H}^{H_1,H_2}$
	with  $\mathbf L$ given  by~: 
	$$
	L_1^{(1)}(\Tt)= |A_2(\Tt)|^{2H_2(\Tt)}|\partial_1 A_1(\Tt)|^{2H_1(\Tt)},\quad L_2^{(1)}(\Tt)=|A_1(\Tt)|^{2H_1(\Tt)}|\partial_1 A_2(\Tt)|^{2H_2(\Tt)},$$
	$$L_1^{(2)}(\Tt)=|A_2(\Tt)|^{2H_2(\Tt)}|\partial_2 A_1(\Tt)|^{2H_1(\Tt)},\quad L_2^{(2)}(\Tt)=|A_1(\Tt)|^{2H_1(\Tt)}|\partial_2 A_2(\Tt)|^{2H_2(\Tt)} .
	$$
\end{corollary}

 Let us note that without domain deformation, \emph{i.e.}, when $A$ is the identity, $\mathbf L=(1,0,0,1)$.  The estimation approach introduced in Section \ref{sec_*} allows to estimate $H_1$, $H_2$ and $\mathbf L$  in general. The  estimation of the domain deformation $A$ is  investigated in the following.


\subsubsection {Estimating equations for the domain  deformation}
When one realization of the process is observed on a dense, regular grid, the estimation of the Hurst function of a multifractional Brownian motion was considered by \cite{hsing2020}. See also \cite{hsing2016}.The use of deformation to model non-stationary processes was first introduced  to the spatial statistics literature by \cite{sampson92}. One dimensional deformations behave locally  as a change of scale. In two dimension, deformations can rotate,  as well as scale local coordinates. See \cite{anderes2008}, \cite{anderes2009consistent} and \cite{Clerc2003} for more details. The fact that the deformation can rotate is mainly  related to the identification problem discussed in Section \ref{sec_*}. 


As a consequence of our  new approach,  we can build a nonparametric estimator of the deformation $A$ under mild technical conditions. We consider that 
\begin{equation}\label{ini_cd}
	 \text{some   $(t_0,s_0)\in\cT$  is  given for which  $A_1(t_0,s_0)$ and $A_2(t_0,s_0)$ are known. }
\end{equation}
This  initial condition avoids identification issues arising in a fully non parametric setup. We also  assume that the time-deformation $A$ 
is such that 
\begin{equation}\label{simpl_A}
\inf_{\Tt\in\cT} A_k(\Tt) >0, \quad  \inf_{\Tt\in\cT} \partial_{i} A_k(\Tt) \geq 0 \quad \text{ and } \quad \inf_{\Tt\in\cT} \{\partial_{1} A_k(\Tt) +\partial_{2} A_k(\Tt) \} >0, \qquad    i,k=1,2. 
\end{equation}
%where $ \partial_{i}$ stands for the partial derivative.
Finally, we set $H_1(\Tt)<H_2(\Tt)$  and focus on the first coordinate $A_1$ of the deformation $A$.  By Corollary \ref{mycoro}, we have   
$$
L_1^{(1)}(\Tt)= A_2(\Tt)^{2H_2(\Tt)}\partial_1 A_1(\Tt)^{2H_1(\Tt)}.
$$
Since the variance of $X$ is given by
\begin{equation} \label{rel_v_A}
v(\Tt)=\EE[X(\Tt)^2]=A_2(\Tt)^{2H_2(\Tt)}A_1(\Tt)^{2H_1(\Tt)},
\end{equation}
it follows that 
$$ \left(\frac{L_1^{(1)}(\Tt)}{v(\Tt)}\right)^{\frac{1}{2H_1(\Tt)}}=\frac{\partial_1A_1(\Tt)}{A_1(\Tt)}.$$
Integrating both sides we obtain
$$
\log A_1(\Tt)=\int_{t_0}^{t_1} f_1(s,t_2){\D} s+h(t_2),\quad \text{ for} \quad \Tt=(t_1,t_2)\in \cT,
$$
where $h$ is a real-valued function of $t_2$ and 
$$
f_1(\Tt)= \left(\frac{L_1^{(1)}(\Tt)}{v(\Tt)}\right)^{\frac{1}{2H_1(\Tt)}}.
$$
The function   $h$ is determined by  
$$
\frac{h^\prime(t_2)}{h(t_2)}=g_1(t_0,t_2):=\left(\frac{L_1^{(2)}(t_0,t_2)}{v(t_0,t_2)}\right)^{\frac{1}{2H_1(t_0,t_2)}}.
$$
This leads us to the following estimating equation~: 
\begin{equation}\label{est_A1}
A_1(\Tt)=\lambda_1\exp\left(\int_{t_0}^{t_1} f_1(s,t_2){\D} s+\int_{s_0}^{t_2}g_1(t_0,s)\D s\right),
\quad \text{ where } \quad \lambda_1=A_1(t_0,s_0).
\end{equation}
An estimator $\widehat{A}_1$ of    the first component of the domain deformation is easily obtained  
by replacing $f_1$ and $g_1$ by their estimates in \eqref{est_A1}. Estimators of $f_1$ and $g_1$ are  naturally obtained by plugging into their expressions  the estimators of    $L_1^{(1)}$, $L_1^{(2)}$,  $H_1$ and  an estimator $\widehat v(\Tt)$ of the variance $v(\Tt)$.


To provide a theoretical result for $\widehat{A}_1$, for simplicity, in addition to \eqref{low_thres}, we assume that constants $\underline\beta$, $\overline\beta$ are known such that 
\begin{equation}\label{simpl_bet}
0< \underline\beta \leq \min_{k=1,2} \inf_{\Tt\in\mathcal T} H_k(\Tt) \qquad \text{ and } \qquad %C_{\mathbf L} 
 \max_{i=1,2}\sup_{\Tt \in \cT} L_1^{(i)}(\Tt) \leq \overline \beta . 
\end{equation}
We then truncate correspondingly  the estimators, \emph{i.e.}, we replace  $\widehat H_k(\Tt)$ and $\widehat{L^{(i)}_1}(\Tt)$  by 
$$
\max \{\widehat H_k(\Tt), \underline \beta \}  \qquad \text{ and  }\qquad  \min\left\{\widehat{L^{(i)}_1}(\Tt), \overline\beta\right\},\qquad \forall \Tt\in \cT,\; k,i=1,2, 
$$  
respectively. Given the relationship \eqref{rel_v_A} and  condition \eqref{simpl_A}, 
the variance $v(\Tt)$ is necessarily  bounded away from zero.
Finally,   for  the estimator of $v(\Tt)$, we assume that, a constant $C_v$ exists such that 
\begin{equation} \label{simpl_v}
\sup_{\Tt\in \cT} \mathbb E\left[ \{v(\Tt)/ \widehat v (\Tt)\}^p\right]	 < C_v^p, \qquad \forall p\geq 1 .
\end{equation}
This  condition   can be satisfied  if, for instance, a positive lower bound $\underline a$ for $A_1$ and $A_2$ is known in \eqref{simpl_A}. By \eqref{rel_v_A}, we then have 
$$
v(\Tt)> \underline v := \min (\underline a ^4,1).
$$
In this case,  $\widehat v (\Tt)$ can be simply defined as maximum between $\underline v$ and the empirical second order moment of the observable approximations $\widetilde X^{(j)}$.

Let
		$$
		F_1 :=\sup_{\Tt\in \cT}\EE\left[|\widehat f_1(\Tt)-f_1(\Tt)|\right]  ,\qquad  G_1 := \sup_{\Tt\in \cT}\EE\left[|\widehat g_1(\Tt)-g_1(\Tt)|\right]  , 
		$$ 
and $\operatorname{diam}(\mathcal T)= \sup_{\Ss^\prime,\Ss\in \cT}\|\Ss^\prime -\Ss\|$.  For the next result,  let  $\Delta = \mathfrak m^{-a}$ and $\rho(\mathfrak m) = \mathfrak m^{-b}$, with $a>0$, $b\geq 0$ and $\rho(\mathfrak m)$ introduced in Assumption \ref{ass_H2}. Moreover, let  
$$
  \chi (\Tt) = a\{2D(\Tt)+1\} - \min\{a\underline H (\Tt), b/2 \} >0. 
$$

\begin{proposition}\label{prop_def_A}
The assumptions of Propositions  \ref{propCH} and \ref{mprop}, and conditions \eqref{ini_cd}, \eqref{simpl_A}, \eqref{simpl_bet} and \eqref{simpl_v} hold true. Moreover, we assume that  constants $\mathfrak a_1$ and $\mathfrak A_1$ exist such that 
$$
\EE[X(\Tt)^{2p}]\leq \frac{p!}{2}\mathfrak a_1\mathfrak A_1^{p-2},\quad \forall p\in\{1,2\ldots\}.
$$ 
Then $F_1+G_1 <\infty$. 
Moreover, let 
\begin{equation}\label{cdt_ell}
\left[ 1/2-   \chi (\Tt)   \liminf_{\mathfrak m, N} \{\log(\mathfrak m)/\log(N)\}\right]_+<\ell < 1/2.
\end{equation}
Then, if $\mathfrak m$ ad $N$ are sufficiently large, positive constants $\mathfrak C_v$, $\tilde q_1$ and $\tilde q_2$ exist such that, 
\begin{multline}		\EE \left[\left|\widehat A_1(\Tt)-A_1(\Tt) \right| \right]\\  \leq \mathfrak C_v A_1(\Tt) \operatorname{diam}(\mathcal T)  \max\left\{1,F_1,G_1\right\}  \left\{  
 \frac{a\log(\mathfrak m) }{	\mathfrak m^{  2aD  (\Tt) - \chi (\Tt)   } } 
	N^{\ell-1/2}+\tilde q_1
	\exp(-\tilde q_2N^\ell) \right\}.
\end{multline}
\end{proposition}

A similar representation can be derived for $A_2$, that is 
\begin{equation}\label{est_A2}
	A_2(\Tt)=\lambda_2\exp\left(\int_{t_0}^{t_1} f_2(s,t_2){\D} s+\int_{s_0}^{t_2}g_2(t_0,s)\D s\right),
\end{equation}
where  
$$
f_2(\Tt)= \left(\frac{L_2^{(1)}(\Tt)}{v(\Tt)}\right)^{\frac{1}{2H_2(\Tt)}},\quad g_2(t_0,t_2)= \left(\frac{L_2^{(2)}(t_0,t_2)}{v(t_0,t_2)}\right)^{\frac{1}{2H_2(t_0,t_2)}}\quad \text{and}\quad \lambda_2 =A_2(t_0,s_0).
$$
Estimators of $f_2$ and $g_2$,  are  obtained by plugging into their expressions  the estimators of $L_2^{(i)}$,  $i=1,2$,  $H_2$, and an estimator of  $v(\Tt)$. Let $\widehat A_2$ be the estimator of $A_2$ obtained by plug-in using \eqref{est_A2}. Under the conditions of Proposition \ref{prop_def_A}, we can show that $F_2+G_2 $ is finite and derive a similar bound for the $\mathbb L^1-$risk of $\widehat A_2$. The arguments are similar and thus omitted. 

 
 
 