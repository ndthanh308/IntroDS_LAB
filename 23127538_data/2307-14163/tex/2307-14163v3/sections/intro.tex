% !TeX root = ../MVFD_arxiv.tex


\section{Introduction}\label{sec:introd}


Functional data analysis (FDA) provides methods for dealing with complex data such that collected by modern sensing devices. See, for instance, the textbooks \cite{ramsay_functional_2005}, \cite{horvath2012inference}, \cite{koko}. The paradigm consists of considering  that data are generated by a sample of functions, realizations of a stochastic process or random field defined over a continuous domain. However, the realizations are practically never observed over a continuous domain, and rarely without error. All the data points generated by such a realization then represent a single observation unit.  The remarkable advantage of functional data analysis is that it can combine information both within and between realizations. Restrictive assumptions, such as stationarity, stationary increments or Gaussianity on the data generating process or random field, can therefore be avoided.

We focus here on the case where the realizations are surfaces, \emph{i.e.}, the realizations are generated by a random scalar  field defined over a multi-dimensional continuous domain. We call this framework \emph{multivariate functional data}, and  focus on the case of a domain in the plane. 
Thus, in a different wat, we use existing FDA terminology, which  usually refers to a vector-valued  processes defined over an interval. In recent years, a wide panel of applications from different areas, including Astrophysics, Climate Sciences, Geophysics and Material Sciences, deal with data which can be considered as generated by random surfaces. 
For instance, it is now well admitted that the world ocean plays a key role in regulating Earth’s climate. Modern tools, such as floating sensors, provide  ocean heat transport measurements , which are made freely available by international programs, such as the Argo Program (http://www.argo.ucsd.edu, http://argo.jcommops.org). If one studies a specific area of the ocean, an observation unit is represented by the measurements collected at random points, sparsely distributed over the area, at some date in the year. See, for instance  \cite{kuusela_2018}, \cite{park2023}, and the references therein, for studies on Argo data. We aim at providing a new perspective for refined and effective analysis of multivariate functional data. 

Our main contribution is a new approach for studying the local regularity of random fields in the context of multivariate functional data. A main example of random field we have in mind is the  multifractional Brownian sheet, see \cite{herbin_06}. In the case of curves, that means for random fields defined over an interval, the local regularity can be defined naturally using the expectation of the squared increments to which one can impose a Hölder-like condition. The local regularity is then determined by the Hölder exponent and the Hölder constant. See \cite{GKP}. See also \cite{fracdim} where the local regularity exponent is linked to the fractal dimension for self-similar Gaussian processes. We here extend the ideas of \cite{GKP} to random fields defined over a domain in the plane. We thus introduce a general notion of  local regularity satisfied by a large class of random fields,  propose simple estimation procedures for the regularity parameters, and prove non-asymptotic results. \cite{fastandexact} consider a related estimation idea for a particular class of random fields, and use it to efficiently and exactly simulate fractional Brownian surfaces. \cite{hsing2020} study the estimation of the local regularity of a multifractional Brownian sheet from one realization of the process observed on a  regular grid, and provide asymptotic theory.


Knowing the regularity of the data generating random field has important consequences for FDA. For instance, there has been increasing interest in the nonparametric  estimation of the characteristics, such as the mean and the covariance structure, of the random field generating the functional data. See \cite{caponera2022} for a valuable review and an interesting approach. It is well-known that the optimal accuracy of nonparametric estimates depends on the regularity of the realizations. See, \cite{cai2010}, \cite{cai2011}, \cite{CMsphere}, \cite{GKP}. Inference methods for functional data should thus adapt to the regularity of the underlying process when aiming at optimality.  \cite{golovkine2023adaptive} and \cite{wei2023adaptive} used  regularity estimators to derive new, easy to implement, adaptive procedures for mean, covariance and functional principal components analysis for data generated by random curves. Adaptation to regularity for multivariate functional 
data seems yet unexplored. 


The paper is organized as follows. In Section \ref{sec2}, the general observation scheme we consider is presented. It encompasses the  scenarios of \textit{common design} 
(the domain points where the random field realizations are  observed, possibly with noise,  are the same for all realizations)  and \textit{random design} (the observation domain points are randomly generated for each realization). Moreover, a general class of bivariate stochastic processes, for which the local regularity is well defined, is introduced. After discussion of some identification issues, in Section \ref{sec_*}, we present the estimation approach for the local regularity exponents, as well as the corresponding Hölder constants. Our estimators adapt to both isotropic and anisotropic settings. In Section \ref{sec4}, we provide  concentration bounds for the estimators of the local regularity, as well as a risk bound for the anisotropy detection. The new results are  of the non-asymptotic type, in the sense that they hold for any  number of random field realizations and observation domain points, provided these numbers are sufficiently large. Our estimation approach to local regularity for multivariate functional data opens the door to a large  array of adaptive procedures. Two applications are proposed. In Section \ref{BfMs}, we  consider the class of multi-fractional Brownian sheets with domain deformation, an example of a random field that belongs to the class defined in Section \ref{sec_*}. Deformed random fields have been studied in the literature, see for instance  \cite{Clerc2003},
\cite{anderes2008},
\cite{anderes2009consistent}, but yet seem little explored in the context of the functional data paradigm. 
As a second application, in Section \ref{sec6} we consider the problem of nonparametric reconstruction of the realizations of a random field from noisy measurements over a discrete set in the domain. A related problem was addressed by \cite{dun_aniso} in the context of Gaussian processes. With  the regularity estimates in hand, we build an adaptive Nadaraya-Watson pointwise estimates, and provide a sharp non-asymptotic bound for the average risk which achieves the optimal minimax rate expected in nonparametric statistics. The proofs of our results are given in the Appendix. Additional proofs and technical lemmas are provided in the Supplement. 


