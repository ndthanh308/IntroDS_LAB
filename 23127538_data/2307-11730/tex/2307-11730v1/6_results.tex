\section{Results}\label{sec:results}

This section assesses the security module performance focused on performance indicators such as the $F_{1} \ score$ for federated models, the percentage of CPU and RAM usage, network traffic quantified in megabytes $(\text{MB})$, and model convergence time.

The diagram depicted in \figurename~\ref{fig:model_performance} demonstrates the average $F_{1} \ score$ for the federated models, using the MNIST dataset, under three separate security configurations: baseline with no security techniques, encryption, and encryption combined with MTD techniques. All three configurations exhibit a consistent growth pattern in the early stages of the federation process ($\approx$10 minutes). The baseline configuration continues upward, achieving an $F_{1} \ score$ of 97\%. This indicates the potential for high performance when security overheads are absent. However, when examining the configurations that include security measures, there is a slight decline in the $F_{1} \ score$. In the encryption configuration, the $F_{1} \ score$ peaks at 94\%, while in the combined encryption and MTD setting, the $F_{1} \ score$ fluctuates between 92.5\%. The variations observed throughout the federation process are likely due to the occasional computational overhead of the security mechanisms during the processing and transmission of data.

% Figure environment removed

A more granular view of the performance is provided by \figurename~\ref{fig:results_physical}, which breaks down the performance into CPU, RAM, and Network metrics across the three security configurations. The observations corroborate the inherent trade-off between security and computational resources. The baseline configuration stands as the baseline, its usage determined by the computational demands of the training process, which averages at 54.6\% ($\pm$1.8\%). Upon the introduction of encryption, CPU usage is expected to rise due to the computational overhead associated with encrypting and decrypting communication. This is manifested in the encryption configuration where the CPU usage averages 60.9\% ($\pm$3.7\%). Further elevations in CPU usage are seen when encryption is combined with MTD. The additional processing required for managing dynamic communication routes leads to a higher average CPU usage of 63.2\% ($\pm$3.5\%).

% Figure environment removed

A similar pattern is observed for RAM usage. The baseline configuration exhibits a lower average of 31.9\% ($\pm$2.3\%), reflecting the lower computational footprint when security measures are absent. However, including encryption mechanisms results in a slight increase in RAM usage due to the additional memory demands of the encryption process. Specifically, the encryption configuration averages 33.8\% ($\pm$2.41\%), and when the MTD technique is added alongside encryption, RAM usage averages 35.9\% ($\pm$1.5\%). This is attributable to the additional memory required for managing dynamic communication routes under MTD. Despite the marginal increase, it underscores the added resource requirements induced by security features.

Furthermore, network traffic provides critical insights into the performance impacts of different security configurations. The average network traffic for the baseline configuration remains fairly modest, averaging around 110.2 MB ($\pm$12 MB). However, the integration of security mechanisms leads to an increase in network usage. The encryption configuration generates an average of 185.2 MB ($\pm$21 MB) of network traffic, while the encryption with MTD configuration pushes the average even higher, reaching 226 MB ($\pm$15 MB). These figures suggest the additional network overhead incurred by transmitting encrypted data and MTD-related messages.

As evidenced by Table \ref{table:results}, these results underscore an inherent tension in securing DFL. While deploying security protocols such as encryption and MTD is indispensable for safeguarding various aspects of the federated learning process, these measures invariably come with additional computational and network overheads. These escalated resource requirements, although a trade-off, provide a safeguard against the pervasive risk of data breaches and cyberattacks. This study thus offers an empirical guide, presenting the performance implications of various security configurations in real-world DFL scenarios. It showcases the balance between achieving high predictive accuracy and maintaining stringent security standards.

\begin{table*}[htb!]
\caption{Security Settings, Protection, and Performance in DFL}
\label{table:results}
\centering
\small
\begin{threeparttable}
\begin{tabular}{|p{3.3cm}|p{4cm}|p{1.5cm}|p{1.2cm}|p{1.2cm}|p{1.5cm}|}
\hline
\makecell{\textbf{Security}\\\textbf{Configuration}} & \makecell{\textbf{Information}\\\textbf{Protected}} & \multicolumn{4}{c|}{\textbf{Performance Metrics}} \\ 
& & \textbf{F1 Score} & \textbf{CPU} & \textbf{RAM} & \textbf{Network} \\
\hline\hline
Baseline (No security) & N/A & 97\% & 54.6\% $\pm$1.8\% & 31.9\% $\pm$2.3\% & 110.2 MB $\pm$12 MB \\
\hline
Encryption & $\sbullet[0.75]$ Model Parameters \newline $\sbullet[0.75]$ Roles \newline $\sbullet[0.75]$ Communication Patterns & 94\% & 60.9\% $\pm$3.7\% & 33.8\% $\pm$2.41\% & 185.2 MB $\pm$21 MB \\
\hline
Encryption + MTD & $\sbullet[0.75]$ Model Parameters \newline $\sbullet[0.75]$ Roles \newline $\sbullet[0.75]$ Communication Patterns \newline $\sbullet[0.75]$ Topology \newline $\sbullet[0.75]$ Activity Periods & 92.5\% & 63.2\% $\pm$3.5\% & 35.9\% $\pm$1.5\% & 226 MB $\pm$15 MB \\
\hline
\end{tabular}
\end{threeparttable}
\end{table*}