%Version 2.1 April 2023
% See section 11 of the User Manual for version history
%
%%%%%%%%%%%%%%%%%%%%%%%%%%%%%%%%%%%%%%%%%%%%%%%%%%%%%%%%%%%%%%%%%%%%%%
%%                                                                 %%
%% Please do not use \input{...} to include other tex files.       %%
%% Submit your LaTeX manuscript as one .tex document.              %%
%%                                                                 %%
%% All additional figures and files should be attached             %%
%% separately and not embedded in the \TeX\ document itself.       %%
%%                                                                 %%
%%%%%%%%%%%%%%%%%%%%%%%%%%%%%%%%%%%%%%%%%%%%%%%%%%%%%%%%%%%%%%%%%%%%%

\RequirePackage{amsthm}

%%\documentclass[referee,sn-basic]{sn-jnl}% referee option is meant for double line spacing

%%=======================================================%%
%% to print line numbers in the margin use lineno option %%
%%=======================================================%%

%%\documentclass[lineno,sn-basic]{sn-jnl}% Basic Springer Nature Reference Style/Chemistry Reference Style

%%======================================================%%
%% to compile with pdflatex/xelatex use pdflatex option %%
%%======================================================%%

%%\documentclass[pdflatex,sn-basic]{sn-jnl}% Basic Springer Nature Reference Style/Chemistry Reference Style


%%Note: the following reference styles support Namedate and Numbered referencing. By default the style follows the most common style. To switch between the options you can add or remove Numbered in the optional parenthesis. 
%%The option is available for: sn-basic.bst, sn-vancouver.bst, sn-chicago.bst, sn-mathphys.bst. %  
 
%%\documentclass[sn-nature]{sn-jnl}% Style for submissions to Nature Portfolio journals
%%\documentclass[sn-basic]{sn-jnl}% Basic Springer Nature Reference Style/Chemistry Reference Style
\documentclass[pdflatex,sn-mathphys,Numbered,iicol]{sn-jnl}% Math and Physical Sciences Reference Style
%%\documentclass[sn-aps]{sn-jnl}% American Physical Society (APS) Reference Style
%%\documentclass[sn-vancouver,Numbered]{sn-jnl}% Vancouver Reference Style
%%\documentclass[sn-apa]{sn-jnl}% APA Reference Style 
%%\documentclass[sn-chicago]{sn-jnl}% Chicago-based Humanities Reference Style
%%\documentclass[default]{sn-jnl}% Default
%%\documentclass[default,iicol]{sn-jnl}% Default with double column layout

%%%% Standard Packages
%%<additional latex packages if required can be included here>

\usepackage{graphicx}%
\usepackage{multirow}%
\usepackage{amsmath,amssymb,amsfonts}%
\usepackage{amsthm}%
\usepackage{mathrsfs}%
\usepackage[title]{appendix}%
\usepackage{textcomp}%
\usepackage{manyfoot}%
\usepackage{booktabs}%
\usepackage{algorithmicx}%
\usepackage{algpseudocode}%
\usepackage{listings}%
%%%%

\usepackage{subcaption}
\usepackage{caption}
\usepackage{natbib}
\usepackage{soul}
\usepackage{color}
\usepackage[dvipsnames,table]{xcolor}
\usepackage{pifont}
\usepackage{graphicx}
\usepackage{svg}
\usepackage[inline]{enumitem}
\usepackage{array}
\usepackage{longtable}
\usepackage{tabularx}
\usepackage{graphicx}
\usepackage{makecell}
\usepackage{lscape}
\usepackage{threeparttable}
\usepackage{xurl}
\usepackage{wasysym}
\usepackage{tikz}
\newcommand*\circled[1]{\tikz[baseline=(char.base)]{
            \node[shape=circle,draw,inner sep=1pt] (char) {#1};}}

\newcommand{\cmark}{\ding{51}}%
\newcommand{\xmark}{\ding{55}}%

\usepackage{algorithm}
\usepackage{algpseudocode}

\usepackage[
  separate-uncertainty = true,
  multi-part-units = repeat
]{siunitx}

\usepackage{tablefootnote}
\usepackage{footnote}
\makesavenoteenv{tabular}
\makesavenoteenv{table}
\makesavenoteenv{table*}

\setstcolor{blue}

\newcommand{\blue}[1]{\textcolor{blue}{#1}}

\newtoggle{finalPaper}
\togglefalse{finalPaper} % Change \toggletrue <-> \togglefalse
%\toggletrue{finalPaper} % Change \toggletrue <-> \togglefalse

\iftoggle{finalPaper} {
    \newcommand{\addtxt}[1]{#1}
    \newcommand{\addtable}[1]{#1}
    \newcommand{\change}[2]{#2}
    \newcommand{\rmvtxt}[1]{}
} {
    \newcommand{\addtxt}[1]{\textcolor{blue}{#1}}
    \newcommand{\addtable}[1]{\color{blue}{#1}}
    \newcommand{\change}[2]{\st{#1}\textcolor{blue}{#2}}
    \newcommand{\rmvtxt}[1]{\st{#1}}
}

\definecolor{ao}{rgb}{0.0, 0.5, 0.0}
\definecolor{amber}{rgb}{1.0, 0.49, 0.0}
\newcommand{\yestick}{{\color{ao}\ding{51}}}
\newcommand{\notick}{{\color{red}\ding{55}}}
\newcommand{\partialtick}{{\textbf{\color{amber}$\mathord{?}$}}}

\newcommand{\specialcell}[2][c]{%
  \begin{tabular}[#1]{@{}c@{}}#2\end{tabular}}

\newcommand\sbullet[1][.5]{\mathbin{\vcenter{\hbox{\scalebox{#1}{$\bullet$}}}}}

%\usepackage{lineno}
%\modulolinenumbers[5]
\definecolor{gray45}{gray}{.45}
\definecolor{gray75}{gray}{.75}
\definecolor{orange-fig}{HTML}{C55A11}

\hyphenation{data-sets}



%\jyear{2021}%

%% as per the requirement, new theorem styles can be included as shown below
\theoremstyle{thmstyleone}%
\newtheorem{theorem}{Theorem}%  meant for continuous numbers
%%\newtheorem{theorem}{Theorem}[section]% meant for section wise numbers
%% optional argument [theorem] produces theorem numbering sequence instead of independent numbers for Proposition
\newtheorem{proposition}[theorem]{Proposition}% 
%%\newtheorem{proposition}{Proposition}% to get separate numbers for theorem and proposition etc.

\theoremstyle{thmstyletwo}%
\newtheorem{example}{Example}%
\newtheorem{remark}{Remark}%

\theoremstyle{thmstylethree}%
\newtheorem{definition}{Definition}%

\raggedbottom
%%\unnumbered% uncomment this for unnumbered level heads

\begin{document}

%\title[Mitigating Communications Threats in Decentralized Federated Learning through MTD Techniques]{Mitigating Communications Threats in Decentralized Federated Learning through MTD Techniques}

\title[Mitigating Communications Threats in Decentralized Federated Learning through Moving Target Defense]{Mitigating Communications Threats in Decentralized Federated Learning through Moving Target Defense}

%%=============================================================%%
%% Prefix	-> \pfx{Dr}
%% GivenName	-> \fnm{Joergen W.}
%% Particle	-> \spfx{van der} -> surname prefix
%% FamilyName	-> \sur{Ploeg}
%% Suffix	-> \sfx{IV}
%% NatureName	-> \tanm{Poet Laureate} -> Title after name
%% Degrees	-> \dgr{MSc, PhD}
%% \author*[1,2]{\pfx{Dr} \fnm{Joergen W.} \spfx{van der} \sur{Ploeg} \sfx{IV} \tanm{Poet Laureate} 
%%                 \dgr{MSc, PhD}}\email{iauthor@gmail.com}
%%=============================================================%%

\author*[1]{\fnm{Enrique Tomás} \sur{Martínez Beltrán}}\email{enriquetomas@um.es}
\author[1]{\fnm{Pedro Miguel} \sur{Sánchez Sánchez}}\email{pedromiguel.sanchez@um.es}
\author[1]{\fnm{Sergio} \sur{López Bernal}}\email{slopez@um.es}
\author[2]{\fnm{Gérôme} \sur{Bovet}}\email{gerome.bovet@armasuisse.ch}
\author[1]{\fnm{Manuel} \sur{Gil Pérez}}\email{mgilperez@um.es}
\author[1]{\fnm{Gregorio} \sur{Martínez Pérez}}\email{gregorio@um.es}
\author[3]{\fnm{Alberto} \sur{Huertas Celdrán}}\email{huertas@ifi.uzh.ch}

%\affil*[1]{\orgdiv{Department of Information and Communications Engineering}, \orgname{University of Murcia}, \orgaddress{\street{Street}, \city{City}, \postcode{30100}, \state{State}, \country{Spain}}}
\affil[1]{\orgdiv{Department of Information and Communications Engineering}, \orgname{University of Murcia}, \orgaddress{\postcode{30100}, \country{Spain}}}
\affil[2]{\orgdiv{Cyber-Defence Campus}, \orgname{Armasuisse Science and Technology}, \orgaddress{\postcode{3602}, \city{Thun}, \country{Switzerland}}}
\affil[3]{\orgdiv{Communication Systems Group, Department of Informatics (IFI)}, \orgname{University of Zurich}, \orgaddress{\postcode{8050}, \city{Zürich}, \country{Switzerland}}}

\abstract{The rise of Decentralized Federated Learning (DFL) has enabled the training of machine learning models across federated participants, fostering decentralized model aggregation and reducing dependence on a server. However, this approach introduces unique communication security challenges that have yet to be thoroughly addressed in the literature. These challenges primarily originate from the decentralized nature of the aggregation process, the varied roles and responsibilities of the participants, and the absence of a central authority to oversee and mitigate threats. Addressing these challenges, this paper first delineates a comprehensive threat model, highlighting the potential risks of DFL communications. In response to these identified risks, this work introduces a security module designed for DFL platforms to counter communication-based attacks. The module combines security techniques such as symmetric and asymmetric encryption with Moving Target Defense (MTD) techniques, including random neighbor selection and IP/port switching. The security module is implemented in a DFL platform called Fedstellar, allowing the deployment and monitoring of the federation. A DFL scenario has been deployed, involving eight physical devices implementing three security configurations: (i) a baseline with no security, (ii) an encrypted configuration, and (iii) a configuration integrating both encryption and MTD techniques. The effectiveness of the security module is validated through experiments with the MNIST dataset and eclipse attacks. The results indicated an average F1 score of 95\%, with moderate increases in CPU usage (up to 63.2\% $\pm$3.5\%) and network traffic (230 MB $\pm$15 MB) under the most secure configuration, mitigating the risks posed by eavesdropping or eclipse attacks.}

\keywords{Federated Learning, Decentralized Network, Collaborative Models, Mitigation Techniques, Cybersecurity}
%%\pacs[JEL Classification]{D8, H51}
%%\pacs[MSC Classification]{35A01, 65L10, 65L12, 65L20, 65L70}

\maketitle


%%%%%%%%%%%%%%%%%%%%%%%%%%%%%%%%%%%%%%%%%%%%%%%%%%%%%%%%%%%%


\section{Introduction}
Deep learning models have been widely used in many applications.
For example, BERT~\citep{devlin_bert_2019}, GPT-3~\citep{brown_language_2020}, and T5~\citep{raffel_exploring_2020} achieved state-of-the-art~(SOTA) results on different natural language processing~(NLP) tasks. 
For computer vision~(CV), Transformer-like models such as ViT~\citep{dosovitskiy_image_2021} and Swin Transformer~\citep{liu_swin_2021} deliver excellent accuracy performance upon multiple tasks. 


At the same time, training deep learning models has been a critical problem troubling the community due to the long training time, especially for those large models with billions of parameters~\citep{brown_language_2020}. 
In order to enhance the training efficiency, researchers propose some manually designed parallel training strategies~\citep{narayanan_efficient_2021,shazeer_mesh-tensorflow_2018,xu_gspmd_2021}. 
However, selecting, tuning, and combining these strategies require extensive domain knowledge in deep learning models and hardware environments. With the increasing diversity of modern hardware architectures~\cite{flynn_very_1966,flynn_computer_1972} and the rapid development of deep learning models, these manually designed approaches are bringing heavier burdens to developers. 
Hence, \emph{automatic parallelism} is introduced to automate the parallel strategy searching for training models.


There are two main categories of parallelism in deep learning models: inter-layer parallelism~\citep{huang_gpipe_2019,narayanan_pipedream_2019,narayanan_memory-efficient_2021,fan_dapple_2021,li_chimera_2021,lepikhin_gshard_2021,du_glam_2022,fedus_switch_2022} and intra-layer parallelism~\citep{li_pytorch_2020,narayanan_efficient_2021,rasley_deepspeed_2020,fairscale_authors_fairscale_2021}. 
Inter-layer parallelism partitions the model into disjoint sets on different devices without slicing tensors. 
Alternatively, intra-layer parallelism partitions tensors in a layer along one or more axes and distributes them across different devices.


Current automatic parallelism techniques focus on optimizing strategies within these two categories. However, they treat these two categories separately. 
Some methods~\citep{zhao_vpipe_2022,jia_exploring_2018,cai_tensoropt_2022,wang_supporting_2019,jia_beyond_2019,schaarschmidt_automap_2021,liu_colossal-auto_2023} overlook potential opportunities for inter- or intra-layer parallelism, the others optimize inter- and intra-layer parallelism hierarchically and sequentially~\citep{narayanan_pipedream_2019,fan_dapple_2021,he_pipetransformer_2021,tarnawski_efficient_2020,tarnawski_piper_2021,zheng_alpa_2022}. 
As a result, current automatic parallelism techniques often fail to achieve the global optima and instead become trapped in local optima. 
Therefore, a unified inter- and intra-layer approach is needed to enhance the effectiveness of automatic parallelism.


This paper aims to find the optimal parallelism strategy while simultaneously considering inter- and intra-layer parallelism. 
It enables us to search in a more extensive strategy space where the globally optimal solution lurk. 
However, unifying inter- and intra-layer parallelism in automatic parallelism brings us two challenges. 
Firstly, to adopt a unified perspective on the inter- and intra-layer automatic parallelism, we should not formalize them with separate formulations as prior works. Therefore, how can we express these parallelism strategies in a unified formulation? 
Secondly, previous methods take a long time to obtain the solution with a limited strategy space. Therefore, how can we ensure that the best solution can be obtained in a reasonable time while expanding the strategy space?


To solve the above challenges, we propose UniAP. For the first challenge, UniAP adopts the mixed integer quadratic programming~(MIQP)~\citep{lazimy_mixed_1982} to search for the globally optimal parallel strategy automatically. 
It unifies the inter- and intra-layer automatic parallelism in a single MIQP formulation. 
For the second challenge, our complexity analysis and experimental results show that UniAP can obtain the globally optimal solution in a significantly shorter time.


The contributions of this paper are summarized as follows: 
\begin{itemize}
    \item We propose UniAP, the first framework to unify inter- and intra-layer automatic parallelism in model training.
    \item The optimal parallel strategies discovered by UniAP exhibit scalability on training throughput and strategy searching time.
    \item The experimental results show that UniAP speeds up model training on four Transformer-like models by up to 1.70$\times$ and reduces the strategy searching time by up to 16$\times$, compared with the SOTA method.
\end{itemize}

\section{Related Work}
\label{sec:related}

\begin{table}[t]
\small
\centering
\caption{Comparison of our method with related settings}
\begin{tabular}{cccc}
\toprule
Setting & Detect Novel OOD Data & Semi-Supervised & Learns from Novel OOD Data \\
\midrule
SSOD & \xmark & \cmark & \xmark \\ 
Open-World OD & \cmark & \xmark & \cmark \\
Open-Set SSOD & \cmark & \cmark & \xmark \\ \midrule
\textbf{Our Method} & \cmark & \cmark & \cmark \\
\bottomrule
\end{tabular}
\label{tab:comparison}
\end{table}

\paragraph{Semi-Supervised Object Detection.} Semi-supervised object detection (SSOD) approaches have become popular to reduce the need for labeling \cite{sohn2020detection, berthelot2019mixmatch, jeong2019consistency}. Pseudo-labeling based methods such as FlexMatch \cite{zhang2021flexmatch}, TSSDL \cite{shi2018transductive}, and others \cite{iscen2019label, luo2018smooth, yan2019semi, liu2021unbiased, xu2021end}, first train a teacher model using only labeled data and then use that model to create pseudo-labels for unlabeled images. The pseudo-labels are then used along with the original labeled data to train a student model. On the other hand, consistency regularization approaches such as \cite{sajjadi2016regularization, laine2017temporal, tarvainen2017mean, liu2021certainty, luo2018smooth, jeong2019consistency, iscen2019label, liu2021unbiased, xu2021end}, aim to minimize a consistency loss between differently augmented versions of an image. All of these semi-supervised learning approaches assume a ``closed-world'' setting with a fixed set of classes in both training and testing, which is not a valid assumption in real-world applications.

\paragraph{Open-World Object Detection.} Open-world object detection enables the detection of novel objects by incrementally adding novel object classes to the set of known classes. Previous work \cite{kim2022learning, kuo2015deepbox, o2015learning, wang2020leads, Maaz2022Multimodal} has studied different methods of object proposals for novel objects by attempting to remove the notion of class (all objects are regarded the same). ORE \cite{joseph2021towards} is the first to propose an open-world object detector that identifies novel classes as ‘unknown’ and proceeds to learn the unknown classes once the labels become available. \cite{han2022expanding} aims to identify unknown objects by separating high/low-density regions in the latent space. Both these approaches work in a fully-supervised setting. Our setup goes a step further and situates the open-world problem in the context of semi-supervised learning, with limited amounts of labeled ID data \textit{only}, that more closely resembles the real-world settings. 

\paragraph{Unsupervised Object Localization.} Recently proposed methods such as CutLER \cite{wang2023cut}, FreeSolo \cite{wang2022freesolo}, LOST \cite{LOST}, and MOST \cite{rambhatla2023most} propose to localize objects in an unsupervised manner, either by segmentation masks or bounding boxes. Some of these \cite{wang2023cut, LOST, rambhatla2023most} use features from self-supervised trained transformers to localize objects in the scene. In our work, we evaluate the capabilities of such methods for localizing OOD objects, as they present open-world capabilities. Based on our evaluation (\ref{sec:expts:ablation}), we use CutLER as part of the OOD Explorer to localize OOD classes. Section \ref{sec:expts} provides the details of our evaluation. 

\paragraph{Open-Set/Open-world Semi-Supervised Object Detection.}
The open-set semi-supervised object detection problem \cite{liuopen} addressed some of the limitation of the above mentioned work. Furthermore, they address like the performance of ID classes in the presence of OOD data, but they do not learn from it or improve OOD performance. They propose an offline OOD detector to filter out OOD data, thus limiting the risk of ID performance in the presence of OOD data. In contrast, our approach \textit{both} improves performance for ID classes \textit{as well as} OOD classes, i.e., our proposed framework solves a strictly stronger problem. Specifically speaking, \cite{liuopen} solves for identifying novel classes and filters it out, but does not re-introduce the classes back into the training pipeline in order to be able to learn its features. \cite{mullappilly2024semi} addresses some of the limitations of the previous mentioned methods by extending the problem to a semi-supervised setting. However, their problem setting is similar to an incremental learning setting, access to unknown class labels is provided in subsequent tasks. Our generalized setting, on the other hand, does not require access to any unknown class labels. 

\section{Communications Threat Model}\label{sec:threatmodel}

The threat model primarily focuses on the communication aspects of DFL, presuming the co-existence of trusted participants who abide by network protocols and malicious participants who pose multilayered threats. The threat landscape in the communication channels of a DFL environment is complex, with malicious entities potentially playing passive or active roles. Passive malicious entities might eavesdrop on network communications, surreptitiously gaining access to sensitive information such as model parameters, aggregated gradients, or participants' metadata. In contrast, active malicious entities could actively interfere with network operations, manipulate data, introduce false information, or disrupt communication channels. These threats can originate from internal and external sources, with internal threats emerging from compromised or malicious network participants and external threats from entities outside the DFL topology.

As outlined in Table~\ref{table:1}, a malicious participant can potentially extract a wide range of sensitive information, each with its distinct implications. One example is model parameters, including the weights and biases for each neural network layer, encoding the knowledge the model has acquired. It's worth noting that while Homomorphic Encryption or Differential Privacy methods may prevent or obfuscate this extraction to some extent, the threat remains similar to the one faced in FL vanilla. The illicit acquisition of these parameters could enable a malicious actor to reconstruct the learning model, resulting in substantial data privacy breaches and potentially revealing sensitive insights. Furthermore, the topology offers valuable insights into the overall structure and interaction within the network. It can provide an adversary with knowledge of the structure, facilitating further targeted attacks.

Additionally, the roles assigned to participants within the DFL network could grant an adversary a comprehensive understanding of functional distribution and control mechanisms. Unlike in FL vanilla, where all clients primarily hold the same role, this aspect of DFL architecture can aid an attacker in identifying which nodes to target for maximum disruption. Moreover, performance metrics and resource usage data could expose system vulnerabilities regarding performance and resource allocation strategies. An attacker might infer these metrics from the patterns and volume of network communications \cite{MartinezBeltran:fedstellar:2023}. Information about participant activity periods and the underlying model architecture could prove invaluable for an attacker. By analyzing communication timings and frequencies, an attacker might discern when specific nodes are most active, providing insights into the operational rhythms of the network. Knowledge of the model architecture, obtained through careful observation of network interactions and data exchanges, can expose the fundamental structure and operational logic of the model, thereby revealing potential weaknesses ripe for exploitation. Finally, understanding communication patterns could prove beneficial for a malicious entity. By scrutinizing the frequency and nature of participant interactions, an attacker could detect valuable patterns, forecast behaviors, and potentially impersonate trusted nodes, thereby gaining unauthorized access or sowing discord within the network.

\begin{table*}[htb!]
\centering
\small
\caption{Information accessible to a malicious participant in DFL}
\begin{tabular}{|p{3.7cm}|p{11.6cm}|}
\hline
\textbf{Information} & \textbf{Description} \\ [0.5ex] 
\hline\hline
Model Parameters & Each layer $l_i$ in a model $M$ with $n$ layers has weight $w_i \in \mathbb{R}^{d_i \times d_{i-1}}$ and bias $b_i \in \mathbb{R}^{d_i}$, where $d_i$ is the number of neurons in layer $i$. The parameters of $M$ are the collection $\{w_i, b_i\}_{i=1}^{n}$. \\ 
\hline
Topology & The graph of the network $G(V,E)$, where $V$ is the set of vertices (participants) and $E$ is the set of edges (connections). If $V = \{v_1, v_2, ..., v_n\}$ and $E = \{(v_i, v_j) | v_i, v_j \in V, i \neq j\}$, the topology is fully connected.\\
\hline
Roles & Each participant $p_i \in V$ has a role $r_i \in \{$idle, trainer, aggregator, proxy$\}$. This can be mathematically represented by a function $R: V \rightarrow \{$idle, trainer, aggregator, proxy$\}$, where $R(p_i) = r_i$. \\
\hline
Metrics & Performance of the model (e.g., accuracy, precision, recall, F1 score) and resource usage (CPU, RAM, network) of the nodes. For resources, let $R$ be the resource, $U_R$ the usage, and $C_R$ the capacity. The usage rate is $R_{rate} = \frac{U_R}{C_R}$.\\
\hline
Activity Periods & If $T = \{t_1, t_2, ..., t_n\}$ represent the set of all time intervals and $A = \{a_1, a_2, ..., a_k\} \subseteq T$ the active intervals, then the activity ratio is $A_{ratio} = \frac{\sum_{i=1}^{k} a_i}{\sum_{i=1}^{n} t_i}$.\\
\hline
Model Architecture & A feedforward neural network with $n$ layers can be represented as a sequence of function compositions $f(x) = f_n(f_{n-1}(...f_2(f_1(x))))$, where $f_i(x) = \sigma(w_i \cdot x + b_i)$ is the operation for layer $i$, and $\sigma$ is the activation function.\\
\hline
Communication Patterns & If $M = \{m_{ij}\}$ is the set of all messages sent from participant $i$ to participant $j$, the frequency of communication between these participants can be quantified as $F_{ij} = \frac{|m_{ij}|}{\sum_{i,j}|m_{ij}|}$, where $|m_{ij}|$ is the number of messages exchanged. \\
\hline
\end{tabular}
\label{table:1}
\end{table*}

Numerous potential security threats can compromise the confidentiality, integrity, and availability of federated data and models. These threats primarily arise from the inherent vulnerabilities presented by the decentralization of learning processes and model sharing without the control of a central authority. The following communications threats have been identified (see \tablename~\ref{table:2}):

\newcommand{\noimportant}{{\color{ao}$\mathord{!}$}}
\newcommand{\important}{{\textbf{\color{amber}$\mathord{!!}$}}}
\newcommand{\critical}{{\color{red}$\mathord{!!!}$}}

\begin{table*}[htb!]
\caption{Attacks, goals, and information at risk in DFL}
\label{table:2}
\centering
\small
\begin{threeparttable}
\begin{tabular}{|p{2.8cm}|p{8cm}|p{4cm}|}
\hline
\textbf{Attack} & \textbf{Goal} & \textbf{Information at Risk} \\ 
\hline\hline
Eavesdropping & Extract sensitive information to undermine integrity and security of the federated participants [\important] & $\sbullet[0.75]$ Model Parameters \newline $\sbullet[0.75]$ Topology \newline $\sbullet[0.75]$ Roles \\
\hline
MitM & Manipulate information or insert malicious data to disrupt federation operations [\critical] & $\sbullet[0.75]$ Communication Patterns \newline $\sbullet[0.75]$ Roles \\
\hline
Network Mapping & Know the network structure to launch more targeted future attacks on the federation [\noimportant] & $\sbullet[0.75]$ Topology \newline $\sbullet[0.75]$ Model Architecture \\
\hline
Eclipse Attacks & Isolate a node or group of nodes to extract information or disrupt DFL communications [\critical] & $\sbullet[0.75]$ Activity Periods \newline $\sbullet[0.75]$ Topology \newline $\sbullet[0.75]$ Roles \newline $\sbullet[0.75]$ Communication Patterns \\
\hline
\end{tabular}
\begin{tablenotes}
\item \noimportant\space Low importance,\space\important\space High importance,\space\critical\space Critical
\end{tablenotes}
\end{threeparttable}
\end{table*}


\begin{itemize}
    \item \textit{TH1. Eavesdropping}. In a DFL setting, an adversary could covertly monitor network communications or infiltrate a participant node to gain unauthorized access to sensitive data. This data could include model parameters, network topology, and participant roles. The adversary could then leverage this information to disrupt the federated process or impersonate a legitimate participant. This threat often persists undetected due to its covert nature, leading to prolonged periods of sensitive data leakage.

    \item \textit{TH2. MitM}. It involves an attacker intercepting and potentially manipulating the communication between two participant nodes. This enables the attacker to alter exchanged model parameters, introduce spurious data, or eavesdrop on the exchanged information, posing significant challenges to the integrity of the federated process.

    \item \textit{TH3. Network Mapping}. It aims to understand the structure of the federated network and the roles of participant nodes. By gaining this knowledge, attackers can predict and interfere with network operations, facilitating more targeted and potentially detrimental exploits.

    \item \textit{TH4. Eclipse}. This attack in DFL seeks to isolate a specific node or a group of nodes from the rest of the network. This isolation distorts the affected nodes' perception of the network state, causing them to act based on inaccurate information and potentially paving the way for additional security breaches.

\end{itemize}

In light of the identified threats, a comprehensive security module for DFL must account for these potential attack vectors and implement countermeasures to ensure robust operation and resilience against attacks. Crucially, achieving this goal involves striking a careful balance between enhancing security and managing the additional computational and network overhead that security measures may introduce.
\section{Security Module}\label{sec:module}

This section details the proposed security module, particularly examining its integration within a novel DFL platform and how it fortifies the network against a broad spectrum of cyber threats.

\subsection{Overview}

The security module comprises a set of cybersecurity strategies designed to safeguard the complex exchange of data and models in DFL. The distinctive features of DFL, such as decentralized aggregation, asynchronous communication, limited visibility to near neighbors, and participant independence, necessitate nuanced and versatile security measures. The limited visibility of DFL nodes, usually only to immediate neighbors, restricts the broader network anomaly detection. Participant independence complicates maintaining a secure environment as nodes decide when to commence model training or aggregation. This proposal responds to the growing need for advanced security mechanisms within the field of DFL, considering the diversity and sensitivity of data involved in these systems. This module employs sophisticated encryption methods and MTD techniques, making it highly adaptable to various DFL platforms:

\begin{itemize}
    \item \textit{Encryption}. Using a combination of symmetric and asymmetric encryption, the module ensures secure model exchanges and efficient key management. This strategy guarantees data confidentiality and provides robust protection against potential breaches.
    \item \textit{MTD Techniques}. These techniques, which include Neighbor Selection and IP/port switching, create a dynamic and unpredictable defensive layer within the system. By continuously changing communication pathways and nodes, these techniques make it increasingly difficult for potential attackers to gain a foothold in the system.
\end{itemize}

\subsection{Security Components}

The components of the security module comprise encryption techniques and MTD strategies. The encryption techniques, designed to ensure data confidentiality during the model exchange, combine the efficiency of symmetric encryption for data protection with the secure key management of asymmetric encryption. MTD techniques, such as Neighbor Selection and IP/port switching strategies, add a dynamic and shifting defensive layer to the system. These techniques introduce unpredictability and fluidity by continuously altering network communication pathways, making the system difficult for potential attackers to decipher due to the increased complexity and resource requirements for successful attacks. The integration of these components in a federated participant cycle within a DFL environment is depicted in Algorithm~\ref{alg:cycle}. This algorithm combines the elements of encryption and MTD, effectively creating a robust security layer within the DFL infrastructure.

\begin{algorithm}[ht!]
\caption{Federated participant cycle with Encryption and MTD Techniques in DFL}
\label{alg:cycle}
\footnotesize
\begin{algorithmic}[1]
\Require{$R$: local round, $\alpha$: learning rate, $\lambda$: regularization parameter, $S_j$: socket to neighbor j, $D$: local dataset, $E_{\text{sym}}$ / $E_{\text{asym}}$: symmetric/asymmetric encryption function, $D_{\text{sym}}$ / $D_{\text{asym}}$: symmetric/asymmetric decryption function, $MTD_{\text{IP}}$: IP/port MTD function, $MTD_{\text{N}}$: neighbor selection MTD function}

\Procedure{$MTD_{\text{N}}$}{$N_{\text{all}}, n$}
\Comment{\textbf{Neighbor Selection}}
    \State Initialize an empty list $N$
    \While{$|N| < n$}
        \State Select a neighbor $i$ from $N_{\text{all}}$ uniformly at random
        \If{$i \notin N$} 
            \State Add $i$ to $N$
        \EndIf
    \EndWhile
    \State \Return{$N$}
\EndProcedure

\Procedure{$MTD_{\text{IP}}$}{$config$}
\Comment{\textbf{IP/Port Switch}}
    \State Fetch a list of available IP and ports: $IP_{\text{avail}}, P_{\text{avail}}$
    \State Select a new IP and port from $IP_{\text{avail}}, P_{\text{avail}}$ uniformly at random
    \State Update $config$ with the new IP address and port
    \State \Return{$config$}
\EndProcedure

\State $D_{\text{Train}}, D_{\text{Test}} \gets split(D)$

\For{$r$ in $R$}
    
    \State $\theta \leftarrow Initialize()$ \Comment{\textbf{Initialize Local Model}}
    \For{each $(x, y)$ in $D_{\text{Train}}$}
        \State $\theta \leftarrow \theta - \alpha (\nabla_\theta J(\theta, x, y) + \lambda \theta)$ \Comment{\textbf{Train}}
    \EndFor
    
    \State $N \leftarrow MTD_{\text{N}}(N_{\text{all}})$
    
    \For{$j$ in $N$} \Comment{\textbf{Send}}
        \State $\theta_{\text{enc}} \leftarrow E_{\text{sym}}(\theta, K_{\text{sym}})$
        \State $K_{\text{sym\_enc}} \leftarrow E_{\text{asym}}(K_{\text{sym}}, K_{j_{\text{pub}}})$
        \State $\text{Send } \theta_{\text{enc}}, K_{\text{sym\_enc}} \text{ to } j \text{ via } S_j$ 
    \EndFor 
    
    \While{$not \; Timeout$}
        \For{$j$ in $N$} \Comment{\textbf{Receive}}
            \State $RP_{j_{\text{enc}}}, K_{j_{\text{sym\_enc}}} \leftarrow \text{Receive from } j \text{ via } S_j$ 
            \State $K_{j_{\text{sym}}} \leftarrow D_{\text{asym}}(K_{j_{\text{sym\_enc}}}, K_{\text{priv}})$
            \State $RP_j \leftarrow D_{\text{sym}}(RP_{j_{\text{enc}}}, K_{j_{\text{sym}}})$
        \EndFor
    \EndWhile
    
    \State $\theta \leftarrow \frac{1}{|N|+1} (\theta + \sum_{j \in N} RP_j)$ \Comment{\textbf{Aggregate} (FedAvg)}
    \State $\text{Update Local Model with } \theta$

\EndFor

\For{each $(x, y)$ in $D_{\text{\textbf{Test}}}$}
    \State $y_{pred} \leftarrow Predict(\theta, x)$ \Comment{\textbf{Test}}
    \State $L \leftarrow \frac{1}{|D_{\text{Test}}|}\sum_{i=1}^{|D_{\text{Test}}|} l(y_i, y_{pred_i})$ \Comment{\textbf{Compute Loss}}
\EndFor

\State $\text{Send metrics to controller}$ \Comment{\textbf{Report Metrics}}

\State $MTD_{\text{IP}}(config) \rightarrow config$

\end{algorithmic}
\end{algorithm}

\subsubsection{Communications Encryption}

The integrity and confidentiality of the information exchanged among participants during the federation is a fundamental requirement in secure DFL systems. This security is achieved by combining symmetric and asymmetric encryption techniques, forming a comprehensive, multi-layered security infrastructure.

The first layer of this security architecture employs symmetric encryption. This method is computationally efficient and uses a single key for data encryption and decryption. The Advanced Encryption Standard (AES) algorithm, provided by the \textit{pycryptodome} library, is utilized for this layer. Known for its robust security and broad acceptance, the AES algorithm is an ideal choice, especially considering the resource constraints often present in many devices.

The second layer of the security architecture employs asymmetric encryption. This technique provides an additional layer of security by using a pair of keys: a public key for encryption and a private key for decryption. The RSA algorithm, also provided by the \textit{pycryptodome} library, is used for this layer. RSA eliminates risks associated with key sharing in symmetric encryption and ensures a secure channel for key exchange, protecting the symmetric keys used in the AES algorithm. Key renewal, or the periodic updating of encryption keys, is another integral security feature of the module. The risk of a key compromise is significantly reduced by continuously renewing the keys during the federated process.

\subsubsection{MTD Techniques}

The MTD techniques disrupt the attack surface of the system, increasing the difficulty for an attacker to exploit vulnerabilities and gain unauthorized access. Specifically, the proposed security module incorporates two MTD techniques: Neighbor Selection and IP/port switching.

The Neighbor Selection MTD technique minimizes network topology exposure to potential attackers. This technique can protect the nodes from targeted attacks by dynamically altering their communication partners in each learning cycle. By continually shifting the communication patterns in the network, the likelihood of an attacker successfully predicting or manipulating these patterns is significantly reduced. The random selection of neighbors is implemented using Python's built-in random library, ensuring unbiased and unpredictable selections for each cycle. The process for the Neighbor Selection MTD is fairly straightforward. In each federated round, a node randomly selects a subset of neighbors from all available participants (see Algorithm~\ref{alg:cycle}). This selection scheme is implemented using the socket library of Python, which provides low-level networking capabilities suitable for various network protocols, including TCP/IP, common in wired and wireless communications. The socket-based communication scheme offers reliability and flexibility, which are vital in a dynamic DFL environment.

The second technique is IP/port switching MTD, adding another layer of security. This method involves routinely changing the IP addresses and ports used by the federated nodes, further complicating the predictability of the attack surface. An attacker finds it difficult to sustain a prolonged attack on a specific node. In the proposed security module, IP/port switching is implemented by regularly selecting a new IP address and port from a pool of available ones. This selection is automated and randomized using the built-in capabilities of Python for network configuration. By dynamically altering the IP addresses and ports, the technique disrupts potential attackers' ability to predict the communication structure or execute targeted attacks.

Building on the elaboration of the implemented security techniques, it is essential to understand their effectiveness, as depicted in \tablename~\ref{table:attacks-mitigations}. Encryption protects against eavesdropping, MitM, and eclipse attacks by protecting data during transmission. As a complement, MTD offers robust defenses against attacks such as Network Mapping or eclipse attacks.

\begin{table}[htb!]
\caption{Potential mitigations for attacks in DFL}
\label{table:attacks-mitigations}
\centering
\footnotesize
\begin{tabular}{|c|p{1.7cm}|p{0.6cm}|c|p{0.8cm}|}
\hline
\multicolumn{1}{|c|}{\makecell{\textbf{Security}\\\textbf{Components}}} & \multicolumn{4}{c|}{\textbf{Attacks}} \\ 
\hline\hline
& Eavesdropping & MitM & \makecell{Network\\Mapping} & Eclipse \\ 
\hline
Encryption & \makecell{\yestick} & \makecell{\yestick} & \makecell{\notick} & \makecell{\yestick} \\
\hline
MTD & \makecell{\notick} & \makecell{\yestick} & \makecell{\yestick} & \makecell{\yestick} \\
\hline
\end{tabular}
\end{table}


\subsection{Fedstellar Platform}

Fedstellar is an innovative platform that facilitates the training of FL models across a wide array of physical and virtual devices. The platform is a hub for developing, deploying, and managing federated applications and provides a standardized approach for executing these processes. The architecture of Fedstellar is composed of three fundamental elements:

\begin{itemize}
    \item \textit{Frontend}. A user-centric interface that offers easy experiment setup and real-time monitoring, thus ensuring an intuitive user experience.
    \item \textit{Controller}. A central command unit orchestrates operations across the platform, ensuring seamless inter-module communication and efficient task execution.
    \item \textit{Core}. This critical component, deployed on each participating device, is responsible for vital functions such as model training and communication.
\end{itemize}

These components establish a robust and resilient architecture that provides sophisticated tools and metrics for federation management. This enables high transparency and efficiency in monitoring the learning process. Moreover, the platform contains extensible modules offering data storage, asynchronous capabilities, and effective model training and communication mechanisms.

The security module is integrated into the Fedstellar platform to demonstrate the proposed effectiveness and compatibility of the module. As depicted in Figure~\ref{fig:module}, the \textit{security module} is a pivotal functionality of the core component responsible for managing secure communications across the platform. Its integration into the core ensures robust protection for the vast and complex communication exchanges characteristic of DFL. To support the overall security structure, enhancements have also been made to the frontend and the controller components of the Fedstellar platform. The front end now encompasses the \textit{security definition} feature, enabling users to set and manage their security parameters conveniently. Also, the controller implements \textit{security measures}, a provision that efficiently manages and enforces the established security settings in real time.

% Figure environment removed

The integration of the security module maintains compatibility through its design, which leverages threaded processing for non-blocking operations and event passing between modules for effective communication. These provisions ensure that the addition of the module does not disrupt the existing functionalities of the platform but rather harmonizes with them, augmenting the capability of Fedstellar to efficiently manage diverse federations comprising various devices, network topologies, and algorithms.
\section{Validation Scenario}\label{sec:scenario}

The validation scenario of this study emulates an authentic edge computing setting, which evaluates the performance of the proposed security module in a DFL environment. As summarized in \tablename~\ref{table:3}, the chosen configuration encompasses a federation of eight physical devices: five Raspberry Pi 4 units and three Rock64 units. These devices are interconnected via a random network topology within the private local network. The Raspberry Pi 4 units, armed with a 1.5GHz quad-core 64-bit ARM Cortex-A72 CPU and 2GB of RAM, present a delicate balance between size, cost-effectiveness, and computational prowess, thereby rendering them a suitable choice for simulating edge nodes. The remaining three devices, Rock64 boards, enhance the system's heterogeneity by contributing slightly lower processing capabilities, characterized by a 64-bit ARM Cortex-A53 with a 1.5 GHz clock speed and up to 2GB RAM. The deployed federation operates under the Fedstellar platform, and each participant utilizes the LeNet5 federated model trained on the MNIST dataset. The MNIST dataset was chosen for its relevance in many FL and pattern recognition research areas, making it a fitting choice for this validation scenario.

\begin{table}[htb!]
\caption{Validation scenario using physical devices and eclipse attack}
\label{table:3}
\centering
\footnotesize
\begin{tabularx}{\columnwidth}{|X|X|}
\hline
\textbf{Characteristic} & \textbf{Description} \\ 
\hline\hline
DFL Platform & Fedstellar \cite{MartinezBeltran:fedstellar:2023} \\
\hline
Federation Architecture & DFL \\
\hline
Participants & 5 Raspberry Pi 4 \newline 3 Rock64 \\
\hline
Network Topology & Random \\
\hline
Federated Model & LeNet5 \\
\hline
Dataset & MNIST \cite{Deng:MNIST:2012} \\
\hline
Security Configuration & \circled{1} Baseline \newline \circled{2} Encryption \newline \circled{3} Encryption and MTD \\
\hline
Attack &
Eclipse attack: \newline $\sbullet[0.75]$ One external attacker \newline $\sbullet[0.75]$ One target participant \\
\hline
\end{tabularx}
\end{table}

The security of the federation is assessed under three different configurations, providing an expansive view of its security posture under varied conditions. Initially, the federation functions with \circled{1} a baseline with no security measures for subsequent security comparisons. Following this, the federation incorporates \circled{2} encryption techniques, forming its primary line of defense. Finally, the system operates with \circled{3} both encryption and MTD techniques, following the design of the proposed security module. To assess the resiliency of the system against cybersecurity threats, the validation scenario simulates an eclipse attack, a significant threat in decentralized networks \cite{Alangot:eclipse_attack_defense:2021, Niu:eclipse_attack:2022}. The choice of this attack is motivated by the number of security measures it requires, as shown in \tablename~\ref{table:attacks-mitigations}. The successful mitigation of this multifaceted attack in the validation scenario implies a high probability of successful defense against other potential attacks, as enumerated in \tablename~\ref{table:2}. \figurename~\ref{fig:eclipse_attack} shows the steps of the eclipse attack deployed: (i) involves isolating a chosen node, (ii) seizing control over its communications, and (iii) extracting valuable information. For the simulation, two nodes are programmed to conduct the attack to quantify potential data theft risks and overall network vulnerability.

% Figure environment removed
\section{Experimental Results}
\label{sec:results} 
%\todo{I removed the word environment after each of them to save on space and hbox badness}
We conduct simulated experiments in three environments---our \emph{J-Intersection} (Sec.~\ref{sec:results:jint}), \emph{Parallel Hallway} (Sec.~\ref{sec:results:hallways}), and \emph{University Building} (Sec.~\ref{sec:results:office})---in which a robot must navigate to a point goal in unseen space.
For each trial, we evaluate performance of 4 planners:
\begin{LaTeXdescription}
\item[Non-Learned Baseline] Optimistically assumes the unseen space to be free and plans via grid-based A$^{\!*}$ search.
\item[LSP-Local (learned baseline)] Plans via Eq.~\eqref{eq:lsp-planning}, estimating subgoal properties via only local features, as in \cite{pmlr-v87-stein18a}.
\item[LSP-GNN (ours)] Plans via Eq.~\eqref{eq:lsp-planning}, yet uses our graph neural network learning backend to estimate subgoal properties using both local and non-local features.
\item[Fully-Known Planner] The robot uses the fully-known map to navigate; a lower bound on cost.
\end{LaTeXdescription}
For each planner, we compute average navigation cost across many (at least 100) random maps from each environment.

\begin{table}[t]
    \begin{center}
    \caption{Avg. Cost over 100 Trials in the J-Intersection Environment}
    \label{table:toy-stats}
        \begin{tabular}{cc}
        \toprule
            \textbf{Planner} & \textbf{Avg. Cost (grid cell units)} \\
            \hline     
            Non-Learned Baseline & $303.03$\\        
            LSP-Local (learned baseline) & $323.46$\\      
            LSP-GNN (ours) & \textbf{204.85}\\
            \hline
            Fully-Known Planner & $204.85$\\   
        \bottomrule
        \end{tabular}
    \end{center}
\end{table}

\subsection{J-Intersection Environment}\label{sec:results:jint}
% Figure environment removed

% To demonstrate the utility of our approach in a simple
We first show results in the J-Intersection environment, described in Sec.~\ref{sec:example-case} to motivate the importance of non-local information for good performance for navigation under uncertainty.
In this environment, the robot must choose where to travel at a fork in the road, yet non-locally observable information is needed to reliably make the correct choice---a blue-colored starting region indicates that the goal can be reached by turning towards the blue hallway at the intersection, and the same for the red-colored regions. We randomly mirror the environment so that the robot cannot learn a systematic policy that quickly reaches the goal without understanding.

We conduct 100 trials for each planner in this environment to evaluate their performance and show the average cost planning strategy in Table~\ref{table:toy-stats}.
Across all trials, our proposed LSP-GNN planner \emph{always} correctly decides where to go at the intersection and achieves near-perfect performance.
By contrast, both the LSP-Local and Non-Learned Baseline planners lack the knowledge to determine which is the correct way to go and perform poorly overall, resulting in poor performance in roughly half of the trials.
We highlight two example trials in Fig.~\ref{fig:example-case-results}.
We do not report the prediction accuracy empirically, because the prediction accuracy does not reflect the actual gain in performance for our work.

\subsection{The Parallel Hallway Environment}\label{sec:results:hallways}
% Figure environment removed

Our \emph{Parallel Hallway} environment (Fig.~\ref{fig:sample-hallway}) consists of parallel hallways connected by rooms.
We procedurally generate maps in this environment with three hallways and two room types: (i) \emph{dead-end} rooms and (ii) \emph{passage} rooms that provide connections between neighboring parallel hallways.
Only one passage room exists between a pair of hallways, and so the robot must identify this room if it is to travel to another hallway.
Environments are generated such that the dead-end rooms all have the same color (red or blue) distinct from the color of the passage rooms, which are thus blue or red, respectively.
We are making the environment such that the relational information, such as recognizing that if a room with certain color is explored as a dead-end, then the other colored room serves as a pass-through room can be learned. 
If the colors were entirely random, there would be no way to make predictions about the unseen space.
Both room types contain obstructions and are otherwise identical, so that it is not possible to tell whether or not a room will connect to a parallel hallway without trial-and-error or by utilizing semantic color information from elsewhere in the map.
Rooms are placed far enough apart that the robot cannot determine from the local observations if a room will lead to the next hallway or will be a dead end.
The start and goal locations are placed in separate hallways, so as to force the robot to understand its surroundings to reach the goal quickly.
Thus, to navigate well in this challenging procedurally-generated environment, the robot must first explore, trying nearby rooms to determine which color belongs to which room type, and then retain this information to inform navigation through the rest of the environment.


\begin{table}[t]
    \begin{center}  
    \caption{Avg. Cost over 500 Trials in the Parallel Hallway Environment}\label{table:hallway-stats}
        \begin{tabular}{cc}
            \toprule
            \textbf{Planner} & \textbf{Avg. Cost  (grid cell units)}\\
            \hline
            Non-Learned Baseline & $205.93$\\
            LSP-Local (learned baseline) & $236.47$\\
            LSP-GNN (ours) & \textbf{141.37}\\
            \hline
            Fully-Known Planner & $108.37$\\
            \bottomrule
        \end{tabular}
    \end{center} 
\end{table}


% Figure environment removed

We train the simulated robot on data from 2,000 distinct procedurally generated maps and evaluate in a separate set of 500 distinct procedurally generated maps. We show the average performance of each planning strategy in Table~\ref{table:hallway-stats} and include scatterplots of the relative performance of different planners for each trial in Fig.~\ref{fig:scatter-plot-hallway}.
The robot planning with our LSP-GNN approach is able to utilize non-local local information to improve its predictions about how best to reach the goal, achieving a 31.3\% improvement in average cost versus the optimistic Non-Learned Baseline planner and a 40.2\% improvement over the LSP-Local planner. In addition, our approach is \emph{reliable}: owing to the LSP planning abstraction, our robot is able to successfully reach the goal in all maps.

% Our approach, by retaining non-local information and using it to make predictions, improves average performance over the baselines.
% In addition, our approach is \emph{reliable}: owing to the LSP planning abstraction, our robot is able to successfully reach the goal in all maps.
% Both the non-learned baseline and the LSP-Local planner cannot take advantage of task-relevant non-local information and so perform poorly.

We highlight one trial in Fig.~\ref{fig:result-hall-path}, in which the robot is tasked to navigate from the top hallway to the bottom hallway, which contains the goal.
After a brief period of trial-and-error exploration in the first (top) hallway, the robot discovers the passage to the neighboring hall and uses the knowledge of the semantic color to quickly locate the passage to the next hallway and reach the goal.
% Our robot quickly discovers the passage to the neighboring hall, 
% Our robot explores very little in the first hallway (top) before discovering the passage to the neighboring hall and uses that knowledge to quickly locate and find the next passage room and reach the goal.
By contrast, the Non-Learned Baseline optimistically assumes unseen space to be free and enters every room in the direction of the goal.
The LSP-Local planner makes predictions using only local information and, unable to use important navigation-relevant information, cannot determine how to reach the goal; its poor predictions result in frequent turning-back behavior as it seeks alternate routes to the goal, reducing performance.

% Figure environment removed



\subsection{University Building Floorplans}\label{sec:results:office}
% Figure environment removed
Finally, we evaluate in large-scale maps generated from real-world floorplans of buildings from the Massachusetts Institute of Technology, including buildings of over 100 meters in extent along either side; see Fig.~\ref{fig:mit-train-example-map} for examples.
We generate data from 2,000 trials across 56 training floorplans and evaluate in 250 trials from 9 held-out test floorplans, each augmented by procedurally generated clutter to add furniture-like obstacles to rooms.
% The training and testing maps are sampled from distinct building.
In addition to occupancy information, \emph{rooms} in the map have a distinct semantic class from \emph{hallways} (and other large or accessible spaces); this semantic information is provided as input node features to the neural networks to inform their predictions.

% These floorplans have rooms
% The MIT building floorplans, which were employed to train our robot, encompass not only diverse shaped configurations, but also comprise hallways that are disconnected and accessible only through passthrough rooms.
% Additionally, there are rooms that can be reached solely via other rooms and not through hallways, and sometimes they can be accessed through both hallways and other rooms. 
% These features of the floorplans present distinct challenges to any robot, requiring it to navigate through complex environments with multiple possible paths to reach its destination. 
% The incorporation of these various elements in the training process enables our robot to learn how to operate efficiently and effectively in practical settings with diverse layouts and structures.
% We train our simulated robot in 2000 maps where we randomly sample the maps from a pool of 320 maps.
% Next, we evaluate the performance on 250 maps again randomly sampled from a different pool of 160 maps (that the robot has not seen during training).
% In both cases, we sample the maps with random initialisation of the robot start pose and goal pose.
% So, the agent always has a new trajectory to follow.

We show the average performance of each planning strategy in Table~\ref{table:office-stats} and include scatterplots of the relative performance of different planners for each trial in Fig.~\ref{fig:scatter-plot-office}.
The robot planning with our LSP-GNN approach achieves improvements in average cost of 9.3\% versus the optimistic Non-Learned Baseline planner and of 14.9\% improvement over the LSP-Local Learned Baseline planner. 
Unlike the LSP-Local planner, which does not have enough information to make good predictions about unseen space, our LSP-GNN approach can make use of non-local information to inform its predictions and thus performs well despite the complexity inherent in these large-scale testing environments. 

% The robot planning with our LSP-GNN approach is able to utilize non-local local information to improve its predictions about how best to reach the goal and achieves a 9.3\% improvement in average cost versus the optimistic Non-Learned Baseline planner and a 14.9\% improvement over the LSP-Local planner.

\begin{table}[t]
    \begin{center}  
    \caption{Avg. Cost over 250 Trials in the University Building Floorplans}\label{table:office-stats}
        \begin{tabular}{cc}
            \toprule
            \textbf{Planner} & \textbf{Avg. Cost (meter)}\\
            \hline
            Non-Learned Baseline & $44.98$\\
            LSP-Local (learned baseline) & $47.93$\\
            LSP-GNN (ours) & \textbf{40.80}\\
            \hline
            Fully-Known Planner & $31.77$\\
            \bottomrule
        \end{tabular}
    \end{center} 
\end{table}

% Figure environment removed

% Figure environment removed

% Figure environment removed

Fig.~\ref{fig:result-office-good} shows a typical navigation example in one of our test environments. 
In this scenario, the shortest possible trajectory involves knowing to follow hallways until near to the goal.
Both learned planners generally exhibit hallway-following behavior---often useful in building-like environments such as these---and improve upon the non-learned (optimistic) baseline. However, our LSP-GNN planner, able to make use of non-local information, can more reliably determine which is the more productive route and more quickly reaches the faraway goal.
Fig.~\ref{fig:result-office-bad} shows two additional examples that highlight the improvements of our LSP-GNN planner made possible by non-locally-available information. In Fig.~\ref{fig:result-office-bad}A, we highlight an example in which both learned planners cannot immediately find the correct path, yet LSP-GNN is able to improve its predictions about where is most likely to lead to the unseen goal and recover more quickly than does LSP-Local. Fig.~\ref{fig:result-office-bad}B shows a more extreme example, in which the LSP-Local planner fails to quickly turn back to seek a promising alternate route immediately identified by LSP-GNN.



% In Fig.~\ref{fig:result-office-bad} we illustrate two more maps, comparing LSP-GNN against LSP-Local.
% We show the ability for quick recovery of our learned planner from going off course. %against the non-learned baseline.
% Since the non-learned baseline uses only local information it struggles to make good predictions and keeps on going off course.
% Despite going off course the same way as LSP-Local in the beginning, LSP-GNN uses non-local information to get prediction that enables it to
%Whereas LSP-GNN utilizes the non-local information going off course and uses that to guide itself and 
% course correct quickly resulting in shorter trajectories.

% Fig.~\ref{fig:result-office-good} and~\ref{fig:result-office-bad} illustrate planning outcomes over two separate floorplans.
% LSP-GNN typically employs hallways as the primary means of reaching distant destinations. 
% It will only venture into rooms located far from its goal if it perceives them to be passthrough rooms.
% As it approaches the goal, it will enter rooms in an attempt to locate the objective. 
% In contrast, both the non-learned planner and the LSP-Local planner struggle to accurately recognize potential passthrough rooms and frequently deviate from the hallway. 
% This often results in suboptimal routes and increased difficulty in reaching the desired location.

% The LSP-GNN planner adheres to the assumption that hallways typically provide the most direct route to distant destinations. 
% Although this approach resulted in higher costs than the other planners in this particular case, the LSP-GNN planner acted in a rational manner based on its prior belief. 
% By sticking to the hallway, it employed a logical and systematic approach to navigating the environment, even though it did not lead to the desired outcome in this specific instance.
\section{Conclusion and Future Work}
\label{sec: Conclusion and Future Work}
This paper explores formal method-based reachability analysis of variable-length time series regression neural networks (NNs) using approximate Star methods in the context of predictive maintenance, which is crucial with the rise of Industry 4.0 and the Internet of Things. The analysis considers sensor noise introduced in the data. Evaluation is conducted on two datasets, employing a unified reachability analysis that handles varying features and variable time sequence lengths while analyzing the output with acceptable upper and lower bounds. Robustness and monotonicity properties are verified for the TEDS dataset. Real-world datasets are used, but further research is needed to establish stronger connections between practical industrial problems and performance metrics. The study opens new avenues for exploring perturbation contributions to the output and extending reachability analysis to 3-dimensional time series data like videos. Future work involves verifying global monotonicity properties as well, and including more predictive maintenance and anomaly detection applications as case studies. \newblue{The study focuses solely on offline data analysis and lacks considerations for real-time stream processing and memory constraints, which present fascinating avenues for future research.}
\paragraph{\textbf{Acknowledgements.}}
The material presented in this paper is based upon work supported by the National Science Foundation (NSF) through grant numbers 1910017, 2028001, 2220418, 2220426, and 2220401, and the Defense Advanced Research Projects Agency (DARPA) under contract number FA8750-18-C-0089 and FA8750-23-C-0518, and the Air Force Office of Scientific Research (AFOSR) under contract number FA9550-22-1-0019 and FA9550-23-1-0135. Any opinions, findings, conclusions, or recommendations expressed in this paper are those of the authors and do not necessarily reflect the views of AFOSR, DARPA, or NSF. We also want to thank our colleagues, Tianshu and Barnie for their valuable feedback.
 


\backmatter

\section*{Declarations}

\begin{itemize}
\item Competing interests: This work has been partially supported by \textit{(a)} 21629/FPI/21, Seneca Foundation - Science and Technology Agency of the Region of Murcia (Spain), \textit{(b)} the Swiss Federal Office for Defense Procurement (armasuisse) with the DEFENDIS and CyberForce projects (CYD-C-2020003), and \textit{(c)} the University of Zürich UZH.
\item Availability of data and materials: Data sharing is not applicable to this article as no datasets were generated during the current study.
\end{itemize}


\bibliography{sn-bibliography}% common bib file
%% if required, the content of .bbl file can be included here once bbl is generated
%%\input sn-article.bbl


\end{document}
