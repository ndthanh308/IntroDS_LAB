\section{Related Work}\label{sec:relatedwork}

This section gives the insights required to understand the concepts used in the following sections and reviews the main works in the literature associated with the present one.

\subsection{Privacy and security in DFL}

The promise of DFL as a tool for collaborative learning in heterogeneous and geographically distributed settings continues to drive robust research into its inherent security implications. A comprehensive understanding of its potential threats and appropriate countermeasures enhances cooperative learning practices. Several ground-breaking research efforts have focused on integrating trust within a DFL context. In this regard, Gholami et al. \cite{Gholami:trusted:2022} proposed an approach that integrates trust as a metric within a DFL context. Their method used a comprehensive mathematical framework to quantify and aggregate the trustworthiness of individual agents. In parallel, Mothukuri et al. \cite{Mothukuri:trusted_blockchain:2022} addressed anomaly detection in Internet of Things (IoT) networks by leveraging the distributed nature of FL. They proposed a FL methodology that optimized anomaly detection by aggregating updates from diverse sources. Their approach hinged on using gated recurrent units (GRUs) in federated training rounds to maximize the accuracy of the overall ML model. Complementing these advancements, Li \cite{Li:trust_dfl:2023} took an innovative leap by proposing a Trustiness-based Hierarchical Decentralized FL (TH-DFL) framework. It employs a Security Robust Aggregation (SRA) rule to ensure privacy and robustness even in the face of malicious nodes. The TH-DFL framework strikes an optimal balance between privacy and robustness, especially as the group size fluctuates, and exhibits superior resilience against varying forms of attacks.

Security concerns related to jamming attacks have also been extensively studied, especially in wireless networks implementing DFL. Shi et al. [3] shed light on the susceptibility of DFL to these attacks, proposing crucial countermeasures. Their algorithms identify and target pivotal network links for attack prevention and optimal placement of jammers to disrupt the federation process. Their findings point to the urgency for sophisticated defense mechanisms in DFL architectures. Further contributing to the body of knowledge on security threats in DFL, Chen et al. \cite{Chen:attacks_detection_historical_gradient:2023} proposed a method called Decentralized FL Historical Gradient (DFedHG). DFedHG utilizes historical gradients to differentiate between regular, untrusted, and malicious users in a DFL environment. This unique solution strengthens the defense against potential threats in DFL systems, accentuating the necessity for sturdy security frameworks. 

Securing wireless networks while implementing DFL is a topic of intensive research. Wang \cite{Wang:dfl_secure_wireless:2023} introduced a method to ensure the security and efficiency of FL in Wireless Computing Power Networks (WCPNs). Their research presents a secure and decentralized FL solution based on blockchain for WCPN, which allows nodes to freely participate or leave the WCPN federated training without authorization and security threats. This approach uses a blockchain with a proof-of-accuracy (PoAcc) consensus scheme and an evolutionary game-based incentive scheme to ensure the consistency and security of FL in WCPN. On the other hand, Salama \cite{Salama:dfl_wireless:2023} proposed a method for Decentralized FL over Slotted ALOHA Wireless Mesh Networking. The approach offers an efficient solution for ML model training without a central server, reducing communication costs and increasing convergence speed. This paper demonstrates how network topologies can impact the performance of ML models, and their results indicate significant promise for DFL in Internet of Things (IoT) systems.

\subsection{Security-based DFL solutions}

Innovative approaches toward enhancing data protection and secure communication within DFL environments have also seen considerable development. For instance, the FusionFedBlock solution, proposed by Singh et al. \cite{Singh:bl_5g_dfl:2023}, merges the strengths of blockchain and DFL to ensure privacy in Industry 5.0. A distributed hash table (DHT) guarantees secure decentralized storage at the cloud layer, while blockchain miners facilitate data verification. FL-SEC, introduced by Qu et al. \cite{Qu:privacy_framework_iot:2022}, stands as a breakthrough framework that addresses potential information leakage due to inference attacks, threats of poisoning attacks via falsified data, and high consumption of communication resources. This model uses a custom incentive mechanism and an enhanced sign gradient descent method to protect the privacy of model parameters and significantly reduce communication resource consumption. Contributing further to privacy preservation and trustworthiness in DFL, Wang \cite{Wang:dfl_privacy_trust:2023} proposed PTDFL, an efficient and novel DFL scheme. This scheme integrates a gradient encryption algorithm to protect data privacy, employs concise proof for the correctness of the gradients, and uses a local aggregation strategy to ensure that the aggregated result is trustworthy. The unique feature of PTDFL is its support for data owners joining in and dropping out during the entire DFL task.

In the enterprise domain, Arakapis et al. \cite{Arapakis:p4l_p2p_telefonica:2023} introduced P4L, a private peer-to-peer learning system. As an asynchronous collaborative learning scheme, P4L allows users to participate in the learning process without depending on a centralized infrastructure. It ensures the confidentiality and utility of shared gradients employing strong cryptographic primitives. Also, it maintains resilience to user dropout and fault tolerance, highlighting the practical applicability and effectiveness of decentralized learning solutions in real-world settings. Finally, on the frontier of sixth-generation (6G) networks, Ridhawi et al. \cite{Ridhawi:digitaltwin_6g_framework:2023} proposed a decentralized zero-trust framework for digital twins. By integrating the zero-trust architecture into digital twin-enabled networks with DFL, they ensured the security, privacy, and authenticity of physical and digital devices. Their approach addresses the challenges of cooperation between devices and network components in a 6G environment, demonstrating the pivotal role of DFL in next-generation networks.