\section{Validation Scenario}\label{sec:scenario}

The validation scenario of this study emulates an authentic edge computing setting, which evaluates the performance of the proposed security module in a DFL environment. As summarized in \tablename~\ref{table:3}, the chosen configuration encompasses a federation of eight physical devices: five Raspberry Pi 4 units and three Rock64 units. These devices are interconnected via a random network topology within the private local network. The Raspberry Pi 4 units, armed with a 1.5GHz quad-core 64-bit ARM Cortex-A72 CPU and 2GB of RAM, present a delicate balance between size, cost-effectiveness, and computational prowess, thereby rendering them a suitable choice for simulating edge nodes. The remaining three devices, Rock64 boards, enhance the system's heterogeneity by contributing slightly lower processing capabilities, characterized by a 64-bit ARM Cortex-A53 with a 1.5 GHz clock speed and up to 2GB RAM. The deployed federation operates under the Fedstellar platform, and each participant utilizes the LeNet5 federated model trained on the MNIST dataset. The MNIST dataset was chosen for its relevance in many FL and pattern recognition research areas, making it a fitting choice for this validation scenario.

\begin{table}[htb!]
\caption{Validation scenario using physical devices and eclipse attack}
\label{table:3}
\centering
\footnotesize
\begin{tabularx}{\columnwidth}{|X|X|}
\hline
\textbf{Characteristic} & \textbf{Description} \\ 
\hline\hline
DFL Platform & Fedstellar \cite{MartinezBeltran:fedstellar:2023} \\
\hline
Federation Architecture & DFL \\
\hline
Participants & 5 Raspberry Pi 4 \newline 3 Rock64 \\
\hline
Network Topology & Random \\
\hline
Federated Model & LeNet5 \\
\hline
Dataset & MNIST \cite{Deng:MNIST:2012} \\
\hline
Security Configuration & \circled{1} Baseline \newline \circled{2} Encryption \newline \circled{3} Encryption and MTD \\
\hline
Attack &
Eclipse attack: \newline $\sbullet[0.75]$ One external attacker \newline $\sbullet[0.75]$ One target participant \\
\hline
\end{tabularx}
\end{table}

The security of the federation is assessed under three different configurations, providing an expansive view of its security posture under varied conditions. Initially, the federation functions with \circled{1} a baseline with no security measures for subsequent security comparisons. Following this, the federation incorporates \circled{2} encryption techniques, forming its primary line of defense. Finally, the system operates with \circled{3} both encryption and MTD techniques, following the design of the proposed security module. To assess the resiliency of the system against cybersecurity threats, the validation scenario simulates an eclipse attack, a significant threat in decentralized networks \cite{Alangot:eclipse_attack_defense:2021, Niu:eclipse_attack:2022}. The choice of this attack is motivated by the number of security measures it requires, as shown in \tablename~\ref{table:attacks-mitigations}. The successful mitigation of this multifaceted attack in the validation scenario implies a high probability of successful defense against other potential attacks, as enumerated in \tablename~\ref{table:2}. \figurename~\ref{fig:eclipse_attack} shows the steps of the eclipse attack deployed: (i) involves isolating a chosen node, (ii) seizing control over its communications, and (iii) extracting valuable information. For the simulation, two nodes are programmed to conduct the attack to quantify potential data theft risks and overall network vulnerability.

% Figure environment removed