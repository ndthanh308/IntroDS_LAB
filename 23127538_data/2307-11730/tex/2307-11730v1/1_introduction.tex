\section{Introduction}\label{sec:introduction}

The rise of the Internet of Things (IoT) has significantly reshaped the digital landscape, defining an era marked by unprecedentedly interconnected devices. IoT devices produce vast volumes of data every second, spanning various sectors, from healthcare and manufacturing to transportation and home automation. Traditionally, Machine Learning (ML) techniques have been employed to derive meaningful insights from these large datasets. However, these techniques often involve the centralized aggregation of data, a process that raises serious concerns about data privacy, data sovereignty, and overhead \cite{idc:iot_number:2022}.

A novel ML approach, known as Federated Learning (FL), has emerged in response to these challenges. FL can train models locally on multiple edge devices, each holding local data samples. This eliminates the need to share raw data, thereby preserving data privacy. Furthermore, Decentralized Federated Learning (DFL) represents a paradigm shift within FL \cite{MartinezBeltran:DFL_survey:2022}. DFL enhances decentralization by facilitating model aggregation at multiple nodes, drastically reducing dependence on a single central server and enabling direct, pairwise model sharing among the network nodes. This innovative approach addresses single points of failures, trust dependencies, and server node bottlenecks inherent in traditional FL. DFL also eliminates the need for a central server by broadening the model aggregation to multiple nodes. Additionally, DFL employs asynchronous communications, a departure from traditional FL. This feature enables individual nodes to communicate their updates independently of others, contributing to system resilience and ensuring the continued learning process even if some nodes encounter delays or disconnections \cite{Shi:personalization_dfl:2023}. The application of DFL to wireless networks has been motivated by the resilience offered by its asynchronous communication, which is crucial in environments with intermittent and unpredictable connectivity \cite{Salama:dfl_wireless:2023}. Specifically, such traits make DFL highly applicable for Unmanned Aerial Vehicle (UAV) networks, where constant and reliable communication is often challenged by diverse factors such as terrain and weather conditions, hence enhancing their cooperative missions \cite{Xiao:military_fullydl:2021}.

Despite the substantial benefits of DFL, it also introduces new challenges. This approach poses different types of sensitive information necessary for the federation, such as the network topology, the roles of the participants, and communication patterns that can be exploited. Besides, in DFL environments where all participants are connected, the absence of a central authority to manage potential threats raises significant security and privacy concerns. With each participant sharing equal threat exposure, adversarial and communication-based attacks become significant concerns. Adversarial attacks can misguide the learning process by manipulating training data or leveraging the shared model updates to infer sensitive information about the other participants. At the same time, communication-based threats can disrupt the model aggregation process or lead to security breaches and privacy infringements \cite{Mothukuri:survey_topology_architecture_privacy_decentralized:2021}. Addressing these challenges could benefit from adopting a dynamic approach like Moving Target Defense (MTD) \cite{Etxezarreta:MTD_wireless:2023}. MTD is a security concept that continuously alters attack surfaces to confuse and mislead adversaries, making it difficult for them to launch successful attacks. Although not yet extensively proposed or validated in the context of DFL, the potential integration of MTD with other security techniques could improve the overall security posture of the system against evolving threats. In recognition of these risks, and with a special emphasis on communication-based attacks that leverage the inherent decentralization of DFL, this paper presents the following contributions:

\begin{itemize}

    \item Create a threat model, identifying and understanding the sensitive information vulnerable to threats affecting the communications in DFL, such as eavesdropping, Man in the Middle (MitM), and eclipse attacks.
    
    \item Develop an advanced security module for DFL platforms providing secure data exchanges through encryption and dynamic proactive defense using MTD. This module mitigates the threats identified in the comprehensive threat model, ensuring efficient system operation despite the integrated security measures.

    \item Implement and deploy the security module within a real-world DFL framework, Fedstellar, integrating it into the frontend, controller, and core components of the platform to enhance the overall security posture of the DFL approach. Furthermore, this work establishes a realistic DFL environment using the Fedstellar platform, consisting of eight interconnected heterogeneous physical devices. Three security configurations are assessed in this setup: a baseline with no security, a configuration with encryption, and a configuration integrating both encryption and MTD techniques.
    
    \item Conduct an in-depth experimental evaluation of the proposed security module using a real-world topology with diverse connections and participants, leveraging the widely used MNIST dataset and eclipse attacks. The evaluation shows an average F1 score of 95\%, reaching 97\% without security measures. Secure configurations, specifically those employing encryption and MTD, induce a slight increase in CPU usage (to 63.2\% $\pm$3.5\%) and network traffic (to 230 MB $\pm$15 MB). Meanwhile, there is a minor RAM rise, peaking at 33.9\% ($\pm$1.5\%) under the encryption and MTD configuration.
\end{itemize}

The remainder of this paper is organized as follows: Section~\ref{sec:relatedwork} provides an in-depth overview of the literature on DFL and its associated security challenges. Section~\ref{sec:threatmodel} introduces the proposed threat model, highlighting the unique security issues that DFL environments face. Section~\ref{sec:module} presents a detailed description of the proposed security module, elucidating its key components and their functionality. Section~\ref{sec:scenario} outlines the experimental setup and evaluation methodology, paving the way for a rigorous assessment of the effectiveness of the security module. Section~\ref{sec:results} presents a comprehensive discussion of the results, and Section~\ref{sec:conclusion} concludes the paper with a summary of the key findings and an exploration of potential avenues for future research.