\section{Security Module}\label{sec:module}

This section details the proposed security module, particularly examining its integration within a novel DFL platform and how it fortifies the network against a broad spectrum of cyber threats.

\subsection{Overview}

The security module comprises a set of cybersecurity strategies designed to safeguard the complex exchange of data and models in DFL. The distinctive features of DFL, such as decentralized aggregation, asynchronous communication, limited visibility to near neighbors, and participant independence, necessitate nuanced and versatile security measures. The limited visibility of DFL nodes, usually only to immediate neighbors, restricts the broader network anomaly detection. Participant independence complicates maintaining a secure environment as nodes decide when to commence model training or aggregation. This proposal responds to the growing need for advanced security mechanisms within the field of DFL, considering the diversity and sensitivity of data involved in these systems. This module employs sophisticated encryption methods and MTD techniques, making it highly adaptable to various DFL platforms:

\begin{itemize}
    \item \textit{Encryption}. Using a combination of symmetric and asymmetric encryption, the module ensures secure model exchanges and efficient key management. This strategy guarantees data confidentiality and provides robust protection against potential breaches.
    \item \textit{MTD Techniques}. These techniques, which include Neighbor Selection and IP/port switching, create a dynamic and unpredictable defensive layer within the system. By continuously changing communication pathways and nodes, these techniques make it increasingly difficult for potential attackers to gain a foothold in the system.
\end{itemize}

\subsection{Security Components}

The components of the security module comprise encryption techniques and MTD strategies. The encryption techniques, designed to ensure data confidentiality during the model exchange, combine the efficiency of symmetric encryption for data protection with the secure key management of asymmetric encryption. MTD techniques, such as Neighbor Selection and IP/port switching strategies, add a dynamic and shifting defensive layer to the system. These techniques introduce unpredictability and fluidity by continuously altering network communication pathways, making the system difficult for potential attackers to decipher due to the increased complexity and resource requirements for successful attacks. The integration of these components in a federated participant cycle within a DFL environment is depicted in Algorithm~\ref{alg:cycle}. This algorithm combines the elements of encryption and MTD, effectively creating a robust security layer within the DFL infrastructure.

\begin{algorithm}[ht!]
\caption{Federated participant cycle with Encryption and MTD Techniques in DFL}
\label{alg:cycle}
\footnotesize
\begin{algorithmic}[1]
\Require{$R$: local round, $\alpha$: learning rate, $\lambda$: regularization parameter, $S_j$: socket to neighbor j, $D$: local dataset, $E_{\text{sym}}$ / $E_{\text{asym}}$: symmetric/asymmetric encryption function, $D_{\text{sym}}$ / $D_{\text{asym}}$: symmetric/asymmetric decryption function, $MTD_{\text{IP}}$: IP/port MTD function, $MTD_{\text{N}}$: neighbor selection MTD function}

\Procedure{$MTD_{\text{N}}$}{$N_{\text{all}}, n$}
\Comment{\textbf{Neighbor Selection}}
    \State Initialize an empty list $N$
    \While{$|N| < n$}
        \State Select a neighbor $i$ from $N_{\text{all}}$ uniformly at random
        \If{$i \notin N$} 
            \State Add $i$ to $N$
        \EndIf
    \EndWhile
    \State \Return{$N$}
\EndProcedure

\Procedure{$MTD_{\text{IP}}$}{$config$}
\Comment{\textbf{IP/Port Switch}}
    \State Fetch a list of available IP and ports: $IP_{\text{avail}}, P_{\text{avail}}$
    \State Select a new IP and port from $IP_{\text{avail}}, P_{\text{avail}}$ uniformly at random
    \State Update $config$ with the new IP address and port
    \State \Return{$config$}
\EndProcedure

\State $D_{\text{Train}}, D_{\text{Test}} \gets split(D)$

\For{$r$ in $R$}
    
    \State $\theta \leftarrow Initialize()$ \Comment{\textbf{Initialize Local Model}}
    \For{each $(x, y)$ in $D_{\text{Train}}$}
        \State $\theta \leftarrow \theta - \alpha (\nabla_\theta J(\theta, x, y) + \lambda \theta)$ \Comment{\textbf{Train}}
    \EndFor
    
    \State $N \leftarrow MTD_{\text{N}}(N_{\text{all}})$
    
    \For{$j$ in $N$} \Comment{\textbf{Send}}
        \State $\theta_{\text{enc}} \leftarrow E_{\text{sym}}(\theta, K_{\text{sym}})$
        \State $K_{\text{sym\_enc}} \leftarrow E_{\text{asym}}(K_{\text{sym}}, K_{j_{\text{pub}}})$
        \State $\text{Send } \theta_{\text{enc}}, K_{\text{sym\_enc}} \text{ to } j \text{ via } S_j$ 
    \EndFor 
    
    \While{$not \; Timeout$}
        \For{$j$ in $N$} \Comment{\textbf{Receive}}
            \State $RP_{j_{\text{enc}}}, K_{j_{\text{sym\_enc}}} \leftarrow \text{Receive from } j \text{ via } S_j$ 
            \State $K_{j_{\text{sym}}} \leftarrow D_{\text{asym}}(K_{j_{\text{sym\_enc}}}, K_{\text{priv}})$
            \State $RP_j \leftarrow D_{\text{sym}}(RP_{j_{\text{enc}}}, K_{j_{\text{sym}}})$
        \EndFor
    \EndWhile
    
    \State $\theta \leftarrow \frac{1}{|N|+1} (\theta + \sum_{j \in N} RP_j)$ \Comment{\textbf{Aggregate} (FedAvg)}
    \State $\text{Update Local Model with } \theta$

\EndFor

\For{each $(x, y)$ in $D_{\text{\textbf{Test}}}$}
    \State $y_{pred} \leftarrow Predict(\theta, x)$ \Comment{\textbf{Test}}
    \State $L \leftarrow \frac{1}{|D_{\text{Test}}|}\sum_{i=1}^{|D_{\text{Test}}|} l(y_i, y_{pred_i})$ \Comment{\textbf{Compute Loss}}
\EndFor

\State $\text{Send metrics to controller}$ \Comment{\textbf{Report Metrics}}

\State $MTD_{\text{IP}}(config) \rightarrow config$

\end{algorithmic}
\end{algorithm}

\subsubsection{Communications Encryption}

The integrity and confidentiality of the information exchanged among participants during the federation is a fundamental requirement in secure DFL systems. This security is achieved by combining symmetric and asymmetric encryption techniques, forming a comprehensive, multi-layered security infrastructure.

The first layer of this security architecture employs symmetric encryption. This method is computationally efficient and uses a single key for data encryption and decryption. The Advanced Encryption Standard (AES) algorithm, provided by the \textit{pycryptodome} library, is utilized for this layer. Known for its robust security and broad acceptance, the AES algorithm is an ideal choice, especially considering the resource constraints often present in many devices.

The second layer of the security architecture employs asymmetric encryption. This technique provides an additional layer of security by using a pair of keys: a public key for encryption and a private key for decryption. The RSA algorithm, also provided by the \textit{pycryptodome} library, is used for this layer. RSA eliminates risks associated with key sharing in symmetric encryption and ensures a secure channel for key exchange, protecting the symmetric keys used in the AES algorithm. Key renewal, or the periodic updating of encryption keys, is another integral security feature of the module. The risk of a key compromise is significantly reduced by continuously renewing the keys during the federated process.

\subsubsection{MTD Techniques}

The MTD techniques disrupt the attack surface of the system, increasing the difficulty for an attacker to exploit vulnerabilities and gain unauthorized access. Specifically, the proposed security module incorporates two MTD techniques: Neighbor Selection and IP/port switching.

The Neighbor Selection MTD technique minimizes network topology exposure to potential attackers. This technique can protect the nodes from targeted attacks by dynamically altering their communication partners in each learning cycle. By continually shifting the communication patterns in the network, the likelihood of an attacker successfully predicting or manipulating these patterns is significantly reduced. The random selection of neighbors is implemented using Python's built-in random library, ensuring unbiased and unpredictable selections for each cycle. The process for the Neighbor Selection MTD is fairly straightforward. In each federated round, a node randomly selects a subset of neighbors from all available participants (see Algorithm~\ref{alg:cycle}). This selection scheme is implemented using the socket library of Python, which provides low-level networking capabilities suitable for various network protocols, including TCP/IP, common in wired and wireless communications. The socket-based communication scheme offers reliability and flexibility, which are vital in a dynamic DFL environment.

The second technique is IP/port switching MTD, adding another layer of security. This method involves routinely changing the IP addresses and ports used by the federated nodes, further complicating the predictability of the attack surface. An attacker finds it difficult to sustain a prolonged attack on a specific node. In the proposed security module, IP/port switching is implemented by regularly selecting a new IP address and port from a pool of available ones. This selection is automated and randomized using the built-in capabilities of Python for network configuration. By dynamically altering the IP addresses and ports, the technique disrupts potential attackers' ability to predict the communication structure or execute targeted attacks.

Building on the elaboration of the implemented security techniques, it is essential to understand their effectiveness, as depicted in \tablename~\ref{table:attacks-mitigations}. Encryption protects against eavesdropping, MitM, and eclipse attacks by protecting data during transmission. As a complement, MTD offers robust defenses against attacks such as Network Mapping or eclipse attacks.

\begin{table}[htb!]
\caption{Potential mitigations for attacks in DFL}
\label{table:attacks-mitigations}
\centering
\footnotesize
\begin{tabular}{|c|p{1.7cm}|p{0.6cm}|c|p{0.8cm}|}
\hline
\multicolumn{1}{|c|}{\makecell{\textbf{Security}\\\textbf{Components}}} & \multicolumn{4}{c|}{\textbf{Attacks}} \\ 
\hline\hline
& Eavesdropping & MitM & \makecell{Network\\Mapping} & Eclipse \\ 
\hline
Encryption & \makecell{\yestick} & \makecell{\yestick} & \makecell{\notick} & \makecell{\yestick} \\
\hline
MTD & \makecell{\notick} & \makecell{\yestick} & \makecell{\yestick} & \makecell{\yestick} \\
\hline
\end{tabular}
\end{table}


\subsection{Fedstellar Platform}

Fedstellar is an innovative platform that facilitates the training of FL models across a wide array of physical and virtual devices. The platform is a hub for developing, deploying, and managing federated applications and provides a standardized approach for executing these processes. The architecture of Fedstellar is composed of three fundamental elements:

\begin{itemize}
    \item \textit{Frontend}. A user-centric interface that offers easy experiment setup and real-time monitoring, thus ensuring an intuitive user experience.
    \item \textit{Controller}. A central command unit orchestrates operations across the platform, ensuring seamless inter-module communication and efficient task execution.
    \item \textit{Core}. This critical component, deployed on each participating device, is responsible for vital functions such as model training and communication.
\end{itemize}

These components establish a robust and resilient architecture that provides sophisticated tools and metrics for federation management. This enables high transparency and efficiency in monitoring the learning process. Moreover, the platform contains extensible modules offering data storage, asynchronous capabilities, and effective model training and communication mechanisms.

The security module is integrated into the Fedstellar platform to demonstrate the proposed effectiveness and compatibility of the module. As depicted in Figure~\ref{fig:module}, the \textit{security module} is a pivotal functionality of the core component responsible for managing secure communications across the platform. Its integration into the core ensures robust protection for the vast and complex communication exchanges characteristic of DFL. To support the overall security structure, enhancements have also been made to the frontend and the controller components of the Fedstellar platform. The front end now encompasses the \textit{security definition} feature, enabling users to set and manage their security parameters conveniently. Also, the controller implements \textit{security measures}, a provision that efficiently manages and enforces the established security settings in real time.

% Figure environment removed

The integration of the security module maintains compatibility through its design, which leverages threaded processing for non-blocking operations and event passing between modules for effective communication. These provisions ensure that the addition of the module does not disrupt the existing functionalities of the platform but rather harmonizes with them, augmenting the capability of Fedstellar to efficiently manage diverse federations comprising various devices, network topologies, and algorithms.