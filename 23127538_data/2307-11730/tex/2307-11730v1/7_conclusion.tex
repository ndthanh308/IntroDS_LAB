\section{Conclusion}\label{sec:conclusion}

This work formulated a threat model for DFL communications, providing a detailed understanding of potential security vulnerabilities and sensitive information that could be exposed during interactions between participating nodes. In response to these challenges, an innovative security module was developed for DFL communications. It incorporates robust defensive mechanisms, including symmetric and asymmetric encryption methods and MTD techniques, tailored to the unique structure and requirements of DFL. This security module was deployed within a real-world DFL framework called Fedstellar to evaluate its efficacy and practicality. The validation scenario involved a random topology of eight physical devices solving an ML task using MNIST and facing eclipse attacks. This scenario allowed to rigorously assess the module under three security configurations: baseline (no security), encryption only, and a composite of encryption and MTD. The assessments validated the performance of the proposed module, demonstrating a satisfactory F1 score of 95\% on average, with an acceptable rise in system overhead. The peak values for CPU usage, network traffic, and RAM utilization were 63.2\% ($\pm$3.5\%), 230 MB ($\pm$15 MB), and 33.9\% ($\pm$1.5\%), respectively, thereby demonstrating the efficiency and practicality in real-world DFL applications.

Future research could consider developing and integrating new security techniques into the current security module to further enhance the resilience of DFL environments. Researchers might assess these enhancements across dynamic network topologies and more participant devices to better understand their efficacy in real-world, large-scale applications. Additionally, simulations with a wider variety of potential attacks would provide valuable insights into the robustness of these defensive methods under diverse threat scenarios. These advancements could significantly contribute to achieving secure, efficient, and scalable deployment of DFL.