\section{Confident Feature Ranking: Step-By-Step}\label{app:full_example}

This section demonstrates how our ranking method constructs simultaneous CIs for the true ranks. We use the bike sharing dataset \citep{fanaee2014event}, the TreeSHAP FI method, and our ranking method with Holm's procedure for $n=50$ base FI values. This demonstration is a detailed description of the example presented in Section \ref{sec:real} (Figure \ref{fig:bike_base_size}).

\paragraph{Base and Global FI Values} We construct a TreeSHAP explainer \citep{lundberg2019Tree} based on the trained XGB model. The base FI values are the absolute values of the local SHAP values the explainer produces for an explanation set of size $n=50$. The global FI values are the average of the base FI values. The values and the order of the global base FI values for this explanation set are presented in Table \ref{tab:full_example}.

\paragraph{Pairwise Differences} The paired-sample t-test is based on the differences between the base FI values of two features $\basematrix_j - \basematrix_k$. The one-sided hypothesis $H_{jk}^0: \trueimp_j \geq \trueimp_k$ is rejected if the difference between the observed global FI values is significantly different from zero. In Figure \ref{fig:pair_diff}, we present the differences between \emph{Workingday} and all other features. The average of the differences between \emph{Workingday} and \emph{Month} and \emph{Temp} is near zero.


\paragraph{Partial Rankings} We set the significance level to $\alpha=0.1$, and for each pair of features, we run two paired one-sided t-tests; then, we adjust the p-values to multiple comparisons using Holm's procedure. In Figure \ref{fig:signs}, gray and black indicate that the observed global FI value of the feature in row $j$ is respectively less and greater than the observed global FI value of the feature in column $k$. White indicates that the difference is zero (neither $H_{jk}^0$  nor $H_{kj}^0$ were rejected). The set of partial rankings $\paranking$ is then obtained. For example, we can conclude that $(Month, Year) \in \paranking$, $(Day, Month) \in \paranking$, and $(Month, Workingday) \not\in \paranking$.


\paragraph{Constructing Simultaneous CIs for the True Ranks}
For each feature, we initialize the lower bound of the CI to one and the upper bound to $p$.
If there are no differences between the features, the CIs for all features are $[1, p]$. Otherwise, there are differences. Without loss of generality, consider the \emph{Workingday} feature. By looking at the row for \emph{Workingday} in Figure \ref{fig:signs}, we can see that the observed global FI value of \emph{Workingday} is significantly higher than the observed global FI values of \emph{Day} and \emph{Weather}, and it is significantly lower than the observed global FI values of \emph{Year} and \emph{Hour}. There is no significant difference between \emph{Workingday} and \emph{Month} and \emph{Temp}. Therefore, we increase the lower bound by two, decrease the upper bound by two, and obtain the confidence set $[3, 5]$ for the true rank of \emph{Workingday}.


We repeat the same process for all features and obtain $90\%$ simultaneous CIs for the true ranks. See lower and upper bounds in Table \ref{tab:full_example} and a visualization of the CIs in Figure \ref{fig:full_example_rank}.

% Figure environment removed


\begin{table}[ht]
\centering
\caption{Ranks and Simultaneous CIs}
\label{tab:full_example}
\begin{tabular}{|l|c|c|c|}
\hline
\textbf{Feature} & \multicolumn{1}{l|}{\textbf{Observed Global FI}} & \multicolumn{1}{l|}{\textbf{Observed Rank}} & \multicolumn{1}{l|}{\textbf{CI}} \\ \hline
Hour             & 129.042                                          & 7                                           & [7, 7]                           \\ \hline
Year             & 44.805                                           & 6                                           & [6, 6]                           \\ \hline
Temp             & 33.777                                           & 5                                           & [4, 5]                           \\ \hline
Workingday       & 28.95                                            & 4                                           & [3, 5]                           \\ \hline
Month            & 23.987                                           & 3                                           & [3, 4]                           \\ \hline
Weather          & 10.865                                           & 2                                           & [2, 2]                           \\ \hline
Day              & 5.673                                            & 1                                           & [1, 1]                           \\ \hline
\end{tabular}
\end{table}


