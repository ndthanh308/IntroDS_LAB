\section{Related Work}\label{sec:related}

\subsection{Uncertainty in Feature Ranking}
Ordering features in terms of their importance to the model's prediction is referred to as feature ranking. It is often infeasible to determine the ``right" order or perfect subset of features, because it requires the examination of all possible feature subsets \citep{prati2012combining}. Many studies suggested overcoming this limitation and obtaining a more stable ranking by applying a two-step procedure in which multiple rankings are generated and the outputs are combined into a single ranking \citep{saeys2008robust}. Each of the rankings generated in the first step is obtained by ordering the global FI values, which are produced using different FI methods \citep{schulz2021uncertainty} or by resampling the data and ranking the output FI values using a single FI method \citep{vettoretti2021variable, alaiz2020information, salman2022stability}. The process of combining all of the rankings into a single ranking is sometimes based on voting \citep{vettoretti2021variable, schulz2021uncertainty}, pairwise comparisons of the rankings \citep{prati2012combining, salman2022stability}, or other techniques \citep{alaiz2020information}. In all of the techniques mentioned above, more stable ranking is achieved by aggregating multiple global scores or rankings, a process that is computationally expensive and requires many explanation sets or FI methods. In contrast, our ranking method produces a stable ranking based on a single FI method and explanation set.


\subsection{Ranking and Selection}
The problem of ranking and selection (R \& S) of items has been well studied by researchers in the field of statistics, and various solutions have been suggested \citep{gupta1965some, boesel2003using}. Some studies focused on ranking items based on noisy data and pairwise comparisons \citep{wright2014ranking, valdeira2022ranking}. Other studies proposed methods that look for the best item \citep{eckman2020revisiting}, select the set of top- or lowest-ranked items, or, most similar to our work, methods that rank all items based on the observed means \citep{zhang2014confidence, klein2020joint, wright2014ranking}. After ranking all of the items, a subset of items might be selected.  \citep{al2022simultaneous, rising2021uncertainty}. Other researchers have proposed methods like ours that deal with the effect of multiple tests and examine how to control the FWER and increase the probability that the correct items are selected \citep{garcia2008extension, holm2013confidence}.


















































