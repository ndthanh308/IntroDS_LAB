\documentclass{article}

\usepackage{parskip}
\usepackage{enumitem}
\usepackage[numbers]{natbib}
\usepackage[english]{babel}
\usepackage{graphicx}
\usepackage{caption}
\usepackage{subcaption}
\usepackage{amsmath, amssymb, mathtools, amsfonts}
\usepackage{multirow}

\usepackage{float}
\usepackage{algorithm}
\usepackage{algorithmic}

\usepackage{xcolor} % highlight
\usepackage{soul} % highlight

\usepackage[colorinlistoftodos]{todonotes}

% set page geometry
\usepackage[verbose=true,letterpaper]{geometry}
\AtBeginDocument{
  \newgeometry{
    textheight=9in,
    textwidth=5.5in,
    top=1in,
    headheight=12pt,
    headsep=25pt,
    footskip=30pt
  }
}


\bibliographystyle{unsrt}
\usepackage[]{hyperref}         % ...clickable refs within pdf...
\hypersetup{                    % ...like so
  colorlinks,
  linkcolor={blue!80!black},
  citecolor={blue!70!black},
  urlcolor={blue!70!black}
}

\newcommand{\real}{\mathbb{R}}
\newcommand{\imp}{\hat{\Phi}}
\newcommand{\localimp}{\hat{\phi}}
\newcommand{\trueimp}{\Phi}
\newcommand{\basevalue}{v}
\newcommand{\basematrix}{\mathbf{v}}
\newcommand{\features}{[p]}
\newcommand{\sampleset}{\mathcal{D}}
\newcommand{\truetopk}{\mathcal{T}_k}
\newcommand{\topk}{\widehat{\mathcal{T}}_k}
\newcommand{\paranking}{D}

\newtheorem{theorem}{Theorem}
\newtheorem{proof}{Proof}
\newtheorem{definition}{Definition}
\newtheorem{assumption}{Assumption}
\newtheorem{lemma}{Lemma}

\title{Confident Feature Ranking}

\author{
  Bitya Neuhof\\
  Department of Statistics and Data Science\\
  The Hebrew University of Jerusalem\\
  \texttt{bitya.neuhof@mail.huji.ac.il}\\
  \and
  Yuval Benjamini\\
  Department of Statistics and Data Science\\
  The Hebrew University of Jerusalem\\
  \texttt{yuval.benjamini@mail.huji.ac.il}
}

\begin{document}

\maketitle

\begin{abstract}
\normalsize
    Machine learning models are widely applied in various fields. Stakeholders often use post-hoc feature importance methods to better understand the input features' contribution to the models' predictions. The interpretation of the importance values provided by these methods is frequently based on the relative order of the features (their ranking) rather than the importance values themselves. Since the order may be unstable, we present a framework for quantifying the uncertainty in global importance values. We propose a novel method for the post-hoc interpretation of feature importance values that is based on the framework and pairwise comparisons of the feature importance values. This method produces simultaneous confidence intervals for the features' ranks, which include the ``true'' (infinite sample) ranks with high probability, and enables the selection of the set of the top-k important features.
\end{abstract}

\section{Introduction}
Current quantum hardware is unable to carry out universal quantum computations due to the buildup of errors that occur during the computation. 
The magnitude of the individual error is currently above the value that the Threshold Theorem requires in order to kick-start quantum error correction and fault-tolerant quantum computation~\cite[Section 10.6]{nielsen_chuang_2010}. 
Although the experimentally achieved fidelity rates are promising and the error bounds are inching closer to the required threshold, we will have to work for the foreseeable future with quantum hardware with errors that build-up during the computation.  This implies that we can only do a limited number of steps before the output of the computation has become completely uncorrelated with the intended one.

For fault-tolerant quantum computing, we repeat four steps: 
1) We apply a number of single and two-qubit quantum gates, in parallel whenever possible; 
2) We perform a syndrome measurement on a subset of the qubits; 
3) We perform fast classical computations to determine which errors have occurred and how to correct them; 
and, 4) We apply correction terms based on the classical computations.
We then repeat these four steps with a next sequence of gates. 
These four steps are essential to fault-tolerant quantum computing. 


The starting point of this work is to use the four steps outlined above, not to carry out error correction and fault-tolerant computation, but to enhance short, constant-depth, {\em uncorrected} quantum circuits that perform single qubit gates and {\em nearest-neighbor} two qubit gates. 
Since in the long run we will have to implement error-correction and fault-tolerant computation anyhow, and this is done by such a four-step process, why not make other use of this architecture? Moreover, on some of the quantum hardware platforms, these operations are already in place.
Embracing this idea we naturally arrive at the question: what is the computational power of \textit{low-depth} quantum-classical circuits organized as in the four steps outlined above? 
We thus investigate circuits that execute a small, ideally constant, number of stages, where at each stage we may apply, in parallel, single qubit gates and {\em nearest-neighbor} two qubit gates, followed by measurements, followed by low-depth classical computations of which the outcome can control quantum gates in later stages. 
It is not clear, at first, whether such circuits, especially with constant depth, can do anything remotely useful. 
But we will see that this is indeed the case: many quantum computations can be done by such circuits in constant depth. 
By parallelizing quantum computations in this way, we improve the overall computational capabilities of these circuits, as we do not incur errors on qubits that are idle, simply because qubits are not idle for a very long time. 
Furthermore, reducing the depth of quantum circuits, at the cost of increasing width, allows the circuit to be run faster even if errors occur.

The first usage of such a four-step layout, not to do error correction, but to perform computations, can be found in the paradigm of measurement-based quantum computing~\cite{gottesman1999demonstrating,raussendorf2001one,jozsa2006introduction,clark2007generalised}: 
A universal form of quantum computing where a quantum state is prepared and operations are performed by measuring qubits in different bases, depending on previous measurements and intermediate measurements.

\citeauthor{PhamSvore2013} were the first to formalize the four-step protocol for performing computations~\cite{PhamSvore2013}. They included specific hardware topologies by considering two-dimensional graphs for imposing constraints on qubit interactions. In their model, they develop circuits for particularly useful multi-qubit gates, including specifying costs in the width, number of qubits, depth, number of concurrent time steps, size, and total number of non-Identity operations.
As a result, they find an algorithm that factors integers in polylogarithmic depth.
\citeauthor{Browne:2011} showed that the main tool in the work by \citeauthor{PhamSvore2013}, the fan-out gate, can also be replaced by additional log-depth classical computations in the measurement-based quantum computing setting~\cite{Browne:2011}.

More recently, \citeauthor{Cirac:2021} introduced a scheme to implement unitary operations involving quantum circuits combined with Local Operations and Classical Communication ($\mathsf{LOCC}$) channels: $\mathsf{LOCC}$-assisted quantum circuits~\cite{Cirac:2021}. Similarly to the four-step scheme we just described, they allow for a short depth circuit to be run on the qubits, followed by one round of $\mathsf{LOCC}$, in which ancilla qubits are measured and local unitaries are applied based on the measurement outcomes. They show that in this model any 1D transitionally invariant matrix-product state (MPS) with fixed bond dimension is in the same phase of matter as the trivial state. Similar ideas can be found in~\cite{TVV_NonAbelianTopologicalOrder_2022, tantivasadakarn2021long}.

In this work, we introduce a new model, called \textit{Local Alternating Quantum-Classical Computations} ($\LAQCC$). In this model we alternate between running quantum circuits (constrained by locality), ending in the measurement of a subset of qubits, and fast classical computations based on the measurement results. The outcome of the classical computations are then used to control future quantum circuits. We allow for flexibility in this model, by giving different constraints to the power of both the quantum circuits and the classical circuits as well as the number of alternations between them. 
Most attention will be given to $\LAQCC$ containing quantum circuits of constant depth, classical circuits of logarithmic depth and at most a constant number of alternations between them. 
Any circuit constructed in this model is considered to be of constant depth. 
We restrict ourselves to logarithmic depth classical computations, as this is the first natural and non-trivial extension beyond constant-depth classical computations. 
Constant-depth classical computations do however also have an equivalent constant-depth quantum implementation.

The definition of $\LAQCC$ sharpens the original definition of \citeauthor{PhamSvore2013} by adding constraints to the intermediate classical computations. This allows us to bound the power of $\LAQCC$ from above. 

The main result of \citeauthor{Cirac:2021}, that 1D translational invariant MPS with fixed bond dimension can be prepared by $\mathsf{LOCC}$-assisted circuits, relies on local symmetries of the MPS. These symmetries allow them to prepare local states (on a constant number of qubits) and glue them together by doing one round of the appropriate entangling measurement and corrections, after which they run a round of local unitaries to get the desired result. This general scheme for preparing states that exhibit an MPS description with the appropriate local symmetries requires only geometrically local unitaries and one round of measurement and corrections an therefore is accessible in $\LAQCC$. Studying different local symmetries, known as Symmetry Protected Topological (SPT) phases of matter, to find measurement-based constant depth circuits for states is a broad ongoing field of research~\cite{TVV_NonAbelianTopologicalOrder_2022, tantivasadakarn2021long, smith2023deterministic}. 
All these schemes have a $\LAQCC$ implementation.

%$\LAQCC$-circuits also exist for general schemes of preparing local states, based on the local tensors, and gluing them together using one round of entangled measurement and corrections, based on the local symmetry. 
%The main result of \citeauthor{Cirac:2021}, that 1D translational invariant MPS with fixed bond dimension can be prepared by $\mathsf{LOCC}$-assisted circuits, relies heavily on local symmetries of the MPS and as a result also has an equivalent $\LAQCC$ implementation. 
%The corrections applied after the measurement round are local unitaries depending on the local symmetries of the MPS. 

 

%This general scheme of preparing local states, based on the local tensors, and gluing it together by doing one round of entangled measurement and corrections, based on the local symmetry, is accessible in $\LAQCC$.
Note however that \citeauthor{Cirac:2021} also suggest a circuit for the $W$-state.
This circuit uses sequentially and dependent measurement-based corrections of the ancilla qubits. 
These dependent measurements translate to sequential alternations between the quantum and classical circuits and therefore increase the total depth to linear depth, exceeding the constant-depth constraints imposed by $\LAQCC$-circuits. 

We study the power of the $\LAQCC$ model with respect to state preparation, showing that even with only constant quantum-depth and logarithmic classical depth it remains possible to prepare states with long-range entanglement.
Another surprising result is that it is unlikely that $\LAQCC$ circuits are classically simulatable. We show that any instantaneous quantum polynomial-time (IQP) circuit~\cite{Bremner2010,Shepherd2009} has an $\LAQCC$ implementation.
Classical simulation of IQP circuits implies the collapse of the polynomial hierarchy to the third level, which is not believed to be true~\cite{Bremner2017}. Therefore, we expect that $\LAQCC$ circuits are unlikely to be classically simulatable. We bound the power of $\LAQCC$ by showing that it is contained in $\QNC^1$, the class of polynomial-size, log-depth circuits.

Next, we also study the power that intermediate classical calculations can add to quantum computations, by considering a new model that alternates between polynomially many polynomial-depth quantum circuits and unbounded classical computations
We study this model by doing a complexity theoretical analysis, where we draw inspiration from the notions of complexity given by \citeauthor{RosenthalYuen:2022}, \citeauthor{MetgerYuen:2023}, and \citeauthor{Aaronson:2004}.
All three complexity notions are based on the notion of state preparation, instead of more traditional definition of complexity such as the decidability of a computational problem. 
The first two consider classes based on sequences of quantum states preparable by a polynomial-sized quantum circuit, where the circuits are uniformly generated by a computational class, for instance, the class $\mathsf{PSPACE}$, which results in the complexity class $\mathsf{StatePSPACE}$~\cite{RosenthalYuen:2022,MetgerYuen:2023}.
The third notion considers a relative complexity, where the complexity is measured between two given states, and is measured by the number of gates, from a given gate-set, required to transform one state in another state~\cite{Aaronson:2004}. 
For our definition of state preparation complexity, we drop the uniformity constraint from~\cite{RosenthalYuen:2022,MetgerYuen:2023} and define a class as $\mathsf{StateX}$, which refers to states preparable by circuits of type $\mathsf{X}$. 
As an example, if $\mathsf{X} = \QNC^0$, this results in the class $\mathsf{StateQNC^0}$, which is the set of states preparable from the $\ket{0}^n$ state by poly-size constant-depth circuits. 
This notion is similar to the relative complexity from~\cite{Aaronson:2004}, where one state is the  $\ket{0}^n$ state and instead of counting the number of gates we consider the set of states preparable by a fixed number of gates. Using this notion of complexity we show that any state preparable by an $\LAQCC^*$ circuit is also preparable by a $\mathsf{PostQPoly}$ circuit, the class of circuits of polynomial depth with an additional post-selection gate. 

All Clifford circuits have a constant-depth $\LAQCC$ implementation, implying that any stabilizer state can be implemented by a constant-depth $\LAQCC$ circuit, see Section~\ref{sec:clifford_circuits} for a proof of this statement. 
Efficient circuits for stabilizer states have been known already through measurement-based quantum computing. Therefore this paper focuses on the preparation of non-stabilizer states, and as a surprising result we find novel constant-depth protocols for four very natural classes of non-stabilizer states.
Despite the extensive research into these four classes of non-stabilizer states and the many applications of them, no efficient constant- or low-depth state preparation protocols are known yet. We specifically consider these four classes as they are all often used as initial states in other algorithms.

The first state is a uniform superposition over an arbitrary number of states. 
This state finds applications in many quantum algorithms, as they often start with a uniform superposition over multiple states. 
This superposition is often achieved by applying Hadamard gates to every qubit due to its simplicity to prepare. 
Yet, the analysis of many algorithms, such as Shor's algorithm~\cite{Shor:1997}, would benefit from a different initial superposition. 
The circuit to prepare the uniform superposition over an arbitrary number of states uses an exact version of Grover search as a subroutine, that turns a probabilistic circuit, with a known constant probability of success, into a deterministic circuit. 
We use the circuit for preparing a uniform superposition over an arbitrary number of states as a subroutine in the next two quantum state preparation protocols. 

The second state is the $W$-state, the uniform superposition over all computational basis states of Hamming-weight~$1$, a natural long-ranged entangled state that displays a fundamentally nonequivalent type of entanglement from the Greenberger–Horne–Zeilinger state~\cite{WState:2000}, for which $\LAQCC$-type constant-depth circuits were previously known~\cite{PhamSvore2013, Cirac:2021}. 
The $W$-state is often used as benchmark for new quantum hardware~\cite{Haffner2005,Neeley2010,GarciaPerez:2021}. 
A novel way to prepare the $W$-state therefore gives a new way to benchmark different quantum devices with each other. 
A circuit for preparing the $W$-state was given in~\cite{Cirac:2021}, but this implementation requires sequentially alternating measurements followed by local unitaries, which in the $\LAQCC$ model is not considered to be of constant depth. 
We improve this protocol by giving an $\LAQCC$ implementation of the $W$-state, based on a compress-uncompress method that links the one-hot and binary encoding of integers.

The third state considered is the Dicke state, a generalization of the $W$-state, a superposition over all computational basis states with Hamming-weight $k$~\cite{Dicke:1954}. 
Dicke states have relevance in various practical settings.
For instance, for quantum game theory~\cite{zdemir2007}, quantum storage~\cite{Bacon_Compress:2006,Plesch:2010}, quantum error correction~\cite{ouyang2014permutation}, quantum metrology~\cite{toth2012multipartite}, and quantum networking~\cite{prevedel2009experimental}. 
Dicke states have been used as a starting state for variational optimization algorithms, most notably Quantum Alternating Operator Ansatz (QAOA)~\cite{Hadfield2019}, to find solutions to problems such as Maximum k-vertex Cover~\cite{Brandhofer2022,cook2020quantum}.
The ground states of physical Hamiltonians describing one-dimensional chains tend to show a resemblance to Dicke states such as states resulting from the Bethe ansatz, making them an ideal starting state when investigating the ground state behavior of these Hamiltonians~\cite{TDL_BetheAnsatzDerivation:2010,B_ExcitedStateQuantumPhaseTransitions:2013,DickeTransitions:2021}. 
For instance, the algorithm by \citeauthor{van2021preparing}, who give an algorithm to prepare the Bethe ansatz eigenstates of the spin-1/2 XXZ spin chain, starts by first preparing a Dicke state~\cite{van2021preparing}. 
A Dicke-state preparation protocol based on the compress-uncompress methodology used in the $W$-state furthermore finds applications in entanglement distillation, where the entanglement of a large state is concentrated on only a few qubits. 
Efficient deterministic circuits for preparing Dicke states have been proposed by \citeauthor{bartschi2019deterministic}~\cite{bartschi2019deterministic, bartschi2022deterministic_short_depth}. 
They provide a quantum circuit of depth $\mathO(k \log(\frac{n}{k}))$, allowing arbitrary connectivity, to prepare a Dicke state, which they conjecture to be optimal when $k$ is constant. 
In this work, we provide a constant-depth $\LAQCC$ circuit below their conjectured bound already for constant $k$. 
However, this does not directly disprove their conjecture, as we allow for intermediate measurements and classical computations. 
More significantly, we even construct constant-depth $\LAQCC$ circuits for $k = \mathO(\sqrt{n})$ greatly improving their bound.
This construction extends the compress-uncompress method for the $W$-state combined with additional subroutines. 

We continue with a log-depth state preparation protocol for the Dicke-state for arbitrary $k$. 
This protocol implements an efficient transformation between the factoradic number representation and the combinatorial number representation of a positive integer. 
The combinatorial number representation relates directly to the Dicke state. 
The provided efficient transformation between number representation systems might be of independent interest. 

We conclude by modifying our protocol for preparing a Dicke-state to a protocol that prepares quantum many-body scar states in constant-depth. 
These states have low entanglement and longer coherence times than states with similar energy density.
These characteristics make many-body scar states interesting to analyze and relevant within physics.
Many-body scar states appear for instance in the AKLT model~\cite{AKLT:1987,MRBAR:2018,MRB:2018} and different spin models~\cite{SI:2019,MOBFR:2020}.
Known methods for preparing these states have polynomial-depth~\cite{Gustafson:2023}, whereas our circuit has constant depth. 

% We conclude by studying the power that intermediate classical calculations can add to quantum computations. 
% In this study, we define a new model that relaxes constant-depth quantum circuits to polynomial depth quantum circuits, log-depth classical calculations to unbounded classical computations and a constant number of alternations to a polynomial number of alternations. 
% We call this model $\LAQCC^*$. 
% We study this model by doing a complexity theoretical analysis, where we draw inspiration from the notions of complexity given by \citeauthor{RosenthalYuen:2022}, \citeauthor{MetgerYuen:2023}, and \citeauthor{Aaronson:2004}.
% All three complexity notions are based on the notion of state preparation, instead of more traditional definition of complexity such as the decidability of a computational problem. 
% The first two consider classes based on sequences of quantum states preparable by a polynomial-sized quantum circuit, where the circuits are uniformly generated by a computational class, for instance, the class $\mathsf{PSPACE}$, which results in the complexity class $\mathsf{StatePSPACE}$~\cite{RosenthalYuen:2022,MetgerYuen:2023}.
% The third notion considers a relative complexity, where the complexity is measured between two given states, and is measured by the number of gates, from a given gate-set, required to transform one state in another state~\cite{Aaronson:2004}. 
% For our definition of state preparation complexity, we drop the uniformity constraint from~\cite{RosenthalYuen:2022,MetgerYuen:2023} and define a class as $\mathsf{StateX}$, which refers to states preparable by circuits of type $\mathsf{X}$. 
% As an example, if $\mathsf{X} = \QNC^0$, this results in the class $\mathsf{StateQNC^0}$, which is the set of states preparable from the $\ket{0}^n$ state by poly-size constant-depth circuits. 
% This notion is similar to the relative complexity from~\cite{Aaronson:2004}, where one state is the  $\ket{0}^n$ state and instead of counting the number of gates we consider the set of states preparable by a fixed number of gates. Using this notion of complexity we show that any state preparable by an $\LAQCC^*$ circuit is also preparable by a $\mathsf{PostQPoly}$ circuit, the class of circuits of polynomial depth with an additional post-selection gate. 

\paragraph{Summary of results}
\begin{itemize}
    \item We give a new definition of a computational model that captures the power of the four step process: applying a constant number of layers of one- and two-qubit gates; performing a syndrome measurement; perform a fast classical computation determining corrections; apply corrections. We call this model \emph{Local Alternating Quantum Classical Computations}, or $\LAQCC$ for short. In this model we bound the allowed quantum operations, intermediate classical calculations, and number of rounds separately. In Section~\ref{sec:LAQCC_model} we define this model and give a list of operations based on results from literature contained in this computational model. In some of these operations we explicitly use that we allow for multiple, but at most constant, rounds  of corrections.
    \item  We show show that there exist $\LAQCC$ circuits that can not be weakly simulated in Section~\ref{sec:IQP_in_LAQCC}. We further show that for every $\LAQCC$ circuit there exists a $\QNC^1$ circuit simulating it perfectly, in Section~\ref{sec:LAQCC_in_QNC1}.
    \item We introduce a new type computational complexity for preparing states and show that the extension of $\LAQCC$ where we allow a polynomial number of rounds and unbounded classical computation, is contained in $\mathsf{PostQPoly}$, the class of polynomial circuits with post-selection, in Section~\ref{sec:Complexity results}.
    \item We show a protocol to prepare the uniform superposition state of size $q$ in $\LAQCC$ using $\mathO(\ceil{\log_2(q)}^2)$ qubits in Section~\ref{sec:superposition_modulo_q}. 
    \item We show a protocol to prepare the $W_n$ state in $\LAQCC$ using $\mathO(n\log(n))$ qubits in Section~\ref{sec:W_state_in_LAQCC}.
    \item We show two ways of preparing the Dicke-$(n,k)$ state. The first method is in $\LAQCC$, works up to $k = \mathO(\sqrt{n})$, uses $\mathO(n^2\log(n))$ qubits, and is found in Section~\ref{sec:dicke:small_k}. The second method is in $\LAQCC\text{-}\mathsf{LOG}$ (an extension of $\LAQCC$ allowing for logarithmic number of alterations instead of constant), works for any $k$, uses $\mathO(\text{poly}(n))$ qubits, and is found in Section~\ref{sec:Dicke_in_LAQCC_LOG}. 
    \item We extend on our $\LAQCC$ method of generating Dicke-$(n,k)$ states for $k = \mathO(\sqrt{n})$ and show a protocol to generate many-body scar states for a particular Hamiltonian in $\LAQCC$ (Section~\ref{sec:many_body_scar}). 
\end{itemize}
Summarized in a table, we provide the following state generation protocols:
\begin{table}[htb]
\centering
\begin{tabular}{l|l|l|l}
\textbf{State description} & \textbf{Width} & \textbf{Depth} & \textbf{Implementation}\\
\hline 
Uniform superposition mod $q$: $\frac{1}{\sqrt{q}} \sum_{i = 0}^{q-1}\ket{i}$ & $\mathO(\ceil{\log^2 q})$ & $\mathO(1)$ & Section~\ref{sec:superposition_modulo_q}\\

$W$-state: $\frac{1}{\sqrt{n}}\sum_{i = 0}^{n-1}\ket{e_i}$ & $\mathO(n \log n)$ & $\mathO(1)$ & Section~\ref{sec:W_state_in_LAQCC}\\

Dicke-$(n,k)$, $k = \mathO(\sqrt{n})$: $\binom{n}{k}^{-1/2}\sum_{x \in \{0,1\}^n: |x| = k} \ket{x}$ &  $\mathO(n^2\log n)$ & $\mathO(1)$ 
&Section~\ref{sec:dicke:small_k}\\

Dicke-$(n,k)$: $\binom{n}{k}^{-1/2}\sum_{x \in \{0,1\}^n: |x| = k} \ket{x}$ & $\mathO(\text{poly}(n))$ & $\mathO(\log n)$ &Section~\ref{sec:Dicke_in_LAQCC_LOG}\\

QMBS: $\ket{S_k} = \frac{1}{k! \sqrt{\mathcal N(n,k)}}(Q^\dagger)^k \ket{\Omega}$ &  $\mathO(n^2\log n)$ & $\mathO(1)$  &  Section~\ref{sec:many_body_scar}
\end{tabular}
\caption{Summary of state preparation protocols given in this paper.}
\label{tab:sate_prep}
\end{table}
In the entry for the quantum many-body scar state $Q$ denotes the raising operator and $\mathcal N(n,k)=\binom{n-k-1}{k}$. 
Section~\ref{sec:many_body_scar} will provide more details on the variables and the implementation. 

\paragraph{Organization of the paper}
\noindent We first introduce relevant preliminaries in Section~\ref{sec:preliminaries}. 
In Section~\ref{sec:LAQCC_model} we formally define the class of Local Alternating Quantum-Classical Computations ($\LAQCC$). We also show that any Clifford circuit can be implemented in constant depth $\LAQCC$ (a result based on a result from measurement-based quantum computing~\cite{jozsa2006introduction}). 
This result allows us to give many useful multi-qubit gates and routines in Section~\ref{sec:gates_created_in_LAQCC}. 
Beyond that we show that constant depth $\LAQCC$ circuits are contained in $\QNC^1$ and that any $\mathsf{IQP}$ circuit has an $\LAQCC$ implementation.
We conclude this section with an analysis of a more powerful instantiation of $\LAQCC$ and show an inclusion with respect to the class $\mathsf{PostQPoly}$, which is the class of circuits of polynomial depth with one additional post-selection gate. 
In Section~\ref{sec:state_prep_in_LAQCC} we give $\LAQCC$ circuit implementations for preparing the uniform superposition over an arbitrary number of states, the $W$-state and the Dicke state up to $k = \mathO(\sqrt{n})$. We furthermore give a log-depth circuit implementation for preparing the Dicke state for any $k$. We conclude by showing a $\LAQCC$ circuit for generating many body scar states of a particular type of Hamiltonian.


% !TeX root = ../MVFD_arxiv.tex


\section{The framework}\label{sec2}
In this section we  present a formal mathematical setup for the local regularity  for bivariate   stochastic processes  (also called scalar random fields, or simply random fields) and the data observed for such processes.

\subsection{Data}\label{sec:data}
Consider $N$ independent realizations, also called sheets, $X^{(1)},\ldots,X^{(j)},\ldots X^{(N)}$   of a stochastic process $X $ defined on a continuous domain $\cT\in\mathbb R^2$. For simplicity, we here focus on domains $\mathcal T$ in the plane, the extension to higher dimensions would not raise different challenges. For the purpose of describing our methodology, we distinguish three observational scenarios of the $N$ realizations. 
First, the ideal, infeasible situation where the sheets $X^{(j)}$ are \emph{completely observed}, that is without error over the entire domain $\cT$.  
The second case is the one where the $X^{(j)}$ are observed (measured) at some \emph{discrete points} in the domain $\cT$, \emph{without noise}. The domain points can be fixed to be the same for all the $X^{(i)}$'s (common design), or can be randomly drawn for each sheets separately (independent design). Finally, the most realistic scenario is the one where in the second case we admit that the realizations of $X$  are observed at discrete domain points \emph{with noise}. 



 To formally describe the second and third scenarios, let  $M_1, \dotsc,M_N$ be an independent sample of an integer-valued random variable $M$ with expectation $\EE[M]=\Mmu$. 
% which increases with $N$. 
In the independent design case, for each $1\leq j \leq N$, and given  $M_j$, let $\Tnm\in\cT$,  $1\leq m \leq  M_j$, be a random sample of a random vector $\TT\in\cT$. The $\Tnm$'s represent the observation points for the realization $\Xp{j}$. We assume that the realizations of $X$, $M$ and $\TT$ are mutually independent. 
In the common design case, $M\equiv \mathfrak m$ and the $\Tnm$'s are the same for all $j$.  Let $\mathcal T_{obs}^{(j)} $ denote the set of observation times $ \Tnm $, $1\leq m \leq M_j$, on the sheet $\Xp{j}$. With common design,  $\mathcal T_{obs}^{(j)} $ does not depend on $j$, while with independent design the expected cardinal of $\mathcal T_{obs}^{(j)} $ can be random with mean $\mathfrak m$.   The following presentation  includes both independent design and common design cases. \color{black}  Finally, the data  consist of  the pairs  $(\Ynm , \Tnm ) \in\mathbb R \times \cT $ where $\Ynm$ is defined as
\begin{equation}\label{model_eq}
	\Ynm = \Xn (\Tnm) + \enm, \quad\text{with}  \quad  \enm = \sigma(\Tnm,\Xn(\Tnm)) \unm, 
	\quad 1\leq i \leq N,  \; 1\leq m \leq M_j.
\end{equation}
Here, the $\unm  \in\mathbb R $ are independent copies of a centered variable $e$ with unit variance, and $\sigma^2(\cdot,\cdot)\geq 0$ is some unknown, bounded conditional variance  function which account for possibly heteroscedastic measurement errors. The case  $\sigma(t,x)\equiv 0$ corresponds to our second scenario, while in the third scenario we have positive conditional variance. 

For each $1\leq j \leq N$,  let $\widetilde X^{(j)}$ denote an observable approximation of $X^{(j)}$. If   the sheets $X^{(j)}$ were  completely observed, as in our infeasible first scenario, $\widetilde X^{(j)} = X^{(j)}$. 
When $X^{(j)}$ are observed  only at  some discrete points $\Tnm$,  arbitrary  $\widetilde X^{(j)}(\Tt)$ can be obtained by simple interpolation or defined equal to the value of $\widetilde X^{(j)}$ at the nearest neighbor of $\Tt$.  Finally, with noisy, discretely observed sheets,   $\widetilde X^{(j)}$  is a pilot nonparametric estimator  of $X^{(j)}$, such as kernel smoothing, splines \emph{etc}.  


Let us next introduce a general class of stochastic processes (random fields)  $X$ with  irregular realizations $X^{(j)}$, for which the regularity can vary over the domain $\mathcal T$. 


\subsection{A class of multivariate processes}
Let $\mathcal{T}$ be an open, bounded bi-dimensional rectangle with the closure included in $(0,\infty)^2$. In the following, $H_1,H_2 : \mathcal T \to (0,1)$ are two continuously differentiable functions such that 
\begin{equation}\label{low_thres}
\min_{i=1,2} \inf_{t\in\mathcal T} H_i(\Tt) >0.
\end{equation} 	
Let $\overline{H} = \max\{H_1 ,H_2\}$.
We also consider the vector-valued function $\mathbf{L}=(L_1^{(1)},L_2^{(1)},L_1^{(2)},L_2^{(2)}),$  where the components  are  non-negative, Lipschitz  continuous functions defined on $\mathcal{T}$ such that 
\begin{equation}\label{id_L}
	L_i^{(1)}(\Tt) +L_i^{(2)}(\Tt) >0,\qquad \forall \Tt\in\mathcal T, \; i=1,2.
\end{equation}
%We denote $\mathbf{L}=(L_1^{(1)},L_2^{(1)},L_1^{(2)},L_2^{(2)}),$ and assume that a constant $C_{\mathbf L}$ exists such that 
%\begin{equation}\label{up_thres}
%C_{\mathbf{L}}=\max_{i,j=1,2} \sup_{t\in\mathcal T} L_j^{(i)}(\Tt) <\infty.
%\end{equation} 	

Let $X$ be a real-valued, second order stochastic process defined on $(0,\infty)^2$. Let $(e_1,e_2)$ be the canonical basis of $\mathbb R ^2,$ and, for sufficiently small scalars $\Delta$, let   
$$
\theta_{\Tt}^{(i)}(\Delta)=\EE\left[\left\{X\left(\Tt-\frac{\Delta}{2}e_i\right)-X\left(\Tt+\frac{\Delta}{2}e_i\right)\right\}^2\right],\quad i=1,2.
$$ 


\begin{definition}\label{def}
Let $H_1$, $H_2$ satisfy \eqref{low_thres}.	The class $\mathcal {H}^{H_1,H_2}(\mathbf{L},\mathcal{T})$ is the set of stochastic processes $X$ satisfying the following condition:  constants $  \Delta_0, C,\beta>0$ exist such that 
	for any $\Tt\in \mathcal T$ and $ 0<\Delta\leq\Delta_0$,  
	\begin{equation}\label{as_repr}
		\left|\theta_{\Tt}^{(i)}(\Delta)-L_1^{(i)}(\Tt)\Delta^{2H_1(\Tt)} -L_2^{(i)}(\Tt)\Delta^{2H_2(\Tt)}\right|\leq C\Delta^{2 \overline{H}(\Tt)+\beta}, \quad i=1,2.
	\end{equation}
	Let  
	$$
	\mathcal{H}^{H_1,H_2} %=\mathcal{H}^{H_1,H_2}(\mathcal{T})
	=\bigcup_{\mathbf{L}}\mathcal {H}^{H_1,H_2}(\mathbf{L},\mathcal{T}) ,
	$$
	where the union is taken over the set of four-dimensional functions $\mathbf{L}$ with non negative positives Lipschitz  continuous components satisfying \eqref{id_L} 
	%and \eqref{up_thres}. 
	The functions $H_1,H_2$ define the local regularity of the process, while $\mathbf{L}$ represent  the local Hölder constants.
\end{definition}

Definition \ref{def}  is general, and extends the local regularity notion considered by \cite{GKP} for processes defined on a compact interval on the real line.  A main example we have in mind is the  multi-fractional Brownian  sheet (MfBs) with a time-deformation. MfBs  is a generalization of the standard fractional Brownian sheet, where the Hurst parameter is allowed to vary along the  domain. The definition of this general class of  processes and some of their properties are provided in Section \ref{BfMs}.




\section{Probability model and target of inference}\label{sec:prob}

Here we develop a probability model for the uncertainty in ranks and identify the parameters of interest in the rank-sets. We then formalize our goal in defining a simultaneous confidence interval set for the rank-sets.

\subsection{True global FI and rank-set}


We model the rows $v_i$ as independent samples from distribution $F_v$ with mean vector $E[v_i] = (\trueimp_1,...,\trueimp_p)$, the \emph{true global FI values}.
For finite $n$, $\imp_j$ would be an unbiased but noisy version of $\trueimp_j$; we are interested in understanding the effects of this deviation on the possible feature rankings.


%\subsubsection{True ranking}

In contrast to the observed noisy rankings, the \emph{true ranking} $r_1,...,r_p$ are based on the true FI values $\trueimp_1,...,\trueimp_p$. 
Whereas in observed data exact ties are unlikely, 
for the true global FI we can imagine ties between equivalent features, or we may want to allow an indifference level.
If there are no ties, the lower and upper ranks are identical and equal to the standard definition.

We follow \cite{al2022simultaneous} in redefining the true ranks to account for ties:
\begin{definition}(Rank-set)
\label{def:setrank}
    Define the lower-rank of $\trueimp_j$ by $
        l_j = 1 + \#\{k: \trueimp_j > \trueimp_k, j \neq k\}
    $
    and the upper-rank of $\trueimp_j$ by  $
        u_j = p - \#\{k: \trueimp_j < \trueimp_k, j \neq k\} $.    Then the rank-set of $\trueimp_j$ is the set of natural numbers $\{l_j, l_{j+1}, \ldots, u_j\}$ denoted here $[l_j, u_j]$.
\end{definition}



\subsection{Confidence intervals of ranks}

Our inferential goal is to construct a confidence interval of ranks, estimated from the data, for each feature's importance such that with high probability, all intervals cover the true rank-sets.

\begin{definition}(Simultaneous coverage) The set of intervals $[L_1,U_1] , \ldots, [L_p, U_p] \subseteq [1, p]$ has simultaneous coverage at level $1 - \alpha$
if 
\begin{align*}
    \mathbb{P}\left([l_j, u_j] \subseteq [L_j, U_j] , \quad \forall j \in \{1, \ldots, p\}\right) \geq 1 - \alpha.
    \end{align*}
\end{definition}

$L_j, U_j$ are functions of $\mathbf{v}$ the observed base FI.
Different sets of observed base FI values would produce different confidence intervals. In a simultaneous $1-\alpha$ set of confidence intervals, the probability that all intervals in the set cover the true rank-sets is at least $1-\alpha$.
Note that simultaneous coverage is conservative and can result in relatively large intervals. 

An advantage of simultaneous coverage requirement is that coverage is maintained for many functions of the observed ranks across features. In particular, simultaneous intervals are valid after any selection of the parameters. 
The confidence interval of the most-important feature may not be valid in standard confidence intervals; but it is valid if the intervals are simultaneous. Extending this, we can select the top-k sets: instead of selecting the most important feature (or most important k features) based on the observed data, we may prefer selecting all features that could be ranked in the top-k as $n$, the number of base FI values, increases based on the confidence intervals. With simultaneous coverage, the error probability for this selection is controlled \cite{hsu1996multiple}. Furthermore, the confidence intervals for the features currently ranked top-k still have marginal coverage. These two properties are not guaranteed by the usual marginal coverage \cite{benjamini2005false}.


\subsection{Top k super-set}
FI values are commonly used to select the most important features for subsequent analysis, such as interventions %\cite{} 
or interpretation. %\cite{}. 
When the observed FI values are uncertain, it may be preferable to analyze a super-set that includes all features which can potentially be in the top $k$ set.

Define the true $Top_k \subseteq [p]$ as the set of features whose true FI is ranked in the top $k$, minding ties; e.g., $Top_k = \{j: u_k \geq p-k+1 \}$. Then, given a simultaneous confidence intervals, it is straightforward to derive a set covering the $Top_k$ with high confidence:

\begin{lemma}
Let $\{ [L_1, U_1],\ldots,[L_p,U_p] \}$ be a $1-\alpha$ simultaneous confidence intervals for the rank-sets. Define a set $\widehat{Top}_k  = \{j: U_j \geq p-k+1 \}$. This set includes all features with an upper bound in the top $k$ ranks. Then 
\begin{equation}
   P( Top_k\subseteq \widehat{Top}_k) \geq 1-\alpha.
\label{eq:topk}
\end{equation}
\end{lemma}
To prove this, consider the case where the event in 
\ref{eq:topk} does not hold. Then there exists some $j\in \widehat{Top}_k$ that is not in $Top_k$, meaning that the estimated upper bound $U_j$ exceeds the true upper rank $u_j$, so the confidence interval $[L_j,U_j]$ does not cover $[l_j, u_j]$. From the definition of $1-\alpha$ simultaneous confidence intervals, the probability of any such event is at most $\alpha.$
\section{Confident simultaneous feature ranking}\label{sec:conf_rank}

In this section, we introduce our ranking method which is designed to rank FI values while taking into account the uncertainty associated with the post-hoc FI method and the sampling process. Using our \emph{base-to-global} framework, we are able to quantify the uncertainty via simultaneous confidence intervals for the ranks.

\subsection{Feature ranking}

Our method uses pairwise hypothesis tests to form lower and upper bounds of the true rank of each feature.

For each feature pair $j,k$, we form two hypothesis tests:
\begin{align*}
H_{jk}^0:& \trueimp_j \geq \trueimp_k \text{ versus }  H^1_{jk}:  \trueimp_j < \trueimp_k,\\
H_{kj}^0:& \trueimp_k \geq \trueimp_j \text{ versus }  H^1_{kj}: \trueimp_k < \trueimp_j.
\end{align*}

The paired test takes vectors $\mathbf{v}_j, \mathbf{v}_k$ and returns a p-value $p_{j,k}$. Denote as
\begin{equation*}
p_{j,k} = pairedOnesidedTest(\mathbf{v}_j, \mathbf{v}_k, <).
\end{equation*}
A partial ranking is obtained by comparing $p_{j,k}$ to a threshold $\alpha$.
In practice, when both tests use the same data and the threshold $\alpha$ is less than $1/2$,  the two null hypotheses will not be rejected together.

To see the relationship between the hypothesis tests and the ranking, consider a feature pair $j,k$ for which we can reject with high probability 
the hypothesis $H_{jk}: \trueimp_j \geq \trueimp_k$. Then, with a high probability $\trueimp_j < \trueimp_k$, a partial ranking of the global FI. In that case, we conclude w.h.p. that feature $j$ is not ranked highest ($u_j < p$) and feature $k$ is not ranked lowest ($l_k > 1$).
Our method is based on \cite{al2022simultaneous}, and we discuss the differences in Sec. \ref{subsec:alg}.


\subsection{Controlling partial ranking error}

We would like to identify intervals that are simultaneous and therefore would still be valid under selection. To achieve this, we will look to control the FWER of all partial rankings.

Define $D$ the set of partial rankings from all pairwise tests 
$D \{(j,k) : H_{jk}^0 \text{was rejected} \}$
 
%\footnote {It is more common to use ``reject the null`` rather than ``accept the alternative''. However, here, we indeed have complimentary sets, hence rejecting $H_0$ means accepting $H_1$.} 
We make an error if there is a pair $(j',k')\in D$ for which $\trueimp_{j'}\geq \trueimp_{k'}$.
A level $\alpha$ calibrated test\footnote{See definition of calibrated test in App. \ref{app:proof}} assures that the probability of this event is less than $\alpha$. However, for simultaneous confidence intervals, we need a stronger condition.

\begin{definition}(Family wise error rate)
The pairwise tests  $\{pairedOnesidedTest(\mathbf{v}_j, \mathbf{v}_k), <\}$
 control the family wise error rate (FWER) at level $\alpha$ if:
 \begin{align*}
     P(\text{Making any error}) \leq \alpha.
 \end{align*}
\end{definition}
Indeed, the probability of making at least one error increases with more tests, and thus we need more conservative tests to control the FWER.


\subsection{Confident simultaneous feature ranking}\label{subsec:alg}

Once a family of pairwise tests is available with FWER control, we derive the simultaneous confidence rank intervals:

\begin{theorem} [Al Mohamad et al, \citep{al2022simultaneous} ]\label{thm}
Let $D$ be the set of partial rankings with FWER control at level $\alpha$. For $j = 1,...,p$, define 
\begin{align*}
    L_j = 1 + \#\{k : (k,j) \in D\}, \\
    U_j = p - \#\{k : (j,k) \in D\}.
\end{align*}
Then the sets $\{[L_j, U_j]$ for $j \in \features\}$ are simultaneous $(1-\alpha)$ confidence intervals for the ranks of the true global FI values.
\end{theorem}
The construction naturally extends the definition of rank-set used in Def. \ref{def:setrank}. The idea in this proof is that a coverage failure means that the set of true (one-sided) differences was smaller than the set of (one-sided) observed differences. This means that at least one partial ranking was an error. Therefore the FWER upper bounds the probability of an error in the confidence intervals. See proof in App. \ref{app:proof}.


Alg. \ref{alg} describes the complete method. The algorithm works directly on the base FI matrix without requiring access to the trained model, FI method, or the explanation data set. The main assumption is that our paired test is calibrated for the family of possible distributions for base FI values.

\begin{algorithm}
\caption{Confidence intervals for rank-sets}\label{alg}
\begin{algorithmic}
\REQUIRE 
\STATE $\mathbf{v}$: base FI values;
\STATE $1 - \alpha>0$: level of confidence;
\STATE $pairedOnesidedTest$: suitable paired test;
\FOR {$j,k \in \features, j \neq k$} {
\STATE    $p_{j,k} \gets pairOnesidedTest(\mathbf{v}_j, \mathbf{v}_k, <)$.
}\ENDFOR
%\Comment{Adjust the p-values $\{p_{j,k}\}_{j<k}$ to control the family-wise error rate}

\STATE $\{p_{j,k}^{adj}\}_{j\neq k} \gets Holm (\{p_{j,k}\}_{j\neq k} )$
\STATE $D \gets  \{(j,k) : p_{j,k}^{adj}< \alpha\}$

\FOR {$j \in \features$} {
\STATE $L_j \gets 1 + \#\{k : (k,j) \in D\}$
\STATE $U_j \gets p - \#\{k : (j,k) \in D\}$
}\ENDFOR

\RETURN $[L_1, U_1], \ldots, [L_p, U_p]$.
\end{algorithmic}
\end{algorithm}

\subsection{Implementation details}

\paragraph{Paired test} We use the parametric paired t-test:
Set $\mathbf{d} = \mathbf{v}_j - \mathbf{v}_k$ to be the vector of differences, 
write $\bar{d}$ for the sample average and $s_d$ for the sample standard deviation. Then the one-sided $\alpha$ level test rejects the null hypothesis if  $|\bar{d}/ (s_d/\sqrt{n})| >  T_{n-1}(1-\alpha)$, 
where the r.h.s. marks the $1-\alpha$ quantile of student-t ($n-1$ df).
The paired t-test is fairly robust to departures from normality\cite{posten1979robustness}.

\paragraph{Correction to multiple comparisons} We use the sequential Holm \cite{holm1979simple} procedure to adjust (increase) the p-values. The adjusted p-values are then compared to the requested level $\alpha$; if the p-values are calibrated, the FWER for the rejected tests (after Holm) is controlled at level $\alpha$, regardless of dependence. We currently implement the Holm on the one-sided tests, though for approximately Normal data the two-sided Holm probably still works well. See Shaffer \cite{shaffer1980control, shaffer1995multiple} on using Holm for pair-wise comparisons.

\paragraph{Comparison to ICranks} Algorithm \ref{alg} is similar to the ICranks \cite{al2022simultaneous}, with some differences. In ICranks, they used the Tukey correction \cite{tukey1953problem} for the normal distribution to control all differences between ranks simultaneously. It assumes normality and independence between features and is hard to replace with non-parametric tests. In contrast, our algorithm applies tests to each feature pair and uses the Holm correction; hence, it can be used with robust or non-parametric location tests \cite{wilcox2011introduction}. 
%\begin{lemma}
%    \todo[inline]{Holm's methods controls FWER ...}
%\end{lemma}


\section{Experiments}
% \haizhou{Follow the same way of introduction as we did in Section2.}
% \noindent In this section, we will introduce datasets and experimental setups that we used. Then we evaluate our method, other self-supervised methods, and supervised methods under different distribution shifts (\ie, concept shifts and covariate shifts) under common settings (\ie, transductive, inductive settings). It has to note that we focus on node-level tasks (\eg, node classification) in this work. As for graph-level tasks, we leave it as our future work and some simple experiments can be found in Appendix~\ref{app:graph_classification}. 
In this section, we first introduce the experimental setup including datasets, training, and evaluation protocol in Section~\ref{sec:dataset}~and~\ref{sec:unsupervised}. 
% Next, we present our experimental setup and conduct extensive experiments to evaluate our method in Section~\ref{sec:unsupervised}. 
We then perform an ablation study to demonstrate the effectiveness of each proposed component in Section~\ref{sec:ablation}. 
Additionally, we analyze the impact of important hyper-parameters in Section~\ref{sec:sensitivity}. 
Subsequently, we integrate our method with various encoding models, showcasing the model-agnostic nature of our recipe in Section~\ref{sec:other_models}. 
Finally, we provide some qualitative results such as feature visualization in Section~\ref{sec:vis}.
It is important to note that we focus on node-level tasks (\eg, node classification) in this work. As for graph-level tasks, we leave it as our future work, while some simple experiments are also provided in Appendix~\ref{app:graph_classification}.

\subsection{Datasets}\label{sec:dataset}
There exist some benchmarks for evaluating graph out-of-distribution generalization~\cite{good,ji2022drugood,gds}. 
Among them, GOOD~\cite{good} is the most representative and comprehensive benchmark that curates more diverse graph datasets with diverse tasks, including single/multi-task graph classification, graph regression, and node classification involving more distribution shifts (\ie, concept shifts and covariate shifts). Hence in this work, we follow the evaluation protocol proposed in \cite{good}. Furthermore, we validate the effectiveness of our method in the datasets (\ie, Amazon-Photo, Elliptic) that are used in EERM~\cite{eerm}. The statistics and detailed introduction to these datasets can be found in Table~\ref{tab:dataset} and Appendix~\ref{app:datasets}.

\begin{table*}[htp]
\caption{The descriptions of datasets. ``Domain-Level'' means splitting by graphs, ``Time-Aware'' denotes splitting according to chronological order.``Word'' and ``Degree'' represent splitting according to word diversity and node degree respectively. ``Language'' means splitting by user language, suggesting the prediction should not be impacted by the language the user use. ``University'' denotes splitting according to the domain university, implying that the prediction of webpages should be based on word contents and link connections rather than university features. ``Color'' means that nodes are split according to node differences in covariate shift and color-label correlations in concept shift.}
\label{tab:dataset}
\centering
\begin{tabular}{cccccccc}
\toprule
Datasets     & Network Type        & \#Nodes & \#Edges & \#Attributes &\#Classes& Train/Val/Test Split     & Metric   \\
% Cora         & Artificial Transformation & 2,703   &         &              &         &                      & Accuracy \\
Amazon-Photo\footnotemark
             & Co-purchasing network      & 7,650   & 119,081   & 755          & 10      & Domain-Level         & Accuracy \\
Elliptic\footnotemark  
             & Bitcoin transactions       & 203,769 & 234,355   & 165          & 2       & Time-Aware           & F1-Score \\
GOOD-Cora    & Scientific publications    & 19,793  & 126,842   & 8,710         & 70      & Word/Degree          & Accuracy \\
% GOOD-Arxiv   & arXiv papers               & 169,343 & 2,315,598 & 128          & 40      & Time/Degree          & Accuracy \\
GOOD-Twitch  & Gamer network              & 34,120  & 892,346   & 128          & 2       & Language             & ROC-AUC  \\
GOOD-CBAS    & A BA-house graph           & 700     & 3,962     & 4             & 4       & Color                & Accuracy \\
GOOD-WebKB   & Webpage network            & 617     & 1,138     & 1,703         & 5       & University           & Accuracy \\
\bottomrule
\end{tabular}
\end{table*}
\footnotetext[5]{This dataset is adopted from~\cite{yang2016revisiting}. \cite{eerm} constructs ten graphs with different environment id’s for each graph.} 
\footnotetext[6]{The original is available on \hyperlink{https://www.kaggle.com/ellipticco/elliptic-data-set}{https://www.kaggle.com/ellipticco/elliptic-data-set}}

\subsection{Unsupervised Representation Learning}\label{sec:unsupervised}
\subsubsection{Transductive Setting}~\label{sec:trans}
% \noindent\textbf{Baselines.}\quad We conduct experiments with 12 baselines which consist of three categories: supervised methods and self-supervised generative methods, self-supervised contrastive methods. Specifically, we compare with three supervised baselines: empirical risk minimization~(ERM)~\cite{erm}, invariant risk minimization (IRM)~\cite{irm}, and a recent proposed graph OOD method dubbed EERM~\cite{eerm}. We also compare various unsupervised node-level representation learning methods: three self-supervised generative methods including GAE~\cite{gae}, VGAE~\cite{gae}, GraphMAE~\cite{gmae} and seven self-supervised contrastive methods: DGI~\cite{dgi}, MVGRL~\cite{mvgrl}, GRACE~\cite{grace}, RoSA~\cite{rosa}, BGRL~\cite{bgrl}, COSTA~\cite{costa}, SwAV~\cite{swav}. The descriptions of these methods can be found in Appendix~\ref{app:baselines}.
In this subsection, we focus on validating our proposed algorithm under the transductive setting, where the test nodes will participate in message passing~\cite{gilmer2017neural} during training following~\cite{good}. 

\noindent\textbf{Baselines.} We conduct experiments with 12 baselines from three categories: (i)~supervised methods, including empirical risk minimization~(\textbf{ERM})~\cite{erm}, invariant risk minimization (\textbf{IRM})~\cite{irm}, and a recent proposed graph OOD method \textbf{EERM}~\cite{eerm}; (ii)~self-supervised generative methods including Graph Autoencoder (\textbf{GAE})~\cite{gae}, Variational Graph Autoencoder (\textbf{VGAE})~\cite{gae}, Self-Supervised Masked Graph Autoencoders (\textbf{GraphMAE})~\cite{gmae}; (iii)~self-supervised contrastive methods including Deep Graph Infomax (\textbf{DGI})~\cite{dgi}, Contrastive Multi-View Representation Learning on Graphs (\textbf{MVGRL})~\cite{mvgrl}, Deep Graph Contrastive Representation Learning (\textbf{GRACE})~\cite{grace}, A Robust Self-Aligned Framework for Node-Node Graph Contrastive Learning (\textbf{RoSA})~\cite{rosa}, Bootstrapped Representation Learning on Graphs (\textbf{BGRL})~\cite{bgrl}, Covariance-Preserving Feature Augmentation for Graph Contrastive Learning (\textbf{COSTA})~\cite{costa}, Unsupervised Learning of Visual Features by Contrasting Cluster Assignments (\textbf{SwAV})~\cite{swav}. The detailed descriptions of these baselines can be found in Appendix~\ref{app:baselines}.

\noindent\textbf{Experimental setup.} We use the same graph encoder across different datasets for a fair comparison following~\cite{good}. We use grid search to find other hyper-parameters (\eg, learning rate, epochs) for different methods. For all experiments, we select the best checkpoints for ID and OOD tests according to results on ID and OOD validation sets following~\cite{good}, respectively. Experimental details and hyper-parameter selections are provided in Appendix~\ref{app:hyper}. For evaluating unsupervised methods, a linear classifier will be built on the frozen trained encoder after finishing pre-training. The reported results are the mean performance with standard deviation after 10 runs following~\cite{good}.

\noindent\textbf{Analysis.}\quad Based on the experimental results listed in Table~\ref{tab:trans_concept} and \ref{tab:trans_covariate}, we can draw the following conclusions: firstly, we find strong self-supervised methods (\eg, GRACE, BGRL, COSTA) are more robust to distribution shifts (concept shift in Table~\ref{tab:trans_concept} and covariate shift in Table~\ref{tab:trans_covariate}) compared to supervised methods. For instance, on GOOD-CBAS and GOOD-WebKB datasets, GRACE surpasses the best supervised method by large margins (over 6\% absolute improvement). Interestingly, we find the methods designed for OOD generalization (\ie, IRM) and graph OOD generalization (\ie, EERM) do not attain superior performance than the standard ERM on most of the datasets. For example, EERM shows superior OOD performance compared to ERM in only one experiment, and IRM outperforms ERM in four out of ten experiments across the conducted evaluations. This phenomenon is also observed in \cite{good,ahuja2020empirical,rosenfeld2021risks}, showcasing the challenge of achieving invariant prediction in non-Euclidean graph settings. 

Furthermore, our method surpasses other SOTA self-supervised methods on the OOD test set of all datasets by a considerable margin while achieving comparable performance in the in-distribution test set. For instance, on small datasets such as GOOD-CBAS and GOOD-WebKB, our method outperforms GRACE\footnote{MARIO is built up on GRACE according to our recipe. So, we make a comparison with GRACE here.} by over 2\% absolute accuracy on the OOD test set. On larger datasets such as GOOD-Cora and GOOD-Twitch, our method still outperforms other methods which shows its superiority. For instance, under covariate shift, MARIO surpasses other methods by over 7\% absolute accuracy on the GOOD-Twitch OOD test set. These statistics prove the effectiveness of our design.


\begin{table*}[htp]
\caption{Experimental results of all methods under concept shift. The bold font means the top-1 performance and the underline represents the second performance across the unsupervised methods. 'ID' represents in-distribution test performance and 'OOD' means out-of-distribution test performance. (OOM: out-of-memory on a GPU with 24GB memory)}
\label{tab:trans_concept}
\centering
\scalebox{0.95}{
\begin{tabular}{l|cc|cc|cc|cc|cc}
\toprule
\toprule
\multirow{3}{*}{concept shift} & \multicolumn{4}{c|}{GOOD-Cora}                   & \multicolumn{2}{c|}{GOOD-CBAS} & \multicolumn{2}{c|}{GOOD-Twitch} & \multicolumn{2}{c}{GOOD-WebKB} \\
                           & \multicolumn{2}{c}{word} & \multicolumn{2}{c|}{degree}& \multicolumn{2}{c|}{color}    & \multicolumn{2}{c|}{language}   & \multicolumn{2}{c}{university} \\
                           & ID         & OOD         & ID          & OOD          & ID            & OOD           & ID             & OOD            & ID            & OOD            \\
\midrule
ERM                        & 66.38±0.45 & 64.44±0.18  & 68.60±0.40  & 60.76±0.34   & 89.79±1.39    & 83.43±1.19    & 80.80±1.00     & 56.92±0.92     & 62.67±1.53    & 26.33±1.09     \\
IRM                        & 66.42±0.41 & 64.29±0.31  & 68.57±0.35  & 61.45±0.24   & 89.64±1.21    & 82.29±1.14    & 78.87±1.04     & 59.30±1.79     & 62.67±1.10    & 26.88±1.42     \\
EERM                       & 65.10±0.44 & 62.45±0.19  & 66.95±0.44  & 56.58±0.25   & 79.07±2.12    & 64.50±1.01    & OOM            & OOM            & 62.50±2.01    & 28.07±3.23      \\
\midrule
% Random-Init                & 37.53±1.74 & 32.12±1.24  & 37.82±1.71  & 27.74±1.14   &               &               &                &                & 60.33±2.21    & 27.07±1.70     \\
GAE                        & 60.65±0.89 & 58.00±0.55  & 62.59±1.11  & 53.44±0.80   & 75.28±1.36    & 68.07±2.05    & 81.25±0.81     & 51.51±1.05     & 62.17±3.34    & 25.78±1.85     \\
VGAE                       & 63.19±0.53 & 60.35±0.47  & 61.65±0.66  & 54.28±0.28   & 76.50±0.50    & 59.07±0.56    & 80.46±0.53     & 55.56±4.53     & 62.50±2.38    & 24.40±2.57     \\
GraphMAE                   & \underline{66.44±0.46} & \underline{64.87±0.30}  & 67.95±0.46  & 59.41±0.39   & 89.14±0.89    & 82.93±0.93    & 80.05±0.64     & 59.38±1.49     & 61.83±3.37    & 29.27±2.15     \\
DGI                        & 63.33±0.56 & 60.71±0.49  & 65.93±1.02  & 55.83±0.53   & 91.22±1.47    & 85.00±1.66    & 80.05±0.87     & 59.16±1.88     & 61.83±2.83    & 28.63±1.92      \\
MVGRL                      & OOM        & OOM         & OOM         & OOM          & 88.57±1.15    & 76.50±1.17    & OOM            & OOM            & 62.00±3.79    & 28.26±4.20     \\
GRACE                      & 65.61±0.61 & 63.92±0.44  & \textbf{68.59±0.35}  & 60.15±0.45   & 92.00±1.39    & 88.64±0.67    & \textbf{83.43±0.63}     & \underline{60.45±1.46}     & 64.00±3.43    & \underline{34.86±3.43}  \\
RoSA                       & 64.06±0.67 & 62.44±0.39  & 67.07±0.65  & 57.68±0.44   & 90.78±2.27    & 85.93±2.14    & 82.39±0.42     & 57.45±2.16     & 64.17±4.10    & 32.20±2.15     \\
BGRL                       & 65.18±0.43 & 63.43±0.45  & 66.83±0.80  & 59.63±0.38   & 92.36±1.16    & 87.14±1.60    & 82.52±0.60     & 55.48±1.48     & 63.67±2.33    & 31.47±3.43     \\
COSTA                      & 65.05±0.80 & 62.37±0.45  & 66.76±0.87  & 55.73±0.36   & \underline{93.50±2.62}    & \underline{89.29±3.11}    & 83.15±0.30 & 55.03±3.22     & 61.66±2.58    & 32.39±2.13 \\
% ArCL                       &            &             & 67.64±0.57  & 59.71±0.44   &               &               &                &                & 65.00±3.94    & 35.41±1.97 \\      
SwAV                       & 62.22±0.53 & 59.79±0.53  & 64.65±0.94  & 55.06±0.39   & 89.00±0.79    & 81.72±0.66    & \underline{83.32±0.15}     & 59.69±1.97     & \underline{65.17±3.76}    & 29.36±2.01    \\
\midrule
MARIO                       & \textbf{67.11±0.46} & \textbf{65.28±0.34}  & \underline{68.46±0.40}  & \textbf{61.30±0.28}   & \textbf{94.36±1.21}    & \textbf{91.28±1.10}    & 82.31±0.54     & \textbf{63.33±1.72}     & \textbf{65.67±2.81}    & \textbf{37.15±2.37}     \\
\bottomrule
\end{tabular}}
\end{table*}

\begin{table*}[htp]
\caption{Experimental results of all methods under covariate shift. The bold font means the top-1 performance and the underline represents the second performance across the unsupervised methods. 'ID' represents in-distribution test performance and 'OOD' means out-of-distribution test performance. (OOM: out-of-memory on a GPU with 24GB memory)}
\label{tab:trans_covariate}
\centering
\scalebox{0.95}{
\begin{tabular}{l|cc|cc|cc|cc|cc}
\toprule
\toprule
\multirow{3}{*}{covariate shift} & \multicolumn{4}{c|}{GOOD-Cora}                                   & \multicolumn{2}{c|}{GOOD-CBAS} & \multicolumn{2}{c|}{GOOD-Twitch} & \multicolumn{2}{c}{GOOD-WebKB} \\
                           & \multicolumn{2}{c}{word} & \multicolumn{2}{c|}{degree}& \multicolumn{2}{c|}{color}    & \multicolumn{2}{c|}{language}   & \multicolumn{2}{c}{university} \\
                           & ID         & OOD         & ID          & OOD          & ID            & OOD           & ID             & OOD            & ID            & OOD            \\
\midrule
ERM                        & 70.50±0.41 & 64.69±0.33  & 72.46±0.49  & 55.53±0.50   & 92.00±3.08    & 77.57±1.29    & 70.98±0.41     & 49.35±5.09     & 39.34±1.79    & 14.52±3.14   \\
IRM                        & 70.48±0.26 & 64.53±0.57  & 71.98±0.34  & 53.72±0.46   & 90.86±2.41    & 78.86±1.67    & 69.81±0.95     & 49.11±2.82     & 38.52±3.30    & 13.97±2.80     \\
EERM                       & OOM        & OOM         & OOM         & OOM          & 65.00±2.57    & 57.43±3.60    & OOM            & OOM            & 46.07±4.55    & 27.40±7.65     \\
\midrule
GAE                        & 56.63±0.79 & 48.93±0.93  & 66.30±0.88  & 34.01±0.87   & 73.00±2.16    & 60.86±3.01    & 67.24±1.23     & 47.65±2.49     & 45.08±6.32    & 28.02±6.29    \\
VGAE                       & 62.02±0.66 & 54.12±0.86  & 69.41±0.57  & 44.20±1.29   & 62.29±2.04    & 63.29±1.11    & 66.99±1.43     & \underline{50.48±4.58}     & 48.85±4.68    & 20.87±6.69     \\
GraphMAE                   & 68.14±0.43 & 64.00±0.33  & \textbf{73.36±0.56}  & 53.75±0.55   & 67.28±3.03    & 67.28±1.49    & 68.84±1.20     & 48.02±2.79     & 48.03±4.34    & 30.00±8.09     \\
DGI                        & 60.85±0.75 & 57.03±0.67  & 68.97±0.41  & 41.75±0.88   & 69.57±4.09    & 59.71±3.43    & 68.43±1.05     & 44.83±1.61     & 48.52±5.04    & 21.11±7.50     \\
MVGRL                      & OOM        & OOM         & OOM         & OOM          & 65.00±1.94    & 64.15±0.77    & OOM            & OOM           & \textbf{54.10±5.39}    & 16.59±6.51     \\
GRACE                      & \underline{68.77±0.33} & \underline{64.21±0.41}  & 72.69±0.34  & \underline{56.10±0.63}   & \underline{93.57±1.83}    & \underline{89.29±3.40}    & \underline{71.12±0.87} & 46.21±1.54 & 49.67±5.82    & 28.10±4.68    \\
RoSA                       & 68.19±0.56 & 62.48±0.61  & 71.04±0.62  & 52.72±0.79   & 84.71±4.14    &79.14±3.51     & 70.58±0.36     & 45.83±1.72     & 52.30±4.24    & \underline{34.24±7.92}     \\
BGRL                       & 67.23±0.43 & 61.33±0.36  & 72.11±0.39  & 49.15±0.73   & 89.00±2.56    & 79.86±3.29    & \textbf{71.43±0.53}     & 43.86±0.94     & 51.80±5.55    & 30.32±7.61    \\
COSTA                      & 65.28±0.60 & 60.33±0.53  & 70.65±0.62  & 54.03±0.28   & 92.29±1.59    & 82.71±2.74    & 69.29±1.37     & 49.07±2.13     & 50.49±3.01    & 29.84±4.75   \\
SwAV                       & 63.29±1.01 & 56.98±0.94  & 70.27±0.73  & 43.00±0.52   & 89.57±1.12    & 81.43±1.69    & 69.19±0.93     & 49.37±2.96     & 49.84±4.82    & 30.55±6.72   \\
\midrule
MARIO                       & \textbf{69.99±0.54} & \textbf{65.06±0.34}  & \underline{72.73±0.43}  & \textbf{57.73±0.45}  & \textbf{94.57±2.46}    & \textbf{91.00±2.48}     & 68.31±0.78 & \textbf{57.37±1.37}     & \underline{53.94±3.23}    & \textbf{35.24±4.98}   \\
\bottomrule
\end{tabular}}

\end{table*}

\subsubsection{Inductive Setting}
In this subsection, we conduct experiments under the inductive settings, where the test nodes are kept unseen during training. This setting is more suitable for domain generalization.
% But we think it is more convincing that conduct experiments under inductive settings which means test nodes are unseen during training. This setting is more appropriate for domain generalization.

\noindent\textbf{Baselines:} For GOOD-WebKB and GOOD-CBAS datasets, we adopt ERM, IRM, GraphMAE, and GRACE as our baselines. And for Amazon-Photo and Elliptic datasets, we select ERM, EERM, and GRACE as our baselines.

\noindent\textbf{Experimental setup:} For GOOD-WebKB and GOOD-CBAS datasets, we use the same model configuration in Section~\ref{sec:trans}.
% Besides, we add experiments on Amazon-Photo dataset~\cite{yang2016revisiting} and Elliptic~\cite{elliptic} dataset in this subsection. 
For Amazon-Photo dataset~\cite{yang2016revisiting} and Elliptic~\cite{elliptic} dataset, they consist of many snapshots (training data and testing data use different snapshots) which are naturally inductive. For Amazon-Photo dataset, we use 2-layer GCN~\cite{gcn} as the encoder and for elliptic dataset, we use 5-layer GraphSAGE~\cite{sage} as encoder following~\cite{eerm}.

% Figure environment removed

\noindent\textbf{Analysis:}
According to Figure~\ref{fig:amazon},\ref{fig:elliptic},\ref{fig:ind_con},\ref{fig:ind_cov}, we can draw following conclusions:
firstly, based on Figure~\ref{fig:amazon}, it is evident that our method outperforms other representative supervised and self-supervised methods on all test graphs (T1$\sim$T8). This superiority is reflected in the larger median value of our method compared to others. For instance, MARIO achieves over a 3\% absolute improvement compared to ERM in terms of the mean value of eight median values. Additionally, our method demonstrates higher stability across different random initializations, as indicated by the closer proximity of the first and third quartile values to the median value~(\eg, the difference of first and third quartile values of ERM, EERM, GRACE and MARIO are 4.2, 3.3, 6.7 and 1.0 on T8 respectively which indicates MARIO is much more stable than other methods). Furthermore, our method exhibits consistent performance across different graphs (\eg, The standard deviation of median values on T1$\sim$T8 for ERM, EERM, GRACE, and MARIO are 0.4, 1.1, 1.2, and 0.3, respectively.), indicating its robustness to environmental variations and its ability to extract invariant features: $g(G^e) \approx g(G^{e'})$ for all $e, e' \in \mathcal{E}^\text{train}$. In summary, our method showcases enhanced OOD generalization capabilities.
% $g(G^e)g(G^e^\prime)$ where $any e, e^\prime in \mathcal{E}^{train}$

Secondly, from the results presented in Figure~\ref{fig:elliptic}, we can observe that our method averagely harvests 10.9\% absolute improvement over GRACE and 12.5\% absolute improvement over EERM in terms of F1 scores on Elliptic dataset. This demonstrates the effectiveness of our method in handling distribution shifts and improving performance compared to existing approaches. It is worth noting that GRACE's performance worsens over time, indicating its inability to handle distribution shifts effectively. In contrast, our method consistently achieves better F1 scores, except for T9, which is caused by the dark market shutdown occurred after T7~\cite{elliptic}. The emergence of such an event introduces significant variations in data distributions, which subsequently results in performance degradation for all methods. Indeed, this event serves as an unpredictable external factor that introduces significant challenges for models trained on limited training data. The results indicate that the performance heavily depends on available training data. Nonetheless, our approach outperforms other methods even in such an extreme case. This highlights the effectiveness of our method in addressing distribution shifts and improving generalization performance.

Finally, based on the observations from Figure~\ref{fig:ind_con} and Figure~\ref{fig:ind_cov} MARIO demonstrates the best performances on both ID and OOD test sets for GOOD-WebKB and GOOD-CBAS datasets, under both concept shift and covariate shift. Notably, MARIO outperforms other methods by more than 3\% and 10\% absolute improvement on GOOD-WebKB and GOOD-CBAS, respectively, under covariate shift. We can draw similar conclusions as discussed in Section~\ref{sec:trans}. Even under the inductive setting, our method continues to demonstrate excellent OOD generalization capabilities and achieves comparable or even improved in-distribution test performance. These statistical results further validate the effectiveness of our method in handling distribution shifts and enhancing generalization performance.

Overall, the observations we have made provide strong evidence of the great capacity of our method for handling distribution shifts, validating its effectiveness and potential for real-world applications.



% Figure environment removed

% Figure environment removed


% Figure environment removed


\subsection{Ablation Studies}\label{sec:ablation}
\noindent Table~\ref{tab:aba} provides a detailed analysis of the effect of each component according to our proposed recipe for improving OOD generalization in graph contrastive learning. Let's examine the different variants of our method and their impact on performance.
Specifically, MARIO~(w/o ad) represents MARIO without  adversarial augmentation. MARIO~(w/o cmi) denotes we only maximize the mutual information between positive pairs without considering conditional mutual information. MARIO~(w/o cmi, ad) means a vanilla graph contrastive method that is similar to GRACE. 

From Table~\ref{tab:aba}, we can find MARIO~(w/o cmi) lags far behind MARIO on OOD test set which demonstrates appropriately minimizing the redundant information (\ie, conditional mutual information) is essential to improve OOD generalization of GCL methods. And adversarial augmentation can also boost OOD generalization because it can approximately serve as a supermum operator to learn more invariant features  discussed in Section~\ref{sec:aug}. Based on the analysis of these variants, it is evident that the proposed improvements on data augmentation and contrastive loss in the recipe are both effective in enhancing graph OOD generalization. Each component contributes to the overall performance improvement, and their combination leads to a stronger self-supervised graph learner in terms of graph OOD generalization. 

In short, the findings from Table~\ref{tab:aba} support the rationale behind your proposed recipe and provide empirical evidence of the effectiveness of each proposed component. By incorporating these enhancements, our method achieves superior performance in handling distribution shifts and improving graph OOD generalization in graph contrastive learning.
\begin{table*}[htp]
\caption{Ablation studies for MARIO by masking each component.}
\label{tab:aba}
\centering
\scalebox{0.9}{
\begin{tabular}{l|cc|cc|cc|cc|cc}
\toprule
\toprule
\multirow{3}{*}{concept shift} & \multicolumn{4}{c|}{GOOD-Cora}                       & \multicolumn{2}{c|}{GOOD-CBAS} & \multicolumn{2}{c|}{GOOD-Twitch} & \multicolumn{2}{c}{GOOD-WebKB} \\
                           & \multicolumn{2}{c}{word} & \multicolumn{2}{c|}{degree}& \multicolumn{2}{c|}{color}    & \multicolumn{2}{c|}{language}   & \multicolumn{2}{c}{university} \\
                           & ID         & OOD         & ID          & OOD          & ID            & OOD           & ID             & OOD            & ID            & OOD            \\
\midrule
MARIO                      & \textbf{67.11±0.46} & \textbf{65.28±0.34}  & \textbf{68.46±0.40}  & \textbf{61.30±0.28}      & \textbf{94.36±1.21}  & \textbf{91.28±1.10}    & 82.31±0.54     & \textbf{63.33±1.72}     & \textbf{65.67±2.81}    & \textbf{37.15±2.37}     \\
MARIO(w/o ad)              & 66.23±0.53 & 64.02±0.18  & 67.88±0.38  & 60.46±0.29   & 93.21±1.25    & 90.29±0.91    & 82.42±0.73     & 60.50±1.02     & 64.83±2.83    & 36.51±3.25    \\
MARIO(w/o cmi)             & 65.32±0.60 & 63.51±0.32  & 68.14±0.32  & 61.19±0.34   & 94.15±1.23    & 90.57±1.96    & \textbf{82.51±0.56}     & 61.41±2.63     & 64.50±4.35    & 35.78±2.53     \\
MARIO(w/o cmi, ad)         & 64.67±0.55 & 63.11±0.32  & 67.95±0.65  & 60.01±0.57   & 93.36±1.66    & 89.64±1.73    & 81.90±0.75     & 60.12±1.60     & 64.17±3.67    & 34.13±2.38     \\
\bottomrule
\end{tabular}}
\end{table*}
% & 65.32±0.60 & 63.51±0.32 exchange 64.67±0.55 & 63.11±0.32
% 68.14±0.32       id ood test: 60.95±0.43       ood ood test: 61.19±0.34


\subsection{Sensitivity Analysis}\label{sec:sensitivity}
\noindent In this subsection, we will analyze some important hyper-parameters of our method. We conduct sensitivity analysis on GOOD-WebKB dataset with concept shift, we chose two sensitive hyper-parameters (\ie, the coefficient $\gamma$ of condition mutual information in Equation~\ref{equ:cmi} and the number of prototypes $|C|$ in Equation~\ref{equ:pq}). The coefficient of CMI range in $[0.001, 0.01, 0.1, 0.5, 1]$ and the number of prototypes $|C|$ ranges in $[10, 50, 100, 200, 300]$. From Figure~\ref{fig:sensitivity}, we can observe that $\gamma$ reaches 0.1 and $|C|$ reaches 100 or 200 can achieve the best OOD test accuracy. Both higher and lower values of $\gamma$ result in suboptimal performance. This finding aligns with previous research such as DIB~\cite{dib}, indicating that an appropriate compression level is crucial for achieving optimal performance. Extremely high or low compression values are not ideal. 

Regarding the number of prototypes $|C|$, based on the results shown in Figure~\ref{fig:sensitivity}, it is found that setting $|C|=100$ leads to the best performance in terms of OOD test accuracy. This choice provides a moderate number of pseudo labels, which is beneficial for the learning process. 

Based on the sensitivity analysis, we determined that setting $\gamma=0.1$ and $|C|=100$ on most datasets. These hyperparameter values strike a balance between compression level and the number of prototypes, resulting in improved graph OOD generalization.
% Figure environment removed


\subsection{Integrated with Other Models}\label{sec:other_models}
% Figure environment removed

\begin{table}[htp]
\caption{Results of different learning approaches with different encoding models (\ie, GCN, GraphSAGE, GAT).}
\label{tab:others}
\centering
\scalebox{0.9}{
\begin{tabular}{cc|cc|cc}
\toprule
\toprule
\multirow{3}{*}{Model}& \multirow{3}{*}{Method} & \multicolumn{2}{c|}{GOOD-CBAS} & \multicolumn{2}{c}{GOOD-WebKB} \\
                & & \multicolumn{2}{c|}{color}    & \multicolumn{2}{c}{university} \\
                &   & ID          & OOD         & ID          & OOD            \\
\midrule
\multirow{3}{*}{GCN} 
&ERM               & 89.79±1.39 & 83.43±1.19  &  62.67±1.53 & 26.33±1.09         \\
&GRACE             & 92.00±1.39 & 88.64±0.67  &  64.00±3.43 & 34.86±3.43        \\
&MARIO             & 94.36±1.21 & 91.28±1.10  &  65.67±2.81 & 37.15±2.37        \\ \bottomrule
\multirow{3}{*}{SAGE} 
&ERM               & 95.07±1.51 & 75.14±1.19  & 73.67±2.08  & 46.33±3.42       \\
&GRACE             & 95.29±1.11 & 74.43±2.36  & 70.50±5.06  & 49.54±3.83        \\
&MARIO             & 96.00±1.07 & 76.29±3.01  & 71.00±3.82  & 51.74±4.63        \\ \bottomrule
\multirow{3}{*}{GAT} 
&ERM               & 78.64±3.63 & 72.93±2.64  & 61.33±3.71  & 28.99±2.63        \\
&GRACE             & 84.57±1.79 & 78.36±1.60  & 59.50±2.36  & 35.78±3.26        \\
&MARIO             & 84.93±1.95 & 80.43±1.89  & 62.17±4.78  & 38.17±3.10        \\
\bottomrule
\end{tabular}}
\end{table}



\noindent In the subsection, we demonstrate the model-agnostic nature of the recipe by integrating it with various graph neural network (GNN) models, including GCN, GraphSAGE, and GAT.

From Table~\ref{tab:others}, it can be observed that regardless of the specific GNN model used as the encoder, our method consistently achieves the best performance on the OOD test set. This indicates the effectiveness and robustness of our method across different GNN models.
By achieving superior performance across different GNN models, MARIO demonstrates its versatility and ability to improve the OOD generalization of various graph neural models. This highlights the broad applicability and effectiveness of our recipe in enhancing the performance of different GNN encoders.

Furthermore, we integrate our recipe with other GCL methods in Appendix~\ref{app:other_methods}. The results demonstrate our recipe can boost the OOD generalization ability of various GCL methods which means our recipe can serve as a plug-in for many current classical GCL methods.

% Figure environment removed

\subsection{Visualization}\label{sec:vis}
\subsubsection{Metric Score Curves}
We present metric score curves for ERM and MARIO, including training, ID validation, ID testing, OOD validation, and OOD testing accuracy, in Figure~\ref{fig:curve2}. Notably, MARIO demonstrates superior convergence with approximately 10\% absolute improvement on the OOD test set compared to ERM. Furthermore, MARIO effectively narrows the performance gap between in-distribution and out-of-distribution performance, showcasing its efficacy in enhancing OOD generalization for graph data. More metric score curves can be found in Appendix~\ref{app:curves}.


\subsubsection{Feature Visualization}
In order to assess the quality of learned embeddings, we adopt t-SNE~\cite{tsne} to visualize the node embedding on GOOD-Cora dataset (concept shift in word domain) using random-init of GCN, EERM, GRACE, and MARIO, where different classes have different colors in Figure~\ref{fig:vis}. For clarity, we select eight classes with the largest number of nodes to enhance the informativeness and interpretability of the visualization. We can observe that the 2D projection of node embeddings learned by MARIO has a better separation of clusters, which indicates the model can help learn representative features for downstream tasks. It has to note that we depict both ID nodes and OOD nodes in the same figure. 

Besides, we also separately visualize ID nodes and OOD nodes in the different figures in the Appendix~\ref{app:feature}. And we can find MARIO performs a clearer separation of clusters whether on ID nodes or OOD nodes compared to other methods.



\section{Related Work}
%\subsection{Cost Volume based Deep Stereo Matching}
%Stereo matching is a typical problem that has been studied for decades and a well-known four-step pipeline \cite{scharstein2002taxonomy} has been established, where cost volume construction is an indispensable step. Current state-of-the-art stereo matching methods are all cost volume based methods and they can be categorized into two types. Typically, a cost volume is a 4D tensor of height, width, disparity, and features. The first category just uses a full correlation to generate a single-feature cost volume. Such methods are usually efficient but lose much information because of the decimation of feature channels. Many previous work, including Dispnet \cite{dispnet}, MADNet \cite{madnet}, IResNet \cite{iresnet} and AANet \cite{aanet}, belong to this category. The second category usually uses concatenation \cite{gcnet} or group-wise correlation \cite{gwcnet} to generate a multi-feature 4D cost volume. Such a method can achieve better performance while requiring higher computational complexity and memory consumption. Actually, a majority of the top-performing networks in public leaderboards belong to this category, such as GANet \cite{ganet}, CSPN \cite{cspn} and ACFNet \cite{acfnet}. These methods generally employ multiple 3D convolution layers to constantly regularize the 4D cost volume and then apply softmax over the disparity dimension to produce a discrete disparity probability distribution. The final predicted disparity is obtained by softly weighting indices according to their probability, which is also called soft argmin in GCNet \cite{gcnet}. However, soft argmin leaves the output susceptible to multi-modal disparity probability distributions. ACFNet \cite{acfnet} observes this problem and proposes to directly supervise the cost volume with unimodal ground truth distributions. In contrast, we define an uncertainty estimation to quantify the degree to which the cost volume tends to be multi-modal distribution, higher implies the higher possibility of estimation error.

\subsection{Multi-scale Cost Volume based Stereo Matching}
Cost volume construction is an indispensable step in the well-known four-step pipeline for stereo matching \cite{scharstein2002taxonomy, pamisurvey1, pamisurvey2}. Typically, current state-of-the-art stereo matching methods can be categorized into two types of cost volume-based methods, where the cost volume is a 4D tensor of height, width, disparity, and features. The first category usually uses the single-feature 3D cost volume generated by full correlation, which is efficient while losing much information due to the decimation of feature channels. Many real-time methods, such as Dispnet \cite{dispnet}, MADNet \cite{madnet, madnet_pami} and AANet \cite{aanet}, belongs to the category. Moreover, two-stage refinement \cite{mcvmfc} and pyramidal towers \cite{madnet} are commonly applied in the single-feature cost volume based network to construct multi-scale cost volume. The second category usually uses the multi-feature 4D cost volume generated by concatenation \cite{gcnet} or group-wise correlation \cite{gwcnet}, which can achieve better performance with higher computational complexity and memory consumption. Most top-performing networks, including GANet \cite{ganet}, CSPN \cite{cspn} and ACFNet \cite{acfnet} belong to this category. 
% In these methods, the 4D cost volume is constantly regularized by multiple 3D convolution layers and then a discrete disparity probability distribution can be produced by softmax. Next, the final predicted disparity can be obtained by softly weighting indices according to their probability \cite{gcnet}. However, such output is susceptible to multimodal disparity probability distributions and ACFNet \cite{acfnet} gives a solution by directly supervising the cost volume with unimodal ground truth distributions to alleviate this problem. 
Recently, to alleviate the high computational complexity and memory consumption when employing multi-feature 4D cost volumes, \cite{cvpmvsnet, cascade, uscnet} propose to use cascade cost volume representation in multi-view stereo. These methods usually first predict an initial disparity at the coarsest resolution of the image and then gradually refine the disparity by narrowing down the disparity search space. More closely related to our approach is Casstereo \cite{cascade}, which first extended such representation to stereo matching. It selected to uniform sample a pre-defined range to generate the next stage’s disparity search range. Instead, we employ pixel-level uncertainty estimation to adaptively adjust the next stage disparity searching range and generate pseudo-labels for subsequent domain adaptation. Our method also shares similarities with UCSNet \cite{uscnet}, which constructs uncertainty-aware cost volume in multi-view stereo while it doesn’t employ uncertainty estimation to generate pseudo-labels.

%\subsection{Multi-scale Cost Volume based Deep Stereo Matching} 
% \subsection{Multi-scale Cost Volume based Stereo Matching} 
%Multi-scale cost volume firstly was applied in the single-feature cost volume based network with the form of two-stage refinement \cite{mcvmfc} and pyramidal towers \cite{madnet}. Recently, cascade cost volume representation \cite{cvpmvsnet, cascade, uscnet} was proposed in multi-view stereo to alleviate the high computational complexity and memory consumption when employing multi-feature 4D cost volumes. These methods generally predict an initial disparity at the coarsest resolution of the image. Then, they will narrow down the disparity search space and gradually refine the disparity. More closely related to our approach is Casstereo \cite{cascade}, which first extended such representation to stereo matching. It selected to uniform sample a pre-defined range to generate the next stage’s disparity search range. Instead, we employ uncertainty estimation to adaptively adjust the next stage pixel-level disparity searching range and push the next stage's cost volume to be predominantly unimodal.

% The single-feature cost volume based network with the form of two-stage refinement \cite{mcvmfc} and pyramidal towers \cite{madnet} first employ multi-scale cost volume for stereo matching. Recently, to alleviate the high computational complexity and memory consumption when employing multi-feature 4D cost volumes, \cite{cvpmvsnet, cascade, uscnet} propose to use cascade cost volume representation in multi-view stereo, which generally predict an initial disparity at the coarsest resolution of the image. Then, the disparity search space is narrowed down and the disparity is gradually refined. More closely related to our approach is Casstereo \cite{cascade}, which first extended such representation to stereo matching. It selected to uniform sample a pre-defined range to generate the next stage’s disparity search range. Instead, we employ uncertainty estimation to adaptively adjust the next stage pixel-level disparity searching range and push the next stage's cost volume to be predominantly unimodal.

% Figure environment removed

\subsection{Robust Stereo Matching} 
There exist three categories of generalization definitions for robust stereo matching. 1) Cross-domain Generalization: the network’s ability to perform well on unseen scenes (cannot see the image pairs of the target domain in advance). Towards this end, Jia et al \cite{sungeneralizaiton} propose to incorporate scene geometry priors into an end-to-end network. Zhang et al \cite{dsmnet} introduce a domain normalization and a trainable non-local graph-based filter to construct a domain-invariant stereo matching network. 2) Adapt Generalization: the network’s ability to adapt pre-trained models to the new domain with unlabeled target data. Previous work usually pre-trains the models on synthetic data and then adapts it to new target domains with Graph Laplacian regularization \cite{zoom}, non-adversarial progressive color transfer \cite{adastereo}, and Knowledge Reverse Distillation \cite{aohnet}. More closely related to our approach are \cite{aohnet, unsuperviseddomainadaptation} in stereo matching and Monoresmatch \cite{monoresmatch} in monocular depth estimation, which also proposes to generate a pseudo-label for domain adaptation. However, these methods all select to employ classical stereo matching methods \cite{sgm} alongside with confidence estimators, e.g., left-right consistency check to generate pseudo-labels. That is all these methods need an independent method to generate corresponding pseudo-labels. Instead, the proposed method is an end-to-end network that can generate the predicted disparity map, corresponding uncertainty map and pseudo-labels jointly, which is a more simple, yet efficient way. 
% Instead, our proposed method can employ pixel-level and area-level uncertainty estimation to self-distill the predicted disparity maps of our pre-training model and generate sparse while reliable pseudo-labels to align the domain gap, which is a more simple, yet efficient way. 
3) Joint Generalization: the network’s ability to perform well on a variety of datasets with the same model parameters. MCV-MFC \cite{mcvmfc} introduces a two-stage finetuning scheme to achieve a good trade-off between generalization and fitting capability on multiple datasets. However, it doesn’t touch the inner difference between diverse datasets, e.g, the unbalanced disparity distribution. To further address this problem, we propose a cascade cost volume to adaptively the next stage disparity searching space, where the pixel-level uncertainty estimation is at the core.

% \subsection{Monocular Depth Estimation}
% Monocular depth estimation aims to estimate depth values from a single image, instead of stereo images or multiple frames in a video. This problem is ill-posed because of the ambiguity of object sizes. However, humans could estimate the depth from a single image with prior knowledge of the scenes. Recently, learning based methods were explored to learn depth values by supervised or unsupervised learning. Eigen et al. first employed Convolutional Neural Networks (CNN) to predict depth in a coarse-to-fine manner and further improved its performance by multi-task learning. Liu et al. presented deep convolutional neural fields model by combining deep model with continuous CRF. Li et al. [22] refined deep CNN outputs with a hierarchical CRF. Multi-scale continuous CRF was formulated into a deep sequential network by Xu et al. [45] to refine depth estimation. Unsupervised methods tried to train monocular depth estimation with stereo
% image pairs or image sequences and test on single images. Garg et al. [9] used novel image view synthesis loss to train a depth estimation network in an unsupervised way. Godard et al. [11] introduced left-right consistency regularization to improve the performance of view synthesis loss. Recently, some work also propose to use the stereo matching network as a proxy to learn depth from synthetic data or directly employ traditional stereo matching methods to distill proxies labels from the target domain, which proves the feasibility of distilling stereo matching networks to learn monocular depth estimation.



\section{Conclusion and Future Work}
In this work, I design corruption-robust algorithms for the Lipschitz contextual search problem. I present the \emph{agnostic checking} technique and demonstrate its effectiveness in designing corruption-robust algorithms. There are several open problems for future research. First, in the algorithm I propose for pricing loss, the schedule for agnostic checks is fixed upfront. Can the learner design an adaptive checking schedule for the pricing loss? Second, this work assumes the learner has knowledge of the Lipschitz constant $L$. Can the learner design efficient no-regret algorithms without knowledge of $L$? 


\subsubsection*{Acknowledgements}
We thank Noga H. Rotman for valuable discussions and advice. This work was supported by the Israel Science Foundation and the Center for Interdisciplinary Research at Hebrew University.

\newpage
\medskip
{
\small
\bibliography{ranking_arxiv}
}


\newpage
\section*{Appendix}
\appendix
\section{PFI Variance Analysis}\label{app:pfi_var}

In Section \ref{sec:framework}, we present two options for defining the base FI values for PFI \citep{breiman2001random}: (1) a single permutation, and (2) a single observation. We obtain the same global FI values from the two base definitions by setting the values of the number of permutations ($B$) and the size of the explanation set ($N$) accordingly. However, the different decomposition of the global FI value to base FI values allows for the analysis of various sources of uncertainty -- the variance of the permutations and the variance of the explanation set. Our framework is limited to quantifying only one source of uncertainty by aggregating base FI values to global FI values. Therefore, using it might be a problem when the uncertainty of the global FI values stems from multiple sources of uncertainty. Nevertheless, if most of the variability comes from one of the sources, it is reasonable to target it and disregard the other sources. In the case of PFI, we expect that the size of the explanation set introduces greater variance than the number of permutations. Our results clearly show this; therefore, we can use our framework to quantify the uncertainty of global PFI values.

\subsection{Experimental Setup}

In this experiment, we use the same DGPs (A and B) described in Section \ref{sec:shap_simulated}, including the definition of $X$, $Y$, and the functions.

\subsubsection{Dummy Prediction Model} 

Instead of training a model, we create a \emph{Dummy} model that predicts $Y$ from $X$ using the DGP's function. We use this approach to control the variability stemming from the training and focus on the variability of the permutations and explanation set.

\subsubsection{Experiment Details}

We sample the data as described above with various configurations of $B$, $p$, and $N$, and the two functions. For each configuration:
\begin{enumerate}[label=(\arabic*)]
    \item We perform $B$ permutations for each observation and calculate the loss difference for each permutation $b$: $L(f(x_{[j]}^b), y) - L(f(x), y)$.
    \item We average all of the permutations for each observation.
    \item We average all of the observations.
\end{enumerate}
The result of steps 1-3 is a set of $p$ global FI values $\imp_1^{PFI}, \ldots, \imp_p^{PFI}$. We repeat this process 100 times and calculate the average and standard deviation (SD) across the repetitions.

\subsection{Results}

For both functions we compare the SD of the global FI values for different values of $B$ and $N$. In Figure \ref{fig:pfi_var_sources}, we can see that different features have different SDs, but in all conditions the SD is almost fixed with respect to $B$ and decreases with $N$. This indicates that the number of observations introduces more variability to the global FI values than the number of permutations.

% Figure environment removed




\newpage
\section{Definitions and Proofs}\label{app:def_proof}

\subsection{Proof of Theorem \ref{thm}}
\label{app:proof}

In this section, we present the detailed proof of Theorem \ref{thm}.

Recall that $D$ is the set of partial rankings and that we assume that the probability of any error in $D$ is less than $\alpha$.
To prove the theorem, we first show that any error in coverage, i.e., a CI that does not cover the true rank, must be caused by at least one partial ranking error in $D$:

Suppose that there is a coverage error. Without loss of generality, assume that the coverage error occurs for feature 1: 
$$ [l_1, u_1] \nsubseteq [L_1,U_1].$$
The coverage error can be in one or both bounds:
\begin{enumerate}[label=(\arabic*)]
    \item $l_1 < L_1$,
    \item $u_1 > U_1$.
\end{enumerate}

If (1), then $L_1 > 1$, and there are $L_1-1>0$ pairs of the type $(1,k) \in D$, meaning that there are $L_1-1$ features with a significantly lower observed global FI than the observed global FI of feature 1.
However, according to Definition \ref{def:setrank}, $l_1-1 = \# \{k: \trueimp_1 > \trueimp_k\}$, meaning that there are only $l_1-1$ features with true global FI lower than the true global FI of feature 1.
Combining these two statements together, there must be at least one feature $k = 2,...,p$ for which 
$(1,k) \in D$ but $\trueimp_1 \ngtr \trueimp_k$, meaning that there is a partial ranking error in $D$.

If (2), then the set $\{k \in 2,...,p : (k,1) \in D\}$ is higher than the set $\{k: \trueimp_k>\trueimp_1\}$. Again, this would mean that for at least one value of $k$ there is a partial ranking error.

The event of at least one coverage error is contained in the event of obtaining a partial ranking error. Given that, the coverage error probability is bounded by $FWER = \alpha$.

\subsection{FWER Adjustment Procedures}\label{app:fwer}

Here, we provide details on the two sequential procedures that we use in our implementation. After adjustments, the p-values are compared to a chosen $\alpha$ level. Note that all p-values are inflated compared to their original level, making it less likely that the null hypothesis will be rejected. Furthermore, the p-values keep their relative order after adjustment. In the procedures below, this is governed by the $\max$ function, which assures that the order is maintained. The resulting process is sequential in that for a given level $\alpha$, after the first non-rejected value, all others would not be rejected. Let $p_1,...,p_K$ be a set of K p-values obtained by testing a family on null hypotheses $H_1^0, \ldots, H_0^K$; below we demonstrate how the two FWER adjustment procedures are used to calculate $p_1^{adg},...,p_K^{adg}$, a set of adjusted p-values.

\subsubsection{Holm's Procedure} 

We implement Holm's procedure \citep{holm1979simple} on one-sided hypothesis tests. The paired t-test is calibrated with this procedure for normally distributed base FI values.

Let $p_{(1)}\leq,...,\leq p_{(K)}\leq 1$ be the sorted set of p-values. Then:
\begin{align*}
    &p^{adj}_{(1)} = K\cdot p_{(1)}, \\
    &p^{adj}_{(2)} = \max\{p^{adj}_{(1)}, (K-1)p_{{(2)}}\}, \\
    &\ldots, \\
    &p^{adj}_{(k)} = \max\{p^{adj}_{(1)},...,p^{adj}_{(k-1)}, (K - (k-1))p_{(k)}\}, \\
    &\ldots, \\
    &p^{adj}_{(K)} = \max\{p^{adj}_{(1)},...,p^{adj}_{(K-1)}, p_{(K)}\}.
\end{align*}

\subsubsection{Min-P Procedure}

Holm's procedure is highly conservative, since it is valid regardless of the structure of dependence between the p-values. To improve it, \cite{westfall1993resampling} suggested the Min-P procedure. The idea is to use bootstrapping to model the structure of dependencies between p-values, obtain lower adjusted p-values, and reject more hypotheses. The details of the Min-P procedure described here are taken from \cite{efron2012large}. 

Here, we also start with the sorted set of p-values $p_{(1)}\leq,...,\leq p_{(K)}\leq 1$. Let $i_1, \ldots i_K$ indicate the corresponding original indices, $p_{(k)} = p_{i_k}$, and define $I_k=\{ i_k, i_{k+1}, \ldots, i_K \}$ and $\pi_k=\mathcal{P}_0 \{ \min_{j \in I_k} (P_j) \leq p_{(k)}\}$. Here, $(P_1, \ldots, P_K)$ indicates a hypothetical realization of the unordered p-values $p_1,...,p_K$ obtained under the complete null hypothesis, meaning all $H_k^0$s are true. 
The adjusted p-values are then defined by:
\begin{align*}
    &p^{adj}_j = \max_{k \leq j} \pi_k.
\end{align*}




\newpage
\section{Confident Feature Ranking: Step-By-Step}\label{app:full_example}

This section demonstrates how our ranking method constructs simultaneous CIs for the true ranks. We use the bike sharing dataset \citep{fanaee2014event}, the TreeSHAP FI method, and our ranking method with Holm's procedure for $n=50$ base FI values. This demonstration is a detailed description of the example presented in Section \ref{sec:real} (Figure \ref{fig:bike_base_size}).

\paragraph{Base and Global FI Values} We construct a TreeSHAP explainer \citep{lundberg2019Tree} based on the trained XGB model. The base FI values are the absolute values of the local SHAP values the explainer produces for an explanation set of size $n=50$. The global FI values are the average of the base FI values. The values and the order of the global base FI values for this explanation set are presented in Table \ref{tab:full_example}.

\paragraph{Pairwise Differences} The paired-sample t-test is based on the differences between the base FI values of two features $\basematrix_j - \basematrix_k$. The one-sided hypothesis $H_{jk}^0: \trueimp_j \geq \trueimp_k$ is rejected if the difference between the observed global FI values is significantly different from zero. In Figure \ref{fig:pair_diff}, we present the differences between \emph{Workingday} and all other features. The average of the differences between \emph{Workingday} and \emph{Month} and \emph{Temp} is near zero.


\paragraph{Partial Rankings} We set the significance level to $\alpha=0.1$, and for each pair of features, we run two paired one-sided t-tests; then, we adjust the p-values to multiple comparisons using Holm's procedure. In Figure \ref{fig:signs}, gray and black indicate that the observed global FI value of the feature in row $j$ is respectively less and greater than the observed global FI value of the feature in column $k$. White indicates that the difference is zero (neither $H_{jk}^0$  nor $H_{kj}^0$ were rejected). The set of partial rankings $\paranking$ is then obtained. For example, we can conclude that $(Month, Year) \in \paranking$, $(Day, Month) \in \paranking$, and $(Month, Workingday) \not\in \paranking$.


\paragraph{Constructing Simultaneous CIs for the True Ranks}
For each feature, we initialize the lower bound of the CI to one and the upper bound to $p$.
If there are no differences between the features, the CIs for all features are $[1, p]$. Otherwise, there are differences. Without loss of generality, consider the \emph{Workingday} feature. By looking at the row for \emph{Workingday} in Figure \ref{fig:signs}, we can see that the observed global FI value of \emph{Workingday} is significantly higher than the observed global FI values of \emph{Day} and \emph{Weather}, and it is significantly lower than the observed global FI values of \emph{Year} and \emph{Hour}. There is no significant difference between \emph{Workingday} and \emph{Month} and \emph{Temp}. Therefore, we increase the lower bound by two, decrease the upper bound by two, and obtain the confidence set $[3, 5]$ for the true rank of \emph{Workingday}.


We repeat the same process for all features and obtain $90\%$ simultaneous CIs for the true ranks. See lower and upper bounds in Table \ref{tab:full_example} and a visualization of the CIs in Figure \ref{fig:full_example_rank}.

% Figure environment removed


\begin{table}[ht]
\centering
\caption{Ranks and Simultaneous CIs}
\label{tab:full_example}
\begin{tabular}{|l|c|c|c|}
\hline
\textbf{Feature} & \multicolumn{1}{l|}{\textbf{Observed Global FI}} & \multicolumn{1}{l|}{\textbf{Observed Rank}} & \multicolumn{1}{l|}{\textbf{CI}} \\ \hline
Hour             & 129.042                                          & 7                                           & [7, 7]                           \\ \hline
Year             & 44.805                                           & 6                                           & [6, 6]                           \\ \hline
Temp             & 33.777                                           & 5                                           & [4, 5]                           \\ \hline
Workingday       & 28.95                                            & 4                                           & [3, 5]                           \\ \hline
Month            & 23.987                                           & 3                                           & [3, 4]                           \\ \hline
Weather          & 10.865                                           & 2                                           & [2, 2]                           \\ \hline
Day              & 5.673                                            & 1                                           & [1, 1]                           \\ \hline
\end{tabular}
\end{table}



\newpage
\section{Experiments Details and Results}\label{app:exp}

\subsection{Synthetic data}\label{app:sim_imp}

\subsubsection{Feature ranking - mock importance values}

\paragraph{Data description}
For this experiment, we generate multivariate-normal data with off-diagonal correlations between the features and multiple levels of correlation $\rho \in [0.1, 0.3, 0.6, 0.9]$. The variance of the features was sampled from chi-squared distribution and was scaled by different factors: 0.2, 1, 5. The structure of the means vector is: $\mu=(1, \ldots, p+1)^{\beta}$, where $\beta$ gets different values that affects the spread of the means: 0.5 (most spread), 0.25, 0.1 (most compact). In general, the ranking maintains simultaneous coverage levels of almost 1, and the efficiency worsening as the variance of the features increases.


Fig. \ref{fig:p_mu_exp} is a detailed version of Fig. \ref{fig:mock_p}, separated by the different values of the exponent.

% Figure environment removed


In addition, we analyzed the efficiency as a function of the number of observations ($n$), see summary in Fig. \ref{fig:mock_n} and the breakdown to different values of $\mu$ exponent in Fig. \ref{fig:n_mu_exp}. To evaluate the efficiency and simultaneous coverage of the ranking methods, we sampled 100 independent explanation sets for each configuration and reported the average across repetitions.

% Figure environment removed

% Figure environment removed



\subsubsection{Feature ranking - TreeSHAP and PFI importance values}\label{app:act_imp}

Here we simulate a regression task with $X$ sampled from the following synthetic data: a multivariate normal distribution, with $\mu=(1, \ldots, p+1)^{0.5}$, a block-wise pairs correlation matrix with $\rho=0.3$, and a chi-squared variance. In addition, we analyzed a different number of total features ($p=15$ and $p=50$) with varying numbers of important features.
The value of $Y$ was calculated by combining the important features linearly and then applying one of (a) adding interactions or (b) transforming the linear combination with a trigonometric function. Finally, we added noise to $Y$ and standardized it.

For each configuration of $(X, Y)$, an XGBoost model was trained and tested (70/30) on 500,000 observations. Then, the true importance values were computed with TreeSHAP and PFI using 50,000 observations for $p=50$ and 10,000 for $p=15$.

To measure our ranking method as a function of the number of base importance values ($n$), we repeatedly (100 times) sample data from the same distribution for each value of $n$, compute the local TreeSHAP and PFI values, and ranked the features with our method. Then, we measure the efficiency and coverage. The reported results are the average across repetitions.

In Fig. \ref{fig:shap_pfi_simulated} we presented the results for a linear $Y$ with interactions. In Fig. \ref{fig:shap_pfi_trigo} you can see the results for the trigonometric $Y$.

% Figure environment removed


\subsection{Real data}\label{app:real_data}

\paragraph{Bike sharing \cite{fanaee2014event}} The data contains 10886 rental bike counts between 2011 and 2012 from the Capital Bikeshare program in Washington, D.C. The regression task is forecasting demand for bike rentals based on time and environmental measures such as month and weather. We created a 60/40 train and test sets and trained an XGBoost regression model (scikit-learn API for XGBoost regression with default hyperparameters; train and test $R^2$ of 0.98 and 0.94, respectively).

\paragraph{COMPAS \cite{angwin2016machine}} The data contains 6172 records of criminal history from Broward County from 2013 and 2014. The classification task is assess a criminal defendant’s likelihood to re-offend based on jail and prison time, and demographics. We created a 60/40 train and test sets and trained an RF classification model (scikit-learn API for XGBoost random forest classification with default hyperparameters; train and test mean accuracy of 0.87 and 0.85, respectively).
\newpage
%-------------------------------------------------------------------------------
\section{Implementation} \label{imple}
%-------------------------------------------------------------------------------

% Figure environment removed

\sys's implementation consists of (\romannumerber{1}) a fully-functional \sys switch prototype which implements the overall in-fabric logic; and (\romannumerber{2}) a set of software APIs exposed to applications. Our prototype is built upon a FPGA-assisted commodity switch while a P4 programmable switch implementation is also provided.

%\sys's implementation consists of (\romannumerber{1}) a FPGA-based fully-functional prototype which implements the self-defined switch logic and (\romannumerber{2}) a set of software APIs exposed to multicast applications. \sys's switch logic can also be implemented on the Tofino P4 switch, as described in $\S$\ref{dis}.

\parab{FPGA-based prototype.} We implement the group registration, data packet duplication, header modification, and feedback aggregation logics on an FPGA board. The board is equipped with a commodity FPGA chip~\cite{ultrascale} and four 100Gbps Ethernet interfaces. The FPGA resource utilization is shown in Table~\ref{tab:overhead}. We build our testbed with the FPGA board, a commodity Ethernet switch, and four servers, as illustrated in Fig.~\ref{fig:fpgaprototype}. Each server is equipped with a commodity RNIC. The FPGA board and four RNICs are connected to the commodity switch through 100Gbps Ethernet interfaces. 

The commodity switch is configured by Access Control List (ACL) to route the servers' multicast traffic to the FPGA board. The FPGA board identifies the multicast data (ACK\footnote{In Fig.~\ref{fig:fpgaprototype}, we use ACK to represent all types of feedback.}) packets through the specific packet header by \textit{Parser} and \textit{Arbiter}. The data (ACK) packets will be duplicated (aggregated) by \textit{Duplicator} (\textit{ACK Aggregator}). The resulting packets will be pushed in \textit{Queue System}, waiting for the \textit{Multiplexer} to schedule in case for queue competition. Finally, the duplicated data (aggregated ACK) packets are sent back to the commodity switch. During processing, the \textit{Multicast Forwarding Table} is accessed when needed. %\todo{describe Fig. 8} Note those resulting packets' destination IP would be unicast IPs, so the switch routes them as normal unicast packets.
%

\parab{P4-based implementation.}
\sys in-fabric logic can be implemented on the P4 switch as well with some special handling. For the one-to-many data forwarding, P4 switch duplicates packets at the Traffic Manager (TM). The extended table states in Fig.~\ref{fig:table} are stored in the egress pipeline, indexed by <GroupID, EgressPort>. Because the lookup key in P4 switch is at most 32bits, we can use the least significant 24bits of GroupIP plus the 8bits port number as the real index. 

For the many-to-one feedback aggregation, there are two challenges due to the limited capability of P4 switch. Commodity P4 switch switch contains many stages, each with minimal computation capability and independent memory. Firstly, a single stage cannot support the \textit{wrapped-around} PSN comparison. To handle it, we simplify the standard PSN comparison to match the stage's capability, resulting in a tighter PSN space reduced from $2^{23}$ to $2^{22}$. Secondly, a single stage cannot iterate the entire table entries and find the minimum PSN. To handle it, we leverage multiple stages, each responsible for partial entries. Thus, the maximum number of entries supported in each multicast group is limited by the total stage number.

Besides, the P4 switch lacks the computation capability to recalculate the Invariant Cyclic Redundancy Checksum (ICRC) for the modified (aggregated) data (ACK) packets. As a result, those packets would violate the ICRC validation and be discarded at the receiver. This is why we choose the FPGA-based prototype to evaluate \sys in this work. However, a recent work shows that some RNICs provide the ability to bypass ICRC validation~\cite{switchML}. 
%Although this can work, it's a compromised method having security risks. 
%So we select the more integrated FPGA-based prototype to evaluate in $\S$\ref{eva}. Other operators can choose their preferred implementation based on their requirements.

%\begin{algorithm}[t]
%	\caption{Update PSN record and find PSN minimum in P4}\label{alg:psncomp}
%	\begin{algorithmic}[1]
%		%\Function{Generation}{}
%		\State $ack.psn, ack.port\gets $ the PSN and port of ACK packet
%		\State $rec.psn, rec.port\gets$ the PSN and port recorded
%		\State $isTrigger\gets$ whether the packet is a trigger packet
%		%\State $last\_ack\_psn\gets$ last aggregated ACK's psn
%		%\State $min\_port\gets$ port with minimum $ack\_psn$ last time
%		\State min.psn = ack.psn;
%		\State \textcolor{purple}{// every stage compare the PSN}
%		\If{$ack.psn > rec.psn$ or $(ack.psn <= 24'b3fffff$ and $rec.psn >= 24'b600000)$}
%			\State min.psn = rec.psn;
%			\If{rec.port == ack.port}
%				\State rec.psn = ack.psn;
%			\EndIf
%		\EndIf
%		\State \textcolor{purple}{// last stage write the min PSN back}
%		\If{isTrigger} 
%			\State ack.psn = min.psn;
%			\State Forward ACK.
%		\EndIf
%	\end{algorithmic}
%\end{algorithm}

\parab{Software APIs.} We provide various communication libraries and middleboxes for \sys multicast support. Take the commonly-used OpenMPI as an example, we modify the OpenMPI (v4.1.1)~\cite{openmpi} and UCX (v2.3)~\cite{ucx} to adapt to \sys's design, as shown in Fig.~\ref{fig:fpgaprototype}. Specifically, we add a new implementation of $MPI\_Bcast$ and modify UCX for multicast QPs creation and data transmission. When the new $MPI\_Bcast$ is called, the MPI process calls the UCX to establish QPs for multicast. Multicast members exchange their QPs information, and the handshake starts, as described in Appendix \ref{apx:regis}. Once the multicast group is successfully established, the UCX finally calls the RDMA primitives defined in the well-known \textit{libibverbs}~\cite{libibverbs} to transmit data. The software modifications at the end-host are transparent to the upper-layer applications and don't require any RNIC or RDMA driver modification.
%\parab{Coalescence of unicast and multicast}
%When we design \sys, there is a question in our mind: \textit{which is better, maintaining unicast and multicast transports separately at end-host, or utilizing the in-network support to enabling them to match the same transport?} Because of the long-standing resource limit in RNICs and the emeging trend of shifting appropriate computation task to programmability network, we believe the latter is the correct selection.

\begin{table}[t]
	\small
    \centering
%	\begin{center}
%    \begin{tabular}{l|c|c|c}
    \begin{tabular}{|p{0.2\linewidth}|p{0.18\linewidth}|p{0.18\linewidth}|p{0.18\linewidth}|}
    \hline
    \textbf{Resource} & \hfil \textbf{LUT} & \hfil \textbf{Register} & \hfil \textbf{BRAM} \\
    \hline
   	\textbf{Usage} & \hfil 53169 & \hfil 15391 & \hfil 188 \\
    \hline
    \end{tabular}
%    \end{center}
    \caption{Resource usage of the \sys in-fabric logic.}
    \label{tab:overhead}
    \vspace{-0.25cm}
\end{table}

%\parab{Resource overhead}  Note that the size of multicast forwarding table is determined by the number of ports of the switch and doesn't scale up with the multicast group size. 2.7MB memory can support upto 1K multicast groups, which is satisfied in datacenter. We provide a detailed calculation of the maxmum group support in Appendix \ref{apx:cal}. 

\end{document}
