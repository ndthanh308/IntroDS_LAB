\section{Related work}\label{sec:related}

\subsection{Uncertainty of post-hoc FI methods}
Recent studies have shown that FI methods are unstable \cite{lakkaraju2020robust, mishra2021survey, slack2021reliable, agarwal2022rethinking}. In particular, researchers have focused on post-hoc FI methods and found that their output is sensitive to input perturbations, imputation or randomness of sampling \cite{covert2020SAGE, slack2021reliable, ishwaran2019standard}, and selection of hyperparameters, such as the choice of reference distribution for Shapley-values-based FI methods \cite{merrick2020explanation}. The standard approach suggested by most studies is to estimate the variance of the FI values and construct confidence intervals for the values.


\subsection{Uncertainty in feature ranking}
The order of the features by their importance for the model's prediction is referred to as feature ranking. Finding the ``right" order or the perfect subset of features is frequently unfeasible because it requires an examination of all possible subsets of features. Therefore, the ranking obtained by the FI methods is an uncertain and sub-optimal approximation. Many studies have suggested overcoming this limitation and getting a more stable ranking by applying a two steps procedure; generating multiple rankings from different FI methods or subsampling, and then combining the different outputs into a single ranking \cite{saeys2008robust, prati2012combining, vettoretti2021variable, alaiz2020information, salman2022stability}. Some of the mentioned techniques are based on a pairwise comparison between the ranks of two features but do not consider the problem of multiple comparisons.


\subsection{Confident ranking and selection}
The problem of ranking and selection (R \& S) is well studied in statistics, and solutions are applied in multiple domains \cite{gupta1965some, boesel2003using}. Current works in this field often deal with choosing parameters in computer experiments and include dynamic choices of sample sizes to control the super-set size and confidence. Most similar to our work are \cite{zhang2014confidence, klein2020joint, al2022simultaneous, rising2021uncertainty} for ranking the observed means, and \cite{garcia2008extension, wright2014ranking, eckman2020revisiting} with multiple comparisons for ranking populations and subset selection.


















































