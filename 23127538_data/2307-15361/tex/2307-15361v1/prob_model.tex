\section{Probability model and target of inference}\label{sec:prob}

Here we develop a probability model for the uncertainty in ranks and identify the parameters of interest in the rank-sets. We then formalize our goal in defining a simultaneous confidence interval set for the rank-sets.

\subsection{True global FI and rank-set}


We model the rows $v_i$ as independent samples from distribution $F_v$ with mean vector $E[v_i] = (\trueimp_1,...,\trueimp_p)$, the \emph{true global FI values}.
For finite $n$, $\imp_j$ would be an unbiased but noisy version of $\trueimp_j$; we are interested in understanding the effects of this deviation on the possible feature rankings.


%\subsubsection{True ranking}

In contrast to the observed noisy rankings, the \emph{true ranking} $r_1,...,r_p$ are based on the true FI values $\trueimp_1,...,\trueimp_p$. 
Whereas in observed data exact ties are unlikely, 
for the true global FI we can imagine ties between equivalent features, or we may want to allow an indifference level.
If there are no ties, the lower and upper ranks are identical and equal to the standard definition.

We follow \cite{al2022simultaneous} in redefining the true ranks to account for ties:
\begin{definition}(Rank-set)
\label{def:setrank}
    Define the lower-rank of $\trueimp_j$ by $
        l_j = 1 + \#\{k: \trueimp_j > \trueimp_k, j \neq k\}
    $
    and the upper-rank of $\trueimp_j$ by  $
        u_j = p - \#\{k: \trueimp_j < \trueimp_k, j \neq k\} $.    Then the rank-set of $\trueimp_j$ is the set of natural numbers $\{l_j, l_{j+1}, \ldots, u_j\}$ denoted here $[l_j, u_j]$.
\end{definition}



\subsection{Confidence intervals of ranks}

Our inferential goal is to construct a confidence interval of ranks, estimated from the data, for each feature's importance such that with high probability, all intervals cover the true rank-sets.

\begin{definition}(Simultaneous coverage) The set of intervals $[L_1,U_1] , \ldots, [L_p, U_p] \subseteq [1, p]$ has simultaneous coverage at level $1 - \alpha$
if 
\begin{align*}
    \mathbb{P}\left([l_j, u_j] \subseteq [L_j, U_j] , \quad \forall j \in \{1, \ldots, p\}\right) \geq 1 - \alpha.
    \end{align*}
\end{definition}

$L_j, U_j$ are functions of $\mathbf{v}$ the observed base FI.
Different sets of observed base FI values would produce different confidence intervals. In a simultaneous $1-\alpha$ set of confidence intervals, the probability that all intervals in the set cover the true rank-sets is at least $1-\alpha$.
Note that simultaneous coverage is conservative and can result in relatively large intervals. 

An advantage of simultaneous coverage requirement is that coverage is maintained for many functions of the observed ranks across features. In particular, simultaneous intervals are valid after any selection of the parameters. 
The confidence interval of the most-important feature may not be valid in standard confidence intervals; but it is valid if the intervals are simultaneous. Extending this, we can select the top-k sets: instead of selecting the most important feature (or most important k features) based on the observed data, we may prefer selecting all features that could be ranked in the top-k as $n$, the number of base FI values, increases based on the confidence intervals. With simultaneous coverage, the error probability for this selection is controlled \cite{hsu1996multiple}. Furthermore, the confidence intervals for the features currently ranked top-k still have marginal coverage. These two properties are not guaranteed by the usual marginal coverage \cite{benjamini2005false}.


\subsection{Top k super-set}
FI values are commonly used to select the most important features for subsequent analysis, such as interventions %\cite{} 
or interpretation. %\cite{}. 
When the observed FI values are uncertain, it may be preferable to analyze a super-set that includes all features which can potentially be in the top $k$ set.

Define the true $Top_k \subseteq [p]$ as the set of features whose true FI is ranked in the top $k$, minding ties; e.g., $Top_k = \{j: u_k \geq p-k+1 \}$. Then, given a simultaneous confidence intervals, it is straightforward to derive a set covering the $Top_k$ with high confidence:

\begin{lemma}
Let $\{ [L_1, U_1],\ldots,[L_p,U_p] \}$ be a $1-\alpha$ simultaneous confidence intervals for the rank-sets. Define a set $\widehat{Top}_k  = \{j: U_j \geq p-k+1 \}$. This set includes all features with an upper bound in the top $k$ ranks. Then 
\begin{equation}
   P( Top_k\subseteq \widehat{Top}_k) \geq 1-\alpha.
\label{eq:topk}
\end{equation}
\end{lemma}
To prove this, consider the case where the event in 
\ref{eq:topk} does not hold. Then there exists some $j\in \widehat{Top}_k$ that is not in $Top_k$, meaning that the estimated upper bound $U_j$ exceeds the true upper rank $u_j$, so the confidence interval $[L_j,U_j]$ does not cover $[l_j, u_j]$. From the definition of $1-\alpha$ simultaneous confidence intervals, the probability of any such event is at most $\alpha.$