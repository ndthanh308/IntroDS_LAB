\documentclass[usenatbib,useAMS]{mnras}
%\documentclass[usenatbib,useAMS,letters]{mnras}
%\documentclass[usenatbib,useAMS,referee]{mn2e}
 
\usepackage{natbib}
\usepackage{amsmath}
\usepackage{graphicx}
\usepackage{color}
\usepackage{mathrsfs}
\usepackage{epsfig}
\usepackage{comment}
\usepackage{ulem}

% 
%\bibliographystyle{mnras}
%\allowdisplaybreaks[1]

%\usepackage{setspace}
%\setstretch{2.5}
%\pdfoutput=1
%\documentclass[fleqn,usenatbib,usedcolumn]{mnras}
%\usepackage{amsmath}
%\usepackage[british]{babel}
%\usepackage{txfonts}
%\let\la=\lesssim

%\usepackage{ae,aecompl}

\usepackage{graphicx}
\usepackage[hang,small,bf]{caption}
\usepackage[subrefformat=parens]{subcaption}
\captionsetup{compatibility=false}
\usepackage{amssymb}
\usepackage[T1]{fontenc}

\setlength\topmargin{-2pc}
\newcommand{\mBH}{M_{\rm BH}}
\newcommand{\msun}{{M}_{\odot}}
\newcommand{\BHNS}{{BH1st--NS2nd}}
\newcommand{\NSBH}{{NS1st--BH2nd}}
\newcommand{\Mc}{M_{\rm chirp}}
\newcommand{\Mt}{M_{\rm total}}
\newcommand{\rsun}{{R}_\odot}
\newcommand{\zsun}{Z_\odot}
\newcommand{\lsun}{{\rm L}_\odot}
\newcommand{\cc}{{\rm cm}^{-3}}
\newcommand{\mdot}{\dot m}
\newcommand{\msunyr}{{\rm M}_\odot~{\rm yr}^{-1}}
\newcommand{\kpc}{{\rm kpc}}
\newcommand{\mpc}{{\rm Mpc}}
\newcommand{\gpc}{{\rm Gpc}}

\newcommand{\yr}{\rm yr}
\newcommand{\Hmol}{\rm H_2}
\newcommand{\K}{{\rm K}}
\newcommand{\beq}{\begin{equation}}
\newcommand{\eeq}{\end{equation}}

\newcommand{\red}[1]{\textcolor{black}{ #1}}
\newcommand{\blue}[1]{\textcolor{blue}{ #1}}
\newcommand{\cl}[1]{\textcolor{red}{ #1}}

\newcommand{\kenta}[1]{\textcolor{cyan}{#1}}
\newcommand{\delkh}[1]{{\color{red}  \sout {#1}}}
\newcommand{\delt}[1]{{\color{red}  \sout {#1}}}

\newcommand{\hn}[1]{{\bf \textcolor{red}{#1}}}
\newcommand{\tn}[1]{{\bf \textcolor{red}{#1}}}
\newcommand{\tk}[1]{{\bf \textcolor{red}{#1}}}

\defcitealias{K14}{K14}
\defcitealias{K16}{K16}
\defcitealias{Visbal_2015}{VHB15}

\usepackage{url} %It provides better support for handling and breaking URLs.

\voffset=-0.4in
%
%
\title[Mass Ratio of BBHs Determined from LIGO/Virgo Data]
{Mass Ratio of Binary Black Holes Determined from LIGO/Virgo Data Restricted to Small False Alarm Rate}

\author[T. Kinugawa, T. Nakamura, and H. Nakano]
{
Tomoya Kinugawa$^{(1)}$\thanks{E-mail: kinugawa@shinshu-u.ac.jp}, Takashi Nakamura$^{(2)}$, and Hiroyuki Nakano$^{(3)}$ \\
\\
$^{1}$Faculty of Engineering, Shinshu University, Nagano, 380-8553, Japan\\
$^{2}$Department of Physics, Graduate School of Science, Kyoto University,
Kyoto 606-8502, Japan\\
$^{3}$Faculty of Law, Ryukoku University, Kyoto 612-8577, Japan
}

\begin{document}

\date{\today}
\maketitle
\begin{abstract}
Binary black-hole mergers up to the third observing run with the minimum false alarm rate smaller than $10^{-5}\,{\rm yr}^{-1}$ tell us that the mass ratio of two black holes follows $m_2/m_1=0.723$ with the chance probability of 0.00301\% for $\Mc > 18 M_{\odot}$ where $\Mc$ ($= (m_1m_2)^{3/5}/(m_1+m_2)^{1/5}$) is called the chirp mass of binary with masses $m_1$ and $m_2$ ($ < m_1$). We show that the relation of $m_2/m_1=0.723$ is consistent if the binaries consist of population III stars which are the first stars in the universe. On the other hand, it is found for $\Mc < 18 M_{\odot}$ that the mass ratio follows $m_2/m_1=0.601$ with the chance probability of 0.117\% if we ignore GW190412 with $m_2/m_1\sim 0.32$. This suggests the different origin from that for $\Mc > 18 M_{\odot}$.
\end{abstract}

\begin{keywords}
stars: population III, binaries: general relativity, gravitational waves, black hole mergers
\end{keywords} 


%%%%%%%%%%%%%%%%%%%%%%%%%%%%%%%%%%%%%%%
\section{Introduction and data with small false alarm rate}
%%%%%%%%%%%%%%%%%%%%%%%%%%%%%%%%%%%%%%%

Possible origin of massive binary black holes (BBHs) with total mass $\sim 65\msun$ like GW150914 which is the world's first observation of gravitational waves (GWs), is Population (Pop) III stars. 
Theoretically, GW events like GW150914 were predicted by~\cite{Kinugawa2014} before the discovery of GW150914 by \cite{Abbott_PRL_2016}. 
After that, various compact object binaries have been observed by LIGO/Virgo GW detectors.

Many of them are BBHs (see, e.g., the third Gravitational-wave Transient Catalog (GWTC-3) of~\cite{2021arXiv211103606T}).
The GW events with ${\rm \bf FAR}_{\rm min} < 1 \times 10^{-5}\,{\rm yr}^{-1}$ in~\cite{2021arXiv211103634T} are summarized in Table~\ref{tab:events}~\footnote{We should note that because there are some updates during this work, we use the latest ones from the online GWTC~\citep{online_GWTC}.}.
Here, ${\rm \bf FAR}_{\rm min}$ means the minimum false alarm rate (FAR) evaluated by various pipelines used in the GW data analysis.
The reason for this restriction of events having the low value of FAR is to treat more accurate tendency of the events. 
In Table~\ref{tab:events}, we show the event name, primary mass $m_1$, secondary mass $m_2$, chirp mass $M_{\rm chirp}$ 
\begin{equation}
M_{\rm chirp} = \frac{(m_1 m_2)^{3/5}}{(m_1+m_2)^{1/5}} \,,
\end{equation}
luminosity distance $D_{\rm L}$, and mass ratio $q$
\begin{equation}
q =\frac{m_2}{m_1} \,.
\end{equation}
BNS, NS-BH, MGCO-BH, and BBH in the column of ``Binary type'' mean binary neutron star, neutron star-black hole binary, mass-gap compact object~\footnote{The mass of MGCOs lies in $2$--$5\,M_{\odot}$. See~\cite{2021PTEP.2021b1E01K} for details.}-black hole binary, and binary black hole, respectively.

In~\cite{2021arXiv211103634T},
the events with ${\rm \bf FAR}_{\rm min} < 1\,{\rm yr}^{-1}$ were used for BBHs, and the merger rate ($16$--$130\,{\rm yr}^{-1}{\rm Gpc}^{-3}$), substructure in the chirp mass distribution (peak around $8\,M_{\odot}$, weak structure around $15\,M_{\odot}$, and peak around $30\,M_{\odot}$), population model to explain the observation etc. were studied. 
As for the mass ratio distribution, a power law was treated to model it (see, e.g.,~\cite{2021arXiv210714239M} for rates of compact object coalescences, and references therein).

\begin{table*}
\caption{Event name (the YYMMDD\_hhmmss format), primary mass $m_1$, secondary mass $m_2$, chirp mass $M_{\rm chirp}$ in unit of the solar mass, $M_{\odot}$, and luminosity distance $D_{\rm L}$ [Mpc] from~\citet{online_GWTC}.
These are expressed by the median and 90\%-symmetric credible interval.
The mass ratio $q$ is evaluated by using the median values
of $m_1$ and $m_2$.
Binary type shows binary neutron star (BNS), neutron star-black hole binary (NS-BH), mass-gap compact object-black hole binary (MGCO-BH), or binary black hole (BBH).
Here, we focus only on events with ${\rm \bf FAR}_{\rm min} < 1 \times 10^{-5}\,{\rm yr}^{-1}$ where ${\rm \bf FAR}_{\rm min}$ means the minimum FAR evaluated by various GW data analyses.
These events have the probability of astrophysical (signal) origin, $p_{\rm astro} > 0.99$.
The data are sorted by $M_{\rm chirp}$.
The first 2 events are BNSs and NS-BH binaries.
The next 14 events are BBHs and a MGCO-BH, and have $M_{\rm chirp} < 18\,M_{\odot}$, and the final 20 events are BBHs and have $M_{\rm chirp} > 18\,M_{\odot}$.}
\label{tab:events}
\begin{center}
\renewcommand{\arraystretch}{1.2}
\begin{tabular}{ccccccc}
\hline
Event name & $m_1$ & $m_2$ & $M_{\rm chirp}$ & $D_{\rm L}$ & $q$ & Binary type \\
\hline
GW170817 & $1.46_{-0.1}^{+0.12}$ & $1.27_{-0.09}^{+0.09}$ & $1.186_{-0.001}^{+0.001}$ & $40_{-15}^{+7}$ & 0.87 & BNS
\\
GW200115\_042309 & $5.9_{-2.5}^{2}$ & $1.44_{-0.29}^{+0.85}$ & $2.43_{-0.07}^{+0.05}$ & $290_{-100}^{+150}$ & 0.24 & NS-BH \\
\hline
GW190924\_021846 & $8.8_{-1.8}^{+4.3}$ & $5.1_{-1.5}^{+1.2}$ & $5.8_{-0.2}^{+0.2}$ & $550_{-220}^{+220}$ & 0.58 & BBH \\
GW190814\_211039 & $23.3_{-1.4}^{+1.4}$ & $2.6_{-0.1}^{+0.1}$ & $6.11_{-0.05}^{+0.06}$ & $230_{-50}^{+40}$ & 0.11 & MGCO-BH \\
GW191129\_134029 & $10.7_{-2.1}^{+4.1}$ & $6.7_{-1.7}^{+1.5}$ & $7.31_{-0.28}^{+0.43}$ & $790_{-330}^{+260}$ & 0.63 & BBH \\
GW200202\_154313 & $10.1_{-1.4}^{+3.5}$ & $7.3_{-1.7}^{+1.1}$ & $7.49_{-0.2}^{+0.24}$ & $410_{-160}^{+150}$ & 0.72 & BBH \\
GW170608 & $11_{-1.7}^{+5.5}$ & $7.6_{-2.2}^{+1.4}$ & $7.9_{-0.2}^{+0.2}$ & $320_{-110}^{+120}$ & 0.69 & BBH \\
GW191216\_213338 & $12.1_{-2.3}^{+4.6}$ & $7.7_{-1.9}^{+1.6}$ & $8.33_{-0.19}^{+0.22}$ & $340_{-130}^{+120}$ & 0.64 & BBH \\
GW190707\_093326 & $12.1_{-2}^{+2.6}$ & $7.9_{-1.3}^{+1.6}$ & $8.4_{-0.4}^{+0.6}$ & $850_{-400}^{+340}$ & 0.65 & BBH \\
GW191204\_171526 & $11.9_{-1.8}^{+3.3}$ & $8.2_{-1.6}^{+1.4}$ & $8.55_{-0.27}^{+0.38}$ & $650_{-250}^{+190}$ & 0.69 & BBH \\
GW190728\_064510 & $12.5_{-2.3}^{+6.9}$ & $8_{-2.6}^{+1.7}$ & $8.6_{-0.3}^{+0.6}$ & $880_{-380}^{+260}$ & 0.64 & BBH \\
GW200316\_215756 & $13.1_{-2.9}^{+10.2}$ & $7.8_{-2.9}^{+1.9}$ & $8.75_{-0.55}^{+0.62}$ & $1120_{-440}^{+470}$ & 0.60 & BBH \\
GW151226 & $13.7_{-3.2}^{+8.8}$ & $7.7_{-2.5}^{+2.2}$ & $8.9_{-0.3}^{+0.3}$ & $450_{-190}^{+180}$ & 0.56 & BBH \\
GW190720\_000836 & $14.2_{-3.3}^{+5.6}$ & $7.5_{-1.8}^{+2.2}$ & $9_{-0.8}^{+0.4}$ & $770_{-260}^{+650}$ & 0.53 & BBH \\
GW190412\_053044 & $27.7_{-6}^{+6}$ & $9_{-1.4}^{+2}$ & $13.3_{-0.5}^{+0.5}$ & $720_{-220}^{+240}$ & 0.32 & BBH \\
GW190512\_180714 & $23.2_{-5.6}^{+5.6}$ & $12.5_{-2.6}^{+3.5}$ & $14.6_{-0.9}^{+1.4}$ & $1460_{-590}^{+510}$ & 0.54 & BBH \\
\hline
GW191215\_223052 & $24.9_{-4.1}^{+7.1}$ & $18.1_{-4.1}^{+3.8}$ & $18.4_{-1.7}^{+2.2}$ & $1930_{-860}^{+890}$ & 0.73 & BBH \\
GW190408\_181802 & $24.8_{-3.5}^{+5.4}$ & $18.5_{-4}^{+3.3}$ & $18.5_{-1.2}^{+1.9}$ & $1540_{-620}^{+440}$ & 0.75 & BBH \\
GW170104 & $30.8_{-5.6}^{+7.3}$ & $20_{-4.6}^{+4.9}$ & $21.4_{-1.8}^{+2.2}$ & $990_{-430}^{+440}$ & 0.65 & BBH \\
GW170814 & $30.6_{-3}^{+5.6}$ & $25.2_{-4}^{+2.8}$ & $24.1_{-1.1}^{+1.4}$ & $600_{-220}^{+150}$ & 0.82 & BBH \\
GW190915\_235702 & $32.6_{-4.9}^{+8.8}$ & $24.5_{-5.8}^{+4.9}$ & $24.4_{-2.3}^{+3}$ & $1750_{-650}^{+710}$ & 0.75 & BBH \\
GW190828\_063405 & $31.9_{-4.1}^{+5.4}$ & $25.8_{-5.3}^{+4.9}$ & $24.6_{-2}^{+3.6}$ & $2070_{-920}^{+650}$ & 0.81 & BBH \\
GW170809 & $35_{-5.9}^{+8.3}$ & $23.8_{-5.2}^{+5.1}$ & $24.9_{-1.7}^{+2.1}$ & $1030_{-390}^{+320}$ & 0.68 & BBH \\
GW190630\_185205 & $35.1_{-5.5}^{+6.5}$ & $24_{-5.2}^{+5.5}$ & $25.1_{-2.1}^{+2.2}$ & $870_{-360}^{+530}$ & 0.68 & BBH \\
GW200311\_115853 & $34.2_{-3.8}^{+6.4}$ & $27.7_{-5.9}^{+4.1}$ & $26.6_{-2}^{+2.4}$ & $1170_{-400}^{+280}$ & 0.81 & BBH \\
GW200129\_065458 & $34.5_{-3.2}^{+9.9}$ & $28.9_{-9.3}^{+3.4}$ & $27.2_{-2.3}^{+2.1}$ & $900_{-380}^{+290}$ & 0.84 & BBH \\
GW200112\_155838 & $35.6_{-4.5}^{+6.7}$ & $28.3_{-5.9}^{+4.4}$ & $27.4_{-2.1}^{+2.6}$ & $1250_{-460}^{+430}$ & 0.79 & BBH \\
GW150914 & $35.6_{-3.1}^{+4.7}$ & $30.6_{-4.4}^{+3}$ & $28.6_{-1.5}^{+1.7}$ & $440_{-170}^{+150}$ & 0.86 & BBH \\
GW170823 & $39.5_{-6.7}^{+11.2}$ & $29_{-7.8}^{+6.7}$ & $29.2_{-3.6}^{+4.6}$ & $1940_{-900}^{+970}$ & 0.73 & BBH \\
GW190503\_185404 & $41.3_{-7.7}^{+10.3}$ & $28.3_{-9.2}^{+7.5}$ & $29.3_{-4.4}^{+4.5}$ & $1520_{-600}^{+630}$ & 0.69 & BBH \\
GW190727\_060333 & $38.9_{-6}^{+8.9}$ & $30.2_{-8.3}^{+6.5}$ & $29.4_{-3.7}^{+4.6}$ & $3070_{-1230}^{+1300}$ & 0.78 & BBH \\
GW200224\_222234 & $40_{-4.5}^{+6.9}$ & $32.5_{-7.2}^{+5}$ & $31.1_{-2.6}^{+3.2}$ & $1710_{-640}^{+490}$ & 0.81 & BBH \\
GW190521\_074359 & $43.4_{-5.5}^{+5.8}$ & $33.4_{-6.8}^{+5.2}$ & $32.8_{-2.8}^{+3.2}$ & $1080_{-530}^{+580}$ & 0.77 & BBH \\
GW191222\_033537 & $45.1_{-8}^{+10.9}$ & $34.7_{-10.5}^{+9.3}$ & $33.8_{-5}^{+7.1}$ & $3000_{-1700}^{+1700}$ & 0.77 & BBH \\
GW190519\_153544 & $65.1_{-11}^{+10.8}$ & $40.8_{-12.7}^{+11.5}$ & $44.3_{-7.5}^{+6.8}$ & $2600_{-960}^{+1720}$ & 0.63 & BBH \\
GW190602\_175927 & $71.8_{-14.6}^{+18.1}$ & $44.8_{-19.6}^{+15.5}$ & $48_{-9.7}^{+9.5}$ & $2840_{-1280}^{+1930}$ & 0.62 & BBH \\
\hline
\end{tabular}
\renewcommand{\arraystretch}{1.0}
\end{center}
\end{table*}

In the following, we focus only on BBHs, and especially the masses~\footnote{There are also studies on the spins, for example, see \cite{2022arXiv220702924F} for limits on hierarchical BH mergers from an effective inspiral spin parameter. To extract more detailed information of spins, we will require multiband GW observations~\citep{Isoyama:2018rjb} with a decihertz GW detector, B-DECIGO~\citep{Nakamura_2016}.}.
Although the mass ratio of binaries is estimated less accurately than the chirp mass, we can find several studies on the mass ratio related to~\cite{2021arXiv211103634T}.
\cite{2021arXiv211113991T} showed that there is no prominent dependence either on the chirp mass or the aligned spin in the mass ratio distribution from the 69 GW events with ${\rm \bf FAR}_{\rm min} < 1\,{\rm yr}^{-1}$.
Here, it should be noted that the estimation of spins is more difficult than the mass ratio.
In our galaxy, \cite{2021arXiv211113704W} give a prediction of the mass ratio distribution with a peak at $q \approx 0.4$ for BBHs observed by a 4\,yr LISA observation~\citep{2017arXiv170200786A} in their fiducial model.
In simulations for hierarchical triples from low-mass young star clusters, \cite{2022MNRAS.511.1362T} found $q \approx 0.3$ which is lower than that from binaries (see, e.g., \cite{2021MNRAS.507.3612R}).
\cite{2021arXiv211205763B} showed that at least 95\% of BBH mergers detectable by LIGO/Virgo/KAGRA (LVK) detector network at design sensitivity have $q \gtrsim 0.25$ in their 560 models.
\cite{2021arXiv211210786S} presented orbital properties (masses, mass ratio, eccentricity, orbital separation etc.) of surviving systems after a BBH has formed in the inner binary and those of BBHs which are formed from an isolated binary population for metallicities, $Z=0.01\,Z_{\odot}$ and $Z_{\odot}$, where $Z_{\odot}$ is the solar metallicity, obtained by using a new triple stellar evolution code.
Using the 69 GW events with ${\rm \bf FAR}_{\rm min} < 1\,{\rm yr}^{-1}$, \cite{2022arXiv220101905L} found that the observed BBHs have a much stronger preference for equal mass binaries in their parameterized primary-mass distribution models.
Using direct $N$-body simulations for star cluster models, \cite{2022arXiv220208924C} found that the distributions of the mass ratio have median values in the range of $0.8$--$0.9$ (except for one model).
\cite{2022arXiv220303651M} have prepared a deep-learning pipeline to constrain properties of hierarchical black-hole mergers.
In the field of galaxies, \cite{2022arXiv220316544S} discussed BBHs starting from hierarchical triple population and isolated binary population, and found that the observed lower mass ratio ($q \lesssim 0.5$) BBHs can be explained by the contribution from the outer binary channel of the triple population.
In mass-ratio reversal systems where the second BH to form in the binary are more massive than the first BH, \cite{2022arXiv220501693B} found that BBHs with $M_{\rm chirp} \gtrsim 10\,M_{\odot}$ and $q \gtrsim 0.6$ are dominant in the GW observation (see also \cite{2022arXiv220512329M}). 
\cite{2022arXiv220613842B} introduced super-Eddington accretion into a population synthesis code.
\cite{2022arXiv220801081A} discussed a scenario of merging BBHs which are formed dynamically in globular clusters, and found that the observed events shown in GWTC-3 cannot be explained in the above scenario (see also~\cite{2022arXiv220905766M}).
In \cite{2022arXiv220905959F,2022arXiv220906196E}, the mass ratio has been discussed by treating a varying equation of state at the QCD epoch in primordial BH formation scenario (see also \cite{2023arXiv230603903C} for a recent review on primordial BHs).
\cite{2022arXiv221012834E} have suggested some possible plateaus at several mass ratios in the distribution by using a data-driven, non-parametric model.
\cite{2023arXiv230306081O} have discussed merging compact binaries with highly asymmetric mass ratios.
\cite{2023arXiv230315511C} discussed BBHs from Pop II ($Z = 10^{-4}$) and III ($Z = 10^{-11}$) stars in their models, and found that the mass ratios for Pop II BBHs are almost $q \sim 1$ and the peak of mass ratios for Pop III BBHs is $q=0.8$ -- $0.9$ in most of their models.
In \cite{2023arXiv230315515S}, the redshift dependence of mass ratio of Pop III BBHs was presented.
In their models, Pop III BBHs merging at low redshift ($z \leq 4$) have low mass ratios, $q \approx 0.5$ -- $0.7$ (the median values), while typically $q \sim 0.9$ at high redshift.
For star clusters, \cite{2023arXiv230704807A} have investigated various types of compact binary mergers by using their database~\citep{2023arXiv230704805A}.


%%%%%%%%%%%%%%%%%%%%%%%%%%%%%%%%%%%%%%%
\section{Analysis}
%%%%%%%%%%%%%%%%%%%%%%%%%%%%%%%%%%%%%%%

% Figure environment removed

In our previous study~\citep{2021MNRAS.504L..28K}, we found a very simple relation, $m_2\simeq 0.7\,m_1$~\footnote{\cite{2022arXiv220205861B} also mentioned this relation for nearly all BBH GW events.}, for BBHs with $M_{\rm chirp} \gtrsim 20\,M_{\odot}$ summarized in GWTC-2~\citep{2021PhRvX..11b1053A}, and that this relation is consistent with the mass distribution in our population synthesis simulations of Pop III stars.

Therefore, first, we focus only on 20 BBH events with $M_{\rm chirp} > 18\,M_{\odot}$ instead of $M_{\rm chirp} > 20\,M_{\odot}$ given in Table~\ref{tab:events}.
This is because there exists a small gap in the chirp mass between GW190512\_180714 and GW191215\_223052. 
Figure~\ref{fig:m1_m2_high} shows $m_1$ and $m_2$ of these events. 
Assuming $m_2=0$ at $m_1=0$, the linear fitting function is obtained as 
\begin{equation}
m_2=0.723\, m_1 \,, 
\end{equation}
and the correlation coefficient is $0.933$ with the chance probability of 0.00301\%.

% Figure environment removed

Next, we treat the remaining BBH events. Figure~\ref{fig:m1_m2_low} shows $m_1$ and $m_2$ of 13 BBH events with $M_{\rm chirp} < 18\,M_{\odot}$ given in Table~\ref{tab:events}.
Here, we have ignored GW190814\_211039 which has a MGCO in the binary.
Assuming $m_2=0$ at $m_1=0$, the linear fitting function is obtained as $m_2=0.527\, m_1$, and the correlation coefficient is $0.741$.
The correlation between the data and this fitting function
is not so high.

% Figure environment removed

Here, we note that GW190412\_053044 has large unequal component masses~\citep{2020PhRvD.102d3015A}.
As an alternative interpretation of masses for this GW event, we may have $q = 0.31^{+0.05}_{-0.04}$ from a prior with a non-spinning primary and a rapidly spinning secondary \citep{2020ApJ...895L..28M} (see also \cite{2020MNRAS.498.3946K}).
Also, \cite{2020ApJ...899L..17Z} found $q \lesssim 0.57$ in the 99\% credible level from various models.
Interestingly, although the GW data did not prefer Model G with a prior assumption, $\chi_1=\chi_2=0$ in the above paper, this model gave $q \approx 0.55$.
\cite{2023arXiv230611088A} have also discussed the most likely formation channel.
When we ignore this GW190412\_053044 event in the analysis of mass ratio, the fitting is improved as Fig.~\ref{fig:m1_m2_low_c}.
Assuming $m_2=0$ at $m_1=0$, the linear fitting function is obtained as
\begin{equation}
m_2=0.601\, m_1 \,, 
\end{equation}
and the correlation coefficient becomes $0.937$ with the chance probability of 0.117\%.
To argue the origin of BBH events for $M_{\rm chirp} < 18\,M_{\odot}$, we need more examples of BBHs similar to GW190412\_053044 in near future by the O4 Observing run~\citep{2020LRR....23....3A}.

In \cite{2020MNRAS.498.3946K}, we performed $10^6$ Pop III binary evolution by using seven different models with initial conditions of mass function, mass ratio, separation, and eccentricity as well as physical models (see Tables 2 and 3 of~\cite{2020MNRAS.498.3946K} for details of each model). 
In this paper, we adopt two models called `M100' and `K14'. 
The former model is the one of the best fit model in our previous paper \citep{2021MNRAS.504L..28K}.
The latter model is the first simulation of the Pop III binary stars performed in 2014 \citep{Kinugawa2014} as the typical example.
The main difference between the two models is the treatment of the mass transfer rate in the stable Roche lobe overflow shown from Eq.~(4) to Eq.~(7) of~\cite{2020MNRAS.498.3946K} in details.

% Figure environment removed

Figure~\ref{fig:Mc_dist} shows the chirp-mass distribution of 20 BBH events with $\Mc > 18\,M_{\odot}$. 
The vertical axis shows the expected number of BBHs for a given range of $\Mc$ so that the total number should be 20. 
The filled (red) circle shows the observed 20 GW events, and the filled square (blue) and diamond (cyan) denote the results of the M100 and K14 models by assuming the total observable number is 20. 
We can see that both the M100 and K14 models fit well with the observed one shown by the red circles. 

% Figure environment removed

Figure~\ref{fig:DL_dist} shows the luminosity distance ($D_{\rm L}$) distribution of 20 observed BBH events. 
We see that $D_{\rm L} \lesssim 3 \,{\rm Gpc}$ for $\Mc\lesssim 30\,\msun$ while $D_{\rm L} \gtrsim 1.5 \,{\rm Gpc}$ for $\Mc \gtrsim 30\,\msun$. But the errors are still large to say the above facts definitely.


%%%%%%%%%%%%%%%%%%%%%%%%%%%%%%%%%%%%%%%
\section{Discussion}
%%%%%%%%%%%%%%%%%%%%%%%%%%%%%%%%%%%%%%%

% Figure environment removed

% Figure environment removed

Figures~\ref{fig:QdistM100} and \ref{fig:QdistK14} show the delay time ($T_{\rm delay}$) distributions for each mass ratio of merging Pop III BBHs in the M100 and K14 models, respectively. 
We present the delay time distributions of Pop III BBHs which merge within the Hubble time in Figs. \ref{fig:QdistM100a} and \ref{fig:QdistK14a}, while the delay time distributions of Pop III BBHs of which the merger time is more than 10$^{3.5}$ Myrs and less than the Hubble time in Figs. \ref{fig:QdistM100d} and \ref{fig:QdistK14d}.
These distributions are normalized by the number of total Pop III binaries.

It is found in Figs. \ref{fig:QdistM100a} and \ref{fig:QdistK14a} that the nearly equal-mass BBH mergers predominate in a very short delay time region ($T_{\rm delay} \lesssim100$ Myrs).
In other words, the nearly equal-mass Pop III BBH mergers predominate in the very early universe ($z\gtrsim10$) since the Pop III stars are born and died at high redshift ($z\gtrsim10$).
These BBH mergers at the high redshift can be observed future GW observatory such as the Einstein telescope (ET)~\citep{ET}, the Cosmic Explorer (CE)~\citep{CE}, and DECIGO~\citep{Seto:2001qf,Nakamura_2016}.

The formation channel is the reason why the equal mass Pop III BBHs merge at the high redshift.
In our previous paper \citep{2020MNRAS.498.3946K}, we classified the Pop III BBH formation channel into the 5 channels such as NoCE, 1CE$_P$, 1CE$_S$, 1CE$_D$, and 2CE.
NoCE means that Pop III BBHs evolve not via a common envelope phase. 1CE$_P$, 1CE$_S$, and 1CE$_D$ are Pop III BBHs evolved via one common envelope phase caused by the primary giant star, the secondary giant star, or double giant stars, respectively. 2CE is the Pop III BBHs experienced more than two common envelope phases.
Figure 5 of~\cite{2020MNRAS.498.3946K} shows that almost all Pop III BBHs merging with very short delay time ($T_{\rm delay} \lesssim100$ Myrs) evolved via the 1CE$_D$ channel.

Figure \ref{fig:equalmassBBH} shows an example of 1CE$_D$ channel.
Pop III binaries of which initial masses are nearly equal tend to evolve via the 1CE$_D$ channel.
If the initial masses of a binary are similar, the timescale of evolution is similar as well. Therefore, they both become giant stars at almost the same time and shed their envelopes simultaneously through a double common envelope process. 
A significant amount of orbital energy is also lost by shedding the outer envelopes of both stars.
After that, a very close binary He star tends to remain, and they can become a close BBH which merges within 100 Myrs.

% Figure environment removed

Next, we focus on Figs. \ref{fig:QdistM100d} and \ref{fig:QdistK14d} where Pop III BBHs with a long delay time are presented for each mass ratio.
Figures \ref{fig:longmergertime_M100} and \ref{fig:longmergertime_K14} show the mass ratio distributions of merging Pop III BBHs of which delay time is more than 10$^{3.5}$ Myrs and less than the Hubble time for M100 and K14 models, respectively.
These BBHs can be detected within the detection range of LVK collaboration.

In the M100 model, the main contribution to the mass ratio distribution is the 1CE$_P$ channel. 
Subdominant ones are NoCE and 2CE channels. 
These channels make BBHs with various mass ratios from 0.4 to 1, unlike the 1CE$_D$ channel.
In the K14 model, the main contribution is the NoCE channel. 
Subdominant ones are 1CE$_P$ and 2CE channels. 
This model also has various mass ratio BBH mergers with a long delay time like the M100 model.

% Figure environment removed

% Figure environment removed

According to these results, it is expected that observations at high redshift, unlike observations at low redshift, would exhibit a higher proportion of equal-mass BBH mergers.
A comparison between the low-redshift results from the current ground-based GW observations and the high-redshift results from future observations such as ET, CE and DECIGO enables us to check the Pop III origin model.


%%%%%%%%%%%%%%%%%%%%%%%%%%%%%%%%%%%%%%%
\section*{Acknowledgment}
%%%%%%%%%%%%%%%%%%%%%%%%%%%%%%%%%%%%%%%

T. K. acknowledges support from JSPS KAKENHI Grant Numbers JP21K13915 and JP22K03630.
H. N. acknowledges support from JSPS KAKENHI Grant Numbers JP21H01082 and JP21K03582,
and also would like to thank to Y. W. for her hospitality.


%%%%%%%%%%%%%%%%%%%%%%%%%%%%%%%%%%%%%%%
\section*{Data Availability}
%%%%%%%%%%%%%%%%%%%%%%%%%%%%%%%%%%%%%%%

Results will be shared on reasonable request to the corresponding author.


%%%%%%%%%%%%%%%%%%%%%%%%%%%%%%%%%%%%%%%
\bibliographystyle{mnras}

\bibliography{ref}
%%%%%%%%%%%%%%%%%%%%%%%%%%%%%%%%%%%%%%%

%%%%%%%%%%%%%%%%%%%%%%%%%%%%%%%%%%%%%%%
\end{document}
%%%%%%%%%%%%%%%%%%%%%%%%%%%%%%%%%%%%%%%


