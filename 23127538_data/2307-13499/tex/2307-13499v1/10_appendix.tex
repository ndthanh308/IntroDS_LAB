\begin{comment}

Three groups (Original count): 
1) Unweighted Neighborhood summary (15 features), 
2) Weighted neighborhood summary  (8 features), 
3) metapath2vec features (64 features). 


\section{Unweighted Neighborhood summary}

\begin{table}[!ht]
    \centering
    \small
    \caption{notSetYet}
    \label{tab:metapaths}
    \begin{tabular}{c}
\hline
\multicolumn{1}{c}{notSetYet} \\
\hline
Outgoing txn \\
\hline
$\texttt{ind} \overset{\texttt{txn}}{\xrightarrow{\hspace*{1.5cm}}} \texttt{ind}$ \\
$\texttt{ind} \overset{\texttt{txn}}{\xrightarrow{\hspace*{1.5cm}}} \texttt{org}$ \\
$\texttt{ind} \overset{\texttt{txn}}{\xrightarrow{\hspace*{1.5cm}}} \texttt{ext}$ \\
\hline
Incoming txn \\
\hline
$\texttt{ind} \overset{\texttt{txn}}{\xrightarrow{\hspace*{1.5cm}}} \texttt{ind}$ \\
$\texttt{org} \overset{\texttt{txn}}{\xrightarrow{\hspace*{1.5cm}}} \texttt{ind}$ \\
$\texttt{ext} \overset{\texttt{txn}}{\xrightarrow{\hspace*{1.5cm}}} \texttt{ind}$ \\
\hline
Other meta-steps \\
$\texttt{ind} \overset{\texttt{role}}{\xrightarrow{\hspace*{1.5cm}}} \texttt{org}$ \\
$\texttt{ind} \overset{\texttt{addr}}{\xrightarrow{\hspace*{1.5cm}}} \texttt{addres}$ \\
$\texttt{ind} \overset{\texttt{mail}}{\xrightarrow{\hspace*{1.5cm}}} \texttt{emaila}$ \\
$\texttt{ind} \overset{\texttt{phne}}{\xrightarrow{\hspace*{1.5cm}}} \texttt{phonen}$ \\
$\texttt{ind} \overset{\texttt{cntr}}{\xrightarrow{\hspace*{1.5cm}}} \texttt{countr}$ \\
    \end{tabular}
\end{table}

These give 11 in/out degree features. In addition, four summary features are added: 1) total degree (sum of all above), 2) sum out-degree (sum of all outgoing), 3) sum in-degree (sum of all incoming), 4) number of meta-steps that the node is involved in (number of the 11 above that is non-zero).

Excluding the four meta-steps that were not mentioned in the article, we are left with 11-4+4=11 features

\end{comment}


%%%%%%%%%%%%%%%%%%%%%%%%%%%%%%%%%%%%%%%%%%%%%%%%%%%%%%%%%%%%%%%%%%%%%%%%%%%%%%%%%%%%%%%%%%%%%%%%%%%%%%%%%%%%%%%%%%%%%%%%%%%%%%%%%%
\appendix

\section{Network features} 
\label{app:network_features}
This appendix provides details of how we created additional node features that capture information about a node's role in the network. 
These were utilized in our entity-based models (and HGraphSage) for benchmarking in Section \ref{sec:usecase}.

We generated a total of 83 additional features, which when combined with the original 11 intrinsic node features, resulted in a total of 94 features. These additional features are categorized into three distinct types: 1) Unweighted Neighborhood summary consisting of 11 features, 2) Weighted Neighborhood summary consisting of 8 features, and 3) Metapath2vec embeddings with a total of 64 features.

\textit{Unweighted Neighborhood Summary}.
The unweighted neighborhood summary features encapsulate the (unweighted) in/out degree for each meta-step that nodes of type individual are part of, see Figure \ref{fig:graph_schema}. This amounts to seven features, six from transaction edges and one from role edges.
%
Further, four summary features are added:
1) The total in-degree, which is the count of all incoming edges, irrespective of the meta-step,
2) The total out-degree, which is the count of all outgoing edges, irrespective of the meta-step,
3) The total degree, which is the count of all edges, irrespective of the meta-step or the direction,
4) The total count of meta-steps in which the node is involved.
In total, this gives 11 additional node features.

\textit{Weighted Neighborhood Summary}.
The weighted neighborhood summary features contain the weighted in/out degree for each meta-step involving transaction edge. Here the edge feature representing the monetary amount was used as edge weight. This provides six features for a node of type individual.  
%
Further, two summary features are added:
1) The total weighted in-degree, 
2) The total weighted out-degree, 
%
In total, this gives 8 additional node features. 


\textit{Metapath2vec Embeddings}.
Metapath2Vec was used to generate embeddings for each of the four meta-paths shown in Table \ref{tab:metapath2vec_metapaths}. 
These are all meta-paths of length 2 that start and end at a node of type individual. 
The dimension of the embedding for each meta-path was set to 8. 
The embeddings were added as features on both the node at the start and end of the meta-path. This amounts to 64 additional node features. 
We used the implementation of MetaPath2Vec in Pytorch Geometric. Table \ref{tab:MetaPath2Vec_settings} lists the parameters used in the creation of the embeddings.  

\begin{table}[ht!]
    \centering
    \small
    \caption{Meta-paths for Individuals}
    \label{tab:metapath2vec_metapaths}
    \begin{tabular}{c}
\hline
$\texttt{ind} \overset{\texttt{txn}}{\xrightarrow{\hspace*{1.5cm}}} \texttt{ind} \overset{\texttt{txn}}{\xrightarrow{\hspace*{1.5cm}}} \texttt{ind}$ \\
$\texttt{ind} \overset{\texttt{txn}}{\xrightarrow{\hspace*{1.5cm}}} \texttt{org} \overset{\texttt{txn}}{\xrightarrow{\hspace*{1.5cm}}} \texttt{ind}$ \\
$\texttt{ind} \overset{\texttt{txn}}{\xrightarrow{\hspace*{1.5cm}}} \texttt{ext}\overset{\texttt{txn}}{\xrightarrow{\hspace*{1.5cm}}} \texttt{ind}$ \\
$\texttt{ind} \overset{\texttt{role}}{\xrightarrow{\hspace*{1.5cm}}} \texttt{org} \overset{\texttt{txn}}{\xrightarrow{\hspace*{1.5cm}}} \texttt{ind} $ \\
\hline
    \end{tabular}
\end{table}

%%%%%%%%%%%%%%%%%%%%%%%%%%%%%%%%%%%%%%%%%%%%%%%%%%%%%%%%%%%%%%%%%%%%%%%%%%%%%%%%%%%%%%%%%%%%%%%%%%
%\vspace{-1cm}
\begin{table}[ht!]
    \centering
    \footnotesize
    \caption{Hyperparameters for the creation of MetaPath2Vec-embeddings.}
    \label{tab:MetaPath2Vec_settings}
    \begin{tabular}{c|c}
    \hline
    Parameter & Value \\
    \hline
    \texttt{embedding\_dim}                 &8\\
    \texttt{walk\_length}                   &20\\
    \texttt{context\_size}                  &10\\
    \texttt{walks\_per\_node}               &10\\
    \texttt{num\_negative\_samples}         &1\\
    \hline
    \end{tabular}
\end{table}
%%%%%%%%%%%%%%%%%%%%%%%%%%%%%%%%%%%%%%%%%%%%%%%%%%%%%%%%%%%%%%%%%%%%%%%%%%%%%%%%%%%%%%%%%%%%%%%%%%

%%%%%%%%%%%%%%%%%%%%%%%%%%%%%%%%%%%%%%%%%%%%%%%%%%%%%%%%%%%%%%%%%%%%%%%%%%%%%%%%%%%%%%%%%%%%%%%%%%%%%%%%%%%%%%%%%%%%%%%%%%%%%%%%%%
\section{Network parameters} 
\label{app:parameters}

Table \ref{tab:num-parameters} shows the number of parameters for each of the models discussed in Section \ref{sec:usecase}. 

\begin{table}[ht!]
\centering
\small
\sisetup{group-separator={,},group-four-digits=true}
\caption{Number of model parameters for the different models}
\label{tab:num-parameters}
\begin{tabular}{lSSS}
\toprule
\multicolumn{1}{c}{\multirow{2}{*}{Model}} &  \multicolumn{3}{c}{Number of Layers}  \\ 
\cmidrule(r{5pt}){2-4}
& \multicolumn{1}{c}{1} & \multicolumn{1}{c}{2} & \multicolumn{1}{c}{3}  \\
\midrule 
%NeuralNetwork & 99 & 9801 & 19503 \\
NeuralNetwork & 95 & 9025 & 17955 \\
HGraphSage & 189 & 3999 & 7809 \\
HGraphSage (extra features) & 329 & 13045 & 25761 \\
HMPNN-sum & 296 & 6536 & 12776 \\
HMPNN-ct & 3071 & 4487 & 6303 \\
\bottomrule
\end{tabular}
\end{table}