\section{AML use case}
\label{data}

This section describes the heterogeneous graph which are subjects of the experiments in this paper.
The graph is established based on custom and transaction data from Norway's largest bank, DNB, in the period ... to ....
%The main ingredient in the graph is all transactions to or from customers in Norway's largest bank from in the period ... to .... Additionally, the graph includes basic data about the customers, and professional role relationships individuals may have in companies in the graph.
The nodes in the graph represent entities that are senders and recipients of the financial transactions.
If two entities participate in a transaction with each other, this is represented by an edge between the two, where the direction of the edge points from the sender to the recipient.
There are in total 5 million nodes and 9 million such edges in the graph. 

There are three types of nodes in the graph. 
The first one is called individual, and represents a human individual’s customer relationship in the bank. 
It includes all of the individual’s accounts in the bank\footnote{ 
Transactions made to or from any of the accounts in the bank belonging to the customer will result in an edge to or from the node representing the individual. 
}.
The second type of node is called organisation, and represents an organisation’s or company’s customer relationship in the bank in the same manner as for a node representing an individual. 
The third type of node is called external, and represents a sender or recipient in a transaction which is outside of the bank. 
%
The majority of the edges in the graph represent presence of a financial transactions between different individuals/organizations/external entities in the edge direction. 
In addition to this edge type, the graph includes a second edge type: role.
This edge points from an individual to an organisation if the individual occupy a position in the board, is the CEO, or holds an ownership in the organisation. 
%
The resulting graph is directed and heterogeneous with respect to both nodes and edges. 
Figure \ref{fig:graph_schema} shows the schema of the graph, including the nine possible meta-steps. 


% Figure environment removed

%%%%%%%%%%%%%%%%%%%%%%%%%%%%%%%%%%%%%%%%%%%%%%%%%%%%%%%%%%%%%%%%%%%%%%%%%%%%%%%%%%%%%%%%%%%%%%%%%%%%

Due to the sensitive nature of these data, we cannot reveal the exact definitions of the features assoicated with the different nodes/edges in our model, but the in the following we give a broad overview.

The three node types have separate set of features which contain basic information about the entity such as age, gender and nationality.
The transaction edges contains aggregated transactions between the sender and recipient during a one year period, and has as features the number and monetary amount of transactions made. I.e.~multiple transactions between to entities are only represented by a single edge (per direction), while the number of transactions and their 
The role edges, on the other hand, have no features. 

The nodes that represent customers of the bank, i.e.~individuals and organisations, are assigned a
binary class (0 for regular customers, 1 for suspicious customers). 
Suspicious customers are defined as customers that have been subject of an AML case (stage 2\footnote{
Note that customers implicated in cases that were not reported to the FIU, are still defined as suspicious.  
This decision was made because our objective is to model suspicious activity, which these customers certainly have conducted, even though the suspiciousness was diminished by a close manual inspection.
} 
in Figure \ref{fig:alert_workflow}) during a certain (non-disclosable) time window. Approximately 0.8\% of the customer nodes belong to class 1.  

In addition to the original node features, we include four types of network features: Degree summary, centrality metrics, DeepWalk embeddings and Metapath2Vec embeddings. 
Combined, they are an attempt to incorporate high-quality features on the node-level that summarize information about their role in the heterogeneous network. 
In total, this results in 79 additional features. 
Combined with the entity features we end up with 87 and 86 features for individuals and organisations, respectively. Precise specification of these network features are provided in \ref{app:network_features}.

