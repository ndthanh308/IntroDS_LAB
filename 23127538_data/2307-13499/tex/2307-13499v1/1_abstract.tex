\begin{abstract}
%Money laundering represents a significant global challenge, as it facilitates criminal activities driven by the pursuit of financial gain. 
Current anti-money laundering (AML) systems, predominantly rule-based, exhibit notable shortcomings in efficiently and precisely detecting instances of money laundering. 
As a result, there has been a recent surge toward exploring alternative approaches, particularly those utilizing machine learning.
Since criminals often collaborate in their money laundering endeavors, accounting for diverse types of customer relations and links becomes crucial. 
In line with this, the present paper introduces a graph neural network (GNN) approach to identify money laundering activities within a large heterogeneous network constructed from real-world bank transactions and business role data belonging to DNB, Norway's largest bank.
Specifically, we extend the homogeneous GNN method known as the Message Passing Neural Network (MPNN) to operate effectively on a heterogeneous graph. As part of this procedure, we propose a novel method for aggregating messages across different edges of the graph. 
%Our findings highlight the significance of utilizing edge features, when available, which most of the heterogeneous GNNs proposed in existing literature do not support.
Our findings highlight the importance of using an appropriate GNN architecture when combining information in heterogeneous graphs.
The performance results of our model demonstrate great potential in enhancing the quality of electronic surveillance systems employed by banks to detect instances of money laundering. To the best of our knowledge, this is the first published work applying GNN on a large real-world heterogeneous network for anti-money laundering purposes.
\end{abstract}


