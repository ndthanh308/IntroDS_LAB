\section{Threat Model}
Information-flow analysis is a part of the security validation
activities~\cite{dorsey2020intel} that take place during the design phase of
the hardware lifecycle~\cite{he2015model}. The goal is to find
weaknesses, vulnerabilities, and flaws in register transfer level (RTL)
designs that may be exploitable post-deployment.

%% Our threat model is an attacker, or colluding attackers, acting in the
%% post-deployment phase, after the hardware has been fabricated and shipped. The
%% attackers aim to exploit a hardware vulnerability to gain unauthorized access to
%% data or execution. For example, a hardware flaw may allow the attacker to learn
%% confidential data stored in memory or hardcoded into a device, or it may allow
%% the attacker to modify the control flow of software running on a processor by
%% leveraging aspects of the processor's behavior. The attacker may access the
%% hardware locally. Remote attacks caused by running attacker-controlled software
%% (e.g., JavaScript on a website visited by the machine) are feasible. Physical
%% attacks, including hardware tampering, fault injection, and power or
%% electromagnetic side channels, are out of scope.
Flaws that result from logic and physical synthesis
tools, manufacturing, or the supply chain cannot be discovered by SEIF. 
We target flaws that occur by benign human error in the specification,
design, or implementation phases. Our analysis may find maliciously
inserted flaws, they will have a lower chance of being uncovered than benign
flaws as the attacker will likely take steps to hide their work,
so that the security engineer does not recognize the
malicious flow of information as dangerous. Flaws maliciously inserted after the
security validation is complete, e.g., analog Trojans~\cite{YangSP2016}, cannot
be discovered by SEIF.

%% Finally, the security engineer who is conducting the information-flow analysis
%% is trusted to act in good faith. The goal of our tool is to help the engineer
%% pare down the set of possible flows of information requiring study, and provide
%% concrete test cases for true flows of information to help the engineer better
%% understand the design. This tool supports the imperfect engineer by helping them
%% to understand the design better and highlight the security implications of
%% various information flows.
