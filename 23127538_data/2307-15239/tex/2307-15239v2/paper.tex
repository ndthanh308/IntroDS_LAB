\documentclass{scrartcl}

\usepackage[style=apa,backend=biber,natbib=true]{biblatex}
\addbibresource{Thesis.bib}
\renewcommand*{\bibfont}{\small}
\patchcmd{\bibsetup}{\interlinepenalty=5000}{\interlinepenalty=10000}{}{}

\usepackage[utf8]{inputenc}
\usepackage[T1]{fontenc}
\usepackage{amsmath}
\usepackage{amsfonts}
\usepackage{amssymb}
\usepackage{lmodern}
\usepackage{color}
\usepackage{graphicx}
\usepackage[lmargin=2cm,rmargin=2.5cm]{geometry}
\usepackage{lscape}
\usepackage{multicol}
\usepackage{subcaption}
\usepackage{empheq}
\usepackage{amsmath}
\usepackage{gensymb}
\usepackage{hyperref}
\usepackage[table]{xcolor}% http://ctan.org/pkg/xcolor
\usepackage{tabularx}
\usepackage{booktabs,siunitx}
\usepackage{multicol}

\RequirePackage{silence}
\WarningFilter{gensymb}{Not defining}

\usepackage{array}
\newcolumntype{x}[1]{>{\centering\let\newline\\\arraybackslash\hspace{0pt}}p{#1}}


\title{\LARGE\textbf Self-recognition generates characteristic responses in \\[1ex] pupil dynamics and microsaccade rate}
\author{Lisa Schwetlick$^1$, Hendrik Graupner$^{2,3}$, Olaf Dimigen$^4$, Ralf Engbert$^1$ \\\small{$^1$Department of Psychology, University of Potsdam, Germany}, \\\small{$^2$Bundesdruckerei GmbH, Berlin, Germany}\\\small{$^3$Hasso Plattner Institute, University of Potsdam, Germany}, \\\small{$^4$Faculty of Behavioural and Social Sciences, University of Groningen, Netherlands}}

\begin{document}
\maketitle

\abstract{
Visual fixation is an active process with pupil dynamics as well as fixational eye movements and microsaccades that support perception. Measures of both pupil contraction and microsaccades are known to be sensitive to ongoing cognition and emotional processing. Here we present experimental results from a visual fixation task demonstrating that pupil size and microsaccade rate respond differently during self-recognition (when seeing one's own face) than when seeing familiar or unfamiliar faces.  First, the pupil response is characterized by an immediate pupil-constriction followed by later dilation in response to stimulus onsets. For one's own face, we observe muted constriction and greater dilation compared to other faces.  Second, microsaccades, which generally show an inhibitory response to incoming stimuli, are more strongly inhibited in response to one's own face compared to other faces.
Our results lend support to the idea that eye-related physiological measures could be useful for authentication or exposing hidden knowledge. 
}

\vspace{1cm}
{ \textit{Keywords:} face perception, self-recognition, pupillometry, microsaccades}

\newpage


Eye-related motor activity is an inseparable and essential component of visual perception. Easy to recognize is the macroscopic sequence of gaze shifts (saccades) and periods of relative rest (fixations) that give rise to the eye's trajectory during scene exploration \citep{Henderson2003}. During visual fixations, movements are involuntary and can only be recorded in the laboratory. First, fixational eye movements serve several visual functions, for example counteracting perceptual fading \citep{MartinezConde2004}. Microsaccades represent the functionally most important component of fixational eye movements: rapid small-amplitude movements that share their kinematic relations with larger saccades \citep{Bahill1975}. Second, the pupillary light reflex adapts the pupil size to the surrounding illumination \citep{Mathot2018}. The involuntary processes of fixational eye movements and dilation and constriction of the pupil are modulated by ongoing cognitive and emotional processing \citep{Hess1960,Winterson1976}. In the current study we set out to investigate specific pupillary and microsaccadic responses during recognition of familiar or unknown faces compared to self-recognition of one's own face.

Pupil size varies between 1.5~mm to 9~mm in diameter \citep{Sirois2014}. Modern pupillometry research distinguishes three types of pupil response: The pupilary light reflex (PLR), the pupil near response (PNR), and psychosensory pupil responses \citep{Mathot2018}.
The PLR is the most dominant factor in determining pupil size, causing the pupil to constrict in bright environments and dilate in darkness. The PLR causes the pupil to constrict within the first 200~ms after stimulus onset, reaching a minimum pupil size between 200~ms \citep{Mathot2018} and 1600~ms \citep{Bradley2008} after stimulus onset. Even covert attention to light or pre-saccadic attention can trigger the PLR, however, these changes tend to be comparatively small \citep{Mathot2018}.
The PNR modulates pupil size with respect to the distance to the focused object. When focusing a close-by object, the pupil tends to constrict, while focus on a far-away object causes it to dilate. Pupil constriction allows greater visual acuity as well as a greater depth of field, i.e., range of distances at which the objects appear in focus \citep{Mathot2018, Charman1977}.

In addition to these external factors that drive pupil size, ongoing cognition and arousal, modulated by variables such as interest or processing load, have also been found to influence pupil size. Physiologically, pupil size is connected to brain regions related to controlling sleep-wake rhythms and general activation of the nervous system \citep{Sirois2014}.
Specifically, the Locus Coeruleus, which is involved in memory retrieval and selective attention, has been found to be highly correlated with pupil size in monkeys \citep{Laeng2012}.
The cognitive effects of pupil size have been reported as early as 1960 \citep{Sirois2014, Hess1960} can be detected within the second after target onset. These early effects are related particularly to  novelty and saliency \citep{Mathot2018}, as well as surprisal, uncertainty and prediction errors \citep{Larsen2018}.
The effects of (positive and negative) arousal and mental effort are associated with slightly longer delay \citep[see][for a review]{Mathot2018}, while emotional responses peak even later, i.e., later than 2~s after target onset.
Thus, cognitive effects of pupil size can be detected within the first few hundred milliseconds after target onset, but peak only after 1 to 2~s \citep{Kinner2017}. 

The second type of involuntary eye movement investigated here are fixational eye movements. The eyes are never fully stationary, even during the relatively stable fixations. Movements made during fixations are typically classified as fixational drift, tremor, and Microsaccades \citep{MartinezConde2004}. Microsaccades occur spontaneously during fixation and share most of the kinematic properties of their macroscopic counterparts. They are distinguished primarily, and somewhat arbitrarily by their size: typically the definition is as a saccade with an amplitudes below 1$^\circ$ of visual angle. The average rate for microsaccades is 1 per second (1~Hz). However, changes in visual input modulate this rate \citep{Engbert2003,Engbert2012}.
Display changes cause a brief and temporary decrease in the rate of microsaccades, known as microsaccadic inhibition. This is subsequently followed by a period of increase, which can sometimes cause microsaccade rate to exceed its baseline level, until the rate eventually returns to its resting state of approximately 1~Hz after 500-1000~ms. 
The display-change-related microsaccade inhibition is typically considered as a reflexive response generates at the superior colliculus (SC) level \citep{Laubrock2013, Rolfs2009}. The SC, aside from generating low-level motor signals, also receives inputs from a variety of regions that may convey more top down information  \citep{Sparks2002, Schall1995}. \citet{Valsecchi2006} showed that the microsaccade inhibition response is modulated by different stimuli. 
From the literature we conclude that both pupil size and microsaccade rate are involuntary measures that are primarily functionally-driven but are modulated by cognition. 

A paradigm that has been used to explore the modulation in pupil size and microsaccade rate, as well as using other measures is the \emph{oddball paradigm}. Typically participants are presented with a sequence of stimuli that are either frequent (or expected), or infrequent (or unexpected) \citep{Sutton1965}. The infrequent target (Oddball stimulus) is presented within a stream of distractors, while participants are generally given a task, e.g. counting. This paradigm was first used in the context of event-related potential measurements \citep[see][for a more extensive review]{Polich2007}, which found that a particular Event related potential (P300) occurs around 300 ms after stimulus presentation \citep{Squires1975, Sutton1965}, particularly when the participant is engaged in target detection \citep{Picton1992}.The latency is related to the discrimination difficulty and the amplitude is affected by the target frequency \citep{Picton1992}. Neurophysiological evidence suggests that the P300 response is comprised of two components, one relating to novelty situated in the frontal lobe activity, and one relating to stimulus processing and memory situated in the central/parietal regions \citep{Linden2005}. Much evidence suggests P300 is an inhibition signal which inhibits irrelevant processes to prioritize the processing of important or salient stimuli \citep{Ranganath2003}. More generally, the most evident psychological correlate of P300 is related to attention, supported by studies that show that the arousal level and availability of attentional resources \citep{Kok2001} affects P300 outcomes. % Kok, 1990;Pribram and McGuinness, 1975


Using an oddball task, further electrophysiological correlates have also been found the human Locus Coeruleus using fmri \citep{Murphy2014}, an area which is generally associated with attention modulation. The authors also report a high correlation of Locus Coeruleus activation with pupil dilation. Effects of the oddball task on pupil dilation  \citep{Kamp2015, Strauch2020} are consistent with this finding. Further, the oddball task also affects microsaccade rate \citep{Valsecchi2006,Valsecchi2009}. 






Here, we explore the idea of face recognition as a kind of oddball task. We suggest that face recognition triggers a response that is similar to the surprisal which is modulated via the frequency in the oddball task. 
We propose that the recognition of one's own face evokes a stronger Oddball response, i.e. reduced microsaccade rate and increased pupil dilation, compared to familiar and unfamiliar faces. More generally, we expect that recognition of familiar faces to cause a stronger response than unfamiliar faces. The differential recognition response is likely to be comprised of a variety of reflexive, cognitive, and emotional components. 
Pupillometry suggests that emotional responses have a longer latency before peaking than cognitive effects. Microsaccade rates, as their expected effect is both much shorter duration and delay, are likely to be primarily driven by the reflexive and cognitive aspects of recognition. 
We expect effects of (self-) recognition to extend from novelty and surprisal to cognitive and emotional, and therefore produce a strong effect on both microsaccades and pupil dilation. In the experimental design we aim to disentangle novelty effects from recognition effects by including a subset of repeated stimuli. 
The relationship between both measures,  Microsaccade rate and pupil size, has not been extensively explored, but it is plausible that attentional processes present a common origin for changes detected by both measures.


The aim of the present study is to investigate whether pupil size and microsaccade rates can indicate whether a participant is looking at an image of her own face compared to the image of a friend's or a stranger's face. 
Robust differences between self recognition and viewing of other faces in our measures derived from pupil dynamics and microsaccade statistics may be applicable in the context of establishing viewer identity or exposing viewer knowledge.






\section{Methods}


\subsection{Participants}
We recruited participants from two graduating high school classes in Potsdam, Germany. The advantage of this setup is that it allows a well-controlled design where each participant personally knows a subset of the other participants. 
Thus, each participant was shown faces that were either their classmates' faces (\emph{Friends}), or students from the other high school (\emph{Strangers}), or their own (\emph{Self}). Of the initial cohort of 127 students who's picture was taken, 118 students came to the eye tracking session. Of these, two individuals were not able to be calibrated in the eye tracker. 116 complete data sets remain. The participants were between the ages of 16 and 18 years old; 55 were male, 63 female and one non-binary.  

The eye tracking data collection was followed by a further questionnaire, where each participant saw the full data set of images and was asked to respond whether they did indeed know the person in each picture. Cases where a participant's response to the question did not match with the expectation, i.e. when they did know someone from the other class or did not know someone from their own class, were removed from the final data set. The final data set therefore represents a balanced experimental design where each image, in principle, appears in each condition.

\subsection{Photographs}
In order to collect the required photographs of the participants a professional photographer visited the schools. Each participating student received an anonymous code. The photographs were taken under consistent illumination and with the heads centered, in a quasi-biometric setup, in order to maximize consistency between the pictures. The photographs were coded with the anonymous code, so that the mapping between participant and photograph was possible without the necessity of saving any identifying information alongside the image. The high resolution photographs were further cropped and scaled to center each face inside a square image. For data security reasons the photographs of the faces were deleted upon completion of the study. In the data set only the anonymous codes remain. 

We computed the relative image luminance for each image and found it to be normally distributed according to a Shapiro Wilk test of normality (\textit{p=0.24}). As each image is shown in each condition (i.e. every image is seen as Friend, Stranger, and Self), differences in image statistics would effect all groups equally and not influence the main effects. 

\subsection{Eye Tracking Data Collection}
For recording eye trajectories we used an Eyelink 1000 eye tracker, which recorded both eyes at 500~Hz at an illumination level of 75\%. The screen (a ViewPIXX monitor with a resolution of 1920$\times$1080) was placed at 70~cm distance from the participant, with the head stabilized in a chin rest. The eyes were centered at $3/4$ of the height of the screen. We used a 5-point calibration grid and subjects were re-calibrated every 14 trials. As the experiment required exclusively fixation in the center (and no exploration of the outlying regions of the screen), a 5-point calibration was sufficient to ensure data quality.

Face images were shown at a resolution of 500$\times$500~px centered on the screen, meaning that each image subtended 11.4$^\circ$ of visual angle. The experimental session proceeded as follows. Participants were informed that they would be seeing pictures of faces and were asked to fixate the central fixation marker for the duration of the trial. 
No task was given apart from fixating the fixation marker. 
The face photograph appeared under the fixation marker for a duration of 300~ms. They were asked not to blink and not to move their eyes. Figure \ref{fig:trial} shows the time course of a single trial. 

The experiment is preceded by three training trials, so that participants get used to the procedure of fixating and not blinking. The three faces for the training block are from the same image data set but are not used again for trials of the same subject. After the training block, the experimental trials followed. Each trial takes roughly 5~s, split into three phases: (a) A fixation check of 2~s, i.e., if the eyes are not centered on the marker a re-calibration is initiated, (b) an interval of 300~ms of presentation of the face photograph, and (c) a fixation cross for 3~s. After each trial participants were encouraged to take a break and to blink. Using a key press they indicated when they were ready for the next trial.

% Figure environment removed

Phases a) and c) showed a grey background which covered the same area as the following image. The shade of grey in the background was designed to be adjusted to the luminance of the stimulus. The luminance was proportional to, but not equal to the stimulus luminance, ranging from a relative luminance of 3.5-6\% while the images varied in relative luminance from 2.5 to 6.5\%. As the distribution of images was well controlled by the experimental design, i.e. each image appeared in each condition, any differences between images can be accounted for by random effect of image in a Linear Mixed Model Analysis.  

Each subject saw the following 100 trials in random order:
{\renewcommand{\theenumi}{\alph{enumi}}
\begin{enumerate}%[a)]
\itemsep0ex
\item 10 repetitions of a photograph of their own face (10 trials)
\item 5 repetitions of photographs of 3 selected friends (15 trials)
\item 5 repetitions of photographs of 3 selected strangers (15 trials)
\item 1 photograph each of 30 strangers (30 trials)
\item 1 photograph each of 30 friends (30 trials)
\end{enumerate}
}
Categories b and c were introduced in order to control for repetition effects that occur during the repeated viewing of one's own face. In order to exclude a pure oddball effect every participant sees is an equal number of familiar and unfamiliar faces. 


\subsection{Pupillometry Analyses}
In order to prepare the data for analysis we closely followed the recommended pre-processing pipeline suggested by \citet{Mathot2022}. Pupil size data was down-sampled to 100~Hz, as a higher resolution is not informative for pupil responses. Missing data were linearly interpolated (and excluded from the analyses). We converted the pupil diameter (which is given by the Eye tracker in arbitrary units) to mm as our base unit and then computed the pupil response of each trial relative to the baseline. The baseline value for each trial was the average pupil size during the 50~ms surrounding target onset (as proposed by \citet{Mathot2022}). We ensured data quality by evaluating the pupil size during the pre-stimulus phase. We found no indication that any trials or participants had to be excluded on the basis of the baseline values. Trials that included a blink after stimulus onset were excluded from the analyses. 

First, as a qualitative analysis, we plotted the pupil response over time,  normalized to the stimulus onset. Second, to statistically support our findings we use linear mixed models (LMM) with pupil size as the dependent variable. While analyses that more efficiently take advantage of the the time series data, such as cluster-based permutation tests, are available and suggested for pupillometry data \citep{Mathot2022}, they were not feasible for this use case, as the we had large differences in the number of samples per group (the "Self, First Presentation" condition exists only a single time for each participant, but many times for "Stranger"). We therefore chose to analyze both measures using LMMs in 5 time windows: Baseline (-50 -- +50~ms), Constriction (200 -- 600~ms), Dilation (600 -- 1200~ms), Late Dilation (1200 -- 2000~ms), and Stability (2000 -- 3000~ms). The baseline window is included as a sanity check, to ensure no effects are found before target onset.

Within each window we computed linear mixed-effect models (LMMs) for the dependent variable "Pupil Size" using the lme4 package \citep{Bates2015a} in R \citep{RCore2021}.
We define the following (custom) contrasts \citep{Schad2020} in our LMM, to test our hypotheses: 
\begin{enumerate}
    \item We compare the Self condition to Stranger and Friend, resulting in a Self--Other comparison.
    \item  We compare the Stranger and Friend conditions (Stranger--Friend).
    \item We compare the repetitions of individual images up to the 5th repetition, using a sliding difference contrast, meaning that we compare presentation 1 to presentation 2, presentation 2 to presentation 3 and so forth.
    \item We compare the interaction of Repetition and Face, which indicates whether the difference in the face conditions is different based on Repetition.
\end{enumerate}

Adding Time within the window and the ordinal trial number as covariates yields the following fixed effects structure for the model formula:

\begin{equation}
Pupil\_Size \sim Face * Repetition + Time + Trial
\end{equation}

The selection process for the random effect structure, is described in Appendix \ref{appx:sec:lmm}. Following \citet{Baayen2008}, we interpret all |t| > 2 as significant fixed effects.


\subsection{Microsaccade Rate Analyses}
Microsaccades were detected with millisecond accuracy from raw data by applying a standard microsaccade detection algorithm using a velocity threshold \citep{Engbert2003,Engbert2006b}. Microsaccade rate can be estimated with the help of a response function that applies filter kernels to the series of onset times \citep{Engbert2021,Engbert2006}. From the series of microsaccade onset times $\{t_1, t_2, t_3, ...\}$, the microsaccade rate $r(t)$ at time $t$ is estimated via \begin{equation}
r_{\mathrm approx}(t) = \int_{-\infty}^{+\infty} {\mathrm d}\tau w(\tau)\rho(t-\tau) \;,
\end{equation}
where the microsaccadic response function $\rho(t)$ \citep{Engbert2021} is defined as 
\begin{equation}
\rho(t)=\sum_{i=1}^N \delta(t-t_i) 
\end{equation}
with Dirac's $\delta$-function $\delta(t)$. We applied a filter kernel known as a causal window, i.e., 
\begin{equation}
w(\tau)=\left[ \alpha^2\tau\exp(-\alpha\tau)  \right]_+
\end{equation}
with parameter $\alpha=1/30$. The microsaccade rate $r(t)$ was computed by averaging over microsaccades from all trials of a participant in a specific experimental condition. The resulting microsaccade rate as a function of time is qualitatively similar the time-courses of pupil size or event-related potentials, which are all stimulus-locked, continuous response functions averaged over many experimental trials. 

We identifyied six relevant time windows for our analyses:
Baseline ($-300$ to 0~ms), Target Onset (0 to 300~ms), Target Offset (300 to 600~ms), Return (600 to 900~ms), and Stability, which we divided into two 300~ms windows for consistency (900 to 1200~ms and 1200 to 1500~ms). The 300~ms window size is consistent with the finding that the microsaccade rate decrease in response to a target onset has a duration of approximately 300~ms before returning to the baseline \citep{Engbert2003}.

For the statistical analysis, we apply the knowledge that microsaccades are Poisson-distributed \citep{Engbert2006a} and  conduct Poisson rate tests in each time window. In this analysis we compare the conditions Self--Other, and Stranger--Friend and report the ratio of the estimated Poisson rate. If the estimated Poisson rate is the the same in both groups the ratio will be 1, indicating no difference in microsaccade rate. 

\subsection{Relating Microsaccades and Pupil Size}
As both microsaccade rate and pupil size are reactive to our face conditions, in an explorative analysis we pose the question: are the two measures related, i.e. do trials with many microsaccades also show reduced dilation and vice versa? A relationship between the two would be an indication of a shared origin. The microsaccade effect precedes the pupil effect in time, with the drop in  microsaccade occurring between 50 and 1000 ms after target onset while pupil size effects peak only after 1000ms after target onset. In order to analyse the relationship of microsaccades and pupil size, we  begin by selecting the most diagnostic window for identity using microsaccades (i.e. the Target Offset Phase, 300-600ms after target onset) and use it as a predictor for pupil size.  The distribution of microsaccade counts in this subset shows that the majority of trials (7919 trials) have a count of 0 microsaccades. 1992 trials had 1 or more saccades. This is a significant difference in sample size and may influence the results. We use a binary variable which encodes the presence of one or more microsaccades in the diagnostic window as an ad-hoc predictor variable of pupil size in an explorative LMM using the formula

\begin{equation}
Pupil\_Size \sim 1 + MS\_in\_phase*Face + (1|VP).
\end{equation}

The selection of the random effects structure is detailed in the appendix.

\section{Results}
In the present study we investigate the effect of seeing one's own face compared to Friend's and Stranger's faces on involuntary eye measures. A subset of faces, including the Self condition, was shown multiple times in order to explore how stimulus repetition influences the effect and to control for any novelty effects.

\subsection{Pupil size}
In the top panel of Figure \ref{fig:pupil_cat} the average pupil reaction for each of the 3 face conditions reveals that there is a qualitative difference between all three. In the following we report the results of the LMM in each phase (refer also to Table \ref{tab:pupillmm}; the coefficients in the Table may be interpreted as the absolute difference in response in mm).

In the Baseline condition, apart from a minimal dilation trend over time, the face conditions behave identically (Figure \ref{fig:pupil_cat} B). This is confirmed by the LMM, as presented in Table \ref{tab:pupillmm}. 

The following window represents the initial constriction, which is a reflexive response to the stimulus presentation. In the constriction phase too, no significant differences for the main effects can be found, with the exception of the comparison between the 4th and 5th presentation. 

In all following time windows, Dilation, Late Dilation and Stability Phases, we observe significant differences between the Face conditions. The effect size is largest in the Late Dilation phase for the Self--Other comparison and in the Stability phase for the Friend--Stranger comparison.


The repetition main effect is is coded by a sliding differences contrast, meaning we compare subsequent presentations. We find a large number of the comparisons to be significant, showing that the pupil response becomes muted over repetitions. Note that we investigate the difference of each presentation to the next; a different contrast (e.g. a treatment contrast, comparing each level to the first repetition) would have likely resulted in more consistently significant effects. It may also be interesting to point out that while the effect becomes more muted over the presentations, the last comparison (5th vs 4th presentation) actually indicates a reversal, perhaps pointing to a limit to the muting.

The interaction terms show a varying pattern of significance. The interactions with the Self--Other comparisons is consistently negative or absent, with the exception of the 2nd-3rd comparison. The interaction term can be interpreted as the pupil muting over repetitions being more or less pronounced. A negative term indicates that the muting in response to the Self condition is stronger than in the other conditions. 

The same interpretation can be applied to the Friend--Stranger and Repetition interactions. Here, a negative term means stronger muting of the friend condition. Note that in Figure \ref{fig:pupil_cat} it is evident that the 2nd--1st : Friend--Stranger interaction shows a stronger muting in the Friend condition, because actually the Stranger condition is amplified: the first presentation of stranger produces a less  strong response than the second. This is a big difference to the repetition muting which is mainly observed and is likely to be related to recognition of that stranger's face from the first repetition.


\begin{table}[p]
\centering
\caption{\textbf{Results of pupil size LMMs.} The analysis was conducted using the same model for 5 time windows: Baseline (-50 -- +50ms), Constriction (200 -- 600ms), Dilation (600 -- 1200ms), Late Dilation (600 -- 1200ms), and Stability (2000 -- 3000 ms). The t values that are marked in red are considered significant.}
\label{tab:pupillmm}

\scalebox{0.7}{
\input{Tables/Table1_1.txt}
}

\scalebox{0.7}{
\input{Tables/Table1_2.txt}
}

\end{table}


% Figure environment removed

Using the terms computed by the LMM, we compare the effect size of the Self-Others comparison for individual subjects, as estimated by the random effects in the LMM. The results are plotted in Figure \ref{fig:subj_effect}. We find that the majority of subjects shows the effect in the expected direction. Some individuals show only a very minor expression of the effect and some even mildly reverse trend, i.e. exhibit less dilation in response to their own face. 


% Figure environment removed

\subsubsection{Microsaccades}

% Figure environment removed

We calculated the microsaccade rate for each subject and condition (Figure \ref{fig:ms_cat}.
The time course of microsaccades is best described as two interacting processes. First, simple display changes cause an inhibition of microsaccades \citep{Engbert2003}. As our paradigm involved two changes in display (image onset and image offset) both display changes independently cause the microsaccade rate to drop. Second, microsaccades can also be inhibited by cognitive factors such as exogeneous or endogeneous shifts of attention \citep{Engbert2006}. In a modeling study, \citet{Engbert2012} suggested that the combination of two processes, (i) the modulation of a threshold for triggering of microsaccades and (ii) a transient reduction of microsaccade, can explain most of the experimental findings for lower-level display changes and higher-level cognitive task manipulations.

As expected by previous research we find that the microsaccade rate responds strongly to display changes. An inhibition of microsaccades is clearly  visible in the rapid drop in microsaccade rate at image onset and image offset. Typically the microsaccade rate recovers quickly, rising to the baseline level. Figure \ref{fig:ms_cat} shows that in the Self condition microsaccades are inhibited for a longer period of time. In all other condition, the rate recovers to the baseline level before the second inhibition occurs at target offset. In the Self condition, the second inhibition is occurs while the first is still active, leading to a further drop in the microsaccade rate and a recovery that is correspondingly longer than in the other conditions. Thus the average time-course of the microsaccade rate is qualitatively different in the Self condition. Thus, our results are compatible with earlier findings on microsaccadic inhibition in the oddball paradigm \citep{Valsecchi2006}.

Table \ref{tab:ms_tests} shows the results of the Poisson rate tests.  The Estimate should be interpreted as the Ratio of the Poisson rates in the compared conditions, e.g.  the estimate of $1.23$ in the Return Phase when comparing Self and Others means that the rate in the Other conditions is higher by a factor of 1.23 than in the Self Condition. We find significant differences between Both Self and Other as well as between Friend and Stranger in the period between 300ms and 900ms.

\begin{table}[p]
\centering
\caption{\textbf{Results of the Microsaccade Poisson rate tests.} The test compared the Poisson rate of two groups. P-values marked in red are considered significant. The t values marked in red are considered significant.}
\label{tab:ms_tests}
\begin{table}
	\centering
	\caption{Characteristics of the unresolved central polarisation for the post-AGB binaries in the sample.}
	\label{tab:unresolved}
	\begin{center}
    	    
    	\begin{tabular}{|c|l|c|c|} 
    		\hline
    		 \#ID & Name & DoLP, \% & AoLP, $^\circ$ \\
    		\hline
    		\multicolumn{3}{c}{Full discs}\\
    		\hline
            1& U\,Mon & 1.63$\pm$0.066&178$\pm$50\\
            2& IRAS\,08544-4431& 0.63$\pm$ 0.09&148$\pm$ 4.3\\
            3& IW\,Car & 0.23$\pm$ 0.17&62$\pm$ 23\\
            4 & HR 4049 & 0.17$\pm$ 0.06& 79$\pm$ 12\\
            5&IRAS 15469-5311 & 1.2$\pm$ 0.06&55$\pm$ 1.4\\
            6&IRAS 17038-4815 & 2.2$\pm$ 0.13&38$\pm$ 2.6\\
            \hline
    		\multicolumn{3}{c}{Transition discs}\\
    		\hline
            7 & RU\,Cen  &0.54$\pm$ 0.17&62$\pm$ 9.9 \\
            8& AC\,Her& 0.35$\pm$ 1.03&69$\pm$ 50\\
            \hline
    	\end{tabular}
    \end{center}
	\begin{tablenotes}
        
    \small
\item \textbf{Notes:}  'DoLP' represents the degree of linear polarisation, 'AoLP' represents the predominant angle of linear polarisation for the unresolved central polarisation (see Section~\ref{sec:unresolved}). We note that for U Mon, results were derived using the observations taken on the 14th of January 2019 as it has better observational conditions (see Table~\ref{tab:weather} in Appendix~\ref{sec:ap_weather}).\\
    \end{tablenotes}
	
\end{table}
\end{table}

\subsection{Relationship of Microsaccades and Pupil Size}
We calculated a model using the presence of microsaccades in the most diagnostic window, as found by the previous analysis, as a predictor for pupil size.  The result of the LMM is presented in Table \ref{tab:pupilmscorr}. First, consistent with the LMMs of Pupil size alone, we find a significant effect of both Self--Other and Friend--Stranger in the Dilation, Late Dilation and Stability time widows. Second, a predictive effect of microsaccades is present only in the earlier phases of the pupil response (Constriction, Dilation). 

\begin{table}[p]
\centering
\caption{\textbf{Estimated terms of the LMM relating microsaccades and pupil size.} Occurrence of microsaccades in the most diagnostic window (300–600~ms) is used as a predictor for pupil size. The predictive effect of microsaccades can only be found in the early phases of the pupil response.}
\label{tab:pupilmscorr}
\scalebox{0.7}{
\input{Tables/Table3_1.txt}
}
\scalebox{0.7}{
\input{Tables/Table3_2.txt}
}

\end{table}
\color{black}
\section{Discussion}
In this study we set out to explore subconscious responses of the eye when participants are looking at their own face or at other persons' faces. In a large-scale eye tracking experiment we recorded  pupil size and eye movements of two groups of participants, two high school classes, where each participant knew the members of their own class (Friends) but not the members of the other (Strangers). Participants saw photographs of the Friend and Stranger groups as well as their own face (Self) in a randomized order, while their eye responses were measured.  In this way the design of the study yields a very neatly controlled data set, as each image appeared for some participants as a Stranger, to some as a Friend and to one individual as the Self condition. We find that both the microsaccade rate and the pupil dilation are affected differently by the Self condition compared to the Friend and Stranger conditions. Moreover, we find differences between the friend and the stranger condition, showing an effect of familiarity in both microsaccade rate and pupil size. Pupil dilation is further influenced by stimulus repetition. 

The change in pupil size is characterized by an initial contraction in response to the display change, followed by a dilation. The  contraction is stereotypical and likely related to the PLR. Dilation, by contrast is modulated by cognitive components (peaking at 1-2s), and emotional effects (peaking after 2s). Consistent with these findings, we find an increased dilation in response to the Self condition, as compared to the other conditions starting in the Dilation phase of the pupil response and growing in effect size over time. We find largely the same effects, but at a smaller effect scale, in the comparison of Strangers and Friends. The recognition paired with the related following associative and emotional effects modulate attention in much the same way as an Oddball stimulus in a detection task. An interesting difference is that here, the recognition is involuntary and not related to task demands. Thus this oddball effect is entirely endogeneneous and reflexive. 


The main effect of stimulus repetition is also significant and shows that further repetitions tend to lead to a reduced dilation. Note that the statistical analysis always compares adjacent repetitions, in order to understand the full extent of this effect we refer to \ref{fig:pupil_cat}. The strongest effect is found in the Late Dilation and Stability Phases. It is perhaps surprising, as neurophysiologically, the novelty response occurs before other attentional effects. Most likely the pupil does not capture the same sort of novelty effect as is discussed in neurophysiology. The P300 component occurs 300~ms after target onset, while the pupil constriction only starts around this time. 
Another noteworthy observation that illustrates the decrease of the response is that the 10th Self presentation causes approximately an average dilation for a stranger (see Figure \ref{fig:pupil_cat}). On its own, this fact seems to indicate that for application purposes, where the picture may be the same over many repetitions. However, this is not true of the early response: in the Dilation phase the faster and stronger dilation in the Self condition is still evidently different from the average Stranger condition.
Whether this effect is short or long-term, i.e. whether it would persist over sessions is an interesting question that remains unanswered.



Several interaction effects of Face and Repetition are noteworthy. The strongest Interaction term is found in the  (Se-O) : (1st-2nd) comparison in the Stabilit time window. The first response to seeing one's own face is comparably much larger than the response of any other condition and decreases particularly strongly on the following repetition. 
Moreover it is interesting to consider the "(F:St) : (1st-2nd)" interaction in the Late Dilation phase. As can also be seen in figure \ref{fig:pupil_cat}, when a stranger's face is repeated, i.e. the first \textit{recognition} of that stranger, the dilation is greater than during the first viewing. This comparison shows the size of the pure recognition of a face, presumably in the absence of any emotional reaction.



Microsaccades tend to be inhibited by display changes, but also by higher cognitive processes such as selective attention. We observe an inhibition following both target onset and offset for all conditions. However in the Self condition this inhibition is both stronger and longer. While in both other conditions the microsaccade rate returns to baseline between the display changes (i.e., image onset and offset), in the Self condition, the second inhibition occurs while the first is still active, and seem to additively combine to inhibit the rate further. We find significantly different microsaccade rates in the time window of 300-900 ms after target onset. 

Our findings are consistent with existing research concerning the oddball paradigm. The oddball stimulus (or target) has been found to increase dilation \citep{Strauch2020} and inhibit microsaccades \citep{Valsecchi2006, Valsecchi2009}. In the context of this study we consider the Self condition to act as an oddball stimulus. We assume that one's own face is seen as qualitatively different from the faces of other individuals. Particularly the faces of strangers are likely to be less salient and capture less attention and to be grouped together. The greater dilation of the pupil and greater inhibition of microsaccades in response to one's own face can be interpreted as a marker of greater attention, consistent with previous research.

At the outset of this study we posed the question of whether it is possible to distinguish self- and other-recognition using involuntary eye movement measures. We find strong differential effects of face recognition on pupil size and microsaccades.  Such reaction differences may present a good basis for applications in the context of security, such as authentication or exposing hidden knowledge.

However, while these effects appear robust when averaged over a large data set, we also observe a high individual variability. We speculate that strong differential reactions to the face stimuli may be related to self-image.
It remains to be seen, whether differences between self and other recognition can be usefully discerned at the level of individual trials. Moreover the effect decreases with presentation repetition, raising the question of whether, in an application context, where the self-recognition is wholly expected and the specific picture is always the same, the differences would stay large enough to detect.
We also find that the microsaccade rate effect and the pupil size effect appear to be only mildly correlated. In other words, trials that have particularly few microsaccades do not serve well as a predictor for particularly large dilation. This could confer a positive effect on models using a conjunction of both features for its prediction, as they capture different aspects of the recognition process. 








We set out to investigate the research question of whether unconscious eye responses systematically vary depending on whether observers are seeing their own faces, a Stranger's face, or a Friend's face. The differences in responses which we report here are consistent with prior research.  Typically, oddball targets, i.e., qualitatively different or unexpected stimuli, evoke a greater dilation and microsaccade inhibition. Due to the high inter-trial and inter-individual variability this trend is mainly visible when averaged over several participants in a large data set. 
Our results point toward various potential applications in the context of identification, security, or exposing hidden knowledge.
\printbibliography

\appendix
\renewcommand{\thefigure}{\Alph{section}\arabic{figure}}
\renewcommand{\thetable}{\Alph{section}\arabic{figure}}
\setcounter{figure}{0}
\setcounter{table}{0}

\section{Appendix}

\subsection{LMM Random Effects Structure}
\label{appx:sec:lmm}
We used Linear mixed effect models (LMMs) to analyze the pupillometry data in 5 time windows. In all time windows we employed the same fixed- and random effect structure. Random effects were estimated for both participants and images, but mainly with the aim of correcting for their influence. We initially formulated a hypothesis-driven random effect structure, including random intercepts for both participants and images, and random slopes for each the Faces comparisons. We included the Repetition comparisons as random slopes for each Subject (but not each image, as there is not all images were necessarily seen in each repetition). We also excluded random interaction slopes. Model reduction was then performed iteratively until convergence was achieved without issues. Specifically, correlation terms were first removed, followed by the least varying random effect terms. In cases where two converging models with different random effect terms were obtained, we used Bayesian-Information-Criterion (BIC) to select the best model. To ensure that none of the models were degenerate, a principal component analysis was performed on the random effect terms \citep{Bates2015a}. The final model structure we arrived at is as follows,
\begin{equation}
+ (1+Face+Rep2+Rep3|VP) + \\
 (1+Face(Se-O)|Img).
\end{equation}


We performed the same procedure for the random effects structure of the explorative LMM that explores the relationship between microsaccade occurrence and pupil size. Here we included random intercepts by subject. As the predictor variable was ad-hoc, group sizes were uneven and a random effect of image was excluded. After the model reduction procedure we arrived at a model which defines only a random intercept for each subject. 

\begin{equation}
+ (1|VP).
\end{equation}


\subsection{Controlling for Knowing}
After the experiment subjects were shown all faces one more time. They were asked to indicate by pressing one of 3 buttons on a \emph{ViewPixx} button box whether they know the person in picture, did not know them, or know them very well. With this information we can exclude trials with subjects what coincidentally did know each other across the schools. It also allows potential exploratory analyses using the distinction of close friends versus acquaintances.

% Figure environment removed

\subsection{Controlling for Self Image}
For the purposes of an exploratory analysis, we asked three questions that were to be answered on a 5 point scale, i.e.,
\begin{enumerate}
\itemsep0ex
\item How did you like the photographs in general? \emph{original: Wie fanden Sie die Fotos allgemein?}
\item How did you like the picture of yourself \emph{original: Wie fanden Sie das Foto von sich selbst?}
\item How happy are you with your own appearance? \emph{original: Wie glücklich sind Sie mit Ihrem eigenen Erscheinungsbild allgemein?}
\end{enumerate}

\noindent Questions 1 and 2 were to be answered with
\begin{enumerate}
\itemsep0ex
\item Liked very much \emph{original: Sehr gelungen}
\item Liked \emph{original: Gelungen}
\item Neutral \emph{original: Durchschnittlich}
\item Did not like much \emph{original: Eher nicht gelungen}
\item Did not like \emph{original: Nicht gelungen}
\end{enumerate}

\noindent and question 3 with
\begin{enumerate}
\itemsep0ex
\item Very happy \emph{original: Sehr glücklich}
\item Happy \emph{original: Glücklich}
\item Neutral \emph{original: neutral}
\item Rather unhappy \emph{original: Eher unglücklich}
\item Unhappy \emph{original: unglücklich}
\end{enumerate}




\end{document}

