% mnras_template.tex 
%
% LaTeX template for creating an MNRAS paper
%
% v3.0 released 14 May 2015
% (version numbers match those of mnras.cls)
%
% Copyright (C) Royal Astronomical Society 2015
% Authors:
% Keith T. Smith (Royal Astronomical Society)

% Change log
%
% v3.0 May 2015
%    Renamed to match the new package name
%    Version number matches mnras.cls
%    A few minor tweaks to wording
% v1.0 September 2013
%    Beta testing only - never publicly released
%    First version: a simple (ish) template for creating an MNRAS paper

%%%%%%%%%%%%%%%%%%%%%%%%%%%%%%%%%%%%%%%%%%%%%%%%%%
% Basic setup. Most papers should leave these options alone.
\documentclass[fleqn,usenatbib]{mnras}

% MNRAS is set in Times font. If you don't have this installed (most LaTeX
% installations will be fine) or prefer the old Computer Modern fonts, comment
% out the following line
\usepackage{newtxtext,newtxmath}
% Depending on your LaTeX fonts installation, you might get better results with one of these:
%\usepackage{mathptmx}
%\usepackage{txfonts}

% Use vector fonts, so it zooms properly in on-screen viewing software
% Don't change these lines unless you know what you are doing
\usepackage[T1]{fontenc}

% Allow "Thomas van Noord" and "Simon de Laguarde" and alike to be sorted by "N" and "L" etc. in the bibliography.
% Write the name in the bibliography as "\VAN{Noord}{Van}{van} Noord, Thomas"
\DeclareRobustCommand{\VAN}[3]{#2}
\let\VANthebibliography\thebibliography
\def\thebibliography{\DeclareRobustCommand{\VAN}[3]{##3}\VANthebibliography}


%%%%% AUTHORS - PLACE YOUR OWN PACKAGES HERE %%%%%

% Only include extra packages if you really need them. Common packages are:
\usepackage{graphicx}	% Including figure files
\usepackage{amsmath}	% Advanced maths commands
\usepackage{subcaption}
\usepackage{xspace}
% \usepackage{amssymb}	% Extra maths symbols

\newcommand{\bill}[1]{{\textcolor{red} {#1}}}
%%%%%%%%%%%%%%%%%%%%%%%%%%%%%%%%%%%%%%%%%%%%%%%%%%

%%%%% AUTHORS - PLACE YOUR OWN COMMANDS HERE %%%%%

% Please keep new commands to a minimum, and use \newcommand not \def to avoid
% overwriting existing commands. Example:
%\newcommand{\pcm}{\,cm$^{-2}$}	% per cm-squared
\newcommand{\jwst}{\textit{JWST}\xspace}
\newcommand{\hst}{\textit{HST}\xspace}

%%%%%%%%%%%%%%%%%%%%%%%%%%%%%%%%%%%%%%%%%%%%%%%%%%

%%%%%%%%%%%%%%%%%%% TITLE PAGE %%%%%%%%%%%%%%%%%%%

% Title of the paper, and the short title which is used in the headers.
% Keep the title short and informative.
\title[Globular Clusters in Abell 2744 from JWST]{\textit{JWST} Photometry of Globular Cluster Populations in Abell 2744 at $z=0.3$}

% The list of authors, and the short list which is used in the headers.
% If you need two or more lines of authors, add an extra line using \newauthor
\author[W.~E.~Harris and M.~Reina-Campos]{
William E. Harris$^{1}$\thanks{E-mail: harris@physics.mcmaster.ca} and 
Marta Reina-Campos$^{1,2}$\thanks{E-mail:reinacampos@mcmaster.ca}
\\
% List of institutions
$^{1}$Department of Physics \& Astronomy, McMaster University, 1280 Main Street West, Hamilton, L8S 4M1, Canada\\
$^{2}$Canadian Institute for Theoretical Astrophysics (CITA), University of Toronto, 60 St George St, Toronto, M5S 3H8, Canada\\
}

% These dates will be filled out by the publisher
\date{Accepted XXX. Received YYY; in original form ZZZ}

% Enter the current year, for the copyright statements etc.
\pubyear{2023}

% Don't change these lines
\begin{document}
\label{firstpage}
\pagerange{\pageref{firstpage}--\pageref{lastpage}}
\maketitle

% Abstract of the paper
\begin{abstract}
\textit{JWST} imaging of the rich galaxy cluster Abell 2744 at $z=0.308$ has been used by the UNCOVER team \citep{bezanson22} to construct mosaic images in the NIRCAM filters. The exceptionally deep images in the ($F115W$, $F150W$, $F200W$) bands reveal a large population of unresolved pointlike sources across the field, the vast majority of which are globular clusters (GCs). To the limits of our photometry, more than 10,000 such objects were measured, most of which are in the halos of the five largest A2744 galaxies but which also include GCs around some satellite galaxies and throughout the IntraCluster Medium.  Their luminosity function follows a lognormal shape, with the data reaching to within one magnitude of the classic GCLF turnover point. The colour index  ($F115W-F200W$) in particular covers a range of $0.5$ mag, clearly resolving the expected internal spread of GC metallicities. The estimated GC masses are systematically higher than in present-day galaxies, consistent with a large, normal GC population seen at a $3.5~$Gyr earlier stage of dynamical  evolution. Lastly, the spatial distribution of the bluer (more metal-poor) GCs resembles the gravitational lensing map of the cluster, consistent with recent theoretical suggestions.
\end{abstract}

% Select between one and six entries from the list of approved keywords.
% Don't make up new ones.
\begin{keywords}
galaxies: clusters -- galaxies: star clusters -- globular clusters
\end{keywords}

%%%%%%%%%%%%%%%%%%%%%%%%%%%%%%%%%%%%%%%%%%%%%%%%%%

%%%%%%%%%%%%%%%%% BODY OF PAPER %%%%%%%%%%%%%%%%%%

\section{Introduction}

Abell 2744 (Pandora's Cluster), an extremely rich cluster of galaxies at redshift $z=0.308$, has been the target for intensive imaging campaigns with \hst (\textit{Hubble Space Telescope}) and now more recently with \jwst (\textit{James Webb Space Telescope}). As a massive lensing cluster and one of the six Hubble Frontier Fields \citep{lotz+2017}, it is an attractive testbed for detection and deep imaging of high-redshift galaxies and for gravitational lensing phenomena in general.  

The UNCOVER project team \citep{bezanson22} has made publicly available a set of deep mosaic images of A2744 generated from the \jwst NIRCAM, \hst ACS, and \hst WFC3 cameras. The central part of this field is shown in Figure \ref{fig:field}.  This material has been used to extract a catalog of 50,000 galaxies in the region \citep{weaver+2023}.  Close inspection of the NIRCAM SWC (Short Wavelength Channel) mosaic images also shows the presence of rich populations of unresolved point sources especially concentrated around the major galaxies in the cluster.  These are the globular clusters (GCs) that are found around every luminous galaxy \citep[e.g.][]{harris2015}, along with some probable Ultra-Compact Dwarfs (UCDs).  A small portion of the field is shown in Figure \ref{fig:sample} to illustrate the presence of these starlike objects.   

The \textit{Hubble Space Telescope} (\hst) has been a powerful instrument for surveying GC populations around galaxies out to limits of roughly twice the Coma cluster distance, i.e.~$\lesssim 200$ Mpc or $z \lesssim 0.05$ \citep[e.g.][]{peng+2011,lim+2018,madrid+2018,amorisco+2018,harris2023,dornan_harris2023}.  For all these cases, cosmological lookback times are $\lesssim 0.6$ Gyr and we are essentially observing within just the Local Universe. In a few rare cases with very long exposures, \hst photometry has been obtained for GCs in more distant systems including an unnamed elliptical galaxy at $z=0.09$ \citep{kalirai+2008}, the rich cluster A1689 at $z=0.18$ \citep{mieske+2004,alamo-martinez13,alamo-martinez+2017}, and A2744 itself \citep{blakeslee+2015,lee_jang2016}.   
Even so, only the very brightest $\sim 1$ magnitude or so of the GC population is recovered with any confidence.  The combination of \jwst and NIRCAM is capable of reaching much deeper.

At the redshift of A2744 the lookback time is $3.5$ Gyr, a quarter of the way back to the very first stages of galaxy formation.  The exceptionally deep images from \jwst give us an opportunity to observe very directly the characteristics of entire GC systems at an earlier stage in their evolution.  The ability to study large populations of GCs like this is entirely complementary to recent \jwst observations of objects like the Sparkler, a lensed galaxy at $z=1.38$ in which a handful of \emph{individual} GCs at even earlier stages of evolution have been {}seen \citep{mowla+2022,claeyssens+2023,forbes_romanowsky2023,adamo+2023}. 

An especially interesting new question is the link between the spatial distribution of GCs and the dark matter (DM) distribution of the galaxy halos and the cluster they are embedded within. Recent theoretical studies \citep[e.g.][]{alonso+2020,reina-campos+2022,reina-campos+2023} reinforce the view that the GC distributions and the IntraCluster Light can be used to recover the DM profile. Observational tests of such links include the large-scale survey of the Virgo cluster \citep{durrell+2014} and \jwst imaging of the cluster SMACS J0723.3-7327 at $z=0.39$ \citep{lee22,diego23}. Carrying out this kind of comparison is 
opened up by the ability to measure large-scale spatial distributions of GCs in clusters at significant redshifts, an observational parameter space well suited to \jwst.

In the present study, we describe photometry of the A2744 NIRCAM SWC images directed specifically at isolating the pointlike sources in the field, which are dominated by the GCs around the member galaxies. Here, we present the measurements and give a brief overview of the system. In following studies, we will analyze the GC spatial distribution and their distributions in metallicity and luminosity more quantitatively. Section \ref{sec:data} describes the data and the photometric measurement steps; Section \ref{sec:colour-distribution} presents the GC colour distribution function and a brief analysis; and Section \ref{sec:luminosity-function} presents the luminosity distribution. Section \ref{sec:adjustments-z0} gives a brief discussion of the adjustments necessary to normalize the data to zero redshift. Finally, Section \ref{sec:summary} gives a summary of our findings and prospects for additional work.

We adopt cosmological parameters $H_0 = 67.8$ km s$^{-1}$ and $\Omega_{\Lambda} = 0.692$, which for $z = 0.308$ gives a luminosity distance $d_L = 1630$ Mpc or $(m-M)_0 = 41.06$ for A2744.  The foreground extinction (from the NED database) is $A_V = 0.036$, which has a negligible effect for the NIR bandpasses used in the present study.

\section{Data and Photometry}\label{sec:data}

The mosaic images of the A2744 field and their construction are thoroughly described in \citet{bezanson22}.  We obtained the NIRCAM SWC images in $F115W$, $F150W$, and $F200W$, all three of which had closely similar depths (limiting magnitudes) for objects like GCs with intermediate colours. The $F090W$ image does not cover as large an area and has shallower depth, and was not used here. Similarly, the NIRCAM images from the Long Wavelength Channel (LWC), with their larger pixel scale and lower resolution, also did not reach sufficient limiting magnitudes for our purposes and were not used. The mosaics, built from NIRCAM exposures in three different programs (GO-2561, ERS-1324, DD-2767), have an irregular overall shape with different total exposure times in various regions \citep[see particularly Fig.~2 of][]{bezanson22}, but the central region thoroughly covering the cluster core is included by all the programs. Our analysis will concentrate on that central region.

The mosaic images in $F115W$, $F150W$, and $F150W$ have a resampled pixel size of $0.02''$, smaller than the native SWC camera scale of $0.03078''$ per pixel. They are reoriented to align with the cardinal directions with EW along the x-axis and NS along the y-axis. Direct measurements of unsaturated stars in the images give PSF (point spread function) sizes of \emph{fwhm} $= 1.9$ px ($F115W$), 2.4 px ($F150W$), and 3.0 px ($F200W$). Total exposure times from all the different programs combined were 162 hours ($F115W$), 130 hours ($F150W$), and 95 hours ($F200W$).

Photometry was carried out with the tools in \texttt{daophot} \citep{stetson1987} in their IRAF implementation.  The first step was to add the images in the three filters to produce a very deep stacked image for object finding. On this coadded image, we used the \emph{daofind} package to generate a finding list of about $10^6$ candidate objects.  Aperture photometry with \emph{phot} and a 3-px ($0.06''$) radius was then carried out with that list on each filter.

The last stage of the photometry is PSF fitting to the object list. Normally, bright but unsaturated foreground stars can be used for constructing a PSF, but they are rare in this high-latitude field. However, the GC population in A2744 itself provides a sample of unresolved (starlike) objects; they are numerous, and the brightest are more than three magnitudes above the photometric limits. At the distance of A2744, a typical GC with a half-light diameter $2 r_h \sim 5$ pc has an angular diameter of $0.00063'' =$ 0.03 px on these images, two orders of magnitude below the resolution limit. Even UCDs with their scale sizes that can be up to $10 \times$ larger \citep[e.g.][]{drinkwater+2000,dabringhausen+2008} can be expected to be unresolved or at worst marginally resolved. About 100 such objects scattered across the central region of the field were carefully selected and confirmed by radial profile measurement; the great majority of these are the brightest GCs themselves. The \emph{pstsel} and \emph{psf} tools were then used to construct an empirical PSF from them for each filter. PSF fitting was then done on the complete photometry list with \emph{allstar}. 
  
Lastly, the lists of output photometry from the three filters were matched by position to within 2 px, accepting only objects that were detected in \emph{all three} filters. Though the three images have very similar depths, the triple-match requirement is useful for rejecting a large proportion of false detections including pixel artifacts, noise clumps, and the like. 

As a final step for reducing sample contamination, the \texttt{daophot} \emph{sharp} parameter was used to reject remaining objects that were not starlike (the rejected objects are mainly small, faint background galaxies, severely crowded objects, or pixel-scale artifacts that do not match the PSF profile) \citep[see][for additional discussion]{harris2009,harris2023}. 

Final measured magnitudes were calibrated to the ABMAG system following \citet{weaver+2023}. The fluxes per pixel on the mosaic images are in units of 10 nJy, from which the magnitudes are defined as
\begin{equation}
    m_{AB} = -2.5 \textrm{log}(f/\textrm{10 nJy}) + 28.90
\end{equation}
where $f$ is the total flux of the object.  We used the PSF encircled-energy tables as published on the NIRCAM webpages 
%\footnote{\texttt{https://jwst-docs.stsci.edu/jwst-near-infrared-camera/nircam-performance/nircam-point-spread-functions}} 
to help estimate the necessary aperture corrections from the the PSF-fitted instrumental magnitudes to large radius:  direct aperture photometry of the isolated PSF stars on the images was used to determine the step from the PSF-fitted magnitudes to a radius of $r=12.5$ px ($0.25''$) that encloses 85\% of the total profile light. Lastly, a further $ -0.18$ mag was added to step to `infinite' radius. 

% Figure environment removed

% Figure environment removed

After the photometry and culling steps, the list of GC candidates consists of 10617 objects. A final close visual inspection of these objects on the images showed that the list included virtually all starlike objects that are clearly detectable, while obviously nonstellar ones were almost all excluded. The final colour-magnitude diagrams (CMDs) for these matched, starlike sources appear in Figure \ref{fig:cmd}, where $F150W$ is plotted versus two different colour indices.\footnote{The second panel of Fig.~\ref{fig:cmd} has the unusual property for a CMD that the x and y axes are independent of each other.} Their spatial distribution across the central field of A2744 is shown in Figure \ref{fig:field}b.  The concentration of the objects around the five biggest galaxies is very evident, but smaller galaxies can be spotted as well, and many GC candidates are also scattered more widely through the ICM (IntraCluster Medium).

To gauge the effective photometric limits of the data and the internal measurement uncertainties, artificial-star tests were run. With the \texttt{daophot} \emph{addstar} tools, 16000 fake stars covering the magnitude range $26 - 34$ were inserted into the three images in series so that no change in the crowding levels on the images would result (though the absolute crowding level is quite low to begin with). The synthesized frames were remeasured with exactly the same procedure as described above.  The resulting recovery completeness is shown in Figure \ref{fig:f}, where the recovery fraction $f$ is defined as the number of artificial stars measured in a given $0.2-$mag bin, divided by the number of stars originally inserted into that bin.  Defined this way, the ratio $f$ implicitly includes the effects of bin-jumping; i.e.~a measured star falling within the bin may have originally been assigned to another adjacent bin \citep{alamo-martinez13}.  Note as well that in Fig.~\ref{fig:f}, $f$ is plotted versus the measured (i.e., recovered) magnitude. For $f > 0.2$, the trend is accurately matched by a modified hyperbolic tangent function 
\begin{equation}
  f(m) = \frac{2}{1+\exp\left[{\beta (m-m_1)}\right]} - 1  
  \label{eq:f}
\end{equation}
with $\beta = 1.70, m_1 = 30.28$.\footnote{This function has the unphysical property that $f$ becomes negative for $m > m_1$, but its most important role is to match the bright end and the downturn region.  No part of our analysis depends on data fainter than the $50~$per cent completeness level.}  The $50~$per cent recovery completeness point is at $F150W$ = 29.63, and $80~$per cent at $F150W$ = 28.98.  The artificial-star tests were also used to gauge the trend of photometric measurement uncertainty $\sigma$ with magnitude, as shown in the second panel of Fig.~\ref{fig:f}.  A simple exponential curve $\sigma(m) \simeq 0.075 \exp\left[{0.7(m-29.0)}\right]$ describes the trend well in each filter, to the useful limits of the data.

\section{Colour Distributions}\label{sec:colour-distribution}

The integrated colours of GCs in the NIRCAM filters are relatively new in this subject, with little available data. By comparison, there are available observations in many nearby galaxies of \hst photometry of GCs in several ACS and WFC3 filters \citep[e.g.][]{harris2023}.
GC colours depend primarily on the cluster \emph{metallicity} and \emph{age}; increases in both quantities drive the cluster to redder colours. However, as will be seen below, the NIRCAM colour indices in adjacent filters such as $(F150W-F200W)$ and $(F115W-F150W)$ are not strongly sensitive to metallicity, so for the CMDs in these colours we should expect to find that old GCs delineate a narrow, vertical sequence exactly like what is seen in Fig.~\ref{fig:cmd}a-b.  The colour $(F115W-F200W)$, however, is about twice as metallicity-sensitive and the GC sequence is distinctly broader (Fig.~\ref{fig:cmd}c).

% Figure environment removed

A basic test of the observed range of colours is shown in Figure \ref{fig:cdf}, displaying the histograms over the magnitude range $27.0 < F150W < 28.5$ where the photometry is the most precise. In this range the recovery completeness is higher than $90~$per cent. From many previous studies drawn primarily from optical colour indices, we expect the GCs to fall into two identifiable metallicity groups, `blue' (metal-poor) and `red' (metal-rich) with a double-Gaussian model usually providing an accurate match to the colour distribution function (CDF) \citep[e.g.][among many others]{larsen+2001,peng+2006,harris2009,faifer+2011,brodie+2012,cho+2012,hargis_rhode2014,harris2023}.  For either the index $(F150W-F200W)$ or $(F115W-F150W)$, the observed total colour range is no more than $\sim 0.2$ mag and is not much larger than what would be produced just by random measurement scatter. In Fig.~\ref{fig:cdf}a, the dashed line shows the shape of the CDF expected if the intrinsic colours of the GCs were dispersionless and the broadening was equal to the $1 \sigma$ colour uncertainty at $F150W=28.0$, i.e.~the \emph{maximum} colour dispersion over the given magnitude range. For comparison, the solid line gives a double-Gaussian fit to the CDF determined from the GMM fitting code \citep{muratov_gnedin2010}; it is scarcely different from the single-Gaussian error distribution, so there is no compelling evidence for more than one component.

The broader colour index $(F115W-F200W)$, however, gives a different story (Fig.~\ref{fig:cdf}b).  The total width of the CDF is $\sim 0.5$ mag and the GMM fit disfavors a single-Gaussian solution at $90~$per cent confidence.  The rms spread of the measured colours is $\pm 0.12$ mag, while the measurement uncertainties alone would yield a dispersion of only $\pm 0.05$ mag (dashed line in the Figure). The best-fit double Gaussian solution gives ($\mu_1 = 0.164\pm0.038, \sigma_1 = 0.046\pm0.024$) for the bluer component and ($\mu_1 = 0.328\pm0.066, \sigma_2 = 0.117\pm0.020$) for the redder component, with a red-component fraction $f_r = 0.89\pm0.21$;  their sum is shown as the solid (magenta) line. But a simpler homoscedastic fit (enforcing equal variances) gives ($\mu_1 = 0.244, \mu_2 = 0.393$) and ($\sigma_1 = \sigma_2 = 0.098$), with $f_r = 0.44$. The quality of fit between the two solutions is quite comparable. The main conclusions drawn from this exercise are that (a) the intrinsic width of the CDF is resolved by the current data; i.e.~its spread is distinctly larger than can be accounted for by measurement scatter alone; and (b) a conventional double-Gaussian model can provide an accurate match to the CDF, though the blue/red fraction in particular is sensitive to the assumptions made about the parameters.  

In Figure \ref{fig:color}, theoretically predicted trends of colour versus metallicity are shown for the three indices used here. These tracks have been generated with the PARSECv1.2 CMD3.7 stellar models\footnote{http://stev.oapd.inaf.it/cgi-bin/cmd} \citep{bressan+2012} for three different ages (12.5 Gyr, 9.0 Gyr, 7.0 Gyr), and adopting the standard \citet{kroupa2001} IMF. Conversions of these models from their normal output Vegamag magnitude scale to the ABMAG scale have been applied\footnote{https://jwst-crds.stsci.edu}. For star clusters in this age range, colour is insensitive to age, even more so than for optical colour indices.

The best-fit single Gaussians to each of the CDFs give $\mu=0.088, \sigma=0.072$ for $(F150W-F200W)$, and $\mu = 0.310, \sigma=0.078$ for $(F115W-F200W)$.  After K-corrections to the colours (See \S\ref{subsec:k-correction} below), these mean colours adjusted to zero redshift become $\mu_0(F150W-F200W) = -0.16$ and $\mu_0(F115W-F200W) = +0.06$.  In both cases, comparison with Fig.~\ref{fig:color} would give mean metallicities near [m/H] $\sim 0$, though the comparison should be viewed with caution since it depends on all three of the K-correction accuracies, the absolute accuracy of the photometry including the aperture corrections, and the zeropoints of the model colour scales themselves.  The \emph{range} of colours may be a more robust comparison with the models.  These are equal to $\Delta(F150W-F200W) \simeq 0.2$ mag and $\Delta(F115W-F200W) \simeq 0.5$ mag (Fig.~ \ref{fig:cdf}).  These correspond to an end-to-end range of 2.0 dex in metallicity, as observed for the GCs in most large galaxies \citep{peng+2006,harris2023}.   

One additional comparison can be made directly with other \jwst photometry: \citet{lee22} have measured the brightest $\sim 1$ mag of the GC population in the $z=0.39$ cluster SMACS J0723.3-7327.  The mean colour (dereddened, but not K-corrected) in $(F150W-F200W)_{0,AB}$ for their sample is in the range $\simeq 0.05-0.15$ depending on location, in close agreement with our results for the A2744 population. The brightest GCs in their data are near $F200W_0 \simeq 27$, again very consistent with our A2744 CMD given that SMACSJ0723 is 0.6 mag more distant.

% Figure environment removed

% Figure environment removed

% Figure environment removed

% Figure environment removed

% Figure environment removed

In Fig.~\ref{fig:cmd}, just to the left of the major GC sequence, a thin sequence of bluer objects can be seen, in the colour range $-0.3 < (F150W-F200W) < 0.0$ and magnitude range $25 < F150W < 27$.  Close inspection of these objects on the images shows that almost half of them are in very distant background galaxy pairs or groups, and some others are located in the bright bulge light of large galaxies. Some of them may be compact star-forming regions. About 20 of them are isolated, unresolved objects scattered around the field without showing any obvious connection with the A2744 galaxies; their distribution across the field is shown in Figure \ref{fig:xyblue}. If guilt by association is relevant, then these may simply represent other background objects such as remote, high$-z$ galaxy nuclei that were able to pass through the photometric culling steps. Bright, unresolved sources sitting to the blue side of the main GC sequence might also be young massive star clusters that would mark much more recent star formation, such as those seen in the embedded filamentary structure in the  Perseus giant NGC 1275 \citep{holtzman+1992,canning+2014,lim+2020}. However, we find no noticeable numbers of such young blue GCs in the current A2744 data.

The bright end of the GC distribution extends upward to high luminosity particularly on the red side of the CDF. The spatial distribution of the luminous, red objects $(F150W < 27.5)$ is shown as well in Fig.~\ref{fig:xyblue}. The majority of these objects group closely around the major galaxies, consistent with their being part of the GC/UCD population.

Assuming that GC colour represents metallicity, then an additional test can be made in terms of their spatial distribution. Much previous photometry for GC systems in large galaxies shows that the redder, more metal-rich GCs have more concentrated, steeper radial profiles around their parent galaxies than do the metal-poor clusters \citep{zinn1985,geisler+1996,bassino+2008,faifer+2011,liu+2011,durrell+2014,harris+2017}.  The A2744 field of view is so large in linear extent (many hundreds of kpc) that this feature can be tested for several of its galaxies at once.  In Figure \ref{fig:xypair}, the A2744 sample has been subdivided into a `blue' group with $(F115W-F200W) < 0.31$ (the midpoint of the CDF as described above), and a `red' group redder than that dividing line. Only objects with $F150W < 29.0$ are included to guarantee high completeness. As seen in the figure, the bluer clusters have shallower, more extended distributions around the major galaxies, and are also found in larger numbers scattered across the ICM than are the redder ones.
The distribution map for the bluer GCs should be closer to marking out the gravitational potential field and the X-ray gas within the cluster \citep{furtak22,lee22,reina-campos+2023,diego23}. In followup work, correlations between the GC distribution, the X-ray light, and the lensing map will be investigated more quantitatively.

\section{Luminosity Function}\label{sec:luminosity-function}

For normal old GCs, the luminosity function (GCLF) follows a lognormal shape with a peak $m_0$ and dispersion $\sigma$ that are known to vary slowly with host galaxy mass \citep{jordan+2007,villegas+2010,harris+2014}. The GCLF peak (turnover) point is near $M_I \simeq -8.3$ and the dispersion $\sigma \simeq 1.2$ mag \citep{harris+2014}, but the turnover luminosity and dispersion both increase slightly with galaxy mass. In the A2744 galaxies, the apparent magnitude $m_0$ of the GCLF turnover point is expected to be fainter than the detection limits of the images, but the photometry penetrates very much deeper than did the previous \hst photometry and it should be possible to make a basic test of the GCLF shape.

Figure \ref{fig:lf} shows the observed numbers of objects as measured in $F150W$.  The basic lognormal shape provides a valid description for the brighter magnitude range, but for $F150W \gtrsim 29$, the recovery completeness starts having an important effect on reducing the curve amplitude. If the data fall well short of the true GCLF turnover point, then the solutions for $m_0$ and $\sigma$ become correlated \citep{hanes_whittaker1987} and both are increasingly uncertain. In Fig.~\ref{fig:lf}, the solid magenta curve shows a solution for $m_0 = 30.79 \pm 0.2, \sigma=1.2 \pm 0.1$, while the dashed curve shows the same lognormal function multiplied by the completeness curve described in Section \ref{sec:data} above.  The solution fits the observed counts well, but the estimated GCLF turnover point is more than a magnitude fainter than the $50~$per cent completeness level. At any magnitudes fainter than $F150W \gtrsim 30$, the observed counts are steeply choked off by the recovery incompleteness of the photometry (Fig.~\ref{fig:f}), and all estimated quantities including the completeness function itself become very uncertain. We view this comparison primarily as a consistency test that the lognormal LF shape provides a reasonable match to the data.

The estimated dispersion $\simeq 1.2$ mag is slightly smaller than the typical $\sigma \simeq 1.3-1.4$ mag for GCs in massive galaxies at zero redshift \citep{harris+2014}.  However, the lookback time for A2744 is large enough that some luminosity evolution of the GCLF shape should be expected. Simulations of evolving GC systems within Milky-Way-sized galaxies with EMP-\textit{Pathfinder} \citep{reina-campos+2022b} show that by an age of $\sim 6-7$ Gyr, the basic lognormal shape of the GCLF is well established, but the peak point and spread (dispersion) continue to evolve slowly as the clusters age and disrupt.

An additional factor to be considered in a more detailed analysis would involve the effects of differential ages. If there is an age-metallicity relation present in A2744 as for GCs in the Milky Way \citep{vandenberg+2013,forbes_bridges2010,dotter+2011}, the metal-poor ones might now be 12-13 Gyr old while the most metal-rich would be 9-10 Gyr old. But at a lookback time of 3.5 Gyr, the metal-poor ones are seen as 9 Gyr old while the metal-rich ones are only 6 Gyr old; i.e.~the relative age difference was larger at that time. The red sequence should then be nearly the same mean colour as at zero redshift (see above), but relatively brighter than the blue sequence for the same GC mass range.  A hint of that effect may be visible in the CMD, where the redder side of the distribution reaches to a much higher level. Further investigation into these effects will be done in later work. Deeper photometry would reveal a much clearer picture, but that is not likely to be achievable in the foreseeable future.

Integrating the GCLF over all magnitudes, with the assumption that the lognormal shape applies at all levels, yields an estimated total GC population $N_{\rm tot} = 60000$ including all the cluster galaxies and the ICM over the central region.  The uncertainty is likely to be $\pm20000$ given the preceding discussion. It should be viewed as a lower limit, since it does not include regions outside the central part of the cluster studied here, and does not account for other locations of area incompleteness such as near the centers of the large galaxies. This total is nevertheless very much smaller than the estimate of $4 \times 10^5$ GCs from \citet{lee16} from the \hst imaging, though later discussions \citep{harris+2017,alamo-martinez+2017} revised that downward by at least a factor of two. The \hst data cover only the brightest magnitude or so of the GCLF and thus require a very large extrapolation compared with our present data. Our estimate, however, is very similar to the total of 67000 GCs in the entire Virgo cluster \citep{durrell+2014}, where the photometry reaches deeper in GC luminosity.

% Figure environment removed

% Figure environment removed

\section{Adjustments to Zero Redshift}\label{sec:adjustments-z0}

Comparison of the A2744 GC population with local ($z=0$) GCs requires estimating the size of three different effects arising from the lookback time and redshift.

\subsection{Passive Luminosity Evolution} 

Any star cluster after its initial formation period will decrease in luminosity over time due to simple stellar evolution. In Figure \ref{fig:agetrend}, predicted luminosities in the three NIRCAM filters as derived from the PARSEC models are shown, at three different metallicities. Although the absolute magnitude depends slightly on metallicity, the relative change in luminosity over the past 3.5 Gyr is nearly independent of metallicity or filter. The net change from $z=0.3$ to $z=0$ is an age-fading of $\simeq 0.21$ mag, or about $20~$per cent decrease in luminosity.

This estimate, however, assumes a similar mean age for all the GCs. If as noted above the red GCs are as much as 3 Gyr younger on average than the blue GCs, then they would decrease by about $30~$per cent in luminosity down to the present time. A mean age difference of $\sim 3$ Gyr may be an upper limit, however, since the age/metallicity relation (AMR) for GCs is expected to be steeper for more massive galaxies \citep{horta+2021}, and most of the A2744 GCs are clearly associated with the giant member galaxies.

\subsection{Mass Loss} \label{subsec:mass-loss}

All GCs experience progressive mass loss due to stellar evaporation and tidal stripping. Clusters located in the ICM far from any major galaxies will be subject to only slow stellar evaporation. However, most of the GCs in our observed sample are satellites of the major galaxies (Fig.~\ref{fig:field}b) and these will experience larger tidal losses. Tidal disruption times for GCs will be several Gyr \citep[e.g.][]{baumgardt_makino2003} but the mass loss rate depends on their own mass and their local host-galaxy potential. For this preliminary study, we adopt a only a simple average correction for mass loss. A useful first-order prescription for the fractional mass loss \citep{choksi_gnedin2019} is
    \begin{equation}
    \frac{\Delta M}{M} = -\frac{\Delta t}{t_{\rm tid}} \simeq -\frac{\Delta t}{\textrm{5 Gyr}} \cdot \left(\frac{M}{2 \times 10^5 M_{\odot}} \right)^{-2/3}
    \end{equation}
where $t_{\rm tid}$ is the tidal disruption time and the GCLF turnover mass is approximately $2 \times 10^5 M_{\odot}$ \citep{villegas+2010,harris+2014}.  \citet{baumgardt_makino2003} or \citet{lamers+2005} give similar relations. As discussed above, the A2744 GCs  observed in this study are on the bright half of the the GCLF, with masses extending very probably up to $\simeq 10^6 M_{\odot}$ and above at the bright end, so with $\Delta t = 3.5$ Gyr, the ratio ($-\Delta M/M$) will range from $\simeq 0.15$ at the high-mass end up to as much as 0.7 at the GCLF peak; i.e. a dimming of 0.2 to 1.3 mag in absolute magnitude.  Stellar evaporation is negligible by comparison, with dissolution times one or two orders of magnitude larger \citep[e.g.][]{baumgardt_makino2003,lamers+2005}.

\subsection{K-Corrections}\label{subsec:k-correction}

The cosmological K-correction is well known in galaxy studies but relatively unfamiliar for GC work. The redshift causes any rest-frame wavelength $\lambda_0$ emitted at the source to be measured at $\lambda = (1+z)\lambda_0$, thus both shifting and stretching any given filter bandpass. By definition, the absolute magnitude $M_{\lambda}$ at wavelength $\lambda$ is
\begin{equation}
    M_{\lambda} = m_{\lambda} - 5 \log_{10}\left( \frac{d_L}{10 pc} \right) - K_{\lambda}  ~.
    \label{eq:kcorr} 
\end{equation}
The redshift factor K$_{\lambda}$ may be positive or negative depending on the shape of the spectrum in the region of a given filter bandpass. For A2744, there is an interesting coincidence at play: the ratios of effective wavelengths between the adjacent NIRCAM/SWC filters almost exactly match its redshift factor of $(1+z) = 1.3$. That is, light observed in $F200W$ ($\lambda_{\rm eff} = 19680$\AA) was emitted very nearly in $F150W$ (14873 \AA); light observed in $F150W$ was emitted in $F115W$ (11434 \AA); and light observed in $F115W$ was emitted in $F090W$ (8985 \AA).  Thus for the colour indices, our most metallicity-sensitive index $(F115W-F200W)$ is actually close to the rest-frame $(F090W-F150W)$, a colour that is similarly metallicity-sensitive.

GCs are good approximations to single stellar populations (SSPs) and their spectral energy distributions (SEDs) are not the same as those of galaxies. For galaxies, the K-values at a given redshift and wavelength depend on galaxy type (essentially, star formation history) \citep[e.g.][]{blanton_roweis2007,o'mill+2011,beare+2014}, whereas for SSPs the important parameters are instead age and metallicity. A more general treatment of the problem is therefore of interest, and this topic is developed further in a separate discussion (Reina-Campos \& Harris 2023, in preparation). For the purposes of this first look at the A2744 data, we apply mean K corrections for only a single age and metallicity.

To estimate the K-values, model template SEDs were first generated with E-MILES\footnote{http://miles.iac.es} \citep{rock+2016}.  These have been used to calculate $K_{\lambda}$ for the three bandpasses here following the method outlined in \citet{hogg+2002,blanton_roweis2007,condon_matthews2018}. For the SEDs, BaSTI isochrones were adopted with an age of 9 Gyr, metallicity [Fe/H] = $-1$, and a Chabrier IMF. With these assumptions we obtain K$_{115} = -0.17$ mag, K$_{150} =-0.17$, and K$_{200} =-0.42$. The values for F115W and F150W are the same because the redshifted and rest-frame SEDs are virtually parallel across those two bandapasses. Use of PARSEC isochrones by comparison gave negligible differences.  The true uncertainties in these estimates are hard to gauge since they rely on model SEDs and also depend on the adopted age and metallicity, but are likely to be at least $\pm0.05$ mag \citep[cf. the discussion of][]{blanton_roweis2007}.

% Figure environment removed

\subsection{A Redshift-Corrected CMD}\label{subsec:z-corrected-cmd}

To the measured magnitude in each filter, we derive the absolute magnitude from Eq.~\ref{eq:kcorr} with the appropriate K correction and $d_L = 1630$ Mpc. The absolute magnitude in $F150W$ can then be converted to a cluster mass with the model curves as shown in Fig.~\ref{fig:agetrend}, where for the present discussion we adopt a single age of 9 Gyr and metallicity [m/H] $= -1$ to represent an average over the population. With these parameters, $1 M_{\odot}$ corresponds to $M_{150,AB} = +5.43$. In Figure \ref{fig:cmd_prob}, the CMD for the system is shown again, but in the form of estimated (model-dependent) GC mass versus intrinsic colour.

The estimated GCLF turnover point at $F150W \simeq 30.9$ after the appropriate K-correction converts to a luminosity $M_{150} = -10.0$ and thus a mass of $1.5 \times 10^6 M_{\odot}$ -- again, as seen at an epoch of $z=0.3$. However, dynamical mass loss (\S\ref{subsec:mass-loss} above) over the 3.5-Gyr interval will reduce M(turnover) by a factor of $\sim3$, bringing it to $\sim 5 \times 10^5 M_{\odot}$ and closer to present-day systems for which $M_{to} \simeq 2-3 \times 10^5 M_{\odot}$ \citep{jordan+2007,villegas+2010,harris+2014}. The remaining discrepancy may reflect the inherent uncertainty in fitting both the GCLF peak and dispersion when the data do not reach the turnover (see \S\ref{sec:luminosity-function} above), but the evolving shape of the LF \citep{reina-campos+2022b} and the assumptions about the model IMF should also be kept in mind. 

At the same time, the CMD of Fig.~\ref{fig:cmd_prob} indicates that the mass distribution reaches to very high levels, with hundreds of objects above $10^7 M_{\odot}$ especially on the red, metal-rich (and perhaps younger) side of the plot. These objects penetrate well into the mass range for UCDs. GC/UCD objects like these are not unprecedented, since they do show up in significant numbers around the most massive present-day Brightest Cluster Galaxies (BCGs) and cDs \citep{harris2023}. As seen in Fig.~\ref{fig:xyblue}, however, most of these very luminous objects concentrate quite closely around the five major galaxies in the system and may experience stronger than average dynamical mass loss as they evolve to zero redshift. More detailed modelling will need to be done to trace the change in the LF and mass distribution.

\section{Summary and Conclusions}\label{sec:summary}

In this study, we have used the deep UNCOVER mosaic images of Abell 2744 taken with \jwst NIRCAM to construct a catalog of 10,000 pointlike (unresolved) objects in the central field of the cluster. The vast majority of the detected objects consist of globular clusters populating the halos of the A2744 galaxies, but many of them also scatter throughout the IntraCluster Medium. PSF fitting with the tools in \texttt{daophot} has been used to complete the photometry in the $F115W$, $F150W$, and $F200W$ filters, in all of which the photometric limits are closely similar. Extensive artificial-star tests are used to determine the recovery completeness and internal measurement uncertainties.  

A brief summary of our findings is as follows:
\begin{enumerate}
    \item[(1)] After careful rejection of nonstellar or crowded sources, the remaining sample of starlike (unresolved) sources defines a sharp, well populated sequence in the CMD. This sample is dominated by GCs, with very little residual contamination.
    \item[(2)] The spatial distribution of the detected sources shows that the majority are located within the halos of the five biggest A2744 galaxies. The map of their distribution also marks out many satellites, but also with a significant population distributed more widely across the IntraCluster Medium.
    \item[(3)] The colour distribution particularly in $(F115W-F200W)$ is much wider (0.5 mag) than the $\lesssim 0.2-$mag spread that would be expected from measurement scatter alone, and is what would be expected if the GCs span the full usual range of metallicities from [Fe/H] $\simeq -2$ to 0. There is no indication of star clusters that were formed much more recently, which would show up as a population of bright clusters much bluer than the normal GC sequence.
    \item[(4)] The GC luminosity function covers a $2.5-$mag range before it is severely damped by photometric incompleteness. Over this range, however, it accurately matches the lognormal shape that is expected for normal old GCs.  
    \item[(5)] The metal-poor GCs (defined as the bluer half of the colour range) are less centrally concentrated to the galaxies than are the metal-richer GCs, and dominate the ICL population. Their distribution resembles the gravitational lensing map of A2744, consistent with predictions from recent theoretical studies that low-metallicity GCs in particular are useful tracers of the dark matter distribution.
    \item[(6)] By comparing the A2744 GC population with GCs in present-day galaxies, we appear to be seeing the clear effects of GC dynamical evolution on the luminosity function. The estimated GCLF turnover point for A2744 corresponds to a GC mass several times higher than in zero-redshift galaxies, very much in line with predictions from current models of progressive mass loss over the intervening time.
\end{enumerate}

The present data strongly suggest that in A2744 we are looking at classic, normal populations of GCs around its member galaxies, seen at an earlier stage but well on their way to evolving into the types of distributions that we see in the Local Universe. Already by this epoch, the lognormal shape of the LF, at least at the high-luminosity end, has been well established, as have their fairly regular distributions around the halos of the member galaxies. The colour distribution function, after adjustment for K-correction, reflects that of the full expected metallicity range for old GCs, and the spatial distribution depends on metallicity in the way expected if the metal-poor GCs are primarily built up by satellite accretion.

Much more detailed followup work is possible and will be addressed in later studies. Normalizations to the data need to be improved, including K-corrections as a function of colour (metallicity) and age, luminosity evolution, and dynamical mass loss as a function of both GC mass and spatial location. Quantitative derivation of the GC radial profiles around the major A2744 galaxies needs to be done as well. Such work will lead to more detailed comparison of the overall GC distribution with other indicators of the DM distribution including the gravitational lensing map and the X-ray gas.

Above all, the A2744 data provide an excellent example of the new ground opened up by \jwst for the study of rich clusters of galaxies and their embedded GC systems. For the first time, it is possible to investigate observationally the evolution of entire GC systems over the past several Gyr leading up to the present day.

\section*{Acknowledgements}

We express our sincere thanks to Laura Parker, John Weaver, Laura Greggio, and Jeremy Webb for helpful discussion, and to Jose Diego for pointing us to this wonderful dataset. We would also like to acknowledge the work of the UNCOVER team to produce the beautiful mosaic images and to make them publicly available at such an early stage. 
MRC gratefully acknowledges the Canadian Institute for Theoretical Astrophysics (CITA) National Fellowship for partial support. This work was supported by the Natural Sciences and Engineering Research Council of Canada (NSERC). 

\emph{Facility:} JWST (NIRCAM)

\emph{Software:} IRAF \citep{tody1986,tody1993}, Daophot \citep{stetson1987}, Matplotlib \citep{hunter2007}, E-MILES \citep{rock+2016}, PARSECv1.2 \citep{bressan+2012}, BaSTI \citep{pietrinferni+2021}.

%%%%%%%%%%%%%%%%%%%%%%%%%%%%%%%%%%%%%%%%%%%%%%%%%%
\section*{Data Availability}

The mosaics for Abell 2744 from the UNCOVER program are publicly available at: \href{https://jwst-uncover.github.io/#}{https://jwst-uncover.github.io/\#}.

%%%%%%%%%%%%%%%%%%%% REFERENCES %%%%%%%%%%%%%%%%%%

% The best way to enter references is to use BibTeX:

\bibliographystyle{mnras}
\bibliography{mybib} % if your bibtex file is called example.bib

% Don't change these lines
\bsp	% typesetting comment
\label{lastpage}
\end{document}

% End of mnras_template.tex
