\section{Motivation}
\label{sec:motivation}


In this section, we motivate our work with
a study to demonstrate the need for a white list of libraries.



Despite existing approaches' limitations, it is unclear whether white lists of libraries are still necessary in practice, mostly due to the usage of obfuscation. 
Hence, a legitimate question is:
\textbf{To what extent does obfuscation jeopardize the use of a white list in Android apps?}
To address this question, we conducted a motivating study examining the prevalence of package name obfuscation in Android apps. 


% Figure environment removed

Let us first introduce name obfuscation, a technique used to remove package names and/or class names' meaning  to prevent or hinder reverse engineering and make malicious code detection more difficult. 
Typically, package and class names have meaningful names to make it easier for developers to understand the app structure and the code's purpose.
Obfuscation is both used to protect apps' intellectual property and to make malicious code harder to detect.
Most code obfuscators replace package or class names with simple letters from the alphabet~\cite{LI2019157,10.1109/SANER.2016.52}. 
For instance, the package name ``com.example.myapp.MyClass" could be obfuscated such as:
\dcircle{1} only the class name could be ofuscated: \texttt{com.example.myapp.a};
\dcircle{2} only the package name could be obfuscated: \texttt{a.b.c.MyClass};
\dcircle{3} both the package name and the class name can be obfuscated: \texttt{a.b.c.d.a}; or
\dcircle{4} the package name can be removed: \texttt{a}.

To conduct this study, we randomly selected \num{10000} apps per year from the Androzoo~\cite{10.1145/2901739.2903508} dataset for each year from 2010 to 2022 and checked whether any of the Fully Qualified Class Names (FQCNs) in these apps either:
\dcircle{1} started with a letter of the alphabet (e.g., \texttt{a.b.*}, or \texttt{g.u.*}); or
\dcircle{2} its class name is a single letter of the alphabet (e.g., \texttt{com.example.a} or \texttt{f.w.i}).
This allows us to cover all the cases cited above.

The results of this study are shown in Figure~\ref{fig:fqcn_motivation}.
This figure presents three categories of FQCNs:
\dcircle{1} the number of fully qualified class names that are within the package name of the app, i.e., the one declared in the manifest (e.g., if the package name of the app is \texttt{org.example} and FQCN is \texttt{org.example.MyClass});
\dcircle{2} the number of FQCNs that have been obfuscated, as defined previously;
\dcircle{3} the number of all remaining FQCNs, i.e., FQCNs that are not within the package name of the apps nor in an obfuscated package name.

This study shows that a low percentage of classes are in the app package (i.e., package with the app package name).
But more importantly, the percentage of obfuscated FQCNs in the apps is even lower, even if this percentage appears to be growing over time.
The largest percentage is made up of FQCNs that are neither within the package name of the app nor obfuscated. 
These results suggest that the use of name obfuscation techniques is not yet widespread enough to render white lists unnecessary for static analysis of Android apps.
Therefore, it appears that white lists 
are still useful for differentiating these FQCNs from developer and library code.
