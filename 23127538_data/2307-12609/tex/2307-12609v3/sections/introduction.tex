\section{Introduction}
\label{sec:introduction}


Static analysis is a popular technique used in the literature to analyze Android apps, it analyzes app code without executing it. 
This approach is widely used to uncover security issues~\cite{10.1016/j.infsof.2017.04.001}. 
For example, researchers apply static analysis techniques to detect privacy leaks~\cite{10.1145/2666356.2594299,10.5555/2818754.2818791,10.1109/ICSE43902.2021.00126,10.1145/3510003.3512766,10.1145/2660267.2660357} and detect malicious code~\cite{10.1109/SP.2016.30,10.1145/3510003.3510135,10.1007/978-0-387-68768-1_4,10.1145/3574158}. 
However, most of these approaches need to differentiate between developer and library code to focus on the app's functionality, which is the relevant code for finding security problems and avoiding scalability issues (due to the widespread use of polymorphism in libraries and the over-approximation of static analyzers).
This is why static analyzers do not dive into the Android framework code during analyses (e.g., FlowDroid discards classes that are within the Android framework, cf. lines 64--69 in FlowDroid's SystemClassHandler class~\cite{flowdroidSystemClassHandler}).
Differentiating between developer and library code is, therefore, a crucial step for static analysis to be effective and more scalable~\cite{9286020,9542854}, as it allows analyzers to focus on the parts of the app that are most likely to contain security issues.

Furthermore, libraries can introduce noise for malware detection.
For example, Mudflow~\cite{10.5555/2818754.2818808} and DroidAPIMiner~\cite{10.1007/978-3-319-04283-1_6} show that discarding libraries in their analyses improves their malware detection performance. 
A reliable list of libraries is thus an important artifact for the research and analyst community. 

Previous studies have employed white lists 
to identify libraries in Android apps.
Chen et al.~\cite{10.1145/2568225.2568286} manually compiled a list of 73 package names from common libraries. 
Similarly, Grace et al.~\cite{10.1145/2185448.2185464} randomly selected apps from a dataset of \num{100000} apps that were manually screened to identify libraries.
With this approach, they created a list of 100 libraries. 
These lists are ad-hoc and incomplete, 
as they only contain 73 and 100 libraries, respectively.
Another method to build a white list of libraries has been proposed by Li et al.~\cite{10.1109/SANER.2016.52}.
Their approach involves using a large dataset of apps to identify candidate libraries. 
The process includes ranking all package names by frequency of appearance in apps and using heuristics to retain libraries.
However, though the approach is considered the \emph{state-of-the-art} white list of libraries in the literature~\cite{9713838},
the list provided is outdated, 
the method used to create the list relies on arbitrary heuristics, 
and the hypothesis to consider a package name as a library according to its occurrence leads to a high rate of false positives (if numerous versions of the same app are present, for instance: \num{20715} different versions of app ``com.slideme.sam.manager" are present in \az).
Hence, creating a comprehensive white list of libraries to discriminate the developer code from libraries accurately remains an open challenge.

In this work, we propose a novel approach to build the first extensive and precise, by construction, white list of libraries by mining software dependencies.
Contrary to the research literature, which often relies upon complex approaches~\cite{10.1145/3324884.3416582},
our method involves mining information from developer habits.
We propose to the research and analyst communities a dataset called \dataset, containing \libsAfterRefinement libraries.
\dataset is accurate by construction, i.e., it only contains libraries.
This dataset is meant to evolve and regularly incorporate new libraries added by third-party vendors.
\dataset aims to facilitate the work of static analysis analysts in terms of scalability and robustness.
We believe this dataset will be a valuable resource for static analysis, we encourage its use and expansion as new libraries become available.

Overall, this paper makes the following contributions:
\begin{itemize}[noitemsep,topsep=0pt]
    \item We show that white list of libraries are still needed.
    \item We propose an approach to build an accurate set of libraries.
    \item We build \dataset, the first version of the library dataset produced by our approach.
\end{itemize}

Our artifacts are available:
\begin{center}
\url{https://github.com/JordanSamhi/AndroLibZoo}
\url{https://zenodo.org/records/10072709}
\end{center}
