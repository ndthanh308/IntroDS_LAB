\section{Limitations}
\label{sec:limitations}

One limitation of our work is that we did not consider all potential sources of third-party libraries publicly available. 
While we extracted libraries from Maven and Google's repositories, as well as open-source Android projects, there may be other sources of libraries.
This limitation is mitigated by the fact that given our hypothesis, we believe that the sources of libraries we relied on are representative of how developers build apps in general.

Another limitation is that we only considered a subset of open-source Android projects when extracting libraries. 
While we used the \az and the F-Droid datasets as sources of open-source projects, there are likely additional open-source projects available.

Another limitation of our work is that our list of libraries is only designed to match non-obfuscated libraries.
Our study has shown that this limitation is also mitigated (cf. Figure~\ref{sec:motivation}) since the majority of package names in Android apps are not obfuscated.
Besides, the detection of obfuscated libraries is another research direction that is actively being explored by the literature~\cite{10.1145/2976749.2978333,8543426,8330204}.