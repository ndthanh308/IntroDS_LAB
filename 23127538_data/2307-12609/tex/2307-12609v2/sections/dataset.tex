\section{Dataset Description}
\label{sec:dataset}


In this section, we describe our dataset. 
\dataset contains \num{1210} unique roots.
A root refers to the first element in a package name. 
For example, in "com.example.mypackage", "com" is the root.

We also collected data on the number of fields in the apps' package names in our dataset. 
A field refers to a dot-separated element in a package name.
For example, in "com.example.mypackage", there are three fields: "com", "example", and "mypackage".
Results are visible in Table~\ref{table:num_of_fields}.
There are 732 package names with only one field, \num{17172} package names with two fields, etc.


\begin{table}
    \centering
    \caption{Number of package names per field in \dataset}
    \begin{adjustbox}{width=\columnwidth,center}
        \begin{tabular}{lr|lr}
            \hline
            Fields & Count & Fields & Count \\ \hline
            with 1 field & \num{732} & with 6 fields & \num{134}\\ 
            with 2 fields & \num{17172} & with 7 fields & \num{40}\\ 
            with 3 fields & \num{13086} & with 8 fields & \num{26}\\ 
            with 4 fields & \num{2979} & with 9 fields & \num{8}\\
            with 5 fields & \num{634} & with 10 fields & \num{2}\\
            \hline \hline
            \multicolumn{2}{l|}{Total} & \multicolumn{2}{r}{\libsAfterRefinement}
        \end{tabular}
    \end{adjustbox}
    \label{table:num_of_fields}
\end{table}



There are significantly fewer package names with four or more fields. 
The number of package names with one field is relatively low compared to the others.
This suggests that many package names in the dataset follow a standard naming convention with a domain name followed by one or more subpackages.
The presence of package names with four or more fields may indicate the use of more complex or specialized naming conventions.

Table~\ref{table:top_ten} presents the top 10 most frequent roots and the top 10 most frequent fields. 
In the first two columns, we see that ``com" is by far the most frequent root, with more than \num{13000} occurrences. 
The second most frequent root is ``org", with \num{5450} occurrences. 
In the second two columns, representing the most frequent fields, including the roots, we see that ``com" field is still the most frequent. 
The second two columns do not differ much from the first two columns, except for the field ``gradle" that now appears.
This could indicate that Android libraries are prevalent (often built with gradle).
It is confirmed in the last two columns, representing the most frequent fields without the roots, in which we see that ``gradle``, and ``android`` fields are the most frequent, with 680 and 443 occurrences respectively. 
After ``gradle" and ``android", the third most frequent field is ``maven", with 352 occurrences.
We see a shift in the most prevalent fields. 
Instead of roots, we now see fields such as ``sdk", ``maven", ``plugin(s)", ``api", and ``tools".
This may be indicative of the types of libraries.
Overall, the results suggest that most libraries are from the ``com" domain and Android libraries are well represented.


\begin{table}
    \centering
    \caption{Top 10 roots and fields present in \dataset}
    \begin{adjustbox}{width=\columnwidth,center}
        \begin{tabular}{c|c|c|c|c|c}
            \hline
            \multicolumn{2}{c|}{\textbf{Top 10 roots}} & \multicolumn{2}{c|}{\textbf{Top 10 most used fields}} & \multicolumn{2}{c}{\textbf{Top 10 most used fields w/o roots}}\\ \hline
            \textbf{Root} & \textbf{Count} & \textbf{Field} & \textbf{Count} & \textbf{Field} & \textbf{Count} \\ \hline \hline
            com & \num{13514} & com & \num{13844} & gradle & \num{680} \\ \hline
            org & \num{5450} & org & \num{5519} & android & \num{443} \\ \hline
            io & \num{2630} & io & \num{2646} & maven & \num{352} \\ \hline
            net & \num{1651} & net & \num{1704} & com & \num{330} \\ \hline
            de & \num{1287} & de & \num{1288} & plugins & \num{269} \\ \hline
            cn & \num{784} & cn & \num{784} & sdk & \num{267} \\ \hline
            dev & \num{562} & gradle & \num{697} & plugin & \num{258} \\ \hline
            me & \num{548} & dev & \num{575} & co & \num{244} \\ \hline
            eu & \num{329} & me & \num{557} & tools & \num{164} \\ \hline
            se & \num{304} & co & \num{460} & api & \num{153} \\ \hline
        \end{tabular}
    \end{adjustbox}
    \label{table:top_ten}
\end{table}


