\section{Limitations}
\label{sec:limitations}

One limitation of our work is that we did not consider all potential sources of third-party libraries publicly available on the Internet. 
While we extracted libraries from Maven and Google's Android library repository, as well as open-source Android projects, there may be other sources of libraries that we did not include in our study. 
This limitation is mitigated by the fact that given our hypothesis, we believe that the sources of libraries we relied on are representative of how developer build  apps in general.
This means that our dataset may not be comprehensive and may not include all potential libraries that could be used in Android app development, but it should be comprehensive enough for better Android app static analysis.

Another limitation is that we only considered a subset of open-source Android projects when extracting libraries. 
While we used the \az and the F-Droid datasets as sources of open-source projects, there are likely additional open-source projects available on the internet that we did not consider. 
As a result, it is possible that we missed some libraries that are used in these other projects.

Another limitation of our work is that our list of libraries is only designed to match non-obfuscated libraries.
Our study has shown that this limitation is also mitigated (cf. Figure~\ref{sec:motivation}) since the majority of package names in Android apps are not obfuscated.
Besides, we believe that the detection of obfuscated libraries is another research direction that is actively being explored by the literature~\cite{10.1145/2976749.2978333,8543426,8330204}.