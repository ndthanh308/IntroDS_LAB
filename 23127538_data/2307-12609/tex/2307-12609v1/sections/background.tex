\section{Background}
\label{sec:background}

This section introduces the terminology used in our study.

\noindent
\textbf{Library}
A library is a collection of pre-written code that can be imported and used in a software project to perform specific tasks.
Libraries are often created by third-party developers and made available to other developers through repositories or package managers.

\noindent
\textbf{Package name}
A package name is a unique identifier for an app or a library.
It is used to differentiate it from other apps or libraries. 
It is typically in the form of a reversed domain name, such as "com.example.myapp", where "com" is the domain, "example" is the company or organization, and "myapp" is the name of the app or library.

\noindent
\textbf{Package name of an app}
The package name of an app is a unique identifier that the Android operating system use to identify the app when it is installed on a device.
This package name can be found in the app's AndroidManifest.xml file.

\noindent
\textbf{Fully Qualified Class Name}
A fully qualified class name (FQCN) is a string that uniquely identifies a class within an app. 
It consists of a package name, followed by the name of the class. 
For example, in the fully qualified class name "com.example.myapp.MyClass", "com.example.myapp" is the package name and "MyClass" is the name of the class.

\noindent
\textbf{Artifact}
In software development, an artifact is a file produced by a build process, such as a compiled software application or a library.

\noindent
\textbf{artifactId}
An artifactId is a unique identifier for a particular artifact within a software repository.
It is typically used to distinguish between different artifacts of the same organization.

\noindent
\textbf{groupId}
A groupId is a unique identifier for a group of related artifacts.
It is typically used to identify the organization that produced the artifacts.
It is usually a reverse domain name, such as "com.example.myproject", where "com.example" represents the organization and "myproject" represents the project.
The groupId is typically used in combination with an artifactId to uniquely identify a specific artifact within a repository.

We use groupId and artifactId to infer whether a third-party library is contained in an Android app. 
To that end, we concatenate groupId and artifactId, and compare the obtained string to the FQCNs within the app.
For example, if a class in an Android app has an FQCN of "com.example.myproject.mylibrary.subpackage.MyClass", we can determine that the app is using the "com.example.myproject.mylibrary" library.

\noindent
\textbf{Transitive dependency}
A transitive dependency is a library or a component that is indirectly required by a project. 
It is a dependency of one of the project's dependencies. 
In other words, a project depends on a library, and this library depends on another library, which means that the project indirectly depends on the second library as well.