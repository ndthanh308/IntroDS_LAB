\subsection{Comparison with state of the art}
\label{sec:comparison}

In this section we intend to answer RQ2 by comparing \dataset with the state-of-the-art while list of libraries generated by Li et al~\cite{10.1109/SANER.2016.52} (RQ2.a) and against LibRadar~\cite{10.1145/2889160.2889178}, a state-of-the-art tool to detect libraries in Android apps (RQ2.b).


\subsubsection{Comparison with state-of-the-art white list}

Since Li et al.'s dataset was generated prior to 2016, it would be unfair to compare both datasets as is, rather than the approaches themselves.
Hence, we applied our approach and only retain libraries that were available before 2016.
In this section, we call our dataset \dataset$_{2016}$

\noindent
\textbf{Comparison dataset:}
To compare with the work of Li et al., we retrieved the package names available in the "libraries" folder of the repository~\cite{commonLibrariesRepo} described in their paper~\cite{LI2019157}.
After removing any duplicates, the resulting list, which we refer to as \textbf{comparison\_dataset}, contains \num{5926} package names.

% Figure environment removed


\noindent
\textbf{Comparison:}
Upon a first examination, we found that the two lists have little in common, as \dataset$_{2016}$ contains \num{9551} package names while \textbf{comparison\_dataset} contains only \num{5926} package names, and the intersection of the two lists is only 66, as shown in Figure~\ref{fig:intersection}.
This suggests that our list is significantly larger and more comprehensive than the state-of-the-art list.
However, it should be noted that this comparison was made using strict one-to-one string matching, so some of our package names might be prefixes of their package names, and vice versa.

As a result, we conducted a follow-up comparison to determine if any of the libraries in \dataset$_{2016}$ might be prefixes of the package names in the \textbf{comparison\_dataset} and vice versa.
Results indicate that 280 of our library names are prefixes of 1636 of their library names.
Additionally, 101 of their library names are prefixes of 194 of our library names.
The total number of common libraries using prefix matching is \num{1722}, as shown in Figure~\ref{fig:intersection}.
These findings show some overlap between the two lists and that \dataset$_{2016}$ covers a larger part of the \textbf{comparison\_dataset}.
However, the overall intersection of the two lists is still relatively small, with only a small proportion of libraries being common to both lists.

To provide further assessment of the \textbf{comparison\_dataset}, a qualitative study was conducted. 
This study aimed to identify potential libraries in the \textbf{comparison\_dataset} that were not included in the package names found in both \dataset$_{2016}$ and the \textbf{comparison\_dataset} with prefix matching (in other words, we want to answer the following question: does the \textbf{comparison\_dataset} contain many libraries that \dataset$_{2016}$ does not?). 
A total of 50 package names were randomly selected. 
We manually searched the Internet to identify any mentions, repositories, or websites that would indicate that they are libraries. 
Results indicate that only 2 out of the 50 package names were found to be actual libraries.
When searching the Internet we found  that these two libraries had dedicated website which were not crawled by our approach.
The 50 package names are available in the project's repository.

We did not further manually check whether the libraries identified by \dataset$_{2016}$ but not by  \textbf{comparison\_dataset} are true libraries or not because, by construction, \dataset only contains "true" libraries.

\highlight{
\textbf{RQ2.a answer:}
Our approach, \dataset$_{2016}$, exhibits a larger coverage of libraries compared to the state-of-the-art dataset by Li Li et al.
Although there is some overlap between the two lists, the overall intersection remains relatively small. 
Our qualitative analysis shows that Li et al. dataset is not reliable, unlike  \dataset.
}

\subsubsection{Comparison with LibRadar}

LibRadar was introduced as a tool to detect third-party libraries in Android apps.
Although its primary purpose is not to generate a dataset of libraries, it can be used to create a list by running it on apps and retaining the libraries identified in the output.
To this end, we executed LibRadar on \num{100000} recent apps (i.e., collected after 2020) and compiled a list of libraries. 
LibRadar successfully completed the analysis for \num{62308} apps, producing a list of \num{12239} unique package names.
However, this list cannot be used as a whitelist since it contains over \num{3000} obfuscated package names (e.g., a.a.a). 
While these package names might represent libraries in apps, they cannot be employed as is for a whitelist, since a.a.a might not be a library in another app. 
Consequently, we cannot directly compare both lists. 

To further investigate which of LibRadar or \dataset would be more useful into filtering libraries in apps, we have randomly selected 100 apps from the \num{62308} previously analyzed apps.
For each of these apps we
\dcircle{1} applied LibRadar to extract the libraries it would detect; and 
\dcircle{2} extracted the package name of each class and used \dataset to check if they were libraries or not.
In total, LibRadar is able to filter 174 non-obfuscated libraries from these 100 apps, whereas \dataset can filter \num{2745}.
The median number of libraries found per app is 1 for LibRadar and 28 for \dataset.
This result shows that LibRadar misses a vast amount of libraries in Android apps that \dataset would not miss.


\highlight{
\textbf{RQ2.b answer:}
Our empirical analysis shows that \dataset covers more non-obfuscated libraries than LibRadar in apps.
}
