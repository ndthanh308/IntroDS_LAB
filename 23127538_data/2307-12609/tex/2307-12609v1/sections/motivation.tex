\section{Motivation}
\label{sec:motivation}

In this section, we motivate our work by highlighting the limitations of existing approaches and providing a study to demonstrate the need for a white list of libraries.


% Figure environment removed

\textbf{Limitation of "Frequency-based approaches"}
Figure~\ref{fig:package_example} shows the package hierarchy of a given app. 
The manifest file informs us about the app's package name, which is, in this case, "com.bmi.calculatorplus". 
Other packages are clearly obfuscated,  e.g., "b.a.a". 
Eventually, two last packages remain: "com.bentenstudio.ui" and "com.cengalabs.flatui.views". 
Each of these two packages can  either be a library or a package containing the code of the app developer (indeed, the developer code is not always only in the app package).
 With no additional information, it is challenging to know if these two packages are libraries or not.

Previous work~\cite{10.1109/SANER.2016.52} has proposed an approach  to discriminate libraries from other packages, by using a frequency-based approach. 
This approach considers package names as potential libraries if they appear in multiple apps 
(10 in the case of the study).
However, this technique is not efficient as it may lead to false positives. 
Indeed, if many versions 
of a single app are analyzed, the same packages may be found, but they may not necessarily be libraries (e.g., \num{20715} different versions of app "com.slideme.sam.manager" are present in AndroZoo).
Additionally, due to the flexibility of package naming in Android apps, different developers can use the same package name, outside the hierarchy of the app package name, to organize their code, and the same company could use the same package names in different apps.




Therefore, an alternative strategy is needed to ensure that only libraries are captured. 
Our approach to addressing this issue will be described in Section~\ref{sec:methodology}.
It is important to note that, despite existing approaches' limitations, it is unclear whether white lists of libraries are still necessary in practice, mostly due to the usage of obfuscation. 
Hence the first research question that we aim to answer in this section is:
\textbf{RQ1: To which extent does obfuscation jeopardize the use of a white list in Android apps?}
To address this question, we conducted a motivating study examining the prevalence of package name obfuscation in Android apps. 


% Figure environment removed

Let us first introduce name obfuscation which is a technique used to remove the meaning from package names and/or class names in order to prevent or hinder reverse engineering and make malicious code detection more difficult. 
Typically, package names and class names are given meaningful names to make it easier for developers to understand the structure of the app and the purpose of the class. 
Most of the code obfuscators replace package names or class names with simple letters from the alphabet~\cite{LI2019157,10.1109/SANER.2016.52}. 
For instance, the package name "com.example.myapp.MyClass" could be obfuscated such as:
\dcircle{1} only the class name could be ofuscated: \texttt{com.example.myapp.a};
\dcircle{2} only the package name could be obfuscated: \texttt{a.b.c.MyClass};
\dcircle{3} both the package name and the class name can be obfuscated: \texttt{a.b.c.d.a}; or
\dcircle{4} the package name can be removed: \texttt{a}.

Our motivating study is designed to examine the prevalence of name obfuscation in Android apps. 
To conduct this study, we randomly selected \num{10000} apps per year from the Androzoo~\cite{10.1145/2901739.2903508} dataset for each year from 2010 to 2022 and checked whether any of the FQCNs in these apps either:
\dcircle{1} started with a letter of the alphabet (e.g., \texttt{a.b.*}, or \texttt{g.u.*}); or
\dcircle{2} its class name is a single letter of the alphabet (e.g., \texttt{com.example.a} or \texttt{f.w.i}). 
This allows us to cover all the cases cited above.

The results of this study is shown in Figure~\ref{fig:fqcn_motivation}.
This figure presents three categories of FQCNs:
\dcircle{1} the number of fully qualified class names (FQCNs) that are within the package name of the app, i.e., the one declared in the manifest (e.g., if the package name of the app is \texttt{org.example} and FQCN is \texttt{org.example.MyClass});
\dcircle{2} the number of FQCNs that have been obfuscated, as defined previously;
\dcircle{3} the number of all remaining FQCNs, i.e., FQCNs that are not within the package name of the apps nor in an obfuscated package name.

This study shows that a low percentage of classes are in the app package (i.e., package with the app package name).
But more importantly, the percentage of obfuscated FQCNs in the apps is even lower, even if this percentage appears to be growing over time.
The largest percentage is made up of FQCNs that are neither within the package name of the app nor obfuscated. 
These results suggest that the use of name obfuscation techniques is not yet widespread enough to render white lists unnecessary for static analysis of Android apps.
Therefore, it appears that white lists can still be useful for differentiating these FQCNs from developer and library code.

\highlight{
\textbf{RQ1 answer:}
This preliminary study shows that obfuscation has a limited impact on the use of a white list of libraries.
}

