\section{Related Work}
\label{sec:related_work}

We aim at providing an overview of previous approaches to identify Android third-party libraries.
We also discuss approaches that aim to detect obfuscated libraries.

\subsection{Identification of libraries}

LibRadar~\cite{10.1145/2889160.2889178} was designed to detect third-party libraries in Android apps. 
The tool uses stable API features to identify libraries, and is able to detect third-party libraries in a given app quickly through simple static analysis and feature comparison. The authors' approach involves three main steps: feature extraction, clustering, and detection.
Contrary to \dataset, LibRadar does not aim at producing a list of library since it is intended to identify third-party library during analysis. 

Similarly to LibRadar, LibSift~\cite{7890569} was proposed to detect third-party libraries in Android apps. 
The authors' approach involves using the natural partitioning of apps into packages and performing module decoupling at the package level.

The closest work to ours is Li Li et al.~\cite{10.1109/SANER.2016.52,LI2019157} which is about collecting and investigating common libraries used by Android apps. 
The authors used a dataset of 1.5 million Android apps from Google Play to identify and collect a large set of common libraries. 
The authors used ad-hoc heuristics to try to discriminate libraries in their dataset of 1.5 million Android apps. 
To do this, they only considered three fields in package names and considered a package as a library if it was used in at least 10 apps. 
This threshold is relatively low given the size of the dataset, as it only represents a small fraction of the total number of apps. 
Additionally, the authors removed package names with only one field, as these were not considered to be relevant for the purpose of the study.
They then performed empirical investigations to confirm the motivations behind the research and to demonstrate how these libraries can be used as whitelists by Android analysis approaches in order to improve their accuracy.

Contrary to other approaches, we took a different approach to collecting common libraries from Android apps. 
Rather than attempting to guess which libraries are being used based on the apps themselves, we instead focused on the sources that developers use when creating apps.
We considered real libraries that are used by developers and mined these sources to construct a white list of libraries.
This approach allowed us to more accurately identify the libraries being used by developers, rather than relying on heuristics or guesses based on the apps. 
As a result, \dataset is more comprehensive and reliable than those produced by other methods.

\subsection{Identification of obfuscated libraries}

LibScout~\cite{10.1145/2976749.2978333} is a technique for detecting third-party libraries in Android apps, even if the code has been obfuscated. 
LibScout is used to perform a study of the usage and evolution of libraries in the top apps on Google Play, and found that app developers are often slow to adopt new library versions, which can leave users vulnerable to security vulnerabilities. 
Their approach involves two steps for detecting third-party libraries in Android apps.
First,  profiles, based on the class hierarchy,  are extracted from the original versions of libraries in the authors' repository. 
Second, the profiles of the libraries are compared to profiles extracted from the app being analyzed. A similarity score is generated, indicating whether a library version was matched or not. 

Orlis~\cite{8543426} was proposed as a new approach for detecting third-party libraries in obfuscated Android apps.
The approach relies on code features such as call graphs and class hierarchies, and it uses similarity digests to improve the detection of third-party libraries in obfuscated Android apps.
The output of the approach is a class-level mapping between the app code and the library code, which provides essential information for tasks such as clone/repackaging detection.

LibPecker~\cite{8330204} is a library detection tool for Android apps that aims to accurately and reliably identify the libraries used in an application, even in the presence of code obfuscation. 
It does this by generating class-dependency-based signatures for both libraries and applications, and using fuzzy class matching and weighted class contributions to account for code customization and elimination.

LibID~\cite{10.1145/3293882.3330563} was designed to identify third-party libraries in Android apps. 
The tool is resistant to code obfuscation techniques, such as identifier renaming, shrinking, control flow randomization, and package modification.
It includes two different library identification schemes, LibID-S and LibID-A, which focus on scalability and accuracy, respectively. 
The authors have developed a novel method for generating synthetic apps containing third-party libraries in order to accurately evaluate and compare the performance of LibID to previous work. 
The experiments conducted using this method and an analysis of hundreds of F-Droid apps show that LibID is able to detect more libraries than other state-of-the-art tools.

Contrary to our approach, these third-party library detection approaches are designed to identify libraries that have been obfuscated. 
In many cases, these approaches rely on a list of known libraries in order to perform matching or learning, whereas our approach does not require such a list. 
Instead, we provide this list as a resource.
Additionally, these approaches often assign trust-like scores to libraries in order to indicate the likelihood of a package to be a library. 
Our approach does not utilize this type of scoring system and instead simply reports the presence or absence of a given library in an app given it is present or not in our white list.