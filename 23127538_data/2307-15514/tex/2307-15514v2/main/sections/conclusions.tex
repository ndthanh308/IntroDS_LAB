%%%%%%%%%%%%%%%%%%%%%%%%%%%%%%%%%%%%%%%%%%%%%%%%%%%%%%%%%%%%%%%
%%%%%%%%%%%%%%%%%%%%%%%%%%%%%%%%%%%%%%%%%%%%%%%%%%%%%%%%%%%%%%%
%%%%%%%%%%%%%%%%%%%%%%%%%%%%%%%%%%%%%%%%%%%%%%%%%%%%%%%%%%%%%%%
\section{Conclusions}\label{sec:conclusions}

We revisited the Fully Convolutional Geometric Feature (FCGF) approach to tackle the problem of object 6D pose estimation.
FCGF uses sparse convolutions to learn point-wise features while optimising a hardest contrastive loss. 
Key modifications to the loss, input data representations, training strategies, and data augmentations to FCGF enabled us to outperform competitors on popular benchmarks. 
A thorough analysis is conducted to study the contribution of each modification to achieve state-of-the-art performance.
Future research directions include the application of our approach to generalisable 6D pose estimation~\cite{onepose}.

\noindent\textbf{Limitations.}
Minkowski engine is computational efficient but has a large memory footprint at training time.
We mitigated this by downsampling the scene point cloud and by adopting quantisation. It would be interesting to understand how not to lose the input point cloud resolution while maintaining a modest memory footprint.

\small{
\section*{Acknowledgements}
\label{sec:acknowledgements}
We are grateful to Andrea Caraffa for his support with the computation of the detection priors and to Nicola Saljoughi for his contributions during the early stage of the project.

This work was supported by the European Union’s Horizon Europe research and innovation programme under grant agreement No 101058589 (AI-PRISM), and by the PNRR project FAIR - Future AI Research (PE00000013), under the NRRP MUR program funded by the NextGenerationEU.
}
