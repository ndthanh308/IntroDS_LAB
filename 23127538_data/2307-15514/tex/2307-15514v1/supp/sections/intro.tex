\section{Appendix Introduction}
\label{sec:intro}

We provide additional material in support of our main paper.
The material is organised as follows:

\setlist{nolistsep}
\begin{itemize}
    \item In Sec.~\ref{sec:qualitative} we show qualitative results on the LMO~\cite{lmo} and YCBV~\cite{ycbv} datasets, divided into success (Figs.~\ref{fig:lmo_success} \& \ref{fig:ycbv_success}) and failure (Figs.~\ref{fig:lmo_fail} \& \ref{fig:ycbv_fail}) cases. 
    We also report examples of how pose registration is affected by correct and incorrect detections.
    To assess the quality of the learnt features, we visualise the distances in the feature space between a point and all the other points (Figs.~\ref{fig:feat_all_lmo} \& \ref{fig:feat_all_ycbv}).
    %
    \item In Sec.~\ref{sec:ablation_ycbv} we report an ablation study on one object of the YCBV dataset (Tab.~\ref{tab:ablation_ycbv}).
    %
    \item In Sec.~\ref{sec:detections} we show the detection metrics on the test set used in our results (Tabs.~\ref{tab:detection_lmo} \& \ref{tab:detection_ycbv}) and highlight problems we found in the ground-truth annotations (Fig.~\ref{fig:scissors}). 
    We also report the percentage of cases in which the detector causes a failure or a success in the pose registration.
\end{itemize}

