%%%%%%%%%%%%%%%%%%%%%%%%%%%%%%%%%%%%%%%%%%%%%%%%%%%%%%%%%%%%%%%
%%%%%%%%%%%%%%%%%%%%%%%%%%%%%%%%%%%%%%%%%%%%%%%%%%%%%%%%%%%%%%%
%%%%%%%%%%%%%%%%%%%%%%%%%%%%%%%%%%%%%%%%%%%%%%%%%%%%%%%%%%%%%%%
\section{Conclusions}\label{sec:conclusions}

We revisited the Fully Convolutional Geometric Feature (FCGF) approach to tackle the problem of object 6D pose estimation.
FCGF uses sparse convolutions and a fully-convolutional network to learn point-wise features while optimising a hardest contrastive loss. 
Key modifications to the loss, input data representations, training strategies, and data augmentations to FCGF enabled us to outperform competitors on popular benchmarks. 
A thorough analysis is conducted to study the contribution of each modification to achieve state-of-the-art performance.
Future research directions include the application of our approach to generalisable 6D pose estimation~\cite{onepose}.

\noindent\textbf{Limitations.}
Minkowski engine is computational efficient but has a large memory footprint at training time.
We mitigated this by downsampling the scene point cloud and by adopting quantisation, at the cost of a reduced output resolution.
It would be interesting to understand how not to lose the resolution of the input point cloud with modest memory footprint.

\section*{Acknowledgements}
\label{sec:acknowledgements}
We would like to thank Nicola Saljoughi for his valuable contributions during the early stage of the project.
