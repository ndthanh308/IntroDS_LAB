%%%%%%%%%%%%%%%%%%%%%%%%%%%%%%%%%%%%%%%%%%%%%%%%%%%%%%%%%%%%%%%%%%
%%%%%%%%%%%%%%%%%%%%%%%%%%%%%%%%%%%%%%%%%%%%%%%%%%%%%%%%%%%%%%%%%%
%%%%%%%%%%%%%%%%%%%%%%%%%%%%%%%%%%%%%%%%%%%%%%%%%%%%%%%%%%%%%%%%%%
\section{Introduction}\label{sec:intro}

Object 6D pose estimation is the problem of finding the Euclidean transformation (i.e.~pose) of an object in a scene with respect to the camera frame~\cite{hodavn2016evaluation}.
This problem is important for autonomous driving~\cite{autodriving1}, augmented reality~\cite{augreality}, space docking~\cite{space}, and robot grasping~\cite{grasping1}.
The main challenges are handling occlusions, structural similarities between objects, and non-informative textures.
Different benchmarks have been designed to study these challenges, such as LineMod-Occluded (LMO)~\cite{lmo}, YCB-Video (YCBV)~\cite{ycbv}, and T-LESS~\cite{tless}. 
LMO includes poorly-textured objects in scenarios with several occlusions.
In YCBV, well-textured objects appear in scenarios with fewer occlusions but more pose variations.
T-LESS includes poorly-textured and geometrically-similar objects in industrial scenarios with occasional occlusions.

% ================================================================
% Figure environment removed
% ================================================================

6D object pose estimation approaches based on deep learning can be classified as \textit{one-stage}~\cite{singlestage,e2ek,deepim} or \textit{two-stage}~\cite{segdriven,pvn3d,ffb6d,geometricaware6d}.
One-stage approaches can directly regress the object pose~\cite{singlestage,e2ek,deepim}.
Two-stage approaches can predict 3D keypoints~\cite{pvn3d,ffb6d} or point-level correspondences between the scene and the object~\cite{geometricaware6d}.
Correspondences can be computed through point-level features~\cite{geometricaware6d}.
One-stage approaches are typically more efficient than their two-stage counterpart, as they require only one inference pass. However, orientation regression is a difficult optimisation task because the space of orientations is non-Euclidean and non-linear, and the definition of correct orientation is ambiguous in case of symmetric objects~\cite{saxena2009learning}.
On the other hand, correspondence-based approaches have to be coupled with registration techniques, such as RANSAC, PnP, or least square estimation~\cite{geometricaware6d}.

We argue that the problem of learning discriminative point-level features is overlooked in the object 6D pose estimation literature.
Moreover, we believe that working at intermediate levels of representation learning, rather than regressing the pose directly, facilitates explainability and enables us to effectively debug algorithms.
The literature about representation learning for general point cloud registration has made great advances~\cite{FCGF,GEDI}, and, to our knowledge, none of the object 6D pose estimation methods have deeply investigated the application of these techniques to the underlying problem (Fig.~\ref{fig:teaser}).
In a landscape dominated by complex networks, our work stands as the first to comprehensively explore and quantify the benefits of this formulation with a simple yet effective solution. Our research addresses fundamental and previously unanswered questions:\\
\textit{a) How to learn features of heterogeneous point clouds (objects and scenes) that align in the same representation space and exhibit cross-domain generalisation (synthetic to real)?\\
b) What training strategies are optimal for this approach?\\
c) What degree of improvement can these strategies bring?}

To answer these questions, we revisit Fully Convolutional Geometric Features (FCGF)~\cite{FCGF} and show that its potential to achieve state-of-the-art results lies in an attentive design of data augmentations, loss negative mining, network architecture, and optimisation strategies.
FCGF is designed to learn point-level features by using a fully-convolutional network optimised through a hardest contrastive loss.
Compared to the original FCGF setting, our setting is asymmetric, i.e.~the two input point clouds have different sizes and resolutions.
Therefore, we modify the hardest contrastive loss to take into account the size of each point cloud for the mining of the hardest negatives.
We use separate architectures to learn specific features for the two (heterogeneous) input data (object and scene), but unlike several state-of-the-art methods we train only a single model for all the objects of each dataset.
We use specific augmentationsto tackle occlusions, which are the main challenge in real-world scenarios and in the considered datasets.
We name our approach \ourmethod.
\ourmethod outperforms state-of-the-art methods (+3.5 ADD(S)-0.1d on LMO, +0.8 ADD-S AUC on YCBV), even when comparing with methods that train one model for each object.
Our ablation study suggests that most of the performance gain is obtained thanks to our changes to the loss and by the addition of the RGB information.
In summary, our contributions are:
\setlist{nolistsep}
\begin{itemize}
    \item We tailor FCGF for object 6D pose estimation in order to 
    i) process entire scenes rather than cropped regions as competitors,
    ii) learn a single network model for all objects instead of a network model for each object,
    iii) process both photometric and geometric information with a single unified deep network model.
    \item A novel version of the hardest contrastive loss that is applied to heterogeneous point clouds and that considers a geometric constraint when mining the hardest negative.
    \item We study data augmentations that enable FCGF to harness generalisation between synthetic and real data.
\end{itemize}
