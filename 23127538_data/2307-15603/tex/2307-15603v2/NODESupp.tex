\documentclass[aps,prb,onecolumn,amsfonts,superscriptaddress,amssymb,amsmath,10pt]{revtex4-2}

\usepackage{graphicx}
\usepackage[separate-uncertainty = false]{siunitx}
\usepackage{color}
\usepackage{lineno}
%\usepackage{palatino}
\usepackage{array}

\newcolumntype{L}[1]{>{\raggedright\let\newline\\\arraybackslash\hspace{0pt}}m{#1}}
\newcolumntype{C}[1]{>{\centering\let\newline\\\arraybackslash\hspace{0pt}}m{#1}}
\newcolumntype{R}[1]{>{\raggedleft\let\newline\\\arraybackslash\hspace{0pt}}m{#1}}

\newcommand\red[1]{{\textbf{\color{red}#1}}}
\newcommand\blue[1]{{\color{black}#1}}
\newcommand{\calP}{$\mathcal{P}$}
\newcommand{\calT}{$\mathcal{T}$}
\newcommand{\ra}{\rightarrow}
\newcommand{\la}{\leftarrow}
\newcommand{\XXX}{\textbf{\red{XXX}}}
\renewcommand{\vector}[1]{\textbf{\textit{#1}}}

\renewcommand\thesection{\arabic{section}}
\renewcommand\thetable{\arabic{table}}

\renewcommand{\baselinestretch}{1.5} 

\setcitestyle{square}

%\renewcommand*\familydefault{\sfdefault}

\renewcommand{\figurename}{\textbf{Supplementary Fig.}}
\renewcommand{\tablename}{\textbf{Supplementary Table}}

\begin{document}
	
	\begin{center}
		\vspace{6cm}\vskip6cm
		\huge{\textbf{Supplementary Information\\}}
		\vspace{3cm}
		\Large{\textbf{Nonlinear optical diode effect in a magnetic Weyl semimetal}}
		\vspace{2cm}
		\flushleft{C. Tzschaschel \textit{et al.}}
	\end{center}
	\tableofcontents
	\clearpage
	
	\section{Materials and Methods}
	
	\subsection{Symmetry considerations in {C\lowercase{e}A\lowercase{l}S\lowercase{i}}}
	
	Throughout the Supplementary Material, all Miller indices $[hkl]$ are given with respect to the conventional tetragonal unit cell of the paramagnetic phase (Fig.~\ref{fig:CeAlSiStructure}). The directions $\hat{\mathbf{x}}$, $\hat{\mathbf{y}}$, $\hat{\mathbf{z}}$ are defined as $[110]=+\hat{\mathbf{x}}$, $[\bar{1}10]=+\hat{\mathbf{y}}$, and $[001]=+\hat{\mathbf{z}}$.
	
	In this section, we describe the symmetry of the paramagnetic and ferromagnetic phases of CeAlSi. The symmetry dictates the nonzero components of the SHG tensor $\chi_{ijk}$ as defined by
	
	\begin{equation}
		P_i(2\omega) = \chi_{ijk} E_j(\omega)E_k(\omega).
		\label{eqn:SHGP}
	\end{equation}
	
	The paramagnetic phase of CeAlSi crystallizes in the same tetragonal lattice as TaAs, with the space group \#109 ($I4_1md$) and the point group $4mm$ (Fig.~\ref{fig:CeAlSiStructure}A). Just like in the case of TaAs, all atoms occupy sites belonging to a 4a-Wyckoff orbit. Table~\ref{Paramag_symm} enumerates the symmetry operations of the paramagnetic phase of CeAlSi.
	
	% Figure environment removed
	
	\begin{center}
		\begin{table}[ht]
			\begin{tabular}{p{3.5cm}|p{1.5cm}|p{1.5cm}|p{1.5cm}|p{1.5cm}|p{1.5cm}|p{1.5cm}|p{1.5cm}}
				Paramagnetic CeAlSi & $\mathcal{E}$ & $\mathcal{C}_4(z)$ & $\mathcal{C}_2(z)$ & $\mathcal{M}_x$ & $\mathcal{M}_y$ & $\mathcal{M}_{xy}$ & $\mathcal{M}_{-xy}$\\
				\hline
				& $\mathcal{T}$ & $\mathcal{C}_4(z)\mathcal{T}$ & $\mathcal{C}_2(z)\mathcal{T}$ & $\mathcal{M}_x\mathcal{T}$ & $\mathcal{M}_y\mathcal{T}$ & $\mathcal{M}_{xy}\mathcal{T}$ & $\mathcal{M}_{-xy}\mathcal{T}$\\
			\end{tabular}
			\caption{\label{Paramag_symm} \textbf{Point group symmetry of the paramagnetic phase of CeAlSi ($4mm$).}}
		\end{table}
	\end{center}
	
	$\mathcal{E}$ is identity, $\mathcal{T}$ is time-reversal symmetry, $\mathcal{C}_4(z)$ and $\mathcal{C}_2(z)$ are the $4$-fold rotation and $2$-fold rotation around the $\hat{\mathbf{z}}$ axis, and $\mathcal{M}_i$ represents the mirror reflection from $i$ to $-i$ ($i=x,y,xy,-xy$). Because the system has time-reversal symmetry $\mathcal{T}$, any spatial symmetry (mirror, rotation, etc) combined with $\mathcal{T}$ is also a good symmetry, as reflected in the second line above.
	
	\vspace{0.5cm}
	The ferromagnetic phase of CeAlSi exhibits a spontaneous magnetization along $\hat{\mathbf{x}}$ or $\hat{\mathbf{y}}$. The magnetization breaks not only time-reversal symmetry $\mathcal{T}$ but also a number of other symmetries. As a result, the point group symmetry is reduced to $m^\prime m 2^\prime$. Supplementary Table~\ref{ferromag_symm} shows the remaining magnetic symmetry operations in the ferromagnetic phase for a domain with $\vector{M}^\mathrm{Ce} \parallel \hat{\mathbf{y}}$ \cite{Birss66}.
	\begin{center}
		\begin{table}[h]
			\begin{tabular}{p{3.5cm}|p{1.5cm}|p{1.5cm}}
				Ferromagnetic CeAlSi & $\mathcal{E}$ & $\mathcal{M}_y$\\
				\hline
				& $\mathcal{C}_2(z)\mathcal{T}$ & $\mathcal{M}_x\mathcal{T}$
			\end{tabular}
			\caption{\label{ferromag_symm} \textbf{Point group symmetry of the ferromagnetic phase of CeAlSi ($m^\prime m 2^\prime$) with $\vector{M} \parallel \hat{\mathbf{y}}$ \cite{Birss66}.}}
		\end{table}
	\end{center}
	
	\subsection{Electrode patterning}\label{Sec:patterning}
	In order to apply current in the crystallographic $xy$ plane, we deposited electrical contacts on a naturally cleaved $(001)$ facet (see Fig.~\ref{fig:SampleImage}). A square pattern of eight gold electrodes was deposited at room temperature by e-beam evaporation through a shadow mask. Aligning the shadow mask with a naturally cleaved edge allows us to apply current along the crystallographic $\langle 110 \rangle$ or $\langle 100 \rangle$ direction, which corresponds to perpendicular/parallel or at an angle of $45^\circ$ relative to the magnetization direction in CeAlSi, respectively. Subsequently, the crystal was glued and wire bonded onto a chip carrier. Current was applied using a Keithley Model 2400 SourceMeter.
	
	% Figure environment removed
	
	\subsection{SHG tensor components of CeAlSi}
	
	The symmetry of the paramagnetic phase of CeAlSi has the point group $4mm$. A symmetry analysis similar to the previous section shows that the allowed components of the SHG tensor introduced in Eq.~(\ref{eqn:SHGP}) are \cite{Birss66}
	
	\begin{equation}
		\chi_{zzz} \qquad \chi_{zxx} = \chi_{zyy} \qquad \chi_{xxz} = \chi_{xzx} = \chi_{yyz} = \chi_{yzy}.
	\end{equation}
	
	All other tensor components vanish. Specifically, no electric-dipole SHG is allowed for light propagating along the $z$ direction.
	
	Below $T_\mathrm{C}$, the magnetic order reduces the symmetry to $Fd^\prime d 2^\prime$ (\#43.3.322, point group $m^\prime m 2^\prime$) \cite{Yang2021}, which gives rise to additional contributions to the second-order susceptibility. In particular, the allowed components below $T_\mathrm{C}$ are (for a magnetization along the $y$ direction):
	
	\begin{eqnarray}
		&\chi_{zzz}& \qquad \chi_{zxx} \qquad \chi_{zyy} \qquad \chi_{xxz} = \chi_{xzx} \qquad \chi_{yyz} = \chi_{yzy}\\
		&\chi_{xxx}& \qquad \chi_{xyy} \qquad \chi_{xzz} \qquad \chi_{zzx} = \chi_{zxz} \qquad \chi_{yyx} = \chi_{yxy}.
	\end{eqnarray}
	
	Note that the allowed $\chi$ tensor components for the four magnetic states of CeAlSi are related by symmetry as shown in Supplementary Table~\ref{Tab:SHGcomp}.
	\begin{center}
		\begin{table}
			\begin{tabular}{c|c c c c}
				\hline
				\hline
				& $\mathbf{M}_\mathrm{+y}$ & $\mathbf{M}_\mathrm{-y}$ & $\mathbf{M}_\mathrm{+x}$ & $\mathbf{M}_\mathrm{-x}$ \\
				\hline
				$\chi_{xxx}$ & $\chi_{xxx}$ & $-\chi_{xxx}$ & 0 & 0 \\
				$\chi_{xyy}$ & $\chi_{xyy}$ & $-\chi_{xyy}$ & 0 & 0 \\
				$\chi_{xzz}$ & $\chi_{xzz}$ & $-\chi_{xzz}$ & 0 & 0 \\
				$\chi_{xyz}$ & 0 & 0 & 0 & 0 \\
				$\chi_{xxz}$ & $\chi_{xxz}$ & $\chi_{xxz}$ & $\chi_{yyz}$ & $\chi_{yyz}$ \\
				$\chi_{xxy}$ & 0 & 0 & $-\chi_{yxy}$ & $\chi_{yxy}$ \\
				\hline
				$\chi_{yxx}$ & 0 & 0 & $-\chi_{xyy}$ & $\chi_{xyy}$ \\
				$\chi_{yyy}$ & 0 & 0 & $-\chi_{xxx}$ & $\chi_{xxx}$ \\
				$\chi_{yzz}$ & 0 & 0 & $-\chi_{xzz}$ & $\chi_{xzz}$ \\
				$\chi_{yyz}$ & $\chi_{yyz}$ & $\chi_{yyz}$ & $\chi_{xxz}$ & $\chi_{xxz}$ \\
				$\chi_{yxz}$ & 0 & 0 & 0 & 0 \\
				$\chi_{yxy}$ & $\chi_{yxy}$ & $-\chi_{yxy}$ & 0 & 0 \\
				\hline
				$\chi_{zxx}$ & $\chi_{zxx}$ & $\chi_{zxx}$ & $\chi_{zyy}$ & $\chi_{zyy}$ \\
				$\chi_{zyy}$ & $\chi_{zyy}$ & $\chi_{zyy}$ & $\chi_{zxx}$ & $\chi_{zxx}$ \\
				$\chi_{zzz}$ & $\chi_{zzz}$ & $\chi_{zzz}$ & $\chi_{zzz}$ & $\chi_{zzz}$ \\
				$\chi_{zyz}$ & 0 & 0 & $-\chi_{zxz}$ & $\chi_{zxz}$ \\
				$\chi_{zxz}$ & $\chi_{zxz}$ & $-\chi_{zxz}$ & 0 & 0 \\
				$\chi_{zxy}$ & 0 & 0 & 0 & 0\\
				\hline
			\end{tabular}
			\caption{Allowed SHG tensor components in the four magnetic states of CeAlSi and their symmetry relationship.}
			\label{Tab:SHGcomp}
		\end{table}
	\end{center}
	
	The polarization dependence of the measured SHG intensity can be simulated as
	
	\begin{equation}
		I_{2\omega} \propto \vert\vector{P}(2\omega)\cdot\vector{A}\vert^2,
	\end{equation}
	
	\noindent where $\vector{P}(2\omega)$ is the second-order polarization as described by Eq.~(\ref{eqn:SHGP}) and $\vector{A}$ is a unit-vector defining the transmission axis of the analyzer crystal for the SHG light. All tensor components are in general complex quantities, i.e. $\chi_{ijk} = \vert\chi_{ijk}\vert \exp\left(i\Phi_{ijk}\right)$. 
	
	\subsection{SHG interference phenomena}
	
	Interference between the various contributions to the nonlinear optical polarization $\vector{P}(2\omega)$ can give rise to several intriguing phenomena including the nonlinear optical diode effect described in the main text. 
	
	In the following we distinguish in particular between three separate, but sometimes overlapping, effects: magnetic domain contrast, nonreciprocal SHG (NR-SHG), and the nonlinear optical diode effect (NODE). We illustrate their relative relationship in a Venn diagram in Fig.~\ref{fig:Venn}
	
	% Figure environment removed
	
	\begin{enumerate}
		\item \textbf{Magnetic domain contrast} describes a difference in SHG intensity between different magnetic domains. A magnetic domain contrast can be observed within the electric-dipole approximation. This is due to interference between i-type (time reversal \calT\ even) and c-type (\calT\ odd) SHG contributions. Below the magnetic ordering temperature $T_\mathrm{C}$, the nonlinear optical susceptibility can be expressed as 
		\begin{equation}
			\chi_{ijk}(T<T_\mathrm{C}) = \underbrace{\chi_{ijk}(T>T_\mathrm{C})}_\mathrm{crystallographic\ (i-type)\ SHG} + \underbrace{\chi_{ijkl}(T>T_\mathrm{C})M_l,}_\mathrm{magnetic\ (c-type)\ SHG}
		\end{equation}
		
		\noindent where the indices $i$,$j$,$k$,$l$ refer to the crystallographic directions $x$, $y$, or $z$. Since the measured SHG intensity is proportional to $\vert\chi_{ijk}(T<T_\mathrm{C})\vert^2$, we find that the SHG intensity depends on the interference between crystallographic and magnetic SHG contributions and therefore on the direction of $\vector{M}$. This gives rise to the observed domain contrast. The domain contrast, in turn, depends on the relative amplitude between the crystallographic and magnetic SHG tensor contributions. The contrast is maximized if crystallographic and magnetic SHG have the same amplitude.  
		
		\item \textbf{Nonreciprocal SHG} can occur if the source term $\vector{S}$ in the electromagnetic wave equation contains contributions that are both even and odd functions of the wave vector \vector{k}, such that 
		\begin{equation}
			\vector{S} = \underbrace{\vector{S}_{\mathrm{even}}}_\mathrm{k-even\ (electric\ dipole)\ SHG} + \underbrace{\vector{S}_{\mathrm{odd}}}_\mathrm{k-odd\ (magnetic\ dipole)\ SHG}.
		\end{equation}
		
		Nonreciprocal SHG is not possible to observe in the electric-dipole approximation as all electric-dipole contributions are $\vector{k}$-even. Often, k-even and k-odd contributions correspond to electric dipole and magnetic dipole SHG, respectively \cite{Toyoda2021, Mund2021}, but also higher order multipole contributions can exist \cite{Shen66}. The precise structure of \vector{S}$_{\mathrm{even}}$ and \vector{S}$_{\mathrm{odd}}$ depends on the involved multipole contribution. In terms of point group symmetry operations,
		k-odd contributions like magnetic-dipole SHG are allowed in centrosymmetric materials and are hence \calP\ even. Moreover, the presence of k-even SHG contributions requires a broken inversion symmetry. They are thus \calP\ odd. Whereas the magnetic domain contrast arises from interference between \calT\ even and \calT\ odd SHG contributions, the observation of NR-SHG requires interference between \calP\ even and \calP\ odd SHG contributions, which highlights the different microscopic origin.
		
		\item We introduce the \textbf{nonlinear optical diode effect (NODE)} as a third SHG interference phenomenon. We will provide here both a phenomenological as well as a microscopic definition. Phenomenologically, we define the NODE as an effect whereby a nonlinear optical response (such as SHG), is stronger for one light propagation direction than for the reversed light propagation direction. Such an effect can be defined both in reflection and in transmission geometry (Supp. Fig.~\ref{ExtDefinition}). In both geometries, we can define a directional contrast $\eta$ as 
			\begin{equation}
				\eta = \left(I^\ra - I^\la \right)/\left(I^\ra + I^\la \right).
			\end{equation} 
		\noindent Microscopically, the reversal of the propagation direction is related to a mirror operation $\mathcal{M}$ (the relevant mirror plane runs vertically in Supp. Fig.~\ref{ExtDefinition}a and b). The NODE arises if the total SHG response is comprised of contributions that are even and odd under that mirror operation $\mathcal{M}$.
			
			\begin{equation}
				\chi_{ijk} = \chi_{ijk}(\mathcal{M}\ even) + \chi_{ijk}(\mathcal{M}\ odd)
				\label{eqn:NODEDef}
			\end{equation}
			
		\noindent While nonreciprocal SHG is crucial for the occurrence of a NODE in transmission, the minimal example in Supplementary Fig.~\ref{ExtMinimal} illustrates that a NODE can be observed in reflection even in the electric dipole approximation. This shows that the NODE and nonreciprocal SHG are independent effects that rely on the transformation behavior of the SHG response under different operations. In this example, the relevant mirror operation that enables the NODE is $\mathcal{M}_x$. The component $\chi_{xxx}$ is odd under $\mathcal{M}_x$, whereas $\chi_{xxz}$ is even. Note that the NODE is present in Supplementary Fig.~\ref{ExtMinimal}a,b even if $\chi_{xxx}$ and $\chi_{xxz}$ are i-type. Thus, the NODE is also distinct from the magnetic domain contrast as a NODE does not strictly require breaking of time-reversal symmetry or c-type SHG contributions \cite{Tokura2018}. The presence of a magnetic domain contrast also does not imply the presence of a NODE \cite{Pavlov03,Toyoda23} (both c-type and i-type SHG contributions may have the same transformation behavior with respect to $\mathcal{M}$). Hence, the magnetic contrast in Supp. Fig.~\ref{fig:Venn} is overlapping, but not contained in the NODE. In the cross section, the magnetic orders allows to control the directionality of the NODE, as demonstrated in the present work.
	\end{enumerate}
	
	% Figure environment removed
	
	% Figure environment removed
	
	To summarize, magnetic contrast, NR-SHG, and NODE are distinct interference phenomena that rely on different microscopic symmetries. Magnetic contrast strictly requires a broken time-reversal symmetry \calT\ and NR-SHG can only be observed in materials with broken spatial inversion \calP. The NODE requires neither symmetry to be broken, but is related to mirror symmetries. Note that $\mathcal{P} = \mathcal{M}_x \circ \mathcal{M}_y \circ \mathcal{M}_z$. Thus, a broken spatial inversion symmetry (as required for NR-SHG) implies a broken mirror symmetry, which enables a NODE. Therefore, NR-SHG is fully contained in the NODE in Supp. Fig.~\ref{fig:Venn}.
	\clearpage
	
	\newpage
	\section{Polarization-resolved SHG measurements}
	
	% Figure environment removed
	
	We show in Supplementary Fig.~\ref{fig:PolScans} the polarization dependence of the SHG intensity for the different magnetic states at $\SI{3}{K} < T_\mathrm{C}$ and in the paramagnetic phase at $\SI{12}{K} > T_\mathrm{C}$ ($T_\mathrm{C} = \SI{8.2}{K}$, see below). We show the polarization dependence for both forward (Fig.~\ref{fig:PolScans}a-e) and backward (Fig.~\ref{fig:PolScans}f-j) propagating light. 
	
	The polarization dependencies were measured such that the incident polarization was rotated, while the outgoing $p$ (grey) and $s$ (red) polarized SHG was detected. All measurements were taken at $\hbar\omega = \SI{715}{meV}$.
	
	In all magnetic states, we observe a NODE  for some light polarization. In the paramagnetic state, we observe an absence of a NODE for all polarizations. Moreover, we note certain symmetries between the polarization dependencies. In particular, $I^{\mathbf{k}}_{\mathbf{M}}(\phi) = I^{-\mathbf{k}}_{-\mathbf{M}} (\phi)$ always holds (see Supplementary Fig.~\ref{ExtSymm} for details).
	
	All 16 measurements in the magnetic phase ('4 magnetic states' x '2 SHG polarizations' x '2 propagation directions') were fit simultaneously with one consistent set of SHG tensor components for ED-SHG according to point group $2^\prime m m^\prime$ (Supplementary Table~\ref{Tab:SHGcomp}). To accommodate experimental uncertainties, we allow for a small rotation of the sample about the $x$, $y$, and $z$ axis corresponding to a pitch of the sample, a deviation of the angle of incidence from $45^\circ$, and an in-plane rotation of the sample, respectively. All deviations from the ideal case are below $2^\circ$. The fit results are shown as solid lines in Supplementary Fig.~\ref{fig:PolScans}. Note that the fit curves are often overlapping with the data points.
	
	The excellent quality of the fit indicates that the detected SHG is dominated by ED-SHG. We provide in table \ref{table:SHGcomponentsbulk} the magnitudes and phases for the fitted tensor $\chi_{ijk} = \vert\chi_{ijk}\vert * e^{i\Phi_{ijk}}$ components relative to $\chi_{zzz}$.
	
	\begin{table}
		\begin{center}
			\begin{tabular}{|c|c|c|}
				\hline
				$\chi_{ijk}$ & $\vert \chi_{ijk}\vert/\vert \chi_{zzz}\vert$ & $(\Phi_{ijk} - \Phi_{zzz})/\pi$ \\
				\hline
				$zzz$ & 1 & 0 \\
				\hline
				$xxz$ & 0.2 & -0.58 \\
				\hline
				$zxx$ & 0.29 & 0.28 \\
				\hline
				$yyz$ & 0.76 & -0.56 \\
				\hline
				$zyy$ & 1.87 & 0.76 \\
				\hline
				$xxx$ & 0.06 & 0.64 \\
				\hline
				$yyx$ & 0.62 & -0.55 \\
				\hline
				$xyy$ & 1.99 & 0.77 \\
				\hline
				$zzx$ & 0.78 & 0.55 \\
				\hline
				$xzz$ & 2.36 & -0.17 \\
				\hline
			\end{tabular}
			\caption{Magnitudes and phases of $\chi$ tensor components relative to $\chi_{zzz}$. Values were found by fitting all 168 measurements below $T_C$ in Supplementary Fig.~\ref{fig:PolScans} simultaneously with one consistent set of $\chi$ tensor components according to point group $2^\prime m m^\prime$.}
			\label{table:SHGcomponentsbulk}
		\end{center}
	\end{table}
	
	Based on the obtained relative magnitudes and phases of all tensor components, we can calculate the SHG intensity and the directional contrast even in configurations that are not easily experimentally accessible. For example, we show in Supplementary Fig.~\ref{fig:AoI} the incident angle dependence of the SHG intensity for s-polarized incident light and p-polarized outgoing SHG (experimental configuration of Fig.~1e in the main text). Although we observe a strong directional contrast at an incident angle of $45^\circ$, this angle is in general not optimal. As the directional contrast must be a continuous function of the incident angle, it is interesting to consider two edge case:
	
	\begin{enumerate}
		\item \underline{Normal incidence (AoI = $0^\circ$)}: In this scenario, incident and outgoing beams overlap. A reversal of the light path (AoI $\rightarrow$ -AoI) does not change the experiment. Hence, a NODE cannot be observed in normal incidence reflection.
		\item \underline{Grazing incidence (AoI = $90^\circ$))}: Also here, a NODE cannot be observed in the electric-dipole approximation. In the coordinate system of Supplementary Fig.~\ref{ExtMinimal}, grazing incidence corresponds to light propagation parallel to $\hat{\mathbf{x}}$. The transversal electric fields thus only couple to tensor components $\chi_{ijk}$ with $i,j,k \in \left\{y,z\right\}$, which are all even with respect to the mirror operation $\mathcal{M}_x$. Hence, the basic symmetry requirement for the observation of a NODE (Eq.~(\ref{eqn:NODEDef})) cannot be realized.
	\end{enumerate}
	
	% Figure environment removed
	\newpage
	\section{Temperature dependent characterization}\label{sec:Temp}
	
	Unless explicitly noted differently, all measurements were performed at a sample temperature of \SI{3}{K}. We provide here additional details on the temperature-dependent characterization of CeAlSi.
	
	% Figure environment removed
	
	We show in the main text that we can switch the magnetization of CeAlSi and control the NODE by applying current (Fig.~3 in the main text). In Supplementary Fig.~\ref{fig:Temperature}a, we show the measured SHG intensity at various temperatures during the application of a DC current. Here, an external magnetic field of \SI{30}{mT} stabilizes a ferromagnetic $\mathbf{M}_\mathrm{+y}$ state such that no current-induced switching can occur. We can clearly distinguish the paramagnetic phase (homogeneously light blue) from the ferromagnetically ordered phase (white to red). 
	
	Note that "Temperature" here refers to the temperature set point instead of the actual sample temperature, which is elevated due to current-induced ohmic heating. The effect of ohmic heating can be seen from the approximately parabolic current-induced temperature increase. This increase is further independent of the applied current direction (consistent with ohmic heating). Under maximum current, the sample enters the paramagnetic phase at a temperature of about \SI{4}{K}. Thus. at \SI{3}{K}, the sample remains in the magnetically ordered phase for all applied DC currents.
	
	Note also that the SHG intensity in the paramagnetic phase is independent of the applied current. In contrast, the SHG intensity at maximum current in Fig.~3c in the main text is different for the opposite polarities. Thus, the sample remained in the magnetically ordered state during the measurement of Fig.~3c.
	
	Supplementary Fig.~\ref{fig:Temperature}b shows the temperature dependence of the SHG intensity in the absence of applied current. We find a transition temperature of \SI{8.2}{K} in agreement with previous characterizations \cite{Yang2021, Sun2021Mapping}. As CeAlSi is noncentrosymmetric above $T_C$, we continue to observe a finite SHG intensity above \SI{8.2}{K}.
	
	\newpage
	\section{Additional NODE spectroscopy}
	
	We report in the main text the observation of a nonlinear optical diode effect (NODE) over a wide spectral range. We demonstrate this observation in the main text on the basis of selected measurements in certain magnetic configurations. In this section, we provide additional systematic experimental evidence in complementary settings. The striking agreement between symmetry-related measurements is evidence for a high data quality and shows that the observations are intrinsic properties of CeAlSi. 
	
	\subsection{Observation of a broadband NODE in the $\mathbf{M}_\mathrm{-y}$ state}
	
	We show in Supplementary Fig.~\ref{fig:NODE} SHG spectra for counter-propagating light beams. The sample is in a magnetic $\mathbf{M}_\mathrm{-y}$ state. Complementary to Fig.~1e of the main text, we again observe a broadband NODE. However, the spectra are inverted relative to Fig.~1e of the main text, which were recorded in the $\mathbf{M}_\mathrm{+y}$ state. This observation is immediate evidence that the magnetization in CeAlSi controls the directionality of the broadband NODE.
	
	Other experimental parameters are identical: spectra were recorded at \SI{3}{K}, incident fundamental light is $s$ polarized, and we detect outgoing $p$ polarized SHG light. 
	
	% Figure environment removed
	
	\subsection{Determination and robustness of directional contrast}
	
	We introduce in the main text the directional contrast 
	
	\begin{equation}
		\eta = \frac{ I^{\ra}_{+y}-I^{\la}_{+y}}{I^{\ra}_{+y}+I^{\la}_{+y}}. \label{eq:eta1}
	\end{equation}
	
	By symmetry, this expression is equivalent to 
	
	\begin{eqnarray}
		\eta &=& \frac{ I^{\ra}_{+y}-I^{\ra}_{-y}}{I^{\ra}_{+y}+I^{\ra}_{-y}} \label{eq:eta2}\\
		&=&  -\frac{ I^{\la}_{+y}-I^{\la}_{+y}}{I^{\la}_{+y}+I^{\la}_{+y}}\label{eq:eta3}\\
		&=&  -\frac{ I^{\ra}_{-y}-I^{\la}_{-y}}{I^{\ra}_{-y}+I^{\la}_{-y}}\label{eq:eta4}\\
	\end{eqnarray}
	
	Here, we experimentally verify this equivalence. We show in Supplementary Fig.~\ref{fig:Mycontrast}a and b 2D maps of the detected $p$ polarized SHG intensity as a function of incident photon energy and polarization angle ($0^\circ/180^\circ$ - $p$ polarized; $90^\circ$ - $s$ polarized fundamental light). Both maps were obtained with light propagating in forward direction, however, the sample is in a magnetic $\mathbf{M}_\mathrm{+y}$ state in Supplementary Fig.~\ref{fig:Mycontrast}a and  in a $\mathbf{M}_\mathrm{-y}$ state in Supplementary Fig.~\ref{fig:Mycontrast}b. We notice clear differences between the two panels.
	
	Supplementary Fig.~\ref{fig:Mycontrast}c shows the directional contrast according to Eq.~\ref{eq:eta2} showing a strong contrast (for certain polarizations) at all photon energies. To verify the robustness of the directional contrast, i.e., the equivalence of Eqs.~\ref{eq:eta1}-\ref{eq:eta4}, we evaluated the SHG intensity also in the reversed direction for both magnetic states (Supplementary Fig.~\ref{fig:Mycontrast}d and e). 
	
	We notice the (expected) similarity between Supplementary Fig.~\ref{fig:Mycontrast}a and e, as well as Supplementary Fig.~\ref{fig:Mycontrast}b and d. Based on these four measurements, we can determine the directional contrast $\eta$ according to all four definitions in Eqs.~\ref{eq:eta1}-\ref{eq:eta4}. Indeed, as we can see from Supplementary Figs.~\ref{fig:Mycontrast}c, f, g, and h, the obtained values for $\eta$ agree well (up to a sign). The absolute value of Supplementary Fig.~\ref{fig:Mycontrast}c is shown in Fig.~4a in the main text.
	
	In Supplementary Fig.~\ref{fig:Mycontrast}i, we display for all photon energies within the considered spectral range the maximum contrast that is achievable by selecting a specific incident polarization. We note that the maximum of $\vert\eta\vert$ determined according to Eqs.~\ref{eq:eta2} and \ref{eq:eta3} show a striking agreement, whereas the values determined according to Eqs.~\ref{eq:eta1} and \ref{eq:eta4} differ slightly. This is due to experimental uncertainties in reversing the beam path, which we can circumvent by determining $\eta$ from measurements in the same propagation direction but opposite magnetic states. Note that this options exists in CeAlSi as reversing the propagation direction is equivalent to reversing the beam path (see Sec.~\ref{sec:Symmetry}).
	
	% Figure environment removed
	
	\subsection{Directional contrast in the $\mathbf{M}_\mathrm{\pm x}$ states}
	
	The discussion of the directional contrast in the main text and in the previous section was centered on the magnetic $\mathbf{M}_\mathrm{\pm y}$ states. In this section, we present an analogous analysis for the $\mathbf{M}_\mathrm{\pm x}$ states.
	
	In Supplementary Fig.~\ref{fig:Mxcontrast}a-d, we show 2D maps of the SHG intensity as a function of incident fundamental photon energy and polarization. The panels show data recorded in forward direction in the $\mathbf{M}_\mathrm{+x}$ state (panel a), in forward direction in the $\mathbf{M}_\mathrm{-x}$ state (panel b), in reversed direction in the $\mathbf{M}_\mathrm{+x}$ state (panel c), and in reversed direction in the $\mathbf{M}_\mathrm{-x}$ state (panel d). We recognize the clear (expected) pairwise similarity between panels a and d, as well as panels b and c.
	
	As we showed in the previous section that a determination of $\eta$ according to Eqs.~\ref{eq:eta1}-\ref{eq:eta4} yields analogous results, we restrict ourselves here to Eq.~\ref{eq:eta2}. The obtained 2D map for the directional contrast is shown in Supplementary Fig.~\ref{fig:Mxcontrast}e. By selecting specific incident light polarizations, we can ensure a directional contrast of at least 72\% for all photon energies and up to 96.6\%, very similar to the achievable contrast in the magnetic $\mathbf{M}_\mathrm{\pm y}$ states (73\% to 97.2\%).
	
	% Figure environment removed
	
	
	\subsection{A note on symmetry}
	
	By comparing the 2D maps of the directional contrast in the $\mathbf{M}_\mathrm{\pm x}$ states (Supplementary Fig.~\ref{fig:Mxcontrast}e) and  $\mathbf{M}_\mathrm{\pm y}$ states (e.g, Supplementary Fig.~\ref{fig:Mycontrast}c), we recognize differences in the symmetry of these magnetic states. In the $\mathbf{M}_\mathrm{\pm y}$ states, the 2D map is mirror symmetric about the $0^\circ$ and $90^\circ$ polarization directions. This is due to the mirror symmetry of the polarization dependence in the $\mathbf{M}_\mathrm{\pm y}$ states (Supplementary Fig.~\ref{fig:PolScans}), which is microscopically a consequence of the $m_y$ mirror symmetry in these states. In contrast, in the $\mathbf{M}_\mathrm{\pm y}$ states, the 2D map is mirror symmetric with a sign change, which is an experimental manifestation of the  $m_y^\prime$ symmetry in these states ($m_y^\prime = m_y \circ \mathcal{T}$).
	
	\subsection{Directional contrast in the paramagnetic phase}
	
	Here, we verify that the NODE vanishes in the paramagnetic phase of CeAlSi as required by symmetry (Sec.~\ref{sec:Symmetry}) and indicated by the polarization-dependent measurements at \SI{715}{meV} (Supplementary Fig.~\ref{fig:PolScans}e,f).
	
	In Supplementary Fig.~\ref{fig:12KPolSpectra}, we show 2D maps showing the detected SHG intensity as a function of incident polarization and photon energy. All measurements were taken at \SI{12}{K}, which is well above $T_C = \SI{8.2}{K}$ (Sec.~\ref{sec:Temp}). We consider here both $p$ and $s$ polarized SHG light for both forward and backward propagating light. For both polarizations, measurements with forward and backward propagating light are strikingly similar. We note that for the $s$ polarized SHG response (Supplementary Fig.~\ref{fig:12KPolSpectra}b,d), the overall SHG intensity is relatively low. Therefore, experimental uncertainties in reversing the light become apparent in the measurement. Moreover, in particular for the $0^\circ$ and $90^\circ$ polarization, the SHG intensity approaches zero (as required by $4mm$ symmetry), which impedes a meaningful determination of the directional contrast. We therefore refrain from showing 2D maps for $\eta$, but note the both qualitative and quantitative similarity between forward and backward propagating SHG responses for both $s$ and $p$ polarized SHG light.
	
	% Figure environment removed
	\clearpage
	
	\newpage
	\section{Domain assignment and mapping}\label{sec:mapping}
	
	For a fixed propagation direction, we can utilize the SHG domain contrast in CeAlSi to visualize the four different magnetic states. In particular, the unique polarization dependence of the SHG in all four magnetic states (Supplementary Fig.~\ref{fig:PolScans}) allows us to identify four distinct combinations of incident fundamental and outgoing SHG polarizations that are uniquely sensitive to each of the magnetic states (Supplementary Fig.~\ref{fig:DomainAssignment}). This enables the capability to image the magnetic domain distribution of CeAlSi and assign the magnetic states. 
	
	% Figure environment removed
	
	We frequently exploit the capability to map and assign domain states in CeAlSi. For example:
	
	\begin{enumerate}
		\item It enables the \textit{operando} visualization of the domain distribution during the application of current (Supplementary Movie 3).
		\item It allows us to verify that the current-induced switching in Fig.~3c of the main text is between $\mathbf{M}_\mathrm{\pm y}$ states.
		\item It allows us to verify that the current-induced switching in Fig.~3g of the main text is between $\mathbf{M}_\mathrm{\pm x}$ states.
		\item It enables the visualization of magnetic-field induced switching (Section~\ref{sec:Bfield})
	\end{enumerate}
	
	To obtain the domain image on the right-hand side of Supplementary Fig.~\ref{fig:DomainAssignment}, we normalized the SHG image recorded at \SI{3}{K} to the SHG image recorded at \SI{12}{K}. Thus, spatial inhomogeneities in the SHG intensity due to the approximately Gaussian beam profile are removed. Since such a normalization is only possible within the illuminated area, the image is cropped to area approximately corresponding to the beam spot.
	\clearpage
	
	\newpage
	\section{Magnetic field dependent SHG characterization}\label{sec:Bfield}
	
	Utilizing the capability to selectively visualize magnetic domains by SHG imaging (Sec.~\ref{sec:mapping}), we determined the domain state for magnetic fields in the range of $\pm\SI{20}{mT}$. The measurement was performed at \SI{1.6}{K}. We show the area fraction of the different states in Supplementary Fig.~\ref{fig:Bfield}a.
	
	We chose here two setting for the laser polarization: (1) the 90/0 configuration that yields a high SHG intensity only for $\mathbf{M}_\mathrm{+y}$ domains and (2) the 90/90 configuration yields a high SHG intensity for both $\mathbf{M}_\mathrm{+x}$ and $\mathbf{M}_\mathrm{-x}$ domains. The area fraction for $\mathbf{M}_\mathrm{-y}$ domains was obtained numerically by completing the determined area fractions to 1.
	
	Consistent with literature, we find a narrow hysteresis from which we deduce a low coercive field of 1-\SI{2}{mT} (estimated from the hysteresis at an area fraction of 50\%). Note that the $\mathbf{M}_\mathrm{+y}$ hysteresis loop is not symmetric around \SI{0}{mT} due to CeAlSi exhibiting four magnetic states. Thus, we expect an area fraction of 25\% for all domains in a fully randomized state at \SI{0}{mT} (i.e., 25\% $\mathbf{M}_\mathrm{+y}$, 25\% $\mathbf{M}_\mathrm{-y}$, and 50\% $\mathbf{M}_\mathrm{+x}$ + $\mathbf{M}_\mathrm{-x}$). Thus, the $\mathbf{M}_\mathrm{+y}$ ($\mathbf{M}_\mathrm{-y}$) hysteresis loop appears shifted towards positive (negative) magnetic fields. Moreover, we find that $\mathbf{M}_\mathrm{\pm x}$ domains are only present at low magnetic fields and fully suppressed for $\vert B_y\vert > \SI{10}{mT}$.
	
	In panels b-e of Supplementary Fig.~\ref{fig:Bfield}, we show SHG domain images at selected magnetic fields for laser polarization setting (1) sensitive only to $\mathbf{M}_\mathrm{+y}$ domains. At \SI{-20}{mT}, the sample appears homogenously dark (panel b). In contrast, at \SI{+20}{mT}, the sample appears homogenously bright (panel d). These states correspond to single domain $\mathbf{M}_\mathrm{+y}$ and $\mathbf{M}_\mathrm{-y}$ states, respectively. Panels c and e show the domain image at \SI{2.7}{mT} taken while increasing and decreasing the magnetic field, respectively. As expected, the area fraction of $\mathbf{M}_\mathrm{+y}$ domains is larger after decreasing the field from \SI{+20}{mT} (panel e). The series of all images for both laser polarization settings are combined into Supplementary Movies 1 (polarization setting 1) and 2 (polarization setting 2).
	
	% Figure environment removed
	\clearpage
	
	\newpage
	\section{Symmetry of the NODE}\label{sec:Symmetry}
	
	In addition to the discussions in the main text, we discuss here  the symmetry of the NODE in more detail. We find that mirror symmetries of the paramagnetic crystal lattice restore certain symmetries in the SHG polarization dependencies (Fig.~2 in the main text and Supplementary Fig.~\ref{fig:PolScans}). To see this, we consider the effect of the mirror operations $m_x$ and $m_y$ on the magnetization, propagation direction, and light polarization (Supplementary Fig.~\ref{ExtSymm}). 
	
	In general, the mirror operation $m_x$ reverses the propagation direction ($\mathbf{k}\rightarrow -\mathbf{k}$) (Supplementary Fig.~\ref{ExtSymm}a and c) while the mirror operation $m_y$ preserves $\mathbf{k}$ (Supplementary Fig.~\ref{ExtSymm}a and c). Both mirror operations individually transform the linear light polarization from an angle $\phi$ to $-\phi$ relative to p-polarized light. We will consider in the following discussion the SHG intensity $I_\mathbf{M}^\mathbf{k}(\phi)$. The arguments below are valid both for $s$ and $p$-polarized SHG light.
	
	In the $\mathbf{M}_{\pm x}$ states (Supplementary Fig.~\ref{ExtSymm}a), the mirror operation $m_x$ preserves the magnetization along $\hat{\mathbf{x}}$ ($\mathbf{M}_{\pm x}\rightarrow \mathbf{M}_{\pm x}$). Therefore, the SHG intensity $I_\mathbf{M}^\mathbf{k}(\phi)$ has the symmetry $I_\mathbf{M}^\mathbf{k}(\phi) = I_\mathbf{M}^\mathbf{-k}(-\phi)$. As a consequence in the experiment, the polarization dependencies in Supplementary Figs.~\ref{fig:PolScans}a and b are mirror symmetric to Supplementary Figs.~\ref{fig:PolScans}f and g.
	
	In contrast to $m_x$, $m_y$ reverses the magnetization along $\hat{\mathbf{x}}$ as shown in Supplementary Fig.~\ref{ExtSymm}b ($\mathbf{M}_{\pm x}\rightarrow \mathbf{M}_{\mp x}$). Thus, $I_\mathbf{M}^\mathbf{k}(\phi) = I_\mathbf{-M}^\mathbf{k}(-\phi)$. Therefore, Supplementary Figs.~\ref{fig:PolScans}a and f are mirror symmetric to Supplementary Figs.~\ref{fig:PolScans}b and g. 
	
	The situation in the $\mathbf{M}_{\pm y}$ states is slightly different. As shown in Supplementary Fig.~\ref{ExtSymm}c, $m_x$ reverses the magnetization along $\hat{\mathbf{y}}$ ($\mathbf{M}_{\pm y}\rightarrow \mathbf{M}_{\mp y}$). Thus, $I_\mathbf{M}^\mathbf{k}(\phi) = I_\mathbf{-M}^\mathbf{-k}(-\phi)$ (Supplementary Figs.~\ref{fig:PolScans}c and d are symmetric to of Supplementary Figs.~\ref{fig:PolScans}i and h. 
	
	Finally, $m_y$ preserves the magnetization along $\hat{\mathbf{y}}$ as illustrated in Supplementary Fig.~\ref{ExtSymm}d ($\mathbf{M}_{\pm y}\rightarrow \mathbf{M}_{\pm y}$). Thus, $I_\mathbf{M}^\mathbf{k}(\phi) = I_\mathbf{M}^\mathbf{k}(-\phi)$. Therefore, Supplementary Figs.~\ref{fig:PolScans}(c,d,h,i) are individually mirror symmetric. 
	
	Irrespective of the magnetic state, the application of both mirror operations (equivalent to a two-fold rotational axis $2_z$ along the $\hat{\mathbf{z}}$ axis) in any magnetic state always enforces $I_\mathbf{M}^\mathbf{k}(\phi) = I_\mathbf{-M}^\mathbf{-k}(\phi)$. Thus, reversing both the magnetization and the propagation direction always preserves the SHG intensity for any polarization dependence. Note that this property hinges on the presence of a two-fold rotational axis $2_z$ along the $\hat{\mathbf{z}}$ axis in the paramagnetic phase. For a general material, it is therefore not automatically guaranteed that reversing the magnetization is equivalent to reversing the light propagation. As a further consequence of the $2_z$ axis in the paramagnetic phase, the NODE is absent in the paramagnetic phase.
	
	% Figure environment removed
	\clearpage
	
	\newpage
	\section{Integrated SHG spectrum}
	
	In analogy to the NDD as a polarization-independent, directional contrast in the absorption coefficient, we introduced in the main text the integrated SHG intensity as a polarization-independent nonlinear optical observable. In this section, we present additional data illustrating properties of the integrated SHG intensity. In particular, we show that the NODE is present even for the integrated SHG intensity and it is thus not restricted to specific incident and outgoing light polarizations.
	
	We define the integrated SHG intensity as the intensity measured without polarization analysis (thus detecting simultaneously $s$ and $p$ polarized SHG light) and averaged over all incident polarization angles $\phi$. The such defined observable is independent of both incoming and outgoing light polarizations.
	
	As we showed in Sec.~\ref{sec:Symmetry} and Supplementary Fig.~\ref{fig:PolScans}, the detected SHG intensity in the $\mathbf{M}_\mathrm{\pm x}$ states exhibits the symmetry $I_\mathrm{\pm x}^\mathbf{k}(\phi) = I_\mathrm{\pm x}^\mathbf{-k}(-\phi) = I_\mathrm{\mp x}^\mathbf{k}(-\phi)$. Averaging over all polarization angles $\phi$, we find $\langle I_\mathrm{\pm x}^\mathbf{k}\rangle = \langle I_\mathrm{\pm x}^\mathbf{-k}\rangle = \langle I_\mathrm{\mp x}^\mathbf{k}\rangle$, i.e., the integrated SHG intensity in the $\mathbf{M}_\mathrm{\pm x}$ states is independent of the propagation direction and magnetization direction. Accordingly, we find in Supplementary Fig.~\ref{fig:IntSpectrum}a equal integrated SHG intensities in the $\mathbf{M}_\mathrm{\pm x}$ states. 
	
	In contrast, the integrated SHG intensity in the $\mathbf{M}_\mathrm{\pm y}$ states is different. This is due to the absence of a symmetry operation that relates the two magnetic states while preserving the propagation direction. We have either $I_\mathrm{\pm y}^\mathbf{k}(\phi) = I_\mathrm{\mp y}^\mathbf{-k}(-\phi)$ or $I_\mathrm{\pm y}^\mathbf{k}(\phi) = I_\mathrm{\pm y}^\mathbf{k}(-\phi)$ (Sec.~\ref{sec:Symmetry}). Thus, for a fixed propagation direction, the integrated SHG intensity can differ for the two magnetic states $\mathbf{M}_\mathrm{+y}$ and $\mathbf{M}_\mathrm{-y}$. 
	
	Reversing the light propagation (Supplementary Fig.~\ref{fig:IntSpectrum}b), we find in agreement with the symmetry considerations equal integrated SHG intensities for the $\mathbf{M}_\mathrm{\pm x}$ states, but the relation between $\langle I_\mathrm{+y}\rangle$ and $\langle I_\mathrm{-y}\rangle$ is inverted relative to panel a. This property is a direct consequence of the relation $I_\mathrm{\pm y}^\mathbf{k}(\phi) = I_\mathrm{\mp y}^\mathbf{-k}(-\phi)$.
	
	In Supplementary Fig.~\ref{fig:IntSpectrum}c, we show the spectral evolution of the integrated SHG intensity for one propagation direction. Very strikingly, we find $\langle I^{\ra}_{+x}\rangle = \langle I^{\ra}_{-x}\rangle$ for all photon energies.
	
	Overall, we note the following properties of the integrated SHG intensity:
	
	\begin{itemize}
		\item if the SHG intensity does not exhibit a NODE (here: SHG response above $T_C$), the integrated SHG intensity does not exhibit a NODE.
		\item if the SHG intensity shows a NODE, mirror symmetries may still enforce a vanishing NODE in the integrated SHG intensity.
	\end{itemize}
	
	% Figure environment removed
	\clearpage
	
	\newpage
	\section{FIB sample}
	
	In an effort to demonstrate the NODE in a device-like structure and minimize the effect of Joule heating observed in the current switching of the bulk single crystal, we microstructured a CeAlSi crystal into a free-standing bar of approximately \SI{126}{\micro\metre} length, \SI{29}{\micro\metre} width, and \SI{2}{\micro\metre} thickness by focused-ion-beam (FIB) milling (see methods for details). A false-colored SEM picture is shown in Fig.~3f in the main text. In order to identify the magnetic state during the application of electrical current (Fig.~3g in the main text), we pre-characterized the SHG response of the FIB device by applying a static magnetic field of approximately \SI{30}{mT} along the four magnetic easy axes. The incident polarization dependence for the $s$ and $p$ polarized SHG light is shown in Supplementary Fig.~\ref{ExtFIB}. 
	
	Although the polarization dependencies differ from the scans on the bulk single crystal (Supplementary Fig.~\ref{fig:PolScans}), we can identify four significantly different polarization patterns for the four magnetic states. Possible reasons for the differences between Supplementary Figs.~\ref{fig:PolScans} and \ref{ExtFIB} include changes to the surface morphology of the material caused by FIB milling (see methods) and deviations of the angle of incidence from $45^\circ$ (a few degree tilt between the surface of the substrate and the surface of the FIB device was unavoidable due to the extreme aspect ratio of the FIB sample).
	
	A comparison of the magnetic-field poled polarization dependencies in Supplementary Fig.~\ref{ExtFIB} and the electric current induced states (Fig.~3h,i in the main text) reveals a current-induced switching between $\mathbf{M}_\mathrm{+x}$ and $\mathbf{M}_\mathrm{-x}$. Solid lines in Supplementary Fig.~\ref{ExtFIB} are fits. In analogy to Supplementary Fig.~\ref{fig:PolScans}, we fit all measurements simultaneously with one consistent set of $\chi$ tensor components. We list the fit parameters relative to $\chi_{zzz}$ in Supplementary Table~\ref{tab:FIBcomponents} below. A surface tilt of approximately $8^\circ$ was included in the fitting procedure. The excellent fit quality is consistent with the $2^\prime m m^\prime$ symmetry for the magnetic state of the FIB sample.
	
	Moreover, we performed a temperature-dependent characterization of the FIB sample. A magnetic field of approximately \SI{30}{mT} stabilizes a magnetic  $\mathbf{M}_\mathrm{+y}$ state. In Supplementary Fig.~\ref{ExtFIBtemp}a, we show the SHG intensity as a function of temperature in the absence of any electrical current. We find a transition temperature in agreement with the bulk sample. 
	
	In Supplementary Fig.~\ref{ExtFIBtemp}b, we show the SHG intensity in the $\mathbf{M}_\mathrm{+y}$ state at \SI{3}{K} under the application of an electric DC current. We extrapolate a critical current of \SI{8.5}{mA} above which Joule heating raises the sample temperature above the transition temperature. Joule heating is negligible for the currents relevant for Fig.~3g in the main text ( $< \SI{3}{mA}$). 
	
	\begin{table}
		\begin{center}
			\begin{tabular}{|c|c|c|}
				\hline
				$\chi_{ijk}$ & $\vert \chi_{ijk}\vert/\vert \chi_{zzz}\vert$ & $(\Phi_{ijk} - \Phi_{zzz})/\pi$ \\
				\hline
				$zzz$ & 1 & 0 \\
				\hline
				$xxz$ & 0.069 & 0.774 \\
				\hline
				$zxx$ & 0.313 & 0.182 \\
				\hline
				$yyz$ & 0.106 & -0.268 \\
				\hline
				$zyy$ & 0.284 & 0.377 \\
				\hline
				$xxx$ & 0.081 & -0.26 \\
				\hline
				$yyx$ & 0.05 & -0.11 \\
				\hline
				$xyy$ & 0.204 & 0.366 \\
				\hline
				$zzx$ & 0.125 & 0.605 \\
				\hline
				$xzz$ & 0.127 & -0.08 \\
				\hline
			\end{tabular}
			\caption{Magnitudes and phases of $\chi$ tensor components relative to $\chi_{zzz}$. Values were found by fitting all 8 measurements below $T_C$ in Supplementary Fig.~\ref{ExtFIB} simultaneously with one consistent set of $\chi$ tensor components according to point group $2^\prime m m^\prime$.}
			\label{tab:FIBcomponents}
		\end{center}
	\end{table}
	
	% Figure environment removed
	
	% Figure environment removed
	\clearpage
	
	\section{First-principles calculations}
	
	\subsection{Electronic structure of CeAlSi}
	
	We provide here further details about the first-principles calculations described in the main text. In particular, in this section we describe the electronic structure of CeAlSi in more detail based on numerical first-principles calculations. We performed calculations both in the paramagnetic and in the ferromagnetically ordered phase of CeAlSi. A comparison of the band structure is shown in Supplementary Fig.~\ref{fig:DFT_band_structure}. Our calculations agree well with other recent calculations and experimental observations \cite{Sakhya2023}.
	
	Due to the noncentrosymmetric lattice structure, CeAlSi is already a Weyl semimetal in the paramagnetic phase. Going to the ferromagnetic phase, CeAlSi remains to be a Weyl semimetal, with relatively small modifications to the details of the band structure. Figure~\ref{fig:DFT_band_structure} shows that the overall band structures of the paramagnetic and ferromagnetic phases are similar: CeAlSi is a semimetal in both states. 
	
	We found a total of 40 Weyl nodes in the band structure of CeAlSi in the paramagnetic phase. We list their positions and energies in Table~\ref{table:PM_WeylL} and illustrate their positions in Supplementary Fig.~\ref{fig:weyl_distribute}. The magnetic order does not change the number of Weyl nodes, but the position of the 40 Weyl nodes shifts both in energy and in momentum upon the inclusion of the magnetic order (Table~\ref{table:FM_WeylL}). 
	
	Based on the electronic structure in the magnetically ordered phase, we calculate the components of the nonlinear optical susceptibility $\chi_{ijk}$ following the diagrammatic approach to nonlinear optical responses \cite{Parker2019,Takasan2021} (see also methods). We show the real and imaginary parts of the obtained susceptibilities in supplementary Fig.~\ref{Ext5} in the energy range of \SI{650}{meV} to \SI{900}{meV} (incident photon energy). We find that only the components that are allowed by symmetry exhibit finite susceptibilities \cite{Birss66}.
	
	We note here also the absence of any sharp resonances in any tensor component. This is in contrast to wide bandgap insulators, where often sharp resonances occur in the energy range close to the band gap \cite{Xiao2023}. Our numerical calculations thus reproduce a broadband SHG response. In Supplementary Fig.~\ref{fig:DFTNODE}, we show the calculated spectrum of the SHG intensity in a setting that reproduces the experiment in Fig.~1e of the main text ($s$ polarized incident light, $p$ polarized SHG response, $45^\circ$ angle of incidence, $\mathbf{M}_\mathrm{+y}$ state). We observe a clear difference in the SHG intensity between the two propagation directions. As a consequence of the absence of resonances in $\chi_{ijk}$, the SHG intensity evolves smoothly as a function of photon energy.
	
	% Figure environment removed
	
	% Figure environment removed
	
	\begin{table}
		\begin{tabular}{cccc}
			\hline\hline
			{Weyl Nodes} & \multicolumn{2}{c}{Position: $(k_x, k_y, k_z) (1/\textrm{\AA})$} & {Energy (meV)} \\
			& Chirality: -1 & Chirality: +1 & \\
			\hline
			$\textrm{W}_1$ & $(0.007,0.744,0.000)$ & $(-0.744,-0.007,0.000)$ & 58 \\
			& $(0.744,-0.007,0.000)$ & $(0.007,-0.744,0.000)$ & 58 \\
			& $ (-0.007,-0.744,0.000)$ & $(0.744,0.007,0.000)$ & 58 \\
			&  $ (-0.744,0.007,0.000)$ & $(-0.007,0.744,0.000)$ & 58 \\
			\hline
			$\textrm{W}_2$  & $(-0.026,0.372,\pm0.295)$ & $(-0.372,0.026,\pm0.295)$ & 31 \\
			& $(0.372,0.026,\pm0.295)$ & $(-0.026,-0.372,\pm0.295)$ &  31 \\
			& $(0.026,-0.372,\pm0.295)$ & $(0.372,-0.026,\pm0.295)$ & 31 \\
			& $(-0.372,-0.026,\pm0.295)$ & $(0.026,0.372,\pm0.295)$ & 31 \\
			\hline
			$\textrm{W}_3$  & $(0.274,-0.367,0.000)$ & $(0.367,-0.274,0.000)$ & 46 \\
			& $(-0.367,-0.274,0.000)$ & $(0.274,0.367,0.000)$ & 46 \\
			& $(0.367,0.274,0.000)$ & $(-0.274,-0.367,0.000)$ & 46 \\
			& $(-0.274,0.367,0.000)$ & $(-0.367, 0.274,0.000)$ & 46 \\
			\hline
			$\textrm{W}_3^{'}$  & $(0.339,0.263,0.000)$ & $(-0.263,-0.339,0.000)$ & 31 \\
			& $(-0.339,-0.263,0.000)$ & $(0.263,0.339,0.000)$ & 31 \\
			& $(0.263,-0.339,0.000)$ & $(0.339,-0.263,0.000)$ & 31 \\
			& $(-0.263,0.339,0.000)$ & $(-0.339,0.263,0.000)$ & 31 \\
			\hline
			\hline
		\end{tabular}
		\caption{\textbf{Weyl node positions and energies in the paramagnetic phase of CeAlSi.} Weyl node energies are given relative to the Fermi energy.}
		\label{table:PM_WeylL}
	\end{table}
	
	\begin{table}
		\begin{tabular}{cccc}
			\hline\hline
			{Weyl Nodes} & \multicolumn{2}{c}{Position: $(k_x, k_y, k_z)(1/\textrm{\AA})$} & {Energy(meV)} \\
			& Chirality: -1 & Chirality: +1 & \\
			\hline
			$\textrm{W}_1$ & $(0.005,0.760,0.000)$ & $(-0.760,-0.005,0.000)$ & 39 \\
			& $(0.757,-0.004,0.000)$ & $(0.004,-0.757,0.000)$ & 65 \\
			& $ (-0.007,-0.738,0.000)$ & $(0.738,0.007,0.000)$ & 70 \\
			&  $ (-0.743,0.004,0.000)$ & $(-0.004,0.743,0.000)$ & 81 \\
			\hline
			$\textrm{W}_2$  & $(-0.023,0.395,\pm0.284)$ & $(-0.395,0.023,\pm0.284)$ & -24 \\
			& $(0.388,0.023,\pm0.282)$ & $(-0.023,-0.388,\pm0.282)$ & -23 \\
			& $(0.018,-0.369,\pm0.305)$ & $(0.369,-0.018,\pm0.305)$ & 23 \\
			& $(-0.372,-0.024,\pm0.294)$ & $(0.024,0.372,\pm0.294)$ & 26 \\
			\hline
			$\textrm{W}_3$  & $(0.276,-0.366,0.000)$ & $(0.366,-0.276,0.000)$ & 8 \\
			& $(-0.355,-0.282,0.000)$ & $(0.282,0.355,0.000)$ & 37 \\
			& $(0.387,0.267,0.000)$ & $(-0.267,-0.387,0.000)$ & 48 \\
			& $(-0.270,0.372,0.000)$ & $(-0.372,0.270,0.000)$ & 69 \\
			\hline
			$\textrm{W}_3^{'}$  & $(0.333,0.257,0.000)$ & $(-0.257,-0.333,0.000)$ & 10 \\
			& $(-0.340,-0.257,0.000)$ & $(0.257,0.340,0.000)$ & 26 \\
			& $(0.270,-0.339,0.000)$ & $(0.339,-0.270,0.000)$ & 33 \\
			& $(-0.278,0.353,0.000)$ & $(-0.353,0.278,0.000)$ & 52 \\
			\hline
			\hline
		\end{tabular}
		\caption{\textbf{Weyl node positions and energies in the ferromagnetic phase of CeAlSi.} Weyl node energies are given relative to the Fermi energy.}
		\label{table:FM_WeylL}
	\end{table}
	
	% Figure environment removed
	
	% Figure environment removed
	\clearpage
	
	\subsection{$k$-space resolved $\chi$ contributions}
	
	As described in the methods section, we calculate $\chi$ as $\chi_{ijk} = \int_{BZ}d^3\mathbf{k} \xi_{ijk}$, where $\xi_{ijk} = \xi^I_{ijk}+\xi^{II}_{ijk}$ with
	
	\begin{eqnarray}
		\xi^I_{ijk} &=& C \sum_{m\neq n} f_{mn}\left(\frac{h^{ij}_{nm}h^k_{mn} + h^{ik}_{nm}h^j_{mn}}{\omega+i\eta - \omega_{mn}} + \frac{h^{jk}_{mn}h^i_{nm}}{2\omega+i\eta - \omega_{mn}}\right) \label{eqn:supp1}, \\
		\xi^{II}_{ijk} &=& C \sum_{m\neq n\neq p} \frac{h^i_{pm}\left(h^j_{mn}h^k_{np}+h^k_{mn}h^j_{np}\right)}{\omega_{mn}+\omega_{np}}\left(\frac{f_{np}}{\omega + i\eta -\omega_{pn}} + \frac{f_{nm}}{\omega + i\eta -\omega_{nm}} + \frac{2f_{mp}}{\omega + i\eta -\omega_{pm}}\right)\label{eqn:supp2}
	\end{eqnarray}  
	
	To gain further insights into the relation of the SHG response and the electronic structure of CeAlSi, we investigate here the $k$ space distribution of $\xi_{ijk}$. As an example, we focus here in particular on $\xi_{xxy}$ in the ferromagnetic state with magnetization $\mathbf{M}_{+x}$ pointing along $\hat{\mathbf{x}} = [110]$. We show in Supplementary Fig.~\ref{SuppFig:DFTkspace}a the magnitude of $\xi_{xxy}$ at an incident photon energy of \SI{650}{meV}. We chose a plane in the Brillouin zone with $k_z = 0.295 \AA^{-1}$. In the paramagnetic phase of CeAlSi, this plane contains the $W_2$ Weyl nodes. In the ferromagnetic phase, the Weyl nodes shift out of the plane (Supplementary Table~\ref{table:FM_WeylL}). Red and blue dots therefore indicate the projection of the Weyl nodes onto the plane $k_z = 0.295 \AA^{-1}$ in the ferromagnetic phase.
	
	We notice a rather complicated distribution of $\vert\xi_{xxy}\vert$ in the plane, but major contributions appear to be related to the Weyl nodes. Note that $\xi_{xxy}$ contains a sum over all bands (Eqs.~\ref{eqn:supp1} and \ref{eqn:supp2}). The main features of Supplementary Fig.~\ref{SuppFig:DFTkspace}a, however, can already be recognized in Supplementary Fig.~\ref{SuppFig:DFTkspace}b, where we only consider contributions from electronic transitions between bands -1 and +1 (marked red and blue in Fig.~4c of the main text) \footnote{According to the numerically determined band structure (Fig.~4c in the main text), bands -1 and 0 follow each other closely throughout the Brillouin zone (as do bands +1 and +2). Thus, any of the transitions $-1 \rightarrow +1$, $-1 \rightarrow +2$, $0 \rightarrow +1$, or $0 \rightarrow +2$ yields a figure similar to Supplementary Fig.~\ref{SuppFig:DFTkspace}b. The choice of bands -1 and +1 for Supplementary Fig.~\ref{SuppFig:DFTkspace}b is arbitrary and serves as an example.}.
	
	Note that each pair of Weyl nodes is surrounded by two lines of strong contributions to $\xi_{xxy}$. The inner line corresponds to electronic transitions that are resonant at the incident photon energy (here: \SI{650}{meV}); the outer line corresponds to transitions resonant at the second-harmonic photon energy (here: $2\times\SI{650}{meV}$).
	
	As we vary the incident photon energy between \SI{550}{meV} (Supplementary Fig.~\ref{SuppFig:DFTkspace}c) and  \SI{950}{meV} (Supplementary Fig.~\ref{SuppFig:DFTkspace}g), we do not observe major changes, but the distance of the inner and outer lines to the Weyl node pair changes. In fact, the distance of the lines of $\xi_{xxy}$ contributions from the Weyl nodes increases approximately linearly with the incident photon energy (dashed black lines in panels c-g), which is due to the linearly dispersive bands in the vicinity of the Weyl nodes.
	
	In Supplementary Fig.~\ref{SuppFig:DFTkspace}h, we show a schematic illustration of a possible band structure near the line with $k_\mathrm{[010]} = 0$, which would qualitatively reproduce our observations from the DFT calculations.
	
	Interestingly, the magnitude of $\xi_{xxy}$ does not change significantly within the considered energy range (in line with the smooth changes of $\chi_{ijk}$, Supplementary Fig. 17). The comparable magnitude of $\xi$ over a wide energy range in combination with smooth changes of the $\xi$ contributions throughout the Brillouin zone rationalize the broadband SHG response that we observed in the experiment.
	
	% Figure environment removed
	
	%\section{Oersted field effects}
	
	%Numerical finite-element simulations indicate that current-induced Oersted fields (https://doi.org/10.21203/rs.3.rs-716302/v1) for currents as large as \SI{120}{mA} may be of the order of \SI{5}{Oe} at the surface of the sample, which is comparable to the coercive magnetic field of \SI{12}{Oe} (SI Sec. XXX). To investigate the role of Oersted fields, we machined a miniaturized sample by focussed-ion beam milling. Indeed, we observed current-induced magnetization control with currents as small as \SI{2.5}{mA} corresponding to surface Oersted fields around \SI{0.5}{Oe}. 
	
	\clearpage
	\renewcommand{\baselinestretch}{1.2}
	\bibliographystyle{Science}	

	\providecommand{\noopsort}[1]{}\providecommand{\singleletter}[1]{#1}%
	\begin{thebibliography}{10}
		
		\bibitem{Birss66}
		R.~Birss, {\it Symmetry and Magnetism\/} (North Holland Publishing Company,
		1966), second edn.
		
		\bibitem{Yang2021}
		H.-Y. Yang, {\it et~al.\/}, Noncollinear ferromagnetic Weyl semimetal with
		anisotropic anomalous Hall effect, {\it Phys. Rev. B\/} {\bf 103}, 115143
		(2021).
		
		\bibitem{Toyoda2021}
		S.~Toyoda, M.~Fiebig, T.~hisa Arima, Y.~Tokura, N.~Ogawa, Nonreciprocal second
		harmonic generation in a magnetoelectric material, {\it Sci. Adv.\/} {\bf 7},
		eabe2793 (2021).
		
		\bibitem{Mund2021}
		J.~Mund, {\it et~al.\/}, Toroidal nonreciprocity of optical second harmonic
		generation, {\it Phys. Rev. B\/} {\bf 103}, L180410 (2021).
		
		\bibitem{Shen66}
		Y.~R. Shen, N.~Bloembergen, Interaction between Light Waves and Spin Waves,
		{\it Phys. Rev.\/} {\bf 143}, 372 (1966).
		
		\bibitem{Tokura2018}
		Y.~Tokura, N.~Nagaosa, Nonreciprocal responses from non-centrosymmetric quantum
		materials, {\it Nat. Commun.\/} {\bf 9}, 3740 (2018).
		
		\bibitem{Pavlov03}
		V.~V. Pavlov, R.~V. Pisarev, M.~Fiebig, D.~Fr\"ohlich, Optical harmonic
		generation in magnetic garnet epitaxial films near the fundamental absorption
		edge, {\it Phys. Solid State\/} {\bf 45}, 662 (2003).
		
		\bibitem{Toyoda23}
		S.~Toyoda, {\it et~al.\/}, Magnetic-field switching of second-harmonic
		generation in noncentrosymmetric magnet {E}u$_2${M}n{S}i$_2${O}$_7$, {\it
			Phys. Rev. Materials\/} {\bf 7}, 024403 (2023).
		
		\bibitem{Sun2021Mapping}
		Y.~Sun, {\it et~al.\/}, Mapping domain-wall topology in the magnetic Weyl
		semimetal {CeAlSi}, {\it Phys. Rev. B\/} {\bf 104}, 235119 (2021).
		
		\bibitem{Sakhya2023}
		A.~P. Sakhya, {\it et~al.\/}, Observation of Fermi arcs and Weyl nodes in a
		noncentrosymmetric magnetic Weyl semimetal, {\it Phys. Rev. Materials\/} {\bf
			7}, L051202 (2023).
		
		\bibitem{Parker2019}
		D.~E. Parker, T.~Morimoto, J.~Orenstein, J.~E. Moore, Diagrammatic approach to
		nonlinear optical response with application to {Weyl} semimetals, {\it Phys.
			Rev. B\/} {\bf 99}, 045121 (2019).
		
		\bibitem{Takasan2021}
		K.~Takasan, T.~Morimoto, J.~Orenstein, J.~E. Moore, Current-induced second
		harmonic generation in inversion-symmetric Dirac and Weyl semimetals, {\it
			Phys. Rev. B\/} {\bf 104}, L161202 (2021).
		
		\bibitem{Xiao2023}
		R.-C. Xiao, {\it et~al.\/}, Classification of second harmonic generation effect
		in magnetically ordered materials, {\it npj Quantum Materials\/} {\bf 8}, 62
		(2023).
		
		\bibitem{Note1}
		According to the numerically determined band structure (Fig.~4c in the main
		text), bands -1 and 0 follow each other closely throughout the Brillouin zone
		(as do bands +1 and +2). Thus, any of the transitions $-1 \rightarrow +1$,
		$-1 \rightarrow +2$, $0 \rightarrow +1$, or $0 \rightarrow +2$ yields a
		figure similar to Supplementary Fig.~\ref {SuppFig:DFTkspace}b. The choice of
		bands -1 and +1 for Supplementary Fig.~\ref {SuppFig:DFTkspace}b is arbitrary
		and serves as an example.
		
	\end{thebibliography}
	
\end{document}


