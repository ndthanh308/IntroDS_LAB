
%% bare_jrnl.tex
%% V1.4b
%% 2015/08/26
%% by Michael Shell
%% see http://www.michaelshell.org/
%% for current contact information.
%%
%% This is a skeleton file demonstrating the use of IEEEtran.cls
%% (requires IEEEtran.cls version 1.8b or later) with an IEEE
%% journal paper.
%%
%% Support sites:
%% http://www.michaelshell.org/tex/ieeetran/
%% http://www.ctan.org/pkg/ieeetran
%% and
%% http://www.ieee.org/

%%*************************************************************************
%% Legal Notice:
%% This code is offered as-is without any warranty either expressed or
%% implied; without even the implied warranty of MERCHANTABILITY or
%% FITNESS FOR A PARTICULAR PURPOSE! 
%% User assumes all risk.
%% In no event shall the IEEE or any contributor to this code be liable for
%% any damages or losses, including, but not limited to, incidental,
%% consequential, or any other damages, resulting from the use or misuse
%% of any information contained here.
%%
%% All comments are the opinions of their respective authors and are not
%% necessarily endorsed by the IEEE.
%%
%% This work is distributed under the LaTeX Project Public License (LPPL)
%% ( http://www.latex-project.org/ ) version 1.3, and may be freely used,
%% distributed and modified. A copy of the LPPL, version 1.3, is included
%% in the base LaTeX documentation of all distributions of LaTeX released
%% 2003/12/01 or later.
%% Retain all contribution notices and credits.
%% ** Modified files should be clearly indicated as such, including  **
%% ** renaming them and changing author support contact information. **
%%*************************************************************************


% *** Authors should verify (and, if needed, correct) their LaTeX system  ***
% *** with the testflow diagnostic prior to trusting their LaTeX platform ***
% *** with production work. The IEEE's font choices and paper sizes can   ***
% *** trigger bugs that do not appear when using other class files.       ***                          ***
% The testflow support page is at:
% http://www.michaelshell.org/tex/testflow/



\documentclass[journal]{IEEEtran}
%
% If IEEEtran.cls has not been installed into the LaTeX system files,
% manually specify the path to it like:
% \documentclass[journal]{../sty/IEEEtran}





% Some very useful LaTeX packages include:
% (uncomment the ones you want to load)


% *** MISC UTILITY PACKAGES ***
%
%\usepackage{ifpdf}
% Heiko Oberdiek's ifpdf.sty is very useful if you need conditional
% compilation based on whether the output is pdf or dvi.
% usage:
% \ifpdf
%   % pdf code
% \else
%   % dvi code
% \fi
% The latest version of ifpdf.sty can be obtained from:
% http://www.ctan.org/pkg/ifpdf
% Also, note that IEEEtran.cls V1.7 and later provides a builtin
% \ifCLASSINFOpdf conditional that works the same way.
% When switching from latex to pdflatex and vice-versa, the compiler may
% have to be run twice to clear warning/error messages.






% *** CITATION PACKAGES ***
%
%\usepackage{cite}
% cite.sty was written by Donald Arseneau
% V1.6 and later of IEEEtran pre-defines the format of the cite.sty package
% \cite{} output to follow that of the IEEE. Loading the cite package will
% result in citation numbers being automatically sorted and properly
% "compressed/ranged". e.g., [1], [9], [2], [7], [5], [6] without using
% cite.sty will become [1], [2], [5]--[7], [9] using cite.sty. cite.sty's
% \cite will automatically add leading space, if needed. Use cite.sty's
% noadjust option (cite.sty V3.8 and later) if you want to turn this off
% such as if a citation ever needs to be enclosed in parenthesis.
% cite.sty is already installed on most LaTeX systems. Be sure and use
% version 5.0 (2009-03-20) and later if using hyperref.sty.
% The latest version can be obtained at:
% http://www.ctan.org/pkg/cite
% The documentation is contained in the cite.sty file itself.




\usepackage{graphicx}% Include figure files
\usepackage{dcolumn}% Align table columns on decimal point
\usepackage{bm}% bold math
\usepackage{float}% bold math
\usepackage{amsmath}
\usepackage{cite}
%\usepackage{hyperref}% add hypertext capabilities
%\usepackage[mathlines]{lineno}% Enable numbe

% *** GRAPHICS RELATED PACKAGES ***
%
\ifCLASSINFOpdf
  %\usepackage[pdftex]{graphicx}
  % declare the path(s) where your graphic files are
  % \graphicspath{{../pdf/}{../jpeg/}}
  % and their extensions so you won't have to specify these with
  % every instance of \includegraphics
  % \DeclareGraphicsExtensions{.pdf,.jpeg,.png}
\else
  % or other class option (dvipsone, dvipdf, if not using dvips). graphicx
  % will default to the driver specified in the system graphics.cfg if no
  % driver is specified.
  % \usepackage[dvips]{graphicx}
  % declare the path(s) where your graphic files are
  % \graphicspath{{../eps/}}
  % and their extensions so you won't have to specify these with
  % every instance of \includegraphics
  % \DeclareGraphicsExtensions{.eps}
\fi
% graphicx was written by David Carlisle and Sebastian Rahtz. It is
% required if you want graphics, photos, etc. graphicx.sty is already
% installed on most LaTeX systems. The latest version and documentation
% can be obtained at: 
% http://www.ctan.org/pkg/graphicx
% Another good source of documentation is "Using Imported Graphics in
% LaTeX2e" by Keith Reckdahl which can be found at:
% http://www.ctan.org/pkg/epslatex
%
% latex, and pdflatex in dvi mode, support graphics in encapsulated
% postscript (.eps) format. pdflatex in pdf mode supports graphics
% in .pdf, .jpeg, .png and .mps (metapost) formats. Users should ensure
% that all non-photo figures use a vector format (.eps, .pdf, .mps) and
% not a bitmapped formats (.jpeg, .png). The IEEE frowns on bitmapped formats
% which can result in "jaggedy"/blurry rendering of lines and letters as
% well as large increases in file sizes.
%
% You can find documentation about the pdfTeX application at:
% http://www.tug.org/applications/pdftex





% *** MATH PACKAGES ***
%
%\usepackage{amsmath}
% A popular package from the American Mathematical Society that provides
% many useful and powerful commands for dealing with mathematics.
%
% Note that the amsmath package sets \interdisplaylinepenalty to 10000
% thus preventing page breaks from occurring within multiline equations. Use:
%\interdisplaylinepenalty=2500
% after loading amsmath to restore such page breaks as IEEEtran.cls normally
% does. amsmath.sty is already installed on most LaTeX systems. The latest
% version and documentation can be obtained at:
% http://www.ctan.org/pkg/amsmath





% *** SPECIALIZED LIST PACKAGES ***
%
%\usepackage{algorithmic}
% algorithmic.sty was written by Peter Williams and Rogerio Brito.
% This package provides an algorithmic environment fo describing algorithms.
% You can use the algorithmic environment in-text or within a figure
% environment to provide for a floating algorithm. Do NOT use the algorithm
% floating environment provided by algorithm.sty (by the same authors) or
% algorithm2e.sty (by Christophe Fiorio) as the IEEE does not use dedicated
% algorithm float types and packages that provide these will not provide
% correct IEEE style captions. The latest version and documentation of
% algorithmic.sty can be obtained at:
% http://www.ctan.org/pkg/algorithms
% Also of interest may be the (relatively newer and more customizable)
% algorithmicx.sty package by Szasz Janos:
% http://www.ctan.org/pkg/algorithmicx




% *** ALIGNMENT PACKAGES ***
%
%\usepackage{array}
% Frank Mittelbach's and David Carlisle's array.sty patches and improves
% the standard LaTeX2e array and tabular environments to provide better
% appearance and additional user controls. As the default LaTeX2e table
% generation code is lacking to the point of almost being broken with
% respect to the quality of the end results, all users are strongly
% advised to use an enhanced (at the very least that provided by array.sty)
% set of table tools. array.sty is already installed on most systems. The
% latest version and documentation can be obtained at:
% http://www.ctan.org/pkg/array


% IEEEtran contains the IEEEeqnarray family of commands that can be used to
% generate multiline equations as well as matrices, tables, etc., of high
% quality.




% *** SUBFIGURE PACKAGES ***
%\ifCLASSOPTIONcompsoc
%  \usepackage[caption=false,font=normalsize,labelfont=sf,textfont=sf]{subfig}
%\else
%  \usepackage[caption=false,font=footnotesize]{subfig}
%\fi
% subfig.sty, written by Steven Douglas Cochran, is the modern replacement
% for subfigure.sty, the latter of which is no longer maintained and is
% incompatible with some LaTeX packages including fixltx2e. However,
% subfig.sty requires and automatically loads Axel Sommerfeldt's caption.sty
% which will override IEEEtran.cls' handling of captions and this will result
% in non-IEEE style figure/table captions. To prevent this problem, be sure
% and invoke subfig.sty's "caption=false" package option (available since
% subfig.sty version 1.3, 2005/06/28) as this is will preserve IEEEtran.cls
% handling of captions.
% Note that the Computer Society format requires a larger sans serif font
% than the serif footnote size font used in traditional IEEE formatting
% and thus the need to invoke different subfig.sty package options depending
% on whether compsoc mode has been enabled.
%
% The latest version and documentation of subfig.sty can be obtained at:
% http://www.ctan.org/pkg/subfig




% *** FLOAT PACKAGES ***
%
%\usepackage{fixltx2e}
% fixltx2e, the successor to the earlier fix2col.sty, was written by
% Frank Mittelbach and David Carlisle. This package corrects a few problems
% in the LaTeX2e kernel, the most notable of which is that in current
% LaTeX2e releases, the ordering of single and double column floats is not
% guaranteed to be preserved. Thus, an unpatched LaTeX2e can allow a
% single column figure to be placed prior to an earlier double column
% figure.
% Be aware that LaTeX2e kernels dated 2015 and later have fixltx2e.sty's
% corrections already built into the system in which case a warning will
% be issued if an attempt is made to load fixltx2e.sty as it is no longer
% needed.
% The latest version and documentation can be found at:
% http://www.ctan.org/pkg/fixltx2e


%\usepackage{stfloats}
% stfloats.sty was written by Sigitas Tolusis. This package gives LaTeX2e
% the ability to do double column floats at the bottom of the page as well
% as the top. (e.g., "% Figure environment removed
%
% Note that often IEEE papers with subfigures do not employ subfigure
% captions (using the optional argument to \subfloat[]), but instead will
% reference/describe all of them (a), (b), etc., within the main caption.
% Be aware that for subfig.sty to generate the (a), (b), etc., subfigure
% labels, the optional argument to \subfloat must be present. If a
% subcaption is not desired, just leave its contents blank,
% e.g., \subfloat[].


% An example of a floating table. Note that, for IEEE style tables, the
% \caption command should come BEFORE the table and, given that table
% captions serve much like titles, are usually capitalized except for words
% such as a, an, and, as, at, but, by, for, in, nor, of, on, or, the, to
% and up, which are usually not capitalized unless they are the first or
% last word of the caption. Table text will default to \footnotesize as
% the IEEE normally uses this smaller font for tables.
% The \label must come after \caption as always.
%
%\begin{table}[!t]
%% increase table row spacing, adjust to taste
%\renewcommand{\arraystretch}{1.3}
% if using array.sty, it might be a good idea to tweak the value of
% \extrarowheight as needed to properly center the text within the cells
%\caption{An Example of a Table}
%\label{table_example}
%\centering
%% Some packages, such as MDW tools, offer better commands for making tables
%% than the plain LaTeX2e tabular which is used here.
%\begin{tabular}{|c||c|}
%\hline
%One & Two\\
%\hline
%Three & Four\\
%\hline
%\end{tabular}
%\end{table}


% Note that the IEEE does not put floats in the very first column
% - or typically anywhere on the first page for that matter. Also,
% in-text middle ("here") positioning is typically not used, but it
% is allowed and encouraged for Computer Society conferences (but
% not Computer Society journals). Most IEEE journals/conferences use
% top floats exclusively. 
% Note that, LaTeX2e, unlike IEEE journals/conferences, places
% footnotes above bottom floats. This can be corrected via the
% \fnbelowfloat command of the stfloats package.

%%%%%%%%%%%%%%%%%%%%%%%%%%%%%%%%%%%%%%%%%%%%%%%%%%%%%%%%%
\section{Conceptual Model} \label{sec2}
%%%%%%%%%%%%%%%%%%%%%%%%%%%%%%%%%%%%%%%%%%%%%%%%%%%%%%%%%
%%%%%%%%%%%%%%%%%%%%%%%%%%%%%%%%%%%%%%%%%%%%%%%%%%%%%%%%%
\subsection{Single-layered MTS antennas} \label{subsec2A}
%%%%%%%%%%%%%%%%%%%%%%%%%%%%%%%%%%%%%%%%%%%%%%%%%%%%%%%%%
To begin with, we briefly review the basic principle behind the design of single-layered MTS antennas. Then, building upon this principle, we will discuss our approach to designing double-layered MTS antennas for dual-frequency operation. An $e^{j\omega t}$ time convention is used and suppressed throughout the paper.
%%%%%%%%%%%%%%%%%%%%  Fig. 2   %%%%%%%%%%%%%%%%%%%%%%%%%%%%%%%%%
% Figure environment removed

The design of single-layered MTS antennas is based on the two-dimensional (2-D) canonical problem shown in Fig.~\ref{two}(a). It consists of a periodically modulated sheet reactance lying on a dielectric substrate backed by a metallic ground plane. The structure is assumed to be invariant along the $y$ direction and we consider propagation along $x$. $\varepsilon_r$ and $h$ denote the relative permittivity and the thickness, respectively, of the dielectric slab. The sheet reactance imposes the following boundary condition 
\begin{equation}\label{IBC}
\boldsymbol{E}_t\left( x,z=0 \right)=jX_s\left( x \right)\left [ \boldsymbol{H}_t\left( x,z=0^{+} \right) - \boldsymbol{H}_t\left( x,z=0^{-} \right) \right]
\end{equation}
and its modulation profile is usually described by a sinusoidal function
\begin{equation}\label{eq1}
Z_s\left( x \right) = jX_s\left( x \right) = jX_0 \left [ 1+M \text{cos}\left(\frac{2\pi x}{d}\right) \right]
\end{equation}
where $\boldsymbol{E}_t$ and $\boldsymbol{H}_t$ are the tangential components of the electric and magnetic fields, respectively, at the sheet level, $X_0<0$ is the average reactance, $M$ is the modulation index, and $d$ is the modulation period, . This structure supports a TM (transverse magnetic) mode without cut-off. Due to the periodic nature of $X_s$, such a mode can be represented through an infinite set of Floquet waves (FWs) \cite{oliner1959,martini2020}. The longitudinal (along $x$) wavenumber of the $n$-indexed FW is
\begin{equation}\label{eq2}
k_{x}^{\left(n\right)} = k_{x}^{\left(0\right)}+\frac{2\pi n}{d}
\end{equation}
where $k_{x}^{\left(0\right)}$ is the longitudinal wavenumber of the $0$-indexed  FM, which is solution of the dispersion equation. For $d$ larger than a critical value \cite{oliner1959}, at least one of the higher-order FMs ends up inside the light cone, and leaky waves occur. In this case, the solution of the dispersion equation takes a complex value and can be represented as 
\begin{equation}\label{eq3}
k_{x}^{\left(0\right)} = \beta_{sw}+\Delta \beta - j\alpha
\end{equation}
where $\beta_{sw}$ is the longitudinal wavenumber of the surface wave propagating on the unmodulated surface impedance ($M=0$), and the positive real numbers $\Delta \beta$ and $\alpha$ account for the perturbation in the phase and amplitude, respectively, induced by the modulation, with $\Delta \beta \ll \beta_{sw}$. In MTS antennas, the modulation period ($d$) is usually chosen such that only the ($-1$)-indexed FM is inside the light cone. As a result, the pointing angle ($\theta$), which
is defined with respect to the z axis, and $d$ are related as
follows
\begin{equation}\label{eq4}
\beta_{sw}+\Delta \beta - \frac{2\pi}{d}= k \text{sin}\theta
\end{equation}
with $k$ being the free-space wavenumber.

As mentioned in Sec.~\ref{sec1}, the desired continuous impedance profile can be accurately synthesized by subwavelength metallic patches smoothly varying along the surface \cite{martini2020} (see Fig.~\ref{two}(b)). The gradual variation of the patches allows for the use of the local periodicity approximation. Namely, the equivalent impedance of each patch can be extracted as if it was embedded in a periodic environment [inset of Fig.~\ref{two}(b)]. A periodic pattern modeled with the equivalent transmission line model shown in Fig.~\ref{two}(c) is studied to build databases linking the equivalent impedance ($Z_p = jX_p$) with the geometrical parameters of the unit cell. Different techniques for an efficient and accurate impedance extraction of periodic metallic claddings printed on a grounded slab are available in the published literature (see, e.g.,\cite{luuk2008,mencagli2015,mencagli2016}).

%%%%%%%%%%%%%%%%%%%%%%%%%%%%%%%%%%%%%%%%%%%%%%%%%%%%%%%%%%%%%%%%%%%%%%%%%%%%%%%%%%%%%%%%%%
\subsection{Dual-band Double-layered MTS antenna} \label{subsec2B}
%%%%%%%%%%%%%%%%%%%%%%%%%%%%%%%%%%%%%%%%%%%%%%%%%%%%%%%%%%%%%%%%%%%%%%%%%%%%%%%%%%%%%%%%%%
Based on the procedure summarized in the previous section, our goal now is to develop a new methodology that allows performing two independent MTS antenna designs working at two different frequencies, that can be merged together, resulting in a single flat dual-band radiator. We assume the two frequencies of interest are $f_1$ and $f_2$, with $f_1<<f_2$. Henceforth, all the frequency-dependent physical quantities introduced in the previous section are assigned with subscripts $1$ and $2$, corresponding to $f_1$ and $f_2$, respectively. 

Let us assume we have the two canonical problems shown in Figs.~\ref{three}(a) and (b). The first one [Fig.~\ref{three}(a)], which is identical to that shown in Fig.~\ref{two}(a) with a surface reactance sheet $X_{s1}$ and slab thickness $h_1$, is set up to perform at $f_1$ with a pointing angle $\theta_1$. The second one [Fig.~\ref{three}(b)] is slightly different than the previous one. That is, the sinusoidally modulated reactance sheet ($X_{s2}$) is embedded in the dielectric slab and located at a distance $h_2$ from the ground plane, with $h_2<h_1$. This difference in the stack-up is rigorously accounted for by properly defining the problem's Green's function. Hence, the antenna design principle remains unchanged and the modulation is set up to operate at $f_2$ with a pointing angle $\theta_2$. As discussed in Sec.~\ref{subsec2A}, the two reactance sheets can be implemented by gradually changing the size and geometry of subwavelength patches [see Figs.~\ref{three}(c) and (d)]. Each patch is linked to an equivalent impedance extracted from an equivalent transmission line model based on the local periodicity approximation. This model can be defined in the two cases as shown in the insets of Figs.~\ref{three}(c) and (d), respectively, with $X_{p1}$ and $X_{p2}$ representing the equivalent reactance of the patches in a locally periodic environment. 

Now, given the two MTSs of Figs.~\ref{three}(c) and (d) operating at $f_1$ and $f_2$, respectively, we wish that their combination [Fig.~\ref{three}(e)] is capable of performing at both frequencies of interest. At first sight, the obtained double-layered MTS cannot perform equally well as the two single-layered MTSs. In fact, when the two metallic patterns are placed closely one to another, their coupling through near-field interactions is likely to imply a change in the equivalent impedance of each pattern. Since the patterns have been designed independently, the radiation performances can be noticeably affected by this change. In principle, the coupling between the two metallic layers can be rigorously accounted for in the design process by extracting the equivalent reactance of the patches in the two-layer configuration. However, this would results in a procedure significantly more complicated than the well-established techniques available for the design of MTSs \cite{luuk2008,mencagli2015,mencagli2016}. Also, in order this approach to be applicable, it is necessary to impose a restriction on the size of the unit cells for the two lattices: $a_1$ and $a_2$ must be commensurable, so that a common period could be identified. 

An alternative approach that does not require the extraction of the equivalent impedance in double-layered periodic structures and does not pose any restriction on the unit cell sizes would be preferable. The proposed approach aims at satisfying these requirement by focusing on a suitable impedance synthesis of the two single-layered MTSs [Figs.~\ref{three}(c) and (d)] such that they perform equally well when they are combined together [Fig.~\ref{three}(e)].  
%%%%%%%%%%%%%%%%%%%%  Fig. 3   %%%%%%%%%%%%%%%%%%%%%%%%%%%%%%%%%
% Figure environment removed
%%%%%%%%%%%%%%%%%%%%%%%%%%%%%%%%%%%%%%%%%%%%%%%%%%%%%%%%%
%%%%%%%%%%%%%%%%%%%%  Fig. 4   %%%%%%%%%%%%%%%%%%%%%%%%%%%%%%%%%
% Figure environment removed
%%%%%%%%%%%%%%%%%%%%%%%%%%%%%%%%%%%%%%%%%%%%%%%%%%%%%%%%%

As is well known, the equivalent reactance provided by a periodic array of metallic patches is purely imaginary and exhibits a capacitive behavior in the low frequency regime. As a result, according to the Foster's reactance theorem, its frequency response presents a pole at zero frequency and then it increases monotonically alternating zeros and poles for increasing frequencies. The unit cells of single-frequency MTS antennas are normally designed to operate between the first zero and the first pole. Hence, we assume $X_{p1}$ and $X_{p2}$ also reside in that region at their operative frequency
(see $X_{p1}$ and $X_{p2}$ in Figs.~\ref{four}(a) and (b), respectively). Now, the question is: what is the equivalent reactance of the patches designed to operate at one frequency as seen from the wave supported by the single-layered MTS operating at the other frequency? To address this question, let's start from the patches designed to operate at $f_2$ (red patches in Fig.~\ref{three}(e)). Since our objective is not to perturb the behaviour of the impedance sheet $X_{p1}$ at $f_1$, we assume to have at that frequency the same wave that would be supported by the single-layered MTS [Fig.~\ref{three}(c)], corresponding to having an infinite reactance (open circuit) in the bottom layer. Hence, to verify the consistency of the assumption, we need to evaluate the equivalent impedance of the patches implementing the profile $X_{p2}$ by simultaneously setting the frequency equal to $f_2$ and the longitudinal wavenumber equal to $\beta_{sw1}$ (see the bottom inset of Fig.~\ref{four}(a)). Notice that this combination does not satisfy the resonance equation, and therefore it does not correspond to a mode supported by the structure. 
Hence, this process implies extracting the reactance of the patches {\it outside the relevant dispersion curve} and can be carried out through, for example, the periodic Method of Moment (MoM) described in \cite{maci2006}.
In practice, since we assumed $f_1<<f_2$, $a_2$ will be much smaller than $a_1$ and so will be the relevant patches. Thus, we can assume that their equivalent reactance evaluated at $f_1$ and $\beta_{sw1}$ will be very close to the pole at zero frequency, corresponding to a quasi open-circuit [Fig.~\ref{four}(a)]. As a result, the MTS operating at $f_2$ [red patches in Fig.~\ref{three}(e)] are almost transparent at $f_1$, with no need for additional design strategies. 

The same procedure must be repeated to evaluate the equivalent reactance of the patches implementing $X_{p1}$ at $f_2$ and $\beta_{sw2}$. However, now $a_1$ and the relevant patches are electrically large at $f_2$. For this reason, the patches  working at $f_1$ [blue patches in Fig.~\ref{four}(b)] need to be properly designed such that their equivalent impedance presents a pole close to $f_2$, thus, behaving like a quasi-open circuit at that frequency [Fig.~\ref{four}(b)]. Note that this pole will be at higher frequency with respect to the first zero; depending on the separation between $f_1$ and $f_2$, it could be the second or the third pole. 

Following this strategy, the double-layered MTS of Fig.~\ref{three}(e) obtained from the combination of the two single-layered MTSs of Figs.~\ref{three}(c) and (d) operating at $f_1$ and $f_2$, respectively, is expected to perform well at both the frequencies of interest. 
%%%%%%%%%%%%%%%%%%%%%%%%%%%%%%%%%%%%%%%%%%%%%%%%%%%%%%%%%
\section{Antenna implementation} \label{sec3}
%%%%%%%%%%%%%%%%%%%%%%%%%%%%%%%%%%%%%%%%%%%%%%%%%%%%%%%%%
To demonstrate the feasibility of the proposed approach for the design of dual-band double-layered MTS antennas, we chose two largely separated popular frequencies in climate science radars such as $35.75$GHz ($f_1$) and $94.05$GHz ($f_2$) \cite{nagaraja2021}. The dielectric substrate is Roger $4003$C with $\varepsilon_r=3.55$ and thickness $h_1=406 \mu$m. The canonical problems at two frequencies [Figs.~\ref{three}(a) and (b)] were studied with the approach proposed in \cite{martini2020}. For the canonical problem at $f_2$ [Fig.~\ref{three}(b)], the Green's function in \cite{martini2020} was suitably changed to consider that the modulated reactance sheet sits inside the dielectric slab ($h_2=h_1/2$) instead of on the top. Defining $X_{s1}$ and $X_{s2}$ in accordance with the modulation profile in \ref{eq1} with $M_{1}=M_{2}=0.3$, $X_{01}=-150\Omega$, $X_{02}=-100\Omega$, $d_{1}=7.4$mm, and $d_{2}=2.15$mm, the canonical problems at $f_1$ and $f_2$ support a fundamental ($0$-indexed) TM FW with wavenumbers $k_{x1}^{\left(0\right)}=778.7-j0.08\,$rad/m and $k_{x2}^{\left(0\right)}=2985.5-j5.5\,$rad/m, respectively. With $M_{1}=M_{2}=0$ (unmodulated surface impedances), $k_{x1}^{\left(0\right)}$ and $k_{x2}^{\left(0\right)}$ reduce to $\beta_{sw1}=777.8\,$rad/m ($\Delta\beta_1=0.9$) and $\beta_{sw2}=2981.27\,$rad/m ($\Delta\beta_2=4.23$), respectively. With this setup, it is straightforward to verify through Eq.~(\ref{eq4}) that the ($-1$)-indexed FW at both frequencies resides inside the light cone generating a broadside beam ($\theta_1=\theta_2=0$). $X_{s1}$ and $X_{s2}$ are synthesized with the square unit cells shown in the top-left and bottom-right insets, respectively, of Fig.~\ref{five}. Both unit cells are based on the same metallic element topology, consisting of a double-anchor patch. The patch sits at the air-dielectric interface and within the dielectric slab ($h_2=h_1/2$) in the unit cell for $f_1$ and $f_2$, respectively (see the insets of Fig.~\ref{five}). The two unit cells with side $a_1=1.29$mm and $a_2=0.43$mm were studied with the MoM in \cite{maci2006} for different patch sizes forcing a longitudinal wavenumber $\beta_{sw1}$ at $f_1$ in the first case and a longitudinal wavenumber $\beta_{sw2}$ at $f_2$ in the second case. The reactances databases ($X_{p1}$ and $X_{p2}$) as a function of the patch size are shown in Fig.~\ref{five}. By using the same MoM, the equivalent TM reactance was extracted by forcing in the periodic problem with the unit cell designed for $f_2$ (bottom-right inset of Fig.~\ref{five}) a surface wave at frequency $f_1$ with longitudinal wavenumber $\beta_{sw1}$ [Fig.~\ref{six}(a)]. With this setup, one can observe that such a unit cell operates in extremely high impedance regime (quasi-open circuit) at $f_1$ for all the patch sizes of interest. The same process was repeated with the unit cell designed for $f_1$ (top-left inset of Fig.~\ref{five}) imposing a surface wave with frequency $f_2$ and longitudinal wavenumber $\beta_{sw2}$. The extracted reactance is shown in Fig.~\ref{six}(b). The range of $L_1$ was extended with respect to the one of Fig.~\ref{five} to show the presence of a pole. One can observe that the location of the pole is right before the range of operation assumed in the reactance database of Fig.~\ref{five}. Thus, the patches designed for $f_1$ operate in extremely high impedance regime (quasi-open circuit) at $f_2$ , as can be seen in the inset of Fig.~\ref{six}(b). It is worth emphasizing that, although in the design under consideration the relation between the sides of the two unit cells ended up being $a_1 = 3a_2$, our approach, as discussed in the previous section, does not prevent using unit cells whose sides are not commensurable. 
%%%%%%%%%%%%%%%%%%%%  Fig. 5   %%%%%%%%%%%%%%%%%%%%%%%%%%%%%%%%%
% Figure environment removed
%%%%%%%%%%%%%%%%%%%%%%%%%%%%%%%%%%%%%%%%%%%%%%%%%%%%%%%%%
%%%%%%%%%%%%%%%%%%%%  Fig. 6   %%%%%%%%%%%%%%%%%%%%%%%%%%%%%%%%%
% Figure environment removed
%%%%%%%%%%%%%%%%%%%%%%%%%%%%%%%%%%%%%%%%%%%%%%%%%%%%%%%%%
%%%%%%%%%%%%%%%%%%%%  Fig. 1   %%%%%%%%%%%%%%%%%%%%%%%%%%%%%%%%%
% Figure environment removed
%%%%%%%%%%%%%%%%%%%%%%%%%%%%%%%%%%%%%%%%%%%%%%%%%%%%%%%%%
%%%%%%%%%%%%%%%%%%%%  Fig. 7   %%%%%%%%%%%%%%%%%%%%%%%%%%%%%%%%%
% Figure environment removed
%%%%%%%%%%%%%%%%%%%%%%%%%%%%%%%%%%%%%%%%%%%%%%%%%%%%%%%%%
By using the reactance databases of Fig.~\ref{five}, we designed a double-layered 3-D MTS antenna radiating a right-handed circular polarized (RHCP) wave at both frequencies. By 3-D, we mean a MTS where the metallic patches are modulated in two directions patterning the following modulated reactance sheets: $X_{s1,2}^{2D}\left( \rho,\phi \right) = X_{01,2} \left [1+M_{1,2} \text{cos}\left(\frac{2\pi }{d_{1,2}}\rho-\phi\right) \right]$ with $\rho$ and $\phi$ representing the position in polar coordinates. The $\phi$-dependence in $X_{s1,2}^{2D}$ ensures that any two sectors on the antennas separated by $90^{\circ}$ radiate an electric field with orthogonal and quadrature-phased components, resulting in a circularly polarized wave. Note that the design of 3-D MTS antennas is usually based on the 2-D canonical problem of Fig.~\ref{two}. Each sector can be seen as a 2-D canonical problem rotated by an angle $\phi$ around the $z$-axis. Hence, the approach proposed in this manuscript can be applied to designing both 2-D and 3-D dual-band double-layered MTS antennas.

The geometry of the designed antenna is illustrated in Fig.~\ref{one}; for ease of visualization, only the central portion of the antenna layout is displayed. The sizes of the two MTSs are equal, although the approach proposed in this manuscript does not prevent the use of MTSs with different sizes. The antenna radius is $3\lambda$ and $5\lambda$ at the operating frequency $f_1$ and $f_2$, respectively. We selected a relatively small radius at the lower frequency in order to be able to complete the full-wave simulation with reasonable accuracy with our computational resources at both frequencies. The antenna is fed from the center by an electrically small dipole immersed in the dielectric, which excites a cylindrical TM mode. The dipole is displaced by $h_2$ upward relative to the center of the ground place. The antenna was simulated with the commercial software Ansys HFSS. Figs.~\ref{seven}(a) and (b) show the obtained co-pol directivity patterns at $f_1$ and $f_2$, respectively, for the single- and double-layered MTSs. One can observe a satisfactory agreement between the patterns generated by the single- and double-layered MTSs at both frequencies.
%%%%%%%%%%%%%%%%%%%%%%%%%%%%%%%%%%%%%%%%%%%%%%%%%%%%%%%%%
\section{Conclusion} \label{sec4}
%%%%%%%%%%%%%%%%%%%%%%%%%%%%%%%%%%%%%%%%%%%%%%%%%%%%%%%%%
A new strategy for the design of dual-band double-layered MTS antennas is presented. The structure of the considered antenna solution consists of a cascade of two MTSs supported by a grounded dielectric slab. Each MTS consisting of a subwavelength metallic cladding controls the radiation at one frequency. The coupling between the two MTSs is drastically reduced by exploiting the Foster's reactance theorem that dictates the frequency behavior of the reactance sheets modeling the metallic layers. The equivalent reactance of the unit cell working at one frequency must be close to a pole at the other frequency and vice-versa. By doing so, the two MTSs are inactive at the other frequency and can be designed independently, avoiding complicated impedance extraction techniques and restrictions on the unit cell sides. A double-layered MTS radiating a circularly polarized broadside beam at $35.75$GHz and $94.05$GHz is designed and numerically tested. The directivity patterns of the single- and double-layered MTSs agree pretty well at both frequencies. Although the concept was presented and verified for a scalar impedance with a constant modulation index, it can be readily extended to the case of anisotropic impedances, and more sophisticated modulation profiles. The proposed dual-frequency antenna solution may find applications in Earth and climate science, remote sensing, and satellite communications.

%\section*{Acknowledgment:} 
%The authors Dimitrios L. Sounas and Mario Junior Mencagli thank Nader Engheta from the University of Pennsylvania for the fruitful discussion on this topic.




% if have a single appendix:
%\appendix[Proof of the Zonklar Equations]
% or
%\appendix  % for no appendix heading
% do not use \section anymore after \appendix, only \section*
% is possibly needed

% use appendices with more than one appendix
% then use \section to start each appendix
% you must declare a \section before using any
% \subsection or using \label (\appendices by itself
% starts a section numbered zero.)
%





% use section* for acknowledgment



% Can use something like this to put references on a page
% by themselves when using endfloat and the captionsoff option.
\ifCLASSOPTIONcaptionsoff
  \newpage
\fi



% trigger a \newpage just before the given reference
% number - used to balance the columns on the last page
% adjust value as needed - may need to be readjusted if
% the document is modified later
%\IEEEtriggeratref{8}
% The "triggered" command can be changed if desired:
%\IEEEtriggercmd{\enlargethispage{-5in}}

% references section

% can use a bibliography generated by BibTeX as a .bbl file
% BibTeX documentation can be easily obtained at:
% http://mirror.ctan.org/biblio/bibtex/contrib/doc/
% The IEEEtran BibTeX style support page is at:
% http://www.michaelshell.org/tex/ieeetran/bibtex/
%\bibliographystyle{IEEEtran}
% argument is your BibTeX string definitions and bibliography database(s)
%\bibliography{IEEEabrv,../bib/paper}
%
% <OR> manually copy in the resultant .bbl file
% set second argument of \begin to the number of references
% (used to reserve space for the reference number labels box)

\bibliographystyle{IEEEtran}
\bibliography{IEEEabrv,mybib}

% biography section
% 
% If you have an EPS/PDF photo (graphicx package needed) extra braces are
% needed around the contents of the optional argument to biography to prevent
% the LaTeX parser from getting confused when it sees the complicated
% \includegraphics command within an optional argument. (You could create
% your own custom macro containing the \includegraphics command to make things
% simpler here.)
%\begin{IEEEbiography}[{% Figure removed}]{Michael Shell}
% or if you just want to reserve a space for a photo:

% if you will not have a photo at all:


% insert where needed to balance the two columns on the last page with
% biographies
%\newpage


% You can push biographies down or up by placing
% a \vfill before or after them. The appropriate
% use of \vfill depends on what kind of text is
% on the last page and whether or not the columns
% are being equalized.

%\vfill

% Can be used to pull up biographies so that the bottom of the last one
% is flush with the other column.
%\enlargethispage{-5in}



% that's all folks
\end{document}


