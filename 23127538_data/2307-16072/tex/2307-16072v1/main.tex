% ****** Start of file aipsamp.tex ******
%
%   This file is part of the AIP files in the AIP distribution for REVTeX 4.
%   Version 4.1 of REVTeX, October 2009
%
%   Copyright (c) 2009 American Institute of Physics.
%
%   See the AIP README file for restrictions and more information.
%
% TeX'ing this file requires that you have AMS-LaTeX 2.0 installed
% as well as the rest of the prerequisites for REVTeX 4.1
% 
% It also requires running BibTeX. The commands are as follows:
%
%  1)  latex  aipsamp
%  2)  bibtex aipsamp
%  3)  latex  aipsamp
%  4)  latex  aipsamp
%
% Use this file as a source of example code for your aip document.
% Use the file aiptemplate.tex as a template for your document.
%\documentclass[aps,preprint,amssymb,superscriptaddress]{revtex4-2}
\documentclass[%
 reprint,
%superscriptaddress,
%groupedaddress,
%unsortedaddress,
%runinaddress,
%frontmatterverbose, 
%preprint,
%preprintnumbers,
%nofootinbib,
%nobibnotes,
%bibnotes,
 amsmath,amssymb,
 aps,
prl,
%prb,
%rmp,
%prstab,
%prstper,
%floatfix,
]{revtex4-2}
\bibliographystyle{apsrev4-2}

\usepackage{floatrow}
\newfloatcommand{capbtabbox}{table}[\FBwidth]
\usepackage{graphicx}% Include figure files
\usepackage{dcolumn}% Align table columns on decimal point
\usepackage{bm}% bold math
%\usepackage[mathlines]{lineno}% Enable numbering of text and display math
%\linenumbers\relax % Commence numbering lines
\usepackage{natbib}
\usepackage{CJK}
\usepackage[utf8]{inputenc}
\usepackage[T1]{fontenc}
\usepackage{mathptmx}
\usepackage{gensymb}
\usepackage{graphicx}% Include figure files
\usepackage{epstopdf}
\usepackage{dcolumn}% Align table columns on decimal point
\usepackage{bm}% bold math
\usepackage{amsmath}
\usepackage{mathtools}
\usepackage{relsize}
\usepackage{multirow}
\usepackage[colorlinks]{hyperref}
\begin{document}

%\preprint{AIP/123-QED}

%\begin{CJK*}{GB}{}
%\preprint{APS/123-QED}
\title{Charge-State Stability of Color Centers in Wide-Bandgap Semiconductors}% Force line breaks with \\
%\thanks{A footnote to the article title}%

\author{Rodrick Kuate Defo}
\email{rkuatedefo@princeton.edu}
\affiliation{Department of Electrical and Computer Engineering, Princeton University, Princeton, NJ 08540}
\author{Alejandro W. Rodriguez}
\affiliation{Department of Electrical and Computer Engineering, Princeton University, Princeton, NJ 08540}
\author{Steven L. Richardson} 
\affiliation{John A. Paulson School of Engineering and Applied Sciences, Harvard University, Cambridge, MA 02138, USA}
\affiliation{Department of Electrical and Computer Engineering, Howard University, Washington, DC 20059}
\date{\today}% It is always \today, today,
             %  but any date may be explicitly specified

\begin{abstract}
The N$V^-$ color center in diamond has been extensively investigated for quantum sensing, computation, and communication applications. Nonetheless, charge-state decay from the N$V^-$ to its neutral counterpart the N$V^0$ detrimentally affects the robustness of the N$V^-$ center and remains to be fully overcome. In this work, we provide an \textit{ab initio} formalism for accurately estimating the rate of charge-state decay of color centers in wide-bandgap semiconductors. Our formalism employs density functional theory calculations in the context of thermal equilibrium. We illustrate the method using the transition of N$V^-$ to N$V^0$ in the presence of substitutional N [see Z. Yuan \textit{et al}., PRR 2, 033263 (2020)].


%We show in this work that when X$V^-$ color centers exhibit a has been shown to mediate entanglement between electronic spins and surrounding nuclear spins. This entanglement leads to proximal nuclear spins enhancing electronic spin coherence and distant nuclear spins destroying it.   This result suggests that if samples are cooled to a low enough temperature that a Jahn-Teller distortion emerges, if such a distortion does not favor enhanced spin coherence, reheating and repeating the process may ultimately coax the system into a distortion that does. Here the enhancement in spin coherence would be the result of the modulation of the hyperfine interaction strength between the X$V^-$ center electronic spin and the neighboring $^{13}$C nuclei when the center is cycled between the three energetically equivalent distortions. 
\end{abstract}

\maketitle
%\preprint{APS/123-QED}

Color centers in wide-bandgap semiconductor hosts have garnered significant interest for potential applications in quantum computation~\cite{Childress2013diamond,Weber2010quantum,Pezzagna2021quantum}, communication~\cite{Childress2013diamond,Su2009high,Bradac2019quantum}, and sensing~\cite{Zhang2021toward}. The N$V^-$ color center in diamond in particular, consisting of a single substitutional N atom adjacent to a C vacancy with an additional negative charge, has enjoyed several research advances including the realization of a coherence time on the order of seconds~\cite{Bar-Gill2013solid} and entanglement of N$V^-$ pairs over a distance greater than a kilometer~\cite{Hensen2015loophole}. A drawback in the utilization of N$V^-$ centers for computation, communication, and sensing applications, however, is their tendency to revert to the neutral state after optical initialization into the singly negatively charged state~\cite{Yuan2020charge}. The implied large hole-capture cross section of optically activated N$V^-$ centers in diamond has been investigated both experimentally and theoretically~\cite{Yuan2020charge,Lozovoi2021optical,Lozovoi2023detection}, where the theory has included a bound-exciton model~\cite{Lozovoi2021optical}, and semi-classical Monte Carlo simulations~\cite{Lozovoi2023detection}. Arriving at a first-principles description of carrier capture that can apply in thermal equilibrium or under the action of an external field is still the subject of much work~\cite{Chen2023semi,Lozovoi2023detection,Alkauskas2014first,Shi2015comparative,Turiansky2021nonrad,Barmparis2015theory}. Herein we provide such a framework which is accurate and demonstrates that ionizing-dopant concentrations along with the electronic structure of the color center of interest and of the ionizing dopant are crucial for the determination of the expected timescale for hole capture by the ionized color centers. To provide an illustration of the method, we investigate charge transfer rates for N$V^-$ centers~\cite{Degen,Doherty2013the,Rondin_2014,Schirhagl,Kurtsiefer,Jelezko,Gruber2012,Balasubramanian2009ultra,Childress,Hao2023first} in the presence of N in diamond. We note that the approach also applies to Si$V^-$~\cite{neu2011single,Kuate2021calculating,haussler2017photoluminescence,ekimov2019effect}, Ge$V^-$~\cite{Ekimov2015germanium,Kuate2021calculating,Iwasaki2015germanium,Ralchenko2015observation,Ekimov2017anharmonicity,Palyanov2016high,Palyanov2015germanium,Bhaskar2017quantum,bray2018single}, Sn$V^-$~\cite{Iwasaki2017tin,Kuate2021calculating,Ekimov2018tin,tchernij2017single,palyanov2019high, Alkahtani2018tin, rugar2019char, wahl2020direct,fukuta2021sn,Gorlitz2020spectroscopic,Trusheim2020transform,Rugar2021quantum}, and Pb$V^-$\cite{Trusheim2019lead,Kuate2021calculating, tchernij2018single} centers in diamond or color centers in other wide-bandgap semiconductors~\cite{Castelletto2014a,Bockstedte2004ab,Kimoto2014fundamentals,Kuate2018energetics,Gadalla2021enhanced,Kuate2019parallel,Bracher2017selective,Soykal2016silicon,Soykal2017quantum,Weber2010quantum,Gali2011time,Awschalom2018quantum,Wolfowicz2021quantum,Whiteley2019spin}.

\textit{Methods.}---We used VASP~\cite{Kresse1993ab,Kresse1996efficient,Kuate2021theor,Kresse1999from} for our formation energy calculations with the screened HSE06 hybrid functional for exchange and correlation ~\cite{Heyd,Krukau}. We terminated our calculations when the forces in the atomic-position relaxations dropped below a threshold of $10^{-2}$ eV$\cdot$\AA$^{-1}$. The wavefunctions were expanded in a planewave basis with a cutoff energy of 430~eV and the size of the supercell was 512 atoms ($4\times4\times4$ multiple of the conventional unit cell). We employed $\Gamma$-point integration to evaluate energies. The elements used in our calculations and the associated ground-state structures and values of their chemical potentials are: N ($\beta$ hexagonal close-packed structure, $-11.39$ eV/atom) and C (diamond structure, $-11.28$ eV/atom). Only ground-state formation energies were computed in this work.
 
\textit{Approach to thermally driven charge transfer.}---As in earlier work~\cite{Kuate2023theor}, we consider the expected rate of transfer for an electron associated with a defect in a crystal, but no longer in the fully dilute limit. Suppose the electron has a definite momentum at each point in time $\hbar\mathbf{k}(t)$. In the non-relativistic limit, we can compute the rate associated with the transfer of the electron with effective $m^*$ across a distance $|\Delta\mathbf{r}|$ between ionized defects A and D as 
\begin{align}
\Gamma &=\int_{t_0}^\infty\text{d}t\frac{1}{t-t_0}\frac{\left|\hbar\mathbf{k}(t^\prime)\cdot\Delta\mathbf{r}\right|}{m^*|\Delta\mathbf{r}|^2}\int_{t_0}^t\text{d}t^\prime\frac{\hbar\mathbf{k}(t^\prime)\cdot\Delta\mathbf{r}}{m^*|\Delta\mathbf{r}|^2}\\&\times\delta\left(1-\frac{\int_{ t_0}^t\text{d}\tilde{t}\hbar\mathbf{k}(\tilde{t})\cdot\Delta\mathbf{r}}{m^*|\Delta\mathbf{r}|^2}\right).\nonumber
\end{align}
Above, we have used the definition of the Dirac delta function to convey that the rate is obtained by considering the average projection of the velocity onto the desired displacement of the electron divided by the distance $|\Delta\mathbf{r}|$ at the time the electron has completed the trajectory. 

The likelihood that the electron has enough initial kinetic energy to overcome the ionizing potential and recover the neutral system is given by a Fermi-Dirac distribution. We consider the donor level $E_{\text{F}_\text{D}}$ to be associated with an ionized donor-dopant D and the acceptor level $E_{\text{F}_\text{A}}$ to be associated with an ionized color center A. We will show below that the required initial kinetic energy $K_{\text{initial}}$ is equal to the difference between the adiabatic charge-transition levels $E_{\text{F}_\text{D}}$ and $E_{\text{F}_\text{A}}$ for defects D and A respectively, up to a correction of order $k_BT$. We can therefore write the expected rate of charge transfer as
 
\begin{align}
\label{eq:Gammakk}
\left<\Gamma\right> &\approx\int_{t_0}^\infty\text{d}t\frac{1}{t-t_0}\frac{\left|\hbar\mathbf{k}(t^\prime)\cdot\Delta\mathbf{r}\right|}{m^*|\Delta\mathbf{r}|^2}\int_{t_0}^t\text{d}t^\prime\frac{\hbar\mathbf{k}(t^\prime)\cdot\Delta\mathbf{r}}{m^*|\Delta\mathbf{r}|^2}\\&\times\frac{\delta\left(1-\frac{\int_{ t_0}^t\text{d}\tilde{t}\hbar\mathbf{k}(\tilde{t})\cdot\Delta\mathbf{r}}{m^*|\Delta\mathbf{r}|^2}\right)}{\exp\left((E_{\text{F}_\text{D}}-E_{\text{F}_\text{A}})/k_BT\right)+1},\nonumber
\end{align}
where $T$ is the temperature and $k_B$ is the Boltzmann constant. 

%and
%\begin{align}
%\label{eq:hGamma}
%    \bar{\Gamma}_h &= \sum_{\text{D}~\in~\text{Donors}}N_{\text{D}}\int_{V_\text{D}}\text{d}\mathbf{r}\rho_\text{D}(\mathbf{r})\sum_{\text{A}~\in~\text{Acceptors}}N_{\text{A}}\int_{V_\text{A}}\text{d}\mathbf{r}^\prime\rho_\text{A}(\mathbf{r}^\prime)\frac{1}{N_s}\sum_{s}\Omega_{PUC}\int\frac{\text{d}\mathbf{k}(t_0)}{(2\pi)^3}\int\frac{\text{d}\mathbf{k}^\prime}{(2\pi)^3}\\&\int_{t_0}^\infty\text{d}t\frac{1}{t-t_0}\frac{1}{\hbar}\frac{|\nabla_\mathbf{k}\epsilon^\text{V}_{\mathbf{k}(t),s}\cdot(\mathbf{r}-\mathbf{r}^\prime)|}{|\mathbf{r}-\mathbf{r}^\prime|^2}\int_{t_0}^t\text{d}t^\prime\frac{1}{\hbar}\frac{\nabla_\mathbf{k}\epsilon^\text{V}_{\mathbf{k}(t^\prime),s}\cdot(\mathbf{r}-\mathbf{r}^\prime)}{|\mathbf{r}-\mathbf{r}^\prime|^2}\nonumber\\&\times\frac{\delta\left(1-\frac{\int_{t_0}^t\text{d}\tilde{t}\frac{1}{\hbar}\nabla_\mathbf{k}\epsilon^\text{V}_{\mathbf{k}(\tilde{t}),s}\cdot(\mathbf{r}-\mathbf{r}^\prime)}{|\mathbf{r}-\mathbf{r}^\prime|^2}\right)}{\exp((\epsilon^\text{V}_{\mathbf{k}^\prime,s}-E_{\text{F}_\text{D}})/k_BT)+1}\nonumber\times\frac{\delta(\mathbf{k}^\prime-\mathbf{k}(t_0)-\frac{1}{\hbar}\int_{t_0}^t\mathbf{F}_h(\tilde{t}^\prime)\text{d}\tilde{t}^\prime)}{\exp((E_{\text{F}_\text{A}}-\epsilon^\text{V}_{\mathbf{k}(t_0),s})/k_BT)+1}\nonumber.
%\end{align}


\textit{Introducing electric forces.}---The calculation of the built-in electric field between defect pairs, which counteracts the electromagnetic potential that results in their ionization, requires the evaluation of electrostatic potentials. In order to compute the total energies of the systems from which the electrostatic potentials will be obtained, we solve for the eigenvalues of the Hamiltonian~\cite{Kaxiras2003atomic}
\begin{align}
\mathcal{H} &= -\sum_I\frac{\hbar^2}{2M_I}\nabla^2_{\mathbf{R}_I} - \sum_i\frac{\hbar^2}{2m_e}\nabla^2_{\mathbf{r}_i}\\&-\sum_{iI}\frac{Z_Ie^2}{|\mathbf{R}_I-\mathbf{r}_i|}+\frac{1}{2}\sum_{ij(j\neq i)}\frac{e^2}{|\mathbf{r}_i-\mathbf{r}_j|}+\frac{1}{2}\sum_{IJ(J\neq I)}\frac{Z_IZ_Je^2}{|\mathbf{R}_I-\mathbf{R}_J|}\nonumber
\end{align}
where $\mathbf{r}_i$ and $m_e$ denote the position and rest mass of electron $i$, respectively, and $\mathbf{R}_I$, $Z_I$, and $M_I$ are the position, valence charge, and rest mass of ion $I$, respectively. We apply the Born-Oppenheimer approximation to decouple the electronic and ionic degrees of freedom and solve for the electronic degrees of freedom given static ionic positions. If the eigenvalues for the electronic Hamiltonian have zero dispersion in reciprocal space, the expectation value of the velocity of the electrons will be zero implying that we can neglect the kinetic terms. Therefore, for defects introduced into a crystal in the dilute limit where the dispersion of the donor and acceptor levels is vanishingly small, our Hamiltonian effectively captures Coulomb or electrostatic potentials. 

The formation energies $\Delta H_f({\rm X^{\rm q}}, \, \{\mu_i^\text{X}\}, \, E_\text{F})$~\cite{zhang1991chemical, Freysoldt2014first,Kuate2018energetics,Kuate2019how,Kuate2021methods,Kuate2021theor,zunger2021under,Yang2015self,Ashcroft1976solid,Kuate2023theor} calculated in the dilute limit for defects capture electrostatic potentials as a result of the aforementioned arguments and would therefore allow for the determination of built-in fields. In $\Delta H_f({\rm X^{\rm q}}, \, \{\mu_i^\text{X}\}, \, E_\text{F})$, X is the defect species for which the formation energy is being calculated, q is the charge state of X, $\{\mu_i^\text{X}\}$ is the set of chemical potentials for the constituents of X, and $E_\text{F}$ is the Fermi level. In order to consistently determine the potentials at the locations of the defect species and to be in line with standard conventions, we use the neutral system in each case as the reference for the zero of the energy. The potential due to each charged defect consequently becomes the difference between the energies of the charged-defect containing system and the neutral system divided by the compensating background charge. The potential associated with a defect X with charge q is then simply
\begin{equation}
\label{eq:phiX}
    \phi(\mathbf{r}_\text{X}) = -\frac{1}{e\text{q}}\left(\Delta H_f({\rm X^{\rm q}}, \, \{\mu_i^\text{X}\}, \, 0)-\Delta H_f({\rm X^{0}}, \, \{\mu_i^\text{X}\}, \, 0)\right)
\end{equation}
and similarly for a defect Y with charge $-$q the potential follows from (\ref{eq:phiX}) upon the substitution $\text{X} \to \text{Y}$ and $\text{q} \to -\text{q}$,
%\begin{equation}
%\label{eq:phiY}
%    \phi(\mathbf{r}_\text{Y}) = \frac{1}{e\text{q}}\left(\Delta H_f({\rm Y^{-\rm q}}, \, \{\mu_i^\text{Y}\}, \, 0)-\Delta H_f({\rm Y^{0}}, \, \{\mu_i^\text{Y}\}, \, 0)\right)
%\end{equation}
with $\mathbf{r}_\text{X}$ and $\mathbf{r}_\text{Y}$ denoting the locations of the respective defects. Here, $E_\text{F}$ is set to zero since its inclusion in the expression for the formation energy subtracts out the energy associated with adding charge to the defect, which is no longer necessary if we are employing the electrostatic energy corresponding to a given charged defect. Above we have also treated the defects in the dilute limit so that the compensating background charge of the computational supercell can be treated as a point charge relative to the entire crystal. The built-in field at the location of the defect Y is then given by
\begin{align}
\label{eq:Efield}
    \vec{\mathcal{E}} = -\nabla(\phi(\mathbf{r})) \approx -\frac{(\phi(\mathbf{r}_\text{Y})-\phi(\mathbf{r}_\text{X}))}{|\Delta\mathbf{r}|}\frac{\Delta\mathbf{r}}{|\Delta\mathbf{r}|},
\end{align}
where $\Delta\mathbf{r} = \mathbf{r}_\text{Y}-\mathbf{r}_\text{X}$. 

We demonstrate the equivalence with our earlier work~\cite{Kuate2023theor} as follows. In our earlier work~\cite{Kuate2023theor}, we provided the expression
\begin{equation}
    \label{eq:Efieldequiv}
    \vec{\mathcal{E}} \approx \frac{1}{e}\frac{E_\text{V}(\mathbf{r}_\text{A})-E_\text{V}(\mathbf{r}_\text{D}))}{|\Delta\mathbf{r}|}\frac{\Delta\mathbf{r}}{|\Delta\mathbf{r}|},
\end{equation}
where $\mathbf{r}_\text{A}$ denoted the location of an acceptor A, $\mathbf{r}_\text{D}$ denoted the location of a donor D, $\Delta\mathbf{r} = \mathbf{r}_\text{A}-\mathbf{r}_\text{D}$, and $E_\text{V}$ indicated the energy of the valence band extremum. We had found that
\begin{align}
    E_\text{V}(\mathbf{r}_\text{A})&-E_\text{V}(\mathbf{r}_\text{D}) =\\&\bigg(\Delta H_f({\rm D^{0}}, \, \{\mu_i^\text{D}\}, \, 0)-\Delta H_f({\rm D^{+}}, \, \{\mu_i^\text{D}\}, \, 0)\bigg)\nonumber\\-&\bigg(\Delta H_f({\rm A^{-}}, \, \{\mu_i^\text{A}\}, \, 0)-\Delta H_f({\rm A^{0}}, \, \{\mu_i^\text{A}\}, \, 0)\bigg).\nonumber
\end{align}
Setting $\text{q} = 1$ implicitly in Eq. (\ref{eq:Efield}), we find agreement between Eq. (\ref{eq:Efield}) and Eq. (\ref{eq:Efieldequiv}) demonstrating the desired equivalence between the `local' Fermi level and built-in electric-field formulations.
%formulation and this formulation in terms of the built-in electric field between two point charges. 

\textit{Computing the momentum due to electric forces.}---At every point in time the charge must have enough energy to sustain motion along the path between the defects. In order to compute the necessary momentum for an electron moving between the defects, we recognize that the momentum must be sufficient to overcome the ionizing electromagnetic potential energy. We note, however, that as the charge approaches defect Y from defect X a built-in field will emerge to cancel the ionizing potential. In order to capture the emergence of the built-in field, we employ an energy conservation argument. Effectively, our argument is that the sum of the potential energy leading to the ionization of the defect pair and the kinetic energy of the electron must be conserved, so that $U + K = U(K=0)$. Thus, one finds
\begin{align}
%U + K &= U(K = 0) \\
U &= U(K=0) - K \\
&= e\left(\phi(\mathbf{r}_\text{X})-\phi(\mathbf{r}_\text{Y})\right) - \frac{\hbar k^2}{2m^*}.\label{eq:Udef}
\end{align}
Above, the electromagnetic potential energy, $U(K=0)$, reflects the relative formation energies associated with the placement of the electron that will travel from defect X to defect Y. A more general potential would allow for the inclusion of arbitrary external fields. The value of the potential energy landscape is defined in the manner given in Eq. (\ref{eq:Udef}) at the location of defect X so that its value at the location of defect Y can be set to zero.

In order to explicitly determine the required momentum as a function of time, we apply the second law of motion and employ a discrete approximation given the small distances with the origin at the location of defect Y to obtain
\begin{align}
\frac{d(\hbar \mathbf{k})}{dt} = -\nabla U \approx -\frac{U}{|\mathbf{r}|}\frac{\mathbf{r}}{|\mathbf{r}|}.
\end{align}
The non-relativistic limit has been applied above, which is justified by the fact that the maximum speed an electron can attain according to our calculations is less than $0.002c$, where $c$ is the speed of light. Once the electron arrives at defect Y from defect X, the kinetic energy would be dissipated in a process akin to the M\"ossbauer effect so that there is no need to produce a large initial change in momentum. Therefore, given an initial speed corresponding to a kinetic energy $E_{\text{thermal}} = k_BT$ in a random direction at an angle $\theta$ with respect to the radial direction
\begin{align}
\label{eq:drdt}
\frac{dr}{dt} \approx \pm \sqrt{2/m^*\frac{\left(k_BT\cos^2(\theta)-U(K=0)\ln(r/r_0)\right)}{1-\ln(r/r_0)}}.
\end{align}
After writing $\hbar \mathbf{k} = m^*\frac{\text{d}\mathbf{r}}{\text{d}t}$, the relation $\frac{d^2r}{dt^2} = \frac{1}{2}\frac{d\left(\frac{dr}{dt}\right)^2}{dr}$ has been used to obtain Eq. (\ref{eq:drdt}). We neglect $\frac{d\theta}{dt}$ and $\frac{d^2\theta}{dt^2}$ since in our work $k_BT \ll U(K=0) = e\left(\phi(\mathbf{r}_\text{X})-\phi(\mathbf{r}_\text{Y})\right) = E_{\text{F}_{\text{D}}} - E_{\text{F}_{\text{A}}}$, where we have used the definition of the donor and acceptor levels~\cite{Kuate2023theor}. Therefore, if the electron is located at defect Y with an initial kinetic energy $E_{\text{F}_{\text{D}}} - E_{\text{F}_{\text{A}}}$ up to a correction of order $k_BT$, then the electron will be able to return to defect X. For a nonzero gradient $U(K=0) > 0$ and zero initial velocity, subsequent $r$ will be less than $r_0 = |\Delta \mathbf{r}|$, requiring a velocity given by applying the negative root in Eq. (\ref{eq:drdt}). 

We can average the reciprocal of Eq. (\ref{eq:Gammakk}) over all possible initial velocity directions, given an initial thermal energy of $E_{\text{thermal}} = k_BT$, which yields 
\begin{align}
\label{eq:tildeGammakk}
    \left<\tau\right> &\approx \frac{1}{\pi}\int_0^\pi\text{d}\theta\Biggl(\frac{1}{\Delta t}\\&\times\frac{1}{\left(\exp\left(\left(E_{\text{F}_\text{D}}-E_{\text{F}_\text{A}}-E_\text{initial}\right)/k_BT\right)+1\right)}\Biggr)^{-1}.\nonumber
\end{align}
Above, $\Delta t = \int_{r_0}^0\text{d}r\left(\frac{dr}{dt}\right)^{-1}$ and $E_\text{initial} = k_BT\cos^2(\theta)\cdot\text{sgn}(\cos(\theta))$.
Care must be applied if $\theta > \pi/2$ in which case we must first integrate with positive velocity from $r = r_0$ to $r = r_0\exp(k_BT\cdot \cos^2(\theta)/U(K=0))$ and then back from $r = r_0\exp(k_BT\cdot \cos^2(\theta)/U(K=0))$ to $r = r_0$ with negative velocity before performing the integral between $r= r_0$ and $r= 0$ with negative velocity. 

In order to obtain the fraction of color centers that have undergone charge-state decay for a given timescale $\left<\tau\right>$, we compute the probability that a color center will have undergone charge-decay as
\begin{align}
\label{eq:prob}
P(\tau) &= \frac{8}{l_\text{X}^3}\bigg|B_{r_0}[\mathbf{0}]\\&\cap\left\{\mathbf{x}\in\mathbb{R}^3 : \max_{i=1,2}\left\{\left|x_i-\frac{l^{3/2}_\text{X}}{\sqrt{32}d^{1/2}_{\text{max}}}\right|\right\}\leq \frac{l^{3/2}_\text{X}}{\sqrt{32}d^{1/2}_{\text{max}}}\right\}\nonumber\\&\cap\left\{\mathbf{x}\in\mathbb{R}^3 : |x_3-d_{\text{max}}/2|\leq d_{\text{max}}/2\right\}\bigg|\nonumber
\end{align}
Above, $|\cdot|$ denotes the volume of the enclosed region, $B_{r_0}[\mathbf{0}]$ is the closed ball of radius $r_0$ centered at the origin,  $l_{\text{X}} = n_\text{X}^{-1/3}$ where $n_\text{X}$ is the concentration of the ionizing dopant X, and $d_\text{max}$ is the maximum implantation depth, since we consider the case where $d_\text{max} < l_{\text{X}}/2$ in this work. The probability reflects the fact that since $\left<\tau\right>$ is monotonic in $r_0$ the fraction of color centers having undergone charge state decay for a given $\left<\tau\right>$ corresponding to a given $r_0$ will be given by the fraction of color centers that are separated from their ionizing dopant by a distance of $r_0$ or less. We evaluate Eqs. (\ref{eq:tildeGammakk}) and (\ref{eq:prob}) for $r_0$ uniformly distributed between $r_0 = a$ and $r_0 = \sqrt{\frac{l_\text{X}^3}{4d_\text{max}}+d_\text{max}^2}$, where $a$ is the lattice constant of the conventional unit cell of diamond ($a = 3.549$~\AA~\cite{Kuate2023theor}). 

\textit{Experimentally investigating charge-state decay of N$V^-$.}---The details of the experiment of Yuan \textit{et al.}~\cite{Yuan2020charge} investigating charge-state instability of near-surface N$V^-$ centers in  diamond are as follows. A green pulse was used to initialize N$V$ centers in the negative charge state. The N$V^-$ centers were then left under darkness for a variable delay time following which they were read out using a charge-state-selective orange pulse. The samples used included a diamond sample labeled A that was implanted with an implantation dose of $5\times10^8$~cm$^{-2}$ at an energy of 3~keV and that was polished, pre-etched, and $^{12}$C enriched and a diamond sample labeled F that was implanted with an implantation dose of $1\times10^9$~cm$^{-2}$ at an energy of 1.5~keV and that was polished and pre-etched. They found that charge-state decay for sample F occurred on a timescale from 11-300~ms, while for sample A much less decay was observed out to 1~s. The accelerated charge conversion for sample F was attributed to the availability of electron traps near the surface, in particular boron impurities. In the following, we therefore concern ourselves with the theoretical determination of the charge-state decay rate in sample A. 

\textit{Effect of defect species and concentration on charge-conversion timescales.}---In order to determine timescales for charge transfer between ionized defect species, we apply Eqs. (\ref{eq:tildeGammakk}) and (\ref{eq:prob}) for an ionized color center with acceptor level $E_{\text{F}_\text{A}}$ transferring an electron to an ionized donor dopant with donor level $E_{\text{F}_\text{D}}$. The acceptor and donor levels serving as $E_{\text{F}_\text{A}}$ and $E_{\text{F}_\text{D}}$, measured relative to the valence band maximum, are given by $\epsilon^{\text{N}V}(0/-)\approx 2.8$~eV and $\epsilon^{\text{N}_\text{C}}(0/+)\approx 3.6$~eV, respectively. An effective mass of $m^*\approx 1.48m_e$ is obtained from fitting the bandstructure in Ref.~\cite{Kuate2023theor}. We account for the fact that the charge-state decay of  individual N$V^-$ centers measured in the Yuan~\textit{et al.} experiment was to a steady-state relative population greater than 0.5~\cite{Yuan2020charge}, where 0.5 is the value corresponding to local pinning of the Fermi level at $\epsilon^{\text{N}V}(0/-)$, by introducing a shift in $E_{\text{F}_\text{A}}$ such that the local Fermi level $E_{\text{F}_\text{A}}$ would produce a relative N$V^-$ population that would correspond to the experimentally measured value. Explicitly, for a final relative population of $p$, the shift is $\Delta E_{\text{F}_\text{A}} = k_BT\ln\left(\frac{(1-p_0)p}{p_0(1-p)}\right)$, where $p_0 = 0.5$. The maximum shift is $\Delta E_{\text{F}_\text{A}} \approx 0.07$~eV and the minimum shift is $\Delta E_{\text{F}_\text{A}} \approx 0.03$~eV. The corresponding values of $p$ are $0.93$ and $0.76$, respectively~\cite{Yuan2020charge}. We also account for the effect of the surface, which we assume to have an ether-like termination~\cite{Sque2006structure}, so that a donor level measured relative to the valence band maximum of 3.4~eV~\cite{Sque2006structure,Broadway2018spat} is induced. For N$V^-$ centers near the surface the number of ionized surface donors should be commensurate with the number of ionized bulk donors~\cite{Broadway2018spat}, so we can employ an effective $E_{\text{F}_\text{D}}$ given by averaging the donor-level values ($E_{\text{F}_\text{D}} \rightarrow (E_{\text{F}_\text{D}} + 3.4~\text{eV})/2)$ since the N$V^-$ centers would have equal probability of exchanging charge with an ionized surface donor or with N$_\text{C}^+$. Such averaging would also bring our earlier results in better agreement with experimental results at shallow implantation depths~\cite{Kuate2023theor,Broadway2018spat}. The result of these corrections produces good agreement with experiment (see Fig. \ref{fig:comparison}).

% Figure environment removed







\textit{Conclusions.}---We have shown that the precise electronic structure of an ionizing-dopant species and of an ionized color center are highly relevant to the charge-state decay characteristics of the ionized color center. The concentrations of dopants and color centers are also integral to the elucidation of charge-transfer rates within a semiconductor sample.  A key implication of our results is that, in order to mitigate charge-state decay for ionized color centers in semiconductors, the color center should be chosen such that its charge-transition level lies much lower in energy than the donor level of the ionizing dopant.  

R.K.D. gratefully acknowledges financial support from the Princeton Presidential Postdoctoral Research Fellowship and from the National Academies of Science, Engineering, and Medicine Ford Foundation Postdoctoral Fellowship program. We additionally acknowledge support by the STC Center for Integrated Quantum Materials, NSF Grant No. DMR-1231319. Finally, we wish to acknowledge insightful feedback from Nathalie P. de Leon and fruitful discussions with Pengning Chao.

% *****************************************************



% Consider at equilibrium a doped or extrinsic semiconductor at finite temperatures. There are electrons in the conduction band (CB) designated by an electron carrier concentration $n_{\text o}$, holes in the valence band (VB) designated by a hole carrier concentration $p_{\text o}$, a concentration of singly-ionized donors (an assumption at the moment) $n_{\text D^+}$, and a concentration of singly-ionized acceptors (an assumption at the moment) $n_{\text A^-}$. The condition of charge neutrality must ensure (R. F. Pierret, Modular Series on Solid State Devices: Advances Fundamentals) that 

% \begin{equation}
% \label{eq:fundamentals-1}
% p_{\text{o}} + n_{\text D^+} = 
% n_{\text{o}}+ n_{\text A^-} 
% \end{equation}

% If the temperature is high enough then all of the donor and acceptors are ionized so that $n_{\text D^+}$ = $n_{\text D}$ and $n_{\text A^-}$ = $n_{\text A}$ and we have

% \begin{equation}
% \label{eq:fundamentals-2}
% p_{\text{o}} + n_D = 
% n_{\text{o}}+ n_A 
% \end{equation}



% Now how do we evaluate the various carrier concentrations discussed in Eq. (\ref {eq:fundamentals-1}) and Eq. (\ref {eq:fundamentals-2})? Let us start with the electron concentration $n_{\text o}$  

% \begin{equation}
% \label{eq:model-2}
%     n_{\text o}(T) = \int_{\text E_g}^\infty f_{\text e(E)}\rho(E)dE,\\
% \end{equation}

% and hole concentration $p_{\text 0}$,  

% \begin{equation}
% \label{eq:model-3}
%     p_{\text o}(T) = \int_{-\infty}^0f_{\text h}(E)\rho(E)dE.
% \end{equation}

% which are integrals which include 
%  $\rho(E)$ is the density of states per unit volume, E$_{\text g}$ is the band gap of the host with the valence band maximum set to zero and all energies are shifted accordingly, and the Fermi-Dirac distributions for electrons $f_{\text e}$ and holes $f_{\text h}$ are given as

%  \begin{equation}
% \label{eq:model-4}
%  f_{\text e}(E) = \frac{1}{\text{exp}((E_{\text{F}}-E)/k_BT+1}
%  \end{equation}


%  and

% \begin{equation}
% \label{eq:model-5}
%  f_{\text h}(E) = 1-f_{\text e}(E).
%  \end{equation}

% The carrier concentrations in Eq. (\ref {eq:fundamentals-1}) and Eq. (\ref {eq:fundamentals-2}) are much complicated to study as they depend on the formation energies for the various charge states of a set of defect species~\cite{Buckeridge2019equilibrium}

% \begin{equation}
% \label{eq:concentration}
%     n_{\text{X}^\text{q}} = N_{\text{X}}g_{\text{X}^{\text{q}}}\exp(-\Delta H_f({\rm X^{\rm q}}, \, \{\mu_i^\text{X}\}, \, E_\text{F})/k_BT),
% \end{equation}

% where $T$ denotes temperature, $k_B$ is Boltzmann's constant, $\text{q}$ is the charge state, $N_\text{X}$ is the concentration of crystal sites on which the defect can form, $g_{\text{X}^{\text{q}}}$ is the degeneracy arising from the symmetry of a given charge state q of the defect, $\{\mu_i^\text{X}\}$ denotes the set of the chemical potential of the  $^{\rm th}$ species of a defect $\text{X}$ that was removed or added to produce the extrinsic semiconductor and $\Delta H_f({\rm X^{\rm q}}$ is the formation energy of the species X$^{\rm q}$ which is given by~\cite{zhang1991chemical, Freysoldt2014first,Kuate2018energetics,Kuate2019how,Kuate2021methods,Kuate2021theor,zunger2021under,Yang2015self,Ashcroft1976solid}

% \begin{equation}
% \label{eq:form_eq-1.0}
% \Delta H_f({\rm X^{\rm q}}, \, \{\mu_i^\text{X}\}, \, E_\text{F}) = E_{\text{def}}({\rm X^{\rm q}}) - E_0 - \sum_i\mu_i^\text{X}n_i + {\rm q}\, E_{\text{F}} + E_{\text{corr}}(\rm X^{\rm q}),
% \end{equation}

% where $E_{\text{def}}({\rm X^{\rm q}})$ is the energy of the charged supercell with the X$^{\rm q}$ species, $E_0$ is the energy of the stoichiometric neutral supercell, $\mu_i^\text{X}$ is the chemical potential of the $i^{\rm th}$ species that was removed or added to produce the supercell with the X$^{\rm q}$ species ($\{\mu_i^\text{X}\}$ denotes the set of all such species), $n_i$ is a positive (negative) integer representing the number of the $i^{\rm th}$ species that was added (removed),  and $E_{\text{F}}$ is Fermi level which is measured with respect to the valence band maximum (VBM) and is treated as a parameter. The importance of the 
% term $E_{\text{corr}}({\rm q})$, which is introduced as a correction to account for a finite supercell when performing a calculation for a charged defect, and the method for calculating it has been outlined by previous authors~\cite{Vinichenko,Freysoldt2011electrostatic,Freysoldt2009fully,Kumagai,Komsa2013finite,Walsh2021}.
% To briefly motivate the importance of the calculation of $E_{\text{corr}}({\rm q})$, the use of a periodic supercell to calculate the total energy of a charged defect naturally leads to divergence of the total energy due to infinitely many uncompensated charges. In order to remedy the issue, a neutralizing background charge is applied which causes spurious terms to arise in the total energy~\cite{Vinichenko,Castleton2006managing,Komsa2012comparison,Alkauskas,Freysoldt2009fully,Freysoldt2011electrostatic,Komsa2012finite,Kumagai,Castleton2009density}. The energy, $E_{\text{corr}}({\rm q})$, is designed precisely to correct these spurious terms. 


% So far in this standard approach of computing formation energies for defects, we have assumed that in Eq. (\ref{eq:concentration}) and Eq. (\ref{eq:form_eq-1.0}), both $E_{\text F}$ and $\mu_{\text i}$ can be treated as independent parameters. This is clearly not true as the following argument shows. For an initial selection of $E_{\text F}^{initial}$ at a particular value of the chemical potential $\mu_i$ and temperature T, we can compute both $n_{\text o}$ and $p_{\text o}$ from Eq. (\ref{eq:model-2},\,\ref{eq:model-3},\,\ref{eq:model-4},\ \ref{eq:model-5}). This choice of $E_{\text F}^{initial}$ also enables us to solve Eq. (\ref{eq:form_eq-1.0}) for an initial value of the formation energy $\Delta H_f({\rm X^{\rm q}}, \, \mu_i, \, E_\text{F}^{initial})$. With a knowledge of $\Delta H_f({\rm X^{\rm q}}, \, \mu_i, \, E_\text{F}^{initial})$, we can compute both $n_{X^q}$, or in particular $n_{D}$, $n_{D^+}$, $n_{A}$, and $n_{A^-}$ from Eq. (\ref{eq:concentration}). While this selection of $E_{\text F}^{initial}$ will not necessarily yield results for the electron, hole, and carrier concentrations which satisfy the conditions for charge neutrality in the extrinsic semiconductor we can consider a subsequent small change $E_{\text F}^{initial}\rightarrow E_{\text F}^{initial} + \,\delta E_{\text F}^{initial}$, which will enable us to repeat this process and solve Eqs. (\ref{eq:model-2},\,\ref{eq:model-3},\,\ref{eq:model-4},\ \ref{eq:model-5}) which we will be able to self-consistently solve until the condition for charge neutrality is satisfied for some small numerical tolerance. At this stage of the process we will have found a self-consistent or equilibrium $E_{\text{F}}^{eq}$ which will enable us to find from Eqs. (\ref{eq:model-2}, \ref{eq:model-3}, \ref{eq:form_eq-1.0}) the appropriate equilibrium values $n_{\text{X}}^{\text{q}}(\mu_i, T)$, $n_0(\mu_i, T)$, and $p_0 (\mu_i, T)$ of our system at a particular chemical potential and temperature. To be consistent with the literature and other texts we shall simply refer to the equilibrium Fermi level $E_{\text F}^{eq}$ as
% $E_{\text F}$ with the {\it caveat} that this equilibrium Fermi level $E_{\text F}$ 
% is not the same as the parametrized Fermi level $E_{\text F}$ used in Eqs. (\ref{eq:model-2}, \ref{eq:model-3}, \ref{eq:form_eq-1.0}).
 

% ****************************************************

% Let me start here with a general derivation of our model as follows for the case of an extrinsic semiconductor. Let us suppose we have an extrinsic semiconductor where the donor concentration $n_{\text D}$ is given by (Please refer here to Ashcroft and Mermin, Ibach and Luth, R. F. Pierret, C. Kittel, and Dalven.)

% \begin{equation}
% \label{eq:fundamentals-1}
% n_{\text D } = 
% n_{\text D_{\text o}}+ n_{\text D^+} 
% \end{equation}

% where $n_{\text D_{\text o}}$ is the total donor concentration of the neutral state and $n_{\text D^+} $ is the donor concentration for the singly-ionized state and where the total acceptor concentration $n_{\text A}$ is given by


% \begin{equation}
% \label{eq:fundamentals-2}
% n_{\text A} = 
% n_{\text A_o }+ n_{\text A^-} 
% \end{equation}

% where $n_{\text A_o }$ is the acceptor concentration of the neutral state and $n_{\text A^-}$ is the acceptor concentration for the singly-ionized state. In both of these cases we are assuming only single ionization of both species.

% The condition for charge neutrality in the extrinsic semiconductor is given by

% \begin{equation}
% \label{eq:fundamentals-3}
% p_{\text{o}} + n_{\text D^+} = 
% n_{\text{o}}+ n_{\text A^-} 
% \end{equation}

% which can be expresses using use Eq. (\ref{eq:fundamentals-1}) and (\ref{eq:fundamentals-2}) as  

% \begin{equation}
% \label{eq:fundamentals-4}
% p_{\text{o}} + n_D - n_{D^+} = 
% n_{\text{o}}+ n_A  - n_{A^-}
% \end{equation}

% RKD: I think Eq. (\ref{eq:fundamentals-4}) should be,
% \begin{equation}
% \label{eq:fundamentals-corr}
% p_{\text{o}} + n_D - n_{D_o} = 
% n_{\text{o}}+ n_A  - n_{A_o}
% \end{equation}

% All of this formalism is true independent of the model used in this draft. Now here come some important questions:


% 1. We could simplify Eq. (\ref{eq:fundamentals-4}) for two cases.

% CASE 1.1

% We could have $n_{\text o}$ = $p_{\text o}$. In this situation Eq. (\ref{eq:fundamentals-4}) becomes

% \begin{equation}
% \label{eq:fundamentals-5}
% n_D - n_{D^+} = 
% n_A  - n_{A^-}
% \end{equation}

% RKD: I think the Eq. (\ref{eq:fundamentals-5}) should read,
% \begin{equation}
% \label{eq:fundamentals-corr}
% n_D - n_{D_o} = 
% n_A  - n_{A_o}
% \end{equation}

% CASE 1.2

% We could have $n_{\text o}$ =  $p_{\text o}$ 
% $ \approx 0$. In this situation Eq. (\ref{eq:fundamentals-4}) also becomes

% \begin{equation}
% \label{eq:fundamentals-6}
% n_D - n_{D^+} = 
% n_A  - n_{A^-}
% \end{equation}

% RKD: I think Eq. (\ref{eq:fundamentals-6}) should read,
% \begin{equation}
% \label{eq:fundamentals-6}
% n_D - n_{D_o} = 
% n_A  - n_{A_o}
% \end{equation}

% The question now is which of these two cases is applicable to our defect model and why. I suspect it is Case 1.2. 

% RKD: Yes, I agree that it should be case 1.2 for the system we study since the defect levels are separated from the band edges.

% 2. Next in our model we seem to make the assumption that the condition of charge neutrality can be carried over to the defect model so that Eq. (\ref{eq:fundamentals-6}) can be recast as 

% \begin{equation}
% \label{eq:fundamentals-7}
% n_D = 
% n_A 
% \end{equation}

% and

% \begin{equation}
% \label{eq:fundamentals-8}
% n_{D^+} = 
% n_{A^-}
% \end{equation}

% You clearly use Eq. (\ref{eq:fundamentals-7}) for and Eq. (\ref{eq:fundamentals-8})
% for your arguments on pp. 14-16. Please check these pages as I have some minor disagreements with some of the equations and we need to be on the same page with these. Assuming my Eq. (\ref{eq:equilFermi0-1.2}) is correct, I will stop here and try to discuss it. I have gone no further in the draft as I want to make sure I understand this equation and how it fits in our model. 

% RKD: Yes, I assume that the chemical potentials of the constituent elements involved in the formation of A and D can be tuned such that the formation energies of the neutral charge states of A and D are equal. It follows that $n_{A_o} = n_{D_o}$ and that $n_A = n_D$ and $n_{A^-} = n_{D^+}$.

% Equation (\ref{eq:equilFermi0-1.2}) assumes singly-ionized donors and acceptors in the extrinsic semiconductor which is an approximation. The question is how to justify in our model ignoring the possibility of multiple-ionized donors and acceptors. I know you give a justification on Page 13 but it is a bit too mathematical, at least for me, and we should do a better job of explaining this assumption in words and pictures if necessary. This is a reasonable and acceptable assumption in our model. We just need to do a better job of justifying it physically for the reader.

% Next everything in the model so far assumes that Eq. (\ref{eq:fundamentals-7}) and Eq. (\ref{eq:fundamentals-8}) are true. We can we not also say that Eq. (\ref{eq:equilFermi0-1.2}) must be valid if 

% \begin{equation}
% \label{eq:fundamentals-9}
% n_D = 
% n_A = 1
% \end{equation}

% and

% \begin{equation}
% \label{eq:fundamentals-8}
% n_{D^+} = 
% n_{A^-} =1?
% \end{equation}

% Is this not simply a special case, namely that of a single defect pair in the extrinsic semiconductor?

% RKD: Yes, I assume that we can treat the special case of a single defect pair in the extrinsic semiconductor, which I will justify by showing that the defects in the defect pair cannot on average communicate to more than one other defect on the time scale of the experiment. The condition $n_{D^+} = n_{A^-} =1$ is not strictly necessary even if $n_D = n_A = 1$ since the neutral state can exist.

% Let me stop here for your feedback.

% **************************************

%  Upon imposing  charge neutrality in the crystal is should be true that 

% \begin{equation}
%     \label{eq:chargebalance}
%     n_0 - \sum_{\text{X}}\sum_{\text{q}}\text{q}n_{\text{X}^{\text{q}}} = p_0.
% \end{equation}

% which will not be initially true. self-consistent or equilibrium $E_{\text{F}}^{eq}$ to calculate the $n_{\text{X}}^{\text{q}}(\mu_i, T)$, $n_0(\mu_i, T)$ and $p_0 (\mu_i, T)$ for a particular temperature and chemical potential.

%  The condition of charge neutrality in the impure semiconductor must be satisfied so that
% \begin{equation}
% \label{eq:model-6}
% n_0 + \sum_{{acceptors}} n_{\text{X}^\text{q}}|\text{q}| = p_o + \sum_{{donors}} n_{\text{X}^\text{q}}|\text{q}| 
% \end{equation}


% \begin{equation}
% p + \sum\limits_{Donors} N_D^+ = 
% n + \sum\limits_{Acceptors} N_A^-
% \end{equation}

% where we have specified here that the defect concentration of species ($n_{\text{X}^\text{q}}$) in a charge state q could be assigned either to a donor state or to an acceptor state and we include the possibility of multiple donors or acceptors in the host semiconductor. 






% such that the total concentration of all charge states of the defect is,
% \begin{equation}
%     n_{\text{X}} = \sum_\text{q}n_{\text{X}^\text{q}}.
% \end{equation}


% We should also note here that for the special case of  $E_\text{F} = 0$, Eq. (\ref{eq:form_eq-1.0}) becomes

% \begin{equation}
% \label{eq:form_eq-1.1}
% \Delta H_f({\rm X^{\rm q}}, \, \{\mu_i^\text{X}\}, \, E_\text{F} = 0) = \Delta H_f({\rm X^{\rm q}}, \, \{\mu_i^\text{X}\}, \, 0) = E_{\text{def}}({\rm X^{\rm q}}) - E_0 - \sum_i\mu_i^\text{X}n_i + E_{\text{corr}}(\rm X^{\rm q}),
% \end{equation}

% so Eq. (\ref{eq:form_eq-1.0}) can be recast as

% \begin{equation}
% \label{eq:form_eq-1.2}
% \Delta H_f({\rm X^{\rm q}}, \, \{\mu_i^\text{X}\}, \, E_\text{F}) = \Delta H_f({\rm X^{\rm q}}, \, \{\mu_i^\text{X}\}, \, 0) + {\rm q}\, E_{\text{F}}.
% \end{equation}


% ***************************************************

% The formation energy $\Delta H_f({\rm X^{\rm q}}, \, \{\mu_i^\text{X}\}, \, E_F)$ depends on the parameterized Fermi level, $E_\text{F}$, and for every charge state there will be a single range of $E_\text{F}$ where that charge state has the minimum formation energy. This allows us to define an adiabatic charge-transition level (ACTL) $\epsilon^{\rm X}(\text{q}/\text{q}^{\prime})$ which is the value of the Fermi level $E_{\text{F}}^*$ for which X$^\text{q}$ and $\text{X}^{\text{q}^{\prime}}$ have equal formation energies or 
% \begin{equation}
% \label{eq:form_eq-1}
% \Delta H_f({\rm X^{\rm q}}, \, \{\mu_i^\text{X}\}, \, E_\text{F}^*)  = \Delta H_f({\rm X^{\rm q^{\prime} }}, \, \{\mu_i^\text{X}\}, \, E_\text{F}^*)
% \end{equation}

% and upon using Eq. (\ref{eq:form_eq-1.0}) we can find an exact expression for our ACTL~\cite{Freysoldt2014first},

% %\begin{equation}
% %\label{eq:CTL_eq-2} 
% %\epsilon^{\rm{X}} (\text{q}/\text{q}^{\prime}) &= %\frac{\Delta H_f({\rm X^{\rm q}}, \, \
% %{\mu_i^\text{X}\}, \, 0)-\Delta H_f({\rm X^{\rm %q^{\prime} }}, \, \{\mu_i^\text{X}\}, \, 0)}
% %{\text{q}^{\prime} - \text{q}},
% %\end{equation}

% \begin{equation}
% \label{eq:CTL_eq-3} 
% \epsilon^{\rm{X}} (\text{q}/\text{q}^{\prime}) \equiv E_F^* =
% \frac{(E_{\text{def}}({\rm X^{\rm q}}) +E_{\text{corr}}({\rm X^{\rm q}})) - (E_{\text{def}}({\rm X^{\rm q^{\prime}}}) +E_{\text{corr}}({\rm X^{\rm q^{\prime}}}))}{\text{q}^{\prime} - \text{q}}.
% \end{equation} 
% Here adiabatic simply reflects the fact that all formation energies that enter into the calculation of the charge-transition level are computed for defects that have been relaxed to their ground state.


% SLR: Depending on how we edit this section I might suggest adding a new Figure X-1 to be determined.

% Let us first consider the general problem of placing a donor D in a supercell as shown in Fig. \ref{fig:donorion} where for simplicity we assume that D can only undergo the chemical reaction
% \begin{equation} 
% \text{D} \rightarrow \text{D}^+ + e^-
% \end{equation}
% To find
% $\epsilon^{\rm{D}}(0/+)$ we can use Eq. (\ref{eq:CTL_eq-3}) where $\text{q}$ = 0 and $\text{q}^{\prime} = +1$. Note here that we assume that the electron removed from D enters the conduction band (CB) of the host semiconductor. 

% % Figure environment removed

% SLR: Depending on how we edit this section I might suggest adding a new Figure X-2 to be determined.

% Similarly, we can study the general problem of placing an acceptor A in a supercell as shown in Fig. \ref{fig:acceptorion} where for simplicity we assume that A can only undergo the chemical reaction
% \begin{equation} 
% \text{A} + e^- \rightarrow \text{A}^-
% \end{equation}
% To find  
% $\epsilon^{\rm{A}}(0/-)$ we can again use Eq. (\ref{eq:CTL_eq-3}) where $\text{q}$ = 0 and $\text{q}^{\prime} = -1$ . Note here that we assume that the electron which A receives comes from the valence band (VB) of the host semiconductor. 

% % Figure environment removed

% Now let us assume that both a donor D and an acceptor A are placed at positions $\mathbf{r}_\text{D}$ and $\mathbf{r}_\text{A}$, respectively, within a supercell and separated by a distance $|\Delta\mathbf{r}| = |\mathbf{r}_\text{A}-\mathbf{r}_\text{D}|$. As in our previous cases we will assume that D can only be ionized to D$^+$ and A can only be ionized to A$^-$,

% \begin{equation} 
% \text{D} \rightarrow \text{D}^+ + e^-
% \end{equation}
% \begin{equation} 
% \text{A} + e^- \rightarrow \text{A}^-
% \end{equation}

% or

% \begin{equation} 
% \text{D} + \text{A} \rightleftharpoons \text{D}^+ + \text{A}^-
% \end{equation}

% Since the net charge of this system in our supercell is now zero, it makes no sense to attempt to use Eq. (\ref{eq:CTL_eq-3}) to define an ACTL for such a case. The question we can pose is whether we can define an ACTL for $\epsilon^{\rm{D}}(0/-)$ in the presence of A and similarly a $\epsilon^{\rm{A}}(0/-)$ in the presence of D. There are two important issues we need to face before we can answer this important question. First, to ensure that we only have charge transfer between D and A we must insist that their energy levels be well separated from the conduction band maximum (CBM) and valence band maximum (VBM) of the host semiconductor, respectively. With this constraint we are only considering charge transfer between D and A directly without having to use the VB and CB at all as mechanism for treating the charge as in the previous two cases. 

% % Figure environment removed

% In order to describe our model, we will present cases that progressively approach our model. The first case is shown in Fig. \ref{fig:donorion1} where we have a random array of donor (D) and acceptor (A) impurities in the the host semiconductor at some temperature $T$. We assume that all impurities are singly ionized. At equilibrium the carrier concentrations, $n_0$ and $p_0$, and the ionized donor concentrations, $n_{\text{D}^+}$ and $n_{\text{A}^-}$, are then well defined. Furthermore, the value of the Fermi level at equilibrium is uniform in the sample and will allow the system to satisfy the condition of charge neutrality,
% \begin{equation}
% \label{eq:chgneutrality}
%     n_0+n_{\text{A}^-} = n_{\text{D}^+}+p_0.
% \end{equation}

% % Figure environment removed

% We can next impose a special restriction on the case described by Fig. \ref{fig:donorion1} to produce the second case shown in Fig. \ref{fig:donorion2}. In this second case, we assume that all of the ionized impurities exist in the form of defect pairs which are closely spaced. Now, while charge neutrality condition in Eq. (\ref{eq:chgneutrality}) still applies, we have the additional condition,
% \begin{equation}
%     n_{\text{A}^-} = n_{\text{D}^+},
% \end{equation}
% which is to say that the semiconductor is fully compensated. We assume that $T$ is sufficiently low that $n_0, p_0 \approx 0$ so charge transfer only exists between pairs of defects and not between the defects and the VB or CB. We note also that the equilibrium Fermi level for the entire semiconductor is still a well-defined quantity.  
% % Figure environment removed

% The case that we are actually investigating is shown in Fig. \ref{fig:donorion3}. For the case that we are investigating, we assume that defects are randomly distributed throughout the sample as in case 1 and that a given acceptor interacts on average with a single donor on the timescale over which the defects preserve their charge, though the donor can be anywhere in the sample. We also assume that $T$ is sufficiently low that $n_0, p_0 \approx 0$.

% % Figure environment removed

% We determine the number of donors that a given acceptor can interact with on the timescale over which the acceptor preserves its charge by considering the time required for $E_\text{F}$ to equilibrate in the sample. For $E_\text{F}$ to reach an equilibrium value over the entire sample the timescale over which the charge in the entire sample equilibrates must be much shorter than the timescale over which perturbations to the sample cause changes in the defect charge states. Suppose we start from a time immediately after some perturbation to the sample has occurred such that an equilibrium value of $E_\text{F}$ must be re-established. If we are at the location of a donor at precisely that time, $E_\text{F}$ will be pinned at the donor charge-transition level since the donor has not had time to interact with the rest of the sample. In general, a dopant can only contribute to charge equilibration in the sample if charge from the dopant has time to enter the conduction or valence band and travel to another defect. In the case of a donor, an electron from the donor must have time to reach the conduction band and travel to another defect. The expectation value for the speed at which an electron can travel in the crystal once in the conduction band is given by~\cite{Kaxiras2003atomic},
% \begin{equation}
%     \left<s_\mathbf{k}\right> = \frac{1}{\hbar}||\nabla_\mathbf{k}\epsilon^c_\mathbf{k}||.
% \end{equation}
% Above, $||\nabla_\mathbf{k}\epsilon^c_\mathbf{k}||$ denotes the norm of the gradient with respect to the wave-vector $\mathbf{k}$ of some conduction-band eigenvalue evaluated at $\mathbf{k}$ and $\hbar$ is the reduced Planck constant. If $E_\text{F}$ lies below $\epsilon^c_\mathbf{k}$ for all $\mathbf{k}$, the band will only be occupied  a fraction of the time given by the Fermi-Dirac distribution, which will be reflected in the expectation value for the speed. Explicitly, $\left<s_\mathbf{k}\right>$ becomes
% \begin{equation}
%     \left<s_\mathbf{k}\right> = \frac{1}{\hbar}||\nabla_\mathbf{k}\epsilon^c_\mathbf{k}||\cdot\frac{1}{\exp((\epsilon^c_\mathbf{k}-E_\text{F})/k_BT)+1}.
% \end{equation}

% The rate at which a donor can transfer charge from its location to another defect at another location is then given by,
% \begin{equation}
%     \Gamma_\mathbf{k} = \frac{1}{\hbar}\frac{||\nabla_\mathbf{k}\epsilon^c_\mathbf{k}||}{||\Delta\mathbf{r}||}\cdot\frac{1}{\exp((\epsilon^c_\mathbf{k}-E_\text{F})/k_BT)+1},
% \end{equation}
% where $\Delta\mathbf{r}$ is the displacement from the donor to the other defect. We take the other defect to be an acceptor in the sample. Assuming interactions between the acceptor and any donor in the sample occur randomly and independently, summing such expressions over all donors in the sample gives the desired effective rate. Given the exponential suppression of the rate if $\epsilon^c_\mathbf{k} - E_\text{F} \gg k_BT$, we only consider $\epsilon^c_\mathbf{k}$ within $k_BT$ of the CBM. The desired effective rate then takes the form,
% \begin{equation}
%     \bar{\Gamma}_\mathbf{k} \approx \frac{1}{\hbar}\sum_{\epsilon^c_\mathbf{k}\leq \text{CBM}+k_BT}\int\text{d}\mathbf{r}\frac{||\nabla_\mathbf{k}\epsilon^c_\mathbf{k}||}{r}\cdot\frac{1}{\exp((\epsilon^c_\mathbf{k}-E_\text{F})/k_BT)+1},
% \end{equation}
% where the integration is performed over the entire crystal. The minimum increment of $\mathbf{k}$ in the sum is constrained by the size of the crystal. The inverse of the effective rate gives the desired timescale for the equilibration of $E_\text{F}$.

% % We will show that this effective rate is much larger than the inverse timescale over which perturbations cause changes in the charge state of acceptor and would therefore not allow for more than two interacting defects over the course of the experiment.
% % The expression can be discretized as,
% % \begin{equation}
% %     \left<s_\mathbf{k}\right> = \frac{1}{\hbar}\frac{\left|\epsilon^c_{\mathbf{k}+\Delta\mathbf{k}}-\epsilon^c_{\mathbf{k}}\right|}{||\Delta\mathbf{k}||}.
% % \end{equation}
% % If $||\Delta\mathbf{r}||$ is the distance between defects, we 


% SLR: Depending on how we edit this section I might suggest adding a new Figure X-3 to be determined.

% **************************************************


% SLR: Given what has been previously written, I still do not understand a lot of the following.


% Why $(\epsilon^{\rm{D}}(0/\text{q}_\text{D}) - \epsilon^{\rm{A}}(0/-)) \gg k_BT$ and $(\epsilon^{\rm{D}}(0/+) - \epsilon^{\rm{A}}(0/\text{q}_\text{A})) \gg k_BT   ~\forall\text{q}_\text{A},\text{q}_\text{D},~\text{s.t.}~ \text{q}_\text{A} > 0,~ \text{q}_\text{D} < 0$.

% RKD: The condition ensures that the D$^-$ and D$^0$ charge states have the lowest formation energies for the D species and that the A$^+$ and A$^0$ charge states have the lowest formation energies for the A species at the value of $E_\text{F}$ that satisfies charge conservation and equality of $n_\text{D}$ and $n_\text{A}$ at some fixed constant value. There are potentially many more conditions that result in the low formation energy for the D$^-$ and D$^0$ charge states for the D species and for the A$^+$ and A$^0$ charge states for the A species at the value of $E_\text{F}$ that satisfies charge conservation and equality of $n_\text{D}$ and $n_\text{A}$ at some fixed constant value. I will change the condition to reflect the more general statement.

% In considering the system of the two defect species,
% we assume that 

% \begin{align}
%     \left(H_f({\rm D^{\text{q}_\text{D}}}, \, \{\mu_i^\text{D}\}, \, E_{\text{F}'}) - \max_{\text{X}^\text{q} = \text{D}^0,\text{D}^+,\text{A}^-,\text{A}^0}H_f({\rm X^{\rm q}}, \, \{\mu_i^\text{X}\}, \, E_{\text{F}'})\right){\bigg/}k_BT \gg 1 \quad\forall\text{q}_\text{D} \neq 0,+1\label{eq:formen_cond1}\\
%     \left(H_f({\rm A^{\text{q}_\text{A}}}, \, \{\mu_i^\text{A}\}, \, E_{\text{F}'}) - \max_{\text{X}^\text{q} = \text{D}^0,\text{D}^+,\text{A}^-,\text{A}^0}H_f({\rm X^{\rm q}}, \, \{\mu_i^\text{X}\}, \, E_{\text{F}'})\right){\bigg/}k_BT \gg 1 \quad\forall\text{q}_\text{A} \neq 0,-1\label{eq:formen_cond2}
% \end{align} 

% Where $E_{\text{F}^{\prime} }$ is the value of 
% $E_{\text{F}}$ that satisfies charge conservation and equality of $n_{\text{D}}$ and $n_{\text{A}}$ at some fixed constant value for $n_{\text{D}}$ and $n_{\text{A}}$. If those conditions are satisfied, we can treat any donor-acceptor system as having a single ACTL, $\epsilon^{\rm{D}}(0/+)$, for the donor and a single ACTL, $\epsilon^{\rm{A}}(0/-)$, for the acceptor.


% Now, consider the case where the total concentrations of donors and acceptors are equal, $n_\text{D} = n_\text{A} = C$ for some constant $C$. If the conditions are met, a charge-conserving solution can be obtained by choosing $E_{\rm F}$, $\{\mu^\text{D}_i\}$, and $\{\mu^\text{A}_i\}$ such that $n_{\text{D}^+} = n_{\text{A}^-}$ and $n_\text{D} = n_\text{A} = C$. The conditions in Eqs. (\ref{eq:formen_cond1}) and (\ref{eq:formen_cond2})
% will ensure that the concentrations of the charge states other than D$^-$, D$^0$, A$^+$, and A$^0$ are suppressed at the value of $E_\text{F}$ that satisfies charge conservation and $n_\text{D} = n_\text{A} = C$. The additional condition that the levels of both defects be separated from the band edges by an amount much greater than $k_BT$ implies $n_0 \approx p_0 \approx 0$, where $n_0$ and $p_0$ are the electron and hole carrier concentrations, respectively. 

% Formally, allowing the total concentration of donors to be equal to the total concentration of acceptors and imposing charge conservation we can write to a good approximation,
% \begin{align}
% \label{eq:eqconc}
%   n_{\text{D}^{+}}+n_{\text{D}^{0}} = n_{\text{A}^{-}}+n_{\text{A}^{0}} = C,\\
% \label{eq:chgbalance}
%   n_{\text{D}^{+}} = n_{\text{A}^{-}}.
% \end{align}


% Starting from Eq. (\ref{eq:chgbalance})

% \begin{equation}
% \label{eq:deriv0}
%     n_{\text{D}^{+}} = n_{\text{A}^{-}},
% \end{equation}
% we can use Eq. (\ref{eq:concentration}) to obtain
% \begin{equation}
% \label{eq:deriv1}
%     N_{\text{D}}g_{\text{D}^+}\exp(-\Delta H_f(\text{D}^+,\{\mu_i^\text{D}\},E_\text{F})/k_BT) = N_{\text{A}}g_{\text{A}^-}\exp(-\Delta H_f(\text{A}^-,\{\mu_i^\text{A}\},E_\text{F})/k_BT).
% \end{equation}
% **************************************************

% SLR: Please check the equation below as I believe there was a previous sign error in the exponentials at the end of end side.

% ****************************************************

% Using Eq. (\ref{eq:form_eq-1.2}), this expression becomes,
% \begin{align}
%     &N_{\text{D}}g_{\text{D}^+}\exp(-\Delta H_f(\text{D}^+,\{\mu_i^\text{D}\},0)/k_BT)\times\exp(E_\text{F}/k_BT)\label{eq:deriv2}\\ &= N_{\text{A}}g_{\text{A}^-}\exp(-\Delta H_f(\text{A}^-,\{\mu_i^\text{A}\},0)/k_BT)\times\exp(-E_\text{F}/k_BT).\nonumber
% \end{align}
% and upon solving for $E_{\text{F}}$  we obtain Eq. (\ref{eq:chgconsFermi})

% RKD: The version,
% \begin{align}
%     &N_{\text{D}}g_{\text{D}^+}\exp(-\Delta H_f(\text{D}^+,\{\mu_i^\text{D}\},0)/k_BT)\times\exp(-E_\text{F}/k_BT)\\ &= N_{\text{A}}g_{\text{A}^-}\exp(-\Delta H_f(\text{A}^-,\{\mu_i^\text{A}\},0)/k_BT)\times\exp(E_\text{F}/k_BT).\nonumber
% \end{align}
% is correct. Breaking down the steps,
% \begin{equation}
%     N_{\text{D}}g_{\text{D}^+}\exp(-\Delta H_f(\text{D}^+,\{\mu_i^\text{D}\},E_\text{F})/k_BT) = N_{\text{A}}g_{\text{A}^-}\exp(-\Delta H_f(\text{A}^-,\{\mu_i^\text{A}\},E_\text{F})/k_BT),
% \end{equation}
% implies
% \begin{equation}
%     N_{\text{D}}g_{\text{D}^+}\exp(-(\Delta H_f(\text{D}^+,\{\mu_i^\text{D}\},0)+E_\text{F})/k_BT) = N_{\text{A}}g_{\text{A}^-}\exp(-(\Delta H_f(\text{A}^-,\{\mu_i^\text{A}\},0)-E_\text{F})/k_BT).
% \end{equation}
% Absorbing the negative sign multiplying the formation energy in the exponential into the $E_\text{F}$ term we have,
% \begin{align}
%     &N_{\text{D}}g_{\text{D}^+}\exp(-\Delta H_f(\text{D}^+,\{\mu_i^\text{D}\},0)/k_BT)\times\exp(-E_\text{F}/k_BT)\\ &= N_{\text{A}}g_{\text{A}^-}\exp(-\Delta H_f(\text{A}^-,\{\mu_i^\text{A}\},0)/k_BT)\times\exp(E_\text{F}/k_BT).\nonumber
% \end{align}
% ****************************************************

% SLR: Please check the following equation again as I believe you had a sign error on the left-hand side and this is now the correct version.


% \begin{equation}
% \label{eq:chgconsFermi}
%     E_\text{F} = (\Delta H_f({\rm D^{+}}, \, \{\mu_i^\text{D}\}, \, 0) - \Delta H_f({\rm A^{-}}, \, \{\mu_i^\text{A}\}, \, 0) )/2 + \frac{k_BT}{2}\ln\left(\frac{N_\text{A}g_{\text{A}^+}}{N_\text{D}g_{\text{D}^-}}\right).
% \end{equation}

% RKD: The original version is correct given the correct sign for $E_\text{F}$ above. That is,

% \begin{equation}
%     E_\text{F} = (\Delta H_f({\rm A^{-}}, \, \{\mu_i^\text{A}\}, \, 0) - \Delta H_f({\rm D^{+}}, \, \{\mu_i^\text{D}\}, \, 0))/2 + \frac{k_BT}{2}\ln\left(\frac{N_\text{D}g_{\text{D}^+}}{N_\text{A}g_{\text{A}^-}}\right).
% \end{equation}
% ***************************************************


% RKD: The $E_\text{F}$ we solve for is the equilibrium value of the Fermi level for the system under consideration. Put otherwise, if we have a single donor with a single charge-transition level that is between the 0 and $+1$ charge states and a single acceptor with a single charge-transition level that is between the zero and $-1$ charge states, the equilibrium value of  $E_\text{F}$ is given by,

% **************************************************

% Starting from Eq. (\ref{eq:chgbalance})

% \begin{equation}
% \label{eq:deriv2}
%   n_{\text{D}^{0}} = n_{\text{A}^{0}},
% \end{equation}

% we can again use Eq. (\ref{eq:concentration}) to obtain

% \begin{equation}
% \label{eq:deriv3}
%     N_{\text{D}} g_{\text{D}^{0}} \exp(-\Delta H_f(\text{D}^{0},\{\mu_i^\text{D}\},E_\text{F})/k_BT) = N_{\text{A}}g_{\text{A}^{0}}\exp(-\Delta H_f(\text{A}^0,\{\mu_i^\text{A}\},E_\text{F})/k_BT).
% \end{equation}

% Using Eq. (\ref{eq:form_eq-1.2}), this expression becomes,

% \begin{equation}
% \label{eq:deriv4}
%     N_{\text{D}}g_{\text{D}^+}\exp(-\Delta H_f(\text{D}^+,\{\mu_i^\text{D}\},0)/k_BT) = N_{\text{A}}g_{\text{A}^-}\exp(-\Delta H_f(\text{A}^-,\{\mu_i^\text{A}\},0)/k_BT)
% \end{equation} 

% RKD: I think Eq. (\ref{eq:deriv4}) should be, 
% \begin{equation}
%     N_{\text{D}}g_{\text{D}^0}\exp(-\Delta H_f(\text{D}^0,\{\mu_i^\text{D}\},0)/k_BT) = N_{\text{A}}g_{\text{A}^0}\exp(-\Delta H_f(\text{A}^0,\{\mu_i^\text{A}\},0)/k_BT)
% \end{equation} 
% which can be rearranged to give

% \begin{equation}
% \label{eq:deriv5}
% \Delta H_f(\rm A^{0}, \{\mu_i^\text{A}\}, \, 0) -\Delta H_f(\rm D^{0}, \{\mu_i^\text{D}\}, \, 0) +\frac{k_BT}{2}\ln\left(\frac{N_{\text{D}^+}g_{\text{D}^0}}{N_{\text{A}}g_{\text{A}^0}}\right) = 0.
% \end{equation}

% RKD: I think Eq. (\ref{eq:deriv5}) should be (small typos),
% \begin{equation}
% \Delta H_f(\rm A^{0}, \{\mu_i^\text{A}\}, \, 0) -\Delta H_f(\rm D^{0}, \{\mu_i^\text{D}\}, \, 0) +k_BT\ln\left(\frac{N_{\text{D}}g_{\text{D}^0}}{N_{\text{A}}g_{\text{A}^0}}\right) = 0.
% \end{equation}
 
% If we divide Eq. (\ref{eq:deriv5}) by a factor of 2, combine this result with Eq. (\ref{eq:chgconsFermi}), and solve for $E_{\text{F}}$  we obtain Eq. (\ref{eq:key-equation})

% ****************************************************

% SLR: Please check the following equation as I am getting something different from what you previously had:

% \begin{align}
% \label{eq:equilFermi0-1.0}
%     E_\text{F} &= (\Delta H_f({\rm A^{-}}, \, \{\mu_i^\text{A}\}, \, 0) -\Delta H_f({\rm A^{0}}, \, \{\mu_i^\text{A}\}, \, 0))/2\\ &+(\Delta H_f({\rm D^{0}}, \, \{\mu_i^\text{A}\}, \, 0) -\Delta H_f({\rm D^{+}}, \, \{\mu_i^\text{D}\}, \, 0))/2\nonumber\\ &+\frac{k_BT}{2}\ln\left(\frac{g_{\text{D}^+}g_{\text{A}^0}}{g_{\text{D}^0}g_{\text{A}^-}}\right)\nonumber
%     \end{align}

% ***************************************************


% \begin{align}
% \label{eq:key-equation}
%     E_\text{F} &= (\Delta H_f({\rm D^{+}}, \, \{\mu_i^\text{D}\}, \, 0) -\Delta H_f({\rm D^{0}}, \, \{\mu_i^\text{D}\}, \, 0))/2\\ &+(\Delta H_f({\rm A^{0}}, \, \{\mu_i^\text{A}\}, \, 0) -\Delta H_f({\rm A^{-}}, \, \{\mu_i^\text{A}\}, \, 0))/2\nonumber\\ &+\frac{k_BT}{2}\ln\left(\frac{g_{\text{A}^-}g_{\text{D}^0}}{g_{\text{A}^0}g_{\text{D}^+}}\right)\nonumber\\
% \end{align}

% RKD: As outlined above (ie. using the correct signs), the original equation is correct.

% Also I am not getting your final result for the ACTLs


% \begin{equation}
% \label{eq:equilFermi0-1.1}
% E_\text{F} = \frac{\epsilon^{\rm{D}} (0/+)+\epsilon^{\rm{A}} (0/-)}{2} + \frac{k_BT}{2}\ln\left(\frac{g_{\text{D}^{+}}g_{\text{A}^{0}}}{g_{\text{D}^{0}}g_{\text{A}^{-}}}\right)
% \end{equation}

% but I am getting


% \begin{equation}
% \label{eq:equilFermi0-1.2}
% E_\text{F} = \frac{\epsilon^{\rm{D}} (+/0)+\epsilon^{\rm{A}} (0/-)}{2} + \frac{k_BT}{2}\ln\left(\frac{g_{\text{A}^{-}}g_{\text{D}^{0}}}{g_{\text{A}^{0}}g_{\text{D}^{+}}}\right)
% \end{equation}

% RKD: As outlined above (ie. using the correct signs), the original equation that I wrote is correct.

% so please check these two equations. Also it would be nice to show in limiting cases that Eq. (\ref{eq:equilFermi0-1.2}) makes physical sense to the reader since it is a new equation which we are introducing to the literature, I think.

% RKD: Yes, it would indeed be good to show some limiting case for the validity of the equation.

% You can stop here for the moment.

% ****************************************************

% We can then adjust the $\{\mu_i^\text{A}\}$ and the $\{\mu_i^\text{D}\}$ in order to satisfy Eq. (\ref{eq:eqconc}), which leads to the expression where we have explicitly used Eq. (\ref{eq:concentration}) and Eq. (\ref{eq:form_eq-1.2}) 


% ****************************************************

% Thus, for the values of $\{\mu_i^\text{A}\}$ and $\{\mu_i^\text{D}\}$ that ensure that $n_\text{D} = n_\text{A} = C$ we have,

% *****************************************************





% \begin{equation}
% \label{eq:equilFermi0-1.1}
% E_\text{F} = \frac{\epsilon^{\rm{D}} (0/+)+\epsilon^{\rm{A}} (0/-)}{2} + \frac{k_BT}{2}\ln\left(\frac{g_{\text{D}^{+}}g_{\text{A}^{0}}}{g_{\text{D}^{0}}g_{\text{A}^{-}}}\right)
% \end{equation}

%  Thus, the result after performing the operation is still equal to the equilibrium Fermi level that we solve for in Eq. (\ref{eq:chgconsFermi}). The equivalent expression for Eq. (\ref{eq:equilFermi0}) in Eq. (\ref{eq:equilFermi}) follows from the definition of the charge-transition level.

% *****************************************************


% where we have used Eq. (\ref{eq:CTL_eq-3}) to arrive at Eq. (\ref{eq:equilFermi}). The expression in Eq. (\ref{eq:equilFermi}) for $E_\text{F}$ is independent of the values of $\{\mu_i^\text{A}\}$ and $\{\mu_i^\text{D}\}$. Consequently, the values of $\{\mu_i^\text{A}\}$ and $\{\mu_i^\text{D}\}$ that should be used in Eq. (\ref{eq:chgconsFermi}) are precisely those values that result in equality with Eq. (\ref{eq:equilFermi}) if the system must obey Eqs. (\ref{eq:eqconc}) and (\ref{eq:chgbalance}).


% We therefore define an ACTL for one defect in the presence of another given by,
% \begin{equation}
% \label{eq:chgtransitionlevel}
%     \epsilon^{\rm{D},\rm{A}} (0/+,0/-) \equiv \frac{\epsilon^{\rm{D}} (0/+)+\epsilon^{\rm{A}} (0/-)}{2} +\frac{k_BT}{2}\ln\left(\frac{g_{\text{D}^{+}}g_{\text{A}^{0}}}{g_{\text{D}^{0}}g_{\text{A}^{-}}}\right).\nonumber
% \end{equation}
% The equality $n_\text{D} = n_\text{A} = C$ is not particularly restrictive for a wide-bandgap semiconductor with defect levels separated from the band edges since in that case proximity will be required for the transfer of charge~\cite{Collins2002}. If the concentration of defects in the crystal is such that defects are spaced on average more than a few lattice constants from one another so that charge can only be exchanged with the nearest defect, then we can take the limit where $C$ is such that exactly one donor and exactly one acceptor exist in the crystal since the remainder of the crystal will not interact appreciably with the system of two defects. Given that thermodynamic principles apply to macroscopic systems, a caveat exists. Namely, the result only rigorously applies when averaged over many such defect pairs. Taking into consideration the caveat, Eq. (\ref{eq:chgtransitionlevel}) captures the energetics of charge transfer for defects in a wide-bandgap semiconductor.

% From the energetics of the charge transfer, we determine the total amount of band bending. From Eq. (\ref{eq:chgtransitionlevel}), we drop the term of order $k_BT$ since it is negligible at room temperature~\cite{Webber2012ab}. Using a generalization of the complex binding energy~\cite{Freysoldt2014first}, we had previously shown that the error associated with assuming the dilute limit when taking the average of the ACTLs is negligible compared to the averaged transition energy~\cite{Kuate2021theor}. Therefore, to a good approximation the energy required to ionize the system of two defects can be taken to be equal to the average of the dilute-limit ACTLs for the two defects. Such a result is consistent with the electronic structure of the semiconductor being modulated by the presence of the two defects in such a manner as to result in a band-bending profile along the line connecting them. We demonstrate this consistency by first noting that $E_\text{F}$ must be pinned at the acceptor and donor levels in the respective parts of the sample if A gains a single electron and D loses a single electron when they are sufficiently far apart that the dilute limit can be applied. In equilibrium, however, $E_\text{F}$ is constant throughout the sample. Thus, the conduction band minimum (CBM) and the valence band maximum (VBM) must be shifted at the positions $\mathbf{r}_\text{A}$ and $\mathbf{r}_\text{D}$ of the respective defects,
% \begin{equation}
% \label{eq:CBMVBMshiftA}
%     \Delta{\text{CBM}}(\mathbf{r}_\text{A}) =  \frac{1}{2}\left(\epsilon^{\rm{A}}(0/-)-\epsilon^{\rm{D}}(0/+)\right) = \Delta{\text{VBM}}(\mathbf{r}_\text{A})
% \end{equation}
% and
% \begin{equation}
% \label{eq:CBMVBMshiftD}
%     \Delta{\text{CBM}}(\mathbf{r}_\text{D}) =  \frac{1}{2}\left(\epsilon^{\rm{D}}(0/+)-\epsilon^{\rm{A}}(0/-)\right) = \Delta{\text{VBM}}(\mathbf{r}_\text{D}).
% \end{equation}
%  In traveling from the location of the donor defect to the location of the acceptor defect, we obtain the result for the total bending of the conduction and valence band extrema,
% \begin{align}
%     {\text{CBM}}(\mathbf{r}_\text{A})-{\text{CBM}}(\mathbf{r}_\text{D}) &= \Delta{\text{CBM}}(\mathbf{r}_\text{A})-\Delta{\text{CBM}}(\mathbf{r}_\text{D}) = \left(\epsilon^{\rm{A}}(0/-)-\epsilon^{\rm{D}}(0/+)\right),\\
%     {\text{VBM}}(\mathbf{r}_\text{A})-{\text{VBM}}(\mathbf{r}_\text{D}) &= \Delta{\text{VBM}}(\mathbf{r}_\text{A})-\Delta{\text{VBM}}(\mathbf{r}_\text{D}) = \left(\epsilon^{\rm{A}}(0/-)-\epsilon^{\rm{D}}(0/+)\right).
% \end{align} 

% SLR: Note that I have inserted the missing deltas in the equation above that were accidentally dropped from Eqs. 21 and 22.

% RKD: The expressions are actually equivalent if we use the final CBM and VBM positions when we drop the deltas. I will need the expressions with the dropped deltas in defining the electric field, so I have included the differences both with and without the deltas.

% SLR: When I made the suggestion of working backwards I meant that it would be best to think backwards. That is what steps are needed for the reader to understand the draft? Once we have these steps spelled out then we need to actually work forward from the beginning in the draft and eventually land at the conclusion. Thus the following material really needs to go towards the end of the draft and not at the beginning!
 
%  The electric field associated with the bending of the conduction and valence bands due to the presence of the defects is given by~\cite{Zhang2012band,Broadway2018spat,Dalven1990introduction},
% \begin{align}
%     \vec{\mathcal{E}} = -\frac{1}{e}\nabla({\text{VBM}}(\mathbf{r})) \approx -\frac{1}{e}\frac{({\text{VBM}}(\mathbf{r}_\text{A})-{\text{VBM}}(\mathbf{r}_\text{D}))}{|\Delta\mathbf{r}|}\frac{\Delta\mathbf{r}}{|\Delta\mathbf{r}|},
% \end{align}
% where VBM is the valence band maximum, $e$ is the elementary charge, $\mathbf{r}_\text{A}$ is the position of defect A, $\mathbf{r}_\text{D}$ is the position of defect D, and $\Delta\mathbf{r} = \mathbf{r}_\text{A} - \mathbf{r}_\text{D}$.  In traveling from the location of the donor defect to the location of the acceptor defect, we have shown that the total bending of the conduction and valence band extrema $\left({\text{CBM}}(\mathbf{r}_\text{A})-{\text{CBM}}(\mathbf{r}_\text{D})~\text{and}~{\text{VBM}}(\mathbf{r}_\text{A})-{\text{VBM}}(\mathbf{r}_\text{D})\right)$ is,
% \begin{equation}
%     {\text{CBM}}(\mathbf{r}_\text{A})-{\text{CBM}}(\mathbf{r}_\text{D}) = {\text{VBM}}(\mathbf{r}_\text{A})-{\text{VBM}}(\mathbf{r}_\text{D}) = \left(\epsilon^{\rm{A}}(0/-)-\epsilon^{\rm{D}}(0/+)\right). 
% \end{equation} 
% Above, $\epsilon^{\rm{A}}(0/-)$ denotes the adiabatic charge-transition level (ACTL) where A in the neutral state gains a single electron and $\epsilon^{\rm{D}}(0/+)$ denotes the ACTL where D in the neutral state loses a single electron. 

% We now consider the case of N$V$ in the presence of N$_\text{C}$ in diamond. The converged lattice constant of diamond for the conventional unit cell was $a=3.539$ \AA. This value is in good agreement with a previous theoretical calculation of $a=3.545$~\AA~\cite{Deak2014formation}. We found an electronic band gap of 5.4~eV, which is in good agreement with previous experimental~\cite{Madelung1991semiconductors} and theoretical~\cite{Szasz2013hyperfine,Deak2014formation} results. Based on our calculations, we find that the $\epsilon^{\text{N}V}(-/-2)$ ACTL for the N$V$ lies at 14.73~eV in absolute while the CBM lies at 14.75~eV in absolute units, meaning that the $\epsilon^{\text{N}V}(-/-2)$ ACTL for the N$V$ is well within $k_BT$ of the CBM at room temperature. Other theory has placed the $\epsilon^{\text{N}V}(-/-2)$ ACTL for the N$V$ at a lower energy relative to the CBM~\cite{Deak2014formation}, likely due to a difference in the form of $E_\text{corr}(\text{q})$~\cite{Vinichenko,lany2008assess,lany2010many}. Experimental evidence~\cite{grotz2012charge} is consistent with our finding that the $\epsilon^{\text{N}V}(-/-2)$ ACTL for the N$V$ lies either very close to or within the conduction band. We can therefore take $2C_{{\text{N}V}^{2-}} = p_0$, which may equal zero if the $\epsilon^{\text{N}V}(-/-2)$ ACTL for the N$V$ lies within the conduction band. Eliminating $2C_{{\text{N}V}^{2-}}$ and $p_0$ through the imposition of charge conservation and setting the total concentration of the remaining charge states equal to one defect per crystal region for both the N$V$ and the N$_\text{C}$ defects, we are left with Eqs. (\ref{eq:eqconc}) and (\ref{eq:chgbalance}). We note that a diamond crystal containing N$V$ and N$_\text{C}$ defects satisfies the other conditions required for the applicability of Eqs. (\ref{eq:eqconc}) and (\ref{eq:chgbalance})~\cite{Deak2014formation,Kuate2021theor}. Thus, measured from the reference frame of a N$V$ in the presence of substitional N in diamond, the total amount of band bending is,
% \begin{equation}
%      {\text{CBM}}(\mathbf{r}_{\text{N}V})-{\text{CBM}}(\mathbf{r}_{\text{N}_\text{C}}) = {\text{VBM}}(\mathbf{r}_{\text{N}V})-{\text{VBM}}(\mathbf{r}_{\text{N}_\text{C}}) =  \left(\epsilon^{{\rm N}V}(0/-)-\epsilon^{\rm{N}_\text{C}}(0/+)\right).
% \end{equation}

% SLR: Note that I have inserted the missing deltas in the equation above that were accidentally dropped from Eqs. 21 and 22.

% RKD: For consistency with the expression for the electric field I am dropping the deltas, but in the other equation I use the fact that keeping or dropping the deltas is equivalent (as long as we take the final CBM position when we drop the deltas).

% We also note that Broadway \textit{et al.}~\cite{Broadway2018spat} investigated the situation after the donor (N) has lost its electron to the acceptor (N$V$) so that the donor and acceptor roles are reversed. Therefore, the value $E_\text{F} = \left(\epsilon^{{\rm N}V}(0/-)+\epsilon^{\rm{N}_\text{C}}(0/+)\right)/2$ parallels the result from solid-state theory that $E_\text{F}$ lies halfway between the electron accepting CBM and the electron donating VBM for equal band curvatures.

% \section{DISCUSSION AND ELUCIDATION OF THE EXPERIMENT OF BROADWAY \textit{ET AL.} \label{sec:imp}}

% ****************************************************


% SLR: A lot of this section probably should occur later in the draft. Please see my comment on Page 5.

% *****************************************************

% We now consider the case of N$V$ in the presence of N$_\text{C}$ in diamond. The converged lattice constant of diamond for the conventional unit cell was $a=3.539$ \AA. This value is in good agreement with a previous theoretical calculation of $a=3.545$~\AA~\cite{Deak2014formation}. We found an electronic band gap of 5.4~eV, which is in good agreement with previous experimental~\cite{Madelung1991semiconductors} and theoretical~\cite{Szasz2013hyperfine,Deak2014formation} results.

% We begin by providing the details of the experiment of Broadway \textit{et al.}~\cite{Broadway2018spat} investigating band bending in the commonly used oxygen-terminated diamond. In that experiment, they performed ODMR spectroscopy on N$V$ centers. They compared the eight resonance frequencies of the N$V^-$ ODMR spectrum to the standard N$V$ spin Hamiltonian including the Zeeman and Stark effects to extract the electric field~\cite{Dolde2011electric,Doherty2012theory}. They found an average electric field in the $z$ direction of $\left<\mathcal{E}_z\right> = 291 \pm 5$~kV~cm$^{-1}$ at an average implantation depth of $\left<d\right> = 35$~nm. This value for the average electric field was not different from the value they obtained in a comparison with N$V$ centers in hydrogen-terminated diamond at the same average implantation depth. The implanted ion was $^{15}$N$^+$ at energies ranging from 4 to 20~keV. The ion dose was 10$^{13}$ ions cm$^{-2}$. The nitrogen ions were implanted to form N$V$ centers following a spatial distribution that could be approximated as uniform over the depth range $d =  0-2\left<d\right>$. The diamond was electronic grade with an intrinsic substitutional N (N$_\text{C}$) concentration less than $1$~ppb. They modeled the electric field as being induced by surface defects with concentrations as high as 1 nm$^{-2}$, which predicts a maximum electric field value at the surface of $\mathcal{E}_z \approx 1.6$~MV~cm$^{-1}$ with a characteristic decay length of roughly 15~nm. They further argued that a positive space charge density exists near the surface such that only N$V$s deeper than approximately 7~nm for $\left<d\right> = 10$~nm exist in the negative charge state usable for sensing. Therefore, averaging over the N$V^-$ distribution, they estimated a maximum average electric field of $\left<\mathcal{E}_z\right> \approx \int_{7}^{20}1.6~\text{MV~cm}^{-1} e^{-z/15}dz{\big /}\int_{7}^{20}dz \approx 600~\text{kV~cm}^{-1}$ for $\left<d\right> = 10$~nm. By contrast, for $\left<d\right> = 35$~nm, their prediction for the maximum average electric field was $\left<\mathcal{E}_z\right> \approx 200~\text{kV~cm}^{-1}$.

% We argue that the Broadway \textit{et al.}~\cite{Broadway2018spat} experiment also captures
% the band bending due to the built-in electric field between N$_\text{C}^+$ and N$V^-$. For the average N$^+$ implantation depth of $\left<d\right> = 35~\text{nm}$, the average electric field measured at the location of N$V^-$ defects should be predominantly due to the built-in electric field between N$_\text{C}^+$ and N$V^-$ rather than due to the surface since hydrogen-terminated and oxygen-terminated samples show no difference in their average electric fields at that average N$^+$ implantation depth~\cite{Broadway2018spat}. Given an implantation dose of 10$^{13}$ ions cm$^{-2}$, the concentration of N$_\text{C}$ is 1.41$\times10^{18}$~cm$^{-3}$ for $\left<d\right> = 35~\text{nm}$~\cite{Broadway2018spat} (see their supplementary material). Therefore, if the implanted diamond region is partitioned into cubes of equal volume each containing on average a single N$_\text{C}$, the side length of one of these cubes will be $l = 8.91$~nm. The average distance between the N$_\text{C}$ will be an upper bound to the average distance between the N$_\text{C}$ defects and the N$V$ defects since the concentration of N$V$ defects produced by the N$^+$ implantation is roughly 1\% of the concentration of N$_\text{C}$ defects produced by the implantation~\cite{Broadway2018spat} (see their supplementary material). Without knowing a priori the cutoff distance beyond which charge transfer cannot occur between the species, we assume that charge transfer can occur for any possible separation between the N$V$ and the N$_\text{C}$ within one of the cubes. Thus, averaging over the possible displacements $\mathbf{r}$ between the positions $\mathbf{r}_{\text{N}V}$ and $\mathbf{r}_{\text{N}_\text{C}}$ of the respective N$V^-$ and N$_\text{C}^+$ defects within one of the cubes we have, 
% \begin{align}
%     \label{eq:partials}\left<\mathcal{E}_z\right>_{\left<d\right> = 35~\text{nm}} &= -\frac{1}{e}\left<\frac{\partial \text{VBM}}{\partial z}\right>_{\left<d\right> = 35~\text{nm}} \\&\approx -\frac{1}{e}\left<\frac{({\text{VBM}}(\mathbf{r}_{\text{N}V})-{\text{VBM}}(\mathbf{r}_{\text{N}_\text{C}}))}{|\mathbf{r}|}\cdot\frac{ z}{|\mathbf{r}|}\right>_{\left<d\right> = 35~\text{nm}}\label{eq:firstexpression}\\&\approx -\frac{1}{e}{\bigg(}\int_{x = 0}^{l/2}\int_{y=0}^{l/2}\int_{z = 0}^{l/2}\frac{({\text{VBM}}(\mathbf{r}_{\text{N}V})-{\text{VBM}}(\mathbf{r}_{\text{N}_\text{C}}))}{(x^2+y^2+z^2)^{1/2}}\cdot\frac{z}{(x^2+y^2+z^2)^{1/2}}dzdydx{\bigg /}\label{eq:firstresult}\\&\quad (l^3/2){\bigg)}\nonumber\\ &\approx 300~\text{kV~cm}^{-1}.
% \end{align}
% As alluded to above, Broadway \textit{et al.}~\cite{Broadway2018spat} provided $\left<\mathcal{E}_z\right>_{\left<d\right> = 35~\text{nm}} = 291\pm5$~kV~cm$^{-1}$, which is in good agreement with our calculated value. Keeping additional significant digits in our calculation, the difference between the results is less than 3\%. By contrast, our application of the same formalism at $\left<d\right> = 7~\text{nm}$ yields $\left<\mathcal{E}_z\right>_{\left<d\right> = 7~\text{nm}} \approx 500~\text{kV~cm}^{-1}$ in comparison with $\left<\mathcal{E}_z\right>_{\left<d\right> = 7~\text{nm}} = 432 \pm 10~\text{kV~cm}^{-1}$ from experiment~\cite{Broadway2018spat}. Keeping additional significant digits, the difference between the results is roughly 12\%. Thus, for $\left<d\right> = 35~\text{nm}$ the band bending due to the built-in field between N$_\text{C}^+$ and N$V^-$ appears to determine the average electric field measured at N$V^-$ centers.

% We now turn to a discussion of our derivation. In moving from Eq. (\ref{eq:firstexpression}) to Eq. (\ref{eq:firstresult}), we have used the fact that there are four possible N$V^-$ orientations in the diamond crystal. Only one of these contributes to an appreciable measured field in the $z$ direction since $\mathcal{E}_x$ and $\mathcal{E}_y$ were set to zero in the reference frame of each N$V^-$ in fitting the measured spectra in the work of Broadway \textit{et al.}~\cite{Broadway2018spat}. To arrive at our results, we have also used  $({\text{VBM}}(\mathbf{r}_{\text{N}V})-{\text{VBM}}(\mathbf{r}_{\text{N}_\text{C}})) \approx -0.8$~eV from our calculations. Finally, we have used the fact that the Broadway \textit{et al.}~\cite{Broadway2018spat} measurements were insensitive to the sign of the electric field so that the contribution from $-z$ does not cancel the contribution from $z$. Furthermore, we can verify explicitly that the system satisfies $1/\bar{\Gamma}^{ii} \ll t_{\text{exp}} \ll 1/\bar{\Gamma}^{ij}$. We consider both $E_\text{F} =$ VBM and $E_\text{F} =$ CBM to bound the possible timescales. From our calculations, the absolute position of the VBM of diamond is approximately 9.4~eV and the band gap of diamond is approximately 5.4~eV. For $1/\bar{\Gamma}^{ii}$, we therefore find timescales on the order of picoseconds, while for $1/\bar{\Gamma}^{ij}$ we find timescales longer than the age of the universe. We also note that our explanation requires no fitted parameters and that our results would be generally applicable to defects in any wide-bandgap semiconductor, such as in the various widely studied polytypes of SiC~\cite{Kraus2014room,Castelletto2014a,Bockstedte2004ab,Koehl2011room,Riedel2012resonant,Kimoto2014fundamentals,Wang2017efficient,Wang2017scalable,Fuchs2015engineering,Kuate2018energetics,Gadalla2021enhanced,Kuate2019parallel,Nagy2018quantum,Widmann2015coherent,Lohrmann2017a,Bracher2017selective,Falk2013polytype,Soykal2016silicon,Soykal2017quantum,Weber2010quantum,Kraus2017three,Gali2011time,Awschalom2018quantum,Wolfowicz2021quantum,Whiteley2019spin}.

 
% \section{CONCLUSION \label{sec:conc}}
% In conclusion, based purely on $ab~initio$ calculations, we have succeeded in providing an explanation for the average electric field of 291$\pm$5~kV~cm$^{-1}$ at the $^{15}$N$^+$ average implantation depth of $\left<d\right> = 35$~nm due to band bending for the commonly used oxygen-terminated diamond. Such a result would be useful for predicting the functioning of semiconductor devices as rectifiers and switching devices. Our result could also be useful for predicting and correcting the spectral diffusion of the optical frequencies of the solid-state single-photon sources used for applications in quantum information and computation.  

% \section*{ ACKNOWLEDGMENTS:}
% R.K.D. gratefully acknowledges financial support from the Princeton Presidential Postdoctoral Research Fellowship and from the National Academies of Science, Engineering, and Medicine Ford Foundation Postdoctoral Fellowship program. We also acknowledge support by the STC Center for Integrated Quantum Materials, NSF Grant No. DMR-1231319. 


% ***************************************************


% **************************************************

% SLR: I want to summarize here my understanding of the model and how it is used in at least the first part of the draft.  My goal here is to get a better physical picture of what is going on here. If we first consider an isolated $H_2$ molecule in a vacuum then the two electrons are equally shared between the two protons forming a nice stable covalent bond. There is no net dipole moment as there is no net charge transfer between the two nuclei. Next let us consider the molecule $HF$ in a vacuum. Since the fluorine atom is far more electronegative than the H atom while the bond is still covalent there is an unequal sharing of the charge between the two nuclei. This results in an electric dipole moment for the $HF$ molecule. If we next consider an isolated $Na$ atom in space and an isolated $Cl$ atom in space nothing really happens between the two atoms provided they are separated by a large distance. When we bring them closer $Na$ has a small ionizaton potential and $Cl$ has a large electron affinity so at some distance they energetically prefer to exist as $Na^+$ $Cl^-$ where there is complete charge transfer of a single electron from the $Na$ atom to the $Cl$ atom. This results in two completely ionized species which are attracted by electrostatics forming an ionic bond. Of course at some point nuclear repulsion kicks in.

% Now in your model for this manuscript you wish to consider a donor species D and an acceptor species A
% introduced as defects in a host material. How do you decide the separation distance between D and A? Are they next to each other or nearest neighbors or what and does the model take into account their separation distance?

% RKD: The separation distance can take on any value as long as charge can be transferred between the two species for the given separation distance. Without knowing what this maximum cutoff would be a priori, when dealing with the Broadway \textit{et al.} experiment we simply consider that every separation distance within the cubes into which we partition the diamond crystal that contain on average a single N$_\text{C}$ allows for the transfer of charge between the N$V$ and N$_\text{C}$ species. I have now said this in the text.

% At some distance between them it is reasonable to assume that there would be charge transfer between D and A to form $D^+$ and $A^-$ or $D\rightarrow D^+$ and $A\rightarrow A^-$. Here we assume that D only has a positive charge state of unity and A only has a negative charge state of minus unity (sloppy terminology indeed but you get the idea). As impurities in a host diamond there is more going on here in this situation that in the previous vacuum cases. First there is the formation energy $\Delta H_f$ for each of these two species $\Delta H_f^{D^+}$ and $\Delta H_f^{A^-}$ given as in Eq. 1. This determines how easy it will be to form these two charge states in the host material. The problem is that we can not apply Eq. 1 to a system of two defects in the supercell to find $\Delta H_f$ so we use the dilute limit approximation to assume that these two formation energies only depend upon a single parametrized Fermi level $E_F$. Is this correct? 

% RKD: Yes, that is correct. We consider the defects separately in the dilute limit since if they were together in the supercell, we would need two separate Fermi levels to describe the defects since the component of the formation of the system as a whole that involves the addition or subtraction of electrons would necessarily need to reflect the energy associated with adding electrons to or subtracting electrons from the neutral states of the two defects (which is what the charged defects are formed from) and these neutral states are necessarily not equilibrium states (since otherwise the charged defects would not form) so that the addition of electrons to or subtraction of electrons from these states cannot be described by a single uniform Fermi level.

% Next I think you are suggesting in your model that $E_F$ is really not an independent parameter but it really depends on the concentrations of D and A, namely $n_D$ and $n_A$. Your model assumes a low T so $n_o$ and $p_o$ are small or negligble. As you suggested on Page 7 of your draft from last night, $E_F$ is really a balance between forming the $D^+$ and $A^-$ and the formation of $D$ and $A$. Is this correct?

% RKD: If by $D$ and $A$ you mean $D^0$ and $A^0$, yes $E_F$ is really a balance between the formation of $D^+$ and $A^-$ and the formation of $D^0$ and $A^0$.

% Next I assume that your model reveals at the end of the day that there is some charge transfer between $D$ and $A$ that is preferred. Is this correct?

% RKD: Yes, ultimately the model determines the field between the $D$ and $A$ defects, which suggests a preferred charge transfer.

% Now let me address the band-bending issue. I do not believe that band-bending comes first and then the electric field when looking at impurities in semiconductors. In fact it is the presence of a built-in electric field in the host material that causes the bending of the bands.  Actually I do not like the term band-bending because it is not precise enough. For the case of a pn-junction with a built-in electric field the effect is to "tilt" not "bend" the CBM and VBM of the host semiconductor over the space charge region. The energy bands are really only "bent" for the case of a semiconductor-vacuum case where there are surface states at the interface.
% I like the discussion of Richard Dalven, "Introduction to Applied Solid State Physics" here and we should include it as a reference in the draft.
% So it is correct to say that band-tilting only occurs in this model when there is an built-in electric field due to the existence of $D^+$ and $A^-$ in the host semiconductor separated by some distance?

% RKD: We use the bending of the bands to calculate the value of the electric field, but physically you may be right that the field itself causes this bending and we are effectively only reading off the field from the bending. I should note though that Dalven also writes that the built-in electric field is given by the gradient of the potential (which ultimately implies that the built-in electric field can be calculated from the amount of band bending). Indeed the bending is associated with the presence of $D^+$ and $A^-$. Also, I would prefer the term band bending since the Dalven uses the expression `tilting the bands' to refer to the the basic effect of an applied electric field and not to the effect of a built-in electric field.

% Now the first part of the draft is the derivation of your Eq. 8 for the built-in electric field. I have not yet read your Refs. 1 and 3, so are you deriving something that they have already come up with or are you deriving it for the first time? If the former is true why didn't Broadway {\it et al.} use it to model their experimental results?

% RKD: The former Eq. 8,
% \begin{equation}
%     \vec{\mathcal{E}} = -\frac{1}{e}\nabla({\text{VBM}}(\mathbf{r})) 
% \end{equation}
% is taken straight from the references. We then make a discrete approximation to the derivative to obtain the expression that the electric field is approximately equal to. Broadway {\it et al.} did use the equation to model their experimental results, but only as relates to the effects of the surface. They were not able to quantitatively capture the effect of the presence of N$_\text{C}$ so they dealt with the effect of N$_\text{C}$ qualitatively. Quantitatively capturing the effect of the presence of N$_\text{C}$ is what we are doing in this paper.

% ***************************************************

% SLR: Is there a difference between D and D$^0$ which you subsequently use and A and A$^0$?

% RKD: D represents all charge states of the defect and D$^0$ represents a specific charge state of the defect. I have made this fact clearer in the text.

% SLR: Fine! Please explicitly say this in the draft at this point where you discuss the model.

% RKD: Yes, sounds good.

% % We further assume that the charge-transition level where D$^0$ loses a single electron is the only charge-transition level for the donor and that the charge-transition level where A$^0$ gains a single electron is the only charge-transition level for the acceptor. Effectively, we assume that no donor charge transitions and no acceptor charge transitions are possible other than those alluded to above. We will see below that the equilibrium position of $E_\text{F}$ does not depend on whether the donor level lies above the acceptor level or the acceptor level lies above the donor level. 

% SLR: The above is good!

% SLR: What is your physical model here? I would think that D lies closest to CBM and not VBM as you stated above. Similarly I would think that A lies closest to VBM and not CBM as you stated above. At least these are the conventional ways of looking at extrinsic semiconductors and their doping so what is going on here?

% RKD: It is indeed a confusing way of putting it. All I'm trying to say is that we can assume that there are no other donor levels and no other acceptor levels. I have simplified the framing of the conditions.

% SLR: Fine! Please explicitly say this in the draft at this point where you discuss the model.

% RKD: Yes, sounds good.

% SLR: There is a lot of information in Fig. 1 and I am afraid that it may be confusing the reader. I have no problem with the six lines as they are a more detailed example of the schematic shown in the paper by S.-H Wei and his colleagues (Cf. Ref. 28). It is the addition of the six expressions like $\epsilon^{\rm{A}}(0/\text{+2})$, etc. at the top of the figure that make it more complicated than it should be. Also why not enlarge the figure a lot? I am not sure I understand the notation $\epsilon^{\rm{D,A}}(0/\text{+},0/\text{-})$. Is not this single line supposed to be charge independent with q$=0$? Also why is there structure below VBM and above CBM in the plots. Don't the figures cut off by convention at these limits? Perhaps this figure could be preceded by two simpler figures where you simply show the three lines for the donor and the three lines for the acceptor separately? The figure is simply too complicated for the reader to digest in its present form. Also in the separated figures it might be easier to actually show the meaning of Fermi level pinning as in Ref. 28. 

% RKD: I have simplified the framing of the conditions so that I think the condition is now clear enough that a figure is not necessary.

% SLR: I understand that you are going to eliminate this figure here as it does not add to adequately describing your narrative.



% SLR: Good!

% SLR: So if these defect levels are not close to either VBM or CBM where are these species either getting their charge from or donating their charge to? Perhaps a nice simple picture of the band structure in real space might help here.

% RKD: I have added a picture showing also the position of the Fermi level.

% SLR: Good!

% SLR: Let me ask the following question. Is the model one where both the donor and acceptor atoms are deep levels such that they do not exchange charge with the respective VBM and CBM levels, but which each other? Thus they should lie closer to mid-gap. Should the acceptor level be above the donor level in this model?

% RKD: Yes, that is the model. Whether the donor level lies above the acceptor level or the acceptor level lies above the donor level, the position of the Fermi level does not change. 

% SLR: Fine! Please explicitly say this in the draft at this point where you discuss the model. I would suggest a general discussion of the model here. You can specialize it for the Broadway experiment later in your draft. 

% RKD: Sounds good.

% RKD: For the Broadway \textit{et al.} experiment that we consider later, they were looking at the N$V^-$ in diamond, which is the situation after the donor (N) has lost its electron to the acceptor (N$V$). That picture more closely resembles what you are describing.

% SLR: I would place this discussion later where you apply the model to the Broadway experiment.

% RKD: Yes, sounds good.


% % Now, consider the case where the total concentrations of donors and acceptors are equal, $n_\text{D} = n_\text{A} = C$ for some constant $C$.  

% SLR: What is the difference between $D$ and $D^0$ and $A$ and $A^0$ here?

% RKD: $D$ is for the total concentration of all charge states of the defect, $D^0$ is for the neutral charge state only. Similarly for $A$.

% % % Figure environment removed

% SLR: Why do you introduce this equation here? How is it helpful for the model and how it is further used in the draft?

% RKD: I have removed the equation and instead painted a more physical picture to motivate what is happening.


% % We can now solve for the $E_\text{F}$ that is required to obtain the equality, which allows us to define an ACTL for one defect in the presence of another given by,
% % \begin{align}
% % \label{eq:chgtransitionlevel}
% %     E_\text{F} &= \frac{1}{2}\left(\Delta H_f({\rm D^{0}}, \, \{\mu_i^\text{D}\}, \, 0) - \Delta H_f({\rm D^{+}}, \, \{\mu_i^\text{D}\}, \, 0)\right)\\ &+ \frac{1}{2}\left(\Delta H_f({\rm A^{-}}, \, \{\mu_i^\text{A}\}, \, 0) - \Delta H_f({\rm A^{0}}, \, \{\mu_i^\text{A}\}, \, 0)\right)\nonumber\\
% %     &+\frac{k_BT}{2}\ln\left(\frac{N_{\text{A}}g_{\text{A}^{-}}N_{\text{D}}g_{\text{D}^{0}}}{N_{\text{A}}g_{\text{A}^{0}}N_{\text{D}}g_{\text{D}^{+}}}\right) \nonumber\\
% %     E_\text{F} &= \frac{1}{2}\left(\epsilon^\text{D}(0/+)+\epsilon^\text{A}(0/-)\right)\\
% %     &+\frac{k_BT}{2}\ln\left(\frac{g_{\text{A}^{-}}g_{\text{D}^{0}}}{g_{\text{A}^{0}}g_{\text{D}^{+}}}\right) .\nonumber
% % \end{align}

% SLR: Given your Eqs. 1, 2, and 3 in the draft so far, I have no idea where the above equation came from.

% RKD: I have added an additional step in the derivation and corrected a missing factor of 1/2.

% SLR: You need to do a better job of explaining where your equations come from if you wish your reader to fully understand the model. For example, starting from your Eq. 1 I see how you can get part of the first part of your Eq. 6, namely the ln dependence. I do not see from Eq. 1 how you break out the $E_F$ term!

% In Eq. 7 you previously had defined your terms like  $\epsilon^\text{A}(0/-)$ in what was your Fig. 1 but you have now eliminated that so you need to define these terms somewhere before just plopping them into Eq. 7 out of nowhere apparently. Perhaps your new Fig. 1 can be of soem help here.

% RKD: I have replaced the equations with discussion and physical intuition for where the value of the Fermi level comes from.

% % The equality $n_\text{D} = n_\text{A} = C$ is not particularly restrictive for a wide-bandgap semiconductor with defect levels separated from the band edges since in that case proximity will be required for the transfer of charge~\cite{Collins2002}. If the concentration of defects in the crystal is such that defects are spaced on average more than a few lattice constants from one another so that charge can only be exchanged with the nearest defect, then we can take the limit where $C$ is such that exactly one donor and exactly one acceptor exist in the crystal since the remainder of the crystal will not interact appreciably with the system of two defects. Given that thermodynamic theory applies to macroscopic systems, a caveat exists. Namely, the result only rigorously applies when averaged over many such defect pairs. Taking into consideration the caveat, $E_\text{F} = \left(\epsilon^\text{D}(0/+)+\epsilon^\text{A}(0/-)\right)/2$ captures the energetics of charge transfer for defects in a wide-bandgap semiconductor.

% % Small world 
% %The generalization to multiple defects is to let $\Delta \text{q}_i$ and $\Delta \text{q}'_j$ be positive charge-state differences for all $i$ and $j$ and solve,
% % \begin{equation}
% % \label{eq:multichgconservation}
% %     \sum_i\Delta \text{q}_iC_{\text{X}_i^{\text{q}_i+\Delta \text{q}_i}}/\sum_i\Delta \text{q}_iC_{\text{X}_i^{\text{q}_i}} - \sum_j\Delta \text{q}'_jC_{\text{Y}_j^{\text{q}'_j-\Delta \text{q}'_j}}/\sum_j\Delta \text{q}'_jC_{\text{Y}_j^{\text{q}'_j}} = 0,
% % \end{equation}
% % where we have chosen the concentration of donors to be related to the concentration of acceptors by,
% % \begin{equation}
% % \label{eq:allreact}
% %     \sum_i\Delta \text{q}_i(C_{\text{X}_i^{\text{q}_i+\Delta \text{q}_i}}+C_{\text{X}_i^{\text{q}_i}}) = \sum_j\Delta \text{q}'_j(C_{\text{Y}_j^{\text{q}'_j-\Delta \text{q}'_j}}+C_{\text{Y}_j^{\text{q}'_j}}).
% % \end{equation}
% % The condition described in Eq. (\ref{eq:allreact}) ensures that charge is conserved if all defects participate in the exchange of electrons. In general, Eq. (\ref{eq:multichgconservation}) does not readily lend itself to a simple analytic solution for the Fermi level. Though, for a single donor and a single acceptor species, Eq. (\ref{eq:multichgconservation}) may be solved analytically for arbitrary nonzero $\Delta \text{q}$ and $\Delta \text{q}'$.

% ****************************************************

% SLR:

% In light of what you have read from the supplementary material in the Broadway paper, I would like to suggest that it would be very helpful at this stage to fill in an outline of the draft. I strongly believe this will help us clarify the structure and questions of the draft in its current form. Please feel free to cut and paste your replies below from various parts of the draft and to add new material as needed. THE IDEA HERE IS TO START WITH THE KEY RESULTS OF THE BROADWAY PAPER THAT YOU ARE ADDRESSING AND ESSENTIALLY WORK BACKWARDS!

% 1. Why is a knowledge of the E-field of the NV$^-$ impurity-vacancy center important to know in diamond? 

% RKD: The precise value of the electric field within a semiconductor device is critical to its functioning~\cite{Broadway2018spat,Iwasaki2017direct,Zhang2012band,Kotadiya2018universal,Simon2010polarization,Stathis2006the,Zhang2006electronic,Kaczer2018a}. Local variations in charge distribution, such as those due to charge traps in the semiconductor bulk (bulk charge traps), affect this electric field~\cite{Broadway2018spat} and can potentially shift semiconductor devices into dangerous operation regimes~\cite{Iwasaki2017direct}. The N$V^-$ in diamond has proven to be a means of measuring the electric field within the wide-bandgap semiconductor that is diamond. Since the N$V^-$ is being used to measure a critically important quantity for semiconductor devices, it is necessary to get some theoretical insight into what it is actually measuring. 

% SLR: You can now take this new paragraph and integrate it into your original first paragraph in the Introduction to create a nice motivation for the problem at hand. Let us now look at your second paragraph in the Introduction. Would not a lot of this be appropriate at the point where you discuss the Broadway model and Yates Chemical Reviews  article? You also introduce the Poole-Frenkel model here. I do not know what it is and I suspect most of your audience will not know either. More importantly why is it important to your model and the subsequent discussion? Is the Poole-Frenkel model important for understanding the Broadway model? 

% RKD: The Poole-Frenkel model is not important to the discussion of either our model or the Broadway one. I will remove mention of it.

% 2. Exactly how do Broadway {\it et al.} measure this internal E-field in their paper?

% RKD: They perform optically detected magnetic resonance spectroscopy on N$V^-$ centers.

% SLR: So what do they obtain as a value? They also seem to provide an estimate of 600 $\frac{km}{cm}$ on the first two pages of their paper. It is important to discuss how they arrived at such an estimate in the next section.

% 3. Exactly how do Broadway {\it et al.} compute this internal E-field in their paper and how does their result compare with their experiment?

% RKD: They compare the eight resonance frequencies of the N$V^-$ spectrum to the standard N$V$ spin Hamiltonian including the Zeeman and Stark effects to extract the electric field. 

% SLR: How does their measurement compare with their previous back-of-the envelope estimate? Exactly what do Broadway {\it et al.} do in their Supplementary Information section to compute the electric? If you are going to suggest an alternative model then you need to take some time to explain exactly what Broadway {\it et al.} exactly do in their Supplementary Information section so that you can later show to your reader that your model is (1) simpler to implement and (2) provides an better agreement with experiment.

% 4. You say that their calculation for the internal E-field is complicated. What exactly is your model and how is it simpler than theirs? I assume that your model gives a much better agreement with experiment.

% RKD: The model I propose is that we can simply consider two interacting defects one labeled as a donor and the other labeled as an acceptor. 

% SLR: It seems that Broadway {\it et al.} also make this assumption and dictate that these levels must be far from the diamond band edges.  

% RKD: These two defects will each have a single charge-transition level (corresponding to loss of a single electron from the neutral state for the donor and gain of a single electron from the neutral state for the acceptor). In that manner, we do not need to consider surface charges or free carriers, which would require numerical integration of Fermi-Dirac distribution functions (both to treat the surface charges and to treat the free carriers). 

% SLR: This preceding sentence is exactly what needs to be spelled out in some detail in the answer to the previous question! This way it will be easier for you to explicitly show how your model is simpler to use! 

% RKD: Indeed our model gives much better agreement with experiment.

% SLR: Indeed this is the central point of our paper but we need to spell out how our model works when compared to the Broadway model! All of this really needs to augment the penultimate paragraph of your Introduction. 

% SLR: Using your own words here you need to spell out what the Broadway model is here and how the model you propose will work. The motivation for our model is that it will be simpler to implement. Indeed to take the time to spell out the Broadway model in some detail here will make it much easier to show our model is different and simpler.

% RKD: Our approaches are pretty much exactly the same. The Broadway paper calculates the Fermi level and then determines the amount of band bending by allowing the bands to bend until they reach the Fermi level and we do exactly the same thing (just that we allow the bands to bend until the defect charge transition levels reach the Fermi level). Since we only treat two defects in a crystal, the calculation is simplified considerably. We show, however, that the accuracy of the approach is not diminished and may even be improved.


% 5. Please think about starting with your Eqs. 21-24 and then working backwards to Eq. 20. From Eq. 20 you should work backwards to explain where all the terms in your improved model come from. If you think a simple diagram would help with any stage of the exposition feel free to build one.

% RKD: OK, sounds good.

% SLR: I am in no way suggesting that we are going to actually need these equations from the Broadway paper but in case we need some of them. At the minimum by going through this exercise I can at least get a better idea of the Broadway model which may be useful.

% \begin{equation}
% \nabla^2 V(\vec{r}) = \frac{-\rho(\vec{r})}{\epsilon \epsilon_o}
% \end{equation}

% To consider the charge distribution in diamond we need to consider the electron and hole carrier concentrations ($n$ and $p$ respectively)and the defect concentrations of negatively charge acceptors ($N_A^-$)  and positively charged donors ($N_D^+$) in thermal equilibrium. Note that $n$, $p$, $N_D^+$, and $N_A-$ are all determined by the filing of the electronic states in the diamond valence bands and conduction bands and the defect states, respectively. The filling of electronic states is calculated by convolving (check on this word) the energy dependent DOS with the probability of a state being occupied by an electron at a given energy. The probability of a state being occupied (unoccupied) is given by the Fermi-Dirac distribution function,

% \begin{equation}
% f_D(E) = \frac{1}{ 1 + exp\left(\frac{E-E_F}{k_B T}\right) }
% \end{equation}

% and its complement

% \begin{equation}
% 1 - f_D(E) = \frac{1}{ 1 + exp\left(\frac{E_F-E}{k_B T}\right) }
% \end{equation}

% where $k_B$ is the Boltzmann constant, T is the temperature, and both functions are centered around the Fermi level ($E_F$) of a material. The Fermi level can be determined by requiring that the material be charge neutral at the Fermi level giving

% \begin{equation}
% p + \sum\limits_{Donors} N_D^+ = 
% n + \sum\limits_{Acceptors} N_A^-
% \end{equation}

% where p(n) is the carrier density of holes (electrons), $N_D^+$ is the defect density of ionized donors, and $N_A^-$ is the defect density of ionized acceptors in the bulk of diamond in thermal equilibrium. We can calculate the electron and hole densities using the effective mass ($m^*$) approximation for the valence and conduction bands DOS

% \begin{equation}
% N_V(E) = \frac{(2m_V^*)^{\frac{3}{2}}}{2 \pi^2 h^3}
% \sqrt{E_V - E_F}
% \end {equation}

% \begin{equation}
% N_C(E) = \frac{(2m_C^*)^{\frac{3}{2}}}{2 \pi^2 h^3}
% \sqrt{E - E_C}
% \end {equation}

% where $h$ is Planck's constant and $m_V^* (m_C^*)$ is the effective mass of the valence (conduction) band. Typically, the Boltzmann approximation of  

% \begin{equation}
% f_D(E) \approx e^{\frac {(E_F -E)}{k_B T}}
% \end {equation}
% for 
% \begin{equation}
% E_F - E \gg k_B T
% \end {equation}

% is used to estimate the electron and hole concentrations 

% \begin{equation}
%     p(z) = 2 \left( \frac{2 \pi m_V^* k_B T}{h^2}\right)^{\frac{3}{2}} e^{\frac{E_V - E_F}{k_B T}} = N_C \, e^{\frac{E_V - E_F}{k_B T}}
% \end{equation}

% \begin{equation}
%     n(z) = 2 \left( \frac{2 \pi m_C^* k_B T}{h^2}\right)^{\frac{3}{2}} e^{\frac{E_F - E_C}{k_B T}} = N_V \, e^{\frac{E_F - E_C}{k_B T}}
% \end{equation}

% where $N_V$ is the effective density of states for holes and where $N_c$ is the effective density of states for electrons. In the surface transfer doped hydrogen-terminated diamond, the conduction band at the surface crosses through the Fermi level (hence becomes a degenerate semiconductor) with a relatively small carrier density of 2.5 $\times$ 10$^{12}$ holes cm$^{-2}$, and the Boltzmann approximation is no longer appropriate we have to solve the full convolution:

% \begin{equation} 
% p(z) = N_V \frac{2}{\sqrt{\pi}}
% \int_0^\infty  \frac{\sqrt{\epsilon}}{1 + e^{\epsilon - \frac {E_V - E_F}{k_B T}}}d\epsilon = N_V F_{\frac{1}{2}} \left (\frac{E_V - E_F}{k_B T}\right)
% \end{equation}

% \begin{equation} 
% n(z) = N_V \frac{2}{\sqrt{\pi}}
% \int_0^\infty  \frac{\sqrt{\epsilon}}{1 + e^{\epsilon - \frac {E_V - E_F}{k_B T}}}d\epsilon = N_V F_{\frac{1}{2}} \left (\frac{E_V - E_F}{k_B T}\right)
% \end{equation}

% where $F$ is the set of Fermi-Dirac integrals, defined as

% \begin{equation} 
% F_j(\eta)=
% \int_0^{\infty}\frac{x^j}{exp (x - \eta) +1 } dx, \,\,\,\, \text{for} \,\,\,\,j > -1
% \end{equation}

% These integrals can be accurately and rapidly approximated numerically where in this work the Python module {\it fdint} was used.

% To calculate the density of ionized defects we approximate the defect DOS as a Dirac delta function
% $\delta(E - E_o)$ around the defect ionization energy, $E_o = E_A$ for acceptor and
% energy, $E_o = E_D$ for donor transitions, with defect densities of $N_A$ and $N_D$


% \begin{equation} 
% N_D^+(z) =
% \int_{-\infty}^{E_D(z)}
% N_D(z) \,\delta(E_D(z) - E)\,[1 -f_D(E)]\,dE = N_D(z)\,[1 -f_D(E_D(z))]
% \end{equation}

% \begin{equation} 
% N_A^-(z) =
% \int_{E_A(z)}^{\infty}
% N_A(z) \, \delta(E - E_A(z))\,f_D(E)\,dE = N_A(z)\, f_D(E_A(z))
% \end{equation}
% Lateral variations in the surface states and defect densities are ignored and only the charge distribution in $z$ needs to be considered where $z$ is defined as positive going into the diamond. The total charge density within the diamond can then be described by

% \begin{equation}\rho(\vec{r}) =\rho(z) = e\, p(z) - e\, n(z) + e\sum\limits_{Donors} N_D^+(z) 
% - e\sum\limits_{Acceptors} N_A^-(z)
% \end{equation}

% \begin{equation} p(z) =
% N_V \exp\left(\frac{-E_F + E_V(z)}{k_B T} \right)
% \end{equation}

% As external potentials affect the bands equally we re-define the z-dependent defect and band energies as a function of a single depth dependent potential. First, we define a potential $\phi(z)$ as the separation between the Fermi level $E_F$ and the intrinsic energy level $E_i$ where the Fermi level $E_F$ would be in an intrinsic diamond

% \begin{equation} p(z) =
% N_V \exp\left(\frac{-E_F - \phi(z) + E_V(z)}{k_B T} \right)
% \end{equation}

% \begin{equation} p(z) =
% N_V \exp\left(\frac{-\frac{1}{2}E_V(z) - \frac{1}{2}E_C(z) + \frac{1}{2}k_B T \log \frac{N_C}{N_V} -e \phi(z) + E_V(z)}{k_B T} \right)
% \end{equation}

% \begin{equation} p(z) =
% N_V \exp\left(\frac{-\frac{1}{2}E_V(z) - \frac{1}{2}E_C(z) + \frac{1}{2}k_B T \log \frac{N_C}{N_V} -e \phi(z) + E_V(z)}{k_B T} \right)
% \end{equation}

% \begin{equation} p(z) = N_V \sqrt{\frac{N_C}{N_V}} \exp^{\frac{-e\phi}{k_B T}} 
% \exp^{\frac{E_V - E_C}{ 2 k_B T}}
% \end{equation}

% \begin{equation}  
% N_D^+(z) = N_D(1 -f_D(E_D(z))) =  N_D \frac{1}{1 + \exp(\nu(z))\exp\left(\frac{E_F - E_{D}(z)}{k_B T}\right)} 
% \end{equation}

% \begin{equation}  
% N_D^+(z) = N_D \frac{1}{1 + \exp(\nu(z)) \exp\left(\frac{E_F - E_{D}(z)}{k_B T}\right)} 
% \end{equation}

% \begin{equation}  
% N_A^-(z) = N_A \frac{1}{1 + \exp(\nu(z)) \exp\left(\frac{E_F - E_{D}(z)}{k_B T}\right)} 
% \end{equation}



% \begin{eqnarray*}  
% p + \sum\limits_{Donors} N_D^+ = N_C \exp \left(\frac{E_V - E_F}{k_B T} \right) + \frac{D_N}{1 + \exp\left(\frac{E_F - E_{N_S}}{k_B T}\right)} + 
% \frac{D_{NV}}{1 + \exp\left(\frac{E_F - E_{NV^+}}{k_B T}\right)} \\ =  n + \sum\limits_{Acceptors} N_A^- = N_V \exp \left(\frac{E_F - E_C}{k_B T} \right) + 
% \frac{D_{NV}}{1 + \exp\left(\frac{ E_{NV^-} - E_F}{k_B T}\right)} 
% \end{eqnarray*}

% \begin{equation}
%     {\text{CBM}}(\mathbf{r}_\text{A})-{\text{CBM}}(\mathbf{r}_\text{D}) = {\text{VBM}}(\mathbf{r}_\text{A})-{\text{VBM}}(\mathbf{r}_\text{D}) = \left(\epsilon^{\rm{A}}(0/-)-\epsilon^{\rm{D}}(0/+)\right). 
% \end{equation}
% *************************************

 
%  Our approach to the determination of the electric field associated with charge transfer between bulk charge traps in a wide-bandgap semiconductor is analogous to the approach to treating the charge transfer induced by a Schottky barrier at a metal-semiconductor interface. To motivate the analogy, we note the similarity to the analogy between the Schottky effect~\cite{Orloff2017handbook} and the Poole-Frenkel effect~\cite{Frenkel1938on}. This latter analogy is made clear when the Schottky effect is viewed as electron emission from a metal and the Poole-Frenkel effect is viewed as electron emission from bulk charge traps within an insulator. Therefore, we treat the charge transfer between bulk charge traps in a wide-bandgap semiconductor as being associated with an effective band bending along the line connecting the two individual bulk charge-trap defects. 
 
 % z-dependent defect and band energies as a function of a single depth dependent potential. First, we define a potential $\phi(z)$ as the separation between the Fermi level $E_\text{F}$ and the intrinsic energy level $E_i$ where the Fermi level $E_\text{F}$ would be in an intrinsic diamond

% \begin{equation} p(z) =
% N_V \exp\left(\frac{-E_\text{F} - \phi(z) + E_V(z)}{k_B T} \right)
% \end{equation}

% \begin{equation} p(z) =
% N_V \exp\left(\frac{-\frac{1}{2}E_V(z) - \frac{1}{2}E_C(z) + \frac{1}{2}k_B T \log \frac{N_C}{N_V} -e \phi(z) + E_V(z)}{k_B T} \right)
% \end{equation}

% \begin{equation} p(z) =
% N_V \exp\left(\frac{-\frac{1}{2}E_V(z) - \frac{1}{2}E_C(z) + \frac{1}{2}k_B T \log \frac{N_C}{N_V} -e \phi(z) + E_V(z)}{k_B T} \right)
% \end{equation}

% \begin{equation} p(z) = N_V \sqrt{\frac{N_C}{N_V}} \exp^{\frac{-e\phi}{k_B T}} 
% \exp^{\frac{E_V - E_C}{ 2 k_B T}}
% \end{equation}

% \begin{equation}  
% N_D^+(z) = N_D(1 -f_D(E_D(z))) =  N_D \frac{1}{1 + \exp(\nu(z))\exp\left(\frac{E_\text{F} - E_{D}(z)}{k_B T}\right)} 
% \end{equation}

% \begin{equation}  
% N_D^+(z) = N_D \frac{1}{1 + \exp(\nu(z)) \exp\left(\frac{E_\text{F} - E_{D}(z)}{k_B T}\right)} 
% \end{equation}

% \begin{equation}  
% N_A^-(z) = N_A \frac{1}{1 + \exp(\nu(z)) \exp\left(\frac{E_\text{F} - E_{D}(z)}{k_B T}\right)} 
% \end{equation}



% \begin{eqnarray*}  
% p + \sum\limits_{Donors} N_D^+ = N_C \exp \left(\frac{E_V - E_\text{F}}{k_B T} \right) + \frac{D_N}{1 + \exp\left(\frac{E_\text{F} - E_{N_S}}{k_B T}\right)} + 
% \frac{D_{NV}}{1 + \exp\left(\frac{E_\text{F} - E_{NV^+}}{k_B T}\right)} \\ =  n + \sum\limits_{Acceptors} N_A^- = N_V \exp \left(\frac{E_\text{F} - E_C}{k_B T} \right) + 
% \frac{D_{NV}}{1 + \exp\left(\frac{ E_{NV^-} - E_\text{F}}{k_B T}\right)} 
% \end{eqnarray*}

% \begin{equation}
%     {\text{CBM}}(\mathbf{r}_\text{A})-{\text{CBM}}(\mathbf{r}_\text{D}) = {\text{VBM}}(\mathbf{r}_\text{A})-{\text{VBM}}(\mathbf{r}_\text{D}) = \left(\epsilon^{\rm{A}}(0/-)-\epsilon^{\rm{D}}(0/+)\right). 
% \end{equation}

% Our approach is to calculate the electric field at the location of an acceptor A due to the presence of a donor D using the gradient of the electric potential. The electric potentials in the acceptor and donor parts of the diamond sample are taken to be the thermal ionization potentials in the corresponding parts of the sample. We use the dilute limit for calculating the ionization potentials, which we showed in earlier work leads to a small error for a 512-atom supercell~\cite{Kuate2021theor}.

% In order to determine the thermal ionization potentials, we begin by presenting the formalism for calculating the formation energy of a species X$^{\rm q}$. This formation energy, $H_f({\rm X^{\rm q}}, \, \{\mu_i^\text{X}\}, \, E_\text{F})$, is given by~\cite{zhang1991chemical, Freysoldt2014first,Kuate2018energetics,Kuate2019how,Kuate2021methods,Kuate2021theor,zunger2021under,Yang2015self,Ashcroft1976solid}
% \begin{equation}
% \label{eq:form_eq-1.0}
% \Delta H_f({\rm X^{\rm q}}, \, \{\mu_i^\text{X}\}, \, E_\text{F}) = E_{\text{def}}({\rm X^{\rm q}}) - E_0 - \sum_i\mu_i^\text{X}n_i + {\rm q}\, E_{\text{F}} + E_{\text{corr}}(\rm X^{\rm q}),
% \end{equation}
% where $E_{\text{def}}({\rm X^{\rm q}})$ is the energy of the charged supercell with the X$^{\rm q}$ species, $E_0$ is the energy of the stoichiometric neutral supercell, $\mu_i^\text{X}$ is the chemical potential of the $i^{\rm th}$ species that was removed or added to produce the supercell with the X$^{\rm q}$ species ($\{\mu_i^\text{X}\}$ denotes the set of all such species), $n_i$ is a positive (negative) integer representing the number of the $i^{\rm th}$ species that was added (removed),  and $E_{\text{F}}$ is absolute position of the Fermi level and is treated as a parameter. The importance of the 
% term $E_{\text{corr}}({\rm q})$, which is introduced as a correction to account for a finite supercell when performing a calculation for a charged defect, and the method for calculating it has been outlined by previous authors~\cite{Vinichenko,Freysoldt2011electrostatic,Freysoldt2009fully,Kumagai,Komsa2013finite,Walsh2021}.
% To briefly motivate the importance of the calculation of $E_{\text{corr}}({\rm q})$, the use of a periodic supercell to calculate the total energy of a charged defect naturally leads to divergence of the total energy due to infinitely many uncompensated charges. In order to remedy the issue, a neutralizing background charge is applied which causes spurious terms to arise in the total energy~\cite{Vinichenko,Castleton2006managing,Komsa2012comparison,Alkauskas,Freysoldt2009fully,Freysoldt2011electrostatic,Komsa2012finite,Kumagai,Castleton2009density}. The energy, $E_{\text{corr}}({\rm q})$, is designed precisely to correct these spurious terms. 

% The thermal ionization potential is given by the ACTL divided by the elementary charge $e$. 

% The electric field resulting from a difference between the ionization potentials in the donor and acceptor parts of the sample is simply,
% \begin{align}
%     \vec{\mathcal{E}} = -\nabla\phi_{\text{thermal}} \approx -\frac{(\phi_{\text{thermal}}(\mathbf{r}_\text{D})-\phi_{\text{thermal}}(\mathbf{r}_\text{A}))}{|\Delta\mathbf{r}|}\frac{\Delta\mathbf{r}}{|\Delta\mathbf{r}|}.
% \end{align}
% Above, $\phi_{\text{thermal}}$ is the thermal ionization potential, $\mathbf{r}_\text{D}$ is the position of defect D, $\mathbf{r}_\text{A}$ is the position of defect A, and $\Delta\mathbf{r} = \mathbf{r}_\text{D} - \mathbf{r}_\text{A}$. 
\bibliography{refs_NV}
\end{document}
