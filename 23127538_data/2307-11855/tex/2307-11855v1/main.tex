%%
%% This is file `sample-sigconf.tex',
%% generated with the docstrip utility.
%%
%% The original source files were:
%%
%% samples.dtx  (with options: `sigconf')
%% 
%% IMPORTANT NOTICE:
%% 
%% For the copyright see the source file.
%% 
%% Any modified versions of this file must be renamed
%% with new filenames distinct from sample-sigconf.tex.
%% 
%% For distribution of the original source see the terms
%% for copying and modification in the file samples.dtx.
%% 
%% This generated file may be distributed as long as the
%% original source files, as listed above, are part of the
%% same distribution. (The sources need not necessarily be
%% in the same archive or directory.)
%%
%%
%% Commands for TeXCount
%TC:macro \cite [option:text,text]
%TC:macro \citep [option:text,text]
%TC:macro \citet [option:text,text]
%TC:envir table 0 1
%TC:envir table* 0 1
%TC:envir tabular [ignore] word
%TC:envir displaymath 0 word
%TC:envir math 0 word
%TC:envir comment 0 0
%%
%%
%% The first command in your LaTeX source must be the \documentclass command.
\documentclass[sigconf, anonymous=False,review=False]{acmart}

%%
%% \BibTeX command to typeset BibTeX logo in the docs
\AtBeginDocument{%
  \providecommand\BibTeX{{%
    \normalfont B\kern-0.5em{\scshape i\kern-0.25em b}\kern-0.8em\TeX}}}
\newcommand{\frank}[1]{\textbf{\textcolor{red}{Frank: #1}}}
%% Rights management information.  This information is sent to you
%% when you complete the rights form.  These commands have SAMPLE
%% values in them; it is your responsibility as an author to replace
%% the commands and values with those provided to you when you
%% complete the rights form.

%\setcopyright{acmcopyright}
%\copyrightyear{2018}
%\acmYear{2018}
%\acmDOI{10.1145/1122445.1122456}

%% These commands are for a PROCEEDINGS abstract or paper.

%\acmConference[Woodstock '18]{Woodstock '18: ACM Symposium on Neural
%  Gaze Detection}{June 03--05, 2018}{Woodstock, NY}
%\acmBooktitle{Woodstock '18: ACM Symposium on Neural Gaze Detection,
%  June 03--05, 2018, Woodstock, NY}
%\acmPrice{15.00}
%\acmISBN{978-1-4503-XXXX-X/18/06}

\copyrightyear{2023}
\acmYear{2023}
\setcopyright{acmlicensed}
\acmConference[GECCO '23]{Genetic and Evolutionary Computation Conference}{July 15--19, 2023}{Lisbon, Portugal}
\acmBooktitle{Genetic and Evolutionary Computation Conference (GECCO '23), July 15--19, 2023, Lisbon, Portugal}
\acmPrice{15.00}
\acmDOI{xxx}
\acmISBN{xxx}


% \usepackage{todonotes}
\usepackage{float}
\usepackage{physics}
\usepackage{tabularx}
    \newcolumntype{Y}{>{\centering\arraybackslash}X}
    \setlength\tabcolsep{1pt}
    \renewcommand{\arraystretch}{1.5}
    \renewcommand\tabularxcolumn[1]{m{#1}}
\usepackage{booktabs}
\usepackage{pdflscape}
\usepackage{csvsimple}
\usepackage{hyperref}[hyperfootnotes=true]
\usepackage{url}
\usepackage{nameref}
\usepackage{tablefootnote}
\usepackage{color, colortbl}
\usepackage{mathtools}
%\usepackage[algo2e, ruled, vlined, linesnumbered]{algorithm2e}
\usepackage{xparse}
\usepackage[vlined,linesnumbered]{algorithm2e}
\RestyleAlgo{ruled}
%\usepackage{algorithm}
\usepackage{makecell,booktabs}
\usepackage{graphicx}
% \usepackage{caption}
% \usepackage{subcaption}
% \captionsetup[figure]{font=small,labelfont=small}
\newcommand{\ignore}[1]{}

\usepackage
[
    noabbrev,   % The words in front of the labels are not abbreviated.
    nameinlink, % Extends the link of a reference to the word in front of it.
]
{cleveref} % This package must be included after ›hyperref‹. It creates clever references that know what they refer to.
\newcommand{\refe}[1]{(\ref{#1})}
\newcommand{\reft}[1]{Theorem \ref{#1}}
\newcommand{\refl}[1]{Lemma \ref{#1}}
\newcommand{\refd}[1]{Definition \ref{#1}}
\newcommand{\refi}[1]{\ref{#1}}

\newcommand{\AND}{{\mbox{ }\wedge\mbox{ }}}
\newcommand{\OR}{{\mbox{ }\vee\mbox{ }}}
\newcommand{\IMP}{{\mbox{ }\rightarrow\mbox{ }}}
\newcommand{\AEQ}{{\mbox{ }\Leftrightarrow\mbox{ }}}
\newcommand{\EQV}{{\mbox{ }\equiv\mbox{ }}}
%\newcommand{\abs}[1]{\lvert#1\rvert}

\DeclareDocumentCommand{\set}{m g o}{
    \ensuremath{
        \IfNoValueTF{#3}{\left}{#3}\{#1
            \IfNoValueTF{#2}{}{
                \ \IfNoValueTF{#3}{\left}{#3}\vert\ \vphantom{#1}#2\IfNoValueTF{#3}{\right.}{}
            } \IfNoValueTF{#3}{\right}{#3}\}
    }\xspace
}

\pdfstringdefDisableCommands{
    \def\\{}
    \def\textnumero{No.}
    \def\todo{}
}

\definecolor{lightgreen}{rgb}{0.75,0.92,0.61}

\newcommand\numberthis{\addtocounter{equation}{1}\tag{\theequation}}
\newcommand{\floor}[1]{\left\lfloor #1 \right\rfloor}
\newcommand{\ceil}[1]{\left\lceil #1 \right\rceil}

\newcommand{\labelname}[1]{
  \def\@currentlabelname{#1}}

\newcommand{\N}{\mathbb{N}}
\newcommand{\R}{\mathbb{R}}
\newcommand{\Z}{\mathbb{Z}}


\newcommand{\ones}[1]{|#1|_1}
\newcommand{\zeros}[1]{|#1|_0}
%\newcommand{\strings}{\set{0,1}^n}

\newcommand{\pflip}{p_{\text{flip}}}

\newcommand{\Bin}{\text{Bin}}

\newcommand{\Tcomment}[1]{\textcolor{red}{\textbf{[TK: #1]}}}
\newcommand{\Acomment}[1]{\textcolor{blue}{\textbf{[AR: #1]}}}
\newcommand{\Jcomment}[1]{\textcolor{green}{\textbf{[JG: #1]}}}
\newcommand{\Scomment}[1]{\textcolor{magenta}{\textbf{[SL: #1]}}}
\newcommand{\Todo}[1]{\textcolor{magenta}{[\textbf{TODO}: #1]}}
\providecommand{\ignore}[1]{} 

\newcommand{\ooea}{(1+1) EA\xspace}
\newcommand{\rlswself}{$RLS_{\alpha,\beta}$\xspace}
\newcommand{\muoea}{($\mu$+1) EA\xspace}
\newcommand{\cGA}{cGA\xspace}
\DeclareMathOperator{\erfc}{erfc}

\newcommand{\om}{\textsc{OM}_{\sigma^2}\xspace}
\newcommand{\OneMax}{\textsc{OneMax}\xspace}
\newcommand*{\LO}{\mbox{\textsc{LeadingOnes}}\xspace}
\newcommand{\BV}{\mbox{\textsc{BinVal}}\xspace}
\newcommand{\Var}{\mathrm{Var}}
\newcommand{\Ber}{\mathrm{Ber}}

\DeclareMathOperator{\Geo}{Geo}

% \DeclareMathOperator{\x1}{|x|_1}


\newcommand{\normal}[1]{\mathcal{N}(#1)}
%\newcommand{\set}[2]{\left\{#1 \mid #2\right\}}
\newcommand{\eqnComment}[2]{\underset{{\scriptstyle \text{#1}}}{#2}}


\newcommand{\realnum}{\mathbb{R}}
\newcommand{\natnum}{\mathbb{N}}
\newcommand{\bitstring}{\{0,1\}^n}
\newcommand{\intstring}{\mathbb{Z}^n}

\newtheorem{thm}{Theorem}
\newtheorem{lem}[thm]{Lemma}
\newtheorem{prop}[thm]{Proposition}
\newtheorem{definition}[thm]{Definition}
\newtheorem{cor}[thm]{Corollary}

\newenvironment{myAlgorithm}%
	{
	%\vspace*{1em}
	\begin{center}
	\begin{algorithm2e}
	%\setstretch{1.33}
	}%
		% there goes the algorithm including labels and captions
	{
	\end{algorithm2e}
	\end{center}
	%\vspace*{1em}
	}
	
\allowdisplaybreaks
%%
%% Submission ID.
%% Use this when submitting an article to a sponsored event. You'll
%% receive a unique submission ID from the organizers
%% of the event, and this ID should be used as the parameter to this command.
%%\acmSubmissionID{123-A56-BU3}

%%
%% The majority of ACM publications use numbered citations and
%% references.  The command \citestyle{authoryear} switches to the
%% "author year" style.
%%
%% If you are preparing content for an event
%% sponsored by ACM SIGGRAPH, you must use the "author year" style of
%% citations and references.
%% Uncommenting
%% the next command will enable that style.
%%\citestyle{acmauthoryear}

%%
%% end of the preamble, start of the body of the document source.
\allowdisplaybreaks

\begin{document}

%%
%% The "title" command has an optional parameter,
%% allowing the author to define a "short title" to be used in page headers.
\title{Run Time Bounds for Integer-Valued OneMax Functions}

\author{Jonathan Gadea Harder}
\email{ jonathan.gadeaharder@hpi.de}
% \orcid{1234-5678-9012}
\affiliation{%
  \institution{Hasso Plattner Institute\\ University of Potsdam}
  \city{Potsdam}
  \country{Germany}
}

\author{Timo Kötzing}
\email{timo.koetzing@hpi.de}
% \orcid{1234-5678-9012}
\affiliation{%
  \institution{Hasso Plattner Institute\\ University of Potsdam}
  \city{Potsdam}
  \country{Germany}
}

\author{Xiaoyue Li}
\email{xiaoyue.li@hpi.de}
% \orcid{1234-5678-9012}
\affiliation{%
  \institution{Hasso Plattner Institute\\ University of Potsdam}
  \city{Potsdam}
  \country{Germany}
}


\author{Aishwarya Radhakrishnan}
\email{aishwarya.radhakrishnan@hpi.de}
% \orcid{1234-5678-9012}
\affiliation{%
  \institution{Hasso Plattner Institute\\ University of Potsdam}
  \city{Potsdam}
  \country{Germany}
}

%%
%% The "author" command and its associated commands are used to define
%% the authors and their affiliations.
%% Of note is the shared affiliation of the first two authors, and the
%% "authornote" and "authornotemark" commands
%% used to denote shared contribution to the research.
% \author{Ben Trovato}
% \authornote{Both authors contributed equally to this research.}
% \email{trovato@corporation.com}
% \orcid{1234-5678-9012}
% \author{G.K.M. Tobin}
% \authornotemark[1]
% \email{webmaster@marysville-ohio.com}
% \affiliation{%
%   \institution{Institute for Clarity in Documentation}
%   \streetaddress{P.O. Box 1212}
%   \city{Dublin}
%   \state{Ohio}
%   \country{USA}
%   \postcode{43017-6221}
% }

% \author{Lars Th{\o}rv{\"a}ld}
% \affiliation{%
%   \institution{The Th{\o}rv{\"a}ld Group}
%   \streetaddress{1 Th{\o}rv{\"a}ld Circle}
%   \city{Hekla}
%   \country{Iceland}}
% \email{larst@affiliation.org}

% \author{Valerie B\'eranger}
% \affiliation{%
%   \institution{Inria Paris-Rocquencourt}
%   \city{Rocquencourt}
%   \country{France}
% }

% \author{Aparna Patel}
% \affiliation{%
%  \institution{Rajiv Gandhi University}
%  \streetaddress{Rono-Hills}
%  \city{Doimukh}
%  \state{Arunachal Pradesh}
%  \country{India}}

% \author{Huifen Chan}
% \affiliation{%
%   \institution{Tsinghua University}
%   \streetaddress{30 Shuangqing Rd}
%   \city{Haidian Qu}
%   \state{Beijing Shi}
%   \country{China}}

% \author{Charles Palmer}
% \affiliation{%
%   \institution{Palmer Research Laboratories}
%   \streetaddress{8600 Datapoint Drive}
%   \city{San Antonio}
%   \state{Texas}
%   \country{USA}
%   \postcode{78229}}
% \email{cpalmer@prl.com}

% \author{John Smith}
% \affiliation{%
%   \institution{The Th{\o}rv{\"a}ld Group}
%   \streetaddress{1 Th{\o}rv{\"a}ld Circle}
%   \city{Hekla}
%   \country{Iceland}}
% \email{jsmith@affiliation.org}

% \author{Julius P. Kumquat}
% \affiliation{%
%   \institution{The Kumquat Consortium}
%   \city{New York}
%   \country{USA}}
% \email{jpkumquat@consortium.net}

%%
%% By default, the full list of authors will be used in the page
%% headers. Often, this list is too long, and will overlap
%% other information printed in the page headers. This command allows
%% the author to define a more concise list
%% of authors' names for this purpose.
\renewcommand{\shortauthors}{ Kötzing, et al.}
%%
%% The abstract is a short summary of the work to be presented in the
%% article.
\begin{abstract}
While most theoretical run time analyses of discrete randomized search heuristics focused on finite search spaces, we consider the search space $\Z^n$. This is a further generalization of the search space of multi-valued decision variables $\{0,\ldots,r-1\}^n$.

We consider as fitness functions the distance to the (unique) non-zero optimum $a$ (based on the $L_1$-metric) and the \ooea which mutates by applying a step-operator on each component that is determined to be varied. For changing by $\pm 1$, we show that the expected optimization time is $\Theta(n \cdot (|a|_{\infty} + \log(|a|_H)))$. In particular, the time is linear in the maximum value of the optimum $a$. Employing a different step operator which chooses a step size from a distribution so heavy-tailed that the expectation is infinite, we get an optimization time of $O(n \cdot \log^2 (|a|_1) \cdot \left(\log (\log (|a|_1))\right)^{1 + \epsilon})$. 

Furthermore, we show that RLS with step size adaptation achieves an optimization time of $\Theta(n \cdot \log(|a|_1))$.

We conclude with an empirical analysis, comparing the above algorithms also with a variant of CMA-ES for discrete search spaces.
\end{abstract}

%%
%% The code below is generated by the tool at http://dl.acm.org/ccs.cfm.
%% Please copy and paste the code instead of the example below.
%%
\ignore{
\begin{CCSXML}
<ccs2012>
 <concept>
  <concept_id>10010520.10010553.10010562</concept_id>
  <concept_desc>Computer systems organization~Embedded systems</concept_desc>
  <concept_significance>500</concept_significance>
 </concept>
 <concept>
  <concept_id>10010520.10010575.10010755</concept_id>
  <concept_desc>Computer systems organization~Redundancy</concept_desc>
  <concept_significance>300</concept_significance>
 </concept>
 <concept>
  <concept_id>10010520.10010553.10010554</concept_id>
  <concept_desc>Computer systems organization~Robotics</concept_desc>
  <concept_significance>100</concept_significance>
 </concept>
 <concept>
  <concept_id>10003033.10003083.10003095</concept_id>
  <concept_desc>Networks~Network reliability</concept_desc>
  <concept_significance>100</concept_significance>
 </concept>
</ccs2012>
\end{CCSXML}

\ccsdesc[500]{Computer systems organization~Embedded systems}
\ccsdesc[300]{Computer systems organization~Redundancy}
\ccsdesc{Computer systems organization~Robotics}
\ccsdesc[100]{Networks~Network reliability}
}
%%
%% Keywords. The author(s) should pick words that accurately describe
%% the work being presented. Separate the keywords with commas.
\keywords{Evolutionary algorithms, integer optimization, run time analysis, theory.}

%% A "teaser" image appears between the author and affiliation
%% information and the body of the document, and typically spans the
%% page.
% \begin{teaserfigure}
%   % Figure removed
%   \caption{Seattle Mariners at Spring Training, 2010.}
%   \Description{Enjoying the baseball game from the third-base
%   seats. Ichiro Suzuki preparing to bat.}
%   \label{fig:teaser}
% \end{teaserfigure}

%%
%% This command processes the author and affiliation and title
%% information and builds the first part of the formatted document.

\maketitle

%\Tcomment{TODO for the final version: add acknowledgment of DFG as in GECCO papers}
%\Scomment{Added. But not appears after compiling}



\section{Introduction}

Optimization problems are formalized as finding the optimal element $x$ from a fixed search space $\mathcal{X}$ given some quality measure $f: \mathcal{X} \rightarrow \realnum$. In the theory of evolutionary search heuristics, the most commonly studied discrete search space is $\mathcal{X} = \{0,1\}^n$, the set of bit strings of fixed length. Other search spaces have been considered, such as permutations \cite{frank21, doerr22}, or multi-valued decision variables \cite{gunter94,timo2017} ($\mathcal{X} = \{0,\ldots,r-1\}^n$). Note that all these search spaces are finite.

In this work we are interested in the infinite (but still discrete) search space $\Z^n$. This models a set of $n$ decision variables with an infinite (totally and discretely ordered) domain. While many search problems can be usefully addressed by translating them into optimization problems using $\mathcal{X} = \{0,1\}^n$ or another of the before-mentioned search spaces, this is impossible in principle for an infinite search space. Furthermore, understanding them in their more natural formulation can lead to more efficient optimization algorithms.

In order to analyze heuristic search in this domain, we generalize fitness functions as well as heuristic search algorithms accordingly. Note that a generalization of the $\{0,1\}^n$ search space to $\{0,\ldots,r-1\}^n$ was done in \cite{timo2017}, and we follow a similar path in the generalization, but now with the added difficulty of an infinite search space.

As a first analysis, we consider the simple setting where, for a given $a \in \Z^n$, we have the fitness function
$$
f_a : \Z^n \rightarrow \Z_{\geq 0}, x \mapsto \sum_{i=1}^n |x_i-a_i|.
$$
Minimizing this function generalizes the well-known OneMax function (class), defined on the search space $\{0,1\}^n$, to the more general space of $\Z^n$.

As algorithms, we consider the \ooea and RLS (Random Local Search), both suitably adjusted to deal with the search space $\Z^n$ as follows (see Section~\ref{sec:prelims} for a detailed description of both algorithms). First, for finite search spaces, it is common to start the search with a uniformly random search point. For the infinite search space $\Z^n$ we make the decision to start deterministically with the all-$0$ string. Since for our fitness functions only the position-wise differences of the starting point to the optimum matter, this choice does not restrict the meaningfulness of our results.

Second, as a variation operator, we consider to change either exactly one position (RLS) or each position independently with probability $1/n$ (\ooea), just as in the common definitions of these algorithms. However, while changing a \emph{bit} from $\{0,1\}$ leaves only one possible choice for the new value, changing a variable from $\Z$ leaves infinitely many choices for the new value. We consider two possible \emph{step operators}, defining how to change a value from $\Z$. The first step operator is the \emph{$\pm 1$ operator}; it either increases or decreases the value by $1$, decided uniformly at random. The second step operator is the \emph{heavy-tailed operator}; it makes a uniform decision to either increase or decrease, but instead of deterministically changing by 1, it changes by a random number. In particular, we consider a distribution of the numbers that is unbounded and is so heavy-tailed that it does not have finite expectation. 

Note that \cite{timo2017} considered a \emph{Harmonic operator} as a step operator for variables on $\{0,\ldots,r-1\}$, which gives a step of size $i$ a weight proportional to $1/i$. This cannot be directly extended to an operator on $\Z$, since the sequence $(1/i)_i$ is not summable. Instead, we take inspiration from \cite{unknown_solution_length}, where very slowly decreasing yet summable sequences were considered, to define our heavy-tailed operator. In effect, a step size of $i$ has a probability proportional to $1/(i \log(i)^{i+\varepsilon})$. The goal of this very heavy tail of the distribution is to have as high as possible a chance to gain a constant fraction of the distance to the optimum, independent of the distance to the optimum. See Section~\ref{sec:prelims} for details on the distribution.

Recently, heavy tailed distributions where used to speed up optimization in various settings. In \cite{fastMutator2017}, the authors proposed to apply a heavy-tailed mutation operator for the first time. In particular, the number of variables to change was chosen from a heavy-tailed distribution (which for us still follows the traditional binomial distribution). In contrast to our work, the heavy-tailed distributions in \cite{fastMutator2017} have finite expectation. This more explorative mutation operator was then shown to optimize so-called jump-functions faster. Since this work, further analyses have shown the use of such heavy-tailed distributions, for example for crossover and the $(1+(\lambda,\lambda))$-GA on OneMax \cite{fastMutatorCrossover} and jump-functions \cite{BenjaminHeavy-TailedGA}.


In our work we consider the expected time of the given algorithms to find the optimum of a fitness function $f_a$. This time naturally depends on $a$ (as well as $n$), specifically on its total weight $|a|_1 = \sum_{i=1}^n |a_i|$, its maximal weight $|a|_{\infty} = \max\set{|a_i|}{i \leq n}$ or its Hamming distance to the all-$0$ string $|a|_H = |\set{i}{a_i \neq 0, i \leq n}|$. Through out our analyses we consider $a$ to be non-zero.

In Section~\ref{sec:unit_mutation_operator} we formally analyze the \ooea with the $\pm 1$ operator. Theorems~\ref{thm:better_upperbound_pm_ operator} and~\ref{thm:better_lowerbound_pm_operator} show that the expected optimization time is $\Theta(n \cdot (|a|_{\infty} + \log(|a|_H)))$ on any given fitness function $f_a$. While the linear dependence on the dimension $n$ constitutes a good performance, the linear dependence on $|a|_{\infty}$ is rather slow. For comparison, note that there are $\Theta(|a|_{\infty}^n)$ many target bit strings of roughly that size. Since the \ooea gains about one bit of information with a comparison of two fitness values, a direct information theoretic lower bound for finding the optimum is at $\Omega(n \cdot \log (|a|_{\infty}))$.

In Section~\ref{sec:heavy_tailed_operator} we turn to the \ooea with the heavy-tailed operator. In Theorem~\ref{thm:heavy_tailed_operator} we show that the expected optimization time is $O(n \cdot \log^2 (|a|_1) \cdot \left(\log (\log (|a|_1))\right)^{1 + \epsilon})$ for a given $f_a$. This is already much closer to the information-theoretic lower bound mentioned above. 

Finally, we consider a version of RLS which adapts the step size during the search. This strategy was proven to be very efficient in \cite{timo2017} for the search space $\{0,\ldots,r-1\}$ and we show that also here the algorithm achieves an expected optimization time of $\Theta(n \cdot \log (|a|_1))$. Note that, for all three operators, we derive central parts of our proofs by carefully adjusting analogous proofs from~\cite{timo2017}.

With our analyses of the search space $\Z^n$, we also aim at bridging the gap towards analyses of heuristic search on $\R^n$ (continuous optimization): if one is interested in approximating the optimum up to a distance of $\varepsilon$, one can discretize the search space accordingly, arriving at the search space $\Z^n$. In optimization problems on $\R^n$, one is frequently interested only in approximating the optimum, since finding it is typically impossible in principle. In Corollary~\ref{cor:heavy_tailed_operator} we show a result about the \ooea with heavy-tailed operator approximating the optimum, showing that finding an approximation ratio of $\alpha$ scales with respect to $\alpha$ as $O(\log(1/\alpha))$.

For $\R^n$, the covariance matrix adaptation evolution strategy (CMA-ES) is a widely used optimization algorithm \cite{CMA-ES}. To extend the use of CMA-ES to address also discrete variables, authors in \cite{CMAESwMargin} propose a variant of CMA-ES (called CMA-ES with margin, CMA-ESwM). We are interested in the performance of this algorithm in optimizing an integer-valued problem in comparison to other discrete optimization algorithms. In Section~\ref{sec:experiments} we experimentally compare our algorithms with CMA-ESwM, finding that the CMA-ESwM fails to optimize with certain probability even with a large time budget, while the \ooea with the heavy-tailed operator and RLS with the self-adjusting operator can handle the instances efficiently.

The remainder of this paper is structured as follows. In Section~\ref{sec:prelims} we introduce algorithms and notation. In Sections~\ref{sec:unit_mutation_operator} and~\ref{sec:heavy_tailed_operator} we give our theoretical analyses of the \ooea. Section~\ref{sec:self_adjusting_rls} addresses the self-adjusting RLS. In Section~\ref{sec:experiments} we present our experiments. 








\section{Preliminaries} \label{sec:prelims}

In this section we define the \ooea and random local search algorithms, along with the different step operators. At the end of this section we also state different drift theorems we use for our analysis.

For any $a \in \intstring$, we let 
\begin{align*}
|a|_1 &= \sum_{i = 1}^{n} |a_i|;\\
|a|_{\infty} &= \max\set{|a_i|}{i \leq n};\\
|a|_H &= |\set{i}{a_i \neq 0, i \leq n}|.
\end{align*}
Also for $a \in \intstring \setminus \{0^n\}$, we define 
$$
f_a: \intstring \rightarrow \realnum, x  \mapsto |a - x|_1.
$$
Our class of target fitness functions is $\mathscr{F} = \{f_a \mid a \in \intstring \setminus \{0^n\} \}$ to be minimized by evaluating the fitness function at any point as chosen by the algorithm.


We consider different \emph{step operators} $\mathrm{step}: \Z\mapsto \Z$. These step operators decide on the update of a mutation in a given component. We consider the following step operators.



\begin{definition}\label{defn:pm_operator}
[$\pm 1$-operator] The \emph{$\pm 1$-operator} takes an integer $x$ as input and makes the following changes: With probability $1/2$ return $x+1$ and otherwise return $x-1$.
\end{definition}




For the next operator we need a definition. Given $\epsilon > 0$, let $ c_{\epsilon} = \sum_{i = 2}^{\infty} \frac{1}{i \cdot (\log i)^{1+\epsilon}}$ (note that this sum is finite \cite{unknown_solution_length}).

\begin{definition}\label{defn:heavy_tailed_operator}
[Heavy-tailed-operator] For a given $\varepsilon > 0$, the \emph{heavy-tailed operator} takes an integer $x$ as input and makes the following changes: First sample a step size of $2^{I - 2}$ using a random variable $I$ which can take a value of any natural number $i \geq 2$ with $P(I=i) = \frac{1}{c_{\epsilon} \cdot i \cdot (\log i)^{1+\epsilon}}$. With probability $\frac{1}{2}$ then return $x+2^{I - 2}$, otherwise return $x-2^{I - 2}$.
\end{definition}



We consider the algorithms RLS and the \ooea as given by \Cref{1+1 ea} and \Cref{rls}. Both start from the initial search point being the all-$0$ string. They then proceed in rounds, each of which consists of a \emph{mutation} and a \emph{selection} step. Throughout the whole optimization process, the algorithms maintain a single individual, which is always the most recently sampled best-so-far solution. The two algorithms differ only in the \emph{mutation operator}. While $RLS$ makes a step in exactly one position (chosen uniformly at random), the \ooea makes, in each position, a step with probability $1/n$. We specify the termination criterion as the point in time when the search point has a fitness of $0$.


\SetKw{KwFrom}{from}

\begin{algorithm}
\textbf{Initialization:} $x\gets 0^n$\;
 \textbf{Optimization:} \While{$f(x) \neq 0$}{
 \For{i \KwFrom 1 \KwTo n}{
 With probability $\frac{1}{n}$ set $y_i=\text{step}(x_i)$ and set\\ $y_i=x_i$ otherwise;
 }
\If{$f(y)\leq f(x)$}{
   $x \gets y$;
   }

 }
 \caption{The \ooea minimizing a function $f:\mathbb{Z}^n \mapsto\mathbb{R}$ with a given step operator $\mathrm{step}:\Z \rightarrow \Z$}
 \label{1+1 ea}
\end{algorithm}


$RLS_{\alpha,\beta}$ maintains a search point $x\in \mathbb{Z}^n$ as well as a real-valued \emph{velocity vector} $v \in [1,\infty]^n$; we use real values for the velocity to circumvent rounding problems. The initial search point is the all-$0$s string and the initial velocity is the all-$1$s string. In one iteration of the algorithm a position $i\in [n]$ is chosen uniformly at random. The entry $x_i$ is replaced by $x_i-\floor{v_i}$ with probability $1/2$ and by $x_i+\floor{v_i}$ otherwise. The entries in positions $j\neq i$ are not subject to mutation. The resulting string $y$ replaces $x$ if its fitness is at least as good as the one of $x$, i.e.\ if $f(y)\leq f(x)$ holds. If the offspring $y$ is strictly better than its parent $x$, i.e.\ if $f(y) < f(x)$, we increase the velocity $v_i$ in the $i$-th component by multiplying it with the constant $\alpha$; we decrease $v_i$ to $\beta v_i$ otherwise. The algorithm proceeds this way until we decide to stop it. To further lighten the notation, we say that the algorithm ``\emph{moves in the right direction}'' or ``\emph{towards the target value}'' if the distance to the target is actually decreased by $\floor{v_i}$. Analogously, we speak otherwise of a step ``\emph{away from the target}'' or ``\emph{in the wrong direction}''.




\begin{algorithm}
\textbf{Initialization:} $x\gets 0^n$, $v \gets 1^n$\;
 \textbf{Optimization:} \While{$f(x) \neq 0$}{
 \For{i \KwFrom 1 \KwTo n}{
 $y\gets x$;\\
 Choose $i\in [n]$ uniformly at random;\\
 With probability $\frac{1}{2}$ set $y_i=x_i-\floor{v_i}$ and set $y_i=x_i+\floor{v_i}$ otherwise;
 }
 \uIf{$f(y)<f(x)$}{
 $v_i \gets \alpha v_i$;
 }\Else{
 $v_i\gets \max\{1,\beta v_i\}$;
 }
\If{$f(y)\leq f(x)$}{
   $x \gets y$;
   }
 }
 \caption{$RLS_{\alpha,\beta}$ with self-adjusting step sizes minimizing a function $f:\mathbb{Z}^n\mapsto \mathbb{R}$}
 \label{rls}
\end{algorithm}






A central tool in many of our proofs is drift analysis, which comprises a number of tools to derive bounds on hitting times from bounds on the expected progress a process makes towards the target. Drift analysis is currently the most powerful tool in run time analysis for evolutionary computation. We briefly collect here the tools we use.


\emph{Additive drift} is the situation that the progress is bounded by any (constant) value. This quite common situation in run time analysis was the first framed into a drift theorem, namely the following one, in \cite{driftthm}.


% \Acomment{Timo, Additive drift theorem is not the actual version in the paper mentioned as reference, should we change it to the one in the paper\cite{driftthm}?} \Tcomment{if its just a modern formulation, just ignore; if you took the formulation from somewhere else, give credit to that other source as well. The drift theorem is incorrect as it is, since the process may be negative}

\begin{thm}[Additive Drift Theorem \cite{driftthm}] \label{thm:additivedrift}
Let $S \subseteq \realnum$ be a finite set of positive numbers and let $({X^t})_{t\in \natnum}$ be a sequence of random variables over $S \cup \{0\}$. Let $T$ be the random variable that denotes the first point in time $t \in \natnum$ for which $X^t \leq 0$.
Suppose that there exists a constant $\delta_1 > 0$ such that

$$ E[X^{t} - X^{t+1} \mid T > t] \geq \delta_1 $$
holds. Then

$$E[T \mid X^0] \leq \frac{X^0}{\delta_1}.$$

If there exists a constant $\delta_2 > 0$ such that

$$ E[X^{t} - X^{t+1} \mid T > t] \leq \delta_2 $$
holds. Then

$$E[T \mid X^0] \geq \frac{X^0}{\delta_2}.$$

\end{thm}

\emph{Multiplicative drift} (first) addresses the situation where progress is proportional to the distance from the target. For this situation quite common in run time analysis we can use the following drift theorem~\cite{multiplicative-drift}.



\begin{thm}[Multiplicative Drift Theorem \cite{firsthit}] \label{thm:multiplicativedrift}
Let $(X_t)_{t \in \natnum}$ be random variables over $\realnum$, $x_{\min} >0$, and let $T = \min\{t \mid X_t < x_{\min}\}$. Furthermore, suppose that\\

(a) $X_0 \geq x_{\min}$ and, for all $t \leq T$, it holds that $X_t \geq 0$, and that

(b) there is some value $\delta >0$ such that, for all $t < T$, it holds that $X_t - E[X_{t+1} \mid X_0, \ldots, X_t] \geq \delta X_t$.\\

Then

$$ E[T \mid X_0] \leq \frac{1+\ln\left(\frac{X_0}{x_{\min}}\right)}{\delta}.$$

% Let $S \subseteq \realnum$ be a finite set of positive numbers with minimum $s_{min}$. Let $({X^t})_{t\in \natnum}$ be a sequence of random variables over $S \cup \{0\}$. Let $T$ be the random variable that denotes the first point in time $t \in \natnum$ for which $X^t < 1$.
% Suppose that there exists a real number $\delta > 0$ such that

% $$ E[X^{t} - X^{t+1} \mid X^{t} = s] \geq \delta s $$
% holds for all $s \in S$ with $Pr[X^t = s] > 0$. Then for all $s_0 \in S$ with $Pr[X^{0} = s_0] > 0$, we have

% $$E[T \mid X^0 = s_0] \leq \frac{1 + \ln(s_0/s_{min})}{\delta}.$$
\end{thm}
% \Tcomment{Twice you write $X(?)$ instead of using a superscript}
% \Tcomment{$ln$ should be $\ln$.}

% Any logarithm for which a base is not explicitly mentioned is the logarithm with base 2.\Tcomment{But $\ln$ is base $e$, no?}


\section{Unit Mutation Strength} \label{sec:unit_mutation_operator}
In this section, we regard the mutation operator that applies only $\pm 1$ changes to each component. We give a tight bound on the expected run time of the \ooea with the $\pm$ operator. We start by stating the following lemma which bounds the expected cost of a given random variable over $[n]$ for $n \in \natnum$, which we later use in our analysis.


\begin{lem}\cite[Lemma 13]{timo2017} \label{cost_bound}
Let $n \in \natnum$ be fixed, let $q$ be a cost function on elements of $[n]$ and let $c$ be a cost function on subsets of $[n]$. Furthermore, let a random variable $S$ ranging over subsets of $[n]$ be given. Then we have

\begin{align}
    \forall T\subseteq [n]: c(T)\leq \sum_{i\in T} q(i)\implies E[c(S)]\leq \sum_{i=1}^n q(i)P(i\in S)\end{align}
    and \begin{align}
    \forall T\subseteq [n]:c(T)\geq \sum_{i\in T} q(i)\implies E[c(S)]\geq \sum_{i=1}^n q(i)P(i\in S)
\end{align}
\end{lem}

The following theorem gives an upper bound on the expected optimization time of the $\ooea$ optimizing any $f_a \in \mathscr{F}$ with the $\pm 1$ operator.


\begin{thm} \label{thm:better_upperbound_pm_ operator}
     Let $f_a \in \mathscr{F}$. Then the expected optimization time of the \ooea with $\pm 1$ operator starting with all $0$ integer string on $f_a$ is $O(n\cdot(|a|_{\infty} +  \log(|a|_H)))$.
\end{thm}



\begin{proof}
Our proof idea is similar to the proof idea in \cite[Theorem 12]{timo2017}. We  will use the multiplicative drift analysis (see \Cref{thm:multiplicativedrift}) to prove this theorem. For our analysis, we make use of two edge cases that both have a fitness of $n$; which would be only one entry being incorrect but $n$ away from its target and the case that every entry in the target is $1$. The former is hard to optimize since it takes long for the algorithm to successively change the incorrect position, while the latter can be solved very fast since there are multiple ways of progressing in each step. We exploit this by giving each position a weight exponential in the amount that is incorrect, and then sum over those weights.
With any search point $x \in\Omega$ we associate a vector $d \in \mathrm{R}^n$ such that, for all $i \leq n, d_i = |a_i - x_i|$.
Given some $\omega>1$ we consider the potential 
\begin{align}
    g(x)\coloneqq \sum_{i=1}^{n} (\omega^{d_i}-1)
\end{align}



Let $x_t$ be the integer string at iteration $t$ when the \ooea with $\pm 1$ operator is optimizing $f_a$. Let $X_t = g(x_t)$ and let $T = \min\{t \geq 0 \mid X_t = 0\}$. Further let $E_1$ be the event that $X_{t+1}$ is obtained by mutating exactly one position and let $E_2$ be the event that $X_{t+1}$ is obtained by mutating at least two positions. Then $X_0 \leq |a|_H\cdot  \omega^{|a|_{\infty}}$, since we start with all $0$ integer string.
We denote $O$ as the set of already optimized positions and $A(S)$ as the set of accepted offspring by only manipulating $S\subseteq [n]$ positions.
Now we bound the expected drift. Since $t < T$, at least one of the position is at least $1$ distant far from the optimum and the probability to mutate this position in the right direction is $\frac{1}{2n}$ and the probability to not mutate any other positions is $\left(1 - \frac{1}{n}\right)^{n - 1}$. Therefore,
\begin{align*}
E[X_{t} - X_{t+1}& \mid t < T, E_1]\cdot P(E_1)\\ &\geq \sum_{i\in [n]\setminus O} \frac{1}{2n}\left(1 - \frac{1}{n}\right)^{n - 1} (\omega^{d_i}-1)-(\omega^{d_i-1}-1) \\
&\geq \frac{1}{2ne}\sum_{i\in [n]\setminus O} (\omega^{d_i}-1)-(\omega^{d_i-1}-1)\\
&= \frac{1}{2ne}\sum_{i\in [n]\setminus O}^{n} \left(1-\frac{1}{\omega}\right) \cdot \omega^{d_i}\\
&\geq \frac{1}{2ne}\sum_{i=1}^{n} \left(1-\frac{1}{\omega}\right) \cdot \left(\omega^{d_i}-1\right)\\
&=  \frac{\omega-1}{2\omega ne}\sum_{i=1}^{n} (\omega^{d_i}-1)
\end{align*}
Let $U$ be the set of all tuples $(y,i),y\in A(S)$ where the $i$-th positions worsens. As we only consider accepted mutations, we have that, for all $y \in A(S), \sum_{i\in S} d(y_i , z_i) - d_i \leq 0$.This implies that there are at least as many improving-pairs as there are worsening-pairs in $A(S) \cross S$.  For every $(y,i)$ with $d_i=0$ where bit position $i$ changes in the wrong direction, there is a $(y',i)\in U$ with bit position $i$ changing in the right direction and the remaining positions behaving the same. The change in potential for both pairs added is $$\omega^{d_i+1}-\omega^{d_i}+\omega^{d_i-1}-\omega^{d_i}= \omega^{d_i}\frac{(\omega-1)^2}{\omega}$$
Since there are in total at least as many improving-pairs as worsening-pairs, we can further map injectively each $y\in A(S)$ with a correct position ($d_i=0$) that changes in the wrong direction to another $y' \in A(S)$ with some improving position. The change in potential for both positions added is $$\omega-1+\omega^{d_{i'}-1}-\omega^{d_{i
'}}= (1-\omega^{d_i'-1})\cdot (\omega-1)< \omega^{d_i}\frac{(\omega-1)^2}{\omega}$$

 Let $Y$ be the random variable describing the search point after one cycle of mutation and selection. The random variable Y is completely determined by choosing a set $S \subseteq [n]$ of bit positions to change in $x$ and then, for each such position $i \in S$, choosing whether to change towards or away from the target. For each possible $S \subseteq [n]$, let $Y(S)$ be the random variable $Y$ conditional on making changes exactly at the bit positions of S. Note that
since we increase/decrease each index by $1$ with the same probability, $Y(S)$ is the uniform distribution on $A(S)$. Further let \begin{align*}
    c(S) &\coloneq E[g(Y(S))-g(x)] \\
    &=\frac{1}{|A(S)|}\sum_{y\in A(S)} g(y)-g(x)\\
    &= \frac{1}{|A(S)|}\sum_{y\in A(S)}\sum_{i=1}^n \left(\omega^{|y_i-a_i|}-\omega^{d_i}\right)\\
    &= \frac{1}{|A(S)|}\sum_{(y,i)\in U} \left(\omega^{|y_i-a_i|}-\omega^{d_i}+\omega^{|y_{i'}-a_{i'}|}-\omega^{d_{i'}}\right)\\
    &\leq \frac{1}{2}\sum_{i\in S} \omega^{d_i}\frac{(\omega-1)^2}{\omega}.
    \end{align*}

    Using Lemma \ref{cost_bound}, we get that
    $$E[g(Y)-g(x)|E_2]\leq \sum_{i=1}^n \frac{1}{n}\omega^{d_i}\frac{(\omega-1)^2}{2\omega}=\frac{(\omega-1)^2}{2\omega n}\sum_{i=1}^n \omega^{d_i}$$
We can use any $\omega>1$ such that $\omega-1-e(\omega-1)^2 >0$ and set $c=(\omega-1-e(\omega-1)^2)/e$. One can verify that for $\omega=1.2$ we obtain $0.0912687>0$, so such $\omega$ exist. In total we get
\begin{align*}
    E[g(x)-g(Y)]\geq \frac{\omega-1}{2\omega ne}\sum_{i=1}^n \omega^{d_i}-\frac{(\omega-1)^2}{2\omega n}\sum_{i=1}^n\omega^{d_i}\\
    =\frac{\omega-1-e(\omega-1)^2}{2\omega n e}\sum_{i=1}^n \omega^{d_i}=\frac{c}{2\omega n}\sum_{i=1}^n\omega^{d_i}\geq \frac{c}{2\omega n} g(x).
\end{align*}
Therefore by the multiplicative drift theorem (\Cref{thm:multiplicativedrift}), we have
\[E[T]  = O((|a|_{\infty}+ \log(|a|_H))\cdot n)= O(|a|_{\infty} \cdot n + n\cdot \log(|a|_H)).\]
\end{proof}


We show in the following theorem that this upper bound on the \ooea is sharp by proving the same asymptotic lower bound.
\begin{thm} \label{thm:better_lowerbound_pm_operator}
Let $f_a \in \mathscr{F}$. Then the expected optimization time of the \ooea with $\pm 1$ operator starting with all $0$ integer string on $f_a$ is $\Omega(|a|_{\infty} \cdot n + n\cdot \log(|a|_H))$.
\end{thm}



\begin{proof}
    The lower bound $|a|_{\infty} \cdot n$ follows from looking at the position with a distance of $|a|_{\infty}$ to the target. The drift in the right direction is at most $\frac{1}{n}$, which is the probability of mutating the position. Therefore by the additive drift theorem (\Cref{thm:additivedrift}), we have a run time of  $\Omega(n\cdot |a|_{\infty})$. For the second part we prove that $P(T \geq (n - 1)\log( |a|_H))$ is at least $\frac{1}{2}$. 
The probability that a particular index does not get modified in any of the $t$ iterations is $\left(1 - \frac{1}{n}\right)^{t}$. The previous statement implies that the probability that it does get modified at least once is $1 - \left(1 - \frac{1}{n}\right)^{t}$. Therefore the probability that $|a|_H$ indices gets modified at least once in $t$ iterations is $\left(1 - \left(1 - \frac{1}{n}\right)^{t}\right)^{|a|_H}$. This in turn implies that the probability that at least one of the $|a|_H$ indices does not get modified in $t$ iterations is  $1 - \left(1 - \left(1 - \frac{1}{n}\right)^{t}\right)^{|a|_H}$. If $ t = (n - 1) \ln(|a|_H)$, then 
\begin{align*}
  1 - \left(1 - \left(1 - \frac{1}{n}\right)^{t}\right)^{|a|_H} &= 1 - \left(1 - \left(1 - \frac{1}{n}\right)^{(n - 1) \ln |a|_H}\right)^{|a|_H}  \\
  &\geq 1 - \left(1 - \left(\frac{1}{e}\right)^{\ln |a|_H}\right)^{|a|_H} \\
  &= 1 - \left(1 - \frac{1}{|a|_H}\right)^{|a|_H} \geq 1 - e^{-1} \geq \frac{1}{2}.\
\end{align*}
Now we have expected time
\begin{align*}
    E[T] 
    &= \sum_{t = 1}^{\infty} t \cdot P(T = t) \\
    &= \sum_{t = 1}^{\infty} P(T \geq t) \\
    &\geq (n - 1) \ln |a|_H \cdot P(T \geq (n - 1) \ln |a|_H)\\
    &\geq (n - 1) \ln |a|_H \cdot \frac{1}{2} = \Omega(n \ln |a|_H).
\end{align*}
\end{proof}



\section{Heavy-Tailed Mutation Strength}\label{sec:heavy_tailed_operator}

In this section we discuss the behavior of the \ooea with the heavy-tailed mutation operator on optimizing any $f_a \in \mathscr{F}$. First, in the following theorem, we give an \emph{upper} bound on the expected optimization time.

\begin{thm} \label{thm:heavy_tailed_operator}
Let $f_a \in \mathscr{F}$. Then the expected optimization time of the \ooea, starting with the all-$0$s integer string, with the \emph{heavy-tailed operator} (see Definition \ref{defn:heavy_tailed_operator}) with parameter $\epsilon > 0 $ on $f_a$ is $O(n \cdot \log^2 |a|_1 \cdot \left(\log (\log |a|_1)\right)^{1 + \epsilon})$.
\end{thm}
\begin{proof}
Our proof idea is similar to the proof idea in \cite[~Theorem 15]{timo2017}. We use multiplicative drift analysis in this proof.

Let $x$ and $x'$ be the integer string at iteration $t$ and $t+1$ when the \ooea with heavy-tailed operator is optimizing $f_a$. Let the potential value $X$ at time $t$ be $f_a(x)$. Then the initial potential value is $|a|_1$, since we start with all $0$ integer string. Let $T = \min\{t \geq 0 \mid X = 0\}$. For any $t \geq 0$ and $i \in \{1, \cdots, n\}$, let $d_i = |a_i - x_i|$. 

For any $i \in \{1, \cdots, n\}$ and $j \in \{0, 1, \cdots, \floor{\log d_i}\}$, let $A_{i, j}$ be the event that the mutation operator only modifies the index $i$ such that $|a_i - x_i| - |a_i - x_i'| = 2^j$ and do not make any other changes. If $j^* = j + 2$, then $P(A_{i, j}) \geq \frac{1}{2ec_{\epsilon}nj^* (\log j^*))^{1 + \epsilon}}$. We get the previous bound on the probability because the probability to make $2^j$ changes to a particular index in the right direction is $\frac{1}{2c_{\epsilon}nj^* (\log j^*))^{1 + \epsilon}}$ and the probability to make exactly this change and no other changes to any other indices is at least $1/e$. Also note that while calculating the probability $P(A_{i, j})$, we did not consider the case that the mutation operator can overshoot, since this will only increase the probability.

\begin{align*}
E[X - X' &\mid X] \geq \sum_{i = 1}^{n} \sum_{j = 0}^{\floor{\log d_i}} E[X - X' \mid A_{i, j}, X] \cdot P(A_{i,j}) \\
&= \sum_{i = 1}^{n} \sum_{j = 0}^{\floor{\log d_i}} 2^j \cdot P(A_{i,j}) \\
&\geq  \sum_{i = 1}^{n} \sum_{j = 0}^{\floor{\log d_i}} \frac{2^j}{2ec_{\epsilon}nj^* (\log j^*))^{1 + \epsilon}}\\
&\geq \frac{1}{2ec_{\epsilon}n} \sum_{i = 1}^{n} \frac{d_i}{(\floor{\log d_i} + 2) \left(\log (\floor{\log d_i} + 2)\right)^{1 + \epsilon}}\\
% &\geq \frac{1}{2ec_{\epsilon}n} \sum_{i = 1}^{n} \frac{d_i^t}{\left(\log (\log d_i^t + 2)\right)^{1 + \epsilon}} \sum_{j = 0}^{\log d_i^t } \frac{1}{(\log d_i^t + 2) }\\
% &= \frac{1}{2ec_{\epsilon}n} \sum_{i = 1}^{n} \frac{d_i^t}{\left(\log (\log d_i^t + 2)\right)^{1 + \epsilon}} \cdot \frac{\log d_i^t + 1}{\log d_i^t + 2 }\\
% &\geq \frac{1}{2ec_{\epsilon}n} \sum_{i = 1}^{n} \frac{d_i^t}{\left(\log (\log d_i^t + 2)\right)^{1 + \epsilon}}\\
&\geq \frac{\sum_{i = 1}^{n} d_i}{2ec_{\epsilon}n \cdot (\log |a|_1 + 2) \cdot \left(\log (\log |a|_1 + 2)\right)^{1 + \epsilon}} \\
&= \frac{X}{2ec_{\epsilon}n \cdot (\log |a|_1 + 2) \cdot \left(\log (\log |a|_1 + 2)\right)^{1 + \epsilon}}.
\end{align*}

Since we have an initial potential value $|a|_1$ and a multiplicative drift value of $\frac{1}{2ec_{\epsilon}n \cdot (\log |a|_1 + 2) \cdot \left(\log (\log |a|_1 + 2)\right)^{1 + \epsilon}}$, by multiplicative drift theorem (\Cref{thm:multiplicativedrift}),
\begin{align*}
E[T] &\leq 2ec_{\epsilon}n \cdot (\log |a|_1 + 2) \cdot \left(\log (\log |a|_1 + 2)\right)^{1 + \epsilon} \cdot (1 + \log |a|_1) \\
& = O(n \cdot \log^2 |a|_1 \cdot \left(\log (\log |a|_1)\right)^{1 + \epsilon}).
\end{align*}

Thus we get the upper bound as claimed.
\end{proof}


As a corollary to the proof, we give an upper bound on the expected time taken by the \ooea with the heavy-tailed mutation operator to find an integer string which is at a distance at most $|a|_1 \cdot \alpha$ from the optimum. Thus, we can inform about the time it takes to approximate the optimum.


\begin{cor} \label{cor:heavy_tailed_operator}
Let $f_a \in \mathscr{F}$ and $\alpha \in (0, 1)$. Then the expected optimization time of the \ooea, starting with all $0$ integer string, with \emph{heavy-tailed operator} with parameter $\epsilon > 0 $ on $f_a$, to find an integer string with weight $|a|_1 \cdot \alpha$ is $O(n \cdot \log |a|_1 \cdot \log \left(\frac{1}{\alpha}\right)\cdot \left(\log (\log |a|_1)\right)^{1 + \epsilon})$.
\end{cor}

\begin{proof}
The proof is similar to the proof of the Theorem \ref{thm:heavy_tailed_operator}.
If we consider the same potential as in Theorem \ref{thm:heavy_tailed_operator}, $X = f_a(x)$, then we have the same value $\frac{1}{2ec_{\epsilon}n \cdot (\log |a|_1 + 2) \cdot \left(\log (\log |a|_1 + 2)\right)^{1 + \epsilon}}$ as the multiplicative drift. Let $T =\{t \geq 0 \mid X_t \leq |a|_1 \cdot \alpha \}$.  The initial potential value is at most $|a|_1$ and the minimum value the potential can take is $|a|_1 \cdot \alpha$. 

Therefore, by multiplicative drift theorem (\Cref{thm:multiplicativedrift}),
\begin{align*}
E[T] &\leq 4ec_{\epsilon}n \cdot (\log |a|_1) \cdot \left(\log (\log |a|_1)\right)^{1 + \epsilon} \cdot \left(1 + \log \left(\frac{|a|_1}{|a|_1 \cdot \alpha}\right) \right)\\
& = O(n \cdot \log |a|_1 \cdot \log \left(\frac{1}{\alpha}\right)\cdot \left(\log (\log |a|_1)\right)^{1 + \epsilon}).
\end{align*}

\end{proof}





\section{Self-Adjusting Mutation Rates} \label{sec:self_adjusting_rls}
In this section, we analyze self-adjusting mutation rates for the $RLS$ algorithm and show how these can outperform the \ooea with the static operators analyzed in the previous sections. The mutation strength for $RLS_{\alpha,\beta}$ is adjusted using the constants $1<\alpha\leq 2$ and $1/2<\beta<1$ (see \Cref{rls} for further details on the algorithm). In Theorem \ref{thm:self_adjusting}, we give a tight bound on the expected run time of $RLS_{\alpha,\beta}$ for suitable $\alpha$ and $\beta$. We start by giving a lower bound on the expected run time in Lemma \ref{lem:self_adjusting_lower}.

% Our main result is the matching lower bound of \Cref{lem:self_adjusting_lower} with the upper bound of \Cref{biglemma}.
\begin{lem}[RLS lower bound] \label{lem:self_adjusting_lower}
Let $f_a \in \mathscr{F}$. For constants $\alpha,\beta$ the expected optimization time of $RLS_{\alpha,\beta}$ starting with all $0$ integer string on $f_a$ is $\Theta(n \cdot \log (|a|_1))$.
\end{lem}
\begin{proof}
A bound of $\Omega(n\log |a|_H)$ easily follows from a coupon collector argument: Since we need to change $|a|_H$ many entries and each one has a probability of $\frac{1}{n}$ of being changed in each iteration.

A bound of $\Omega(n\log |a|_{\infty})$ follows from analyzing the entry $j$ with the highest distance to the target. First observe that since the velocity doubles at most each time that an entry is selected and we start at $0$, we need at least $\log(a_{\infty})-1$ changes for this entry.
Let $X_t$ be the random variable counting the number of changes on~$j$.
Let $Y_t$ be another random variable with $Y_t\coloneqq \log(a_{\infty})-1-X_t$. Since $Y_0=\log(|a|_{\infty})-1$ and we have an additive drift of at most $\frac{1}{n}$, the expected run time is of order $\Omega(n\cdot\log(|a|_{\infty}))$. We obtain this drift because the probability of changing the entry $j$ is $\frac{1}{n}$ and we can only change it or not change it.

The lower bound of $\Omega(n\log|a|_1)$ is obtained by adding both run times (this is asymptotically the same as taking the max) to get $$n\log |a|_{\infty}+n\log |a|_H=n\log(|a|_{\infty}|a|_H)\geq n\log(|a|_1).$$
\end{proof}
 

The proof of the upper bound in the following lemma is essentially the same as in \cite[Theorem 17]{timo2017}, only omitting parts that are not necessary for our setting. For the sake of self-containment, we present the modified proof here.

\begin{lem}[RLS upper bound] \label{biglemma}
Let $f_a \in \mathscr{F}$. For constants $\alpha,\beta$ satisfying $1<\alpha\leq 2,1/2<\beta\leq 0.9,2\alpha\beta-\beta-\alpha>0,\alpha+\beta>2$ and $\alpha^2\beta>1 $ the expected optimization time of $RLS_{\alpha,\beta}$ starting with all $0$ integer string on $f_a$ is $O(n \cdot \log (|a|_1))$.
\end{lem}

\begin{proof}

To simplify the notation for a given search point $x$ and the target integer string $z$ and the chosen metric $d$, we let $d_i=d(x_i,z_i)$ for all $(i\leq n)$ be the distance vector of $x$ to $z$. Thus, the goal is to reach a state in which the distance vector is $(0,...,0)$. We now want to define a potential function in dependence on $(d,v)$ (where of course $d$ is dependent on $x$) such that it is $0$ when $d$ is $(0,...,0)$ and strictly positive for any $x\neq (0,...,0)$.

We use as potential function the following map $g:\mathbb{Z}^n\mapsto \mathbb{R}, (x,v)\mapsto \sum_{i=1}^n g_i(d_i,v_i)$ where $g_i(d_i,v_i)\coloneqq 0$ for $d_i=0$ and for $d_i\geq 1$
\begin{align*}
    g_i(d_i,v_i) \coloneqq  d_i + \begin{cases}
			cd_i\max\{2v_i/d_i,d_i/(2v_i)\},& \text{ if $v_i\leq 2\beta d_i$};\\
            cd_i\max\{2v_i/d_i,d_i/(2v_i)\}+pd_i, & \text{otherwise.}
		 \end{cases}
\end{align*}
and $c,p$ are (small) constants specified below. For further motivation on the potential see \cite[~Theorem 17]{timo2017}.

Summarizing all the conditions needed below, we require that the constants $\alpha,\beta,c,p$ satisfy $1<\alpha\leq 2,1/2<\beta\leq 0.9, 2\alpha\beta-\beta-\alpha>0,\alpha+\beta>2,\alpha^2\beta>1,8\alpha\beta c+2p+4c/\beta\leq 1/16, p > 8c\left(\frac{\alpha+\beta}{2}-1\right),$ and $p>4(\alpha-1)c>0$.

We can thus choose, for example, $\alpha=1.7, \beta=0.9, p=0.01,$ and $c=0.001$.

Let $d\neq (0,...,0)$ and $v\in \mathbb{N}^n$. Let $(d',v')$ be the state of \Cref{rls} started in $(d,v)$ after one iteration (i.e., after a possible update of $x$ and $v$). First we show that the expected difference in potential satisfies
\begin{align*}
    E[g(d,v)-g(d',v')|d,v]\geq \frac{\delta}{n} g(d,v)
\end{align*}
for some positive constant $\delta$. Any fixed index $i$ is chosen by \Cref{rls} for mutation with probability $1/n;$ for all $i$, let $A_i$ be the event that index $i$ was chosen. We show that there is a constant $\delta$ such that, for all indices $i$ with $d_i\neq 0$
\begin{align*}
    E[g(d_i,v_i)-g(d_i',v_i') \mid d,v]\geq \delta g_i(d_i,v_i)
\end{align*}
thus proving the claim using $P(A_i)=1/n$.

We regard several cases, depending on how $d_i$ and $v_i$ relate.

Case $1$: $v_i \leq d_i/8$.\\
First we observe that $\max\{2v_i/d_i,d_i/(2v_i)\}=d_i/(2v_i)$. The contribution of the $i$-th position to the current potential is thus
\begin{align*}
    g_i(d_i,v_i) = d_i + cd_i^2/(2v_i).
\end{align*}
With probability $1/2$ the algorithm decides to move in the right direction. In this case we make progress with respect to the fitness function and the velocity. That is, after the iteration we have $d_i'=d_i - \floor{v_i}<d_i$ and $v_i' = \alpha v_i> v_i$.

To bound the progress in the second component of $g_i,$ we observe that
\begin{align*}
    cd_i'\max\{2\alpha v_i / d_i', d_i'/(2\alpha v_i)\}&=\max\{2c\alpha v_i, cd_i'^2/(2\alpha v_i)\}\\&=cd_i'^2/(2\alpha v_i),
\end{align*}
where the second equality follows from $2\alpha v_i\leq d_i/2 < d_i'$. We thus obtain that for this case the difference in potential is at least
\begin{align}\label{case1_1}
    g_i(d_i,v_i)-g_i(d_i',v_i') &= d_i + cd_i^2/(2vi)-d_i' - cd_i'^2 /(2\alpha v_i)\\ &\geq \frac{c d_i^2}{2v_i}-\frac{cd_i^2}{2\alpha v_i}.
\end{align}

With probability $1/2$ the algorithm decides to go in the wrong direction, then $d_i' > d_i$ holds and the new individual is discarded while the velocity $v_i$ at position $i$ is further decreased to $\max\{\beta v_i,1\}\geq \beta v_i$. Hence the difference in potential for this case is at least
\begin{align}\label{case1_2}
    g_i(d_i,v_i)-g_i(d_i,\beta v_i) = \frac{cd_i^2}{2v_i}-\frac{cd_i^2}{2\beta v_i}.
\end{align}

Combining (\ref{case1_1}) and (\ref{case1_2}), we thus obtain that the expected difference in potential is at least
\begin{align*}
    &\frac{1}{2}\left(\frac{cd_i^2}{2v_i}-\frac{cd_i^2}{2\alpha v_i}+\frac{cd_i^2}{2v_i}-\frac{cd_i^2}{2\beta v_i}\right)=\frac{cd_i^2}{2v_i}\left(\frac{2\alpha\beta-\beta-\alpha}{2\alpha\beta}\right)\\
    &= \left(\frac{2\alpha\beta-\beta-\alpha}{4\alpha\beta}\right)\left(\frac{cd_i^2}{2v_i}+\frac{cd_i^2}{2v_i}\right)\geq \left(\frac{2\alpha\beta-\beta-\alpha}{4\alpha\beta}\right)\left(4cd_i + \frac{cd_i^2}{2v_i}\right)\\
    &\geq \left(\frac{2\alpha\beta-\beta-\alpha}{4\alpha\beta}\right)\min\{4c,1\}\left(d_i+\frac{cd_i^2}{2v_i}\right)\\
    &=  \left(\frac{2\alpha\beta-\beta-\alpha}{4\alpha\beta}\right)\min\{4c,1\} g_i(d_i,v_i)
\end{align*}
where in the third step we have used the requirement that $v_i \leq d_i/8$.\\
Case $2$: $d_i/8 < v_i \leq 2\beta d_i$.\\
Now we are in a range of velocity which is well-suited to make progress. In fact, every step towards the optimum decreases the distance to the optimum by at least the minimum of $\floor{d_i}/{8}$ (if $v_i$ is close to $d_i/8$ and we hence do not overshoot the target) and $\floor{(2-2\beta)d_i}$ (if $v_i=2\beta d_i\geq d_i$ in which case we overshoot the target and the distance to it from $d_i$ to at most $\floor{2\beta d_i}-d_i)$. In case of moving towards the target value, the change in the first term of $g_i$ is thus at least
$$\min\{\floor{d_i/8},\floor{(2-2\beta)d_i}\}=\floor{d_i/8},$$
using $\beta \leq 0.9$. However, note that the decrease is at least $1$ (since $v_i$ is at least $1$). Furthermore, we have, for all $z\geq 8, z/16\leq \floor{z/8}.$ Thus, we always have a decrease of at least $d_i/16$.

We now compute the change in the second term of $g_i$. Regard first the case that $\max\{2v_i'/d_i',d_i'/(2v_i')\} = 2v_i'/d_i'$. In this case, we pessimistically assume that the previous contribution of the second term in $g_i(d_i,v_i)$ was zero. This contribution increases to at most\begin{align}
    2cv_i' + p d_i' \leq 2\alpha cv_i + pd_i'\leq 2\alpha cv_i + pd_i\leq (4\alpha\beta c+p)d_i.
\end{align}
On the other hand, $\max\{2v_i'/d_i',d_i'/(2v_i')\} = d_i'/(2v_i')$ and the previous contribution of the second term in $g_i(d_i,v_i)$ was $cd_i^2/(2v_i)$, then the contribution of this second term has been decreased to $c(d_i')^2/(2\alpha v_i)\leq cd_i^2/(2v_i)$. The change in contribution is thus positive in this case, and therefore in particular strictly larger than $-(4\alpha \beta c+p)d_i$. We finally need to regard the case that $$\max\{2v_i'/d_i',d_i'/(2v_i')\}=d_i'/(2v_i')$$ and $$\max\{2v_i/d_i,d_i/(2v_i)\}=2v_i/d_i.$$ In this case the contribution in the second term of $g_i$ increases by at most
\begin{align*}
    \frac{cd_i'^2}{2\alpha v_i}\leq \frac{cd_i^2}{2\alpha(d_i/8)}\leq \frac{4cd_i}{\alpha}\leq 4\alpha\beta c d_i,
\end{align*}
where the last step follows from $\alpha^2\beta\geq 1$.

Summarizing this discussion, we see that in case of stepping towards the target the change in progress satisfies
\begin{align}
    g_i(d_i,v_i)-g_i(d_i',v_i')\geq d_i(1/16 -(4\alpha\beta c+p)),
\end{align}
which is positive by our conditions on $c$ and $p$.

Let us now regard the case of stepping away from the optimum, which happens with probability $1/2$ and the velocity is decreased to $\max\{\beta v_i,1\}$. Assume first that $\max\{\beta v_i,1\} = \beta v_i$. Then,
\begin{align}\label{case2,2}
    g_i(d_i,v_i)-g_i(d_i',v_i')=\max\{2cv_i,\frac{cd_i^2}{2v_i}\}-\max\{2c\beta v_i,\frac{cd_i^2}{2\beta v_i}\}.
\end{align}
If $\max\{2c\beta v_i,cd_i^2/(2\beta v_i)\} = cd_i^2/(2\beta v_i),$ then the term in (\ref{case2,2}) is at least $-cd_i^2/(2\beta v_i)\geq -4cd_i/\beta$ by our condition $d_i/8\leq v_i$. Furthermore, if $\max\{2c\beta v_i,cd_i^2/(2\beta v_i)\}=2c\beta v_i,$ then (\ref{case2,2}) is strictly positive as can be seen by the following observation\begin{align}
    \max\{2cv_i,\frac{cd_i^2}{2v_i}\}-2c\beta v_i \geq 2cv_i - 2c\beta v_i > 0.
\end{align}
Putting everything together we thus obtain that for $d_i/8\leq v_i \leq 2\beta d_i$
\begin{align}
    E[g_i(d_i,v_i)-g_i(d_i',v_i')]\leq \frac{d_i}{2}(1/16-2(4\alpha\beta c+p)-4c/\beta)
\end{align}
which is positive if $8\alpha\beta c+2p+4c/\beta\leq 1/16$. Since $v_i= \Theta(d_i)$ this also shows that there is a positive constant $\delta$ such that $E[g_i(d_i,v_i)-g_i(d_i',v_i')]\geq \delta g_i(d_i,v_i)$.

We finally need to regard the case that $\max\{\beta v_i,1\}=1$. Intuitively, the cap can only make our situation better. This is formalized by the following computations. We need to bound
\begin{align}\label{case2,3}
    g_i(d_i,v_i)-g_i(d_i',v_i')=\max\{2cv_i,\frac{cd_i^2}{2v_i}\}-\max\{2c,\frac{cd_i^2}{2}\}.
\end{align}
As above we obtain positive drift for the case $\max\{2c,cd_i^2\}=2c$ by observing that $\max\{2cv_i,\frac{cd_i^2}{2v_i}\}-2c\geq 2cv_i-2c\geq 0$ (using that $v_i\geq 1$). For the case $\max\{2c,cd_i^2\} = cd_i^2$ the term in (\ref{case2,3}) is at least $-cd_i^2\geq -cd_i^2/(2\beta v_i)\geq -4cd_i/\beta$ as above. The same computation as above thus shows a positive multiplicative gain in $g_i$.


Case $3$: $2\beta d_i< v_i < 2d_i$.\\
Under these conditions $g_i(d_i,v_i)=d_i+2cv_i+pd_i$ holds.

As before, we first regard the case that the algorithm moves towards the target value. Since $\beta\geq 1/2$ it holds that $d_i\leq 2\beta d_i < v_i$ and the target value is thus overstepped. However, due to the requirement $v_i < 2d_i,$ the distance of the offspring is strictly smaller than the previous distance. The velocity is hence increased to $\alpha v_i$.

With probability $1/2$ the algorithm does a step away from the goal and thus the velocity is reduced to $v_i'=\max\{\beta v_i,1\}$. Regard first the case that $v_i'=\beta v_i$. Then, due to $\beta v_i < 2\beta d_i$, the penalty term $pd_i$ is no longer applied and the resulting potential at component $i$ is thus $g_i(d_i',v_i')=d_i+2c\beta v_i$.

Ignoring any possible gains in $d_i$, we therefore obtain that the expected difference in the potential is at least$$2cv_i\left(1-\frac{\alpha+\beta}{2}\right)+\frac{p}{2}d_i$$
Note that $1-(\alpha+\beta)/2$ is negative, since we require $\alpha+\beta>2.$ Using $v_i\leq 2d_i$ we see that the drift is at least
$$4cd_i\left(1-\frac{\alpha+\beta}{2}\right)+\frac{p}{2}d_i=d_i\left(\frac{p}{2}-4c\left(\frac{\alpha+\beta}{2}-1\right)\right).$$
Since $p>8c(\frac{\alpha+\beta}{2}-1)$ this expression is positive. Furthermore, we have $g_i(d_i,v_i)=\Theta(d_i),$ yielding the desired multiplicative drift.

For $v_i'= 1$ we first observe that $v_i'=1\leq d_i \leq 2\beta d_i$ and the penalty term $pd_i$ is thus not in force. Furthermore, we have $\beta d_i < \beta v_i\leq 1$ and thus $d_i\leq 1/\beta \leq 2,$ showing that $\max\{2/d_i,d_i/2\}\leq \max\{2,1\}=2.$ We obtain
$$E[g_i(d_i,v_i)-g(d_i',v_i')\geq\frac{p}{2}d_i-cv_i(\alpha-1)\geq \frac{p}{2}d_i-2(\alpha-1)cd_i,$$
which is positive for $p/2-2(\alpha-1)c>0$.


Case $4$: $v_i=2d_i.$ \\
Steps away from the target are not accepted, thus regardless of whether or not we move towards or away from the target, the fitness does not decrease; therefore, the velocity is decreased to $\beta v_i$ (note that $v_i\geq 2$ and hence $\beta v_i\geq 1)$. The previous contribution of the $i$-th component to $g(x)$ being $d_i+2cv_i+pd_i=d_i(1+4c+p),$ and the new potential at the $i$-th component being $d_i(1+4\beta c)$, we obtain
$$E[g_i(d_i,v_i)-g_i(d_i',v_i')]=d_i(4c+p-4\beta c),$$
which is strictly positive and linear in $g_i(d_i,v_i)$.

Case $5$: $2d_i<v_i$.\\
Steps towards the optimum are now also not accepted, since they overstep the optimum by too much. Therefore, we always decrease the velocity to $\max\{\beta v_i,1\}=\beta v_i$ (note that $v_i>2$ and thus $\beta v_i >1$) and the gain in potential is
$$2cv_i + pd_i - (2c\beta v_i+pd_i)=2cv_i(1-\beta)>4(1-\beta)cd_i$$
showing that we have the multiplicative drift as desired.

Together with the observation that
the initial potential is of order at most 
$$\sum_{i=1}^n d_i^2 \leq \left(\sum_{i=1}^n d_i\right)^2 = |a|_1^2 $$ plugged into the multiplicative drift theorem (\Cref{thm:multiplicativedrift}) proves the desired overall expected run time of $O(n \log(|a|_1))$.
\end{proof}  
Combining both results we get a sharp run time result in the following theorem.

\begin{thm} \label{thm:self_adjusting}
Let $f_a \in \mathscr{F}$. For constants $\alpha,\beta$ satisfying $1<\alpha\leq 2,1/2<\beta\leq 0.9,2\alpha\beta-\beta-\alpha>0,\alpha+\beta>2$ and $\alpha^2\beta>1 $ the expected optimization time of $RLS_{\alpha,\beta}$ starting with all $0$ integer string on $f_a$ is $\Theta(n \cdot \log (|a|_1))$.
\end{thm}
\begin{proof}
    The result follows by the matching lower bound in \Cref{lem:self_adjusting_lower} with the upper bound in \Cref{biglemma},
\end{proof}
\section{Comparison with {CMA-ESwM}}
\label{sec:experiments}


\section{Experiments}
% In this section, we will detail the settings of our experiments and present the experimental results.
To fully demonstrate the superiority of MHCPL,
we conduct experiments\footnote{https://github.com/Snnzhao/MHCPL} on two public datasets to explore the following questions:
\begin{itemize}
    \item \textbf{RQ1:} How does MHCPL perform compared with the state-of-the-art methods?
    \item \textbf{RQ2:} How do different components (social influence, hypergraph based state encoder, and cross-view contrastive learning) affect the results of MHCPL?
    \item \textbf{RQ3:} How do parameters (the layer number of Hypergraph based State Encoder) influence the results of MHCPL?
    \item \textbf{RQ4:} Can our MHCPL effectively leverage the interactive conversation, item knowledge, and social influence to learn the dynamic user preferences?
\end{itemize}

\subsection{Datasets}\label{sec:standalone}
To evaluate the proposed method, we adapt two existing
MCR benchmark datasets, named Yelp and LastFM. The statistics of these datasets are presented in Table \ref{tab:data}.
\begin{itemize}
    \item {\textbf{LastFM}}~\cite{lei2020estimation}: LastFM dataset is the music listening dataset collected from Last.fm online music systems. As Zhang \etal \shortcite{zhang2022multiple}, We define the 33 coarse-grained groups as attribute types for the 8,438 attributes.
    \item{\textbf{Yelp}}~\cite{lei2020estimation}: Yelp dataset is adopted from the 2018 edition of the Yelp challenge. Following Zhang \etal \shortcite{zhang2022multiple}, we define the 29 first-layer categories as attribute types, and 590 second-layer categories as attributes.
\end{itemize}
Following Zhang \etal \shortcite{zhang2022multiple}, we sample two items with partially overlapped attributes as the user's acceptable items for each conversation episode.
\begin{table}[ht]
    \setlength{\tabcolsep}{5pt}
 \centering
 \small
    \begin{tabularx}{0.45\textwidth}{p{3cm}|X|X}
    \toprule
    \makecell[c]{\text{Dataset}}&\makecell[c]{\text{Yelp}}&\makecell[c]{\text{LastFM}}\cr
    \hline
    \hline
    \makecell[c]{\text{Users}}&\makecell[c]{27,675}&\makecell[c]{1,801}\cr

    \makecell[c]{\text{Items}}&\makecell[c]{70,311}&\makecell[c]{7,432}\cr
    \makecell[c]{\text{Attributes}}&\makecell[c]{590}&\makecell[c]{8,438}\cr
    \makecell[c]{\text{Attribute types}}&\makecell[c]{29}& \makecell[c]{34}\cr
    \hline
    \makecell[c]{\text{User-Item}}&\makecell[c]{1,368,606}&\makecell[c]{76,693}\cr
    \makecell[c]{\text{User-User}}&\makecell[c]{688,209}&\makecell[c]{23,958}\cr
    \makecell[c]{\text{Item-Attribute}}&\makecell[c]{477,012}&\makecell[c]{94,446}\cr
    \bottomrule[0.8pt]
    \end{tabularx}
    \caption{Statistics of two utilized datasets}
    \label{tab:data}
\end{table}
\subsection{Experiments Setup}

\subsubsection{User Simulator}
MMCR is a system that is trained and evaluated based on interactive conversations with users. Following the user simulator adopted in \cite{zhang2022multiple}, we simulate a interactive session for each user-item set interaction pair $(u, \mathcal{V}_u)$. Each item in the item set $v \in \mathcal{V}_u$ is treated as an acceptable item for the user. Each session is initialized with a user $u$ specifying an attribute $p_0 \in \mathcal{P}_{joint}$, where $\mathcal{P}_{joint}$ is the set of attributes that are shared by the items in $\mathcal{V}_u$. Then the session follows the process of "System Ask or Recommend, User response" \cite{zhang2022multiple} as described in Section \ref{sec:def}.

\begin{table*}[t]
    \centering
    \begin{tabular}{p{2.0cm}<{\centering}p{1cm}<{\centering}p{1cm}<{\centering}p{1cm}<{\centering}p{1cm}<{\centering}p{1cm}<{\centering}p{0.01cm}p{1cm}<{\centering}p{1cm}<{\centering}p{1cm}<{\centering}p{1cm}<{\centering}p{1cm}<{\centering}p{0.01cm}p{1cm}<{\centering}p{1cm}<{\centering}p{1cm}<{\centering}p{0.01cm}p{1cm}<{\centering}p{1cm}<{\centering}p{1cm}<{\centering}}
    \toprule
    \multirow{2}{*}{\bfseries Models }&\multicolumn{5}{c}{\bfseries Yelp }&&\multicolumn{5}{c}{\bfseries LastFM }\\
    \cline{2-6}
    \cline{8-12}
    &SR@5&SR@10&SR@15&AT&hDCG&&SR@5&SR@10&SR@15&AT&hDCG\\
    \midrule

    Abs Greedy &0.078&0.124&0.150&13.65&0.065&&0.292&0.436&0.512&10.10&0.237\\
    Max Entropy&0.046&0.200&0.390&12.97&0.117&&0.280&0.560&0.680&9.34&0.263\\
    CRM& 0.026&0.100&0.188&13.99&0.059&&0.092&0.240&0.372&12.56&0.130\\
    EAR& 0.120&0.198&0.240&12.91&0.094&&0.298&0.436&0.508&10.08&0.237\\
    SCPR &0.146&0.188&0.436&12.29&0.169&&0.322&0.630&0.764&8.47&0.322\\
    UNICORN &\underline{0.200}&0.338&0.430&11.33&0.175&&0.444&0.774&0.846&7.10&0.348\\
     MCMIPL&{0.162}&{0.366}&{0.522}&{11.25}&{0.184}&&\underline{0.448}&{0.809}&{0.884}&{6.87}&{0.353}\\
    \midrule
    S*-UNICORN &0.120&0.478&0.696&10.59&0.223&&0.412&0.850&0.912&6.69&0.363\\
     S*-MCMIPL&{0.126}&\underline{0.490}&\underline{0.722}&\underline{10.51}&\underline{0.230}&&{0.442}&\underline{0.872}&\underline{0.940}&\underline{6.43}&\underline{0.368}\\
    \midrule
    MHCPL&{0.142}&{\bfseries 0.592}&{\bfseries 0.854}&{\bfseries 9.96}&{\bfseries 0.261}&&{\bfseries 0.470}&{\bfseries 0.938}&{\bfseries 0.982}&{\bfseries 5.87}&{\bfseries 0.427}\\
    % \midrule
     Improv. &-&20.82$\%$&18.28$\%$&5.23$\%$&13.48$\%$&&4.91$\%$&7.57$\%$&4.47$\%$&8.71$\%$&16.03$\%$\\
    \bottomrule
    \end{tabular}
    \caption{Performance comparison of different models on the two datasets. The bold number represents the improvement of our model over baselines is statistically significant with p-value $< 0.01$. hDCG stands for hDCG@($15,10$).}
    \label{tab:results}
\end{table*}

\subsubsection{Baselines}
To demonstrate the effectiveness of the proposed MHCPL, the state-of-the-art methods are chosen for comparison : 
\begin{itemize}
    \item \textbf{Max Entropy.} This method employs a rule-based strategy to ask and recommend. It chooses to select an attribute with maximum entropy based on the current state, or recommends the top-ranked item with certain probabilities \cite{lei2020estimation}.
    \item \textbf{Greedy\cite{christakopoulou2016towards}.} This method only makes item recommendations and updates the model based on the feedback. It keeps recommending items until the successful recommendation is made or the pre-defined round is reached. 

    \item \textbf{CRM\cite{sun2018conversational}.} A reinforcement learning-based method that records the users' preferences into a belief tracker and learns the policy deciding when and which attributes to ask based on the belief tracker.
    \item \textbf{EAR\cite{lei2020estimation}.} This method proposes a three-stage solution to enhance the interaction between the conversational component and the recommendation component.
    \item \textbf{SCPR\cite{lei2020interactive}.} This method learns user preferences by reasoning the path on the user-item-attribute graph via the user’s feedback and accordingly chooses actions.
    \item \textbf{UNICORN\cite{deng2021unified}.} This work builds a weighted graph to model dynamic relationships between the user and the candidate action space, and proposes a graph-based Markov Decision Process (MDP) environment to learn dynamic user preferences and chooses actions from the candidate action space.
    \item \textbf{MCMIPL\cite{zhang2022multiple}.} This approach proposes a multi-interest policy learning framework that captures the multiple interests of the user to decide the next action.
    \item \textbf{S*-UNICORN and S*-MCMIPL.} For a more comprehensive and fair performance comparison, we adapt UNICORN and MCMIPL by timely selecting helpful social information and incorporating it into the weighted graph of the model. We name the two adapted methods S*-UNICORN and S*-MCMIPL.
\end{itemize}

\subsubsection{Parameters Setting}
Following \cite{zhang2022multiple}, we recommend top $K=10$ items or ask $K_a= 2$ attributes in each turn. We employ the Adam optimizer with a learning rate of $1e-4$. Discount factor $\gamma$ is set to be $0.999$. Following \cite{deng2021unified}, we adopt TransE \cite{TransE} via OpenKE \cite{OpenKE} to pretrain the node embeddings with 64 dimensions in the constructed KG with the training set. We make use of Nvidia Titan RTX graphics cards equipped with AMD r9-5900x CPU (32GB Memory).
For the action space, we select $K_p=10$ attributes and $K_v=10$ items.
To maintain a fair comparison, we adopt the same reward settings as previous works  \cite{lei2020estimation, lei2020interactive,deng2021unified,zhang2022multiple}: $r_{rec\_suc}=1, r_{rec\_fail}=-0.1, r_{ask\_suc}=0.01, r_{ask\_fail}=0.1, r_{quit}=-0.3$. For MHCPL, we select the number of layers from {1, 2, 3, 4}.

\subsubsection{Evaluation Metrics}
Following previous works \cite{lei2020estimation, lei2020interactive,deng2021unified}, we adopt success rate (SR@t) to measure the cumulative ratio of successful recommendations by the turn t, average turns (AT) to evaluate the average number of turns for all sessions, and hDCG@(T, K) to additionally evaluate the ranking performance of recommendations. 
Therefore, the higher SR@t and hDCG@(T, K) indicate better performance, while the lower AT means an higher efficiency.


\subsection{Performance Comparison (RQ1)}
\subsubsection{Overall Performance}
The comparison experimental results of the baseline models and our models are shown in Table \ref{tab:results}. 
We can summarize our observations as follows:

\begin{itemize}[leftmargin=*]
    %our对比正常
    \item \textbf{Our proposed MHCPL achieves the best performance.} MHCPL significantly outperforms all the baselines on the metrics of SR@15, AT and hDCG by over 4.47$\%$, 5.23$\%$ and 13.48$\%$, respectively.  We attribute the improvements to the following reasons: 1) The proposed dynamic multi-view hypergraph could effectively capture multiplex relations from three views. And the proposed hierarchical hypergraph neural network is able to well learn dynamic user preferences by integrating the information of graph structure and sequential modeling from the dynamic multi-view hypergraph; 
    2) MHCPL timely selects helpful social information and effectively integrates the interactive conversation, item knowledge, and social influence for better dynamic user preference learning; 3) MHCPL designs a cross-view contrastive learning method to help maintain the inherent characteristics and the correlations of user preferences from different views.
    
    \item \textbf{The learning of the dynamic user preferences is crucial for conversational recommendation.} The graph-based methods (MHCPL, MCMIPL, UNICORN, SCPR) outperforms the factorization-based methods (EAR, CRM) since they learn user preferences from the collaborative information in the graph. MCMIPL achieves the best performance among the graph-base baselines since it further considers the multiple interests of the user preferences. Our proposed MHCPL further outperforms these methods since we leverage multiplex relations to integrate interactive conversation, item knowledge, and social influence to help learn the dynamic user preferences.

    \item \textbf{Social influence is effective in helping learn dynamic user preferences for conversational recommendation when well filtered.} The socially adapted methods (\ie S*-UNICORN and S*-MCMIPL) outperform their original versions in the final performances. We attribute this to the reason that social influence is an important factor that affects user preferences and could help learn dynamic user preferences with friends' preferences that satisfy the interactive conversation. But the socially adapted methods perform worse than their original version in the early turns (\eg SR@5). This happens because the information in the interactive conversation is not sufficient to filter out the noise from the social information in the early turn of the conversation.
    
\end{itemize}

% Figure environment removed

\subsubsection{Comparison at Different Conversation Turns} Besides the performance in the final turn, we also present success rates at different turns in \autoref{fig:overall}.
In order to better observe the differences among different models, we use the relative success rate compared with the most competitive baseline $\text{S*-MCMIPL}$, where the blue line of $\text{S*-MCMIPL}$ is set to zero in the figures. From the \autoref{fig:overall}, we following observations:
\begin{itemize}[leftmargin=*]
    %our对比正achieve
    \item \textbf{} The proposed MHCPL outperforms these baseline methods across all the datasets and almost all the turns in the conversational recommendation. This is because our proposed MHCPL could better learn dynamic user preferences with multiplex relations that integrate interactive conversation, item knowledge, and social influence.
    
    \item \textbf{} The recommendation success rate of the proposed socially-aware methods (\ie MHCPL, S*-MCMIPL, and S*-UNICORN) could not surpass all the baselines in the early turns of the conversational recommendation, especially on the dataset Yelp with a larger candidate space of items and attributes. This is because the information in the interactive conversation is not sufficient to filter out the noise from the social information at the early turn of the conversation.
    Furthermore, socially-aware methods prefer to ask rather than recommend in the early turns when the user's preference is not certain enough. This will effectively reduce the action space and better learn user preferences, but lead to a lower recommendation success rate in the early turns. %This is because they may recommend items with unclear user preferences in the early turns. This can increase early turns' success rates but is not effective since it is helpless in learning user preferences.
\end{itemize}
\begin{table}[t]
    \small
    \centering
    \begin{tabular}{p{2.7cm}<{\centering}p{0.6cm}<{\centering}p{0.5cm}<{\centering}p{0.6cm}<{\centering}p{0.6cm}<{\centering}p{0.5cm}<{\centering}p{0.6cm}<{\centering}}
    \toprule
    % \hline
    \multirow{2}{*}{\bfseries Models }& \multicolumn{3}{c}{\textbf{Yelp}}& \multicolumn{3}{c}{\textbf{LastFM}}\\
    &SR@15&AT&hDCG&SR@15&AT&hDCG\\
    \midrule
    Ours&{\bfseries0.854}&{\bfseries9.96}&{\bfseries0.261}&{\bfseries0.982}&{\bfseries5.87}&{\bfseries0.427}\\
    \midrule
    -w/o social&0.592&10.80&0.208&0.908&6.63&0.365\\
    -w/o hypergraph&0.726&10.68&0.346&0.938&6.58&0.382\\
    -w/o contrastive&0.762&10.37&0.237&0.962&6.17&0.403\\
    \bottomrule
    \end{tabular}
    \caption{Results of the Ablation Study. }
    \label{tab:ablation_study}
\end{table}
\subsection{Ablation Studies (RQ2)}
To investigate the underline mechanism of MHCPL, we conduct ablation experiments on the Yelp and LastFM datasets with three ablated methods including: $\text{MHCPL}_{\wo social}$ that ablates the social influence, $\text{MHCPL}_{\wo hypergraph}$ that replaces the hypergraph neural networks with graph neural networks, and $\text{MHCPL}_{\wo contrastive}$ that ablates the cross-view contrastive learning. From results shown in Table~\ref{tab:ablation_study}, we have the following observations:
\begin{itemize}
    \item $\text{MHCPL}_{\wo social}$ is the least  competitive. This demonstrates the importance of social influence in alleviating the data sparsity problem and helping learn dynamic user preferences. And it is effective to accordingly choose helpful social information based on interactive conversation. $\text{MHCPL}_{\wo social}$ still outperforms all the baselines that ignore the social information in Table~\ref{tab:results}, which proves the effectiveness of MHCPL in learning dynamic user preferences with multiplex relations.
    
    \item MHCPL outperforms $\text{MHCPL}_{\wo hypergraph}$. 
    We contribute this to the importance of multiplex relations in learning dynamic user preferences. This also proves the effectiveness of our proposed multi-view hypergraph-based state encoder in learning user preferences by integrating the information of graph structure and sequential modeling from the dynamic multi-view hypergraph.
    \item MHCPL outperforms $\text{MHCPL}_{\wo contrastive}$. This demonstrates the effectiveness of the cross-view contrastive learning module in helping maintain the inherent characteristics and correlations of user preferences from different views. 
\end{itemize}

% Figure environment removed

\subsection{Hyper-parameter Sensitivity Analysis (RQ3)}
\subsubsection{Impact of Layer Number} The hypergraph-based state encoder learns dynamic user preferences from the multiplex relations in the hypergraph. By stacking more layers, collaborative information from multi-hop neighbors is distilled. We investigate how the layer number $L$ influences the performance of MHCPL. Specifically, we conduct experiments with $L$ in the range $\{1, 2, 3, 4\}$, and the results are shown in Figure \ref{fig:layers}. There are some observations:
\begin{itemize}
    \item Increasing the number of layers can improve the performance of our model. MHCPL-2 highly outperforms MHCPL-1. The reason is that MHCPL-1 only gains information from the one-hop neighbors and neglects high-order collaborative information. 
    \item When increasing the layer of number, the performance does not always improve. MHCPL-3 outperforms MHCPL-4 on data LastFM. This can be attributed to the noise which increases along with the hop of neighbors.
\end{itemize}

% Figure environment removed

\subsection{Case Study (RQ4)}
To show the effectiveness of our proposed MHCPL in leveraging multiplex relations to integrate interactive conversation, item knowledge, and social influence to learn dynamic user preferences, we present a case of conversational recommendation generated by our framework in \autoref{fig:case}. As illustrated in the figure, by integrating the information from the interactive conversation, item knowledge, and social information with multiplex relations from different views, MHCPL is able to effectively ask attributes and recommend user-preferred items, reaching success in five turns. Furthermore, the social information selected according to the interactive conversation is helpful in learning dynamic user preferences. With the help of selected social information, MHCPL could accurately select the target item when the information from the interactive history is limited in distinguishing user preferences towards the seventy candidate items.



\bibliographystyle{ACM-Reference-Format}
\bibliography{main.bib}



% \section{Purgatory}

% \input{purgatory}



\begin{acks}
This work is supported by grant FR 2988/17-1 by the German Research Foundation (DFG).
\end{acks}

\begin{comment}
\section{System Architecture}
\label{appendix:architecture}
\system has a novel modularized system architecture with three key components: 
\emph{StreamManager}, 
\emph{TxnManager} and \emph{TxnScheduler}. 
These components are instantiated in each thread locally.
The execution outline of \system is presented in Algorithm~\ref{alg:algo}.
Transactional stream processing is continuous and potentially never ends (Line 1$\sim$8).
The dependency resolution and execution of state transactions are separated into two non-overlapping phases by punctuations~\cite{Tucker:2003:EPS:776752.776780} (Line 2 and 5), which guarantees that no subsequent input event will have a smaller timestamp. 
Effectively, a batch of state transactions is collected during the first phase, and processed during the second phase.

In the first phase (i.e., stream processing phase), 
the \emph{StreamManager} conducts preprocessing for every input event ($e$). Similar to some prior works~\cite{tstream}, state transactions may be issued but not immediately processed during preprocessing (Line 3).
The \emph{pre\_processing} and \emph{post\_processing} functions are exposed as APIs to users.
The \emph{TxnManager} handles dependency resolution (Line 4) among state transactions and insert decomposed operations to construct a \tpg. We discuss the detailed two-phase \tpg construction process in Section~\ref{subsec:construction}.

In the second phase  (i.e., transaction processing phase), 
the \emph{TxnManager} is first involved again to refine (Line 6) the constructed \tpg with further dependency resolution.
The \emph{TxnScheduler} 
schedules operations for concurrent execution based on the constructed \tpg according to the three dimensions of scheduling decisions (Line 7). 
In particular, a scheduling decision model $M$ is instantiated based on the constructed \tpg (Line 14).
\textbf{\circled{1}} Guided by $M$, execution threads adopt an exploration strategy (Section~\ref{subsec:explore}) to explore the constructed \tpg for operations available to be scheduled constrained by dependencies. 
\textbf{\circled{2}} 
During exploration, one or multiple operations may be treated as the 
% basic 
unit of scheduling (Section~\ref{subsec:granularity}). 
Subsequently, \textbf{\circled{3}} every thread executes operation(s) in the unit of scheduling with various abort handling mechanisms (Section~\ref{subsec:abort_handling}).
Only when state transactions are processed (i.e., committed or aborted) can the associated input events be postprocessed (Line 8) by the \emph{StreamManager} based on transaction processing results.
\end{comment}

\begin{comment}
\begin{algorithm}
\footnotesize
    \KwData{$e$ \tcp{Input event}}
    \KwData{$txn_{ts}$ \tcp{State transaction}}
    \KwData{$G$ \tcp{The currently constructed TPG}}
    \While{!finish processing of input streams}{
        \eIf(\tcp*[h]{Phase 1}){\text{$e$ is not a $punctuation$}}{
                $txn_{ts}$ $\gets$ PRE\_Processing($e$)\;
                \textbf{TPG\_Construction}($G$, $txn_{ts}$)\; 
          }(\tcp*[h]{Phase 2}){
                \textbf{TPG\_Refinement}($G$)\; 
                \textbf{TXN\_Scheduling}($G$)\; 
                POST\_Processing()\;
          }
    }
    
    \SetKwFunction{FMain}{TPG\_Construction}
    \SetKwProg{Fn}{Function}{:}{}
    \Fn{\FMain{$G$, $txn_{ts}$}}{
        $O_{1..k}$ $\gets$ \textbf{Partition} $txn_{ts}$\;
        \ForEach{\text{operation $O_{i}$ $\in$ $O_{1..k}$}}{
            \textbf{Identify} its \ld\;
            $G$ $\gets$ $G$ + $O_{i}$ \;
        }
    }
    \SetKwFunction{FMain}{TPG\_Refinement}
    \SetKwProg{Fn}{Function}{:}{}
    \Fn{\FMain{$G$}}{
        \ForEach{\text{vertex $e_{i}$ $\in$ $G$}}{
            \textbf{Identify} its \td, \pd\;
        }
    }
    
    \SetKwFunction{FMain}{TXN\_Scheduling}
    \SetKwProg{Fn}{Function}{:}{}
    \Fn{\FMain{$G$}}{
        $M$ $\gets$ Instantiated with $G$;\tcp{A decision model}
        \While{!finish scheduling of $G$
        }{
          \textbf{\circled{2}} $Scheduling Unit$ $\gets$ \textbf{\circled{1}} \emph{Explore}($G$, $M$)\; 
            \textbf{\circled{3}} \emph{Execute with Abort Handling} ($Scheduling Unit$)\; 
        }
    }
  \caption{Execution Outline of \system}
  \label{alg:algo}
\end{algorithm}
\end{comment}
\end{document}

