Estimation-of-Distribution Algorithms for Multi-Valued Decision Variables by Firas Ben Jedidia, Benjamin Doerr, Martin S. Krejca

Abstract:

With apparently all research on estimation-of-distribution algorithms (EDAs) concentrated on pseudo-Boolean optimization and per- mutation problems, we undertake the first steps towards using EDAs for problems in which the decision variables can take more than two values, but which are not permutation problems. To this aim, we propose a natural way to extend the known univariate EDAs to such variables. Different from a naive reduction to the binary case, it avoids additional constraints.
Since understanding genetic drift is crucial for an optimal parameter choice, we extend the known quantitative analysis of genetic drift to EDAs for multi-valued variables. Roughly speaking, when the variables take r different values, the time for genetic drift to become significant is r times shorter than in the binary case. Consequently, the update strength of the probabilistic model has to be chosen r times lower now.
To investigate how desired model updates take place in this framework, we undertake a mathematical runtime analysis on the r-valued LeadingOnes problem. We prove that with the right parameters, the multi-valued UMDA solves this problem efficiently in $O(r log(r)^2n^2 log(n))$ function evaluations.
Overall, our work shows that EDAs can be adjusted to multi-valued problems, and it gives advice on how to set the main parameters.

BibTex citation:

@misc{jedidia2023estimationofdistribution,
      title={Estimation-of-Distribution Algorithms for Multi-Valued Decision Variables}, 
      author={Firas Ben Jedidia and Benjamin Doerr and Martin S. Krejca},
      year={2023},
      eprint={2302.14420},
      archivePrefix={arXiv},
      primaryClass={cs.NE}
}

pdf link: https://arxiv.org/pdf/2302.14420.pdf

Note: Only found a preprint version in Arxiv.

2. An Evolutionary Algorithm for Integer Programming by Gunter Rudolph

Abstract:The mutation distribution of evolutionary algorithms usually is oriented at the type of the search space. Typical examples are binomial distributions for binary strings in genetic algorithms or normal distributions for real valued vectors in evolution strategies and evolutionary programming. This paper is devoted to the construction of a mutation distribution for unbounded integer search spaces. The principle of maximum entropy is used to select a specific distribution from numerous potential candidates. The resulting evolutionary algorithm is tested for five nonlinear integer problems.

Note: The construction of the mutation distribution is non-trivial and it uses entropy of the distribution.

BibTex citation:

@InProceedings{10.1007/3-540-58484-6_258,
author="Rudolph, G{\"u}nter",
editor="Davidor, Yuval
and Schwefel, Hans-Paul
and M{\"a}nner, Reinhard",
title="An evolutionary algorithm for integer programming",
booktitle="Parallel Problem Solving from Nature --- PPSN III",
year="1994",
publisher="Springer Berlin Heidelberg",
address="Berlin, Heidelberg",
pages="139--148",
abstract="The mutation distribution of evolutionary algorithms usually is oriented at the type of the search space. Typical examples are binomial distributions for binary strings in genetic algorithms or normal distributions for real valued vectors in evolution strategies and evolutionary programming. This paper is devoted to the construction of a mutation distribution for unbounded integer search spaces. The principle of maximum entropy is used to select a specific distribution from numerous potential candidates. The resulting evolutionary algorithm is tested for five nonlinear integer problems.",
isbn="978-3-540-49001-2"
}

pdf link: https://ls11-www.cs.tu-dortmund.de/people/rudolph/publications/papers/PPSN94.pdf


3. Discrete Multi-Valued Particle Swarm Optimization by Pugh, Jim; Martinoli, Alcherio

Abstract:

We present a new optimization technique based on the Particle Swarm Optimization algorithm to be used for solving problems with unordered discrete-valued solution elements. The algorithm achieves comparable performance to both the previous modification of PSO for binary optimization and to genetic algorithms on a standard set of benchmark problems. Performance is maintained on these problems re-encoded with ternary solution elements. The algorithm is then tested on a simulated discrete-valued unsupervised robotic learning problem and obtains competitive results. Various potential improvements, modifications, and uses of the algorithm are discussed.

BibTex citation:

@article{article,
author = {Pugh, Jim and Martinoli, A.},
year = {2006},
month = {01},
pages = {},
title = {Discrete Multi-Valued Particle Swarm Optimization},
volume = {1}
}

pdf link: https://infoscience.epfl.ch/record/83373?ln=en


4. Probabilistically Driven Particle Swarms for Optimization of Multi Valued Discrete Problems : Design and Analysis by Kalyan Veeramachaneni

Abstract:

A new particle swarm optimization (PSO) algorithm that is more effective for discrete, multi-valued optimization problems is presented. The new algorithm is probabilistically driven since it uses probabilistic transition rules to move from one discrete value to another in the search for an optimum solution. Properties of the binary discrete particle swarms are discussed. The new algorithm for discrete multi-values is designed with the similar properties. The algorithm is tested on a suite of benchmarks and comparisons are made between the binary PSO and the new discrete PSO implemented for ternary, quaternary systems. The results show that the new algorithm’s performance is close and even slightly better than the original discrete, binary PSO designed by Kennedy and Eberhart. The algorithm can be used in any real world optimization problems, which have a discrete, bounded field.

BibTex citation:

@INPROCEEDINGS{4223167,
  author={Veeramachaneni, Kalyan and Osadciw, Lisa and Kamath, Ganapathi},
  booktitle={2007 IEEE Swarm Intelligence Symposium}, 
  title={Probabilistically Driven Particle Swarms for Optimization of Multi Valued Discrete Problems : Design and Analysis}, 
  year={2007},
  volume={},
  number={},
  pages={141-149},
  doi={10.1109/SIS.2007.368038}}


pdf link:
https://ieeexplore.ieee.org/stamp/stamp.jsp?tp=&arnumber=4223167

—————————————————————————————————————————————————————————————————————————————————————————————————————————————————————————————————



Multiple-Valued Minimization for PLA(programmable logic arrays) Optimization by R.L. Rudell; A. Sangiovanni-Vincentelli.

citation:
@ARTICLE{1270318,
  author={Rudell, R.L. and Sangiovanni-Vincentelli, A.},
  journal={IEEE Transactions on Computer-Aided Design of Integrated Circuits and Systems}, 
  title={Multiple-Valued Minimization for PLA Optimization}, 
  year={1987},
  volume={6},
  number={5},
  pages={727-750},
  doi={10.1109/TCAD.1987.1270318}}

pdf link: https://ieeexplore.ieee.org/stamp/stamp.jsp?tp=&arnumber=1270318

2. Design of multivalued circuits using genetic algorithms by Wenjun Wang; C. Moraga.

citation:

@INPROCEEDINGS{508361,
  author={Wenjun Wang and Moraga, C.},
  booktitle={Proceedings of 26th IEEE International Symposium on Multiple-Valued Logic (ISMVL'96)}, 
  title={Design of multivalued circuits using genetic algorithms}, 
  year={1996},
  volume={},
  number={},
  pages={216-221},
  doi={10.1109/ISMVL.1996.508361}}

pdf link: https://ieeexplore.ieee.org/stamp/stamp.jsp?tp=&arnumber=508361

3. A Random Forest Using a Multi-valued Decision Diagram on an FPGA by Hiroki Nakahara; Akira Jinguji; Simpei Sato; Tsutomu Sasao.

citation: 

@INPROCEEDINGS{7965002,
  author={Nakahara, Hiroki and Jinguji, Akira and Sato, Simpei and Sasao, Tsutomu},
  booktitle={2017 IEEE 47th International Symposium on Multiple-Valued Logic (ISMVL)}, 
  title={A Random Forest Using a Multi-valued Decision Diagram on an FPGA}, 
  year={2017},
  volume={},
  number={},
  pages={266-271},
  doi={10.1109/ISMVL.2017.40}}

pdf link: https://ieeexplore.ieee.org/stamp/stamp.jsp?tp=&arnumber=7965002

And few more related to multi-valued logic trees...