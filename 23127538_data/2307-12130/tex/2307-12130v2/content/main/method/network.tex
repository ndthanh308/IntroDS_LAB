\noindent This work focuses on the physical modeling of the IR camera and the use of a physical constraint to improve the temperature estimation by using a single-frame.
To highlight the effect of the physical constraint, we use the well-known U-Net architecture~\cite{unet} because it is a well-established architecture for image-to-image translation tasks. The neural network itself is a backbone to test the physical constraint, and the results are not dependent on the network architecture.
More information on the network architecture is provided in \cref{sec:supp:network} of the supplementary material.
The architecture is presented in \cref{fig:supp:network} of the supplementary material.

The network shown in \cref{fig:supp:network} operates as an end-to-end function:
\begin{equation}\label{eq:netEnd2End}
    \tobjapprox = F(I(\tobj), \tamb)
\end{equation} where $I(\tamb)$ is a gray-level map taken at known ambient temperature $\tamb$, and $F$ is the output of the network blocks. We name this configuration \textbf{E2E} (end to end).

\cref{eq:stefanBoltzmannTaylor} shows that the radiance is a linear function of the scene temperature, and \cref{eq:finalWithFPA} shows a linear dependence on the ambient temperature.
To plug this prior knowledge into the network, the final block in the backbone network was replaced with two identical blocks.
Both blocks have the same input, which is the output of the layer before the split.
These blocks extract the estimated object temperature $\tobjapprox$ from the linear approximation of the radiance shown in \cref{eq:finalWithFPA}:
\begin{equation}
    \begin{split}\label{eq:netPysical}
        I(\tamb) &= G(\tamb)\cdot\tobj+D(\tamb) \longrightarrow \\
        \tobjapprox &= \mathcal{G} \cdot I(\tamb) + \mathcal{D}
    \end{split}
\end{equation}
where $I(\tamb)$ is the input to the network, and $\mathcal{G}\approx\frac{1}{G}, \mathcal{D}\approx\frac{D}{G}$ and the outputs of the respective blocks. We name this configuration \textbf{GxPD} (Gx + D).

The network is trained on the synthetic dataset created by the simulator described in \cref{sec:method:characterize}, and the inference is performed on a single IR gray-level frame.

The effects of both networks are elaborated in \cref{sec:experiments}.