\noindent A method to characterize the physical behavior of a system was demonstrated (\cref{sec:method:characterize}).
The characterization process allowed for supervised training of a neural network (\cref{sec:method:net}).
The temperature estimation performed by the network can be generalized to real data and different cameras (\cref{sec:experiments:realdata}).
We also showed that previously-suggested methods does not generalize well to different cameras (\cref{fig:results:realDataCmp}).
This allows for a faster NUC process that only requires a single collection of calibration data.
Moreover, the method not only estimates the temperature of the object, but also corrects the nonuniformity of the camera.

When using the proposed camera characterization and training, the backbone network (E2E) shows a significant improvement of roughly $1^\circ C$ compared to previous works~\cite{He2018}, producing a mean temperature error (MAE) of only $0.42^\circ C$ over the validation dataset.

We proposed a physical constraint on neural networks that incorporates the physical behavior of the system into the network (\cref{sec:method:net}).
The physically-constrained network (GxPD) achieved a MAE of $0.37^\circ C$, a significant improvement of $12\%$ in the accuracy of the temperature estimation compared to the backbone network (E2E).
Another notable result is that the ambient temperature of the camera has a significant effect on the accuracy of the temperature estimation, reducing the MAE by $13\%$.

Results on real-world experimental data achieve a MAE ranging in $0.15^\circ C-0.93^\circ C$ with a \emph{different camera} than the one used for training and validation. This shows that the proposed method can generalize to different cameras.