\IEEEPARstart{I}{nfrared} (IR) imagery in the $8_{\mu m}-14_{\mu m}$ atmospheric window measures the thermal radiation emitted from an object. IR imagery is extensively used for various applications, such as - military night vision \cite{bolometer_night_vision}, medical fever screening \cite{bolometer_medial_screening} and machinery fault diagnosis \cite{bolometer_fault_diagnosis}, among many others. One interesting utilization of such an imaging system is agriculture, because the temperature of a plant is important in deducing information on its well-being \cite{ir_agri_est_crop_water, ir_agri_dought}.
% Figure environment removed

Low-cost IR cameras are usually uncooled, and rely on microbolometer arrays as sensors. The microbolometer array enables the construction of inexpensive IR cameras with low energy requirements. Unlike the photon-counting detector arrays (e.g., CMOS in the visible range), microbolometers measure changes in the electrical resistance caused by the incident thermal radiation originating from an object \cite{bolometer}. The thermal radiation heats each microbolometer to a temperature that depends on the scene, and each microbolometer in the array has a slightly different temperature depending on the observed scene and the incident angle of the radiation. The resistance of each microbolometer changes according to the scene temperature. The minuscule changes in the resistance of each microbolometer in the array are used to construct an image corresponding to the temperature of the observed scene.

While accurate microbolometer-based IR camera exists (e.g., \scientificCamera which will be used extensively in this work), they are expensive and require complex calibration procedures~\cite{kruseIR}.
Low-cost IR cameras temperature estimation accuracy is usually low.
\cref{fig:intro:GraylevelsAsFunctionOfFPAs} demonstrates the effect of the camera's ambient temperature on the measurements made by a typical low-cost IR camera \taucamera. Notice that the change is dependent on the ambient temperature, as well as non-linear.
% Figure environment removed

\subsection{Nonuniformity and noise}\label{sec:intro:nonuniformity}
Microbolometer arrays are subject to space-variant nonuniformity and noise from various sources.
The microbolometer array is uncooled, and so a prominent source of nonuniformity is thermal radiation emitted by the camera itself~\cite{kruseIR}.
Another parasitic thermal radiation source is the narcissus effect, where unfocused reflection of the detector returns from the optical surfaces~\cite{infraredTheramlImaging}. The effect of the internal self-radiation (red lines) mixed with the incident thermal radiation from the scene (green lines) is schematically presented in \cref{fig:intro:selfHeating}.

These parasitic effects are dependent on ambient temperatures, meaning that their influence on the measurements changes with the environmental conditions of the camera.
\cref{fig:intro:ConstBlackbodyDiffFPAs} demonstrates the nonuniformity effect. The images presented are of a scientific-grade blackbody target set at a constant temperature of $40^\circ C$, and were taken with a \taucamera camera.
\cref{fig:intro:ConstBlackbodyDiffFPAs:18} was taken at an ambient temperature of $18.1^\circ C$, and \cref{fig:intro:ConstBlackbodyDiffFPAs:53} was taken at an ambient temperature of $53.7^\circ C$.
Two effects are visible in the images. First, the gray-levels of the image are dependent on the ambient temperature. Second, the nonuniformity is spatially variant.
The effect of the ambient temperature on the middle horizontal line of both images is illustrated in \cref{fig:intro:ConstBlackbodyDiffFPAs:lineCmp}, with both lines on the same axis.
The nonuniformity exhibits spatial variation, as evidenced by the different relations between the lines for each pixel.
\newcommand{\heightCmpGlDiffFPAs}{10em}
% Figure environment removed

Another source of nonuniformity is fixed-pattern noise (FPN). The readout circuitry of the microbolometer array is usually line-based (similar to charge coupled devices). Slight changes between readers on the same array can lead to considerable disparity between lines on the image \cite{Riou2004}.

Finally, the signal-to-noise ratio of the camera is often low due to readout and electronic noise~\cite{kruseIR}.
% These noises affect the minimum detectable change in scene temperature, known in the literature as noise equivalent differential temperature (NEDT). Lower NETD values are preferable, and the noise in the camera increases this value~\cite{Riou2004}.

\subsection{Image acquisition model}
The thermal radiation emitted by a body for all wavelengths can be found using the Stefan-Boltzmann law, whereby the emitted radiation can be approximated by the fourth power of the object's temperature~\cite{kruseIR}:
\begin{equation}\label{eq:stefanBoltzmann}
    L(T)\approx\epsilon\sigma T^4  \quad \left[   \frac{W}{m^2}   \right]
\end{equation}
where $T$ is the object's temperature, $\epsilon$ is a proportional constant and $\sigma$ is the Boltzmann constant.

In a small environment near a reference temperature $T_0$, the Stefan-Boltzmann law can be expanded by Taylor series:
\begin{equation}\label{eq:stefanBoltzmannTaylor}
    \begin{split}
        L(T)&=\epsilon\cdot\sigma T^4=\epsilon\cdot\sigma(T_0+\Delta T)^4\\
        &\approx\epsilon\cdot\sigma(T_0)^4+4\epsilon\cdot\sigma(T_0)^3\Delta T
        \approx a_1\cdot\tobj + a_0
    \end{split}
\end{equation}
where $a_0$, $a_1$ are the coefficients and $T_0$ is a reference temperature. $\Delta T$ was changed to $\tobj$ for brevity.

\cref{eq:stefanBoltzmannTaylor} demonstrates that the radiation can be approximated as linear in scene temperature for a small environment around a reference temperature. This result means that the incident thermal radiation on the sensor has a temperature-\textit{dependent} element and a temperature-\textit{independent} element.

The ambient temperature of the camera has a profound effect on the measurements that it produces. 
%\cref{fig:intro:drift} demonstrates the drift in measurements caused by a change in ambient temperature.
Thus, the model in \cref{eq:stefanBoltzmannTaylor} must also account for changes in ambient temperature.
The linear approximation of the overall reading of the camera depends on both the ambient temperature and the object temperature:
\begin{equation}\label{eq:finalWithFPA}
    L(\tobj, \tamb) = G(\tamb)\cdot L(\tobj)+D(\tamb)
\end{equation} where $\tamb,\tobj$ are the ambient and object temperatures, respectively.

$G(\tamb)$ and $D(\tamb)$ in \cref{eq:finalWithFPA} are polynomials of $\tamb$. The polynomial model has been previously shown to be representative of the underlying physical thermal radiation model (e.g,\cite{Nugent2013,Tempelhahn2016}). For the remainder of this work, higher-order polynomials, mainly quadratic, will be used for approximations.

Separating the coefficients from the object temperature in \cref{eq:finalWithFPA} is complicated when only the camera response is given \cite{Papini2018}.
However, some mathematical functions can separate a product into a summation, such as the $\log()$ function \cite{math_handbook}. The existence of a separation function suggests the use of neural networks, which can approximate any function \cite{nn_alg_book}. Thus, this work attempts to leverage a neural network for representing the physical model in \cref{eq:finalWithFPA}.