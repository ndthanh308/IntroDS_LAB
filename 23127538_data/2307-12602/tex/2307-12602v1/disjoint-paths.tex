\documentclass[11pt,a4paper]{article}
\usepackage{fullpage}
\usepackage[utf8]{inputenc}
\usepackage{amsmath,amssymb,mathtools}
\usepackage{amsthm}
\usepackage{xcolor}
%%%
\usepackage{tikz}
\usepackage{graphicx,color}
\definecolor{blue}{RGB}{0,2,127} 
\definecolor{green}{RGB}{30,150,34} 
\definecolor{red}{rgb}{220,0,0}
\colorlet{grey}{black!30}
%%%
\usepackage{enumitem}
\usepackage{thmtools}
\usepackage{subcaption,thm-restate}
\usepackage[sort]{cite}
\usepackage{hyperref}
\hypersetup{
    colorlinks=true,       % false: boxed links; true: colored links
    linkcolor=blue,        % color of internal links (change box color with linkbordercolor)
    citecolor=red,         % color of links to bibliography
    filecolor=magenta,     % color of file links
    urlcolor=cyan,         % color of external links
    linktocpage=true
}
\usepackage{cleveref}
\usepackage{algorithm}
\usepackage{algorithmicx}
\usepackage[noend]{algpseudocode}
\usetikzlibrary {arrows.meta}

\algrenewcommand\algorithmicrequire{\textbf{Input:}}
\algrenewcommand\algorithmicensure{\textbf{Output:}}

\newenvironment{varalgorithm}[1]
  {\algorithm\renewcommand{\thealgorithm}{#1}}
  {\endalgorithm}



\DeclareRobustCommand*{\ora}{\overrightarrow}
\DeclareRobustCommand*{\ola}{\overleftarrow}
\DeclareRobustCommand*{\olra}{\overleftrightarrow}
\DeclareRobustCommand*{\orla}{\overleftrightarrow}

%%%%%%%%%%%%%%%%%%%%%%%%%%%%%%%%%%%%%%%%%%%%%%%%%%%%%%%%%%%%
%%%% for wrapping text in algorithmx environment
\makeatletter
\newcommand{\algmargin}{\the\ALG@thistlm}
\makeatother
\algnewcommand{\parState}[1]{\State%
    \parbox[t]{\dimexpr\linewidth-\algmargin}{\strut\hangindent=\algorithmicindent \hangafter=1 #1\strut}}


%%%%%%%%%%%%%%%%%%%%%%%%%%%%%%%%%%%%%%%%%%%%%%%%%%%%%%%%%%%%
\theoremstyle{plain}
\newtheorem{thm}{Theorem}
\newtheorem{lem}[thm]{Lemma}
\newtheorem{cor}[thm]{Corollary}
\newtheorem{conj}[thm]{Conjecture}
\newtheorem{claim}[thm]{Claim}
\newtheorem{observation}[thm]{Observation}

\theoremstyle{definition}
\newtheorem{defn}[thm]{Definition}
%\newtheorem{note}[theorem]{Notes}
%\newtheorem{example}[theorem]{Example}
%\newtheorem{assumption}[theorem]{Assumption}
\newtheorem{rem}[thm]{Remark}

%%%%% CLAIM PROOF
%\newenvironment{claimproof}[1]{\par\noindent\underline{Proof:}\space#1}{\hfill $\lhd$}
\newenvironment{claimproof}[1]{\par\noindent\textit{Proof:}\space#1}{\hfill $\lhd$}


\def\DISP{\textsc{Shortest Two Disjoint Paths}}
\def\Q{\widetilde{Q}}
\def\R{\widetilde{R}}
\def\T{\mathcal{T}}
\def\F{\mathcal{F}}
\def\dist{\textup{dist}}
\def\Shared{\textup{Sh}}
\def\Private{\textup{Pr}}
\def\Child{\textup{Ch}}
\def\NP{\mathsf{NP}}
\def\Amend{\texttt{Amend}}



%%%%%%%%%%%%%%%%%%%%%%%%%%%%%%%%%%%%%%%%%%%%%%%%%%%%%%%%%%%%
\author{Ildik\'o Schlotter$^{1,2}$}
\title{Shortest two disjoint paths in conservative graphs}
\date{
\normalsize
    $^1$Centre for Economic and Regional Studies, Hungary \\%
    $^2$Budapest University of Technology and Economics, Hungary \\[2ex]%
    }
%%%%%%%%%%%%%%%%%%%%%%%%%%%%%%%%%%%%%%%%%%%%%%%%%%%%%%%%%%%%
\begin{document}

\maketitle

%%%%%%%%%%%%%%%%%%%%%%%%%%%%%%%%%%%%%%%%%%%%%%%%%%%%%%%%%%%%

\begin{abstract}
We consider the following problem that we call the \textsc{Shortest Two Disjoint Paths} problem: 
given an undirected graph~$G=(V,E)$ with edge weights $w:E\rightarrow \mathbb{R}$, two terminals $s$ and~$t$ in~$G$, 
find two internally vertex-disjoint paths between $s$ and $t$ with minimum total weight.
As shown recently by Schlotter and Seb\H{o} (2023), this problem becomes $\NP$-hard if edges can have negative weights, 
even if the weight function is conservative, i.e., there are are no cycles in~$G$ with negative weight. 
We propose a polynomial-time algorithm that solves the \textsc{Shortest Two Disjoint Paths} problem for conservative weights 
in the case when the negative-weight edges form a single tree in~$G$. 
\end{abstract}

\section{Introduction}

Finding disjoint paths between given terminals is a fundamental problem in algorithmic graph theory and combinatorial optimization.
Besides its theoretical importance, it is also motivated by numerous applications in transportation, VLSI design, and network routing.  
In the \textsc{Disjoint Paths} problem, we are given $k$ terminal pairs $(s_i,t_i)$ for $i \in \{1,\dots,k\}$ in an undirected graph~$G$, 
and the task is to find pairwise vertex-disjoint paths $P_1,\dots, P_k$ 
so that $P_i$ connects $s_i$ with $t_i$ for each $i \in \{1,\dots,k\}$. 
This problem was shown to be $\NP$-hard by Karp~\cite{Karp75} when $k$ is part of the input, and remains $\NP$-hard even on planar graphs~\cite{Lynch1975}.
Robertson and Seymour~\cite{RS1995} proved that there exists an $f(k) n^3$ algorithm for \textsc{Disjoint Paths} with $k$ terminal pairs, 
where $n$ is the number of vertices in~$G$ and $f$ some computable function; this celebrated result is among the most important achievements of graph minor theory.
In the \textsc{Shortest Disjoint Paths} problem we additionally require that $P_1, \dots, P_k$ have minimum total length (in terms of the number of edges). 
For fixed $k$, the complexity of this problem is one of the  most important open questions in the area. 
Even the case for $k=2$ had been open for a long time, until Bj\"orklund and Husfeldt~\cite{BH2019} gave a randomized polynomial-time algorithm for it in 2019. 
For directed graphs the problem becomes much harder: the \textsc{Directed Disjoint Paths} problem is $\NP$-hard already for $k=2$.
The \textsc{Disjoint Paths} problem and its variants have also received considerable attention 
when restricted to planar graphs~\cite{DSS1992,VS2011,KS2010,AKKLST17,CMPP13,Schrijver1994,LMPSZ20}.

The variant of \textsc{Disjoint Paths} when $s_1=\dots =s_k=s$ and $t_1=\dots =t_k=t$ is considerably easier, since one can find $k$ pairwise 
(vertex- or edge-) disjoint paths between $s$ and $t$ using a max-flow computation. 
Applying standard techniques for computing a minimum-cost flow (see e.g.~\cite{schrijver-book}), 
one can even find $k$ pairwise disjoint paths between $s$ and $t$ with minimum total weight, 
given non-negative weights on the edges. 
Notice that if negative weights are allowed, then flow techniques break down for undirected graphs: 
in order to construct an appropriate flow network based on our undirected graph~$G$, 
the standard technique is to direct each edge of $G$ in both directions;
however, if edges can have negative weight, then this operation creates negative cycles consisting of two arcs, an obstacle for computing a minimum-cost flow.
Recently, Schlotter and Seb\H{o}~\cite{SS2023} have shown that this issue is a manifestation of a complexity barrier: 
finding two openly disjoint paths with minimum total weight between two vertices in an undirected edge-weighted graph is $\NP$-hard, even if weights are \emph{conservative} (i.e., no cycle has negative total weight) and each edge has weight in $\{-1,1\}$.\footnote{In fact, Schlotter and Seb\H{o} use an equivalent formulation for the problem where, instead of finding two openly disjoint paths between $s$ and $t$, the task  is to find two vertex-disjoint paths between $\{s_1,s_2\}$ and $\{t_1,t_2\}$ for four vertices $s_1,s_2,t_1,t_2 \in V$.} 
Note that negative edge weights occur in network problems due to various reasons:
for example, they might arise as a result of some reduction 
(e.g., deciding the feasibility of certain scheduling problems with deadlines translates into finding negative-weight cycles), 
or as a result of data that is represented on a logarithmic scale.
We remark that even the \textsc{Single-Source Shortest Paths} problem is the subject of active research for the case when negative edges are allowed; 
see Bernstein et al.~\cite{BNWN22} for an overview of the area and their state-of-the-art algorithm running in near-linear time.



\medskip
\noindent
{\bf Our contribution.}
We consider the following problem: 
\begin{center}
\fbox{ 
\parbox{13.6cm}{
\begin{tabular}{l}\DISP{}:  \end{tabular} \\
\begin{tabular}{p{1cm}p{11.5cm}}
Input: & An undirected graph $G=(V,E)$, a weight function~$w\colon E \to \mathbb{R}$ that is conservative on~$G$, and two vertices $s$ and $t$ in~$G$. \\
Task: & Find two paths~$P_1$ and $P_2$ between $s$ and $t$ with $V(P_1) \cap V(P_2)=\{s,t\}$ that minimizes $w(P_1)+w(P_2)$.
\end{tabular}
}}
\end{center}
A \emph{solution} for an instance $(G,w,s,t)$ of \DISP{} is a pair of openly disjoint $(s,t)$-paths.

From the $\NP$-hardness proof for \DISP{} by Schlotter and Seb\H{o}~\cite{SS2023} it follows that the problem remains $\NP$-hard even if the set of negative-weight edges forms a perfect matching. 
Motivated by this intractability, we focus on the ``opposite'' case when 
the subgraph of~$G$ spanned by the set $E^-=\{e \in E:w(e)<0\}$ of negative-weight edges, denoted by $G[E^-]$, is connected.\footnote{See 
Section~\ref{sec:prelim} for the precise definition of a subgraph spanned by an edge set.}
Note that since $w$ is conservative on~$G$, the graph $G[E^-]$ is acyclic.
Hence, our assumption that $G[E^-]$ is connected amounts to assuming that $G[E^-]$ is a single tree.
Ideally, one would aim for an algorithm that is efficient when the number of connected components in $G[E^-]$ is small;
our paper can be seen as a first step towards such an algorithm.

We prove the following result:
\begin{thm}
\label{thm:DISP-main}
\DISP{} can be solved in polynomial time on instances where the set of negative edges spans a tree in~$G$.
\end{thm} 


Our algorithm first applies standard flow techniques to find minimum-weight solutions among those that have a simple structure:
either because only one of the paths contains negative-weight edges, or 
because both paths contain negative-weight edges but these paths do not ``interfere'' with each other in a certain sense.
To deal with more complex solutions where both paths heavily use negative edges and they need to take care to avoid each other, 
we apply an intricate dynamic programming method
that is based on significant insight into the structural properties of such solutions.

\medskip
\noindent
{\bf Organization.}
We give all necessary definitions in Section~\ref{sec:prelim}. 
In Section~\ref{sec:tree-init} we make initial observations about optimal solutions for an instance~$(G,w,s,t)$ of \DISP{},
and we also present a lemma of key importance that will enable us to create solutions by 
combining partial solutions that are easier to find (Lemma~\ref{lem:uncrossing}).
We prove our main result, Theorem~\ref{thm:DISP-main}, in Section~\ref{sec:negtree}. 
We start this in Section~\ref{sec:mainAlg} by giving a general description of our algorithm, 
and explaining which types of solutions can be found using flow-based techniques. 
We proceed in Section~\ref{sec:properties} by establishing structural observation that we need to exploit in 
order to find those types of solutions where more advanced techniques are necessary. 
Section~\ref{sec:partsol} contains our dynamic programming method for finding partial solutions which, together with Lemma~\ref{lem:uncrossing}, 
form the heart of our algorithm.
We assemble the proof of Theorem~\ref{thm:DISP-main} in Section~\ref{sec:permdis} using our findings in the previous sections, 
and finally pose some questions for further research in Section~\ref{sec:conclusion}.




\section{Notation}
\label{sec:prelim}

For a positive integer $\ell$, we use $[\ell]\coloneqq\{1,2,\dots,\ell\}$.

 Let a graph~$G$ be a pair $(V,E)$ where $V$ and $E$ are the set of vertices and edges, respectively.
For two vertices $u$ and $v$ in~$V$, an edge connecting $u$ and $v$ is denoted by $uv$ or $vu$.
A \emph{walk}~$W$ in~$G$ is a series $e_1, e_2, \dots, e_\ell$ of edges in~$G$ for which there exist 
vertices $v_0, v_1, \dots, v_\ell$ in~$G$ such that $e_i= v_{i-1} v_i$ for each $i \in [\ell]$;
note that both vertices and edges may appear repeatedly on a walk. 
We denote by $V(W)$ the set of vertices \emph{contained by} or \emph{appearing on}~$W$, that is, $V(W)=\{v_0, v_1, \dots, v_\ell\}$. 
%We define the \emph{length} of~$W$ as $|W|=\ell$, and we say that $W$ is \emph{odd} (or \emph{even}) if $|W|$ is odd (or even, respectively). 
The \emph{endpoints} of $W$ are $v_0$ and $v_\ell$, or in other words, it is a  \emph{$(v_0,v_\ell)$-walk}, while all vertices on~$W$ that are not endpoints are \emph{inner vertices}.
If $v_0=v_\ell$, then we say that $W$ is a \emph{closed walk}.

A \emph{path} is a walk on which no vertex appears more than once.
By a slight abuse of notation, we will usually treat a path as a \emph{set} $\{e_1,e_2,\dots, e_\ell\}$ of edges for which there exist distinct vertices $v_0, v_1, \dots, v_\ell$ in~$G$ such that $e_i= v_{i-1} v_i$ for each $i \in [\ell]$.
For any $i$ and $j$ with $0 \leq i \leq j \leq \ell$ we will write $P[v_i,v_j]$ for the \emph{subpath} of $P$ between~$v_i$ and~$v_j$, consisting of edges $e_{i+1}, \dots, e_j$. 
Note that since we associate no direction with~$P$, we have $P[v_i,v_j]=P[v_j,v_i]$.
Given two vertices $s$ and~$t$, an $(s,t)$-path is a path whose endpoints are~$s$ and~$t$;
similarly for two subsets~$S$ and~$T$ of vertices, 
and $(S,T)$-path is a path with one endpoint in~$S$, and the other endpoint in~$T$. 
%For indices $i_1, i_2, \dots, i_r$ we say that the vertices $v_{i_1}, v_{i_2}, \dots, v_{i_r}$ \emph{follow each other in this order} if $i_1 \leq i_2 \leq \dots \leq i_r$ or $i_1 \geq i_2 \geq \dots \geq i_r$.

We say that two paths are \emph{vertex-} or \emph{edge-disjoint}, if they do not share a common vertex or edge, respectively. Two paths are \emph{openly disjoint}, if they share no common vertices apart from possibly their endpoints. 
Given vertices $s_1,s_2,t_1,$ and $t_2$, we say that two $(\{s_1,s_2\},\{t_1,t_2\})$-paths are \emph{permissively disjoint}, if a vertex $v$ can only appear on both paths if either $v=s_1=s_2$ or $v=t_1=t_2$. 
Two paths properly intersect, if they share edges, but neither is the subpath of the other.


A \emph{cycle} in~$G$ is a set $\{e_1, e_2, \dots, e_\ell\}$ of distinct edges in~$G$ such that $e_1, e_2, \dots, e_{\ell-1}$ form a path in~$G - e_\ell$ whose endpoints are connected by $e_\ell$. 

Given a set $U \subseteq V$ of vertices in~$G$, the subgraph of $G$ \emph{induced by} $U$ is the graph on vertex set~$U$ that contains an edge between two vertices of~$U$ exactly if they are connected in~$G$ by an edge.
Given a set $F \subseteq E$ of edges in~$G$, the subgraph of $G$ \emph{spanned by} $F$ is the graph whose edge set is~$F$ and whose vertex set contains vertices incident to some edge of~$F$; we denote this subgraph as $G[F]$.

A set $T \subseteq E$ of edges in~$G$ is \emph{connected}, if for every pair of edges $e$ and $e'$ in~$T$, there is a path contained in~$T$ containing both $e$ and $e'$.
If $T$ is connected and \emph{acyclic}, i.e., contains no cycle, then $T$ is a \emph{tree} in~$G$. 
Given two vertices $a$ and $b$ in a tree~$T$, we denote by $T[a,b]$ the unique path contained in~$T$ whose endpoints are $a$ and $b$, 
and we denote by $\overrightarrow{T}[a,b]$ the directed path obtained by orienting $T[a,b]$ from~$a$ towards~$b$.
For an edge $uv \in T$ and a path $P$ within~$T$ such that $uv \notin P$,
we say that $v$ \emph{is closer to~$P$} in~$T$ than~$u$, if $v \in V(T[u,p])$ for any vertex $p$ on~$P$.


Given a weight function~$w\colon E \to \mathbb{R}$ on the edge set of~$G$, we define the \emph{weight} of any edge set $F \subseteq E$ as $w(F)=\sum_{e \in F} w(e)$.
We say that $w$ (or, to make the dependency on~$G$ explicit, the weighted graph $(G,w)$) is \emph{conservative}, if no cycle in~$G$ has a negative weight. 
\iffalse
For a set~$F \subseteq E$ of edges in~$G$, we define the weight function $w_F:E \rightarrow \mathbb{R}$ based on~$w$ as
\[
w_F(a)=\left\{
\begin{array}{cc}
    w(a), & \textrm{ if $a \notin F$,} \\
    -w(a) & \textrm{ if $a \in F$.}
\end{array}\right.
\]
In a directed graph with arc set~$A$, we define the weight function $w_F$ on~$A$ for some arc set~$F \subseteq A$ in the same way.
\fi

\section{Initial observations}
\label{sec:tree-init}

Let $G=(V,E)$ be an undirected graph with a conservative weight function~$w:E \rightarrow \mathbb{R}$. 
Let $E^-=\{e \in E:w(e)<0\}$ denote the set of negative edges, and $\T$ the set of negative trees they form. More precisely, let~$\T$ be the set of connected components in the subgraph $G[E^-]$; the acyclicity of each $T \in \T$ follows from the conservativeness of~$w$.
Although our main result concerns the case when $\T$ contains only a single tree, 
in this section we will prove our claims in the general setting when there is no restriction on the instance other than the conservativeness of~$w$.

We start with the following simple observation.

\begin{lem}
\label{lem:closed-walk}
If $W$ is a closed walk that does not contain any edge with negative weight more than once, then $w(W) \geq 0$.
\end{lem}

\begin{proof}
Let $Z$ be the set of edges that includes an edge $e$ of the graph if and only if $e$ appears an odd number of times on~$W$. Note that since any edge used at least twice in~$W$ has non-negative weight, we know $w(Z) \leq w(W)$. Moreover, since $W$ is a walk, each vertex $v \in V$ has an even degree in the graph $(V,Z)$. 
Therefore, $(V,Z)$ is the union of edge-disjoint cycles, and the conservativeness of~$w$ implies $w(Z) \geq 0$.
\end{proof}

The next lemma establishes further implications of the conservativeness of our weight function, 
considering the weights of paths running between two vertices on some negative tree in~$\T$. 

\begin{lem}
\label{lem:T-min-pathlength}
Let $x,y,x',y'$ be four distinct vertices on a tree~$T$ in~$\T$. \begin{itemize}
    \item[(1)] If $Q$ is an $(x,y)$-walk in~$G$ that does not use any edge with negative weight more than once, then $w(Q) \geq w(T[x,y])$.
    \item[(2)] If $Q$ is an $(x,y)$-path, $Q'$ is an $(x',y')$-path, and $Q$ and $Q'$ are vertex-disjoint,  then \\
    $w(Q)+w(Q')\geq w(T[x,y] \Delta T[x',y'])$. \\
    Moreover, if $T[x,y]$ and $T[x',y']$ properly intersect, then the inequality is strict.
\end{itemize} 
\end{lem}


\begin{proof}
We first start with describing a useful procedure. The input of this procedure is a set~$F \subseteq E$ that can be partitioned into a set $\F$ of mutually vertex-disjoint paths, each with both endpoints on~$T$, and its output is a new weight function~$w_F$ on $E$ fulfilling $w(F)=w_F(F)$. 

Initially, we set $w_F \equiv w$.
A \emph{$T$-leap} in~$F$ is a path in~$\F$ that has both endpoints on~$T$, has no inner vertices on~$T$, and contains no edge of~$T$. 
We consider each $T$-leap in~$F$ one-by-one. 
So let $L$ be a $T$-leap in~$F$ with endpoints $a$ and~$b$.
We define the \emph{shadow} of~$L$ as the edge set $F \cap T[a,b]$. 
Then for each edge $f$ in the shadow of $L$ for which $w_Q(f)=w(f)$, we set $w_F(f):=0$ and we decrease $w_F(L)$ by $|w(f)|$; we may do this by decreasing $w_F$ on the edges of~$L$ in any way as long as the total weight of~$L$ is decreased by $|w(f)|$. 
Since $w$ is conservative, $w(L) + w(T[a,b]) \geq 0$ by Lemma~\ref{lem:closed-walk}, and thus $w(L)\geq |w(T[a,b])| \geq \sum \{ |w(f)|:f \textrm{ is in the shadow of }L\}$.
Therefore, after performing this operation for each edge in the shadow of~$L$, $w_F(L) \geq 0$ remains true.
Applying these changes for each $T$-leap in~$F$, the resulting weight function~$w_F$ fulfills 
(i) $w_F(F) = w(F)$, 
(ii) $w_F(L) \geq 0$ for each $T$-leap~$L$ in~$F$, 
and that (iii) $w_F(f)=0$ for each edge~$f$ in the shadow of~$F$.

To show statement~(1) of the lemma, let $Q$ be an $(x,y)$-walk in~$G$. Observe that we can assume w.l.o.g.\ that $Q$ is an $(x,y)$-path: if $Q$ contains cycles, then we can repeatedly delete any cycle from~$Q$, possibly of length~2, so that in the end we obtain an $(x,y)$-path whose weight 
(since $w$ is conservative and no edge with negative weight is contained more than once in~$Q$)
is at most the weight of~$Q$.
Observe that any edge of~$Q \cap T$ that is not in the shadow of any $T$-leap on~$Q$ must lie on~$T[x,y]$: 
indeed, for any edge $uv \in Q \cap T$ that lies outside~$T[x,y]$, with $v$ being closer to $T[x,y]$ than~$u$ in~$T$, 
either $Q[u,x]$ or $Q[u,y]$ needs to use  a $T$-leap that contains~$uv$ in its shadow. 
Hence,  for the function~$w_Q$ obtained by the above procedure (for $F = Q$) we get 
$$w(Q) = w_Q(Q) \geq w(T[x,y])+\sum_{\textrm{$L$: $L$ is a $T$-leap on~$Q$}} w_F(L) \geq  w(T[x,y]).$$

To prove statement~(2), assume that $Q$ and $Q'$ are as in the statement.
Observe that any edge $f$ in~$(Q \cup Q') \cap T$ that is not contained in~$T[x,y] \Delta T[x',y']$ must be in the shadow of some $T$-leap in~$Q \cup Q'$: indeed,
 if $f$ is not in $T[x,y] \cup T[x',y']$, then this follows by the same arguments we used for statement~(1), and if $f \in T[x,y] \cap T[x',y']$, then either $f \in Q$ in which case it must be in the shadow of a $T$-leap on~$Q'$, or $f \in Q'$ in which case it is in the shadow of a $T$-leap on~$Q$, as $Q$ and $Q'$ are vertex-disjoint. Considering the function~$w_{Q \cup Q'}$ obtained by the above procedure, we therefore know
\begin{align}
\label{eqn:Tminpath}
w(Q \cup Q') \, &= \,w_{Q \cup Q'}(Q \cup Q') \geq w(T[x,y] \Delta T[x',y'])+\sum_{\textrm{$L$: $L$ is a $T$-leap in~$Q \cup Q'$}} w_{Q \cup Q'}(L)  \notag \\
\, &\geq \,   w(T[x,y] \Delta T[x',y']). %\qedhere
\end{align}
To prove the last statement of the lemma, assume that $T[x,y]$ properly intersects $T[x',y']$. 
Since $Q$ and $Q'$ are vertex-disjoint, 
this implies that there exists an edge $f^\star$ in $T[x,y] \cap  T[x',y']$ that is not contained in $Q \cup Q'$ 
(because neither~$Q$ nor~$Q'$ can entirely contain $T[x,y] \cap  T[x',y']$). 
Note that $Q$ must contain a leap~$L^\star$ for which the cycle induced by~$L$ contains~$f^\star$ (in fact, the same is true for~$Q'$).
Since $f^\star \notin Q_1 \cup Q_2$, we have that $f^\star$ is not in the shadow of~$L^\star$,  which implies that
$w_{Q \cup Q'}(L^\star) \geq  -w(f^\star) > 0$. Consequently,  inequality~(\ref{eqn:Tminpath}) is strict.
\end{proof}

\begin{defn}[\bf Locally cheapest path pairs]
Let $s_1,s_2,t_1,t_2$ be vertices in~$G$, and let $P_1$ and $P_2$ be two permissively disjoint $(\{s_1,s_2\},\{t_1,t_2\})$-paths. 
A path~$T[u,v]$ in some~$T \in \T$ is called a \emph{shortcut} for $P_1$ and $P_2$, if $u$ and $v$ both appear on the same path, either $P_1$ or~$P_2$, 
and there is no inner vertex or edge of~$T[u,v]$ is contained in $P_1 \cup P_2$. 
We will call $P_1$ and $P_2$ \emph{locally cheapest}, if there is no shortcut for them.
\end{defn}

The idea behind this concept is the following. 
Suppose that $P_1$ and $P_2$ are permissively disjoint $(\{s_1,s_2\},\{t_1,t_2\})$-paths, and 
$T[u,v]$ is a shortcut for $P_1$ and $P_2$.
Suppose that $u$ and~$v$ both lie on $P_i$ (for some $i \in [2]$), and let $P'_i$ be the path obtained by replacing $P_i[u,v]$ with $T[u,v]$; 
we refer to this operation as \emph{amending} the shortcut $T[u,v]$.
Then $P'_i$ is also permissively disjoint from~$P_{3-i}$ and, since Lemma~\ref{lem:closed-walk} implies $w(P_i[u,v]) \geq -w(T[u,v]) >0$,
has weight less than $w(P_i)$. 
Hence, we have the following observation.

\begin{observation}
\label{obs:locally-cheapest}
Let  $P_1$ and $P_2$ be two permissively disjoint $(\{s_1,s_2\},\{t_1,t_2\})$-paths,
admitting a shortcut $T[z,z']$. 
Suppose that $z$ and $z'$ are on the path, say,~$P_1$.
Let $P'_1$ be the path obtained by amending~$T[z,z']$ on~$P_1$. Then 
$P'_1$ and $P_2$ are permissively disjoint $(\{s_1,s_2\},\{t_1,t_2\})$-paths and
    $w(P'_1) < w(P_1)$.
\end{observation}


Observe also that amending all shortcuts on two paths~$P_1$ and~$P_2$ results always in the same pair of paths, denoted by $\Amend(P_1,P_2)$,
independently from the order in which we amend the shortcuts. 
Note that shortcuts can be found simply by traversing all trees~$T \in \T$ 
and checking their intersection with $P_1$ and $P_2$. This way, $\Amend(P_1,P_2)$ can be computed in linear time.

\smallskip
The following lemma will be a crucial ingredient in our algorithm, as it enables us to combine ``partial solutions'' 
without violating the requirement of vertex-disjointness.

\begin{lem}%[Uncrossing Lemma]
\label{lem:uncrossing}
Let $p_1,p_2,q_1,q_2$ be vertices in~$G$, and let $T \in \T$ contain vertices~$v_1$ and~$v_2$ with $v_1 \neq v_2$.
Let $P_1$ and $P_2$ be two permissively disjoint $(\{p_1,p_2\},\{v_1,v_2\})$-paths in $G$, 
and let $Q_1$ and $Q_2$ be two permissively disjoint $(\{v_1,v_2\},\{q_1,q_2\})$-paths in $G$ that are locally cheapest. 
Assume also that we can partition~$\T$ into two sets $\T_1$ and $\T_2$ with~$T \in \T_2$ such that 
\begin{itemize}
    \item[(i)]
    $V(T) \cap V(P_1 \cup P_2) =\{v_1,v_2\}$, and 
    \item[(ii)]
     $P_1 \cup P_2$ contains no edge of~$\bigcup \T_2$, and 
    $Q_1 \cup Q_2$  contains no edge of~$\bigcup \T_1$.
\end{itemize}
Then there exist two permissively disjoint $(\{p_1,p_2\},\{q_1,q_2\})$-paths $S_1$ and $S_2$ in~$G$ 
such that 
$w(S_1)+w(S_2) \leq w(P_1)+w(P_2) + w(Q_1)+w(Q_2)$;
moreover, such paths can be found in linear time.
\end{lem}

\begin{proof}
W.l.o.g.\ we may assume that $P_i$ is a $(p_i,v_i)$-path 
and $Q_i$ is a $(v_i,q_i)$-path for both $i \in [2]$.
Let $y_i$ be the first vertex on~$P_i$ when traversed from~$p_i$ to~$v_i$ that is contained in~$V(Q_1 \cup Q_2)$.
We distinguish between three cases as follows.

{\bf Case A:} $y_1 \in V(Q_1)$ and $y_2 \in V(Q_2)$. 
In this case let $S_i=P_i[p_i,y_i] \cup Q_i[y_i,q_i]$
for $i \in [2]$. Observe that $S_1$ and $S_2$ are two permissively disjoint $(\{p_1,p_2\},\{q_1,q_2\})$-paths: 
this follows from the definition of~$y_1$ and $y_2$, and our assumptions on the disjointness of the paths~$P_1$ and $P_2$, as well as that of $Q_1$ and $Q_2$.
Note also that for any $i \in [2]$, the path~$S_i$ can be obtained from $P_i \cup Q_i$ by deleting the paths~$P_i[y_i,v_i]$ and $Q_i[v_i,y_i]$. Observe that $W_i=P_i[y_i,v_i] \cup Q_i[v_i,y_i]$ is a closed walk. 
Due to~(ii), no edge of~$E^-$ may appear both on~$P_i$ and on~$Q_i$. Hence, 
the walk~$W_i$ does not contain any edge of~$E^-$ more than once, and so by Lemma~\ref{lem:closed-walk} we get $w(W_i) \geq 0$.  This implies
$w(S_i) = w(P_i) +w(Q_i) -w(W_i) \leq w(P_i) + w(Q_i)$, proving the lemma for Case~A.

{\bf Case B:} $y_1 \in V(Q_2)$ and $y_2 \in V(Q_1)$. 
In this case let $S_i=P_i[s_i,y_i] \cup Q_{3-i}[y_i,q_{3-i}]$
for $i \in [2]$. One can observe as in the previous case that $S_1$ and $S_2$ are two permissively disjoint $(\{p_1,p_2\},\{q_1,q_2\})$-paths.
Note also that for any $i \in [2]$, the path~$S_i$ can be obtained from $P_i \cup Q_{3-i}$ by deleting the paths~$P_i[y_i,v_i]$ and $Q_{3-i}[v_{3-i},y_i]$. Hence we get
\[ S_1 \cup S_2= P_1 \cup P_2 \cup Q_1 \cup Q_2 \setminus
(P_1[y_1,v_1] \cup Q_1[v_1,y_2] \cup P_2[y_2,v_2] \cup Q_2[v_2,y_1]).
\]
Observe that $W=P_1[y_1,v_1] \cup Q_1[v_1,y_2] \cup P_2[y_2,v_2] \cup Q_2[v_2,y_1]$ is a closed walk.
Due to the disjointness of~$P_1$ and~$P_2$, we know
that $P_1$ shares no edge with~$P_2$; 
similarly, $Q_1$ shares no edge with~$Q_2$. 
Moreover, $P_1 \cup P_2$ shares no edge of~$E^-$ with $Q_1 \cup Q_2$ due to~(ii). Therefore, no edge of~$E^-$ may appear more than once on~$W$.
Thus by Lemma~\ref{lem:closed-walk} we get $w(W) \geq 0$.
 This implies
\[w(S_1)+w(S_2) = w(P_1)+w(P_2) + w(Q_1)+w(Q_2) -w(W) 
\leq w(P_1)+w(P_2) + w(Q_1)+w(Q_2),\] proving the lemma for Case~B.

{\bf Case C:} If $y_1$ and $y_2$ both lie on the same path from~$Q_1$ and $Q_2$. By symmetry, we may assume that they both lie on~$Q_1$. Suppose that $y_1$ comes before~$y_2$ when $Q_1$ is traversed starting from~$v_1$;
the case when $y_2$ precedes $y_1$ can be dealt with in an analogous manner. 

%We need the following claim.
\begin{claim}
\label{clm:connection-in-T}
We can find a path $T[u,u']$ in linear time such that~$u$ lies on~$Q_1 \setminus Q_1[y_2,q_1]$, $u'$ lies on~$Q_2$, and no other vertex of~$T[u,u']$ appears on~$Q_1 \cup Q_2$.  
\end{claim}

\begin{claimproof}
Recall that $y_1 \in V(P_1)$ implies that either $y_1=v_1$ or $y_1 \notin V(T)$. 

First assume $y_1=v_1$, and consider the path $T[v_1,v_2]$. Since $Q_1$ and $Q_2$ may only share $q_1$ and $q_2$ (in case they coincide), but $y_2 \notin V(T)$, it must be the case that 
when travelling through $T[v_1,v_2]$ from $v_1$, we exit $Q_1$ at some vertex~$u$, meaning that the edge following~$u$ on this path does not belong to~$Q_1$. Proceeding within~$T[v_1,v_2]$ towards $v_2$, at some point we enter a vertex of $Q_2$, let $u'$ denote this vertex. 
Observe that $T[u,u']$ contains no vertex of~$Q_2$ as an inner vertex by definition. Moreover, it also cannot contain a vertex of~$Q_1$ as an inner vertex. Indeed, supposing that some vertex $z \in V(Q_1)$ is an inner vertex of~$T[u,u']$, the path $T[u,z]$ would be a shortcut, contradict our assumption that $Q_1$ and $Q_2$ are locally cheapest. 
Note also that by $y_2 \notin V(T)$ we also know that $u \notin Q_1[y_2,q_1]$.
Hence, the claim holds.
 
Assume now $y_1 \notin V(T)$. Then we start by walking on~$Q_1$ from~$y_1$ towards $v_1$ until we reach a vertex of~$T$ (note that, by our assumption, $y_2$ is not on this path, and note also $v_1 \in V(T)$), and from that point on, we travel towards $v_2 \in V(T)$ within~$T$. Following the procedure described in the previous case, we can define $u$ and $u'$ as the vertices where we exit $Q_1$ and enter $Q_2$, respectively, and we can argue $T[u,u']$ is disjoint from~$Q_1 \cup Q_2$ apart from its endpoints. Again, notice that $y_2$ cannot lie on the walk from~$y_1$ towards $u$, and so $u$ lies on $Q_1 \setminus Q_1[y_2,q_1]$ and the claim holds.
\end{claimproof}

\smallskip
Using Claim~\ref{clm:connection-in-T}, we can now define $S_1$ and $S_2$. Let $T[u,u']$ be the path guaranteed by the claim. We distinguish between two cases, depending on the place of~$u$ on~$Q_1$; see Figure~\ref{fig:uncrossing} for an illustration.

% Figure environment removed


{\bf Case C/1:}
Assume that $u$ lies on $Q_1[v_1,y_1]$. Define
\begin{align}
\label{eq:uncrossing-defS1}
S_1 &=P_1[p_1,y_1] \cup Q_1[y_1,u] \cup T[u,u'] \cup Q_2[u',q_2], \\
\label{eq:uncrossing-defS2}
S_2 &=P_2[p_2,y_2] \cup Q_1[y_2,q_1].
\end{align} 
Observe that $S_1$ and $S_2$ are two permissively disjoint $(\{p_1,p_2\},\{q_1,q_2\})$-paths: 
this follows from the definition of~$y_1$ and $y_2$, our assumptions on the disjointness of the paths~$P_1$ and $P_2$, as well as that of $Q_1$ and $Q_2$, from the properties of~$T[u,u']$, and from our assumption~(i).

Consider the closed walk 
\[W:=P_1[y_1,v_1] \cup Q_1[v_1,u] \cup T[u,u'] \cup Q_2[u',v_2] \cup P_2[v_2,y_2] \cup
Q_1[y_2,y_1]. \]
Note that $W$ does not contain any edge of~$E^-$ more than once due to our assumption~(ii), the properties of~$T[u,u']$, and the disjointness of~$P_1$ and $P_2$, as well as that of $Q_1$ and~$Q_2$.
Hence we get $w(W)\geq 0$ by Lemma~\ref{lem:closed-walk}. 
Observe that Equations~\ref{eq:uncrossing-defS1} and~\ref{eq:uncrossing-defS2} yield
\begin{align*}
S_1 \cup S_2 &=  (P_1 \cup P_2 \cup Q_1 \cup Q_2) \setminus  (W \setminus T[u,u']) \cup T[u,u'].
\end{align*}
By $w(W) \geq 0$ and $w(T[u,u'])<0$ this implies 
$w(S_1)+w(S_2) <(P_1)+w(P_2) + w(Q_1)+w(Q_2)$, proving the lemma for Case C/1.

{\bf Case C/2:}
Assume now that $u$ lies on $Q_1[y_1,y_2]$.\footnote{The reader might observe that the proof of Claim~\ref{clm:connection-in-T} implies that in this case only $v_1=y_1$ is possible, but we do not use this fact in the proof of Lemma~\ref{lem:uncrossing}.}
Define $S_1$ and $S_2$ exactly as in Equations~\ref{eq:uncrossing-defS1} and~\ref{eq:uncrossing-defS2}; again, $S_1$ and $S_2$ are two permissively disjoint $(\{p_1,p_2\},\{q_1,q_2\})$-paths.
Consider the closed walks 
\begin{align*}
W_1 &=P_1[y_1,v_1] \cup Q_1[v_1,y_1], \\
W_2 &= Q_1[u,y_2] \cup P_2[y_2,v_2] \cup Q_2[v_2,u'] \cup T[u',u],
\end{align*}
and observe that, owing to the same reasons as in the previous case, neither~$W_1$ nor~$W_2$ contains an edge of~$E^-$ more than once.
Hence, we get $w(W_1 \cup W_2)\geq 0$ by Lemma~\ref{lem:closed-walk}.
Observe that Equations~\ref{eq:uncrossing-defS1} and~\ref{eq:uncrossing-defS2} yield
\begin{align*}
S_1 \cup S_2 &=  (P_1 \cup P_2 \cup Q_1 \cup Q_2) \setminus  (W_1 \cup W_2 \setminus T[u,u']) \cup T[u,u'].
\end{align*}
By $w(W_1 \cup W_2) \geq 0$ and $w(T[u,u'])<0$ this implies 
$w(S_1)+w(S_2) < w(P_1)+w(P_2) + w(Q_1)+w(Q_2)$, proving the lemma for Case C/2.
\end{proof}

\section{Polynomial-time algorithm: when negative edges form a tree}
\label{sec:negtree}
This section contains the proof of our main result, Theorem~\ref{thm:DISP-main}.
Let $(G,w,s,t)$ be our instance of \DISP{} with input graph~$G=(V,E)$, 
and assume that the set~$E^-$ of negative edges spans a tree~$T$ in~$G$.
We present a polynomial-time algorithm that computes a solution for $(G,w,s,t)$ with minimum total weight, 
or correctly concludes that no solution exists for~$(G,w,s,t)$.
Let $n=|V|$ and $m=|E|$. 


\subsection{The algorithm}
\label{sec:mainAlg}
We distinguish between two types of solutions for our instance $(G,w,s,t)$ of \DISP{}. 

Consider a solution for $(G,w,s,t)$, i.e., two openly disjoint $(s,t)$-paths $P_1$ and $P_2$. 
We say that they have \emph{type~$1$} if at most one of them contains a vertex of~$T$; otherwise they have \emph{type~$2$}.
We further refine the categorization of type~2 solutions as follows.
Assuming that $P_1$ and $P_2$ both intersect~$T$, let $a_1$ and~$b_1$ be the first and the last vertex on~$P_1$ when traversed from~$s$ to~$t$, and define $a_2$ and $b_2$ for the path~$P_2$ analogously. 
Now, if $T[a_1,b_1]$ and $T[a_2,b_2]$ share no vertices, then we say that $P_1$ and $P_2$ form a solution of \emph{type~$2a$},
otherwise, they have \emph{type~$2b$}.
Henceforth, we will call solutions of type~1 or type~$2a$ \emph{separable}, and solutions of type~2b as \emph{non-separable}.


\medskip
\noindent
{\bf Finding separable solutions.}
We first explain how we can find a cheapest separable solution for our instance.
Roughly speaking, we first guess the vertices $a_1$, $a_2$, and $b_1$, and then compute a minimum-cost flow in an appropriately defined network. 

More precisely, for each possible choice of vertices $a_1$, $b_1$, and $a_2$ in~$T$, 
we build a network~$N_{(a_1,b_1,a_2)}$ as follows. We direct each non-negative edge in~$G$ in both directions. 
Then we direct the edges of the path~$T[a_1,b_1]$ in the direction from~$a_1$ to~$b_1$, 
and we direct all remaining edges in~$T$ ``away'' from~$a_2$ 
(i.e., an edge~$uv$ in~$T$ becomes an arc~$(u,v)$ in the network if $T[a_2,u]$ has fewer edges than $T[a_2,v]$).  
We assign a capacity of~$1$ to each arc and to each vertex\footnote{The standard network flow model can be adjusted by well-known techniques to allow for vertex capacities.} in the network except for~$s$ and~$t$, 
and we retain the cost function~$w$ (meaning that we define $w(\overrightarrow{e})$ as $w(e)$ for any arc~$\overrightarrow{e}$ obtained by directing some edge~$e$). 
Finally, we find a minimum-cost flow of value~$2$ in the network~$N_{(a_1,b_1,a_2)}$
in $O(nm)$ time using the Successive Shortest Path algorithm~\cite{Jewell62,Iri60,BG60}.


\begin{lem}
\label{lem:separablesol}
If there exists a separable solution for~$(G,w,s,t)$ with weight~$k$, then there exists a flow of value~2 having 
cost~$k$ in the network~$N_{(a_1,b_1,a_2)}$ for some $a_1, b_1, a_2 \in V(T)$. 
Conversely, a flow of value~2 and cost~$k$ in~$N_{(a_1,b_1,a_2)}$ for some $a_1, b_1, a_2 \in V(T)$ yields a solution for~$(G,w,s,t)$ with weight at most~$k$. 
\end{lem}

\begin{proof}
Suppose first that $P_1$ and $P_2$ are two paths forming a separable solution for our instance. 
We are going to define openly disjoint directed $(s,t)$-paths $\overrightarrow{S_1}$ and $\overrightarrow{S_2}$ 
with weight at most $k=w(P_1)+w(P_2)$
so that 
there exists some network~$N_{(a_1,b_1,a_2)}$ containing both $\overrightarrow{S_1}$ and~$\overrightarrow{S_2}$.
If such paths exist, then they clearly determine a flow of value~2 and cost at most~$k$ in~$N_{(a_1,b_1,a_2)}$, as required.

Let $\overrightarrow{P_1}$ and $\overrightarrow{P_2}$ denote the paths obtained by directing the edges $P_1$ and $P_2$, respectively, 
as they are traversed from~$s$ to~$t$. 
First, if $P_i$ for some $i \in [2]$ does not contain any edge of~$T$, then we simply set $\overrightarrow{S_i}=\overrightarrow{P_i}$. 
Otherwise let $a_i$ and $b_i$ be defined as the first and last vertex of $\overrightarrow{P_i}$ that lies on~$T$,
and define $\overrightarrow{S_i} = \overrightarrow{P_i}[s,a_i] \cup \overrightarrow{T}[a_i,b_i] \cup \overrightarrow{P_i}[b_i,t]$.
We claim that the paths~$\overrightarrow{S_1}$ and~$\overrightarrow{S_2}$ thus defined fulfill the required conditions.

First we show that their total cost does not exceed~$k$.
If $\overrightarrow{S_i}=\overrightarrow{P_i}$ for some $i \in [2]$, then $w(\overrightarrow{S_i})\leq w(P_i)$ is trivial.
Otherwise, by Lemma~\ref{lem:T-min-pathlength} we know $w(\overrightarrow{T}[a_i,b_i]) \leq w(\overrightarrow{P_i}[a_i,b_1])$, implying
\[w(\overrightarrow{S_i})= 
w(\overrightarrow{P_i}[s,a_i])+ w(\overrightarrow{T}[a_i,b_i]) + w(\overrightarrow{P_i}[b_i,t]) \leq w(\overrightarrow{P_i}).
\] 
Hence, in all cases $w(\overrightarrow{S_1}) + w(\overrightarrow{S_2}) \leq w(P_1)+w(P_2)=k$.

We next show that $\overrightarrow{S_1}$ and $\overrightarrow{S_2}$ are  openly disjoint and are contained in some network~$N_{(x,y,z)}$.
We distinguish between three cases. 
First, if neither $P_1$ nor $P_2$ uses any edge of~$T$, then $\overrightarrow{S_1}$ and~$\overrightarrow{S_2}$ are clearly 
openly disjoint, and contained in every network~$N_{(x,y,z)}$.
Second, assume that exactly one of the paths uses an edge of~$T$; 
by symmetry we may assume that $P_1 \cap E(T) \neq \emptyset$ and $P_2 \cap E(T) = \emptyset$. 
Observe that $\overrightarrow{S_1}$ is then contained in the network~$N_{(a_1,b_1,z)}$ for each $z \in V(T)$ by construction, and 
so is $\overrightarrow{S_2}$.
Since $P_1$ and $P_2$ form a separable solution, we know that $P_2$ cannot contain any vertex of $T[a_1,b_1]$, 
and therefore $\overrightarrow{S_1}$ and $\overrightarrow{S_2}$ are openly disjoint.

Third, if both~$P_1$ and~$P_2$ use edges of~$T$, then the vertices $a_1$, $a_2$, $b_1$ and $b_2$ are all defined.
Since $P_1$ and $P_2$ form a separable solution, we have that $T[a_1,b_1]$ and $T[a_2,b_2]$ do not intersect. 
Consequently, $\overrightarrow{S_1}$ and~$\overrightarrow{S_2}$ are openly disjoint. 
Furthermore, both~$\overrightarrow{S_1}$ and~$\overrightarrow{S_2}$ are contained in the network~$N_{(a_1,b_1,a_2)}$ by its construction,
proving our claim. 

\smallskip
For the other direction, assume that there is a flow of value~2  in $N_{(a_1,b_1,a_2)}$ for some vertices~$a_1,b_1,$ and $a_2$ in~$T$. 
Such a flow must flow through two openly disjoint path from~$s$ to~$t$, due to the unit capacities in~$N_{(a_1,b_1,a_2)}$.
Moreover, by the conservativeness of~$w$ (and since each edge of~$T$ is directed in only one direction), there are no cycles with negative total cost in the network.
This implies that the cost of these two paths is at most the cost of our flow. 
Hence, the lemma follows.  
\end{proof}

%\medskip
\noindent
{\bf Finding non-separable solutions.}
To find a cheapest one among all non-separable solutions, we need a more involved approach based on important structural observations that allow for efficient dynamic programming. 

Let $P_1$ and $P_2$ be two paths forming a non-separable solution we aim to find. 
We start by guessing the vertices~$a_1, b_1, a_2$, and $b_2$, similarly as before.
Note that by the definitions of these four vertices, either $a_1\neq a_2$ or $s=a_1=a_2$, and similarly, either $b_1 \neq b_2$ or $t=b_1=b_2$.
Let~$X$ denote the intersection of~$T[a_1,b_1]$ and~$T[a_2,b_2]$; for simplicity, we will assume that deleting the edges of the path~$X$ from~$T$ leaves $a_1$ and $a_2$ in the same connected component, as well as~$b_1$ and~$b_2$, otherwise we simply rename the vertices~$a_2$ and~$b_2$ (switching their definitions).  
Our goal is to compute two $(\{a_1,a_2\},\{b_1,b_2\})$-paths $Q_1$ and $Q_2$ in $G$ with minimum total weight  that are permissively disjoint. 
This computation is at the heart of our algorithm, and takes up most of the remainder of this section. 

Given paths~$Q_1$ and $Q_2$, we next compute four paths from~$\{s,t\}$ to~$\{a_1,b_1,a_2,b_2\}$ in the graph $G_{(a_1,b_1,a_2,b_2)}=G-E(T)-(V(T) \setminus \{a_1,a_2,b_1,b_2\})$ with minimum length such that two paths have~$s$ as an endpoint, the other two have~$t$ as an endpoint, and no other vertex appears on more than one path. 
This can be done by constructing a network $N_{(a_1,b_1,a_2,b_2)}$: we direct each edge of $G_{(a_1,b_1,a_2,b_2)}$ in both directions, add a new source vertex~$s^\star$ as well as a new sink vertex~$t^\star$. We also add the arcs~$(s^\star,s)$ and $(s^\star,t)$ of capacity~$2$, and the arcs $(a_i,t^\star)$ and~$(b_i,t^\star)$ for $i=1,2$  with capacity~1.\footnote{In the degenerate case when $s=a_1=a_2$ or $t=b_1=b_2$ this yields two parallel edges from~$s$ or $t$ to $t^\star$.
} We also assign capacity~1 to all other arcs, and also to each vertex of~$V(G)\setminus \{s,t\}$. 
We retain the cost function~$w$, and we compute a minimum-cost flow of value~$4$ in this network. 

\begin{lem}
\label{lem:type2bsol}
Let $Q_1$ and $Q_2$ be two permissively disjoint $(\{a_1,a_2\},\{b_1,b_2\})$-paths in~$G$ with minimum total weight, and 
let $w^\star$ be the minimum cost of a flow of value~4 in the network~$N_{(a_1,b_1,a_2,b_2)}$. 
If there exists a non-separable solution for~$(G,w,s,t)$ with cost~$k$, then $w^\star+w(Q_1)+w(Q_2) \leq k$. 
Conversely, 
given a flow of value~4 in the network~$N_{(a_1,b_1,a_2,b_2)}$ with cost~$w^\star$, 
we can find a solution with cost at most 
$w^\star+w(Q_1)+w(Q_2)$ in linear time. 
\end{lem}
\begin{proof}
Suppose that $P_1$ and $P_2$ are two paths forming a non-separable solution for our instance having cost~$k$. 
Clearly, $P_1[a_1,b_1]$ and $P_2[a_2,b_2]$ are openly disjoint, so 
$w(Q_1)+w(Q_2) \leq w(P_1[a_1,b_1])+w(P_2[a_2,b_2])$.
Moreover, the four paths $P_i[s,a_i]$ and $P_i[b_i,t]$, $i \in [2]$, together with the arcs incident to~$s^\star$ and~$t^\star$ yield a flow of value~4 in~$N_{(a_1,b_1,a_2,b_2)}$, which implies
$w^\star \leq \sum_{i \in [2]} w(P_i[s,a_i])+w(P_i[b_i,t])$. This yields $w^\star + w(Q_1)+w(Q_2) \leq w(P_1)+w(P_2)=k$ as required.

\smallskip
Suppose now that $Q_1$ and $Q_2$ are two permissively disjoint $(\{a_1,a_2\},\{b_1,b_2\})$-paths in~$G$, and let $P^s_1$ and $P^s_2$ ($P^t_1$ and $P^t_2$) be the two paths
indicated by the flow of value~4 in~$N_{(a_1,b_1,a_2,b_2)}r$ from~$s$ (from~$t$, respectively) to two vertices  of $\{a_1,b_1,a_2,b_2\}$.

First, if the set of the endpoints of $P^s_1$ and $P^s_2$ is $\{s,a_i,b_j\}$ for some $i,j \in [2]$ then, clearly, the set of endpoints of $P^t_1$ and $P^t_2$ is $\{t,a_{3-i},b_{3-j}\}$. Hence, the union of the paths $T[a_1,a_2]$ and $T[b_1,b_2]$ (using only edges of~$E(T)$) as well as the four paths $P^s_i$ and $P^t_i$ for $i \in [2]$ (using only edges of~$E(G) \setminus E(T)$) yields two openly disjoint $(s,t)$-paths $S_1$ and $S_2$. Moreover,
by the second statement of Lemma~\ref{lem:T-min-pathlength}
we get
\[
w(S_1)+w(S_2) =w^\star + w(T[a_1,a_2])+w(T[b_1,b_2]) \leq w^\star + w(Q_1)+w(Q_2).
\]
Note that the paths~$S_1$ and~$S_2$ can be computed in linear time, given our flow of value~4. Hence, the second statement of the lemma holds.

Second, suppose that the set of the endpoints of $P^s_1$ and $P^s_2$ is $\{s,a_1,a_2\}$ (the case when the two paths starting from~$s$ lead to~$b_1$ and~$b_2$ is symmetric). 
Then the set of the endpoints of $P^t_1$ and $P^t_2$ is $\{t,b_1,b_2\}$. 
We need to apply Lemma~\ref{lem:uncrossing} twice. 

First, if $a_1 \neq a_2$, then we apply Lemma~\ref{lem:uncrossing} for $P^s_1$, $P^s_2$, and $Q_1$ and $Q_2$. 
Note that the conditions of Lemma~\ref{lem:uncrossing} hold, since $V(P^s_1 \cup P^s_2) \cap V(T)=\{a_1,a_2\}$, and  $Q_1$ and $Q_2$ are locally cheapest $(\{a_1,a_2\},\{b_1,b_2\})$-paths, as they have minimum total weight (recall Observation~\ref{obs:locally-cheapest}).
Hence, we obtain in linear time two permissively disjoint $(s,\{b_1,b_2\})$-paths $Q'_1$ and $Q'_2$ in~$G$ whose weight is at most $\sum_{i \in [2]} w(P^s_i)+w(Q_i)$. 
If $a_1=a_2$, then we also know $s=a_1=a_2$, and thus 
$P^s_1$ and $P^s_2$ are paths of length~0. In this case we simply define $Q'_1=Q_1$ and $Q'_2=Q_2$.

We next check whether $Q'_1$ and $Q'_2$ are locally cheapest, and if not, we apply Observation~\ref{obs:locally-cheapest}  to compute $\Amend(Q'_1,Q'_2)$,
obtaining two locally cheapest, permissively disjoint $(s,\{b_1,b_2\})$-paths $Q''_1$ and $Q''_2$ with weight at most $w(Q'_1)+w(Q'_2)$.
By our remarks following Observation~\ref{obs:locally-cheapest}, this can be done in linear time.

If $b_1\neq b_2$, then we apply Lemma~\ref{lem:uncrossing} once again, for paths  $P^t_1$, $P^t_2$, and $Q''_1$ and $Q''_2$. Observe again that the conditions of Lemma~\ref{lem:uncrossing} hold, since $V(P^t_1 \cup P^t_2) \cap V(T)=\{b_1,b_2\}$, and  $Q''_1$ and $Q''_2$ are locally cheapest $(\{b_1,b_2\},s)$-paths.
Hence, we obtain in linear time two permissively disjoint $(t,s)$-paths $S_1$ and $S_2$ in~$G$ such that
\begin{align}
\label{eq:sol2b-weights}
w(S_1)+w(S_2) & \leq \sum_{i \in [2]} w(P^t_i)+w(Q''_i) \leq 
\sum_{i \in [2]} w(P^t_i)+w(Q'_i) \leq 
\sum_{i \in [2]} w(P^t_i)+w(Q_i) + w(P^s_i) 
\\ \notag
&= w^\star + w(Q_1)+w(Q_2).   
\end{align}
Thus, $S_1$ and $S_2$ form a solution for our instance of \DISP{} with weight at most $w^\star + w(Q_1)+w(Q_2)$, as promised.

If $b_1=b_2$, then we also know $t=b_1=b_2$, and thus 
$P^t_1$ and $P^t_2$ are paths of length~0. In this case $Q''_1$ and $Q''_2$ are two permissively disjoint $(s,t)$-paths, so we simply define $S_1=Q''_1$ and $S_2=Q''_2$.
Note that Inequality~\ref{eq:sol2b-weights} still holds, and thus $S_1$ and $S_2$ form a solution for~$(G,w,s,t)$ as promised.

Note that applying Lemma~\ref{lem:uncrossing} (twice) and amending all shortcuts in a pair of paths can be performed in linear time.
This finishes the proof of the lemma.
\end{proof}


Using Lemmas~\ref{lem:separablesol} and~\ref{lem:type2bsol} we immediately obtain that Algorithm~\ref{alg:DISP} 
solves \DISP{} in polynomial time on inputs where the negative-weight edges form a tree, proving Theorem~\ref{thm:DISP-main}. 
Algorithm~\ref{alg:DISP} relies on a subroutine that computes, for any four vertices $a_1$, $b_1$, $a_2$, and $b_2$ in~$T$ 
two permissively disjoint $(\{a_1,a_2\},\{b_1,b_2\})$-paths of minimum total length; 
%Sections~\ref{sec:partsol} and~\ref{sec:permdis}, building on structural observations from Section~\ref{sec:properties}, describe this subroutine in detail.
Section~\ref{sec:partsol}, building on structural observations from Section~\ref{sec:properties}, describes this subroutine in detail;
its existence is stated in Corollary~\ref{cor:perm-disjoint-paths}. 


\begin{varalgorithm}{STDP}
\caption{Solving \DISP{} if negative edges span a tree~$T$. 
}
\label{alg:DISP}
\begin{algorithmic}[1]
\Require{An instance $(G,w,s,t)$ where $w$ is conservative on~$G$, and $G[E^-]$ is a tree.}
\Ensure{A solution for $(G,w,s,t)$ with minimum weight, or $\varnothing$ if no solution exist.}
\State Let $\mathcal{S}=\emptyset$.
\ForAll{$a_1 \in V(T)$}
	\ForAll{$b_1 \in V(T)$}
		\ForAll{$a_2 \in V(T)$}
			\If{$\exists$ a flow~$f$ of value~$2$ in $N_{(a_1,b_1,a_2)}$}
				\State Construct a solution~$(S_1,S_2)$ from~$f$ using Lemma~\ref{lem:separablesol}. \label{line:DISP-separable-output}
				\State $\mathcal{S} \leftarrow (S_1,S_2)$.
			\EndIf 
			\ForAll{$b_2 \in V(T)$}
				\If{$\exists$ a flow~$f$ of value~$4$ in $N_{(a_1,b_1,a_2,b_2)}$}
					\If{$\exists$ two permissively disjoint $(\{a_1,a_2\},\{b_1,b_2\})$-paths} 
						\State Compute two permissively disjoint $(\{a_1,a_2\},\{b_1,b_2\})$-paths $Q_1$ and $Q_2$. \\
													\Comment{Use Algorithm~\ref{alg:PermDisj} in Section~\ref{sec:partsol}} 
						\State Construct a solution~$(S_1,S_2)$ from~$f$, $Q_1$, and $Q_2$ using Lemma~\ref{lem:type2bsol}. \label{line:DISP-nonseparable-output}
						\State $\mathcal{S} \leftarrow (S_1,S_2)$.						
					\EndIf
				\EndIf
			\EndFor			
		\EndFor
	\EndFor
\EndFor 
\If{$\mathcal{S}=\emptyset$} {\bf return} $\varnothing$.
\Else{ Let $S^\star$ be the cheapest pair among those in $\mathcal{S}$, and {\bf return} $S^\star$.} 
\EndIf
\end{algorithmic}
\end{varalgorithm}



\subsection{Properties of a non-separable solution}
\label{sec:properties}
%Let us now turn our attention to the core of our algorithm for \DISP{}. 
Let $a_1, a_2, b_1,$ and $b_2$ be vertices on~$T$ such that $T[a_1,b_1]$ and $T[a_2,b_2]$ intersect in a path~$X$ with one component of~$T-X$ containing $a_1$ and $a_2$, and the other containing $b_1$ and $b_2$.
Let the vertices on~$X$ be $x_1,\dots,x_r$ with $x_1$ being the closest to~$a_1$ and~$a_2$. 
We will use the notation $A_i=T[a_i,x_1]$ and $B_i=T[b_i,x_r]$ for each~$i \in [2]$.

For convenience, for any path $Q$ that has $a_i \in \{a_1,a_2\}$ as its endpoint,  we will say that $Q$ \emph{starts} at $a_i$ and \emph{ends} at its other endpoint. 
Accordingly, for vertices $u,v \in V(Q)$ we say that $u$ \emph{precedes} $v$  on~$Q$, or equivalently, $v$ \emph{follows} $u$ on~$Q$, if $u$ lies on $Q[a_i,v]$. 
When defining a vertex as the ``first'' (or ``last'') vertex with some property on~$Q$ or on a subpath~$Q'$ of $Q$ then, unless otherwise stated, we mean the vertex on~$Q$ or on~$Q'$ that is closest to~$a_i$ (or farthest from~$a_i$, respectively) that has the given property.

We begin with the following observation about the paths $A_1$, $A_2$, $B_1$, and $B_2$.
\begin{lem}
\label{lem:starting}
Let $Q_1$ and $Q_2$ be two permissively disjoint $(\{a_1,a_2\},\{b_1,b_2\})$-paths in~$G$ with minimum total weight.
For each $i \in [2]$, neither $V(A_i) \setminus \{x_1\}$ nor
  $V(B_i) \setminus \{x_r\}$ can contain vertices both from~$Q_1$ and~$Q_2$.
\end{lem}

\begin{proof}
W.l.o.g.\ we suppose $a_i,b_i \in V(Q_i)$ for $i \in [2]$. See Figure~\ref{fig:starting} for an illustration of the proof.
For the sake of contradiction, assume first that there is a vertex on~$A_i$ other than~$x_1$ that is contained in~$Q_{3-i}$; 
by symmetry, we may also assume that $i=1$.
Note that in this case $a_1 \neq a_2$.
Let $z$ be the vertex of~$Q_2$ that is closest to~$a_1$ on~$T[a_1,x_1]$; then $V(T[a_1,z])\cap V(Q_2)=\{z\}$. 
Moreover, for each vertex $z'$ on~$T[a_1,z]$ that is on~$Q_1$, we must have $T[a_1,z'] \subseteq Q_1$, because $Q_1$ and $Q_2$ are locally cheapest.

% Figure environment removed


Since $z \in V(Q_2)$ and $z$ is not on~$T[a_2,b_2]$, using that $Q_1$ and $Q_2$ are locally cheapest, 
we know that there exists a vertex of~$Q_1$ on~$T[a_2,b_2]$. 
Let $y$ be the vertex of~$Q_1$ on~$T[a_2,b_2]$ that is closest to~$a_2$; then $V(T[a_2,y]) \cap V(Q_1)=\{y\}$.
As before, for each vertex $y'$ on~$T[a_2,y]$ that is on~$Q_2$, we must have $T[a_2,y'] \subseteq Q_2$, because $Q_1$ and $Q_2$ are locally cheapest.

These facts imply that $Q_1[y,b_1]$ shares no vertices with~$T[a_1,z]$, and similarly, $Q_2[z,b_2]$ shares no vertices with~$T[a_2,y]$. 
Define paths 
\begin{align*}
    S_1 &=T[a_1,z] \cup Q_2[z,b_2], \\
    S_2 &=T[a_2,y] \cup Q_1[y,b_1]. 
\end{align*}
Observe that $S_1$ and $S_2$ are permissively disjoint $(\{a_1,a_2\},\{b_1,b_2\})$-paths due to the permissive disjointness of $Q_1$ and $Q_2$ 
our observation above. 

Using statement~(2) of Lemma~\ref{lem:T-min-pathlength} for paths~$Q_1[a_1,y]$ and~$Q_2[a_2,z]$, we obtain
\begin{align*}
 w(S_1)+w(S_2) &= w(T[a_1,z])+ w(Q_2[z,b_2])+w(T[a_2,y])+ w(Q_1[y,b_1])  \\
  & < w(Q_1[a_1,y])+w(Q_2[a_2,z]) + w(Q_2[z,b_2])+ w(Q_1[y,b_1])  = w(Q_1)+w(Q_2)
\end{align*}
where the inequality follows from the fact that $T[a_1,y]$ and $T[a_2,z]$ properly intersect each other (since both contain~$x_1$).
This contradicts our definition of $Q_1$ and $Q_2$. 
Thus, for each $i \in [2]$, we have proved that $Q_1$ and~$Q_2$ cannot both contain vertices of $V(A_i) \setminus \{x_1\}$. 

The second claim of the lemma follows by symmetry (i.e., by switching the roles of~$s$ and~$t$, which amounts to switching the roles of $A_i$ and~$B_i$ as well as that of~$x_1$ and~$x_r$).
\end{proof}

To define the next important property of non-separable solutions for our instance, we need some additional notation.
For each $i \in [r]$, let $T_i$ be the maximal subtree of~$T$ containing~$x_i$ but no other vertex of~$X$.
For some $i$ and~$j$ with~$1 \leq i\leq j \leq r$, we also define $T_{(i,j)}=\bigcup_{i \leq h \leq j} T_h$.

\begin{defn}[\bf $X$-monotone path]
A path~$Q$ starting at $a_1$ or $a_2$ is \emph{$X$-monotone} 
if for any vertices $u_1$ and $u_2$ on~$Q$ such that $u_1 \in V(T_{j_1})$ and $u_2 \in V(T_{j_2})$ for some $j_1<j_2$ 
it holds that $u_1$ precedes $u_2$ on~$Q$.
\end{defn}
The following lemma states that our solution must use $X$-monotone paths.

\begin{lem}
\label{lem:X-monotone}
Let $Q_1$ and $Q_2$ be two permissively disjoint $(\{a_1,a_2\},\{b_1,b_2\})$-paths in~$G$ with minimum total weight.
Then both $Q_1$ and $Q_2$ are $X$-monotone.
%, and suppose $a_i,b_i \in V(Q_i)$ for $i \in [2]$. 
%Suppose that $Q_i$ contains vertices $u_1$ and $u_2$, with $u_2$ not on $Q_i[a_i,u_1]$.  
%If $u_1 \in T_{j_1}$ and $u_2 \in T_{j_2}$, then $j_1  \leq j_2$.
\end{lem}

\begin{proof}
W.l.o.g.\ we suppose $a_i,b_i \in V(Q_i)$ for $i \in [2]$.
For vertices $u_1 \in V(T_{j_1})$ and $u_2 \in V(T_{j_2})$ for some indices $j_1  < j_2$, we  
say that $(u_1,u_2)$ is a  \emph{reversed pair}, if $u_1$ and $u_2$ both appear on~$Q_i$  for some $i \in [2]$, but $u_2$ precedes $u_1$ on~$Q_i$. The statement of the lemma is that no reversed pair exists for $Q_1$ or $Q_2$. 
Assume the contrary, and choose $u_1$ and $u_2$ so that 
$u_1$ is as close to~$x_1$ as possible (i.e., it minimizes $|T[u_1,x_1]|$),
and subject to that, $u_2$ is as close to~$x_r$ as possible. By symmetry, we may assume that $u_1$ and $u_2$ are both on~$Q_1$. 
See Figure~\ref{fig:monotone} for an illustration.

First note that $T[a_1,u_1] \not\subseteq Q_1$ (because $u_2$ precedes~$u_1$), and therefore
$T[a_1,u_1]$ must contain a vertex of $Q_2$, since $Q_1$ and $Q_2$ are locally cheapest.
Let $v$ be a vertex on $T[a_1,u_1]$ closest to~$u_1$ among those contained in~$V(Q_2)$.  
Similarly, note that $T[u_2,b_1] \not\subseteq Q_2$, and therefore 
$T[u_2,b_1]$ must contain a vertex of $Q_2$, since $Q_1$ and $Q_2$ are locally cheapest.
Let $v'$ be a vertex on $T[u_2,b_1]$ closest to~$u_2$ among those contained in~$V(Q_1)$.  

% Figure environment removed


By Lemma~\ref{lem:starting} we know that $v$ lies on~$T[x_1,u_1]$, and $v'$ lies on~$T[u_2,x_r]$.
Note that no inner vertex of~$T[v,u_1]$ is contained in $V(Q_2)$ by the definition of~$v$. 
Let $e_1$ denote the edge $T[x_1,u_1]$ incident to~$u_1$, and let $u'_1$ be the other endpoint of~$e_1$. 
Observe that $e_1 \in Q_1$ is not possible, because then $(u'_1,u_2)$ would also be a reversed pair, which would contradict our choice of~$(u_1,u_2)$
(since $u'_1$ is closer to~$x_1$ than~$u_1$). 
Using that $Q_1$ and $Q_2$ are locally cheapest, we have 
that no inner vertex of $T[v,u_1]$ is contained in $V(Q_1 \cup Q_2)$.
Reasoning analogously, it also follows that no inner vertex of $T[u_2,v']$ is contained in $V(Q_1 \cup Q_2)$.

Note that $v \in V(T_j)$ for some $j\leq j_1$, and $v' \in V(T_{j'})$ for some $j' \geq j_2$, implying $j<j'$. 
Moreover, $v$ is closer to~$x_1$ than~$u_1$. Hence, by our choice of $(u_1,u_2)$, the pair $(v,v')$ is not a reversed pair. 
This implies that $v$ is on $Q_2[a_2,v']$.
In particular, $Q_2[a_2,v]$ and $Q_2[v',b_2]$ are disjoint (because $v \neq v'$).
Define paths 
\begin{align*}
    S_1 &=Q_1[a_1,u_2] \cup T[u_2,v'] \cup Q_2[v',b_2], \\
    S_2 &=Q_2[a_2,v] \cup T[v,u_1] \cup Q_1[u_1,b_1]. 
\end{align*}
Observe that $S_1$ and $S_2$ are permissively disjoint $(\{a_1,a_2\},\{b_1,b_2\})$-paths due to the permissive disjointness of $Q_1$ and $Q_2$ 
and our observations on $T[v,u_1]$ and $T[u_2,v']$. 
Moreover, the walk 
\[ W=T[v,u_1] \cup Q_1[u_1,u_2] \cup T[u_2,v'] \cup Q_2[v',v]
\]
is closed and does not contain any edge of~$T$ more than once, since $Q_1$ and $Q_2$ share no edges, and no edge of $T[v,u_1] \cup T[u_2,v']$ 
is contained in~$Q_1 \cup Q_2$. Thus, $w(W)\geq 0$ by Lemma~\ref{lem:closed-walk}. This implies
\begin{align*}
w(S_1)+w(S_2) &= w(Q_1)+w(Q_2)-w(W \setminus (T[v,u_1] \cup T[u_2,v'])) + w(T[v,u_1] \cup T[u_2,v']) \\
&< w(Q_1)+w(Q_2),
\end{align*} 
because $w(T[v,u_1] \cup T[u_2,v']) <0$. This contradicts our definition of $Q_1$ and $Q_2$.
Hence, we have proved that no reversed pair exists, and thus both~$Q_1$ and~$Q_2$ are $X$-monotone.
\end{proof}

The next property we establish is the following. 

\begin{defn}[\bf Plain path]
\label{def:plain}
A path~$Q$ is \emph{plain}, if whenever $Q$ contains some $x_i \in V(X)$, 
then the vertices of~$Q$ in~$T_i$ induce a path in~$T_i$. 
In other words, if vertices $x_j \in V(X)$ and $u \in V(T_j)$  
both appear on~$Q$, then $T[u,x_j] \subseteq Q$.
\end{defn}

\begin{lem}
\label{lem:plain}
Let $Q_1$ and $Q_2$ be two permissively disjoint $(\{a_1,a_2\},\{b_1,b_2\})$-paths in~$G$ with minimum total weight.
Then both~$Q_1$ and~$Q_2$ are plain.
\end{lem}

\begin{proof}
W.l.o.g.\ we suppose $a_i,b_i \in V(Q_i)$ for $i \in [2]$.
Suppose for contradiction that $x_j \in V(Q_1)$ and $u \in V(Q_1) \cap V(T_j)$, but $T[u,x_j]  \not \subseteq Q_1$;
the case when $u$ and $x_j$ are on~$Q_2$ is symmetric. 
Clearly, $u \neq x_j$.
Since $Q_1$ and $Q_2$ are locally cheapest, by $T[u,x_j] \not \subseteq Q_1$ we know that
$T[u,x_j]$ must contain a vertex of~$Q_2$. 

Let $z$ be the closest vertex on $T[u,x_j]$ to~$u$ that appears on~$Q_2$, and let $z'$ be the closest vertex on~$T[u,z]$ to~$z$  that appears on~$Q_1$. Then no inner vertex of $T[z,z']$ contained in $V(Q_1 \cup Q_2)$. 
Since $Q_1$ and $Q_2$ are locally cheapest, it follows from the definition of~$z$ and $z'$ that $T[u,z'] \subseteq Q_1$. 

We now distinguish between two cases; see Figure~\ref{fig:plain} for an illustration. 

% Figure environment removed


{\bf Case A.} First assume that $u$ precedes $x_j$ on~$Q_1$.
By $T[u,z'] \subseteq Q_1$ this implies that $z'$ also precedes $x_j$ on~$Q_1$. 

Using again that $Q_1$ and $Q_2$ are locally cheapest, by $T[a_1,x_j] \notin Q_1$ we know that $[Tx_1,x_j]$ must contain a vertex of~$Q_2$. Let $x_h$ be the vertex on~$T[x_1,x_j]$ closest to~$x_j$. We show that neither $Q_1$ nor $Q_2$ contains an inner vertex of $T[x_h,x_j]$. For~$Q_2$, this follows from the definition of~$x_h$. For $Q_1$, this follows from  Lemma~\ref{lem:X-monotone}: 
since $u$ precedes~$x_j$ on~$Q_1$, the $X$-monotonicity of~$Q_1$ implies that $Q_1$ enters $x_j$ through an edge with both endpoints in~$T_j$, 
so in particular, $x_j x_{j-1} \notin Q_1$; using that $Q_1$ and $Q_2$ are locally cheapest, it follows that $Q_1$ contains no inner vertices of~$T[x_h,x_j]$.
By the $X$-monotonicity of~$Q_2$, we also know that $z$ follows $x_h$ on~$Q_2$.

Define the paths
\begin{align*}
S_1 &= Q_1[a_1,z'] \cup T[z',z] \cup Q_2[z,b_2], \\
S_2 &= Q_2[a_2,x_h] \cup T[x_h,x_j] \cup Q_1[x_j,b_1].
\end{align*}
Observe that $S_1$ and $S_2$ are permissively disjoint $(\{a_1,a_2\},\{b_1,b_2\})$-paths due to the permissive disjointness of $Q_1$ and $Q_2$, and our observations on $T[z,z']$ and $T[x_h,x_j]$. 
Moreover, the walk 
\[ W=Q_1[z',x_j] \cup T[x_j,x_h] \cup Q_2[x_h,z] \cup T[z,z']
\]
is closed and does not contain any edge of~$T$ more than once, since $Q_1$ and $Q_2$ share no edges, and no edge of $T[z,z'] \cup T[x_j,x_h]$ 
is contained in~$Q_1 \cup Q_2$. Thus, $w(W)\geq 0$ by Lemma~\ref{lem:closed-walk}. This implies
\begin{align*}
w(S_1)+w(S_2) &= w(Q_1)+w(Q_2)-w(W \setminus (T[z,z'] \cup T[x_j,x_h])) + w(T[z,z'] \cup T[x_j,x_h]) \\
&< w(Q_1)+w(Q_2),
\end{align*} 
because $w(T[z,z'] \cup T[x_j,x_h]) <0$. This contradicts our definition of $Q_1$ and $Q_2$.

{\bf Case B.} 
Assume now that $u$ follows $x_j$ on~$Q_1$; our reasoning is very similar to the previous case.
By $T[u,z'] \subseteq Q_1$ this implies that $z'$ also follows $x_j$ on~$Q_1$. 

Using again that $Q_1$ and $Q_2$ are locally cheapest, by $T[x_j,b_1] \notin Q_1$ we know that $T[x_j,x_r]$ must contain a vertex of~$Q_2$. Let $x_h$ be the vertex on~$T[x_j,x_r]$ closest to~$x_j$. We show that neither $Q_1$ nor $Q_2$ contains an inner vertex of $T[x_j,x_h]$. For~$Q_2$, this follows from the definition of~$x_h$. For $Q_1$, this follows from  Lemma~\ref{lem:X-monotone} (implying $x_j x_{j+1} \notin Q_1$) 
and the fact that $Q_1$ and $Q_2$ are locally cheapest.
Note also that by the $X$-monotonicity of~$Q_2$, we know that $z$ precedes $x_h$ on~$Q_2$.

Define the paths
\begin{align*}
S_1 &= Q_1[a_1,x_j] \cup T[x_j,x_h] \cup Q_2[x_h,b_2], \\
S_2 &= Q_2[a_2,z] \cup T[z,z'] \cup Q_1[z,b_1].
\end{align*}
Observe that $S_1$ and $S_2$ are permissively disjoint $(\{a_1,a_2\},\{b_1,b_2\})$-paths due to the permissive disjointness of $Q_1$ and $Q_2$, and our observations on $T[z',z]$ and $T[x_j,x_h]$. 
Moreover, the walk 
\[ W=Q_1[x_j,z'] \cup T[z',z] \cup Q_2[z,x_h] \cup T[x_h,x_j] 
\]
is closed and does not contain any edge of~$T$ more than once, since $Q_1$ and $Q_2$ share no edges, and no edge of $T[z',z] \cup T[x_h,x_j]$ 
is contained in~$Q_1 \cup Q_2$. Thus, $w(W)\geq 0$ by Lemma~\ref{lem:closed-walk}. This implies
\begin{align*}
w(S_1)+w(S_2) &= w(Q_1)+w(Q_2)-w(W \setminus (T[z',z] \cup T[x_h,x_j])) + w(T[z',z] \cup T[x_h,x_j]) \\
&< w(Q_1)+w(Q_2),
\end{align*} 
because $w(T[z,z'] \cup T[x_j,x_h]) <0$. This contradicts our definition of $Q_1$ and $Q_2$.
\end{proof}

The following observation completes our understanding on how a solution uses paths $A_1$, $A_2$, $B_1$, and~$B_2$.

\begin{lem}
\label{lem:contains-A1orA2}
If $Q_1$ and $Q_2$ are two permissively disjoint $(\{a_1,a_2\},\{b_1,b_2\})$-paths that are locally cheapest and also plain, then one of them contains $A_1$ or~$A_2$, and one of them contains~$B_1$ or~$B_2$.    
\end{lem}

\begin{proof}
W.l.o.g.\ we suppose $a_i,b_i \in V(Q_i)$ for $i \in [2]$.
First, we show that both $x_1$ and $x_r$ are contained in $Q_1 \cup Q_2$. For each $i \in [2]$, let $u_i$ denote the vertex closest to~$x_1$ on~$A_i$ that is contained in~$Q_1 \cup Q_2$.
Similarly, let $v_i$ denote the vertex closest to~$x_r$ on~$B_i$ that is contained in~$Q_1 \cup Q_2$. 
If $V(X) \cap V(Q_1 \cup Q_2) =  \emptyset$, then $u_1, u_2, v_1$ and $v_2$ are four distinct vertices, with at least two of them belonging to the same path~$Q_1$ or $Q_2$, contradicting our assumption that $Q_1$ and $Q_2$ are locally cheapest. 
Consider the vertex $z$ of $V(X) \cap V(Q_1 \cup Q_2)$ closest to~$x_1$ on~$X$.
If $x_1 \notin V(Q_1 \cup Q_2)$, then $u_1, u_2$ and $z$ are three distinct vertices, so at least two of them belong to the same path, $Q_1$ or $Q_2$
%. As $x_1$ lies on the path of~$T$ connecting these two vertices, $x_1 \notin V(Q_1 \cup Q_2)$ 
which contradicts our assumption that $Q_1$ and $Q_2$ are locally cheapest. 
Hence, $x_1 \in V(Q_1 \cup Q_2)$, and an analogous argument shows $x_r \in V(Q_1 \cup Q_2)$.

Now, let $Q_i$ contain $x_1$. As $Q_i$ is plain and $a_i \in V(Q_i)$ we get that $V(A_i) \subseteq V(Q_i)$. As $Q_1$ and $Q_2$ are locally cheapest, we get that $A_i \subseteq Q_i$. The analogous argument shows that $B_i$ is contained in $Q_1$ or $Q_2$. 
\end{proof}


\subsection{Computing partial solutions}
\label{sec:partsol}

In Section~\ref{sec:properties} we have established that two permissively disjoint $(\{a_1,a_2\},\{b_1,b_2\})$-paths of minimum total weight 
are necessarily $X$-monotone, plain, and they form a locally cheapest pair.
A natural approach would be to require these same properties from a partial solution that we aim to compute. 
However, it turns out that the property of $X$-monotonicity is quite hard to ensure when building subpaths of a solution. 
The following relaxed version of monotonicity can be satisfied much easier, and still suffices for our purposes:

\begin{defn}[\bf Quasi-monotone path]
\label{def:quasimonotone}
A path~$P$ starting from~$a_1$ or $a_2$ is \emph{quasi-monotone}, if the following holds:
if $x_i \in V(P)$ for some $i \in [r]$, then any vertex in $V(P) \cap \bigcup_{h \in [i-1]} V(T_h)$ precedes $x_i$ on~$P$,
and any vertex in $V(P) \cap \bigcup_{h \in [r] \setminus [i]} V(T_h)$ follows $x_i$ on~$P$.
\end{defn}

\begin{defn}[\bf Well-formed path pair]
Two paths $P_1$ and $P_2$ form a \emph{well-formed pair}, if they are locally cheapest, and both are plain and quasi-monotone.
\end{defn}

We are now ready to define partial solutions, the central notion that our dynamic programming algorithm relies on. 

\begin{defn}[\bf Partial solution]
\label{def:partsol}
Given vertices $u \in  V(T_i)$ and $v \in V(T_j)$ for some $i \leq j$,
two paths~$Q_1$ and $Q_2$ form a \emph{partial solution} $(Q_1,Q_2)$ for $(u,v)$, if 
\begin{itemize}
    \item[(a)] $Q_1$ and $Q_2$ are permissively disjoint $(\{a_1,a_2\},\{u,v\})$-paths; 
    \item[(b)] $Q_1$ and $Q_2$ are a well-formed pair;
    \item[(c)] $Q_1$ ends with the subpath $T[x_i,u]$;
%    \item[(d)] $Q_2$ ends with the subpath $T[v,x_j]$;
    \item[(d)] $V(T_{(i+1,r)}) \cap V(Q_2)\subseteq \{v\}$.
\end{itemize}
\end{defn}

Before turning our attention to the question of how to find partial solutions, let us first show 
how partial solutions enable us to find two permissively disjoint $(\{a_1,a_2\},\{b_1,b_2\})$-paths.
\begin{lem}
\label{lem:partsol-to-permdisj}
Paths $P_1$ and~$P_2$ are permissively disjoint $(\{a_1,a_2\},\{b_1,b_2\})$-paths of minimum weight if and only if
they form a partial solution for~$(b_h,b_{3-h})$ of minimum weight for some \mbox{$h \in [2]$}.
\end{lem}

\begin{proof}
First assume that $P_1$ and~$P_2$ are permissively disjoint $(\{a_1,a_2\},\{b_1,b_2\})$-paths. 
By Lemma~\ref{lem:contains-A1orA2} we know that one of them contains~$B_1$ or~$B_2$, so define~$h$ such that $B_h \subseteq P_1 \cup P_2$.
By Observation~\ref{obs:locally-cheapest} and Lemmas~\ref{lem:X-monotone},~\ref{lem:plain} we know that $P_1$ and~$P_2$ are a well-formed pair.
Note also that condition~(d) holds vacuously, since $b_1,b_2 \in V(T_r)$. 
Therefore, if $P_1$ ends with~$B_h$, then $(P_1,P_2)$ is a partial solution for~$(b_h,b_{3-h})$,
and if $P_2$ ends with~$B_h$, then $(P_2,P_1)$ is a partial solution for~$(b_h,b_{3-h})$.

Since a partial solution for $(b_h,b_{3-h})$ for some~$h \in [2]$ is by definition a pair of two permissively disjoint $(\{a_1,a_2\},\{b_1,b_2\})$-paths,
the lemma follows.
\end{proof}



We are now ready to present our approach for computing partial solutions using dynamic programming.
\medskip

\noindent
{\bf Computing partial solutions: high-level view.}
For each $u \in  V(T_i)$ and $v \in V(T_j)$ for some $i\leq j$,
we are going to compute a partial solution for $(u,v)$ of minimum weight, 
denoted by $F(u,v)$, using dynamic programming;
if there exists no partial solution for $(u,v)$, 
we set $F(u,v)=\varnothing$. We now give an overview of our method; see Algorithm~\ref{alg:PartSol} for more details in pseudo-code.

To compute $F(u,v)$ in a recursive manner, we use the observation that
either the partial solution has a fairly simple structure, 
or it strictly contains a partial solution for~$(u',v')$ for some vertices~$u'$ and~$v'$ with $u' \in V(T_{i'})$ and $i'<i$. 
We can thus try all possible values for~$u'$ and~$v'$, and use the partial solution~$(Q'_1,Q'_2)$ 
we have already computed and stored in $F(u',v')$.
To obtain a partial solution for~$(u,v)$ based on~$Q'_1$ and~$Q'_2$, we append paths to~$Q'_1$ and to~$Q'_2$ so that 
they fulfill the requirements of Definition~\ref{def:partsol}---most importantly, that 
$Q_1$ ends with~$T[x_i,u]$ and that 
$Q_2$ ends at~$v$. 
To this end, we create a path $P_1=Q'_2 \cup T[v',u]$ 
and a path $P_2=Q'_1 \cup R$ where
$R$ is a shortest $(u',v)$-path in a certain auxiliary graph.
Essentially, we use the tree~$T$ for getting from~$v'$ to~$u$, and 
we use the ``remainder of the graph'' for getting from~$u'$ to~$v$; 
the precise definition of the auxiliary subgraph of~$G$ that we use for this purpose is given below in Definition~\ref{def:auxgraph}.
If the obtained path pair $(P_1,P_2)$ is indeed a partial solution for~$(u,v)$, then we store it. 
After trying all possible values for~$u'$ and~$v'$, we select a partial solution that has minimum weight among those we computed.


\begin{defn}[\bf Auxiliary graph]
\label{def:auxgraph}
Let $P \subseteq T$  be a path within~$T$, and let $u$ and $v$ be two vertices. 
Then the \emph{auxiliary graph} $G \langle P,u,v \rangle$ %for $(P,u,v)$ 
denotes the graph 
obtained by deleting the vertex 
set $\bigcup\{ V(T_h): V(T_h) \cap V(P)=\emptyset\}  \cup  V(P) \setminus \{u,v\}$, i.e., 
\[
G \langle P,u,v \rangle = G - \left(\bigcup\{ V(T_h): V(T_h) \cap V(P)=\emptyset\}  \cup  V(P) \setminus \{u,v\} \right).
\]
In other words, we obtain $G \langle P,u,v \rangle$ from~$G$ by deleting all trees~$T_h$ that do not intersect~$P$, and deleting $P$ itself as well,
while taking care not to delete $u$ or $v$.
\end{defn}

\begin{varalgorithm}{{\textsc{PartSol}}}
\caption{Computes a partial solution $F(u,v)$ of minimum weight for $(u,v)$ where $u \in V(T_i)$ and $v \in V(T_j)$ with $i\leq j$. 
}
\label{alg:PartSol}
\begin{algorithmic}[1]
\Require{Vertices $u$ and $v$ where $u \in V(T_i)$ and $v \in V(T_j)$ for some $i\leq j$.}
\Ensure{A partial solution $F(u,v)$ for~$(u,v)$ of minimum weight, or $\varnothing$ it not existent.}
\State Let $\mathcal{S}=\emptyset$.
\ForAll{$h \in [2]$}
	\If{$A_h \cup T[x_1,u]$ is a path}
		\If{$v$ is reachable from~$a_{3-h}$ in~$G \langle A_h \cup T[x_1,u],a_{3-h},v \rangle$} 
			\State Compute a shortest $(a_{3-h},v)$-path $R$ in~$G \langle A_h \cup T[x_1,u],u,a_{3-h},v \rangle$.
			\If{$(A_h \cup T[x_1,u], R )$ is a partial solution for~$(u,v)$}				
				\State $\mathcal{S} \leftarrow (A_h \cup T[x_1,u], R)$.			\label{line:PS-comp1}
			\EndIf				
		\EndIf
	\EndIf 
\EndFor 
\ForAll{$i' \in [i-1]$ and $u' \in V(T_{i'})$}
	\ForAll{$j' \in [i] \setminus [i']$ and $v' \in V(T_{j'})$ such that $T[x_{j'},v'] \cap T[x_i,u]=\emptyset$} \label{line:choosev'}
		\If{$F(u',v') = \varnothing$}  {\bf continue;} 	\label{line:PS-callF}
		\EndIf
		\State Let $(Q'_1,Q'_2)=F(u',v')$.					\label{line:PS-defQ'}
		\If{$v$ is not reachable from~$u'$ in~$G \langle T[v',u],u',v \rangle$}  {\bf continue;}  \label{line:PS-reach}
		\EndIf
		\State Compute a shortest $(u',v)$-path $R$ in~$G \langle  T[v',u],u',v \rangle$.			\label{line:PS-shortest}
		\State Let $P_1=Q'_2 \cup T[v',u]$ and $P_2=Q'_1 \cup R$. 
		\If{$(P_1,P_2)$ is a partial solution for~$(u,v)$}
			\State $\mathcal{S} \leftarrow (P_1,P_2)$. 						%\label{•}-comp2}

		\EndIf
	\EndFor		
\EndFor
\If{$\mathcal{S}=\emptyset$} {\bf return} $\varnothing$.
\Else{ Let $S^\star$ be the cheapest pair among those in $\mathcal{S}$, and {\bf return} $F(u,v):=S^\star$.} \label{line:SP-final}
\EndIf
\end{algorithmic}
%\end{algorithm}
\end{varalgorithm}

Working towards a proof for the correctness of Algorithm~\ref{alg:PartSol},
we start with two simple observations. 
The first one, stated by Lemma~\ref{lem:paths-vs-X} below, essentially says that a path in a partial solution that uses a subtree~$T_h$ for some~$h \in [r]$
should also go through the vertex~$x_h$ whenever possible, that is, unless the other path uses~$x_h$.

\begin{lem}
\label{lem:paths-vs-X}
Let $Q_1$ and $Q_2$ be two permissively disjoint, locally cheapest $(\{a_1,a_2\},\{x_i,v\})$-paths for some $v \in V(T_j)$ where $1 \leq i\leq j \leq r$.
Let $z \in V(T_h)$ for some $h \leq j$ such that $h<j$ or $x_h \in V(T[z,v])$. 
If $z\in V(Q_1 \cup Q_2)$, then $x_h \in V(Q_1 \cup Q_2)$.
\end{lem}

\begin{proof}
W.l.o.g.\ we may assume $z \in V(Q_1)$; let $a$ denote the starting vertex ($a_1$ or $a_2$) of $Q_1$.
Suppose for contradiction that $x_h \notin V(Q_1 \cup Q_2)$; then either $1 \leq h<i$ or $i<h\leq j$. 
In the former case, define $P_1$, $P_2$, and $P_3$ as the three openly disjoint paths leading within~$T$ from $x_h$ to~$a$, to~$x_i$, and to~$z$, respectively. 
In the latter case, define $P_1$, $P_2$, and $P_3$ as the three openly disjoint paths leading within~$T$ from $x_h$ to~$x_i$, to~$v$, and to~$z$, respectively;
observe that such paths exist due to the condition that either $h<j$ or $x_h \in V(T[z,v])$. 
Let $z_1$, $z_2$, and $z_3$ denote the vertex closest to~$x_h$ on $T_1$, $T_2$, and $T_3$, respectively, that is contained in~$V(Q_1 \cup Q_2)$; 
note that $z_1$, $z_2$ and $z_3$ are three distinct vertices, with no vertex of~$V(Q_1 \cup Q_2)$ lying between any two of them on~$T$. 
Since at least two vertices from $\{z_1,z_2,z_3\}$ belong to the same path $Q_1$ or~$Q_2$, this contradicts the assumption that $Q_1$ and $Q_2$ are locally cheapest.
\end{proof}

As a consequence of Lemma~\ref{lem:paths-vs-X}, applied with $a_1$ taking the role of~$z$ and $x_1$ taking the role of~$x_h$, 
we get that every partial solution for some~$(u,v)$ must contain~$x_1$; using that both paths in a partial solution must be plain, 
we get the following fact.
\begin{observation}
\label{obs:contains-A1orA2}
Given two vertices $u \in  V(T_i)$ and $v \in V(T_j)$ for some $i\leq j$,
let $(Q_1,Q_2)$ be a partial solution for~$(u,v)$. 
Then either $Q_1$ or $Q_2$ contains $A_1$ or $A_2$.
\end{observation}
%%%%%%%%%%%%%%%%%%%%%%%%%%%%%%%%%%%%%%%%%%%%%%%%%%%%%%%%%%%%%%%%%%%%%%%%%%

We are now ready to prove that Algorithm~\ref{alg:PartSol} is correct.

\begin{lem}
\label{lem:partsol-alg}
Let $i,j \in [r]$ with $i\leq j$. For any $u \in V(T_i)$ and $v \in V(T_j)$,
if there exists a partial solution for $(u,v)$, then 
Algorithm~\ref{alg:PartSol} puts a partial solution into~$\mathcal{S}$ with minimum weight.
If no partial solution for~$(u,v)$ exists, then Algorithm~\ref{alg:PartSol} sets $F(u,v)=\varnothing$.
\end{lem}

\begin{proof} 
Using induction, we will assume that the lemma holds for all index pairs~$(i',j')$ where $i'<i$; this holds vacuously for $i=1$.
Let $(Q_1,Q_2)$ be a partial solution for~$(u,v)$ with minimum weight. 
We distinguish between two cases. 

% Figure environment removed

\noindent
{\bf Case A:} $Q_2$ contains no vertex of~$T[x_1,x_i]$. See Figure~\ref{fig:partsolA} for an illustration.
Due to Observation~\ref{obs:contains-A1orA2}, 
we know that $Q_1$ contains $A_h$ for some $h \in [2]$; let us fix this value of~$h$.
Since $Q_1$ and~$Q_2$ are locally cheapest and~$Q_1$ ends with~$T[x_i,u]$, we also know that $Q_1$ must contain~$T[x_1,x_i]$.
Therefore, we obtain $Q_1=A_h \cup T[x_1,u]$. 

%Let $\R=Q_2 \setminus T[v,x_j]$. 
First, $Q_1$ and $Q_2$ may share a vertex only if $x_1=a_1=a_2$.
We also know that $Q_2$ may only contain vertices of~$T_{(1,i)}$ that are not on the path~$A_h \cup T[x_1,u]$, 
except possibly for the starting vertex~$a_{3-h}$ (in case $a_1=a_2$). 
%Second, since $Q_2$ is quasi-monotone, it contains no vertices from~$T_h$ for any $h>j$.
%Third,
Second, by condition~(d) in the definition of a partial solution, we also have that $Q_2$ contains no vertices 
from~$T_h$ for any $i<h \leq r$ other than its endpoint~$v$.
This means that $Q_2$ is a path in the auxiliary graph~$G \langle A_h \cup T[x_1,u],a_{3-h},v \rangle$.

Conversely, we claim that $(A_h \cup T[x_1,u], R)$ is a partial solution for~$(u,v)$
for any shortest $(a_{3-h},v)$-path~$R$ in~$G \langle A_h \cup T[x_1,u],a_{3-h},v \rangle$.
First, $A_h \cup T[x_1,u]$ is an $(a_h,u)$-path, as this is explicitly checked by the algorithm, and it is a subpath of~$T_{(1,i)}$ by $u \in V(T_i)$. 
By the definition of $G \langle A_h \cup T[x_1,u],a_{3-h},v \rangle$, 
we also know that  $A_h \cup T[x_1,u]$ and $R$ are two permissively disjoint $(\{a_1,a_2\},\{u,v\})$-paths.

Let us show that $A_h \cup T[x_1,u]$ and $R$ are also locally cheapest: 
assuming that there exists a shortcut $T[q,q']$ witnessing the opposite,
then we get that both~$q$ and~$q'$ must be vertices of~$R$ on~$T_{(1,i)}$, and 
the path~$T[q,q']$ must share no vertices with~$A_h \cup T[x_1,u]$.
%we get $q,q' \in V(R) \cap V(T)$ and thus $V(A_h \cup T[x_1,u]) \cap V(T[q,q'])=\emptyset$, 
Hence, replacing $R[q,q']$ with $T[q,q']$ would yield a path in~$G \langle A_h \cup T[x_1,u],a_{3-h},v \rangle$ whose weight is less than~$w(R)$,
contradicting the fact that $R$ is a shortest path. 
Thus, $A_h \cup T[x_1,u]$ and $R$ are locally cheapest.
They also form a well-formed pair, since 
their quasi-monotonicity and plainness is obvious.
They also satisfy conditions~(c) and~(d) in Definition~\ref{def:partsol}, and thus form a partial solution for~$(u,v)$. 

Therefore, for the right choice of~$h$, 
Algorithm~\ref{alg:PartSol} will put $(A_h \cup T[x_1,u], R)$ into~$\mathcal{S}$ on line~\ref{line:PS-comp1}
where $R$ is a shortest $(a_{3-h},v)$-path in~$G \langle A_h \cup T[x_1,u],a_{3-h},v \rangle$.
Observe that 
\begin{align*}
w(A_h \cup T[x_1,u])+w( R) &=w(Q_1) + w(R)  
 \leq w(Q_1) + w(\R) = w(Q_1)+w(Q_2),
\end{align*}
implying that $(A_h \cup T[x_1,u], R)$ is indeed a partial solution for $(u,v)$ with minimum total weight.
This proves the lemma for Case~A.
%%%%%%%%%%%%%%%%%%%%

\medskip
\noindent
{\bf Case B:} $Q_2$ contains a vertex of~$T[x_1,x_i]$.
Consider the vertex~$x_{i'}$ on $T[x_1,x_{i-1}]$ closest to~$x_i$ that appears on~$Q_2$, and let $u'$ be the last vertex of $Q_2$ in~$T_{i'}$;
since $Q_2$ is plain, we know $T[x_{i'},u'] \subseteq Q_2$.
Let $x_{j'}$ denote the vertex on~$T[x_{i'},x_i]$ closest to $x_{i'}$ that is contained in~$Q_1$; then $i' <j' \leq i$.
As $Q_1$ and $Q_2$ are locally cheapest, we have $T[x_{j'},x_i] \subseteq Q_1$. 
Let $v'$ denote the first vertex of~$Q_1$ in~$T_{j'}$. Since $Q_1$ is plain, we know $T[v',x_{j'}] \subseteq Q_1$.
Note also that $T[x_{j'},v'] \cap  T[x_i,u]=\emptyset$: 
this is trivial if $x_{j'}\neq x_i$, and it follows from the plainness of~$Q_1$ if $x_{j'}= x_i$, because 
then $T[v',x_{j'}]$ and~$T[x_{j'},u]$ are both subpaths of~$Q_1$ and therefore can share no edge.

Let $\Q_1=Q_2 \setminus Q_2[u',v]$ and $\Q_2=Q_1 \setminus Q_1[v',u]$.

% Figure environment removed


\begin{claim} 
\label{clm:partsol-recurse} $(\Q_1,\Q_2)$ is a partial solution for~$(u',v')$.
\end{claim}
\begin{claimproof}
Observe that $\Q_1$ and $\Q_2$ are permissively disjoint $(\{a_1,a_2\},\{u',v'\})$-paths.
It is straightforward to verify that they form a well-formed pair and satisfy condition~(c)
in the definition of partial solutions.
Observe that neither~$Q_1$ nor~$Q_2$ passes through an inner vertex of~$T[x_{i'},x_{j'}]$, by the definition of~$x_{i'}$ and $x_{j'}$.
By Lemma~\ref{lem:paths-vs-X} %(and since $\Q_1 \cup \Q_2$ contains $x_{i'}$ and $x_{j'}$) 
this implies that $V(\Q_2) \cap V(T_h)=\emptyset$ for any $i' < h <j'$.
Recall also that $V(\Q_2) \cap V(T_{j'})=\{v'\}$ by the definition of~$v'$.% and the plainness of~$Q_1$ (and hence of~$\Q_2$). 
Furthermore, since $Q_1$ is quasi-monotone, we also have that $\Q_2$ contains no vertices from~$T_{(j'+1,r)}$,
so $\Q_2$ satisfies condition~(d) as well.
Thus $(\Q_1,\Q_2)$ is a partial solution for~$(u',v')$.
\end{claimproof}

\begin{claim} 
\label{clm:partsol-path}
$\R=Q_2[u',v]$ is a $(u',v)$-path in~$G \langle T[v',u], u',v \rangle$. 
\end{claim}
\begin{claimproof}
It is clear that $\R \subseteq Q_2$ has no common vertices with~$T[v',u] \subseteq Q_1$ by the (permissive) disjointness of~$Q_1$ and~$Q_2$.
Hence it suffices to prove the following:
if $\R$ contains a vertex in~$T_\ell$ other than~$u'$ or~$v$, then $j' \leq \ell \leq i$.
First, $\ell<i'$ is not possible by the quasi-monotonicity of~$Q_2$ (using that $x_{i'}$ is on $Q_2$). 
Second, $\ell>i$ is also not possible, due to condition~(d) for~$Q_2$ in the definition of a partial solution.
Third, since $Q_2$ is plain, we get $\ell \neq i'$ from the definition of~$u'$.
Fourth, recall that 
neither~$Q_1$ nor~$Q_2$ passes through an inner vertex of~$T[x_{i'},x_{j'}]$, by the definition of~$x_{i'}$ and~$x_{j'}$.
Hence, by Lemma~\ref{lem:paths-vs-X}, no vertex of~$Q_1$ or $Q_2$
can be contained in~$\bigcup_{i' < h <j'} V(T_h)$, so $\ell$ is not between $i'$ and $j'$.
Thus, $j' \leq \ell \leq i$ as claimed, and so 
$\R$ is a path in~$G \langle T[v',u],u',v \rangle$. 
\end{claimproof}

\smallskip
Consider the iteration when Algorithm~\ref{alg:PartSol} picks the values for $i'$, $j'$, $u'$ and $v'$ as defined above 
(note that the condition on line~\ref{line:choosev'} holds for these values);
we call this the \emph{lucky} iteration.
In this iteration, the algorithm will find on line~\ref{line:PS-callF} that $F(u',v') \neq \varnothing$:
due to Claim~\ref{clm:partsol-recurse} and our inductive hypothesis for~$i'<i$, 
we know that $F(u',v')=(Q'_1,Q'_2)$ is a partial solution for~$(u',v')$ with minimum total weight. 
Moreover, by Claim~\ref{clm:partsol-path} Algorithm~\ref{alg:PartSol} will find on line~\ref{line:PS-reach} 
that $v$ is reachable from~$u'$ in~$G \langle T[v',u],u',v \rangle$, 
and so the algorithm will not continue with the next iteration but will proceed with computing a pair~$(P_1,P_2)$.

In the remainder of the proof we show that $(P_1,P_2)$ is a  partial solution for $(u,v)$, and moreover, its weight is at most~$w(Q_1)+w(Q_2)$.
Clearly, this finishes the proof of the lemma for Case B.
We start with Claim~\ref{clm:caseB-Rdisjoint} saying that  $(P_1,P_2)$ is a  partial solution for $(u,v)$ if $R$ is disjoint from~$Q'_1 \cup Q'_2$;
as we will show later in Claims~\ref{clm:caseB-Q1intersectsR} and~\ref{clm:caseB-Q2intersectsR}, this is always the case.

\begin{claim}
\label{clm:caseB-Rdisjoint}
If $V(R) \cap V(Q'_1 \cup Q'_2)=\emptyset$, then $(P_1,P_2)$ is a  partial solution for $(u,v)$. 
\end{claim}
\begin{claimproof}
Recall that $P_1=Q'_2 \cup T[v',u]$ and $P_2=Q'_1 \cup R$. 
Note that condition~(c) for being a partial solution for~$(u,v)$ holds, because $T[v',x_{j'}] \cap T[x_i,u] = \emptyset$  
is guaranteed by the algorithm's choice for~$v'$, and so we have $T[x_i,u] \subseteq T[v',u]$.
Condition~(d) holds by the definition of~$R$. 
Observe also that both~$P_1$ and~$P_2$ are plain and quasi-monotone.

Let us show now that $(P_1,P_2)$ is a locally cheapest and, hence, a well-formed pair.
Suppose for contradiction that $T[q,q']$ is a shortcut for $P_1$ and $P_2$; since $Q'_1$ and $Q'_2$ are locally cheapest, it follows that 
$q$ and $q'$ must lie on~$R$. If $T[q,q']$ is contained in the auxiliary graph~$G \langle T[v',u],u',v \rangle$, 
then this contradicts the optimality of~$R$. If the auxiliary graph does not contain $T[q,q']$ even though it contains both $q$ and~$q'$, 
then $q$ and~$q'$ must be contained in different components of~$T_{(j',i)} - V(T[v',u])$: in this case, however, by $T[v',u] \subseteq P_1$ 
we obtain that there is a vertex of~$P_1$ on~$P_2[q,q']=R[q,q']$, contradicting our assumption that $T[q,q']$ is a shortcut for~$(P_1,P_2)$.

It remains to prove that 
$P_1$ and $P_2$ are permissively disjoint $(\{a_1,a_2\},\{u,v\})$-paths. 
By definition, $Q'_1$ and $Q'_2$ are permissively disjoint $(\{a_1,a_2\},\{u',v'\})$-paths.
Moreover, they do not share any vertex with~$T[v',u] $ except for $v'$:
for $Q'_1$ this follows from its quasi-monotonicity and $i'<j'$;
for $Q'_2$ this follows from condition~(d) on~$Q'_2$ for being a partial solution for~$(u',v').$
%for $Q'_2$ this follows from its quasi-monotonicity, $i<j$, and the fact that $T[v,x_{j'}] \cap T[x_i,u] = \emptyset$  
%which is guaranteed by the algorithm's choice for~$v'$.
Note that $R$ cannot share a vertex with $T[v',u]$ either,
by the definition of~$G \langle T[v',u],u',v \rangle$. %and the facts $i<j$ and $T[v',u] \supset T[x_{j'},u]$. 
Hence, our assumption $V(R) \cap V(Q'_1 \cup Q'_2)=\emptyset$ implies that 
$P_1$ and $P_2$ are permissively disjoint $(\{a_1,a_2\},\{u,v\})$-paths, as claimed.
\end{claimproof}

\smallskip
Using the optimality of~$(Q'_1,Q'_2)$ and that of~$R$, we get
\begin{align}
\begin{split}
\label{eq:qpqp}
w(Q_1) + w(Q_2) & = w(\Q_2) +  w(T[v',u]) + w(\Q_1) + w(\R)  \\
& \geq w(Q'_1)+w(Q'_2) + w(T[v',u]) + w(R) \geq  w(P_1) + w(P_2)
\end{split}
\end{align}
where the last inequality follows from the fact that $Q'_1$, $Q'_2$, and~$T[v',u]$ are 
pairwise edge-disjoint, as we have proved in Claim~\ref{clm:caseB-Rdisjoint},
and moreover, $R$ cannot share a negative-weight edge with any of these four paths, 
due to our definition of the auxiliary graph~$G \langle T[v',u],u',v \rangle$ and the quasi-monotonicity of~$Q'_1$ and~$Q'_2$.
 

%\smallskip
It remains to show that $(P_1,P_2)$ is indeed a partial solution for~$(u,v)$; Inequality~\ref{eq:qpqp} then implies its optimality.
%To prove this, we distinguish between three cases based on the intersection of~$R$ with~$Q'_1 \cup Q'_2$.
%First we show that if $R$ shares no vertices with $Q'_1 \cup Q'_2$, then Claim~\ref{clm:caseB-Rdisjoint} shows that $(P_1,P_2)$ is a partial solution for~$(u,v)$. 
To this end, we show that $R$ must be vertex-disjoint from $Q'_1 \cup Q'_2$: 
as Claims~\ref{clm:caseB-Q1intersectsR} and~\ref{clm:caseB-Q2intersectsR} prove, assuming otherwise  contradicts the optimality of~$(Q_1,Q_2)$.


%%%%%%%%%%%%%%%%%%%%%%%%%%%%%%%%%%%%%%%%%

\smallskip 
Let $y$ be the vertex on~$R$ closest to~$v$ that is contained in~$Q'_1 \cup Q'_2$, assuming that such a vertex exists. 

\begin{claim}
\label{clm:caseB-Q1intersectsR}
If $y \in V(Q'_1)$, then 
$(u,v)$ admits a partial solution of weight less than $w(Q_1)+w(Q_2)$. 
\end{claim}

\begin{claimproof}
Define $P'_2=P_2 \setminus W$, 
for the closed walk $W=Q'_1[y,u'] \cup R[u',y]$; see Figure~\ref{fig:yQ1} for an illustration. 
Note that $Q'_1$ and $R$ cannot share vertices of~$T$ other than~$u'$, because
$Q'_1$ contains no vertex of $T$ outside $T_{(1,i')}$ due to its plainness and quasi-monotonicity, 
and $R$ contains no vertex of~$T$ outside $T_{(j',i)}$ except for~$u'$ and~$v$, because $R$ is in~$G \langle T[v',u],u',v \rangle$;
from $i'<j'$ thus follows $V(Q'_1) \cap V(R) \cap V(T)=\{u'\}$. 
Therefore, $W$ does not contain  any edge of~$T$ more than once, so by Lemma~\ref{lem:closed-walk} has non-negative weight.
Using Inequality~\ref{eq:qpqp} this implies 
\begin{align}
\label{eq:p1p2}
w(P_1) + w(P'_2) &= w(P_1)+w(P_2) -w(W) \leq  w(P_1)+ w(P_2) \leq w(Q_1) + w(Q_2).
\end{align}

% Figure environment removed

It is clear by our choice of~$y$ that $R[y,v]$ shares no vertices with~$Q'_1$ or $Q'_2$.
Using the same arguments as in the proof of Claim~\ref{clm:caseB-Rdisjoint}, 
we obtain that $P'_1$ and $P_2$ are permissively disjoint $(\{a_1,a_2\},\{u,v\})$-paths. 
It is also clear that they satisfy conditions~(c) and~(d) of being a partial solution for~$(u,v)$, 
and that $P_1$ is plain and quasi-monotone.
To see that $P'_2$ is also plain and quasi-monotone, recall that $y$ cannot be on~$T$.
Since $P'_2$ is obtained by deleting the subpath of~$Q'_1$ starting at~$y$, and then appending a subpath of~$R$ (not containing any vertex of~$X$) 
it follows that $P'_2$ is quasi-monotone and plain.

Therefore, $(P_1,P'_2)$ satisfies all conditions for being a partial solution for~$(u,v)$, except the condition of being locally cheapest.
\smallskip

We are going to show that $(P_1,P'_2)$ is \emph{not} locally cheapest, but $\Amend(P_1,P'_2)$ is a partial solution for~$(u,v)$. 
Observe that this suffices to prove the claim: 
if $(P_1,P'_2)$ is \emph{not} locally cheapest, 
then by Observation~\ref{obs:locally-cheapest} and Inequality~\ref{eq:p1p2} we get that 
$\Amend(P_1,P'_2)$ has weight lass than $w(P_1)+w(P'_2) \leq w(Q_1)+w(Q_2)$.

\smallskip
Recall first that $P_1$ and~$P_2$ are locally cheapest, 
as we argued in the proof Claim~\ref{clm:caseB-Rdisjoint} (note that our argument there did not rely on $Q'_1 \cup Q'_2$ and $R$ being vertex-disjoint). 
Hence, any shortcut for $P_1$ and $P'_2$ must be the result of deleting the closed walk~$W=Q'_1[y,u'] \cup R[u',y] \subseteq P_2$.
This means that if $T[q,q']$ is a shortcut for~$(P_1,P'_2)$, then $T[q,q']$ contains an inner vertex $\hat{q} \in V(W)$. 
In addition, note that $\hat{q}$ cannot be on~$R$: since $V(P_1)\cap V(T_{(j',i)})=V(T[v',u])$, 
we know that $P_1$ contains no two vertices that may be separated on~$T$ 
by any vertex of~$T_{(j',i)}$. 
Hence, any shortcut for~$(P_1,P'_2)$ is of the form $T[q,q']$ where 
$q$ and $q'$ are on~$P_1$ (in fact, on $Q'_2$), 
and $T[q,q']$ contains a vertex $\hat{q} \in V(Q'_1[y,u'])$.

Let $(\widetilde{P}_1,\widetilde{P}_2)=\Amend(P_1,P'_2)$. 
By the above arguments we have $\widetilde{P}_2=P'_2 = Q'_1 \cup R \setminus W$. 
Observe that by amending shortcuts we cannot violate permissive disjointness, 
and it is easy to see that conditions~(c) and~(d) remain true as well.
%We don't need the line below, for we already know that P_1 ends with $T[x_i,u]$, a property that cannot be destroyed by amending shortcuts.
%(to see that $\widetilde{P}_1$ ends with $T[x_i,u]$, in the case $x_{j'}=x_i$ we need that $T[x_i,u]\cap T[x_{j'},v']=\emptyset$). 
Moreover, after amending all shortcuts, the resulting pair of paths must be locally cheapest. 
Recall that we already proved that $\widetilde{P}_2=P'_2$ is plain and quasi-monotone. 
Hence, to prove that $(\widetilde{P}_1,\widetilde{P}_2)$ is a partial solution for~$(u,v)$
it suffices to check the plainness and quasi-monotonicity of $\widetilde{P}_1$.


%Consider any vertex~$x$ of~$X$ on~$\widetilde{P}_1$. If $x \in V(P_1)$ as well, then it is obvious that amending shortcuts on~$P_1$ 
%does not violate the plainness and quasi-monotonicity conditions at~$x$ satisfied by~$P_1$. Thus, we only need to consider vertices of~$X$ 
%that are contained in~$\widetilde{P}_1$ but not in~$P_1$.

Let $x_\ell$ be the first vertex on~$Q'_1$ after $y$ that is on~$X$.
By $y \notin V(T)$ we know that $Q'_1$ does not contain $T[x_1,x_\ell]$, and since $Q'_1$ and $Q'_2$ are locally cheapest,
$T[x_1,x_\ell]$ must contain a vertex of~$Q'_2$. Let $x_{\ell'}$ denote the vertex on $T[x_1,x_\ell]$ closest to~$x_\ell$ that belongs to~$Q'_2$. 
%%%%%%%%%%%
We claim that $\widetilde{P}_1= Q'_2 \setminus Q'_2[x_{\ell'},x_{j'}] \cup T[x_{\ell'},x_{j'}]$. 
To see this, note that the deletion of the vertices of~$Q'_1[x_\ell,u']$ 
removes all vertices of~$Q'_1$ from $T[x_{\ell'},x_{j'}]$; these vertices (weakly) follow $x_\ell$ on~$Q'_1$ by its quasi-monotonicity. 
Hence, as $x_{\ell'}$ and $x_{j'}$ are both on~$Q'_2$, 
we obtain $T[x_{\ell'},x_{j'}] \subseteq \widetilde{P}_1$.
Observe also that if a vertex~$z$ of $Q'_1[y,x_\ell]$  is contained in some~$T_h$ with $h<\ell$, then $x_h \in V(Q'_1 \cup Q'_2)$ by Lemma~\ref{lem:paths-vs-X},
which implies $x_h \in V(Q'_2)$ by the definition of~$x_\ell$, and thus we get $h \leq \ell'$ from the definition of~$x_{\ell'}$. 
However, then the removal of~$z$ from~$Q'_1$ cannot result in a shortcut, since $Q'_2$ is plain and thus its intersection with~$T_h$ is a path. 
Therefore, $\widetilde{P}_1= Q'_2 \setminus Q'_2[x_{\ell'},x_{j'}] \cup T[x_{\ell'},x_{j'}]$ indeed holds.
%%%%%%%%%%%%%%%
As a consequence, the plainness and quasi-monotonicity of~$P_1$ immediately implies that $\widetilde{P}_1$ is plain and quasi-monotone as well. 
Hence, $(\widetilde{P}_1,\widetilde{P}_2)$ is a partial solution for~$(u,v)$.

To see that $(P'_1,P_2)$ is not locally cheapest,
observe at least one shortcut is created when deleting~$W$: we know that $T[x_{\ell'},x_{j'}] \subseteq \widetilde{P}_1$,
but $x_{i'}$ lies on the path~$T[x_{\ell'},x_{j'}]$ and is a vertex of~$W$. Hence $w(\widetilde{P}_1)+w(\widetilde{P}_2)<w(P_1)+w(P_2) \leq w(Q_1)+w(Q_2)$.
\end{claimproof}

\smallskip
The proof of Claim~\ref{clm:caseB-Q2intersectsR} is similar to the proof of Claim~\ref{clm:caseB-Q1intersectsR}, 
but it is not entirely symmetric.

\begin{claim}
\label{clm:caseB-Q2intersectsR}
If $y \in V(Q'_2)$, then 
$(u,v)$ admits a partial solution of weight less than $w(Q_1)+w(Q_2)$. 
\end{claim}

\begin{claimproof}
Define $P'_1=Q'_1 \setminus T[x_{i'},u'] \cup T[x_{i'},u]$ and $P'_2=Q'_2 \setminus Q'_2[y,v'] \cup R[y,v]$, and
consider the closed walk
\[W= R[u',y] \cup Q'_2[y,v'] \cup T[u',v'].\]

%Then $P'_1 \cup P'_2=P_1 \cup P_2 \setminus W$;
See Figure~\ref{fig:yQ2} for an illustration.
We argue that $W$ does not contain any edge of~$T$ more than once. 
First, by the definition of $G \langle T[v',u],u',v \rangle$ we know that 
any vertex of~$T$ contained in the path $R$ belongs to~$T_{(j',i)}$, but not to~$T[v',u]$. 
Hence, $R$ shares no edges with~$T[u',v']$.
Condition~(d) on~$Q'_2$ for being a partial solution for~$(u',v')$
requires $V(T_{(i'+1,r)}) \cap V(Q'_2) \subseteq \{v'\}$.
This shows that $Q'_2$ cannot share an edge of~$T$ with~$R$ or with~$T[x_{i'},v']$; note that $y \notin V(T)$ also follows.
%Hence, such an edge cannot be contained in~$T[u',v']$  by $i'<j'$, 
%and it also cannot appear on~$Q'_2[y,x_{j'}]$ by condition
%by the definition of~$v'$ and the quasi-monotonicity of~$Q'_2$. 
Lastly, $Q'_2[y,v']$ shares no edge of~$T$ with $T[u',x_{i'}]$ by
the (permissive) disjointness of $Q'_1$ and $Q'_2$. 
Therefore, by Lemma~\ref{lem:closed-walk} we know $w(W)\geq 0$.
By Inequality~\ref{eq:qpqp} this implies 
\begin{align} 
%\begin{split}
w(P'_1) + w(P'_2)
& = w(P_1)+ w(P_2)- w(T[x_{i'},u'] \cup R[u',y]  \cup Q'_2[y,v'] \cup T[v',x_{j'}]) + w(T[x_{i'},v']) \notag \\
& = w(P_1)+w(P_2)- w(W \setminus T[x_{i'},x_{j'}]) + w(T[x_{i'},x_{j'}]) \notag \\
& < w(P_1)+w(P_2) \leq w(Q_1) + w(Q_2). \label{eq:vmi}
%\end{split}
\end{align}


% Figure environment removed

It is clear by our choice of~$y$ that $R[y,v]$ shares no vertices with~$Q'_1$ or $Q'_2$.
Using the same arguments as in the proof of Claim~\ref{clm:caseB-Rdisjoint}, 
we obtain that $P'_1$ and $P'_2$ are permissively disjoint $(\{a_1,a_2\},\{u,x_j\})$-paths. 
It is also clear that they satisfy conditions~(c) and~(d), 
and that both $P'_1$ and $P'_2$ are plain and quasi-monotone.
Therefore, $(P'_1,P'_2)$ satisfies all conditions for being a partial solution for~$(u,v)$, except for the condition of being locally cheapest.


\smallskip
We are going to show that $\Amend(P'_1,P'_2)$ is a partial solution for~$(u,v)$. 
Observe that this suffices to prove the claim, as
by Observation~\ref{obs:locally-cheapest} and Inequality~\ref{eq:vmi}, 
$\Amend(P'_1,P'_2)$ has weight at most $w(P'_1)+w(P'_2) < w(Q_1)+w(Q_2)$.

\smallskip
Recall first that $Q'_1$ and $Q'_2$ are partial solutions for~$(u',v')$ and thus are locally cheapest.
This means that if $T[q,q']$ is a shortcut for~$(P'_1,P'_2)$, then $W$ contains an inner vertex of~$T[q,q']$. 
In addition, note that $\hat{q}$ cannot be on~$R$: since the vertices of~$P'_1$ induce a path in~$T_{(j',i)}$, 
we know that $P'_1$ contains no two vertices that may be separated on~$T$ 
by any vertex of~$T_{(j',i)}$. 
Hence, any shortcut~$T[q,q']$ for $P'_1$ and $P'_2$ must be the result of deleting $Q'_2[y,v']$,
that is, $q$ and $q'$ must both lie on~$Q'_1$, with $T[q,q']$ some vertex of $Q'_2[y,v']$.

Let $(\widetilde{P}_1,\widetilde{P}_2)=\Amend(P_1,P'_2)$. 
By the above arguments we have $\widetilde{P}_2=P'_2$. 
Observe that by amending shortcuts we cannot violate permissive disjointness, 
and it is obvious that conditions~(c) and~(d) remain true as well. 
Moreover, after amending all shortcuts, the resulting pair of paths must be locally cheapest. 
Recall that we already proved that $\widetilde{P}_2=P'_2$ is plain and quasi-monotone. 
Hence, to prove that $(\widetilde{P}_1,\widetilde{P}_2)$ is a partial solution for~$(u,v)$
it suffices to check the plainness and quasi-monotonicity of $\widetilde{P}_1$.



Suppose for contradiction that the condition of plainness or quasi-monotonicity is violated for~$\widetilde{P}_1$ at some vertex~$x_k$; 
by the definition of these properties (see Definitions~\ref{def:plain} and~\ref{def:quasimonotone}) 
we have $x_k \in V(\widetilde{P}_1)$.
If $x_k$ is also contained in $P'_1$, then exactly by the plainness and quasi-monotonicity of~$P'_1$, these properties
will not get violated at~$x_k$ when amending shortcuts.
Hence, $x_k \notin V(P'_1)$ and thus must be contained in a shortcut for~$(P'_1,P'_2)$. 
Then $x_k \notin V(T[x_{i'},x_i])$ is obvious from the fact that $T[x_{i'},u] \subseteq \widetilde{P}_1$. 
Note that $k > i$ is not possible either, since $R$ and thus $P'_2$ contains no vertex of $T_{(i+1,r)}$ other than~$v$.
%(because the same holds for $P'_2$ by the definition of~$R$).
%As $P'_1 \cup P'_2$ has no vertices in $T_{(j+1,r)}$, 
Thus, we get $k<i'$. Consequently, $Q'_2$ must contain a vertex of~$T[x_1,x_{i'}]$, as otherwise
$x_k$ could not be contained in a shortcut resulting from the deletion of~$Q'_2[y,v']$ 
(here we also rely on the fact that there are no shortcuts on~$Q'_2$ with respect to~$Q'_1$).

Let $x_\ell$ be the first vertex on~$Q'_2$ after $y$ that is on~$X$ (by our previous sentence, such a vertex exists). 
By $y \notin V(T)$ we know that $Q'_2$ does not contain $T[x_1,x_\ell]$, and since $Q'_1$ and $Q'_2$ are locally cheapest,
it must contain a vertex of~$Q'_1$. Let $x_{\ell'}$ denote the vertex on $T[x_1,x_\ell]$ closest to~$x_\ell$ that belongs to~$Q'_1$. 
%%%%%%%%%%%%%%
We claim that $\widetilde{P}_1= Q'_1 \setminus Q'_1[x_{\ell'},x_{i'}] \cup T[x_{i'},u]$. 
To see this, note that the deletion of the vertices of~$Q'_2[x_\ell,v']$ 
removes all vertices of~$Q'_2$ from $T[x_{\ell'},x_{i'}]$; these vertices (weakly) follow $x_\ell$ on~$Q'_2$ by its quasi-monotonicity. 
Hence, as $x_{\ell'}$ and $x_{i'}$ are both on~$Q'_2$, 
we obtain $T[x_{\ell'},x_{i'}] \subseteq \widetilde{P}_1$.
Observe also that if a vertex~$z$ of $Q'_2[y,x_\ell]$  is contained in some~$T_h$ with $h<\ell$, then $x_h \in V(Q'_1 \cup Q'_2)$ by Lemma~\ref{lem:paths-vs-X},
which implies $x_h \in V(Q'_1)$ by the definition of~$x_\ell$, and thus we get $h \leq \ell'$ from the definition of~$x_{\ell'}$. 
However, then the removal of~$z$ from~$Q'_2$ cannot result in a shortcut, since $Q'_1$ is plain and thus its intersection with~$T_h$ is a path. 
Therefore, $\widetilde{P}_1= Q'_1 \setminus Q'_1[x_{\ell'},x_{i'}] \cup T[x_{i'},u]$ indeed holds.
%%%%%%%%%%%%%
As a consequence, the plainness and quasi-monotonicity of~$P_1$ immediately implies that $\widetilde{P}_1$ is plain and quasi-monotone as well. 
a contradiction to the definition of~$x_k$. 
Hence, $(\widetilde{P}_1,\widetilde{P}_2)$ is a partial solution for~$(u,v)$.
\end{claimproof}

\smallskip
A direct consequence of Claims~\ref{clm:caseB-Q1intersectsR} and~\ref{clm:caseB-Q2intersectsR}
is that neither $y \in V(Q'_1)$ nor $y \in V(Q_2)$ is possible, as that would contradict the optimality of~$(Q_1,Q_2)$.
Hence, $V(Q'_1 \cup Q'_2) \cap V(R)=\emptyset$, so by Claim~\ref{clm:caseB-Rdisjoint}
we know that $(P_1,P_2)$ is a partial solution for~$(u,v)$, as required. 
\end{proof}

\subsection{Assembling the parts}
\label{sec:permdis}

Using Algorithm~\ref{alg:PartSol} as a subroutine, 
it is now straightforward to construct an algorithm that computes
two permissively disjoint $(\{a_1,a_2\},\{b_1,b_2\})$-paths of minimum total weight; see Algorithm~\ref{alg:PermDisj}.


\begin{varalgorithm}{\textsc{PermDisjoint}}
\caption{Computing two permissively disjoint $(\{a_1,a_2\},\{b_1,b_2\})$-paths of minimum total weight in~$G$. 
}
\label{alg:PermDisj}
\begin{algorithmic}[1]
\Require{Vertices $a_1,a_2,b_1,$ and $b_2$  on~$T$.}
\Ensure{Two permissively disjoint $(\{a_1,a_2\},\{b_1,b_2\})$-paths of minimum total weight, or $\varnothing$ if no such paths exist. }
\ForAll{$u,v \in V(T)$} $F(u,v)\leftarrow \varnothing$.
\EndFor
\For{$i=1$ to $r$}
	\For{$j=i$ to $r$}
	   \ForAll{$u \in V(T_i)$ and $v \in V(T_j)$}
	       \State $F(u,v) \leftarrow $ \textsc{PartSol}$(u,v)$.					
	   \EndFor
	\EndFor		
\EndFor
\If{$F(b_1,b_2)=\varnothing$ and $F(b_2,b_1)=\varnothing$} {\bf return} $\varnothing$.
\Else{ {\bf return} a (non-$\varnothing$) path pair in $\{F(b_1,b_2),F(b_2,b_1)\}$ with minimum weight.} \label{line:PD-final}
\EndIf
\end{algorithmic}
\end{varalgorithm}

Using the correctness of Algorithm~\ref{alg:PartSol}, as established by Lemma~\ref{lem:partsol-alg}, 
and the observation in Lemma~\ref{lem:partsol-to-permdisj} on how partial solutions can be used to 
find two permissively disjoint $(\{a_1,a_2\},\{b_1,b_2\})$-paths of minimum total weight, we immediately obtain the following.

\begin{cor}
\label{cor:perm-disjoint-paths}
Algorithm~\ref{alg:PermDisj} in polynomial time finds two permissively disjoint $(\{a_1,a_2\},\{b_1,b_2\})$-paths of minimum total weight if such paths exist, 
otherwise outputs $\varnothing$. 
\end{cor}

\begin{proof}
Observe that Algorithm~\ref{alg:PartSol} called for a pair $(u,v)$ with~$u \in V(T_i)$ 
relies only on partial solutions~$F(u',v')$ where $u' \in V(T_{i'})$ for some~$i'<i$. 
Hence, when computing the values~$F(u,v)$ for increasing values of~$i$ (where $i$ denotes the index for which~$u \in V(T_i)$),
as done in Algorithm~\ref{alg:PermDisj}, the partial solutions required by Algorithm~\ref{alg:PartSol} will be available.
Hence, as a consequence of Lemma~\ref{lem:partsol-alg}, Algorithm~\ref{alg:PermDisj} correctly computes the values~$F(u,v)$ in polynomial time, 
and by Lemma~\ref{lem:partsol-to-permdisj} returns two permissively disjoint $(\{a_1,a_2\},\{b_1,b_2\})$-paths of minimum total weight 
whenever such paths exist.
\end{proof}

Using Corollary~\ref{cor:perm-disjoint-paths}, we can now give the proof of Theorem~\ref{thm:DISP-main}.

\bigskip
\noindent
{\bf Proof of Theorem~\ref{thm:DISP-main}.}
We show that given an instance $(G,w,s,t)$ of \DISP{} where $w$ is conservative on~$G$ and $G[E^-]$ is a tree, 
Algorithm~\ref{alg:DISP} computes a solution for~$(G,w,s,t)$ of minimum weight whenever a solution exists. 
Suppose that $(P_1,P_2)$ is an optimal solution for~$(G,w,s,t)$ with total weight~$w^\star$. 
If $(P_1,P_2)$ is separable, then by Lemma~\ref{lem:separablesol}, Algorithm~\ref{alg:DISP} 
will find a solution of weight~$w^\star$ on Line~\ref{line:DISP-separable-output}.
If $(P_1,P_2)$ is not separable, then by Lemma~\ref{lem:type2bsol}, Algorithm~\ref{alg:DISP} 
will find a solution of weight~$w^\star$ on Line~\ref{line:DISP-nonseparable-output}; 
note that here we rely on the correctness of Algorithm~\ref{alg:PermDisj}, as stated in Corollary~\ref{cor:perm-disjoint-paths}.

The algorithm clearly runs in polynomial time: for each choice of vertices~$a_1$, $a_2$, $b_1$, $b_2$ 
we compute a flow of constant value (which can be done in linear time), 
and perform a call of Algorithm~\ref{alg:PermDisj}. 
The bottleneck is therefore Algorithm~\ref{alg:PermDisj}: for each pair of vertices on~$(u,v)$, 
it iterates over possible values for vertices~$u'$ and~$v'$, and 
for each pair~$(u',v')$ computes a shortest path in the auxiliary graph $G \langle T[v',u],u',v \rangle$. 
Since $G \langle T[v',u],u',v \rangle$ implicitly depends also on~$x_1$ and~$x_r$, this amounts to 
$O(n^6)$ shortest-path computations in total on undirected graphs with negative edge-weights;
this can be solved, e.g., using the algorithm by Bernstein et al.~\cite{BNWN22} running in near-linear time.
\hfill$\qedsymbol$


\section{Conclusion}
\label{sec:conclusion}

We have presented a polynomial-time algorithm for solving the \DISP{} problem on graphs where negative-weight edges form a tree.
Is it possible to give a substantially faster algorithm for this problem?

We believe that our approach can be generalized to obtain a polynomial-time algorithm 
for instances where the number of connected components in the subgraph~$G[E^-]$ 
spanned by all negative-weight edges is a fixed constant~$c$. Is it possible to give an algorithm for \DISP{} on conservative graphs 
that is fixed-parameter tractable when parameterized by~$c$?

Finally, is it possible to find in polynomial time $k$ openly disjoint $(s,t)$-paths with minimum total weight for some fixed~$k \geq 3$ in conservative graphs?

\subsection*{Acknowledgement}
Ildik\'o Schlotter acknowledges the support of the Hungarian Academy of Sciences under its Momentum Programme (LP2021-2), 
and the Hungarian Scientific Research Fund (OTKA grants K128611 and K124171). 

\bibliographystyle{abbrv}
\bibliography{odd}

\end{document}
