\documentclass[reqno, 11pt]{amsart}
\usepackage[utf8]{inputenc}
\usepackage{geometry}
\geometry{top=3cm, bottom=3cm, left=2cm, right=2cm}
\usepackage{amsfonts}
\usepackage{hyperref}
\usepackage[backend=biber, style=numeric-comp]{biblatex}
\hypersetup{
pdftitle={TBA},
pdfsubject={},
pdfauthor={Shijia Jin},
pdfkeywords={}
}
\usepackage{amsmath}
\usepackage{xcolor}
\usepackage{ulem}
\usepackage{amsthm}
\usepackage{pdflscape}
\usepackage{pgfplots}
\usepackage{mathrsfs}
\setlength{\textwidth}{\paperwidth}
\addtolength{\textwidth}{-2.2in}
\calclayout
\DeclareMathOperator*{\argmax}{arg\,max}
\DeclareMathOperator*{\argmin}{arg\,min}

\newtheorem{theorem}{Theorem}[section]
\newtheorem{corollary}[theorem]{Corollary}
\newtheorem{proposition}[theorem]{Proposition}
\newtheorem{lemma}[theorem]{Lemma}
\newtheorem{conjecture}[theorem]{Conjecture}
\newtheorem{question}[theorem]{Question}
\newtheorem{claim}[theorem]{Claim}
\newtheorem{property}[theorem]{Property}
\newtheorem{assumption}[theorem]{Assumption}
\newtheorem{remark}[theorem]{Remark}


\theoremstyle{definition}
\newtheorem{definition}[theorem]{Definition}
\newtheorem{example}[theorem]{Example}
\newtheorem{notation}[theorem]{Notation}
\usepackage{relsize}

\usepackage{mlmodern}
\usepackage[T1]{fontenc}

\newcommand{\e}{\mathfrak{e}}
\newcommand{\x}{\mathfrak{x}}
\newcommand{\cc}{\mathfrak{c}}
\newcommand{\kong}{\vspace{0.08cm}}

\renewcommand{\proofname}{\normalfont{\normalfont{P}}\footnotesize{ROOF}}


\addbibresource{ref.bib} 


%--------Meta Data: Fill in your info------
\title{Macroscopic Market Making}

\author{Ivan Guo}
\address{Ivan Guo, Centre for Quantitative Finance and Investment Strategies, School of Mathematics, 
Monash University}
\email{ivan.guo@monash.edu}
\author{Shijia Jin}
\address{Shijia Jin, School of Mathematics, 
Monash University}
\email{shijia.jin@monash.edu}
\author{Kihun Nam}
\address{Kihun Nam, Centre for Quantitative Finance and Investment Strategies, School of Mathematics, 
Monash University}
\email{kihun.nam@monash.edu}

\begin{document}


\maketitle

\begin{abstract}
We propose a macroscopic market making model \`a la Avellaneda-Stoikov, using continuous processes for orders instead of discrete point processes. The model intends to bridge the gap between market making and optimal execution problems, while shedding light on the influence of order flows on the optimal strategies. We demonstrate our model through three problems. The study provides a comprehensive analysis from Markovian to non-Markovian noises and from linear to non-linear intensity functions, encompassing both bounded and unbounded coefficients. Mathematically, the contribution lies in the existence and uniqueness of the optimal control, guaranteed by the well-posedness of the strong solution to the Hamilton-Jacobi-Bellman equation and the (non-)Lipschitz forward-backward stochastic differential equation.\\

\noindent \textbf{Keywords}: Market making, Stochastic optimal control, Forward-backward
stochastic differential equation
\end{abstract}

\tableofcontents


\section{Introduction}
\label{intro}
\noindent In this paper, we focus on the strategic behaviour of the market maker, which is defined as a liquidity provider in the financial market. More specifically, the market maker provides bid and ask prices on one or several assets, and makes profit by the bid–ask spread (the price difference between buy and sell orders). Pioneered by the work of \cite{ho1981optimal} and subsequently \cite{avellaneda2008high}, market making---as a stochastic control problem---has been the subject of extensive literature in market microstructure. We propose a macroscopic model \`a la Avellaneda-Stoikov \cite{avellaneda2008high} in view of two major motivations.

Firstly, while the market making problem focuses on the provision of liquidity, the other side of the coin studies how liquidity can be consumed in an optimal fashion. This is known as the optimal execution problem, set in motion by \cite{bertsimas1998optimal} and \cite{almgren2001optimal}. We refer to \cite{gueant2016financial} and \cite{cartea2015algorithmic} for excellent overviews of both topics. Although market making and optimal execution problems are highly correlated, they have been studied separately in two streams of literature. One of the main reasons is the difference in the underlying order characteristics: the total volume of (market) orders in the optimal execution is absolutely continuous with respect to time, allowing for the consideration of trading rates; However, in the market making problem, orders follow prescribed point processes. To bridge the gap between these two topics, we study a market making problem similar to the framework in \cite{avellaneda2008high}, but replace discrete point processes by continuous processes, aligning with the order rate concept prevalent in the optimal execution literature. In practical terms, we expand the time horizon of the market making from seconds, as traditionally approached, to minutes or hours (see the introduction section in \cite{guo2017optimal}). This expanded horizon enables us to safely model orders using continuous processes, resulting in what we refer to as a \textit{macroscopic} model.

Secondly, our macroscopic model is expected to address several deficiencies of the original model, while offering new insights simultaneously. Pointed out by \cite{law2019market}, in certain liquid stocks, the execution probability for limit orders located more than one tick away from the best quote is significantly low in short time scales. Under the expanded time horizon, we anticipate a higher and hence more meaningful execution probability. The assumption that all market orders are of the same size is no longer needed in the macroscopic model, and our order dynamics are allowed to be non-Markovian. Furthermore, akin to the fact that investments are made based on the belief of future prices, the market maker should also keep an eye on the order dynamics to devise appropriate strategies. This is made possible in our setting. Additionally, the solution of the optimal control problem enables the observation of quotes price changes corresponding to the order flows. This can offer a market-making explanation to the concave price impact---see \cite{nadtochiy2022simple} for a microstructural explanation.

The goal of this paper is to propose the macroscopic market making problems as stochastic control problems, and to study the associated mathematical issues---the existence and uniqueness of the optimal control. The study on each problem is conducted in three steps: we introduce the market making setup, providing a clear description of the stochastic control problem; the stochastic control problem is then transformed to either a partial differential equation (PDE) or a forward-backward stochastic differential equation (FBSDE), depending on the specific context; we finally establish the well-posedness of the equation, hence ensuring the existence and uniqueness of the optimal control.

We start with the case when noises are Markovian, and the intensity function---which is used to model the execution probability---is linear. Applying the dynamic programming principle, the resulting PDE exhibits similarities to the main PDEs studied in \cite{barger2019optimal}, \cite{fouque2022optimal} and \cite{souza2022regularized}, all of which investigate the optimal execution problems with stochastic price impact. While \cite{barger2019optimal}, \cite{fouque2022optimal} study the approximation solutions and \cite{souza2022regularized} deduces the well-posedness of viscosity solutions, our contribution lies in the existence and uniqueness of classical solutions. Besides the connection with optimal execution problem, we also examine the case of linear intensity function due to the applicability of the technique employed in proving the well-posedness, which will be extended to the next problem. We conclude the first problem by presenting a simple example with an explicit solution, demonstrating how order flows are incorporated into the market maker's strategy.

Since order flows are known to be non-Markovian, the second problem adopts non-Markovian noises and also more general non-linear intensity functions introduced in \cite{gueant2017optimal}. By utilizing a truncation function and a version of the stochastic maximum principle, we obtain a Lipschitz FBSDE, the well-posedness of which is proved via the decoupling method pioneered by \cite{ma2015well}. Furthermore, the solution of the Lipschitz FBSDE is used to verify the well-posedness of a non-Lipschitz FBSDE that corresponds to the case when the truncation is not applied. 

Moving beyond bounded noises and coefficients considered in the previous two problems, the third and final problem explores unbounded coefficients, along with non-Markovian noises and the linear intensity function. By leveraging a convex-analytic approach, we obtain a non-Lipschitz FBSDE and establish its well-posedness by constructing the solution using a monotonic sequence. By organizing the problems in this manner, we gradually extend the analysis from Markovian to non-Markovian noises, and from linear to non-linear intensity functions, encompassing situations with both bounded and unbounded coefficients.

The organization of the article aligns with the three problems outlined in preceding paragraphs: Section \ref{section_2} delves into the first problem, Section \ref{section_3} explores the second problem, and Section \ref{section_4} focuses on the third problem. Appendix \ref{section_5} encompasses a version of the stochastic maximum principle.

\textit{Notation:} Throughout the present work, we fix $T > 0$ to represent our finite trading horizon. We denote by $(\Omega, \mathcal{F}, \mathbb{F}=(\mathcal{F}_t)_{0\leq t\leq T}, \mathbb{P})$ a complete filtered probability space, with $\mathcal{F}_T = \mathcal{F}$. An $m$-dimensional Brownian motion $W=(W^1,\dots,W^m)$ is defined on such space, for a fixed positive integer $m$, and the filtration $\mathbb{F}$ is generated by $W$ and augmented. Let $\mathcal{G}$ represents an arbitrary $\sigma$-algebra contained in $\mathcal{F}$ and consider the following spaces:
\begin{gather*}
    L^p(\Omega, \mathcal{G}):=\big\{X: X \text{ is } \mathcal{G} \text{-measurable and } \mathbb{E}|X|^p<\infty \big\};\\
    \mathbb{H}^p:=\Big\{X: X \text{ is } \mathbb{F} \text{-progressively measurable and } \mathbb{E}\Big[\big(\int_0^T |X_t|^2\,dt\big)^{p/2}\Big]<\infty\, \Big\};\\
    \mathbb{S}^p:=\Big\{X\in \mathbb{H}^p: \mathbb{E}\big[\sup_{0\leq t\leq T} |X_t|^p\big]<\infty \Big\};\\
    \mathcal{M}:= \big\{ M : M_t \in L^2(\Omega, \mathcal{F}_t) \text{ for a.e. } t \in [0, T] \text{ and } \{M_t, \mathcal{F}_t\}_{0\leq t\leq T}\text{ is a martingale} \big\}.
\end{gather*}

\kong

\section{Markovian Order Flows with Linear Intensity Function}  \label{section_2}

\noindent This section is devoted to introducing the macroscopic market making as a stochastic control problem. We will begin with the linear intensity function and Markovian order flows, solving the problem by the dynamic programming principle. The reason is twofold: it reveals several interesting connections with optimal execution problems under stochastic price impact; the same technique used in the problem will be also applied in the next section, where we will look at the non-Markovian context.

Let $S=(S_t)_{0\leq t\leq T}$ be the mid-price of the asset being continuously traded throughout the time horizon $T$. We follow the framework in \cite{becherer2005classical} and consider the following stochastic differential equation for $L:=(L_t)_{t\in[0,T]}$ in $\mathbb{R}^d$:
\begin{equation}
\begin{aligned}
    L_0&=l_0\in\mathbb{R}^d,\\
    dL_t&=\Gamma(t,L_t)\,dt+\sum_{j=1}^m\Sigma_j(t,L_t)\,dW_t^j, \hspace{1cm} t\in[0,T],
    \label{markov_sde}
\end{aligned}
\end{equation}
for continuous functions $\Gamma : [0, T] \times \mathbb{R}^d \to \mathbb{R}^d$ and $\Sigma_j : [0, T] \times \mathbb{R}^d \to \mathbb{R}^d$, $j =1, \dots, m$. Define the matrix-valued function
$\Sigma : [0, T]\times \mathbb{R}^d \to \mathbb{R}^{d\times m}$ by $\Sigma^{i, j}
:= (\Sigma_j)^i$. The process $L$ serves as the underlying noise. We introduce the following assumptions.

\kong

\begin{assumption}
(1) The mid-price process $S$ belongs to $\mathcal{M}$. Hence, for any $K\in\mathbb{H}^2$, it holds 
\begin{equation}
    \mathbb{E}\int_0^T K_t\,dS_t=0.
    \label{martingale_price}
\end{equation}
(2) For $G\in\{\Gamma, \Sigma_1, \dots, \Sigma_m\}$, there exists a constant $C_L>0$ such that
\begin{equation*}
    |G(t,x)-G(t,y)|\leq C_L |x-y|,
\end{equation*}
for all $t\in[0,T]$ and $x,y\in\mathbb{R}^d$. Further, $\det(\Sigma(t,l)\,\Sigma^{tr}(t,l))\neq 0$ for all $(t,l)\in [0,T]\times \mathbb{R}^d$.
\label{sde_assumption}
\end{assumption}

\kong

\begin{remark}
The first assumption is common in optimal execution and market making literatures, where the mid-price $S$ usually has no effect on the optimization problem. In this case, the emphasis is placed on how the market maker deals with the order flows. The last condition on $\Sigma$ is added for technical reasons.
\end{remark}

\kong

A market maker keeps trading such an asset to profit from the bid-ask spread by posting bid and ask quotes during the time interval $[0, T]$. More precisely, she posts the ask orders and bid orders at the price levels
\begin{equation}
    S_t^a:=S_t+\delta_t^a \text{\quad and \quad} S_t^b:=S_t-\delta_t^b
    \label{admiss_control}
\end{equation}
respectively, where $\boldsymbol{\delta}:=(\delta^a, \delta^b)\in\mathbb{H}^2\times\mathbb{H}^2$ represents the control. While market orders are described by point processes in the discrete setting, here we regard them as \textit{flows}---continuous processes adapted to the Brownian filtration $\mathbb{F}$. The market order flows at the ask and bid side are modelled by
\begin{equation}
    a_t:=a(L_t) \text{\quad and \quad} b_t:=b(L_t)
    \label{admiss_control}
\end{equation}
accordingly, for some bounded positive functions $a, b:\mathbb{R}^d\to\mathbb{R}_+$. Denote by $\bar{a}, \bar{b}>0$ the upper bounds for functions $a$ and $b$ accordingly.
On a macroscopic scale, the execution probability in the original Avellaneda-Stoikov framework is now interpreted as the portion of order flows captured by the agent. More specifically, suppose that execution probability is $p\in(0,1)$ in the discrete setting, implying the agent has a probability of $p$ to execute each market order. Then, given a sufficient amount of orders, one expects that $p$ portion of total orders is matched by the agent \textit{on average}. Since such probability is determined by the \textit{intensity function} $\Lambda$, we first look at a linear function
\begin{equation}
    \Lambda(\delta)=\zeta-\gamma\,\delta,
    \label{linear_intensity}
\end{equation}
where $\zeta, \gamma>0$ represents the competition level among market makers and we also assume the bid-ask symmetry. Intuitively, as the agent posts further away from the mid-price, the chance of executions decreases in a linear fashion. While the action $\delta$ makes the most sense in $[0,\zeta/\gamma]$, we do not restrict the action to this closed interval, which just requires a small modification overall. Consequently, the inventory and cash processes of the market maker now follow
\begin{gather*}
        Q_t = q_0-\int_0^t a_u\cdot(\zeta-\gamma \delta^a_u)\,du+\int_0^t b_u\cdot(\zeta-\gamma \delta^b_u)\,du,\\
        \nonumber
    X_t = x_0+\int_0^t a_u\cdot(\zeta-\gamma \delta^a_u)\,(S_u+\delta_u^a)\,du-\int_0^t b_u\cdot(\zeta-\gamma \delta^b_u)\,(S_u-\delta_u^b)\,du.
    \nonumber
\end{gather*}
The term $a_u(\zeta-\gamma \delta^a_u)$ shows the (passive) selling rate of the agent, and $S_u+\delta_t^u$ represents the selling price at time $u$. The agent then aims at maximizing the following objective functional 
\begin{equation}
    \mathbb{E}\big[X_T+S_T\,Q_T-\int_0^T \phi_t\,(Q_t)^2\,dt-A\,(Q_T)^2 \big]
    \label{obj_functional}
\end{equation}
by controlling $\boldsymbol{\delta}\in\mathbb{H}^2\times\mathbb{H}^2$, where $A:=A(L_T)$ and $\phi_t:=\phi(L_t)$ for some non-negative bounded functions $A, \phi:\mathbb{R}^d\to\mathbb{R}_+$ with the upper bounds $\bar{A}, \bar{\phi}>0$, accordingly.

\kong

\begin{assumption}
Function A is a continuous non-negative function with the upper bound $\bar{A}$. Functions $a, b$ are positive functions bounded above by $\bar{a}, \bar{b}$ accordingly; function $\phi$ is a non-negative function bounded above by $\bar{\phi}$. Moreover, functions $a, b, \phi$ are all Lipschitz continuous in any bounded domain of $\mathbb{R}^d$.
\label{lip_ord_flow}
\end{assumption}

\kong

\begin{remark}
In the functional \eqref{obj_functional}, the random variable $A$ stands for the terminal penalty. It can be interpreted as the overnight risk as in \cite{adrian2020intraday} or the terminal clearance cost as in the optimal execution literatures. The integral term is the well-known reduced-form model of risk and process $\phi$ could be understood as the product of a risk parameter and the stochastic volatility of the mid-price process (see \cite{souza2022regularized} for an example).
\end{remark}

\kong

By It\^o’s formula, we compute that
\begin{equation}
    \begin{aligned}
    X_T+Q_TS_T&=x_0+\int_0^T(S_t+\delta_t^a)\,a_t(\zeta-\gamma\delta_t^a)\,dt-\int_0^T(S_t-\delta_t^b)\,b_t(\zeta-\gamma\delta_t^b)\,dt\\
    &\hspace{1.45cm} +q_0s_0+\int_0^TS_t\,b_t(\zeta-\gamma\delta_t^b)\,dt-\int_0^TS_t\,a_t(\zeta-\gamma\delta_t^a)\,dt+\int_0^TQ_t\,dS_t\\
    &=x_0+q_0s_0+\int_0^T\delta_t^a\,a_t(\zeta-\gamma\delta_t^a)\,dt+\int_0^T\delta_t^b\,b_t(\zeta-\gamma\delta_t^b)\,dt+\int_0^TQ_t\,dS_t.
    \nonumber
    \end{aligned}
\end{equation}
While the last term will vanish after taking the expectation due to \eqref{martingale_price}, we can then rewrite the functional \eqref{obj_functional} as
\begin{equation}
    \mathbb{E}\big[\int_0^T\delta_t^a\,a_t(\zeta-\gamma\delta_t^a)\,dt+\int_0^T\delta_t^b\,b_t(\zeta-\gamma\delta_t^b)\,dt-\int_0^T \phi_t\,(Q_t)^2\,dt-A\,(Q_T)^2 \big],
    \label{equ_obj_functional}
\end{equation}
where the mid-price and cash no longer play a role. Define
\begin{equation}
\begin{aligned}
    H(t, l, q):=\sup_{\delta^a,\, \delta^b\in \mathbb{H}^2}\mathbb{E}\big[\int_t^T\delta_u^a\,a_u^{t, l}(\zeta-\gamma\delta_u^a)\,&du+\int_t^T\delta_u^b\,b_u^{t, l}(\zeta-\gamma\delta_u^b)\,du\\
    &-\int_t^T \phi_u^{t, l}\,(Q_u^{t, l, q, \boldsymbol{\delta}})^2\,du-A\,(Q_T^{t, l, q, \boldsymbol{\delta}})^2 \big].
\end{aligned}
\end{equation}
We use the superscript $(t, l)$ to show the initial condition of the underlying Markov diffusions, and $(t, l, q, \boldsymbol{\delta})$ to display the additional dependence on the control. Let $\mathcal{L}$ be the generator of $L$ as
\begin{equation*}
    (\mathcal{L}f)(t,l,q):=\sum_{i=1}^d\Gamma^i(t,l)\,\frac{\partial f}{\partial l^i}(t,l,q)+\frac{1}{2}\sum_{i,j=1}^d (\Sigma\Sigma^{tr})^{i,j}(t,l)\,\frac{\partial^2 f}{\partial l^i\partial l^j}(t,l,q)
\end{equation*}
for some sufficiently smooth function $f:[0,T]\times\mathbb{R}^d\times\mathbb{R}\to\mathbb{R}$, then the dynamic programming principle suggests the following partial differential equation (PDE)
\begin{equation}
\begin{aligned}
    \frac{\partial H}{\partial t}+\mathcal{L}H(t,l,q)&+\sup_{\delta^a\in\mathbb{R}}\big\{(\delta^a-\frac{\partial H}{\partial q})\,a(l)(\zeta-\gamma\delta^a)\big\}\\
    &+\sup_{\delta^b\in\mathbb{R}}\big\{(\delta^b+\frac{\partial H}{\partial q})\,b(l)(\zeta-\gamma\delta^b)\big\}-\phi(l)\,(q)^2=0
    \label{pde}
\end{aligned}
\end{equation}
for $(t,l,q)\in[0,T]\times\mathbb{R}^d\times\mathbb{R}$, subjecting to the boundary condition $H(T,l,q)=-A(l)\,( q)^2$. The optimal control in feedback form reads
\begin{equation}
    \delta_t^{b,*}=\frac{\zeta}{2\gamma}-\frac{1}{2}\frac{\partial H}{\partial q}, \hspace{1cm} \delta_t^{a,*}=\frac{\zeta}{2\gamma}+\frac{1}{2}\frac{\partial H}{\partial q}.
    \nonumber
\end{equation}
Combined with the affine ansatz widely used in optimal execution literatures (thanks to the linear structure of our problem):
\begin{equation}
    H(t,l,q)=h_0(t,l)+h_1(t,l)\,q+h_2(t,l)\,(q)^2,
    \nonumber
\end{equation}
we plug them back to \eqref{pde} to obtain, for $(t,l)\in[0,T]\times\mathbb{R}^d$, that
\begin{equation}
    \frac{\partial h_2}{\partial t}+\mathcal{L}h_2(t,l)+\gamma\big(a(l)+b(l)\big)\,h_2(t,l)^2-\phi(l)=0,
    \label{non_lin_pde}
\end{equation}
\begin{equation}
    \frac{\partial h_1}{\partial t}+\mathcal{L}h_1(t,l)+\zeta\big(b(l)-a(l)\big)\,h_2(t,l)+\gamma\big(a(l)+b(l)\big)\,h_2(t,l)\,h_1(t,l)=0,
    \label{lin_pde_1}
\end{equation}
\begin{equation}
    \frac{\partial h_0}{\partial t}+\mathcal{L}h_0(t,l)+\zeta\big(b(l)-a(l)\big)\,h_1(t,l)+\frac{\gamma\, b(l)}{4}\big(\frac{\zeta}{\gamma}-h_1(t,l)\big)^2+\frac{\gamma\, a(l)}{4}\big(\frac{\zeta}{\gamma}+h_1(t,l)\big)^2=0,
    \label{lin_pde_2}
\end{equation}
subjecting to the boundaries $h_2(T,l)=-A(l)$,  $h_1(T,l)=0$ and $h_0(T,l)=0$. One can see that the main challenge lies in the nonlinear PDE \eqref{non_lin_pde}, while the others are all linear and the well-posedness is direct given $h_2$ with sufficient regularity. We summarize this result in the next theorem with the verification step. Hence, the remaining of this section is devoted to \eqref{non_lin_pde}.

\kong

\begin{remark}
We compare our PDE system with the ones from related literatures:\\
\indent (1) Regarding the optimal execution problems that take the order flow into account (see \cite{cartea2015algorithmic} section 7.3), in the PDE for $h_2$, the quadratic term of $h_2$ has a constant coefficient, that is,
    \begin{equation}
        \frac{\partial h_2}{\partial t}+\mathcal{L}h_2(t,l)-\phi+\frac{1}{k}\,h_2(t, l)^2=0,
        \nonumber
    \end{equation}
where $k, \phi>0$ represent the permanent price impact and risk parameter accordingly. As there are no source terms $a$ and $b$, they deduce that $h_2$ is independent of the order flow and solve a Riccati equation. However, there is a source term of order flows in our case.

(2) With respect to execution problems with stochastic  parameters (\cite{barger2019optimal}, \cite{fouque2022optimal} and \cite{souza2022regularized}), they all consider PDEs for $h_2$ that are similar to ours. While \cite{barger2019optimal} and \cite{fouque2022optimal} study the approximation solutions, the existence and uniqueness of the viscosity solution is proven by \cite{souza2022regularized}. Our contribution lies in the well-posedness of the classical solution. Let us look at the equation from \cite{barger2019optimal} (equation (2.17) with no permanent price impact):
    \begin{equation}
        \frac{\partial h_2}{\partial t}+\mathcal{L}h_2(t,l)-\varphi+\frac{1}{f(l)}\,h_2(t,l)^2=0,
        \nonumber
    \end{equation}
where $\varphi>0$ stands for the risk and $f(l)$ is the stochastic impact parameter. Comparing $1/f(l)$ with $\gamma(a(l)+b(l))$ of our equation, it is interesting to see that the lower friction in optimal executions is analogous to the higher market activity in market makings. Furthermore, our method can be applied to prove the well-posedness of (2.17) by a change of variable.
\end{remark}

\kong

\begin{theorem}
\label{verification}
Suppose \eqref{non_lin_pde} admits a bounded classical solution denoted by $h_2$, it holds: 

(1) Both \eqref{lin_pde_1} and \eqref{lin_pde_2} also possess bounded classical solutions, written as $h_1$ and $h_0$ accordingly;

(2) Define $\tilde{H}:[0,T]\times\mathbb{R}^d\times\mathbb{R}\to\mathbb{R}$ as
\begin{equation*}
    \tilde{H}(t,l,q)=h_0(t,l)+h_1(t,l)\,q+h_2(t,l)\,(q)^2,
\end{equation*}
then $H=\tilde{H}$ on $[0,T]\times\mathbb{R}^d\times\mathbb{R}$;

(3) Let $\boldsymbol{\delta}^*=(\delta^{a,*}, \delta^{b,*})$ be defined by
\begin{equation}
    \delta_t^{b,*}=\frac{\zeta}{2\gamma}-\frac{1}{2}\frac{\partial \tilde{H}}{\partial q} \text{ \; and \; } \delta_t^{a,*}=\frac{\zeta}{2\gamma}+\frac{1}{2}\frac{\partial \tilde{H}}{\partial q},
    \label{uniq_Mark_cont}
\end{equation}
then $\boldsymbol{\delta}^*$ is the unique optimal control.
\end{theorem}

\begin{proof}
(1) The fact that both \eqref{lin_pde_1} and \eqref{lin_pde_2} accepts classical solutions follows from \cite{becherer2005classical}. It suffices to apply the Feynman–Kac formula to derive the boundedness.

(2) It is straightforward to see that $\tilde{H}$ is sufficiently regular to apply the It\^o's formula and satisfies the quadratic growth condition in $q$:
\begin{equation*}
    |\tilde{H}(t,l,q)|\leq C\,(1+q^2),
\end{equation*}
for a constant $C>0$. We follow the arguments in \cite{pham2009continuous}. First, we will show that $\tilde{H}\geq H$. For all $(t,l,q)\in[0,T)\times\mathbb{R}^d\times\mathbb{R}$, $\boldsymbol{\alpha}\in\mathbb{H}^2\times \mathbb{H}^2$, and any stopping time $\tau$ valued in $[t,\infty)$, it holds that
\begin{equation*}
\begin{aligned}
    \tilde{H}(s\wedge\tau, L_{s\wedge\tau}^{t,l}, Q_{s\wedge\tau}^{t,l,q,\boldsymbol{\alpha}})=\tilde{H}&(t,l,q)+\int_t^{s\wedge\tau}\frac{\partial \tilde{H}}{\partial l}(u, L_{u}^{t,l}, Q_{u}^{t,l,q,\boldsymbol{\alpha}})^{tr}\,\Sigma(u, L_{u}^{t,l})\,dW_u\\
    &+\int_t^{s\wedge\tau}\big[\frac{\partial \tilde{H}}{\partial t}(u, L_{u}^{t,l}, Q_{u}^{t,l,q,\boldsymbol{\alpha}})+\mathcal{L}\tilde{H}(u, L_{u}^{t,l}, Q_{u}^{t,l,q,\boldsymbol{\alpha}})\\
    &\hspace{1.8cm}-\frac{\partial \tilde{H}}{\partial q}\,a(L_{u}^{t,l})\,(\zeta-\gamma\,\alpha_u^a)\\
    &\hspace{2.8cm}+\frac{\partial \tilde{H}}{\partial q}\,b(L_{u}^{t,l})\,(\zeta-\gamma\,\alpha_u^b)\big]\,du.
\end{aligned}
\end{equation*}
If we choose
\begin{equation*}
    \tau=\tau_n=\inf\Big\{r\geq t:\; \int_t^{r}\big|\frac{\partial \tilde{H}}{\partial l}(u, L_{u}^{t,l}, Q_{u}^{t,l,q,\boldsymbol{\alpha}})^{tr}\,\Sigma(u, L_{u}^{t,l})\big|^2\,du\geq n \Big\},
\end{equation*}
the stopped stochastic integral is then a martingale and hence

\begin{equation*}
\begin{aligned}
    &\mathbb{E}\big[\tilde{H}(s\wedge\tau_n, L_{s\wedge\tau_n}^{t,l}, Q_{s\wedge\tau_n}^{t,l,q,\boldsymbol{\alpha}})\big]\\
    &=\tilde{H}(t,l,q)+\mathbb{E}\int_t^{s\wedge\tau_n}\big[\frac{\partial \tilde{H}}{\partial t}(u, L_{u}^{t,l}, Q_{u}^{t,l,q,\boldsymbol{\alpha}})+\mathcal{L}\tilde{H}(u, L_{u}^{t,l}, Q_{u}^{t,l,q,\boldsymbol{\alpha}})\\
    &\hspace{2cm}-\frac{\partial \tilde{H}}{\partial q}\,a(L_{u}^{t,l})\,(\zeta-\gamma\,\alpha_u^a)+\frac{\partial \tilde{H}}{\partial q}\,b(L_{u}^{t,l})\,(\zeta-\gamma\,\alpha_u^b)\big]\,du\\
    &\leq \tilde{H}(t,l,q)+\mathbb{E}\int_t^{s\wedge\tau_n}\big[\frac{\partial \tilde{H}}{\partial t}(u, L_{u}^{t,l}, Q_{u}^{t,l,q,\boldsymbol{\alpha}})+\mathcal{L}\tilde{H}(u, L_{u}^{t,l}, Q_{u}^{t,l,q,\boldsymbol{\alpha}})\\
    &\hspace{2cm}+\sup_{\delta^a\in\mathbb{R}}\big\{(\delta^a-\frac{\partial \tilde{H}}{\partial q})\,a(L_{u}^{t,l})\,(\zeta-\gamma\,\delta^a)\big\}\\
    &\hspace{2cm}+\sup_{\delta^b\in\mathbb{R}}\big\{(\delta^b+\frac{\partial \tilde{H}}{\partial q})\,b(L_{u}^{t,l})\,(\zeta-\gamma\,\delta^b)\big\}\\
    &\hspace{2cm}-\alpha_u^a\,a(L_{u}^{t,l})\,(\zeta-\gamma\,\alpha_u^a)-\alpha_u^b\,b(L_{u}^{t,l})\,(\zeta-\gamma\,\alpha_u^b)\big]\,du\\
    &= \tilde{H}(t,l,q)+\mathbb{E}\int_t^{s\wedge \tau_n}\Big[\phi(L_{u}^{t,l})\,(Q_{u}^{t,l,q,\boldsymbol{\alpha}})^2\\
    &\hspace{2cm}-\alpha_u^a\,a(L_{u}^{t,l})\,(\zeta-\gamma\,\alpha_u^a)-\alpha_u^b\,b(L_{u}^{t,l})\,(\zeta-\gamma\,\alpha_u^b)\big]\,du.
\end{aligned}
\end{equation*}
Note that
\begin{equation*}
\begin{aligned}
    &\big|\int_t^{s\wedge \tau_n}\Big[\phi(L_{u}^{t,l})\,(Q_{u}^{t,l,q,\boldsymbol{\alpha}})^2-\alpha_u^a\,a(L_{u}^{t,l})\,(\zeta-\gamma\,\alpha_u^a)-\alpha_u^b\,b(L_{u}^{t,l})\,(\zeta-\gamma\,\alpha_u^b)\big]\,du\,\big|\\
    &\hspace{2cm}\leq \int_t^{T}\big|\phi(L_{u}^{t,l})\,(Q_{u}^{t,l,q,\boldsymbol{\alpha}})^2-\alpha_u^a\,a(L_{u}^{t,l})\,(\zeta-\gamma\,\alpha_u^a)\\
    &\hspace{3cm}-\alpha_u^b\,b(L_{u}^{t,l})\,(\zeta-\gamma\,\alpha_u^b)\big|\,du
\end{aligned}
\end{equation*}
and
\begin{equation*}
    \tilde{H}(s\wedge\tau, L_{s\wedge\tau_n}^{t,l}, Q_{s\wedge\tau_n}^{t,l,q,\boldsymbol{\alpha}})\leq C\,(1+\sup_{u\in[t,T]}|Q_{u}^{t,l,q,\boldsymbol{\alpha}}|^2).
\end{equation*}
The right hand sides of both above terms are integrable and we apply the dominated convergence theorem twice to see 
\begin{equation*}
\begin{aligned}
    \mathbb{E}\big[\tilde{H}(s, &L_{s}^{t,l}, Q_{s}^{t,l,q,\boldsymbol{\alpha}})\big]\leq \tilde{H}(t,l,q)\\
    &+\mathbb{E}\int_t^{s}\Big[\phi(L_{u}^{t,l})\,(Q_{u}^{t,l,q,\boldsymbol{\alpha}})^2-\alpha_u^a\,a(L_{u}^{t,l})\,(\zeta-\gamma\,\alpha_u^a)-\alpha_u^b\,b(L_{u}^{t,l})\,(\zeta-\gamma\,\alpha_u^b)\big]\,du,
\end{aligned}
\end{equation*}
and further
\begin{equation*}
\begin{aligned}
    \mathbb{E}\big[ -A(L_{T}^{t,l})\,(Q_{T}^{t,l,q,\boldsymbol{\alpha}})^2 \big] &\leq \tilde{H}(t,l,q)\\
    &+\mathbb{E}\int_t^{T}\Big[\phi(L_{u}^{t,l})\,(Q_{u}^{t,l,q,\boldsymbol{\alpha}})^2-\alpha_u^a\,a(L_{u}^{t,l})\,(\zeta-\gamma\,\alpha_u^a)\\
    &\hspace{3.5cm}-\alpha_u^b\,b(L_{u}^{t,l})\,(\zeta-\gamma\,\alpha_u^b)\big]\,du.
\end{aligned}
\end{equation*}
Therefore, the arbitrariness of $\boldsymbol{\alpha}$ implies
\begin{equation*}
    H(t,l,q)\leq \tilde{H}(t,l,q).
\end{equation*}
To see the reversed inequality, let $Q^{t,l,q,\boldsymbol{\delta}^*}$ be the associated inventory of the control $\boldsymbol{\delta}$ defined in \eqref{uniq_Mark_cont} and apply the It\^o's formula to $\tilde{H}(u, L_{u}^{t,l},  Q_{u}^{t,l,q,\boldsymbol{\delta}^*})$ between $t\in[0,T)$ and $s\in[t,T)$. Through similar localizing and limiting procedures, one can obtain
\begin{equation*}
\begin{aligned}
    -A(L_{T}^{t,l})\,(Q_{T}^{t,l,q,\boldsymbol{\delta}^*})^2&=\tilde{H}(t,l,q)+\mathbb{E}\int_t^{T}\big[\frac{\partial \tilde{H}}{\partial t}(u, L_{u}^{t,l}, Q_{u}^{t,l,q,\boldsymbol{\delta}^*})+\mathcal{L}\tilde{H}(u, L_{u}^{t,l}, Q_{u}^{t,l,q,\boldsymbol{\delta}^*})\\
    &\hspace{3cm}-\frac{\partial \tilde{H}}{\partial q}\,a(L_{u}^{t,l})\,(\zeta-\gamma\,\delta_u^{a,*})\\
    &\hspace{3.8cm}+\frac{\partial \tilde{H}}{\partial q}\,b(L_{u}^{t,l})\,(\zeta-\gamma\,\delta_u^{b,*})\big]\,du\\
    &= \tilde{H}(t,l,q)+\mathbb{E}\int_t^{T}\Big[\phi(L_{u}^{t,l})\,(Q_{u}^{t,l,q,\boldsymbol{\delta}^*})^2\\
    &\hspace{2cm}-\alpha_u^a\,a(L_{u}^{t,l})\,(\zeta-\gamma\,\delta_u^{a,*})-\alpha_u^b\,b(L_{u}^{t,l})\,(\zeta-\gamma\,\delta_u^{b,*})\big]\,du.
\end{aligned}
\end{equation*}
To finally conclude that
    \begin{equation*}
    H(t,l,q)= \tilde{H}(t,l,q),
\end{equation*}
it suffices to show that  $\boldsymbol{\delta}^*\in\mathbb{H}^2$. Since the inventory satisfies the linear random ODE
\begin{equation*}
\begin{aligned}
    dQ_{u}^{t,l,q,\boldsymbol{\delta}^*}&=-a_u\,(\zeta-\gamma\,\delta_u^{a,*})\,du+b_u\,(\zeta-\gamma\,\delta_u^{b,*})\,du\\
    &=\frac{1}{2}\,[\zeta\,(b_u-a_u)+\gamma\,h_1(u,l)\,(a_u-b_u)]\,du+\gamma\,(a_u+b_u)\,h_2(u,l)\,Q_{u}^{t,l,q,\boldsymbol{\delta}^*}\,du,
\end{aligned}
\end{equation*}
the process $Q^{t,l,q,\boldsymbol{\delta}^*}$ turns out to be bounded due to the boundedness of $a, b, h_1,$ and $h_2$. Consequently, the control $\boldsymbol{\delta}^*$ is also bounded and thus admissible.

(3) It suffices to prove the uniqueness of the optimal control. The statement is true if the objective functional \eqref{equ_obj_functional} is strictly concave. The strict concavity of \eqref{equ_obj_functional} can be deduced from the strict concavity of $\delta_u^a\,a_u(\zeta-\gamma\delta_u^a)+\delta_u^b\,b_u(\zeta-\gamma\delta_u^b)$ and the convexity of $Q_t^2$ with respect to the control. The strict concavity of $\delta_u^a\,a_u(\zeta-\gamma\delta_u^a)+\delta_u^b\,b_u(\zeta-\gamma\delta_u^b)$ is clear because they are second order polynomials with negative signs for the leading terms. Let $\boldsymbol{\delta}, \boldsymbol{\beta}\in\mathbb{H}^2\times\mathbb{H}^2$ be two generic controls, and set $\lambda\in[0,1]$. We check the convexity of the quadratic inventory through
\begin{equation*}
\begin{aligned}
    (Q_u^{\lambda\,\boldsymbol{\delta}+(1-\lambda)\,\boldsymbol{\beta}})^2&-\lambda\,(Q_u^{\boldsymbol{\delta}})^2-(1-\lambda)\,(Q_u^{\boldsymbol{\beta}})^2\\
    &=\big(\lambda\,Q_u^{\boldsymbol{\delta}}+(1-\lambda)\,Q_u^{\boldsymbol{\beta}}\big)^2-\lambda\,(Q_u^{\boldsymbol{\delta}})^2-(1-\lambda)\,(Q_u^{\boldsymbol{\beta}})^2\\
    &\leq0,
\end{aligned}
\end{equation*}
where we have applied the linearity of the inventory with respect to the control. 
\end{proof}

\kong

We intend to prove the existence and uniqueness of the classical solution to \eqref{non_lin_pde} on the basis of \cite{becherer2005classical}, the result of which cannot be applied directly because of the quadratic term $h_2(t,l)^2$.  To begin with, we introduce two auxiliary independent linear PDEs
\begin{gather*}
    \frac{\partial h}{\partial t}+\mathcal{L}h(t,l)+\gamma\big(a(l)+b(l)\big)\,g(t,l)\,h(t,l)-\phi(l)=0, \quad \text{such that}\;\; h(T,l)=-A; \label{pde_1}\\
    \frac{\partial h}{\partial t}+\mathcal{L}h(t,l)+\gamma\big(a(l)+b(l)\big)\,\big[\psi\big(g(t,l)\big)\big]^2-\phi(l)=0, \quad \text{such that}\;\; h(T,l)=-A,
\end{gather*}
where $\psi:\mathbb{R}\to\mathbb{R}$ is a truncation function with the parameter $\xi>0$ defined by
\begin{equation*}
    \psi(x)=
    \begin{cases}
    -\xi, \quad &\text{if } x<-\xi\\
    x, \quad &\text{if } -\xi\leq x\leq\xi\\
    \xi, \quad &\text{if } x>\xi,
    \end{cases}
\end{equation*}
and $g$ is a bounded continuous function to be specified later. If the function $g$ and the solution of the linear PDE are considered as input and output accordingly, we can define two mappings $F_1$ and $F_2$ for the two equations by the Feynman-Kac formula:
\begin{gather*}
    (F_1\,g)(t,l)=\mathbb{E}\Big[-A\,e^{\int_t^T \gamma (a_s+b_s)\,g(s, L_s^{t,l})\,ds}-\int_t^T \phi_s\,e^{\int_t^s \gamma (a_s+b_s)\,g(u, L_u^{t,l})\,du}\,ds\Big],\\
    (F_2\,g)(t,l)=\mathbb{E}\Big[-A+\int_t^T \gamma (a_s+b_s)\,\big(\psi(g(s,L_s^{t,l}))\big)^2\,ds-\int_t^T \phi_s\,ds\Big].
\end{gather*}

\noindent It turns out that $F_1$ and $F_2$ are contraction mappings in the space of bounded functions.

\kong

\begin{lemma}
Denote by $C_b^-$ the set of bounded, non-positive and continuous functions on $[0,T]\times \mathbb{R}^d$, then the following statements hold:\\
\indent (1) $F_1$ defines a contraction mapping on $C_b^-$ with respect to the norm
\begin{equation}
    \|v\|_{\zeta_1}:=\sup_{(t,x)\in[0,T]\times \mathbb{R}^d}e^{-\zeta_1\,(T-t)}\cdot|v(t,x)|,
    \nonumber
\end{equation}
for some $\zeta_1\in\mathbb{R}_+$ large enough.\\
\indent (2) $F_2$ defines a contraction mapping on $C_b$ with respect to the norm
\begin{equation}
    \|v\|_{\zeta_2}:=\sup_{(t,x)\in[0,T]\times \mathbb{R}^d}e^{-\zeta_2\,(T-t)}\cdot|v(t,x)|,
    \nonumber
\end{equation}
for some $\zeta_2\in\mathbb{R}_+$ large enough.
\end{lemma}

\begin{proof}
We first introduce an inequality that will be used for both statements. For $\zeta_1>0$ and some bounded functions $v$ and $w$, one can see
\begin{equation}
    \begin{aligned}
        \int_t^T |v(s, L_s)-w(s, L_s)|\,ds &= \int_t^T |v(s, L_s)-w(s, L_s)| e^{-\zeta_1\,(T-s)}\,e^{\zeta_1\,(T-s)}\,ds\\
        &\leq\|v-w\|_{\zeta_1}\int_t^T e^{\zeta_1\,(T-s)}\,ds\\
        &=\frac{e^{\zeta_1\,(T-t)}-1}{\zeta_1}\cdot\|v-w\|_{\zeta_1}.
    \label{aux_ineq}
    \end{aligned}
\end{equation}
For any $v\in C_b^-$, it is clear that $F_1v$ is also a bounded negative function. Since the term inside the conditional expectation is continuous and uniformly bounded by $\bar{A}+\bar{\phi}\,T$, the continuity of $F_1v$ follows from the dominated convergence theorem. Consider any $v$, $w\in C_b^-$ and it follows that, for all $(t,l)\in[0,T]\times \mathbb{R}^d$, 
\begin{equation}
    \begin{aligned}
    e^{-\zeta_1\,(T-t)}\,&|(F_1\,v)(t, l)-(F_1\,w)(t, l)|\\
    &\leq e^{-\zeta_1\,(T-t)}\, \mathbb{E}\Big[A\,|e^{\int_t^T \gamma (a_s+b_s)\,v(s, L_s^{t,l})\,ds}-e^{\int_t^T \gamma (a_s+b_s)\,w(s, L_s^{t,l})\,ds}|\\
    &\hspace{2.5cm}+\bar{\phi}\int_t^T|e^{\int_t^s \gamma (a_u+b_u)\,v(u, L_u^{t,l})\,du}-e^{\int_t^s \gamma (a_u+b_u)\,w(u, L_u^{t,l})\,du}|\,ds\Big]\\
    &\leq e^{-\zeta_1\,(T-t)}\, \mathbb{E}\Big[A\,\big|\int_t^T \gamma (a_s+b_s)\,v(s, L_s^{t,l})\,ds-\int_t^T \gamma (a_s+b_s)\,w(s, L_s^{t,l})\,ds\big|\\
    &\hspace{2.5cm}+\bar{\phi}\int_t^T\big|\int_t^s \gamma (a_u+b_u)\,v(u, L_u^{t,l})\,du\\
    &\hspace{4,5cm}-\int_t^s \gamma (a_u+b_u)\,v(u, L_u^{t,l})\,du\big|\,ds\Big]\\
    &\leq e^{-\zeta_1\,(T-t)}\cdot \gamma\,(\bar{a}+\bar{b})\,(\bar{A}+\bar{\phi}\,T)\cdot \mathbb{E}\Big[\int_t^T |v(s, L_s^{t,l})-w(s, L_s^{t,l})|\,ds\Big]\\
    &\leq \frac{\gamma\,(\bar{a}+\bar{b})\,(\bar{A}+\bar{\phi}\,T)}{\zeta_1}\cdot\|v-w\|_{\xi_1}
    \nonumber
    \end{aligned}
\end{equation}
where the positiveness of $a, b$ and negativeness of $v$, $w$ are used to derive the second inequality, and the proposed inequality \eqref{aux_ineq} is applied in the last line. For a contraction mapping, it suffices to pick $\zeta_1\in\mathbb{R}_+$ such that 
\begin{equation}
    \zeta_1>\gamma\,(\bar{a}+\bar{b})\,(\bar{A}+\bar{\phi}\,T).
    \nonumber
\end{equation}
\indent Similarly, we look at the second statement and compute that
\begin{equation}
    \begin{aligned}
    e^{-\zeta_2\,(T-t)}\,|(F_2\,v)(t, l)&-(F_2\,w)(t, l)|\\
    &=e^{-\zeta_2\,(T-t)}\, \mathbb{E}\big|\int_t^T\gamma (a_s+b_s)\,\big[\psi(v(s,L_s^{t,l}))^2-\psi(w(s, L_s^{t,l}))^2\big]\,ds\,\big|\\
    &\leq2\xi\,e^{-\zeta_2\,(T-t)}\, \mathbb{E}\int_t^T\big|\gamma (a_s+b_s)\,\big[\psi(v(s,L_s^{t,l}))-\psi(w(s,L_s^{t,l}))\big]\,\big|\,ds\\
    &\leq \frac{2\xi\,\gamma\,(\bar{a}+\bar{b})}{\zeta_2}\cdot\|v-w\|_{\xi_2}.
    \nonumber
    \end{aligned}
\end{equation}
Again, it suffices to pick $\zeta_2$ large enough such that $\zeta_2>2\xi\,\gamma\,(\bar{a}+\bar{b})$ for a contraction mapping. The boundedness and continuity of $F_2v$ follows from the uniform boundedness and continuity inside the conditional expectation. 
\end{proof}

\kong

\noindent Due to the Banach fixed-point theorem, both $F_1$ and $F_2$ have their own fixed points. One reason why we consider both two mappings is that they actually share the same fixed point if the constant $\xi$ is sufficiently large.

\kong

\begin{lemma}
Let $v\in C_b^-$ be the fixed point of the $F_1$, then it is also the one of $F_2$, provided that the constant $\xi$ of the truncation $\psi$ is large enough.
\end{lemma}
\begin{proof}
Since $v$ is the fixed point of $F_1$, for any $(t,l)\in[0,T]\times \mathbb{R}^d$ it holds
\begin{equation}
    v(t, l)=\mathbb{E}\Big[-A\,e^{\int_t^T \gamma\,(a_s+b_s)\,v(s, L_s^{t,l})\,ds}-\int_t^T\phi_s\, e^{\int_t^s \gamma\,(a_u+b_u)\,v(u,L_u^{t,l})\,du}\,ds\Big].
    \nonumber
\end{equation}
We take the derivative of $\exp(\int_t^T\gamma (a_s+b_s)\,v(s,L_s^{t,l})\,ds)$ to obtain
\begin{equation*}
\begin{aligned}
    de^{\int_r^T \gamma (a_s+b_s)\,v(s,L_s^{t,l})\,ds}&=-\gamma (a_r+b_r)\,v(r,L_r^{t,l})\,e^{\int_r^T\gamma (a_s+b_s)\,v(s,L_s^{t,l})\,ds}\,dr,\\
    A-A\,e^{\int_t^T\gamma (a_s+b_s)\,v(s,L_s^{t,l})\,ds}&=-A\int_t^T\gamma (a_s+b_s)\,v(s,L_s^{t,l})\,e^{\int_s^T\gamma (a_u+b_u)\,v(u,L_u^{t,l})\,du}\,ds,
\end{aligned}
\end{equation*}
and then take the expectation to see
\begin{equation}
\begin{aligned}
   \mathbb{E}\big[ A-A\,e^{\int_t^T\gamma (a_s+b_s)\,v(s,L_s^{t,l})\,ds}\big]&=\mathbb{E}\int_t^T\gamma (a_s+b_s)\,v(s,L_s^{t,l})\\
    &\hspace{0.5cm}\cdot\mathbb{E}_s\big[(-A)\,e^{\int_s^T\gamma (a_u+b_u)\,v(u,L_u^{t,l})\,du}\big]\,ds,\\
    v(t, l)+\mathbb{E}\Big[\int_t^T\phi_s\,e^{\int_t^s \gamma\,(a_u+b_u)\,v(u,L_u^{t,l})\,du}\,ds\Big]&=\mathbb{E}[-A] + \mathbb{E}\int_t^T\gamma (a_s+b_s)\,v(s,L_s^{t,l})\\
    &\hspace{0.5cm}\cdot\mathbb{E}_s\big[(-A)\,e^{\int_s^T\gamma (a_u+b_u)\,v(u,L_u^{t,l})\,du}\big]\,ds.
    \label{fix_1}
\end{aligned}
\end{equation}
If one notes that
\begin{equation}
\begin{aligned}
    &\mathbb{E}\int_t^T\gamma (a_s+b_s)\,v(s,L_s^{t,l})\,\mathbb{E}_s\big[(-A)\,e^{\int_s^T\gamma (a_u+b_u)\,v(u,L_u^{t,l})\,du}\big]\,ds\\
    &=\mathbb{E}\int_t^T\gamma (a_s+b_s)\,v(s,L_s^{t,l})\,\big(v(s,L_s^{t,l})+\mathbb{E}_s\int_s^T \phi_r\,e^{\int_s^r \gamma (a_u+b_u) v(u,L_u^{s,L_s^{t,l}})\,du}\,dr\big)\,ds,
    \label{fix_2}
\end{aligned}
\end{equation}
which follows from the fixed point property, and also
\begin{equation}
\begin{aligned}
    \mathbb{E}\Big[\int_t^T&\gamma\,(a_s+b_s)\,v(s, L_s^{t,l})\,\mathbb{E}_s\big(\int_s^T\phi_r\, e^{\int_s^r \gamma\,(a_u+b_u)\,v(u, L_u^{s,L_s^{t,l}})\,du}\,dr\big)\,ds\Big]\\
    &=\mathbb{E}\Big[\int_t^T\phi_r\,\big(\int_t^r \gamma\,(a_s+b_s)\,v(s, L_s^{t,l})\,e^{\int_s^r \gamma\,(a_u+b_u)\,v(u, L_u^{t,l})\,du}\,ds\big)\,dr\Big]\\
    &=\mathbb{E}\Big[\int_t^T\phi_r\,\big(e^{\int_t^r \gamma\,(a_u+b_u)\,v(u, L_u^{t,l})\,du}-1\big)dr\Big],
\end{aligned}
\label{fix_3}
\end{equation}
one can observe the following result via \eqref{fix_1}--\eqref{fix_3}:
\begin{equation}
    v(t,l)=\mathbb{E}\Big[-A+\int_t^T \gamma\,(a_s+b_s)\,v(s, L_s^{t,l})^2\,ds-\int_t^T \phi_s\,ds\Big].
    \nonumber
\end{equation}
Since $v$ is the fixed point of $F_1$, it's clear that $v$ is uniformly bounded by $\bar{A}+\bar{\phi}\,T$ and, if we pick $\xi\geq \bar{A}+\bar{\phi}\,T$, the equation above is equivalent to
\begin{equation}
    v(t, l)=\mathbb{E}\Big[-A+\int_t^T \gamma\,(a_s+b_s)\,\psi(v(s,L_s^{t,l}))^2\,ds-\int_t^T \phi_s\,ds\Big],
    \nonumber
\end{equation}
showing that $v$ is also the fixed point of $F_2$. 
\end{proof}

\kong

\noindent The other reason why we consider both two mappings is that the mapping $F_2$ has been studied by \cite{becherer2005classical} under some relative general assumptions. We can now extend its result to our equation \eqref{non_lin_pde}.

\kong

\begin{theorem}
\label{pde_thm}
The equation \eqref{non_lin_pde} accepts a unique classical solution in $C_b^{1,2}$. More specifically, the solution takes value in $[-\bar{A}-\bar{\phi}\,T,0)$.
\end{theorem}

\begin{proof}
We start to look at the PDE \eqref{non_lin_pde} with a regularization:
\begin{equation}
        \frac{\partial h_2}{\partial t}+\mathcal{L}h_2(t,l)+\gamma\big(a(l)+b(l)\big)\,\psi(h_2(t,l))^2-\phi(l)=0, \quad \text{such that}\;\; h_2(T, l)=-A(l).
    \label{loc_pde}
\end{equation}
In order to apply the proposition 2.3 from \cite{becherer2005classical}, for convenience we set $\iota(t,l,v)=\gamma\, (a(l)+b(l))\,\psi(v)^2-\phi(l)$ and check the following set of conditions:

\begin{itemize}
    \item [1.] functions A and $\iota$ are continuous and bounded, with $\iota$ being Lipschitz in $v$ uniformly in $t$ and $l$;\\
    \vspace{-0.2cm}

    \item [2.] there exists a sequence $(D_n)_{n\in\mathbb{N}}$ of bounded domains with closure $\bar{D}_n \subseteq \mathbb{R}^d$ such that $\bigcup^\infty_
    {n=1}D_n = \mathbb{R}^d$ and each $D_n$ has a $C^2$-boundary;\\
    \vspace{-0.2cm}
    
    \item [3.] the functions $\Gamma$ and $\Sigma\,\Sigma^{tr}$ are uniformly Lipschitz-continuous on $[0,T] \times\bar{D}_n$;\\
        \vspace{-0.2cm}
    
    \item [4.] $\det
    (\Sigma(t,l)\,\Sigma^{tr}(t,l))\neq0$ for all $(t,l)\in[0,T] \times \mathbb{R}^d$;\\
        \vspace{-0.2cm}

    \item [5.] $(t, l, v)\mapsto \iota(t, l, v)$ is uniformly H\"{o}lder-continuous on $[0,T]\times \bar{D}_n\times \mathbb{R}$.
\end{itemize}

\noindent The first two conditions can be verified directly based on the Assumption \ref{lip_ord_flow}, while the third and fourth ones follow from the Assumption \ref{sde_assumption}. For any $(t, l, v), (t', l', v')\in [0,T]\times \bar{D}_n\times \mathbb{R}$, note that
\begin{equation}
\begin{aligned}
    |\iota(t', l', v')-\iota(t, l, v)|&\leq\gamma\,(a(l')+b(l'))\,|\psi(v')^2-\psi(v)^2|\\
    &\hspace{1.8cm}+\gamma\,\psi(v)^2\,|a(l')+b(l')-a(l)-b(l)|+|\phi(l')-\phi(l)|\\
    &\leq2\,\gamma\,\xi\, (\bar{a}+\bar{b})\,|v'-v|+2\,\gamma\,\xi^2\,C_{n}\,|l'-l|+C_n\,|l'-l|,
    \label{lip_iota}
\end{aligned}
\end{equation}
where $C_n$ denotes the Lipschitz coefficient of $a$, $b$ and $\phi$ in the domain $\bar{D}_n$ using the Assumption \ref{lip_ord_flow}. Equation \eqref{lip_iota} shows that $\iota$ is Lipschitz-continuous on $[0,T]\times \bar{D_n}\times \mathbb{R}$, and hence uniformly H\"{o}lder-continuous due to the boundedness. Consequently, the Proposition 2.3 from \cite{becherer2005classical} tells us that the `regular' PDE \eqref{loc_pde} has a unique classical solution in $C_b^{1,2}$, which is given by the fixed point $\hat{v}$ of the mapping $F_2$. Since $F_1$ and $F_2$ share the same fixed point, $\hat{v}$ is also the fixed point of $F_1$ and we already know that $\hat{v}$ is bounded by $\bar{A}+\bar{\phi}\,T$. Therefore, if $\xi$ is chosen to be larger than $\bar{A}+\bar{\phi}\,T$, it follows that 
\begin{equation}
\begin{aligned}
        \frac{\partial \hat{v}}{\partial t}+\mathcal{L}\hat{v}(t,l)&+\gamma\big(a(l)+b(l)\big)\,\psi(\hat{v}(t,l))^2-\phi(l)\\
        &=\frac{\partial \hat{v}}{\partial t}+\mathcal{L}\hat{v}(t,l)+\gamma\big(a(l)+b(l)\big)\,\hat{v}(t,l)^2-\phi(l)=0,
    \nonumber
\end{aligned}
\end{equation}
establishing the existence of solutions.

For the uniqueness, suppose that $u\in C_b^{1,2}$ is another solution to the PDE \eqref{non_lin_pde}. Consider another truncation function $\tilde{\psi}$ with the parameter 
$\tilde{\xi}$ satisfying
\begin{equation}
    \tilde{\xi}\;\geq\;(\bar{A}+\bar{\phi}\,T)\;\vee\sup_{(t, l)\in[0,T]\times \mathbb{R}^d}|u(t, l)|.
    \nonumber
\end{equation}
Note that both $\hat{v}$ and $u$ solve the following `regular' PDE:
\begin{equation}
        \frac{\partial h}{\partial t}+\mathcal{L}h(t,l)+\gamma\big(a(l)+b(l)\big)\,\tilde{\psi}(h(t,l))^2-\phi(l)=0, \quad \text{such that}\;\; h(T, l)=-A(l).
    \nonumber
\end{equation}
which has a unique solution according to previous discussions. Hence, $\hat{v}\equiv u$ and the uniqueness is obtained. 
\end{proof}

\kong

\begin{remark}
\label{unbound_order_flow}
\noindent Our method can be further generalized to unbounded functions $a$ and $b$. If $a, b$ satisfy proper growth conditions, consider equations
\begin{equation*}
    \frac{\partial h_n}{\partial t}+\mathcal{L}h_n(t,l)+\gamma\big(a(l)\wedge n+b(l)\wedge n\big)\,h_n(t,l)^2-\phi(l)=0, \quad \text{such that}\;\; h_n(T, l)=-A(l),
\end{equation*}
for $n\in \mathbb{N}$. A sequence of solutions $(h_n)_{n\in\mathbb{N}}$, being uniformly bounded, is guaranteed by the Theorem \ref{pde_thm} and is non-decreasing by the comparison result. If we denote by $g$ as the limit of the sequence, it turns out that $g$ satisfies the conditional representation
\begin{equation*}
    g(t,l)=\mathbb{E}\Big[-A+\int_t^T \gamma (a_s+b_s)\,\big(\psi(g(s,L_s^{t,l}))\big)^2\,ds-\int_t^T \phi_s\,ds\Big]
\end{equation*}
for a truncation function $\psi$ with the parameter $\bar{A}+\bar{\phi}\,T$. Then, one can use the proof of \cite{becherer2005classical} to show that $g\in C^{1,2}$ solves the PDE. We do not cover it here, because the boundedness of order flows guarantees the square integrability of the inventory (and thus  \eqref{martingale_price}). But, the idea will be illustrated in the later section through some other equations.
\end{remark}

\kong

\begin{example}
We conclude this section with a particular example of \eqref{non_lin_pde} that has an explicit solution. Some economic intuition is also helpful in a later proof. First, the boundedness of $\partial h_2/\partial l$ needs to be derived. For this purpose, several additional conditions are imposed:
\begin{itemize}
    \item each element in the matrix-valued function $\Sigma(t,l)$ is a bounded function;\\
    \vspace{-0.2cm}

    \item $\phi\equiv0$, i.e., the agent is risk neutral;\\
    \vspace{-0.2cm}
    
    \item functions $a, b$, and $A$ are all uniformly Lipschitz, the Lipschitz coefficient of which is denoted by $C_{\text{Lip}}$.
\end{itemize}
Define the function $g:[0,T]\times\mathbb{R}^d\to \mathbb{R}$ as
\begin{equation*}
    g(t,l)= \frac{h_2(t,l+\epsilon\,e^i)-h_2(t,l)}{\epsilon},
\end{equation*}
where $\epsilon>0$ is a constant and $e^i$ is the unit vector of the $i$-coordinate in $\mathbb{R}^d$ for $i\in\{1,\dots, d\}$. Applying a similar transform to the equation \eqref{non_lin_pde}, one can observe that $g$ satisfies the following linear PDE:
\begin{equation}
\begin{aligned}
    \frac{\partial g}{\partial t}+\mathcal{L}g(t,l)&+\gamma\big(a(l+\epsilon\,e^i)+b(l+\epsilon\,e^i)\big)\,\big(h_2(t,l+\epsilon\,e^i)+h_2(t,l)\big)\,g(t,l)\\
    &+\big(\tilde{a}(l)+\tilde{b}(l)\big)\,h_2(t,l)=0,
\end{aligned}
\label{linear_pde_diff}
\end{equation}
such that $g(T,l)=-\tilde{A}(l)$, where we have defined 
\begin{equation*}
    \tilde{\theta}(l)=\frac{\theta(t,l+\epsilon\,e^i)-\theta(t,l)}{\epsilon}
\end{equation*}
for $\theta\in\{a, b, A\}$. To verify the term $(a(l+\epsilon\,e^i)+b(l+\epsilon\,e^i))\,(h_2(t,l+\epsilon\,e^i)+h_2(t,l))$ is uniformly H\"{o}lder-continuous, it suffices to notice that:
\begin{itemize}
    \item functions $a$ and $b$ are uniformly Lipschitz;
    \vspace{0.2cm}

    \item function $h_2(t,l)$ is uniformly Lipschitz on $[0,T]\times D$ for any compact set $D\subset \mathbb{R}^d$, since partial derivatives of $h_2$ are continuous;
    \vspace{0.2cm}

    \item functions $a, b$, and $h_2$ are uniformly bounded.
\end{itemize}
While the other conditions can be checked in a similar way, one can again learn from \cite{becherer2005classical} that the PDE \eqref{linear_pde_diff} accepts a unique solution $g\in C_b^{1,2}$. The solution $g$ possesses the representation
\begin{equation*}
\begin{aligned}
    &g(t,l)=\mathbb{E}\Big[-\Tilde{A}(L_T^{t,l})\,e^{\int_t^T \gamma(a(L_s^{t,l}+\epsilon\,e^i)+b(L_s^{t,l}+\epsilon\,e^i))\,(h_2(s, L_s^{t,l}+\epsilon\,e^i)+h_2(s, L_s^{t,l}))\,ds}\\
    &\hspace{1cm} +\int_t^T (\tilde{a}(L_s^{t,l})+\tilde{b}(L_s^{t,l}))\,h_2(s,L_s^{t,l})\\
    &\hspace{4cm}\cdot e^{\int_t^s \gamma(a(L_u^{t,l}+\epsilon\,e^i)+b(L_u^{t,l}+\epsilon\,e^i))\,(h_2(u,L_u^{t,l}+\epsilon\,e^i)+h_2(u,L_u^{t,l}))\,du}\Big],
\end{aligned}
\end{equation*}
which further implies
\begin{equation}
    |g(t,l)|\leq C_{\text{Lip}}\,e^{2\gamma T(\bar{a}+\bar{b})(\bar{A}+\bar{\phi}\,T)}+2\,T\,C_{\text{Lip}}(\bar{A}+\bar{\phi}\,T)\,e^{2\gamma T(\bar{a}+\bar{b})(\bar{A}+\bar{\phi}\,T)}
    \label{upper_bound_linear_pde}
\end{equation}
for all $(t,l)\in[0,T]\times\mathbb{R}^d$. We emphasize that the right hand side of \eqref{upper_bound_linear_pde} is independent of $\epsilon$. It follows $\partial h_2/\partial l^i$ is uniformly bounded and the same is true for all elements in the vector $\partial h_2/\partial l$.

Let $Y_t:=h_2(t,L_t)$, $Z_t:=(\partial h_2/\partial l)(t,L_t)\,\Sigma(t,L_t)$, and then apply It\^o's formula to obtain
\begin{equation*}
    Y_t=-A+\int_t^T \gamma\,(a_s+b_s)\,Y_s^2\,ds-\int_t^TZ_s\,dW_s.
\end{equation*}
Because $Y_t<0$, to observe the following we use It\^o's formula again:
\begin{equation}
\begin{aligned}
    d(Y_t^{-1})&=-A^{-1}-\int_t^T \gamma\,(a_s+b_s)\,ds-\int_t^T\frac{Z_s}{Y_s^2}\big(dW_s-\frac{Z_s}{Y_s}\,ds\big)\\
    &=-A^{-1}-\int_t^T \gamma\,(a_s+b_s)\,ds-\int_t^T\frac{Z_s}{Y_s^2}\,d\tilde{W}_s,
\end{aligned}
\label{chang_measur}
\end{equation}
where $\tilde{W}$ is a standard Brownian motion with respect to some probability measure $\mathbb{Q}$, guaranteed by the Girsanov theorem. Indeed, the Girsanov transform can be applied since $Z_t$ is bounded and $Y_t<-A\,\exp{(-\gamma\,(\bar{a}+\bar{b})\,A\,T)}$ by the mapping $F_1$. An explicit form of $h_2$ is obtained via taking the conditional expectation of \eqref{chang_measur}:
\begin{equation*}
    h_2(t,L_t)=Y_t=\bigg\{\,\mathbb{E}_t^\mathbb{Q} \Big[-A^{-1}-\int_t^T \gamma\,(a_s+b_s)\,ds\Big]\,\bigg\}^{-1}.
\end{equation*}
The first-order approximation, mentioned in \cite{barger2019optimal} and \cite{fouque2022optimal}, turns out to be the rectangle approximation of the integral $\int_t^T (a_s+b_s)\,ds$. In our context, the integral $\int_t^T (a_s+b_s)\,ds$ stands for the total volume of future order flows. While $h_2$ is the coefficient of the second-order term in $H$, the coefficient of the first-order term $h_1$---by the Feynman–Kac Formula---can be represented by
\begin{equation*}
    h_1(t, l )=\zeta\,\mathbb{E}_{t, l}\Big[\int_t^T (b_s-a_s)\cdot h_2(s, L_s)\,e^{\gamma\int_t^s (a_u+b_u)\,h_2(u, L_u)\,du}\,ds\Big],
\end{equation*}
which estimates a weighted sum of future order imbalances.
\end{example}

\vspace{0.2cm}

\section{Non-Markovian Order Flows: General Intensity Functions} \label{section_3}
\noindent In this section, we extend the market making problem in two directions. First, the order flows and penalty parameters are not necessarily Markovian.

\kong

\begin{assumption}
Let $a$, $b\in\mathbb{H}^2$ be positive processes and $\phi\in\mathbb{H}^2$ be a non-negative process,  such that they are bounded by constants $\bar{a}$, $\bar{b},$ $\bar{\phi}>0$ respectively. Let $A\in L^2(\Omega,\mathcal{F}_T)$ be a non-negative random variable, bounded by $\bar{A}>0$.
\end{assumption}

\kong

\noindent On the other hand, more general intensity functions are considered. We borrow the following notations, definitions, and associated results from \cite{gueant2017optimal}.


\kong

\begin{assumption}[\cite{gueant2017optimal}]
\label{general_inten}
A function $\Lambda:\mathbb{R}\to\mathbb{R_+}$ belongs to the class of intensity functions $\boldsymbol{\Lambda}$ if:
\begin{itemize}
    \item[1.] $\Lambda$ is twice continuously differentiable;\\
    \vspace{-0.2cm}
    
    \item[2.] $\Lambda$ is strictly decreasing and hence $\Lambda'(x)<0$ for any $x\in\mathbb{R}$;\\
    \vspace{-0.2cm}
    
    \item[3.] $\lim_{x\to\infty}\Lambda(x)=0\,$ and $\;\sup_{x\in\mathbb{R}}\frac{\Lambda(x)\,\Lambda''(x)}{(\Lambda'(x))^2}<2$.
\end{itemize}
\label{inten_assu}
\end{assumption}

\kong

\begin{lemma}[\cite{gueant2017optimal}]
\label{inten_fun}
For any $\Lambda\in\boldsymbol{\Lambda}$, define the function $\mathcal{W}:\mathbb{R}\to\mathbb{R}$ as $\mathcal{W}(p)=\sup_{\delta\in\mathbb{R}}\Lambda(\delta)\,(\delta-p)$. Then, the following holds: 
\begin{itemize}
    \item[1.] $\mathcal{W}$ is a decreasing function of class $C^2$;\\
    \vspace{-0.2cm}
    
    \item[2.] The supremum in the definition of $\mathcal{W}$ is attained at a unique $\delta^*(p)$ characterized by 
    \begin{equation}
        \delta^*(p)=\Lambda^{-1}\big(-\mathcal{W}'(p)\big),
        \nonumber
    \end{equation}
    where $\Lambda^{-1}$ denotes the inverse function of $\Lambda$;\\
    \vspace{-0.2cm}
    
    \item[3.] The function $p\mapsto\delta^*(p)$ belongs to $C^1$ and is increasing. Its derivative reads
    \begin{equation}
        (\delta^{*})'(p)=\Big[2-\frac{\Lambda(\delta^*(p))\,\Lambda''(\delta^*(p))}{\Lambda'(\delta^*(p))^2}\Big]^{-1}>0.
        \nonumber
    \end{equation}
\end{itemize}
\end{lemma}

\kong

\noindent Denote by $(\Lambda^a,\Lambda^b)\in\boldsymbol{\Lambda}\times\boldsymbol{\Lambda}$ the intensity functions of the ask and bid sides. Additionally, a strategy $\boldsymbol{\delta}:=(\delta^a,\delta^b)$ is admissible if both two entries are bounded by $\xi$ for some large $\xi>0$, i.e., the admissible space is defined as
\begin{equation*}
    \mathbb{A}:=\{\delta\in \mathbb{H}^2: |\delta_t|\leq \xi \text{ for all } t\in[0,T]\,\}.
\end{equation*}    
The inventory and cash processes of the market maker are then given by
\begin{gather*}
        Q_t = q_0-\int_0^t a_u\Lambda^a(\delta_u^a)\,du+\int_0^t b_u\Lambda^b(\delta_u^b)\,du,\\
        \nonumber
    X_t = x_0+\int_0^t a_u\Lambda^a(\delta_u^a)\,(S_u+\delta_u^a)\,du-\int_0^t b_u\Lambda^b(\delta_u^b)\,(S_u-\delta_u^b)\,du.
    \nonumber
\end{gather*}
The agent intends to maximise the same objective functional:
\begin{equation*}
\begin{aligned}
    &\mathbb{E}\big[X_T+S_T\,Q_T-\int_0^T \phi_t\,(Q_t)^2\,dt-A\,(Q_T)^2 \big]\\
   &=\mathbb{E}\Big[\int_0^T\delta_t^a\,a_t\,\Lambda^a(\delta_t^a)\,dt+\int_0^T\delta_t^b\,b_t\,\Lambda^b(\delta_t^b)\,dt-\int_0^T \phi_t\,(Q_t)^2\,dt-A\,(Q_T)^2 \Big]
\end{aligned}
\end{equation*}
by controlling $\boldsymbol{\delta}\in\mathbb{A}\times\mathbb{A}$. To apply the Pontryagin maximum principle, the Hamiltonian of the agent with respect to the objective functional reads
\begin{equation}
    \mathscr{H}(t,Q_t,Y_t,\boldsymbol{\delta}_t)=\big[b_t\,\Lambda^b(\delta_t^b)-a_t\,\Lambda^a(\delta_t^a)\big]\,Y_t+b_t\,\delta_t^b\Lambda^b(\delta_t^b)+a_t\,\delta_t^a\Lambda^a(\delta_t^a)-\phi\,Q_t^2.
    \nonumber
\end{equation}
While the function $\mathscr{H}$ is concave in the state variable $Q$, the concavity with respect to the control $\boldsymbol{\delta}$ is not guaranteed, which is required by the usual stochastic maximum principle (for example \cite{carmona2016lectures}). However, the separation between the state variable and control enables us to still apply this principle, the proof of which is placed in the appendix. Denote by $(\delta^{a*},\delta^{b*})$ the maximisers associated with the intensity functions $(\Lambda^a, \Lambda^b)$ according to Lemma \ref{inten_fun}, and further define 
\begin{equation*}
    \tilde{\delta}^{i*}(p):=\delta^{i*}(p)\wedge\xi\vee(-\xi)
\end{equation*}
for $i\in\{a,b\}$. Then the optimal feedback controls on the ask and bid sides are given by 
\begin{equation}
    \tilde{\delta}^{a*}(Y_t) \quad\text{and}\quad \tilde{\delta}^{b*}(-Y_t).
    \nonumber
\end{equation}
Indeed, on the ask side, we look for the maximizer of 
\begin{equation*}
    \sup_{|\delta_t^a|\leq \xi}g(\delta_t^a):=\sup_{|\delta_t^a|\leq \xi}\Lambda^a(\delta_t^a)\cdot(\delta_t^a-Y_t).
\end{equation*}
Without the constraint $|\delta_t^a|\leq \xi$, the maximiser simply reads $\delta^{a*}(Y_t)$ according to Lemma \ref{inten_fun}. However, being aware of
\begin{equation*}
    g'(\delta_t^a)=\Lambda^{a'}(\delta_t^a)\,\big[\delta_t^a+\frac{\Lambda^a(\delta_t^a)}{\Lambda^{a'}(\delta_t^a)}-Y_t\big]
\end{equation*}
and the fact that $\delta_t^a+\Lambda^a(\delta_t^a)/\Lambda^{a'}(\delta_t^a)$ is an increasing function because
\begin{equation*}
    \big[\delta_t^a+\frac{\Lambda^a(\delta_t^a)}{\Lambda^{a'}(\delta_t^a)}\big]'=2-\frac{\Lambda^a(\delta_t^a)\,\Lambda^{a''}(\delta_t^a)}{\Lambda^{a'}(\delta_t^a)^2}>0,
\end{equation*}
one can see that the function $g$ first increases then decreases. Hence, the maximiser is $\tilde{\delta}^{a*}(Y_t)$ under the constraint stated in $\mathbb{A}$. The bid side can be derived in the same way. Combined with the stochastic maximum principle, the above discussion yields the next result.

\kong

\begin{theorem}
Let $\boldsymbol{\delta}\in\mathbb{A}\times\mathbb{A}$ be an admissible control, 
$Q = Q^{\boldsymbol{\delta}}$ be the corresponding controlled inventory, and $(Y, Z)$ be the adjoint processes. Then $\boldsymbol{\delta}$ is an optimal control if and only if it holds $\mathbb{P}$-a.s. that
\begin{equation}
    \mathscr{H}(t,Q_t,Y_t,\boldsymbol{\delta}_t)=\sup_{\boldsymbol{\alpha}\in [-\xi, \xi]^2}\mathscr{H}(t,Q_t,Y_t,\boldsymbol{\alpha}),  \qquad \text{a.e. in } t\in[0,T],
    \nonumber
\end{equation}
or equivalently
\begin{equation}
    \delta^a_t=\tilde{\delta}^{a*}(Y_t), \quad\text{and}\quad \delta^b_t=\tilde{\delta}^{b*}(-Y_t),  \qquad \text{a.e. in } t\in[0,T].
    \nonumber
\end{equation}
Further, the optimal inventory $Q$ together with the adjoint processes $(Y,Z)$ solves the FBSDE 
\begin{equation}
\left\{
\begin{aligned}
\;& dQ_t  = -a_t\,\Lambda^a\big(\tilde{\delta}^{a*}(Y_t)\big)dt+b_t\,\Lambda^b\big(\tilde{\delta}^{b*}(-Y_t)\big)dt, \\
& dY_t=2\phi_t\,Q_t\,dt+Z_t\,dW_t,\\
& Q_0=q_0,\quad Y_T=-2A\,Q_T.
\label{Lip_den_FBSDE}
\end{aligned}
\right.
\end{equation}
\end{theorem}
\begin{proof}
    See the \hyperref[section_5]{Appendix} for the stochastic maximum principle. 
\end{proof}

\kong

\noindent Equation \eqref{Lip_den_FBSDE} is a degenerate FBSDE with Lipschitz coefficients. To see this, note that $(\Lambda^a,\Lambda^b)$ have bounded derivatives in the admissible action space, and the same is true for $(\delta^{a*},\delta^{b*})$ since
\begin{equation}
        0<(\delta^{i*})'(p)=\Big[2-\frac{\Lambda^i(\delta^{i*}(p))\,\Lambda^{i''}(\delta^{i*}(p))}{\Lambda^{i'}(\delta^{i*}(p))^{2}}\Big]^{-1}\leq\Big[2-\sup_{u\in\mathbb{R}}\frac{\Lambda^i(u)\,\Lambda^{i''}(u)}{\Lambda^{i'}(u)^{2}}\Big]^{-1}<\infty,
    \nonumber
\end{equation}
for $i\in\{a,b\}$. The Lipschitz property of the forward equation of \eqref{Lip_den_FBSDE} then follows from the boundedness of these two derivatives and flows $a$, $b$. The remaining of section is devoted to the (global) well-posedness of the FBSDE \eqref{Lip_den_FBSDE}.

\kong

\begin{remark}
(1) The constant $\xi$ in the definition of $\mathbb{A}$ can be also interpreted as a regularizer for the equation \eqref{Lip_den_FBSDE} to make it Lipschitz. Later, we will see how it can be removed.\\
\indent (2) If one additionally imposes that $a$, $b$ and $\phi$ are bounded away from $0$, the well-posedness of \eqref{Lip_den_FBSDE} can be verified by the method of continuation (see \cite{peng1999fully}).
\end{remark}

\kong

We first introduce an auxiliary result on a linear degenerate FBSDE, the proof of which is based on the same technique used for \eqref{non_lin_pde}.

\kong

\begin{lemma}
\label{growth_lip_fbsde}
Let $(\mu_t)_{0\leq t\leq T}\in \mathbb{H}^2$ be a non-negative bounded process with the upper bound $\bar{\mu}>0$, and $\nu_T\in L^2(\Omega,\mathcal{F}_T)$ be a non-negative bounded random variable bounded by $\bar{\nu}>0$. The FBSDE
\begin{equation}
\left\{
\begin{aligned}
\;& dQ_t  = \mu_t\,Y_t\,dt, \\
& dY_t=2\phi_t\,Q_t\,dt+Z_t\,dW_t,\\
& Q_0=q_0,\quad Y_T=-\nu_T\,Q_T
\label{Lin_noncont_FBSDE}
\end{aligned}
\right.
\end{equation}
admits a unique solution in $\mathbb{S}^2\times\mathbb{S}^2\times\mathbb{H}^2$. In particular, the solution accepts the representation
\begin{equation}
    Y_t=P_t\cdot Q_t \text{\quad and \quad} Q_t=q_0\,\exp\big(\int_0^t\mu_s\,P_s\,ds\big),
\nonumber
\end{equation}
with $(P_t)_{0\leq t\leq T}\in\mathbb{H}^2$ being non-positive and bounded by $\bar{\nu}+2\bar{\phi}\,T$.
\end{lemma}

\begin{proof}
Motivated by the linearity of \eqref{Lin_noncont_FBSDE}, we adopt the affine ansatz $Y_t=P_t\cdot Q_t$ for some $P:=(P_t)_{0\leq t\leq T}$ to be determined. Through applying It\^o's formula and matching the coefficients, one can see that $P$ satisfies the BSDE
\begin{equation}
    dP_t=\big(-\mu_t\,(P_t)^2+2\phi_t\big)\,dt+\tilde{Z}_t\,dW_t, \quad \text{such that}\;\; P_T=-\nu_T.
    \label{nonlip_bsde}
\end{equation}
To study this non-Lipschitz BSDE, we similarly look at two auxiliary linear BSDEs:
\begin{gather*}
    dP_t=\big(-\mu_t\,(G_t\wedge(-\tilde{\xi})\vee \tilde{\xi})^2+2\phi_t\big)\,dt+Z_t^1\,dW_t, \quad \text{such that}\;\; P_T=-\nu_T;\\
    dP_t=\big(-\mu_t\,G_t P_t+2\phi_t\big)\,dt+Z_t^2\,dW_t, \quad \text{such that}\;\; P_T=-\nu_T,
\end{gather*}
and their associated mappings 
\begin{gather*}
    (F_3\,G)_t=\mathbb{E}_t\Big[-\nu_T+\int_t^T \mu_s\,(G_s\wedge(-\tilde{\xi})\vee \tilde{\xi})^2\,ds-\int_t^T 2\phi_s\,ds\Big],\\
    (F_4\,G)_t=\mathbb{E}_t\Big[-\nu_T\,e^{\int_t^T \mu_u\,G_u\,du}-\int_t^T2\phi_s\,e^{\int_t^s\mu_u\,G_u\,du}\,ds\Big],
\end{gather*}
for some constant $\tilde{\xi}>0$ and process $G\in\mathbb{S}^2$ to be specified. As before, $F_3$ is a contraction mappings on the space $\mathbb{S}^2$ with respect to the norm
\begin{equation*}
    \|G\|_{\zeta_3}:=\mathbb{E}\big[\sup_{0\leq t\leq T}e^{-2\zeta_3\,(T-t)}\cdot |G_t|^2\,\big]^{1/2},
\end{equation*}
and $F_4$ is a contraction mappings on the space of non-positive processes in $\mathbb{S}^2$ with respect to the norm
\begin{equation*}
    \|G\|_{\zeta_4}:=\mathbb{E}\big[\sup_{0\leq t\leq T}e^{-2\zeta_4\,(T-t)}\cdot |G_t|^2\,\big]^{1/2},
\end{equation*}
for some $\zeta_3, \zeta_4>0$ large enough. We learn from the Banach fixed point theorem that both $F_3$ and $F_4$ have their own unique fixed points. Given $\tilde{\xi}>\bar{\nu}+2\bar{\phi}\,T$, they further share the same fixed point. Finally, the common fixed point solves the equation \eqref{nonlip_bsde} and guarantees the existence of the solution. To see that \eqref{nonlip_bsde} accepts a unique bounded solution, let $P^1$, $P^2\in\mathbb{S}^2$ be two solutions with bounds $\bar{P}^1$, $\bar{P}^2>0$ respectively. If we let $\Tilde{\xi}\geq\bar{P}^1\vee\bar{P}^2$, both $P^1$ and $P^2$ are then fixed points of the mapping $F_3$, implying $P^1=P^2$ due the uniqueness of the fixed point.

The uniqueness of \eqref{Lin_noncont_FBSDE} follows from the method of continuation as in \cite{peng1999fully}. Given two solutions $(Q, Y, Z )$ and $(\tilde{Q}, \tilde{Y}, \tilde{Z})$ in $\mathbb{S}^2\times\mathbb{S}^2\times\mathbb{H}^2$, then $(\mathcal{Q}, \mathcal{Y}, \mathcal{Z}):=(\tilde{Q}-Q, \tilde{Y}-Y, \tilde{Z}-Z)$ solves the FBSDE
\begin{equation}
\left\{
\begin{aligned}
\;& d\mathcal{Q}_t  = \mu_t\,\mathcal{Y}_t dt, \\
& d\mathcal{Y}_t=2\phi_t\,\mathcal{Q}_t\,dt+\mathcal{Z}_t\,dW_t,\\
& \mathcal{Q}_0=0,\quad \mathcal{Y}_T=-\nu_T\,\mathcal{Q}_T.
\nonumber
\end{aligned}
\right.
\end{equation}
While It\^o's formula implies
\begin{equation*}  d(\mathcal{Q}_t\,\mathcal{Y}_t)=\big[\mu_t(\mathcal{Y}_t)^2+2\phi_t(\mathcal{Q}_t)^2\big]\,dt+\mathcal{Q}_t\,\mathcal{Z}_t\,dW_t,
\end{equation*}
through observing
\begin{equation*}
        0\geq\mathbb{E}[-\nu_T\,(\mathcal{Q}_T)^2]=\mathbb{E}\int_0^T\big[\mu_t\,(\mathcal{Y}_t)^2+2\phi_t\,(\mathcal{Q}_t)^2\big]\,dt\geq \mathbb{E}\int_0^T\mu_t\,(\mathcal{Y}_t)^2\,dt\geq 0
\end{equation*}
we can see $\mu_t\,\mathcal{Y}_t=0$ a.e. in $t$ and thus $\mathcal{Q}=0$. The fact $\mathcal{Y}=\mathcal{Z}=0$ then follows immediately. 
\end{proof}

\kong

\begin{remark}
The arguments of the continuation method in proving the uniqueness of the solution will be applied to some other equations later. We emphasize that such arguments do not need the boundedness of $\mu$, $\phi$ and $\nu_T$.
\end{remark}

\kong

\noindent Since the FBSDE \eqref{Lip_den_FBSDE} is Lipschitz, denote by $\iota>0$ the Lipschitz constant for the coefficient functions of forward and backward equations of $\eqref{Lip_den_FBSDE}$ (excluding the terminal condition), i.e., it holds $\mathbb{P}$-a.s. for any $y, \Tilde{y}, q, \Tilde{q}$ that
\begin{equation*}
\begin{aligned}
    \Big|-a_t\,\Lambda^a\big(\tilde{\delta}^{a*}(y)\big)+b_t\,\Lambda^b\big(\tilde{\delta}^{b*}(-y)\big)+a_t\,\Lambda^a\big(\tilde{\delta}^{a*}(\tilde{y})\big)-b_t\,\Lambda^b\big(\tilde{\delta}^{b*}(-\tilde{y})\big)\Big|&\leq \iota\,|\tilde{y}-y|,\\
    \big|2\phi_t\,q-2\phi_t\,\tilde{q}\big|&\leq  \iota\,|\tilde{q}-q|,\\
    \big|2A\,q-2A\,\tilde{q}\big|&\leq  \iota\,|\tilde{q}-q|.
\end{aligned}
\end{equation*}
To prove the well-posedness, we apply the decoupling approach introduced by \cite{ma2015well}. Actually, Lemma \ref{growth_lip_fbsde} studies the growth of the Lipschitz coefficient---a critical step for such technique.

\kong

\begin{theorem}
\label{control_mono}
The FBSDE \eqref{Lip_den_FBSDE} accepts a unique solution in $\mathbb{S}^2\times\mathbb{S}^2\times\mathbb{H}^2$. Further, the optimal control is monotonic with respect to the initial (inventory) condition, i.e., if $q_1>q_2$, then it holds that
\begin{equation}
  \hat{\delta}_{1,t}^{a}\leq\hat{\delta}_{2,t}^{a}         \quad\text{and}\quad \hat{\delta}_{1,t}^{b}\geq\hat{\delta}_{2,t}^{b}
    \nonumber
\end{equation}
for  $t\in[0,T]$ $\mathbb{P}$-a.s., where $(\hat{\delta}_{i,t}^{a}, \hat{\delta}_{i,t}^{b})_{0\leq t\leq T}$ represents the optimal control associated with the initial condition $q_i$ for $i\in\{1, 2\}$.
\end{theorem}

\begin{proof}
\uline{\textit{Short time analysis}}: Since the FBSDE is of Lipschitz type, we start with the short time analysis. Fix any $q_0\in\mathbb{R}$, if $Q=(Q_t)_{0\leq t\leq T}\in\mathbb{S}^2$ is given such that $Q_0=q_0$, denote by $(Y, Z)\in\mathbb{S}^2\times\mathbb{H}^2$ the unique solution of the BSDE:
\begin{equation}
    dY_t=2\phi_t\,Q_t\,dt+Z_t\,dW_t, \text{\; such that \;} Y_T=-2A\,Q_T.
    \nonumber
\end{equation}
Subsequently, given $(Y, Z)$, we let $Q'_t$ be the uniqueness solution of the SDE
\begin{equation}
    dQ'_t=-a_t\,\Lambda^a\big(\tilde{\delta}^{a*}(Y_t)\big)dt+b_t\,\Lambda^b\big(\tilde{\delta}^{b*}(-Y_t)\big)dt,\text{\; such that \;} Q'_0=q_0.
    \nonumber
\end{equation}
In such way, we have defined a mapping
\begin{equation}
    \mathbb{S}^2\ni Q\hookrightarrow \Phi(Q)=Q'\in\mathbb{S}^2
    \nonumber
\end{equation}
and proceed to show that $\Phi$ is a contraction mapping when $T$ is small enough. For any $Q$, $\tilde{Q}\in\mathbb{S}^2$, write $Q'=\Phi(Q)$, $\tilde{Q}'=\Phi(\tilde{Q})$ and note that
\begin{equation}
\begin{aligned}
    |\tilde{Q}'_t-Q'_t|&\leq \iota \, T \sup_{0\leq s \leq T}|\tilde{Y}_s-Y_s|\\ \mathbb{E}\sup_{0\leq s \leq T}|\tilde{Q}'_s-Q'_s|^2&\leq (\iota \, T)^2\,\mathbb{E}\sup_{0\leq s \leq T}|\tilde{Y}_s-Y_s|^2.
\label{control_stab_1}
\end{aligned}
\end{equation}
To obtain the stability for the $\tilde{Y}$ and $Y$, we further compute
\begin{equation}
    \begin{aligned}
    |\tilde{Y}_t-Y_t|^2&\leq (2\bar{A}+2\iota T)^2\,\mathbb{E}_t\big[\sup_{0\leq s\leq T}|\tilde{Q}_s-Q_s|\big]^2,\\
    \mathbb{E}\sup_{0\leq s\leq T}|\tilde{Y}_s-Y_s|^2&\leq (2\bar{A}+2\iota T)^2\,\mathbb{E}\Big[\sup_{0\leq t\leq T}\mathbb{E}_t\big[\sup_{0\leq s\leq T}|\tilde{Q}_s-Q_s|\big]^2\Big],\\
    &\leq 4\,(2\bar{A}+2\iota T)^2\,\mathbb{E}\sup_{0\leq s\leq T}|\tilde{Q}_s-Q_s|^2,
    \end{aligned}
    \label{control_stab_2}
\end{equation}
where the Doob's $L^p$ inequality is applied in the last line.  Combining \eqref{control_stab_1} and \eqref{control_stab_2}, we find that
\begin{equation}
    \mathbb{E}\sup_{0\leq s \leq T}|\tilde{Q}'_s-Q'_s|^2\leq 4\,(\iota T)^2\,(2\bar{A}+2\iota T)^2\,\mathbb{E}\sup_{0\leq s\leq T}|\tilde{Q}_s-Q_s|^2
    \nonumber   
\end{equation}
and it suffices to pick $T$ satisfying
\begin{equation}
    4\,(\iota T)^2\,(2\bar{A}+2\iota T)^2<1
    \nonumber
\end{equation}
for a contraction mapping. The FBSDE \eqref{Lip_den_FBSDE} accepts a unique solution in $\mathbb{S}^2\times\mathbb{S}^2\times\mathbb{H}^2$ with respect to the such $T$.

\uline{\textit{Decoupling approach}}: we then extend the local result to any finite time horizon $T$. Define time step $\Delta>0$ as
\begin{equation}
    \Delta=\frac{1}{\sqrt{2}}\cdot \big[4\iota^2\,(2\bar{A}+2\bar{\phi}T+2\iota T)^2\big]^{-1/2}.
    \label{control_time_step}
\end{equation}
Note that 
\begin{equation*}
    4\,(\iota \Delta)^2\,(2\bar{A}+2\iota \Delta)^2\leq4\,(\iota \Delta)^2\,(2\bar{A}+2\bar{\phi}T+2\iota T)^2=\frac{1}{2}
\end{equation*}
and thus the following FBSDE is well-posed by the previous discussion:
\begin{equation}
\left\{
\begin{aligned}
\;& dQ_t  = -a_t\,\Lambda^a\big(\tilde{\delta}^{a*}(Y_t)\big)dt+b_t\,\Lambda^b\big(\tilde{\delta}^{b*}(-Y_t)\big)dt, \\
& dY_t=2\phi_t\,Q_t\,dt+Z_t\,dW_t,\\
& Q_{T-\Delta}=q_{T-\Delta},\quad Y_T=-2A\,Q_T.
\end{aligned}
\right.
\label{control_fbsde_1}
\end{equation}
We can then define the decoupling field $u: [T-\Delta,T]\times\Omega\times\mathbb{R}\to\mathbb{R}$ by
\begin{equation}
    u(t,q):=Y_t^{t,q},
    \label{control_decoupling}
\end{equation}
where the superscript $(t,q)$ denotes the initial time and condition of \eqref{control_fbsde_1}.  Let $q_1$ and $q_2$ be two initial conditions for the FBSDE \eqref{control_fbsde_1}, and---due to the well-posedness---denote by $(Q^1,Y^1,Z^1)$ and $(Q^2,Y^2,Z^2)$ the solutions corresponding to these initial conditions. For $t$ such that $Y_t^1\neq Y_t^2$, let
\begin{equation}
\begin{aligned}
    d(Q_t^1-Q_t^2)/dt&=-a_t\big[\Lambda^a\big(\tilde{\delta}^{a*}(Y_t^1)\big)-\Lambda^a\big(\tilde{\delta}^{a*}(Y_t^2)\big)\big]+b_t\,\big[\Lambda^b\big(\tilde{\delta}^{b*}(-Y_t^1)\big)-\Lambda^b\big(\tilde{\delta}^{b*}(-Y_t^2)\big)\big]\\
    &=:\mathcal{U}_t\cdot(Y_t^1-Y_t^2).
\end{aligned}
    \nonumber
\end{equation}
Naturally, for those $t$ such that $Y_t^1= Y_t^2$, we let $\mathcal{U}_t=0$. Note that $\mathcal{U}\in \mathbb{H}^2$ is bounded by $\iota$ due to the Lipschitz continuity. It is further non-negative since the expression
\begin{equation}
    -a\,\Lambda^a\big(\tilde{\delta}^{a*}(y)\big)+b\,\Lambda^b\big(\tilde{\delta}^{b*}(-y)\big)
    \nonumber   
\end{equation}
is non-decreasing with respect to $y$, which can be deduced from the increasing property of $(\delta^{a*},\delta^{b*})$ and decreasing property of $(\Lambda^a,\Lambda^b)$. One can then observe that $(Q^1-Q^2, Y^1-Y^2, Z^1-Z^2)$ satisfies the following FBSDE
\begin{equation}
    \left\{
\begin{aligned}
\;& d\mathcal{Q}_t  = \mathcal{U}_t\,\mathcal{Y}_tdt, \\
& d\mathcal{Y}_t=2\phi_t\,\mathcal{Q}_t\,dt+\mathcal{Z}_t\,dW_t,\\
& \mathcal{Q}_{T-\Delta}=q_1-q_2,\quad \mathcal{Y}_T=-2A\,\mathcal{Q}_T,
\end{aligned}
\right.
    \label{var_FBSDE}
\end{equation}
for $t\in[T-\Delta,T]$, which is called the variational FBSDE associated with the original \eqref{Lip_den_FBSDE}. While \eqref{var_FBSDE} is a special case of the result in Lemma \ref{growth_lip_fbsde}, equation \eqref{var_FBSDE} admits a unique solution 
\begin{equation}
    (\mathcal{Q}_t,\mathcal{Y}_t,\mathcal{Z}_t)=\big(Q_t^1-Q_t^2, Y_t^1-Y_t^2, Z_t^1-Z_t^2\big)
    \nonumber
\end{equation}
and, more importantly, process $\mathcal{Y}$ is bounded by $|q_1-q_2|\cdot(2\bar{A}+2\bar{\phi}\,\Delta)$. According to the definition of the decoupling field, we deduce that $u$ is uniformly Lipschitz continuous $\mathbb{P}$-a.s. with respect to variable $q$  with Lipschitz coefficient $2\bar{A}+2\bar{\phi}\,\Delta$. Consider the FBSDE in the next interval
\begin{equation}
\left\{
\begin{aligned}
\;& dQ_t  = -a_t\,\Lambda^a\big(\tilde{\delta}^{a*}(Y_t)\big)dt+b_t\,\Lambda^b\big(\tilde{\delta}^{b*}(-Y_t)\big)dt, \\
& dY_t=2\phi_t\,Q_t\,dt+Z_t\,dW_t,\\
& Q_{T-2\Delta}=q_{T-2\Delta},\quad Y_{T-\Delta}=u(T-\Delta,Q_{T-\Delta}).
\end{aligned}
\right.
\end{equation}
With the same short time analysis, it suffices to make sure
\begin{equation*}
    4\,(\iota \Delta)^2\,(2\bar{A}+2\bar{\phi}\Delta+2\iota \Delta)^2<1,
\end{equation*}
and our choice \eqref{control_time_step} is clearly qualified. Referring to Lemma \ref{growth_lip_fbsde}, we study the variational FBSDE and observe the linear growth of the Lipschitz coefficient of the decoupling field. Because the potential Lipschitz coefficient of the decoupling field can not exceed $2\bar{A}+2\bar{\phi}T$, the time step $\Delta$ will always qualify, and hence the field $u$ can be extended to the whole time horizon $[0,T]$ by a finite number of iterations. The global existence of the solution then can be inferred from the (global) decoupling field $u$. Because the uniqueness of solution can be proved in the same fashion as Lemma \ref{growth_lip_fbsde}, the well-posedness of \eqref{Lip_den_FBSDE} is established.

\uline{\textit{Monotonicity}}: to verify the monotonicity property, we let $q_1>q_2$ be two initial conditions, denoting by $(Q^1,Y^1,Z^1)$ and $(Q^2,Y^2,Z^2)$ the associated solutions. Since $(Q^1-Q^2, Y^1-Y^2, Z^1-Z^2)$ solves the variational FBSDE \eqref{var_FBSDE} with initial condition $q_1-q_2>0$, again by Lemma \ref{growth_lip_fbsde} we see $Y^1-Y^2$ is negative. Recalling that the optimal controls read
\begin{equation}
    \hat{\delta}_t^a=\tilde{\delta}^{a*}(Y_t) \quad\text{and}\quad \hat{\delta}_t^b=\tilde{\delta}^{b*}(-Y_t),
    \nonumber
\end{equation}
the smaller value of $Y_t$ results in a smaller $\hat{\delta}_t^a$ but a larger $\hat{\delta}_t^b$. 
\end{proof}

\kong

We finish this section by removing the constant $\xi$. It turns out that, provided $\xi$ is picked large enough, the corresponding constraint will have no impact on the control. That is to say, we are able to find a (unique) solution to the equation
\begin{equation}
\left\{
\begin{aligned}
\;& dQ_t  = -a_t\,\Lambda^a\big(\delta^{a*}(Y_t)\big)dt+b_t\,\Lambda^b\big(\delta^{b*}(-Y_t)\big)dt, \\
& dY_t=2\phi_t\,Q_t\,dt+Z_t\,dW_t,\\
& Q_0=q_0,\quad Y_T=-2A\,Q_T.
\end{aligned}
\right.
\label{non_lip_fbsde}
\end{equation}
Note that this FBSDE is non-Lipschitz because the derivatives of $(\Lambda^a, \Lambda^b)$ are not necessarily bounded.

\kong

\begin{proposition}
The FBSDE \eqref{non_lip_fbsde} accepts a unique solution in $\mathbb{S}^2\times\mathbb{S}^2\times \mathbb{H}^2$. Further, the optimal control is monotonic with respect to the initial (inventory) condition, i.e., if $q_1>q_2$, then it holds that
\begin{equation}
  \hat{\delta}_{1,t}^{a}<\hat{\delta}_{2,t}^{a}         \quad\text{and}\quad \hat{\delta}_{1,t}^{b}>\hat{\delta}_{2,t}^{b}
    \nonumber
\end{equation}
for  $t\in[0,T]$ $\mathbb{P}$-a.s., where $(\hat{\delta}_{i,t}^{a}, \hat{\delta}_{i,t}^{b})_{0\leq t\leq T}$ represents the optimal control associated with the initial condition $q_i$ for $i\in\{1, 2\}$.
\label{proof_control_non_lip_fbsde}
\end{proposition}

\begin{proof}
We intend to construct a solution to \eqref{non_lip_fbsde} via the `truncated' version:
\begin{equation}
\left\{
\begin{aligned}
\;& dQ_t  = -a_t\,\Lambda^a\big(\tilde{\delta}^{a*}(Y_t)\big)dt+b_t\,\Lambda^b\big(\tilde{\delta}^{b*}(-Y_t)\big)dt, \\
& dY_t=2\phi_t\,Q_t\,dt+Z_t\,dW_t,\\
& Q_0=q_0,\quad Y_T=-2A\,Q_T.
\label{nonLip_den_FBSDE}
\end{aligned}
\right.
\end{equation}
The idea comes from the solution of the linear case. Recall that the value function $H$ is composed of three terms with different order (see Theorem \ref{verification}): the one of second order is induced from the terminal inventory penalty and is uniformly bounded by the penalty parameters; the one of first order is the estimation of the integral of future order imbalance. We are then motivated to look at two extreme unbalanced situations: the case when the buy market order dominates and the case when the dominance is achieved by the sell market order. Thus, we look at two associate FBSDEs:
\begin{equation}
\left\{
\begin{aligned}
\;& dQ_t  = b_t\,\Lambda^b\big(\tilde{\delta}^{b*}(-Y_t)\big)dt, \\
& dY_t=2\phi_t\,Q_t\,dt+Z_t\,dW_t,\\
& Q_0=q_0,\quad Y_T=-2A\,Q_T;
\label{nonLip_dom_FBSDE_1}
\end{aligned}
\right.
\end{equation}
\begin{equation}
\left\{
\begin{aligned}
\;& dQ_t  = -a_t\,\Lambda^a\big(\tilde{\delta}^{a*}(Y_t)\big)dt, \\
& dY_t=2\phi_t\,Q_t\,dt+Z_t\,dW_t,\\
& Q_0=q_0,\quad Y_T=-2A\,Q_T.
\label{nonLip_dom_FBSDE_2}
\end{aligned}
\right.
\end{equation}
We hope the solutions of these two equations would reveal certain properties of \eqref{nonLip_den_FBSDE}, so that the parameter $\xi$ can be removed safely. Denote by $(\tilde{Q},\tilde{Y},\tilde{Z})$ and $(\hat{Q},\hat{Y},\hat{Z})$ the (unique) solutions of \eqref{nonLip_dom_FBSDE_1} and \eqref{nonLip_dom_FBSDE_2} accordingly. By definition, process $\tilde{Y}$ has the representation
\begin{equation*}
    -\tilde{Y}_t=\mathbb{E}_t\Big[2A\,\tilde{Q}_T+ 2\int_t^T\phi_s\,\tilde{Q}_s\,ds\Big]\geq\mathbb{E}_t\Big[2A\, \tilde{Q}_0+ 2\int_t^T\phi_s\,\tilde{Q}_0\,ds\Big]\geq(-2)\,|q_0|\,(\bar{A}+\bar{\phi}\,T)
\end{equation*}
for all $t\in[0,T]$, where we have used the fact that $\Tilde{Q}$ is non-decreasing. Consequently, a uniform upper bound for $\tilde{Q}$ can be derived by
\begin{equation}
\begin{aligned}
    q_0\leq\tilde{Q}_t&=q_0+\int_0^t b_s\,\Lambda^b\big(\tilde{\delta}^{b*}(-\tilde{Y}_s)\big)\,ds\\
    &\leq q_0+\bar{b}\int_0^T \Lambda^b\Big(\delta^{b*}\big((-2)\,|q_0|\,(\bar{A}+\bar{\phi}\,T)\big)\Big)\,ds\\
    &=q_0+\bar{b}\,T\, \Lambda^b\Big(\delta^{b*}\big((-2)\,|q_0|\,(\bar{A}+\bar{\phi}\,T)\big)\Big).
\end{aligned}
\label{q^tilde_bound}
\nonumber
\end{equation}
In light of this, the lower bound for $\tilde{Y}_t$ can also be found by
\begin{equation}
    -\tilde{Y}_t=\mathbb{E}_t\Big[2A\,\tilde{Q}_T+ 2\int_t^T\phi_s\,\tilde{Q}_s\,ds\Big]\leq 2\,(\bar{A}+\bar{\phi}\,T)\cdot\bigg[|q_0|+\bar{b}\,T\, \Lambda^b\Big(\delta^{b*}\big((-2)\,|q_0|\,(A+\bar{\phi}\,T)\big)\Big)\bigg].
\label{y^tilde_bound}
\end{equation}
We remark that these bounds for $(\tilde{Q},\tilde{Y})$ are independent of the parameter $\xi$, implying that we can already find a solution for this `one-sided' non-Lipschitz FBSDE when $\xi$ is large enough. In a similar fashion, the corresponding bounds for $(\hat{Q},\hat{Y})$ read
\begin{gather}
    \hat{Y}_t=\mathbb{E}_t\Big[-2A\,\hat{Q}_T- 2\int_t^T\phi_s\,\hat{Q}_s\,ds\Big]\geq(-2)\,|q_0|\,(\bar{A}+\bar{\phi}\,T),\nonumber\\
    q_0\geq\hat{Q}_t=q_0-\int_0^t a_s\,\Lambda^a\big(\tilde{\delta}^{a*}(\hat{Y}_s)\big)\,ds\geq q_0-\bar{a}\,T\,\Lambda^a\Big(\delta^{a*}\big((-2)\,|q_0|\,(\bar{A}+\bar{\phi}\,T)\big)\Big),
\label{q^hat_bound}\\
    \hat{Y}_t\leq 2\,(\bar{A}+\bar{\phi}\, T)\cdot\bigg[|q_0|+\bar{a}\,T\,\Lambda^a\Big(\delta^{a*}\big((-2)\,|q_0|\,(\bar{A}+\bar{\phi}\,T)\big)\Big)\bigg],
\end{gather}
holding for all $t\in[0,T]$. Again, such bounds are independent of the parameter $\xi$.

Set $(Q,Y,M)$ as the solution to the original FBSDE \eqref{nonLip_den_FBSDE}, and we then compute the difference of two equations \eqref{nonLip_den_FBSDE} and \eqref{nonLip_dom_FBSDE_1}. Note that
\begin{equation}
\begin{aligned}
    d(\tilde{Q}_t-Q_t)&=b_t\,\Big[\Lambda^b\big(\tilde{\delta}^{b*}(-\tilde{Y}_t)\big)-\Lambda^b\big(\tilde{\delta}^{b*}(-Y_t)\big)\Big]\,dt+a_t\,\Lambda^a\big(\tilde{\delta}^{a*}(Y_t)\big)\,dt\\
    &=b_t\,\Big[\Lambda^b\big(\tilde{\delta}^{b*}(-\tilde{Y}_t)\big)-\Lambda^b\big(\tilde{\delta}^{b*}(-Y_t)\big)\Big]\,dt\\
    &\hspace{2cm}+a_t\,\Big[\Lambda^a\big(\tilde{\delta}^{a*}(Y_t)\big)-\Lambda^a\big(\tilde{\delta}^{a*}(\tilde{Y}_t)\big)\Big]\,dt+a_t\,\Lambda^a\big(\tilde{\delta}^{a*}(\tilde{Y}_t)\big)\,dt\\
    &=\pi_t\,(\tilde{Y}_t-Y_t)\,dt+a_t\,\Lambda^a\big(\tilde{\delta}^{a*}(\tilde{Y}_t)\big)\,dt,
\end{aligned}
\nonumber
\end{equation}
where we let
\begin{equation}
    \pi_t=\frac{b_t\,\Big[\Lambda^b\big(\tilde{\delta}^{b*}(-\tilde{Y}_t)\big)-\Lambda^b\big(\tilde{\delta}^{b*}(-Y_t)\big)\Big]+a_t\,\Big[\Lambda^a\big(\tilde{\delta}^{a*}(Y_t)\big)-\Lambda^a\big(\tilde{\delta}^{a*}(\tilde{Y}_t)\big)\Big]}{\tilde{Y}_t-Y_t}
    \nonumber
\end{equation}
when $\tilde{Y}_t-Y_t\neq0$, and $\pi_t=0$ in the case of $\tilde{Y}_t-Y_t=0$. Due to the monotonicity properties of $(\Lambda^a,\Lambda^b)$ and $(\delta^{a*},\delta^{b*})$, we know that $\pi\in\mathbb{H}^2$ is a bounded non-negative process. Afterwards, the triplet $(\tilde{Q}-Q,\tilde{Y}-Y,\tilde{Z}-Z)$ solves the following linear FBSDE:
\begin{equation}
\left\{
\begin{aligned}
\;& d\mathcal{Q}_t  = \pi_t\,\mathcal{Y}_t dt+a_t\,\Lambda^a\big(\tilde{\delta}^{a*}(\tilde{Y}_t)\big)\,dt, \\
& d\mathcal{Y}_t=2\phi_t\,\mathcal{Q}_t\,dt+\mathcal{Z}_t\,dW_t,\\
& \mathcal{Q}_0=0,\quad \mathcal{Y}_T=-2A\,\mathcal{Q}_T,
\nonumber
\end{aligned}
\right.
\end{equation}
which accepts a unique solution as a simple variation of Lemma \ref{growth_lip_fbsde}. To solve this equation, its linear structure suggests the affine ansatz $\mathcal{Y}_t=\mathscr{A}_t\,\mathcal{Q}_t+\mathscr{B}_t$, where processes $\mathscr{A}$ and $\mathscr{B}$ satisfy two coupled BSDEs:
\begin{gather*}
    d\mathscr{A}_t=\big(-\pi_t\,(\mathscr{A}_t)^2+2\phi_t\big)\,dt+\mathscr{Z}^{\mathscr{A}}_t\,dW_t, \quad\text{with}\quad \mathscr{A}_T=-2A;\\
    d\mathscr{B}_t=-\Big[\pi_t\,\mathscr{A}_t\,\mathscr{B}_t+\mathscr{A}_t\,a_t\,\Lambda^a\big(\tilde{\delta}^{a*}(\tilde{Y}_t)\big)\Big]\,dt+\mathscr{Z}^{\mathscr{B}}_t\,dW_t, \quad\text{with}\quad \mathscr{B}_T=0.
\nonumber
\end{gather*}
We already know the first BSDE accepts a solution $\mathscr{A}$ that is negative and bounded by $2(\bar{A}+\Bar{\phi}\,T)$. In turn, the solution of the second linear BSDE is given explicitly by
\begin{equation} \mathscr{B}_t=\mathbb{E}_t\Big[\int_t^T\mathscr{A}_s\,a_s\,\Lambda^a\big(\tilde{\delta}^{a*}(\tilde{Y}_s)\big)\,e^{\int_t^s\pi_u\,\mathscr{A}_u\,du}\,ds\Big]\leq0.
\nonumber
\end{equation}
Note that the expression inside the conditional expectation is uniformly bounded on $[0,T]$ by some constant independent of $\xi$. Since $\mathcal{Q}_t$ satisfies the simple ODE
\begin{equation*}
    d\mathcal{Q}_t=\pi_t\,(\mathscr{A}_t\,\mathcal{Q}_t+\mathscr{B}_t)\,dt+a_t\,\Lambda^a\big(\tilde{\delta}^{a*}(\tilde{Y}_t)\big)\,dt,
\end{equation*}
the solution can be also given precisely by
\begin{equation*}
    \mathcal{Q}_t=\int_0^t \Big[\pi_s\,\mathscr{B}_s+a_s\,\Lambda^a\big(\tilde{\delta}^{a*}(\tilde{Y}_s)\big)\Big]\,e^{\int_s^t\pi_u\,\mathscr{A}_u\,du}\,ds.
\end{equation*}
By definition, we further deduce
\begin{equation*}
\begin{aligned}
    \tilde{Q}_t-Q_t&=\int_0^t \Big[\pi_s\,\mathscr{B}_s+a_s\,\Lambda^a\big(\tilde{\delta}^{a*}(\tilde{Y}_s)\big)\Big]\,e^{\int_s^t\pi_u\,\mathscr{A}_u\,du}\,ds\\
    &\leq\int_0^t a_s\,\Lambda^a\big(\tilde{\delta}^{a*}(\tilde{Y}_s)\big)\,ds,\\
    Q_t&\geq\tilde{Q}_t-\int_0^t a_s\,\Lambda^a\big(\tilde{\delta}^{a*}(\tilde{Y}_s)\big)\,ds.
\end{aligned}
\end{equation*}
The uniform boundedness of $(\tilde{Q},\tilde{Y})$ implies that $Q$ is bounded below by a constant that is independent of $\xi$. On the other hand, to search for an upper bound for $Q$, we study the relation between $(Q,Y,Z)$ and  $(\hat{Q},\hat{Y},\hat{Z})$. Taking the difference of \eqref{nonLip_den_FBSDE} and \eqref{nonLip_dom_FBSDE_2}, we apply a similar transform to see
\begin{equation}
\begin{aligned}
    d(\hat{Q}_t-Q_t)&=-b_t\,\Lambda^b\big(\tilde{\delta}^{b*}(-Y_t)\big)\,dt-a_t\,\Big[\Lambda^a\big(\tilde{\delta}^{a*}(\hat{Y}_t)\big)-\Lambda^a\big(\tilde{\delta}^{a*}(Y_t)\big)\Big]\,dt\\
    &=b_t\,\Big[\Lambda^b\big(\tilde{\delta}^{b*}(-\hat{Y}_t)\big)-\Lambda^b\big(\tilde{\delta}^{b*}(-Y_t)\big)\Big]\,dt\\
    &\hspace{2cm}+a_t\,\Big[\Lambda^a\big(\tilde{\delta}^{a*}(Y_t)\big)-\Lambda^a\big(\tilde{\delta}^{a*}(\hat{Y}_t)\big)\Big]\,dt-b_t\,\Lambda^b\big(\tilde{\delta}^{b*}(\hat{Y}_t)\big)\,dt\\
    &=\kappa_t\,(\hat{Y}_t-Y_t)\,dt-b_t\,\Lambda^b\big(\tilde{\delta}^{b*}(\hat{Y}_t)\big)\,dt,
\end{aligned}
\nonumber
\end{equation}
where we let
\begin{equation*}
    \kappa_t=\frac{b_t\,\Big[\Lambda^b\big(\tilde{\delta}^{b*}(-\hat{Y}_t)\big)-\Lambda^b\big(\tilde{\delta}^{b*}(-Y_t)\big)\Big]+a_t\,\Big[\Lambda^a\big(\tilde{\delta}^{a*}(Y_t)\big)-\Lambda^a\big(\tilde{\delta}^{a*}(\hat{Y}_t)\big)\Big]}{\hat{Y}_t-Y_t}
\end{equation*}
when $\hat{Y}_t-Y_t\neq0$, and $\kappa_t=0$ in the case of $\hat{Y}_t-Y_t=0$. One can see that $\kappa\in\mathbb{H}^2$ is non-negative and bounded as well. Similarly, the triplet $(\hat{Q}_t-Q_t,\hat{Y}_t-Y_t,\hat{Z}_t-Z_t)$ solves the following (well-posed) linear FBSDE:
\begin{equation}
\left\{
\begin{aligned}
\;& d\mathcal{Q}_t  = \kappa_t\,\mathcal{Y}_t dt-b_t\,\Lambda^b\big(\tilde{\delta}^{b*}(\hat{Y}_t)\big)\,dt, \\
& d\mathcal{Y}_t=2\phi_t\,\mathcal{Q}_t\,dt+\mathcal{Z}_t\,dW_t,\\
& \mathcal{Q}_0=0,\quad \mathcal{Y}_T=-2A\,\mathcal{Q}_T.
\nonumber
\end{aligned}
\right.
\end{equation}
The linear structure again advocates the affine ansatz $\mathcal{Y}_t=\mathscr{P}_t\,\mathcal{Q}_t+\mathscr{H}_t$, with processes $\mathscr{P}$ and $\mathscr{H}$ solving two coupled BSDEs
\begin{gather*}
    d\mathscr{P}_t=(-\kappa_t\,\mathscr{P}_t^2+2\phi_t)\,dt+\mathscr{Z}^{\mathscr{P}}_t\,dW_t, \quad\text{with}\quad \mathscr{P}_T=-2A;\\
    d\mathscr{H}_t=-\Big[\kappa_t\,\mathscr{P}_t\,\mathscr{H}_t-\mathscr{P}_t\,b_t\,\Lambda^b\big(\tilde{\delta}^{b*}(\hat{Y}_t)\big)\Big]\,dt+\mathscr{Z}^{\mathscr{H}}_t\,dW_t, \quad\text{with}\quad \mathscr{H}_T=0.
\nonumber
\end{gather*}
We know the first BSDE accepts a solution $\mathscr{P}\in\mathbb{H}^2$ that is negative and bounded. In turn, the solution of the second linear BSDE is given explicitly by
\begin{equation}
    \mathscr{H}_t=-\,\mathbb{E}_t\Big[\int_t^T\mathscr{P}_s\,b_s\,\Lambda^b\big(\tilde{\delta}^{b*}(\hat{Y}_s)\big)\,e^{\int_t^s\kappa_u\,\mathscr{P}_u\,du}\,ds\Big]\geq 0.
\nonumber
\end{equation}
Note that $\mathscr{H}$ is also uniformly bounded by some constant being independent of $\xi$. As $\mathcal{Q}$ can be represented by
\begin{equation*}
    \mathcal{Q}_t=\int_0^t \Big[\kappa_s\,\mathscr{H}_s-b_s\,\Lambda^b\big(\tilde{\delta}^{b*}(\hat{Y}_s)\big)\Big]\,e^{\int_s^t\kappa_u\,\mathscr{P}_u\,du}\,ds,
\end{equation*}
by definition we observe
\begin{equation*}
\begin{aligned}
    \hat{Q}_t-Q_t&=\int_0^t \Big[\kappa_s\,\mathscr{H}_s-b_s\,\Lambda^b\big(\tilde{\delta}^{b*}(\hat{Y}_s)\big)\Big]\,e^{\int_s^t\kappa_u\,\mathscr{P}_u\,du}\,ds\\
    &\geq-\int_0^t b_s\,\Lambda^b\big(\tilde{\delta}^{b*}(\hat{Y}_s)\big)\,ds,\\
    Q_t&\leq\hat{Q}_t+\int_0^t b_s\,\Lambda^b\big(\tilde{\delta}^{b*}(\hat{Y}_s)\big)\,ds.
\end{aligned}
\end{equation*}
The uniform boundedness of $(\hat{Q},\hat{Y})$ suggests that $Q$ is bounded above by a constant that is independent of $\xi$. While now we know $Q$ is uniformly lower bounded, it leads to the uniform boundedness of $Y$ since
\begin{equation*}
    Y_t=\,\mathbb{E}_t\Big[-2A\,Q_T- 2\int_t^T\phi_s\, Q_s\,ds\Big].
\end{equation*}
In consequence, by picking $\xi$ large enough, the solution $(Q,Y,Z)$ to the Lipschitz FBSDE \eqref{nonLip_den_FBSDE} also solves the non-Lipschitz \eqref{non_lip_fbsde}. The uniqueness again follows from the continuation method in Lemma \ref{growth_lip_fbsde}.

With respect to the monotonicity, it suffices to apply Theorem \ref{control_mono} and the fact that the truncation has been removed. 
\end{proof}

\vspace{0.2cm}

\section{Non-Markovian Order Flows: Unbounded Coefficients} \label{section_4}
\noindent In the previous two cases, the coefficients---order flows $(a, b)$ and penalisation parameters $(\phi, A)$---are all bounded. Since we have mentioned the possibility of unbounded order flows in Remark \ref{unbound_order_flow}, this section is devoted to such situations. For convenience of notation, we will present the result in the case of linear intensity function, i.e., $\Lambda(\delta)=\zeta-\gamma\,\delta$ for some constants $\zeta, \gamma>0$. Regarding the general intensity function, it suffices to add a polynomial growth condition in Assumption \ref{general_inten}, while the exponential function will be excluded.

We start by introducing proper spaces for the order flows $(a, b)$ and penalisation parameters $(\phi, A)$.

\kong

\begin{assumption}
Let $A\in L^{8\mathscr{D}}(\mathcal{F}_T, \mathbb{R}_+)$, $a$, $b\in\mathbb{S}^{8\mathscr{D}}$, and $\phi\in\mathbb{H}^{8\mathscr{D}}$, where $\mathscr{D}\in\mathbb{N}$ will be specified later.
\end{assumption}

\kong

\noindent Since the intensity function is $\Lambda(\delta)=\zeta-\gamma\,\delta$, we recall the inventory and cash processes of the market maker as follows:
\begin{gather*}
        Q_t = q_0-\int_0^t a_u(\zeta-\gamma \delta^a_u)\,du+\int_0^t b_u(\zeta-\gamma \delta^b_u)\,du,\\
        \nonumber
    X_t = x_0+\int_0^t a_u(\zeta-\gamma \delta^a_u)\,(S_u+\delta_u^a)\,du-\int_0^t b_u(\zeta-\gamma \delta^b_u)\,(S_u-\delta_u^b)\,du,
    \nonumber
\end{gather*}
where $\delta^a,\delta^b\in\mathbb{H}^8$ represent the control. Note that here we seek strategies in the space $\mathbb{H}^8$ to make sure the associated inventory $Q$ is still in $\mathbb{H}^4$ (and indeed in $\mathbb{S}^4$). Hence, the following objective functional is well-defined and can be simplified as before:
\begin{equation*}
\begin{aligned}
    J(\boldsymbol{\delta}):&=\mathbb{E}\big[X_T+S_T\,Q_T-\int_0^T \phi_t\,(Q_t)^2\,dt-A\,(Q_T)^2 \big]\\
   &=\mathbb{E}\Big[\int_0^T\delta_t^a\,a_t(\zeta-\gamma\delta_t^a)\,dt+\int_0^T\delta_t^b\,b_t(\zeta-\gamma\delta_t^b)\,dt-\int_0^T \phi_t\,(Q_t)^2\,dt-A\,(Q_T)^2 \Big],
\end{aligned}
\end{equation*}
by controlling $\boldsymbol{\delta}\in\mathbb{H}^8\times\mathbb{H}^8$. Indeed, it suffices to observe
\begin{equation*}
\begin{aligned}
    \big|\int_0^T\delta_t^a\,a_t(\zeta-\gamma\delta_t^a)\,dt\big|&\leq C\,\sup_{0\leq t\leq T}|a_t|^2+C\,\big|\int_0^T(\delta_t^a)^2\,dt\big|+C\,\big|\int_0^T(\delta_t^a)^2\,dt\big|^2,\\
    \big|\int_0^T \phi_t\,(Q_t)^2\,dt\big|&\leq C\,\sup_{0\leq t\leq T}|Q_t|^4+C\,\big|\int_0^T(\phi_t)^2\,dt\big|
\end{aligned}
\end{equation*}
to justify the integrability. Since the convex nature of the objective functional $J$ is revealed in the proof of Theorem \ref{verification}, the convex-analytic method is used to characterize the (unique) optimal control as the solution of a FBSDE.

\kong

\begin{theorem}
    The functional $\boldsymbol{\delta}\hookrightarrow J(\boldsymbol{\delta})$ is G\^ateaux differentiable and the derivative in the direction of $\boldsymbol{\beta}:=(\beta^a, \beta^b)\in\mathbb{H}^8\times\mathbb{H}^8$ reads
\begin{equation}
\begin{aligned}
    \frac{d}{d\epsilon}&J(\boldsymbol{\delta}+\epsilon\boldsymbol{\beta})\big|_{\epsilon=0}:=\lim_{\epsilon\searrow0}\frac{1}{\epsilon}\,\big[J(\boldsymbol{\delta}+\epsilon\boldsymbol{\beta})-J(\boldsymbol{\delta})\big]\\
    &=\mathbb{E}\Big[\int_0^T\big(a_t\,\beta_t^a\,(\zeta-2\gamma\,\delta_t^a)+b_t\,\beta_t^b\,(\zeta-2\gamma\,\delta_t^b)\big)\,dt-2\,\int_0^T\phi_t\,V_t\,Q_t\,dt-2\,A\,V_T\,Q_T\Big],
    \label{liner_gat_deriv}
\end{aligned}
\end{equation}
where $V_t=\gamma\int_0^t(a_s\,\beta_s^a-b_s\,\beta_s^b)\,ds$. A control $(\delta^{*,a},\delta^{*,b})\in\mathbb{H}^8\times\mathbb{H}^8$ maximises the functional $J$ if it can be represented by
\begin{equation}
    \delta_t^{b,*}=\frac{\zeta}{2\gamma}-\frac{1}{2}Y_t, \text{\; and \;} \delta_t^{a,*}=\frac{\zeta}{2\gamma}+\frac{1}{2}Y_t,
    \label{optimal_feedback_unbound}
\end{equation}
where $Y$ solves the following FBSDE:
\begin{equation*}
\left\{
\begin{aligned}
\;& dQ_t  = \zeta\,(b_t-a_t)\,dt\,/\,2+\gamma\,(a_t+b_t)\,Y_t\,dt\,/\,2, \\
& dY_t=2\,\phi_t\,Q_t\,dt+Z_t\,dW_t,\\
& Q_0=q_0,\quad Y_T=-2A\,Q_T.
\end{aligned}
\right.
\end{equation*}
The representation \eqref{optimal_feedback_unbound} is also necessary when $a,b>0$.
\end{theorem}

\begin{proof}
Fix an admissible control $\boldsymbol{\delta}=(\delta^a, \delta^b)\in \mathbb{H}^8\times\mathbb{H}^8$ and denote by $Q = Q^{\boldsymbol{\delta}}\in\mathbb{S}^2$ the corresponding controlled inventory. Next, consider $\boldsymbol{\beta}=(\beta^a, \beta^b)\in \mathbb{H}^8\times\mathbb{H}^8$---that is also uniformly bounded---as the direction in which we are computing the
G\^ateaux derivative of $J$. For each $\epsilon > 0$,
we consider the admissible control $\boldsymbol{\delta}^\epsilon\in\mathbb{H}^8\times\mathbb{H}^8$ defined by $\boldsymbol{\delta}^\epsilon_t = \boldsymbol{\delta}_t + \epsilon\,\boldsymbol{\beta}_t$, and the corresponding controlled state is denoted by $Q^\epsilon:= Q^{\boldsymbol{\delta}^\epsilon}\in\mathbb{S}^2$. Note that $V\in\mathbb{S}^{8\mathscr{D}}$. Direct calculations yield
\begin{equation}
\begin{aligned}
\label{compute_deriva}
    \frac{1}{\epsilon}\,&\big[J(\boldsymbol{\delta}+\epsilon\boldsymbol{\beta})-J(\boldsymbol{\delta})\big]\\
    &=\mathbb{E}\Big[\int_0^Ta_t\,\beta_t^a\,(\zeta-2\,\gamma\,\delta_t^a-\epsilon\,\gamma\,\beta_t^a)\,dt\Big]+\mathbb{E}\Big[\int_0^Tb_t\,\beta_t^b\,(\zeta-2\,\gamma\,\delta_t^b-\epsilon\,\gamma\,\beta_t^b)\,dt\Big]\\
    &\hspace{1cm}-\mathbb{E}\Big[\int_0^T \phi_t\,(Q_t^\epsilon+Q_t)\,V_t\,dt\Big]-\mathbb{E}\Big[A\,(Q_T^\epsilon+Q_T)\,V_T\Big].
\end{aligned}
\end{equation}
We look at the first term on the right hand side of \eqref{compute_deriva}. To perform the limiting procedure, note that
\begin{equation*}
\begin{aligned}
    a_t\,\beta_t^a\,(\zeta-2\,\gamma\,\delta_t^a-\epsilon\,\gamma\,\beta_t^a)\,dt & \leq a_t\,|\beta_t^a|\,\big(\zeta+2\,\gamma\,|\delta_t^a|+\gamma\,|\beta_t^a|\big),\\
\end{aligned}
\end{equation*}
the right hand side of which is $d\mathbb{P}\times dt$-integrable. Therefore, it follows from the dominated convergence theorem that
\begin{equation*}
    \lim_{\epsilon\to 0}\,\mathbb{E}\Big[\int_0^Ta_t\,\beta_t^a\,(\zeta-2\,\gamma\,\delta_t^a-\epsilon\,\gamma\,\beta_t^a)\,dt\Big]=\mathbb{E}\Big[\int_0^Ta_t\,\beta_t^a\,(\zeta-2\,\gamma\,\delta_t^a)\,dt\Big].
\end{equation*}
The second term of \eqref{compute_deriva} can be verified in the same way. With respect to third term, we are aware of
\begin{equation*}
\begin{aligned}
    \int_0^T \phi_t\,(Q_t^\epsilon+Q_t)\,V_t\,dt&\leq C\int_0^T \phi_t^2\,\big(|Q_t^\epsilon|+|Q_t|\big)\,dt+C\int_0^T V_t^2\,\big(|Q_t^\epsilon|+|Q_t|\big)\,dt,\\
    &\leq C\sup_{t\in[0,T]}\big(|Q_t^\epsilon|^2+|Q_t|^2\big)+C\,\Big(\int_0^T \phi_t^2\,dt\Big)^2+C\,\Big(\int_0^T V_t^2\,dt\Big)^2\\
    &\leq C\sup_{t\in[0,T]}|Q_t|^2+C\,\Big(\int_0^T (a_t+b_t)\,dt\Big)^2+C\,\Big(\int_0^T \phi_t^2\,dt\Big)^2\\
    &\hspace{2.8cm}+C\,\Big(\int_0^T V_t^2\,dt\Big)^2\\
    \phi_t\,(Q_t^\epsilon+Q_t)\,V_t\,&\leq C\,|Q_t|^2+C\,\Big(\int_0^T (a_t+b_t)\,dt\Big)^2+C\,\phi_t^2+C\,V_t^2\\
\end{aligned}
\end{equation*}
and the integrability on the right hand sides, and hence can conclude
\begin{equation*}
    \lim_{\epsilon\to 0}\,\mathbb{E}\Big[\int_0^T \phi_t\,(Q_t^\epsilon+Q_t)\,V_t\,dt\Big]=2\,\mathbb{E}\Big[\int_0^T \phi_t\,Q_t\,V_t\,dt\Big].
\end{equation*}
Since the final term can be proved similarly, the equation \eqref{compute_deriva} is verified and an integration by parts further yields
\begin{equation}
\begin{aligned}
    \frac{d}{d\epsilon}J(\boldsymbol{\delta}+\epsilon&\boldsymbol{\beta})\big|_{\epsilon=0}\\
    &=\mathbb{E}\bigg\{\int_0^T\Big[a_t\,\beta_t^a\,\big(\zeta-2\gamma\,\delta_t^a-2\gamma\int_t^T\phi_s\,Q_s\,ds-2\gamma\, A \, Q_T\big)\Big]\,dt\bigg\}\\
    &\hspace{2cm}+\mathbb{E}\bigg\{\int_0^T\Big[b_t\,\beta_t^b\,\big(\zeta-2\gamma\,\delta_t^b-2\gamma\int_t^T\phi_s\,Q_s\,ds+2\gamma\, A\, Q_T \big)\Big]\,dt\bigg\}.
    \label{derive_fbsde}
\end{aligned}
\end{equation}
Since the functional $J$ is concave, a control $\boldsymbol{\delta}^*=(\delta^{*,a}, \delta^{*,b})\in\mathbb{H}^8\times\mathbb{H}^8$ is optimal if and only if \eqref{derive_fbsde} is equal to $0$ for any bounded $\boldsymbol{\beta}\in\mathbb{H}^8\times\mathbb{H}^8$. With the law of total expectation, this infers that
\begin{equation*}
\begin{aligned}
    \gamma\,Y_t=\zeta-2\gamma\,\delta_t^{*,a}&=\mathbb{E}_t\Big[2\gamma\int_t^T\phi_s\,Q_s\,ds+2\gamma\, A \, Q_T\Big],\\
    \gamma\,Y_t=2\gamma\,\delta_t^{*,b}-\zeta&=\mathbb{E}_t\Big[2\gamma\int_t^T\phi_s\,Q_s\,ds+2\gamma\, A\, Q_T\Big]
\end{aligned}
\end{equation*}
$d\mathbb{P} \times dt$-almost everywhere. The FBSDE can be obtained by `differentiating' $Y$. 
\end{proof}

\kong

\noindent The remaining of this section is devoted the well-posedness of the FBSDE. Before looking at the FBSDE associated with the stochastic maximum principle, we first investigate the cornerstone of its well-posedness theory---a quadratic BSDE: 
\begin{equation}
    dY_t=\big(-\gamma\,(a_t+b_t)\,(Y_t)^2+2\,\phi_t\big)\,dt+Z_t\,dW_t, \text{\quad with \quad} Y_T=-2\,A,
    \label{quad_bsde_unbound_coeff}
\end{equation}
the coefficients $a, b, \phi, A$ of which are not bounded. Note that the $(Y,Z)$ here is different from the one in the FBSDE, and here we only focus on the existence of solutions. We localize this equation via the truncation:
\begin{equation}
    dY_t=\big(-\gamma\,\big((a_t+b_t)\wedge m\big)\,(Y_t)^2+2\,(\phi_t\wedge n)\big)\,dt+Z_t\,dW_t, \text{\quad with \quad} Y_T=-2\,(A\wedge n),
    \label{trun_bsde_unbound_coeff}
\end{equation}
where $m, n\in\mathbb{N}$. First, we try to remove the constant $n$.

\kong

\begin{lemma}
The BSDE
\begin{equation}
    dY_t=\big(-\gamma\,\big((a_t+b_t)\wedge m\big)\,(Y_t)^2+2\,\phi_t\big)\,dt+Z_t\,dW_t, \text{\quad with \quad} Y_T=-2\,A,
        \label{half_trun_bsde_unbound_coeff}
\end{equation}
accepts a solution $(Y, Z)\in\mathbb{S}^{8\mathscr{D}}\times\mathbb{H}^2$.
\end{lemma}    

\begin{proof}
Recall that \eqref{trun_bsde_unbound_coeff} is well-posed for any pair $(m, n)$ by Lemma \ref{growth_lip_fbsde}. Fixing any $m\in\mathbb{N}$, equation \eqref{trun_bsde_unbound_coeff} accepts a unique solution $(Y^{(n)}, Z^{(n)})\in\mathbb{S}^2\times\mathbb{H}^2$ for all $n\in\mathbb{N}$. Because $|Y_t^{(n)}|\leq 2n+2nT$, the standard comparison theorem for Lipschitz BSDEs implies
\begin{equation*}
    0 \geq Y_t^{(n)} \geq Y_t^{(n+1)}, \qquad 0 \leq t \leq T, \qquad \mathbb{P}\text{-a.s.}
\end{equation*}
Consequently, for almost every $\omega\in\Omega$, we can define the limit as follows:
\begin{equation*}
    U_t(\omega):=\lim_{n\to\infty}Y_t^{(n)}(\omega),\qquad 0 \leq t \leq T.
\end{equation*}
Note that $-U_t(\omega)$ may be infinity for some $t$ and $\omega$. Thanks to the monotonicity of $(Y^{(n)})_{n\in\mathbb{N}}$, we observe that
\begin{equation*}
    \lim_{n\to\infty}|Y_t^{(n)}|^{8\mathscr{D}}\leq \lim_{n\to\infty}\,\sup_{0\leq u\leq T}|Y_u^{(n)}|^{8\mathscr{D}}, \qquad 0 \leq t \leq T,
\end{equation*}
the right hand side of which is well-defined (possibly infinite). It further reveals the integrability of $U$ as follows
\begin{equation}
\begin{aligned}
    \mathbb{E}\Big[\sup_{0\leq t\leq T}|U_t|^{8\mathscr{D}}\Big]&\leq \lim_{n\to\infty}\,\mathbb{E}\Big[\sup_{0\leq t\leq T}|Y_t^{(n)}|^{8\mathscr{D}}\Big]\\
    &= \lim_{n\to\infty}\,\mathbb{E}\Big[\sup_{0\leq u\leq T}\Big|\,\mathbb{E}_u\big[-2\,(A\wedge n)-2\int_t^T (\phi_s\wedge n)\,ds\\
    &\hspace{3.5cm}+\int_t^T\gamma\,\big((a_u+b_u)\wedge m\big)\,(Y_u)^2\,du\big]\,\Big|^{8\mathscr{D}}\Big]\\
    &\leq \lim_{n\to\infty}\,\mathbb{E}\Big[\sup_{0\leq u\leq T}\mathbb{E}_u\big[2\,(A\wedge n)+2\int_t^T (\phi_s\wedge n)\,ds\big]^{8\mathscr{D}}\Big]\\
    &\leq \lim_{n\to\infty}\,\mathbb{E}\Big[\sup_{0\leq u\leq T}\mathbb{E}_u\big[2\,A+2\int_0^T \phi_s\,ds\big]^{8\mathscr{D}}\Big]\\
    &\leq C\,\mathbb{E}\Big[\big(2\,A+2\int_0^T \phi_s\,ds\big)^{8\mathscr{D}}\Big]\\
    &<\infty,
    \label{integ_Y}
\end{aligned}
\end{equation}
where we have applied the monotone convergence theorem, the fact $Y_t^{(n)}\leq 0$, and the Doob's inequality. For any $u\in[0,T]$, since two random variables
\begin{equation*}
    \mathbb{E}_u\Big[2\,A+2\int_u^T \phi_s\,ds\Big] \text{\quad and \quad} \mathbb{E}_u\Big[\int_u^T \big((a_s+b_s)\wedge m\big)\,(U_s)^2\,ds\Big]
\end{equation*}
are both integrable, then they are almost surely finite and one can use the monotone convergence theorem again to see
\begin{equation*}
\begin{aligned}
    U_t&=\lim_{n\to\infty}Y_t^{(n)}\\
    &=\lim_{n\to\infty}\mathbb{E}_t\Big[-2\,(A\wedge n)-2\int_t^T (\phi_s\wedge n)\,ds+\gamma\int_t^T \big((a_s+b_s)\wedge m\big)\,\big(Y_s^{(n)}\big)^2\,ds\Big]\\
    &=\mathbb{E}_t\Big[-2\,A-2\int_t^T \phi_s\,ds+\gamma\int_t^T \big((a_s+b_s)\wedge m\big)\,(U_s)^2\,ds\Big].
\end{aligned}
\end{equation*}
While the random variable
\begin{equation*}
    -2\,A-2\int_0^T \phi_s\,ds+\gamma\int_0^T \big((a_s+b_s)\wedge m\big)\,(Y_s)^2\,ds
\end{equation*}
is square integrable, we finally conclude that $U$ satisfies the BSDE
\begin{equation*}
    dU_t=\big(-\gamma\,\big((a_t+b_t)\wedge m\big)\,(U_t)^2+2\,\phi_t\big)\,dt+Z_t^U\,dW_t, \text{\quad with \quad} U_T=-2\,A,
\end{equation*}
where $Z^U\in\mathbb{H}^2$ is obtained via the martingale representation theorem. 
\end{proof}

\kong

\noindent Note that the solution of \eqref{half_trun_bsde_unbound_coeff} may not be bounded, and hence the standard comparison theorem for Lipschitz BSDEs can not be applied. We summarize the comparison result in the following lemma.

\kong

\begin{lemma}
For each $m\in\mathbb{N}$, denote by $(Y^{(m)}, Z^{(m)})$ the solution of equation \eqref{half_trun_bsde_unbound_coeff}. Then, $(Y^{(m)})_{m\in\mathbb{N}}$ is non-decreasing with respect to $m$.
\end{lemma}

\begin{proof}
Define $\Delta Y:=Y^{(m+1)}-Y^{(m)}$ and $\Delta Z:=Z^{(m+1)}-Z^{(m)}$. The proof starts with taking the difference of two BSDEs:
\begin{equation*}
\begin{aligned}
    d\Delta Y_t&=\Big(-\gamma\,\big((a_t+b_t)\wedge (m+1)\big)\,(Y_t^{(m+1)})^2+\gamma\,\big((a_t+b_t)\wedge (m+1)\big)\,(Y_t^{(m)})^2\\
    &\hspace{1cm}-\gamma\,\big((a_t+b_t)\wedge (m+1)\big)\,(Y_t^{(m)})^2+\gamma\,\big((a_t+b_t)\wedge m\big)\,(Y_t^{(m)})^2\Big)\,dt+\Delta Z_t\,dW_t\\
    &=\Big(-\gamma\,\big((a_t+b_t)\wedge (m+1)\big)\,(Y_t^{(m+1)}+Y_t^{(m)})\,\Delta Y_t\\
    &\hspace{1cm}-\gamma\, (Y_t^{(m)})^2\,\big[(a_t+b_t)\wedge (m+1)-(a_t+b_t)\wedge m\big]\Big)\,dt+\Delta Z_t\,dW_t.
\end{aligned}
\end{equation*}
Set $\iota_t=\gamma\,\big((a_t+b_t)\wedge (m+1)\big)\,(Y_t^{(m+1)}+Y_t^{(m)})$, and observe that $\iota_t\leq 0$ since any $Y^{(m)}$ is non-positive. The integration factor gives
\begin{equation*}
    \Delta Y_t e^{\int_0^t \iota_u\,du}=\gamma\,\mathbb{E}_t\Big[\int_t^Te^{\int_0^s\iota_u\,du}\,(Y_s^{(m)})^2\,\big[(a_s+b_s)\wedge (m+1)-(a_s+b_s)\wedge m\big]\,ds\Big]\geq 0,
\end{equation*}
where we have used the fact that $\Delta Y_T=0$. 
\end{proof}

\kong

\noindent Given the comparison result, we can finally remove the constant $m$ as well.

\kong

\begin{theorem}
The BSDE \eqref{quad_bsde_unbound_coeff} accepts a solution $(Y, Z)\in\mathbb{S}^{8\mathscr{D}}\times\mathbb{H}^2$.
\label{well_posed_quad_bsde}
\end{theorem}

\begin{proof}
\noindent Denote by $(Y^{(m)}, Z^{(m)})\in\mathbb{S}^{8\mathscr{D}}\times\mathbb{H}^2$ the solution of equation \eqref{half_trun_bsde_unbound_coeff} for each $m\in\mathbb{N}$. Given the monotonicity of $(Y^{(m)})_{m\in\mathbb{N}}$, for almost every $\omega\in\Omega$, let us define the limit similarly as
\begin{equation*}
    Y_t(\omega):=\lim_{m\to\infty}Y_t^{(m)}(\omega)\leq 0,\qquad 0 \leq t \leq T.
\end{equation*}
Its integrability can be directly seen from
\begin{equation*}
    \mathbb{E}\Big[\sup_{0\leq t\leq T}|Y_t|^{8\mathscr{D}}\Big]\leq \mathbb{E}\Big[\sup_{0\leq t\leq T}|Y_t^{(1)}|^{8\mathscr{D}}\Big]<\infty,
\end{equation*}
since $Y^{(m)}\leq Y^{(m+1)}\leq0$ for any $m\in\mathbb{N}$. Thanks to the integrability of the random variable
\begin{equation*}
    \mathbb{E}_u\Big[\int_u^T (a_s+b_s)\,\big(Y_s^{(1)}\big)^2\,ds\Big],
\end{equation*}
the dominated convergence theorem implies
\begin{equation*}
\begin{aligned}
    Y_t=\lim_{m\to\infty}Y_t^{(m)}&=\lim_{m\to\infty}\mathbb{E}_t\Big[-2\,A-2\int_t^T \phi_s\,ds+\gamma\int_t^T \big((a_s+b_s)\wedge m\big)\,\big(Y_s^{(m)}\big)^2\,ds\Big]\\
    &=\mathbb{E}_t\Big[-2\,A-2\int_t^T \phi_s\,ds+\gamma\int_t^T (a_s+b_s)\,(Y_s)^2\,ds\Big].
\end{aligned}
\end{equation*}
Finally, the square integrability of the random variable
\begin{equation*}
    -2\,A-2\int_0^T \phi_s\,ds+\gamma\int_0^T (a_s+b_s)\,(Y_s)^2\,ds
\end{equation*}
helps us conclude that $Y$ solves the BSDE
\begin{equation*}
    dY_t=\big(-\gamma\,(a_t+b_t)\,(Y_t)^2+2\,\phi_t\big)\,dt+Z_t\,dW_t, \text{\quad with \quad} Y_T=-2\,A.
\end{equation*}
Note that $Z$ is again obtained via the martingale representation theorem. 
\end{proof}

\kong

Subsequently, we look at the FBSDE associated with the stochastic maximum principle
\begin{equation}
\left\{
\begin{aligned}
\;& dQ_t  = \zeta\,(b_t-a_t)\,dt\,/\,2+\gamma\,(a_t+b_t)\,Y_t\,dt\,/\,2, \\
& dY_t=2\,\phi_t\,Q_t\,dt+Z_t\,dW_t,\\
& Q_0=q_0,\quad Y_T=-2A\,Q_T.
\label{linear_fbsde_unbound}
\end{aligned}
\right.
\end{equation}
The next theorem presents the well-posedness result of the main FBSDE, and then specifies the value of $\mathscr{D}$ in line with the admissibility of the control.

\kong

\begin{theorem}
The FBSDE \eqref{linear_fbsde_unbound} accepts a unique solution $(Q, Y, Z)\in \mathbb{S}^{2\mathscr{D}}\times\mathbb{S}^{\mathscr{D}}\times\mathbb{H}^{2}$. It suffices to let $\mathscr{D}=8$ to ensure the resulting control is admissible.
\end{theorem}

\begin{proof}
The linear structure of \eqref{linear_fbsde_unbound} suggests the affine ansatz
\begin{equation}
    Y_t = \mathscr{P}_t\,Q_t+\mathscr{H}_t,
    \label{unbound_affine_ansatz}
\end{equation}
where processes $\mathscr{P}$ and $\mathscr{H}$ solve the BSDEs
\begin{equation*}
    d\mathscr{P}_t=\big(-\gamma\,(a_t+b_t)\,(\mathscr{P}_t)^2\,/\,2+2\,\phi_t\big)\,dt+Z_t^1\,dW_t, \text{\quad with \quad} \mathscr{P}_T=-2\,A,
\end{equation*}
\begin{equation*}
    d\mathscr{H}_t=\big(-\gamma\,(a_t+b_t)\,\mathscr{P}_t\,\mathscr{H}_t-\zeta\,(b_t-a_t)\,\mathscr{P}_t\big)\,dt\,/\,2+Z_t^2\,dW_t, \text{\quad with \quad} \mathscr{H}_T=0.
\end{equation*}
The well-posedness of $\mathscr{P}\in \mathbb{S}^{8\mathscr{D}}$ is guaranteed by Theorem \ref{well_posed_quad_bsde}, and the linear BSDE of $\mathscr{H}$ can be solved explicitly by
\begin{equation*}
\begin{aligned}
    \mathscr{H}_t=\mathbb{E}_t\Big[\int_t^T\zeta\,(b_s-a_s)\,\mathscr{P}_s\,e^{\int_s^T\gamma\,(a_u+b_u)\,\mathscr{P}_u\,du}\,ds\Big].
\end{aligned}
\end{equation*}
Direct estimations and Doob’s inequality then yield
\begin{equation*}
\begin{aligned}
    |\mathscr{H}_t| &\leq C\sup_{0\leq u\leq T}\mathbb{E}_u\Big[\int_0^T(a_s-b_s)^2\,ds\Big]+C\sup_{0\leq u\leq T}\mathbb{E}_u\Big[\int_0^T(\mathscr{P}_s)^2\,ds\Big],\\
    \mathbb{E}\big[\sup_{0\leq t\leq T}|\mathscr{H}_t|^{4\mathscr{D}}]&\leq C\,\mathbb{E}\Big[\big(\int_0^T(a_s-b_s)^2\,ds\big)^{4\mathscr{D}}\Big]+C\,\mathbb{E}\Big[\big(\int_0^T(\mathscr{P}_s)^2\,ds\big)^{4\mathscr{D}}\Big]<\infty,
\end{aligned}
\end{equation*}
from which one can see $\mathscr{H}\in\mathbb{S}^{4\mathscr{D}}$. On the other hand, we can calculate the corresponding inventory
\begin{equation*} Q_t=q_0\,e^{\gamma\,\int_0^t(a_s+b_s)\,\mathscr{P}_s\,ds\,/\,2}+\int_0^t e^{\gamma\,\int_s^t(a_u+b_u)\,\mathscr{P}_u\,du\,/\,2}\big[\zeta\,(b_s-a_s)\,/\,2+\gamma\,(a_s+b_s)\,\mathscr{H}_s\,/\,2\big]\,ds,
\end{equation*}
and deduce that $Q\in\mathbb{S}^{2\mathscr{D}}$ from the following estimation:
\begin{equation*}
\begin{aligned}
    \mathbb{E}\big[\sup_{0\leq t\leq T}|Q_t|^{2\mathscr{D}}\big]&\leq C\, q_0^{2\mathscr{D}}+C\,\mathbb{E}\Big[\big(\int_0^T(b_t-a_t)^2\,dt\big)^{\mathscr{D}}\Big]+C\,\mathbb{E}\Big[\big(\int_0^T(a_t+b_t)\,|\mathscr{H}_t|\,dt\big)^{2\mathscr{D}}\Big]\\
    &\leq C\, q_0^{2\mathscr{D}}+C\,\mathbb{E}\Big[\big(\int_0^T(b_t-a_t)^2\,dt\big)^{\mathscr{D}}\Big]+C\,\mathbb{E}\Big[\big(\int_0^T(a_t+b_t)^2\,dt\big)^{2\mathscr{D}}\Big]\\
    &\hspace{3cm}+C\,\mathbb{E}\Big[\big(\int_0^T|\mathscr{H}_t|^2\,dt\big)^{{2\mathscr{D}}}\Big]\\
    &<\infty.
\end{aligned}
\end{equation*}
Since $Y_t = \mathscr{P}_t\,Q_t+\mathscr{H}_t$ and
\begin{equation*}
    \mathbb{E}\big[\sup_{0\leq t\leq T} |\mathscr{P}_t\,Q_t|^\mathscr{D}\big]\leq C\,\mathbb{E}\big[\sup_{0\leq t\leq T} |\mathscr{P}_t|^{2\mathscr{D}}\big]+C\,\mathbb{E}\big[\sup_{0\leq t\leq T} |Q_t|^{2\mathscr{D}}\big]<\infty,
\end{equation*}
it yields $Y\in\mathbb{S}^{\mathscr{D}}$. The uniqueness of solution again follows from the continuation argument in Lemma \ref{growth_lip_fbsde}.

Recalling that the optimal control is given by \eqref{optimal_feedback_unbound} and the admissibility requires that the control is in $\mathbb{H}^8$, it suffices to take $\mathscr{D}=8$. 
\end{proof}

\section{Appendix: Stochastic Maximum Principle} \label{section_5}

\noindent This section is devoted to the stochastic maximum principle for the general intensity function. We follow the argument in \cite{carmona2016lectures}, and start with the \textit{necessary} condition. Fix an admissible control $\boldsymbol{\delta}\in \mathbb{A}\times\mathbb{A}$ and denote by $Q = Q^{\boldsymbol{\delta}}\in\mathbb{S}^2$ the corresponding controlled inventory, which is also uniformly bounded. Next, consider $\boldsymbol{\beta}\in \mathbb{A}\times\mathbb{A}$---that is uniformly bounded and satisfies that $\boldsymbol{\delta}+\boldsymbol{\beta}\in\mathbb{A}\times\mathbb{A}$---as the direction in which we are computing the
G\^ateaux derivative of $J$. For each $\epsilon > 0$ small enough,
we consider the admissible control $\boldsymbol{\delta}^\epsilon\in \mathbb{A}\times\mathbb{A}$ defined by $\boldsymbol{\delta}^\epsilon_t = \boldsymbol{\delta}_t + \epsilon\,\boldsymbol{\beta}_t$, and the corresponding controlled state $Q^\epsilon:= Q^{\boldsymbol{\delta}^\epsilon}\in\mathbb{S}^2$. Define $V$ as the solution of the equation
\begin{equation*}
    dV_t=\partial_{\boldsymbol{\delta}}\mu(t, \boldsymbol{\delta}_t)\cdot\boldsymbol{\beta}_t\,dt,
\end{equation*}
with the initial condition $V_0 = 0$, where
\begin{equation*}
    \mu(t,\boldsymbol{\delta}):=-a_t\,\Lambda^a(\delta_t^a)+b_t\,\Lambda^b(\delta_t^b).
\end{equation*}
It is clear that $V$ is uniformly bounded in the finite time horizon. We then look at the G\^ateaux differentiability of the functional $J$.

\kong

\begin{lemma}
The functional $\boldsymbol{\delta}\hookrightarrow J(\boldsymbol{\delta})$ is G\^ateaux differentiable and the derivative reads
\begin{equation}
\begin{aligned}
    \frac{d}{d\epsilon}J(\boldsymbol{\delta}+\epsilon\boldsymbol{\beta})\big|_{\epsilon=0}:&=\lim_{\epsilon\searrow0}\frac{1}{\epsilon}\,\big[J(\boldsymbol{\delta}+\epsilon\boldsymbol{\beta})-J(\boldsymbol{\delta})\big]\\
    &=\mathbb{E}\Big[\int_0^T\big(\partial_qf(t,Q_t,\boldsymbol{\delta}_t)\,V_t+\partial_{\boldsymbol{\delta}}f(t,Q_t,\boldsymbol{\delta}_t)\cdot\beta_t\big)\,dt-2\,A\,V_T\,Q_T\Big],
    \label{gen_gat_deriv}
\end{aligned}
\end{equation}
where $f$ is the running payoff defined by
\begin{equation*}
    f(t, Q_t, \boldsymbol{\delta}_t)=a_t\,\delta_t^a\,\Lambda^a(\delta_t^a)+b_t\,\delta_t^b\,\Lambda^b(\delta_t^b)-\phi_t\,Q_t^2.
\end{equation*}
\end{lemma}

\begin{proof}
Regarding the ask side, we compute that
\begin{equation*}
\begin{aligned}
\frac{1}{\epsilon}\,&\mathbb{E}\Big[\int_0^Ta_t\,(\delta_t^a+\epsilon\,\beta_t^a)\,\Lambda^a(\delta_t^a+\epsilon\,\beta_t^a)\,dt-\int_0^T a_t\,\delta_t^a\,\,\Lambda^a(\delta_t^a)\,dt\Big]\\
&=\mathbb{E}\Big[\int_0^Ta_t\,\delta_t^a\,\frac{1}{\epsilon}\,\big(\Lambda^a(\delta_t^a+\epsilon\,\beta_t^a)-\Lambda^a(\delta_t^a)\big)\,dt\Big]+\mathbb{E}\Big[\int_0^Ta_t\,\beta_t^a\,\Lambda^a(\delta_t^a+\epsilon\,\beta_t^a)\,dt\Big].
\end{aligned}
\end{equation*}
To perform the limiting procedure, note that
\begin{equation*}
\begin{aligned}
    a_t\,\delta_t^a\,\frac{1}{\epsilon}\,\big(\Lambda^a(\delta_t^a+\epsilon\,\beta_t^a)-\Lambda^a(\delta_t^a)\big)&\leq C\,a_t\,|\delta_t^a|,\\
    a_t\,\beta_t^a\,\Lambda^a(\delta_t^a+\epsilon\,\beta_t^a) &\leq C\, a_t,
\end{aligned}
\end{equation*}
the right hand sides of which are $d\mathbb{P}\times dt$-integrable on the time horizon. Therefore, it follows by the dominated convergence theorem that
\begin{equation*}
\begin{aligned}
    \lim_{\epsilon\to 0}\frac{1}{\epsilon}\,\mathbb{E}\Big[\int_0^Ta_t\,(\delta_t^a+\epsilon\,\beta_t^a)\,&\Lambda^a(\delta_t^a+\epsilon\,\beta_t^a)\,dt-\int_0^T a_t\,\delta_t^a\,\,\Lambda^a(\delta_t^a)\,dt\Big]\\
    &=\mathbb{E}\Big[\int_0^Ta_t\,\delta_t^a\,\big(\Lambda^{a}(\delta_t^a)\big)'\,\beta_t^a\,dt\Big]+\mathbb{E}\Big[\int_0^Ta_t\,\beta_t^a\,\Lambda^a(\delta_t^a)\,dt\Big]\\
    &=\mathbb{E}\Big[\int_0^T\partial_{\delta^a}f(t,Q_t,\boldsymbol{\delta}_t)\,\beta_t^a\,dt\Big].
\end{aligned}
\end{equation*}
The bid side can be computed in the same way:
\begin{equation*}
\begin{aligned}
    \lim_{\epsilon\to 0}\frac{1}{\epsilon}\,\mathbb{E}\Big[\int_0^Tb_t\,(\delta_t^b+\epsilon\,\beta_t^b)\,&\Lambda^b(\delta_t^b+\epsilon\,\beta_t^b)\,dt-\int_0^T b_t\,\delta_t^b\,\,\Lambda^b(\delta_t^b)\,dt\Big]\\
    &=\mathbb{E}\Big[\int_0^T\partial_{\delta^b}f(t,Q_t,\boldsymbol{\delta}_t)\,\beta_t^b\,dt\Big].
\end{aligned}
\end{equation*}
With respect to the running penalty, we further calculate that
\begin{equation*}
\begin{aligned}
    \frac{1}{\epsilon}\,\mathbb{E}\Big[\int_0^T &\phi_t\,(Q_t^\epsilon)^2\,dt-\int_0^T \phi_t\,(Q_t)^2\,dt\Big]\\
    &=\mathbb{E}\Big[\int_0^T \phi_t\,(Q_t^\epsilon+Q_t)\,\Big(-\int_0^ta_u\,\frac{1}{\epsilon}\,\big[\Lambda^a(\delta_u^a+\epsilon\,\beta_u^a)-\Lambda^a(\delta_u^a)\big]\,du\\
    &\hspace{4cm}+\int_0^tb_u\,\frac{1}{\epsilon}\,\big[\Lambda^b(\delta_u^b+\epsilon\,\beta_u^b)-\Lambda^b(\delta_u^b)\big]\,du\Big)\,dt\Big].
\end{aligned}
\end{equation*}
Being aware of the boundedness of $\phi$, $Q$, $Q^\epsilon$, $a$,  $b$ as well as the one of
\begin{equation*}
 \frac{1}{\epsilon}\,\big[\Lambda^a(\delta_u^a+\epsilon\,\beta_u^a)-\Lambda^a(\delta_u^a)\big],
 \text{\quad and \quad} \frac{1}{\epsilon}\,\big[\Lambda^b(\delta_u^b+\epsilon\,\beta_u^b)-\Lambda^b(\delta_u^b)\big],
\end{equation*}
we can conclude
\begin{equation*}
    \lim_{\epsilon\to 0}\,\frac{1}{\epsilon}\,\mathbb{E}\Big[\int_0^T \phi_t\,(Q_t^\epsilon)^2\,dt-\int_0^T \phi_t\,(Q_t)^2\,dt\Big]=2\,\mathbb{E}\Big[\int_0^T \phi_t\,Q_t\,V_t\,dt\Big].
\end{equation*}
The terminal penalty part can be justified in the same way and the proof is complete. 
\end{proof}

\kong

\noindent Let $(Y,Z)$ be the adjoint processes associated with $\boldsymbol{\delta}$, i.e., processes $(Y,Z)$ solve the BSDE
\begin{equation*}
    dY_t=2\,\phi_t\,Q_t\,dt+Z_t\,dW_t, \text{\quad and \quad} Y_T=-2A\,Q_T.
\end{equation*}
The well-posedness of such equation is guaranteed due to the boundedness of $A$, $\phi$ and $Q$. The duality relation provides an expression of the G\^ateaux derivative of the cost functional in terms of the Hamiltonian of the system.

\kong

\begin{lemma}
The duality relation is given by
\begin{equation*}
    \mathbb{E}[Y_T\,V_T]=\mathbb{E}\Big[\int_0^T\big(\partial_{\boldsymbol{\delta}}\mu(t,\boldsymbol{\delta}_t)\cdot\boldsymbol{\beta}_t\,Y_t-\partial_qf(t,Q_t,\boldsymbol{\delta}_t)\,V_t\big)\,dt\Big].
\end{equation*}
The duality further implies
\begin{equation*}
    \frac{d}{d\epsilon}\,J(\boldsymbol{\delta}+\epsilon\boldsymbol{\beta})\big|_{\epsilon=0}=\mathbb{E}\Big[\int_0^T\partial_{\boldsymbol{\delta}}\mathscr{H}(t,Q_t,Y_t,\boldsymbol{\delta}_t)\cdot\boldsymbol{\beta}_t\,dt\Big],
\end{equation*}
where $\mathscr{H}$ denotes the Hamiltonian as follows
\begin{equation*}
    \mathscr{H}(t, q, y, \boldsymbol{\delta}):=\big(-a_t\,\Lambda^a(\delta^a)+b_t\,\Lambda^b(\delta^b)\big)\,y+a_t\,\delta^a\,\Lambda^a(\delta^a)+b_t\,\delta^b\,\Lambda^b(\delta^b)-\phi_t\,(q)^2.
\end{equation*}
\end{lemma}

\begin{proof}
The integration by parts gives
\begin{equation*}
\begin{aligned}
    Y_T\,V_T&=Y_0\,V_0+\int_0^T Y_t\,dV_t+\int_0^T V_t\,dY_t\\
    &=\int_0^TY_t\, \partial_{\boldsymbol{\delta}}b(t, \boldsymbol{\delta}_t)\cdot\boldsymbol{\beta}_t\,dt+\int_0^T 2\,\phi_t\,V_t\,Q_t\,dt+\int_0^T V_t\,Z_t\,dW_t.
\end{aligned}
\end{equation*}
The duality is verified after taking the expectation, since the last term is a square integrable martingale. Using this relation, we can further compute
\begin{equation*}
\begin{aligned}
    \frac{d}{d\epsilon}&J(\boldsymbol{\delta}+\epsilon\boldsymbol{\beta})\big|_{\epsilon=0}\\
    &=\mathbb{E}\Big[\int_0^T\big(\partial_qf(t,Q_t,\boldsymbol{\delta}_t)\,V_t+\partial_{\boldsymbol{\delta}}f(t,Q_t,\boldsymbol{\delta}_t)\cdot\boldsymbol{\beta}_t\big)\,dt+V_T\,Y_T\Big]\\
    &=\mathbb{E}\Big[\int_0^TY_t\, \partial_{\boldsymbol{\delta}}b(t, \boldsymbol{\delta}_t)\cdot\boldsymbol{\beta}_t\,dt+\int_0^T\partial_{\boldsymbol{\delta}}f(t,Q_t,\boldsymbol{\delta}_t)\cdot\boldsymbol{\beta}_t\,dt\Big]\\
    &=\mathbb{E}\Big[\int_0^T\partial_{\boldsymbol{\delta}}\mathscr{H}(t,Q_t,Y_t,\boldsymbol{\delta}_t)\cdot\boldsymbol{\beta}_t\,dt\Big].
\end{aligned}
\end{equation*}

\end{proof}

\kong

\noindent With these preliminaries, we finally reach the necessary condition for optimality.

\kong

\begin{theorem}
If the admissible control $\boldsymbol{\delta}$ is optimal, $Q$ is the associated controlled inventory, and $(Y, Z)$ are the associated adjoint
processes, then
\begin{equation*}
    \mathscr{H}(t,Q_t,Y_t,\boldsymbol{\delta}_t)\geq\mathscr{H}(t,Q_t,Y_t,\beta), \qquad \text{a.e. in } t\in[0,T],\; \mathbb{P} \text{-} a.s.,
\end{equation*}
for any $\beta\in[-\xi, \xi]\times[-\xi, \xi]$.
\end{theorem}

\begin{proof}
Due to the convexity of the admissible control, given any admissible and bounded $\boldsymbol{\beta}\in\mathbb{A}\times\mathbb{A}$, we can choose the perturbation $\boldsymbol{\delta}^\epsilon=\boldsymbol{\delta}+\epsilon\,(\boldsymbol{\beta}-\boldsymbol{\delta})$ which is still admissible. Since $\boldsymbol{\delta}$ is optimal, we have the inequality
\begin{equation*}
    \frac{d}{d\epsilon}\,J(\boldsymbol{\delta}+\epsilon\,(\boldsymbol{\beta}-\boldsymbol{\delta}))\big|_{\epsilon=0}=\mathbb{E}\Big[\int_0^T\partial_{\boldsymbol{\delta}}\mathscr{H}(t,Q_t,Y_t,\boldsymbol{\delta}_t)\cdot(\boldsymbol{\beta}_t-\boldsymbol{\delta}_t)\,dt\Big]\leq0.
\end{equation*}
From this we can see 
\begin{equation}
    \partial_{\boldsymbol{\delta}}\mathscr{H}(t,Q_t,Y_t,\boldsymbol{\delta}_t)\cdot(\boldsymbol{\beta}-\boldsymbol{\delta}_t)\leq 0, \qquad \text{a.e. in } t\in[0,T],\; \mathbb{P}-a.s.,
    \label{smp_gener}
\end{equation}
for all $\beta\in[-\xi, \xi]\times[-\xi, \xi]$. Regarding the condition \eqref{smp_gener}, it can only happen at time $t$ if one of the following three cases holds:
\begin{equation*}
\begin{aligned}
    \delta_t^j\in(-\xi, +\xi) &\text{\quad with \quad} 
    \partial_{\boldsymbol{\delta}^j}\mathscr{H}(t,Q_t,Y_t,\boldsymbol{\delta}_t)=0;\\
    \delta_t^j=-\xi\leq \beta^j &\text{\quad with \quad} \partial_{\boldsymbol{\delta}^j}\mathscr{H}(t,Q_t,Y_t,\boldsymbol{\delta}_t)\leq0;\\
    \delta_t^j=+\xi\geq \beta^j &\text{\quad with \quad} \partial_{\boldsymbol{\delta}^j}\mathscr{H}(t,Q_t,Y_t,\boldsymbol{\delta}_t)\geq0,
\end{aligned}
\end{equation*}
where $j\in\{a, b\}$ representing the bid-ask side. Take the ask side for example. If
\begin{equation*}
    \partial_{\boldsymbol{\delta}^a}\mathscr{H}(t,Q_t,Y_t,\boldsymbol{\delta}_t)=-a_t\,(\Lambda^a(\delta_t^a))'\,\Big[Y_t-\delta_t^a-\frac{\Lambda^a(\delta_t^a)}{(\Lambda^a(\delta_t^a))'}\Big]=0,
\end{equation*}
the monotonicity of $\delta+\frac{\Lambda^a(\delta)}{(\Lambda^a(\delta))'}$ (see Assumption \ref{inten_assu}) implies that $\delta_t^a$ maximizes the corresponding part of the Hamiltonian. On the other hand, we look at the case $\delta_t^a=-\xi$ and  $\partial_{\boldsymbol{\delta}^a}\mathscr{H}(t,Q_t,Y_t,\boldsymbol{\delta}_t)\leq0$. Since  $\mathscr{H}$ is first increasing and then decreasing, such condition indicates that $\delta_t^a=-\xi$ is the maximizer of the Hamiltonian on the interval $[-\xi, \xi]$. While the final case can be discussed in the same way, all three cases imply that the optimal control $\boldsymbol{\delta}$ maximizes the Hamiltonian along the optimal paths, as stated by the theorem. 
\end{proof}

\kong

The \textit{sufficient} condition is more straightforward. We summarize the result in the following theorem.

\kong


\begin{theorem}
Let $\boldsymbol{\delta}$ be an admissible control, 
$Q = Q^{\boldsymbol{\delta}}$ be the corresponding controlled inventory, and $(Y, Z)$ be the adjoint processes. If it holds $\mathbb{P}$-a.s. that
\begin{equation}
    \mathscr{H}(t,Q_t,Y_t,\boldsymbol{\delta}_t)=\sup_{\boldsymbol{\alpha}\in [-\xi,\xi]^2}\mathscr{H}(t,Q_t,Y_t,\boldsymbol{\alpha}),  \qquad \text{a.e. in } t\in[0,T],
    \nonumber
\end{equation}
then $\boldsymbol{\delta}$ is the optimal control, that is to say, $J(\boldsymbol{\delta})=\sup_{\boldsymbol{\alpha}\in \mathbb{A}\times\mathbb{A}}J(\boldsymbol{\alpha})$.
\end{theorem}

\begin{proof}
Since the inventory process---that is associated with any admissible strategy---is bounded, the adjoint BSDE
\begin{equation}
    dY_t=2\phi_t\, Q_t\,dt+Z_t\,dW_t, \quad \text{with} \quad Y_T=-2A\,Q_T,
    \nonumber
\end{equation}
admits a unique solution $(Y,Z)\in\mathbb{S}^2\times\mathbb{H}^2$. Let $\boldsymbol{\alpha}\in \mathbb{A}\times\mathbb{A}$ be a generic admissible strategy and $Q'$ be the associated state process. Due to the concaveness of the terminal penalty, we have
\begin{equation}
    \begin{aligned}
        \mathbb{E}[-A\,&(Q_T)^2 + A\,(Q'_T)^2]\\
        &\geq \mathbb{E}\big[-2 A\, Q_T\,(Q_T-Q'_T)\big]\\
        &=\mathbb{E}\big[Y_T\,(Q_T-Q'_T)\big]\\
        &=\mathbb{E}\Big[\int_0^T2\phi_t\,Q_t\cdot(Q_t-Q'_t)\,dt+\int_0^T(Q_t-Q'_t)\cdot Z_t\,dW_t+\int_0^TY_t\,d(Q_t-Q'_t)\Big]\\
        &=\mathbb{E}\Big[\int_0^T2\phi_t\,Q_t\cdot(Q_t-Q'_t)\,dt+\int_0^TY_t\,d(Q_t-Q'_t)\Big],
    \end{aligned}
    \nonumber
\end{equation}
where we have used the integration by parts and the fact that the stochastic integral is a bona fide martingale. By the definition of the Hamiltonian, one can also obtain
\begin{equation}
    \begin{aligned}
    \mathbb{E}\int_0^T\Big[a_t\,&\delta_t^a\,\Lambda^a(\delta_t^a)+b_t\,\delta_t^b\,\Lambda^b(\delta_t^b)\\
    &\hspace{2cm}-\phi_t\,(Q_t)^2-\big(a_t\,\alpha_t^a\,\Lambda^a(\alpha_t^a)+b_t\,\alpha_t^b\,\Lambda^a(\alpha_t^a)-\phi_t\,(Q'_t)^2\big)\Big]\,dt\\
    &=\mathbb{E}\int_0^T\big[\mathscr{H}(t,Q_t,Y_t,\boldsymbol{\delta}_t)-\mathscr{H}(t,Q'_t,Y_t,\boldsymbol{\alpha}_t)\big]\,dt-\mathbb{E}\int_0^TY_t\,d(Q_t-Q'_t).
    \end{aligned}
    \nonumber
\end{equation}
Combining two results above, finally we can get
\begin{equation}
\begin{aligned}
    J&(\boldsymbol{\delta})-J(\boldsymbol{\alpha})\\
    &=\mathbb{E}\int_0^T\Big[a_t\,\delta_t^a\,\Lambda^a(\delta_t^a)+b_t\,\delta_t^b\,\Lambda^b(\delta_t^b)-\phi_t\,(Q_t)^2\\
    &\hspace{1cm}-\big(a_t\,\alpha_t^a\,\Lambda^a(\alpha_t^a)+b_t\,\alpha_t^b\,\Lambda^b(\alpha_t^b)-\phi_t\,(Q'_t)^2\big)\Big]\,dt+\mathbb{E}\big[-2A\,(Q_T)^2 +2A\, (Q'_T)^2\big]\\
    &\geq\mathbb{E}\int_0^T\big[\mathscr{H}(t,Q_t,Y_t,\boldsymbol{\delta}_t)-\mathscr{H}(t,Q'_t,Y_t,\boldsymbol{\alpha}_t)+2\phi_t\,Q_t\cdot(Q_t-Q'_t)\big]\,dt\\
    &\geq\mathbb{E}\int_0^T\big[\mathscr{H}(t,Q_t,Y_t,\boldsymbol{\alpha}_t)-\mathscr{H}(t,Q'_t,Y_t,\boldsymbol{\alpha}_t)+2\phi_t\,Q_t\cdot(Q_t-Q'_t)\big]\,dt\\
    &=\mathbb{E}\int_0^T\phi_t\,(Q_t-Q'_t)^2\,dt\\
    &\geq 0.
\end{aligned}
    \nonumber
\end{equation} 
\end{proof}

\emergencystretch=2em

\printbibliography
\vspace{1cm}



\end{document}

