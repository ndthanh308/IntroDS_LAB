\documentclass[journal=jacsat,manuscript=article,layout=traditional]{achemso}

\usepackage[version=3]{mhchem} 
\newcommand*\mycommand[1]{\texttt{\emph{#1}}}
\usepackage{amssymb}
\usepackage{graphicx}% Include figure files
\usepackage{dcolumn}% Align table columns on decimal point
\usepackage{bm}% bold math
\usepackage{color}
\usepackage{tabularx}
\usepackage{multirow}




\author{Chiara Guidolin}
\affiliation[BIOMETRA]
{Dipartimento di Biotecnologie Mediche e Medicina Traslazionale, Università
degli Studi di Milano, Via. F.lli Cervi 93, Segrate (MI) I-20090,
Italy}
\author{Christopher Heim}
\affiliation{NanoTemper Technologies GmbH, Munich, Germany}
\author{Nathan B P Adams}
\affiliation{NanoTemper Technologies GmbH, Munich, Germany}
\author{Philipp Baaske}
\affiliation{NanoTemper Technologies GmbH, Munich, Germany}
\author{Valeria Rondelli}
\affiliation[BIOMETRA]
{Dipartimento di Biotecnologie Mediche e Medicina Traslazionale, Università
degli Studi di Milano, Via. F.lli Cervi 93, Segrate (MI) I-20090,
Italy}
\author{Roberto Cerbino}
\affiliation{Faculty of Physics, University of Vienna, Boltzmanngasse 5, Vienna 1090, Austria}
\email{roberto.cerbino@univie.ac.at}
\author{Fabio Giavazzi}
\email{fabio.giavazzi@unimi.it}
\affiliation[BIOMETRA]
{Dipartimento di Biotecnologie Mediche e Medicina Traslazionale, Università
degli Studi di Milano, Via. F.lli Cervi 93, Segrate (MI) I-20090,
Italy}


\title[]
  {Supporting Information}



\def\REV#1{\textcolor{black}{#1}}

\def\REVV#1{\textcolor{red}{#1}}

%\title{Supporting Information}

\begin{document}

\maketitle


\renewcommand\thefigure{S\arabic{figure}}
\setcounter{figure}{0}

\renewcommand\thetable{S\arabic{table}}
\setcounter{table}{0}

\renewcommand{\theequation}{S\arabic{equation}}

\REV{\section{Estimation of the critical signal-to-noise ratio and of the detection limit}}

\REV{
In this section, we derive a simplified expression to calculate the critical signal-to-noise ratio, below which our method may cease to provide reliable sample dynamics estimates.
Here, the signal-to-noise ratio (SNR) is defined as the ratio between the mean square amplitude $A(q)$ (intensity fluctuation linked to the diffusing molecules at a particular wavevector $q$) and the mean square amplitude $B$ of the noise.
In the final part of the section, we also provide practical guidance on estimating the SNR value for a particular combination of experimental parameters. The equations derived in this section serve a dual purpose.
Firstly, they can predict whether DDM will be suitable for characterizing a specific sample under given experimental conditions.
Secondly, they provide recommendations for optimizing the experimental protocol to achieve the necessary performance for detecting a sample of interest.
We start by representing the intensity distribution in each image $I(\mathbf{x},t)$ as the sum of three components,
\begin{equation*} I(\mathbf{x},t)=I_0(\mathbf{x})+S(\mathbf{x},t)+N(\mathbf{x},t)  ,
\end{equation*}
where $I_0(\mathbf{x})$ is a time-independent background, $S(\mathbf{x},t)$ represents the intensity fluctuation associated with diffusing molecules, and $N(\mathbf{x},t)$ is the detection noise.
The Fourier transform of the difference $\Delta I(\mathbf{x},t,\Delta t)$ between two images taken at times $t+\Delta t$ and $t$ respectively, can be written as follows
\begin{equation*}
\Delta \hat{I}(\mathbf{q},t,\Delta t)=\Delta \hat{S}(\mathbf{q},t,\Delta t)+\Delta \hat{N}(\mathbf{q},t,\Delta t).
\end{equation*}
We assume that $\Delta \hat{S}(\mathbf{q},t,\Delta t)$ and $\Delta \hat{N}(\mathbf{q},t,\Delta t)$ are independent complex Gaussian variables with zero mean.
By introducing the image structure function $d(q,t)=\langle |\Delta \hat{I}(\mathbf{q},t,\Delta t)|^2 \rangle $ as per the definition given in the main text, we obtain
\begin{equation*}
d(q,t)=\langle{|\Delta \hat{S}(\mathbf{q},t,\Delta t)|^2}\rangle+
\langle{|\Delta \hat{N}(\mathbf{q},t,\Delta t)|^2}\rangle,
\end{equation*}
where (as per Eq.1 in the main text)
\begin{equation*}
\langle{|\Delta \hat{S}(\mathbf{q},t,\Delta t)|^2}\rangle=A(q) [1-f(q,\Delta t)],
\end{equation*}
and
\begin{equation*}
\langle{|\Delta \hat{N}(\mathbf{q},t,\Delta t)|^2}\rangle=B.
\end{equation*}
We can also calculate $d_2(q,t)=\langle |\Delta \hat{I}(\mathbf{q},t,\Delta t)|^4 \rangle $
\begin{equation*}
 d_2(q,\Delta t)=\langle{|\Delta \hat{S}(\mathbf{q},t,\Delta t)|^4}\rangle+
 \langle{|\Delta \hat{N}(\mathbf{q},t,\Delta t)|^4}\rangle+4\langle{|\Delta \hat{S}(\mathbf{q},t,\Delta t)|^2}\rangle \langle{|\Delta \hat{N}(\mathbf{q},t,\Delta t)|^2}\rangle.
 \end{equation*}
Since the relation $\langle |g|^4 \rangle =2 \langle |g|^2 \rangle^2 $ holds for a generic complex Gaussian variable $g$, the last equation can be rewritten as
\begin{equation*}
 d_2(q,\Delta t)=2A^2(q) [1-f(q,\Delta t)]^2 + 2B^2 +4 A(q) B [1-f(q,\Delta t)]\simeq 2 B^2,
 \end{equation*}
where the last equivalence holds if $A(q) \ll B$, a condition that is always fulfilled  for the measurements presented in this work.
In experiments, the image structure function is estimated as the mean value of $|\Delta \hat{I}(\mathbf{q},t,\Delta t)|^2$ over a finite number $N_{\Delta t}$ of image differences and a finite number $N_q$ of 2-dimensional wavevectors $\mathbf{q}$ corresponding to the same modulus $q$. For example, if the azimuthal average in the Fourier plane is performed by considering ring-shaped regions of width $q_{\text{min}}=2\pi/L$, where $L$ is the image size, the number $N_q$ of independent contributions is given by $N_q \simeq \pi q/q_{\text{min}}$. We can thus estimate the uncertainty $\sigma_d$ associated with the measured structure function as
\begin{equation}
\label{sigmad}
    \sigma_d^2 = \frac{1}{N_{\Delta t}N_{q}}(d_2(q,\Delta t)-d(q,\Delta t)^2)\simeq \frac{B^2}{N_{\Delta t}N_{q}}.
\end{equation}
A simple criterion to predict the ability of DDM to reliably measure the dynamics of the sample at a given $q$ (provided that the acquisition frame rate $f_0$ is large enough to correctly sample the relation dynamics, namely $f_0 \gg Dq^2$) corresponds to the requirement that the expected overall variation over time of the image structure function is at least three times larger than the associated errorbar $\sqrt{2}\sigma_d$,  $A>3\sqrt{2}\sigma_d$. Using Eq. \ref{sigmad}, this condition can be written as
\begin{equation}
\label{ineq}
    \text{SNR} > \text{SNR}_{\text{c}},
\end{equation}
where 
\begin{equation}
\label{SNRC}
    \text{SNR}_{\text{c}}=\frac{3\sqrt{2}}{\sqrt{N_{\Delta t}N_{q}}}     
\end{equation}
is the critical signal-to-noise ratio. We can use Eq. \ref{SNRC} to estimate $\text{SNR}_\text{c}$ for the measurements presented in this article using $N_{\Delta t} \simeq 2 \cdot 10^4$,
$N_{q}\simeq 3 \cdot 10^2$ for $q=1$ $\mu$m. We get  $\text{SNR}_{\text{c}} \simeq 1\cdot 10^{-3}$, a value that is rather close to the signal-to-noise ratio corresponding to the lowest measured concentrations for all the proteins considered in this work.
Let us now consider the key factors that determine the values of $A$ and $B$ and therefore of $\text{SNR}$.
In a typical working condition for spatially coherent bright-field imaging, a strong transmitted beam is present, and the optical signal in the images is mainly due to the interference between the transmitted field and the field scattered by  the molecules \cite{giavazzi2009scattering}. In this condition, the mean square amplitude of the signal $A$ is proportional to the square of the average intensity (grayscale value) $I_0$ recorded by the camera sensor. 
On the other hand, since the main contribution to the noise comes from the shot noise, \textit{i.e.}, from the Poissonian fluctuation in the number of collected photoelectrons, the noise mean square amplitude $B$ is expected to scale linearly with  $I_0$, with a prefactor that depends on the full well capacity $\text{FWC}$ of the sensor pixel (in our case corresponding to about 30000 electrons per pixel)
\begin{equation*}
B=I_\text{max}/\text{FWC} \cdot I_0.    
\end{equation*}
In the last equation, $I_\text{max}$ is the grayscale value corresponding to saturation.
In our measurements, the average grayscale value was about one-third of the saturation value $I_0\simeq I_\text{max}/3$, as can be seen from the intensity histograms reported in Figure \ref{fig:Histograms_raw}.\\
Let us now consider a suspension of diluted molecules of molecular weight $M$ at concentration $c$.
If we assume fixed values for the density and for d$n$/d$c$, in the Rayleigh regime, the scattered intensity is proportional to the product $c\cdot M$.
Another important factor that affects the amplitude $A(q)$, and which is not easy to accurately predict \textit{a priori}, is the optical transfer function $T(q)$ of the imaging system.
The optical transfer function depends on the coherence of the illumination beam and the numerical aperture of the collection optics and incorporates also the effect of the finite thickness $h$ of the sample, which can be relevant in particular for low $q$\cite{giavazzi2009scattering}.
Combining all the aforementioned contributions, we obtain the following expression for the signal-to-noise ratio
\begin{equation}
\label{SNR}
\text{SNR}\approx a \cdot (\text{FWC}\cdot I_0/I_\text{max})\cdot c \cdot M,
\end{equation}
where the prefactor $a$ depends on the optical transfer function and the optical contrast (d$n$/d$c$) of the molecules. For our setup, $a$ can be estimated by measuring $\text{SNR}$ for a given combination of experimental parameters.
Let us consider, for example, the most diluted BSA sample ($c=1$ mg mL$^{-1}$, $M=6.6\cdot10^4$ Da). In this case, for $q=0.7$ $\mu$m$^{-1}$, we obtain $\textit{SNR}\simeq 2 \cdot 10^{-3}$ (see also Figure 3(a)). This provides the estimate $a\approx 3\cdot 10^{-12}$ mL mg$^{-1}$Da$^{-1}$.
Although this value is not expected to be of general validity, it could provide an order-of-magnitude reference for experiments performed under similar conditions.
The combination of Eqs. \ref{ineq}, \ref{SNRC}, and \ref{SNR} represents the main result of this section, as it specifies the role of all relevant experimental parameters in determining the feasibility of the measurement and enables the prediction of DDM performance under prescribed conditions.}


\section{\REV{Supplementary Movies}}
\REV{For each protein, we provide two movies showing an example of image acquisition at the highest ($c_1$) and the lowest ($c_4$) concentration investigated, as listed below.
Each movie shows the first 500 frames of the image acquisition, before (top) and after (bottom) background removal. Frames are displayed at a playback speed of 50 fps.
The corresponding intensity histograms are shown in Figures \ref{fig:Histograms_raw} and \ref{fig:Histograms_BGrmvd}.\\
List of supplementary movies:\\
- \textbf{SM1}: BSA sample at concentration $c=c_1$\\
- \textbf{SM2}: BSA sample at concentration $c=c_4$\\
- \textbf{SM3}: Lysozyme sample at concentration $c=c_1$\\
- \textbf{SM4}: Lysozyme sample at concentration $c=c_4$\\
- \textbf{SM5}: Pepsin sample at concentration $c=c_1$\\
- \textbf{SM6}: Pepsin sample at concentration $c=c_4$\\}

% Figure environment removed

% Figure environment removed



\section{\REV{Supplementary Data}}
\REV{For each protein solution we report in Table \ref{tab:ExpConc} the concentration values measured with the UV spectrophotometer, together with the extinction coefficients used for their calculation. The hydrodynamic radii reported in the literature for the proteins under study, as well as the values obtained from DLS measurements, are summarised in Table \ref{tab:ProteinSizeDLS}.
The corresponding protein size distributions are shown in Figure \ref{fig:DLS_dist}.
For each sample, we furthermore report in Figures \ref{fig:BSA_A1}-\ref{fig:LYSO_C4} examples of intermediate scattering functions at different wavevectors, as well as the amplitude, noise and relaxation rate values obtained from the fit.}\\

\begin{table}
\centering
\begin{tabular}{cccccc} 
 Protein & $c_1$ & $c_2$ & $c_3$ & $c_4$ & $\varepsilon$\\ 
 \hline
 BSA & 34 $\pm$ 2 & 11.7 $\pm$ 0.6 & 5.8 $\pm$ 0.3 & 1.17 $\pm$ 0.06 & 43824\\
 Pepsin & 23 $\pm$ 1 & 11.5 $\pm$ 0.6 & 3.9 $\pm$ 0.2 & 1.55 $\pm$ 0.08 & 52920\\
 Lysozyme & 28 $\pm$ 1 & 11.7 $\pm$ 0.6 & 8.3 $\pm$ 0.4 & 4.8 $\pm$ 0.2 & 36000\\ 
\end{tabular}
\caption{\textbf{Protein concentrations and extinction coefficients.} Concentration values are expressed in mg/mL. The extinction coefficients $\varepsilon$ are expressed in M$^{-1}$ cm$^{-1}$.}
\label{tab:ExpConc}
\end{table}

% measured values:
%\begin{table}
%\centering
%\begin{tabular}{cccccc} 
 %Protein & $c_1$ & $c_2$ & $c_3$ & $c_4$ & $\varepsilon$\\ 
 %\hline
 %BSA & 33.52 & 11.67 & 5.79 & 1.28 & 43824\\
 %Pepsin & 23.18 & 11.49 & 3.92 & 1.55 & 52920\\
 %Lysozyme & 28.37 & 11.72 & 8.27 & 4.80 & 36000\\ 
%\end{tabular}
%\caption{\textbf{Protein concentrations and extinction coefficients.} Concentration values are expressed in mg/mL. The extinction coefficients $\varepsilon$ are expressed in M$^{-1}$ cm$^{-1}$.}
%\label{tab:ExpConc}
%\end{table}



\begin{table}
\centering
\begin{tabular}{cccc} 
 Protein &  $R_h$ (literature) [nm] & $c$ & $R_h$ (DLS) [nm]\\ 
\hline
\multirow{4}{*}{BSA} & \multirow{4}{*}{3.3 - 4.3 \cite{jachimska2008characterization}} & $c_1$ &  3.03 $\pm$ 0.02\\% 3.0 - 3.5\\
%BSA & 3.3 - 4.3 & $c_1$ &  3.03 $\pm$ 0.02\\% 3.0 - 3.5\\
& & $c_2$ & 3.28 $\pm$ 0.01 \\
 & & $c_3$ & 3.37 $\pm$ 0.01 \\
 & & $c_4$ & 3.49 $\pm$ 0.01\\
 \hline
 \multirow{4}{*}{Pepsin} & \multirow{4}{*}{3.04 \cite{gtari2017impact}} & $c_1$ &  3.11 $\pm$ 0.03\\% 3.0 - 3.5\\
 & & $c_2$ & 3.29 $\pm$ 0.03 \\
 & & $c_3$ & 3.46 $\pm$ 0.07 \\
 & & $c_4$ & 3.51 $\pm$ 0.06\\
 \hline
 \multirow{4}{*}{Lysozyme} & \multirow{4}{*}{2.05 \cite{wilkins1999hydrodynamic}} & $c_1$ & 2.13 $\pm$ 0.01\\% 2.0 - 2.1\\
 & & $c_2$ & 2.04 $\pm$ 0.02 \\
 & & $c_3$ & 2.01 $\pm$ 0.01 \\
 & & $c_4$ & 1.96 $\pm$ 0.01\\
 \hline
\end{tabular}
\caption{\textbf{Protein size.} Mean hydrodynamic radii estimated with DLS for the three different proteins.}
\label{tab:ProteinSizeDLS}
\end{table}

\newpage
% Figure environment removed













% BSA ----------------------------------------------

% sample A1
% Figure environment removed


%% Figure environment removed


% sample A2
% Figure environment removed



% sample A3
% Figure environment removed


%% sample A4 (original, with double decay)
%% Figure environment removed


% sample A2/10 new point rescaled with the noise and with new A3 amplitude
% Figure environment removed


%% Figure environment removed



%%% PEPSIN ---------------------------

% sample B1
% Figure environment removed



% sample B2
% Figure environment removed

% sample B3
% Figure environment removed

% sample B4
% Figure environment removed






%%% LYSOZYME ---------------------------

% sample C1
% Figure environment removed


% sample c2
% Figure environment removed


% sample c3
% Figure environment removed


% sample c4
% Figure environment removed



\clearpage

\bibliography{bibl}


\end{document}