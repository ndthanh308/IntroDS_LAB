%% LyX 2.3.2 created this file.  For more info, see http://www.lyx.org/.
%% Do not edit unless you really know what you are doing.
\documentclass[10pt]{amsart}

\oddsidemargin0.25in
\evensidemargin0.25in
\textwidth6.00in
\topmargin0.00in
\textheight8.50in

\renewcommand{\baselinestretch}{1.10}

\newcommand{\indentalign}{\hspace{0.3in}&\hspace{-0.3in}}


\usepackage[latin9]{inputenc}
\pagestyle{headings}
% \setcounter{secnumdepth}{1}
\usepackage{color}
\usepackage{amstext}
\usepackage{amsthm}
\usepackage{amssymb}
\usepackage[numbers,sort]{natbib}
\PassOptionsToPackage{normalem}{ulem}
\usepackage{ulem}
\usepackage[unicode=true,
 bookmarks=false,
 breaklinks=false,pdfborder={0 0 1},backref=section,colorlinks=true]
 {hyperref}
\hypersetup{
 linkcolor=red,citecolor=blue,filecolor=dullmagenta,urlcolor=blue}

\makeatletter
%%%%%%%%%%%%%%%%%%%%%%%%%%%%%% Textclass specific LaTeX commands.
\numberwithin{equation}{section}
\numberwithin{figure}{section}

%%%%%%%%%%%%%%%%%%%%%%%%%%%%%% User specified LaTeX commands.





\newtheorem{thm}{Theorem}[section]
\newtheorem{lemma}[thm]{Lemma}
\newtheorem{prop}[thm]{Proposition}
\newtheorem{cor}[thm]{Corollary}
\newtheorem{defn}[thm]{Definition}
\newtheorem{rem}[thm]{Remark}
\newtheorem{ass}[thm]{Assumption}

\numberwithin{equation}{section}





\newcommand{\les}{\lesssim}
\newcommand{\lam}{{\lambda}}
\newcommand{\Lam}{{\Lambda}}
\newcommand{\gam}{{\gamma}}
\newcommand{\Gam}{{\Gamma}}
\newcommand{\vp}{{\varphi}}
\newcommand{\ve}{{\varepsilon}}
\newcommand{\de}{{\delta}}
\newcommand{\De}{{\Delta}}
\newcommand{\al}{{\alpha}}
\newcommand{\ka}{{\kappa}}
\newcommand{\ze}{{\zeta}}

\newcommand{\R}{{\mathbb R}}
\newcommand{\Z}{{\mathbb Z}}

\newcommand{\rt}{\mathbb{R}^3}
\newcommand{\brad}{\langle D\rangle}
\newcommand{\braxi}{\langle\xi\rangle}
\newcommand{\brat}{\langle t \rangle}
\newcommand{\barpsi}{\overline{\psi}}

\newcommand{\phase}{{\mathbf{p}_\lam}}
\newcommand{\phasep}{\mathbf p_\lam}
\newcommand{\phaseq}{{\mathbf q_\lam(\xi,\eta,\sigma)}}
\newcommand{\jsev}{{J_{\lam,\textbf{N}}^7}}


\newcommand{\thez}{{\theta_0}}
\newcommand{\theo}{{\theta_1}}
\newcommand{\thet}{{\theta_2}}
\newcommand{\theth}{{\theta_3}}
\newcommand{\thef}{{\theta_4}}
\newcommand{\thej}{{\theta_j}}

\newcommand{\supp}{{\left(  \textbf{L}, \textbf{N}\right)}}


\def\normo#1{\left\|#1\right\|}
\def\normb#1{\Big\|#1\Big\|}
\def\norm#1{\|#1\|}
\def\abs#1{\left|#1\right|}
\def\bra#1{{\langle#1\rangle}}
\def\bigbra#1{\Big\langle#1\Big\rangle}
\def\wb#1{\overline{#1}}
\def\wt#1{\widetilde{#1}}
\def\wh#1{\widehat{#1}}
\def\set#1{\left\{#1\right\}}


\def\inho#1{\left(1+ |#1|\right)}
\def\normcm#1{\left\|#1\right\|_{\rm CM}}




\global\long\def\R{\mathbf{\mathbb{R}}}%
\global\long\def\C{\mathbf{\mathbb{C}}}%
\global\long\def\Z{\mathbf{\mathbb{Z}}}%
\global\long\def\N{\mathbf{\mathbb{N}}}%
\global\long\def\D{\mathbf{D}}%
\global\long\def\Im{\mathrm{Im}}%
\global\long\def\Re{\mathrm{Re}}%
\global\long\def\rad{\mathrm{rad}}%
\global\long\def\tilde#1{\widetilde{#1}}%
\global\long\def\H{\mathcal{H}}%
\global\long\def\env{\mathrm{Env}}%
\global\long\def\re{\mathrm{re}}%
\global\long\def\im{\mathrm{im}}%
\global\long\def\err{\mathrm{Err}}%
\global\long\def\d{\partial}%
\global\long\def\jp#1{\langle#1\rangle}%
\global\long\def\norm#1{\|#1\|}%
\global\long\def\ol#1{\overline{#1}}%
\global\long\def\wt#1{\widetilde{#1}}%
\global\long\def\tilde#1{\widetilde{#1}}%
\global\long\def\br#1{(#1)}%
\global\long\def\Bb#1{\Big(#1\Big)}%
\global\long\def\bb#1{\big(#1\big)}%
\global\long\def\lr#1{\left(#1\right)}%
\global\long\def\ve{\varepsilon}%
\global\long\def\la{\lambda}%
\global\long\def\al{\alpha}%
\global\long\def\be{\beta}%
\global\long\def\ga{\gamma}%
\global\long\def\La{\Lambda}%
\global\long\def\De{\Delta}%
\global\long\def\na{\nabla}%
\global\long\def\ep{\epsilon}%
\global\long\def\fl{\flat}%
\global\long\def\sh{\sharp}%

\makeatother

\begin{document}
\title[The modified scattering on 2d semi-relativistic equations]{The modified scattering of 2 dimensional semi-relativistic Hartree equations}
\author{Soonsik Kwon}
\address{Department of Mathematical Sciences, Korea Advanced Institute of Science
and Technology, 291 Daehak-ro, Yuseong-gu, Daejeon 34141, Republic
of Korea}
\email{soonsikk@kaist.edu}
\author{Kiyeon Lee}
\address{Stochastic Analysis and Application Research Center(SAARC), Korea
Advanced Institute of Science and Technology, 291 Daehak-ro, Yuseong-gu,
Daejeon, 34141, Republic of Korea}
\email{kiyeonlee@kaist.ac.kr}
\author{Changhun Yang}
\address{Department of Mathematics, Chungbuk National University, Chungdae-ro1,
Seowon-gu, Cheongju-si, Chungcheongbuk-do, Republic of Korea}
\email{chyang@chungbuk.ac.kr}
\begin{abstract}
In this paper, we consider the asymptotic behaviors of small solutions
to the semi-relativistic Hartree equations in two dimension. The nonlinear
term is convolved with the Coulomb potential $|x|^{-1}$, and it produces
the\emph{ long-range interaction} in the sense of scattering phenomenon.
From this observation, one anticipates that small solutions converge
to a modified scattering states, although they decay as linear solutions.
We show the global well-posedness and the modified scattering for
small solutions in weighted Sobolev spaces. Our proof follows a road
map of exploiting the space-time resonance by \cite{gemasha2008,pusa}.
Compared to the result in three dimensional case \cite{pusa}, weaker
time decay in two dimension is one of the main obstacles.
\end{abstract}

\maketitle

\section{Introduction}
% The subject of this paper is the global well-posedness of 2 dimensional semi-relativistic Hartree equations and an asymptotic behavior of the global solution. In order to handle the nonlinearity that possesses long-range interaction, we give the phase modification which plays a role of getting rid of the resonant interaction and this phase modification yields the modified scattering results. Compared to the 3 dimensional problem \cite{pusa}, weaker time decay leads us main obstacles. To obtain the extra time decay, we exploit the space-time resonant argument. See Sections \ref{sec:main-idea} and \ref{sec:Weighted-Energy-estimate} for the details.
\subsection{The equation and previous results}
We consider the semi-relativistic Hartree equations
with {\em Coulomb} potential
\begin{align}
\left\{ \begin{aligned}-i\partial_{t}u+\sqrt{m^{2}-\Delta}u & =\lam\left(|x|^{-1}*|u|^{2}\right)u\qquad &&\mathrm{in}\;\;\R\times\mathbb{R}^{d},\\
u(0,\cdot) & =u_{0} &&\mathrm{in}\;\;\mathbb{R}^{d},
\end{aligned}
\right.\label{main-eq:semi}
\end{align}
where the unknown $u:\mathbb{R}^{1+d}\to\mathbb{C}$ and some fixed
constant $\lam\in\R$. The nonlocal differential operator $\sqrt{m^{2}-\Delta}$
is defined as a Fourier multiplier operator associated to the symbol
$\sqrt{m^{2}+|\xi|^{2}}$ and $\ast$ denotes the convolution in $\mathbb{R}^d$. Here we consider the mass parameter
$m>0$, so we normalize $m=1$ throughout the paper. For three dimensional case $d=3$, \eqref{main-eq:semi} is often referred to as \textit{Boson star equation} which describes the dynamics and collapse of relativistic Boson stars. It was rigorously derived as the mean-field limit of large systems of bosons. See \cite{Michelangeli2012,frjonlenz2007-nonlinearity,Lieb1987} and references therein. 
% In this paper we focus on two dimensional case $d=2$ and study asymptotic behaviors of solutions.

The mass and energy of solution to \eqref{main-eq:semi} are defined by  
\begin{align}
    M(u)(t)&= \|u(t)\|_{L^2(\R^d)}, \label{mass conservation}\\ 
    E(u)(t)&= \frac12\int_{\R^d}\overline{u}\sqrt{1-\Delta} u dx 
    + \frac{\lambda}{4}\int_{\R^d}\left(|x|^{-1}\ast |u|^2\right) |u|^2 dx \nonumber
\end{align}
respectively and conserved as time evolves. From the conservation laws, one sees that $H^\frac12$ is the energy space. Furthermore,
in massless case $m=0$, we have the scaling symmetry. If $u$ is solution to \eqref{main-eq:semi}, $u_\alpha$ defined by 
$u_\alpha(t,x)=\alpha^{\frac{d}{2}}u(\alpha t,\alpha x)$ are also solutions and the mass is invariant under the scaling, i.e., $\|u(t)\|_{L^2(\R^d)}=\|u_\alpha(t)\|_{L^2(\R^d)}$, thus \eqref{main-eq:semi} is $L^2$-critical.


% The semi-relativistic Hartree equation is written as
% \begin{align}
% -i\partial_{t}u+\sqrt{m^{2}-\Delta}u=\lam\left(V_{\mu}*|u|^{2}\right)u\label{eq:g-semi-relativistic}
% \end{align}
% where\textcolor{red}{{} the potential 
% \begin{align}
% V_{\mu}(x)=\int_{0}^{\infty}e^{-\mu^{2}r-{|x|^{2}}/{4r}}\frac{dr}{r}\sim\left\{ \begin{aligned}\frac{e^{-\mu|x|}}{|\mu x|^{\frac{1}{2}}} & \qquad|x|\gtrsim1,\\
% -\log|x| & \qquad|x|\ll1.
% \end{aligned}
% \right.\label{eq:g-semi-potential}
% \end{align}
% }When $\mu=0$, we refer to that $V_{0}$ is {\em Coulomb} potential
% and when $\mu>0$, we refer to that $V_{\mu}$ is {\em Yukawa}
% potential. Semi-relativistic equation \eqref{eq:g-semi-relativistic}
% is a model to describe the dynamic and collapse of relativistic Boson
% stars. It is often referred to as {\em the Boson star equation}.
% Our main equation \eqref{main-eq:semi} corresponds to the case $V_{0}$
% of \eqref{eq:g-semi-relativistic}.

There are numerous local and global well-posedness results for the semi-relativistic
Hartree equations \eqref{main-eq:semi}. 
A first result was obtained in \cite{lenz2007} where the local well-posedness in $H^{s}(\R^{3})$ for $s\ge\frac{1}{2}$ and global well-posedness in the energy space $H^{\frac12}(\R^{3})$ were established. This result was extended to other dimensions $d\ge2$ in \cite{choz2006-siam}. Also, the authors in \cite{choz2006-siam} established the low regularity well-posedness below the energy space, more precisely, the local well-posedness in $H^s(\R^d)$ for $s>\frac12-\frac{d-1}{4d}$.
This result was later improved in \cite{hele2014} and \cite{lee2021-bkms} where the 
local well-posendess in $H^s(\R^3)$ for $s\ge\frac14$ and $H^s(\R^2)$ for $s>\frac38$ were proved, respectively. 

% extend the result to global in the energy space the global well-posedness for $H^{\frac{1}{2}}(\R^{3})$
% with sufficiently small initial data. The local well-posedness result
% was improved up to \emph{almost optimal regularity} $H^{s}(\R^{3})$
% with $s>\frac{1}{4}$ ($s>0$, when radial symmetric initial data)
% by Herr and Lenzmann \citet{hele2014} via Bilinear estimates and
% $X^{s,b}$-spaces. In two dimensional case, we refer to \citet{lee2021-bkms}.
% As a corollary of theorems in \citet{lee2021-bkms}, we obtain the
% local well-posedness for $H^{s}(\R^{2})$ with $s>\frac{3}{8}$ and
% a lack of uniform continuity of the flow map below $L_{x}^{2}(\R^{2})$.


The aim of this paper is to study the asymptotic behaviors of solutions to \eqref{main-eq:semi} when $d=2$. By a \textit{scattering}, we mean a solution to nonlinear PDEs converges to a solution of the linear equation as time goes to infinity. This phenomenon has been observed in various dispersive equations. Concerning our equation,  let us consider the following generalized model 
\begin{align}\label{eq:gamma}
-i\partial_{t}u+\sqrt{1-\Delta}u =\lam\left(|x|^{-\gamma}*|u|^{2}\right)u,\quad 0<\gamma<d, \quad \mathrm{in}\;\;\R\times\mathbb{R}^{d}.
\end{align}
The asymptotic behavior of solutions to \eqref{eq:gamma} varies depending on the potential, i.e., the range of $\gamma$. To see this, 
by Duhamel's principle, we write \eqref{eq:gamma} as the integral equation
\begin{align*}
u(t) = e^{it\sqrt{1-\Delta}}u_0 + \lam\int_0^t e^{i(t-s)\sqrt{m^2-\Delta}} \left(|x|^{-\gamma}*|u(s)|^{2}\right)u(s)ds.
\end{align*}
Observe that if \eqref{eq:gamma} scatters a linear profile, it would be defined as
\begin{align*}
    u_0+\lim_{t\rightarrow\infty}
    \lam\int_0^t e^{-is\sqrt{1-\Delta}} \left(|x|^{-\gamma}*|u(s)|^{2}\right)u(s)ds.
\end{align*}
By using the well-known time decay estimates of linear solution (see \cite[Lemma~3]{MNO2003})
\begin{align}\label{standard time decay}
    \| e^{it\sqrt{1-\Delta}} u_0 \|_{L^\infty(\R^d)} \les \langle t\rangle^{-\frac d2} \quad \text{for} \quad u_0\in C_0^\infty(\R^d),
\end{align}
one verifies that the time decay of $L^2$ norm of the nonlinearity, computed on a solution to the linear equation, is $t^{-\gamma}$  \footnote{We refer to \cite[Section~4]{choz2006-siam} for the precise statement and detailed proof.}
\begin{align*}
    \normo{(|x|^{-\gamma}*|e^{it\sqrt{1-\Delta}} u_0|^{2})\,e^{it\sqrt{1-\Delta}} u_0}_{L^{2}(\R^d)}\sim |t|^{-\gamma} \quad \text{ for } |t|\gg 1.
\end{align*}
Thus, one may infer that there can not exist a linear profile if $0<\gamma\le1$ with which 
the nonlinearity  is called a \textit{long-range interaction}. Indeed, 
the authors in \cite{choz2006-siam} proved that \eqref{eq:gamma} failed to scatter 
when $0<\gamma\le1$  for  $d\ge3$  and $0<\gamma<\tfrac{d}{2}$  for  $d=1$ or $2$.
On the other hand, for the case $1<\gamma<d$ which is called a \textit{short-range interaction} we may expect the scattering. The first scattering result for the case $2<\gamma<d$ and $d\ge3$ was obtained in \cite{choz2006-siam} and the gap corresponding to $1<\gamma\le 2$ was later closed in \cite{pusa} for $d=3$ and in \cite{hanaog2015-die} for $d\ge3$. 
Recently, scattering result for two dimensional case when $1<\gamma<2$ was established in \cite{YCH}.
Lastly, we refer to \cite{choz2007-jkms,choz2008-dcds-s,chozhishim2009-dcds} for related works.


% In the appendix, we provide the scattering results for the case $d=2$ and $1<\gamma<2$ as a byproduct of main results.

Now, let us focus on the case $\gamma=1$, which corresponds to our main equation. We refer to this as the ``scattering-critical'' case, 
because the time integration barely fails to integrable, or diverges logarithmically. 
We generally anticipate a \textit{modified scattering} result for solutions
to equations which have the scattering critical nonlinearity. The
modified scattering means that a global solution decays as linear
solutions, but converges to a linear solution \emph{with a suitable 
correction} (eg. a phase modification). In the area of nonlinear dispersive
equations, the first modified scattering results was established in \cite{ozawa1991} for one dimensional cubic nonlinear Schr\"odinger equations (NLS). This result was extended to higher
dimension in \cite{hayashi-naumkin1998} and the authors also proved the modified scattering for NLS with Hartree nonlinear terms for $d\ge2$
\begin{align}\label{eq:schrodinger}
    -i\partial_{t}u+\Delta u =\lam\left(|x|^{-1}*|u|^{2}\right)u,\quad \mathrm{in}\;\;\R\times\mathbb{R}^{d}.
\end{align}
Later, in \cite{kapu}, the authors reproved the modified scattering for \eqref{eq:schrodinger}, the same equations addressed in \cite{hayashi-naumkin1998}, by the different technique called \textit{space-time resonance argument} which was introduced in \cite{gemasha2008,gemasha2012-jmpa,gemasha2012-annals}. 
We should mention that the algebraic structure of Schr\"odinger symbol plays a crucial role in their proof. 
Concerning our equation \eqref{main-eq:semi} where the linear operator is nonlocal, the structure of resonance is more involved, so we have to induce a different asymptotic behavior of a solution. Also, relatively higher regularity assumption on initial data is required.

The modified scattering result of \eqref{main-eq:semi} for three dimensional case was proved in \cite{pusa}. We also refer to \cite{hayashi-naumkin2017-henri,iopu2014,sautwang2021-compde} 
where the nonlinear equation with non-local differential operator was studied.
Inspired by work \cite{pusa}, we investigate an asymptotic behavior of solution to \eqref{main-eq:semi} when $d=2$.


% \color{red}
% Then   \cite{pusa} in which
% three dimensional asymptotic behavior of solution to \eqref{main-eq:semi}
% is considered, we also expect the modified scattering of solution
% to \eqref{main-eq:semi} on two dimension.

% \color{black}


% We are concerned with asymptotic behaviors of solutions, that is,
% the scattering. Here is a simple observation of the scattering phenomenon.
% In view of time decay \eqref{global-bound:semi}, we may consider
% the decay of nonlinearity $N(u(t))=(|x|^{-\gamma}*|u(t)|^{2})u(t)$
% as follows: 
% \begin{align}
% \normo{(|x|^{-\gamma}*|u(t)|^{2})u(t)}_{L_{x}^{2}}\les\||x|^{-\gamma}*|u(t)|^{2}\|_{L_{x}^{\infty}}\normo{u(t)}_{L_{x}^{2}}\les\bra{t}^{-\gamma}\|u_{0}\|_{L_{x}^{2}}^{3-\gamma}.\label{eq:scattering-range}
% \end{align}
% Then $L_{x}^{2}$-norm of nonlinearity is integrable in time when
% $1<\gamma<d$. In this case, the nonlinearity $N(u(t))$ is refered
% to as a {\em short-range} interaction. On the contrary, we call
% the nonlinearity a {\em long-range} interaction when $0<\gamma\le1$.



% There also have been the scattering results of our equation \eqref{main-eq:semi}.
% In \citet{choz2006-siam,choz2007-jkms,choz2008-dcds-s,chozhishim2009-dcds},
% the authors showed linear scattering result for {\em short-range}
% nonlinearity $V=|x|^{-\gamma}$ with $1<\gamma<d$ on dimension $d\ge2$.
% Pusateri \citet{pusa} proved the modified scattering result for the
% Coulomb potential $V=|x|^{-1}$ that becomes scattering critical nonlinearity
% in three dimensions. In \citet{choz2006-siam}, the authors showed
% that a linear scattering phenomenon of our main equation \eqref{main-eq:semi}
% fails in $L_{x}^{2}(\R^{2})$.\footnote{More precisely, In \citet{choz2006-siam}, the authors worked on three
% dimensional case. But, thier argument work here. See Theorem 4.1 of
% \citet{choz2006-siam}.}

\medskip 

\subsection{Main results and ideas}\label{sec:main-idea}
We now state our main theorem for the two dimensional semi-relativistic Hartree equations \eqref{main-eq:semi}:

\begin{thm} \label{main-thm:semi} Let $n\ge1000$ and $k=\frac{n}{100}$.
There exists $\overline{\ve_{0}}>0$ satisfying the following:

Suppose that the initial data $u_{0}$ is sufficiently small in a weighted space.
In other words, for any $\ve_{0}\le\overline{\ve_{0}}$, $u_{0}$
satisfies 
\begin{align}
\|u_{0}\|_{H^{n}(\R^2)}+\|\langle x\rangle^{2}u_{0}\|_{H^{2}(\R^2)}+\|\jp{\xi}^{k}\widehat{u_{0}}\|_{L^{\infty}(\R^2)}\le\ve_{0}.\label{condition-initial:semi}
\end{align}
Then the Cauchy problem \eqref{main-eq:semi} with the initial data $u_{0}$
has a unique global solution $u$ to \eqref{main-eq:semi} decaying
as 
\begin{align}
\|u(t)\|_{L^{\infty}(\R^2)}\les\ve_{0}\bra{t}^{-1}.\label{global-bound:semi}
\end{align}
Moreover, there exists a scattering profile $u_{\infty}$ such that
\begin{align}
\left\Vert \bra{\xi}^{k}\mathcal{F}\left[u(t)-e^{iB(t,D)}e^{-it\bra{D}}u_{\infty}\right]\right\Vert _{L_{\xi}^{\infty}(\R^2)}\les\ve_{0}\bra{t}^{-\de},\label{eq:modified-scattering}
\end{align}
for some $0<\de< \frac{1}{100}$. 
Here, the phase modification is defined by 
\begin{align}\label{formula of B}
    B(t,\xi)=\frac{\lam}{(2\pi)^{2}}\int_{0}^{t}\left( \int_{\R^{2}}\left|\frac{\xi}{\bra{\xi}}-\frac{\sigma}{\bra{\sigma}}\right|^{-1}|\wh{u}(\sigma)|^{2}d\sigma \right) \frac{\rho(s^{-\frac{2}{n}}\xi)}{\bra{s}}ds,    
\end{align}
where $\rho$ is a smooth compactly supported function.
\end{thm}
\begin{rem}
We do not pursue to optimize the regularity
indices $n$ and $k$  and time decay $\delta>0$ in Theorem \ref{main-thm:semi}.
% Throughout this paper, we fix 
% % \begin{equation}
% % n\ge1000,\qquad k=\frac{n}{100}.\label{eq:definition n k}
% % \end{equation}
\end{rem}
\begin{rem}
We prefer to express the formula of phase modification \eqref{formula of B} in the fourier space because it can be seen not only from some heuristic consideration (see \eqref{HD} below) but also in our rigorous proof.

Furthermore, we observe that convergence in the weighted $L^\infty$ norm in \eqref{global-bound:semi} immediately implies the convergence in $L^2$ spaces.
\end{rem}
\begin{rem}
The time decay rate of solutions in \eqref{global-bound:semi} is optimal in the sense that 
the nonlinear solutions decay as the same rate with the linear ones \eqref{standard time decay}.
\end{rem}

Theorem \ref{main-thm:semi} contains the global existence and asymptotic
behaviors of small solutions to \eqref{main-eq:semi}. Our proof to obtain the global existence of solutions is basically based on the bootstrap argument in a weighted Sobolev space, and then the next crucial part is to perform a suitable phase correction and find a modified scattering state.
Briefly, the proof of Theorem \ref{main-thm:semi} consists of threefold.
First, we find the time decay of solutions to \eqref{main-eq:semi}, from which we construct a function space which consists of the weighted energy norm and scattering norm. Then, the second step is to show that the small solutions stay small as long as they exist by performing the weighted energy estimates. Our strategy is based on the method of space-time resonance which was introduced in \cite{gemasha2008,gemasha2012-jmpa,gemasha2012-annals,pusa}. 
The final step is to obtain the bound for the scattering norm in the function space.  It is in this step that a suitable correction of the phase based on the Taylor expansion is required to close the bootstrap argument.
Collecting all from the three steps, we can finally obtain the modified scattering results for \eqref{main-eq:semi}.

Let us explain in detail the ideas of proof in each step.
In the first step, we use the standard stationary phase method on  oscillatory integral to derive the time decay of linear solutions, $t^{-1}$. 
By a direct proof, without resorting to well-known $L^p-L^q$ estimates (e.g \cite[Lemma~3]{MNO2003}), we manage to obtain the time decay of solutions up to higher ($k$ th) order derivative, which is essential in the course of weighted energy estimates to overcome the lack of time decay compared to higher dimensional cases. 
To fully utilize the time decay of solutions, we construct our solution space based on the weighted $L^2$-norms. 


In the second step, we show that the small nonlinear solutions stay small during the existing time by performing the weighted energy estimates. We introduce the interaction representation of solutions $u(t)$ so as to track the scattering states 
\begin{align}
f(t,x):=e^{it\langle D\rangle}u(t,x).\label{eq:interation}
\end{align}
Then we can express $f$ via Duhamel's representation
\begin{align}
\begin{aligned}\widehat{f}(t,\xi) & =\widehat{u_{0}}(\xi)+i\lam\mathcal{I}(t,\xi),\\
\mathcal{I}(t,\xi) & = \frac{1}{2\pi}\int_{0}^{t}\int_{\mathbb{R}^{2}\times\mathbb{R}^{2}}e^{is\phi(\xi,\eta)}|\eta|^{-1}\widehat{|u|^2}(\eta)\wh{f}(s,\xi-\eta)d\eta ds
\end{aligned}
\label{eq:duhamel}
\end{align}
with the resonance function 
\begin{align}
\phi(\xi,\eta)=\braxi-\langle\xi-\eta\rangle.\label{eq:resonance-ftn}
\end{align}
In the course of weighted energy estimates, we should bound the $xf$ and $x^2f$ in $L^2$ which are converted to $\nabla \widehat{f}$ and $\nabla^2 \widehat{f}$ in the fourier space, respectively.
The main task is to not only bound the singularity $|\eta|^{-1}$, but also recover the time growth resulting from the derivative $\nabla_\xi$ taken to exponential term. 
Indeed, the most delicate term occurs from $\nabla^2 \widehat{f}$ when two derivatives both fall on $e^{is\phi(\xi,\eta)}$
\begin{align}\label{delicate term}
    \frac{1}{2\pi}\int_{0}^{t}\int_{\mathbb{R}^{2}}s^2\big( \nabla_\xi \phi(\xi,\eta) \big)^2 e^{is\phi(\xi,\eta)}|\eta|^{-1}\widehat{|u(s)|^2}(\eta)\wh{f}(s,\xi-\eta) d\eta ds,
\end{align}
where we have to compensate the time growth $s^{2}$. Here, we encounter the main difficulty from two dimensional nature, i.e., the weaker time decay $|s|^{-1}$ of solutions in contrast to three or higher dimensional problem, 
because $L^{2}$-norm of cubic nonlinearity in \eqref{delicate term} enjoys at most $s^{-2}$ decay which is not sufficient to compensate the time growth for \eqref{delicate term} being integrable in time. 
Nevertheless, since the singularity $|\eta|^{-1}$ near the origin is weaker compared to three or higher dimensional case where $\mathcal{F}(|x|^{-1})(\eta) = C_d |\eta|^{-d+1}$, we anticipate that this advantage leads us to obtaining an extra time decay. One of key observations, as already observed in \cite{pusa}, is the null structure from the phase function
\begin{align}\label{Null structure}
    \nabla_\xi \phi(\xi,\eta) \Big|_{\eta=0}
    =\nabla_\xi \Big( \braxi-\langle\xi-\eta\rangle \Big) \Big|_{\eta=0} = 0.
\end{align}
The null structure removes the singularity near the origin, more precisely, the multiplier in \eqref{delicate term} behaves as 
$$|\nabla_\xi \phi(\xi,\eta)|^2|\eta|^{-1} \sim |\eta|, \; \; \text{ if } \; |\eta|,|\xi|\les 1.$$
Using this, we can heuristically regard $|\eta| \approx |s|^{-1}$ in the analysis respect. Indeed, for $|\eta|\ge |s|^{-1}$, we can exploit the space resonance, in other words, we can apply an integration by parts to the following quadratic term  
\begin{align*}
    \widehat{ |u(s)|^2 } (\eta)
    = \frac{1}{(2\pi)^2}\int_{\R^2} e^{is(\langle \sigma+\eta\rangle - \langle\sigma\rangle )}\widehat{f}(s,\sigma)\overline{\widehat{f}(s,\eta+\sigma}) d\sigma,
\end{align*}
by using $|\nabla_\sigma(\langle \sigma+\eta\rangle - \langle\sigma\rangle )|\sim |\eta|$ for $|\eta|,|\sigma|\les1$ to derive an additional time decay at the cost of $|\eta|^{-1}$.
In the rigorous proof below, we then can control the $xf$ and $x^2f$ in $L^2$, allowing a small growth in $t$. 


As mentioned above, since our main equation has the {\em long-range}
nonlinearity, the nonlinearity occurs the logarithm divergence
in terms of time integration. To overcome this difficulty, we employ
a phase modification \eqref{eq:modified-scattering} from the singular potential $|x|^{-1}$ and obtain an extra logarithm time decay by following the argument in \cite{pusa,kapu}. We begin with writing the nonlinear term as 
\begin{align*}
    & \mathcal{I}(t,\xi)=\frac{1}{(2\pi)^3}\int_{0}^{t}\iint_{\R^{2}\times\R^{2}}e^{isp(\xi,\eta,\sigma)}|\eta|^{-1}\wh{f}(s,\xi+\eta)\wh{f}(s,\xi+\sigma)\overline{\wh{f}(s,\xi+\eta+\sigma)}d\eta d\sigma ds,
   \end{align*} 
   where 
   \[
    p(\xi,\eta,\sigma)=\braxi-\langle\xi+\eta\rangle-\langle\xi+\sigma\rangle+\bra{\xi+\eta+\sigma}.
   \]
   Let us assume that $|\xi|\les 1$. By Taylor expansion, the phase function is approximated by
 \begin{align*}
    p(\xi,\eta,\sigma)=\eta\cdot\left(\frac{\xi}{\braxi}-\frac{\xi+\sigma}{\bra{\xi+\sigma}}\right)+O\left(|\eta|^{2}\right).
 \end{align*}  
 Then, neglecting all contributions that decay faster than $|s|^{-1}$, we can approximate the above integration as 
\begin{align}\begin{aligned}\label{HD}
  &\frac{1}{(2\pi)^3}\iint_{|\eta|\lesssim |s|^{-1+}}e^{is\eta\cdot\left(\frac{\xi}{\braxi}-\frac{\xi+\sigma}{\bra{\xi+\sigma}}\right)}|\eta|^{-1}\wh{f}(s,\xi)\wh{f}(s,\xi+\sigma)\overline{\wh{f}(s,\xi+\sigma)}d\eta d\sigma  \\
  &=\frac{1}{(2\pi)^3}\widehat{f}(s,\xi) 
  \iint_{\R^{2}\times\R^{2}} e^{is\eta\cdot\left(\frac{\xi}{\braxi}-\frac{\xi+\sigma}{\bra{\xi+\sigma}}\right)}|\eta|^{-1} d\eta \big|\widehat{f}(s,\xi+\sigma)\big|^2 d\sigma + O(s^{-1-}) \\ 
  &=\frac{1}{2\pi}\widehat{f}(s,\xi)\int_{\R^2} \mathcal{F}^{-1}(|\eta|^{-1}) \left(s\big(\tfrac{\xi}{\braxi}-\tfrac{\sigma}{\bra{\sigma}}\big)\right)\big|\widehat{f}(s,\sigma)\big|^2 d\sigma+ O(s^{-1-}),
\end{aligned}\end{align}
under the suitable assumption on $f$. Then, we obtain 
\begin{align*}
    \partial_t\widehat{f}(t,\xi) = it^{-1}\frac{1}{(2\pi)^2}\widehat{f}(t,\xi)\int_{\R^2}\left|\frac{\xi}{\braxi}-\frac{\sigma}{\bra{\sigma}}\right|^{-1}\big|\widehat{f}(t,\sigma)\big|^2 d\sigma + O(t^{-1-})
\end{align*}
which implies the modified scattering property \eqref{eq:modified-scattering} and \eqref{formula of B}. The rigorous analysis for error terms will be achieved by identifying suitable scale in $\eta$ with respect to time, say $s^{-1+}$, and then, by exploiting the space resonance for $|\eta|\gtrsim s^{-1+}$.

   
\subsection{Motivation: 2d models}
The two dimensional semi-relativistic equation \eqref{main-eq:semi} might be regarded as a simplified model of the Chern-Simons-Dirac system under the Coulomb gauge condition \footnote{ \, We refer to \cite{bourcanma2014-dcds} for its derivation.}
\begin{align}
    (-i\partial_{t}+\al\cdot D+m\beta)\psi & =N(\psi,\psi)\psi\qquad\mathrm{in}\;\;\R\times\mathbb{R}^{2},\tag{CSD-C}\label{eq:csd-coulomb}
    \end{align}
where the unknown $\psi:\R^{1+2}\to\C^{2}$ and the nonlinear term is given as 
\[
N(\psi,\psi)=\frac{1}{\Delta}\left[\Big(\partial_{1}(\psi^{\dagger}\al^{2}\psi)-\partial_{2}(\psi^{\dagger}\al^{1}\psi)\Big)+\Big(\partial_{2}(|\psi|^{2})\al^{1}-\partial_{1}(|\psi|^{2})\al^{2}\Big)\right]
\]
with Dirac matrices $\al^{j},\beta$ defined as 
\begin{align*}
\al^{1}=\begin{bmatrix}0 & i\\
-i & 0
\end{bmatrix},\quad\al^{2}=\begin{bmatrix}0 & 1\\
1 & 0
\end{bmatrix},\quad\beta=\begin{bmatrix}1 & 0\\
0 & -1
\end{bmatrix}.
\end{align*} 
One of strategy to deal with Dirac operator is, as introduced in \cite{anfosel}, to diagonalize the system using the following identity
\begin{align*}
    \al\cdot D+m\beta = \langle D\rangle \Pi_+(D) - \langle D\rangle \Pi_-(D),
    \end{align*}
where $\Pi_{\pm}(D) = \frac12 \left( I_2\pm \frac{1}{\langle D\rangle}\big[ \alpha\cdot D + \beta\big]\right)$ are the projection operators.
Letting $\psi_{\pm}=\Pi_{\pm}(D)\psi$, \eqref{eq:csd-coulomb} is indeed diagonalized into 
\begin{align*}
    -i\partial_t \psi_{\pm} \pm \langle D\rangle \psi_{\pm}= \sum_{\theta_1,\theta_2,\theta_3\in\{\pm\}}N(\psi_{\theta_1},\psi_{\theta_2})\psi_{\theta_3},
\end{align*}
which consists of the nonlocal differential operator and cubic Hartree nonlinear term as in our main equations, \eqref{main-eq:semi}. 
Especially, the potentials in Hartree term are given as $\frac{\eta_j}{|\eta|^2}$ for $j=1,2$ in the fourier space, which has the similar singularity near the origin and decay property  with order $-1$ as the one given in  \eqref{main-eq:semi}, so \eqref{eq:csd-coulomb} can be also regarded as the scattering critical equation and modified scattering would be expected. 
However, not only long time behaviors but the global existence of solutions to \eqref{eq:csd-coulomb} are still unknown and only the local results have been intensively studied including other choices of gauge \cite{Okamoto2013,bourcanma2014-dcds,Huh2016,Pecher2016}.
One of main difficulty in studying global solutions, compared to our equation \eqref{main-eq:semi}, arises from analysis of the following various resonance functions 
\begin{align}
    p_{(\theta_1,\theta_2,\theta_3)}(\xi,\eta,\sigma)=\braxi-\theta_1\langle\xi-\eta\rangle-\theta_2\langle\eta+\sigma\rangle+\theta_3\langle\sigma\rangle, \quad \theta_i \in \{\pm\}.\label{eq:resonance-ftn}
    \end{align}
Indeed, one can see that the key null structure \eqref{Null structure} to remove  the singularity is no longer valid when $\theta_1=-$.
Nevertheless, we believe that the methodology in this paper with the help of careful analysis of resonance set together with null structures from Dirac operator will play a crucial role in studying the global behavior of solutions to \eqref{eq:csd-coulomb}. 


The similar structure also can be observed in the following Dirac equation 
\begin{align}
(-i\partial_{t}+\al\cdot D+m\beta)\psi & =\lam\left(|x|^{-1}*|\psi|^{2}\right)\psi\qquad\mathrm{in}\;\;\R\times\mathbb{R}^{2},\tag{DE}\label{eq:dirac-hartree}
\end{align}
where $\psi:\R^2\rightarrow \C^2$ is the spinor.
\eqref{eq:dirac-hartree} describe the relativistic dynamics of electrons in graphene  and can be derived from the nonlinear Schr\"odinger equation with a potential which is periodic with respect to honeycomb structure \cite{arbuspar2018-jmp}.
 \eqref{eq:dirac-hartree} also can be referred to as the scattering critical equation, but the only local results were established in \cite{hajjmehats2014,lee2021-bkms}. 
 The global well-posedness and modified scattering for \eqref{eq:dirac-hartree}
will be treated in future work. Finally, we refer to \cite{CKLY2022,Cloos2020} for global results for \eqref{eq:dirac-hartree} in three dimension. 


% where $V_{\mu}$ is in \eqref{eq:g-semi-potential}. When $\mu=0$,
% \eqref{eq:dirac-hartree} is simplified models of Dirac equations
% with nonlinearity derived from honeycomb structure and when $\mu>0$,
% \eqref{eq:dirac-hartree} is derived from Dirac-Klein-Gordon system
% by combining a standing wave (see the \citet{yang2019,tesfahun2020-2d}).
% Regarding the asymptotic behavior results for \eqref{eq:dirac-hartree}
% when $\mu>0$, in Cho and the second author \citet{chle2021-die}
% have shown the global well-posedness and small data scattering results
% for $L_{x}^{2}(\R^{2})$. For 3 dimension, global well-posedness and
% small data scattering was established for $H^{\ve}(\R^{3})$ by the
% third author \citet{yang2019} and independently by Tesfahun \citet{tesfahun2020-3d}.
% We also refer to \citet{arbuspar2018-jmp} for the derivation of \eqref{eq:dirac-hartree}
% when $\mu=0$. Given the results for the 3 dimensional modified scattering
% between semi-relativistic equations \citet{pusa} and Dirac equations
% \citet{CKLY2022}, our main theorem will play a crucial role in the
% proof of the global well-posedness and modified scattering for \eqref{eq:csd-coulomb}
% and \eqref{eq:dirac-hartree} with potential $V_{0}$. These topics
% will be treated in future work.


% The Dirac equation
% \eqref{eq:DiracHartree} with $A=B=\beta$ and $V(x)=\int_{0}^{\infty}e^{-\mu^{2}r-{|x|^{2}}/{4r}}\frac{dr}{r} $ (which is called Yukawa potential) can be derived from the Dirac-Klein-Gordon system {} and the scattering for this equation was proved in \cite{tesfahun2020-2d,chle2021-die}.
% While, \eqref{eq:DiracHartree} with $A=B=I_2$ and $V=|x|^{-1}$ can be derived from the nonlinear Schr\"odinger equation with a potential which is periodic with respect to honeycomb structure \cite{arbuspar2018-jmp}.







% \subsection{Proof of Theorem \ref{main-thm:semi}}

% To show the decay rate \eqref{global-bound:semi} and global well-posedness,
% we have to handle a weighted energy norm and Fourier amplitude in
% $\Sigma_{T}$ by a bootstrap argument (see \eqref{assumption-apriori}).
% For our main proof, it takes precedence to show the existence of a
% local solution to \eqref{main-eq:semi} in $\Sigma_{T}$. However,
% since it is straightforward, we may omit the proof (for instance see
% \citet{lee2021-bkms,hele2014}).

% Given $T>0$, we assume that $\psi$ be a solution to \eqref{main-eq:semi}
% on $[0,T]$ with initial data condition \eqref{condition-initial:semi}.
% Then it suffices to prove that for sufficiently small $\ve_{1}>0$,
% there exists C such that 
% \begin{align}
% \|u\|_{\Sigma_{T}}\le\ve_{0}+C\ve_{1}^{3}.\label{eq:contraction}
% \end{align}
% The propositions \ref{prop-energy} and \ref{prop:scattering} below
% induce \eqref{eq:contraction}. This completes the proof of global
% well-posedness. For the proof of the modified scattering, we may refer
% to the beginning of Section \ref{sec:scattering}.




\subsection*{Notations}
\; 

\noindent $\bullet$ (Fourier transform)
$\mathcal F g(\xi)=\widehat{g}(\xi):=\int_{\R^2}e^{-ix\cdot\xi}g(x)dx$ and $g(x)=\frac{1}{(2\pi)^2}\int_{\R^2}e^{ix\cdot\xi}\widehat{g}(\xi)d\xi$.

\noindent $\bullet$
$\langle x\rangle:=(1+|x|^{2})^{\frac{1}{2}}$ for $x\in\mathbb{R}^{2}$.

\noindent $\bullet$ (Mixed-normed spaces) For a Banach space $X$
and an interval $I$, $u\in L_{I}^{q}X$ iff $u(t)\in X$ for a.e. $t\in I$
and $\|u\|_{L_{I}^{q}X}:=\|\|u(t)\|_{X}\|_{L_{I}^{q}}<\infty$. Especially,
we denote $L_{I}^{q}L_{x}^{r}=L_{t}^{q}(I;L_{x}^{r}(\mathbb{R}^2))$, $L_{I,x}^{q}=L_{I}^{q}L_{x}^{q}$,
$L_{t}^{q}L_{x}^{r}=L_{\mathbb{R}}^{q}L_{x}^{r}$.

\noindent $\bullet$ As usual, different positive constants are denoted by the same letter $C$, if not specified. $A\lesssim B$ and $A\gtrsim B$ means that
$A\le CB$ and $A\ge C^{-1}B$, respectively for some $C>0$. $A\sim B$
means that $A\lesssim B$ and $A\gtrsim B$.

\noindent $\bullet$ (Fourier multiplier) $D=-i\nabla$. For $m:\R^2\rightarrow\R$, $m(D)f:=\mathcal{F}^{-1}\big( m(\xi)\widehat{f}(\xi) \big)$.


\noindent $\bullet$ (Littlewood-Paley operators) Let $\rho$ be a
bump function such that $\rho\in C_{0}^{\infty}(B(0,2))$
with $\rho(\xi)=1$ for $|\xi|\le1$ and define $\rho_{N}(\xi):=\rho\left(\frac{\xi}{N}\right)-\rho\left(\frac{2\xi}{N}\right)$
for $N\in2^{\mathbb{Z}}$ and also $\rho_{\le N_{0}}:=1-\sum_{N>N_{0}}\rho_{N}$. We define the frequency projection
$P_{N}$ by $\mathcal{F}(P_{N}f)(\xi)=\rho_{N}(\xi)\widehat{f}(\xi)$. In addition
$P_{N_{1}\le\cdot\le N_{2}}:=\sum_{N_{1}\le N\le N_{2}}P_{N}$ and
$P_{\sim N_{0}}:=\sum_{N\sim N_{0}}P_{N}$. For $N\in2^{\mathbb{Z}}$
we denote $\widetilde{\rho_{N}}=\rho_{N/2}+\rho_{N}+\rho_{2N}$. In
particular, $\widetilde{P_{N}}P_{N}=P_{N}\widetilde{P_{N}}=P_{N}$
where $\widetilde{P_{N}}=\mathcal{F}^{-1}\widetilde{\rho_{N}}\mathcal{F}$.
Especially, we denote $P_{N}f$ by $f_{N}$ for any measurable function
$f$.

\noindent $\bullet$ Let $\textbf{v}=(v_{i}),\textbf{w}=(w_{i})\in\R^{2}$.
Then $\textbf{v}\otimes\textbf{w}$ denotes the usual tensor product
such that $(\textbf{v}\otimes\textbf{w})_{ij}=v_{i}w_{j}$ for $i,j=1,2$. We also
denote a tensor product of $\textbf{v}\in\C^{n}$ and $\textbf{w}\in\C^{m}$
by a matrix $\textbf{v}\otimes\textbf{w}=(v_{i}w_{j})_{\substack{i=1,\cdots,n\\
i=1,\cdots,m
}
}$. For simplicity, we use the simplified notation 
\[
\mathbf{v}^{k}=\overbrace{\mathbf{v}\otimes\cdots\otimes\mathbf{v}}^{k\;\text{times}},\qquad\nabla^{k}=\overbrace{\nabla\otimes\cdots\otimes\nabla}^{k\;\text{times}}.
\]
The product of $\mathbf{v}$ and $f\in\C$ is given by $\mathbf{v}f=\mathbf{v}\otimes f$.

\noindent $\bullet$ For the distinction between a vector and a scalar,
we use the bold letter for a vector-valued function and the normal
letter for a scalar-valued function.


%%%%%%%%%%%%%%%%%%%%%%%%%%%%%%%%%%%%%%%%%%%%%%%%%%%%%%%%%%%%%%%%%%%%%%%%%%%%%%%%%%%%%%%%%%%%%%%%%%%%%%%%%%%%%%%%%%%%%%%%%%%%%%%%%%%%%%%%%%%%%%%%%%%%%%%%%%%%%%%%%%

\section{Time decay estimates}
In this section, we find sharp time decay estimates of solutions to linear equation. We define an a priori assumption incorporating time decay to get global solutions to \eqref{main-eq:semi}. 
Define 
\begin{align}
    f(t,x):=e^{it\langle D\rangle}u(t,x),\label{eq:interation}
\end{align}
where $u(t)$ is a solution to \eqref{main-eq:semi}. Let us set
\begin{align}\label{eq:bundle}
	n\ge1000,\;\; k=\frac{n}{100},\;\; \mbox{ and }\;\; \delta_0=\frac{1}{100}.
\end{align} For $\ve_{1}>0$
to be chosen later, we assume a priori smallness of solutions: for
a large time $T>0,$
\begin{align}
\|u\|_{\Sigma_{T}}\les\ve_{1},\label{assumption-apriori}
\end{align}
with 
\begin{align*}
\begin{aligned}\|u\|_{\Sigma_{T}} & :=\sup_{t\in[0,T]}\Big[\bra{t}^{-\de_{0}}\|u(t)\|_{E_{1}}+\bra{t}^{-2\de_{0}}\|u(t)\|_{E_{2}}+\|u(t)\|_{S}\Big],\end{aligned}
\end{align*}
where 
\begin{align*}
\|u(t)\|_{E_{1}} & :=\|u(t)\|_{H^{n}(\R^2)}+\|xe^{it\jp D}u(t)\|_{H^{2}(\R^2)},\\
\|u(t)\|_{E_{2}} & :=\|x^{2}e^{it\jp D}u(t)\|_{H^{2}(\R^2)},\\
\|u(t)\|_{S_{\ }} & :=\left\Vert \braxi^{k}\widehat{u}(t)\right\Vert _{L_{\xi}^{\infty}(\R^2)}.
\end{align*}
We compute the pointwise time decay of the semi-relativistic equation by assuming the a priori assumption \eqref{assumption-apriori}. We refer to \cite[Proposition~3.1]{pusa} for three dimensional case where the sharp time decay $|t|^{-\frac32}$ was obtained. Here, we obtain sharp time decay estimates for two dimensional case by following the similar strategy in \cite[Proposition~3.1]{pusa}, 
but we have to bound the linear solution with the differential order up to $k$.
 \begin{prop}[Time decay]\label{timedecay-prop}
Assume that $u$ satisfies the a priori assumption \eqref{assumption-apriori} for $\ve_{1}$ and $T$ with the index conditions \eqref{eq:bundle}.  Then for small $\ve_{1}$, There exists $C$
satisfying that for $0\le t\le T$ and $0\le\ell\le k$ 
\begin{align}
\|u(t)\|_{W^{\ell,\infty}}\le C\langle t\rangle^{-1}\ve_{1},\label{eq:time-decay}
\end{align}
where the index $k$ is in the a priori assumption \eqref{assumption-apriori}.
\end{prop}

\begin{proof}
We prove that 
\begin{align*}
  \| \bra{D}^{\ell} e^{-it\langle D\rangle} f \|_{L^{\infty}} \lesssim 1,
\end{align*}    
whenever $f$ satisfies
\begin{align}
    \langle t\rangle^{-1}\|\braxi^{k}\widehat{f}\|_{L_{\xi}^{\infty}}+\langle t\rangle^{-1-\de_{0}}\Big[\|xf\|_{L_{x}^{2}}+\|f\|_{H^{n}}\Big]+\bra{t}^{-1-2\de_{0}}\|x^{2}f\|_{L_{x}^{2}}\le\ve_{1}.\label{f-condition}
\end{align}
We begin with writing     
\[
\bra{D}^{\ell}e^{-it\langle D\rangle} f (t,x)=\sum_{N\in2^{\mathbb{Z}}}I_N(t,x)
\]
where 
\begin{align}
  I_N(t,x) &:= \frac{1}{(2\pi)^2}\int_{\mathbb{R}^{2}}\bra{\xi}^{\ell}e^{it\phi(\xi)}\widehat{f}(\xi)\rho_{N}(\xi)d\xi, \nonumber \\ 
    \phi(\xi) &:= -\braxi+\xi\cdot\frac{x}{t}.\label{theta-phase}
\end{align}
% It suffices to prove that a priori assumption \eqref{assumption-apriori}
% implies that for $f(t),$ 
% \begin{align}
% \langle t\rangle^{-1}\|\braxi^{k}\widehat{f}\|_{L_{\xi}^{\infty}}+\langle t\rangle^{-1-\de_{0}}\Big[\|xf\|_{L_{x}^{2}}+\|f\|_{H^{n}}\Big]+\bra{t}^{-1-2\de_{0}}\|x^{2}f\|_{L_{x}^{2}}\le\ve_{1}.\label{f-condition}
% \end{align}
% Under this, it suffices to show 
% \begin{equation}
% \left|\bra{D}^{\ell}u(t,x)\right|\le\sum_{N\in2^{\mathbb{Z}}}I_{N}(t,x)\le C\ve_{1}\label{eq:claim2.1}
% \end{equation}
% where 
% \[
% I_{N}(t,x)=\left|\int_{\mathbb{R}^{2}}\bra{\xi}^{\ell}e^{it\phi(\xi)}\widehat{f}(\xi)\rho_{N}(\xi)d\xi\right|.
% \]
We decompose 
\[
\sum_{N\in2^{\mathbb{Z}}}I_{N}(t,x)=\left(\sum_{N\le\langle t\rangle^{-\frac{1}{2}}}+\sum_{N\ge\langle t\rangle^{\frac{2}{n}}}+\sum_{\langle t\rangle^{-\frac{1}{2}}\le N\le\langle t\rangle^{\frac{2}{n}}}\right)I_{N}(t,x).
\]
The low-frequency part can be estimated as 
\[
\sum_{N\le\langle t\rangle^{-\frac{1}{2}}}I_{N}(t,x)\les\sum_{N\le\langle t\rangle^{-\frac{1}{2}}}\|\rho_{N}\|_{L^{1}}\normo{\bra{\xi}^{\ell}\widehat{f}}_{L_{\xi}^{\infty}}\les\langle t\rangle^{-1}\|\bra{\xi}^{\ell}\widehat{f}\|_{L_{\xi}^{\infty}}\les\ve_{1}.
\]
For the high-frequency, we exploit the high Sobolev norm bound as follows:
\begin{align*}
\sum_{N\ge\langle t\rangle^{\frac{2}{n}}}I_{N}(t,x) & \les\sum_{N\ge\langle t\rangle^{\frac{2}{n}}}\bra{N}^{\ell+1}\|\rho_{N}\widehat{f}\|_{L_{x}^{2}} \les\sum_{N\ge\langle t\rangle^{\frac{2}{n}}}N^{\ell+1-n}\|f\|_{H^{n}}\\
 &  \les\langle t\rangle^{-1-\de_{0}}\|f\|_{H^{n}}\les\ve_{1}. 
\end{align*}

For the remaining mid-frequency part, we apply the non-stationary phase method. %\[
%\sum_{\langle t\rangle^{-\frac{1}{2}}\le N\le\langle t\rangle^{\frac1{15}}}\left|\int_{\mathbb{R}^{3}}e^{it\phi (\xi)}\widehat{f }(\xi)\rho_{N}(\xi)d\xi\right|\les \ve_1.
%\]
One verifies that when $|x|>t$, the phase $\phi$ is non-stationary, i.e. 
\begin{align}
|\nabla_{\xi}\phi(\xi)|\ge \left|\frac{|x|}{t}-\frac{|\xi|}{\braxi}\right|\ge 1-\frac{|\xi|}{\braxi}\gtrsim\braxi^{-2}.\label{non-stat-1}
\end{align}
On the other hand, when $|x|<t$, the phase $\phi$ could be stationary around $\xi_{0}$:
\[
\nabla_{\xi}\phi(\xi_{0})=0\;\;\mbox{where}\;\;\xi_{0}=-\frac{x}{\sqrt{t^{2}-|x|^{2}}}.
\]
We now set $N_0\sim |\xi_{0}|$. 
First, we consider the non-stationary case $N\nsim N_{0}$. Then one can find that  
\begin{align}
\Big|\nabla_{\xi}\phi(\xi)\Big|\gtrsim\max\left(\frac{|\xi-\xi_{0}|}{\bra{N}^{3}},\frac{|\xi-\xi_{0}|}{\bra{N_{0}}^{3}}\right), \text{ for } |\xi|\sim N. \label{non-stat-2}
\end{align}
We perform an integration by parts twice to write $I_N$ as 
\[
\int_{\mathbb{R}^{2}}\bra{\xi}^{\ell}e^{it\phi(\xi)}\widehat{f}(\xi)\rho_{N}(\xi)d\xi=I_{N}^{1}(t,x)+I_{N}^{2}(t,x)+I_{N}^{3}(t,x)
\]
where 
\begin{align}
\begin{aligned}I_{N}^{1}(t,x) & =-t^{-2}\int_{\mathbb{R}^{2}}\bra{\xi}^{\ell}e^{it\phi(\xi)}\frac{\nabla_{\xi}\phi}{|\nabla_{\xi}\phi|^{4}}\cdot\nabla_{\xi}\phi\nabla_{\xi}^{2}\left(\widehat{f_{N}}(\xi)\right)d\xi,\\
I_{N}^{2}(t,x) & =-2t^{-2}\int_{\mathbb{R}^{2}}\bra{\xi}^{\ell}e^{it\phi(\xi)}\nabla_{\xi}\cdot\left(\frac{\nabla_{\xi}\phi}{|\nabla_{\xi}\phi|^{2}}\right)\frac{\nabla_{\xi}\phi}{|\nabla_{\xi}\phi|^{2}}\cdot\nabla_{\xi}\widehat{f_{N}}(\xi)d\xi\\
I_{N}^{3}(t,x) & =-t^{-2}\int_{\mathbb{R}^{2}}\bra{\xi}^{\ell}e^{it\phi(\xi)}\nabla_{\xi}\cdot\left[\nabla_{\xi}\cdot\left(\frac{\nabla_{\xi}\phi}{|\nabla_{\xi}\phi|^{2}}\right)\frac{\nabla_{\xi}\phi}{|\nabla_{\xi}\phi|^{2}}\right]\widehat{f_{N}}(\xi)d\xi.
\end{aligned}
\end{align}
By \eqref{non-stat-1} and \eqref{non-stat-2}, we obtain the following bounds (independent
of $N_{0}$): for $|\xi|\sim N$
\begin{align}
\begin{aligned}
\left|\frac{1}{\nabla_{\xi}\phi(\xi)}\right| & \les N^{-1}\bra{N}^{3},\\
\left|\nabla_{\xi}\cdot\left(\frac{\nabla_{\xi}\phi}{|\nabla_{\xi}\phi|^{2}}\right)\right| & \les N^{-2}\bra{N}^{5},\\
\left|\nabla_{\xi}\cdot\left[\nabla_{\xi}\cdot\left(\frac{\nabla_{\xi}\phi}{|\nabla_{\xi}\phi|^{2}}\right)\frac{\nabla_{\xi}\phi}{|\nabla_{\xi}\phi|^{2}}\right]\right| & \les N^{-4}\bra{N}^{10}.
\end{aligned}
\label{bound-phase}
\end{align}
Using \eqref{bound-phase} and the Sobolev embedding, we see that 
\begin{align}
    \begin{aligned}
    \label{computation}
\Big|I_{N}^{1}(t,x)\Big| & \les t^{-2}N^{-2}\bra{N}^{\ell+6}\left\Vert \nabla_{\xi}^{2}\left(\widehat{f_{N}}\;\right)\right\Vert _{L_{\xi}^{1}}\\
 & \les t^{-2}\bra{N}^{\ell+6}\Big(N^{-1}\|x^{2}f\|_{L^{2}}+N^{-2}\left\Vert \nabla\rho_N\right\Vert _{L^{\frac{4}{3}}}\|\langle x\rangle^{2}f\|_{L^{2}} \\ 
 &\qquad\qquad \qquad\qquad +N^{-2}\min\big( \langle N\rangle^{-n}N^{-1}\| f\|_{H^n}, \langle N\rangle^{-k}\|\wh{f}\|_{L_{\xi}^{\infty}}\big)\Big)\\
 & \les t^{-2}\bra{N}^{\ell+6}\left((N^{-1}+N^{-\frac{3}{2}})t^{1+2\de_{0}}+N^{-2}\min\big(\bra{N}^{-n+1} t^{1+\delta_0},t\big)\right),
    \end{aligned}\end{align}
which implies that 
\[
\sum_{\langle t\rangle^{-\frac{1}{2}}\le N\le\langle t\rangle^{\frac{2}{n}}}\Big|I_{N}^{1}(t,x)\Big|\les\ve_{1}.
\]
$I_{N}^{2}$ can be estimated similarly. Indeed, one has 
\begin{align*}
&|I_{N}^{2}(t,x)|  \les t^{-2}N^{-3}\bra{N}^{\ell+8}\normo{\nabla_{\xi}\left(\rho_{N}\widehat{f}\;\right)}_{L_{\xi}^{1}}\\
&\; \les t^{-2}\bra{N}^{\ell+8}\left(N^{-2}\left\Vert \nabla\rho_N\right\Vert _{L^{\frac{4}{3}}}\|\langle x\rangle^{2}f\|_{L^{2}}+N^{-2}\min\big( \langle N\rangle^{-n}N^{-1}\| f\|_{H^n}, \langle N\rangle^{-k}\|\wh{f}\|_{L_{\xi}^{\infty}}\big)\right),
\end{align*}
which leads us that 
\[
\sum_{\langle t\rangle^{-\frac{1}{2}}\le N\le\langle t\rangle^{\frac{2}{n}}}|I_{N}^{2}(t,x)|\les\ve_{1}.
\]
Lastly, we estimate 
\begin{align*}
    \sum_{\langle t\rangle^{-\frac{1}{2}}\le N\le\langle t\rangle^{\frac{2}{n}}} |I_N^3(t,x)| &\lesssim \sum_{\langle t\rangle^{-\frac{1}{2}}\le N\le\langle t\rangle^{\frac{2}{n}}} t^{-2} N^{-4}\langle N\rangle^{\ell+10} \|\rho_N\widehat{f} \|_{L_{\xi}^1} \\ 
    &\lesssim \sum_{\langle t\rangle^{-\frac{1}{2}}\le N\le\langle t\rangle^{\frac{2}{n}}}t^{-2}\langle N\rangle^{\ell+10}N^{-2}\min\big( \langle N\rangle^{-n}N^{-1}\| f\|_{H^n}, \langle N\rangle^{-k}\|\wh{f}\|_{L_{\xi}^{\infty}}\big)  \\ 
    &\lesssim \ep_1.
\end{align*}

We remain to consider the stationary phase case $N\sim N_{0}$. We
further decompose dyadically the frequency space around $\xi_0$. Let $L_{0}\in2^{\mathbb{Z}}$
such that $\frac{L_{0}}{2}<t^{-\frac{1}{2}}\le L_{0}$. We write 
\[
\left|\int_{\mathbb{R}^{2}}\bra{\xi}^{\ell}e^{it\phi(\xi)}\rho_{N}(\xi)\widehat{f}(\xi)d\xi\right|\le\sum_{L=L_{0}}^{2^{10}N}|J_{L}|
\]
where 
\begin{align*}
J_{L}(t,x)=\left\{ \begin{aligned} & \int_{\mathbb{R}^{2}}\bra{\xi}^{\ell}e^{it\phi(\xi)}\rho_{\le L_{0}}(\xi-\xi_{0})\rho_{N}(\xi)\widehat{f}(\xi)d\xi\qquad\mbox{when }L=L_{0},\\
 & \int_{\mathbb{R}^{2}}\bra{\xi}^{\ell}e^{it\phi(\xi)}\rho_{L}(\xi-\xi_{0})\rho_{N}(\xi)\widehat{f}(\xi)d\xi\quad\qquad\mbox{when }L>L_{0}.
\end{aligned}
\right.
\end{align*}
We bound $J_{L_0}$ the support of which contains the stationary point by just the measure 
\[
|J_{L_{0}}|\les L_{0}^{2}\bra{N}^{l}\|\rho_{N}\widehat{f}\|_{L_{\xi}^{\infty}}\les t^{-1}\|\langle \xi\rangle^k\widehat{f}\|_{L_{\xi}^{\infty}}\les\ve_{1}.
\]
When $L>L_{0}$, we return to the non-stationary phase cases. By
integrating by parts twice, we decompose $J_{L}(t,x)$ into  
\[
J_{L}(t,x)=J_{L}^{1}(t,x)+J_{L}^{2}(t,x)+J_{L}^{3}(t,x),
\]
where 
\begin{align*}
\begin{aligned}J_{L}^{1}(t,x) & =-t^{-2}\int_{\mathbb{R}^{2}}\bra{\xi}^{l}e^{it\phi(\xi)}\frac{\nabla_{\xi}\phi}{|\nabla_{\xi}\phi|^{4}}\cdot\nabla_{\xi}\phi\nabla_{\xi}^{2}\left(\widehat{f}(\xi)\rho_{N}(\xi)\rho_{L}(\xi-\xi_{0})\right)d\xi,\\
J_{L}^{2}(t,x) & =-2t^{-2}\int_{\mathbb{R}^{2}}\bra{\xi}^{l}e^{it\phi(\xi)}\nabla_{\xi}\cdot\left(\frac{\nabla_{\xi}\phi}{|\nabla_{\xi}\phi|^{2}}\right)\frac{\nabla_{\xi}\phi}{|\nabla_{\xi}\phi|^{2}}\cdot\nabla_{\xi}\left(\widehat{f}(\xi)\rho_{N}(\xi)\rho_{L}(\xi-\xi_{0})\right)d\xi,\\
J_{L}^{3}(t,x) & =-t^{-2}\int_{\mathbb{R}^{2}}\bra{\xi}^{l}e^{it\phi(\xi)}\nabla_{\xi}\cdot\left[\nabla_{\xi}\cdot\left(\frac{\nabla_{\xi}\phi}{|\nabla_{\xi}\phi|^{2}}\right)\frac{\nabla_{\xi}\phi}{|\nabla_{\xi}\phi|^{2}}\right]\widehat{f}(\xi)\rho_{N}(\xi)\rho_{L}(\xi-\xi_{0})d\xi.
\end{aligned}
\end{align*}
We estimate $J_{L}^{i}$ for $i=1,2,3$ similarly to the above, but the following bounds are employed instead of \eqref{bound-phase}: for $|\xi|\sim N$ and $|\xi-\xi_0|\sim L$,  
\begin{align*}
\begin{aligned}\left|\nabla_{\xi}\phi(\xi)\right|^{-1} & \les L^{-1}\bra{N}^{3},\\
\left|\nabla_{\xi}\cdot\left(\frac{\nabla_{\xi}\phi}{|\nabla_{\xi}\phi|^{2}}\right)\right| & \les L^{-2}\bra{N}^{5},\\
\left|\nabla_{\xi}\cdot\left[\nabla_{\xi}\cdot\left(\frac{\nabla_{\xi}\phi}{|\nabla_{\xi}\phi|^{2}}\right)\frac{\nabla_{\xi}\phi}{|\nabla_{\xi}\phi|^{2}}\right]\right| & \les L^{-4}\bra{N}^{10}.
\end{aligned}
\end{align*}
The computation as in \eqref{computation} gives the desired results.
We omit the details and complete the proof of \eqref{eq:time-decay}
\end{proof}



Since we have to handle the multipliers to exploit the time decay
\eqref{eq:time-decay} in our main proof, we introduce some useful
estimates in the rest of this section. %\begin{lemma}[Lemma 6.3 of \cite{chleoz}]\label{lem:hls-infty} For
%	any $\mathbb{C}$-valued functions $u_{1},u_{2}\in L_{x}^{2}(\mathbb{R}^{2})\cap L_{x}^{\infty}(\mathbb{R}^{2})$,
%	we get
%	\begin{align*}
%		\||x|^{-1}*(u_{1}\overline{u_{2}})\|_{L_{x}^{\infty}(\mathbb{R}^{2})}\les\|u_{1}\|_{L_{x}^{\frac{4}{1-\ve}}(\mathbb{R}^{2})}\|u_{2}\|_{L_{x}^{\frac{4}{1+\ve}}(\mathbb{R}^{2})}
%	\end{align*}
%	for some $0<\ve\ll1$. \end{lemma}
\begin{lemma}[Coifman-Meyer operator estimates]\label{lem:coif}
Assume that a multiplier $\textbf{m}(\xi,\eta)$ satisfies that 
\begin{align}\label{multiplier bound}
    C_{\mathbf{m}}:=\left\Vert \iint_{\mathbb{R}^{2}\times\R^{2}}\mathbf{m}(\xi,\eta)e^{ix\cdot\xi}e^{iy\cdot\eta}\,d\eta d\xi\right\Vert _{L_{x,y}^{1}(\mathbb{R}^{2}\times\R^{2})}<\infty.    
\end{align}
Then, for $\frac{1}{p}+\frac{1}{q}=\frac{1}{2}$, 
\begin{align}\label{eq:coif-1}
\left\Vert \int_{\mathbb{R}^{2}}\mathbf{m}(\xi,\eta)\widehat{u}(\xi\pm\eta)\widehat{v}(\eta)\,d\eta\right\Vert _{L_{\xi}^{2}(\R^2)}\les C_{\mathbf{m}}\|u\|_{L^{p}}\|v\|_{L^{q}},
\end{align}
and for $\frac{1}{p}+\frac{1}{q}+\frac{1}{r}=1$, 
\begin{align}\label{eq:coif-1-1}
\left|\iint_{\mathbb{R}^{2}\times\R^{2}}\mathbf{m}(\eta,\sigma)\widehat{u}(\eta\pm\sigma)\widehat{v}(\eta)\wh{w}(\sigma)\,d\sigma d\eta\right|\les C_{\mathbf{m}}\|u\|_{L^{p}}\|v\|_{L^{q}}\|w\|_{L^{r}}.
\end{align}
% Moreover, if the multiplier $\textbf{m}(\xi,\eta,\sigma)$ satisfies
% that 
% \[
% \left\Vert \iiint_{\mathbb{R}^{2}\times\R^{2}\times\mathbb{R}^2}\mathbf{m}(\xi,\eta,\sigma)e^{ix\cdot\xi}e^{iy\cdot\eta}e^{iz\cdot\sigma}\,d\sigma d\eta d\xi\right\Vert _{L_{x,y,z}^{1}(\mathbb{R}^{2}\times\R^{2}\times\mathbb{R}^2)}\le C_{\mathbf{m}},
% \]
% for $\frac{1}{p}+\frac{1}{q}+\frac{1}{r}=\frac{1}{2}$, one has 
% \begin{align}
% \left\Vert \iint_{\mathbb{R}^{2}\times\R^{2}}\mathbf{m}(\xi,\eta,\sigma)\widehat{u}(\xi-\eta)\widehat{v}(\eta-\sigma)\wh{w}(\sigma)\,d\sigma d\eta\right\Vert _{L_{\xi}^{2}(\R^2)}\les C_{\mathbf{m}}\|u\|_{L^{p}}\|v\|_{L^{q}}\|w\|_{L^{r}}.\label{eq:coif-2}
% \end{align}
\end{lemma}

% In view of the interpolation between high Sobolev norm $\|u\|_{H^{n}}$
% and second order weighted norm $\|x^{2}e^{it\bra{D}}u\|_{L_{x}^{2}}$
% in a priori assumption \eqref{assumption-apriori}, we may consider
% a high Sobolev norm of first order weighted norm in \eqref{assumption-apriori},
% as the following lemma shows: 
% \begin{lemma} Let $u$ satisfy a priori
% assumption \eqref{assumption-apriori} for some $\ve_{1}>0$. Then
% we have 
% \begin{align}
% \left\Vert P_{N}\left(xe^{it\bra{D}}u(t)\right)\right\Vert _{H^{\frac{n}{2}}}\les\bra{t}^{\frac{3\de_{0}}{2}}\ve_{1}.\label{eq:esti-x-hn}
% \end{align}
% \end{lemma} \begin{proof} As the representation \eqref{eq:interation},
% we use the $f=e^{it\bra{D}}u$. By Plancherel's theorem, the left
% hand-side of \eqref{eq:esti-x-hn} can be rewritten as 
% \begin{align*}
% \|P_{N}(xf(t))\|_{H^{\frac{n}{2}}} & =\left\Vert \bra{\xi}^{\frac{n}{2}}\rho_{N}\nabla_{\xi}\wh{f}(t)\right\Vert _{L_{\xi}^{2}}\\
%  & =\left(\int_{\R^{2}}\bra{\xi}^{n}\rho_{N}^{2}(\xi)\nabla_{\xi}\wh{f}(t,\xi)\cdot\nabla_{\xi}\wh{f}(t,\xi)d\xi\right)^{\frac{1}{2}}.
% \end{align*}
% Using the integration by parts, we see that 
% \begin{align}
% \begin{aligned} & \int_{\R^{2}}\bra{\xi}^{n}\rho_{N}^{2}(\xi)\nabla_{\xi}\wh{f}(t,\xi)\cdot\nabla_{\xi}\wh{f}(t,\xi)d\xi\\
%  & \hspace{2cm}=-\int_{\R^{2}}\nabla_{\xi}\left(\bra{\xi}^{n}\rho_{N}^{2}(\xi)\right)\cdot\nabla_{\xi}\wh{f}(t,\xi)\wh{f}(t,\xi)d\xi\\
%  & \hspace{5cm}-\int_{\R^{2}}\bra{\xi}^{n}\rho_{N}^{2}(\xi)\wh{f}(t,\xi)\Delta_{\xi}\wh{f}(t,\xi)d\xi.
% \end{aligned}
% \label{eq:proof-lemma-xf}
% \end{align}
% The former integration in \eqref{eq:proof-lemma-xf} can be handle
% as follows: Since 
% \[
% \nabla_{\xi}(\bra{\xi}^{n}\rho_{N}(\xi))=n\bra{\xi}^{n-2}\xi\rho_{N}(\xi)+N^{-1}\bra{\xi}^{n}\rho_{N}(\xi)\rho'\left(\frac{\xi}{N}\right)\frac{\xi}{|\xi|},
% \]
% we have 
% \begin{align*}
%  & \left|\int_{\R^{2}}\nabla_{\xi}\left(\bra{\xi}^{n}\rho_{N}^{2}(\xi)\right)\cdot\nabla_{\xi}\wh{f}(t,\xi)\wh{f}(t,\xi)d\xi\right|\\
%  & \qquad\les\|xf(t)\|_{L_{x}^{2}}\|u(t)\|_{H^{n-1}}+N^{-1}\|\rho_{N}\|_{L_{\xi}^{2}}\|xf(t)\|_{L_{x}^{2}}\|\bra{\xi}^{n}\wh{f}(t)\|_{L_{x}^{\infty}}\\
%  & \qquad\les\bra{t}^{3\de_{0}}\ve_{1}^{2}.
% \end{align*}
% The remainder term in \eqref{eq:proof-lemma-xf} 
% \[
% \left|\int_{\R^{2}}\bra{\xi}^{n}\rho_{N}^{2}(\xi)\wh{f}(t,\xi)\Delta_{\xi}\wh{f}(t,\xi)d\xi\right|\les\|u(t)\|_{H^{n}}\|x^{2}f(t)\|_{L_{x}^{2}}\les\bra{t}^{3\de_{0}}\ve_{1}^{2}.
% \]
% This finishes the proof of \eqref{eq:esti-x-hn}. \end{proof}

%\begin{proof} Using Proposition \ref{timedecay-prop}, we see that
%\begin{align*}
%\left\|P_{N}\Big(|u(s)|^2\Big)\right\|_{L_{x}^{\infty}} &=\sup_{x \in \R^2} \left| \int_{\R^2} e^{ix\cdot \xi} \rho_N(\xi) \wh{|u(s)|^2}(\xi) \,d\xi \right|\\
%&=\sup_{x \in \R^2} \left| \int_{\R^2} e^{ix\cdot \xi} \rho_N(\xi) \bra{\xi}^{-k}\wh{\bra{D}^k|u(s)|^2}(\xi) \,d\xi \right|\\
%&\les \normo{ \mathcal F^{-1}(\rho_N \bra{\xi}^{-k}) * \bra{D}^k(|u(s)|^2)}_{L_x^\infty}\\
%&\les \normo{ \mathcal F^{-1}(\rho_N \bra{\xi}^{-k})}_{L_x^1} \normo{\bra{D}^k(|u(s)|^2)}_{L_x^\infty}\\
%& \les\bra{N}^{-k} \|u(s)\|_{W^{k,\infty}}^2 \les \bra{N}^{-k}\bra{s}^{-2}\ve_1^2.
%\end{align*}
%By Young's inequality, we also get
%\begin{align*}
%\|P_N\left(|u(s)|^2\right)\| \les \normo{\rho_N \wh{|u(s)|^2}}_{L_\xi^1} \les \| \rho_N \|_{L_\xi^1} \left\|\wh{|u(s)|^2}\right\|_{L_\xi^\infty}\les N^2 \|u_0\|_{L_x^2}^2.
%\end{align*}
%We move on to the proof of \eqref{eq:norm-two}.By Plancherel's theorem and H\"older inequality, we see that
%\begin{align*}
%	&\left\|P_{N}\Big(|u(s)|^2\Big)\right\|_{L_x^{2}}^{2} 
%	\les\int_{\R^2}\rho_{N}^{2}(\xi)\abs{\mathcal{F}\Big(|u(s)|^2\Big)(\xi)}^2 d\xi\\
%	& \les \int_{\R^2}\rho_{N}^{2}(\xi) \left| \int_{\R^3} \langle\eta\rangle^{-k}\langle\xi-\eta\rangle^{-k}  \widehat{ \bra{D}^{k}u}(s,\eta)\widehat{\langle D\rangle^{k}u}(s,\xi-\eta) \, d\eta \right|^2 d\xi\\ 
%	& \les \left\Vert \widehat{\langle D\rangle^{k}u(s)}\right\Vert _{L_x^{\infty}}^2
%	 \int_{\R^2}\rho_{N}^{2}(\xi) \left|\int_{\R^2}\langle\eta\rangle^{-10}\langle\xi-\eta\rangle^{-10} d\eta \right|^{2}d\xi\\
%	& \les N^{2}\langle N\rangle^{-k} \left\Vert \widehat{\langle D\rangle^{k}u(s)}\right\Vert _{L_x^{\infty}}^2.
%\end{align*}
%This finishes the proof.
%\end{proof}	

%%%%%%%%%%%%%%%%%%%%%%%%%%%%%%%%%%%%%%%%%%%%%%%%%%%%%%%%%%%%%%%%%%%%%%%%%%%%%%%%%%%%%%%%%%%%%%%%%%%%%%%%%%%%%%%%%%%%%%%%%%%%%%%%%%%%%%%%%%%%%
%%%%%%%%%%%%%%%%%%%%%%%%%%%%%%%%%%%%%%%%%%%%%%%%%%%%%%%%%%%%%%%%%%%%%%%%%%%%%%%%%%%%%%%%%%%%%%%%%%%%%%%%%%%%%%%%%%%%%%%%%%%%%%%%%%%%%%%%%%%%%

\section{\label{sec:Weighted-Energy-estimate}Weighted Energy estimate}
In this section, we prove the energy estimate
which plays an important role in the bootstrap argument. In the following Proposition \ref{prop-energy}, we bound the weighted norms $\|u\|_{E_{1}},\|u\|_{E_{2}}$ in the a priori assumption \eqref{assumption-apriori}.

\begin{prop}[Weighted energy estimate]\label{prop-energy} Assume
that $u\in C([0,T],H^{n})$ satisfies the a priori assumption \eqref{assumption-apriori}
for some $\ve_{1}>0$ with initial data condition \eqref{condition-initial:semi}
for $\ve_{0}>0$. Suppose the index conditions \eqref{eq:bundle}. Then, we have the following estimates
\begin{align}
 & \sup_{t\in[0,T]}\langle t\rangle^{-\de_{0}}\|u(t)\|_{E_{1}}\le\ve_{0}+C\ve_{1}^{3},\label{eq:first-moment}\\
 & \sup_{t\in[0,T]}\langle t\rangle^{-2\de_{0}}\|u(t)\|_{E_{2}}\le\ve_{0}+C\ve_{1}^{3},\label{eq:second-moment}
\end{align}
%\begin{align}
% & \sum_{t\in[0,T]}\langle t\rangle^{-\de_{0}}\|u(t)\|_{H^{n}}\le\ve_{0}+C\ve_{1}^{3},\label{estimate-n}\\
% & \sum_{t\in[0,T]}\langle t\rangle^{-\de_{0}}\|x e^{ it\brad}u(t)\|_{L_x^2}\le\ve_{0}+C\ve_{1}^{3},\label{estimate-1}\\
% & \sum_{t\in[0,T]}\langle t\rangle^{-2\de_0}\|x^2 e^{ it\brad}u(t)\|_{L_x^2}\le\ve_{0}+C\ve_{1}^{3},\label{estimate-2}
%\end{align}
with $\delta_0=\frac{1}{100}$.
\end{prop}


\subsection{Useful inequalities}\label{sub:norm bound}
 For the purpose of proving weighted energy estimates, we introduce some useful inequalities.
\begin{lemma}[Lemma 3.2 in \cite{choz2006-siam}] 
    For any $\mathbb{C}$-valued functions $u\in L^{2}(\R^{2})\cap L^{\infty}(\R^{2})$,
    we get 
    \begin{align}\label{eq:hls}
    \left\||x|^{-1}*(|u|^{2})\right\|_{L^{\infty}(\mathbb{R}^{2})} & \les\|u\|_{L^{2}}\|u\|_{L^{\infty}}.
    \end{align}
    \end{lemma} 
Under the a priori assumption, we find the bounds for the frequency localized terms.
\begin{lemma}\label{lem:norm-esti} Let
    $u$ satisfy the a priori assumption \eqref{assumption-apriori} for some   $\ve_{1}>0$. Suppose the index conditions \eqref{eq:bundle}. Then, for $0\le\gamma\le1$ and a dyadic number $N\in2^{\mathbb{Z}}$,
    \begin{align}\label{PNuinfty}
  \|P_Nu(t)\|_{L^\infty(\mathbb{R}^{2})} \les N^{\gamma}\langle N\rangle^{-k(1-\gamma)}\langle t\rangle^{-(1-\gamma)}\ep_1.
    \end{align}
We also have 
\begin{align}
     \|P_Nf(t)\|_{L^2(\mathbb{R}^{2})}  &\les\min( N^\frac12 \langle t\rangle^{\frac12\delta_0}, \langle N\rangle^{-n})\ep_1  \label{PNxf},\\ 
    \|P_{N}xf(t)\|_{L^2(\mathbb{R}^{2})} &\les \min( N^\frac12 \langle t\rangle^{\frac32\delta_0}, \langle N\rangle^{-2}\langle t\rangle^{\delta_0})\ep_1.  \label{PNx2f}
\end{align}
\end{lemma}
\begin{proof}
By Young's inequality, we have from \eqref{eq:time-decay} that 
\begin{align*}
 \|P_N u(t)\|_{L^\infty(\mathbb{R}^{2})} \le  \left\|\mathcal{F}^{-1}\left( \rho_N\langle \xi\rangle^{-k}\right) \right\|_{L^1}\|u(t)\|_{W^{k,\infty}}
 \les \langle N \rangle^{-k}\langle t\rangle^{-1}\ep_1.
\end{align*} 
On the other hand, interpolating time decay estimates \eqref{eq:time-decay} and the mass conservation law \eqref{mass conservation}, we get for $2\le p\le\infty$, 
    \[
    \|P_Nu(t)\|_{L^{p}(\mathbb{R}^{2})}\les\|P_Nu(t)\|_{L^{\infty}}^{1-\frac{2}{p}}\|u_0\|_{L^{2}}^{\frac{2}{p}}\les\langle N\rangle^{-k(1-\frac2p)}\bra{t}^{-\left(1-\frac{2}{p}\right)}\ve_{1}^{2}.
    \]
Then, by Bernstein's inequality, we obtain for $2\le p<\infty$
\begin{align*}
\|P_N u(t)\|_{L^\infty(\R^2)} \les N^{\frac{2}{p}}\| P_Nu(t)\|_{L^p(\R^2)}
\les N^{\frac{2}{p}}\langle N\rangle^{-k(1-\frac2p)}\langle t\rangle^{-(1-\frac{2}{p})}\ep_1.
\end{align*}

Next, by using Bernstein's inequality and interpolating weighted norms, one can obtain  
\begin{align*}
    \| P_N f\|_{L^2(\R^2)} \les N^\frac12\|  f\|_{L^{\frac43}(\R^2)}
    \les N^\frac12 \|u_0\|_{L^2}^\frac12\|xf\|_{L^2}^\frac12
    \les N^\frac12\langle t\rangle^{\frac12\delta_0}\ep_1^2,
\end{align*}
or, one has
\begin{align*}
    \| P_N f\|_{L^2(\R^2)} \les \langle N\rangle^{-n}\|u\|_{H^n(\R^2)}.
\end{align*}
Then, \eqref{PNxf} follows by interpolating above two estimates.
The last inequality \eqref{PNx2f} can be obtained similarly.
\end{proof}

Next, we consider the quadratic terms.  
%The following lemma is 2 dimensional version of  \cite[Lemma~2.4]{CKLY2022}. 
\begin{lemma}\label{lem:quadratic terms} Let
    $u$ satisfy the a priori assumption \eqref{assumption-apriori} for some
    $\ve_{1}>0$ with the index conditions \eqref{eq:bundle}. For a dyadic number $N\in2^{\Z}$, we have 
    \begin{align}
    \left\Vert P_{N}\Big(|u(t)|^{2}\Big)\right\Vert _{L^{\infty}(\mathbb{R}^{2})} & \les\min(\langle N\rangle^{-k}\langle t\rangle^{-2},N^{2})\ve_{1}^{2},\label{eq:norm-infty}\\
    \left\Vert P_{N}\Big(|u(t)|^{2}\Big)\right\Vert _{L^{2}(\mathbb{R}^{2})} & \les N\langle N\rangle^{-\frac{k}{2}}\ve_{1}^{2}.\label{eq:norm-two}
    \end{align}
\end{lemma}
\begin{proof}
We refer to \cite[Lemma~2.4]{CKLY2022} where the three dimensional case is proved. 
\end{proof}
We observe that no time decay was obtained in \eqref{eq:norm-two}. In the following lemma, 
we find the time decay in \eqref{eq:norm-two} at the cost of derivative. 
Especially, the space resonance is employed to obtain the almost second order time decay $|t|^{-2+}$.
\begin{lemma} Let
    $u$ satisfy the a priori assumption \eqref{assumption-apriori} for some
    $\ve_{1}>0$ with the index conditions \eqref{eq:bundle}. Then, for a dyadic number $L\in2^{\Z}$, we have
\begin{align} 
    \| P_{N}|u(t)|^2 \|_{L^2(\R^2)} &\les \langle t \rangle^{-1+\delta_0}\langle N\rangle^{-k}\ep^2,  \nonumber \\ 
\| P_{N}|u(t)|^2 \|_{L^2(\R^2)} &\les \langle t \rangle^{-2+\frac{3}{2}\delta_0}N^{-1}\langle N\rangle^{-1}\ep^2. \label{ineq:PLu2}
\end{align}
\end{lemma}
\begin{proof}The Plancherel's identity yields that 
    \begin{align*}
        \| P_{N}|u(t)|^2 \|_{L^2(\R^2)}
        = \| \rho_N \mathcal{F} \big( |u(t)|^2\big) \|_{L^2(\R^2)}.
    \end{align*}
We write 
 \begin{align*}
    \rho_N(\eta)\mathcal{F} \big( |u(t)|^2\big) (\eta) &= \frac{1}{(2\pi)^2}\int_{\R^2}\rho_N(\eta) \widehat{u}(\sigma)\overline{\widehat{u}(\sigma+\eta)} d\sigma \\
    &=\frac{1}{(2\pi)^2}\int_{\R^2} \frac{\rho_N(\eta)}{\langle \sigma\rangle^k\langle\sigma+\eta\rangle^k}\widehat{\langle D\rangle^k u}(\sigma)\overline{\widehat{\langle D\rangle^k u}(\sigma+\eta)} d\sigma. 
\end{align*}
If we let 
$$
\mathbf{m}(\eta,\sigma):=\frac{\rho_N(\eta)}{\langle \sigma\rangle^k\langle\sigma+\eta\rangle^k},
$$ 
 one can verify that 
\begin{align*}
    C_{\mathbf{m}} \lesssim \langle N\rangle^{-k},
\end{align*}
where the constant $C_{\mathbf{m}}$ is defined in \eqref{multiplier bound}. Thus, by the Coifman-Meyer estimates \eqref{eq:coif-1}, we have 
\begin{align*}
    \| \rho_N\mathcal{F} \big( |u(t)|^2\big) \|_{L^2(\R^2)}
    \lesssim \langle N\rangle^{-k}\|u\|_{H^k}\|\langle D\rangle^{k} u\|_{L^\infty}  
    \lesssim \langle N\rangle^{-k} \langle t\rangle^{-1+\delta_0}\ep_1^2.
\end{align*}

%  We begin with the proof of \eqref{ineq:PLu2}.  The Plancherel's identity yields that 
%  \begin{align*}
%      \| P_{L}|u(t)|^2 \|_{L^2(\R^2)}
%      = \| \rho_L \mathcal{F} \big( |u(t)|^2\big) \|_{L^2(\R^2)}.
%  \end{align*}

 Next, we consider \eqref{ineq:PLu2}. we write 
\begin{align}\label{FT |u|^2}
    \mathcal{F} \big( |u(t)|^2\big) (\eta)
    = \frac{1}{(2\pi)^2}\int_{\R^2} e^{it(\langle \sigma+\eta\rangle - \langle\sigma\rangle )}\widehat{f}(t,\sigma)\overline{\widehat{f}(t,\eta+\sigma}) d\sigma,
\end{align}
where $f(t,x)=e^{it\langle D\rangle}u(t,x)$.
We perform an integration by parts to obtain 
\begin{align*}
    \mathcal{F} \big( |u(t)|^2\big) (\eta)
    =\frac{1}{(2\pi)^2}\Big( I_1(t,\eta)+I_2(t,\eta)+I_3(t,\eta) \Big),
\end{align*}
where
\begin{align*}
    I_1(t,\eta)&=\frac{i}{t}\int_{\R^2} \frac{\nabla_\sigma(\langle \sigma+\eta\rangle - \langle\sigma\rangle)}{|\nabla_\sigma(\langle \sigma+\eta\rangle - \langle\sigma\rangle)|^2}e^{-it\langle\sigma\rangle} \cdot  \widehat{xf}(t,\sigma)\overline{\widehat{u}(t,\eta+\sigma}) d\sigma , \\
    I_2(t,\eta)&=\frac{i}{t}\int_{\R^2} \frac{\nabla_\sigma(\langle \sigma+\eta\rangle - \langle\sigma\rangle)}{|\nabla_\sigma(\langle \sigma+\eta\rangle - \langle\sigma\rangle)|^2}e^{it\langle \sigma+\eta\rangle } \cdot  \widehat{u}(t,\sigma)\overline{\widehat{xf}(t,\eta+\sigma})  d\sigma, \\ 
    I_3(t,\eta)&= \frac{i}{t}\int_{\R^2} \nabla_\sigma \cdot \left(  \frac{\nabla_\sigma(\langle \sigma+\eta\rangle - \langle\sigma\rangle)}{|\nabla_\sigma(\langle \sigma+\eta\rangle - \langle\sigma\rangle)|^2} \right) \widehat{u}(t,\sigma)\overline{\widehat{u}(t,\eta+\sigma}) d\sigma.
\end{align*}

% \begin{align*}
%   \mathcal{F} \big( |u(t)|^2\big) (\eta)
%     &= I_1+I_2+I_3, \\ 
%     \text{ where } I_1&=\frac{i}{t}\int_{\R^2} \frac{\nabla_\sigma(\langle \sigma+\eta\rangle - \langle\sigma\rangle)}{|\nabla_\sigma(\langle \sigma+\eta\rangle - \langle\sigma\rangle)|^2}e^{it(\langle \sigma+\eta\rangle - \langle\sigma\rangle )} \cdot \nabla_\sigma \Big\{ \widehat{f}(t,\sigma)\overline{\widehat{f}(t,\eta+\sigma}) \Big\} d\sigma \\
%     &\qquad + \frac{i}{t}\int_{\R^2} \nabla_\sigma \cdot \left(  \frac{\nabla_\sigma(\langle \sigma+\eta\rangle - \langle\sigma\rangle)}{|\nabla_\sigma(\langle \sigma+\eta\rangle - \langle\sigma\rangle)|^2} \right) e^{it(\langle \sigma+\eta\rangle - \langle\sigma\rangle )}\widehat{f}(t,\sigma)\overline{\widehat{f}(t,\eta+\sigma}) d\sigma 
% \end{align*}

First, we consider $I_1$. We perform dyadic decomposition in the variables $\sigma$ and $\eta+\sigma$ to write 
\begin{align}\label{I1dyadic}
    \rho_N (\eta)I_1(t,\eta)&=\frac{i}{t}\sum_{(N_1,N_2)\in(2^{\Z})^2} I_1^{(N,N_1,N_2)}(t,\eta), \\ 
    I_1^{(N,N_1,N_2)}(t,\eta)&:=\frac{i}{t}\int_{\R^2}\mathbf{m}_{(N,N_1,N_2)}(\eta,\sigma)e^{-it\langle \sigma\rangle} \widehat{P_{N_1}xf}(t,\sigma) \overline{\widehat{P_{N_2}u}(t,\eta+\sigma}) d\sigma, \nonumber
\end{align}
where 
\begin{align*}
    \mathbf{m}_{(N,N_1,N_2)}(\eta,\sigma)=\frac{\nabla_\sigma(\langle \sigma+\eta\rangle - \langle\sigma\rangle)}{|\nabla_\sigma(\langle \sigma+\eta\rangle - \langle\sigma\rangle)|^2}\rho_N (\eta)\rho_{N_1} (\eta+\sigma)\rho_{N_2} (\sigma). 
\end{align*}
Since
\begin{align*}
    \left| \nabla_\sigma(\langle \sigma+\eta\rangle - \langle\sigma\rangle)\right|
  =\left|\frac{\eta+\sigma}{\bra{\eta+\sigma}}-\frac{\sigma}{\bra{\sigma}}\right|
    \gtrsim \frac{|\eta|}{\max(\langle \eta+\sigma\rangle, \langle\sigma\rangle) \min(\langle \eta+\sigma\rangle, \langle\sigma\rangle)^2}, 
\end{align*}
a direct computation yields that \footnote{For detailed computation, we refer to \cite{CKLY2022}.} 
\begin{align*}
  |\mathbf{m}_{(N,N_1,N_2)}(\eta,\sigma)| \les N^{-1}\max(\langle N_1\rangle, \langle N_2\rangle) \min(\langle N_1\rangle, \langle N_2\rangle)^2 
\end{align*}
and 
\begin{align}\label{cmbound}
    C_{\mathbf{m}_{(N,N_1,N_2)}} \les N^{-1}\max(\langle N_1\rangle, \langle N_2\rangle) \min(\langle N_1\rangle, \langle N_2\rangle)^{10}.
\end{align}
% From now, we denote $\max(\langle N_1\rangle, \langle N_2\rangle)$ and $\min(\langle N_1\rangle, \langle N_2\rangle)$ by $N_{min}$ and $N_{max}$, respectively.
Applying the operator inequality \eqref{eq:coif-1} with \eqref{cmbound}, we obtain 
\begin{align*}
    & \left\|I_1^{(N,N_1,N_2)}(t)\right\|_{L^2(\R^2)} \\ 
    &\qquad \les |t|^{-1}   N^{-1}\max(\langle N_1\rangle, \langle N_2\rangle) \min(\langle N_1\rangle, \langle N_2\rangle)^{10}
    \|P_{N_1}xf(t)\|_{L^2} \|P_{N_2}u(t)\|_{L^\infty}
\end{align*}
We observe that the sums in \eqref{I1dyadic} are actually taken over those indexes $(N_1,N_2)$ satisfying  
\begin{align*}
    N\les N_1\sim N_2 \; \text{ or } \; N_{\min}\ll N_{\max} \sim N,
\end{align*}
where $N_{\max} = \max(N_1, N_2)$ and $N_{\min} = \min(N_1, N_2)$. Thus, using Lemma~\ref{lem:norm-esti}, we estimate
\begin{align*}
 &\sum_{ N_1 \les N_2} \left\| I_1^{(N,N_1,N_2)}(t) \right\|_{L^2(\R^2)} \\    
 &\les |t|^{-1} N^{-1}  
 \Big( \sum_{N_{1}\ll N_2\sim N}  + \sum_{ N\les N_1\sim N_2} \Big)\langle N_2\rangle^{1-k} \langle N_{1}\rangle^{10} \min( N_1^\frac12 \langle t\rangle^{\frac32\delta_0}, \langle N_1\rangle^{-2} \langle t\rangle^{\delta_0}) \langle t\rangle^{-1}\ep_1^2 \\ 
 &\les \langle t\rangle^{-2+\frac32\delta_0}N^{-1}\langle N\rangle^{-1}
\end{align*}
and 
\begin{align*}
	\sum_{N_{2}\ll N_{1}\sim N} \left\| I_1^{(N,N_1,N_2)}(t) \right\|_{L^2(\R^2)}  & \les |t|^{-1}N^{-1}   
\sum_{N_{2}\ll N_{1}\sim N}  \langle N_1\rangle^{-1} \langle N_2\rangle^{10-k+k\frac{\delta_0}{2}}  \langle t\rangle^{\delta_0} N_2^{\frac{\delta_0}{2}}\langle t\rangle^{-1+\frac{1}{2}\delta_0}\ep_1^2 \\ 
&\quad \les \langle t\rangle^{-2+\frac{3}{2}\delta_0}N^{-1}   \langle N\rangle^{-1} 
\sum_{N_{2}\ll  N}  \langle N_2\rangle^{10-k+k\frac{\delta_0}{2}}  N_2^{\frac{\delta_0}{2}}\ep_1^2 \\ 
&\quad  \les \langle t\rangle^{-2+\frac32\delta_0}N^{-1}\langle L\rangle^{-1}.
\end{align*}
 By the symmetry, we may omit the proof of estimates for $I_2$.

For $I_3$, we also perform dyadic decomposition and write 
\begin{align*}
    \rho_N (\eta)I_3(t,\eta)&=\frac{i}{t}\sum_{N_1,N_2\in(2^{\Z})^2} I_3^{(N,N_1,N_2)}(t,\eta), \\ 
    I_3^{(N,N_1,N_2)}(t,\eta)&=\frac{i}{t}\int_{\R^2}\mathbf{m'}_{(N,N_1,N_2)}(\eta,\sigma)\widehat{P_{N_1}u}(t,\sigma) \overline{\widehat{P_{N_2}u}(t,\eta+\sigma}) d\sigma,
\end{align*}
where 
\begin{align*}
    \mathbf{m'}_{(N,N_1,N_2)}(\eta,\sigma)=\nabla_{\sigma}\cdot \left( \frac{\nabla_\sigma(\langle \sigma+\eta\rangle - \langle\sigma\rangle)}{|\nabla_\sigma(\langle \sigma+\eta\rangle - \langle\sigma\rangle)|^2}\right)\rho_N(\eta)\rho_{N_1} (\eta+\sigma)\rho_{N_2} (\sigma). 
\end{align*}
Then, one easily verifies that 
\begin{align*}
    |\mathbf{m'}_{(N,N_1,N_2)}(\eta,\sigma)| \les N^{-1}\max(\langle N_1\rangle, \langle N_2\rangle)\min(\langle N_1\rangle, \langle N_2\rangle)^3 
  \end{align*}
  and 
  \begin{align}\label{cmbound2}
      C_{\mathbf{m'}_{(N,N_1,N_2)}} \les N^{-1} \max(\langle N_1\rangle, \langle N_2\rangle)\min(\langle N_1\rangle, \langle N_2\rangle)^{11}.
  \end{align}
Applying the operator inequality \eqref{eq:coif-1} with \eqref{cmbound2}, we obtain 
\begin{align*}
	\begin{aligned}
		&\sum_{(N_1,N_2)\in(2^{\Z})^2}  \left\| I_3^{(N,N_1,N_2)}(t) \right\|_{L^2(\R^2)} \\ 
		&\;\les |t|^{-1}  N^{-1} \sum_{(N_1,N_2)\in(2^{\Z})^2}  \max(\langle N_1\rangle, \langle N_2\rangle)\min(\langle N_1\rangle, \langle N_2\rangle)^{11}
		\|P_{N_1}u\|_{L^2} \|P_{N_2}u(t)\|_{L^\infty}\\
    &\; \les |t|^{-2+\frac32\delta_0}N^{-1}\sum_{(N_1,N_2)\in(2^{\Z})^2}\
    \max(\langle N_1\rangle, \langle N_2\rangle)\min(\langle N_1\rangle, \langle N_2\rangle)^{11}
    N_1^\frac12  \langle N_1\rangle^{-n+\frac12} 
    N_2^{\delta_0}\langle N_2\rangle^{-k+k\delta_0}\ep_1^2 \\ 
    &\; \les \langle t\rangle^{-2+\delta_0}N^{-1}\langle N\rangle^{-1}.
\end{aligned}
\end{align*} 
Here we used Lemma~\ref{lem:norm-esti} in the second inequality. This finishes the proof.
\end{proof}

%%%%%%%%%%%%%%%%%%%%%%%%%%%%%%%%%%%%%%%%%%%%%%%%%%%%%%%%%%%%%%%%%%%%%%%%%%%%%%%%%%%%%%%%%%%%%%%%%%%%%%%%%%%%%%%%%%%%%%%%%%%%%%%%%%%%%%%%%%%%%%%%%%%%%%%%%%%%%%%%%%%%%%%%%%%%%%%%%%%%%%%%%%%%%%%%%%%%%%%%%%%%%%%%%%%%%%%%%%%%%%%%%%%%%%%%%%%%%%%%%%%%%%%%%%%%%%%%%%%%%%%%%%%%%%%%%%%%%%%%%%%%%%%%%%%%%%%%%%%%%%%%%%%%%%%%%%%%%%%%%%%%%%%%%%

\subsection{Proof of Proposition~\ref{prop-energy}} Now, we begin with the proof of Proposition~\ref{prop-energy}. 
\begin{proof}[Proof of \eqref{eq:first-moment}]
Let us first handle the high Sobolev norm in $\|u(t)\|_{E_{1}}$.
This can be bounded by Hardy-Littlewood-Sobolev inequality and \eqref{eq:hls}. Indeed, we first observe that by interpolation of the time decay estimates \eqref{eq:time-decay} and the conservation law \eqref{mass conservation}, we get for $2\le p\le\infty$, 
\[
\|u(t)\|_{L_{x}^{p}(\R^2)}\les\|u(t)\|_{L_{x}^{\infty}}^{1-\frac{2}{p}}\|u(t)\|_{L_{x}^{2}}^{\frac{2}{p}}\les\bra{t}^{-\left(1-\frac{2}{p}\right)}\ve_{1}^{2}.
\]
Then, we estimate 
\begin{align*}
\normo{|x|^{-1}*|u(t)|^{2}}_{L_{x}^{\infty}(\R^2)}\les\|u_{0}\|_{L_{x}^{2}}\|u(t)\|_{L_{x}^{\infty}}\les\bra{t}^{-1}\ve_{1}^{2}
\end{align*}
and 
\begin{align*}
\normo{\bra{D}^{n}\left(|x|^{-1}*|u(t)|^{2}\right)}_{L_x^4(\R^2)} & \les\|u(t)\|_{H^{n}}\|u(t)\|_{L_{x}^{4}}\les\bra{t}^{-\frac{1}{2}+\de_{0}}\ve_{1}^{2},
\end{align*}
which imply 
\[
\|u(t)\|_{H^{n}(\R^2)}\les\ve_{0}+C\bra{t}^{\de_{0}}\ve_{1}^{3}.
\]

Let us consider $\|xe^{it\bra{D}}u\|_{H^{2}(\R^2)}$. Note that $$\|xe^{it\jp D}u\|_{H^{2}}\sim\left\|\langle\xi\rangle^2\mathcal{F}\left(xe^{it\jp D}u\right)\right\|_{L_{\xi}^{2}}\sim\left\Vert \langle\xi\rangle^2\nabla_{\xi}\widehat{f}\,\right\Vert _{L_{\xi}^{2}}.$$
By the Duhamel's formula \eqref{eq:duhamel},  $\nabla_{\xi}\widehat{f}$ can be represented by 
\begin{align*}
\nabla_{\xi}\widehat{f}(t,\xi) & =\nabla_{\xi}\widehat{u_{0}}(\xi)+\frac{i\lam}{2\pi}\int_0^t\Big[\mathcal{I}^{1}(s,\xi)+\mathcal{I}^{2}(s,\xi)\Big]ds,
\end{align*}
where 
\begin{align*}
\mathcal{I}^{1}(s,\xi) & =\int_{\mathbb{R}^{2}}e^{is{\phi}(\xi,\eta)}|\eta|^{-1}\nabla_{\xi}\widehat{f}(\xi-\eta)\mathcal{F}(|u|^{2})(\eta)\,d\eta,  \\
\mathcal{I}^{2}(s,\xi) & =is\int_{\mathbb{R}^{2}}\nabla_{\xi}\phi(\xi,\eta)e^{is\phi(\xi,\eta)}|\eta|^{-1}\widehat{f}(\xi-\eta)\mathcal{F}(|u|^{2})(\eta)\,d\eta.
\end{align*}
Here we defined a resonant function $\phi$:
\begin{align} \label{eq:pm}\phi(\xi,\eta) & =\langle\xi\rangle-\langle\xi-\eta\rangle, \;\;\text{ and }\;\;
 \nabla_{\xi}\phi(\xi,\eta)=\frac{\xi}{\langle\xi\rangle}-\frac{\xi-\eta}{\langle\xi-\eta\rangle}.
\end{align}
We estimate the contribution from $\mathcal{I}^{1}$ and $\mathcal{I}^{2}$
under the a priori assumption \eqref{assumption-apriori} as follows:
\[
\normo{\langle\xi\rangle^2\mathcal{I}^{1}(s,\xi)}_{L_{\xi}^{2}}+\normo{\langle\xi\rangle^2\mathcal{I}^{2}(s,\xi)}_{L_{\xi}^{2}}\les\langle s\rangle^{-1+\de_{0}}\ve_{1}^{3}.
\]
\noindent \uline{Estimates for \mbox{$\mathcal{I}^{1}$}}. By Lemma \ref{lem:norm-esti}
and a priori assumption \eqref{assumption-apriori}, we get 
\begin{align}\begin{aligned}\label{I1}
\normo{\langle\xi\rangle^2\mathcal{I}^{1}(s,\xi)}_{L_{\xi}^{2}} 
& \les  \| u(s)\|_{L^\infty} \| u(s)\|_{H^2}\|xf(s)\|_{H^2}   \les \bra{s}^{-1+\de_{0}}\ve_{1}^{3}.
\end{aligned} \end{align}

\noindent \uline{Estimates for \mbox{$\mathcal{I}^{2}$}}. We decompose the frequency variables $|\xi|,|\xi-\eta|$ into the dyadic pieces $N_{0},N_{1}\in2^{\Z}$, respectively.
We also divide $|\eta|$ associated to the potential into $N_2\in2^{\Z}$. Then, we write 
\begin{align*}
    \langle \xi\rangle^2\mathcal{I}^{2}(s,\xi)&=\sum_{\mathbf{N}:=(N_{0},N_{1},N_2)\in (2^{\Z})^3}\mathcal{I}_{\mathbf{N}}^{2}(s,\xi),\\
 \mathcal{I}_{\mathbf{N}}^{2}(s,\xi) &=is\int_{\mathbb{R}^{2}}\mathbf{m}_{\mathbf{N}}(\xi,\eta)e^{is\phi(\xi,\eta)}\widehat{P_{N_1}f}(\xi-\eta)\widehat{P_{N_2}(|u|^{2})}(\eta)\,d\eta,
\end{align*}
where 
\begin{align*}
    \mathbf{m}_{\mathbf{N}}(\xi,\eta) 
    = \langle \xi\rangle^2 |\eta|^{-1} \nabla_{\xi}\phi(\xi,\eta)\rho_{N_{0}}(\xi)\rho_{L}(\eta)\rho_{N_{1}}(\xi-\eta).
\end{align*}
We observe that the sums are actually taken over those indices $(N_{0},N_{1},N_2)$ in the following set 
\begin{align*}
 \mathcal{N}:= \left\{ (N_{0},N_{1},N_2)\in (2^{\Z})^3 \; | \; N_0\les N_1\sim N_2 \text{ or } N_0\sim \max(N_1,N_2)\right\}.
\end{align*}
Indeed,  the integral $\mathcal{I}_{\mathbf{N}}^{2}$ is zero for $(N_{0},N_{1},N_2)\notin \mathcal{N}$.


We can estimate $\mathcal{I}_{\mathbf{N}}^{2}(s)$ in two ways. 
First, we have by H\"older inequality
\begin{align}\label{I2:holder}
    \normo{\mathcal{I}_{\textbf{N}}^{2}(s,\xi)}_{L_{\xi}^{2}} 
\les |s|\normo{\textbf{m}_{\mathbf{N}}(\xi,\eta)}_{L_{\xi,\eta}^{\infty}}\|\rho_{N_{0}}\|_{L^{2}}\normo{P_{N_2}|u(s)|^{2}}_{L^{2}}\normo{P_{N_{1}}{f}(s)\,}_{L^{2}}.
\end{align}
On the other hand, by using the operator inequality \eqref{eq:coif-1} with $C_{\mathbf{m}_{\mathbf{N}}}$ satisfying
\begin{align*}
    C_{\mathbf{m}_{\mathbf{N}}}:= \left\Vert \iint_{\mathbb{R}^{2}\times\R^{2}}\mathbf{m}_{\mathbf{N}}(\xi,\eta)e^{ix\cdot\xi}e^{iy\cdot\eta}\,d\eta d\xi\right\Vert _{L_{x,y}^{1}(\mathbb{R}^{2}\times\R^{2})}<\infty ,
\end{align*}
we have 
\begin{align}\label{I2:CM}
    \normo{\mathcal{I}_{\textbf{N}}^{2}(s,\xi)}_{L_{\xi}^{2}} 
    \les  |s|C_{\mathbf{m}_{\mathbf{N}}}\normo{P_{N_2}|u(s)|^{2}}_{L^{\infty}}\normo{P_{N_{1}}{f}(s)\,}_{L^{2}}.
\end{align}
% Let us recall from Lemma~\ref{lem:quadratic terms} that 
% \begin{align*}
%     \left\Vert P_{N}\Big(|u(t)|^{2}\Big)\right\Vert _{L^{\infty}} & \les\min(\langle N\rangle^{-k}\langle t\rangle^{-2},N^{2})\ve_{1}^{2}, \\
%     \left\Vert P_{N}\Big(|u(t)|^{2}\Big)\right\Vert _{L^{2}} & \les N\langle N\rangle^{-\frac{k}{2}}\ve_{1}^{2},
% \end{align*}
% We see from the a priori assumption \eqref{assumption-apriori} that 
% \begin{align*}
%     \normo{P_{N_{1}}{f}(s)\,}_{L^{2}} 
%     \les N_1\langle N_1\rangle^{-k}\ep_1.
% \end{align*}
% In addition, 
From the following inequality
\begin{align}\label{bound of grad phi}
   \left|  \nabla_{\xi}\phi(\xi,\eta) \right| =\left|  \frac{\xi}{\langle\xi\rangle}-\frac{\xi-\eta}{\langle\xi-\eta\rangle} \right| 
   \les \frac{|\eta|}{\max(\langle\xi\rangle,\langle\xi-\eta\rangle)},
\end{align}
one can readily verify that 
\begin{align}\label{pointwise bound of m}
    \sup_{\xi,\eta\in\R^2}\left|\mathbf{m}_{\mathbf{N}}(\xi,\eta)\right|\les \langle N_0\rangle^2\max(\langle N_0\rangle, \langle N_1\rangle)^{-1}  \; \text{ and }  \;  C_{\mathbf{m}_{\mathbf{N}}}\les  \langle N_0\rangle^2\max(\langle N_0\rangle, \langle N_1\rangle)^{-1} .
\end{align} 
Now, we estimate the sum over those indexes $N_0$ such that $N_0\le\langle s\rangle^{-2}$ by applying \eqref{I2:holder} together with \eqref{eq:norm-two} and \eqref{PNxf}
\begin{align*}
 \sum_{\substack{(N_{0},N_{1},N_2)\in\mathcal{N} \\ N_0\le \langle s\rangle^{-2}}}\normo{\mathcal{I}_{\textbf{N}}^{2}(s)}_{L^{2}}
 &\les |s|\sum_{\substack{(N_{0},N_{1},N_2)\in\mathcal{N} \\ N_0\le \langle s\rangle^{-2}}}\langle N_0\rangle^2\normo{\textbf{m}_{\mathbf{N}}}_{L^{\infty}}\normo{P_{N_2}|u|^{2}(s)}_{L^{2}}\normo{P_{N_{1}}{f}(s)\,}_{L^{2}} \\ 
 &\les |s|\sum_{\substack{(N_{0},N_{1},N_2)\in\mathcal{N} \\ N_0\le \langle s\rangle^{-2}}}\langle N_0\rangle^2N_0\max(\langle N_0\rangle, \langle N_1\rangle)^{-1}  N_2\langle N_2\rangle^{-\frac{k}{2}}N_1\langle N_1\rangle^{-k}\ep_1^3 \\ 
 &\les \langle s\rangle^{-1+\delta_0}\ep_1^3,
\end{align*}
where in the second inequality we used Lemma~\ref{lem:quadratic terms}.
For the remaining contribution, we utilize \eqref{I2:CM} together with \eqref{eq:norm-infty} and \eqref{PNxf} to obtain 
\begin{align*}
    \sum_{\substack{(N_{0},N_{1},N_2)\in\mathcal{N} \\ N_0\ge \langle s\rangle^{-2}}}\normo{\mathcal{I}_{\textbf{N}}^{2}(s)}_{L^{2}}
    &\les |s|\sum_{N_0\ge \langle s\rangle^{-2}} \langle N_0\rangle^2\max(\langle N_0\rangle, \langle N_1\rangle)^{-1} \normo{P_{L}|u|^{2}(s)}_{L^{\infty}}\normo{P_{N_{1}}{f}(s)}_{L^{2}} \\ 
    &\les|s|\sum_{\substack{(N_{0},N_{1},N_2)\in\mathcal{N} \\N_0\ge \langle s\rangle^{-2},\;  N_2\le \langle s\rangle^{-1}}}\langle N_0\rangle^2\max(\langle N_0\rangle, \langle N_1\rangle)^{-1}N_2^2N_1\langle N_1\rangle^{-k}\ep_1^3\\ 
    &\;\;\; + \langle s\rangle^{-1}\sum_{\substack{(N_{0},N_{1},N_2)\in\mathcal{N} \\N_0\ge \langle s\rangle^{-2},\;  N_2\ge \langle s\rangle^{-1}}}\langle N_0\rangle^2\max(\langle N_0\rangle, \langle N_1\rangle)^{-1}\langle N_2\rangle^{-k}N_1\langle N_1\rangle^{-k}\ep_1^3 \\ 
    &\les \langle s\rangle^{-1+\delta_0}\ep_1^3.
\end{align*}
\end{proof}

% We suffice to estimate the summation over three cases: \textbf{Case(i)} $N_{0}\les N_{1}\sim L$, \textbf{Case (ii)} $N_{1}\ll N_{0}\sim L$,
% and \textbf{Case (iii)} $L\ll N_{0}\sim N_{1}$.

% \noindent\textbf{Case~(i)}: $N_{0}\les N_{1}\sim L$.
% From the above two inequalities \eqref{I2:holder} and \eqref{I2:CM} with \eqref{pointwise bound of m}, we estimate 
% \begin{align*}
%  & \sum_{ \textbf{Case (i)}}
%  \normo{\mathcal{I}_{\textbf{N}}^{2}(t,\xi)}_{L_{\xi}^{2}}\\
%  & \les\int_{0}^{t}s\sum_{\substack{\textbf{Case (i)}\\
% \{\textbf{N}:N_{0}\le\bra{s}^{-2}\}
% }
% }\langle N_1\rangle^{-1}\|\rho_{N_{0}}\|_{L^{2}}\normo{\textbf{m}_{\mathbf{N}}}_{L^{\infty}}\normo{P_{L}|u|^{2}(s)}_{L^{2}}\normo{P_{N_{1}}{f}(s)\,}_{L^{2}}ds\\
%  & \hspace{3cm}+\int_{0}^{t}s\sum_{\substack{\textbf{Case (i)}\\
% \{\textbf{N}:\bra{s}^{-2}\le N_{0}\}
% }
% }\langle N_1\rangle^{-1}\normo{P_{L}|u|^{2}(s)}_{L^{\infty}}\normo{P_{N_{1}}{f}(s)}_{L^{2}}ds\\
%  & \les\int_{0}^{t}\ve_{1}^{3} \bigg( s\sum_{\substack{\textbf{Case (i)}\\
% \{\textbf{N}:N_{0}\le\bra{s}^{-2}\}
% }
% }N_{0}L\bra{L}^{-\frac{k}{2}}\bra{N_{1}}^{-k-1}
% +\bra{s}^{-1}\sum_{\substack{\textbf{Case (i)}\\
% \{\textbf{N}:\bra{s}^{-2}\le N_{0}\}
% }
% }\bra{L}^{-k}N_{1}\bra{N_{1}}^{-k-1}\bigg) ds\\
%  & \les\int_{0}^{t}\bra{s}^{-1+\delta_0}\ve_{1}^{3}ds\les\bra{t}^{\de_{0}}\ve_{1}^{3},
% \end{align*}
% where we used \eqref{eq:norm-two} and \eqref{eq:norm-infty} in the second inequality.

% \noindent\textbf{Case (ii)}: $N_{1}\ll N_{0}\sim L$.
% From  \eqref{I2:CM} with \eqref{pointwise bound of m}, we estimate using \eqref{eq:norm-infty}
% \begin{align*}
%   \sum_{\textbf{Case (ii)}}\normo{\mathcal{I}_{\textbf{N}}^{2}(t,\xi)}_{L_{\xi}^{2}}
%  & \les\int_{0}^{t}s\ve_{1}^{3}\sum_{\textbf{Case (ii)}}\langle N_0\rangle^{-1}\min\left(L^{2},\bra{L}^{-k}\bra{s}^{-2}\right)N_{1}\bra{N_{1}}^{-k}ds\\
%  & \les\int_{0}^{t}s\ve_{1}^{3}\left(\sum_{L\le\bra{s}^{-1}}L^{2-\de_{0}}+\sum_{\bra{s}^{-1}\le L}L^{-\de_{0}}\bra{L}^{-k}\bra{s}^{-2}\right)ds\\
%  & \les\int_{0}^{t}\bra{s}^{-1+\de_{0}}\ve_{1}^{3}ds\les\bra{t}^{\de_{0}}\ve_{1}^{3}.
% \end{align*}
% \textbf{Case (iii)}: $L\ll N_{0}\sim N_{1}$.
% This can be similarly estimated as the above. We omit the details.

%\textcolor{red}{
%\begin{align*}
%	&\sum_{L \ll N_0 \sim N_1} \normo{ \mathcal I_{\textbf{S}}^2(s,\xi)}_{L_\xi^2}\\
%	&\les s \ve_1^3\sum_{L \ll N_0 \sim N_1}  \min \left( L^2 , \bra{L}^{-k}\bra{s}^{-2}\right)  N_1\bra{N_1}^{-k}\\
%	& \les s \ve_1^3  \left(\sum_{L \le \bra{s}^{-1}}  L^2 +  \sum_{ \bra{s}^{-1} \le L} \bra{s}^{-2} \right)\\
%	&\les \bra{s}^{-1+\de_0} \ve_1^3.  
%\end{align*}
%}
Let us move on to the proof of second weighted estimates \eqref{eq:second-moment}. 
\begin{proof}[Proof of \eqref{eq:second-moment}]
By Plancherel's theorem, we have
\[
\|x^{2}e^{it\jp D}u\|_{H^2}\sim\|\langle\xi\rangle^2\mathcal{F}(x^{2}e^{it\jp D}u)\|_{L^{2}}\sim\left\Vert \langle\xi\rangle^2\nabla_{\xi}^{2}\widehat{f}\,\right\Vert _{L^{2}}.
\]
The Duhamel's formula \eqref{eq:duhamel} implies that $\nabla_{\xi}^{2}\widehat{f}$
can be represented by 
\begin{align*}
    \langle\xi\rangle^2\nabla_{\xi}^{2}\widehat{f}(t,\xi) & =\langle\xi\rangle^2\nabla^{2}\widehat{u_{0}}(\xi)+\frac{i\lam}{2\pi}\sum_{j=1}^{4}\int_0^t\mathcal{J}^{j}(s,\xi)ds,
\end{align*}
where, by abusing the notation,
\begin{align}
\mathcal{J}^{1}(s,\xi) & =\langle\xi\rangle^2\int_{\mathbb{R}^{2}}e^{is\phi(\xi,\eta)}|\eta|^{-1}\nabla^{2}\widehat{f}(\xi-\eta)\mathcal{F}(|u|^{2})(\eta)d\eta ,\nonumber \\
\mathcal{J}^{2}(s,\xi) & =2is\langle\xi\rangle^2\int_{\mathbb{R}^{2}}\nabla_{\xi}\phi(\xi,\eta)e^{is\phi(\xi,\eta)}|\eta|^{-1}\nabla\widehat{f}(\xi-\eta)\mathcal{F}(|u|^{2})(\eta)d\eta ,\nonumber \\
\mathcal{J}^{3}(s,\xi) & =is\langle\xi\rangle^2\int_{\mathbb{R}^{2}}\nabla_{\xi}^2\phi(\xi,\eta)(\xi,\eta)e^{is\phi(\xi,\eta)}|\eta|^{-1}\widehat{f}(\xi-\eta)\mathcal{F}(|u|^{2})(\eta)d\eta ,\nonumber \\
\mathcal{J}^{4}(s,\xi) & =-s^{2}\langle\xi\rangle^2\int_{\mathbb{R}^{2}}\big( \nabla_{\xi}\phi(\xi,\eta)\big)^{2}e^{is\phi(\xi,\eta)}|\eta|^{-1}\widehat{f}(\xi-\eta)\mathcal{F}(|u|^{2})(\eta)d\eta. \label{eq:j4-esti}
\end{align}
Then we prove that 
\[
\sum_{j=1}^{4}\normo{\mathcal{J}^{j}(s,\xi)}_{L_{\xi}^{2}}\les\langle s\rangle^{-1+2\de_{0}}\ve_{1}^{3}, \quad \text{ for } j=1,\cdots,4.
\]
\noindent \uline{Estimates for \mbox{$\mathcal{J}^{1}$}}. By Lemma \ref{lem:norm-esti},
$\mathcal{J}^1$ can be bounded as in \eqref{I1}. Indeed, 
\begin{align*}
\normo{\langle\xi\rangle^2\mathcal{J}^{1}(s,\xi)}_{L_{\xi}^{2}(\R^2)} 
& \les  \| u(s)\|_{L^\infty} \| u(s)\|_{H^2}\|x^2f(s)\|_{H^2}   \les \bra{s}^{-1+2\de_{0}}\ve_{1}^{3}.
\end{align*}

\noindent \uline{Estimates for \mbox{$\mathcal{J}^{2}$}}.
Estimates for $\mathcal{J}^{2}$ can be done almost similarly to those for $\mathcal{I}^{2}$. 
If one follows the argument, the only difference is that the norm $\|P_{N_1}f(s)\|_{L^2}$ in inequalities \eqref{I2:holder} and \eqref{I2:CM} is replaced by the weighted norm $\|P_{N_1}xf(s)\|_{L^2}$, which can be easily dealt with once one utilizes \eqref{PNx2f}.


% Here, we encounter $\|P_{N_1}xf(s)\|_{L^2}$
% in \eqref{I2:holder} and \eqref{I2:CM} instead of $\|P_{N_1}f(s)\|_{L^2}$. 
% But, the desired bound can be obtained if one utilizes \eqref{PNxf}
% \begin{align*}
%     \|P_{N}xe^{it\langle D\rangle}u(t)\|_{L^2} \les \min( N^{\delta_0} \langle t\rangle^{\frac32\delta_0}, \langle N\rangle^{-2}\langle t\rangle^{\delta_0})\ep_1.
% \end{align*}

\medskip

\noindent \uline{Estimates for \mbox{$\mathcal{J}^{3}$}}.
% Estimates for $\mathcal{J}^{1}$ and $\mathcal{J}^{2}$ can be done almost similarly to those for $\mathcal{I}^{1}$ and $\mathcal{I}^{2}$, respectively. The only
% difference between them is the order of derivatives which falls on $\wh{f}$ and the $L^2$ norm of derivative of $f$ up to the second order is bounded in our function space. Thus, we
% omit the proof of estimates for $\mathcal{J}^{1}$ and $\mathcal{J}^{2}$. In addition, 
$\mathcal{J}^{3}$ also can be handled in a similar manner to $\mathcal{I}^{2}$.  
Indeed, one finds that the multiplier $\nabla_{\xi}^2\phi(\xi,\eta)$ in $\mathcal{J}^{3}$ verifies the smaller bound than the one given in \eqref{bound of grad phi} satisfied by the multiplier $\nabla_{\xi}\phi(\xi,\eta)$ in $\mathcal{I}^{2}$. 
More precisely, the following bound holds
\begin{align*}
    \left| \nabla_{\xi}^2\phi(\xi,\eta)\right| 
    \les  \frac{|\eta|}{\max(\langle\xi\rangle,\langle\xi-\eta\rangle)^2}.
\end{align*}
Thus, if we let 
\begin{align*}
    \widetilde{\mathbf{m}}_{\mathbf{N}}(\xi,\eta) 
    =\langle \xi \rangle^2  |\eta|^{-1}  \nabla_{\xi}^2\phi(\xi,\eta)\rho_{N_{0}}(\xi)\rho_{N_2}(\eta)\rho_{N_{1}}(\xi-\eta) ,
\end{align*}
one can show that 
\begin{align*}
    \sup_{\xi,\eta\in\R^2}\left|\widetilde{\mathbf{m}}_{\mathbf{N}}(\xi,\eta)\right|&\les \langle N_0\rangle^2 \max(\langle N_0\rangle, \langle N_1\rangle)^{-2}, \\ C_{\widetilde{\mathbf{m}}_{\mathbf{N}}} &\les  \langle N_0\rangle^2\max(\langle N_0\rangle, \langle N_1\rangle)^{-2}.
\end{align*} 
Applying these bounds into \eqref{I2:holder} and \eqref{I2:CM}, one can obtain the desired bounds.

\medskip

\noindent \uline{Estimates for \mbox{$\mathcal{J}^{4}$}}.
It remains to estimate the main case $\mathcal{J}^{4}$. 
As before, we decompose 
\begin{align*}
    \mathcal{J}^{4 }(s,\xi)&=\sum_{\textbf{N}=(N_{0},N_{1},N_2)\in (2^{\Z})^3}\mathcal{J}_{\textbf{N}}^{4}(s,\xi),\\
 \mathcal{J}_{\textbf{N}}^{4}(s,\xi) &:=-s^2\int_{\mathbb{R}^{2}}\mathbf{m}_{\mathbf{N}}(\xi,\eta)e^{is\phi(\xi,\eta)}\widehat{P_{N_1}f}(\xi-\eta)\widehat{P_{N_2}(|u|^{2})}(\eta)\,d\eta,
\end{align*}
where 
\begin{align*}
    \mathbf{m}_{\mathbf{N}}(\xi,\eta) 
=\langle \xi\rangle^2   |\eta|^{-1} \left( \nabla_{\xi}\phi(\xi,\eta)\right)^2\rho_{N_{0}}(\xi)\rho_{N_2}(\eta)\rho_{N_{1}}(\xi-\eta).
\end{align*}
One can readily verify that  
\begin{align}\begin{aligned}\label{pointwise bound of m2}
    \sup_{\xi,\eta\in\R^2}\left|\mathbf{m}_{\mathbf{N}}(\xi,\eta)\right|&\les N_2\langle N_0\rangle^2\max(\langle N_0\rangle, \langle N_1\rangle)^{-2}, \\ 
     C_{\mathbf{m}_{\mathbf{N}}}&\les  N_2\langle N_0\rangle^2\max(\langle N_0\rangle, \langle N_1\rangle)^{-2}.
\end{aligned}\end{align} 
We see that the sum over those indexes $N_0$ such that $N_0\le \langle s\rangle^{-3}$ can be dealt with:
\begin{align*}
    \|\mathcal{J}_{\textbf{N}}^{4}(s)\|_{L^2(\R^2)} &\les |s|^2\sum_{\substack{ (N_0,N_1,L)\in \mathcal{N} \\ N_0 \le \langle s\rangle^{-3}}}\|\rho_{N_{0}}\|_{L^{2}}\normo{\textbf{m}_{\mathbf{N}}}_{L^{\infty}}\normo{P_{N_2}|u|^{2}(s)}_{L^{2}}\normo{P_{N_{1}}{f}(s)\,}_{L^{2}} \\ 
    &\les   |s|^2   \sum_{N_0 \le \langle s\rangle^{-3}}N_0N_2\langle N_0\rangle^2\max(\langle N_0\rangle, \langle N_1\rangle)^{-2}N_2\langle N_2\rangle^{-\frac{k}{2}}N_1\langle N_1\rangle^{-k}\ep_1^3\\ 
    &\les \langle s\rangle^{-1}\ep_1^3.
\end{align*}
On the other hand, by the multiplier inequalities with \eqref{pointwise bound of m2},
 one has 
\begin{align}\label{J4}
    \|\mathcal{J}_{\textbf{N}}^{4}(s)\|_{L^2(\R^2)} 
    \les   |s|^2   \sum_{\substack{ (N_0,N_1,L)\in \mathcal{N} \\ N_0 \ge \langle s\rangle^{-3}}}N_2\langle N_0\rangle^2\max(\langle N_0\rangle, \langle N_1\rangle)^{-2}\normo{P_{N_2}|u(s)|^{2}}_{L^{2}}\normo{P_{N_{1}}{u}(s)\,}_{L^{\infty}}.
\end{align}
Using \eqref{eq:norm-two}, we can bound the sum in \eqref{J4} for $N_2 \le \langle s\rangle^{-1+\frac34\delta_0}$ by 
\begin{align*}
    &|s|^2   \sum_{ \substack{ N_0 \ge \langle s\rangle^{-3}, \\  N_2 \le \langle s\rangle^{-1+\frac34\delta_0} }}N_2\langle N_0\rangle^2\max(\langle N_0\rangle, \langle N_1\rangle)^{-2} N_2
     N_1^{\frac14\delta_0}\langle N_1\rangle^{-k-\frac14\delta_0}\langle t\rangle^{-1+\frac14\delta_0}\ep_1^3  \\ 
     &\les \langle s\rangle^{-1+\frac74\delta_0} 
     \sum_{N_0\ge \langle s\rangle^{-3},\; N_0\les N_1 } \langle N_0\rangle^2\max(\langle N_0\rangle, \langle N_1\rangle)^{-2} 
     N_1^{\frac14\delta_0}\langle N_1\rangle^{-k-\frac14\delta_0}\ep_1^3 \\
     &\qquad +\langle s\rangle^{-1+\frac74\delta_0} \sum_{N_1\ll N_0 \les \langle s\rangle^{-1+\frac34\delta_0}} 
     N_1^{\frac14\delta_0} \ep_1^3 \\ 
     &\les  \langle s\rangle^{-1+2\delta_0}\ep_1^3.
   \end{align*}
To estimate the sum in \eqref{J4} for $L \ge \langle s\rangle^{-1+\frac34\delta_0}$, we use \eqref{ineq:PLu2} to obtain 
\begin{align*}
    &\langle s\rangle^{-1+\frac74\delta_0}   \sum_{ \substack{ N_0 \ge \langle s\rangle^{-3}, \\ N_2 \ge \langle s\rangle^{-1+\frac34\delta_0} }}\langle N_0\rangle^2\max(\langle N_0\rangle, \langle N_1\rangle)^{-2} \langle N_2\rangle^{-1}
    N_1^{\frac14\delta_0}\langle N_1\rangle^{-k-\frac14\delta_0}\ep_1^3  \\ 
    &\les \langle s\rangle^{-1+2\delta_0}\ep_1^3.
\end{align*}
\end{proof}

% We begin with rewriting  $\mathcal{J}^{4}$ as 
% \begin{align*}
% \mathcal{J}^{4}(s,\xi) =-s^{2}\iint_{\mathbb{R}^{2}\times \mathbb{R}^2}\big( \nabla_{\xi}\phi(\xi,\eta)\big)^{2}e^{isp(\xi,\eta,\sigma)}|\eta|^{-1}\widehat{f}(\xi-\eta)\wh{f}(\eta+\sigma)\overline{\wh{f}(\sigma)}d\eta d\sigma ,
% \end{align*}
% where the resonance function is 
% \[
% p(\xi,\eta,\sigma)=\bra{\xi}-\bra{\xi-\eta}-\bra{\eta+\sigma}+\bra{\sigma}
% \]
% which was already introduced in \eqref{eq:resonance-ftn}.
% We observe non-resonance of
% phase function $p$, in other words, we verify that 
% \begin{align}\label{lower bound of nabla p}
%     \left|\nabla_{\sigma}p(\xi,\eta,\sigma)\right|  
%     =\left|\frac{\eta+\sigma}{\bra{\eta+\sigma}}-\frac{\sigma}{\bra{\sigma}}\right|
%     \gtrsim \frac{|\eta|}{\max(\langle \eta+\sigma\rangle, \langle\sigma\rangle) \min(\langle \eta+\sigma\rangle, \langle\sigma\rangle)^2}.
% \end{align}
% We proceed by decomposing the support of variables $|\xi|,|\xi-\eta|,|\eta+\sigma|,|\sigma|,|\eta|$  in the integral into dyadic numbers $N_{0},N_{1},N_{2},N_{3},L\in2^{\Z}$, respectively. 
% \begin{align*}
% \mathcal{J}^{4}(s,\xi)=\sum_{\textbf{S}=(N_{0},N_{1},N_{2},N_{3},L)\in\Z^5}\mathcal{J}_{\textbf{S}}^{4}(s,\xi),
% \end{align*}
% where 
% \begin{align*}
% %\mathcalJ_{\textbf{N}}^{2}(t,\xi)
% %
% \mathcal{J}_{\textbf{S}}^{4}(s,\xi) & =-s^{2}\iint_{\mathbb{R}^{2}\times\R^{2}}\rho_{N_{0}}(\xi)\big( \nabla_{\xi}\phi(\xi,\eta)\big)^{2}e^{is{p}(\xi,\eta,\sigma)}\rho_{L}(\eta)|\eta|^{-1}\\
%  & \hspace{4cm}\times\widehat{f_{N_{1}}}(s,\xi-\eta)\wh{f_{N_{2}}}(s,\eta+\sigma)\overline{\wh{f_{N_{3}}}(s,\sigma)}d\sigma d\eta,
% \end{align*}
% and $f_{N_j}$ denote $P_{N_j}f$ for $j=1,2,3$.
% % Simple calculation implies that when
% % $N_{2}\sim N_{3}$, by the mean value theorem, 
% % \begin{align}
% % \begin{aligned}\left|\nabla_{\sigma}p(\xi,\eta,\sigma)\right| & =\left|\frac{\eta+\sigma}{\bra{\eta+\sigma}}-\frac{\sigma}{\bra{\sigma}}\right|\gtrsim\frac{L}{\bra{N_{2}}^{3}},\end{aligned}
% % \label{eq:nonresonance-space}
% % \end{align}
% % and when $N_{2}\nsim N_{3}$, 
% % \begin{align*}
% % \begin{aligned}\left|\nabla_{\sigma}p(\xi,\eta,\sigma)\right| & \gtrsim\left|\frac{N_{2}}{\bra{N_{2}}}-\frac{N_{3}}{\bra{N_{3}}}\right|\ge\frac{|N_{2}^{2}-N_{3}^{2}|}{\bra{N_{2}}\bra{N_{3}}\left(N_{2}\bra{N_{3}}+N_{3}\bra{N_{2}}\right)}\\
% %  & \sim\frac{\max(N_{2},N_{3})}{\bra{N_{2}}\bra{N_{3}}\min(\bra{N_{2}},\bra{N_{3}})}.
% % \end{aligned}
% % \end{align*}
% % If $N_{2}\nsim N_{3}$, by the support relation of $\eta,\sigma,\eta+\sigma$,
% % we have 
% % \[
% % \min(N_{2},N_{3})\ll\max(N_{2},N_{3})\sim L.
% % \]
% % Hence we get 
% % \begin{align}
% % |\nabla_{\sigma}p(\xi,\eta,\sigma)|\gtrsim\frac{L}{\min(\bra{N_{2}},\bra{N_{3}})^{3}}.\label{eq:phase-sigma}
% % \end{align}
% We utilize the space resonance \eqref{lower bound of nabla p} via the integration
% by parts as follows: Using the relation 
% \[
% e^{isp(\xi,\eta,\sigma)}=-i\frac{1}{s}\frac{(\nabla_{\sigma}p)\cdot\nabla_{\sigma}e^{isp}}{|\nabla_{\sigma}p|^{2}},
% \]
% we perform an integration by parts in $\sigma$ variable
% \begin{align*}
%   \mathcal{J}_{\textbf{S}}^{4}(s,\xi)
%  & =is\int_{\mathbb{R}^{2}}\rho_{N_{0}}(\xi)\big( \nabla_{\xi}\phi(\xi,\eta)\big)^{2}\frac{\nabla_{\sigma}p(\xi,\eta,\sigma)}{|\nabla_{\sigma}p(\xi,\eta,\sigma)|^{2}}\cdot\nabla_{\sigma}e^{is{p}(\xi,\eta,\sigma)}\\
%  & \hspace{3cm}\times\rho_{L}(\eta)|\eta|^{-1}\widehat{f_{N_{1}}}(s,\xi-\eta)\wh{f_{N_{2}}}(s,\eta+\sigma)\overline{\wh{f_{N_{3}}}(\sigma)}d\sigma d\eta \\
%  & =-is\int_{\mathbb{R}^{2}}\rho_{N_{0}}(\xi)\big( \nabla_{\xi}\phi(\xi,\eta)\big)^{2}\nabla_{\sigma}\cdot\left(\frac{\nabla_{\sigma}p(\xi,\eta,\sigma)}{|\nabla_{\sigma}p(\xi,\eta,\sigma)|^{2}}\right)e^{is{p}(\xi,\eta,\sigma)}\\
%  & \hspace{3cm}\times\rho_{L}(\eta)|\eta|^{-1}\widehat{f_{N_{1}}}(s,\xi-\eta)\wh{f_{N_{2}}}(s,\eta+\sigma)\overline{\wh{f_{N_{3}}}(\sigma)}d\sigma d\eta \\
%  & \quad-is\int_{\mathbb{R}^{2}}\rho_{N_{0}}(\xi)\big( \nabla_{\xi}\phi(\xi,\eta)\big)^{2}\frac{\nabla_{\sigma}p(\xi,\eta,\sigma)}{|\nabla_{\sigma}p(\xi,\eta,\sigma)|^{2}}e^{is{p}(\xi,\eta,\sigma)}\\
%  & \hspace{3cm}\times\rho_{L}(\eta)|\eta|^{-1}\widehat{f_{N_{1}}}(s,\xi-\eta)\nabla_{\sigma}\wh{f_{N_{2}}}(s,\eta+\sigma)\overline{\wh{f_{N_{3}}}(\sigma)}d\sigma d\eta \\
%  & \quad-is\int_{\mathbb{R}^{2}}\rho_{N_{0}}(\xi)\big( \nabla_{\xi}\phi(\xi,\eta)\big)^{2}\frac{\nabla_{\sigma}p(\xi,\eta,\sigma)}{|\nabla_{\sigma}p(\xi,\eta,\sigma)|^{2}}e^{is{p}(\xi,\eta,\sigma)}\\
%  & \hspace{3cm}\times\rho_{L}(\eta)|\eta|^{-1}\widehat{f_{N_{1}}}(s,\xi-\eta)\wh{f_{N_{2}}}(s,\eta+\sigma)\nabla_{\sigma}\overline{\wh{f_{N_{3}}}(\sigma)}d\sigma d\eta \\
%  & =:\mathcal{J}_{\textbf{S}}^{4,1}(s,\xi)+\mathcal{J}_{\textbf{S}}^{4,2a}(s,\xi)+\mathcal{J}_{\textbf{S}}^{4,2b}(s,\xi).
% \end{align*}
% We define 
% \begin{align*}
% \mathfrak{M}_{1,\textbf{S}}(\xi,\eta,\sigma) & =\big( \nabla_{\xi}\phi(\xi,\eta)\big)^{2}|\eta|^{-1}\nabla_{\sigma}\left(\frac{\nabla_{\sigma}p(\xi,\eta,\sigma)}{|\nabla_{\sigma}p(\xi,\eta,\sigma)|^{2}}\right)\\
%  & \hspace{3cm}\times\rho_{N_{0}}(\xi)\rho_{N_{1}}(\xi-\eta)\rho_{N_{2}}(\eta+\sigma)\rho_{N_{3}}(\sigma)\rho_{L}(\eta),\\
% \mathfrak{M}_{2,\textbf{S}}(\xi,\eta,\sigma) & =\big( \nabla_{\xi}\phi(\xi,\eta)\big)^{2}|\eta|^{-1}\frac{\nabla_{\sigma}p(\xi,\eta,\sigma)}{|\nabla_{\sigma}p(\xi,\eta,\sigma)|^{2}}\\
%  & \hspace{3cm}\times\rho_{N_{0}}(\xi)\rho_{N_{1}}(\xi-\eta)\rho_{N_{2}}(\eta+\sigma)\rho_{N_{3}}(\sigma)\rho_{L}(\eta).
% \end{align*}
% One can verify that 
% \begin{align*}
% \sup_{\xi,\eta,\sigma\in\R^2}|\mathfrak{M}_{1,\textbf{S}}(\xi,\eta,\sigma)| &\les \max(\langle N_0\rangle,\langle N_1\rangle )^{-2}  \min(N_2,N_3)^4,  \\ 
% \sup_{\xi,\eta,\sigma\in\R^2}|\mathfrak{M}_{2,\textbf{S}}(\xi,\eta,\sigma)| &\les \max(\langle N_0\rangle,\langle N_1\rangle )^{-2}  \max(N_2,N_3)\min(N_2,N_3)^2 
% \end{align*}
% and 
% %\textcolor{red}{
% %\begin{align*}
% %	&\left| \nabla_\sigma p(\xi,\eta,\sigma) \right| = \left| \frac{\eta +\sigma}{\bra{\eta +\sigma}} - \frac{\sigma}{\bra{\sigma}} \right| = \left| \frac{(\eta+\sigma)\bra{\sigma} - \sigma \bra{\eta +\sigma}}{\bra{\eta+\sigma}\bra{\sigma}} \right| = \left| \frac{\eta}{\bra{\eta+\sigma}} - \frac{\sigma(\eta \cdot (\eta +2\sigma))}{\bra{\eta+\sigma}\bra{\sigma}(\bra{\eta+\sigma} + \bra{\sigma})}  \right|\\
% %	&\ge \frac{|\eta|}{\bra{\eta+\sigma}} - \frac{|\sigma||\eta|}{\bra{\eta+\sigma}\bra{\sigma}}  \ge \frac{|\eta|}{\bra{\eta+\sigma}\bra{\sigma}}, \qquad (\because \mbox{triangle  inequality})   \\
% %	&\left| \nabla_\sigma^2 p(\xi,\eta,\sigma) \right| = \left| \frac1{\bra{\eta+\sigma}^3} - \frac1{\bra{\sigma}^3} \right|   \les \frac{L}{\max(\bra{N_2}^4, \bra{N_3}^4)} \qquad (\because  \mbox{the mean value theorem})
% %\end{align*}}
% \begin{align}
% \begin{aligned}\normo{\iiint_{\R^{2}\times\R^{2}\times\R^{2}}\mathfrak{M}_{1,\textbf{S}}(\xi,\eta,\sigma)e^{ix\cdot\xi}e^{iy\cdot\eta}e^{iz\cdot\sigma}d\sigma d\eta d\xi}_{L_{x,y,z}^{1}} & \les\frac{\min(\bra{N_{2}},\bra{N_{3}})^{10}}{\max(\bra{N_{0}},\bra{N_{1}})^{2}},\\
% \normo{\iiint_{\R^{2}\times\R^{2}\times\R^{2}}\mathfrak{M}_{2,\textbf{S}}(\xi,\eta,\sigma)e^{ix\cdot\xi}e^{iy\cdot\eta}e^{iz\cdot\sigma}d\sigma d\eta d\xi}_{L_{x,y,z}^{1}} & \les\frac{\min(\bra{N_{2}},\bra{N_{3}})^{9}}{\max(\bra{N_{0}},\bra{N_{1}})^{2}}.
% \end{aligned}
% \label{eq:multiplier}
% \end{align}
% In the view of \eqref{eq:multiplier}, the multiplier bound for $\mathfrak{M}_{1,\textbf{S}}$
% is similar to that for $\mathfrak{M}_{2,\textbf{S}}$ except for order
% of $\bra{N_{2}}$ and $\bra{N_{3}}$ which does not play a crucial role.
% Since the derivative $\nabla_{\sigma}$ does not fall on any of $\wh{f_{N_{1}}}$,$\wh{f_{N_{2}}}$,
% or $\wh{f_{N_{3}}}$ in the integrand of $\mathcal{J}_{\textbf{S}}^{4,1}$,
% we can estimate $\mathcal{J}_{\textbf{S}}^{4,1}$ easier than $\mathcal{J}_{\textbf{S}}^{4,2a}$
% and $\mathcal{J}_{\textbf{S}}^{4,2b}$. For these reasons, we omit
% the estimates for $\mathcal{J}_{\textbf{S}}^{4,1}$. In addition,
% by a symmetry between $f_{N_{2}}$ and $f_{N_{3}}$, we may only focus
% on the estimates for $\mathcal{J}_{\textbf{S}}^{4,2a}$ as the estimates
% for $\mathcal{J}_{\textbf{S}}^{4,2b}$ can be obtained in a similar
% manner.

% Now we perform the estimates for $\mathcal{J}_{\textbf{S}}^{4,2a}$.
% \begin{align*}
%   \normo{\mathcal{J}_{\textbf{S}}^{4,2a}(s,\xi)}_{L_{\xi}^{2}} \les |s|
%   \|\mathfrak{M}_{2,\textbf{S}}\|_{L^{\infty}}\|\rho_{N_{0}}\|_{L^{2}}\normo{\rho_{N_{1}}\wh{f}}_{L^{1}}\normo{\rho_{N_{2}}\wh{xf}}_{L^{2}}\normo{\rho_{N_{3}}\wh{f}}_{L^{2}}
% \end{align*}
% or
% \begin{align*}
%     \normo{\mathcal{J}_{\textbf{S}}^{4,2a}(s,\xi)}_{L_{\xi}^{2}} \les |s|   
%     C_{\mathfrak{M}_{2,\textbf{S}}}
%     \normo{P_{N_1}u}_{L^{\infty}}\normo{P_{N_{2}}(xf)}_{L^{2}}\normo{P_{N_3}u}_{L^{\infty}}
% \end{align*}

% \begin{align*}
%     \|P_Nu(s)\|_{L^\infty} \les \langle N\rangle^{-k}\|\langle D\rangle^k u\|_{L^\infty}.
% \end{align*}

% Using Bernstein inequality and a priori assumption \eqref{assumption-apriori},
% we see that for $2\le p\le\infty$, 
% \begin{align}
% \|P_{N}u(s)\|_{L_{x}^{\infty}}\les N^{\frac{2}{p}}\|P_{N}u(s)\|_{L_{x}^{p}}\les N^{\frac{2}{p}}\bra{N}^{-\left(1-\frac{2}{p}\right)k}\bra{s}^{-\left(1-\frac{2}{p}\right)}\ve_{1}.\label{eq:bernstein}
% \end{align}

% We suffice to consider three cases: \textbf{Case (i) $N_{0}\les N_{1}\sim L$},
% \textbf{Case (ii) $N_{1}\ll N_{0}\sim L$}, and \textbf{Case (iii)
% $L\ll N_{0}\sim N_{1}$}. In each case, we further take a closer look at the relations between the support of $|\eta|,|\eta+\sigma|$, and $|\sigma|$.

% \medskip

% \noindent\textbf{Case (i)}:$N_{0}\les N_{1}\sim L$.
% We further divide the case into \textbf{ Case (ia)} $N_{0}\les N_{1}\sim L\ll N_{2}\sim N_{3}$,
% \textbf{Case (ib)} $N_{0},N_{2}\les N_{1}\sim L\sim N_{3}$, and \textbf{Case
% (ic)} $N_{0},N_{3}\les N_{1}\sim L\sim N_{2}$. %   \begin{align*}
% %   	\textbf{Case (ia): }& N_0 \les N_1 \sim L \les N_2 \sim N_3,\\
% %   	\textbf{Case (ib): }& N_0, N_2 \les N_1 \sim L \sim N_3, \\
% %   	\textbf{Case (ic): }& N_0, N_3 \les N_1 \sim L \sim N_2,\\
% %   	\textbf{Case (iia): }& N_1 \ll N_0 \sim L \les N_2 \sim N_3,\\
% %   	\textbf{Case (iib): }& N_1,N_2 \les  N_0 \sim L \sim N_3,\\	\textbf{Case (iic): }& N_1,N_3 \les  N_0 \sim L \sim N_2,\\
% %   	\textbf{Case (iiia): }& L \ll N_0 \sim N_1 \mbox{ and } L \les N_2 \sim N_3\\
% %   	\textbf{Case (iiib): }& N_2 \ll L \sim N_3 \ll N_0 \sim N_1,\\
% %   	\textbf{Case (iiic): }& N_3 \ll L \sim N_2  \ll  N_0 \sim N_1.
% %   \end{align*}

% Let us treat the \textbf{Case (ia) $N_{0}\les N_{1}\sim L\ll N_{2}\sim N_{3}$}.
% By \eqref{eq:multiplier} and \eqref{eq:coif-2} with \eqref{eq:multiplier},
% we obtain 
% \begin{align*}
%  & \sum_{\textbf{Case (ia)}}\normo{\mathcal{J}_{\textbf{S}}^{4,2a}(s,\xi)}_{L_{\xi}^{2}}\\
%  & \les |s|\sum_{\substack{\textbf{Case (ia)}\\
% \{\textbf{S}:N_{0}\le\bra{s}^{-2}\}
% }
% }\|\mathfrak{M}_{2,\textbf{S}}\|_{L^{\infty}}\|\rho_{N_{0}}\|_{L^{2}}\normo{\rho_{N_{1}}\wh{f}}_{L^{1}}\normo{\rho_{N_{2}}\wh{xf}}_{L^{2}}\normo{\rho_{N_{3}}\wh{f}}_{L^{2}} \\
%  & \qquad+ |s| \sum_{\substack{\textbf{Case (ia)}\\
% \{\textbf{S}:N_{0}\ge\bra{s}^{-2}\}
% }
% }\frac{\bra{N_{2}}^{9}}{\bra{N_{1}}^{2}}\bra{N_{1}}^{-k}\bra{N_{2}}^{-k}\normo{u}_{W^{k,\infty}}\normo{P_{N_{2}}(xf)}_{L_{x}^{2}}\normo{u}_{W^{k,\infty}}\\
%  & \les \ve_{1}^{3} \bra{s}^{1+\de_{0}}\sum_{\substack{N_{0}\les N_{1}\les N_{2}\\
% N_{0}\le\bra{s}^{-2}
% }
% }N_{0}N_{1}^{2}\bra{N_{1}}^{-2-k}N_{2}\bra{N_{2}}^{3-k} \\ 
% &\qquad +\ve_{1}^{3}\bra{s}^{-1+\de_{0}}\sum_{\substack{N_{0}\les N_{1}\les N_{2}\\
% \bra{s}^{-2}\le N_{0}
% }
% }\bra{N_{1}}^{-2-k}\bra{N_{2}}^{9-k}  \les\bra{s}^{-1+2\de_{0}}\ve_{1}^{3}.
% \end{align*}
% To estimate for \textbf{Case (ib) $N_{0},N_{2}\les N_{1}\sim L\sim N_{3}$},
% we use H\"older inequality and Lemma \ref{lem:coif} for low and high
% frequency regime, respectively. Indeed, we see that 
% \begin{align*}
%  & \sum_{\textbf{Case (ib)}}\normo{\mathcal{J}_{\textbf{S}}^{4,2a}(t,\xi)}_{L_{\xi}^{2}}\\
%  & \les\int_{0}^{t}s\sum_{\substack{\textbf{Case (ib)}\\
% \{\textbf{S}:N_{0}\le\bra{s}^{-2}\}
% }
% }\|\mathfrak{M}_{2,\textbf{S}}\|_{L_{\eta}^{\infty}}\|\rho_{N_{0}}\|_{L_{\xi}^{2}}\normo{\rho_{N_{1}}\wh{f}}_{L_{\eta}^{1}}\normo{\rho_{N_{2}}}_{L_{\sigma}^{2}}\normo{\rho_{N_{2}}\wh{xf}}_{L_{\sigma}^{2}}\normo{\rho_{N_{3}}\wh{f}}_{L_{\sigma}^{\infty}}ds\\
%  & \hspace{0.5cm}+\int_{0}^{t}s\sum_{\substack{\textbf{Case (ib)}\\
% \{\textbf{S}:N_{0}\ge\bra{s}^{-2}\}
% }
% }\frac{\bra{N_{3}}^{9}}{\bra{N_{1}}^{2}}\bra{N_{1}}^{-k}\bra{N_{3}}^{-k}\normo{u}_{W^{k,\infty}}\normo{P_{N_{2}}(xf)}_{L_{x}^{2}}\normo{u}_{W^{k,\infty}}ds\\
%  & \les\int_{0}^{t}\ve_{1}^{3}\left(\bra{s}^{1+\de_{0}}\sum_{\substack{N_{0},N_{2}\les N_{1}\\
% N_{0}\le\bra{s}^{-2}
% }
% }N_{0}N_{1}^{2}\bra{N_{1}}^{1-2k}N_{2}ds\right.\\
%  & \hspace{4cm}\left.+\int_{0}^{t}\bra{s}^{-1+\de_{0}}\sum_{\substack{N_{0}\les N_{1}\les N_{2}\\
% \bra{s}^{-2}\le N_{0}
% }
% }\bra{N_{1}}^{-2-k}\bra{N_{3}}^{9-k}\right)ds\\
%  & \les\bra{t}^{2\de_{0}}\ve_{1}^{3}.
% \end{align*}
% %\begin{align*}
% %	&\sum_{\substack{N_0 \les N_1 \sim L\\ N_2 \sim N_3 \\ \bra{t}^{-2} \le N_0}}\normo{ \mathcal J_\textbf{S}^{4,2a}(t,\xi)}_{L_\xi^2}\\
% %	& \les \int_0^t s \sum_{\substack{N_0 \les N_1 \sim L\\ N_2 \sim N_3 \\ \bra{s}^{-2} \le N_0}}  \frac{\bra{N_2}^9}{\bra{N_1}^{2}}\bra{N_1}^{-k} \bra{N_2}^{-k} \normo{u}_{W^{k,\infty}}\normo{x f_{2}}_{L_x^2} \normo{u}_{W^{k,\infty}} ds\\
% %	& \les \int_0^t \bra{s}^{-1+\de_0} \ve_1^3\sum_{\substack{N_0 \les N_1 \les N_2 \\ \bra{s}^{-2} \le N_0}}  \bra{N_1}^{-2-k} \bra{N_2}^{9-k}  ds\\
% %	&\les \bra{t}^{2\de_0} \ve_1^3.  
% %\end{align*}
% By the symmetry between $f_{2}$ and $f_{3}$, we may omit the estimates
% for the \textbf{Case (ic)}.

% \noindent\textbf{\uline{Estimate of Case (ii).}}

% We further decompose it into subcases; \textbf{Case (iia) $N_{1}\ll N_{0}\sim L\les N_{2}\sim N_{3}$},
% \textbf{Case (iib) $N_{1},N_{2}\les N_{0}\sim L\sim N_{3}$}, and
% \textbf{Case (iic) $N_{1},N_{3}\les N_{0}\sim L\sim N_{2}$}.

% Using Bernstein inequality and a priori assumption \eqref{assumption-apriori},
% we see that for $2\le p\le\infty$, 
% \begin{align}
% \|P_{N}u(s)\|_{L_{x}^{\infty}}\les N^{\frac{2}{p}}\|P_{N}u(s)\|_{L_{x}^{p}}\les N^{\frac{2}{p}}\bra{N}^{-\left(1-\frac{2}{p}\right)k}\bra{s}^{-\left(1-\frac{2}{p}\right)}\ve_{1}.\label{eq:bernstein}
% \end{align}
% Then, by \eqref{eq:coif-2} with pair $(\infty,2,\infty)$ and \eqref{eq:bernstein}
% with $p=\frac{2}{\de_{0}}$, we estimate 
% \begin{align}
% \begin{aligned} & \sum_{\textbf{Case (iia)}}\normo{\mathcal{J}_{\textbf{S}}^{4,2a}(t,\xi)}_{L_{\xi}^{2}}\\
%  & \les\int_{0}^{t}s\sum_{\substack{N_{1}\ll N_{0}\sim L\\
% L\les N_{2}\sim N_{3}
% }
% }\frac{\bra{N_{2}}^{9}}{\bra{N_{0}}^{2}}\bra{N_{3}}^{-k}\normo{P_{N_{1}}u}_{L_{x}^{\infty}}\normo{P_{N_{2}}(xf)}_{L_{x}^{2}}\normo{P_{N_{3}}u}_{W^{k,\infty}}ds\\
%  & \les\int_{0}^{t}\bra{s}^{-1+2\de_{0}}\ve_{1}^{3}\sum_{N_{1}\ll N_{0}\les N_{2}}\bra{N_{0}}^{-2}N_{1}^{\de_{0}}\bra{N_{1}}^{-(1-\de_{0})k}\bra{N_{2}}^{9-k}ds\\
%  & \les\bra{t}^{2\de_{0}}\ve_{1}^{3}.
% \end{aligned}
% \label{eq:caseiia}
% \end{align}
% For \textbf{Case (iib)}, \eqref{eq:coif-2} with the same pair above
% implies the desired bound. To handle \textbf{Case (iic)}, we also
% utilize \eqref{eq:bernstein} with $p=\frac{8}{\de_{0}}$ for $P_{N_{1}}u$
% and $P_{N_{3}}u$, respectively. Indeed, by \eqref{eq:esti-x-hn},
% we deduce that 
% \begin{align*}
%  & \sum_{\textbf{Case (iic)}}\normo{\mathcal{J}_{\textbf{S}}^{4,2a}(t,\xi)}_{L_{\xi}^{2}}\\
%  & \les\int_{0}^{t}s\sum_{\substack{N_{1},N_{3}\les N_{0}\\
% N_{0}\sim L\sim N_{2}
% }
% }\frac{\bra{N_{2}}^{9}}{\bra{N_{0}}^{2}}\normo{P_{N_{1}}u}_{L_{x}^{\infty}}\normo{P_{N_{2}}(xf)}_{L_{x}^{2}}\normo{P_{N_{3}}u}_{L_{x}^{\infty}}ds\\
%  & \les\int_{0}^{t}\bra{s}^{-1+2\de_{0}}\ve_{1}^{3}\sum_{N_{1},N_{3}\les N_{2}}N_{1}^{\frac{\de_{0}}{4}}\bra{N_{1}}^{-(1-\frac{\de_{0}}{4})k}\bra{N_{2}}^{7-\frac{n}{2}}N_{3}^{\frac{\de_{0}}{4}}\bra{N_{3}}^{-(1-\frac{\de_{0}}{4})k}\ ds\\
%  & \les\bra{t}^{2\de_{0}}\ve_{1}^{3}.
% \end{align*}
% \textbf{\uline{Estimate of Case (iii).}}

% As before we decompose it into subcases; \textbf{Case (iiia)} $L\ll N_{0}\sim N_{1}$
% and $L\les N_{2}\sim N_{3}$, \textbf{Case (iiib) $N_{2}\ll L\sim N_{3}\ll N_{0}\sim N_{1}$},
% and \textbf{Case (iiic) $N_{3}\ll L\sim N_{2}\ll N_{0}\sim N_{1}$}.
% By the pointwise and multiplier bounds, \eqref{eq:coif-2} leads us
% that 
% \begin{align*}
%  & \sum_{\textbf{Case (iiia)}}\normo{\mathcal{J}_{\textbf{S}}^{4,2a}(t,\xi)}_{L_{\xi}^{2}}\\
%  & \les\int_{0}^{t}s\sum_{\substack{\textbf{Case (iiia)}\\
% \{\textbf{S}:L\le\bra{s}^{-1}\}
% }
% }\frac{\bra{N_{2}}^{3}}{\bra{N_{0}}^{2}}\|\rho_{L}\|_{L_{\eta}^{1}}\normo{\rho_{N_{1}}\wh{f}}_{L_{\xi}^{2}}\normo{\rho_{N_{2}}\wh{xf}}_{L_{\sigma}^{2}}\normo{\rho_{N_{3}}\wh{f}}_{L_{\sigma}^{2}}ds\\
%  & \hspace{2cm}+\int_{0}^{t}s\sum_{\substack{\textbf{Case (iiia)}\\
% \{\textbf{S}:L\ge\bra{s}^{-1}\}
% }
% }\frac{\bra{N_{2}}^{9}}{\bra{N_{0}}^{2}}\bra{N_{1}}^{-k}\bra{N_{3}}^{-k}\normo{u}_{W^{k,\infty}}^{2}\normo{xf}_{L_{x}^{2}}ds\\
%  & \les\int_{0}^{t}\ve_{1}^{3}\left(\bra{s}^{1+\de_{0}}\sum_{\substack{N_{1},N_{3}\\
% L\le\bra{s}^{-1}
% }
% }L^{2}N_{1}\bra{N_{1}}^{-k}N_{3}\bra{N_{3}}^{-k}\right.\\
%  & \hspace{4cm}\left.+\bra{s}^{-1+\de_{0}}\sum_{\substack{\bra{s}^{-1}\le L\les N_{1},N_{3}}
% }\bra{N_{1}}^{-2-k}\bra{N_{3}}^{9-k}\right)ds\\
%  & \les\bra{t}^{2\de_{0}}\ve_{1}^{3}.
% \end{align*}
% To get the estimates for \textbf{Case (iiib)} and \textbf{Case (iiic)},
% we also utilize \eqref{eq:coif-2} with pair $(\infty,2,\infty)$
% and \eqref{eq:bernstein} with $p=\frac{2}{\de_{0}}$. Since the proof
% is similar to \eqref{eq:caseiia}, we omit the details.


% %\textcolor{red}{
% %Lemma \ref{lem:coif} leads us that
% %\begin{align*}
% %	&\sum_{\textbf{Case (iiib)}}\normo{ \mathcal L_\textbf{S}^2(t,\xi)}_{L_\xi^2}\\
% %	&\les \int_0^t s \sum_{\textbf{Case (iiib)} }  \frac{\bra{N_2}\bra{N_3}}{\bra{N_0}^{2}} \bra{N_1}^{-k} \normo{u_1}_{W^{k,\infty}}\normo{P_{N_2}(xf)}_{L_x^\frac2{1-\de_0}} \normo{u_5                              }_{L_x^{\frac2{\de_0}}} ds\\
% %	&\les \int_0^t \ve_1^3 \bra{s}^{-1+ \de_0} \sum_{N_2 \ll N_3 \ll N_1}  \bra{N_1}^{-2-k} N_2^{\de_0}\bra{N_2}N_3^{\de_0} \bra{N_3}^{1-\de_0k} ds\\
% %	&\les \bra{t}^{2\de_0}\ve_1^3
% %\end{align*}
% %and
% %\begin{align*}
% %	&\sum_{\textbf{Case (iiic)}}\normo{ \mathcal L_\textbf{S}^2(t,\xi)}_{L_\xi^2}\\
% %	&\les \int_0^t s \sum_{\textbf{Case (iiic)} }  \frac{\bra{N_2}\bra{N_3}}{\bra{N_0}^{2}} \bra{N_1}^{-k} \normo{u_1}_{W^{k,\infty}}\normo{P_{N_2}(xf)}_{L_x^\frac2{1-\de_0}} \normo{u_5                              }_{L_x^{\frac2{\de_0}}} ds\\
% %	&\les \int_0^t \ve_1^3 \bra{s}^{-1+ \de_0} \sum_{N_3 \ll N_2 \ll N_1}  \bra{N_1}^{-2-k} N_2^{\de_0}\bra{N_2}N_3^{\de_0} \bra{N_3}^{1-\de_0k} ds\\
% %	&\les \bra{t}^{2\de_0}\ve_1^3.
% %\end{align*}
% %}

\section{Modified Scattering : Proof of Theorem \ref{main-thm:semi}}\label{sec:scattering}

\global\long\def\freq{{(\xi,\eta,\sigma)}}%
In Section \ref{sec:Weighted-Energy-estimate}, we have shown that the small solutions stay small in weighted Sobolev norm. In this section we prove \eqref{eq:modified-scattering}, the asymptotic behaviors of solutions, 
and complete the proof of Theorem~\ref{main-thm:semi}. We assume that $u$ satisfies
the a priori assumption \eqref{assumption-apriori}.
The modified scattering profile is defined by 
\begin{align*}
    \mathsf{v}(t,\xi)=e^{-iB(t,\xi)}e^{it\braxi}\wh{u}(t,\xi),
\end{align*}
where the phase correction is given by 
\begin{align*}
B(t,\xi) & =\frac{\lam}{(2\pi)^{2}}\int_{0}^{t}\int_{\mathbb{R}^{2}}\left|\frac{\xi}{\langle\xi\rangle}-\frac{\sigma}{\langle\sigma\rangle}\right|^{-1}\abs{\wh{u}(\sigma)}^{2}d\sigma\frac{\rho(s^{-\frac{2}{n}}\xi)}{\bra{s}}ds.
\end{align*}
\begin{prop}\label{prop:scattering} 
Assume that $u\in C([0,T],H^{n})$
satisfies the a priori assumption \eqref{assumption-apriori} with the index conditions \eqref{eq:bundle}. Then we
get 
\begin{align}
    \normo{\braxi^{k}\Big(\mathsf{v}(t_{2},\xi)-\mathsf{v}(t_{1},\xi)\Big)}_{L_{\xi}^{\infty}}\les\ve_{1}^3\bra{t_{1}}^{-\de}.\label{eq:scattering}
\end{align}
for $t_{1}\le t_{2}\in[0,T]$ and some $0<\de\le\frac{1}{100}$. \end{prop} 

By assuming Proposition~\ref{prop:scattering}, we first prove our main theorem.

\smallskip

\begin{proof}[Proof of Theorem \ref{main-thm:semi}]
For the proof of global behavior of solutions, it takes precedence to show the existence of a local solution to \eqref{main-eq:semi} in $\Sigma_{T}$. However,
since it is straightforward from the contraction mapping principle, we may omit the proof (for instance see
\cite{choz2006-siam,lee2021-bkms,hele2014}).
Now, given $T>0$, we assume that $\psi$ is a solution to \eqref{main-eq:semi}
on $[0,T]$ with initial data condition \eqref{condition-initial:semi}.
Then, by the bootstrap argument, it suffices to prove that for sufficiently small $\ve_{1}>0$,
there exists $C>0$ such that if $\|u\|_{\Sigma_{T}}\le \ep_1$,
\begin{align}
\|u\|_{\Sigma_{T}}\le\ve_{0}+C\ve_{1}^{3}.\label{eq:contraction}
\end{align}
From \eqref{eq:scattering}, one sees that the scattering norm stays bounded
\[
\|u(t)\|_{S}\le\ve_{0}+C\ve_{1}^{3}, \;\;\text{ for } t\in[0,T].
\]
Thus, together with the weighted energy estimates \eqref{eq:first-moment}
and \eqref{eq:second-moment}, we can close the bootstrapping argument and obtain the global existence of solution.
Concerning \eqref{eq:modified-scattering}, the asymptotic behavior of the solution,
we define a scattering profile by 
\[
u_{\infty}:=\mathcal{F}^{-1}\left(\lim_{t\to\infty}\mathsf{v}(t,\cdot)\right).
\]
Then Proposition \ref{prop:scattering} immediately yields that for $t\in[0,T]$,
\begin{align*}
\left\Vert \bra{\xi}^{k}\left(\widehat{u}(t,\xi)-e^{iB(t,\xi)}e^{-it\bra{\xi}}\wh{u_{\infty}}(\xi)\right)\right\Vert _{L_{\xi}^{\infty}}\les\ve_{1}^3\bra{t}^{-\de}.
\end{align*}    
\end{proof}



% The propositions \ref{prop-energy} and \ref{prop:scattering} induce \eqref{eq:contraction}. This completes the proof of global well-posedness. 


% To show the decay rate \eqref{global-bound:semi} and global well-posedness,
% we have to handle a weighted energy norm and Fourier amplitude in
% $\Sigma_{T}$ by a bootstrap argument (see \eqref{assumption-apriori}).


% Given $T>0$, we assume that $\psi$ be a solution to \eqref{main-eq:semi}
% on $[0,T]$ with initial data condition \eqref{condition-initial:semi}.
% Then it suffices to prove that for sufficiently small $\ve_{1}>0$,
% there exists C such that 
% \begin{align}
% \|u\|_{\Sigma_{T}}\le\ve_{0}+C\ve_{1}^{3}.\label{eq:contraction}
% \end{align}
% The propositions \ref{prop-energy} and \ref{prop:scattering} induce \eqref{eq:contraction}. This completes the proof of global well-posedness. For the proof of the modified scattering, we may refer
% to the beginning of Section \ref{sec:scattering}.



%   From \eqref{eq:scattering}, one sees that  the scattering norm stays bounded
% \[
% \|u(t)\|_{S}\le\ve_{0}+C\ve_{1}^{3}, \text{ for } t\in[0,T].
% \]
% Thus, together with the weighted energy estimates \eqref{eq:first-moment}
% and \eqref{eq:second-moment}, we can close the bootstrapping argument and obtain the global existence of solution.
% Concerning the asymptotic behavior of the solution \eqref{eq:modified-scattering},
% we define a scattering profile by 
% \[
% u_{\infty}:=\mathcal{F}^{-1}\left(\lim_{t\to\infty}\mathsf{v}(t,\cdot)\right).
% \]
% Then Proposition \ref{prop:scattering} immediately yields that for $t\in[0,T]$,
% \begin{align*}
% \left\Vert \bra{\xi}^{k}\left(\widehat{u}(t,\xi)-e^{iB(t,\xi)}e^{-it\bra{\xi}}\wh{u_{\infty}}(\xi)\right)\right\Vert _{L_{\xi}^{\infty}}\les\ve_{1}^3\bra{t}^{-\de}.
% \end{align*}    

\smallskip 

\noindent\emph{Proof of Proposition \ref{prop:scattering}.}
We will proceed as in \cite{pusa}. We prove that if $t_1\le t_{2}\in [M-2,2M]\cap[0,T]$ for
a dyadic number $M\in2^{\mathbb{N}}$
\begin{align}
\normo{\braxi^{k}\Big(\mathsf{v}(t_{2},\xi)-\mathsf{v}(t_{1},\xi)\Big)}_{L_{\xi}^{\infty}}\les\ve_{1}^{3}M^{-\de},\label{goal-modi}
\end{align}
for some $0<\delta\le\frac{1}{100}$.
We begin with writing $\mathsf{v}$ as 
\[
\mathsf{v}(t,\xi)=e^{-iB(t,\xi)}\wh{f}(t,\xi)=e^{-iB(t,\xi)}\wh{u_{0}}(\xi)+i\lam e^{-iB(t,\xi)}\mathcal{I}(t,\xi),
\]
where, after change of variables, the nonlinear term $\mathcal{I}$ is given by 
\begin{align*}
 & \mathcal{I}(t,\xi)=\frac{1}{(2\pi)^3}\int_{0}^{t}\iint_{\R^{2}\times\R^{2}}e^{isq\freq}|\eta|^{-1}\wh{f}(s,\xi+\eta)\wh{f}(s,\xi+\sigma)\overline{\wh{f}(s,\xi+\eta+\sigma)}d\eta d\sigma ds,
\end{align*}
% where we applied 
% The nonlinear part comes from \eqref{eq:duhamel} by the change of
% variables with 
% \[
% \sigma\to\xi-\eta+\sigma\;\;\mbox{ and }\;\;\eta\to-\eta.
% \]
% In the view of \eqref{function-resonance}, 
where a resonant function
\[
q\freq=\braxi-\langle\xi+\eta\rangle-\langle\xi+\sigma\rangle+\bra{\xi+\eta+\sigma}.
\]
Let $L_{0}\in2^{\Z}$ such that 
\begin{align}
L_{0}\sim M^{-\frac{9}{10}}.\label{eq:lzero}
\end{align}
We write  
\begin{align}
\begin{aligned}\label{eq:duhamel-lzero}\mathcal{I}(t,\xi)=\int_{0}^{t}\bigg(\mathcal{K}_{L_{0}}(s,\xi)+\sum_{L\in2^{\Z},L>L_{0}}\mathcal{K}_{L}(s,\xi)\bigg)ds,\end{aligned}
\end{align}
where 
\begin{align*}
\mathcal{K}_{L_{0}}(s,\xi) & :=\frac{1}{(2\pi)^3}\iint_{\R^{2}\times\R^{2}}e^{isq\freq}\rho_{\le L_{0}}(\eta)|\eta|^{-1}\wh{f}(s,\xi+\eta)\wh{f}(s,\xi+\sigma)\overline{\wh{f}(s,\xi+\eta+\sigma)}d\eta d\sigma,\\
\mathcal{K}_{L}(s,\xi) & :=\frac{1}{(2\pi)^3}\iint_{\R^{2}\times\R^{2}}e^{isq\freq}\rho_{L}(\eta)|\eta|^{-1}\wh{f}(s,\xi+\eta)\wh{f}(s,\xi+\sigma)\overline{\wh{f}(s,\xi+\eta+\sigma)}d\eta d\sigma.
\end{align*}
The first term $\mathcal{K}_{L_{0}}$, the integral around the singular point,
is the one responsible for the correction of scattering, whereas the second term $\mathcal{K}_{L}$ is remainder term. The profile $\mathsf{v}$ verifies  
\begin{align*}
\partial_{t}\mathsf{v}(t,\xi) & =\partial_{t}\left[e^{-iB(t,\xi)}\wh{f}(t,\xi)\right]\\
 & =i\lam e^{-iB(t,\xi)}\left[\left(\mathcal{K}_{L_{0}}(t,\xi)+\sum_{L>L_{0}}\mathcal{K}_{L}(t,\xi)\right)-\frac{1}{\lam}\left[\partial_{t}B(t,\xi)\right]\wh{f}(t,\xi)\right].
\end{align*}
Thus,
\begin{align}
\begin{aligned}\label{eq:deriv-v}  \mathsf{v}(t_{2},\xi)-\mathsf{v}(t_{1},\xi)
  &=\int_{t_{1}}^{t_{2}}\partial_{s}\mathsf{v}(s,\xi)ds\\
 & =i\lam\int_{t_{1}}^{t_{2}}e^{-iB(s,\xi)}\left[\left(\mathcal{K}_{L_{0}}(s,\xi)+\sum_{L>L_{0}}\mathcal{K}_{L}(s,\xi)\right)-\frac{1}{\lam}\left[\partial_{s}B(s,\xi)\right]\wh{f}(s,\xi)\right]ds.
\end{aligned}
\end{align}
In order to prove \eqref{goal-modi}, we use the cancellation effect
between $ \mathcal{K}_{L_{0}} $ and $\partial_{s}B(s,\xi)$, specifically, we show that for each $\xi$ with $|\xi|\sim N\in2^{\Z}$,
\begin{align}
\begin{aligned} & \left|\int_{t_{1}}^{t_{2}}e^{-iB(s,\xi)}\left(\mathcal{K}_{L_{0}}(s,\xi)-\frac{1}{\lam}\left[\partial_{s}B(s,\xi)\right]\wh{f}(s,\xi)\right)ds\right|\les\ve_{1}^{3}M^{-\de}\bra{N}^{-k}\label{eq:part-modification}\end{aligned}
\end{align}
and 
\begin{align}
\left|\int_{t_{1}}^{t_{2}}e^{-iB(s,\xi)}\sum_{L>L_{0}}\mathcal{K}_{L}(s,\xi)ds\right| & \les\ve_{1}^{3}M^{-\de}\bra{N}^{-k},\label{eq:part-scattering}
\end{align}
for some $0<\delta\le \frac{1}{100}$.

\smallskip
\noindent\emph{Proof of \eqref{eq:part-modification}.} 
% In the estimate \eqref{eq:part-modification}, when $M^{\frac{3}{n}}\les N$,
% we can get the desired decay \eqref{goal-modi} without the phase
% modification $B(s,\xi)$. Then we progress our proof to separate into
% $M^{\frac{3}{n}}\les N$ and $N\ll M^{\frac{3}{n}}$ via $\rho\in C_{0}^{\infty}(B(0,2))$.
We prove the following two bounds:
\begin{align}
\left|\int_{t_{1}}^{t_{2}}e^{-iB(s,\xi)}\mathcal{K}_{L_{0}}(s,\xi)\left(1-\rho\left(s^{-\frac{2}{n}}\xi\right)\right)ds\right| & \les\ve_{1}^{3}M^{-\de}\bra{N}^{-k},\label{eq:decay-part}\\
\left|\int_{t_{1}}^{t_{2}}e^{-iB(s,\xi)}\left[\mathcal{K}_{L_{0}}(s,\xi)\rho\left(s^{-\frac{2}{n}}\xi\right)-\frac{1}{\lam}\left[\partial_{s}B(s,\xi)\right]\wh{f(s,\xi)}\right]ds\right| & \les\ve_{1}^{3}M^{-\de}\bra{N}^{-k},\label{eq:crucial-part}
\end{align}
where $\rho\in C_{0}^{\infty}(B(0,2))$.
We remark that the phase correction will be derived in the proof of \eqref{eq:crucial-part}.

We first consider \eqref{eq:decay-part}. It suffices to show the integrand bound 
\begin{align*}
\Big|\mathcal{K}_{L_{0}}(s,\xi)\Big|\les\ve_{1}^{3}M^{-(1+\de)}\bra{N}^{-k}.
\end{align*}
%By a priori assumption \eqref{assumption-apriori}, we estimate
%\begin{align*}
%	& \left|\mathcal K_{L_{0}}(s,\xi)\right|\\
%	& \;\;\les \bra{\xi}^{-k}\int\!\!\!\! \int_{\R^2 \times \R^2}\left|\bra{N}^k\rho_{\le L_{0}}(\eta)|\eta|^{-1}\wh{f}(s,\xi+\eta)\wh{f}(s,\xi+\sigma)\overline{\wh{f}(s,\xi+\eta+\sigma)}\right|d\eta d\sigma\\
%	& \;\;\les L_{0}\bra{N}^{-k}\left\|\bra{\xi}^k\wh{f}\,\right\|_{L_{\xi}^{\infty}}\|u\|_{H^{k}}\|u\|_{H^k}\\
%	& \;\;\les \ve_{1}^{3} L_{0}M^{2\de_{0}}\bra{N}^{-k}.
%\end{align*}
%Then, \eqref{eq:lzero} implies our desired result.
We further split $\mathcal{K}_{L_{0}}$ dyadically as follows:
\begin{align*}
\mathcal{K}_{L_{0}}(s,\xi)=\sum_{L_{1}\le L_{0}+10}\mathcal{K}_{L_{0},L_{1}}(s,\xi),
\end{align*}
where 
\begin{align*}
\mathcal{K}_{L_{0},L_{1}}(s,\xi)  =\frac{1}{(2\pi)^3}\int\!\!\!\!\int e^{isq\freq}\rho_{L_{1}}(\eta)\rho_{\le L_{0}}(\eta)|\eta|^{-1}\wh{f}(s,\xi+\eta)\wh{f}(s,\xi+\sigma)\overline{\wh{f}(s,\xi+\eta+\sigma)}d\eta d\sigma.
\end{align*}
Then, by the a priori assumption \eqref{assumption-apriori}, we estimate
\begin{align*}
 & \left|\mathcal{K}_{L_{0},L_{1}}(s,\xi)\right|\\
 & \;\;\les L_{1}^{-1}\bra{\xi}^{-k}\int\!\!\!\!\int\left|\bra{N}^{k}\rho_{L_{1}}(\eta)\rho_{\le L_{0}}(\eta)\wh{f}(s,\xi+\eta)\wh{f}(s,\xi+\sigma)\overline{\wh{f}(s,\xi+\eta+\sigma)}\right|d\eta d\sigma\\
 & \;\;\les L_{1}^{-1}\bra{N}^{-k}L_{1}^{2}\left\Vert \bra{\xi}^{k}\wh{f}\,\right\Vert _{L_{\xi}^{\infty}}\|u\|_{H^{n}}\|u\|_{H^{n}}\\
 & \;\;\les\ve_{1}^{3}L_{1}M^{2\delta_0}\bra{N}^{-k}.
\end{align*}
On the other hand, by H\"older inequality, we get 
\begin{align*}
\left|\mathcal{K}_{L_{0},L_{1}}(s,\xi)\right|\;\;\les L_{1}^{-1}\|\rho_{L_{1}}\|_{L^{2}}N^{-n}\|f\|_{H^{n}}^{3}\les\ve_{1}^{3}M^{-2}M^{3\delta_0}.
\end{align*}
These two estimates induce that 
\begin{align*}
 & \sum_{L_{1}\le L_{0}+10}\Big|\mathcal{K}_{L_{0},L_{1}}(s,\xi)\Big|\\
 & \les\ve_{1}^{3}\left(\sum_{L_{1}\le M^{-2}}L_{1}M^{2\delta_0}\bra{N}^{-k}+\sum_{M^{-2}< L_{1}<L_0+10}M^{-2+3\delta_0}\right)\\
 & \les\ve_{1}^{3}M^{-(1+\delta_0)}\bra{N}^{-k}.
\end{align*}


Next, consider \eqref{eq:crucial-part}. 
Due to the cut-off $\rho(s^{-\frac{2}{n}}\xi)$, we may assume $N\le M^{\frac{2}{n}}$.
It suffices to show that the integrand satisfies 
\begin{align}
\left|\mathcal{K}_{L_{0}}(s,\xi)-\frac{1}{\lam}\left[\partial_{s}B(s,\xi)\right]\wh{f}(s,\xi)\right| & \les\ve_{1}^{3}M^{-(1+\de)}\bra{N}^{-k}.\label{eq:proof-part2}
\end{align}
The correction term will be achieved after  three steps. 

\noindent\textit{Step1: Phase approximation.}
We approximate the phase function by a simpler one in the support of the integrand in \eqref{eq:crucial-part}. Let us observe that 
\begin{align*}
q\freq & =\Big(\braxi-\bra{\xi+\eta}\Big)-\Big(\bra{\xi+\sigma}-\bra{\xi+\eta+\sigma}\Big)\\
 & =\left(\frac{-|\eta|^{2}-2\eta\cdot\xi}{\braxi+\bra{\xi+\eta}}-\frac{-|\eta|^{2}-2\eta\cdot(\xi+\sigma)}{\bra{\xi+\sigma}+\bra{\xi+\eta+\sigma}}\right)\\
 & =\eta\cdot\left(\frac{\xi}{\braxi}-\frac{\xi+\sigma}{\bra{\xi+\sigma}}\right)+O\left(|\eta|^{2}\right)\\
 & =:r(\xi,\eta,\sigma)+O\left(|\eta|^{2}\right).
\end{align*}
We now set 
\begin{align*}
\mathcal{K}_{L_{0}}'(s,\xi) & :=\frac{1}{(2\pi)^3}\iint_{\R^{2}\times\R^{2}}e^{isr\freq}\rho_{\le L_{0}}(\eta)|\eta|^{-1}\wh{f}(s,\xi+\eta)\wh{f}(s,\xi+\sigma)\overline{\wh{f}(s,\xi+\eta+\sigma)}d\eta d\sigma.
\end{align*}
Then we estimate 
\begin{align*}
 & \left|\mathcal{K}_{L_{0}}(s,\xi)-\mathcal{K}_{L_{0}}'(s,\xi)\right|\\
 & \les\iint_{\R^{2}\times\R^{2}}|s|\left|q(\xi,\eta,\sigma)-r(\xi,\eta,\sigma)\right||\eta|^{-1}\\
 & \hspace{2.2cm}\times\left|\rho_{\le L_{0}}(\eta)\wh{f}(s,\xi+\eta)\wh{f}(s,\xi+\sigma)\overline{\wh{f}(s,\xi+\eta+\sigma)}\right|d\eta d\sigma\\
 & \les M\iint_{\R^{2}\times\R^{2}}|\eta|\left|\rho_{\le L_{0}}(\eta)\wh{f}(s,\xi+\eta)\wh{f}(s,\xi+\sigma)\overline{\wh{f}(s,\xi+\eta+\sigma)}\right|d\eta d\sigma\\
 & \les ML_{0}^{3}\normo{f}_{L^{2}}^{2}\norm{\wh{f}}_{L_{\xi}^{\infty}}\\
 & \les\ve_{1}^{3}M^{-\frac{17}{10}}\les\ve_{1}^{3}M^{-(1+\de)}\bra{N}^{-k},
\end{align*}
where we used \eqref{eq:lzero} and $\bra{N}^{k} \le M^{\frac{2k}{n}}\le M^{\frac{1}{100}}$ in the last inequality. 

\noindent\textit{Step2: Profiles approximation.}
We now approximate
$\mathcal{K}_{L_{0}}'$ by $\wt{\mathcal{K}_{L_{0}}\;}$
which is defined by 
\begin{align*}
\wt{\mathcal{K}_{L_{0}}\;}(s,\xi) & :=\frac{1}{(2\pi)^3}\iint_{\R^{2}\times\R^{2}}e^{isr\freq}\rho_{\le L_{0}}(\eta)|\eta|^{-1}\wh{f}(\xi)\left|\wh{f}(\xi+\sigma)\right|^{2}d\eta d\sigma.
\end{align*}
By setting $R=L_{0}^{-\frac{1}{2}}$, we see that 
\begin{align*}
 \left|\wh{f}(\zeta+\eta)-\wh{f}(\zeta)\right|
 & \les\left|\wh{\rho_{>R}f}(\zeta+\eta)-\wh{\rho_{>R}f}(\zeta)\right|+\left|\wh{\rho_{\le R}f}(\zeta+\eta)-\wh{\rho_{\le R}f}(\zeta)\right|\\
 & \les\normo{\wh{\rho_{>R}f}}_{L_{\xi}^{\infty}}+L_{0}\normo{\nabla_{\xi}\wh{\rho_{\le R}f}}_{L_{\xi}^{\infty}}\\
 & \les R^{-1}\normo{\bra{x}^{2}f}_{L^{2}}+L_{0}\|\rho_{\le R}\|_{L_{x}^{2}}\normo{xf}_{L^{2}}\\
 & \les L_{0}^{\frac{1}{2}}M^{2\de_{0}}.
\end{align*}
From this and \eqref{eq:lzero}, we estimate 
\begin{align*}
 & \abs{\mathcal{K}_{L_{0}}'(s,\xi)-\wt{\mathcal{K}_{L_{0}}\;}(s,\xi)}\\
 & \les\iint_{\R^{2}\times\R^{2}}\rho_{\le L_{0}}(\eta)|\eta|^{-1}\abs{\wh{f}(\xi+\eta)\wh{f}(\xi+\sigma)\overline{\wh{f}(\xi+\eta+\sigma)}-\wh{f}(\xi)\left|\wh{f}(\xi+\sigma)\right|^{2}}d\eta d\sigma\\
 & \les\ve_{1}^{3}L_{0}^{\frac{3}{2}}M^{2\de_{0}}\les\ve_{1}^{3}M^{-(1+\de)}\bra{N}^{-k}.
\end{align*}


\noindent\textit{Step3: final approximation.}
We conclude the proof of \eqref{eq:crucial-part} by showing that 
\begin{align}
\left|\wt{\mathcal{K}_{L_{0}}\;}(s,\xi)-\frac{1}{\lam}\left[\partial_{s}B(s,\xi)\right]\wh{f}(s,\xi)\right|\les\ve_{1}^{3}M^{-(1+\de)}\bra{N}^{-k}.\label{eq:cancel}
\end{align}
Setting $\mathbf{z}:=\left(\frac{\xi}{\bra{\xi}}-\frac{\sigma}{\bra{\sigma}}\right)$,
by the change of variables, we get 
\begin{align*}
\wt{\mathcal{K}_{L_{0}}\;}(s,\xi) & =\frac{1}{(2\pi)^3}\iint_{\R^{2}\times\R^{2}}e^{is\eta\cdot\mathbf{z}}\rho_{\le L_{0}}(\eta)|\eta|^{-1}\wh{f}(\xi)\left|\wh{f}(\sigma)\right|^{2}d\eta d\sigma.
\end{align*}
Then, \eqref{eq:cancel} can be reduced to showing that 
\begin{align}\begin{aligned}\label{eq:cancel2}
    &\left|\frac{1}{(2\pi)^3}\iint_{\R^{2}\times\R^{2}}e^{is\eta\cdot\mathbf{z}}\rho_{\le L_{0}}(\eta)|\eta|^{-1}\abs{\wh{f}(\sigma)}^{2}d\eta d\sigma-\frac{1}{(2\pi)^{2}|s|}\int_{\R^{3}}|\mathbf{z}|^{-1}\abs{\wh{f}(\sigma)}^{2}d\sigma\right| \\
    &\quad \les \ep_1^2 M^{-(1+\de)}.
\end{aligned}\end{align}
% Then, by the definition of $B(s,\xi)$, we deduce that 
% \begin{align*}
%  & \left|\wt{\mathcal{K}_{L_{0}}^{q}\;}(s,\xi)-\frac{1}{\lam}\left[\partial_{s}B(s,\xi)\right]\wh{f}(s,\xi)\right|\\
%  & \les\abs{\wh{f}(\xi)}\left|\iint_{\R^{2}\times\R^{2}}e^{is\eta\cdot\mathbf{z}}\rho_{\le L_{0}}(\eta)|\eta|^{-1}\abs{\wh{f}(\sigma)}^{2}d\eta d\sigma-\frac{(2\pi)^{2}}{s}\int_{\R^{3}}|\mathbf{z}|^{-1}\abs{\wh{f}(\sigma)}^{2}d\sigma\right|\\
%  & \les\abs{\wh{f}(\xi)}\left|\int_{\R^{2}}\left|\int_{\R^{2}}e^{is\eta\cdot\mathbf{z}}|\eta|^{-1}\rho_{\le L_{0}}(\eta)d\eta-(2\pi)^{2}|s\mathbf{z}|^{-1}\right|\abs{\wh{f}(\sigma)}^{2}d\sigma\right|.
% \end{align*}
We first claim that 
\begin{align}\label{bound1}
    \left|\int_{\R^{2}}e^{is\eta\cdot\mathbf{z}}|\eta|^{-1}\rho_{\le L_{0}}(\eta)d\eta-(2\pi)^{2}|s\mathbf{z}|^{-1}\right|
    \les   M^{-\frac{16}{15}}|\mathbf{z}|^{-\frac53}.
\end{align}
Observe that since $\mathcal{F}(|x|^{-1})=2\pi|\eta|^{-1}$, the following formula hols 
\[
\frac{2\pi}{ |s\mathbf{z}|}=\lim_{A\to\infty}\int_{\R^2} e^{is\mathbf{z}\cdot\eta}\rho_{\le A}(\eta)\frac{1}{|\eta|}d\eta.
\]
Then we get that for $L_{0}\ll A$, 
\begin{align}
\begin{aligned}\label{eq:final-approx-1} & \left|\int_{\R^{2}}e^{is\eta\cdot\mathbf{z}}|\eta|^{-1}\rho_{\le L_{0}}(\eta)d\eta-2\pi|s\mathbf{z}|^{-1}\right|\\
 & =\left|\int_{\R^{2}}e^{is\eta\cdot\mathbf{z}}|\eta|^{-1}\left(\rho_{\le L_{0}}(\eta)-\rho_{\le A}(\eta)\right)d\eta\right|\\
 & =|s\mathbf{z}|^{-2}\left|\int_{\R^{2}}\left(\nabla_{\eta}^{2}e^{is\eta\cdot\mathbf{z}}\right)|\eta|^{-1}\left(\rho_{\le L_{0}}(\eta)-\rho_{\le A}(\eta)\right)d\eta\right| \\ 
 & \les M^{-2}|\mathbf{z}|^{-2}L_{0}^{-1}
\end{aligned}
\end{align}
and the trivial bounds 
\begin{align}
\begin{aligned}\label{eq:final-approx-2}\left|\int_{\R^{2}}e^{is\eta\cdot\mathbf{z}}|\eta|^{-1}\rho_{\le L_{0}}(\eta)d\eta-(2\pi)^{2}|s\mathbf{z}|^{-1}\right| & \les L_{0}+|s\mathbf{z}|^{-1}.
\end{aligned}
\end{align}
Then, \eqref{bound1} follows by interpolating \eqref{eq:final-approx-1} and \eqref{eq:final-approx-2}, since $L_0\sim M^{-\frac{9}{10}}$.
Now, the left-hand side of \eqref{eq:cancel2} is bounded by 
\begin{align*}
    M^{-\frac{16}{15}} \left|\int_{\R^{2}}|\mathbf{z}|^{-\frac{5}{3}}\abs{\wh{f}(\sigma)}^{2}d\sigma \right|.
\end{align*}
Since $|\mathbf{z}|\gtrsim\min\left\{ 1,|\sigma|,\frac{|\xi-\sigma|}{\bra{\sigma}^{3}}\right\} $,
the a priori assumption \eqref{assumption-apriori} yields that 
\begin{align*}
 \left|\int_{\R^{2}}|\mathbf{z}|^{-\frac{5}{3}}\abs{\wh{f}(\sigma)}^{2}d\sigma\right|\les\ve_{1}^{2}.    
\end{align*}
which completes the proof of \eqref{eq:cancel2}. 

\medskip

\noindent\emph{Proof of \eqref{eq:part-scattering}.} We further localize the frequencies as follows: 
\begin{align*}
\mathcal{K}_{L}(s,\xi) & =\frac{1}{(2\pi)^3}\sum_{\textbf{N}=(N_{1},N_{2},N_{3})\in(2^{\Z})^{3}}\mathcal{K}_{L,\mathbf{N}}(s,\xi),\\
\mathcal{K}_{L,\textbf{N}}(s,\xi) & :=\iint_{\R^{2}\times\R^{2}}e^{isq\freq}\rho_{L}(\eta)|\eta|^{-1}\wh{f_{N_{1}}}(s,\xi+\eta)\wh{f_{N_{2}}}(s,\xi+\sigma)\overline{\wh{f_{N_{3}}}(s,\xi+\eta+\sigma)}d\eta d\sigma.
\end{align*}
We prove that 
\begin{align}
\sum_{L>L_{0},\textbf{N}=(N_1,N_2,N_3)}\left|\mathcal{K}_{L,\textbf{N}}(s,\xi)\right|\les\ve_{1}^{3}M^{-(1+\de)}\bra{N}^{-k}.\label{eq:goal-high}
\end{align}
By H\"older inequality, we readily have 
\[
\left|\mathcal{K}_{L,\mathbf{N}}(s,\xi)\right|\les\prod_{j=1}^{3}\normo{\wh{f_{N_{j}}}(s)}_{L^{2}}.
\]
Using the a priori assumption \eqref{assumption-apriori}, we know that 
\[
\normo{\wh{f_{N_{j}}}(s)}_{L^{2}}\les\min{\left(N_{j}\bra{N_{j}}^{-k},\bra{N_{j}}^{-n}M^{\de_{0}}\right)}\ve_{1}\;\;\mbox{ for }\;\;j=1,2,3.
\]
The last two estimates above imply the summation in \eqref{eq:goal-high} over those indexes $(N_1,N_2,N_3)$ with 
$\max( N_{1},N_{2},N_{3})\ge M^{\frac{2}{n}}$ or 
$\min( N_{1},N_{2},N_{3})\le M^{-(1+\de)}$ satisfy the desired bound.
Thus, we suffice to estimate the sum over those indexes $(N_1,N_2,N_3)$ with 
\begin{align}
M^{-(1+\de)}\le N_{1},N_{2},N_{3}\le M^{\frac{2}{n}}.\label{eq:condi-supp}
\end{align}
Let us further localize $\sigma$ variable with respect to $L'\in2^{\Z}$ to write 
\begin{align*}
&\mathcal{K}_{\textbf{L},\mathbf{N}}(s,\xi) = \\
&\; \iint_{\R^{2}\times\R^{2}}e^{isq\freq}\rho_{L}(\eta)\rho_{L'}(\sigma)|\eta|^{-1}\wh{f_{N_{1}}}(s,\xi+\eta)\wh{f_{N_{2}}}(s,\xi+\sigma)\overline{\wh{f_{N_{3}}}(s,\xi+\eta+\sigma)}d\eta d\sigma,
\end{align*}
where we denoted $\textbf{L}=(L,L')\in (2^{\Z})^2$. 
Then, \eqref{eq:goal-high} reduces to showing that 
\begin{align}
    \sum_{\supp\in\mathcal{A}}\left|\mathcal{K}_{\textbf{L},\textbf{N}}(s,\xi)\right|\les\ve_{1}^{3}M^{-(1+\de)}\bra{N}^{-k},
\label{eq:goal2-high} 
\end{align}
where the summation runs over 
\begin{align}\begin{aligned}\label{eq:condi-piece}
    \mathcal{A}=\Big\{  \left(\textbf{L},\textbf{N}\right)\in (2^{\Z})^5 &: M^{-(1+\de)}\le N_{1},N_{2},N_{3}\le M^{\frac{2}{n}},\; \max(N_1,N_2,N_3)\sim N, \\ 
   &\hspace{3cm} L_{0}\le L \le \max(N_2,N_3), \; L'\le \max(N_1,N_3)\Big\} .
\end{aligned}\end{align}
By H\"older inequality, we also readily have 
\begin{align*}
    \left|\mathcal{K}_{L,\mathbf{N}}(s,\xi)\right| &\les
    L^{-1}\|\rho_{L}\|_{L^{1}}\|\rho_{L'}\|_{L^{1}}\langle N_1\rangle^{-k}\langle N_2\rangle^{-k}\langle N_3\rangle^{-k}\normo{\bra{\xi}^{k}\wh{f}(s,\xi)}_{L_{\xi}^{\infty}}^3 \\ 
    &\les L(L')^{2}\bra{N}^{-k}\normo{\bra{\xi}^{k}\wh{f}(s,\xi)}_{L_{\xi}^{\infty}}^3.
\end{align*}
Hence we may obtain the desired estimates \eqref{eq:goal2-high} 
whenever the summation runs over those indexes in $\mathcal{A}$ satisfying $L(L')^{2}\le M^{-(1+2\de)}$. Indeed, we have
\global\long\def\nl{{\textbf{L},\textbf{N}}}%
\begin{align*}
    \sum_{\mathcal{A}\cap\left\{ L(L')^{2}\le M^{-(1+2\de)}\right\}}|\mathcal{K}_{\nl}(s,\xi)|  
    \les \ve_{1}^{3}M^{-(1+\de)}\bra{N}^{-k}.
\end{align*}
We are left with a summation over 
\begin{align*}
    \mathcal{B}=\mathcal{A}\cap\left\{ L(L')^{2}\ge M^{-(1+2\de)}\right\}.
\end{align*}
We perform an integration by parts in $\eta$ twice, with a slight abuse of notation, to obtain 
\begin{align*}  
&|\mathcal{K}_{\nl}(s,\xi)|  \le |\mathcal{K}_{1}(s,\xi)|+|\mathcal{K}_{2}(s,\xi)|+|\mathcal{K}_{3}(s,\xi)|,\\
 &   \mathcal{K}_{1}(s,\xi)  :=\frac{1}{s^{2}}\int\!\!\!\!\int e^{isq\freq}\textbf{m}_{1}(\xi,\eta,\sigma)\nabla_{\eta}^{2}\left(\wh{f_{N_{1}}}(s,\xi+\eta)\overline{\wh{f_{N_{3}}}(s,\xi+\eta+\sigma)}\right)\wh{f_{N_{2}}}(s,\xi+\sigma)d\eta d\sigma, \\
  &      \mathcal{K}_{2}(s,\xi)  :=\frac{1}{s^{2}}\int\!\!\!\!\int e^{isq\freq}\textbf{m}_{2}(\xi,\eta,\sigma)\nabla_{\eta}\left(\wh{f_{N_{1}}}(s,\xi+\eta)\overline{\wh{f_{N_{3}}}(s,\xi+\eta+\sigma)}\right)\wh{f_{N_{2}}}(s,\xi+\sigma)d\eta d\sigma,\\
   & \mathcal{K}_{3}(s,\xi) :=\frac{1}{s^{2}}\int\!\!\!\!\int e^{isq\freq}\textbf{m}_{3}(\xi,\eta,\sigma)\wh{f_{N_{1}}}(s,\xi+\eta)\overline{\wh{f_{N_{3}}}(s,\xi+\eta+\sigma)}\wh{f_{N_{2}}}(s,\xi+\sigma)d\eta d\sigma, 
    \end{align*}
    where 
    \begin{align*}
        \textbf{m}_{1}(\xi,\eta,\sigma) & :=\left( \frac{\nabla_{\eta}q\freq}{|\nabla_{\eta}q\freq|^{2}}\right)^2|\eta|^{-1}\rho_{L}(\eta)\rho_{L'}(\sigma)\rho_{N_{1}}(\xi+\eta)\rho_{N_{3}}(\xi+\eta+\sigma),\\  
    \textbf{m}_{2}(\xi,\eta,\sigma) & :=\frac{\nabla_{\eta}q\freq}{|\nabla_{\eta}q\freq|^{2}}\nabla_{\eta}\left(\frac{\nabla_{\eta}q\freq}{|\nabla_{\eta}q\freq|^{2}}|\eta|^{-1}\rho_{L}(\eta)\rho_{N_{1}}(\xi+\eta)\rho_{N_{3}}(\xi+\eta+\sigma)\right)\rho_{L'}(\sigma),\\ 
    \textbf{m}_{3}(\xi,\eta,\sigma) & :=\nabla_{\eta}\textbf{m}_{2}\freq.
    \end{align*}
A direct computation yields that 
\begin{align}
    \begin{aligned}
    \label{lower bound of nabla q}
    \left|\nabla_{\eta}q(\xi,\eta,\sigma)\right|  
    &=\left|\frac{\xi+\eta+\sigma}{\bra{\xi+\eta+\sigma}}-\frac{\xi+\eta}{\bra{\xi+\eta}}\right| \\ 
    &\gtrsim \frac{|\sigma|}{\max(\langle \xi+\eta+\sigma\rangle, \langle\xi+\eta\rangle) \min(\langle \xi+\eta+\sigma\rangle, \langle\xi+\eta\rangle)^2}.
\end{aligned}\end{align}
With the help of \eqref{lower bound of nabla q}, one can verify that the symbols satisfy the following bounds:
\begin{align}
\left\Vert \iint_{\R^{2}\times\R^{2}}\textbf{m}_{1}\freq e^{iy\cdot\eta}e^{iz\cdot\sigma} d\eta d\sigma\right\Vert _{L_{y,z}^{1}(\R^{2}\times\R^{2})} & \les  L^{-1}(L')^{-2}\langle\max(N_1,N_3)\rangle\langle\min(N_1,N_3)\rangle^{10},\label{eq:k_1 bound} \\ 
\normo{\iint_{\R^{2}\times\R^{2}}\textbf{m}_{2}(\xi,\eta,\sigma)e^{iy\cdot\eta}e^{iz\cdot\sigma}\,d\sigma d\eta}_{L_{y,z}^{1}(\R^{2}\times\R^{2})}
&\les L^{-2}(L')^{-2}\langle\max(N_1,N_3)\rangle^2\langle\min(N_1,N_3)\rangle^{12}, \label{eq:k_2 bound}
\end{align}
and 
\begin{align}
    \begin{aligned}\label{eq:pw-bound-k3}
    \left|\textbf{m}_{3}\freq\right| & \les L^{-3}(L')^{-2}\langle\max(N_1,N_3)\rangle^2\langle\min(N_1,N_3)\rangle^{4}.
    \end{aligned}
    \end{align} 
By applying the operator inequality \eqref{eq:coif-1-1} with \eqref{eq:k_1 bound}, we estimate 
\begin{align*}
 & \sum_{(\nl)\in\mathcal{B}}\left|\mathcal{K}_{1}(s,\xi)\right|\\
 & \les M^{-2}\sum_{(\nl)\in\mathcal{B}}L^{-1}(L')^{-2}\langle\max(N_1,N_3)\rangle\langle\min(N_1,N_3)\rangle^{10} \\ 
 &\qquad\qquad \qquad\qquad\times \normo{P_{N_{2}}u}_{L_{x}^{\infty}}\left( \normo{x^{2}f}_{L^{2}}\normo{f_{N_{3}}}_{L^{2}} + \normo{xf}_{L^{2}}^2 + \normo{x^{2}f}_{L^{2}}\normo{f_{N_{1}}}_{L^{2}}\right)\\
 & \les \ve_1^3 M^{-2+2\delta+2\delta_0}\sum_{ \substack{M^{-(1+\de)}\le N_{1},N_{2},N_{3}\le M^{\frac{2}{n}} \\   L_0\le L \le \max(N_2,N_3)}}\langle\max(N_1,N_3)\rangle\langle\min(N_1,N_3)\rangle^{10} \langle N_2\rangle^{-k}\\
 &\les  \ve_1^3 M^{-2+3\delta+2\delta_0} \bra{N}^{12}  \les\ve_{1}^{3}M^{-(1+\de)}\bra{N}^{-k}.
\end{align*}
Similarly, \eqref{eq:k_2 bound} implies that
\begin{align*}
 & \sum_{(\nl)\in\mathcal{B}}\left|\mathcal{K}_{2}(s,\xi)\right|\\
 & \les M^{-2}\sum_{(\nl)\in\mathcal{B}}L^{-2}(L')^{-2}\langle\max(N_1,N_3)\rangle^2\langle\min(N_1,N_3)\rangle^{12} \\ 
 &\qquad\qquad\qquad\qquad\qquad\qquad\qquad\qquad\times \normo{P_{N_{2}}u}_{L^{\infty}}\normo{xf}_{L^{2}}\left( \normo{f_{N_{3}}}_{L^{2}} + \normo{f_{N_{1}}}_{L^{2}}\right)\\
 & \les \ve_1^3 M^{-2+\delta_0}\sum_{ \substack{M^{-(1+\de)}\le N_{1},N_{2},N_{3}\le M^{\frac{2}{n}} \\   L_0\le L \le \max(N_2,N_3)}}L^{-1}\langle\max(N_1,N_3)\rangle^2 \langle\min(N_1,N_3)\rangle^{12} \langle N_2\rangle^{-k}\\
 &\hspace{9cm}\times\Big(N_1^{\frac12}\langle N_1\rangle^{-k}+N_3^{\frac12}\langle N_3\rangle^{-k}\Big ) \\
  & \les \ve_1^3 M^{-\frac{11}{10}+\delta_0}\sum_{ M^{-(1+\de)}\le N_{1},N_{2},N_{3}\le M^{\frac{2}{n}} }\langle\max(N_1,N_3)\rangle^2 \langle\min(N_1,N_3)\rangle^{12} \langle N_2\rangle^{-k}\\
  &\hspace{9cm}\times \Big(N_1^{\frac12}\langle N_1\rangle^{-k}+N_3^{\frac12}\langle N_3\rangle^{-k}\Big ) \\
 & \les \ve_1^3 M^{-\frac{11}{10}+\delta_0}\langle N\rangle^{14}  \les\ve_{1}^{3}M^{-(1+\de)}\bra{N}^{-k}.
 \end{align*}
Finally, using the pointwise bound in \eqref{eq:pw-bound-k3}, we estimate 
\begin{align*}
    & \sum_{(\nl)\in\mathcal{B}}\left|\mathcal{K}_{3}(s,\xi)\right|\\
    & \les M^{-2}\sum_{(\nl)\in\mathcal{B}}L^{-3}(L')^{-2}\langle\max(N_1,N_3)\rangle^2\langle\min(N_1,N_3)\rangle^{4}\|\rho_{L}\|_{L^{1}}\|\rho_{L'}\|_{L^{1}}
    \normo{\wh{f_{N_{1}}}}_{L^{\infty}}\normo{\wh{f_{N_{2}}}}_{L^{\infty}}\normo{\wh{f_{N_{3}}}}_{L^{\infty}}\\
    & \les\ve_{1}^{3}M^{-2}\sum_{(\nl)\in\mathcal{B}}L^{-1}\langle\max(N_1,N_3)\rangle^2\langle\min(N_1,N_3)\rangle^{4}\bra{N_{1}}^{-k}\bra{N_{2}}^{-k}\bra{N_{3}}^{-k}\\
    & \les\ve_{1}^{3}M^{-\frac{11}{10}}\sum_{ \substack{M^{-(1+\de)}\le N_{1},N_{2},N_{3}\le M^{\frac{2}{n}} \\ L'\sim \max(N_1,N_3)} }\langle\max(N_1,N_3)\rangle^2\langle\min(N_1,N_3)\rangle^{4}\bra{N_{1}}^{-k}\bra{N_{2}}^{-k}\bra{N_{3}}^{-k}\\
    & \les\ve_{1}^{3}M^{-(1+\de)}\bra{N}^{-k}.
   \end{align*}



%We further divide the above set into two cases:
%\begin{align*}
%\textbf{Case 1: }\mathcal{A}_{1} & =\mathcal{A}\cap\left\{ L(L')^{2}\ge M^{-(1+2\de_{0})}\;\;\mbox{and}\;\;\max{(N_{1},N_{3})}\le L\right\}, \\ 
%\textbf{Case 2: }\mathcal{A}_{2} & =\mathcal{A}\cap\left\{ L(L')^{2}\ge M^{-(1+2\de_{0})}\;\;\mbox{and}\;\;\max(N_{1},N_{3})\ge L \right\}.
%\end{align*}
%In the remainder of this section, we are devoted to verifying two cases. Since the singularity from $|\eta|^{-1}$ is excluded in these cases, we can adopt normal form approach by repeatedly exploiting the space non-resonance in $\eta$ integration.
%
%\medskip
%
%\noindent\textbf{Case 1.}
%Using the following relation 
%\[
%e^{isq\freq}=-i\frac{1}{s}\frac{(\nabla_{\eta}q)\cdot\nabla_{\eta}e^{isq}}{|\nabla_{\eta}q|^{2}},
%\]
%we perform an integration by parts in $\eta$ variable, up to irrelevant constants, and with a slight abuse of notation, to obtain 
%\begin{align*}
%    \mathcal{K}_{\nl}(s,\xi) & =\mathcal{K}_{1}(s,\xi)+\mathcal{K}_{2}(s,\xi),\\
%    \mathcal{K}_{1}(s,\xi) & :=\frac{i}{s}\iint_{\R^{2}\times\R^{2}}e^{isq\freq}\textbf{m}_{1}(\xi,\eta,\sigma)\nabla_{\eta}\left( \wh{f_{N_{1}}}(s,\xi+\eta)\overline{\wh{f_{N_{3}}}(s,\xi+\eta+\sigma)}\right)\\
%     & \hspace{6cm}\times\wh{f_{N_{2}}}(s,\xi+\sigma)d\eta d\sigma,\\
%    \mathcal{K}_{2}(s,\xi) & :=\frac{i}{s}\iint_{\R^{2}\times\R^{2}}e^{isq\freq}\textbf{m}_{2}(\xi,\eta,\sigma)\wh{f_{N_{1}}}(s,\xi+\eta)\wh{f_{N_{2}}}(s,\xi+\sigma)\\
%     & \hspace{6cm}\times\overline{\wh{f_{N_{3}}}(s,\xi+\eta+\sigma)}d\eta d\sigma,
%    \end{align*}
%% \begin{align*}
%% \mathcal{K}_{\nl}(s,\xi) & =\mathcal{K}_{1a}(s,\xi)+\mathcal{K}_{1b}(s,\xi)+\mathcal{K}_{2}(s,\xi),\\
%% \mathcal{K}_{1a}(s,\xi) & :=\frac{i}{s}\iint_{\R^{2}\times\R^{2}}e^{isq\freq}\textbf{m}_{1}(\xi,\eta,\sigma)\nabla_{\eta}\wh{f_{N_{1}}}(s,\xi+\eta)\wh{f_{N_{2}}}(s,\xi+\sigma)\\
%%  & \hspace{6cm}\times\overline{\wh{f_{N_{3}}}(s,\xi+\eta+\sigma)}d\eta d\sigma,\\
%% \mathcal{K}_{1b}(s,\xi) & :=\frac{i}{s}\iint_{\R^{2}\times\R^{2}}e^{isq\freq}\textbf{m}_{1}(\xi,\eta,\sigma)\wh{f_{N_{1}}}(s,\xi+\eta)\wh{f_{N_{2}}}(s,\xi+\sigma)\\
%%  & \hspace{6cm}\times\nabla_{\eta}\overline{\wh{f_{N_{3}}}(s,\xi+\eta+\sigma)}d\eta d\sigma,\\
%% \mathcal{K}_{2}(s,\xi) & :=\frac{i}{s}\iint_{\R^{2}\times\R^{2}}e^{isq\freq}\textbf{m}_{2}(\xi,\eta,\sigma)\wh{f_{N_{1}}}(s,\xi+\eta)\wh{f_{N_{2}}}(s,\xi+\sigma)\\
%%  & \hspace{6cm}\times\overline{\wh{f_{N_{3}}}(s,\xi+\eta+\sigma)}d\eta d\sigma,
%% \end{align*}
%where the multipliers $\textbf{m}_{1}$ and $\textbf{m}_{2}$ are defined by
%\begin{align}
%\begin{aligned}\label{eq:multi-m}\textbf{m}_{1}(\xi,\eta,\sigma) & =\frac{\nabla_{\eta}q\freq}{|\nabla_{\eta}q\freq|^{2}}|\eta|^{-1}\rho_{L}(\eta)\rho_{L'}(\sigma)\\
% & \hspace{3cm}\times\rho_{N_{1}}(\xi+\eta)\rho_{N_{3}}(\xi+\eta+\sigma),\\
%\textbf{m}_{2}(\xi,\eta,\sigma) & =\nabla_{\eta}\left(\frac{\nabla_{\eta}q\freq}{|\nabla_{\eta}q\freq|^{2}}|\eta|^{-1}\rho_{L}(\eta)\right)\rho_{L'}(\sigma)\\
% & \hspace{3cm}\times\rho_{N_{1}}(\xi+\eta)\rho_{N_{3}}(\xi+\eta+\sigma).
%\end{aligned}
%\end{align}
%By using the following estimates
%\begin{align}
%    \begin{aligned}
%    \label{lower bound of nabla q}
%    \left|\nabla_{\eta}q(\xi,\eta,\sigma)\right|  
%    &=\left|\frac{\xi+\eta+\sigma}{\bra{\xi+\eta+\sigma}}-\frac{\xi+\eta}{\bra{\xi+\eta}}\right| \\ 
%    &\gtrsim \frac{|\sigma|}{\max(\langle \xi+\eta+\sigma\rangle, \langle\xi+\eta\rangle) \min(\langle \xi+\eta+\sigma\rangle, \langle\xi+\eta\rangle)^2},
%\end{aligned}\end{align}
%one can verify that 
%\begin{align*}
%    & \normo{\iint_{\R^{2}\times\R^{2}}\textbf{m}_{1}(\xi,\eta,\sigma)e^{iy\cdot\eta}e^{iz\cdot\sigma}\,d\sigma d\eta}_{L_{y,z}^{1}(\R^{2}\times\R^{2})} \\ 
%    &\qquad\qquad\qquad\qquad\qquad\qquad \les L^{-1}(L')^{-1}\langle\max(N_1,N_3)\rangle\langle\min(N_1,N_3)\rangle^{10} ,\\
%    & \normo{\iint_{\R^{2}\times\R^{2}}\textbf{m}_{2}(\xi,\eta,\sigma)e^{iy\cdot\eta}e^{iz\cdot\sigma}\,d\sigma d\eta}_{L_{y,z}^{1}(\R^{2}\times\R^{2})}\\ 
%    &\qquad\qquad\qquad\qquad\qquad\qquad \les L^{-2}(L')^{-1}\langle\max(N_1,N_3)\rangle^2\langle\min(N_1,N_3)\rangle^{12} .
%   \end{align*}
%   Without loss of generally, we may assume that $\max(N_1,N_3)=N_1$.   Applying the multiplier estimates \eqref{eq:coif-1-1} and using the a priori bounds, we estimate 
%   \begin{align*}
%       &\sum_{(\nl)\in\mathcal{A}_{1}}\left|\mathcal{K}_{1}(s,\xi)\right| \\ & \les M^{-1}\sum_{(\nl) \in\mathcal{A}_{1}}L^{-1}(L')^{-1}\langle N_1\rangle\langle N_3\rangle^{10}
%        \|xf\|_{L_{x}^{2}}\left(\|\wh{f_{N_{1}}}\|_{L^{2}}+ \|\wh{f_{N_{3}}}\|_{L^{2}}  \right) \|P_{N_{2}}u\|_{L^{\infty}}\\     & \les\ve_{1}^{3}M^{-2+\de_{0}}\sum_{(\nl)\in\mathcal{A}_{1}}L^{-1}(L')^{-1}\langle N_3\rangle^{11}\bra{N_{2}}^{-k}(N_{1}\bra{N_{1}}^{-k}+N_{3}\bra{N_{3}}^{-k})\\
%       & \les\ve_{1}^{3}M^{-\frac32+2\de_{0}}
%       \sum_{ \substack{M^{-(1+\de_{0})}\le N_{1},N_{2},N_{3}\le M^{\frac{2}{n}} \\  N_1\le L }}
%       L^{-\frac12}\langle N_1\rangle\langle N_3\rangle^{10}\bra{N_{2}}^{-k}(N_{1}\bra{N_{1}}^{-k}+N_{3}\bra{N_{3}}^{-k})\\
%       & \les\ve_{1}^{3}M^{-\frac32+2\de_{0}}
%       \sum_{ M^{-(1+\de_{0})}\le N_{1},N_{2},N_{3}\le M^{\frac{2}{n}} }
%       \langle N_1\rangle\langle N_3\rangle^{10}\bra{N_{2}}^{-k}(N_{1}^\frac12\bra{N_{1}}^{-k}+N_{3}^\frac12\bra{N_{3}}^{-k})\\
%        & \les\ve_{1}^{3}M^{-(1+\delta_{0})}\bra{N}^{-k}.
%   \end{align*}
%% Applying the multiplier estimates \eqref{eq:coif-1-1}, we estimate 
%% \begin{align*}
%%     &\sum_{(\nl)\in\mathcal{A}_{1}}\left|\mathcal{K}_{1a}(s,\xi)\right| \\ & \les M^{-1}\sum_{(\nl) \in\mathcal{A}_{1}}L^{-1}(L')^{-1}\langle\min(N_1,N_3)\rangle^{11}\|xf\|_{L_{x}^{2}}\|P_{N_{2}}u\|_{L^{\infty}}\|\wh{f_{N_{3}}}\|_{L^{2}}\\     & \les\ve_{1}^{3}M^{-2+\de_{0}}\sum_{(\nl)\in\mathcal{A}_{1}}L^{-1}(L')^{-1}\langle\min(N_1,N_3)\rangle^{11}\bra{N_{2}}^{-k}N_{3}\bra{N_{3}}^{-k}\\
%%     & \les\ve_{1}^{3}M^{-\frac32+2\de_{0}}
%%     \sum_{ \substack{M^{-(1+\de_{0})}\le N_{1},N_{2},N_{3}\le M^{\frac{2}{n}} \\  \max(N_1,N_3)\le L }}
%%     L^{-\frac12}\langle\min(N_1,N_3)\rangle^{11}\bra{N_{2}}^{-k}N_{3}\bra{N_{3}}^{-k}\\
%%     & \les\ve_{1}^{3}M^{-\frac32+2\de_{0}}
%%     \sum_{ M^{-(1+\de_{0})}\le N_{1},N_{2},N_{3}\le M^{\frac{2}{n}} }\max(N_1,N_3)^{-\frac12}
%%     \langle\min(N_1,N_3)\rangle^{11}\bra{N_{2}}^{-k}N_{3}\bra{N_{3}}^{-k}\\
%%      & \les\ve_{1}^{3}M^{-(1+\delta_{0})}\bra{N}^{-k}.
%% \end{align*}
%Similarly, we estimate 
%\begin{align*}
%    \sum_{(\nl)\in\mathcal{A}_{1}}\left|\mathcal{K}_{2}(s,\xi)\right| & \les M^{-1}\sum_{(\nl)\in\mathcal{A}_{1}}L^{-2}(L')^{-1}
%    \langle N_1\rangle\langle N_3\rangle^{12}\|\wh{f_{N_{1}}}\|_{L^{2}}\|P_{N_{2}}u\|_{L^{\infty}}\|\wh{f_{N_{3}}}\|_{L^{2}}\\
%    & \les\ve_{1}^{3}M^{-2}\sum_{(\nl)\in\mathcal{A}_{1}}L^{-2}(L')^{-1}\langle N_1\rangle\langle N_3\rangle^{12}N_{1}\bra{N_{1}}^{-k}\bra{N_{2}}^{-k}N_{3}\bra{N_{3}}^{-k}\\
%    & \les\ve_{1}^{3}M^{-\frac32+2\de_{0}}
%    \sum_{ M^{-(1+\de_{0})}\le N_{1},N_{2},N_{3}\le M^{\frac{2}{n}} }
%    N_1^{-\frac12}\langle N_1\rangle^{1-k}\bra{N_{2}}^{-k}N_{3}\bra{N_{3}}^{12-k}\\
%     & \les\ve_{1}^{3}M^{-(1+\delta_{0})}\bra{N}^{-k}.
%    \end{align*}
%
%% Especially, by \eqref{eq:multipl-bound},
%% \eqref{eq:multi-kab} and \eqref{eq:k_1 bound}, we have 
%% \begin{align*}
%%  & \normo{\iint_{\R^{2}\times\R^{2}}\textbf{m}_{1}(\xi,\eta,\sigma)e^{iy\cdot\eta}e^{iz\cdot\sigma}\,d\sigma d\eta}_{L_{y,z}^{1}}\les L^{-1}(L')^{-1}M^{\frac{18}{n}},\\
%%  & \normo{\iint_{\R^{2}\times\R^{2}}\textbf{m}_{2}(\xi,\eta,\sigma)e^{iy\cdot\eta}e^{iz\cdot\sigma}\,d\sigma d\eta}_{L_{y,z}^{1}}\les L^{-2}(L')^{-1}M^{\frac{22}{n}}.
%% \end{align*}
%% If $N_{1}\nsim N_{3}$, by the support condition $\min(N_{1},N_{3})\ll\max(N_{1},N_{3})\sim L'$,
%% one has the same multiplier bounds in either case $N_{1}\nsim N_{3}$
%% or case $N_{1}\sim N_{3}$. Thus we only treat the case $N_{1}\sim N_{3}$.
%% Using the multiplier estimates above, Lemma \ref{lem:coif}, and the
%% fact that $L(L')^{2}\ge M^{-(1+\de_{0})}$, we estimate 
%% \begin{align*}
%% \sum_{(\nl)\in\mathcal{A}_{4}}\left|\mathcal{K}_{1a}(s,\xi)\right| & \les M^{-1+\frac{30}{n}}\sum_{(\nl)\in\mathcal{A}_{4}}L^{-1}(L')^{-1}\|xf\|_{L_{x}^{2}}\|P_{N_{2}}u\|_{L_{x}^{\infty}}\|\wh{f_{N_{3}}}\|_{L_{\xi}^{2}}\\
%%  & \les\ve_{1}^{3}M^{-2+\frac{30}{n}+\de_{0}}\sum_{(\nl)\in\mathcal{A}_{4}}L^{-1}(L')^{-1}\bra{N_{1}}^{-k}\bra{N_{2}}^{-k}N_{3}\bra{N_{3}}^{-k}\\
%%  & \les\ve_{1}^{3}M^{-(1+\delta_{0})}\bra{N}^{-k}
%% \end{align*}
%% and 
%% \begin{align*}
%% \sum_{(\nl)\in\mathcal{A}_{4}}\left|\mathcal{K}_{2}(s,\xi)\right| & \les M^{-1+\frac{30}{n}}\sum_{(\nl)\in\mathcal{A}_{4}}L^{-2}(L')^{-1}\|\wh{f_{N_{1}}}\|_{L_{\xi}^{2}}\|P_{N_{2}}u\|_{L_{x}^{\infty}}\|\wh{f_{N_{3}}}\|_{L_{\xi}^{2}}\\
%%  & \les\ve_{1}^{3}M^{-2+\frac{30}{n}}\sum_{(\nl)\in\mathcal{A}_{4}}L^{-2}(L')^{-1}N_{1}\bra{N_{1}}^{-k}\bra{N_{2}}^{-k}N_{3}\bra{N_{3}}^{-k}\\
%%  & \les\ve_{1}^{3}M^{-(1+\delta_{0})}\bra{N}^{-k}.
%% \end{align*}
%% Therefore, we finish the proof of \eqref{eq:part-scattering}.
%
%\medskip
%
%\noindent\textbf{Case 2.} In this case, we perform an integration by parts in $\eta$ twice to obtain 
%\begin{align*}
%    |\mathcal{K}_{\nl}(s,\xi)| & \le |\wt{\mathcal{K}_{1}}(s,\xi)|+|\wt{\mathcal{K}_{2}}(s,\xi)|+|\wt{\mathcal{K}_{3}}(s,\xi)|,\\
%    \wt{\mathcal{K}_{1}}(s,\xi) & :=\frac{1}{s^{2}}\iint_{\R^{2}\times\R^{2}}e^{isq\freq}\wt{\textbf{m}_{1}}(\xi,\eta,\sigma)\nabla_{\eta}^{2}\left(\wh{f_{N_{1}}}(s,\xi+\eta)\overline{\wh{f_{N_{3}}}(s,\xi+\eta+\sigma)}\right)\\
%     & \hspace{6cm}\times\wh{f_{N_{2}}}(s,\xi+\sigma)d\eta d\sigma,\\
%    \wt{\mathcal{K}_{2}}(s,\xi) & :=\frac{1}{s^{2}}\iint_{\R^{2}\times\R^{2}}e^{isq\freq}\wt{\textbf{m}_{2}}(\xi,\eta,\sigma)\nabla_{\eta}\left(\wh{f_{N_{1}}}(s,\xi+\eta)\overline{\wh{f_{N_{3}}}(s,\xi+\eta+\sigma)}\right)\\
%     & \hspace{6cm}\times\wh{f_{N_{2}}}(s,\xi+\sigma)d\eta d\sigma,\\
%    \wt{\mathcal{K}_{3}}(s,\xi) & :=\frac{1}{s^{2}}\iint_{\R^{2}\times\R^{2}}e^{isq\freq}\wt{\textbf{m}_{3}}(\xi,\eta,\sigma)\left(\wh{f_{N_{1}}}(s,\xi+\eta)\overline{\wh{f_{N_{3}}}(s,\xi+\eta+\sigma)}\right)\\
%     & \hspace{6cm}\times\wh{f_{N_{2}}}(s,\xi+\sigma)d\eta d\sigma,
%    \end{align*}
%    where 
%    \begin{align*}
%    \wt{\textbf{m}_{1}}(\xi,\eta,\sigma) & :=\frac{\nabla_{\eta}q\freq}{|\nabla_{\eta}q\freq|^{2}}\textbf{m}_{1}\freq,\\
%    \wt{\textbf{m}_{2}}(\xi,\eta,\sigma) & :=\frac{\nabla_{\eta}q\freq}{|\nabla_{\eta}q\freq|^{2}}\textbf{m}_{2}(\xi,\eta,\sigma),\\
%    \wt{\textbf{m}_{3}}(\xi,\eta,\sigma) & :=\nabla_{\eta}\left(\frac{\nabla_{\eta}q\freq}{|\nabla_{\eta}q\freq|^{2}}\textbf{m}_{2}\freq\right),
%    \end{align*}
%where the symbols $\mathbf{m}_1$ and $\mathbf{m}_2$ were defined in \eqref{eq:multi-m}.
%A direct computation yields that 
%\begin{align}
%\left\Vert \iint_{\R^{2}\times\R^{2}}e^{iy\cdot\eta}e^{iz\cdot\sigma}\wt{\textbf{m}_{1}}\freq d\eta d\sigma\right\Vert _{L_{y,z}^{1}} & \les  L^{-1}(L')^{-2}\langle\max(N_1,N_3)\rangle^2\langle\min(N_1,N_3)\rangle^{12},\label{eq:k_1 bound}
%\end{align}
%and 
%\begin{align}
%\begin{aligned}\label{eq:pw-bound-k23}\left|\wt{\textbf{m}_{2}}\freq\right| & \les L^{-2}(L')^{-2}\langle\max(N_1,N_3)\rangle^2\langle\min(N_1,N_3)\rangle^{6},\\
%\left|\wt{\textbf{m}_{3}}\freq\right| & \les L^{-3}(L')^{-2}\langle\max(N_1,N_3)\rangle^2\langle\min(N_1,N_3)\rangle^{8}.
%\end{aligned}
%\end{align}
%By applying the operator inequality \eqref{eq:coif-1-1} with \eqref{eq:k_1 bound}, we estimate 
%\begin{align*}
% & \sum_{(\nl)\in\mathcal{A}_{2}}\left|\wt{\mathcal{K}_{1}}(s,\xi)\right|\\
% & \les M^{-2}\sum_{(\nl)\in\mathcal{A}_{2}}L^{-1}(L')^{-2}\normo{P_{N_{2}}u}_{L_{x}^{\infty}}\left( \normo{x^{2}f}_{L^{2}}\normo{f_{N_{3}}}_{L_{x}^{2}} + \normo{xf}_{L^{2}}^2 + \normo{x^{2}f}_{L^{2}}\normo{f_{N_{1}}}_{L_{x}^{2}}\right)\\
% & \les \ve_1^3 M^{-2+\delta_0}\sum_{ \substack{M^{-(1+\de_{0})}\le N_{1},N_{2},N_{3}\le M^{\frac{2}{n}} \\   L_0\le L \le \max(N_1,N_3)}} \langle N_2\rangle^{-k}\\
% & \les\ve_{1}^{3}M^{-(1+\de_{0})}\bra{N}^{-k}.
%\end{align*}
%Next, by using the pointwise bound \eqref{eq:pw-bound-k23}, one can see that 
%\begin{align*}
% & \sum_{(\nl)\in\mathcal{A}_{2}}\left|\wt{\mathcal{K}_{2}}(s,\xi)\right|\\
% & \les M^{-2}\sum_{(\nl)\in\mathcal{A}_{2}}L^{-2}(L')^{-2}\langle\max(N_1,N_3)\rangle^2\langle\min(N_1,N_3)\rangle^{6}\|\rho_{L}\|_{L_{\eta}^{1}}\|\rho_{L'}\|_{L_{\sigma}^{1}} \\ 
% &\qquad\qquad \times 
% \normo{\wh{f_{N_{1}}}}_{L^{\infty}}\normo{\wh{f_{N_{2}}}}_{L^{\infty}}\normo{\wh{f_{N_{3}}}}_{L^{\infty}}\\
% & \les\ve_{1}^{3}M^{-2}\sum_{(\nl)\in\mathcal{A}_{2}}\langle\max(N_1,N_3)\rangle^2\langle\min(N_1,N_3)\rangle^{6}\bra{N_{1}}^{-k}\bra{N_{2}}^{-k}\bra{N_{3}}^{-k}\\
% & \les\ve_{1}^{3}M^{-2+\de_{0}}\bra{N}^{-k}\les\ve_{1}^{3}M^{-(1+\de_{0})}\bra{N}^{-k}
%\end{align*}
%and 
%\begin{align*}
% & \sum_{(\nl)\in\mathcal{A}_{3}}\left|\wt{\mathcal{K}_{3}}(s,\xi)\right|\\
% & \les M^{-2}\sum_{(\nl)\in\mathcal{A}_{3}}L^{-3}(L')^{-2}\bra{N_{1}}^{6}\|\rho_{L}\|_{L_{\eta}^{1}}\|\rho_{L'}\|_{L_{\sigma}^{1}}\normo{\wh{f_{N_{1}}}}_{L_{\xi}^{\infty}}\normo{\wh{f_{N_{2}}}}_{L_{\xi}^{\infty}}\normo{\wh{f_{N_{3}}}}_{L_{\xi}^{\infty}}\\
% & \les\ve_{1}^{3}M^{-2}\sum_{(\nl)\in\mathcal{A}_{3}}L^{-1}\bra{N_{1}}^{6-k}\bra{N_{2}}^{-k}\bra{N_{3}}^{-k}\\
% & \les\ve_{1}^{3}M^{-2}M^{1-3\de_{0}}\bra{N}^{-k}\les\ve_{1}^{3}M^{-(1+\de_{0})}\bra{N}^{-k}.
%\end{align*}
%
%
%
%\begin{align}
%\begin{aligned}\mathcal{K}_{\nl}(s,\xi) & =\mathcal{K}_{1a}(s,\xi)+\mathcal{K}_{1b}(s,\xi)+\mathcal{K}_{2}(s,\xi),\\
%\mathcal{K}_{1a}(s,\xi) & :=\frac{i}{s}\iint_{\R^{2}\times\R^{2}}e^{isp\freq}\textbf{m}_{1}(\xi,\eta,\sigma)\nabla_{\eta}\wh{f_{N_{1}}}(s,\xi+\eta)\wh{f_{N_{2}}}(s,\xi+\sigma)\\
% & \hspace{6cm}\times\overline{\wh{f_{N_{3}}}(s,\xi+\eta+\sigma)}d\eta d\sigma,\\
%\mathcal{K}_{1b}(s,\xi) & :=\frac{i}{s}\iint_{\R^{2}\times\R^{2}}e^{isp\freq}\textbf{m}_{1}(\xi,\eta,\sigma)\wh{f_{N_{1}}}(s,\xi+\eta)\wh{f_{N_{2}}}(s,\xi+\sigma)\\
% & \hspace{6cm}\times\overline{\nabla_{\eta}\wh{f_{N_{3}}}(s,\xi+\eta+\sigma)}d\eta d\sigma,\\
%\mathcal{K}_{2}(s,\xi) & :=\frac{i}{s}\iint_{\R^{2}\times\R^{2}}e^{isp\freq}\textbf{m}_{2}(\xi,\eta,\sigma)\wh{f_{N_{1}}}(s,\xi+\eta)\wh{f_{N_{2}}}(s,\xi+\sigma)\\
% & \hspace{6cm}\times\overline{\wh{f_{N_{3}}}(s,\xi+\eta+\sigma)}d\eta d\sigma,
%\end{aligned}
%\label{eq:k-decompose}
%\end{align}
%where 
%\begin{align}
%\begin{aligned}\label{eq:multi-m}\textbf{m}_{1}(\xi,\eta,\sigma) & =\frac{\nabla_{\eta}p\freq}{|\nabla_{\eta}p\freq|^{2}}|\eta|^{-1}\rho_{L}(\eta)\rho_{L'}(\sigma)\\
% & \hspace{3cm}\times\rho_{N_{1}}(\xi+\eta)\rho_{N_{3}}(\xi+\eta+\sigma),\\
%\textbf{m}_{2}(\xi,\eta,\sigma) & =\nabla_{\eta}\left(\frac{\nabla_{\eta}p\freq}{|\nabla_{\eta}p\freq|^{2}}|\eta|^{-1}\rho_{L}(\eta)\right)\rho_{L'}(\sigma)\\
% & \hspace{3cm}\times\rho_{N_{1}}(\xi+\eta)\rho_{N_{3}}(\xi+\eta+\sigma).
%\end{aligned}
%\end{align}
%
%By the observation \eqref{eq:nonresonance-eta} and $L'\sim\max\left(N_{1},N_{3}\right)$,
%we obtain, for $0\le m,l\le3$, 
%\begin{align}
%\begin{aligned} & \nabla_{\eta}^{m}\nabla_{\sigma}^{l}\left(\frac{\nabla_{\eta}p\freq}{|\nabla_{\eta}p\freq|^{2}}\right)\\
% & \qquad\les\max\left(N_{1},N_{3}\right)^{-1}\max(\bra{N_{1}},\bra{N_{3}})^{2m+l+3}L^{-m}(L')^{-l}.
%\end{aligned}
%\label{eq:deri-phase}
%\end{align}
%%Since $\max(N_1,N_3) \ge L$, we obtain
%%\[
%%\nabla_\eta \Big[\rho_{L}(\eta)\rho_{L'}(\sigma)\rho_{N_{1}}(\xi+\eta)\rho_{N_{3}}(\xi+\eta+\sigma)\Big] \les L^{-1} + N_1^{-1}  + N_3^{-1} \les L^{-1}.
%%\]
%%By repeating this derivation similarly, we see that
%%\[
%%\nabla_\eta^m \nabla_\sigma^l \Big[\rho_{L}(\eta)\rho_{L'}(\sigma)\rho_{N_{1}}(\xi+\eta)\rho_{N_{3}}(\xi+\eta+\sigma)\Big] \les L^{-m} (L')^{-l}.
%%\]
%Then we estimate 
%\begin{align}
%\begin{aligned} & \left\Vert \iint_{\R^{2}\times\R^{2}}\textbf{m}_{1}(\xi,\eta,\sigma)e^{iy\cdot\eta}e^{iz\cdot\sigma}d\eta d\sigma\right\Vert _{L_{y,z}^{1}}\\
% & \qquad\qquad\les L^{-1}\max(N_{1},N_{3})^{-1}\max\left(\bra{N_{1}},\bra{N_{3}}\right)^{9}\\
% & \qquad\qquad\les L^{-1}\max(N_{1},N_{3})^{-1}M^{\frac{18}{n}}.
%\end{aligned}
%\label{eq:multipl-bound}
%\end{align}
%Hence Lemma \ref{lem:coif} leads us that 
%\begin{align*}
% & \sum_{(\nl)\in\mathcal{A}_{2}}|\mathcal{K}_{1a}(s,\xi)|\\
% & \les M^{-1+\frac{18}{n}}\sum_{(\nl)\in\mathcal{A}_{2}}L^{-1}\max(N_{1},N_{3})^{-1}\normo{\nabla_{\xi}\wh{f_{N_{1}}}}_{L_{\xi}^{2}}\|P_{N_{2}}u\|_{L_{x}^{\infty}}\normo{\wh{f_{N_{3}}}}_{L_{\xi}^{2}}\\
% & \les\ve_{1}^{3}M^{-2+\frac{18}{n}+\de_{0}}\sum_{(\nl)\in\mathcal{A}_{2}}L^{-1}\max(N_{1},N_{3})^{-1}\bra{N_{2}}^{-k}N_{3}\bra{N_{3}}^{-k}\\
% & \les\ve_{1}^{3}M^{-2+\frac{18}{n}+\de_{0}}M^{1-3\de_{0}}M^{\frac{1}{n}}M^{\frac{2k}{n}}\bra{N}^{-k}\les\ve_{1}^{3}M^{-(1+\de_{0})}\bra{N}^{-k},
%\end{align*}
%Here we used $M^{-1+3\de_{0}}\sim L_{0}\le L$, and \eqref{eq:time-decay}.
%Since $\mathcal{K}_{1b}(s,\xi)$ is estimated similarly, we omit the
%proof.
%
%To estimate $\mathcal{K}_{2}(s,\xi)$ we further use the integration
%by parts in $\eta$. Indeed, we get 
%\begin{align*}
%\mathcal{K}_{2}(s,\xi) & =\mathcal{K}_{3a}(s,\xi)+\mathcal{K}_{3b}(s,\xi)+\mathcal{K}_{4}(s,\xi),\\
%\mathcal{K}_{3a}(s,\xi) & :=-\frac{1}{s^{2}}\iint_{\R^{2}\times\R^{2}}e^{isp\freq}\textbf{m}_{3}(\xi,\eta,\sigma)\nabla_{\eta}\wh{f_{N_{1}}}(s,\xi+\eta)\wh{f_{N_{2}}}(s,\xi+\sigma)\\
% & \hspace{6cm}\times\overline{\wh{f_{N_{3}}}(s,\xi+\eta+\sigma)}d\eta d\sigma,\\
%\mathcal{K}_{3b}(s,\xi) & :=-\frac{1}{s^{2}}\iint_{\R^{2}\times\R^{2}}e^{isp\freq}\textbf{m}_{3}(\xi,\eta,\sigma)\wh{f_{N_{1}}}(s,\xi+\eta)\wh{f_{N_{2}}}(s,\xi+\sigma)\\
% & \hspace{6cm}\times\overline{\nabla_{\eta}\wh{f_{N_{3}}}(s,\xi+\eta+\sigma)}d\eta d\sigma,\\
%\mathcal{K}_{4}(s,\xi) & :=-\frac{1}{s^{2}}\iint_{\R^{2}\times\R^{2}}e^{isp\freq}\textbf{m}_{4}(\xi,\eta,\sigma)\wh{f_{N_{1}}}(s,\xi+\eta)\wh{f_{N_{2}}}(s,\xi+\sigma)\\
% & \hspace{6cm}\times\overline{\wh{f_{N_{3}}}(s,\xi+\eta+\sigma)}d\eta d\sigma,\\
%\end{align*}
%where 
%\begin{align}
%\begin{aligned}\textbf{m}_{3}(\xi,\eta,\sigma) & :=\frac{\nabla_{\eta}p\freq}{|\nabla_{\eta}p\freq|^{2}}\textbf{m}_{2}\freq,\\
%\textbf{m}_{4}(\xi,\eta,\sigma) & :=\nabla_{\eta}\left(\frac{\nabla_{\eta}\phase\freq}{|\nabla_{\eta}\phase\freq|^{2}}\textbf{m}_{2}\freq\right).
%\end{aligned}
%\label{eq:multi-kab}
%\end{align}
%A similar computation to \eqref{eq:multipl-bound} also shows 
%\begin{align}
%\begin{aligned} & \normo{\iint_{\R^{2}\times\R^{2}}e^{iy\cdot\eta}e^{iz\cdot\sigma}\textbf{m}_{3}(\xi,\eta,\sigma)d\eta d\sigma}_{L_{y,z}^{1}}\\
% & \qquad\qquad\les L^{-2}\max(N_{1},N_{3})^{-2}\max({\bra{N_{1}},\bra{N_{3}}})^{14}\\
% & \qquad\qquad\les L^{-2}\max(N_{1},N_{3})^{-2}M^{\frac{28}{n}}.\label{eq:k_{a}-bound}
%\end{aligned}
%\end{align}
%By \eqref{eq:k_a-bound}, Lemma \ref{lem:coif}, and Proposition \ref{timedecay-prop},
%we estimate 
%\begin{align*}
% & \sum_{(\nl)\in\mathcal{A}_{2}}|\mathcal{K}_{3a}(s,\xi)|\\
% & \les M^{-2+\frac{28}{n}}\sum_{(\nl)\in\mathcal{A}_{2}}L^{-2}\max(N_{1},N_{3})^{-2}\normo{\nabla_{\xi}\wh{f_{N_{1}}}}_{L_{\xi}^{2}}\normo{P_{N_{2}}u}_{L_{x}^{\infty}}\normo{\wh{f_{N_{3}}}}_{L_{\xi}^{2}}\\
% & \les\ve_{1}^{3}M^{-3+\frac{28}{n}+\de_{0}}\sum_{(\nl)\in\mathcal{A}_{2}}L^{-2}(L')^{-1}\max(N_{1},N_{3})^{-1}\bra{N_{2}}^{-k}N_{3}\bra{N_{3}}^{-k}\\
% & \les\ve_{1}^{3}M^{-3+\frac{28}{n}+\de_{0}}M^{\frac{3}{2}-\frac{9}{2}\de_{0}}M^{\frac{1}{2}+\frac{1}{2}\de_{0}}M^{\frac{2k}{n}}\bra{N}^{-k}\les\ve_{1}^{3}M^{-(1+\de_{0})}\bra{N}^{-k}.
%\end{align*}
%In the third inequality, we used the fact that $M^{-1+3\de_{0}}\le L$
%and $L(L')^{2}\ge M^{-(1+\de_{0})}$. Estimates for $\mathcal{K}_{3b}(s,\xi)$
%can be treated similarly to the above.
%
%From the condition of dyadic pieces $L\les M^{\frac{2}{n}}$ and $L(L')^{2}\ge M^{-(1+\de_{0})}$,
%we have 
%\begin{align}
%M^{-\frac{1}{2}-\frac{\de_{0}}{2}-\frac{1}{n}}\les L',\qquad L'\sim|\sigma|\les\max(N_{1},N_{3}).\label{eq:l'}
%\end{align}
%As considered around \eqref{eq:deri-phase} and \eqref{eq:multipl-bound},
%we have the pointwise multiplier bound 
%\begin{align*}
%|\textbf{m}_{4}(\xi,\eta,\sigma)| & \les L^{-3}\max(N_{1},N_{3})^{-2}\max(\bra{N_{1}}^{6},\bra{N_{3}}^{6})\les L^{-3}\max(N_{1},N_{3})^{-2}M^{\frac{12}{n}},
%\end{align*}
%which implies that 
%\begin{align*}
% & \sum_{(\nl)\in\mathcal{A}_{2}}|\mathcal{K}_{4}(s,\xi)|\\
% & \les M^{-2+\frac{12}{n}}\sum_{(\nl)\in\mathcal{A}_{2}}L^{-3}\max(N_{1},N_{3})^{-2}\|\rho_{L}\|_{L_{\eta}^{1}}\normo{\wh{f_{N_{1}}}}_{L_{\xi}^{2}}\normo{\wh{f_{N_{2}}}}_{L_{\xi}^{\infty}}\normo{\wh{f_{N_{3}}}}_{L_{\xi}^{2}}\\
% & \les\ve_{1}^{3}M^{-2+\frac{12}{n}}\sum_{(\nl)\in\mathcal{A}_{2}}L^{-1}\max(N_{1},N_{3})^{-2}N_{1}\bra{N_{1}}^{-k}\bra{N_{2}}^{-k}N_{3}\bra{N_{3}}^{-k}\\
% & \les\ve_{1}^{3}M^{-2+\frac{12}{n}}M^{1-3\de_{0}}\bra{N}^{-k}\les\ve_{1}^{3}M^{-(1+\de_{0})}\bra{N}^{-k}.
%\end{align*}
%
%\noindent\textbf{Case 3.}
%
%The estimate \eqref{eq:goal-high} will be derived from integrating
%by parts twice for $\mathcal{K}_{\textbf{L},\textbf{N}}(s,\xi)$ and
%a priori assumption \eqref{assumption-apriori}. Indeed, by the mean
%value theorem with the fact $N_{1}\sim N_{3}$, we have 
%\begin{equation}
%|\nabla_{\eta}p(\xi,\eta,\sigma)|=\left|\frac{\xi+\eta}{\bra{\xi+\eta}}-\frac{\xi+\eta+\sigma}{\bra{\xi+\eta+\sigma}}\right|\gtrsim\frac{L'}{\bra{N_{1}}^{3}}.\label{eq:p_lam bound case d}
%\end{equation}
%Then we consider 
%\[
%e^{isp}=-i\frac{1}{s}\frac{\nabla_{\eta}p\cdot\nabla_{\eta}e^{isp}}{|\nabla_{\eta}p|^{2}},
%\]
%which enables us to take the integration by parts in $\eta$ twice
%for $\mathcal{K}_{\nl}(s,\xi)$: 
%\begin{align*}
%\mathcal{K}_{\nl}(s,\xi) & =\wt{\mathcal{K}_{1}}(s,\xi)+\wt{\mathcal{K}_{2}}(s,\xi)+\wt{\mathcal{K}_{3}}(s,\xi),\\
%\wt{\mathcal{K}_{1}}(s,\xi) & :=-\frac{1}{s^{2}}\iint_{\R^{3+3}}e^{isp\freq}\wt{\textbf{k}_{1}}(\xi,\eta,\sigma)\nabla_{\eta}^{2}\left(\wh{f_{N_{1}}}(s,\xi+\eta)\overline{\wh{f_{N_{3}}}(s,\xi+\eta+\sigma)}\right)\\
% & \hspace{6cm}\times\wh{f_{N_{2}}}(s,\xi+\sigma)d\eta d\sigma,\\
%\wt{\mathcal{K}_{2}}(s,\xi) & :=-\frac{1}{s^{2}}\iint_{\R^{3+3}}e^{isp\freq}\wt{\textbf{k}_{2}}(\xi,\eta,\sigma)\nabla_{\eta}\left(\wh{f_{N_{1}}}(s,\xi+\eta)\overline{\wh{f_{N_{3}}}(s,\xi+\eta+\sigma)}\right)\\
% & \hspace{6cm}\times\wh{f_{N_{2}}}(s,\xi+\sigma)d\eta d\sigma,\\
%\wt{\mathcal{K}_{3}}(s,\xi) & :=-\frac{1}{s^{2}}\iint_{\R^{3+3}}e^{isp\freq}\wt{\textbf{k}_{3}}(\xi,\eta,\sigma)\left(\wh{f_{N_{1}}}(s,\xi+\eta)\overline{\wh{f_{N_{3}}}(s,\xi+\eta+\sigma)}\right)\\
% & \hspace{6cm}\times\wh{f_{N_{2}}}(s,\xi+\sigma)d\eta d\sigma,
%\end{align*}
%where 
%\begin{align*}
%\wt{\textbf{k}_{1}}(\xi,\eta,\sigma) & =\frac{\nabla_{\eta}p\freq}{|\nabla_{\eta}p\freq|^{2}}\textbf{m}_{1}\freq,\\
%\wt{\textbf{k}_{2}}(\xi,\eta,\sigma) & :=\frac{\nabla_{\eta}p\freq}{|\nabla_{\eta}p\freq|^{2}}\textbf{m}_{2}(\xi,\eta,\sigma),\\
%\wt{\textbf{k}_{3}}(\xi,\eta,\sigma) & :=\textbf{m}_{3}(\xi,\eta,\sigma),
%\end{align*}
%where $\textbf{m}_{2}$ and $\textbf{m}_{3}$ are defined as in \eqref{eq:multi-m},
%\eqref{eq:multi-kab}, respectively. In a similar way to \eqref{eq:k_a-bound},
%we get the multiplier bound 
%\begin{align}
%\left\Vert \iint_{\R^{2}\times\R^{2}}e^{iy\cdot\eta}e^{iz\cdot\sigma}\wt{\textbf{k}_{1}}\freq d\eta d\sigma\right\Vert _{L_{y,z}^{1}} & \les L^{-1}(L')^{-2}\bra{N_{1}}^{12}\les L^{-1}(L')^{-2}M^{\frac{24}{n}},\label{eq:k_1 bound}
%\end{align}
%and the pointwise bounds 
%\begin{align}
%\begin{aligned}\label{eq:pw-bound-k23}\left|\wt{\textbf{k}_{2}}\freq\right| & \les L^{-2}(L')^{-2}\bra{N_{1}}^{6},\\
%\left|\wt{\textbf{k}_{3}}\freq\right| & \les L^{-3}(L')^{-2}\bra{N_{1}}^{6}.
%\end{aligned}
%\end{align}
%Together with \eqref{eq:condi-piece}, \eqref{eq:l'}, Proposition
%\ref{timedecay-prop}, a priori assumption \eqref{assumption-apriori},
%one shows 
%\begin{align*}
% & \sum_{(\nl)\in\mathcal{A}_{3}}\left|\wt{\mathcal{K}_{1}}(s,\xi)\right|\\
% & \les M^{-2+\frac{24}{n}}\sum_{(\nl)\in\mathcal{A}_{3}}L^{-1}(L')^{-2}\normo{x^{2}f_{N_{1}}}_{L_{x}^{2}}\normo{P_{N_{2}}u}_{L_{x}^{\infty}}\normo{f_{N_{3}}}_{L_{x}^{2}}\\
% & \les\ve_{1}^{3}M^{-2+\frac{24}{n}}M^{1+\de_{0}}M^{2\de_{0}}M^{-1}\bra{N}^{-k}\les\ve_{1}^{3}M^{-(1+\de_{0})}\bra{N}^{-k}.
%\end{align*}
%Here we also used 
%\[
%L\ge M^{-1+3\de_{0}}\;\;\mbox{and}\;\;L(L')^{2}\ge M^{-(1+\de_{0})}
%\]
%for the last inequality for $\wt{\mathcal{K}_{1}}(s,\xi)$.
%
%To complete the proof for this case, we use \eqref{eq:pw-bound-k23}
%and H�lder inequality. Then we readily see that 
%\begin{align*}
% & \sum_{(\nl)\in\mathcal{A}_{3}}\left|\wt{\mathcal{K}_{2}}(s,\xi)\right|\\
% & \les M^{-2}\sum_{(\nl)\in\mathcal{A}_{3}}L^{-2}(L')^{-2}\bra{N_{1}}^{6}\|\rho_{L}\|_{L_{\eta}^{1}}\|\rho_{L'}\|_{L_{\sigma}^{1}}\normo{\wh{f_{N_{1}}}}_{L_{\xi}^{\infty}}\normo{\wh{f_{N_{2}}}}_{L_{\xi}^{\infty}}\normo{\wh{f_{N_{3}}}}_{L_{\xi}^{\infty}}\\
% & \les\ve_{1}^{3}M^{-2}\sum_{(\nl)\in\mathcal{A}_{3}}\bra{N_{1}}^{6-k}\bra{N_{2}}^{-k}\bra{N_{3}}^{-k}\\
% & \les\ve_{1}^{3}M^{-2+\de_{0}}\bra{N}^{-k}\les\ve_{1}^{3}M^{-(1+\de_{0})}\bra{N}^{-k}
%\end{align*}
%and 
%\begin{align*}
% & \sum_{(\nl)\in\mathcal{A}_{3}}\left|\wt{\mathcal{K}_{3}}(s,\xi)\right|\\
% & \les M^{-2}\sum_{(\nl)\in\mathcal{A}_{3}}L^{-3}(L')^{-2}\bra{N_{1}}^{6}\|\rho_{L}\|_{L_{\eta}^{1}}\|\rho_{L'}\|_{L_{\sigma}^{1}}\normo{\wh{f_{N_{1}}}}_{L_{\xi}^{\infty}}\normo{\wh{f_{N_{2}}}}_{L_{\xi}^{\infty}}\normo{\wh{f_{N_{3}}}}_{L_{\xi}^{\infty}}\\
% & \les\ve_{1}^{3}M^{-2}\sum_{(\nl)\in\mathcal{A}_{3}}L^{-1}\bra{N_{1}}^{6-k}\bra{N_{2}}^{-k}\bra{N_{3}}^{-k}\\
% & \les\ve_{1}^{3}M^{-2}M^{1-3\de_{0}}\bra{N}^{-k}\les\ve_{1}^{3}M^{-(1+\de_{0})}\bra{N}^{-k}.
%\end{align*}
%
%\noindent\textbf{Case 4.}
%
%This case can be treated similarly to the estimates for $\mathcal{A}_{2}$.
%In view of \eqref{eq:nonresonance-eta}, we use the integration by
%parts with respect to $\eta$ in a same way to \eqref{eq:k-decompose}.
%Then we get 
%\begin{align*}
%\mathcal{K}_{\nl}(s,\xi) & =\mathcal{K}_{1a}(s,\xi)+\mathcal{K}_{1b}(s,\xi)+\mathcal{K}_{2}(s,\xi),\\
%\mathcal{K}_{1a}(s,\xi) & :=\frac{i}{s}\iint_{\R^{2}\times\R^{2}}e^{isp\freq}\textbf{m}_{1}(\xi,\eta,\sigma)\nabla_{\eta}\wh{f_{N_{1}}}(s,\xi+\eta)\wh{f_{N_{2}}}(s,\xi+\sigma)\\
% & \hspace{6cm}\times\overline{\wh{f_{N_{3}}}(s,\xi+\eta+\sigma)}d\eta d\sigma,\\
%\mathcal{K}_{1b}(s,\xi) & :=\frac{i}{s}\iint_{\R^{2}\times\R^{2}}e^{isp\freq}\textbf{m}_{1}(\xi,\eta,\sigma)\wh{f_{N_{1}}}(s,\xi+\eta)\wh{f_{N_{2}}}(s,\xi+\sigma)\\
% & \hspace{6cm}\times\overline{\nabla_{\eta}\wh{f_{N_{3}}}(s,\xi+\eta+\sigma)}d\eta d\sigma,\\
%\mathcal{K}_{2}(s,\xi) & :=\frac{i}{s}\iint_{\R^{2}\times\R^{2}}e^{isp\freq}\textbf{m}_{2}(\xi,\eta,\sigma)\wh{f_{N_{1}}}(s,\xi+\eta)\wh{f_{N_{2}}}(s,\xi+\sigma)\\
% & \hspace{6cm}\times\overline{\wh{f_{N_{3}}}(s,\xi+\eta+\sigma)}d\eta d\sigma.
%\end{align*}
%Here the multipliers $\textbf{m}_{1}$ and $\textbf{m}_{2}$ are defined
%in \eqref{eq:multi-m}. Especially, by \eqref{eq:multipl-bound},
%\eqref{eq:multi-kab} and \eqref{eq:k_1 bound}, we have 
%\begin{align*}
% & \normo{\iint_{\R^{2}\times\R^{2}}\textbf{m}_{1}(\xi,\eta,\sigma)e^{iy\cdot\eta}e^{iz\cdot\sigma}\,d\sigma d\eta}_{L_{y,z}^{1}}\les L^{-1}(L')^{-1}M^{\frac{18}{n}},\\
% & \normo{\iint_{\R^{2}\times\R^{2}}\textbf{m}_{2}(\xi,\eta,\sigma)e^{iy\cdot\eta}e^{iz\cdot\sigma}\,d\sigma d\eta}_{L_{y,z}^{1}}\les L^{-2}(L')^{-1}M^{\frac{22}{n}}.
%\end{align*}
%If $N_{1}\nsim N_{3}$, by the support condition $\min(N_{1},N_{3})\ll\max(N_{1},N_{3})\sim L'$,
%one has the same multiplier bounds in either case $N_{1}\nsim N_{3}$
%or case $N_{1}\sim N_{3}$. Thus we only treat the case $N_{1}\sim N_{3}$.
%Using the multiplier estimates above, Lemma \ref{lem:coif}, and the
%fact that $L(L')^{2}\ge M^{-(1+\de_{0})}$, we estimate 
%\begin{align*}
%\sum_{(\nl)\in\mathcal{A}_{4}}\left|\mathcal{K}_{1a}(s,\xi)\right| & \les M^{-1+\frac{30}{n}}\sum_{(\nl)\in\mathcal{A}_{4}}L^{-1}(L')^{-1}\|xf\|_{L_{x}^{2}}\|P_{N_{2}}u\|_{L_{x}^{\infty}}\|\wh{f_{N_{3}}}\|_{L_{\xi}^{2}}\\
% & \les\ve_{1}^{3}M^{-2+\frac{30}{n}+\de_{0}}\sum_{(\nl)\in\mathcal{A}_{4}}L^{-1}(L')^{-1}\bra{N_{1}}^{-k}\bra{N_{2}}^{-k}N_{3}\bra{N_{3}}^{-k}\\
% & \les\ve_{1}^{3}M^{-(1+\delta_{0})}\bra{N}^{-k}
%\end{align*}
%and 
%\begin{align*}
%\sum_{(\nl)\in\mathcal{A}_{4}}\left|\mathcal{K}_{2}(s,\xi)\right| & \les M^{-1+\frac{30}{n}}\sum_{(\nl)\in\mathcal{A}_{4}}L^{-2}(L')^{-1}\|\wh{f_{N_{1}}}\|_{L_{\xi}^{2}}\|P_{N_{2}}u\|_{L_{x}^{\infty}}\|\wh{f_{N_{3}}}\|_{L_{\xi}^{2}}\\
% & \les\ve_{1}^{3}M^{-2+\frac{30}{n}}\sum_{(\nl)\in\mathcal{A}_{4}}L^{-2}(L')^{-1}N_{1}\bra{N_{1}}^{-k}\bra{N_{2}}^{-k}N_{3}\bra{N_{3}}^{-k}\\
% & \les\ve_{1}^{3}M^{-(1+\delta_{0})}\bra{N}^{-k}.
%\end{align*}
%Therefore, we finish the proof of \eqref{eq:part-scattering}.

%\begin{lemma}
%	Let $u$ satisfy a priori assumption \eqref{assumption-apriori}. Then we have
%	\begin{align}\label{eq:time-derivative}
%		\left\| P_N \partial_s \left(|u(s)|^2\right)\right\|_{L_x^\infty} &\les \min (N \bra{N}^{-2}\bra{s}^{-2}, N^3) \ve_1^2,\\
%			\left\| P_N \partial_s \left(|u(s)|^2\right)\right\|_{L_x^\infty} &\les \min (N \bra{N}^{-2}\bra{s}^{-2}, N^3) \ve_1^2.
%	\end{align}
%\end{lemma}
%\begin{proof} By direct calculation, we see that
%	\begin{align*}
%		&\partial_s \left(|u(s)|^2\right) \\
%		&=  \left(\partial_su(s) \right) \overline{u(s)} +   u(s) \overline{\partial_s u(s)}\\
%		&= i \left(\bra{D} u(s) \right) \overline{u(s)} -i u(s) \overline{\bra{D} u(s)} +  e^{is\bra{D}}\partial_s f(s) \overline{u(s)} +  u(s)e^{-is\bra{D}}\overline{\partial_s f(s)}.
%	\end{align*}
%\begin{align*}
%	\mathcal F \left[ i \left(\bra{D} u(s) \right) \overline{u(s)} -i u(s) \overline{\bra{D} u(s)} \right](\xi)= i \Big(\bra{\xi+\eta} - \bra{\eta}\Big)\wh{u}(\xi) \overline{\wh{u}(\eta)} 
%\end{align*}
%Then we get
%\begin{align*}
%	&\left\|P_N \left(\bra{D} u(s)  \overline{u(s)} - u(s) \overline{\bra{D} u(s)}  \right)\right \|_{L_x^\infty} \\
%	&\les \left\|\rho_N \mathcal F \left[  \left(\bra{D} u(s) \right) \overline{u(s)} -u(s) \overline{\bra{D} u(s)} \right](\xi) \right\|_{L_\xi^1}\\
%	&\les \Big\|\rho_N(\xi) |\xi|\Big\|_{L_\xi^1}  \left\| \wh{|u(s)|^2} \right\|_{L_\xi^\infty} \\
%	&\les N^4 \ve_1^2
%\end{align*}
%and
%\begin{align*}
%	&\left\|P_N \left(\bra{D} u(s) \right) \overline{u(s)} - u(s) \overline{\bra{D} u(s)}  \right\|_{L_x^\infty} \\
%	&\les \left\|\rho_N \mathcal F \left[ i \left(\bra{D} u(s) \right) \overline{u(s)} -i u(s) \overline{\bra{D} u(s)} \right](\xi) \right\|_{L_\xi^1}\\
%	&\les \Big\|\rho_N(\xi) |\xi|\Big\|_{L_\xi^1}  \left\| \wh{|u(s)|^2} \right\|_{L_\xi^\infty} \\
%	&\les N^4 \ve_1^2
%\end{align*}
%\end{proof}

%%%%%%%%%%%%%%%%%%%%%%%%%%%%%%%%%%%%%%%%%%%%%%%%%%%%%%%%%%%%%%%%%%%%%%%%%%%%%%%%%%%%%%%%%%%%%%%%%%%%%%%%%%%%%%%%%%%%%%%%%%%%%%%%%%%%%%%%%%%%%%%%%%%%%%%%%%%%%%%%%%%%%%%%%%%%%%%%%%%%


% \appendix
% %dummy comment inserted by tex2lyx to ensure that this paragraph is not empty


% \section{GWP and scattering results for the {\it short-range} case}
% In this appendix, we consider the Cauchy problem of the semi-relativistic Hartree equations with the {\em short-range} nonlinearity 
% \begin{align}
% 	\left\{ \begin{aligned} -i\partial_t u + \sqrt{m^2 -\Delta} u &= \lam \left(|x|^{-\gamma}*|u|^2\right) u  \qquad &&\mathrm{in}\;\;\R \times \mathbb{R}^{2},\\
% 		u(0,\cdot) &= u_0 &&\mathrm{in}\;\; \R^2,
% 	\end{aligned}\label{appen-maineq}
% 	\right.
% \end{align}
% where $1<\gamma <2$.  In \cite{pusa,hanaog2015-die}, they independently established the linear scattering results for semi-relativistic equations with {\em
% 	short-range} nonlinearity when dimension is greater than 3. In view of these results, as a byproduct
% of our main result, we give the global well-posedness and linear
% scattering of the {\em short-range} semi-relativistic Hartree equations in two dimension.
% \begin{thm}\label{appen-mainthm}
% 	Assume that an initial data condition
% 	\[
% 	\|u_0\|_{H^{50}} + \|\bra{x}^2u_0\|_{H^{10}} \les \ve_0.
% 	\]
% 	Then we get the global well-posedness for \eqref{appen-maineq} and global decay
% 	\[
% 	\|u(t)\|_{W^{10,\infty}} \les \bra{t}^{-1} \ve_0.
% 	\]
% 	Moreover, there exists a scattering profile $\textbf{f} \in H^{10}(\bra{x}^4dx)$ such that	
% 	\begin{align}\label{appen-scattering}
% 		\left\| \bra{x}^2\Big( e^{it\bra{D}}u - \textbf{f}\Big)  \right\|_{H^{10}} \xrightarrow{t \to \infty} 0.
% 	\end{align}
% \end{thm}


% We now brief the proof of Theorem \ref{appen-mainthm}. As we proved the GWP and asymptotic behavior for {\it long-range} case throughout this paper, we perform a similar approach, the bootstrap argument based on the space-time resonance method. To this end, let us set an a priori assumption
% \begin{align}\label{appen-apriori}
% 	\sup_{t\in[0,T]}\left( \|u(t)\|_{H^{50}} + \|\bra{x}^2 e^{it\bra{D}}u(t)\|_{H^{10}} \right) \les \ve_1,
% \end{align}
% for some $\ve_1 >0$. Note that there is no contribution of $\|\cdot\|_S$ compared to the a priori assumption \eqref{assumption-apriori}, which was considered to exploit the phase modification. Under the a priori assumption \eqref{appen-apriori}, we may have the time decay with appropriate regularity: 
% \begin{align}\label{appen-timedecay}
% 	\|u(t)\|_{W^{10,\infty}} \les  \bra{t}^{-1}\ve_1.
% \end{align}
% \begin{rem}
% 	The proof of \eqref{appen-timedecay} can be shown by following the procedure of the proof of Proposition \ref{timedecay-prop}. Since second weighted norm does not require the time growth unlike the critical case, one can overcome the absence of the scattering norm $\| \cdot \|_S$ in \eqref{assumption-apriori} by using Sobolev embedding $H_\xi^2 \to L_\xi^\infty$ in Fourier space as we frequently used in Section throughout this paper.
% \end{rem}

% Now we prove the global well-posedness of the Cauchy problem of \eqref{appen-maineq}. Then it suffices to show that
% \begin{align}
% 	\sup_{t\in[0,T]}\|u(t)\|_{H^{50}} &\le \ve_0 + C\ve_1^3,\label{appen-ener-high}\\
% 	\sup_{t\in[0,T]}\|x e^{it\bra{D}}u(t)\|_{H^{10}} &\le \ve_0 + C\ve_1^3, \label{appen-ener-wei1}\\
% 	\sup_{t\in[0,T]}\|x^2 e^{it\bra{D}}u(t)\|_{H^{10}} &\le \ve_0 + C\ve_1^3. \label{appen-ener-wei2}
% \end{align}
% The energy estimates \eqref{appen-ener-high} and \eqref{appen-ener-wei1} can be treated similarly or more simply to those in Section \ref{sec:Weighted-Energy-estimate}. More precisely, by Hardy-Littlewood-Sobolev type inequality, a priori assumption \eqref{appen-apriori}, and time decay effect \eqref{appen-timedecay}, we estimate
% \begin{align*}
% 	\int_0^t	\normo{(|x|^{-\gamma}*|u(s)|^2 )u(s)}_{H^{50}} ds &\les \int_0^t \|u(s)\|_{H^{50}}\|u(s)\|_{L_x^2}^{2-\gamma}\|u(s)\|_{L_x^\infty}^\gamma ds\\
% 	& \les \ve_1^{3} \int_0^t\bra{s}^{-\gamma} ds \les \ve_1^3.
% \end{align*}
% This finishes the proof of \eqref{appen-ener-high}. The proof of \eqref{appen-ener-wei1} is estimated easier than that of  \eqref{appen-ener-wei2}. Hence, we now move on to the estimates \eqref{appen-ener-wei2}. Similarly to the proof of \eqref{eq:second-moment}, we consider
% \begin{align}\label{appen-goal}
% 	\|\bra{\xi}^{10}\nabla_\xi^2 \wh{f}(\xi)\|_{L_\xi^2} \le \ve_0 + C\ve_1^3,
% \end{align}
% where the interaction function $f(t) = e^{it\bra{D}}u(t)$. In view of the estimates for $\mathcal J^{k}(t,\xi)(k=1,\cdots,4)$ in Section \ref{sec:Weighted-Energy-estimate}, the most delicate term occurs when the two derivative $\nabla_\xi^2$ falls on only the phase function $e^{is\phi(\xi,\eta)}$. Let us denote this case by
% \begin{align*}
% 	\mathcal J_{\gamma}(t,\xi) &= \int_{0}^{t} s^2 \int_{\R^2} [\nabla_\xi \phi(\xi,\eta)]^2 |\eta|^{-2+\gamma} e^{is\phi(\xi,\eta)} \wh{f}(\xi-\eta) \mathcal F(|u|^2)(\eta) \, d\eta ds \\
% 	&=: \int_{0}^{t} \mathcal K_{\gamma}(s,\xi)\,ds 
% \end{align*}
% Note that $\mathcal J_\gamma$ has the singularity $|\eta|^{-2+\gamma}$ instead of $|\eta|^{-1}$ as in \eqref{eq:j4-esti}. To estimate this case, we have to get an extra time decay by using the fact that the singularity is weaker than that of \eqref{eq:j4-esti}. To this end, we introduce the following lemma.
% \begin{lemma}\label{lem:esti-l2-decay}
% 	Assume that $\psi$ satisfies a priori assumption \eqref{appen-apriori}. Then we get
% 	\begin{align}
% 		%		\|P_N \left(|u(s)|^2\right)\|_{L_x^2} &\les N^{-1} \bra{s}^{-2}\ve_1^2, \label{appen:esti-l2-decay1}\\
% 		\|P_N \left(|u(s)|^2\right)\|_{L_x^2} &\les N^{-2-\de} \bra{s}^{-3+\de}\ve_1^2, \label{appen:esti-l2-decay2}
% 	\end{align}
% 	for $0 < \de \ll 1$.
% \end{lemma}
% The proof of Lemma \ref{lem:esti-l2-decay} is placed in the end of this appendix. Thus we assume the validity of this lemma. 

% In the rest of this paper, we focus on proving that for some $\zeta >0$,
% \begin{align}\label{appen-j4-esti}
% 	\|\bra{\xi}^{10}\mathcal K_\gamma(s,\xi) \|_{L_\xi^2}  \les  \bra{s}^{-1 -\zeta}\ve_1^3,
% \end{align}
% which implies \eqref{appen-goal}. For the purpose of proving \eqref{appen-j4-esti}, let us decompose $|\xi|,|\xi-\eta|,|\eta|$ into the dyadic pieces $N_0,N_1,N_2 \in 2^\Z$ as follows:
% \begin{align*}
% 	&\mathcal K_{\gam, \textbf{N}}(s,\xi) \\
% 	&\;\;= \sum_{\textbf{N}}\rho_{N_0}(\xi) s^2 \int_{\R^2} [\textbf{m}_{\textbf{N}}(\xi,\eta)]^2 |\eta|^{-2+\gamma} e^{is\phi(\xi,\eta)} \wh{f_{N_1}}(\xi-\eta) \mathcal F[P_{N_2}\left(|u|^2 \right)](\eta) \,d\eta,
% \end{align*}
% where $\textbf{N} = (N_0,N_1,N_2)$ and 
% \begin{align*}
% 	\textbf{m}_{\textbf{N}}(\xi,\eta) = [\nabla_\xi \phi(\xi,\eta)]^2 \rho_{N_0}(\xi)\rho_{N_1}(\xi-\eta)\rho_{N_2}(\eta).
% \end{align*}
% Then we also consider three cases \textbf{Case (i)} $N_0 \les N_1 \sim N_2$, \textbf{Case (ii)} $N_1 \ll N_0 \sim N_2$, \textbf{Case (iii)} $N_2 \ll N_0 \sim N_1 $. 

% Let us begin with \textbf{Case (i)}. To this, we further divide by using the frequency condition set
% \begin{align*}
% 	\mathcal A = \{ \textbf{N} : N_0 \le \bra{s}^{-4} \mbox{ or } N_1 \le \bra{s}^{-1}  \}.
% \end{align*} 
% By H\"older inequality, Hausdorff-Young's inequality, and  \eqref{appen:esti-l2-decay2}, we have
% \begin{align*}
% 	&\sum_{\textbf{Case (i)}} \normo{ \bra{\xi}^{10} \mathcal K_{\gamma,\textbf{N}}(s,\xi) }_{L_\xi^2} \\
% 	&\les 	 s^2 \sum_{\substack{\textbf{Case (i)}\\ \textbf{N} \in \mathcal A}}N_2^{\gamma} \|\rho_{N_0}\|_{L_\xi^2} \|\mathcal F \left(|u|^2\right)\|_{L_\eta^\infty} \|\rho_{N_2}\|_{L_\eta^1} \|u_{N_1}\|_{L^2} ds\\
% 	&\hspace{2cm} +  s^2 \sum_{\substack{\textbf{Case (i)}\\ \textbf{N} \in \mathcal A^c}}\bra{N_0}^8 N_2^{\gamma} \|P_{N_2}\left(|u|^2\right)\|_{L^2} \bra{N_1}^{-10}\|u\|_{W^{10,\infty}}  \\
% 	&\les 	\left( s^2 \sum_{\substack{\textbf{Case (i)}\\ \textbf{N}\in \mathcal A}}N_0 N_1^{\gamma+2} \bra{N_1}^{-50}  +  \bra{s}^{-2+\de} \sum_{\substack{\textbf{Case (i)}\\ \textbf{N}\in \mathcal A^c}}N_2^{\gamma-2-\de}  \bra{N_1}^{-10} \right) \ve_1^3\\
% 	&\les \bra{s}^{-\gamma+3\de}\ve_1^3.
% \end{align*}

% Similarly, we estimate \textbf{Case (ii)} as follows: by setting 
% \begin{align*}
% 	\mathcal B = \{ \textbf{N} : N_1 \le \bra{s}^{-4} \mbox{ or } N_0 \le \bra{s}^{-1}  \},
% \end{align*} 
% H\"older inequality and Lemmas \ref{lem:coif} and \ref{lem:esti-l2-decay} yield that
% \begin{align*}
% 	&\sum_{\textbf{Case (ii)}} \normo{  \bra{\xi}^{10} \mathcal K_{\gamma,\textbf{N}}(s,\xi) }_{L_\xi^2}  \\
% 	&\les 	 s^2 \sum_{\substack{\textbf{Case (ii)}\\ \textbf{N}\in \mathcal B}} \bra{N_0}^{8} N_2^{\gamma} \|\rho_{N_1}\|_{L_\xi^2} \|\rho_{N_2}\|_{L_\eta^1} \bra{N_2}^{-50} \left\|\mathcal F \left(|\bra{D}^{50}u|^2\right) \right\|_{L_\eta^\infty} \|u_{N_1}\|_{L^2} \\
% 	&\hspace{3cm} +  s^2 \sum_{\substack{\textbf{Case (ii)}\\ \textbf{N}\in \mathcal B^c}} \bra{N_0}^8 N_2^{\gamma} \|P_{N_2}\left(|u|^2\right)\|_{L^2} \bra{N_1}^{-10}\|u_{N_1}\|_{W^{10,\infty}} \\
% 	&\les 	\left( s^2\sum_{\substack{\textbf{Case (ii)}\\ \textbf{N}\in \mathcal B}}N_1 N_0^{\gamma+2} \bra{N_0}^{-42}  +  \bra{s}^{-2+\de} \sum_{\substack{\textbf{Case (ii)}\\ \textbf{N}\in \mathcal B^c}}\bra{N_1}^{-10}N_2^{\gamma-2-\de}  \bra{N_2}^{-2}  \right)\ve_1^3\\
% 	&\les\bra{s}^{-\gamma +3\de}\ve_1^3.
% \end{align*}


% For the \textbf{Case (iii)}, we utilize H\"older inequality and Lemma \ref{lem:esti-l2-decay} as follows:
% \begin{align*}
% 	&\sum_{\textbf{Case (iii)}} \normo{ \bra{\xi}^{10}  \mathcal K_{\gamma,\textbf{N}}(s,\xi) }_{L_\xi^2}\\
% 	&\les 	 s^2 \sum_{\substack{\textbf{Case (iii)}\\ N_2 \le \bra{s}^{-1}}}\bra{N_0}^{8}N_2^{\gamma} \|\rho_{N_2}\|_{L_\eta^1} \|u\|_{H^{50}}^2 \|u_{N_1}\|_{L^2} \\
% 	&\qquad +  s^2 \sum_{\substack{\textbf{Case (iii)}\\ N_2 \ge \bra{s}^{-1}}}\bra{N_0}^{8}N_2^{\gamma} \|P_{N_2}\left(|u|^2\right)\|_{L^2} \bra{N_1}^{-10}\|u\|_{W^{10,\infty}}  \\
% 	&\les 	\left( s^2 \sum_{\substack{\textbf{Case (iii)}\\ N_2 \le \bra{s}^{-1}}}N_2^{\gamma+2} \bra{N_1}^{8-n}  +  \bra{s}^{-2+\de} \sum_{\substack{\textbf{Case (iii)}\\ N_2 \ge \bra{s}^{-1}}}N_2^{\gamma-2-\de}  \bra{N_1}^{-2} \right) \ve_1^3\\
% 	&\les \bra{s}^{-\gamma+2\de}\ve_1^3.
% \end{align*}
% Therefore, gathering the estimates for the \textbf{Case (i)--(iii)} and setting $0 <\de < \frac{\gamma -1}3$, we finish the proof of \eqref{appen-j4-esti}


% Let us move on to the proof of scattering results. By Duhamel's formula, we see that for $0 < t_1 \le t_2 \in [0,T]$,
% \[
% f(t_2,\cdot) - f(t_1,\cdot) = \int_{t_1}^{t_2} e^{is\bra{D}} \left[(|x|^{-\gamma}*|u(s)|^2)u(s)  \right] ds.
% \]
% Then the estimates \eqref{appen-j4-esti} yields that for  $0 < t_1 \le t_2 \in [0,T]$,
% \[
% \left\|\bra{x}^2\Big( f(t_2,x) - f(t_1,x) \Big) \right\|_{L_x^2} \les \bra{t_1}^{-1-\zeta},
% \]
% where $\zeta = \gamma-1-3\de$ for $0< \de < \frac{ \gamma -1}3$. Hence, this completes the proof of the linear scattering \eqref{appen-scattering} by setting  
% {\it \[
% 	\textbf{f}\, := \lim_{t\to\infty} e^{it\bra{D}}f(t,\cdot).
% 	\]}

% \subsection{Proof of Lemma \ref{lem:esti-l2-decay}}
% Taking Fourier transform into the left-hand side of \eqref{appen:esti-l2-decay2}, we see that
% \begin{align}\label{eq:inte1}
% 	\mathcal F \left[P_N \left(|u(t)|^2\right)  \right] (\xi)= \rho_N(\xi) \int_{\R^2} e^{is\varphi(\xi,\eta)} \overline{\wh{f}(\eta)} \wh{f}(\xi+\eta) \,d\eta ,
% \end{align}
% where
% \[
% \varphi(\xi,\eta) =  \bra{\eta} -  \bra{\xi + \eta}.
% \]
% To obtain an extra time decay in \eqref{eq:inte1}, we perform the space resonance approach. For the purpose, we note that
% \[
% \left|\nabla_\eta\vp(\xi,\eta) \right| = \left| \frac{\eta}{\bra{\eta}} -  \frac{\xi+\eta}{\bra{\xi+\eta}} \right| \gtrsim \begin{cases}
% 	\frac{N}{\bra{N_1}^3}             & \mbox{ for } N_1 \sim N_2,\\
% 	\frac{N_{1}}{\bra{N_{1}}} 	 & \mbox{ for } N_1 \gg N_2.
% \end{cases}
% \]
% where $|\eta| \sim N_1$ and $|\xi+\eta| \sim N_2$ for the dyadic numbers $N_1,N_2 \in 2^\Z$. Here we do not consider $N_1 \ll N_2$ due to the symmetry between $\eta$ and $\xi+\eta$. Let us decompose the frequencies $\eta,\xi+\eta$ into $N_1,N_2$. Using a relation 
% \[
% e^{is \vp} = -i \frac1s \frac{\nabla_\sigma \varphi \cdot \nabla_\sigma e^{is\vp}}{|\nabla_\sigma \varphi|^2},
% \]
% we apply integration by parts in $\eta$ twice to \eqref{eq:inte1}, obtaining the following terms:
% \begin{subequations}
% 	\begin{align}
% 		&  s^{-2}\rho_N(\xi) \int_{\R^2} e^{is\varphi(\xi,\eta)} \nabla_\eta m_{1,\textbf{N}}(\xi,\eta)\overline{\wh{f_{N_1}}(\eta)} \wh{f_{N_2}}(\xi+\eta) \,d\eta, \label{eq:biliear1}\\
% 		& s^{-2}\rho_N(\xi) \int_{\R^2} e^{is\varphi(\xi,\eta)} m_{1,\textbf{N}}(\xi,\eta)\overline{\rho_{N_1}\nabla_\eta\wh{f}(\eta)} \wh{f_{N_2}}(\xi+\eta) \,d\eta,\label{eq:biliear2}\\
% 		& s^{-2}\rho_N(\xi) \int_{\R^2} e^{is\varphi(\xi,\eta)} m_{1,\textbf{N}}(\xi,\eta)\overline{\wh{f_{N_1}}(\eta)} \rho_{N_2}\nabla_\eta \wh{f}(\xi+\eta) \,d\eta,\label{eq:biliear3}\\
% 		& s^{-2}\rho_N(\xi) \int_{\R^2} e^{is\varphi(\xi,\eta)} \nabla_\eta m_{2,\textbf{N}}(\xi,\eta)\overline{\rho_{N_1}\nabla_\eta\wh{f}(\eta)} \wh{f_{N_2}}(\xi+\eta) \,d\eta,\label{eq:biliear4}\\
% 		& s^{-2}\rho_N(\xi) \int_{\R^2} e^{is\varphi(\xi,\eta)}  m_{2,\textbf{N}}(\xi,\eta)\overline{\rho_{N_1}\nabla_\eta^2\wh{f}(\eta)} \wh{f_{N_2}}(\xi+\eta) \,d\eta,\label{eq:biliear5}\\
% 		& s^{-2}\rho_N(\xi) \int_{\R^2} e^{is\varphi(\xi,\eta)}  m_{2,\textbf{N}}(\xi,\eta)\overline{\rho_{N_1}\nabla_\eta\wh{f}(\eta)} \rho_{N_2}\nabla_\eta \wh{f}(\xi+\eta) \,d\eta,\label{eq:biliear6}\\
% 		&+ \mbox{``similar terms"} \nonumber
% 	\end{align}
% \end{subequations}
% where
% \begin{align*}
% 	m_{1,\textbf{N}}(\xi,\eta) &= \frac{\nabla_\eta \varphi(\xi,\eta)}{|\nabla_\eta \varphi(\xi,\eta)|^2} \nabla_\eta \left(\frac{\nabla_\eta \varphi(\xi,\eta)}{|\nabla_\eta \varphi(\xi,\eta)|^2} \right) \rho_N(\xi)\rho_{N_1}(\eta) \rho_{N_2}(\xi+\eta),\\
% 	m_{2,\textbf{N}}(\xi,\eta) &=  |\nabla_\eta \varphi(\xi,\eta)|^{-2}  \rho_N(\xi)\rho_{N_1}(\eta) \rho_{N_2}(\xi+\eta).
% \end{align*}
% Then we get the multiplier bounds 
% \begin{align}\label{appen:multi-1}
% 	\|\nabla_\eta^\ell m_{1,\textbf{N}}\|_{\rm CM} \les \begin{cases}
% 		N^{-1}N_1^{-\ell} \bra{N_1}^{14+2\ell}            & \mbox{ for } N_1 \sim N_2,\\
% 		N_1^{-1-\ell} \bra{N_1} 	 & \mbox{ for } N_1 \gg N_2.
% 	\end{cases},
% \end{align}
% and 
% \begin{align}\label{appen:multi-2}
% 	\|\nabla_\eta^{\ell}m_{2,\textbf{N}}\|_{\rm CM} \les \begin{cases}
% 		N^{-2}N_1^{-\ell} \bra{N_1}^{14+2\ell}            & \mbox{ for } N_1 \sim N_2,\\
% 		N_1^{-2-\ell} \bra{N_1}^2 	 & \mbox{ for } N_1 \gg N_2.
% 	\end{cases}.
% \end{align}
% for $\ell =0,1$. To bound \eqref{eq:biliear1}, we use Lemma \ref{lem:coif} with \eqref{appen:multi-1}. Indeed,  
% \begin{align*}
% 	\|\eqref{eq:biliear1}\|_{L^2} &\les \sum_{N_1 \sim N_2}s^{-2} N^{-1} N_1^{-1} \bra{N_1}^{16-60}  \|u_{N_1}\|_{H^{50}} \|u_{N_2}\|_{W^{10,\infty}}\\
% 	&\hspace{2cm}+\sum_{N_1 \gg N_2}s^{-2} N^{-2} \bra{N}^{-9} \|u_{N_1}\|_{W^{10,\infty}} \|u_{N_2}\|_{L^2}\\
% 	& \les  s^{-3} N^{-2} \ve_1^2.
% \end{align*}
% Using the similar estimates and Bernstein inequality, we estimate
% \begin{align*}
% 	\|\eqref{eq:biliear2}\|_{L^2} &\les \sum_{N_1 \sim N_2} s^{-2} N^{-1}  \bra{N_1}^{-6}  \|xf\|_{H^{10}} \|u_{N_2}\|_{W^{10,\infty}}\\
% 	&\hspace{1cm} + \sum_{N_1 \gg N_2} s^{-2} N^{-1}  \bra{N}^{-9} N_2^{\de} \|xf\|_{H^{10}} \|u_{N_2}\|_{L^{\frac2\de}}\\
% 	& \les  s^{-3+\de} N^{-1} \ve_1^2.
% \end{align*}
% The estimate for \eqref{eq:biliear3} is obtained similarly to that for \eqref{eq:biliear2}. We also estimate
% \begin{align*}
% 	\|\eqref{eq:biliear4}\|_{L^2} &\les \sum_{N_1 \sim N_2} s^{-2} N^{-2}N_1^{-1} \bra{N_1}^{6} \|\rho_{N_1}\|_{L_\eta^{\frac2{1-\de}}} \|\rho_{N_1} \wh{x f}\|_{L_\eta^\frac2\de} \|u_{N_2}\|_{W^{10,\infty}} \\
% 	& \hspace{1cm}+ \sum_{N_1 \gg N_2} s^{-2} N^{-3} \bra{N}^{2} N_2^{\de} \|\rho_{N_1}\|_{L_\eta^{\frac2{1-\de}}} \|\rho_{N_1} \wh{x f}\|_{L_\eta^\frac2\de} \|u_{N_2}\|_{L^{\frac2\de}} \\
% 	&\les s^{-3+\de} N^{-2-\de}  \ve_1^2,\\
% 	\|\eqref{eq:biliear5}\|_{L^2} &\les  \sum_{N_1 \sim N_2} s^{-2} N^{-2} \bra{N_1}^{-4} \|P_{N_1}(x^2 f)\|_{H^{10}} \|u_{N_2}\|_{W^{10,\infty}}\\
% 	&\hspace{1cm} +  \sum_{N_1 \gg N_2} s^{-2} N^{-2} \bra{N}^{-8}N_2^{\de} \|P_{N_1}(x^2 f)\|_{H^{10}} \|u_{N_2}\|_{L^{\frac2\de}}\\
% 	&\les s^{-3+\de} N^{-2}  \ve_1^2.
% \end{align*}
% To handle \eqref{eq:biliear6}, we further perform the integration by parts in $\eta$, to obtaining
% \begin{subequations}
% 	\begin{align}
% 		& s^{-3}\rho_N(\xi) \int_{\R^2} e^{is\varphi(\xi,\eta)}  \nabla_\eta m_{3,\textbf{N}}(\xi,\eta)\overline{\rho_{N_1}\nabla_\eta\wh{f}(\eta)} \rho_{N_2}\nabla_\eta \wh{f}(\xi+\eta) \,d\eta,\label{eq:biliear6-1}\\
% 		& s^{-3}\rho_N(\xi) \int_{\R^2} e^{is\varphi(\xi,\eta)}  m_{3,\textbf{N}}(\xi,\eta)\overline{\rho_{N_1}\nabla_\eta^2 \wh{f}(\eta)} \rho_{N_2}\nabla_\eta \wh{f}(\xi+\eta) \,d\eta,\label{eq:biliear6-2}\\
% 		& s^{-3}\rho_N(\xi) \int_{\R^2} e^{is\varphi(\xi,\eta)}  m_{3,\textbf{N}}(\xi,\eta)\overline{\rho_{N_1}\nabla_\eta\wh{f}(\eta)} \rho_{N_2}\nabla_\eta^2 \wh{f}(\xi+\eta) \,d\eta,\label{eq:biliear6-3}
% 	\end{align}
% \end{subequations}
% where 
% \begin{align*}
% 	m_{3,\textbf{N}}(\xi,\eta) &=  \frac{\nabla_\eta \vp(\xi,\eta)}{|\nabla_\eta \varphi(\xi,\eta)|^4}  \rho_N(\xi)\rho_{N_1}(\eta) \rho_{N_2}(\xi+\eta).
% \end{align*}
% Then we have the pointwise bounds
% \begin{align*}
% 	\|\nabla_\eta^{\ell}m_{3,\textbf{N}}\|_{L_{\xi,\eta}^\infty} \les \begin{cases}
% 		N^{-3}N_1^{-\ell} \bra{N_1}^{9+2\ell}            & \mbox{ for } N_1 \sim N_2,\\
% 		N_1^{-3-\ell} \bra{N_1}^3 	 & \mbox{ for } N_1 \gg N_2,
% 	\end{cases}
% \end{align*}
% for $\ell =0,1$. These bounds with H\"older inequality imply that
% \begin{align*}
% 	\|\eqref{eq:biliear6-1}\|_{L^2} &\les \sum_{N_1 \sim N_2} s^{-3}  N^{-3}N_1^{-1}\bra{N_1}^{11} \|\rho_N\|_{L_\xi^2}\|\rho_{N_1}\|_{L_\eta^2}\|\rho_{N_1} \wh{xf}\|_{L_\eta^4}\|\rho_{N_2}\wh{xf}\|_{L_\eta^4}\\
% 	&\hspace{1cm}  +\sum_{N_1 \gg N_2} s^{-3}  N^{-4}\bra{N_1}^{3} \|\rho_N\|_{L_\xi^2} \|\rho_{N_2}\|_{L_\eta^2}\|\rho_{N_1} \wh{xf}\|_{L_\eta^4}\|\rho_{N_2}\wh{xf}\|_{L_\eta^4}\\
% 	&\les s^{-3} N^{-2-\ve} \ve_1^2,
% \end{align*}
% and
% \begin{align*}
% 	\|\eqref{eq:biliear6-2}\|_{L^2} &\les  \sum_{N_1 \sim N_2} s^{-3}  N^{-3}\bra{N_1}^{-11} \|\rho_N\|_{L_\xi^2} \| x^2 f\|_{H^{10}}\|xf\|_{H^{10}}\\
% 	&\hspace{1cm} + \sum_{N_1 \gg N_2} s^{-3}  N^{-3}\bra{N_1}^{-7} \|\rho_N\|_{L_\xi^2} \| x^2 f\|_{H^{10}}\|\rho_{N_2}\|_{L_\xi^4} \|\wh{xf}\|_{L_\xi^4}\\
% 	&\les s^{-3} N^{-2} \ve_1^2.
% \end{align*}
% Lastly, \eqref{eq:biliear6-3} can be estimated similarly to the estimates for \eqref{eq:biliear6-2}.



% %
% %
% %\section{GWP and scattering results for the {\em short-range} case}
% %
% %
% %\subsection{}
% %In \cite{pusa,hanaog2015-die}, they independently have considered
% %the linear scattering for semi-relativistic equations with {\em
% %short-range} nonlinearity. In view of these results, as a byproduct
% %of our main result, we give the global well-posedness and the linear
% %scattering of the semi-relativistic Hartree equations in two dimension.
% %\begin{rem} Related to our main results, one can consider the short-range
% %case semi-relativistic Hartree equations: 
% %\begin{align*}
% %\left\{ \begin{aligned}-i\partial_{t}u+\sqrt{m^{2}-\Delta}u & =\lam\left(|x|^{-\gamma}*|u|^{2}\right)u\qquad\mathrm{in}\;\;\R\times\mathbb{R}^{2},\\
% %u(0) & =u_{0},
% %\end{aligned}
% %\right.
% %\end{align*}
% %where $1<\gamma<2$. Under an initial data condition 
% %\[
% %\|u_{0}\|_{H^{10}}+\|\bra{x}^{2}u_{0}\|_{H^{5}}\les\ve_{0},
% %\]
% %Then we can obtain the weighted energy estimates without much difficulty,
% %as the convolved potential is less singular (with Fourier multiplier
% %$|\eta|^{-2+\gamma}$). In fact, we have better estimates than those
% %for \eqref{eq:first-moment} and \eqref{eq:second-moment}. As a result,
% %we have the global well-posedness and linear scattering results in
% %short-range case $1<\gamma<2.$ We sketch the proof in an appendix.
% %\end{rem}
% %
% %
% %In this appendix, we consider Cauchy problem of the {\em short-range}
% %case semi-relativistic Hartree equations: 
% %\begin{align}
% %\left\{ \begin{aligned}-i\partial_{t}u+\sqrt{m^{2}-\Delta}u & =\lam\left(|x|^{-\gamma}*|u|^{2}\right)u\qquad\mathrm{in}\;\;\R\times\mathbb{R}^{2},\\
% %u(0) & =u_{0},
% %\end{aligned}
% %\label{appen-maineq}\right.
% %\end{align}
% %where $1<\gamma<2$. \begin{thm}\label{appen-mainthm} Assume that
% %an initial data condition 
% %\[
% %\|u_{0}\|_{H^{10}}+\|\bra{x}^{2}u_{0}\|_{H^{5}}\les\ve_{0}.
% %\]
% %Then we get the global well-posedness for \eqref{appen-maineq} and
% %global decay 
% %\[
% %\|u(t)\|_{W^{2,\infty}}\les\bra{t}^{-1}\ve_{0}.
% %\]
% %Moreover, there exists a scattering profile $\textbf{f}\in L_{x}^{2}(\bra{x}^{2})$
% %such that 
% %\begin{align}
% %\left\Vert \bra{x}^{2}\Big(e^{it\bra{D}}u-\textbf{f}\Big)\right\Vert _{L_{x}^{2}}\xrightarrow{t\to\infty}0.\label{appen-scattering}
% %\end{align}
% %\end{thm} We now brief the proof of Theorem \ref{appen-mainthm}.
% %Let us set a priori assumption 
% %\begin{align}
% %\sup_{t\in[0,T]}\left(\|u(t)\|_{H^{10}}+\|\bra{x}^{2}e^{it\bra{D}}u(t)\|_{H^{5}}\right)\les\ve_{1}\label{appen-apriori}
% %\end{align}
% %for some $\ve_{1}>0$. Note that there is no contribution of $\|\cdot\|_{S}$
% %compared to the a priori assumption \eqref{assumption-apriori}, which
% %was considered throughout this paper. Under the assumption \eqref{appen-apriori},
% %we may have the time decay 
% %\begin{align}
% %\|u(t)\|_{W^{2,\infty}}\les\ve_{1}\bra{t}^{-1}.\label{appen-timedecay}
% %\end{align}
% %Then it suffices to show that 
% %\begin{align}
% %\sup_{t\in[0,T]}\|u(t)\|_{H^{10}} & \le\ve_{0}+C\ve_{1}^{3},\label{appen-ener-high}\\
% %\sup_{t\in[0,T]}\|xe^{it\bra{D}}u(t)\|_{H^{5}} & \le\ve_{0}+C\ve_{1}^{3},\label{appen-ener-wei1}\\
% %\sup_{t\in[0,T]}\|x^{2}e^{it\bra{D}}u(t)\|_{H^{5}} & \le\ve_{0}+C\ve_{1}^{3}.\label{appen-ener-wei2}
% %\end{align}
% %
% %By Hardy-Littlewood-Sobolev type inequality, a priori assumption \eqref{appen-apriori},
% %and \eqref{appen-timedecay}, we estimate 
% %\begin{align*}
% %\normo{(|x|^{-\gamma}*|u(s)|^{2})u(s)}_{L_{x}^{2}}\les\|u(s)\|_{L_{x}^{\frac{4}{2-\gamma}}}^{2}\|u(s)\|_{L_{x}^{2}}\les\bra{s}^{-\gamma},
% %\end{align*}
% %which induces that 
% %\begin{align*}
% %\int_{0}^{t}\normo{(|x|^{-\gamma}*|u(s)|^{2})u(s)}_{H^{10}}ds & \les\int_{0}^{t}\|u(s)\|_{H^{10}}\|u(s)\|_{L_{x}^{2}}^{2-\gamma}\|u(s)\|_{L_{x}^{\infty}}^{\gamma}ds\\
% % & \les\ve_{1}^{3}\int_{0}^{t}\bra{s}^{-\gamma}ds\les\ve_{1}^{3}.
% %\end{align*}
% %This finishes the proof of \eqref{appen-ener-high}.
% %
% %The proof of \eqref{appen-ener-wei1} can be estimated easier than
% %\eqref{appen-ener-wei2}. Hence, we now move on to the estimates \eqref{appen-ener-wei2}.
% %Similarly to the proof of \eqref{eq:second-moment}, we have to handle
% %\begin{align}
% %\|\bra{\xi}^{5}\nabla_{\xi}^{2}\wh{f}(\xi)\|_{L_{\xi}^{2}}\le\ve_{0}+C\ve_{1}^{3}\label{appen-goal}
% %\end{align}
% %The most delicate term occurs when the derivative $\nabla_{\xi}^{2}$
% %falls on the phase function $e^{isp(\xi,\eta,\sigma)}$. Let $\mathcal{J}_{\gamma}$
% %denote this case as follows: 
% %\begin{align*}
% %\mathcal{J}_{\gamma}(t,\xi)=\int_{0}^{t}s^{2}\iint_{\R^{2}\times\R^{2}}[\textbf{m}(\xi,\eta)]^{2}|\eta|^{-2+\gamma}e^{isp\freq}\wh{f}(\xi-\eta)\wh{f}(\eta+\sigma)\overline{\wh{f}(\sigma)}\,d\sigma d\eta ds.
% %\end{align*}
% %Since $\mathcal{J}_{\gamma}$ has the singularity $|\eta|^{-2+\gamma}$
% %instead of $|\eta|^{-1}$ as in \eqref{eq:j4-esti}. Then, by using
% %the normal form approach with respect to $\sigma$-direction three
% %times, we see that 
% %\begin{align}
% %\|\bra{\xi}^{5}\mathcal{J}_{\gamma}(t,\xi)\|_{L_{\xi}^{2}}\les\int_{0}^{t}\bra{s}^{-1-(\gamma-1)}\ve_{1}^{3}ds\les\ve_{1}^{3},\label{appen-j4-esti}
% %\end{align}
% %which implies \eqref{appen-goal}.
% %
% %By Duhamel's formula, we see that for $0<t_{1}\le t_{2}\in[0,T]$,
% %\[
% %f(t_{2},\cdot)-f(t_{1},\cdot)=\int_{t_{1}}^{t_{2}}e^{is\bra{D}}(|x|^{-\gamma}*|u(s)|^{2})u(s)ds.
% %\]
% %Then the estimates \eqref{appen-j4-esti} yields that for $0<t_{1}\le t_{2}\in[0,T]$,
% %\[
% %\left\Vert \bra{x}^{2}\Big(f(t_{2},x)-f(t_{1},x)\Big)\right\Vert _{L_{x}^{2}}\les\bra{t_{1}}^{-(\gamma-1)}.
% %\]
% %Therefore, this completes the proof of the linear scattering \eqref{appen-scattering}
% %by setting \textit{
% %\[
% %\textbf{f}\,(x):=\lim_{t\to\infty}e^{it\bra{D}}f(t,x).
% %\]
% %} %%%%%%%%%%%%%%%%%%%%%%%%%%%%%%%%%%%%%%%%%%%%%%%%%%%%%%%%%%%%%%%%%%%%%%%%%%%%%%%%%%%%%%%%%%%%%%%%%%%%%%%%%%%%%%%%%%%%%%%%%%%%%%%%
% \medskip{}


\subsection*{Acknowledgement}

The first and second authors were supported by the National Research
Foundation of Korea(NRF) grant funded by the Korea government(MSIT)
(No. NRF-2019R1A5A1028324). The first author was supported in part
by NRF-2022R1A2C1091499. The second author was supported in part by
NRF-2022R1I1A1A0105640812. The third author was supported in part by
NRF-2021R1C1C1005700. 

%%%%%%%%%%%%%%%%%%%%%%%%%%%%%%%%%%%%%%%%%%%%%%%%%%%%%%%%%%%%%%%%%%%%%%%%%%%%%%%%%%%%%%%%%%%%%%%%%%%%%%%%%%%%%%%%%%%%%%%%

 \bibliographystyle{plain}
 \begin{thebibliography}{10}

    \bibitem{arbuspar2018-jmp}
    Jack Arbunich and Christof Sparber.
    \newblock Rigorous derivation of nonlinear {D}irac equations for wave
      propagation in honeycomb structures.
    \newblock {\em Journal of Mathematical Physics}, 59(1):011509, 2018.
    
    \bibitem{bourcanma2014-dcds}
    Nikolaos Bournaveas, Timothy Candy, and Shuji Machihara.
    \newblock A note on the {C}hern-{S}imons-{D}irac equations in the {C}oulomb
      gauge.
    \newblock {\em Discrete and Continuous Dynamical Systems}, 34(7):2693--2701,
      2014.
    
    \bibitem{CKLY2022}
    Yonggeun Cho, Soonsik Kwon, Kiyeon Lee, and Changhun Yang.
    \newblock The modified scattering for {D}irac equations of scattering-critical
      nonlinearity.
    \newblock {\em preprint, arXiv:2208.12040}.
    
    \bibitem{choz2006-siam}
    Yonggeun Cho and Tohru Ozawa.
    \newblock On the semirelativistic {H}artree--type equation.
    \newblock {\em SIAM Journal on Mathematical Analysis}, 38(4):1060--1074, 2006.
    
    \bibitem{choz2007-jkms}
    Yonggeun Cho and Tohru Ozawa.
    \newblock Global solutions of semirelativistic {H}artree type equations.
    \newblock {\em Journal of the Korean Mathematical Society}, 44(5):1065--1078,
      2007.
    
    \bibitem{choz2008-dcds-s}
    Yonggeun Cho and Tohru Ozawa.
    \newblock On radial solutions of semi-relativistic {H}artree equations.
    \newblock {\em Discrete and Continuous Dynamical Systems - S}, 1(1):71--82,
      2008.
    
    \bibitem{chozhishim2009-dcds}
    Yonggeun Cho, Tohru Ozawa, Hironobu Sasaki, and Yongsun Shim.
    \newblock Remarks on the semirelativistic {H}artree equations.
    \newblock {\em Discrete and Continuous Dynamical Systems}, 23(4):1277--1294,
      2009.
    
    \bibitem{Cloos2020}
    Cai~Constantin Cloos.
    \newblock On the long-time behavior of the three-dimensional dirac-maxwell
      equation with zero magnetic field.
    \newblock 2020.
    
    \bibitem{anfosel}
    Piero D'Ancona, Damiano Foschi, and Sigmund Selberg.
    \newblock Null structure and almost optimal local regularity for the
      {D}irac-{K}lein-{G}ordon system.
    \newblock {\em J. Eur. Math. Soc. (JEMS)}, 9(4):877--899, 2007.
    
    \bibitem{hajjmehats2014}
    Raymond El~Hajj and Florian M\'{e}hats.
    \newblock Analysis of models for quantum transport of electrons in graphene
      layers.
    \newblock {\em Mathematical Models and Methods in Applied Sciences},
      24(11):2287--2310, 2014.
    
    \bibitem{frjonlenz2007-nonlinearity}
    J\"urg Fr\"ohlich, B.~Lars~G. Jonsson, and Enno Lenzmann.
    \newblock Effective dynamics for boson stars.
    \newblock {\em Nonlinearity}, 20(5):1031--1075, mar 2007.
    
    \bibitem{gemasha2008}
    Pierre Germain, Nader Masmoudi, and Jalal Shatah.
    \newblock {Global Solutions for 3D Quadratic {S}chr\"odinger Equations}.
    \newblock {\em International Mathematics Research Notices}, 2009(3):414--432,
      12 2008.
    
    \bibitem{gemasha2012-jmpa}
    Pierre Germain, Nader Masmoudi, and Jalal Shatah.
    \newblock Global solutions for 2d quadratic {S}chr\"odinger equations.
    \newblock {\em Journal de Mathematiques Pures et Appliquees}, 97(5):505--543,
      2012.
    
    \bibitem{gemasha2012-annals}
    Pierre Germain, Nader Masmoudi, and Jalal Shatah.
    \newblock Global solutions for the gravity water waves equation in dimension 3.
    \newblock {\em Annals of Mathematics}, 175:691--754, 2012.
    
    \bibitem{hayashi-naumkin1998}
    Nakao Hayashi and Pavel~I. Naumkin.
    \newblock Asymptotics for large time of solutions to the nonlinear
      {S}chr\"{o}dinger and {H}artree equations.
    \newblock {\em Amer. J. Math.}, 120(2):369--389, 1998.
    
    \bibitem{hayashi-naumkin2017-henri}
    Nakao Hayashi and Pavel~I. Naumkin.
    \newblock Large time asymptotics for the fractional order cubic nonlinear
      {S}chr\"odinger equations.
    \newblock {\em Annales Henri Poincar\'e}, 18:1025--1054, 2017.
    
    \bibitem{hanaog2015-die}
    Nakao Hayashi, Pavel~I. Naumkin, and Takayoshi Ogawa.
    \newblock {Scattering operator for semirelativistic Hartree type equation with
      a short range potential}.
    \newblock {\em Differential and Integral Equations}, 28(11/12):1085--1104,
      2015.
    
    \bibitem{hele2014}
    Sebastian Herr and Enno Lenzmann.
    \newblock The {B}oson star equation with initial data of low regularity.
    \newblock {\em Nonlinear Analysis: Theory, Methods {\rm \&} Applications},
      97:125--137, 2014.
    
    \bibitem{Huh2016}
    Hyungjin Huh and Sung-Jin Oh.
    \newblock Low regularity solutions to the {C}hern-{S}imons-{D}irac and the
      {C}hern-{S}imons-{H}iggs equations in the {L}orenz gauge.
    \newblock {\em Comm. Partial Differential Equations}, 41(3):375--397, 2016.
    
    \bibitem{iopu2014}
    Alexandru~D. Ionescu and Fabio Pusateri.
    \newblock Nonlinear fractional {S}chr\"odinger equations in one dimension.
    \newblock {\em Journal of Functional Analysis}, 266(1):139--176, 2014.
    
    \bibitem{kapu}
    Jun Kato and {Fabio} Pusateri.
    \newblock A new proof of long-range scattering for critical nonlinear
      {S}chr{\"o}dinger equations.
    \newblock {\em Differential and Integral Equations}, 24(9-10):923--940, 2011.
    
    \bibitem{lee2021-bkms}
    Kiyeon Lee.
    \newblock Local well-posedness of {D}irac equations with nonlinearity derived
      from honeycomb structure in 2 dimensions.
    \newblock {\em Bulletin of the Korean Mathematical Society}, 58(6):1445--1461,
      2021.
    
    \bibitem{lenz2007}
    Enno Lenzmann.
    \newblock Well-posedness for semi-relativistic {H}artree equations of critical
      type.
    \newblock {\em Mathematical Physics, Analysis and Geometry}, 10:43--64, 2007.
    
    \bibitem{Lieb1987}
    Elliott~H. Lieb and Horng-Tzer Yau.
    \newblock The {C}handrasekhar theory of stellar collapse as the limit of
      quantum mechanics.
    \newblock {\em Comm. Math. Phys.}, 112(1):147--174, 1987.
    
    \bibitem{MNO2003}
    Shuji Machihara, Kenji Nakanishi, and Tohru Ozawa.
    \newblock {Small global solutions and the nonrelativistic limit for the
      nonlinear Dirac equation}.
    \newblock {\em Revista Matem\'atica Iberoamericana}, 19(1):179 -- 194, 2003.
    
    \bibitem{Michelangeli2012}
    Alessandro Michelangeli and Benjamin Schlein.
    \newblock Dynamical collapse of boson stars.
    \newblock {\em Comm. Math. Phys.}, 311(3):645--687, 2012.
    
    \bibitem{Okamoto2013}
    Mamoru Okamoto.
    \newblock Well-posedness of the {C}auchy problem for the
      {C}hern-{S}imons-{D}irac system in two dimensions.
    \newblock {\em J. Hyperbolic Differ. Equ.}, 10(4):735--771, 2013.
    
    \bibitem{ozawa1991}
    Tohru Ozawa.
    \newblock Long range scattering for nonlinear {S}chr\"{o}dinger equations in
      one space dimension.
    \newblock {\em Comm. Math. Phys.}, 139(3):479--493, 1991.
    
    \bibitem{Pecher2016}
    Hartmut Pecher.
    \newblock Low regularity solutions for {C}hern-{S}imons-{D}irac systems in the
      temporal and {C}oulomb gauge.
    \newblock {\em Electron. J. Differential Equations}, pages Paper No. 174, 16,
      2016.
    
    \bibitem{pusa}
    Fabio Pusateri.
    \newblock Modified scattering for the {B}oson star equation.
    \newblock {\em Comm. Math. Phys.}, 332(3):1203--1234, 2014.
    
    \bibitem{sautwang2021-compde}
    Jean-Claude Saut and Yuexun Wang.
    \newblock {Global dynamics of small solutions to the modified fractional
      Korteweg-de Vries and fractional cubic nonlinear Schr\"odinger equations}.
    \newblock {\em Communications in Partial Differential Equations},
      46(10):1851--1891, 2021.
    
    \bibitem{YCH}
    Changhun Yang.
    \newblock Scattering for 2d semi-relativistic hartree equations with short
      range potential.
    \newblock {\em preprint}.
    
    \end{thebibliography}

\medskip{}

\end{document}
