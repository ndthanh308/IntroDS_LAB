%============================================================ 
% AA vers. 9.1, LaTeX class for Astronomy & Astrophysics
%============================================================

% Document class, letter
\documentclass[longauth]{aa}  
%\documentclass[longauth]{aa}  

%%%%%%%%%%%%%%%%%%%%%%%%%%%%%%%%%%%%%%%%%%%%%%%%%%%%%%%%%%%%%%%%%%%%%%%%%%%%%%%
% PACKAGES
%%%%%%%%%%%%%%%%%%%%%%%%%%%%%%%%%%%%%%%%%%%%%%%%%%%%%%%%%%%%%%%%%%%%%%%%%%%%%%%

\usepackage{graphicx}
\usepackage{txfonts}
\usepackage{siunitx}						
\usepackage{physics}
\usepackage{hyperref}
\usepackage{multicol}
\usepackage{lipsum}

%If not using the letter format
\usepackage{xcolor}

%%%%%%%%%%%%%%%%%%%%%%%%%%%%%%%%%%%%%%%%%%%%%%%%%%%%%%%%%%%%%%%%%%%%%%%%%%%%%%%
% USEFULL COMMANDS
%%%%%%%%%%%%%%%%%%%%%%%%%%%%%%%%%%%%%%%%%%%%%%%%%%%%%%%%%%%%%%%%%%%%%%%%%%%%%%%

\newcommand\solarmass{M_\odot}

%%%%%%%%%%%%%%%%%%%%%%%%%%%%%%%%%%%%%%%%%%%%%%%%%%%%%%%%%%%%%%%%%%%%%%%%%%%%%%%
% DOCUMENT SETUP
%%%%%%%%%%%%%%%%%%%%%%%%%%%%%%%%%%%%%%%%%%%%%%%%%%%%%%%%%%%%%%%%%%%%%%%%%%%%%%%

%Setup packages
\hypersetup{
	colorlinks			= true			, % Set to false to disable coloring links
	citecolor			= blue!80!black			, % The color of citations
	linkcolor			= blue			, % The color of references to document (sections ,figures, etc)
	urlcolor			=blue!80!black		, % The color of hyperlinks (URLs)
	breaklinks			=true           ,
	%unicode								,
}

%Setup bibliography
\usepackage{natbib}
\bibpunct{(}{)}{;}{a}{}{,}             
\bibliographystyle{./aa_url}

%============================================================

\begin{document}

\title{Polarimetry and Astrometry of NIR Flares as Event Horizon Scale, Dynamical Probes for the Mass of Sgr~A*}
 	
 	\author{
  	    The GRAVITY Collaboration\fnmsep\thanks
  	    {
  	    GRAVITY is developed in a collaboration by MPE, LESIA of Paris Observatory / CNRS / Sorbonne Université / Univ.
  	    Paris Diderot and IPAG of Université Grenoble Alpes / CNRS, MPIA, Univ. of Cologne, CENTRA - Centro de
  	    Astrofisica e Gravitação, and ESO. Corresponding authors:
  	    A.~Drescher (drescher@mpe.mpg.de), D.~C.~Ribeiro (dribeiro@mpe.mpg.de) \& N.~Aimar (nicolas.aimar@obspm.fr). 
  	    }:
  	    R.~Abuter                   \inst{4}            \and
  	    N.~Aimar                    \inst{2}            \and
  	    P.~Amaro Seoane             \inst{18,1,25,19}   \and
  	    A.~Amorim                   \inst{8,7}          \and
  	    M.~Bauböck                  \inst{16}           \and
  	    J.P.~Berger                 \inst{3,4}          \and
  	    H.~Bonnet                   \inst{4}            \and
  	    G.~Bourdarot                \inst{1}            \and
  	    W.~Brandner                 \inst{5}            \and
  	    V.~Cardoso                  \inst{7,21}         \and
  	    Y.~Clénet                   \inst{2}            \and
  	    R.~Davies                   \inst{1}            \and
  	    P.T.~de~Zeeuw               \inst{24}           \and
  	    J.~Dexter                   \inst{11}           \and
  	    A.~Drescher                 \inst{1}            \and
  	    A.~Eckart                   \inst{6,17}         \and
  	    F.~Eisenhauer               \inst{1,23}         \and
  	    H.~Feuchtgruber             \inst{1}            \and
  	    G.~Finger                   \inst{1}            \and
  	    N.M.~Förster~Schreiber      \inst{1}            \and
  	    A.~Foschi                   \inst{7,10}         \and
  	    P.~Garcia                   \inst{12,7}         \and
  	    F.~Gao                      \inst{13,1}         \and
  	    Z.~Gelles                   \inst{26}           \and
  	    E.~Gendron                  \inst{2}            \and
  	    R.~Genzel                   \inst{1,14}         \and
  	    S.~Gillessen                \inst{1}            \and
  	    M.~Hartl                    \inst{1}            \and
  	    X.~Haubois                  \inst{9}            \and
  	    F.~Haussmann                \inst{1}            \and
  	    G.~Heißel                   \inst{20,2}         \and
  	    T.~Henning                  \inst{5}            \and
  	    S.~Hippler                  \inst{5}            \and
  	    M.~Horrobin                 \inst{6}            \and
  	    L.~Jochum                   \inst{9}            \and
  	    L.~Jocou                    \inst{3}            \and
  	    A.~Kaufer                   \inst{9}            \and
  	    P.~Kervella                 \inst{2}            \and
  	    S.~Lacour                   \inst{2}            \and
  	    V.~Lapeyrère                \inst{2}            \and
  	    J.-B.~Le~Bouquin            \inst{3}            \and
  	    P.~Léna                     \inst{2}            \and
  	    D.~Lutz                     \inst{1}            \and
  	    F.~Mang                     \inst{1}            \and
  	    N.~More                     \inst{1}            \and
  	    T.~Ott                      \inst{1}            \and
  	    T.~Paumard                  \inst{2}            \and
  	    K.~Perraut                  \inst{3}            \and
  	    G.~Perrin                   \inst{2}            \and
  	    O.~Pfuhl                    \inst{4,1}          \and
  	    S.~Rabien                   \inst{1}            \and
  	    D.~C.~Ribeiro                \inst{1}            \and
  	    M.~Sadun Bordoni            \inst{1}            \and
  	    S.~Scheithauer              \inst{5}            \and
  	    J.~Shangguan                \inst{1}            \and
  	    T.~Shimizu                  \inst{1}            \and
  	    J.~Stadler                  \inst{15,1}         \and
  	    O.~Straub                   \inst{1,22}         \and
  	    C.~Straubmeier              \inst{6}            \and
  	    E.~Sturm                    \inst{1}            \and
  	    L.J.~Tacconi                \inst{1}            \and
  	    F.~Vincent                  \inst{2}            \and
  	    S.~von~Fellenberg           \inst{17,1}         \and
  	    F.~Widmann                  \inst{1}            \and
  	    M.~Wielgus                  \inst{17}           \and
  	    E.~Wieprecht                \inst{1}            \and
  	    E.~Wiezorrek                \inst{1}            \and
  	    J.~Woillez                  \inst{4}            
    }    
  
  
    \institute{
  	    %1 
  	    Max Planck Institute for Extraterrestrial Physics, Giessenbachstraße 1, 85748 Garching, Germany \and
  	    %2
  	    LESIA, Observatoire de Paris, Université PSL, CNRS, Sorbonne Université, Université de Paris, 5 place Jules Janssen, 92195 Meudon, France \and 	
  	    %3
  	    Univ. Grenoble Alpes, CNRS, IPAG, 38000 Grenoble, France \and
  	    %4
  	    European Southern Observatory, Karl-Schwarzschild-Straße 2, 85748 Garching, Germany \and
  	    %5
  	    Max Planck Institute for Astronomy, Königstuhl 17, 69117 Heidelberg, Germany \and
  	    %6
  	    1st Institute of Physics, University of Cologne, Zülpicher Straße 77, 50937 Cologne, Germany \and
  	    %7
  	    CENTRA - Centro de Astrofísica e Gravitação, IST, Universidade de Lisboa, 1049-001 Lisboa, Portugal \and
  	    %8
  	    Universidade de Lisboa - Faculdade de Ciências, Campo Grande, 1749-016 Lisboa, Portugal \and
  	    %9
  	    European Southern Observatory, Casilla 19001, Santiago 19, Chile \and
  	    %10
  	    Universidade do Porto, Faculdade de Engenharia, Rua Dr. Roberto, Frias, 4200-465 Porto, Portugal \and
  	    %11
  	    Department of Astrophysical \& Planetary Sciences, JILA, Duane Physics Bldg., 2000 Colorado Ave, University of Colorado, Boulder, CO 80309, USA 
  	    \and
  	    %12
  	    Faculdade de Engenharia, Universidade do Porto, rua Dr. Roberto Frias, 4200-465 Porto, Portugal \and
  	    %13
  	    Hamburger Sternwarte, Universität Hamburg, Gojenbergsweg 112, 21029 Hamburg, Germany \and
  	    %14
  	    Departments of Physics \& Astronomy, Le Conte Hall, University of California, Berkeley, CA 94720, USA \and
  	    %15
  	    Max Planck Institute for Astrophysics, Karl-Schwarzschild-Straße 1, 85748 Garching, Germany \and
  	    %16
  	    Department of Physics, University of Illinois, 1110 West Green Street, Urbana, IL 61801, USA \and
  	    %17
  	    Max Planck Institute for Radio Astronomy, auf dem Hügel 69, 53121 Bonn, Germany \and
  	    %18
  	    Institute of Multidisciplinary Mathematics, Universitat Politècnica de València, València, Spain \and
  	    %19
  	    Kavli Institute for Astronomy and Astrophysics, Beijing, China \and
  	    %20
  	    Advanced Concepts Team, ESA, TEC-SF, ESTEC, Keplerlaan 1, 2201 AZ Noordwijk, The Netherlands \and
  	    %21
  	    Niels Bohr International Academy, Niels Bohr Institute, Blegdamsvej 17, 2100 Copenhagen, Denmark \and
  	    %22
  	    ORIGINS Excellence Cluster, Boltzmannstraße 2, 85748 Garching, Germany \and
  	    %23
  	    Department of Physics, Technical University of Munich, 85748 Garching, Germany \and
  	    %24
  	    Leiden University, 2311EZ Leiden, The Netherlands \and
  	    %25
  	    Higgs Centre for Theoretical Physics, Edinburgh, UK \and
  	    %26
  	    Department of Physics, Jadwin Hall, Washington Road, Princeton, New Jersey 08544, USA
    }  

    \date{July 20, 2023}
  
    \abstract{
    We present new astrometric and polarimetric observations of flares from Sgr~A* obtained with GRAVITY, the near-infrared interferometer at ESO’s Very Large Telescope Interferometer (VLTI), bringing the total sample of well-covered astrometric flares to four and polarimetric ones to six, where we have for two flares good coverage in both domains. All astrometric flares show clockwise motion in the plane of the sky with a period of around an hour, and the polarization vector rotates by one full loop in the same time. Given the apparent similarities of the flares, we present a common fit, taking into account the absence of strong Doppler boosting peaks in the light curves and the EHT-measured geometry. Our results are consistent with and significantly strengthen our model from 2018: We find that a) the combination of polarization period and measured flare radius of around nine gravitational radii ($9 R_g \approx 1.5 R_{ISCO}$, innermost stable circular orbit) is consistent with Keplerian orbital motion of hot spots in the innermost accretion zone. The mass inside the flares’ radius is consistent with the \SI{4.297e6}{\solarmass} measured from stellar orbits at several thousand $R_g$. This finding and the diameter of the millimeter shadow of Sgr~A* thus support a single black hole model. Further, b) the magnetic field configuration is predominantly poloidal (vertical), and the flares’ orbital plane has a moderate inclination with respect to the plane of the sky, as shown by the non-detection of Doppler-boosting and the fact that we observe one polarization loop per astrometric loop. Moreover, c) both the position angle on sky and the required magnetic field strength suggest that the accretion flow is fueled and controlled by the winds of the massive, young stars of the clockwise stellar disk 1-5 $\arcsec$ from Sgr~A*, in agreement with recent simulations.
    }
  
    \keywords{Galaxy: nucleus -- Black hole physics -- Gravitation -- Relativistic processes }
  
    \titlerunning{Dynamical Probes for the Mass of Sgr~A*}
    \authorrunning{GRAVITY Collaboration}
    \maketitle

%============================================================

\section{Introduction}
\label{sec:introduction}

The presence of a massive black hole (MBH) in the Galactic Center (GC) has been established, beyond any reasonable doubt, in four-decade-long research by several teams \citep[e.g.][]{Genzel_2021}.
Sgr~A* is not only the best case for the existence of black holes, but unlike any other known black hole, we can resolve Sgr~A* both spatially and temporally, allowing for dynamical tests long considered unfeasible. The combination of precise stellar astrometry and spectroscopy has revealed the gravitational redshift signature in the orbit of the star S2 \citep{GRAVITY_2018, Do_2019a}.
Furthermore, the prograde relativistic precession of the orbit yielding an astrometric signature of around 0.5 mas per revolution was detected employing the ultra-high resolution of the GRAVITY interferometer \citep{GRAVITY_2020a,GRAVITY_2022}. Emission from a distance of just a few $R_g$ ($R_g = G M / c^2$ corresponds to \SI{5.12}{\micro as} at our assumed $R_0 = \SI{8.277}{kpc}$ and for $M = \SI{4.297e6}{\solarmass}$) was detected in the X-ray \citep{Baganoff_2001,Neilsen_2013}, near-infrared (NIR, \cite{Genzel_2003a,Eisenhauer_2005,Dodds_Eden_2011,Do_2019b,Witzel_2021}), and radio bands \citep{Balick_1974,Lo_1998,Krichbaum_1998,Shen_2005,Yusef_Zadeh_2006,Doeleman_2008,EHT_2022} The small spatial scales imply that the light arises deep in the strong gravitational field of the MBH, and that its propagation is subject to strong relativistic effects.

The NIR emission of Sgr~A* is highly variable, with \emph{flares} of ten times the quiescent flux occurring once to twice per day \citep{Dodds_Eden_2011,Witzel_2018,GRAVITY_2020b}. The flare emission most likely comes from locally heated electrons emitting synchrotron emission from the NIR, possibly up to the X-ray band \citep{Yuan_2003,Dodds_Eden_2009,Ponti_2017,GRAVITY_2021b}. The synchrotron model with emission coming from a compact source also explains the observed variable NIR polarization of a few tens of percent linear polarization and a negligible amount of circular polarization \citep{Eckart_2006,Eckart_2008,Trippe_2007,Zamaninasab_2010,Witzel_2010,Shahzamanian_2015,GRAVITY_2020a}, as well as the co-evolution of the light curves in the two bands. The estimated source sizes of around 1 $R_g$ and the required magnetic field strengths of around \SI{30}{}-\SI{80}{G} \citep{Dodds_Eden_2010,Eatough_2013,Ripperda_2020} are in accordance with the so-called hot spot model with orbital radii close to the innermost circular orbit \citep{Broderick_2005,Broderick_2006,Meyer_2006,Hamaus_2009,Ripperda_2020,GRAVITY_2020d}. This model has observable signatures in the astrometry and polarimetry of the NIR emission, where the astrometric amplitudes of a few \SI{10}{\micro as} were one key driver for the development of GRAVITY \citep{Paumard_2006,GRAVITY_2017}. Moreover, \cite{Dexter_2020} show by means of general relativistic magnetohydrodynamic simulations that for magnetically arrested disks (MAD), flares with properties similar to the observed ones might arise from magnetic reconnection events.

GRAVITY combines the light from either the four Auxiliary Telescopes (\SI{1.8}{\meter} diameter) or the four Unit Telescopes (\SI{8}{\meter} diameter) of the Paranal observatory. The spatial resolution in K-band is around \SI{3}{mas}, and the accuracy by which sources can be located is of a few \SI{10}{ \micro as} -- good enough to follow the motions of orbiting hot spots. In 2018, we observed in three nights flares from Sgr~A*, all of which showed astrometrically clockwise motions at radii compatible with the hot spot model. For one flare, we also obtained polarimetry in 2018, showing a full rotation of the polarization vector \citep{GRAVITY_2018b}.

Clearly, the number of such high-precision flares was still low. Hence, we embedded in our nightly observing strategy regular visits to SgrA* and analysed in real-time the incoming data to monitor for flux increases. If a flare occurred, we continued to observe Sgr~A*. 

Here, we report on one new flare, for which the astrometric and polarimetric data are good enough to resolve orbital motion, and on four additional ones, for which only the polarimetry is of sufficient quality.	

%============================================================

\section{Observations}
\label{sec:observations}

We observe the GC typically during six months per year, with a total allocation of around 90 hours. The data serve for the astrometric monitoring of the stellar orbits \citep{GRAVITY_2018,GRAVITY_2022} and to observe the emission from Sgr~A* itself \citep{GRAVITY_2018b}.We report on so far unpublished data obtained from observations in 2019, 2021, and 2022 during which Sgr~A* exhibited flares. We analyze them together with previous flare observations. Specifically, we use the observations listed in Table \ref{tab:observations}.
%
\begin{table}
	\caption[]{List of observations with Astrometric(A) and Polarimetric (P) data. The Programme IDs for the 2018 data are 0101.B-0576(C) and 0101.B-0576(E). For 2019, 2020, and 2022 they are respectively 0103.B-0032(C), and 105.20B2.004, 109.22ZA.002. }
	\label{tab:observations}
	\centering
	\begin{tabular}{cccclc}
		\hline \hline
		\noalign{\smallskip}
		Date of night & UTC interval & \# of frames & A & P\\
		\noalign{\smallskip}
		\hline
		\noalign{\smallskip}
		2018-05-27 & 07:02 – 09:15 & 19 & x	& 	\\
		2018-07-22 & 02:44 – 04:51 & 19 & x & 	\\
		2018-07-28 & 00:02 – 04:16 & 36 & x & x	\\
		2019-08-18 & 23:46 – 03:46 & 32 &  	& x	\\
		2021-07-26 & 23:25 – 03:47 & 36 &  	& x	\\
		2021-07-27 & 23:29 – 03:21 & 33 &  	& x	\\
		2022-05-19 & 04:14 – 09:41 & 45 & x	& x	\\
		2022-08-16 & 23:40 – 02:06 & 26 &   & x	\\
		\noalign{\smallskip}
		\hline 
	\end{tabular}
\end{table}
%

%============================================================

\section{Data}
\label{sec:data}

Our GRAVITY data deliver position and polarization measurements simultaneously. The former are obtained from the phase measurements, the latter by comparing (interferometric) fluxes behind a Wollaston prism.

%------------------------------
\subsection{Astrometry}
\label{subsec:data_astrometry}
%------------------------------

For the GC, GRAVITY uses optical fibers to simultaneously observe two objects and record the interference of both targets on the six baselines formed by the four \SI{8}{\meter} telescopes. One target is the star IRS 16C ($m_K = 9.7$) used as a phase reference, and the other one is the science target. Each GRAVITY fiber has a field of view of approximately \SI{70}{mas}, comparable to the diffraction limit of the individual telescopes. 

Since the beginning of the GRAVITY observations, stars would always be close enough to SgrA* to be observed in the same field of view, yielding binary, triple, or multiple signatures in the interference pattern. In 2018, the star S2 ($m_K = 14.0$) was observed together with Sgr~A* in that way. In this so-called single beam measurement, information %about the sources, 
such as the position vector and flux ratio between the two sources, is directly inferred from the measured closure phases and amplitudes. Since 2019, S2 has moved so far out of the interferometric field of view, that other fainter stars have been detected close to Sgr~A* \citep{GRAVITY_2021a,GRAVITY_2022}. However, during a flare, Sgr~A* outshines these fainter stars and effectively appears as a single
source. In this case the astrometry of Sgr~A* has to be derived from the visibility phases, which are linked to the visibility phases of the phase reference target via GRAVITY’s metrology system \citep[Appendix A therein]{GRAVITY_2020a}.

Hence, for the new, dual-beam data sets, the astrometry is obtained as a sum of three measured phases: The one of Sgr~A*, the one of IRS 16C, and the metrology linkage. The source models are simply unary fits in this case. In practice, we measure the Sgr~A* phases and every few exposures, that of a local phase reference, for which we chose S2. Since
%
\begin{equation}
	\left( \varphi_{SgrA*} - \varphi_{IRS16C} \right) 
	- 
	\left( \varphi_{S2} - \varphi_{IRS16C} \right)
	=
	\left( \varphi_{SgrA*} - \varphi_{S2} \right) \; , 
\end{equation}
%
we reference the Sgr~A* positions to S2. The proper motion of S2 during a flare is small enough not to affect our measurements.

A significant systematic error in our dual-beam flare astrometry is due to high-order imperfections in the GRAVITY optics. As the metrology sensors cover only four points of the pupil, the high-order aberrations translate into phase errors and, thus, astrometric uncertainty. These errors average out effectively by the pupil-rotation over the course of the night, so that stellar orbits are routinely measured with an accuracy down to \SI{20}{}-\SI{50}{\micro as} \cite[e.g.][]{GRAVITY_2022}. For the individual, short exposures of flare observations, however, these metrology footprint errors dominate over the statistical errors. Since 2022, we circumvent this source of error by simultaneously modulating the telescope pupils relative to the metrology receivers such that the high-order aberrations average out. This restores the astrometry to a level similar to what we achieved in the single-beam case. This new observing mode was not yet in place in 2019 and 2021, and the accuracy of the dual-beam astrometry was not at the level needed for flare orbits during this period.

%------------------------------
\subsection{Polarimetry}
\label{subsec:data_polarimetry}
%------------------------------

In its polarimetric mode, GRAVITY splits the light post-beam combination by means of a Wollaston prism into two orthogonal, linear polarization states. Employing a half-wave plate, we rotate the polarization by 45° between exposures, such that for each pair of exposures, we measure the polarized flux in the \SI{0}{\degree}, \SI{45}{\degree}, \SI{90}{\degree} and \SI{135}{\degree} directions. Note that polarization angles are only defined between \SI{0}{\degree} and \SI{180}{\degree}. From the polarized fluxes, the Stokes parameter $Q$ and $U$ are calculated as
%
\begin{equation}
	Q^\prime = \frac{f_0 - f_{90}}{f_{0} + f_{90}} \quad ,\quad U^\prime = \frac{f_{135} - f_{45}}{f_{135} + f_{45}} \quad ,
\end{equation}
%
where the primes indicate that the values are measured in the detector coordinate system. To get the on-sky polarization state $Q$ and $U$ one has to correct for the geometric beam propagation and the instrumental polarization. This is done with the model developed by \cite[submitted]{GRAVITY_2023b}. As GRAVITY cannot measure circular polarization, we correct for birefringence with the model presented by \cite{Witzel_2010}, using the fact that the polarized NIR emission of Sgr~A* is predominantly linearly polarized.

Our $Q$ and $U$ values are fractional values, normalized to the observed intensity. Unlike at longer millimeter wavelengths, the variations of the polarized fluxes are completely dominated by the overall brightness variation -- a result of the much shorter heating and cooling time scales in the NIR \citep{von_Fellenberg_2023}. Thus, the fractional values carry the information on the magnetic field and geometry of the Sgr~A* system.


Following \cite{GRAVITY_2020b} there are two different ways to measure the required fluxes:
%
\begin{enumerate}
	\item If Sgr~A* is the dominant (or only) source in the (interferometric) field of view, the coherent flux directly measures Sgr~A*’s flux.
	\item If there are multiple sources in the field of view, a multiple source fit to these sources includes the flux ratio between Sgr~A* and the stars, which multiplied with the coherent flux yields the desired Sgr~A* flux.
\end{enumerate}
%
Flux measurements for observations up to 2020 are reported in \cite{GRAVITY_2020b}. For 2021 and 2022 we use binary fits with the stars S29 (2021) and S38 (2022) as second object beyond Sgr~A*.

%============================================================

\section{Analysis}
\label{sec:analysis}


% Figure environment removed

% Figure environment removed

% Figure environment removed

Our sample of flares with polarimetric coverage has grown from one \citep{GRAVITY_2018b,GRAVITY_2020c} to six, all of which show a single, clockwise loop in the $Q-U$-plane (Fig. \ref{fig:polarization_flares}).

In contrast to the flare astrometry, where we need a precision of $\approx \SI{10}{}-\SI{30}{\micro as}$ per 5-minute exposure, polarimetry is less sensitive to observing conditions and is not penalized by the optical imperfections. As a result, we got valid polarimetry for flares even if the conditions were not good enough for astrometry. This is the case for the flare in 2019, the two from 2021 and the one from August 2022.

Our sample of astrometric measurements consists now of four events, adding to the three flares from \cite{GRAVITY_2020b} the one from 2022-05-19 (Fig. \ref{fig:astrometry_flares}).

The similarity between the flares motivates us to look at them in an averaged way. We combine the data of the six polarimetric and four astrometric flares by allowing only a relative time shift between the individual events and finding the solution that best represents a sinusoidal in both sky coordinates for the combined flare data. This yields the data shown in Fig. \ref{fig:averaged_flares}. Both averaged data sets are very well described by elliptical figures with a common period of around 60 minutes, and constant angular velocity of \SI{6}{\degree/\minute}.

The mean astrometric loop size in Fig.~\ref{fig:averaged_flares} of around $r=61\pm9\; \SI{ }{\micro as}$ (corresponding to $0.50 \pm 0.07 \; \SI{}{AU}$ for our assumed $R_0$) and the polarization period of $P = 60 \pm 3 \; \SI{}{\min}$ allows for estimating the enclosed mass $M_{enc}$. Assuming Keplerian orbital motion, this yields $M_{enc,estimate} = (5.1 \pm 1.6) \times \SI{e6}{\solarmass}$ (Appendix \ref{appendix:mass_estimate}). Within the error, this value agrees with the one known from stellar orbits, $M = (4.297 \pm 0.012) \times \SI{e6}{\solarmass}$ \citep{GRAVITY_2022} and  shows that (within the uncertainties) the mass of Sgr~A* is enclosed within the flare orbit.

In the following, we refine the value of $M_{encl}$ by modeling the astrometry and polarimetry of the flares. We model the astrometric motion of the combined data set as in \cite{GRAVITY_2020d} by an orbiting relativistic hot spot, including the effects of lensing via ray tracing. However, here, we enhance the pure astrometric fit by including more information, mainly from the polarimetry:
%
\begin{enumerate}
	\item We identify the period of the polarization loop with the astrometric one. Technically this sets a prior on the orbital period. We use $P_{pol} = 60 \pm 3 \; \SI{}{\minute}$.
	%
	\item In Sect.~\ref{subsec:analysis_modeling_the_polarization_loop} we show that observing a single polarization loop per astrometric loop yields a strong prior on the allowed range of inclinations. 
	%
	The angular velocity of \SI{6}{\degree/min} constrains the inclination to $i = \SI{157}{\degree} \pm \SI{5}{\degree}$.
	%
	
	%
	\item The absence of strong spikes from Doppler beaming in the light curves also limits the inclination to be close to face-on. Following \cite{Hamaus_2009} and \cite[Appendix C therein]{GRAVITY_2018} we place a conservative prior of $i = \SI{180}{\degree} \pm \SI{40}{\degree}$.
	%
	\item Finally, the EHT image of Sgr~A* constrains the inclination to $i = \SI{180}{\degree} \pm \SI{50}{\degree}$ \citep{EHT_2022}.
\end{enumerate}

We take into account these additional constraints by including them in the likelihood function and determine a globally best-fitting orbiting hot spot model. The parameters of the fit are the hot spot’s orbital radius $R$, the inclination $i$, the position angle $\Omega$, and the enclosed mass $M_{enc}$.

%------------------------------
\subsection{Modeling the polarization loop}
\label{subsec:analysis_modeling_the_polarization_loop}
%------------------------------

The polarization signal of an orbiting hot spot probes the magnetic field around Sgr~A* \citep{Broderick_2005,GRAVITY_2018b,GRAVITY_2020c,GRAVITY_2020d,Wielgus_2022}. Analytical approaches and simulations of the
polarization changes have been investigated by \cite{GRAVITY_2020c} and \cite{Narayan_2021}. A dominantly toroidal (i.e., with field lines horizontally going around the black hole) or radial field leads, in general, to a full rotation of the emission geometry per astrometric revolution, which corresponds to two full $Q-U$ loops. A dominantly poloidal field (with field lines vertically crossing the orbital plane) can lead to different polarization loops, with either no rotation (purely vertical field and/or edge-on geometry, $i = \SI{90}{\degree}$), one loop (dominantly vertical field and moderate inclination $i \approx \SI{30}{\degree}, \; \SI{150}{\degree}$) or two loops (inclination face-on, $i = \SI{0}{\degree}, \; \SI{180}{\degree}$). Hence, observing one loop per astrometric revolution severely constrains the configuration space and argues for a vertical magnetic field and moderate inclination.

To derive the inclination constraint, we use the model introduced by \cite{Gelles_2021}. This semi-analytic model implements an equatorial, point-like hot spot orbiting on a Keplerian orbit around a Kerr black hole with a local magnetic field. The model uses the semi-analytic solution for light rays \citep{Gralla_2020} to obtain a polarized image of the synchrotron emitting source. 

Since the model describes purely linear polarized emission, we extend it to take into account the observed polarization fraction in the data. Although coming from a point-source emitter, the model captures the fundamental features of extended emission regions \citep{GRAVITY_2020c,Vos_2022}. A more detailed description of the model can be found in Appendix~\ref{appendix:depolarized-hot-spot}.
	
The local magnetic field is given in cylindrical coordinates, and allows a clear distinction between radial, toroidal, and poloidal magnetic field configurations. Other free parameters are the inclination of the orbit, the orbital radius, the local boost, and the spin of the black hole. The latter only has a minor impact on the polarization signal \citep{Gelles_2021}, and hence we choose to work with zero-spin models. Further, we fix the radius to $9 R_g$ as this matches the observed astrometric and polarimetric periods. Using radii in the $8-10 R_g$ range does not alter our results (see Fig.\ref{fig:gelles_model_mural}). Fixing the radius also fixes the Lorentz boosting factor to $\beta = 0.37$. The model does not take radiative transfer effects into account. Further, time delays and luminosity changes due to the combination of a non-flat spectral energy distribution and Doppler boosting are neglected.

As expected from the discussion above, the rotation speed distinguishes very well between the poloidal and toroidal/radial fields. Observing one polarimetric loop per astrometric one only occurs for poloidal fields. The inclination can be constrained from the observed 
angular velocity with which the $Q-U$ loop gets traversed (Fig. \ref{fig:morphology_constraints}). The observed angular velocity of $\SI{6}{\degree / min}$ for the astrometry is only compatible with poloidal field configurations with an inclination of  $i = \SI{157}{\degree} \pm \SI{5}{\degree}$. 
%
% Figure environment removed

Figure \ref{fig:polarization_fit} shows that the model for $i = \SI{157}{\degree}$ is an excellent description the observed polarization changes. We allowed the position angle on sky and the arbitrary zero point in time to be adjusted and picked the combination which yields the smallest $\chi^2$, to the data (Fig. \ref{fig:chi_squared_fit}). The best-matching position angle is $PA = \SI{25}{\degree}$. These values are consistent with the results from \cite{GRAVITY_2020c} and \cite{Wielgus_2022}, as well as the inclination measured by \cite{GRAVITY_2018b} and the EHT ring \citep{EHT_2022}.
%
% Figure environment removed

In summary, the analysis of the polarization loops gives robust constraints and a result compatible with the astrometric signature. In the following, we combine this information in a combined fit.

%------------------------------
\subsection{Combined fit}
\label{subsection:analysis_combined_fit}
%------------------------------
%
% Figure environment removed
%
% Figure environment removed
%
% Figure environment removed

Our explicit, combined fit (Fig. \ref{fig:astrometry_fit} and Fig. \ref{fig:corner_plot}) uses the astrometric data and the constraints derived from the polarimetry. We obtain the set of parameters presented in Table \ref{tab:best_fit_values}. 
%
\begin{table}
	\caption[]{Posterior mean values of the astrometry fit. The model corresponding to this set of parameters is displayed in Fig. \ref{fig:astrometry_fit}.}
	\label{tab:best_fit_values}
	\centering
	\begin{tabular}{cc}
	\hline \hline
	\noalign{\smallskip}
	Parameter 				& Posterior mean                       						\\
	\noalign{\smallskip}
	\hline
	\noalign{\smallskip}
	Radius             		& $R = 8.9^{+1.5}_{-1.3} R_g$                 				\\[3pt]
	Inclination          	& $i = 154.9^{ +\SI{4.6}{\degree} }_{ \SI{-4.6}{\degree}}$  \\[3pt]
	Position Angle          & $\Omega = 177.3^{ +\SI{24}{\degree} }_{ \SI{-23}{\degree}}$ \\[3pt]
	Enclosed Mass          	& $M_{enc} = 4.2^{+1.2}_{-0.9} \times 10^6\solarmass$                				\\ 
	\noalign{\smallskip}
	\hline
\end{tabular}
\end{table}
%
Within the uncertainties, the flare radius and the orientation agree with what \cite{GRAVITY_2018,GRAVITY_2020c,GRAVITY_2020d} have found from the previous, smaller data set. New is the mass constraint, which stems from the new data, and in particular from the strong, independent period constraint from the polarimetry. The value for the mass inside $9 R_g$ matches the simple estimate, and again also the mass derived using stellar
orbits.

%============================================================

\section{Results}
\label{sec:results}


The most important result we find is that the mass of Sgr~A* measured from stellar orbits, $M = (4.297 \pm 0.012) \times \SI{e6}{\solarmass}$, is enclosed within 9 $R_g$, corresponding to 0.38 AU, or roughly the orbital radius of planet Mercury. A similarly strong constraint is obtained from the radio-VLBI image of Sgr~A* \citep{EHT_2022}. 

The stellar orbits with the sub-\% precise mass measurement, on the other hand, limit the size of Sgr~A* to be smaller than the smallest observed pericenter passages, which are those of S29 with 100 AU in 2021 (directly observed by \cite{GRAVITY_2022}) and S14 with 26 AU in 2000 (well-determined orbit, but poorer coverage, \cite{Gillessen_2017} and newer data). We show an updated version of the enclosed mass versus radius in Fig. \ref{fig:genzel_plot}.

Other than the mass determination from the stellar orbits, both inner constraints are not assumption-free. The NIR flare constraint assumes that the motion of the emission is near Keplerian, i.e. governed by the forces of gravity, which is well motivated by 
%some 
accretion flow models \citep{Yuan_2003,Narayan_2023}. The EHT image needs to assume a certain (again well-motivated) structure of the accretion flow, as other gas configurations could mimic the ring-like appearance. Taken together, these two independent estimates make the case for a black hole ever more convincing. Completely excluded are, for example, self-gravitating fermionic dark matter models \citep{Arguelles_2019,Becerra_Vergara_2020} that have typical core radii of $\approx \SI{e-3}{pc} \approx \SI{200}{AU}$.

The fact that the azimuthal speed of the hot spots is near Keplerian is, however, non-trivial. In low density Advection-Dominated Accretion Flows (ADAFs), the gas motions generally are assumed to be sub-Keplerian \citep[e.g.][]{Yuan_2014}. The same is not true for magnetically dominated flows, where the speed can depend on the spin of the black hole, and both sub- and super-Keplerian motions occur. Also, we note that a number of non-Keplerian models have been proposed. \cite{Aimar_2023} consider the outward motion of a hot spot along a conical trajectory, and \cite{Lin_2023} propose that, similar to coronal mass ejections on the Sun, Sgr~A* flux ropes are ejected out and filled with energetic electrons. \cite{Matsumoto_2020} discuss, whether the observed motions are actually pattern motions.


Besides the enclosed mass, we also measure the  orientation of the flare orbits (see Fig. \ref{fig:orientation_plot}). The measured orientation, which is the direction of orbital angular momentum, is close to that of the clockwise stellar disk \citep[][Fig. 9]{Levin_2003,Genzel_2003b,Paumard_2006,Lu_2008,von_Fellenberg_2022}, suggesting a physical connection. 

The stars that make up the disk are predominantly O- and Wolf-Rayet-type stars that have strong winds. Simulations by \cite{Ressler_2020} have indeed shown that the almost spherical accretion flow, fueled by the stellar winds from disk stars carries the initial angular momentum down to event horizon scales, where the gas settles into a disk-like structure. This disk can tilt by a moderate angle with respect to the initial angular momentum, perhaps responsible for the misalignment between stellar disk and flare angular momentum direction in Fig.~\ref{fig:orientation_plot}.
%
The simulations also predict that the infalling gas settles into a magnetically arrested disk (MAD), which naturally carries a poloidal field geometry. Only for strong fields can the geometry withstand the dragging with the fluid which would lead to a toroidal field geometry.Our data, favoring a poloidal field, directly support the MAD state. It is further supported by the estimated magnetic field strength (\SI{30}{}-\SI{80}{G}, \cite{Ripperda_2020}) and measured density profile (see Fig. 6 in \cite{Gillessen_2019} and references given there). \cite{McKinney_2013} and \cite{Ressler_2023} find that for MAD situations, the innermost disk-like structure tends to align with the spin axis of the black hole. The dynamics of NIR flares may in that case actually carry information on the black hole spin rather than just on the initial conditions of the inflow.

%============================================================

\begin{acknowledgements}
    We are very grateful to our funding agencies (MPG, ERC, CNRS [PNCG, PNGRAM], DFG, BMBF, Paris Observatory [CS, PhyFOG], Observatoire des Sciences de l’Univers de Grenoble, and the Fundação para a Ciência e Tecnologia), to ESO and the Paranal staff, and to the many scientific and technical staff members in our institutions, who helped to make NACO, SINFONI, and GRAVITY a reality. JS is supported by the Deutsche Forschungsgemeinschaft (DFG, German Research Foundation) under Germany’s Excellence Strategy – EXC- 2094 – 390783311. A.A., A.F., P.G. and V.C. were supported by Fundação para a Ciência e a Tecnologia, with grants reference SFRH/BSAB/142940/2018, UIDB/00099/2020 and PTDC/FIS-AST/7002/2020. F.W. has received funding from the European Union’s Horizon 2020 research and innovation programme under grant agreement No 101004719. Based on observations collected at the European Southern Observatory under the ESO programme IDs 109.22ZA.005, 109.22ZA.002, 105.20B2.004, 0103.B-0032(C), 0101.B-0576(E), 0101.B-0576(C).
    \end{acknowledgements}
    

\bibliography{./bibliography_short_journals}

%============================================================
%============================================================
\begin{appendix}
	
%------------------------------
\section{Rough mass estimate}
\label{appendix:mass_estimate}
%------------------------------

The period $P$ for an emitter on a circular orbit with radius $r_e$ is simply given by the common Kepler formula
%
\begin{equation}
	P^2 = \frac{4 \pi^2 r_e^3 }{GM} = \frac{4 \pi^2}{c^2} \frac{r_e^3}{R_g}
\end{equation}
%
Following \cite{Gralla_2019} and \cite{Gates_2020}, the observed loop radius on-sky $r_o$ is (to very good approximation) $r_o = r_e +R_g$, meaning that the loop appears enlarged by roughly $1 R_g$ due to the gravitational lensing. This yields
%
\begin{equation}
	R_g = \left(\frac{2 \pi}{P c}\right)^2 (r_o - R_g)^3 \; .
\end{equation}
%
which, for an observed period and radius can be solved for $R_g$ and thus yields a mass $M$.

%------------------------------
\section{Depolarized hot spot model}
\label{appendix:depolarized-hot-spot}	
%------------------------------

The model used in Sec.\ref{sec:analysis} builds on the model presented in \cite{Gelles_2021}. 
The original model describes the non fractional stokes parameters $Q$ and $U$ for completely polarized light such that the total intensity of the hot spot is given by $\sqrt{Q^2 +U^2}$. 
Experimentally we measure both polarized and non-polarized intensities such that the total intensity is $I = I_p + I_{np}$. While we assume that the flux is dominated by the hot spot, it still is only $10-40$ $\%$ polarized~\citep{GRAVITY_2020c}. If the non-polarized intensity comes from some stochastic process (turbulence in the flow, chaotic local magnetic fields, etc) the averaging process of the non-polarized intensity will average to a constant value $\langle I_{np}\rangle$ if enough flares are considered. The average fractional polarization will then be
%
\begin{equation}
	\frac{Q}{I} = \frac{Q}{I_p + \langle I_{np}\rangle} \quad , \quad \frac{U}{I} = \frac{U}{I_p + \langle I_{np}\rangle} \; .
\end{equation}
%
Although a simplification this approach captures the fundamental features of the complex dynamics seen in more realistic hot spot models \citep[see ][]
%and Fig. \ref{fig:gelles_model_mural}
{Vos_2022,Dexter_2020}. 

An example of how one can use the non-polarized emission to match the observational average fractional polarization 
%\
\begin{equation}
	\left \langle \frac{I_p}{I} \right \rangle= 
	\left \langle \frac{\sqrt{Q^2 + U^2}}{I} \right \rangle
\end{equation}
%
is shown in Fig. \ref{fig:fractional_polarization_example}. Matching the average polarization fraction of the data naturally represents the data points.

%\FloatBarrier
% Figure environment removed

%%%%%%%%%%%%%%%%%%%%%%%%%%%%%%%%%%%%%%%%%%%
\section{QU loop morphology}
\label{appendix:QU_loop_morphology}
%%%%%%%%%%%%%%%%%%%%%%%%%%%%%%%%%%%%%%%%%%%

Regardless of the depolarization fraction used in the depolarized hot spot model (see Appendix \ref{appendix:depolarized-hot-spot}), the predicted Electric Field Position Angle (EVPA) is always given by
%
\begin{equation}
	\text{EVPA} = \frac{1}{2} \arctan \left( \frac{U}{Q}   \right)
\end{equation}
%
and depends solely on the Stokes parameters. The angular velocity with which the $Q-U$ loops get traversed (see Fig. \ref{fig:gelles_model_mural}) is thus independent of the non-polarized emission and makes it a robust value for the inclination constraint presented Fig. \ref{fig:morphology_constraints}. By taking the inclination that matches the $\SI{6}{\degree/min}$ and fixing the polarization emission such that the average fractional polarization matches the observed one, we can fit for the position angle and time starting point of the loop. The $\chi^2$ fit is displayed in Fig. \ref{fig:chi_squared_fit} and the corresponding model in Fig. \ref{fig:polarization_fit}.

%\FloatBarrier
% Figure environment removed
%	
%\FloatBarrier	
%
% Figure environment removed
%
%%%%%%%%%%%%%%%%%%%%%%%%%%%%%%%%%%%%%%%%%%%		
\section{Fitting model}
\label{appendix:fit}
The astrometric fit presented in Sect. \ref{subsection:analysis_combined_fit} is similar to the one presented in~\cite{GRAVITY_2020d}. Besides the new constraints from the polarimetric data, we replace the previously used $\chi^2$-fit by a nested sampling one. The posterior distributions are shown in Fig. \ref{fig:corner_plot} and the corresponding model in Fig.\ref{fig:astrometry_fit}.
%
% Figure environment removed

\end{appendix}



\end{document}