\section{Modulation instability in a ring cavity with a tap coupler} \label{sec:ringMI}

We consider a ring resonator composed of two spans of identical fiber connected to an input coupler $1$ and a tap coupler $2$, as illustrated in Fig. \ref{fig:ring_resonator}.
If there is no coupling between forward and backward fields, it is easy to obtain a map wich describes the behavior of the system at each roundtrip \cite{zezyulin2011modulational}.
The fields propagating in the two spans satisfies NLSE~:
\begin{align}
    i\frac{\partial F_n}{\partial z} \label{eq:propFring}
    -\frac{\beta_2}{2}\frac{\partial^2 F_n}{\partial t^2}+\gamma|F_n|^2F_n &=0,\; 0<z<L,\\
    i\frac{\partial B_n}{\partial z} \label{eq:propBring}
    -\frac{\beta_2}{2}\frac{\partial^2 B_n}{\partial t^2}+\gamma|B_n|^2B_n &=0,\; L<z<2L,
\end{align}
and they are coupled by the following boundary conditions at couplers:
\begin{align}
    F_{n+1}(0,t)&=\theta_1 E_{in} + \rho_1 e^{i\phi_0} B_n(2L,t),\label{eq:BC1ring}\\
    B_n(L,t)&= \rho_2 F_n(L,t).\label{eq:BC2ring}
\end{align}
The total linear phase $\phi_0$ accounts for propagation and phase from the couplers and the index $n$ counts the  number of roundtrips.

\subsection{Steady states}
Steady state solutions of Eqs.~(\ref{eq:propFring}-\ref{eq:propBring}) reads as
\begin{align}
    F_n(z,t)&=F_0e^{i\gamma P_F z},\;P_F=|F_0|^2,\\
    B_n(z,t)&=B_0e^{i\gamma P_B z},\;P_B=|B_0|^2.
\end{align}
By using the boundary conditions, we find the cavity transfer function:
\begin{equation}\label{eq:steady_ring}
    F_0=\frac{\theta_1 E_{in}}{1-\rho_1\rho_2\exp[i(\phi_0+\phi_{NL})]},
\end{equation}
which permits to write the input power $P_{in}=|E_{in}|^2$ as a function of intracavity forward power $P_F=|F_0|^2$ as:
\begin{equation}
    P_{in}=\frac{P_F}{\theta_1^2}\left(1+(\rho_1\rho_2)^2-2\rho_1\rho_2\cos(\theta_0)\right),
\end{equation}
with $\theta_0=\phi_0+\phi_{NL}=\phi_0+\gamma P_F L(1+\rho^2_2)$.

It is worth noting that Eq.~(\ref{eq:steady_ring}) is equivalent to the steady-state of a FP resonator with $G=0$ and it is also equivalent to the steady state of a ring resonator of length $2L$ if $\rho_2=1$.

\subsection{Linear stability analysis}
We consider a perturbation of the steady state in the following form
\begin{align}
    F_n(z,t)&=(\sqrt{P_F}+\eta)e^{i\gamma P_F z},\\
    B_n(z,t)&=(\rho_2\sqrt{P_F}+\varepsilon)e^{i\gamma \rho_2^2 P_F z}e^{i\gamma \phi_B},
\end{align}
where we have assumed without loss of generality $F_0$ real, which fixes the phase $\phi_B=\gamma LP_F(1-\rho_2^2)$ through boundary condition (\ref{eq:BC2ring}).
Linearization around steady solutions gives the equations for the perturbations:
\begin{align}
    \label{eq:pert_eta_map} i\eta_z-\frac{\beta_2}{2}\eta_{tt}+\gamma P_F(\eta+\eta^*)&=0,\\ 
    \label{eq:pert_eps_map} i\varepsilon_z-\frac{\beta_2}{2}\varepsilon_{tt}+\gamma \rho_2^2P_F(\varepsilon+\varepsilon^*)&=0. 
\end{align}
 
We split perturbations into real and imaginary parts, $\eta=a+ib$ and $\varepsilon=c+id$ , we substitute into Eqs.~(\ref{eq:pert_eta_map}-\ref{eq:pert_eps_map}) and Fourier transform to get
\begin{align}\label{eq:evopert}
    \nonumber\begin{pmatrix}
    \hat a \\ \hat b
    \end{pmatrix}_z &=
    \begin{pmatrix}
    0 &-\frac{\beta_2\omega^2}{2}\\
    \frac{\beta_2\omega^2}{2}+2\gamma P_F & 0
    \end{pmatrix}
    \begin{pmatrix}
    \hat a \\ \hat b
    \end{pmatrix},\\
%
    \begin{pmatrix}
    \hat c \\ \hat d
    \end{pmatrix}_z &=
    \begin{pmatrix}
    0 &-\frac{\beta_2\omega^2}{2}\\
    \frac{\beta_2\omega^2}{2}+2\gamma \rho_2^2P_F & 0
    \end{pmatrix}
    \begin{pmatrix}
    \hat c \\ \hat d
    \end{pmatrix}.
\end{align}
The fundamental matrix solutions of systems Eqs.~(\ref{eq:evopert}) are
\begin{align}
    M(z) &=
    \begin{pmatrix}
    \cos kz &-\frac{\beta_2\omega^2}{2k}\sin kz\\
    \frac{2k}{\beta_2\omega^2}\sin kz & \cos kz
    \end{pmatrix},\\
    N(z) &=
    \begin{pmatrix}
    \cos k_\rho z &-\frac{\beta_2\omega^2}{2k\rho}\sin k_\rho z\\
    \frac{2k_\rho}{\beta_2\omega^2}\sin k_\rho z & \cos k_\rho z
    \end{pmatrix},\;\;
\end{align}
with $k,k_\rho$ defined in Eqs.~(\ref{eq:k_omega}) with $P=P_F$ and $\lambda=i\omega$.
The boundary conditions give the following relations:
\begin{align}\label{eq:BC_abcd}
    \nonumber\begin{pmatrix}
    \hat c_n(L) \\ \hat d_n(L)
    \end{pmatrix}&=
    \rho_2
    \begin{pmatrix}
    \hat a_n(L) \\ \hat b_n(L)
    \end{pmatrix},\\
    \begin{pmatrix}
    \hat a_{n+1}(0) \\ \hat b_{n+1}(0)
    \end{pmatrix}&=
    \rho_1
    \begin{pmatrix}
    \cos \theta_0 & -\sin\theta_0\\
    \sin\theta_0 & \cos\theta_0
    \end{pmatrix}
    \begin{pmatrix}
    \hat c_n(2L) \\ \hat d_n(2L)
    \end{pmatrix}.
\end{align}

By combining propagation and boundary conditions, we get the following difference equation:
\begin{equation}
    \begin{pmatrix}
    \hat a_{n+1}(0) \\ \hat b_{n+1}(0)
    \end{pmatrix}=S
    \begin{pmatrix}
    \hat a_{n}(0) \\ \hat b_{n}(0)
    \end{pmatrix},\;\; S=\rho_1\rho_2 R N(L) M(L),
\end{equation}
and $R$ is the rotation matrix defined in Eq.~(\ref{eq:BC_abcd})

The eigenvalues $\lambda_{1,2}$ of matrix $S$ determines the stability of the steady solution.
We find
\begin{equation}
    \lambda_{1,2}=\frac{\tilde\Delta}{2}\pm\sqrt{\frac{\tilde\Delta^2}{4} - |\rho_1\rho_2|^2},
\end{equation}
with $\tilde\Delta$ as defined in Eq.~(\ref{eq:FP_gain}).
Instability takes places if $|\lambda_{1,2}|>1$ and the MI gain is
\begin{equation}\label{eq:gain_map}
    g(\omega)=\frac{1}{2L}\ln\max\left|\frac{\tilde\Delta}{2}\pm\sqrt{\frac{\tilde\Delta^2}{4} - |\rho_1\rho_2|^2}\right|,
\end{equation}
which coincides with the gain for the FP resonator found before in Eq.~(\ref{eq:FP_gain}).

\section{FP-LLE derivation} \label{sec:LLE}
We derive a mean field model, which generalises the Lugiato-Lefever equation, for the description of a passive driven fiber Fabry-Perot cavity.
We follow an approach similar to the one developed in Ref.~\cite{cole2018theory} but with a different starting point, namely coupled NLS  [Eqs.~(\ref{eq:FP_2NLS}, \ref{eq:FP_BC})] rather than Maxwell-Bloch equations.
The main steps are~: (i) change variables to make the boundary conditions periodic and to include the pump term in the propagation equation; (ii) take the good-cavity (or mean field) approximation; (iii) derive a partial differential equation using the modal equations.
We start by defining the following change of variables~\cite{lugiato1988nonlinear,lugiato2015nonlinear}~:
\begin{subequations}\label{eq:transf}
    \begin{eqnarray}
        \nonumber \tilde{F}(z,t) &=& \exp\left[\frac{z-L}{L}\left(\ln{\rho_1}+i\frac{\phi_0}{2}\right)-\sigma z\right]F(z,t)\\
       &+& \frac{\theta_1}{\rho_1}\exp\left(-i\frac{\phi_0}{2}\right) \frac{z-L}{2L}E_{in}(t-\beta_1 z)\;,\\
        \nonumber \tilde{B}(z,t) &=& \exp\left[-\frac{z}{L}\left(\ln{\rho_2}+i\frac{\phi_0}{2}\right)-\sigma z-i\frac{\phi_0}{2}\right ]B(z,t)\\
         &-& \frac{\theta_1}{\rho_1}\exp\left(-i\frac{\phi_0}{2}\right) \frac{z-L}{2L}E_{in}(t+\beta_1 z),
\end{eqnarray}
\end{subequations}
with $\sigma = \frac{1}{2L} \ln(\rho_1/\rho_2)$.
This transformation is more general than the one proposed in~\cite{lugiato1988nonlinear,lugiato2015nonlinear} because we allow the two mirrors to be different and the pump may vary in time.
The boundary conditions given by Eqs.~(\ref{eq:FP_BC}) for the new variables are simplified to~:
\begin{equation}\label{eq:FP_BC_nv}
\begin{split}
    \tilde{F}(0,t) =& \tilde{B}(0,t),\\
    \tilde{F}(L,t) =& \tilde{B}(L,t)
\end{split}
\end{equation}
The simplification of the boundary conditions is payed by an increase in complexity of the propagation equations.
We thus restrict our analysis to good cavities ($\rho_{1,2}\rightarrow 1$ and $\phi_0\rightarrow 0$), for which we can obtain a mean field description.
From Eqs.~(\ref{eq:transf}) we calculate $\partial_z F, \partial_t F, \partial_z B ,\partial_t B$ as a function of $\tilde F,\tilde B$ and their derivatives.
We truncate the obtained expressions at first order in $\rho_{1,2}$ and $\phi_0$ and insert them into Eqs.~(\ref{eq:FP_2NLS}).
By considering that dispersion and nonlinearity are weak (assumptions already used to derive NLS ), we can use zero order expansion ($\tilde F=F$, $\tilde B=B$) in the dispersive and nonlinear terms.
These approximations permit to greatly simplify the propagation equations as follows~:
\begin{subequations}\label{eq:NLS_nv}
    \begin{align}
        \nonumber \frac{\partial \tilde{F}}{\partial z} + \beta_1 \frac{\partial \tilde{F}}{\partial t} + i\frac{\beta_2}{2}\frac{\partial^2 \tilde{F}}{\partial t^2}-\frac{1}{L}\left( \ln{\rho_1\rho_2}+i\frac{\phi_0}{2} \right)\tilde{F} \\
        -\frac{\theta_1}{2L} E_{in}(t-\beta_1 z) = i\gamma \left( |\tilde{F}|^2 + G|\tilde{B}|^2 \right)\tilde{F}, \\
        \nonumber  -\frac{\partial \tilde{B}}{\partial z} + \beta_1 \frac{\partial \tilde{B}}{\partial t} + i\frac{\beta_2}{2}\frac{\partial^2 \tilde{B}}{\partial t^2}-\frac{1}{L}\left( \ln{\rho_1\rho_2}+i\frac{\phi_0}{2} \right)\tilde{B} \\
        -\frac{\theta_1}{2L} E_{in}(t+\beta_1 z) = i\gamma \left( |\tilde{B}|^2 + G|\tilde{F}|^2 \right)\tilde{B}.
    \end{align}
\end{subequations}

\subsection{Modal equations}
We start by finding the modes of the empty and undriven (cold) cavity, then we expand the fields of the hot cavity in terms of the modes of the cold cavity and derive the equations ruling the slow evolution of the modal amplitudes. 
By taking $\beta_2=\phi_0=E_{in}=\gamma=0$, we solve Eqs.~(\ref{eq:NLS_nv}) with boundary conditions Eqs.~(\ref{eq:FP_BC_nv}), to find
\begin{subequations}
    \begin{eqnarray}
        \tilde{F}(z,t) &=&  A \exp\left[ \left(\beta_1\lambda + \frac{\ln{\rho_1\rho_2}}{2L} \right)z\right] e^{-\lambda t}, \\
        \tilde{B}(z,t) &=&  A \exp\left[-\left(\beta_1\lambda + \frac{\ln{\rho_1\rho_2}}{2L}\right)z \right] e^{-\lambda t},
    \end{eqnarray}
\end{subequations}
with
\begin{equation}\label{eq:boundary}
    \exp[2\beta_1\lambda L + \ln(\rho_1\rho_2)] = 1,
\end{equation}
where $A$ and $\lambda$ are constants.
By defining $\lambda=\kappa+i\omega$, we get from Eq.~(\ref{eq:boundary})
\begin{equation}
    \omega_m = \frac{m\pi}{\beta_1 L}\;,\quad \text{and} \quad \kappa = -\frac{\ln(\rho_1\rho_2)}{2\beta_1 L},
\end{equation}
which are the frequencies and the decay rate of the cavity modes.
We may write the modes of the cold cavity as 
\begin{subequations}
    \begin{eqnarray}
        \tilde{F}_m(z,t) &=&  e^{-\kappa t}e^{-i\omega_m(t-\beta_1 z)}\\
        \tilde{B}_m(z,t) &=&  e^{-\kappa t}e^{-i\omega_m(t+\beta_1 z)}.
    \end{eqnarray}
\end{subequations}
The fields in the full model can now be written as the sum of the loss-less cold cavity modes, allowing for a slow temporal variation of the modal amplitudes, which is induced by pumping, nonlinear and dispersive effects.
Note that the small damping  $\kappa$ is also accounted for in the slowly varying modal amplitudes.
We thus may write:
\begin{subequations}\label{eq:apdx_field}
    \begin{eqnarray}
        \tilde{F}(z,t) &=&  \sum_m a_m(t) e^{-i\omega_m(t-\beta_1 z)}\\
        \tilde{B}(z,t) &=&  \sum_m a_m(t) e^{-i\omega_m(t+\beta_1 z)}.
    \end{eqnarray}
\end{subequations}
We consider a periodic input, synchronised with the cavity repetition rate,  which can be expanded in Fourier series as follows
\begin{equation}\label{eq:apdx_pump}
    E_{in}(t) = \sum_m S_m e^{-i\omega_m t}.
\end{equation}
We insert Eq.~(\ref{eq:apdx_field}) and Eq.~(\ref{eq:apdx_pump}) in (\ref{eq:NLS_nv})a, multiply by $e^{i\omega_n (t-\beta_1 z)}$ and integrate in $z \in [-L,L]$, to obtain :
\begin{align}\label{eq:modal}
    \nonumber &\beta_1 \dot{a_n} - \left(\frac{\ln(\rho_1\rho_2)}{2L}+i\frac{\phi_0}{2L}\right)a_n \\
    \nonumber &+ i\frac{\beta_2}{2} \left(\ddot{a}_n -2i \omega_n \dot{a}_n -\omega_n^2 a_n\right) - \frac{\theta_1}{2L} S_n = \\
    &i\gamma \sum_{n',n''} a_{n'}a_{n''}^* (a_{n-n'+n''} + G a_{n+n'-n''}e^{-2i(\omega_{n'}-\omega_{n''})t}).
\end{align}
We assume that the modal amplitudes change slowly over a roundtrip, i.e. $|\dot{a}_n| \ll |\omega_n a_n|$.
This assumption permits to simplify the dispersive contribution, by neglecting the time derivatives of the modal amplitudes in the third term of Eq.~(\ref{eq:modal}). Moreover, by integrating Eq.~(\ref{eq:modal}) in time over one roundtrip, and considering $a_n(t)$ constant in this range,  the fast oscillations in the second nonlinear term are averaged out. We eventually obtain~:
\begin{align}
    \nonumber \dot{a}_n + \left(\kappa -i\frac{\phi_0}{2\beta_1 L} -i\frac{\beta_2}{2\beta_1}\omega_n^2\right)a_n 
    - \frac{\theta_1}{2\beta_1 L} S_n = \\
    i\frac{\gamma}{\beta_1} \left( \sum_{n',n''}a_{n'}a_{n''}^*a_{n-n'+n''} + G a_n \sum_{n'} |a'_n|^2 \right).
\end{align}
The same equation is also obtained by following a similar procedure starting from (\ref{eq:NLS_nv})b.

\subsection{Mean field FP-LLE}
We may now define the slowly varying envelope of the forward and backward fields in the laboratory frame as
\begin{subequations}
    \begin{eqnarray}
        \psi(z,t) &=& \sum_m a_m(t) e^{-i\omega_m t}e^{i\beta_1 \omega_m z} \\
        \psi_B(z,t) &=& \sum_m a_m(t) e^{-i\omega_m t}e^{-i\beta_1 \omega_m z}.
    \end{eqnarray}
\end{subequations}
It is apparent that the fields are periodic in space of period $2L$ and they satisfy $\psi(z, t) = \psi_B(-z, t)$.
Thanks to this relation we can relate the fields in the ’nonphysical’ cavity $-L < z < 0$ to the real cavity $0 < z < L$ to their conter-propagating counterparts~\cite{cole2018theory}.
By using
\begin{eqnarray*}
    \frac{\partial \psi}{\partial t} &=& \sum_m (\Dot{a}_m(t) -i\omega_m a_m) e^{-i\omega_m t}e^{i\beta_1 \omega_m z} \;, \\
    \frac{\partial^n \psi}{\partial z^n} &=& \sum_m (i\beta_1 \omega_m)^n a_m(t) e^{-i\omega_m t}e^{i\beta_1 \omega_m z}\;,
\end{eqnarray*}
we easily get
\begin{align}\label{eq:FP_LLE_t1}
    \nonumber \frac{\partial \psi}{\partial t} &+ \frac{1}{\beta_1} \frac{\partial \psi}{\partial z} + \left(\kappa -i\frac{\phi_0}{2\beta_1 L}\right)\psi\\
    \nonumber &+i\frac{\beta_2}{2\beta_1^3}\frac{\partial^2\psi}{\partial z^2} - \frac{\theta_1}{2\beta_1 L}E_{in}(t-\beta_1 z) \\
    &= i\frac{\gamma}{\beta_1} \left( |\psi|^2 + \frac{G}{2L} \int_{-L}^L |\psi(z',t)|^2 dz'\right)\psi.
\end{align}
By means of the change of variable $z \rightarrow -z + t/\beta_1$ [mod $2L$] and multiplying by the roundtrip time $t_R = 2\beta_1L$ we get~:
\begin{align}\label{eq:FP_LLE_t11}
    \nonumber   t_R \frac{\partial \psi}{\partial t} &= -(\alpha-i\phi_0)\psi -2iL \frac{\beta_2}{2\beta_1^2}\frac{\partial^2 \psi}{\partial z^2} +\theta_1 E_{in}(\beta_1 z)\\
    &+2iL\gamma \left( |\psi|^2 + \frac{G}{2L} \int_{-L}^L |\psi(z',t)|^2 dz' \right)\psi,
\end{align}
where $\alpha = \kappa t_R=-\ln(\rho_1 \rho_2) \approx 1 - \rho_1\rho_2$.
This form of FP-LLE reduces to the one obtained by Cole \textit{et al}.~\cite{cole2018theory} for the case of CW pumping and identical mirrors.
Its structure is usual in the context of microresonators~\cite{chembo2013spatiotemporal}.
More precisely, the evolution is in time and the transverse dimension is the space with periodic boundary conditions.

In fiber ring resonators it is customary to have evolution in space (also called slow time) and a temporal transverse coordinate \cite{haelterman1992additive,coen2013modeling}.
The role of time and space can be swapped at first order if we consider that the most important effect is the translation at the group velocity~\cite{chabchoub2016hydrodynamic}.
Indeed, in (\ref{eq:FP_LLE_t1}) the first two terms are of order one, while the remaining ones are first order corrections.
This means that, at the lowest order, we have
\begin{equation}
    \frac{\partial \psi}{\partial z} \approx -\beta_1 \frac{\partial \psi}{\partial t}\;, \quad \text{and} \quad \frac{\partial^2 \psi}{\partial z^2} \approx \beta_1^2 \frac{\partial \psi}{\partial t^2} 
\end{equation}
By using the second of the relations above in Eq. (\ref{eq:FP_LLE_t1}) and making the change of variable $t \rightarrow t-\beta_1z$, we get the space propagated version of the FP-LLE.
\begin{align}\label{eq:FP_LLE_t2}
   \nonumber  2L \frac{\partial \psi}{\partial z} &= -(\alpha-i\phi_0)\psi -iL\beta_2 \frac{\partial^2 \psi}{\partial t^2} + \theta_1 E_{in} \\
   &+2i\gamma L \left( |\psi|^2 + \frac{G}{t_R} \int_{-t_R/2}^{t_R/2} |\psi(z,t')|^2 dt' \right)\psi,
\end{align}
where $z>0$ and $-t_R/2 <t<t_R/2$.
Even if Eq.~(\ref{eq:FP_LLE_t2}) and Eq.~(\ref{eq:FP_LLE_t1}) have the same degree of approximation, only the time-propagated version has the correct boundary conditions.
Indeed, in Eq.~(\ref{eq:FP_LLE_t2}) we have assumed that the field is periodic in time, which is not strictly true.
This also implies that the modes have a constant frequency spacing (free spectral range, FSR), while in reality the FSR changes slightly because of dispersion.
Conversely, in Eq.~(\ref{eq:FP_LLE_t11}) the modes have equally spaced wavenumbers, but their frequencies are fixed by the dispersion relation.
These facts are almost irrelevant in standard (i.e. 'long', tens of meters) fiber ring resonators, because the roundtrip time is usually much longer than the pulse circulating in the resonator.
This usually allows one to consider an infinite roundtrip time with constant boundary conditions.
The field is no more considered as periodic and its spectrum, which is now continuous, gives the envelope of the discrete-spectrum of the full optical field circulating in the cavity.

\section{Exact solution for square pulse pumping}\label{sec:pulses_exact}

In this section we report the explicit expressions of the cross-phase modulation terms Eqs.~(\ref{eq:x_int}), for  $t\in[0,t_R]$, given the periodicity of the functions.
The expressions are different depending if the duty-cycle $f_r$ of the square pulse is greater or lesser than $0.5$.
The fundamental period $[0,t_R]$ is divided into six intervals, where the functions Eqs.~(\ref{eq:x_int}) have different forms.
For each time interval, there exist three different spatial intervals where the functions (\ref{eq:x_int}) are different in general.
Tables \ref{tab:table1} and \ref{tab:table2} report the explicit expressions of Eqs.~(\ref{eq:x_int}) for $f_r>0.5$ and $f_r<0.5$.
\begin{widetext}
    
\begin{table*}[htb]
    \label{tab:table2}
    \begin{tabular}{|c|c|c|c|c|}
    \hline
    \multicolumn{4}{|c|}{$f_r<0.5$}\\ \hline
    Time interval & Space interval & $\phi_{XF}(z,t)$ & $\phi_{XB}(z,t)$\\ \hline
    \multirow{2}{*}{$\displaystyle 0<t<\frac{t_R}{2}f_r$} & $\displaystyle 0<z<\frac{t}{\beta_1}$ & $|B_0|^2z $ &  $|F_0|^2z $\\
        & $\displaystyle\frac{t}{\beta_1}<z<2f_rL-\frac{t}{\beta_1}$ &0 &$\displaystyle|F_0|^2\left(\frac{z}{2}+\frac{t}{2\beta_1}\right)$\\
        & $\displaystyle 2f_rL-\frac{t}{\beta_1}<z<L$ &0 &$0$\\ \hline
    \multirow{3}{*}{$\displaystyle \frac{t_R}{2}f_r<t<t_Rf_r$} & $\displaystyle 0<z<2f_rL-\frac{t}{\beta_1}$ & $|B_0|^2z $ &  $|F_0|^2z $\\
        & $\displaystyle 2f_rL-\frac{t}{\beta_1}<z<\frac{t}{\beta_1}$ & $\displaystyle|B_0|^2\left(\frac{z}{2}-\frac{t}{2\beta_1}+f_rL\right)$ & $0$\\
        & $\displaystyle \frac{t}{\beta_1}<z<L$ & $0$ & $0$\\ \hline
    \multirow{3}{*}{$\displaystyle t_Rf_r<t<\frac{t_R}{2}$} & $\displaystyle 0<z<\frac{t}{\beta_1}-2f_rL$ & $0$ &  $0$\\
        & $\displaystyle \frac{t}{\beta_1}-2f_rL<z<\frac{t}{\beta_1}$ & $\displaystyle|B_0|^2\left(\frac{z}{2}-\frac{t}{2\beta_1}+f_rL\right)$ &$0$\\
        & $\displaystyle \frac{t}{\beta_1}<z<L$ & $0$ & $0$\\\hline
    \multirow{3}{*}{$\displaystyle \frac{t_R}{2}<t<\frac{t_R}{2}(1+f_r)$} & $\displaystyle 0<z<\frac{t}{\beta_1}-2f_rL$ & $0$ &  $0$\\
        & $\displaystyle \frac{t}{\beta_1}-2f_rL<z<2L-\frac{t}{\beta_1}$ & $\displaystyle|B_0|^2\left(\frac{z}{2}-\frac{t}{2\beta_1}+f_rL\right)$ &$0$\\
        & $\displaystyle 2L-\frac{t}{\beta_1}<z<L$ & $|B_0|^2(z-L(1-f_r))$ & $|F_0|^2(z-L(1-f_r))$\\\hline
    \multirow{3}{*}{$\displaystyle \frac{t_R}{2}(1+f_r)<t<t_R\left(\frac{1}{2}+f_r\right)$} & $\displaystyle 0<z<2L-\frac{t}{\beta_1}$ & $0$ &  $0$\\
        & $\displaystyle 2L-\frac{t}{\beta_1}<z<\frac{t}{\beta_1}-2f_rL$ & $0$ &$\displaystyle|F_0|^2\left(\frac{z}{2}+\frac{t}{2\beta_1}-L\right)$\\
        & $\displaystyle \frac{t}{\beta_1}-2f_rL<z<L$ & $|B_0|^2(z-L(1-f_r))$ & $|F_0|^2(z-L(1-f_r))$\\\hline
    \multirow{3}{*}{$\displaystyle t_R\left(\frac{1}{2}+f_r\right)<t<t_R$} & $\displaystyle 0<z<2L-\frac{t}{\beta_1}$ & $0$ &  $0$\\
        & $\displaystyle 2L- \frac{t}{\beta_1}<z<2L(1+f_r)-\frac{t}{\beta_1}$ & $0$ &$\displaystyle|F_0|^2\left(\frac{z}{2}+\frac{t}{2\beta_1}-L\right)$\\
        & $\displaystyle 2L(1+f_r)-\frac{t}{\beta_1}<z<L$ & $0$ & $0$\\\hline
    \end{tabular}
    \caption{Cross-phase terms for $f_r<0.5$.}
\end{table*}

\begin{table*}
    \label{tab:table1}
    \begin{tabular}{|c|c|c|c|c|}
        \hline
        \multicolumn{4}{|c|}{$f_r>0.5$}\\ \hline
        Time interval & Space interval & $\phi_{XF}(z,t)$ & $\phi_{XB}(z,t)$\\ \hline
        \multirow{3}{*}{$\displaystyle 0<t<t_R\left(f_r-\frac{1}{2}\right)$}  & $\displaystyle 0<z<\frac{t}{\beta_1}$ & $|B_0|^2z $ & $|F_0|^2z $\\
            & $\displaystyle\frac{t}{\beta_1}<z<\frac{t}{\beta_1}+2L(1-f_r)$ &0 &$\displaystyle|F_0|^2\left(\frac{z}{2}+\frac{t}{2\beta_1}\right)$\\
            & $\displaystyle\frac{t}{\beta_1}+2L(1-f_r)<z<L$ & $|B_0|^2(z-L(1-f_r)) $ & $|F_0|^2(z-L(1-f_r))$\\ \hline
        \multirow{3}{*}{$\displaystyle t_R\left(f_r-\frac{1}{2}\right)<t<\frac{t_R}{2}f_r$} & $\displaystyle 0<z<\frac{t}{\beta_1}$ & $|B_0|^2z $ &  $|F_0|^2z $\\
            & $\displaystyle\frac{t}{\beta_1}<z<2f_rL-\frac{t}{\beta_1}$ &0 &$\displaystyle|F_0|^2\left(\frac{z}{2}+\frac{t}{2\beta_1}\right)$\\
            & $\displaystyle 2f_rL-\frac{t}{\beta_1}<z<L$ & $0$ & $0$\\ \hline
        \multirow{3}{*}{$\displaystyle \frac{t_R}{2}f_r<t<\frac{t_R}{2}$} & $\displaystyle 0<z<2f_rL-\frac{t}{\beta_1}$ & $|B_0|^2z $ &  $|F_0|^2z $\\
            & $\displaystyle 2f_rL-\frac{t}{\beta_1}<z<\frac{t}{\beta_1}$ & $\displaystyle|B_0|^2\left(\frac{z}{2}-\frac{t}{2\beta_1}+f_rL\right)$ &$0$\\
            & $\displaystyle \frac{t}{\beta_1}<z<L$ & $0$ & $0$\\ \hline
        \multirow{3}{*}{$\displaystyle \frac{t_R}{2}<t<t_Rf_r$} & $\displaystyle 0<z<2f_rL-\frac{t}{\beta_1}$ & $|B_0|^2z $ &  $|F_0|^2z $\\
            & $\displaystyle 2f_rL-\frac{t}{\beta_1}<z<2L-\frac{t}{\beta_1}$ & $\displaystyle|B_0|^2\left(\frac{z}{2}-\frac{t}{2\beta_1}+f_rL\right)$ &$0$\\
            & $\displaystyle 2f_rL-\frac{t}{\beta_1}<z<L$ & $|B_0|^2(z-L(1-f_r))$ & $|F_0|^2(z-L(1-f_r))$\\ \hline
        \multirow{3}{*}{$\displaystyle t_Rf_r<t<\frac{t_R}{2}(1+f_r)$} & $\displaystyle 0<z<\frac{t}{\beta_1}-2f_rL$ & $0$ &  $0$\\
            & $\displaystyle \frac{t}{\beta_1}-2f_rL<z<2L-\frac{t}{\beta_1}$ & $\displaystyle|B_0|^2\left(\frac{z}{2}-\frac{t}{2\beta_1}+f_rL\right)$ &$0$\\
            & $\displaystyle 2L-\frac{t}{\beta_1}<z<L$ & $|B_0|^2(z-L(1-f_r))$ & $|F_0|^2(z-L(1-f_r))$\\ \hline
        \multirow{3}{*}{$\displaystyle \frac{t_R}{2}(1+f_r)<t<t_R$} & $\displaystyle 0<z<2L-\frac{t}{\beta_1}$ & $0$ &  $0$\\
            & $\displaystyle 2L- \frac{t}{\beta_1}<z<\frac{t}{\beta_1}-2f_rL$ & $0$ &$\displaystyle|F_0|^2\left(\frac{z}{2}+\frac{t}{2\beta_1}-L\right)$\\
            & $\displaystyle \frac{t}{\beta_1}-2f_rL<z<L$ & $|B_0|^2(z-L(1-f_r))$ & $|F_0|^2(z-L(1-f_r))$\\ \hline
    \end{tabular}
    \caption{Cross-phase terms for $f_r>0.5$.}
\end{table*}

\end{widetext}
