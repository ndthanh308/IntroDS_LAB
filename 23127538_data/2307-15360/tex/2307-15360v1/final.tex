

\documentclass[%
 notitlepage,
 reprint,
 superscriptaddress,
%groupedaddress,
%unsortedaddress,
%runinaddress,
%frontmatterverbose, 
%preprint,
%showpacs,preprintnumbers,
%nofootinbib,
%nobibnotes,
%bibnotes,
 amsmath,amssymb,
 aps,
%pra,
%prb,
%rmp,
%prstab,
%prstper,
%floatfix,
]{revtex4-2}


\usepackage{lipsum}
%
%\usepackage{bbold}
%
\usepackage{graphicx}% Include figure files
\usepackage{dcolumn}% Align table columns on decimal point
\usepackage{bm}% bold math
%\usepackage{hyperref}% add hypertext capabilities
%\usepackage[mathlines]{lineno}% Enable numbering of text and display math
%\linenumbers\relax % Commence numbering lines

%\usepackage[showframe,%Uncomment any one of the following lines to test 
%%scale=0.7, marginratio={1:1, 2:3}, ignoreall,% default settings
%%text={7in,10in},centering,
%%margin=1.5in,
%%total={6.5in,8.75in}, top=1.2in, left=0.9in, includefoot,
%%height=10in,a5paper,hmargin={3cm,0.8in},
%]{geometry}
\usepackage{epstopdf}
\usepackage{xcolor}

\newcommand{\bra}[1]{\langle #1 \vert}
\newcommand{\ket}[1]{\vert #1 \rangle}
%\usepackage{epstopdf}



\begin{document}

\preprint{APS/123-QED}

\title{Flat-band quantum communication induced by disorder} 



\author{G. M. A. Almeida}
\email{gmaalmeida@fis.ufal.br}
\affiliation{%
 Instituto de F\'{i}sica, Universidade Federal de Alagoas, 57072-900 Macei\'{o}, AL, Brazil
}%

\author{R. F. Dutra}
\affiliation{%
 Instituto de F\'{i}sica, Universidade Federal de Alagoas, 57072-900 Macei\'{o}, AL, Brazil
}%

\author{A. M. C. Souza}
\affiliation{%
 Departamento de F\'{i}sica, Universidade Federal de Sergipe, 49100-000 S\~{a}o Crist\'{o}v\~{a}o, SE, Brazil
}%

\author{M. L. Lyra}
\affiliation{%
 Instituto de F\'{i}sica, Universidade Federal de Alagoas, 57072-900 Macei\'{o}, AL, Brazil
}%

\author{F. A. B. F. de Moura}
\affiliation{%
 Instituto de F\'{i}sica, Universidade Federal de Alagoas, 57072-900 Macei\'{o}, AL, Brazil
}%

\begin{abstract}


We show that a qubit transfer protocol can be realized through a flat band hosted by a disordered $XX$ spin-1/2 diamond chain. In the absence of disorder, the transmission becomes impossible due to the compact localized states forming the flat band. 
%
When off-diagonal disorder is considered, the degeneracy of the band is preserved but the associated states are no longer confined to the unit cells. 
%
By perturbatively coupling the sender and receiver to the flat band, we derive a general effective Hamiltonian resembling a star network model with two hubs.
%
The effective couplings correspond to wavefunctions associated with the flat-band modes.
%
Specific relationships between these parameters define the quality of the quantum-state transfer which, in turn, are related to the degree of localization in the flat band. 
%
Our findings establish a framework for further studies of flat bands in the context of quantum communication. 
\end{abstract}  


\maketitle


\section{Introduction}

Progress in 
quantum information processing have led us to the so-called noisy-intermediate-scale quantum era \cite{preskill18}. This means that the technology for assembling dozens of qubits to perform simple and proof-of-principle tasks is available \cite{arute19}. Yet, practical issues persist such as
faulty quantum gates and limited control over the processing units. 
A prominent
source of errors, besides decoherence, comes from the manufacturing process of quantum devices. For example, the parameters of a qubit network -- such as coupling strengths and local transition energies -- might deviate
from their original design, culminating in disorder.
%
This can consequently lead to Anderson localization of quantum information \cite{burrell07, allcock09}.
Since disorder cannot be fully suppressed, 
it is important to consider its influence on the quantum dynamics.
%
%So addressing matters of large-scale quantum computation, disorder effects should not be put aside. 
%It is therefore of great importance to assess the performance of quantum information protocols against static flucutuations.

Quantum-state transfer (QST) and entanglement distribution are essential tasks to be performed on quantum networks
\cite{kimble08}.
A quantum communication channel can be set by a collection of spin-1/2 particles acting as qubits and linked via engineered exchange interactions. 
An arbitrary qubit state prepared at one end
of a 1D chain can be transmitted to the other end by the unitary evolution of the Hamiltonian. This idea was introduced by Bose in Ref. \cite{bose03} and many other schemes have since been proposed \cite{christandl04, plenio04, osborne04, wojcik05, wojcik07, li05, huo08, liu08, gualdi08, wang09,  banchi10, apollaro12, lorenzo13, paganelli13, lorenzo15, almeida16, almeida18, apollaro19}. 

A particular class of spin chains for QST relies on a complete engineering of their couplings. 
This approach results in a linear spectrum that supports end-to-end perfect state transfer 
in arrays of any size \cite{christandl04, plenio04}. 
High-fidelity QST protocols can also be designed 
with lower engineering costs by tuning the boundaries of a homogeneous spin chain \cite{wojcik05, wojcik07, banchi10}.
Other methods involve the application of strong local magnetic fields in order to effectively decouple the sender and receiver spins from the channel \cite{lorenzo13}. 
%A similar approach can taken by setting very weak outer couplings \cite{wojcik05,wojcik07}.

Under the influence of disorder, the overall performance 
of any QST scheme is expected to be reduced
to some degree. \cite{dechiara05,fitzsimons05,burgarth05,tsomokos07,giampaolo10,yao11,zwick12,zwick15, bruderer12, ashhab15, kay16,estarellas17}.
%
%based on speed-fidelity requirements as well as on the profile of
%the fluctuations being considered  that their
% It was shown in \cite{zwick12,zwick15} fidelity decays $\propto e^{-cNW^{\beta}}$, where factor $c$ and exponent $\beta$ (of the order of $2$) depend on the spin chain configuration, and $W$ is the disorder strength. 
%
Chains featuring modified boundary couplings 
are more robust against static noise \cite{zwick12, almeida18}. When the communicating parties are perturbatively coupled to the channel, the conditions for 
an end-to-end effective interaction become more flexible because 
most of the channel modes barely interfere.
As such, it is possible to tune the end spins in a way that shields the dynamics from the influence of the strongly localized states \cite{almeida18pla,almeida19qinp}.
%In the scenario where the outer spins are perturbatively coupled to the channel and their frequencies do not match any its normal modes, they end up setting their own subspace with renormalized parameters.
%Such Rabi-like QST protocol has been shown to be very resilient to various kinds of correlated disorder, given the communicating parties can only ``feel'' the channel spectrum locally \cite{almeida18}. 
%
%High-fidelity QST can thus be realized even at moderate levels of disorder \cite{almeida18pla,almeida19qinp}.
%
Alternatively, one can harness topological protection against disorder \cite{almeida16, estarellas17}.

In this paper, we go beyond the usual 1D schemes to explore a QST protocol on 
a diamond-like spin-1/2 chain [see Fig. \ref{fig1}] described by the $XX$ model.
Disordered quasi-1D materials, such as nanowires, have been shown to display peculiar strongly correlated phenomena \cite{petrovic16,gligori20, roy20}. 
%
Many quasi-1D networks are known to support flat bands \cite{hyrkas13,flach14,maimaiti17}. These are dispersionless Bloch bands
hosting macroscopic degeneracy, diverging density of states, zero-group velocity, and infinite effective mass \cite{derzhko15rev}.
A rich variety of transport regimes, including exotic Anderson transitions \cite{goda06,chalker10}, can emerge,
specially when the system is under the influence of perturbations that slightly lift the degeneracy \cite{souza09,vicencio13,leykam13,ramachandran17,khomeriki16,roy20,bouzerar21}. 
%

Here we consider a pair of communicating spins weakly coupled to a diamond channel that hosts a flat band.
%and tuned to the flat band so that their effective, end-to-end interaction is only mediated by the whole degenerate level. 
In the ordered case, 
the band hosts a set of compact localized states, each restricted to one unit cell of the diamond lattice \cite{flach14,maimaiti17}. 
When off-diagonal disorder is present,
the flat band is preserved but 
is formed by a distinct set of eigenstates. These modes can be extended and thus mediate QST between the end cells.
Remarkably, we observe that
the competition between the (topological) compact localization and Anderson localization benefits the QST. Furthermore, an effective model is derived to explain that property. By solving it analytically, we highlight 
the key ingredients responsible for achieving better fidelities.
% 
Our findings can be readily extended to other flat band bipartite lattices. 

%
%one of the downsides being
%the transfer time, that scales with $O(g^{-2})$, $g$ being the perturbative coupling  \cite{wojcik05,wojcik07}.
%
%the communicating parties can only ``feel'' the channel spectrum locally, what gives room for bypassing localized wavefunctions. 
 
\section{Model and flat-band structure}


% Figure environment removed

We consider a quantum channel that consists of $3N$ spin-1/2 particles arranged in a diamond-like configuration with open boundary conditions as shown in Fig. \ref{fig1}. They interact through a $XX$ Hamiltonian of the form ($\hbar = 1$) \begin{equation}
   H_{\mathrm{ch}} = \frac{1}{2}\sum_{\left< i, j \right>} J_{ij} (\hat{\sigma}_{i}^x\hat{\sigma}_{j}^x + \hat{\sigma}_{i}^y\hat{\sigma}_{j}^y) ,
\end{equation}
where $\hat{\sigma}_{i}^{x,y}$ 
are the usual Pauli operators for spin $i$ and
$J_{ij}$ is the nearest-neighbour interaction strength between spins $i$ and $j$. 
%
Herein, it is more convenient to visualize the channel as being composed of $N$ coupled vertical trimer cells. 
In the single-excitation sector, the Hamiltonian reads
\begin{align}
H_{\mathrm{ch}} &= \sum_{n=1}^{N} (J_{1,n}\ket{a_n}\bra{b_n}+J_{2,n}\ket{b_n}\bra{c_n}) \nonumber \\  
&\,\,\,\, +\sum_{n=1}^{N-1}(J'_{1,n}\ket{a_n}\bra{b_{n+1}}+J'_{2,n}\ket{c_n}\bra{b_{n+1}})+h.c.,
\label{Hch}
\end{align}
where $\ket{\ell_{n}}$ denotes a single spin flipped in the $n$-th cell at leg $\ell \in \lbrace a,b,c \rbrace$.

Let us now see how the flat band emerges.
%We now evaluate the spectrum of the channel and see how it supports a flat band.
Each cell contributes with the local eigenstates 
$\ket{v_0^{(n)}} = \frac{1}{\lambda_n}(J_{2,n},0,-J_{1,n})$ and $\ket{v_{\pm 1}^{(n)}} =\frac{1}{\lambda_n\sqrt{2}} (J_{1,n},\pm \lambda_n,J_{2,n})$, with corresponding eigenvalues $0$ and $\pm \lambda_n$, where $\lambda_n = \sqrt{J_{1,n}^2+J_{2,n}^2}$.
%
%Due to the coupling between the cells, 
%these states overlap to form the structures seen in Figs. \ref{fig1}(b) and \ref{fig1}(c). 
Within this basis set, transitions between different cells are given by:
\begin{align}
\bra{v_{\nu}^{(n+1)}}H_{\mathrm{ch}}\ket{v_{\pm 1}^{(n)}}&=\frac{\nu}{2\lambda_n}\left( J_{1,n}J'_{1,n}+J_{2,n}J'_{2,n} \right),\\
\bra{v_{\nu}^{(n+1)}}H_{\mathrm{ch}}\ket{v_{0}^{(n)}}&=\frac{\nu}{\sqrt{2}\lambda_n}\left( J_{2,n}J'_{1,n}-J_{1,n}J'_{2,n} \right),\label{transition}\\
\bra{v_{0}^{(n+1)}}H_{\mathrm{ch}}\ket{v_{0}^{(n)}}&=0,
\end{align}
where $\nu=\pm 1$. 

In the ordered case ($J_{i,n} = J'_{i,n} = J$) a quick look at the expressions above tells us that $\ket{v_0^{(n)}}$ are eigenstates of $H_{\mathrm{ch}}$ with the same energy $E=0$.
Indeed, they form a complete orthogonal basis at the center of the band. Given each $\ket{v_0^{(n)}}$ is spatially confined to the $n$-th unit cell, 
they are classified as compact localized states \cite{flach14,maimaiti17}. 
%
Therefore, a diamond channel with $N$ cells hosts a $N$-fold degenerate flat band at $E=0$.   

%The remaining states $\ket{v_{\pm 1}}$ are coupled between adjacent cells resulting in the graph structure depicted in Fig. \ref{fig2}(a). 

%each trimer contributes with \textit{local} eigenvectors of the form $\ket{v_0} = (1/\sqrt{2},0,-1/\sqrt{2})^{T}$, $\ket{v_{\pm 1}} = (1/2,\pm 1/\sqrt{2},1/2)^{T}$ with corresponding eigenvalues $E_{0} = 0$, $E_{\pm 1}= \pm \sqrt{2}J$. 
 
%
%
%Defining $a_{k,i}^{(l)}=\langle i+3(l-1) \vert v_k^{(l)} \rangle$ ($i=1,2,3$ and $k=0,\pm 1$) for the l-$th$ cell, we obtain $H_{\mathrm{ch}}\ket{v_{0}^{(l)}} = \sum_k\mu_k^{(l)}\ket{v_k^{(l+1)}}$, where $\mu_k^{(l)} = J a_{k,2}^{(l+1)} [a_{0,1}^{(l)}+a_{0,3}^{(l)}]$. It is immediate to note that $a_{0,2}^{(l)} = 0$ for every cell (it also holds in the disordered case) and $a_{0,1}^{(l)} = - a_{0,3}^{(l)}$. Then, $H_{\mathrm{ch}}\ket{v_{0}^{(l)}} = 0$ indeed.

The ordered flat band cannot 
mediate a resonant QST \cite{wojcik07} between two external spins weakly coupled to the outermost cells (see Fig. \ref{fig1}).  
%
The compact localized states $\ket{v_{0}^{(n)}}$ forbid excitation transport between any pair of cells. 
%
This scenario changes, however, when $\ket{v_{0}^{(n)}}$ are no longer eigenstates of $H_{\mathrm{ch}}$. Any disorder in the channel will promote transitions between those and the dispersive modes $\ket{v_{\nu}^{(n)}}$ [Eq. (\ref{transition})]. 
%This translates into coupling those states to the dispersive ones [see Fig. \ref{fig1}(c)] thereby destroying their compact localization. 
In this work, instead of devising an engineering scheme for $H_{\mathrm{ch}}$ we will 
see how random fluctuations in the spin-spin couplings 
can assist a QST protocol.

% disordered diamonds and the case of bipartite lattices
Flat-band diamond lattices 
have been studied in the presence of diagonal as well of off-diagonal disorder \cite{leykam13, leykam17, roy20}. 
In the weak disorder regime, a general result is that the mixing between flat-band states and the others leads to a scaling of the localization length of the form $\xi \sim W^{-\gamma}$ at the flat band, with $W$ being the disorder width and the exponent $\gamma$ 
depending on the flat band class \cite{leykam13, leykam17}. 
%

One property
that deserves particular attention here is the bipartite nature of the diamond lattice. This means that we can
group the $N$ states $\lbrace \ket{b_n} \rbrace$ into one 
sublattice
and the remaining
$2N$ states $\lbrace \ket{a_n},\ket{c_n} \rbrace$ into another.
%
A known theorem \cite{sutherland86, inui94} states that bipartite lattices featuring only off-diagonal disorder sustains at least $M$ linearly independent states at $E=0$, where $M$ is the
difference between the number of sites 
of both sublattices.  
%
In addition, these $M$ states have no amplitude on the minority sublattice. For the diamond lattice, $M=N$.
%

The flat band is thereby preserved if we set
$J_{i,n}\rightarrow J_{i,n}(1+\delta_{i,n})$ and $J'_{i,n}\rightarrow J'_{i,n}(1+\delta'_{i,n})$, where  
$\delta_{i,n},\delta'_{i,n}$ are uncorrelated random numbers uniformly distributed in $[-W/2,W/2]$. However, note that while the 
$N$-fold degeneracy is maintained at $E=0$,
its corresponding modes are no longer
$\ket{v_0^{(n)}}$. Instead, we get another set of flat-band eigenstates 
$\ket{E_{FB,k}}$ which involves linear 
combinations of $\ket{v_0^{(n)}}$ and $\ket{v_\nu^{(n)}}$ but still 
have no amplitude on $\ket{b_n}$ \cite{sutherland86, inui94}.
%
Note that as our lattice is finite, there will always be one compact localized state left for the end cell, $\ket{v_0^{(N)}}$. Its contribution is negligible for our purposes. 
In the following section we show how the disordered flat band can mediate a QST protocol between the outermost cells.    

\section{Results}

\subsection{Effective Hamiltonian}

Let us now add two extra spins to the diamond channel, one at each end, to play the role of sender ($S$) and receiver ($R$). As shown in Fig. \ref{fig1}, they
are coupled to the sites $\ket{a_1}$ and $\ket{a_N}$, respectively. The full Hamiltonian of the system is now given by $H_{\mathrm{ch}}+H_{I}$, with the interaction Hamiltonian
\begin{equation}
H_I = g(\ket{S}\bra{a_1}+\ket{R}\bra{a_N}+ h.c.).
\end{equation}
Here we assume that $g$ is much smaller than the gap between the flat band and the 
non-zero energy states. 
%
The dispersion relation of the delocalized
states for an infinite ordered diamond lattice
reads $E(k)=\pm 2J\sqrt{1+\cos k}$, with $k$ being the typical wavenumber \cite{leykam13}. Then, the energy of the next non-zero energy states decreases as $\sim N^{-1}$ and so $g\ll J N^{-1}$. 
%
This ensures that only the flat-band modes will contribute to the QST dynamics. 
%

The perturbative coupling set by $g$ delivers a first-order resonant interaction involving $\ket{S}$, $\ket{E_{FB,k}}$ ($k=1,\ldots,N$), and $\ket{R}$. 
By generalizing the framework of single-mode resonant QST proposed in \cite{wojcik07}, we obtain the effective Hamiltonian
\begin{equation} \label{Heff}
H_{\mathrm{eff}} = g\sum_{k=1}^{N}(\mu_{1,k}\ket{S}\bra{E_{FB,k}}+\mu_{N,k}\ket{R}\bra{E_{FB,k}}+ h.c.),
\end{equation}
where the wavefunctions $\mu_{n,k} =\langle a_n \vert E_{FB,k} \rangle$ are now effective couplings.
Note that 
the states $\ket{c_1}$ and $\ket{c_N}$ can also host
the communicating parties without loss of generality. The states $\ket{b_n}$ have null amplitudes in the flat band and therefore are not suited for the QST protocol.

%
In matrix form, the Hamiltonian above reads
\begin{equation}\label{starnet}
H_{\mathrm{eff}}=
g\begin{pmatrix}
0 & 0 & \mu_{1,1} & \mu_{1,2} & \cdots & \mu_{1,N} \\
0 & 0 & \mu_{N,1} & \mu_{N,2} & \cdots & \mu_{N,N} \\
\mu_{1,1} & \mu_{N,1} & 0 & 0 &  & \vdots \\
\mu_{1,2} & \mu_{N,2} & 0 & 0 &   &  \\
\vdots & \vdots &  &  & \ddots &   \\
\mu_{1,N} & \mu_{N,N} & \cdots &  &  & 0
\end{pmatrix}.
\end{equation}
%
The effective QST model can be seen as a disordered star network with two hubs. 
%
We remark that the disorder featuring in the parameters
$\mu_{n,k}$ traces back to the fluctuations in $J_{i,n}$ and $J'_{i,n}$, with the constraint that $\eta_n = \sum_{k}|\mu_{n,k}|^2 \leq 1$.
%
%An ordered ($W=0$) diamond channel gives $\eta_1=\eta_N=1/2$ as $\mu_{n,k}=\langle a_n\ket{v_{0}^{(k)}}=1/\sqrt{2}$ provided $n=k$. 
The presence of disorder generally lead to 
$\eta_1\neq\eta_N$. We will see shortly that this population imbalance and the degree of localization of the flat band modes dictate the quality of the QST.
 
%If equality holds, it means that wavefunctions carrying
%some amplitude at sites $a_{n}$ all belong to the flat band. 

\subsection{Quantum-state transfer via flat bands}

We now analyze the transfer of an arbitrary qubit state $\ket{\psi} = \alpha \ket{0_S}+\beta\ket{1_S}$ prepared at site $S$ (see Fig. \ref{fig1}). The remaining spins are set in the ferromagnetic ground state such that $\ket{\Psi(t=0)}=\ket{\psi}\ket{0_{1}0_{2}\cdots 0_{3N}0_{R}}$. To evaluate the QST performance at site $R$, we compute the input-averaged (over all $\alpha$ and $\beta$) transfer fidelity \cite{bose03}:
\begin{equation}\label{avF}
F(t) = \frac{1}{2}+\frac{|f_{R}(t)|}{3}+\frac{|f_{R}(t)|^2}{6}.
\end{equation}
Hence, it suffices to track the evolution of the transition amplitude $f_{R}(t) = \langle R \vert U(t) \vert S \rangle$ over time, with $U(t) = e^{-iHt}$ being the quantum time evolution operator. 
%

We now turn our attention back to the effective two-hub star Hamiltonian [Eq. (\ref{starnet})] to obtain an expression for $f_{R}(t)$. Note that it embodies \textit{another} bipartite network featuring $N$ nodes in one sublattice (the flat band itself) and two nodes ($\ket{S}$ and $\ket{R}$) in the other. As such, 
by symmetry arguments \cite{sutherland86, inui94}, the spectrum 
is composed of the set of eigenvalues $\lbrace \pm\epsilon_1,\pm\epsilon_2, \lbrace 0 \rbrace_{N-2}  \rbrace$. The $N-2$ states at level $E=0$
do not have amplitude on the minority sublattice and thus will not contribute to the QST protocol.
The remaining four eigenstates can be written as
\begin{align}
\ket{\epsilon_{1}^{\pm}}&=\frac{1}{\sqrt{2}}\left( x_S\ket{S}+x_R\ket{R} \right) \pm \frac{1}{\sqrt{2}}\ket{\phi_1},\\
\ket{\epsilon_{2}^{\pm}}&=\frac{1}{\sqrt{2}}\left( y_S\ket{S}+y_R\ket{R} \right) \pm \frac{1}{\sqrt{2}}\ket{\phi_2},
\end{align}
where $\ket{\phi_i}$ are linear combinations of the states $\ket{E_{FB,k}}$, satisfying $\langle \phi_1 \vert \phi_2 \rangle = 0$. In each of those eigenstates 
the probability to find the excitation in either of the sublattices is 1/2. This is another remarkable symmetry property of bipartite networks \cite{souza17}. As $\lbrace \ket{\epsilon_{1}^{\pm}},\ket{\epsilon_{2}^{\pm}} \rbrace$ must be an orthonormal set, we also have that $y_S=-x_R^{*}$ and $y_R=x_S^{*}$, with $|x_R|^2+|x_S|^2=1$.

Now, expanding $U(t) = e^{-iHt}$ we obtain
\begin{align}
f_{R}(t)&=x_S^{*}x_R(\cos\epsilon_1 t-\cos\epsilon_2 t)\nonumber\\
&=-2x_S^{*}x_R \left[ \sin\left(\frac{\epsilon_1+\epsilon_2}{2}t\right)\sin\left(\frac{\epsilon_1-\epsilon_2}{2}t\right) \right].
\label{effFR}
\end{align}
The primary QST timescale will be dictated
by the slower sine function that depends on the gap $\delta \epsilon = \epsilon_1-\epsilon_2$.
The transition amplitude
$|f_{R}(t)|$ reaches its maximum at times 
$\tau=m\pi/\delta \epsilon$, where $m=n\delta\epsilon/(\epsilon_1 +\epsilon_2)$, with $n$ and $m$ being odd integers. (Strictly, $m$ is an integer only if  $\delta\epsilon/(\epsilon_1 +\epsilon_2)$ is rational.)
%
Given the QST time $\tau \propto g^{-1}$ and recalling that
$g\ll J N^{-1}$, in order to validate the effective Hamiltonian, then $\tau$ is typically larger than $O(N)$.

The maximum amplitude that $f_{R}(t)$ can achieve is related to a correlation parameter $C_{S,R}=2|x_S^{*}x_R|=4|\langle \epsilon_{i}^{\pm} \ket{S}\bra{R} \epsilon_{i}^{\pm} \rangle|$, which ranges from 0 to 1. It can thus be used to assess the quality of the QST.
%
Figure \ref{fig2} depicts the time evolution 
of the transition amplitude during a QST cycle as obtained by exact diagonalization of the full Hamiltonian for a single disorder sample. The wave envelope given by $C_{S,R}|\sin (\delta \epsilon t/2)|$
is also plotted for comparison.

% Figure environment removed

A perfect QST [within the effective framework of Eq. (\ref{starnet})], with $F(\tau)=1$ [Eq. (\ref{avF})], can only be achieved provided $C_{S,R}=1$, which implies $|x_{S}|=|x_{R}|=1/\sqrt{2}$.
With regard to the effective couplings $\mu_{n,k}$, a particular condition must be fulfilled. 
To see this, we can solve the eigenvalue equation $H_{\mathrm{eff}}\ket{\epsilon_{i}^{\pm}}=\pm\epsilon_{i}\ket{\epsilon_{i}^{\pm}}$ analytically to obtain:
\begin{align}
\epsilon_{1,2}&=g\left[\frac{1}{2}\left(A\pm\sqrt{A^2-4B}\right)\right]^{1/2},\\
x_{S}&=\left( 1-\frac{\Lambda^2}{(\tilde{\epsilon_i}^2-\eta_N)^2+\Lambda^2} \right)^{1/2},\\
x_{R}&=\frac{\sqrt{2}\Lambda}{\tilde{\epsilon_i}^2-\eta_N} x_{S},
\end{align}
where $A=\eta_1+\eta_N$, $B=\eta_1\eta_N-\Lambda^2$, $\Lambda=\sum_{k}(\mu_{1,k}\mu_{N,k})$, and $\tilde{\epsilon_{i}}=\epsilon_{i}/g$. Hence, the correlation parameter can be written as
$C_{S,R}=2/\sqrt{4+\Delta^2}$, where $\Delta=(\eta_1-\eta_N)/\Lambda$. We immediately see that $C_{S,R}=1$ whenever $\eta_1=\eta_N$ (as long as $\Lambda \neq 0$). 
%
In such a case, the fluctuations in the parameters $\mu_{n,k}$ are irrelevant. 
This is remarkable from the standpoint of the effective Hamiltonian in Eq. (\ref{starnet}). It means that 
in principle one can realize an almost perfect QST despite any level 
of disorder in $\mu_{n,k}$ by tuning a \textit{single} parameter.

% Figure environment removed

Here, we cannot manipulate $\mu_{n,k}$ directly, though. These parameters are attached to the flat-band modes of the physical lattice. Our goal now is to harness the randomness present in the exchange couplings $J_{i,n}$ and $J'_{i,n}$. 
%
By doing so, we generally obtain $\eta_1 \neq \eta_N$ and then
$\Delta \rightarrow 0$ is required to attain
$C_{S,R}\rightarrow 1$. 
Note that $\Delta$ is inversely proportional to the parameter 
$|\Lambda|=|\sum_{k}(\mu_{1,k}\mu_{N,k})|$. The latter scans 
the whole flat band modes via their amplitudes on states $\ket{a_1}$ and $\ket{a_N}$ (the ones to which spins $S$ and $R$ are coupled). In a disordered system, we expect that the product of two wavefunctions such as $\mu_{1,k}\mu_{N,k}$ be typically close to zero, especially for spins
residing at distant locations.
%

In Fig. \ref{fig3} we show probability density functions (PDFs) 
of $|\Lambda|$ for $N=20,40$ cells and selected values of the disorder width $W$. Indeed,
the distributions become more peaked around zero as $W$ grows. This trend is more severe for 
larger system sizes [Fig. \ref{fig3}(b)]. Despite
the correlation $C_{S,R}$ is a function of the ratio $\Delta=(\eta_1-\eta_N)/\Lambda$, the factor $\Lambda$ is the one that counts as far as $N$ is concerned. This can be observed from the fact that $\eta_n$ is a local quantity.   

% Figure environment removed

Things are different, however, as $W\rightarrow 0$. In this regime, we observe the opposite behavior for $\Lambda$. The increasing of $W$ actually benefits the formation of $C_{S,R}$.
The PDF for very low disorder widths is shown in Fig. \ref{fig4}. Such a reverse trend can be understood as a reminiscent influence of the compact localized states, to which $\Lambda=0$, as we depart from $W=0$. 
As the disorder increases, the flat-band modes become strongly localized again, but are no longer restricted to each cell. Instead, their localization length scales as $\xi \sim W^{-\gamma}$ \cite{leykam13,leykam17}.
It is between those two regimes that the QST will occur.  


%In the following section, we will see how the QST scheme responds to that.   

\subsection{Protocol performance}

% Figure environment removed

Now that all the relevant quantities that govern the speed and quality of the QST have been presented, 
we are ready to  
the test its performance against $W$.
%

Any disorder induces fluctuations in the energy spectrum and affects the transfer time $\tau \propto \delta\epsilon^{-1}$. In Fig. \ref{fig5} we show
how the gap $\delta\epsilon = \epsilon_1-\epsilon_2$ (in units of $g$) responds to $W$. It grows roughly linear with $W$, not being affected by the number of cells $N$. 
%
Another caveat to the limit $W\rightarrow 0$ is that a vanishing gap 
implies in a extremely slow QST, which is not a desirable feature. Therefore, disorder is needed to bypass the compact localized states of the flat band and also 
to make the QST faster. 

Because the fluctuations in the gap increase with $W$, we track the QST fidelity over a given time window (instead of a specific time).  
%
 Let us define $F_{\mathrm{max}} = \mathrm{max}\lbrace F(t) \rbrace$ as the maximum fidelity achieved for $t \in [0,t_{\mathrm{max}}]$, where $t_{\mathrm{max}}=20\pi/g$. Given $\tau = m \pi/\delta \epsilon$, we remark that
 the value of $t_{\mathrm{max}}$ corresponds to $m=1$ and $\delta\epsilon / g =0.05$. In this way, 
 the QST fidelity for disorder widths slightly below $W\approx 0.2J$ (cf. Fig. \ref{fig5}) may be underestimated. 
 %
 For $W>0.2J$, $t_{\mathrm{max}}$ is enough for the first QST cycle to occur [see Eq. (\ref{effFR})].  

% Figure environment removed

% Figure environment removed

%
The results for the QST fidelity are displayed in Fig. \ref{fig6}(a) considering $N=10$ and $N=20$ cells and fixed $g=0.01J$. 
Note that this value of $g$ is 
compatible with the system sizes considered.  
%
All the elements discussed so far are manifested 
through the fidelity performances. Indeed, 
the overall QST quality declines in 
the larger system size. This have been predicted 
in the analysis of the parameter $\Lambda$ (Fig. \ref{fig3}).
Yet, it is possible to reach fidelities above the classical threshold of $2/3$ \cite{bose03,horodecki99} at intermediate disorder levels.
%
Figure \ref{fig7} 
displays histograms of the maximum fidelities for $N=10$ and some selected values of $W$.

We remark that the 
poor performances associated to the 
lower values of $W$ in Fig. \ref{fig6}(a)
is a consequence of the chosen time interval. 
%
To confirm this, Fig. \ref{fig6}(b) shows the behavior of correlation $C_{S,R}$ against $W$. As we have seen, $C_{S,R}$ ultimately determines the quality of the QST.
Therefore, in this case higher fidelities \textit{can} be achieved at times $t>t_{\mathrm{max}}$. But if we were to consider disorder levels $W\ll 0.001J$ [inset of Fig. \ref{fig6}(b)], then
the QST would be unfeasible because of the reverse localization trend discussed earlier (see Fig \ref{fig4}).  
%
Indeed, $C_{S,R}$ must vanish as $W\rightarrow 0$ so as to conform with the development of the compact localized states. 

%To conclude, one should note that the effective Hamiltonian given by Eq. (\ref{starnet}) applies to any situation involving a pair of nodes in strict resonance with a flat band. The effective couplings $\mu_{n,k}$ are naturally set according to the underlying tight-binding network. As such, the model embodies rich physics on its own. Indeed, the quantity $C_{S,R}$ can serve as a localization measure for the flat band. As seen in Fig. \ref{fig6}(b), it manifests the transition from compact (at $W=0$) to an Anderson-type localization \cite{leykam13,leykam17}.

\section{Concluding remarks}


We studied a resonant QST through a flat band hosted by a disordered diamond lattice. In particular, off-diagonal disorder was considered 
for preserving the flat band due to the bipartite topology of the lattice.
%
Our findings revealed that the QST protocol
yields
good fidelities when a certain amount of disorder is present. 
%
The underlying phenomenon is a transition
of the flat-band modes from compact localization, when $W=0$, to Anderson localization as $W$ increases \cite{leykam13,leykam17}. For intermediate levels of disorder, those modes can 
jointly sustain significant amplitudes on distant cells.
%
 While we considered a simple uniform distribution for the disorder, similar results are obtained for uncorrelated Gaussian-distributed disorder (which is a more realistic situation).  
 
We also derived and solved an effective Hamiltonian [Eq. (\ref{starnet})] that applies for a pair of sites perturbatively coupled to any flat-band bipartite lattice. The model resembles a star network with two hubs and disordered couplings associated to the flat-band wavefunctions.
By deriving analytical expressions for the relevant eigenstates, we were able to 
identify the parameters that control the QST. 
%
Interestingly, if $\eta_1 = \eta_N$ then an almost perfect QST can occur for very small $g$ despite the level of fluctuations associated to the effective couplings $\mu_{n,k}$. 
%

The effective model we addressed is a powerful tool to study quantum transport via flat bands.  
It may be explored on its own as a synthetic network aiming for the remarkable relationship between the couplings. 
%
%The framework is thus established to apply in other classes of flat bands \cite{flach14,leykam17}. 
%An interesting variation of the diamond lattice, for instance, involves transverse couplings between sites $\ket{a_n}$ and $\ket{c_n}$. This parameter can be used to move the flat band along the spectrum \cite{leykam17}. 
We hope that our results encourage further research on quantum communication in other classes of flat bands \cite{flach14,leykam17}.

%Flat bands in general are known to display some degree of robustness against disorder and other unique features. As such, they can be appealing resources for quantum information processing.

\section{Acknowledgments}

This work is supported by CNPq,
CAPES, FINEP, CNPq-Rede Nanobioestruturas, 
and FAPEAL. 

%\bibliography{refs}

%apsrev4-2.bst 2019-01-14 (MD) hand-edited version of apsrev4-1.bst
%Control: key (0)
%Control: author (8) initials jnrlst
%Control: editor formatted (1) identically to author
%Control: production of article title (0) allowed
%Control: page (0) single
%Control: year (1) truncated
%Control: production of eprint (0) enabled
\begin{thebibliography}{58}%
\makeatletter
\providecommand \@ifxundefined [1]{%
 \@ifx{#1\undefined}
}%
\providecommand \@ifnum [1]{%
 \ifnum #1\expandafter \@firstoftwo
 \else \expandafter \@secondoftwo
 \fi
}%
\providecommand \@ifx [1]{%
 \ifx #1\expandafter \@firstoftwo
 \else \expandafter \@secondoftwo
 \fi
}%
\providecommand \natexlab [1]{#1}%
\providecommand \enquote  [1]{``#1''}%
\providecommand \bibnamefont  [1]{#1}%
\providecommand \bibfnamefont [1]{#1}%
\providecommand \citenamefont [1]{#1}%
\providecommand \href@noop [0]{\@secondoftwo}%
\providecommand \href [0]{\begingroup \@sanitize@url \@href}%
\providecommand \@href[1]{\@@startlink{#1}\@@href}%
\providecommand \@@href[1]{\endgroup#1\@@endlink}%
\providecommand \@sanitize@url [0]{\catcode `\\12\catcode `\$12\catcode
  `\&12\catcode `\#12\catcode `\^12\catcode `\_12\catcode `\%12\relax}%
\providecommand \@@startlink[1]{}%
\providecommand \@@endlink[0]{}%
\providecommand \url  [0]{\begingroup\@sanitize@url \@url }%
\providecommand \@url [1]{\endgroup\@href {#1}{\urlprefix }}%
\providecommand \urlprefix  [0]{URL }%
\providecommand \Eprint [0]{\href }%
\providecommand \doibase [0]{https://doi.org/}%
\providecommand \selectlanguage [0]{\@gobble}%
\providecommand \bibinfo  [0]{\@secondoftwo}%
\providecommand \bibfield  [0]{\@secondoftwo}%
\providecommand \translation [1]{[#1]}%
\providecommand \BibitemOpen [0]{}%
\providecommand \bibitemStop [0]{}%
\providecommand \bibitemNoStop [0]{.\EOS\space}%
\providecommand \EOS [0]{\spacefactor3000\relax}%
\providecommand \BibitemShut  [1]{\csname bibitem#1\endcsname}%
\let\auto@bib@innerbib\@empty
%</preamble>
\bibitem [{\citenamefont {Preskill}(2018)}]{preskill18}%
  \BibitemOpen
  \bibfield  {author} {\bibinfo {author} {\bibfnamefont {J.}~\bibnamefont
  {Preskill}},\ }\bibfield  {title} {\bibinfo {title} {Quantum {C}omputing in
  the {NISQ} era and beyond},\ }\href
  {https://doi.org/10.22331/q-2018-08-06-79} {\bibfield  {journal} {\bibinfo
  {journal} {{Quantum}}\ }\textbf {\bibinfo {volume} {2}},\ \bibinfo {pages}
  {79} (\bibinfo {year} {2018})}\BibitemShut {NoStop}%
\bibitem [{\citenamefont {Arute~\textit{et al}}(2019)}]{arute19}%
  \BibitemOpen
  \bibfield  {author} {\bibinfo {author} {\bibfnamefont {F.}~\bibnamefont
  {Arute~\textit{et al}}},\ }\bibfield  {title} {\bibinfo {title} {Quantum
  supremacy using a programmable superconducting processor},\ }\href
  {https://doi.org/10.1038/s41586-019-1666-5} {\bibfield  {journal} {\bibinfo
  {journal} {Nature}\ }\textbf {\bibinfo {volume} {574}},\ \bibinfo {pages}
  {505} (\bibinfo {year} {2019})}\BibitemShut {NoStop}%
\bibitem [{\citenamefont {Burrell}\ and\ \citenamefont
  {Osborne}(2007)}]{burrell07}%
  \BibitemOpen
  \bibfield  {author} {\bibinfo {author} {\bibfnamefont {C.~K.}\ \bibnamefont
  {Burrell}}\ and\ \bibinfo {author} {\bibfnamefont {T.~J.}\ \bibnamefont
  {Osborne}},\ }\bibfield  {title} {\bibinfo {title} {Bounds on the speed of
  information propagation in disordered quantum spin chains},\ }\href
  {https://doi.org/10.1103/PhysRevLett.99.167201} {\bibfield  {journal}
  {\bibinfo  {journal} {Phys. Rev. Lett.}\ }\textbf {\bibinfo {volume} {99}},\
  \bibinfo {pages} {167201} (\bibinfo {year} {2007})}\BibitemShut {NoStop}%
\bibitem [{\citenamefont {Allcock}\ and\ \citenamefont
  {Linden}(2009)}]{allcock09}%
  \BibitemOpen
  \bibfield  {author} {\bibinfo {author} {\bibfnamefont {J.}~\bibnamefont
  {Allcock}}\ and\ \bibinfo {author} {\bibfnamefont {N.}~\bibnamefont
  {Linden}},\ }\bibfield  {title} {\bibinfo {title} {Quantum communication
  beyond the localization length in disordered spin chains},\ }\href
  {https://doi.org/10.1103/PhysRevLett.102.110501} {\bibfield  {journal}
  {\bibinfo  {journal} {Phys. Rev. Lett.}\ }\textbf {\bibinfo {volume} {102}},\
  \bibinfo {pages} {110501} (\bibinfo {year} {2009})}\BibitemShut {NoStop}%
\bibitem [{\citenamefont {Kimble}(2008)}]{kimble08}%
  \BibitemOpen
  \bibfield  {author} {\bibinfo {author} {\bibfnamefont {H.~J.}\ \bibnamefont
  {Kimble}},\ }\bibfield  {title} {\bibinfo {title} {The quantum internet},\
  }\href@noop {} {\bibfield  {journal} {\bibinfo  {journal} {Nature}\ }\textbf
  {\bibinfo {volume} {453}},\ \bibinfo {pages} {1023} (\bibinfo {year}
  {2008})}\BibitemShut {NoStop}%
\bibitem [{\citenamefont {Bose}(2003)}]{bose03}%
  \BibitemOpen
  \bibfield  {author} {\bibinfo {author} {\bibfnamefont {S.}~\bibnamefont
  {Bose}},\ }\bibfield  {title} {\bibinfo {title} {Quantum communication
  through an unmodulated spin chain},\ }\href
  {https://doi.org/10.1103/PhysRevLett.91.207901} {\bibfield  {journal}
  {\bibinfo  {journal} {Phys. Rev. Lett.}\ }\textbf {\bibinfo {volume} {91}},\
  \bibinfo {pages} {207901} (\bibinfo {year} {2003})}\BibitemShut {NoStop}%
\bibitem [{\citenamefont {Christandl}\ \emph {et~al.}(2004)\citenamefont
  {Christandl}, \citenamefont {Datta}, \citenamefont {Ekert},\ and\
  \citenamefont {Landahl}}]{christandl04}%
  \BibitemOpen
  \bibfield  {author} {\bibinfo {author} {\bibfnamefont {M.}~\bibnamefont
  {Christandl}}, \bibinfo {author} {\bibfnamefont {N.}~\bibnamefont {Datta}},
  \bibinfo {author} {\bibfnamefont {A.}~\bibnamefont {Ekert}},\ and\ \bibinfo
  {author} {\bibfnamefont {A.~J.}\ \bibnamefont {Landahl}},\ }\bibfield
  {title} {\bibinfo {title} {Perfect state transfer in quantum spin networks},\
  }\href {https://doi.org/10.1103/PhysRevLett.92.187902} {\bibfield  {journal}
  {\bibinfo  {journal} {Phys. Rev. Lett.}\ }\textbf {\bibinfo {volume} {92}},\
  \bibinfo {pages} {187902} (\bibinfo {year} {2004})}\BibitemShut {NoStop}%
\bibitem [{\citenamefont {Plenio}\ \emph {et~al.}(2004)\citenamefont {Plenio},
  \citenamefont {Hartley},\ and\ \citenamefont {Eisert}}]{plenio04}%
  \BibitemOpen
  \bibfield  {author} {\bibinfo {author} {\bibfnamefont {M.~B.}\ \bibnamefont
  {Plenio}}, \bibinfo {author} {\bibfnamefont {J.}~\bibnamefont {Hartley}},\
  and\ \bibinfo {author} {\bibfnamefont {J.}~\bibnamefont {Eisert}},\
  }\bibfield  {title} {\bibinfo {title} {Dynamics and manipulation of
  entanglement in coupled harmonic systems with many degrees of freedom},\
  }\href {http://stacks.iop.org/1367-2630/6/i=1/a=036} {\bibfield  {journal}
  {\bibinfo  {journal} {New Journal of Physics}\ }\textbf {\bibinfo {volume}
  {6}},\ \bibinfo {pages} {36} (\bibinfo {year} {2004})}\BibitemShut {NoStop}%
\bibitem [{\citenamefont {Osborne}\ and\ \citenamefont
  {Linden}(2004)}]{osborne04}%
  \BibitemOpen
  \bibfield  {author} {\bibinfo {author} {\bibfnamefont {T.~J.}\ \bibnamefont
  {Osborne}}\ and\ \bibinfo {author} {\bibfnamefont {N.}~\bibnamefont
  {Linden}},\ }\bibfield  {title} {\bibinfo {title} {Propagation of quantum
  information through a spin system},\ }\href
  {https://doi.org/10.1103/PhysRevA.69.052315} {\bibfield  {journal} {\bibinfo
  {journal} {Phys. Rev. A}\ }\textbf {\bibinfo {volume} {69}},\ \bibinfo
  {pages} {052315} (\bibinfo {year} {2004})}\BibitemShut {NoStop}%
\bibitem [{\citenamefont {W\'ojcik}\ \emph {et~al.}(2005)\citenamefont
  {W\'ojcik}, \citenamefont {\L{}uczak}, \citenamefont
  {Kurzy\ifmmode~\acute{n}\else \'{n}\fi{}ski}, \citenamefont {Grudka},
  \citenamefont {Gdala},\ and\ \citenamefont {Bednarska}}]{wojcik05}%
  \BibitemOpen
  \bibfield  {author} {\bibinfo {author} {\bibfnamefont {A.}~\bibnamefont
  {W\'ojcik}}, \bibinfo {author} {\bibfnamefont {T.}~\bibnamefont {\L{}uczak}},
  \bibinfo {author} {\bibfnamefont {P.}~\bibnamefont
  {Kurzy\ifmmode~\acute{n}\else \'{n}\fi{}ski}}, \bibinfo {author}
  {\bibfnamefont {A.}~\bibnamefont {Grudka}}, \bibinfo {author} {\bibfnamefont
  {T.}~\bibnamefont {Gdala}},\ and\ \bibinfo {author} {\bibfnamefont
  {M.}~\bibnamefont {Bednarska}},\ }\bibfield  {title} {\bibinfo {title}
  {Unmodulated spin chains as universal quantum wires},\ }\href
  {https://doi.org/10.1103/PhysRevA.72.034303} {\bibfield  {journal} {\bibinfo
  {journal} {Phys. Rev. A}\ }\textbf {\bibinfo {volume} {72}},\ \bibinfo
  {pages} {034303} (\bibinfo {year} {2005})}\BibitemShut {NoStop}%
\bibitem [{\citenamefont {W\'ojcik}\ \emph {et~al.}(2007)\citenamefont
  {W\'ojcik}, \citenamefont {\L{}uczak}, \citenamefont
  {Kurzy\ifmmode~\acute{n}\else \'{n}\fi{}ski}, \citenamefont {Grudka},
  \citenamefont {Gdala},\ and\ \citenamefont {Bednarska}}]{wojcik07}%
  \BibitemOpen
  \bibfield  {author} {\bibinfo {author} {\bibfnamefont {A.}~\bibnamefont
  {W\'ojcik}}, \bibinfo {author} {\bibfnamefont {T.}~\bibnamefont {\L{}uczak}},
  \bibinfo {author} {\bibfnamefont {P.}~\bibnamefont
  {Kurzy\ifmmode~\acute{n}\else \'{n}\fi{}ski}}, \bibinfo {author}
  {\bibfnamefont {A.}~\bibnamefont {Grudka}}, \bibinfo {author} {\bibfnamefont
  {T.}~\bibnamefont {Gdala}},\ and\ \bibinfo {author} {\bibfnamefont
  {M.}~\bibnamefont {Bednarska}},\ }\bibfield  {title} {\bibinfo {title}
  {Multiuser quantum communication networks},\ }\href
  {https://doi.org/10.1103/PhysRevA.75.022330} {\bibfield  {journal} {\bibinfo
  {journal} {Phys. Rev. A}\ }\textbf {\bibinfo {volume} {75}},\ \bibinfo
  {pages} {022330} (\bibinfo {year} {2007})}\BibitemShut {NoStop}%
\bibitem [{\citenamefont {Li}\ \emph {et~al.}(2005)\citenamefont {Li},
  \citenamefont {Shi}, \citenamefont {Chen}, \citenamefont {Song},\ and\
  \citenamefont {Sun}}]{li05}%
  \BibitemOpen
  \bibfield  {author} {\bibinfo {author} {\bibfnamefont {Y.}~\bibnamefont
  {Li}}, \bibinfo {author} {\bibfnamefont {T.}~\bibnamefont {Shi}}, \bibinfo
  {author} {\bibfnamefont {B.}~\bibnamefont {Chen}}, \bibinfo {author}
  {\bibfnamefont {Z.}~\bibnamefont {Song}},\ and\ \bibinfo {author}
  {\bibfnamefont {C.-P.}\ \bibnamefont {Sun}},\ }\bibfield  {title} {\bibinfo
  {title} {Quantum-state transmission via a spin ladder as a robust data bus},\
  }\href {https://doi.org/10.1103/PhysRevA.71.022301} {\bibfield  {journal}
  {\bibinfo  {journal} {Phys. Rev. A}\ }\textbf {\bibinfo {volume} {71}},\
  \bibinfo {pages} {022301} (\bibinfo {year} {2005})}\BibitemShut {NoStop}%
\bibitem [{\citenamefont {Huo}\ \emph {et~al.}(2008)\citenamefont {Huo},
  \citenamefont {Li}, \citenamefont {Song},\ and\ \citenamefont {Sun}}]{huo08}%
  \BibitemOpen
  \bibfield  {author} {\bibinfo {author} {\bibfnamefont {M.~X.}\ \bibnamefont
  {Huo}}, \bibinfo {author} {\bibfnamefont {Y.}~\bibnamefont {Li}}, \bibinfo
  {author} {\bibfnamefont {Z.}~\bibnamefont {Song}},\ and\ \bibinfo {author}
  {\bibfnamefont {C.~P.}\ \bibnamefont {Sun}},\ }\bibfield  {title} {\bibinfo
  {title} {The {P}eierls distorted chain as a quantum data bus for quantum
  state transfer},\ }\href@noop {} {\bibfield  {journal} {\bibinfo  {journal}
  {Europhysics Letters}\ }\textbf {\bibinfo {volume} {84}},\ \bibinfo {pages}
  {30004} (\bibinfo {year} {2008})}\BibitemShut {NoStop}%
\bibitem [{\citenamefont {Liu}\ \emph {et~al.}(2008)\citenamefont {Liu},
  \citenamefont {Zhang},\ and\ \citenamefont {Chen}}]{liu08}%
  \BibitemOpen
  \bibfield  {author} {\bibinfo {author} {\bibfnamefont {J.}~\bibnamefont
  {Liu}}, \bibinfo {author} {\bibfnamefont {G.-F.}\ \bibnamefont {Zhang}},\
  and\ \bibinfo {author} {\bibfnamefont {Z.-Y.}\ \bibnamefont {Chen}},\
  }\bibfield  {title} {\bibinfo {title} {Quantum state transfer via a two-qubit
  {H}eisenberg xxz spin model},\ }\href
  {https://doi.org/https://doi.org/10.1016/j.physleta.2008.01.017} {\bibfield
  {journal} {\bibinfo  {journal} {Physics Letters A}\ }\textbf {\bibinfo
  {volume} {372}},\ \bibinfo {pages} {2830} (\bibinfo {year}
  {2008})}\BibitemShut {NoStop}%
\bibitem [{\citenamefont {Gualdi}\ \emph {et~al.}(2008)\citenamefont {Gualdi},
  \citenamefont {Kostak}, \citenamefont {Marzoli},\ and\ \citenamefont
  {Tombesi}}]{gualdi08}%
  \BibitemOpen
  \bibfield  {author} {\bibinfo {author} {\bibfnamefont {G.}~\bibnamefont
  {Gualdi}}, \bibinfo {author} {\bibfnamefont {V.}~\bibnamefont {Kostak}},
  \bibinfo {author} {\bibfnamefont {I.}~\bibnamefont {Marzoli}},\ and\ \bibinfo
  {author} {\bibfnamefont {P.}~\bibnamefont {Tombesi}},\ }\bibfield  {title}
  {\bibinfo {title} {Perfect state transfer in long-range interacting spin
  chains},\ }\href {https://doi.org/10.1103/PhysRevA.78.022325} {\bibfield
  {journal} {\bibinfo  {journal} {Phys. Rev. A}\ }\textbf {\bibinfo {volume}
  {78}},\ \bibinfo {pages} {022325} (\bibinfo {year} {2008})}\BibitemShut
  {NoStop}%
\bibitem [{\citenamefont {Wang}\ \emph {et~al.}(2009)\citenamefont {Wang},
  \citenamefont {Byrd}, \citenamefont {Shao},\ and\ \citenamefont
  {Zou}}]{wang09}%
  \BibitemOpen
  \bibfield  {author} {\bibinfo {author} {\bibfnamefont {Z.-M.}\ \bibnamefont
  {Wang}}, \bibinfo {author} {\bibfnamefont {M.}~\bibnamefont {Byrd}}, \bibinfo
  {author} {\bibfnamefont {B.}~\bibnamefont {Shao}},\ and\ \bibinfo {author}
  {\bibfnamefont {J.}~\bibnamefont {Zou}},\ }\bibfield  {title} {\bibinfo
  {title} {Quantum communication through anisotropic {H}eisenberg $xy$ spin
  chains},\ }\href
  {https://doi.org/https://doi.org/10.1016/j.physleta.2008.12.016} {\bibfield
  {journal} {\bibinfo  {journal} {Physics Letters A}\ }\textbf {\bibinfo
  {volume} {373}},\ \bibinfo {pages} {636} (\bibinfo {year}
  {2009})}\BibitemShut {NoStop}%
\bibitem [{\citenamefont {Banchi}\ \emph {et~al.}(2010)\citenamefont {Banchi},
  \citenamefont {Apollaro}, \citenamefont {Cuccoli}, \citenamefont {Vaia},\
  and\ \citenamefont {Verrucchi}}]{banchi10}%
  \BibitemOpen
  \bibfield  {author} {\bibinfo {author} {\bibfnamefont {L.}~\bibnamefont
  {Banchi}}, \bibinfo {author} {\bibfnamefont {T.~J.~G.}\ \bibnamefont
  {Apollaro}}, \bibinfo {author} {\bibfnamefont {A.}~\bibnamefont {Cuccoli}},
  \bibinfo {author} {\bibfnamefont {R.}~\bibnamefont {Vaia}},\ and\ \bibinfo
  {author} {\bibfnamefont {P.}~\bibnamefont {Verrucchi}},\ }\bibfield  {title}
  {\bibinfo {title} {Optimal dynamics for quantum-state and entanglement
  transfer through homogeneous quantum systems},\ }\href
  {https://doi.org/10.1103/PhysRevA.82.052321} {\bibfield  {journal} {\bibinfo
  {journal} {Phys. Rev. A}\ }\textbf {\bibinfo {volume} {82}},\ \bibinfo
  {pages} {052321} (\bibinfo {year} {2010})}\BibitemShut {NoStop}%
\bibitem [{\citenamefont {Apollaro}\ \emph {et~al.}(2012)\citenamefont
  {Apollaro}, \citenamefont {Banchi}, \citenamefont {Cuccoli}, \citenamefont
  {Vaia},\ and\ \citenamefont {Verrucchi}}]{apollaro12}%
  \BibitemOpen
  \bibfield  {author} {\bibinfo {author} {\bibfnamefont {T.~J.~G.}\
  \bibnamefont {Apollaro}}, \bibinfo {author} {\bibfnamefont {L.}~\bibnamefont
  {Banchi}}, \bibinfo {author} {\bibfnamefont {A.}~\bibnamefont {Cuccoli}},
  \bibinfo {author} {\bibfnamefont {R.}~\bibnamefont {Vaia}},\ and\ \bibinfo
  {author} {\bibfnamefont {P.}~\bibnamefont {Verrucchi}},\ }\bibfield  {title}
  {\bibinfo {title} {99$\%$-fidelity ballistic quantum-state transfer through
  long uniform channels},\ }\href {https://doi.org/10.1103/PhysRevA.85.052319}
  {\bibfield  {journal} {\bibinfo  {journal} {Phys. Rev. A}\ }\textbf {\bibinfo
  {volume} {85}},\ \bibinfo {pages} {052319} (\bibinfo {year}
  {2012})}\BibitemShut {NoStop}%
\bibitem [{\citenamefont {Lorenzo}\ \emph {et~al.}(2013)\citenamefont
  {Lorenzo}, \citenamefont {Apollaro}, \citenamefont {Sindona},\ and\
  \citenamefont {Plastina}}]{lorenzo13}%
  \BibitemOpen
  \bibfield  {author} {\bibinfo {author} {\bibfnamefont {S.}~\bibnamefont
  {Lorenzo}}, \bibinfo {author} {\bibfnamefont {T.~J.~G.}\ \bibnamefont
  {Apollaro}}, \bibinfo {author} {\bibfnamefont {A.}~\bibnamefont {Sindona}},\
  and\ \bibinfo {author} {\bibfnamefont {F.}~\bibnamefont {Plastina}},\
  }\bibfield  {title} {\bibinfo {title} {Quantum-state transfer via resonant
  tunneling through local-field-induced barriers},\ }\href
  {https://doi.org/10.1103/PhysRevA.87.042313} {\bibfield  {journal} {\bibinfo
  {journal} {Phys. Rev. A}\ }\textbf {\bibinfo {volume} {87}},\ \bibinfo
  {pages} {042313} (\bibinfo {year} {2013})}\BibitemShut {NoStop}%
\bibitem [{\citenamefont {Paganelli}\ \emph {et~al.}(2013)\citenamefont
  {Paganelli}, \citenamefont {Lorenzo}, \citenamefont {Apollaro}, \citenamefont
  {Plastina},\ and\ \citenamefont {Giorgi}}]{paganelli13}%
  \BibitemOpen
  \bibfield  {author} {\bibinfo {author} {\bibfnamefont {S.}~\bibnamefont
  {Paganelli}}, \bibinfo {author} {\bibfnamefont {S.}~\bibnamefont {Lorenzo}},
  \bibinfo {author} {\bibfnamefont {T.~J.~G.}\ \bibnamefont {Apollaro}},
  \bibinfo {author} {\bibfnamefont {F.}~\bibnamefont {Plastina}},\ and\
  \bibinfo {author} {\bibfnamefont {G.~L.}\ \bibnamefont {Giorgi}},\ }\bibfield
   {title} {\bibinfo {title} {Routing quantum information in spin chains},\
  }\href {https://doi.org/10.1103/PhysRevA.87.062309} {\bibfield  {journal}
  {\bibinfo  {journal} {Phys. Rev. A}\ }\textbf {\bibinfo {volume} {87}},\
  \bibinfo {pages} {062309} (\bibinfo {year} {2013})}\BibitemShut {NoStop}%
\bibitem [{\citenamefont {Lorenzo}\ \emph {et~al.}(2015)\citenamefont
  {Lorenzo}, \citenamefont {Apollaro}, \citenamefont {Paganelli}, \citenamefont
  {Palma},\ and\ \citenamefont {Plastina}}]{lorenzo15}%
  \BibitemOpen
  \bibfield  {author} {\bibinfo {author} {\bibfnamefont {S.}~\bibnamefont
  {Lorenzo}}, \bibinfo {author} {\bibfnamefont {T.~J.~G.}\ \bibnamefont
  {Apollaro}}, \bibinfo {author} {\bibfnamefont {S.}~\bibnamefont {Paganelli}},
  \bibinfo {author} {\bibfnamefont {G.~M.}\ \bibnamefont {Palma}},\ and\
  \bibinfo {author} {\bibfnamefont {F.}~\bibnamefont {Plastina}},\ }\bibfield
  {title} {\bibinfo {title} {Transfer of arbitrary two-qubit states via a spin
  chain},\ }\href {https://doi.org/10.1103/PhysRevA.91.042321} {\bibfield
  {journal} {\bibinfo  {journal} {Phys. Rev. A}\ }\textbf {\bibinfo {volume}
  {91}},\ \bibinfo {pages} {042321} (\bibinfo {year} {2015})}\BibitemShut
  {NoStop}%
\bibitem [{\citenamefont {Almeida}\ \emph {et~al.}(2016)\citenamefont
  {Almeida}, \citenamefont {Ciccarello}, \citenamefont {Apollaro},\ and\
  \citenamefont {Souza}}]{almeida16}%
  \BibitemOpen
  \bibfield  {author} {\bibinfo {author} {\bibfnamefont {G.~M.~A.}\
  \bibnamefont {Almeida}}, \bibinfo {author} {\bibfnamefont {F.}~\bibnamefont
  {Ciccarello}}, \bibinfo {author} {\bibfnamefont {T.~J.~G.}\ \bibnamefont
  {Apollaro}},\ and\ \bibinfo {author} {\bibfnamefont {A.~M.~C.}\ \bibnamefont
  {Souza}},\ }\bibfield  {title} {\bibinfo {title} {Quantum-state transfer in
  staggered coupled-cavity arrays},\ }\href
  {https://doi.org/10.1103/PhysRevA.93.032310} {\bibfield  {journal} {\bibinfo
  {journal} {Phys. Rev. A}\ }\textbf {\bibinfo {volume} {93}},\ \bibinfo
  {pages} {032310} (\bibinfo {year} {2016})}\BibitemShut {NoStop}%
\bibitem [{\citenamefont {Almeida}(2018)}]{almeida18}%
  \BibitemOpen
  \bibfield  {author} {\bibinfo {author} {\bibfnamefont {G.~M.~A.}\
  \bibnamefont {Almeida}},\ }\bibfield  {title} {\bibinfo {title} {Interplay
  between speed and fidelity in off-resonant quantum-state-transfer
  protocols},\ }\href {https://doi.org/10.1103/PhysRevA.98.012334} {\bibfield
  {journal} {\bibinfo  {journal} {Phys. Rev. A}\ }\textbf {\bibinfo {volume}
  {98}},\ \bibinfo {pages} {012334} (\bibinfo {year} {2018})}\BibitemShut
  {NoStop}%
\bibitem [{\citenamefont {Apollaro}\ \emph {et~al.}(2019)\citenamefont
  {Apollaro}, \citenamefont {Almeida}, \citenamefont {Lorenzo}, \citenamefont
  {Ferraro},\ and\ \citenamefont {Paganelli}}]{apollaro19}%
  \BibitemOpen
  \bibfield  {author} {\bibinfo {author} {\bibfnamefont {T.~J.~G.}\
  \bibnamefont {Apollaro}}, \bibinfo {author} {\bibfnamefont {G.~M.~A.}\
  \bibnamefont {Almeida}}, \bibinfo {author} {\bibfnamefont {S.}~\bibnamefont
  {Lorenzo}}, \bibinfo {author} {\bibfnamefont {A.}~\bibnamefont {Ferraro}},\
  and\ \bibinfo {author} {\bibfnamefont {S.}~\bibnamefont {Paganelli}},\
  }\bibfield  {title} {\bibinfo {title} {Spin chains for two-qubit
  teleportation},\ }\href {https://doi.org/10.1103/PhysRevA.100.052308}
  {\bibfield  {journal} {\bibinfo  {journal} {Phys. Rev. A}\ }\textbf {\bibinfo
  {volume} {100}},\ \bibinfo {pages} {052308} (\bibinfo {year}
  {2019})}\BibitemShut {NoStop}%
\bibitem [{\citenamefont {De~Chiara}\ \emph {et~al.}(2005)\citenamefont
  {De~Chiara}, \citenamefont {Rossini}, \citenamefont {Montangero},\ and\
  \citenamefont {Fazio}}]{dechiara05}%
  \BibitemOpen
  \bibfield  {author} {\bibinfo {author} {\bibfnamefont {G.}~\bibnamefont
  {De~Chiara}}, \bibinfo {author} {\bibfnamefont {D.}~\bibnamefont {Rossini}},
  \bibinfo {author} {\bibfnamefont {S.}~\bibnamefont {Montangero}},\ and\
  \bibinfo {author} {\bibfnamefont {R.}~\bibnamefont {Fazio}},\ }\bibfield
  {title} {\bibinfo {title} {From perfect to fractal transmission in spin
  chains},\ }\href {https://doi.org/10.1103/PhysRevA.72.012323} {\bibfield
  {journal} {\bibinfo  {journal} {Phys. Rev. A}\ }\textbf {\bibinfo {volume}
  {72}},\ \bibinfo {pages} {012323} (\bibinfo {year} {2005})}\BibitemShut
  {NoStop}%
\bibitem [{\citenamefont {Fitzsimons}\ and\ \citenamefont
  {Twamley}(2005)}]{fitzsimons05}%
  \BibitemOpen
  \bibfield  {author} {\bibinfo {author} {\bibfnamefont {J.}~\bibnamefont
  {Fitzsimons}}\ and\ \bibinfo {author} {\bibfnamefont {J.}~\bibnamefont
  {Twamley}},\ }\bibfield  {title} {\bibinfo {title} {Superballistic diffusion
  of entanglement in disordered spin chains},\ }\href
  {https://doi.org/10.1103/PhysRevA.72.050301} {\bibfield  {journal} {\bibinfo
  {journal} {Phys. Rev. A}\ }\textbf {\bibinfo {volume} {72}},\ \bibinfo
  {pages} {050301} (\bibinfo {year} {2005})}\BibitemShut {NoStop}%
\bibitem [{\citenamefont {Burgarth}\ and\ \citenamefont
  {Bose}(2005)}]{burgarth05}%
  \BibitemOpen
  \bibfield  {author} {\bibinfo {author} {\bibfnamefont {D.}~\bibnamefont
  {Burgarth}}\ and\ \bibinfo {author} {\bibfnamefont {S.}~\bibnamefont
  {Bose}},\ }\bibfield  {title} {\bibinfo {title} {Perfect quantum state
  transfer with randomly coupled quantum chains},\ }\href
  {https://doi.org/10.1088/1367-2630/7/1/135} {\bibfield  {journal} {\bibinfo
  {journal} {New Journal of Physics}\ }\textbf {\bibinfo {volume} {7}},\
  \bibinfo {pages} {135} (\bibinfo {year} {2005})}\BibitemShut {NoStop}%
\bibitem [{\citenamefont {Tsomokos}\ \emph {et~al.}(2007)\citenamefont
  {Tsomokos}, \citenamefont {Hartmann}, \citenamefont {Huelga},\ and\
  \citenamefont {Plenio}}]{tsomokos07}%
  \BibitemOpen
  \bibfield  {author} {\bibinfo {author} {\bibfnamefont {D.~I.}\ \bibnamefont
  {Tsomokos}}, \bibinfo {author} {\bibfnamefont {M.~J.}\ \bibnamefont
  {Hartmann}}, \bibinfo {author} {\bibfnamefont {S.~F.}\ \bibnamefont
  {Huelga}},\ and\ \bibinfo {author} {\bibfnamefont {M.~B.}\ \bibnamefont
  {Plenio}},\ }\bibfield  {title} {\bibinfo {title} {Entanglement dynamics in
  chains of qubits with noise and disorder},\ }\href
  {http://stacks.iop.org/1367-2630/9/i=3/a=079} {\bibfield  {journal} {\bibinfo
   {journal} {New Journal of Physics}\ }\textbf {\bibinfo {volume} {9}},\
  \bibinfo {pages} {79} (\bibinfo {year} {2007})}\BibitemShut {NoStop}%
\bibitem [{\citenamefont {Giampaolo}\ and\ \citenamefont
  {Illuminati}(2010)}]{giampaolo10}%
  \BibitemOpen
  \bibfield  {author} {\bibinfo {author} {\bibfnamefont {S.~M.}\ \bibnamefont
  {Giampaolo}}\ and\ \bibinfo {author} {\bibfnamefont {F.}~\bibnamefont
  {Illuminati}},\ }\bibfield  {title} {\bibinfo {title} {Long-distance
  entanglement in many-body atomic and optical systems},\ }\href@noop {}
  {\bibfield  {journal} {\bibinfo  {journal} {New Journal of Physics}\ }\textbf
  {\bibinfo {volume} {12}},\ \bibinfo {pages} {025019} (\bibinfo {year}
  {2010})}\BibitemShut {NoStop}%
\bibitem [{\citenamefont {Yao}\ \emph {et~al.}(2011)\citenamefont {Yao},
  \citenamefont {Jiang}, \citenamefont {Gorshkov}, \citenamefont {Gong},
  \citenamefont {Zhai}, \citenamefont {Duan},\ and\ \citenamefont
  {Lukin}}]{yao11}%
  \BibitemOpen
  \bibfield  {author} {\bibinfo {author} {\bibfnamefont {N.~Y.}\ \bibnamefont
  {Yao}}, \bibinfo {author} {\bibfnamefont {L.}~\bibnamefont {Jiang}}, \bibinfo
  {author} {\bibfnamefont {A.~V.}\ \bibnamefont {Gorshkov}}, \bibinfo {author}
  {\bibfnamefont {Z.-X.}\ \bibnamefont {Gong}}, \bibinfo {author}
  {\bibfnamefont {A.}~\bibnamefont {Zhai}}, \bibinfo {author} {\bibfnamefont
  {L.-M.}\ \bibnamefont {Duan}},\ and\ \bibinfo {author} {\bibfnamefont
  {M.~D.}\ \bibnamefont {Lukin}},\ }\bibfield  {title} {\bibinfo {title}
  {Robust quantum state transfer in random unpolarized spin chains},\ }\href
  {https://doi.org/10.1103/PhysRevLett.106.040505} {\bibfield  {journal}
  {\bibinfo  {journal} {Phys. Rev. Lett.}\ }\textbf {\bibinfo {volume} {106}},\
  \bibinfo {pages} {040505} (\bibinfo {year} {2011})}\BibitemShut {NoStop}%
\bibitem [{\citenamefont {Zwick}\ \emph {et~al.}(2012)\citenamefont {Zwick},
  \citenamefont {\'Alvarez}, \citenamefont {Stolze},\ and\ \citenamefont
  {Osenda}}]{zwick12}%
  \BibitemOpen
  \bibfield  {author} {\bibinfo {author} {\bibfnamefont {A.}~\bibnamefont
  {Zwick}}, \bibinfo {author} {\bibfnamefont {G.~A.}\ \bibnamefont
  {\'Alvarez}}, \bibinfo {author} {\bibfnamefont {J.}~\bibnamefont {Stolze}},\
  and\ \bibinfo {author} {\bibfnamefont {O.}~\bibnamefont {Osenda}},\
  }\bibfield  {title} {\bibinfo {title} {Spin chains for robust state transfer:
  Modified boundary couplings versus completely engineered chains},\ }\href
  {https://doi.org/10.1103/PhysRevA.85.012318} {\bibfield  {journal} {\bibinfo
  {journal} {Phys. Rev. A}\ }\textbf {\bibinfo {volume} {85}},\ \bibinfo
  {pages} {012318} (\bibinfo {year} {2012})}\BibitemShut {NoStop}%
\bibitem [{\citenamefont {Zwick}\ \emph {et~al.}(2015)\citenamefont {Zwick},
  \citenamefont {\'Alvarez}, \citenamefont {Stolze},\ and\ \citenamefont
  {Osenda}}]{zwick15}%
  \BibitemOpen
  \bibfield  {author} {\bibinfo {author} {\bibfnamefont {A.}~\bibnamefont
  {Zwick}}, \bibinfo {author} {\bibfnamefont {G.~A.}\ \bibnamefont
  {\'Alvarez}}, \bibinfo {author} {\bibfnamefont {J.}~\bibnamefont {Stolze}},\
  and\ \bibinfo {author} {\bibfnamefont {O.}~\bibnamefont {Osenda}},\
  }\bibfield  {title} {\bibinfo {title} {Quantum state transfer in disordered
  spin chains: how much engineering is reasonable?},\ }\href@noop {} {\bibfield
   {journal} {\bibinfo  {journal} {Quant. Inf. Comp.}\ }\textbf {\bibinfo
  {volume} {15}},\ \bibinfo {pages} {852} (\bibinfo {year} {2015})}\BibitemShut
  {NoStop}%
\bibitem [{\citenamefont {Bruderer}\ \emph {et~al.}(2012)\citenamefont
  {Bruderer}, \citenamefont {Franke}, \citenamefont {Ragg}, \citenamefont
  {Belzig},\ and\ \citenamefont {Obreschkow}}]{bruderer12}%
  \BibitemOpen
  \bibfield  {author} {\bibinfo {author} {\bibfnamefont {M.}~\bibnamefont
  {Bruderer}}, \bibinfo {author} {\bibfnamefont {K.}~\bibnamefont {Franke}},
  \bibinfo {author} {\bibfnamefont {S.}~\bibnamefont {Ragg}}, \bibinfo {author}
  {\bibfnamefont {W.}~\bibnamefont {Belzig}},\ and\ \bibinfo {author}
  {\bibfnamefont {D.}~\bibnamefont {Obreschkow}},\ }\bibfield  {title}
  {\bibinfo {title} {Exploiting boundary states of imperfect spin chains for
  high-fidelity state transfer},\ }\href
  {https://doi.org/10.1103/PhysRevA.85.022312} {\bibfield  {journal} {\bibinfo
  {journal} {Phys. Rev. A}\ }\textbf {\bibinfo {volume} {85}},\ \bibinfo
  {pages} {022312} (\bibinfo {year} {2012})}\BibitemShut {NoStop}%
\bibitem [{\citenamefont {Ashhab}(2015)}]{ashhab15}%
  \BibitemOpen
  \bibfield  {author} {\bibinfo {author} {\bibfnamefont {S.}~\bibnamefont
  {Ashhab}},\ }\bibfield  {title} {\bibinfo {title} {Quantum state transfer in
  a disordered one-dimensional lattice},\ }\href
  {https://doi.org/10.1103/PhysRevA.92.062305} {\bibfield  {journal} {\bibinfo
  {journal} {Phys. Rev. A}\ }\textbf {\bibinfo {volume} {92}},\ \bibinfo
  {pages} {062305} (\bibinfo {year} {2015})}\BibitemShut {NoStop}%
\bibitem [{\citenamefont {Kay}(2016)}]{kay16}%
  \BibitemOpen
  \bibfield  {author} {\bibinfo {author} {\bibfnamefont {A.}~\bibnamefont
  {Kay}},\ }\bibfield  {title} {\bibinfo {title} {Quantum error correction for
  state transfer in noisy spin chains},\ }\href
  {https://doi.org/10.1103/PhysRevA.93.042320} {\bibfield  {journal} {\bibinfo
  {journal} {Phys. Rev. A}\ }\textbf {\bibinfo {volume} {93}},\ \bibinfo
  {pages} {042320} (\bibinfo {year} {2016})}\BibitemShut {NoStop}%
\bibitem [{\citenamefont {Estarellas}\ \emph {et~al.}(2017)\citenamefont
  {Estarellas}, \citenamefont {D'Amico},\ and\ \citenamefont
  {Spiller}}]{estarellas17}%
  \BibitemOpen
  \bibfield  {author} {\bibinfo {author} {\bibfnamefont {M.~P.}\ \bibnamefont
  {Estarellas}}, \bibinfo {author} {\bibfnamefont {I.}~\bibnamefont
  {D'Amico}},\ and\ \bibinfo {author} {\bibfnamefont {T.~P.}\ \bibnamefont
  {Spiller}},\ }\bibfield  {title} {\bibinfo {title} {Robust quantum
  entanglement generation and generation-plus-storage protocols with spin
  chains},\ }\href {https://doi.org/10.1103/PhysRevA.95.042335} {\bibfield
  {journal} {\bibinfo  {journal} {Phys. Rev. A}\ }\textbf {\bibinfo {volume}
  {95}},\ \bibinfo {pages} {042335} (\bibinfo {year} {2017})}\BibitemShut
  {NoStop}%
\bibitem [{\citenamefont {Almeida}\ \emph {et~al.}(2018)\citenamefont
  {Almeida}, \citenamefont {de~Moura},\ and\ \citenamefont
  {Lyra}}]{almeida18pla}%
  \BibitemOpen
  \bibfield  {author} {\bibinfo {author} {\bibfnamefont {G.~M.~A.}\
  \bibnamefont {Almeida}}, \bibinfo {author} {\bibfnamefont {F.~A. B.~F.}\
  \bibnamefont {de~Moura}},\ and\ \bibinfo {author} {\bibfnamefont {M.~L.}\
  \bibnamefont {Lyra}},\ }\bibfield  {title} {\bibinfo {title} {High-fidelity
  state transfer through long-range correlated disordered quantum channels},\
  }\href@noop {} {\bibfield  {journal} {\bibinfo  {journal} {Phys. Lett. A}\
  }\textbf {\bibinfo {volume} {382}},\ \bibinfo {pages} {1335} (\bibinfo {year}
  {2018})}\BibitemShut {NoStop}%
\bibitem [{\citenamefont {Almeida}\ \emph {et~al.}(2019)\citenamefont
  {Almeida}, \citenamefont {de~Moura},\ and\ \citenamefont
  {Lyra}}]{almeida19qinp}%
  \BibitemOpen
  \bibfield  {author} {\bibinfo {author} {\bibfnamefont {G.~M.~A.}\
  \bibnamefont {Almeida}}, \bibinfo {author} {\bibfnamefont {F.~A. B.~F.}\
  \bibnamefont {de~Moura}},\ and\ \bibinfo {author} {\bibfnamefont {M.~L.}\
  \bibnamefont {Lyra}},\ }\bibfield  {title} {\bibinfo {title} {Transmission of
  quantum states through disordered channels with dimerized defects},\
  }\href@noop {} {\bibfield  {journal} {\bibinfo  {journal} {Quant. Inf.
  Proc.}\ }\textbf {\bibinfo {volume} {18}},\ \bibinfo {pages} {350} (\bibinfo
  {year} {2019})}\BibitemShut {NoStop}%
\bibitem [{\citenamefont {Petrovi{\'{c}}}\ \emph {et~al.}(2016)\citenamefont
  {Petrovi{\'{c}}}, \citenamefont {Ansermet}, \citenamefont {Chernyshov},
  \citenamefont {Hoesch}, \citenamefont {Salloum}, \citenamefont {Gougeon},
  \citenamefont {Potel}, \citenamefont {Boeri},\ and\ \citenamefont
  {Panagopoulos}}]{petrovic16}%
  \BibitemOpen
  \bibfield  {author} {\bibinfo {author} {\bibfnamefont {A.~P.}\ \bibnamefont
  {Petrovi{\'{c}}}}, \bibinfo {author} {\bibfnamefont {D.}~\bibnamefont
  {Ansermet}}, \bibinfo {author} {\bibfnamefont {D.}~\bibnamefont
  {Chernyshov}}, \bibinfo {author} {\bibfnamefont {M.}~\bibnamefont {Hoesch}},
  \bibinfo {author} {\bibfnamefont {D.}~\bibnamefont {Salloum}}, \bibinfo
  {author} {\bibfnamefont {P.}~\bibnamefont {Gougeon}}, \bibinfo {author}
  {\bibfnamefont {M.}~\bibnamefont {Potel}}, \bibinfo {author} {\bibfnamefont
  {L.}~\bibnamefont {Boeri}},\ and\ \bibinfo {author} {\bibfnamefont
  {C.}~\bibnamefont {Panagopoulos}},\ }\bibfield  {title} {\bibinfo {title} {A
  disorder-enhanced quasi-one-dimensional superconductor},\ }\href
  {https://doi.org/10.1038/ncomms12262} {\bibfield  {journal} {\bibinfo
  {journal} {Nature Communications}\ }\textbf {\bibinfo {volume} {7}},\
  \bibinfo {pages} {12262} (\bibinfo {year} {2016})}\BibitemShut {NoStop}%
\bibitem [{\citenamefont {Gligori\ifmmode~\acute{c}\else \'{c}\fi{}}\ \emph
  {et~al.}(2020)\citenamefont {Gligori\ifmmode~\acute{c}\else \'{c}\fi{}},
  \citenamefont {Leykam},\ and\ \citenamefont {Maluckov}}]{gligori20}%
  \BibitemOpen
  \bibfield  {author} {\bibinfo {author} {\bibfnamefont {G.}~\bibnamefont
  {Gligori\ifmmode~\acute{c}\else \'{c}\fi{}}}, \bibinfo {author}
  {\bibfnamefont {D.}~\bibnamefont {Leykam}},\ and\ \bibinfo {author}
  {\bibfnamefont {A.}~\bibnamefont {Maluckov}},\ }\bibfield  {title} {\bibinfo
  {title} {Influence of different disorder types on aharonov-bohm caging in the
  diamond chain},\ }\href {https://doi.org/10.1103/PhysRevA.101.023839}
  {\bibfield  {journal} {\bibinfo  {journal} {Phys. Rev. A}\ }\textbf {\bibinfo
  {volume} {101}},\ \bibinfo {pages} {023839} (\bibinfo {year}
  {2020})}\BibitemShut {NoStop}%
\bibitem [{\citenamefont {Roy}\ \emph {et~al.}(2020)\citenamefont {Roy},
  \citenamefont {Ramachandran},\ and\ \citenamefont {Sharma}}]{roy20}%
  \BibitemOpen
  \bibfield  {author} {\bibinfo {author} {\bibfnamefont {N.}~\bibnamefont
  {Roy}}, \bibinfo {author} {\bibfnamefont {A.}~\bibnamefont {Ramachandran}},\
  and\ \bibinfo {author} {\bibfnamefont {A.}~\bibnamefont {Sharma}},\
  }\bibfield  {title} {\bibinfo {title} {Interplay of disorder and interactions
  in a flat-band supporting diamond chain},\ }\href
  {https://doi.org/10.1103/PhysRevResearch.2.043395} {\bibfield  {journal}
  {\bibinfo  {journal} {Phys. Rev. Research}\ }\textbf {\bibinfo {volume}
  {2}},\ \bibinfo {pages} {043395} (\bibinfo {year} {2020})}\BibitemShut
  {NoStop}%
\bibitem [{\citenamefont {Hyrk\"as}\ \emph {et~al.}(2013)\citenamefont
  {Hyrk\"as}, \citenamefont {Apaja},\ and\ \citenamefont
  {Manninen}}]{hyrkas13}%
  \BibitemOpen
  \bibfield  {author} {\bibinfo {author} {\bibfnamefont {M.}~\bibnamefont
  {Hyrk\"as}}, \bibinfo {author} {\bibfnamefont {V.}~\bibnamefont {Apaja}},\
  and\ \bibinfo {author} {\bibfnamefont {M.}~\bibnamefont {Manninen}},\
  }\bibfield  {title} {\bibinfo {title} {Many-particle dynamics of bosons and
  fermions in quasi-one-dimensional flat-band lattices},\ }\href
  {https://doi.org/10.1103/PhysRevA.87.023614} {\bibfield  {journal} {\bibinfo
  {journal} {Phys. Rev. A}\ }\textbf {\bibinfo {volume} {87}},\ \bibinfo
  {pages} {023614} (\bibinfo {year} {2013})}\BibitemShut {NoStop}%
\bibitem [{\citenamefont {Flach}\ \emph {et~al.}(2014)\citenamefont {Flach},
  \citenamefont {Leykam}, \citenamefont {Bodyfelt}, \citenamefont {Matthies},\
  and\ \citenamefont {Desyatnikov}}]{flach14}%
  \BibitemOpen
  \bibfield  {author} {\bibinfo {author} {\bibfnamefont {S.}~\bibnamefont
  {Flach}}, \bibinfo {author} {\bibfnamefont {D.}~\bibnamefont {Leykam}},
  \bibinfo {author} {\bibfnamefont {J.~D.}\ \bibnamefont {Bodyfelt}}, \bibinfo
  {author} {\bibfnamefont {P.}~\bibnamefont {Matthies}},\ and\ \bibinfo
  {author} {\bibfnamefont {A.~S.}\ \bibnamefont {Desyatnikov}},\ }\bibfield
  {title} {\bibinfo {title} {Detangling flat bands into {F}ano lattices},\
  }\href {https://doi.org/10.1209/0295-5075/105/30001} {\bibfield  {journal}
  {\bibinfo  {journal} {EPL}\ }\textbf {\bibinfo {volume} {105}},\ \bibinfo
  {pages} {30001} (\bibinfo {year} {2014})}\BibitemShut {NoStop}%
\bibitem [{\citenamefont {Maimaiti}\ \emph {et~al.}(2017)\citenamefont
  {Maimaiti}, \citenamefont {Andreanov}, \citenamefont {Park}, \citenamefont
  {Gendelman},\ and\ \citenamefont {Flach}}]{maimaiti17}%
  \BibitemOpen
  \bibfield  {author} {\bibinfo {author} {\bibfnamefont {W.}~\bibnamefont
  {Maimaiti}}, \bibinfo {author} {\bibfnamefont {A.}~\bibnamefont {Andreanov}},
  \bibinfo {author} {\bibfnamefont {H.~C.}\ \bibnamefont {Park}}, \bibinfo
  {author} {\bibfnamefont {O.}~\bibnamefont {Gendelman}},\ and\ \bibinfo
  {author} {\bibfnamefont {S.}~\bibnamefont {Flach}},\ }\bibfield  {title}
  {\bibinfo {title} {Compact localized states and flat-band generators in one
  dimension},\ }\href {https://doi.org/10.1103/PhysRevB.95.115135} {\bibfield
  {journal} {\bibinfo  {journal} {Phys. Rev. B}\ }\textbf {\bibinfo {volume}
  {95}},\ \bibinfo {pages} {115135} (\bibinfo {year} {2017})}\BibitemShut
  {NoStop}%
\bibitem [{\citenamefont {Derzhko}\ \emph {et~al.}(2015)\citenamefont
  {Derzhko}, \citenamefont {Richter},\ and\ \citenamefont
  {Maksymenko}}]{derzhko15rev}%
  \BibitemOpen
  \bibfield  {author} {\bibinfo {author} {\bibfnamefont {O.}~\bibnamefont
  {Derzhko}}, \bibinfo {author} {\bibfnamefont {J.}~\bibnamefont {Richter}},\
  and\ \bibinfo {author} {\bibfnamefont {M.}~\bibnamefont {Maksymenko}},\
  }\bibfield  {title} {\bibinfo {title} {Strongly correlated flat-band systems:
  The route from {H}eisenberg spins to {H}ubbard electrons},\ }\href
  {https://doi.org/10.1142/S0217979215300078} {\bibfield  {journal} {\bibinfo
  {journal} {International Journal of Modern Physics B}\ }\textbf {\bibinfo
  {volume} {29}},\ \bibinfo {pages} {1530007} (\bibinfo {year}
  {2015})}\BibitemShut {NoStop}%
\bibitem [{\citenamefont {Goda}\ \emph {et~al.}(2006)\citenamefont {Goda},
  \citenamefont {Nishino},\ and\ \citenamefont {Matsuda}}]{goda06}%
  \BibitemOpen
  \bibfield  {author} {\bibinfo {author} {\bibfnamefont {M.}~\bibnamefont
  {Goda}}, \bibinfo {author} {\bibfnamefont {S.}~\bibnamefont {Nishino}},\ and\
  \bibinfo {author} {\bibfnamefont {H.}~\bibnamefont {Matsuda}},\ }\bibfield
  {title} {\bibinfo {title} {Inverse {A}nderson transition caused by
  flatbands},\ }\href {https://doi.org/10.1103/PhysRevLett.96.126401}
  {\bibfield  {journal} {\bibinfo  {journal} {Phys. Rev. Lett.}\ }\textbf
  {\bibinfo {volume} {96}},\ \bibinfo {pages} {126401} (\bibinfo {year}
  {2006})}\BibitemShut {NoStop}%
\bibitem [{\citenamefont {Chalker}\ \emph {et~al.}(2010)\citenamefont
  {Chalker}, \citenamefont {Pickles},\ and\ \citenamefont
  {Shukla}}]{chalker10}%
  \BibitemOpen
  \bibfield  {author} {\bibinfo {author} {\bibfnamefont {J.~T.}\ \bibnamefont
  {Chalker}}, \bibinfo {author} {\bibfnamefont {T.~S.}\ \bibnamefont
  {Pickles}},\ and\ \bibinfo {author} {\bibfnamefont {P.}~\bibnamefont
  {Shukla}},\ }\bibfield  {title} {\bibinfo {title} {{A}nderson localization in
  tight-binding models with flat bands},\ }\href
  {https://doi.org/10.1103/PhysRevB.82.104209} {\bibfield  {journal} {\bibinfo
  {journal} {Phys. Rev. B}\ }\textbf {\bibinfo {volume} {82}},\ \bibinfo
  {pages} {104209} (\bibinfo {year} {2010})}\BibitemShut {NoStop}%
\bibitem [{\citenamefont {Souza}\ and\ \citenamefont
  {Herrmann}(2009)}]{souza09}%
  \BibitemOpen
  \bibfield  {author} {\bibinfo {author} {\bibfnamefont {A.~M.~C.}\
  \bibnamefont {Souza}}\ and\ \bibinfo {author} {\bibfnamefont {H.~J.}\
  \bibnamefont {Herrmann}},\ }\bibfield  {title} {\bibinfo {title} {Flat-band
  localization in the {A}nderson-{F}alicov-{K}imball model},\ }\href
  {https://doi.org/10.1103/PhysRevB.79.153104} {\bibfield  {journal} {\bibinfo
  {journal} {Phys. Rev. B}\ }\textbf {\bibinfo {volume} {79}},\ \bibinfo
  {pages} {153104} (\bibinfo {year} {2009})}\BibitemShut {NoStop}%
\bibitem [{\citenamefont {Vicencio}\ and\ \citenamefont
  {Johansson}(2013)}]{vicencio13}%
  \BibitemOpen
  \bibfield  {author} {\bibinfo {author} {\bibfnamefont {R.~A.}\ \bibnamefont
  {Vicencio}}\ and\ \bibinfo {author} {\bibfnamefont {M.}~\bibnamefont
  {Johansson}},\ }\bibfield  {title} {\bibinfo {title} {Discrete flat-band
  solitons in the kagome lattice},\ }\href
  {https://doi.org/10.1103/PhysRevA.87.061803} {\bibfield  {journal} {\bibinfo
  {journal} {Phys. Rev. A}\ }\textbf {\bibinfo {volume} {87}},\ \bibinfo
  {pages} {061803} (\bibinfo {year} {2013})}\BibitemShut {NoStop}%
\bibitem [{\citenamefont {Leykam}\ \emph {et~al.}(2013)\citenamefont {Leykam},
  \citenamefont {Flach}, \citenamefont {Bahat-Treidel},\ and\ \citenamefont
  {Desyatnikov}}]{leykam13}%
  \BibitemOpen
  \bibfield  {author} {\bibinfo {author} {\bibfnamefont {D.}~\bibnamefont
  {Leykam}}, \bibinfo {author} {\bibfnamefont {S.}~\bibnamefont {Flach}},
  \bibinfo {author} {\bibfnamefont {O.}~\bibnamefont {Bahat-Treidel}},\ and\
  \bibinfo {author} {\bibfnamefont {A.~S.}\ \bibnamefont {Desyatnikov}},\
  }\bibfield  {title} {\bibinfo {title} {Flat band states: Disorder and
  nonlinearity},\ }\href {https://doi.org/10.1103/PhysRevB.88.224203}
  {\bibfield  {journal} {\bibinfo  {journal} {Phys. Rev. B}\ }\textbf {\bibinfo
  {volume} {88}},\ \bibinfo {pages} {224203} (\bibinfo {year}
  {2013})}\BibitemShut {NoStop}%
\bibitem [{\citenamefont {Ramachandran}\ \emph {et~al.}(2017)\citenamefont
  {Ramachandran}, \citenamefont {Andreanov},\ and\ \citenamefont
  {Flach}}]{ramachandran17}%
  \BibitemOpen
  \bibfield  {author} {\bibinfo {author} {\bibfnamefont {A.}~\bibnamefont
  {Ramachandran}}, \bibinfo {author} {\bibfnamefont {A.}~\bibnamefont
  {Andreanov}},\ and\ \bibinfo {author} {\bibfnamefont {S.}~\bibnamefont
  {Flach}},\ }\bibfield  {title} {\bibinfo {title} {Chiral flat bands:
  Existence, engineering, and stability},\ }\href
  {https://doi.org/10.1103/PhysRevB.96.161104} {\bibfield  {journal} {\bibinfo
  {journal} {Phys. Rev. B}\ }\textbf {\bibinfo {volume} {96}},\ \bibinfo
  {pages} {161104} (\bibinfo {year} {2017})}\BibitemShut {NoStop}%
\bibitem [{\citenamefont {Khomeriki}\ and\ \citenamefont
  {Flach}(2016)}]{khomeriki16}%
  \BibitemOpen
  \bibfield  {author} {\bibinfo {author} {\bibfnamefont {R.}~\bibnamefont
  {Khomeriki}}\ and\ \bibinfo {author} {\bibfnamefont {S.}~\bibnamefont
  {Flach}},\ }\bibfield  {title} {\bibinfo {title} {{L}andau-{Z}ener {B}loch
  oscillations with perturbed flat bands},\ }\href
  {https://doi.org/10.1103/PhysRevLett.116.245301} {\bibfield  {journal}
  {\bibinfo  {journal} {Phys. Rev. Lett.}\ }\textbf {\bibinfo {volume} {116}},\
  \bibinfo {pages} {245301} (\bibinfo {year} {2016})}\BibitemShut {NoStop}%
\bibitem [{\citenamefont {Bouzerar}\ and\ \citenamefont
  {Mayou}(2021)}]{bouzerar21}%
  \BibitemOpen
  \bibfield  {author} {\bibinfo {author} {\bibfnamefont {G.}~\bibnamefont
  {Bouzerar}}\ and\ \bibinfo {author} {\bibfnamefont {D.}~\bibnamefont
  {Mayou}},\ }\bibfield  {title} {\bibinfo {title} {Quantum transport in flat
  bands and supermetallicity},\ }\href
  {https://doi.org/10.1103/PhysRevB.103.075415} {\bibfield  {journal} {\bibinfo
   {journal} {Phys. Rev. B}\ }\textbf {\bibinfo {volume} {103}},\ \bibinfo
  {pages} {075415} (\bibinfo {year} {2021})}\BibitemShut {NoStop}%
\bibitem [{\citenamefont {Leykam}\ \emph {et~al.}(2017)\citenamefont {Leykam},
  \citenamefont {Bodyfelt}, \citenamefont {Desyatnikov},\ and\ \citenamefont
  {Flach}}]{leykam17}%
  \BibitemOpen
  \bibfield  {author} {\bibinfo {author} {\bibfnamefont {D.}~\bibnamefont
  {Leykam}}, \bibinfo {author} {\bibfnamefont {J.~D.}\ \bibnamefont
  {Bodyfelt}}, \bibinfo {author} {\bibfnamefont {A.~S.}\ \bibnamefont
  {Desyatnikov}},\ and\ \bibinfo {author} {\bibfnamefont {S.}~\bibnamefont
  {Flach}},\ }\bibfield  {title} {\bibinfo {title} {Localization of weakly
  disordered flat band states},\ }\href
  {https://doi.org/10.1140/epjb/e2016-70551-2} {\bibfield  {journal} {\bibinfo
  {journal} {The European Physical Journal B}\ }\textbf {\bibinfo {volume}
  {90}},\ \bibinfo {pages} {1} (\bibinfo {year} {2017})}\BibitemShut {NoStop}%
\bibitem [{\citenamefont {Sutherland}(1986)}]{sutherland86}%
  \BibitemOpen
  \bibfield  {author} {\bibinfo {author} {\bibfnamefont {B.}~\bibnamefont
  {Sutherland}},\ }\bibfield  {title} {\bibinfo {title} {Localization of
  electronic wave functions due to local topology},\ }\href
  {https://doi.org/10.1103/PhysRevB.34.5208} {\bibfield  {journal} {\bibinfo
  {journal} {Phys. Rev. B}\ }\textbf {\bibinfo {volume} {34}},\ \bibinfo
  {pages} {5208} (\bibinfo {year} {1986})}\BibitemShut {NoStop}%
\bibitem [{\citenamefont {Inui}\ \emph {et~al.}(1994)\citenamefont {Inui},
  \citenamefont {Trugman},\ and\ \citenamefont {Abrahams}}]{inui94}%
  \BibitemOpen
  \bibfield  {author} {\bibinfo {author} {\bibfnamefont {M.}~\bibnamefont
  {Inui}}, \bibinfo {author} {\bibfnamefont {S.~A.}\ \bibnamefont {Trugman}},\
  and\ \bibinfo {author} {\bibfnamefont {E.}~\bibnamefont {Abrahams}},\
  }\bibfield  {title} {\bibinfo {title} {Unusual properties of midband states
  in systems with off-diagonal disorder},\ }\href
  {https://doi.org/10.1103/PhysRevB.49.3190} {\bibfield  {journal} {\bibinfo
  {journal} {Phys. Rev. B}\ }\textbf {\bibinfo {volume} {49}},\ \bibinfo
  {pages} {3190} (\bibinfo {year} {1994})}\BibitemShut {NoStop}%
\bibitem [{\citenamefont {Souza}\ \emph {et~al.}(2017)\citenamefont {Souza},
  \citenamefont {Andrade}, \citenamefont {Ara\'ujo}, \citenamefont {Vezzani},\
  and\ \citenamefont {Herrmann}}]{souza17}%
  \BibitemOpen
  \bibfield  {author} {\bibinfo {author} {\bibfnamefont {A.~M.~C.}\
  \bibnamefont {Souza}}, \bibinfo {author} {\bibfnamefont {R.~F.~S.}\
  \bibnamefont {Andrade}}, \bibinfo {author} {\bibfnamefont {N.~A.~M.}\
  \bibnamefont {Ara\'ujo}}, \bibinfo {author} {\bibfnamefont {A.}~\bibnamefont
  {Vezzani}},\ and\ \bibinfo {author} {\bibfnamefont {H.~J.}\ \bibnamefont
  {Herrmann}},\ }\bibfield  {title} {\bibinfo {title} {How the site degree
  influences quantum probability on inhomogeneous substrates},\ }\href
  {https://doi.org/10.1103/PhysRevE.95.042130} {\bibfield  {journal} {\bibinfo
  {journal} {Phys. Rev. E}\ }\textbf {\bibinfo {volume} {95}},\ \bibinfo
  {pages} {042130} (\bibinfo {year} {2017})}\BibitemShut {NoStop}%
\bibitem [{\citenamefont {Horodecki}\ \emph {et~al.}(1999)\citenamefont
  {Horodecki}, \citenamefont {Horodecki},\ and\ \citenamefont
  {Horodecki}}]{horodecki99}%
  \BibitemOpen
  \bibfield  {author} {\bibinfo {author} {\bibfnamefont {M.}~\bibnamefont
  {Horodecki}}, \bibinfo {author} {\bibfnamefont {P.}~\bibnamefont
  {Horodecki}},\ and\ \bibinfo {author} {\bibfnamefont {R.}~\bibnamefont
  {Horodecki}},\ }\bibfield  {title} {\bibinfo {title} {General teleportation
  channel, singlet fraction, and quasidistillation},\ }\href
  {https://doi.org/10.1103/PhysRevA.60.1888} {\bibfield  {journal} {\bibinfo
  {journal} {Phys. Rev. A}\ }\textbf {\bibinfo {volume} {60}},\ \bibinfo
  {pages} {1888} (\bibinfo {year} {1999})}\BibitemShut {NoStop}%
\end{thebibliography}%


\end{document}

%It is easy to show that the eigenvalue equation $H_{\mathrm{eff}}\ket{\epsilon_{i}^{\pm}}=\pm\epsilon_{i}\ket{\epsilon_{i}^{\pm}}$, assuming $x_{S}=x_{R}=1/\sqrt{2}$, is satisfied if $\eta_{1}=\eta_{N}=\eta$ and $\epsilon_1=g\sqrt{\eta+\sum_{k}\mu_{1,k}\mu_{N,k}}$. Following the same reasoning, we obtain $\epsilon_2=g\sqrt{\eta-\sum_{k}\mu_{1,k}\mu_{N,k}}$.


