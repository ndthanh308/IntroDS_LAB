\section{Related Work}

\textbf{Scientific Machine Learning Datasets.} \update{There have been several efforts to develop benchmark datasets for scientific machine learning tasks \cite{PDEBench2022,simphys2021, chung2022blastnet, bonnet2022airfrans, Stachenfeld2021LearnedCM, hersbach2020era5}. Notably, the ERA5 atmospheric reanalysis dataset \cite{hersbach2020era5}, curated by the European Center for Medium-Range Weather Forecasting (ECMWF) provides hourly estimates of a large number of atmospheric, land, and oceanic climate variables since 1940. It is the most popular publicly available source for weather forecasting, facilitating the training of neural weather models such as FourCastNet \cite{pathak2022fourcastnet}, GraphCast \cite{lam2022graphcast}, and ClimaX \cite{nguyen2023climax}. PDEBench \cite{PDEBench2022} provides an impressive collection of datasets for 11 PDEs commonly encountered in computational fluid dynamics. Boundary conditions in scientific simulations play a crucial role in capturing the dynamics of the underlying physical systems. The majority of datasets in PDEBench utilize periodic boundary conditions. Although some datasets encompass Neumann or Dirichlet boundary conditions, none consider a combination of both which presents a noteworthy gap in accurately modeling real-world scenarios. 
Another challenging problem is the modeling of turbulent Kolmogorov flows and the dataset generated using JAX-CFD \cite{kochkov2021machine} is gaining popularity in benchmarking neural flow models \cite{lippe2023modeling, sun2023neural}.
BlastNet \cite{chung2022blastnet} generated using DNS solver, S3D \cite{chen2009terascale} focuses on simulating the behavior of a single fluid phase solving for compressible fluid dynamics, combustion, and heat transfer.
AirfRANS \cite{bonnet2022airfrans} is a dataset for studying the 2D incompressible steady-state Reynolds-Averaged Navier–Stokes equations over airfoils. Current datasets have made commendable strides in addressing single- and multiphysics scenarios, and provide a valuable foundation for developing and evaluating SciML algorithms. Nonetheless, their scope falls short in capturing the range of behaviors and phenomena encountered in phase change physics.}

In contrast, BubbleML focuses on capturing the complex dynamics and physics associated with multiphase phenomena, particularly in the context of phase change simulations. Unlike many existing datasets that predominantly utilize a single type of boundary condition, BubbleML incorporates a combination of Dirichlet and Neumann boundary conditions \cite{StraussPDE}. This inclusion enables researchers to explore and model scenarios where multiple boundary conditions coexist, enhancing the realism and applicability of the dataset. Moreover, the presence of ``jump'' conditions along the liquid-vapor interface adds an additional layer of complexity. These conditions arise due to surface tension effects and require careful modeling to accurately capture the interface behavior \cite{DHRUV2019, DHRUV2021}. By incorporating such challenges, BubbleML provides a realistic and demanding testbed for ML models.

\textbf{Optical Flow Datasets.} Optical flow estimation, a classical ill-posed problem \cite{10.1145/212094.212141} in image processing, has witnessed a shift from traditional methods to data-driven deep learning approaches. 

Middlebury \cite{4408903} is a dataset with dense ground truth for small displacements, while KITTI2015 \cite{menze2015joint} provides sparse ground truth for large displacements in real-world scenes. MPI-Sintel \cite{10.1007/978-3-642-33783-3_44} offers synthetic data with very large displacements, up to 400 pixels per frame.
However, these datasets are relatively small for training deep neural networks. FlyingChairs \cite{dosovitskiy2015flownet}, a large synthetic dataset, contains around 22,000 image pairs generated by applying affine transformations to rendered chairs on random backgrounds. FlyingThings3D \cite{mayer2016large} is another large synthetic dataset with approximately 25,000 stereo frames of 3D objects on different backgrounds. 

While these datasets have been instrumental in advancing data-driven optical flow methods, they primarily focus on rigid object motion in visual scenes and do not address the specific challenges posed by multiphase simulations. Efforts have been made to capture non-rigid motion in nature, such as piece-wise rigid motions seen in animals \cite{LeARAP2018}. In boiling, the non-rigid dynamics of bubbles and the motion of liquid-vapor interfaces play a crucial role in the distribution and transfer of thermal energy. The BubbleML dataset provides a unique opportunity to explore and develop optical flow algorithms tailored to such dynamics. Unlike existing datasets, it offers a diverse range of bubble behaviors, including merging, growing, splitting, and complex interactions (see Figure \ref{fig:dataset-variables}). As a result, BubbleML fills a gap by providing challenging scenarios that involve phase change dynamics. The ability to accurately predict and forecast bubble dynamics has practical implications in various fields.

