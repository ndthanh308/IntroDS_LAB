\section{BubbleML: A Multiphase Multiphysics Dataset for ML}
\vspace{-1em}
In this section, we start by introducing the preliminary concepts underlying the SciML learning problem and give insights into the types of simulations and PDEs involved in this domain. Then, we present an overview of the dataset pipeline along with its validation against real-world experiments. 

\subsection{Preliminaries}
\label{sec:bvp-bc}
A common application for SciML is approximating the solution of \textit{boundary value problems} (BVPs). BVPs are widely used to model various physical phenomena, including fluid dynamics, heat transfer, electromagnetics, and quantum mechanics \cite{StraussPDE, stakgold2000boundary, doi:10.1137/0135035, 5357513}. BVPs take the form: $L[u(x)] = f(x), x \in \Omega$ and $B[u(x)] = g(x), x \in \partial \Omega$. 
The goal is to determine the vector-valued solution function, $u$. $x$ is a point in the domain $\Omega$ and may include a temporal component.
The boundary of the domain is denoted as $\partial\Omega$. The governing equation is described by the PDE operator $L$, and the forcing function is denoted as $f$. The \emph{boundary condition} (BC) is given by the boundary operator $B$ and the boundary function $g$. $B[u] = g$ ensures the existence and uniqueness of the solution. 

There are three common types of BCs: periodic, Dirichlet, and Neumann. Periodic BCs enforce the equality of the solution at distinct points in the domain: $u(x_1) = u(x_2)$. Dirichlet BCs specify the values of the solution on the boundary: $u(x) = g(x)$. Neumann BCs enforce constraints on the derivatives of the solution: $\partial_{n}u(x) = g(x)$ \cite{StraussPDE}. As seen in Figure \ref{fig:dataset-variables}, BubbleML combines both Dirichlet (no-slip walls, heater, inflow) and Neumann (outflow) boundaries, which impose constraints on flow and temperature dynamics. Additionally, the ``jump conditions'' that govern the transitions between the liquid and vapor phases use Dirichlet and Neumann boundaries \cite{DHRUV2019}.

\subsection{Overview of PDEs and Flash-X Simulation} \label{sec:pde-sim-overview}
A comprehensive description of the simulations is well beyond the scope of this paper and can be found in \cite{DHRUV2019,DHRUV2021}. We provide a concise description here as knowledge of the PDEs is important when training physics-informed models. 

The liquid ($l$) and vapor ($v$) phases of a boiling simulation are characterized by differences in fluid and thermal properties: density, $\rho$; viscosity, $\mu$; thermal diffusivity, $\alpha$; and thermal conductivity $k$. The phases are tracked using a level-set function,
$\phi$, which is positive inside the vapor and negative in the liquid. $\phi=0$ provides implicit representation of the liquid-vapor interface, $\Gamma$ (see Figure \ref{fig:dataset-variables}). The transport equations are non-dimensionalized and scaled using the values in liquid and are given as, 
%
\begin{subequations} \label{eq:phase2}
\begin{equation} \label{eq:momt2}
\frac{\partial \vec u}{\partial t} + \vec u \boldsymbol{\cdot}  \nabla \vec u = - \frac{1}{\rho'}\nabla P + \nabla \boldsymbol{\cdot} \Big[\frac{\mu'}{\rho'}\frac{1}{\text{Re}}\nabla \vec u \Big] + \frac{\vec g}{\text{Fr}^2} + {\vec{S}^\Gamma_u} + {S}^{\Gamma}_P
\end{equation}
\begin{equation} \label{eq:temp2}
\frac{\partial T}{\partial t} + \vec u \boldsymbol{\cdot}  \nabla T = \nabla \boldsymbol{\cdot} \Big[\frac{\alpha'}{\text{Re}\:\text{Pr}}\nabla T \Big] + {S}^{\Gamma}_T
\end{equation}
\end{subequations}
%

where, $\vec u$, is the velocity, ${P}$ is the pressure, and ${T}$ is the temperature everywhere in the domain. The Reynolds number ($\text{Re}$), Froude number ($\text{Fr}$), and Prandtl Number ($\text{Pr}$) are constants set for each simulation. Scaled fluid properties like, $\rho'$, represent the local value of the phase scaled by the corresponding value in liquid. Therefore, $\rho'$ is $1$ in liquid phase, and $\rho_v/\rho_l$ for vapor phase. The effect of surface tension is modeled using Weber number ($\text{We}$) and incorporated by a sharp pressure jump, ${S}^{\Gamma}_P$, at the liquid-vapor interface, $\Gamma$. The effects of evaporation and saturation conditions on velocity and temperature, ${\vec{S}^\Gamma_u}$, and ${S}^{\Gamma}_T$, are modeled using a ghost fluid method \cite{DHRUV2019}. For a more detailed discussion of non-dimensional parameters and values, we refer the reader to Appendix \ref{asec:non-dim}. 
 
The continuity equation is given by, $\nabla \cdot \vec{u} = -\dot{m}\nabla (\rho')^{-1} \cdot \vec{n}$,
where the mass transfer $\dot{m}$ is computed using local temperature gradients in liquid and vapor phase, $\dot{m} = \text{St}(\text{Re}\:\text{Pr})^{-1} 
\big[
    \nabla T_l \cdot {\vec{n}^\Gamma} -
    k'\nabla T_v\cdot {\vec{n}^\Gamma}
\big]$
%
%
where, ${\vec{n}^\Gamma}$ is the surface normal vector to the liquid-vapor interface. The Stefan number $\text{St}$, is another constant defined for the simulation and depends on the the temperature scaling given by, $\Delta T = T_{wall} - T_{bulk}$, and latent heat of evaporation, $h_{lv}$. Simulation data is scaled to dimensional values using the characteristic length $l_0$, velocity $u_0$, and temperature $({T-T_{bulk}})/{\Delta T}$ scale. Temporal integration is implemented using a fractional step predictor-corrector formulation to enforce incompressible flow constraints. The solver has been extensively validated and demonstrates an overall second-order accuracy in space \cite{DHRUV2019,DHRUV2021}.

In thermal science, \emph{heat flux} measured as the integral of the temperature gradient across the heater surface ($\overline{q} = \partial T/ \partial y$) serves as a vital indicator of boiling efficiency. It reflects the contribution from multiple sub-processes such as conduction, convection, microlayer evaporation, and bubble induced turbulence. Identifying and managing each sub-process's impact to enhance $\overline{q}$ is an open challenge \cite{ebadian2011review, hughes2021status}. \emph{Critical heat flux} (CHF) signifies peak heat flux before a sharp drop in efficiency occurs due to the formation of a vapor barrier (see Figure  \ref{fig:validation-data}b). It is arguably the most important design and safety parameter for any heat-flux controlled boiling application \cite{liang2018pool}. Accurate heat flux modeling and prediction of boiling crisis are paramount for the reliability of heat transfer systems \cite{rassoulinejad2021deep, zhao2020prediction, sinha2021deep}.

The simulations in this study are implemented within the Flash-X framework \cite{DUBEY2022, DHRUV2019}, and a dedicated environment is provided for running new simulations \footnote{\label{git_simul} \url{https://github.com/Lab-Notebooks/Outflow-Forcing-BubbleML}}. The repository contains example configuration files for various multiphase simulations, including those used in this dataset. To ensure reproduciblity, a lab notework has been designed that organizes each study using configuration files for data curation. The lab notebook and Flash-X source code are open-source to allow for community development and contribution, enabling creation of new datasets beyond the scope of this paper. The simulation archives store HDF5 output files and bash scripts that document software environment and repository tags for reproducibility. The lab notebook also provides an option to package Flash-X simulations as standalone Docker/Singularity containers which can be deployed on cloud and supercomputing platforms without the need for installing third-party software dependencies. The latter is ongoing work towards software sustainability \cite{flashx_workflows}.

\subsection{Dataset Overview}

The study encompasses two types of boiling namely, pool boiling and flow boiling. Pool boiling represents fluid confined in a tank above a heater, resembling scenarios like cooling nuclear waste. The BCs for pool boiling include walls on the left and right, an outlet at the top, and a heater at the bottom. In contrast, flow Boiling models water flowing through a channel with a heater, simulating liquid cooling of data center GPUs. There is an inlet BC modeling flow into the system and an outlet. The fluid used for the simulations is FC-72 (perfluorohexane), an electrically insulating and stable fluorocarbon-based fluid commonly used for cooling applications in electronics operating at low temperatures (ranging from $50^\circ$C to $100^\circ$C). To explore various phenomena, different parameters such as heater temperature, liquid temperature, inlet velocity, and gravity scale are adjusted in each simulation. A summary of the dataset is presented in Table~\ref{tab:datset-summary}. Appendix \ref{asec:sim-bc-details} provides detailed illustrations of the boundary conditions and descriptions of each simulation for reference.

\begin{table}[h]
  \caption {\update{Summary of BubbleML datasets and their parameters. $\Delta t$ is the temporal resolution in non-dimensional time ($\Delta t = 1 = 0.008$ seconds). For rationale behind the parameter choices, refer to appendix \ref{appendix:bml-param-choices}. PB: pool boiling. FB: flow boiling.}}
  \label{tab:datset-summary}
  \centering
  \newcommand{\results}[8]{ #1 & #2 & #3 & #5 & #4 & #7 & #6 & #8\\}

  \begin{tabular}{llllllll}
    \toprule
    Dim & Type - Physics & Sims & Domain  & \multicolumn{2}{c}{Resolution} & Timesteps  & \update{Size} \\
     & & & ($mm^d$) & Spatial & $\Delta t$ & & \update{(GB)} \\
    \midrule
     \results{2D}{PB - Single Bubble}{1}{$192 \times 288$}{$4.2 \times 6.3$}{500}{0.5}{0.5}
     \results{2D}{PB - Saturated}{13}{$512 \times 512$}{$11.2 \times 11.2$}{200}{1}{24.2}
     \results{2D}{PB - Subcooled}{10}{$384 \times 384$}{$8.4 \times 8.4$}{200}{1}{10.3}
     \results{2D}{PB - Gravity}{9}{$512 \times 512$}{$11.2 \times 11.2$}{200}{1}{16.5}
     \results{2D}{FB - Inlet Velocity}{7}{$1344 \times 160$}{$29.4 \times 3.5$}{200}{1}{10.7}
     \results{2D}{FB - Gravity}{6}{$1600 \times 160$}{$35 \times 3.5$}{200}{1}{10.9}
     \results{2D}{PB - \update{Subcooled$_\text{0.1}$}}{15}{$384 \times 384$}{$8.4 \times 8.4$}{2000}{0.1}{155.1}
     \results{2D}{PB - \update{Gravity$_\text{0.1}$}}{9}{$512 \times 512$}{$11.2 \times 11.2$}{2000}{0.1}{163.8}
     \results{2D}{FB - \update{Gravity$_\text{0.1}$}}{6}{$1600 \times 160$}{$35 \times 3.5$}{2000}{0.1}{108.6}
     \results{3D}{PB - Earth Gravity}{1}{$400^3$}{$8.75^3$}{57}{1}{122.2}
     \results{3D}{PB - ISS Gravity}{1}{$400^3$}{$8.75^3$}{29}{1}{62.6}
     \results{3D}{FB - Earth Gravity}{1}{$1600 \times 160^2$}{$35 \times 3.5^2$}{55}{1}{93.9}
    \bottomrule
  \end{tabular}
  \vspace{-0.5em}
\end{table}

BubbleML stores simulation output in HDF5 files. Each HDF5 file corresponds to the state of a simulation at a specific instant in time and can be directly loaded into popular tensor types (e.g., PyTorch tensors or NumPy arrays) using BoxKit. BoxKit is a custom Python API designed for efficient management and scalability of block-structured simulation datasets \cite{HDF5, boxkit}. It leverages multiprocessing and cache optimization techniques to improve read/write efficiency of data between disk and memory. Figure \ref{fig:simulation-schematic} provides an example of a boiling dataset and the corresponding workflow for enabling downstream tasks like scientific machine learning and optical flow. By operating on simulation data in manageable chunks that fit into memory, BoxKit significantly improves computational performance, particularly when handling large quantities of datasets.

% Figure environment removed

Each simulation within the BubbleML dataset tracks the velocities in the x and y directions, temperature, and a signed distance function (SDF), $\phi$, which represents the distance from the bubble interface. The SDF can be used to get a mask of the bubble interfaces or determine if a point is in the liquid or vapor phase. These variables are stored in HDF5 datasets. For instance, the temperature is stored in a tensor with a shape $t\times x\times y\times z$ , which allows indexing with $xyz$-spatial coordinates or time. For 2D simulation datasets, the shape becomes $t\times x\times y$. The HDF5 files also include any constants or runtime parameters provided to the simulation.  Some of these parameters, such as thermal conductivity or Reynolds number, are constants used in the PDEs that govern the system. The inclusion of these variables and parameters in the dataset enables comprehensive analysis and modeling of the boiling phenomena.

BubbleML follows the FAIR data principles \cite{wilkinson2016fair} as outlined in appendix \ref{appendix:bml-fair}.
To ensure accuracy of scientific simulations, it is also essential to validate against experimental observations due to inherent approximations in numerical solvers and simplified models of real-world phenomena. Appendix \ref{appendix:data_validation} provides a comprehensive validation of the BubbleML dataset.