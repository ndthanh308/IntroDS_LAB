\section{Simulation Details}
\label{asec:sim-bc-details}
\subsection{Multiphase Simulations}
 Numerical simulations of multiphase flows with phase changes have been studied using various techniques to model the behavior at the liquid-vapor interface and track its evolution over time.
 Two commonly used methods for handling boundary conditions related to surface tension and evaporation are the ghost fluid method (GFM) and the continuum surface force method (CSF). The GFM enforces a sharp jump in pressure, velocity, and temperature across the interface, while the CSF diffuses the forcing within the vicinity of the interface for a smoother transition. The choice between these approaches involves a trade-off between accuracy and stability, with GFM offering higher accuracy but lower stability compared to CSF. Interface tracking is typically achieved implicitly using level-set or volume of fluid (VOF) techniques.

Researchers have employed these methods to study and model various aspects of multiphase flows with phase changes. For example, Gibou et al. \cite{Gibou2007} used a level-set method with sharp interfacial jump conditions within the framework of the GFM to model homogeneous two-dimensional evaporation and film boiling. Son and Dhir \cite{Son2007, Son2008} extended this approach to perform heterogeneous pool boiling calculations involving single and multiple bubbles. Majority of their initial work focused on model development and verification using two-dimensional (2D) simulations, since real world three-dimensional (3D) calculations were expensive due to limitations of the software framework.

Efforts have also been made to perform high-fidelity 3D simulations of pool boiling by combining different techniques.  Yazadani et al. \cite{Yazdani2016} conducted critical heat flux (CHF) calculations on earth gravity using a combination of VOF and CSF methods. Sato et al. \cite{SATO2013127,SATO2018876}, on the other hand used the level-set method in combination with CSF for their simulations. These studies highlighted the computational cost associated with boiling simulations which had to be mitigated by performing low resolution calculations. More recently, Dhruv et al. \cite{DHRUV2019,DHRUV2021} used a combination of level-set and GFM methods for gravity scaling analysis of pool boiling at finer resolution than previous studies using adaptive mesh refinement (AMR) within the framework of FLASH. These simulations were applied to study effects of gravity on boiling heat transfer which lead to verification of experiment based heat flux models and enabled the quantification of turbulent heat flux associated with bubble dynamics during bubbly and slug flow \cite{DHRUV2021}. The implementation of multiphase models within FLASH has transitioned to Flash-X, which leverages state-of-art AMR techniques and heterogeneous supercomputing architectures to significantly improve performance of boiling calculations. The BubbleML dataset is curated from simulations carried out with the Flash-X framework.

An important note is that simulations still heavily rely on experimental observations to determine input conditions such as nucleation site distribution, bubble nucleation frequency, and solid-liquid-vapor contact angle dynamics \cite{DHRUV2019}. As a result, simulations serve as an effective tool to understand and quantify trends in boiling regimes,  rather than attempting to replicate experiments precisely. This opens up an opportunity for the integration of data-driven ML techniques, which can leverage diverse datasets to make informed predictions for boiling phenomena.

\subsection{Fluid parameters}
The input configuration file of a simulation requires the inclusion of certain parameters, which remain constant for a specific fluid. For our simulations using FC-72 (Perfluorohexane, $C_6F_{14}$), the non-dimensional values for this fluid are provided in Table \ref{tab:fluid_non_dimparams}. It is important to note that parameters corresponding to any specific real-world fluid must be converted to these non-dimensional values before being input into the simulation configuration files. This conversion allows for consistent and standardized representation of fluid properties in the simulation, enabling accurate and meaningful results to be obtained.

\begin{table}[h]
    \caption{Non-dimensional constants for FC-72, the fluid used in BubbleML}
    \centering
    \begin{tabular}{l l c}
    \toprule
    Parameter & Variable Name & Non-dimensional Value\\
    \midrule
    Inverse Reynolds Number, $\frac{1}{Re}$ & ins\_invReynolds & 0.0042 \\
    Non-dimensional density, $\rho^*$ & mph\_rhoGas & 0.0083 \\
    Non-dimensional viscosity, $\mu^*$ & mph\_muGas & 1.0 \\
    Non-dimensional thermal conductivity, $k^*$& mph\_thcoGas & 0.25\\
    Non-dimensional specific heat capacity, $C_p^*$ & mph\_CpGas & 0.83 \\
    Inverse Weber Number, $\frac{1}{We}$ & mph\_invWeber & 1.0 \\
    Prandtl Number, $Pr$ & ht\_Prandtl & 8.4 \\
    \bottomrule
    \end{tabular}
    \label{tab:fluid_non_dimparams}
\end{table}