\section{Additional BubbleML details}
\label{appendix:bml}

\subsection{Dataset URLs and Links}
\label{appendix:bml-links}

\textbf{\update{Dataset:}} \update{Links to download all the BubbleML datasets in Table \ref{tab:datset-summary} are available on the \href{https://github.com/HPCForge/BubbleML}{GitHub} homepage. The dataset homepage is hosted at \href{https://hpcforge.github.io/BubbleML/}{https://hpcforge.github.io/BubbleML/}. All future versions of the dataset with the new links will be uploaded here.}

\textbf{\update{Code:}} \update{Code for training and evaluation of all the benchmark models are available within the same \href{https://github.com/HPCForge/BubbleML}{GitHub} repository.}

\textbf{\update{Model Weights:}}
\update{Weights for all the benchmark models are available in the \href{https://github.com/HPCForge/BubbleML/tree/main/model-zoo}{model zoo}. All the relevant benchmark results can be accessed on the same page.}

\textbf{\update{DOI:}}
\update{The BubbleML dataset has a DOI from Zenodo: \href{https://zenodo.org/record/8039786}{https://doi.org/10.5281/zenodo.8039786}.}

\textbf{\update{Documentation:}}
\update{We provide detailed descriptions for the fields tracked in the simulation data: \href{https://github.com/HPCForge/BubbleML/blob/main/bubbleml\_data/DOCS.md}{https://github.com/HPCForge/BubbleML/blob/main/bubbleml\_data/DOCS.md}. The documentation discusses the layout of the data, important metadata, and some potential pitfalls.}


\textbf{\update{Tutorials:}}
\update{We provide example Jupyter notebook to enable reproducibility of the benchmarks and findings in this study: \href{https://github.com/HPCForge/BubbleML/tree/main/examples}{https://github.com/HPCForge/BubbleML/tree/main/examples}. These tutorials cover accessing simulation data, dataset schema, querying simulation parameters, and training a Fourier Neural Operator---discussed in Section \ref{appendix:sciml-benchmark-models}---using PyTorch.} 

\subsection{Maintenance and Long Term Preservation}
\label{appendix:bml-fair}
\update{
The authors of BubbleML are committed to maintaining and preserving this dataset. It is likely that the authors will make extensions as we use BubbleML for our own research. Ongoing maintenance also encompass tracking and resolving issues identified by the broader community after release. User feedback will be closely monitored via the GitHub issue tracker. All data is hosted on AWS, which guarantees reliable and stable storage. Depending on usage we may migrate to archival storage for long-term preservation.
}

\update{\textbf{Findable:} All data is stored in an Amazon AWS S3 instance. All present and future data will share a global and persistent DOI \href{https://zenodo.org/record/8039786}{https://doi.org/10.5281/zenodo.8039786}}.

\update{\textbf{Accessible:} All data and descriptive metadata can be downloaded from the public links listed on the GitHub homepage. For added convenience, we provide a bash script for users to download the entire dataset at once.}

\update{\textbf{Interoperable:} All BubbleML data is provided in the form of standard HDF5 files that can be read using many common libraries, such as h5py for Python. Relevant metadata is stored with each simulation.}

\update{\textbf{Reusable:} BubbleML is released under the  Creative Commons Attribution 4.0 International License.}

\subsection{Justification of simulation parameters}
\label{appendix:bml-param-choices}

\update{We discuss the rationale guiding the selection of simulation parameters in Table \ref{tab:datset-summary}, addressing both the quantity of simulations and their resolutions/time-steps.}
 
\update{
\textbf{Number of simulations:} The studies performed in BubbleML encompass diverse two-phase boiling phenomena. The number of simulations is determined based on the distribution of the variable being studied, keeping other factors constant. For example, in the case of saturated boiling, we choose a wall temperature range from $60^{\circ} C$ to $120^{\circ} C$, with uniform intervals of $5^\circ C$, resulting in 13 simulations. This range captures the boiling transition from the bubbly regime to the slug regime as the heat flux approaches criticality. However when studying the effects of gravity, the scaling factor, $\text{Fr}^{-2}$ (Fr is the Froude Number), is chosen from the range $10^{-4}$ to 1 using a logarithmic scale resulting in 9 simulations. This scale is essential to cover the vastly different gravity conditions spanning from the Earth's surface to the International Space Station.
}

\update{
\textbf{Domain size:} Exploring phenomena across varying scales and geometries is common practice. Such variations in sizes enable the exploration of a broad spectrum of heat transfer dynamics, bubble dynamics, and phase change behaviors. The domain and heater sizes are chosen to replicate typical ranges in experiments \cite{Raj2012, Raj} while also taking into account the computational costs of simulations.
}

\update{
\textbf{Spatial resolution:} The domain resolution depends on the grid size of an individual 2D block. A block with spatial dimensions of $0.5 \times 0.5$ (in non-dimensionalized units) is discretized into a grid of size $16 \times 16$. This resolution is determined based on grid sensitivity studies conducted for a single bubble case to ensure high-fidelity simulations \cite{DHRUV2019}. This results in the spatial resolution sizes in Table \ref{tab:datset-summary}.
}

\update{
\textbf{Temporal resolution and timesteps:} The temporal discretization in the majority of BubbleML datasets is set at 1 non-dimensional unit, equivalent to 0.008 seconds for FC-72. Additionally, we include several datasets with a finer discretization of 0.1 non-dimensional time. In both cases, we intentionally chose a discretization that is much larger than what the CFL-condition mandates for the Flash-X solver. A significant advantage of neural PDE solvers is their ability to maintain reasonable approximations while taking much larger timesteps than traditional numerical simulations. However, this choice introduces a trade-off between dataset size and ease of use. The datasets with a 1 time unit discretization are potentially more accessible, but might present challenges in achieving accurate results. Conversely, the datasets with a 0.1 time unit discretization are less accessible and may require distributed training, yet they are likely to achieve more accurate results due to both more training data (i.e., more timesteps) and presence of fine-grained physics.}

\subsection{Dataset Validation}
\label{appendix:data_validation}
\paragraph{Saturated and subcooled boiling.} We first validate two distinct boiling phenomena. Saturated boiling refers to the state of a liquid when it reaches its boiling point, known as the saturation temperature, $T_{sat}$. At this temperature, the liquid is in equilibrium with its vapor phase, and bubbles start forming at the heated surface. The liquid is on the verge of vaporization, and any further increase in temperature can lead to the formation of vapor bubbles. This bubble formation is called nucleate boiling; a far more effective way to transfer heat than natural convection on its own. On the other hand, subcooled boiling is a complex process with evaporation and condensation occurring simultaneously. The heat flux imposed on the wall produces a thermal layer around it in which bubbles may nucleate and grow. However, condensation occurs as a bubble migrates into the bulk liquid region with temperature under saturation point,  $T_{bulk} < T_{sat}$. Experimental findings \cite{boilingcurvefc72} indicate that the heat flux increases linearly with the heater temperature, $T_{wall}$ until reaching CHF, marking the transition from nucleate pool boiling to film boiling. 

% Figure environment removed

Figure \ref{fig:validation-data} presents the bubble dynamics in two different regimes of saturated pool boiling: onset of nucleate boiling (ONB) and CHF. ONB occurs at low wall superheat and exhibits structured bubbly flow with consistent shape and size of departing bubbles from the heater surface. In contrast, the CHF regime is characterized by chaotic slug flow.  Remarkably, the heat flux, $\overline{q}$ at the heater surface for both subcooled and saturated boiling shown in Figure \ref{fig:validation-data}c closely match the experimental boiling curve, specifically for Si(100) surface \cite{boilingcurvefc72}. Beyond CHF, the heat flux reaches a plateau, indicating a stasis marked by presence of large pockets of vapor cover on the heater surface as shown in Figure  \ref{fig:validation-data}b. 

% Figure environment removed

\paragraph{Boiling at low gravity.} Boiling is the most efficient mode of heat transfer on Earth and serves as a cooling mechanism for various thermal applications. However, in low gravity environments such as the International Space Station (ISS), the dynamics of bubble growth, merger, and departure, which significantly impact thermal efficiency, are influenced by the interplay between surface tension and gravity. Quantifying these effects is crucial for developing phase change heat transfer systems in such environments.

Figure \ref{fig:gravity-scaling} provides an overview of simulations that were conducted to compare against the gravity-based heat flux model proposed in \cite{Raj2012,Raj}. 

These simulations cover a range of gravity levels, $Fr = 1 - 100$. The gravity scaling, $a/g = \frac{1}{Fr^2}$, is used to scale values relative to the Earth gravity ($g$). Gravity separates pool boiling into two distinct regimes: buoyancy dominated boiling (BDB) and surface tension dominated boiling (SDB). The transitional acceleration, $a_{trans}$, which depends on the size of the heater, serves as the boundary between these two regimes.

In BDB, bubbles periodically detach from the heater surface when buoyancy takes over surface tension.  Figure \ref{fig:gravity-scaling} illustrates the temporal evolution of temperature and the liquid-vapor interface during bubble departure events for $a/g=1$. After departing from the surface, the bubbles undergo condensation due to subcooling, and generate vortices that gradually dissipate as they move upstream toward the outflow. This dissipation occurs as the vapor completely condenses into liquid. Decreasing gravitational acceleration results in larger departing bubbles and reduces the wall heat flux, $\overline{q}$. The scaling of $\overline{q}$ with respect to gravity, $\overline{q}_a/\overline{q}_g$, follows the slope $m_{BDB}$. In SDB, the dynamics are dominated by the presence of a central bubble that remains on the heater surface and acts as a vapor sink for smaller satellite bubbles, as depicted in Figure \ref{fig:gravity-scaling} for $a/g=0.001$. The figure also captures the transient behavior of the central bubble which fluctuates in size due to the balance between evaporation and condensation leading to different type of vortical structures. The heat flux drops sharply by a value $K_{jump}$, which depends on the size of the central bubble \cite{DHRUV2021}, and the slope, $m_{SDB}=0$. The gravity scaling of heat flux computed from simulations accurately matches the expected trend from the model \cite{Raj2012,Raj}, providing validation for the simulation results.