% qcircuit file
% example of classical polar code of length N=8
%
\Qcircuit @C=0.5em @R=0.62em {
\lstick{0}&\qw & \targ      &\qww & \targ     & \qw        &\qww & \targ     & \qw        & \qw       & \qw        & \qww \\
\lstick{0}&\qw & \ctrl{-1}  &\qww & \qw       & \targ      &\qww & \qw       & \targ      & \qw       & \qw        & \qww \\
\lstick{0}&\qw & \targ      &\qww & \ctrl{-2} & \qw        &\qww & \qw       & \qw        & \targ     & \qw        & \qww \\
&\qw & \ctrl{-1}  &\qww & \qw       & \ctrl{-2}  &\qww & \qw       & \qw        & \qw       & \targ      & \qww \\
&\qw & \targ      &\qww & \targ     & \qw        &\qww & \ctrl{-4} & \qw        & \qw       & \qw        & \qww \\
&\qw & \ctrl{-1}  &\qww & \qw       & \targ      &\qww & \qw       & \ctrl{-4}  & \qw       & \qw        & \qww \\
&\qw & \targ      &\qww & \ctrl{-2} & \qw        &\qww & \qw       & \qw        & \ctrl{-4} & \qw        & \qww \\
&\qw & \ctrl{-1}  &\qww & \qw       & \ctrl{-2}  &\qww & \qw       & \qw        & \qw       & \ctrl{-4}  & \qww 
  \inputgrouph{1}{3}{1.4em}{\mathcal{F}\left\lbrace\raisebox{-1.3em}[2.2em]{\,}\right.}{2.2em}
  \inputgrouph{4}{8}{2.7em}{\mathcal{I}\left\lbrace\raisebox{-1em}[3.4em]{\,}\right.}{2.2em}
}