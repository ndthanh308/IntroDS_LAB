% qcircuit file
% example of quantum polar code of length N=8
%
\Qcircuit @C=0.7em @R=0.8em {
\lstick{\ket{0}}&\qw & \targ      &\qww & \targ     & \qw        &\qww & \targ     & \qw        & \qw       & \qw        & \qww \\
\lstick{\ket{0}}&\qw & \ctrl{-1}  &\qww & \qw       & \targ      &\qww & \qw       & \targ      & \qw       & \qw        & \qww \\
\lstick{\ket{0}}&\qw & \targ      &\qww & \ctrl{-2} & \qw        &\qww & \qw       & \qw        & \targ     & \qw        & \qww \\
&\qw & \ctrl{-1}  &\qww & \qw       & \ctrl{-2}  &\qww & \qw       & \qw        & \qw       & \targ      & \qww \\
&\qw & \targ      &\qww & \targ     & \qw        &\qww & \ctrl{-4} & \qw        & \qw       & \qw        & \qww \\
\lstick{\ket{+}} &\qw & \ctrl{-1}  &\qww & \qw       & \targ      &\qww & \qw       & \ctrl{-4}  & \qw       & \qw        & \qww \\
\lstick{\ket{+}} &\qw & \targ      &\qww & \ctrl{-2} & \qw        &\qww & \qw       & \qw        & \ctrl{-4} & \qw        & \qww \\
\lstick{\ket{+}}&\qw & \ctrl{-1}  &\qww & \qw       & \ctrl{-2}  &\qww & \qw       & \qw        & \qw       & \ctrl{-4}  & \qww 
  \inputgrouph{1}{3}{1.6em}{\mathcal{Z}\left\lbrace\raisebox{0em}[2.8em]{\,}\right.}{3em}
  \inputgrouph{4}{5}{.9em}{\mathcal{I}\left\lbrace\raisebox{0em}[1.4em]{\,}\right.}{3em}
  \inputgrouph{6}{8}{1.6em}{\mathcal{X}\left\lbrace\raisebox{0em}[2.8em]{\,}\right.}{3em}
}