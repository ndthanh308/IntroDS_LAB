% qcircuit file
% example of quantum polar code of length N=8 
% -- classical polar code induced in X basis
%
\Qcircuit @C=0.7em @R=0.8em {
&\qw & \ctrl{+1}  &\qww & \ctrl{+2} & \qw        &\qww & \ctrl{+4} & \qw        & \qw       & \qw        & \qww \\
&\qw & \targ      &\qww & \qw       & \ctrl{+2}  &\qww & \qw       & \ctrl{+4}  & \qw       & \qw        & \qww \\
%
&\qw & \ctrl{+1}  &\qww & \targ     & \qw        &\qww & \qw       & \qw        & \ctrl{+4} & \qw        & \qww \\
&\qw & \targ      &\qww & \qw       & \targ      &\qww & \qw       & \qw        & \qw       & \ctrl{+4}  & \qww \\
%
&\qw & \ctrl{+1}  &\qww & \ctrl{+2} & \qw        &\qww & \targ     & \qw        & \qw       & \qw        & \qww \\
&\qw & \targ      &\qww & \qw       & \ctrl{+2}  &\qww & \qw       & \targ      & \qw       & \qw        & \qww \\
%
&\qw & \ctrl{+1}  &\qww & \targ     & \qw        &\qww & \qw       & \qw        & \targ     & \qw        & \qww \\
&\qw & \targ      &\qww & \qw       & \targ      &\qww & \qw       & \qw        & \qw       & \targ      & \qww 
  \inputgrouph{6}{8}{1.6em}{(\mathcal{X}, \bm{v})\left\lbrace\raisebox{0em}[2.4em]{\,}\right.}{1.8em}
  \inputgrouph{6}{8}{2.6em}{\text{\footnotesize frozen}}{2.4em}
  \inputgrouph{1}{3}{3.1em}{\mathcal{Z} \cup \mathcal{I} \left\lbrace\raisebox{0em}[3.8em]{\,}\right.}{1.8em}
  \inputgrouph{1}{3}{4.1em}{\text{\footnotesize info.}}{2.3em}
}