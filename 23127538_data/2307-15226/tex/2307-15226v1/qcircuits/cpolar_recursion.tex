% qcircuit file
% the classical polar transform reccursion: P_N = P_{N/2} \otimes P_2
%
\Qcircuit @C=1em @R=0.5em {
& \qw\A{d1} & \multigate{3}{P_{N/2}} & \targ\A{A1}     & \qw & \qw & \qw             & \qw \\
&           & \nghost{P_{N/2}}       &   \\
&           & \nghost{P_{N/2}}       &   \\
& \qw\A{d2} & \ghost{P_{N/2}}        & \qw             & \qw & \qw & \targ\A{D1}     & \qw \\
& \qw\A{d3} & \multigate{3}{P_{N/2}} & \ctrl{-4}\A{A2} & \qw & \qw & \qw             & \qw  \\
&           & \nghost{P_{N/2}}       &   \\
&           & \nghost{P_{N/2}}       &   \\
& \qw\A{d4} & \ghost{P_{N/2}}        & \qw             & \qw & \qw & \ctrl{-4}\A{D2} & \qw  \\
   \ar@{.}"d1"+<0pt,-5pt>;"d2"+<0pt,5pt>
   \ar@{.}"d3"+<0pt,-5pt>;"d4"+<0pt,5pt>
   \ar@{.}"A1"+<5pt,-5pt>;"D1"+<-5pt,5pt>
   \ar@{.}"A2"+<5pt,-5pt>;"D2"+<-5pt,5pt>
}