\documentclass[aps,prd,amsmath,amssymb,11pt]{revtex4}
\usepackage{graphicx}% Include figure files
\usepackage{bm}% bold math
\def\journal#1#2#3#4{{#1} {\bf #2}, #3 (#4)}
\newcommand{\be}{\begin{equation}}
\newcommand{\ee}{\end{equation}}
\newcommand{\bea}{\begin{eqnarray}}
\newcommand{\eea}{\end{eqnarray}}
\newcommand{\hf}{\frac12}
\newcommand{\nn}{\cr}
\def\eq#1{(\ref{#1})}
\def\fd#1#2{\frac{\delta#1}{\delta#2}}
\def\fdd#1#2#3{\frac{\delta^2#1}{\delta#2\delta#3}}
\def\pd#1#2{\frac{\partial#1}{\partial#2}}
\def\la{\langle}
\def\ra{\rangle}
\def\mr#1{{\mathrm{#1}}}
\def\ord#1{{\cal O}(#1)}

\def\dt{{\Delta t}}
\def\hj{{\hat j}}
\def\hx{\hat x}
\def\hy{\hat y}
\def\hD{{\hat D}}
\def\ih{\frac{i}{\hbar}}
\def\sign{\mr{sign}}
\def\v#1{{\bm{#1}}}
\def\Tr{{\mr{Tr}}}

\begin{document}
\title{Action for classical, quantum, closed and open systems}
\author{Janos Polonyi}
\affiliation{Strasbourg University, CNRS-IPHC,23 rue du Loess, BP28 67037 Strasbourg Cedex 2 France}
\date{\today}
\begin{abstract}
The action functional can be used to define classical, quantum, closed, and open dynamics in a generalization of the variational principle and in the path integral formalism in classical and quantum dynamics, respectively. These schemes are based on an unusual feature, a formal redoubling of the degrees of freedom. Five arguments to motivate such a redoubling are put forward to demonstrate that such a formalism is natural. The common elements of the different arguments is the causal time arrow. Some lessons concerning decoherence, dissipation and the classical limits are mentioned, too.
\end{abstract}
\maketitle


\section{Introduction}
To understand the transition between the classical and the quantum physics one needs at least a common formalism, applicable in both domains. This is a wonderful problem because the classical limit of quantum system is driven by the interactions with a large environment, in other word the quantum system obeys open dynamics. Hence we need a CQCO formalism in mechanics which can handle Classical, Quantum, Closed and Open systems.

The search for simpler schemes, in particuler for a CCO formalism covering classical closed and open systems, has started a century ago by attempting to describe the forces acting on electric charges by the variational principle without the electromagnetic field \cite{schwarzschild,ritz,tetrode}. However an inconsistency arises because the force obtained in such a manner is the sum of retarded and advanced contributions and to recover the usual retarded interactions one has to give up the starting point, the variational principle. The idea of the electromagnetic force resulting from a time reversal invariant action at a distance has been advanced further by putting the burden on the absorber, by arguing that the charges completely absorb the in-falling electromagnetic radiation \cite{wheeler}. This assumption bears the imprint of open dynamics and the original problem, the untenability of the variational principle returns. Summarising in contemporary terms: The naive relativistic generalization of local forces without fields is doomed to a failure due to the lack of retardation \cite{currie,cannon,leutwyler}. Though the problem can formally be solved by the introduction of constraints \cite{kerner, pauri} the existence of electromagnetic radiation requires the introduction of non-mechanical degrees of freedom. The use of the variational principle to capture their contributions leads to the time reversal invariant near field interaction \cite{schwarzschild,ritz,tetrode}. To incorporate retardation one needs the far field component but the full representation of the field degrees of freedom renders the charge dynamics open, non-accessible for the traditional variational principle. In another standard CQC formalism the action is used in the variational principle and in the path integral formalism in the classical and the quantum case, respectively. However this bridge between the quantum and the classical domains is formal since the quantum dynamics must contain open channels to reach the classical limit.

The extension of a formalism over open systems represents a challenge both on classical and quantum levels. In the deterministic world of classical mechanics one is tempted to retreat to probabilistic description like in kinetic theory. The construction of an effective open quantum theory starts with the assumption that the observed system and its environment together obey a closed dynamics with known quantization rules. The next step is the extraction of the time dependence of the reduced density matrix of the observed sub-system by projection operators and the result is an integro-differential equation of motion \cite{feshbach,nakajima,zwanzig,mori}. The complexity of this equation restricts its application for Markovian weak coupling expansion \cite{redfield,gaspard,chruscinski,breuerk,preverzev,timm,kidon}. Another level of difficulties appears by recalling that the density matrix has to satisfy more stringent relations than the wave function of a pure state and the positivity can not be assured in the non-Markovian case \cite{barnett,shabani,maniscalco}. The usual solution of this problem is the phenomenological characterization of the most general Markovian master equation to produce physically acceptable density matrix \cite{lindblad}.

Another approach to effective quantum dynamics is the Closed Time Path (CTP) scheme \cite{schw,keldysh} and the resulting QCO formalism allows us to employ the standard perturbation expansion based on the physically appealing Feynman graphs \cite{hu,kamenev}. This method can be applied even in the projector operator formalism \cite{gu,breuerm}, as well. To appreciate the importance of this scheme one should realize that the usual UV divergences of quantum field theory make the introduction of an UV cutoff necesary which in turn opens the bare cutoff theory \cite{openqft}. A distinguished feature of this formalism is a redoubling of the degrees of freedom. This step is non-intuitive and renders the mathematics unusually involved and slowed down the spread of the applications. But any practitioner of CTP can convince himself or herself that the complications of this scheme always represent true physical elements of the rich dynamics of open systems.

Our intuition arises from the macroscopic world which is supposed to be derived from the underlying and only formally known quantum dynamics. If the redoubling is indeed an inherent part of quantum physics then its trace should be visible in classical mechanics, too. The CTP formalism with redoubling has been introduced in classical Lagrangian \cite{envindta} and Hamiltonian \cite{galley} formalism. Hence we already have the elements of a CQCO scheme, the CTP formalism, where the dynamics is defined by the action functional. The goal of the present work is to identify the points showing the need of redoubling in the CQCO formalism based on the action functional and highlight some insights gained from such a scheme.

To find an intuitive explanation of the redoubling within classical physics we start with the standard variational principle, a CC scheme, and underline the need of redoubling during its extensions to a CCO formalism in five steps. The time arrow plays an important role in the arguments as we have seen in the discussion of the action at a distance interaction of charges above hence we start in section \ref{auxtarrs} with the discussion of the direction of time and its relation with the auxiliary conditions of the Newton equations. In section \ref{atas} the redoubling is used (i) to replace the acausal auxiliary conditions of the traditional variational principle by the causal initial conditions. The way the time arrow is encoded by the modified variation principle is demonstrated by the Green functions, introduced in section \ref{stpgreenfs}. The action of the modified variation principle is generalized in section \ref{opeclmechs} for open systems and the redoubling is found crucial (ii) to parametrize the non-conservative forces, and (iii) to represent these forces in the action. The vision of the redoubling as an ancilla (iv) to preserve the Noether theorem in a non-conservative dynamics, and (v) to obtain any equation of motion from the variational principle is described in section \ref{ancillas}.

The extension CCO $\to$ CQCO over the quantum domain starts in section \ref{taqds} with pointing out the need of the introduction of internal time reversal parity followed by the identification of the quantum origin of argument (i). The effective dynamics for the reduced density matrix of an open system, introduced in section \ref{quanteffacts}, shows clearly the origin of arguments (iii) and (v) in the quantum dynamics. In section \ref{wns} the Ward identities are used to derive the Noether theorem for the expectation value of conserved quantities in open systems, the latter being the key point of step (iv). Section \ref{quexts} is devoted to some genuine quantum issues, namely the complexification of the action, the decoherence, its relation to dissipation, and finally the nature of the semiclassical limit. The results are briefly summarised in section \ref{summs}. Some technical details are collected in three appendices, namely appendix \ref{clgrfncta} contains the calculation of the CTP Green function for an open classical harmonic oscillator, \ref{eomqms} is about the the derivation of the equation of motion in quantum mechanics, and appendix \ref{interps} outlines the construction of an interpolating trajectory needed in deriving the energy balance equation in quantum mechanics.




\section{Classical equations of motion and their time arrows}\label{auxtarrs}
Physical laws have two components, an equation of motion, and some auxiliary conditions. The latter installs a time arrow for the former. These two components and their relations are briefly surveyed in this section.

\subsection{Auxiliary conditions}
An equation of motion alone is not sufficient to  make predictions in physics because it contains time derivatives hence one has to impose auxiliary conditions. These two components, the equations of motion and its auxiliary conditions, are strictly separate. The former comes from our theories and the latter is chosen by the experimentalists. Such a strong separation may explain that a slight inconsistency of the variational principle about causality remained unnoticed: On the one hand, we use non-causal auxiliary conditions in the variational method by fixing the initial and the final coordinates in classical mechanics. On the other hand, the Euler-Lagrange equations are used with causal initial conditions in physical applications. No problem arises from the change of the auxiliary conditions as long as they are indeed independent of the equations of motions.

However one can never observe a genuinely closed system where the auxiliary conditions are fully under our control. When a subsystem of a closed dynamics is observed then the equations of motion of the observed system and the auxiliary conditions of its invisible environment are irrevocably mixed. In fact, let us denote the observed and the unobserved coordinates of a closed bipartite system of classical particles by $x$ and $y$, respectively and assume the equations of motion $\ddot x=F(x,y)$, $\ddot y=G(x,y)$ with the auxiliary conditions $y(t_a)=y_i$, and $\dot y(t_a)=v_i$. The effective equation of motion of the observed system is found by first solving the environment equation of motion for an arbitrary system trajectory $y=y[x;y_i,v_i]$ and inserting the result into the system equation of motion $\ddot x=F(x,y[x;y_i,v_i])$. The resulting effective equation displays an explicit dependence on the environment auxiliary conditions. If the open dynamics is derived from an action principle, the best scheme to describe a large system of particles, then that principle must be based on initial conditions.







\subsection{Time arrows}\label{tasf}
Several time arrows can be defined \cite{zeh,halliwell,houghton,albeverio} and they are usually classified according to the domains of physics where they appear, time arrows are known in cosmology, quantum mechanics, a thermodynamics, and electrodynamics. It remains to be seen if these time arrows are independent of each other or they stem from a common origin. A time arrow can be informational or causal. The former points in the direction we loose information, eg. the auxiliary conditions degrade, examples being the quantum mechanical and the thermodynamical time arrows. The causal time arrow is directed from the cause to its effects, an example being the electrodynamical time arrow, and will simply be called ``time arrow'' below. Finally, the time arrow can be internal or external relative to the dynamics where it is observed. The latter is introduced by the auxiliary conditions and the former corresponds to equations of motions with broken time reversal symmetry.

An external time arrow which is generated by the auxiliary conditions imposed at a given time $t_c$ points away from $t_c$. Such a double, time-dependent time arrow is a characteristic feature of the solution of local equations of motion and has caused some complications in finding the origin of the thermodynamical time arrow since the second law of thermodynamics applies in either directions. To avoid such a pathological cases one employs causal auxiliary conditions, either initial or final.

One would think that the experimental determination of the time arrow is trivial but it is actually rather challenging. The reason is that the concept of cause is not defined in physics since it implies an external intervention into the physical world. For instance the Newton equation describes a correlation between the states of a particle at different times rather than referring to cause and consequence. The cause is usually replaced by the supposedly the free will of the physicist in selecting the initial conditions for the experiment. In fact, the choice of the initial conditions must be arbitrary in some range to prove or to disprove an equation of motion.

After having granted the independence of the experimentalists from the observed system one can identify the causal time arrow by the help of a time dependent external source $j(t)$ coupled to the system in a finite time interval identified by some reference clock. The causal time arrow $\tau=\pm1$ relative to the reference time, is determined by the direction of the time the external intervention leads to changes in the state of the system.

The existence of an internal time arrow, irreversibility, can be observed by recording the motion and by checking wether the time reversed motion, seen by the replaying the recording backward, satisfy the same equation of motion. The orientation, usually defined by the direction of the stable, relaxing motion, is used to define the internal time arrow.



\section{Action principle with time arrow}\label{atas}
The goal of the variational principle is the selection of the observed trajectory from a set of possible trajectories, called variational trajectory space. This space consists of trajectories which are at least twice differentiable and satisfy the desired auxiliary conditions to make the choice unique and well defined. The problematic feature of this scheme is that the variational trajectory space is defined by non-causal auxiliary conditions, by fixing the initial and the final coordinates, hence no time arrow can be introduced in this scheme. The generalization of the action principle to allow the dynamics to handle its time arrow is presented in this section for the Lagrangian $L=m\dot x^2/2-U(x)$ of a one dimensional particle for the sake of simplicity.

To keep track of the independent equations we discretize the time interval $t_i\le t\le t_f$ by introducing a small time step $\dt$ and represent the trajectories $x(t)$ as vectors $\vec x$ with components $x_n=x(t_n)$, where $t_n=t_i+n\dt$, $n=0,\ldots,N+1$, $\dt=(t_f-t_i)/(N+1)$. The action
\be
S(\vec x)=\dt\sum_{n=1}^{N+1}\left[\frac{m}2\left(\frac{x_n-x_{n-1}}\dt\right)^2-U(x_n)\right]
\ee
yields the variational equations
\be
0=\frac{\partial S(\vec x)}{\partial x_n}=\begin{cases}x_0-x_1&n=0,\cr
2x_n-x_{n+1}-x_{n-1}-\frac{\dt^2}mU'(x_n)&1\le n\le N,\cr
x_{N+1}-x_N-\dt^2 U'(x_{N+1})/m&n=N+1.\end{cases}
\ee
It is easy to check that the evaluation of the potential energy at an intermediate point $x_n^{(\eta)}=(1-\eta)x_n+\eta x_{n-1}$ leads to $\ord{\dt^2}$ changes in the equation of motion at the end points and to a $\ord{\dt^3}$ correction at the intermediate points and brings no changes in the limit $\dt\to0$.

The velocity is vanishing at the end points as $\dt\to0$ because there is no kinetic energy contribution in the Lagrangian before the initial and after the final time. This is not a problem if $x_0$ and $x_N$ are provided by the auxiliary conditions and the variational equation is used only for the remaining $N$ intermediate points to find $N$ unknowns. But in the case of initial conditions $x_0$ and $x_1$ are fixed by the initial coordinate and velocity and we must use the variational equation for $x_{N+1}$ which is incomplete. How can we complete it?

% Figure environment removed

One can not follow the time evolution in the absence of trajectory therefore we must turn back, make a time inversion $\dt\to-\dt$, and follow the motion backward in time. The $x_{N+1}$-dependent part of the action,
\be\label{turningpoint}
S(x_N)=\dt\left[\frac{m}2\left(\frac{x_{N+1}-x_N}\dt\right)^2-U(x_{N+1})-\frac{m}2\left(\frac{x_{N+2}-x_{N+1}}\dt\right)^2+U(x_{N+2})\right],
\ee
yields $x_{N+2}=x_N+\dt^2U'(x_{N+1})/m$, in other words turns the motion back in time with $\ord\dt$ precision in the velocity. However we still have a problem with the last point, $x_{N+2}$, since there is no kinetic energy and its variational equation from the action \eq{turningpoint}, $x_{N+2}=x_{N+1}$, stops the motion. Thus we have to add more and more points $x_n$, $n>N+2$ to the backward moving part of the trajectory,
\be
S\to S-\dt\sum_n\left[\frac{m}2\left(\frac{x_n-x_{n-1}}\dt\right)^2-U(x_n)\right].
\ee
The last coordinate still enters into an with an incomplete equation but luckily we arrive back to the initial condition at $n=2(N+1)$. We stop there and take the last two coordinates from the time reversed form of the known initial condition rather than solving variational equations.

Let us check quickly the consistency. The number of the coordinates of the two trajectories is $2(N+2)$ which is reduced by the two initial conditions and the common end point to $2N-1$. Since we have $2N$ variational equation this seems to be an overdetermined problem. However the recursive solution of the equation of motion from the two end points, starting with $n=0,1$ and $n=2N+3,2N+2$ forward and backward in time, respectively, yields the same end points thus the number of independent equations is indeed $2N-1$.

Therefore the proposal is that we trace the trajectory twice: first forward in time then we make a time reversal and revisit the motion backward in time until we arrive back to the time reversed initial conditions as depicted in Fig. \ref{ctppathf}. Redoubling (i) arises from breaking the trajectory of the roundtrip into two pieces,
\be
\tilde x(t)=\begin{cases}x_+(t)=x(t)&t_i\le t\le t_f,\cr x_-(t)=x(2t_f-t)&t_f\le t\le 2t_f-t_i,\end{cases}
\ee
where the time as a parameter is reversed, $t\to-t$, in the second phase of the motion creating the false illusion that the time flows in the same direction in both trajectories. The action for the trajectory doublet $x_\pm(t)$ is
\be\label{cctpact}
S[\hx]=S[x_+]-S[x_-].
\ee
The variational trajectory space is defined by identical initial conditions for $x_+(t)$ and $x_-(t)$ and the final condition
\be\label{finalctp}
x_+(t_f)=x_-(t_f).
\ee
Note that the choice of the final time does not matter, the trajectory is $t_f$-independent for $t<t_f$.  The common final point justifies the name Closed Time Path of this scheme. The traditional action principle will be called Single Time Path (STP) formalism. Note that the introduction of the CTP doublet $x\to\hx=(x_+,x_-)$ is {\em not} a redoubling of the physical degrees of freedom since we observe a single physical degree of freedom for twice and long time and even an irreversible the equation of motion sets $x_+(t)=x_-(t)$.

It is instructive to check the presence of a time arrow. Since the solution of the equation of motion makes the trajectories of the CTP copies identical an external source $j(t)=j_0\delta(t-t_0)$ with $t_i<t_0<t_f$ in the action induces a response for $t_0<t<2t_f-t_0$ on the trajectory $\tilde x(t)$ of Fig. \ref{ctppathf}, for $t_0<t<t_f$ in $\hx(t)$ and a causal structure is formed.

While the redoubling makes the use of the causal initial conditions possible in the variational principle it comes with a surprising high price. In fact, the redoubling of the coordinates seems to be out of proportion compared with the original problem, the change of the auxiliary conditions. But this is actually a reasonable price since the two trajectories satisfy the same equation of motion hence the effort to obtain them remains the same as in the traditional scheme.






\section{Green functions}\label{stpgreenfs}
A system of infinitely many Green function can be introduced for a classical dynamics and it offers two important advantages: It shows the role of the time arrow in a specially clear manner, and incorporates the initial conditions within the action. We discuss the case of a one dimensional system for the sake of simplicity where $x$ denotes the coordinate of a particle or the component of a plane wave of a field with a given wave vector whose dynamics is defined by the action $S[x]$.


\subsection{STP Green functions}
We start with the traditional variation method where we perform a functional Legendre transformation, $x(t)\to j(t)$, $S[x]\to W[j]$,
\be
W[j]=S[x]+\int_{t_i}^{t_f}dtx(t)j(t),
\ee
where the trajectory $x(t)$ is chosen by solving the equation of motion
\be
\fd{S[x]}{x(t)}=-j(t)
\ee
with fixed auxiliary conditions. The variational equation for $j$,
\be
\fd{W[j]}{j(t)}=x(t),
\ee
can be used to express $j$ in terms of $x$ and to construct the inverse functional Legendre transform
\be
S[x]=W[j]-\int_{t_i}^{t_f}dtx(t)j(t).
\ee
By restricting the book-keeping variable $j$ infinitesimal the functional $W[j]$ can be considered as formal functional power series,
\be
W[j]=\sum_{n=1}^\infty\frac1{n!}\int_{t_i}^{t_f}dt_1\cdots dt_nD_n(t_1,\ldots,t_n)j(t_1)\cdots j(t_n),
\ee
where the coefficient functions $D_n$ define the Green functions.

To find the physical roles of the Green functions let us consider the action
\be\label{weakint}
S[x]=\hf\int_{t_i}^{t_f}dtdt'x(t)K(t,t')x(t')-\frac{g}4\int_{t_i}^{t_f}dtx^4(t)+\int_{t_i}^{t_f}dtj(t)x(t)
\ee
where the kernel of the first integral is local in time with time translation invariance, $K(t,t')=K(d/dt)\delta(t-t')$, $K(z)$ being a polynomial of order $2n_d$ in $z$. The kernel is assumed to be symmetrical, $K(z)=K(-z)$ because the odd powers of the time derivative produce a boundary term in the action and drop out from the variational equations. The iterative solution of the equation of motion
\be\label{intmeom}
\int_{t_i}^{t_f}dtdt'K(t,t')x(t')=g\int_{t_i}^{t_f}dtx^3(t)-j(t)
\ee
which is reliable for sufficiently short time is the infinite sum of tree graphs, the first three are shown in Fig. \ref{treegr}. Such a representation reveals that the Green functions $D_n$ describe the $\ord{j^{n-1}}$ contribution to the trajectory.

% Figure environment removed


\subsection{Mass-shell and off-shell modes}
The null space of the kernel $K$ consists of the trajectories $K(d/dt)x_h(t)=0$, the general solutions of the homogeneous equation of motion. The trajectories in the null-space will be called mass-shell modes as in field theory. To assess the role of the mass-shell modes we restrict our attention to a harmonic dynamics with $g=0$. The mass-shell modes drop out from the action and thereby from the variational principle, they remain present only to assure the auxiliary conditions. The variational trajectory space is now defined by the generalized Dirichlet condition $x(t_i)=x(t_f)=0$ and $d^nx(t_i)/dt^n=d^nx(t_f)/d^n=0$ with $n=1,\ldots,n_d-1$, called off-shell modes. To regain the desired auxiliary conditions we write the physical trajectory as the sum of the general solution of the homogeneous equation of motion and a particular solution of the inhomogeneous case
\be\label{sephih}
x(t)=x_h(t)+x_{ih}(t)
\ee
where $x_h(t)$ and $x_{ih}(t)$ is a mass-shell and an off-shell mode, respectively.

The kernel is invertible in the space of the off-shell modes where we can define the near Green function $D^n(t,t')=D_2(t,t')$ by
\be\label{Ginverse}
\delta(t_1-t_2)=\int_{t_i}^{t_f}dt'D^n(t_1,t')K(t',t_2)=\int_{t_i}^{t_f}dt'K(t_1,t')D^n(t',t_2).
\ee
Being the inverse of a symmetric operator $D^n$ is symmetric as well, $D^n(t,t')=D^n(t',t)$. The solution of the equation of motion of the harmonic model is given by
\be\label{harmsol}
x_{ih}(t)=-\int_{t_i}^{t_f}dt'D^n(t,t')j(t').
\ee

To recover translation symmetry in time we perform the limits $t_i\to-\infty$ and $t_f\to\infty$ where $D^n(t,t')=D^n(t-t')$ and
\be\label{ftr}
D^n(t)=\int_{-\infty}^\infty\frac{d\omega}{2\pi}e^{-it\omega}D^n(\omega).
\ee
Partial fraction decomposition can be used to bring the inverse of the kernel within the off-shell modes into the form
\be\label{ftrgfnt}
D^n(\omega)=\sum_{j=1}^{2n_d}\frac{Z_j}{\omega-\omega_j}
\ee
where $\omega_j$ are the normal frequencies. The extension of this expression over the whole frequency axis is carried out by the help of the principal value prescription $1/(\omega-\omega_j)\to(\omega-\omega_j)/[(\omega-\omega_j)^2+\epsilon^2]$,
\be
D^n(\omega)=\sum_{\omega_j\in C}\frac{Z_{cj}}{\omega^2-\omega_j^2}+\sum_{\omega_j\in R}\frac{Z_{rj}(\omega-\omega_j)}{(\omega-\omega_j)^2+\epsilon^2}
\ee
where the limit $\epsilon\to0$ is to be performed after the Fourier interal over the frequency, and the first and the second sum includes the complex and real normal frequencies, respectively. The latter come in doublets $(\omega_j,-\omega_j)$ and the formers in quadruplets $(\omega_j,-\omega_j,\omega^*_j,-\omega^*_j)$ and we have
\be\label{neargf}
D^n(t)=2i\sum_{\mr{Re}\omega_{cj},\mr{Im}\omega_{cj}>0}Z_{\omega_{cj}}\cos\mr{Re}\omega_{cj}te^{-\mr{Im}\omega_{cj}|t|}+\sum_{\omega_{rj}>0}Z_{\omega_{rj}}\sin\omega_{rj}|t|
\ee
with imaginary $Z_{\omega_{cj}}$ and real $Z_{\omega_{rj}}$.




\subsection{Causal Green functions}
The variational dynamics for the off-shell modes has non-oriented time due to the symmetry $D^n(t,t')=D^n(t',t)$. The time arrow arises from the the mass-shell modes in the decomposition \eq{sephih} which are introduced ``by hand'', beyond the STP variational scheme. This decomposition can be realized by the use of the retarded Green function $D^r=D^n+D^f$ where the far Green function $D^f$ acts within the null-space.

According to eqs. \eq{Ginverse}
\be
K\left(\frac{d}{dt}\right)D^n(t)=0,~~~t\ne0
\ee
implying that $D^n(t)$ is given by two linear superpositions of the mass-shell modes, one for $t>0$ and another for $t<0$, cf. \eq{neargf}. There are two symmetrical and real combinations for each mode, $\cos\bar\omega t$ and $\sin\bar\omega|t|$ where $\bar\omega$ stands for the normal frequency. The former is regular at $t=0$ and is omitted but the latter can reproduce the singularity in eq. \eq{Ginverse} and the retarded Green function with time arrow $\tau_c=1$ results by the choice $D^f(t)=\sign(t)D^n(t)$ and the restriction that the normal frequencies must be real.




\subsection{CTP Green functions}
The variational space of the CTP formalism is defined by the initial conditions thus the time arrow can be introduced on the level of the variational trajectory space. One employs independent sources $\hj=(j_+,j_-)$ for the CTP copies and defines the functional Legendre transformation
\be\label{legendre}
W[\hj]=S[\hx]+\int_{t_i}^{t_f}dt\hx(t)\hj(t),
\ee
where $\hx\hj=x_+j_++x_-j_-$ and $\hx(t)$ solves the equation of motion
\be\label{legendrev}
\fd{S[\hx]}{\hx(t)}=-\hj(t)
\ee
with identical initial conditions for $x_+$ and $x_-$ at $t_i$ and the final condition $x_+(t_f)=x_-(t_f)$.

The Green functions are defined by the functional Taylor series
\be\label{cgrfnct}
W[\hj]=\sum_{n=0}^\infty\frac1{n!}\int_{t_i}^{t_f}dt_1\cdots dt_nD_{n,\sigma_1,\ldots,\sigma_n}(t_1,\ldots,t_n)j_{\sigma_1}(t_1)\cdots j_{\sigma_n}(t_n).
\ee
The iterative solution of the equation of motion results the series of the tree graph of Fig. \ref{treegr} as in the case of closed dynamics. The variational equation for $\hj$,
\be\label{ilegendrev}
\fd{W[\hj]}{\hj(t)}=\hx(t),
\ee
and its successive derivatives establish the same interpretation of the Green functions as in the case of closed dynamics. The two external sources $j_\pm$, are treated as independent in the funciotnal Legendre transformation but physical systems where $x_+(t)=x_-(t)$ holds for the solution of the equation of motion one has to use $j_+(t)=-j_-(t)$.

A harmonic dynamics is defined by the action
\be\label{hoactcn}
S[\hx]=\hf\int_{t_i}^{t_f}dtdt'\hx(t)\hat K(t-t')\hx(t')+\int_{t_i}^{t_f}dt\hx(t)\hj(t)
\ee
and the solution of the equation of motion is
\be\label{ctpsol}
\hx(t)=-\int_{t_i}^{t_f}dt'\hD(t-t')\hj(t').
\ee
where $\hD=\hD_2$. The Green function $\hD$ is symmetrical, $D_{\sigma_1,\sigma_2}(t_1,t_2)=D_{\sigma_2,\sigma_1}(t_2,t_1)$. To find identical and real copies of $x$ for physical sources, $j_\pm=\pm j$, the Green function must satisfy the conditions $D^{++}+D^{--}=D^{+-}+D^{-+}$, $\rm{Im}(D^{++})=\rm{Im}(D^{+-})$, and $\rm{Im}(D^{-+})=\rm{Im}(D^{--})$. These equations restrict the Green function to the form
\be\label{cslgr}
D_{\sigma\sigma'}=\begin{pmatrix}D^n+iD^i&-D^f+iD^i\cr D^f+iD^i&-D^n+iD^i\end{pmatrix}
\ee
in terms of three real functions $D^n(t,t')=D^n(t',t)$, $D^i(t,t')=D^i(t',t)$, and $D^f(t,t')=-D^f(t',t)$. The solution \eq{ctpsol} for physically realizable source is
\be
x(t)=-\int_{t_i}^{t_f}dt'D^r(t-t')j(t')
\ee
and the time arrow is properly installed even in the iterative solution of the anharmonic model.

The inverse can be written in a similar form
\be
K_{\sigma\sigma'}=\begin{pmatrix}K^n+iK^i&K^f-iK^i\cr-K^f-iK^i&-K^n+iK^i\end{pmatrix}
\ee
with
\bea\label{dtok}
K^{\stackrel{r}{a}}&=&K^n\pm K^f=(D^{\stackrel{r}{a}})^{-1},\nn
%
K^i&=&-D^{r-1}D^iD^{a-1},
\eea
and
\bea\label{ktod}
D^{\stackrel{r}{a}}&=&D^n\pm D^f=(K^{\stackrel{r}{a}})^{-1},\nn
%
D^i&=&-K^{r-1}K^iK^{a-1}.
\eea


\subsection{Generalized $\epsilon$-prescription}\label{epresrs}
To complete the CTP variational principle guided by the Green functions and use the inversions \eq{dtok}-\eq{ktod} one needs an invertible kernel of the harmonic model. Such a regularization of the Green functions is achieved in the traditional STP scheme by the usual $\epsilon$-prescription, the introduction of an infinitesimal imaginary term in the action $S[x]\to S[x]+i\epsilon\int dtx^2(t)/2$ which implements a particular treatment of the discrete spectrum embedded into the continuum. However the CTP action \eq{cctpact} possesses a much larger degeneracy, it is vanishing for arbitrary $x_+(t)=x_-(t)$. To lift this degeneracy the imaginary part of the action is used with the same sign in the action,
\be\label{cctpactf}
S[\hx]=S[x_+]-S[x_-]+i\frac\epsilon2\int_{t_i}^{t_f}dt[x_+^2(t)+x_-^2(t)].
\ee
The Green function \eq{gfctinft} of a harmonic oscillator becomes time translation invariant in the limits $t_i\to-\infty$, $t_f\to\infty$,
\be
\hD(\omega)=\frac1m\begin{pmatrix}\frac1{\omega^2-\omega_0^2+i\epsilon}&-i2\pi\Theta(-\omega_0)\delta(\omega^2-\omega_0^2)\cr-i2\pi\Theta(\omega_0)\delta(\omega^2-\omega_0^2)&-\frac1{\omega^2-\omega_0^2-i\epsilon}\end{pmatrix}.
\ee
The variational trajectory space is defined by the generalized Dirichlet boundary conditions because the desired initial conditions at finite time can be achieved by an appropriate adiabatic turning on of the external source.

The use of the Lorentzian regulated Dirac-delta gives
\bea
D^n(\omega)&=&P\frac1{m(\omega^2-\omega_0^2)},\nn
%
D^f(\omega)&=&-i\frac{\sign(\omega)\epsilon}{m[(\omega^2-\omega_0^2)^2+\epsilon^2]},\nn
%
D^i(\omega)&=&-\frac\epsilon{m[(\omega^2-\omega_0^2)^2+\epsilon^2]},
\eea
and the inversion \eq{dtok} can be used to arrive at the kernel
\be\label{ctpkernel}
\hat K(\omega)=m\begin{pmatrix}\omega^2-\omega_0^2+i\epsilon&-2i\epsilon\Theta(-\omega)\cr-2i\epsilon\Theta(\omega)&-\omega^2+\omega_0^2+i\epsilon \end{pmatrix},
\ee
where $K^n=m(\omega^2-\omega_0^2)$, $K^f=im\epsilon\sign(\omega)$, and $K^i=m\epsilon$ and to define the action of the harmonic oscillator
\bea\label{hot}
S&=&\frac{m}2\int_{-\infty}^\infty dt[\dot x^2_+(t)-\dot x^2_+(t)-\omega_0^2(x_+^2(t)-x_-^2(t))]\nn
&&+\frac\epsilon\pi\int_{-\infty}^\infty dtdt'\frac{x^-(t)x^+(t')}{t-t'+i\epsilon}+\frac{i\epsilon}2\int_{-\infty}^\infty dt[x^{+2}(t)+x^{-2}(t)]
\eea
with the generalized $\epsilon$-prescription terms in the second line. The time translation symmetry breaking coupling \eq{finalctp} of the CTP doublet trajectories at the final time is spread over an infinitesimal time translation symmetrical coupling between the trajectories.




\section{Open classical mechanics}\label{opeclmechs}
The traditional variational principle was extended in closed dynamics to incorporate the time arrow. The next step is the generalization of the action for open dynamics.

\subsection{External time arrow}\label{inttarrws}
We start at the bipartite system mentioned in the Introduction. By imposing initial or final conditions one can install time arrow independently for the system and for the environment as long as they do not interact. Hence we can formally prepare parallel or anti-parallel flow of time for the non-interacting subsystems. The four possible orientation of the time are displayed in Fig. \ref{sectas} where the arrows indicate the system and the environment time arrows, $\tau_s$ and $\tau_e$, respectively in the absence of interactions.

When the system-environment interactions are turned on then the effective dynamics remains causal for $\tau_s=\tau_e$ but becomes acausal for $\tau_s=-\tau_e$. Expressed in another manner, the causal structure of the interactive system is the result of two time arrows, an external one set by the system auxiliary conditions the and an internal one inherited from the environment. The effective dynamics is causal only for identical external and internal time arrows. Yet another way of summarising the lesson of Fig. \ref{sectas} is that a distinguished feature of open interaction channels is that they transfer an external time arrow from the environment into an internal one of the system.

% Figure environment removed

The lesson is that the open dynamics always breaks the time reversal invariance. This actually follows from the proposed experimental detection of the time arrow mentioned in section \ref{tasf}, as well, since the recording does not show the environment. In fact, the time appears to run in opposite direction in the observed system and in the environment in the inverted replaying unlike to the original physical case.




\subsection{Semi-holonomic forces}
The conservative holonomic forces of the action principle are represented by a potential $U(x,\dot x)$ and are of the form
\be\label{hol}
F(x,\dot x)=-\partial_xU(x,\dot x)+\frac{d}{dt}\partial_{\dot x}U(x,\dot x).
\ee
To find their generalization, the open semi-holonomic forces, we start with the full closed dynamics of the observed system and its environment characterized by the action $S[x,y]$. The elimination of the environment is achieved by solving the equation of motion $\delta S[x,y]/\delta y(t)=0$ together with the environment initial conditions imposed at $t_i$ for a general system trajectory $x(t)$. The effective system action is obtained by inserting the solution, $y[x]$, into the action, $S_{eff}[x]=S[x,y[x]]$. The resulting effective equation of motion
\be\label{effeom}
\fd{S_{eff}[x]}{x(t)}=\fd{S[x,y]}{x(t)}_{|y=y[x]}+\int_{t_t}^{t_f}dt'\fd{S[x,y[x]]}{y(t')}\fd{y[t';x]}{x(t)}=\fd{S[x,y]}{x}_{|y=y[x]}=0
\ee
shows clearly the double role the system coordinate plays in the effective dynamics: It appears twice on the list of variables of $S[x,y[x]]$, first as a virtual variational parameter to deduce forces and second as a position defining parameter and only the second role is taken up outside of the variational calculation. This is redoubling (ii) and suggests the generalization of eq. \eq{hol}
\be\label{semihol}
F(x,\dot x)=-\partial_xU(x,\dot x,x',\dot x')_{|x'=x}+\frac{d}{dt}\partial_{\dot x}U(x,\dot x,x',\dot x')_{|x'=x}
\ee
for semi-holonomic forces. The structure $S[x,y[x]]$ of the effective action assures that the semi-holonomic forces cover all possible open forces existing within a subsystem of a closed dynamics.






\subsection{Action of an open system}\label{aopsyss}
To find the action and the variational trajectory space in the presence of semi-holonomic forces we write the full action in the form $S[x,y]=S_s[x]+S_e[y]+S_i[x,y]$ and assume a simple system-environment interaction corresponding to the interaction Lagrangian $L_i=gxy$ $g$ being a coupling constant. We shall use initial conditions for the system and seek the action corresponding to the case of either parallel or antiparallel system and environment time arrows, shown in Fig. \ref{sectas} (a) and (d), respectively.

In the case of a causal effective dynamics of Fig. \ref{sectas} (a) the perturbative elimination of the environment, the iterative solution of the environment equation of motion, generates the non-local potential energy $U^{(a)}=g^2x'(t_2)D_e^r(t_2,t_1)x'(t_1)$ in the leading order to the effective action where $t_1$ and $t_2$ correspond to the time of right and the left oriented horizontal dashed lines in Fig. \ref{sectas} (a), respectively and $D_e^r$ stands for the retarded Green function of the environment. In the case of an acausal effective dynamics, shown in Fig. \ref{sectas} (d) the leading order interaction is represented by $U^{(d)}=gx'(t_2)D_e^a(t_2,t_1)x'(t_1)$, $D_e^a$ being the advanced environment Green function. Thus the leading order interaction is given by the action
\be\label{intact}
S_i^{\stackrel{(a)}{(d)}}=g\int_{t_i}^{t_f}dt_1dt_2x'(t_2)[D_e^n(t_2,t_1)\pm D_e^f(t_2,t_1)]x'(t_1).
\ee

This expression shows the need of redoubling (iii):  The contribution of the far Green function is vanishing because the Green function is sandwiched between the same trajectory. This is a well known problem of the traditional STP scheme which is having difficulties in supporting interactions with odd time reversal parity. For instance the Lorentz force in electrodynamics has negative time reversal parity however it can be derived from an STP variation principle owing to its negative space inversion parity. The representation of the time inversion asymmetric part of a one dimensional harmonic force needs ``another'' system trajectory handled independently in the variation.

The action for the trajectory doublet has to be introduced in such a manner that the solution of the equation of motion brings the two trajectory to overlap. For this end we follow the procedure outlined in section \ref{atas} and introduce the action $S[\hx,\hy]=S[x_+,y_+]-S[x_-,y_-]$ up to the infinitesimal generalized $\epsilon$-prescription terms for the full closed system and define the variational trajectory space with the same initial conditions for the two trajectories and with the identification of the final coordinates, $x_+(t_f)=x_-(t_f)$, $y_+(t_f)=y_-(t_f)$. The derivation of the effective action follows the steps outlined above and one arrives at
\be\label{ctpeffact}
S_{eff}[\hx]=S_s[x_+]+S_e[y_+[\hx]]+S_i[x_+,y_+[\hx]]-S^*_s[x_-]-S^*_e[y_-[\hx]]-S_i[x_-,y_-[\hx]]
\ee
where
\be\label{elim}
\fd{}{\hy(t)}\{S_e[y_+]+S_i[x_+,y_+]-S^*_e[y_-]-S^*_i[x_-,y_-]\}=0.
\ee
The effective action \eq{ctpeffact} can be written as
\be\label{infleffact}
S_{eff}[\hx]=S_s[x_+]-S^*_s[x_-]+S_{infl}[x_+,x_-],
\ee
the sum of the closed system CTP action and the influence functional \cite{feynman}
\be\label{inflact}
S_{infl}[\hx]=S_e[y_+[\hx]]+S_i[x_+,y_+[\hx]]-S^*_e[y_-[\hx]]-S^*_i[x_-,y_-[\hx]]
\ee
representing the effective interactions. The simplest open system, a damped harmonic oscillator, corresponds to the Lagrangian
\be\label{dho}
L=\frac{m}2[\dot x_+^2-\dot x_-^2-\omega_0^2(x_+^2-x_-^2)+\nu(\dot x_+x_--x_+\dot x_-)],
\ee
with the equation of motion \cite{bateman}
\be
\ddot x_\pm=-\omega_0^2x_\pm-\nu\dot x_\mp.
\ee

Few remarks are in order at this point:

(1) One can now better understand the necessity of the time inversion performed at $t_f$ in the intuitively introduced scheme of section \ref{atas}: It yields the opposite sign in front of the action of the two CTP doublet trajectories, needed to retain the contribution of the far Green function term in eq. \eq{intact}.

(2) There is a clear difference between closed and open dynamics: While the action \eq{cctpact} of a closed dynamics has no finite $\ord\epsilon$ coupling between the copies $x_+$ and $x_-$ the closing of the final environment coordinates $y_+(t_f)=y_-(t_f)$ during the elimination \eq{elim} couples $x_+$ and $x_-$ with a finite strength.

(3) The effective action can be written by separating the STP and the genuine CTP terms,
\be\label{clopch}
S_{eff}[\hx]=S_1[x_+]-S^*_1[x_-]+S_2[\hx]
\ee
where $\delta^2S_2[\hx]/\delta x_+(t)\delta x_-(t')\ne0$. The role of $S_1$ and $S_2$ can be revealed by the help of the parametrization $x_\pm=x\pm x_d/2$: The equation of motion arising from the variation of $x$ by ignoring the infinitesimal imaginary terms,
\be\label{triveom}
\fd{S_1[x+\frac{x_d}2]}{x}-\fd{S_1[x-\frac{x_d}2]}{x}+\fd{S_2[x+\frac{x_d}2,x-\frac{x_d}2]}{x}=0,
\ee
is trivially satisfied since $x_+(t)=x_-(t)$ hence $x_d(t)=0$. The variation of $x_d$ yields the physical equation of motion,
\be
\fd{S_1[x]}{x}+\fd{S_2[x_+,x]}{x_+}_{|x_+=x}=0,
\ee
and shows that the holonomic and the semi-holonomic forces arise from $S_1$ and $S_2$, respectively, shown in Fig. \ref{dfpath} where the dashed line stands to the environment far Green function. The interactions are separated into two classes by the expressions \eq{infleffact} and \eq{clopch} the difference being that the system interaction is defined by the original system dynamics ($S_s$) in the former and by all closed system interactions ($S_1$) in the latter. The classification of \eq{clopch} is more natural since the separation of the full system into the observed subsystem and its environment is introduced only by us.

(4) The time reversal can be realized in two different manners in open systems: The full time reversal acts on both the system and its environment and is a trivial symmetry owing to a reparametrization of the motion. It is represented by the exchange of the trajectories, $(x_+,x_-)\to(x_-,x_+)$ and $S[x_+,x_-]\to-S^*[x_-,x_+]$ where the complex conjugation corresponds to the flip $\epsilon\to-\epsilon$, the swap of the initial and the final conditions. Therefore
\be\label{ctpsym}
S[x_+,x_-]=-S^*[x_-,x_+]
\ee
is a formal symmetry of the CTP scheme. A partial time reversal effects is performed only on the observed system and is represented by $t\to-t$ in and a complex conjugation of the effective action. The symmetry with respect to partial time reversal transformation is violated by $S_2$ in \eq{clopch} in agreement with remark made in section \ref{inttarrws} that the time of an open dynamics is always oriented.

% Figure environment removed


\section{Ancilla}\label{ancillas}
The argument of redoubling (iii) of the previous section suggests that the CTP copy of the system actually represents the environment. We follow this point of view and seek a non-trivial extension of the original variational principle to cover open systems by treating the copy as an ancilla.



\subsection{Environment as a copy}
If a small system is interacting with a large environment then it is not necessary to possess the full information about the environment to find its imprint on the dynamics of the system. The selection and the representation of the necessary information can be achieved by introducing a sufficiently simple ancilla as a new, reduced environment. How to select the ancilla and its interaction with the system?

The solution comes from looking into the conservation laws whose violation is the hallmark of open dynamics. The energy is decreased by a Newtonian friction force which is proportional to the velocity despite the explicit time translation invariance of the equation of motion. How can we modify the variational principle in order to reproduce such a non-conservative force with unbroken time translation symmetry of the effective dynamics?

The answer is rather obvious, the ancilla should absorb the lost energy. Rather than trying to circumvent the second law of thermodynamics and accumulate the dissipated energy in a small ancilla we can arrive at a solution in two simple steps. First, it is trivial to compare the energy stored in the observed system and the ancilla if the latter is chosen to be a copy of the former. This is redoubling (iv). As a side remark, such an {\em ancilla = observed system} construction is optimal for the representation of the effective dynamics since the environment is reduced to an ancilla with the same complexity as the system. Next, the energy conservation can trivially be achieved by assuring that the energy is defined with the opposite sign in the original system and its copy. Since the conserved quantities are linear in the Lagrangian according to the Noether theorem the Lagrangian of the full system is chosen to be antisymmetric with respect to the exchange of the system and the ancilla as in eq. \eq{inflact}.

The procedure can easily be demonstrated by the derivation of the Noether theorem for the Lagrangian $L=L_1(x_+,\dot x_+)-L_1(x_-,\dot x_-,t)+L_2(x_+,\dot x_+,x_-,\dot x_-,t)$. By analogy with field theory the trajectory $x(t)$ can be viewed as a mapping of the external space, the time, into the internal space, the coordinate space. Hence the infinitesimal changes $x\to x+\delta x$ and $t\to t+\delta t$ are called internal and external space transformations, respectively.

The momentum balance equation arises by performing an infinitesimal translation in the internal space, $\hx\to\hx+\hat\xi$ and by treating the time dependent $\delta\hx=\hat\xi$ with $\hat\xi(t_i)=\hat\xi(t_f)=0$ as a special variation. The corresponding linearized action
\be
S[\hat\xi]=\int dt\hat\xi\left(\frac{d}{dt}\pd{L}{\dot{\hx}}-\pd{L}{\hx}\right)
\ee
has a trivial $0=0$ equation of motion for $\xi=(\xi_++\xi_-)/2$ in agreement with eq. \eq{triveom}. This triviality indicates that the total momentum of both copies is vanishing because the momentum is defined with the opposite sign in the copies. The Noether theorem corresponding to the opposite variation of the two trajectories, the equation of motion for $\xi_d=\xi_+-\xi_-$, adds rather than subtracts the contributions of the two momenta and is actually a balance equation,
\be
\frac{d}{dt}\pd{L_1}{\dot x}=\pd{L_1}{x}+\left(\pd{L_2}{x_+}-\frac{d}{dt}\pd{L_2}{\dot x_+}\right)_{|\dot x_\pm=\dot x,x_\pm=x}.
\ee
The change of the generalized momentum is due to the violation of the translation invariance of the closed dynamics and the semi-holonomic forces. Another form of this equation states that the change of the momentum renormalized by the open interactions,
\be
p_r=\pd{L_1}{\dot x}+\pd{L_2}{\dot x_+}_{|\dot x_\pm=\dot x,x_\pm=x}
\ee
is due to the breakdown of the translation invariance of the effective dynamics,
\be\label{momcons}
\dot p_r=\pd{L_1}{x}+\pd{L_2}{x_+}_{|\dot x_\pm=\dot x,x_\pm=x}.
\ee
The second term in this expression represents the image of the system, the ``polarization''  of the environment, generated by the system-environment interactions which moves with the system. In particular, the momentum balance equation for the damped harmonic oscillator \eq{dho} is $p_r=m(\dot x+\nu x/2)$ is $\dot p_r=-m(\omega_0^2x+\nu\dot x/2)$.

The balance equation for angular momentum comes from infinitesimal internal space rotations of a multi-component coordinate vector, $\delta x_\sigma=\xi_\sigma\tau x_\sigma$ where $\tau$ stands for a generator of the rotation group. The linearized action of the infinitesimal angle $\xi$,
\be
S[\hat\xi]=-\sum_\sigma\int dt\left[\xi_\sigma\pd{L}{x_\sigma}\tau x_\sigma+\pd{L}{\dot x_\sigma}(\xi_\sigma\tau x_\sigma+\xi_\sigma\tau\dot x_\sigma)\right]
\ee
results the equation of motion for $\xi_d$
\be\label{angmombe}
\frac{d}{dt}(p_r\tau x)=\pd{L_1}{x}\tau x+\pd{L_1}{\dot x}\tau\dot x+\left(\pd{L_2}{x_+}\tau x+\pd{L_2}{\dot x_+}\tau\dot x\right)_{|\dot x_\pm=\dot x,x_\pm=x}
\ee
showing that the non-conservation of the renormalized angular momentum arises from the non-rotational invariant part of the effective Lagrangian.

The energy equation is obtained by performing an infinitesimal external space translation, $x_\pm(t)\to x_\pm(t-\xi_\pm)$, $\xi_\pm(t_i)=\xi_\pm(t_f)=0$, and treating $\delta x_\sigma=-\xi_\sigma\dot x_\sigma$ as a variation with the liearized action
\be
S[\hat\xi]=-\sum_\sigma\int dt\left(\xi_\sigma\dot x_\sigma\pd{L}{x_\sigma}+\frac{d}{dt}(\xi_\sigma\dot x_\sigma)\pd{L}{\dot x_\sigma}\right).
\ee
The triviality of the equation of motion for $\xi$ renders the linearized action $\xi$-independent and its $\xi_d$-dependence,
\be
S[\xi_d]=\int dt\left[\xi_d\left(\partial_tL_1-\frac{d}{dt}L_1\right)-\dot\xi_d\dot x\pd{L_1}{\dot x}\right]-\hf\sum_\sigma\sigma\int dt\left(\xi_d\dot x_\sigma\pd{L_2}{x_\sigma}+\dot\xi_d\dot x_\sigma\pd{L_2}{\dot x_\sigma}+\xi_d\ddot x_\sigma\pd{L_2}{\dot x_\sigma}\right),
\ee
is obtained by the help of the relation
\be\label{legy}
\frac{d}{dt}L_1=\dot x\pd{L_1}{x}+\ddot x\pd{L_1}{\dot x}+\partial_tL_1.
\ee
The equation of motion
\be
\frac{d}{dt}\left(\dot x\pd{L_1}{\dot x}-L_1\right)=-\partial_tL_1+\dot x\left(\pd{L_2}{x_+}-\frac{d}{dt}\pd{L_2}{\dot x_+}\right)_{|\dot x_\pm=\dot x,x_\pm=x}
\ee
is the energy equation showing that the energy defined by the closed dynamics is changed by the explicite time-dependence of the closed dynamics and the work of the semi-holonomic forces.

It follows from the structure of the semi-holonomic forces that the internal symmetry conservation laws are retained by an appropriate renormalization of the conserved quantities if the open interactions, encoded by $L_2$ do not violate the underlying symmetry. This is not the case with external symmetries which are always broken by the semi-holonomic forces.





\subsection{Generalized variational equation}
The equations of motion of closed systems are restricted by the variational principle, they are the canonical equations of classical mechanics. Open equations of motion are non-conservative hence non-canonical. It is natural to ask the question whether the possible equations of motion of a subsystem of closed systems cover the set of all imaginable equation or they belong to a restricted class.

Let us assume that the equation of motion for the coordinate $x$ at the time $t'$ can be written in the form $F[x,t']=0$ where $F[x,t']$ is an arbitrary functional of the trajectory $x(t)$. We introduce another coordinate $x_d$ and define the action
\be
S_F[x,x_d]=\int dt'x_d(t')F[x,t']+S'[x_d]
\ee
for the two trajectories where $S'$ is an arbitrary odd functional, $S'[x_d]=-S'[x_d]$. It easy to see that the action
\be
S[\hx]=S_F\left[\hf(x_++x_-),x_+-x_-\right]
\ee
equipped with the generalized $\epsilon$-prescription terms generates the variational equation $F[x,t']=0$ and $x_d=0$ within the variational space of section \ref{atas}. Argument (v) for the redoubling is that it introduces a copy of the system as an ancilla to obtain any equation of motion by the variational principle.


\section{Time arrow in quantum mechanics}\label{taqds}
Perhaps the first indication of the redoubling in quantum mechanics is the way the time arrow is introduced. The auxiliary conditions for the Newton equation, a second order differential equation, can be used to orient the time for the classical motions. In fact, neither the boundary conditions, the initial and the final coordinates, nor the initial conditions, the initial coordinate and the velocity, are time reversal invariant. The initial condition for the first order Schrödinger equation is the initial state. How to encode the direction of time in a state?

The solution of this problem is well knwown, it is the introduction of an internal time reversal parity for the quantum states. The usual procedure is to use positive time reversal parity coordinate eigenstates with real wave functions and to represent the time reversal transformation by complex conjugation. The real and imaginary part of the wave function in coordinate representation stand for the positive and the negative time reversal parity components of the states. As a result the bra and the ket develop in opposite direction in time.

This remarks brings us to the quantum origin of argument (i) mentioned above in closed classical dynamics. To recover time independent physical quantities in the eigenstate of the Hamiltonian the time dependence of the ket expectation values must be neutralized by a bra. The appearance of the bra and the ket components as multiplicative factors in the expectation values is required by Gleason's theorem, as well. The expectation value of the coordinate dependent operator $A(x)$ at time $t$ is given by
\be\label{pintexpval}
\la\psi_i|U^\dagger(t,t_i)AU(t,t_i)|\psi_i\ra=\int d\hx_idx_fA(x_f)\la x_{i+}|\psi_i\ra\la\psi_i|x_{i-}\ra\int_{\hx(t_i)=\hx_i}^{x_\pm(t)=x_f}D[\hx]e^{\ih(S[x_+]-S^*[x_-])}
\ee
in the path integral representation. Therefore the integration over the trajectories $x_+(t)$ and $x_-(t)$ of Fig. \ref{ctppathf} represents the quantum fluctuations within $U$ and $U^\dagger$ and the action is given by eq. \eq{cctpactf}. The redoubling of the coordinate is the result of the non-trivial representation of the time arrow on the space of pure states, $\psi\to(|\psi\ra,\la\psi|)$. The CTP formalism was actually first introduced to deal with the perturbation expansion in the Heisenberg representation \cite{schw,keldysh} where the redoubling arises from the independent perturbation series of the time evolution operators $U(t)$ and $U^\dagger(t)$. Similar argument applies to Thermal Field Theory \cite{umezawa}.




\section{Quantum effective actions}\label{quanteffacts}
The distribution of the quantum fluctuations in a state is described by the density matrix, defined by  $\rho(x_ +,x_-)=\la x_+|\rho|x_-\ra$ in the coordinate representation and the generalization of the expectation value \eq{pintexpval} for mixed state is
\be
\Tr[\rho A]=\int dx_\pm\rho(x_+,x_-)\la x_-|A|x_+\ra.
\ee
The density matrix is factorizable in a pure state $\rho(x_ +,x_-)=\la x_+|\psi\ra\la\psi|x_-\ra$ making the quantum fluctuations in the bra and the ket independent. These fluctuations become correlated in a mixed state where the density matrix is not factorizable,
\be\label{mixeddm}
\rho(x_ +,x_-)=\sum_n\la x_+|\psi_n\ra p_n\la\psi_n|x_-\ra
\ee
where the sum extends over more than one pairwise orthogonal states in the sum. Therefore the mixing, the uncertainty about the actual state of the system appears as a correlation between the quantum fluctuations of the bra and the ket. The correlations appear as a coupling between the CTP copies, represented by the contribution $S_2$ in \eq{clopch}.

The construction of the effective action, point (iii) in classical mechanics, is based on the reduced density matrix of the coordinate $x$ within a closed system of section \ref{aopsyss}
\bea\label{reddensm}
\rho(t,\hx)&=&\la x_+|\Tr_e[U(t,t_i)\rho_iU^\dagger(t,t_i)]|x_-\ra\nn
&=&\int d\hx_id\hy_idy_f\rho_i(x_{i+},y_{i+},x_{i-},y_{i-})\int_{\hx(t_i)=\hx_i,\hy(t_i)=\hy_i}^{\hx(t)=\hx,y_\pm(t)=y_f}D[\hx]D[\hy]e^{\ih(S[x_+,y_+]-S^*[x_-,y_-])},
\eea
where the integration of the final coordinate of the environment trajectory stands for the trace $\Tr_e$ over the environment Hilbert space. One can introduce the influence functional by the equation
\be\label{qinflf}
e^{\ih S_{infl}[\hx]}=\int d\hy_idy_f\rho_e(y_{i+},y_{i-})\int_{\hy(t_i)=\hy_i}^{y_\pm(t)=y_f}D[\hy]e^{\ih(S_e[y_+]+S_i[x_+,y_+]-S_e^*[y_-]-S_i^*[x_-,y_-])},
\ee
where the initial density matrix is assumed to be factorizable, $\rho_i(x_{i+},y_{i+},x_{i-},y_{i-})=\rho_s(x_{i+},x_{i-})\rho_e(y_{i+},y_{i-})$ and rewrite the reduced density matrix as
\be\label{otprdm}
\rho(t,\hx)=\int d\hx_id\hy_idy_f\rho_s(x_{i+},x_{i-})\int_{\hx(t_i)=\hx_i}^{\hx(t)=\hx}D[\hx]e^{\ih S_{Beff}[\hx]},
\ee
where the bare effective action $S_{Beff}$ is given by \eq{infleffact}. The system and the environment are treated in the CTP scheme with identical final points of the doublet trajectories and in the Open Time Path (OTP) scheme with different, fixed end points, respectively.

In the case of a pure state in closed, unitary dynamics the bra and the ket follow independent time evolution related by a trivial time inversion thus it is sufficient to solve the Schrödinger equation for one of them to find the time dependence of the expectation values. This is similar to classical mechanics where it is sufficient to solve the Euler-Lagrange equation for the unique trajectory in the traditional variational principle. However the correlation between the bra and the ket fluctuations in an open dynamics makes it necessary to find the non-factorizable density matrix. In a similar manner one has to solve the equation of motion for the trajectories of both copies in classical open systems.

The origin of point (v), the possibility of reproducing any classical time dependence by an appropriately chose open dynamics is not easy to find in quantum mechanics because the positivity of the reduced density matrix restricts the effective equation of motion in a rather complicated manner. But at least a partial analogy can be found: The wave function of a pure state of a closed dynamics is given by the dependence of the path integral on the final point. In a similar manner the density matrix should be represented by a path integral over redoubled trajectories. This is just what the double CTP path integral \eq{otprdm} is doing. To assure the Hermiticity of the density matrix one needs condition \eq{ctpsym}.

The relation with the classical dynamics can be easily be established by the help of the generator functional
\bea\label{genfunct}
e^{\ih W[\hj]}&=&\Tr[U(t,t_i;j_+)\rho_iU^\dagger(t,t_i;-j_i)]\nn
&=&\int D[\hx]e^{\ih S[\hx]+\ih\int dt\hj(t)\hx(t)}
\eea
for the connected Green functions defined by using \eq{cgrfnct} for $W[\hj]$ rather than $Z[\hj]$ where $U(t,t_i;j)$ is the time evolution operator for the closed dynamics of the system and its environment in the presence of the external source $j$ coupled to $x$. The inverse functional Legendre transform $W[\hj]\to S[\hx]$ defined by eqs. \eq{legendre} and \eq{ilegendrev} can be used to introduce the effective action $S_{eff}[\hx]$. The trajectory $\hx(t)$ defined by \eq{ilegendrev} at $\hj=0$ yields the physical trajectory
\be
x(t)=\int D[\hx]e^{\ih S_{Beff}[\hx]}x_\pm(t)=\la x(t)\ra
\ee
which satisfies the variational equation of $S_{eff}[\hx]$ according to eq. \eq{legendrev}. Hence the effective action defined by the Legendre transformation \eq{legendre} and \eq{ilegendrev} is the classical CTP action.


\section{From Ward identities to Noether theorem}\label{wns}
Point (iv) of redoubling in classical mechanics is about the reconciliation of the violation of conservation laws of open  dynamics with Noether theorem. To understand the relation of this argument with quantum mechanics one has to start at an earlier point, at the difference between conservation laws in quantum and classical mechanics. An obvious difference is the existence of quantum conservation laws related to discrete symmetries. This is easy to understand by noting that the symmetry such a symmetry is realized by discrete linear superposition. The absence of such states in classical mechanics explains the lack of classical conservation laws belonging to discrete symmetry. The relation between Noether theorem and quantum conservation laws is a more difficult question in the case of a continuous symmetry. In fact, the proof of Noether theorem goes by considering the parameters of the continuous symmetry as new coordinates and the reparametrization invariant Euler-Lagrange equation, written in this new coordinate system, yields the conservation of the generalized momentum. One can not translate this argument into quantum mechanics because the quantization rules are not invariant under non-linear coordinate transformations and the equation of motion can not be used as an operator equation. The relation between the classical and quantum conservation laws goes rather by comparing the Ward identities, the expression of a continuous symmetry in terms of Green functions, with Noether theorem.

The variation of the trajectory is interpreted as a change of the integration variable in \eq{genfunct} and invariance of the integral, the generator functional produces a functional equation in the source. The successive functional derivatives of this equation generates a hierarchical set of equations. This procedure is briefly summarized in appendix \ref{eomqms} for a generic variation. The resulting hierarchical set of equations gives the insertion of the equation of motion into the Green functions. In the case of a continuous symmetry the variation is chosen to be the local, gauged version of the symmetry and the resulting hierarchical set of equations are called Ward identities. They describe the insertion of the conservation law resulting from Noether theorem into the Green functions.

The translation $\delta x_\pm=\pm\xi$ treated as a time-dependent variation yields the momentum balance equation \eq{momcons} in classical mechanics. Treated as a change of variable the repetition of the steps \eq{varstart}-\eq{expvaleom} yields
\bea
&&\frac{d}{dt}\la T[p_r(t)x_{\sigma_1}(t_1)\cdots x_{\sigma_n}(t_n)]\ra=\la T\left[\left(\pd{L_1}{x(t)}+\pd{L_2}{x_+(t)}_{|\dot x_\pm=\dot x,x_\pm=x}\right)x_{\sigma_1}(t_1)\cdots x_{\sigma_n}(t_n)\right]\ra\nn
&&-i\hbar\sum_{j=1}^n\delta(t-t_j)\la T[x_{\sigma_1}(t_1)\cdots x_{\sigma_{j-1}}(t_{j-1})x_{\sigma_{j+1}}(t_{j+1})\cdots x_{\sigma_n}(t_n)]\ra.
\eea
The linear transformation of a multi-component coordinate $\delta x_\pm=\pm\xi\tau x_\sigma$ leads to the classical balance equation \eq{angmombe} for the renormalized angular momentum but treated as a change of variable in the path integral generates the equations
\bea
&&\frac{d}{dt}\la T[p_r(t)\tau x(t)x_{\sigma_1}(t_1)\cdots x_{\sigma_n}(t_n)]\ra\nn
&=&\la T\left[\left(\pd{L_1}{x}\tau x+\pd{L_1}{\dot x}\tau\dot x+\left(\pd{L_2}{x_+}\tau x+\pd{L_2}{\dot x_+}\tau\dot x\right)_{|\dot x_\pm=\dot x,x_\pm=x}\right)x_{\sigma_1}(t_1)\cdots x_{\sigma_n}(t_n)\right]\ra\nn
&&-i\hbar\sum_{j=1}^n\delta(t-t_j)\la T[x_{\sigma_1}(t_1)\cdots x_{\sigma_{j-1}}(t_{j-1})x_{\sigma_{j+1}}(t_{j+1})\cdots x_{\sigma_n}(t_n)]\ra.
\eea
The expectation value of the renormalized momentum or angular momentum is non-conserved for translation or rotational non-invariant Lagrangian, respectively or when the instantaneous creation or annihilation of elementary excitations violate the equation of motion.

The energy conservation is a more subtle issue because the underlying external symmetry is broken by the $\dt\ne0$ regulator of the path integral which restricts the time into a discrete set. To recover the continuous translations, used to derrive \eq{legy}, we replace the regulated bare action of the discrete trajectory defined by the set of points $\{(t_i+n\dt,x_n)\}$ by that of the continuous action of the interpolating trajectory defined in appendix \ref{interps} which is constructed in such a manner that the relative difference of the two actions is bounded by a freely chosen small number. The steps followed in appendix \ref{eomqms} can now be repeated for the interpolating trajectories with eq. \eq{legy} holding with the result
\bea
\frac{d}{dt}\int D[x]e^{\ih S[x]+\ih\int xj}(H-jx)&=&\int D[x]e^{\ih S[x]+\ih\int xj}\left[\dot x\left(\frac{d}{dt}\pd{L}{\dot x}-\pd{L}{x}-j\right)-\partial_tL\right]\nn
%
\frac{d}{dt}\la T[\left(\dot x\pd{L_1}{\dot x(t)}-L_1\right)x_1\cdots x_n]\ra&=&\la T\left[\left[\dot x\left(\frac{d}{dt}\pd{L}{\dot x}-\pd{L}{x}\right)-\partial_tL\right]_tx_1\cdots \cdots x_n\right]\ra
\eea





\section{Quantum extensions}\label{quexts}
We turn finally to the aspects of open dynamics which are specific of the quantum level, the complexification of the action, decoherence, dissipation and the semiclassical limit.

\subsection{Finite life-time and preservation of the total probability}
The open interaction channels lead to more radical changes in the quantum dynamics than in the classical case: While the action of a classical open system is real up to the infinitesimal terms of the generalized $\epsilon$-prescription the bare quantum action whose parameters are given in terms of Green functions may assume complex values. The physical origin of the complexification in a unitary full dynamics of a many-body system is identified by the optical theorem as the forward scattering, the presence of the mass-shell excitations in the intermediate states of the perturbation series.

The complexification of the action enriches the dynamics which now contains more free parameters: The number of the closed interaction parameters is doubled since any parameters of $S_1$ can be complex and new the open interaction channels in $S_2$ represent additional free parameters restricted by the formal symmetry \eq{ctpsym}. The imaginary part of the closed parameters introduces finite life-time for excitations which renders the time evolution non-unitary. One would expect a violation the conservation of the norm of the state, the total probability in that case. However the open channels restore the conservation of the total probability. In fact, the system and its environment together obey closed unitary dynamics hence $W[\hj]=0$ for physically realisable sources, $j_+=-j_-$, which implies $\Tr\rho_s=1$ for the system reduced density matrix according to eq. \eq{genfunct}.




\subsection{Decoherence and dissipation}
The excitations of the environment not only destabilize the state of a subsystem, they generate system-environment entanglement and decohere the sub-systems. The non-factorisablilty of the density matrix \eq{mixeddm} arises in this formalism when the excitations of the environment produce several contributions to the trace $\Tr_e$ in \eq{reddensm}. The decoherence follows since the imaginary on-shell contributions which dominate the parameters of the influence action in eq. \eq{qinflf} for long time are positive according to the optical theorem and suppress the integrand on the right hand side of eq. \eq{otprdm}. A detailed space-time picture of the build up of the decoherence can be obtained from the path integral \eq{otprdm}: The integrand $\exp iS[\hx]/\hbar$ is the contribution of the pair of trajectories $x_\pm$ to the density matrix and the decoherence in the coordinate basis consists of the suppression of contributions with $x_d=x_+-x_-\ne0$, induced by $\mr{Im}S>0$.

The physical origin of the decoherence of a subsystem in a large, closed many-body system is identical with that of dissipation. In fact, the internal on-shell excitations establish interactions among well separated spatial regions and generate long range open dynamics. The instability of the subsystem states appears as an irresistible ``leakage`` into the environment and generates dissipative effective dynamics. A condition for strong decoherence/dissipation is a dense excitation spectrum of the system and the environment. The small and frequent energy exchanges make the mixing contributions of the reduced density matrix large and thereby the system-environment entanglement and the decoherence become strong.

It is instructive to consider the case of a harmonic oscillator where the second order Green functions are identical in the classical and the quantum case. The action \eq{hot} of the $\epsilon$-prescription can in be used for finite $\epsilon$ and the the first and the second $\ord\epsilon$ term describes decoherence and dissipation, respectively but an $\ord\epsilon$ acausal contribution arise in $D^r$. A better strategy is to use the most general quadratic Lagrangian compatible with the full time inversion symmetry \eq{ctpsym} \cite{ines}
\be\label{hod}
L=\frac{m}2(\dot x_+^2-\dot x_-^2)-\frac{m\omega^2}2(x_+^2-x_-^2)+\frac{m\nu}2(\dot x_+x_--\dot x_-x_+)+\frac{i}2[d_0(x_+-x_-)^2+d_2(\dot x_+-\dot x_-)^2]
\ee
with $K^n=m(\omega^2-\omega_0^2)$, $K^i=d_0+d_2\omega^2$,  and $K^f=im\nu\omega$ gives
\bea
D^{\stackrel{r}{a}}&=&\frac1{m[\omega^2-\omega_0^2\pm i\omega\nu]},\nn
%
D^i&=&-\frac{d_0+d_2\omega^2}{m[(\omega^2-\omega_0^2)^2+\omega^2\nu^2]},
\eea
where the Heaviside function is smeared. The closed limit is $\nu=\epsilon/\omega_0$, $d_0=m\epsilon$, and $d_2=0$. This Lagrangian is better suited for phenomenological applications because $\nu$, $d_0$ and $d_2$ may be finite without acausality. The open oscillator has a Newton friction force $F_f=-m\nu\dot x$ and the decoherence is controlled by the parameters $d_0$ and $d_2$ which drop out from the classical equation of motion. The dissipative time scale, the life-time of the excitations, is given by the imaginary part of the pole of $D^r$ which depends only on $\nu$, a parameter of the real part of the action in agreement with the classical origin of the friction force. However the decoherence, being a genuine quantum effect, is controlled by the imaginary part of the action. Hence dissipation and decoherence may have independent scales despite their common dynamical origin \cite{timescales}. Another remark is that the CTP symmetry \eq{ctpsym} makes the Green functions for $x_d$ vanishing, rendering $x_d$ ''invisible`` on the level of expectation values. However in the mixed Green function for $x$ and $x_d$ the latter brings an $\ord{\sqrt{\hbar}}$ multiplication factor.


\subsection{Semiclassical limit}
It is a widespread view that in phenomenons where $\hbar$ can be treated as a small parameter the quantum effects are weak up to some macroscopic quantum effects. One usually argues by mentioning that the Heisenberg commutation relations are $\ord\hbar$ or pointing out that the traditional path integral expression for the transition amplitude between coordinate eigenstates
\be\label{tradpint}
\la x_f|e^{-\ih Ht}|x_i\ra=\int_{x(0)=x_i}^{x(t)=x_f}D[x]e^{\ih S[x]}
\ee
is dominated by the classical trajectory as $\hbar\to0$. Such a view is too naive \cite{ballantine,klein} and one may look for more reliable signatures of the classical limit in a CQCO scheme.

Though the necessary conditions for the classical limit remain unknown there are few well known necessary conditions, such as the decoherence, the suppression of the interference between macroscopically different states \cite{zehd,zurekd,joos,zurekt}, and the return of the determinism, the narrowing of the probability distributions for the observables. The first condition excludes closed systems where the time evolution is unitary and their fully resolved dynamics remains forever quantum. The classical description can be efficient for open dynamics where only a partial information is available. We have argued above that the decoherence is  strong when the system and the environment have dense excitation spectrum. The second condition can be satisfied if the observable is the macroscopic average of microscopic quantities according to the central limit theorem \cite{macr}.

The conflict between the simplistic $\hbar\to0$ condition and the more involved arguments about need of the decoherence and the macroscopic limit can partially be understood by comparing the path integrals \eq{otprdm} and \eq{tradpint}. The strong decoherence in the coordinate basis implies strong suppression of trajectory pairs with large $x_d=x_+-x_-$ for long time in \eq{otprdm}. Hence the trajectory pairs with $x_d\sim0$ dominate the path integral and the two copies ''stick together`` in the classical limit yielding the action $S_{Beff}[x,x]=\mr{Im}S_{Beff}[x,x]$ according to \eq{ctpsym} for the common trajectory $x=x_+=x_-$. The imaginary part of the bare effective action arises from the system-environment interactions hence it is  assumed to be small and the trajectories contributes to the functional integral \eq{otprdm} in an approximately identical manner. In other words the fluctuations are large and the dynamics is soft. The limit $\hbar\to0$ localizes the dominant contributions in \eq{tradpint} around the classical trajectory hence the fluctuations are small and the dynamics is hard. We are lead to the question whether the classical limit is soft or hard.

The answer depends on the auxiliary conditions. When the initial state of the system and its environment is fixed as in a realistic situation then the soft path integral of the CTP formalism reflects the easy excitability owing to the dense excitation spectrum. When a pure initial and final states are fixed then the transition is dominated by classical physics if the corresponding action is large compared to $\hbar$. But notice that \eq{tradpint}, a transition amplitude in a closed dynamics between pure states is neither an observable nor relevant for the classical limit of open systems. A measurable transition probability between initial and final coordinate states is given by the help of the CTP path integral \eq{otprdm} by integrating over trajectories with given initial and final points. The naive argument about the dominance of the classical trajectory may remain valid for short time evolution in agreement with the general expectation that short time, high energy motion of particles is semiclassical. However the decoherence may build up in during a long time evolution invalidating the naive argument about \eq{tradpint}.



\section{Summary}\label{summs}
To understand realistic mechanical systems from first principles we need a CQCO formalism which is equally applicable for classical, quantum, closed and open dynamics. This condition is satisfied by the CTP scheme where the dynamics can be defined by an action functional. However such a wide applicability is achieved by an unusual feature, a formal redoubling of degrees of freedom. The following possible oringins of redoubling were presented in classical mechanics:

(i): A generalization of the traditional variational principle of classical mechanics can be constructed for causal initial conditions. The variational determination of the final coordinate of the motion is possible by following the motion backward in time from the final to the initial time. The description of the motion in both direction in time yields the redoubling.

(ii): The necessary and sufficient generalization of the classical conservative interactions, the semi-holonomic forces, is based by assigning a double role to the coordinate. It stands for the location of the particle and denotes the variational parameter. The separation of these two roles leads to the redoubling.

(iii): Causal interactions with the environment generate time reversal odd terms to the effective action which can be retained by the redoubling.

(iv): A simple way to encode the relevant information of a large environment for a small system is to represent the environment by an ancilla which is comparable with the system in its complexity. In particular, the energy exchange with the environment can trivially be reproduced by using a copy of the system as ancilla where the energy is defined with the opposite sign.

(v): The effective equation of motion of an open system is non-canonical. Any differential equation can be derived by the variational principle with redoubling.

These points can be justified by starting from quantum mechanics. Point (i) follows from the presence of the bra and the ket components in the expectation values, the origin of argument (iii) and (v) can be identified in the path integral representation of the reduced density matrix and the Noether theorem for classical open system, point (iv) can be derived from the Ward identities.

The CTP formalism offers a simple way to derive decoherence, shows its common origin with dissipation, and is helpful is establishing the classical limit. An important difference between the quantum and classical levels is that the redoubling is purely formal in classical physics because the copies are separate only in the virtual variations and the equations of motion send them along the same trajectory. This is not the case anymore in quantum dynamics where the difference between the two trajectory contains the quantum fluctuations and is $\ord{\sqrt{\hbar}}$. Another words, the quantum fluctuations appear as the difference between the copies and quantum and thermal averages have the right number of degrees of freedom. The redoubling makes a wide class of physical phenomenons accessible and offers new points of view suggesting that it should be included into our standard tool box of mechanics.



 \begin{thebibliography}{99}
\bibitem{schwarzschild} K. Schwarzschild, \journal{Göttinger Nachr.}{128}{132}{1903}.
\bibitem{ritz} W. Ritz, \journal{Ann. Chem. Physique}{13}{145}{1908}.
\bibitem{tetrode} H. Tetrode, \journal{Zeits. Physik}{10}{317}{1922}.
\bibitem{wheeler} J. A. Wheeler, R. P. Feynman, \journal{Rev. Mod. Phys.}{17}{157}{1945}; \journal{Rev. Mod. Phys.}{21}{425}{1949}.
\bibitem{currie} D. G. Currie, T. F. Jordan, E. Sudarshan, \journal{Rev. Mod. Phys.}{35}{350}{1963}.
\bibitem{cannon} J. T. Cannon, T. F. Jordan, \journal{J. Math. Phys.}{5}{299}{1964}.
\bibitem{leutwyler} H. Leutwyler, \journal{Il Nuov. Cim.}{37}{556}{1965}.
\bibitem{kerner} E. H. Kerner, \journal{J. Math. Phys.}{6}{1218}{1965}.
\bibitem{pauri} M. Pauri, G. M. Prosperi, \journal{J. Math. Phys.}{17}{1468}{1976}.
\bibitem{feshbach} H. Feshbach, \journal{Ann. Phys. (N.Y.)}{5}{357}{1958}; \journal{Ann. Phys. (N.Y.)}{19}{287}{1962}.
\bibitem{nakajima}S. Nakajima, \journal{Progr. Theor. Phys.}{20}{948}{1958}.
\bibitem{zwanzig} R. Zwanzig, \journal{J. Chem. Phys.}{33}{1338}{1960}.
\bibitem{mori} H. Mori, \journal{Progr. Theor. Phys.}{33}{423}{1965}
\bibitem{redfield} A. G. Redfield, \journal{Adv. Magn. Reson.}{1}{1}{1965}.
\bibitem{gaspard} P. Gaspard, M. Nagaoka, \journal{J. Chem. Phys.}{111}{5676}{1999}.
\bibitem{chruscinski} D. Chruscinski, A. Kossakowski, \journal{Phys. Rev. Lett.}{111}{050402}{2013}.
\bibitem{breuerk} H. P. Breuer, B. Kappler, F. Petruccione, \journal{Ann. Phys.}{291}{36}{2001}.
\bibitem{preverzev} A. Preverzev, E. Bittner, \journal{J. Chem. Phys.}{125}{104906}{2006}.
\bibitem{timm} C. Timm, \journal{Phys. Rev.}{B83}{115416}{2011}.
\bibitem{kidon} L. Kidon, E. Y. Wilner, E. Rabani, \journal{J. Chem. Phys.}{143}{234110}{2015}.
\bibitem{barnett} S. M. Barnett, S. Stenholm, \journal{Phys. Rev.}{A64}{033808}{2001}.
\bibitem{shabani} A. Shabani, D. A. Lidar, \journal{Phys. Rev.}{A71}{020101(R)}{2005}.
\bibitem{maniscalco} S. Maniscalco, \journal{Phys. Rev.}{A72}{024103}{2005}.
\bibitem{lindblad} G. Lindblad, \journal{Comm. Math. Phys.}{48}{119}{1976}.
\bibitem{schw} J. Schwinger, \journal{J. Math. Phys.}{2}{407}{1961};
{\em Particles and Sources}, vol. I., II., and III., Addison-Wesley, Cambridge, Mass. 1970-73.
\bibitem{keldysh} L. V. Keldysh, \journal{Zh. Eksp. Teor. Fiz.}{47}{1515}{1964}
(\journal{Sov. Phys. JETP}{20}{1018}{1965}).
\bibitem{hu} A. E. Calzetta, {\em Nonequilibrium Quantum Field Theory}, Cambridge Univ. Press, Cambridge UK, (2008).
\bibitem{kamenev} A. Kamenev, {\em Non-Equilibrium Systems}, Cambridge Univ. Press, Cambridge UK, (2011).
\bibitem{gu} B. Gu, \journal{Phys. Rev.}{A101}{012121}{2020}.
\bibitem{breuerm} H. P. Breuer, A. Ma, F. Petrussione, in {\em Quantum Computing and Quantum Bits in Mesoscopic Systems} A. J. Leggett, B. Ruggerio, P. Silvestrini eds. Kluwer, New York (2004).
\bibitem{openqft} S. Nagy, J. Polonyi, \journal{Universe}{8}{127}{2022}.
\bibitem{envindta} J. Polonyi, {\em Environment induced time arrow}, arXiv:1206.5781.
\bibitem{galley} C. R. Galley, \journal{Phys. Rev. Lett.}{110}{17}{174301}{2013}.
\bibitem{zeh} H. D. Zeh, {\em The Physical Basis of the Directin of Time}, Springer, Berlin, 1989.
\bibitem{halliwell} J. J. Halliwell, J. Pérez-Mercader, W. H. Zurek eds., {\em Physical origins of Time Asymmetry}, Cambridge Univ. Press, Cambridge UK, (1994).
\bibitem{houghton} L. Mersini-Hougthon, R. Vaas wds., {\em The Arrows of Time}, Springer, Heidelberg, (2012).
\bibitem{albeverio} S. Albeverio, P. Blanchard eds. {\em Direction of Time}, Springer, Heidelberg, (2014).
\bibitem{feynman} R. P. Feynman, F. L. Vernon, \journal{Ann. Phys.}{24}{118}{1963}.
\bibitem{bateman} H. Bateman, \journal{Phys. Rev.}{38}{815}{1931}.
\bibitem{umezawa} H. Umezawa, H. Matsumoto, M. Tachiki, {\em Thermo Field Dynamics and Condensed States}, Norht-Holland, Amsterdam, New York, Oxford (1982).
\bibitem{ines} J. Polonyi, I. Rachid, \journal{Phy. Rev.}{D107}{056010}{2023}.
\bibitem{timescales} J. Polonyi, \journal{Phys. Rev.}{A96}{012104}{2016}.
\bibitem{ballantine} L. E. Ballentine, {\em Quantum Mechanics} World Scientific, Singapore (1998).
\bibitem{klein} U. Klein, \journal{Am. J. Phys.}{80}{1009}{2012}.
\bibitem{zehd} H. D. Zeh, \journal{Found. Phys.}{1}{69}{1970}.
\bibitem{zurekd} W. H. Zurek, \journal{Phys. Rev.}{D24}{1516}{1981}.
\bibitem{joos} E. Joos, H. D. Zeh, \journal{Z. Phys.}{B59}{223}{1985}.
\bibitem{zurekt} W. H. Zurek, in {\em Frontiers of Nonequilibrium Statistical Physics}, ed. G. T. Moore, M. T. Scully Plenum (1986).
\bibitem{macr} J. Polonyi, \journal{Universe}{7}{315}{2021}.
\end{thebibliography}



\appendix
\section{CTP Green function for a classical closed harmonic oscillator}\label{clgrfncta}
The Green function of the CTP formalism is calculated in this appendix for the harmonic oscillator, defined by the Lagrangian $L=m(\dot x^2-\omega_0^2x^2)/2$. Since the external source $j$ is supposed to generate the non-trivial dynamics the trivial initial conditions with vanishing coordinate an velocity are assumed. To facilitate the inversion of the kernel of the action we discretize the time interval $-T<t<0$ by introducing $t_n=n\dt-T$, $n=0,\ldots,N-1$, $\dt=T/N$ and write the CTP action in the form
\be
S=\frac{m}2\sum_{\sigma=\pm}\sigma\sum_{n=0}^{N-1}\left[\frac{(x_{n+1,\sigma}-x_{n,\sigma})^2}\dt-\dt\omega_0^2x_{n,\sigma}^2\right].
\ee
The rule of the partial integration with the finite difference operators $\nabla^+f_n=f_{n+1}-f_n$, $\nabla^-f_n=f_n-f_{n-1}$,
\be
\sum_{n=1}^{N-1}\nabla^+f_ng_n=-\sum_{n=1}^{N-1}f_n\nabla^-g_n+f_Ng_{N-1}-f_1g_0,
\ee
yields the action
\be
S=\hf\sum_{\sigma,\sigma'}\sum_{n,n'=1}^{N-1}x_{n,\sigma}D^{-1}_{0(n,\sigma),(n'\sigma')}x_{n',\sigma'}+\sum_\sigma\sum_{n=1}^{N-1}x_{n,\sigma}B_{n,\sigma}z
\ee
for two separate trajectory segments $(x_{0,\pm},\ldots,x_{N-1,\pm})$ which become the CTP trajectory doublet with the common final point $z=x_{N,\pm}$. The Green function $\hD_0$ is defined by
\be
D^{-1}_{0(n,\sigma),(n'\sigma')}=-\delta_{\sigma,\sigma'}m\left[\sigma\left(\frac1\dt\Delta_{n,n'}+\dt\omega_0^2\delta_{n,n'}\right)-i\dt\epsilon\delta_{n,n'}\right]
\ee
with $\Delta_{n,n'}=(\nabla^-\nabla^+)_{n,n'}=\delta_{n,n'+1}+\delta_{n,n'-1}-2\delta_{n,n'}$ and describes the response to the external source within the time interval $-T<t<0$ with Dirichlet boundary condition. The time-dependent internal source
\be
B^\sigma_t=-\delta_{t,T-\dt}\frac{\sigma m}{\dt}
\ee
represents the common end point. The calculation of $\hD_0$ is easiest by the help of the complete set of eigenfunctions of $\hD^{-1}_0$,
\be
\phi_n(t)=\sqrt{\frac2{T}}\sin\pi\frac{t}{T}n,
\ee
resulting
\be
D_{0++}(t,t')=\frac{2}{Tm}\sum_{n=1}^N\frac{\sin\omega_nt\sin\omega_nt'}{\hat\omega^2_n-\omega_0^2+i\epsilon}
\ee
where $\hat\omega_n=\frac2\dt\sin\frac{\pi}{2N}n$ and $\omega_n=\frac\pi{T}n$. The continuum limit $\dt\to0$ is straightforward,
\bea\label{dpp0}
D_{0++}(t,t')&=&-\frac1{2Tm}\sum_{n=1}^N\frac{(e^{i\frac\pi{T}nt}-e^{-i\frac\pi{T}nt})(e^{i\frac\pi{T}nt'}-e^{-i\frac\pi{T}nt'})}{\frac4{\dt^2}\sin^2\pi\frac{\dt n}{2T}-\omega_0^2+i\epsilon}\nn
&=&\frac1{4\pi m}\int_{-\infty}^\infty d\omega\frac{e^{i\omega(t-t')}-e^{i\omega(t+t')}}
{(\omega-\omega_0+i\epsilon)(\omega+\omega_0-i\epsilon)}\nn
&=&\frac{i}{2m\omega_0}[e^{i(\omega_0-i\epsilon)(t+t')}-e^{-i(\omega_0-i\epsilon)|t-t'|}].
\eea

The full CTP Green function can be obtained by performing the Legendre transformation from the action $S=\hx\hD^{-1}\hx/2+\hj\hx$ written in condensed vector notation by suppressing the indices to the generator functional
\be
W=\hf\hx\hD_0^{-1}\hx+\hx(\hat Bz+\hj)
\ee
where the source $j$ is coupled to the CTP trajectories. The coordinates are eliminated in two steps, first the CTP trajectories are replaced by the solution $\hx=-\hD_0(\hat Bz+\hj)$ leading to
\be
W=-\hf(z\hat B+\hj)\hD_0(\hat Bz+\hj).
\ee
This is followed by the elimination of the common end point, $z=-\hat B\hD_0\hj/\hat B\hD_0\hat B$, resulting $W=-\hj\hD\hj/2$ with
\be\label{ctpgfnt}
\hD=\hD_0-\hD_0\hat B\frac1{\hat B\hD_0\hat B}\hat B\hD_0.
\ee

The last term on the right hand side describes the impact of the coupling of the separate trajectory segments by the common end point and contains
\bea
D_{0++}(t,-\dt)&=&-\frac{2}{Tm}\sum_{n=1}^N\frac{\sin\frac\pi{T}nt\sin\frac\pi{T}n\dt}{(\frac\pi{T}n)^2-\omega_0^2+i\epsilon}\nn
&=&\frac{i}{Tm}\sum_{n=1}^N\frac{(e^{i\frac\pi{T}nt}-e^{-i\frac\pi{T}nt})\frac\pi{T}n\dt}{(\frac\pi{T}n)^2-\omega_0^2+i\epsilon}\nn
&=&\frac{i\dt}{\pi m}\int_{-\infty}^\infty d\omega\frac{\omega e^{i\omega t}}{(\omega-\omega_0+i\epsilon)(\omega+\omega_0-i\epsilon)}\nn
&=&-\frac{\dt}{m}e^{i(\omega_0-i\epsilon)t}
\eea
and
\bea
D_{0++}(-\dt,-\dt)&=&\frac2{Tm}\sum_{n=1}^N\frac{\sin^2\frac\pi{T}n\dt}{\frac{4}{\dt^2}\sin^2\pi\frac{\dt n}{2T}-\omega_0^2+i\epsilon}\nn
&=&\frac{2\dt^2}{mT}\sum_{n=1}^N\frac{\frac4{\dt^2}\sin^2\frac{\dt\omega_n}2(1-\sin^2\frac{\dt\omega_n}2)}{\frac4{\dt^2}\sin^2\frac{\dt\omega_n}2-\omega_0^2+i\epsilon}.
\eea
It is advantageous to split the latter sum into two parts,
\be
D_{0++}(-\dt,-\dt)=\frac{2\dt^2}{mT}\sum_{n=1}^N\left(1-\sin^2\frac{\dt\omega_n}{2}\right)
+\frac{2\dt^2\omega_0^2}{mT}\sum_{n=1}^N\frac{1-\sin^2\frac{\dt\omega_n}2}{\frac{4}{\dt^2}\sin^2\frac{\dt\omega_n}2-\omega_0^2+i\epsilon},
\ee
giving
\bea
D_{0++}(-\dt,-\dt)&=&\frac{2\dt}{m\pi}\int_0^\pi d\omega\left(1-\sin^2\frac{\omega}{2}\right)+\frac{2\dt^2\omega_0^2}{m\pi}\int_0^\infty d\omega\frac1{\omega^2-\omega_0^2+i\epsilon}\nn
&=&\frac\dt{m}-i\frac{\dt^2\omega_0}m
\eea
for small $\dt$. We can finally assemble the CTP Green function \eq{ctpgfnt},
\bea
\hD(t,t')&=&-\frac{i}{2m\omega_0}\Biggl[\begin{pmatrix}e^{-i\omega_0|t-t'|}-e^{i\omega_0(t+t')}&0\cr0&e^{i\omega_0|t-t'|}-e^{-i\omega_0(t+t')}\end{pmatrix}\nn
&&+\begin{pmatrix}e^{i\omega_0(t+t')}&e^{i\omega_0(t-t')}\cr e^{-i\omega_0(t-t')}&e^{-i\omega_0(t+t')}\end{pmatrix}\Biggr].
\eea

Two comments are in order at this point, the first concerns the time arrow. The first matrix describes the response of the CTP trajectory with Dirichlet boundary conditions to the external source. This is the sum of two terms: The first is an acausal signal which is non-oriented in time and becomes the result of the near Green function. The second just cancels the response to the common final point propagating back in time, given by the second matrix. The off-diagonal terms couple the two trajectories with a mass-shell mode, to become the effect of the far Green function and carries a time arrow. The other comment is about the violation of the translation invariance in time. The trivial initial condition is obviously translation invariant in time however the final conditions, the vanishing of the trajectory segments at $-\dt$ and the common final value break this symmetry one by one but their sum turns out to be symmetrical.  In other words, the time inversion, displayed in Fig. \ref{ctppathf} preserves the translation invariance in time and the CTP Green function is diagonal in the frequency space,
\be\label{gfctinft}
\hD(t,t')=-\frac{i}{2m\omega_0}\begin{pmatrix}e^{-i\omega_0|t-t'|}&e^{i\omega_0(t-t')}\cr e^{-i\omega_0(t-t')}&-e^{i\omega_0|t-t'|}\end{pmatrix}.
\ee
Note that the explicit translation invarince in time, the independence of the Green function on $t+t'$, is a necessary condition of having a time arrow.



\section{Equation of motion in quantum mechanics}\label{eomqms}
The variation of the trajectory of classical mechanics $x(t)\to x(t)+\delta x(t)$ serves to identify the unique physical trajectory by the help of the variational principle $\delta S[x]=\ord{\delta x^2}$. Thought there is no unique trajectory in quantum mechanics the variation of the trajectories in the path integral can be considered as a change of integral variable and the invariance of the path integral yields important relations among physical quantities.

The most detailed information about the dynamics is contained in the generator functional for the Green functions \eq{genfunct}. The CTP path integral for a closed system with a functional of the trajectory $F[\hx]$ in the integrand can be converted into an expectation value
\be\label{schraver}
\int D[\hx]e^{\ih S[\hx]}F[\hx]=\Tr[T[F[\hx]U(t,t_i)\rho_iU^\dagger(t,t_i)]
\ee
where $T$ denotes the CTP time ordering which places right (left) the $+$ ($-$) operators in (anti)chronological order in the Schrödinger representation. The right hand side can be written in a simpler form of an expectation value in the Heisenberg representation
\be\label{heisav}
\int D[\hx]e^{\ih S[\hx]}F[\hx]=\la T[F[\hx]]\ra.
\ee
In the case of closed bipartite system with the coordinates $x$ and $y$ one integrates the $\hy$ trajectories and the remaining integration over $\hx$ is done by the help of the bare effective action as in \eq{otprdm}. There is no Heisenberg representation in open dynamics nevertheless the simplified notation of \eq{heisav} can be used to have formal but more readably equations.

The independence of the generator functional \eq{genfunct} on the variation of the trajectories defined on the infinite real axis $-\infty<x<\infty$ yields
\be\label{varstart}
0=\int D[\hx]e^{\ih S[\hx]+\ih\int dt\hj(t)\hx(t)}\int dt\left[\delta\hx(t)\pd{L}{\hx(t)}+\delta\dot{\hx}(t)\pd{L}{\dot{\hx}(t)}+\delta\hx(t)\hj(t)\right]
\ee
for arbitrary $\delta\hx(t)$. A partial integration results
\be
\frac{d}{dt}\int D[\hx]e^{\ih S[\hx]+\ih\int dt\hj(t)\hx(t)}\pd{L}{\dot{\hx}(t)}=\int D[\hx]e^{\ih S[\hx]+\ih\int dt\hj(t)\hx(t)}\left[\pd{L}{\hx(t)}+\delta\hx(t)\hj(t)\right].
\ee
The successive functional derivatives generate a hierarchical set of equations for the physical case $\hj=0$,
\bea
\frac{d}{dt}\int D[\hx]e^{\ih S[\hx]}\pd{L}{\dot{\hx}(t)}x_{\sigma_1}(t_1)\cdots x_{\sigma_n}(t_n)&=&\int D[\hx]e^{\ih S[\hx]}\pd{L}{\hx(t)}x_{\sigma_1}(t_1)\cdots x_{\sigma_n}(t_n)\nn
&&\hskip-5cm-i\hbar\sum_{j=1}^n\delta(t-t_j)\int D[\hx]e^{\ih S[\hx]}x_{\sigma_1}(t_1)\cdots x_{\sigma_{j-1}}(t_{j-1})x_{\sigma_{j+1}}(t_{j+1})\cdots x_{\sigma_n}(t_n)
\eea
which can be written in terms of expectation values as
\bea\label{expvaleom}
&&\la T\left[\pd{L}{\hx(t)}x_{\sigma_1}(t_1)\cdots x_{\sigma_n}(t_n)\right]\ra-\frac{d}{dt}\la T\left[\pd{L}{\dot{\hx}(t)}x_{\sigma_1}(t_1)\cdots x_{\sigma_n}(t_n)\right]\ra\nn
&&=i\hbar\sum_{j=1}^n\delta(t-t_j)\la T[x_{\sigma_1}(t_1)\cdots x_{\sigma_{j-1}}(t_{j-1})x_{\sigma_{j+1}}(t_{j+1})\cdots x_{\sigma_n}(t_n)]\ra
\eea
with $n=0,1,2,\ldots$.

The extraction of the time derivative from the chronological product is a left over of the hierarchical structure: In the first step ($n=0$) the generalized momentum appears alone in the expectation value hence there is no chronological product and the extraction is trivial. Any further functional derivative with respect to the source ($n\ge1$) brings a coordinate down into the integrand but the time derivative is already outside of the chronological product.

While the expectation value of the equation of motion is vanishing ($n=0$) its insertion into the Green functions ($n\ge1$) in such a manner that the time derivative acts on the expectation value leads to non-vanishing result. This follows from the chronological product translating the path integral into expectation value in eq. \eq{schraver}, namely there is a non-trivial, non-intuitive contribution to the expectation value whenever one of the legs of the Green function coincide with the local equation of motion. Its origin can be understood by recalling that in the interaction representation of closed dynamics the coordinate operator contains the creation and annihilation operator of the harmonic oscillator defined by the quadratic part of the action. The right hand side of eq. \eq{expvaleom} is non-vanishing whenever an elementary excitation is created or annihilated, a process not covered by the equation of motion.






\section{Interpolating trajectories}\label{interps}
The path integral contains the bare action
\be
S_B=\dt\sum_{n=0}^{N-1}\left[\frac{m}2\left(\frac{x_{n+1}-x_n}\dt\right)^2-U(x_n)\right]
\ee
for the discrete trajectory defined by the set of points $\{x_n\}$, $n=0,\ldots,N$. To recover the external symmetries of the action one replaces the discrete trajectory with a twice continuously differentiable function $x(t)$ constructed in such a manner that its action
\be
S_I=\int dt\left[\frac{m}2\dot x(t)-U(x(t))\right]
\ee
remains close to $S_B$. This construction is possible only for ''reasonable`` discrete trajectories which dominate the path integral in the limit $\dt\to0$. The $x_n$-dependence of the  bare action $S_B$ comes dominantly from the kinetic energy for small $\dt$. Thus the contribution of a trajectory to the path integral survives the integration over $x_n$ if the jumps $\Delta x_n=x_{n+1}-x_n$ satisfy the inequality $m\Delta x_n^2/2\hbar\dt\ll2\pi$. The interpolating trajectory is introduced for discrete set of points $\{x_n\}$ satisfying the inequality $\Delta x_n\ll\sqrt{\hbar\dt/m}$.

First we introduce an ellipsoid around each edge of the discrete trajectory $\{(t_n,x_n)\}$ with horizontal and vertical main radius $r_t=\eta\dt$, $r_x=\eta\sqrt{\hbar\dt/m}$,  $\eta$ being a dimensionless small number. The interpolating trajectory agrees with the piecewise linear approximation of the discrete trajectory $\{(t_n,x_n)\}$ outside of the ellipsoids. Within the the ellipsoid around $(t_n,x_n)$ we use the interpolating curve $t=t_n+\rho(\alpha)\cos\alpha$, $x=x_n+\rho(\alpha)\sin\alpha$ where $\rho(\alpha)$ is a 7-th order polynomial chosen in such a manner that the interpolating trajectory is twice continuously differentiable.

The velocity independent part of the bare discrete action obviously converges to the continuous expression in the limit $\dt\to0$. The kinetic energy of the discretized and the continuous action differ only within the contributions from the ellipsoid. The contribution of the interpolating trajectory to the kinetic energy,
\be
\frac{m}2\int_{\alpha_n}^{\alpha_{n+1}}d\alpha\frac{(\rho'(\alpha)\sin\alpha+\rho(\alpha)\cos\alpha)^2}{\rho'(\alpha)\cos\alpha-\rho(\alpha)\sin\alpha},
\ee
is $\ord{\eta\dt}$ since the integrand remains bounded. Similar bound holds for the contribution between the ellipsoids. Hence the relative difference between the contribution of the kinetic energy to the discretized and the interpolating action is $\ord\eta$. It is not difficult to generalize the argument for any action as long as the bound $\Delta x_n\ll\sqrt{\hbar\dt/m}$ holds.


\end{document}

