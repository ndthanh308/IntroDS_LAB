\section{Web User Interface}\label{sec:ui}
Based on our empirical study results and observations, a fully automated ontology distillation process is possible. However, it may lead to unpredictable and irrelevant ontology results due to the randomness in the responses and the butterfly effect. Manual supervision and early intervention are still required to guarantee distillation quality, improve efficiency and save potential costs (e.g., from repeated trials). To facilitate this, we develop a web-based domain ontology distillation assistant as shown in Figure~\ref{fig:website-ui}. The website has four sub-pages corresponding to the four distillation tasks. In the prompt engineering section, all the essential components are rendered as independent editable text areas for maximum flexibility, e.g., the user may change the instruction part from ``\textit{Add 10 new relevant concepts, ..., to the ontology}'' to ``\textit{Add 10 new concepts under the Vehicle class}''. The execution log contains the complete history of both prompts and ChatGPT's responses in each iteration. After ChatGPT's response is logged, the entire log is parsed, the ontology is updated, the visualization is refreshed, and the prompt for the next iteration will be generated. To facilitate manual supervision and early intervention, the user can then decide whether to continue the next step or make necessary adjustments to the ontology or prompt during the entire execution loop. Currently, extensive engineering effort is underway to improve the assistant tool's usability and design across transportation application domains, and we are pleased to open-source it soon.

% Figure environment removed
