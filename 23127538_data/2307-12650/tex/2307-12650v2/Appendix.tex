\appendix

\section{Details of simulation environment} \label{App:Sim_details}

The simulation environment for solving the governing Navier-Stokes equations is adapted from \citet{rabault_artificial_2019} to the flow past a square bluff body. The boundary condition at the inflow boundary $\Gamma_I$ is set as a uniform velocity profile, and a zero-pressure condition is used at the outflow boundary $\Gamma_O$. Free-stream condition is used at the top and bottom boundary $\Gamma_D$ of the domain. The boundary on the bluff body is separated into body surface $\Gamma_W$ and jet area $\Gamma_j$, with a non-slip boundary condition and jet velocity profile, respectively. All the boundary conditions are formulated as
\begin{equation}
\begin{aligned}
u & = && U_{\infty} & \text { on } &\Gamma_I, \\
p & = && 0 & \text { on } &\Gamma_O, \\
u & = && U_{\infty} & \text { on } &\Gamma_D, \\
u & = && 0 & \text { on } &\Gamma_W, \\
\boldsymbol{u} & = && \boldsymbol{U_{j}} & \text { on } &\Gamma_j, j=1,2.
\end{aligned}
\label{eq:BCs}
\end{equation}

The mesh of the simulation domain and a zoom-in view of the mesh around the square bluff body are presented in Fig. \ref{fig:Mesh}. The mesh is refined in the wake region with a ratio of 0.45 and near the body wall with a ratio of 0.075, with respect to the mesh size of the far field. Near the jet area, the mesh is further refined with a ratio of 0.015. More details can be found in the source code (see GitHub repository).

% Figure environment removed


\section{Hyperparameters of RL} \label{App:Hyperparameters}

The RL hyperparameters to reproduce the result section (\S\ref{sec:Results}) are listed in table \ref{tab:hyperparams}.
%
\begin{table}
  \begin{center}
\def~{\hphantom{0}}
\begin{tabular}{l|l}
Hyperparameter & Value \\
\hline optimiser & Adam \\
learning rate & $10^{-4}$ \\
discount $(\gamma)$ & 0.99 \\
replay buffer size & $10^5$ \\
number of hidden layers (both actor and critic) & 3 \\
number of hidden units per layer & 512 \\
number of samples per minibatch & 128 \\
entropy target & $-\operatorname{dim}(\mathcal{A})$ \\
activation & $\operatorname{ReLU}$ \\
target smoothing coefficient $(\tau)$ & $5 \cdot 10^{-3}$ \\
target update interval & 1 \\
gradient steps & 48 \\
top quantiles to drop per net & 2 \\
number of quantiles per net & 25 \\

\end{tabular}
  \caption{   {Hyperparameters used by default in TQC. For SAC, ``top quantiles to drop per net'' is not used, and other parameters remain the same. For the entropy target, $-\operatorname{dim}(\mathcal{A})$ denotes the dimension of action space $\mathcal{A}$.}}
  \label{tab:hyperparams}
  \end{center}
\end{table}

\section{A long-run test of RL-trained controller} \label{App:LongEva}

In figure \ref{fig:LongEva}, the trained policy is tested for a longer time (400 time units) than training (200 time units) to show the control stability outside the training timeframe for the dynamic controller in the PM environment.
The initial condition of this long-run test is different compared to figure \ref{fig:TQC_FMPM}, indicating the adaptability of the controller to different initial conditions. Other parameters in this run are consistent with the results in figure \ref{fig:TQC_FMPM}.

% Figure environment removed

The control performance and behaviour in this test are consistent with the results shown in figure \ref{fig:TQC_FMPM} both in the transient stage and asymptotic stage. The drag coefficient $C_D$ starts from the condition of steady vortex shedding and drops to the value of the stabilised flow in around 120 time units, with minor fluctuations. After training time (200 time units), the controller is still able to prevent triggering vortex shedding and preserve the drag coefficient near the baseflow values (minimum drag without vortex shedding). The behaviour of the controller is further presented in the subfigures of $Q_1$. The controller creates negligible random mass flow after stabilizing the vortex shedding due to the maximum entropy in training.


\section{Base flow simulation} \label{App:BaseFlow}

The base flow corresponds to a steady equilibrium of the governing Navier-Stokes equations. This fixed point is unstable to infinitesimal perturbations, giving rise to vortex shedding. The base flow is obtained by simulating only half of the domain, as shown in figure \ref{fig:Baseflow}, which prevents antisymmetric vortex shedding. The boundary conditions are consistent with Eq. \req{eq:BCs} while a symmetric boundary condition is applied on the bottom boundary (symmetry line) of the domain, i.e. on $y = 0$. The symmetric boundary condition is imposed as $v = 0$, $\frac{\partial u}{\partial y} = 0$ and $\frac{\partial p}{\partial y} = 0$.

In this case, the vortex shedding is not triggered, as shown in the contour of figure \ref{fig:Baseflow}, and the only cause of the pressure drag is flow separation. Therefore, comparing the pressure drag in a full-domain simulation of uncontrolled flow, where the vortex shedding is triggered, with this base flow, the amount of pressure drag due to flow unsteadiness can be estimated. As only the unsteady component of pressure drag can be effectively reduced by flow control \citep{bergmann_optimal_2005}, the control performance can be evaluated with respect to this base flow (Eq.\req{eq:drag_reduction}). The drag coefficient of the half square body measures $C_{Dh} = 0.618$, and the base flow drag coefficient of the whole body can be obtained as $C_{Db} = 2C_{Dh} = 1.236$. 

% Figure environment removed
