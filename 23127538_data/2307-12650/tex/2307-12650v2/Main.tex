% This is file JFM2esam.tex
% first release v1.0, 20th October 1996
%       release v1.01, 29th October 1996
%       release v1.1, 25th June 1997
%       release v2.0, 27th July 2004
%       release v3.0, 16th July 2014
%   (based on JFMsampl.tex v1.3 for LaTeX2.09)
% Copyright (C) 1996, 1997, 2014 Cambridge University Press

\documentclass{jfm}
% \usepackage{hyperref}
\usepackage{xcolor}
\usepackage{ulem}
\normalem
\usepackage{graphicx}
\usepackage{epstopdf, epsfig}
\usepackage{caption}
\usepackage{subcaption}
\usepackage{mathtools}
\def\req#1{(\ref{#1})} % Equation reference

\newtheorem{lemma}{Lemma}
\newtheorem{corollary}{Corollary}

\shorttitle{AFC by RL with Partial Measurements}
\shortauthor{C. Xia, J. Zhang, E. C. Kerrigan and  G. Rigas}

\title{Active Flow Control for Bluff Body Drag Reduction Using Reinforcement Learning with Partial Measurements}

\author{Chengwei Xia\aff{1},
  Junjie Zhang\aff{1},
  Eric C. Kerrigan\aff{1,2},
  \and Georgios Rigas\aff{1}\corresp{\email{g.rigas@imperial.ac.uk}}}

\affiliation{\aff{1}Department of Aeronautics, Imperial College London,
London, SW7 2AZ, UK
\aff{2}Department of  Electrical and Electronic Engineering, Imperial College London, London, SW7 2AZ, UK}

\begin{document}

\maketitle

\begin{abstract}
Active flow control for drag reduction with reinforcement learning (RL) is performed in the wake of a 2D square bluff body at laminar regimes with vortex shedding. Controllers parameterised by neural networks are trained to drive two blowing and suction jets that manipulate the unsteady flow. RL with full observability (sensors in the wake) successfully discovers a control policy which reduces the drag by suppressing the vortex shedding in the wake. However, a non-negligible performance degradation ($\sim$ 50\% less drag reduction) is observed when the controller is trained with partial measurements (sensors on the body). To mitigate this effect, we propose an energy-efficient, dynamic, maximum entropy RL control scheme. First, an energy-efficiency-based reward function is proposed to optimise the energy consumption of the controller while maximising drag reduction. Second, the controller is trained with an augmented state consisting of both current and past measurements and actions, which can be formulated as a nonlinear autoregressive exogenous model, to alleviate the partial observability problem. Third, maximum entropy RL algorithms  (Soft Actor Critic and Truncated Quantile Critics) which promote exploration and exploitation in a sample efficient way are used and discover near-optimal policies in the challenging case of partial measurements. Stabilisation of the vortex shedding is achieved in the near wake using only surface pressure measurements on the rear of the body, resulting in similar drag reduction as in the case with wake sensors. The proposed approach opens new avenues for dynamic flow control using partial measurements for realistic configurations.


\end{abstract}

%\begin{keywords}
%Authors should not enter keywords on the manuscript, as these must be chosen by the author during the online submission process and will then be added during the typesetting process (see http://journals.cambridge.org/data/\linebreak[3]relatedlink/jfm-\linebreak[3]keywords.pdf for the full list)
%\end{keywords}


The problem of the presence or absence of phase transition is central in statistical mechanics. To prove the existence of phase transition, the standard idea is to define a notion of contour and use \textit{Peierls' argument} \cite{Peierls.1936}. In the usual Ising model \cite{Ising_25}, particles of the system interact only with their nearest-neighbors. On ferromagnetic long-range Ising models \cite{Anderson_Yuval_69}, there is interaction between each pair of spins in the lattice. The Hamiltonian of the model is given formally by
\begin{equation*}
    H(\sigma) = - \sum_{x,y\in \Z^d}J_{xy}\sigma_x\sigma_y,
\end{equation*}
where $J_{xy}=J|x-y|^{-\alpha}$, $J>0$, $\alpha > d$. It is well-known that the Peierls' argument in dimension 2 implies phase transition for Ising models with nearest-neighbors or long-range interactions when $d\geq 2$, using correlation inequalities. For the unidimensional lattice, it was known that short-range models do not present phase transition. In the long-range case, a different behavior was expected depending on the exponent $\alpha$ (see \cite{Kac_Thompson_69}), but the problem was challenging since contours were first created as multidimensional objects.

In dimension $d=1$, phase transition was proved first in 1969 by Dyson \cite{Dyson.69}, for $\alpha \in (1,2)$, by proving phase transition in an auxiliary model and then using correlation inequalities. In 1982, Fr{\"o}hlich and Spencer \cite{Frohlich.Spencer.82} introduced a notion of one-dimensional contours and then applied the Peierls' argument to show phase transition for the critical value $\alpha = 2$. These contours were inspired by the multiscale techniques previously introduced to study the Berezinskii-Kosterlitz-Thouless transition in two-dimensional continuous spin systems \cite{FS81}. Later, Cassandro, Ferrari, Merola and Presutti  \cite{Cassandro.05} extended the contour argument previously available for $\alpha=2$ to exponents $\alpha\in (3-\frac{\ln 3}{\ln 2}, 2)$, with the additional restriction that the nearest-neighbor interaction is strong, i.e.,  ${J(1)\gg 1}$; this restriction was removed for a subclass of interactions in \cite{Bissacot.Endo.18}. Further results were obtained using contour arguments, such as the decay of correlations, cluster expansions, phase transition with random interactions, etc; some references with these results are \cite{ Cassandro.Merola.Picco.17, Cassandro.Merola.Picco.Rozikov.14, Imbrie.82, Imbrie.Newman.88, Johansson.91}. 

In the multidimensional setting ($d\geq 2$), Ginibre, Grossmann, and Ruelle, in \cite{Ginibre.Grossmann.Ruelle.66}, proved the phase transition for $\alpha > d+1$, using an enhanced version of Peierls' argument and the usual contours. Park proposed a different notion of contour for long-range systems in \cite{Park.88.I, Park.88.II}, extending the Pirogov-Sinai theory available for short-range interactions assuming $\alpha > 3d+1$, although he can also consider Potts models with his methods. Some results in the literature suggest that truly long-range effects appear only when $d < \alpha \leq d+1$, see for instance, \cite{Biskup_Chayes_Kivelson_07}. Recently, Affonso, Bissacot, Endo and Handa \cite{Affonso.2021}, inspired by the ideas from Fr{\"o}hlich and Spencer in \cite{FS81, Frohlich.Spencer.82}, introduced a version of multiscale multidimensional contour and proved phase transition by a contour argument in the whole region $\alpha > d$. They can consider long-range Ising models with deterministic decaying fields, first introduced in the context of nearest-neighbor interactions in \cite{Bissacot_Cioletti_10}. For these models, the lack of analyticity of the free energy does not imply phase transition since these models have the same free energy as the models with zero field. It is expected that fields decaying slowly imply uniqueness. In this setting, a contour argument is useful for proofs of phase transitions as well for uniqueness, some papers with models with deterministic decaying fields are \cite{Aoun_Ott_Velenik_23, Bissacot_Cass_Cio_Pres_15, Bissacot.Endo.18, Cioletti_Vila_2016}.

The Random Field Ising model (RFIM) \cite{Imry.Ma.75} is the nearest-neighbor Ising model with an additional external field acting on each site $(h_x)_{x\in\Z^d}$ that is a family of i.i.d. Gaussian random variable with mean 0 and variance 1. Formally, the Hamiltonian of the model is given by
\begin{equation*}
    H(\sigma) = - \sum_{\substack{x,y\in \Z^d \\|x-y|=1}}J\sigma_x\sigma_y  - \varepsilon\sum_{x\in\Z^d}h_x\sigma_x,
\end{equation*}
where $J>0$, $\varepsilon>0$, $\alpha > d$ and $d \geq 1$. A detailed account of the history of the phase transition problem for this model, as well as detailed proofs, was given in \cite{Bovier.06}. Here we present a brief overview.

During the 1980s, the question of the specific dimension where phase transition for the RFIM should happen attracted much attention and was a topic of heated debate. Two convincing arguments were dividing the physics community. One of them, due to Imry and Ma \cite{Imry.Ma.75}, was a non-rigorous application of the Peierls' argument together with the use of the isoperimetric inequality. The key idea of Peierls' argument is to define a notion of contour and calculate the energy cost of "erasing" each contour, i.e., the energy cost of flipping all spins inside the contour. When there is no external field, that energy necessary to flip the spins in a region $A\subset \Z^d$ is of the order of the boundary $|\partial A|$. When we add an external field, we get an extra cost depending on this field. Imry and Ma argued that this cost should be approximately $\sqrt{|A|}$, which is smaller than $|\partial A|$ for all regions only when $d\geq 3$, so this should be the region where phase transition occurs. The other argument, due to Parisi and Sourlas \cite{Parisi.Sourlas.79}, based on dimensional reduction, predicted that the $d$-dimensional RFIM would behave like the $d-2$-dimensional nearest-neighbor Ising model, therefore presenting phase transition only when $d\geq 4$. 

The question was settled by two celebrated papers showing that Imry and Ma's prediction was correct. First, in 1988, Bricmont and Kupiainen \cite{Bricmont.Kupiainen.88} showed that there is phase transition almost surely in $d\geq3$, for low temperatures and variance $\varepsilon$ small enough. Their proof uses a rigorous renormalization group analysis for the short-range case and it is considered involved. Still, they claimed that the result works for any model with a suitable contour representation and centered sub-gaussian external field. Later on, Aizenman and Wehr \cite{Aizenman.Wehr.90} proved uniqueness for $d\leq 2$. For detailed proofs of these results, we refer the reader to \cite{Bovier.06} (see also \cite{Berretti.85, Camia.18, Frohlich.Imbre.84,  Klein.Masooman.97} for more uniqueness results). 

Recently, Ding and Zhuang, see \cite{Ding2021}, provided a simpler proof of the phase transition, not using RGM. And in  \cite{Ding.Liu.Xia.22}, Ding, Liu and Xia proved that if $\beta_c(d)$ is the critical inverse of the temperature of the Ising model with no field, for all $\beta>\beta_c(d)$ there exists a critical value $\varepsilon_0(d, \beta)$ such that the RFIM with $\varepsilon \leq \varepsilon_0$ presents phase transition. 

In the present paper, we are considering a long-range Ising model with a random field, whose Hamiltonian is given formally by
\begin{equation*}
    H(\sigma) = - \sum_{x,y\in \Z^d}J_{xy}\sigma_x\sigma_y - \varepsilon\sum_{x\in\Z^d}h_x\sigma_x,
\end{equation*}
where $J_{xy}=J|x-y|^{-\alpha}$, $J, \varepsilon>0$, $\alpha > d$ and $h_x\in\mathbb{R}$, $d\geq 3$.
Until now, the only known result in the long-range setting is for the one-dimensional long-range Ising model with a random field, by Cassandro, Orlandi, and Picco \cite{Cassandro.Picco.09}. They used the contours of \cite{Cassandro.05} to show the phase transition for the model when $\alpha\in (3-\frac{\ln 3}{\ln 2}, \frac{3}{2})$, under the assumption $J(1) \gg 1$. We stress that, as remarked by Aizenman, Greenblatt, and Lebowitz \cite{Aizenman_Greenblatt_Lebowitz_2012}, although their argument does not work for the whole region for the exponent $\alpha$, the phase transition holds for values close to the critical value $\alpha=3/2$, since by the Aizenman-Wehr theorem we know that there is uniqueness for $\alpha>3/2$.

The argument from Ding and Zhuang in \cite{Ding2021}, for $d\geq3$, involves controlling the probability of a bad event, which is closely related to controlling the quantity $$\sup_{\substack{0\in A\subset\Z^d \\ A \text{ connected }}}\frac{\sum_{x\in A}h_x}{|\partial A|},$$ known as the greedy animal lattice normalized by the boundary. The greedy animal lattice normalized by the size, instead of the boundary, was extensively studied for general distributions of $(h_x)_{x\in\Z^d}$, see \cite{Cox_Gandolfi_Griffin_Kesten_93, Gandolfi_Kesten_94, Hammond_06, Martin_02}. When we normalize by the boundary, an argument by Fisher, Fr\"{o}hlich and Spencer \cite{FFS84} shows that the expected value of the greedy animal lattice is constant. In dimension $d=2$, the expected value is not finite, see \cite{Ding.Wirth.20}. The supremum is taken over connected regions containing the origin since the interiors of the usual Peierls contours are of this form.


For the long-range model, the interior of contours is not necessarily connected. In fact, long-range contours may have considerably large diameters with respect to their size, so their interiors can be very sparse. To avoid this, we define contours, strongly inspired by the $(M,a,r)$-partition in \cite{Affonso.2021}, using a multiscaled procedure that assures that the contours have no cluster with small density.  With them, we generalize the arguments by Fisher-Fr\"{o}hlich-Spencer \cite{FFS84}, and prove that the expected value of the greedy animal lattice is constant, even considering regions not necessarily connected in the supremum. Then, we prove the phase transition for $d\geq 3$. The main result of this paper is the following.
\begin{theorem*}Given $d\geq 3$, $\alpha>d$, there exists $\beta_c\coloneqq\beta(d, \alpha)$ and $\varepsilon_c\coloneqq\varepsilon(d, \alpha)$ such that, for $\beta >\beta_c$ and $\varepsilon\leq \varepsilon_c$, the extremal Gibbs measures $\mu_{\beta, \varepsilon}^+$ and $\mu_{\beta, \varepsilon}^-$ are distinct, that is, $\mu_{\beta, \varepsilon}^+ \neq \mu_{\beta, \varepsilon}^-$ $\mathbb{P}$-almost surely. Therefore the long-range random field Ising model presents phase transition.
\end{theorem*}

This paper is divided as follows. In Section 2, we define the model and the contours, and suitable generalizations to the constructions in \cite{Affonso.2021} are introduced.  In Section 3, we define two bad events of the external field and prove that they occur with a small probability.  In Section 4, we present the proof of the phase transition.

\section{Methodology}
\label{sec:method}
% Figure environment removed


%\vspace*{0.2cm}\noindent\textbf{Problem formulation.} 
% Let $O$ be the set of Objects,  $S$ be the set of
%  States and $\textit{I}$ the set of
%   Images which consists of the disjoint sets ${I^{S}}$ and ${I^{U}}$ that are used during the training and testing phase respectively. 
%    Each image $i_{k} \in \textit{I}$  contains an object $o_{i} \in O$ which  is situated in a state $s_{j} \in S$. The OSC task deals with the yielding  of a  predicted state label $sp_{j} \in S$ for an image $i_{k} \in {I^U}$ that has been given as an input. In the zero-shot variation of OSC, ${S^{S}} 
%   \not \supseteq {S^{U}}$, i.e. some of the states contained in the testing images do not appear in the training images. 
% Let $O$ denote a set of objects, 
% \textcolor{red}{$S^S$ as the set of known object states found in the training images, $S^U$ as the test, yet unknown, object state labels} and $I$ the set of images, which is partitioned into the training set $I^T$ and the test set $I^U$. 
% % Each image $i \in I$ contains an object $o \in O$ in a state $s \in S$. 
% \textcolor{red}{Each image $i \in I^T$ contains an object $o \in O$ in a state $s \in S^S$, while an image $i \in I^U$ contains an object $o \in O$ in any state $s \in S = \{S^S \cup S^U\}$. }
% The goal of OSC is to predict the state $s \in S$, given the object $o$ in $i \in I^U$. In the zero-shot variation of OSC, the set of states observed in the test images $S^U$ is not a subset of the set of states observed in the training images $S^S$, i.e., there exists some states in the test image set that do not appear in the training set. Furthermore, the task should be addressed in an object-agnostic manner, i.e. no information concerning the object classes is to be utilized explicitly.  However,  although the set of object classes does not directly affect the task of OaSC, its size is proportional to the complexity of the problem. 
% The workflow of the proposed method is shown in \autoref{fig:pipeline}.

Let $O$ denote a set of objects, $S$ denote the set of states and $I$ denote the set of images, which is partitioned into the training set $I^T$ and the testing set $I^U$. Each image $i \in I$ contains an object $o \in O$ in a state $s \in S$. 
The goal of OSC is to predict the state $s \in S$, given the object $o$ in $i \in I^U$. In the zero-shot variation of OSC, the set of states observed in the test images $S^U$ is not a subset of the set of states observed in the training images $S^S$, i.e., there exists some states in the test image set that do not appear in the training set. Furthermore, the task should be addressed in an object-agnostic manner, i.e. no information concerning the object classes is to be utilized explicitly.  However,  although the set of object classes does not directly affect the task of OaSC, its size is proportional to the complexity of the problem. 
The workflow of the proposed method is shown in \autoref{fig:pipeline}.




% and will be analyzed in the following sections.

% \vspace*{0.2cm}\noindent\textbf{Approach.}

\subsection{Overview}
% Our method is inspired by works that address the problem of zero-shot object classification \cite{}. The main idea behind this line of work is that the necessary information for the classification of the unseen classes can be found in a Knowledge Graph (KG) if processed appropriately by a Graph Neural Network (GNN). Obviously, the most crucial component of this approach lies in the combination of the visual information stemming from the training images and referring to the seen classes with the semantic information stemming from the KG  and referring to the unseen classes.

We are inspired by prior research on zero-shot object classification and leverage the potential of KGs and GNNs to classify previously unseen objects~\cite{Kampffmeyer2019,nayak:tmlr22}. 
The core idea is that semantic information that is stored in the KG can be used by GNNs to learn graph embeddings that can be utilized jointly with visual information extracted from training images. 
This enables the model to generalize to new object classes by leveraging the semantic and contextual information encoded in the graph embeddings of the KG.

% More in detail, the GNN architecture is adopted to the architecture of the Classifier  that is used for the training on seen classes, the GNN last layer has the same size  with the Classifier last layer. This way the GNN can produce semantic embedding features that correspond to all the classes, both seen and unseen, that will be encountered during the inference. These embedding features  replace the last layer of the Classifier. Holding this layer fixed, the body of the Classifier is then fine-tuned with the training images.

GNNs are designed to operate on graph-structured data, such as KGs~\cite{kipf2016semi,Monka2022}. KGs are typically represented as labeled multi-graphs, where nodes correspond to entities, and edges represent entity relationships. GNNs process this graph by iteratively aggregating information from neighboring nodes, using neural network-based operations.

At each iteration, a GNN receives a feature vector for each graph node, which is initially set to the node's embedding vector. Then, the GNN performs a message-passing step that aggregates information from neighboring nodes, based on the edge weights and the features of the nodes. This message-passing operation can be formulated as a neural network layer, which applies a learnable function to the features of the neighboring nodes and returns an aggregated message for each node. After the message-passing step, the GNN updates the node features by applying a learnable transformation that takes into account the original features of the node and the received messages from its neighbors. This updated feature vector is then passed to the next iteration of the message-passing step. The process continues until a fixed number of epochs or convergence.
%%%AAA: Endexetai na mas rethrown gia tis times aytwn twn parametrwn?
% KP edw anaferetai genika mia diadikasia GNN training. Na anaferoume edw times parametrwn h sto 4 - see implementation details ?

The proposed method leverages GNN training using a visual classifier that is trained on seen state classes as supervision. In particular, the last layer of the GNN is designed to have the same size as the last layer of the classifier. This enables the GNN to generate semantic embedding features that correspond to all classes, including both seen and unseen classes that will be encountered during inference. Subsequently, the semantic embedding features replace the last layer of the classifier while this layer is kept fixed. The body of the classifier is then fine-tuned with the training images to optimize the overall model for state recognition.

% \vspace*{0.2cm}\noindent\textbf{GNN Details.} 
Overall, we experimented with four different model architectures and opted for the Transformer Graph Convolutional network (Tr-GCN)~\cite{nayak:tmlr22}. Further details are provided in Section~\ref{sec:abl} and the supplementary material of this work. 
The Tr-GCN mode is capable of combining input sets non-linearly by utilizing multilayer perceptrons and self-attention. Tr-GCN refers to an inductive model that can learn node representations by aggregating local neighborhood features allowing the trained model to make predictions on new graph structures without retraining. 
We leverage the aforementioned property of the Tr-GCN to train a permutation invariant non-linear aggregator that captures the intricate structure of a common sense knowledge graph. 
% , rendering it well-suited for zero-shot learning. 
% It is worth noting that a similar network architecture has been effectively employed for zero-shot object classification~\cite{nayak:tmlr22}.

% A critical aspect of the proposed method involves calibrating the weights of the GNN in a manner that its predictions in the semantic space are useful for the classifier deployed in the visual space. To accomplish this, we adopt an approach based on prior research \cite{Kampffmeyer2019, Wang2018b, nayak:tmlr22} that involves learning the semantic class representations by minimizing the L2 distance between the learned class representations and the weights of a fully connected layer in a ResNet classifier pre-trained on the ILSVRC 2012 dataset \cite{russakovsky2015imagenet}. Once the class representations are learned, we fix them and fine-tune the ResNet backbone using the training images from the dataset.




% \vspace*{0.2cm}\noindent\textbf{Building of the KG.}
% The KG is created by the querying  of a common sense repository. The repositories that we are ConceptNet \cite{} and WordNet\cite{}. The procedure takes place as follows. Initially we create a set of nodes that correspond to the target stace classes. Subsequently, the repository is queried for each of these nodes and its neighbours in the repository of  added to the KG if  certain criteria are met (see ablation section for more details). This procedure is repeated for the newly added nodes and henceforth until a number of hops has been reached.  

\subsection{The proposed OaSC approach}
\label{sec:pipeline}
Overall, the proposed method consists of four stages, as shown in \autoref{fig:pipeline}: (1) construction of the KG, (2) GNN training and learning of semantic graph embeddings, (3) fine-tuning of the visual classifier and (4) deployment of the fine-tuned state classifier.

\vspace*{0.0cm}\noindent\textbf{Construction of the KG (Stage 1)}:
To create the KG, we query a common sense repository to compile a generic solution and to avoid the construction of a task-specific KG, tailored to the entities at hand and their relationships. First, a set of nodes that correspond to the words of the target state classes $S^U$ and $S^S$ is generated. Then, we query the repository for each of these nodes and add their neighbors in the KG, if they meet specific criteria (see also Section~\ref{sec:abl}). This process is repeated for the newly added nodes until a specified number of node hops is reached.

This technique for building a generic KG offers several advantages in comparison to other problem-specific approaches. First, it allows the same KG to be used for different variations of the task. It also enables transfer learning since KGs can be reused to tackle other related problems. Moreover, the construction of such a KG does not rely on expert knowledge. Besides, the structured representation of relationships between entities and concepts that KGs provide can be leveraged to generate robust embeddings for zero-shot learning.
% which is expensive and time-consuming.  
The trade-off is that such KGs are prone to noisy information in the used repositories. 

% In comparison, language models, such as BERT~\cite{devlin2018bert}, often rely on large amounts of unstructured text data to generate embeddings. While language models are highly effective at capturing semantic relationships between words and phrases, they can also be prone to create associations between concepts that are not actually related. This can lead to noisy or unreliable embeddings, which can in turn degrade the performance of zero-shot learning models. By contrast, the structured nature of KGs allows for more accurate and precise capture of relationships between entities and concepts, leading to more robust embeddings that can improve the accuracy and reliability of zero-shot learning models~\cite{brown2020language}.


\vspace*{0.0cm}\noindent\textbf{Computation of  Graph Embeddings (Stage 2)}:
% Given the KG constructed in Stage 1, a word features embedding matrix corresponding to the KG nodes is created by utilizing the pre-computed word features of GloVe~\cite{pennington2014glove}. 
% % Subsequently,  random walks are performed in the KG and a sample of neighbors for each node is obtained.
% By taking the word features embedding matrix, the KG topology, and a target node as inputs, the GNN estimates the node's embeddings: the features of the node and its neighbors are  aggregated and submitted to a series of convolutions and pooling operations before the  output is produced in the form of a feature vector, the length of which is tailored to be the same as the size dimension of the last layer of a ResNet-101 classifier. 
% This procedure is repeated for all KG nodes and results in the computation of the semantic embeddings for all target state classes with each embedding being a feature vector of length equal to 2048. By combining these embeddings for the \mathcal{d} target classes  a  $ d \times 2048$ features matrix is created which serves as the last layer of a CNN classifier that is utilized during Stages 3 and 4.
% which serves as the last layer of a CNN classifier that is utilized during Stages 3 and 4 .
% We employ an established approach~\cite{Kampffmeyer2019, Wang2018b} that involves training of a transformer-based Graph Convolutional Model using graph embeddings of a set of semantic entities acquired by a common sense repository by minimizing the L2 distance between the learned class representations and the weights of a fully connected layer in a ResNet classifier, pre-trained on the ILSVRC 2012 dataset~\cite{russakovsky2015imagenet}, ensuring that the semantic class representations are meaningfully embedded.
We employ an established approach~\cite{Kampffmeyer2019, Wang2018b} that involves the training of a transformer-based Graph Convolutional Network (GCN)
 \textcolor{black}{ that utilizes a KG as input  %Training is performed %using features of a set of semantic entities acquired by a common sense repository, \textcolor{red}{(e.g. the ConceptNet, CSKG, or other)}  
 and generates an embedding vector for each node of the  KG. %. For the production of the embeddings vectors the GCM employs a sequence of transformations to the semantic features that correspond to the concepts linked to each node.
This process defines pre-computed GloVe word, i.e. semantic features~\cite{pennington2014glove}, for the KG nodes with each node representing a concept class.
% To compute node embeddings, the GNN is applied to encode the KG topology and the word feature embedding matrix. 
The GNN  aggregates each node's and its neighbors' features through a sequence of convolutions and pooling operations. %This results in the generation of a feature vector having a length equal to the dimension of the last layer in a visual CNN-based classifier that is instantiated using a ResNet-101 model. 
%By pre-training the visual classifier in a set of target classes 
The visual classifier is pre-trained on a set of target classes and using the weights of its fully connected layer, the GCN learns to produce visual feature representations, i.e. visual embeddings,  corresponding to the concept classes of the KG`s nodes.}
\textcolor{black}{
Formally,  the training involves the minimization of the L2 distance   $\mathcal{L_G}$ between the generated visual embeddings and the ground truth visual embeddings stemming from the visual classifier.} 
\textcolor{black}{In notation, 
\begin{equation}
\mathcal{L_G} = \frac{1}{2N} \sum_{n \in N} \sum_{p \in P} (W_{n,p} - \tildea{W}_{n,p})^2,
 \end{equation}
where $\tildea{W} \in \mathbb{R}^{|N|xP}$ denotes the weights of the GCN for the set of known concept classes $N$ and the dimensionality $P$ of the weight vector. Similar to~\cite{Kampffmeyer2019}, the ground truth weights, denoted as $W \in \mathbb{R}^{|N|xP}$, are obtained by extracting the last layer weights of a pre-trained CNN.}
% This process is repeated for all KG nodes corresponding to $S^U$ and $S^S$, generating semantic graph embeddings for all target state classes. 
%Each embedding comes in the form of a feature vector of length 2048. 


%By combining these embeddings for the $d$ target classes, a  $d \times 2048$ features matrix is defined that is integrated as the final layer of the visual CNN-based classifier that is employed in Stages~3-4.
%A critical aspect of this process is adjusting the GNN weights to align its predictions with the semantic space. This ensures that the semantic embeddings effectively aid the classifier used in Stages 3 and 4, operating in the visual space. 



\textcolor{black}{ 
The KG  given as an input to the GCN model is a hierarchical graph created for the requirements of the   ILSVRC 2012 dataset~\cite{russakovsky2015imagenet} and represents the WordNet hierarchical structure of the $1,000$ classes comprising the dataset. These 1,000 concept labels constitute the set of classes upon which the visual classifier used for the extraction of the ground truth visual embeddings is pre-trained.
}
% A critical aspect of this process is adjusting the GNN weights to align its predictions with the semantic space. This ensures that the semantic embeddings effectively aid the classifier used in Stages 3 and 4, operating in the visual space. 
% The concepts  that are used for the training refer to a set of 1K object classes of the ILSVRC 2012 dataset~\cite{russakovsky2015imagenet}, while the pre-trained ResNet101-based classifier is used for supervision to ensure that the GNN outputs, thus the semantic object class representations, are meaningfully embedded into the visual feature space. 
After the training is completed, the GCN model is employed to process the KG (constructed in Stage 1) and generate visual embeddings for the KG nodes that correspond to the object state classes,  by taking as input the  KG that was constructed during Stage 1. Each embedding comes in the form of a feature vector of length 2048, i.e. dimension of the last layer of the  pre-trained visual CNN-based classifier.
By combining these embeddings for the $d$ target classes, a  $d \times 2048$ features matrix is defined that is integrated as the final layer of the visual CNN-based classifier that is employed in Stages~3-4.
% First, using their pre-computed GloVe word features~\cite{pennington2014glove}, a matrix of word, i.e. semantic, features embeddings is defined for each of the KG nodes.
% % To compute node embeddings, the GNN is applied to encode the KG topology and the word feature embedding matrix. 
% Subsequently,  the GNN takes as input every target node that corresponds to any class in $S^U$ and $S^S$ and aggregates the features about the node and its neighbors through a sequence of convolutions and pooling operations. This results in the generation of a feature vector having a length equal to the dimension of the last layer in the visual CNN-based classifier that is instantiated using a ResNet-101 model.
% % This process is repeated for all KG nodes corresponding to $S^U$ and $S^S$, generating semantic graph embeddings for all target state classes. 
% Each embedding comes in the form of a feature vector of length 2048. By combining these embeddings for the $d$ target classes, a  $d \times 2048$ features matrix is defined that is integrated as the final layer of the visual CNN-based classifier that is employed in Stages~3-4.
% A critical aspect of this process is adjusting the GNN weights to align its predictions with the semantic space. This ensures that the semantic embeddings effectively aid the classifier used in Stages 3 and 4, operating in the visual space. 

% A critical aspect of this procedure involves calibrating the weights of the GNN to embed its predictions in the semantic space, i.e. semantic embeddings, are useful for the classifier deployed in the visual space during Stages 3 and 4. To accomplish this, we adopt an approach based on prior research~\cite{Kampffmeyer2019, Wang2018b} that involves learning the semantic class representations by minimizing the L2 distance between the learned class representations and the weights of a fully connected layer in a ResNet classifier pre-trained on the ILSVRC 2012 dataset~\cite{russakovsky2015imagenet}.  

\vspace*{0.0cm}\noindent\textbf{Fine-tuning of the Visual Classifier (Stage 3)}:
The estimated semantic embeddings are integrated into a visual CNN classifier that relies on the ResNet backbone and is initially pre-trained for object classification. The embeddings serve as the final layer of the network, encapsulating the representations essential for predicting the train state classes $S^S$. To enable this adaptation, the visual classifier undergoes re-training, specifically tailored to the classification of the train classes. 
During this fine-tuning process, input images $I^T$ contain states sourced exclusively from the training set $S^S$, i.e. ``seen states''. The primary objective is to harness the classifier capabilities to classify these familiar states, accurately. Notably, fine-tuning involves keeping the weights of the last layer fixed, safeguarding the integrity of the acquired semantic representations from Stage 2. Consequently, adjustments are only applied to the weights of preceding layers to ensure they effectively match the ``frozen'' last-layer weights.
% Apart from this detail, the procedure takes place in the same manner as the training of a CNN classifier.
% in every training epoch a loss is computed the value of which guides the update of all layers weights except the last one. 
Following the notation introduced 
\textcolor{black}{in the beginning of Section~\ref{sec:method}, the loss function is defined as:}
\begin{equation}
% \mathcal{L} = -\sum_{i \in S^{S}} y_i \cdot \log(P(y=i|X))
\mathcal{L_V} = -\sum_{s \in S^S, i \in I^{T}} y_s \cdot \log(P(s|i)),
 \end{equation}
\textcolor{black}{for the predicted \textit{$y_s$} state label in the \textit{$S^S$} set of state labels. $P(s|i)$ denotes the probability of state label \textit{s} based on the softmax vector given an image \textit{i} from the $I^T$ training set.}

\noindent\textbf{Zero-shot OaSC (Stage 4)}:
Upon the completion of fine-tuning, the visual state classifier can be utilized for  prediction by choosing the most likely class
\begin{equation}
% \^y = \arg\max_{i \in S} \left( P(y=i|X) \right)
\hat{y} = \arg\max_{s \in S^U i \in I^{U}} \left( P(s|i) \right),
\end{equation}
\textcolor{black}{where $I^U$ denotes the test image set and $S^U$ the test state classes respectively.} 
We highlight that the classifier is well-suited for predicting either only unseen classes, i.e. zero-shot classification, or both seen and unseen classes, i.e. generalized zero-shot classification.
\vspace{-.15cm}
% \subsection{Pipeline}

% Overall, the pipeline of our method consists of four stages (\autoref{fig:pipeline}}). During \textbf{Stage 1}, the KG is constructed.

% \vspace*{0.2cm}\noindent\textbf{Construction of the KG (Stage 1)}:
% The KG creation process involves querying a common sense repository to enable generalization instead of creating a custom KG tailored to specific entities and relationships. Initially, nodes corresponding to the target state classes are generated. The repository is then queried for each node, and neighbors meeting specific criteria are added to the knowledge graph. This process continues for the newly added nodes until a specified number of hops is reached. More details can be found in the ablation section.


% \vspace*{0.2cm}\noindent\textbf{Computation of semantic embeddings (Stage 2)}:


% \vspace*{0.2cm}\noindent\textbf{Finetuning of the Classfier (Stage 3)}:

% \vspace*{0.2cm}\noindent\textbf{Deployment  (Stage 4)}:

\section{Results of RL active flow control}\label{sec:Results}

In this section, we discuss the converge of the RL algorithms for the three FM and PM cases (\S\ref{subsec:Convergence}) and evaluate their drag reduction performance (\S\ref{Result_drag_reduction}). A parametric analysis of the effect of NARX memory length is presented (\S\ref{subsec:Nfs}) and the isolated effect of including past actions as observations during the RL training and control (\S\ref{subsec:past_actions}). Studies of reward function (\S\ref{subsec:Rewards_Study}), sensor placement (\S\ref{subsec:Sensor_study}) and generalisability to Reynolds number changes (\S\ref{subsec:Res}) are presented, followed by a comparison of SAC and TQC algorithms (\S\ref{subsec:SACvsTQC}). 

\subsection{Convergence of learning}\label{subsec:Convergence}

We perform RL with the maximum entropy TQC algorithm to discover control policies for the three cases shown in figure \ref{fig:Case_Demo}, which maximise the net-power-saving reward function given by \req{eq: PowerR}. During the learning stage, each episode (1 DNS simulation) corresponds to $200$ non-dimensional time units.  To accelerate learning, $65$ environments run in parallel.


Figure \ref{fig:Learning_Curve} shows the learning curves of the three cases.  Table \ref{tab:LearningConvergence} shows the number of episodes needed for convergence and relevant parameters for each case.
It can be observed from the curve of episode reward that the RL agent is updated after every 65 episodes, i.e. $1$ iteration, where the episode reward is defined as 
\begin{equation}
R_{ep} = \sum_{k=1}^{N_k} r_{k},
\label{eq:Epi_R}
\end{equation}
where $k$ denotes the $k^{th}$ RL step in one episode and $N_k$ is the total number of samples in one episode.
The root mean square (RMS) value of the drag coefficient, $C_D^{RMS}$, at the asymptotic regime of control, is also shown to demonstrate convergence, defined as 
$C_D^{RMS} = \sqrt { (\mathcal{D}(\langle C_D\rangle_{env}))^2 }$,
where the operator $\mathcal{D}$ detrends the signal with a $9^{th}$-order polynomial and removes the transient part, and $\langle ~ \rangle_{env}$ denotes the average value of parallel environments in a single iteration. 

% Figure environment removed

\begin{table}
  \begin{center}
\def~{\hphantom{0}}
  \begin{tabular}{lcccccc}
    
      Environment  & Algorithm  &  $N_{c}$ & $R_{ep,c}$ & (Layers, Neurons) & $N_{fs}$ & Number of Inputs \\ 
       FM-Static   & TQC & $325$ & $37.72$ & (3,512) & $0$ & $64p_t+2a_{t-1}$\\
       PM-Static   & TQC & $1235$ & $21.87$ & (3,512) & $0$ & $64p_t+2a_{t-1}$\\
       PM-Dynamic  & TQC & $715$ & $34.35$ & (3,512) & $27$ & $N_{fs} (64p_t+2a_{t-1})$\\
  \end{tabular}
  \caption{Number of episodes $N_{c}$ required for RL convergence in different environments. The episode reward $R_{ep,c}$ at the convergence point, the configuration of NN and the dimension of inputs are presented for each case. $N_{fs}$ is the finite-horizon length of past actions-measurements.}
  \label{tab:LearningConvergence}
  \end{center}
\end{table}

In figure \ref{fig:Learning_Curve}, it can be noticed that in the FM environment, RL converges after approximately $325$ episodes ($5$ iterations) to a   {nearly} optimal policy using a static   {feedback} controller. As will be shown in \S\ref{Result_drag_reduction}, this policy is globally optimal since the vortex shedding is fully attenuated and the jets converge to zero mass flow actuation, thus recovering the unstable base flow and the minimum drag state.  However, with the same static   {feedback} controller in a PM environment (POMDP), the RL agent fails to discover the   {nearly} optimal solution, requiring around $1235$ episodes for convergence but only obtaining a relatively low episode reward.
Introducing a dynamic   {feedback} controller in the PM environment, the RL agent convergences to a near-optimal solution in 735 episodes. The dynamic   {feedback} controller trained by RL achieves a higher episode reward (34.35) than the static   {feedback} controller in the PM case (21.87), which is close to the FM case (37.72). The learning curves illustrate that using a finite horizon of past actions-measurements ($N_{fs} = 27$) to train a dynamic   {feedback} controller in the PM case improves learning in terms of speed of convergence and accumulated reward achieving nearly optimal performance with only wall pressure measurements. 


\subsection{Drag reduction with dynamic RL controllers} \label{Result_drag_reduction}

% Figure environment removed

The trained controllers for the cases shown in figure \ref{fig:Case_Demo} are evaluated to obtain the results shown in figure \ref{fig:TQC_FMPM}.   {Evaluation tests are performed for 120 non-dimensional time units to show both transient and asymptotic dynamics of the closed-loop system.}
Control is applied at $t=0$ with the same initial condition for each case, i.e. steady vortex shedding with average drag coefficient $\langle C_{D0}\rangle \approx 1.45$ (baseline without control). Consistent with the learning curves, the difference in control performance in the three cases can be observed both from the drag coefficient $C_D$ and the actuation $Q_1$.
  {The drag reduction is quantified by a ratio $\eta$ using the asymptotic time-averaged drag coefficient with control $C_{Da} = \langle C_{D}\rangle_{t \in [80,120]}$, the drag coefficient $C_{Db}$ of the base flow (details presented in Appendix \ref{App:BaseFlow}), and the baseline time-averaged drag coefficient without control $\langle C_{D0}\rangle$, as
\begin{equation}
\eta = \frac{\langle C_{D0}\rangle - C_{Da}}{\langle C_{D0}\rangle - C_{Db}} \times 100\%.
\label{eq:drag_reduction}
\end{equation}}

\begin{itemize}

\item {\bf FM-Static:} With a static   {feedback} controller trained in a full-measurement environment, a drag reduction of $\eta = 101.96\%$ is obtained with respect to the base flow (steady unstable fixed point; maximum drag reduction). This indicates that an RL controller informed with full-state information can entirely stabilise the vortex shedding and cancel the unsteady part of the pressure drag.

\item {\bf PM-Static:} A static/memoryless controller in a partial-measurement environment leads to performance degradation and a drag reduction of   {$\eta = 56.00\%$} in the asymptotic control stage, i.e. after $t=80$, compared to the performance of ``FM-Static''. This performance loss can also be observed from the control actuation curve, as $Q_1$ oscillates with a relatively large fluctuation in ``PM-Static'' while it stays about zero in the ``FM-Static'' case. 
The discrepancy between FM and PM environments using a static   {feedback} controller reveals the challenge of designing a controller with a POMDP environment. The RL agent cannot fully identify the dominant dynamics with only partial measurements on the   {downstream} surface of the bluff body, resulting in sub-optimal control behaviour.

\item{\bf PM-Dynamic:} With a dynamic   {feedback} controller (NARX model presented in \S\ref{subsec:PM_Dynamic}) in a partial-measurement environment, the vortex shedding is stabilised and the dynamic   {feedback} controller achieves   {$\eta = 97.00\%$} of the maximum drag reduction after time $t=60$. Although there are minor fluctuations in the actuation $Q_1$, the energy spent in the synthetic jets is significantly lower compared to the ``PM-Static'' case. Thus, a dynamic   {feedback} controller in PM environments can achieve nearly optimal drag reduction, even if the RL agent only collects information from pressure sensors on the   {downstream} surface of the body. The improvement in control indicates that the POMDP due to the PM condition of the sensors can be reduced to an approximate MDP by training a dynamic   {feedback} controller with a finite horizon of past actions-measurements. Furthermore, high-frequency action oscillations, which can be amplified with static   {feedback} controllers, are attenuated in the case of dynamic   {feedback} control. These encouraging and unexpected results support the effectiveness and robustness of model-free RL control in practical flow control applications, in which sensors can only be placed on a solid surface/wall.

\end{itemize}


% Figure environment removed

In figure \ref{fig:Contour}, snapshots of the velocity magnitude   {$|\boldsymbol{u}| = \sqrt{u^2+v^2}$} are presented for ``Baseline'' without control, ``PM-Static'', ``PM-Dynamic'' and ``FM-Static'' control cases. Snapshots are captured at $t=100$ in the asymptotic regime of control. A vortex-shedding structure of different strengths can be observed in the wake of all three controlled cases. In ``PM-Static'', the recirculation area is lengthened compared to the baseline flow, corresponding to base pressure recovery and pressure drag reduction. A longer recirculation area can be noticed in ``PM-Dynamic'' due to the enhanced attenuation of vortex shedding and pressure drag reduction. The dynamic   {feedback} controller in the PM case renders a $326.22\%$ increase of recirculation area with respect to the baseline flow, while only a $116.78\%$ increase is achieved by a static   {feedback} controller. The ``FM-Static'' case has the longest recirculation area, and the vortex shedding is almost fully stabilised, which is consistent with the drag reduction shown in figure \ref{fig:TQC_FMPM}.

% Figure environment removed

Figure \ref{fig:Obs} presents first- and second-order base pressure statistics for the baseline case without control and PM cases with control. In figure \ref{fig:Obs}(a), the time-averaged value of base pressure, $\overline{p}$, demonstrates the base pressure recovery after control is applied. Due to flow separation and recirculation, the time-averaged base pressure is higher at the middle of the   {downstream surface}, which is retained with control. The base pressure increase is directly linked to pressure drag reduction, which quantifies the control performance of both static and dynamic   {feedback} controllers. Up to $49.56\%$ of pressure increase at the centre of the   {downstream surface}  is obtained in the ``PM-Dynamic'' case, while only $21.15\%$ can be achieved by a static   {feedback} controller. In figure \ref{fig:Obs}(b), the base pressure RMS is shown. For the baseline flow, strong vortex-induced fluctuations of the base pressure can be noticed around the top and bottom   {on the downstream surface} of the bluff body. In the ``PM-Static'' case, the RL controller   {partially suppresses} the vortex shedding, leading to a sub-optimal reduction of the pressure fluctuation. The sensors close to the top and bottom corners are also affected by the synthetic jets, which change the RMS trend for the two top and bottom measurements. In the ``PM-Dynamic'' case,  the pressure fluctuations are nearly zero for all the measurements on the   {downstream surface}, highlighting the success of vortex shedding suppression by a dynamic RL controller in a PM environment.

% Figure environment removed

The differences between static and dynamic controllers in PM environments are further elucidated in figure \ref{fig:Action_analysis} by examining  the time series of pressure differences $\Delta p_t$ from surface sensors (control input) and control actions $a_{t-1}$ (output). The pressure differences are calculated from sensor pairs at $y=\pm y_{sensor}$, where $y_{sensor}$ is defined in Eq. \req{eq:Probe_base}. For $N=64$, there are 32 time series of $\Delta p_t$ for each case. 
%
During  the initial stages of control ($t \in [0,11]$), the control actions are similar  for the two PM cases and they deviate for $t>11$, resulting in discernible control performance at the asymptotic regime. 
At the initial stages, the controllers operate in nearly anti-phase to $\Delta p_t$, in order to eliminate the antisymmetric pressure component due to vortex shedding. The inability of the static controller to have a frequency dependent amplitude (and phase), manifests as well through the amplification of high frequency noise. For $t>11$, the static feedback controller continues to operate in nearly anti-phase to the pressure difference, resulting in partial stabilisation of unsteadiness. However, the dynamic feedback controller adjusts its phase and amplitude significantly, which attenuates the antisymmetric fluctuation of base pressure and drives $\Delta p_t$ to near zero. 

% Figure environment removed

Figure \ref{fig:ContourComparision} shows instantaneous vorticity contours for PM-Dynamic and PM-Static cases, showing both the similarities and discrepancies between the two cases. At $t=2$, flow is expelled from the bottom jet for both cases, generating a clockwise vortex, termed V1. This V1 vortex, shown in black, works against the primary counter-clockwise vortex labelled as P1, depicted in red, emerging from the bottom surface. At $t=5.5$, a secondary vortex, V2, forms from the jets to oppose the primary vortex shedding from the top surface (labelled as P2). 
%
 At $t=13$, the suppression of the two primary vortices near the bluff body is evident in both cases, indicated by their less tilted shapes compared to the previous time instances. At $t=13$, the PM-Dynamic adjusted the phase of the control signal, which corresponds to a marginal action at this time instance at figure \ref{fig:Action_analysis}. Consequently, no additional counteracting vortex is formed in PM-Dynamic. However, in the PM-Static scenario, the jets generate a third vortex, labelled V3, which emerges from the top surface. This corresponds to a peak in the action of the PM-Static controller at this time. The inability of the PM-Static controller to adapt the amplitude/phase of the input/output behaviour results in suboptimal performance.

\subsection{Horizon of the finite-history sufficient statistic}\label{subsec:Nfs}

A parametric study on the horizon of the finite history in NARX (equation \req{eq:NARX}), i.e. the number of frames stacked $N_{fs}$, is presented in this section. Since the NARX model uses a finite horizon of past actions-measurements in  \req{eq:Sufficient_statistic}, the horizon of the finite history affects the convergence of the approximation \citep{yu_near_2008}. This approximation affects the optimisation during the learning of RL because it determines whether the RL agent can observe sufficient information to converge to an optimal policy. 

Since vortex shedding is the dominant instability to be controlled, the choice of $N_{fs}$ should intuitively link to the timescale of the vortex shedding period. The ``frames'' of observations are obtained every RL step ($0.5$ time units), while the vortex shedding period is $t_{vs}\approx6.85$ time units. Thus, $N_{fs}$ is rounded to integer values for different numbers of vortex shedding periods, as shown in table \ref{tab:Frame_Stack}.


% Figure environment removed

\begin{table}
  \setlength{\tabcolsep}{12pt}
  \begin{center}
\def~{\hphantom{0}}
  \begin{tabular}{ccc}
      Number of  & Non-dimensional &  History length \\
      VS periods &    time units          &  ($N_{fs}$)         \\ [3pt]
      \hline
       0.5   & 3.43 & 7 \\
       1   & 6.85 & 14 \\
       2  & 13.70 & 27 \\
       3 & 20.55 & 41\\
       4 & 27.40 & 55\\
       5 & 34.25 & 68\\
  \end{tabular}
  \caption{Correspondence between the number of vortex shedding (VS) periods and frame stack (history) length in samples $N_{fs}$. The RL control step size is $t_a =0.5$, and $N_{fs}$ is rounded to an integer.}
  \label{tab:Frame_Stack}
  \end{center}
\end{table}

The results of time-averaged drag coefficients $\langle C_{D}\rangle$ after control and the average episode rewards $\langle R_{ep}\rangle$ in the final stage of training are presented in figure \ref{fig:Frame_Stack}. As $N_{fs}$ increases from 0 to 27, the performance of RL control improves, resulting in a lower $\langle C_{D}\rangle$ and a higher $\langle R_{ep}\rangle$. $N_{fs}=2$ is specially examined because the latent dimension of the vortex shedding limit cycle is 2. However, the control performance with $N_{fs}=2$ is marginally improved to the one with $N_{fs}=0$, i.e. a static   {feedback} controller. This result indicates that the horizon consistent with the vortex shedding dimension is not long enough for the finite horizon of past action measurements. The optimal history length to achieve stabilisation of the vortex shedding   {in PM environments} is 27 samples, which are equivalent to 13.5 convective time units or $\sim 2$ vortex shedding periods. 

With $N_{fs}=41$ and $N_{fs}=55$, the drag reduction and episode rewards drop slightly compared to $N_{fs}=27$. The decline in performance is non-negligible as $N_{fs}$ increases further to 68. This decline shows that excessive inputs to the neural networks (see table \ref{tab:LearningConvergence}), may impede training because more parameters need to be tuned or larger neural networks need to be trained. 

\subsection{Observation sequence with past actions}\label{subsec:past_actions}

Past actions (exogenous terms in NARX) facilitate reducing a POMDP to an MDP problem, as discussed in \S\ref{subsec:PM_Dynamic}. In the near-optimal control of a PM environment using a dynamic   {feedback} controller with inputs $\left( o_t, o_{t-1}, ..., o_{t-N_{fs}} \right)$, a sequence of observations $o_t = \left \{ p_t, a_{t-1}\right \}$ at step $t$ is constructed to include pressure measurements and actions. In the FM environment, due to the introduction of one-step delayed action due to the first-order-hold interpolation given by \req{eq:FOH_action}, the inclusion of the past action along with the current pressure measurement, meaning $o_t = \left \{ p_t, a_{t-1} \right \}$, is required even when the sensors are placed in the wake and cover the wavemaker region. 

Figure \ref{fig:ActionInObs} presents the control performance for the same environment with and without past actions included.
In the FM case, there is no apparent difference between RL control with $o_t = \left \{ p_t, a_{t-1} \right \}$ or $o_t = \left \{ p_t \right \}$, which indicates that the inclusion of the past action is negligible to the performance. This is the case when the RL sampling frequency is sufficiently faster than the timescale of the vortex shedding dynamics. 
In PM cases, if exogenous action terms are not included in the observations but only the finite history of pressure measurements is used, the RL control fails to converge to a near-optimal policy, with only   {$\eta = 67.45\%$}  drag reduction. With past actions included, the drag reduction of the same environment increases up to   {$\eta = 97.00\%$}. 

The above results show that in PM environments, sufficient statistics cannot be constructed only from the finite history of measurements. Missing state information needs to be reconstructed by both state-related measurements and control actions. 

% Figure environment removed

\subsection{Reward study}
\label{subsec:Rewards_Study}

In \S\ref{Result_drag_reduction}, a power-based reward function given by \req{eq: PowerR} has been implemented, and stabilising controllers can be learned by RL, as shown. In this section, RL control results with other forms of reward functions (introduced in \S\ref{subsec:Reward}) are provided and discussed.

% Figure environment removed

The control performance of RL control with the different reward functions is evaluated based on the drag coefficient $C_D$ shown in figure \ref{fig:Reward_Study}. Static   {feedback} controllers are trained in FM environments, and dynamic   {feedback} controllers are trained in PM environments. In FM cases, control performance is not sensitive to the choice of reward function (power or force-based).  
In PM cases, the discrepancies between RL-step time-averaged and instantaneous rewards can be observed in the asymptotic regime of control. The controllers with both rewards (power or force-based) achieve nearly optimal control performance, but there is some unsteadiness in the cases using instantaneous rewards due to slow statistical convergence of the rewards and limited correlation to the partial observations.

All four types of reward functions studied in this work achieve nearly optimal drag reduction around $100\%$. However, the energy-based reward (``PowerR'') offers an intuitive reward design, attributable to its physical properties and the dimensionally consistent addition of the constituent terms of the reward function. Further enhancing its practicality, since the power of the actuator can be directly measured, it avoids the necessity for hyperparameter tuning, as in the force-based reward. Additionally, the results show similar performance with both time-averaged between RL steps and instantaneous rewards, avoiding the necessity for faster sampling for the calculation of the rewards. This choice of reward function can be extended to various RL flow control problems and can be beneficial to experimental studies.


\subsection{Sensor configuration study with partial measurements}\label{subsec:Sensor_study}

% Figure environment removed

In the PM environment, the configuration of sensors (number and location on the downstream surface) may also affect the information contained in the observations and thus control performance. 
Control results of drag coefficient $C_D$ for different sensor configurations in PM-dynamic cases are presented in figure \ref{fig:Sensor_config}. In the configuration with $N = 2$, two sensors are placed at $y=\pm 0.25$, and for $N = 1$, only one sensor is placed at $y = 0.25$. Other configurations are consistent with equation \req{eq:Probe_base}. 

The $C_D$ curves in figure \ref{fig:Sensor_config} show that, as the number of sensors is reduced from 64 to 2, RL control achieves the same level of performance with minor discrepancies due to randomness in different learning cases. However, if RL control uses observations from only one sensor at $y = 0.25$, performance degradation can be observed in the asymptotic stage with 19.79\% on average less drag reduction. The sub-figure presents the relationship between the number of sensors and asymptotic drag coefficient $\langle C_D \rangle$. These results indicate a limit on sensor configuration for the use of the NARX-modeled controller to stabilise the vortex shedding. 

% Figure environment removed

To understand the cause of performance degradation in the $N=1$ case, the pressure measurements from two sensors in both baseline and PM-Dynamic cases are presented in figure \ref{fig:Pressure2Sensors}. In the baseline case, two sensors are placed at the same location as the $N=2$ case ($y=\pm 0.25$) only for observations. It can be observed that the pressure measurements from two sensors are anti-symmetric since they are placed symmetrically on the downstream surface.
In the PM-Dynamic case, the NARX controller is used, and control is applied at $t=0$. In this closed-loop system, the anti-symmetric relationship between two sensors (from the symmetric position) is broken by the control actuation, and no correlation is evident. This can be seen during the transient dynamics, e.g. in $t \in [0,10]$. Therefore, when the number of sensors is reduced to $N=1$ by removing one sensor from the $N=2$ case, the dynamic feedback from the removed sensor cannot be fully reflected by the remaining sensor in the closed-loop system. This loss of information affects the fidelity of the control response to the dynamics of the sensor-removing side, causing suboptimal drag reduction in the $N=1$ scenario.

It should be noted that the configuration of 64 sensors is not necessary for control, as $N = 2$ or $N = 16$ also achieves nearly optimal performance. The number of sensors $N = 64$ in PM-Static environments is used for comparison with the FM-Static configuration (Eq. \ref{eq:Probe_wake}), which eliminates the effect from different input dimensions between two static cases. Also, 64 sensors sufficiently cover the downstream surface of the bluff body to avoid missing spatial information. 
The optimal configuration of sensors can be tuned with optimisation techniques such as \cite{paris_robust_2021}, but the results in figure \ref{fig:Sensor_config} indicate that RL adapts with nearly optimal performance to non-optimised sensor placement in the present environment.

\subsection{Performance of RL controllers to unseen $Re$} \label{subsec:Res}

% Figure environment removed

The RL controller is tested at different Reynolds numbers, in order to examine its generalisability to environment changes. The controllers have been trained at $Re=100$ with both FM and PM conditions, and tested at $Re= 80, 90, 100, 110, 120, 150$. The controllers were further trained at $Re=150$, denoted as continual learning (CL), and tested again at $Re=150$. 

As shown in figure \ref{fig:Res}, in both ``PM-Dynamic'' and ``FM-Static'' cases, the RL controllers are able to reduce drag by $\eta=64.68\%$ in the worst case, when $Re$ is close to the training point at $Re=100$, i.e. the test cases with $Re= 80, 90, 100, 110, 120$. 
However, when applying the controllers trained at $Re=100$ to an environment at $Re=150$, the drag reduction drops to $\eta=41.98\%$ and $\eta = 74.04\%$ in PM-Dynamic and FM-Static cases, respectively.

Performing CL at $Re=150$, the drag reduction is improved to $\eta = 78.07\%$ in PM-Dynamic after 1105 training episodes while $\eta = 88.13\%$ in FM-Static after 390 episodes, with the same RL parameters as the training at $Re=100$.
Overall, the results of these tests indicate that the RL-trained controllers can achieve significant drag reduction in the vicinity of the training point (i.e. $\pm\%20$ $Re$ change). If the test point is far from the training point, a CL procedure can be implemented to achieve nearly optimal control.

\subsection{TQC vs SAC}\label{subsec:SACvsTQC}

% Figure environment removed

Control results with TQC and SAC are presented in figure \ref{fig:TQCvsSAC} in terms of $C_D$. TQC shows a more robust control performance. In the case of FM, SAC might demonstrate a slightly more stable transient behaviour attributed to the fact that the quantile regression process in TQC introduced complexity to the optimisation process. Both controllers achieved an identical level of drag reduction in the FM case. 

However, in the context of the PM cases, it is observed that TQC outperforms SAC in drag reduction with both static and dynamic   {feedback} controllers. For static   {feedback} control, TQC achieved an average drag reduction of   {$\eta = 56.00\%$}, compared to the   {$\eta = 46.31\%$}  reduction achieved by SAC. The performance under dynamic   {feedback} control conditions is more compelling, where TQC fully reduced the drag, achieving   {$\eta = 97.00\%$}  of drag reduction, reverting it to a near-base-flow scenario. In contrast, SAC managed to achieve an average drag reduction of   {$\eta = 96.52\%$}.

The fundamental mechanism for updating Q-functions in RL involves selecting the maximum expected Q-functions among possible future actions. This process, however, can potentially lead to overestimation of certain Q-functions \citep{hasselt_double_2010}. In POMDP, this overestimation bias might be exacerbated due to the inherent uncertainty arising from the partial-state information. Therefore, the Q-learning-based algorithm, when applied to POMDPs, might be more prone to choosing these overestimated values, thereby affecting the overall learning and decision-making process.

As mentioned in \S\ref{subsec:SACTQC}, the core benefit of TQC under these conditions can be attributed to its advanced handling of the overestimation bias of rewards. By constructing a more accurate representation of possible returns, TQC provides a more accurate Q-function approximation than SAC. This process of modulating the probability distribution of the Q-function assists TQC in managing the uncertainties inherent in environments with only partial-state information. In this case, TQC can adapt more robustly to changes and uncertainties, leading to better performance in both static and dynamic feedback control tasks.

\section{Conclusion}\label{sec:conclusion}

This paper presents our empirical domain knowledge distillation framework using ChatGPT and discusses our observations from the framework application experiments in the autonomous driving domain. The key finding is that: 1) with proper design of prompt engineering and execution flow, fully automated domain knowledge (in the ontology format) distillation is possible. However, due to the randomness in the response and the butterfly effect, the quality of fully automated distillation results is not guaranteed. To address this, we develop a web-based assistant to enable manual supervision and early intervention at runtime. We hope our findings and tools inspire future research toward revolutionizing the engineering processes of knowledge-based systems across domains.

\appendices
\section{The Proof of Proposition \ref{prop2}}
\label{appa}
For the jointly Gaussian random vectors $\bm{x}$ and $\bm{y}$, we have
\begin{equation}
\begin{aligned}
&    \left[\begin{matrix}\bm{x}\\\bm{y}\\\end{matrix}\right] \sim \mathcal{N}\left(\left[\begin{matrix}\bm{\mu}_x\\\bm{\mu}_y\\\end{matrix}\right],\left[\begin{matrix}A&C\\C^T&B\\\end{matrix}\right]\right) \\
& = \mathcal{N}\left(\left[\begin{matrix}\bm{\mu}_x\\\bm{\mu}_y\\\end{matrix}\right],\left[\begin{matrix}\widetilde{A}&\widetilde{C}\\{\widetilde{C}}^T&B\\\end{matrix}\right]^{-1}\right)
\end{aligned}
\end{equation}
then the marginal and conditional distribution of $\bm{x}$ are shown as follows according to \cite{williams2006gaussian}.
\begin{equation}
    \bm{x} \sim \mathcal{N}\left(\bm{\mu}_x,A\right)
\end{equation}
% and
\begin{equation}
\label{app2-1}
    \bm{x}|\bm{y} \sim \mathcal{N}\left(\bm{\mu}_x+CB^{-1}\left(\bm{y}-\bm{\mu}_y\right),A-CB^{-1}C^T\right)
\end{equation}
% or
\begin{equation}
\label{app2-2}
    \bm{x}|\bm{y} \sim \mathcal{N}\left(\bm{\mu}_x-{\widetilde{A}}^{-1}\widetilde{C}\left(\bm{y}-\bm{\mu}_y\right),{\widetilde{A}}^{-1}\right)
\end{equation}

Thus, \textbf{Proposition \ref{prop2}} is proved.










\section{The Proof of Proposition \ref{prop3}}
\label{appb}
The product of two Gaussian distributions is represented as
\begin{equation}
\mathcal{N}\left(\bm{x}\middle|\bm{a},A\right)\mathcal{N}\left(\bm{x}\middle|\bm{b},B\right)=Z^{-1}\mathcal{N}\left(\bm{x}\middle|\bm{c},C\right)
\end{equation}
where
\begin{equation}
\label{app4}
    \bm{c}=C\left(A^{-1}\bm{a}+B^{-1}\bm{b}\right)
\end{equation}
\begin{equation}
\label{app5}
    C=\left(A^{-1}+B^{-1}\right)^{-1}
\end{equation}
\begin{equation}
\label{app6}
    Z^{-1}=\left(2\pi\right)^{-\frac{D}{2}}\left|A+B\right|^{-\frac{1}{2}}\exp{\left(-\frac{\left(\bm{a}-\bm{b}\right)^T\left(\bm{a}-\bm{b}\right)}{2\left(A+B\right)}\right)}
\end{equation}

Thus, through multiplying the cavity distribution by $t_i$ from (\ref{11}), \textbf{Proposition \ref{prop3}} is proved.


\section{The Proof of Proposition \ref{prop4}}
\label{appc}
Consider
\begin{equation}
\label{app7}
Z=\int_{-\infty}^{\infty}{\Phi\left(\frac{x-m}{v}\right)\mathcal{N}(x|\mu,\sigma^2)dx}
\end{equation}
% where
% \begin{equation}
%     \Phi\left(x\right)=\int_{-\infty}^{x}{\mathcal{N}\left(y\right)dy}
% \end{equation}
When $v>0$, by combining$ z=y-x+\mu-m$ and $w=x-\mu$ we can get
\begin{equation}
\begin{aligned}
& Z_{v>0}=\frac{\int_{-\infty}^{\infty}\int_{-\infty}^{x}\exp{\left(-\frac{\left(y-m\right)^2}{2v^2}-\frac{\left(x-\mu\right)^2}{2\sigma^2}\right)}}{2\pi\sigma v}dydx \\
& =\frac{\int_{-\infty}^{\mu-m}\int_{-\infty}^{\infty}\exp{\left(-\frac{\left(z+w\right)^2}{2v^2}-\frac{w^2}{2\sigma^2}\right)}}{2\pi\sigma v}dwdz
\end{aligned}
\end{equation}
% and
\begin{equation}
\begin{aligned}
& Z_{v>0} \\
& =\frac{\int_{-\infty}^{\mu-m}\int_{-\infty}^{\infty}\exp{\left(-\frac{1}{2}\left[\begin{matrix}w\\z\\\end{matrix}\right]^T\left[\begin{matrix}\frac{1}{v^2}+\frac{1}{\sigma^2}&\frac{1}{v^2}\\\frac{1}{v^2}&\frac{1}{v^2}\\\end{matrix}\right]\left[\begin{matrix}w\\z\\\end{matrix}\right]\right)}}{2\pi\sigma v}dwdz \\
& =\int_{-\infty}^{\mu-m}\int_{-\infty}^{\infty}\mathcal{N}\left(\left[\begin{matrix}w\\z\\\end{matrix}\right]|\mathbf{0},\left[\begin{matrix}\sigma^2&-\sigma^2\\-\sigma^2&v^2+\sigma^2\\\end{matrix}\right]\right)dwdz
\end{aligned}
\end{equation}
According to (\ref{app2-1}) and (\ref{app2-2}), we can get
\begin{equation}
\label{app11}
    Z_{v>0}=\frac{\int_{-\infty}^{\mu-m}\exp{\left(-\frac{z^2}{2\left(v^2+\sigma^2\right)}\right)}dz}{\sqrt{2\pi(v^2+\sigma^2)}}=\Phi\left(\frac{\mu-m}{\sqrt{v^2+\sigma^2}}\right)
\end{equation}
When $v<0$, by combining $\Phi\left(-z\right)=1-\Phi\left(z\right)$ and (\ref{app7}),
% we can obtain
\begin{equation}
\label{app12}
Z_{v<0}=1-\Phi\left(\frac{\mu-m}{\sqrt{v^2+\sigma^2}}\right)=\Phi\left(-\frac{\mu-m}{\sqrt{v^2+\sigma^2}}\right)
\end{equation}

By collecting (\ref{app11}) and (\ref{app12}), we can get
\begin{equation}
\label{app13}
Z=\int\Phi\left(\frac{x-m}{v}\right)\mathcal{N}\left(x\middle|\mu,\sigma^2\right)dx=\Phi\left(z\right)
\end{equation}
where $z=\frac{\mu-m}{v\sqrt{1+\sigma^2/v^2}} (v\neq0)$. 
% We aim to get the moments of
% \begin{equation}
% q\left(x\right)=Z^{-1}\Phi\left(\frac{x-m}{v}\right)\mathcal{N}\left(x\middle|\mu,\sigma^2\right)
% \end{equation}
By differentiating with respect to $\mu$ on (\ref{app13}), we can obtain
\begin{equation}
\begin{aligned}
& \frac{\partial Z}{\partial\mu}=\int{\frac{x-\mu}{\sigma^2}\Phi\left(\frac{x-m}{v}\right)}\mathcal{N}\left(x\middle|\mu,\sigma^2\right)dx =\frac{\partial}{\partial\mu}\Phi\left(z\right) \\
& \Longleftrightarrow \frac{1}{\sigma^2}\int x\Phi\left(\frac{x-m}{v}\right)\mathcal{N}\left(x\middle|\mu,\sigma^2\right)dx-\frac{\mu Z}{\sigma^2} \\
& =\frac{\mathcal{N}(z)}{v\sqrt{1+\sigma^2/v^2}}
\end{aligned}
\end{equation}
where $\partial\Phi\left(z\right)/\partial\mu=\mathcal{N}(z)\partial z/\partial\mu$ is utilized. Multiplying through by $\sigma^2/Z$, (\ref{app16}) is obtained.
\begin{equation}
\label{app16}
\mathbb{E}_q\left[x\right]=\mu+\frac{\sigma^2\mathcal{N}\left(z\right)}{\Phi\left(z\right)v\sqrt{1+\frac{\sigma^2}{v^2}}}
\end{equation}
Similarly, we can obtain the second moment as
\begin{equation}
\label{app17}
\begin{aligned}
 & \frac{\partial^2Z}{\partial\mu^2} \\
 & =\int{[\frac{x^2}{\sigma^4}-\frac{2\mu x}{\sigma^4}+\frac{\mu^2}{\sigma^4}-\frac{1}{\sigma^2}] \Phi\left(\frac{x-m}{v}\right)\mathcal{N}\left(x\middle|\mu,\sigma^2\right)} dx  \\
 & =-\frac{z\mathcal{N}(z)}{v^2+\sigma^2} \Longleftrightarrow \\
 & \mathbb{E}_q\left[x^2\right]=2\mu\mathbb{E}_q\left[x\right]-\mu^2+\sigma^2-\frac{\sigma^4z\mathcal{N}\left(z\right)}{\Phi\left(z\right)\left(v^2+\sigma^2\right)}
\end{aligned}
\end{equation}
By combining (\ref{app16}) and (\ref{app17}), we can get
\begin{equation}
\begin{aligned}
& \mathbb{E}_q\left[{(x-\mathbb{E}_q\left[x\right])}^2\right]=\mathbb{E}_q\left[x^2\right]-\mathbb{E}_q[x]^2 \\
& =\sigma^2-\frac{\sigma^4\mathcal{N}\left(z\right)}{\left(v^2+\sigma^2\right)\Phi\left(z\right)}\left(z+\frac{\mathcal{N}\left(z\right)}{\Phi\left(z\right)}\right)
\end{aligned}
\end{equation}

Thus, \textbf{Proposition \ref{prop4}} is proved.

\section{The Proof of Proposition \ref{prop5}}
\label{appd}
We can obtain (\ref{19-1}), (\ref{19-2}), and (\ref{19-3}) according to (\ref{app4}), (\ref{app5}), and (\ref{app6}). Hence, \textbf{Proposition \ref{prop5}} is proved.



\section{The Proof of Proposition \ref{prop6}}
\label{appe}
The approximated mean for $f_\ast$ can be denoted as
\begin{equation}
\begin{aligned}
& \mathbb{E}_q\left[f_\ast|X,\bm{y},\bm{x}_\ast\right]=\bm{k}_\ast^TK^{-1}\bm{\mu} \\
& =\bm{k}_\ast^TK^{-1}\left(K^{-1}+{\widetilde{\Sigma}}^{-1}\right)^{-1}{\widetilde{\Sigma}}^{-1}\widetilde{\bm{\mu}} \\
& =\bm{k}_\ast^T\left(K+\widetilde{\Sigma}\right)^{-1}\widetilde{\bm{\mu}}
\end{aligned}
\end{equation}

The variance of $f_\ast|(X,\bm{y})$ under the Gaussian approximation can be denoted as
\begin{equation}
\begin{aligned}
& \mathbb{V}_q\left[f_\ast\middle| X,\bm{y},\bm{x}_\ast\right] = \mathbb{E}_{p(f_\ast|X,\bm{x}_\ast,\bm{f})} {f_\ast-\mathbb{E}[f_\ast|X,\bm{x}_\ast,\bm{f}]}^2 \\
& =k\left(\bm{x}_\ast,\bm{x}_\ast\right)-\bm{k}_\ast^TK^{-1}\bm{k}_\ast+\bm{k}_\ast^TK^{-1}\left(K^{-1}+\widetilde{\Sigma}\right)^{-1}K^{-1}\bm{k}_\ast \\
& =k\left(\bm{x}_\ast,\bm{x}_\ast\right)-\bm{k}_\ast^T\left(K^{-1}+\widetilde{\Sigma}\right)^{-1}\bm{k}_\ast
\end{aligned}
\end{equation}

Then, we can obtain
\begin{equation}
\begin{aligned}
& q\left(y_\ast\middle| X,\bm{y},\bm{x}_\ast\right)=\mathbb{E}_q\left[\pi_\ast|X,\bm{y},\bm{x}_\ast\right] \\
& =\int\Phi\left(f_\ast\right)q\left(f_\ast\middle| X,\bm{y},\bm{x}_\ast\right)df_\ast
\end{aligned}
\end{equation}

According to (\ref{app11}), we can obtain
\begin{equation}
\label{app22}
\begin{aligned}
& q\left(y_\ast\middle| X,\bm{y},\bm{x}_\ast\right) \\
& =\Phi\left(\frac{\bm{k}_\ast^T\left(K+\widetilde{\Sigma}\right)^{-1}\widetilde{\bm{\mu}}}{\sqrt{1+k\left(\bm{x}_\ast,\bm{x}_\ast\right)-\bm{k}_\ast^T\left(K+\widetilde{\Sigma}\right)^{-1}\bm{k}_\ast}}\right)
\end{aligned}
\end{equation}

By combining (\ref{13}) and (\ref{app22}), \textbf{Proposition \ref{prop6}} is proved.




\section{The Proof of Proposition \ref{prop7}}
\label{appf}
Given $f_s$ and $f_\ast$, $y_s$ and $y_\ast$ are conditionally independent. Hence, $p\left(y_s,y_\ast\middle|\bm{x}_s,\bm{x}_\ast\right)$ can be represented as
\begin{equation}
\begin{aligned}
& p\left(y_s=1,y_\ast=1\middle|\bm{x}_s,\bm{x}_\ast\right) \\
& =\iint{\Phi\left(f_s\right)\Phi\left(f_\ast\right)\phi\left(f_s,f_\ast\middle|\mu_{s\ast},\Sigma_{s\ast}\right)}df_sdf_\ast \\
& =\iint{\Phi\left(f_\ast\right)\phi\left(f_\ast\middle|{\widetilde{\mu}}_\ast\left(f_s\right),{\widetilde{\sigma}}_{\ast\ast}\right)df_\ast\Phi\left(f_s\right)}\phi\left(f_s\middle|\mu_s,\sigma_{ss}\right)df_s \\
& =\int\Phi\left(\frac{{\widetilde{\mu}}_\ast\left(f_s\right)}{\sqrt{{\widetilde{\sigma}}_{\ast\ast}+1}}\right)\Phi\left(f_s\right)\phi\left(f_s\middle|\mu_s,\sigma_{ss}\right)df_s
\end{aligned}
\end{equation}

Hence, \textbf{Proposition \ref{prop7}} is proved.

% \section{The Proof of Lemma \ref{lem}}
% \label{appg}
% \begin{equation}
% \begin{aligned}
% & R_e=\frac{1}{N_a}\sum_{n=1}^{N_a}\mathbb{I}\left(\bm{L}_n \neq \bm{Y}_n\right) \\
% & =\displaystyle\frac{FA+FL}{TL+TA+FL+FA} \\
% & =\displaystyle\frac{1}{\displaystyle\frac{TL+TA+FL+FA}{FA+FL}} \\
% & =\displaystyle\frac{1}{1+\displaystyle\frac{TL+TA}{FA+FL}} \\
% & =\displaystyle\frac{1}{1+\displaystyle\frac{\displaystyle\frac{TL}{TA}+1}{\displaystyle\frac{FA}{TA}+\displaystyle\frac{FL}{TA}}} \\
% & =\frac{1}{1+\displaystyle\frac{\displaystyle\frac{TL}{TA}+1}{\displaystyle\frac{1}{P_{md}-1}+\displaystyle\frac{1}{P_{fa}-1}}}
% \end{aligned}
% \end{equation}

% Hence, \textbf{Lemma \ref{lem}} is proved.



\bibliographystyle{jfm}
% Note the spaces between the initials
\bibliography{jfm-instructions}


\end{document}
