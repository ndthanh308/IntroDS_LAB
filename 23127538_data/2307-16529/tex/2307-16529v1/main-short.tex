\documentclass[aps,prb,floatfix,superscriptaddress,twocolumn,footinbib,longbibliography]{revtex4-2}
\usepackage{amssymb}
\usepackage{amsmath}
\usepackage{amsfonts}
\usepackage{appendix}
\usepackage{bm}
\usepackage[caption=false]{subfig}
\usepackage{graphicx}
\usepackage{epsfig}
\usepackage{epstopdf}
\usepackage{balance}
\usepackage[dvipsnames]{xcolor}
\usepackage{calc}
\usepackage{natbib}
%\usepackage[numbers,sort&compress]{natbib}
\usepackage[colorlinks,
linkcolor=blue,
anchorcolor=blue,
citecolor=blue,
urlcolor=blue]{hyperref}
\usepackage{lipsum}
\usepackage{braket}
\usepackage{subfig}
\usepackage{hyperref}% add hypertext capabilities
\usepackage{verbatim}
\newcommand{\mT}{{\mathcal{T}}}
\newcommand{\revision}[1]{{\color{blue}{#1}}}
%%%%%%%%%%%%%%%%%%%%%%%%%%%%%%%%%%%
% Main Text
%%%%%%%%%%%%%%%%%%%%%%%%%%%%%%%%%%%
%+++++++++++++++++++++++++++++++++++
% 1. Simulating condensed matter systems:
%     (1) topological systems, 
%     (2) wave transport, i.e., localization
%     (3) ...
%
%  the combination of particle and thermal
%  currents can be explored in a regime that 
%  is not accessible to experiments with 
%  metallic superconductors.
%+++++++++++++++++++++++++++++++++++
% 2. Separation of particle and energy transport
%   (1) different relaxation behavior, cf. Fig. 2
%
%+++++++++++++++++++++++++++++++++++
% 3. Introduction starts with W-F law in bulk metals then narrow down to meso-, nano- Fermions, then to bosons. 
%
% 
%%%%%%%%%%%%%%%%%%%%%%%%%%%%%%%%%%%
\begin{document}
%\title{Violation of the Wiedemann-Franz law in coupled heat and power transport of multimode optical fiber systems}
\title{Violation of the Wiedemann-Franz law in coupled thermal and power transport of optical waveguide arrays}
\author{Meng Lian}
\affiliation{School of Physics, Institute for Quantum Science and Engineering and Wuhan National High Magnetic Field Center,
Huazhong University of Science and Technology, Wuhan 430074,  China}
\author{Yin-Jie Chen}
\affiliation{School of Physics, Institute for Quantum Science and Engineering and Wuhan National High Magnetic Field Center,
Huazhong University of Science and Technology, Wuhan 430074,  China}
\author{Yue Geng}
\affiliation{School of Physics, Institute for Quantum Science and Engineering and Wuhan National High Magnetic Field Center,
Huazhong University of Science and Technology, Wuhan 430074,  China}
%%%%%%%%%%%%%%%%%%%%%
\author{Yuntian Chen}     
\affiliation{School of Optical and Electronic Information, Huazhong University of Science and Technology, Wuhan 430074, China}
\affiliation{Wuhan National Laboratory for Optoelectronics, Huazhong University of Science and Technology, Wuhan 430074, China}
\author{Jing-Tao L{\"u}}     
\email{jtlu@hust.edu.cn}
\affiliation{School of Physics, Institute for Quantum Science and Engineering and Wuhan National High Magnetic Field Center,
Huazhong University of Science and Technology, Wuhan 430074,  China}
%+%%%%%%%%%%%%%%%%%%%%%%%%%%%%%%%%%%%%%%%%%%%%%%%%%%%%%%%%
\begin{abstract}
In isolated nonlinear optical waveguide arrays with bounded energy spectrum,  simultaneous conservation of energy and power of the optical modes enables study of coupled thermal and particle transport in the negative temperature regime. 
%+%which leads to a steady-state distribution of mode occupancies as the Rayleigh-Jeans distribution. 
%+%Based on this, 
%We study coupled particle and heat transport across junctions made from weakly nonlinear multimode optical fibers. 
Here, based on exact numerical simulation and rationale from Landauer formalism, we predict generic violation of the Wiedemann-Franz law in such systems. This is rooted in the spectral decoupling of thermal and power current of optical modes, and their different temperature dependence. 
Our work extends the study of coupled thermal and particle transport into unprecedented regimes, not reachable in natural condensed matter and atomic gas systems. 
%While all modes contribute equally to heat current, their contribution to particle current decays quadratically with energy deviation from the chemical potential. 
%These results 
%shed light on the nonequilibrium transport property of multimode optical fibers. 
%pave the way of exploring coupled particle and heat transport in unprecedented regimes.
%
%+%We extracted the conductance, thermal conductance, Seebeck coefficient and Lorenz number under linear response through the image-only model given by the Landauer theory, and found that the system violates the Wiedemann-Franz law due to the different contributions of quasi-particles to the "electric" and thermal transport.
%+%Our results are consistent in order of magnitude with those obtained using the nonequilibrium Green's function method.
\end{abstract}
%+%%%%%%%%%%%%%%%%%%%%%%%%%%%%%%%%%%%%%%%%%%%%%%%%%%%%%%%%
%\begin{abstract}
%\end{abstract}

\maketitle

%\section{Introduction}	
%W-F law:
%Semiconductor point contact\cite{van_thermo_1992}
Precisely-engineered optical systems have enabled study of problems originated from condensed matter and other branches of physics in clean and controllable setups\cite{Yablonovitch_1987,Sajeev_1987,Haldane_2008}. Recently, such effort has been extended from linear, Hermitian to nonlinear, non-Hermitian systems\cite{miri2019exceptional,el2018non,wright_physics_2022}. An important insight is description of the complex behavior of weakly nonlinear coupled optical waveguide arrays within the framework of statistical thermodynamics\cite{wu_thermodynamic_2019,parto_thermodynamic_2019,makris_statistical_2020}. It provides fresh new insight in understanding puzzling optical phenomena, such as beam self-cleaning \cite{krupa_spatial_2017,liu_kerr_2016,niang_spatial_2019}, 
spatiotemporal solitons\cite{herr_temporal_2014,wright_controllable_2015,kalashnikov_stabilization_2022},
mode-locking \cite{wright_spatiotemporal_2017},  
 and in designing novel devices to manipulate optical waves\cite{wu_thermodynamic_2019}, from the point of view of statistical mechanics
\cite{jung_thermal_2022,pyrialakos_thermalization_2022,shi_controlling_2021,leykam_probing_2021,bloch_non-equilibrium_2022,xiong_k-space_2022,baudin_classical_2020}.


When an isolated system initially deviates from equilibrium, it relaxes to equilibrium through thermal and particle transport mediated by nonlinear interactions within the system.
As a cornerstone result of linear irreversible thermodynamics\cite{callen1951irreversibility}, such coupled transport of thermal and particle current is characterized by the transport matrix. Its diagonal and off-diagonal elements correspond to particle ($G$), thermal ($G_T$) conductance (or conductivity) and thermo-particle (Seebeck and Peltier) transport coefficients, respectively. Study of these transport properties has gained invaluable information on the equilibrium and nonequilibrium properties of condensed matter and atomic gases\cite{brantut_thermoelectric_2013,nietner_transport_2014,gallego-marcos_nonequilibrium_2014,hausler_interaction-assisted_2021,2013arXiv1306.4018H,PhysRevLett.113.170601,rancon_bosonic_2014,uchino_bosonic_2020,husmann_connecting_2015,husmann_breakdown_2018}. The celebrated Wiedemann-Franz (W-F) law links the thermal and particle transport coefficients of the system, with the proportionality constant $L=G_T/(GT)$ being constant in Fermi liquids $L_0=k_B^2\pi^2/3$ with $k_B$ the Boltzmann constant. Break down of the W-F law indicates decoupling of thermal and particle transport\cite{lee_anomalously_2017,husmann_breakdown_2018}, whose origin ranges from strong inter-particle interaction to singular quasi-particle density of states. 


In this work, by numerically tracking the thermalization process of isolated transport junctions made from weakly nonlinear coupled waveguides within the microcanonical ensemble, we study the coupled power\footnote{It plays the role of particle number here. Thus, power conservation corresponds to particle number conservation.}  and thermal transport in such junctions. This is made possible due to simultaneous power and energy conservation. The same condition is difficult to realize in other bosonic systems like phonons or magnons, where there is no particle number conservation. We predict a generic violation of the W-F law in such junctions, with the Lorenz number $L \propto 1/T^2$ (Fig.~\ref{fig:1a}). It does not depend on details of the lattice structures, model parameters, 
 and is valid in both positive and negative temperature regimes. We provide a rationale of this result using the Landauer transport theory in the classical limit.
These results illustrate the opportunity to study fundamental problems of statistical thermodynamics in unprecedented regimes using engineered optical systems \cite{baldovin2021statistical,marques2023observation}. 
%The MOF system provides experimental opportunity to realize such temperature regimes and resolve the debate. 

%use temperature and chemical potential as independently tunable parameters to study the coupled transport behavior of optical power and energy in symmetric junctions composed of waveguides at negative temperatures.
%We extracted the conductance, thermal conductance, and Seebeck coefficients in the optical system using the method of extracting transport coefficients in cold atomic gases\cite{brantut_thermoelectric_2013,grenier_thermoelectric_2016} and then performed a qualitative analysis by non-equilibrium Green's functions, 
%revealing signs that the system violates the W-F law.	


%In MOFs, exchange of power and energy between different modes leads to rich nonlinear dynamical behaviors such as beam self-cleaning \cite{krupa_spatial_2017,liu_kerr_2016,niang_spatial_2019}, 
%spatiotemporal solitons\cite{herr_temporal_2014,wright_controllable_2015,kalashnikov_stabilization_2022},
%mode-locking \cite{wright_spatiotemporal_2017}.
%They can be ultimately linked with thermalization\cite{jung_thermal_2022,pyrialakos_thermalization_2022,shi_controlling_2021,leykam_probing_2021}, condensation\cite{bloch_non-equilibrium_2022,xiong_k-space_2022,baudin_classical_2020} and other thermodynamic processes of optical modes. 
%Wherein, the weak nonlinearity plays the role of collisions among ideal gas molecules, allowing the system to reach the equilibrium Rayleigh-Jeans (RJ) distribution for conserved total power and energy\cite{pourbeyram_direct_2022}.
%These studies provide strong support for studying the nonequilibrium dynamics of such multimode systems within the framework of statistical mechanics.

%Based on this, the thermalization of orbital angular momentum in multimode fibers\cite{podivilov_thermalization_2022,wu_thermalization_2022},  Bose-Einstein condensation, and the dynamics of thermalization in optical lattice systems\cite{jung_thermal_2022,pyrialakos_thermalization_2022,shi_controlling_2021,leykam_probing_2021} have been studied successively,
%which provides support for studying the nonequilibrium dynamics of such multimode systems.


%When an isolated system initially deviates from equilibrium, it relaxes to equilibrium through particle and heat exchange within the system.
%This behavior can be characterized by the `diagonal' particle $G$ and thermal conductance $G_T$, and the `off-diagonal' Seebeck and Peltier-type coefficients within linear irreversible thermodynamics\cite{callen1951irreversibility}. 
%Coupled particle and energy transport has been 
%extensively studied in cold atomic systems\cite{brantut_thermoelectric_2013,nietner_transport_2014,gallego-marcos_nonequilibrium_2014,hausler_interaction-assisted_2021},
%bosonic systems\cite{2013arXiv1306.4018H,PhysRevLett.113.170601,rancon_bosonic_2014,uchino_bosonic_2020}, and resonantly interacting Fermi gases connected by quantum dot contact \cite{husmann_connecting_2015,husmann_breakdown_2018}.
%The ratio of conductance to thermal conductance, $L=G_T/\left( TG \right)$, is known as the Lorenz number, is robust in low-temperature Fermi liquids, and is known as the Wiedemann-Franz law.
%However, in strongly correlated systems, the Wiedemann-Franz law breaks down\cite{lee_anomalously_2017,husmann_breakdown_2018}.
%
%Since particles are carriers of energy, transport is usually coupled together\cite{iubini_nonequilibrium_2012,iubini_coupled_2019}, 
%and this coupled transport behavior has been extensively studied in cold atomic systems\cite{brantut_thermoelectric_2013,nietner_transport_2014,gallego-marcos_nonequilibrium_2014,hausler_interaction-assisted_2021},
%bosonic systems\cite{2013arXiv1306.4018H,PhysRevLett.113.170601,rancon_bosonic_2014,uchino_bosonic_2020}, and resonantly interacting Fermi gases connected by quantum dot contact \cite{husmann_connecting_2015,husmann_breakdown_2018}.
%The ratio of conductance to thermal conductance, $L=G_T/\left( TG \right)$, is known as the Lorenz number, is robust in low-temperature Fermi liquids, and is known as the Wiedemann-Franz law.
%However, in strongly correlated systems, the Wiedemann-Franz law breaks down\cite{lee_anomalously_2017,husmann_breakdown_2018}.

%In this work, we use temperature and chemical potential as independently tunable parameters to study the coupled transport behavior of optical power and energy in symmetric junctions composed of waveguides at negative temperatures.
%We extracted the conductance, thermal conductance, and Seebeck coefficients in the optical system using the method of extracting transport coefficients in cold atomic gases\cite{brantut_thermoelectric_2013,grenier_thermoelectric_2016} and then performed a qualitative analysis by non-equilibrium Green's functions, 
%revealing signs that the system violates the Wiedemann-Franz law.	

%In Sec.~\ref{Model_Theory}, we introduce the computational model and the theoretical framework used throughout this paper. 
%We first present our computational model and parameters and review the theory of photothermodynamics in Sec.~\ref{Model}. In Sec.~\ref{microcanonical}, 
%we introduce a phenomenological model to extract the transport coefficients of the system, and in Sec.~\ref{canonical}, we introduce the method of computing transport parameters by nonequilibrium Green's functions (NEGF) in the giant regular system synthesis.
%In Sec.~\ref{Results}, we calculated the phenomenological model for symmetric junctions made from square reservoirs (Sec.~\ref{square}) and hexagonal honeycomb reservoirs ((Sec.~\ref{honeycomb})) and analyze them in comparison with the NEGF results to explore the reasons why the system violates the Wiedemann-Franz law.
%Finally, our results are summarized in Sec.~\ref{CONCLUSION}.
%
%\section{Model \& Theory} \label{Model_Theory}
%\subsection{Model}  \label{Model}
{\bf Model --}
We consider transport junction made from one-dimensional waveguide arrays. The junction consists of two large but finite-size reservoirs connected by a chain.  We study coupled heat and power transport properties across the chain. As shown in Fig.~\ref{fig:1a}, each node represents a waveguide, along which ($z$ direction) the optical wave can propagate. In the reservoir the nodes are periodically arranged. This is a prototypical junction structure to study transport properties of solid-state \cite{agrait2003quantum} and cold atom systems\cite{krinner2017two}. 
%
%HAMILTONIAN
%Total Hamiltonian of the system contains a linear part $H_L$ and a nonlinear part $H_{NL}$, written as\cite{wu_thermodynamic_2019,rasmussen_statistical_2000}
%\begin{equation} 
%  H\left( \psi _m,i\psi _{m}^{*} \right) =\sum_{m,n}^{\mathcal{M}}{\kappa _{mn}\psi _{m}^{*}\psi _n}+\frac{1}{2}\chi \sum_m^{\mathcal{M}}{\left| \psi _m \right|^4}.
%      \label{equ:H}
%\end{equation}
%Here, $\psi _{m}$  is the normalized optical field complex amplitude at node $m$, which forms the canonical conjugate pairs with $i\psi _{m}^{*}$, $\mathcal{M}$ is the set of nodes, $\kappa _{mn}=\kappa _{mn}^{*}$ is the normalized coupling coefficient between node $m$ and node $n$. We consider only the nearest neighbour coupling. 
%Finally, $\chi$ is a parameter associated with the Kerr type nonlinearality of the system.
%HAMILTONIAN

%Here we consider two types of symmetric junctions whose reservoirs are made from square and hexagonal honeycomb lattices, respectively.
%We choose size of $N_x*N_y=19*19$ for square lattices, and $N_x*N_y=16*31$ for honeycomb lattices, where $N_x$ and $N_y$ are the numbers of waveguides arranged in the $x$-direction and $y$-direction, respectively.
%Initially, the two reservoirs are in their respective equilibrium states. After turning on their coupling to the chain simultaneously, transfer of energy and optical power (particle number) occurs. 
%Eventually, the whole system evolves to a new equilibrium state.
%%%%%%%%%%%%%%%%%%%%%%%%%%%%%%%%%%%%%%%%%%%%%%%%%%%%%%%%%%%%%%%%%%
\begin{comment}
% Figure environment removed
\end{comment}
%%%%%%%%%%%%%%%%%%%%%%%%%%%%%%%%%%%%%%%%%%%%%%%%%%%%%%%%%%%%%%%%%%

% Figure environment removed



%Using Eq.~(\ref{equ:H}), 
Propagation of the optical modes along the waveguides is described by 
the discrete nonlinear Schr\"odinger equation (DNSE)\cite{lederer_discrete_2008,agrawal_applications_2001}
\begin{equation} 
i\frac{d\psi _{m}}{dz}+\kappa \sum_{\{n\}}{\psi _{n}}+\chi \left| \psi _{m} \right|^2\psi _{m}=0,
  \label{equ:DNES}
\end{equation}
with dimensionless parameters $\psi_m$, $\kappa$, $\chi$ representing the complex wave amplitude, the nearest neighbour coupling between waveguides and Kerr-type nonlinear coefficient, respectively.  
Here the coordinate $z$ along waveguide plays the role of time in standard Schr\"odinger equation, and $\{n\}$ includes all the nearest neighbours of $m$. 
%

For weak nonlinear interactions, we can  diagonalize the linear term and obtain the corresponding eigenmode (supermode) propagating constant $\beta_k$ and corresponding vector $\varphi_k$. For a given state $\psi$, the modal occupancy is obtained by projection onto each supermode 
$\left|c_k\right|^2=\left| \langle \varphi _k | \psi \rangle \right|^2$.
%
The system evolves with conserved internal energy
%\begin{align}
$U=\sum_{k=1}^M{\varepsilon _k\left| c_k \right|^2}	$
%  \label{eq:ud}
%\end{align}
and optical power
%\begin{align}
$P=\sum_{k=1}^M{\left| c_k \right|^2}$,
%        \label{eq:pd}
%\end{align}
with the total number of modes $M$, the eigen energy defined by the negative of $\beta$ as $\varepsilon_k = -\beta_k$, and $\beta _1\ge \beta _2\ge \cdots \ge \beta _M$. The system thus has a bounded energy spectrum between $[\varepsilon_1, \varepsilon_M]$.
The weak nonlinear part introduces coupling among different supermodes, such that the system can thermalize to an equilibrium state through energy and power redistribution among supermodes. It plays a role similar to molecular collisions in the ideal gas model.
%\revision{
%The mode occupancy $\left|c_i\right|^2=\left| \langle \varphi _i | \Psi \rangle \right|^2$ is obtained by projecting the optical field complex amplitude vector $\Psi =\left( \psi _{11},\psi _{12},\cdots ,\psi _{ij},\cdots \right) ^T$ onto the supermode eigenvector $\left| \varphi _i \right> $, 
%and $\varepsilon _i=-\beta _i$ is the negative of the eigenvalue of ${H}_L$, $\beta$ is the mode propagation constant and satisfies $\beta _1\ge \beta _2\ge \cdots \ge \beta _M$.
%The role of nonlinear Hamiltonian is to introduce interaction between different supermodes, such that energy and power can redistribute among them, similar to collisions between gas molecules in the ideal gas model.
%}

{\bf Equilibrium thermodynamic theory --} 
We can use a recently developed optical thermodynamic theory to describe each reservoir\cite{wu_thermodynamic_2019}. The system internal energy can be written as (omitting the reservoir index)
\begin{align}
	U = M T + \mu P,
 \label{eq:eos}
\end{align}
where $T$ is the dimensionless optical temperature, $\mu$ is the chemical potential. An optical entropy can be defined as 
\begin{align}
S=\sum_{k=1}^M{\ln \left| c_k \right|^2}. 
\end{align}
Using the maximum entropy principle, at thermal equilibrium, the mode occupancy follows the classical Rayleigh-Jeans (R-J) distribution 
\begin{align}
	f(\varepsilon_k) = \left| c_k \right|^2=\frac{T}{\varepsilon _k-\mu}.
	\label{eq:rj}
\end{align}


One prominent feature of the system is that, for given power $P$, the 
optical entropy $S$ no longer varies monotonically with the internal energy $U$ due to bounded energy spectrum. Consequently, the system can reach the  negative optical temperature regime\cite{wu_thermodynamic_2019,wu_entropic_2020}.
%
From Eqs.~(\ref{eq:eos}-\ref{eq:rj}), we have plotted the phase diagram of the reservoir with square lattice. As shown in Fig.~\ref{fig:1b}, only $\left\{ \left( T,\mu \right) |T<0,\mu >\varepsilon _M \right\}$ and $\left\{ \left( T,\mu \right) |T>0,\mu <\varepsilon _1 \right\}$ can be visited, and the other regimes are forbidden($R_f$). This follows from the requirement $\left| c_k \right|^2 \ge 0$.  Moreover, the chemical potential is out of the band spectrum of the system, i.e., $\mu$ is below (above) the lowest (highest) energy of the spectrum for positive (negative) temperature. The positive and negative temperature regimes are anti-symmetric with each other in the phase diagram when neglecting the nonlinear effect.  
In the following, we focus on the negative temperature regime and numerically check that this anti-symmetry is still approximately valid in the weakly nonlinear regime (Fig.~\ref{fig:3}). 

%\begin{table}[b]
%	\caption{\label{tab:1}
%	Correspondence between variables in standard and optical thermodynamic theory.}
%	\begin{ruledtabular}
%	\begin{tabular}{cc}
%	 Standard thermodynamics&Optical thermodynamics\\
%	\hline
%	\begin{tabular}[c]{@{}c@{}}Phase space coordinate: \\ $\left( q_1,q_2,\cdots q_M,p_1,p_2,\cdots p_M \right)$\end{tabular}
%	&\begin{tabular}[c]{@{}c@{}}Phase space coordinate: \\$\left( \left| c_1 \right|^2,\left| c_2 \right|^2,\cdots ,\left| c_M \right|^2 \right) $\end{tabular} \\
%	Number of particles: $N$& Optical power: $P$ \\
%	Energy: $E$& EM momentum: $U$   \\
%	Volume: $V$& Number of modes: $M$ \\
%	%\begin{tabular}[c]{@{}c@{}}Equation of state: \\ $PV=Nk_BT$\end{tabular}
%	%&\begin{tabular}[c]{@{}c@{}}Equation of state: \\ $U=MT+\mu P$\end{tabular}  \\
%	\begin{tabular}[c]{@{}c@{}}Interactions: \\ molecular forces\end{tabular}
%	&\begin{tabular}[c]{@{}c@{}}Interactions: \\ optical nonlinearity\end{tabular}  \\
%	\end{tabular}
%	\end{ruledtabular}
%\end{table}

%The presence of nonlinearity breaks the symmetry of the system in the positive ($R_p$) and negative ($R_n$) temperature region. Larger input power in $R_p$ region may generate modulation instability, resulting in a significant transfer of internal energy to the nonlinear part\cite{xiong_k-space_2022}.
%This leads to states that deviates from prediction from the thermodynamic theory.
%For $R_n$, there is no restriction on the input power. Therefore, in the present work we mainly focus on the $R_n$ region of the phase space.

%\subsection{Transport coefficients in the microcanonical ensemble}  \label{microcanonical}
{\bf Transport coefficients in microcanonical ensemble --} 
Transport theory is normally formulated with open boundaries using grand canonical ensemble.  Due to the finite size of present junction, it is convenient to utilize the microcanonical setup. We extract the transport coefficients by following the thermalization process of a microcanonical system from given initial conditions.
%In order to extract the transport coefficients from the evolution equation, we employ the irreversible thermodynamic theory to 
Using the theory of irreversible thermodynamics, 
%we can make connection between the driving affinities $\varDelta \mu$, $\varDelta T$ with the corresponding power ($I_P$) and entropy ($I_S$) fluxes, which follow the Onsager symmetry
%%$I_P=\partial \left( P_R-P_L \right) /\left( 2\partial z \right) $ and $I_S=\partial \left( S_R-S_L \right) /\left( 2\partial z \right)$.
%\begin{equation} 
%\left( \begin{array}{c}
%	I_P\\
%	I_S\\
%\end{array} \right) =\left( \begin{matrix}
%	K_0&		K_1/T\\
%	K_1/T&		K_2/T^2\\
%\end{matrix} \right) \left( \begin{array}{c}
%	\Delta \mu\\
%	\Delta T\\
%\end{array} \right). 
%\label{equ:1}
%\end{equation}
%The positive direction of the current is chosen as from $L$ to $R$ reservoir.
%%
%The matrix elements $K_0$, $K_1$ and $K_2$ depend on the state of the system and can be expressed in terms of the power conductance $G$, the thermal conductance $G_T$, the Seebeck coefficient $\alpha _{c}$ and the Lorenz number $L=G_T/TG$,
%\begin{equation} 
%  \frac{\partial}{\partial z}\left( \begin{array}{c}
%    \Delta P\\
%    \Delta S\\
%  \end{array} \right) =-2G\left( \begin{matrix}
%    1&		\alpha _{{c}}\\
%    \alpha _{{c}}&		L+\alpha _{{c}}^{2}\\
%  \end{matrix} \right) \left( \begin{array}{c}
%    \Delta \mu\\
%    \Delta T\\
%  \end{array} \right) 
%  \label{equ:2}
%\end{equation}
%%
%%These transport coefficients characterize the process by which the two reservoirs exchange optical power and energy through the chain to relax the system to an equilibrium state.
%%Where the transport process occurs mainly in the chain, and the reservoirs as macroscopic systems can be described by thermodynamic methods, the transport coefficients should not depend on the size of the reservoirs, i.e., the structure should be convergent.
%%The convergence tests of the symmetric junctions used in the calculations of this work are shown in Appendix~\ref{B}.
%%
%%Since the optical power proportion in the chain at equilibrium is $P_c/P_s<M_c/M_s$, where $M_s$ and $M_c$ are the waveguide numbers of the system and the chain, respectively. So we always keep $M_c/M_s<0.01$ in all our simulations to ignore the effect of the optical power in the chain.
%
%%We now describe the evolution of the optical power and internal energy of the two reservoirs caused by the temperature and chemical potential differences. 
%Due to the weak link between the two reservoirs, the bottleneck of the thermalization process is at the central chain. 
%For the two reservoirs, the thermalization can be considered as a quasi-static process, \revision{which can be verified by tracking the evolution of the system.} 
%This allows us to introduce the thermodynamic equation of the reservoir to extend the above results to the time evolution of the biases.
%Using the Maxwell's relations $\left. \left( \partial S/\partial \mu \right) \right|_T=\left. \left( \partial P/\partial T \right) \right|_{\mu}$ and $-\left. \left( \partial S/\partial P \right) \right|_T=\left. \left( \partial \mu /\partial T \right) \right|_P$, we can obtain
%\begin{eqnarray}
%  dP=\left. \frac{\partial P}{\partial \mu} \right|_Td\mu +\left. \frac{\partial P}{\partial T} \right|_{\mu}dT=\kappa d\mu +\alpha _r\kappa dT,
%  \label{equ:3}
%  \\
%  dS=\left. \frac{\partial S}{\partial P} \right|_TdP+\left. \frac{\partial S}{\partial T} \right|_PdT=\alpha _rdP+\kappa ldT
%  \label{equ:4}
%\end{eqnarray}
%where $\kappa =\left. \left( \partial P/\partial \mu \right) \right|_T$, $\alpha _r=-\left. \left( \partial \mu /\partial T \right) \right|_P$, $l=C_P/T\kappa $ are the compression coefficient, Seebeck coefficient and Lorenz number of a single reservoir, respectively.
%Here $C_P=\left. T\left( \partial S/\partial T \right) \right|_P$ is the reservoir heat capacity at constant power.  They can be determined by the definition of optical power and optical entropy.
%
%With Eqs.~(\ref{equ:2}-\ref{equ:4}), 
we can obtain the evolution equations of $\Delta P$ and $\Delta T$ (See Appendix~\ref{C}). 
For symmetric junctions with identical left and right reservoirs, 
they read
\begin{equation} 
  \tau _0\frac{\partial}{\partial z}\left( \begin{array}{c}
    \Delta P\\
    \Delta T\\
  \end{array} \right) =-\left( \begin{matrix}
    1&		-\kappa \alpha\\
    -\frac{\alpha}{\ell \kappa}&		\frac{L+\alpha ^2}{\ell}\\
  \end{matrix} \right) \left( \begin{array}{c}
    \Delta P\\
    \Delta T\\
  \end{array} \right) 
  \label{equ:5}
\end{equation}
where we have defined a transport timescale $\tau _0=\kappa /\left( 2G \right)$ .
%the Seebeck coefficient $\alpha =\alpha _r-\alpha _c$. 
%\revision{what is $\alpha_c$?}
Here, the reservoir properties $\kappa =\left. \left( \partial P/\partial \mu \right) \right|_T$, $\alpha _r=-\left. \left( \partial \mu /\partial T \right) \right|_P$, $C_P=\left. T\left( \partial S/\partial T \right) \right|_P$ are the compression coefficient, Seebeck coefficient and the heat capacity at constant power, respectively,
with $l=C_P/T\kappa$.
They can be obtained from the thermodynamic relations [Eqs.~(\ref{equ:kappa}-\ref{equ:CP})]. 
The effective Seebeck coefficient $\alpha=\alpha_r-\alpha_c$ characterizes the difference between the reservoir itself $\alpha_r$ and the whole junction $\alpha_c$.

%\revision{This form of the transport matrix highlights that the optical power and temperature dynamics result from a competition between the transport properties of the constriction and the thermodynamical properties of the reservoirs.}
%When $\Delta P_0=0$, the optical power flow can be rewritten as $I_P=-2G\alpha \Delta T$, and its direction is determined by the sign of $G\alpha$: if $G\alpha>0$, the optical power is transferred from the cold to the hot reservoir; otherwise from hot to cold.
%Since the equation above is the first-order differential equation, the time evolutions can analytically be determined. 
The general solution of the above equations are
%is expressed as 
\begin{widetext}
	\begin{align}
		\Delta P\left( z \right) =\left( \frac{\Omega +\delta -1}{2\Omega}\Delta P_0+\frac{\alpha \kappa}{2\Omega}\Delta T_0 \right) e^{-\frac{z}{\tau _>}}\left[ 1+\frac{\left( \Omega -\delta +1 \right) \Delta P_0-\alpha \kappa \Delta T_0}{\left( \Omega +\delta -1 \right) \Delta P_0+\alpha \kappa \Delta T_0}e^{-\frac{z}{\tau _<}} \right],
  \label{equ:6}\\
  \Delta T\left( z \right) =\left( \frac{\Omega -\delta +1}{2\Omega}\Delta T_0+\frac{\alpha}{2l\kappa \Omega}\Delta P_0 \right) e^{-\frac{z}{\tau _>}}\left[ 1+\frac{l\kappa \left( \Omega +\delta -1 \right) \Delta T_0-\alpha \Delta P_0}{l\kappa \left( \Omega -\delta +1 \right) \Delta T_0+\alpha \Delta P_0}e^{-\frac{z}{\tau _<}} \right]
  \label{equ:7}
	\end{align}
\end{widetext}
where $\delta =\left[ 1+\left( L+\alpha ^2 \right) /l \right] /2$ and $\varOmega =\sqrt{\delta ^2-L/l}$ form the eigenvalues of the evolutionary matrix $\lambda _{\pm}=\delta \pm \varOmega $. 
The fast and slow timescales $\tau_<=\tau _0/2\varOmega$ and $\tau_>=\tau _0/\left( \delta -\varOmega \right)$ indicate the evolution of the system in two stages (Fig.~\ref{fig:2}): (1) A saturation process is characterized by the time scale $\tau_<$, which dominates the initial short time period. This process leads to an initial increase in the absolute value of the particle number deviation until it reaches a maximum . (2) 
A decay process is characterized by the time scale $\tau_>$, which dominates the longer period of time after saturation. 
The three transport coefficients $G$, $\alpha _c$ and $L$ (or $G_T$) can then be acquired by fitting the numerical results . 
Figure~\ref{fig:2} shows results of fitting the transient evolution of the rectangular junction. The details can be found in Appendix~\ref{D}.



% Figure environment removed


%% Figure environment removed

%%%%%%%%%%%%%%%%%%%%%%%%%%%%%%%%%%%%%%%%%%%%%%%%%%%%%%%%%%%%%%%%%%
\begin{comment}

It is worth noting that the NEGF is calculated for the case of two reservoirs coupled separately to the bathes, so the thermodynamic coefficients obtained characterize the whole system. 
However, it can be found that when the reservoir is large enough, the thermodynamic coefficients depend significantly on the structure of the chain, so we approximate them as parameters of the chain.

\end{comment}
%%%%%%%%%%%%%%%%%%%%%%%%%%%%%%%%%%%%%%%%%%%%%%%%%%%%%%%%%%%%%%%%%%

%\section{Results \& Discussions} \label{Results}
%\subsection{Symmetric junction with square reservoirs} \label{square}
{\bf Numerical Results --} 
Figure~\ref{fig:3a}-\ref{fig:3d} shows temperature dependence of the transport coefficients for a symmetric junction with square-lattice reservoirs. 
We notice that, in the negative temperature regime, conductance $G<0$, indicating that the power is transferred from the low to the high chemical potential reservoir when $\varDelta T=0$. This seemingly surprising result does not violate the second law of thermodynamics.
Considering two subsystems that can exchange particles with each other under constant temperature, the second law of thermodynamics requires that the entropy does not decrease during the particle exchange
\begin{equation} 
  dS=\left( \frac{\mu _L}{T_L}-\frac{\mu _R}{T_R} \right) dP \ge 0
  \label{equ:9}
\end{equation}
where $dP$ is the power lost from the left reservoir or gained from the right reservoir.
When $T_L=T_R>0$, the power is transferred from high to low chemical potential. However, when $T_L=T_R<0$, the direction is reversed, meaning $G<0$. Since heat is transferred from high to low temperature in both positive and negative temperature regions, we have $G_T>0$. 

Figure~\ref{fig:3a} shows an approximated linear $T$ dependence of $G$ for all the nonlinear parameters considered. This can be understood qualitatively from the Landauer formalism (Appendix~\ref{sec:lt}). 
%
%The results are in the same order of magnitude as the NEGF calculations and observe that the conductance expected by the Landauer model is proportional to temperature. 
%As the nonlinearity decreases, the power exchange process between the modes weakens, resulting in the smaller $\left| G \right|$, while deviating from the results predicted by the Landauer model. 
%This is because the $\gamma$ terms coupled to the bath in the NEGF make the calculation deviate from the microcanonical system's relevant results. It can be found that if $\gamma$ is reduced, the $\left| G \right|$ calculated by the Landauer model is also reduced.
%
%Unlike the fermionic system in the low temperature regime where only the energy states near the Fermi energy level contribute to the conductance, according to
The conductance can be approximated as 
\begin{equation} 
  G=\int {\frac{d\varepsilon}{2\pi}{\mathcal{T}}(\varepsilon) \frac{T}{\left( \varepsilon -\mu \right) ^2}},
  \label{equ:G}
\end{equation}
indicating $G \propto T$ for temperature independent transmission $\mathcal{T}(\varepsilon)$. Moreover, the optical modes contribute to the transport with a weighting factor inversely proportional to the square of the energy deviation from the chemical potential.  
%Similar analysis using the Landauer formalism shows that the thermal conductance $G_T$ does not depend on temperature. The numerical results shows relatively weak dependence, due to the weak nonlinear effect. 
%As shown in Fig.~(\ref{fig:3a}), as the temperature decreases, i.e., $\left| T \right|$ increases, the weighting factor $\left| T/\left( \varepsilon -\mu \right) ^2 \right|$ of the conductance increases, causing $\left| G \right|$ to increase.

% Figure environment removed



Temperature dependence of the Seebeck coefficient $\alpha_c$ is shown in Fig.~\ref{fig:3b}. It can be well fitted by an inverse $T$ dependence $\alpha_c \propto -1/T$ predicted by the Landauer model. 
%\revision{Since only one quasiparticle contributes to the thermoelectric behavior, this makes $\alpha _c$ always positive within $R_p$ and $R_n$.}
%However, the conductance in the positive and negative temperature regions flips, and flip the direction of the thermoelectric power current by $\underset{\Delta T\rightarrow 0}{\lim}\left. I_P/\Delta T \right|_{\Delta \mu =0}=\alpha _{c}G$, the power will be transported from the hot end to the cold end at positive temperature and from the cold end to the hot end at negative temperature.
%This result is natural, as can also be seen from the power $P=\sum_{i=1}^M{T/\left( \varepsilon _i-\mu \right)}$. Initially, when the two reservoirs have the same chemical potential, when $T_L>T_R>0$,there is $P_L>P_R$; when $0>T_L>T_R$,there is $P_L<P_R$.
%
The little effect of nonlinear interaction on $\alpha_c$ can also be captured by the Landuer result. In fact, $\alpha_c$ can be written as $\alpha _c = K_1/(K_0T)$. $K_0$ and $K_1$ have similar dependence on the nonlinear interaction, they cancel with each other in $\alpha_c$. 

Similar to conductance, the thermal conductance $G_T$ also increases with nonlinearity (Fig.~\ref{fig:3c}). From the Landauer formalism, we get
\begin{equation} 
  G_T=\int {\frac{d\varepsilon}{2\pi}}\mathcal{T}(\varepsilon) -T\alpha _{c}^{2}G.
  \label{equ:15}
\end{equation}
This gives a temperature independent $G_T$. Similar to $G$, the temperature dependence in the numerical results is due to nonlinear interaction.  
The second term on the right side of Eq.~(\ref{equ:15}) represents correction to the thermal conductance due to Seebeck coefficient. Here, its magnitude is comparable to the first term. This is in contrast to the case of electrons in condensed matter system, where the thermoelectric correction is often negligible.  
%When the optical power and temperature difference are uncoupled, $G_T$ is a quantity that depends only on the structure of the system.
%For the symmetric junction made up of square reservoirs in our setup, the first term at the right end of Eq.~(\ref{equ:15}) is 0.49.
%However, in the results of Fig.~\ref{fig:3c}, $G_T$ is one order of magnitude less, indicating that the thermopower strongly suppresses the heat transfer.
%This is different from the result that the correction term is negligible in traditional materials.
%The reason is that $T$ and $\mu$ are not equivalent in conventional materials, and their units are K and J, respectively, leading to the thermal potential $\alpha _c=-\partial \mu /\partial T$ that is often of the same order as $k_B$, but in multimode fiber systems, both $T$ and $\mu$ characterize the disorder of the optical field, so $\alpha _c$ is on the order of $10^0$, which makes the correction term non-negligible.
%This means that the temperature relaxation is accomplished primarily through power exchange rather than heat.

%Next, we focus on the the Lorenz number $L$. 
The variation of $L$ with temperature and chemical potential is shown in Fig.~\ref{fig:3d}, and is once again captured by Landauer's theory. 
It can be found that the Lorenz number is more sensitive to temperature than the chemical potential (Inset).
In the Landauer picture, $G_T \propto T^0$ together with $G \propto T$ leads to $L \propto 1/T^2$, which violates the W-F law. The decrease in $L$ with decreasing temperature comes from the increase in $\left| G \right|$.
Similar to $\alpha _{c}$, we also observe a rather weak dependence on the strength of nonlinear interaction. Thus, the violation of W-F law here is not due to nonlinear interaction. Rather, its origin is similar to that in coherent mescoscopic conductors, which can be fully accounted by the single particle Landauer formalism.


%\subsection{Symmetric junction with hexagonal honeycomb reservoirs} \label{honeycomb}

% Figure environment removed

To further illustrate the violation of the W-F law, we considered a symmetric junction made from hexagonal honeycomb reservoirs with the same bandwidth (Fig.~\ref{fig:1a}).
Junction with hexagonal reservoirs has a broader transmission spectrum compared to that with square reservoirs (Fig.~\ref{fig:T}).
This is due to the two van Hove singularities unique to hexagonal lattice, particularly evident in the DOS (Fig.~\ref{fig:DOS}), giving higher transmission to the modes near the band edges.
This makes the thermal conductivity increase to about twice that of the square case.
As a result, we observe a more severe violation of the W-F law (Fig.~\ref{fig:5}), albeit with the same $1/T^2$ dependence. 



%\section{CONCLUSION} \label{CONCLUSION}
{\bf Conclusions --}
We have developed a theory based on microcanonical ensemble 
to extract the transport coefficients of the coupled thermal and power transport in coupled waveguide arrays. We found a systematic violation of the Wiedemann-Franz law with the Lorenz number $L \propto 1/T^2$. 
The numerical results can be understood  from the classical limit of Landauer formula. Our theory based on microcanonical ensemble is compatible to experimental realizations.   
Thus, it paves the way of studying fundamental nonequilibrium thermodynamic processes  in parameters regimes that are unreachable in condensed matter and atomic gas systems. 

%Using the nonequilibrium Green's function method, we reveal that the Lorentz number $L\propto 1/T^2$ of the system, which violates the Wiedemann-Franz law, is caused by the different ways of mode contributions to the conductance and thermal conductance.
%The two Van Hove singularities unique to hexagonal honeycomb symmetric junction lead to the thermal conductance twice as high as that of the square reservoir symmetric junction, which makes the Lorentz number of the system vary more significantly.
%As the nonlinearity gradually increases, $\left| G \right|$ and $G_T$ will increase, indicating that nonlinearity promotes the thermalization of the system. However, the composite parameters $\alpha _c$ and L are insensitive to the change of nonlinearity.
%Our work provides the possibility to measure the transport coefficients of optical systems in experiments, and by changing the structure of the reservoir, many interesting coupled transport phenomena can be observed.


%\begin{acknowledgments}
This work was supported by the National Natural Science Foundation of China (Grant Nos. 22273029 and 21873033).
%\end{acknowledgments}

\appendix
\renewcommand\thefigure{\thesection\arabic{figure}}  
%\section{The discrete nonlinear Schrödinger equation} \label{A}
%\setcounter{figure}{0} 
%We briefly illustrate the derivation of DNSE following closely Ref.~\onlinecite{agrawal_applications_2001}. 
%Neglecting the loss of the waveguide, the transmission of the optical field is described by the following Helmholtz equation
%%In discrete fiber systems with identical cores, from the coupled mode theory one can obtain the following control equation to describe \revision{XXXX}
%\begin{equation} 
%  \nabla ^2\boldsymbol{\tilde{E}}\left( \boldsymbol{r,}\omega \right) +\tilde{n}^2\left( x,y \right) k_{0}^{2}\boldsymbol{\tilde{E}}\left( \boldsymbol{r,}\omega \right) =0,
%  \label{equ:Hel}
%\end{equation}
%where $k_{0}$ is the wave vector in the vacuum, $\tilde{n}$ is the refractive index associated with the field strength, and $\boldsymbol{\tilde{E}}$ is the electric field distribution over the frequency domain.
%Assuming that the direction of electric field polarization $\boldsymbol{\hat{e}}$ remains constant during the transmission, the electric field can be approximated as
%\begin{equation} 
%  \boldsymbol{\tilde{E}}\left( \boldsymbol{r,}\omega \right) \approx \boldsymbol{\hat{e}}\sum_{m=1}^M{\tilde{A}_m\left( Z,\omega \right) F_m\left( x,y \right)}e^{i\beta Z},
%  \label{equ:E}
%\end{equation}
%where $Z$ is the transmission distance, $m$ is the waveguide indicator, $\beta$ is the propagation constant, and $\tilde{A}$ is the complex amplitude of the optical field. Substituting Eq.~(\ref{equ:E}) into Eq.~(\ref{equ:Hel}), since $\tilde{A}$ is a slow-varying function, $\partial ^2\tilde{A}/\partial Z^2$ can be neglected, we can then obtain
%\begin{equation} 
%  \sum_{m=1}^M{i\frac{\partial \tilde{A}_m}{\partial Z}F_m+\left[ \frac{\left( \tilde{n}^2-n_{m}^{2} \right) k_{0}^{2}+\bar{\beta}_{m}^{2}-\beta ^2}{2\beta} \right] \tilde{A}_mF_m}=0,
%  \label{equ:A1}
%\end{equation}
%where $\bar{\beta}_{m}$ is the mode propagation constant.
%
%Assuming that $F_m$ has been normalized and the overlap integral between different waveguides is zero, i.e., $\iint_{-\infty}^{+\infty}{F_{n}^{*}\left( x,y \right) F_m\left( x,y \right) dxdy}=\delta _{m,n}$, the coupled-mode equations can be obtained
%\begin{equation} 
%  \frac{\partial \tilde{A}_m}{\partial Z}=i\left( \bar{\beta}_m+\Delta \beta _{m}^{NL}-\beta \right) \tilde{A}_m+i\sum_{n\ne m}^M{\mathcal{K}_{mn}\tilde{A}_n},
%  \label{equ:A2}
%\end{equation}
%the coupling coefficient $\mathcal{K}$ and the nonlinear interaction $\Delta \beta _{m}^{NL}$ are defined as
%	\begin{align}
%    \mathcal{K}_{mn}=\frac{k_{0}^{2}}{2\beta}\iint_{-\infty}^{+\infty}{\left( \tilde{n}^2-n_{n}^{2} \right) F_{m}^{*}F_ndxdy},
%  \label{equ:K}\\
%  \Delta \beta _{m}^{NL}=\frac{k_{0}^{2}}{2\beta}\iint_{-\infty}^{+\infty}{\left( \tilde{n}^2-n_{m}^{2} \right) F_{m}^{*}F_mdxdy}.
%  \label{equ:Beta}
%	\end{align}
%
%Expanding $\bar{\beta}_m\left( \omega \right)$ to second order near the center frequency $\omega_0$ of the quasi-monochromatic optical field and neglecting the frequency dependence of $\mathcal{K}$, the time domain form of Eq.~(\ref{equ:A2}) is
%\begin{equation}
%  \begin{aligned}
%  & \frac{\partial A_m}{\partial Z}+\beta_{1 m} \frac{\partial A_m}{\partial t}+\frac{i}{2} \beta_{2 m} \frac{\partial^2 A_m}{\partial t^2} \\
%  & =i\sum_{n\ne m}^M{\mathcal{K} _{mn}A_n}+i\left( \delta _m+k_0\iint_{-\infty}^{+\infty}{\Delta n_m\left| F_m \right|^2dxdy} \right) A_m
%  \label{equ:A3}
%  \end{aligned}
%\end{equation}
%where $\delta _m=\beta _{0m}-\beta$ is the amount of asymmetry between the waveguides, and is set to $\delta _m=0$, $\beta _{1m}$ is the reciprocal of the group velocity, and $\beta _{2m}$ is the group velocity dispersion of the waveguides. $\Delta n_m$ is the amount of refractive index change due to third-order nonlinearity $\chi ^{\left( 3 \right)}$. Assuming that the phase matching condition is not satisfied, the nonlinear effect does not produce new frequencies and $\Delta n_m$ can be expressed as
%\begin{equation} 
%  \Delta n_m=
%  n_2 \left( \left| E_m \right|^2+2\sum_{n\ne m}^M{\left| E_n \right|^2} \right), 
%  \label{equ:Dn}
%\end{equation}
%here, $n_2=3\text{Re}\left( \chi ^{\left( 3 \right)} \right)/{8n} $ is the nonlinear refractive index coefficient. Combining Eqs.~(\ref{equ:A3}-\ref{equ:Dn}), the coupled-mode equations for symmetric couplers can be obtained
%
%\begin{equation}
%  \begin{aligned}
%  & \frac{\partial A_m}{\partial z}+\beta _{1m}\frac{\partial A_m}{\partial t}+\frac{i}{2}\beta _{2m}\frac{\partial ^2A_m}{\partial t^2} \\
%  & =i\sum_{n\ne m}^M{\mathcal{K} _{mn}A_n}+i\left( \mathcal{X} \left| A_m \right|^2+\sum_{n\ne m}^M{C_{mn}\left| A_n \right|^2} \right) A_m
%  \label{equ:DN}
%  \end{aligned}
%\end{equation}
%where $\mathcal{X} _m$ is the self-phase modulation (SPM) parameter and $C_{mn}$ is the cross-phase modulation (XPM) parameter, respectively, defined as
%
%\begin{align}
%  \mathcal{X}=n_2k_0\iint_{-\infty}^{\infty}{\left| F_m \right|^4dxdy},
%\label{equ:spm}\\
%C_{mn}=2n_2k_0\iint_{-\infty}^{\infty}{\left| F_m \right|^2\left| F_n \right|^2dxdy}.
%\label{equ:xpm}
%\end{align}
%When a continuous beam is injected from one end, the time derivative term of Eq.~(\ref{equ:DN}) is negligible. $C_{mn}$ leads to a small value due to the overlap integral involving different waveguides, which is usually neglected. At this point, Eq.~(\ref{equ:DN}) is reduced to
%\begin{equation} 
%  i\frac{dA_m}{dZ}+\mathcal{K}\sum_{n\ne m}^{\mathcal{M}}{A_{n}}+\mathcal{X}\left| A_m \right|^2A_m=0
%  \label{equ:DN2}
%\end{equation}
%
%It has the form of the DNSE. We introduce dimensionless parameters $\psi _m=A_m/\rho$ and $z=\xi Z$, the normalized equation is obtained
%\begin{equation} 
%  i\frac{d\psi _{m}}{dz}+\kappa \sum_{n}^{\mathcal{N}_{m}}{\psi _{n}}+\chi \left| \psi _{m} \right|^2\psi _{m}=0
%  \label{equ:DN3}
%\end{equation}
%with $\kappa =\mathcal{K}/\xi $, $\chi =\rho ^2\mathcal{X}/\xi $ the corresponding dimensionless parameters.


\section{Evolution equations for $\Delta P$ and $\Delta T$} \label{C}
\setcounter{figure}{0} 
Using the theory of irreversible thermodynamics, we can make connection between the driving affinities $\varDelta \mu$, $\varDelta T$ with the corresponding power ($I_P$) and entropy ($I_S$) fluxes, which follow the Onsager symmetry
\begin{equation} 
\left( \begin{array}{c}
	I_P\\
	I_S\\
\end{array} \right) =\left( \begin{matrix}
	K_0&		K_1/T\\
	K_1/T&		K_2/T^2\\
\end{matrix} \right) \left( \begin{array}{c}
	\Delta \mu\\
	\Delta T\\
\end{array} \right). 
\label{equ:1}
\end{equation}
Positive direction of the current is chosen as from $L$ to $R$ reservoir.
The matrix elements $K_0$, $K_1$ and $K_2$ depend on the state of the system and can be expressed in terms of the power conductance $G$, the thermal conductance $G_T$, the Seebeck coefficient $\alpha _{c}$ and the Lorenz number $L=G_T/TG$,
\begin{equation} 
  \frac{\partial}{\partial z}\left( \begin{array}{c}
    \Delta P\\
    \Delta S\\
  \end{array} \right) =-2G\left( \begin{matrix}
    1&		\alpha _{{c}}\\
    \alpha _{{c}}&		L+\alpha _{{c}}^{2}\\
  \end{matrix} \right) \left( \begin{array}{c}
    \Delta \mu\\
    \Delta T\\
  \end{array} \right) 
  \label{equ:2}
\end{equation}
%
%These transport coefficients characterize the process by which the two reservoirs exchange optical power and energy through the chain to relax the system to an equilibrium state.
%Where the transport process occurs mainly in the chain, and the reservoirs as macroscopic systems can be described by thermodynamic methods, the transport coefficients should not depend on the size of the reservoirs, i.e., the structure should be convergent.
%The convergence tests of the symmetric junctions used in the calculations of this work are shown in Appendix~\ref{B}.
%
%Since the optical power proportion in the chain at equilibrium is $P_c/P_s<M_c/M_s$, where $M_s$ and $M_c$ are the waveguide numbers of the system and the chain, respectively. So we always keep $M_c/M_s<0.01$ in all our simulations to ignore the effect of the optical power in the chain.
%
%We now describe the evolution of the optical power and internal energy of the two reservoirs caused by the temperature and chemical potential differences. 

Due to the weak link between the two reservoirs, bottleneck of the thermalization process is at the central chain. 
For the two reservoirs, the thermalization can be considered as a quasi-static process, where the two reservoirs can reach local equilibrium much faster than the whole junction. This can be verified by tracking the evolution of the system, and  
allows us to use the thermodynamic equation of the reservoirs.
Using the Maxwell's relations $\left. \left( \partial S/\partial \mu \right) \right|_T=\left. \left( \partial P/\partial T \right) \right|_{\mu}$ and $-\left. \left( \partial S/\partial P \right) \right|_T=\left. \left( \partial \mu /\partial T \right) \right|_P$, we can obtain
\begin{eqnarray}
  dP=\left. \frac{\partial P}{\partial \mu} \right|_Td\mu +\left. \frac{\partial P}{\partial T} \right|_{\mu}dT=\kappa d\mu +\alpha _r\kappa dT,
  \label{equ:3}
  \\
  dS=\left. \frac{\partial S}{\partial P} \right|_TdP+\left. \frac{\partial S}{\partial T} \right|_PdT=\alpha _rdP+\kappa ldT
  \label{equ:4}
\end{eqnarray}
where $\kappa =\left. \left( \partial P/\partial \mu \right) \right|_T$, $\alpha _r=-\left. \left( \partial \mu /\partial T \right) \right|_P$, $l=C_P/T\kappa $ are the compression coefficient, Seebeck coefficient and Lorenz number of the reservoir, respectively,
with $C_P=\left. T\left( \partial S/\partial T \right) \right|_P$ the reservoir heat capacity at constant power.  
In the linear response regime, they can be obtained from the equilibrium thermodynamic state at $\bar{T}$ and $\bar{\mu}$ as 
\begin{equation} 
  \kappa =\left. \sum_i^M{\frac{T}{\left( \varepsilon _i-\mu \right) ^2}} \right|_T,
  \label{equ:kappa}
\end{equation}
\begin{equation} 
  \alpha _r=\frac{1}{\kappa}\left. \sum_{i=1}^M{\frac{1}{\varepsilon _i-\mu}} \right|_{\mu},
  \label{equ:alpha}
\end{equation}
\begin{equation} 
  C_P=M-T\alpha _{r}^{2}\kappa.
  \label{equ:CP}
\end{equation}

%With Eqs.~(\ref{equ:2}-\ref{equ:4}), we can obtain the evolution equations of $\Delta P$ and $\Delta T$. 
%% The general case of different types of reservoirs is discussed in the appendix.
%For symmetric junctions with identical left and right reservoirs, 
%they read
%\begin{equation} 
%  \tau _0\frac{\partial}{\partial z}\left( \begin{array}{c}
%    \Delta P\\
%    \Delta T\\
%  \end{array} \right) =-\left( \begin{matrix}
%    1&		-\kappa \alpha\\
%    -\frac{\alpha}{\ell \kappa}&		\frac{L+\alpha ^2}{\ell}\\
%  \end{matrix} \right) \left( \begin{array}{c}
%    \Delta P\\
%    \Delta T\\
%  \end{array} \right) 
%  \label{equ:5}
%\end{equation}
%where we have defined a transport timescales $\tau _0=\kappa /\left( 2G \right)$ and the Seebeck coefficient $\alpha =\alpha _r-\alpha _c$. 
%\revision{This form of the transport matrix highlights that the optical power and temperature dynamics result from a competition between the transport properties of the constriction and the thermodynamical properties of the reservoirs.}
%%
%

According to the thermodynamic equations, Eqs.~(\ref{equ:3}-\ref{equ:4}), the optical power $P(z)$ and entropy $S(z)$ of the reservoir at position $z$ can be written as
\begin{align} 
  P\left( z \right) &=\kappa \mu \left( z \right) +\alpha _r\kappa T\left( z \right) +A,
  \label{AP_B_1}\\
  S\left( z \right) &=\alpha _rP\left( z \right) +\kappa lT\left( z \right) +B
  \label{AP_B_2}
\end{align}
where $A$ and $B$ are constants related to the initial conditions. We have dropped the reservoir index. When the left and right reservoirs are identical, their thermodynamic parameters are also identical, $A_L=A_R$ and $B_L=B_R$.
Then, the optical power ($\Delta P\left( z \right) $) and entropy ($\Delta S\left( z \right) $) difference can be written as
\begin{align} 
  \Delta P\left( z \right)&=\kappa \Delta \mu \left( z \right) +\alpha _r\kappa \Delta T\left( z \right), 
  \label{AP_B_3}\\
  \Delta S\left( z \right)&=\alpha _r\Delta P\left( z \right)+\kappa l\Delta T\left( z \right).
  \label{AP_B_4}
\end{align}
Using $\Delta P$ and $\Delta T$ to eliminate $\Delta S$ and $\Delta \mu$ in Eq.~(\ref{equ:2}), the evolution equations are obtained
\begin{equation} 
  \tau _0\frac{\partial}{\partial z}\left( \begin{array}{c}
    \Delta P\\
    \Delta T\\
  \end{array} \right) =-\left( \begin{matrix}
    1&		-\kappa \alpha\\
    -\frac{\alpha}{\kappa l}&		\frac{\alpha ^2+L}{l}\\
  \end{matrix} \right) \left( \begin{array}{c}
    \Delta P\\
    \Delta T\\
  \end{array} \right) \equiv -\varLambda \left( \begin{array}{c}
    \Delta P\\
    \Delta T\\
  \end{array} \right) ,
  \label{AP_B_5}
\end{equation}
where we have defined a transport timescale $\tau _0=\kappa /\left( 2G \right)$ and an effective Seebeck coefficient $\alpha =\alpha _r-\alpha _c$. 
The eigen values of matrix $\varLambda $ are 
\begin{equation} 
  \lambda _{\pm}=\frac{1}{2}\left( 1+\frac{\alpha ^2+L}{l} \right) \pm \sqrt{\left( \frac{1}{2}+\frac{\alpha ^2+L}{2l} \right) ^2-\frac{L}{l}}=\delta \pm \varOmega .
  \label{AP_B_6}
\end{equation}
General solutions of Eq.~(\ref{AP_B_5}) can be expressed using the eigen values
\begin{align} 
  \Delta P\left( z \right) &=C_1e^{-\frac{\lambda _+}{\tau _0}z}+C_2e^{-\frac{\lambda _-}{\tau _0}z},
  \label{AP_B_7}\\
  \Delta T\left( z \right) &=C_3e^{-\frac{\lambda _+}{\tau _0}z}+C_4e^{-\frac{\lambda _-}{\tau _0}z},
  \label{AP_B_8}
\end{align}
where the coefficients $C_1$, $C_2$, $C_3$ and $C_4$ are determined by the initial conditions.
Assuming that the initial power and temperature difference are $\Delta P_0$ and $\Delta T_0$, from Eqs.~(\ref{AP_B_5}),  (\ref{AP_B_7}) and  (\ref{AP_B_8}) we get
\begin{align} 
  C_1&=\frac{1+\Omega -\delta}{2\Omega}\Delta P_0-\frac{\kappa \alpha}{2\Omega}\Delta T_0,
  \label{AP_B_9}\\
  C_2&=\frac{\Omega +\delta -1}{2\Omega}\Delta P_0+\frac{\kappa \alpha}{2\Omega}\Delta T_0,
  \label{AP_B_10}\\
  C_3&=-\frac{\alpha}{2\kappa l\Omega}\Delta P_0+\frac{\delta +\Omega -1}{2\Omega}\Delta T_0,
  \label{AP_B_11}\\
  C_4&=\frac{\alpha}{2\kappa l\Omega}\Delta P_0+\frac{\Omega -\delta +1}{2\Omega}\Delta T_0.
  \label{AP_B_12}
\end{align}
Substituting into Eqs.~(\ref{AP_B_7}) and  (\ref{AP_B_8}), we can obtain the evolution equations for $\Delta P$ and $\Delta T$.

\subsection{Numerical simulation}
\setcounter{figure}{0} 
We choose the initial condition of each reservoir by fixing its optical temperature $T_\alpha$ and chemical potential $\mu_\alpha$.  
%We choose the optical temperature $T$ and the chemical potential $\mu$ as adjustable parameters in the initial input.
The corresponding optical power and internal energy can then be obtained. We perform calculations in the linear response regime with $T_L-T_R=\Delta T\ll \bar{T}$, $\mu _L-\mu _R=\Delta \mu \ll \bar{\mu}$. When the left and right reservoirs are identical, the average temperature and chemical potential are $\bar{T}=\left( T_L+T_R \right) /2$ and $\bar{\mu}=\left( \mu _L+\mu _R \right) /2$, respectively. 
%In addition, the initial input in the main text also satisfies $P_L=P_R$ to facilitate the observation of the coupled energy and particle transport.
Figure~\ref{fig:1b} shows seven sets of parameters at fixed chemical potential, all selected on iso-power lines while ensuring the small temperature and chemical potential difference.
The fourth-order Runge-Kutta method was used to solve Eq.~(\ref{equ:DNES}).  
The final result for each given initial condition is ensemble averaged by choosing different random phases for $c_{i}$. 
%For each given initial condition, with different initial conditions, i.e. random phases of $c_{i0}^S$ in the reservoirs. The final result is obtained by performing the ensemble average.
%
%
%and by averaging the evolutionary results of a large number of system integrations for reservoirs with different random phases $\text{arg}\left( c_{i0}^{S} \right)$ in the initial stage to obtaining the $\varPsi \left( z \right) $, where $S\in \left\{ L,R \right\} $.
%For the structure shown in Fig.~\ref{fig:1a}, we trace the modal occupancies evolution of each reservoir through $\left| c_{i}^{S} \right|^2\left( z \right) =\left| \left< \varphi _{i}^{S}\mid \varPsi ^S \right> \right|^2\left( z \right)$, where $\left| \varphi _{i}^{S} \right> $ is the eigenvector of the reservoir and $\varPsi ^S$ is the part of $\varPsi$ that describes the $S$ reservoir.
%When the system is in equilibrium, they will have the same temperature and chemical potential as the system (see the yellow dot in Fig.~\ref{fig:1b}), although they may have different power and internal energy due to differences in reservoir structure.
%
Fig.~\ref{fig:2} shows the calculated results for a symmetric junction with square-lattice reservoirs at $\bar{T}=-1.065$ and $\bar{\mu}=7$.
At the initial stage, driven by temperature and chemical potential deviations, the optical power is rapidly transferred from the cold reservoir to the hot reservoir. After $\Delta P$ reaches saturation, the system is brought to equilibrium by a slow relaxation process.





\subsection{Data Fitting} \label{D}
% Figure environment removed

% Figure environment removed
We use Eqs.~(\ref{equ:6}) and (\ref{equ:7}) to fit the evolution of $\varDelta P$ and $\varDelta T$ simultaneously by scanning the chain parameter space so that the residual $R$ is minimized, which is defined as\cite{hausler_interaction-assisted_2021}
\begin{equation} 
  R=\sum_i{\left[ \left( \frac{\Delta P\left( z_i \right) -\Delta P_i}{\bar{P}} \right) ^2+\left( \frac{\Delta T\left( z_i \right) -\Delta T_i}{\bar{T}} \right) ^2 \right]}.
  \label{equ:RES}
\end{equation}
Here, $\Delta P\left( z_i \right)$ and $\Delta T\left( z_i \right)$ are obtained from Eqs.~(\ref{equ:6}) and (\ref{equ:7}) at position $z_i$, $\Delta P_i$ and $\Delta T_i$ are the numerically calculated results from Eq.~(\ref{equ:DNES}) at $z_i$, and $\bar{P}$ and $\bar{T}$ are averages of $\Delta P_i$ and $\Delta T_i$.
%
Since the calculation is in the linear response regime, our results are only related to the equilibrium temperature and chemical potential of the system, independent of the initial biases $\Delta P_0$ and $\Delta T_0$. This is checked in Fig.~\ref{fig:test} where evolution from two sets of different initial conditions is fitted by the same set of transport coefficients. 
%Therefore, while ensuring the same average temperature and chemical potential, \revision{we set: (1). the temperature-driven case with $\Delta \mu _0=0$; (2). the chemical-potential-driven case with $\Delta T_0=0$, and the initial deviation and the fitting parameters of Fig.~\ref{fig:2} are substituted into Eqs.~(\ref{equ:6}) and (\ref{equ:7}) to verify the correctness of the results.
%The results in Fig.~\ref{fig:test} show that the evolution curve fits well with the calculated results, indicating the correctness of the fitting method.}


\subsection{Convergence tests} \label{B}
We performed convergence tests for symmetric junctions formed by square and hexagonal honeycomb lattices, respectively, to ensure that the fitted transport coefficients do not depend on the size of the reservoirs. 
The results are summarized in Fig.~(\ref{fig:test}).
%We \revision{choose $\left( -1.06,7 \right) $ as the representative point} in the parameter space shown in Fig.~(\ref{fig:1b}) for the calculation, with nonlinear parameter $\chi =0.2$.
%The results are shown in Fig.~\ref{fig:test_S} and \ref{fig:test_H}. 
Parameters used in the main text are enclosed by dotted boxes.


\section{Landauer transport theory}
\label{sec:lt}
\setcounter{figure}{0} 
The DNSE corresponds to an effective tight-binding Hamiltonian for the junction\cite{wu_thermodynamic_2019,rasmussen_statistical_2000}
%Total Hamiltonian of the system contains a linear part $H_L$ and a nonlinear part $H_{NL}$, written as\cite{wu_thermodynamic_2019,rasmussen_statistical_2000}
\begin{equation} 
  H\left( \psi _m,i\psi _{m}^{*} \right) =\sum_{\{m,n\}}{\kappa \psi _{m}^{*}\psi _n}+\frac{1}{2}\chi \sum_m {\left| \psi _m \right|^4}.
      \label{equ:H}
\end{equation}
Here, $\psi _{m}$ and $i\psi _{m}^{*}$ form the canonical conjugate pairs,  $\{m,n\}$ are nearest neighbour pairs. The first and the second term corresponds to the linear and nonlinear part of the Hamiltonian.  

% Figure environment removed



We can also analyze the transport coefficients of the junction in the grand canonical ensemble using the Landauer formalism. For this purpose, 
we introduce two fictitious baths coupling to the boundary sites of the two reservoirs, respectively. 
The linear transport coefficients $G$, $\alpha _c$ and $G_T$ of the junction are given by the following integrals

\begin{equation}
  K_n=\int_{\varepsilon _{\min}}^{\varepsilon _{\max}}{\frac{d\varepsilon}{2\pi}\left( \varepsilon -\mu \right) ^n\mT \left( \varepsilon \right) \left( -\frac{\partial f}{\partial \varepsilon} \right)}
  \label{equ:13}
\end{equation}
where $f=T/\left( \varepsilon-\mu \right)$ is the R-J distribution function and $\mT \left( \varepsilon \right)$ is the energy-dependent transmission coefficient, a central quantity in the Landauer formalism.
It is given by the Caroli formula from the nonequilibrium Green's function (NEGF) method as
\begin{equation} 
  \mT \left( \varepsilon \right) =\text{Tr}\left[ G_{d}^{\dag}\left( \varepsilon \right) \Gamma _L\left( \varepsilon \right) G_d\left( \varepsilon \right) \Gamma _R\left( \varepsilon \right) \right] 
  \label{equ:14}
\end{equation}
where $G_d=1/\left[ \left( E+i\eta \right) I-H-\Sigma _L-\Sigma _R \right] $ and $G_d^\dagger$ are the retarded and advanced Green's functions of the system, respectively.
$\eta$ is a small positive quantity,
$\Sigma _\alpha \left( \varepsilon \right)$ is the retarded self-energy due to coupling to bath $\alpha$, and $\Gamma _\alpha\left( \varepsilon \right) =-2\text{Im}\left[ \Sigma _\alpha\left( \varepsilon \right) \right]$ is the corresponding broadening function.
Under the wide band approximation, $\Sigma _\alpha$ can be regarded as an energy-independent constant, $\Sigma _\alpha=-i\gamma $, and $\gamma$ is a real parameter, which we choose as $\gamma = 0.1$.
The power and thermal current are written respectively as
\begin{align}
\label{eq:jp}
    J_P &= \int_0^{+\infty}d\varepsilon \mathcal{T}(\varepsilon) (f_L(\varepsilon)-f_R(\varepsilon)),\\
    J_Q &= \int_0^{+\infty}d\varepsilon (\varepsilon-\mu) \mathcal{T}(\varepsilon) (f_L(\varepsilon)-f_R(\varepsilon)).
\label{eq:jq}
\end{align}
The density of states (DOS) of the system is obtained by 
$\rho \left( \varepsilon \right) =i\text{Tr}\left( G_d-G_{d}^{\dag} \right) /M$.
Figure~\ref{fig:3t} shows the transmission spectra and DOS of the two symmetric junctions, from which the transport coefficients of the systems can be calculated and analyzed.
 

\documentclass[a4paper,11pt]{article}
\pdfoutput=1 % if your are submitting a pdflatex (i.e. if you have
             % images in pdf, png or jpg format)

%\usepackage[utf8]{inputenc}
%\usepackage{mathrsfs, amssymb, amsmath}  
%\usepackage{comment}
%\usepackage{dcolumn}
%\usepackage{multirow}
%\usepackage{color}
%\usepackage{amsfonts,amssymb,amsmath, txfonts}
%\usepackage{float}

\usepackage{jcappub} % for details on the use of the package, please
                     % see the JCAP-author-manual

\usepackage[T1]{fontenc} % if needed

\hypersetup{ linktoc=all,
    colorlinks=true, linkcolor={blue},  
       citecolor={red}, urlcolor={darkred}
}
\definecolor{Redgreen}{RGB}{153,76,0}
\definecolor{vividviolet}{rgb}{0.62, 0.0, 1.0}
\definecolor{green}{RGB}{11,98,17}
\definecolor{darkgreen}{RGB}{40,150,65}
\definecolor{darkblue}{rgb}{0,0,0.3}
\definecolor{darkred}{rgb}{0.7,0,0}

\def\blue{\textcolor{blue}}
\def\red{\textcolor{red}}
\def\be{\begin{equation}}
\def\ee{\end{equation}}
\def\bea{\begin{eqnarray}}
\def\eea{\end{eqnarray}}


\title{MCMC Marginalisation Bias and $\Lambda$CDM tensions}
%\title{Overcoming bias in MCMC marginalisation to elucidate $\Lambda$CDM tensions}
%\title{Temp}

%%Markov Chain Monte Carlo

%% %simple case: 2 authors, same institution
%% \author{A. Uthor}
%% \author{and A. Nother Author}
%% \affiliation{Institution,\\Address, Country}

% more complex case: 4 authors, 3 institutions, 2 
\author[a]{Eoin \'O Colg\'ain}
\author[b]{Saeed Pourojaghi}
\author[b, c]{M. M. Sheikh-Jabbari}
\author[a]{Darragh Sherwin}

% The "\note" macro will give a warning: "Ignoring empty anchor..."
% you can safely ignore it.

\affiliation[a]{Atlantic Technological University, Ash Lane, Sligo, Ireland}
\affiliation[b]{School of Physics, Institute for Research in Fundamental Sciences (IPM), P.O.Box 19395-5531, Tehran, Iran}
\affiliation[c]{The Abdus Salam ICTP, Strada Costiera 11, I-34014 Trieste, Italy}

% e-mail addresses: one for each author, in the same order as the authors
\emailAdd{eoin.ocolgain@atu.ie}
\emailAdd{pourojaghi@ipm.ir}
\emailAdd{jabbari@theory.ipm.ac.ir}
\emailAdd{darragh.sherwin@research.atu.ie}




\abstract{Probability distributions become non-Gaussian when the flat $\Lambda$CDM model is fitted to redshift binned data in the late Universe. We explain mathematically why this non-Gaussianity arises and confirm that Markov Chain Monte Carlo (MCMC) marginalisation leads to biased inferences in observational Hubble data (OHD). In particular, in high redshift bins we find that $\chi^2$ minima, as identified from both least squares fitting and the MCMC chain, fall outside of the $1 \sigma$ confidence intervals. We resort to profile distributions to correct this bias. Doing so, we observe that $z \gtrsim 1$ cosmic chronometer (CC) data currently prefers a non-evolving (constant) Hubble parameter over a Planck-$\Lambda$CDM cosmology at $\sim 2 \sigma$. We confirm that both mock simulations and profile distributions agree on this significance. Moreover, on the assumption that the Planck-$\Lambda$CDM cosmological model is correct, using profile distributions we confirm  a $> 2 \sigma$ discrepancy with Planck-$\Lambda$CDM in a combination of  CC and baryon acoustic oscillations (BAO) data beyond $ z \sim 1.5$ that was noted earlier through comparison of least square fits of observed and mock data.}



\begin{document}
\maketitle
\flushbottom

\section{Introduction}
\label{sec:intro}
The flat $\Lambda$CDM model is the minimal model that fits Cosmic Microwave Background (CMB) data. Remarkably, CMB data from the Planck satellite \cite{Planck:2018vyg} constrains the $\Lambda$CDM model to sub-percent errors, thereby not only providing the strongest constraints, but also a concrete prediction for cosmological probes in the late Universe. The unmitigated success of the $\Lambda$CDM model is that CMB, Type Ia supernovae (SN) \cite{Riess:1998cb, Perlmutter:1998np} and baryon acoustic oscillations (BAO) \cite{Eisenstein:2005su} agree on a $\Lambda$CDM Universe that is approximately $30 \%$ matter. Thus, one key prediction of the Planck-$\Lambda$CDM model agrees across early and late Universe cosmological probes. Given this non-trivial agreement, any discrepancies that arise elsewhere constitute challenging puzzles. 

Nevertheless, one cannot define any \textit{model} for a dynamical system, especially a complicated system like the Universe, using data from a cosmic snapshot.\footnote{Here, we mean CMB data with an effective redshift $z \sim 1100$.} At best, one has a \textit{prediction} and not a model. In recent years, key predictions of Planck data have been challenged by late Universe determinations of the Hubble constant $H_0$ \cite{Riess:2021jrx, Freedman:2021ahq, Pesce:2020xfe, Blakeslee:2021rqi, Kourkchi:2020iyz} and the $S_8:= \sigma_8 \sqrt{\Omega_m/0.3}$ parameter \cite{HSC:2018mrq, KiDS:2020suj, DES:2021wwk, Boruah:2019icj, Said:2020epb}. Given the diversity of the late Universe probes (see reviews \cite{Perivolaropoulos:2021jda, Abdalla:2022yfr}), it is highly unlikely that any single systematic can be found to explain the discrepancies. That being said, in astrophysics one can never preclude systematics; 3 decades after Phillips' seminal paper \cite{Phillips:1993ng}, we are still debating an ad hoc correction for the mass of the host galaxy in Type Ia SN \cite{NearbySupernovaFactory:2018qkd, Kang:2019azh, Brout:2020msh, Lee:2021txi}. Bearing in mind that Type Ia SN are one of our best understood cosmological probes, one quickly understands that any systematics debate may be endless. 

Thus, it is far more expedient to assume that the $\Lambda$CDM model is breaking down and to look for tell-tale signatures of model breakdown. If signatures cannot be found, one arrives at a contradiction, and revisits the assumption that the model is breaking down. For physicists, \textit{model breakdown comes about when model fitting parameters return discrepant values at different time slices or epochs}. Translated into astronomy, this equates to discrepant cosmological parameters in different redshift ranges. The usual $H_0, S_8$ tensions  may also be viewed in the same light: a discrepancy between high and low redshift inferences/measurements of the parameters \cite{Perivolaropoulos:2021jda, Abdalla:2022yfr}. Nevertheless, early and late Universe observables are typically not the same, so one is confronted with a rich set of potential systematics. 

Within the context of $\Lambda$CDM tensions, it was recently observed that the integration constant from the Friedmann equations, aka the Hubble constant $H_0$, picks up redshift dependence whenever our model assumption - required to close the Friedmann equations - disagrees with the Hubble parameter $H(z)$ extracted from observations \cite{Krishnan:2020vaf, Krishnan:2022fzz}. %\footnote{One is free to speculate about the nature of the missing physics \cite{Liao:2020zko, Montani:2023xpd}.} 
Similarly, $\rho_{m0}=H_0^2\Omega_m$, an integration constant of the matter continuity equation, implies matter density $\Omega_m$ is a mathematically constant quantity. 
These are irrefutable predictions from mathematics, i. e. a prediction that is \textit{robust to systematics}. However, observationally $H_0$ and $\Omega_{m}$ are model fitting parameters and nothing precludes them picking up redshift dependence (except of course if one assumes they do not!), and providing a signature of model breakdown. If this happens in the late Universe within the $\Lambda$CDM model, $H_0$ is correlated with matter density $\Omega_m$, 
while $\Omega_m$ is correlated with $S_8 \propto \sigma_8 \sqrt{\Omega_m}$. Thus, there is at least one simple scenario, namely redshift evolution of cosmological parameters in the late Universe, where ``$H_0$ tension'' and ``$S_8$ tension'' are not independent and simply symptoms of $\Lambda$CDM model breakdown. 

The next relevant question is, where is the evidence for evolving cosmological parameters in the late Universe? Starting with strong lensing time delay \cite{Wong:2019kwg, Millon:2019slk},\footnote{Systematics are explored in \cite{Millon:2019slk} and the descending trend is not an obvious systematic. The lensed system RXJ1131-1231 \cite{Sluse:2003iy}, which partly drives the trend, has recently been re-analysed using spatially resolved stellar kinematics of the host galaxy \cite{Shajib:2023uig}, and the higher $H_0$ value remains robust, admittedly with inflated errors. As TDCOSMO project to analyse 40 lenses, the prospect of a discovery of a descending $H_0$ trend assuming the $\Lambda$CDM model remain strong.} descending trends of $H_0$ with redshift have been reported in Type Ia SN \cite{Dainotti:2021pqg, Colgain:2022nlb, Colgain:2022rxy,  Malekjani:2023dky, Hu:2022kes, Jia:2022ycc} and combinations of data sets \cite{Krishnan:2020obg, Dainotti:2022bzg}. On the other hand, larger values of $\Omega_m$ have been noted in high redshift observables, primarily quasars (QSOs) \cite{Risaliti:2015zla, Risaliti:2018reu, Lusso:2020pdb, Yang:2019vgk, Khadka:2020vlh, Khadka:2020tlm, Khadka:2021xcc, Pourojaghi:2022zrh},\footnote{Just as with Type Ia SN, the systematics of QSOs are being investigated \cite{Zajacek:2023qjm}.} but also Type Ia SN \cite{Colgain:2022nlb, Colgain:2022rxy, Malekjani:2023dky, Pasten:2023rpc} (see also \cite{Wagner:2022etu, Sakr:2023hrl}). Note, as emphasised earlier, if $H_0$ evolves at the background level, correlated fitting parameters are expected to also evolve. Moreover, mock analysis within the $\Lambda$CDM setting reveals that evolution of best fit $(H_0, \Omega_m)$ parameters cannot be precluded, and conversely possesses a finite likelihood, in either observational Hubble data (OHD) \textit{or} angular diameter distance data \textit{or} luminosity distance data \cite{Colgain:2022tql}. We stress that this result \textit{rests on mock analysis}; it represents a purely mathematical statement about the $\Lambda$CDM model that is independent of systematics. 

Separately, at the perturbative level, redshift evolution of $S_8$ or $\sigma_8$ has been reported in galaxy cluster number counts and Lyman-$\alpha$ spectra \cite{Esposito:2022plo}, $f \sigma_8$ constraints from peculiar velocities and redshift space distortions (RSD) 
 \cite{Adil:2023jtu}, comparison between weak \cite{HSC:2018mrq, KiDS:2020suj, DES:2021wwk} and CMB lensing \cite{ACT:2023dou, ACT:2023kun}. What is important here is that these observations appear to restrict the evolution in $S_8$ to the late Universe. In \cite{ACT:2023ipp} the possibility was raised that \textit{``tracers at higher redshift and probing larger scales prefer higher $S_8$''}.\footnote{There are also conflicting observations of high redshift $\sigma_8$ or $S_8$ values that are lower than Planck in the late Universe \cite{Miyatake:2021qjr, Alonso:2023guh}, so either this trend is not universal, or systematics are at play.} Nevertheless, one can argue against evolution with scale on the grounds that cosmic shear \cite{HSC:2018mrq, KiDS:2020suj, DES:2021wwk}, which is sensitive to smaller scales (larger $k$), and peculiar velocity constraints \cite{Boruah:2019icj, Said:2020epb}, which are sensitive to larger scales (smaller $k$), both prefer lower values of $S_8$. Moreover, both galaxy clusters and Lyman-$\alpha$ spectra are expected to probe similar scales.\footnote{We thank Matteo Viel for correspondence on this point.} Thus, if systematics are not impacting results, then redshift evolution is the only point of agreement in the observations \cite{Esposito:2022plo, Adil:2023jtu, HSC:2018mrq, KiDS:2020suj, DES:2021wwk, ACT:2023dou, ACT:2023kun, ACT:2023ipp}. Note also that redshift is more fundamental than scale in FLRW cosmology; one must solve the Friedmann equations in either time or redshift before one contemplates any discussion of scale.  

 The purpose of this letter is to revisit the analysis presented in \cite{Colgain:2022rxy,Colgain:2022tql}, where the evidence for evolution was quantified on the basis of mock simulations and not Markov Chain Monte Carlo (MCMC), the technique most familiar in cosmology. The fundamental problem is that once one bins low redshift data and studies evolution of cosmological parameters with bin redshift, one quickly encounters projection effects in MCMC analyses. These effects are not just the preserve of exotic models \cite{Herold:2021ksg, Gomez-Valent:2022hkb, Meiers:2023gft}, such as Early Dark Energy (EDE) \cite{Poulin:2018cxd, Niedermann:2019olb}, and happen in the simplest model when one bins data. The most striking demonstration of the resulting bias is that the peaks of MCMC posteriors no longer coincide with the minimum of the likelihood (see \cite{Gomez-Valent:2022hkb}). Ultimately, this bias is expected  because one is working in a regime of the $\Lambda$CDM model with non-Gaussian probability distributions   \cite{Colgain:2022tql}  (see also \cite{Colgain:2022rxy}).

 The structure of this paper is as follows. In section \ref{sec:MCMC_bias} we confirm the bias in MCMC marginalisation. In section \ref{sec:PD} we introduce profile distributions (PDs) \cite{Gomez-Valent:2022hkb} as a means of addressing the bias and confirm that the statistical significance of discrepancies from mock simulations agree well with PD analysis. In section \ref{sec:tension}, we revisit and confirm the high redshift OHD tensions reported in \cite{Colgain:2022rxy}. We end in section \ref{sec:discussion} with concluding remarks. 
 %A short appendix is also added on Fisher matrix for $\Lambda$CDM mdoel. 

\section{A bias in MCMC marginalisation}
\label{sec:MCMC_bias}
In this section we illustrate a bias in MCMC marginalisation that arises in the (flat) $\Lambda$CDM model when data is binned by redshift. This bias can be traced to a regime of the $\Lambda$CDM model with non-Gaussian distributions and is independent of systematics  \cite{Colgain:2022rxy, Colgain:2022tql}. 

\subsection{Mathematical Foundations}
\label{sec:math}
Consider an exercise where one bins OHD and confronts it to the $\Lambda$CDM Hubble Parameter $H(z)$ in the late Universe, a setting where the radiation sector can be safely decoupled. In high redshift bins ($z \gg 0$) in the matter-dominated regime, the Hubble parameter becomes insensitive to the dark energy (DE) sector: 
\be
\label{eq:lcdm}
H(z) = H_0 \sqrt{1-\Omega_m + \Omega_m (1+z)^3} \xrightarrow[z \gg 0]{} H_0 \sqrt{\Omega_m} (1+z)^{\frac{3}{2}}.  
\ee
More concretely, taking $z \rightarrow \infty$ we see that data can only constrain the combination $\rho_{m0}=H_0^2{\Omega_m}$. For \textit{hypothetical} data in a redshift bin with effective redshift $z = \infty$, this means that one can only constrain the combination $\Omega_m h^2$ ($h:= H_0/100)$, but $H_0$ and $\Omega_m$ remain unconstrained. Alternatively put, for any given $\Omega_m h^2$ constraint, there is an infinite number of corresponding $(H_0, \Omega_m)$ pairs. Translated into a probability density function (PDF), this is simply the statement that in a very high redshift bin at $z = \infty$, one expects uniform or flat distributions for $H_0$ and $\Omega_m$ with the model (\ref{eq:lcdm}).  

Of course, observed data resides at finite $z$ and not $z = \infty$. As a result, one does not encounter \textit{exactly} flat PDFs in $H_0$ and $\Omega_m$ at high redshift, but \textit{almost} flat PDFs. More important to us is the observation that these PDFs must flatten in a non-Gaussian manner. To appreciate this fact, we observe that high redshift OHD only constrains $\Omega_m h^2$ well.\footnote{Note that observables like SN or QSO that measure $D_L(z)=c (1+z)\int_0^z \textrm{d} z'/H(z')$ are mainly sensitive to the low redshift part of $H(z)$, i. e. the combination $H_0^2 (1-\Omega_m)$, and in this sense they are complementary to the OHD data which is more sensitive to high redshift part of $H(z)$, $H_0^2\Omega_m$. The complementarity can be demonstrated by combining $H(z)$ and $D_{L}(z)$ constraints and checking that one recovers mock data input parameters in all redshift bins \cite{Colgain:2022tql}. } For this reason, best fit parameters are constrained to a $\Omega_m h^2 = \textrm{constant}$ curve in the $(H_0, \Omega_m)$-plane. The almost flat $H_0$ and $\Omega_m$ PDFs can only arise if this curve stretches in the $(H_0, \Omega_m)$-plane. As a result of this stretching, one ends up with a relatively uniform distribution on a curve. At the extremes of the curve, one finds a distribution of large $H_0$ values, which do not differ greatly in $\Omega_m$, and they get projected to a peak at small values on the $\Omega_m$ axis. Conversely, at the other end of the curve, one finds a distribution of small $\Omega_m$ values, which do not differ greatly in $H_0$, and they get projected onto a peak at large values on the $H_0$ axis.  This is a ``projection effect'' in common cosmology parlance.  It is driven by the irrelevance of the DE sector at high redshift and the constraint $\Omega_m h^2 = \textrm{constant}$ from the $\Lambda$CDM model (\ref{eq:lcdm}). Together these features distort the distribution away from a Gaussian configuration. 

Thus, simply by binning and fitting OHD to the $\Lambda$CDM model one enters a non-Gaussian regime as the effective redshift of the bin increases. This effect, which is expected from the purely mathematical arguments above, has been confirmed in mock data \cite{Colgain:2022rxy, Colgain:2022tql}, and in line with expectations, we demonstrate that it impacts MCMC inferences with observed data in the next subsection.  

% Figure environment removed

\subsection{Cosmic Chronometer (CC) Data}
\label{sec:CCbias}
Here we work with OHD from the cosmic chronometer (CC) program \cite{Jimenez:2001gg}. Concretely, we work with 34 $H(z)$ constraints spanning the redshift range $0.07 \leq z \leq 1.965$ \cite{Stern:2009ep, Moresco:2012jh, Zhang:2012mp, Moresco:2016mzx, Ratsimbazafy:2017vga, Borghi:2021rft, Jiao:2022aep, Tomasetti:2023kek}. We illustrate the data in Fig.~\ref{fig:CC}, where it is consistent with Fig. 9 of \cite{Tomasetti:2023kek} {modulo the fact that we have an additional data point at $z = 0.8$, which is not independent. See Table 1.1 of \cite{Moresco:2023zys}. While CC data may eventually be good enough to arbitrate on Hubble tension \cite{Moresco:2023zys}, the data is not good enough on its own to do cosmology. To put this comment in context, we observe that the errors in Fig.~\ref{fig:CC} do not include systematic errors (see \cite{Moresco:2020fbm} for an account of the systematics). As a result the constraints we get on cosmological parameters will be underestimated. Thus, from our perspective the data in Fig.~\ref{fig:CC} is simply some representative cosmological data in the OHD class.}

\paragraph{Methodology:} We impose a low redshift cut-off on the OHD $z_{\textrm{min}}$, removing all data points with redshifts $z_i < z_{\textrm{min}}$, and then extremising the $\chi^2$ likelihood, 
\be
\label{eq:chi2}
\chi^2 = Q^{T} \cdot C^{-1} \cdot Q, 
\ee
where $C$ is the covariance matrix, which is simply the square of the $H_i$ errors on the diagonal, and $Q$ is the vector, 
\be
\label{eq:Q}
Q_i = H_i - H_{\textrm{model}}(z_i), 
\ee
where $H_i:=H(z_i)$ denotes OHD and $H_{\textrm{model}}(z)$ is the model (\ref{eq:lcdm}) without the high redshift limit. The best fit $(H_0, \Omega_m)$ parameters correspond to the minumum of the $\chi^2$, while on the assumption of Gaussian errors, we estimate the errors from a Fisher matrix (appendix \ref{sec:fisher}). In parallel, we perform MCMC marginalisation through \textit{emcee} \cite{Foreman-Mackey:2012any}. More concretely, subject to the priors $H_0 \in [0, 200 ]$ and $\Omega_m \in [ 0, 1]$, the latter restricting us to a physical regime, we record $16^{\textrm{th}}$, $50^{\textrm{th}}$ and $84^{\textrm{th}}$ percentiles for MCMC posteriors, as is common practice with Gaussian distributions. Thus, both techniques are tailored to Gaussian posteriors, yet non-Gaussianities will be evident in MCMC posteriors. By comparing the output from these two techniques in Table \ref{tab:LCDM_CC} for different values of $z_{\textrm{min}}$ we observe that error estimates from Fisher matrix and MCMC quickly disagree as $z_{\textrm{min}}$ increases. 

From Table \ref{tab:LCDM_CC}, we see that MCMC inferences lead to non-Gaussian $1 \sigma$ confidence intervals, where in line with the expectations from \cite{Colgain:2022tql}, $H_0$ errors are larger for smaller values, and $\Omega_m$ errors are larger for larger values, respectively. This is expected if the $H_0$ and $\Omega_m$ posteriors are peaked at larger and smaller values, respectively, in line with our earlier mathematical argument. Only for the full data set with $z_{\textrm{min}} = 0$  do we find reasonable agreement between the Fisher matrix and MCMC $1 \sigma$ confidence intervals. As can be seen from the lopsided MCMC confidence intervals, the non-Gaussianity becomes more pronounced with increasing $z_{\textrm{min}}$. Interestingly, beyond $z_{\textrm{min}} = 1$, the minimum of the $\chi^2$ falls outside of the MCMC $1 \sigma$ confidence intervals. Nevertheless, by evaluating the MCMC chains on the $\chi^2$ likelihood (\ref{eq:chi2}), we confirm that the parameters corresponding to the minimum $\chi^2$ value are tracking the best fit. Note, the peak of the MCMC posterior is no longer a measure of goodness of fit and inferences have become biased in a regime of model parameter space where distributions are expected to be inherently non-Gaussian. Our analysis here underscores potential problems with a blind MCMC analysis with the traditional $16^{\textrm{th}}$, $50^{\textrm{th}}$ and $84^{\textrm{th}}$ percentiles.       



\begin{table}[htb]
    \centering
    \begin{tabular}{c|c|c|c|c|c}
    \rule{0pt}{3ex} $z_{\textrm{min}}$ & \# CC & \multicolumn{2}{c}{Fisher Matrix}  & \multicolumn{2}{|c}{MCMC} \\
    \hline
    \rule{0pt}{3ex} & & $H_0$ (km/s/Mpc) & $\Omega_m$ & $H_0$ (km/s/Mpc) & $\Omega_m$ \\
    \hline
    \rule{0pt}{3ex} $0$ & $34$ & $68.14 \pm 3.07$ & $0.320 \pm 0.059$ & $67.76^{+3.03}_{-3.09}$  ($68.12$) & $0.328^{+0.065}_{-0.055}$ ($0.321$) \\
    \hline 
    \rule{0pt}{3ex} $0.2$ & $27$ & $65.03 \pm 6.65$ & $0.368 \pm 0.118$ & $63.05^{+6.64}_{-7.23}$ ($64.98$) & $0.405^{+0.170}_{-0.111}$ ($0.369$) \\
    \hline 
    \rule{0pt}{3ex} $0.4$ & $22$ & $62.42 \pm 8.38$ & $0.411 \pm 0.161$ & $59.54^{+8.30}_{-8.22}$ ($62.39$) & $0.470^{+0.229}_{-0.151}$ ($0.411$)\\
    \hline 
    \rule{0pt}{3ex} $0.6$ & $15$ & $59.83 \pm 17.21$ & $0.454 \pm 0.338$ & $56.45^{+13.16}_{-9.33}$ ($59.86$) & $0.526^{+0.288}_{-0.225}$ ($0.453$) \\
    \hline 
    \rule{0pt}{3ex} $0.7$ & $14$ & $79.11 \pm 19.40$ & $0.222 \pm 0.162$ & $67.59^{+19.19}_{-16.57}$ ($79.18$) & $0.344^{+0.344}_{-0.178}$ ($0.222$) \\
    \hline 
    \rule{0pt}{3ex} $0.8$ & $11$ & $103.97 \pm 24.94$ & $0.097 \pm 0.088$ & $82.43^{+28.33}_{-27.03}$ ($104.02$) & $0.206^{+0.357}_{-0.131}$ ($0.096$) \\
    \hline 
    \rule{0pt}{3ex} $1$ & $8$ & $150.37 \pm 31.21$ & $0.010 \pm 0.035$ & $108.92^{+33.94}_{-44.47}$ ($150.38$) & $0.087^{+0.304}_{-0.068}$ ($0.010$) \\
    \hline 
    \rule{0pt}{3ex} $1.2$ & $7$ & $154.35 \pm 42.95$ & $0.006 \pm 0.042$ & $83.07^{+48.52}_{-32.19}$ ($154.47$) & $0.194^{+0.439}_{-0.159}$ ($0.006$) \\
    \hline 
    \rule{0pt}{3ex} $1.4$ & $4$ & $125.41 \pm 79.55$ & $0.039 \pm 0.132$ & $65.32^{+44.88}_{-20.30}$ ($125.44$) & $0.320^{+0.423}_{-0.250}$ ($0.039$) \\
    \hline 
    \rule{0pt}{3ex} $1.5$ & $3$ & $36.12 \pm 72.69$ & $1.000 \pm 4.269$ & $55.19^{+34.64}_{-14.73}$ ($36.16$) & $0.393^{+0.387}_{-0.283}$ ($0.999$)
    \end{tabular}
    \caption{Comparison between Fisher matrix and MCMC analysis for CC data with a low redshift cut-off $z_{\textrm{min}}$. We record the number of data points, the extremum of the $\chi^2$ and $1 \sigma$ confidence interval estimated from the Fisher matrix,  $16^{\textrm{th}}$, $50^{\textrm{th}}$ and $84^{\textrm{th}}$ percentiles from MCMC posteriors corresponding to $1 \sigma$ confidence intervals, and the minimum $\chi^2$ from the MCMC chain in brackets. MCMC marginalisation exhibits non-Gaussian $1 \sigma$ confidence intervals, and for $z_{\textrm{min}} > 1$, the minimum value of the $\chi^2$ from the MCMC chain falls outside of this interval. The latter tracks the best fit up to small numbers in line with expectations. }
    \label{tab:LCDM_CC}
\end{table}

\subsection{Features in CC Data}
\label{sec:features}
Once one accounts for biases, it is clear from Table \ref{tab:LCDM_CC} that there are trends in CC data when it is binned. Starting from $z_{\textrm{min}} = 0$ through to $z_{\textrm{min}} = 0.6$ we see a decreasing trend in best fit values of $H_0$ (also central $H_0$ values from MCMC), which is compensated by a increasing trend in $\Omega_m$ best fit values. From Fig.~\ref{fig:CC} it is difficult to visibly discern any trend from the raw data. From $z_{\textrm{min}} = 0.7$ through to $z_{\textrm{min}} = 1.4$, there is in contrast a preference for larger $H_0$ and smaller $\Omega_m$ values. This trend is evident from the raw data, where at higher redshifts one sees large scatter and large fractional errors in the data. For $z_{\textrm{min}} = 1$, it is clear that the best fit line in magenta corresponding to $(H_0, \Omega_m) = (150.4, 0.01)$ (Table \ref{tab:LCDM_CC}) is closer to horizontal line than the Planck-$\Lambda$CDM cosmology in red. To be more explicit, for $z_{\textrm{min}} = 0$, $\rho_{m0}:=H_0^2\Omega_m\simeq 1500$ which is close to the Planck value, whereas for $z_{\textrm{min}} = 1$, $\rho_{m0}\simeq 225$. The sharp drop in $\rho_{m0}$ means the magenta line should be almost horizontal. For $z_{\textrm{min}} = 1.5$, we switch to an opposite regime of parameter space with unexpectedly low and high values of $H_0$ and $\Omega_m$, respectively, a trend which is evident in the data, but there are only three data points. Despite, the small number of data points, the tendency for smaller $H_0$ and larger $\Omega_m$ inferences within $\Lambda$CDM cosmology at high redshifts has been documented across three independent observables \cite{Colgain:2022rxy}. We will come back to this claim in section \ref{sec:tension}. Finally, it is worth noting that for large $z_{\textrm{min}}$ and samples with few data points, one expects broad MCMC posteriors. These posteriors are severely impacted by the prior on $\Omega_m$, as is evident from Table \ref{tab:LCDM_CC}. 

For the moment we leave physical speculations to the discussion and return to the trend in CC data above $z=1$ favouring less evolution in the Hubble parameter than the Planck-$\Lambda$CDM model. We would like to quantify the significance of this trend, but since we are working in a non-Gaussian regime of the model, we can expect both Fisher matrix and MCMC to give biased results. In Fig.~\ref{fig:CCsplit1} we show MCMC posteriors for $z>1$ CC data in blue alongside posteriors for low redshift ($z < 1$) CC data, which is simply added to aid comparison and also highlight the Gaussianity of the low redshift posteriors. One notes that the peaks of the $z > 1$ distributions are a little displaced from to the values minimising the $\chi^2$. However, the emergence of the lower peak in the $H_0$ posterior at $H_0 \sim 50$ km/s/Mpc has the hallmarks of a projection effect. To appreciate this, note that the configurations in the blue curve in the top left corner of the 2D posterior are projected onto the lower $H_0$ peak. Moreover, if one shifts the $H_0$ peak from $H_0 \sim 150$ to $H_0 \sim 50$ km/s/Mpc while maintaining $\Omega_m \sim 0$, this shifts the magenta curve in Fig. \ref{fig:CC} outside of all the data points, so the lower $H_0$ peak is a phantom artefact unrelated to the goodness of fit. We also observe a shift in the higher $H_0$ peak away from the minimum of the $\chi^2$.

Ignoring these features, one could attempt to interpret the overlap in the 2D posteriors in Fig. \ref{fig:CCsplit1}. Doing so, one may conclude that low and high redshift CC data are consistent within $1 \sigma$. However, since Hubble tension is a 1D problem (local $H_0$ determinations are insensitive to other parameters), to compare with locally observed values of $H_0$ one needs to project onto the $H_0$ axis. Alternatively put, Hubble tension is a problem in 1D posteriors. Projecting onto the $H_0$ axis by determining $16^{\textrm{th}}$, $50^{\textrm{th}}$ and $84^{\textrm{th}}$ percentiles, one sees from Table \ref{tab:LCDM_CC} that the $z_{\textrm{min}} = 1$ MCMC confidence interval encloses the $z_{\textrm{min}} = 0$ central values within $1 \sigma$,\footnote{Note, removing the eight high redshift data points from the $z_{\textrm{min}} = 0$ sample will not shift the central values much.} but not the point in parameter space that best fits the data!


% Figure environment removed



Evidently, given the non-Gaussian posteriors, care is required when interpreting the significance of the trend towards a non-evolving (horizontal) $H(z)$ at higher redshifts in Fig.~\ref{fig:CC}. We cannot use the errors from the Fisher matrix as we are clearly in a non-Gaussian regime, whereas MCMC inferences are impacted by projection effects to the extent that the minimum of the $\chi^2$ (confirmed from the MCMC chain) falls outside of the $1 \sigma$ confidence interval. For this reason, we resort to mock simulations. While this may seem a little redundant if we are going to employ profile distributions in section \ref{sec:PD}, there is motivation for this exercise. In \cite{Colgain:2022rxy} the significance of a descending $H_0$/increasing $\Omega_m$ trend with effective redshift in OHD, Type Ia SN and QSOs was estimated to be a $\sim 3 \sigma$ effect on the basis of combining $\sim 2 \sigma$ effects in each of the \textit{independent} data sets using Fisher's method. Here, working with the same data throughout, we can directly compare the significance of a discrepancy estimated through mock simulations from the significance of a discrepancy estimated through profile distributions. In particular, we will address the question: how significant is a constant $H(z)$ with $z_{\textrm{min}}=1$ (8 data points) against the Planck consistent cosmology favoured by the full data set ($z_{\textrm{min}}=0$ entry in Table \ref{tab:LCDM_CC})? Note, the significance will be overestimated due to missing systematic uncertainties (see \cite{Moresco:2020fbm}), but we can still make comparison between the two techniques.

\paragraph{{Mock simulations:}} To address this question using mock simulations, we begin with the MCMC chains for the full sample. For each entry in the MCMC chain (approximately 15,000 entries in total), we generate a new realisation of the 8 high redshift data points $(z > 1)$ that are by construction statistically consistent with both the best fits from the full sample and also the Planck-$\Lambda$CDM values \cite{Planck:2018vyg}. More concretely, for each $(H_0, \Omega_m)$ entry in our MCMC chain, we displace the data points to the corresponding $\Lambda$CDM Hubble parameter before generating new data points in a normal distribution where the errors serve as standard deviations. We then fit back the $\Lambda$CDM model to each realisation of the mock data and record the best fit $(H_0, \Omega_m)$ values, which give us a distribution of expected $(H_0, \Omega_m)$ best fits. The distributions are presented in Fig.~\ref{fig:CCsims} alongside the best fits from observed data. Throughout, we assume canonical values $(H_0, \Omega_m) = (70, 0.3)$ for the initial guess of the fitting algorithm. Best fits can saturate our bounds, i. e. $\Omega_m = 0$ and $\Omega_m = 1$, and this leads to an unsightly pile up of best fits at $\Omega_m = 0$ and $\Omega_m = 1$ in Fig.~\ref{fig:CCsims} \cite{Colgain:2022rxy}. It is important to retain all the configurations, otherwise one is not accounting for the probability that a best fit falls outside our priors. As a consistency check, we see that the median or 50$^{\textrm{th}}$ percentile, $(H_0, \Omega_m) = (68.32, 0.321)$ agrees well with the mock input parameters, thereby demonstrating that there are an equal number of best fits with values above and below the injected parameters in the mocks. We find that probability of a more extreme (larger) $H_0$ value to be $p = 0.022$, while the probability of a more extreme (smaller) $\Omega_m$ value to be $p = 0.035$, respectively. Converted into a Gaussian statistic, these correspond to $2 \sigma$ and $1.8 \sigma$, respectively, for a one-sided normal distribution. Thus, on the basis of mock simulations, we estimate the non-evolving constant $H(z)$ with $z_{\textrm{min}} = 1$ as a $\sim 2 \sigma$ effect. In the next section we will recover this number more or less from the profile distribution analysis. 

% Figure environment removed


\section{Profile Distributions}
\label{sec:PD}
Having explained the mathematics behind the bias, which gives rise to a projection effect, in subsection \ref{sec:math}, and having illustrated how it affects MCMC inferences in subsection \ref{sec:CCbias} - the minimum of the $\chi^2$ may fall outside of $1 \sigma$ confidence intervals - we turn to profile distributions (PDs) \cite{Gomez-Valent:2022hkb}, an extension of the profile likelihood, e. g. \cite{Trotta:2017wnx}, in order to address the bias. Consider two sets of parameters $\theta_1$ and $\theta_2$ and a normalised distribution $\mathcal{P}(\theta_1, \theta_2)$. The basic idea \cite{Gomez-Valent:2022hkb} is to study the ratio 
\be
\label{R}
R(\theta_1) = \frac{\tilde{\mathcal{P}}(\theta_1)}{\max_{\theta_1} \tilde{\mathcal{P}}(\theta_1) } = \frac{\tilde{\mathcal{P}}(\theta_1)}{\max_{\theta_1, \theta_2} \mathcal{P}(\theta_1, \theta_2) },  
\ee
where $\tilde{\mathcal{P}}(\theta_1)$ is the PD, defined to be the maximum of $\mathcal{P}$ for each $\theta_1$ along the $\theta_2$ direction: 
\be
\label{PD}
\tilde{\mathcal{P}} (\theta_1) = \max_{\theta_2} \mathcal{P}(\theta_1, \theta_2). 
\ee
The advantage of this approach is that $R(\theta_1)$ can serve as a probability distribution function (up to an overall normalization), however we do not need to perform any integration, so $R(\theta_1)$ is not prone to volume or projection effects. At this juncture, given the simplicity of our setup with only two parameters $(H_0, \Omega_m)$, we can be more explicit. Consider the probability distribution,   
\be
\mathcal{P}(\theta_1, \theta_2) = \exp \left( - \frac{1}{2} \chi^2(\theta_1, \theta_2) \right), 
\ee
where $\theta_i \in \{H_0, \Omega_m \}$  and $\chi^2(H_0, \Omega_m)$ is our earlier likelihood (\ref{eq:chi2}). The maximum value of $\mathcal{P}$ occurs for the minimum value of $\chi^2$ from the MCMC chain, $\mathcal{P}_{\textrm{max}} = e^{-\frac{1}{2} \chi^2_{\textrm{min}}}$. In this concrete setting, the PD becomes 
\be
\tilde{\mathcal{P}}(\theta_1) = e^{-\frac{1}{2} \chi^2_{\textrm{min}}(\theta_1)}, 
\ee
where $\chi^2_{\textrm{min}}(\theta_1)$ denotes the minimum value of the $\chi^2$ along the $\theta_2$ direction for a fixed $\theta_1$ value. It should not be confused with the overall minimum $\chi^2_{\textrm{min}}$, which can be extracted easily from the MCMC chain. In practice, one can also determine $\chi^2_{\textrm{min}}(\theta_1)$ from the MCMC chain by breaking the $\theta_1$ direction up into bins and finding the minimum of the $\chi^2$ for each bin. Having done so, we are in a position to define a PDF \cite{Gomez-Valent:2022hkb}: 
\be
\label{eq:w}
w(\theta_1) = \frac{e^{-\frac{1}{2} \chi^2_{\textrm{min}}(\theta_1)}}{\int e^{-\frac{1}{2} \chi^2_{\textrm{min}}(\theta_1)} \, \textrm{d} \theta_1} = \frac{R(\theta_1)}{\int R(\theta_1) \, \textrm{d} \theta_1}, 
\ee
where in the second equality we have divided top and bottom by $\mathcal{P}_{\textrm{max}} = e^{-\frac{1}{2} \chi^2_{\textrm{min}}}$. As a result, $R(\theta_1) = e^{-\frac{1}{2} \Delta \chi_{\textrm{min}}^2}$, where $\Delta \chi^2_{\textrm{min}} := \chi_{\textrm{min}}^2(\theta_1) - \chi^2_{\textrm{min}}$, so that $R(\theta_1)$ peaks at $R(\theta_1) = 1$. Note that $\int_{-\infty}^{+\infty} w(\theta_1) \, \textrm{d} \theta_1 = 1$ by construction, so $w(\theta_1)$ describes a properly normalised PDF. Thus we can identify the $1 \sigma, 2 \sigma$ and $3 \sigma$ confidence intervals corresponding to the 68\%, 95\% and 99.7\% confidence level, respectively, by simply identifying $\theta_1^{(1)}$ and $\theta_1^{(2)}$ such that \cite{Gomez-Valent:2022hkb}
\be
\label{eq:wsigma}
\int_{\theta_1^{(1)}}^{\theta_1^{(2)}} w(\theta_1) \, \textrm{d} \theta_1 = I, \quad w(\theta_1) = w(\theta_2), \quad I \in \{0.68, 0.95, 0.997\}. 
\ee
We will outline how these conditions can most easily be satisfied when we turn to explicit examples. 

Our first port of call is making sure that the PD methodology gives sensible results. This can be best judged by applying it to the CC data with $z_{\textrm{min}} = 0$, since this is where we expect a distribution closest to a Gaussian distribution, as is evident from the agreement between Fisher matrix and MCMC results in Table \ref{tab:LCDM_CC}. In particular, we will be interested in a comparison between $1 \sigma$ confidence intervals to make sure that (\ref{eq:wsigma}) is not underestimating or overestimating the $1 \sigma$ confidence interval. 

% Figure environment removed

We start by running a long MCMC chain (100,000 iterations) in order to ensure bins are well populated, and begin by analysing $\theta_1 = H_0$ with $\theta_2 = \Omega_m$. From the MCMC chain we identify the smallest and largest value of $H_0$ in the chain and break up this range into approximately 200 uniform bins, which we label using the $H_0$ value at the centre of the bin. We omit any empty bins. One can increase the number of bins by simply running a longer MCMC chain. In each $H_0$ bin we identify the minimum value of the $\chi^2$, $\chi^2_{\textrm{min}}(H_0)$, and calculate $R(H_0)$. One then repeats the steps for $\Omega_m$. In Fig.~\ref{fig:R_zmin0} we plot $R(H_0)$ against $H_0$ and $R(\Omega_m)$ against $\Omega_m$, noting that the distributions are Gaussian to first approximation. 

Since the distributions from the MCMC chain are sparse in the tails, empty bins are evident in Fig.~\ref{fig:R_zmin0}. Nevertheless, with 200 bins, modulo any empty bins, we have sufficient density of points to calculate the total area under the $R(H_0)$ and $R(\Omega_m)$ curve using Simpson's rule. Any concern about precision can simply be mitigated by running a longer MCMC chain and increasing the number of bins. 
One may directly use $R(H_0)\leq 1$ and $R(\Omega_m)\leq 1$   to find $68$, $95$ and $99.7$ percentiles,  respectively corresponding to $1 \sigma, 2 \sigma$ and $3 \sigma$ confidence intervals. Consider $F_\kappa:= \int_{R\geq \kappa} R (\theta_1) \, \textrm{d} \theta_1$, where $\kappa \leq 1$. Observe that $F_{\kappa=1}=0$ and $F_{\kappa=0}:=F_0=\int R(\theta_1) \textrm{d} \, \theta_1$. Then move $\kappa$ through and terminate the process when $F_\kappa/F_0$ is equal to $0.68$, $0.95$ and $0.997$. This gives the corresponding range for $\theta_1$ that defines the confidence interval.
Working with the precision afforded to us by approximately 200 bins, the $H_0$ and $\Omega_m$ $1 \sigma$ confidence intervals are presented in Fig.~\ref{fig:R_zmin0} and the first entry in Table \ref{tab:LCDM_CC_PD}. The outcome is in excellent agreement with both Fisher matrix and MCMC analysis. In particular, a mild non-Gaussianity in $\Omega_m$ is evident in both Fig.~\ref{fig:R_zmin0} and the errors. 
Thus, we have succeeded in recovering results in the (almost) Gaussian regime that are consistent with Fisher matrix and MCMC analysis and this provides an important check of the methodology.  

% Figure environment removed

We now apply the same PD methodology to the non-Gaussian regime where MCMC marginalisation leads to biased results. To be concrete, we focus on the eight data points in the range $1 < z < 2$ where a non-evolving $H(z)$ trend is evident in the raw data in Fig.~\ref{fig:CC}. Our goal here is to quantify the disagreement with the full data set, where one infers $H_0 \sim 68$ km/s/Mpc and $\Omega_m \sim 0.32$. A similar exercise was performed in subsection \ref{sec:features} with mock simulations and the disagreement was estimated to be approximately $2 \sigma$. Repeating the steps outlined above for the CC data with $z_{\textrm{min}} = 1$ we find the distributions in Fig.~\ref{fig:R_zmin1}. The first observation is that the distributions are non-Gaussian, but a comparison to the MCMC posteriors from the same data in blue in Fig.~\ref{fig:CCsplit1} reveals that there is no secondary $H_0$ peak at $H_0 \sim 50$ km/s/Mpc. Thus, we confirm the secondary peak to be a projection effect. That being said, the primary $H_0$ peak from Fig.~\ref{fig:CCsplit1} has shifted to the dashed line corresponding to the minimum of the $\chi^2$, since the peak of the distribution and $\chi^2$ minimum agree by construction. Comparing the blue $\Omega_m$ distribution from Fig.~\ref{fig:CCsplit1} to the $R(\Omega_m)$ distribution in Fig.~\ref{fig:R_zmin1}, we see that the peak is close to $\Omega_m = 0$ and that the tails continue to $\Omega_m = 1$. In both plots we see that there is a non-zero probability of inferring $\Omega_m = 1$. In some sense, this is not so surprising, the reason being that one is free to adopt generous priors for $H_0$, so that probability of large and small $H_0$ values is zero, but the priors on $\Omega_m$ in the flat $\Lambda$CDM model are restricted. For this reason, as a distribution spreads one invariably finds that distributions are impacted by the $\Omega_m$ priors.\footnote{Note, this is a problem for the flat $\Lambda$CDM model. In particular, one may easily find that the peak of the $\Omega_m$ distribution is larger than $\Omega_m=1$, as is the case with Hubble Space Telescope SN with redshifts $z > 1$ in the Pantheon+ sample \cite{Malekjani:2023dky}.}

It is evident from Fig.~\ref{fig:R_zmin1} that any tension that exists is confined to the $H_0$ parameter. Moreover, since there may be only one binned value of $\Omega_m$ below the $R(\Omega_m)$ peak, at the precision afforded to us by 200 bins, the $R(\Omega_m)$ distribution in Fig.~\ref{fig:R_zmin1} is essentially one-sided and the $1 \sigma$ confidence interval stretches beyond $\Omega_m \sim 0.32$, so there is no disagreement in the $\Omega_m$ parameter. Nevertheless, in the $H_0$ parameter we see that $H_0 \sim 68$ km/s/Mpc, the value favoured by the full data set is just under $2 \sigma$ removed from the peak. The main point here is that, as is obvious from the raw data, current CC data with $z > 1$ has a preference for a non-evolving Hubble parameter $H(z)$ with a large constant $H_0 \sim 150$ km/s/Mpc. The disagreement is just under $2 \sigma$, more accurately $1.9 \sigma$ from $R(H_0)$, and only $0.9 \sigma$ from $R(\Omega_m)$. Although this may not be a serious discrepancy, essentially because of the poor data quality (8 data points), this disagreement supports the $\sim 2 \sigma$ discrepancy seen in the mock simulations. It should be borne in mind that systematic uncertainties have been omitted and these will reduce this discrepancy once properly propagated. Given the agreement between the PD and mock simulation analysis, there is nothing to suggest that the three independent trends highlighted in \cite{Colgain:2022rxy} across OHD, Type Ia SN and QSOs are not \textit{bona fide} disagreements and that redshift evolution is present in the sample. The task remains to combine them at the level of a $\chi^2$ likelihood instead of combining them using Fisher's method on the basis that they are independent probabilities. We leave this exercise for a forthcoming paper, but revisit the tension in OHD data in the following section.  %\ref{sec:tension}. 
For completeness, in Table \ref{tab:LCDM_CC_PD} we perform a reanalysis of CC data subsets with the PD approach and record the $1 \sigma$ intervals.  

\begin{table}[htb]
    \centering
    \begin{tabular}{c|c|c|c}
    \rule{0pt}{3ex} $z_{\textrm{min}}$ & \# CC & \multicolumn{2}{c}{PD}  \\
    \hline
    \rule{0pt}{3ex} & & $H_0$ (km/s/Mpc) & $\Omega_m$ \\
    \hline
    \rule{0pt}{3ex} $0$ & $34$ & $68.15^{+3.04}_{-3.11}$ & $0.320^{+0.065}_{-0.055}$ \\
    \hline 
    \rule{0pt}{3ex} $0.2$ & $27$ & $65.03^{+6.52}_{-7.03}$ & $0.368^{+0.167}_{-0.110}$ \\
    \hline 
    \rule{0pt}{3ex} $0.4$ & $22$ & $62.42^{+7.78}_{-8.74}$ & $0.411^{+0.236}_{-0.113}$ \\
    \hline
    \rule{0pt}{3ex} $0.6$ & $15$ & $59.75^{+11.73}_{-13.97}$ & $0.455^{+0.355}_{-0.160}$ \\
    \hline
    \rule{0pt}{3ex} $0.7$ & $14$ & $79.10^{+16.42}_{-20.56}$ & $0.222^{+0.386}_{-0.117}$ \\
    \hline
    \rule{0pt}{3ex} $0.8$ & $11$ & $103.94^{+22.88}_{-28.54}$ & $0.097^{+0.378}_{-0.074}$ \\
    \hline
    \rule{0pt}{3ex} $1$ & $8$ & $150.35^{+17.12}_{-35.95}$ & $ < 0.339$ \\
    \hline
    \rule{0pt}{3ex} $1.2$ & $7$ & $154.26^{+14.88}_{-54.82}$ & $ < 0.570$ \\
    \hline
    \rule{0pt}{3ex} $1.4$ & $4$ & $124.81^{+35.38}_{-52.60}$ & $ < 0.661$ \\
    \hline
    \rule{0pt}{3ex} $1.5$ & $3$ & $36.11^{+72.87}_{-2.43}$ & $ > 0.354$
    \end{tabular}
    \caption{Same as Table \ref{tab:LCDM_CC} but with the PD methodology in lieu of Fisher matrix and MCMC analysis. The high redshift $R(\Omega_m)$ distributions are typically one-sided, so one encounters $1 \sigma$ upper and lower bounds.}
    \label{tab:LCDM_CC_PD}
\end{table}




\section{A tension with Planck}
\label{sec:tension}
A $2 \sigma$ ($p = 0.021$) tension with Planck has been reported in OHD through best fits and mock simulations in \cite{Colgain:2022rxy}. In particular, it was noted that a combination of 7 CC and BAO data points above $z = 1.45$ resulted in a $(H_0, \Omega_m) = (37.8, 1)$ best fit, where in line with analysis here, an $\Omega_m \in [0, 1]$ uniform prior was assumed. Based on mock simulations, the probability of such a best fit configuration arising by chance in mocks assuming input parameters consistent with Planck was estimated to be $p = 0.021$ \cite{Colgain:2022rxy}. A similar best fit appears in the last entry of Table \ref{tab:LCDM_CC} and Table \ref{tab:LCDM_CC_PD}, but there is no tension with Planck within the errors, even with our PD analysis, because CC data is inherently of poorer quality than BAO data. One further difference between the analysis is that \cite{Colgain:2022rxy} imposes a Gaussian Planck prior $\Omega_m h^2 = 0.1430 \pm 0.0011$ \cite{Planck:2018vyg} \footnote{This prior essentially prevents high redshift CC data from tracking a non-evolving $H(z)$.} to fix the high redshift behaviour of $H(z)$, whereas our analysis here so far has not introduced a prior. 

% Figure environment removed

Nevertheless, armed with a new PD methodology, we are in a position to revisit the earlier result and see if we can recover the $2 \sigma$ tension with Planck. Since \cite{Colgain:2022rxy} made use of older BAO data, here we replace QSO and Lyman-$\alpha$ BAO with the latest eBOSS results \cite{Hou:2020rse, Neveux:2020voa, duMasdesBourboux:2020pck}. Moreover, we work directly with the $D_{H}/r_d$ constraints and do not invert them. This entails assuming a value for the radius of the sound horizon, which we take to be the Planck value, $r_d = 147.09 \pm 0.26$ Mpc \cite{Planck:2018vyg}. In addition, we reinstate the prior $\Omega_m h^2 = 0.1430 \pm 0.0011$, so that the only difference with \cite{Colgain:2022rxy} is simply to update OHD BAO to the latest constraints. We stress that the priors we introduce are consistent with the Planck cosmology, so \textit{they cannot be driving any disagreement}. Moreover, the $\Omega_m h^2$ prior restricts one to a curve in the $(H_0, \Omega_m)$, but it cannot dictate where one is on the curve, this is done by the remaining 3 CC and 3 BAO data points.  

We again marginalise over the free parameters $(H_0, \Omega_m, r_d)$ with MCMC. In Fig.~\ref{fig:CC_BAO_MCMC} we present the posteriors. While $r_d$ is Gaussian and peaked on our Planck prior, as expected, the $\Omega_m$ posterior is peaked at $\Omega_m \sim 0.6$ and the fact that the fall off in the distribution is gradual beyond the peak leads to a pile up of configurations in the top left corner of the $(H_0, \Omega_m)$-plane. This fall off continues beyond $\Omega_m = 1$ and if the prior is relaxed, the $H_0$ peak shifts to smaller values. So,  once again all the hallmarks of projection effects are present. That being said, given the sharp fall off in the $\Omega_m$ distribution to smaller $\Omega_m$ values, some tension appears to be evident with the Planck values (dashed lines). 

% Figure environment removed

We now run the MCMC chain through our PD methodology. From Fig.~\ref{fig:CC_BAO}, we can see that the $R(H_0)$ and $R(\Omega_m)$ distributions prefer smaller values of $H_0$ and larger values of $\Omega_m$. The peak of the distributions occurs at $H_0 = 42.40$ km/s/Mpc and $\Omega_m = 0.795$.  The lone dot in the $R(H_0)$ distribution at low values of $H_0$ tells us that the distribution falls off sharply below $H_0 = 40$ km/s/Mpc. Note, since we employed generous uniform priors $H_0 \in [0, 200]$, the priors are not impacting the $R(H_0)$ distribution, so it is expected that the distribution falls off to zero on both sides. In contrast, the $R(\Omega_m)$ distribution is one-sided and fails to fall off in the direction of larger values within the uniform priors $\Omega_m \in [0, 1]$. The tension with Planck falls between $2 \sigma$ and $3 \sigma$. By integrating the PDF as far as the black lines corresponding to the Planck values in Fig.~\ref{fig:CC_BAO}, we estimate that the Planck $H_0$ is located at $2.1 \sigma$ from the peak, while the Planck $\Omega_m$ value is $2.5 \sigma$ from the peak.

The main take-away from this section is that OHD data comprising CC and BAO data points beyond $z=1.45$ is inconsistent with the Planck cosmology at in excess of $2 \sigma$. We have employed Planck priors to arrive at this result, but these priors cannot drive the disagreement. Moreover, independent analysis based on least squares fitting and mock simulations presented in \cite{Colgain:2022rxy} also points to a $2 \sigma$ tension, albeit with less up-to-date high redshift BAO data. In summary, different methodologies agree on a $2 \sigma$ discrepancy with Planck, which is robust to interchanging older and newer BAO data. 

\section{Concluding remarks}
\label{sec:discussion}
A $\chi^2$ likelihood is a metric or measure of how well a model fits data. The point in model parameter space that fits the data the best possesses the lowest $\chi^2$. Once one has identified this point, the problem remains to establish $1 \sigma$, $2 \sigma$, etc, confidence intervals in parameter space. In cosmology and astrophysics, MCMC is the prevailing technique for estimating confidence intervals. Its great advantage is that it allows one to i) globally sample the parameter space and ii) arrive at posteriors that serve as an estimate of the errors even with non-Gaussian distributions. In contrast, if one minimises the $\chi^2$ by gradient descent, there is always a risk that one ends up in a local minimum, i. e. the global minimum is missed, while error estimation through Fisher matrix assumes any distribution is Gaussian. The appeal of MCMC marginalisation is that it is widely applicable. However, the point of this paper is that limitations exist, even in the simplest model. 

Indeed, what happens when the MCMC posterior no longer tracks points in parameter space that fit the data better? Traditionally, volume effects are seen as the preserve of higher-dimensional models, e. g. \cite{Herold:2021ksg, Gomez-Valent:2022hkb, Meiers:2023gft}, but projection effects also occur in the minimal $\Lambda$CDM model when one fits the model to data binned by redshift in the late Universe \cite{Colgain:2022tql}. As explained in \cite{Colgain:2022tql}, this ``projection effect'' is driven by OHD, $H(z_i)$, and angular diameter or luminosity distance data, $D_{A}(z_i)$ or $D_{L}(z_i)$, {respectively} only constraining the combinations $\Omega_m h^2$ and $ (1-\Omega_m) h^2$ well, with high redshift data $z_i \gg 0$. In practice, this restricts MCMC configurations to constant $\Omega_m h^2$ and constant $(1-\Omega_m) h^2$ curves in the $(H_0, \Omega_m)$ plane, and as the curves stretch due to DE or matter being less well constrained in high redshift bins, projection effects lead to shifts in the peaks of MCMC posteriors and the emergence of non-Gaussian tails \cite{Colgain:2022tql}. We stress that one sees the same effect in PDFs of best fit $(H_0, \Omega_m)$ parameters in a large number of mock data realisations \cite{Colgain:2022tql}, so the problem is more general than MCMC; there is an inherent bias in the $\Lambda$CDM model when one fits it to redshift binned $H(z)$ \textit{or} $D_{A}(z)$ \textit{or} $D_{L}(z)$ data. Within MCMC, one sees this effect in the errors, but also in the drift of the parameters corresponding to the $\chi^2$ minimum outside of the $1 \sigma$ confidence intervals. Highlighting this (expected) bias in MCMC using OHD is the opening salvo (result) of this paper.     

Why should one care? This is evidently only a problem if one bins data and confronts the $\Lambda$CDM model. First, note that some data sets are inherently binned. For example, effective redshifts are assigned to CC and BAO analysed in a given redshift bin, while each strongly lensed system constitutes its own bin. Working with binned data is unavoidable. Secondly, $\Lambda$CDM tensions point to a problem with the $\Lambda$CDM model once the tensions become widespread and persistent. As explained in \cite{Krishnan:2020vaf}, if the minimal $\Lambda$CDM model is too simple, one expects redshift evolution of $\Lambda$CDM cosmological parameters as it is confronted to redshift binned data. Hints of these trends are now evident in $H_0$ \cite{Wong:2019kwg, Millon:2019slk, Dainotti:2021pqg, Colgain:2022nlb, Colgain:2022rxy, Malekjani:2023dky, Hu:2022kes, Jia:2022ycc, Krishnan:2020obg, Dainotti:2022bzg}, $\Omega_m$ \cite{Risaliti:2015zla, Risaliti:2018reu, Lusso:2020pdb, Yang:2019vgk, Khadka:2020vlh, Khadka:2020tlm, Khadka:2021xcc, Pourojaghi:2022zrh, Colgain:2022nlb, Colgain:2022rxy, Malekjani:2023dky, Pasten:2023rpc, Sakr:2023hrl} and $S_8$/$\sigma_8$ \cite{Esposito:2022plo, Adil:2023jtu, ACT:2023dou, ACT:2023kun} (also \cite{Miyatake:2021qjr, Alonso:2023guh}) across a host of different observables. This evolution is an expected hallmark of model breakdown, which must happen at some redshift if systematics are not universally at play. 

The main problem with redshift dependent $\Lambda$CDM cosmological parameters\footnote{There is a separate interpretation problem as the cosmology literature works with  parameters ``defined today''. In more mathematical language, this is simply the statement that one solves an ordinary differential equation (ODE), namely the Friedmann equation or equivalent, by specifying an integration constant, e.g. $H_0 = H(z=0)$ or $\rho_m(z=0)=\rho_{m0}=H_0^2\Omega_{m}$. However, this is a mathematical statement and it still needs to be confirmed observationally that $H_0$ or $\rho_{m0}$ are \textit{bona fide} constants. This cannot be \textit{a priori} assumed, because it is mathematical prediction of the model. If the model is correct, a constant $H_0$ and $\Omega_m$  will be supported by the data. See \cite{Krishnan:2020vaf} for further discussion.} is one needs to assign a statistical significance to any trend. At a purely practical level, this entails constructing bins centered on different redshifts and identifying discrepancies in $\Lambda$CDM parameters between bins, \textit{ideally in the same observable}, so that the potential systematics are under greatest control. As demonstrated both mathematically and observationally with the CC data in section \ref{sec:MCMC_bias}, MCMC marginalisation leads to biased inferences when one bins the data. In this paper we have resorted to profile distributions \cite{Gomez-Valent:2022hkb} to overcome this bias and have applied the technique to a setting where $\Lambda$CDM distributions are expected to be non-Gaussian for the reasons outlined above and in section \ref{sec:MCMC_bias}. This new technique, provides a complementary perspective that confirms the least square fits of observed and mock data presented in \cite{Colgain:2022nlb, Colgain:2022rxy, Malekjani:2023dky}, where evidence for redshift evolution in $H_0$ and $\Omega_m$ was presented. Regardless of the methodology, the objective is to drill down on the prevailing \textit{assumption} that cosmological parameters are constants. \textit{In the era of tensions in cosmology, nothing can be assumed, especially noting that the tensions are in essence showing an example of evolution of these parameters with redshift.}

More concretely, in this paper with both mock simulations and profile distributions we have shown that high redshift CC data has a preference for a non-evolving $H(z)$ over Planck-$\Lambda$CDM at approximately $\sim 2 \sigma$. This trend, which constitutes the second result of the paper, is unquestionable, as it is visible in the data. Note, we have not propagated systematic uncertainties, so the significance will be less when these are properly propagate. Nevertheless, low and high redshift CC data currently have a preference for different $\Lambda$CDM cosmological parameters. This is important because if the CC program is claiming an 8\% constraint on the Hubble constant, $H_0 = 66.7 \pm 5.5$ km/s/Mpc \cite{Moresco:2023zys}, it is imperative that \textit{all subsets of the data are consistent with this result}. If they are not, then we are staring at either systematics or model breakdown. Admittedly, demanding self-consistency of subsets of a data set confronted to a model is a high bar, but it is important that data sets result in overlapping constraints on $\Lambda$CDM parameters, otherwise this makes cosmological inferences moot. Note, the $\Lambda$CDM model is largely only well tested in the DE dominated regime $z \lesssim 1$ and at very high redshifts $z \sim 1100$, which leaves a wide expanse of redshifts to be explored in order to confirm or refute the model. Given the existing $\Lambda$CDM tensions \cite{Perivolaropoulos:2021jda, Abdalla:2022yfr}, and the hints of evolution in $H_0$, $\Omega_m$ and $S_8$ across assorted probes in the late Universe $z \lesssim 5$, it would be surprising if all discrepancies could be explained away by systematics.\footnote{We are open to the possibility, we just consider it a bad bet at the moment. The odds can of course change as observations improve.}

As an aside, it is intriguing that CC data has a preference for larger best fit values of $H_0$ and smaller best fit values of $\Omega_m$ beyond $z_{\textrm{min}} = 0.7$, as this is traditionally the transition redshift between decelerated and accelerated expansion. % where $\ddot{a} = 0$. 
Moreover, at higher redshifts $z \sim 2.3$, there is not only a longstanding anomaly in Lyman-$\alpha$ BAO \cite{duMasdesBourboux:2020pck}, but QSOs also show a preference for a lower luminosity distance, $D_{L}(z)$, relative to Planck-$\Lambda$CDM \cite{Risaliti:2015zla, Risaliti:2018reu}. Translated into $\Lambda$CDM parameters, this corresponds to conversely larger $\Omega_m$ values, e. g.  \cite{Yang:2019vgk, Khadka:2020vlh, Khadka:2020tlm, Khadka:2021xcc, Pourojaghi:2022zrh}, and consequently smaller $H_0$ values. Thus, the emerging probes CC and QSOs  \cite{Moresco:2022phi} do not appear to be in sync on high redshift $\Lambda$CDM inferences. Nevertheless, neither may be inconsistent with the anomaly in Lyman-$\alpha$ BAO. Relative to Planck-$\Lambda$CDM, Lyman-$\alpha$ BAO prefers \textit{smaller} values of $D_{M}(z) := c \int_{0}^z 1/H(z^{\prime}) \, \textrm{d} z$ and \textit{smaller} values of $H(z)$ (larger values of $D_{H}(z) := c/H(z)$).\footnote{In this statement we assumed the Planck value $r_d \sim 147$ Mpc \cite{Planck:2018vyg} If we reinstate the radius of the sound horizon in these expressions, one recognises that changing the sound horizon, as advocated by early Universe resolutions to Hubble tension, cannot consistently address the Lyman-$\alpha$ BAO anomaly. In general, even for the Planck-$\Lambda$CDM sound horizon, one cannot get both a smaller $D_{M}(z)$ and smaller $H(z)$ from a strictly increasing function, such as the $\Lambda$CDM $H(z)$. As a result, deviations from the Planck-$\Lambda$CDM model that address this anomaly are expected to lead to wiggles in $H(z)$ \cite{Akarsu:2022lhx}, which are unsurprisingly seen in data reconstructions \cite{Zhao:2017cud, Wang:2018fng, Escamilla:2021uoj}. Finally, evolution in $H_0, \Omega_m$ discussed here cannot be explained or accommodated by early resolutions to Hubble tension relying on a change in the $r_d$ at very high $z$.}. If CC data prefer less evolution in $H(z)$ in the matter-dominated regime, then this is consistent with the preference for a smaller $H(z)$ from Lyman-$\alpha$ BAO. Furthermore, QSO data prefers smaller luminosity distances $D_{L}(z)$ relative to Planck, which are consistent with the smaller $D_{M}(z) \propto D_{L}(z)$ values preferred by Lyman-$\alpha$ BAO. Thus, even if CC and QSOs appear to be showing diverging behaviour in the cosmological parameters $(H_0, \Omega_m)$, this may still turn out to be consistent with Lyman-$\alpha$ BAO. We await future DESI \cite{DESI:2023ytc} data releases to ascertain if the non-evolving $H(z)$ trend in high redshift CC data is physical or not. 

Finally, we come to our third and main result outlined in section \ref{sec:tension}. We have revisited a $\sim 2 \sigma$ tension between high redshift CC and BAO data reported in \cite{Colgain:2022rxy}, where the significance was estimated through mock simulations. Here, we have upgraded the BAO data to the latest constraints and again  recover a $>2 \sigma$ discrepancy in $(H_0, \Omega_m)$ with different methodology. This provides a consistency check that there is evolution in OHD between low and high redshifts in the late Universe. Note, this evolution runs contrary to the non-evolving $H(z)$ seen in high redshift CC data because it assumes Planck has accurately constrained the high redshift behaviour of the Hubble parameter in (\ref{eq:lcdm}). Nevertheless, both with and without a Planck prior on $\Omega_m h^2$, evolution at $ \gtrsim 2 \sigma$ is evident in OHD data. It should be stressed that evolution is evident in PDFs of best fit $\Lambda$CDM parameters fitted to a large number of Planck-$\Lambda$CDM mocks \cite{Colgain:2022tql}, so evolution in observed data can be expected. It is imperative to revisit the remaining observations in \cite{Colgain:2022rxy, Malekjani:2023dky} in order to confirm the significance of $\sim 2 \sigma$ hints of evolution found separately in Type Ia SN and QSO data sets. 




\acknowledgments
We would like to thank Adri\`a G\'omez-Valent for discussions and comments on the draft. We thank Gabriela Marques, Mike Hudson and Matteo Viel for related discussions on late Universe evolution in $S_8$. E\'OC thanks Yonsei University and Asia Pacific Center for Theoretical Physics for hospitality. 
This article/publication is based upon work from COST Action CA21136 – “Addressing observational tensions in cosmology with systematics and fundamental physics (CosmoVerse)”, supported by COST (European Cooperation in Science and Technology). SP and MMShJ acknowledge SarAmadan grant No. ISEF/M/401332. MMShJ thanks the support from ICTP associates office (under Senior Associate program) and ICTP HECAP section for hospitality.  


\appendix
\section{Fisher Matrix}
\label{sec:fisher}
Consider the $\chi^2$ (\ref{eq:chi2}). 
Defining $H_{\textrm{model}}(z) = H_0 \sqrt{1-\Omega_m + \Omega_m (1+z)^3}$ and $Q_i$ as in \eqref{eq:Q}, we can now work out the derivatives
\begin{equation}
    \begin{split}
\partial_{H_0} Q_i &= -\sqrt{1-\Omega_m + \Omega_m (1+z_i)^3}, \\  \partial_{\Omega_m} Q_i &= - \frac{1}{2} H_0 (z_i^3 + 3 z_i^2 + 3 z_i)/\sqrt{1-\Omega_m + \Omega_m (1+z_i)^3}, \\
\partial^2_{H_0} Q_i &= 0, \\
\partial_{H_0} \partial_{\Omega_m} Q_i &= - \frac{1}{2} (z_i^3 + 3 z_i^2 + 3 z_i)/\sqrt{1-\Omega_m + \Omega_m (1+z_i)^3}, \\
\partial^2_{\Omega_m} Q_i =& \frac{1}{4} H_0 (z_i^3 + 3 z_i^2 + 3 z_i)^2/(1-\Omega_m + \Omega_m (1+z_i)^3)^{\frac{3}{2}}.      
    \end{split}
\end{equation}
We can then define the Fisher matrix 
\be
F_{ij} = \frac{1}{2} \frac{\partial^2 \chi^2(H_0, \Omega_m)}{\partial p_i \partial p_j}
\ee
where $p_i \in \{ H_0, \Omega_m \}$. Note that the Fisher matrix is evaluated on the best fit parameters. The result is a $2 \times 2$ matrix, which one inverts and the estimated errors are the square root of the diagonal entries. 








\begin{thebibliography}{99}

\bibitem{Planck:2018vyg}
N.~Aghanim \textit{et al.} [Planck],
``Planck 2018 results. VI. Cosmological parameters,''
Astron. Astrophys. \textbf{641} (2020), A6
% doi:10.1051/0004-6361/201833910
%[arXiv:1807.06209 [astro-ph.CO]].

\bibitem{Riess:1998cb}
A.~G.~Riess \textit{et al.} [Supernova Search Team],
``Observational evidence from supernovae for an accelerating universe and a cosmological constant,''
Astron. J. \textbf{116} (1998), 1009-1038
% doi:10.1086/300499
%[arXiv:astro-ph/9805201 [astro-ph]].
%13031 citations counted in INSPIRE as of 02 Feb 2021

\bibitem{Perlmutter:1998np}
S.~Perlmutter \textit{et al.} [Supernova Cosmology Project],
``Measurements of $\Omega$ and $\Lambda$ from 42 high redshift supernovae,''
Astrophys. J. \textbf{517} (1999), 565-586
% doi:10.1086/307221
%[arXiv:astro-ph/9812133 [astro-ph]].
%13057 citations counted in INSPIRE as of 02 Feb 2021

\bibitem{Eisenstein:2005su}
D.~J.~Eisenstein \textit{et al.} [SDSS],
``Detection of the Baryon Acoustic Peak in the Large-Scale Correlation Function of SDSS Luminous Red Galaxies,''
Astrophys. J. \textbf{633} (2005), 560-574
%doi:10.1086/466512
%[arXiv:astro-ph/0501171 [astro-ph]].
%3380 citations counted in INSPIRE as of 08 Oct 2020

\bibitem{Riess:2021jrx}
A.~G.~Riess, W.~Yuan, L.~M.~Macri, D.~Scolnic, D.~Brout, S.~Casertano, D.~O.~Jones, Y.~Murakami, L.~Breuval and T.~G.~Brink, \textit{et al.}
``A Comprehensive Measurement of the Local Value of the Hubble Constant with 1 km s$^{?1}$ Mpc$^{?1}$ Uncertainty from the Hubble Space Telescope and the SH0ES Team,''
Astrophys. J. Lett. \textbf{934} (2022) no.1, L7
%doi:10.3847/2041-8213/ac5c5b
%[arXiv:2112.04510 [astro-ph.CO]].
%370 citations counted in INSPIRE as of 09 Jan 2023

\bibitem{Freedman:2021ahq}
W.~L.~Freedman,
``Measurements of the Hubble Constant: Tensions in Perspective,''
Astrophys. J. \textbf{919} (2021) no.1, 16
%doi:10.3847/1538-4357/ac0e95
%[arXiv:2106.15656 [astro-ph.CO]].
%179 citations counted in INSPIRE as of 09 Jan 2023

\bibitem{Pesce:2020xfe}
D.~W.~Pesce, J.~A.~Braatz, M.~J.~Reid, A.~G.~Riess, D.~Scolnic, J.~J.~Condon, F.~Gao, C.~Henkel, C.~M.~V.~Impellizzeri and C.~Y.~Kuo, \textit{et al.}
%``The Megamaser Cosmology Project. XIII. Combined Hubble constant constraints,''
Astrophys. J. Lett. \textbf{891} (2020) no.1, L1
%doi:10.3847/2041-8213/ab75f0
%[arXiv:2001.09213 [astro-ph.CO]].
%96 citations counted in INSPIRE as of 12 Jul 2021

\bibitem{Blakeslee:2021rqi}
J.~P.~Blakeslee, J.~B.~Jensen, C.~P.~Ma, P.~A.~Milne and J.~E.~Greene,
%``The Hubble Constant from Infrared Surface Brightness Fluctuation Distances,''
Astrophys. J. \textbf{911} (2021) no.1, 65
%doi:10.3847/1538-4357/abe86a
%[arXiv:2101.02221 [astro-ph.CO]].
%11 citations counted in INSPIRE as of 12 Jul 2021

\bibitem{Kourkchi:2020iyz}
E.~Kourkchi, R.~B.~Tully, G.~S.~Anand, H.~M.~Courtois, A.~Dupuy, J.~D.~Neill, L.~Rizzi and M.~Seibert,
%``Cosmicflows-4: The Calibration of Optical and Infrared Tully\textendash{}Fisher Relations,''
Astrophys. J. \textbf{896} (2020) no.1, 3
%doi:10.3847/1538-4357/ab901c
%[arXiv:2004.14499 [astro-ph.GA]].
%15 citations counted in INSPIRE as of 12 Jul 2021

\bibitem{HSC:2018mrq}
C.~Hikage \textit{et al.} [HSC],
``Cosmology from cosmic shear power spectra with Subaru Hyper Suprime-Cam first-year data,''
Publ. Astron. Soc. Jap. \textbf{71}, 43  (2019).
%doi:10.1093/pasj/psz010

\bibitem{KiDS:2020suj}
M.~Asgari \textit{et al.} [KiDS],
``KiDS-1000 Cosmology: Cosmic shear constraints and comparison between two point statistics,''
Astron. Astrophys. \textbf{645} (2021), A104
%doi:10.1051/0004-6361/202039070
%[arXiv:2007.15633 [astro-ph.CO]].
%113 citations counted in INSPIRE as of 18 Aug 2021

\bibitem{DES:2021wwk}
T.~M.~C.~Abbott \textit{et al.} [DES],
``Dark Energy Survey Year 3 results: Cosmological constraints from galaxy clustering and weak lensing,''
Phys. Rev. D \textbf{105} (2022) no.2, 023520
%doi:10.1103/PhysRevD.105.023520
%[arXiv:2105.13549 [astro-ph.CO]].
%519 citations counted in INSPIRE as of 14 Jul 2023

\bibitem{Boruah:2019icj}
S.~S.~Boruah, M.~J.~Hudson and G.~Lavaux,
``Cosmic flows in the nearby Universe: new peculiar velocities from SNe and cosmological constraints,''
Mon. Not. Roy. Astron. Soc. \textbf{498} (2020) no.2, 2703-2718
%doi:10.1093/mnras/staa2485
%[arXiv:1912.09383 [astro-ph.CO]].
%54 citations counted in INSPIRE as of 14 Jul 2023

\bibitem{Said:2020epb}
K.~Said, M.~Colless, C.~Magoulas, J.~R.~Lucey and M.~J.~Hudson,
``Joint analysis of 6dFGS and SDSS peculiar velocities for the growth rate of cosmic structure and tests of gravity,''
Mon. Not. Roy. Astron. Soc. \textbf{497} (2020) no.1, 1275-1293
%doi:10.1093/mnras/staa2032
%[arXiv:2007.04993 [astro-ph.CO]].
%49 citations counted in INSPIRE as of 14 Jul 2023

\bibitem{Perivolaropoulos:2021jda}
L.~Perivolaropoulos and F.~Skara,
``Challenges for \ensuremath{\Lambda}CDM: An update,''
New Astron. Rev. \textbf{95}, 101659  (2022).
%doi:10.1016/j.newar.2022.101659
%\href{https://arxiv.org/abs/2105.05208}{2105.05208}

\bibitem{Abdalla:2022yfr}
E.~Abdalla, G.~Franco Abell\'an, A.~Aboubrahim, A.~Agnello, O.~Akarsu, Y.~Akrami, G.~Alestas, D.~Aloni, L.~Amendola and L.~A.~Anchordoqui, \textit{et al.}
``Cosmology intertwined: A review of the particle physics, astrophysics, and cosmology associated with the cosmological tensions and anomalies,''
JHEAp \textbf{34}, 49  (2022).
%doi:10.1016/j.jheap.2022.04.002
%\href{https://arxiv.org/abs/2203.06142}{2203.06142}

\bibitem{Phillips:1993ng}
M.~M.~Phillips,
``The absolute magnitudes of Type IA supernovae,''
Astrophys. J. Lett. \textbf{413} (1993), L105-L108
%doi:10.1086/186970
%1245 citations counted in INSPIRE as of 24 Aug 2021

\bibitem{NearbySupernovaFactory:2018qkd}
M.~Rigault \textit{et al.} [Nearby Supernova Factory],
``Strong Dependence of Type Ia Supernova Standardization on the Local Specific Star Formation Rate,''
Astron. Astrophys. \textbf{644} (2020), A176
%doi:10.1051/0004-6361/201730404
%[arXiv:1806.03849 [astro-ph.CO]].
%143 citations counted in INSPIRE as of 20 Jul 2023

\bibitem{Kang:2019azh}
Y.~Kang, Y.~W.~Lee, Y.~L.~Kim, C.~Chung and C.~H.~Ree,
``Early-type Host Galaxies of Type Ia Supernovae. II. Evidence for Luminosity Evolution in Supernova Cosmology,''
Astrophys. J. \textbf{889} (2020) no.1, 8
%doi:10.3847/1538-4357/ab5afc
%[arXiv:1912.04903 [astro-ph.GA]].
%56 citations counted in INSPIRE as of 20 Jul 2023

\bibitem{Brout:2020msh}
D.~Brout and D.~Scolnic,
``It\textquoteright{}s Dust: Solving the Mysteries of the Intrinsic Scatter and Host-galaxy Dependence of Standardized Type Ia Supernova Brightnesses,''
Astrophys. J. \textbf{909} (2021) no.1, 26
%doi:10.3847/1538-4357/abd69b
%[arXiv:2004.10206 [astro-ph.CO]].
%82 citations counted in INSPIRE as of 20 Jul 2023

\bibitem{Lee:2021txi}
Y.~W.~Lee, C.~Chung, P.~Demarque, S.~Park, J.~Son and Y.~Kang,
``Evidence for strong progenitor age dependence of type Ia supernova luminosity standardization process,''
Mon. Not. Roy. Astron. Soc. \textbf{517} (2022) no.2, 2697-2708
%doi:10.1093/mnras/stac2840
%[arXiv:2107.06288 [astro-ph.GA]].
%5 citations counted in INSPIRE as of 20 Jul 2023


\bibitem{Krishnan:2020vaf}
C.~Krishnan, E.~\'O~Colg\'ain, M.~M.~Sheikh-Jabbari and T.~Yang,
``Running Hubble Tension and a H0 Diagnostic,''
Phys. Rev. D \textbf{103} (2021) no.10, 103509
%doi:10.1103/PhysRevD.103.103509
%[arXiv:2011.02858 [astro-ph.CO]].
%65 citations counted in INSPIRE as of 14 Jul 2023 

\bibitem{Krishnan:2022fzz}
C.~Krishnan and R.~Mondol,
``$H_0$ as a Universal FLRW Diagnostic,''
[arXiv:2201.13384 [astro-ph.CO]].
%12 citations counted in INSPIRE as of 14 Jul 2023

%\bibitem{Liao:2020zko}
%K.~Liao, A.~Shafieloo, R.~E.~Keeley and E.~V.~Linder,
%``Determining Model-independent H 0 and Consistency Tests,''
%Astrophys. J. Lett. \textbf{895} (2020) no.2, L29
%doi:10.3847/2041-8213/ab8dbb
%[arXiv:2002.10605 [astro-ph.CO]].
%51 citations counted in INSPIRE as of 14 Jul 2023

%\bibitem{Montani:2023xpd}
%G.~Montani, M.~De Angelis, F.~Bombacigno and N.~Carlevaro,
%``Metric $f(R)$ gravity with dynamical dark energy as a paradigm for the Hubble Tension,''
%[arXiv:2306.11101 [gr-qc]].
%1 citations counted in INSPIRE as of 14 Jul 2023

\bibitem{Wong:2019kwg}
K.~C.~Wong, S.~H.~Suyu, G.~C.~F.~Chen, C.~E.~Rusu, M.~Millon, D.~Sluse, V.~Bonvin, C.~D.~Fassnacht, S.~Taubenberger and M.~W.~Auger, \textit{et al.}
``H0LiCOW \textendash{} XIII. A 2.4 per cent measurement of H0 from lensed quasars: 5.3\ensuremath{\sigma} tension between early- and late-Universe probes,''
Mon. Not. Roy. Astron. Soc. \textbf{498} (2020) no.1, 1420-1439
%doi:10.1093/mnras/stz3094
%[arXiv:1907.04869 [astro-ph.CO]].
%804 citations counted in INSPIRE as of 18 May 2023

\bibitem{Millon:2019slk}
M.~Millon, A.~Galan, F.~Courbin, T.~Treu, S.~H.~Suyu, X.~Ding, S.~Birrer, G.~C.~F.~Chen, A.~J.~Shajib and D.~Sluse, \textit{et al.}
``TDCOSMO. I. An exploration of systematic uncertainties in the inference of $H_0$ from time-delay cosmography,''
Astron. Astrophys. \textbf{639} (2020), A101
%doi:10.1051/0004-6361/201937351
%[arXiv:1912.08027 [astro-ph.CO]].
%114 citations counted in INSPIRE as of 18 May 2023

\bibitem{Sluse:2003iy}
D.~Sluse, J.~Surdej, J.~F.~Claeskens, D.~Hutsemekers, C.~Jean, F.~Courbin, T.~Nakos, M.~Billeres and S.~V.~Khmil,
``A Quadruply imaged quasar with an optical Einstein ring candidate: 1RXS J113155.4-123155,''
Astron. Astrophys. \textbf{406} (2003), L43-L46
%doi:10.1051/0004-6361:20030904
%[arXiv:astro-ph/0307345 [astro-ph]].
%83 citations counted in INSPIRE as of 14 Jul 2023

\bibitem{Shajib:2023uig}
A.~J.~Shajib, P.~Mozumdar, G.~C.~F.~Chen, T.~Treu, M.~Cappellari, S.~Knabel, S.~H.~Suyu, V.~N.~Bennert, J.~A.~Frieman and D.~Sluse, \textit{et al.}
``TDCOSMO. XIII. Improved Hubble constant measurement from lensing time delays using spatially resolved stellar kinematics of the lens galaxy,''
Astron. Astrophys. \textbf{673} (2023), A9
%doi:10.1051/0004-6361/202345878
%[arXiv:2301.02656 [astro-ph.CO]].
%3 citations counted in INSPIRE as of 18 May 2023

\bibitem{Dainotti:2021pqg}
M.~G.~Dainotti, B.~De Simone, T.~Schiavone, G.~Montani, E.~Rinaldi and G.~Lambiase,
``On the Hubble constant tension in the SNe Ia Pantheon sample,''
Astrophys. J. \textbf{912}, 150  (2021).
%doi:10.3847/1538-4357/abeb73


\bibitem{Colgain:2022nlb}
E.~\'O~Colg\'ain, M.~M.~Sheikh-Jabbari, R.~Solomon, G.~Bargiacchi, S.~Capozziello, M.~G.~Dainotti and D.~Stojkovic,
``Revealing intrinsic flat \ensuremath{\Lambda}CDM biases with standardizable candles,''
Phys. Rev. D \textbf{106}, L041301  (2022).
%doi:10.1103/PhysRevD.106.L041301

\bibitem{Colgain:2022rxy}
E.~\'O~Colg\'ain, M.~M.~Sheikh-Jabbari, R.~Solomon, M.~G.~Dainotti and D.~Stojkovic,
``Putting Flat $\Lambda$CDM In The (Redshift) Bin,''
[arXiv:2206.11447 [astro-ph.CO]].
%42 citations counted in INSPIRE as of 14 Jul 2023

%\cite{Colgain:2022tql}
%\bibitem{Colgain:2022tql}
%E.~\'O.~Colg\'ain, M.~M.~Sheikh-Jabbari and R.~Solomon,
%``High redshift \ensuremath{\Lambda}CDM cosmology: To bin or not to bin?,''
%Phys. Dark Univ. \textbf{40} (2023), 101216
%doi:10.1016/j.dark.2023.101216
%[arXiv:2211.02129 [astro-ph.CO]].
%10 citations counted in INSPIRE as of 25 Jul 2023


\bibitem{Malekjani:2023dky}
M.~Malekjani, R.~M.~Conville, E.~\'O.~Colg\'ain, S.~Pourojaghi and M.~M.~Sheikh-Jabbari,
``Negative Dark Energy Density from High Redshift Pantheon+ Supernovae,''
[arXiv:2301.12725 [astro-ph.CO]].
%13 citations counted in INSPIRE as of 17 Jul 2023

\bibitem{Hu:2022kes}
J.~P.~Hu and F.~Y.~Wang,
``Revealing the late-time transition of H0: relieve the Hubble crisis,''
Mon. Not. Roy. Astron. Soc. \textbf{517}, 576  (2022).

\bibitem{Jia:2022ycc}
X.~D.~Jia, J.~P.~Hu and F.~Y.~Wang,
``Evidence of a decreasing trend for the Hubble constant,''
Astron. Astrophys. \textbf{674} (2023), A45
%doi:10.1051/0004-6361/202346356
%[arXiv:2212.00238 [astro-ph.CO]].
%10 citations counted in INSPIRE as of 17 Jul 2023

\bibitem{Krishnan:2020obg}
C.~Krishnan, E.~\'O~Colg\'ain, Ruchika, A.~A.~Sen, M.~M.~Sheikh-Jabbari and T.~Yang,
``Is there an early Universe solution to Hubble tension?,''
Phys. Rev. D \textbf{102} (2020) no.10, 103525
%doi:10.1103/PhysRevD.102.103525
%[arXiv:2002.06044 [astro-ph.CO]].
%69 citations counted in INSPIRE as of 17 Jul 2023

\bibitem{Dainotti:2022bzg}
M.~G.~Dainotti, B.~De Simone, T.~Schiavone, G.~Montani, E.~Rinaldi, G.~Lambiase, M.~Bogdan and S.~Ugale,
``On the Evolution of the Hubble Constant with the SNe Ia Pantheon Sample and Baryon Acoustic Oscillations: A Feasibility Study for GRB-Cosmology in 2030,''
Galaxies \textbf{10}, 24  (2022).
%doi:10.3390/galaxies10010024

\bibitem{Risaliti:2015zla}
G.~Risaliti and E.~Lusso,
``A Hubble Diagram for Quasars,''
Astrophys. J. \textbf{815} (2015), 33
%doi:10.1088/0004-637X/815/1/33
%[arXiv:1505.07118 [astro-ph.CO]].
%146 citations counted in INSPIRE as of 16 Jun 2023

\bibitem{Risaliti:2018reu}
G.~Risaliti and E.~Lusso,
``Cosmological constraints from the Hubble diagram of quasars at high redshifts,''
Nature Astron. \textbf{3}, 272  (2019).

\bibitem{Lusso:2020pdb}
E.~Lusso, G.~Risaliti, E.~Nardini, G.~Bargiacchi, M.~Benetti, S.~Bisogni, S.~Capozziello, F.~Civano, L.~Eggleston and M.~Elvis, \textit{et al.}
``Quasars as standard candles III. Validation of a new sample for cosmological studies,''
Astron. Astrophys. \textbf{642}, A150  (2020).


\bibitem{Yang:2019vgk}
T.~Yang, A.~Banerjee and E.~\'O~Colg\'ain,
``Cosmography and flat $\Lambda$CDM tensions at high redshift,''
Phys. Rev. D \textbf{102}, 123532  (2020).

\bibitem{Khadka:2020vlh}
N.~Khadka and B.~Ratra,
``Using quasar X-ray and UV flux measurements to constrain cosmological model parameters,''
Mon. Not. Roy. Astron. Soc. \textbf{497}, 263  (2020).


\bibitem{Khadka:2020tlm}
N.~Khadka and B.~Ratra,
``Determining the range of validity of quasar X-ray and UV flux measurements for constraining cosmological model parameters,''
Mon. Not. Roy. Astron. Soc. \textbf{502}, 6140  (2021).


\bibitem{Khadka:2021xcc}
N.~Khadka and B.~Ratra,
``Do quasar X-ray and UV flux measurements provide a useful test of cosmological models?,''
Mon. Not. Roy. Astron. Soc. \textbf{510}, 2753  (2022).
%doi:10.1093/mnras/stab3678

\bibitem{Pourojaghi:2022zrh}
S.~Pourojaghi, N.~F.~Zabihi and M.~Malekjani,
``Can high-redshift Hubble diagrams rule out the standard model of cosmology in the context of cosmography?,''
Phys. Rev. D \textbf{106}, 123523  (2022).


\bibitem{Zajacek:2023qjm}
M.~Zaja\v{c}ek, B.~Czerny, N.~Khadka, R.~Prince, S.~Panda, M.~L.~Mart\'\i{}nez-Aldama and B.~Ratra,
``Extinction biases quasar luminosity distances determined from quasar UV and X-ray flux measurements,''
[arXiv:2305.08179 [astro-ph.GA]].
%0 citations counted in INSPIRE as of 17 Jul 2023

\bibitem{Pasten:2023rpc}
E.~Past\'en and V.~H.~C\'ardenas,
``Testing \ensuremath{\Lambda}CDM cosmology in a binned universe: Anomalies in the deceleration parameter,''
Phys. Dark Univ. \textbf{40} (2023), 101224
%doi:10.1016/j.dark.2023.101224
%[arXiv:2301.10740 [astro-ph.CO]].

\bibitem{Wagner:2022etu}
J.~Wagner,
``Casting the $H_0$ tension as a fitting problem of cosmologies,''
[arXiv:2203.11219 [astro-ph.CO]].
%5 citations counted in INSPIRE as of 28 Jul 2023

\bibitem{Sakr:2023hrl}
Z.~Sakr,
``One matter density discrepancy to alleviate them all or further trouble for $\Lambda$CDM model,''
[arXiv:2305.02846 [astro-ph.CO]].
%0 citations counted in INSPIRE as of 24 Jul 2023


\bibitem{Colgain:2022tql}
E.~\'O~Colg\'ain, M.~M.~Sheikh-Jabbari and R.~Solomon,
``High redshift \ensuremath{\Lambda}CDM cosmology: To bin or not to bin?,''
Phys. Dark Univ. \textbf{40} (2023), 101216
%doi:10.1016/j.dark.2023.101216
[arXiv:2211.02129 [astro-ph.CO]].
%10 citations counted in INSPIRE as of 28 Jun 2023

\bibitem{Esposito:2022plo}
M.~Esposito, V.~Ir\v{s}i\v{c}, M.~Costanzi, S.~Borgani, A.~Saro and M.~Viel,
``Weighing cosmic structures with clusters of galaxies and the intergalactic medium,''
Mon. Not. Roy. Astron. Soc. \textbf{515}, 857  (2022).
%doi:10.1093/mnras/stac1825
[arXiv:2202.00974 [astro-ph.CO]].

\bibitem{Adil:2023jtu}
S.~A.~Adil, \"O.~Akarsu, M.~Malekjani, E.~\'O~Colg\'ain, S.~Pourojaghi, A.~A.~Sen and M.~M.~Sheikh-Jabbari,
``$S_8$ increases with effective redshift in $\Lambda$CDM cosmology,''
[arXiv:2303.06928 [astro-ph.CO]].
%1 citations counted in INSPIRE as of 14 Jul 2023

\bibitem{ACT:2023dou}
F.~J.~Qu \textit{et al.} [ACT],
``The Atacama Cosmology Telescope: A Measurement of the DR6 CMB Lensing Power Spectrum and its Implications for Structure Growth,''
[arXiv:2304.05202 [astro-ph.CO]].
%10 citations counted in INSPIRE as of 14 Jul 2023

\bibitem{ACT:2023kun}
M.~S.~Madhavacheril \textit{et al.} [ACT],
``The Atacama Cosmology Telescope: DR6 Gravitational Lensing Map and Cosmological Parameters,''
[arXiv:2304.05203 [astro-ph.CO]].
%10 citations counted in INSPIRE as of 14 Jul 2023

\bibitem{ACT:2023ipp}
G.~A.~Marques \textit{et al.} [ACT and DES],
``Cosmological constraints from the tomography of DES-Y3 galaxies with CMB lensing from ACT DR4,''
[arXiv:2306.17268 [astro-ph.CO]].
%0 citations counted in INSPIRE as of 14 Jul 2023

\bibitem{Miyatake:2021qjr}
H.~Miyatake, Y.~Harikane, M.~Ouchi, Y.~Ono, N.~Yamamoto, A.~J.~Nishizawa, N.~Bahcall, S.~Miyazaki and A.~A.~Plazas Malag\'on,
``First Identification of a CMB Lensing Signal Produced by 1.5~Million Galaxies at z\ensuremath{\sim}4: Constraints on Matter Density Fluctuations at High Redshift,''
Phys. Rev. Lett. \textbf{129} (2022) no.6, 061301
%doi:10.1103/PhysRevLett.129.061301
[arXiv:2103.15862 [astro-ph.CO]].
%7 citations counted in INSPIRE as of 25 Jul 2023

\bibitem{Alonso:2023guh}
D.~Alonso, G.~Fabbian, K.~Storey-Fisher, A.~C.~Eilers, C.~Garc\'\i{}a-Garc\'\i{}a, D.~W.~Hogg and H.~W.~Rix,
``Constraining cosmology with the Gaia-unWISE Quasar Catalog and CMB lensing: structure growth,''
[arXiv:2306.17748 [astro-ph.CO]].
%0 citations counted in INSPIRE as of 25 Jul 2023


\bibitem{Herold:2021ksg}
L.~Herold, E.~G.~M.~Ferreira and E.~Komatsu,
``New Constraint on Early Dark Energy from Planck and BOSS Data Using the Profile Likelihood,''
Astrophys. J. Lett. \textbf{929} (2022) no.1, L16
%doi:10.3847/2041-8213/ac63a3
%[arXiv:2112.12140 [astro-ph.CO]].
%43 citations counted in INSPIRE as of 17 Jul 2023

\bibitem{Gomez-Valent:2022hkb}
A.~G\'omez-Valent,
``Fast test to assess the impact of marginalization in Monte~Carlo analyses and its application to cosmology,''
Phys. Rev. D \textbf{106} (2022) no.6, 063506
%doi:10.1103/PhysRevD.106.063506
%[arXiv:2203.16285 [astro-ph.CO]].
%20 citations counted in INSPIRE as of 11 Jul 2023

\bibitem{Meiers:2023gft}
M.~Meiers, L.~Knox and N.~Sch\"oneberg,
``Exploration of the Pre-recombination Universe with a High-Dimensional Model of an Additional Dark Fluid,''
[arXiv:2307.09522 [astro-ph.CO]].
%0 citations counted in INSPIRE as of 22 Jul 2023

\bibitem{Poulin:2018cxd}
V.~Poulin, T.~L.~Smith, T.~Karwal and M.~Kamionkowski,
``Early Dark Energy Can Resolve The Hubble Tension,''
Phys. Rev. Lett. \textbf{122} (2019) no.22, 221301
%doi:10.1103/PhysRevLett.122.221301
%[arXiv:1811.04083 [astro-ph.CO]].
%608 citations counted in INSPIRE as of 17 Jul 2023

\bibitem{Niedermann:2019olb}
F.~Niedermann and M.~S.~Sloth,
``New early dark energy,''
Phys. Rev. D \textbf{103} (2021) no.4, L041303
%doi:10.1103/PhysRevD.103.L041303
[arXiv:1910.10739 [astro-ph.CO]].
%140 citations counted in INSPIRE as of 24 Jul 2023

\bibitem{Jimenez:2001gg}
R.~Jimenez and A.~Loeb,
``Constraining cosmological parameters based on relative galaxy ages,''
Astrophys. J. \textbf{573} (2002), 37-42
%doi:10.1086/340549
%[arXiv:astro-ph/0106145 [astro-ph]].
%598 citations counted in INSPIRE as of 28 Jun 2023

\bibitem{Stern:2009ep}
D.~Stern, R.~Jimenez, L.~Verde, M.~Kamionkowski and S.~A.~Stanford,
``Cosmic Chronometers: Constraining the Equation of State of Dark Energy. I: H(z) Measurements,''
JCAP \textbf{02} (2010), 008
%doi:10.1088/1475-7516/2010/02/008
%[arXiv:0907.3149 [astro-ph.CO]].
%740 citations counted in INSPIRE as of 20 May 2022

\bibitem{Moresco:2012jh}
M.~Moresco, A.~Cimatti, R.~Jimenez, L.~Pozzetti, G.~Zamorani, M.~Bolzonella, J.~Dunlop, F.~Lamareille, M.~Mignoli and H.~Pearce, \textit{et al.}
``Improved constraints on the expansion rate of the Universe up to z\textasciitilde{}1.1 from the spectroscopic evolution of cosmic chronometers,''
JCAP \textbf{08} (2012), 006
%doi:10.1088/1475-7516/2012/08/006
%[arXiv:1201.3609 [astro-ph.CO]].
%508 citations counted in INSPIRE as of 20 May 2022

\bibitem{Zhang:2012mp}
C.~Zhang, H.~Zhang, S.~Yuan, T.~J.~Zhang and Y.~C.~Sun,
``Four new observational $H(z)$ data from luminous red galaxies in the Sloan Digital Sky Survey data release seven,''
Res. Astron. Astrophys. \textbf{14} (2014) no.10, 1221-1233
%doi:10.1088/1674-4527/14/10/002
%[arXiv:1207.4541 [astro-ph.CO]].
%425 citations counted in INSPIRE as of 20 May 2022

\bibitem{Moresco:2016mzx}
M.~Moresco, L.~Pozzetti, A.~Cimatti, R.~Jimenez, C.~Maraston, L.~Verde, D.~Thomas, A.~Citro, R.~Tojeiro and D.~Wilkinson,
``A 6\% measurement of the Hubble parameter at $z\sim0.45$: direct evidence of the epoch of cosmic re-acceleration,''
JCAP \textbf{05} (2016), 014
%doi:10.1088/1475-7516/2016/05/014
%[arXiv:1601.01701 [astro-ph.CO]].
%505 citations counted in INSPIRE as of 17 May 2022

\bibitem{Ratsimbazafy:2017vga}
A.~L.~Ratsimbazafy, S.~I.~Loubser, S.~M.~Crawford, C.~M.~Cress, B.~A.~Bassett, R.~C.~Nichol and P.~V\"ais\"anen,
``Age-dating Luminous Red Galaxies observed with the Southern African Large Telescope,''
Mon. Not. Roy. Astron. Soc. \textbf{467} (2017) no.3, 3239-3254
%doi:10.1093/mnras/stx301
%[arXiv:1702.00418 [astro-ph.CO]].
%162 citations counted in INSPIRE as of 17 May 2022

\bibitem{Borghi:2021rft}
N.~Borghi, M.~Moresco and A.~Cimatti,
``Toward a Better Understanding of Cosmic Chronometers: A New Measurement of H(z) at z \ensuremath{\sim} 0.7,''
Astrophys. J. Lett. \textbf{928} (2022) no.1, L4
%doi:10.3847/2041-8213/ac3fb2
%[arXiv:2110.04304 [astro-ph.CO]].
%10 citations counted in INSPIRE as of 17 May 2022

\bibitem{Jiao:2022aep}
K.~Jiao, N.~Borghi, M.~Moresco and T.~J.~Zhang,
``New Observational H(z) Data from Full-spectrum Fitting of Cosmic Chronometers in the LEGA-C Survey,''
Astrophys. J. Suppl. \textbf{265} (2023) no.2, 48
%doi:10.3847/1538-4365/acbc77
%[arXiv:2205.05701 [astro-ph.CO]].
%14 citations counted in INSPIRE as of 17 Jul 2023

\bibitem{Tomasetti:2023kek}
E.~Tomasetti, M.~Moresco, N.~Borghi, K.~Jiao, A.~Cimatti, L.~Pozzetti, A.~C.~Carnall, R.~J.~McLure and L.~Pentericci,
``A new measurement of the expansion history of the Universe at z=1.26 with cosmic chronometers in VANDELS,''
[arXiv:2305.16387 [astro-ph.CO]].
%1 citations counted in INSPIRE as of 28 Jun 2023

\bibitem{Moresco:2023zys}
M.~Moresco,
``Addressing the Hubble tension with cosmic chronometers,''
[arXiv:2307.09501 [astro-ph.CO]].
%0 citations counted in INSPIRE as of 24 Jul 2023

\bibitem{Moresco:2020fbm}
M.~Moresco, R.~Jimenez, L.~Verde, A.~Cimatti and L.~Pozzetti,
``Setting the Stage for Cosmic Chronometers. II. Impact of Stellar Population Synthesis Models Systematics and Full Covariance Matrix,''
Astrophys. J. \textbf{898} (2020) no.1, 82
%doi:10.3847/1538-4357/ab9eb0
[arXiv:2003.07362 [astro-ph.GA]].
%57 citations counted in INSPIRE as of 28 Jul 2023

\bibitem{Foreman-Mackey:2012any}
D.~Foreman-Mackey, D.~W.~Hogg, D.~Lang and J.~Goodman,
``emcee: The MCMC Hammer,''
Publ. Astron. Soc. Pac. \textbf{125} (2013), 306-312
%doi:10.1086/670067
%[arXiv:1202.3665 [astro-ph.IM]].
%3393 citations counted in INSPIRE as of 17 Jul 2023


\bibitem{Hou:2020rse}
J.~Hou, A.~G.~S\'anchez, A.~J.~Ross, A.~Smith, R.~Neveux, J.~Bautista, E.~Burtin, C.~Zhao, R.~Scoccimarro and K.~S.~Dawson, \textit{et al.}
``The Completed SDSS-IV extended Baryon Oscillation Spectroscopic Survey: BAO and RSD measurements from anisotropic clustering analysis of the Quasar Sample in configuration space between redshift 0.8 and 2.2,''
Mon. Not. Roy. Astron. Soc. \textbf{500} (2020) no.1, 1201-1221
%:10.1093/mnras/staa3234
%[arXiv:2007.08998 [astro-ph.CO]].
%135 citations counted in INSPIRE as of 28 Jun 2023

\bibitem{Neveux:2020voa}
R.~Neveux, E.~Burtin, A.~de Mattia, A.~Smith, A.~J.~Ross, J.~Hou, J.~Bautista, J.~Brinkmann, C.~H.~Chuang and K.~S.~Dawson, \textit{et al.}
``The completed SDSS-IV extended Baryon Oscillation Spectroscopic Survey: BAO and RSD measurements from the anisotropic power spectrum of the quasar sample between redshift 0.8 and 2.2,''
Mon. Not. Roy. Astron. Soc. \textbf{499} (2020) no.1, 210-229
%doi:10.1093/mnras/staa2780
%[arXiv:2007.08999 [astro-ph.CO]].
%133 citations counted in INSPIRE as of 28 Jun 2023

\bibitem{duMasdesBourboux:2020pck}
H.~du Mas des Bourboux, J.~Rich, A.~Font-Ribera, V.~de Sainte Agathe, J.~Farr, T.~Etourneau, J.~M.~Le Goff, A.~Cuceu, C.~Balland and J.~E.~Bautista, \textit{et al.}
``The Completed SDSS-IV Extended Baryon Oscillation Spectroscopic Survey: Baryon Acoustic Oscillations with Ly\ensuremath{\alpha} Forests,''
Astrophys. J. \textbf{901} (2020) no.2, 153
%doi:10.3847/1538-4357/abb085
%[arXiv:2007.08995 [astro-ph.CO]].
%172 citations counted in INSPIRE as of 28 Jun 2023

\bibitem{Trotta:2017wnx}
R.~Trotta,
``Bayesian Methods in Cosmology,''
[arXiv:1701.01467 [astro-ph.CO]].
%96 citations counted in INSPIRE as of 18 Jul 2023

\bibitem{Moresco:2022phi}
M.~Moresco, L.~Amati, L.~Amendola, S.~Birrer, J.~P.~Blakeslee, M.~Cantiello, A.~Cimatti, J.~Darling, M.~Della Valle and M.~Fishbach, \textit{et al.}
``Unveiling the Universe with emerging cosmological probes,''
Living Rev. Rel. \textbf{25} (2022) no.1, 6
%doi:10.1007/s41114-022-00040-z
%[arXiv:2201.07241 [astro-ph.CO]].
%71 citations counted in INSPIRE as of 16 Jun 2023

\bibitem{DESI:2023ytc}
G.~Adame \textit{et al.} [DESI],
``The Early Data Release of the Dark Energy Spectroscopic Instrument,''
%doi:10.5281/zenodo.7964161
[arXiv:2306.06308 [astro-ph.CO]].
%13 citations counted in INSPIRE as of 26 Jul 2023

\bibitem{Akarsu:2022lhx}
O.~Akarsu, E.~\'O~Colg\'ain, E.~\"Ozulker, S.~Thakur and L.~Yin,
``Inevitable manifestation of wiggles in the expansion of the late Universe,''
Phys. Rev. D \textbf{107} (2023) no.12, 123526
%doi:10.1103/PhysRevD.107.123526
%[arXiv:2207.10609 [astro-ph.CO]].
%6 citations counted in INSPIRE as of 17 Jul 2023

\bibitem{Zhao:2017cud}
G.~B.~Zhao, M.~Raveri, L.~Pogosian, Y.~Wang, R.~G.~Crittenden, W.~J.~Handley, W.~J.~Percival, F.~Beutler, J.~Brinkmann and C.~H.~Chuang, \textit{et al.}
``Dynamical dark energy in light of the latest observations,''
Nature Astron. \textbf{1} (2017) no.9, 627-632
%doi:10.1038/s41550-017-0216-z
%[arXiv:1701.08165 [astro-ph.CO]].
%356 citations counted in INSPIRE as of 17 Jul 2023

\bibitem{Wang:2018fng}
Y.~Wang, L.~Pogosian, G.~B.~Zhao and A.~Zucca,
``Evolution of dark energy reconstructed from the latest observations,''
Astrophys. J. Lett. \textbf{869} (2018), L8
%doi:10.3847/2041-8213/aaf238
%[arXiv:1807.03772 [astro-ph.CO]].
%92 citations counted in INSPIRE as of 17 Jul 2023

\bibitem{Escamilla:2021uoj}
L.~A.~Escamilla and J.~A.~Vazquez,
``Model selection applied to reconstructions of the Dark Energy,''
Eur. Phys. J. C \textbf{83} (2023) no.3, 251
%doi:10.1140/epjc/s10052-023-11404-2
%[arXiv:2111.10457 [astro-ph.CO]].
%13 citations counted in INSPIRE as of 17 Jul 2023

\end{thebibliography}
\end{document}


\end{document}
