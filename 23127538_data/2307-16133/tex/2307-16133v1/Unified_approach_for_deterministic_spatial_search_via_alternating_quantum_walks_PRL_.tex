%\documentclass[aps,physrev,reprint,]{revtex4-2}
\documentclass[aps,prl,twocolumn,superscriptaddress,showpacs,altaffillsymbol,nofootinbib]{revtex4-1} %longbibliography

%%aps,prl,twocolumn,superscriptaddress,showpacs,altaffillsymbol,nofootinbib
\usepackage{amsmath, amsthm, amssymb,amsfonts, graphicx, subfigure}
\usepackage[colorlinks,citecolor=blue,linkcolor=blue, urlcolor=blue, anchorcolor=blue]{hyperref}
\usepackage{makecell}
\usepackage{dcolumn}
\usepackage{mathrsfs}
\usepackage{float}
\usepackage{multirow}
\usepackage{braket}
\usepackage{color}
\usepackage{tcolorbox}
\usepackage[ruled]{algorithm2e}
\usepackage{tikz}
\newtheorem{lemma}{Lemma}
\newtheorem{theorem}{Theorem}
\newtheorem{remark}{Remark}
\newtheorem{definition}{Definition}
\newtheorem{example}{Example}


\begin{document}
\title{Universal  approach to deterministic spatial search via alternating quantum walks}
\author{Qingwen Wang}%
\author{Ying Jiang}%
\author{Shiguang Feng}%
\author{Lvzhou Li}%
\email{Email: lilvzh@mail.sysu.edu.cn}
\affiliation{School of Computer Science and Engineering, Sun Yat-sen University, Guangzhou 510006, China}
	
\begin{abstract}
Spatial search is an important problem in quantum computation, which aims to find a marked vertex on a graph. %We solve the spatial search problem via alternating quantum walks,  
We propose a novel and universal approach for designing deterministic quantum search algorithms on a variety of graphs via alternating quantum walks. The approach divides the search space into a series of subspaces and performs deterministic quantum searching on these subspaces. We highlight the flexibility of our approach by  proving that for Johnson graphs, rook graphs, complete-square graphs and complete bipartite graphs, our quantum algorithms can find the marked vertex with $100\%$ success probability and achieve quadratic speedups over classical algorithms. 
This not only gives an alternative succinct way to prove the existing results, but also leads to new findings on more general graphs.
		
\end{abstract}
	
\maketitle
	%\section{Introduction}
	
\textit{Introduction}.---
 The continuous-time quantum walk (CTQW)  was introduced by Farhi and Gutmann~\cite{RN1}  in 1998, and has been one of the key components in quantum computation ~\cite{Childs2002,Ambainis2003QUANTUMWA,Farhi2008,Childs2010,Reichardt2012,Dheeraj2015,RN3,Dadras2019,Wang2020,Delvecchio2020,Kadian2021,Silva2023}. 
% The spatial search problem introduced by Benioff~\cite{Benioff2000Space} aims to find a marked element in a spatially structured database. %It is an important algorithmic application of quantum walks~\cite{RN2,Aaronson2005,Portugal2013,Chakraborty2016,Osada2020,Benedetti2021,Tanaka2022,RN4}.
In 2004, Childs and Goldstone~\cite{RN2} presented a algorithmic framework using  CTQW to solve the  spatial search problem which aims to find a unknown marked vertex on an underlying graph of specified topology. They showed that the algorithm has $O(\sqrt{N})$ searching time on complete graphs, hypercubes, and $d$-dimensional periodic lattices for $d>4$, where $N$ is the number of vertices in the graph.
In this framework, a Hamiltonian $H$ is constructed using the adjacency matrix of the graph and information about the location of the marked vertex. The algorithm is then to make a quantum system evolve $T$ time from an initial state under the Hamiltonian $H$, where $T$ can be set arbitrarily.
 Since then, many kinds of graphs, e.g., strong regular graphs~\cite{Janmark2014}, complete bipartite graphs~\cite{RN10}, balanced trees~\cite{RN3}, and Johnson graphs~\cite{RN4}, have been studied in this framework. All of these graphs can admit a quadratic quantum speedup by using CTQW. Specially, it is worth mentioning that an exponential algorithmic speedup can be achieved via CTQW for the welded tree problem~\cite{10.1145/780542.780552}.
    
    Recently, Marsh and Wang~\cite{RN5,RN6} presented a framework called \textit{alternating phase-walk} for spatial search that uses an alternating series of CTQWs and marked-vertex phase shifts. In the framework, two Hamiltonians are constructed. One uses the Laplacian matrix of the graph and the other uses the information of the marked element. 
    Then the quantum system evolves alternately under the two Hamiltonians, which is similar to the model of the quantum approximate optimization algorithm (QAOA).
    The authors designed quantum algorithms that achieve quadratic speedups on Johnson graphs $J(n,2)$, rook graphs, and complete-square graphs. Using this framework, Qu et al. studied the case of star graphs and also obtained a quadratic speedup~\cite{RN8}.
 
    In this article, we present a novel approach based on alternating phase-walks to design deterministic quantum spatial search algorithms on a variety of graphs. Taking advantage of the approach, we obtain deterministic quantum search algorithms on Johnson graphs $J(n,k)$ for any fixed $k$, rook graphs, complete-square graphs, and complete bipartite graphs $K(N_1, N_2)$, respectively. All of these algorithms can find the marked vertex with 100\% success probability theoretically and achieve quadratic speedups over classical ones. 
    These results not only cover those about rook graphs and complete-square graphs in~\cite{RN5,RN6} but also generalize from Johnson graphs $J(n,2)$ to $J(n,k)$ for any fixed $k$. A star graph is a kind of complete bipartite graph, so the results in~\cite{RN8} can also be obtained as a direct conclusion of this paper. Our approach is universal in the sense that it subsumes the consequences of~\cite{RN5,RN6,RN8} by giving a more succinct formulation of alternating phase-walks, which significantly simplifies the proofs of previous works. Also, the idea behind the approach may give inspiration for inventing potential techniques that are applicable to more graph classes.
    
	%Therefore, we not only prove the already existing results in~\cite{RN5,RN6,RN8} using a simple alternative way, but also obtain new findings on more general graphs. 
	
%The article is organized as follows: Some preliminaries are first introduced. Then, our search framework and its applications are proposed in sequence. Finally, we conclude the article.
	
	\textit{Preliminaries}.--- 
For quantum search in an unstructured database, one frequently-used way is to perform alternately the %following 
two unitary operators 
\begin{align*}
    U_1(\alpha) & =I-(1-e^{-i\alpha})|s\rangle\langle s|,\\
    U_2(\beta) & =I-(1-e^{-i\beta})|m\rangle\langle m|,
\end{align*}
where $\alpha$ and $\beta$ are real numbers, on the initial state $|s\rangle$ to find the marked element $|m\rangle$.
%$U_1(\alpha)$ and $U_2(\beta)$ on the initial state $|s\rangle$ to find the marked element $|m\rangle$:
%\begin{equation}\label{e1}
%	\begin{aligned}
%		U_1(\alpha) & =I-(1-e^{-i\alpha})|s\rangle\langle s|,\\
%		U_2(\beta) & =I-(1-e^{-i\beta})|m\rangle\langle m|,
%		\nonumber
%	\end{aligned}
%\end{equation}
%where $\alpha$ and $\beta$ are real numbers.
In~\cite{RN9,RN7}, the authors showed that if the value of $|\langle m|s\rangle|$ is known, then we can choose $\alpha=\pi$ and appropriate values for $\beta$ to carry out the search deterministically.
\begin{lemma}[\cite{RN9,RN7}]\label{l1}
	Given two unitary operators $U_1(\pi)$, $U_2(\beta)$, and a positive number $|\langle m|s\rangle|$, where $U_1(\pi) =I-2|s\rangle\langle s|$, $U_2(\beta)=I-(1-e^{-i\beta})|m\rangle\langle m|$, we can find an integer $p \in O(\frac{1}{|\langle m|s\rangle|})$, and real numbers $\gamma$, $\beta_1,\dots,\beta_p$
	%$\beta_k$ $(k\in \{1,2,\dots,p\})$
	such that 
	\[|m\rangle=e^{-i\gamma}\prod_{k=1}^{p}U_1(\pi) U_2(\beta_k)|s\rangle.\]
\end{lemma}


Let $G=(V, E)$ be an undirected connected graph with no loops, where $V=\{1,\dots,N\}$ is the set of vertices. We consider a quantum system whose Hilbert space is an $N$-dimensional space consisting of the vectors $|1\rangle,\dots,|N\rangle$, where each $|v\rangle$ for $v \in \{ 1, 2, \dots, N\}$ denotes the vertex $v$ in $G$. The state of the quantum system at time $t$ is denoted by $|\psi(t)\rangle$. The continuous time quantum walk on $G$ from the initial state $|\psi(0)\rangle$ can be described by the following Schr\"{o}dinger equation of the system:
\begin{equation}\label{e2}
	i\cdot \frac{d\langle v|\psi(t)\rangle}{dt}=\sum_{u\in V}\langle v|H|u\rangle \\ \langle u|\psi(t)\rangle,
\end{equation}
where $H$ is a Hamiltonian satisfying $\langle v|H|u\rangle=0$ when $v$ and $u$ are not adjacent in $G$. This equation means that at time $t$, the change of the amplitude of $|v\rangle$ is only related to the amplitudes of its adjacent vertices. From \eqref{e2}, we see that the continuous time quantum walk over a graph $G$ at time $t$ can be defined by the unitary transformation $U=e^{-iHt}$. One choice of $H$ is the Laplacian matrix $L=D-A$, where $D$ is the degree matrix (a diagonal matrix with $D_{jj}=deg(j)$) and $A$ is the adjacency matrix of $G$. The other choice is to let $H = A$.
	
	\textit{Search framework}.--- 
	Here we prove a theorem that can be used to design deterministic quantum search algorithms on  a variety of graphs. 
	Let $G=(V,E)$ be an undirected connected graph with Laplacian matrix $L$ and no loops, and $m$ a marked vertex in $G$. Our aim is to find the location of $m$ via performing alternately the continuous time quantum walk operator $e^{-iLt}$ and the search operator $e^{-i\theta |m\rangle\langle m|}$ on the initial state $|s\rangle$.
	
	The idea behind our algorithm is to prove that there exist an integer $p$ and real numbers $\gamma$, $\theta_k$, $t_k$ $(k\in \{1, 2, \dots, p\})$ such that the following equation holds:
	\begin{equation}\label{e3}
		|m\rangle=e^{-i\gamma}\prod_{k=1}^{p}e^{-i\theta_k |m\rangle\langle m|}e^{-iLt_k} |s\rangle.\nonumber
	\end{equation}
	%where $|s\rangle$ is the initial state, $H$ is the Hamiltonian that we use to construct the walk operator, $p$ is an integer, and $\theta_k$, $t_k$ $(k\in \{1, 2, \dots, p\})$ are all real numbers.
	
	Let $S$ be a finite multiset of integers. Define $gcd(S)$ to be the greatest common divisor of all nonzero elements in $S$. If there is no nonzero element in $S$, then we define $gcd(S)$ to be 1.
	
	
	\begin{definition}
		Let $M$ be an $N\times N$ Hermitian matrix  with spectral decomposition $M=\sum_{i=1}^{N} \lambda_i |\eta_i \rangle \langle \eta_i|$, where $\lambda_1,\dots,\lambda_N$ are integers such that $\prod_{1\leq i \leq N} \lambda_i =0$. 
		
		(\romannumeral 1) Define $\Lambda_0=\{\lambda_1,\dots,\lambda_N\}$. For $k\geq 0$, we recursively define $\Lambda_{k+1}$ and $\overline{\Lambda}_{k+1}$ as follows
		\begin{equation}\label{e4}
			\Lambda_{k+1}=\{\lambda\in\Lambda_ {k} \mid e^{-i\lambda \frac{\pi}{gcd(\Lambda_ k)}}=1\},
			\nonumber
		\end{equation}
		and \begin{equation}\label{e5}
			\overline{\Lambda}_{k+1}=\{ \lambda\in\Lambda_ {k} \mid e^{-i\lambda \frac{\pi}{gcd(\Lambda_ k)}}=-1\}.
			\nonumber
		\end{equation}
		We use $d(M)$ to denote the least $k$ such that $\Lambda_k$ contains only 0.  
		
		(\romannumeral 2) Let $|m\rangle=\sum_{i=1}^{N}\alpha_i |\eta_i\rangle$ be a vector in span$\{|\eta_1\rangle,\dots,|\eta_N\rangle\}$ where each $\alpha_i$ for $i\in \{1,\dots,N\}$ is a real number such that
		\[
		(\sum_{\lambda_i \in {\Lambda}_k} \alpha_i^2)(\sum_{\lambda_i \in \overline{\Lambda}_k} \alpha_i^2)\neq 0
		\]
		for any $k\in\{1,\dots,d(M)\}$. We define $|w_0\rangle=|m\rangle$, and for $k\in\{1,\dots,d(M)\}$ define
		\begin{equation}\label{e6}
			|w_k\rangle =\frac{1}{\sqrt{\sum_{\lambda_i \in {\Lambda}_k} \alpha_i^2}} \sum_{\lambda_i \in {\Lambda}_k} \alpha_i |\eta_i\rangle
			\nonumber
		\end{equation}
		and \begin{equation}\label{e7}
			|\overline{w}_k\rangle =\frac{1}{\sqrt{\sum_{\lambda_i \in \overline{\Lambda}_k} \alpha_i^2}} \sum_{\lambda_i \in \overline{\Lambda}_k} \alpha_i |\eta_i\rangle.
			\nonumber
		\end{equation}
		\label{de1}
	\end{definition}
	
	
	\begin{example}
		Given the following $6\times 6$ Hermitian matrix
		\begin{equation}
	    \begin{aligned}
		M &=  0|\eta_1 \rangle \langle \eta_1|+1|\eta_2 \rangle \langle \eta_2|+3|\eta_3 \rangle \langle \eta_3|+6|\eta_4 \rangle \langle \eta_4|
	+64|\eta_5 \rangle \langle \eta_5|\\
	&+64|\eta_6 \rangle \langle \eta_6|,
	    \end{aligned}
        \nonumber
		\end{equation}
		the process of computing $d(M)$, $\Lambda_0$, $|w_0\rangle$, and $\Lambda_k$, $\overline{\Lambda}_{k}$, $|w_k\rangle$, $|\overline{w}_{k}\rangle$ for $k\in\{1,2,\dots,d(M)\}$ with $|m\rangle=\sum_{i=1}^{6}\alpha_i |\eta_i\rangle$ where $\alpha_1,\dots,\alpha_6$ are not zero is shown in Fig.~\ref{fig1}.
		% Figure environment removed
	\end{example}
	
	Now we assume that $G$ is both vertex transitive and periodic. Recall that $L=D-A$, where both $D$ and $A$ are symmetric. Obviously $L$ is Hermitian. We can also see that
	\begin{equation}\label{e8}
		L\cdot r=(D-A)\cdot r = 0,
	\end{equation}
	where $r= (1,1,\dots,1)^T$. It implies that 0 is an eigenvalue of $L$. Since  $G$ is periodic, each eigenvalue $\lambda_i$ of $L$ is rational.  In this article, we assume that they are all integers, and this assumption will not affect the generality of our result, which  will be illustrated later. For $L$ with spectral decomposition  $\sum_{i=1}^{N} \lambda_i |\eta_i \rangle \langle \eta_i|$, we define $d(L)$, $\Lambda_0$, and $\Lambda_k$, $\overline{\Lambda}_{k}$, $(k\in\{1,2,\dots,d(L)\})$ as in (\romannumeral 1) of Definition~\ref{de1}.
	In search space span$\{|v_1\rangle,\dots,|v_N\rangle\}$ where each $|v_i\rangle$ denotes a vertex of $G$, $|\eta_1\rangle,\dots,|\eta_N\rangle$ constitute a set of standard orthogonal basis,
	and the marked vertex $|m\rangle$ can be represented by the basis $|m\rangle=\sum_{i=1}^{N}\alpha_i |\eta_i\rangle$.
	Since each $|\eta_i\rangle$ is an eigenvector of $L$ that is both real and symmetric, it follows that each $|\eta_i\rangle$ is a real vector and $\langle \eta_i|m\rangle=\alpha_i$ is a real number. We denote the cardinality of a set $S$ by $|S|$.
	Since $G$ is vertex transitive, which has symmetry for every pair of vertices, for any $k\in\{1,\dots,d(L)\}$, $\sqrt{\sum_{\lambda_i \in \Lambda_k} \alpha_i^2}$ and $\sqrt{\sum_{\lambda_i \in \overline{\Lambda}_k} \alpha_i^2}$ are independent of the location of the marked vertex. Actually, we make some preliminary comments that
	\begin{equation}\label{e14}
		\begin{aligned}
			\sqrt{\sum\nolimits_{\lambda_i \in \Lambda_k} \alpha_i^2}=\sqrt{ \frac{|\Lambda_k|}{N}}, \\
			\sqrt{\sum\nolimits_{\lambda_i \in \overline{\Lambda}_k} \alpha_i^2}=\sqrt{ \frac{|\overline{\Lambda_k}|}{N}},
		\end{aligned}
	\end{equation} 
	where $|\Lambda_k|$ and $|\overline{\Lambda_k}|$ ($k\in\{1,\dots,d(L)\}$) are both positive. 
	For $|m\rangle=\sum_{i=1}^{N}\alpha_i |\eta_i\rangle$, we define $|w_0\rangle$,  $|w_k\rangle$, $|\overline{w}_{k}\rangle$ $(k\in\{1,2,\dots,d(L)\})$ as in (\romannumeral 2) of Definition~\ref{de1}.
	We can observe that $\Lambda_{k}=\Lambda_{k+1} \cup \overline{\Lambda}_{k+1}$, $|w_k\rangle \in span\{|w_{k+1}\rangle, |\overline{w}_{k+1\rangle}\}$ $(k\in \{0,1,\dots,d(L)-1\})$ and $d(L)$ is less than the number of distinct eigenvalues of $L$.
	In the following, we present our main theorem.
	
	
	\begin{theorem}
		Given a periodic vertex transitive graph $G$ with $N$ vertices and Laplacian matrix $L$, we can find an integer $p \in O(2^{d(L)-1}\sqrt{N})$, and real numbers $\gamma$, $\theta_k$, $t_k$ $(k\in \{1,2,\dots,p\})$ such that $|s\rangle=e^{-i\gamma}\prod_{k=1}^{p}e^{-i\theta_k |m\rangle\langle m|} e^{-iLt_k}|m\rangle$, where $|s\rangle $ is the uniform superposition state over all vertices of the graph.
		\label{th1}                       
	\end{theorem}
	
	\begin{proof}
		The idea for the proof is to divide the search space into a series of subspaces and use $e^{-i\theta |m\rangle\langle m|}$ and $e^{-iLt}$ to construct $U_1(\pi)$ and $U_2(\beta)$ (given in Lemma \ref{l1}) on these subspaces.
		First we illustrate the generality of the assumption that each $\lambda_i$ is an integer.
		Since $G$ is periodic, each eigenvalue $\lambda_i$ of $L$ is rational. In fact, for rational numbers $\lambda_i$ $(i\in\{1,2,\dots,N\})$, we can always find a number $q$ such that $q\lambda_i$ $(i\in\{1,2,\dots,N\})$ are all integers and
		\begin{equation}
			e^{-iL t}=\sum_{i=1}^{N} e^{-i\lambda_i  t}  |\eta_i \rangle \langle \eta_i|=\sum_{i=1}^{N} e^{-i(q\lambda_i)  (\frac{1}{q}t)}|\eta_i \rangle \langle \eta_i|,
			\nonumber
			\label{e9}
		\end{equation}
		where $|\eta_i\rangle$ is the corresponding eigenvalue of $\lambda_i$.
		To make the operator be the same as that when the eigenvalues of $L$ are integers $q\lambda_1,\dots,q\lambda_N$, we just need to multiply $t$ by $q$. Therefore we can assume that each $\lambda_i$ is an integer. 
		
		We consider the walk operator $e^{-iLt}$ and let $t=\frac{\pi}{gcd(\Lambda_0)}$, then
		\begin{equation}\label{e10}
			\begin{aligned}
		e^{-iL  \frac{\pi}{gcd(\Lambda_0)}}&=\sum_{i=1}^{N} e^{-i\lambda_i  \frac{\pi}{gcd(\Lambda_0)}}  |\eta_i \rangle \langle \eta_i|\\
		&=\sum_{\lambda_i \in \Lambda_1} |\eta_i \rangle \langle \eta_i|-\sum_{\lambda_i \in \overline{\Lambda}_1} |\eta_i \rangle \langle \eta_i|.
			\end{aligned}
		\end{equation}
		The second equality in \eqref{e10} holds by the following equations:
		\begin{equation}\begin{aligned}
				\Lambda_{0}  & =\{\lambda_1,\dots,\lambda_N\},\\
				\Lambda_{1}  & =\{\lambda \in \Lambda_0 \mid e^{-i\lambda \frac{\pi}{gcd(\Lambda_0)}}=1\},\\
				\overline{\Lambda}_{1} & =\{\lambda \in \Lambda_0 \mid e^{-i\lambda \frac{\pi}{gcd(\Lambda_0)}}=-1\}.
				\label{e11}
				\nonumber
			\end{aligned}	
		\end{equation}
		Recall that \begin{equation}
			|m\rangle=|w_0\rangle \in span\{|w_{1}\rangle ,|\overline{w}_{1}\rangle \}
			\label{e12} 
			\nonumber
		\end{equation}
		and $|w_1\rangle$ ($|\overline{w}_{1}\rangle$) are linear combinations of eigenvectors $|\eta_1\rangle,\dots,|\eta_N\rangle$ whose corresponding eigenvalues are in $\Lambda_1$  ($\overline{\Lambda}_{1}$). In the subspace span$\{|w_{1}\rangle ,|\overline{w}_{1}\rangle \}$, we have
		\begin{equation}\label{e13}
			\begin{aligned}		
				e^{-iL \frac{\pi}{gcd(\Lambda_0)}} & =2|{w}_{1}\rangle \langle {w}_{1}|-I=e^{i\pi}(I-2|{w}_{1}\rangle \langle {w}_{1}|) ,\\
				e^{-i\theta|m\rangle \langle m| } & =I- (1-e^{-i\theta})|w_0\rangle \langle w_0| .
			\end{aligned}
		\end{equation}
		From \eqref{e14}, we have
		\begin{equation}\label{e15}
			\langle w_k|w_{k+1}\rangle=\frac{\sqrt{\sum_{\lambda_i \in \Lambda_{k+1}} \alpha_i^2}}{\sqrt{\sum_{\lambda_i \in \Lambda_k} \alpha_i^2}} =\frac{\sqrt{|\Lambda_{k+1}|} }{\sqrt{|\Lambda_{k}|}},
		\end{equation}
		which is a number independent of $|m\rangle$. According to \eqref{e13}, \eqref{e15} and Lemma~\ref{l1}, we can find parameters $p \in O(\frac{1}{|\langle w_{1}|w_0\rangle|})$, $\gamma$, $t=\frac{\pi}{gcd( \Lambda_0)}$ and $\theta_k$ $(k\in \{1,2,\dots,p\})$ such that
		\begin{equation}
			|w_1\rangle=e^{-i\gamma}\Big(\prod_{k=1}^{p}e^{-i\theta_k |m\rangle\langle m|} e^{-iLt}\Big)|w_0\rangle.
			\label{e16}
		\end{equation}
		Next, we shall prove the following fact.
		
		If there are parameters $p$, $\gamma$, and $\theta_k$, $t_k$ $(k\in \{1,2,\dots,p\})$ satisfying
		\begin{equation}\label{e17}
			|w_i\rangle=e^{-i\gamma}\prod_{k=1}^{p}e^{-i\theta_k |m\rangle\langle m|} e^{-iLt_k}|w_0\rangle,
		\end{equation}
		where $1\leq i \leq d(L)-1$, then we can find parameters $p'\in O(\frac{2p}{|\langle w_i|w_{i+1}\rangle|})$, $\gamma'$, and $\theta'_k$, $t'_k$, $(k\in \{1,2,\dots, p'\})$ such that
		\begin{equation}	|w_{i+1}\rangle=e^{-i\gamma'}\prod_{k=1}^{p'}e^{-i\theta'_k |m\rangle\langle m|} e^{-iLt'_k}|w_0\rangle.
			\label{e18}
		\end{equation}
		We denote $e^{-i\gamma}\prod_{k=1}^{p}e^{-i\theta_k |m\rangle\langle m|} e^{-iLt_k}$ in \eqref{e17} by $A_i$. We have
		\begin{equation}\label{e19}
			\begin{aligned}
				A_i e^{-i\theta |m\rangle\langle m|} A_i^{\dagger} &=  A_i(I- (1-e^{-i\theta})|w_0\rangle \langle w_0|) A_i^{\dagger} \\
				& =I- (1-e^{-i\theta})|w_i\rangle \langle w_i|.
				\nonumber
			\end{aligned}
		\end{equation}
		Recall that $ |w_i\rangle \in span\{|w_{i+1}\rangle ,|\overline{w}_{i+1}\rangle \}$. In this subspace,
		\begin{equation}
			e^{-iL \frac{\pi}{gcd(\Lambda_ {i})}}=2|{w}_{i+1}\rangle \langle {w}_{i+1}|-I=e^{i\pi}(I-2|{w}_{i+1}\rangle \langle {w}_{i+1}|) .
			\label{20}
			\nonumber
		\end{equation}
		By Lemma~\ref{l1}, there are parameters $ p''\in O(\frac{1}{|\langle w_i|w_{i+1}\rangle|})$, $\gamma''$, $t=\frac{\pi}{gcd(\Lambda_ {i})}$ and $\theta''_k$ $(k\in \{1,2,\dots,p''\})$ such that 
		\begin{equation}\label{e21}
			\begin{aligned}
				|w_{i+1}\rangle & =e^{-i\gamma''}\Big(\prod_{k=1}^{p''} A_i e^{-i\theta''_k |m\rangle\langle m|} A_i^{\dagger}e^{-iLt}\Big)|w_i\rangle \\
				&=e^{-i\gamma''}\Bigr(\prod_{k=1}^{p''}A_i e^{-i\theta''_k |m\rangle\langle m|} A_i^{\dagger}	e^{-iLt}\Bigl) A_i|w_0\rangle.
			\end{aligned}
		\end{equation}
		By changing $A_i$ to $e^{-i\gamma}\prod_{k=1}^{p}e^{-i\theta_k |m\rangle\langle m|} e^{-iLt_k}$ in \eqref{e21}, we can get the parameters satisfy \eqref{e18}. This proves the fact. By \eqref{e16} and using the fact recursively, we can find parameters $p \in O(\frac{1}{2}\prod_{k=0}^{d(L)-1}\frac{2}{|\langle w_{k+1}|w_k\rangle|})$, $\gamma$, and $\theta_k$, $t_k$ $(k\in \{1,2,\dots,p\})$ such that
		\begin{equation} 
			\begin{aligned}
			|w_{d(L)}\rangle&=e^{-i\gamma}\prod_{k=1}^{p}e^{-i\theta_k |m\rangle\langle m|} e^{-iLt_k}|w_0\rangle\\
			&=e^{-i\gamma}\prod_{k=1}^{p}e^{-i\theta_k |m\rangle\langle m|} e^{-iLt_k}|m\rangle.
			\label{e22}
		\end{aligned}
		 \end{equation}
		For an undirected connected graph, $0$ is a simple eigenvalue of $L$, and by \eqref{e8}, $|s\rangle$ is its eigenvector.
		%By Definition~\ref{de1}, 
		We have
		\begin{equation}\label{e23} 
			\begin{aligned}
			|w_{d(L)}\rangle&=\frac{1}{\sqrt{\sum_{\lambda_i \in {\Lambda}_d(L)} \alpha_i^2}} \sum_{\lambda_i \in {\Lambda}_d(L)}\alpha_i |\eta_i\rangle\\
			&=\frac{1}{\sqrt{\sum_{\lambda_i =0} \alpha_i^2}} \sum_{\lambda_i =0} \alpha_i |\eta_i\rangle=e^{i\gamma_0}|s\rangle, 
			\nonumber
		\end{aligned}
		\end{equation} where $e^{i\gamma_0}=1$ or $-1$
		and we can ignore it since it is a global phase.
		The upper bound of times that we call the search operator is
		\begin{equation}\label{e24} 
			\begin{aligned}
				\frac{1}{2}\prod_{k=0}^{d(L)-1}\frac{2}{|\langle w_{k+1}|w_k\rangle|} & =\frac{1}{2}\prod_{k=0}^{d(L)-1}\frac{2\sqrt{\sum_{\lambda_ \in \Lambda_k} \alpha_i^2}}{\sqrt{\sum_{\lambda_i \in \Lambda_{k+1}} \alpha_i^2}} \\
				& = \frac{2^{d(L)-1}\sqrt{\sum_{\lambda_ \in \Lambda_0} \alpha_i^2}}{\sqrt{\sum_{\lambda_i \in \Lambda_{d(L)}} \alpha_i^2}}\\
				& =\frac{2^{d(L)-1}\sqrt{|\Lambda_{0}|}}{\sqrt{|\Lambda_{d(L)}}| } \\
				& =2^{d(L)-1}\sqrt{N}.
				\nonumber
			\end{aligned}
		\end{equation}
		This completes the proof.
	\end{proof}
	
	
	
	
	
	
\textit{ Applications}.---
It will be shown that by applying Theorem~\ref{th1} we can not only obtain easily the results in \cite{RN5,RN6,RN8},  but also design deterministic quantum spatial search algorithms on some more general graphs.

We first take Johnson graphs as an example. The Johnson graph $J(n, k)$ has vertices given by the $k$-subsets of $\{1,\cdots,n\}$, with two vertices connected when their intersection has size $k-1$. It has many interesting properties and connections with many important problems. Marsh and Wang~\cite{RN6} showed that a quadratic speedup spatial search algorithm on $J(n, 2)$ can be designed using  alternating quantum walks. We prove that, for any fixed positive integer $k$, the quadratic speedup can be generalized to the Johnson graphs $J(n, k)$.


\begin{theorem}		\label{th2}
	Let $k$ be a fixed positive integer. For any Johnson graph $J(n, k)$ with Laplacian matrix $L$ and $N$ vertices in which there is a marked vertex $|m\rangle$, we can design a quantum search algorithm that deterministically finds the marked vertex using $O(\sqrt{N})$ calls to the search operator $e^{-i\theta|m\rangle\langle m|}$.
	
\end{theorem}
\begin{proof}
	The Johnson graph $J(n,k)$ is both periodic and vertex transitive, and $L$ has $min(k,n-k)+1$ distinct eigenvalues that are all integers.
	By Theorem~\ref{th1}, we can construct a quantum algorithm $A$ satisfying $A|m\rangle=|s\rangle$ that uses $O(2^{d(L)-1}\sqrt{N})$ calls  to the search operator where $|s\rangle$ is the uniform superposition state over all vertices of the graph and $d(L)$ is defined as in Definition \ref{de1}. Obviously, $A^\dagger|s\rangle=|m\rangle$, which means that the algorithm can find the marked vertex from $|s\rangle$. Recall that $d(L)$ is less than the number of distinct eigenvalues of $L$ and we have $d(L)<min(k,n-k)+1\leq k+1$.  Since $k$ is fixed, the number of queries for the algorithm $A^\dagger$ to call the search operator is bounded by $O(\sqrt{N})$. Therefore, the algorithm achieves the quadratic speedup.
\end{proof}
Rook graphs and complete-square graphs are both vertex transitive and periodic.
An $m \times n$ rook graph is the graph Cartesian product $K_m \mathbin{\square} K_n$ of complete graphs, having a total of $N = mn$ vertices. The Laplacian matrix of a rook graph has at most four distinct eigenvalues which are all integers. 
The $K_n \mathbin{\square} Q_2$ graph that is called complete-square graph is the graph Cartesian product of a complete graph $K_n$ and a square graph $Q_2$. The Laplacian matrix of a complete-square graph has at most six distinct eigenvalues which are all integers. Using Theorem~\ref{th1} on rook graphs and complete-square graphs as we do in Theorem~\ref{th2}, we can get similar quantum algorithms that deterministically find the marked vertex with quadratic speedups, which cover the results in~\cite{RN5,RN6}.

Next, we consider the complete bipartite graph $K(N_1, N_2)$, which is usually neither vertex transitive nor periodic, so Theorem~\ref{th1} does not apply to this situation directly. However, we can still design a deterministic quantum search algorithm that achieves the quadratic speedup in a similar way as we do in Theorem~\ref{th1}.

A complete bipartite graph $K(N_1, N_2)$ is an undirected graph that has its vertex set partitioned into two subsets $V_1$ of size $N_1$ and $V_2$ of size $N_2$ such that there is an edge from every vertex in $V_1$ to every vertex in $V_2$. A special case of complete bipartite graphs is star graphs, where there is only one vertex in $V_1$ or $V_2$. For this case, Qu et\,al.~\cite{RN8} gave a quantum search algorithm using alternating quantum walks that has $O(\sqrt{N})$ calls to the search operators.  We shall give a generalized algorithm for all complete bipartite graphs.

\begin{theorem}
	For any complete bipartite graph $K(N_1, N_2)$ with the adjacency matrix $H$ in which there is a marked vertex $|m\rangle$, we can design a quantum search algorithm that deterministically finds the marked vertex using  $O(\sqrt{N_1+N_2})$ calls to the search operator $e^{-i\theta|m\rangle\langle m|}$.
\end{theorem}
\begin{proof}
	For a complete bipartite graph $K(N_1, N_2)$, its adjacency matrix $H$ has three distinct eigenvalues: 0, $\sqrt{N_1N_2}$, $-\sqrt{N_1N_2}$. The algebraic multiplicity of 0 is $N_1+N_2-2$, and both $\sqrt{N_1N_2}$, $-\sqrt{N_1N_2}$ are simple eigenvalues. By the Spectral Decomposition Theorem, we can write $H$ in the following form
	\begin{equation}\label{e25}
		\begin{aligned}
	    H&=\sqrt{N_1N_2}|\eta_+\rangle \langle \eta_+| -\sqrt{N_1N_2}|\eta_-\rangle \langle \eta_-|\\
		&+\sum_{i=1}^{N_1+N_2-2}0|\eta_i\rangle \langle \eta_i|, 
		\nonumber
	\end{aligned}
	\end{equation}
	where %$|\eta_+\rangle$, $|\eta_-\rangle$ and $|\eta_i\rangle$ $(i\in \{1,2,\dots, N_1+N_2-2\})$ are the eigenvectors of $\sqrt{N_1N_2}$, $-\sqrt{N_1N_2}$, $0$ and form an orthogonal basis of the entire search space.
	%(\sqrt{N_2},\sqrt{N_2},\cdots,\sqrt{N_2},\sqrt{N_1},\cdots,\sqrt{N_1})
	\begin{equation} \label{e26}
		\begin{aligned}
			&\ |\eta_+\rangle=\frac{1}{\sqrt{2N_1N_2}}(\underbrace  {\sqrt{N_2},\dots,\sqrt{N_2}}_{N_1},\underbrace{\sqrt{N_1},\dots,\sqrt{N_1}}_{N_2})^T,\\	&\ |\eta_-\rangle=\frac{1}{\sqrt{2N_1N_2}}(\underbrace  {\sqrt{N_2},\dots,\sqrt{N_2}}_{N_1},\underbrace{-\sqrt{N_1},\dots,-\sqrt{N_1}}_{N_2})^T,
			\nonumber
		\end{aligned}
	\end{equation}
	and $|\eta_i\rangle$ $(i\in \{1,2,\dots, N_1+N_2-2\})$ are the eigenvectors of $\sqrt{N_1N_2}$, $-\sqrt{N_1N_2}$, $0$ and they form an orthogonal basis of the entire search space. We write the marked vertex $|m\rangle$ on this basis as
	\begin{equation}
		|m\rangle=\sum_{i=1}^{N_1+N_2-2}\alpha_i |\eta_i\rangle+\alpha_+|\eta_+\rangle+\alpha_-|\eta_-\rangle.
		\nonumber
		\label{e27}
	\end{equation}
	If the marked vertex is in $V_1$, then 
 \[\langle\eta_+|m\rangle=\langle\eta_-|m\rangle=\frac{1}{\sqrt{2N_1}},\]
	and
	\begin{equation}
		|m\rangle=\sum_{i=1}^{N_1+N_2-2}\alpha_i |\eta_i\rangle+\frac{1}{\sqrt{2N_1}}|\eta_+\rangle+\frac{1}{\sqrt{2N_1}}|\eta_-\rangle.
		\nonumber
		\label{e28}
	\end{equation}
	Define 
	\begin{equation}
		\begin{aligned}
			&\	|\eta_0\rangle=\frac{1}{\sqrt{\sum_{i=1}^{N_1+N_2-2}\alpha_i^2}}\sum_{i=1}^{N_1+N_2-2}\alpha_i |\eta_i\rangle, \\
			&\	|s\rangle=\frac{1}{\sqrt{2}}(|\eta_+\rangle+|\eta_-\rangle)=\frac{1}{\sqrt{N_1}}(1,1,\cdots,1,0,\cdots,0).
			\nonumber
			\label{e29}
		\end{aligned}
	\end{equation}
	We can see $|m\rangle \in span\{|\eta_0\rangle,|s\rangle\}$, and in this subspace,
	\begin{equation}
		e^{-iH\frac{\pi}{\sqrt{N_1N_2}}}=I-2|s\rangle \langle s|.
		\nonumber
		\label{e30}
	\end{equation}
	Obviously, $\langle m|s\rangle=\frac{1}{\sqrt{N_1}}$. And by Lemma~\ref{l1}, we can find parameters $p \in O(\sqrt{N_1})$, $\gamma$, $\theta_k$ $(k\in \{1,2,\dots,p\})$ such that
	\begin{equation}
		|m\rangle=e^{-i\gamma}\prod_{k=1}^{p}e^{-iH\frac{\pi}{\sqrt{N_1N_2}}}e^{-i\theta_k |m\rangle\langle m|}|s\rangle.
		\nonumber
		\label{e31}
	\end{equation} 
	Here we get a deterministic quantum search algorithm that uses $O({\sqrt{N_1}})$ calls to the search operator and finds the marked vertex from the uniform superposition state over all vertices in $V_1$. Similarly, if the marked vertex is in $V_2$, we can construct a deterministic quantum search algorithm that find the marked vertex from the uniform superposition state over all vertices in $V_2$ and has $O({\sqrt{N_2}})$ calls to the search operator. Combining the two algorithms above, we can get a quantum algorithm that has $O({\sqrt{N_1}+\sqrt{N_2}})$ calls to the search operator and finds the marked vertex exactly. Since ${\sqrt{N_1}+\sqrt{N_2}}\leq \sqrt{2N_1+2N_2}$, the algorithm has $O(\sqrt{N_1+N_2})$ calls to the search operator %achieves a quadratic speedup 
	and the theorem is proved completely.
\end{proof}
	
	

	
	%\section{Conclusions and Discussions}
	\textit{Conclusion and Discussion}.--- 
In this article, we have presented a universal approach that can be used to design deterministic quantum  algorithms for  spatial search on a variety of graphs based on alternating quantum walks. Using this approach, we  have obtained deterministic quantum search algorithms with quadratic speedups on Johnson graphs $J(n,k)$ for any fixed $k$, rook graphs, complete-square graphs, and complete bipartite graphs, which not only cover the previous results obtained in \cite{RN5,RN6,RN8}, but also result in some new and general results. Our algorithms are concise and easy to understand, which are simply to perform alternately the continuous-time quantum walk operator $e^{-iHt}$ and the search operator $e^{-i\theta |m\rangle\langle m|}$ on the initial state $|s\rangle$.

For future work, we will consider generalizing our approach to aperiodic graphs and designing algorithms for the case of multiple marked vertices.
	
	\bibliographystyle{apsrev4-1}
	\bibliography{references.bib}
	
	%\appendix
	%\begin{widetext}
	%\newpage
	
	% \end{widetext} 




%\bibliography{refs}
\end{document} 