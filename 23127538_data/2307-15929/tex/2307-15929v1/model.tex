\documentclass[conference]{IEEEtran}
%\IEEEoverridecommandlockouts
\usepackage[utf8]{inputenc}
\usepackage{amsmath}
\usepackage{graphicx}
\usepackage{caption}
\captionsetup[figure]{labelfont={normal},labelformat={default},labelsep=period,name={Fig.}}
\usepackage{epstopdf}
\usepackage{amsmath,amsfonts,amssymb,amsthm,epsfig,epstopdf,url,array}
%\usepackage{algorithmic}
\usepackage{graphicx}
\usepackage{color}
\usepackage{textcomp}
\newtheorem{theorem}{Theorem}
\newtheorem{lemma}{Lemma}
\usepackage{cite}
\usepackage{amsthm}
\usepackage{amsmath,amsthm}
\usepackage{marvosym}
\usepackage{array}
%\usepackage{biblatex}
%\addbibresource{ref.bib}
\def\BibTeX{{\rm B\kern-.05em{\sc i\kern-.025em b}\kern-.08em
    T\kern-.1667em\lower.7ex\hbox{E}\kern-.125emX}}
\begin{document}

\IEEEoverridecommandlockouts
\title{Secure HARQ-IR-Aided Terahertz Communications}
\author{
\IEEEauthorblockN{
Yongkang Li\IEEEauthorrefmark{1}, Ziyang Song\IEEEauthorrefmark{1},
Zheng Shi\textsuperscript{\Letter}\IEEEauthorrefmark{1},
Qingping Dou\IEEEauthorrefmark{1},
Hongjiang Lei\IEEEauthorrefmark{2},
Jinming Wen\IEEEauthorrefmark{1}
and Junbin Fang\IEEEauthorrefmark{1}
\thanks {This work was supported in part by National Natural Science Foundation of China under Grants 62171200 and 61971080, in part by Chongqing Key Laboratory of Mobile Communications Technology under Grant cqupt-mct-202204, in part by Guangdong Basic and Applied Basic Research Foundation under Grant 2023A1515010900, and in part by Zhuhai Basic and Applied Basic Research Foundation under Grant ZH22017003210050PWC. (\emph{Corresponding Author: Zheng Shi.})}
}
\IEEEauthorblockA{\IEEEauthorrefmark{1}School of Intelligent Systems Science and Engineering, Jinan University, Zhuhai 519070, China}
\IEEEauthorblockA{\IEEEauthorrefmark{2}Chongqing Key Lab of Mobile Communications Technology,\\
Chongqing University of Posts and Telecommunications, Chongqing 400065, China}} %
\date{December 2022}



\maketitle
\begin{abstract}
%Terahertz (THz) communications has become a hot topic in recent years. The main reason is that THz can provide higher data transmission rate and higher-frequency bands to meet the demand of people. Meanwhile, in the past low-rate frequency band communication system, hybrid automatic repeat request (HARQ) technology is considered to be an effective way to improve the transmission reliability of communication system. Therefore,
% Terahertz(THz) communication is one of the development propects in the 6G field. However, the large path loss, antenna misalignment, and atmospheric influence of THZ communications severely deteriorate its reliability. At the same time, since terahertz communication is also wireless communication, anyone in the communication range can eavesdrop on the transmission signal and extract private information, so it is not only reliable, but also security is a problem that must be solved when it matures. To address this, the physical layer security technology is proposed to improve the security of THZ communication. At the same time, hybrid automatic repeat request with incremental redundancy (HARQ-IR) protocols can effectively further enhance the reliability of THZ communication. This paper delves into the performance analysis of HARQ-IR-aided THZ communication in the presence of an eavesdropper, where the secrecy of the transmission is ensured via introduction of dummy messages. We evaluate the performance according to connection outage probability, secrecy outage probability and secrecy throughput. Finally, number results demonstrate the throughput and security benefits of the HARQ-IR protocol.
%There have been some studies on the reliability of THz communications, in which hybrid automatic repeat request(HARQ) can ensure reliable THz communications. But there are still few studies on the security of THz communication. Unlike prior analyses, physical layer security based on information theory is considered as an effective technique.
Terahertz (THz) communication is one of the most promising candidates to accommodate high-speed mobile data services. This paper proposes a secure hybrid automatic repeat request with incremental redundancy (HARQ-IR) aided THz communication scheme, where the transmission secrecy is ensured by confusing the eavesdropper with dummy messages. The connection and secrecy outage probabilities are then derived in closed-form. Besides, the tail behaviour of the connection outage probability in high signal-to-noise ratio (SNR) is examined by carrying out the asymptotic analysis, and the upper bound of the secrecy outage probability is obtained in a simple form by capitalizing on large deviations. With these results, we take a step further to investigate the secrecy long term average throughput (LTAT). By noticing that HARQ-IR not only improves the reliability of the legitimate user, but also increases the probability of being eavesdropped, a robust rate adaption policy is finally proposed to maximize the LTAT while restricting the connection and secrecy outage probabilities within satisfactory requirements.

%It should be noticed that HARQ-IR not only enhances the reliability of the legitimate user, but also increases the probability of being eavesdropped. To address this issue, the main code rate $R_0$ and confidential information rate $R_s$ are optimally designed to maximize the secure LTAT meanwhile guaranteeing the connection and the secrecy outage probabilities

  %More specifically, we evaluate the performance according to connection outage probability, secrecy outage probability. Then the secrecy throughput is expressed in terms of the outage probability based on renewal theory. Moreover, we analyze the secrecy degradation caused by retransmissions based on the asymptotic results. Finally, through the simple expression of the asymptotic result, the optimal rate selection is realized and the complexity is reduced. The analytical results are eventually validated via Monte-Carlo simulations.


%In this paper, we investigated the performance of secure HARQ-IR assisted THz communications. More specifically, the connection outage probability are derived in closed-form, with which its asymptotic analysis in high SNR was performed. Then the secrecy outage performance was examined by conducting exact and approximate analyses. With these fundamental results, the secrecy LTAT was obtained accordingly. Finally, a robust rate adaption strategy was proposed to maximize the LTAT while guaranteeing the outage constraints.
\end{abstract}
% Note that keywords are not normally used for peerreview papers.
\begin{IEEEkeywords}
Hybrid automatic repeat request (HARQ), incremental redundancy, physical layer security, terahertz (THz).
\end{IEEEkeywords}

\section{Introduction}\label{sec:int}
%To support upcoming for the sixth generation (6G),
In order to offer higher capacity, ultra-high frequencies are foreseen to be used in the sixth generation (6G). Terahertz (THz) communications are recognized as one of key enabling technologies to deliver a peak data rate of 1 Tbps.
%What restricts people's development of THz communications is that
Unfortunately, in contrast to the low-frequency communications, THz communications suffer from severe path-loss attenuation, and are susceptible to the atmospheric turbulence, the pointing errors, the molecular absorption, etc. These negative effects degrade the reception reliability of THz communications. In \cite{8610080,9039743}, the error performance of THz communications was examined to account for the antenna misalignment and hardware imperfections. To remedy these defects, several approaches have been put forward to fulfill reliable THz communications. To be specific, an offloading strategy was proposed to minimize the energy consumption under ultra-reliable and low latency constraints in \cite{THz2_2}. In \cite{10005197}, reconfigurable reflecting surface (RIS) was leveraged to ensure the reliability and latency requirements of THz communications.
Moreover, hybrid automatic repeat request (HARQ) has been acknowledged as an effective transmission technique to boost the reliability of signal reception, albeit at the price of additional transmission delay. Thereon, the authors in \cite{THz2_3,2304.11341} thoroughly investigated the outage performance of three different HARQ types assisted THz communications.% analyze the performance of HARQ-aided THz communications. What's more, incremental redundancy HARQ scheme can achieves throughput efficiency by adapting its error correcting code redundancy to fluctuating channel conditions.


%designed the transmission power and the offloading strategy aimed to provide reliable end-to-end THz communication while minimizing power.

%Specifically, in \cite{8610080} the combined effects of misalignment fading and hardware imperfections in multipath fading are analyzed, and in \cite{9039743} the combined effects of antenna misalignment and local oscillator (LO) phase noise (PHN) are studied.
% In order to solve the problem that the terahertz channel is susceptible to external conditions and affects reliability, there is a lot of research aimed at improving the reliability of THz communication.
%In \cite{THz2_1}, the reliability analysis of integrated circuit technology based on physics method was studied.
% Among the many ways to improve reliability, hybrid automatic retransmission request (HARQ) protocols is a method that can effectively improve reliability by powerful channel coding with retransmission. The authors in \cite{THz2_3} analyze the performance of HARQ-aided THz communications. What's more, incremental redundancy HARQ scheme can achieves throughput efficiency by adapting its error correcting code redundancy to fluctuating channel conditions.



 %THz communications are susceptible to atmospheric turbulence, pointing errors, rain attenuation, etc, which is harmful to the reception reliability.

%The arrival of the era of the sixth generation (6G) of mobile communication is unstoppable, and we need the intelligent connection of everything. To this end, we need ultra-large bandwidth and ultra-low-latency communication to transmit massive amounts of data. The current bit rates does not support us to do this, so we need to develop new technologies.
%The authors of \cite{8766143} envision the architecture of 6G networks while proposing some promising technologies. THz communications is considered as one of these technologies,
%Due to factors such as physical device size constraints and policies, it is natural to develop to higher frequency bands, such as terahertz(THz) band.
%which have emerged as a potential technique %这里加点6G和THz背景% ,9040264,9269507
%$to support the peak data rate of more than 1~Tbps. % \cite{tataria20216g,9662195}. %by pushing the rates from the Gigabits per second (Gbps) to the Terabits per second (Tbps)
%\cite{9558848}  \cite{9368251,9541155}%,9473756%这里加两个中断性能分析

Another inherent advantage of THz communications is the provision of physical layer security owing to its high antenna directionality \cite{9482609}, which has received considerable attentions recently. To name a few, Qiao {\it et al.} in \cite{PLS0_2} proposed a RIS-assisted secure THz transmission scheme. In \cite{9497766}, an artificial noise based mechanism was developed to address the in-beam security issue. In addition, a secure two-phase transmission strategy with unmanned aerial vehicles (UAV) relaying was devised in \cite{9709673} to safeguard THz communications. %In \cite{9784907}, the atmosphere attenuation was utilized to enhance the security of THz communications by choosing the carrier frequency based on the peak of a water vapor absorption line.
%another challenge of THz communications is security.
%Terahertz communication is very easy to be eavesdropped during the communication process, because it belongs to wireless communication, anyone within the communication range can eavesdrop on the transmitted signal and extract private information.
%For a large network, the use of traditional encryption methods to provide confidentiality has a problem that needs to be solved is the distribution and maintenance of keys. For this problem, Wyner pioneered a method that does not require a key, but uses the difference between the legitimate channel and the eavesdropper channel \cite{PLS0_1}. In this way, users can transmit information reliably and securely. This is called physical layer security. %In [0],they research on Retransmission Protocols for Reliable Packet Communication under Confidentiality Constraints. on the basis of [0],the dummy-message rate is split, these extra degrees of freedom improve the match between dummy-message rate and eavesdropper channel implementation[1].In [2],the HARQ-assisted cognitive non-orthogonal multiple access (NOMA) scheme is studied to realize secure transmission in the Internet of Things network.
%The authors in \cite{9482609}  surveyed how security may impact the envisioned 6G wireless systems, one potential solution was physical layer security.
%Zhihong Liu {\it et al.} in \cite{8964397} investigated secure THz communications in internet-of-things (IoT) networks.
%The inherent multi-path nature of the THz communications was considered to decrease the message eavesdropping probability \cite{8845312}.
 %The authors in \cite{PLS0_3} derived the expression for the secrecy capacity of IRS-Assisted THz wireless communications with pointing errors.
However, so far, there have been almost no readily available results concerned with the physical layer security of HARQ-aided THz communications in the literature. Notably, the retransmission strategy of HARQ will yield increasing probability of being eavesdropped.  This motivates us to study the performance of secure HARQ-aided THz communications from the information-theoretical perspective, with which useful system design guidelines can be extracted for THz communications. Moreover, unlike previous literature with inaccurate and complicated outage expressions, this paper provides precise expressions and deep insights into the benefits of physical layer security through approximate analysis.

  %HARQ-IR not only improves the reliability of the legitimate user, but also increases the probability of being eavesdropped

 %Due to retransmission, secure communications are more difficult in HARQ-IR over THz channels. %But The outage analysis thus involves the probability distribution analysis of a product of multiple correlated shifted Gamma random variables and hinders previous scholars from analyzing the confidentiality of HARQ-IR over THz channel.

%The closed-form expressions are derived for the outage probabilities of two types of secure HARQ over THz communications including connection outage probability and secrecy outage probability. And we derived the closed-form solution of secrecy throughput, and then optimized the secrecy throughput with connection interruption constraints and secrecy interruption constraints. With the analytical results, the asymptotic outage analysis is performed to uncover meaningful insights. For example, when the number of retransmission increases, HARQ over THz communications will still retain a certain degree of confidentiality.


In this paper, we focus on secure HARQ with incremental redundancy (HARQ-IR) aided THz communications, where the eavesdropper is confused through the introduction of dummy messages. At first, the connection and secrecy outage probabilities are derived in closed-form. With these results, the asymptotic/approximate expression of the connection/secrecy outage probability in high signal-to-noise ratio (SNR) was derived by capitalizing on the asymptotic analysis/large deviation. Besides, the secrecy long term average throughput (LTAT) is expressed in terms of the outage metrics. Furthermore, although HARQ-IR is able to improve the reliability of the legitimate user, the eavesdropping probability is increased. Therefore, we eventually develop a robust rate adaption policy to maximize the LTAT while ensuring the connection and secrecy outage constraints. %Consequently, our thereotical analysis


The reminder of this paper is outlined as follows. In Section II, we introduce the system model for secure HARQ-IR aided THz communications. The outage and throughput performance metrics are then analyzed in Section III. In Section IV, the numerical results are presented for verification and a robust rate adaption policy is proposed. Section V finally concludes this paper.

\section{System Model}
\subsection{Secure HARQ-IR Transmissions}
In this paper, we consider a HARQ-IR aided THz communication system in the presence of a single passive eavesdropper. The  transmitter (i.e., Alice) sends the confidential information to the receiver (i.e., Bob) through the main channel, and the eavesdropper (i.e., Eve) overhears the transmitted signal. Herein, we assume that Alice does not know the instantaneous channel state information (CSI), which may result in the possibility of communication interruptions. In order to ensure the reception reliability of confidential message at Bob, HARQ-IR is adopted. More specifically, if the outage event occurs, Bob will request the retransmission of the message by feeding back a non-acknowledge (NACK) message. According to HARQ-IR, a new packet with different redundancy will be delivered in the next HARQ round once receiving the NACK request.

To implement the secure HARQ protocol between Alice and Bob, the Wyner codes are used \cite{4802331,8355527,6844902}. The confidential information is first encoded into a mother code of length $ML$, which is then split into $M$ sub-codewords, each with length $L$. The $M$ sub-codewords will be conveyed one by one upon request, and $M$ refers to the maximum number of transmissions. The basic idea of Wyner code is leveraging random binning approach, in which dummy message is randomly inserted into the confidential message so as to increase the secrecy level. Particularly, in order to convey the confidential message in the set ${\mathcal W} = \{1,2,\cdots,2^{LR_s}\}$, we adopt a Wyner code ${\mathcal C}(R_0/M,R_s/M,ML)$ of size $2^{LR_0}$ codewords. Therein, the two rate parameters of the Wyner code, i.e., $R_0$ and $R_s$, are the main channel code rate and the secrecy information rate, respectively. Moreover, the difference $R_0-R_s$ is termed as the secrecy gap (or called dummy message rate), which is introduced to confuse the eavesdropper. During the first HARQ round, the sub-codeword ${\bf x}_1$ is formed by a punctured Wyner code of length $L$, i.e., ${\mathcal C}(R_0,R_s,L)$. Furthermore, after $m$ HARQ rounds, all the transmitted sub-codewords constitute $[{\bf x}_1,\cdots,{\bf x}_m]$ that corresponds to a punctured code of length $mL$, i.e., ${\mathcal C}(R_0/m,R_s/m,mL)$.

By considering block fading wiretap THz channels, the received signals at the legitimate user (i.e., Bob) and eavesdropper (i.e., Eve), i.e., ${\bf y}_{B,m}$ and ${\bf y}_{E,m}$, in the $m$-th HARQ round can be expressed as
\begin{align}\label{model}
{\bf y}_{\beta,m} &= \sqrt{P_m}{h_{\beta,m}}{\bf x}_m + {{\bf n}_{\beta,m}},\,\beta\in \{B,E\},%\notag\\
%{\bf y}_{E,m} &= \sqrt{P_m}{h_{E,m}}{\bf x}_m + {{\bf n}_{E,m}},
\end{align}
%, denoted by ${\bf y}_m$, ${\bf z}_m$, respectively.  and $\bf{x}_m$
where $P_m$ represents the transmit power in the $m$-th HARQ round, ${\bf{n}}_{\beta,m}$ corresponds to the complex additive white Gaussian noises (AWGN) with zero mean and variance of $N_0$, ${h_{B,m}}$ and ${h_{E,m}}$ are the THz channel coefficient of the main channel and the eavesdropper's channel, respectively.
%, We assume block-fading channels where each block of signals experiences independent fading. During  the $m$-th HARQ round, where $M$ denotes maximum number of transfers, the THz channel coefficients of the main channel and the eavesdropping channel are expressed as $h_m$ and $g_m$, respectively.
%the received signals at the legitimate user and eavesdropper, i.e., ${\bf y}_{B,m}$ and ${\bf y}_{E,m}$, in the $m$-th HARQ round are respectively written as
%\begin{align}\label{model}
%{\bf y}_{B,m} &= \sqrt{P_m}{h_{B,m}}{\bf x}_m + {{\bf n}_{B,m}},\notag\\
%{\bf y}_{E,m} &= \sqrt{P_m}{h_{E,m}}{\bf x}_m + {{\bf n}_{E,m}},
%\end{align}
%, denoted by ${\bf y}_m$, ${\bf z}_m$, respectively.  and $\bf{x}_m$
%where $P_m$ represents the transmit power in the $m$-th HARQ round, $\bf{w}_m$ and $\bf{e}_m$ are complex additive white Gaussian noises (AWGN) with zero mean and variance of $N_0$, $h_m$ and $g_m$ are the THz channel coefficients of the main channel and the eavesdropper's channel, respectively.

\subsection{THz Channel Model}
 %For main channel and eavesdropper's channel, the state parameters $\beta \buildrel \Delta \over = (h, g) \in  {\mathcal {B}}$ are channel coefficients, where $\mathcal {B}$ is the channel state coefficient space. What's more,
 %the state parameters can be expressed as
 By following the THz channel modeling in \cite{9039743}, the THz channel coefficient ${h_{\beta,m}}$ can be modeled as %$\beta \triangleq (h, g)$
 \begin{equation}\label{hm}
{{h_{\beta,m}}} = {h_{\beta,l}}{h_{\beta,pf,m}},\, \beta\in \{B,E\},
 \end{equation}
 %${h_m} = {h_l}{h_{pf,m}}({g_m} = {g_l}{g_{pf,m}})$,
 where ${h_{\beta,l}}$ is the deterministic THz path gain and remains constant during all HARQ rounds, and ${h_{\beta,pf,m}}$ quantifies the combining influence of antenna misalignment and multipath fading. According to \cite{8610080}, ${h_{\beta,l}}$ is given by
 \begin{equation}\label{eqn:path gain}
{h_{\beta,l}} = \frac{{c\sqrt {{G_{t}}{G_{\beta}}} }}{{4\pi {f}{d_\beta}}}\exp \left( { - \frac{1}{2}\kappa ({f},T,\psi ,p){d_\beta}} \right),
\end{equation}
%\begin{equation}\label{eqn:path gain of g}
%{g_l} = \frac{{c\sqrt {{G_{t}}{G_{e}}} }}{{4\pi {f_1}{d_2}}}\exp \left( { - \frac{1}{2}\kappa ({f_1},T,\psi ,p){d_2}} \right),
%\end{equation}
where $c$ and $f$ stand for the light speed and the carrier frequency, respectively, ${{G_{t}}}$ represents the transmit antenna gain, $d_\beta$ and ${{G_{\beta}}}$ are the transmission distance and receive antenna gains, respectively. %of main or eavesdropper channel which depended on the channel coefficients
$\kappa ({f},T,\psi ,p)$ characterizes the molecular absorption coefficient that is decided by the temperature $T$, the relative humidity $\psi$, and the atmospheric pressure $p$. As proved in \cite{8610080}, $\kappa ({f_1},T,\psi ,p)$ is explicitly obtained as
\begin{equation}
    \kappa ({f_1},T,\psi ,p) = \kappa_1(f_1,\upsilon)+\kappa_2(f_1,\upsilon)+\Lambda(f_1),
\end{equation}
where $\upsilon={\psi p_w(T,p)}/{(100p)}$,
%\begin{equation}
 %   \upsilon = \frac{\psi}{100} \frac{p_w(T,p)}{p},
%\end{equation}
$\upsilon$ is the volume mixing ratio of the water vapor, $p_w(T,p)$ refers to the partial pressure of saturated water vapor that depends on the temperature $T$ and pressure $p$. Besides, the terms $\kappa_1(f_1,\upsilon)$, $\kappa_2(f_1,\upsilon)$, and $\Lambda(f_1)$ can be calculated by using the simplified model of molecular absorption loss as \cite{8417891}
\begin{equation}
    \kappa_1(f_1,\upsilon) = \frac {q_1\upsilon (q_2\upsilon+q_3)} {(q_4\upsilon+q_5)^2+(\frac{f} {100c}-c_1)^2},
\end{equation}
\begin{equation}
    \kappa_2(f_1,\upsilon) = \frac {q_6\upsilon(q_7\upsilon+q_8)} {(q_9\upsilon+q_{10})^2+(\frac{f} {100c}-c_2)^2},
\end{equation}
\begin{equation}
    \Lambda(f_1) = j_1{f_1^3} + j_2{f_1^2} + j_3{f_1} + j_4,
\end{equation}
where $q_1=0.2205$, $q_2 = 0.1303$, $q_3 = 0.0294$, $q_4 = 0.4093$,
$q_5 = 0.0925$, $q_6 = 2.014$, $q_7 = 0.1702$, $q_8 = 0.0303$,
$q_9 = 0.537$, $q_{10} = 0.0956$, $c_1= 10.835$cm$^{-1}$, $c_2=12.664$cm$^{-1}$, $j_1=5.54  \times 10^{-37}$Hz$^{-3}$, $j_2=-3.94  \times 10^{-25}$Hz$^{-2}$, $j_3=9.06  \times 10^{-14}$Hz$^{-1}$, $j_4=-6.36  \times 10^{-3}$Hz$^{-3}$.
Moreover, as derived in \cite{8610080}, the probability density function (PDF) of $|{h_{\beta,pf,m}}|$ can be expressed as
\begin{equation}\label{eqn:PDF}
{f_{|{h_{\beta,pf,m}}|}}(x) = \frac{{\phi_\beta {\mu ^{\frac{\phi_\beta}{\alpha }}}{x^{\phi_\beta - 1}}}}{{S_\beta^{\phi_\beta} \hat h_{f,\beta}^{\phi_\beta} \Gamma (\mu )}}\Gamma \left( {\frac{{\alpha \mu  - \phi_\beta }}{\alpha },\frac{{\mu {x^\alpha }}}{{S_\beta^\alpha \hat h_{f,\beta}^\alpha }}} \right),
\end{equation}
% \begin{equation}\label{eqn:PDF of g}
% {g_{\left| {{g_{pf,m}}} \right|}}(x) = \frac{{\phi_e {\mu ^{\frac{\phi_e }{\alpha }}}{x^{\phi_e  - 1}}}}{{S_e^{\phi_e} \hat h_f^{\phi_e} \Gamma (\mu )}}\Gamma \left( {\frac{{\alpha \mu  - \phi }}{\alpha },\frac{{\mu {x^\alpha }}}{{S_e^\alpha \hat h_f^\alpha }}} \right),
% \end{equation}
%\begin{equation}\label{eqn:CDF}
%{F_{\left| {{h_{\beta,pf,m}}} \right|}}(x) = 1 - \frac{{\phi_\beta {\mu ^{\frac{\phi_\beta }{\alpha }}}{x^{\phi_{\beta} }}}}{{\alpha S_\beta^{\phi_{\beta}} \hat h_{f,\beta}^{\phi_\beta }}}\sum\limits_{n = 0}^{\mu  - 1} {\frac{1}{{n!}}} \Gamma \left( {\frac{{\alpha n - \phi_\beta }}{\alpha },\frac{{\mu {x^\alpha }}}{{S_0^\alpha \hat h_{f,\beta}^\alpha }}} \right),
%\end{equation}
 % \begin{equation}\label{eqn:CDF of g}
 % {F_{\left| {{\beta_{pf,m}}} \right|}}(x) = 1 - \frac{{{\phi_\beta} {\mu ^{\frac{\phi_\beta }{\alpha }}}{x^{\phi_\beta} }}}{{\alpha S_\beta^{\phi_\beta} \hat h_f^{\phi_\beta} }}\sum\limits_{n = 0}^{\mu  - 1} {\frac{1}{{n!}}} \Gamma \left( {\frac{{\alpha n - {\phi_\beta} }}{\alpha },\frac{{\mu {x^\alpha }}}{{S_\beta^\alpha \hat h_f^\alpha }}} \right),
 % \end{equation}
where $\Gamma(a) $ and $ \Gamma \left( a,x \right)$ %$\Gamma(a) = \int_0^\infty  {{t^{x - 1}}{e^{ - t}}dt} $ and $ \Gamma \left( a,x \right) =\int_x^\infty  {{t^{a - 1}}{e^{ - t}}dt} $
denote Gamma function and the upper incomplete Gamma function, respectively,
%$\alpha$ is the distribution parameter,
$\mu$ and ${\hat h_{f,\beta}}$ denote the variance and the $\alpha$-root mean value of the fading channel envelope, respectively, ${S_\beta} = {\left| {{\rm erf}(\zeta_\beta)} \right|^2}$ is the fraction of the maximum collected power over THz channels and $\zeta_\beta  = \sqrt \pi  {r_\beta}/\left( {\sqrt 2 {w_{d_\beta}}} \right)$, $r_\beta$ and ${w_{d_\beta}}$ stand for the radius of the receive antenna effective area and the transmission beam footprint radius at reference distance ${d_\beta}$, respectively, ${\phi _\beta } = w_{{d_\beta }}^2\sqrt \pi  {\rm{erf}}\left( {{\zeta _\beta }} \right)\exp (\zeta _\beta ^2)/(8{\zeta _\beta }\sigma _\beta ^2)$ and $\sigma_\beta$ are
the ratio of normalized beam-width to the jitter and the doubled spatial jitter standard deviation of THz channels, respectively.
%$\phi_\beta  = {w_\beta}^2/(4\sigma_\beta ^2)$ and $\sigma_\beta$ are
%the ratio of normalized beam-width to the jitter and the doubled spatial jitter standard deviation of THz channels, respectively, ${w_\beta} = \sqrt{ {w_{d_\beta}^2\sqrt \pi \rm erf\left( \zeta_\beta  \right)} \exp ( {  {\zeta_\beta ^2}} )/( {2\zeta_\beta } )}$ is the equivalent beam width radius.

\subsection{Achievable Mutual Information}
% From the information-theoretical perspective\cite{930931}, by selecting state
% parameters $\beta \in (h, g) \in  {\mathcal {B}}$ which is depending on the random fading coefficients channel coefficients,
By following the information theory of secure HARQ-IR \cite{8355527}, the accumulated mutual information of HARQ-IR aided THz communications achieved by the legitimate user and eavesdropper after $M$ HARQ rounds can be obtained as
\begin{equation}\label{eqn:harq IR}
I_{\beta}(M) = \sum\limits_{m = 1}^M {{{\log }_2}\left(1 + {\rho _m}{{\left| {{h_{\beta,l}}} \right|}^2}{{\left| {{h_{\beta,pf,m}}} \right|}^2}\right)}, \beta\in \{B,E\},
\end{equation}
where $\rho_m=P_m/N_0$ denotes the transmit signal-to-noise ratio.
%\begin{equation}\label{eqn:harq IR of g}
%I_{XZ}(M) = \sum\limits_{m = 1}^M {{{\log }_2}(1 + {\rho _m}{{\left| {{g_l}} \right|}^2}{{\left| {{g_{pf,m}}} \right|}^2})}.
%\end{equation}
% and depend on the random fading coefficients channel coefficients $\beta = (h, g) \in  {\rm B}$. \par
It is worth noting that the introduction of HARQ-IR is not only beneficial to enhance reception reliability for legitimate users, but also is vulnerable to eavesdropping. %for eavesdroppers to successfully overhear secret messages,
Therefore, it is imperative to examine both the connection outage and the secrecy outage. According to \cite{4802331}, the connection outage occurs if the accumulated mutual information attained by the legitimate user is below the code rate $R_0$, i.e., $I_{B}(M)< R_0$. Whereas, the secrecy outage occurs if the accumulated mutual information is larger than the dummy message rate $R_0 - R_s$, i.e., $ I_{E}(M)> R_0 - R_s$.

%for a given pair of rates$(R_{0},R_{s})$, there are two types of outage: connection outage and secrecy outage.  Specifically, a connection outage occurs if
%\begin{equation}\label{PCO_1}
%  I_{XY}(M)< R_0,
%\end{equation}
%a secrecy outage occurs if
%\begin{equation}\label{PSO}
%  I_{XZ}(M)> R_0 - R_s,
%\end{equation}
%the parameter $R_s$ represents the amount of secret information that needs to be transmitted to legitimate user, which satisfies $R_s \geq 0$; and the parameter $R_0$ represents the total amount of information transmitted, which satisfies $R_0 \geq R_s$. What's more, $R_0-R_s$ represents the amount of randomness introduced to confuse eavesdroppers, which is called dummy messages rate.
% The connection outage probability $P_{CO}$ and secrecy outage probability $P_{SO}$ are defined by:
% %The definition of $Pco$ and $Pso$ can be written as follows:
% % \begin{equation}\label{Pe}
% %   P_{CO} = P(I_{XY}(M)<R_0)
% % \end{equation}
% \begin{equation}\label{Ps}
%   P_{SO} = \sum\limits_{m = 1}^M{P(M=m)P({I_{XZ}(m)>R_0 - R_s})}
% \end{equation}
% where $P(M=m)$denote the probability mass function of the number of transmissions $M$, can be expressed as
% \begin{align}\label{PDF_M}
%   P(M=m)&=P({I_{XY}(m-1)<R_0})\\
%   \nonumber&-P({I_{XY}(m)<R_0}),m=1,...,M-1
% \end{align}
% \begin{align}
% P(M=M)=P({I_{XY}(M-1)<R_0})
% \end{align}

% To prevent network congestion in the presence of possible deep fading, the maximum number of HARQ rounds is limited up to $M$. The accumulated mutual information of HARQ-IR-aided THz communications after $M$ HARQ rounds is given by
% \begin{equation}\label{eqn:harq IR}
% {I}_M = \sum\limits_{m = 1}^M {{{\log }_2}(1 + {\rho _m}{{\left| {{h_l}} \right|}^2}{{\left| {{h_{pf,m}}} \right|}^2})}.
% \end{equation}
% \subsection{Thz Channels}
% In this paper, we consider a point-to-point HARQ-IR-aided THz communication system. To enhance the reception reliability, HARQ-IR technique is utilized to assist THz communications. The received signal at the $m$-th HARQ round can be expressed as
% \begin{equation}\label{eqn:channel_model}
% {{\bf y}_m} = \sqrt{P}{h_m}{\bf s}_m + {{\bf w}_m},
% \end{equation}
% where ${P}$ and $h_m$ represents the transmit power and the equivalent THz channel coefficient in the $m$-th HARQ round, respectively, ${\bf s}_m$ is the transmitted symbols with unity power, and ${{\bf w}_m}$ is the complex additive white Gaussian noise (AWGN) with mean zero and variance $N_0$.  The channel coefficient $h_m$ is modeled according to ${h_m} = {h_l}{h_{pf,m}}$, where $h_l$ is the deterministic THz path gain and keeps constant during all HARQ rounds, and ${h_{pf,m}}$ captures the joint effect of antenna misalignment and multipath fading. ${h_l}$ is given by
% \begin{equation}\label{eqn:path gain}
% {h_l} = \frac{{c\sqrt {{G_{t,1}}{G_{r,1}}} }}{{4\pi {f_1}{d_1}}}\exp \left( { - \frac{1}{2}\kappa ({f_1},T,\psi ,p){d_1}} \right),
% \end{equation}
% where $c$, $f_1$, and $d_1$ stand for the speed of light, the carrier frequency, and the transmission distance, respectively, ${{G_{t,1}}}$ and ${{G_{r,1}}}$ represent the transmit and receive antenna gains, respectively, $\kappa ({f_1},T,\psi ,p)$ characterizes the molecular absorption coefficient, which is determined by the temperature $T$, the relative humidity $\psi$ and the atmospheric pressure $p$. The explicit expression of $\kappa ({f_1},T,\psi ,p)$ is  omitted here to conserve space.
% Moreover, the probability density function (PDF) and cumulative distribution function (CDF) of ${h_{pf,m}}$ are respectively expressed as
% \begin{equation}\label{eqn:PDF}
% {f_{\left| {{h_{pf,m}}} \right|}}(x) = \frac{{\phi {\mu ^{\frac{\phi }{\alpha }}}{x^{\phi  - 1}}}}{{S_0^\phi \hat h_f^\phi \Gamma (\mu )}}\Gamma \left( {\frac{{\alpha \mu  - \phi }}{\alpha },\frac{{\mu {x^\alpha }}}{{S_0^\alpha \hat h_f^\alpha }}} \right),
% \end{equation}
% \begin{equation}\label{eqn:CDF}
% {F_{\left| {{h_{pf,m}}} \right|}}(x) = 1 - \frac{{\phi {\mu ^{\frac{\phi }{\alpha }}}{x^\phi }}}{{\alpha S_0^\phi \hat h_f^\phi }}\sum\limits_{n = 0}^{\mu  - 1} {\frac{1}{{n!}}} \Gamma \left( {\frac{{\alpha n - \phi }}{\alpha },\frac{{\mu {x^\alpha }}}{{S_0^\alpha \hat h_f^\alpha }}} \right),
% \end{equation}
% where $ \Gamma \left( a,x \right)$ denotes the upper incomplete Gamma function, $\alpha$ is the distribution parameter, $\mu$ and ${\hat h_f}$ denote the variance and the $\alpha$-root mean value of the fading channel envelope, respectively, ${S_0}$ is the fraction of the maximum collected power and is given by ${S_0} = {\left| {{\rm erf}(\zeta )} \right|^2}$ and $\zeta  = \sqrt \pi  {w_{d1}}/\left( {\sqrt 2 {r_1}} \right)$, $r_1$ and ${w_{d1}}$ denote the radius of the receive antenna effective area and the transmission beam footprint radius at reference distance ${d_1}$, respectively, $\phi  = w_e^2/4\sigma _s^2$, ${w_e}$ and ${\sigma _s}$ are the equivalent beam width radius and the doubled spatial jitter standard deviation, respectively, $w_e^2 = \left( {w_{d1}^2\sqrt \pi \rm erf\left( \zeta  \right)} \right)/\left( {2\zeta \exp \left( { - {\zeta ^2}} \right)} \right)$.
% To prevent network congestion in the presence of possible deep fading, the maximum number of HARQ rounds is limited up to $M$. The accumulated mutual information of HARQ-IR-aided THz communications after $M$ HARQ rounds is given by
% \begin{equation}\label{eqn:harq IR}
% {I}_M = \sum\limits_{m = 1}^M {{{\log }_2}(1 + {\rho _m}{{\left| {{h_l}} \right|}^2}{{\left| {{h_{pf,m}}} \right|}^2})}.
% \end{equation}
% \subsection{secure channels}
% Then,we consider the HARQ-IR-Aided THz channel model in the presence of eavesdroppers. The  transmitter  sends confidential information through the main channel, and the eavesdropper eavesdrops on the eavesdropper channel. or convenience, we assume block-fading channels where each block of signals experiences independent fading. During  the $m$ th re-transmission, the channel coefficients of the main channel and the eavesdropping channel are expressed as $h_m$ and $g_m$, respectively. Then, the received signals at user and eavesdropper, denoted by $y_m$, $z_m$, respectively, are given by
% \begin{equation}\label{model}
% \begin{aligned}
% y_m = \sqrt{P_m}{h_m}{\bf x}_m + {{\bf w}_m},\\
% z_m = \sqrt{P_m}{g_m}{\bf x}_m + {{\bf e}_m}.
% \end{aligned}
% \end{equation}
% where $P_m$, $h_m$($g_m$), and $x_m$ represent the transmit power, the main(eavesdropper) THz channel coefficient, and the signal vector in the m-th HARQ round, respectively, $w_m$($e_m$) is the complex additive white Gaussian noise (AWGN) with variance $N_0$ in main(eavesdropper) THz channel.
\section{Performance Analysis of Secure HARQ}\label{sec:opa}
 In this section, we first study the outage performance of secure HARQ-IR over THz fading channels. It has been mentioned that there are two types of outage events, i.e., the connection outage and the secrecy outage. Hence, these two types of outage probabilities are studied individually. With the analytical results, the secrecy long term average throughput (LTAT) is then evaluated.%Note that in the system model, the transmitter does not have any information about the instantaneous channel state other than channel statistics. So, code rate $(R_{0},R_{s})$ is fixed.
%  In the following, for a given pair of code rates, we describe channel conditions for secure THz channels.

%  \subsection{Definitions}
%  For a given pair of rates$(R_{0},R_{s})$and a fixed input distribution $p(x)$, the secure channel set $P$ is the union of all channel pairs $(h,g)$ satisfying
%  \begin{equation}\label{Al}
%   \sum\limits_{m = 1}^M{I(X;Y|h_m)}\geq R_0
% \end{equation}
% \begin{equation}\label{Bl}
%   \sum\limits_{m=1}^M{I(X;Y|g_m)}\leq R_0 - R_s
% \end{equation}
% In physical layer security, there are two types of outage: connection outage and secrecy outage. When the channel pair does not satisfy \eqref{Al} and \eqref{Bl}. Specifically, a connection outage occurs if
% \begin{equation}\label{PCO_1}
%   \sum\limits_{m = 1}^M{I(X;Y|h_m)}< R_0
% \end{equation}
% a secrecy outage occurs if
% \begin{equation}\label{PSO}
%   \sum\limits_{m=1}^M{I(X;Z|g_m)}> R_0 - R_s
% \end{equation}
% The accumulated mutual information of HARQ-IR-aided THz communications of main channel and eavesdropper channel after $M$ HARQ rounds can be written as
% \begin{equation}\label{eqn:harq IR}
% I_{XY}(M) = \sum\limits_{m = 1}^M {{{\log }_2}(1 + {\rho _m}{{\left| {{h_l}} \right|}^2}{{\left| {{h_{pf,m}}} \right|}^2})}.
% \end{equation}
% \begin{equation}\label{eqn:harq IR of g}
% I_{XZ}(M) = \sum\limits_{m = 1}^M {{{\log }_2}(1 + {\rho _m}{{\left| {{g_l}} \right|}^2}{{\left| {{g_{pf,m}}} \right|}^2})}.
% \end{equation}

% \newtheorem{myDef}{Definition}
% \begin{myDef}
% The connection outage probability $P_{CO}$ and secrecy outage probability $P_{SO}$ are defined by:
% %The definition of $Pco$ and $Pso$ can be written as follows:
% \begin{equation}\label{Pe}
%   P_{co} = Pr\left\{I_{XY}(M)<R_0\right\}
% \end{equation}
% \begin{equation}\label{Ps}
%   P_{so} = \sum\limits_{m = 1}^M{\Pr\{M=m\}\Pr\left\{{I_{XZ}(m)>R_0 - R_s}\right\}}
% \end{equation}
% where $Pr\left\{M=m\right\}$denote the probability mass function of the number of transmissions $M$, can be expressed as
% \begin{align}\label{PDF_M}
%   Pr\left\{M=m\right\}&=Pr\left\{{I_{XY}(m-1)<R_0}\right\}\\
%   \nonumber&-Pr\left\{{I_{XY}(m)<R_0}\right\},m=1,...,M-1
% \end{align}
% \begin{align}
% p[M=M]=Pr\left\{{I_{XY}(M-1)<R_0}\right\}
% \end{align}
% \end{myDef}

\subsection{Connection Outage Probability}
As aforementioned, the connection outage occurs if $I_{B}(M)< R_0$. Accordingly, the connection outage probability $P_{co}$ can be obtained by averaging over all the realizations of the channel process, i.e.,
%The definition of $Pco$ and $Pso$ can be written as follows:
\begin{equation}\label{Pe}
  P_{co} = \Pr\left\{I_{B}(M)<R_0\right\}.
\end{equation}
%According to (\ref{Pe}), the connection outage probability does not need to consider the eavesdropper channel, that is, the probability of connection outage occurs.
In what follows, the exact analysis of $P_{co}$ is firstly performed, and the asymptotic connection outage probability is then derived in the high SNR regime, i.e., $\rho_1,\cdots,\rho_M\to \infty$.
\subsubsection{Exact Analysis}
% In \cite{2304.11341}, based on the hypothesis of the conditional independence of the received signal-to-noise ratios, the outage probability was exactly derived by using conditional Mellin transform. More specifically, the inner integral was then identified as a Mellin-Barnes integral and then as the Fox’s H-function. The outage probability of HARQ-IR over THz channel is consequently derived as (\ref{eqn:IR_8}), as shown at the top of the next page.
By substituting \eqref{eqn:harq IR} into \eqref{Pe}, the derivations of the connection outage probability amount to determining the distribution of the product of multiple random variables. This inspires us to capitalize on the Mellin transform. Fortunately, as proved in \cite{2304.11341}, this method was applied to derive the distribution of the accumulated mutual information of HARQ-IR aided THz communications in closed-form, as given by the following theorem.
 \begin{theorem} \label{the:clo_c}{\cite[eq.(11)]{2304.11341}}
 %Considering after $M$ HARQ rounds, the initial transmission rate is $R_0$, the channel coefficients chooses $h$, the closed form of outage probability can expressed by (\ref{eqn:IR_8}), written as $\Psi _h(M,R_0)$, as shown at the top of the next page, where $c_1 < 0$, ${\rm i} = \sqrt{-1}$, and the definition of the Fox’s H function is given by (\ref{eqn:HF}) at the top of next page.
The cumulative distribution function (CDF) of ${I_{\beta}(M)}$ can be expressed in terms of an inverse Laplace transform as {\eqref{eqn:IR_8}}, as shown at the top of the next page, where $\rm{c} < 0$, ${\rm i} = \sqrt{-1}$, and $ H_{p,q}^{m,n}(\cdot)$ refers to the Fox’s H function {\cite{ansari2017new}}. For the notational convenience, $\Psi _\beta^M(x)$ is used to represent the CDF of ${I_{\beta}(M)}$.
 % Figure environment removed
 \end{theorem}
It is noteworthy that \eqref{eqn:IR_8} can be evaluated fast and accurately by adopting the numerical inversion of Laplace transform {\cite{2304.11341}}.
% So that the the connection outage probability can be written as
% \begin{equation}
%     P_{CO} = \delta(M,R_0,h).
% \end{equation}
% $P_{co}$ can be written as $P_{XY}^{M,R_0}$.
% The closed-form solution of $P_{co}$ that means $Pr\left\{I_{XY}(M)<R_0\right\}$ in THz channels can find in [], as shown at the top of the next page, written as $P_{XY}^{M,R_0}$.The same is true for the eavesdropping channel representation, written as $P_{XZ}^{M,R_0-R_s}$.
%% Figure environment removed
Accordingly, the above theorem can be applied to express \eqref{Pe} in closed-form as
%So that the the connection outage probability can be written as
\begin{equation}\label{eqn:cn_cf}
    P_{co} = \Psi _B^M(R_0).
\end{equation}
\subsubsection{Asymptotic Analysis}
Clearly, the exact expression of the connection outage probability in \eqref{eqn:cn_cf} for $m>1$ is too complex to extract useful insights as well as ease the optimal design. Therefore, we perform the asymptotic analysis of the outage probability in the high SNR regime. %(\ref{eqn:IR_8}), because the form of (\ref{eqn:IR_8}) is too complex.
As proved in \cite{2304.11341}, the asymptotic behaviour of (\ref{eqn:IR_8}) under the conditions of $\rho_1,\cdots,\rho_M\to \infty$ was investigated by using the residue theorem and the dominant term approximation, as given by the following theorem.
 \begin{theorem}\label{the:asy}
In the high SNR regime, the asymptotic expression of $\Psi _\beta^M(x)$ can be written as {\eqref{eqn:IR_101}} at the top of the next page, where $\theta  = \min \left\{ {\mu \alpha ,\phi_\beta } \right\}$, $\Theta = \max \left\{ {\mu \alpha ,\phi_\beta } \right\}$, ${\bf 1}_{A}(x)$ denotes the indicator function such that ${\bf 1}_{A}(x)=1$ if $x\in A$, and ${\bf 1}_{A}(x)=0$ otherwise, and $ G_{p,q}^{m,n}(\cdot)$ represents the Meijer G-function {\cite{ansari2017new}}. For the notational convenience, $\Psi _\beta^{M,\infty}(x)$ is adopted to denote the asymptotic expression of $\Psi _\beta^M(x)$ in high SNR. %{\color{red}modify the relevant parameters in this theorem according to the definition in system model.}
% where ${{\cal D}}$ denotes the diversity order and is given by
% \begin{align}\label{eqn:d_I}
% {{\cal D}} = \frac{K\min\{\phi,\alpha \mu\}}{2}.%\left\{ \begin{array}{l}
% %\frac{{\phi M}}{2},\alpha \mu  - \phi  > 0\\
% %\frac{{\alpha \mu M}}{2},\alpha \mu  - \phi  < 0
% %\end{array} \right. .
% \end{align}
% where $B$ and $C$ are given by
% \begin{equation}\label{eqn:B}
% B = \Gamma \left( {\frac{\phi }{2}} \right)\Gamma (\mu  - \frac{\phi }{\alpha }){{\left( {{{\left| {{h_l}} \right|}^{ - 1}}{{\left( {\frac{\mu }{{\hat h_f^\alpha S_{\rm{0}}^\alpha }}} \right)}^{\frac{1}{\alpha }}}} \right)}^\phi },
% \end{equation}
% \begin{equation}\label{eqn:C}
% C = \frac{\alpha }{{\phi  - \mu \alpha }}\Gamma \left( {\frac{{\mu \alpha }}{2}} \right){{\left( {{{\left| {{h_l}} \right|}^{ - 1}}{{\left( {\frac{\mu }{{\hat h_f^\alpha S_{\rm{0}}^\alpha }}} \right)}^{\frac{1}{\alpha }}}} \right)}^{\mu \alpha }}.
% \end{equation}
% Figure environment removed
 \end{theorem}
%  {p_{M,R_0}^{asy}} &\simeq \left\{ \begin{gathered}原来的渐近在这里
%   {\left( {\frac{\phi }{{2\Gamma (\mu )}}} \right)^M}{{\hat \rho }^{\frac{{ - \phi M}}{2}}}\prod\limits_{m = 1}^M {q_m^{\frac{{ - \phi }}{2}}B\frac{1}{{2\pi {\text{i}}}}\int_{{{\text{c}}_1} - {\text{i}}\infty }^{{{\text{c}}_1} + {\text{i}}\infty } {{{\left( {{2^{ - R_0}}} \right)}^t}\frac{{\Gamma \left( { - t} \right)}}{{\Gamma \left( {1 - t} \right)}}\prod\limits_{m = 1}^M {\frac{{\Gamma \left( { - t - \frac{\phi }{2}} \right)}}{{\Gamma \left( { - t} \right)}}} dt,\mu \alpha  - \phi  > 0} }  \hfill \\
%   {\left( {\frac{\phi }{{2\Gamma (\mu )}}} \right)^M}{{\hat \rho }^{\frac{{ - \mu \alpha M}}{2}}}\prod\limits_{m = 1}^M {q_m^{\frac{{ - \mu \alpha }}{2}}C\frac{1}{{2\pi {\text{i}}}}\int_{{{\text{c}}_1} - {\text{i}}\infty }^{{{\text{c}}_1} + {\text{i}}\infty } {{{\left( {{2^{ - R_0}}} \right)}^t}\frac{{\Gamma \left( { - t} \right)}}{{\Gamma \left( {1 - t} \right)}}\prod\limits_{m = 1}^M {\frac{{\Gamma \left( { - t - \frac{{\mu \alpha }}{2}} \right)}}{{\Gamma \left( { - t} \right)}}} dt,\mu \alpha  - \phi  < 0} }  \hfill \\
% \end{gathered}  \right.\\
% &\simeq \left\{ \begin{gathered}
% {\left( {\frac{{{\mu ^{\frac{\phi }{\alpha }}}\Gamma (\mu  - \frac{\phi }{\alpha })}}{{{{\left| {{h_l}} \right|}^\phi }\hat h_f^\phi S_{\rm{0}}^\phi \Gamma (\mu )N_0^{\frac{\phi }{2}}}}} \right)^M}{\left( {\prod\limits_{m = 1}^M {{P_m}} } \right)^{ - \frac{\phi }{2}}}\left(\left(\Gamma \left( {\frac{\phi }{2}+1} \right)\right)^M  G_{M,M}^{0,M}{\left( {{2^{R_0}}\left| {_{0,1,...,1}^{1 + \frac{\phi }{2},1 + \frac{\phi }{2},...,1 + \frac{\phi }{2}}} \right.} \right)}\right),\mu \alpha  - \phi  > 0\\
% {\left( {\frac{{\phi {\mu ^{\mu  - 1}}}}{{\left( {\phi  - \mu \alpha } \right){{\left| {{h_l}} \right|}^{\mu \alpha }}\hat h_f^{\mu \alpha }S_{\rm{0}}^{\mu \alpha }\Gamma (\mu )N_0^{\frac{{\mu \alpha }}{2}}}}} \right)^M}{\left( {\prod\limits_{m = 1}^M {{P_m}} } \right)^{ - \frac{{\mu \alpha }}{2}}}\left(\left(\Gamma \left( {\frac{{\mu \alpha }}{2}+1} \right)\right)^M G_{M,M}^{0,M}{\left( {{2^{R_0}}\left| {_{0,1,...,1}^{1 + \frac{{\mu \alpha }}{2},1 + \frac{{\mu \alpha }}{2},...,1 + \frac{{\mu \alpha }}{2}}} \right.} \right)}\right),\\
% \quad\quad\quad\quad\quad\quad\quad\quad\quad\quad\quad\quad\quad\quad\quad\quad\quad\quad\quad\quad\quad\quad\quad\quad\quad\quad\quad\quad\quad\quad\quad\quad\quad\quad\quad\quad\quad\quad\quad\quad\quad\quad\mu \alpha  - \phi  < 0
% \end{gathered} \right.\\
 % The asymptotic outage analysis in the high SNR regime in THz channels also can find in [], as shown at the top of the this page,

% % Figure environment removed
With the asymptotic result in Theorem \ref{the:asy}, the connection outage probability $P_{co}$ is asymptotic to
\begin{equation}\label{eqn:asy}
    P_{co} \simeq \Psi _B^{M,\infty}(R_0).
\end{equation}

To conserve space, the in-depth discussions of the asymptotic expression \eqref{eqn:asy} are omitted here and interested readers are referred to \cite{2304.11341} for more details.
%From (\ref{eqn:IR_101}), the connection outage probability  is determined by the pointing error and the fading channels, the power allocation, the modulation and coding gain, the diversity order. The authors in \cite{2304.11341} made a more specific analysis on this.

% Which is written as
% \begin{align}\label{eqn:IR_1000}
% {p_{M,R_0}^{asy}} =  {\cal A}  {{\cal L}\left( P \right)} \left( \mathcal C(R_0)\right)^{-\mathcal D}
% \end{align}
% where ${\cal A}$ is the impact factor that combines the pointing error and fading channels, and is expressed as%which takes on different values depending on the sign of ${\mu \alpha  - \phi }$ and is denoted as
% \begin{align}\label{eqn:A}
% {\cal A} =
% \left\{ \begin{array}{l}
% {\left( {\frac{{{\mu ^{\frac{\phi }{\alpha }}}\Gamma (\mu  - \frac{\phi }{\alpha })}}{{{{\left| {{h_l}} \right|}^\phi }\hat h_f^\phi S_{\rm{0}}^\phi \Gamma (\mu )N_0^{\frac{\phi }{2}}}}} \right)^M},\mu \alpha  - \phi  > 0\\
% {\left( {\frac{{\phi {\mu ^{\mu  - 1}}}}{{\left( {\phi  - \mu \alpha } \right){{\left| {{h_l}} \right|}^{\mu \alpha }}\hat h_f^{\mu \alpha }S_{\rm{0}}^{\mu \alpha }\Gamma (\mu )N_0^{\frac{{\mu \alpha }}{2}}}}} \right)^M},\mu \alpha  - \phi  < 0
% \end{array} \right.,
% \end{align}
% the term ${{\cal L}\left( \bf P \right)}$ refers to the impact factor from the power allocation, which is obtained as
% \begin{align}\label{eqn:P}
% {\cal L}\left( \bf P \right) = \left\{ \begin{array}{l}
% {\left( {\prod\limits_{m = 1}^M {{P_m}} } \right)^{ - \frac{\phi }{2}}},\mu \alpha  - \phi  > 0\\
% {\left( {\prod\limits_{m = 1}^K {{P_m}} } \right)^{ - \frac{{\mu \alpha }}{2}}},\mu \alpha  - \phi  < 0
% \end{array} \right.,
% \end{align}
% ${\bf P} =(P_1,\cdots,P_M)$, $\mathcal C(R_0) = {{{\cal G}_M}\left( {{2^{R_0}}} \right)}^{-1/\mathcal D} $ stands for the modulation and coding gain, and ${{{\cal G}_M}\left( {{2^R_0}} \right)}$ is expressed as \eqref{eqn:g_m}, as shown at the top of this page.
% %% Figure environment removed
% ${{\cal D}}$ denotes the diversity order and is given by
% \begin{align}\label{eqn:d_I}
% {{\cal D}} = \frac{M\min\{\phi,\alpha \mu\}}{2}.%\left\{ \begin{array}{l}
% %\frac{{\phi M}}{2},\alpha \mu  - \phi  > 0\\
% %\frac{{\alpha \mu M}}{2},\alpha \mu  - \phi  < 0
% %\end{array} \right. .
% \end{align}

\subsection{Secrecy Outage Probability}
According to \cite{4802331}, by using the law of total probability, we can get the secrecy outage probability $P_{so}$ as
 \begin{equation}\label{Ps}
  P_{so} = \sum\limits_{m = 1}^M{\Pr\left\{\mathcal{M}=m\right\}\Pr\left\{{I_{E}(m)>R_0 - R_s}\right\}},
\end{equation}
where ${\mathcal{M}}$ denotes the number of transmissions within one HARQ cycle, that is, the number of HARQ rounds required to convey a single information message. In what follows, the exact and the approximate expressions of $P_{so}$ are derived. %, and we denote by $\Pr\left\{\mathcal{M}=m\right\}$ the probability mass function (pmf) of $\mathcal{M}$.
\subsubsection{Exact Analysis}
Clearly, since the distribution of $\mathcal{M}$ depends on whether Bob successfully receives the message over the main channels or the maximum number of transmissions is reached, and the probability mass function (pmf) of $\mathcal{M}$ can be obtained as \eqref{PDF_M},
% Figure environment removed
as shown at the top of the next page. By applying Theorem \ref{the:clo_c} to \eqref{PDF_M}, the pmf of $\mathcal{M}$ can be evaluated.

%\begin{equation}
 %   \Pr\left\{\mathcal{M}=m\right\} = \Pr\left\{I_{B}(m-1)<R_0,I_{B}(m)\ge R_0\right\},
%\end{equation}
%More specifically, it can be written as \eqref{PDF_M}, as shown at the top of the next page.
% where $\Pr(\mathcal{M}=m)$denote the probability mass function of the number of transmissions $\mathcal{M}$, can be expressed as
% % \begin{equation}\label{PDF_M}
% % \begin{aligned}
% %     &P(\mathcal{M}=m)=\\
% %   &\left\{ {\begin{array}{*{20}{c}}
% %   {\Pr\left\{{I_{XY}(m-1)<R_0}\right\}-\Pr\left\{{I_{XY}(m)<R_0}\right\}},{m=1,...,M-1}\\
% %   {\Pr\left\{{I_{XY}(M-1)<R_0}\right\}},{m=M}
% %   \end{array}} \right.
% %   \end{aligned}
% % \end{equation}
% \begin{equation}\label{PDF_M}
% \begin{aligned}
%     \Pr(\mathcal{M}=m)&=\Pr\left\{I_{XY}(m-1)<R_0,I_{XY}(m)\ge R_0\right\}\\
%     &=\Pr\left\{{I_{XY}(m-1)<R_0}\right\}\\
%     &\quad -\Pr\left\{{I_{XY}(m)<R_0}\right\},m=1,...,M-1  \\
%     \Pr(\mathcal{M}=M)&=\Pr\left\{{I_{XY}(M-1)<R_0}\right\}
%   % \left\{ {\begin{array}{*{20}{c}}
%   % {\Pr\left\{{I_{XY}(m-1)<R_0}\right\}-\Pr\left\{{I_{XY}(m)<R_0}\right\}},{m=1,...,M-1}\\
%   % {\Pr\left\{{I_{XY}(M-1)<R_0}\right\}},{m=M}
%   % \end{array}} \right.
%   \end{aligned}
% \end{equation}
%From (\ref{Ps}) and (\ref{PDF_M}), $P_{so}$ is defined as the probability that the eavesdropper can correctly decode the information transmitted in a given HARQ round.
% \begin{align}
% P(M=M)=P({I_{XY}(M-1)<R_0})
% \end{align}
Moreover, with regard to the term $\Pr\left\{I_{E}(M)> R_0 - R_s\right\}$ in \eqref{Ps}, note that this complementary CDF (CCDF) follows as $\Pr\left\{I_{E}(M) > R_0 - R_s\right\} = 1-\Pr\left\{I_{E}(M) < R_0 - R_s\right\}$. Hence, $\Pr\left\{I_{E}(M) > R_0 - R_s\right\}$ can also be obtained by using Theorem \ref{the:clo_c} as
%Considering the eavesdropper channel, a secrecy transfers occurs when $\Pr\left\{I_{XZ}(M) < R_0 - R_s\right\}$, and secrecy transfers and secrecy outage are exclusive events, so
% when the cumulative mutual information of the eavesdropper exceeds the dummy-message rate $R_0-R_s$, we consider the eavesdropper to have successfully eavesdropped, that is (\ref{Ps}), a secrecy outage occurs. What's more, secrecy transfers and secrecy outage are exclusive events, so
\begin{equation}\label{Pr}
\begin{aligned}
     \Pr\left\{I_{E}(M)> R_0 - R_s\right\} &= 1 - \Psi _E^M(R_0-R_s).
\end{aligned}
\end{equation}
% where $\Pr\left\{I_{XZ}(M) < R_0 - R_s\right\}$ can be denoted by $\Psi _g(M,R_0-R_s)$, in other words,
% \begin{equation}\label{Pr}
%     \Pr\left\{I_{XZ}(M)> R_0 - R_s\right\} = 1 - \Psi _g(M,R_0-R_s).
% \end{equation}\\
By substituting (\ref{Pr}) into (\ref{Ps}), $P_{so}$ can be consequently expressed as
\begin{equation}
    \begin{aligned}\label{PSO_1}
        P_{so} =& \sum\limits_{m = 1}^{M-1}(\Psi_B^{m-1}(R_0)-\Psi_B^{m}(R_0))
        (1-\Psi_E^m(R_0-R_s))\\
        &+\Psi_B^{M-1}(R_0)(1-\Psi_E^{M}(R_0-R_s)),
    \end{aligned}
\end{equation}
where we stipulate $\Psi _B^0(R_0)=1$.
%modify it according to the new definition.}
% \begin{equation}
%     \begin{aligned}
%         P_{so} =& \sum\limits_{m = 1}^{M-1}\left\{P_{XY}^{m-1,R_0}-P_{XY}^{m,R_0}\right\}
%         \left\{1-P_{XZ}^{m,R_0-R_s}\right\}\\
%         &+P_{XY}^{M-1,R_0}\left\{1-P_{XZ}^{M,R_0-R_s}\right\}
%     \end{aligned}
% \end{equation}
\subsubsection{Approximate Analysis}
It is obvious that the secrecy outage probability in \eqref{PSO_1} is too cumbersome to facilitate the system design. In the meantime, it is different from the analysis of $P_{co}$ that the investigations into the asymptotic expression of $P_{so}$ in high SNR are meaningless. This is because $P_{so}$ tends to 1 under high SNR (i.e., $\rho_1,\cdots,\rho_M\to \infty$) if we directly apply Theorem \ref{the:clo_c} to \eqref{PSO_1}. In other words, the confidential message will be almost surely intercepted by eavesdropper in the high SNR regime. To overcome the shortcoming of the asymptotic analysis, an approximate result of the secrecy outage probability can be obtained based on the large deviation \cite{400650}. By assuming uniform power allocation, i.e., $\rho_1=\cdots=\rho_M$, the CCDF of ${I_{E}}(M)$ is upper bounded by using the large deviation as
%The goal of physical layer security is to prevent passive eavesdroppers from stealing communication content, and passive eavesdroppers include not only attackers outside the communication system but also legitimate users within the system. In some cases, even legitimate users may attempt to eavesdrop on communication signals, such as attempting to eavesdrop on others' communication content to gain advantage information. Therefore, the channel characteristics of passive eavesdroppers that we consider should also be sufficiently good.
%When the SNR is high enough, the secrecy outage probability will tend to 1, so considering asymptotic results at high SNR will be meaningless. We need to extract useful physical insights from other aspects.
%It is worth noting that the use of the HARQ-IR method is not only beneficial for legitimate users to successfully receive information, but also for eavesdroppers to successfully eavesdrop on secret messages, so it is necessary to consider the probability of a secrecy outage occurs, $\Pr\left\{I_{XZ}(M)>R_0-R_s\right\}$, when $M$ is very large.
%By using large deviations \cite{400650}, for a sufficiently large $M$, we can obtain the upper bound of the probability of a secrecy outage occurs.
\begin{align}\label{large}
\Pr\left\{{I_{E}}(M) > {R_0} - {R_s}\right\} &= \Pr\left(\sum\limits_{m = 1}^M { Z_m  > Mr_M}\right) \notag\\
&\le {e^{ - M\mathcal{I}(r_M)}},
\end{align}
where $r_M = (R_0-R_s)/M$, $Z_m={\log_2 (1 + \rho_m|{h_{E,l}}{|^2}|{h_{E,pf,m}}{|^2})}$, $\mathcal{I}(r_M) = \mathop {\max }\nolimits_{s \ge 0} \{ sr_M - \lambda (s)\}$ denotes the rate function,
%\begin{align}
%    \mathcal{I}(r) &= \mathop {\max }\limits_{s \ge 0} \{ sr - \lambda (s)\} \\
%    \nonumber      &=- \mathop {\min }\limits_{s \ge 0} \{  - sr + \lambda (s)\} .
%\end{align}
$\lambda (s) = \ln \mathbb E[{e^{  s{Z_m}}}]$ is the logarithmic moment generating function (MGF) of $Z_m$, and $\mathbb E$ refers to the expectation operator.
% \begin{equation}
%     \lambda (s) = \log \mathbb E[{e^{  s{Z_m}}}].
% \end{equation}
% \[\lambda (s) = \log \mathbb E[{e^{  s{Z_m}}}].\]
As proved in Appendix \ref{eqn:8}, the MGF of $Z_m$ can be derived as
{\begin{multline}\label{E}
    \mathbb E[{e^{  s{Z_m}}}] =
    \frac{{\xi {{\rho _m}^{ - \frac{\phi_E }{2}}{{\left| {{h_{E,l}}} \right|}}^{ - {\phi_E }}}}}{{\Gamma ({s/\ln2})}}\\
    \times H_{2,3}^{3,1}\left[{\frac{\mathcal{C}}{{{{\rho _m}^{\frac{\alpha }{2}}{{\left| {{h_{E,l}}} \right|}}^{{\alpha }}}}}\left|{_{(0,1),(K,1),( - \frac{\phi_E }{2} - {s/\ln2},\frac{\alpha }{2})}^{(1 - \frac{\phi_E }{2},\frac{\alpha }{2}),(1,1)}}\right.}\right].
\end{multline}
where
%$\mathcal{P} = {\rho _m}{{\left| {{h_{E,l}}} \right|}^2}$,
$\mathcal{C} ={u}{{\left({S_E}{\hat h_{f,E}}\right)^{-\alpha} }}$,
 $\xi  ={{\phi_E {u^{\frac{{ \phi_E }}{\alpha }}}}} {{{S_E}^{-\phi_E} {\hat h_{f,E}}^{-\phi_E} {\Gamma (u)^{-1}}}}$, $K ={ u - \alpha^{-1}\phi_E } $.  }
 %don't use $\times $ if unnecessary! change the symbols according to the definitions in the system model. I have used $L$ before, replace it with another symbol.} %\par
%Proof: See Appendix \ref{eqn:8}.\par
% l(r)  &= \mathop {\max }\limits_{s \ge 0} \{ sr - \lambda (s)\} \\
%       &=- \mathop {\min }\limits_{s \ge 0} \{  - sr + \lambda (s)\} .
% \end{array}
% \end{equation}
%The rate function relates the probability to the parameter r, it describes the relationship between the degree of deviation from expected value and the probability.
According to the theory of large deviation, for a sufficiently large $M$, the secrecy outage probability approaches to the upper bound if $r_M < \mathbb E[Z_m]$. This justifies the significance of the upper bound given by \eqref{large}. %Specifically, in the case that the state of the main channel is not much better than the eavesdropper channel, we need to weigh the connection outage probability against the secrecy outage probability.\par
% Specifically,
% \begin{equation}
%     \lambda (s) = \log \mathbb E[{e^{  s{Z_m}}}].
% \end{equation}
% % \[\lambda (s) = \log \mathbb E[{e^{  s{Z_m}}}].\]
% What's more,
% \begin{multline}\label{E}
%     \mathbb E[{e^{  s{Z_m}}}] = \frac{{\varepsilon {P^{ - \frac{\phi }{2}}}}}{{\Gamma (s{{\log }_2}e)}}H_{2,3}^{3,1}\left[\frac{L}{{{P^{\frac{\alpha }{2}}}}}|_{(0,1),(K,1),( - \frac{\phi }{2} - s{{\log }_2}e,\frac{\alpha }{2})}^{(1 - \frac{\phi }{2},\frac{\alpha }{2}),(1,1)}\right].
% \end{multline}
% where
%  $P = \frac{P}{{{N_0}}}|{g_l}{|^2}$, $L = \frac{u}{{{S_0}^\alpha {{\mathord{\buildrel{\lower3pt\hbox{$\scriptscriptstyle\frown$}}\over h} }_f}^\alpha }}$,
%  $\varepsilon  = \frac{{\phi {u^{u - \frac{{u\alpha  - \phi }}{\alpha }}}}}{{{S_0}^\phi {{\mathord{\buildrel{\lower3pt\hbox{$\scriptscriptstyle\frown$}}\over h} }_f}^\phi \Gamma (u)}}$,
% $K = \frac{{\alpha u - \phi }}{\alpha }$\par
% Proof: See Appendix \eqref{eqn:8}.
% \begin{equation}
%     \begin{aligned}\label{eqn:Pr}
%         Pr\left\{I_{xz}(M)>R_0-R_s\right\}\simeq {e^{ - Ml(r) + o(M)}}
%     \end{aligned}
% \end{equation}
% Proof: See Appendix \eqref{eqn:8}.
The same approach also applies to derive the upper bound of the CCDF of ${I_{E}}(m)$ for $m<M$. By considering the upper bound of the CCDF of ${I_{E}}(m)$ for $m>1$, the secrecy outage probability of \eqref{PSO_1} is upper bounded as
\begin{align}\label{eqn:up_so}
        %P_{so} \le &\sum\limits_{m = 1}^{M-1}(\Psi _h(m-1,R_0)-\Psi _h(m,R_0,h))
        %{e^{ - m\mathcal{I}(r)+ o(M)}}\\
        %&+\Psi _h(M-1,R_0){e^{ - M\mathcal{I}(r)+ o(M)}}.
        {P_{so}} \le& \left( {1 - \Psi _B^1\left( {{R_0}} \right)} \right)\left( {1 - \Psi _E^1\left( {{R_0} - {R_s}} \right)} \right)  \notag\\
        &+ \sum\limits_{m = 2}^{M - 1} {\left( {\Psi _B^{m - 1}\left( {{R_0}} \right) - \Psi _B^m\left( {{R_0}} \right)} \right){e^{ - m{\cal I}({r_m})}}}  \notag\\
        &+ \Psi _B^{M - 1}\left( {{R_0}} \right){e^{ - M{\cal I}({r_M})}},%\notag\\
        % = &{e^{ - {\cal I}(r)}} - \left( {1 - {e^{ - {\cal I}(r)}}} \right)\sum\limits_{m = 1}^{M - 1} {{e^{ - m{\cal I}(r)}}\Psi _B^m\left( {{R_0}} \right)}.
\end{align}
where $r_m = (R_0-R_s)/m$. To further reduce the computational complexity, by substituting the asymptotic result of $ {\Psi _B^{m }\left( {{R_0}} \right)}$ in \eqref{eqn:asy} for $m>1$ into \eqref{eqn:up_so}, the secrecy outage probability can be approximated as
\begin{align}\label{eqn:se_appr}
    {P_{so}} \approx &
\left( {1 - \Psi _B^{1}\left( {{R_0}} \right)} \right)\left( {1 - \Psi _E^{1}\left( {{R_0} - {R_s}} \right)} \right) \notag\\
 & + \sum\limits_{m = 2}^{M - 1} {\left( {\Psi _B^{m - 1,\infty }\left( {{R_0}} \right) - \Psi _B^{m,\infty }\left( {{R_0}} \right)} \right){e^{ - m{\cal I}({r_m})}}} \notag\\
 & + \Psi _B^{M - 1,\infty }\left( {{R_0}} \right){e^{ - M{\cal I}({r_M})}}.
\end{align}
% and We wish to consider the asymptotic leak probability as m increases,this property can be characterized by the large deviation decay rate function I defined as follows,
% \begin{equation}
% {l^*} = \mathop {\lim }\limits_{m \to \infty }  - \frac{1}{m}\log \Pr \{ \sum\limits_{i = 1}^m {{Z_i} > mr\} }
% \end{equation}
% the exponent $l^*$characterizes the tail distribution of the worst case transmission times $m \to \infty $.It illustrates how quickly the probability of being tapped successfully rise when the m grows. A small rate $l^*$corresponds to better performance of the system in the sense of increased secrecy.


\subsection{Secrecy LTAT}\label{sec:asy_I}
According to {\cite{4802331}}, the secrecy long term average throughput (LTAT) can be expressed as
%The secrecy throughput $\eta$ can be obtained by updating the reward method, which is defined as
\begin{equation}\label{sec}
\begin{aligned}
  \eta = %\mathop {\lim }\limits_{t \to \infty } \frac{{b_t^{suc}}}{t} =
  \frac{R_s(1-\Psi _B^M\left( {{R_0}} \right))}{\mathbb E[\mathcal{M}]},
\end{aligned}
\end{equation}
where $\mathbb E[\mathcal{M}]$ indicates the expected number of transmissions and is given by
\begin{equation}
    \mathbb E[\mathcal{M}] = \sum\limits_{m=1}^{M}m \Pr\left\{\mathcal{M}=m\right\} = 1+\sum\limits_{m=1}^{M-1}\Psi _B^m\left( {{R_0}} \right).
\end{equation}
Furthermore, the asymptotic expression of the secure LTAT can be calculated by replacing $\Psi _B^m\left( {{R_0}} \right)$ in \eqref{sec} with $\Psi _B^{M,\infty}(R_0)$. It can be proved that the asymptotic expression of $\eta$ actually offers a lower bound of the actual secure LTAT, and interested readers are referred to \cite{7959548} for the details.
%where $b_t^{suc}$ denotes the total number of successfully received information bits till time $t$.
% What's more, $1+\sum\limits_{m=1}^{M-1}\Psi _h(m,R_0)$ indicates the expected number of transmissions, which is given by
% \begin{equation}
%     \mathbb E[\mathcal{M}] = \sum\limits_{m=1}^{M}m\cdot \Pr(\mathcal{M}=m) = 1+\sum\limits_{m=1}^{M-1}\Psi _h(m,R_0).
% \end{equation}


% \par Under the connection interruption constraint and the secrecy constraint, we can get the maximum secrecy throughput by choosing the parameters $R_0$ and $R_s$. Hence, we consider the following problem:
% \begin{equation}\label{eqn:rate}
% \begin{array}{*{20}{c}}
% {\mathop {\max }\limits_{R_0,R_S} }&{{\eta(R_0,R_s)}}\\
% {{\rm{s}}{\rm{.t.}}}&P_{CO}\leq\xi_c,P_{SO}\leq\xi_e
% \end{array}
% \end{equation}
% where $\xi_s$ and $\xi_e$ represent a target connection outage probability and a target secrecy outage probability. They represent minimum communication quality of service requirements and security confidentiality requirements. That is to say, at least a fraction $1-\xi_e$ of HARQ-IR sessions are successful and at least a fraction $1-\xi_e$ of confidential message bits are kept completely secret.


\section{Numerical Results}\label{sec:NR}
In this section, Monte Carlo simulations are conducted to verify our analytical results. Unless otherwise noted, the system parameters are set as follows, $\alpha  = 1$, $\mu  = 2$, $\psi  = 50\% $, $T = 296{\rm{^\circ }}$~K, $p = 101325$~Pa, ${\sigma _{\rm{s}}} = 1$, and the carrier frequency is $f=275$~GHz. Moreover, both the transmit and the Bob's receive antenna gains are equal to $55$~dBi, the Eve's receive antenna gain is $50$~dBi, the transmission distances from Alice to Bob and Eve are assumed to be $20$~m and $40$~m, respectively. Besides, the main code rate and the confidential rate are set as $R_0$ = 3 bps/Hz and $R_s$ = 2 bps/Hz, respectively, and the transmit SNRs during all HARQ rounds are assumed to be the same, i.e., $\rho_1=\cdots=\rho_M\triangleq \gamma_T$.


\subsection{Performance Evaluation}
% Figure environment removed

% \par Fig.\ref{fig:2} depicts the  secrecy outage probability Pso versus the average transmit SNR  for secrecy HARQ-aided THz communication schemes.It is clearly seen that the exact and simulated results are in perfect agreement. And they converge to the asymptotic ones in the high SNR regime. This observation clearly confirms the validity of our asymptotic results.

% Figure environment removed





% % Figure environment removed

Figs. \ref{fig:1} and \ref{fig:2} depict the connection outage probability $P_{co}$ and the secrecy outage probability $P_{so}$ %of secure HARQ-IR-aided THz communications
versus the average transmit SNR $\gamma_T$, respectively. It is clearly seen from both figures that the exact and simulated results are in perfect agreement. One can also observe from Fig. \ref{fig:1} that the asymptotic results tightly approach to the exact ones with the increase of the SNR. Furthermore, the upper bound of the secrecy outage probability (labeled as ``upp.'') in Fig. \ref{fig:2} is plotted according to \eqref{eqn:up_so}, which justifies the validity of our analysis. Moreover, it can be observed that the proposed HARQ-aided communications can significantly reduce the connection outage probability, albeit at the cost of increasing the secrecy outage probability by comparing to THz communications without HARQ  (labeled as ``No HARQ'') \cite{9039743}. Hence, it is necessary to properly devise the HARQ-aided scheme to guarantee both the secrecy and the connection outage requirements. More discussions are deferred to the next subsection.%is that the utilization of HARQ-IR technology reduces the $P_{co}$ and increases the $P_{so}$

% fig.3 represents the secrecy throughput depending on the rate
% parameter Rs  for different constraints $\xi_s$. The shape of the
% curve depends on the secrecy outage constraint $P_{so}<\xi_s$.
% Figure environment removed

Fig. \ref{fig:3} presents the secrecy LTAT $\eta$ versus the average transmit SNR. It can be seen from Fig. \ref{fig:3} that the analytical results coincide with the simulated ones. %Moreover, it is observed that the asymptotic results of the secrecy LTAT in Fig.\ref{fig:3} are obtained by replacing $\Psi _B^m\left( {{R_0}} \right)$ with $\Psi _B^{M,\infty}(R_0)$ in \eqref{sec}.
As expected, the asymptotic results of the secrecy LTAT in Fig. \ref{fig:3} indeed act as a lower bound of $\eta$, which is consistent with the theoretical analysis in \cite{7959548}. Furthermore, it can be observed that the proposed HARQ-IR-aided THz communications outperform THz communications without HARQ in terms of LTAT.
%What's more, increasing the number of retransmissions can lead to a decrease in $P_{co}$ and an increase in $P_{so}$. However, in general, increasing the number of retransmissions is beneficial for increasing the secrecy throughput.

\subsection{Robust Design of Rate Adaption}
It should be noticed that HARQ-IR not only enhances the reliability of the legitimate user, but also increases the probability of being eavesdropped. To address this issue, the main code rate $R_0$ and confidential information rate $R_s$ are optimally designed to maximize the secure LTAT meanwhile guaranteeing the connection and the secrecy outage probabilities. The problem can be mathematically formulated as% The advantage of physical layer security is that the maximum secrecy throughput can be obtained under the condition of satisfying the connection requirement and the secrecy requirement by choosing the parameters $R_0$ and $R_s$. Hence, we consider the following problem:
 % Under the connection interruption constraint and the secrecy constraint, we can get the maximum secrecy throughput by choosing the parameters $R_0$ and $R_s$. Hence, we consider the following problem:
\begin{equation}\label{eqn:rate}
\begin{array}{*{20}{c}}
{\mathop {\max }\limits_{R_0,R_s} }&{{\eta}}\\
{{\rm{s}}{\rm{.t.}}}&P_{ co}\leq\varepsilon_c,\\
                    &P_{so}\leq\varepsilon_e
\end{array}
\end{equation}
where $\varepsilon_c$ and $\varepsilon_e$ represent the maximum allowable connection outage probability and secrecy outage probability, respectively. %They represent minimum communication quality of service requirements and security confidentiality requirements. That is to say, at least a fraction $1-\xi_c$ of HARQ-IR sessions are successful and at least a fraction $1-\xi_e$ of confidential message bits are kept completely secret.
In order to reduce the computational complexity, the asymptotic connection outage probability in \eqref{eqn:asy} and the approximate secrecy outage probability in \eqref{eqn:se_appr} are used to replace the exact expressions in \eqref{eqn:rate}. Since both the asymptotic results and the approximate results can be regarded as the worst-case performance benchmark, the proposed rate adaption method is actually a robust design for secure HARQ-IR-aided THz communications. %, Among them, the secrecy outage probability is calculated using the upper bound result, which ensures the robustness of the chosen code rate

% Figure environment removed

Fig. \ref{fig:4} illustrates the optimal secrecy LTAT versus the the average transmit SNR under different $M$. It is also observed that the increase of the maximum number of transmission is favorable for the improvement of the secure LTAT. In particular, it is clearly observed that the proposed HARQ-IR-aided scheme achieves a considerable LTAT improvement by comparing to No-HARQ, especially under a strict maximum allowable outage tolerance. %that in the presence of constraints, by selecting appropriate parameters, an increase in the number of retransmissions will increase the optimal secrecy  throughput.


%the maximal secrecy throughput $\eta$ increases when the outage constraint reduces.
%It shows that $P_{SO}<\xi_e$ determines the optimal secrecy throughput that can be achieved, and $P_{CO}<\xi_c$ determines the maximum $R_S$ that can be transmitted.

% % Figure environment removed

% % \par Fig.\ref{fig:2} depicts the  secrecy outage probability Pso versus the average transmit SNR  for secrecy HARQ-aided THz communication schemes.It is clearly seen that the exact and simulated results are in perfect agreement. And they converge to the asymptotic ones in the high SNR regime. This observation clearly confirms the validity of our asymptotic results.

% % Figure environment removed


% % fig.3 represents the secrecy throughput depending on the rate
% % parameter Rs  for different constraints $\xi_s$. The shape of the
% % curve depends on the secrecy outage constraint $P_{so}<\xi_s$.
% % Figure environment removed


% % Figure environment removed

\section{Conclusions}\label{sec:con}
In this paper, we confined our analysis to the outage and throughput performance of secure HARQ-IR assisted THz communications. More specifically, the connection outage probability was derived in closed-form, with which its asymptotic analysis in high SNR was performed. Then the secrecy outage performance was examined by conducting exact and approximate analyses. With these basis results, the secrecy LTAT was calculated accordingly. Finally, a robust rate adaption policy was proposed to maximize the LTAT while guaranteeing the outage constraints.
%, including the connection outage probability, the secrecy outage probability, and the secrecy LTAT
%The main contribution of this paper is to quantify the performance of HARQ-IR over THz channels in physical layer security through exact analysis. This result filled the gap for the performance of achieving physical layer security from the perspective of information theory in the HARQ-IR-Aided THz channel. By setting the rate parameter, we can ensure confidentiality and reliability. The analytical results were finally validated through Monte Carlo simulations.




\appendices


\section{Proof of \eqref{E}}\label{eqn:8}
The MGF of $Z_m$, i.e., $\mathbb E[{e^{  s{Z_i}}}]$, can be written as
\begin{equation}\label{eqn:mgf_deff1}
    %C_g = \mathbb E[{e^{  s{{\log }_2}(1 + \mathcal{P}|{g_{f,{x_i}}}{|^2})}}]\\
    \mathbb E[{e^{  s{Z_i}}}]    = \mathbb E[{(1 + {\rho _m}{{\left| {{h_{E,l}}} \right|}^2}|{g_{f,{x_i}}}{|^2})^{  {s/\ln2}}}].
\end{equation}
By putting the PDF of \eqref{eqn:PDF} into \eqref{eqn:mgf_deff1}, it follows that
\begin{equation}\label{eqn:mgf_deff2}
   \mathbb E[{e^{  s{Z_i}}}] = \xi \int_0^\infty  ({1 + } {\rho _m}{{\left| {{h_{E,l}}} \right|}^2}{x^2}){^{ - {s/\ln2}}}{x^{\phi_E  - 1}}\Gamma (K,\mathcal{C}{x^\alpha })dx.
\end{equation}
Moreover, by employing the representations of the Meijer G-functions \cite[eq.(8.4.2.5), eq.(8.4.16.2)]{brychkov1986integrals}, \eqref{eqn:mgf_deff2} can be rewritten as
\begin{align}\label{z}
    & \mathbb E[{e^{  s{Z_i}}}] = \frac{\xi }{{\Gamma ({s/\ln2})}} \times\notag\\
    &  \int_0^\infty  {{x^{\phi_E  - 1}}G_{1,1}^{1,1}\left[{\rho _m}{{\left| {{h_{E,l}}} \right|}^2}{x^2}\left|{_0^{ - {s/\ln2} + 1}}\right.\right]G_{1,2}^{2,0}\left[\mathcal{C}{x^\alpha }\left|{_{0,K}^1}\right.\right]dx}.
\end{align}
%where %$G_{p,q}^{m,n}\left[ {\rm{\cdot}} \right]$ is Meijer's G function, which is given by (\ref{MeijerG}).
%\begin{multline}\label{MeijerG}
%G_{p,q}^{m,n}\left( {{{x}}\left| {_{{b_{1,...,{b_p}}}}^{{a_{1,...}},{a_p}}} \right.} \right)
%= \\
%\frac{1}{{2\pi i}}\int_L^{} {\frac{{\prod\limits_{j = 1}^m {\Gamma ({b_j} - s)\prod\limits_{j = 1}^n {\Gamma (1 - {a_j} %+ s)} } }}{{\prod\limits_{j = m + 1}^q {\Gamma (1 - {b_j} + s)\prod\limits_{j = n + 1}^p {\Gamma ({a_j} - s)} } }}%{x^s}ds}
%\end{multline}
By change of variable $z = x^2$ and identifies this integral with the representation of Fox's H function {\cite[eq.(21)]{adamchik1990algorithm} and \cite [eq.(8.3.2.21)]{brychkov1986integrals}}, (\ref{z}) can be consequently derived as \eqref{E}.

%\begin{equation}\label{92}
%    C_g = \frac{\varepsilon }{{\Gamma (s{{\log }_2}e)}}\int_0^\infty  {{z^{\frac{\phi }{2} - 1}}G_{1,1}^{1,1}\left[\mathcal{P}z|_0^{  s{{\log }_2}e + 1}\right]G_{1,2}^{2,0}\left[L{z^{\frac{\alpha }{2}}}|_{0,K}^1\right]dz}
%\end{equation}
%Finally, by using \cite{onLine} into (\ref{92}), it can be written as
%\begin{equation}
%    C_g = \frac{{\varepsilon {\mathcal{P}^{ - \frac{\phi }{2}}}}}{{\Gamma (s{{\log }_2}e)}}H_{2,3}^{3,1}\left[\frac{L}{{{\mathcal{P}^{\frac{\alpha }{2}}}}}|_{(0,1),(K,1),( - \frac{\phi }{2} - s{{\log }_2}e,\frac{\alpha }{2})}^{(1 - \frac{\phi }{2},\frac{\alpha }{2}),(1,1)}\right]
%\end{equation}
%Then the proof directly follows.

% \begin{equation}
% \begin{array}{l}
% E[{e^{  s{Z_i}}}]\\
%  = E[{e^{  s{{\log }_2}(1 + P|{h_{f,{x_i}}}{|^2})}}]\\
%  = E[{(1 + P|{h_{f,{x_i}}}{|^2})^{  s{{\log }_2}e}}]\\
%  = \varepsilon \int_0^\infty  {(1 + } P{x^2}{)^{ - s{{\log }_2}e}}{x^{\phi  - 1}}\Gamma (K,L{x^\alpha })dx\\
%  = \frac{\varepsilon }{{\Gamma (s{{\log }_2}e)}}\int_0^\infty  {{x^{\phi  - 1}}G_{1,1}^{1,1}[P{x^2}|_0^{  s{{\log }_2}e + 1}]G_{1,2}^{2,0}[L{x^\alpha }|_{0,K}^1]dx} \\
%  = \frac{\varepsilon }{{\Gamma (s{{\log }_2}e)}}\int_0^\infty  {{z^{\frac{\phi }{2} - 1}}G_{1,1}^{1,1}[Pz|_0^{  s{{\log }_2}e + 1}]G_{1,2}^{2,0}[L{z^{\frac{\alpha }{2}}}|_{0,K}^1]dz} \\
%  = \frac{{\varepsilon {P^{ - \frac{\phi }{2}}}}}{{\Gamma (s{{\log }_2}e)}}H_{2,3}^{3,1}[\frac{L}{{{P^{\frac{\alpha }{2}}}}}|_{(0,1),(K,1),( - \frac{\phi }{2} - s{{\log }_2}e,\frac{\alpha }{2})}^{(1 - \frac{\phi }{2},\frac{\alpha }{2}),(1,1)}]
% \end{array}
% \end{equation}

% \begin{thebibliography}{99}
% \bibitem{8766143}Z. Zhang, Y. Xiao, Z. Ma, M. Xiao, Z. Ding, X. Lei, G. K. Karagiannidis,
% and P. Fan, ‘‘6G wireless networks: Vision, requirements, architecture, and
% key technologies,’’ \textit{IEEE Veh. Technol. Mag.}, vol. 14, no. 3, pp. 28–41,
% Sep. 2019.
% \bibitem{8610080}A.-A. A. Boulogeorgos and A. Alexiou, "Error analysis of mixed THz-RF wireless systems," \textit{IEEE Commun. Lett.}, vol. 24, no. 2, pp. 277-281, Feb. 2020.
% \bibitem{9039743}E. N. Papasotiriou, A.-A. A. Boulogeorgos, and A. Alexiou,
% “Performance analysis of THz wireless systems in the presence
% of antenna misalignment and phase noise,” \textit{IEEE Commun. Lett.},
% vol. 24, no. 6, pp. 1211–1215, Jun. 2020.
% \bibitem{THz2_1} C. Mukherjee, B. Ardouin, J.Y. Dupuy, V. Nodjiadjim, M.
% Riet, T. Zimmer, F. Marc and C. Maneux, "Reliability-Aware
% Circuit Design Methodology for Beyond-5G Communication
% Systems", \textit{IEEE Transactions on Device and Materials
% Reliability}, vol. 17, no. 3, pp. 490-506, 2017.
% \bibitem{THz2_2}S. Xie, H. Li, L. Li, Z. Chen, and S. Li, “Reliable and energy-aware job
% offloading at terahertz frequencies for mobile edge computing,” \textit{China
% Commun.}, vol. 17, no. 12, pp. 17–36, Dec. 2020.
% \bibitem{THz2_3}Z. Song, J. Feng, Z. Shi, Q. Dou, G. Yang, Y. Li, and S. Ma, “Outage
% probability analysis of Harq-aided terahertz communications,” in Proc. Int. Conf. Wireless Commun. and Signal
% Processing (WCSP). IEEE, 2021, pp. 1–6.
% \bibitem{PLS0_1} A. D. Wyner, ‘‘The wire-tap channel,’’ \textit{Bell Syst. Tech. J.}, vol. 54, no. 8,
% pp. 1355–1387, 1975.
% \bibitem{9482609}P. Porambage, G. Gür, D. P. M. Osorio, M. Liyanage, and
% M. Ylianttila, “6G security challenges and potential solutions,” \textit{in
% Proc. IEEE Joint Eur. Conf. Netw. Commun. (EuCNC) 6G Summit},
% 2021, pp. 1–6.
% \bibitem{8964397} Z. Liu, J. Liu, Y. Zeng, and J. Ma, “Covert wireless communication
% in IoT network: From AWGN channel to THz band,” \textit{IEEE Internet
% Things J.}, vol. 7, no. 4, pp. 3378–3388, Apr. 2020.
% \bibitem{8845312} V. Petrov, D. Moltchanov, J. M. Jornet, and Y. Koucheryavy,
% “Exploiting multipath terahertz communications for physical layer
% security in beyond 5G networks,” \textit{in Proc. IEEE INFOCOM Conf.
% Comput. Commun. Workshops (INFOCOM WKSHPS)}, Paris, France,
% 2019, pp. 865–872.
% \bibitem{PLS0_2}J. Qiao and M.-S. Alouini, “Secure transmission for intelligent reflecting
% surface-assisted mmWave and terahertz systems,” \textit{IEEE Wireless Commun. Lett.}, vol. 9, no. 10, pp. 1743–1747, Oct. 2020.
% % \bibitem{PLS0_3}A. M. T. Khel and K. A. Hamdi, “Secrecy capacity of irs-assisted
% % terahertz wireless communications with pointing errors,” IEEE Com-
% % munications Letters, 2023.
% \bibitem{930931} G. Caire and D. Tuninetti, “The throughput of hybrid-ARQ protocols
% for the Gaussian collision channel,” \textit{IEEE Trans. Inf. Theory}, vol. 47,
% no. 5, pp. 1971–1988, Jul. 2001.
% \bibitem{4802331} X. Tang, R. Liu, P. Spasojevic, and H. V. Poor, “On the throughput
% of secure hybrid-ARQ protocols for Gaussian block-fading channels,”
% \textit{IEEE Trans. Inf. Theory}, vol. 55, no. 4, pp. 1575–1591, Apr. 2009.
% \bibitem{400650} A. Weiss, “An introduction to large deviations for communication networks,” \textit{IEEE J. Sel. Areas Commun.}, vol. 13, no. 6, pp. 938–952, Aug.
% 1995.
% \bibitem{prudnikov1988integrals}A. P. Prudnikov, Y. A. Brychkov, and O. I. Marichev, Integral Series: More
% Special Functions, vol. 3. Boca Raton, FL, USA: CRC Press, 1990.
% \bibitem{onLine}The Wolfram Functions Site. Accessed: Aug. 8, 2018. [Online]. Available:
% http://functions.wolfram.com/07.34.21.0012.01
%\end{thebibliography}

% \nocite{*}
	\bibliographystyle{IEEEtran}
 	\bibliography{ref} %.bib文件名字
	
 	 %.bst模板

% \bibliographystyle{unsrt}




% \bibliography{ref}




\end{document}
