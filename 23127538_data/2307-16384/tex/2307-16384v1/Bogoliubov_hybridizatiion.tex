
\documentclass[preprint,aps]{revtex4}
\usepackage{graphicx}
%\usepackage{lineno}
%\DeclareUnicodeCharacter

\begin{document}

%\linenumbers
\title{ARPES Detection of Superconducting Gap Sign in Unconventional Superconductors}

\author{Qiang Gao$^{1,\sharp}$, Jin Mo Bok$^{2,\sharp}$, Ping Ai$^{1,\sharp}$, Jing Liu$^{1,3,\sharp}$, Hongtao Yan$^{1}$, Xiangyu Luo$^{1,4}$, Yongqing Cai$^{1}$, Cong Li$^{1}$, Yang Wang$^{1}$, Chaohui Yin$^{1,4}$, Hao Chen$^{1,4}$, Genda Gu$^{5}$, Fengfeng Zhang$^{6}$, Feng Yang$^{6}$, Shenjin Zhang$^{6}$, Qinjun Peng$^{6}$, Zhihai Zhu$^{1,4,7}$, Guodong Liu$^{1,4,7}$, Zuyan Xu$^{6}$, Tao Xiang$^{1,3,4}$, Lin Zhao$^{1,4,7,*}$, Han-Yong Choi$^{8,*}$ and X. J. Zhou$^{1,4,7,*}$}
\affiliation{
\\$^{1}$Beijing National Laboratory for Condensed Matter Physics, Institute of Physics, Chinese Academy of Sciences, Beijing 100190, China.
\\$^{2}$Department of Physics, Pohang University of Science and Technology (POSTECH), Pohang 37673, Korea
\\$^{3}$Beijing Academy of Quantum Information Sciences, Beijing 100193, China
\\$^{4}$School of Physical Sciences, University of Chinese Academy of Sciences, Beijing 100049, China.
\\$^{5}$Condensed Matter Physics and Materials Science Department, Brookhaven National Laboratory, Upton, New York, 11973, USA
\\$^{6}$Technical Institute of Physics and Chemistry, Chinese Academy of Sciences, Beijing 100190, China.
\\$^{7}$Songshan Lake Materials Laboratory, Dongguan, Guangdong 523808, China.
\\$^{8}$Department of Physics and Institute for Basic Science Research, SungKyunKwan University, Suwon 440-746, Korea
\\$^{\sharp}$These people contributed equally to the present work.
\\$^{*}$Corresponding authors: LZhao@iphy.ac.cn, hychoi@skku.edu, XJZhou@iphy.ac.cn
}

\date{\today}

\begin{abstract}

\textbf{Superconductivity is realized by opening a gap in the superconducting state. The gap symmetry is crucial in understanding the underlying superconductivity mechanism. The magnitude and the phase are essential in fully characterizing the superconducting gap. Angle-resolved photoemission spectroscopy (ARPES) has played a key role in determining the gap symmetry in unconventional superconductors. However,  it has been considered so far that ARPES can only measure the magnitude of the superconducting gap but not its phase; the phase has to be detected by other phase-sensitive techniques. Here we propose a new method to directly detect the superconducting gap sign by using ARPES. This method is successfully validated in a cuprate superconductor with a well-known $d$-wave gap symmetry. When two bands are nearby in momentum space and have a strong interband interaction, the resulted electronic structures in the superconducting state are sensitive to the relative gap sign between the two bands which can be captured by ARPES measurements. Our present work provides a new way to detect the gap sign and can be applied to various superconductors, particularly those with multiple orbitals like the iron-based superconductors. It also makes ARPES more powerful to determine both the gap magnitude and the phase that are significant in understanding the superconductivity mechanism of unconventional superconductors.
}




\end{abstract}



\maketitle

\newpage


The superconducting gap is the most basic and important physical quantity of superconductors which is characterized by the magnitude and sign. Its determination is essential for understanding the mechanism of superconductivity. While the conventional superconductors exhibit an \textit{s}-wave gap symmetry that has the same sign along the entire Fermi surface, in unconventional superconductors, the superconducting gap may have different signs on different parts of the Fermi surface\cite{DJScalapino2012}. High temperature cuprate superconductors have been extensively studied for more than thirty years due to its unusually high critical temperature (\textit{T}$\rm_{c}$), anomalous normal state, and challenging mechanism of high temperature superconductivity\cite{JRKirtley2000CCTsuei,BStatt1999TTimusk,ZXShen2003ADamascelli,XGWen2006PALee,JZaanen2015BKeimer}.  
One of the most significant achievements is the establishment of the \textit{d}-wave pairing symmetry that pinpoints the unconventional superconductivity mechanism in the cuprate superconductors\cite{JRKirtley2000CCTsuei}. For the magnitude of the superconducting gap, angle-resolved photoemission spectroscopy (ARPES) played an important role in directly determining the anisotropic gap size in the momentum space that is consistent with the \textit{d}-wave symmetry\cite{CHPark1993ZXShen,KKadowaki1996HDing,ZXShen2003ADamascelli}. However, the ARPES measurements do not provide the sign information that is necessary in pinning down the pairing symmetry. For the phase information of the \textit{d}-wave gap, it was obtained later on by the phase-sensitive experiments based on Josephson tunnelling and flux quantization\cite{AJLeggett1993DAWollman,MBKetchen1994CCTsuei,JRKirtley2000CCTsuei} that were specially designed for the cuprate superconductors which have relatively simple Fermi surface and \textit{d}-wave superconducting gap (Fig. 1a). These methods have not been successfully applied to the other superconductors like the iron-based superconductors which possess multiple Fermi surface sheets and possible unconventional pairing with gap sign changes (Fig. 1b)\cite{DJScalapino2009SGraser, IIMazin2011PJHirschfeld, KKuroki2015HHosono}. So far, experimental extraction of the sign information in the gap function has proven to be significant but challenging in unconventional superconductors\cite{DJScalapino2012}. It has been attempted in the scanning tunnelling microscopy (STM) measurements utilizing quasiparticle interference\cite{HTakagi2009THanaguri,HTakagi2010THanaguri,JCDavis2013MPAllan,JCSDavis2017POSprau,HHWen2019QQGu}.
 Although ARPES is a powerful tool to directly measure the superconducting gap magnitude, it has long been believed that it can not probe the sign of the superconducting gap and therefore there has been no ARPES report in the sign detection of the superconducting gap.


In this paper, we develop a new method to detect superconducting gap sign by ARPES. It is motivated by our observation of an unusual Bogoliubov band hybridization in the cuprate superconductor Bi$_2$Sr$_2$CaCu$_2$O$_{8+\delta}$ (Bi2212)\cite{XJZhou2020QGao}. The gap sign manifests itself in the resulted electronic structures in the superconducting state including the Fermi momentum shift, the strong Bogoliubov band hybridization and the abnormal superconducting gap behaviors. The proposed method is well tested in the ARPES measurements of Bi2212 with a \textit{d}-wave gap symmetry. The present work provides a new way to detect the superconducting gap sign that is significant to understand the mechanism of unconventional superconductors. \\



\noindent\textbf{Proposed method}
%figure1

To detect the relative sign of the superconducting gap between two bands, we propose a method based on the Bogoliubov band hybridization. For a system with two bands ($\alpha$ and $\beta$) which are close in momentum space (Fig. 1c-f), its superconducting state can be described by a phenomenological Hamiltonian
\begin{eqnarray}
\hat{H}(\mathbf{k}) =
\left(
\begin{array}{cccc}
\varepsilon_{i\alpha} & V & \Delta_{i\alpha} & 0 \\
V & \varepsilon_{i\beta} & 0 & \Delta_{i\beta} \\
\Delta_{i\alpha} & 0 & -\varepsilon_{i\alpha} & -V \\
0 & \Delta_{i\beta} & -V & -\varepsilon_{i\beta} \\
\end{array}
\right)
\end{eqnarray}

{\noindent}where $\varepsilon_{i\alpha}$ and $\varepsilon_{i\beta}$ represent the initial $\alpha$ and $\beta$ bare bands, $V$ is the coupling strength between the two bands, and $\Delta_{i\alpha}$ and $\Delta_{i\beta}$ are the initial superconducting gap of the $\alpha$ and $\beta$ bands\cite{TKondo2017SKunisada,AFujimori2021SIdeta}. Such a Hamiltonian can also describe the normal state when the initial superconducting gaps are taken as zeros.
   
Figure 1c-i show the simulated band structures of the two band system in the normal state and in the superconducting state (see Supplementary Materials for the details of the simulation). To be typical and for simplicity, we started with the two initial bare bands which are degenerate (Fig. 1c). When there is an interband coupling ($V$=10\,meV) between the two bands, the band structure in the normal state exhibits a band splitting (Fig. 1d). In the superconducting state, the relative sign of the initial superconducting gap between the two bands dramatically affects the resulted band structures. When the initial superconducting gap of the two bands is taken as the same both in the magnitude and in its sign (Fig. 1e), superconducting gap opens in the normal way at the two Fermi momenta and there is no noticeable hybridization between the Bogoliubov backbending band of $\alpha$ and the $\beta$ band at the crossing point k$\rm_H$. In contrast, when the initial superconducting gap of the two bands takes the opposite sign, the resulted band structures (Fig. 1f) become totally different from those in Fig. 1e in terms of the unusual change of the Fermi momentum (Fig. 1g), the superconducting gap (Fig. 1h) and the opening of a hybridization gap at k$\rm_H$ (Fig. 1i). While the two Fermi momenta keep fixed in the superconducting state when the two initial superconducting gaps have the same sign, they can be dramatically shifted in the case of the opposite gap sign (Fig. 1g). Depending on the relative magnitude of the initial superconducting gap ($\Delta_i$) and the interband coupling ($V$), the two Fermi momenta in the normal state may even evolve into the same one in the superconducting state. In the case of the superconducting gap, although it keeps the same with the initial supercondcuting gap in the superconducting state when the two initial superconducting gaps have the same sign, it is completely altered when the two initial superconducting gaps take the opposite sign (Fig. 1h). The superconducting gap may even become zero in the superconducting state when the initial superconducting gap ($\Delta_i$) is relatively smaller than the interband coupling ($V$). Whereas the band hybridization gap is nearly zero when the two initial superconducting gaps have the same sign, it can become significant in the case of the opposite gap sign in the superconducting state (Fig. 1i). These fundamental differences in the resulted band structures form the basis of our proposed method to probe the relative gap sign between the two bands through the ARPES measurements. 

We just discussed one typical and extreme case where the initial two bare bands are degenerate. We also carried out simulations on the cases that the initial two bare bands are different, in particular, how the Bogoliubov band hybridization evolves with the seperation of the two bare bands (Fig. S1 in Supplementary Materials). It is found that the dramatic difference of the electronic structures in the superconducting state between the same-gap-sign and the opposite-gap-sign cases still persists. Even if the initial two bare bands separate, when the initial two superconducting gaps have the opposite sign, the unusual behaviors like the Fermi momentum shift, the pronounced Bogoliubov band hybridization and the abnormal superconducting gap remain present. This indicates that our proposed method is more general which can be used for both cases that the initial two bare bands are degenerate and separate.\\



  
\noindent\textbf{Test of the method: Manifestations of gap sign in Bogoliubov band hybridization}


In order to test the above proposed method, it is necessary to find a superconductor that is unconventional and its superconducting gap symmetry is well established with a sign change. Such superconductors are rare and cuprate superconductors are essentially the only known case that can satisfy the stringent requirements. It is well known that the cuprate superconductors have a $d$-wave superconducting gap that changes sign on different parts of the Fermi surface (Fig. 1a). Furthermore, to test the method in the cuprate superconductors, it is also necessary to find two bands that are close in momentum space and have the opposite gap sign. 



In the process of studying the origin of the superstructure bands in Bi2212\citep{XJZhou2020QGao}, we came across an ideal case that can test our proposed method. In Bi2212, because of the presence of the incommensurate structural modulations, in addition to the main bonding and antibonding bands, superstructure bands are formed by shifting the main bands with the superstructure wavevector, ${\pm}$\textbf{Q}, as shown in Fig. 2a\cite{KKadowaki1995JOsterwalder,XJZhou2019JLiu,XJZhou2020QGao}. 
While the main Fermi surface and the superstructure Fermi surface are well separated in the first quadrant, they cross each other in the second quadrant (Fig. 2a).  Fig. 2b shows a constant energy contour near the Fermi level covering the band crossing area in the second quadrant. Here mainly two Fermi surface sheets are observed due to the photoemission matrix element effects: the main antibonding Fermi surface sheet(AB) and the superstructure antibonding Fermi surface sheet(AB$\_$SS). It has been found that the main bands and the superstructure bands exhibit a selective band hybridization, \textit{i.e.}, the initial main antibonding band (AB, red line in Fig. 2b) hybridizes with the initial superstructure bonding band (BB$\_$SS, dashed blue line in Fig. 2b)\citep{XJZhou2020QGao,TValla2019}, as shown in Fig. 2b.  The main AB Fermi surface is then broken into two branches (Branch1 and Branch2) at the crossing point MS induced by the hybridization. Therefore, we have found a rare but ideal case to test our method in Bi2212: (1), There are two bands, the main antibonding band (AB) and the superstructure bonding band (BB$\_$SS), that are close-by in momentum space; (2), These two bands exhibit strong interband coupling; (3), The superconducting gap sign of the two bands is opposite; (4), The superconducting gap magnitude of the two bands is similar.

Figure 2 shows the band structures of Bi2212 measured in the normal state (Fig. 2c) and the superconducting state (Fig. 2d) along a typical momentum cut near the crossing region of the AB and BB$\_$SS Fermi surface. In the normal state, two bands are mainly observed, labelled as BR1 and BR2 in Fig. 2c, that correspond to Branch1 and Branch2 Fermi surface in Fig. 2b. In the superconducting state, the observed band structure (Fig. 2d) is strikingly different from generally expected picture that superconducting gaps open at the two Fermi momenta. It is unusual in several aspects. First, there is an obvious Fermi momentum shift in the superconducting state. The Fermi momentum seperation between the two bands shrinks from 0.023$\pi/a$ in the normal state to 0.018$\pi/a$ in the superconducting state. Second, the gap opening at the two Fermi momenta is quite unusual. The particle-hole symmetry is not conserved at the Fermi momentum as seen from the BR2 band in Fig. 2d and the photoemission spectrum (energy distribution curve, EDC) at the BR2 Fermi momentum (purple line in Fig. 2i). Third, below the Fermi level, the BR2 band breaks into two parts with a strong spectral weight suppression around the binding energy of 14\,meV, as marked by red arrows in Fig. 2d. Such a band breaking and the dramatic spectral weight suppression of the BR2 band can also be seen from the corresponding EDCs in the superconducting state (Fig. 2i).

The unusual behaviors observed in the superconducting state (Fig. 2d) can be understood in terms of the two band model we proposed in Eq. 1 by taking proper bare bands, initial superconducting gaps and the coupling strength between the two bands. Fig. 2e-h shows the simulated band structure in the normal state and superconducting state. In the simulation process, the relative sign of the initial superconducting gap  between the two bands plays a decisive role in dictating the band structures in the superconducting state. Fig. 2g shows the simulated band structure in the superconducting state by considering the opposite sign of the superconducting gaps on the two bands. The corresponding EDCs are shown in Fig. 2j. The simulated results (Fig. 2g and 2j) are highly consistent, even on the quantitative level, with the measured band structure (Fig. 2d) and EDCs (Fig. 2i). All the unusual behaviors observed in the superconducting state are well captured by the simulations. In contrast, if the same sign of the superconducting gap is taken for the two bands, the simulated band structure (Fig. 2h) deviates far from the measured result (Fig. 2d). The observation of the unusual band structures in the superconducting state and their quantitative understanding based on the two band model indicate unambiguously that the superconducting gaps of the two bands have the opposite sign. It demonstrates the feasibility of our proposed method in detecting the relative sign of the superconducting gaps on two bands. This is the first time that the superconducting gap sign is detected from ARPES measurements.




The relative gap sign also manifests itself in the Fermi surface topology, momentum dependence of the band structure, and the associated momentum dependence of the Bogoliubov band hybridization. Fig. 3a shows the detailed Fermi surface mapping in the superconducting state around the crossing point MS of the initial main AB and the superstructure BB$\_$SS Fermi surface. The hybridization between the AB and BB$\_$SS bands results in breaking the main AB Fermi surface into two branches (Branch1 and Branch2) with the spectral weight enhanced near the crossing point. Fig. 3b shows band structures measured along different cuts in the covered momentum space in Fig. 3a (more complete momentum-dependent band structures are shown in Fig. S2 in Supplementary Materials). Within a narrow momentum space, the observed band structures exhibit a dramatic and systematic momentum dependence. Moreover, like the band structure for $\theta$=32 that is analyzed in detail in Fig. 2, these observed bands are also unusual in the gap opening and Bogoliubov band hybridization. The BR1 band shows a strong Bogoliubov hybridization with the BR2 band near the crossing point MS (Cut1 and Cut2 in Fig. 3b) and the hybridization gets weaker with the momentum cuts moving away from  MS (Cut3 to Cut5 in Fig. 3b). The quantitative evolution of the Bogoliubov hybridization gap with momentum, which is extracted in Fig. S3 in Supplementary Materials, is plotted in Fig. 3h. 

The measured Fermi surface mapping in Fig. 3a and the unusual momentum-dependent band structure evolution (Fig. 3b and 3h) can be understood by the two band model (Eq. 1) only when the relative sign of the superconducting gaps on the two bands are taken opposite. We note that, in the particular case of Bi2212, the initial two bare bands and the initial two superconducting gap sizes are known beforehand which can be determined from the measurements on the two main bands (AB and BB) in the first quadrant (see Fig. S4, Fig. S5 and the tight binding fitting in Supplementary Materials). In principle, the coupling strength $V$ between the main AB band and the superstructure BB$\_$SS band can also be determined from the band structure measurements in the normal state\cite{XJZhou2020QGao}. Here we take it as a constant in the small covered momentum space of the Fermi surface crossing region. Under the condition that all the parameters in Eq. 1 are known, we globally simulated the Fermi surface and momentum-dependent band structures by considering the opposite gap sign (Fig. 3d and 3e) and the same gap sign (Fig. 3f and 3g) for the two bands. When the two gaps take the opposite sign, the simulated Fermi surface mapping (Fig. 3d) well reproduces the measured Fermi surface in Fig. 3a. The momentum-dependent band structures and the associated Bogoliubov band hybridization (Fig. 3b) are well captured in the simulated results (Fig. 3e). In particular, the measured momentum-dependent Bogoliubov hybridization gap (black circles in Fig. 3h) shows a quantitative agreement with the simulations (red line in Fig. 3h). In contrast, if the same gap sign is taken for the two bands, the simulated Fermi surface (Fig. 3f), the overall momentum-dependent band structures (Fig. 3g) and the Bogoliubov hybridization gap (blue line in Fig. 3h) deviate significantly from the measured results.  These results lend further decisive evidence that the relative sign of the superconducting gap on the two bands is opposite in the covered momentum space.\\




\noindent\textbf{Test of the method: Manifestations of gap sign in the abnormal superconducting gap behaviors}
%figure4

As shown from the simulations of the two band model in Fig. 1 and Fig. S1, one of the main signatures of the relative gap sign is the unusual superconducting gap behaviors. Such an anomaly of the superconducting gap is observed in Bi2212 in the crossing area of two bands with opposite gap sign. Fig. 4b highlights the measured band structure near the Fermi level along the momentum cut of ${\theta}$=32 in the superconducting state. The corresponding original EDCs and the EDCs after dividing the Fermi distribution function at the two Fermi momenta (BR1$\_$k$\rm_F$ and BR2$\_$k$\rm_F$) are shown in Fig. 4(c,d) and Fig. 4(e,f), respectively. The superconducting gaps of the two bands are unusual in several aspects. First, the BR1 band crosses the Fermi level with a gap size that is nearly zero. Second, the particle-hole symmetry for the BR2 band is apparently broken where the spectral weight at the Fermi momentum is not symmetrical with respect to the Fermi level. Third, compared with the gap size at the equivalent momentum position in the first quadrant, the gap size of the BR1 and BR2 bands after the hybridization in the second quadrant is much reduced from the initial 5.8\,meV and 11.6\,meV to 0\,meV and 5\,meV, respectively. As shown in Fig. 2g, these unusual gap behaviors can be well reproduced when the relative gap sign of the two bands is opposite. 

Figure 4g and 4h show EDCs measured along the BR1 and BR2 Fermi surface in the superconducting state. The momentum dependent superconducting gap along the two branches of Fermi surface is plotted Fig. 4i. Near the crossing area, the measured superconducting gap strongly deviates from the standard $d$-wave form with the gap size much reduced and even new accidental nodes can be produced. The unusual momentum dependence of the superconducting gap can be well understood by considering the two bands hybridization and the opposite gap sign on the two bands. As shown in Fig. 4i, the superconducting gap obtained from the same global band structure simulations (Fig. 3e) is quantitatively consistent with the measured results. These further demonstrates that the relative sign of the superconducting gap on the two bands is opposite in the covered momentum space.\\




\noindent\textbf{Discussion}

Our present results show that the hybridization between two bands with opposite superconducting gap sign can induce unusual behaviors. First, the Fermi momenta can be shifted in the superconducting state relative to those in the normal state (Fig. 2c and 2d). Second, the density of states at the Fermi momenta may become unsymmetrical with respect to the Fermi level which indicates the particle-hole symmetry is no longer conserved in the superconducting state (Fig. 2i). Third, it can produce gap nodes even though the initial gap of the two bands is non-zero (Fig. 4e). Our results indicate that, because of the superstructures and band hybridization in Bi2212, additional accidental gap nodes or even segments of gapless Fermi surface can be produced besides the usual nodes along the nodal directions in the first Brillouin zone. These should be considered in detecting the gap symmetry of cuprate superconductors\cite{QKXue2019YYZhu, PKim2021SYFZhao, HHWen2019QQGu}. Our results also point to the possibility that some unusual superconductors may be designed and produced. As an extreme case, we consider a material that consists of two Fermi surface sheets originated from the initial two degenerate bands due to the interband interaction. As shown in Fig. S6, if the superconducting gap sign is opposite on the two bands, it is possible that the superconducting gap becomes zero on both Fermi surface sheets even though the initial gap is non-zero. This would produce a gapless superconductor with zero superconducting gap on the entire Fermi surface. It is interesting to explore whether such unusual superconductors can be realized both theorically and experimentally.




%Advantage

%both magnitude and phase (ARPES) 

In unconventional superconductors, the determination of the gap symmetry, particularly the phase, is both significant and challenging \cite{DJScalapino2012,IIMazin2011PJHirschfeld}. In the cuprate superconductors, the phase information of the \textit{d}-wave gap is obtained by the phase-sensitive experiments based on Josephson tunnelling and flux quantization\cite{AJLeggett1993DAWollman,MBKetchen1994CCTsuei,JRKirtley2000CCTsuei}. However, these methods were specifically designed for the cuprate superconductors which have relatively simple Fermi surface and \textit{d}-wave superconducting gap (Fig. 1a). For the iron-based superconductors that possess multiple Fermi surface sheet and possible unconventional pairing with gap sign changes (Fig. 1b)\cite{DJScalapino2009SGraser, IIMazin2011PJHirschfeld, KKuroki2015HHosono}, the conventional phase-sensitive methods based on Josephson tunneling and flux quantization have not been successful in determining the phase of the superconducting gap. The STM measurements have been tried to extract the phase information of the superconducting gap by utilizing quasiparticle interference but this method is limited to only a few materials \cite{HTakagi2009THanaguri,HTakagi2010THanaguri,JCDavis2013MPAllan,JCSDavis2017POSprau,HHWen2019QQGu}.
In the present work, we proposed a new method to detect the phase of the superconducting gap. It provides direct information of the relative gap sign on the two bands that are studied, regardless of the complexity from the multiple orbital systems. It is a general method that can be applied in various superconductors, including cuprate superconductors, iron-based superconductors, heavy Fermion superconductors and so on. Since ARPES is good at directly determining the anisotropic gap size in the momentum space, with its new capability to provide the phase information, it will become a more powerful tool to determine the pairing symmetry of superconductors.



%Disadvantage
As shown in Fig. 1 and S1, the applicability of our method depends on the significant difference of the band structures in the superconducting state when the initial superconducting gaps of the two bands have the same or opposite signs. In particular, when the initial superconducting gaps have the opposite sign, some distinct and unusual signatures may appear such as the Fermi momentum shift, the strong Bogoliubov band hybridization and the anomalous gap behaviors. In addition to the relative gap sign, the band structure in the superconducting state is determined by the relative position of the two bare bands, the interband coupling strength and the initial superconducting gap size. The most ideal case to use the method is that the two bare bands are degenerate or adjacent, the two initial gap sizes are the same or very close and the interband coupling is relatively strong. The momentum cut of $\theta$=33 is close to such an ideal case (the second panel in Fig. 3b). In less ideal cases when the two bare bands are separated or the two initial gaps are different, the signatures of the relative gap sign become less prominent and can not be identified qualitatively. Careful quantitative analyses are needed to distinguish the relative gap sign. The momentum cuts of $\theta$=34 and 32 belong to such less ideal cases (the first and third panels in Fig. 3b). In the cases when the two bare bands are far apart, or the two initial superconducting gaps are significantly different or the interband coupling strength is rather small, the difference induced by the relative gap sign becomes small in the resulted band structures and the method becomes ineffective. The utilization of the method requires an overall understanding of the band structure and superconducting gap structure of the involved two bands.

Our proposed method works well when the two bands are close in the momentum space with a strong interaction between them. It can be used to test whether the two hole-like pockets around the zone center in the iron-based superconductors (Fig. 1b) have the same or opposite gap sign \cite{AVChubukov2012SMaiti, AVChubukov2017OVafek}. It can also be used to check whether the two electron-like pockets around M (Fig. 1b) have the same or opposite gap sign, particularly in the iron-based superconductors with only electron pockets \cite{DJScalapino2011TAMaier, DHLee2011FWang, AVChubukov2012MKhodas, DLFeng2015QFan,HHWen2016ZYDu}. When the two bands are far away in the momentum space, it is possible to use the method by taking advantage of the band folding from the superstructure or surface reconstruction. In the present work, although the gap sign change occurs on different parts of the Fermi surface, it is detected from the interaction of the main band and the folded band from the superstructures in Bi2212. In the iron-based superconductors like (Ba,K)Fe$_{2}$As$_{2}$, in order to determine the relative gap sign between $\Gamma$ and M, i.e., whether the gap symmetry is s$_{\pm}$ or s$_{++}$ \cite{MHDu2008IIMazin,HAoki2008KKuroki,QMSi2008EAbrahams,JPHu2008PRL,SOnari2010HKontani,YOno2010YYanagi}, the $\sqrt{2}\times\sqrt{2}$ surface reconstruction can be used to fold the bands between the $\Gamma$ and M points\cite{XJZhou2021YQCai,XJZhou2022DSWu}. In the case that no natural band folding mechanisms are available, it is possible to engineer the band structures such as those realized in twisted graphene\cite{AHMacdonald2011,YLChen2017HPeng,JHPablo2018YCao} or substrate-controlled film growth\cite{KMShen2015JWHarter,YKuk2018SKim}.

In summary, we have proposed a new method to detect superconducting gap sign by ARPES. The method is well tested in the ARPES measurements of Bi2212 with a known \textit{d}-wave gap symmetry. The gap sign manifests itself in the resulted electronic structures in the superconducting state including the Fermi momentum shift, the Bogoliubov band hybridization and the abnormal superconducting gap behaviors. Our present work provides a new powerful way to detect the superconducting gap sign that is significant to understand the gap symmetry and the superconductivity mechanism of unconventional superconductors.\\



\noindent{\bf Methods}

{\noindent}ARPES measurements were carried out on a vacuum ultraviolet (VUV) laser-based ARPES system equipped with an electron energy analyzer (Scienta Omicron DA30L)\cite{XJZhou2008GDLiu,XJZhou2018,XJZhou2020QGao}. The photon energy of the VUV laser is 6.994\,eV. The overall energy resolution was set at 1.0\,meV. The angular resolution was $\sim$0.3$^\circ $, corresponding to a momentum resolution of 0.004\,$\rm\mathring{A}^{-1}$ with the photon energy of 6.994\,eV. The Fermi level is referenced by measuring on the Fermi edge of a clean polycrystalline gold that is electrically connected to the sample. High quality overdoped Bi$_2$Sr$_2$CaCu$_2$O$_{8+\delta}$ (\textit{T}$\rm_{c}$=78\,K) single crystals were grown by the floating zone method and then annealed in oxygen atmosphere\cite{XJZhou2016YXZhang}. The \textit{T}$\rm_{c}$ was measured using a Quantum Design SQUID magnetometer. The sample was cleaved $in$ $situ$ and measured in vacuum with a base pressure better than $3\times10^{-11}$\,mbar.\\





\noindent{\bf Acknowledgements}

{\noindent}This work is supported by the National Key Research and Development Program of China (No. 2021YFA1401800, 2017YFA0302900, 2018YFA0305600, 2018YFA0704200, 2019YFA0308000, 2022YFA1604200 and 2022YFA1403900), the National Natural Science Foundation of China (Grant No. 11888101, 11922414, 11974404 and 12074411), the Strategic Priority Research Program (B) of the Chinese Academy of Sciences (XDB25000000 and XDB33000000), the Innovation Program for Quantum Science and Technology (Grant No. 2021ZD0301800), the Youth Innovation Promotion Association of CAS (Grant No. Y2021006) and Synergetic Extreme Condition User Facility (SECUF). J.M.B was supported by National Research Foundation (NRF) of Korea through Grants  No. NRF-2022R1C1C2008671. The work at Brookhaven was supported by the Office of Basic Energy Sciences, U.S. Department of Energy (DOE) under Contract No de-sc0012704.\\

\noindent{\bf Author Contributions}

{\noindent}X.J.Z., T.X., L.Z and Q.G. proposed and designed the research. G.D.G., Q.G. and P.A. prepared single crystal. Q.G. carried out the experiment with J.L. and P.A.. Y.Q.C., C.L., Y.W., Q.G., H.T.Y., X.Y.L., C.H.Y., C.H., Z.H.Z, L.Z., G.D.L., F.F.Z., F.Y., S.J.Z., Q.J.P., Z.Y.X. and X.J.Z. contributed to the development and maintenance of Laser ARPES system. Q.G., L.Z. and X.J.Z. analyzed the data. J.M.B., Q.G., and H.Y.C. contributed to model calculations. X.J.Z., L.Z. and Q.G. wrote the paper. All authors discussed the results and commented on the manuscript.








\newpage

\bibliographystyle{unsrt}

\begin{thebibliography}{10}

\bibitem{DJScalapino2012}
D. J. Scalapino.
\newblock {A common thread: The pairing interaction for unconventional
  superconductors}.
\newblock {\em Rev. Mod. Phys.} 84, 1383, 2012.

\bibitem{JRKirtley2000CCTsuei}
C. C. Tsuei and J. R. Kirtley.
\newblock {Pairing symmetry in cuprate superconductors}.
\newblock {\em Rev. Mod. Phys.} 72, 969, 2000.

\bibitem{BStatt1999TTimusk}
T. Timusk and B. Statt.
\newblock {The pseudogap in high-temperature superconductors: an experimental
  survey}.
\newblock {\em Rep. Prog. Phys.} 62, 61, 1999.

\bibitem{ZXShen2003ADamascelli}
A. Damascelli, Z. Hussain, and Z. X. Shen.
\newblock {Angle-resolved photoemission studies of the cuprate
  superconductors}.
\newblock {\em Rev. Mod. Phys.} 75, 473, 2003.

\bibitem{XGWen2006PALee}
P. A. Lee, N. Nagaosa, and X. G. Wen.
\newblock {Doping a Mott insulator: Physics of high-temperature
  superconductivity}.
\newblock {\em Rev. Mod. Phys.} 78, 17, 2006.

\bibitem{JZaanen2015BKeimer}
B. Keimer, S. A. Kivelson, M. R. Norman, S. Uchida, and J. Zaanen.
\newblock {From quantum matter to high-temperature superconductivity in copper
  oxides}.
\newblock {\em Nature} 518, 179, 2015.

\bibitem{CHPark1993ZXShen}
Z. X. Shen, D. S. Dessau, B. O. Wells, D. M. King, W. E. Spicer, A. J. Arko,
  D. Marshall, L. W. Lombardo, A. Kapitulnik, P. Dickinson, S. Doniach,
  J. DiCarlo, T. Loeser, and C. H. Park.
\newblock {Anomalously large gap anisotropy in the a-b plane of
  Bi$_2$Sr$_2$CaCu$_2$O$_{8+\delta}$}.
\newblock {\em Phys. Rev. Lett.} 70, 1553, 1993.

\bibitem{KKadowaki1996HDing}
H. Ding, M. R. Norman, J. C. Campuzano, M. Randeria, A. F. Bellman, T. Yokoya,
  T. Takahashi, T. Mochiku, and K. Kadowaki.
\newblock {Angle-resolved photoemission spectroscopy study of the
  superconducting gap anisotropy in
  ${\mathrm{Bi}}_{2}{\mathrm{Sr}}_{2}\mathrm{Ca}{\mathrm{Cu}}_{2}{\mathrm{O}}_{8+x}$}.
\newblock {\em Phys. Rev. B} 54, R9678, 1996.

\bibitem{AJLeggett1993DAWollman}
D. A. Wollman, D. J. Van Harlingen, W. C. Lee, D. M. Ginsberg, and A. J.
  Leggett.
\newblock {Experimental determination of the superconducting pairing state in
  YBCO from the phase coherence of YBCO-Pb dc SQUIDs}.
\newblock {\em Phys. Rev. Lett.} 71, 2134, 1993.

\bibitem{MBKetchen1994CCTsuei}
C. C. Tsuei, J. R. Kirtley, C. C. Chi, Lock See Yu-Jahnes, A. Gupta, T. Shaw,
  J. Z. Sun, and M. B. Ketchen.
\newblock {Pairing symmetry and flux quantization in a tricrystal
  superconducting ring of
  $\mathrm{Y}{\mathrm{Ba}}_{2}{\mathrm{Cu}}_{3}{\mathrm{O}}_{7\ensuremath{-}\ensuremath{\delta}}$}.
\newblock {\em Phys. Rev. Lett.} 73, 593, 1994.

\bibitem{DJScalapino2009SGraser}
S. Graser, T. A. Maier, P. J. Hirschfeld, and D. J. Scalapino.
\newblock {Near-degeneracy of several pairing channels in multiorbital models
  for the Fe pnictides}.
\newblock {\em New J. Phys.} 11, 025016, 2009.

\bibitem{IIMazin2011PJHirschfeld}
P. J. Hirschfeld, M. M. Korshunov, and I. I. Mazin.
\newblock {Gap symmetry and structure of Fe-based superconductors}.
\newblock {\em Rep. Prog. Phys.} 74, 124508, 2011.

\bibitem{KKuroki2015HHosono}
H. Hosono and K. Kuroki.
\newblock {Iron-based superconductors: Current status of materials and pairing
  mechanism}.
\newblock {\em Phys. C: Supercond} 514, 399, 2015.

\bibitem{HTakagi2009THanaguri}
T. Hanaguri, Y. Kohsaka, M. Ono, M. Maltseva, P. Coleman, I. Yamada, M. Azuma,
  M. Takano, K. Ohishi, and H. Takagi.
\newblock {Coherence factors in a high-\textit{T}$\rm_{c}$ cuprate probed by quasi-particle
  scattering off vortices}.
\newblock {\em Science} 323, 923, 2009.

\bibitem{HTakagi2010THanaguri}
T. Hanaguri, S. Niitaka, K. Kuroki, and H. Takagi.
\newblock {Unconventional s-wave superconductivity in Fe(Se,Te)}.
\newblock {\em Science} 328, 474, 2010.

\bibitem{JCDavis2013MPAllan}
M. P. Allan, F. Massee, D. K. Morr, J. Van Dyke, A. W. Rost, A. P. Mackenzie,
  C. Petrovic, and J. C. Davis.
\newblock {Imaging cooper pairing of heavy fermions in CeCoIn$_{5}$}.
\newblock {\em Nat. Phys.} 9, 468, 2013.

\bibitem{JCSDavis2017POSprau}
P. O. Sprau, A. Kostin, A. Kreisel, A. E. Böhmer, V. Taufour, P. C. Canfield,
  S. Mukherjee, P. J. Hirschfeld, B. M. Andersen, and J. C. Séamus Davis.
\newblock {Discovery of orbital-selective Cooper pairing in FeSe}.
\newblock {\em Science} 357, 75, 2017.

\bibitem{HHWen2019QQGu}
Q. Gu, S. Wan, Q. Tang, Z. Du, H. Yang, Q. H. Wang, R. Zhong, J. Wen, G. D. Gu,
  and H. H. Wen.
\newblock {Directly visualizing the sign change of d-wave superconducting gap
  in ${\mathrm{Bi}}_{2}{\mathrm{Sr}}_{2}{\mathrm{CaCu}}_{2}{\mathrm{O}}_{8+\ensuremath{\delta}}$ by phase-referenced quasiparticle interference}.
\newblock {\em Nat. Commun.} 10, 1603, 2019.

\bibitem{XJZhou2020QGao}
Q. Gao, H. T. Yan, J. Liu, P. Ai, Y. Q. Cai, C. Li, X. Y. Luo,
  C. Hu, C. Y. Song, J. W. Huang, H. T. Rong, Y. Huang, Q. Y. Wang, G. D. Liu, G. D. Gu, F. F. Zhang, F. Yang, S. J. Zhang, Q. J. Peng, Z. Y. Xu, L. Zhao, T. Xiang, and X. J. Zhou.
\newblock {Selective hybridization between the main band and the superstructure
  band in the
  ${\mathrm{Bi}}_{2}{\mathrm{Sr}}_{2}{\mathrm{CaCu}}_{2}{\mathrm{O}}_{8+\ensuremath{\delta}}$
  superconductor}.
\newblock {\em Phys. Rev. B} 101, 014513, 2020.

\bibitem{TKondo2017SKunisada}
S. Kunisada, S. Adachi, S. Sakai, N. Sasaki, M. Nakayama, S. Akebi, K. Kuroda,
  T. Sasagawa, T. Watanabe, S. Shin, and T. Kondo.
\newblock {Observation of bogoliubov band hybridization in the optimally doped
  trilayer Bi$_{2}$Sr$_{2}$Ca$_{2}$Cu$_{3}$O$_{10+\delta}$}.
\newblock {\em Phys. Rev. Lett.} 119, 217001, 2017.

\bibitem{AFujimori2021SIdeta}
S. Ideta, S. Johnston, T. Yoshida, K. Tanaka, M. Mori, H. Anzai, A. Ino, M. Arita, H. Namatame, M. Taniguchi, S. Ishida, K. Takashima, K. M Kojima, T. P Devereaux, S. Uchida, and A. Fujimori.
\newblock {Hybridization of bogoliubov quasiparticles between adjacent
  ${\mathrm{CuO}}_{2}$ layers in the triple-layer cuprate
  ${\mathrm{Bi}}_{2}{\mathrm{Sr}}_{2}{\mathrm{Ca}}_{2}{\mathrm{Cu}}_{3}{\mathrm{O}}_{10+\ensuremath{\delta}}$ studied by angle-resolved photoemission spectroscopy}.
\newblock {\em Phys. Rev. Lett.} 127, 217004, 2021.

\bibitem{KKadowaki1995JOsterwalder}
J. Osterwalder, P. Aebi, P. Schwaller, L. Schlapbach, M. Shimoda, T. Mochiku,
  and K. Kadowaki.
\newblock {Angle-resolved photoemission experiments
  on Bi$_2$Sr$_2$CaCu$_2$O$_{8+\delta}$(001)}.
\newblock {\em Appl. Phys. A} 60, 247, 1995.

\bibitem{XJZhou2019JLiu}
J. Liu, L. Zhao, Q. Gao, P. Ai, L. Zhang, T. Xie, J. W. Huang, Y. Ding, C. Hu, H. T. Yan, C. Y. Song, Y. Xu, C. Li, Y. Q. Cai, H. T. Rong, D. S. Wu, G. D. Liu, Q. Y. Wang, Y. Huang,
  F. F. Zhang, F. Yang, Q. J. Peng, S. L. Li, H. X. Yang,
  J. Qi Li, Z. Y. Xu, and X. J. Zhou.
\newblock {Evolution of incommensurate superstructure and electronic structure
  with Pb substitution in (Bi$_{2-x}$Pb$_ x$)Sr$_2$CaCu$_2$O$_{8+\delta}$
  superconductors}.
\newblock {\em Chin. Phys. B} 28, 7, 2019.

\bibitem{TValla2019}
T. Valla, I. Pletikosić, I. K. Drozdov, and G. D. Gu.
\newblock {Reconstruction of the
  ${\mathrm{Bi}}_{2}{\mathrm{Sr}}_{2}{\mathrm{CaCu}}_{2}{\mathrm{O}}_{8+\ensuremath{\delta}}$
  Fermi surface}.
\newblock {\em Phys. Rev. B} 100, 241112, 2019.

\bibitem{QKXue2019YYZhu}
Y. Zhu, M. Liao, Q. Zhang, H. Xie, F. Meng, Y. Liu, Z. Bai, S. Ji, J. Zhang, K. Jiang, R. Zhong, J. Schneeloch, G. Gu, L. Gu, X. Ma, D. Zhang, and Q. K. Xue.
\newblock {Presence of $s$-wave pairing in Josephson junctions made of twisted
  ultrathin
  ${\mathrm{Bi}}_{2}{\mathrm{Sr}}_{2}{\mathrm{CaCu}}_{2}{\mathrm{O}}_{8+x}$
  flakes}.
\newblock {\em Phys. Rev. X} 11, 031011, 2021.

\bibitem{PKim2021SYFZhao}
S. Y. Frank Zhao, N. Poccia, X. Cui, P. A. Volkov, H. Yoo,
  R. Engelke, Y. Ronen, R. Zhong, G. Gu, S. Plugge, T. Tummuru, M. Franz, J. H. Pixley, and P. Kim.
\newblock {Emergent interfacial superconductivity between twisted cuprate
  superconductors}.
\newblock {\em arXiv} 2108.13455, 2021.

\bibitem{AVChubukov2012SMaiti}
S. Maiti, M. M. Korshunov, and A. V. Chubukov.
\newblock {Gap symmetry in KFe${}_{2}$As${}_{2}$ and the $\mathrm{cos}$
  4$\ensuremath{\theta}$ gap component in LiFeAs}.
\newblock {\em Phys. Rev. B} 85, 014511, 2012.

\bibitem{AVChubukov2017OVafek}
O. Vafek and A. V. Chubukov.
\newblock {Hund interaction, spin-orbit coupling, and the mechanism of
  superconductivity in strongly hole-doped iron pnictides}.
\newblock {\em Phys. Rev. Lett.} 118, 087003, 2017.

\bibitem{DJScalapino2011TAMaier}
T. A. Maier, S. Graser, P. J. Hirschfeld, and D. J. Scalapino.
\newblock {$d$-wave pairing from spin fluctuations in the
  ${\mathrm{K}}_{x}$Fe${}_{2\ensuremath{-}y}$Se${}_{2}$ superconductors}.
\newblock {\em Phys. Rev. B} 83, 100515, 2011.

\bibitem{DHLee2011FWang}
Fa Wang, Fan Yang, Miao Gao, Zhong-Yi Lu, Tao Xiang, and Dung-Hai Lee.
\newblock {The electron pairing of K$_{x}$Fe$_{2-y}$Se$_{2}$}.
\newblock {\em Europhys. Lett.} 93, 57003, 2011.

\bibitem{AVChubukov2012MKhodas}
M. Khodas and A. V. Chubukov.
\newblock {Interpocket pairing and gap symmetry in Fe-based superconductors
  with only electron pockets}.
\newblock {\em Phys. Rev. Lett.} 108, 247003, 2012.

\bibitem{DLFeng2015QFan}
Q. Fan, W. H. Zhang, X. Liu, Y. J. Yan, M. Q. Ren, R. Peng, H. C. Xu, B. P.
  Xie, J. P. Hu, T. Zhang, and D. L. Feng.
\newblock {Plain s-wave superconductivity in single-layer FeSe on SrTiO$_{3} $ probed by scanning tunnelling microscopy}.
\newblock {\em Nat. Phys.} 11, 946, 2015.

\bibitem{HHWen2016ZYDu}
Z. Y. Du, X. Yang, H. Lin, D. L. Fang, G. Du, J. Xing, H. Yang, X. Y. Zhu, and H. H. Wen.
\newblock {Scrutinizing the double superconducting gaps and strong coupling
  pairing in (Li$_{1-x}$Fe$_{x}$)OHFeSe}.
\newblock {\em Nat. Commun.} 7, 10565, 2016.

\bibitem{MHDu2008IIMazin}
I. I. Mazin, D. J. Singh, M. D. Johannes, and M. H. Du.
\newblock {Unconventional superconductivity with a sign reversal in the order Parameter of ${\mathrm{LaFeAsO}}_{1\ensuremath{-}x}{\mathrm{F}}_{x}$}.
\newblock {\em Phys. Rev. Lett.} 101, 057003, 2008.

\bibitem{HAoki2008KKuroki}
K. Kuroki, S. Onari, R. Arita, H. Usui, Y. Tanaka, H. Kontani, and H. Aoki.
\newblock {Unconventional pairing originating from the disconnected Fermi
  surfaces of superconducting
  ${\mathrm{LaFeAsO}}_{1\ensuremath{-}x}{\mathrm{F}}_{x}$}.
\newblock {\em Phys. Rev. Lett.} 101, 087004, 2008.

\bibitem{QMSi2008EAbrahams}
Q. M. Si and E. Abrahams.
\newblock {Strong correlations and magnetic frustration in the high ${T}_{c}$
  iron pnictides}.
\newblock {\em Phys. Rev. Lett.} 101, 076401, 2008.

\bibitem{JPHu2008PRL}
K. Seo, B. Andrei Bernevig, and J. P. Hu.
\newblock {Pairing symmetry in a two-orbital exchange coupling model of
  oxypnictides}.
\newblock {\em Phys. Rev. Lett.} 101, 206404, 2008.

\bibitem{SOnari2010HKontani}
H. Kontani and S. Onari.
\newblock {Orbital-fluctuation-mediated superconductivity in iron pnictides:
  analysis of the five-orbital Hubbard-Holstein model}.
\newblock {\em Phys. Rev. Lett.} 104, 157001, 2010.

\bibitem{YOno2010YYanagi}
Y. Yanagi, Y. Yamakawa, and Y. Ōno.
\newblock {Two types of $s$-wave pairing due to magnetic and orbital
  fluctuations in the two-dimensional 16-band $d\text{\ensuremath{-}}p$ model
  for iron-based superconductors}.
\newblock {\em Phys. Rev. B} 81, 054518, 2010.

\bibitem{XJZhou2021YQCai}
Y. Q. Cai, T. Xie, H. Yang, D. S. Wu, J. W. Huang, W. S. Hong,
  L. Cao, C. Liu, C. Li, Y. Xu, Q. Gao, T. M. Miao, G. D. Liu,
  S. L. Li, L. Huang, H. Q. Luo, Z. Y. Xu, H. J. Gao, L. Zhao, and
  X. J. Zhou.
\newblock {Common ($\pi$,$\pi$) band folding and surface reconstruction in
  FeAs-based superconductors}.
\newblock {\em Chin. Phys. Lett.} 38, 057404, 2021.

\bibitem{XJZhou2022DSWu}
D. S. Wu, J. J. Jia, J. G. Yang, W. S. Hong, Y. J. Shu, T. M. Miao,
  H. T. Yan, H. T. Rong, P. Ai, X. Zhang, C. H. Yin, C. L. Li,
  S. J. Zhang, F. F. Zhang, F. Yang, Z. M. Wang, N. Zong, L. J. Liu,
  R. K. Li, X. Y. Wang, Q. J. Peng, H. Q. Mao, G. D. Liu, S. L. Li,
  H. Q. Luo, X. X. Wu, Z. Y. Xu, L. Zhao, and X. J. Zhou.
\newblock {Nodal s$_\pm$ pairing symmetry in an iron-based superconductor with
  only hole pockets}.
\newblock {\em arXiv} 2212.03472, 2022.

\bibitem{AHMacdonald2011}
R. Bistritzer and A. H. MacDonald.
\newblock {Moiré bands in twisted double-layer graphene}.
\newblock {\em Proc. Natl. Acad. Sci.}
  108, 12233, 2011.

\bibitem{YLChen2017HPeng}
H. Peng, N. B. M. Schröter, J. B. Yin, H. Wang, T. F. Chung,
  H. F. Yang, S. Ekahana, Z. K. Liu, J. Jiang, L. X. Yang, T. Zhang, C. Chen, H. Ni, A. Barinov, Y. P. Chen, Z. F. Liu,
  H. L. Peng, and Y. L. Chen.
\newblock {Substrate doping effect and unusually large angle van Hove
  singularity evolution in twisted bi- and multilayer graphene}.
\newblock {\em Adv. Mater.} 29, 1606741, 2017.

\bibitem{JHPablo2018YCao}
Y. Cao, V. Fatemi, S. Fang, K. Watanabe, T. Taniguchi,
  E. Kaxiras, and P. J. Herrero.
\newblock {Unconventional superconductivity in magic-angle graphene
  superlattices}.
\newblock {\em Nature} 556, 43, 2018.

\bibitem{KMShen2015JWHarter}
J.  W. Harter, L. Maritato, D.  E. Shai, E.  J. Monkman, Y. F. Nie,
  D.  G. Schlom, and K.  M. Shen.
\newblock {Doping evolution and polar surface reconstruction of the
  infinite-layer cuprate
  ${\mathrm{Sr}}_{1\ensuremath{-}x}{\mathrm{La}}_{x}{\mathrm{CuO}}_{2}$}.
\newblock {\em Phys. Rev. B} 92, 035149, 2015.

\bibitem{YKuk2018SKim}
S. Kim, S. Yi, M. Oh, B. G. Jang, W. Nam, Y. C. Yoo, M. Lee, H. Jeon, I. Zoh,
  H. Lee, C. Zhang, K. H. Kim, J. Seo, J. H. Shim, J. Chae, and Y. Kuk.
\newblock {Surface reconstruction and charge modulation in BaFe$ _{2} $As$ _{2} $ superconducting film}.
\newblock {\em J. Condens. Matter Phys.} 30, 315001, 2018.

\bibitem{XJZhou2008GDLiu}
G. Liu, G. Wang, Y. Zhu, H. Zhang, G. Zhang, X. Wang, Y. Zhou, W. Zhang,
  H. Liu, L. Zhao, J. Meng, X. Dong, C. Chen, Z. Xu, and X. J. Zhou.
\newblock {Development of a vacuum ultraviolet laser-based angle-resolved
  photoemission system with a superhigh energy resolution better than 1 meV}.
\newblock {\em Rev. Sci. Instrum.} 79, 023105, 2008.

\bibitem{XJZhou2018}
X. Zhou, S. He, G. Liu, L. Zhao, L. Yu, and W. Zhang.
\newblock {New developments in laser-based photoemission spectroscopy and its
  scientific applications: a key issues review}.
\newblock {\em Rep. Prog. Phys.} 81, 062101, 2018.

\bibitem{XJZhou2016YXZhang}
Y. X. Zhang, L. Zhao, G. D. Gu, and X. J. Zhou.
\newblock {A reproducible approach of preparing high-quality overdoped
  Bi$_2$Sr$_2$CaCu$_2$O$_{8+\delta}$ single crystals by oxygen annealing and
  quenching method}.
\newblock {\em Chin. Phys. Lett.} 33, 6, 2016.

\end{thebibliography}








\newpage


% Figure environment removed



% Figure environment removed



% Figure environment removed


% Figure environment removed





\end{document} 