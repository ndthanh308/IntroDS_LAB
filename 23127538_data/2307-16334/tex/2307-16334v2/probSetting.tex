\section{Problem setting and parametric multi-domain formulation}
\label{sec:setting}

Let $\Omega \subset \R^d$ ($d = 1, 2, 3$) be an open bounded domain with Lipschitz boundary $\partial\Omega = \Gamma^D \cup \Gamma^N$, such that $\Gamma^D \cap \Gamma^N = \emptyset$. Let $\bmu =(\mu^1,\ldots,\mu^P) \in \mathcal{P}$ be a tuple of $P \in \N$ problem parameters with $\mathcal{P} = \mathcal{I}^1 \times \dots \times \mathcal{I}^P \subset \R^P$ and each $\mathcal{I}^p$ compact ($p=1,\ldots,P$).
Consider the linear elliptic parametric operator
\begin{equation*}
  L(u(\bmu);\bmu) = -\nabla\cdot(\nu(\bmu)\nabla u(\bmu)) + \boldsymbol{\alpha}(\bmu)\cdot\nabla u(\bmu) + \gamma(\bmu) u(\bmu) \, ,
\end{equation*}
and the parametric boundary value problem: for all $\bmu \in \mathcal{P}$, find $u(\bmu)$ such that 
\begin{equation}
	\label{eq:globalProb}
	\begin{array}{rcll}
		L(u(\bmu);\bmu) &=& s(\bmu) &\quad \text{in } \Omega,\\
		u(\bmu) &=& g^D(\bmu)& \quad \text{on } \Gamma^D,\\
		\Neum{u(\bmu)}{} &=& g^N(\bmu) & \quad \text{on } \Gamma^N,
	\end{array}
\end{equation}
where $s(\bmu)$ denotes the source term and $g^D(\bmu)$ and $g^N(\bmu)$ are given functions that prescribe Dirichlet and Neumann boundary conditions on $\Gamma^D$ and $\Gamma^N$, respectively, with $\bn$ the unit normal vector to $\Gamma^N$, pointing outwards of the domain. Note that all the above material data, physical quantities and boundary conditions are functions of the parameters $\bmu$. For the sake of readability, the domain and its boundary are assumed to be independent of $\bmu$, although the framework presented in this work can be applied also to geometric parameters, as shown in the numerical example of Sect.~\ref{sec:testRozza}.

To guarantee the well posedness of~\eqref{eq:globalProb}, we assume that, for all $\bmu \in \mathcal{P}$, there exists $\nu_0 >0$ such that $\nu(\bmu) \geq \nu_0$, that $\boldsymbol{\alpha}(\bmu), \gamma(\bmu) \in L^\infty(\Omega)$, and that there exists $\gamma_0 \geq 0$ such that
\begin{equation*}
-\frac{1}{2} \nabla \cdot \boldsymbol{\alpha}(\bmu) + \gamma(\bmu) \geq \gamma_0 \qquad \mbox{a.e. in } \Omega \, ,
\end{equation*}
with
\begin{equation*}
\| \boldsymbol{\alpha}(\bmu) \cdot \bn \|_{L^\infty(\Gamma^N)} < \frac{2}{C_t} \min \left( \frac{\nu_0}{2}, \frac{\nu_0}{2} C_\Omega^{-1} + \gamma_0 \right) \, ,
\end{equation*}
and $C_t,C_\Omega>0$ being the trace and Poincar\'e constants, respectively (see, e.g., \cite{Quarteroni:1994}).

Consider a decomposition of the domain $\Omega$ into two overlapping subdomains $\Omega_i \subset \Omega$ ($i = 1, 2$) such that $\Omega_1 \cup \Omega_2 = \Omega$ and $\Omega_1 \cap \Omega_2 = \Omega_{12} \neq \emptyset$. For $i=1,2$, let $\Gamma_i = \partial\Omega_i \setminus \partial\Omega$ as shown in Fig.~\ref{fig:partition}, let $\Gamma = \cup_{i} \Gamma_i$ be the union of all interfaces, and let $\Gamma_i^D = \Gamma^D \cap \partial\Omega_i$ and $\Gamma_i^N = \Gamma^N \cap \partial\Omega_i$.

For clarity of exposition, we henceforth focus on the case of two subdomains, but the approach can be straightforwardly extended to the case of more than two subdomains without cross-points, as shown in Sect.~\ref{sec:testPatera}.

% Figure environment removed

Problem~\eqref{eq:globalProb} can be rewritten in the equivalent multi-domain formulation:
%
for all $\bmu \in \mathcal{P}$, find $u_i(\bmu)$ ($i = 1, 2$) such that
\begin{equation}
	\label{eq:multiDom}
	\begin{array}{rcll}
		L(u_i(\bmu);\,\bmu) &=& s_i(\bmu) & \quad \text{in } \Omega_i,\\
		u_i(\bmu) &=& g^D_i(\bmu) & \quad \text{on } \Gamma_i^D,\\
		\Neum{u_i(\bmu)}{} &=& g^N_i(\bmu) & \quad \text{on } \Gamma_i^N,\\
		u_1(\bmu) &=& u_2(\bmu) & \quad \text{on } \Gamma\,,
	\end{array}
\end{equation}
%
where $s_i(\bmu)$, $g^D_i(\bmu)$ and $g^N_i(\bmu)$ denote the restrictions of $s(\bmu)$, $g^D(\bmu)$ and $g^N(\bmu)$ to $\Omega_i$, $\Gamma_i^D$ and $\Gamma_i^N$, respectively.

It is worth noticing that the continuity of the local solutions $u_1(\bmu)$ and $u_2(\bmu)$ across the interfaces $\Gamma_1$ and $\Gamma_2$ follows from the last condition in~\eqref{eq:multiDom}. 
%
The equivalence of~\eqref{eq:globalProb} and~\eqref{eq:multiDom} can be proved by straightforwardly extending the argument of Proposition 2.1 of \cite{Discacciati:2013:SICON} to take into account the parameters $\bmu \in \mathcal{P}$. Due to the equivalence, there holds $u_i(\bmu) = u(\bmu)|_{\Omega_i}$ for $i=1,2$ and for all $\bmu \in \mathcal{P}$, and, in particular, $u_1(\bmu) = u_2(\bmu)$ in $\Omega_{12}$.
%
\rev{Moreover, the continuity of the local solutions in the overlapping region $\Omega_{12}$ straightforwardly guarantees the continuity of the normal fluxes at the interfaces, thus avoiding imposing this condition explicitly}.


%==========================================================================
\subsection{Parametric overlapping Schwarz method}
%==========================================================================
The multi-domain problem~\eqref{eq:multiDom} can be used to formulate the overlapping Schwarz method (see, e.g., \cite{Smith:1996}) reported in Algorithm~\ref{alg:Schwarz}. Problems~\eqref{eq:localProblem1} and~\eqref{eq:localProblem2} are solved on the local subdomains $\Omega_1$ and $\Omega_2$, respectively (Algorithm~\ref{alg:Schwarz}, steps 2 and 4), with appropriate Dirichlet-type boundary conditions at the interfaces (Algorithm~\ref{alg:Schwarz}, steps 3 and 5). The algorithm stops when the discrepancy between the local solutions at the interfaces is below a user-defined tolerance $\texttt{tol}$ (Algorithm~\ref{alg:Schwarz}, step 6) and the solution $u(\bar{\bmu})$ of the parametric problem~\eqref{eq:globalProb} for the fixed set of parameters $\bar{\bmu} \in \mathcal{P}$ is retrieved by gluing the local solutions (Algorithm~\ref{alg:Schwarz}, step 7).
%
For the definition of the stopping criterion, a suitable norm $\|\cdot\|_{\Gamma}$ on $\Gamma$ must be defined. In the numerical simulations of Sect.~\ref{sec:results}, the $l^{\infty}(\Gamma)$ norm is employed.
%
\begin{subequations}
\begin{algorithm}[!ht]
\caption{Overlapping Schwarz method for a two-domain parametric problem}\label{alg:Schwarz}
\begin{algorithmic}[1]
\REQUIRE{Fixed set of parametric values $\bar{\bmu} \in \mathcal{P}$, tolerance $\texttt{tol}$ for the stopping criterion and initial function $\lambda_1^{(0)}$ defined on the interface $\Gamma_1$.}
\STATE{Set $k=1$.}
\STATE{
Find $u_1^{(k)}\!(\bar{\bmu})$ such that 
	\begin{equation}\label{eq:localProblem1}
		\begin{array}{rcll}
			L (u_1^{(k)}\!(\bar{\bmu}); \, \bar{\bmu}) &=& s_1(\bar{\bmu}) & \quad \text{in } \Omega_1,\\
			u_1^{(k)}\!(\bar{\bmu}) &=& g_1^D(\bar{\bmu}) & \quad \text{on } \Gamma_1^D,\\
			\displaystyle
			\nu(\bar{\bmu}) \nabla u_1^{(k)}\!(\bar{\bmu}) \cdot \bn &=& g_1^N(\bar{\bmu}) & \quad \text{on } \Gamma_1^N,\\
			u_1^{(k)}\!(\bar{\bmu}) &=& \lambda_1^{(k-1)} & \quad \text{on } \Gamma_1 .
		\end{array}
	\end{equation}
	\label{alg:firstStep}
}
\STATE{Set $\lambda_2^{(k)} = u_1^{(k)}\!(\bar{\bmu})|_{\Gamma_2}$.}
\STATE{
Find $u_2^{(k)}\!(\bar{\bmu})$ such that 
	\begin{equation}\label{eq:localProblem2}
		\begin{array}{rcll}
			L (u_2^{(k)}\!(\bar{\bmu}); \, \bar{\bmu}) &=& s_2(\bar{\bmu}) & \quad \text{in } \Omega_2,\\
			u_2^{(k)}\!(\bar{\bmu}) &=& g_2^D(\bar{\bmu}) & \quad \text{on } \Gamma_2^D,\\
			\displaystyle
			\nu(\bar{\bmu}) \nabla u_2^{(k)}\!(\bar{\bmu}) \cdot \bn &=& g_2^N(\bar{\bmu}) & \quad \text{on } \Gamma_2^N,\\
			u_2^{(k)} &=& \lambda_2^{(k)} & \quad \text{on } \Gamma_2.
		\end{array}
	\end{equation}
}
\STATE{Set $\lambda_1^{(k)}= u_2^{(k)}\!(\bar{\bmu})|_{\Gamma_1}$.}
\IF{$\|u_1^{(k)}\!(\bar{\bmu})_{\vert\Gamma} - u_2^{(k)}\!(\bar{\bmu})_{\vert\Gamma}\|_\Gamma < \texttt{tol}$}
\STATE{Construct the solution on the entire domain as
\begin{equation}\label{eq:finalSol}
u(\bar{\bmu}) = \begin{cases}
u_1^{(k)}\!(\bar{\bmu}) & \text{in } \Omega_1 \\
u_2^{(k)}\!(\bar{\bmu}) & \text{in } \Omega_2 \setminus \Omega_{12}
\end{cases}
\end{equation}
}
\ELSE
\STATE{$k \gets k+1$.}
\STATE{\textbf{go to} step~\ref{alg:firstStep}.}
\ENDIF
\ENSURE{Solution $u(\bar{\bmu})$ in the entire domain $\Omega$.}
\end{algorithmic}
\end{algorithm}
\end{subequations}


%==========================================================================
\subsection{Algebraic formulation of the Schwarz method}
\label{sec:schwarzAlg}
%==========================================================================

Consider a finite element discretization of problems~\eqref{eq:localProblem1} and~\eqref{eq:localProblem2}.
%
More precisely, let $\mat{A}_{\Omega_i}$ ($i=1,2$) be the invertible matrix associated with the finite element approximation of the local problem in the subdomain $\Omega_i$,  whose rows and columns correspond to the degrees of freedom inside $\Omega_i$, excluding the unknowns at the interface $\Gamma_i$. 
%
Moreover, let $\mat{A}_{\Gamma_i}$ ($i=1,2$) be the finite element matrix with rows corresponding to the degrees of freedom inside $\Omega_i$ and columns associated with the degrees of freedom on $\Gamma_i$, whereas let $\mathbf{f}_{\Omega_i}$ ($i=1,2$) denote the finite element vector accounting for the contributions of the source term and the Neumann boundary conditions.
%
Finally, let $\mat{R}_{\Omega_i\to\Gamma_j}$ ($i,j=1,2$, $i\not= j$) be the restriction matrix that, for any vector of nodal values inside $\Omega_i$, returns the vector of nodal values at the interface $\Gamma_j$ internal to $\Omega_i$. 
% 
Hence, following, e.g., \cite[Sect. 1.1.1]{Smith:1996},  the procedure in Algorithm~\ref{alg:Schwarz} can be rewritten in algebraic form as a block Gauss-Seidel method for the linear system
%
\begin{equation}\label{eq:systemSchwarz}
\begin{pmatrix}
\mat{A}_{\Omega_1} & \mat{A}_{\Gamma_1} & \mat{0} & \mat{0} \\
\mat{0} & \mat{I}_{\Gamma_1} & - \mat{R}_{\Omega_2\to\Gamma_1} & \mat{0} \\
\mat{0} & \mat{0} & \mat{A}_{\Omega_2} & \mat{A}_{\Gamma_2} \\
- \mat{R}_{\Omega_1\to\Gamma_2} & \mat{0} & \mat{0} & \mat{I}_{\Gamma_2}
\end{pmatrix}
\begin{pmatrix}
\mathbf{u}_{\Omega_1}\!(\bar{\bmu}) \\
\mathbf{u}_{\Gamma_1}\!(\bar{\bmu}) \\
\mathbf{u}_{\Omega_2}\!(\bar{\bmu}) \\
\mathbf{u}_{\Gamma_2}\!(\bar{\bmu})
\end{pmatrix}
=
\begin{pmatrix}
\mathbf{f}_{\Omega_1}\!(\bar{\bmu}) \\
\mathbf{0} \\
\mathbf{f}_{\Omega_2}\!(\bar{\bmu}) \\
\mathbf{0}
\end{pmatrix}\,,
\end{equation}
%
where $\mathbf{u}_{\Omega_i}(\bar{\bmu})$ and $\mathbf{u}_{\Gamma_i}(\bar{\bmu})$ denote the vectors of nodal values inside the domain $\Omega_i$ and at the interface $\Gamma_i$, respectively, whereas $\mat{I}_{\Gamma_i}$ represents the identity matrix at $\Gamma_i$ ($i=1,2$).

By computing the Schur complement of the linear system~\eqref{eq:systemSchwarz},  the degrees of freedom $\mathbf{u}_{\Omega_1}(\bar{\bmu})$ and $\mathbf{u}_{\Omega_2}(\bar{\bmu})$ internal to each subdomain can be eliminated, expressing them in terms of the unknowns $\mathbf{u}_{\Gamma_1}(\bar{\bmu})$ and $\mathbf{u}_{\Gamma_2}(\bar{\bmu})$ on $\Gamma_1$ and $\Gamma_2$, respectively,  yielding the so-called \emph{interface system}
%
\begin{equation}\label{eq:systemSchwarzInterface}
\hspace*{-3mm}
\begin{pmatrix}
\mat{I}_{\Gamma_1} & \mat{R}_{\Omega_2\to\Gamma_1} \mat{A}_{\Omega_2}^{-1} \mat{A}_{\Gamma_2} \\[3pt]
\mat{R}_{\Omega_1\to\Gamma_2} \mat{A}_{\Omega_1}^{-1} \mat{A}_{\Gamma_1} & \mat{I}_{\Gamma_2}
\end{pmatrix}
\begin{pmatrix}
\mathbf{u}_{\Gamma_1}\!(\bar{\bmu}) \\[3pt]
\mathbf{u}_{\Gamma_2}\!(\bar{\bmu})
\end{pmatrix}
=
\begin{pmatrix}
\mat{R}_{\Omega_2\to\Gamma_1} \mat{A}_{\Omega_2}^{-1} \mathbf{f}_{\Omega_2}\!(\bar{\bmu}) \\[3pt]
\mat{R}_{\Omega_1\to\Gamma_2} \mat{A}_{\Omega_1}^{-1} \mathbf{f}_{\Omega_1}\!(\bar{\bmu})
\end{pmatrix} \,  ,
\end{equation}
%
which can be solved using a suitable matrix-free Krylov method, e.g., GMRES~\cite{Saad:1986:SISSC}. 

It is worth noticing that problem~\eqref{eq:systemSchwarzInterface} corresponds to steps 2--5 of Algorithm~\ref{alg:Schwarz}, namely
%
\begin{equation}\label{eq:SchwarzAlg}
\mat{I}_{\Gamma_j}\mathbf{u}_{\Gamma_j}\!(\bar{\bmu})
=
\mat{R}_{\Omega_i\to\Gamma_j} \mat{A}_{\Omega_i}^{-1} \mathbf{f}_{\Omega_i}\!(\bar{\bmu}) 
+ \mat{R}_{\Omega_i\to\Gamma_j} \mat{A}_{\Omega_i}^{-1} (-\mat{A}_{\Gamma_i}\mathbf{u}_{\Gamma_i}\!(\bar{\bmu})) \, ,
\end{equation}
where the matrix-vector operations on the right-hand side of equation~\eqref{eq:SchwarzAlg} are the algebraic counterpart of the following operations:
%
\begin{enumerate}[label=(\Alph*)]
\item extension into subdomain $\Omega_i$ of the Dirichlet datum $\mathbf{u}_{\Gamma_i}\!(\bar{\bmu})$ at interface $\Gamma_i$ by the matrix-vector product $-\mat{A}_{\Gamma_i} \mathbf{u}_{\Gamma_i}\!(\bar{\bmu})$;
%
\item computation of the local solution $\mathbf{u}_{\Omega_i}\!(\bar{\bmu})$ in subdomain $\Omega_i$ as the superposition of $\mat{A}_{\Omega_i}^{-1} \mathbf{f}_{\Omega_i}\!(\bar{\bmu})$ and $\mat{A}_{\Omega_i}^{-1} \left(- \mat{A}_{\Gamma_i}\mathbf{u}_{\Gamma_i}\!(\bar{\bmu}) \right)$, solving a linear system with matrix $\mat{A}_{\Omega_i}$;
%
\item restriction of the computed solution $\mathbf{u}_{\Omega_i}\!(\bar{\bmu})$ to the internal interface $\Gamma_j$, $j \not= i$, through the restriction matrix $\mat{R}_{\Omega_i\to\Gamma_j}$.
\end{enumerate}

Note that the computational effort required by the above Schwarz method is proportional to the cost of solving the local problems~\eqref{eq:localProblem1} and~\eqref{eq:localProblem2}, that is, the cost of step (B). This can become demanding when a new set of parameters $\bar{\bmu} \in \mathcal{P}$ is to be tested since the entire procedure needs to be executed from scratch. Indeed,  the matrices $\mat{A}_{\Omega_i}$ and $\mat{A}_{\Gamma_i}$ may themselves depend on the parameters $\bar{\bmu}$ defining the novel configuration under analysis.

%==========================================================================
\subsection{The DD-PGD strategy}
\label{sec:DDPGDstrategy}
%==========================================================================

To reduce the computational cost, in this paper the Schwarz algorithm is combined with a PGD-based surrogate model to efficiently obtain the solution of the parametric problem~\eqref{eq:globalProb}, for any set of parameters $\bmu \in \mathcal{P}$. To this aim, the Schwarz algorithm is reformulated by identifying an offline phase and an online phase as follows.
\begin{enumerate}
\item
In the \emph{offline} phase, the local parametric problems~\eqref{eq:localProblem1} and~\eqref{eq:localProblem2} are solved using the PGD method to devise a set of surrogate solutions $\upgd_i$ ($i=1,2$) explicitly depending on space, $\bx$, on problem parameters, $\bmu$, and on arbitrary, problem-relevant functions $\lambda_i$\rev{, which represent the traces of the unknown solution on the interfaces $\Gamma_i$.} 
% 
The arbitrariness of the functions $\lambda_i$ is dealt with by parametrizing them through a set of auxiliary parameters, say $\bLambda_i$, as detailed in Sect.~\ref{sec:offline}. The output of the offline phase is the set of local surrogate models $\upgd_i$, featuring arbitrary traces at the interfaces, which are thus suitable for efficient evaluations during the Schwarz algorithm. This procedure is meant to replace the computationally demanding step (B).
\item
In the \emph{online} phase, the Schwarz algorithm is performed using the interface formulation \eqref{eq:systemSchwarzInterface}. For a fixed set of parametric values $\bar{\bmu} \in \mathcal{P}$, at each iteration of the algorithm, the extension of Dirichlet interface data and the solution of the local problems~\eqref{eq:localProblem1} and~\eqref{eq:localProblem2} in steps (A) and (B) are replaced by the evaluation of the precomputed local surrogate models at specific instances of the auxiliary interface parameters $\bLambda_i$. It is worth noticing that, contrary to alternative \emph{a posteriori} ROMs requiring the solution of small problems in the online phase, the evaluation of the PGD surrogate model for a specific value of the parameters only relies on interpolation procedures, thus allowing the Schwarz algorithm to be executed in real time. Details of the online phase are provided in Sect.~\ref{sec:online}.
\end{enumerate}

\rev{
\begin{rem}
The overlapping Schwarz method used in the online phase has been mainly chosen for computational efficiency in the offline phase. Indeed, while non-overlapping DD techniques such as, e.g., Neumann-Neumann or FETI methods~\cite{Toselli:2005} could alternatively be used to rewrite the parametric problem~\eqref{eq:globalProb} into an equivalent multi-domain formulation, in the online phase both the continuity of traces and the continuity of fluxes would have to be imposed through a suitably preconditioned interface equation.
%
In the offline phase, this would entail the solution of local parametric problems with arbitrary fluxes at the interfaces, besides those with arbitrary traces employed in the present strategy. Therefore, although alternative DD strategies are possible, in this work only the overlapping Schwarz method is considered to avoid increasing the overall computational cost of the DD-ROM procedure.
\end{rem}
}