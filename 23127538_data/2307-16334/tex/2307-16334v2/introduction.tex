\section{Introduction}
\label{sec:introduction}

Model order reduction (MOR) techniques~\cite{Chinesta:2017:ECM} represent established methodologies for the solution of multi-queries problems, e.g., parametric partial differential equations (PDEs), arising from computationally intensive applications such as uncertainty quantification, optimization, data assimilation and real-time control~\cite{Gunzburger-PWG-18}.
%
While these techniques have achieved full maturity, their employment in the construction of digital twins of large-scale, multi-physics, multi-disciplinary, systems is still limited by the computational cost of the high-fidelity simulations required during the offline phase of the reduced order model (ROM).
%
In this context, the last decade has witnessed a growing interest towards the combination of domain decomposition (DD) methods \cite{Smith:1996,Quarteroni:1999,Toselli:2005,Dolean-DJN-15} with ROMs, to reduce the number of coupled parameters and/or degrees of freedom of parametric surrogate models, see, e.g., the recent reviews~\cite{Buhr:2021,Klawonn-HKLW-21}. This is particularly critical in the context of multi-physics phenomena~\cite{Hesthaven-DH-23}, such as micro-electro-mechanical systems~\cite{Corigliano-CDM-13} and the Stokes-Darcy problem~\cite{Martini:2015:ACM}.

Stemming from the seminal works by Maday and R\o nquist on the reduced basis element (RBE) method~\cite{Maday:2002:JSC,Maday:2004:SISC,Lovgren:2006:M2AN}, many strategies have been proposed to (i) split the computational domain of a parametric PDE into subdomains, possibly characterized by simpler geometrical shapes, (ii) compute local approximations of the parametric solutions and (iii) efficiently recompose them to obtain a surrogate model of the original problem.
%
Existing approaches in the literature propose a combination of overlapping and non-overlapping strategies in step (i), different MOR techniques for step (ii), including reduced basis (RB), proper orthogonal decomposition (POD) and proper generalized decomposition (PGD), whereas step (iii) is performed via Lagrange multipliers and Schwarz iterations, just to name a few.

In the context of RB, non-overlapping DD techniques have been mainly adopted. Indeed, the RBE method relies on a non-overlapping DD approach, where the local solutions in the subdomains are constructed using the RB method and glued together using a mortar approach.
%
To reduce the cost of RBE, in~\cite{Patera-HKP-13,Eftang:2013:IJNME}, the authors leverage the idea of static condensation by expressing the local ROM as a function of the reduced approximation of the interface solution, or port, and a strategy to derive optimal port spaces is proposed in~\cite{Patera-SP-16}.
%
A variation of the RBE method relying on parametric boundary conditions at the subdomain interfaces is presented in~\cite{Iapichino:2016:CMA} and strategies to couple non-conforming meshes in the RBE context have been studied in~\cite{Antonietti:2016:M2AN} and~\cite{Gervasio-ZMGQ-22}.

Other ROM solutions inspired by classical DD techniques have been proposed in the literature: the RB hybrid method~\cite{Iapichino:2012:CMAME} employs a coarse mesh strategy with a global high-fidelity solution to ensure the continuity of normal fluxes across the interfaces; the hierarchical model reduction~\cite{Perotto-PEV-10} approach uses a non-overlapping DD method to glue local ROMs obtained from separated representations in the longitudinal and transversal directions; a stabilized local POD-ROM is constructed with overlapping and non-overlapping penalizations in~\cite{Baiges-BCI-13}; \cite{Maier:2014:ANM} couples a Dirichlet-Neumann method on conforming meshes with local POD solutions; in~\cite{Barnett:2022:Sandia}, overlapping and non-overlapping Schwarz alternating methods are discussed in the context of POD approximations.
%
Recently, local surrogate models have also been coupled using optimization-based DD approaches.  
%
This is the case of~\cite{Iollo-IST-23}, where a constrained optimization procedure is formulated by minimizing the $L^2$-norm of the jump of the local surrogate models at the interfaces between overlapping subdomains, and of~\cite{Rozza-PNTBR-22} that proposes a non-overlapping variational PDE-constrained optimization approach which minimizes the $L^2$-norm of the distance between the local ROMs at the interface, with an appropriate penalization for the fluxes.

An alternative approach to reduce the cost of ROMs stems from rethinking the parametric PDE as a multi-scale problem.
%
For example, the localized RB multi-scale method~\cite{Ohlberger-OS-15} employs a coarse non-overlapping partition to devise local ROMs, which are later coupled using a discontinuous Galerkin ansatz at the interface.
%
Following the variational multi-scale rationale, \cite{Veroy-DVRU-23} proposes an additive splitting of the solution into a coarse and a fine scale, and, using the fine-scale information, the coarse local surrogate models -- obtained by solving an oversampling problem with random boundary conditions~\cite{Smetana-BS-18} -- are coupled via a reduced interface basis.

Despite the extensive efforts to couple DD methods and ROMs, existing approaches still present different shortcomings.
%
On the one hand, non-overlapping strategies are currently limited by the high cost of devising a parametric representation of the global solution at the interface, leading to high-dimensional spaces to be explored using surrogate models.
%
On the other hand, most of the existing MOR techniques rely on intrusive implementations with respect to the high-fidelity solver employed to compute the snapshots and require the solution of additional, low-dimensional, problems during the online phase to evaluate the surrogate model for a new set of parameters.
%
This is particularly critical in the context of industrial problems, where commercial and proprietary software is commonly employed for simulations and source codes are not accessible.
%
To tackle these issues, increasing attention has been recently devoted to overlapping DD strategies~\cite{Barnett:2022:Sandia,Iollo-IST-23} and non-intrusive solutions to build local ROMs with DD techniques, e.g., by relying on purely algebraic formulations~\cite{Hoang:2021:CMAME} or by combining POD with radial basis functions~\cite{Pain-XFHNP-19}, Gaussian process regression~\cite{Pain-XHFMHBANP-19} and autoencoders~\cite{Pain-HWTKSNNMSP-22}.
%
Although purely data-driven approaches offer appealing solutions for non-intrusive surrogate models, it is well known that they may lack physical interpretability. For this reason, non-intrusive solutions incorporating physical information, e.g., via physics-informed neural networks~\cite{Raissi-RPK-19}, have gained increasing attention in recent years, being also successfully coupled with non-overlapping~\cite{Karniadakis-JKK-20,Karniadakis-JK-21} and overlapping~\cite{Li-LTWL-19,NissenMeyer-MMN-21,Dolean-DHMM-23} DD approaches.

An alternative solution to circumvent the above mentioned issues is represented by PGD~\cite{Chinesta-AMCK-06,Chinesta:2014}.
%
PGD offers a physics-based \emph{a priori} MOR framework,  with an \emph{offline} phase constructing a rank-one approximation with no prior knowledge of the solution and an \emph{online} phase where efficient evaluations of the surrogate model are performed by simple interpolation in the parametric space.  
%
This allows a seamless integration of the resulting ROM with any full-order solver, without the need for any extra solution step in the online phase.
%
Indeed, non-intrusive PGD implementations, paired with software such as SAMCEF, Abaqus, OpenFOAM and MSC-Nastran, have been presented in~\cite{Ladeveze-CNLB-16,Zou-ZCDA-18,Tsiolakis-TGSOH-20,Tsiolakis-TGSOH-22,Cavaliere-CZSLD-22}.
%
Moreover, a fully algebraic, non-intrusive framework -- the so-called \emph{encapsulated PGD} -- has been recently proposed in~\cite{Diez:2020:ACME}.

In the context of PGD, DD strategies were first introduced in~\cite{Nazeer:2014:CM} for the overlapping case and in~\cite{Huerta:2017:IJNME} for the non-overlapping one.
%
More precisely, the Arlequin method~\cite{Nazeer:2014:CM} constructs a local PGD solution in each subdomain and exploits Lagrange multipliers, defined as separated functions in the overlapping regions, to couple the surrogate models, thus leading to a global system involving both local unknowns and Lagrange multipliers.
%
The approach in~\cite{Huerta:2017:IJNME} relies on a non-overlapping Dirichlet-Dirichlet method: during the offline phase, the local surrogate models are computed in each subdomain as a function of a suitable representation of the trace of the unknown at the interface; in the online phase, the interface problem is solved to impose the continuity of fluxes.
%
Due to the separated representation of the PGD solution, it follows that the resulting interface equation is nonlinear, even when the original problem is a linear PDE,  thus requiring an appropriate iterative scheme, such as the Newton-Raphson method.
%

In this work,  a DD-PGD computational framework is devised in the context of linear elliptic PDEs to remedy the shortcomings of existing approaches.  
%
The strategy relies on an overlapping Schwarz algorithm, executed online, to couple the local surrogate models constructed offline in each subdomain. Parametric Dirichlet boundary conditions are employed at the subdomain level and the linearity of the operator is exploited to reduce the dimensionality of each local ROM via superimposition.  This allows to devise a set of low-dimensional problems which can be easily parallelized to enhance the performance of the method. Differently from~\cite{Huerta:2017:IJNME},  the proposed method does not require the definition of any auxiliary basis functions at the interfaces, but it can rely, e.g., on the traces of the finite element functions used for the spatial discretization within each subdomain. 
%
Moreover,  thanks to an \emph{ad hoc} multi-domain reformulation of the original parametric problem,  the coupling in the online phase occurs only at the interfaces,  instead of across the whole overlapping region as in~\cite{Nazeer:2014:CM}.  The continuity of the solution and of its fluxes is indeed guaranteed by imposing the equality of the traces of the local PGD solutions at the interfaces, without Lagrange multipliers (with separated representations) to glue the local ROMs in the entire overlap. This is practically achieved by formulating the overlapping Schwarz algorithm as a parametric interface linear system, which can be efficiently solved in real time in the online phase by standard matrix-free Krylov methods, while the local surrogate models are evaluated via interpolation in the parametric domain, with no extra solution step.
%
Finally, the proposed methodology provides a physics-based PGD-ROM,  non-intrusive with respect to the high-fidelity spatial solver and featuring a reduced number of interface parameters in each subproblem solved in the offline phase.

The remainder of this paper is structured as follows. Section~\ref{sec:setting} introduces the parametric elliptic problem, its multi-domain formulation and the main idea of the proposed DD-PGD approach. In Sect.~\ref{sec:offline},  the offline phase of the method is presented,  explaining the rationale for constructing local PGD surrogate models with parametric boundary conditions at the interfaces between subdomains,  while reducing the dimensionality of the resulting problem by exploiting the linearity of the underlying operator. The online phase accounting for the parametric overlapping Schwarz method to solve the linear interface system is described in Sect.~\ref{sec:online}, while in Sect.~\ref{sec:results} numerical tests are presented to assess the accuracy, robustness and efficiency of the proposed methodology. Finally,  Sect.~\ref{sec:Conclusions} summarizes the conclusions of this work and two appendices provide technical details on the encapsulated PGD framework and the implementation of the presented approach.