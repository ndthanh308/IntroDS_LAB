\section{Encapsulated proper generalized decomposition solver}\label{sec:encapsulated}
\label{append:encapsulatedPGD}
%==========================================================================

In this appendix, some technical aspects related to the solution of the parametric linear systems~\eqref{eq:algebraicA} and~\eqref{eq:algebraicB} using the encapsulated PGD toolbox are presented. For a detailed description of the method, the interested reader is referred to~\cite{Diez:2020:ACME}.

The goal is to compute the solutions $\bvpgd_{i,0}(\bmu)$ and $\bvpgd_{i,j}(\bmu, \bLambda_i^j)$ in separated form, that is, as the sum of rank-one approximations.
%
It follows that $M_0$ pairs $(\bV_{i,0}^m,\phi_{i,0}^m(\bmu))$ need to be determined for the PGD expansion~\eqref{eq:algebraicSolA} of $\bvpgd_{i,0}(\bmu)$. Similarly, the triplets $(\bV_{i,j}^m,\phi_{i,j}^m(\bmu),\psi_{i,j}^m(\bLambda_i^j))$, for $m=1,\ldots,M_j$, are sought to approximate $\bvpgd_{i,j}(\bmu, \bLambda_i^j)$ according to~\eqref{eq:algebraicSolB}.
%
To this end, the encapsulated PGD method relies on a greedy procedure to compute the $m$-th spatial and parametric modes, assuming that the terms up to $m-1$ are known, thus leading to the parametric problems
%
\begin{subequations}\label{eq:algebraicGreedy}
\begin{align}
\left(
\sum_{\ell=1}^{n_{\nu}} \xi_{\nu}^\ell(\bmu) \mat{K}_{\nu}^\ell
+ \sum_{\ell=1}^{n_{\alpha}} \xi_{\alpha}^\ell(\bmu) \mat{K}_{\alpha}^\ell
+ \sum_{\ell=1}^{n_{\gamma}} \xi_{\gamma}^\ell(\bmu) \mat{K}_{\gamma}^\ell
\right)
\bV_{i,0}^m \phi_{i,0}^m(\bmu) 
=
\mathbf{R}_{i,0}^{m-1}(\bmu) &
\notag \\[-1em]
\qquad \forall \bmu \in \mathcal{P} \, , &
\label{eq:greedyA} \\
\left(
\sum_{\ell=1}^{n_{\nu}} \xi_{\nu}^\ell(\bmu) \mat{K}_{\nu}^\ell
+ \sum_{\ell=1}^{n_{\alpha}} \xi_{\alpha}^\ell(\bmu) \mat{K}_{\alpha}^\ell
+ \sum_{\ell=1}^{n_{\gamma}} \xi_{\gamma}^\ell(\bmu) \mat{K}_{\gamma}^\ell
\right) 
\bV_{i,j}^m \phi_{i,j}^m(\bmu) \psi_{i,j}^m(\bLambda_i^j ) 
=
\mathbf{R}_{i,j}^{m-1}(\bmu, \bLambda_i^j ) &
\notag \\[-1em]
\forall \bmu \in \mathcal{P} \, \text{ and } \forall \bLambda_i^j \in \mathcal{Q}_i^j \, , &
\label{eq:greedyB}
\end{align}
\end{subequations}
%
where $\mathbf{R}_{i,0}^{m-1}(\bmu)$ and $\mathbf{R}_{i,j}^{m-1}(\bmu, \bLambda_i^j )$ denote the residuals
%
\begin{subequations}\label{eq:residualsGreedy}
\begin{align}
\mathbf{R}_{i,0}^{m-1}(\bmu) :=&
\sum_{\ell=1}^{n_s} \xi_s^\ell(\bmu) \mathbf{f}_s^\ell
+ \sum_{\ell=1}^{n_N} \xi_N^\ell(\bmu) \mathbf{f}_N^\ell
+ \sum_{\ell=1}^{n_D} \xi_D^\ell(\bmu) \mathbf{f}_D^\ell
\notag \\
&-
\sum_{k=1}^{m-1} 
\left(
\sum_{\ell=1}^{n_{\nu}} \xi_{\nu}^\ell(\bmu) \mat{K}_{\nu}^\ell
+ \sum_{\ell=1}^{n_{\alpha}} \xi_{\alpha}^\ell(\bmu) \mat{K}_{\alpha}^\ell
+ \sum_{\ell=1}^{n_{\gamma}} \xi_{\gamma}^\ell(\bmu) \mat{K}_{\gamma}^\ell
\right)
\bV_{i,0}^k \phi_{i,0}^k(\bmu) \, ,
\label{eq:resGreedyA} \\
\mathbf{R}_{i,j}^{m-1}(\bmu, \bLambda_i^j ) :=&
\sum_{q \in \mathcal{N}_i^j} \Lambda^q_i \mathbf{f}_{\Lambda}^q 
\notag \\
&-
\sum_{k=1}^{m-1} 
\left(
\sum_{\ell=1}^{n_{\nu}} \xi_{\nu}^\ell(\bmu) \mat{K}_{\nu}^\ell
+ \sum_{\ell=1}^{n_{\alpha}} \xi_{\alpha}^\ell(\bmu) \mat{K}_{\alpha}^\ell
+ \sum_{\ell=1}^{n_{\gamma}} \xi_{\gamma}^\ell(\bmu) \mat{K}_{\gamma}^\ell
\right) 
\bV_{i,j}^k \phi_{i,j}^k(\bmu) \psi_{i,j}^k(\bLambda_i^j ) \, .
\label{eq:resGreedyB}
\end{align}
\end{subequations}

In order to make the solution of high-dimensional equations~\eqref{eq:algebraicGreedy} feasible, the encapsulated PGD framework uses a fixed-point iteration scheme, namely the alternating directions algorithm, whose non-intrusive implementation is described in~\cite{Diez:2020:ACME}.
%
In particular, this library employs a purely algebraic formulation of the PGD, relying on the so-called \texttt{separatedTensor} structure, where all information is stored in the form of either matrices or vectors, as detailed in Appendix~\ref{append:Storage}.
%
An example of the setup and computation performed by the encapsulated PGD solver is presented in Appendix~\ref{append:Setup} for the test case in Sect.~\ref{sec:testAnalytical}.


%==========================================================================
\subsection{Encapsulated PGD: solution and data storage}
\label{append:Storage}		
%==========================================================================

Consider a finite element discretization for the spatial subproblems and a pointwise collocation approach for each parameter. For the sake of simplicity, parameters $\mu$ and $\Lambda_i^j$ are assumed to be scalar but this can be straightforwardly generalized to the vectorial case according to~\eqref{eq:vectParam}.
%
More precisely, let $\Nfem$ be the number of unknowns arising from the finite element discretization of the spatial variables defined on $\Omega_i$. Similarly,  $\Nmu$ and $\Nlam$ denote the number of unknowns in the uniform meshes introduced for the parametric domains $\mathcal{P}$ and $\mathcal{Q}_i^j$, respectively.

As previously explained, the computation of the $m$-th mode of the PGD expansion of $\bvpgd_{i,0}$ requires determining the pair $(\bV_{i,0}^m,\phi_{i,0}^m)$, depending on $\bx$ and $\mu$, respectively. Similarly, the triplet $(\bV_{i,j}^m,\phi_{i,j}^m,\psi_{i,j}^m)$ of functions of $\bx$, $\mu$ or $\Lambda_i^j$ needs to be computed for $\bvpgd_{i,j}$.
%
This leads to the spatial modes of the solution being stored in the vectors $\bV_{i,0}^m, \bV_{i,j}^m \in \R^{\Nfem}$ of finite element nodal unknowns in $\Omega_i$, whereas vectors $\bphi_{i,0}^m, \bphi_{i,j}^m \in \R^{\Nmu}$ and $\bpsi_{i,j}^m \in \R^{\Nlam}$ contain the parametric modes of the solution defined for all $\mu \in \mathcal{P}$ and for all $\Lambda_i^j \in \mathcal{Q}_i^j$, respectively.

Similarly, the parametric modes of problem data (i.e.,   $\xi_{\nu}^\ell$, $\xi_{\alpha}^\ell$, $\xi_{\gamma}^\ell$, $\xi_s^\ell$, $\xi_N^\ell$, $\xi_D^\ell$) are discretized for all $\mu \in \mathcal{P}$ and the resulting values are stored in vectors $\bxi_{\nu}^\ell$, $\bxi_{\alpha}^\ell$, $\bxi_{\gamma}^\ell$, $\bxi_s^\ell$, $\bxi_N^\ell$, $\bxi_D^\ell$ of dimension $\Nmu$.
%
Following the same rationale, for all values of $\Lambda_i^j \in \mathcal{Q}_i^j$, the parametric modes of the active boundary parameters are stored in a set of vectors $\bups_{\Lambda}^1, \bups_{\Lambda}^2, \ldots \in \R^{\Nlam}$.


%==========================================================================
\subsection{Encapsulated PGD: setup of the test in Section~\ref{sec:testAnalytical}}
\label{append:Setup}
%==========================================================================

\begin{table}[!ht]
\begin{tabularx}{\textwidth}{lcX}
\hline
\hline\\[-10pt]
Terms in the   & Algebraic   & \\
weak form      & counterpart & 
\\[2pt]
%
\hline
\multicolumn{3}{c}{Spatial modes} \\
\hline\\[-10pt]
%
$\displaystyle \int_{\Omega_i} \nabla v \cdot \nabla \deV \,d\bx$ & $\mat{K}_{\nu}^1$ & $\Nfem \times \Nfem$ finite element stiffness matrix with constant diffusion coefficient $b_{\nu}^1 = 1$
\\
$\displaystyle \int_{\Omega_i} x\,\nabla v \cdot \nabla \deV\,d\bx$ & $\mat{K}_{\nu}^2$ & $\Nfem \times \Nfem$ finite element stiffness matrix with space-dependent diffusion coefficient $b_{\nu}^2 = x$
\\
$\displaystyle \int_{\Omega_i} b_s^1(\bx) \deV\,d\bx$ & $\mathbf{f}_s^1$ & $\Nfem \times 1$ vector arising from finite element discretization of the source term $b_s^1$
\\
$\displaystyle \int_{\Omega_i} b_s^2(\bx) \deV\,d\bx$ & $\mathbf{f}_s^2$ & $\Nfem \times 1$ vector arising from finite element discretization of the source term $b_s^2$
\\
$\displaystyle \int_{\Omega_i} b_s^3(\bx) \deV\,d\bx$ & $\mathbf{f}_s^3$ & $\Nfem \times 1$ vector arising from finite element discretization of the source term $b_s^3$
\\
$\displaystyle \int_{\Omega_i} \nabla \varphi^q \cdot \nabla \deV \, d\bx$ & $\mathbf{f}_{\Lambda}^{q,1}$ & $\Nfem \times 1$ vector arising from finite element imposition of Dirichlet boundary conditions
\\[10pt]
$\displaystyle \int_{\Omega_i} x\,\nabla \varphi^q \cdot \nabla \deV \, d\bx$ & $\mathbf{f}_{\Lambda}^{q,2}$ & $\Nfem \times 1$ vector arising from finite element  imposition of Dirichlet boundary conditions
\\[20pt]
%
\hline
\multicolumn{3}{c}{Parametric modes in $\mu$} \\
\hline\\[-10pt]
%
$\displaystyle 1$ & $\bxi_{\nu}^1, \,  \bxi_{s}^1$ & $\Nmu \times 1$ vector of ones
\\[10pt]
$\displaystyle \mu$ & $\bxi_{\nu}^2, \,  \bxi_{s}^2$ & $\Nmu \times 1$ vector with discrete values of $\mu \in (\muMin,\muMax)$
\\[10pt]
$\displaystyle \mu^2$ & $\bxi_{s}^3$ & $\Nmu \times 1$ vector with discrete values of $\mu^2$ with $\mu \in (\muMin,\muMax)$
\\[20pt]
%
\hline
\multicolumn{3}{c}{Parametric modes in $\Lambda$} \\
\hline\\[-10pt]
%
$1$ & $\bups_{\Lambda}^1$ & $\Nlam \times 1$ vector of ones
\\[10pt]
$\Lambda_i^q$ & $\bups_{\Lambda}^2$ & $\Nlam \times 1$ vector with discrete values of $\Lambda_i^q \in (\LamMin,\LamMax)$
\\[5pt]
\hline
\hline
\end{tabularx}
\caption{Differential and algebraic formulation of the terms in the parametric problem for the setup of the encapuslated PGD solver.}
\label{table:Implementation}
\end{table}

In this appendix, the construction of the \texttt{separatedTensor} structure required by the encapsulated PGD solver is detailed for the test case presented in Sect.~\ref{sec:testAnalytical}.
%
In this context, a diffusion problem with homogeneous Dirichlet boundary conditions is considered. It follows that $\balpha = \bm{0}$, $\gamma = 0$,  $g^D = 0$, while the Neumann datum $g^N$ is redundant.
%
From~\eqref{eq:separatedData}, the separated data features $n_{\nu} = 2$ and $n_s=3$ terms, with the spatial and parametric modes given by
%
\begin{equation}\label{eq:sepDataEx}
\begin{aligned}
\xi_{\nu}^1(\mu) &= 1
\, , \quad
&& b_{\nu}^1(\bx) = 1
\, , \\
\xi_{\nu}^2(\mu) &= \mu
\, , \quad
&& b_{\nu}^2(\bx) = x
\, , \\
\xi_{s}^1(\mu) &= 1
\, , \quad
&& b_{s}^1(\bx) = 8\pi^2\sin(2\pi x)\sin(2\pi y)
\, , \\
\xi_{s}^2(\mu) &= \mu
\, , \quad
&& b_{s}^2(\bx) = 2\pi(4\pi x\sin(2\pi x) - \cos(2\pi x))\sin(2\pi y) - x(x-2) -y(y-1)
\, , \\
\xi_{s}^3(\mu) &= \mu^2
\, , \quad
&& b_{s}^3(\bx) = y(y-1)(1-2x)-x^2(x-2)
\, .
\end{aligned}
\end{equation}

The resulting bilinear form for problems~\eqref{eq:pbApgd} and~\eqref{eq:pbBpgd} is
\begin{equation}\label{eq:bilinearApgdEx}
\Apgd(v, \deV; \mu) 
= \int_{\Omega_i}   \nabla v\cdot \nabla \deV\,d\bx 
+ \mu \int_{\Omega_i}  x \nabla v\cdot \nabla \deV\,d\bx \, ,
\end{equation}
%
while the linear form for problem~\eqref{eq:pbApgd} is
%
\begin{subequations}\label{eq:linearPPGDex}
\begin{equation}\label{eq:linearApgdEx}
\Fpgd_0(\deV; \mu)
=  \int_{\Omega_i} b_s^1(\bx) \deV\,d\bx 
+ \displaystyle \mu \int_{\Omega_i} b_s^2(\bx) \deV\,d\bx 
+ \displaystyle \mu^2 \int_{\Omega_i} b_s^3(\bx) \deV\,d\bx
%- \Apgd \left(0, \deV; \mu \right) 
\, ,
\end{equation}
%
for any value of $\mu \in \mathcal{P}$, and the linear form for problem~\eqref{eq:pbBpgd} is
%
\begin{equation}\label{eq:linearBpgdEx}
\Fpgd_j(\deV; \mu;  \bLambda_i^j ) = 
- \sum_{q \in \mathcal{N}_i^j} \Lambda^q_i  \left( \int_{\Omega_i}   \nabla \varphi^q_i \cdot \nabla \deV\,d\bx 
+ \mu \int_{\Omega_i}  x \nabla \varphi^q_i\cdot \nabla \deV\,d\bx \right) \,  ,
\end{equation}
%
for all $\mu \in \mathcal{P}$ and for all $\bLambda_i^j \in \mathcal{Q}_i^j$.
\end{subequations}

It is worth noticing that the integrals in equation~\eqref{eq:bilinearApgdEx} correspond to the standard finite element matrices for the Poisson equation, the first one with a constant unitary diffusion coefficient and the second one with a space-dependent diffusion equal to $x$. Similarly,  the integrals in~\eqref{eq:linearPPGDex} yield the standard finite element vectors on the right-hand side of the linear system, accounting for the source term and the imposition of the Dirichlet boundary conditions.
%
Table~\ref{table:Implementation} reports a summary of the matrices and vectors required for the construction of the spatial and parametric terms of the \texttt{separatedTensor} structure employed by the encapsulated PGD solver. 

Finally, the setup of the encapsulated PGD solver for problems~\eqref{eq:algebraicGreedy} is briefly presented.
%
Algorithm~\ref{alg:setupA} details the implementation for problem~\eqref{eq:greedyA}, whereas the solver for problem~\eqref{eq:greedyB} is presented in Algorithm~\ref{alg:setupB}. It is worth recalling that the latter problem features the parametric description of the subdomain boundary conditions, with $\mbox{card}(\mathcal{N}_i^j)$ denoting the number of active boundary parameters.
%
\begin{algorithm}[!ht]
\caption{Encapsulated PGD solver for problem~\eqref{eq:greedyA}.}\label{alg:setupA}
\begin{algorithmic}[1]
\REQUIRE{Spatial ($\mat{K}_{\nu}^1, \mat{K}_{\nu}^2, \mathbf{f}_s^1, \mathbf{f}_s^2, \mathbf{f}_s^3$) and parametric ($\bxi_{\nu}^1, \bxi_{\nu}^2, \bxi_{s}^1, \bxi_{s}^2, \bxi_{s}^3$) modes of the linear system, tolerances $\varepsilon$ for the PGD enrichment and $\varepsilon^\star$ for the PGD compression.}
\STATE{Initialization of the separated tensor for the left-hand side of problem: \\
{\small \verb|Ki0 = separatedTensor;|}
}
\STATE{Setup of the spatial modes: \\% for the left-hand side of the problem. \\
{\small \verb|Ki0.sectionalData{1,1} =|} $\mat{K}_{\nu}^1${\small \verb|;|} \\
{\small \verb|Ki0.sectionalData{1,2} =|} $\mat{K}_{\nu}^2${\small \verb|;|}
}
\STATE{Setup of the parametric modes: \\% for the left-hand side of the problem. \\
{\small \verb|Ki0.sectionalData{2,1} =|} $\bxi_{\nu}^1${\small \verb|;|} \\
{\small \verb|Ki0.sectionalData{2,2} =|} $\bxi_{\nu}^2${\small \verb|;|} \\
}
\STATE{Initialization of the separated tensor for the right-hand side of the problem: \\
{\small \verb|Fi0 = separatedTensor;|}
}
\STATE{Setup of the spatial modes: \\% for the right-hand side of the problem. \\
{\small \verb|Fi0.sectionalData{1,1} =|} $\mathbf{f}_s^1${\small \verb|;|} \\
{\small \verb|Fi0.sectionalData{1,2} =|} $\mathbf{f}_s^2${\small \verb|;|} \\
{\small \verb|Fi0.sectionalData{1,3} =|} $\mathbf{f}_s^3${\small \verb|;|} \\
}
\STATE{Setup of the parametric modes: \\% for the right-hand side of the problem. \\
{\small \verb|Fi0.sectionalData{2,1} =|} $\bxi_{s}^1${\small \verb|;|} \\
{\small \verb|Fi0.sectionalData{2,2} =|} $\bxi_{s}^2${\small \verb|;|} \\
{\small \verb|Fi0.sectionalData{2,3} =|} $\bxi_{s}^3${\small \verb|;|} \\
}
\STATE{Solution of the problem via the alternating directions method: \\
{\small \verb|vi0 = pgdLinearSolve(Ki0,Fi0,`tolModes',|}$\varepsilon${\small \verb|});|}
}
\IF{PGD compression is active}
\STATE{Compress the computed PGD solution: \\
{\small \verb|vi0 = pgdCompression(vi0,`tolModes',|}$\varepsilon^\star${\small \verb|});|}
}
\ENDIF
\ENSURE{Solution $\bvpgd_{i,0}$ stored in the form of a \texttt{separatedTensor} structure containing the vectors $\bV_{i,0}^m$ and $\bphi_{i,0}^m$ of the spatial and parametric modes.}
\end{algorithmic}
\end{algorithm}
%
\begin{algorithm}[!ht]
\caption{Encapsulated PGD solver for problem~\eqref{eq:greedyB}.}\label{alg:setupB}
\begin{algorithmic}[1]
\REQUIRE{Spatial ($\mat{K}_{\nu}^1, \mat{K}_{\nu}^2, \mathbf{f}_\Lambda^{q, 1}, \mathbf{f}_\Lambda^{q, 2}$) and parametric ($\bxi_{\nu}^1, \bxi_{\nu}^2, \bups_{\Lambda}^1, \bups_{\Lambda}^2$) modes of the linear system, tolerances $\varepsilon$ for the PGD enrichment and $\varepsilon^\star$ for the PGD compression.}
\STATE{Initialization of the separated tensor for the left-hand side of problem: \\
{\small \verb|Kij = separatedTensor;|}
}
\STATE{Setup of the spatial modes: \\% for the left-hand side of the problem. \\
{\small \verb|Kij.sectionalData{1,1} =|} $\mat{K}_{\nu}^1${\small \verb|;|} \\
{\small \verb|Kij.sectionalData{1,2} =|} $\mat{K}_{\nu}^2${\small \verb|;|}
}
\STATE{Setup of the parametric modes depending on the physical parameter $\mu$: \\% for the left-hand side of the problem. \\
{\small \verb|Kij.sectionalData{2,1} =|} $\bxi_{\nu}^1${\small \verb|;|} \\
{\small \verb|Kij.sectionalData{2,2} =|} $\bxi_{\nu}^2${\small \verb|;|} \\
}
\FOR{{\small \texttt{q = 1:}}card$(\mathcal{N}_i^j)$}
\STATE{Setup of the parametric modes depending on the active boundary parameters $\bLambda_i^j$: \\% for the left-hand side of the problem. \\
{\small \verb|Kij.sectionalData{2+q,1} =|} $\bups_{\Lambda}^1${\small \verb|;|} \\
{\small \verb|Kij.sectionalData{2+q,2} =|} $\bups_{\Lambda}^1${\small \verb|;|} \\
}
\ENDFOR
\STATE{Initialization of the separated tensor for the right-hand side of the problem: \\
{\small \verb|Fij = separatedTensor;|}
}
\FOR{{\small \texttt{q = 1:}}card$(\mathcal{N}_i^j)$}
\STATE{Setup of the spatial modes: \\% for the right-hand side of the problem. \\
{\small \verb|Fij.sectionalData{1,2*q-1} =|} $\mathbf{f}_\Lambda^{q, 1}${\small \verb|;|} \\
{\small \verb|Fij.sectionalData{1,2*q}   =|} $\mathbf{f}_\Lambda^{q, 2}${\small \verb|;|}
}
\STATE{Setup of the parametric modes depending on the physical parameter $\mu$: \\% for the right-hand side of the problem. \\
{\small \verb|Fij.sectionalData{2,2*q-1} =|} $\bxi_{\nu}^1${\small \verb|;|} \\
{\small \verb|Fij.sectionalData{2,2*q}   =|} $\bxi_{\nu}^2${\small \verb|;|}
}
\FOR{{\small \texttt{r = 1:}}card$(\mathcal{N}_i^j)$}
\STATE{Setup of the parametric modes depending on the active boundary parameters $\bLambda_i^j$:}% for the right-hand side of the problem.}
\IF{{\small \texttt{r == q}}}
\STATE{
{\small \verb|Fij.sectionalData{2+r,2*q-1} =|} $\bups_{\Lambda}^2${\small \verb|;|} \\
{\small \verb|Fij.sectionalData{2+r,2*q}   =|} $\bups_{\Lambda}^2${\small \verb|;|}
}
\ELSE
\STATE{
{\small \verb|Fij.sectionalData{2+r,2*q-1} =|} $\bups_{\Lambda}^1${\small \verb|;|} \\
{\small \verb|Fij.sectionalData{2+r,2*q}   =|} $\bups_{\Lambda}^1${\small \verb|;|}
}
\ENDIF
\ENDFOR
\ENDFOR
\STATE{Solution of the problem via the alternating directions method: \\
{\small \verb|vij = pgdLinearSolve(Kij,Fij,`tolModes',|}$\varepsilon${\small \verb|});|}
}
\IF{PGD compression is active}
\STATE{Compress the computed PGD solution: \\
{\small \verb|vij = pgdCompression(vij,`tolModes',|}$\varepsilon^\star${\small \verb|});|}
}
\ENDIF
\ENSURE{Solution $\bvpgd_{i,j}$ stored in the form of a \texttt{separatedTensor} structure containing the vectors $\bV_{i,j}^m$, $\bphi_{i,j}^m$ and $\bpsi_{i,j}^m$ of the spatial and parametric modes.}
\end{algorithmic}
\end{algorithm}