\section{Surrogate-based overlapping Schwarz method}
\label{sec:online}

In this section,  an efficient strategy to construct the global solution of problem~\eqref{eq:globalProb} for a fixed set of parametric values $\bar{\bmu} \in \mathcal{P}$ is presented.
%
The goal is to devise a procedure, suitable for real time execution, to appropriately glue the parametric solutions of the local subproblems, thus drastically reducing the cost of the overall DD algorithm.
%
To this end, the overlapping Schwarz method presented in Sect.~\ref{sec:schwarzAlg} is adapted to exploit the local PGD surrogate models constructed in Sect.~\ref{sec:offline}.

\begin{rem}
To easily perform the coupling between subdomains $\Omega_1$ and $\Omega_2$ in the online phase and to avoid expensive interpolation procedures among different grids, the meshes used for the spatial discretization of the local subdomains are assumed to be conforming with the interfaces (i.e., the interfaces do not cut through any elements of the meshes) and to coincide in the overlapping region $\Omega_{12}$.
\end{rem}

First, note that by construction (see~\eqref{eq:lambdaRep}),  the vector $\mathbf{u}_{\Gamma_i}\!(\bar{\bmu})$ of the nodal values of the solution at $\Gamma_i$ corresponds to the vector of parameters $\bLambda_i$. Hence, for any $\bar{\bmu} \in \mathcal{P}$, equation~\eqref{eq:SchwarzAlg} can be rewritten as
%
\begin{equation}\label{eq:SchwarzAlgLambda}
\mat{I}_{\Gamma_j}\bLambda_j
=
\mat{R}_{\Omega_i\to\Gamma_j} \mat{A}_{\Omega_i}^{-1} \mathbf{f}_{\Omega_i}\!(\bar{\bmu}) 
+ \mat{R}_{\Omega_i\to\Gamma_j} \mat{A}_{\Omega_i}^{-1} (-\mat{A}_{\Gamma_i}\bLambda_i )\, ,
\end{equation}
%
where $\mat{A}_{\Omega_i}^{-1} \mathbf{f}_{\Omega_i}\!(\bar{\bmu})$ corresponds to the solution of problem~\eqref{eq:sourceProb} and $\mat{A}_{\Omega_i}^{-1} (-\mat{A}_{\Gamma_i}\bLambda_i)$ denotes the solution of problem~\eqref{eq:boundaryProbs}.
% 
Exploiting the local surrogate models constructed in the offline phase, equation~\eqref{eq:SchwarzAlgLambda} reduces to
%
\begin{equation}\label{eq:SchwarzAlgPGD}
\mat{I}_{\Gamma_j}\bLambda_j
=
\mat{R}_{\Omega_i\to\Gamma_j} \bupgd_{i,0}\!(\bar{\bmu}) 
+ \mat{R}_{\Omega_i\to\Gamma_j} \bupgd_{i,\Lambda}\!(\bar{\bmu},\bLambda_i)  \, ,
\end{equation}
%
where $\bupgd_{i,0}\!(\bar{\bmu})$ and $\bupgd_{i,\Lambda}\!(\bar{\bmu},\bLambda_i)$ denote the vectors of the nodal values of the PGD solutions $\upgd_{i,0}$ and $\upgd_{i,\Lambda}$, respectively, evaluated for the target values $\bar{\bmu} \in \mathcal{P}$ and $\bLambda_i \in \mathcal{Q}_i$.

Let $\mat{A}_{i,\Lambda_i}^\texttt{PGD}$ be the local PGD operator 
%
\begin{equation}\label{eq:PGDlambdaOper}
\mat{A}_{i,\Lambda_i}^\texttt{PGD} : \bLambda_i \to \bupgd_{i,\Lambda}(\bar{\bmu},\bLambda_i) \, 
\end{equation}
%
that, given a set of boundary parameters $ \bLambda_i$, returns the nodal values of the PGD surrogate model $\upgd_{i,\Lambda}$ of problem~\eqref{eq:boundaryProbs} for the set of parameters $\bar{\bmu}$. Hence, equation \eqref{eq:SchwarzAlgPGD} can be rewritten as
\begin{equation}\label{eq:SchwarzAlgPGD_Op}
\mat{I}_{\Gamma_j}\bLambda_j
=
\mat{R}_{\Omega_i\to\Gamma_j} \bupgd_{i,0}\!(\bar{\bmu}) 
+ \mat{R}_{\Omega_i\to\Gamma_j} \mat{A}_{i, \Lambda_i}^\texttt{PGD} \bLambda_i  \, ,    
\end{equation}
%
and the surrogate-based overlapping Schwarz method is finally obtained by rewriting the interface system~\eqref{eq:systemSchwarzInterface} as
%
\begin{equation}\label{eq:systemSchwarzInterfacePGD}
\begin{pmatrix}
\mat{I}_{\Gamma_1} & -\mat{R}_{\Omega_2\to\Gamma_1} \mat{A}_{2,\Lambda_2}^\texttt{PGD} \\[3pt]
-\mat{R}_{\Omega_1\to\Gamma_2} \mat{A}_{1,\Lambda_1}^\texttt{PGD} & \mat{I}_{\Gamma_2}
\end{pmatrix}
\begin{pmatrix}
\bLambda_1 \\[3pt]
\bLambda_2
\end{pmatrix}
=
\begin{pmatrix}
\mat{R}_{\Omega_2\to\Gamma_1} \bupgd_{2,0}\!(\bar{\bmu}) \\[3pt]
\mat{R}_{\Omega_1\to\Gamma_2} \bupgd_{1,0}\!(\bar{\bmu})
\end{pmatrix} \, .
\end{equation}

Therefore, the online phase of the method consists of an iterative strategy to solve equation~\eqref{eq:systemSchwarzInterfacePGD}, e.g., by GMRES.  At convergence, say, at iteration $k=k^*$,  the approximation of the solution of the global problem~\eqref{eq:globalProb} for $\bar{\bmu} \in \mathcal{P}$ is thus given by
%
\begin{equation}\label{eq:globalSolnA}
	\bupgd(\bar{\bmu}) = 
	\begin{cases}
		\bupgd_{1,0}(\bar{\bmu}) + 
		\bupgd_{1,\Lambda}(\bar{\bmu},\bLambda_1^{(k^*)} ) \quad \text{in } \Omega_1,\\
		\noalign{\vskip5pt}
		\bupgd_{2,0}(\bar{\bmu}) + 
		\bupgd_{2,\Lambda}(\bar{\bmu},\bLambda_2^{(k^*)} ) \quad \text{in } \Omega_2 \setminus \Omega_{12}.
	\end{cases}
\end{equation}

It is worth noticing that the values $\bLambda_i^{(k)}$ computed by GMRES iterations to solve problem~\eqref{eq:systemSchwarzInterfacePGD} may not coincide with any of the values obtained from the discretization of the parametric domain $\mathcal{Q}_i$.  
%
If this is the case, the solution $\bupgd_{i,\Lambda}(\bar{\bmu},\bLambda_i^{(k)})$ provided by the operator $\mat{A}_{i,\Lambda}^\texttt{PGD}$ is obtained by performing a linear interpolation of the parametric modes depending on $\bLambda_i$ and associated with the available values closest to $\bLambda_i^{(k)}$.

\begin{rem}
Following Algorithm~\ref{alg:Schwarz}, at the beginning of the online phase,  an instance $\bar{\bmu}$ is selected in the set of parameters $\mathcal{P}$.
%
This is not strictly necessary and the described algorithm can be adapted to handle arbitrary parameters $\bmu$. 
%
In the latter case, at the end of the online phase, one would obtain a global surrogate model that represents a family of solutions of the global problem~\eqref{eq:globalProb} depending on $\bmu \in \mathcal{P}$, instead of an instance of such model for $\bmu = \bar{\bmu}$. 
\end{rem}