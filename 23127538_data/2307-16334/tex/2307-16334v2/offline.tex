\section{Local surrogate models using proper generalized decomposition}
\label{sec:offline}

In this section, the procedure to construct the PGD local surrogate models in the offline phase is presented.
%
The local parametric problem to be solved in the generic subdomain $\Omega_i$ is: for all $\bmu \in \mathcal{P}$, find $u_i(\bmu)$ such that 
%
\begin{equation}
	\label{eq:localProb}
	\begin{array}{rcll}
		L_{}({u}_{i}(\bmu); \bmu) &=& s_i(\bmu) & \quad \text{in } \Omega_i\,,\\
		u_i(\bmu) &=& g^D_i(\bmu) & \quad \text{on } \Gamma_i^D,\\
		\Neum{u_i(\bmu)}{} &=& g^N_i(\bmu) & \quad \text{on } \Gamma_i^N,\\
		u_i(\bmu) &=& \lambda_i & \quad \text{on } \Gamma_i \, ,
	\end{array}
\end{equation}
%
for arbitrary Dirichlet data $\lambda_i$ at the interface $\Gamma_i$, where $\lambda_i=\lambda_i(\bx)$ is a space-dependent function. 

The arbitrariness of the boundary function $\lambda_i$ is dealt with by an appropriate parametrization.
%
Considering that problem \eqref{eq:localProb} is solved in a finite dimensional context, e.g., by the finite element method, the boundary function $\lambda_i$ can be expressed as a linear combination of suitable basis functions on $\Gamma_i$, say, $\eta^q_i(\bx), \ q=1,\ldots,N_{\Gamma_i}$, with coefficients $\Lambda^q_i$:
%
\begin{equation}\label{eq:lambdaRep}
 \lambda_i = \lambda_i(\bx) = \sum_{q = 1}^{N_{\Gamma_i}} \Lambda^q_i \, \eta^q_i(\bx) \, .
\end{equation}
%
For example, upon introducing a finite element space of continuous piecewise polynomial functions in $\Omega_i$, if $\varphi^q_i(\bx)$ are the finite element basis functions with non-null support at $\Gamma_i$, one can choose $\eta^q_i(\bx)$ to be the restriction of $\varphi^q_i(\bx)$ to $\Gamma_i$, i.e., $\eta^q_i(\bx) = \varphi^q_i(\bx)|_{\Gamma_i}$. Note that, while this is the approach used in the present work, other suitable bases can be considered on $\Gamma_i$. The dependence of the basis functions upon space is henceforth omitted, unless in the case of ambiguity.

The arbitrary coefficients $\bLambda_i = (\Lambda^1_i, \ldots, \Lambda^{N_{\Gamma_i}}_i) \in \mathcal{Q}_i$ thus become additional parameters of the local problem \eqref{eq:localProb}, with values in $ \mathcal{Q}_i = \mathcal{J}_i^1 \times \dots \times \mathcal{J}_i^{N_{\Gamma_i}}$,  where each $\mathcal{J}_i^q \subset \mathbb{R}$, $q = 1, \dots , N_{\Gamma_i}$ is a compact set.

\smallskip  

\begin{rem}
The sets of admissible boundary values introduced above need to be appropriately selected to ensure that the linear combination~\eqref{eq:lambdaRep} can approximate the trace $\lambda_i$ of the solution for all parameters $\bmu \in \mathcal{P}$. Therefore, the choice of the minimum and maximum values of $\mathcal{J}_i^q$ depends on the parameters $\bmu$.
\end{rem}

\smallskip

Although the introduction of the subdomains $\Omega_i$ in the DD procedure in Algorithm~\ref{alg:Schwarz} allows to work locally with a reduced number of spatial and parametric degrees of freedom, the parametrization of the boundary condition along the interface $\Gamma_i$ in the local problem~\eqref{eq:localProb} leads to a growth of the dimensionality of the local parametric problem, namely by introducing $N_{\Gamma_i}$ new dimensions, each associated with a coefficient $\Lambda^q_i, \ q = 1, \dots , N_{\Gamma_i}$. Unfortunately, it is well known that if $N_{\Gamma_i} \gg 1$ the solution of the local problem~\eqref{eq:localProb} with parametrized data~\eqref{eq:lambdaRep} might become unfeasible.

To overcome this difficulty, the linearity of the operator $L$ is exploited and the local problem~\eqref{eq:localProb} is split into a family of $N_i$ subproblems, each involving a \emph{sufficiently small} set of parameters $\mathcal{N}_i^j$, gathering the so-called \emph{active boundary parameters} (see Fig.~\ref{fig:activeBdryNodes}). More precisely,  $\{\mathcal{N}_i^j\}_{j=1,\ldots,N_i}$ denotes a disjoint partition of the set of indices $1,\ldots,N_{\Gamma_i}$ such that $\text{card}(\mathcal{N}_i^j) \ll N_{\Gamma_i}$, for all $j$.  Hence, the coefficients $\bLambda_i$ employed to characterize the trace functions can be split into subsets $\bLambda_i^j = (\Lambda_i^q)_{q \in \mathcal{N}_i^j} \in \mathcal{Q}_i^j$, with $\mathcal{Q}_i^j = \bigtimes_{q \in \mathcal{N}_i^j} \mathcal{J}_i^q \subset \mathcal{Q}_i$, and equation~\eqref{eq:lambdaRep} is rewritten as
%
\begin{equation}\label{eq:splittingBoundaryParameters}
\lambda_i = 
\sum_{q \in \mathcal{N}^1_i} \Lambda^q_i \, \eta^q_i + 
\sum_{q \in \mathcal{N}^2_i} \Lambda^q_i \, \eta^q_i + \ldots +
\sum_{q \in \mathcal{N}^{N_i}_i} \Lambda^q_i \, \eta^q_i .
\end{equation}

%
%
% Figure environment removed
%
%

The solution of the local problem~\eqref{eq:localProb} is thus expressed in terms of both the problem parameters $\bmu \in \mathcal{P}$ and the active boundary parameters $\bLambda_i^j \in \mathcal{Q}_i^j, \ j=1,\ldots,N_i$. By linearity, for all $\bmu \in \mathcal{P}$, the resulting solution is given by
%
\begin{equation}\label{eq:solutionSplit}
u_i(\bmu, \bLambda_i) = u_{i,0} (\bmu) + \sum_{j=1}^{N_i} u_{i,j}(\bmu, \bLambda_i^j) ,
\end{equation}
%
where $u_{i,0} (\bmu)$ satisfies the equation
%
\begin{subequations}\label{eq:subProb}
\begin{equation}\label{eq:sourceProb}
\begin{array}{rcll}
L({u}_{i, 0}(\bmu); \bmu) &=& s_i(\bmu) & \quad \text{in } \Omega_i,\\
u_{i, 0}(\bmu) &=& g^D_i(\bmu) & \quad \text{on } \Gamma_i^D,\\
\Neum{u_{i, 0}(\bmu)}{} &=& g^N_i(\bmu) &\quad \text{on } \Gamma_i^N,\\
u_{i, 0}(\bmu) &=& 0 & \quad \text{on } \Gamma_i ,
\end{array}
\end{equation}
%
whereas each $u_{i, j}(\bmu, \bLambda_i^j)$, with $j=1,\ldots,N_i$, is solution of
%
\begin{equation}\label{eq:boundaryProbs}
\begin{array}{rcll}
L(u_{i, j}(\bmu, \bLambda_i^j); \, \bmu) &=& 0 &\quad \text{in } \Omega_i,\\
u_{i, j}(\bmu, \bLambda_i^j) &=& 0 &\quad \text{on } \Gamma_i^D,\\
\Neum{u_{i, j}(\bmu, \bLambda_i^j)}{} &=& 0 & \quad \text{on } \Gamma_i^N,\\
u_{i, j}(\bmu, \bLambda_i^j) &=& \displaystyle\sum\limits_{q \in \mathcal{N}_i^j} \Lambda^q_i\,\eta^q_i & \quad \text{on } \Gamma_i ,
\end{array}
\end{equation}
%
for all $\bLambda_i^j \in \mathcal{Q}_i^j$.
\end{subequations}


%==========================================================================
\subsection{Separated representation of data and local solutions}
%==========================================================================

For the sake of readability and without any loss of generality, in this section the subindex $i$ is omitted in the description of problem data in each subdomain: for instance, the Dirichlet datum $g^D(\bmu)$ is employed to seamlessly describe $g^D_i(\bmu)$ on $\Gamma_i^D$ for any subdomain $\Omega_i$.

\smallskip

In order to construct a PGD approximation of problems~\eqref{eq:subProb}, data are assumed to be given in separated form, that is,
%
\begin{equation}\label{eq:separatedData}
\begin{array}{c}
 \nu = \displaystyle\sum_{\ell=1}^{n_{\nu}} \xi_{\nu}^\ell(\bmu)b_{\nu}^\ell(\bx) \,, \qquad
 \balpha = \displaystyle\sum_{\ell=1}^{n_{\alpha}} \xi_{\alpha}^\ell(\bmu) \bb_{\alpha}^\ell(\bx) \,, \qquad
 \gamma = \displaystyle\sum_{\ell=1}^{n_{\gamma}} \xi_{\gamma}^\ell(\bmu) b_{\gamma}^\ell(\bx) \,, \\
 s = \displaystyle\sum_{\ell=1}^{n_s} \xi_s^\ell(\bmu) b_s^\ell(\bx) \,, \qquad
  g^N = \displaystyle\sum_{\ell=1}^{n_N}  \xi_N^\ell(\bmu) b_N^\ell(\bx) ,
\end{array}
\end{equation}
%
where each term of the expressions~\eqref{eq:separatedData} is the product of a function depending on the spatial coordinate $\bx$ and a function of the parameters $\bmu$. Moreover, the parametric modes are assumed to be the product of one-dimensional functions of the parameters $\mu^1,\ldots,\mu^P$, e.g.,
%
\begin{equation}\label{eq:vectParam}
\xi_{\nu}^\ell(\bmu) = \prod_{p=1}^P \xi_{\nu,p}^\ell(\mu^p) .
\end{equation}
%
Although data are not directly given in the form~\eqref{eq:separatedData}, it is possible to numerically construct a good approximation in a separated form, see~\cite{DM-MZH:15}.
In addition, the PGD rationale assumes that the solutions $u_{i, 0}$ and $u_{i, j}$ of the local subproblems can be written in separated form. 

Consider the Hilbert space
\begin{equation*}
    \mathcal{V}_i = \{w_{i} \in {H}^1(\Omega_i) \, : \, w_{i} = 0 \text{ on }\partial \Omega_i \setminus \Gamma_i^N\}\,.
\end{equation*}
The solution of the local subproblem~\eqref{eq:sourceProb} depends only on space and on the parameters $\bmu$, and it can be written as
%
\begin{equation}\label{eq:solProbA}
 u_{i, 0}(\bmu) = v_{i, 0}(\bmu) + G^D(\bmu) ,
\end{equation}
%
where $G^D(\bmu) \in H^1(\Omega_i)$ is a suitable extension of the boundary datum $g^D(\bmu)$ at $\Gamma_i^D$ such that $G^D(\bmu) = g^D(\bmu)$ at $\Gamma_i^D$ and $G^D(\bmu) = 0$ at $\Gamma_i$. Similarly to~\eqref{eq:separatedData}, also the function $G^D(\bmu)$ can be written in separated form as
\begin{equation}\label{eq:separatedDataGD}
G^D = \displaystyle\sum_{\ell=1}^{n_D} \xi_D^\ell(\bmu) b_D^\ell(\bx) \, .
\end{equation}

By construction, it follows that $v_{i, 0}(\bmu) \in \mathcal{V}_i$, for all $\bmu \in \mathcal{P}$.  Similarly, the solution of each subproblem~\eqref{eq:boundaryProbs} for $j=1,\ldots,N_i$ depends both on the parameters $\bmu$ and on the active boundary parameters $\bLambda^j_i$ at the interface, and it can be expressed as
%
\begin{equation}\label{eq:solProbB}
u_{i, j}(\bmu, \bLambda_i^j) = v_{i, j}(\bmu, \bLambda_i^j) + \sum_{q \in \mathcal{N}_i^j} \Lambda^q_i\,\varphi^q_i ,
\end{equation}
%
with $v_{i, j}(\bmu, \bLambda_i^j) \in \mathcal{V}_i$,  for all $\bmu \in \mathcal{P}$ and for all $\bLambda_i^j \in \mathcal{Q}_i^j$. 

Following the standard procedure in PGD~\cite{Chinesta:2014}, the contributions of Dirichlet boundary conditions are handled by introducing \emph{ad-hoc}, sufficiently smooth modes.  The remaining terms $v_{i, 0}(\bmu)$ and $v_{i, j}(\bmu, \bLambda_i^j)$ are computed with homogeneous Dirichlet data, under the assumption of a separated representation of all the variables, that is, $\bx$ and $\bmu$ for $v_{i,0}$, and $\bx$, $\bmu$ and $\bLambda_i^j$ for $v_{i,j}$. This yields the PGD expansions
%
\begin{subequations}\label{eq:nonNormalisedPGD}
\begin{align}
    v_{i,0} \approx \vpgd_{i,0} &= \sum_{m=1}^{M_0} {V}_{i,0}^m(\bx) {\phi}_{i,0}^m(\bmu) \,  ,  \label{eq:nonNormalisedPGDA} \\
    v_{i,j} \approx \vpgd_{i,j}  &= \sum_{m=1}^{M_j} {V}_{i,j}^m(\bx) {\phi}_{i,j}^m(\bmu) {\psi}_{i,j}^m(\bLambda_i^j) \, , \label{eq:nonNormalisedPGDB}
\end{align}
\end{subequations}
%
where ${V}_{i,0}^m$ and ${V}_{i,j}^m$ are the $m$-th spatial modes,  whereas ${\phi}_{i,0}^m$, ${\phi}_{i,j}^m$ and ${\psi}_{i,j}^m$ denote the corresponding parametric modes. It is worth noticing that the numbers of modes $M_0$ and $M_j$ are \emph{a priori} unknown and are automatically determined by a greedy procedure, see~\cite{Diez:2020:ACME}.

In the following Sects.~\ref{sect:localPGD} and \ref{sec:algebraic}, the strategy to compute \eqref{eq:nonNormalisedPGD} is presented. The result is then employed to construct $\upgd_{i, 0}(\bmu)$ and $\upgd_{i, j}(\bmu,\bLambda_i^j)$ according to~\eqref{eq:solProbA} and \eqref{eq:solProbB}. The former is the surrogate model of the data-dependent parametric problem~\eqref{eq:sourceProb}, whereas the latter are employed to define the surrogate model $\upgd_{i,\Lambda}(\bmu, \bLambda_i)$ associated with the boundary parameters, namely,
%
\begin{equation}\label{eq:surrogateBdry}
\upgd_{i,\Lambda}(\bmu, \bLambda_i) = \sum_{j=1}^{N_i} \upgd_{i,j}(\bmu, \bLambda_i^j) \, .
\end{equation}
%
Finally, the complete surrogate model for subdomain $\Omega_i$ is obtained from~\eqref{eq:solutionSplit} as
%
\begin{equation}\label{eq:pgdFINALsol}
\upgd_i(\bmu, \bLambda_i) = \upgd_{i,0} (\bmu) + \upgd_{i,\Lambda}(\bmu, \bLambda_i) \, .
\end{equation}

\begin{rem}
The surrogate models $\upgd_{i,j}(\bmu, \bLambda_i^j)$ feature different supports $\mathcal{Q}_i^j$ in the space of parameters $\bLambda_i^j$. 
%
Hence, for the summation on the right-hand side of equation~\eqref{eq:surrogateBdry} to be well-defined, each surrogate model associated with the active boundary parameters needs to be appropriately extended to have support on the entire parametric space $\mathcal{Q}_i$.
%
This can be straightforwardly achieved in the framework of PGD approximations by defining the modal functions for the \emph{inactive} boundary parameters $\Lambda_i^q$, $q \not\in \mathcal{N}_i^j$ to be constant and equal to $1$.
\end{rem}


%==========================================================================
\subsection{Parametric weak form of the local subproblems}
\label{sect:localPGD}
%==========================================================================
A continuous Galerkin finite element strategy is employed to construct the PGD approximations of the solutions of the local subproblems. To this end, the weak forms of the parametric problems~\eqref{eq:subProb} are first presented.

For all $v,\deV \in \mathcal{V}_i$ and for all $\bmu \in \mathcal{P}$, let $\mathcal{A}$ be the bilinear form
\begin{equation}\label{eq:bilinearA}
\mathcal{A}(v, \deV; \bmu) 
= \int_{\Omega_i} \nu(\bmu)\nabla v\cdot \nabla \deV\,d\bx 
+ \int_{\Omega_i} \balpha(\bmu) {\cdot} \nabla v \, \deV \, d\bx
+ \int_{\Omega_i} \gamma(\bmu) \, v\, \deV \,d\bx \, .
\end{equation}

The parametric weak form of problem~\eqref{eq:sourceProb} becomes: find $v_{i, 0}(\bmu) \in \mathcal{V}_i$ such that
%
\begin{subequations}\label{eq:pbA}
\begin{equation}\label{eq:varfA}
 \mathcal{A}(v_{i,0}(\bmu), \deV; \bmu) = \mathcal{F}_0(\deV; \bmu)\qquad \forall \deV \in \mathcal{V}_i\, \text{ and } \forall \bmu \in \mathcal{P} \, ,
\end{equation}
%
where
%
\begin{equation}\label{eq:linearA}
\mathcal{F}_0(\deV; \bmu)
= \int_{\Omega_i} s(\bmu) \deV\,d\bx
+ \int_{\Gamma_i^N} g^N(\bmu) \deV\, d\bx
- \mathcal{A}(G^D(\bmu), \deV; \bmu) 
\, .
\end{equation}
\end{subequations}

In a similar fashion,  for each problem~\eqref{eq:boundaryProbs} for $j=1,\ldots,N_i$, the parametric weak formulation is: find $v_{i, j}(\bmu, \bLambda_i^j) \in \mathcal{V}_i$ such that
%
\begin{subequations}\label{eq:pbB}
\begin{equation}\label{eq:varfB}
	\mathcal{A}(v_{i,j}(\bmu, \bLambda_i^j), \deV; \bmu) = \mathcal{F}_j(\deV; \bmu; \bLambda_i^j )\quad \forall \deV \in \mathcal{V}_i \, , \forall \bmu \in \mathcal{P} \, \text{ and } \forall \bLambda_i^j \in \mathcal{Q}_i^j \, ,
\end{equation}
%
with
%
\begin{equation}\label{eq:linearB}
\mathcal{F}_j(\deV; \bmu; \bLambda_i^j ) = 
- \mathcal{A} \left(\sum_{q \in \mathcal{N}_i^j} \! \Lambda^q_i\,\varphi^q_i, \deV; \bmu \right) \, .
\end{equation}
%
\end{subequations}


%==========================================================================
\subsection{Parametric linear systems}
\label{sec:algebraic}
%==========================================================================
Under the assumption of an affine parameter dependence of the bilinear form $\mathcal{A}$ and of the linear forms $\mathcal{F}_0$ and $\mathcal{F}_j$ (see, e.g.,~\cite{Rozza:14}), the separated approximations~\eqref{eq:nonNormalisedPGD} are constructed using a greedy approach~\cite{Chinesta:2014}. In particular,  the non-intrusive implementation provided by the encapsulated PGD solver~\cite{Diez:2020:ACME} is employed.

This approach relies on rewriting the local problems~\eqref{eq:pbA} and~\eqref{eq:pbB} in algebraic form, as parametric linear systems. To this end,  the separated representation of data~\eqref{eq:separatedData} is substituted in the parametric weak forms~\eqref{eq:varfA} and~\eqref{eq:varfB} and the unknown solutions $v_{i,0}(\bmu)$ and $v_{i,j}(\bmu, \bLambda_i^j)$ are replaced by the their corresponding PGD approximations $\vpgd_{i,0}$ and $\vpgd_{i,j}$, see~\eqref{eq:nonNormalisedPGD}.

For all $v,\deV \in \mathcal{V}_i$ and for all $\bmu \in \mathcal{P}$, let $\Apgd$ be the bilinear form
\begin{equation}\label{eq:bilinearApgd}
\begin{array}{rcl}
\Apgd(v, \deV; \bmu) 
&=& \displaystyle \sum_{\ell=1}^{n_{\nu}} \xi_{\nu}^\ell(\bmu) \int_{\Omega_i}  b_{\nu}^\ell(\bx) \nabla v\cdot \nabla \deV\,d\bx \\[3pt]
&& \displaystyle + \sum_{\ell=1}^{n_{\alpha}} \xi_{\alpha}^\ell(\bmu) \int_{\Omega_i} \bb_{\alpha}^\ell(\bx) {\cdot} \nabla v \,  \deV \, d\bx\\[3pt]
&& \displaystyle + \sum_{\ell=1}^{n_{\gamma}} \xi_{\gamma}^\ell(\bmu) \int_{\Omega_i} b_{\gamma}^\ell(\bx) v \,  \deV \,d\bx \, .
\end{array}
\end{equation}

The PGD solution $\vpgd_{i,0}$ of problem~\eqref{eq:pbA} is computed by solving the parametric equation
%
\begin{subequations}\label{eq:pbApgd}
\begin{equation}\label{eq:varfApgd}
 \Apgd(\vpgd_{i,0}, \deV; \bmu) = \Fpgd_0(\deV; \bmu)\qquad \forall \deV \in \mathcal{V}_i\, \text{ and } \forall \bmu \in \mathcal{P} \, ,
\end{equation}
%
with
%
\begin{equation}\label{eq:linearApgd}
\begin{array}{rcl}
\Fpgd_0(\deV; \bmu)
&=& \displaystyle \sum_{\ell=1}^{n_s} \xi_s^\ell(\bmu) \int_{\Omega_i} b_s^\ell(\bx) \deV\,d\bx \\[3pt]
&& \displaystyle + \sum_{\ell=1}^{n_N} \xi_N^\ell(\bmu) \int_{\Gamma_i^N} b_N^\ell(\bx) \deV\, d\bx \\[3pt]
&& \displaystyle - \Apgd \left(\,\sum_{\ell=1}^{n_D} \xi_D^\ell(\bmu) b_D^\ell(\bx), \deV; \bmu \right) \, .
\end{array}
\end{equation}
\end{subequations}

The PGD approximation~\eqref{eq:nonNormalisedPGDA} is constructed using a continuous Galerkin finite element discretization for each spatial mode $V_{i,0}^m(\bx)$ and a pointwise collocation approach for the parametric modes $\phi_{i,0}^m(\bmu)$. More precisely,  a finite element mesh is introduced in each subdomain $\Omega_i$ and a spatial polynomial approximation $\mathbb{Q}_r$ of degree $r \geq 1$ is selected, with basis functions $\varphi_i^n$, $n=1,\ldots,\Nfem_i$. It follows that each spatial mode can be written as
%
\begin{equation}\label{eq:spatialMode}
V_{i,0}^m(\bx) = \sum_{n=1}^{\Nfem_i} V_{i,0}^{m,n} \varphi_i^n(\bx) \, ,
\end{equation}
%
where the coefficients $V_{i,0}^{m,n}$, $n=1,\ldots,\Nfem_i$ determine the vector of spatial finite element unknowns $\mathbf{V}_{i,0}^m$. 
%
Therefore, the integrals appearing in the bilinear and linear forms~\eqref{eq:bilinearApgd} and~\eqref{eq:linearApgd} give rise to standard finite element matrices and vectors, appropriately weighted by means of parametric functions stemming from the separated form of data~\eqref{eq:separatedData} and~\eqref{eq:separatedDataGD}.

The resulting parametric linear system for problem~\eqref{eq:sourceProb} is
%
\begin{equation}\label{eq:algebraicA}
\begin{aligned}
\left(
\sum_{\ell=1}^{n_{\nu}} \xi_{\nu}^\ell(\bmu) \mat{K}_{\nu}^\ell
+ \sum_{\ell=1}^{n_{\alpha}} \xi_{\alpha}^\ell(\bmu) \mat{K}_{\alpha}^\ell
+ \sum_{\ell=1}^{n_{\gamma}} \xi_{\gamma}^\ell(\bmu) \mat{K}_{\gamma}^\ell
\right) &
\bvpgd_{i,0}(\bmu) \\
=
\sum_{\ell=1}^{n_s} \xi_s^\ell(\bmu) \mathbf{f}_s^\ell
+ \sum_{\ell=1}^{n_N} & \xi_N^\ell(\bmu) \mathbf{f}_N^\ell
+ \sum_{\ell=1}^{n_D} \xi_D^\ell(\bmu) \mathbf{f}_D^\ell
\qquad \forall \bmu \in \mathcal{P} \, ,
\end{aligned}
\end{equation}
%
where the PGD separated solution is defined as 
%
\begin{equation}\label{eq:algebraicSolA}
\bvpgd_{i,0}(\bmu) = \sum_{m=1}^{M_0} \mathbf{V}_{i,0}^m \, \phi_{i,0}^m(\bmu) \, ,
\end{equation}
%
whereas $ \mat{K}_{\nu}^\ell$, $ \mat{K}_{\alpha}^\ell$ and $ \mat{K}_{\gamma}^\ell$ are weighted finite element matrices stemming from the diffusion, convection and reaction term, respectively, and $\mathbf{f}_s^\ell$, $\mathbf{f}_N^\ell$ and $\mathbf{f}_D^\ell$ denote the finite element vectors accounting for the source,  Neumann and Dirichlet data, respectively.

\smallskip

The parametric linear system associated with problem~\eqref{eq:pbB} is derived with an analogous procedure.  More precisely, let $\vpgd_{i,j}$ be the solution of the parametric equation
%
\begin{subequations}\label{eq:pbBpgd}
\begin{equation}\label{eq:varfBpgd}
	\Apgd(\vpgd_{i,j}, \deV; \bmu) = \Fpgd_j(\deV; \bmu; \bLambda_i^j )\quad \forall \deV \in \mathcal{V}_i\, ,  \forall \bmu \in \mathcal{P} \, \text{ and } \forall \bLambda_i^j \in \mathcal{Q}_i^j \, ,
\end{equation}
%
with
%
\begin{equation}\label{eq:linearBpgd}
\Fpgd_j(\deV; \bmu;  \bLambda_i^j ) = 
- \Apgd \left( \, \sum_{q \in \mathcal{N}_i^j} \! \Lambda^q_i\,\varphi^q_i, \deV; \bmu \right) \, .
\end{equation}
\end{subequations}

The continuous Galerkin finite element discretization introduced in~\eqref{eq:spatialMode} is employed also for the spatial modes $V_{i,j}^m(\bx)$, leading to the vector of spatial unknowns $\mathbf{V}_{i,j}^m$, whereas pointwise collocation is used for the parametric modes $\phi_{i,j}^m(\bmu)$ and $\psi_{i,j}^m(\bLambda_i^j )$. Hence, the PGD approximation~\eqref{eq:nonNormalisedPGDB} is determined by computing 
%
\begin{equation}\label{eq:algebraicSolB}
\bvpgd_{i,j}(\bmu, \bLambda_i^j) = \sum_{m=1}^{M_j} \mathbf{V}_{i,j}^m \, \phi_{i,j}^m(\bmu) \, \psi_{i,j}^m(\bLambda_i^j )
\end{equation}
%
as the solution of the parametric linear system
%
\begin{equation}\label{eq:algebraicB}
\left(
\sum_{\ell=1}^{n_{\nu}} \xi_{\nu}^\ell(\bmu) \mat{K}_{\nu}^\ell
+ \sum_{\ell=1}^{n_{\alpha}} \xi_{\alpha}^\ell(\bmu) \mat{K}_{\alpha}^\ell
+ \sum_{\ell=1}^{n_{\gamma}} \xi_{\gamma}^\ell(\bmu) \mat{K}_{\gamma}^\ell
\right)
\bvpgd_{i,j}(\bmu,\bLambda_i^j) 
=
\sum_{q \in \mathcal{N}_i^j} \Lambda^q_i \mathbf{f}_{\Lambda}^q \, ,
\end{equation}
%
for any value of $\bmu \in \mathcal{P}$ and $\bLambda_i^j \in \mathcal{Q}_i^j$.  In equation~\eqref{eq:algebraicB}, the vector $\mathbf{f}_{\Lambda}^q$ stems from imposing the parametric Dirichlet boundary condition at the interface $\Gamma_i$ in equation~\eqref{eq:boundaryProbs} within the finite element setting.

\smallskip

The encapsulated PGD library~\cite{Diez:2020:ACME} is utilized to solve equations~\eqref{eq:algebraicA} and \eqref{eq:algebraicB}. Technical details on the setup of problems~\eqref{eq:algebraicA} and~\eqref{eq:algebraicB} in the encapsulated PGD framework for a sample test case are presented in Appendix~\ref{append:encapsulatedPGD}.

\smallskip

\begin{rem}
For the case of two subdomains, the cost of the offline phase stems from the computation of two surrogate models accounting for the data-dependent parametric problems~\eqref{eq:sourceProb} and $N_1+N_2$ surrogate models related to problems~\eqref{eq:boundaryProbs} with active boundary parameters.
%
It is worth noticing that all PGD approximations mentioned above are independent from one another and can be efficiently computed in parallel. 
%
Moreover, the computational effort during the offline phase can be further reduced, e.g.,  by identifying a reference subdomain where local surrogate models are computed before being suitably mapped to the physical subdomains of the problems under consideration, as demonstrated in the example in Sect.~\ref{sec:testPatera}.
\end{rem}