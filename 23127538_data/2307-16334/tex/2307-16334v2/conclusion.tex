\section{Concluding remarks}
\label{sec:Conclusions}

In this work, a DD-PGD approach combining the overlapping Schwarz algorithm with a physics-based PGD-ROM was proposed to solve parametric linear elliptic PDEs.
%
The method constructs local surrogate models with low dimensionality by exploiting the linearity of the problems to define disjoint sets of active boundary parameters that allow to represent arbitrary Dirichlet boundary conditions at the interfaces between subdomains.
%
In the online phase, the coupling of the subdomains is performed via an interface equation, leading to a linear system for the nodal values of the parametric solution at the interfaces.
%
The resulting system can be efficiently solved in real time using standard Krylov methods.

The advantages of the proposed approach are multiple.
%
First, by relying on a fully algebraic formulation, the DD-PGD strategy provides a non-intrusive framework for the construction of local ROMs via the encapsulated PGD library.
%
In the offline phase, the traces of the finite element functions used for the spatial discretization within each subdomain are employed to define parametric Dirichlet boundary conditions at the subdomain level, without the need to introduce auxiliary basis functions for the solution at the interface. 
%
In addition, the linearity of the operators is exploited to reduce the overall dimensionality of the problem, by devising a set of subproblems with only a few active boundary parameters.
%
For the coupling procedure, the discussed parametric multi-domain formulation allows to seamlessly glue the local ROMs only at the interfaces, without introducing extra variables (i.e., Lagrange multipliers) or enforcing the continuity of the local solutions in the entire overlapping region.
%
Finally, the solution of the parametric interface equation only entails the interpolation of the previously computed surrogate models in the parametric space, with no additional problems to be solved during the online phase.

The resulting DD-PGD approach was tested on a set of numerical benchmarks, including one and multiple parameters (both physical and geometrical), two and multiple subdomains, to assess accuracy, robustness and efficiency of the method.
%
The strategy showed accurate results comparable with the high-fidelity solutions, and robustness in different scenarios (from pure diffusion to convection-dominated convection-diffusion equations) while outperforming the standard non-overlapping DD-FEM in terms of computing time.
%
\rev{
It is worth noticing that the CPU times reported for the DD-PGD simulations could be further optimized by accelerating the convergence of the online phase via tailored preconditioning strategies~\cite{Toselli:2005,Dolean-DJN-15}. Other robust overlapping DD algorithms as well as non-overlapping DD strategies, which are outside the scope of this work, should also be investigated. These aspects play a crucial role in guaranteeing the applicability of the described methodology to real-world, three-dimensional cases. Indeed, although the presented approach can be seamlessly employed to construct local surrogate models of 3D problems, the partition of complex domains of industrial interest relies on state-of-the-art graph partitioning software, such as, e.g., METIS, KaHIP and Scotch~\cite{Metis,Kahip,Scotch}, and their efficient, non-intrusive coupling with local PGD-ROMs will require further study.
}