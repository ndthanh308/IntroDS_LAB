\section{SUPG formulation on a reference domain}
\label{append:Graetz}
%==========================================================================

The problem in Sect.~\ref{sec:testRozza} features a convection-diffusion equation in a parametric domain. To construct the PGD surrogate model, a stabilized SUPG formulation is employed and the subproblem in $\Omega_2(\bmu) = [1, 1+\mu_2] \times [0,1]$ is rewritten on a parameter-independent subdomain $\hOmega_2 = [0,1] \times [0,1]$ via a reference domain configuration.

First, the variational form of problems~\eqref{eq:pbApgd} and~\eqref{eq:pbBpgd} with SUPG stabilization and $\mathbb{Q}_1$ Lagrangian finite elements are recalled. More precisely, the bilinear form is given by
%
\begin{equation}\label{eq:bilinearSUPG}
\begin{aligned}
\Apgd(v, \deV; \bmu) 
=& \frac{1}{\mu_1} \int_{\Omega_2(\bmu)}   \nabla v\cdot \nabla \deV\,d\bx 
+ \int_{\Omega_2(\bmu)}  \balpha \cdot \nabla v \, \deV\,d\bx 
\\
&+ \sum_{T \in \Omega_2(\bmu)} \int_{T} \tSUPG (\balpha \cdot \nabla v) (\balpha \cdot \nabla \deV)\,d\bx 
\, ,
\end{aligned}
\end{equation}
%
while the linear forms for problems~\eqref{eq:pbApgd} and~\eqref{eq:pbBpgd} respectively account for the parameter-independent Dirichlet boundary condition on $\Gamma^{D,2}_2(\bmu)$ and for the Dirichlet condition associated with the active boundary parameters on the interface.

Following~\cite{Huerta-GCCDH-13} and exploiting the horizontal direction of the studied convection field, the stabilization parameter is defined as
%
\begin{equation}\label{eq:tauSPUG}
    \tSUPG = h_{x,1} \left(1 + \frac{9}{\mathrm{Pe}^2}\right)^{-\tfrac{1}{4 |\alpha_1|}}.
\end{equation}
%
Of course, the stabilization coefficient $\tSUPG$ depends upon the parameters. A detailed discussion on the choice of such stabilization in the context of PGD-ROM is available in~\cite{Huerta-GCCDH-13}. Nonetheless, in the present study, it was observed that the value of $\tSUPG$ computed according to the definition~\eqref{eq:tauSPUG} does not significantly vary with the parameters. Hence, 
for all the computations, a space-dependent stabilization coefficient is obtained by setting in~\eqref{eq:tauSPUG} the value of the P\'eclet number associated with $\mu_1 = 2 \times 10^4$.

In order to construct a PGD surrogate model starting from the SUPG formulation presented above, all terms in the bilinear and linear forms need to be appropriately rewritten in a parameter-independent domain.
%
To this end, the parametric mapping
%
\begin{equation}\label{eq:mapping}
\begin{aligned}
\Map : \
& \hOmega_2 \times \mathcal{I}^2 \rightarrow \Omega_2(\bmu) \\[0.5em]
& (\hx,\hy,\mu_2) \mapsto (x,y)
\end{aligned}
\end{equation}
%
between the reference and the physical subdomains is defined as
%
\begin{equation}\label{eq:transformation}
x = 
\begin{cases}
    1+ \hx \quad & \text{for } \hx \leq \bar{h}\,,\\[0.5em]
    \displaystyle\frac{1 - \bar{h} \hx}{1-\bar{h}} + \mu_2 \frac{\hx - \bar{h}}{1-\bar{h}}\quad &\text{for } \hx > \bar{h} \, ,
\end{cases}
%
\qquad
y = \hy \, ,
\end{equation}
%
with $\bar{h} = 5 \times 10^{-2}$. A sketch of the transformation is displayed in Figure~\ref{fig:mapGraetz}.
%
% Figure environment removed

Hence, following~\cite{Ammar-AHCCL-14}, the integrals in~\eqref{eq:bilinearSUPG} are mapped to the reference subdomain by inverting the transformation~\eqref{eq:mapping}. Let $\Jaco$ denote the Jacobian of the mapping, $\detJ$ its determinant and $\adjJ = \detJ \Jaco^{-1}$ its adjoint.
%
The resulting SUPG bilinear form on the reference subdomain is given by
%
\begin{equation}\label{eq:bilinearSUPGfixed}
\begin{aligned}
\Apgd(v, \deV; \bmu) 
=& \frac{1}{\mu_1} \int_{\hOmega_2}   \nabla v\cdot \left(\frac{\adjJt \adjJ}{\detJ} \nabla \deV \right) d\bhx 
+ \int_{\hOmega_2}  \balpha \cdot \left( \adjJ \nabla v \right) \, \deV\,d\bhx 
\\
&+ \sum_{\hT \in \hOmega_2} \int_{\hT} \tSUPG (\balpha \cdot \adjJ \nabla v) (\balpha \cdot \Jaco^{-1}\nabla \deV)\,d\bhx 
\, ,
\end{aligned}
\end{equation}
%
where all integrals are computed on the parameter-independent subdomain $\hOmega_2$ and the dependence on $\mu_2$ is encapsulated in the mapping.
%
Introducing the definition of $\Jaco$, $\detJ$ and $\adjJ$ for the transformation~\eqref{eq:transformation} into~\eqref{eq:bilinearSUPGfixed}, the resulting bilinear form features an affine dependence on $\bmu$ (see, e.g.,~\cite{Rozza:14}), yielding
%
standard finite element matrices with non-constant parameters. Hence, the strategy described in Appendix~\ref{append:Setup} can be straightforwardly applied to equation~\eqref{eq:bilinearSUPGfixed} to construct the local surrogate model using the encapsulated PGD.
%
For alternative strategies to construct PGD surrogate models of geometrically parametrized problems featuring more general transformations, the interested reader is referred to~\cite{Zlotnik:2015:IJNME,Sevilla-SZH-20,Sevilla-SBGH-20}.

