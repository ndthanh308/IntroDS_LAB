\section{Approximate histogram under linear aggregation}
\label{sec:ahist}
In this section, we study the task of obtaining an approximate histogram in the multi-round linear aggregation model. The first observation we make is that using \cref{alg:subsample_IBLT} with threshold $\dist$, we are able to return a list $H$ of heavy hitters such that with high probability, the list contains all $x$'s with frequency more than $\dist$ and no tail elements. The approximate histogram algorithm builds on this and further asks each user to send a linear sketching of the their unsampled local data alongside the IBLT data structures in \cref{alg:subsample_IBLT}. The server would then use the aggregation of these linear sketches as a frequency oracle to estimate the frequency of elements in $H$. 

The above protocol leads to near optimal performance in the single-round case.
However, the $R$-round case is trickier since the error will build up along all $R$ rounds and the naive application of the sketching algorithm will lead to an error that depends linearly in $R$. This can be solved by carefully designing the correlation among hash functions in all $R$ rounds and we show that the dependence on $R$ can be reduced to $\sqrt{R}$.
We further show that the $\sqrt{R}$ dependence is in fact optimal by proving a matching lower bound, stated in \cref{thm:ahist_lower}.











\new{To improve the dependence on $R$, we use the \textsc{HybridSketch} idea from \citet{wu2023private}}. More precisely, 
the location hashes are fixed across rounds while the sign hashes are generated with fresh randomness. The details of the algorithm are described in \cref{alg:ahist_r}. The proof follows from the guarantee in \cref{thm:ahh} and standard analysis for the Count-sketch algorithm. We defer the complete proof to \cref{sec:ahist_app}.  %

\begin{theorem}\label{thm:ahist_r}
      In the $R$-round setting, there exists a linear aggregation protocol with communication cost %
      \new{$\tilde{O}\Paren{
      \min\{ \frac{mn\sqrt{R}}{\dist}, mn\}}$}
      per user, which solves the \textbf{$\dist$-approximate histogram}
problem. Moreover, the running time of the algorithm is %
      \new{$\tilde{O}\Paren{
      \min\{ \frac{mn\sqrt{R}}{\dist}, mn\}}$}. 
\end{theorem}

\begin{algorithm}[h]
\caption{$R$-round \ahist~with \linagg}
\begin{algorithmic}[1]
\STATE \textbf{Input:} $\{h_i\}_{i \in B_r, r \in [R]}:$ local histograms; $d:$ alphabet size; $R:$ number of rounds; $m:$ per-user contribution bound; $n:$ number of users per round; $\dist:$ error for approximate histogram; $\beta:$ failure probability.
\new{\IF{$\tau \le \sqrt{R}$}
\STATE Users implement \cref{alg:subsample_IBLT} with $\tau = 1$ and \textbf{return} the histogram obtained in Line 11.
\ENDIF}
\STATE Let $w = \ceil{ \frac{10\nspu \ns\sqrt{R}}{\dist}}$ and $\rep = \ceil{\log\Paren{\frac{4\nspu\ns R}{\tau \beta}}}$.

\STATE Get the same set of location hash functions $\{g_j: [d] \rightarrow [w]\}_{j \in [w]}$ for all rounds. And the independent sets of sign hashes $\{s_{j,r}: [d] \rightarrow \{\pm 1\}\}_{j \in [w], r \in [R]}$ across rounds.
\FOR{$r \in [R]$}
\STATE (\emph{In Parallel}) Each user $i \in B_r$ implements the protocol in \cref{alg:subsample_IBLT} and sends messages $Y_i$.

\STATE (\emph{In Parallel}) User $i \in B_r$ encode  $j \in [b]$ and $k \in [w]$,  
\[
    T_{i}(j,k) =\sum_{x}\indic{ \hash_{j}(x) = k} s_{j, r}(x) \cdot h_i(x).
\]
\ENDFOR
\STATE Server obtains a list $H$ of heavy hitters from the the messages $Y_i$'s.

\STATE Server obtains $\forall r \in [R], T^{(r)} = \sum_{i \in B_r} T_i$ and constructs $\hat{h}$, where $\forall x \in H$
\[
   \hat{h}(x) = {\rm Median}\Paren{ \{\sum_{r \in [R]} T^{(r)}(j, \hash_{j}(x)) \cdot s_{j, r}(x)  \}_{j \in [\rep]}},
\]
and $\forall x \notin H, \hat{h}(x) = 0$.
\STATE \textbf{Return} $\hat{h}.$
\end{algorithmic}
\label{alg:ahist_r}
\end{algorithm}

\paragraph{Lower bound for \ahist} %
We prove the following lower bound on \ahist, which shows that the bound in \cref{thm:ahist_r} is tight up to logarithmic factors, establishing the seperation between the sample complexities from \ahh~and \ahist.


\begin{theorem}
\label{thm:ahist_lower}
      For any $\dist$ and $R$-round \ahist~protocol with per-user communication cost \new{$o\Paren{\min\{\frac{mn\sqrt{R}}{\dist}, mn\}}$}, there exists a dataset $\{ h_i\}_{i \in B_r, r \in [R]}$, such that the protocol cannot solve $\dist$-\textbf{approximate histogram} with error probability at most 1/5.
\end{theorem}

