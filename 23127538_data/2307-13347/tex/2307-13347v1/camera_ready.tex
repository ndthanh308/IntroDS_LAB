
\documentclass{article}

\usepackage{microtype}
\usepackage{graphicx}
\usepackage{subfigure}
\usepackage{booktabs} %
\usepackage{bbm}
\usepackage{hyperref}


\newcommand{\theHalgorithm}{\arabic{algorithm}}

\def\withcolors{1}
\def\withnotes{1}

\ifnum\withnotes=1
\renewcommand{\th}[1]{ {\noindent \textit{\small\textcolor{blue}{theertha: #1}}}}
\newcommand{\zs}[1]{{\noindent \textit{\small\textcolor{red}{ziteng: #1}}}}
\newcommand{\znote}[1]{ [\textcolor{purple}{Ziteng: #1}] }
\newcommand{\tnote}[1]{ [\textcolor{blue}{Theertha: #1}] }
\newcommand{\anote}[1]{ [\textcolor{orange}{Adria: #1}] }
\newcommand{\todo}[1]{ [\textcolor{red}{TODO: #1}] }
\else
\renewcommand{\th}[1]{{}}
\newcommand{\zs}[1]{{}}
\newcommand{\znote}[1]{}
\newcommand{\tnote}[1]{}
\newcommand{\anote}[1]{}
\newcommand{\todo}[1]{}
\fi

\ifnum\withcolors=1
\newcommand{\new}[1]{{\color{red}{#1}}}
\newcommand{\hl}[1]{{\textcolor{red}{#1}}}
\else
\newcommand{\new}[1]{{{#1}}}
\newcommand{\hl}[1]{{{#1}}}
\fi

\usepackage[accepted]{icml2023}

\usepackage{amsmath}
\usepackage{amssymb}
\usepackage{mathtools}
\usepackage{amsthm}

\usepackage[capitalize,noabbrev]{cleveref}

\theoremstyle{plain}
\newtheorem{theorem}{Theorem}[section]
\newtheorem{proposition}[theorem]{Proposition}
\newtheorem{lemma}[theorem]{Lemma}
\newtheorem{corollary}[theorem]{Corollary}
\theoremstyle{definition}
\newtheorem{definition}[theorem]{Definition}
\newtheorem{assumption}[theorem]{Assumption}
\theoremstyle{remark}
\newtheorem{remark}[theorem]{Remark}



\input{ICML_camera_ready/glodef}
\DeclareMathOperator*{\tr}{tr}
\DeclareMathOperator*{\diag}{diag}%
\DeclareMathOperator*{\argmin}{argmin}
\DeclareMathOperator*{\argmax}{argmax}
\DeclareMathOperator*{\N}{N}
\DeclareMathOperator*{\h}{H}
\DeclareMathOperator*{\NMSE}{NMSE}
\DeclareMathOperator*{\SNR}{SNR}
\DeclareMathOperator*{\avg}{avg}
\DeclareMathOperator*{\cov}{Cov}
\DeclareMathOperator*{\KL}{KL}
\DeclareMathOperator*{\var}{Var}
\newcommand{\norm}[1]{\lVert{#1}\rVert}
\newcommand{\ip}[1]{\langle {#1}\rangle}

\newcommand*{\unity}{\textrm{{\usefont{U}{fplmbb}{m}{n}1}}}

\newcommand{\dtv}[2]{d_{TV}(#1, #2)}
\newcommand{\hellinger}[2]{d_{H}^2(#1, #2)}
\renewcommand{\algorithmicrequire}{\textbf{Input:}}
\newcommand{\bigO}{\ensuremath{O}}

\newcommand{\Epsilon}{E}

\newcommand{\X}{X}
\newcommand{\x}{x}
\newcommand{\Y}{Y}
\newcommand{\y}{y}
\newcommand{\kk}{k}
\newcommand{\n}{l}
\newcommand{\D}{D}
\newcommand{\Q}{Q}
\newcommand{\q}{q}
\newcommand{\s}{S}
\newcommand{\R}{R}
\newcommand{\T}{T}

\newcommand{\pp}{p}
\newcommand{\MM}{M}
\newcommand{\mm}{m}


\newcommand{\prob}{\mathbb{P}}

\newcommand{\Po}{P_1}
\newcommand{\po}{p_1}
\newcommand{\Mo}{M_1}
\newcommand{\mo}{m_1}


\newcommand{\Pt}{P_2}
\newcommand{\pt}{p_2}
\newcommand{\Mt}{M_2}
\newcommand{\mt}{m_2}


\newcommand{\B}{\Theta}
\newcommand{\bb}{\theta}
\newcommand{\dd}{\alpha}

\newcommand{\Dkl}{D_{\rm kl}}

\newcommand{\proj}{\text{Proj}}
\newcommand{\Dk}{\Delta_k}
\newcommand{\Du}{\Delta_u}

\newcommand{\absz}{k}
\newcommand{\ab}{k}

\newcommand{\ns}{n}
\newcommand{\nspu}{m}
\newcommand{\dist}{\tau}
\newcommand{\p}{p}


\newcommand{\smb}{x}
\newcommand{\psmb}{\p_\smb}
\newcommand{\ps}[1]{\p_{#1}}
\newcommand{\Xon}{X_1^\ns}
\newcommand{\Yon}{Y_1^\ns}



\newcommand{\Mestxon}{\Mest\Paren{\Xon}}
\newcommand{\Mest}{{M}}
\newcommand{\Mestgtxon}{\Mest^{\texttt{GT}}\Paren{\Xon}}


\newcommand{\Moxon}{M_0(\Xon)}

\newcommand{\prf}[2]{\Phi_{#1}(#2)}
\newcommand{\prfi}[1]{\Phi_{#1}}

\newcommand{\kp}{k^{'}}

\newcommand{\eps}{\varepsilon}

\newcommand{\tmix}[1]{t_{\text{mix}}\Paren{#1}}
\newcommand{\trelax}{t_{\text{relax}}}
\newcommand{\tm}{t_{\text{m}}}

\newcommand{\st}{n}
\newcommand{\sign}{\ensuremath{\text{sgn}}}

\newcommand{\sdist}[1]{$\textrm{DIST}_{#1}$}
\newcommand{\supp}[1]{{\rm supp}\Paren{#1}}
\newcommand{\bino}[2]{{\rm Bin}\Paren{#1, #2}}

\newcommand{\ahist}{ApproxHist}
\newcommand{\ahh}{ApproxHH}


\newcommand{\linagg}{LinSketch}
\newcommand{\thr}{\tau}
\newcommand{\deciblt}{{\rm Dec}}
\newcommand{\rep}{b}
\newcommand{\heavy}{H}
\newcommand{\hash}{g}
\newcommand{\size}{C}
\newcommand{\width}{W}
\newcommand{\length}{H}
\newcommand{\mmax}{M_{\rm max}}
\newcommand{\lmax}{\ell_{\rm max}}
\newcommand{\indic}[1]{\mathbbm{1}\left\{ #1 \right\}}
\begin{document}


\twocolumn[
\icmltitle{Federated Heavy Hitter Recovery under Linear Sketching}




\begin{icmlauthorlist}
\icmlauthor{Adria Gascon}{goo}
\icmlauthor{Peter Kairouz}{goo}
\icmlauthor{Ziteng Sun}{goo}
\icmlauthor{Ananda Theertha Suresh}{goo}
\end{icmlauthorlist}

\icmlaffiliation{goo}{Google Research. Authorship is in alphabetical order}
\icmlcorrespondingauthor{Ziteng Sun}{zitengsun@google.com}

\icmlkeywords{Machine Learning, ICML}

\vskip 0.3in
]



\printAffiliationsAndNotice{}  %

\begin{abstract}
Motivated by real-life deployments of multi-round federated analytics with secure aggregation, we investigate the fundamental communication-accuracy tradeoffs of the heavy hitter discovery and approximate (open-domain) histogram problems under a linear sketching constraint. We propose efficient algorithms based on local subsampling and invertible bloom look-up tables (IBLTs).  We also show that our algorithms are information-theoretically optimal for a broad class of interactive schemes. The results show that the linear sketching constraint does increase the communication cost for both tasks by introducing an extra linear dependence on the number of users in a round. Moreover, our results also establish a separation between the communication cost for heavy hitter discovery and approximate histogram in the multi-round setting. The dependence on the number of rounds $R$ is at most logarithmic for heavy hitter discovery whereas that of approximate histogram is $\Theta(\sqrt{R})$. We also empirically demonstrate our findings.
\end{abstract}


% Figure environment removed

\section{Introduction}
Automatic 3D reconstruction of clothed humans using image inputs has gained increasing significance due to its potential applications in a wide array of AR/VR scenarios. High-fidelity reconstructions typically depend on sophisticated capture systems, which are developed with dense camera arrays~\cite{collet2015high,joo2015panoptic,joo2018total}, programmable light-stages~\cite{Vlasic2009, guo2019relightables}, and depth sensors~\cite{newcombe2011kinectfusion,DoubleFusion,BodyFusion,dou2016fusion4d,newcombe2015dynamicfusion}. However, stringent capture environments equipped with complex hardware pose significant challenges for consumer-level applications.


In this context, considerable research effort has been dedicated to developing methods that allow for more flexible capture configurations, such as utilizing a few RGB inputs. Among these works, learning implicit functions \cite{iccv2020PIFu, saito2020pifuhd, hong2021stereopifu} has proven effective in achieving highly detailed reconstructions by integrating the advancements of deep neural networks. These methods employ large multi-layer perceptrons (MLPs) to predict the occupancy probability or truncated signed distance function (TSDF) value of every queried 3D point based on its associated local feature, which is extracted from images. They can recover a continuous surface at arbitrary resolutions without topology restrictions.


However, in typical MLP-based implicit networks, the occupancy or TSDF value at each location is solved independently with planar image features, rendering them less capable of addressing challenging cases such as occlusions. Consequently, these methods suffer from generalization and robustness issues, particularly when tackling strong occlusions caused by large motion or multiple interacting humans. 
Some follow-up studies  \cite{zheng2021deepmulticap,zheng2021pamir,huang2020arch} utilize an extra geometric model, SMPL~\cite{Loper2015}, to improve robustness by introducing strong shape priors. 
Their success typically relies on the assumption of geometrical similarity \cite{huang2020arch} between the shape prior and target reconstruction, making them intractable for handling complex cases with loose clothes and sensitive to errors in SMPL model fitting.



%\ping{this paragraph sounds like `TSDF is better than MLP/SMPL, and we use TSDF to solve the problem'. But in Sec 3, we are telling a different story, saying `MLP needs a 3D convolutional encoder'. We need to make these two sections consistent.}\sicong{I think in this paragraph we claim that the TSDF}


%We opt for Trucated Signed Distance Funtion (TSDF) volumetric representations as they are naturally suitable for convolution operations, which have shown remarkable performance for learning hierarchical features on 2D visual perception tasks \cite{SunXLW19}. 
%Meanwhile, TSDF also describes the gradual geometry change around shape surface, which is not reflected by occupancy volume. 

We instead revisit the 3D volumetric representation and resort to 3D convolutional neural networks (CNNs) for feature learning, due to their impressive performance in feature learning and the ability to incorporate spatial context. However, volumetric methods and 3D convolution involve discretization, which might raise concerns regarding whether a discretized volume can preserve subtle geometric details as continuous representations learned in implicit functions. We investigate the relationship between volume resolution and quantization error on synthetic data by converting target mesh objects to TSDF volumes, as shown in Figure~\ref{fig:quantization_error}. We observe that the quantization errors are significantly reduced by increasing volume resolution and become nearly negligible when reaching a relatively high resolution (e.g., 512 or higher). In other words, achieving fine-detailed reconstruction is not supposed to be restricted by the use of volume representations as long as a proper volume resolution is utilized. Therefore, we present a method with high-resolution feature volumes, e.g., 256 and 512, while traditional volumetric methods \cite{varol18_bodynet,gilbert2018volumetric} are often limited to much lower resolutions, such as 32 or 128.



On the other hand, an increase in volume resolution may lead to a cubic growth of memory overhead \cite{8100085}. Reducing memory costs while guaranteeing the granularity of volumetric representations is necessary for pursuing high-quality reconstruction. Thus, we adopt a coarse-to-fine approach and cull away irrelevant voxels to build a sparse high-resolution feature volume. At the coarse level, the network computes an initial TSDF by applying a U-Net with sparse 3D CNN \cite{3DSemanticSegmentationWithSubmanifoldSparseConvNet} on the sparse feature volume, which is carved by a visual hull. Through our experiments, it turns out that more than 95\% of the volume grids are discarded by the visual hull culling, making the sparse 3D CNN efficient. At the fine level, the network focuses on a narrow band near the zero-level set of the initial TSDF and discretizes the narrow band with smaller voxels. By employing this narrow-band culling, we further shrink the sampling space, resulting in a relatively small range of grid numbers (usually 300K--500K in our experiments) even with a high volume resolution of 512. The remaining voxels in the narrow band are associated with features that fuse high-frequency information from the computed normal maps upon the low-frequency shape from the coarse level to compute the TSDF at high resolution. The final mesh is then extracted from the TSDF using the Marching-Cube algorithm ~\cite{Lorensen87marchingcubes}.
% Different from the u-net sturcture to preserve global topology context, we then apply a shallow 3dcnn to compute the final TSDF $D_{final}$ which contain more local geometry detail.




% \ping{this paragraph can be expanded. It is an important contribution and often ignored by other works. stress on the novel idea of regressing blending weights instead of colors}

In addition to geometry, high-quality mesh texture is also a crucial factor contributing to visual appearance. Directly computing a color field in 3D space, as in \cite{iccv2020PIFu}, struggles to capture high-frequency texture details, while the neural radiance field (NeRF) \cite{yu2020pixelnerf} or the DoubleField~\cite{shao2022doublefield} require expensive per-instance optimization and are often unstable for sparse input images. In contrast, we adopt an image-based rendering approach to compute a texture atlas map, which is efficient and widely supported in existing computer graphics tools. 
Specifically, we compute a blending weight at each 3D point on the mesh surface to determine its color as a weighted average of the colors at its image projections. The blending weights can be computed at a relatively coarse resolution, e.g., 512 volume resolution in our case, and leave texture details to the high-resolution images, such as 1K or 2K. Unlike previous methods that generate blurry texturing results under sparse input, our method generalizes well on both synthetic and real data with just a few input views. 
Figure~\ref{fig:teaser} shows two examples reconstructed by our method. Despite the challenging garment, pose, and occlusion, our method recovers faithful shape, normal, and texture on the right.

%with a wide variety of poses and clothing styles, and it is also adaptive to handle input image with arbitrary resolutions.
%\sicong{For this concern we claim that when the resolution of dicretized volume meets certain threshold (which is 256 in our experiment), the quantization error can be neglected.} 



In summary, the main contributions of this paper are as follows:
\begin{itemize}
\vspace{-0.1in}
  \item 
  We revisit the 3D volumetric representation and demonstrate that it can support clothed human reconstruction with equal or even better performance compared to implicit representation. 
  \item 
  We develop a memory and computation-efficient method for high-resolution volumetric reconstruction using sophisticated sparse 3D CNN, coarse-to-fine estimation, and voxel culling by visual hull and narrow bands. 
  \item 
  We introduce a novel method to compute a texture atlas map, which captures rich appearance details from high-resolution input images.
  \item 
  We achieve impressive results on standard benchmark datasets Twindom and MultiHuman, significantly reducing the point-2-surface (P2S) precision to approximately 0.2cm from just six input views, with more than $50\%$ error reduction compared to the state-of-the-art methods, including DoubleField~\cite{shao2022doublefield} and PIFuHD~\cite{saito2020pifuhd}.
\end{itemize}
\section{Background and Problem Statement}
\label{sec:setup}
We consider the problem of an agent interacting with an SCM for $T$ rounds in order to maximize the value of a reward variable. We start by introducing SCMs, the soft intervention model used in this work, and then define the adversarial sequential decision-making problem we study. In the following, we denote with $[m]$ the set of integers $\{0, \dots, m\}$. \looseness-1

\paragraph{Structural Causal Models}
Our SCM is described by a tuple $\langle \G,  Y, \bX, \fs, \snoiserv \rangle$ of the following elements: $\G$ is a \emph{known} DAG; $Y$ is the reward variable; $\bX = {\{X_i\}_{i=0}^{m-1}}$ is a set of observed scalar random variables; the set $\fs = \{\fofi\}_{i=0}^m$ defines the \emph{unknown} functional relations between these variables; and $\snoiserv = \{\snoiserv_i \}_{i=0}^{m}$ is a set of independent noise variables with zero-mean and known distribution. % \looseness-1
 We use the notation $Y$ and $X_m$ interchangeably and assume the elements of $\bX$ are topologically ordered, i.e., $X_0$ is a root and $X_m$ is a leaf.  We denote with $\pa_i \subset \{0, \dots, m\}$ the indices of the parents of the $i$th node, and use the notation $\bZi = \{ X_j\}_{j \in \pa_i}$ for the parents this node. We sometimes use $X_i$ to refer to both the $i$th node and the $i$th random variable. \looseness-1\looseness-1

Each $X_i$ is generated according to the function $\fofi: \calZ_i \rightarrow \calX_i$, taking the parent nodes $\bZi$ of $X_i$ as input: $\si =\fofi(\zi) + \noisei$, where lowercase denotes a realization of the corresponding random variable. The reward is a scalar $x_m \in [0,1]$ while observation $X_i$ is defined over a compact set $\si \in \calX_i \subset \R$, and its parents are defined over $\calZ_i = \prod_{j \in pa_i} \calX_j$ for $i\in [m-1]$.\footnote{Here we consider scalar observations for ease of presentation, but we note that the methodology and analysis can be easily extended to vector observations as in \citet{sussex2022model}}  \looseness-1

\paragraph{Interventions}

\looseness -1 In our setup, an agent and an adversary both perform \emph{interventions} on the SCM~\footnote{Our framework allows for there to be potentially multiple adversaries, but since we consider everything from a single player's perspective, it is sufficient to combine all the other agents into a single adversary.}. 
We consider a soft intervention model \citep{eberhardt2007interventions} where interventions are parameterized by controllable \emph{action variables}. A simple example of a soft intervention is a shift intervention, where actions affect their outputs additively \citep{zhang2021matching}.

First, consider the agent and its action variables $\bm a = {\{ \ai\}_{i=0}^{m}}$. Each action $a_i$ is a real number chosen from some finite set. That is, the space $\calA_i $  of action $a_i$ is   $\calA_i \subset \R_{[0, 1]}$ where $\abs{\calA_i} = K_i$  for some $K_i \in \nN$. Let $\calA$ be the space of all actions $\bm a = {\{ \ai\}_{i=0}^{m}}$. 
% Let $\calA$ be the space of all actions $\bm a = {\{ \ai\}_{i=0}^{m}}$.
We represent the actions as additional nodes in $\G$ (see \cref{fig:overview}): $\ai$ is a parent of only $X_i$, and hence an additional input to $\fofi$. Since $\fofi$ is unknown, the agent does not know apriori the functional effect of $\ai$ on $X_i$. Not intervening on a node $X_i$ can be considered equivalent to selecting $\ai = 0$. For nodes that cannot be intervened on by our agent, we set $K_i = 1$ and do not include the action in diagrams, meaning that without loss of generality we consider the number of action variables to be equal to the number of nodes $m$.
\footnote{There may be constraints on the actions our agent can take. We refer the reader to \citet{sussex2022model} for how our setup can be extended to handle constraints.}

For the adversary we consider the same intervention model but denote their actions by $\a'$ with each $\ai'$ defined over $\calA_i' \subset \R_{[0, 1]}$ where $\abs{\calA_i'} = K_i'$ and $K_i'$ is not necessarily equal to $K_i$. 

According to the causal graph, actions $\a, \a'$ induce a realization of the graph nodes: 
\begin{align}
\label{eq:groud_truth}
& \si = \fofi(\zi, \ai, \ai') + \noisei, \ \ \forall i \in [m].
\end{align}
 
If an index $i$ corresponds to a root node, the parent vector $\zi$ denotes an empty vector, and the output of $\fofi$ only depends on the actions.

\looseness-1

\paragraph{Problem statement}
Over multiple rounds, the agent and adversary intervene simultaneously on the SCM, with known DAG $\calG$ and fixed but unknown functions $\fs = \{\fofi\}_{i=1}^m$ with $\fofi: \calZ_i \times \A_i \times \A_i' \rightarrow \calX_i$. \looseness-1
At round $t$ the agent selects actions $\at = \{\ait\}_{i=0}^m$ and obtains observations $\st = \{\sit\}_{i=0}^m$, where we add an additional subscript to denote the round of interaction. When obtaining observations, the agent also observes what actions the adversary chose $\at' = \{\ait'\}_{i=0}^m$.  We assume the adversary does not have the power to know $\at$ when selecting $\at'$, but only has access to the history of interactions until round $t$. The agent obtains a reward given by \looseness-1
\begin{align}
\label{eq:groud_truth_target}
& y_t = f_m(\bm z_{m, t}, a_{m, t}, a_{m, t}') + \noise_{m, t},
\end{align}
which implicitly depends on the whole action vector $\at$ and adversary actions $\at'$. 

The agent's goal is to select a sequence of actions that maximizes their cumulative expected reward $\sum_{t=1}^T 
r(\at, \at')$ where $r(\at, \at') = \E{y_t\mid \at, \at'}$ and expectations are taken over $\snoise$ unless otherwise stated. The challenge for the agent lies in not knowing a-priori neither the causal model (i.e., the functions $\fs = \{\fofi\}_{i=1}^m$), nor the sequence of adversarial actions $\{\at'\}_{t=1}^{\cdots}$.

\paragraph{Performance metric} 

After $T$ timesteps, we can measure the performance of the agent via the notion of regret:
\begin{align}
    R(T) = \max_{\a \in \A} \sum_{t=1}^T r(\a, \at') - \sum_{t=1}^T r(\at, \at'),
    \label{eq:regret}
\end{align}
\ie, the difference between the best cumulative expected reward obtainable by playing a single fixed action if the adversary's action sequence and $\fs$ were known in hindsight, and the agent's cumulative expected reward. We seek to design algorithms for the agent that are \emph{no-regret}, meaning that $R(T)/T \rightarrow 0$ as $T\rightarrow \infty$, for any sequence $\at'$. We emphasize that while we use the term `adversary', our regret notion encompasses all strategies that the adversary could use to select actions. This might include cooperative agents or mechanism non-stationarities. \looseness -1


 For simplicity, we consider only adversary actions observed after the agent chooses actions. Our methods can be extended to also consider adversary actions observed \emph{before} the agent chooses actions, i.e., a \textit{context}. This results in learning a policy that returns actions depending on the context, rather than just learning a fixed action. This extension is straightforward and we briefly discuss it in~\Cref{app:contextual}. \looseness-1

\textbf{Regularity assumptions} We consider standard smoothness assumptions for the unknown functions $\fofi:\mathcal{S} \rightarrow \X_i$ defined over a compact domain $\mathcal{S}$ \citep{srinivas10}. In particular, for each node $i \in [m]$, we assume that $\fofi(\cdot)$ belongs to a reproducing kernel Hilbert space (RKHS) $\mathcal{H}_{k_i}$, a space of smooth functions defined on $\calS = \calZ_i \times \calA_i \times \calA_i'$.
This means that $\fofil \in \mathcal{H}_{k_i}$ is induced by a kernel function $k_i: \calS \times  \calS \rightarrow \mathbb{R}$. 
We also assume that $k_i(s,s') \leq 1$ for every $s, s' \in \calS$\footnote{This is known as the bounded variance property, and it holds for many common kernels.}. Moreover, the RKHS norm of $\fofi(\cdot)$ is assumed to be bounded $\|\fofi\|_{k_i} \leq \mathcal{B}_i$ for some fixed constant $\mathcal{B}_i>0$.  Finally, to ensure the compactness of the domains $\Z_i$, we assume that the noise $\snoise$ is bounded, i.e., $\noisei \in \left[-1,1\right]^{d}$. \looseness-1

\section{Experimental Results}\label{sec:results}
    \subsection{General Results}
        The basic ResSAN model is used to determine reference results which our expanded model can be compared to as it is structurally similar to ResLAN but does not possess the Lidar adaptive components of it. Further, we compare with the full-size PackNet-SAN and the unmodified NLSPN architecture. 
        As it can be seen from Tab.\,\ref{tab:sota-results}, our LiDAR-adaptive ResLAN achieves competitive performance compared to state-of-the-art standard depth completion methods, which are specialized to the unfiltered 64-beam-LiDAR. The performance differences are in the range of a few centimetres in terms of MAE, which is acceptable given the practical advantage that ResLAN can generalize to different beam patterns as will be shown below.

        Furthermore, we compared the architectures for a set of three different input types that contained 64, 32 or 16 LiDAR channels using both filter types on the metrics from the KITTI benchmark. The NLSPN model was trained for the standard depth completion task and then evaluated with different input data. As for the ResSAN models, we trained one model for each input type and tested it for the corresponding one which serve serve as the \emph{Baseline} in Tab.\,\ref{tab:overall-results}. Our ResLAN model was jointly trained for all three settings. As listed in Tab.\,\ref{tab:overall-results}, the ResLAN models outperform the challenging baseline in all metrics for FOV filtering and all but one for sparse filtering. This implies that our LiDAR adaptive model is able to outperform dedicated models in case of very sparse input depth. Fig.\,\ref{fig:comp-plot} shows this is indeed the case for 32 and even more for 16 channels. For FOV-filtered inputs with 16 channels, the ResLAN exhibits approx. $10\%$ smaller MAE than the baseline. As for the NLSPN, it becomes apparent that it is not capable of generalizing to other input types since it shows clearly worse results. The difference is especially pronounced for the FOV filtering where on average more than every fourth predicted pixel is more than $25 \%$ deviating from the ground truth\,($\delta_{1.25}$). Therefore, using a weight-adapting network in combination with differently filtered input depths allows us to train models that outperform their non-adaptive counterparts.

        \begin{table}[]
            \centering
    	    \small
            \vspace{0.4cm}
            \caption{\textbf{Depth estimation result for standard depth completion} when the ResSAN model was only trained for 64 channels and the ResLAN model for multiple tasks. The PackNet-SAN and NLSPN models were trained with the setup that was also used for our model architecture.}
            \footnotesize
            \setlength{\tabcolsep}{5pt}
            \begin{tabular}{@{}lrrrrl@{}}
            \toprule
            \multicolumn{6}{c}{\textbf{Standard LiDAR Depth Completion}}                                                                                                                         \\ \midrule
            \multicolumn{1}{l|}{Method}          & RMSE $\downarrow$            & MAE  $\downarrow$            & iRMSE $\downarrow$             & iMAE $\downarrow$ & $\delta_{1.25}$ $\uparrow$ \\
            \multicolumn{1}{l|}{}                & \multicolumn{1}{l}{{[}mm{]}} & \multicolumn{1}{l}{{[}mm{]}} & \multicolumn{1}{l}{{[}1/km{]}} & {[}1/km{]}        &                            \\ \midrule
            \multicolumn{1}{l|}{PackNet-SAN}     &  914                            &  298                            &  2.78                              &  1.4                 &  99.65 \%                          \\
            \multicolumn{1}{l|}{NLSPN}           &  \textbf{889}                            &   \textbf{263}                           &  \textbf{2.62}                              &   \textbf{1.3}                &   \textbf{99.61} \%                         \\ \midrule
            \multicolumn{1}{l|}{ResSAN (Ours)}   & 948                             &  275                            &  2.75                              &    1.4               &   99.58 \%                         \\
            \multicolumn{1}{l|}{ResLAN (Ours)} &   969                           &  283                            &   2.83                             &   1.4                &  99.56 \%                          \\ \bottomrule
            \end{tabular}
            \vspace{0.2cm}
            \label{tab:sota-results}
        \end{table}

        \begin{table}[]
    	    \centering
    	    \small
    	    \caption{\textbf{Depth estimation results of the two baseline setups and the explicit and implicit ResSAN} when evaluated on a combination of 16, 32 and 64 channel depth inputs. Please note that Specialist Methods need to train three specialized networks, one for each of the three types of inputs while our method only uses one network.}
            \footnotesize
            \setlength{\tabcolsep}{4.8pt}
            \begin{tabular}{@{}lrrrrl@{}}
                \toprule
                \multicolumn{6}{c}{\textbf{Sparse Channel Filter}}                                                                                                                                  \\ \midrule
                \multicolumn{1}{l|}{Method}        & RMSE $\downarrow$            & MAE  $\downarrow$            & iRMSE $\downarrow$             & iMAE $\downarrow$ & $\delta_{1.25}$ $\uparrow$  \\
                \multicolumn{1}{l|}{}              & \multicolumn{1}{l}{{[}mm{]}} & \multicolumn{1}{l}{{[}mm{]}} & \multicolumn{1}{l}{{[}1/km{]}} & {[}1/km{]}        &                             \\ \midrule
                \multicolumn{1}{l|}{NLSPN}         &  1396                            &  437                            & 5.54                               &  2.2                 &  98.82 \%                           \\
                \multicolumn{1}{l|}{Baseline}      & \textbf{1207}                             &  381                            & 4.41                               &  1.8                 &  \textbf{99.37} \%                           \\
                \multicolumn{1}{l|}{ResLAN (Ours)} &  1215                            &  \textbf{378}                            &  \textbf{4.27}                              &  \textbf{1.7}                 &  99.31 \%                           \\ \toprule
                \multicolumn{6}{c}{\textbf{Field-of-View Filter}}                                                                                                                                   \\ \midrule
                \multicolumn{1}{l|}{Method}        & RMSE $\downarrow$            & MAE  $\downarrow$            & iRMSE $\downarrow$             & iMAE $\downarrow$ & $\delta_{1.25}$ $\uparrow$ \\
                \multicolumn{1}{l|}{}              & \multicolumn{1}{l}{{[}mm{]}} & \multicolumn{1}{l}{{[}mm{]}} & \multicolumn{1}{l}{{[}1/km{]}} & {[}1/km{]}        &                             \\ \midrule
                \multicolumn{1}{l|}{NLSPN}         &  2738                            &  1702                            & 12.3                              &  4.3                 &  74.69 \%                           \\
                \multicolumn{1}{l|}{Baseline}      &  1556                            &  525                            &  6.8                              &  3.0                 & 98.14 \%                            \\
                \multicolumn{1}{l|}{ResLAN (Ours)} &  \textbf{1548}                            &  \textbf{519}                            &  \textbf{6.44}                              &  \textbf{2.8}                 & \textbf{98.52 \%}                            \\ \bottomrule
            \end{tabular}
            \label{tab:overall-results}
        \end{table}

        
        
        % Figure environment removed
        
        % Figure environment removed

    \subsection{Filter Effects}
        Comparing the effect of the two different types of depth input filters on the model performance, it becomes apparent that FOV filtering is the more challenging task. In that setting, reducing LiDAR channels is more detrimental to the performance than sparse filtering as it creates regions where no depth information is available. Effectively, the model is forced to perform depth prediction in these regions. These effects are highlighted in the depth images in Fig.\,\ref{fig:dense-maps} where the effect of a 16-channel sparse depth filter and a 16-channel FOV can be compared.

    \subsection{Generalization Capabilities}
        We trained three models for both filter types eaach, so the combinations and number of filtered depth inputs they receive are different. This serves the purpose of testing the generalization capabilities of the ResLAN architecture as well as the robustness to different filter settings. After training, the models were evaluated for the depth input settings they were trained for, as well as for ones they weren't exposed to. Overall, ResLAN shows good generalization capabilities. As one can gather from Fig.\,\ref{fig:explicit-comp} and Fig.\,\ref{fig:implicit-comp}, the consequences of slightly varying sets of input depth settings are limited. The most considerable deviations can be seen when the model is tasked to extrapolate. For instance, the model $\{64, 32, 16\}$ shows a noticeably higher MAE for eight-channel depth inputs than the model that was trained for it. Similar behaviour can be seen for the FOV filtering case as well for the model $\{64, 48, 32\}$ when tasked to generalize for a 16-channel input. There is no such pronounced effect for generalization tasks that lie between two filter settings the model was trained for. At most, it can be observed that models that were trained for a smaller range of filter values perform slightly better than ones that have to cover a wider range. The number of filter settings used in a fixed range does not relevantly influence the model performance, as can be seen, when comparing the two models in Fig.\,\ref{fig:implicit-comp}, which are both trained for a range of 64 to 32 channels but one with three filter settings and the other one with five.
    
    % Figure environment removed
    
    
    % Figure environment removed
\section{Approximate heavy hitter under linear aggregation}
\label{sec:ahh}
In this section, we study the approximate heavy hitter problem and show that the problem can be solved with per-user communication complexity $\tilde{O}\Paren{\frac{mn}{\thr} \log d}$, stated in \cref{thm:ahh}.

A natural comparison to make is the heavy hitter recovery algorithm obtained from getting a frequency oracle up to accuracy $\Theta(\thr)$. Since there are $R$ rounds, the naive approach would require an accuracy of $\Theta(\thr/R)$ in each round and classic methods such as Count-min and Count-sketch would require a per-user communication complexity of $\tilde{\Theta}(\nspu\ns R/\thr)$. In the $R$-round case, our result improves upon this by a factor of $R$. In fact, as we show in \cref{thm:ahist_lower}, any frequency oracle-based approach would require per-user communication complexity of at least $\Omega(\nspu\ns \sqrt{R}/\thr)$. Our result improves upon these and show that the dependence on $R$ is at most logarithmic.

\begin{theorem} \label{thm:ahh}
There exists a non-interactive linear sketching protocol with communication cost %
$\tilde{O}\Paren{\frac{mn}{\thr} }$
bits per user, %
which solves the \textbf{$\thr$-approximate heavy hitter}
problem. Moreover, the running time of the algorithm is
$\tilde{O}\Paren{\frac{mn}{\thr}}$. 
\end{theorem}

The next theorem shows that the above communication complexity is minmax optimal up to logarithmic factors. 
\begin{theorem} \label{thm:ahh_lower}
    For any $\thr$ and \new{interactive linear sketching %
    protocol} $\cA$ with per-user communication cost $o\Paren{\frac{mn}{\thr}}$, there exists a dataset $h_i, i \in B_r, r \in [R]$, such that $\cA$ cannot solve $\thr$-heavy hitter (HH) with success probability at least 4/5.
\end{theorem}



Next we will present the protocol that achieves \cref{thm:ahh} in \cref{sec:ahh_upper} and discuss the proof of the lower bound \cref{thm:ahh_lower} in \cref{sec:ahh_lower}.

\label{sec:ahh_upper}

At a high level, the protocol relies on two main components: (i) a probabilistic data structure called Invertible Bloom Lookup Table (IBLT) introduced by \citet{Goodrich2011iblt}, and (ii) local subsampling. We start by introducing IBLTs, starting from the more standard (counting) Bloom filters. 

\paragraph{IBLT: Bloom filters with efficient listing.} Note that each user's local histogram $h_i$ can be viewed as a sequence of key-value pairs $(x, h_i(x))$.
The Bloom filter data structure
is a standard linear data structure to represent a 
set of key-value pairs with keys coming from a large domain. 
IBLT is a version of Bloom filter that %
supports an efficient listing operation -- while preserving the other nice properties of Bloom counting filters, namely linearity (and thus mergeable by summation), and succintness (linear size in number of indices it holds).\footnote{\new{In our algorithm, IBLT could be replaced by other data structures with these properties.}}
These properties are summarized in the following Lemma. 



\begin{lemma}[\cite{Goodrich2011iblt}] \label{lem:iblt}
Consider a collection of local histograms $(h_i)_{i\in [n]}$ over $[d]$ such that $\norm{\sum_{i \in [\ns]} h_i}_0 \le L_0$.

For any $\gamma  > 0$, there exist local linear sketches $\{f_i\}_{i \in [\ns]}$ of length $\ell = \tilde{O}(\gamma L_0)$ and an $O(\ell)$ time decoding procedure $\deciblt(\cdot)$~such that 
\[
    \deciblt\paren{\sum_{i \in [\ns]} f_i(h_i)} = \sum_{i \in [\ns]} h_i
\]
 succeeds except with probability at most $O\Paren{L_0^{2 -\gamma}}$.
\end{lemma}


For the purpose of this paper we can focus on 
the two main operations supported by an IBLT instance $\mathcal{B}$ (see~\cite{Goodrich2011iblt} for details on deletions and look-ups):
\begin{itemize}
    \item $\texttt{Insert}(k, v)$, which inserts the pair $(k, v)$ into 
$\mathcal{B}$.
\item
$\texttt{ListEntries}()$,
which enumerates the set of 
key-value pairs in $\mathcal{B}$.
\end{itemize} Note that
$f_i(h_i)$ in Lemma~\ref{lem:iblt} corresponds to the IBLT $\mathcal{B}_i$ resulting from inserting the set $\{(x,h_i(x)) ~|~ h_i(x) > 0\}$ into an empty IBLT.
Also, $\texttt{ListEntries}()$
corresponds to $\deciblt$
in Lemma~\ref{lem:iblt}.

Finally, 
$\sum_i^n f_i(h_i)$ corresponds to 
the encoding of 
the IBLT resulting 
from inserting the set $\{(x, \sum_i^n h_i(x)) ~|~ \exists i\in[n]: h_i(x) > 0\}$ into an empty IBLT.
In other words, each client $i\in [n]$
computes {\em local}
IBLT $\mathcal{B}_i := f_i(h_i)$,
and the (secure) aggregation of the 
$\mathcal{B}_i$'s results in the
{\em global} IBLT 
$\mathcal{B}:= \sum_i^n f_i(h_i)$. Further details on IBLT are stated in \cref{sec:iblt_app}.

\paragraph{Reducing capacity via threshold sampling.}
The second tool in our main protocol is threshold sampling. Note that
the guarantee in \cref{lem:iblt} relies on the number of unique elements in $\sum_{i \in [\ns]} h_i$, which can be at most $\nspu \ns$ in the worst-case, leading to an $O(mn)$ worst-case communication cost, not matching our lower bound in Lemma~\ref{thm:ahh_lower}. For heavy hitter recovery, we reduce the communication cost by local subsampling. More precisely, we use the threshold sampling algorithm from \cite{duffield05}, detailed in \cref{alg:threshold_sampling}
to achieve the (optimal) dependency $O(mn/\tau)$. 

\new{
\begin{remark}
Threshold sampling can be replaced by any unbiased local subsampling method that offers sparsity, \eg binomial sampling where $p \cdot h'(x) \sim \text{Binomial}(h(x), p)$ for some $p \in (0,1)$, and similar theoretical guarantee will hold. In this work, we choose threshold sampling due to the property that it minimizes the total variance of $h'$ under an expected sparsity constraint (see \citet{duffield05} for details).
\end{remark}
}
\begin{algorithm}[h]
\caption{Threshold sampling.}
\begin{algorithmic}[1]
\STATE \textbf{Input:} $h:$ local histogram. $t \in \RR_{+}:$ threshold.

\FOR{$x \in {\rm supp}(h)$}
    \IF {$h(x) \ge t$, }
        \STATE $h'(x) = h(x)$.
    \ELSE
        \STATE
        \[
            h'(x) = \begin{cases}
                t & \text{ with prob } \frac{h(x)}{t}, \\
                0 & \text{ otherwise.}
            \end{cases}
        \]
    \ENDIF
\ENDFOR
\STATE \textbf{Return:} $h'$.
\end{algorithmic}
\label{alg:threshold_sampling}
\end{algorithm}

The protocol that achieves the desired communication complexity in Theorem~\ref{thm:ahh} is detailed in \cref{alg:subsample_IBLT}. 

\begin{algorithm}[h!]
\caption{Subsampled IBLT with \linagg.}
\begin{algorithmic}[1]
\STATE \textbf{Input:} $\{h_i\}_{i \in B_r, r \in [R]}:$ local histograms; $d:$ alphabet size; $R:$ number of rounds; $m:$ per-user contribution bound; $n:$ number of users per round; $\thr:$ threshold for heavy hitter recovery; $\beta:$ failure probability.
\STATE Let $t = \max\{\tau/2, 1\}$, $\rep = \ceil{10\log(\frac{4\nspu\ns R}{\tau\beta})}$ and $L_0 = 20 \frac{mn}{\tau} \log R, \gamma = \log R$.

\FOR{$r \in [R]$}
\FOR{$j \in [b]$}
\STATE Each user $i \in B_r$ applies \cref{alg:threshold_sampling} with threshold $t$ in to their local histogram with fresh randomness to get $h'_{i, j}$. \label{line:iblt_encoding}
\STATE Each user sends message
$
    Y_{i, j} = f_{i,j}(h'_{i, j})
$
where $f_{i, j}$'s are mappings from \cref{lem:iblt} with parameter $L_0, \gamma$ and fresh randomness.
\STATE Server observes $\sum_{i \in B_r} Y_{i,j}$ and computes $$\hat{h}_{r, j} = \deciblt (\sum_{i \in B_r} Y_{i,j}).$$
If the decoding is not successful, we set $\hat{h}_{r, j}$ be the all-zero vector.\label{line:iblt_decoding}
\ENDFOR
\ENDFOR
\FOR{$j \in [b]$}\STATE Server computes $\hat{h}^{[R]}_j = \sum_{r \in [R]} \hat{h}_{r,j},$
and obtain list 
\[
    \heavy_j = \{x \in [d] \mid  \hat{h}^{[R]}_j > 0\}.
\]
\ENDFOR
\STATE \textbf{Return: } 
\[
    \heavy = \{x \mid \sum_{j \in [\rep]} \idc{x \in \heavy_j} \ge \frac{\rep}{2} \}.
\]
\end{algorithmic}
\label{alg:subsample_IBLT}
\end{algorithm}



The algorithm can be viewed as $\rep \eqdef \ceil{20\log(\frac{40\nspu\ns R}{\tau\beta})}$ independent runs of a basic protocol, each of which returns a list $\heavy_i$ of potential heavy hitters. And the repetition is to boost the error probability.

In each basic protocol, users first apply \cref{alg:threshold_sampling} to subsample to the data, which reduces the number of unique elements while maintaining the heavy hitters upon aggregation. Then the user encodes their samples using IBLTs, whose aggregation is then sent to the server to decode. Since the number of unique elements is reduced through subsampling, the decoding of the aggregated IBLT will be successful with high probabiltiy, hence recovering the aggregation of subsampled local histograms. The detailed proof of \cref{thm:ahh} is presented in \cref{sec:proof_ahh}.




\section{Approximate histogram under linear aggregation}
\label{sec:ahist}
In this section, we study the task of obtaining an approximate histogram in the multi-round linear aggregation model. The first observation we make is that using \cref{alg:subsample_IBLT} with threshold $\dist$, we are able to return a list $H$ of heavy hitters such that with high probability, the list contains all $x$'s with frequency more than $\dist$ and no tail elements. The approximate histogram algorithm builds on this and further asks each user to send a linear sketching of the their unsampled local data alongside the IBLT data structures in \cref{alg:subsample_IBLT}. The server would then use the aggregation of these linear sketches as a frequency oracle to estimate the frequency of elements in $H$. 

The above protocol leads to near optimal performance in the single-round case.
However, the $R$-round case is trickier since the error will build up along all $R$ rounds and the naive application of the sketching algorithm will lead to an error that depends linearly in $R$. This can be solved by carefully designing the correlation among hash functions in all $R$ rounds and we show that the dependence on $R$ can be reduced to $\sqrt{R}$.
We further show that the $\sqrt{R}$ dependence is in fact optimal by proving a matching lower bound, stated in \cref{thm:ahist_lower}.











\new{To improve the dependence on $R$, we use the \textsc{HybridSketch} idea from \citet{wu2023private}}. More precisely, 
the location hashes are fixed across rounds while the sign hashes are generated with fresh randomness. The details of the algorithm are described in \cref{alg:ahist_r}. The proof follows from the guarantee in \cref{thm:ahh} and standard analysis for the Count-sketch algorithm. We defer the complete proof to \cref{sec:ahist_app}.  %

\begin{theorem}\label{thm:ahist_r}
      In the $R$-round setting, there exists a linear aggregation protocol with communication cost %
      \new{$\tilde{O}\Paren{
      \min\{ \frac{mn\sqrt{R}}{\dist}, mn\}}$}
      per user, which solves the \textbf{$\dist$-approximate histogram}
problem. Moreover, the running time of the algorithm is %
      \new{$\tilde{O}\Paren{
      \min\{ \frac{mn\sqrt{R}}{\dist}, mn\}}$}. 
\end{theorem}

\begin{algorithm}[h]
\caption{$R$-round \ahist~with \linagg}
\begin{algorithmic}[1]
\STATE \textbf{Input:} $\{h_i\}_{i \in B_r, r \in [R]}:$ local histograms; $d:$ alphabet size; $R:$ number of rounds; $m:$ per-user contribution bound; $n:$ number of users per round; $\dist:$ error for approximate histogram; $\beta:$ failure probability.
\new{\IF{$\tau \le \sqrt{R}$}
\STATE Users implement \cref{alg:subsample_IBLT} with $\tau = 1$ and \textbf{return} the histogram obtained in Line 11.
\ENDIF}
\STATE Let $w = \ceil{ \frac{10\nspu \ns\sqrt{R}}{\dist}}$ and $\rep = \ceil{\log\Paren{\frac{4\nspu\ns R}{\tau \beta}}}$.

\STATE Get the same set of location hash functions $\{g_j: [d] \rightarrow [w]\}_{j \in [w]}$ for all rounds. And the independent sets of sign hashes $\{s_{j,r}: [d] \rightarrow \{\pm 1\}\}_{j \in [w], r \in [R]}$ across rounds.
\FOR{$r \in [R]$}
\STATE (\emph{In Parallel}) Each user $i \in B_r$ implements the protocol in \cref{alg:subsample_IBLT} and sends messages $Y_i$.

\STATE (\emph{In Parallel}) User $i \in B_r$ encode  $j \in [b]$ and $k \in [w]$,  
\[
    T_{i}(j,k) =\sum_{x}\indic{ \hash_{j}(x) = k} s_{j, r}(x) \cdot h_i(x).
\]
\ENDFOR
\STATE Server obtains a list $H$ of heavy hitters from the the messages $Y_i$'s.

\STATE Server obtains $\forall r \in [R], T^{(r)} = \sum_{i \in B_r} T_i$ and constructs $\hat{h}$, where $\forall x \in H$
\[
   \hat{h}(x) = {\rm Median}\Paren{ \{\sum_{r \in [R]} T^{(r)}(j, \hash_{j}(x)) \cdot s_{j, r}(x)  \}_{j \in [\rep]}},
\]
and $\forall x \notin H, \hat{h}(x) = 0$.
\STATE \textbf{Return} $\hat{h}.$
\end{algorithmic}
\label{alg:ahist_r}
\end{algorithm}

\paragraph{Lower bound for \ahist} %
We prove the following lower bound on \ahist, which shows that the bound in \cref{thm:ahist_r} is tight up to logarithmic factors, establishing the seperation between the sample complexities from \ahh~and \ahist.


\begin{theorem}
\label{thm:ahist_lower}
      For any $\dist$ and $R$-round \ahist~protocol with per-user communication cost \new{$o\Paren{\min\{\frac{mn\sqrt{R}}{\dist}, mn\}}$}, there exists a dataset $\{ h_i\}_{i \in B_r, r \in [R]}$, such that the protocol cannot solve $\dist$-\textbf{approximate histogram} with error probability at most 1/5.
\end{theorem}


\section{Practical adaptive tuning for instance-specific bounds} \label{sec:adaptive}
In practical scenarios, the per-user communication cost $\ell$ is often determined by system constraints (\eg delay tolerance, bandwidth constraint) and the goal is to recovery heavy hitters with the small enough $\thr$ under a fixed communication cost $\lmax$. While we have shown in \cref{thm:ahh_lower}, in the worst case, we can only reliably recover heavy hitters with frequency at least $\Omega(\frac{\nspu\ns}{\lmax})$. However, since the successful decoding of IBLTs only requires the number of \emph{unique} elements in a round to be small, when users' data is more favorable, it is possible to obtain better instance-specific bounds when the data is more concentrated on ``heavy'' elements.

When interactivity across rounds is allowed, we give an adaptive tuning algorithm for the subsampling parameter, which can be implemented when interactivity is allowed. The details of the algorithm are described in \cref{alg:adaptive_iblt}. At a high level, our algorithm is based on an estimate for $\norm{\sum_{i \in B_r} h_i'}_0$ where $h_i'$s are the subsampled histograms. When the decoding is successful, we can compute $\norm{\sum_{i \in B_r} h_i'}_0$ exactly from the recovered histogram. When the decoding is not successful, we rely on an analysis based on the ``core size'' of a random hypergraph~\citep{molloy2005cores} introduced by the hashing process to get an estimate of $\norm{\sum_{i \in B_r} h_i'}_0$. We discuss this in details in \cref{sec:iblt_app}.
Under the assumption that for a fixed subsampling parameter $t$, $\norm{\sum_{i \in B_r} h_i'}_0$  will be relatively stable across rounds, we can then increase/decrease $t$ based on past estimates of the data process.

We will empirically demonstrate the effectiveness of our tuning procedure. We leave proving rigorous guarantees on the adaptive tuning algorithm as an interesting future direction.


\begin{algorithm}[h]
\caption{Adaptive subsampled IBLT}\label{alg:adaptive_iblt}
\begin{algorithmic}[1]
\REQUIRE{Communication budget $C$, number of users $\ns$, user contribution bound $\nspu$. \\
\textbf{Update}: A tuning function that updates the subsampling parameter based on past observations.}
\STATE Set $
    t_0 = \Theta\Paren{\frac{\ns m}{C}}.
$
\FOR{$r = 0 , 1, 2, \ldots, R$}
        \STATE Each user $i \in B_r$ applies \cref{alg:threshold_sampling} with threshold $t$ in to their local histogram with fresh randomness to get $h'_{i}$. \label{line:iblt_encoding}
\STATE Each user sends message
$
    Y_{i} = f_{i}(h'_{i})
$
where $f_{i}$'s are mappings from \cref{lem:iblt} with parameter $L_0, \gamma$ and fresh randomness.
\STATE Server observes $\sum_{i \in B_r} Y_{i}$ and computes $$\hat{h}_{r} = \deciblt \paren{\sum_{i \in B_r} Y_{i}}$$
If the decoding is not successful, we let $\hat{h}_{r, j}$ be the all-zero vector.\label{line:iblt_decoding} 
\IF{The decoding is successful,}
    \STATE Set $\hat{s}_r = \norm{\hat{h}_{r}}_0$.
    \ELSE
        \STATE Get an estimate $\hat{s}_r$ for $\norm{\sum_{i \in B_r} h_i'}_0$ based on $\sum_{i \in B_r} Y_{i}$ using \eqref{eq:cardinality-from-core} and \eqref{eq:cardinality-from-core-2} (\cref{sec:iblt_app}). 
\ENDIF
        \STATE Set
        \[
        t_{r + 1} = \textbf{Update}(t_r, C, \hat{s}_r).
        \]
\ENDFOR
\end{algorithmic}
\end{algorithm}


\section{Experiment}

\subsection{Datasets and metrics}

% \noindent\textbf{Dataset.}
\subsubsection{Dataset}
% We adopt three datasets in our experiments, i.e., ClearGrasp \cite{sajjan2020clear}, TransCG \cite{fang2022transcg} and ClearPose \cite{chen2022clearpose}. The ClearGrasp dataset is the pioneering large-scale synthetic dataset that specifically focused on transparent objects. It provids a large-scale synthetic dataset as well as a real-world benchmark. The TransCG dataset comprises 57K RGB-D images from 130 different real-world scenes. 
% ClearPose dataset contains 350K RGB-D images of 63 household objects in real-world settings. Depth completion experiments and generalization verification (reported respectively in Section \ref{sec:depth} and \ref{sec:generalization}) are conducted on ClearGrasp, TransCG and ClearPose. Ablation study (reported in Section \ref{sec:ablation}) is performed on TransCG.
We use three datasets including ClearGrasp \cite{sajjan2020clear}, TransCG \cite{fang2022transcg}, and ClearPose \cite{chen2022clearpose}. The ClearGrasp dataset is a pioneering large-scale synthetic dataset that specifically focuses on transparent objects. It provides a large-scale synthetic dataset as well as a real-world benchmark. The TransCG dataset comprises 57K RGB-D images from 130 different real-world scenes. The ClearPose dataset contains 350K RGB-D images of 63 household objects in real-world settings. 
% We conducted depth completion experiments and generalization verification on ClearGrasp, TransCG, and ClearPose, reported respectively in Section \ref{sec:depth} and \ref{sec:generalization}. We performed an ablation study on TransCG, which is reported in Section \ref{sec:ablation}.

% ClearGrasp\cite{sajjan2020clear} is the first large-scale synthetic dataset as well as a real-world test benchmark focusing on transparent objects. TransCG\cite{fang2022transcg} is a large-scale real-world dataset, which contains 57K RGB-D images from 130 different scenes. ClearPose\cite{chen2022clearpose} is a recentily proposed real-world dataset, containing 350K RGB-D images covering 63 household objects.

% \newgeometry{letterpaper,top=60pt,bottom=43pt,left=48pt,right=48pt}
% \begin{table*}[!t]
% \caption{Ablation study. We show the impact of progressively substituting the components of the DFNet with ours. \label{tab:table1}
% }
% \centering
% \resizebox{\linewidth}{!}{%
% \begin{tabular}{cccccccccc}
% \toprule
% Model/Metric    & RMSE  & REL   & MAE   & $\delta$1.05 & $\delta$1.10 & $\delta$1.25          & Inference time (s)& Parameters & Size (MB)   \\ \midrule
% DFNet\cite{fang2022transcg}          & 0.018 & 0.027 & 0.012 & 83.76 & 95.67 & 99.71          & 0.0244s        & 1.25M & 4.819MB \\ \midrule
% New Loss        & 0.017 & 0.026 & 0.012 & 84.42 & 96.30 & \textbf{99.81} & 0.0244s        & 1.25M & 4.819MB \\ \midrule
% Shortcut Fusion & 0.017 & 0.024 & 0.011 & 86.18 & 96.67 & 99.79          & 0.0218s        & 1.02M & 3.919MB \\ \midrule
% Ours(slim) & 0.016          & 0.024          & 0.011          & 86.22          & 96.64          & \textbf{99.81} & \textbf{0.0143s} & \textbf{0.39M} & \textbf{1.518MB} \\ \midrule
% Ours       & \textbf{0.015} & \textbf{0.022} & \textbf{0.010} & \textbf{88.18} & \textbf{97.15} & \textbf{99.81} & 0.0153s          & 1.25M          & 4.803MB          \\
% \bottomrule
% \end{tabular}%
% }
% \end{table*}
\begin{table}[!t]
\renewcommand{\arraystretch}{1.05}
\setlength{\tabcolsep}{5pt}
\caption{Ablation study. We show the impact of progressively substituting the components of the DFNet with ours. \label{tab:table1}
}
\centering
\resizebox{\linewidth}{!}{%
\begin{threeparttable}
\begin{tabular}{cccccccccc}
\toprule
Model   & RMSE  & REL   & MAE   & $\delta$1.05 & $\delta$1.10 & $\delta$1.25          & Time(s)& Para(M) & Size (MB)   \\ \midrule
DFNet\cite{fang2022transcg}          & 0.018 & 0.027 & 0.012 & 83.76 & 95.67 & 99.71          & 0.0244        & 1.25 & 4.819 \\ \midrule
Huber Loss &0.017   &0.027  &0.012  &84.10  &95.82  &99.74 &0.0244  &1.25   &4.819  \\ \midrule
New Loss        & 0.017 & 0.026 & 0.012 & 84.42 & 96.30 & \textbf{99.81} & 0.0244        & 1.25 & 4.819 \\ \midrule
SF* & 0.017 & 0.024 & 0.011 & 86.18 & 96.67 & 99.79          & 0.0218        & 1.02 & 3.919 \\ \midrule
Ours(s)* & 0.016          & 0.024          & 0.011          & 86.22          & 96.64          & \textbf{99.81} & \textbf{0.0143} & \textbf{0.39} & \textbf{1.518} \\ \midrule
Ours       & \textbf{0.015} & \textbf{0.022} & \textbf{0.010} & \textbf{88.18} & \textbf{97.15} & \textbf{99.81} & 0.0153          & 1.25          & 4.803          \\
\bottomrule
\end{tabular}%
% \multicolumn{10}{l}{Note: NL* represents New Loss, SF* represents Shortcut Fusion and Ours(s)* represents Ours(slim).}
\begin{tablenotes}
\footnotesize
\item Note: SF* represents Shortcut Fusion and Ours(s)* represents Ours(slim).
\end{tablenotes}

\end{threeparttable}
}


\end{table}
% \vspace{-0.5cm}
\subsubsection{Metrics}
For evaluating the performance of our depth completion model, we employ four common metrics: RMSE, REL, MAE and Threshold $\delta$ (where $\delta$ is set to 1.05, 1.10, and 1.25). These metrics are calculated only on the transparent areas, as determined by transparent masks.
% Me use common metrics RMSE, REL, MAE and Threshold $\delta$ ($\delta$ is set to 1.05, 1.10 and 1.25) to evaluate our model. All metrics are calculated on the transparent areas according to transparent masks.


% We use three metrics to evaluate performance on pose estimation task. The average closest point distance (ADD-S)\cite{xiang2017posecnn} calculates the mean distance from each 3D model point to its closest neighbor on the target model. Followed DenseFusion\cite{wang2019densefusion} we report the area under the ADD-S curve (AUC) and the percentage of ADD-S smaller than 2cm ($<$2cm).

\subsection{Implementation Details}
% \noindent
% \textbf{Network configuration.}
\subsubsection{\bf Network Configuration}
% \textcolor{blue}{
In the network architecture, the number of hidden channels, \textbf{$C$}, is set to 64. Each FFEB/DFCB contains a single OSA module. Each OSA module is composed of 5 layers with stage channels of 20. The SFM module maintains \textbf{$C$} channels throughout the pipeline, while cross-layer shortcuts have only 1 channel. Residual connections between the encoder and decoder retain only \textbf{$C$} channels. The input head module and output head module use $3\times3$ convolution to adjust the number of channels and resolution (with resolution changes only occurring in the input head module). For the slim version, \textbf{$C$} is set to 32, and the OSA block contains 4 layers with stage channels of 16.
% }
% The hidden channels \textbf{$C$} in the network is set to 64. Each FFEB/DFCB contains one OSA module, in which, we use 5 layers per block and set stage channels \textbf{$C'$} to 20. SFM keeps \textbf{$C$} channels throughout the pipeline while cross-layer shortcuts take 1 channel only. Residual connections between encoder and decoder just keep channel \textbf{$C$}. $3\times3$ convolution is used in the input head module and the output head module to modify channels and resolution (resolution modified in the input head module only). For slim version, \textbf{$C$} is set to 32, \textbf{$C'$} is set to 16 and uses 4 layers per OSA block.

\subsubsection{\bf Training Details}
% \noindent
% \textbf{Training details.}
All experiments are carried out using the AdamW optimizer with an initial learning rate of $10^{-3}$. The learning rate is reduced by half after 5, 15, 25, and 35 epochs, and training continues for a total of 40 epochs with a batch size of 32. The threshold $\delta$ is kept constant at 0.1 during the training process. The weights $\alpha$ and $\beta$ for the loss function are set to 0.1 and 0.001, respectively. The images are resized to $320\times240$ for both training and testing. The experiments were conducted using an NVIDIA GeForce RTX 3090 GPU.
% We use AdamW optimizer with initial learning rate of $10^{-3}$ and multi-step learning rate scheduler which decays the learning rate by half after 5, 15, 25, 35 epochs. We train the model for 40 epochs with the batch size of 32. Threshold $\delta$ keeps 0.1 during training. Considering loss, we set $\alpha=0.1$, $\beta=0.001$. For all methods, we scale the images to $320\times240$ during training and testing. We use NVIDIA GeForce RTX 3090 for training and testing. 

 % Depth completion task and generalization ability are tested on ClearGrasp, TransCG and ClearPose. Pose estimation task is carried out on the set1 of ClearPose, since Clearpose has an accurate pose annotation without sticker. We use typical network DenseFusion\cite{wang2019densefusion} as pose estimation network. Following the learning strategy of DenseFusion, we train the network on 12G NVIDIA TITAN Xp GPU for 5 epochs with batch size of 128. The margin of refinement is set to 0.03. For fair comparison, we evaluate others works using their released source codes and optimal hyper-parameters or statistics reported in their paper.

\subsection{Ablation study} \label{sec:ablation}
We conduct an ablation study to investigate the effectiveness of our proposed components, including  new loss function, fusion branch, cross-layer shortcut and backbone structure. We take DFNet as baseline method since it is constructed following UNet structure. We  gradually replace its original components by our proposed ones and show the influence of using our proposed components. All the experiments of the ablation study are conducted on TransCG dataset.

% In view that DFNet is also constructed based on UNet, We here gradually replace its original components by our proposed. This study is conducted on TransCG dataset.
% To study the impact of each component in our proposed method, we perform experiments with different configurations of loss functions, network architecture, and backbones. Our method is compared against the recent transparent object depth completion work DFNet, which serves as our baseline. The ablation study experiments are all performed on the TransCG dataset.
% To verify the effectiveness of each component in our method, we evaluate the performance w.r.t. different configurations of loss functions, network architecture, and backbones. We use recently proposed transparent objects depth completion work DFNet as baseline. Ablation study is carried out on TransCG.




\subsubsection{\bf Loss Function}
The training of DFNet employs the mean squared error (MSE) and smooth loss as its loss function. However, these simple loss functions can lead to overfitting to local features, which makes the model more sensitive to the noise from low-level features such as edges and positions, negatively impacting its accuracy. To validate our proposed loss function, we first replaced the MSE loss with Huber loss in DFNet and termed it as Huber Loss. And then we replaced the loss function of DFNet with ours, leaving all other aspects unchanged and termed it as New Loss in Table \ref{tab:table1}. It can be observed by comparing New Loss with DFNet that all metrics showed improvement without requiring any additional parameters. 

% Qualitatively, the use of our proposed loss function can let the network to concentrate on the global structure rather than local details. By comparing the rows 3 and 4 of Figure \ref{fig:figure5}, the boundaries become smoother and even less distinct.
% The training of DFNET uses MSE and cosine distance. The simple loss function may lead to overfit to local features during training. This makes the model more sensitive to the noise of low-level features such as edge and position, which in turn affects its accuracy. So we propose a loss function consisting of Huber loss, SSIM loss and Smooth loss to suppress it. To verify its validity, we replaced the loss function of DFNet with ours and remain its other parts unchanged, then compared the results output by the mixed model (New Loss in Table \ref{tab:table1}) with the original one.
% All metrics are improved without extra parameters. Furthermore, we manually designed a feature to describe those pixels by computing the gradient of depth image and doing Gaussian blur to form an 'edge mask'. As their wights drop, the performance of the model is improved (Edge weight modified in Table \ref{tab:table2}), suggesting that it is necessary to treat pixels differently.
%and lower their weight during training. Specifically, we compute the gradient of depth image and do gaussian blur to form an 'edge mask'. Result (Edge weight modified in Table \ref{tab:table2}) supports our idea and shows it is necessary to treat pixels differently. 

\subsubsection{\bf Fusion Branch and Cross-layer Shortcuts}
In order to evaluate the impact of our proposed fusion branch and cross-layer shortcuts, we make changes to DFNet's architecture. First, we remove the redundant CDC blocks in DFNet from its skip connections, in line with our insight of preserving low-level features and the purpose of light weighting. Then, we added cross-layer shortcuts and a fusion branch to the modified network. It can be seen in Table \ref{tab:table1} that adopting this new architecture (referred to as Shortcut Fusion), almost all metrics show improvement with fewer parameters. 

\subsubsection{\bf Backbone}
We finally replace the denseblock in DFNet with our OSA module and utilized max pooling as the downsampling method. This final modification has transformed DFNet into our network. As shown in Table \ref{tab:table1}, our network outperforms the previous state-of-the-art (SOTA) by at least 16\% on difference-based metrics and improves ratio-based metrics by up to 4.42\%, resulting in a new SOTA performance. To make it practical for low-power robots, we created a slim version to balance speed and accuracy. 


% Qualitatively, figure \ref{fig:figure5} shows our method predicts clearer edges and is better handling crowded area.

% The fusion branch in our proposed network introduces a rich collection of low-level features, while the OSA module promotes feature reuse. Additionally, raw depth information is provided throughout the network, which enhances the representation of low-level features but may also hinder the learning of high-level semantic information. Our hypothesis is that the use of max pooling as a less aggressive downsampling method can mitigate these side effects while also reducing the number of parameters. The results in Table \ref{tab:table2} support our viewpoint.
% We fianlly relace the denseblock in DFNet by our used OSA module, and use max pooling as downsampling method. After this final modification, DFNet is tranformed to our proposed network. We thus show the performance by :Our"  in Table \ref{tab:table1}. It can be observed that ours outperforms previous SOTA by at least 16\% on difference-based metrics and improves ratio-based metrics by 0.1\% to 4.42\%, achieving the new state-of-the-art performance. In order to be capable in real applications, we also construct a slim version for speed/accuracy trade-off. 
% As we mentioned above, fusion branch introduces abundant low-level features and OSA encourages feature reuse. Furthermore, Raw depth is provided throughout the network. They enrich the representation of low-level features but may also harm to the learning of high-level semantic information. We suppose that using maxpooling to loosely downsampling may reduce their side effects as well as parameters saving. Result in Table \ref{tab:table2} proved our point of view.

% For summary, with our loss function, network tend to learn high-level features, with fusion branch, raw depth image and shortcuts, network can take advantage of low-level features. These components working together gives the network ability to take into account both local details and global structures. OSA module and max-pooling downsampling accelerate inference speed and reduce side effects.




% To intuitively show the impact of the proposed components, we visualize the predicted depth on TransCG and CleargGrasp dataset in Figure \ref{fig:figure5}. All networks are trained on TransCG dataset. Qualitatively, with our loss function, network is likely to focus on global structure rather than local detail. Red rectangle in row 3 and 4 show that with our loss function, boundaries become smoothy and even ambiguous, and outliers in the bottom right corner of the second column are suppressed. 



% FDCT performs domain adaption to the concatenation of raw depth and deep features and adopts maxpooling to lossly downsampling. It is supposed to reduce the disadvantage of the inaccuracy of raw depth. Our method predicts more accuracy and smooth edge as shown by the red circle on the left and the black square on the right. And even correct the ground truth as depicted in black circle on the right. The light spot reflected on the apple significantly affects the performance in row 2,3,5, but has little impact on row 4,6. Our methods successfully overcome the side effect of the raw depth information.

\subsection{Depth Completion Experiments} \label{sec:depth}

We compare our method with others on synthetic dataset ClearGrasp and real-world dataset TransCG. The quantitative results are respectively reported in Table \ref{tab:table2} and Table \ref{tab:table3}. Our proposed network surpasses others in almost every metric on these datasets which contain  synthetic and real-world scenes. Our method achieves a new state-of-the-art performance with a smaller model size and faster inference time, making it a highly competitive solution in this field.
%except on ClearGrasp synthetic validation set. It may be result of that the local implicit depth function which is environment-dependent, as well as the extra training data. 

% {\color{blue}
Specifically, our method outperforms the other methods by a larger margin in terms of REL and $\delta1.05$ metrics. This indicates its robustness to noise in the raw depth information, as these metrics are computed based on relative values and are sensitive to noise. Additionally, the gap between our method and others is larger in tests involving novel objects in ClearGrasp (CG Syn-novel in Table \ref{tab:table4} and the ClearGrasp column in Figure \ref{fig:figure5}), indicating that our method has a better ability to generalize to unseen objects. The qualitative results is reported in Figure \ref{fig:figure5}. The prediction of our method exhibits a clearer boundary and finer details than DFNet.
% }
% Specifically, our method has a bigger gap in REL and $\delta1.05$ to others most of the time. It demonstrates that our method is more stable to the noise in raw depth information of pixels, because these metrics are computed by relative value and significantly affected by noise. Noteworthy, the gap between our method and others getting bigger in the test of novel objects in most cases, indicates our method is able to generalize better to unseen objects.

\begin{table}[!t]
\caption{Depth Completion Result on TransCG dataset.}
\label{tab:table2}

\centering
\resizebox{\linewidth}{!}{%
\begin{tabular}{ccccccccc}
\toprule
Model & RMSE  & REL   & MAE   & $\delta1.05$ & $\delta1.10$ & $\delta1.25$ & Time ($\second$)   & Size ($\mega$B)    \\ \midrule
ClearGrasp\cite{sajjan2020clear}   & 0.054 & 0.083 & 0.037 & 50.48 & 68.68 & 95.28 & 2.281          & 934          \\
LIDF-Refine\cite{zhou2021pr}  & 0.019 & 0.034 & 0.015 & 78.22 & 94.26 & 99.80 & 0.018          & 251          \\
DFNet\cite{fang2022transcg}        & 0.018 & 0.027 & 0.012 & 83.76 & 95.67 & 99.71 & 0.024          & 4.8          \\
Ours (slim)   & 0.017 & 0.025 & 0.011 & 85.53 & 96.46 & 99.79 & \textbf{0.014} & \textbf{1.6} \\
Ours & \textbf{0.015} & \textbf{0.022} & \textbf{0.010} & \textbf{88.18} & \textbf{97.15} & \textbf{99.81} & 0.015 & 4.8 \\ \bottomrule
\end{tabular}}
% \vspace{-0.5cm}
\end{table}


\begin{table}[!t]
\renewcommand{\arraystretch}{0.9}
\caption{Depth Completion Results on ClearGrasp dataset\label{tab:table3}}
\centering
\resizebox{\linewidth}{!}{%
\begin{tabular}{ccccccc}
\toprule
\multicolumn{1}{c}{Model/Metric} &
  \multicolumn{1}{c}{RMSE} &
  \multicolumn{1}{c}{REL} &
  \multicolumn{1}{c}{MAE} &
  \multicolumn{1}{c}{$\delta$1.05} &
  \multicolumn{1}{c}{$\delta$1.10} &
  $\delta$1.25 \\ \midrule
\multicolumn{7}{c}{Train CG Test CG Syn-novel} \\ \midrule
\multicolumn{1}{c}{ClearGrasp} &
  \multicolumn{1}{c}{0.040} &
  \multicolumn{1}{c}{0.071} &
  \multicolumn{1}{c}{0.035} &
  \multicolumn{1}{c}{42.95} &
  \multicolumn{1}{c}{80.04} &
  98.10 \\ 
\multicolumn{1}{c}{Local Implicit} &
  \multicolumn{1}{c}{\underline{0.028}} &
  \multicolumn{1}{c}{\underline{0.045}} &
  \multicolumn{1}{c}{\underline{0.023}} &
  \multicolumn{1}{c}{\underline{68.62}} &
  \multicolumn{1}{c}{\underline{89.10}} &
  \underline{99.20} \\ 
\multicolumn{1}{c}{DFNet} &
  \multicolumn{1}{c}{0.032} &
  \multicolumn{1}{c}{0.051} &
  \multicolumn{1}{c}{0.027} &
  \multicolumn{1}{c}{62.59} &
  \multicolumn{1}{c}{84.37} &
  98.39 \\ 
\multicolumn{1}{c}{FDCT (Ours)} &
  \multicolumn{1}{c}{\textbf{0.025}} &
  \multicolumn{1}{c}{\textbf{0.040}} &
  \multicolumn{1}{c}{\textbf{0.021}} &
  \multicolumn{1}{c}{\textbf{71.66}} &
  \multicolumn{1}{c}{\textbf{92.95}} &
  \textbf{99.64} \\ \midrule
\multicolumn{7}{c}{Train CG Test CG Syn-known} \\ \midrule
\multicolumn{1}{c}{Local Implicit} &
  \multicolumn{1}{c}{\textbf{0.012}} &
  \multicolumn{1}{c}{\textbf{0.017}} &
  \multicolumn{1}{c}{\textbf{0.009}} &
  \multicolumn{1}{c}{\textbf{94.79}} &
  \multicolumn{1}{c}{\textbf{98.52}} &
  99.67 \\ 
\multicolumn{1}{c}{ClearGrasp} &
  \multicolumn{1}{c}{0.044} &
  \multicolumn{1}{c}{0.047} &
  \multicolumn{1}{c}{0.033} &
  \multicolumn{1}{c}{71.23} &
  \multicolumn{1}{c}{92.60} &
  98.24 \\ 
\multicolumn{1}{c}{DFNet} &
  \multicolumn{1}{c}{0.018} &
  \multicolumn{1}{c}{0.023} &
  \multicolumn{1}{c}{0.013} &
  \multicolumn{1}{c}{88.85} &
  \multicolumn{1}{c}{97.57} &
  \underline{99.92} \\ 
\multicolumn{1}{c}{FDCT (Ours)} &
  \multicolumn{1}{c}{\underline{0.015}} &
  \multicolumn{1}{c}{\underline{0.020}} &
  \multicolumn{1}{c}{\underline{0.012}} &
  \multicolumn{1}{c}{\underline{90.53}} &
  \multicolumn{1}{c}{\underline{98.21}} &
  \textbf{99.99} \\ \bottomrule

\end{tabular}%
% \tablen}
}
\end{table}



\subsection{Generalization Experiment} \label{sec:generalization}
% The generalization capability of a network is essential for practical applications. We evaluated the generalization ability of our proposed method from two perspectives: from synthetic images to real-world images and from one real-world dataset to another. The results of our experiments, shown in Table \ref{tab:table6}, indicate that our method (FDCT) has a comparable generalization capability to the state-of-the-art methods in cross-dataset evaluations, and it outperforms similar works in the synthetic-to-real test. However, it lags behind methods that focus solely on sim-to-real (noted as "local implicit*").
% The generalization ability of a network is critical for real-world application. The proposed method has a generalization ability that can be trained on synthetic data and aply to real world scene (syn-to-real) or trained on one real world dataset TransCG and adap to ClearGrasp (real-to-real). Comparison result is reported in Table \ref{tab:table4}. It shows that although there is still a certain gap compared with the method Local Implicit designed for syn-to-real; compared with the similar method DFNet, our method achieves a better result in the syn-to-real setting, and a competitive result in the syn-to-syn setting.
The generalization ability of a network is critical for real-world application. Our proposed method exhibits a high degree of generalization, being able to be trained on synthetic data and applied to real-world scenes (syn-to-real), or trained on one real-world dataset TransCG and adapted to the other real-world dataset (real-to-real), such as ClearGrasp. Comparison results are reported in Table \ref{tab:table4}, which show that while there is still a certain gap compared to the syn-to-real method (Local Implicit \cite{zhu2021rgb}), our method achieves better results in the syn-to-real setting when compared to the similar method DFNet, and competitive results in the real-to-real setting.

% We inspect the generalization ability of our proposed method from two aspects, from synthetic image to real-world image and from one real-world dataset to another. Experiment results in Table \ref{tab:table5} show that FDCT has a similar generalization ability to previous SOTA in cross-dataset and get better result in synthetic-to-real test compared to similar work, but is far below to methods focusing on sim-to-real.

% Since both datasets comprise real-world image, we train models on TransCG and test it on ClearGrasp real-world set for cross-dataset test. DFNet outperformed other method with a huge gap in generalization test and is chosen to be compared with ours. Comparison result is reported in Table \ref{tab:table5}. Our method outperforms the closest work in all metrics both for known and novel objects in synthetic-to-real test. There is a bigger gap between DFNet and ours in terms of novel objects. It might owe to a better utilization of RGB cues. Our method gets similar results to DFNet in cross dataset test, showing that our method has the ability to generalize from real-world dataset to another. With a series of real-world transparent objects datasets being proposed, we believe that the generalization ability in real-world is more important than sim-to-real.



% {\color{blue}
% Figure environment removed

\begin{table}[!t]
\caption{
% Result of Synthetic to Real and Cross Dataset Generalization Experiment
Generalization test on syn-to-real and real-to-real.}
\label{tab:table4}
\renewcommand{\arraystretch}{0.95}
\centering
\resizebox{\linewidth}{!}{%
% \begin{threeparttable}
\begin{tabular}{ccclclclclcl}
\toprule
\multicolumn{1}{c}{Model/Metric} &
  \multicolumn{1}{c}{RMSE} &
  \multicolumn{2}{c}{REL} &
  \multicolumn{2}{c}{MAE} &
  \multicolumn{2}{c}{$\delta$1.05} &
  \multicolumn{2}{c}{$\delta$1.10} &
  \multicolumn{2}{c}{$\delta$1.25} \\ \midrule
\multicolumn{12}{c}{Train CG Test CG Real-known (syn-to-real)} \\ \midrule
\multicolumn{1}{c}{Local Implicit\cite{zhu2021rgb}} &
  \multicolumn{1}{c}{\textbf{0.028}} &
  \multicolumn{2}{c}{\textbf{0.033}} &
  \multicolumn{2}{c}{\textbf{0.020}} &
  \multicolumn{2}{c}{\textbf{82.37}} &
  \multicolumn{2}{c}{\textbf{92.98}} &
  \multicolumn{2}{c}{\textbf{98.63}} \\ 
\multicolumn{1}{c}{DFNet} &
  \multicolumn{1}{c}{0.068} &
  \multicolumn{2}{c}{0.107} &
  \multicolumn{2}{c}{0.059} &
  \multicolumn{2}{c}{32.42} &
  \multicolumn{2}{c}{56.88} &
  \multicolumn{2}{c}{91.47} \\ 
\multicolumn{1}{c}{FDCT (Ours)} &
  \multicolumn{1}{c}{\underline{0.065}} &
  \multicolumn{2}{c}{\underline{0.103}} &
  \multicolumn{2}{c}{\underline{0.057}} &
  \multicolumn{2}{c}{\underline{33.08}} &
  \multicolumn{2}{c}{\underline{59.81}} &
  \multicolumn{2}{c}{\underline{91.70}} \\ \midrule
\multicolumn{12}{c}{Train CG Test CG Real-novel (syn-to-real)} \\ \midrule
\multicolumn{1}{c}{Local Implicit\cite{zhu2021rgb}} &
  \multicolumn{1}{c}{\textbf{0.025}} &
  \multicolumn{2}{c}{\textbf{0.036}} &
  \multicolumn{2}{c}{\textbf{0.020}} &
  \multicolumn{2}{c}{\textbf{76.21}} &
  \multicolumn{2}{c}{\textbf{94.01}} &
  \multicolumn{2}{c}{\textbf{99.35}} \\ 
\multicolumn{1}{c}{DFNet} &
  \multicolumn{1}{c}{0.051} &
  \multicolumn{2}{c}{0.088} &
  \multicolumn{2}{c}{0.046} &
  \multicolumn{2}{c}{31.23} &
  \multicolumn{2}{c}{64.66} &
  \multicolumn{2}{c}{97.77} \\ 
\multicolumn{1}{c}{FDCT (Ours)} &
  \multicolumn{1}{c}{\underline{0.043}} &
  \multicolumn{2}{c}{\underline{0.073}} &
  \multicolumn{2}{c}{\underline{0.038}} &
  \multicolumn{2}{c}{\underline{39.42}} &
  \multicolumn{2}{c}{\underline{75.54}} &
  \multicolumn{2}{c}{\underline{99.09}} \\ \midrule
\multicolumn{12}{c}{Train TCG Test CG Real-novel (real-to-real)} \\ \midrule
\multicolumn{1}{c}{Local Implicit\cite{zhu2021rgb}} &
  \multicolumn{1}{c}{0.152} &
  \multicolumn{2}{c}{0.225} &
  \multicolumn{2}{c}{0.139} &
  \multicolumn{2}{c}{9.86} &
  \multicolumn{2}{c}{20.63} &
  \multicolumn{2}{c}{46.02} \\ 
\multicolumn{1}{c}{DFNet} &
  \multicolumn{1}{c}{\textbf{0.041}} &
  \multicolumn{2}{c}{\textbf{0.054}} &
  \multicolumn{2}{c}{\textbf{0.031}} &
  \multicolumn{2}{c}{\textbf{62.74}} &
  \multicolumn{2}{c}{\textbf{83.31}} &
  \multicolumn{2}{c}{\textbf{97.33}} \\ 
\multicolumn{1}{c}{FDCT (Ours)} &
  \multicolumn{1}{c}{\textbf{0.041}} &
  \multicolumn{2}{c}{\underline{0.055}} &
  \multicolumn{2}{c}{\underline{0.032}} &
  \multicolumn{2}{c}{\underline{61.23}} &
  \multicolumn{2}{c}{\underline{82.84}} &
  \multicolumn{2}{c}{\underline{97.28}} \\ \bottomrule
\end{tabular}
%     \begin{tablenote}
%         \footnotesize
%         \item [*]Local Implicit is method aiming at sim-to-real.
%     \end{tablenote}
% \end{threeparttable}
}
%\vspace{-0.5cm}
\end{table}
% Figure environment removed
\subsection{Analysis} \label{sec:analysis}
In our proposed method, the loss function plays a crucial role in enabling the network to focus on structural information and alleviate the effects of unstable pixels. However, this focus on structural information may come at the expense of some details. On the other hand, the fusion branch and shortcuts draw attention to the details, which can introduce extra redundancy. Nonetheless, the use of maxpooling facilitates lossy and aggressive downsampling, which can reduce redundancy and improve robustness. The convolution based fusion method make better use of the raw depth image. All components work together and complement each other to achieve the best possible balance between structural information and details. In this section, we analyze the four critical components of our method and demonstrate their effectiveness.

\subsubsection{Influence of loss term}
% As we mentioned above, some unstable pixels can unwantedly make big penalty to the loss. By computing the gradient of the depth image and applying Gaussian blur, we manually created a feature to represent these pixels. As the weights of these pixels were reduced, the model's performance improved (as seen in Experiment of weight in Table \ref{tab:table5}), indicating the importance of treating pixels differently and pointing out the necessity of the so designed loss function. However, the side effect of such loss function is that the network pays too much attention to the structure and ignores some details. The highlighted area of the feature map changes from dotted to regional in the Loss column in Figure \ref{fig:figure6}.
As mentioned in \ref{section:Loss}, unstable pixels can have a significant negative influence on the calculation of the training loss. To illustrate this issue, we manually created a feature to represent these pixels by computing the gradient of the depth image and applying a Gaussian blur. By reducing the weights of these pixels, we observed an improvement in the model's performance (as seen in the Experiment of weight in Table \ref{tab:table5}), highlighting the importance of treating pixels differently and emphasizing the necessity of the used loss functions (especially the Huber Loss). Qualitatively, as shown in Figure \ref{fig:figure6}, the New Loss model places greater emphasis on the overall structure of transparent objects, as compared to DFNet, which primarily focuses on local information. The downside of such a loss function is that the network may ignore some details.
% Figure environment removed

\subsubsection{Low-level feature preservation}
% Fusion branch and cross-layer shortcuts alleviate the indistinct boundaries and perceptual details by taking more low-level cues into consideration. The highlighted area of the feature map changes from regional to scattered in the Fusion column in Figure \ref{fig:figure6}. Loss function and low-level feature awareness components together make a good trade-off between detail and structure information.
The fusion branch and cross-layer shortcuts help alleviate the issue of blurry boundaries and low perceptual details by incorporating more low-level cues. As a result, more low-level features such as object edges and holes are preserved in the feature map of Fusion model in Figure \ref{fig:figure6}. The combination of the loss function and low-level feature awareness components strikes a good balance between detail and structural information.

\subsubsection{Influence of downsampling}
Our hypothesis is that the use of max pooling as a lossy downsampling method can mitigate the side effects of the low-level awareness components while reducing the number of parameters. The results in Table \ref{tab:table5} that are noted as ``Experiment of downsampling'' support our viewpoint. It can be observed that the performance of using convolutional downsampling and average pooling is slightly worse than that of using max pooling.

% The loss function makes the network focus on structural information and alleviating the affects of unstable pixels, but may harming to the details. The fusion branch and shortcuts draws the attention to details, but may introduce extra redundancy. Maxpooling is used to lossy and aggressively downsampling. It can reduce redundancy and improve robustness. These components work together and complement each other.
% }

\subsubsection{Fusion method of depth image}
To demonstrate that fusing the raw depth image with feature map via convolution is better than directly concatenation. We removed the convolution layers used for fusion in the model Ours and named it Ours(concat). The result labeled Table ``Experiment on fusion method'' in Table \ref{tab:table5} support our viewpoint.

\begin{table}[!ht]
\centering
\caption{Experiment Result on Weight Modification, Downsampling Implementation and Fusion Method\label{tab:table5}}

\resizebox{\linewidth}{!}{%
\begin{tabular}{ccccccc}
\toprule
\multicolumn{1}{c}{Model/Metric} &
  \multicolumn{1}{c}{RMSE} &
  \multicolumn{1}{c}{REL} &
  \multicolumn{1}{c}{MAE} &
  \multicolumn{1}{c}{$\delta$1.05} &
  \multicolumn{1}{c}{$\delta$1.10} &
  $\delta$1.25 \\ \midrule
\multicolumn{7}{c}{Experiment on weight} \\ \midrule
\multicolumn{1}{c}{Baseline} &
  \multicolumn{1}{c}{0.018} &
  \multicolumn{1}{c}{0.027} &
  \multicolumn{1}{c}{0.012} &
  \multicolumn{1}{c}{83.76} &
  \multicolumn{1}{c}{95.67} &
  99.71 \\ 
\multicolumn{1}{c}{Edge Weight Modified} &
  \multicolumn{1}{c}{\textbf{0.017}} &
  \multicolumn{1}{c}{\textbf{0.025}} &
  \multicolumn{1}{c}{\textbf{0.011}} &
  \multicolumn{1}{c}{\textbf{85.34}} &
  \multicolumn{1}{c}{\textbf{96.26}} &
  \textbf{99.75} \\ \midrule
\multicolumn{7}{c}{Experiment on downsampling} \\ \midrule
\multicolumn{1}{c}{Conv Down} &
  \multicolumn{1}{c}{0.016} &
  \multicolumn{1}{c}{0.023} &
  \multicolumn{1}{c}{0.011} &
  \multicolumn{1}{c}{87.16} &
  \multicolumn{1}{c}{96.83} &
  99.80 \\ 
\multicolumn{1}{c}{AvgPooling Down} &
  \multicolumn{1}{c}{0.016} &
  \multicolumn{1}{c}{0.024} &
  \multicolumn{1}{c}{0.011} &
  \multicolumn{1}{c}{87.16} &
  \multicolumn{1}{c}{96.93} &
  99.80 \\ 
\multicolumn{1}{c}{MaxPooling Down} &
  \multicolumn{1}{c}{\textbf{0.015}} &
  \multicolumn{1}{c}{\textbf{0.022}} &
  \multicolumn{1}{c}{\textbf{0.010}} &
  \multicolumn{1}{c}{\textbf{88.18}} &
  \multicolumn{1}{c}{\textbf{97.15}} &
  \textbf{99.81} \\ \midrule
  \multicolumn{7}{c}{Experiment on fusion method} \\ \midrule
  \multicolumn{1}{c}{Ours(concat)} &
  \multicolumn{1}{c}{\textbf{0.015}} &
  \multicolumn{1}{c}{0.023} &
  \multicolumn{1}{c}{0.011} &
  \multicolumn{1}{c}{87.90} &
  \multicolumn{1}{c}{96.68} &
  99.80 \\ 
\multicolumn{1}{c}{Ours} &
  \multicolumn{1}{c}{\textbf{0.015}} &
  \multicolumn{1}{c}{\textbf{0.022}} &
  \multicolumn{1}{c}{\textbf{0.010}} &
  \multicolumn{1}{c}{\textbf{88.18}} &
  \multicolumn{1}{c}{\textbf{97.15}} &
  \textbf{99.81} \\ 
\bottomrule
\end{tabular}%
}
%\vspace{-0.5cm}
\end{table}
\vspace{-0.2cm}


\subsection{Pose Estimation Experiment}
In this experiment, we aim to demonstrate the applicability of our network for downstream tasks and to show that it can improve the accuracy of pose estimate.
To evaluate the performance of pose estimation, we use three evaluation metrics, i.e, the average closest point distance (ADD-S), the area under the ADD-S curve (AUC), and the percentage of ADD-S values that are smaller than 2 \centi\meter.
%\cite{xiang2017posecnn}
% The higher the metrics the stronger the performance.

% This experiment is carried out on the set1 of ClearPose, since Clearpose has an accurate pose annotation without sticker. We use typical network DenseFusion \cite{wang2019densefusion} as pose estimation network. Following the learning strategy of DenseFusion, we train the network on 12G NVIDIA TITAN Xp GPU for 5 epochs with batch size of 128. The margin of refinement is set to 0.03. For fair comparison, we evaluate others works using their released source codes and optimal hyper-parameters or statistics reported in their paper.
Both our method and DFNet are trained on the ClearPose Set 1 and are used to predict the depth of Set 1-Scene 5 for pose estimation purposes. The depth completion result is reported in Table \ref{tab:table6} and a screenshot of the live demonstration is reported in Figure \ref{fig:figure7}. In our experiments, we use DenseFusion \cite{wang2019densefusion}  as the pose estimation method. We trained DenseFusion with the restored depth and tested it on 3,000 randomly selected images. Ideally, a more accurate depth prediction can lead to improved performance in pose estimation. The results of our evaluations, presented in Table \ref{tab:table7}, indicate that the depth restored by our method outperforms DFNet in almost every object in the pose estimation task. This results validate that the depth map given by our method is more appropriate for addressing the downstream task, i.e., pose estimation.
% Depth completion models are trained on ClearPose set 1 and predict the depth of set 1-scene 5 for pose estimation. We train DenseFusion with the restored depth and test on 3k randomly chosen images. Metrics for each object are reported in Table \ref{tab:table7}. Result shows that the depth restored by FDCT outperforms DFNet's in almost every object in pose estimation task.
% \todo{format of tablehead!!}
\begin{table}[!t]
\caption{Depth Completion Results on ClearPose dataset.}
\label{tab:table6}
\centering
\begin{tabular}{ccccccc}
\toprule
Model & RMSE           & REL            & MAE            & $\delta$1.05          & $\delta$1.10          & $\delta$1.25          \\ \midrule
DFNet        & 0.048          & 0.038          & 0.033          & 76.36          & 94.22          & \textbf{99.40} \\
Ours         & \textbf{0.045} & \textbf{0.033} & \textbf{0.028} & \textbf{82.15} & \textbf{94.43} & 99.25          \\
\bottomrule
\end{tabular}%
\end{table}



\begin{table}[!t]
\caption{Pose Estimation Results on ClearPose dataset\label{tab:table7}}
\centering
\resizebox{\linewidth}{!}{%
\begin{tabular}{ccccccc}
\toprule
Models &
  \multicolumn{3}{c}{DFNet} &
  \multicolumn{3}{c}{Ours} \\ \midrule
Object/Metirc &
  \multicolumn{1}{c}{AUC} &
  \multicolumn{1}{c}{\textless{}2cm} &
  ADD-S(10\%) &
  \multicolumn{1}{c}{AUC} &
  \multicolumn{1}{c}{\textless{}2cm} &
  ADD-S(10\%) \\ 
beaker\_1 &
  \multicolumn{1}{c}{79.07} &
  \multicolumn{1}{c}{\textbf{0.00}} &
  0.68 &
  \multicolumn{1}{c}{\textbf{80.44}} &
  \multicolumn{1}{c}{\textbf{0.00}} &
  \textbf{7.53} \\ 
dropper\_1 &
  \multicolumn{1}{c}{\textbf{67.76}} &
  \multicolumn{1}{c}{61.00} &
  \textbf{48.00} &
  \multicolumn{1}{c}{31.70} &
  \multicolumn{1}{c}{\textbf{65.33}} &
  0.00 \\ 
dropper\_2 &
  \multicolumn{1}{c}{81.09} &
  \multicolumn{1}{c}{\textbf{33.10}} &
  1.78 &
  \multicolumn{1}{c}{\textbf{84.24}} &
  \multicolumn{1}{c}{0.00} &
  \textbf{9.61} \\ 
flask\_1 &
  \multicolumn{1}{c}{84.96} &
  \multicolumn{1}{c}{60.33} &
  42.33 &
  \multicolumn{1}{c}{\textbf{86.71}} &
  \multicolumn{1}{c}{\textbf{68.33}} &
  \textbf{68.00} \\ 
funnel\_1 &
  \multicolumn{1}{c}{78.85} &
  \multicolumn{1}{c}{91.33} &
  0.00 &
  \multicolumn{1}{c}{\textbf{82.91}} &
  \multicolumn{1}{c}{\textbf{98.33}} &
  \textbf{12.33} \\ 
cylinder\_1 &
  \multicolumn{1}{c}{78.77} &
  \multicolumn{1}{c}{48.33} &
  28.67 &
  \multicolumn{1}{c}{\textbf{79.83}} &
  \multicolumn{1}{c}{\textbf{77.00}} &
  \textbf{33.33} \\ 
cylinder\_2 &
  \multicolumn{1}{c}{62.75} &
  \multicolumn{1}{c}{54.67} &
  3.33 &
  \multicolumn{1}{c}{\textbf{75.68}} &
  \multicolumn{1}{c}{\textbf{58.67}} &
  \textbf{29.33} \\ 
pan\_1 &
  \multicolumn{1}{c}{86.76} &
  \multicolumn{1}{c}{13.67} &
  33.33 &
  \multicolumn{1}{c}{\textbf{89.37}} &
  \multicolumn{1}{c}{\textbf{53.67}} &
  \textbf{50.00} \\ 
pan\_2 &
  \multicolumn{1}{c}{88.71} &
  \multicolumn{1}{c}{84.67} &
  44.00 &
  \multicolumn{1}{c}{\textbf{89.73}} &
  \multicolumn{1}{c}{\textbf{90.33}} &
  \textbf{56.00} \\ 
pan\_3 &
  \multicolumn{1}{c}{\textbf{88.90}} &
  \multicolumn{1}{c}{87.67} &
  \textbf{53.33} &
  \multicolumn{1}{c}{88.10} &
  \multicolumn{1}{c}{\textbf{91.00}} &
  48.00 \\ 
bottle\_1 &
  \multicolumn{1}{c}{86.05} &
  \multicolumn{1}{c}{91.53} &
  24.41 &
  \multicolumn{1}{c}{\textbf{88.71}} &
  \multicolumn{1}{c}{\textbf{93.22}} &
  \textbf{31.53} \\ 
bottle\_2 &
  \multicolumn{1}{c}{71.81} &
  \multicolumn{1}{c}{83.16} &
  4.04 &
  \multicolumn{1}{c}{\textbf{77.01}} &
  \multicolumn{1}{c}{\textbf{88.22}} &
  \textbf{13.47} \\ 
stick\_1 &
  \multicolumn{1}{c}{69.53} &
  \multicolumn{1}{c}{32.32} &
  32.66 &
  \multicolumn{1}{c}{\textbf{79.60}} &
  \multicolumn{1}{c}{\textbf{57.58}} &
  \textbf{58.92} \\ 
syringe\_1 &
  \multicolumn{1}{c}{73.03} &
  \multicolumn{1}{c}{31.67} &
  25.67 &
  \multicolumn{1}{c}{\textbf{80.15}} &
  \multicolumn{1}{c}{\textbf{57.00}} &
  \textbf{47.00} \\ 
MEAN &
  \multicolumn{1}{c}{78.43} &
  \multicolumn{1}{c}{55.25} &
  24.45 &
  \multicolumn{1}{c}{\textbf{79.58}} &
  \multicolumn{1}{c}{\textbf{64.19}} &
  \textbf{33.22} \\


  \bottomrule
  \end{tabular}%
}
\vspace{-0.5cm}
\end{table}
\section{Conclusion and Future Work}
In this work, I design corruption-robust algorithms for the Lipschitz contextual search problem. I present the \emph{agnostic checking} technique and demonstrate its effectiveness in designing corruption-robust algorithms. There are several open problems for future research. First, in the algorithm I propose for pricing loss, the schedule for agnostic checks is fixed upfront. Can the learner design an adaptive checking schedule for the pricing loss? Second, this work assumes the learner has knowledge of the Lipschitz constant $L$. Can the learner design efficient no-regret algorithms without knowledge of $L$? 

\bibliography{references}
\bibliographystyle{icml2023}

\newpage
\appendix 
\onecolumn
\section{Proof of \cref{thm:ahh}}
\label{sec:proof_ahh}
Note that the algorithm can be viewed as $\rep \eqdef \ceil{20\log(\frac{40\nspu\ns R}{\tau\beta})}$ independent runs of a basic protocol, each of which returns a list $\heavy_i$ of potential heavy hitters. We assume $b \ge 260$, else we take $b' = \max\{b, 260\}$ and the result will change by at most a constant factor.

The next lemma states that the probabilities of heavy elements and tail elements falling in the list.
\begin{lemma} \label{lem:single_run}
    All $\heavy_j$ defined in \cref{alg:subsample_IBLT} satisfy that, if $h^{[R]}(x) \ge \thr$,
    \[
    \probof{x \in \heavy_j} \ge 4/5.
    \]
    Else if $h^{[R]}(x) \le \thr/10$, 
    \[
        \probof{x \in \heavy_j} \le \frac{2h^{[R]}(x)}{\thr}.
    \]
\end{lemma}

Before proving the lemma, we first show how \cref{thm:ahh} can be implied by \cref{lem:single_run}. 

By \cref{lem:single_run}, for $x$ with $h^{[R]}(x) \ge \thr$, we have 
\[
    \probof{x \in \heavy} \ge \probof{{\rm Binom}\Paren{\rep, 4/5} \ge \rep/2} \ge 1 - \frac{\beta \tau}{40 \nspu \ns R},
\]
where the last inequality follows from standard concentration bounds for Binomial random variables  (\eg Chernoff bound \cite{Mitzenmacher2017probability}).

Hence by union bound, we have 
\[
    \probof{\{x \in [d] \mid h^{[R]}(x) \ge \thr \} \subset \heavy } \ge 1 - \frac{\beta}{40}.
\]

For any $x$, with $h^{[R]}(x) \le \thr/10$, by \cref{lem:single_run}, we have
\begin{align*}
\probof{x \in \heavy} \le \probof{{\rm Binom}\Paren{\rep, \frac{2h^{[R]}(x)}{\thr}} \ge \rep/2} \le \frac{\rep + 1}{2} \Paren{\frac{8e}{5} \cdot \frac{2h^{[R]}(x)}{\thr}}^{\rep/2},
\end{align*}
where the last inequality follows from Binomial tail bound (see \cref{lem:binomial_tail}).

Hence by union bound we have 
\begin{align}
    & \qquad \probof{\{x \in [d] \mid h^{[R]}(x) \le \thr/10 \} \cap H \neq \emptyset}  \nonumber \\
    & \le \sum_{x: h^{[R]}(x) \le \thr/10} \frac{\rep+1}{2} \Paren{ \frac{16e h^{[R]}(x)}{5\thr}}^{\rep/2}  \nonumber  \\
    & \le \frac{20\nspu \ns R}{\thr} \frac{\rep+1}{2} \Paren{\frac{8e}{25}}^{\rep/2} \label{eqn:combine_elements}\\
    & \le  \frac{20\nspu \ns R}{\thr} e^{-\frac{b}{20}} \label{eqn:algebra}\\
    & \le \frac{\beta}{2}, \nonumber
\end{align}
where \eqref{eqn:combine_elements} follows from $x^{\rep/2} + y^{\rep/2} \le (x + y)^{\rep/2}$, and hence we can combine symbols to increase the sum of tail probability and end up with at most $\frac{20\nspu \ns R}{\thr}$ symbols with frequencies at most $\thr/10$. \eqref{eqn:algebra} follows from the inequality $(x + 1/2) (8e/25)^x \le e^{-x/10}$ for $x \ge 130$.

By union bound, we get the guarantee claimed in \cref{thm:ahh}.

\begin{proofof}{\cref{lem:single_run}}
The proof mainly consists of two parts. We will first show that local subsampling will keep each heavy hitter with a high probability and each tail element with a low probability, stated in \cref{lem:threshold}. We will then show that after local subsampling, the number of unique elements in each round will decrease so that the decoding in \cref{alg:subsample_IBLT} will succeed with high probability.


\begin{lemma}\label{lem:threshold}
    Let $h_j^{'[R]}$ be the aggregation of locally subsampled histogram for run $j$, \ie
    \[
        h_j^{'[R]} = \sum_{r \in [R]} \sum_{i \in B_r} h_{i, j}.
    \]
    Then if $h^{[R]}(x) \ge \tau$, 
    \[
        \probof{h_j^{'[R]}(x) > 0 } \ge 1 - \frac{1}{e^2}. 
    \]
    Else if $h^{[R]}(x) \le \tau/10$,
    \[
        \probof{x \in \heavy_j} \le \frac{2h^{[R]}(x)}{\thr}.
    \]
\end{lemma}
\begin{proof}
When $h^{[R]}(x) \ge \tau$,
\begin{align*}
    \probof{h_j^{'[R]}(x) > 0 } = 1 - \Pi_{r \in [R], i \in B_r} \min \{1 -  \frac{2 h_{i, j}(x)}{\thr}, 0\} \ge 1 - \Pi_{r \in [R], i \in B_r} e^{-\frac{2 h_{i, j}(x)}{\thr}} =  1 -e^{-\frac{2 h^{[R]}(x)}{\thr}} \ge 1 - \frac{1}{e^2}.
\end{align*}
When $h^{[R]}(x) \le \tau/10$
\begin{align*}
     \probof{h_j^{'[R]}(x) > 0 } = 1 - \Pi_{r \in [R], i \in B_r} \Paren{1 -  \frac{2 h_{i, j}(x)}{\thr}} \le 1 - \Paren{1 - \sum_{r \in [R], i \in B_r}\frac{2 h_{i, j}(x)}{\thr}} = \frac{2h^{[R]}(x)}{\thr}.
\end{align*}
\end{proof}

The next lemma shows that with high probability, the number of elements in each round will decrease by least a factor of $\tau$. 
\begin{lemma} \label{lem:max_zero}
    With probability at least $1 - 1/32$, we have
    \[
        \max_{r \in [R]}\left\{ \norm{h'_r}_0 \right\} = O \Paren{\frac{\nspu \ns}{\tau} \log R}.
    \]
\end{lemma}
\begin{proof}
Since all rounds are independent, it would be enough to show that $\forall i$, with probability at least $1 - 1/32R$, we have 
\[
    \norm{h'_r}_0  = O \Paren{\frac{\nspu \ns}{\tau} \log R}.
\]
To see this, we have
\[
    \probof{\norm{h'_r}_0 \ge \frac{2\nspu \ns}{\tau} \log R} \le \probof{{\rm Binom}\Paren{mn, \frac1{\tau}}\ge \frac{2\nspu \ns}{\tau} \log R} \le \frac{1}{32R},
\]
where the first step follows from that the left hand side is maximized when all $\nspu\ns$ elements in $h_r$ are distinct, and the second step follows from standard binomial tail bound when $mn > 4 \tau$ and $R > 32$.
\end{proof}

Finally, it would be enough to show that when the condition in \cref{lem:max_zero} holds, the decoding of the aggregated IBLT will succeed with high probability. This is true since by \cref{lem:iblt} and union bound, we have
\[
    \probof{\forall j, \hat{h}_j^{[R]} = h_j^{'[R]}} \ge 1 - R \cdot (\frac{\nspu \ns}{\tau} \log R)^{2-\gamma} \ge 1 - 1/32,
\]
where the last inequality holds when $mn > 4 \tau$ and $R > 32$.
Combining the above and \cref{lem:max_zero,lem:threshold}, we conclude the proof since $1/e^2 + 1/32 + 1/32 \le 1/5$.
\end{proofof}

\section{Proof of \cref{thm:ahist_r}} \label{sec:ahist_app}

\new{We start with the case when $\tau \le \sqrt{R}$. In this case, \cref{alg:ahist_r} implements \cref{alg:subsample_IBLT} with $\tau = 1$ and returns the obtained histogram in Line 11. Notice that when $\tau = 1$, the subsampling step is trivial and each user encodes their entire histogram. Hence as long as long the decoding of IBLT succeeds (as promised in the performance analysis of \cref{alg:subsample_IBLT}), we recover the histogram perfectly, \ie $\hat{h}^{[R]} = h^{[R]}.$ And the communication cost will be $\tilde{\Theta}(mn)$.
}

Next we focus on the case when $\tau \ge \sqrt{R}.$ We will condition on the event that the list $H$ obtained in Line 8 of \cref{alg:ahist_r} is a $\thr$ approximate heavy hitter set and hence setting $\hat{x} = 0$ for $x \notin H$ won't introduce error larger than $\thr$.

The rest of the proof follows similarly as the standard proof for Count-sketch. Since $\rep = \ceil{\log\Paren{\frac{4\nspu\ns R}{\tau \beta}}}$, it would be enough to prove that $\forall x \in \cX$, with probability at least 2/3, we have
\[
    |\sum_{r \in [R]} T_r(j, \hash_{j}(x)) \cdot s_{j, r}(x)  - h^{[R]}(x)|  = O(\thr).
\]
Let
\begin{align*}
       \hat{h}_j(x) \eqdef & \sum_{r \in [R]} T^{(r)}(j, \hash_{j}(x))  \cdot s_{j, r}(x) \\ =  & \sum_{r \in [R]}  \sum_{x'}\indic{ \hash_{j}(x') = \hash_{j}(x)} s_{j, r}(x') s_{j, r}(x) \cdot h^{(r)}(x') \\
        = & \sum_{x'}\indic{ \hash_{j}(x') = \hash_{j}(x)} \sum_{r \in [R]} s_{j, r}(x') s_{j, r}(x) \cdot h^{(r)}(x') \\
        = & h^{[R]}(x) +  \sum_{x'\neq x}\indic{ \hash_{j}(x') = \hash_{j}(x)} \sum_{r \in [R]} s_{j, r}(x') s_{j, r}(x) \cdot h^{(r)}(x') 
\end{align*}

Then we have $\expectation{ \hat{h}_j(x) = h^{[R]}(x)}$. Next we provide a bound on the variance.  
Let $H_{10\tau/\sqrt{R}}$ be the set of elements with frequency at least $10\tau/\sqrt{R}$, then we have $|H_{10\tau/\sqrt{R}}| \le \frac{mn\sqrt{R}}{10\tau}$. Since $w = \ceil{\frac{10mn\sqrt{R}}{\tau}}$, we have with probability at least 5/6, 
\[
    \sum_{x' \in H_{10\tau/\sqrt{R}}, x' \neq x}\indic{ \hash_{j}(x') = \hash_{j}(x)}  = 0.
\]
Conditioned on this event, we have
\begin{align*}
    \expectation{\Paren{\hat{h}_j(x)  - h^{[R]}(x)}^2}  & = \expectation{\Paren{\sum_{x' \notin H_{10\tau/\sqrt{R}}, x' \neq x}\indic{ \hash_{j}(x') = \hash_{j}(x)} \sum_{r \in [R]} s_{j, r}(x') s_{j, r}(x) \cdot h^{(r)}(x')}^2}
    \\
    & \le \frac{\max_{x' \notin H_{10\tau/\sqrt{R}}} h^{[R]}(x) \sum_{x' \notin H_{10\tau/\sqrt{R}}} h^{[R]}(x) }{w} \\
    & \le \thr^2.
\end{align*}

Hence with probability at least $5/6$, we have
\[
     \expectation{\absv{\hat{h}_j(x)  - h^{[R]}(x)}} \le \sqrt{6}\tau.
\]
We conclude the proof by a union bound over the two events.
\section{Additional details on IBLT}
\label{sec:iblt_app}
\paragraph{Intuition on \texttt{ListEntries} for IBLT.} The intuition behind the IBLT construction is as follows:
Start with an array $\mathcal{B}$
of length $\ell$ containing 4-tuples
of the form $(0,0,0,0)$.
To insert pair $(x, v)$
hash the tuple ($x$, $\tilde{x}$, $v$, $1$) into $k$ locations $l_1, \ldots, l_k$ in $\mathcal{B}$ based on the key $x$, where $\tilde{x} := G(x)$ is a 
hash of $x$ into a sufficiently large domain so that collision probability is sufficiently unlikely. %
Then add, using component-wise sum, ($x$, $\tilde{x}$, $v$, $1$) to the contents of 
$\mathcal{B}$ in all locations $l_1, \ldots, l_k$.
The $\texttt{ListEntries}/\deciblt$ operation corresponds to the result of the following procedure: (1) find an
entry ($x_{sum}$, $\tilde{x}_{sum}$, $v_{sum}$, $j$) such that $G(x_{sum}/j) = \tilde{x}_{sum}/j$ holds, 
(2) add $(x_{sum}/j, v_{sum})$ to the output, and
(3) remove the pair $(x_{sum}/j, v_{sum})$
by subtracting ($x_{sum}$, $\tilde{x}_{sum}$, $v_{sum}$, $j$)
from the entries $l_1', \ldots, l_k'$ in the array $\mathcal{B}$ to which an insertion would add the tuple for key $x_{sum}/j$ and get back to step (1).
The process of listing entries a.k.a ``peeling off" $\mathcal{B}$.
might terminate before the IBLT is empty.
This is the failure procedure in 
Lemma~\ref{lem:iblt}, which corresponds to the natural procedure to find a 2-core in a random graph \cite{Goodrich2011iblt}.

\paragraph{Sketch size.} 
The above intuition corresponds to the IBLT construction variant from~\cite{Goodrich2011iblt}
that can handle duplicates.
It can be implemented with four length $\ell$ vectors with entries in 
$[d], \texttt{Im}(G), [mn], [mn]$,
respectively. 
In terms of concrete parameters (see ~\cite{Goodrich2011iblt} for details), $k = 3, \ell > 1.3 L_0$,
and $G = \mathbb{Z}_p$ with $p = 2^{31}-1$ give good performance, and require
$1.3L_0(32 + \log_2 d + 2\log_2(mn))$ bits. For the experiment setting considered in \cref{sec:exp}, this is will take at most $8L_0$ words.

\paragraph{Cardinality estimation from saturated IBLT.} 
\cref{lem:iblt} tells us that 
a tight bound $L_0$ on the number of distinct non-zero indices in the intended histogram, 
can save us space in an IBLT encoding. %
However, getting that bound wrong results in an undecodable IBLT. 
While in the single round case all is lost, in the multi-round setting
we leverage a property of undecodable IBLTs that helps update our $L_0$ bound for subsequent rounds
after a failed round. This is the main ingredient for our adaptive tuning heuristic presented in Section~\ref{sec:adaptive}.

Let $\mathcal{B}$ be an undecodable IBLT, and let $S$ be the size of the undecoded graph of $\mathcal{B}$. Also let $\ell$ be the size of $\mathcal{B}$, and let $N$ the (unknown) 
number of distinct elements inserted in $\mathcal{B}$ 
(note that $N$ corresponds to the correct bound $L_0$ that enables decoding).
By \cite{molloy2005cores}, we have the following relation: 
For large enough $N$, if $S < \ell$, we have 
\begin{equation}\label{eq:cardinality-from-core}
    \frac{S}{C} = 1 - e^{-x}(1+x) + o(1),
\end{equation}
where $x$ is the greatest solution to 
\begin{equation}\label{eq:cardinality-from-core-2}
    \frac{6N}{C} = \frac{2x}{(1 - e^{-x})^2}.
\end{equation}
Hence we can have an estimate for $N$ (and thus a correct choice for $L_0$ in a subsequent round) 
based on  $S$ and $C$. We first solve \eqref{eq:cardinality-from-core} ignoring the $o(1)$ term to get $x$ and then plug $x$ and $C$ into \eqref{eq:cardinality-from-core-2} to get an estimate for $N$.
As mentioned above we leverage this fact in Section~\ref{sec:adaptive}.


\section{Proof of lower bounds.}

\subsection{Proof of \cref{thm:ahh_lower}}
\label{sec:ahh_lower}
We will focus on the case when $R = 1$ since the claimed bound doesn't depend on $R$ and we can assume there is no data in other $R - 1$ rounds. We will consider the case when $10< \tau < n/4$.


\new{We prove the theorem by a reduction to the set disjointness problem \citep{BARYOSSEF2004702, Jayram09}. The set disjointness problem ($\textsc{Dist}_{t, d}$) considers the setting where $t$ users where user $i$ has a set of elements $S_i \subset \{1, 2, \ldots, d\}$. The goal is to distinguish between the following two chase with success probability at least $4/5$.
\begin{enumerate}
    \item All $S_i$'s disjoint.
    \item There exists $x \in [d]$ such that for all $i, j \in [t]$, $S_i \cap S_j = \{x\}$.
\end{enumerate}
And the goal is to minimize the size of the transcript of all communications among all users. More specifically, we will use the following lemma:
\begin{lemma}[\citep{Jayram09}]\label{lem:set_disjoint}
Any protocol that solves $\textsc{Dist}_{t, d}$ must have a transcript of size at least $\Theta(d/t)$.
\end{lemma}

Next we show that $\textsc{Dist}_{t, d}$ with $t = \tau$ and $d = mn/2$ can be reduced to the approximate heavy hitter problem. We divide users into $\tau + 1$ groups. For $i \in [\tau]$, the $i$th group has $n_i = \ceil{|S_i|/m}$ users. And let $\tilde{S}_i$ be set of all elements held by users in group $i$. We  partition $S_i$ to subsets of size at most $m$ and distribute them to users in group $i$ arbitrarily. This can be done since $mn_i \ge |S_i|$. The total number of users in the first $\tau$ groups is $\sum_{i \in [\tau]}n_i \le \frac{d + \tau}{m} + \tau \le n$. 
The $\tau + 1$ group has $n - \sum_{i \in [\tau]}n_i$ users and each user has zero element. 

Suppose there exists a $\tau$-\ahh~ linear sketch algorithm with communication cost per-user $o(\frac{mn}{\tau})$. When $S_i$'s are disjoint, all elements in $[d]$ will have frequency $1 < \tau/10$. The algorithm should output an empty list. When $S_i$'s have an unique intersection, the element will have frequency $\tau$, and hence the algorithm should output a list with size 1. By distinguishing between the two cases, the \ahh~algorithm can be used to solve $\textsc{Dist}_{t, d}$.

Moreover, under linear sketching constraint, the size of the transcript is the same as the per-user communication. Hence we conclude the proof by noticing that this violates \cref{lem:set_disjoint}.}




\subsection{Proof of \cref{thm:ahist_lower}}

Here we prove a stronger version of the lower bound where in each round $r$, the communication among users is not limited but
the users in $B_r$ must compress $h^{(r)}$ to an element $Y^{(r)} \in G_r$ with $|G_r| \le 2^\ell$, which is observed by the server. And the server will then obtain an approximate histogram $\widehat{h}^{[R]}$ based on $\Pi = (Y^{(1)}, \ldots, Y^{(R)}, U)$. \new{For a given $\dist$, next we show that any protocol with $\ell = o\Paren{ \min\{ \frac{\nspu \ns\sqrt{R}}{\thr}, \nspu \ns \} }$ won't solve $\dist$-approximate heavy hitter with error probability at most 1/5. We will focus on the case when $\tau \ge \sqrt{R}$ and $\ell = o\Paren{\frac{\nspu \ns\sqrt{R}}{\thr}}$. When $\tau < \sqrt{R}$, the bound follows by setting $\tau = \sqrt{R}$ and the fact that the problem gets harder as $\tau$ decreases.} To simply the proof, we assume $R \ge 400$ without loss of generality.

We consider histograms $h^{(r)}, \forall r \in [R]$ supported over the domain $10\ell$ and are generated \iid~from a distribution $P$.
Let $Z$ be uniformly distributed over $\{\pm 1\}^{5\ell}$, and under distribution $P_Z$, we have $\forall r \in [R], i \in [5\ell]$,
\[
    h^{(r)}(2i) = \begin{cases}
    \frac{mn}{5\ell} & \qquad \text{ with prob } \frac{1}{2} + \frac{10}{\sqrt{R}} Z_i. \\
    0 & \qquad \text{ with prob } \frac{1}{2} - \frac{10}{\sqrt{R}} Z_i.
    \end{cases}
\]
and $$h^{(r)}(2i - 1) = 1 - h^{(r)}(2i).$$
It can be check that $\norm{h^{(r)}}_1 = \nspu \ns$ with probability 1. 
We prove the theorem by contradiction. If the protocol solves $\dist$-approximate heavy hitter with error probability at most 1/5, let
\[
    \hat{Z}_i = \indic{\hat{h}^{[R]}(2i) > \frac{mnR}{10\ell}}.
\]

We have
\begin{align*}
    \probof{\hat{Z}_i \neq Z_i} & \le   \probof{{\absv{\hat{h}^{[R]}(2i) - h^{[R]}(2i)}}  \ge \frac{\nspu \ns \sqrt{R}}{\ell}}  + \probof{\absv{ h^{[R]}(2i) - \frac{\nspu\ns R}{5\ell} \Paren{\frac12 + \frac{10}{\sqrt{R}}Z_i}} \ge \frac{\nspu \ns \sqrt{R}}{\ell}} \\
    & \le \frac{1}{5} + \frac{1}{25} = \frac{6}{25},
\end{align*}

where the first probability is bounded by the success probability of the algorithm and the second probability is bounded using Hoeffding bound. Hence we have
\[
    \sum_{i \in [5\ell]} I(Z_i; \Pi) \ge  \sum_{i \in [5\ell]} I(Z_i; \hat{Z}_i) \ge \sum_{i \in [5\ell]} (1 - H(\frac{5}{26})) \ge 2 \ell,
\]
where $H(p)$ is the Shannon entropy of a Bernoulli random variable with success probability $6/25$.

\new{To upper bound $\sum_{i \in [5\ell]} I(Z_i; \Pi)$, we notice that the vector \[\Paren{\frac{5\ell h^{(r)}(2)}{mn}, \frac{5\ell h^{(r)}(4)}{mn}, \ldots, \frac{5\ell h^{(r)}(10\ell)}{mn}}.\]
follows a product distribution with the marginal of each coordinate being a Bernoulli distribution. Hence by standard arguments on communication-limited estimation of product of Bernoulli random variables (\eg in \cite{braverman2016communication, han2021geometric, acharya2020unified}). In particular, following almost the same steps as in \citet[Section 7.1]{acharya2020unified}, }
\[
     \sum_{i \in [5\ell]} I(Z_i; \Pi) \le R \cdot \Paren{\frac{1}{\sqrt{R}}}^2 \ell = \ell,
\]
which leads to a contradiction. This concludes the proof.


\section{Binomial tail bound.}
\begin{lemma} \label{lem:binomial_tail}
Let $X \sim {\rm Binom}(n, p)$ be a binomial distribution, when $n > 10$ and $p < 1/5$, we have
\[
    \probof{X \ge n/2} \le \frac{n + 1}{2} \Paren{\frac{8ep}{5}  }^{n/2}.
\]
\end{lemma}
\begin{proof}
    \begin{align}
        \probof{X \ge n/2} & = \sum_{i = \floor{(\ns+1)/2}}^\ns  \probof{X = i} \nonumber  \\ 
        & = \sum_{i = \floor{(\ns+1)/2}}^\ns  {n \choose i} \Paren{1 - p}^{n - i} p^i \nonumber  \\
        & \le \frac{n + 1}{2} {n \choose \ceil{n/2}}  \Paren{(1 - p)p}^{n/2}  \label{eqn:n_2}\\
        & \le  \frac{n + 1}{2} (2e)^{n/2} \Paren{\frac{4p}5}^{n/2} \label{eqn:stirling}\\
        & = \frac{n + 1}{2} \Paren{\frac{8ep}{5}  }^{n/2}, \nonumber 
    \end{align}
    where \eqref{eqn:n_2} follows from $\probof{X = i}$ is monotonically decreasing when $i \ge n/2$ and \eqref{eqn:stirling} follows from standard bounds on binomial coefficients.
\end{proof}
\end{document}


