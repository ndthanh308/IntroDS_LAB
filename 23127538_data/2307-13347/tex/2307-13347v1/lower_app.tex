\section{Proof of lower bounds.}

\subsection{Proof of \cref{thm:ahh_lower}}
\label{sec:ahh_lower}
We will focus on the case when $R = 1$ since the claimed bound doesn't depend on $R$ and we can assume there is no data in other $R - 1$ rounds. We will consider the case when $10< \tau < n/4$.


\new{We prove the theorem by a reduction to the set disjointness problem \citep{BARYOSSEF2004702, Jayram09}. The set disjointness problem ($\textsc{Dist}_{t, d}$) considers the setting where $t$ users where user $i$ has a set of elements $S_i \subset \{1, 2, \ldots, d\}$. The goal is to distinguish between the following two chase with success probability at least $4/5$.
\begin{enumerate}
    \item All $S_i$'s disjoint.
    \item There exists $x \in [d]$ such that for all $i, j \in [t]$, $S_i \cap S_j = \{x\}$.
\end{enumerate}
And the goal is to minimize the size of the transcript of all communications among all users. More specifically, we will use the following lemma:
\begin{lemma}[\citep{Jayram09}]\label{lem:set_disjoint}
Any protocol that solves $\textsc{Dist}_{t, d}$ must have a transcript of size at least $\Theta(d/t)$.
\end{lemma}

Next we show that $\textsc{Dist}_{t, d}$ with $t = \tau$ and $d = mn/2$ can be reduced to the approximate heavy hitter problem. We divide users into $\tau + 1$ groups. For $i \in [\tau]$, the $i$th group has $n_i = \ceil{|S_i|/m}$ users. And let $\tilde{S}_i$ be set of all elements held by users in group $i$. We  partition $S_i$ to subsets of size at most $m$ and distribute them to users in group $i$ arbitrarily. This can be done since $mn_i \ge |S_i|$. The total number of users in the first $\tau$ groups is $\sum_{i \in [\tau]}n_i \le \frac{d + \tau}{m} + \tau \le n$. 
The $\tau + 1$ group has $n - \sum_{i \in [\tau]}n_i$ users and each user has zero element. 

Suppose there exists a $\tau$-\ahh~ linear sketch algorithm with communication cost per-user $o(\frac{mn}{\tau})$. When $S_i$'s are disjoint, all elements in $[d]$ will have frequency $1 < \tau/10$. The algorithm should output an empty list. When $S_i$'s have an unique intersection, the element will have frequency $\tau$, and hence the algorithm should output a list with size 1. By distinguishing between the two cases, the \ahh~algorithm can be used to solve $\textsc{Dist}_{t, d}$.

Moreover, under linear sketching constraint, the size of the transcript is the same as the per-user communication. Hence we conclude the proof by noticing that this violates \cref{lem:set_disjoint}.}




\subsection{Proof of \cref{thm:ahist_lower}}

Here we prove a stronger version of the lower bound where in each round $r$, the communication among users is not limited but
the users in $B_r$ must compress $h^{(r)}$ to an element $Y^{(r)} \in G_r$ with $|G_r| \le 2^\ell$, which is observed by the server. And the server will then obtain an approximate histogram $\widehat{h}^{[R]}$ based on $\Pi = (Y^{(1)}, \ldots, Y^{(R)}, U)$. \new{For a given $\dist$, next we show that any protocol with $\ell = o\Paren{ \min\{ \frac{\nspu \ns\sqrt{R}}{\thr}, \nspu \ns \} }$ won't solve $\dist$-approximate heavy hitter with error probability at most 1/5. We will focus on the case when $\tau \ge \sqrt{R}$ and $\ell = o\Paren{\frac{\nspu \ns\sqrt{R}}{\thr}}$. When $\tau < \sqrt{R}$, the bound follows by setting $\tau = \sqrt{R}$ and the fact that the problem gets harder as $\tau$ decreases.} To simply the proof, we assume $R \ge 400$ without loss of generality.

We consider histograms $h^{(r)}, \forall r \in [R]$ supported over the domain $10\ell$ and are generated \iid~from a distribution $P$.
Let $Z$ be uniformly distributed over $\{\pm 1\}^{5\ell}$, and under distribution $P_Z$, we have $\forall r \in [R], i \in [5\ell]$,
\[
    h^{(r)}(2i) = \begin{cases}
    \frac{mn}{5\ell} & \qquad \text{ with prob } \frac{1}{2} + \frac{10}{\sqrt{R}} Z_i. \\
    0 & \qquad \text{ with prob } \frac{1}{2} - \frac{10}{\sqrt{R}} Z_i.
    \end{cases}
\]
and $$h^{(r)}(2i - 1) = 1 - h^{(r)}(2i).$$
It can be check that $\norm{h^{(r)}}_1 = \nspu \ns$ with probability 1. 
We prove the theorem by contradiction. If the protocol solves $\dist$-approximate heavy hitter with error probability at most 1/5, let
\[
    \hat{Z}_i = \indic{\hat{h}^{[R]}(2i) > \frac{mnR}{10\ell}}.
\]

We have
\begin{align*}
    \probof{\hat{Z}_i \neq Z_i} & \le   \probof{{\absv{\hat{h}^{[R]}(2i) - h^{[R]}(2i)}}  \ge \frac{\nspu \ns \sqrt{R}}{\ell}}  + \probof{\absv{ h^{[R]}(2i) - \frac{\nspu\ns R}{5\ell} \Paren{\frac12 + \frac{10}{\sqrt{R}}Z_i}} \ge \frac{\nspu \ns \sqrt{R}}{\ell}} \\
    & \le \frac{1}{5} + \frac{1}{25} = \frac{6}{25},
\end{align*}

where the first probability is bounded by the success probability of the algorithm and the second probability is bounded using Hoeffding bound. Hence we have
\[
    \sum_{i \in [5\ell]} I(Z_i; \Pi) \ge  \sum_{i \in [5\ell]} I(Z_i; \hat{Z}_i) \ge \sum_{i \in [5\ell]} (1 - H(\frac{5}{26})) \ge 2 \ell,
\]
where $H(p)$ is the Shannon entropy of a Bernoulli random variable with success probability $6/25$.

\new{To upper bound $\sum_{i \in [5\ell]} I(Z_i; \Pi)$, we notice that the vector \[\Paren{\frac{5\ell h^{(r)}(2)}{mn}, \frac{5\ell h^{(r)}(4)}{mn}, \ldots, \frac{5\ell h^{(r)}(10\ell)}{mn}}.\]
follows a product distribution with the marginal of each coordinate being a Bernoulli distribution. Hence by standard arguments on communication-limited estimation of product of Bernoulli random variables (\eg in \cite{braverman2016communication, han2021geometric, acharya2020unified}). In particular, following almost the same steps as in \citet[Section 7.1]{acharya2020unified}, }
\[
     \sum_{i \in [5\ell]} I(Z_i; \Pi) \le R \cdot \Paren{\frac{1}{\sqrt{R}}}^2 \ell = \ell,
\]
which leads to a contradiction. This concludes the proof.

