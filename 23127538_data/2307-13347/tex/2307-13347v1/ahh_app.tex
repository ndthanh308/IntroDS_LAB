\section{Proof of \cref{thm:ahh}}
\label{sec:proof_ahh}
Note that the algorithm can be viewed as $\rep \eqdef \ceil{20\log(\frac{40\nspu\ns R}{\tau\beta})}$ independent runs of a basic protocol, each of which returns a list $\heavy_i$ of potential heavy hitters. We assume $b \ge 260$, else we take $b' = \max\{b, 260\}$ and the result will change by at most a constant factor.

The next lemma states that the probabilities of heavy elements and tail elements falling in the list.
\begin{lemma} \label{lem:single_run}
    All $\heavy_j$ defined in \cref{alg:subsample_IBLT} satisfy that, if $h^{[R]}(x) \ge \thr$,
    \[
    \probof{x \in \heavy_j} \ge 4/5.
    \]
    Else if $h^{[R]}(x) \le \thr/10$, 
    \[
        \probof{x \in \heavy_j} \le \frac{2h^{[R]}(x)}{\thr}.
    \]
\end{lemma}

Before proving the lemma, we first show how \cref{thm:ahh} can be implied by \cref{lem:single_run}. 

By \cref{lem:single_run}, for $x$ with $h^{[R]}(x) \ge \thr$, we have 
\[
    \probof{x \in \heavy} \ge \probof{{\rm Binom}\Paren{\rep, 4/5} \ge \rep/2} \ge 1 - \frac{\beta \tau}{40 \nspu \ns R},
\]
where the last inequality follows from standard concentration bounds for Binomial random variables  (\eg Chernoff bound \cite{Mitzenmacher2017probability}).

Hence by union bound, we have 
\[
    \probof{\{x \in [d] \mid h^{[R]}(x) \ge \thr \} \subset \heavy } \ge 1 - \frac{\beta}{40}.
\]

For any $x$, with $h^{[R]}(x) \le \thr/10$, by \cref{lem:single_run}, we have
\begin{align*}
\probof{x \in \heavy} \le \probof{{\rm Binom}\Paren{\rep, \frac{2h^{[R]}(x)}{\thr}} \ge \rep/2} \le \frac{\rep + 1}{2} \Paren{\frac{8e}{5} \cdot \frac{2h^{[R]}(x)}{\thr}}^{\rep/2},
\end{align*}
where the last inequality follows from Binomial tail bound (see \cref{lem:binomial_tail}).

Hence by union bound we have 
\begin{align}
    & \qquad \probof{\{x \in [d] \mid h^{[R]}(x) \le \thr/10 \} \cap H \neq \emptyset}  \nonumber \\
    & \le \sum_{x: h^{[R]}(x) \le \thr/10} \frac{\rep+1}{2} \Paren{ \frac{16e h^{[R]}(x)}{5\thr}}^{\rep/2}  \nonumber  \\
    & \le \frac{20\nspu \ns R}{\thr} \frac{\rep+1}{2} \Paren{\frac{8e}{25}}^{\rep/2} \label{eqn:combine_elements}\\
    & \le  \frac{20\nspu \ns R}{\thr} e^{-\frac{b}{20}} \label{eqn:algebra}\\
    & \le \frac{\beta}{2}, \nonumber
\end{align}
where \eqref{eqn:combine_elements} follows from $x^{\rep/2} + y^{\rep/2} \le (x + y)^{\rep/2}$, and hence we can combine symbols to increase the sum of tail probability and end up with at most $\frac{20\nspu \ns R}{\thr}$ symbols with frequencies at most $\thr/10$. \eqref{eqn:algebra} follows from the inequality $(x + 1/2) (8e/25)^x \le e^{-x/10}$ for $x \ge 130$.

By union bound, we get the guarantee claimed in \cref{thm:ahh}.

\begin{proofof}{\cref{lem:single_run}}
The proof mainly consists of two parts. We will first show that local subsampling will keep each heavy hitter with a high probability and each tail element with a low probability, stated in \cref{lem:threshold}. We will then show that after local subsampling, the number of unique elements in each round will decrease so that the decoding in \cref{alg:subsample_IBLT} will succeed with high probability.


\begin{lemma}\label{lem:threshold}
    Let $h_j^{'[R]}$ be the aggregation of locally subsampled histogram for run $j$, \ie
    \[
        h_j^{'[R]} = \sum_{r \in [R]} \sum_{i \in B_r} h_{i, j}.
    \]
    Then if $h^{[R]}(x) \ge \tau$, 
    \[
        \probof{h_j^{'[R]}(x) > 0 } \ge 1 - \frac{1}{e^2}. 
    \]
    Else if $h^{[R]}(x) \le \tau/10$,
    \[
        \probof{x \in \heavy_j} \le \frac{2h^{[R]}(x)}{\thr}.
    \]
\end{lemma}
\begin{proof}
When $h^{[R]}(x) \ge \tau$,
\begin{align*}
    \probof{h_j^{'[R]}(x) > 0 } = 1 - \Pi_{r \in [R], i \in B_r} \min \{1 -  \frac{2 h_{i, j}(x)}{\thr}, 0\} \ge 1 - \Pi_{r \in [R], i \in B_r} e^{-\frac{2 h_{i, j}(x)}{\thr}} =  1 -e^{-\frac{2 h^{[R]}(x)}{\thr}} \ge 1 - \frac{1}{e^2}.
\end{align*}
When $h^{[R]}(x) \le \tau/10$
\begin{align*}
     \probof{h_j^{'[R]}(x) > 0 } = 1 - \Pi_{r \in [R], i \in B_r} \Paren{1 -  \frac{2 h_{i, j}(x)}{\thr}} \le 1 - \Paren{1 - \sum_{r \in [R], i \in B_r}\frac{2 h_{i, j}(x)}{\thr}} = \frac{2h^{[R]}(x)}{\thr}.
\end{align*}
\end{proof}

The next lemma shows that with high probability, the number of elements in each round will decrease by least a factor of $\tau$. 
\begin{lemma} \label{lem:max_zero}
    With probability at least $1 - 1/32$, we have
    \[
        \max_{r \in [R]}\left\{ \norm{h'_r}_0 \right\} = O \Paren{\frac{\nspu \ns}{\tau} \log R}.
    \]
\end{lemma}
\begin{proof}
Since all rounds are independent, it would be enough to show that $\forall i$, with probability at least $1 - 1/32R$, we have 
\[
    \norm{h'_r}_0  = O \Paren{\frac{\nspu \ns}{\tau} \log R}.
\]
To see this, we have
\[
    \probof{\norm{h'_r}_0 \ge \frac{2\nspu \ns}{\tau} \log R} \le \probof{{\rm Binom}\Paren{mn, \frac1{\tau}}\ge \frac{2\nspu \ns}{\tau} \log R} \le \frac{1}{32R},
\]
where the first step follows from that the left hand side is maximized when all $\nspu\ns$ elements in $h_r$ are distinct, and the second step follows from standard binomial tail bound when $mn > 4 \tau$ and $R > 32$.
\end{proof}

Finally, it would be enough to show that when the condition in \cref{lem:max_zero} holds, the decoding of the aggregated IBLT will succeed with high probability. This is true since by \cref{lem:iblt} and union bound, we have
\[
    \probof{\forall j, \hat{h}_j^{[R]} = h_j^{'[R]}} \ge 1 - R \cdot (\frac{\nspu \ns}{\tau} \log R)^{2-\gamma} \ge 1 - 1/32,
\]
where the last inequality holds when $mn > 4 \tau$ and $R > 32$.
Combining the above and \cref{lem:max_zero,lem:threshold}, we conclude the proof since $1/e^2 + 1/32 + 1/32 \le 1/5$.
\end{proofof}
