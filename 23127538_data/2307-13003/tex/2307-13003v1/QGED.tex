\documentclass[a4paper]{article}
\usepackage{typearea}
\typearea{13}
%\usepackage{hyperref}
\usepackage{makeidx}
\usepackage{graphicx}
\usepackage{amsmath}
%\usepackage{mathtools}
\usepackage{amssymb}
%\usepackage{amsthm}
\usepackage{mathrsfs}
\usepackage{color}
\usepackage{hyperref}
%\usepackage{ulem}
%\usepackage{booktabs}
\usepackage{multirow}
\usepackage{tikz}

\let\corresponds\relaxElectroweak
\usepackage{mathabx}
\setcounter{MaxMatrixCols}{30}
\providecommand{\U}[1]{\protect\rule{.1in}{.1in}}


% TikZ libraries
\usepackage{tikz}
\usetikzlibrary{positioning}
\usetikzlibrary{calc}
\usetikzlibrary{shapes.geometric}
\usetikzlibrary{angles}
\usetikzlibrary{decorations.pathreplacing}
\usetikzlibrary{calligraphy}

% Valve settings (TikZ figures)
\newcommand{\valveheight}{ 7pt}
\newcommand{\valvewidth} {10pt}

% Formatting of a standard/default box for TikZ drawings
\tikzstyle{box} = [
    draw,                       % Create a box around the text
    minimum width   =  25pt,    % Minimum width of the box
    minimum height  =  25pt,    % Minimum height of the box
    rounded corners =   5pt]    % Round the corners
    
% Formatting of a large (square) box for TikZ drawings
\tikzstyle{bigbox} = [
    draw,                       % Create a box around the text
    minimum width   =  55pt,    % Minimum width of the box
    minimum height  =  55pt,    % Minimum height of the box
    rounded corners =   5pt]    % Round the corners
    
% Formatting of a long (rectangular) box for TikZ drawings
\tikzstyle{longbox} = [
    draw,                       % Create a box around the text
    minimum width   = 125pt,    % Minimum width of the box
    minimum height  =  25pt,    % Minimum height of the box
    rounded corners =   5pt]    % Round the corners
%%%%%%%%%%%
\begin{document}
\title{
Quantum GravitoElectromagnetic Dynamics
}
\author{Yoshimasa Kurihara\footnote{yoshimasa.kurihara@kek.jp}
\\
{\it\footnotesize The High Energy Accelerator Organisation (KEK), 
Tsukuba, Ibaraki 305-0801, Japan}
}
\date{}
\maketitle
\begin{abstract} % abstract
We propose a renormalisable quantum theory of gravity (QGED) based on the standard BRST quantisation used to quantise the Yang--Mills theory.
The BRST-invariant Lagrangian of the gravitationally interacting $U(1)$-gauge theory, including gauge fixing and ghost parts, is provided.
From this Lagrangian, we extract a set of Feynman rules in the local inertial frame where gravity vanishes locally.
Utilising Feynman rules of the QGED prepared here, we construct all renormalisation constants and show that the theory is perturbatively renormalisable in one-loop order.
We replace infinite-valued bare objects in the bare Lagrangian with experimentally measured ones.
In addition to standard QED parameters, we show that the gravitational coupling constant is measurable experimentally.

We also discuss a running effect of the gravitational coupling constant and the perturbative estimation of the Hawking radiation as examples of the perturbative QGED.
\end{abstract}
\maketitle
\tableofcontents
%\newpage
%%% ----------------------------------------------------------------------
% Introduction
%%% ----------------------------------------------------------------------
\section{Introduction}
For the microscopic aspect of Nature, the standard theory of particle physics based on quantum field theory has many experimental supports.
On the other hand, general relativity is among the fundamental theories concerning the large space-time structure of the Universe, and many experiments have established its correctness.
Thus, our understanding of Nature covers a wide range of lengthscales and timescales, from the Universe's large-scale structure to the microscopic behaviour of subatomic particles.
However, two fundamental theories are inconsistent; the former is quantum, and the latter is a classical theory.
Owing to the uncertainty principle, the fundamental physical theory must be a quantum theory; thus, constructing a quantum theory of gravity is one of the fundamental goals of modern physics.

We understand the \emph{quantum theory of gravity} as a theory that describes the behaviour of four-dimensional space-time (the metric tensor) in regions where the uncertainty principle is essential and a theory consistent with well-established general relativity at large space-time scales.
Moreover, a theory that can provide experimentally measurable predictions is desirable.
Immediately after the establishment of general relativity and quantum mechanics during the 1920s, the development of quantum gravity began in the 1930s.
For a detailed history, see Refs.\cite{Rovelli:2000aw,doi:10.1142qg} and references therein.
It is commonly recognised that perturbative expansion of general relativity in four-dimensional space-time is not renormalisable.
Although trials to construct a perturbatively renormalised theory of general relativity have failed, its non-existence has yet to be proven.
The current author discussed the renormalisability of four-dimensional general relativity in\cite{BERENDS197599,Goroff198581,GOROFF1986709}.
On the other hand, the quantisation of the Yang--Mills theory in flat space-time is well established, and the theory is proven to be renormalisable in all orders of perturbation expansions.
However, it is not proven in curved space-time and is not trivial.
A primary objective of this report is to construct the renormalisable quantum gravity with the Yang--Mills theory simultaneously.

General relativity is a kind of gauge theory, which is at first developed in a pioneering work by Utiyama\cite{PhysRev.101.1597} (see also Ref.\cite{ blagojevi2013gauge}).
Common aspects between general relativity and the Yang--Mills theory may give a hint for renormalisable perturbation for general relativity.
We summarise correspondences of mathematical properties between them in Table-\ref{table2}.
%%%%%%%%%%%%%%%%%%%%%%%%%%%%%%%%%%%%%%%%%%%%%%%%%%%%%%%
\begin{table}[t]
\begin{center}
\caption{\label{table2}\small
Comparison between general relativity and the Yang--Mills theory at a classical level.}
\vskip 2mm
\begin{tabular}{l||llll|c}
\hspace{2em}Property & \multicolumn{2}{c}{General Relativity} & \multicolumn{2}{c|}{Electromagnetism}&dimension \\
\hline
structural group & \multicolumn{2}{c}{$SO(1,3)$}& \multicolumn{2}{c|}{$Spin(1,3)\oplus\SU(N)$} &~\\
%
connection & spin-connection:& $\omega_\mu^{~ab}$ 
& gauge potential:& $\Aa^\mu$&$L^{-1}$ \\
%
curvature & curvature:& $R^{ab}_{~~cd}/\kE\hbar$ 
& field strength:& $\f^{\mu\nu}$ &$L^{-2}$\\
%
section & vierbein:& $\E_a^\mu/\sqrt{\kE\hbar}$ 
& Dirac spinor:& $\psi^{2/3}$&$L^{-1}$ \\
%
%mass of section& $\hbar\Lambda_c^{\hlf}$ & $m_e$ &$M$/$M$\\
%
coupling constant & & $\cGR$ & &$e$&${\it 1}\hspace{1em}$
%
\end{tabular}
\end{center}
\end{table}
%%%%%%%%%%%%%%%%%%%%%%%%%%%%%%%%%%%%%%%%%%%%%%%%%%%%%%%
Comparing two gauge theories, it is clear that the spin-connection (gauge potential) and vierbein field (section) are appropriate targets of quantisation\cite{Kurihara_2020}.
Moreover, the gravitational coupling constant must be included in the Lagrangian, and it must be renormalised, even if it is unity valued after renormalisation, which is not considered in previous studies.
Renormalizability of the Yang-Mills theory must be ensured, including gravitational fields $\E$ and $\omega$.
In the curved space-time, the Yang-Mills Lagrangian also has an interaction term of Dirac spinor $\psi$, spin-connection $\omega$ and vierbein $\E$\cite{fre2012gravity}.
In this report, we develop quantum general relativity quantised with the U(1)-gauge theory, namely the ``Quantum GravitoElectromagnetic Dynamics (QGED)''.

The QGED does not quantise space-time itself; thus, the space-time coordinates $x^\mu$ are not the quantum operator but the classical continuous object\cite{nakanishi1990covariant,NakanishiSK2009}.
The subject for quantisation is the vierbein field (equivalently, the metric tensor $g^{(c)}_{\hspace{.3em}\mu\nu}(x)$), which is obtained as a solution of the Einstein equation.
In classical general relativity, the geometrical metric tensor $g^{(g)}_{\hspace{.3em}\mu\nu}(x)$ is given by the solution of the classical Einstein equation, such as $g^{(c)}_{\mu\nu}(x)$=$g^{(g)}_{\hspace{.3em}\mu\nu}(x)$, i.e., it is Einstein's equivalence principle.
This relation is not simply true at the quantum level, and the geometrical metric tensor is given as the expected value of the quantum metric tensor $g^{(g)}_{\hspace{.3em}\mu\nu}(x)=\langle g^{(q)}_{\hspace{.3em}\mu\nu}(x)\rangle$.

This report is organised as follows:
After this introduction section, \textbf{section 2} provides mathematical preliminaries describing geometrical backgrounds of general relativity and the Yang--Mills theory based on Ref.\cite{Kurihara:2022sso}.
We introduce a gravitational coupling constant in a covariant differential concerning a local $SO(1,3)$ group.
Although this method is not standard in general relativity, it is commonly utilised in the Yang--Mills theory.
\textbf{Section 3} introduces the classical (un-renormalised) QGED Lagrangian consisting of the pure and interaction Lagrangian for general relativity and the Yang--Mills theory.
The interaction Lagrangian includes a gravitational interaction between a fermion and a gravitational gauge boson.
In addition to that, we have to include a gauge fixing and ghost Lagrangian to quantise this system.
In this section, we give all necessary Lagrangians and show their BRST invariance, ensuring the gauge theory's quantisation.
From the QGED Lagrangian given in \textbf{section 3}, \textbf{section 4} extracts Feynman rules for the perturbative calculation of amplitudes.
Successively, \textbf{section 4} develops the renormalised QGED using Feynman rules given in \textbf{section 3}.
We prepare all necessary renormalisation constants and show that all ultraviolet divergences are absorbed in a finite number of renormalisation constants at the one-loop level.
Consequently, the renormalisation-group equation provides an energy scale dependence of the effective gravitational coupling (a running gravitational coupling).
\textbf{Section 5} provides an application of the QGED, i.e., the Hawking radiation from the Schwartzschild black hole in contrast with the Schwinger effect of the QED.
Finally, the summary section summarises the main results of this report and discusses essential aspects of the QGED.
In addition, we provide two appendixes, a proof of nilpotent for all fields appearing in the Lagrangian in \textbf{Appendix \ref{app1}} and the possible method to measure the gravitational coupling constant in particle physics experiments by \textbf{Appendix \ref{appB}}.

We use the following physical units in this study:
The speed of light and the vacuum permittivity are set to unity, $c=1$ and $\epsilon^{~}_0=1$.
Einstein's constant of gravitation, $\kE$, and reduced Plank-constant, $\hbar=h/2\pi$, are written explicitly.
We define the elementary charge, $e$, appearing in the Lagrangian as a dimensionless object; thus, the fine-structure constant is defined as $\alpha=e^2/4\pi$. 
In these units, physical dimensions of fundamental constants are $[\hbar\hspace{.1em}\kE]_\text{p.d.}=L^2=T^2$ and $[\hbar/\kE]_\text{p.d.}=E^2=M^2$, where $L$, $T$, $E$ and $M$ are respectively length, time, energy and mass dimensions. 
Here, $[\bullet]_\text{p.d.}$ gives a physical dimension of the quantity $\bullet$.
Planck mass $m_{\textrm{p}}$ and Planck length $\lp$ are respectively defined as $m_{\textrm{p}}:=\sqrt{\hbar/\kE}$ and $\lp:=\sqrt{\hbar\kE}$.

\section{Mathematical preliminaries}
\subsection{Space-time manifold}
We introduce a four-dimensional Riemannian manifold $(\MM,\bm{g})$, where $\MM$ is a smooth and oriented four-dimensional manifold, and $\bm{g}$ is a metric tensor with a negative signature in $\MM$.
We refer to $\MM$ as the global manifold.
In an open  neighbourhood $U_{p{\in}\MM}\subset\MM$, we introduce the standard coordinate $x^\mu$. 
Orthonormal bases in $T\MM_{p}$ and $T^*\MM_{p}$ are denoted as $\partial/\partial x^\mu$ and $dx^\mu$, respectively.
Abbreviation $\partial_\mu:=\partial/\partial x^\mu$ is used throughout this report.
Two trivial vector bundles $\TMM:=\bigcup_{p} T\MM_{p}$ and $\TsMM:=\bigcup_{p} T^*\MM_{p}$ are referred to as  a tangent and cotangent bundles in $\MM$, respectively.

An inertial system, in which the Levi-Civita connection vanishes, exists locally at any point in $\MM$.
An inertial frame at point $p\in\MM$ is denoted as $\M_{p}$; namely, a local inertial manifold at point $p$.
$\M_{p}$ has a $SO(1,3)$ symmetry.
A trivial bundle $\M:=\bigcup_{p\in\M}\M_p$ is referred to as the inertial bundle.
Trivial bundles $\TM:=\bigcup_{p} T\M_{p}$ and $\TsM:=\bigcup_{p} T^*\M_{p}$ also exist over $\M$.
In an open  neighbourhood $U_{p{\in}\M}\subset\M$, we introduce the standard coordinate $\xi^a$. 
Roman letter suffixes are used for components of the standard basis in $\TM_{p}$ throughout this study; on the other hand, Greek letters are used for them in $\MM_{p}$.
This convention allows us to distinguish two abbreviated vectors, such as $\partial_\mu\in V(\TMM)$ and $\partial_a=\partial~/\partial\xi^a\in V(\TM)$.
The metric tensor in $\M_{p}$ is $\bm{\eta}=\textup{diag}(1,-1,-1,-1)$.
The Levi-Civita tensor (complete anti-symmetric tensor) $\bm\epsilon$, whose component is $[\bm\epsilon]_{0123}=\epsilon_{0123}=+1$, and $\bm\eta$ are constant tensors in $\M$.

We define the vierbein $\E^a_\mu(x)\in{C^\infty(\MM)}$ as a map transferring a vector in $\TsMM$ to that in $\TsM$ such that: 
\begin{align*}
\E^a_\mu(x\in\TsMM)dx^\mu\big|_{p\in\MM_p}=d\xi^a\big|_{p\in\M_p}
\in\Omega^1(\TsM){\otimes}\sss\ooo(1,3),
\end{align*}
where $\Omega^p(\bullet)$ is a space of $p$-form objects defined in bundle $\bullet$.
The vierbein is a smooth and invertible function globally defined in $\MM$.
The standard bases in $\TsM$ is referred to as the vierbein one-form denoted as $\eee^a:=\E^a_\mu dx^\mu$. 
The vierbein inverse $[\E^{-1}]_a^\mu=\E_a^\mu\in{C^\infty(\M)}$, which is also called the vierbein, is an inverse transformation such that
\begin{align*}
\E^a_\mu(x)\E_a^\nu\(\xi(x)\)=\delta^\nu_\mu\in\TsMM,~&\textrm{and}~~
\E_a^\mu\(\xi\)\E^b_\mu(x\(\xi)\)=\delta_a^b\in\TsM.%\label{EE}
\end{align*}
Metric tensors in $\MM$ and $\M$ are related to each other such that: $\eta_{ab}=\E_a^\mu\E_b^\nu g_{\mu\nu}$.
The $GL(4)$ invariant two-, three- and four-dimensional volume forms are, respectively, defined using vierbein forms as 
\begin{align*}
\SSS^{~}_{ab}&:=\frac{1}{2}\epsilon_{ab\bcdots}\hspace{.1em}
\eee^{\bcdot}\wedge\eee^{\bcdot},\\
\VVV^{~}_{a}\hspace{.2em}&:=\frac{1}{3!}\epsilon_{a\bcdot\bcdots}\hspace{.1em}
\eee^{\bcdot}\wedge\eee^{\bcdot}\wedge\eee^{\bcdot},\\
\vvv\hspace{.5em}&:=\frac{1}{4!}\epsilon_{\bcdots\bcdots}\hspace{.1em}\eee^\bcdot\wedge\eee^\bcdot\wedge\eee^\bcdot\wedge\eee^\bcdot
=\deteps\hspace{.2em}dx^0{\wedge}dx^1{\wedge}dx^2{\wedge}dx^3,
\end{align*}
which give the standard orthonormal bases of  two-, three- and four-forms.

\subsection{Principal bundles}
We introduce a spin structure into $\M$ using a spinor group $Spin(1,3)$.
A $Spin(1,3)$ group doubly covers a $SO(1,3)$ group such that $Spin(1,3)\cong SO(1,3)\times_{\Z_2}U(1)$ under covering map $\tau_\cov:Spin(1,3)\rightarrow SO(1,3)\otimes\{\pm1\}$.
The principal spinor bundle is a tuple of
\begin{align*}
\(\S^{~}_\Sp,\pi_\Sp,\M, Spin(1,3)\), 
\end{align*}
whose total space is Clifford module $\S_\Sp=\Cl(1,3)\otimes\C$, where $\Cl$ is Clifford algebra.
Projection map $\pi_\Sp$ is provided owing to the covering map as
\begin{align*}
\pi_\Sp:=\tau_\cov/\{\pm1\}:
\S_\Sp\otimes \M\rightarrow\M/\{\pm1\}:\left.\psi\right|_p\hspace{.1em}\mapsto p\in\M,
\end{align*}
where $\psi\in\Gamma\(\M,\S_\Sp\)$.
Here, $\Gamma\(\bullet,\star\)$ is a section belonging object $\star$ defined in manifold $\bullet$. 
An $SO(1,3)$ group of the inertial bundle is lifted to $Spin(1,3)$ owing to projection map $\pi_\Sp$; thus, a spin structure is induced globally in $\M$.
We assume the existence of a spin structure globally.

A principal gauge bundle is defined as a tuple such that:
\begin{align*}
\left(\TsM^{\otimes N},\pi_{\SU},\M,SU(N)\right),~&\textrm{where}~
\TsM^{\otimes N}:=\overbrace{\TsM\otimes\cdots\otimes\TsM}^{N}.
\end{align*}
A space of complex-valued scalar fields, namely the Higgs field $\bm\phi=(\phi^1,\phi^2,\cdots,\phi^N)^T$, is introduced as a section $\bm\phi\in\bm\Phi=\Gamma\left(\M,\Omega^0(\TsM^{\otimes N})\otimes{SU(N)}\right)$ belonging to the fundamental representation of the $SU(N)$ symmetry.
$SU(N)$ group operator $\Gsu$  acts on the section as
\begin{align*}
\left[\Gsu\left(\bm\phi\right)\right]^I=
\left[\bm{g}_\SU^{~}\right]^I_{J}\phi^J,
\end{align*}
where $\bm{g}_\SU^{~}$ is a unitary matrix with det$[\bm{g}_\SU^{~}]=1$.
Lie algebra of $SU(N)$ group is 
\begin{align*}
\left[\tau^{~}_I,\tau^{~}_J\right]
:=i\sum_K\hspace{.1em}f^{~}_{IJK}\hspace{.2em}\tau^{~}_K,
\end{align*}
where $f_{\bullets\bullet}$ is a structure constant of $SU(N)$, and ${\tau}=(\tau_1,\tau_2,\cdots,\tau_{N^2-1})\in\sss\uuu(N)$.

A spinor-gauge bundle is a Whitney sum of spinor and gauge bundles which is a tuple such as
\begin{align*}
\left(\S_\Sp^{\otimes N},\pi_{\Sp}\oplus\pi_{\SU},\M, Spin(1,3)\otimes{SU(N)}\right)
~&\textrm{where}~
\S_\Sp^{\otimes N}:=\overbrace{\S_\Sp\otimes\cdots\otimes\S_\Sp}^{N}.
\end{align*}
The total space of the gauge bundle is lifted to the spin manifold.

%Connections and curvatures
\subsection{Connections and curvatures}
This section introduces connections and curvatures in principal bundles given in previous section.
In physics, a connection provides a gauge-boson field mediating a force between matter fields, and a curvature is named a field strength and provides, e.g., electromagnetic fields in the $U(1)$ gauge theory.
The Lagrangian consists of the covariant differential of matter and gauge fields, ensuring the gauge invariance of the theory.
The covariant differential includes a \emph{coupling constant} in physics, which provides relative strength among fundamental forces in Nature.

On the other hand, the standard treatment of the connection and the curvature in mathematics does not include a coupling constant in their definitions.
We introduce the gravitational coupling constant to simultaneously discuss a gravitational force and other forces.
We first provide the standard definitions of the connection and the curvature in general relativity; then, we introduce the gravitational coupling constant in our theory.

%Standard definition
\subsubsection{Standard definition}
We introduce connection one-form $\hat\www$ in the spinor bundle, namely the spin-connection form. 
An object with a hat, $\hat\bullet$, represents one by the standard definition in this section.
The $SO(1,3)$-covariant differential of the $p$-form object $\aaa^{a_1a_2\cdots a_q}\in\Omega^p(\TsMM)\otimes{V}^q(\TM)$ is defined using the spin-connection form as
\begin{subequations}
\begin{align}
d_{\hat{\www}}\aaa^{a_1\cdots a_q}&:=d\aaa^{a_1\cdots a_q}+
\hat\www^{a_1}_{\hspace{.5em}\bcdot}\wedge\aaa^{\bcdot a_2\cdots a_p}+
\hat\www^{a_2}_{\hspace{.5em}\bcdot}\wedge\aaa^{a_1\bcdot\cdots a_p}+\cdots+
\hat\www^{a_q}_{\hspace{.5em}\bcdot}\wedge\aaa^{a_1\cdots\bcdot},
\label{dwwwST}
\intertext{The spin-connection form has a component representation as }
\hat\www^a_{~b}&=
\hat\omega^{\hspace{.3em}a}_{\mu\hspace{.3em}b}\hspace{.1em}dx^\mu, 
\end{align}
\end{subequations}
concerning the standard bases.
Dummy \emph{Roman} indices are often abbreviated to a small circle $\bcdot$ (and $\star$, $\times$, $\cdots$) when the dummy-index pair of the Einstein convention is obvious as above.
When multiple circles appear in an expression, the pairing must be in a left-to-right order at the subscripts and superscripts.
Raising and lowering indices are done using metric tensor $\bm\eta$ in $\M$.
Two-form object 
\begin{subequations}
\begin{align}
\hat\TTT^a&:=d_{\hat\www}\eee^a=
d\eee^a+\hat\www^{a}_{\hspace{.3em}\bcdot}\wedge\eee^\bcdot
\in\Omega^2(\TsMM)\otimes V^1(\TM)\label{torsionFMST}
\intertext{
is referred to as a torsion form.
We define a component of the torsion two-form using the standard bases as
}
\text{(\ref{torsionFMST})}&=:
\frac{1}{2}\hat\TT^a_{\hspace{.3em}\mu\nu}\hspace{.1em}dx^\mu{\wedge}dx^\nu.\label{torsionFMST2}
\end{align}
\end{subequations}

Local $SO(1,3)$ group action $\Gso:\TsM\rightarrow\TsM$ is known as the Lorentz transformation.
That for the vierbein form and the spin-connection form are
\begin{subequations}
\begin{align}
\Gso&:\eee\mapsto\Gso(\eee)=\eee'=\LLambda\eee,\label{GsoST1}\\
\Gso&:\hat\www\mapsto\Gso(\hat\www)=\hat\www'=
\LLambda\hat\www\LLambda^{-1}+\LLambda\hspace{.1em}d\hspace{-.1em}\LLambda^{\hspace{-.2em}-1}.\label{GsoST2}
\end{align}
\end{subequations}
The section belongs to the fundamental representation, and the connection does to adjoint one. 
A space-time curvature (or simply curvature) is defined owing to the structure equation as
\begin{align}
\hat\RRR^{ab}&:=d\hat\www^{ab}+\hat\www^a_{~\bcdot}\wedge\hat\www^{\bcdot b}
\in\Omega^2(\TsMM)\otimes V^2(\TM),\label{RRRST}
\end{align}
which is a two-form valued rank-$2$ tensor  represented using the trivial basis as
\begin{align*}
\hat\RRR^{ab}&=
\sum_{\mu<\nu}\hat{\Ri}^{ab}_{\hspace{.7em}\mu\nu}\hspace{.1em}dx^\mu\wedge dx^\nu =
\frac{1}{2}\hat{\Ri}^{ab}_{\hspace{.7em}\mu\nu}\hspace{.1em}dx^\mu\wedge dx^\nu
\end{align*}
Tensor coefficient $\hat{R}^{ab}_{\hspace{.7em}cd}$ is referred to as the Riemann-curvature tensor.
Ricci-curvature tensor and scalar curvature are defined, respectively, owing to the Riemann-curvature tensor as
\begin{align*}
\hat{R}^{ab}:=\hat\Ri^{\bcdot a}_{\hspace{.7em}\mu\nu}\hspace{.1em}
\E^\mu_\bcdot\E^\nu_\star\eta^{\star b}~~\textrm{and }~~
\hat{R}:=\hat\Ri^{\bcdot\star}_{\hspace{.7em}\mu\nu}\hspace{.1em}
\E^\mu_\bcdot\E^\nu_\star.%\label{RicciR}
\end{align*}
The first and second Bianchi identities are
\begin{align}
d_{\hat\www}\hat\TTT^a&=d_{\hat\www}(d_{\hat\www}\eee^a)=
\eta_{\bcdots}\hat\RRR^{a\bcdot}\wedge\eee^{\bcdot}
~~~\text{and}~~
d_{\hat\www}\hat\RRR^{ab}=0.\label{BianchiRST}
\end{align}

%Coupling constant
\subsubsection{Coupling constant}\label{CC}
We re-define connection and curvature with the gravitational coupling constant $\cGR$.
Objects without a hat represent one defined with the gravitational coupling constant corresponding to those with a hat.
We re-define the covariant differential (\ref{dwwwST}) utilising the scaling 
\begin{align}
{\cGR}\hspace{.1em}\www^{ab}:=\hat\www^{ab}\implies
{\cGR}\hspace{.1em}\omega_\mu^{\hspace{.3em}ab}=\hat\omega_\mu^{\hspace{.3em}ab},\label{cgrwww}
\end{align}
which provides
\begin{align}
d_{{\www}}\aaa^{a_1\cdots a_q}&=d\aaa^{a_1\cdots a_q}+\cGR\(
\www^{a_1}_{\hspace{.5em}\bcdot}\wedge\aaa^{\bcdot\cdots a_p}+\cdots\).
\tag*{(\ref{dwwwST})\texttt{'}}\label{dwww}
\end{align}
Accordingly, the scaled torsion form owing to $\cGR$ is
\begin{align}
\TTT^a&=d_\www\eee^a=d\eee^a+\cG\hspace{.1em}\www^a_{~\bcdot}\wedge\eee^\bcdot.
\tag*{(\ref{torsionFMST})\texttt{'}}\label{torsionFM}
\end{align}
The scaling also provides the scaled curvature as
\begin{align}
\RRR^{ab}&:=d\www^{ab}+\cG\www^a_{~\bcdot}\wedge\www^{\bcdot b},
\tag*{(\ref{RRRST})\texttt{'}}\label{RRR}
\end{align}
with 
\begin{align}
\cGR\hspace{.1em}\RRR^{ab}=\hat\RRR^{ab}\implies
\cGR\hspace{.1em}\Ri^{ab}_{\hspace{.7em}\mu\nu}=\hat\Ri^{ab}_{\hspace{.7em}\mu\nu}.\label{cdiE}
\end{align}
The Lorentz transformation for the scaled spin-connection scales as
\begin{align}
\text{(\ref{GsoST2})}&\implies
\Gso:\www\mapsto\Gso(\www)=\www'=
\LLambda\www\LLambda^{-1}+{\cG}^{\hspace{-.5em}-1}\LLambda\hspace{.1em}d\hspace{-.1em}\LLambda^{\hspace{-.2em}-1},
\tag*{(\ref{GsoST2})\texttt{'}}\label{Gso2}
\intertext{
and the first Bianchi identity as
}
\text{(\ref{BianchiRST})}&\implies
d_\www\TTT^a=d_\www(d_\www\eee^a)=\cG\hspace{.2em}
\eta_{\bcdots}\RRR^{a\bcdot}\wedge\eee^{\bcdot}.
\tag*{(\ref{BianchiRST})\texttt{'}}\label{BianchiT}
\end{align}

As seen in (\ref{RRRST}) and \ref{RRR}, a simultaneous re-scaling of the connection and curvature eliminates the coupling constant in the system.
When treating only the gravitational force, the gravitational coupling constant has no physical meaning.
However, when considering the Yang--Mills theory gauge forces in curved space-time, coupling constants provide a relative strength among forces, including the gravitational one.

%Spinor-gauge bundle
\subsection{Spinor-gauge bundle}
A connection in the gauge bundle $\AAA^{~}_{\SU}:=\AAA_\SU^I\hspace{.2em}\tau^{~}_I\in\Omega^1(\TsM)\otimes\sss\uuu(N)$ is an $SU(N)$ Lie-algebra valued one-form object.
We define covariant differential $d^{~}_\SUO$ on $p$-form object $\aaa\in\Omega^p(\TsM)$ concerning $SU(N)$ and $SO(3,1)$ as
\begin{align*}% Peskin--Schroeder p.487 eq. (15.24)
d^{~}_\SUO\hspace{.2em}\aaa\hspace{.1em}&:=
\bm{1}_\SU\hspace{.1em}d_\www\aaa-i\hspace{.1em}\cSU\hspace{.1em}
[\AAA^{~}_\SU,\aaa]_\wedge,
\intertext{where}
[\AAA^{~}_\SU,\aaa]_\wedge&:=\AAA^{~}_\SU\wedge\aaa-(-1)^{p}\hspace{.1em}\aaa\wedge\AAA^{~}_\SU,
%\label{cdSU}
\end{align*}
and $\cSU$ is a dimensionless coupling constant of the gauge interaction. 
Connection $\AAA^{~}_\SU$ belongs to an adjoint representation of the gauge group as
\begin{align*}
\Gsu\hspace{-.2em}\left(\AAA^{~}_\SU\right)&=
\bm{g}^{-1}_\SU\hspace{.2em}\AAA^{~}_\SU\hspace{.2em}\bm{g}_\SU^{~}
+i\hspace{.1em}{c^{-1}_\SU}\gdg,%\label{pisu}
\end{align*}
which ensures $SU(N)$ covariance of the covariant differential.
At the same time, $\AAA^{~}_\SU$ is a vector in $\TsM$, which is $SO(1,3)$-transformed.
Corresponding gauge curvature two-form $\FFF_\SU$ is defined through a structure equation such that:
\begin{align*}
\FFF_\SU=\FFF_\SU^I\hspace{.2em}\tau^{~}_I&:=d_\www\AAA^{~}_\SU
-i\hspace{.1em}{{\cSU}}\hspace{.1em}\AAA^{~}_\SU\wedge\AAA^{~}_\SU%,\nonumber \\&
=\left(d\AAA_\SU^I+\cG\hspace{.1em}\www\wedge\AAA_\SU^I
+\frac{{\cSU}}{2}f^I_{~JK}\hspace{.1em}\AAA_\SU^J\wedge\AAA_\SU^K
\right)\tau^{~}_I,%\label{SUCRV}
\end{align*}
where $f^I_{~JK}=f^{~}_{IJK}$.
We utilise the scaled spin-connection; thus, the gravitational coupling constant is appearing in the Lagrangian, which provides the relative strength between gravitational and $SU(N)$ gauge forces.

The gauge connection and gauge curvature are represented using the trivial bases in $\TsM$, respectively, as
\begin{align*}
\AAA^{~}_\SU&=\AAA_\SU^I\hspace{.2em}\tau^{~}_I=:\Aa^I_{a}\eee^a\hspace{.2em}\tau^{~}_I,~~\textrm{and}~~
\FFF_\SU=\FFF_\SU^I\hspace{.2em}\tau^{~}_I=:\frac{1}{2}\f^I_{ab}\hspace{.1em}\eee^a\hspace{-.1em}\wedge\eee^b\hspace{.2em}\tau^{~}_I.%\label{Fan}
\end{align*}
In this expression, tensor coefficients of the gauge curvature is provided using those of the gauge connection such that: 
\begin{align*}
\f^I_{ab}&=\partial_a\Aa^I_b-\partial_b\Aa^I_a+
{{\cSU}}\hspace{.1em}f^I_{~JK}\Aa^J_a\Aa^K_b+\Aa^I_\bcdot\TT^\bcdot_{~ab},%\label{FabISU}
\end{align*}
where $\TT$ is a tensor coefficient of the torsion form defined as  $\TTT^\bullet=:\TT^\bullet_{\hspace{.5em}\bcdots}\eee^\bcdot\wedge\eee^\bcdot/2$.
When the space-time manifold is torsion-less, the gauge curvature has the same representation as it in the flat space-time.

The $Spin(1,3)$ covariant differential acts on the spinor as\cite{fre2012gravity, moore1996lectures,Nakajima:2015rhw}
\begin{align*}
d_\www\psi:=d\psi-i\frac{\cG}{2}\hspace{.1em}\eta_\bcdots\eta_\stars\www^{\bcdot\star}\frac{\sigma^{\bcdot\star}}{2}\psi,
\end{align*}
where $\sigma^{ab}/2:=i[\gamma^a,\gamma^b]/4$ is a generator of $Spin(1,3)$.
We define the Dirac operator as
\begin{align*}
\dSp:\Gamma\left(\M,\Omega^0(\S_\Sp)\right)\rightarrow
\Gamma\left(\M,\Omega^0(\S_\Sp)\right).
\end{align*}
More precisely, it is a transformation such that:
\begin{align*}
\dSp\psi&:=\iota_{\gamma}d_\www:
\Gamma\left(\M,\Omega^0(\S_\Sp)\right)
\overset{d_\www}{\longrightarrow}
\Gamma\left(\M,\Omega^1(\S_\Sp)\right)
\overset{\iota_\gamma}{\longrightarrow}
\Gamma\left(\M,\Omega^0(\S_\Sp)\right),%\label{DSG}
\intertext{
which has an expression owing to the trivial basis as
}
\dSp(\psi)&=\left(\gamma^\bcdot\partial_\bcdot
-i\frac{\cG}{2}\gamma^\bcdot\E_\bcdot^\mu\omega_\mu^{~\stars}\frac{\sigma_\stars}{2}
\right)\psi,
\end{align*}
where $\iota_{\gamma}$ is a contraction with the Clifford algebra $\gamma^a$.

A connection and curvature of the spinor-gauge bundle are provided, respectively, as
\begin{align*}
\AAA^{ab}_\SG=\www^{ab}\otimes\bm{1}_\SU+[\bm{1}_\Sp]^{ab}\otimes\AAA^{~}_\SU,~~\textrm{and}~~
\FFF^{ab}_\SG=\RRR^{ab}_\Sp\otimes\bm{1}_\SU+[\bm{1}_\Sp]^{ab}\otimes\FFF_\SU,%\label{SGcc}
\end{align*}
We introduce the $SU(N)$ symmetric $N$ spinor as $\bm{\psi}=(\psi^1,\cdots,\psi^N)^T$.
The Dirac operator on $\bm{\psi}\in\S_\Sp^{\otimes N}$ concerning the spinor-gauge bundle is provided as
\begin{align*}
\ds_{\SG}&:\Gamma\left(\M,\Omega^0\left(\S_\Sp^{\otimes{N}}\otimes{SU(N)}\right)\right)\rightarrow
\Gamma\left(\M,\Omega^0\left(\S_\Sp^{\otimes{N}}\otimes{SU(N)}\right)\right)\\&:
\bm{\psi}\mapsto
\ds_{\SG}\bm{\psi}:=\(
\gamma^\bcdot\partial_\bcdot
-i\frac{\cG}{2}\gamma^\bcdot\E_\bcdot^\mu\omega_\mu^{~\stars}\frac{\sigma_\stars}{2}
-i{\cSU}\gamma^\bcdot\Aa^I_\bcdot \tau^{~}_I\)\bm{\psi}.
\end{align*}

%%%%%%%%%%%%%
% QGED Lagrangian %
%%%%%%%%%%%%%
\section{QGED Lagrangian}
\subsection{Bare Lagrangian}
We introduce the Lagrangian four-form in the inertial manifold as $\LLL^\bare:=\LL^\bare(\xi\in\M)\hspace{.1em}\vvv\in\Omega^4(\TsM)$, where $\LL^\bare(\xi)\in\Omega^0(\TsM)$ is a Lagrangian density.
Superscript ``$\bare$'' indicates that a given object is un-renormalised, hereafter.
We define the Lagrangian form as the null physical dimensional object; thus, the Lagrangian density has a $L^{-4}$ dimension.
A bear (un-renormalised) Lagrangian form of the QGED consists of seven parts as follows:
\begin{align*}
\LL_\QGED^\bare\hspace{.3em}:=&
 \LL_\GR^\bare+\LL_{\GR;\gfix}^\bare
+\LL_{\GR;gh}^\bare\hspace{.4em}
+\LL_{\SU}^\bare+\LL_{\SU;\gfix}^\bare
+\LL_{\SU;gh}^\bare\hspace{.4em}
+\LL_{\MT}^\bare,
\end{align*}
where
\begin{itemize}
\item Gravitational part:
\begin{itemize}
\item $\LL_\GR^\bare$ : a pure gravitational term, 
\item $\LL_{\GR;\gfix}^\bare$ : a $Spin(1,3)$ gauge fixing term,
\item $\LL_{\GR;gh}^\bare$ : a Fadeev--Poppov ghost term for the $Spin(1,3)$ gauge group,
\end{itemize}
\item $SU(N)$ gauge part:
\begin{itemize}
\item  $\LL_{\SU}^\bare$: an $SU(N)$ gauge boson term,
\item $\LL_{\SU;\gfix}^\bare$ : an $SU(N)$ gauge fixing term,
\item $\LL_{\SU;gh}^\bare$ : a Fadeev--Poppov ghost term for for the $SU(N)$ gauge group,
\end{itemize}
\item Matter part:
\begin{itemize}
\item $\LL_{\MT}^\bare$ : a matter (fermion) field term.
\end{itemize}
\end{itemize}
A pure gravitational part is provided by the Einstein--Hilbert gravitational Lagrangian such as
\begin{align}
\LLL_\GR^\bare&:=
\frac{1}{4{\lp}^{\hspace{-.2em}2}}\epsilon_{\bcdots\bcdots}
\hat\RRR^{\bare\hspace{.1em}\bcdots}\wedge\eee^{\bare\hspace{.1em}\bcdot}\wedge\eee^{\bare\hspace{.1em}\bcdot}.
\label{Lgr}
\end{align}
We define the Lagrangian using the non-scaled curvature. 
Component representations of the gravitational Lagrangian in $\TsMM$ and $\TsM$ are, respectively,  
\begin{subequations}
\begin{align}
\LLL_\GR^\bare&=
\frac{1}{4{\lp}^{\hspace{-.2em}2}}\epsilon_{\bcdots\bcdots}
\(\cGz\partial^{~}_\mu\omega_{\hspace{.7em}\nu}^{\bare\hspace{.1em}\bcdots}+
\cGz^2\omega_{\hspace{.7em}\mu\hspace{.3em}\star}^{\bare\hspace{.1em}\bcdot}\hspace{.1em}
\omega_{\hspace{.7em}\nu}^{\bare\hspace{.1em}\star\bcdot}\)
\E^{\bare\hspace{.1em}\bcdot}_{\hspace{.7em}\rho}\E^{\bare\hspace{.1em}\bcdot}_{\hspace{.7em}\sigma} 
dx^\mu{\wedge}dx^\nu{\wedge}dx^\rho{\wedge}dx^\sigma,\label{Lgr1}\\
&=
\frac{1}{4{\lp}^{\hspace{-.2em}2}}\epsilon_{\bcdots\bcdots}
\(\cGz\partial^{~}_a\omega_{\hspace{.7em}b}^{\bare\hspace{.1em}\bcdots}+
\cGz^2\omega_{\hspace{.7em}a\hspace{.3em}\star}^{\bare\hspace{.1em}\bcdot}\hspace{.1em}
\omega_{\hspace{.7em}b}^{\bare\hspace{.1em}\star\bcdot}\)
\eee^{\bare\hspace{.1em}a}\wedge\eee^{\bare\hspace{.1em}b}\wedge
\eee^{\bare\hspace{.1em}\bcdot}\wedge\eee^{\bare\hspace{.1em}\bcdot}.\label{Lgr2}
\end{align}
\end{subequations}
We note that
\begin{align*}
\frac{\partial\omega^{~\bullets}_{b}(\xi)}{\partial \xi^a}\eee^a\wedge \eee^b&=
\E^\mu_a(\xi)\(\frac{\partial~}{\partial x^\mu}\omega^{~\bullets}_{\nu}(x)\E_b^\nu(\xi)\)
d\xi^a\wedge d\xi^b,\\
&=\(\frac{\partial~}{\partial x^\mu}\omega^{~\bullets}_{\nu}(x)\)\E^\mu_a(\xi)\E_b^\nu(\xi)d\xi^a\wedge d\xi^b,\\
&=\frac{\partial\omega^{~\bullets}_{\nu}(x)}{\partial x^\mu}dx^\mu\wedge dx^\nu,
\end{align*}
and thus, the covariant differential gives the same expression both in $\TsMM$ and $\TsM$ as shown in (\ref{Lgr1}) and (\ref{Lgr2}). 
Other Lagrangian forms are provided using the standard bases as:
\begin{align}
\LLL_{\MT}^\bare&:=\bar{\psi}^\bare\(i\hspace{.1em}\gamma^\bcdot\partial_\bcdot
-\frac{1}{\hbar}{m^\bare}
-i\frac{\cGz}{2}\gamma^\bcdot\E_{\hspace{.7em}\bcdot}^{\bare\hspace{.1em}\mu}
\omega_{\hspace{.7em}\mu}^{\bare\hspace{.1em}\stars}\frac{\sigma_\stars}{2}
+\cSUz\hspace{.1em}\gamma^\bcdot\Aa^{\bare I}_{\hspace{.7em}\bcdot}\hspace{.1em}\tau_I
\)\psi^\bare\vvv,\\
\LLL_\SU^\bare&:=\frac{1}{4}
\sum_{I=1}^N
\f^{\bare I}_{\hspace{.7em}\bcdots}\hspace{.2em}
\f^{\bare\hspace{.1em}I\bcdots} %\text{Tr}[\tau_I\bullet\tau_I]\)
\hspace{.1em}\vvv.
\end{align}
Lagrangian densities are defined in $\TsM$.
We decompose the bare QGED Lagrangian into free and interaction parts according to the standard perturbative quantization method such as
\begin{subequations}
\begin{align}
&\left\{
\begin{array}{cl}
\LL_{\GR;free}^\bare&:=\cGz
\(\partial^{~}_a\hspace{.1em}\omega_{~~b}^{\bare\hspace{.1em}ab}-
\partial^{~}_b\hspace{.1em}\omega_{~~a}^{{\bare}ab}\)/2{\lp}^{\hspace{-.2em}2},\\
\LL_{\GR;int}^\bare&:=\cGz^2\eta_{\bcdots}
\(\omega_{~~a}^{{\bare}a\bcdot}\hspace{.1em}\omega_{~~b}^{{\bare}\bcdot b}-
\omega_{~~b}^{{\bare}a\bcdot}\hspace{.1em}\omega_{~~a}^{{\bare}\bcdot b}\)/2{\lp}^{\hspace{-.2em}2},
\end{array}
\right.\label{LGrfreeint}\\
%%
&\left\{
\begin{array}{cl}
\LL_{\MT;free}^\bare&:=
\bar{\psi}^\bare\(i\hspace{.1em}\gamma^\bcdot\partial^{~}_\bcdot
-\frac{1}{\hbar}{m_e^\bare}\)\psi^\bare,\\
\LL_{\MT;int}^\bare&:=
\bar{\psi}^\bare\(
-i\cGz\hspace{.1em}\gamma^\bcdot\E_\bcdot^\mu\omega_\mu^{~\stars}\sigma_\stars/4
+\cSUz\gamma^\bcdot\Aa^{\bare I}_{\hspace{.7em}\bcdot}\hspace{.1em}\tau_I\)\psi^\bare,
\end{array}
\right.\label{LMTfreeint}\\
%%
&\left\{
\begin{array}{cl}
\LL_{\SU;free}^\bare&:=
\sum_{I=1}^N\eta^\bcdots\eta^\stars
\(\partial^{~}_\bcdot\Aa^{\bare I}_{\hspace{.7em}\star}-\partial^{~}_\star\Aa^{\bare I}_{\hspace{.7em}\bcdot}\)
\(\partial^{~}_\bcdot\Aa^{\bare I}_{\hspace{.7em}\star}-\partial^{~}_\star\Aa^{\bare I}_{\hspace{.7em}\bcdot}\)
/4,\\
\LL_{\SU;int}^\bare&:=\cSUz\sum_{I=1}^N\eta^\bcdots\eta^\stars
\(\partial^{~}_\bcdot\Aa^{\bare I}_{\hspace{.7em}\star}-\partial^{~}_\star\Aa^{\bare I}_{\hspace{.7em}\bcdot}\)
\(f^I_{~JK}\Aa^{\bare J}_{\hspace{.7em}\bcdot}\Aa^{\bare k}_{\hspace{.7em}\star}\)/2\\
~&+\cSUz^2\sum_{I=1}^N\eta^\bcdots\eta^\stars
\(f^I_{~JK}\Aa^{\bare J}_{\hspace{.7em}\bcdot}\Aa^{\bare K}_{\hspace{.7em}\star}\)
\(f^I_{~LM}\Aa^{\bare L}_{\hspace{.7em}\bcdot}\Aa^{\bare M}_{\hspace{.7em}\star}\)/4.
\end{array}
\right.\label{LSUfreeint}
\end{align}
\end{subequations}
where $\psi^\bare$ is an $N$-spinors $\psi^\bare:=(\psi^\bare_i,\cdots,\psi^\bare_N)$. 
Trivial $\bm{1}_\SU$ is omitted in formulae.
Later after providing ghost Lagrangian, we decompose it into free and interaction parts, too.
We note that the identity
\begin{align*}
\epsilon_{abcd}\hspace{.1em}\eee^a\wedge\eee^b\wedge\eee^{c'}\wedge\eee^{d'}
=2\(\delta^c_{c'}\delta^d_{d'}-\delta^c_{d'}\delta^d_{c'}\)\vvv
\end{align*}
is utilised in (\ref{LGrfreeint}).


%
% BRST transformation
%
\subsection{BRST transformation}\label{BRST}
The standard procedure of canonical quantization based on the BRST symmetry\cite{Becchi:1974md,Tyutin:1975qk} is well established.
In this study, we utilise a quantization method proposed by Nakanishi\cite{Nakanishi01061966}, Kugo and Ojima\cite{kugo1979local,Kugo1978459}.
The Nakanishi--Kugo--Ojima method introduces the auxiliary field and the Fadeev--Popov ghost field and sets the BRST transformation on all fields that appear in theory.
This section provides a set of the BRST transformations for the QGED.

The BRST transformations for un-renormalised fields are common to those for renormalised fields; thus, we omit suffix ``$\hspace{.1em}\bare$'' in the following.

%Yang--Mills theory
\subsubsection{Yang--Mills theory}
In the standard quantum field theory textbooks, we can find the detailed procedure of quantising the Yang--Mills theory based on the BRST symmetry, e.g., Ref.\cite{peskin1995introduction,weinberg1996quantum}.
Even though that, this section summarises the BRST transformation for the Yang--Mils theory in the inertial manifold and provides a guide to constructing the BRST transformation of the gravitational fields.

We introduce classical auxiliary and Faddeev-Popov ghost/anti-ghost fields as follows:
\begin{itemize}
\item auxiliary field:\hspace{2.9em}$B^{I}(\xi)$,
\item ghost field:\hspace{4.5em}$\chi^{I}(\xi)$,
\item anti-ghost field:\hspace{2.4em}$\bar{\chi}^{I}(\xi)$,
\end{itemize}
where suffix $I$ concerns the $SU(N)$ gauge group.
Each number of auxiliary, ghost and anti-ghost fields is the same as that of the degree of freedom of the gauge group.
For the $SU(N)$ symmetry, it is $N^2-1$.
The connection field (gauge boson) belongs to the adjoint representation of the gauge group; thus, the number of the above fields is the same as that of gauge bosons. 
The ghost/anti-ghost fields are the Grassmann number following an anti-commutable product. 
%We omit ``$(\xi)$'' in fields, hereafter.
This study denotes the BRST transformation concerning the $SU(N)$ gauge group as $\delBRST^\SU[\bullet]$.
The BRST transformations for classical fermion and $SU(N)$ gauge fields are
\begin{align*}
\delBRST^\SU[\psi]&:=i\cSU\hspace{.1em}\chi^{I}\tau^{~}_I\psi,\\
\delBRST^\SU[\AAA_\SU^{I}]&:=d^{~}_\SUO\hspace{.1em}\chi^I
=\(\partial^{~}_\bcdot\chi^I+\cSU\hspace{.1em}f^I_{\hspace{.2em}JK}\Aa^J_\bcdot\chi^K\)\eee^\bcdot.
\end{align*}
We note that  $d_\www s(\xi)=ds(\xi)$ for a local scalar function $s(\xi)$.
The BRST transformation on the Grassmann numbers $X$ and  $Y$ fulfils the Leibniz rule such that
\begin{align*}
\delBRST^\SU[XY]=\delBRST^\SU[X]\hspace{.1em}Y+\epsilon_X X\hspace{.1em}\delBRST^\SU[Y],
\end{align*}
where the signature $\epsilon_X=-1$ for $X\in\{\chi^{I},\bar{\chi}^{I}\}$, and $\epsilon_X=+1$ otherwise.
We define the BRST transformations for auxiliary and ghost/anti-ghost fields as
\begin{align*}
\delBRST^\SU[B^I]&:=0,\\
\delBRST^\SU[\chi^I]&:=-\frac{1}{2}\cSU\hspace{.1em}f^I_{\hspace{.2em}JK}\hspace{.1em}\chi^J\chi^k,\\
\delBRST^\SU[\bar\chi^I]&:=B^I.
\end{align*}
The BRST transformation for vierbein and spin-connection fields gives zero.
Simple calculations show the BRST transformation is nilpotent for any fields.

%General relativity
\subsubsection{General relativity}
The current author has developed the canonical quantization of general relativity in the Heisenberg picture utilising the Nakanishi--Kugo--Ojima quantization method\cite{doi:10.1140/epjp/s13360-021-01463-3}.
This section discusses the BRST transformation of gravitational fields in the interaction picture following the method in the preceding section.

We introduce the auxiliary field and Faddeev--Popov ghost/anti-ghost fields as follows:
\begin{itemize}
\item auxiliary field:\hspace{2.8em}$\beta_{\mu}^{~a}(x)$,
\item ghost fields:\hspace{4em}$\chi^{a}_{\hspace{.3em}b}(\xi)$ and $\chi^\mu(x)$,
\item anti-ghost field:\hspace{2.5em}$\bar{\chi}_{\mu}^{\hspace{.3em}a}(x)$.
\end{itemize}
Here, ghost fields are Grassmannian.
Auxiliary field  $\beta_{\mu}^{\hspace{.3em}a}(x)$ plays a role in fixing $Spin(1,3)$ gauge symmetry in the quantization of the spin-connection and is decoupled from the physical system after the gauge fixing.
Ghost fields $\chi^{a}_{\hspace{.3em}b}(\xi)$ and $\chi^\mu(x)$ were introduced to preserve the unitarity of the scattering amplitude, corresponding to the local Lorentz and global coordinate transformations, respectively.
We assign a ghost number $+1$ for $\chi^{a}_{\hspace{.3em}b}$ and $-1$ for $\bar{\chi}_{\mu}^{\hspace{.3em}a}$. 
An anti-ghost field corresponding to the global ghost is unnecessary since it is decoupled from the physical system in the inertial space.

In this study, we denote the BRST transformation for gravitational fields as $\delBRST^\GR[\bullet]$, and require rules introduced by Nakanishi\cite{Nakanishi01071978} as follows: 
The BRST transformation of the coordinate vector in $\MM$ should obey the general linear transformation as follows:\begin{align}
\delBRST^\GR\left[x^\mu\right]&=\chi^\mu.\label{BRSTx}
\end{align}
In addition, we require the postulate given in \cite{Nakanishi01071978}, such as
\begin{align}
\delBRST^\GR\left[\partial^{~}_\mu X\right]
&=\partial^{~}_\mu\delBRST^\GR\left[X\right]
-\(\partial^{~}_\mu\delBRST^\GR\left[x^\nu\right]\)\partial^{~}_\nu X
=\partial^{~}_\mu\delBRST^\GR\left[X\right]
-\(\partial^{~}_\mu\chi^\nu\)\partial^{~}_\nu X,\label{brstdelX}
\end{align}
where $X$ is any field defined in $\TMM$; thus, the BRST transformation acts on a one-form object in $\TsMM$ as
\begin{align*}
\delBRST^\GR\left[dx^\mu\right]&=\left(\partial^{~}_\nu\delBRST^\GR\left[x^\mu\right]\right)dx^\nu
=d\left(\delBRST^\GR\left[x^\mu\right]\right)
=~d\chi^\mu.
\end{align*}
Consequently, the BRST transformation and external derivative are commute with each other, i.e., 
\begin{align*}
\left[\delBRST^\GR,d\right]\bullet&=\delBRST^\GR\left[d\bullet\right]-
d\left(\delBRST^\GR\left[\bullet\right]\right)=~0,%\label{dBRST}
\end{align*}

We define the BRST transformations of auxiliary and ghost/anti-ghost fields by reference to those of the Yang--Mills theory as
\begin{subequations}
\begin{align}
\delBRST^\GR\left[\beta_{\mu}^{\hspace{.3em}a}\right]&=\delBRST^\GR\left[\chi^\mu\right]~=~0,\label{BRSTchi1}\\
\delBRST^\GR\left[\chi^{a}_{\hspace{.3em}b}\right]\hspace{.3em}&=\cGR\hspace{.1em}
\chi^{a}_{\hspace{.3em}\bcdot}\hspace{.1em}\chi^{\bcdot}_{\hspace{.3em}b},\label{BRSTchi2}\\
\delBRST^\GR\left[\bar{\chi}_{\mu}^{~a}\right]&=\lp\hspace{.1em}\beta_{\mu}^{~a}.\label{BRSTchi3}
\end{align}
\end{subequations}
Here, we have set the BRST transformations to preserve the physical dimension of each field.
The BRST transformation of the position vector given in (\ref{BRSTx}) provides the global ghost field with a Length dimension.
The auxiliary field has the same physical dimension as the spin-connection field; it is $L^{-1}$ dimension.
The ghost field must be a null dimension object owing to (\ref{BRSTchi2}).
We also set anti-ghost fields as a null physical dimension; thus, we introduce the Planck length in (\ref{BRSTchi3}) to adjust a physical dimension.

The BRST transformation satisfies the following Leibniz rule:
\begin{align}
\delBRST^\GR\left[XY\right]&=\delBRST^\GR\left[X\right]Y+\epsilon_XX\delBRST^\GR\left[Y\right],\label{Leib}
\end{align}
where the signature $\epsilon_X=-1$ for $X\in\{\chi^\mu,\chi_a^{\hspace{.3em}b},\bar{\chi}_{\mu}^{\hspace{.3em}a}\}$, and $\epsilon_X=+1$ otherwise.
The BRST transformations of vierbein and spin-connection forms are defined as
\begin{align*}
% 09/June/2030 change sign of derivative terms !
\delBRST^\GR\left[\eee^a\right]&:=
\cGR\hspace{.1em}\chi^{a}_{~\bcdot}\hspace{.1em}\eee^\bcdot,\\
\delBRST^\GR\left[\www^{ab}\right]
&:=
%d_\www\chi^{ab}=
% 06/June/2030 change sign of derivative terms !
d\chi^{ab}
-\cGR\(\www^{a}_{~\bcdot}\hspace{.1em}\chi^{\bcdot b}
+\www^{b}_{\hspace{.3em}\bcdot}\hspace{.1em}\chi^{a \bcdot}\),
\end{align*}
where $\chi^{ab}:=\chi^{a}_{\hspace{.3em}\bcdot}\eta^{\bcdot b}$.
The local ghost field is anti-symmetric owing to the above definition.
They induce the BRST transformations of vierbein and spin-connection fields as
\begin{subequations}
\begin{align}
\delBRST^\GR\left[\E^a_\mu\right]&=
-\E^a_\nu\partial^{~}_\mu\chi^{\nu}
+\cGR\hspace{.1em}\chi^{a}_{\hspace{.3em}\bcdot}\hspace{.1em}\E^{\bcdot}_\mu,\\
\delBRST^\GR\left[\omega^{\hspace{.3em}ab}_{\mu}\right]&=
% 06/June/2030 change sign of derivative terms !
\E_\mu^\bcdot\partial^{~}_\bcdot\chi^{a b}+\(\partial^{~}_\mu\chi^\nu\)\omega^{\hspace{.3em}ab}_{\nu}
-\cGR\(
\omega_{\mu\hspace{.3em}\bcdot}^{~a}\hspace{.2em}\chi^{\bcdot b}+
\omega_{\mu\hspace{.3em}\bcdot}^{\hspace{.3em}b}\hspace{.2em}\chi^{a\bcdot}_{~}\).\label{brstwabc}
\intertext{
The BRST transformation of the vierbein inverse is obtained after simple calculations as
}
\delBRST^\GR\left[\E_a^\mu\right]&=
\E_a^\nu\partial^{~}_\nu\chi^{\mu}
-\cGR\chi^{\bcdot}_{\hspace{.3em}a}\hspace{.1em}\E_{\bcdot}^{\mu},
\end{align}
where $\delBRST^\GR\left[\E^a_\mu\hspace{.1em}\E_a^\nu\right]=\delBRST^\GR\left[\delta_\mu^\nu\right]=0$ is used.
\end{subequations}

We introduce one-form objects of auxiliary and ghost/anti-ghost fields as
\begin{subequations}
\begin{align}
\bbb^{a}&:=\(\lp\hspace{.1em}\beta_{\mu}^{~a}-\bar\chi^{~a}_{\nu}\hspace{.1em}\partial^{~}_\mu\chi^\nu\)dx^\mu,\label{bfGR}\\
\ccc^{a}&:=\chi^{a}_{\hspace{.3em}\bcdot}\hspace{.1em}\eee^\bcdot,\label{cfGR}\\
\bar{\ccc}^{a}&:=\bar{\chi}_{\mu}^{~a}\hspace{.1em}dx^\mu=\bar\chi^{~a}_{\bcdot}\hspace{.1em}\eee^\bcdot,\label{cbfGR}
\end{align}
where $\bar\chi_{b}^{~a}:=\bar\chi_\mu^{~a}\hspace{.1em}\E^\mu_{b}$.
\end{subequations}
They have a physical dimension as $\left[\bbb^{a}\right]=\left[\ccc^{a}\right]=\left[\bar{\ccc}^{a}\right]=L$, and their BRST transformations are
\begin{align*}
\delBRST^\GR\left[\bbb^{a}\right]=\delBRST^\GR\left[\ccc^a\right]=0~~&\textrm{and}~~
\delBRST^\GR\left[\bar{\ccc}^{a}\right]=\bbb_{~}^a.
\end{align*}
Auxiliary form has representations in the inertial space as
\begin{align*}
\text{(\ref{bfGR})}=\(\lp\beta_{\bcdot}^{\hspace{.3em}a}-\bar\chi^{\hspace{.3em}a}_{\star}\hspace{.1em}
\partial^{~}_\bcdot\chi^\star\)\eee^\bcdot\in\TsM,
~~&\text{where}~~
\beta_{b}^{\hspace{.3em}a}:=\beta_{\mu}^{\hspace{.3em}a}\hspace{.1em}\E^\mu_{b},~~
\chi^a:=\E^a_\mu\hspace{.1em}\chi^\mu.
\end{align*}
We represent the auxiliary form in $\TsM$ as $\bbb^{a}=B_{\hspace{.2em}\bcdot}^{a}\eee^\bcdot$ using components defined as
\begin{align*}
B_{\hspace{.2em}b}^{a}&:=\(\lp\beta_{b}^{\hspace{.3em}a}-
\bar\chi^{\hspace{.3em}a}_{\bcdot}\hspace{.1em}
\partial^{~}_b\chi^\bcdot\).
\end{align*}
Since we define the Lagrangian form in $\TsM$, Romanised forms appear in the Lagrangian.
We provide the BRST transformation of (\ref{brstdelX}) and (\ref{brstwabc}) in the inertial space for future convenience:
\begin{align*}
\delBRST^\GR\left[\partial^{~}_aX(\xi)\right]
&=\partial^{~}_a\(\delBRST^\GR\left[X\right](\xi)\)
-\cGR\hspace{.1em}\chi^\bcdot_{\hspace{.3em}a}\(\partial^{~}_\bcdot X(\xi)\),\\
\delBRST^\GR\left[\omega_c^{\hspace{.3em}ab}\right]
&=\partial^{~}_c\chi^{ab}-\cGR\(
 \omega_{c\hspace{.3em}\bcdot}^{\hspace{.3em}a}\hspace{.1em}\chi^{\bcdot b}
+\omega_{c\hspace{.3em}\bcdot}^{\hspace{.3em}b}\hspace{.1em}\chi^{a\bcdot}
+\omega_{\bcdot}^{\hspace{.3em}ab}\hspace{.1em}\chi^{\bcdot}_{\hspace{.3em}c}
\).
\end{align*}
We note that the BRST transformation on the object defined in the inertial space does not include the global ghost field, as shown above.
Consequently, the global ghost field does not appear in the ghost Lagrangian defined in the inertial space.

Direct calculations show that the BRST transformations are nilpotent for all forms and fields.
Consequently, the classical Lagrangian $\LL_\GR$ is BRST-invariant.
 In this study, we simplified definitions of auxiliary and ghost/anti-ghost fields from those in Ref.\cite{doi:10.1140/epjp/s13360-021-01463-3}.
Although calculations are almost identical to those in Ref.\cite{doi:10.1140/epjp/s13360-021-01463-3}, we provide proof of nilpotency for all necessary forms in \textbf{Appendix~\ref{app1}} for completeness.

%
% Ghost and gauge fixing Lagrangian
%
\subsection{Ghost and gauge fixing Lagrangian}\label{Lghost}
This section introduces gauge fixing and ghost Lagrangians to the bare quantum Lagrangian according to the standard prescription:
Prepare the necessary number of functions $F[\Aa, B, \chi, \bar\chi]$, which includes the same number of ghost and anti-ghost fields. 
Then, add a term 
\begin{align*}
\LL_{\bullet;\gfix}+\LL_{\bullet;gh}:=\delBRST^\bullet[\bar{\chi}\hspace{.1em}F]
\end{align*}
to the Lagrangian.
As a result, the total Lagrangian is BRST-invariant, and after fixing all gauge degrees, the Lagrangian is still keeping the BRST symmetry.
We omit suffix ``$\hspace{.1em}\bare$'' suffix for the bare fields in this section, too.

%Yang--Mills theory
\subsubsection{Yang--Mills theory}
For the $SU(N)$ gauge part, we introduce the scalar field as
\begin{align*}
H^{I}&:=\eta^\bcdots\partial^{~}_\bcdot\Aa^{I}_{\hspace{.4em}\bcdot}+
\frac{1}{2}\xi_AB^{I},
\intertext{which gives the gauge fixing and ghost Lagrangians, such that}
\LL_{\SU;\gfix}+\LL_{\SU;gh}&:=\sum_{I=1}^N\delBRST^\SU[\bar{\chi}^{I}H^{I}].
\end{align*}
It is decomposed into
\begin{subequations}
\begin{align}
\LL_{\SU;\gfix}&:=\sum_{I=1}^N\delBRST^\SU[\bar{\chi}^{I}]H^{I}=\sum_{I=1}^N\(
B^{I}\eta^\bcdots\partial^{~}_\bcdot\Aa^{I}_{\hspace{.4em}\bcdot}
+\frac{\xi_A}{2}\hspace{.1em}B^{I}\hspace{-.1em}B^{I}\),\label{Lsugf1}\\
\LL_{\SU;gh}&:=\sum_{I=1}^N\bar{\chi}^{I}\delBRST^\SU[H^{I}]
=\bar\chi^{I}\eta^\bcdots\partial^{~}_\bcdot\(
\partial^{~}_\bcdot\chi^{I}+\cSU\hspace{.1em}f^I_{\hspace{.2em}JK}\Aa^{J}_{\hspace{.4em}\bcdot}\chi^{K}\).\label{LsuFP}
\end{align}
We can write the gauge fixing Lagrangian as
\begin{align}
\text{(\ref{Lsugf1})}&=\sum_{I=1}^N\frac{1}{2\xi_A}\left\{
\(C^{I}\)^2-\(\eta^\bcdots\partial^{~}_\bcdot\Aa^{I}_{\hspace{.4em}\bcdot}\)^2
\right\},\label{Lsugf}
\intertext{where}
C^{I}&:=\eta^\bcdots\partial^{~}_\bcdot\Aa^{I}_{\hspace{.4em}\bcdot}+
\xi_AB^{I}.\notag
\end{align}
\end{subequations}
Therefore, scalar field $C^{I}$ (and thus $B^{I}$, too) is decoupled from the other parts, and the standard gauge fixing term 
\begin{align}
\LL_{\SU;\gfix}&:=-\frac{1}{2\xi^{~}_{\hspace{-.1em}A}}\sum_{I=1}^N
\left(\eta^{\bcdots}\partial^{~}_{\bcdot}\Aa^{I}_{\hspace{.4em}\bcdot}\right)^2,\label{Acovg}
\end{align}
and interaction vertices among gauge bosons and ghosts remain in the Lagrangian.
The gauge fixing condition provides one constraint on four components of $\Aa^a$.
Together with an on-shell (mass-less) condition, two dynamical degrees (transverse polarization) remain on a $SU(N)$ gauge boson. 

Owing to definitions, we can confirm that the Lagrangian is BRST-invariant:
\begin{align*}
\delBRST^\SU\left[\LL_{\SU}+\LL_{\SU;\gfix}
+\LL_{\SU;gh}
+\LL_{\MT}\right]=0.
\end{align*}

%General relativity
\subsubsection{General relativity}
This section provides gauge fixing and ghost Lagrangians for a gravity part according to the same prescription used for the Yang--Mills theory.
We introduce a one-form object
\begin{align*}
\HHH^{a}&:=
\ddd\hspace{-.1em}\www^{a}
+\frac{1}{2\lpt}\xi_\omega\hspace{.1em}\bbb^{a}
,~\text{with}~~
\ddd\hspace{-.1em}\www^{a}:=\(\partial^{~}_\bcdot\omega^{\hspace{.3em}a\bcdot}_{\star}\)\eee^\star.
\end{align*}
We note that $\ddd\hspace{-.1em}\www^{a}$ has a length-inverse dimension and $\bbb^{a}$ has length dimension; thus, $\HHH^{a}$ has a length-inverse dimension. 
We define gauge fixing and ghost Lagrangians using this one-form object as
\begin{align*}
\LLL_{\GR;\gfix}+\LLL_{\GR;gh}&:=\frac{1}{\lpt}\delBRST^\GR\left[
\bar\ccc^{ \bcdot}\hspace{-.2em}\wedge\HHH^{ \bcdot}
\hspace{-.2em}\wedge\SSS_{\bcdots}
\right],
\end{align*}
where the Planck length appears to adjust the Lagrangian forms to a null-dimension.
Owing to the definition and nilpotent of each field, this is BRST invariant. 
We decompose it into the gauge fixing and ghost parts, such that:
\begin{subequations}
\begin{align}
\lpt\hspace{.1em}\LLL_{\GR;\gfix}&:=\delBRST^\GR\left[\bar\ccc^{\bcdot}\right]\wedge\HHH^{\bcdot}
\hspace{-.2em}\wedge\SSS_{\bcdots}
=\bbb^{\bcdot}\hspace{-.2em}\wedge\(\ddd\hspace{-.1em}\www^{\bcdot}
+\frac{1}{2\lpt}\xi_\omega\hspace{.1em}\bbb^{\bcdot}
\hspace{-.2em}\)\wedge\SSS_{\bcdots}
,\label{Lgrgf}\\
\lpt\hspace{.1em}\LLL_{\GR;gh}&:=-\bar\ccc^{\bcdot}\wedge
\delBRST^\GR\left[\HHH^{\bcdot}\hspace{-.2em}\wedge\SSS_{\bcdots}\right].\label{LgrFP}
\end{align}

First, we look at the gauge fixing Lagrangian.
We define  a new one-form object 
\begin{align*}
\CCC^{a}:=\ddd\hspace{-.1em}\www^{a}+\frac{\xi_\omega}{\lpt}\hspace{.1em}\bbb^{a} 
\end{align*}
yielding 
\begin{align}
\LLL_{\GR;\gfix}&=\frac{\text{(\ref{Lgrgf})}}{\lpt}=\frac{1}{2\xi_\omega}\left\{
\CCC^{\bcdot}\wedge\CCC^{\bcdot}\wedge\SSS_{\bcdots}-
\(\(\hspace{.1em}\partial^{~}_\bcdot\omega_\star^{~\star\bcdot}\)^2
-\(\partial^{~}_\bcdot\omega_\star^{\hspace{.3em}\times\bcdot}\)
 \(\partial^{~}_\bcdot\omega_\times^{\hspace{.3em}\star\bcdot}\)\)\vvv
\right\}.\label{Lgrgf2}
\end{align}
\end{subequations}
The first term of (\ref{Lgrgf2}) has representation in the global space like
\begin{align*}
\CCC^{\bcdot}\wedge\CCC^{\bcdot}\wedge\SSS_{\bcdots}
=C^{\bcdot}_{\hspace{.2em}\star}\hspace{.1em}C^{\times}_{\hspace{.2em}\ast}
\(\delta^\star_\bcdot\hspace{.1em}\delta^\ast_\times-\delta^\ast_\bcdot\hspace{.1em}\delta^\star_\times\)
\vvv,
%\deteps\hspace{.2em}dx^0{\wedge}dx^1{\wedge}dx^2{\wedge}dx^3,
\end{align*}
where $\CCC^{a}=:C^{a}_{\hspace{.2em}\bcdot}\hspace{.1em}\eee^{\bcdot}$.
Thus, $\CCC^{a}$ is isolated from the physical system. 

The spin-connection $\omega_{c}^{\hspace{.3em}ab}$ has $24$ components owing to anti-symme concerning superscripts. 
In addition, the torsion-less condition provides twelve constraints on the spin-connection field and implies a relation on the Riemann curvature such that:
\begin{align}
R_{ab;cd}&=R_{cd;ab},~~\text{where}~~
R_{ab;cd}:=R^\bcdots_{\hspace{.7em}cd}\hspace{.1em}\eta^{~}_{\bcdot a}\hspace{.1em}\eta_{\bcdot b}.\label{Rabcdcdab}
\end{align}
The Riemann curvature fulfiling (\ref{Rabcdcdab}) can be constructed using a half of the spin-connection fields of
\begin{align}
\omega_{\dot{a}}^{\hspace{.4em}b}&:=\omega_{\hat a}^{\hspace{.4em}\hat{a}b}.\label{wpolvec}
\end{align}
We do not apply the Einstein convention of repeated suffixes for those with a hat. 
There are twelve independent components in (\ref{wpolvec}) owing to four constraints of $\hat{a}\neq b$.
We set remnant twelve components to 
\begin{align}
\omega_{a}^{\hspace{.4em}bc}&=0~~\text{for}~~a\neq b\land a\neq c,\label{wconst} 
\end{align}
without loss of generality.
Therefore, we re-define the gauge fixing Lagrangian as
\begin{subequations}
\begin{align}
\LLL_{\GR;\gfix}&:=-\frac{1}{2\xi_\omega}
\(\partial^{~}_\bcdot\omega^{\hspace{.3em}\star\bcdot}_{\star}\)^2\vvv,
\label{Wgauge}
\intertext{which corresponds to set the covariant gauge fixing condition as}
\partial^{~}_\bcdot\omega_{\hat a}^{\hspace{.3em}\hat a\bcdot}&=0.\label{Wgauge2}
\end{align}
\end{subequations}
Equation (\ref{Wgauge2}) provides four constraints on $\omega_{\dot{a}}^{\hspace{.4em}b}$.
In addition, there are six on-shell conditions; thus, two physical degrees of freedom remain in theory.
The graviton is the quantized spin-connection field in our theory and is a massless spin-two particle.
The little group concerning the graviton in four-dimensional space-time is $SO(2)$, which has two degrees of freedom and is consistent with the above discussion.

Next, we construct the ghost Lagrangian, which provides an interaction term among ghosts and physical fields.
We require further simplification for the ghost/anti-ghost fields.
The role of the Faddeev--Popov ghost is to preserve the Ward identity and to ensure the unitarity of gravitational scattering amplitudes.
The quantum theory requires ghost fields as many as a dimension of the gauge group; that of the Lorentz group is six.
However, all six components of the graviton are not independent, but it has only two independent components, as discussed above.
Thus, we need a scalar field each for ghost and anti-ghost fields.

Owing to the BRST transformation of the spin-connection field, the local ghost is anti-symmetric concerning its two suffixes.
We parameterize the local ghost field as
\begin{subequations}
\begin{align}
\eta_{a\bcdot}\hspace{.1em}\chi^\bcdot_{\hspace{.3em}b}(\xi)&=\chi_{ab}(\xi):=\chi(\xi)\hspace{.2em}c_ac_b,
\end{align}
where $c_a:=(c_0,c_1,c_2,c_3)$ is a Grassmannian vector.
On the other hand, we set the anti-ghost fields a symmetric tensor:
\begin{align}
\bar\chi^{\hspace{.3em}a}_{b}(\xi)=\bar\chi(\xi)\hspace{.2em}\delta^{a}_{b}.\label{scalarchi}
\end{align}
Both $\chi(\xi)$ and $\bar\chi(\xi)$ are Grassmannian scalar fields.
The auxiliary form also has the same degree of freedom and the same convention as that of the anti-ghost field, such that
\begin{align}
B_{\hspace{.2em}b}^{a}(\xi)&=B(\xi)\hspace{.2em}\delta^{a}_{b}.\label{scalarB}
\end{align}
\end{subequations}
In the following calculations, we keep all components of the spin-connection field without requiring the torsion-less condition.

The ghost Lagrangian splits into two parts owing to the Leibniz rule of the BRST transformation, such that:
\begin{align}
-(\ref{LgrFP})&=
\bar\ccc^{\bcdot}\wedge
\delBRST^\GR\left[\HHH^{\bcdot}\right]\wedge\SSS_{\bcdots}
+\bar\ccc^{\bcdot}\wedge\HHH^{\bcdot}\hspace{-.2em}\wedge
\delBRST^\GR\left[\SSS_{\bcdots}\right].\label{LgrFP2}
\end{align}
Simple calculations provide the BRST transformation of the surface form and $\HHH^a$, respectively,  as
\begin{align*}
\delBRST^\GR\left[\SSS_{ab}\right]
&=\frac{1}{2}\epsilon_{ab\bcdots}\delBRST^\GR\left[\eee^{\bcdot}\wedge\eee^{\bcdot}\right]
=-\cGR\hspace{.1em}\epsilon_{ab\bcdot\star}
\chi\hspace{.1em}c_{~}^\bcdot c^{~}_\times\hspace{.1em}\eee^\times\hspace{-.2em}\wedge\eee^\star,\\
\delBRST^\GR\left[\HHH^a\right]&=
\left\{
\(\partial^{~}_\bcdot\partial^{~}_\times\chi\){c^a}{c^\bcdot}+
\cGR
\(\hspace{.1em}\omega_{\times\hspace{.2em}\star}^{\hspace{.3em}a}\(\partial^{~}_\bcdot\chi\){c^\star}{c^\bcdot}+\(\partial^{~}_\bcdot\omega_{\times\hspace{.2em}\star}^{\hspace{.3em}\bcdot}\hspace{.1em}\chi\){c^a}{c^\star}
+\(\partial^{~}_\bcdot\omega_{\star}^{\hspace{.3em}a\bcdot}\hspace{.1em}\chi\){c^\star}{c_\times}
\)\right\}\eee^\times
\end{align*}
Thus, we obtain each term of (\ref{LgrFP2}) as
\begin{align*}
\bar\ccc^{\bcdot}\wedge\HHH^{\bcdot}\hspace{-.2em}\wedge
\delBRST^\GR\left[\SSS_{\bcdots}\right]&=
\bar{\chi}\(\partial^{~}_\bcdot\omega_{\star}^{\hspace{.3em}\times\bcdot}\)
\chi\hspace{.1em}{c^\star}{c_\times}\vvv=0,
%
\intertext{and}
%
\bar\ccc^{\bcdot}\wedge
\delBRST^\GR\left[\HHH^{\bcdot}\right]\wedge\SSS_{\bcdots}&=
3\(\({c^\bcdot}\partial^{~}_\bcdot\bar{\chi}\)
   \({c^\star}\partial^{~}_\star\chi\)+
\cGR\hspace{.1em}\bar{\chi}\hspace{.1em}\omega_{\star\hspace{.2em}\bcdot}^{\hspace{.3em}\star}
\(\partial^{~}_\times\chi\){c^\bcdot}{c^\times}\)\vvv.
\end{align*}
As a result, we obtain the ghost Lagrangian such that:
\begin{align}
\lpt\LLL_{\GR;gh}&=-\(\({c^\bcdot}\partial^{~}_\bcdot\bar{\chi}\)
   \({c^\star}\partial^{~}_\star\chi\)+\cGR\hspace{.1em}\bar{\chi}\hspace{.1em}\omega_{\star\hspace{.2em}\bcdot}^{\hspace{.3em}\star}
\(\partial^{~}_\times\chi\){c^\bcdot}{c^\times}\)\vvv\label{Lgrgh},
\end{align}
after the re-scaling of scalar ghost/anti-ghost fields.
%At the last step of the above calculation, we use the integration by parts with appropriate boundary condition.
 
Although we keep all components of the spin-connection fields in the above calculations, the result includes only components appearing in (\ref{wpolvec}); thus, our parameterizations of ghost field, (\ref{scalarchi}), is consistent with  that of the connection fields.
Consequently, the interaction terms consist of scalar ghost/anti-ghost fields and polarization vectors $\omega_{\dot{a}}^{\hspace{.3em}b}$.
 
The nilpotent of gravitational fields ensures the BRST invariance of the gravitational Lagrangian, such that:
\begin{align*}
\delBRST^\GR\left[\LLL_{\GR}+\LLL_{\GR;\gfix}+\LLL_{\GR;gh}\right]=0.
\end{align*}

%
% QGED Feynman rules 
%
\section{QGED Feynman rules}\label{FeynmanRules}
This section formulates Feynman rules concerning the QGED Lagrangian.
Hereafter, we treat only $U(1)$ gauge as a gauge group and one spinor field (electron field).
Thus, the theory is electrodynamics, and the dimensionless coupling constant is an electric charge; $\cSU=e$.
For curved space-time, a momentum space, in which Feynman rules are defined, is not trivial.
In this study, we utilise the definition of the momentum space introduced in Ref.\cite{Kurihara:2022green} and denote it by $\TstM$.
We follow a phase-convention of the Feynman rules in Ref.\cite{10.1143/PTPS.73.1}.  

Feynman rules for bare fields are common to those for renormalised fields. Therefore, continuing from the previous section, this section also omits the suffix ``$\hspace{.1em}\bare$''.

%
% Green's function of vierbein and spin-connection fields
%
\subsection{Green's function of vierbein and spin-connection fields}\label{greensfunctionWE}
This section provides Green's function of vierbein and spin-connection fields in the configuration space.
The Bianchi identity \ref{RRR} has a tensor representation like
\begin{align*}
\partial^{~}_\bcdot\(R^{a\bcdot}-\frac{1}{2}\eta^{a\bcdot}R\)=
\partial_{~}^\bcdot\Ri^{a\star}_{\hspace{.7em}\bcdot\star}
-\frac{1}{2}\partial_{~}^a\Ri^{\bcdot\star}_{\hspace{.7em}\bcdot\star}
=0.
\end{align*}
We use anti-symmetry of $\Ri$ concerning both the subscripts and superscripts. 
Owing to the definition of the curvature, we obtain an equation for the classical spin-connection field as\cite{Kurihara:2022green}
\begin{align}
\(\partial_{~}^\bcdot\partial^{~}_\bcdot\omega_{\star}^{~a\star}-
\partial_{~}^\bcdot\partial^{~}_\star\omega_{\bcdot}^{~a\star}-
\partial_{~}^a\partial^{~}_\star\omega_\bcdot^{~\star\bcdot}\)+
\cG\partial_{~}^\bcdot\(
\omega_\bcdot^{~a\times}\omega_\star^{~\times\star}-
\omega_\star^{~a\times}\omega_\bcdot^{~\times\star}-
\delta^a_\bcdot\omega_\star^{~\star\times}\omega_\star^{~\times\star}
\)
\eta_{\times\times}=0.\label{EinsteinEq3}
\end{align}
Using gauge fixing condition (\ref{Wgauge2}) and free field condition $\cG=0$  in (\ref{EinsteinEq3}), we obtain an equation of motion for the free spin-connection in the inertial space, such that:
\begin{align}
\(\partial_{~}^\bcdot\partial^{~}_\bcdot\omega_{\star}^{~\star a}\)\(\xi\)&=0.\label{deltaomega}
\end{align}
The fundamental solution of (\ref{deltaomega}) provides Green's function of the spin-connection as
\begin{align*}
{G^{~}_{\omega}}(\xi):=\frac{\omega_{\hspace{1em}\dot{a}}^{(\lambda)\hspace{.4em}b}}
{-(\xi^0)^2+(\xi^1)^2+(\xi^2)^2+(\xi^3)^2-i\delta},%\label{GW}
\end{align*}
where $\omega_{\hspace{1em}\dot{a}}^{(\lambda)\hspace{.4em}b}$ is polarization vectors of the spin-connection field.
Infinitesimal constant $\delta>0$ is a parameter to determine the analyticity of Green's function.

The torsion two-form has a tensor representation in $\TsM$ and $\TsMM$ such that:
\begin{align*}
\TTT^a&=:\frac{1}{2}\TT^a_{\hspace{.6em}\bcdots}\hspace{.2em}\eee^\bcdot\wedge\eee^\bcdot=
\frac{1}{2}\TT^a_{\hspace{.6em}\bcdots}\hspace{.2em}\E^\bcdot_\mu\E^\bcdot_\nu
\hspace{.2em}dx^\mu{\wedge}dx^\nu.
\end{align*}
The torsion-less equation, which is obtained as the Euler--Lagrange equations owing to Lagrangian (\ref{Lgr}),  is 
\begin{align}
\TT^a_{\hspace{.6em}bc}&=\E^a_\mu
\(\partial^{~}_c\E_b^\mu-\partial^{~}_b\E_c^\mu\)+
\cG\(\omega_{b\hspace{.3em}c}^{\hspace{.3em}a}-\omega_{c\hspace{.3em}b}^{\hspace{.3em}a}\)=0.\label{torsionlesscp}
\end{align}
Divergence concerning the superscript of the torsion form, $\partial^{~}_\bcdot\TT^\bcdot_{\hspace{.6em}bc}$, and gauge fixing condition (\ref{Wgauge}) provides constraints such as\cite{Kurihara:2022green}
\begin{align}
\partial^{~}_\mu\E^\mu_a&=0,
\end{align}
which is consistent with the de\hspace{.15em}Donder gauge condition: $\partial^{~}_\mu(\sqrt{-g}g^{\mu\nu})=0$.
On the other hand, divergence concerning the subscript of the torsion-less equation provides an equation of motion for a free vierbein field in the global space-time manifold such as
\begin{align}
\TMM\ni g^{\mu\nu}\partial^{~}_{\mu}\partial_{\nu}\E^{a}_{\rho}(x)=0.\label{deltaE}
\end{align}
The fundamental solution of equation (\ref{deltaE}) is
\begin{align}
{G^a_{\E}}(x):=\frac{\E_\mu^{(a)}}
{-(x^0)^2+(x^1)^2+(x^2)^2+(x^3)^2-i\delta}\label{GE}
\end{align}
where ${\E}_{\mu}^{(a)}$ is polarization vectors of the vierbein field.
Polarization vectors are components of a spin-one vector defined in $\TsMM$ such that
\begin{align*}
{\eee}^{(a)}:={\E}^{(a)}_\mu(x)dx^\mu,
\end{align*}
where $a=1,2$ shows two independent polarization vectors.
We interpret that the subscript represents a vector component concerning the global space-time manifold, and the superscript points to an inner degree of freedom corresponding to two polarization vectors.
When we interpret a vierbein field $\E^a_\mu$ as a component of the global vector defined in $\TsMM$ concerning the subscript $\mu$, a local $SO(1,3)$ symmetry is an internal degree of freedom concerning the superscripts $a$, which is interpreted as a polarization degree of a vierbein field.
We can construct a spin-two polarization vector of the metric tensor using the coupling of angular momenta\cite{Kurihara:2022green}.

%We note that the free spin-connection field propagates in the inertial manifold; on the other hand, the free vierbein field does in the global space-time manifold.

%
% external field
%
\subsection{External field}
The QGED Lagrangian consists of four external physical fields: vierbein field, spin-connection field, photon, and electron.
In addition to that, the QGED has ghost/anti-ghost fields.
Among them, photons and electrons are defined purely in the inertial space.
The spin-connection and local ghost/anti-ghost fields are also objects defined in the inertial space after the Romanisation of the subscript.
The spin-connection field propagates in the inertial bundle as a wave according to equation (\ref{deltaomega}).
On the other hand, the vierbein field $\E_\mu^a(x\in\MM)$ is an object essentially defined in the global space-time and propagating in it according to the wave equation (\ref{deltaE}).
The role of the vierbein is twofold in the QGED: one is a dynamic field propagating in global space-time, and the other is a transformation function converting a Greek index into a Roman index.
An electron and a photon do not couple to the vierbein field directly through the interaction Lagrangian.
Therefore, this study treats the vierbein field as a transformation function between the global and inertial space and does not discuss an external vierbein field in curved space-time in this study.

\subsubsection{Electromagnetic part}
\paragraph{spinor:}
According to the standard perturbation method, we introduce electron and positron wave functions as follows:
Spinor fields of electron and positron, $u(p,s)$ and  $v(p,s)$, are positive  and negative frequency parts of the plane-wave solution of the Dirac equation with momentum $p^{~}_a=(p^{~}_0,\vec{p})\in V^1(\TstM)$  and spin $s$ as follows:
\begin{align}
\hbar^{3/2}\psi(\xi)=:\frac{1}{(2\pi\hbar)^{3}}\int\frac{d^3\vec{p}}{\sqrt{2E_p}}
\sum_{s=1,2}\(
a^{~}_e(p,s)u(p,s)e^{-i\xi\cdot p/\hbar}+
b^{\dagger}_e(p,s)v(p,s)e^{+i\xi\cdot p/\hbar}
\),\label{spnorm}
\end{align}
where $E_p:=\sqrt{\vec{p}\cdot\vec{p}+m^2}$ is an on-shell energy, and $a^{~}_e(p,s)$ and $b^{\dagger}_e(p,s)$ are an electron's creation and annihilation operators, respectively.
The canonical commutation relation of the creation/annihilation operators is
\begin{align}
\left\{a^{~}_e(\vec{p},s),a^{\dagger}_e(\vec{p}\hspace{.1em}',s')\right\}=
\left\{b^{~}_e(\vec{p},s),b^{\dagger}_e(\vec{p}\hspace{.1em}',s')\right\}=\hbar^3
\delta(\vec{p}-\vec{p}\hspace{.1em}')\delta_{ss'},\label{ccrpsi}
\end{align}
and otherwise zero.
Electron's creation and annihilation operators have $L^{3/2}$ dimension owing to the commutationrelation (\ref{ccrpsi}); thus,
spinors $u(p,s)$ and $v(p,s)$ have a physical dimension of $E^{1\hspace{-.11em}/\hspace{-.09em}2}$. 
%The annihilation operator of an electron is reinterpreted as the creation operator of a positron.
Spinors are normalized as 
\begin{align*}
\bar{u}(p,s)\hspace{.1em}\gamma_0\hspace{.1em}u(p,s')=2E_p\hspace{.1em}\delta_{ss'},~&\textrm{and}~~
\bar{v}(p,s)\hspace{.1em}\gamma_0\hspace{.1em}v(p,s')=2E_p\hspace{.1em}\delta_{ss'}.
\end{align*}
The polarization sums of an electron and a positron are
\begin{align}
\sum_su(p,s)\bar{u}(p,s)=\slash{p}+m,~&\textrm{and}~~
\sum_sv(p,s)\bar{v}(p,s)=\slash{p}-m.\label{fnorm}
\end{align}

\paragraph{photon:}
The Fourier transformation of transversely polarised $U(1)$ gauge field $\Aa^{T}_a(\xi)$ provides a polarization vector of a physical photon, such that: 
\begin{align*}
\hbar^{-1/2}\hspace{.1em}\Aa^{T}_a(\xi)=:\frac{1}{(2\pi\hbar)^{3}}
\int\frac{d^3\vec{k}}{\sqrt{2k_0}}
\sum_{\lambda=1,2}\epsilon_a^{(\lambda)} \(
a^{~}_\gamma(k,\lambda)e^{-i\xi\cdot k/\hbar}+
a_\gamma^\dagger(k,\lambda)e^{+i\xi\cdot k/\hbar}
\),
\end{align*}
where $k_0$ is an on-shell energy such that $k_0^2-\vec{k}\cdot\vec{k}=0$, and $a^{~}_\gamma(k,\lambda)$ and $a_\gamma^\dagger(k,\lambda)$ are, respectively, creation and annihilation operator of a photon with momentum $k$ and polarization $\lambda$.
Creation and annihilation operators has $L^{3/2}$ physical dimension owing to the canonical quantization conditions, e.g.,
\begin{align}
\left[a^{~}_\gamma(\vec{k},\lambda),a_\gamma^\dagger(\vec{k}\hspace{.1em}',\lambda')\right]&=
\hbar^3\delta^3\hspace{-.2em}\(\vec{k}-\vec{k}\hspace{.1em}'\)\delta_{\lambda'\lambda}.\label{CQe}
\end{align}
Consequently, polarization vector $\epsilon_a^{(\lambda)}$ has null physical dimension.
A photon polarization vector is normalized as
\begin{align}
\sum_{\lambda=1,2}\epsilon_a^{(\lambda)}(k)\epsilon_b^{(\lambda)}(k)&=
-\eta_{ab}+\(1-\xi_A\)\frac{k_ak_b}{\eta_\bcdots k^\bcdot k^\bcdot},\label{Anorm}
\end{align}
in the covariant gauge (\ref{Acovg}).

\subsubsection{Gravitational part:}
\paragraph{vierbein:}
We consider an asymptotic state of the vierbein field in the asymptotically flat space-time with $g_{\mu\nu}=\textup{diag}\(1,-1,-1,-1\)$. 
%We consider a vierbein's polarization vector for a plane wave solution in a flat global space whose metric tensor is $g_{\mu\nu}=\textup{diag}\(1,-1,-1,-1\)$.
The Fourier transformation of the vierbein field of the fundamental solution (\ref{GE}) provides a polarization vector in the momentum space, such that:
\begin{align}
\(\kE\hbar\)^{-1}\E_\mu^{a}(x)=:\frac{1}{(2\pi\hbar)^{4}}
\left.\int{d^4p}\hspace{.2em}
%\sum_{\lambda=0}^{4}
\tilde{\E}_\mu^{(\lambda)} \(
{a}_{\E}^{~}(p,\lambda)e^{-ix\cdot p/\hbar}+
{a}_{\E}^\dagger(p,\lambda)e^{+ix\cdot p/\hbar}\label{FourierE}
\)\right|_{\lambda=a}.
\end{align}
We note that the vacuum concerning the vierbein field is not a flat metric in general, and the Fourier transformation and the momentum space in the curved space-time is not trivial\cite{Kurihara:2022green}.
The vierbein fields appear in the integral kernel of the Fourier transformation as
\begin{align*}
x\cdot p:=g_{\mu\nu}(x)\hspace{.1em}x^\mu p^\nu=
\eta_\bcdots\hspace{.1em}\E^\bcdot_\mu(x)\E^\bcdot_\nu(x)x^\mu p^\nu,
\end{align*}
and thus, the integration (\ref{FourierE}) is not well-defined in the curved space-time.
We consider the Fourier transformation only in the asymptotic frame assumed to be flat.
%Therefore, this report restricts our target only to the flat space, and a question about the vierbein's vacuum for the curved space-time is open for future studies.

The equation of motion of the free vierbein field in the flat space-time shown in (\ref{deltaE})  is the same as that of a photon; thus, the Fock space of a free (asymptotic) vierbein operator is the same as that of a photon.
%We note that equation (\ref{deltaE}) is not a wave function when $g_{\mu\nu}\neq\textrm{diag}(1,-1,-1,-1)$.
The canonical quantization condition requires an equal-time commutation relation on creation and annihilation operators of the physical vierbein fields, such as\cite{nakanishi1990covariant}
\begin{align}
\left[a^{~}_\E(p,\lambda),a_\E^\dagger(p',\lambda')\right]&=\hbar^4
\delta\hspace{-.2em}\(p_0-p_0'\)
\delta^3\hspace{-.2em}\(\vec{p}-\vec{p}\hspace{.1em}'\)
\delta_{\lambda'\lambda}.\label{CQv}
\end{align}
Consequently, annihilation and creation operators of the vierbein field have the length squared dimension, and a polarization vector in the momentum space has a null physical dimension.

We introduce four polarization vectors of the asymptotic vierbein field propagating along, e.g., the $x^1$-axis, such as
\begin{align*}
&{\tilde{\E}}_{\mu}^{(\lambda)}:=\frac{1}{\sqrt{2}}
\left\{
\begin{array}{lc}
(1,\hspace{.5em}i,~0,\hspace{.8em}0), & (\lambda=0),\\
(i,\hspace{.5em}1,~0,\hspace{.8em}0), & (\lambda=1),\\
(0,~0,~1,\hspace{.8em}1), & (\lambda=2),\\
(0,~0,~1,-1), & (\lambda=3).
\end{array}
\right.
\end{align*}
with a circular polarization.
The vierbein field is constructed from polarization vectors, such as 
\begin{align*}
\E_\mu^a:=&\tilde{\E}_\mu^{(\lambda=a)},
\intertext{which provides a flat metric tensor}
g_{\mu\nu}\hspace{.3em}=&\eta^{~}_\bcdots\E_\mu^\bcdot\E_\nu^\bcdot=
\textup{diag}\(1,-1,-1,-1\).
\end{align*}
This relation implies the normalisation of the polarization vector as
\begin{align}
\sum_{\lambda,\lambda'}\eta^{~}_{\lambda'\lambda}\tilde{\E}_\mu^{(\lambda)}(p)\hspace{.1em}
\tilde{\E}_\nu^{(\lambda')}(p)&=g_{\mu\nu}.\label{EEnorm}
\end{align}
Each vierbein polarization vector represents the spin-one state, and a symmetric product (\ref{EEnorm}) concerning two Lorentz indexes constructs a spin-two state of the quantised metric tensor.

Two polarization vectors, 
\begin{align*}
\ppp^{(\lambda=\bm{+})}:={\tilde{\E}}_{\mu}^{(2)}dp^\mu=\frac{1}{\sqrt{2}}\(dp^2+dp^3\)
~&\textrm{and }~~
\ppp^{(\lambda=\bm{\times})}:={\tilde{\E}}_{\mu}^{(3)}dk^\mu=\frac{1}{\sqrt{2}}\(dp^2-dp^3\),
\end{align*}
have a dynamic degree and are physically observable in the quantum field theory, similar to the polarization vector of the transversely polarised photon in the QED.
We use representations $\lambda=2=:\bm{+}$ and $\lambda=3=:\bm{\times}$ according to the standard convention.
Polarization vectors  
\begin{align*}
\tilde{\E}^{(s)}_\mu:=\frac{1}{\sqrt{2}}\(\tilde{\E}^{(0)}_\mu-i\tilde{\E}^{(1)}_\mu\),~&\textrm{and}~~
\tilde{\E}^{(l)}_\mu:=\frac{1}{\sqrt{2}i}\(\tilde{\E}^{(0)}_\mu+i\tilde{\E}^{(1)}_\mu\),
\end{align*}
correspond to a scalar vierbein and a longitudinally polarised vierbein, respectively.
polarization vectors satisfy the covariant gauge condition for the on-shell momentum $\tilde{p}_\mu=(\tilde{p},0,0,\tilde{p})$ as
\begin{align*}
g^{\mu\nu}\tilde{p}_\mu\hspace{.1em}\tilde{\E}^{(\lambda)}_\nu=0
\end{align*}
for $\lambda=\bm{+},\bm{\times},s$.

\paragraph{spin-connection:}
The Fourier transformation of the spin-connection field provides a polarization vector in the momentum space, such that:
\begin{align*}
\hbar^{-1/2}\hspace{.1em}\omega^{(\lambda)\hspace{.3em}b}_{\hspace{1.em}\dot{a}}(\xi)=:\frac{1}{(2\pi\hbar)^{3}}
\int\frac{d^3\vec{p}}{\sqrt{2p_0}}
\sum_{\lambda=1,2}\tilde\omega^{(\lambda)\hspace{.3em}b}_{\hspace{1.em}\dot{a}}(p) \(
a_{\omega}^{~}(p,\lambda)e^{-i\xi\cdot p/\hbar}+
a_{\omega}^\dagger(p,\lambda)e^{+i\xi\cdot p/\hbar}%\label{FourierW}
\).
\end{align*}
In contrast to the Vierbein field defined on the global manifold, the spin connection field is defined on the local inertial manifold after Romanising the subscripts. Its Fourier transform is, therefore, well-defined in inertial space.
The equation of motion for the free spin-connection field $\omega_{\bcdot}^{\hspace{.3em}\bcdot a}$  shown in (\ref{deltaomega}) is the same as that of a photon; thus, the Fock space of a free vierbein operator is the same as that of a photon, too. 
We set a canonical quantisation condition of an equal-time commutation relation on creation and annihilation operators of physical spin-connection fields  as 
\begin{align*}
\left.\left[a^{~}_\omega(p,\lambda),a_\omega^\dagger(p',\lambda')\right]\right|_{p^{~}_0={p^{~}_0}'}&=
\hbar^3\hspace{.1em}\delta^3\hspace{-.2em}\(\vec{p}-\vec{p'}\)\delta_{\lambda'\lambda}.
\end{align*}
Consequently, a polarisation vector of the spin-connection field in the momentum space has null physical dimension.

We can construct polarization vectors of the spin-two field for the spin-connection in the momentum space.
For a spin-connection momentum $p^{~}_a:=(p^{~}_0,p\sin{\vartheta}\cos{\varphi},p\sin{\vartheta}\sin{\varphi},p\cos{\vartheta})$, we introduce three linearly polarised  vectors, such that:
\begin{align*}
\tilde{\varepsilon}_{a}^{(\lambda=1)}&:=
\(0,\cos{\vartheta}\cos{\varphi},\cos{\vartheta}\sin{\varphi},-\sin{\theta}\),\\
\tilde{\varepsilon}_{a}^{(\lambda=2)}&:=\(0,-\sin{\varphi},\cos{\varphi},0\),\\
\tilde{\varepsilon}_{a}^{(\lambda=3)}&:=
\frac{1}{\sqrt{p_0^2-p^2}}\(p,p_0\sin{\vartheta}\cos{\varphi},p_0\sin{\vartheta}\sin{\varphi},p_0\cos{\vartheta}\).
\end{align*}
The polarizations $\lambda=1,2$ are the transverse components of the physical spin-connection, and $\lambda=3$ corresponds longitudinal one appearing in the virtual state in the propagator.
The circularly polarised vectors are constructed using them as
\begin{subequations}
\begin{align}
\tilde{\varepsilon}_{a}^{(+)}&:=\frac{1}{\sqrt{2}}\(\tilde{\varepsilon}_{1}^{(1)}+i\tilde{\varepsilon}_{a}^{(2)}\)\label{wlambda1},\\
\tilde{\varepsilon}_{a}^{(-)}&:=\frac{1}{\sqrt{2}}\(\tilde{\varepsilon}_{1}^{(1)}-i\tilde{\varepsilon}_{a}^{(2)}\)\label{wlambda2},\\
\tilde{\varepsilon}_{a}^{(l)}&:=\tilde{\varepsilon}_{1}^{(3)}\label{wlambda3}.
\end{align}
\end{subequations}
They satisfy the covariant gauge fixing conditions of $\eta^\bcdots p_\bcdot\tilde{\varepsilon}_\bcdot^{(\lambda)}=0$.

We can construct polarization vectors of the spin-connection fields owing to spin-one polarization vectors (\ref{wlambda1})$\sim$(\ref{wlambda3}) as
\begin{subequations}
\begin{align}
\tilde{\omega}^{(0)}_{\hspace{1em}\dot{a}b}&:=
\tilde{\varepsilon}_{a}^{(+)}\tilde{\varepsilon}_{b}^{(-)}-
\tilde{\varepsilon}_{a}^{(-)}\tilde{\varepsilon}_{b}^{(+)},\label{wlambda21}\\
\tilde{\omega}^{(+1)}_{\hspace{1em}\dot{a}b}&:=
\tilde{\varepsilon}_{a}^{(+)}\tilde{\varepsilon}_{b}^{(l)}-
\tilde{\varepsilon}_{a}^{(l)}\tilde{\varepsilon}_{b}^{(+)},\label{wlambda22}\\
\tilde{\omega}^{(-1)}_{\hspace{1em}\dot{a}b}&:=
\tilde{\varepsilon}_{a}^{(-)}\tilde{\varepsilon}_{b}^{(l)}-
\tilde{\varepsilon}_{a}^{(l)}\tilde{\varepsilon}_{b}^{(-)},\label{wlambda23}
\end{align}
where a value of $\dot{a}$ equals to that of  $a$.
\end{subequations}
Here, we utilise a convention
\begin{align*}
\tilde{\omega}^{(\lambda)}_{\hspace{1em}\dot{a}b}:=\tilde{\omega}^{(\lambda)\hspace{.3em}\hat{a}\bcdot}_{\hspace{1em}\hat{a}}\eta^{~}_{\bcdot b}\big|_{\hat{a}\rightarrow\dot{a}}
~&\text{and}~~
\tilde{\omega}^{(\lambda)}_{\hspace{1em}a\dot{b}}:=\tilde{\omega}^{(\lambda)\hspace{.3em}\bcdot\hat{b}}_{\hspace{1em}\hat{b}}\eta^{~}_{\bcdot a}\big|_{\hat{b}\rightarrow\dot{b}}
\end{align*}
to distinguish two different polarization vectors concerning two subscripts. 
They are anti-symmetric concerning two suffixes and satisfy the covariant gauge condition as
\begin{align*}
\tilde{\omega}^{(\lambda)}_{\hspace{1em}\dot{a}b}=-\tilde{\omega}^{(\lambda)}_{\hspace{1em}b\dot{a}}~&\text{and}~~
\eta^\bcdots p^{~}_\bcdot\hspace{.1em}\tilde{\omega}^{(\lambda)}_{\hspace{1em}\dot{a}\bcdot}=0.
\end{align*}
Polarization vectors $\tilde{\omega}^{(+1)}_{\hspace{1em}\dot{a}b}$, $\tilde{\omega}^{(0)}_{\hspace{1em}\dot{a}b}$ and $\tilde{\omega}^{(-1)}_{\hspace{1em}\dot{a}b}$ provide the spin-two polarization tensors with a helicity state $-1$, $0$ and $+1$. 

Polarization vectors (\ref{wlambda21})$\sim$(\ref{wlambda23}) provide the polarization sum such that
\begin{align}
\sum_{\lambda=-1,0,1}
\tilde{\omega}^{(\lambda)}_{\hspace{1em}\dot{a}b}\hspace{.2em}
\tilde{\omega}^{(\lambda)}_{\hspace{1em}\dot{c}d}&=
\(\eta_{ac}^{~}+\(1-\xi^{~}_\omega\)\frac{p^{~}_a p^{~}_c}
{\eta_{~}^\bcdots\hspace{.1em} p^{~}_\bcdot p^{~}_\bcdot}\)
\(\eta_{bd}^{~}+\(1-\xi^{~}_\omega\)\frac{p^{~}_b p^{~}_d}
{\eta_{~}^\bcdots\hspace{.1em} p^{~}_\bcdot p^{~}_\bcdot}\)\notag\\
&-
\(\eta_{ad}^{~}+\(1-\xi^{~}_\omega\)\frac{p^{~}_a p^{~}_d}
{\eta_{~}^\bcdots\hspace{.1em} p^{~}_\bcdot p^{~}_\bcdot}\)
\(\eta_{bc}^{~}+\(1-\xi^{~}_\omega\)\frac{p^{~}_b p^{~}_c}
{\eta_{~}^\bcdots\hspace{.1em} p^{~}_\bcdot p^{~}_\bcdot}\)
\label{Wnorm}
\end{align}
in the covariant gauge fixing (\ref{Wgauge}) with the unitary gauge ($\xi^{~}_\omega=0$).

\paragraph{ghosts:}
Although a ghost field for the $U(1)$ gauge does not appear in scattering processes, that for gravity is indispensable in the QGED.
We define generalised four-momenta concerning ghost/anti-ghost fields as
\begin{align*}
\pi^a_\chi:=\frac{\delta\LL_{\GR;gh}}{\delta\(\partial^{~}_a\chi\)}=
-\eta^{a\bcdot}\partial^{~}_\bcdot\bar{\chi}~~&\text{and}~~
\pi^a_{\bar\chi}:=\frac{\delta\LL_{\GR;gh}}{\delta\(\partial^{~}_a\bar{\chi}\)}
=\eta^{a\bcdot}\partial^{~}_\bcdot\chi
-\cGR\hspace{.1em}
\omega^{\hspace{.3em}a\star}_{\star}\chi,
\end{align*}
owing to ghost Lagrangian (\ref{Lgrgh}).
The Fourier transformation of ghost/anti-ghost fields provides those in the momentum space, such that:
\begin{align}
\(\kE\hbar\)^{-1}\hspace{.1em}\chi(\xi)=:\frac{1}{(2\pi\hbar)^{4}}
\int{d^4p}\hspace{.2em}
\tilde{\chi}(p)\(
{a}_{\chi}^{~}(p)e^{-i\xi\cdot p/\hbar}+
{a}_{\chi}^\dagger(p)e^{+i\xi\cdot p/\hbar}
\)~~\text{and}~~(\chi\rightarrow\bar\chi,~\tilde\chi\rightarrow\tilde{\bar\chi}).
\end{align}
The equation of motion for ghost/anti-ghost fields are the Klein--Gordon equation; thus, the Fock space of a free ghost/anti-ghost operators is the same as that of the standard scalar field. 
Commutation relations for creation and annihilation operators of ghost/anti-ghost fields are
\begin{align*}
\left\{a^{~}_\chi(p),a_\chi^\dagger(p')\right\}&=
\left\{a^{~}_{\bar\chi}(p),a_{\bar\chi}^\dagger(p')\right\}=
\hbar^4
\delta\hspace{-.2em}\(p_0-p_0'\)
\delta^3\hspace{-.2em}\(\vec{p}-\vec{p}\hspace{.1em}'\).
\end{align*}
Consequently, ghost/anti-ghost fields in the momentum space have a null physical dimension.
We normalised ghost/anti-ghost fields in the momentum space such that
\begin{align}
\tilde\chi(p)^2=\bar{\tilde\chi}(p)^2=1.\label{chinorm}
\end{align}
Fields appearing in the QGED and their physical dimensions are summarised in Table \ref{table3}.
%%%%%%%%%%%%%%%%%%%%%%%%%%%%%%%%%%%%%%%%%%%
% Figure environment removed
%%%%%%%%%%%%%%%%%%%%%%%%%%%%%%%%%%%%%%%%%%%
%%%%%%%%%%%%%%%%%%%%%%%%%%%%%%%%%%%%%%%%%%%%%%%%%%%%%%%
\begin{table}[b]
\begin{center}
\caption{\label{table3}\small
Fields in the QGED and its physical dimension in the Lagrangian and Feynman rules.}
\vskip 2mm
\begin{tabular}{cl||ll|ll}	
\multicolumn{2}{c||}{Field} & configurations space &dim. & momentum space &dim.\\
\hline
electron&\multirow{2}{*}{(section)}
& \hspace{3em}$\psi(\xi)$ & $L^{-3/2}$& \hspace{3em}$u(p,s)$ & $E^{\hlf}$\\
vierbein&  & \hspace{3em}$\E^a_\mu(x)$ &{\it 1} & \hspace{3em}$\tilde{\E}^{(\lambda)}_\mu(p)$&{\it 1} \\
\hline
photon&\multirow{2}{*}{(connection)}
& \hspace{3em}$\Aa_a(\xi)$ & $L^{-1}$ & \hspace{3em}$\epsilon^{(\lambda)}_a(p)$ & {\it 1}\\
spin-connection& & \hspace{3em}$\omega_c^{~ab}(\xi)$ & $L^{-1}$ &\hspace{3em}$\tilde{\omega}^{(\lambda)a}(p)$  &{\it 1}\\
\hline
\multicolumn{2}{c||}{ghost/anti-ghost}&\hspace{2.5em}$\chi(\xi)$/$\bar\chi(\xi)$&{\it 1}&\hspace{2.5em}$\tilde\chi(p)$/$\tilde{\bar\chi}(p)$&{\it 1}
\end{tabular}
\end{center}
\end{table}
%%%%%%%%%%%%%%%%%%%%%%%%%%%%%%%%%%%%%%%%%%%%%%%%%%%%%%%

%
% Vertices
%
\subsection{Vertices}
In our convention, we define the Feynman rule for the interaction vertex owing to the Fourier transform of the Lagrangian without any extra phase (imaginary unit).
The vertex rule provides a multiplicable factor after truncated wave functions and creation/annihilation operators.
We provide a gravitational interaction vertex of an electron as
\begin{subequations}
\begin{align}
\widetilde{V}_{\omega\psi}&:=-i\frac{\cG}{2}\gamma^\bcdot\E_\bcdot^\mu\tilde\omega_\mu^{~\stars}\frac{\sigma_\stars}{2}
=\frac{\cG}{4}\gamma^\bcdot\tilde\omega_\bcdot^{~\stars}\gamma_\star\gamma_\star,\label{Vwpsi0}\\
&=\frac{\cG}{2}\eta^\bcdots\sum_{a=0}^3\(\gamma^a\gamma_\bcdot\gamma_a\)
\tilde\omega_{\dot{a}\bcdot}
=\cG\frac{D-2}{2}\eta^\bcdots\gamma_\bcdot\tilde\omega_{\dot\Sigma\bcdot}.\label{Vwpsi}
\end{align}
where $\tilde\omega_{\dot\Sigma b}:=\sum_a\tilde\omega_{\dot{a}b}$ and $D=4-2\varepsilon^{~}_{UV}$ is a space-time dimension.
\end{subequations}
After summing up one tensor component, $\tilde\omega_{\dot\Sigma b}$ is a Lorentz vector.
The spin-connection form is the one-form object in $\TsM$ and the rank-two tensor in $\TM$.
Using the component representation of the coordinate-free object $\www$ such that:
\begin{align*}
 \www:=\frac{1}{2}\www^{ab}\(\partial^{~}_a\times\partial^{~}_b\)=
\frac{1}{2}\omega_c^{\hspace{.4em}ab}\eee^c\(\partial^{~}_a\times\partial^{~}_b\),
\end{align*}
we can write the definition of $\tilde\omega_{\dot\Sigma b}$ in the configuration space as $\omega^{\hspace{.5em}b}_{\dot\Sigma}:=\omega_{a}^{\hspace{.4em}ab}$,
where ``$\times$'' is the anti-symmetric product of two vectors in $\TM$.
Therefore, $\omega_{a}^{\hspace{.4em}ab}$ is the Lorentz vector in $\TM$, and  $\eta^\bcdots\gamma_\bcdot\tilde\omega_{\dot\Sigma\bcdot}$ is the Lorentz invariant.

The gravitational Lagrangian density in (\ref{LGrfreeint}) provides the interaction vertex among vierbein and spin-connection fields as
\begin{align}
\widetilde{V}_{\omega\E}&=
-\frac{\cGR^{\hspace{-.3em}2}}{2}
\eta_\bcdots^{~}
\(\delta^{a}_{c}\hspace{.1em}\delta^{b}_{d}-\delta^{b}_{c}\hspace{.1em}\delta^{a}_{d}\)
\omega_{a}^{\hspace{.4em}c\bcdot}\hspace{.1em}
\omega_{b}^{\hspace{.4em}d\bcdot}\hspace{.1em},\label{Lintddww}
\end{align}
where the  factor ${\lp}^{\hspace{-.2em}-2}$ is omitted since an overall factor of the Lagrangian does not contribute to the Feynman rules.
This vertex term is defined in the inertial space.
Vierbein functions are factored out and give determinant det$[\E]$, which is absorbed in a part of the volume form.
After the Fouries transformation, the second term of (\ref{Lintddww}) is
\begin{align*}
\eta_\bcdots^{~}\delta^{b}_{c}\hspace{.1em}\delta^{a}_{d}\hspace{.1em}
\tilde\omega_{\hspace{.2em}a}^{\dagger\hspace{.4em}c\bcdot}\hspace{.1em}
\tilde\omega_{b}^{\hspace{.4em}d\bcdot}\hspace{.1em}&=
\sum_{\dot{a}}\tilde\omega^\dagger_{\hspace{.2em}\dot{a}\bcdot}\hspace{.1em}
\tilde\omega^{~}_{\dot{a}\bcdot}\eta^\bcdots_{~},
\end{align*}
which provides a summation of the normalization constant of the connection field; thus, it does not contribute to scattering amplitudes.

We apply a further modification to the Lagrangian using the torsion-less condition.
Given the vierbein field, we can algebraically solve  the torsion-less equation (\ref{torsionlesscp}) easily using our choice of independent spin-connection components (\ref{wconst}).
\begin{align*}
\text{(\ref{torsionlesscp})}&\implies
\cG\(\omega_{b\hspace{.3em}c}^{\hspace{.3em}a}-\omega_{c\hspace{.3em}b}^{\hspace{.3em}a}\)
=-\E^a_\mu
\(\partial^{~}_c\E_b^\mu-\partial^{~}_b\E_c^\mu\),\\
&\implies
\sum_{a=c}\cG\(\omega_{b\hspace{.3em}c}^{\hspace{.3em}a}-\omega_{c\hspace{.3em}b}^{\hspace{.3em}a}\)
=-\sum_{a=c}\E^a_\mu
\(\partial^{~}_c\E_b^\mu-\partial^{~}_b\E_c^\mu\),\\
&\implies
\cG\omega^{\hspace{.6em}\bcdot}_{\dot\Sigma}\eta_{\bcdot a}=\E^\bcdot_\mu\partial^{~}_a\E_\bcdot^\mu.
\end{align*}
Thus, we obtain after the Fourier transformation that
\begin{align*}
\text{(\ref{Lintddww})}&\implies
\frac{\cGR}{2}\hspace{.1em}
\tilde\omega^{\hspace{.6em}\bcdot}_{\dot\Sigma}
\({\tilde\E^\star_\mu}\hspace{.1em}\partial^{~}_\bcdot{\tilde\E_\star^\mu}\).
\end{align*}
Therefore, we obtain two vertex terms using polarization vectors (\ref{wlambda21})$\sim$(\ref{wlambda23}) in the momentum space, such that:
\begin{align*}
\widetilde{V}_{\omega\E}&=\cG\hspace{.2em}
\tilde\omega^{\hspace{.6em}\bcdot}_{\dot\Sigma}
\({\E^\star_\mu}\hspace{.1em}\partial^{~}_\bcdot{\E_\star^\mu}\)
=-\cG\hspace{.2em}
\tilde\omega^{\hspace{.6em}\bcdot}_{\dot\Sigma}
\({\E_\star^\mu}\hspace{.1em}\partial^{~}_\bcdot{\E^\star_\mu}\),
\end{align*}
for (spin--connection)--(vierbein) vertex owing to the interaction Lagrangian (\ref{LGrfreeint}).

The Feynman rule for (spin-connection)-(ghost) vertex is obtained from (\ref {Lgrgh}) as
\begin{align*}
\tilde{V}_{\bar{\chi}\omega\chi}&=\cG\hspace{.1em}
\(\partial_\bcdot^{~}\tilde{\bar\chi}\)
\tilde\omega^{\hspace{.5em}\bcdot}_{\dot\Sigma}\hspace{.1em}\tilde{\chi}
\end{align*}
for the ghost vertex owing to the Lagrangian (\ref{Lgrgh}).
%In this interaction Lagrangian, all four vierbein fields are absorbed in the Romanisation of ghost/anti-ghost and spin-connection fields and differential operator; thus, no vierbein field has a dynamic degree to carry momentum.
We are performing perturbative expansion of the gravitational Lagrangian, treating the interaction Lagrangian as the small perturbation concerning the kinetic term (free Lagrangian).
Therefore, an overall factor commonly appearing on the gravitational Lagrangian does not contribute to the perturbative expansion.
Consequently, the vertex rule has no factor $(\kE\hbar)^{-1}$. 
 %%%%%%%%%%%%%%%%%%%%%%%%%%%%%%%%%%%%%%%%%%%
% Figure environment removed
%%%%%%%%%%%%%%%%%%%%%%%%%%%%%%%%%%%%%%%%%%%

The electromagnetic interaction vertex is the same as the QED; thus, we obtain the QGED interaction vertices shown in Figure \ref{fig2} as follows: 
\begin{subequations}
\begin{align}
\bullet~&\textrm{(photon)--(electron)$^2$}& \textrm{(Figure \ref{fig2}-(a)):} & &e\hspace{.1em}\gamma_{~}^a&\label{v1}\\
\bullet~&\textrm{(spin-connection)--(electron)$^2$}& \textrm{(Figure \ref{fig2}-(b)):}& &\(1-\epsilon^{~}_{UV}\)\cG\gamma^{~}_a&\label{v2}\\
\bullet~&\textrm{(spin-connection)--(vierbein)$^2$}& \textrm{(Figure \ref{fig2}-(c)):}&
&\cG\hspace{.1em}p^{\E}_{b}%\sum_{\dot{a}}
&\label{v3}\\
\bullet~&\textrm{(spin-connection)--(ghost)$^2$}& \textrm{(Figure \ref{fig2}-(d)):}& 
&\cG\hspace{.1em}p^{\chi}_{b}%\sum_{\dot{a}}
&\label{v4}
\end{align}
where $p^{\E}_{\bullet}$ and $p^{\chi}_{\bullet}$ are incoming momenta of the vierbein and ghost fields, respectively.
We set the space-time dimension as $D=4-2\epsilon^{~}_{UV}$.
\end{subequations}
%In the Feynman rules, $\sum_{\dot{a}}$ must be understood as
%\begin{align*}
%\sum_{\dot{a}}\text{(an external field) or (a numerator of the propagator)}
%\rightarrow\sum_{\dot{a}}
%\tilde\omega^{~}_{\dot{a}b}=\tilde\omega^{~}_{\dot{\Sigma}b}=\tilde\omega_{a\hspace{.3em}b}^{\hspace{.3em}a}.
%\end{align*}

%
% Propagators
%
\subsection{Propagators}
A quantum propagator is the vacuum-to-vacuum transition amplitude of the time-ordered product of quantum fields consisting of a sum of creation and annihilation operators, such that:
\begin{align}
D_{\hspace{-.1em}F}(\xi-\eta)&:=i\langle0|T\phi(\xi)\phi(\eta)|0\rangle,\label{Dfxy}
\end{align}
where $\phi(\xi)$ shows a general quantum field at $\xi\in\TsM$.
Additional degrees of freedom, e.g., a Lorentz index of vector bosons or a spinor index of electrons, are omitted here.  
A suffix ``$F$'' of $D_{\hspace{-.1em}F}$ stands for the \emph{Feynman propagator} and is omitted hereafter.
The Feynman propagator naturally appears in some extended Riemannian metric (the $\theta$-metric\cite{Kurihara:2022green}).
The classical counterpart of a quantum propagator is Green's function for the equation of motion for the free gauge field, which is a wave equation; thus, their Fock space of creation and annihilation operators is the same as that of the QED gauge field. 

The Fourier transformation of (\ref{Dfxy}) provides a quantum propagator in the momentum space such that:
\begin{align}
D^\phi(p)&=\frac{i}{(2\pi\hbar)^{2}}\int_{\TsM}\vvv\hspace{.2em}e^{-ip\cdot\xi/\hbar}
\langle0|T^*\phi(\xi)\phi(0)|0\rangle=
\frac{\sum_{\lambda}\tilde{\phi}^{(\lambda)}(p)\tilde{\phi}^{(\lambda)}(p)}{-\eta_\bcdots p^\bcdot p^\bcdot-i\delta},
\label{Dphi}
\end{align}
 where $p$ is a momentum of the field and $\tilde{\phi}^{(\lambda)}(p)$ is a wave function  in the momentum space with polarization $\lambda$.
 A contraction of vectors in $p\in\TstM$ and $\xi\in\TsM$ is $p\cdot\xi:=\eta^\bcdots p^{~}_\bcdot\xi_\bcdot$.

%A propagator of the ghost field is trivial.
We obtain propagators for QGED fields owing to  (\ref{Dphi}) as follows:
\begin{itemize}
\begin{subequations}
\item Photon propagator:
\begin{align}
D^{\Aa}_{\hspace{.2em}ab}(p)&=
\frac{\sum_{\lambda}\epsilon_a^{(\lambda)}(p)\epsilon_b^{(\lambda)}(p)}
{-\eta_\bcdots p^\bcdot p^\bcdot-i\delta}
\end{align}
\item Electron propagator: 
\begin{align}
S_{\psi}(p)=
\frac{\sum_{s}u(p,s)\bar{u}(p,s)}
{-\eta_\bcdots p^\bcdot p^\bcdot+m^2-i\delta}
\end{align}
\item Vierbein propagator:
\begin{align}
D^\E_{\hspace{.1em}\mu\nu}(p)=
\frac{\sum_{\lambda}\tilde{\E}_\mu^{(\lambda)}(p)\tilde{\E}_\nu^{(\lambda)}(p)}
{-\eta_\bcdots p^\bcdot p^\bcdot-i\delta}\label{DE}
\end{align}
\item Spin-connection propagator:
\begin{align}
D_{\hspace{.2em}\dot{a}b;\dot{c}d}^{\omega}(p)=
\frac{\sum_{\lambda}
\tilde{\omega}^{(\lambda)}_{\hspace{1em}\dot{a}b}\hspace{.2em}
\tilde{\omega}^{(\lambda)}_{\hspace{1em}\dot{c}d}}
{-\eta_\bcdots p^\bcdot p^\bcdot-i\delta}
\end{align}
\item Ghost propagator:
\begin{align}
D_{\chi}(p)=
\frac{\tilde\chi(p)^2}
{-\eta_\bcdots p^\bcdot p^\bcdot-i\delta}
\end{align}
\end{subequations}
\end{itemize}
Polarization summations for the QGED fields are given in  (\ref{fnorm}), (\ref{Anorm}), (\ref{EEnorm}), (\ref{Wnorm}) and (\ref{chinorm}).
The vierbein propagator (\ref{DE}) is defined in the momentum space corresponding to the (flat) asymptotic space-time.

%%%%%%%%%%%%%%%%%%%%%%%%%%%%%%%%%%%%%%%%%%%
% Figure environment removed
%%%%%%%%%%%%%%%%%%%%%%%%%%%%%%%%%%%%%%%%%%%

%
% Integration measure and S-matrix
%
\subsection{Integration measure and S-matrix}
An integration measure for each loop momentum $l_j$ and outgoing external momentum $k_j$ are, respectively, taken to be\begin{align}
\prod_j\(-1\)^{\sigma_j}\(\mu_R^2\)^{\epsilon^{~}_{UV}}\frac{d^Dl_j}{i(2\pi)^D } ~~&{\rm and}~~
\prod_j\frac{d^3k_j}{(2\pi)^3}\frac{1}{2k^0_j},
\end{align}
where $k^0_j$ is the zeroth component of the $j$'th outgoing momentum $k_j$.
In the dimensional regularisation method, we need to put an arbitrary parameter with energy dimension, $\mu_R^{~}$, to adjust the physical dimension of the loop contribution.
We set $\sigma_j=1$, if a $j$'th loop particle is an electron or a ghost; otherwise, $\sigma_j=0$. 
An imaginary unit in the loop integration is eliminated owing to the analytic continuation (the Wick rotation) concerning the zeroth component $dl_j^0$.

The $S$-matrix is defined owing to the interaction Lagrangian as
\begin{align*}
S:=T^* \exp{\left[i\int d^4\xi\LL_{int}\(\xi\)\right]},
\end{align*}
and its perturbative expansion is
\begin{align*}
S=1+\sum_{N=1}^{\infty}\frac{i^N}{N!}\int d^4\xi_1\cdots\int d^4\xi_N
T^*\left[\LL_{int}\(\xi_1\)\cdots\LL_{int}\(\xi_N\)\right].
\end{align*}
The scattering matrix, namely $T$-matrix, is defined owing to the $S$-matrix as
\begin{align}
S=1+i T.\label{TMatrix}
\end{align}
Matrix elements $S_{fi}:=\langle f|S|i\rangle$ is expressed as
\begin{align*}
S_{fi}=\delta_{fi}+i(2\pi)^4\delta(p^{~}_f-p^{~}_i)\sum_{pol}\big| T_{fi}\big|^2,
\end{align*}
 where $|i\rangle$ and $|f\rangle$ are, respectively,  initial and final states, and $p^{~}_i$ and $p^{~}_f$ are their total momentum.
Our convention of Feynman rules, including the gravitational bosons, gives a factor of the scattering matrix in total, such as \begin{align*}
i_{~}^N&=i\times i_{~}^{N+L-1}\times i_{~}^{-L},
\end{align*}
where the second factor is absorbed into the propagators and the last factor into loop integrals; thus, the first $i$ gives a correct factor of $iT$.

%%%%%%%%%%%%%%%%%%%%%
% Renormalisation  of the QGED  %
%%%%%%%%%%%%%%%%%%%%%
\section{Renormalisation of the QGED}
%After section \ref{greensfunctionWE}, we omits a suffix ``$\bare$'' for bare objects appearing in Feynman rules.
%We put it back to expressions.
This section demonstrates a renomalisation of the QGED at a one-loop order owing to the Feynman rues provided in previous section.
In following calculations, We exploit the Feynman gauge for both photon and spin-connection fields, such that $\xi^{~}_{\hspace{-.1em}A}=\xi_{\hspace{-.1em}\omega}=1$.

%
% Power counting
%
\subsection{Power counting}
Before starting the one-loop renormalisation of the QGED, we apply the power counting on the QGED to confirm its divergence degree.
For details of the power counting, see, e.g. Ref.\cite{kaku1993quantum}.
For each vertex in the interaction Lagrangian, we can judge an ultraviolet divergence degree using
\begin{align*}
\rho=D-\frac{D-1}{2}N_f-\frac{D-2}{2}N_b-N_\delta,
\end{align*}
where $N_f$, $N_b$ and $N_\delta$ are the numbers of electrons, bosons and derivative couplings in the vertex, respectively. 
When the degree exceeds zero, it induces an ultraviolet divergence, and the theory is un-renormalisable.
On the other hand, when the degree is equal to zero, it induces a logarithmic divergence, and the theory is possibly renormalisable.
The degrees of vertices in Figure \ref{fig2} in four-dimensional space-time are 
\begin{align*}
\rho_a&=\rho_b=4-\frac{3}{2}\times2-1-0=0,~~~
\rho_c=\rho_d=4-\frac{3}{2}\times0-3-1=0;
\end{align*}
thus, the QGED may be renormalisable.

We can assign the superficial degree of divergence $d$ for any Feynman diagrams.
When some Feynman diagrams have $d\ge0$, these diagrams give ultraviolet-divergent integrals.
%It is known that when $D>0$ ($D<0$), the theory is un-renormalisable (super-renormalisable), respectively.
%When $D=0$, the theory is possibly renormalisable, such as the Yang--Mills theory in four-dimensional space-time.
We denote the number of loops, external/internal fields and vertices as shown in Table \ref{table33}.
%%%%%%%%%%%%%%%%%%%%%%%%%%%%%%%%%%%%%%%%%%%%%%%%%%%%%%%
\begin{table}[bt]
\begin{center}
\caption{\label{table33}\small
The number of ingredients of the Feynman diagrams.}
\vskip 2mm
\begin{tabular}{lcc}
\multicolumn{2}{c}{Type}& the number in diagrams\\
\hline
\multirow{4}{*}{External lines} & electron& $E_\psi$\\
 &photon& $E_\gamma$\\
 &vierbein& $E_\E$\\
 &spin-connection& $E_\omega$\\
\hline
\multirow{4}{*}{Internal lines}& electron& $I_\psi$\\
 &photon& $I_\gamma$\\
 &vierbein& $I_\E$\\
 &spin-connection& $I_\omega$\\
 &ghost/anti-ghost& $I_\chi$\\
\hline
 \multirow{3}{*}{Vertex} &(electron)$^2$-(photon)& $V_\gamma$\\
 &(electron)$^2$-(spin-connection)& $V_\omega$\\
 &(vierbein)$^2$-(spin-connection)$^{*)}$& $V_\E$\\
 &(ghost)-(anti-ghost)-(spin-connection)$^{*)}$& $V_\chi$\\
\hline
 \multicolumn{2}{l}{Loops}& $L$
\end{tabular}
\end{center}
\begin{center}$*)$ derivative coupling\end{center}
\end{table}
%%%%%%%%%%%%%%%%%%%%%%%%%%%%%%%%%%%%%%%%%%%%%%%%%%%%%%%
In the four-dimensional space-time, the superficial degree of divergence is provided as
\begin{align}
d=DL  -2I_\gamma  -2I_\omega  -2I_\chi  -2I_\E  -I_\psi+V_\E+V_\chi.
\end{align}
We can express the superficial degree of divergence only by the number of external particles using identities such as
\begin{align*}
V_\omega+V_\psi&=I_\psi+\frac{1}{2}E_\psi,\\
V_\psi&=2I_\gamma+E_\gamma,\\
V_\E&=I_\E+\frac{1}{2}E_\E,\\
V_\chi&=I_\chi,\\
E_\omega+2I_\omega&=V_\E+V_\omega+V_\chi,\\
L-1&=I_\gamma+I_\psi+I_\E+I_\omega+I_\chi-V_\E-V_\chi-V_\omega-V_\psi.
\end{align*}
Consequently, we obtain the superficial degree of divergence in four-dimensional space-time, such that:
\begin{align*}
d=4-E_\gamma-E_\E-E_\omega-\frac{3}{2}E_\psi.
\end{align*}
Therefore, ultraviolet-divergent diagrams have external particles as 
\begin{align*}
E_\gamma+E_\E+E_\omega+\frac{3}{2}E_\psi<4;
\end{align*}
thus, the finite number of diagrams gives an ultraviolet divergent, which can be eliminated by the finite number of counter terms.

%
% One-loop renormalisation
%
\subsection{One-loop renormalisation}
This section provides renormalisation constants and counter terms of the QGED at a one-loop level.
Although the QED's renormalisation procedure is established and known well, we repeat it as a guide for the renormalisation of a gravitational part.
We exploit the on-shell renormalisation conditions such that
\begin{enumerate}
\item the pole position of propagators should locate at physical mass,
\item residues of propagators at the pole should be unity,
\end{enumerate}
and the charge renormalisation utilised in Refs.\cite{10.1143/PTPS.73.1, 10.1143/PTPS.100.1,BELANGER2006117}.
This report refers to charge renormalisation as the coupling constant renormalisation.
Owing to the second condition of the on-shell condition, we can omit self-energy correction diagrams on the external line.

Throughout this study, we exploit dimensional regularisation to regulate the ultraviolet divergence and put fictitious masses into massless bosons to treat infrared divergence.

% Renormlisation constants
\subsubsection{Renormlisation constants}
For a QED part, renormalisation constants $Y^{~}_{\hspace{-.2em}\Aa}$, $Z^{~}_\psi$, $Z^{~}_{\hspace{-.1em}\Aa3}$ and $\delta{m^{~}_{e\hspace{-.1em}\Aa}}$ are introduced to the Lagrangian.
In the QGED, additional renormalisation constants,  $Y^{~}_{\hspace{-.2em}\omega}$, $Z^{~}_{\hspace{-.1em}\omega3}$, $Z^{~}_{\hspace{-.1em}\Etr}$ and $\delta{m^{~}_{e\omega}}$ appears.
Bare objects in the Lagrangian are replaced as 
\begin{align*}
\psi^\bare&=
{Z^{~}_{\psi\Aa}}^{\hspace{-.2em}\hlf}\hspace{.1em}
{Z^{~}_{\psi\omega}}^{\hspace{-.2em}\hlf}\hspace{.1em}\psi,\\
\Aa^\bare_{\hspace{.7em}a}\hspace{.1em}&=
{Z^{~}_{\hspace{-.1em}\Aa}}^{\hlf}\hspace{.2em}\Aa^{~}_a,~~~
\omega^{{\bare}ab}_{\hspace{.7em}c}\hspace{.1em}=\hspace{.2em}
{Z^{~}_{\hspace{-.1em}\omega}}^{\hlf}\hspace{.2em}\omega^{\hspace{.3em}ab}_{c},~~~
\E^{{\bare}a}_{\hspace{.8em}\mu}\hspace{.1em}=\hspace{.2em}
{Z^{~}_{\hspace{-.1em}\E}}^{\hlf}\hspace{.2em}\E^a_\mu\\
e^\bare&=Y^{~}_{\hspace{-.2em}\Aa}\hspace{.2em}e,~~~\cGz=Y^{~}_{\hspace{-.2em}\omega}\cG,\\
m_e^\bare&=m_e+\(\delta{m^{\Aa}_{e}}+\delta{m^{\omega}_{e}}\).
\end{align*}
In addition to that, there are several mixed corrections at $\mathcal{O}(e\hspace{.1em}\cGR)$.
Then, we add the counter-term Lagrangian to the renormalised one. 
%
%Renormalisation constants and the counter-term Lagrangian are summarised in {\bf Appendix \ref{app2}}.

%Photon and spin-connection renormalisation
\subsubsection{Photon and spin-connection renormalisation}\label{PSR}
%%%%%%%%%%%%%%%%%%%%%%%%%%%%%%%%%%%%%%%%%%%
% Figure environment removed
%%%%%%%%%%%%%%%%%%%%%%%%%%%%%%%%%%%%%%%%%%%t
First, we provide a gauge boson renormalisation in the QGED.
Vacuum polarisation diagrams in the QGED are depicted in Figures \ref{figvacpol} and \ref{figvacpol2}; the former shows the electron-loop diagrams, and the latter shows boson-loop diagrams.
We calculate vacuum polarization at $\mathcal{O}(e^2)$, $\mathcal{O}(\cGR^{\hspace{-.3em}2})$ and $\mathcal{O}(e\hspace{.1em}\cGR)$.

Owing to the Lorentz invariance of the photon wave function, the vacuum polarization diagram of a photon shown in Figure \ref{figvacpol}-(a) can be expressed as
\begin{subequations}
\begin{align}
\text{Figure \ref{figvacpol}-(a)}&=:\Pi^{(\gamma{e}\gamma)}_{ab}(q^2)=\(\eta^{~}_{ab}-\frac{q_aq_b}{q^2}\)a^{~}_{\gamma{e}\gamma}(q^2)
+\frac{q_aq_b}{q^2}b^{~}_{\gamma{e}\gamma}(q^2)\label{PIgeg}
\end{align}
in general, where $q_a$ is a photon momentum.
After standard calculations, we obtain the results as
\begin{align}
a^{~}_{\gamma{e}\gamma}(q^2)&=-\frac{\alpha}{4\pi}\frac{4}{3}q^2
\(C_{UV}-\log{\frac{m_e^2}{\mu^2_R}}\),\label{PIgeg2}\\
b^{~}_{\gamma{e}\gamma}(q^2)&=0,\label{PIgeg3}
\end{align}
\end{subequations}
where $C_{UV}:=1/\varepsilon^{~}_{UV}-\gamma^{~}_E+\log{4\pi}$ and $\alpha:={{e}^2}/{4\pi}$.
The vacuum polarisation fulfils a current conservation such as
\begin{align*}
\eta^\bcdots q_\bcdot\Pi^{(\gamma{e}\gamma)}_{a\bcdot}(q^2)&=0.
\end{align*}
We obtain the renormalisation constant for a photon wave function as
\begin{align}
{Z^{~}_{\hspace{-.1em}\Aa}}^{\hlf}&=
1-\frac{1}{2}\left.\frac{\partial}{\partial q^2}a^{~}_{\gamma{e}\gamma}(q^2)\right|_{q^2=0}=
1-\frac{1}{2}\frac{\alpha}{4\pi}
\frac{4}{3}\(C_{UV}-\log{\frac{m_e^2}{\mu^2_R}}\).\label{ZAA}
\end{align}
 
Vertex Feynman rules for a (photon)-(electron) and a (spin-connection)-(electron) are the same as each other except the coupling constant, as shown in (\ref{v1}) and (\ref{v2}); thus, vacuum polarisations, including the spin-connection fields, are easily provided from that for a photon as follows:
\begin{align*}
\text{Figure \ref{figvacpol}-(b)}&=:\Pi^{(\gamma{e}\omega)}_{ab}(q^2)=\(\eta^{~}_{ab}-\frac{q_aq_b}{q^2}\)a^{~}_{\gamma{e}\omega}(q^2)
+\frac{q_aq_b}{q^2}b^{~}_{\gamma{e}\omega}(q^2),\\
\text{Figure \ref{figvacpol}-(c)}&=:\Pi^{(\omega{e}\omega)}_{ab}(q^2)=\(\eta^{~}_{ab}-\frac{q_aq_b}{q^2}\)a^{~}_{\omega{e}\omega}(q^2)
+\frac{q_aq_b}{q^2}b^{~}_{\omega{e}\omega}(q^2),
%
\intertext{where}
%
a^{~}_{\gamma{e}\omega}(q^2)&=
-\frac{\(\alpha\aGR\)^{1/2}}{4\pi}
\frac{4}{3}q^2\(C_{UV}-1-\log{\frac{m_e^2}{\mu^2_R}}\),\\
a^{~}_{\omega{e}\omega}(q^2)&=
-\frac{\aGR}{4\pi}
\frac{4}{3}q^2\(C_{UV}-2-\log{\frac{m_e^2}{\mu^2_R}}\),\\
b^{~}_{\gamma{e}\omega}(q^2)&=b^{~}_{\omega{e}\omega}(q^2)=0,
\end{align*}
where $\aGR:={{\cGR}^{\hspace{-.2em}2}}/{4\pi}$.
Again, they fulfil the current conservation by themselves. 
%%%%%%%%%%%%%%%%%%%%%%%%%%%%%%%%%%%%%%%%%%%
% Figure environment removed
%%%%%%%%%%%%%%%%%%%%%%%%%%%%%%%%%%%%%%%%%%%t

The spin-connection has two other vacuum polarisation diagrams; a vierbein loop shown in Figure \ref{figvacpol2}-(a) and a ghost loop shown in Figure \ref{figvacpol2}-(b). 
The calculations of both diagrams are very similar owing to their derivative couplings.
A difference comes from scalar or vector propagators in loops.
The Feynman rules provide the results as
\begin{align*}
\text{Figure \ref{figvacpol2}-(a)}=:\Pi^{(\omega\E\omega)}_{ab}(q^2)&=
\frac{1}{2}\cGR^{\hspace{-.2em}2}\(\mu_R^2\)^{\epsilon^{~}_{UV}}\int\frac{d^Dk}{i(2\pi)^D }
\frac{\eta^\bcdots k_\bcdot(k_\bcdot+q_\bcdot)}{k^2(k-q)^2}\eta^{~}_{ab},\\
&=\(\eta^{~}_{ab}-\frac{q_aq_b}{q^2}\)a^{~}_{\omega\E\omega}(q^2)
+\frac{q_aq_b}{q^2}b^{~}_{\omega\E\omega}(q^2),\\
%
\text{Figure \ref{figvacpol2}-(b)}=:\Pi^{(\omega\chi\omega)}_{ab}(q^2)
&=-\cGR^{\hspace{-.2em}2}\(\mu_R^2\)^{\epsilon^{~}_{UV}}\int\frac{d^Dk}{i(2\pi)^D }
\frac{k_a(k_b+q_b)}{k^2(k-q)^2},\\
&=\(\eta^{~}_{ab}-\frac{q_aq_b}{q^2}\)a^{~}_{\omega\chi\omega}(q^2)
+\frac{q_aq_b}{q^2}b^{~}_{\omega\chi\omega}(q^2).
\end{align*}
A factor of one-half in the vierbein loop is due to the symmetric factor of two identical particles in the loop.\begin{align*}
&a^{~}_{\omega\E\omega}(q^2)=0,~~~
a^{~}_{\omega\chi\omega}(q^2)=-\frac{\aGR}{4\pi}q^2\hspace{.2em}
\frac{1}{6}\(C_{UV}+\frac{5}{3}-\log{\frac{-q^2}{\mu^2_R}}\),\\
%
&b^{~}_{\omega\E\omega}(q^2)=-b^{~}_{\omega\chi\omega}(q^2)=-\frac{\aGR}{4\pi}q^2\hspace{.2em}
\frac{1}{4}\(C_{UV}+2-\log{\frac{-q^2}{\mu^2_R}}\)
\implies b^{~}_{\omega\E\omega}(q^2)+b^{~}_{\omega\chi\omega}(q^2)=0.
\end{align*}
Thus, the vacuum polarisation of boson loops fulfils current conservation after summing up two contributions, even though each of them does \emph{NOT}.

At last in the section, we provide a vacuum polarization of a vierbein field depicted in Figure \ref{figvacpol2}-(c).
After calculations following the Feynman rules, we obtain that
\begin{align*}
\text{Figure \ref{figvacpol2}-(c)}=:\Pi^{(\E\omega\E)}_{ab}(q^2)&=
\cGR^{\hspace{-.2em}2}\(\mu_R^2\)^{\epsilon^{~}_{UV}}\int\frac{d^Dk}{i(2\pi)^D }
\frac{\eta_{~}^{\bcdots}q_\bcdot(k_\bcdot+q_\bcdot)\eta^{~}_{ab}-q_a(q_b+k_b)}{k^2(k-q)^2},\\
&=\(\eta^{~}_{ab}-\frac{q_aq_b}{q^2}\)a^{~}_{\E\omega\E}(q^2)
+\frac{q_aq_b}{q^2}b^{~}_{\E\omega\E}(q^2),
\end{align*}
where
\begin{align*}
a^{~}_{\E\omega\E}(q^2)&=-\frac{\aGR}{4\pi}q^2
\(\frac{1}{2}C_{UV}+1-\frac{1}{2}\log{\frac{-q^2}{\mu^2_R}}\),\\
b^{~}_{\E\omega\E}(q^2)&=0,
\end{align*}
which fulfils the current conservation by itself. 

We introduce total $\Nf$ fermions, in which $\Ncf$ fermions are electromagnetically charged, in the QGED and summarise the renormalisation constant for bosonic fields:
\begin{subequations}
\begin{align}
{Z^{~}_{\hspace{-.1em}\Aa}}^{\hlf}&=
1-\frac{1}{2}\left.\frac{\partial}{\partial q^2}\sum_{i=1}^\Ncf a^{~}_{\gamma{e}\gamma}(m^{~}_i;q^2)\right|_{q^2=0},\notag \\
&=
1-\frac{1}{2}\frac{\alpha}{4\pi}
\frac{4}{3}\(C_{UV}-\sum_{i=1}^\Ncf\log{\frac{m_i^2}{\mu^2_R}}\),\label{ZAA2}\\
%
{Z^{~}_{\hspace{-.1em}\omega}}^{\hspace{-.2em}\hlf}&=
1-\frac{1}{2}\left.\frac{\partial}{\partial q^2}\(
\sum_{i=1}^\Nf\hspace{.1em}a^{~}_{\omega{e}\omega}\(q^2;m^{~}_i\)+
a^{~}_{\omega\chi\omega}(q^2)+
a^{~}_{\omega\E\omega}(q^2)
\)\right|_{q_{~}^2=-\mu_{I\hspace{-.2em}R}^2},\notag
\\&=
1-\frac{1}{2}\frac{\aGR}{4\pi}\frac{1}{12}
\(\(1-8\Nf\)C_{UV}-\frac{1}{3}+24N_f
+8\sum_{i=1}^\Nf\log{\frac{m_i^2}{\mu^2_R}}-\log{\frac{\mu_{I\hspace{-.2em}R}^2}{\mu^2_R}}\),\label{ZWW}\\
%
{Z^{~}_{\Aa\hspace{-.1em}\omega}}^{\hspace{-.2em}\hlf}&=
1-\frac{1}{2}\left.\frac{\partial}{\partial q^2}\sum_{i=1}^\Ncf a^{~}_{\gamma{e}\omega}(m^{~}_i;q^2)
\right|_{q_{~}^2=0},\notag
\\&=
1-\frac{1}{2}\frac{\(\alpha\aGR\)^{1/2}}{4\pi}
\frac{4}{3}\(\Ncf\(C_{UV}-1\)-\sum_{i=1}^\Ncf\log{\frac{m_i^2}{\mu^2_R}}\),\label{ZAW}\\
%
{Z^{~}_{\hspace{-.1em}\E}}^{\hspace{-.2em}\hlf}&=1-\frac{1}{2}\left.\frac{\partial}{\partial q^2}
a^{~}_{\E\omega\E}(q^2)\right|_{q_{~}^2=-\mu_{I\hspace{-.2em}R}^2},\notag\\
&=1-\frac{1}{2}\frac{\aGR}{4\pi}
\(\frac{1}{2}C_{UV}-\frac{1}{2}\log{\frac{\mu_{I\hspace{-.2em}R}^2}{\mu^2_R}}\),\label{ZEE}
\end{align}
where $m^{~}_i$ is a mass of $i$'th fermion appearing in the QGED and $\mu^{~}_{\hspace{-.1em}I\hspace{-.2em}R}$ is an artificial energy scale to regularise an infrared divergence, which is eliminated by adding real spin-connection radiation diagrams.
We align fermions such that the first $\Ncf$ charged fermions are followed by $(\Nf-\Ncf)$ neutral ones. 
\end{subequations}

% Electron field and mass renormalisation
\subsubsection{Electron field and mass renormalisation}
%%%%%%%%%%%%%%%%%%%%%%%%%%%%%%%%%%%%%%%%%%%
% Figure environment removed
%%%%%%%%%%%%%%%%%%%%%%%%%%%%%%%%%%%%%%%%%%%
Electron self-energy diagrams in Figure \ref{fig4} provide electron field and mass renormalisation constants. 
A QED part obtained from Figure \ref{fig4}-(a) with electron momentum $p_a$ is
\begin{align*}
&\text{Figure \ref{fig4}-(a)}=:\Sigma^{~}_{\Aa}(\slash{p})=\Sigma^{1}_\Aa(p^2)\bm{1}+
\Sigma^{\gamma}_\Aa(p^2)\slash{p},
\intertext{with}
\Sigma^{1}_\Aa(p^2)&=-\frac{\alpha}{4\pi}\hspace{.2em}4m^{~}_e
\(C_{UV}-\frac{1}{2}+-F_0(m_e^2,\lambda^2,p^2)\)\rightarrow
\Sigma^{1}_\Aa(m_e^2)=
-\frac{\alpha}{4\pi}\hspace{.2em}4m^{~}_e\(C_{UV}+\frac{3}{2}-\log{\frac{m_e^2}{\mu_R^2}}\),\\
%
\Sigma^{\gamma}_\Aa(p^2)&=\hspace{.7em}\frac{\alpha}{4\pi}
\(C_{UV}-1-2F_1(m_e^2,\lambda^2,p^2)\)\hspace{3.0em}\rightarrow
\Sigma^{\gamma}_\Aa(m_e^2)=\hspace{.8em}
\frac{\alpha}{4\pi}\(C_{UV}+2-\log{\frac{m_e^2}{\mu_R^2}}\),
\end{align*}
where $\slash{\hspace{-.13em}p}:=\eta^\bcdots\gamma_\bcdot p_\bcdot$ and \bm{1} is a unit spinor.
We put a fictitious photon mass $\lambda$ to regularise an infrared divergence.
The above functions have no infrared divergence at $p^2_{~}=m^2_e$; thus, the fictitious photon mass is set to zero in the rightmost expressions.
The Feynman parameter integration of the two-point function $F_n$ is defined as
\begin{align*}
F_n(\mu_1,\mu_2,\mu_3):=\int dx\hspace{.2em}x^n 
\log{\(\mu_1\hspace{.1em}(1-x)+\mu_2\hspace{.1em}x-\mu_3\hspace{.1em}x(1-x)\)}.
\end{align*}
Owing to these results, we can obtain a derivative of these functions as
\begin{align*}
\partial\Sigma^{1}_\Aa(m_e^2)&:=\left.\frac{\partial}{\partial\hspace{-.3em}\slash{p}}
\Sigma^{1}_\Aa(p^2)\right|_{p^2=m_e^2}=\hspace{.8em}
\frac{\alpha}{4\pi}\frac{1}{m_e^{~}}\(8+4\log{\frac{\lambda^2}{m_e^{2}}}\),\\
%
\partial\Sigma^{\gamma}_\Aa(m_e^2)&:=\left.\frac{\partial}{\partial\hspace{-.3em}\slash{p}}\Sigma^{\gamma}_\Aa(p^2)\right|_{p^2=m_e^2}=
-\frac{\alpha}{4\pi}\frac{1}{m_e^2}\(6+2\log{\frac{\lambda^2}{m_e^{2}}}\).\\
\end{align*}
The fictitious photon mass will be eliminated from scattering amplitudes by adding real-photon radiation diagrams.
The QED renormalisation constants of the electron field and mass are obtained using the above functions, such that
\begin{subequations}
\begin{align}
{Z^{~}_{\psi\Aa}}^{\hlf}&=1-\frac{1}{2}\(
\Sigma^{\gamma}_\Aa\(m_e^2\)+m_e^{~}\(m_e^{~}\hspace{.1em}\partial\Sigma^{\gamma}_\Aa\(m_e^2\)+
\partial\Sigma^{1}_\Aa\(m_e^2\)\)
\),\notag\\
&=1-\frac{1}{2}\frac{\alpha}{4\pi}\(
C_{UV}+4-\log{\frac{m_e^{2}}{\mu_R^2}}+2\log{\frac{\lambda^2}{m_e^{2}}}\label{ZpsiA}
\)
\intertext{and}
\delta{m^{\Aa}_{e}}&=m_e^{~}\hspace{.1em}\Sigma^{\gamma}_\Aa\(m_e^2\)+\Sigma^{1}_\Aa\(m_e^2\),\notag\\
&=-\frac{\alpha}{4\pi}{m_e^{~}}\(3C_{UV}+4-3\log{\frac{m_e^{2}}{\mu_R^2}}\).\label{dmeA}
\end{align}
\end{subequations}

Similarly, we obtain the electron self-energy in the QGED as
\begin{align*}
\text{Figure \ref{fig4}-(b)}&=:\Sigma^{~}_{\omega}(\slash{p}),\\&=
\(1-\epsilon^{~}_{UV}\)^2\cGR^{\hspace{-.2em}2}\(\mu_R^2\)^{\epsilon^{~}_{UV}}\int\frac{d^Dk}{i(2\pi)^D }
\frac{\gamma_\bcdot(\slash{p}+\slash{k})\gamma_\bcdot}{k^2(k-q)^2}
\(\eta^\bcdots\eta^\stars-\eta^{\star\bcdot}\eta^{\star\bcdot}\)\eta_\stars,\\
&=
\Sigma^{1}_\omega(p^2)\bm{1}+
\Sigma^{\gamma}_\omega(p^2)\slash{p}.
\end{align*}
After similar calculations with the QED one, we obtain that
\begin{align*}
\Sigma^{1}_\omega(p^2)&=-\frac{\aGR}{4\pi}\hspace{.2em}12m^{~}_e\(
C_{UV}-\frac{19}{6}-F_0\(m^2_e,\lambda_\omega^2,p^2\)
\)\rightarrow
\Sigma^{1}_\omega(m_e^2)=
-\frac{\alpha}{4\pi}\hspace{.1em}12m^{~}_e\(C_{UV}-\frac{7}{6}-\log{\frac{m_e^2}{\mu_R^2}}\),\\
%
\Sigma^{\gamma}_\omega(p^2)&=\hspace{.7em}\frac{\aGR}{4\pi}\hspace{.2em}3
\(C_{UV}-\frac{11}{3}-2F_1\(m^2_e,\lambda_\omega^2,p^2\)\)
\hspace{1.3em}\rightarrow
\Sigma^{\gamma}_\omega(m_e^2)=\hspace{.7em}
\frac{\aGR}{4\pi}\hspace{.2em}3\(C_{UV}-\frac{2}{3}-\log{\frac{m_e^2}{\mu_R^2}}\),
%
\intertext{and}
%
\partial\Sigma^{1}_\omega(m_e^2)&:=
\left.\frac{\partial}{\partial\hspace{-.3em}\slash{p}}\Sigma^{1}_\omega(p^2)\right|_{p^2=m_e^2}=
\frac{\aGR}{4\pi}\frac{1}{m_e^{~}}\(24+12\log{\frac{\lambda_\omega^2}{m_e^{2}}}\),\\
%
\partial\Sigma^{\gamma}_\omega(m_e^2)&:=
\left.\frac{\partial}{\partial\hspace{-.3em}\slash{p}}\Sigma^{\gamma}_\omega(p^2)\right|_{p^2=m_e^2}=
\frac{\aGR}{4\pi}\frac{1}{m_e^2}\(-18-6\log{\frac{\lambda_\omega^2}{m_e^{2}}}\)
\end{align*}
where $\lambda_\omega$ is the fictitious spin-connection mass.
Consequently, electron field and mass renormalisation constants, respectively, are provided as follows: 
\begin{subequations}
\begin{align}
{Z^{~}_{\psi\omega}}^{\hlf}&=1-\frac{1}{2}\(
\Sigma^{\gamma}_\omega\(m_e^2\)+m_e^{~}\(m_e^{~}\hspace{.1em}\partial\Sigma^{\gamma}_\omega\(m_e^2\)+
\partial\Sigma^{1}_\omega\(m_e^2\)\)
\),\notag\\
&=1-\frac{1}{2}\frac{\aGR}{4\pi}\hspace{.2em}3\(
C_{UV}+\frac{4}{3}-\log{\frac{m_e^{2}}{\mu_R^2}}+2\log{\frac{\lambda_\omega^2}{m_e^{2}}}
\)\label{ZpsiW}
\intertext{and}
\delta{m^{\omega}_{e}}&=m_e^{~}\hspace{.1em}\Sigma^{\gamma}_\omega\(m_e^2\)+\Sigma^{1}_\omega\(m_e^2\),\notag\\
&=-\frac{\aGR}{4\pi}\hspace{.2em}9\(
C_{UV}-\frac{4}{3}-\log{\frac{m_e^{2}}{\mu_R^2}}
\)\label{dmeW}
\end{align}
\end{subequations}

% Vertex correction and coupling constant renormalisation
\subsubsection{Coupling constant renormalisation}
Coupling constant renormalisaion constants are provided using vertex diagrams in Figures \ref{fig7} and \ref{fig8} as
\begin{align*}
\text{Figure \ref{fig7}-(a)}=:&Z^{V}_{\hspace{-.2em}\Aa\hspace{-.2em}\Aa}
\rightarrow Y^{~}_{\hspace{-.2em}\Aa\hspace{-.2em}\Aa}=
Z^{V}_{\hspace{-.2em}\Aa\hspace{-.2em}\Aa}\hspace{.1em}
{Z^{~}_{\psi\Aa}}^{\hspace{-.2em}-1}\hspace{.1em}{Z^{~}_{\hspace{-.1em}\Aa}}^{\hspace{-.2em}-\hlf},\\
%
\text{Figure \ref{fig7}-(b)}=:&Z^{V}_{\hspace{-.2em}\Aa\hspace{-.1em}\omega}
\rightarrow Y^{~}_{\hspace{-.2em}\Aa\hspace{-.1em}\omega}=
Z^{V}_{\hspace{-.2em}\Aa\hspace{-.1em}\omega}\hspace{.1em}
{Z^{~}_{\psi\omega}}^{\hspace{-.2em}-1}\hspace{.1em}
{Z^{~}_{\Aa\hspace{-.1em}\omega}}^{\hspace{-.2em}-\hlf},\\
%
\text{Figure \ref{fig8}-(a)}=:&Z^{V}_{\hspace{-.2em}\omega\Aa}
\rightarrow Y^{~}_{\hspace{-.2em}\omega\Aa}=
Z^{V}_{\hspace{-.2em}\omega\Aa}\hspace{.1em}
{Z^{~}_{\psi\Aa}}^{\hspace{-.2em}-1}\hspace{.1em}
{Z^{~}_{\Aa\hspace{-.1em}\omega}}^{\hspace{-.2em}-\hlf},\\
%
\text{Figure \ref{fig8}-(b)}=:&Z^{V}_{\omega\omega}
\rightarrow Y^{~}_{\omega\omega}=
Z^{V}_{\omega\omega}\hspace{.1em}
{Z^{~}_{\psi\omega}}^{\hspace{-.2em}-1}\hspace{.1em}
{Z^{~}_{\hspace{-.1em}\omega}}^{\hspace{-.2em}-\hlf}.
\end{align*}
To complete the one-loop renormalisation of the QGED, vertex renormalisation constants are needed. 
However, current conservation induces the Ward--Takahashi identity\cite{PhysRev.78.182,1957NCim....6..371T}, such that
\begin{subequations}
\begin{align}
Z^{V}_{\hspace{-.2em}\Aa\hspace{-.2em}\Aa}&=Z_{\psi\Aa}^{~},\label{ZVAA}\\
Z^{V}_{\hspace{-.2em}\Aa\hspace{-.1em}\omega}&=Z_{\psi\omega}^{~},\hspace{1.0em}
Z^{V}_{\hspace{-.2em}\omega\Aa}=\(1-\epsilon^{~}_{UV}\)Z_{\psi\Aa}^{~},\label{ZVAW}\\
Z^{V}_{\omega\omega}&=\(1-\epsilon^{~}_{UV}\)Z_{\psi\omega}^{~},\label{ZVWW}
\end{align} 
in the QGED; thus, we immediately obtain the coupling renormalisation as
\end{subequations}
\begin{subequations}
\begin{align}
Y^{~}_{\hspace{-.2em}\Aa\hspace{-.2em}\Aa}&=
1+\frac{\alpha}{4\pi}\hspace{.2em}
\frac{2}{3}\(C_{UV}-\sum_{i=1}^\Ncf\log{\frac{m_i^2}{\mu^2_R}}\),\label{YAA}\\
Y^{~}_{\hspace{-.2em}\Aa\hspace{-.1em}\omega}&=
1+\frac{\(\alpha\aGR\)^{1/2}}{4\pi}\hspace{.2em}
\frac{8}{3}\(C_{UV}-\frac{1}{4}-\log{\frac{m_e^2}{\mu^2_R}}+3\log{\frac{\lambda^2}{m_e^2}}\),\label{YAW}\\
Y^{~}_{\hspace{-.2em}\omega\Aa}&=1-\frac{\(\alpha\aGR\)^{1/2}}{4\pi}\hspace{.2em}
\frac{4}{3}\(C_{UV}-\frac{5}{4}-\log{\frac{m_e^2}{\mu^2_R}}+3\log{\frac{\lambda^2}{m_e^2}}\),\label{YWA}\\
Y^{~}_{\omega\omega}&=
1+\frac{\aGR}{4\pi}\hspace{.2em}
\frac{1}{12}\(\(8N_f-1\)C_{UV}-24N_f+\frac{1}{3}-8\sum_{i=1}^\Nf\log{\frac{m_i^2}{\mu^2_R}}+
\log{\frac{\mu^2_{I\hspace{-.2em}R}}{\mu^2_R}}\).\label{YWW}
\end{align}
\end{subequations}
%%%%%%%%%%%%%%%
% Figure environment removed
%%%%%%%%%%%%%%%

% Tadpole renormalisation
\paragraph{Tadpole renormalisation:}
At last, we note that the QGED has no fundamental scalar particle; thus, the tadpole diagram does not appear in the physical processes.

% Determination of coupling constants and electron mass
\subsubsection{Determination of coupling constants and electron mass}\label{DCCEM}
We provide all renormalisation constants appearing in one-loop diagrams in the QGED and summarise them in Table \ref{tableRC}.
All divergent loop corrections are encapsulated in three physical parameters in the bare Lagrangian, $e^\bare$, $\cGR^\bare$ and $m^\bare_e$.
When we replace them with measured values, we can numerically evaluate probabilities to observe gravioelectromagnetic phenomena.
This section discusses how to measure such parameters experimentally.

% Electron charge and mass
\paragraph{Electron charge and mass:}
First, we determine the electromagnetic parameters of electric charge and electron mass. 
As discussed in {\bf section \ref{CC}}, a coupling constant can be absorbed in definitions of a connection and a curvature when we look at single gauge interaction.
For the pure Yang--Mills theory given by the Lagrangian (\ref{LSUfreeint}), coupling constant $\cSU$ can be set to unity.
We can fix the coupling constant through the interaction Lagrangian (\ref{LMTfreeint}) as a relative value between the coupling constant and an electron mass.

Formally, an electron charge (equivalently, the fine structure constant) is fixed due to a free electron's forward-scattering of photons, i.e., the Thomson scattering.
A ratio of the fine structure constant and an electron mass appears in the Thomson scattering cross section at $\theta=0$, such that
\begin{align*}
\frac{d\sigma^{~}_{\text{Th}}}{d\cos{\theta}}&=\pi\hbar\frac{\alpha^2}{m^{2}_e}\(1+\cos^2{\theta}\)
\xrightarrow{\theta\rightarrow0} 2\pi\hbar\frac{\alpha^2}{m^{2}_e}.
\end{align*}
In reality, the fine structure constant is provided by measuring the Lamb shift of atomic spectra.
An essential relation to determining the electromagnetic parameters is
\begin{align*}
\alpha^2=\frac{R_\infty}{c}\times\frac{m^{~}_{\hspace{-.2em}A}}{m^{~}_e}\times\frac{h}{m^{~}_{\hspace{-.2em}A}},
\end{align*}
where $R_\infty$ is the Rydberg constant and $m^{~}_{\hspace{-.2em}A}$ is an atomic mass used in the experiment.
The Rydberg constant appears in an atom's electromagnetic spectra and has been measured using the optical frequency of the 2S-12D transitions in hydrogen and deuterium\cite{PhysRevLett.82.4960}.
This measurement does not depend on a measured value of an electron mass.
A mass ratio between electron and ionized atomic masses is provided by a ratio of the cyclotron frequency of electron and ionized atom as
\begin{align*}
\frac{m^{~}_{\hspace{-.2em}A}}{m^{~}_e}&=\frac{2}{g}\frac{q}{e}\frac{\nu^{~}_{\hspace{-.2em}A}}{\nu^{~}_e},
\end{align*}
where $g$ is a spin $g$-factor (the Land\'{e} $g$-factor) of an electron, $q$ is a charge of an ion, and $\nu^{~}_{\hspace{-.2em}A}(\nu^{~}_e)$ is an ion (electron) cyclotron frequency, respectively.
The most precise measured value of ${m^{~}_{\hspace{-.2em}A}}/{m^{~}_e}$ to date is provided in Ref.\cite{Sturm_2014}.
This relative atomic mass is dimensionless observable and independent from the precise value of an electric charge.
%We note that a ratio ${q}/{e}$ is integer-valued.
A rubidium atomic mass has been measured using the combination of Bloch oscillations with a Ramsey-Bord\'{e} interferometer\cite{Cadoret_2008}, which measures the inertial mass of the atom.
Consequently, the electron inertial mass is provided through the measured value of the relative electron mass.

% Gravitational coupling constant
\paragraph{Gravitational coupling constant:}
Next, we consider how we can fix the gravitational coupling constant experimentally.
All masses discussed so far in this section are the inertial mass appearing in the classical equation of motion and the interaction Lagrangian.
On the other hand, when we treat $\widetilde{V}_{\omega\psi}$ defined by (\ref{Vwpsi0}), we may fix the gravitational coupling constant with respect to the gravitational mass of the fermion since the Born approximation of the electron scattering with $\widetilde{V}_{\omega\psi}$  provides the amplitude as
\begin{align*}
i\bm\tau&=
\langle f|i\widetilde{V}_{\omega\psi}|i\rangle\simeq
-i\(2m^{~}_e\)\bm\xi'^\dagger\widetilde{V}_{\omega\psi}\hspace{.1em}\bm\xi,
\end{align*}
where $\bm\xi$ is a two-component spinor.
The normalisation of the external electron field (\ref{fnorm}) provides the mass in the above amplitude, which must be the gravitational mass.
The weak equivalent principle insists that a gravitational mass and an inertial mass are equivalent.
In the macroscopic scale, it is confirmed experimentally that both gravitational and inertial masses of a proton are equivalent (proportional) to each other.
On a microscopic scale, we can rephrase the equivalent principle as \emph{``the pole mass of an electron in the quantum field theory is equivalent to the gravitational mass in general relativity''}.
Experimental results, e.g., the proton pole mass equals to its gravitational mass, support the microscopic equivalent principle.

We cannot measure the gravitational mass and a coupling constant simply using Newton's law of universal gravity since we formulate a perturbative approach in the inertial system, where gravity is eliminated locally.
Consider an experiment measuring the gravitational coupling constant using the Earth's gravity, other than Newton's law of universal gravity.
We propose to utilise the \emph{gravitomagnetic effect}\cite{de2010classical,pfister2015inertia}.
The spin-connection couples to the electron through a tensor coupling; thus, the gravitational field directly interacts with an electron spin.
This effect is measurable using existing\cite{Muong-2:2021ojo} and future\cite{10.1093/ptep/ptz030} experimental apparatus.
In reality, it gives a possible resolution of the muon $\gmt$ anomaly and $\cG$ is consistent with unity\cite{Kurihara:2022g-2}.
We discuss the measurement of $\cG$ in detail in \textbf{Appendix \ref{appB}}.

% Gravitational beta-function 
\subsection{Gravitational running coupling and the Landau pole}
This section discusses an energy scale dependence of an effective gravitational coupling using a gravitational beta function owing to the renormalisation group equation.
In the perturbative quantum field theory, this effect, known as the running coupling constant, provides a stronger coupling at a higher energy scale than lower energy for the QED and vice versa for the QCD.
Consequently, an effective coupling may diverge at some specific energy, namely the \emph{Landau pole}.
The Landau pole is the energy where the perturbative calculations of the quantum field theory brake down, which does not mean a brake down of the theory itself.
E.g., the non-perturbative QCD, such as the lattice QCD, still works below the QCD scale parameter $\Lambda^{~}_{\hspace{-.1em}\text{Q}\hspace{-.1em}\text{C}\hspace{-.1em}\text{D}}\simeq200$MeV.
We apply the running coupling to the QGED to discuss the effective gravitational coupling and the applicability of the perturbative QGED for higher energies. 

The $\beta$-function of the QGED governs the renormalised gravitational coupling $\aGR\hspace{-.15em}\(\mu^{~}_{\hspace{-.15em}R}\)$ with respect to the renormalisatin scale $\mu^{~}_{\hspace{-.15em}R}$ through the renormalisation group equation (see, e.g., Ref.\cite{ellis2003qcd}) such as 
\begin{align*}
\mu^{2}_{\hspace{-.15em}R}\hspace{.1em}
\frac{\partial~}{\partial\mu^{2}_{\hspace{-.15em}R}}\aGR\hspace{-.15em}\(\mu^{2}_{\hspace{-.15em}R}\)
&=\bGR\(\aGR;\mu^{2}_{\hspace{-.15em}R}\).
\end{align*}
The renormalised gravitational coupling $\aGR\hspace{-.15em}\(\mu^{~}_{\hspace{-.15em}R}\)$ at a one-loop order is provided using the  renormalisation constant $Y^{~}_{\omega\omega}$ in (\ref{YWW}) as
\begin{align*}
\frac{\aGR\hspace{-.15em}\(\mu^{2}_{\hspace{-.15em}R}\)}{4\pi}
=2\hspace{.1em}\delta Y^{~}_{\omega\omega},~~~\text{where}~~
\delta Y^{~}_{\omega\omega}:=Y^{~}_{\omega\omega}-1, 
\end{align*}
yielding
\begin{align*}
\bGR\(\aGR;\mu^{2}_{\hspace{-.15em}R}\)&=
-b^{~}_{\hspace{-.1em}g\hspace{-.1em}r}
\hspace{-.15em}\(\mu^{2}_{\hspace{-.15em}R}\)
\aGR^{\hspace{-.4em}2}\hspace{-.15em}\(\mu^{2}_{\hspace{-.15em}R}\),~~\text{with}~~~
b^{~}_{\hspace{-.1em}g\hspace{-.1em}r}\hspace{-.15em}\(\mu^{2}_{\hspace{-.15em}R}\):=-
\frac{8N^{~}_f\hspace{-.15em}\({\mu^{2}_{\hspace{-.15em}R}}\)-1}{24\pi}.
\end{align*}
Consequently, we obtain a solution with the boundary condition $\aGR(\LGRt)=\aGRz$ as
\begin{align*}
\aGR\hspace{-.15em}\(\mu^{2}_{\hspace{-.15em}R}\)
&=\frac{\aGRz}{1+\aGRz\hspace{.1em}b^{~}_{\hspace{-.1em}g\hspace{-.1em}r}\hspace{.1em}\log{\({\mu^{2}_{\hspace{-.15em}R}}/{\LGRt}\)}}.
\end{align*}
We set an energy dependent number of fermions as $N^{~}_f\hspace{-.15em}\({\mu^{2}_{\hspace{-.15em}R}}\)$ since only fermions heavier than the energy transfer contribute to the $\beta$-function.
When we replace $b^{~}_{\hspace{-.1em}g\hspace{-.1em}r}\hspace{-.15em}\(\mu^{2}_{\hspace{-.15em}R}\)$ to
\begin{align*}
b^{~}_\QED\hspace{-.15em}\(\mu^{2}_{\hspace{-.15em}R}\)&=
-\hspace{-.5em}\sum_{i=1}^{\Ncf\({\mu^{2}_{\hspace{-.15em}R}}\)}\frac{Q_i^2}{3\pi},
\end{align*}
 we can obtain the QED running coupling $\alpha^{~}_\QED\hspace{-.15em}\(\mu^{2}_{\hspace{-.15em}R}\)$ owing to fermions with the electric charge $Q_i^{~}$.
 %%%%%%%%%%%%%%%%%%%%%%%%%%%%%%%%%%%%%%%%%%%%%%%%%%%%%%%
\begin{table}[bt]
\begin{center}
\caption{\label{tableRC}\small
Table of renormalisation constants}
\vskip 2mm
\begin{tabular}{lllc}
\multicolumn{2}{c}{renormalisation constants}&order of coupling(s)&equation number\\
\hline
\multirow{6}{*}{Firld renomalisation}
&${Z^{~}_{\hspace{-.1em}\Aa}}^{\hlf}$&$\hspace{3em}\mathcal{O}(e^2)$&(\ref{ZAA})\\
&${Z^{~}_{\omega}}^{\hlf}$&$\hspace{3em}\mathcal{O}(\cGR^{\hspace{-.3em}2})$&(\ref{ZWW})\\
&${Z^{~}_{\Aa\hspace{-.1em}\omega}}^{\hspace{-.2em}\hlf}$&$\hspace{3em}\mathcal{O}(e\hspace{.1em}\cGR)$&(\ref{ZAW})\\
&${Z^{~}_{\hspace{-.1em}\E}}^{\hspace{-.2em}\hlf}$&\hspace{3em}$\mathcal{O}(\cGR^{\hspace{-.3em}2})$&(\ref{ZEE})\\
&${Z^{~}_{\psi\Aa}}^{\hlf}$&\hspace{3em}$\mathcal{O}(\cGR^{\hspace{-.3em}2})$&(\ref{ZpsiA})\\
&${Z^{~}_{\psi\omega}}^{\hlf}$&\hspace{3em}$\mathcal{O}(\cGR^{\hspace{-.3em}2})$&(\ref{ZpsiW})\\
\hline
\multirow{2}{*}{Mass renomalisation}
&$\delta{m^{\Aa}_{e}}$&\hspace{3em}$\mathcal{O}(e^2)$&(\ref{dmeA})\\
&$\delta{m^{\omega}_{e}}$&\hspace{3em}$\mathcal{O}(\cGR^{\hspace{-.3em}2})$&(\ref{dmeW})\\
\hline
\multirow{4}{*}{Vertex renomalisation}
&$Z^{V}_{\hspace{-.2em}\Aa\hspace{-.2em}\Aa}$ &\hspace{3em}$\mathcal{O}(e^2)$&(\ref{ZVAA})\\
&$Z^{V}_{\hspace{-.2em}\Aa\hspace{-.1em}\omega}$ &$\hspace{3em}\mathcal{O}(e\hspace{.1em}\cGR)$&(\ref{ZVAW})\\
&$Z^{V}_{\hspace{-.2em}\omega\Aa}$ &$\hspace{3em}\mathcal{O}(e\hspace{.1em}\cGR)$&(\ref{ZVAW})\\
&$Z^{V}_{\omega\omega}$ &\hspace{3em}$\mathcal{O}(\cGR^{\hspace{-.3em}2})$&(\ref{ZVWW})\\
\hline
\multirow{4}{*}{Coupling constant renomalisation}
&$Y^{~}_{\hspace{-.2em}\Aa\hspace{-.2em}\Aa}$& \hspace{3em}$\mathcal{O}(e^2)$& (\ref{YAA})\\
&$Y^{~}_{\hspace{-.2em}\Aa\hspace{-.1em}\omega}$& \hspace{3em}$\mathcal{O}(e\hspace{.1em}\cGR)$& (\ref{YAW})\\
&$Y^{~}_{\hspace{-.2em}\omega\Aa}$& \hspace{3em}$\mathcal{O}(e\hspace{.1em}\cGR)$& (\ref{YWA})\\
&$Y^{~}_{\omega\omega}$&\hspace{3em}$\mathcal{O}(\cGR^{\hspace{-.3em}2})$& (\ref{YWW})\\
\hline
\end{tabular}
\end{center}
\end{table}
%%%%%%%%%%%%%%%%%%%%%%%%%%%%%%%%%%%%%%%%%%%%%%%%%%%%%%%
 %%%%%%%%%%%%%%%%%%%%%%%%%%%%%%%%%%%%%%%%%%%
% Figure environment removed
%%%%%%%%%%%%%%%%%%%%%%%%%%%%%%%%%%%%%%%%%%%

As we discussed in {\bf section \ref{DCCEM}} and {\bf appendix \ref{appB}}, we renormalise  electromagnetic and gravitational coupling constants at $\Lambda^{~}_\QED=2m^{~}_e$ and $\LGR=\pb^{~}_\Earth\times{m^{~}_\mu}$, such that:
\begin{align*}
\alpha^{0}_\QED=\frac{1}{137.036}=0.00729735\cdots~~\text{and}~~~
\aGRz=\left.\frac{\cG^{\hspace{-.4em}2}}{4\pi}\right|_{\cG=1}=0.0795775\cdots,
\end{align*}
respectively.
An energy dependence of gravitational and electromagnetic effective couplings with the threshold effect is shown in Figure \ref{runC}.
For the gravitational coupling, the running effect concerning only an electron is also shown in the figure.
For the QED, an electron is the lightest charged particle; thus, the coupling stays constant below $\Lambda^{~}_\QED$ and, above that, increases according to an increase in the energy scale.
On the other hand, light neutral fermions, i.e. neutrinos, the vierbein boson and ghosts, also contribute to the gravitational $\beta$-function.
Consequently, the effective coupling decreases from the renormalisation energy scale to the threshold of neutrino pair creation and then increases according to the energy.
The number of particles contributing to the gravitational $\beta$-function is larger than those for the electromagnetic one; thus, the gravitational coupling runs faster than the electromagnetic one.
In reality, the renormalisation group equation for the gravitational interaction with all standard model particles has the Landau pole around $10$ MeV, and the perturbative QGED calculation loses its validity above this energy, even though the perturbative QED is extendable over the Planck energy.
As mentioned above, it does not mean a brake down of the QGED itself.
In the perturbative QGED, the Planck energy plays no significant role.
Although the Landau pole is at a relatively low energy region, we still have many applications of the perturbative QGED.
When we treat only one species of fermions, the Landau pole of the QGED is above the Planck energy.
%\cite{Symanzik:1973pp,PhysRevD.2.1541}

%%%%%%%%%%%%%%%%%%%%%%%%
% Particle creation in the strong field %
%%%%%%%%%%%%%%%%%%%%%%%%
\section{Particle creation in the strong field}
This section treats particle production due to the strong electromagnetic field or the strong gravitational field as an application of the perturbative QGED. 
The former is referred to as the Schwinger effect, and the latter is nothing other than the Hawking radiation\cite{1974Natur.248...30H}.

%Schwinger effect in the QED
\subsection{Schwinger effect in the QED}
The Schwinger effect, or the Sauter--Schwinger effect, is a phenomenon that the strong electric field creates a pair of charged particles\cite{Sauter:1931zz,Heisenberg:1936nmg,PhysRev.82.664}, e.g., an electron-positron pair.
A strong electric field separates the created pair from each other, preventing the pair annihilating back into the vacuum.
The Schwinger effect is not confirmed experimentally to date.
The effective action of the Schwinger effect, $S^{~}_\QED$,  is provided as\cite{Heisenberg:1936nmg,Medina:2015qzc}
\begin{align*}
S^{~}_\QED&=\frac{\FF^{2}_e}{\(2\pi\)^3}
\sum_{n=1}^\infty\frac{1}{n^2}\exp{\(-n\pi\frac{\tilde{m}^{2}_e}{\FF^{~}_e}\)},\\
\intertext{with}
\FF^{~}_e&:=\frac{eE}{\hbar}~~\text{and}~~~\tilde{m}^{~}_e=\frac{m^{~}_e}{\hbar},
\end{align*}
where $E$ is a electric field strength and $\FF^{~}_e$ is an electric force on an electron.
We note on the physical dimension that $[\FF^{~}_e]_\text{p.d.}=L^{-2}$ and $[\tilde{m}^{~}_e]_\text{p.d.}=L^{-1}$ in our units, thus the argument of exponential has a null physical dimension, and the action shall be dimensionless after four-dimensional spatial integrate.

Consider the Schwinger effect owing to the Coulomb-type electric field.
The critical electric-field strength to the electron-positron pair creation owing to the Schwinger effect is  $V^{~}_c=1\times10^{18}$ V/m\cite{PhysRev.82.664}, yielding the critical radius ($r^{~}_{\hspace{-.1em}c}$) and the critical energy ($\E^{~}_c$) of the Coulomb potential as
\begin{align*}
V^{~}_c=\frac{1}{4\pi\epsilon^{~}_0}\frac{e}{r^2_{\hspace{-.1em}c}}\implies
r^{~}_{\hspace{-.1em}c}\simeq37.9~\text{fm}~~{\implies}~~\E^{~}_c=
\frac{\hbar}{r^{~}_{\hspace{-.1em}c}}\simeq5.19~\text{MeV},
\end{align*}
where critical radius is larger than the proton charge radius.
We note that the critical radius is smaller than the atomic scale and larger than the nucleus scale.
Elementary charged particles cannot occur the Schwinger effect due to the energy-momentum conservation. 
However, this simple estimation suggests that the Schwinger effect may occur around a charged composite particle if the composite system can supply enough energy to create an electron-positron pair.

The infinite sum of vacuum-to-vacuum transition diagrams gives the effective action of the Schwinger effect in the QED\cite{PhysRev.82.664}.
We re-calculate the vacuum-decay rate of the Coulomb electric field at a one-loop order to compare with the Hawking radiation owing to the QGED.
We write the vacuum-to-vacuum transition amplitude in the QED as
\begin{align*}
\langle0_{\text{out}}|0_{\text{in}}\rangle&=
e^{iS^{~}_{\hspace{-.1em}\text{eff}}},~~~\text{yielding}~~
P^{~}_\text{v.d.}:=\left|\langle0_{\text{out}}|0_{\text{in}}\rangle\right|^2=
e^{-2\hspace{.15em}\text{Im}\hspace{-.1em}\left[S^{~}_{\hspace{-.1em}\text{eff}}\right]},
\end{align*}
where $P^{~}_\text{v.d.}$ is the vacuum decay probability.
The effective action of the vacuum-to-vacuum transition, denoted as $S^{~}_{\hspace{-.1em}\text{eff}}$, is related to the vacuum polarization (\ref{PIgeg})$\sim$(\ref{PIgeg3}) such that\cite{Dunne:2008kc}
\begin{align}
\Pi^{(\gamma{e}\gamma)}_{ab}(q^2_{i})\delta{q^2_{i}}&=\frac{\delta^2S^{~}_{\hspace{-.1em}\text{eff}}(\Aa)}
{\delta\Aa^a_{~}\hspace{.1em}\delta\Aa^b_{~}},\notag
\intertext{which has a solution}
S^{~}_{\hspace{-.1em}\text{eff}}(\Aa)&=
\(\Aa_{~}^{\bcdot}\hspace{.1em}\Pi^{(\gamma{e}\gamma)}_{\bcdots}\Aa_{~}^\bcdot\)\(q^2_{i}\)\delta{q^2_{i}},
\label{Seff}
\end{align}
where $\delta{q^2_i}$ is the possible energy spectrum due to the Schwinger effect.
We consider the Coulomb electric field that provides the electric potential in the momentum space as
\begin{align}
\Aa_{~}^a&=\(\frac{1}{\tilde{q}^{\hspace{.1em}2}},0,0,0\),\label{Aa}
\end{align}
where $\tilde{q}:=|\vec{q}\hspace{.1em}|$ is a virtual photon three-momentum.
We exploit an ansatz for the virtual photon four-vector such that $\eta_\bcdots^{~}q^\bcdot_{~}q^\bcdot_{~}=q^2=\tilde{q}^2$, yielding
\begin{align*}
q^a=\(\sqrt{2}\hspace{.1em}\tilde{q},0,0,\tilde{q}\).
\end{align*}
Consequently, we obtain the effective action as
\begin{align*}
-2\hspace{.15em}
\text{Im}\hspace{-.2em}\left[S^{~}_{\hspace{-.1em}\text{eff}}(\Aa)\right]&=
-2\times(4\pi)\sum_i\text{Im}
\left[\(\Aa_{~}^{\bcdot}\hspace{.1em}\Pi^{(\gamma{e}\gamma)}_{\bcdots}
\Aa_{~}^\bcdot\)\(q^{2}_{i}\)\right]{\delta q^2_{i}},\\
&\simeq-2\times(4\pi)\int_{4m^{2}_e}^{E^2_{\text{Max}}}\text{Im}
\left[\(\Aa_{~}^{\bcdot}\hspace{.1em}\Pi^{(\gamma{e}\gamma)}_{\bcdots}
\Aa_{~}^\bcdot\)\(q^{2}_{~}\)\right]{dq^2_{~}},\\
&=-\int_{2m^{~}_e}^{E^{~}_{\text{Max}}}\Pi_{\gamma^*\rightarrow e^+e^-}\(\tilde{q}\)d\tilde{q},
%
\intertext{with}
%
\Pi_{\gamma^*\rightarrow e^+e^-}\(\tilde{q}\)&:=4\pi\alpha
\frac{\hspace{.1em}\sqrt{\tilde{q}_{~}^2-4m^{2}_e}\(\tilde{q}^2_{~}+2m^2_e\)}{\tilde{q}_{~}^4}.
\end{align*}
%%%%%%%%%%%%%%%%%%%%%%%%%%%%%%%%%%%%%%%%%%%
% Figure environment removed
%%%%%%%%%%%%%%%%%%%%%%%%%%%%%%%%%%%%%%%%%%%

We consider the effect on the electromagnetic decay of the $\Delta^{\hspace{-.2em}+}$ baryon in Figure \ref{Ddecay}.
The critical radius is larger than proton and $\Delta^{\hspace{-.2em}+}$ baryon charge radii; thus, the Schwinger effect may occur due to the Coulomb electric field induced by a total charge of the $\Delta^{\hspace{-.2em}+}$ baryon rather than individual quarks.
We estimate the effective action with the $\Delta^+\rightarrow p\hspace{.2em}e^+e^-$ decay parameters as 
\begin{align*}
&\hspace{-2em}
-2\hspace{.1em}\text{Im}\hspace{-.1em}\left[S_\text{eff}\hspace{-.2em}\(\Delta^+\rightarrow p\hspace{.2em}e^+e^-\)\right]\\
&=
\int_{2m^{~}_e}^{E^{~}_{\text{Max}}}\Pi_{\gamma^*\rightarrow e^+e^-}\hspace{-.2em}\(\tilde{q}\)d\tilde{q},\\
&=\frac{16\pi\alpha}{3}\left\{
\frac{\sqrt{E^{2}_{\text{Max}}-4m^{2}_e}\(5E^2_\text{Max}+4m^2_e\)}{6E^{3}_{\text{Max}}}
+\log{\(\frac{2m^{~}_e}{E^{~}_\text{Max}+\sqrt{E^2_\text{Max}-4m^2_e}}\)}
\right\},\\
&\simeq6.75\times10^{-1},
\end{align*}
where $E^{~}_{\text{Max}}:=m^{~}_{\hspace{-.1em}\Delta}-m^{~}_p-2m^{~}_e$.

A Feynman diagramatic derivation of the $\Delta^+$ decay is as follows:
The decay amplitude of the $\Delta^+\rightarrow p\hspace{.2em}e^+e^-$ decay is
\begin{align*}
\M\(\Delta^+\rightarrow p\hspace{.2em}e^+e^-\)&=
J_{\Delta^+\rightarrow p+\gamma^*}\(q^2_{\gamma^*}\)\frac{1}{q^2_{\gamma^*}}
J_{\gamma^*\rightarrow e^+e^-}\(q^2_{\gamma^*}\)dq^2_{\gamma^*},
\end{align*}
where $J_{\Delta^+\rightarrow p+\gamma^*}(q^2_{\gamma^*})$ and $J_{\gamma^*\rightarrow e^+e^-}(q^2_{\gamma^*})$ are hadronic and electromagnetic currents such that
\begin{align*}
J_{\Delta^+\rightarrow p+\gamma^*}&:=
e \hspace{.1em}\bar\psi^{~}_p\Gamma_\mu\psi^{~}_\Delta\hspace{.1em}\epsilon^\mu_{\gamma^*},
~~\text{and}~~~
J_{\gamma^*\rightarrow e^+e^-}:=e\hspace{.1em}\bar\psi^{~}_e\gamma_\mu\psi^{~}_e\hspace{.1em}\epsilon^\mu_{\gamma^*},
\end{align*}
respectively, where, $\bar\psi^{~}_p\Gamma_\mu\psi^{~}_\Delta$ is the hadron current including an electric form-factor.
They have physical dimension of $[J]_\text{p.d.}=E$.
We obtain the decay width using the optical theorem and the decoupling approximation such that 
\begin{align}
\Gamma\(\Delta^+\rightarrow p\hspace{.2em}e^+e^-\)&=\frac{1}{2m^{~}_{\hspace{-.1em}\Delta}}\int
\left|\M\(\Delta^+\rightarrow p\hspace{.2em}e^+e^-\)\right|^2d\Omega,\notag \\
&\simeq\frac{1}{2m^{~}_{\hspace{-.1em}\Delta}}
\left|J_{\Delta^+\rightarrow p+\gamma^*}\hspace{.1em}dq^2_{\gamma^*}\right|^2
2\hspace{.1em}\text{Im}\hspace{-.1em}\left[S_\text{eff}\hspace{-.2em}\(\gamma^*\rightarrow e^+e^-\)\right],
\label{G2pee}
\end{align}
where $d\Omega$ is a phase space integration of the three body decay.
$q^{-2}_{\gamma^*}$ is absorbed by $\Aa^2$ (see (\ref{Aa})) in $S_\text{eff}$.
For the hadronic part, we do not have any detailed information.
We treat the hadronic current as a source of particle pair creation owing to the strong electric field; thus, we replace it simply with the critical energy of the Schwinger effect of the Coulomb electric field, such that:
\begin{align*}
\left|J_{\Delta^+\rightarrow p+\gamma^*}\hspace{.1em}dq^2_{\gamma^*}\right|^2\simeq
\E^{2}_c,
\end{align*}
due to the typical energy scale of the $e^+e^-$ pair creation; thus, we obtain
\begin{align*}
\text{(\ref{G2pee})}&\simeq
\frac{1}{2m^{~}_{\hspace{-.1em}\Delta}}\E^{2}_c
\hspace{.2em}\text{Im}\hspace{-.1em}\left[S_\text{eff}\hspace{-.2em}\(\gamma^*\rightarrow e^+e^-\)\right]
\simeq7.39\times10^{-3}~\text{MeV},
\end{align*}
In reality, a partial width of this decay branch is measured as\cite{10.1093/ptep/ptac097} 
\begin{align*}
\Gamma^{\text{exp}}_{~}\(\Delta^{\hspace{-.2em}+}\rightarrow
p\hspace{.2em}e^+e^-\)=\(4.91\pm0.82\)\times10^{-3}~\text{MeV}.
\end{align*}
Consequently, we obtain a numerical value consistent with the experimental measurement.
We can interpret the electromagnetic $\Delta^+$ decay as the Schwinger effect due to the Coulomb-type electric field.

%Hawking radiation in the QGED
\subsection{Hawking radiation in the QGED}
The Schwinger effect is a phenomenon that a strong static electric field creates pairs of charged particles that are then separated by the same strong electric field against the attraction between them.
On the other hand, when a strong gravitational field produces a pair of charged particles, both particles are attracted by the gravitational and electric fields, after which the pair annihilates back into the vacuum.
Fortunately, exceptional circumstances can exist on the event horizon of a black hole.
The black hole's strong gravitational field produces a pair of charged particles close enough to its event horizon that one of the particles in the pair falls into the hole, allowing the rest to reach the asymptotic observer without annihilation.
This phenomenon is known as the \emph{Hawking radiation}.
We estimate the thermodynamical temperature of the Hawking radiation from the Schwarzschild black hole utilising the same method used for the Schwinger effect.

Start from the effective action (\ref{Seff}) for the spin-connection vacuum polarization, which is 
\begin{align*}
S^{~}_{\hspace{-.1em}\text{eff}}(\omega)&=
\(\tilde\omega_{~}^{\bcdot}\hspace{.1em}
\Pi^{(\omega{e}\omega)}_{\bcdots}\tilde\omega_{~}^\bcdot\)\(q^2_{i}\)\delta{q^2_{i}},
\end{align*}
where $\Pi^{(\omega{e}\omega)}_{ab}(q^2_{~})$ is given in \textbf{section \ref{PSR}} and $\tilde\omega_{~}^a$ is a polarization vector of the spin-connection in the momentum space, which for the Schwarzschild metric in the frame fixed on a hole are given in (\ref{WSE}) and (\ref{WSx}).
In this section, we set the spin-connection and the momentum vector as
\begin{align*}
\left.{\tilde\omega}^{\text{S}}_a(\tilde{q})\right|_{\text{lab}}&=
\(\left.{\tilde\omega}^{\text{S}}_E(\tilde{q})\right|_{\text{lab}},\left.{\tilde\omega}^{\text{S}}_x(\tilde{q})\right|_{\text{lab}},0,0\),\\
q^a&=\(\sqrt{2}\hspace{.1em}\tilde{q},\tilde{q}\sin{\theta}\cos{\phi},\tilde{q}\sin{\theta}\sin{\phi},\tilde{q}\cos{\theta}\),
\end{align*}
where the laboratory frame is defined as in \textbf{Appendix \ref{appB}} concerning the hole instead of the Earth.
 The Hawking radiation is possible only at the event horizon, thus, the momentum must be
\begin{align*}
\tilde{q}^\text{S}=\frac{\hbar}{R_\bullet^\text{S}}~~\text{with}~~~
 R_\bullet^\text{S}=\frac{\kE M^{~}_\bullet}{\hbar},
 \end{align*}
where $R_\bullet^\text{S}$ is the Schwarzschild radius and $M^{~}_\bullet$ is a hole mass.
Consequently, the effective action is
\begin{align*}
-2\hspace{.15em}
\text{Im}\hspace{-.2em}\left[S^{~}_{\hspace{-.1em}\text{eff}}(\omega)\right]&=
-2\int_{-1}^{1}d\cos{\theta}\int_{0}^{2\pi}\hspace{-.3em}d\phi\hspace{.3em}2\tilde{q}
\left.\text{Im}\left[\omega_{~}^{\bcdot}\hspace{.1em}\Pi^{(\omega{e}\omega)}_{\bcdots}
\omega_{~}^\bcdot\right]\right|_{\tilde{q}=\tilde{q}^\text{S}}\delta\tilde{q},
%
\intertext{with}
%
\text{Im}\left[\omega_{~}^{\bcdot}\hspace{.1em}\Pi^{(\omega{e}\omega)}_{\bcdots}
\omega_{~}^\bcdot\right]&=
{\aGR}
\frac{\hspace{.1em}\sqrt{q_{~}^2-4m^{2}_e}\(q^2_{~}+2m^2_e\)}{3\(q^2\)^{1/2}}\(
\left.{\tilde\omega}^{\text{S}}_\bcdot(\tilde{q})\right|_{\text{lab}}
\(\eta^{~}_\bcdots-\frac{q^{~}_\bcdot q^{~}_\bcdot}{q^2}\)
\left.{\tilde\omega}^{\text{S}}_\bcdot(\tilde{q})\right|_{\text{lab}}\).
\end{align*}
After the integration with an approximation $m^{~}_e=0$,  we interpret the result thermodynamically as
\begin{align*}
&-\text{Im}\hspace{-.2em}\left[S^{~}_{\hspace{-.1em}\text{eff}}(\omega)\right]=
-\beta^{~}_\bullet\hspace{.1em}\delta\tilde{q},
%
\intertext{yielding}
%
\beta^{-1}_\bullet&=\cGR^{\hspace{-.4em}-1}
\frac{9\pi}{4+3\pi^2}T_\bullet^\text{Hawking},\\
&\simeq\cGR^{\hspace{-.4em}-1}\(8.41\times10^{-1}\)T_\bullet^\text{Hawking}
~~~
\text{with}
~~
T_\bullet^\text{Hawking}:=\frac{\hbar}{\kE M^{~}_\bullet},
\end{align*}
where $T_\bullet^\text{Hawking}$ is the Hawking temperature of the Schwarzschild black hole.
{As we expected}, the QGED provides the hole temperature proportional to the hole mass inverse.
When we set $\cGR=1$ after the coupling constant renormalisation, we obtain the correction factor ${9\pi}/{(4+3\pi^2)}$ owing to the QGED.

%%%%%%%%%%%%%%%%%%
% Discussions and summary %
%%%%%%%%%%%%%%%%%%
\section{Summary}
This report proposed a renormalisable quantum theory of gravity (QGED) based on the standard method used to quantise the Yang--Mills theory.
In our theory, objects quantised are spin-connection (gauge boson) and vierbein (section) fields defined in the local inertial manifold, together with the electromagnetic field (gauge boson) and electron (section) fields of the $U(1)$ gauge theory.
In classical general relativity, the solution of the Einstein equation provides the metric tensor, which determines the structure of space-time as a (pseudo-)Riemannian manifold.
In the QGED, an expected value of quantised vierbein fields provides the metric tensor, instead of a solution of the classical equation of motion. 
The space-time manifold itself is not a quantisation target; thus, it is a smooth manifold even after quantisation.
To be brief, we quantise not the space-time but the ruler to measure it.
The author has discussed a space whose metric tensor is provided as the expected value of a stochastic process in Ref.\cite{Kurihara_2018}.

To quantise both general relativity and the Yang--Mills theory simultaneously, the current author has reformulated both theories as the geometrical theory on an equal footing in the previous study\cite{Kurihara:2022sso}.
Moreover, the author has been applying the BRST quantisation non-perturbatively for pure gravity in the Heisenberg picture to them\cite{doi:10.1140/epjp/s13360-021-01463-3}.
This report took them further and provided a perturbative extension of quantum gravity.
Although the quantisation method utilised for general relativity in this report is faithful to the standard BRST quantisation of the Yang--Mills theory, it is not standard compared with traditional methods of quantum general relativity in some aspects; we identify a spin-connection as a gauge boson instead of the metric tensor and utilise the covariant differential including a gravitational coupling constant.
The unconventional gravitational coupling constant provides several benefits to the theory:
\begin{enumerate}
\item We can quantise general relativity completely parallel to the Yang--Mills theory.
\item The bare gravitational coupling constant can absorb the ultraviolet divergence of a scattering vertex.
\item Free gravitational fields with vanishing coupling constant provide a linearised Einstein equation, which is also parallel to the Yang--Mills theory.
\end{enumerate}
The old quantum theoretical consideration provides the evidence of the spin-connection to be the gauge boson
\cite{Kurihara_2020}, and a gravitational coupling constant is natural to be there to quantise the gauge theory\cite{doi:10.1140/epjp/s13360-021-01463-3}.
Then we proved all fields are nilpotent and the Lagrangian is BRST invariant, which ensured the renormalisation with ghosts is anomaly free.
Moreover, after defining the physical states of gravitation appropriately\cite{doi:10.1140/epjp/s13360-021-01463-3}, it ensures the unitarity of the physical amplitude.

We extracted a set of Feynman rules based on the QGED Lagrangian with gauge fixing and ghost parts.
Propagators in the momentum space for the curved space-time are not trivial because the Fourier transformation kernel includes the metric tensor.
This report defined all Feynman rules in the local inertial frame in which the gravity is eliminated locally; thus, the transformation kernel has a flat metric, and the Fourier transformation is well-defined\cite{Kurihara:2022green}.
Utilising Feynman rules of the QGED prepared here, we provided all renormalisation constants and showed that the theory is perturbatively renormalisable at a one-loop level.
In the renormalisation theory, we must replace infinite-valued bare objects with experimentally measured ones.
We showed that the gravitational coupling constant is measurable experimentally\cite{Kurihara:2022g-2}.

Above all, the existence of the gravitational coupling constant allows us to discuss the running effect of gravitational coupling.
The Einstein (Newtonian) gravitational constant is a fundamental constant of Nature and does not change depending on the energy scale we are looking at.
On the other hand, the effective coupling may depend on the energy scale of observables through the renormalisation group equation.
Since a boson loop provides an opposite sign of a fermion loop, a boson loop decreases the effective coupling according to an increased energy scale when the fermion loop is vice versa.
In the QED, only charged fermions contribute to the beta function and provide an increasing effective coupling according to the increase in the energy scale. 
In general, the number of fermions contributing to the gravitational $\beta$-function is larger than those for the electromagnetic one; thus, the gravitational coupling runs faster than the electromagnetic one.
Moreover, $\aGR$ is larger than $\alpha$ at the energy scale $\mu^{~}_R\simeq m^{~}_e$; thus, two effective couplings are never coincide at any energy scale.

The gravitational Landau pole of the QGED with one fermion species is above the Planck energy, and it, with all standard model particles, is at around $10$ MeV.
Although the perturbative QGED calculation loses its validity above the Landau pole, it does not mean a brake-down of the QGED itself.
For the QCD, even though the perturbative QCD fails under the $\Lambda_\text{QCD}$, non-perturbative treatment works appropriately, i.e., the lattice QCD.
One of the primary candidates of non-perturbative quantum theories of gravity is a loop quantum gravity (see, e.g., Refs.\cite{rovelli_vidotto_2014,Ashtekar:2021kfp}).
Although the loop quantum gravity has common aspects with the lattice QCD, one of the main differences is that the former does not assume the existence of a smooth manifold as a background.
In contrast, the latter takes the continuum limit of the lattice space to simulate a smooth space-time.
The loop quantum gravity keeps a size parameter finite, which protects the theory from UV divergence.
On the other hand, the lattice QCD provides finite results at the short-distance limit since the QCD is asymptotic free and non-perturbative QCD is well-defined at zero distance.
The lattice approximation of the QGED, if any, is different from both theories.
For a lattice approximation of the QGED, we have to take a short-distance limit to obtain realistic physical quantities since the QGED theory is constructed based on the smooth manifold.
However, the QGED is not an asymptotic-free theory; thus, it may have a divergence.
We need yet another formulation of the non-perturbative quantum theory than utilising the discrete space-time.

This report also provided a perturbative estimation of the Hawking radiation using the QGED, with reference to the Schwinger effect of the QED.
The Hawking radiation and the Schwinger effect are the particle-pair creation owing to the strong static fields; the former is due to the gravitational field, and the latter is due to the electric field.
The Schwinger effect owing to the electric force has yet to be observed experimentally since the critical field strength is enormous.
On the other hand, a quark pair creation owing to the strong force is an indispensable part of the hadronisation models in jet simulation programs, e.g., PYTHIA\cite{Sjostrand:2014zea} and HERWIG\cite{Bahr:2008pv}.
These hadronisation models can describe a jet structure quite well; thus, the Schwinger effect owing to the QCD has experimental support.
We provided a possible interpretation of the $\Delta^+\rightarrow{p}\hspace{.1em}e^+e^-$ decay owing to the Schwinger effect.
In contrast with the electric force, which has attractive and repulsive forces, the gravitational force is attractive only.
Consequently, though a strong electric field separates a created charged particle pair, they re-combine immediately for the gravitational field case.
One possible loophole in this problem is using the black hole's event horizon, i.e., Hawking radiation.
The perturbative QGED shows that the hole temperature is proportional to the hole mass for the Schwarzschild black hole.

Ultimately, we emphasise that the proposed perturbative QGED is an experimentally testable theory, e.g., future high-precision measurements of a muon anomalous magnetic moment.
Other applications, such as the state of arts measurements of the atomic energy spectrum, the Rydberg constant and others, are also possible testbench of the QGED.
A quest for the non-perturbative approach of the QGED is also an essential direction of future study.

%
%  Acknowledgements
%
\section*{Acknowledgements}
I would like to thank Dr Y$.$ Sugiyama, Prof$.$ J$.$ Fujimoto and Prof$.$ T$.$ Ueda for their continuous encouragement and fruitful discussions.
%%%%%%%%%%% end Acknowledgements

%%% ----------------------------------------------------------------------
% Appendix
%%% ----------------------------------------------------------------------
\numberwithin{equation}{section}
%\newpage
\vskip 2cm
\noindent
{\Large\bf Appendix:}\\
\vspace{-0.8cm}
\begin{appendix}
%%%%%%%%%%% appendix A %%%%%%%%%%%
\section{Proof of nilpotent}\label{app1}
%%%%%%%%%%%%%%%%%%%%%%%%%%%%%%%%
This section summarizes proof of nilpotent for the gravitational BRST transformation..\\

\noindent\fbox{{\bf Coordinate vector}}\\
The coordinate vectors are fundamental vectors on $T\MM$, which is nilpotent as
\begin{align*}
\delBRST^\GR\left[\delBRST^\GR\left[x^{\mu}\right]\right]
&=\delBRST^\GR\left[\chi^\mu\right]=0.
\end{align*}
\noindent
%%%%%%%%%%%%%%%%%%%%%%%%%%
\fbox{{\bf Ghost field}}\\
Ghost  $\chi^\mu$ is trivially nilpotent.
For $\chi^a_{~b}:=\eta_{b\bcdot}\chi^{a\bcdot}$, we obtain
\begin{align*}
\delBRST^\GR\left[\delBRST^\GR\left[\chi^a_{~b}\right]\right]&=
\delBRST^\GR\left[\chi^a_{~c}\chi^{c}_{~b}\right]
=
\chi^a_{~c_2}\chi^{c_2}_{~~c_1}\chi^{c_1}_{~~b}-\chi^a_{~c_1}\chi^{c_1}_{~~c_2}\chi^{c_2}_{~~b}=0,
\end{align*}
due to anticommutativity of the ghost field. A tensor $\partial^{~}_{\mu}\chi^{\nu}$ is also nilpotent as
\begin{align*}
\delBRST^\GR\left[\delBRST^\GR\left[\partial^{~}_{\mu}\chi^{\nu}\right]\right]&=
-\delBRST^\GR\left[\partial^{~}_{\mu}\chi^{\rho}\partial^{~}_{\rho}\chi^{\nu}\right]=
-\partial^{~}_{\mu}\chi^{\rho_1}\partial^{~}_{\rho_1}\chi^{\rho_2}\partial^{~}_{\rho_2}\chi^{\nu}
+\partial^{~}_{\mu}\chi^{\rho_1}\partial^{~}_{\rho_1}\chi^{\rho_2}\partial^{~}_{\rho_2}\chi^{\nu}=0.
\end{align*}

\noindent
\fbox{{\bf Vierbein form}}
\begin{align*}
\delBRST^\GR\left[\delBRST^\GR\left[\eee^a\right]\right]&=
-\cGR\delBRST^\GR\left[\eee^b\chi^a_{~b}\right]=\cGR^2\(
\eee^{b_1}\chi^{b_2}_{~~b_1}\chi^a_{~b_2}+\eee^{b_2}\chi^a_{~b_1}\chi^{b_1}_{~~b_2}\)=0.
\end{align*}

\noindent
\fbox{{\bf Spin form}}\\
\begin{align*}
\delBRST^\GR\left[\delBRST^\GR\left[\www^{ab}\right]\right]&=
\delBRST^\GR\left[
d\chi^{ab}
-\cGR\(\www^{a}_{~\bcdot}\hspace{.1em}\chi^{\bcdot b}
+\www^{b}_{\hspace{.3em}\bcdot}\hspace{.1em}\chi^{a \bcdot}\)
\right],\\
&=\cG\hspace{.3em}d\(\chi^a_{\hspace{.3em}\bcdot}\chi^{\bcdot b}\)
-\cG\left\{
\(d\chi^a_{\hspace{.3em}\bcdot}\)\chi^{\bcdot b}+
\(d\chi^b_{\hspace{.3em}\bcdot}\)\chi^{a\bcdot}
-\cG\www^\bcdot_{\hspace{.3em}\star}
\(
\chi^{a\star}\chi_\bcdot^{\hspace{.3em}b}+
\chi^{b\star}\chi_{\hspace{.3em}\star}^a
\)
\right\},\\&=0.
\end{align*}

\noindent
\fbox{{\bf Surface form}}\\
Applying the BRST-transformation on it again, one can get
\begin{align*}
\delBRST^\GR\left[\delBRST^\GR\left[\SSS_{ab}\right]\right]&=
\cGR\hspace{.1em}\epsilon_{abc_1c_2}\delBRST^\GR\left[\chi^{c_1}_{~c_3}\eee^{c_3}\wedge\eee^{c_2}\right],\\
&=\cGR^2\hspace{.1em}\epsilon_{abc_1c_2}\Bigl\{
\chi^{c_1}_{~c_4}\chi^{c_4}_{~c_3}\eee^{c_3}\wedge\eee^{c_2}-
\chi^{c_1}_{~c_3}\chi^{c_3}_{~c_4}\eee^{c_4}\wedge\eee^{c_2}-
\chi^{c_1}_{~c_3}\chi^{c_2}_{~c_4}\eee^{c_3}\wedge\eee^{c_4}
\Bigr\},\\&=0,
\end{align*}
because first term is the same as the second term and the third term is symmetric with $c_1$ and $c_2$ exchange.\\

\noindent
\fbox{{\bf Volume form}}\\
The volume form is global scalar and their BRST transformation is expected to vanish, which can be confirmed as
\begin{align*}
\delBRST^\GR\left[\vvv\right]&=\frac{1}{4!}\epsilon_{\bcdott}
\delBRST^\GR\left[\eee^\bcdot\wedge\eee^\bcdot\wedge\eee^\bcdot\wedge\eee^\bcdot\right]=
\frac{1}{3!}\epsilon_{a_1\bcdot\bcdots}
\delBRST^\GR\left[\chi^{a_1}_{~~a_2}\eee^{a_2}\wedge\eee^\bcdot\wedge\eee^\bcdot\wedge\eee^\bcdot\right]=0,
\end{align*}
due to $\eee^\bcdot\wedge\eee^\bcdot\wedge\eee^\bcdot\wedge\eee^\bcdot\propto\epsilon^{\bcdott}$ and $\chi^{a_1}_{~~a_2}=0$ when $a_1=a_2.$\\

\noindent
\fbox{{\bf ghost forms}}\\
The BRST transformation of $\ccc^a$ is given by
\begin{eqnarray*}
\delBRST^\GR\left[\ccc^a\right]&=&
\delBRST^\GR\left[\chi^a_{~b}~\E^b_\mu~dx^\mu\right],\\
&=&\chi^a_{~b_1}\chi^{b_1}_{~b_2}\E^{b_2}_\mu dx^\mu-
\chi^a_{~b_1}~\E^{b_2}_\mu \chi^{b_2}_{~b_1}dx^\mu
+\chi^a_{~b}~\left(\partial^{~}_\mu\chi^\nu\right)\E^b_\nu~dx^\mu
-\chi^a_{~b}~\E^b_\mu d\chi^\mu=0,
\end{eqnarray*} 

\noindent
\fbox{{\bf Other forms}}\\
Nilpotent of other forms are trivial and the proof is omitted here.\\

\vskip 2mm
\noindent
%%%%%%%%%%%%%%%%%%%%%%%%%%
\fbox{{\bf Gravitational Lagrangian}}\\
The quantum Lagrangian must be the BRST-null.
Gauge-fixing and Fadeef--Popov Lagrangians are constructed to satisfy the BRST-null condition in section {6-2}.
Nilpotent of only the gravitational Lagrangian is given here.
The BRST transformation of the gravitational Lagrangian is provided as
\begin{eqnarray*}
\delBRST^\GR\left[\LLL_G\right]&=&
\frac{1}{2}\delBRST^\GR\left[\left(
d\www^\bcdots
+\cGR\hspace{.1em}\www^\bcdot_{~\star}\wedge\www^{\star\bcdot}\right)\wedge{\SSS}_\bcdots
-\frac{\Lambda}{3!}\vvv\right].
\end{eqnarray*}
The BRST transformation for the volume form is vanished by itself. 
For the derivative term,
\begin{eqnarray*}
\delBRST^\GR\left[
d\www^\bcdots\wedge{\SSS}_\bcdots
\right]&=&
\epsilon_{abc_2c_3}\chi_{~c_1}^{b}d\www^{ac_1}\wedge\eee^{c_2}\wedge\eee^{c_3}+
\epsilon_{abc_2c_3}\www^{ac_1}\wedge d\chi_{~c_1}^{b}\wedge\eee^{c_2}\wedge\eee^{c_3}\\&~&+
\epsilon_{abc_1c_2}\chi^{c_1}_{~c_3}~d\www^{ab}\wedge\eee^{c_3}\wedge\eee^{c_2}\\
&=&2~\www^{ac_1}\wedge d\chi^{b}_{~c_1}\wedge\SSS_{ab},
\end{eqnarray*}
where first- and third-terms are cancelled each other.
Remnant term is transformed as
\begin{eqnarray*}
\delBRST^\GR\left[\www^\bcdot_{~\star}\wedge\www^{\star\bcdot}\wedge{\SSS}_\bcdots\right]&=&
\epsilon_{abc_2c_3}\chi^{c_2}_{~c_4}\www^{ac_1}\wedge\www_{c_1}^{~~b}\wedge\eee^{c_4}\wedge\eee^{c_3}+
\epsilon_{abc_3c_4}\chi^{c_2}_{~c_1}\www^{ac_1}\wedge\www_{c_2}^{~~b}\wedge\eee^{c_3}\wedge\eee^{c_4}\\&~&+
\epsilon_{abc_3c_4}\chi^{b}_{~c_2}\www^{ac_1}\wedge\www_{c_1}^{~~c2}\wedge\eee^{c_3}\wedge\eee^{c_4}-2{\cGR}^{-1}\www^{ac_1}\wedge d\chi^b_{~c_1}\wedge\SSS_{ab},\\
&=&-2{\cGR}^{-1}\www^{ac_1}\wedge d\chi^b_{~c_1}\wedge\SSS_{ab}.
\end{eqnarray*}
In the r.h.s of the first line, the second term is zero as itself, and first- and third-terms are cancelled each other.
Therefore one can confirmed $\delBRST^\GR\left[\LLL_G\right]=0$ after summing up all terms.
%

If we use a following remake, we can give simpler proofs for above forms.\\
\noindent
\fbox{{\bf Remark}}\\
If both of two fields, $\alpha$ and $\beta$, are nilpotent, $\alpha\beta$ is also nilpotent.\\
\noindent
{\it Proof:}\\
If a field $X$ is nilpotent, signatures of the Leibniz rule satisfy $\epsilon_{X}=-\epsilon_{\delta X}$ due to $\delBRST^\GR[\delBRST^\GR[X]]=0$ and (\ref{Leib}), where
$\epsilon_{X}$ ($\epsilon_{\delta X}$) is a signature of $X$ ($\delBRST^\GR[X]$), respectively.
Therefore
\begin{eqnarray*}
\delBRST^\GR\left[\delBRST^\GR\left[\alpha\beta\right]\right]&=&
\epsilon_{\alpha}\delBRST^\GR\left[\alpha\right]\delBRST^\GR\left[\beta\right]+
\epsilon_{\delta\alpha}\delBRST^\GR\left[a\right]\delBRST^\GR\left[\beta\right]
~=~0.
\end{eqnarray*} 

%%%%%%%%%%% appendix B %%%%%%%%%%%
\section{Measurement of the gravitational coupling constant}\label{appB}
We consider an electron scattering with the Earth's gravity in Figure \ref{figahc} to measure the gravitational coupling constant.

Morishima, Futamase and Shimizu\cite{Morishima_2018} proposed a possible resolution to the muon  $\gmt$ anomaly owing to the static gravitational potential of the Earth using the post-Newtonian approximation.
Visser\cite{Visser:2018omi} immediately criticised their proposal as contradicting Einstein's equivalence principle and pointed out also that they missed counting the effect of the Sun's and the Galaxy's static potential, which are much stronger than the Earth's.
However, a frame fixed on the Earth is the inertial system concerning the Sun's and the Galaxy's gravity.
Due to Einstein's equivalence principle, the static gravitational potential does not contribute to any local observable obtained in the inertial system.
On the other hand, experimental apparatus on the Earth is not in an inertial system but in an accelerating system concerning the Earth's gravity.
A finite effect on the muon anomalous coupling owing to the Earth's gravity is measurable\cite{Kurihara:2022g-2}. 

%%%%%%%%%%%%%%%%%%%%%%%%%%%%%%%%%%%%%%%%%%%
% Figure environment removed
%%%%%%%%%%%%%%%%%%%%%%%%%%%%%%%%%%%%%%%%%%%

% Schwarzscild spin-connection
\paragraph{Schwarzscild spin-connection:}
The Schwarzschild solution provides a gravitational field induced by the Earth. 
We first calculate a classical spin-connection of the Schwarzschild solution in the inertial space.
We set the Earth at rest in the global manifold and the origin of the standard basis at the Earth's centre, as shown in Figure  \ref{sphare}.
We denote the standard basis in the global manifold as $dx^\mu=(dt,dr,d\theta,d\phi)$ in the polar coordinate.
We equate the configuration space to the inertial space and utilise the polar coordinate of $d\xi^a=(d\tau,d\rho,d\vartheta,d\varphi)$ as the local standard basis.  
The local inertial manifold located at the Earth's surface at time $t=\tau=0$. 
It has a standard basis whose spatial axes are parallel to the global ones.
At first, we set the origin of the local standard basis at the Earth's centre.
After the Fourier transformation, we set the origin of the momentum coordinate at the Earth's surface. 
(See Figure \ref{sphare}.)
The Schwarzscild spin-connection in the global manifold is given as\cite{fre2012gravity}
\begin{align*}
{\omega}^{\text{S}\hspace{.2em}tr}_{\hspace{.5em}t}&=
-{\omega}^{\text{S}\hspace{.2em}rt}_{\hspace{.5em}t}\hspace{.3em}=-\frac{{\Rs}}{2r^2},\hspace{4em}
{\omega}^{\text{S}\hspace{.2em}r\theta}_{\hspace{.5em}\theta}=
-{\omega}^{\text{S}\hspace{.2em}\theta r}_{\hspace{.5em}\vartheta}=\fsch(r),\\
{\omega}^{\text{S}\hspace{.2em}r\phi}_{\hspace{.5em}\phi}&=
-{\omega}^{\text{S}\hspace{.2em}\phi r}_{\hspace{.5em}\phi}=\fsch(\rho)\sin{\theta},\hspace{2.2em}
{\omega}^{\text{S}\hspace{.2em}\theta\phi}_{\hspace{.5em}\phi}=
-{\omega}^{\text{S}\hspace{.2em}\phi\theta}_{\hspace{.5em}\phi}=\cos{\theta},
%
\intertext{otherwise zero, where}
%
\fsch(\rho)&:=\sqrt{1-\frac{\Rs}{\rho}},\hspace{3.em}
\Rs:=\frac{\kE M_\Earth}{4\pi},
\end{align*}
and $M_\Earth$ and $\Rs$ are a mass and the Schwarzschild radius of the Earth, respectively.
The spin-connection in the inertial space is provided using $\omega^{\text{S}\hspace{.2em}ab}_{\hspace{.2em}c}(\xi)=\omega^{\text{S}\hspace{.2em}ab}_{\hspace{.2em}\mu}(x(\xi))\E^{\text{S}\hspace{.1em}\mu}_{\hspace{.2em}c}(x(\xi))$, such that:%\cite{Kurihara:2022green}
\begin{align*}
{\omega}^{\text{S}\hspace{.2em}\tau\rho}_{\hspace{.5em}\tau}&=
-{\omega}^{\text{S}\hspace{.2em}\rho\tau}_{\hspace{.5em}\tau}\hspace{.3em}=-\frac{{\Rs}}{2\fsch(\rho)\rho^2},\hspace{2em}
{\omega}^{\text{S}\hspace{.2em}\rho\vartheta}_{\hspace{.5em}\vartheta}=
-{\omega}^{\text{S}\hspace{.2em}\vartheta\rho}_{\hspace{.5em}\vartheta}=\frac{\fsch(\rho)}{\rho},\\
{\omega}^{\text{S}\hspace{.2em}\rho\varphi}_{\hspace{.5em}\varphi}&=
-{\omega}^{\text{S}\hspace{.2em}\varphi\rho}_{\hspace{.5em}\varphi}=\frac{\fsch(\rho)}{\rho},\hspace{4.2em}
{\omega}^{\text{S}\hspace{.2em}\vartheta \varphi}_{\hspace{.5em}\varphi}=
-{\omega}^{\text{S}\hspace{.2em}\varphi\vartheta}_{\hspace{.5em}\varphi}=\frac{\cos\vartheta}{\rho\sin{\vartheta}},
\end{align*}
After the Fourier transformation, we obtain the Schwarzscild spin-connection in the momentum space as\cite{Kurihara:2022green}
\begin{subequations}
\begin{align}
%%%%
\tilde{\omega}^{\text{S}\hspace{.1em}\rho\tau}_{\hspace{.4em}\tau}(E,\pb)
&\simeq\frac{\pi}{4}\Rs^2\(
\frac{\pi}{\pb}-4\log{\pb}-4\gamma^{~}_E-4i\pi+{\cal O}\(\pb\)
\)\(2\pi\hbar\hspace{.1em}\delta\hspace{-.1em}E\),\label{wb}\\
%%%
\tilde{\omega}^{\text{S}\hspace{.1em}\rho\vartheta}_{\hspace{.4em}\vartheta}(E,\pb)=
\tilde{\omega}^{\text{S}\hspace{.1em}\rho\varphi}_{\hspace{.4em}\varphi}(E,\pb)=
&\simeq
\frac{\pi}{4}\Rs^2\(
\frac{1}{\pb^2}-\frac{\pi}{\pb}
+2\log{\pb}+2\gamma^{~}_E-1-2i\pi
+{\cal O}\(\pb\)\)\(2\pi\hbar\hspace{.1em}\delta\hspace{-.1em}E\),\label{wd}
%%%
\intertext{and otherwise zero, where}
\pb&:=\frac{\Rs}{\hbar}|\vec{q}\hspace{.08em}|,~~~
\delta{E}:=\left.\hspace{.1em}\delta(E+i\epsilon)\right|_{\epsilon\rightarrow+0},\notag
\end{align}
\end{subequations}
and $q^a_{~}=(E,\vec{q})$ is a momentum of a spin-connection.
%%%%%%%%%%%%%%%%%%%%%%%%%%%%%%%%%%%%%%%%%%%
% Figure environment removed
%%%%%%%%%%%%%%%%%%%%%%%%%%%%%%%%%%%%%%%%%%%

As shown in Figure \ref{sphare}, we set a local coordinate system such that the origin is at the Earth's centre, the test particle is on the $\xi^{~}_3$-axis, and its three-momentum $\vec{p}$ lies in a $\xi_1$-$\xi_3$ plane.
We refer to it as the laboratory frame.
%The experimental apparatus is fixed on the Earth; thus, a laboratory frame is equated to the global frame.
We set the $\xi_1$-axis to a beam momentum $p^{~}_e$ in the laboratory frame for experiments using an electron beam\footnote{Here, ``electron'' includes other leptons used in experiments such as a muon.}, such that:
\begin{align*}
\left.p\right|_{\text{lab}}&=\(E^{~}_{e},p^{~}_{e},0,-\frac{q^{~}_e}{2}\) ,~~~~
\left.p'\right|_{\text{lab}}=\(E^{~}_{e},p^{~}_{e},0,\frac{q^{~}_e}{2}\).
\end{align*}
After changing the polar coordinate to the Cartesian coordinate, we obtain the Schwarzschild spin-connection as
\begin{align*}
%{\tilde\omega}^{\text{S}}_a(\pb)&:=
\({\tilde\omega}^{\text{S}}_E(\pb),{\tilde\omega}^{\text{S}}_x(\pb),{\tilde\omega}^{\text{S}}_y(\pb),
{\tilde\omega}^{\text{S}}_z(\pb)\)
=f_\omega\(
\hspace{.1em}\tilde{\omega}^{\text{S}\hspace{.1em}\rho\tau}_{\hspace{.4em}\tau}(0,\pb),
\hspace{.1em}\pb\hspace{.2em}\tilde{\omega}^{\text{S}\hspace{.1em}\rho\vartheta}_{\hspace{.4em}\vartheta}(0,\pb)
,0,0\),
\end{align*}
where $f_\omega$ is a normalisation factor to be determined.
The Schwarzschild spin-connection only with an index type such that $\tilde{\omega}^{\text{S}\hspace{.1em}\rho\bullet}_{\hspace{.4em}\bullet}$ gives a non-zero value, consistent with the observation in (\ref{wpolvec}); thus, the spin-connection has a vector representation, even if it is a tensor object.
We set the normalisation constant as
\begin{align}
f_\omega=\(\frac{\pi\hbar^2}{4}\)^{-1},\label{fw}
\end{align}
whose legitimacy of the normalization will be shown later in this section.
Consequently, we obtain the normalised spin-connection for the Schwarzschild solution in the laboratory frame as
\begin{subequations}
\begin{align}
\left.{\tilde\omega}^{\text{S}}_E(\pb)\right|_{\text{lab}}&=\(\frac{\Rs}{\hbar}\)^2
\(\frac{\pi}{\pb}-4\log{\pb}-4\gamma^{~}_E+{\cal O}\(\pb\)\),\label{WSE}\\
\left.{\tilde\omega}^{\text{S}}_x(\pb)\right|_{\text{lab}}&=\(\frac{\Rs}{\hbar}\)^2\(\frac{1}{\pb}+2\pb\log{\pb}-\pi+
\pb\(2\gamma^{~}_E-1\)+{\cal O}\(\pb\)\),\label{WSx}
\end{align}
\end{subequations}
which has an inverse energy dimension, such that $[{\tilde\omega}^{\text{S}}_E]_\text{p.d.}=[{\tilde\omega}^{\text{S}}_x]_\text{p.d.}=E^{-2}$, which is the same as that of the spin-connection propagator.

To estimate the gravitomagnetic effect quantitatively for the $\cG$ measurement, we utilise the Breit frame setting a momentum vector of initial and final electrons as
\begin{align*}
\left.p^a\right|_{\text{Br}}:=\(E_{\text{Br}},0,0,-\frac{q^{~}_e}{2}\) ~~\text{and}~~~
\left.p'^a\right|_{\text{Br}}=\(E_{\text{Br}},0,0,\frac{q^{~}_e}{2}\),
\end{align*}
with a forward scattering approximation neglecting $\mathcal{O}\(q^{~}_e/E^{~}_{\text{Br}}\)^2$.
Moreover,  we require the ideal experimental set-up such that the Earth rests in the Universe; thus, the centrifugal and Coriolis forces do not contribute a particle motion.
In the Breit frame, the spin-connection of the Schwarzschild solution in the momentum space is
\begin{align*}
\left.{\tilde\omega}^{\text{S}}_E(\pb)\right|_{\text{Br}}&=
\gamma^{~}_e\left.{\tilde\omega}^{\text{S}}_E(\pb)\right|_{\text{Lab}}-
\gamma^{~}_e\beta^{~}_e\left.{\tilde\omega}^{\text{S}}_x(\pb)\right|_{\text{Lab}},~~~
\left.{\tilde\omega}^{\text{S}}_x(\pb)\right|_{\text{Br}}=
\gamma^{~}_e\left.{\tilde\omega}^{\text{S}}_x(\pb)\right|_{\text{Lab}}-
\gamma^{~}_e\beta^{~}_e\left.{\tilde\omega}^{\text{S}}_E(\pb)\right|_{\text{Lab}}
%
\intertext{after the Lorentz boost, where}
%
\beta_e&=\frac{p^{~}_e}{E^{~}_e},~~~~~
\gamma_e=\frac{E^{~}_e}{m^{~}_e}.
\end{align*}

% Scattering amplitude
\paragraph{Scattering amplitude:}
We can obtain the scattering amplitude by applying the Feynman rule (\ref{v2}) with $\epsilon^{~}_{UV}=0$ to Figure \ref{figahc}, such that:
\begin{subequations}
\begin{align}
i{\tau}^{\text{S}}_\Earth&:=\text{Figure \ref{figahc}}
=i\cG\left.\tilde{{\omega}}^{\text{S}}_\bcdot(\pb)\right|_{\text{Br}}
\hspace{.1em}\bar{u}(p')\gamma^\bcdot u(p)\label{tauSE0},
%
\intertext{yielding due to the Gordon identity that:}
%
\text{(\ref{tauSE0})}&=
\cG\hspace{.1em}\tilde{{\omega}}^{\text{S}}_\bcdot(\pb)\hspace{.1em}
\bar{u}(p')\(
\frac{p^\bcdot+p'^{\bcdot}}{2m^{~}_e}+
\frac{i\sigma^{\bcdot\star}\(p-p'\)_\star}{2m^{~}_e}\)u(p),\notag\\
&\simeq
i\cG\hspace{.2em}
\bar{u}(p')\(
E^{~}_{\text{Br}}\frac{\tilde{{\omega}}^{\text{S}}_E(\pb)|_{\text{Br}}}{m^{~}_e}-iq^{~}_e
\frac{
-\tilde{{\omega}}^{\text{S}}_E(\pb)|_{\text{Br}}\sigma^{03}+
\tilde{{\omega}}^{\text{S}}_x(\pb)|_{\text{Br}}\sigma^{13}}{2m^{~}_e}\)u(p),\notag\\
&\simeq i\cGR\(
m^{~}_e\hspace{.2em}\tilde{{\omega}}^{\text{S}}_E(\pb)|_{\text{Br}}\hspace{.2em}\(2+\frac{15}{16}\frac{q^{2}_e}{m^{2}_e}\)
\bm\xi'^\dagger\bm\xi-
q^{~}_e
\hspace{.2em}\tilde{{\omega}}^{\text{S}}_x(\pb)|_{\text{Br}}\hspace{.1em}\(2+\frac{q^{2}_e}{m^{2}_e}\)
\bm\xi'^\dagger\frac{\sigma^2}{2}\bm\xi\),\label{tauSE1}
\end{align}
\end{subequations}
up to $\mathcal{O}({q^{~}_e}/{m^{~}_e})^2$.
We note that $\bar{u}(p')\sigma_{~}^{03}{u}(p)=0$ in our Breit system.
Here, we exploit the Chiral representation for the gamma matrices and obtain a nonrelativistic Dirac spinor using a two-component spinor $\bm\xi$ as
\begin{align*}
u(p)&\simeq\(
\begin{array}{c}
\sqrt{\eta^{~}_{\bcdots}\hspace{.1em}p^\bcdot\sigma^\bcdot}\hspace{.2em}\bm{\xi}\\
\sqrt{\eta^{~}_{\bcdots}\hspace{.1em}p^\bcdot\bar\sigma^\bcdot}\hspace{.2em}\bm{\xi}
\end{array}
\),
%
\intertext{where}
%
\sigma^a&:=\(I_2,\sigma^1,\sigma^2,\sigma^3\)~~\text{and}~~\bar\sigma^a:=\(I_2,-\sigma^1,-\sigma^2,-\sigma^3\),
\end{align*}
and $I_2$ is a ($\TxT$) unit-matrix.
We expand $u\(\left.p^a\right|_{\text{Br}}\)$ concerning ${q^{~}_e}/{m^{~}_e}$ up to $\mathcal{O}({q^{~}_e}/{m^{~}_e})^2$, since $\tilde{{\omega}}^{\text{S}}_\bullet|_{\text{Br}}$ has a $\pb^{-1}$ term. 
Two-component spinor $\bm{\xi}$ is normalised as $\sum_\lambda{\bm\xi}^\lambda\cdot{\bm\xi}^{\lambda\hspace{.1em}\dagger}=\bm{1}$ owing to (\ref{spnorm}).

When an electron is at rest in the inertial frame with $q^{~}_e=0$, we obtain  
\begin{align}
&iq^{~}_e
\bar{u}(0)\frac{
\tilde{{\omega}}^{\text{S}}_E(0)|_{\text{Br}}\sigma^{03}-
\tilde{{\omega}}^{\text{S}}_x(0)|_{\text{Br}}\sigma^{13}}{2m^{~}_e}u(0)=
-2\frac{\Rs}{\hbar}\(\bm\xi'^\dagger\frac{\sigma^2}{2}\bm\xi\),\notag
%
\intertext{yielding}
%
&\left.i{\tau}^{\text{S}}_\Earth\right|_{\pb=0}\Rightarrow-i(2m^{~}_e)
\left[
-\frac{2\cG}{m^{~}_e}
\hbar\(\bm\xi'^\dagger\frac{\sigma^2}{2}\bm\xi\)
\right]
\BG,\label{tauSE}
%
~~\intertext{with}~~
%
&\BG:=-\frac{\Rs}{2\hbar^2}=-\frac{G^{~}_{\hspace{-.1em}\text{N}}M_\Earth}{\hbar^2},
\notag
\end{align}
from (\ref{tauSE1}) omitting a $\bm\xi'^\dagger\bm\xi$ term, which does not couple to an electron spin.
Here, $\BG$ is a gravitomagnetoc field in the momentum space generated by the Earth along the negative $z$-axis.
$G^{~}_{\hspace{-.1em}\text{N}}=\kE/8\pi$ is the Newtonian constant of gravitation.

% Alternative derivation of scattering amplitude
\paragraph{Alternative derivation of scattering amplitude:}
We obtained the scattering amplitude owing to the classical Schwarzschild field using the simplified Feynman rule (\ref{Vwpsi}), which is derived under the constraint (\ref{wpolvec}).
This result is critical to determine the gravitational coupling constant; thus, we confirm the derivation from an original vertex term without the constraint.
We give the scattering amplitude as
%\begin{subequations}
\begin{align}
{\tau}^{\text{S}}_\Earth=
\bar{u}^{\lambda'}\hspace{-.2em}\hspace{-.2em}(p')\hspace{.1em}V^{\textrm{S}}\hspace{.1em}u^{\lambda}_{~}\hspace{-.1em}(p),
~~&\text{with}~~~
V^{\textrm{S}}:=-i\frac{\cG}{2}
\tilde{\omega}^{\text{S}\hspace{.2em}\stars}_{\hspace{.4em}\bcdot}\gamma^\bcdot\frac{\sigma_\stars}{2},
\label{tauS}
\end{align}
using (\ref{Vwpsi0}).
%Hereafter, we utilise only the Breit frame and omit $\bullet|_{\text{Br}}$ for simplicity.
We insert an identity\footnote{
When $p$ is an on-shell momentum, ${\eta_{\diamonds}^{~}\gamma^\diamond p^\diamond+m^{~}_e}$ does not have an inverse.
We consider $p$ to be slightly off-shell and perform the following calculations.
In the end, we put it back to unity.
}
\begin{align*}
1&=\frac{\sum_{\lambda''}u^{\lambda''}\hspace{-.2em}(p)\hspace{.2em}\bar{u}^{\lambda''}\hspace{-.2em}(p)}
{\eta_{\diamonds}^{~}\gamma^\diamond p^\diamond+m^{~}_e}
\end{align*}
into (\ref{tauS}), yielding
\begin{align}
\text{(\ref{tauS})}&=-i\frac{\cG}{2}
\tilde{\omega}^{\text{S}\hspace{.2em}\stars}_{\hspace{.4em}\bcdot}
\sum_{\lambda''}\bar{u}^{\lambda'}\hspace{-.2em}(p')\gamma^\bcdot
\frac{u^{\lambda''}\hspace{-.2em}(p)\hspace{.2em}\bar{u}^{\lambda''}\hspace{-.2em}(p)}
{\eta_{\diamonds}^{~}\gamma^\diamond p^\diamond+m^{~}_e}
\frac{\sigma_\stars}{2}{u}^\lambda(p).\label{Vwpsi2}
\end{align}
We look at an electron current in (\ref{Vwpsi2}) and use the Gordon identity again, yielding
\begin{align}
\bar{u}^\lambda(p')\gamma^au^{\lambda''}\hspace{-.2em}(p)&=
\bar{u}^\lambda(p')\(\frac{p^a+p'^a}{2m^{~}_e}+
\frac{i\sigma^{a\bcdot} (p-p')^\bcdot\eta_\bcdots}{2m^{~}_e}
\)u^{\lambda''}\hspace{-.2em}(p).\label{Gordon}
\end{align}
The second term is in $\mathcal{O}(q^{~}_e/m^{~}_e)$ and negligible after multiplying the overall factor in $\mathcal{O}(q^{~}_e/m^{~}_e)$.
Thus, we have
\begin{align*}
\text{(\ref{Gordon})}&\simeq\frac{1}{2m^{~}_e}\bar{u}^\lambda(p')\(p^a+p'^a\)u^{\lambda''}\hspace{-.2em}(p)
\simeq
\bar{u}^{\lambda}(p')\hspace{.1em}u^{\lambda''}\hspace{-.2em}(p)
\end{align*}
We put this relation back into the interaction vertex and rearrange order of a polarisation sum again, yielding that
\begin{align}
\text{(\ref{Vwpsi2})}&\simeq
{\tau}^{\text{S}}_\Earth\simeq-i\frac{\cG}{2}\hspace{.1em}
\tilde{\bm{\omega}}^{\text{S}}_{\hspace{.2em}\bcdot}\cdot\bm{S}.\label{tauS2}
%
\intertext{where}
%
\tilde{\bm{\omega}}^{\text{S}}_{\hspace{.2em}a}\cdot\bm{S}&:=
\tilde{\omega}^{\text{S}\hspace{.2em}\bcdots}_{\hspace{.2em}a}S_\bcdots~~\text{with}~~~
S_{ab}:=\bar{u}^{\lambda'}\hspace{-.2em}(p')\frac{\sigma^{~}_{ab}}{2}u^{\lambda}(p).\notag
\end{align}
Owing to the normalised Schwarzschild spin-connection,  we write the amplitude as 
\begin{align}
\text{(\ref{tauS2})}&=
-i\frac{\cG}{2}\(\tilde{{\omega}}^{\text{S}}_E(\pb)\hspace{.1em}S_{30}
-\tilde{{\omega}}^{\text{S}}_x(\pb)\hspace{.1em}S_{31}
-\tilde{{\omega}}^{\text{S}}_y(\pb)\hspace{.1em}S_{32}
\).\label{vtx}
\end{align}
In this representation with the forward scattering approximation, we obtain an electron current as
\begin{align*}
S^{~}_{30}&\simeq\hspace{.8em}\bm\xi'^\dagger\(p^{~}_x\sigma^2-p^{~}_y\sigma^1\)\bm\xi,\\
S^{~}_{31}&\simeq\hspace{.8em}{m^{~}_e}\hspace{.1em}\bm\xi'^\dagger\sigma^2\bm\xi,\\
S^{~}_{32}&\simeq-{m^{~}_e}\hspace{.1em}\bm\xi'^\dagger\sigma^1\bm\xi.
\end{align*}
Thus, the Schwarzschild spin-connection in the Breit frame provides the same amplitude as (\ref{tauSE}).

% Gravitomagnetic-moment
\paragraph{Gravitomagnetic-moment:}
We define a gravitomagnetic-moment vector of an electron as
\begin{align*}
\bmuG:=\frac{\cG\hspace{.1em}\gG}{m^{~}_e}\vec{S}
~~~\text{with}~ ~~\vec{S}=\hbar\hspace{.2em}\bm\xi'^\dagger\frac{\vec{\sigma}}{2}\hspace{.1em}\bm\xi,
\end{align*}
where $\vec{S}$ is a spin angular-momentum vector and $\gG$ is a \emph{gravitomagnetic $g$-factor}.
We put constants $\cG$ and $\gG$ in the definition by analogy of the magnetic moment of an electron.
A gravitomagnetic potential of an electron in a gravitational field $\bmBG$ is 
\begin{align}
\VG(r)&=-\bmuG\cdot\bmBG=-\frac{\cG\hspace{.1em}\gG}{m^{~}_e}\vec{S}\cdot\bmBG.\label{Hgr}
\end{align}

We note similarities between electromagnetic and gravitomagnetic interactions of an electron.
An electromagnetic field is not a Lorentz vector but a component of an electromagnetic tensor.
However, we can treat an electromagnetic field as a three-dimensional vector.
Similarly, a gravitational field is not a vector but a tensor field in general relativity.
However,  we can treat it as a vector field obtained using the \emph{gravitational vector potential} such as $\omega^a_{~}:=\omega_\bcdot^{\hspace{.4em}\bcdot a}$. 
In an electromagnetic field, a parallel spin electron to an electromagnetic field has a smaller potential energy than that of an anti-parallel one.
Therefore, a small perturbation makes an electron spin-flip from the anti-parallel to parallel by emitting a photon with gap energy of two states.
Similarly, an up-spin electron concerning the Earth's gravitational field has smaller potential energy than the down-spin one owing to the potential (\ref{Hgr}).
Thus, a small perturbation (a small energy transfer from the Earth) may rotate an electron spin to an upper direction in a coordinate frame fixed on the Earth's surface.


Amplitude ${\tau}^{\text{S}}_\Earth$ may give the Born approximation of the scattering of an electron with the potential due to the Earth's gravity:
 \begin{align*}
 i\bm\tau^{~}_{\text{Born}}=-i\(2m^{~}_e\)\bm\xi'^\dagger\tilde{V}(\pb)\bm\xi\xrightarrow{~\text{Figure \ref{figahc}}~}
 i{\tau}^{\text{S}}_\Earth=-i(2m^{~}_e)\bm\xi'^\dagger\widetilde\VG(\pb)\bm\xi.
\end{align*}
By comparing (\ref{tauSE}) with (\ref{Hgr}), we obtain that
\begin{align*}
\widetilde\VG(0)&=-\frac{\cG\hspace{.1em}\gG}{m^{~}_e}{S^y_{~}}\BG~~\text{with}~~
\gG=2,
\end{align*}
in our Breit frame.

A coefficient of an angular momentum operator $S^2=\bm\xi'^\dagger(\sigma^2/2)\hspace{.1em}\bm\xi$ gives a rotation angle of an electron spin; thus, the amplitude (\ref{tauSE}) generates a counter-clockwise spin rotation around a $y$-axis such that an initial horizontal (along a $x$-axis) spin vector to a $z$-axis (upward).
The gravitational force $\BG$ induce an orbital motion with the orbital angular frequency $\omega_O^2=|\BG|$.
On the other hand, an angular frequency of a spin rotation $\omega^{~}_S$ is provided using a relation between a rotation energy of an angular momentum and  an angular frequency, such that:
\begin{align*}
\frac{1}{2}|\bmuG|\omega^{2}_S&={\cG}\gG\left|\BG\right|\rightarrow\omega^{}_{S}=\pm\omega^{~}_{O}.
\end{align*}

In a clockwise orbital motion of an electron, it provides a $2\pi$ counter-clockwise spin rotation to a spin in the inertial system during one cycle.  
On the other hand, the electron spin point in the same direction during orbital motion in the global system, which is fixed on the Earth.
The spin rotation in the inertial system is spurious due to the observation of the rotational coordinate system.
The first-order approximation under the weak gravitational field provides $\gG=2$ to induce no spin precession.
We note in the above result is owing to the normalisation (\ref{fw}) yielding
\begin{align*}
\lim_{q^{~}_e\rightarrow0}q^{~}_e\hspace{.1em}{\tilde\omega}^{\text{S}}_x(\pb)=\frac{\Rs}{\hbar}.
\end{align*}
This first order result is consistent with a classical spin precession of a gyroscope in the weak Schwarzschild gravitational field owing to the Earth\cite{hoyng2007relativistic}. 
It is known that the higher order correction induces a geodesic precession\cite{hoyng2007relativistic}.


For the finite momentum transfer case with $\pb\neq0$, the second term in (\ref{tauSE1}) proportional to $\bm\xi'^\dagger(\sigma^2/2)\hspace{.1em}\bm\xi$ provides a gravitomagnetic interaction of the Earth's gravity to an electron spin, which gives a gravitomagnetic moment slightly differ from $\gG=2$ owing to the gravitomagnetic effect.
More precisely, this term generates a counterclockwise spin rotation around a $y$-axis such that an initial horizontal (along a $x$-axis) spin vector to a negative $z$-axis (upward).
We obtain an anomalous gravitomagnetic moment as 
\begin{align}
\aG:=
\frac{\gG-2}{2}&=\frac{1}{2}\(\gG|_{\pb\neq0}-\gG|_{\pb=0}\)\simeq
-4\cG\hspace{.2em}{\pb}\hspace{.1em}\gamma^{~}_e\(\(\log{\pb}+\gamma^{~}_E\)\beta^{~}_e
-\frac{\pi}{4}
\).\label{gg-2}
\end{align}
The anomalous magnetic-moment measurements may include an effect of the above contribution.

% Experimental measurements
\paragraph{Experimental measurements:}
In the electron anomalous magnetic-moment measurements\cite{PhysRevLett.100.120801,Fan:2022eto}, an electron has a small Lorentz factor $\gamma^{~}_e$ and $\gamma^{~}_e\beta^{~}_e$.
Moreover, these experiments utilise free-falling electrons with $\pb\simeq0$.
Thus, the spin precession owing to the gravitomagnetic effect is negligible compared with the magnetic one. 

On the other hand, the BNL--FNAL type $\gmt$ measurement used the \emph{magic momentum}\cite{Muong-2:2006rrc,Muong-2:2021ojo} of $p=4.094$GeV  corresponding to Lorentz factors $\gamma=29.4$ and $\beta=0.999421$ to eliminate the spin precession due to the focusing electric field; thus, we can expect a sizable precession owing to the gravitomagnetic moment with the Earth's gravity.
The electrostatic quadrupole (ESQ) covers $13/30$ of the muon storage ring\cite{Muong-2:2015xgu}.
Muons in the storage ring are kept horizontally due to the electric field of ESQ on average. 
Therefore, we estimate each muon receives a momentum transfer of $\pb^{~}_\Earth{\simeq}\kappa M^{~}_\Earth/r^{~}_\Earth{\simeq}6.95\times10^{-10}$ on average, which induces the spin precession owing to (\ref{gg-2}), such that:
\begin{align*}
\aG/\cG&=173\hspace{.2em}954\times10^{-11}.
%
\intertext{On the other hand, the muon anomalous magnetic moment is estimated theoretically\cite{Aoyama:2020ynm} as}
%
\aSM&:=\frac{\gSM-2}{2}=
116\hspace{.2em}591\hspace{.2em}810\times10^{-11}.
\end{align*}
A total precesstion effect is estimated as
\begin{align*}
a^{2}_{T}&:=\aSM^2+\(\frac{\aG}{\cG}\)^2+2\frac{\aSM\hspace{.1em}\aG}{\cG}\cos{\theta_a},
\end{align*}
where $\theta_a$ is a angle between two precession axes.
The precession owing to the anomalous grativomagnetic moment is around the $y$-axis and that owing to the anomalous magnetic moment is around the $z$-axis in the lab-frame.
Consequently, $\theta_a=\pi/2$ and the grativomagnetic contribution to the anomalous magnetic moment measurements is
\begin{align*}
\delta\aG&:=\(\left.a^{~}_{T}-\aSM\)\right|_{\cG=1}=1.3\times10^{-9}.
\end{align*}
The measured muon anomalous magnetic moment is reported\cite{Muong-2:2021ojo} as
\begin{align*}
\aExp&=116\hspace{.2em}592\hspace{.2em}061\times10^{-11},
%
\intertext{yielding}
%
\delta{a}&:=\aExp-\aSM=\(2.8\pm0.8\)\times10^{-9},
\end{align*}
which suggests the gravitational coupling constant is consistent with unity.
\end{appendix}
% ------------------------------------------------------------------------
\bibliographystyle{bmc-mathphys} % Style BST file (bmc-mathphys, vancouver, spbasic).
\bibliography{QGED}

\end{document}
%
%
% back up note
%
%
We need appropriate normalisation for the spin-connection. 
In the QED, the coulomb potential provides the external field after the Fourier transformation as
\begin{align*} 
V_c(r)=A^0=\frac{e}{4\pi}\frac{1}{r}&\xrightarrow{~~\text{F.T.}~~}\tilde{V}_c(p)=\hbar^2\frac{e}{p^2},
\end{align*}
where $e$ is a dimensionless electric charge.
Similarly, we require that the Newtonian gravitational potential due to the Earth provides the time component of the spin-connect, such as
\begin{align*} 
V_{\text{N}}(r)=\frac{\kE M^{~}_\Earth}{8\pi}\frac{1}{r}
&\xrightarrow{~~\text{F.T.}~~}\tilde{V}_{\text{N}}(p)=\frac{\hbar^2}{2}\frac{\kE M^{~}_\Earth}{p^2}.
\end{align*}
 Consequently, we obtain the normalisation constant of the spin connection that
\begin{align}
f_\omega=\(\frac{\pi\hbar\Rs}{4}\)^{-1},\label{fw}
\end{align}
 


Physical dimensions of fundamental constants as well as fields appearing in this study are summarised in advance below:
%%%%%%%%%%%%%%%%%%%%%%%%%%%%%%%%%%%%%%%%%%%%%%%%%%%%%%%
	\begin{table}[t]
		\begin{center}
			\begin{tabular}{|cc||cc|}
\hline
constant/field & dim. &constant/field & dim. \\
\hline
%%
$\kE$ & $L/M$ & $\hbar$ & $ML$ \\
%
${\lp}^{\hspace{-.2em}2}$ & $L^2=T^2$ & $\hbar/\kE$ & $E^2=M^2$ \\
%
$\E$ & \it{1} & $\omega$ & $L^{-1}$\\
%
$\psi$ & $L^{-3/2}$ & $\Aa$ & $L^{-1}$ \\ 
%
$\Lambda_c$ & $L^{-2}$ & $m_e$ & $M$\\
\hline
%%
			\end{tabular}
\caption{Physical dimensions of natural constants and fields are summarised in this table. Definitions of fields are given later in this study. Here, ``{\it 1}'' denotes a null physical-dimension.}
		\end{center}
	\end{table}
%%%%%%%%%%% appendix C %%%%%%%%%%%
\section{Renormalisation constants of QED and QGED}\label{app2}
\subsection{QED}
Space-time dimension is set to $D=4-2\epsilon_{UV}$  ($\epsilon_{UV}>0$) and an ultra-violet divergence is encapsulated in $C_{UV}:=1/\epsilon_{UV}-\gamma_E+\log{4\pi}$, where $\gamma_E=0.57721\cdots$ is the Euler's constant.
For regularisation for an infra-red divergence, space-time dimension is taken to $D=4+2\epsilon_{I\hspace{-.2em}R}$  ($\epsilon_{I\hspace{-.2em}R}>0$) and an infra-red divergence is encapsulated in $C_{I\hspace{-.2em}R}:=1/\epsilon_{I\hspace{-.2em}R}-\gamma_E+\log{(4\pi\mu^2_{\hspace{-.1em}R}/m_e^2})$, where $\mu_{\hspace{-.1em}R}$ is an arbitrary energy scale where an infra-red divergence is separated.
Renormalisation constants for a pure QED is summarised as follows;
\begin{subequations}
\begin{align}
Z_2^{\hlf}&=1-\frac{1}{2}\frac{\alpha}{4\pi}
\left(C_{UV}+4-\log{m_e^2}+C_{I\hspace{-.2em}R}\right),\\
Z_3^{\hlf}&=1-\frac{1}{2}\frac{\alpha}{4\pi}
\left(-\frac{2}{3}+\frac{4}{3}\left(C_{UV}-\log{m_e^2}\right)\right),\\
Y&=1-\frac{\alpha}{4\pi}
\left(\frac{1}{3}-\frac{2}{3}\left(C_{UV}-\log{m_e^2}\right)\right),\\
\delta m&=\frac{\alpha}{4\pi}m_e\left(-4-3C_{UV}+3\log{m_e^2}\right).
\end{align}
\end{subequations}

%QGED
\subsection{QGED}



%%%%%%%%%%%%%%%%%%%%%%%%%%%%%%%%%%%
% Quantum effects of gravitoelectromagnetic processes %
%%%%%%%%%%%%%%%%%%%%%%%%%%%%%%%%%%%
\section{Quantum effects of GravitoElectroMagnetic processes}
%
% Hawking radiation
%
\subsection{Hawking radiation}

%
%Gravitational bremsstrahlung
%
\subsection{Gravitational bremsstrahlung}

%
%Photon radiation
%
\subsubsection{Photon radiation}

%
%Graviton radiation
%
\subsubsection{Graviton radiation}

