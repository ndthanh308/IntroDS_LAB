% % % % % % % % % % % % % % % % % % % % % % % % % % %
% IS&T Template 
% Patrick Vandewalle
% January 2006
% % % % % % % % % % % % % % % % % % % % % % % % % % %

%%%%%%%%%%%%%%%%%%%%%%%%%%%%%%%%%%
% Document class
%%%%%%%%%%%%%%%%%%%%%%%%%%%%%%%%%%
\documentclass[letterpaper,twocolumn,fleqn]{article} 

%%%%%%%%%%%%%%%%%%%%%%%%%%%%%%%%%%
% Packages
%%%%%%%%%%%%%%%%%%%%%%%%%%%%%%%%%%
\usepackage{ist}
\usepackage{multirow}
% add other packages here
\usepackage{mwe}
\usepackage{subcaption}
\usepackage{subfiles}
\usepackage{times}
\usepackage{epsfig}
\usepackage{graphicx}
\usepackage{amsmath}
\usepackage{amssymb}
\usepackage{cleveref}

\pagestyle{empty}                % no page numbers is default
\usepackage[style=ieee]{biblatex}
% \usepackage{viper-bibtex-check}
\addbibresource{ref.bib}

%%%%%%%%%%%%%%%%%%%%%%%%%%%%%%%%%%
% Title and Authors
%%%%%%%%%%%%%%%%%%%%%%%%%%%%%%%%%%
\title{A Visual Quality Assessment Method for Raster Images in Scanned Document}
\author{Justin Yang$^{*}$, Peter Bauer$^{\dagger}$, Todd Harris$^{\dagger}$, Changhyung Lee$^{\ddagger}$, Hyeon Seok Seo$^{\ddagger}$, Jan P Allebach$^{*}$, Fengqing Zhu$^{*}$;\\
$^{*}$Elmore School of Electrical and Computer Engineering, Purdue University, West Lafayette, Indiana, USA\\ $^{\dagger}$HP Inc., Boise, Idaho, USA\\ $^{\ddagger}$HP Printing Korea Co Ltd, Suwon, Korea}

\date{} % date has an empty field.

% correct for bad hyphenation here
\hyphenation{}

%%%%%%%%%%%%%%%%%%%%%%%%%%%%%%%%%%
% Begin document
%%%%%%%%%%%%%%%%%%%%%%%%%%%%%%%%%%
\begin{document} 

\maketitle 

\thispagestyle{empty} % prevents the first page to be numbered

%%%%%%%%%%%%%%%%%%%%%%%%%%%%%%%%%%
% Abstract
%%%%%%%%%%%%%%%%%%%%%%%%%%%%%%%%%%

\begin{abstract}
\label{sec:abstract}
Image quality assessment (IQA) is an active research area in the field of image processing. Most prior works focus on visual quality of natural images captured by cameras. In this paper, we explore visual quality of scanned documents, focusing on raster image areas. Different from many existing works which aim to estimate a visual quality score, we propose a machine learning based classification method to determine whether the visual quality of a scanned raster image at a given resolution setting is acceptable. We conduct a psychophysical study to determine the acceptability at different image resolutions based on human subject ratings and use them as the ground truth to train our machine learning model. However, this dataset is unbalanced as most images were rated as visually acceptable. To address the data imbalance problem, we introduce several noise models to simulate the degradation of image quality during the scanning process. Our results show that by including augmented data in training, we can significantly improve the performance of the classifier to determine whether the visual quality of raster images in a scanned document is acceptable or not for a given resolution setting.
\end{abstract}


%%%%%%%%%%%%%%%%%%%%%%%%%%%%%%%%%%%%
% Overall Document Guidelines: Head
%%%%%%%%%%%%%%%%%%%%%%%%%%%%%%%%%%%%
\section{Introduction}
\label{sec:intro}
\subfile{sections/01_introduction.tex}

% \section{Related Works}
% \label{sec:related}
% \subfile{sections/02_Related_Works.tex}



\section{Dataset Collection}
\label{sec:datacollection}
\subfile{sections/03_Dataset_Collection.tex}
\vspace{-0.1cm}
\section{Psychophysical Image Quality Assessment Experiment}
\label{sec:IQA}
\subfile{sections/04_IQA_Experiment.tex}
\vspace{-0.1cm}
\section{Method}
\label{sec:method}
\subfile{sections/05_Methods.tex}

\section{Data Analysis and Augmentation}
\label{sec:dataanaysis}
\subfile{sections/06_Data_Analysis_Augmentation.tex}

% \section{Dataset Augmentation}
% \label{sec:dataaug}
% \subfile{sections/07_Data_Augmentation}

\section{Experimental Results}
\label{sec:result}
\subfile{sections/07_Results.tex}
\vspace{-0.3cm}
\section{Conclusion}
\label{sec:conclusion}
\subfile{sections/08_Conclusion.tex}
\printbibliography


%%%%%%%%%%%%%%%%%%%%%%%%%%%%%%%%%%
% Biography
%%%%%%%%%%%%%%%%%%%%%%%%%%%%%%%%%%

% \begin{biography}
% Please submit a brief biographical sketch of no more than 75 words. 
% Include relevant professional and educational information as shown 
% in the example below.

% Jane Doe received her BS in physics from the University of Nevada (1977) 
% and her PhD in applied physics from Columbia University (1983). Since 
% then she has worked in the Research and Technology Division at Xerox 
% in Webster, NY. Her work has focused on the development of toner adhesion 
% and transport issues. She is on the Board of  IS\&T and a member of APS 
% and SPIE.
% \end{biography}

\end{document} 
