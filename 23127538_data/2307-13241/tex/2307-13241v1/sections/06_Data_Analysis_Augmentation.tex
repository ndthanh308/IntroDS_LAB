
\subsection{Noise Models}
Table \ref{tab:data_summary} summarizes the participant ratings from the psychophysical experiment. Because the quality scores resulted from our dataset is significantly unbalanced, \textit{i.e.}, more images are rated at ``Acceptable" compared to ``Unacceptable", we introduce several noise models to simulate the quality degradation during scanning process in order to mitigate the data imbalance issue. 

\begin{table}[t]
\caption{Subjective Rating Summary}
    \centering
    \begin{tabular}{|c|c|c|} 
        \hline 
        Resolution (dpi)& \shortstack{Data Rated as \\ Acceptable}& \shortstack{Data Rated as \\ Unacceptable}\\ \hline
        300&174&0\\ \hline
        200&173&1\\ \hline
        150&154&20\\ \hline
        100&76&98\\ \hline
    \end{tabular}
\label{tab:data_summary}
\end{table}


We adopt several noise models to augment visually unacceptable images to make the dataset more balanced, including Gaussian noise, salt and pepper noise, and generative adversarial networks. The parameter settings for the former two noise models are determined based on the Structural Similarity Index (SSIM) statistics of the images rated as ``Unacceptable", while the latter is trained using a subset of the images rated as ``Unacceptable". 

% Figure environment removed

\textbf{Gaussian Noise.}
As we discovered that Gaussian noise can simulate the visual quality degradation effect of the scanned documents during the scanning process, we decided to use it as one of our augmentation methods. The variance of the Gaussian noise applied to images is set to $0.0005$ as it matches the mean SSIM value of the ``Unacceptable" images in our dataset. %Example images with and without Gaussian noise can be found in Figure \ref{fig:figure_gaussian}.

% % Figure environment removed

\textbf{Salt and Pepper Noise.}
Impulse noise may be introduced during the document scanning process, which can be used simulate visual quality degradation. Therefore, We use salt and pepper noise to augment our dataset. The noise density of the salt and pepper noise applied to the images is set to $0.008$ in order to match the mean SSIM value of the ``Unacceptable" images in our dataset.% Example images with and without salt and pepper noise can be found in Figure \ref{fig:figure_salt}.
% % Figure environment removed

\textbf{Generative Adversarial Network (GAN).}
Generative adversarial Networks, or GANs, are popular methods for image generation and synthesis \cite{NIPS2014_5ca3e9b1}. 
% A GAN consists of two main parts: the generator $G$ and the discriminator $D$. The main idea to train a GAN is to let the generator and discriminator play a mini-max game where the generator tries to synthesize a realistic looking image, while the discriminator tries to distinguish the real image from the synthesized one. %The loss function is described as below where $p_{data}(x)$ serves as the real data distribution while the $p_z(z)$ represents a random noise distribution.
% \begin{equation}
%     \begin{array}{r}
%         \min _{G} \max _{D} V(G, D)=E_{x \sim P_{d a t a}(x)}[\log D(x)] \\
%         +E_{z \sim P_{z}(z)}[\log (1-D(G(z)))]
%     \end{array}
% \end{equation}
% where $G$ is the generator while $D$ is the discriminator. The generator maps a random noise $z$ into a the real data $X$ and results in the generated data $G(z)$. $D$ is a classifier that takes both the real data $X$ and the generated $G(z)$ to determine if $G(z)$ is real or not. When the sample comes from $G(z)$, $D$ would minimize the loss. When the sample comes from $X$, $D$ would maximize the loss.
For our noise model, we formulate the task of generating low quality images into an image-to-image translation task, where we utilize conditional GANs \cite{NIPS2014_5ca3e9b1} to generate low quality versions of the given images. The loss function is slightly different from the original GAN's loss function as we also take the base resolution image $X$ as input for our network. The loss function is described as below.
\begin{equation}
    \begin{array}{r}
        \min _{G} \max _{D} V(G, D)=E_{x \sim P_{data}(x)}[\log D(y)] \\
        +E_{z \sim P_{z}(z)}[\log (1-D(G(x,z)))]
    \end{array}
\end{equation}
where $G$ is the generator and $D$ is the discriminator. The generator takes the base-res image $x$ and a random noise $z$ as input and generates the low-res data $G(x, z)$. $D$ is a classifier that takes both the real low-res data $y$ and the generated $G(x, z)$ as inputs to determine if $G(x, z)$ is real or not. When the sample comes from $G(x, z)$, $D$ would minimize the loss. When the sample comes from $y$, $D$ would maximize the loss.
% % Figure environment removed
%An example original image and a degraded version of that image generated with a conditional GAN can be found in Figure \ref{fig:figure_gan}.

\subsection{Analysis on Augmented data}
To determine the noise level used in the data augmentation for each noise model, we derive statistics from the images that are labeled as ``Unacceptable" from the psychophysical experiment. In particular, we calculate the mean and standard deviation of the SSIM (Structure Similarity Index Measure) between the original images (with acceptable quality) and their corresponding low resolution images (with unacceptable quality), since SSIM is partially related to image content. Table \ref{tab:data_iqa_summary} shows the mean and standard deviation of SSIM values for the participants ratings are equal to 0.63 and 0.06, respectively. Therefore, we use an SSIM value around 0.63 as a criterion to determine how much noise we should apply for each noise model. The generated noisy images are considered as the images with unacceptable quality and are included in the training of the classifier.

% As we went through the augmentation methods in the previous sections, we will discuss how the noise level is determined in this section. From the previous Figure \ref{fig:figure_gaussian} and Figure \ref{fig:figure_salt}, we know that the amount of noise, added into the original image for generating the image with unacceptable quality, depends on the content of the image. Therefore, SSIM (Structure Similarity Index Measure) is used for analysis since SSIM is partially related to image content.

\begin{table}[t]
    \caption{SSIM Statistics of Unacceptable and Augmentation Data.}
    \centering
    \begin{tabular}{|c|c|c|} 
        \hline 
        & Mean& Standard Deviation\\ \hline
        Unacceptable (human)&0.63&0.06\\ \hline
        Gaussian Noise (model)&0.59&0.001\\ \hline
        S\&P Noise (model)&0.62&0.002\\ \hline
        GAN-based (model) & 0.738& 0.006\\ \hline
    \end{tabular}

    \label{tab:data_iqa_summary}
\end{table}

% The SSIM values, between the original images (with acceptable quality) and their corresponding low resolutions image (with unacceptable quality), are calculated.  A statistical analysis is done on these SSIM values and the result is as follows. 


% For the SSIM values between original images and noisy images, we also calculate the mean and standard deviation.
% It is known that SSIM can be adjusted to any value if the noise level is controlled properly. However, the standard deviation is harder (or even impossible) to control while fixing the SSIM mean value. From Table \ref{tab:data_iqa_summary}, we observe that the standard deviation for the additive noise model is far less than the one in the original dataset. This allows us to understand that additive noise models alone are not enough to simulate all images of unacceptable quality. Therefore, we also use GAN to generate images of unacceptable quality. The SSIM standard deviation values, between original images and GAN generated images, is closer to the real situation compared to those generated by the  additive noise models.