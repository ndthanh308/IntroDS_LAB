Scanners on mulit-functional printers (MFPs) are widely used for both personal and commercial purposes to digitize printed documents. The scan resolution is usually set by the user or by the default setting in the scanner. This leads to the same output file size regardless of the content of the scanned document. 
% the fact that when scanning, disregard of the image content of the whole page, the final output file size will be the same. 
However, most printed documents do not require scanning all regions in one page at a high resolution setting. This can lead to a unnecessarily large file size for those regions where details are less relevant, such as images that do not contain salient objects or are mostly homogeneous. On the other hand, when scanning at a lower resolution, for image regions where details are important, information could be lost and visual quality degraded. Striking a good balance between visual quality and compression level to optimize user experience is a valuable yet challenging problem since it depends not only on the content of the document but also the purpose of scanning the document. For a scanned document that contains multiple raster image regions on the same document, the best strategy is to scan the raster image regions with different resolutions according to the individual image content in order to maintain desired image quality with minimum file size. 

We are interested in developing methods that would be implemented into MFPs, which uses only the CPU to perform computation. Given this setup, we need to design an efficient image quality assessment method that can balance the performance and complexity suitable for product deployment. Previous works on document image quality assessment focus on Optical Character Recognition (OCR) accuracy \cite{DIQ_DL_Le, DIQ_Kanungo} instead of visual pleasantness. Other works on optimal resolution determination for scanned documents mainly focus on the text regions of a scanned document \cite{litao_IQA,litao_TextQA} such as bar-codes/QR-codes, hand-writings, and printed text. In these methods, the focus is on determining suitable resolution to achieve certain level of accuracy in OCR, and treat OCR accuracy as the metric to determine the visual quality of the document. However, in our scenario, we want to focus on raster image content, which does not have such metric to utilize for quality assessment. Since there is no objective evaluation more suitable than the human eyes, a psychophysical experiment is needed to determine the visual acceptance of raster images in scanned documents.

Due to the limited computation power, we propose to extract a set of features from the scanned document that are relevant to visual quality, require relatively simple computation for CPUs, and then find the appropriate combinations. Under a simple machine learning based framework, we then fuse the information from different features to determine the minimum required scan resolution. 
% The selected metrics/features should require relatively simple computation for CPUs. 
% We propose to combine multiple features to determine the minimum required resolution because each feature contains visual quality information from different aspects. By using machine learning techniques, we are able to fuse the information from different features and then properly combine them to achieve the desired performance.  

The main contributions of this paper are as follows:
\begin{itemize}
    \item We formulate visual quality assessment for document images into full-reference task, as it is different from natural camera image quality assessment where no reference image can be provided.
    \item We propose a simple machine learning method to estimate image quality for different image regions in a scanned document page to determine the optimal scanning resolution for each image region. 
    \item We collect a scanned document dataset containing images scanned at different resolution settings paired with human subjective ratings from a psychophysical experiment.
    \item We address the data imbalance problem of subjective ratings by introducing noise models that simulate the degradation of image quality during the scanning process.  
\end{itemize}
% 1. We have collected a dataset containing pairs of human subjective ratings and image with different resolution settings
% 2. We dealt with the data imbalance issue of subjective ratings by introducing noise model that simulate the degradation of image quality during the scanning process.  
% 3. We proposed a model based on SVM to estimate image quality for different image regions in a single scanned document which can further infer the optimal scanning resolution for each image region. 

% The remainder of the paper is organized as follows. In Section~\ref{sec:related}, we review prior works related to image quality assessment, and optimal resolution determination for scanned documents. In Section~\ref{sec:datacollection}, we introduce the collection of a scanned documents dataset including the dataset preparation and composition. In Section~\ref{sec:IQA}, a detailed description of the psychophysical experiment is provided. Our proposed method to determine the optimal scanned document resolution is described in Section~\ref{sec:method}. Section~\ref{sec:dataanaysis} summarizes the statistics of the psychophysical experiment and introduces three noise models to address the data imbalance problem from the human subject ratings. Evaluation of proposed method and discussion are presented in Section~\ref{sec:result}. Section~\ref{sec:conclusion} presents a conclusion of our work.

