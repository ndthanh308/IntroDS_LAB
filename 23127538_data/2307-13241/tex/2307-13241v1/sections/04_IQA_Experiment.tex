
To determine the minimum resolution for our purpose, we want to minimize the dots per inch (dpi) resolution setting for a multifunction printer (MFP) when scanning a document page. Our goal is to reduce the resolution setting as much as possible without noticeable visual quality degradation, so as to meet the end user's expectations. We design a psychophysical image quality evaluation experiment as the subjective quality metric to compare the visual quality between different dpi settings of the raster images in scanned documents.

The experimental procedure is as follows. In order to keep the experiment in a reasonable time length, we first partition our dataset into five subsets so that each participant only need to rate one subset instead of all images in the dataset. The size of the whole document would be set at 8.5 by 11 inches for display. There were a total of 35 participants with experience in the image processing field.
% which included four different HP engineers with background in image quality assessment, two professors with background in image processing and computer vision, and 29 graduate students 
For each given mixed raster document, the participant can select the raster images one-by-one, and set the resolution of that image areas to one of the following: 100, 150, 200, and 300 dpi, while the remaining raster images on the page, \textit{i.e.}, the text and vector regions, are fixed at 300 dpi, which is the base resolution setting. Participants are asked to rate each raster image individually in the document according to the naturalness, smoothness and detailed rendition of the raster image area of the document. Participants can select from four quality scores following the standard mean opinion score (MOS) setting for quality assessment: A (Visually Pleasant), B (Visually Okay), C (Visually Okay with Some Artifacts), and D (Visually Unacceptable). Each participant rates approximately 30 separate raster images in 15 to 20 pages, where each page contains 1 to 4 raster images. The whole procedure took on average 25 minutes to complete. Each raster image in the scanned document is evaluated by seven different participants, and assigned a quality rating. The final quality rating assigned to each raster images is calculated by averaging the ratings ($A=4$, $B=3$, $C=2$, $D=1$) and then quantize the averaged result to the closest category after removing outliers.

%Figure \ref{fig:GUI_example} shows an example of the GUI application used by the participants to provide quality scores. The window on the left shows the entire document page. The right side of the GUI contains a separate column for each raster image on the page. Clicking the box in the second row highlights the corresponding raster image in the full page display. Then, by clicking the boxes next to each resolution in that column, the selected raster image is re-rendered at that resolution following the procedure described earlier; and the participant can select A, B, C, or D from the drop-down menu to rate that raster image at the given resolution.