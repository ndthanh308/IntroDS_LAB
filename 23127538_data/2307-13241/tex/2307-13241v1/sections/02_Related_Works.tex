\subsection{Image Quality Assessment}
As described in the introduction, the aim of this work is to determine the minimum resolution requirement for scanned documents while maintaining acceptable visual quality. 
% We are interested in image quality assessment methods that use a reference image and assume the reference image has the best image quality. 
% We understand that the final judgement whether or not an image quality is good is done by human.  
Since no proposed IQA (Image Quality Assessment) method can fully replace the subjective evaluation by the human viewer, the goal of all proposed method is trying to match the assessment performed by humans of visual quality of an image.

Considering the number of images currently browsed on the internet every day, it is practically impossible to manually adjust the image processing parameters for each image. Hence, Image Quality Assessment (IQA) has been an active research topic for the past few decades \cite{IQA_Bovik,IQA_Bovik2}. Image Quality Assessment (IQA) evaluates the quality of an image processing system so that it can be optimized. Depending on whether and how reference images are used, IQA can be further divided into three categories: full-reference (FR) image quality assessment \cite{SSIM, Multi_SSIM, VIF, MAD}, reduced-reference (RR) image quality assessment \cite{RRIQA_SSIM, RRIQA1, RREDI, RRIQA_DN}, and no-reference (NR) image quality assessment \cite{NRIQA}. As far as these three image quality assessment methods are concerned, full-reference IQA is relatively mature, but this method requires reference images, which limits its scope of use. Therefore, people also work on finding good methods under reduced-reference and no-reference image quality assessment criteria.

% FR image quality assessment techniques compare a given image to a reference image and predict the perceptual quality of the given image based on an objective score. The most commonly used methods for calculating the objective score include mean square error (MSE) and peak signal-to-noise ratio (PSNR).  MSE and PSNR are used because these two metrics are simple and intuitive. However, MSE and PSNR do not fully capture the complexity of the human vision system. For example, two images with the same MSE or PSNR can show very different visual effects \cite{gonzalez2008digital}. Hence, other evaluation methods have been proposed to address these issues. Among them, commonly used ones include Structural Similarity Index (SSIM) \cite{SSIM}, Multi-scale Structural Similarity \cite{Multi_SSIM}, Visual Information Fidelity (VIF) \cite{VIF}, Most Apparent Distortion (MAD) \cite{MAD}, etc. These methods are proposed to better reflect the human perceptual ratings. 

% For reduced-reference image quality assessment, these methods \cite{RRIQA_SSIM} \cite{RRIQA1} \cite{RREDI} \cite{RRIQA_DN} usually require partial information from the reference image and then compare it with the information extracted from the processed /distorted image. The required information from the reference image can be the statistical parameters of some distribution after proper computation. For example, in the paper \cite{RREDI}, the distribution of the wavelet coefficients from the processed image is fitted to a Gaussian scale mixture distribution, and the corresponding difference of the entropy is used to predict the quality. 

% In some occasions, such as imaging systems and user terminals, it is difficult to obtain reference images.  Only no-reference image quality assessment methods can be used.  No-reference image quality assessment methods estimate image quality of a given image without any information from the reference image \cite{NRIQA}.  Actually, one of our selected features, named ``Edge Density”, belongs to the category of NR image quality assessment methods.


\subsection{Optimal Resolution Determination for Scanned Documents}
%As mentioned in \cite{DIQ_Survey_Ye}, during the generation of a scanned document, there are several stages that might bring in artifacts, and therefore degrade the image quality of scanned documents. These stages ranges from the creation of the printed document, such as noise in printing, to external degradation after the document is created, such as paper aging and stains, to the digitalization process using the scanners. These can all cause document image quality to degrade. Since the raster images from scanned documents are different from photos or videos captured from a natural scene \cite{DIQ_Survey_Ye}, they have different properties as well. In addition,
The purpose for document image quality assessment is different from that of natural images\cite{DIQ_Survey_Ye}. This implies that methods developed to assess visual quality of images captured from a natural scene do not necessarily work on images of scanned document. For example, We used VMAF \cite{VMAF}, which is a video quality assessment model based on human perception, to evaluate the quality of raster images of scanned documents, but the results are inconsistent with subjective evaluation scores. Thus, there is a need to develop methods that can evaluate quality of scanned document for different contents. 

% In this paper, we propose a full-reference image quality assessment model to estimate raster image quality for scanned documents which can further infer the optimal scanning resolution of each image region. 
Previous works on document image quality assessment focus on Optical Character Recognition (OCR) accuracy \cite{DIQ_DL_Le, DIQ_Kanungo} instead of visual pleasantness. Other works on optimal resolution determination for scanned documents mainly focused on the text regions of a scanned document \cite{litao_IQA,litao_TextQA}. These text regions include bar-codes/QR-codes, hand-writings, and printed text. In their case, they mainly focus on developing algorithms to determine suitable resolution to achieve certain level of accuracy in OCR, and treat OCR accuracy as the metric to determine the visual quality of the document. However, in our scenario, we mainly focus on raster image content, which does not have such metric to utilize for quality assessment. Since there is no objective evaluation more suitable than the human eyes, we conducted a psychophysical experiment to determine the visual acceptance of raster images in scanned documents.
% To the best of our knowledge, there were no similar works focusing on raster image quality for scanned documents in terms of resolution determination. 

% Deep learning-based approaches have showed significant improvement in many research areas of image processing/ computer vision in the recent years. However, most of these methods require large computational power and complex calculation to approximate non-linear functions. In our scenario, developed method would be implemented into MFPs, which uses only CPU to perform computation. Given this setup, we need to design an efficient image quality assessment method that can balance the performance and complex suitable for product deployment.


% As for which features are extracted and the accuracy of predictions will be discussed in the later sections.
