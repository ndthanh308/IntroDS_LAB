
\subsection{Composition of Our Dataset}
\label{Composition_Of_Dataset}
% Figure environment removed
We collected a dataset of mixed content scanned documents containing raster images and possibly other elements such as text, barcodes, line art, graphs etc.
We grouped the documents into 4 subsets: Prima Dataset \cite{Prima_Paper},  Mixed Raster Content (MRC) Dataset, Magazine Scans (MS), and High Quality Scans (HQS). The Prima dataset \cite{Prima_Paper} is a public dataset which contains documents, magazines, and technical journals that are scanned at 300 dpi. The MRC subset is an internal dataset from HP Inc. This includes documents with mixed raster content scanned at 300 dpi. For the MS subset, it consists of 300 dpi scans from Time Magazine using the HP OfficeJet MFP X585.  The HQS subset contains high quality document sources from the internet, which includes annual reports from companies and universities. These documents were first printed using an HP LaserJet 500 color MFP M575 and then scanned at 300 dpi using an HP OfficeJet MFP X585. 
Figure \ref{fig:data} shows some examples from each component of our dataset. For our dataset, we use 50 documents from the Prima dataset, which contains a total of 105 raster images; 5 from MRC dataset, which contains 7 raster images; 5 documents from the MS dataset, which contains 5 raster images; and 40 documents from the HQS dataset, which contains 57 raster images. For the Prima dataset, there are 1 to 4 raster images in each page. For the other subsets of our dataset, there are 1 to 2 raster images in each page.

\subsection{Generation of different dpi images}
% Figure environment removed
% In this subsection, we will first describe the overall pipeline of our system. 
Given a mixed content document page scanned at 300 dpi, which we term as the base resolution, our goal is to determine the optimal resolution setting for each individual raster image region in this document. Figure \ref{fig:downsample_diagram} shows an example of this process. Assume the original image consists of 4 regions, namely E, F, G, and H.  Each region may a have different optimal scan resolution setting. In our system, segmentation is first applied to the full scanned document to separate each region (\textit{e.g.}, text region or image region).  We manually label the segmentation map using the LabelMe tool \cite{wada2018labelme}.  

To determine the optimal scan resolution of a given raster image region in a document page, we generate low dpi versions of each region in addition to having the base resolution, which is then assessed by our proposed method (described in Section \ref{sec:method}).
% Then the optimal resolution is assigned based on the raster image content .  
In order to view the down-sampled image on the original scanned document, region-based post-processing is applied to the images to emulate the visual effects of low-dpi images. Two steps are included in this process to generate the low dpi images, the down-sampling phase and the up-sampling phase. Depending on the chosen optimal resolution, if it is other than base resolution, \textit{i.e.,} 300 dpi, box averaging is applied to the original image region. This is called the down-sampling phase. After the down-sampling phase, the decimated image is then interpolated back to the same size as the original image, termed the up-sampling phase which uses the nearest neighbor algorithm. 

