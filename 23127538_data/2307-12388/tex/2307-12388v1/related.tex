\section{Related Work}
This section will introduce the related work from three aspects, regarding the traffic signal control methods, simulation-transfer methods, and uncertainty quantification techniques. 

\subsection{Traffic Signal Control Methods} 
Optimizing traffic signal control to alleviate traffic congestion has been a challenge in the transportation field for a long time. Different approaches have been extensively studied, including rule-based methods~\cite{dion2002rule, chen2020toward} and RL-based methods~\cite{wei2019colight,wei2019presslight,wei2018intellilight} to optimize vehicle travel time or delay. Most of these studies achieved significant improvement compared to pre-designed time control methods. 
While most of the existing RL-based traffic signal control methods do not consider the sim-to-real gap problem, a few recent studies start to tackle the sim-to-real gap by modifying the simulator directly~\cite{muller2021towards, mei2022libsignal}, which requires the parameters of the simulator can be easily modified to perfectly match the real world. Instead of modifying the simulator, this paper takes a different approach by modifying the output of the policies learned in the simulator.

\subsection{Sim-to-real Transfer}
The literature on sim-to-real transfer~\cite{zhao2020sim}  can be generally categorized into three groups. The first group is \textbf{\emph{domain randomization}}~\cite{tobin2019real, andrychowicz2020learning}, which aims to learn policies that are resilient to changes in the environment. Domain randomization is based mostly on the data from simulation and can be useful in scenarios where the target domain is not known or is non-stationary. The second group is \textbf{\emph{domain adaptation}}\cite{tzeng2019deep, han2019learning}, which focuses on tackling the domain distribution shift problem by unifying the source domain and the target domain features. Most of the domain adaptation methods deal with the gap in the perception of robots~\cite{tzeng2015towards,fang2018multi,bousmalis2018using,james2019sim}, whereas in traffic signal control domain, the gap is mainly from the dynamics rather than perception because most TSC methods directly take vectorized representations like lane-level number of vehicles or queue length as observation.

The third group of approaches, known as \textbf{\emph{grounding methods}}, focuses on improving the accuracy of the simulator concerning the real world by correcting for simulator bias. Unlike system identification approaches~\cite{6907423, Cully_2015} that try to learn the precise physical parameters, Grounded Action Transformation (GAT)~\cite{hanna2017grounded} does not require a parameterized simulator that can be modified. It induces the dynamics of the simulator to match the real world with grounded action, which has shown promising results for sim-to-real transfer in robotics.  Our \ours is based on GAT, with novel designs on leveraging uncertainty quantification to enhance action transformation.



% In order to realize the sim-to-real transfer,
% \cite{real-world-robot, inhand} propose to use domain randomization which tries to randomize the simulation to cover the real-world data distribution.
% \cite{domain_confusion, transfer_feature, domain_ada} leverage domain adaptation, which focuses on tackling the domain distribution shift problem by unifying the source domain and the target domain features. Other approaches either focus on system identification to calibrate existing simulators or exploit the simulation experience to improve learning efficiency~\cite{cutler2014,cully2015}. 

% In contrast to those who learn from experience, Grounded Action Transformation (GAT) \cite{gat_aaai} performs all learning in a grounded simulator, it directly optimizes the simulator dynamics within a modified simulator and only uses the physical robot for policy evaluation, eventually shows high feasibility and effectiveness. The above methods are all proposed for specific robotic tasks, while few efforts were made in the TSC domain. Our \ours is the first to solve the TSC sim-to-real problems, it not only benefits from transforming the grounded action but also from quantifying the policy's uncertainty.

% One direction for realizing the sim-to-real is by domain randomization. Instead of carefully modeling all the parameters of the real world, it focuses on highly randomizing the simulation to cover the real-world data distribution \cite{real-world-robot}. \cite{inhand} tries to randomize source domain dynamics by some physical parameters, and \cite{grasp} proposes to translate the randomized simulation domain features and real-world features into the canonical feature, which are proven to be effective, however, sometimes it is hard to select the relevant parameters to randomize, and there exists the risk of creating biased models.

% Another valuable direction is by leveraging the domain adaptation to realize sim-to-real. It aims to unify the source domain features and the target domain ones, to tackle the domain distribution shift problem.  Dividing by different ways of aligning the two domains' feature spaces, there are Discrepancy- based methods to measure the feature distance \cite{domain_confusion, transfer_feature, domain_ada}, reconstruction-based methods intend to find the invariant or shared features between domains by constructing one auxiliary reconstruction task \cite{reconstruct, sim-to-real_survey}.
% % Introduce the domain adaptation and how it would solve the simulation to real-world problems, the current solutions of domain adaption, etc.


\subsection{Uncertainty Quantification}
For current deep learning methods, effectively quantifying the uncertainty would allow one to understand the models' limitations, and artfully adopting the quantification results could benefit the model's acceleration and accuracy. Gaussian Process (GPs) \cite{seeger2004gaussian, houlsby2012collaborative} is a direction of multiple uncertainty-measure research, which has the ability to calculate the prediction variance in a closed form. But GPs are non-parametric methods, which could benefit from the model parameters in NNs. Another line of research employs prior distribution~\cite{xue2019reliable, zou2018subsampled, kingma2015variational} on model parameters and infers the posterior to estimate the uncertainty while training. Evidential Deep Learning (EDL) \cite{sensoy2018evidential} method is one of the representatives, it directly models a Dirichlet posterior from a deterministic neural net and provides reliable uncertainty measurement. MC dropout \cite{gal2017concrete}, and Deep Essembles \cite{lakshminarayanan2017simple} also exploit parametric models, thus, this paper conduct experiment on EDL, MC dropout, and Deep Ensembles to understand the benefits.

% \textcolor{red}{emphazie EDL, MC dropout}
