%%%%%%%%%%%%%%%%%%%%%%%%%%%%%%%%%%%%%%%%%%%%%%%%%%%%%%%%%%%%%%%%%%%%%%%%%%%%%%%%
%2345678901234567890123456789012345678901234567890123456789012345678901234567890
%        1         2         3         4         5         6         7         8

\documentclass[letterpaper, 10 pt, conference]{ieeeconf}  % Comment this line out
                                                          % if you need a4paper
%\documentclass[a4paper, 10pt, conference]{ieeeconf}      % Use this line for a4
                                                          % paper

\IEEEoverridecommandlockouts                              % This command is only
                                                          % needed if you want to
                                                          % use the \thanks command
\overrideIEEEmargins

% \usepackage[bottom=0.5in]{geometry}
% See the \addtolength command later in the file to balance the column lengths
% on the last page of the document



% The following packages can be found on http:\\www.ctan.org
%\usepackage{graphics} % for pdf, bitmapped graphics files
%\usepackage{epsfig} % for postscript graphics files
%\usepackage{mathptmx} % assumes new font selection scheme installed
%\usepackage{times} % assumes new font selection scheme installed
%\usepackage{amsmath} % assumes amsmath package installed
%\usepackage{amssymb}  % assumes amsmath package installed

\usepackage{microtype}
\usepackage{subfigure}
\usepackage{subcaption}
\usepackage{caption}
\usepackage{booktabs} % for professional 
\usepackage{graphicx}
\usepackage{booktabs}
\usepackage{amsmath}
\usepackage{amssymb}
\usepackage{color}
\usepackage{longtable}
\usepackage{etoolbox}
\usepackage[ruled,linesnumbered,lined]{algorithm2e}
\usepackage{algpseudocode} 
\usepackage{hyperref}
\makeatletter
\patchcmd{\@makecaption}
  {\scshape}
  {}
  {}
  {}
\makeatother


\newcommand{\hua}[1]{{\bf\color{blue} [#1]}}
\newcommand{\hao}[1]{{\bf\color{red} [#1]}}

\newcommand{\longchao}[1]{{\bf\color{green} [#1]}}

\newcommand{\datasetFont}{\text}
\newcommand{\ours}{\datasetFont{UGAT}\xspace}

% \urldef{\myurl}\url{https://sumo.dlr.de/docs/Definition_of_Vehicles,_Vehicle_Types,_and_Routes.html}

\title{\LARGE \bf
Uncertainty-aware Grounded Action Transformation \\ towards Sim-to-Real Transfer for Traffic Signal Control
}

%\author{ \parbox{3 in}{\centering Huibert Kwakernaak*
%         \thanks{*Use the $\backslash$thanks command to put information here}\\
%         Faculty of Electrical Engineering, Mathematics and Computer Science\\
%         University of Twente\\
%         7500 AE Enschede, The Netherlands\\
%         {\tt\small h.kwakernaak@autsubmit.com}}
%         \hspace*{ 0.5 in}
%         \parbox{3 in}{ \centering Pradeep Misra**
%         \thanks{**The footnote marks may be inserted manually}\\
%        Department of Electrical Engineering \\
%         Wright State University\\
%         Dayton, OH 45435, USA\\
%         {\tt\small pmisra@cs.wright.edu}}
%}

\author{Longchao Da, Hao Mei, Romir Sharma and Hua Wei% <-this % stops a space
% \thanks{This work was supported by xxxxx}% <-this % stops a space
\thanks{\boldmath Hua Wei with Longchao Da, Hao Mei, are at  
School of Computing and Augmented Intelligence (SCAI), Arizona State University, USA,
        {\tt\small \{hua.wei, longchao, hmei7\}@asu.edu.} Romir Sharma is with the West Windsor-Plainsboro High School South, West Windsor, USA, {\tt\small sharmaromir@gmail.com}}%
\thanks{The work was partially supported by NSF award \#2153311. The views and conclusions contained in this paper are those of the authors and should not be interpreted as representing any funding agencies.}
}


\begin{document}



\maketitle
\thispagestyle{empty}
\pagestyle{empty}


We present a self-stabilizing algorithm for the (asynchronous) unison
problem which achieves an efficient trade-off between time, workload,
and space in a weak model.
%
Precisely, our algorithm is defined in the atomic-state model and
works in anonymous networks in which even local ports are unlabeled.
It makes no assumption on the daemon and thus stabilizes under the
weakest one: the distributed unfair daemon.

In a $n$-node network of
diameter $D$ and assuming a period $B \geq 2D+2$, our algorithm only
requires $O(\log B)$ bits per node to achieve full polynomiality as it
stabilizes in at most $2D-2$ rounds and $O(\min(n^2B, n^3))$ moves.
In particular and to the best of our knowledge, it is the first
self-stabilizing unison for arbitrary anonymous networks achieving an
asymptotically optimal stabilization time in rounds using a bounded
memory at each node.

Finally, we show that our solution allows to efficiently simulate
synchronous self-stabilizing algorithms in an asynchronous
environment.  This provides a new state-of-the-art algorithm solving
both the leader election and the spanning tree construction problem in
any identified connected network which, to the best of our knowledge,
beat all existing solutions of the literature.

% 1/4

\section{Introduction}
Traffic Signal Control (TSC) is vital for enhancing traffic flow, alleviating congestion in contemporary transportation systems, and providing widespread societal benefits. It remains an active research area due to the intricate nature of the problem. TSC must cope with dynamic traffic scenarios, necessitating the development of adaptable algorithms to respond to changing conditions.

Recent advances in reinforcement learning (RL) techniques have shown superiority over traditional approaches in TSC~\cite{noaeen2022reinforcement}. In RL, an agent aims to learn a policy through trial and error by interacting with an environment to maximize the cumulative expected reward over time. The biggest advantage of RL is that it can directly learn how to generate adaptive signal plans by observing the feedback from the environment. 


One major issue of applying current RL-based TSC approaches in the real world is that these methods are mostly trained in simulation and suffer from the performance gap between simulation and the real world. 
While training in simulations offers a cost-effective means to develop RL-based policies, it may not fully capture the complexities of real-world dynamics, limiting RL-based TSC models' practical performance~\cite{jiang2021simgan}. Simulators often employ static vehicle settings, such as default acceleration and deceleration, whereas real-world conditions introduce substantial variability influenced by factors like weather and vehicle types. These inherent disparities between simulation and reality impede RL-based models trained in simulations from achieving comparable real-world performance, as depicted in Figure~\ref{fig:intro}.

To bridge this gap, prior research has concentrated on enhancing traffic simulators to align more closely with real-world conditions, using real-world data~\cite{zhang2019cityflow}. This enables smoother policy or model transfer from simulation to reality, minimizing performance disparities. Yet, altering internal simulator parameters can be challenging in practice. To tackle this issue, Grounded Action Transformation (GAT) has emerged as a popular technique, aiming to align simulator transitions more closely with reality. However, GAT has predominantly been applied to robotics, with limited exploration in the context of traffic signal control.
%Unlike most sim-to-real transfer techniques in robotics which mainly tackle the domain gap in visual representations, traffic signal control mainly faces a domain gap in the transition dynamics, as real-world traffic dynamics can be complex and hard to simulate accurately.
% Grounded Simulation Learning is a promising paradigm for addressing the sim-to-real problem in robotics which aims to modify the simulator to better match the real world based on data from the real world. However, in practice, the internal parameters of the simulator cannot be easily modified. To address this challenge, the state-of-the-art approach is Grounded Action Transformation (GAT). Unlike GSL, GAT performs grounding not by modifying the simulator but rather by augmenting it with a learned action transformer that seeks to induce simulator transitions that more closely match the real world. GAT has shown promising results in bridging the gap between simulation and reality, making it a popular technique for sim-to-real transfer in robotics.
%the state-of-the-art approach is Grounded Action Transformation (GAT)
%much effort has been made to transfer a learned policy or model from a simulated environment to a real-world environment with simulation-to-real-world (sim-to-real) transfer techniques~\cite{}. Among them, 


In this paper, we present Uncertainty-aware Grounded Action Transformation (\ours), an approach that bridges the domain gap of transition dynamics by dynamically transforming actions in the simulation with uncertainty. 
\ours learns to mitigate the discrepancy between the simulated and real-world dynamics under the framework of grounded action transformation (GAT), which learns an inverse model that can generate an action to ground the next state in the real world with a desired next state predicted by the forward model learned in simulation. Specifically, to avoid enlarging the transition dynamics gap induced by the grounding actions with high uncertainty, \ours dynamically decides when to transform the actions by quantifying the uncertainty in the forward model. Our experiments demonstrate the existence of the performance gap in traffic signal control problems and further show that \ours has a good performance in mitigating the gap with higher efficiency and stability.

% Figure environment removed


% TSC problem--> RL method good --> real-world gap --> sim2real 

% 1

\section{Preliminaries}

% \subsection{\textcolor{red}{Formalize the sim-to-real TSC problems}}


This section will formalize the traffic signal control (TSC) problem and its RL solutions and introduce the grounded action transformation (GAT) framework for sim-to-real transfer.

\subsection{Concepts of TSC and  RL Solutions}
    % \paragraph{\textcolor{red}{Traffic Signal Control(TSC) problem define}}

    % \textcolor{blue}{Given a set of traffic demands at an intersection, the objective of TSC details...
    
    % . }
In the TSC problem, each traffic signal controller decides the phase of an intersection, which is a set of pre-defined combinations of traffic movements that do not conflict while passing through the intersection. Given the current condition of this intersection, the traffic signal controller will choose a phase for the next time interval $\Delta t$ to minimize the average queue length on lanes around this intersection. Following existing work~\cite{wei2018intellilight,chen2020toward,wei2019colight}, an agent is assigned to each traffic signal, and the agent will choose actions to decide the phase in the next $\Delta t$. The TSC problem is defined as an MDP which could be characterized by $\mathcal{M}  = \langle \mathcal{S}, \mathcal{A}, P, r, \gamma \rangle$ where the state space $\mathcal{S}$ contains each lane's number of vehicles and the current phase of the intersection in the form of the one-hot code, $s_t \in \mathcal{S}$. Action space (discrete) $\mathcal{A}$ is the phase chosen for the next time interval $\Delta t$. Transition dynamics $P(s_{t+1}| s_t, a_t)$ maps $\mathcal{S} \times \mathcal{A} \rightarrow \mathcal{S}$, describing the probability distribution of next state $s_{t+1} \in \mathcal{S}$. Reward $r$ is an immediate scalar return from the environment calculated as $r_t = -\sum_l w_t^l$, where $l$ is the lane belonging to the intersection and $w_t^l$ is the queue length at each lane. And the $\gamma$ is the discount factor. Policy $\pi_{\theta}$ could be represented as the logic of: $\mathcal{S} \rightarrow \mathcal{A}$. An RL approach solves this problem by maximizing the long-term expectation of discounted accumulation reward adjusted $\gamma$. The discounted accumulated reward is $\mathbb{E}_{(s_t, a_t)\sim (\pi_{\theta},\mathcal{M})}[ \sum_{t=0}^T \gamma^{T-t} r_t(s_t, a_t) ]$. Since the action space $\mathcal{A}$ is discrete, we follow the past work using Deep Q-network (DQN)~\cite{wei2018intellilight} to optimize the $\pi_{\theta}$, the above procedure is conducted in simulation environment $E_{sim}$.


    % \item System: The traffic state space consists of the number of vehicles on each lane belonging to the intersection and the current phase of the intersection in the form of the one-hot code. Thus the state at time t could be represented as $s_t \in \mathcal{S}$.
    % \item Set of action space $\mathcal{A}$: In this paper, we define the action as the phase for the next time interval $\Delta t$. The action space $\mathcal{A}$ is discrete since, in practice, the phases are a set of pre-defined combinations of traffic movements. %The action $a_t \in \mathcal{A}$ at time $t$ is discrete, indicating a certain phase from a set of predefined phases $\mathcal{A}$ for the intersection from $t$ to $t+\Delta t$.
    % \item Transition dynamics $P$: Given current state $s_t \in \mathcal{S}$ and action $a_t \in \mathcal{A}$, the transition dynamics $P(s_{t+1}| s_t, a_t)$ maps $\mathcal{S} \times \mathcal{A} \rightarrow \mathcal{S}$, describing the probability distribution of next state $s_{t+1} \in \mathcal{S}$. 
    % In this paper, we use transition dynamics and transition probability distribution interchangeably. 
    % \item \label{pre:rewarddefine}Reward $r$: The reward is typically a scalar return from the environment which can be described as $\mathcal{S} \times \mathcal{A} \rightarrow \mathbb{R}$. The reward we used in the traffic signal control problem is an immediate return from the environment. The reward at time t could be calculated from $r_t = -\sum_l w_t^l$, where $l$ is the lane belonging to the intersection and $w_t^l$ is the queue length at each lane.
    % \item Policy $\pi_{\theta}$ and discount factor $\gamma$: The policy $\pi_\theta$ parameterized by $\theta$ aims to  minimize the expected queue length $r_t$. Specifically, it maps the given state $s_t$ to the action $a_t$ taken at this time step. It could be represented $\pi_{\theta}: \mathcal{S} \rightarrow \mathcal{A}$.


% \subsection{\textcolor{red}{Domain Gap between Simulation and Real-world}}

% Domain gap impedes the generalization of models trained in the source domain to the target domain. In RL, the gap between the simulation (the source domain to improve a policy) and the real world (the target domain where trained policy is implemented) hinders most research transacted from the simulation into the real world concerning the cost and public safety. Here we define the sim-to-real transfer problem as below:
% \begin{itemize}
%     \item Modeling simulation and real-world environment: We first define simulation and real-world environment as $E_{sim}$ and $E_{real}$. We consider the two environments to share the same MDP except for the transition dynamics characterized by the transition probability distribution. The MDP in $E_{sim}$ could be characterized as $\mathcal{M}_{sim}  = \langle \mathcal{S}, \mathcal{A}, P_{\phi}, r, \gamma \rangle$, where the simulation's transition dynamic is controlled by a set of parameters $\phi$.
%     And the MDP in $E_{real}$ could be characterized as $\mathcal{M}_{real}  = \langle \mathcal{S}, \mathcal{A}, P^*, r, \gamma \rangle$, where $P^*$ is the transition dynamics in the real world.
% \end{itemize}
% The goal of sim-to-real transfer is to find the best policy $\pi_{\theta^*}$ such that:
%     \begin{equation}
%         \pi_{\theta^*} = \arg \max_{\pi_{\theta}} \eta_{P^*}(\pi_{\theta}), P^* \neq P_{\phi}
%     \end{equation}
%     where $\eta_P(\pi_{\theta})$ is used to evaluate the policy's performance in the $E_{real}$:
%     \begin{equation}
%         \eta_{P}(\pi_{\theta}) = \mathbb{E}_{(s_t,a_t) \sim (\pi_{\theta}, \mathcal{M}_{real}}) [ \sum^T_{t=0} \gamma^{T-t}r_t(s_t, a_t) ]
%     \end{equation}


     % \paragraph{\textcolor{blue}{Describe why the gap exists with reality examples and refer to our preliminary study results to justify the problem's seriousness. And mathematically model how the gap appears and causes problems.}}  \\
    


     % \paragraph{}

% \subsection{Potential Solutions}
% Domain Randomization, Transfer Learning, Data Augmentation, Multi-Task Learning, etc.

\subsection{Grounded Action Transformation}

% 3. GAT is trying to bridge the gap between transition dynamics, which suits our TSC problem. Explain how grounded action solves the dynamics differences. 

Grounded action transformation (GAT) is a framework originally proposed in robotics to improve robotic learning by using trajectories from the physical world $E_{real}$ to modify $E_{sim}$. Under the GAT framework, MDP in $E_{sim}$ is imperfect and modifiable, and it can be parameterized as a transition dynamic $P_{\phi}(\cdot|s, a)$. Given real-world dataset $\mathcal{D}_{real} =  \{\tau^{1}, \tau^{2}, \dots, \tau^{I}\}$, where $\tau^{i} = (s_0^{i}, a_0^{i}, s_1^{i}, a_1^{i}, \dots, s_{T-1}^{i}, a_{T-1}^{i}, s_T^{i})$ is a trajectory collected by running a policy $\pi_{\theta}$ in $E_{real}$, GAT aims to minimize differences between transition dynamics by finding $\phi^*$:
\vspace{-2mm}
\begin{equation}
\phi^* = \arg \min_{\phi} \sum_{\tau^i \in \mathcal{D}_{real}} \sum_{t=0}^{T-1} d(P^*(s^i_{t+1}|s^i_t, a^i_t), P_{\phi}(s^i_{t+1}|s^i_t, a^i_t))
\end{equation}

% \textcolor{red}{wirte full function $g_{\phi}()$}

\noindent where $d(\cdot)$ is the distance between two dynamics, $P^*$ is the real world transition dynamics, and $P_{\phi}$ is the simulation transition dynamics. 

To find $\phi$ efficiently, GAT takes the agent's state $s_t$ and action $a_t$ predicted by policy $\pi_\theta$ as input and outputs a grounded action $\hat{a}_t$. Specifically, it uses an action transformation function parameterized with $\phi$: 
\vspace{-2mm}
\begin{equation}
\label{eq:gat}
    \hat{a}_t = g_{\phi}(s_t, a_t) = h_{\phi^{-}}(s_t, f_{\phi^{+}}(s_t, a_t))
    \vspace{-2mm}
\end{equation}
 which includes two specific functions: a forward model $f_{\phi^{+}}$, and an inverse model $h_{\phi^{-}}$, as is shown in Fig.~\ref{fig:intro}.   
 
\noindent $\bullet$ \textit{The forward model} $f_{\phi^{+}}$ is trained with the data from $E_{real}$, aiming to predict the next possible state $\hat{s}_{t+1}$ given current state $s_t$ and action $a_t$:
\vspace{-2mm}
\begin{equation}
\label{eq:forward}
     \hat{s}_{t+1} =  f_{\phi^{+}}(s_t, a_t)
     \vspace{-2mm}
\end{equation}
% $s_t, a_t$ are current state and action and $\hat{s}_{t+1}$ is the predicted next state in $\mathcal{M}_{real}$ 

\noindent $\bullet$ \textit{The inverse model} $h_{\phi^{-}}$ is trained with the data from $E_{sim}$, aiming to predict the possible action $\hat{a}_t$ that could lead the current state $s_t$ to the given next state. Specifically, the inverse model in GAT takes $\hat{s}_{t+1}$, the output from the forward model, as its input for the next state: 
\vspace{-2mm}
\begin{equation}
\label{eq:inverse}
     \hat{a}_t =  h_{\phi^{-}}(\hat{s}_{t+1}, s_t)
     \vspace{-2mm}
\end{equation}

%takes the next state $\hat{s}_{t+1}$ calculated from the forward model and the current state $s_t$ in simulation as input, and predicts a grounded action $\hat{a}_t$ under the transition dynamics in $E_{sim}$:

% The output from the inverse model, $\hat{a}_t$, is the grounded action which makes the resulted $s_{t+1}$ under the simulation transition dynamics $P_{\phi}(s_{t+1}|s_t, \hat{a}_t)$ close to the predicted next state $\hat{s}_{t+1}$ under real world transition dynamics $P^*(\hat{s}_{t+1}|s_t, a_t)$.

Given current state $s_t$ and the action $a_t$ predicted by the policy $\pi_\theta$, the grounded action $\hat{a}_t$ takes place in $E_{sim}$ will make the resulted $s_{t+1}$ in $E_{sim}$ close to the predicted next state $\hat{s}_{t+1}$ in $E_{real}$, which makes the dynamics $P_{\phi}(s_{t+1}|s_t, \hat{a}_t)$ in simulation close to the real-world dynamics $P^*(\hat{s}_{t+1}|s_t, a_t)$. Therefore, the policy $\pi_\theta$ is learned in $E_{sim}$ with $P_{\phi}$ close to $P^*$ will have a smaller performance gap when transferred to $E_{real}$ with $P^*$.


% The new grounded action $\hat{a}_t$ in the $E_{sim}$ will result in the agents transitioning close to the next states, which would happen in $E_{real}$ given the current states and actions. \hua{need more intuition}
% Given current state $s_t$, the grounded action $\hat{a}_t$ will make the resulted $s_{t+1}$ under the simulation transition dynamics $P_{\phi}(s_{t+1}|s_t, \hat{a}_t)$ close to the predicted next state $\hat{s}_{t+1}$ under real world transition dynamics $P^*(\hat{s}_{t+1}|s_t, a_t)$. GAT then proceeds to improve $p_\theta$ within the grounded simulator.


% 1. We formalize tsc as a reinforcement learning (RL) problem (Sutton and Barto 1998).

    

%2. Deinfe \pi_{sim, \theta} for sim2real use later: We assume π is parameterized by a vector θ and denote the
%parameterized policy as πθ

%3. Raise sim2real problem E_sim to E_real: In this paper, learning takes place in a simulator which is in environment, Esim, that models E. Specifically Esim has the same state-action space as E but inevitably a different dynamics distribution, Psim.. Thus θ with Jsim(θ) < Jsim(θ0) does not imply J(θ) < J(θ0) – in fact J(θ) could even be worse than J(θ0). In practice and in the literature, learning in a simulator frequently leads to catastrophic degradation of J. This paper explores methods for learning in Esim that result in lower policy cost. 


% 3/4 (together)
\vspace{-4mm}
\section{Methods}

% To mitigate the gap in the transition dynamics between the simulation and the real world, we propose an uncertainty

% \textcolor{blue}{we use \ours to mitigate the gap between sim and real transition dynamics and add into uncertainty to help mitigate the transformation model overconfidence problem and improve its performance.}
To mitigate the gap in the transition dynamics between traffic simulations and real-world traffic systems, we use the vanilla GAT and analyze its limitations. To overcome the problem in vanilla GAT, we 
propose \ours to further leverage uncertainty quantification to take grounded action dynamically. %In the rest of this section, we will illustrate GAT in TSC problem, \ours, and its corresponding implementations. 

\subsection{Vanilla GAT for TSC}

% Though TSC problem has been studied a lot recently and many researches have shown RL-based method has achieved promising results, there are still concerns about the these methods performances when they are deployed in real life. We will discuss current RL-based researches limitation on generalizability and propose an effective way to mitigate this problem in the below paragraphs. 
% % \textcolor{blue}{introduce the problem of transition dynamic gaps in TSC in the past work}

% % 1. there is a dynamic transition gap existing between sim and real in TSC. TSC exploration is at a high cost, so we use the simulation. 

% \subsubsection{\textcolor{red}{Generalization problem in past RL-based Traffic signal control}}

 % The gap between simulation and the real world hinders the application of RL-based  methods in the real world.
% 3. To improve the policy trained in the simulation and generalize it to the real-word, we need to modify the simulation‘s $\mathcal{M}_{sim}$ to make it higher fidelity and close to real-world $\mathcal{M}_{real}$.
% To establish a closer resemblance between the simulation and the physical world, minimizing the disparity between the transition dynamics in both domains is essential. The procedure enabling the transition dynamics in simulation $P_{\phi}$ close to the real-world transition dynamics $P^*$ is commonly called grounding.



% \textcolor{green}{formula to define forward and inverse model}

% \subsubsection{\textcolor{red}{GAT for Traffic Signal Control}}


% \noindent1. Initialize policy $\pi_0$, real-world traffic dataset $\mathcal{D}_{real}$, simulation traffic dataset $\mathcal{D}_{sim}$, and train policy in the $E_{sim}$ till converge. 

% \noindent2. Execute policy $\pi_i$ to control traffic light in both $E_{sim} \text{ and } E_{real}$ and append collected traffic states and actions (phases) pairs ${(s^i_0, a^i_0), \dots, (s^i_t, a^i_t)}$ into $\mathcal{D}_{real}$ and $\mathcal{D}_{sim}$.


% \noindent3. Optimize action transformation function $g_{\phi} = f^{-1}_{sim}(s^i_t, f_{real}(s^i_t, a^i_t))$ to find the $\phi^*$ which minimize equation~\ref{eq:gat}. 

% \noindent4. Optimize policy $\pi_i$ in $E_{sim}$. At each simulation step $t$, if there is an $a_t^* = \arg \max_{a_{t}} Q(s_t, a_t)$ returned from $\pi_i$, replace $a_t^*$ with grounded action $\hat{a}_t = g_{\phi}(s_t, a^*_t)$.

% \noindent5. After optimization, return updated policy $\pi_{i+1}$ got to the step 2 until the policy did not improve in the $E_{real}$.


We use the vanilla GAT for the traffic signal control problem by specifying the learning of $f_{\phi^+}$ and $h_{\phi^-}$: 

\noindent $\bullet$ \textit{The forward model} $f_{\phi^+}(s_t, a_t)$ in traffic signal control problem predicts the next traffic state $\hat{s}_{t+1}$ in the real world given taken action $a_t$ and the current traffic state $s_t$. We approximate $f_{\phi^+}$ with a deep neural network and optimize $\phi^+$ by minimizing the Mean Squared Error (MSE) loss:
\begin{equation}
\label{eq:forward-loss}
     \mathcal{L}(\phi^+) = MSE(\hat{s}^i_{t+1}, s^i_{t+1}) = MSE(f_{\phi^+}(s^i_t, a^i_t), s^i_{t+1})
\end{equation}
where $s^i_t$, $a^i_t$, $s^i_{t+1}$ are sampled from the trajectories collected from $E_{real}$.

\noindent $\bullet$ \textit{The inverse model} $h_{\phi^-}(\hat{s}_{t+1}, s_t)$ in traffic signal control predicts the grounded action $\hat{a}^i_t$ in simulation $E_{sim}$ to reproduce the same traffic states $\hat{s}_{t+1}$. 
We approximate $h_{\phi^-}$ with a deep neural network and optimize $\phi^-$ by minimizing the Categorical Cross-Entropy (CCE) loss since the target $a^i_t$ is a discrete value:
\begin{equation}
\label{eq:inverse-loss}
    \mathcal{L}(\phi^-) = CCE(\hat{a}^i_t, a^i_t) = CCE(h_{\phi^-}(s^i_{t+1}, s^i_t), a^i_t)
\end{equation}
where $s^i_t$, $a^i_t$, $s^i_{t+1}$ are sampled from the trajectories collected from $E_{sim}$.

% Details of \ours are shown in Algorithm~\ref{algo:UGAT}.

% an inverse model $f^{-1}_{sim}(\hat{s}_{t+1}, s_t)$, which predicts the grounded action in the simulation to reproduce the same traffic states $\hat{s}_{t+1}$ in the real world. Thus the optimization on parameter $\phi$ could be formulated as a supervised learning problem, where the forward model is trained with the gradient from $MSE(f_{real}(s^i_t, a^i_t), s^i_{t+1})$ by data collected from real-world. And the inverse model is trained with the gradient from $CrossEnropy(f^{-1}_{sim}(\hat{s}^i_{t+1}, s^i_t), a^i_t)$ with data collected from low-fidelity simulation environment. Details of UGAT are shown in Algorithm~\ref{}

% 4. How TSC could benefit from grounded action transformation.


% Following the definition in the preliminary. We optimize our policy $\pi$ in the $E_{sim}$ with MDP $\mathcal{M}_{sim} = \left \langle \mathcal{S, A, r, P_{sim}}, \pi, \gamma \right \rangle$ to archive optimal in $E_{real}$: 

% \begin{align}
%     \pi^* &= \arg \max_{\pi} \eta_{P_{real}}(\pi), \\
%     \eta_{P_{real}}(\pi) &= \mathbb{E}_{\tau\sim(\pi)} \left[ R(\tau) \right] 
% \end{align}

% We achieve this by taking grounded action to make trajectory $\tau$ in $E_{sim}$ close to $E_{real}$:

% \begin{equation}
% \phi^* = \arg \min \sum_{\tau_{i} \in \mathcal{D}} \sum^{L}_{t=0} d(P(s_{t+1}^{i}| s^i_t, a^i_t), P_{\phi}(s^i_{t+1}|s^i_t, a^i_t))
% \end{equation} 

% \textcolor{red}{introduce gat module and how }


% \textcolor{red}{a picture to show how grounded action find $\theta^*$ }
% % 1/6
% % Figure environment removed


% where $d(c\dot)$ is the distance between two distributions, $s^i_t \text{ and } s^i_{t+1} \in \mathcal{S}$ are current state and next time state in state space, $a^i_t \in \mathcal{A}$ is the action taken at current time step belong to action space.

% To find $\phi$ efficiently, GAT uses parameterized action transformation function, which takes in agents' states and actions and outputs new actions. The new actions, grounded actions, will result in the agents in the $E_{sim}$ transitioning close to the next states, which would happen in $E_{real}$ given the current states and actions. This transition is called the grounded procedure.

% In the GAT grounding procedure is achieved by training an action transformation function with supervised learning:
% \begin{equation}
% g_{\phi}(s^i_t, a^i_t) = f^{-1}_{sim}(s^i_t, f_{real}(s^i_t, a^i_t))
% \end{equation}
% % where $f^{-1}_{sim}() \text{ and } f_{real}()$ are forward and inverse models explained below:


% Our implementation of GAT in TSC follows the steps described below:

% \begin{itemize}
%     \item[1] First train $\pi$ in the $E_{sim}$ to improve and get initial policy $\pi^0$.
%     \item[2] Execute policy $\pi$ in $E_{real}$ and $E_{sim}$ to collect data $D_{sim}$ and $D_{real}$
%     \item[3] train forward transition model $f_{real}$ with $D_{real}$.
%     \item[4] Train inverse transition model $f^-1_{sim}$ with $D_{sim}$.
%     \item[5] Update policy $\pi$ in $S_{sim}$ 
%     \item[6] iteratively run between steps 2-5 till performance stop improving in $E_{real}$
    
% \end{itemize}


% The forward model is trained with the gradient from $MSE(f_{real}(s^i_t, a^i_t), s^i_{t+1})$ by data collected from real-world. The inverse model is trained with the gradient from $CrossEnropy(f^{-1}_{sim}(s^i_{t+1}, s^i_t), a^i_t)$ with data collected from low-fidelity simulation environment.

% \textcolor{blue}{Introduce the problem that we have, and make the connection with what GAT is trying to solve, then we changed something of GAT to fit the task.
% (when explaining the methods, use defined TSC actions/states/terms. }

\subsection{Uncertainty-aware GAT}
In this section, we will introduce the limitations of the vanilla GAT and propose an uncertainty-aware method on GAT that can benefit from quantifying model uncertainty.

% \textcolor{blue}{introduce the existing GATs' flaws and challenges, and then introduce how we think about uncertainty could address such a challenge}

\subsubsection{\textbf{Model Uncertainty on $g_{\phi}$}}

% \textcolor{red}{add in Annotation with description}
% 1. Current \ours has model uncertainty and overconfidence in some actions. This will result in enlarging the gap instead of mitigating it. Thus learning is not stable and hard to converge.

The vanilla GAT takes supervised learning to train the action transformation function $g_{\phi}$, and grounded action transformation $\hat{a}$ is taken at each step while improving in the $E_{sim}$. However, the action transformation function $g_{\phi}$ could have high model uncertainty on unseen state and action inputs, which is likely to happen during the exploration of RL. With high model uncertainty on $g_{\phi}$, the grounded action $\hat{a}$ in Equation~\eqref{eq:gat} is likely to enlarge the performance gap instead of mitigating it if the high uncertainty action is taken because it will make policy learning unstable and hard to converge.


% \textcolor{red}{uncertatinty decrease performance of }
% To overcome this problem, inspired by uncertainty-aware control methods (reference here), we proposed an uncertainty-induced action transformation mechanism that will only execute grounded action while the inference has high certainty. By doing so, we can 1) keep the exploration rate close to the original $\epsilon$ and thus improve the policy learning step in the GAT framework. 2) By doing grounded action only at high certainty, the new actions will only be taken in $E_{sim}$ when the resulting next state is closed to the next state, which would be reached in $E_{real}$.


 % \textcolor{red}{model uncertainty exists, overconfident- we introduce uncertainty to avoid overconfident}
% 2. Using uncertainty quantification to evaluate model uncertainty can help distinguish these actions and avoid these grounded actions with high model uncertainty.

To overcome the enlarged gap induced by $\hat{a}$ with high model uncertainty in $g_{\phi}$, we need uncertainty quantification methods~\cite{kabir2018neural} to keep track of the uncertainty of $g_{\phi}$. Specifically, we would like the action transformation function to output an uncertainty value $u_t$ in addition to $\hat{a}_t$:
\begin{equation}
\label{eq:ugat-uncertainty}
    \hat{a}_t, u_t = g_{\phi}(s_t, a_t) = h_{\phi^-}(f_{\phi^+}(s_t, a_t), s_t)
\end{equation}

In general, any methods capable of quantifying the uncertainty of a predicted class from a deep neural network (since $h_{\phi^-}$ is implemented with deep neural networks) could be utilized, like evidential deep learning (EDL), Concrete Dropout~\cite{gal2017concrete}, Deep Ensembles~\cite{lakshminarayanan2017simple}, etc. In this paper, we explored different state-of-the-art uncertainty quantification methods and found out that they all perform well with our method (their experimental results can be found in Section~\ref{sec:exp:uncertainty}). We adopted EDL as the default in our method as it performs the best with our method.

Intuitively, during action grounding, whenever model $g_{\phi}(s_t, a_t)$ returns a grounded action $\hat{a}_t$, if the uncertainty $u_t$ is less than the threshold $\alpha$, the grounded action $\hat{a}_t$ will be taken in the simulation environment $E_{sim}$ for policy improvement; otherwise, we will reject $\hat{a}_t$ and take the original $a_t$. This uncertainty quantification allows us to evaluate the reliability of the transformation model and take grounded actions $\hat{a}$ when the model is certain that the resulting transition $P_{\phi}(s_t, \hat{a}_t)$ would mirror that of the real-world environment $E_{real}$ transition $P^*(s_t, a_t)$. This process enables us to minimize the gap in Equation~\eqref{eq:gat} between the policy training environment $E_{sim}$ and the policy testing environment $E_{real}$, thereby mitigating the performance gap. 


%Details are described in Algorithm~\ref{algo:UGAT}.

% and propose an uncertainty-aware GAT method that will only execute grounded action $\hat{a}$ while the inference $g_{\phi}(s_t, a_t)$ has low model uncertainty. Such uncertainty quantification allows us to 1) take grounded action only at low uncertainty. 2) stabilize the policy improvement and converge the learning policy. 

% Introducing uncertainty quantification allows us to evaluate the reliability of the transformation model and take grounded actions $\hat{a}$ when the model is certain that the resulting transition $P_{\phi}(s_t, \hat{a}_t)$  would mirror that of the real-world environment $E_{real}$ transition $P^*(s_t, a_t)$. This process enables us to minimize the gap in Equation~\eqref{eq:gat} between the simulation environment $E_{sim}$ and the real-world environment $E_{real}$, thereby improving the fidelity of the simulation. Details are described in Algorithm~\ref{algo:UGAT}.

% 3. We improve GAP-TSC to uncertainty-aware \ours to mitigate the dynamic gaps and, at the same time, achieve stable learning. We keep track of uncertainty and plugin the uncertainty into GAT. (Details description of how we set alpha)

% To make GAP-TSC avoid executing grounded action $a_grounding$ while model uncertainty is high, we extend our framework to uncertainty-aware \ours. 

% \subsubsection{\textbf{Choice of Uncertainty Model}}
% Since the uncertainty model is highly correlated to the performance of the uncertainty method, we explored different state-of-the-art uncertainty quantification methods and found UGAT performs well across different methods. We use the evidential deep learning (EDL) method to demonstrate how we integrate the uncertainty model into the inverse model of \ours.

% % Inspired by the Theory of Evidence (DST) \cite{TheoryE}, Subjective Logic (SL) allows one to quantify belief masses and uncertainty through a well-defined theoretical framework and the overall uncertainty mass of $u$ can be written as $u = \frac{K}{S}$,  where $ S = \sum_{i=1}^K (e_i+ 1) $, evidence $e_i$ is a measure of the amount of support collected from data in favor of a sample to be classified into a certain class and $K$ refers to mutually exclusive class labels $k = 1, . . . , K$.

% Evidential deep learning (EDL)~\cite{sensoy2018evidential} is an uncertainty quantification method from the theory of evidence perspective for classification problems, quantifying belief mass and uncertainty. Specifically, in a K-class multi-classification problem, subjective logic (SL) considers class labels mutually exclusive singletons by providing belief mass $b_k$ for each class $k$. And it, at the same time, provides an overall uncertainty mass of $u$. The relationship between belief mass and uncertainty is: 
% \begin{equation}
%     u = \frac{K}{S} \text{ and } b_k = \frac{e_k}{S}
% \end{equation}
% where $S = \sum_{i=1}^K(e_i +1)$. In the EDL, the $Softmax$ activation on the output layer of a neural network is replaced with $ReLU$ to calculate the evidence vector $[e_1, e_2, \dots, e_K ]$ for each class. And the uncertainty is calculated by:
% \begin{equation}
% \label{eq:edl}
%     u = \frac{K}{\sum_{i =1}^K (e_i +1)}
% \end{equation}


% To track the model uncertainty of grounded transformation function $g_{\phi}(\cdot, \cdot)$, we see the action space $\mathcal{A}$ as classes and integrate uncertainty into the inverse model $h_{\phi^-}(\hat{s}_{t+1}, s_t)$ to quantify the uncertainty of output grounded action $\hat{a} \in \mathcal{A}$. At the last layer for predicting the output $logits$, we use $ReLU$ as the activation layer \cite{sensoy2018evidential} to calculate the evidence. Model uncertainty $u$ can be easily computed from Equation~\eqref{eq:edl}. After adding EDL to the inverse model, we modify the inverse model into:
% \begin{equation}
% \label{eq:update inverse}
%     \hat{a}_t, u_t =  h^u_{\phi^-}(\hat{s}_{t+1}, s_t)
% \end{equation}
% and grounded transformation could is modified into:
% \begin{equation}
%     \hat{a}_t, u_t = g_{\phi}(s_t, a_t) = h_{\phi^-}(f_{\phi^+}(s_t, a_t), s_t)
% \end{equation}


\subsubsection{\textbf{Dynamic Grounding Rate $\alpha$}}
The threshold $\alpha$, which we referred to as the grounding rate,  helps us to decide when to filter out $\hat{a}_t$ with uncertainty $u_t$. One naive approach of deciding the grounding rate $\alpha$ is to treat it as a hyperparameter for training and keep it fixed during the training process. However, since $g_{\phi}(s_t, a_t)$ keeps being updated during the training process, the model uncertainty of $g_{\phi}$ is dynamically changing. Even with the same $s_t$ and $a_t$, the output $u_t$ and $\hat{a}_t$ from $g_{\phi}(s_t, a_t)$ could be different in different training iterations. 

An alternative yet feasible approach is to set grounding rate $\alpha$ dynamically changing with the model uncertainty during different training iterations.
% Hereby, we use an $\alpha$ to represent the grounding rate which is the threshold for taking $\hat{a}$. In the uncertainty-aware \ours, we dynamically adjust $\alpha$, since the action transformation model $g_{\phi}(\cdot, \cdot)$ is updating; as a result, the model uncertainty will adjust with the learning sample increasing. A fixed grounding rate cannot synchronously adjust with the updating $g_{\phi}(\cdot, \cdot)$, which would not be able to achieve the regularization on grounded action $\hat{a}$. Also, a dynamically adjusted grounding rate will release the efforts on hyper-parameter tuning. \hua{gabbish}
To dynamically adjust the grounding rate with the changing of model uncertainty, we keep track of the model uncertainty $u_t$ of $g_{\phi}(s_t, a_t)$ during each training iteration. 
 % there is a high chance of making the next states $s_{t+1}$ in $E_{sim}$ diverge to the next state $\hat{s}_{t+1}$, which would happen in $E_{real}$. As a result, we reject to take it. And low uncertainty means that by taking $\hat{a}_t$, we have a high chance of pushing the next state in $E_{sim}$ and $E_{real}$ close. 
%When we work in the uncertainty-aware \ours framework, in the first iteration, during the policy improvement step, whenever model $g_{\phi}(s_t, a_t)$ returns a grounded action $\hat{a}_t$, we track the model uncertainty $u_t$ and take the original action $a_t$. 
At the end of each iteration $i$, we update the grounding rate $\alpha$ for the next iteration based on the past record of model uncertainty by calculating the mean
\begin{equation}
\label{eq:u-update}
    \alpha = \frac{\sum^E_{e=1} \sum^{T-1}_{t=0} u^e_t}{T \times E}
\end{equation}
from the logged uncertainties in the last $E$ epochs.
This dynamic grounding rate $\alpha$ can synchronously adjust $\alpha$ with the update of $g_{\phi}$ and relief efforts on hyper-parameter tuning.


% \subsection{\textcolor{red}{Implementations}}
% In this section, we introduce the detailed implementation of model uncertainty quantification and dynamically adjusted grounding rate.



% % In the policy improvement step, at round $r$, assume for each epoch $i \in N$, containing $T$ steps, we can calculate $u_r$ from estimation on each step $u_i$:
% % \begin{equation}
% %     u_r = \frac{\sum_{n=1}^N{\sum_{i=1}^M} u_i }{N*M}
% % \end{equation}
% % And then, we update the parameter $\alpha = u_r$; if the uncertainty at round $t$: $u_t < u_r$, the model conducts action grounded. 

% \subsubsection{\textcolor{red}{Calculation of Grounding Rate}}



% Thus introducing uncertainty-aware grounding action is beneficial to learn a policy in $E_{sim}$, which could be generalized to $E_{real}$.

% \textcolor{red}{why dynamically change $\alpha$}

% \textcolor{red}{fixed cannot tune with updated transformation model. no need to tune hyper-parameter  }
%  In the first iteration, during the policy improvement step, we track the uncertainty of each simulation step and calculate the first quantile as $\alpha$. At this round, we do not take grounded actions. In the rest iterations, during the policy improvement step, we calculate grounded action with its uncertainty. If the uncertainty is less than the threshold $\alpha$, we take grounded action; if less, then not. And take the new first quantile as $\alpha$ for the next iteration. 

% Since the uncertainty model is highly correlated to the performance of the uncertainty quantification method, we explored different state-of-the-art methods, including {Methods}.



% \subsubsection{\textcolor{red}{Example: evidential deep learning (EDL) in \ours}}
% 4. We provide one example of an uncertainty modeling method (how we use uncertainty modeling here.




% In the Algorithm~\ref{} step, when we calculate grounded action from grounded action transformation function $g_{\phi}$, at the same time, we simultaneously calculate the uncertainty of taking this action. High uncertainty means by taking the grounded action; there is a high chance of making the next states in $E_{sim}$ diverge to the next state, which would happen in $E_{real}$. As a result, we reject to take it. And low uncertainty means that by taking grounded action, we have a high chance of pushing the next state in $E_{sim}$ and $E_{real}$ close. Introducing uncertainty-aware grounding action is beneficial to learn a policy in $E_{sim}$, which could be generalized to $E_{real}$.

% We set a dynamically changing threshold value $\alpha$ to decide whether to take a grounded action. In the first iteration, during the policy improvement step, we track the uncertainty of each simulation step and calculate the first quantile as $\alpha$. At this round, we do not take grounded actions. In the rest iterations, during the policy improvement step, we calculate grounded action with its uncertainty. If the uncertainty is less than the threshold $\alpha$, we take grounded action; if less, then not. And take the new first quantile as $\alpha$ for the next iteration. 

% Since the uncertainty model is highly correlated to the performance of the GAT method, we explored different state-of-the-art methods including {Methods}.

% \textcolor{red}{taking one and combine the above part for demonstration later.}

% % 1 / 6 pseudo code 

% \textcolor{blue}{strengthen how we use uncertainty with exploited various uncertainty quantification methods.}

% During the action grounding phase, due to the imperfection of the forward and inverse models, the grounded action is prone to enlarge instead of mitigating the difference of states between the simulation and real-life environment. A more severe situation is the grounded action with low belief will make this transformation behave like random exploration, resulting in instability during training. To quantify belief masses and uncertainty, we introduce a dynamically adjusted hyperparameter $\alpha$ to dynamically determine taking grounded action with a qualified uncertainty value during inference. In the framework, the uncertainty of $u^r_t$ at time $t$ after training round $r$ is quantified based on the Evidential Deep Learning method~\cite{EDL}. If $u^r_t < u^r$, where $u^r$ is the average uncertainty of all $u$ during the training round $r$, the model conducts action grounding. 
% At round $r$, assume for each epoch $n \in N$, containing $M$ steps, we can calculate $u_r$ from estimation on each step $u_i$:
% \begin{equation}
%     u_r = \frac{\sum_{n=1}^N{\sum_{i=1}^M} u_i }{N*M}
% \end{equation}
% Based on $\alpha = u_r$, if the uncertainty at round $t$: $u_t < u_r$, the model conducts action grounding. 
% Uncertainty quantification involves identifying the sources of uncertainty in the model, propagating the uncertainties through the model, and quantifying the uncertainty in the model output. This process can involve a variety of techniques, including statistical analysis, probabilistic modeling, and sensitivity analysis.

% \paragraph{\textcolor{red}{introduce one specific uncertainty and how we use it in our method, leave an open space to explore the uncertainty quantification methods}
% }


% 1/4
\subsection{Training Algorithm}
The overall algorithm for \ours is shown in Algorithm~\ref{algo:UGAT}.
% We first initialize a policy with the DQN algorithm~\cite{wei2018intellilight}, a feed-forward neural network as the forward model, and another feed-forward neural network as the inverse model with $ReLU$ activation for the last layer to output uncertainty $u$. And we initialize a real-world dataset $D_{real}$, a simulation dataset $D_{sim}$, and a dynamic grounding rate $\alpha$ as $\inf$.
We firstly pre-train the RL policy $\pi_{\theta}$ for $M$ epochs in the simulation environment $E_{sim}$. Then each training iteration of \ours starts with collecting datasets for $E_{sim}$ and $E_{real}$. Note that here we follow the data collection process in~\cite{hanna2017grounded} while the data collection in $E_{real}$ does not necessarily happen in the training process and could be done from existing offline logged data. With the collected data, we can update $g_{\phi}$ by training the forward model $f_{\phi^+}$ and inverse model $h_{\phi^-}$. With the updated $g_{\phi}$, we start to use the policy $\pi_\theta$ to interact with $E_{sim}$ for policy training. Before the action $a_t$ outputted by $\pi_\theta(s_t)$ is taken into the environment $E_{sim}$, \ours grounds the actions through $\hat{a}_t$ and $u_t$ from $g_{\phi}(s_t, a_t)$. If the model uncertainty $u_t$ is greater than the grounding rate $\alpha$, the grounded action $\hat{a}_t$ is rejected and we execute origin action $a_t$ in the simulation $E_{sim}$. Then $u_t$ is added into logged uncertainty $U$. The RL policy $\pi_{\theta}$ updates during the interaction with $E_{sim}$. After $E$ rounds of intersections, we update $\alpha$ with Equation~\eqref{eq:u-update} for the next round of policy training.


\begin{algorithm}[h!]
\DontPrintSemicolon
\caption{Algorithm for \ours with model uncertainty quantification}
\label{algo:UGAT}
\KwIn{Initial policy $\pi_{\theta}$, forward model $f_{\phi^+}$, inverse model $h_{\phi^-}$, real-world dataset $\mathcal{D}_{real}$, simulation dataset $\mathcal{D}_{sim}$, grounding rate $\alpha = \inf$}
\KwOut{Policy $\pi_{\theta}$, $f_{\phi^+}$, $h_{\phi^-}$}

    % \For {e = 1,2, \dots}{
    % \# \textbf{\textit{Improving initial policy  in $E_{sim}$}} \;
    %     Update policy $\pi$ in the $E_{sim}$ \;
    % }
    Pre-train policy $\pi_{\theta}$ for M iterations in $E_{sim}$ \;
    
	\For {i = 1,2, \dots, I}
	{
        % Reset $E_{real}$ \;
        % \For {t = 1,2, \dots}
        % {
        % \# \textbf{\textit{Generate real-world rollout with current $\pi$}} \;
        % Take actions with $a_t = \pi(s_t) $ \;
        % Add $(s_t, a_t, s_{t+1})$ into $\mathcal{D}_{real}$ \;
        % }
        Rollout policy $\pi_{\theta}$ in $E_{sim}$ and add data to $\mathcal{D}_{sim}$ \;
        
        % Reset $E_{sim}$ \;
        % \For {t = 1,2, \dots}
        % {
        % \# \textbf{\textit{Generate simulation rollout with current $\pi$ }} \;
        % Take actions with $a_t = \pi(s_t)  $ \;
        % Add $(s_t, a_t, s_{t+1})$ into $\mathcal{D}_{sim}$ \;
        % }
        Rollout policy $\pi_{\theta}$ in $E_{real}$ and add data to $\mathcal{D}_{real}$ \\

        % \For {l = 1,2, \dots} {
        % \# \textbf{\textit{Train forward model $f_{real}$}} \;
        %     Sample minibatch $B_l$ from $D_{real}$ \;
        %     Update $f_{real}$ with $MSE(f_{real}(s_t,a_t), s_{t+1}$) \;
        % }
        \# \textbf{\textit{Transformation function update step}} \;
        
        Update $f_{\phi^+}$ with Equation~\eqref{eq:forward-loss} \;
        % \For {l = 1,2, \dots} {
        % \# \textbf{\textit{Train inverse model $f^{-1}_{sim}$ \;
        %     Sample minibatch $B_l$ from $D_{sim}$ }} \;
        %     Update $f_{sim}$ with $CrossEntropy(f^{-1}_{sim}(s_t, s_{t+1|}), a_t)$ \;
        % }
        Update $h_{\phi^-}$ with Equation~\eqref{eq:inverse-loss}  \;
        
        Reset logged uncertainty $U^i = List()$ \;
        
        \# \textbf{\textit{Policy training}}\;
        \For {e = 1, 2, \dots, E}
        {
        \# \textbf{\textit{Action grounding step}} \;
            \For {t = 0, 1 ,\dots, T-1}
            {
            $a_t = \pi(s_t)$ \;
            Calculate $\hat{a}_t$ and $u_t$ with Equation~\eqref{eq:ugat-uncertainty} \;
            % $\hat{a}_t, u_t = g^u_{\phi}(s_t, a_t) = h_{\phi^-}(f_{\phi^+}(s_t, a_t), s_t)$\;
            \If{$u^e_t \geq \alpha $ }   
                {
                
                $\hat{a}_t = a_t$  \textbf{\textit{\# Reject grounded action}}
                }
            $U.append(u^e_t)$ \;
            }
            \# \textbf{\textit{Policy update step}}\;
            Improve policy $\pi_{\theta}$ with reinforcement learning\;
        }  
       Update $\alpha$ with Equation~\eqref{eq:u-update} \;
    }
\end{algorithm} 
% 2.5
% section title (report first)

% 1. introduce our sim and the real environment - CityLlow: low-fidelity, deterministic simulator(sim), SUMO: high-fidelity stochastic simulator (real), and how we construct a believable sim2real environment.
% 1.1 we used simulators to construct the environment of the simulator and real-world - this follows the procedure of past Sim2Real works. 

% 1.2 simulator introduction: why cityflow - sim, sumo -real

% 1. Inspired by grounded action transformation (GAT), how we solve the problem in TSC domain, (when explain the methods, use defined TSC actions/states/terms. 

% 3/4 - 1

% 3. uncertainty-induced grounded transformation
% 3.1 high probability of more exploration if taking grounding action if underfitting reward and inverse model
% (we take $\epsilon$-greedy exploration which $\epsilon$ = 0.1 while training policy. but if taking grounding action, the actual $\epsilon$ will be higher than the setting if our forward and inverse model is underfitting. (grounding action with less certainty equals random exploration). Thus learning will not be stable and will be hard to converge. 

% 3.2 we use two schemes to lower the grounding action rate to improve stability and improve the action transformation quality (high certainty action is allowed) 

% % 3.2.1 fixed 
% % we set $\alpha$ fixed as a threshold and grounding action when uncertainty at inference time is lower than this value.

% 3.2.2 dynamic 
% we track uncertainty at the last policy updating step and take the first quantile as a threshold $\alpha$ for the next iterations. Thus this threshold will be dynamically changing. We take grounding action when uncertainty is smaller than this threshold.

% strengthen how we use uncertainty with exploited various uncertainty quantification methods.
%1


% # introduce how uncertainty could help to benefit the GAT
% # justify without uncertainty and with uncertainty the action grounding effiencecy's improvement
% # what we applied (one example of uncertainty) briefly introduce
% # other uncertaity methods' exploratiob in experiments






\section{Experimental Results}\label{sec:results}
    \subsection{General Results}
        The basic ResSAN model is used to determine reference results which our expanded model can be compared to as it is structurally similar to ResLAN but does not possess the Lidar adaptive components of it. Further, we compare with the full-size PackNet-SAN and the unmodified NLSPN architecture. 
        As it can be seen from Tab.\,\ref{tab:sota-results}, our LiDAR-adaptive ResLAN achieves competitive performance compared to state-of-the-art standard depth completion methods, which are specialized to the unfiltered 64-beam-LiDAR. The performance differences are in the range of a few centimetres in terms of MAE, which is acceptable given the practical advantage that ResLAN can generalize to different beam patterns as will be shown below.

        Furthermore, we compared the architectures for a set of three different input types that contained 64, 32 or 16 LiDAR channels using both filter types on the metrics from the KITTI benchmark. The NLSPN model was trained for the standard depth completion task and then evaluated with different input data. As for the ResSAN models, we trained one model for each input type and tested it for the corresponding one which serve serve as the \emph{Baseline} in Tab.\,\ref{tab:overall-results}. Our ResLAN model was jointly trained for all three settings. As listed in Tab.\,\ref{tab:overall-results}, the ResLAN models outperform the challenging baseline in all metrics for FOV filtering and all but one for sparse filtering. This implies that our LiDAR adaptive model is able to outperform dedicated models in case of very sparse input depth. Fig.\,\ref{fig:comp-plot} shows this is indeed the case for 32 and even more for 16 channels. For FOV-filtered inputs with 16 channels, the ResLAN exhibits approx. $10\%$ smaller MAE than the baseline. As for the NLSPN, it becomes apparent that it is not capable of generalizing to other input types since it shows clearly worse results. The difference is especially pronounced for the FOV filtering where on average more than every fourth predicted pixel is more than $25 \%$ deviating from the ground truth\,($\delta_{1.25}$). Therefore, using a weight-adapting network in combination with differently filtered input depths allows us to train models that outperform their non-adaptive counterparts.

        \begin{table}[]
            \centering
    	    \small
            \vspace{0.4cm}
            \caption{\textbf{Depth estimation result for standard depth completion} when the ResSAN model was only trained for 64 channels and the ResLAN model for multiple tasks. The PackNet-SAN and NLSPN models were trained with the setup that was also used for our model architecture.}
            \footnotesize
            \setlength{\tabcolsep}{5pt}
            \begin{tabular}{@{}lrrrrl@{}}
            \toprule
            \multicolumn{6}{c}{\textbf{Standard LiDAR Depth Completion}}                                                                                                                         \\ \midrule
            \multicolumn{1}{l|}{Method}          & RMSE $\downarrow$            & MAE  $\downarrow$            & iRMSE $\downarrow$             & iMAE $\downarrow$ & $\delta_{1.25}$ $\uparrow$ \\
            \multicolumn{1}{l|}{}                & \multicolumn{1}{l}{{[}mm{]}} & \multicolumn{1}{l}{{[}mm{]}} & \multicolumn{1}{l}{{[}1/km{]}} & {[}1/km{]}        &                            \\ \midrule
            \multicolumn{1}{l|}{PackNet-SAN}     &  914                            &  298                            &  2.78                              &  1.4                 &  99.65 \%                          \\
            \multicolumn{1}{l|}{NLSPN}           &  \textbf{889}                            &   \textbf{263}                           &  \textbf{2.62}                              &   \textbf{1.3}                &   \textbf{99.61} \%                         \\ \midrule
            \multicolumn{1}{l|}{ResSAN (Ours)}   & 948                             &  275                            &  2.75                              &    1.4               &   99.58 \%                         \\
            \multicolumn{1}{l|}{ResLAN (Ours)} &   969                           &  283                            &   2.83                             &   1.4                &  99.56 \%                          \\ \bottomrule
            \end{tabular}
            \vspace{0.2cm}
            \label{tab:sota-results}
        \end{table}

        \begin{table}[]
    	    \centering
    	    \small
    	    \caption{\textbf{Depth estimation results of the two baseline setups and the explicit and implicit ResSAN} when evaluated on a combination of 16, 32 and 64 channel depth inputs. Please note that Specialist Methods need to train three specialized networks, one for each of the three types of inputs while our method only uses one network.}
            \footnotesize
            \setlength{\tabcolsep}{4.8pt}
            \begin{tabular}{@{}lrrrrl@{}}
                \toprule
                \multicolumn{6}{c}{\textbf{Sparse Channel Filter}}                                                                                                                                  \\ \midrule
                \multicolumn{1}{l|}{Method}        & RMSE $\downarrow$            & MAE  $\downarrow$            & iRMSE $\downarrow$             & iMAE $\downarrow$ & $\delta_{1.25}$ $\uparrow$  \\
                \multicolumn{1}{l|}{}              & \multicolumn{1}{l}{{[}mm{]}} & \multicolumn{1}{l}{{[}mm{]}} & \multicolumn{1}{l}{{[}1/km{]}} & {[}1/km{]}        &                             \\ \midrule
                \multicolumn{1}{l|}{NLSPN}         &  1396                            &  437                            & 5.54                               &  2.2                 &  98.82 \%                           \\
                \multicolumn{1}{l|}{Baseline}      & \textbf{1207}                             &  381                            & 4.41                               &  1.8                 &  \textbf{99.37} \%                           \\
                \multicolumn{1}{l|}{ResLAN (Ours)} &  1215                            &  \textbf{378}                            &  \textbf{4.27}                              &  \textbf{1.7}                 &  99.31 \%                           \\ \toprule
                \multicolumn{6}{c}{\textbf{Field-of-View Filter}}                                                                                                                                   \\ \midrule
                \multicolumn{1}{l|}{Method}        & RMSE $\downarrow$            & MAE  $\downarrow$            & iRMSE $\downarrow$             & iMAE $\downarrow$ & $\delta_{1.25}$ $\uparrow$ \\
                \multicolumn{1}{l|}{}              & \multicolumn{1}{l}{{[}mm{]}} & \multicolumn{1}{l}{{[}mm{]}} & \multicolumn{1}{l}{{[}1/km{]}} & {[}1/km{]}        &                             \\ \midrule
                \multicolumn{1}{l|}{NLSPN}         &  2738                            &  1702                            & 12.3                              &  4.3                 &  74.69 \%                           \\
                \multicolumn{1}{l|}{Baseline}      &  1556                            &  525                            &  6.8                              &  3.0                 & 98.14 \%                            \\
                \multicolumn{1}{l|}{ResLAN (Ours)} &  \textbf{1548}                            &  \textbf{519}                            &  \textbf{6.44}                              &  \textbf{2.8}                 & \textbf{98.52 \%}                            \\ \bottomrule
            \end{tabular}
            \label{tab:overall-results}
        \end{table}

        
        
        % Figure environment removed
        
        % Figure environment removed

    \subsection{Filter Effects}
        Comparing the effect of the two different types of depth input filters on the model performance, it becomes apparent that FOV filtering is the more challenging task. In that setting, reducing LiDAR channels is more detrimental to the performance than sparse filtering as it creates regions where no depth information is available. Effectively, the model is forced to perform depth prediction in these regions. These effects are highlighted in the depth images in Fig.\,\ref{fig:dense-maps} where the effect of a 16-channel sparse depth filter and a 16-channel FOV can be compared.

    \subsection{Generalization Capabilities}
        We trained three models for both filter types eaach, so the combinations and number of filtered depth inputs they receive are different. This serves the purpose of testing the generalization capabilities of the ResLAN architecture as well as the robustness to different filter settings. After training, the models were evaluated for the depth input settings they were trained for, as well as for ones they weren't exposed to. Overall, ResLAN shows good generalization capabilities. As one can gather from Fig.\,\ref{fig:explicit-comp} and Fig.\,\ref{fig:implicit-comp}, the consequences of slightly varying sets of input depth settings are limited. The most considerable deviations can be seen when the model is tasked to extrapolate. For instance, the model $\{64, 32, 16\}$ shows a noticeably higher MAE for eight-channel depth inputs than the model that was trained for it. Similar behaviour can be seen for the FOV filtering case as well for the model $\{64, 48, 32\}$ when tasked to generalize for a 16-channel input. There is no such pronounced effect for generalization tasks that lie between two filter settings the model was trained for. At most, it can be observed that models that were trained for a smaller range of filter values perform slightly better than ones that have to cover a wider range. The number of filter settings used in a fixed range does not relevantly influence the model performance, as can be seen, when comparing the two models in Fig.\,\ref{fig:implicit-comp}, which are both trained for a range of 64 to 32 channels but one with three filter settings and the other one with five.
    
    % Figure environment removed
    
    
    % Figure environment removed

% 1. experimental settings with explanation
% 2. main table showing the gap and improvement
% 
% 3. The ablation study shows each part of the method's effectiveness
% 4. The case study shows the usefulness on specific crossing
% 5. 


% 2.5
\vspace{-5mm}
\section{Related Work}
\label{appsec: related work}
Bayesian causal discovery literature has primarily focused on inference in linear models with closed-form posteriors or marginalized parameters. Early works considered sampling directed acyclic graphs (DAGs) for discrete~\cite{cooper1992bayesian, madigan1995bayesian, heckerman2006bayesian} and Gaussian random variables~\cite{friedman2003being, tong2001active} using Markov chain Monte Carlo (MCMC) in the DAG space. However, these approaches exhibit slow mixing and convergence~\cite{eaton2012bayesian,grzegorczyk2008improving}, often requiring restrictions on number of parents~\cite{kuipers2017partition}. %Alternative exact dynamic programming methods are limited to small settings~\cite{koivisto2012advances}. 

Recent advances in variational inference~\cite{zhang2018advances} have facilitated graph inference in DAG space, with gradient-based methods employing the NOTEARS DAG penalty \cite{zheng2018dags}.\cite{annadani2021variational} samples DAGs from autoregressive adjacency matrix distributions, while \cite{lorch2021dibs} utilizes Stein variational approach \cite{liu2016stein} for DAGs and causal model parameters. \cite{cundy2021bcd} proposed a variational inference framework on node orderings using the gumbel-sinkhorn gradient estimator \cite{mena2018learning}. \cite{deleu2022bayesian,nishikawa2022bayesian} employ the GFlowNet framework \cite{bengio2021gflownet} for inferring the DAG posterior. Most methods, except\cite{lorch2021dibs} are restricted to linear models, while \cite{lorch2021dibs} has high computational costs and lacks DAG generation guarantees compared to our method.
% at least quadratic scaling complexity, both with respect to the number of nodes (due to the DAG penalty) as well as number of posterior samples. Our proposed approach instead has linear complexity with respect to number of posterior samples and does not require any additional DAG penalty.     

In contrast, \emph{quasi-Bayesian} methods, such as DAG bootstrap \cite{friedman2013data}, demonstrate competitive performance. DAG bootstrap resamples data and estimates a single DAG using PC \cite{spirtes2000causation}, GES \cite{chickering2002optimal}, or similar algorithms, weighting the obtained DAGs by their unnormalized posterior probabilities. Recent neural network-based works employ variational inference to learn DAG distributions and point estimates for nonlinear model parameters \cite{charpentier2022differentiable,geffner2022deep}.

\vspace{-4.5mm}
\section{Conclusion}
% In this paper, we present the existence of the performance gap in traffic signal control problems and propose an uncertainty-aware grounding action transformation method (\ours) that can dynamically transform actions in the simulation with uncertainty to mitigate the discrepancy between simulated and real-world dynamics. The experiments demonstrate that \ours has an excellent performance in bridging the performance gap with higher efficiency and stability. This research is a step towards improving the real-world applicability of RL-based TSC models, 
In this paper, we identify the performance gap in traffic signal control problems and introduce \ours, an uncertainty-aware grounding action transformation method, to dynamically adapt actions in simulations. Our experiments confirm that \ours effectively reduces the performance gap with improved stability. This work represents progress in enhancing the real-world applicability of RL-based traffic signal control models, our code can be found at  \url{https://github.com/DaRL-LibSignal/UGAT.git}.
\vspace{-5mm}
% \section*{ACKNOWLEDGMENTS}

% \bibliography{software}

% \input{bib}
\bibliographystyle{IEEEtran}
\vspace{-5mm}
\bibliography{software.bib}

\end{document}
