\section{Related Work}
% \paragraph{Traffic Signal Control Methods} 

% Addressing traffic congestion through traffic signal control has been a long-standing challenge in transportation. Rule-based methods~\cite{chen2020toward, dion2002rule} and RL-based methods~\cite{wei2018intellilight, wei2019colight, wei2019presslight} have shown significant improvements over pre-designed time control methods. While many RL-based methods have not addressed the sim-to-real gap, recent studies attempt to bridge this gap by directly modifying the simulator~\cite{mei2022libsignal, muller2021towards, muller2023bridging}. Instead, this paper takes a unique approach by modifying the output of policies learned in the simulator.

\paragraph{Sim-to-real Transfer}
The literature on sim-to-real transfer~\cite{zhao2020sim} can be generally categorized into three groups. The \textbf{\emph{domain randomization}}~\cite{tobin2019real} aims to learn policies that are resilient to changes in the environment. The \textbf{\emph{domain adaptation}}~\cite{tzeng2019deep} tackles the domain distribution shift problem by unifying the source domain and the target domain features that mainly applied in the perception of robots~\cite{james2019sim}, whereas in TSC, the gap is mainly from the dynamics. The \textbf{\emph{grounding methods}}, improve the accuracy of the simulator concerning the real world by correcting simulator bias. 
% Unlike system identification~\cite{6907423, Cully_2015} learning the precise physical parameters, 
Grounded Action Transformation~\cite{hanna2017grounded} induces the dynamics of the simulator to match reality-grounded action,~\cite{IROS20-Desai, IROS20-Karnan, NEURIPS20-Karnan} further explore modeling the stochasticity when grounding,  applying RL, and Imitation from observation (IfO) techniques to advance grounding. Inspired by the above work, \ours leverages uncertainty quantification to enhance action transformation.

% In order to realize the sim-to-real transfer,
% \cite{real-world-robot, inhand} propose to use domain randomization which tries to randomize the simulation to cover the real-world data distribution.
% \cite{domain_confusion, transfer_feature, domain_ada} leverage domain adaptation, which focuses on tackling the domain distribution shift problem by unifying the source domain and the target domain features. Other approaches either focus on system identification to calibrate existing simulators or exploit the simulation experience to improve learning efficiency~\cite{cutler2014,cully2015}. 

% In contrast to those who learn from experience, Grounded Action Transformation (GAT) \cite{gat_aaai} performs all learning in a grounded simulator, it directly optimizes the simulator dynamics within a modified simulator and only uses the physical robot for policy evaluation, eventually shows high feasibility and effectiveness. The above methods are all proposed for specific robotic tasks, while few efforts were made in the TSC domain. Our \ours is the first to solve the TSC sim-to-real problems, it not only benefits from transforming the grounded action but also from quantifying the policy's uncertainty.

% One direction for realizing the sim-to-real is by domain randomization. Instead of carefully modeling all the parameters of the real world, it focuses on highly randomizing the simulation to cover the real-world data distribution \cite{real-world-robot}. \cite{inhand} tries to randomize source domain dynamics by some physical parameters, and \cite{grasp} proposes to translate the randomized simulation domain features and real-world features into the canonical feature, which are proven to be effective, however, sometimes it is hard to select the relevant parameters to randomize, and there exists the risk of creating biased models.

% Another valuable direction is by leveraging the domain adaptation to realize sim-to-real. It aims to unify the source domain features and the target domain ones, to tackle the domain distribution shift problem.  Dividing by different ways of aligning the two domains' feature spaces, there are Discrepancy- based methods to measure the feature distance \cite{domain_confusion, transfer_feature, domain_ada}, reconstruction-based methods intend to find the invariant or shared features between domains by constructing one auxiliary reconstruction task \cite{reconstruct, sim-to-real_survey}.
% % Introduce the domain adaptation and how it would solve the simulation to real-world problems, the current solutions of domain adaption, etc.


\paragraph{Uncertainty Quantification}
Effective uncertainty quantification (UQ) is essential in current deep learning methods to grasp model limitations and enhance model acceleration and accuracy. Gaussian Process (GPs) \cite{seeger2004gaussian} is a non-parametric approach for quantifying uncertainty, while another line of research involves using prior distributions on model parameters to estimate uncertainty during training \cite{xue2019reliable}. Evidential Deep Learning (EDL) \cite{sensoy2018evidential}, MC dropout \cite{gal2017concrete}, and Deep Ensembles \cite{lakshminarayanan2017simple} are representative methods that leverage parametric models. This paper experiments on EDL, MC dropout, and Deep Ensembles to explore their benefits.

% \textcolor{red}{emphazie EDL, MC dropout}
