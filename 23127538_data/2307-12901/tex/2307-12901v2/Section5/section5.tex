
In this section we prove Theorem \ref{ker} and \ref{main}. In particular, we study the topological monodromy homomorphisms of the projective strata $\mathbb{P}\mathcal{H}^{nh}(4)$ and $\mathbb{P}\mathcal{H}(3,1)$. In Theorem \ref{ham} we recall the definition of some standard generators of $\pi_1^{orb}(\mathbb{P}\mathcal{H}^{nh}(4))$ that map via the homomorphisms $$\rho_{\mathbb{P}\mathcal{H}^{nh}(4)}:\pi_1^{orb}(\mathbb{P}\mathcal{H}^{nh}(4))\rightarrow\operatorname{Mod}_{3,1}$$ to Dehn twists. Discussions related to Theorem \ref{ham} are subject of an ongoing work by Calderon-Cuadrado-Salter; for instance, see \cite{Cuadrado2021}. The main ideas that underline Theorem \ref{ham} come from the theory of versal deformation spaces for plane curve singularities \cite{arnold}. For the sake of completeness, we are going to include a similar description of $\rho_{\mathbb{P}\mathcal{H}(3,1)}$ closely following the work of Calderon-Cuadrado-Salter. In particular, we prove that the monodromy $$\rho_{\mathbb{P}\mathcal{H}(3,1)}:\pi_1^{orb}(\mathbb{P}\mathcal{H}(3,1))\rightarrow\operatorname{Mod}_{3,2}$$ is geometric. However, the result of Calderon-Cuadrado-Salter gives explicit generators for $\pi_1^{orb}(\mathbb{P}\mathcal{H}^{nh}(4))$. This finite set of generators arises from the algebro-geometric theory of versal deformation spaces and can be described as a finite set of cylinder shears; see Figure \ref{shear} and \ref{e6trsu}. On the other hand, Theorem \ref{main} is enough to prove the existence of a non-abelian free group of rank $2$ in the kernel of $\rho_{\mathbb{P}\mathcal{H}(3,1)}$.

\begin{proof}[Proof of Theorem \ref{ker}]
    The copy of the non-abelian free group of rank $2$ we constructed in Theorem \ref{thmb} is in the kernel of any geometric homomorphism $A(E_6)\rightarrow\operatorname{Mod}_{3,1}$, but it is also a non-abelian free group in the kernel of the geometric map $A(E_6)_\Delta\rightarrow\operatorname{Mod}_{3,1}$, which is the same as $\rho_{\mathbb{P}\mathcal{H}^{nh}(4)}$. Similarly, the same copy of the non-abelian free group of rank $2$ can be defined in the kernel of the geometric homomorphism $A(E_7)_\Delta\rightarrow\operatorname{Mod}_{3,2}$, which is the same as $\rho_{\mathbb{P}\mathcal{H}(3,1)}$.
    
\end{proof}

\textbf{The monodromy of the stratum $\bm{\mathbb{P}\mathcal{H}^{nh}(4)}$.}  A \textit{cylinder} $\xi$ on a translation surface is an isometric embedding of a Euclidean cylinder whose boundary is a union of saddle connections. In particular, the interior of $\xi$ does not contain any singular point.

If the embedded cylinder $\xi$ is isometric to $(\mathbb{R}/a\mathbb{Z})\times[0,b]$ for some $a,b\in\RR^+$, the core curve of $\xi$ on the translation surface $(X,\omega)$ is the isotopic class of the simple closed curve which is the image of $(\mathbb{R}/a\mathbb{Z})\times\{t\}$ in $(X,\omega)$ for some $t\in(0,b)$. 

Suppose $\xi$ is a horizontal cylinder on a translation surface $(X,\omega)$. In particular, the cylinder $\xi$ can be represented as a rectangle $[0,b]\times[0,a]$ embedded in a defining polygon of $(X,\omega)$ with a pair of sides identified. Suppose that the ratio between its height $a$ and its weight $b$ is $R$. If $t\in[0,R]$, a \textit{cylinder shear} along $\xi$ is the result of the action by the matrix 
\begin{align*}
S_t=
    \begin{bmatrix}
    1& t\\
    0& 1
    \end{bmatrix}
\end{align*}
on the embedded parallelogram of the polygon representative. Analogously, by taking a suitable conjugate of $S_t$ one can define a cylinder shear along non-horizontal cylinders.

% Figure environment removed

Let now $f:\Sigma_g\rightarrow X$ be a marking of $(X,\omega)$. The full shear $S_R$ acts on $(X,f,\omega)$ preserving the translation structure of $X$, as the resulting polygon differs from the initial one by a scissor move, as in Figure (\ref{shear}). However, the matrix $S_R$ changes the marking $f$ by a Dehn twist along the core curve of the cylinder $\xi$. Hence, a cylinder shear is an orbifold loop, and it is mapped via the topological monodromy map of the connected component containing $(X,\omega)$ to a Dehn twist.

The following result is known by experts. It appears as a consequence of Henry Pinkham's thesis \cite{Pinkham} and can also be found in \cite[Proposition 6.2]{ham}. 

\begin{theorem}
\label{ham}
    Let $\{\xi_1,\dots,\xi_6\}$ be a collection of embedded cylinders of a translation surface $(X,\omega)\in\mathcal{H}^{nh}(4)$ such that the family of the associated core curves have an $E_6$-type intersection graph. Then, there exists a map $\Theta:A(E_6)\rightarrow \pi_1^{orb}(\mathbb{P}\mathcal{H}^{hn}(4))$ that associates each standard generator of $A(E_6)$ to a full cylinder shear and can be extended to a well-defined surjective homomorphism with kernel the center of $A(E_6)$. 
\end{theorem}

% Figure environment removed

The homomorphism $\Theta$ is well-defined.  Every pair of adjacent standard generators in $A(E_6)$ is mapped to cylinder shears along embedded cylinders with core curves intersecting once; every pair of standard generators that commute is mapped to cylinder shears along disjoint flat cylinders.

Theorem \ref{ham} shows that $\pi_1^{orb}(\mathbb{P}\mathcal{H}^{nh}(4))$ is generated by a finite family of cylinder shears. However, the group $\pi_1^{orb}(\mathcal{H}^{nh}(4))$ contains orbifold loops that cannot be generated by cylinder shears only. These are loops that cyclically permute the prongs around the singularity. Calderon-Salter \cite[Corollary 7.6]{CalderonSalterFramed2022} showed that there exists an epimorphism $\pi_1^{orb}(\mathcal{H}^{nh}(4))\twoheadrightarrow\ZZ_2$ with kernel containing those orbifold loops that do not permute the prongs at the singularity. In particular, cylinder shears cannot cyclically permute any prong configuration. %However, the map $\mathcal{H}^{nh}(4)\rightarrow\mathbb{P}\mathcal{H}^{nh}(4)$ induces a short exact sequence
%\begin{align}
%\label{LMses}
%%    0\rightarrow\ZZ\rightarrow\pi_1^{orb}(\mathcal{H}^{nh}(4))\rightarrow\pi_1^{orb}(\mathbb{P}\mathcal{H}^{nh}(4))\rightarrow0
%\end{align}
%and any cylinder shear is also an orbifold loop of $\mathcal{H}^{nh}(4)$. Hence, we found a right-inverse homomorphism for $\pi_1^{orb}(\mathcal{H}^{nh}(4))\rightarrow\pi_1^{orb}(\mathbb{P}\mathcal{H}^{nh}(4))$ and prong cyclic permutations are the only missing group elements needed to generate $\pi_1^{orb}(\mathcal{H}^{nh}(4))$, since in this case 
%$$\pi_1^{orb}(\mathcal{H}^{nh}(4))\cong \ZZ \rtimes A(E_6)_\Delta,$$ where the infinite cyclic subgroup is generated by a $\pi$-rotation of the translation surface in Figure \ref{e6trsu} that cyclically permutes the prongs at the singularity of order $4$. However, we are not aware of a similar result for the stratum $\mathcal{H}(3,1)$.



\textbf{The monodromy of the stratum $\bm{\mathbb{P}\mathcal{H}(3,1)}$.} 
Every genus $3$ non-hyperelliptic Riemann surface $X$ can be embedded in $\mathbb{C}\mathbb{P}^2$ as the vanishing locus of a smooth plane quartic \cite[Chapter VII, Proposition 2.5]{Miranda}. Such embedding is defined as the unique projective embedding of $X$ in $\mathbb{C}\mathbb{P}^2$ corresponding to the linear system of positive canonical divisors on $X$, up to linear change of coordinates. By abuse of notations, we identify every genus $3$ non-hyperelliptic Riemann surface $X$ with its image in $\mathbb{C}\mathbb{P}^2$. 

A \textit{flex} of a smooth quartic $X$ is a point $p\in X$ where the intersection multiplicity of $X$ with its tangent space is exactly $3$. A plane quartic $X$ with a flex point $p$ can always be reparametrized in such a way that $p$ is the point at infinity $[0:0:1]$ and its vanishing polynomial is of the form 
$$Q_s=x^3z+y^3x+s_1xyz^2+s_2xz^3+s_3y^4+s_4y^3z+s_5y^2z^2+s_6yz^3+s_7z^4\in\CC[x,y,z],$$
for some $s=(s_1,\dots,s_7)\in\CC^7$ \cite[Proposition 1]{Shioda1993}.

However, there are some strata $\mathcal{H}(k_1,\dots,k_n)$ where all the underlying Riemann surfaces are non-hyperelliptic. This is the case if all the odd numbers in the partition $(k_1,\dots,k_n)$ appear an odd number of times, since every positive canonical divisor on a hyperelliptic Riemann surface is the pullback of a divisor on the Riemann sphere $\mathbb{CP}^1$ \cite[Chapter IV, Proposition 5.3]{hartshorne}. In particular, the Riemann surfaces in the stratum $\mathcal{H}(3,1)$ are all non-hyperelliptic. 

\begin{proposition}

\label{flex}
Let $(X,\omega)$ be a translation surface in $\mathcal{H}(3,1)$. Then $X$ has a flex. In particular, the Riemann surfaces at each point in $\mathcal{H}(3,1)$ are vanishing loci $\mathbb{V}(Q_s)$ of quartics of the form $Q_s$, up to isomorphism.
    
\end{proposition}
\begin{proof}
    Since $X$ is a genus $3$ projective smooth curve, the positive canonical divisors associated with an abelian differential $(X,\omega)\in\mathcal{H}(3,1)$ coincide with divisors coming from lines $L_\omega$ in $\mathbb{CP}^2$ that intersects $X$ in two points. One of these points, say $p$, has multiplicity $3$; then $L_\omega$ is necessarily the tangent line to $X$ in $p$. In particular, $X$ has a flex at $p$ and is isomorphic to the vanishing locus of a quartic $Q_s$.
    
\end{proof}

On the other hand, every smooth vanishing locus $\mathbb{V}(Q_s)$ comes with an abelian differential in $\mathcal{H}(3,1)$ as follows.

The vanishing loci $\mathbb{V}(Q_s)$ are compact Riemann surfaces and the points at infinity can be removed to get a surface diffeomorphic to $\Sigma_{3,2}$. Equivalently, we can evaluate the homogeneous polynomial $Q_s$ at $z=1$ to get a polynomial $q_s\in\CC[x,y]$ and the respective affine vanishing locus $\mathbb{V}(q_s)$ in $\CC^2$.

Since every $\mathbb{V}(q_s)$ is the zero level set of a holomorphic function, the two complex derivatives $\partial_xq_s$ and $\partial_yq_s$ satisfy $\partial_xq_sdx+\partial_yq_sdy=0$. Moreover, the derivatives $\partial_xq_s$ and $\partial_yq_s$ cannot simultaneously vanish since $ \mathbb{V}(q_s)$ is smooth. Hence, the abelian differential 
\begin{align*}
    \omega_{s}(x_0,y_0)=
\begin{cases} 
       \frac{dx}{\partial_yq_s(x_0,y_0)} & \text{ if } \partial_yq_s(x_0,y_0)\neq 0 \\
      -\frac{dy}{\partial_xq_s(x_0,y_0)} & \text{ if } \partial_xq_s(x_0,y_0)\neq 0
   \end{cases}
\end{align*}
is well-defined and non-vanishing at every point $(x_0,y_0)\in\mathbb{V}(q_s)$. In particular, the volume form $\omega_s$ can be holomorphically extended to be zero on the two points at infinity $[1:0:0]$ and $[0:1:0]$, where $\omega_s$ vanishes with multiplicity $3$ and $1$, respectively. 

Looijenga observed that a line intersecting $\mathbb{V}(Q_s)$ with multiplicity $3$ is determined solely by the parameter $s$ \cite[Introduction]{Looijenga93}. In particular, up to a rescaling factor, the abelian differential $\omega_s$ is the unique holomorphic $1$-form on $\mathbb{V}(Q_s)$ such that the pair $(\mathbb{V}(Q_s),\omega_s)$ is a translation surface in $\mathcal{H}(3,1)$. 

In what follows we are going to denote by $\mathcal{M}_3^{\operatorname{flex}}$ the moduli space of non-hyperelliptic genus $3$ Riemann surfaces with $2$ marked points given by a flex $p\in X$ and the unique point of intersection between $T_pX$ and $X$ with multiplicity $1$ (Figure \ref{flexpic}).

% Figure environment removed

\begin{proposition}
\label{lo}
The forgetful map 
    \begin{align*}
        \mathbb{P}\mathcal{H}(3,1)&\rightarrow\mathcal{M}_3^{\operatorname{flex}}\\
        (X,[\omega])&\mapsto X
    \end{align*}
is an isomorphism on orbifolds. In particular, it induces an isomorphism $$\theta_1:\pi_1^{orb}(\mathbb{P}\mathcal{H}(3,1))\rightarrow\pi_1^{orb}(\mathcal{M}_3^{\operatorname{flex}})$$ that commutes with the monodromies $\rho^{\operatorname{flex}}:\pi_1^{orb}(\mathcal{M}_3^{\operatorname{flex}})\rightarrow\operatorname{Mod}_{3,2}$ and $\rho_{\mathbb{P}\mathcal{H}(3,1)}:\pi_1^{orb}(\mathbb{P}\mathcal{H}(3,1))\rightarrow\operatorname{Mod}_{3,2}$ of the respective moduli spaces.
    
    %Let $\mathbb{P}\mathcal{H}^{\operatorname{Teich}}(3,1)$ be the Teichm\"{u}ller cover of the projective stratum $\mathbb{P}\mathcal{H}(3,1)$. If $\mathcal{T}_3^{\operatorname{flex}}$ is the Teichm\"{u}ller cover of $\mathcal{M}_3^{\operatorname{flex}}$, then the map
    %%   \mathbb{P}\mathcal{H}^{\operatorname{Teich}}(3,1)&\rightarrow\mathcal{T}_3^{\operatorname{flex}}\\
     %   (X,f,[\omega])&\mapsto(X,f)
    %\end{align*}
    %is an isomorphism. In particular, there is an isomorphism $\theta_1:\pi_1^{orb}(\mathbb{P}\mathcal{H}(3,1))\rightarrow\pi_1^{orb}%(\mathcal{M}_3^{\operatorname{flex}})$ that commutes with the monodromies of the moduli spaces.
\end{proposition}

\begin{proof}

    Let $\mathbb{P}\mathcal{H}^{\operatorname{Teich}}(3,1)$ be the Teichm\"{u}ller cover of the projective stratum $\mathbb{P}\mathcal{H}(3,1)$. If $\mathcal{T}_3^{\operatorname{flex}}$ is the Teichm\"{u}ller cover of $\mathcal{M}_3^{\operatorname{flex}}$, then the forgetful map
    \begin{align*}
        \mathbb{P}\mathcal{H}^{\operatorname{Teich}}(3,1)&\rightarrow\mathcal{T}_3^{\operatorname{flex}}\\
        (X,f,[\omega])&\mapsto(X,f)
    \end{align*}
    is a bijective quotient map and therefore a homeomorphism. Moreover, the monodromies $\rho^{\operatorname{flex}}$ and $\rho_{\mathbb{P}\mathcal{H}(3,1)}$ share the same image in $\operatorname{Mod}_{3,2}$ as every marking of a Riemann surface in $\mathbb{P}\mathcal{H}(3,1)$ appears as a marking of a Riemann surface in $\mathcal{M}_3^{\operatorname{flex}}$, and viceversa. Therefore, the forgetful map $\mathbb{P}\mathcal{H}^{\operatorname{Teich}}(3,1)\rightarrow\mathcal{T}_3^{\operatorname{flex}}$ induces an orbifold isomorphism and in particular an isomorphism between the respective orbifold fundamental groups.
    
\end{proof}

The collection of parameters $s\in\CC^7$ representing smooth quartics $Q_s$ is an Eilenberg-Maclane space for the spherical-type Artin group $A(E_7)$. In particular, the space $\{s\in\CC^7\mid  \mathbb{V}(Q_s)\text{ is smooth}\}$ has fundamental group isomorphic to $A(E_7)$ and can be homeomorphically realized as the complement of the complexified hyperplane arrangement $\cup_{i\in I}H_i$ of the root system $E_7$ modulo its reflection group $W(E_7)$ \cite[Proposition 9.3]{arnold}. We will briefly describe the homeomorphism between the space $\{s\in\CC^7\mid  \mathbb{V}(Q_s)\text{ is smooth}\}$ and the quotient of $\CC^7\setminus\cup_{i\in I}H_i$ by the group $W(E_7)$, as the construction is going to be used in Lemma \ref{sper}. %For simplicity, the finite Coxeter group $W(E_7)$ will be denoted by $W$.

The $\CC$-algebra of $W(E_7)$-invariant polynomials in $\CC[x_1,\dots,x_7]$ is generated by some homogeneous polynomials $q_1,\dots,q_7$ with degrees $d_i=\operatorname{deg}(q_i)$ uniquely determined by the finite group $W(E_7)$. The basis $\{q_1,\dots,q_7\}$ maps (in a neighborhood of zero) the quotient space $\CC^7/W(E_7)$ to $\CC^7$ by an isomorphism $\tau:\CC^7/W(E_7)\rightarrow\CC^7$ of complex manifolds. In particular, the image of the hyperplane arrangement $\cup_{i\in I}H_i$ modulo $W(E_7)$ is the hypersurface $\Pi=\{s\in\CC^7\mid  \mathbb{V}(Q_s)\text{ is singular}\}$ defined as the vanishing locus of a weighted homogeneous polynomial with weights given by the degrees $(d_1,\dots,d_7)$ of the homogeneous polynomials $\{q_1,\dots,q_7\}$; see, for example, \cite[Introduction]{discriminant} or \cite[Chapter 3]{arnoldbook}. 


The complement $\mathbb{C}^7\setminus{\Pi}$ comes with a surface bundle with fibers diffeomorphic to $\Sigma_3^2$, as follows; for more details, see \cite[Section 2]{Cuadrado2021}. The intersection of the space $\{(p,s)\in\CC^2\times(\CC^7\setminus\Pi)\mid p\in\mathbb{V}(q_s)\}$ with a sufficiently small closed polydisk $\mathbb{D}^2\times\mathbb{D}^7$ in $\CC^2\times\CC^7$ is the total space of a fiber bundle with base space $\mathbb{C}^7\setminus{\Pi}$ and fibers diffeomorphic to $\Sigma_3^2$. It turns out that the monodromy $$\rho:\pi_1(\mathbb{C}^7\setminus{\Pi})\rightarrow\operatorname{Mod}_3^2$$ is a geometric homomorphism.  

%\begin{align}
 %   \{(p,s)\in\CC^2\times(\CC^7\setminus\Pi)\mid p\in\mathbb{V}(q_s)\}\cap\mathbb{D}^2\times\mathbb{D}^7\rightarrow \mathbb{C}^7\setminus{\Pi}
%\end{align}



If we glue a pair of open punctured disks to the boundary components of $\Sigma_3^2$ we obtain a punctured surface diffeomorphic to $\Sigma_{3,2}$. This procedure defines the capping homomorphism $\operatorname{Cap}:\operatorname{Mod}^2_3\rightarrow\operatorname{Mod}_{3,2}$ by extending the mapping classes in $\operatorname{Mod}_3^2$ to the be identity on the glued punctured disks. The proof of Theorem \ref{main} relies on the existence of a surjective homomorphism $$\theta:\pi_1(\CC^7\setminus\Pi)\rightarrow\pi_1^{orb}(\mathbb{P}\mathcal{H}(3,1))$$ such that the two monodromies $\rho:\pi_1(\CC^7\setminus\Pi)\rightarrow\operatorname{Mod}_3^2$ and $\rho_{\mathbb{P}\mathcal{H}(3,1)}:\pi_1^{orb}(\mathbb{P}\mathcal{H}(3,1))\rightarrow\operatorname{Mod}_{3,2}$ fit inside the following commutative diagram

\begin{center}
\begin{equation}
\label{comm}
\begin{tikzcd}
\pi_1(\CC^7\setminus\Pi) \arrow[r, "\theta"] \arrow[d, "\rho"]
& \pi_1^{orb}(\mathbb{P}\mathcal{H}(3,1)) \arrow[d, "\rho_{\mathbb{P}\mathcal{H}(3,1)}"] \\
\operatorname{Mod}_3^2 \arrow[r, "\operatorname{Cap}"]
& \operatorname{Mod}_{3,2}.
\end{tikzcd}  
\end{equation}
\end{center}

Let us define $\theta:\pi_1(\CC^7\setminus\Pi)\rightarrow\pi_1^{orb}(\mathbb{P}\mathcal{H}(3,1))$. We do so by composing two homomorphisms, where one of them has already been given in Proposition \ref{lo}. In what follows, we construct a surjective homomorphism $\theta_2:\pi_1(\CC^7\setminus\Pi)\rightarrow\pi_1^{orb}(\mathcal{M}^{\operatorname{flex}}_3)$. Then, the composition $\theta_1^{-1}\circ\theta_2$ will be the homomorphism $\theta$ we need in order to prove Theorem \ref{main}.

A pair of smooth quartics $Q_s$ and $Q_t$ might define the same isomorphism class of a Riemann surface. This is the case if and only if the parameters $s$ and $t$ are related by a weighted projective relation \cite[Proposition 1]{Shioda1993}. In particular, the vanishing loci $\mathbb{V}(Q_s)$ and $\mathbb{V}(Q_t)$ are isomorphic if and only if there exists $\lambda\in\CC^*$ such that 
\begin{equation}
\label{rel}
(s_1,s_2,s_3,s_4,s_5,s_6,s_7)=(\lambda t_1,\lambda^3t_2,\lambda^4t_3,\lambda^5t_4,\lambda^6t_5,\lambda^7t_6,\lambda^9t_7).
\end{equation}
The above relation is well-defined on $\Pi$. Indeed, the defining weighted polynomial of $\Pi$ has weights compatible with the weights of the relation in (\ref{rel}); we can see it by noticing that the weights given in (\ref{rel}) coincide with half the degrees $(d_1,\dots,d_7)$ of the homogeneous polynomials $q_1,\dots,q_7$ \cite[Section 3.7]{humphr}. In particular, the above relation is also well defined on $\CC^7\setminus\Pi$. 

%Let us denote by $\mathbb{P}_\mathbb{W}(\CC^7\setminus\Pi)$ the weighted projective space obtained from the quotient of $\CC^7\setminus\Pi$ by the relation in (\ref{rel}).

Topologically, the weighted projective space obtained from the quotient of $\CC^7\setminus\Pi$ by the relation in (\ref{rel}) can be realized as a moduli space. In particular, it can be seen as the moduli space $\mathcal{M}^{\operatorname{flex}}_{3,\partial}$ of genus $3$ Riemann surfaces with $2$ boundary components and isomorphism classes given by the fibers of the surface bundle associated with $\CC^7\setminus\Pi$.

\begin{lemma}
\label{sper}
    The quotient map $l:\CC^7\setminus\Pi\rightarrow\mathcal{M}^{\operatorname{flex}}_{3,\partial}$ induces a surjective homomorphism $l_*:\pi_1(\CC^7\setminus\Pi)\rightarrow\pi_1(\mathcal{M}^{\operatorname{flex}}_{3,\partial})$ on the respective fundamental groups. 
\end{lemma}
\begin{proof}

The weighted projective relation defined in (\ref{rel}) on $\CC^7$ pulls back to a projective relation on the quotient $\CC^7/W(E_7)$ via the isomorphism $\tau:\CC^7/W(E_7)\rightarrow\CC^7$. In other words, the isomorphism $\tau$ induces a homeomorphism between the weighted projective space defined by (\ref{rel}) and $\mathbb{CP}^6$ modulo the induced linear action of $W(E_7)$.

Then, the quotient map $l$ can also be seen as the map
\begin{align*}
    l:\faktor{\CC^7\setminus\cup_{i\in I}H_i}{W(E_7)}\longrightarrow\faktor{\mathbb{P}(\CC^7\setminus\cup_{i\in I}H_i)}{W(E_7),}
\end{align*}
where $\mathbb{P}(\CC^7\setminus\cup_{i\in I}H_i)$ is the projectivization of the space $\CC^7\setminus\cup_{i\in I}H_i$. 

The map $l$ descends from the fiber bundle $\CC^7\setminus\cup_{i\in I}H_i\rightarrow\mathbb{P}(\CC^7\setminus\cup_{i\in I}H_i)$ via the free action of the finite group $W(E_7)$ and has connected fibers. In particular, the map $l$ is a fiber bundle with connected fibers and the induced homomorphism on the fundamental groups is surjective by applying the long exact sequence associated with $l$.


%The hypersurface $\Pi$ is the vanishing locus of a homogeneous polynomial and in particular an affine cone in $\CC^7$. The multiplication by $\CC^*$ is well defined on $\Pi$ and hence on its complement $\CC^7\setminus\Pi$. The fiber bundle $\CC^7\setminus\{(0,\dots,0)\}\rightarrow\mathbb{CP}^6$ restricts to a fiber bundle with connected fibers on $\CC^7\setminus\Pi$ and the quotient map $\CC^7\setminus\Pi\rightarrow\mathbb{P}(\CC^7\setminus\Pi)$ induces a surjection on the fundamental groups. 

%Taking a further quotient by a finite group realizes $\mathbb{P}(\CC^7\setminus\Pi)$ as a weighted projective space $\mathbb{P}_\mathbb{W}(\CC^7\setminus\Pi)$ \cite[page 127]{harris2013algebraic}. In particular, the space $\mathbb{P}_\mathbb{W}(\CC^7\setminus\Pi)$ is the quotient of $\mathbb{P}(\CC^7\setminus\Pi)$ by the diagonal action of a finite group $\xi$, product of a finite number of groups of roots of unity. The quotient map $\mathbb{P}(\CC^7\setminus\Pi)\rightarrow\mathbb{P}_\mathbb{W}(\CC^7\setminus\Pi)$ is then open, surjective and proper. Hence, the induced map on the fundamental groups has finite index image.

%Then, the homomorphism $l_*:\pi_1(\CC^7\setminus\Pi)\rightarrow\pi_1(\mathbb{P}_\mathbb{W}(\CC^7\setminus\Pi))$
%has finite index image since it can be realized as the composition of a surjection and a homomorphism with finite index image.

\end{proof}


%Topologically, the space $\mathcal{M}_3^{\operatorname{flex}}$ can be realized as the weighted projective space $\mathbb{P}_\mathbb{W}(\CC^7\setminus\Pi)$.  However, we are interested in the orbifold structure given by the associated Teichm\"{u}ller cover. Its monodromy map is going to be denoted by $\rho^{\operatorname{flex}}:\pi_1^{orb}(\mathcal{M}_3^{\operatorname{flex}})\rightarrow\operatorname{Mod}_{3,2}$.

The Teichm\"{u}ller cover of $\mathcal{M}^{\operatorname{flex}}_{3,\partial}$ will be denoted by $\mathcal{T}_{3,\partial}^{\operatorname{flex}}$ in the following proof.


%However, we are interested in the orbifold structure given by the associated Teichm\"{u}ller cover. For this purpose, we are going to denote by $\mathcal{M}^{\operatorname{flex}}_{3,\partial}$ the orbifold topologically given by $\mathbb{P}_\mathbb{W}(\CC^7\setminus\Pi)$ but with Teichm\"{u}ller cover denoted by $\mathcal{T}_{3,\partial}^{\operatorname{flex}}$.

\begin{proposition}
\label{li}
    Let $\rho^{\operatorname{flex}}:\pi_1^{orb}(\mathcal{M}_3^{\operatorname{flex}})\rightarrow\operatorname{Mod}_{3,2}$ be the monodromy of $\mathcal{M}_3^{\operatorname{flex}}$ and let $\rho:\pi_1(\CC^7\setminus\Pi)\rightarrow\operatorname{Mod}_3^2$ denote the monodromy of $\CC^7\setminus\Pi$.
    There exists a surjective homomorphism $\theta_2:\pi_1(\CC^7\setminus\Pi)\rightarrow\pi_1^{orb}(\mathcal{M}_3^{\operatorname{flex}})$ that commutes with the respective monodromies. In particular, the following diagram commutes

\begin{center}
\begin{equation*}
\label{comm}
\begin{tikzcd}
\pi_1(\CC^7\setminus\Pi) \arrow[r, "\theta_2"] \arrow[d, "\rho"]
& \pi_1^{orb}(\mathcal{M}_3^{\operatorname{flex}}) \arrow[d, "\rho^{\operatorname{flex}}"] \\
\operatorname{Mod}_3^2 \arrow[r, "\operatorname{Cap}"]
& \operatorname{Mod}_{3,2}.
\end{tikzcd}  
\end{equation*}
\end{center}
\end{proposition}

\begin{proof}

The surjective homomorphism $l_*:\pi_1(\CC^7\setminus\Pi)\rightarrow\pi_1(\mathcal{M}^{\operatorname{flex}}_{3,\partial})$ is induced by the quotient map, and in particular it is induced by a bundle map between surface bundles with isomorphic fibers. Therefore, the monodromies of $\CC^7\setminus\Pi$ and $\mathcal{M}^{\operatorname{flex}}_{3,\partial}$ must commute through $l_*$.

The group $\operatorname{Mod}_3^2$ is torsion-free and therefore the orbifold structure of $\mathcal{M}^{\operatorname{flex}}_{3,\partial}$ is not singular. In particular, the orbifold fundamental group of $\mathcal{M}^{\operatorname{flex}}_{3,\partial}$ can be identified with its fundamental group $\pi_1(\mathcal{M}_{3,\partial}^{\operatorname{flex}})$. In other words, if $\rho^{\operatorname{flex}}_{\partial}$ is the monodromy of the moduli space $\mathcal{M}^{\operatorname{flex}}_{3,\partial}$, the diagram

\[
    \begin{tikzcd}[column sep=1em]
\pi_1(\CC^7\setminus\Pi) \arrow[rr, "l_*"] \arrow[dr, "\rho"']& & \pi_1(\mathcal{M}_{3,\partial}^{\operatorname{flex}})\arrow[ld, "\rho^{\operatorname{flex}}_{\partial}"]\\
 & \operatorname{Mod}_3^2
\end{tikzcd}
    \]

must commutes.

Suppose now $\mathcal{T}_{3,\partial}^{\operatorname{flex}}$ and $\mathcal{T}_3^{\operatorname{flex}}$ are the Teichm\"{u}ller covers of $\mathcal{M}^{\operatorname{flex}}_{3,\partial}$ and $\mathcal{M}^{\operatorname{flex}}_3$, respectively. There exists a map $\mathcal{T}_{3,\partial}^{\operatorname{flex}}\rightarrow\mathcal{T}_3^{\operatorname{flex}}$ given by collapsing the lengths of the boundary components to zero. In particular, this map is the restriction of the classic projection given on the respective global Teichm\"{u}ller spaces where the preimage of $\mathcal{T}_3^{\operatorname{flex}}$ is exactly $\mathcal{T}_{3,\partial}^{\operatorname{flex}}$. Hence, the induced map $\pi_1(\mathcal{T}_{3,\partial}^{\operatorname{flex}})\rightarrow\pi_1(\mathcal{T}_{3}^{\operatorname{flex}})$ on the fundamental groups is surjective.

Consider the images of $\rho^{\operatorname{flex}}_{\partial}$ in $\operatorname{Mod}_{3}^2$ and of $\rho^{\operatorname{flex}}$ in $\operatorname{Mod}_{3,2}$. Every marking of a Riemann surface in $\mathcal{M}^{\operatorname{flex}}_3$ appears as the image of a marking associated with a Riemann surface in $\mathcal{M}^{\operatorname{flex}}_{3,\partial}$. Then, the restriction of the homomorphism $\operatorname{Cap}:\operatorname{Mod}_3^2\rightarrow\operatorname{Mod}_{3,2}$ on $\operatorname{im}\rho^{\operatorname{flex}}_{\partial}$ is surjective onto $\operatorname{im}\rho^{\operatorname{flex}}$ and the homomorphism $\pi_1^{orb}(\mathcal{M}^{\operatorname{flex}}_{3,\partial})\rightarrow\pi_1^{orb}(\mathcal{M}^{\operatorname{flex}}_3)$ must be surjective too.

 %\[
 %\begin{tikzcd}
 % 1 \arrow[r] & \pi_1(\mathcal{T}_{3,\partial}^{\operatorname{flex}}) \arrow[d ,->>] \arrow[r] & \pi_1^{orb}(\mathcal{M}^{\operatorname{flex}}_{3,\partial}) \arrow[d] \arrow[r, "\rho^{\operatorname{flex}}_{\partial}"] & \operatorname{im}\rho^{\operatorname{flex}}_{\partial} \arrow[d, "\operatorname{Cap}",->>] \arrow[r] & 1  \\
  %1 \arrow[r] & \pi_1(\mathcal{T}_3^{\operatorname{flex}}) \arrow[r] & \pi_1^{orb}(\mathcal{M}^{\operatorname{flex}}_3) \arrow[r, "\rho^{\operatorname{flex}}"] & \operatorname{im}\rho^{\operatorname{flex}} \arrow[r] & 1 \\
%\end{tikzcd}
%\]



\end{proof}

Our final goal is to prove Theorem \ref{main}. In particular, we will show that the monodromy $\rho_{\mathbb{P}\mathcal{H}(3,1)}$ is geometric. We are going to prove Theorem \ref{main} using the following lemma.% For more details about the following lemma, see \cite{hirshon_1977} or \cite[Lemma 4.1]{bridson2010cofinitely}.

\begin{lemma}
\label{saveass}
Every surjective endomorphism of $A(E_7)_\Delta$ is an isomorphism.
%Let $G$ be a finitely generated and residually finite group with no non-trivial normal finite subgroups. Then, every endomorphism $G\rightarrow G$ with finite index image is an injection. 
\end{lemma}
\begin{proof}
Both $A(E_7)$ and the automorphism group $\operatorname{Aut}(A(E_7))$ are residually finite \cite[Theorem 1]{BaumslagResidually1963} because $A(E_7)$ is linear \cite{Cohen2002}. Hence, the inner subgroup $A(E_7)_\Delta$ of $\operatorname{Aut}(A(E_7))$ is both finitely generated and residually finite. In particular, we can conclude that every surjective endomorphism of $A(E_7)_\Delta$ is an isomorphism \cite[Chapter III, Proposition 7.5]{metricbridson}.

\end{proof}
%The group $A(E_7)_\Delta$ satisfies the hypotheses of Lemma \ref{saveass}. Indeed, both $A(E_7)$ and the automorphism group $\operatorname{Aut}(A(E_7))$ are residually finite \cite[Theorem 1]{BaumslagResidually1963} because $A(E_7)$ is linear \cite{Cohen2002}. Hence, the inner subgroup $A(E_7)_\Delta$ of $\operatorname{Aut}(A(E_7))$ is both finitely generated and residually finite. It is a result of Bestvina that there are no non-trivial normal finite subgroups in $A(E_7)_\Delta$ \cite[Proposition 4.13]{bestvina1999non}.

\begin{proof}[Proof of Theorem \ref{main}] Since $\CC^7\setminus\Pi$ is an Eilenberg-Maclane space for the Artin group $A(E_7)$, the fundamental group $\pi_1(\CC^7\setminus\Pi)$ is isomorphic to $A(E_7)$. Moreover, from Theorem \ref{lomo} we know that $\pi_1^{orb}(\mathbb{P}\mathcal{H}(3,1))$ is isomorphic to $A(E_7)_\Delta$. 

Let us consider the surjective homomorphism $\theta:\pi_1(\CC^7\setminus\Pi)\rightarrow\pi_1^{orb}(\mathbb{P}\mathcal{H}(3,1))$ as the composition $$\pi_1(\CC^7\setminus\Pi)\xrightarrow{\theta_2}\pi^{orb}(\mathcal{M}^{\operatorname{flex}}_3)\xrightarrow{\theta_1^{-1}}\pi_1^{orb}(\mathbb{P}\mathcal{H}(3,1)).$$

The group $A(E_7)_\Delta$ is centerless. Therefore, the kernel of the homomorphism $\theta:A(E_7)\rightarrow A(E_7)_\Delta$ contains the subgroup $\langle\Delta\rangle$. This implies that the induced map $$\overline{\theta}:A(E_7)_\Delta\rightarrow A(E_7)_\Delta$$
is a well-defined surjective endomorphism of $A(E_7)_\Delta$ and therefore an isomorphism by Lemma \ref{saveass}. 

Since $\theta$ commutes with the monodromies $\rho:\pi_1(\CC^7\setminus\Pi)\rightarrow\operatorname{Mod}_3^2$ and $\rho_{\mathbb{P}\mathcal{H}(3,1)}:\pi_1^{orb}(\mathbb{P}\mathcal{H}(3,1))\rightarrow\operatorname{Mod}_{3,2}$ through the capping homomorphism $\operatorname{Cap}:\operatorname{Mod}_3^2\rightarrow\operatorname{Mod}_{3,2}$, we can conclude that $\rho_{\mathbb{P}\mathcal{H}(3,1)}$ is geometric since $\rho$ is.

\end{proof}





%\begin{lemma}
%\label{saveass}
%Let $G$ be a finitely generated and residually finite group with no non-trivial normal finite subgroups. Then, every endomorphism $G\rightarrow G$ with finite index image is an injection. 
%\end{lemma}




