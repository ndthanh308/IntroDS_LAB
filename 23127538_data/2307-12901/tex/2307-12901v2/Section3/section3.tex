In this section we construct a non-abelian free subgroup of rank 2 in the kernel of any geometric homomorphism of Artin group with defining graph containing $E_6$.

Recall that $\kappa\in A(\Gamma)_\Delta$ is the loxodromic isometry of $\mathcal{C}_{AL}(E_6)$ in (\ref{kappa}). In view of the Abbott-Dahmani result \cite[Proposition 2.1]{AbbottDahmaniPnaive2019}, we show that the Wajnryb element $w$ is an elliptic isometry of $\mathcal{C}(E_6)$, that none of its powers preserve the quasi-axis $A_{10\delta}(\kappa)$ and that $\operatorname{Fix}_{50\delta}(w)$ is a bounded set. It will follow that there is a power $n\in\ZZ$ such that the subgroup $\langle w,\kappa^n\rangle$ is a non-abelian free group of rank $2$. 

\begin{proof}[Proof of Theorem \ref{thmb}]
Let $\Omega$ be a collection of isotopy classes of non-essential simple closed curves on $\Sigma_g^b$ that pairwise intersect at most once, and suppose that its intersection graph $\Lambda_\Omega$ contains $E_6$. 

The hypotheses of  the Abbott-Dahmani result \cite[Proposition 2.1]{AbbottDahmaniPnaive2019} are satisfied for $w$ and $\kappa$ by Lemma \ref{l1},  Lemma \ref{l2} and  Lemma \ref{l3} below. However, the loxodromic isometry $\kappa$ is not in the kernel of $\varphi_\Omega$. Nevertheless, if we denote by $w^{\kappa^n}$ the conjugate $\kappa^{-n}w\kappa^n$, the group $\langle w,w^{\kappa^n}\rangle$ is contained in $\ker\varphi_\Omega$ and it is also isomorphic to $F_2$, as any combination of letters in $\{w,w^{\kappa^n}\}$ that represents a trivial word is also a combination of letters in $\{w,\kappa^n\}$.

\end{proof}

\begin{lemma}
\label{l1}
The projection of $w$ in $A(E_6)_\Delta$ is an elliptic isometry of the additional length graph $\mathcal{C}_{AL}(E_6)$.
\end{lemma}
\begin{proof}
We would like to apply the Antolin-Cumplido criterion from Theorem \ref{ell}. It is enough to show that the subgroup $\langle a_1,b \rangle$ normalizes the parabolic subgroup $\langle a_2,a_5\rangle$. The action of $b$ by conjugation on $A(E_6)$ permutes $a_2$ and $a_5$ (see Figure \ref{bnorm}). Since $a_1$ is in the centralizer of both $a_2$ and $a_5$, we can conclude that the group generated by $a_1$ and $b$ preserves the $2$-simplex $\{\langle a_2\rangle\ ,\langle a_5\rangle\}$ of the complex $\mathcal{P}(E_6)$.

\end{proof}

% Figure environment removed
%We recall that points on the Gromov boundary of a $\delta$-hyperbolic space are equivalence classes of sequences
%$(x_n)_{n\in\NN}$ in $X$, where twosequences are identified if and only if their difference is uniformly bounded. 
\begin{lemma}
\label{l2}
    No non-trivial power of the Wajnryb element $w\in A(E_6)$ preserves the $10\delta$-quasi fixed axis $A_{10\delta}(\kappa)$ of $\kappa$.
\end{lemma}
\begin{proof}
    Let $x\in\operatorname{A}_{10\delta}(\kappa)$ be a vertex of $\mathcal{C}_{AL}(E_6)$. If we suppose that $w$, or any of its non trivial power, preserves $\operatorname{A}_{10\delta}(\kappa)$ we would have that
\begin{align*}
    d(w\kappa^nx,\kappa^nx)&\leq d(w\kappa^nx,\kappa^nwx)+d(\kappa^nw x,\kappa^nwx)& \text{(triangular inequality)}\\
    &=d(w^{\kappa^n} x, w x)+d(w x, x)&(\kappa\text{ is an isometry})\\
    &\leq  d(w^{\kappa^n} x, x)+2d(w x, x)&\text{(triangular inequality)}\\
    &\leq \inf_{y\in\mathcal{C}_{AL}(E_6)}d(y,\kappa y)+10\delta+2d(wx, x),&\text{(definition of }A_{10\delta}(\kappa))
\end{align*}
for any $n\in\ZZ$, where the last inequality follows from the fact that also $\kappa$ preserves $\operatorname{A}_{10\delta}(\kappa)$. However, the inequality (\ref{tech}) implies that it cannot happen, as $\kappa$ is loxodromic.

\end{proof}

Every spherical-type Artin group has a finite $K(\pi,1)$ space given by the complement of a hyperplane arrangement associated with the respective Coxeter group (see, for example, \cite{Deligne1972}). In particular, the Artin group $A(E_6)$ is torsion-free \cite[Proposition 2.45]{hatcher2002algebraic}. However,  the quotient $A(E_6)_\Delta$ has torsion elements but the Wajnryb element $w$ is not a periodic isometry of $\mathcal{C}_{AL}(E_6)$. 

In order to prove the following lemma, we recall that  standard generators $\{a_1,\dots,a_n\}$ of an Artin group $A(\Gamma)$ are related by length-preserving relations and the map \begin{align*}
    \operatorname{deg}:A(\Gamma)&\rightarrow\ZZ \\
    a_{i_1}^{n_1}\dots a_{i_k}^{n_k}&\mapsto \sum_{j=1}^k n_k
\end{align*}
is a homomorphism. More precisely, the commutator subgroup of $A(\Gamma)$ is exactly the kernel of the length homomorphism $\operatorname{deg}:A(\Gamma)\rightarrow\ZZ$ \cite[Proposition 3.1]{Mulholland2002}. 

\begin{lemma}
\label{l3}
    The Wajnryb element $w$ is not torsion in $A(E_6)_\Delta$.
\end{lemma}
\begin{proof}
Suppose there is some $m\in\ZZ$ such that $w^m$ is central in $A(E_6)$ and can be written as $\Delta^{k}$ for some integer $k$. The degree $\operatorname{deg}(w)$ is zero and therefore we can write $w$ as a commutator. However, the Garside element of $A(E_6)$ is $\Delta=(a_1a_3a_5a_2a_4a_6)^6$ and has positive length. Hence, we have that$$0=\operatorname{deg}(w^m)=\operatorname{deg}(\Delta^{k})=k\cdot\operatorname{deg}(\Delta)$$ and $k$ is then forced to be equal to zero. Since $A(E_6)$ is torsion-free, the only possibility for the  $m^{th}$-power of $w$ to be trivial is that $m=0$. 

\end{proof}

The set $\operatorname{Fix}_{50\delta}(w)$ is then necessarily bounded.

\begin{lemma}
    Let $G$ be a group acting acylindrically on a $\delta$-hyperbolic space $X$. If $\operatorname{Fix}_{K}(g)$ is unbounded, then $g$ has finite order.   
\end{lemma}
\begin{proof}
    Let $x,y\in \operatorname{Fix}_{K}(g)$ be two points of $X$ such that $d(x,y)$ is greater than the constant $R(K)$ from the definition of acylindrical hyperbolicity of a group (Theorem \ref{acyartin}). Then, the set $\Gamma_K(x,y)$ is finite and contains any power of $g$. 
    
\end{proof}