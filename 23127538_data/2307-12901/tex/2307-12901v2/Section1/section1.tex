
Artin groups are finitely presented groups where the generators and the relations are given by a finite graph, as in (\ref{presArt}). For example, a braid group $\mathcal{B}_n$ is an Artin group with defining graph $A_{n-1}$, as in Figure \ref{spherical}. 

The quotient of a braid group $\mathcal{B}_n$  by the subgroup normally generated by the squares of the standard generators is the symmetric group of size $n$ \cite[Section 9.3]{farb2011primer} . Similarly, any Artin group $A(\Gamma)$ comes with a Coxeter group $W(\Gamma)$ given by the additional relations $a_i^2=1$ for every standard generator $a_i$. An Artin group $A(\Gamma)$ is of \textit{spherical-type} if $W(\Gamma)$ is finite. 
% Figure environment removed

Spherical-type Artin groups are well understood and, for example, admit a solution to the word problem. The algorithm is given by the Garside structure; see \cite{BrieskornArtin1972} for more details.

\textbf{Garside groups.} Let $G$ be a finitely generated group and $G^+$ the submonoid generated by the same finite generating set of $G$. Furthermore, suppose that $G^+$ trivially intersects $(G^+)^{-1}$. The prefix order on $G$ is the partial order $(G,\preceq)$ where  $a\preceq b$ if and only if $a^{-1}b\in G^+$.

If $\mathcal{A}$ is the set of elements in $G^+$ that cannot be written as a product of other non-trivial elements of $G^+$, the monoid $G^+$ is \textit{Noetherian} if for every $x\in G^+$ we have that $\sup\{n\in\NN\mid x=a_1\dots a_n, a_i\in\mathcal{A}\}$ is finite. Then, the group $G$ is \textit{Garside} if its monoid $G^+$ is Noetherian, the prefix order admits greater common divisors and lower common multiples and there exists  an element $\Delta\in G^+$ such that:
\begin{itemize}
    \item the conjugation action by $\Delta$ fixes the monoid $G^+$;
    \item the set $\{s\in G\mid 1\preceq s\preceq \Delta\}$ of simple elements is finite and generates $G$.
\end{itemize}

All the above assumptions guarantee that every $x\in G$ in a Garside group can be uniquely written in its \textit{(left) normal form} as $x=\Delta^ks_1\dots s_n,$ where each $s_i$ is a simple element that is not $\Delta$, and for every pair $\{s_i,s_{i+1}\}$ of adjacent simple elements the greater common divisor between $s_is_{i+1}$ and $\Delta$ is exactly $s_i$. The integer $k$ is denoted by $\inf(x)$, while $n$ is denoted by $\sup(x)$. 

Similarly, the suffix order on a Garside group $G$ provides the group with right normal forms, where elements in $G$ can be uniquely written in the form $t_1\dots t_n\Delta^k$ with $t_i$ simple elements different from $\Delta$; see \cite[Section 2]{AntolinCumplidoParabolic21} for more details. In what follows, we will only adopt the prefix order.

We say that $x\in G$ absorbs $y\in G$ if either $\sup(y)=0$ or $\inf(y)=0$ and both the equalities $\sup(xy)=\sup(x)$ and $\inf(xy)=\inf(x)$ hold. In this case, $y$ is absorbed by $x$ and we say that the group element $x$ is \textit{absorbable}. 

Spherical-type Artin groups admit a Garside structure through the submonoid $A(\Gamma)^+$ generated by the same standard generators of $A(\Gamma)$. Simple elements of $A(\Gamma)^+$ are words with free-square subwords. Absorbable elements are not classified but, for example, if $A(\Gamma)$ is the Braid group $\mathcal{B}_4$, any $n$-th powers of a generator absorb the $n$-th power of a non-adjacent generator. More precisely, we can write $$\sigma_1^n\sigma_3^n=(\sigma_1\sigma_3)^n,$$ and observe that $\sigma_1^n$ absorb $\sigma_3^n$.

The \textit{Garside element} $\Delta$ is the least common multiple of all the standard generators and any spherical type Artin group has infinite cyclic center generated by a power of $\Delta$ \cite[Théorème 7.1]{BrieskornArtin1972}. In what follows, we denote by $A(\Gamma)_\Delta$ the $\Gamma$-type spherical Artin group modulo its center. 

\textbf{Acylindrical hyperbolicity.} Let $\mathcal{S}$ be the set of simple and absorbable elements of a spherical-type Artin group $A(\Gamma)$. The vertices of the \textit{additional length graph} $\mathcal{C}_{AL}(\Gamma)$ are the left cosets of the subgroup $\langle\Delta\rangle$ in $A(\Gamma)$. Two cosets $g_1\langle\Delta\rangle$ and $g_2\langle\Delta\rangle$ are adjacent if they differ by the left multiplication of some element in $\mathcal{S}\setminus\{\Delta\}$.  As usual, the graph comes equipped with the metric where each edge has length one.

Calvez-Wiest proved that $\mathcal{C}_{AL}(\Gamma)$ is a $\delta$-hyperbolic geodesic metric space \cite[Theorem 1]{CalvezWiestCurve2017} and $A(\Gamma)_\Delta$ is an \textit{acylindrically hyperbolic} group for its action on $\mathcal{C}_{AL}(\Gamma)$  \cite[Theorem 1.3]{CalvezWiestAcyArt2017}. More precisely, they proved the following theorem.

\begin{theorem}
\label{acyartin}
If $A(\Gamma)$ is a spherical-type Artin group, then the action of $A(\Gamma)_{\Delta}$ on $\mathcal{C}_{AL}(\Gamma)$ is cobounded and non-elementary. Moreover, for every $\varepsilon>0$ there is a positive real numbers $R(\varepsilon)$ such that for each $x,y\in \mathcal{C}_{AL}(\Gamma)$ with $d(x,y)>R(\varepsilon)$, the set 
$$\Gamma_{\varepsilon}(x,y)=\{g\in A(\Gamma)_\Delta\mid d(x,gx)<\varepsilon, d(y,gy)<\varepsilon\}$$
is finite. 
\end{theorem}

It follows from a more general theorem \cite[Theorem 1.1]{Osin2016} that every $g\in A(\Gamma)_\Delta$ is either a \textit{loxodromic} or an \textit{elliptic} isometry of $\mathcal{C}_{AL}(\Gamma)$. In other words, either
\begin{itemize}
    \item the map $n\mapsto g^n\cdot x$ is a quasi-isometry between $\ZZ$ and the orbit of some (equivalently any) point $x\in\mathcal{C}_{AL}(\Gamma)$, and then $g$ is \textit{loxodromic}, or
    \item the action by $g$ has bounded orbits, and $g$ is then\textit{ elliptic}.
\end{itemize}

Any loxodromic element $g\in A(\Gamma)_\Delta$ is always \textit{weakly properly discontinuous}: for every $\varepsilon>0$ and $x\in \mathcal{C}_{AL}(\Gamma)$ there exists some $n\in\ZZ$ such that the set
$$\{g\in G\mid d(x,gx)<\varepsilon, d(\kappa^nx,g\kappa^nx)<\varepsilon\}$$ is finite. It turns out that the existence of a loxodromic and weakly properly discontinuous group element is also a sufficient property to show that a non virtually-cyclic group is acylindrically hyperbolic \cite[Theorem 1.2]{Osin2016}. Calvez-Wiest proved Theorem \ref{acyartin} by showing that 
\begin{equation}
\label{kappa}
   \kappa=a_4a_1a_3a_2a_4a_5a_4a_1a_3a_2a_6a_5a_5a_6a_2a_3a_1a_4a_5a_4a_2a_3a_1a_4 
\end{equation}

projects to a loxodromic and weakly properly discontinuous isometry in $A(E_6)_\Delta$.

However, there is no known sufficient and necessary criterion to determine if a given isometry of an acylindrically hyperbolic group is loxodromic or elliptic. Nevertheless, Antolin-Cumplido gave a sufficient condition for an isometry of the additional length graph to have bounded orbits \cite[Theorem 2]{AntolinCumplidoParabolic21}.  We will describe this criterion below. 

A \textit{parabolic subgroup} $P$ of an Artin group $A(\Gamma)$ is the conjugate of a subgroup generated by some strict subset of the standard generators. If $P$ is not a direct product of non-trivial parabolic subgroups, we say that it is \textit{irreducible}. The complex of irreducible parabolic subgroups $\mathcal{P}(\Gamma)$ is defined to have irreducible parabolic subgroups as vertices. A set of vertices $\{P_1,\dots,P_n\}$ is an $n$-simplex if one of the following properties is satisfied for all $i\neq j$:
\begin{itemize}
    \item $P_i\subset P_j$ or $P_j\subset P_i$;
    \item $P_i\cap P_j=\{1\}$ and $[P_i,P_j]=1$.
\end{itemize}
The complex $\mathcal{P}(\Gamma)$ can detect elliptic isometries of $\mathcal{C}_{AL}(\Gamma)$.

\begin{theorem}
\label{ell}
Suppose $A(\Gamma)$ is an irreducible spherical-type Artin group with more than two standard generators. The elements preserving some simplex of $\mathcal{P}(\Gamma)$ act elliptically on $\mathcal{C}_{AL}(\Gamma)$.  In particular, the normalizers of parabolic subgroups act elliptically on $\mathcal{C}_{AL}(A)$.
\end{theorem}

In Section $3$, we are also going to use a technical lemma borrowed from Antolin-Cumplido paper \cite[Lemma 25]{AntolinCumplidoParabolic21}. This lemma gives the following estimate for $g\in A(E_6)$ infinite order element in the normalizer of a proper standard parabolic subgroup and $x\in\mathcal{C}_{AL}(E_6)$:
\begin{equation}
\label{tech}
    d(g\kappa^n x,\kappa^nx)\geq d(x,\kappa^n x)+K,
\end{equation}
for some constant $K>0$ and $\lvert n\rvert$ big enough. 
