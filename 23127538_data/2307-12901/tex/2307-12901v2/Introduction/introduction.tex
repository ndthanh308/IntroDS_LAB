
Translation surfaces and their moduli spaces naturally arise in the interplay of topology, algebraic geometry, dynamics and number theory as shown through the work of Veech \cite{Veech}, Masur \cite{Masur82}, Thurston \cite{Thurston} and many subsequent authors. The topology of the moduli spaces of Riemann surfaces is somewhat understood; see, for example, the work of Harer-Zagier \cite{HarerZagier} and Maclachlan \cite{Maclachlan}. Much less is known about the topology of the moduli space of translation surfaces.

In this article, we analyze the topological monodromy map of the connected components $\mathcal{H}^{nh}(4)$ and $\mathcal{H}(3,1)$ of the moduli space of translation surfaces in genus $3$. We show that the topological monodromy maps from the orbifold fundamental groups onto the images in the respective mapping class groups are far from being isomorphisms. In particular, we prove that the kernels contain a non-abelian free group of rank $2$ by relating the topological monodromy maps to some geometric homomorphisms of Artin groups.  

\textbf{Translation surfaces.} Let $\Sigma_g$ denote a closed oriented surface of genus $g$ and let $\mathcal{Z}\subset\Sigma_g$ be a finite set of points. A \textit{translation structure} on $\Sigma_g$ is an atlas of charts with values in $\CC$ where the transition maps of $\Sigma_g\setminus\mathcal{Z}$ are translation, points in $\mathcal{Z}$ are cone type singularities and the holonomy $\pi_1(\Sigma_g\setminus\mathcal{Z})\rightarrow \operatorname{SO}(2)$ is trivial. In particular, the complex structure on $\Sigma_g\setminus\mathcal{Z}$ can be extended to $\Sigma_g$ by Riemann's removable singularity theorem and the metric around each point $p\in\mathcal{Z}$ can be given by cyclically gluing half-planes around $p$.

A translation structure on $\Sigma_g$ can also be given by pairs of the form $(X,\omega)$, where $X$ is genus $g$ Riemann surface and $\omega$ is a non-zero holomorphic one form on $X$. The finite set $\mathcal{Z}$ is identified with $\mathcal{Z}(\omega)=\{p\in X\mid\omega_p\equiv0\}$. Since the holonomy around every cone singularity is trivial, the number $k_p$ of half-planes glued around each point $p\in\mathcal{Z}$ is even. The multiplicity of $\omega$ at the respective vanishing point is exactly $\frac{k_p}{2}+1$.

\textbf{Strata of translation surfaces.} The moduli space of genus $g$ translation surfaces is the set of all translation structures $(X,\omega)$ of $\Sigma_g$ up to isomorphisms. The whole moduli space can be stratified in orbifolds $\mathcal{H}(k_1,\dots,k_n)$ characterized by the combinatorial data given by the orders of $\omega$ at its zeros. 

Even though the topology of the strata of translation surfaces is poorly understood, our knowledge has improved in the past years. Costantini-M\"{o}ller-Zoachhuber gave a recursive computable formula for the Euler characteristic of the moduli space of translation surfaces \cite{Costantini2022}. Further, Zykoski has constructed a finite simplicial complex with the same homotopy type of the strata $\mathcal{H}(k_1,\dots,k_n)$, motivated by Harer's construction of a simplicial complex those quotient by the mapping class group is homotopic equivalent to the moduli space of Riemann surfaces \cite{zykoski2022isodelaunay}. 

Kontsevich-Zorich showed that each stratum has at most $3$ connected components and in every genus some components are \textit{hyperelliptic} \cite{Kontsevich2003}. Namely, hyperelliptic components consist of translation surfaces $(X,\omega)$ where $X$ is a hyperelliptic Riemann surface and $\iota^*(X,\omega)=(X,-\omega)$ where $\iota$ is the hyperelliptic involution of $X$. These connected components are orbifold classifying spaces for finite extensions of braid groups; see \cite[Section 1.4]{Looijenga2014} for a proof. Kontsevich-Zorich also conjectured that the other non-hyperelliptic components have orbifold fundamental groups commensurable with some mapping class group \cite{Kontsevich1997}; this conjecture is still open.

Our focus is to shed some light on the Kontsevich-Zorich conjecture for some non-hyperelliptic components in small complexity, by showing that the topological monodromy maps of some exceptional connected components in the mapping class group are far from being injective.

\textbf{Monodromy maps.} Let $\Sigma_{g,n}$ be a closed surface with $n$ marked points. The mapping class group $\operatorname{Mod}_{g,n}$ is the group of all orientation preserving self-diffeomorphism of $\Sigma_{g,n}$ that leave the set of marked points invariant, up to isotopies relative to the set of marked points. If $\mathcal{C}$ is a connected component of a stratum $\mathcal{H}(k_1,\dots,k_n)$, then any (orbifold) homotopy class of loops based at $(X,\omega)$ gives rise to some self-diffeomorphism of $X$ that preserves the zeros of $\omega$. These data are recorded by the (punctured) topological monodromy map:
$$\rho_\mathcal{C}:\pi_1^{orb}(\mathcal{C})\rightarrow\operatorname{Mod}_{g,n}.$$

Calderon studied these homomorphisms and described the connected components of the strata of \textit{marked translation surfaces} for genus $g\geq 5$, which cover the strata of translation surfaces  \cite{CalderonConnected2020}.  Then, Calderon-Salter's work resulted in a complete description of the images of the monodromy maps associated with all non-hyperelliptic connected components of the strata $\mathcal{H}(k_1,\dots,k_n)$ in genus $g\geq 5$. In other words, the orbifold fundamental groups of all non-hyperelliptic connected components in genus $g\geq 5$ are projected onto subgroups of the mapping class group called \textit{framed mapping class groups} \cite{CalderonSalterFramed2022}. 

\textbf{The kernel of the punctured monodromy.} If $\mathcal{C}$ is hyperelliptic then $\operatorname{Im}\rho_\mathcal{C}$ is isomorphic to the symmetric mapping class group $\operatorname{SMod}_{g,n}$, and $\ker\rho_\mathcal{C}$ is finite \cite[Section 2.1]{CalderonConnected2020}. In view of Kontsevich-Zorich conjecture, it is natural to ask whether or not the topological monodromy is the right homomorphism to look at in order to prove the conjecture. For this reason, we are interested in estimating the size of the kernels of the monodromies $\rho_\mathcal{C}$ for non-hyperelliptic connected components. The first main result of this paper is that in some cases the kernel is large.

\begin{alphateo}
\label{ker}
    Let $\rho_{\mathcal{H}^{nh}(4)}:\pi_1^{orb}(\mathcal{H}^{nh}(4))\rightarrow\operatorname{Mod}_{3,1}$ and $\rho_{\mathcal{H}(3,1)}:\pi_1^{orb}(\mathcal{H}(3,1))\rightarrow\operatorname{Mod}_{3,2}$ be the topological monodromy maps of the non-hyperelliptic connected components of $\mathcal{H}(4)$ and of $\mathcal{H}(3,1)$, respectively. The kernels of both the monodromies contain a non-abelian free group of rank 2. 
\end{alphateo}

The orbifold fundamental groups involved in the statement of  Theorem \ref{ker} are closely related to Artin groups. Looijenga-Mondello showed that the groups $\pi_1^{orb}(\mathcal{H}^{nh}(4))$ and $\pi_1^{orb}(\mathcal{H}(3,1))$ are infinite-cyclic central extensions of the inner automorphism groups of some Artin groups \cite{Looijenga2014}. It turns out that Theorem \ref{ker} is an example of a more general phenomenon related to \textit{geometric homomorphisms} from Artin groups to mapping class groups.

\textbf{Geometric homomorphisms.} If $\Gamma$ is a finite, connected and undirected simple graph with $\mathcal{V}(\Gamma)$ as its set of vertices, an {Artin group} is a group that admits a presentation of the following form
\begin{align}
    A(\Gamma):=\Bigg\langle a_1,\dots,a_n\in\mathcal{V}(\Gamma)\Bigg\mid
    \text{ }
    \begin{matrix}
        a_ia_ja_i=a_ja_ia_j & \text{if }a_i\text{ and }a_j\text{ are adjacent}\\
        a_ia_j=a_ja_i & \text{otherwise}
    \end{matrix}
    \text{ }
    \Bigg\rangle. \label{presArt} 
\end{align}

Roughly speaking, a geometric homomorphism $A(\Gamma)\rightarrow\operatorname{Mod}_{g,n}$ arises as the correspondence between the vertices of the defining graph $\Gamma$ and a family of simple closed curves on the surface $\Sigma_{g,n}$.  The standard Artin generators in the presentation (\ref{presArt}) map to Dehn twists about curves that respect the intersection pattern given by the defining graph; see Figure \ref{perron1}.


% Figure environment removed

Possibly, there might exist relations between Dehn twists that do not hold for standard generators of Artin groups. However, there is no known algorithm that can solve the word problem for a generic Artin group (for further details, see, for example \cite[Conjecture 5.2]{McCammond2017}), and this is the main obstruction to characterize kernels of geometric homomorphisms. 

Wajnryb proved that if the graph $\Gamma$ contains $E_6$ as a subgraph, any geometric homomorphism cannot be an injection \cite{Wajnryb1999}. In particular, Wajnryb found an element $w$ given explicitly in terms of the standard generators in the presentation (\ref{presArt}) and adopted the following strategy: as every inclusion of graphs induces a monomorphism of the respective Artin groups \cite{lek}, it is enough to find a non-trivial element $w$ in $A(E_6)$ which can be written in $\operatorname{Mod}_{3,1}$ as a braid relator of Dehn twists. The group $A(E_6)$ is a spherical-type Artin group, a class of groups for which the word problem has been solved by means of their \textit{Garside structure}. Our next result builds on Wajnryb's work and Theorem \ref{ker} can be thought of as a corollary of the following theorem.

\begin{alphateo}
\label{thmb}
    Let $\Gamma$ be any finite and undirect simple graph with $E_6$ as a subgraph. Any geometric homomorphism of $A(\Gamma)$ in $\operatorname{Mod}_{g,b}$ has a large kernel that contains a non-abelian free group $F_2$ of rank $2$. In particular, there is some $g\in A(\Gamma)$ such that $F_2$ is generated by the Wajnryb element $w$ and its conjugate $g^{-1}wg$.
\end{alphateo}

Theorem \ref{thmb} follows from the acylindrical hyperbolicity of spherical-type Artin groups modulo their center. Here, the Ping-Pong strategy can be adopted to detect non-abelian free groups.

\textbf{Acylindrical hyperbolicity.} Let $A(\Gamma)_\Delta$ denote the spherical-type Artin group $A(\Gamma)$ quotient by its center. Calvez-Wiest proved that the group $A(\Gamma)_\Delta$ acts \textit{acylindrically} on a $\delta$-hyperbolic graph, which is known in the literature as the \textit{additional length graph} $\textit{C}_{AL}(\Gamma)$ \cite[Theorem 1.3]{CalvezWiestAcyArt2017}. 

Calvez-Wiest found a group element $\kappa\in A(\Gamma)$ representing a loxodromic isometry of $\textit{C}_{AL}(\Gamma)$ that acts weakly properly discontinuously. By Osin's criterion \cite[Theorem 1.2]{Osin2016} the existence of the \textit{Calvez-Wiest} element $\kappa$ is enough to conclude the acylindrical hyperbolicity of $A(\Gamma)_\Delta$.

We prove that the infinite order Wajnryb element acts elliptically on $\textit{C}_{AL}(\Gamma)$ and a classical result shows that $w\in A(\Gamma)_\Delta$ cannot fix $\kappa\in A(\Gamma)_\Delta$ in the Gromov boundary of the additional length graph \cite[Lemma 25]{AntolinCumplidoParabolic21}. The following is due to Abbott-Dahmani and is the key ingredient we need to prove Theorem \ref{thmb}.

\begin{proposition*}[{\cite[Proposition 2.1]{AbbottDahmaniPnaive2019}}]
\label{quasi}
Let $G$ be a group acting acylindrically hyperbolic on a geodesic $\delta$-hyperbolic space $X$. Suppose $\sigma\in G$ is elliptic and $\gamma\in G$ is loxodromic. If 
\begin{enumerate}
    \item the set $\operatorname{A}_{10\delta}(\gamma)=\{x\in X\mid d(x,\gamma x)\leq \inf_{y\in X}d(y,\gamma y)+10\delta\}$ is not preserved by any non-trivial power of $\sigma$ and
    \item the diameter of $\operatorname{Fix}_{50\delta}(\sigma)=\{x\in X\mid d(x,\sigma^nx)\leq 50\delta\mbox{ for all } n\in\ZZ\}$ is finite,
\end{enumerate}
then there is some $n\in\ZZ$ such that the group generated by $\sigma$ and $\gamma^n$ is a non-abelian free group of rank $2$.
\end{proposition*}

We conclude that there exists a positive integer $n$ such that the group generated by $\kappa^{-n}w\kappa^n$ and $w$ is a non-abelian free group of rank $2$. 

\textbf{Projective strata.} We now explain how Artin groups arise in the context of the non-hyperelliptic components of the strata mentioned in Theorem \ref{ker}. 

The multiplicative group $\CC^*$ acts on the cotangent bundle of each Riemann surface $X$ by multiplication. The action preserves the multiplicity at the cone points of each holomorphic $1$-form and is well-defined of each connected component $\mathcal{C}$ of a stratum $\mathcal{H}(k_1,\dots,k_n)$. The resulting quotient is denoted by $\mathbb{P}\mathcal{C}$ and is known as a \textit{projective stratum} of translation surfaces. 

Looijenga-Mondello proved that the orbifold fundamental groups of $\mathbb{P}\mathcal{H}^{nh}(4)$ and $\mathbb{P}\mathcal{H}(3,1)$ are the inner automorphism groups of the $E_6$-type and $E_7$-type spherical Artin groups, respectively \cite{Looijenga2014}. A result of Pinkham implies that the monodromy map of $\mathbb{P}\mathcal{H}^{nh}(4)$ is geometric \cite{Pinkham}, meaning that standard Artin generators representing classes of elements in $\operatorname{Inn}(A(E_6))$ are mapped to some Dehn twists. We prove that the same holds for the monodromy of $\mathbb{P}\mathcal{H}(3,1)$. %, up to considering some finite index copy of the inner automorphism groups of the $E_7$-type Artin groups in $\pi_1^{orb}(\mathbb{P}\mathcal{H}(3,1))$.

\begin{alphateo}
\label{main}
The topological monodromy $\rho_{\mathbb{P}\mathcal{H}(3,1)}:\pi_1^{orb}(\mathbb{P}\mathcal{H}(3,1))\rightarrow\operatorname{Mod}_{3,2}$ maps the classes of the standard generators to Dehn twists.
    %There exists a finite index copy of $A(E_7)_\Delta$ in $\pi_1^{orb}(\mathbb{P}\mathcal{H}(3,1))$ such that the restriction of the topological monodromy map $\rho_{\mathbb{P}\mathcal{H}(3,1)}:\pi_1^{orb}(\mathbb{P}\mathcal{H}(3,1))\rightarrow\operatorname{Mod}_{3,2}$ to this copy of $A(E_7)_\Delta$ maps the classes of the standard generators to Dehn twists.
%The topological monodromy map of the projective stratum $\mathbb{P}\mathcal{H}(3,1)$ is a homomorphism from the $E_7$-type Artin group $A(E_7)$ quotient by its infinite-cyclic center onto the mapping class group $\operatorname{Mod}_{3,2}$, such that there exists a finite index copy of $A(E_7)_\Delta$ where the classes of the standard generators are mapped to Dehn twists.
\end{alphateo}

Theorem \ref{main} and Pinkham's result are then enough to conclude that the kernels of the monodromies associated with the strata $\mathcal{H}^{nh}(4)$ and $\mathcal{H}(3,1)$ both contain a copy of a non-abelian free group $F_2$ of rank $2$.

\textbf{Structure of the paper} The paper is organized into five sections. In Section $1$ we describe the additional length graph associated with the Garside structure of a spherical-type Artin group $A(\Gamma)$, which serves as $\delta$-hyperbolic metric space for the acylindrical action of $A(\Gamma)$. In Section $2$ we define geometric homomorphisms and describe the Wajnryb element, while in Section $3$ we prove Theorem \ref{thmb}. In Sections $4$ and $5$ we draw the consequences that the existence of a non-abelian free group of rank $2$ implies for the topological monodromy of strata of abelian differentials. In particular, in Section $5$ we prove Theorem \ref{ker} and \ref{main}.


\textbf{Acknowledgments.} I am grateful for the patient guidance and support provided by my supervisors Tara Brendle and Vaibhav Gadre throughout working on this paper. I would also like to thank Aaron Calderon for the many helpful conversations, as well as Matt Bainbridge and Bradley Zykoski for patiently answering my questions. I am grateful to Aaron Calderon and Nick Salter for pointing me out the relation between versal deformation spaces of plane curve singularities and the flat geometry in genus $3$, which is the content of their joint work with Pablo Portilla Cuadrado. I would also thank Dawei Chen and the anonymous referee for their comments on the initial draft of this work. Finally, I would like to express my gratitude for the comments and suggestions provided by Aitor Azemar, Philipp Bader, Jim Belk, Rachael Boyd, Tudur Lewis, Miguel Orbegozo Rodriguez, Francesco Pagliuca and Franco Rota. 