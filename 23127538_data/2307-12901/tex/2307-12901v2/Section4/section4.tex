

In this section we define the topological monodromy $\rho_\mathcal{C}:\pi_1^{orb}(\mathcal{C})\rightarrow\operatorname{Mod}_{g,n}$ of the connected components $\mathcal{C}$ of strata of translation surfaces. The relation between the connected components $\mathcal{H}^{nh}(4)$ and $\mathcal{H}(3,1)$ and the Artin groups of type $E_6$ and $E_7$ is stated at the end of this section.

\subsection{Translation surfaces as polygons}

As mentioned in the introduction, a translation surface on $\Sigma_g$ is defined by a genus $g$ Riemann surface $X$ and a holomorphic non-zero global section $\omega$ of the cotangent bundle of $X$, called \textit{abelian differential}. Each $\omega$ has $2g-2$ zeros on $X$ counted with multiplicity \cite[Theorem 1.2]{Wright2015}.

By means of a developing map, an isomorphic class of translation structure on $\Sigma_g$ is equivalently an equivalence class of polygons on the complex plane. The sides of the polygon are identified in pairs via translations, such that the quotient space is $\Sigma_g$. Two such polygons define isomorphic translation structures if one can be obtained from the other by a \textit{scissor move}, as shown in Figure (\ref{scissor}). This operation is performed by cutting one of the two polygons along a straight segment joining two vertices and gluing back the two cut pieces along identified sides via a translation. 

In Section $6$ we are going to use this alternative definition of translation surface to describe a generating set for the orbifold fundamental group $\pi_1^{orb}(\mathbb{P}\mathcal{H}^{nh}(4))$.

% Figure environment removed


\subsection{The strata of marked abelian differentials} 

Let $P$ be a finite set of points on $\Sigma_g$. If $(X,\omega)$ is a genus $g$ translation surface with $\mathcal{Z}(\omega)$ as set of zeros for $\omega$, a marking $f:(\Sigma_g,P)\rightarrow(X,\mathcal{Z}(\omega))$ is the isotopy class rel $P$ of a diffeomorphisms. The marked stratum $\mathcal{T}\mathcal{H}(\underline{k})$ is the set of triples $(X,f,\omega)$ where $(X,\omega)\in\mathcal{H}(\underline{k})$ and $f$ is a marking of $(X,\omega)$.

The topology of each stratum $\mathcal{H}(\underline{k})$ is inherited by its cover $\mathcal{T}\mathcal{H}(\underline{k})$.  Indeed, every \textit{marked stratum} $\mathcal{T}\mathcal{H}(\underline{k})$ is equipped with an atlas of charts in $\CC^{2g+n-1}$. Let $\tau$ be a triangulation of $\Sigma_g$ where the vertices are points in $P$. The set $U_\tau$ of triples $(X,f,\omega)\in\mathcal{T}\mathcal{H}(\underline{k})$ such that $f(\tau)$ is a triangulation of $(X,\omega)$ via saddle connections, namely geodesic arcs intersecting the zeros of $\omega$ only at the endpoints. If $\{\gamma_1,\dots,\gamma_{2g+n-1}\}$ is a fixed basis for the relative homology group $H_1(\Sigma_g,P,\ZZ)$, the charts are given by the maps
\begin{align*}
    U_\tau&\rightarrow H^1(\Sigma_g, P,\CC)\\
    (X,f,\omega)&\mapsto (\gamma_i\mapsto\int_{f_*\gamma_i}\omega)_{i=1}^{2g-n+1}
\end{align*}
\cite[Proposition 2.1]{BainbridgeSmillieWeissHorocycle2022}. 

The mapping class group $\operatorname{Mod}_g$ acts on the marked strata by precomposition on the markings and the resulting quotient space gives the quotient topology to the stratum $\mathcal{H}(\underline{k})$. However, the action of the mapping class group $\operatorname{Mod}_g$ is not free on the marked strata, but the point-stabilizers are finite groups \cite[Section 12.1]{farb2011primer}. In particular, the strata of translation surfaces are orbifolds.

In general, each $\mathcal{H}(\underline{k})$ is not connected and its number of connected components is at most $3$ \cite[Theorem 1]{Kontsevich2003}. The strata $\mathcal{H}(2g-2)$ and $\mathcal{H}(g-1,g-1)$ both
have a hyperelliptic connected component, which is isomorphic to quotients of configuration spaces of points on the Riemann sphere by the action of the group of some roots of unity \cite[Theorem 2.3]{CalderonConnected2020}. On the other hand, studying the topology of the non-hyperelliptic components proves to be more intricate.


\subsection{The topological monodromy map}

If $M$ is a connected manifold and $G$ acts smoothly and properly discontinuously on $M$, the quotient space $M/G$ is a (good) orbifold. The\textit{ orbifold fundamental group} $\pi_1^{orb}(M/G,p)$ based at $p\in M$ is the group of pairs $(\eta,g)$, where $g\in G$ and $\eta$ is a homotopy class of arcs with endpoints $p$ and $g\cdot p$. The group operation on $\pi_1^{orb}(M/G,p)$ is given by the composition law $(\eta_1,g_1)(\eta_1,g_1)=(\eta_1*(g_1\cdot\eta_2),g_1g_2)$.

Let $(X,\omega)$ be a translation surface in some connected components $\mathcal{C}$ of a stratum $\mathcal{H}(\underline{k})$ and let us fix a marked translation surface $(X,f,\omega)\in\mathcal{T}\mathcal{H}(\underline{k})$. If $\mathcal{M}_{g,n}$ is the moduli space of genus $g$ Riemann surfaces with $n$ marked points, the forgetful map $\mathcal{C}\rightarrow\mathcal{M}_{g,n}$ induces a homomorphism between orbifold fundamental groups
$$\rho_\mathcal{C}:\pi_1^{orb}(\mathcal{C},(X,f,\omega))\rightarrow\pi_1^{orb}(\mathcal{M}_{g,n},X),$$

where $\pi_1^{orb}(\mathcal{M}_g,X)$ is just $\operatorname{Mod}_{g,n}$ \cite[Section 12.5.3]{farb2011primer}. Geometrically, the homomorphism $\rho_C$ keeps track of the change of marking performed along a loop $(\eta,g)$. The translation structure carried along $\eta$ coincides at the endpoints of the path in $\mathcal{T}\mathcal{H}(\underline{k})$ but the marking might change. Indeed, every path $\eta$ in $\mathcal{T}\mathcal{H}(\underline{k})$ with endpoints $(X,f,\omega)$ and $(X,f',\omega)$ is mapped to the mapping class $f^{-1}f'$ by $\rho_C$.

Calderon-Salter proved that in genus $g\geq 5$ and for every non-hyperelliptic component $\mathcal{C}$, the image of $\rho_\mathcal{C}$ in $\operatorname{Mod}_{g,n}$ is a \textit{framed mapping class group} \cite[Theorem A]{CalderonSalterFramed2022}. The framed mapping class groups are stabilizers of winding number functions defined on the marked translation surface $(X,f,\omega)$ that serves as a base point for the monodromy $\rho_\mathcal{C}$. Calderon-Salter result does not cover the case of the strata $\mathcal{H}^{nh}(4)$ and $\mathcal{H}(3,1)$. However, in this article the focus is on the kernels of the monodromy maps and not on the images. %If $\gamma$  is the isotopy class of an oriented simple closed curves on $\Sigma_{g,n}$, the value $\phi(\gamma)\in\ZZ$ is the winding number of $f(\gamma)$ on $X$ with respect to the unique unitary and locally constant vector field on $X\setminus\mathcal{Z}(\omega)$ defined by $\omega$.

\subsection{Projective strata of abelian differentials}

Any non-zero complex number is a composition of a rotation and a homothety that acts on the strata of translation surfaces by rotating and dilating the defining polygons. The $\CC^*$-action on the strata $\mathcal{H}(\underline{k})$ is continuous and preserves $\mathcal{Z}(\omega)$ pointwise for every $(X,\omega)\in\mathcal{H}(\underline{k})$. A \textit{projective stratum} $\mathbb{P}\mathcal{H}(\underline{k})$ is the quotient of $\mathcal{H}(\underline{k})$ by the action of $\CC^*$. Equivalently, every projective stratum $\mathbb{P}\mathcal{H}(\underline{k})$ parameterizes pairs $(X,D)$ where $X$ is a smooth projective curve and $D$ is a canonical positive divisor with prescribed multiplicities given by representative abelian differentials \cite{Looijenga2014}.
 
In particular, for any connected component $\mathcal{C}$ of the stratum $\mathcal{H}(\underline{k})$, the quotient map $q:\mathcal{C}\rightarrow\mathbb{P}\mathcal{C}$ induces a homomorphism between orbifold fundamental groups $q_*:\pi_1^{orb}(\mathcal{C})\rightarrow\pi_1^{orb}(\mathbb{P}\mathcal{C})$. The topological monodromy map $\rho_{\mathbb{P}\mathcal{C}}:\pi_1^{orb}(\mathbb{P}\mathcal{C})\rightarrow\operatorname{Mod}_g^n$ can then be also defined for $\mathbb{P}\mathcal{C}$ as for $\mathcal{C}$. The monodromies $\rho_\mathcal{C}$ and $\rho_{\mathbb{P}\mathcal{C}}$ fit inside the commutative diagram below

\[
    \begin{tikzcd}[row sep=2em]
\pi_1^{orb}(\mathcal{C}) \arrow[rr, "q_*"] \arrow[dr, "\rho_C"']& & \pi_1^{orb}(\mathbb{P}\mathcal{C})\arrow[ld, "\rho_{{\mathbb{P}\mathcal{C}}}"]\\
 & \operatorname{Mod}_g^n.
\end{tikzcd}
    \]

For the stratum components in question, Looijenga-Mondello proved the following \cite{Looijenga2014}.

\begin{theorem}
\label{lomo}
    The orbifold fundamental groups of the projective connected components $\mathbb{P}\mathcal{H}^{nh}(4)$ and $\mathbb{P}\mathcal{H}(3,1)$ are isomorphic to $A(E_6)_\Delta$ and $A(E_7)_\Delta$, respectively.
\end{theorem}