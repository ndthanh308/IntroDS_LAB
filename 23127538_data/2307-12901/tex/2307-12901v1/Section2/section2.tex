Let $\Sigma_g^b$ a closed and orientable genus $g$ surface with $b$ boundary components. The mapping class group $\operatorname{Mod}_{g}^b$ is the group of isotopy classes of orientation-preserving diffeomorphism of $\Sigma_g^b$ that pointwise fix the boundary and where the isotopies are required to fix the components of $\partial \Sigma_g^b$ pointwise.


% Figure environment removed

\textit{Dehn twists} are diffeomorphisms of $\Sigma_g^b$ supported on the tubular neighborhood of some simple closed curve, as in Figure $3$. Two Dehn twists $T_{\gamma_1}$ and $T_{\gamma_2}$ commute if and only if $\gamma_1$ and $\gamma_2$ are disjoint, and satisfy the braid relation $T_{\gamma_1}T_{\gamma_2}T_{\gamma_1}=T_{\gamma_2}T_{\gamma_1}T_{\gamma_2}$ if and only if the geometric intersection number of $\gamma_1$ and $\gamma_2$ is exactly $1$. 

Suppose that $\Omega$ is a finite family of isotopy classes of non-essential simple closed curves on $\Sigma_g^b$. Moreover, suppose that the geometric intersection number of each pair of curves in $\Omega$ is at most $1$. The \textit{intersection graph} $\Lambda_\Omega$ of $\Omega$ is the graph with set of vertices $\Omega$ and edges for any pair of intersecting curves. Then, any pair of Dehn twists about curves in $\Omega$ either commute or satisfy the Braid relation. The group $$\operatorname{Mod}_g^b(\Omega)=\langle T_\gamma\mid\gamma\in\Omega\rangle$$ is then the quotient of the Artin group $A(\Lambda_\Omega)$ by some normal subgroup. We will call the quotient map
$$\varphi_{\Omega}:A(\Lambda_\Omega)\twoheadrightarrow\operatorname{Mod}_g^b(\Omega)$$ a \textit{geometric homomorphism}.

It is known that the $A_n$ and $D_n$ type Artin groups can be embedded into the mapping class group of some surfaces via a geometric homomorphism  \cite[Théorème 1]{Perron1996}. However, Wajnryb proved that for the $E_6$-type Artin group this is never the case \cite[Theorem 3]{Wajnryb1999}. 

\begin{theorem}
    Any geometric homomorphism of an Artin group $A(\Gamma)$ is not injective as long as $\Gamma$ contains $E_6$ as a subgraph.
\end{theorem}

Wajnryb found a non-trivial element $w\in A(E_6)$ that maps trivially in $\operatorname{Mod}_{3}^1$ via a geometric homomorphism $\varphi_\Omega$, where the set of curves $\Omega$ has intersection graph $E_6$. However, the result can be extended to Artin group with defining graph $\Gamma$ containing $E_6$, since the following is a commutative diagram (see \cite{lek} and \cite[Theorem 3.18]{farb2011primer}).

\begin{center}
\begin{tikzcd}
A(E_6) \arrow[r, hook] \arrow[d]
& A(\Gamma) \arrow[d] \\
\operatorname{Mod}_3^1 \arrow[r, hook]
& \operatorname{Mod}_g^b.
\end{tikzcd} 
\end{center}


With respect to Figure \ref{perron1}, the \textit{Wajnryb element} $w$ can be written as a word in the alphabet $\{a_1,b\}\subset A(E_6)$, where $$b=a_4a_5a_3a_4a_2a_6a_5a_3a_4$$ is contained in a parabolic subgroup isomorphic to $\mathcal{B}_6$. The element $b$ has the following braid representation.

% Figure environment removed

The image of $b$ via the geometric homomorphism $\varphi_\Omega$ is represented by a diffeomorphism which maps the simple closed curve $\gamma_1$ to a simple closed curve $\beta$ that intersects $\gamma_1$ once. Hence, the Dehn twists $T_{\gamma_1}$ and $T_\beta$ satisfy the braid relation and the Wajnryb element  
\begin{align*}
\label{wajnryb}
    w=a_1a_1^ba_1\cdot (a_1^{-1})^ba_1^{-1}(a_1^{-1})^b,
\end{align*}
acts trivially on $\Sigma_{3,1}$ as a mapping class. Indeed, its image via $\varphi_{\Omega}$ is precisely $T_{\gamma_1}T_\beta T_{\gamma_1}\cdot T_\beta^{-1}T_{\gamma_1}^{-1}T_\beta^{-1}=1$. Wajnryb proved that the group element $w$ is non-trivial applying the Garside algorithm. However, it is less clear why $w$ should geometrically describe a non-trivial group element in $A(E_6)$. 

% Figure environment removed
