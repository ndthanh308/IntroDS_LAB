
In this section we consider the topological monodromy homomorphisms of the projective strata $\mathbb{P}\mathcal{H}^{nh}(4)$ and $\mathbb{P}\mathcal{H}(3,1)$. In Theorem \ref{ham} we recall the definition of some standard generators of $\pi_1^{orb}(\mathbb{P}\mathcal{H}^{nh}(4))$ that map via the homomorphisms $$\rho_{\mathbb{P}\mathcal{H}^{nh}(4)}:\pi_1^{orb}(\mathbb{P}\mathcal{H}^{nh}(4))\rightarrow\operatorname{Mod}_{3,1}$$ to Dehn twists. Discussions related to Theorem \ref{ham} are subject of an ongoing work by Calderon-Cuadrado-Salter; for instance, see \cite{Cuadrado2021}. The main ideas that underline Theorem \ref{ham} come from the theory of versal deformation spaces for plane curve singularities \cite{arnold}. For the sake of completeness, we are going to include a similar description of $\rho_{\mathbb{P}\mathcal{H}(3,1)}$ closely following the work of Calderon-Cuadrado-Salter. In particular, we prove that the restriction of $$\rho_{\mathbb{P}\mathcal{H}(3,1)}:\pi_1^{orb}(\mathbb{P}\mathcal{H}(3,1))\rightarrow\operatorname{Mod}_{3,2}$$ to a finite index copy of $A(E_7)_\Delta$ in $\pi_1^{orb}(\mathbb{P}\mathcal{H}(3,1))$ is geometric. However, the result of Calderon-Cuadrado-Salter gives explicit generators for $\pi_1^{orb}(\mathbb{P}\mathcal{H}^{nh}(4))$. This finite set of generators arises from the algebro-geometric theory of versal deformation spaces and can be described as a finite set of cylinder shears; see Figure \ref{shear} and \ref{e6trsu}. On the other hand, Theorem \ref{main} is enough to prove the existence of a non-abelian free group of rank $2$ in the kernel of $\rho_{\mathbb{P}\mathcal{H}(3,1)}$.

\begin{proof}[Proof of Theorem \ref{ker}]
    The copy of the non-abelian free group of rank $2$ we constructed in Theorem \ref{thmb} is in the kernel of any geometric homomorphism $A(E_6)\rightarrow\operatorname{Mod}_{3,1}$, but it is also a non-abelian free group in the kernel of the geometric map $A(E_6)_\Delta\rightarrow\operatorname{Mod}_{3,1}$, which is the same as $\rho_{\mathbb{P}\mathcal{H}^{nh}(4)}$. Similarly, the same copy of the non-abelian free group of rank $2$ can be defined in the kernel of the geometric homomorphism $A(E_7)_\Delta\rightarrow\operatorname{Mod}_{3,2}$, which is the same as $\rho_{\mathbb{P}\mathcal{H}(3,1)}$ restricted to a finite index copy of $A(E_7)_\Delta$.
    
\end{proof}

\textbf{The monodromy of the stratum $\bm{\mathbb{P}\mathcal{H}^{nh}(4)}$.}  A \textit{cylinder} $\xi$ on a translation surface is an isometric embedding of a Euclidean cylinder whose boundary is a union of saddle connections. In particular, the interior of $\xi$ does not contain any singular point.

If the embedded cylinder $\xi$ is isometric to $(\mathbb{R}/a\mathbb{Z})\times[0,b]$ for some $a,b\in\RR^+$, the core curve of $\xi$ on the translation surface $(X,\omega)$ is the isotopic class of the simple closed curve which is the image of $(\mathbb{R}/a\mathbb{Z})\times\{t\}$ in $(X,\omega)$ for some $t\in(0,b)$. 

Suppose $\xi$ is a horizontal cylinder on a translation surface $(X,\omega)$. In particular, the cylinder $\xi$ can be represented as a rectangle $[0,b]\times[0,a]$ embedded in a defining polygon of $(X,\omega)$ with a pair of sides identified. Suppose that the ratio between its height $a$ and its weight $b$ is $R$. If $t\in[0,R]$, a \textit{cylinder shear} along $\xi$ is the result of the action by the matrix 
\begin{align*}
S_t=
    \begin{bmatrix}
    1& t\\
    0& 1
    \end{bmatrix}
\end{align*}
on the embedded parallelogram of the polygon representative. Analogously, by taking a suitable conjugate of $S_t$ one can define a cylinder shear along non-horizontal cylinders.

% Figure environment removed

Let now $f:\Sigma_g\rightarrow X$ be a marking of $(X,\omega)$. The full shear $S_R$ acts on $(X,f,\omega)$ preserving the translation structure of $X$, as the resulting polygon differs from the initial one by a scissor move, as in Figure (\ref{shear}). However, the matrix $S_R$ changes the marking $f$ by a Dehn twist along the core curve of the cylinder $\xi$. Hence, a cylinder shear is an orbifold loop, and it is mapped via the topological monodromy map of the connected component containing $(X,\omega)$ to a Dehn twist.

The following result is known by experts. It appears as a consequence of Henry Pinkham's thesis \cite{Pinkham} and can also be found in \cite[Proposition 6.2]{ham}. 

\begin{theorem}
\label{ham}
    Let $\{\xi_1,\dots,\xi_6\}$ be a collection of embedded cylinders of a translation surface $(X,\omega)\in\mathcal{H}^{nh}(4)$ such that the family of the associated core curves have an $E_6$-type intersection graph. Then, there exists a map $\Pi:A(E_6)\rightarrow \pi_1^{orb}(\mathbb{P}\mathcal{H}^{hn}(4))$ that associates each standard generator of $A(E_6)$ to a full cylinder shear and can be extended to a well-defined surjective homomorphism with kernel the center of $A(E_6)$. 
\end{theorem}

% Figure environment removed

The homomorphism $\Pi$ is well-defined.  Every pair of adjacent standard generators in $A(E_6)$ is mapped to cylinder shears along embedded cylinders with core curves intersecting once; every pair of standard generators that commute is mapped to cylinder shears along disjoint flat cylinders.

Theorem \ref{ham} shows that $\pi_1^{orb}(\mathbb{P}\mathcal{H}^{nh}(4))$ is generated by a finite family of cylinder shears. However, the group $\pi_1^{orb}(\mathcal{H}^{nh}(4))$ contains orbifold loops that cannot be generated by cylinder shears only. These are loops that cyclically permute the prongs around the singularity. Calderon-Salter \cite[Corollary 7.6]{CalderonSalterFramed2022} showed that there exists an epimorphism $\pi_1^{orb}(\mathcal{H}^{nh}(4))\twoheadrightarrow\ZZ_2$ with kernel containing those orbifold loops that do not permute the prongs at the singularity. In particular, cylinder shears cannot cyclically permute any prong configuration. %However, the map $\mathcal{H}^{nh}(4)\rightarrow\mathbb{P}\mathcal{H}^{nh}(4)$ induces a short exact sequence
%\begin{align}
%\label{LMses}
%%    0\rightarrow\ZZ\rightarrow\pi_1^{orb}(\mathcal{H}^{nh}(4))\rightarrow\pi_1^{orb}(\mathbb{P}\mathcal{H}^{nh}(4))\rightarrow0
%\end{align}
%and any cylinder shear is also an orbifold loop of $\mathcal{H}^{nh}(4)$. Hence, we found a right-inverse homomorphism for $\pi_1^{orb}(\mathcal{H}^{nh}(4))\rightarrow\pi_1^{orb}(\mathbb{P}\mathcal{H}^{nh}(4))$ and prong cyclic permutations are the only missing group elements needed to generate $\pi_1^{orb}(\mathcal{H}^{nh}(4))$, since in this case 
%$$\pi_1^{orb}(\mathcal{H}^{nh}(4))\cong \ZZ \rtimes A(E_6)_\Delta,$$ where the infinite cyclic subgroup is generated by a $\pi$-rotation of the translation surface in Figure \ref{e6trsu} that cyclically permutes the prongs at the singularity of order $4$. However, we are not aware of a similar result for the stratum $\mathcal{H}(3,1)$.



\textbf{The monodromy of the stratum $\bm{\mathbb{P}\mathcal{H}(3,1)}$.} Every genus $3$ non-hyperelliptic Riemann surface is the vanishing locus of a degree $4$ homogeneous polynomial in $\CC[x,y,z]$ \cite[Chapter VII, Proposition 2.5]{Miranda}.

Homogeneous quartics with $3$ complex variables have explicit normal forms \cite{TT} and in particular every genus $3$ non-hyperelliptic Riemann surface is the vanishing locus of one of the two following polynomials, up to isomorphism:
\begin{enumerate}
\item[a)]$F_t=x^3z+y^4+t_1xy^2z+t_2xyz^2+t_3xz^3+t_4y^2z^2+t_5yz^3+t_6z^4$ or
    \item[b)]$F_s=x^3z+y^3x+s_1xyz^2+s_2xz^3+s_3y^4+s_4y^3z+s_5y^2z^2+s_6yz^3+s_7z^4,$
\end{enumerate}
 where $t=(t_1,\dots,t_6)\in\CC^6$ and $s=(s_1,\dots,s_7)\in\CC^7$.

Consider the homogeneous polynomial $F_s$ as in $b)$ and denote by $f_s$ the polynomial obtained by de-homogenize in the $z$-variable the polynomial $F_s$. The collection of parameters $s\in\CC^7$ such that $f_s$ defines a smooth Riemann surface is the base space of a fiber bundle. Let  $B=\{s\in\CC^7\mid f_s^{-1}(0)\text{ is smooth}\}$ be the parameter space. The set $\{s\in\CC^7\mid  f_s^{-1}(0)\text{ is singular}\}$ is a hypersurface in $\CC^7$ and $B$ is an Eilenberg-Maclane space for the spherical-type Artin group $A(E_7)$.  After intersecting $E=\{(s,p)\in\CC^7\times\CC^2\mid f_s^{-1}(0)\text{ is smooth, }p\in f_s^{-1}(0) \}$ by a closed polydisk in $\CC^7\times\CC^2$, the natural projection $E\rightarrow B$ is a fiber bundle with fibers diffeomorphic to $\Sigma_3^2$ and the monodromy $\rho:\pi_1(B)\rightarrow\operatorname{Mod}_3^2$ is a geometric homomorphism. For more details, see \cite[Section 2]{Cuadrado2021} and \cite[Proposition 9.3]{arnold}. 

If we glue a pair of open punctured disks to the boundary components of $\Sigma_3^2$ we obtain $\Sigma_{3,2}$. This procedure defines the capping homomorphism $\operatorname{Cap}:\operatorname{Mod}^2_3\rightarrow\operatorname{Mod}_{3,2}$ by extending the mapping classes in $\operatorname{Mod}_3^2$ to the identity on the glued punctured disks. The proof of Theorem \ref{main} relies on the existence of a homomorphism $\Theta:\pi_1(B)\rightarrow\pi_1^{orb}(\mathbb{P}\mathcal{H}(3,1))$ with finite index image that relates the two monodromies $\rho:\pi_1(B)\rightarrow\operatorname{Mod}_3^2$ and $\rho_{\mathbb{P}\mathcal{H}(3,1)}:\pi_1^{orb}(\mathbb{P}\mathcal{H}(3,1))\rightarrow\operatorname{Mod}_{3,2}$.  In particular, we are going to prove that the following diagram

\begin{center}
\begin{equation}
\label{comm}
\begin{tikzcd}
\pi_1(B) \arrow[r, "\Theta"] \arrow[d, "\rho"]
& \pi_1^{orb}(\mathbb{P}\mathcal{H}(3,1)) \arrow[d, "\rho_{\mathbb{P}\mathcal{H}(3,1)}"] \\
\operatorname{Mod}_3^2 \arrow[r, "\operatorname{Cap}"]
& \operatorname{Mod}_{3,2}
\end{tikzcd}  
\end{equation}
\end{center}

commutes. Theorem \ref{main} then follows directly from Lemma \ref{saveass} below. For more details, see \cite{hirshon_1977}.
%or \cite[Lemma 4.1]{bridson2010cofinitely}.

\begin{lemma}
\label{saveass}
Let $G$ be a finitely generated residually finite group with no non-trivial normal finite subgroups. Then, every endomorphism $G\rightarrow G$ with finite index image is an injection. 
\end{lemma}

The group $A(E_7)_\Delta$ satisfies the hypotheses of Lemma \ref{saveass}. Indeed, since $A(E_7)$ is linear \cite{Cohen2002}, the group $A(E_7)$ and its automorphism group $\operatorname{Aut}(A(E_7))$ are both residually finite \cite[Theorem 1]{BaumslagResidually1963}. Hence, the subgroup $\operatorname{Inn}(A(E_7))\cong A(E_7)_\Delta$ of $\operatorname{Aut}(A(E_7))$ is both finitely generated and residually finite. Moreover, there are no non-trivial normal finite subgroups in $A(E_7)_\Delta$ \cite[Proposition 4.13]{bestvina1999non}.

\begin{proof}[Proof of Theorem \ref{main}] The kernel of the homomorphism $\Theta:A(E_7)\rightarrow A(E_7)_\Delta$ contains the center of $A(E_7)$, since the center of every group is characteristic and $A(E_7)_\Delta$ is centerless. Then, we deduce that the induced map $$\overline{\Theta}:A(E_7)_\Delta\rightarrow A(E_7)_\Delta$$
is well-defined and injective by Lemma \ref{saveass}.    

\end{proof}

Let us now construct the map $\Theta:\pi_1(B)\rightarrow\pi_1^{orb}(\mathbb{P}\mathcal{H}(3,1))$ as the composition of epimorphisms and monomorphisms with finite index images. We divide the construction in $3$ steps. 

\textit{Step 1.} Pair of quartics $F_{s}$ and $F_{r}$ as in $b)$ define the same smooth Riemann surface if and only if there exists $\lambda\in\CC^*$ such that the parameters $s,r\in\CC^7$ satisfy the following relation \cite[Proposition 1]{Shioda1993}:
\begin{equation}
\label{rel}
(r_1,r_2,r_3,r_4,r_5,r_6,r_7)=(\lambda s_1,\lambda^3s_2,\lambda^4s_3,\lambda^5s_4,\lambda^6s_5,\lambda^7s_6,\lambda^9s_7).
\end{equation}

We denote by $\mathbb{P}_w^{B}$ the quotient of $B$ by the relation in (\ref{rel}) and by $\mathbb{P}_w^{E}$ the respective quotient of $E$. The projection $q_w:\mathbb{P}_w^E\rightarrow\mathbb{P}_w^B$ is a fiber bundle that fits inside the following commutative diagram
%In other words, the quotient of $B$ by the relation in (\ref{rel}) is the complement of a hypersurface in a weighted projective space. We will denote such a quotient by $\mathbb{P}_w B$, while the space $\mathbb{P}B$ will always denote the quotient of $B$ by the usual projective relation. Analogously, we denote by $\mathbb{P}_w E$ and $\mathbb{P}E$ the respective projectivized total spaces. The weighted fiber bundle $q_w:\mathbb{P}_w E\rightarrow\mathbb{P}_w B$ fits inside the following commutative diagram
\begin{center}
\begin{tikzcd}
E \arrow[r, "p_E"] \arrow[d, "q"]
& \mathbb{P}_w^E \arrow[d, "q_w"] \\
B \arrow[r, "p_B"]
& \mathbb{P}_w^B,
\end{tikzcd}  
\end{center}

where $p_E$ and $p_B$ are the obvious quotient maps. In particular, the map $p_E:E\rightarrow\mathbb{P}_w^E$ is a bundle map between fiber bundles with diffeomorphic fibers and therefore the respective monodromies commute.

The induced homomorphism $(p_B)_*:\pi_1(B)\rightarrow\pi_1(\mathbb{P}_w^B)$ has a finite index image. To see this, we recall that any complex weighted projective space is the quotient of a projective space by a finite group $G$ \cite[page 127]{harris2013algebraic}. If $\mathbb{P}B$ is the projectivization of $B$, then $\mathbb{P}_w B$ can be seen as $\mathbb{P}B/G$. However, the action might not be free and therefore there exists a short exact sequence
$$1\rightarrow\pi_1(\mathbb{P}B)\rightarrow\pi_1^{orb}(\mathbb{P}_w^B)\rightarrow G\rightarrow1.$$
By post-composing the inclusion $\pi_1(\mathbb{P}B)\rightarrow\pi_1^{orb}(\mathbb{P}_w^B)$ with the natural epimorphism $p_{\mathbb{P}_w}:\pi_1^{orb}(\mathbb{P}_w^B)\twoheadrightarrow \pi_1(\mathbb{P}_w^B)$, we obtain a map $\pi(\mathbb{P}B)\rightarrow\pi(\mathbb{P}_w^B)$ with finite index image. The homomorphism $p_\mathbb{P}:\pi_1(B)\rightarrow\pi_1(\mathbb{P}B)$ is surjective (see, for example, \cite[Main Theorem]{Zariski1937} and \cite[Lemma 6.1]{Lonne06}) and we can then describe the homomorphism induced by $p_B$ on the fundamental groups as the composition of $p_{\mathbb{P}}$ and $p_{\mathbb{P}_w}$.


\textit{Step 2.} Let us denote by $ \mathbb{V}(f_s)$ and $ \mathbb{V}(F_s)$ the Riemann surfaces $ f_s^{-1}(0)$ and $ F_s^{-1}(0)$,  respectively. As the Riemann surface $ \mathbb{V}(f_s)$ is the level set of a holomorphic function, we have that $f_xdx+f_ydy=0.$ Moreover, the derivatives $f_x$ and $f_y$ cannot simultaneously vanish since $ \mathbb{V}(f_s)$ is smooth. Hence, the abelian differential 
\begin{align*}
    \omega_{s}(p)=
\begin{cases} 
       f_y(p)^{-1}dx & \text{ if } f_y(p)\neq 0 \\
      -f_x(p)^{-1}dy & \text{ if } f_x(p)\neq 0
   \end{cases}
\end{align*}
 is well-defined and non-vanishing at every $p\in   \mathbb{V}(f_s)$. 

\begin{lemma}
The abelian differential $\omega_s$ can be extended to $ \mathbb{V}(F_s)$. In particular, the pair $( \mathbb{V}(F_s),\omega_s)$ is a translation surface in the stratum $\mathcal{H}(3,1)$.
\end{lemma}
\begin{proof}
The hyperplane at infinity $\{[x:y:z]\in\CC\mathbb{P}^2\mid z=0\}$ intersects $ \mathbb{V}(F_s)$ at the points $p_1=[1:0:0]\in\CC\mathbb{P}^2$ and $p_2=[0:1:0]\in\CC\mathbb{P}^2$. For $\varepsilon>0$, let $U_\varepsilon$ be the open set $\{[x:y:z]\in  \mathbb{V}(F_s)\mid\|x\|>\varepsilon,\|y\|>\varepsilon\}\setminus\{p_2\}$ of $ \mathbb{V}(F_s)$. As $U_\varepsilon$ is contained in $\{z\in  \mathbb{V}(F_s)\mid f_y(z)\neq 0\}$ for $\varepsilon$ big enough, we can write $\omega_s$ as $f_y(z)^{-1}dx$ for every $z\in U_\varepsilon$. After a change of coordinate, the abelian differential $\omega_s$ can be rewritten as $y^{-1}g(y)dy$ where $g:U_\varepsilon\rightarrow\CC$ does not have a zero at the origin. Similarly, the $1$-form $\omega_s$ can be extended to $p_1\in  \mathbb{V}(F_s)$ with multiplicity $3$.

\end{proof}

Let now $\mathscr{H}\subset\mathbb{P}\mathcal{H}(3,1)$ be the locus of hyperelliptic translation surfaces is genus $3$ and $\mathscr{H}^c$ its complement in $\mathbb{P}\mathcal{H}(3,1)$.  

 \begin{lemma}
 \label{corr}
   There is a one-to-one correspondence between $\mathbb{P}_\omega B$ and  $\mathscr{H}^c$.
 \end{lemma} 
 \begin{proof}
Given $s\in B$, we can compactify $ \mathbb{V}(f_s)$ to get the smooth Riemann surface $ \mathbb{V}(F_s)$. From what we have seen before, the pair $(  \mathbb{V}(F_s),\omega_s)$ is a non-hyperelliptic translation surface in $\mathcal{H}(3,1)$. Let us now consider an arbitrary non-hyperelliptic translation surface $(X,\omega)\in\mathcal{H}(3,1)$. As for the above discussion, there exists $s\in\CC^7$ such that $X=  \mathbb{V}(F_s)$ up to an isomorphism. The space of volume forms on $ \mathbb{V}(f_s)$ is a complex line bundle and therefore there exists a holomorphic function $u:  \mathbb{V}(f_s)\rightarrow\CC$ such that $\omega_s=u\cdot\omega$ on $ f_s^{-1}(0)$. Since $\omega_s$ can holomorphically be extended at infinity, this holds for $u\cdot\omega$ too and in particular for the holomorphic function $u$. However, any holomorphic function on a compact Riemann surface is constant and the abelian differentials $\omega_s$ and $\omega$ differ by a non-zero complex number. Hence, the map $  \mathbb{V}(f_s)\mapsto(  \mathbb{V}(F_s),\omega_s)$ is a bijection from $\mathbb{P}_\omega B$ to $\mathscr{H}^c$.

 \end{proof}

The space $\mathbb{P}_w^B$ parameterizes genus $3$ Riemann surfaces with $2$ boundary components and can be given the orbifold structure inherited by the moduli space $\mathcal{M}_3^2$ of Riemann structures on $\Sigma_3^2$. 

The correspondence from Lemma \ref{corr} can be upgraded to a homomorphism of the Teichm\"{u}ller covers $\mathcal{T}(\mathbb{P}_w^B)$ and $\mathcal{T}\mathscr{H}^c$. If we denote by $\rho_{\mathscr{H}^c}$ the monodromy of $\mathscr{H}^c$, then the Five Lemma applies to the following commutative diagram
 
 \[
 \begin{tikzcd}
  1 \arrow[r] & \pi_1(\mathcal{T}(\mathbb{P}_w^B)) \arrow[d ,phantom, "\nvisom"] \arrow[r] & \pi_1^{orb}(\mathbb{P}_w^B) \arrow[d] \arrow[r, "\rho"] & \operatorname{im}\rho \arrow[d, "\operatorname{Cap}",->>] \arrow[r] & 1 \\
  1 \arrow[r] & \pi_1(\mathcal{T}(\mathscr{H}^c)) \arrow[r] & \pi_1^{orb}(\mathscr{H}^c) \arrow[r, "\rho_{\mathscr{H}^c}"] & \operatorname{im}\rho_{\mathscr{H}^c} \ar[r] & 1\\
\end{tikzcd}
\]

 and the homomorphism $\pi_1^{orb}(\mathbb{P}_w^B)\rightarrow\pi_1^{orb}(\mathscr{H}^c)$ is surjective. However, the orbifold structure of $\mathbb{P}_w^B$ inherited by $\mathcal{M}_3^2$ is not singular, as $\operatorname{Mod}_3^2$ is torsion-free. In particular, we have defined a homomorphism $\pi_1(B)\rightarrow\pi_1^{orb}(\mathscr{H}^c)$ with finite index image that commutes with the monodromy maps. 

\textit{Step 3.} Lastly, we prove that the homomorphism $\pi_1^{orb}(\mathscr{H}^c)\rightarrow\pi_1^{orb}(\mathbb{P}\mathcal{H}(3,1))$ induced by the inclusion is surjective. To see this, we observe that the complex codimension of the hyperelliptic locus in a non-hyperelliptic stratum is non-trivial \cite[Section 9.4]{ZorichFlat} and therefore every loop can be homotoped away from $\mathscr{H}$. In particular,  the inclusion $\mathscr{H}^c\hookrightarrow \mathbb{P}\mathcal{H}(3,1)$ induces an epimorphism of the fundamental groups at the level of the respective Teichm\"{u}ller covers. Hence, the homomorphism $\pi_1^{orb}(\mathscr{H}^c)\rightarrow\pi_1^{orb}(\mathbb{P}\mathcal{H}(3,1))$ is surjective and $\Theta:\pi_1(B)\rightarrow\pi_1^{orb}(\mathbb{P}\mathcal{H}(3,1))$ has been constructed.

%If $m$ is the number of zeros occurring as Weierstrass points, the subvariety of hyperelliptic translation surfaces in $\mathcal{H}(\underline{k})$ has dimension $2g+(n-m)/2$ \cite[Secton 2.2]{Mullane2017}. If $(X,\omega)$ is in $\mathscr{H}\cap\mathcal{H}(3,1)$ and $P$ is a polygon representative for $(X,\omega)$, the hyperelliptic involution $\iota\in\operatorname{Aut}(X)$ changes the sign of $\omega$ and therefore acts as a $\pi$-rotation on $P$ that preserves the vertices. However, the two singularities of $(X,\omega)$ have different multiplicities and cannot be swapped. Since $m=2$, the intersection $\mathcal{H}(3,1)\cap\mathscr{H}$ has complex codimension $1$ in $\mathcal{H}(3,1)$. 





