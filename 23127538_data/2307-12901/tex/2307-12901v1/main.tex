% LaTeX Template For MATH 490 @ VCU
\documentclass[12pt]{article}

\newcommand{\bmmax}{0}  
\newcommand{\hmmax}{0}  

%\usepackage{hyperref}
\usepackage{amsmath}
\usepackage{amsthm}
\usepackage{amssymb}
\documentclass{article}
\usepackage[utf8]{inputenc}
\usepackage{multicol}
\usepackage{xcolor}
\usepackage{graphicx}
\usepackage{amsmath}
\usepackage{amsfonts}
\usepackage{float}


\definecolor{dorange}{rgb}{1,0.4,0}
\newcommand{\sdorange}[1]{\textcolor{dorange}{{#1}}}
\definecolor{olivegreen}{rgb}{0.2,.8,0.36}
\newcommand{\ardgreen}[1]{\textcolor{olivegreen}{{#1}}}
\newcommand{\yascyan}[1]{\textcolor{cyan}{{#1}}}
\newcommand{\red}[1]{\textcolor{red}{{#1}}}
\newcommand{\blue}[1]{\textcolor{blue}{{#1}}}


\title{Everpresent Lamda}
\author{sanjone }
\date{January 2022}

\begin{document}

\maketitle

\section{Questions}
\begin{itemize}
    \item What is a good time parameter? (proper time, conformal time, scale factor, spacetime volume, inverse or inverse square root of the spacetime volume, etc.). Whichever one we choose, we may want to consider equal or un-equal spacing between the steps.
    \item What starting $a$ or $z$ did previous works use and why? How far back do we want to start?
    \item Were any inflationary models for initial perturbations used when running CAMB in the 2018 paper?
    \item Why do there appear different expressions for the volume of the past lightcone in different references? (see Section 4 below) How different are the volumes given by the two expressions?
    \item Santanu had some question about $\Omega_b$ \sdorange{ My issue was not with $\Omega_b$ in particular. The issue is that we are beginning with some values of $\Omega_L$ and $H_0$. Then we are applying the standard LCDM model to get the energy density at $a=10^-8$ or so. We are evolving the Everpresent Lamda model and finding some value of $\Omega_L$, and $H_0$ at  $a=1$, which of course does not match with the initial values that we set. May be we can check it in the literature once. Its not such a big problem proably. However, we need to check it properly while doing the calculations. }
\end{itemize}

\section{Meeting 18 Jan 2020}

\begin{itemize}
    \item If I calculate the volume of the universe from $\Lamda$CDM model and calculate the $\rho_{\Lambda} = \frac{\frac{\hbar}{l_p^2}}{\sqrt{\mathcal{V}}}$, then the value of $\frac{\rho_\Lambda}{\rho_\text{cr}}$  is coming to be about $22$, which is not expected (Atleast I was not expecting it - \sdorange{Santanu}). \yascyan{In the Everpresent $\Lambda$ model, $\Lambda$ is a random number with mean value $0$ (we are assuming this)  and standard deviation $1/\sqrt{V}$ (in appropriate units). Therefore the model only says which range of values is more probable than which others, i.e. within one standard deviation is more natural or typical and outside it is less so. The model does not say that the value we observe needs to be exactly at the standard deviation of the distribution (which is $1/\sqrt{V}$). The currently observed value of $\Lambda$ therefore lies within the expected range.}
    \item \ardgreen{Plot of $\frac{\rho_\Lamda}{H^2}$ from Arad using his python code.} 
    % Figure environment removed
\end{itemize}


\section{Speeding up the Volume calculation}
The volume is given by 
\begin{equation}
\mathcal{V}(t)=\frac{4 \pi}{3} \int_{0}^{t} d t^{\prime} a\left(t^{\prime}\right)^{3}\left[\int_{t^{\prime}}^{t} d t^{\prime \prime} / a\left(t^{\prime \prime}\right)\right]^{3}
\end{equation}

Now $d\tau = \frac{dt}{a}$. Therefore, we should have 

\begin{eqnarray}
\mathcal{V}(t)=\frac{4 \pi}{3} \int_{0}^{\tau} d \tau^{\prime} a\left(t^{\prime}\right)^{4}\left( \tau - \tau' \right)^{3} \\
= \frac{4 \pi}{3} \left[ \tau^3 \int_{0}^{\tau} d \tau^{\prime} a\left(t^{\prime}\right)^{4}
- 3 \tau^2 \int_{0}^{\tau} d \tau^{\prime} a\left(t^{\prime}\right)^{4}\tau^{\prime}
+ 3 \tau \int_{0}^{\tau} d \tau^{\prime} a\left(t^{\prime}\right)^{4}\tau^{\prime 2} 
- \int_{0}^{\tau} d \tau^{\prime} a\left(t^{\prime}\right)^{4}\tau^{\prime 3}\right]
\end{eqnarray}

The value of all these 4 integrals can be stored in some variables in some steps. So if we want the value of the integration we can break it as follows

\begin{eqnarray}
\int_{0}^{\tau+\tau^{\prime}} d \tau^{\prime} a\left(t^{\prime}\right)^{4}
=\int_{0}^{\tau} d \tau^{\prime} a\left(t^{\prime}\right)^{4} + \int_{\tau}^{\tau+\tau^{\prime}} d \tau^{\prime} a\left(t^{\prime}\right)^{4}
\end{eqnarray}

The first part will be stored from previous integration. And we need to calculate the second part only. So this integration will be done in constant time or $O(1)$. As we need the values of  $\mathcal{V}(t)$ at different $t$s, lets say $n$ number of $t$s, the full process can be done in $O(n)$.  

\yascyan{\section{Volume Expressions}}

We are interested in the spacetime volume $V(t)$ of the past lightcone of a point at time $t$. This will be 

\begin{equation}
V(t)=\frac{4\pi}{3}\int_0^t dt'a(t')^3 r(t')^3,
\end{equation}
where $r(t')$ is the radial coordinate at time $t'$ for incoming null rays reaching the point at $t$, which is at the vertex of the lightcone. The expression for this is $r(t')=-\int_{t}^{t'} \frac{dt''}{a(t'')}$. If we were to instead take $r(t')=-\int_{t'}^{0} \frac{dt''}{a(t'')}$, we are considering the radius at $t=0$ of light rays that reach the observer at time $t'$.
\\ \\
\section{first two runs for the expansion history- Arad}
With the faster code for integration,  I could get one complete run in $\sim1$ hour. tolerance=1.0e-2 in rombint. However, the code hit the $H^2<0$ problem 50 and 250 times respectively before being able to run one clean run for the whole expansion history.\\
Here are two plots for $\rho_{\Lambda}/H^2$ vs time (in Mpc):

\begin{center}
    

     % Figure removed
\end{center}
\begin{center}
  % Figure removed
\end{center}
\\ \\
And here are the same functions vs redshift:\\ \\
  \begin{center}
     % Figure removed
    \end{center}
   

\begin{center}
  % Figure removed
  \end{center}
\begin{itemize}
    \item The final values for $\Omega_\Lambda$ are 0.70 and 0.64.
    \item Regarding Santanu's question on what time coordinate to use, I used equal steps in "a" to update volume, not in cosmic time. Fay made a very interesting comment on this. She said that the most natural time coordinate for us would be the volume itself, as it is directly proportional to the number of elements which are contributing to the stochastic lambda.
    \item I think "a" and cosmic time yield similar results because they have an almost liner relation; see the plot of scale factor vs time:
    \begin{center}
  % Figure removed
  \end{center}
However, volume (in $Mpc^4$) vs time behaves differently:
\begin{center}
  % Figure removed
  \end{center}
\end{itemize}

  \section{some statistics of the data- 23/03/2022}
  
  Here are the histograms for $H_0$ and $\Omega^0_\Lambda=\frac{\Lambda(a=1)}{3H_0^2}$ with different values for $\alpha$ and different steps. $da=1$ means steps of $\Delta a =10^{-4}$ were taken, and $dt=1$ means steps of 1 Mpc in time were taken.\\
  As you can see, for increasing $H_0$ to make it closer (statistically) to the observed value, we should choose $\alpha>0.015$.\\
Also there are rare cases of $\frac{\Lambda(a=1)}{3H_0^2}$ becoming as much negative as $-74$. We need to understand these cases better. In particular, why don't they terminate? (\sdorange{Please list those cases. We need to check those properly.})
  % Figure environment removed
\\ \\ \\ \\ \\ \\ \\ \\ \\ \\ \\ \\ \\ \\ \\ \\ \\ \\ \\
Next, we take a constant volume, like $10^{12}Mpc^4$, and look for the first place in an expansion history that it reaches that volume (which usually happens at $a\sim\frac{1}{2}$) and record the value of $\Lambda$. We can plot several histograms. We can either plot $\frac{\Lambda}{3H^2}$ which is the energy budget of cosmological constant at that moment, we can plot $\frac{\Lambda}{3H_0^2}$ where $H_0$ is the final hubble constant for that history, or we can plot $\frac{\Lambda}{3h_0^2}$ where $h_0$ is just a normalization constant, taken to be $70.74 km/(s.Mpc)$ and is the same for all histories. \\

\sdorange{I think only the last plot is relavent as the theory sayd that $\rho_\Lambda$ will be gaussian}

\sdorange{Also take 3-4 other Volume and plot the distribution of $\rho_\Lambda$. Variance should be inversely proportional to volume.}

Q-Q plots show that only the last one is Gaussian:

% Figure environment removed

Next, we plot the histogram of the scale factor for which a run terminates. All such plots look the same, and here is one of them:


\sdorange{I think you need to remove all the run which are getting terminated at the beginning. Only then it will provide us some relevant plots. Remove all the runs which are crashing within lets say first 10 steps and then make the plots.}

% Figure environment removed
  
\section{Number of terminations}
$\alpha=0.01$, $\delta_t=3$ :\\
---------------------------------\\
Run:                     1770 \\
Crashed:              8230   \\
Crashed after 5: 1153\\
Run/Crash5 :      1.54\\
\\
\\
$\alpha=0.01$, $\delta_t=2$ : \\
---------------------------------\\
Run:                     2823 \\
Crashed:              7147   \\
Crashed after 5: 1587\\
Run/Crash5 :      1.77\\
\\
\\
$\alpha=0.01$, $\delta_t=1$ :\\
-------------------------------\\
Run:                     4748\\
Crashed:             5252   \\
Crashed after 5: 2369\\
Run/Crash5 :      2.0\\
\\
\\
$\alpha=0.015$ $\delta_t=1$:\\
-------------------------------\\
Run:                     256\\
Crashed:              9744   \\
Crashed after 5: 4275\\
Run/Crash5 :      0.0599\\
\\
\\
$\alpha=0.02$, $\delta_t=1$ :\\
-------------------------------\\
Run:                     9\\
Crashed:              9991  \\ 
Crashed after 5:  3090\\
Run/Crash5 :      0.0029\\

  
\section{\sdorange{Our discussion on 1st April and the issue in Ever present Lambda model}}
\subsection{\sdorange{Issue with the initial volume}}
In our equation 
\begin{eqnarray}
\mathcal{V}(t) &=&\frac{4 \pi}{3} \int_{0}^{a} da^{\prime} \left(\frac{d \tau^{\prime}}{da^{\prime}}\right) a\left(t^{\prime}\right)^{4}\left( \tau - \tau' \right)^{3} \nonumber \\
&=& \frac{4 \pi}{3} \int_{0}^{10^{-8}} da^{\prime} \left(\frac{d \tau^{\prime}}{da^{\prime}}\right) a\left(t^{\prime}\right)^{4}\left( \tau - \tau' \right)^{3} + \frac{4 \pi}{3} \int_{10^{-8}}^{a} da^{\prime} \left(\frac{d \tau^{\prime}}{da^{\prime}}\right) a\left(t^{\prime}\right)^{4}\left( \tau - \tau' \right)^{3} \nonumber
\end{eqnarray}

\begin{enumerate}
\item  Without any inflationary field its not possible to calculate the first integral. Now we are taking the first integral to be $0$ which is definitely now correct. In fact due to inflation it will take very large value. 
\item As $a\longrightarrow 0$, $\tau$ will become infinite. (I think that the integration will converge because there is a $a^4$ factor. However, we need to calculate the integration analytically. )
\end{enumerate}

\subsection{\sdorange{Issue with energy conservation}}


Einstein equations and the Newtons equations are the same except in 3 cases. 1) There is a strong gravitation field, 2) Gravitation field rapidly varying (for GW) 3) Concerned speeds are close to $c$. In our universe none of these are the case. So we can check whats going on in the Newtons model, because its easier to visualize. 

Consider a comoving test particle of mass $m$ located at a position $\mathbf{r}=a(t) \mathbf{x}$ relative to a comoving observer defining the origin. Next we examine the gravitational attraction experienced by the test particle due to the ball of fluid of radius $|r|$ centred at the observer. Conservation of energy requires that
\begin{equation}
    \frac{1}{2} m \dot{\mathbf{r}}^{2}-\frac{G M m}{|\mathbf{r}|}=\frac{1}{2} m\left(\dot{a}^{2}-\frac{8}{3} \pi G \rho a^{2}\right) \mathbf{x}^{2}=-\frac{1}{2} m k c^{2} \mathbf{x}^{2}
\end{equation}
since $M=\frac{4}{3} \pi \rho|\mathbf{r}|^{3}$, the gravitational attraction being determined only by the mass within the ball, here $k$ is a constant proportional to the conserved energy, for consistency it is independent of $\mathrm{x}$. Rearranging this equation leads to the Friedmann equation:
$$
\dot{a}^{2}+k c^{2}=\frac{8 \pi G \rho a^{2}}{3}
$$
The first law of theomodynamics, $\mathrm{d} E=\mathrm{d}\left(\rho c^{2} V\right)=-p \mathrm{~d} V$ and thus
$$
\dot{\rho}=-\frac{\dot{V}}{V}\left(\rho+\frac{p}{c^{2}}\right)=-\frac{3 \dot{a}}{a}\left(\rho+\frac{p}{c^{2}}\right)
$$
since comoving volumes are proportional to $a^{3}$. So, differentiating the first equation and replacing the second we get  
$$
\ddot{a}=-\frac{4 \pi G}{3}\left(\rho+\frac{3 p}{c^{2}}\right) a
$$

----------------------

We must not add energy in one side of the equation. We need to add same amount of energy in both the sides of the equation if we want a consistent equation. Right hand side will give us a overall dark energy. However it will not satisfy the energy conservation equation. 

\section{On pressure of the dark energy- 14/04/2022}

Assuming the 1st Friedmann equation and the continuity equation for matter and radiation, we can derive the following equation:
$$\frac{\ddot{a}}{a}=-\frac{4\pi G}{3}(\rho +3p)+\frac{1}{6}(2\Lambda+\frac{\dot{\Lambda}}{H}) $$
Therefore, if we want to keep the second Friedmann equation, in order to be consistent with the perturbation theory which also uses the full Einstein equation, we have to say that the pressure of dark energy is:
$$8\pi G \rho_{\Lambda}=\Lambda, \hspace{1cm}8\pi G p_{\Lambda}=-\Lambda-\frac{\dot{\Lambda}}{3H}\\
\hspace{1cm} \Rightarrow\hspace{5mm} w=-1-\frac{\dot{\Lambda}}{3H\Lambda}$$
Here are the plots of $H$, $\ddot{a}$, and $w$ for a run that terminated at $a=0.9$: 

% Figure environment removed

I have omitted the initial 100 data point for the plot of $H$ and $\Omega_{\Lambda}$. For computing $\ddot{a}$, I used spline to make $\frac{da}{dt}$ smooth and then take derivatives. $w$ oscillates badly and its mean is NOT -1 (in this case it is -14). Since the constraints people have put on $w$ are tight, we should discuss if we really want to take this pressure expression seriously. After all, this pressure has no interpretation in the gravitational action (?).  

\section{Marginalizing over seeds- 21/04/2022}
The true physical parameters of the model (at least those relevant for supernova) are $(\alpha, \Omega_m h^2)=\theta$. There is also the auxiliary parameter $seed=s$, which we need for calculating likelihoods.
$$\begin{aligned}L\left( \Omega _{m}h^{2},\alpha \right) =L\left( \theta \right) =P\left(  D| \theta \right) =\sum _{seed}P\left(  D| \theta ,seed\right) P\left( seed\right) \\
=\dfrac{1}{N}\sum _{s}P\left(  D| \theta ,s\right) =\dfrac{1}{N}\sum _{s}L\left( \theta ,s\right) \end{aligned}$$
By $D$ I mean data, and N is the number of seeds. So the likelihood on $\theta$ is an average over $L(\theta,s)\propto \exp(-1/2\chi^2(\theta,s))$.\\
Therefore, for a correct MCMC in $\theta$ space, we have to use $$\dfrac{L\left( \theta _{i+1}\right) }{L\left( \theta i\right) }=\dfrac{\sum _{s}\exp \left( -\dfrac{1}{2}\chi ^{2}\left( \theta _{i+1},s\right) \right) }{\sum _{s'}\exp \left( -\dfrac{1}{2}\chi ^{2}\left( \theta _{i},s'\right) \right) }.$$ This suggests the following recipe: Take each step of MCMC only in $(\alpha, \Omega_m h^2)=\theta$ space; for each $\theta$, take 1000 seeds, and compute the $\chi^2$ for those 1000 seeds in \boldsymbol{parallel} in the cluster, then sum over likelihoods to find the transition probabilities. We will be sampling the posterior: $P(\theta\sdorange{|D})=\frac{L(\theta)}{Z}$.\\
One might want to work things out in 3d $(\theta,s)$ parameter space instead. 2 ways come to mind:\\
1. Change seed at each step. The ratio of transition probabilities would be:
$$\dfrac{T\left(  \theta _{i+1},s_{i+1}| \theta _{i},s_{i}\right) }{T\left(  \theta _{i},s_{i}| \theta _{i+1},s_{i+1}\right) }=\dfrac{L\left( \theta _{i+1},s_{i+1}\right) }{L\left( \theta _{i},s_{i}\right) }=\dfrac{\exp \left( -\dfrac{1}{2}\chi^{2}\left( \theta _{i+1},s_{i+1}\right) \right) }{\exp \left( -\dfrac{1}{2}\chi^{2}\left( \theta _{i},s_{i}\right) \right) }.$$
By updating $s$ at each step, we will be sampling the posterior $P_1(\theta,s|D)=\frac{1}{N}\frac{L(\theta,s)}{Z}$ ($\frac{1}{N}$ is due to the prior on seeds); summing over $seed$ will give us the desired 2d posterior: $P(\theta|D)=\sum_{s} P_1(\theta,s|D)$. The downside is that this should be done in series in the cluster which takes too long.
\\2. Keep the seed constant at each run of MCMC. So take a constant $s$ and update $\theta$. The transition probabilities are:
$$\dfrac{T\left(  \theta _{i+1},s| \theta _{i},s\right) }{T\left(  \theta _{i},s| \theta _{i+1},s\right) }=\dfrac{L\left( \theta _{i+1},s\right) }{L\left( \theta _{i},s\right) }=\dfrac{\exp \left( -\dfrac{1}{2}\chi^{2}\left( \theta _{i+1},s\right) \right) }{\exp \left( -\dfrac{1}{2}\chi ^{2}\left( \theta _{i},s\right) \right) }$$
After one run, we will sample the posterior $P_2(\theta|s,D)=\frac{L(\theta,s)}{Z_2(s)}$ for a constant $s$, which this time satisfies: $\int d\theta P_{2}\left( \theta |s,D\right) =1$; therefore $Z_{2}\left( s\right) =\int d\theta L\left( \theta ,s\right) $. So $$\Rightarrow P_{2}\left( \theta |s,D\right) =\dfrac{L\left( \theta ,s\right) }{\int d\theta 'L\left( \theta ',s\right) }$$Hence, unlike $P_1$, here $P(\theta|D)\neq\sum_{s} P_2(\theta|s,D)$. In fact there is no simple algebraic relation between our desired posterior $P$ and this $P_2$ (\sdorange{Wrong. There exists}).(Note that in the previous method, since we were also searching in seed numbers, we had $\sum_s\int d\theta P_{1}\left( \theta ,s\right) =1$ and so $Z =\frac{1}{N}\sum_s\int d\theta L\left( \theta ,s\right)$ which is s-independent.)

\sdorange{I think what you point out is correct. However, I think it can be solved easily. We can each of the  chains independently and then combine those.} 

\sdorange{$Z = \int d\theta' L(\theta)=\int d\theta' P(D|\theta)=P(D) = 1$ ... as data is given}

\sdorange{$Z = \int d\theta' L(\theta,s)=\int d\theta' P(D|\theta,s)=P(D|s)$}

\sdorange{$P(D|s)$ is the probability of data given a particular seed. When all the chains are running for equal number of chain points the number of accepted data points are giving the same thing. When you are taking all the samples of equal size we are calculating the same probability. I mean for each seed P(D|s) is just $\sum exp(-\chi^2/2)$}

\sdorange{Now $P(\theta|D) = \sum P(\theta, s |D) P(D|s) = \sum L(\theta,s) = L(\theta)$ as prior on $s$ is constant.} 

Update on method 2:\\
After finding $P_2$, we can do a summation with proper coefficients to find the real posterior $P$. Just note that:
$$P(\theta|D)=\sum_{s} P_2(\theta|s,D)P(s|D)$$
On the other hand $P(s|D)=\frac{P(D|s)\pi(s)}{\sum_{s'}P(D|s')\pi(s')}=\frac{\int d\theta P(D|s,\theta)\pi(\theta)}{\sum_{s'}\int d\theta' P(D|s',\theta')\pi(\theta ')}=\frac{\int d\theta L(\theta,s)}{\sum_{s'}\int d\theta' L(\theta',s')}$



\section{MCMC results for supernovae}
Here are the three seed numbers that give results better than $\Lambda$CDM.
\subsection{seed -10892}
% Figure environment removed

% Figure environment removed
Why is $\alpha$ vs $\Omega_m$ perfectly correlated? The reason is that if you keep the seed and $\alpha$ constant, and then multiply $\rho_m^0$ by some constant $c_0$, then all $\Lambda$, $H^2$, and $\frac{1}{\sqrt{V}}$ also get multiplied by $c_0$. In particular, this is saying that the ratio $\frac{\rho_m}{\Lambda}$ and hence $\Omega_m$ remain constant if we keep $\alpha$ and the series of random numbers fixed. Therefore, for a fixed seed number, one will be moving along some fixed curve in the $\alpha-\Omega_m$ plane.

\subsection{seed -11126}
% Figure environment removed

% Figure environment removed


\subsection{seed -11927}
% Figure environment removed

% Figure environment removed

\section{CMB Comments}

\begin{itemize}
    \item There are seeds which are coming close to LCDM ( $\chi^2$ diff about 20 ).  The seeds for which $(\Omega_m)_{min}$ is very high is not working that well.  
    \item One issue is that for most of the cases the Hubble parameter is becoming very low, sometimes even less than 50. 
    \item  The seeds are not matching with the supernova seeds. 
    \item Based on the data we can do a joint analysis. ( Supernova + CMB )
\end{itemize}


% Figure environment removed

% Figure environment removed




\end{document}
\usepackage[english]{babel}
\usepackage[utf8]{inputenc}
\usepackage{wasysym, stackengine, makebox, graphicx}
\newcommand\isom{\mathrel{\stackon[-0.1ex]{\makebox*{\scalebox{1.08}{\AC}}{=\hfill\llap{=}}}{{\AC}}}}
\newcommand\nvisom{\rotatebox[origin=cc] {-90}{$ \isom $}}
\usepackage{sectsty}
\sectionfont{\large}
\subsectionfont{\small}
\usepackage[font=small,labelfont=bf]{caption}
\newtheorem*{theorem*}{Theorem}
\newtheorem*{proposition*}{Proposition}
\usepackage{amsthm}
\newtheorem{alphateo}{Theorem}
\renewcommand{\thealphateo}{\Alph{alphateo}}
\usepackage{layout}
\usepackage{geometry}

\setlength{\paperwidth}{21cm}   % A4
\setlength{\paperheight}{29.7cm}% A4
\setlength\topmargin{-0.5cm}    
\setlength\oddsidemargin{0cm}   
\setlength\textheight{24.7cm} 
\setlength\textwidth{16.0cm}
\setlength\columnsep{0.6cm}  
\newlength\titlebox 
\setlength\titlebox{5cm}
\setlength\headheight{5pt}   
\setlength\headsep{0pt}


\title{Monodromy kernels for strata of translation surfaces}

\author{Riccardo Giannini}

\date{} %leave blank


\begin{document}

\maketitle

\begin{abstract} 
\noindent 
    {\footnotesize The non-hyperelliptic connected components of the strata of translation surfaces are conjectured to be orbifold classifying spaces for some groups commensurable to some mapping class groups. The topological monodromy map of the non-hyperelliptic components projects naturally to the mapping class group of the underlying punctured surface and is an obvious candidate to test commensurability. In the present article, we prove that for the components $\mathcal{H}(3,1)$ and $\mathcal{H}^{nh}(4)$ in genus 3 the monodromy map fails to demonstrate the conjectured commensurability. In particular, building on work of Wajnryb, we prove that the kernels of the monodromy maps for $\mathcal{H}(3,1)$ and $\mathcal{H}^{nh}(4)$ are large, as they contain a non-abelian free group of rank $2$.} \end{abstract}

\section*{Introduction}
\section{Introduction}
Current quantum hardware is unable to carry out universal quantum computations due to the buildup of errors that occur during the computation. 
The magnitude of the individual error is currently above the value that the Threshold Theorem requires in order to kick-start quantum error correction and fault-tolerant quantum computation~\cite[Section 10.6]{nielsen_chuang_2010}. 
Although the experimentally achieved fidelity rates are promising and the error bounds are inching closer to the required threshold, we will have to work for the foreseeable future with quantum hardware with errors that build-up during the computation.  This implies that we can only do a limited number of steps before the output of the computation has become completely uncorrelated with the intended one.

For fault-tolerant quantum computing, we repeat four steps: 
1) We apply a number of single and two-qubit quantum gates, in parallel whenever possible; 
2) We perform a syndrome measurement on a subset of the qubits; 
3) We perform fast classical computations to determine which errors have occurred and how to correct them; 
and, 4) We apply correction terms based on the classical computations.
We then repeat these four steps with a next sequence of gates. 
These four steps are essential to fault-tolerant quantum computing. 


The starting point of this work is to use the four steps outlined above, not to carry out error correction and fault-tolerant computation, but to enhance short, constant-depth, {\em uncorrected} quantum circuits that perform single qubit gates and {\em nearest-neighbor} two qubit gates. 
Since in the long run we will have to implement error-correction and fault-tolerant computation anyhow, and this is done by such a four-step process, why not make other use of this architecture? Moreover, on some of the quantum hardware platforms, these operations are already in place.
Embracing this idea we naturally arrive at the question: what is the computational power of \textit{low-depth} quantum-classical circuits organized as in the four steps outlined above? 
We thus investigate circuits that execute a small, ideally constant, number of stages, where at each stage we may apply, in parallel, single qubit gates and {\em nearest-neighbor} two qubit gates, followed by measurements, followed by low-depth classical computations of which the outcome can control quantum gates in later stages. 
It is not clear, at first, whether such circuits, especially with constant depth, can do anything remotely useful. 
But we will see that this is indeed the case: many quantum computations can be done by such circuits in constant depth. 
By parallelizing quantum computations in this way, we improve the overall computational capabilities of these circuits, as we do not incur errors on qubits that are idle, simply because qubits are not idle for a very long time. 
Furthermore, reducing the depth of quantum circuits, at the cost of increasing width, allows the circuit to be run faster even if errors occur.

The first usage of such a four-step layout, not to do error correction, but to perform computations, can be found in the paradigm of measurement-based quantum computing~\cite{gottesman1999demonstrating,raussendorf2001one,jozsa2006introduction,clark2007generalised}: 
A universal form of quantum computing where a quantum state is prepared and operations are performed by measuring qubits in different bases, depending on previous measurements and intermediate measurements.

\citeauthor{PhamSvore2013} were the first to formalize the four-step protocol for performing computations~\cite{PhamSvore2013}. They included specific hardware topologies by considering two-dimensional graphs for imposing constraints on qubit interactions. In their model, they develop circuits for particularly useful multi-qubit gates, including specifying costs in the width, number of qubits, depth, number of concurrent time steps, size, and total number of non-Identity operations.
As a result, they find an algorithm that factors integers in polylogarithmic depth.
\citeauthor{Browne:2011} showed that the main tool in the work by \citeauthor{PhamSvore2013}, the fan-out gate, can also be replaced by additional log-depth classical computations in the measurement-based quantum computing setting~\cite{Browne:2011}.

More recently, \citeauthor{Cirac:2021} introduced a scheme to implement unitary operations involving quantum circuits combined with Local Operations and Classical Communication ($\mathsf{LOCC}$) channels: $\mathsf{LOCC}$-assisted quantum circuits~\cite{Cirac:2021}. Similarly to the four-step scheme we just described, they allow for a short depth circuit to be run on the qubits, followed by one round of $\mathsf{LOCC}$, in which ancilla qubits are measured and local unitaries are applied based on the measurement outcomes. They show that in this model any 1D transitionally invariant matrix-product state (MPS) with fixed bond dimension is in the same phase of matter as the trivial state. Similar ideas can be found in~\cite{TVV_NonAbelianTopologicalOrder_2022, tantivasadakarn2021long}.

In this work, we introduce a new model, called \textit{Local Alternating Quantum-Classical Computations} ($\LAQCC$). In this model we alternate between running quantum circuits (constrained by locality), ending in the measurement of a subset of qubits, and fast classical computations based on the measurement results. The outcome of the classical computations are then used to control future quantum circuits. We allow for flexibility in this model, by giving different constraints to the power of both the quantum circuits and the classical circuits as well as the number of alternations between them. 
Most attention will be given to $\LAQCC$ containing quantum circuits of constant depth, classical circuits of logarithmic depth and at most a constant number of alternations between them. 
Any circuit constructed in this model is considered to be of constant depth. 
We restrict ourselves to logarithmic depth classical computations, as this is the first natural and non-trivial extension beyond constant-depth classical computations. 
Constant-depth classical computations do however also have an equivalent constant-depth quantum implementation.

The definition of $\LAQCC$ sharpens the original definition of \citeauthor{PhamSvore2013} by adding constraints to the intermediate classical computations. This allows us to bound the power of $\LAQCC$ from above. 

The main result of \citeauthor{Cirac:2021}, that 1D translational invariant MPS with fixed bond dimension can be prepared by $\mathsf{LOCC}$-assisted circuits, relies on local symmetries of the MPS. These symmetries allow them to prepare local states (on a constant number of qubits) and glue them together by doing one round of the appropriate entangling measurement and corrections, after which they run a round of local unitaries to get the desired result. This general scheme for preparing states that exhibit an MPS description with the appropriate local symmetries requires only geometrically local unitaries and one round of measurement and corrections an therefore is accessible in $\LAQCC$. Studying different local symmetries, known as Symmetry Protected Topological (SPT) phases of matter, to find measurement-based constant depth circuits for states is a broad ongoing field of research~\cite{TVV_NonAbelianTopologicalOrder_2022, tantivasadakarn2021long, smith2023deterministic}. 
All these schemes have a $\LAQCC$ implementation.

%$\LAQCC$-circuits also exist for general schemes of preparing local states, based on the local tensors, and gluing them together using one round of entangled measurement and corrections, based on the local symmetry. 
%The main result of \citeauthor{Cirac:2021}, that 1D translational invariant MPS with fixed bond dimension can be prepared by $\mathsf{LOCC}$-assisted circuits, relies heavily on local symmetries of the MPS and as a result also has an equivalent $\LAQCC$ implementation. 
%The corrections applied after the measurement round are local unitaries depending on the local symmetries of the MPS. 

 

%This general scheme of preparing local states, based on the local tensors, and gluing it together by doing one round of entangled measurement and corrections, based on the local symmetry, is accessible in $\LAQCC$.
Note however that \citeauthor{Cirac:2021} also suggest a circuit for the $W$-state.
This circuit uses sequentially and dependent measurement-based corrections of the ancilla qubits. 
These dependent measurements translate to sequential alternations between the quantum and classical circuits and therefore increase the total depth to linear depth, exceeding the constant-depth constraints imposed by $\LAQCC$-circuits. 

We study the power of the $\LAQCC$ model with respect to state preparation, showing that even with only constant quantum-depth and logarithmic classical depth it remains possible to prepare states with long-range entanglement.
Another surprising result is that it is unlikely that $\LAQCC$ circuits are classically simulatable. We show that any instantaneous quantum polynomial-time (IQP) circuit~\cite{Bremner2010,Shepherd2009} has an $\LAQCC$ implementation.
Classical simulation of IQP circuits implies the collapse of the polynomial hierarchy to the third level, which is not believed to be true~\cite{Bremner2017}. Therefore, we expect that $\LAQCC$ circuits are unlikely to be classically simulatable. We bound the power of $\LAQCC$ by showing that it is contained in $\QNC^1$, the class of polynomial-size, log-depth circuits.

Next, we also study the power that intermediate classical calculations can add to quantum computations, by considering a new model that alternates between polynomially many polynomial-depth quantum circuits and unbounded classical computations
We study this model by doing a complexity theoretical analysis, where we draw inspiration from the notions of complexity given by \citeauthor{RosenthalYuen:2022}, \citeauthor{MetgerYuen:2023}, and \citeauthor{Aaronson:2004}.
All three complexity notions are based on the notion of state preparation, instead of more traditional definition of complexity such as the decidability of a computational problem. 
The first two consider classes based on sequences of quantum states preparable by a polynomial-sized quantum circuit, where the circuits are uniformly generated by a computational class, for instance, the class $\mathsf{PSPACE}$, which results in the complexity class $\mathsf{StatePSPACE}$~\cite{RosenthalYuen:2022,MetgerYuen:2023}.
The third notion considers a relative complexity, where the complexity is measured between two given states, and is measured by the number of gates, from a given gate-set, required to transform one state in another state~\cite{Aaronson:2004}. 
For our definition of state preparation complexity, we drop the uniformity constraint from~\cite{RosenthalYuen:2022,MetgerYuen:2023} and define a class as $\mathsf{StateX}$, which refers to states preparable by circuits of type $\mathsf{X}$. 
As an example, if $\mathsf{X} = \QNC^0$, this results in the class $\mathsf{StateQNC^0}$, which is the set of states preparable from the $\ket{0}^n$ state by poly-size constant-depth circuits. 
This notion is similar to the relative complexity from~\cite{Aaronson:2004}, where one state is the  $\ket{0}^n$ state and instead of counting the number of gates we consider the set of states preparable by a fixed number of gates. Using this notion of complexity we show that any state preparable by an $\LAQCC^*$ circuit is also preparable by a $\mathsf{PostQPoly}$ circuit, the class of circuits of polynomial depth with an additional post-selection gate. 

All Clifford circuits have a constant-depth $\LAQCC$ implementation, implying that any stabilizer state can be implemented by a constant-depth $\LAQCC$ circuit, see Section~\ref{sec:clifford_circuits} for a proof of this statement. 
Efficient circuits for stabilizer states have been known already through measurement-based quantum computing. Therefore this paper focuses on the preparation of non-stabilizer states, and as a surprising result we find novel constant-depth protocols for four very natural classes of non-stabilizer states.
Despite the extensive research into these four classes of non-stabilizer states and the many applications of them, no efficient constant- or low-depth state preparation protocols are known yet. We specifically consider these four classes as they are all often used as initial states in other algorithms.

The first state is a uniform superposition over an arbitrary number of states. 
This state finds applications in many quantum algorithms, as they often start with a uniform superposition over multiple states. 
This superposition is often achieved by applying Hadamard gates to every qubit due to its simplicity to prepare. 
Yet, the analysis of many algorithms, such as Shor's algorithm~\cite{Shor:1997}, would benefit from a different initial superposition. 
The circuit to prepare the uniform superposition over an arbitrary number of states uses an exact version of Grover search as a subroutine, that turns a probabilistic circuit, with a known constant probability of success, into a deterministic circuit. 
We use the circuit for preparing a uniform superposition over an arbitrary number of states as a subroutine in the next two quantum state preparation protocols. 

The second state is the $W$-state, the uniform superposition over all computational basis states of Hamming-weight~$1$, a natural long-ranged entangled state that displays a fundamentally nonequivalent type of entanglement from the Greenberger–Horne–Zeilinger state~\cite{WState:2000}, for which $\LAQCC$-type constant-depth circuits were previously known~\cite{PhamSvore2013, Cirac:2021}. 
The $W$-state is often used as benchmark for new quantum hardware~\cite{Haffner2005,Neeley2010,GarciaPerez:2021}. 
A novel way to prepare the $W$-state therefore gives a new way to benchmark different quantum devices with each other. 
A circuit for preparing the $W$-state was given in~\cite{Cirac:2021}, but this implementation requires sequentially alternating measurements followed by local unitaries, which in the $\LAQCC$ model is not considered to be of constant depth. 
We improve this protocol by giving an $\LAQCC$ implementation of the $W$-state, based on a compress-uncompress method that links the one-hot and binary encoding of integers.

The third state considered is the Dicke state, a generalization of the $W$-state, a superposition over all computational basis states with Hamming-weight $k$~\cite{Dicke:1954}. 
Dicke states have relevance in various practical settings.
For instance, for quantum game theory~\cite{zdemir2007}, quantum storage~\cite{Bacon_Compress:2006,Plesch:2010}, quantum error correction~\cite{ouyang2014permutation}, quantum metrology~\cite{toth2012multipartite}, and quantum networking~\cite{prevedel2009experimental}. 
Dicke states have been used as a starting state for variational optimization algorithms, most notably Quantum Alternating Operator Ansatz (QAOA)~\cite{Hadfield2019}, to find solutions to problems such as Maximum k-vertex Cover~\cite{Brandhofer2022,cook2020quantum}.
The ground states of physical Hamiltonians describing one-dimensional chains tend to show a resemblance to Dicke states such as states resulting from the Bethe ansatz, making them an ideal starting state when investigating the ground state behavior of these Hamiltonians~\cite{TDL_BetheAnsatzDerivation:2010,B_ExcitedStateQuantumPhaseTransitions:2013,DickeTransitions:2021}. 
For instance, the algorithm by \citeauthor{van2021preparing}, who give an algorithm to prepare the Bethe ansatz eigenstates of the spin-1/2 XXZ spin chain, starts by first preparing a Dicke state~\cite{van2021preparing}. 
A Dicke-state preparation protocol based on the compress-uncompress methodology used in the $W$-state furthermore finds applications in entanglement distillation, where the entanglement of a large state is concentrated on only a few qubits. 
Efficient deterministic circuits for preparing Dicke states have been proposed by \citeauthor{bartschi2019deterministic}~\cite{bartschi2019deterministic, bartschi2022deterministic_short_depth}. 
They provide a quantum circuit of depth $\mathO(k \log(\frac{n}{k}))$, allowing arbitrary connectivity, to prepare a Dicke state, which they conjecture to be optimal when $k$ is constant. 
In this work, we provide a constant-depth $\LAQCC$ circuit below their conjectured bound already for constant $k$. 
However, this does not directly disprove their conjecture, as we allow for intermediate measurements and classical computations. 
More significantly, we even construct constant-depth $\LAQCC$ circuits for $k = \mathO(\sqrt{n})$ greatly improving their bound.
This construction extends the compress-uncompress method for the $W$-state combined with additional subroutines. 

We continue with a log-depth state preparation protocol for the Dicke-state for arbitrary $k$. 
This protocol implements an efficient transformation between the factoradic number representation and the combinatorial number representation of a positive integer. 
The combinatorial number representation relates directly to the Dicke state. 
The provided efficient transformation between number representation systems might be of independent interest. 

We conclude by modifying our protocol for preparing a Dicke-state to a protocol that prepares quantum many-body scar states in constant-depth. 
These states have low entanglement and longer coherence times than states with similar energy density.
These characteristics make many-body scar states interesting to analyze and relevant within physics.
Many-body scar states appear for instance in the AKLT model~\cite{AKLT:1987,MRBAR:2018,MRB:2018} and different spin models~\cite{SI:2019,MOBFR:2020}.
Known methods for preparing these states have polynomial-depth~\cite{Gustafson:2023}, whereas our circuit has constant depth. 

% We conclude by studying the power that intermediate classical calculations can add to quantum computations. 
% In this study, we define a new model that relaxes constant-depth quantum circuits to polynomial depth quantum circuits, log-depth classical calculations to unbounded classical computations and a constant number of alternations to a polynomial number of alternations. 
% We call this model $\LAQCC^*$. 
% We study this model by doing a complexity theoretical analysis, where we draw inspiration from the notions of complexity given by \citeauthor{RosenthalYuen:2022}, \citeauthor{MetgerYuen:2023}, and \citeauthor{Aaronson:2004}.
% All three complexity notions are based on the notion of state preparation, instead of more traditional definition of complexity such as the decidability of a computational problem. 
% The first two consider classes based on sequences of quantum states preparable by a polynomial-sized quantum circuit, where the circuits are uniformly generated by a computational class, for instance, the class $\mathsf{PSPACE}$, which results in the complexity class $\mathsf{StatePSPACE}$~\cite{RosenthalYuen:2022,MetgerYuen:2023}.
% The third notion considers a relative complexity, where the complexity is measured between two given states, and is measured by the number of gates, from a given gate-set, required to transform one state in another state~\cite{Aaronson:2004}. 
% For our definition of state preparation complexity, we drop the uniformity constraint from~\cite{RosenthalYuen:2022,MetgerYuen:2023} and define a class as $\mathsf{StateX}$, which refers to states preparable by circuits of type $\mathsf{X}$. 
% As an example, if $\mathsf{X} = \QNC^0$, this results in the class $\mathsf{StateQNC^0}$, which is the set of states preparable from the $\ket{0}^n$ state by poly-size constant-depth circuits. 
% This notion is similar to the relative complexity from~\cite{Aaronson:2004}, where one state is the  $\ket{0}^n$ state and instead of counting the number of gates we consider the set of states preparable by a fixed number of gates. Using this notion of complexity we show that any state preparable by an $\LAQCC^*$ circuit is also preparable by a $\mathsf{PostQPoly}$ circuit, the class of circuits of polynomial depth with an additional post-selection gate. 

\paragraph{Summary of results}
\begin{itemize}
    \item We give a new definition of a computational model that captures the power of the four step process: applying a constant number of layers of one- and two-qubit gates; performing a syndrome measurement; perform a fast classical computation determining corrections; apply corrections. We call this model \emph{Local Alternating Quantum Classical Computations}, or $\LAQCC$ for short. In this model we bound the allowed quantum operations, intermediate classical calculations, and number of rounds separately. In Section~\ref{sec:LAQCC_model} we define this model and give a list of operations based on results from literature contained in this computational model. In some of these operations we explicitly use that we allow for multiple, but at most constant, rounds  of corrections.
    \item  We show show that there exist $\LAQCC$ circuits that can not be weakly simulated in Section~\ref{sec:IQP_in_LAQCC}. We further show that for every $\LAQCC$ circuit there exists a $\QNC^1$ circuit simulating it perfectly, in Section~\ref{sec:LAQCC_in_QNC1}.
    \item We introduce a new type computational complexity for preparing states and show that the extension of $\LAQCC$ where we allow a polynomial number of rounds and unbounded classical computation, is contained in $\mathsf{PostQPoly}$, the class of polynomial circuits with post-selection, in Section~\ref{sec:Complexity results}.
    \item We show a protocol to prepare the uniform superposition state of size $q$ in $\LAQCC$ using $\mathO(\ceil{\log_2(q)}^2)$ qubits in Section~\ref{sec:superposition_modulo_q}. 
    \item We show a protocol to prepare the $W_n$ state in $\LAQCC$ using $\mathO(n\log(n))$ qubits in Section~\ref{sec:W_state_in_LAQCC}.
    \item We show two ways of preparing the Dicke-$(n,k)$ state. The first method is in $\LAQCC$, works up to $k = \mathO(\sqrt{n})$, uses $\mathO(n^2\log(n))$ qubits, and is found in Section~\ref{sec:dicke:small_k}. The second method is in $\LAQCC\text{-}\mathsf{LOG}$ (an extension of $\LAQCC$ allowing for logarithmic number of alterations instead of constant), works for any $k$, uses $\mathO(\text{poly}(n))$ qubits, and is found in Section~\ref{sec:Dicke_in_LAQCC_LOG}. 
    \item We extend on our $\LAQCC$ method of generating Dicke-$(n,k)$ states for $k = \mathO(\sqrt{n})$ and show a protocol to generate many-body scar states for a particular Hamiltonian in $\LAQCC$ (Section~\ref{sec:many_body_scar}). 
\end{itemize}
Summarized in a table, we provide the following state generation protocols:
\begin{table}[htb]
\centering
\begin{tabular}{l|l|l|l}
\textbf{State description} & \textbf{Width} & \textbf{Depth} & \textbf{Implementation}\\
\hline 
Uniform superposition mod $q$: $\frac{1}{\sqrt{q}} \sum_{i = 0}^{q-1}\ket{i}$ & $\mathO(\ceil{\log^2 q})$ & $\mathO(1)$ & Section~\ref{sec:superposition_modulo_q}\\

$W$-state: $\frac{1}{\sqrt{n}}\sum_{i = 0}^{n-1}\ket{e_i}$ & $\mathO(n \log n)$ & $\mathO(1)$ & Section~\ref{sec:W_state_in_LAQCC}\\

Dicke-$(n,k)$, $k = \mathO(\sqrt{n})$: $\binom{n}{k}^{-1/2}\sum_{x \in \{0,1\}^n: |x| = k} \ket{x}$ &  $\mathO(n^2\log n)$ & $\mathO(1)$ 
&Section~\ref{sec:dicke:small_k}\\

Dicke-$(n,k)$: $\binom{n}{k}^{-1/2}\sum_{x \in \{0,1\}^n: |x| = k} \ket{x}$ & $\mathO(\text{poly}(n))$ & $\mathO(\log n)$ &Section~\ref{sec:Dicke_in_LAQCC_LOG}\\

QMBS: $\ket{S_k} = \frac{1}{k! \sqrt{\mathcal N(n,k)}}(Q^\dagger)^k \ket{\Omega}$ &  $\mathO(n^2\log n)$ & $\mathO(1)$  &  Section~\ref{sec:many_body_scar}
\end{tabular}
\caption{Summary of state preparation protocols given in this paper.}
\label{tab:sate_prep}
\end{table}
In the entry for the quantum many-body scar state $Q$ denotes the raising operator and $\mathcal N(n,k)=\binom{n-k-1}{k}$. 
Section~\ref{sec:many_body_scar} will provide more details on the variables and the implementation. 

\paragraph{Organization of the paper}
\noindent We first introduce relevant preliminaries in Section~\ref{sec:preliminaries}. 
In Section~\ref{sec:LAQCC_model} we formally define the class of Local Alternating Quantum-Classical Computations ($\LAQCC$). We also show that any Clifford circuit can be implemented in constant depth $\LAQCC$ (a result based on a result from measurement-based quantum computing~\cite{jozsa2006introduction}). 
This result allows us to give many useful multi-qubit gates and routines in Section~\ref{sec:gates_created_in_LAQCC}. 
Beyond that we show that constant depth $\LAQCC$ circuits are contained in $\QNC^1$ and that any $\mathsf{IQP}$ circuit has an $\LAQCC$ implementation.
We conclude this section with an analysis of a more powerful instantiation of $\LAQCC$ and show an inclusion with respect to the class $\mathsf{PostQPoly}$, which is the class of circuits of polynomial depth with one additional post-selection gate. 
In Section~\ref{sec:state_prep_in_LAQCC} we give $\LAQCC$ circuit implementations for preparing the uniform superposition over an arbitrary number of states, the $W$-state and the Dicke state up to $k = \mathO(\sqrt{n})$. We furthermore give a log-depth circuit implementation for preparing the Dicke state for any $k$. We conclude by showing a $\LAQCC$ circuit for generating many body scar states of a particular type of Hamiltonian.



\section{Spherical-type Artin groups}
\section{Introduction}

Automatic Speech Recognition (ASR) is a key component in processing audio materials such as audio translation and voice assistant \cite{federmann-lewis-2016-microsoft}, and speech information extraction \cite{cho-etal-2021-streamhover}. Typical ASR systems produce chunks of transcription without any text structures such as sentence and phrase boundaries \cite{jones2003measuring}. As a result, it lowers the readability of the generated ASR texts \cite{jones2003measuring} and severely affects the performance of systems for downstream tasks over this type of text, e.g., information extraction \cite{alam2015comparing}. To address this issue, the Punctuation Restoration (PR) task has been added to the ASR systems \cite{tilk2015lstm} to improve the text readability and the performance of downstream tasks for ASR-generated texts such as question answering \cite{pouran-ben-veyseh-etal-2022-behanceqa}, chitchat detection \cite{lai-etal-2022-behancecc}, and tutorial recommendation \cite{Veyseh2022TutorialRF}. The most recent successful work for PR was all built on top of transformer-based PLMs such as BERT \cite{devlin-etal-2019-bert} and ELECTRA \cite{clark2020electra}.

%together with other post-processing tasks such as true-casing \cite{lita-etal-2003-truecasing}.

% Modern PR systems model the PR task as a word-level sequence labeling task, in which each word is labeled with either a punctuation mark or a NULL label. These models depend on  

% Due to its importance, many studies have been conducted for PR in recent years. 
%Early work employed various combinations of both text-feature and audio features \cite{}. More recent studies mainly develop advanced neural network architectures \cite{}, and integrate external knowledge \cite{}. 
Despite such progress, lacking domain-specific training data is still a major obstacle that hinders the research and development of PR systems for real-world applications \cite{lai-etal-2022-behancepr}. We identify two factors accounting for this issue. First, speech topics involve a unique set of keywords as well as slang in spoken languages. The ASR system and PR system without topic knowledge can be severely affected by the shift of topics in the source audio. Second, unlike other tasks where the unlabeled data is created by humans, the input of PR is generated by an ASR system. This creates a unique dependency that must be addressed by the PR model. Consequently, creating cost-effective datasets for a wide range of domains for PR is highly challenging.

Moreover, naive adoption of available punctuated data is problematic. While large-scale punctuated texts corpora are available, they are mostly written texts (REF texts), which are substantially well-punctuated. In contrast, ASR-generated texts (ASR texts) inherit a substantial amount of noise from both spoken language (e.g., verbal pauses) and the transcription process (e.g., word errors). Accordingly, prior studies have shown that a PR model that was trained on REF texts performed poorly on real-world ASR texts \cite{alam-etal-2020-punctuation}. In other words, directly using readily available written texts does not help to improve the PR model.


To overcome these issues, we introduce a novel data generation method to automatically generate large-scale, high-quality labeled data for PR. In particular, instead of manual annotation, we employ a pre-trained language model, namely GPT2 \cite{radford2019language}, to create synthetic labeled data for PR because generative models like GPT2 can generate punctuated texts that can be converted to labeled data for PR easily. Since the GPT2 model was trained on written texts across diverse topics, this leads to two issues that need to be addressed. 

First, the topics in the generated texts are unconstrained, which is suboptimal for some specific applications, such as gaming livestreaming. As such, we propose a method to control the topic of the generated texts. Instead of unconditional text generation, we feed the GPT2 model with an in-topic seed text, which was sampled from an in-topic unsupervised source. Hence, we encourage the GPT2 model to generate more texts within the initial topic. As a result, we can leverage GPT2's knowledge to obtain unlimited in-topic labeled texts for PR.

Second, the disconnection of the GPT2 model and the target PR model might cause a discrepancy between GPT2-generated texts and the target PR text. Therefore, to improve the quality of the  GPT2-generated data for PR, we propose to further finetune the GPT2 model in parallel with the training of the PR model to generate optimal customized texts for PR. Particularly, we propose a meta-learning framework to consider the GPT2 model as a meta-parameter for the training of the PR model, in which the GPT2 model will be fine-tuned based on the performance of the PR model on the development set. A trivial solution is reinforcement learning, where the reward can be calculated directly from the evaluation metrics of the PR model on the development set, e.g., the F1-score. However, obtaining a reliable, fast reward is challenging due to either the small scale of the evaluation or the computational cost of the evaluation that has to be done at every single iteration. To alleviate this issue, we propose a novel reward function that relies on the gradients of the PR model obtained from the generated texts and the development set. Intuitively, a generated sample should have a higher reward if the PR model's gradients derived from the sample follows the PR expected gradients derived from the development set. Toward this end, in each iteration, after generating synthetic PR data, we compute an average gradient of the PR model over the generated data for each training example. Then, we compute another average gradient of the PR model over a sampled subset of the development set. Finally, the reward for each generated sample is computed using the cosine similarity score between the two gradients. We evaluate the effectiveness of the proposed methods on two benchmark datasets for PR. The experiments show that our model outperforms the strongest baseline on both datasets.






% The GPT2 model is fed with a small chunk of in-topic text to generate more in-topic texts for PR.

% To overcome these issues, instead of annotating more data for some particular domains, we introduce a novel domain-agnostic data generation method to generate labeled PR data automatically. The generated data can inherit a wide range of keywords from the PLM, while the PLM-generated text is also customized to the domain of interest. In particular, we propose to employ a generative PLM named GPT2 to create synthetic data for PR. The GPT2 model is trained on the augmented in-topic data, and then the fine-tuned GPT2 is asked to generate a large amount of labeled semi-in-domain data. This data is then combined with the real in-domain PR data to train a PR model. Moreover, this method allows us to control the seed text that is fed to the GPT2 model, hence controlling the topic in the PLM-generated texts. Hence, this method brings the topic gap between the PLM-generated data and the target PR data.

%  As such, it is necessary that the GPT2 and the PR models can interact with each other so that the PR model can guide the GPT2 model to generate spoken texts instead of written texts. In particular, we proposed a reinforcement learning method to allow the PR model to give feedback to the GPT2 model. So, the GPT2 model can generate texts to optimize the PR model's performance.



\section{Geometric homomorphisms and the Wajnryb element}
\section{Related Work}

Early PR studies employed syntactic features and prosodic features\cite{szaszak2019leveraging} to train graphical models such as HMM and CRF \cite{tilk2015lstm}. Recent models for PR employed artificial neural networks to model the PR problem as a sequence-to-sequence problem using various network architectures such as convolutional neural network \cite{che-etal-2016-punctuation}, recurrent neural network \cite{tilk2015lstm,kim2019deep}, and transformer \cite{alam-etal-2020-punctuation}. Pretrained language models stand at the core of the recent PR models. There have been variants of pre-trained language models used for PR such as BERT \cite{fu-etal-2021-improving}, RoBERTa \cite{alam-etal-2020-punctuation,courtland-etal-2020-efficient}, ELECTRA \cite{hentschel2021making,chen2021discriminative}, XLM-RoBERTa \cite{chordia-2021-punktuator}, and funnel-transformer \cite{shi2021incorporating}. Recent advance in training and preprocessing leads to many training techniques such as data augmentation \cite{alam-etal-2020-punctuation}, adversarial training \cite{yi2020adversarial}, multitask learning \cite{lin2020joint,hentschel2021making}, self-training \cite{chen2021discriminative}, two-stage training \cite{fu-etal-2021-improving}, and contrastive learning \cite{huang2021token}. External knowledge was also incorporated into the PR model including external punctuated data \cite{fu-etal-2021-improving}, syntactic features \cite{shi2021incorporating} and acoustic features \cite{zhu2022unified}.

\section{Kernels of geometric homomorphisms}
\section{Numerical implementation}
\label{sec:3}

The collagen density evolution is computed by implicitly integrating Eq.\ (\ref{eq:2-10}). A standard Backward-Euler scheme is implemented to compute the collagen density ${\rho}^{0}_{\mathrm{co}}$. The collagen density ${\rho}^{0}_{\mathrm{co}}$ is multiplied by the mass-specific Helmholtz free energy $\mathit{\psi}_{\mathrm{co, m}}$ to obtain the corresponding energy for the collagen fibers per unit reference volume $\mathit{\psi}_{\mathrm{co}}$ as described by Eq.\ (\ref{eq:2-20}). 

\subsection{Time integration}
To apply the time-discretization of the collagen evolution equation, we define the time step as $\Delta t = (t_{\mathrm{n+1}} - t_{\mathrm{n}})$ where $t_{\mathrm{n+1}}$ refers to the current time-step and $t_{\mathrm{n}}$ refers to the previous time-step. Variables from time-step $t_{\mathrm{n}}$ are denoted by the subscript $\mathrm{n}$, and variables from the current time $t_{\mathrm{n+1}}$ are without a subscript. Consequently, the implicit Backward-Euler integration scheme can be expressed as following \begin{equation}
\label{eq:3-1}
{\rho}^{0}_{\mathrm{co}} =  ({\rho}^{0}_{\mathrm{co}})_{\mathrm{n}} + {\Delta t} \, \dot{\rho}^{0}_{\mathrm{co}},
\end{equation}
and the corresponding residual equation is 
\begin{equation}
\label{eq:3-2}
r_{\mathrm{\rho}} = {\rho}^{0}_{\mathrm{co}} - ({\rho}^{0}_{\mathrm{co}})_{\mathrm{n}} - {\Delta t} \, \dot{\rho}^{0}_{\mathrm{co}}  = 0.
\end{equation}

Solving the residual equation using the Newton-Raphson scheme gives us the densification rate $\dot{\rho}^{0}_{\mathrm{co}}$ at the current time-step $t_{\mathrm{n+1}}$. By substituting with $\dot{\rho}^{0}_{\mathrm{co}}$ in Eq.\ (\ref{eq:3-1}), we can find the corresponding collagen density ${\rho}^{0}_{\mathrm{co}}$. Then, ${\rho}^{0}_{\mathrm{co}}$ is inserted in Eq.\ (\ref{eq:2-20}) to compute the corresponding strain energy $\mathit{\psi}_{\mathrm{co}}$. The time-integration steps are explained in Algorithm \ref{densification_algorithm}.

\vspace{0.5cm}
\setlength\belowcaptionskip{1ex}
\begin{algorithm}[H]
	\begin{algorithmic}[1]
		\State Initialize Backward-Euler integration scheme
		\State Input $a_{1}$, $a_{2}$, $c_{\mathrm{cell}}$,  $h$, $\tau$, $t$, ${\psi}_{\mathrm{crit}}$
		\State Output $\rho^{0}_{\mathrm{co}}$ 
		%\For{$t \leq t_{end}$} 
		\State Compute $\dot{\alpha}_{\mathrm{bio}} \gets  \frac{h}{\tau}\, e^{-(t / \tau)^{h}}  \, (\frac{t}{\tau})^{h - 1}$  
		\State Compute $\dot{\rho}^{0}_{\mathrm{bio}} \gets a_{1} \, c_{\mathrm{cell}} \, \dot{\alpha}_{\mathrm{bio}}$
		\State Compute ${\psi}_{\mathrm{co, m}}$
		\If{${\psi}_{\mathrm{co, m}}$ $\geq {\psi}_{\mathrm{crit}} $}
		\State Compute $\dot{\alpha}_{\mathrm{mech}} \gets  e^{-({\rho}^{0}_{\mathrm{co}} / \rho_{\mathrm{th}})}$ 
		\State Compute $\dot{\rho}^{0}_{\mathrm{mech}} \gets a_{2} \, c_{\mathrm{cell}} \, \dot{\alpha}_{\mathrm{mech}} \, {\rho}^{0}_{\mathrm{co}} \, \frac{({\psi}_{\mathrm{co, m}} - {\psi}_{\mathrm{crit}})}{{\psi}_{\mathrm{crit}}} $
		\State $r_{\mathrm{\rho}} \gets {\rho}^{0}_{\mathrm{co}} - ({\rho}^{0}_{\mathrm{co}})_{\mathrm{n}} - {\Delta t} \, \dot{\rho}^{0}_{\mathrm{co}}  = 0 $
		\While  {$|r_{\mathrm{\rho}}| \leq tolerance$}
		\State Compute $\rho^{0}_{\mathrm{co}}$ using Newton-Raphson method
		\EndWhile
		\EndIf
	\end{algorithmic} 
	\caption{Computing collagen fiber density}
	\label{densification_algorithm}
\end{algorithm}


\subsection{Computing stresses and material tangents}
To construct a global finite element system, first we compute the second Piola-Kirchhoff stresses for each constituent and then sum up their individual contribution to get the total second Piola-Kirchhoff stress $\mathbf{S}$ as described by the Eq.\ \ref{eq:2-27}. The computation of second Piola-Kirchhoff stress of the textile scaffold $\mathbf{S}_{\mathrm{tex}}$ and the isotropic matrix $\mathbf{S}_{\mathrm{matrix}}$ can be performed in a straightforward way as already described by the Eq.\ \ref{eq:2-28} and Eq.\ \ref{eq:2-29} respectively. However, the computation of the second Piola-Kirchhoff stress for the collagen part $\mathbf{S}_{\mathrm{co}}$ requires taking into consideration the influence of the collagen evolution equations. The collagen density ${\rho}^{0}_{\mathrm{co}}$ depends on the mass-specific energy ${\psi}_{\mathrm{co, m}}$ and consequently on the Cauchy-Green tensor $\mathbf{C}$, which
introduces additional terms to compute the derivatives as shown in Eq.\ \ref{eq:2-9}. By summing up the contributions from each constituent, we end with an expression for the second Piola-Kirchhoff stress that reads  
\begin{equation}
\label{eq:3-3}
\mathbf{S} = 2 \, \left(\frac{\partial {\psi}_{\mathrm{tex}}}{\partial \mathbf{C}} + \frac{\partial {\psi}_{\mathrm{matrix}}}{\partial \mathbf{C}} + \frac{\partial {\rho}^{0}_{\mathrm{co}}}{\partial \mathbf{C}} \, {\psi}_{\mathrm{co, m}} + {\rho}^{0}_{\mathrm{co}} \, \frac{\partial {\psi}_{\mathrm{co, m}}}{\partial \mathbf{C}} \right).
\end{equation}

 The next step is to compute the material tangent operator ${\bf \mathbb{C}}$. By applying the same approach used to compute $\mathbf{S}$, ${\bf \mathbb{C}}$ is computed by summing up the tangents of each constituent as described by the following expression
\begin{equation}
\label{eq:3-4}
{ \mathbb{C}} = {\mathbb{C}_{\mathrm{tex}}  + { \mathbb{C}_{\mathrm{matrix}}} + { \mathbb{C}_{\mathrm{co}}}}.
\end{equation}

The tangent tensor for each constituent is computed by taking the partial derivative of the second Piola-Kirchhoff stress with respect to the right Cauchy-Green tensor. This gives us the following term for the textile part
\begin{equation}
\label{eq:3-5}
{ \mathbb{C}_{\mathrm{tex}}} = 2 \, \frac{\partial \mathbf{S}_{\mathrm{tex}}}{\partial \mathbf{C}} = 4 \, \frac{\partial^{2} {\psi}_{\mathrm{tex}}}{\partial \mathbf{C}^{2}}.
\end{equation}

Analogously, for the matrix part, we compute the following term
\begin{equation}
\label{eq:3-6}
{ \mathbb{C}_{\mathrm{matrix}}} = 4 \, \frac{\partial^{2} {\psi}_{\mathrm{matrix}}}{\partial \mathbf{C}^{2}}.
\end{equation}

For the collagen part, computing the tangent operator is more complex, because the Helmholtz free energy ${\psi}_{\mathrm{co}}$ depends on the collagen density ${\rho}^{0}_{\mathrm{co}}$. Similar to the other constituents, we describe the tangent operator by the following expression
\begin{equation}
\label{eq:3-7}
{ \mathbb{C}_{\mathrm{co}}} = 4 \, \frac{\partial^{2} {\psi}_{\mathrm{co}}}{\partial \mathbf{C}^{2}}.
\end{equation}

The derivatives in Eq.\ \ref{eq:3-3} and Eqs.\ \ref{eq:3-5}-\ref{eq:3-7} are computed using the code generated by the automatic differentiation software package AceGen \cite{Korelc_2002, Korelc_2009}.

\subsection{Finite element implementation}
In our structural computations, we used a special finite element technology with reduced integration, namely the solid-shell element Q1STs \cite{Reese_2007, Barfusz_2021b}. Q1STs is a low-order isoparametric element with eight nodes. Due to the application of the reduced integration concept, the element contains only one Gauss point within the shell plane. It is especially beneficial for modeling thin shell structures such as heart valves because we can use an arbitrary number of Gauss points through the element's thickness. Furthermore, Q1STs element formulation offers a remedy to volumetric and shear locking. Locking treatment and hourglass stabilization in Q1STs are achieved by applying the concepts of enhanced assumed strain and assumed natural strain. Using elements capable of treating such locking phenomena is essential for us to compute the examples presented in Section \ref{sec:5} because: (i) soft biological tissues are almost incompressible materials which makes them susceptible to volumetric locking, and (b) the structure presented in our work are under severe bending which would cause shear locking in case of using standard low-order element formulation. An additional advantage of using a solid-shell formulation is its ability to model the non-linear material behavior along the thickness direction using only one element. Consequently, we can drastically reduce the number of elements needed in our computations compared to using a standard continuum solid element.



\section{The topological monodromy for strata of translation surfaces}
\section{Experiments}

\newcommand{\tablett}[1]{\multicolumn{1}{|c|}{\textbf{#1}}}

\begin{table*}[t]
\centering
\begin{tabular}{|l|ccc|ccc|ccc|ccc|}
\hline
    \mtrb{2}{Model} & 
    \mytitle{3}{Comma} & 
    \mytitle{3}{Period} & 
    \mytitle{3}{Question} & 
    \mytitle{3}{Overall} \\
    \cline{2-13}
    & P & R & F & P & R & F & P & R & F  & P & R & F  \\
    \hline 
    \hline
    % \myrotate{6}{\textbf{ASR}}
    RoBERTA-large \cite{lai-etal-2022-behancepr} & - & - & - & - & - & - & - & - & - &  62.0 & 61.4 & 61.7 \\
    + Augmentation & - & - & - & - & - & - & - & - & - &  63.8 & 60.7 & 62.2 \\
    + CRF & - & - & - & - & - & - & - & - & - &62.2 & 63.5 & 62.9 \\
    + CRF + Augmentation & - & - & - & - & - & - & - & - & - & 61.1 & 62.8 & 62.0 \\
    % ELECTRA-base & 63.3 & 62.4 & 62.9 \\
    \hline
    DeBERTa-large & 61.8 & 58.3 & 60.0 & 65.1 & \textbf{74.6} & \textbf{69.5} & 72.1 & \textbf{56.7} & \textbf{63.5} & 63.7 & 64.8 & 64.2 \\
    % DeBERTa-large + Augmentation &  \\
    \textbf{+ RL (Ours)} &  \textbf{62.1} &\textbf{63.0} & \textbf{62.5} & \textbf{65.9} & 72.4 & 69.0 & \textbf{73.0} & 53.1 & 61.4 & \textbf{64.1} & \textbf{66.2} & \textbf{65.2} \\
    \hline
\end{tabular}
\caption{Performances on the BehancePR test set. Note that \cite{lai-etal-2022-behancepr} did not report the breakdown performance for each type.}
\label{table-result-behance}
\end{table*}



%\newcommand{\tablett}[1]{\multicolumn{1}{|c|}{\textbf{#1}}}

% \begin{table}[t]
% \centering
% % \resizebox{0.4\textwidth}{!}{
% \begin{tabular}{|l|r|r|r|}
%     \hline
%     \tablett{Model} & \tablett{P} & \tablett{R}& \tablett{F1} \\
%     \hline
%     % \myrotate{6}{\textbf{ASR}}
%     RoBERTA-large &  62.0 & 61.4 & 61.7 \\
%     + Augmentation &  63.8 & 60.7 & 62.2 \\
%     + CRF & 62.2 & 63.5 & 62.9 \\
%     + CRF + Augmentation & 61.1 & 62.8 & 62.0 \\
%     % ELECTRA-base & 63.3 & 62.4 & 62.9 \\
%     \hline
%     DeBERTa-large  & 63.7 & 64.8 & 64.2 \\
%     % DeBERTa-large + Augmentation &  \\
%     \textbf{+ RL (Ours)} &  \textbf{64.1} & \textbf{66.2} & \textbf{65.2} \\
%     \hline
% \end{tabular}
% % }
% \caption{Performances on the BehancePR test set. Note that \cite{lai-etal-2022-behancepr} did not report the breakdown performance for each type.}
% \label{table-result-behance}
% \end{table}



\textbf{Settings}: In this paper, we evaluate our proposed model on two available English datasets that have been used in previous studies. 
\textbf{IWSLT} is the benchmark dataset for the PR task in English. It annotates three prominent punctuation marks: \textit{PERIOD, COMMA, QUESTION}. The IWSLT corpus contains texts derived from TED Talks, which are mainly monologues. The testing set of this corpus contains both reference text (REF), which is well-written text, and transcribed text (ASR) with manually inserted punctuation. Whereas the training set consists of only REF text. The training, development, and test sets contain approximately 2.1M, 300K, and 12K words, respectively. 
\textbf{BehancePR} is a human-annotated dataset for livestreaming videos. It features multiple speakers as well as interaction with a large number of audiences. BehancePR corpus contains only ASR text. The training/development/testing sets contain approximately 1.2M, 34K, and 44K words, respectively. 
The models are evaluated using the standard precision, recall, and F1-score (micro).


\textbf{Hyperparameters}: In this paper, each input word is tokenized using the word-piece tokenizer provided in the PLM. The representation of the first word-piece is collected as the input of the classifier head, which is a fully connected layer, to predict the punctuation.
We employed the DeBERTa-large PLM \cite{he2021deberta} as the encoder of the PR model. The hidden states of the top 8 layers are used as the representation of a token, searched from a pool of \{1,4,8,12\} layers. The GPT2-medium is used to generate the text. The seed texts for the GPT2 model contain 64 consecutive words randomly sampled from these pools. Both models are trained using the Adam optimizer with a learning rate in \{2e-5, 5e-5\}. The augmentation ratios $\alpha_1,\alpha_2,\alpha_3$ are set to 5\%, similar to \cite{alam-etal-2020-punctuation}. We concatenate $C=20$ context words to the head and tail of each chunk. Due to the high cost of evaluating the PR model on the whole development set, in each iteration, we only sample a subset $|B_j|=16$ chunks from $\mathcal{D}_{dev}$ to compute the reward.



\subsection{IWSLT corpus}

\textbf{Baselines}: We compared our model with the state-of-the-art PR models: \textbf{RoBERTa-large+Augmentation} model employs a RoBERTa-large PLM \cite{alam-etal-2020-punctuation}. The input data is augmented using three augmentation strategies: insertion, substitution, and deletion. \textbf{ELECTRA-base+Multitask} \cite{hentschel2021making} is finetuned using additional augmentation detection loss and knowledge distillation loss. \textbf{ELECTRA-large+Discriminative Self-Training} \cite{chen2021discriminative} was self-trained with a discriminator to detect human-annotated data and pseudo-machine-labeled data. \textbf{Funnel-transformer-xlarge+POSFusion} \cite{shi2021incorporating} incorporates additional part-of-speech features from an external neural-network based POS tagger.

\textbf{Results}: Table \ref{table-result-ted} compares the examined models' performance on both the REF test set and the ASR test set. The performance on the REF test set shows us the performance in case the ASR text is close to the written text, while the ASR test shows the actual performance on ASR text.

On the REF test set, ELECTRA-large is the best model among the five examined PLMs with an F1 score of 84.4\%, and it is closely followed by DeBERTa-large (0.3\%  lower). These models leave a large margin to the smaller models such as ELECTRA-base (approx. 3\% lower). Comparing the full models, our DeBERTA-large + RL model gains 1\% over the DeBERTa-large model, achieving 85.1\%. This performance is on par with the ELECTRA-large + Discriminative Self-Training model with a mere margin of 0.1\%. 


% On the REF test set, ELECTRA-large is the best model among the five examined PLMs with an F1 score of 84.4\%, and it is closely followed by DeBERTa-large (0.3\%  lower) and funnel-transformer-xlarge (0.7\% lower). This is reasonable given their similar sizes and architectures \cite{clark2020electra,he2021deberta}. These models leave a large margin to the smaller models: RoBERTa-large (approx. 2\% lower) and ELECTRA-base (approx. 3\% lower). Second, comparing the full models, our DeBERTA-large + RL model gains 1\% over the DeBERTa-large model, achieving 85.1\%. This performance is on par with the ELECTRA-large + Discriminative Self-Training model with a mere margin of 0.1\%. Third, comparing the performances for each punctuation, even though the DeBERTa-large + RL loses on QUESTION with substantially lower performance compared to ELECTRA-base and ELECTRA-large models, it yields an identical F1 score to the F1 score of the ELECTRA-large+Discriminative Self-Training model on both COMMA and PERIOD, which account for more than 90\% of the dataset. This experiment shows the effectiveness of the RL training process in providing helpful examples for training the PR model on reference data.


For ASR text, comparing the full models, our DeBERTa-large + RL model (77\% in terms of overall F1) outperforms all the other models at a large margin of 3\% to the highest competitor, RoBERTa-large + Augmentation,  with $p<0.01$. Moreover, without additional training signals or external features, the DeBERTa-large model yields similar performance to other PLMs (e.g., RoBERTa-large and funnel-transformer-xlarge). Furthermore, our proposed model outperforms the other models on all three punctuation marks with a consistently large margin ranging from 1.8\% to 5.1\%, compared to the next highest. These results clearly show the robustness of our proposed RL method to boost the performance of real-world ASR data significantly. The improvement suggests that the RL method has provided helpful training examples to help the model bridge the gap between the REF text and the ASR text in the training and testing data, respectively.


\subsection{BehancePR corpus}

\textbf{Baselines}: We compare our models with the state-of-the-art models that have been evaluated on this corpus. These models include the \textbf{RoBERTa-large} model and its variants with \textbf{Data Augmentation} and \textbf{Conditional Random Field} \cite{alam-etal-2020-punctuation}.

\textbf{Results}: First, we found that data augmentation does not improve the performance of the model trained on the BehancePR dataset. The reason is that the BehancePR dataset's training and testing data are all ASR texts, which is different from the IWSLT corpus in which the training texts are REF texts, and the testing texts are ASR texts. As such, introducing data augmentation skewed the distribution of training and testing data in the BehancePR corpus. Hence, hurting the model's performance. Table \ref{table-result-behance} presents the overall performance of our proposed models on the BehancePR corpus. The DeBERTa-large outperforms the current state-of-the-art  RoBERTa-large+CRF model (62.2\% versus 62.9\%). Furthermore, the DeBERTa-large + RL improves the F1 score from 64.2\% to 65.2\% (+1.0) (statistically significant with $p<0.01$). This again shows the effectiveness of the proposed reinforcement learning methods.



\begin{table}[t]
\centering
% \resizebox{0.83\linewidth}{!}{
\begin{tabular}{|l|l|r|r|r|}
    \hline
    \tablett{Model} & \tablett{P} & \tablett{R}& \tablett{F1} \\
    \hline
    % \myrotate{7}{\textbf{ASR}} 
    RoBERTa-large                   & 66.5 & 76.7 & 71.3  \\
    \hspace{0.2cm}+ Augmentation    & 72.0 & 76.2 & 74.0  \\
    \hspace{0.4cm} + GPT + RL             & 73.3 & 76.7 & 75.0 \\
    \hline
    DeBERTa-large                       & 66.1 & 77.6 & 71.4  \\
    \hspace{0.2cm}+ Augmentation        & 73.0 & 77.1 & 75.0  \\    
    \hspace{0.4cm} + GPT & 74.9 & 76.3 & 75.6 \\
    \hspace{0.6cm} + RL (Full model)  & \textbf{74.6} & \textbf{79.4} & \textbf{77.0} \\
    \hline
    DeBERTa-large + GPT + RL & & & \\
    + PR pretraining (1 epoch)                & 74.5 & 78.5 & 76.4   \\
    + PR pretraining (2 epochs)                & 74.2 & 78.3 & 76.2   \\
    + GPT2 pretraining (1 epoch)               & 75.7 & 77.0 & 76.3   \\
    + GPT2 pretraining (2 epochs)              & 74.5 & 77.2 & 75.8 \\
    \hline
\end{tabular}
% }


\caption{Performances on the IWSLT ASR test set.}
\label{table-ablation}
\end{table}



\subsection{Ablation study}

We perform an ablation study to examine the contribution of each component of the model on the IWSLT ASR test set as shown in Table \ref{table-ablation} (Rows 1-7). Adding the augmentation to the DeBERTa-large model boosts the performance from 71.4\% to 75.0\% (\textbf{+3.6\%}), while \textit{GPT} improves the F1 score from 75.0\% to 75.6\% (\textbf{+0.6\%}). Finally, when we add \textit{RL}, the F1 score jumps from 75.6\% to 77.0\%. These demonstrate that all the proposed components contribute to the improvement. However, data augmentation and RL contribute largely to the performance gain on the IWSLT ASR test set. Finally, to further show the effectiveness of the \textit{RL}, we add it to the RoBERTa-large+Augmentation, resulting in an increase of 1\% in the F1 score. This experiment shows that our RL method is model-agnostic that can be applied to any PR model.

The PR model and the GPT2 model could be finetuned/pre-trained with different strategies. To examine whether finetuned or pre-trained model before the reinforcement learning could further improve the performance of the model. We used the configuration of the full model with GPT2 and RL. However, for the PR model, we trained the PR alone with the same training data for 1 and 2 epochs. Similarly, we pre-trained the GPT2 model on the unsupervised text derived from the training set for the same epochs. Table \ref{table-ablation} (Rows 8-12) reports the performance of these runs. As can be seen from the performance, training/finetuning the model using only PR or GPT2 data significantly hurts the performance of the model. In particular, pretraining a single epoch on PR or GPT2  reduced the performance by 0.4\% to 0.7\%, respectively. Further training the model for one more epoch decreased the performance by 0.4\% to 0.5\%, respectively.


% \input{tab:pretraining}



\section{The geometric monodromy maps of $\mathbb{P}\mathcal{H}^{nh}(4)$ and $\mathbb{P}\mathcal{H}(3,1)$}
\section{Conclusion}

This paper focuses on generating helpful training data for the punctuation restoration task, especially for real-world ASR texts.
We devise a reinforcement learning method to use the GPT2 model to generate additional data to train the punctuation restoration model. This method allows the GPT2 model to learn from real-world ASR text to generate more helpful training examples based on gradient feedback from the PR model. Our model improves PR performance on real-world ASR tests on IWSLT and BehancePR  (+3\% and +2.3\%, respectively). In the future, we would like to extend this research with more advanced gradient feedback to improve the generated data.

\section{Acknowledgement}: This research has been supported by the Army Research Office (ARO) grant W911NF-21-1-0112, the NSF grant CNS-1747798 to the IUCRC Center for Big Learning, and the NSF grant \# 2239570. This research is also supported in part by the Office of the Director of National Intelligence (ODNI), Intelligence Advanced Research Projects Activity (IARPA), via the HIATUS Program contract 2022-22072200003. The views and conclusions contained herein are those of the authors and should not be interpreted as necessarily representing the official policies, either expressed or implied, of ODNI, IARPA, or the U.S. Government. The U.S. Government is authorized to reproduce and distribute reprints for governmental purposes notwithstanding any copyright annotation therein.

\begin{thebibliography}{99}
\footnotesize

\bibitem{AbbottDahmaniPnaive2019} \textbf{C. R. Abbott and F. Dahmani}, \emph{Property $P_{naive}$ for acylindrically hyperbolic groups}, Mathematische Zeitschrift 291 (1-2 Feb. 2019), pp. 555–568.

\bibitem{AntolinCumplidoParabolic21} \textbf{Y. Antolín  and M. Cumplido}, \emph{Parabolic subgroups acting on the additional length graph}, Algebraic \& Geometric Topology 21 (4 Aug. 2021), pp. 1791–1816.

\bibitem{arnold} \textbf{V. I. Arnol’d}, \emph{Normal forms of functions near degenerate critical points, the Weyl groups $A_k$, $D_k$, $E_k$ and Lagrangian singularities}, Funkcional. Anal. i Priložen. (no. 4, 1972), pp. 3–25.

\bibitem{arnoldbook} \textbf{V. I. Arnold, S. N. Gusein-Zade and A. N. Varchenko} Singularities of differentiable maps, volume 2: Monodromy and asymptotics of integrals, vol. 2. \emph{Springer} 64 (2012): 65.

\bibitem{BainbridgeSmillieWeissHorocycle2022} \textbf{M. Bainbridge, J. Smillie and B. Weiss}, \emph{Horocycle Dynamics: New Invariants and Eigenform Loci in the Stratum $\mathcal{H}(1,1)$}, Memoirs of the American Mathematical Society 280 (1384 Nov. 2022).

\bibitem{BaumslagResidually1963} \textbf{G. Baumslag}, \emph{Automorphism Groups of Residually Finite Groups}, Journal of the London Mathematical Society s1-38 (1 1963), pp. 117–118.

%\bibitem{bestvina1999non} \textbf{M. Bestvina}, \emph{Non-positively curved aspects of Artin groups of finite type}, Geometry \& Topology 3.1 (1999), pp. 269–302.

\bibitem{BrieskornArtin1972} \textbf{E. Brieskorn and K. Saito}, \emph{Artin-Gruppen und Coxeter-Gruppen}, Inventiones Mathematicae 17 (4 Dec. 1972), pp. 245–271.

%\bibitem{bridson2010cofinitely} \textbf{M. R. Bridson}, \emph{Cofinitely Hopfian groups, open mappings and knot complements}, Groups, Geometry, and Dynamics 4.4, (2010), pp. 693-707.

\bibitem{metricbridson} \textbf{M. R. Bridson and A. Haefliger} \emph{Metric Spaces of Non-Positive Curvature}, Vol. 319. Springer Science \& Business Media, 2013.

\bibitem{CalderonConnected2020} \textbf{A. Calderon}, \emph{Connected components of strata of Abelian differentials over Teichmüller space}, Commentarii Mathematici Helvetici 95 (2 June 2020), pp. 361–420.

\bibitem{CalderonSalterFramed2022} \textbf{A. Calderon and N. Salter}, \emph{Framed mapping class groups and the monodromy of strata of abelian differentials}, Journal of the European Mathematical Society (Nov.
2022).

\bibitem{CalvezWiestAcyArt2017} \textbf{M. Calvez and B. Wiest}, \emph{Acylindrical hyperbolicity and Artin-Tits groups of spherical type}, Geometriae Dedicata 191 (1 Dec. 2017), pp. 199–215.

\bibitem{CalvezWiestCurve2017} \textbf{M. Calvez and B. Wiest}, \emph{Curve graphs and Garside groups}, Geometriae Dedicata 188 (1 June 2017), pp. 195–213.
 
\bibitem{Cohen2002} \textbf{A. M. Cohen and D. B. Wales}, \emph{Linearity of artin groups of finite type}, Israel
Journal of Mathematics 131 (1 Dec. 2002), pp. 101–123.

\bibitem{Costantini2022} \textbf{M. Costantini, M. Möller, and J. Zachhuber}, \emph{The Chern classes and Euler characteristic of the moduli spaces of Abelian differentials}, Forum of Mathematics, Pi 10 (July 2022), e16.

\bibitem{Cuadrado2021} \textbf{P. P. Cuadrado and N. Salter}, \emph{Vanishing cycles, plane curve singularities and framed
mapping class groups}, Geometry \& Topology 25 (6 2021), pp. 3179–3228.
 
\bibitem{Deligne1972} \textbf{P. Deligne}, \emph{Les immeubles des groupes de tresses généralisés}, Inventiones Mathematicae 17 (4 Dec. 1972), pp. 273–302.

\bibitem{farb2011primer} \textbf{B. Farb and D. Margalit}, A primer on mapping class groups. Vol. 41. \emph{Princeton university press}, 2011.

\bibitem{ham} \textbf{U. Hamenstädt}, \emph{On the orbifold fundamental group of the odd component of the stratum $\mathcal{H}(2,\dots,2)$} Preprint, 2020.

\bibitem{HarerZagier} \textbf{J. Harer and D. Zagier}, \emph{The Euler characteristic of the moduli space of curves} Invent. Math. 85.3 (1986), pp. 457–485.

%\bibitem{harris2013algebraic} \textbf{J. Harris}, Algebraic geometry: a first course. Vol. 133. \emph{Springer Science \& Business Media}, 2013.

\bibitem{hartshorne} \textbf{R. Hartshorne}, Algebraic geometry. \emph{Springer-Verlag, New York-Heidelberg}, 1977.

\bibitem{hatcher2002algebraic} \textbf{A. Hatcher}, Algebraic Topology. \emph{Cambridge University Press}, 2002.

%\bibitem{hirshon_1977} \textbf{R. Hirshon}, \emph{Some properties of endomorphisms in residually finite groups}, Journal of the Australian Mathematical Society 24.1 (1977), pp. 117–120.

\bibitem{humphr} \textbf{J. E. Humphreys} Reflection groups and Coxeter groups. \textit{Cambridge university press}, 1992.

\bibitem{Kontsevich1997} \textbf{M. Kontsevich and A. Zorich}, \emph{Lyapunov exponents and Hodge theory}, The mathematical beauty of physics (Saclay, 1996). Vol. 24. Adv. Ser. Math. Phys. World Sci. Publ., River Edge, NJ, 1997, pp. 318–332

\bibitem{Kontsevich2003} \textbf{M. Kontsevich and A. Zorich}, \emph{Connected components of the moduli spaces of Abelian differentials with prescribed singularities}, Inventiones Mathematicae 153 (3 Sept. 2003), pp. 631–678.

\bibitem{lek} \textbf{H. der Lek}, \emph{The homotopy type of complex hyperplane complements}, Katholieke Universiteit te Nijmegen, Ph.D. Thesis, 1983.

%\bibitem{Lonne06} \textbf{Michael Lönne}, \emph{Fundamental Groups of Spaces of Smooth Projective Hypersurfaces}, Duke Mathematical Journal 150 (2 Aug. 2006), pp. 357–405.

\bibitem{Looijenga2014} \textbf{E. Looijenga and G. Mondello}, \emph{The fine structure of the moduli space of abelian differentials in genus 3}, Geometriae Dedicata 169 (1 Apr. 2014), pp. 109–128.

\bibitem{Looijenga93} \textbf{E. Looijenga} \emph{Cohomology of $\mathcal{M}_3$ and $\mathcal{M}_3^1$}, Contemp. Math 150 (1993) pp. 205-228.

\bibitem{Maclachlan} \textbf{C. Maclachlan}, \emph{Modulus space is simply-connected}, Proc. Amer. Math. Soc. 29 (1971), pp. 85–86.

\bibitem{Masur82} \textbf{H. Masur}, \emph{Interval exchange transformations and measured foliations}, Ann. of Math. (2) 115.1 (1982), pp. 169–200.

\bibitem{McCammond2017} \textbf{J. McCammond}, \emph{The mysterious geometry of Artin groups}, Winter Braids Lecture Notes 4 (Feb. 2017), pp. 1–30.

\bibitem{Miranda} \textbf{R. Miranda}, Algebraic curves and Riemann surfaces. Vol. 5. \emph{Graduate Studies in Mathematics}, American Mathematical Society, Providence, RI, 1995.

\bibitem{Mulholland2002} \textbf{J. T. Mulholland}, \emph{Artin groups and local indicability}, arXiv preprint, arXiv:math/0606116 (2002).

\bibitem{discriminant} \textbf{P. Orlik and L. Solomon,}  \emph{Discriminants in the invariant theory of reflection groups.} Nagoya Math. J. 109 (1988): 23-45.

\bibitem{Osin2016} \textbf{D. Osin}, \emph{Acylindrically hyperbolic groups}, Transactions of the American Mathematical Society 368 (2016), pp. 851–888.

\bibitem{Perron1996} \textbf{B. Perron and J. P. Vannier}, \emph{Groupe de monodromie géométrique des singularités simples}, Mathematische Annalen 306 (1 1996), pp. 231–245.

\bibitem{Pinkham} \textbf{H. C. Pinkham}, Deformations of algebraic varieties with $\mathbb{G}_m$ action, \emph{Société Mathématique de France}, Paris, 1974, pp. i+131.

\bibitem{Shioda1993} \textbf{T. Shioda}, \emph{Plane quartics and Mordell-Weil lattices of type E7}, Comment. Math. Univ. St. Paul. 42.1 (1993), pp. 61–79.

%\bibitem{TT} \textbf{Takahashi Tadashi}, \emph{Normal forms of smooth plane quartics and their restrictions}, ScienceAsia 42S (2016), pp. 26–33.

\bibitem{Thurston} \textbf{W. P. Thurston}, \emph{On the geometry and dynamics of diffeomorphisms of surfaces}, Collected works of William P. Thurston with commentary. Vol. I. Foliations, surfaces and differential geometry. Reprint. Amer. Math. Soc., Providence, RI, 2022, pp. 495–509.

\bibitem{Veech} \textbf{W. A. Veech}, \emph{Interval exchange transformations}, J. Analyse Math. 33 (1978), pp. 222–272.

\bibitem{Wajnryb1999} \textbf{B. Wajnryb}, \emph{Artin groups and geometric monodromy}, Inventiones Mathematicae 138 (3 Dec. 1999), pp. 563–571.

\bibitem{Wright2015} \textbf{A. Wright}, \emph{Translation surfaces and their orbit closures: An introduction for a broad audience}, EMS Surveys in Mathematical Sciences 2 (1 2015), pp. 63–108.

%\bibitem{Zariski1937} \textbf{Oscar Zariski}, \emph{A Theorem on the Poincare Group of an Algebraic Hypersurface}, TheAnnals of Mathematics 38 (1 Jan. 1937), p. 131.

%\bibitem{ZorichFlat} \textbf{Anton Zorich}, \emph{Flat surfaces}, Frontiers in number theory, physics, and geometry. I. Springer, Berlin, 2006, pp. 437–583.

\bibitem{zykoski2022isodelaunay} \textbf{B. Zykoski}, \emph{The l-isodelaunay decomposition of strata of abelian differentials}, arXiv preprint, arXiv:2206.04143 (2022).


\end{thebibliography}

\end{document}
