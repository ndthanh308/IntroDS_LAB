We now have all the tools to produce in-place polynomial algorithms.
We start, in~\cref{alg:doublebilin}, with a version
of~\cref{alg:bilin} that regroups the intermediate computations into
consecutive blocks.

\begin{algorithm}[htbp]
  \caption{In place bilinear $2$ by $2$ formula}\label{alg:doublebilin}
  \begin{algorithmic}[1]\small
    \REQUIRE $\vec{a}\in\F^m$, $\vec{b}\in\F^n$, $\vec{c}\in\F^s$;
    $\mat{\alpha}\in\F^{t{\times}m}$, $\mat{\beta}\in\F^{t{\times}n}$,
    $\mat{\mu}\in\F^{s{\times}(2t)}=\begin{smatrix}M_1&\ldots&M_t\end{smatrix}$,
      with no zero-rows in $\alpha$, $\beta$, $\mu$,
      s.t. $(a_i\cdot{b_j})$ fits two result variables $c_k$, $c_l$
      and s.t. $M_i\in\F^{s{\times}2}$ is of full-rank $2$ for $i=1..t$.
    \READONLY$\mat{\alpha},\mat{\beta},\mat{\mu}$.
    \ENSURE $\vec{c}\pe\mat{\mu}\vec{m}$, for
    $\vec{m}=(\mat{\alpha}\vec{a})\odot(\mat{\beta}\vec{b})$
    \FOR{$\ell=1$ \To $t$}
    \STATE Let $i$ s.t. $\alpha_{\ell,i}\neq{0}$;
    $a_i\fe\alpha_{\ell,i}$;
    \ForDoEnd[lin:doublealpha]{$\lambda=1$ \To $m$, $\lambda\neq{i}$,
      $\alpha_{\ell,\lambda}\neq{0}$
    }{$a_i\pe\alpha_{\ell,\lambda}a_\lambda$}
    \STATE Let $j$ s.t. $\beta_{\ell,j}\neq{0}$;
    $b_j\fe\beta_{\ell,j}$;
    \ForDoEnd[lin:doublebeta]{$\lambda=1$ \To $n$, $\lambda\neq{j}$,
      $\beta_{\ell,\lambda}\neq{0}$}{$b_j\pe\beta_{\ell,\lambda}b_\lambda$}
    \STATE Let $k,f$
    s.t. $M=\begin{smatrix}\mu_{k,2\ell}&\mu_{k,2\ell+1}\\\mu_{f,2\ell}&\mu_{f,2\ell+1}\end{smatrix}$
    is invertible;
    \STATE\label{lin:invmul}$\begin{smatrix} c_k\\c_f \end{smatrix} \leftarrow M^{-1}
    \begin{smatrix} c_k\\c_f \end{smatrix}$
    \hfill\COMMENT{Via~\cref{eq:twobytwomul,rq:zerotopleft}}
    \ForDoEnd[lin:divsube]{$\lambda=1$ \To $s$,
      $\lambda\not\in\{f,k\}$,
      $\mu_{\lambda,2\ell}\neq{0}$}{$c_\lambda\me\mu_{\lambda,2\ell}{c_k}$}
    \ForDoEnd[lin:divsubo]{$\lambda=1$ \To $s$,
      $\lambda\not\in\{f,k\}$,
      $\mu_{\lambda,2\ell+1}\neq{0}$}{$c_\lambda\me\mu_{\lambda,2\ell+1}{c_f}$}
    \STATE\label{lin:doubleproduct}$\begin{smatrix} c_k\\c_f
    \end{smatrix}\pe{a_i\cdot{b_j}}$\hfill\COMMENT{This is the
      accumulation of the product $\begin{smatrix} m_k\\m_f\end{smatrix}$}
    \ForDoEnd{$\lambda=1$ \To $s$, $\lambda\not\in\{f,k\}$,
      $\mu_{\lambda,2\ell+1}\neq{0}$}{$c_\lambda\pe\mu_{\lambda,2\ell+1}{c_f}$}
    %
    \ForDoEnd{$\lambda=1$ \To $s$,$\lambda\not\in\{f,k\}$,
      $\mu_{\lambda,2\ell}\neq{0}$}{$c_\lambda\pe\mu_{\lambda,2\ell}{c_k}$}
    %
    \STATE$\begin{smatrix} c_k\\c_f \end{smatrix} \leftarrow M\begin{smatrix} c_k\\c_f \end{smatrix}$
    \hfill\COMMENT{Via~\cref{eq:twobytwomul,rq:zerotopleft}, undo~\ref{lin:invmul}}
    \ForDoEnd{$\lambda=1$ \To $n$, $\lambda\neq{j}$,
      $\beta_{\ell,\lambda}\neq{0}$}{$b_j\me\beta_{\ell,\lambda}b_\lambda$};~$b_j\de\beta_{\ell,j}$;
    %
    \ForDoEnd{$\lambda=1$ \To $m$, $\lambda\neq{i}$,
      $\alpha_{\ell,\lambda}\neq{0}$}{$a_i\me\alpha_{\ell,\lambda}a_\lambda$};~$a_i\de\alpha_{\ell,i}$;
    %
    \ENDFOR
    \RETURN $\vec{c}$.
  \end{algorithmic}
\end{algorithm}
%
%
\begin{theorem}\label{thm:doublebilin}
\Cref{alg:doublebilin} is correct, in-place, and requires
$t$ \MUL-2D,
$2(\#\alpha+\#\beta+\#\mu-t)$ \ADD and
$2(\sharp\alpha+\sharp\beta+\sharp\mu+2t)$ \SCA operations.
\end{theorem}
\begin{proof}
  Thanks to~\cref{eq:twobytwo,eq:twobytwomul,rq:zerotopleft},
  correctness is similar to that of~\cref{alg:bilin}
  in~\cref{thm:bilin}.
Then, \cref{eq:twobytwomul} requires $4$ \SCA
and $2$ \ADD operations and is called $2t$ times.
The rest is similar to~\cref{alg:bilin} and amounts to
$2t+2(\#\alpha-t+\#\beta-t+\#\mu-2t)+(2t)2$ \ADD and
$2(\sharp\alpha+\sharp\beta+\sharp\mu-2t)+(2t)4$ \SCA operations.
\end{proof}

There remains to use a double expansion of the output
$\mu\in\F^{s{\times}t}$ to simulate the double size of the
intermediate products (\MUL-2D), producing
$\mu^{(2)}\in\F^{s{\times}(2t)}$, as in~\cref{eq:bilinkarasplit}, that
is used as an input in~\cref{alg:doublebilin}.
This double expansion matrix is obtained by the following duplication:
$\mu^{(2)}(i,2j)=\mu(i,j)$ and $\mu^{(2)}(i+1,2j+1)=\mu(i,j)$ for
$i=1..s$ and $j=1..t$. We prove, in \cref{lem:fullrank},
that in fact any such double expansion of a representative matrix
is suitable for the in-place computation of~\cref{alg:doublebilin}.

%
%
%
%
%
%
%
%
%
%
%
%
%
%
%

%
%
%
%
%
%
%
%
%

\begin{lemma}\label{lem:fullrank}
  If $\mu$ does not contain any zero column, then each pair of columns
  of $\mu^{(2)}$, resulting from the expansion of a single column in $\mu$,
  contains an invertible lower triangular $2{\times}2$ submatrix.
\end{lemma}
\begin{proof}
Consider the top most non-zero element of a column.
It is expanded as a $2{\times}2$ identity matrix whose second row is
merged with the first row of the next identity matrix: in
picture, $\begin{smatrix} a \\ b\end{smatrix}$ is expanded to
$\begin{smatrix} a &0 \\ b & a\\ * & b\end{smatrix}$.
\end{proof}
