\section{Experimental details}
\label{sec:expdetails}

Here we provide more details about the four different conducted experiments: 2D composition, guided diffusion for CIFAR-100, guided diffusion for ImageNet, and Image tapestry. 

\subsection{2D composition}
\label{sec:expdetails:2d_comp}

The composed distribution is defined by a product of two components, a Gaussian mixture and a uniform distribution with non-zero values on
\begin{align} \label{eq:bar}
    \square = \{ x \in \mathbb{R}^2: -s_i \leq x_i \leq s_i, i = 1,2 \},
\end{align}
where $s_1$ and $s_2$ are equal to $0.2$ and $1.0$, respectively. The eight modes of the Gaussian mixture are evenly distributed on a circle with a radius of 0.5 at the angles $\frac{\pi}{4}i$ for $i = 0, \ldots, 7$, respectively. The covariance matrix at each mode is $0.03^2 \cdot I$, where $I$ is the identity matrix.

% log.-likelihood or log-likelihood?
The metric log-likelihood is ill-defined as we may generate samples where the true distribution has no support (due to the uniform distribution). We address this problem by expanding the definition set of the uniform distribution and redistributing one percent of the probability mass into this extended region. The whole set is defined as \eqref{eq:bar} except $s_1=s_2=1.1$. Note that 99 percent probability mass remains inside the original definition set $\square$.

We use the same neural network architectures as the base for both the score and energy models. It is a residual network consisting of a linear layer (dim $2 \rightarrow 128$) followed by four blocks, and concluding with a linear layer (dim $128 \rightarrow 2$).
Within each block, the input $x$ passes through a normalization layer, a SiLU activation, and a linear layer (dim $128 \rightarrow 256$). Subsequently, it is added with an embedded $t$ (dim 32) that has undergone a linear layer transformation (dim $32 \rightarrow 256$). The resulting sum passes through a SiLU activation and is further processed by a linear layer (dim $256 \rightarrow 256$). After that, another SiLU activation is applied, followed by a final linear layer (dim $256 \rightarrow 128$). The output of this linear layer is then added to the original input $x$ within the block. The embedding of $t$ is also learnable. 

The parameter $\beta_t$ follows the cosine schedule proposed in \cite{nichol2021improved}. For (U-)HMC, the damping coefficient is set to $0.5$, the mass diagonal matrix has all diagonal elements equal to $1$, and the stepsize for each $t$ is $0.03$. For (U-)LA, the stepsize for each $t$ is set to $0.001$.

%, and the hyperparameter $\betaTilde_t$ is calculated as $\beta_t \frac{1 - \bar{\alpha}_{t-1}}{1 - \bar{\alpha}_t}$, as mentioned in \cite{ho2020denoising}, where $\bar{\alpha}_t = \prod_{s=1}^t \alpha_s$ and $\alpha_t = 1 - \beta_t$. Additionally, $\sigma_t$, in \eqref{eq:score_noise_connection}, is set to $\sqrt{1 - \bar{\alpha}_t}$. 
% The parameters for the different MCMC methods can be found in \cref{tab:mcmcparam2d}.
%\begin{table}
%    \centering
%    \caption{Parameter values for the different MCMC methods used in the 2D composition experiment.}
%  \begin{tabular}{ | c | c | c |}
 %   \hline
 %   MCMC & Parameter & Value \\ \hline
 %   \multirow{5}{*}{(U-)HMC}  & \# steps per $t$ & 10 \\ \cline{2-3} 
  %  & \# leapfrogs & 3 \\ \cline{2-3}
  %  & Step size $\forall t$ & 0.03 \\ \cline{2-3}
  %  & Damping-coefficient & 0.5 \\ \cline{2-3}
  %   & Mass-diagonal-matrix & 1 \\ \hline
  %  \multirow{2}{*}{(U-)LA } & \# steps per $t$ & 10 \\ \cline{2-3} & Step size $\forall t$ & 0.001 \\ \hline
  %\end{tabular}
  %\label{tab:mcmcparam2d}
%\end{table}

\subsection{Guided diffusion for CIFAR-100}
\label{sec:expdetails:cifar100_guid}

 The parameter $\beta_t$ has a linear schedule as originally proposed in \cite{ho2020denoising}. For (U)-HMC is the damping coefficient equal to 0.9 and the diagonal elements in the massmatrix are equal to $\beta_t$ for each $t$. The values of the stepsize parameters $a$ and $b$ were determined through a simple parameter search for the different MCMC methods and they can be found in \cref{tab:mcmc_stepsize_cifar100}. This was done for both the score and energy parameterizations, where the stepsize is defined as $\langStep_t = a \beta_t^b$. 

\begin{table}[h!]
    \centering
     \caption{The values of the stepsize parameters $a$ and $b$ obtained from a random parameter search for the different MCMC methods for both score and energy parameterization in the CIFAR-100 experiment, where the stepsize is defined as $\langStep_t = a \beta_t^b$.}
    \label{tab:mcmc_stepsize_cifar100}
    \begin{tabular}{|c|c|>{\centering\arraybackslash}m{1.8cm}|>{\centering\arraybackslash}m{1.8cm}|}
        \hline
        \multirow{2}{*}{Model} & \multirow{2}{*}{MCMC} & \multicolumn{2}{c|}{Stepsize Parameters} \\ \cline{3-4}
         &  & a & b \\ \hline
        \multirow{4}{*}{Energy} & U-LA & 9.22 & 1.40 \\ \cline{2-4}
         & LA & 9.84 & 0.83 \\ \cline{2-4}
         & U-HMC & 0.26 & 1.53 \\ \cline{2-4}
         & HMC & 9.33 & 1.48 \\ \hline
        \multirow{4}{*}{Score} & U-LA & 1.96 & 1.04 \\ \cline{2-4}
         & LA & 9.84 & 0.83 \\ \cline{2-4}
         & U-HMC & 0.26 & 1.53 \\ \cline{2-4}
         & HMC & 4.03 & 1.34 \\ \hline
    \end{tabular}
\end{table}


\subsection{Guided diffusion for ImageNet}
\label{sec:expdetails:imagenet_guid}

Again, the parameter $\beta_t$ follows a linear schedule. The hyperparameters for the HMC include a damping coefficient set at $0.9$, with the diagonal elements of the mass matrix being equal to $\beta_t$ for each $t$. The stepsize parameters for HMC, obtained from a simple parameter search, are $a=1.87$ and $b=1.51$.


\subsection{Image tapestry}
\label{sec:expdetails:image_tapestry}

A cosine schedule is used for the parameter $\beta_t$. The stepsize parameters in this case is simply $a=1$ and $b = 1$, i.e., $\langStep_t = \beta_t$.