\section{Discussion}
The choice between score and energy parameterisations remains intriguing.
We have provided additional empirical evidence that the score parameterisation performs better in the standard reverse process, but that accurate energy estimates can improve more complex sampling methods. We have also demonstrated that most of the gains that may be achieved by using an energy parameterisation can be obtained directly in a score-based model by approximating pseudo-energy differences (probability ratios) using a curve integral of the score function. That is, even though we do not explicitly train an energy-based model, we can perform MH-like correction steps in different MCMC samplers and obtain comparable performance gains.  

Particularly interesting is that our trick of using a curve only containing the score evaluations from the standard HMC method seems to perform just as well as the straight line.
This means we achieve the correction at practically no extra cost. However, in higher dimensions, additional intermediate steps may be needed, increasing the computational burden. This issue might be mitigated with more efficient integral approximation techniques. Furthermore, it is addressed in latent diffusion, whose primary purpose is to reduce dimensionality. Note, the energy parameterisation always requires an extra differentiation to obtain the score.

The score parameterization is not a proper score function in general, because the vector field it produces is not guaranteed to be conservative. Nevertheless, sampling with the estimated score still performs well in practice.
Likewise, using it for the MH-like correction step proposed in our work, seems to achieve significant improvement to the reverse process.
However, the invalid assumption of a conservative vector field might explain the small performance edge of the energy parameterisation in the CIFAR-100 experiment,
and seeking improved methods for estimating the pseudo-energy difference from a score model is a highly relevant future topic.