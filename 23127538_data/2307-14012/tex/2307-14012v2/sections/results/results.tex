\section{Results}

We investigate the impact of our MH-like correction step on different forms of model composition, aiming to draw samples from a composed distribution using diffusion models trained solely on the individual components' distributions. That is, the model components are fixed, and we sample from the composed distribution, incorporating MCMC steps at each diffusion step. Our code is available at our GitHub repo\footnote{https://github.com/FraunhoferChalmersCentre/mcmc\_corr\_score\_diffusion}.

In the first two experiments, we train the diffusion models ourselves with both energy and score parameterization. The experimental setups for these cases are similar: the score parameterization is defined by a noise prediction model $\epsP(\x_t, t)$. For the energy parameterization, we follow the setup in \cite{du2023reduce} and estimate the energy function as
\begin{equation}
\label{eq:ebm:param}
\energy(\x_t, t) = \twoNorm{x_t - \score(\x_t, t)}^2,
\end{equation}
where $\score$ is a vector-valued output of a neural network with the same dimension as $\x_t$. The score-based $\epsilon_\theta$ has an identical network architecture to $s_\theta$. Both score and energy-parameterized models are trained with the standard diffusion loss \cite{ho2020denoising}, with the score function of the energy-based model obtained through explicit differentiation.

In the third and fourth experiments, we utilize larger pre-trained score-based diffusion models to test our proposed method. However, we do not have energy-based models for comparison in these cases.

We evaluate both unadjusted and MH-corrected versions of LA and HMC. We compare the sampling performance of the score and energy-parameterized models, when available, to the standard reverse process, which serves as the baseline.

For the MH-like correction, we evaluate two types of curves along which the score is integrated: \textit{line} and \textit{curve}. \textit{Line} corresponds to a straight line from $\x^\tau$ to $\xCand$ (see \cref{eq:fake_energy}), while \textit{curve} corresponds to integration along a curve that incorporates the internal points of an MCMC method (in this case, the leapfrog steps of HMC). Furthermore, the number of points used for the trapezoidal rule’s mesh for the line integral per MCMC step is treated as a hyperparameter. However, since several points are evaluated regardless in the MCMC method, e.g., $\x^\tau$ and $\xCand$ , we let the hyperparameter describe the number of points in addition to those we get for free. These additional points are evenly distributed along the curve.

\subsection{2D Composition} \label{section:2dcomp}
To investigate the effectiveness of our MH-like correction in a controlled yet expressive setting, we replicate the 2D composition experiment introduced by~\citet{du2023reduce}, using their publicly available codebase\footnote{\url{https://github.com/yilundu/reduce_reuse_recycle}} as a foundation. Our experimental setup mirrors theirs unless otherwise specified.

A 2D density pair is composed via multiplication into a complex distribution, as in (\ref{eq:comp:prod:ebm}): a Gaussian mixture with 8 modes in a circle and a uniform distribution covering two of the modes. For a visual representation
of the two individual distributions and their resulting product distribution together with samples from the reverse diffusion and HMC corrected samples, see Figure \ref{fig:exp:2d_toy}. The baseline reverse diffusion process uses $T=100$ steps. In the MCMC variants, following \cite{du2023reduce}, we omit the optional reverse step for a fair comparison. MCMC sampling runs for $\mcSteps = 10$ at each $t$, with (U-)HMC using 3 leapfrog steps per MCMC step.
% Figure environment removed
%
We evaluate performance using three metrics. The first is negative log-likelihood (NLL), which assesses the likelihood of generated samples under the true data distribution. To address potential samples outside the true distribution’s support, we extend it by adding a small uniform probability.
The second metric is a Gaussian mixture model (GMM), where we fit bi-modal GMMs to samples from both the true and model distributions and compute the Frobenius norm of the variance mean difference.
Finally, we use the Wasserstein-2 distance ($W_2$) to measure the discrepancy between the data and model distributions by computing the optimal assignment between sampled sets~\citep{villani2009opttransp}.

We present quantitative results for the 2D composition in Table~\ref{tab:exp:2d_combined}\hyperlink{tab:2d_perf}{(a)}, averaged over 10 independent trials. In each trial, we train the diffusion models from scratch and sample 2000 points using different MCMC methods. The results show that the corrected sampling methods outperform the unadjusted ones. HMC variants yield better results than Langevin, while the reverse process performs worse. Score and energy parameterizations exhibit similar NLL and GMM performance within their respective methods. However, with HMC, the score parameterization significantly outperforms the energy parameterization in $W_2$. Performance also saturates with as few as three points in the trapezoidal rule.

Additionally, we measured memory usage and runtime during this experiment, see Table~\ref{tab:exp:2d_combined}\hyperlink{tab:2d_runtime}{(b)}. Score-based parameterization was more than twice as memory-efficient as energy-based parameterization and, with the exception of LA with 8 extra trapezoidal evaluations, faster for the corresponding MCMC methods. Notably, HMC curve was nearly three times faster. While our approach requires more model evaluations, this does not necessarily make it slower or more memory-intensive than using an energy-based model. However, these results are implementation-dependent, and further investigation is needed to confirm whether these trends generalize to other setups. 
\begin{table}[ht]
\centering
\caption{
Quantitative results for different samplers in the 2D composition experiment. 
(a) shows performance metrics (NLL, GMM, and $W_2$) based on 10 independent trials, with lower values indicating better performance. 
(b) reports average runtime (in seconds) and peak memory consumption (in MiB). 
For the score parameterization, we include variants with different numbers of additional points in the trapezoidal rule (e.g., 1L, 3L, 8L) and different integration paths (“L” for a straight line and “C” for the HMC trajectory).
}
\label{tab:exp:2d_combined}
\vspace{0.5em}

\begin{minipage}[t]{0.62\linewidth}
\centering
\textbf{(a) Performance metrics}\hypertarget{tab:2d_perf}{}\\[0.5ex]
\begin{tabular}{|c|c|c|c|c|}
\hline
 & Sampler & NLL\textdownarrow & GMM\textdownarrow & $W_2$\textdownarrow \\
\hline
\multirow{5}{*}{\rotatebox[origin=c]{90}{Energy}} 
& Reverse   & $8.22 \pm 0.21$ & $27.01 \pm 1.34$ & $5.81 \pm 0.19$ \\
& U-LA      & $7.52 \pm 0.22$ & $14.61 \pm 1.35$ & $4.19 \pm 0.45$ \\
& LA        & $6.50 \pm 0.30$ & $14.66 \pm 1.46$ & $4.24 \pm 0.55$ \\
& U-HMC     & $5.72 \pm 0.18$ & $6.53 \pm 0.91$  & $4.19 \pm 1.25$ \\
& HMC       & $\pmb{4.09 \pm 0.14}$ & $\pmb{3.33 \pm 0.65}$ & $\pmb{4.12 \pm 1.44}$ \\
\hline
\multirow{10}{*}{\rotatebox[origin=c]{90}{Score}} 
& Reverse   & $8.15 \pm 0.24$ & $26.88 \pm 1.20$ & $5.80 \pm 0.20$ \\
& U-LA      & $7.57 \pm 0.12$ & $14.99 \pm 0.62$ & $4.44 \pm 0.63$ \\
& LA-1L     & $6.45 \pm 0.20$ & $14.28 \pm 1.07$ & $4.03 \pm 0.52$ \\
& LA-3L     & $6.61 \pm 0.17$ & $15.19 \pm 0.92$ & $4.22 \pm 0.46$ \\
& LA-8L     & $6.53 \pm 0.17$ & $14.75 \pm 0.91$ & $4.20 \pm 0.51$ \\
& U-HMC     & $5.77 \pm 0.12$ & $6.90 \pm 0.71$  & $3.39 \pm 0.77$ \\
& HMC-1L    & $4.29 \pm 0.13$ & $3.72 \pm 0.61$  & $2.92 \pm 1.02$ \\
& HMC-3L    & $\pmb{4.07 \pm 0.13}$ & $3.08 \pm 0.69$ & $\pmb{2.68 \pm 1.20}$ \\
& HMC-8L    & $\pmb{4.07 \pm 0.14}$ & $3.17 \pm 0.56$ & $2.87 \pm 0.89$ \\
& HMC-C     & $\pmb{4.07 \pm 0.12}$ & $\pmb{3.06 \pm 0.54}$ & $2.94 \pm 0.90$ \\
\hline
\end{tabular}
\end{minipage}
\hfill
\begin{minipage}[t]{0.35\linewidth}
\centering
\textbf{(b) Runtime and memory usage}\hypertarget{tab:2d_runtime}{}\\[0.5ex]
\begin{tabular}{|c|c|c|c|}
\hline
 & Sampler & Time & Memory \\
\hline
\multirow{5}{*}{\rotatebox[origin=c]{90}{Energy}} 
& Reverse   & $2.1$ & $5252$ \\
& U-LA      & $2.7$ & $5252$ \\
& LA        & $10.4$ & $5252$ \\
& U-HMC     & $19.3$ & $5254$ \\
& HMC       & $22.8$ & $5256$ \\
\hline
\multirow{10}{*}{\rotatebox[origin=c]{90}{Score}} 
& Reverse   & $1.6$ & $2178$ \\
& U-LA      & $2.2$ & $2180$ \\
& LA-1L     & $6.1$ & $2180$ \\
& LA-3L     & $8.8$ & $2180$ \\
& LA-8L     & $13.6$ & $2180$ \\
& U-HMC     & $9.5$ & $2180$ \\
& HMC-1L    & $8.8$ & $2180$ \\
& HMC-3L    & $11.6$ & $2180$ \\
& HMC-8L    & $16.4$ & $2180$ \\
& HMC-C     & $7.1$ & $2180$ \\
\hline
\end{tabular}
\end{minipage}
\end{table}

\subsection{Guided diffusion for CIFAR-100}
We evaluate our proposed sampling methods for guided diffusion sampling on the CIFAR-100 image dataset \cite{krizhevsky2009cifar}.
The sampling process is based on a score function defined in \cref{eq:guidance:score}, composed of the score functions of a marginal distribution $\gradX \log \pEbm(\x_t, t)$ and a likelihood of the class $y$, $\gradX \log p(y \mid \x_t, t)$, at each diffusion step $t = 1, \dots, T$.

The score of the marginal distribution $\gradX \log \pEbm(\x_t, t)$ is estimated with an unconditional diffusion model.
The score parameterised model is $\epsP: \reals^{D_x \times 1} \to \reals^{D_x}$, where $D_x = 3 \cdot 32^2$, takes a noisy image and $t$ as input and outputs a noise prediction of the same shape as $\x_t$.
It is parameterised by a neural network with a UNet architecture.
We use the same architecture and training settings as \cite{ho2020denoising} used for the CIFAR-10 image dataset \cite{krizhevsky2009cifar}.
The energy parameterised model uses the same architecture and is parameterised as in \cref{eq:ebm:param}.

For the guidance model, we use classifier-full guidance, that is, we train a classifier model to predict the class label of an image at all diffusion steps $\pCfull(y \mid \x_t, t)$, where the classifier parameters $\pF$ are independent of $\param$.
The classifier model is parameterised by a neural network with the first half of the UNet structure used for the unconditional diffusion model extended with a dense layer. To train the classifier, we use labelled pairs $(\x_t, y)$, where $y$ is the class and $\x_t \sim q(\x_t \mid \x_0)$ is a sample from the forward diffusion process in \cref{eq:diff:forward_proc_x0}, conditioned on a sample $\x_0$ from the data distribution.

The sampling is based on the standard reverse process with $T=1000$.
The MCMC samplers add $\mcSteps = 2$ or $6$ extra MCMC steps at each diffusion step $t$ for (U-)HMC and (U-)LA, respectively, whereas (U-)HMC uses 3 leapfrog steps per MCMC step.
All sampling methods use the same guidance scale $\lambda = 20.0$.

For this experiment, we need to use more points in the trapezoidal rule's mesh than in the 2D experiment. Based on the insights from that experiment, for HMC we integrate only along the curve obtained from the leapfrog steps. However, we also evaluate a point in the middle of each leapfrog step to obtain a better energy estimation, resulting in three extra model evaluations per HMC step. For LA, we use seven evaluation points along the line, which means eight extra evaluations per step.

Recognising the impact of the step length on MCMC methods in general, we parameterise the step length as a function of the beta-schedule $\langStep_t = a \beta_t^b$. We conducted a simple parameter search for parameters $a$ and $b$, to determine  suitable step length for each MCMC variant.
Further details are provided in \cref{sec:expdetails:cifar100_guid}.

We sample 50k images with each sampling method, for both the energy and score parameterisations and compute the FID score \cite{heusel2017gans} (based on the validation set) and the average accuracy\footnote{We classify an image as correctly generated if the classifier has predicted the specified class and is 50\% certain or greater.} of a separate classifier model, trained only on noise-free pairs $(\x_0, y)$ from the CIFAR-100 dataset.
The model architecture of the classifier is VGG-13-BN from \cite{simonyan2014very}.
The results are shown in \cref{tab:exp:cifar100}.
%
\begin{table}
\centering
\caption{Average accuracy and FID score for classifier-full guidance on CIFAR-100.
The metrics are based on 50k generated samples for each sampling method with both energy and score parameterisations.
We use the guidance scale $\lambda = 20.0$, the (U-)LA methods use $\mcSteps = 6$ MCMC steps, and (U-)HMC use $\mcSteps = 2$ with $3$ leapfrog steps for the variants, the step length at diffusion step is $a \beta_t^b$.
The accuracy is based on a separate model, which has only been trained on noise-free samples, i.e., it predicts $p(y \mid \x_0)$.
Both parameterizations benefit from the added MCMC steps, especially the MH-corrected versions.
The energy parameterisation appears to perform worse in the standard reverse process but sees a larger improvement from the extra MCMC steps.
}
\begin{tabular}{|c|c|c|c|}
\hline
Model & Sampler & Accuracy [\%]\textuparrow & FID\textdownarrow \\ \hline
\multirow{5}{*}{Energy} 
& Reverse   & 72.6 & 33.4 \\
& U-LA      & $\pmb{87.3}$ & 24.6 \\
& LA        & 80.0 & 12.7 \\
& U-HMC     & 87.2 & 25.4 \\
& HMC       & 84.9 & $\pmb{12.4}$ \\
\hline
\multirow{5}{*}{Score}
& Reverse   & $74.2$ & $31.8$ \\
& U-LA      & $\pmb{82.9}$ & $25.9$\\
& LA-8-line        & $75.2$ & $15.5$ \\
& U-HMC     & $79.0$ & $28.6$ \\
& HMC-3-curve       & $75.8$ & $\pmb{13.3}$ \\
\hline
\end{tabular}
\label{tab:exp:cifar100}
\end{table}
From the table, we note a general trend of improvement of the baseline reverse process when additional MCMC steps are added.
In particular, the MH-corrected samplers LA and HMC show significant improvement in the FID score, which is arguably the more important metric for image generation. 

Comparing the score and energy parameterisations, the respective performances have largely shared characteristics.
Interestingly, the basic reverse process favours the score parameterisation supporting the claim that this less restricted parameterisation better models the score function of the marginal distribution.
However, the energy parameterisation sees larger improvements from the added MCMC steps.
This indicates, perhaps, that the direct energy estimation provides a better correction step compared to our method of approximating the pseudo-energy difference from $\epsP$, though it should be noted that the same difference is also observed in the unadjusted samplers U-LA and U-HMC.
Despite the performance edge of the energy parameterisation, our proposed MH-corrected sampling methods can provide essentially the same improvement, without having to train an energy parameterised diffusion model.
\subsection{Guided diffusion for ImageNet}
We extend the evaluation of our proposed method for guided diffusion sampling on the ImageNet dataset \cite{deng2009imagenet}. We utilize pre-trained score models from \cite{dhariwal2021diffbeatgan}, available on the OpenAI GitHub repository\footnote{https://github.com/openai/guided-diffusion}.

As in the CIFAR-100 experiments, the score of the marginal distribution $\gradX \log \pEbm(\x_t, t)$ is estimated using an unconditional diffusion model. For the ImageNet dataset, the score parameterized model, $\epsilon_\theta : \mathbb{R}^{D_x} \times 1 \rightarrow \mathbb{R}^{D_x}$, where $D_x = 3 \cdot 256^2$, is also parameterized by a neural network with a UNet architecture. Again, we use classifier-full guidance, and the classifier model is parameterised by a neural network with the structure of the corresponding encoder part of the UNet.

Given the high computational demands due to both the large models and high-dimensional input, we have chosen to focus solely on evaluating HMC (with our MH-like correction) and compare it to the baseline, which is the standard reverse process with $T=1000$. The HMC sampler adds $L=2$ extra MCMC steps at each diffusion step $t$, where each MCMC step constitutes three leapfrog steps. Both sampling methods use the same
guidance scale $\lambda = 20.0$. Again, we incorporate the points given by the leapfrog steps, but due to the high dimension, two additional points between each leapfrog step are needed for the line integration. We conduct the same type of parameter search of the step length for the HMC method as in the CIFAR-100 experiment. Further details are provided in \cref{sec:expdetails:imagenet_guid}.

We sample 50k images with both sampling methods and compute the FID score \cite{heusel2017gans} (based on the validation set), the average accuracy\footnote{Again, an image is considered correctly generated if the classifier's prediction for the specified class is at least 50\%.}, and the top-5 average accuracy, i.e., correct prediction if the specified label is within the top-5 predictions. Once again, we utilize a separate classifier model, in this case, a RegNetX-8.0GF \cite{radosavovic2020designing}, for evaluation. This time, however, the model is pre-trained. The results are shown in \cref{tab:exp:imagenet}.

\begin{table}
\centering
\caption{Average accuracy, top-5 accuracy, and FID score for classifier-full guidance on ImageNet.
The metrics are based on 50k generated samples for both sampling methods with score parameterisations.
We use the guidance scale $\lambda = 20.0$,  $\mcSteps = 2$ with $3$ leapfrog steps for HMC, the step length at diffusion step is $a \beta_t^b$.
The accuracy is based on an independent classifier model.
}
\begin{tabular}{|c|c|c|c|c|}
\hline
Model & Sampler & Accuracy [\%]\textuparrow & Top-5 Accuracy [\%]\textuparrow & FID\textdownarrow \\ \hline
\multirow{2}{*}{Score}
& Reverse   & $\pmb{50.0}$ & $83.9$ & $14.5$ \\
& HMC-6-curve       & $49.9$ & $\pmb{85.1}$ & $\pmb{11.6}$ \\
\hline
\end{tabular}
\label{tab:exp:imagenet}
\end{table}
The reverse process and HMC perform very similarly in average accuracy, but our method shows a slight improvement in top-5 average accuracy. However, augmenting with some extra MCMC steps with MH-like correction significantly improves the FID score.
\subsection{Image tapestry}
We conduct a so-called image tapestry experiment, similar to the one in \cite{du2023reduce} and based on their code\footnote{https://github.com/yilundu/reduce\_reuse\_recycle}, as our final experiment. This experiment involves not only the composition of guidance---in this case, classifier-free guidance---but also the composition of combining multiple overlapping text-to-image models. This approach allows us to construct an image with specified content at different spatial locations. Here, we use a pre-trained DeepFloyd-IF\footnote{Available at https://huggingface.co/DeepFloyd/IF-I-XL-v1.0} model. For each diffusion step ($T=100$), 15 extra LA steps were added, with three additional evaluation points for line integration for each step. The guidance scale $\lambda = 20.0$. For more details, see the \cref{sec:expdetails:image_tapestry}. In \cref{fig:tapestry}, we can see a generated tapestry image, and in \cref{fig:tapestry_content}, we can see the specified content at the corresponding spatial locations. There are, in total, nine overlapping content boxes: four are positioned in each corner with different content, while the remaining five are arranged to create a unified image using the same content prompt.

% Figure environment removed