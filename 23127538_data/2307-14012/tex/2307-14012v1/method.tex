\section{MCMC-correction with the score parameterisation}
We propose to combine the properties of the energy parameterisation with the performance and greater availability of the score parameterisation.
Instead of computing the score as the gradient of an energy function, we estimate the change in energy from the score.

\subsection{Recovering the energy from the score}
The MH acceptance probability in \cref{eq:mh:acc_prob} is based on the relative probability of the new candidate $\xCand$ and the current sample $\x^\tau$.
The transition probabilities given by the kernel $\mcKern{\cdot \mid \cdot}$ is assumed to be simple to compute and we focus on the quotient $\pEbm(\xCand) / \pEbm(\x^\tau)$.
For an EBM, we have from \cref{eq:mh:rel_energy} that it can be expressed in terms of the difference in energy at $\xCand$ and $\x^\tau$.

We assume that the energy function is not directly available and we write the difference as a line integral on the curve $\curve$
\begin{align}
    \label{eq:energy_diff:def}
    \energy(\xCand, t) - \energy(\x^\tau, t)
    = \int_{\curve} \gradR \energy(\rr, t) \cdot \dd \rr
    = \int_0^1 \gradR \energy(\rr(s), t) \cdot \rr'(s) \, \dd s,
    % = \frac{1}{\sigma_t} \int_0^1 \epsP(\rr(s)) \cdot \rr'(s) \, \dd s,
\end{align}
where $\rr(s)$ is a parameterisation of $\curve$ such that $\rr(0) = \x^\tau$ and $\rr(1) = \xCand$.
The curve parameterisation is arbitrary (under mild conditions), since $\energy$ is a scalar field. Thus, for an energy-based model, the MH ratio can expressed as a path integral of the score.

For a score parametrised diffusion model, we thus propose to calculate a MH-like ratio as
\begin{align}
    f(\hat{x}, x^\tau, t) &= \int_0^1 -\frac{\epsP(r(s), t)}{\sigma_t} \cdot r'(s) \,\dd s \label{eq:fake_energy}\\
    \alpha &= \min \left(1, \exp\left[f(\hat{x}, x^\tau, t)\right] \frac{\, \mcKern{\x^\tau \mid \xCand}}{ \mcKern{\xCand \mid \x^\tau }} \right). \label{eq:fake_mh}
\end{align}
%
Note that if the score $-\frac{\epsP(x, t)}{\sigma_t} = \gradX F(x, t)$ for some function $F$, \eqref{eq:fake_energy} can be interpreted as recovering an (unknown) energy function, and in this case \eqref{eq:fake_mh} agrees with \eqref{eq:mh:acc_prob}. In general, however, no such function $F$ exists, and the expression \eqref{eq:fake_energy} depends on the path $r$ that is integrated over. Nevertheless, we propose to use \eqref{eq:fake_mh} to directly model a MH-like acceptance probability, to be used in an MCMC sampling scheme.

Since \eqref{eq:fake_energy} in general depends on the path $r$ between $\x^\tau$ and $\xCand$, we propose two ways of such a selection:
first, the obvious option of a straight line connecting the two points and second, a curve running through points where we have already evaluated the score function.
The motivation for the latter option is to reduce the computational burden, since evaluating the score is a bottle-neck.
Some MCMC methods (notably HMC), require the score at some additional points apart from the current sample.
By choosing a curve which incorporates these points, we achieve greater numerical accuracy, essentially for free.
 
Specifically, we approximate the line integral with the trapezoidal rule, where the number of line segments used to approximate the curve $\curve$ is treated as a hyperparameter.
Note that we have to evaluate $\epsP$ at some internal points on $\curve$, incurring an additional computational burden (except for those we can re-use in the HMC case).
The energy parameterisation in contrast, only evaluates the energy at $\x^\tau$ and $\xCand$.

\subsection{Composition}
Our approach works well for the product and negation compositions.
For a product distribution in \cref{eq:comp:prod:ebm}, we have a sum of energies in the exponent, which we can approximate from the score models $\epsP^i(\x_t, t)$, using the line integral in \cref{eq:energy_diff:def}
\begin{align}
    \label{eq:energy_diff:prod}
    \eProd(\xCand, t) - \eProd(\x^\tau, t)
    &= \sum_i \energy^i(\xCand, t) - \sum_i \energy^i(\x^\tau, t)
    = \int_0^1 \sum_i \gradR \energy^i(\rr(s), t) \cdot r'(s) \, \dd s \nonumber \\
    &\approx \frac{1}{\sigma_t} \int_0^1 \sum_i \epsP^i(\rr(s), t) \cdot r'(s) \, \dd s.
\end{align}
Negation (as formulated in \cite{du2023reduce}) can be computed in an analogous way.

Mixtures are less suitable for this method as they cannot be described as an energy difference. One could argue that a similar integration idea can be employed to "recover" an energy at $x$, simply, by integrating from an arbitrary fixed point $x^0$ to $x$. However, this would involve integration along a potentially lengthy curve, which comes with several disadvantages.
Firstly, it requires a more refined mesh for the numerical integation.
Secondly, as the estimated score is not a proper conservative field, the choice of curve will have a greater effect over long distances, making the method less robust.
