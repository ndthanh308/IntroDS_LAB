\section{Discussion}
The choice between score and energy parameterizations remains an intriguing and nuanced topic within diffusion-based generative modeling. In this work, we have provided additional empirical evidence suggesting that the score parameterization performs better in the standard reverse process.

At the same time, we have shown that performance gains often attributed to the energy parameterization can, in fact, be recovered within a score-based framework. This is achieved by approximating pseudo-energy differences using a line integral of the model's noise predictions. Notably, this allows us to incorporate MH-like correction steps into a variety of MCMC samplers—without the need to explicitly train an energy-based model—yet still attain comparable improvements in sample quality.

A particularly interesting observation is that using a curve composed only of model evaluations from the HMC sampler appears to perform on par with using a straight-line path. This suggests that the proposed correction comes at virtually no additional computational cost in this case. However, it is worth noting that in higher-dimensional settings, additional intermediate points along the integration path may be required to maintain accuracy, which could increase the computational burden. This challenge might be addressed through more efficient numerical integration techniques, or by working in a lower-dimensional latent space, as is done in latent diffusion models. One persistent drawback of the energy parameterization is that it always requires an explicit gradient computation to recover the score function.

One limitation of the score parameterization is that the learned vector field is not guaranteed to be conservative. In other words, it does not, in general, correspond to the gradient of a scalar energy function. Nonetheless, recent work by~\cite{horvat2024gauge} demonstrates that a vector field does not necessarily need to be strictly conservative to generate accurate samples or estimate a density effectively. This perspective aligns with the empirical success of score-based generative models that operate without explicitly modeling an energy function. Likewise, our MH-like correction mechanism, though built on a non-conservative field, yields significant improvements when applied to the reverse process.

Still, the lack of exact conservativity may explain the slightly superior performance of the energy parameterization observed in the CIFAR-100 experiment. Developing better techniques for estimating pseudo-energy differences from score-based models—without requiring an explicitly trained energy function—thus remains a highly relevant and promising direction for future research.



