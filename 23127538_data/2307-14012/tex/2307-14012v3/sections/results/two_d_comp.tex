\subsection{2D Composition} \label{section:2dcomp}
To investigate the effectiveness of our MH-like correction in a controlled yet expressive setting, we replicate the 2D composition experiment introduced by~\citet{du2023reduce}, using their publicly available codebase\footnote{\url{https://github.com/yilundu/reduce_reuse_recycle}} as a foundation. Our experimental setup mirrors theirs unless otherwise specified.

A 2D density pair is composed via multiplication into a complex distribution, as in (\ref{eq:comp:prod:ebm}): a Gaussian mixture with 8 modes in a circle and a uniform distribution covering two of the modes. For a visual representation
of the two individual distributions and their resulting product distribution together with samples from the reverse diffusion and HMC corrected samples, see Figure \ref{fig:exp:2d_toy}. The baseline reverse diffusion process uses $T=100$ steps. In the MCMC variants, following \cite{du2023reduce}, we omit the optional reverse step for a fair comparison. MCMC sampling runs for $\mcSteps = 10$ at each $t$, with (U-)HMC using 3 leapfrog steps per MCMC step.
% Figure environment removed
%
We evaluate performance using three metrics. The first is negative log-likelihood (NLL), which assesses the likelihood of generated samples under the true data distribution. To address potential samples outside the true distribution’s support, we extend it by adding a small uniform probability.
The second metric is a Gaussian mixture model (GMM), where we fit bi-modal GMMs to samples from both the true and model distributions and compute the Frobenius norm of the variance mean difference.
Finally, we use the Wasserstein-2 distance ($W_2$) to measure the discrepancy between the data and model distributions by computing the optimal assignment between sampled sets~\citep{villani2009opttransp}.

We present quantitative results for the 2D composition in Table~\ref{tab:exp:2d_combined}\hyperlink{tab:2d_perf}{(a)}, averaged over 10 independent trials. In each trial, we train the diffusion models from scratch and sample 2000 points using different MCMC methods. The results show that the corrected sampling methods outperform the unadjusted ones. HMC variants yield better results than Langevin, while the reverse process performs worse. Score and energy parameterizations exhibit similar NLL and GMM performance within their respective methods. However, with HMC, the score parameterization significantly outperforms the energy parameterization in $W_2$. Performance also saturates with as few as three points in the trapezoidal rule.

Additionally, we measured memory usage and runtime during this experiment, see Table~\ref{tab:exp:2d_combined}\hyperlink{tab:2d_runtime}{(b)}. Score-based parameterization was more than twice as memory-efficient as energy-based parameterization and, with the exception of LA with 8 extra trapezoidal evaluations, faster for the corresponding MCMC methods. Notably, HMC curve was nearly three times faster. While our approach requires more model evaluations, this does not necessarily make it slower or more memory-intensive than using an energy-based model. However, these results are implementation-dependent, and further investigation is needed to confirm whether these trends generalize to other setups. 
\begin{table}[ht]
\centering
\caption{
Quantitative results for different samplers in the 2D composition experiment. 
(a) shows performance metrics (NLL, GMM, and $W_2$) based on 10 independent trials, with lower values indicating better performance. 
(b) reports average runtime (in seconds) and peak memory consumption (in MiB). 
For the score parameterization, we include variants with different numbers of additional points in the trapezoidal rule (e.g., 1L, 3L, 8L) and different integration paths (“L” for a straight line and “C” for the HMC trajectory).
}
\label{tab:exp:2d_combined}
\vspace{0.5em}

\begin{minipage}[t]{0.62\linewidth}
\centering
\textbf{(a) Performance metrics}\hypertarget{tab:2d_perf}{}\\[0.5ex]
\begin{tabular}{|c|c|c|c|c|}
\hline
 & Sampler & NLL\textdownarrow & GMM\textdownarrow & $W_2$\textdownarrow \\
\hline
\multirow{5}{*}{\rotatebox[origin=c]{90}{Energy}} 
& Reverse   & $8.22 \pm 0.21$ & $27.01 \pm 1.34$ & $5.81 \pm 0.19$ \\
& U-LA      & $7.52 \pm 0.22$ & $14.61 \pm 1.35$ & $4.19 \pm 0.45$ \\
& LA        & $6.50 \pm 0.30$ & $14.66 \pm 1.46$ & $4.24 \pm 0.55$ \\
& U-HMC     & $5.72 \pm 0.18$ & $6.53 \pm 0.91$  & $4.19 \pm 1.25$ \\
& HMC       & $\pmb{4.09 \pm 0.14}$ & $\pmb{3.33 \pm 0.65}$ & $\pmb{4.12 \pm 1.44}$ \\
\hline
\multirow{10}{*}{\rotatebox[origin=c]{90}{Score}} 
& Reverse   & $8.15 \pm 0.24$ & $26.88 \pm 1.20$ & $5.80 \pm 0.20$ \\
& U-LA      & $7.57 \pm 0.12$ & $14.99 \pm 0.62$ & $4.44 \pm 0.63$ \\
& LA-1L     & $6.45 \pm 0.20$ & $14.28 \pm 1.07$ & $4.03 \pm 0.52$ \\
& LA-3L     & $6.61 \pm 0.17$ & $15.19 \pm 0.92$ & $4.22 \pm 0.46$ \\
& LA-8L     & $6.53 \pm 0.17$ & $14.75 \pm 0.91$ & $4.20 \pm 0.51$ \\
& U-HMC     & $5.77 \pm 0.12$ & $6.90 \pm 0.71$  & $3.39 \pm 0.77$ \\
& HMC-1L    & $4.29 \pm 0.13$ & $3.72 \pm 0.61$  & $2.92 \pm 1.02$ \\
& HMC-3L    & $\pmb{4.07 \pm 0.13}$ & $3.08 \pm 0.69$ & $\pmb{2.68 \pm 1.20}$ \\
& HMC-8L    & $\pmb{4.07 \pm 0.14}$ & $3.17 \pm 0.56$ & $2.87 \pm 0.89$ \\
& HMC-C     & $\pmb{4.07 \pm 0.12}$ & $\pmb{3.06 \pm 0.54}$ & $2.94 \pm 0.90$ \\
\hline
\end{tabular}
\end{minipage}
\hfill
\begin{minipage}[t]{0.35\linewidth}
\centering
\textbf{(b) Runtime and memory usage}\hypertarget{tab:2d_runtime}{}\\[0.5ex]
\begin{tabular}{|c|c|c|c|}
\hline
 & Sampler & Time & Memory \\
\hline
\multirow{5}{*}{\rotatebox[origin=c]{90}{Energy}} 
& Reverse   & $2.1$ & $5252$ \\
& U-LA      & $2.7$ & $5252$ \\
& LA        & $10.4$ & $5252$ \\
& U-HMC     & $19.3$ & $5254$ \\
& HMC       & $22.8$ & $5256$ \\
\hline
\multirow{10}{*}{\rotatebox[origin=c]{90}{Score}} 
& Reverse   & $1.6$ & $2178$ \\
& U-LA      & $2.2$ & $2180$ \\
& LA-1L     & $6.1$ & $2180$ \\
& LA-3L     & $8.8$ & $2180$ \\
& LA-8L     & $13.6$ & $2180$ \\
& U-HMC     & $9.5$ & $2180$ \\
& HMC-1L    & $8.8$ & $2180$ \\
& HMC-3L    & $11.6$ & $2180$ \\
& HMC-8L    & $16.4$ & $2180$ \\
& HMC-C     & $7.1$ & $2180$ \\
\hline
\end{tabular}
\end{minipage}
\end{table}
