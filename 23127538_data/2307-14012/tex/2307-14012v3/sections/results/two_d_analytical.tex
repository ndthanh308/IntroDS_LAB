\subsection{Evaluating Pseudo-Energy Differences} \label{section:pseudo-energy}

To evaluate the accuracy of pseudo-energy differences, we conducted experiments on a synthetic 2D dataset, generated from a bivariate Gaussian distribution to allow access to analytical solutions, and a higher-dimensional dataset, MNIST~\citep{mnist}. For each experiment, we trained 10 score and energy models independently from scratch. For evaluation, we sampled 2k pairs of points $(x_t^1, x_t^2)$ via the forward process at various diffusion steps $t$, and these pairs were used to compute the (pseudo-)energy difference $\Delta E$ for both the score and energy models (and analytically when available). The pseudo-energy difference was computed along a linear curve connecting the two points, using five points for numerical integration.


\textbf{2D Gaussian:}  
For the 2D Gaussian dataset, the relative error metric is defined as 
$|\Delta E_{\text{pred}} - \Delta E_{\text{true}}| / |\Delta E_{\text{true}}|$, 
where $\Delta E_{\text{pred}}$ is the predicted energy difference and $\Delta E_{\text{true}}$ is the analytical energy difference. The median relative error was calculated across all sampled pairs for each trained model, and the mean and standard deviation of this metric were computed across the 10 models. Interestingly, the score model achieved a lower relative error $0.071 \pm 0.005$ compared to the energy model $0.084 \pm 0.004$, demonstrating better alignment with the true energy differences.

\textbf{MNIST:}  
For the MNIST dataset, where analytical energy differences are unavailable, we used a symmetric relative error metric defined as 
$2 |\Delta E_{\text{score}} - \Delta E_{\text{energy}}| / (|\Delta E_{\text{score}}| + |\Delta E_{\text{energy}}|)$. 
The median relative error was calculated across all sampled pairs for each trained model, and the mean and standard deviation were computed across the 10 models. This yielded a mean relative error of $0.030 \pm 0.002$ indicating that the energy differences predicted by the score and energy models align closely, even in this higher-dimensional setting.

