\section{Results}

In this section, we present an empirical evaluation of our MH-like correction method, examining both the accuracy of the pseudo-energy differences and the quality of the generated samples. The experiments are designed to span a spectrum of difficulty: from controlled, low-dimensional setups where models can be trained from scratch and analytical solutions are available, to more realistic high-dimensional scenarios involving pre-trained models. Our two primary objectives are (1) to compare our proposed approach against a true energy parameterization when available, and (2) to assess the sampling improvements achieved over the standard reverse process when augmented with MCMC steps.

The experiments in Sections~\ref{section:pseudo-energy}, \ref{section:2dcomp}, and the first part of \ref{section:guideddiffusion} involve training diffusion models using both energy and score parameterizations. The score parameterization follows a noise prediction model, $\epsP(\x_t, t)$, while the energy parameterization defines an energy function as $\energy(\x_t, t) = \twoNorm{\x_t - \score(\x_t, t)}^2$, as in~\citep{du2023reduce}. We use identical network architectures for $\epsilon_\theta$ and $s_\theta$. Both models are trained with the standard diffusion loss~\citep{ho2020denoising}, with the energy model’s score function obtained through explicit differentiation.

The later experiments utilize only pre-trained score-based diffusion models, as pre-trained energy-based models are unavailable for direct comparison. We evaluate both unadjusted and MH-corrected versions of Langevin and Hamiltonian Monte Carlo, comparing them against the standard reverse process, which serves as the baseline.

For the MH-like correction, we examine two types of integration paths: line and curve. The line follows a direct path between $\x^\tau$ and $\xCand$, while the curve integrates along the trajectory formed by HMC leapfrog steps. The number of points for the trapezoidal rule’s mesh is treated as a hyperparameter. Since points like $\x^\tau$ and $\xCand$ are already included, the hyperparameter refers to the additional points, which are evenly distributed along the curve.

Complete training details, hyperparameter settings, and implementation specifics are deferred to Appendix~\ref{sec:expdetails}.


\subsection{Evaluating Pseudo-Energy Differences} \label{section:pseudo-energy}

To evaluate the accuracy of pseudo-energy differences, we conducted experiments on a synthetic 2D dataset, generated from a bivariate Gaussian distribution to allow access to analytical solutions, and a higher-dimensional dataset, MNIST~\citep{mnist}. For each experiment, we trained 10 score and energy models independently from scratch. For evaluation, we sampled 2k pairs of points $(x_t^1, x_t^2)$ via the forward process at various diffusion steps $t$, and these pairs were used to compute the (pseudo-)energy difference $\Delta E$ for both the score and energy models (and analytically when available). The pseudo-energy difference was computed along a linear curve connecting the two points, using five points for numerical integration.


\textbf{2D Gaussian:}  
For the 2D Gaussian dataset, the relative error metric is defined as 
$|\Delta E_{\text{pred}} - \Delta E_{\text{true}}| / |\Delta E_{\text{true}}|$, 
where $\Delta E_{\text{pred}}$ is the predicted energy difference and $\Delta E_{\text{true}}$ is the analytical energy difference. The median relative error was calculated across all sampled pairs for each trained model, and the mean and standard deviation of this metric were computed across the 10 models. Interestingly, the score model achieved a lower relative error $0.071 \pm 0.005$ compared to the energy model $0.084 \pm 0.004$, demonstrating better alignment with the true energy differences.

\textbf{MNIST:}  
For the MNIST dataset, where analytical energy differences are unavailable, we used a symmetric relative error metric defined as 
$2 |\Delta E_{\text{score}} - \Delta E_{\text{energy}}| / (|\Delta E_{\text{score}}| + |\Delta E_{\text{energy}}|)$. 
The median relative error was calculated across all sampled pairs for each trained model, and the mean and standard deviation were computed across the 10 models. This yielded a mean relative error of $0.030 \pm 0.002$ indicating that the energy differences predicted by the score and energy models align closely, even in this higher-dimensional setting.


\subsection{2D Composition} \label{section:2dcomp}
To investigate the effectiveness of our MH-like correction in a controlled yet expressive setting, we replicate the 2D composition experiment introduced by~\citet{du2023reduce}, using their publicly available codebase\footnote{\url{https://github.com/yilundu/reduce_reuse_recycle}} as a foundation. Our experimental setup mirrors theirs unless otherwise specified.

A 2D density pair is composed via multiplication into a complex distribution, as in (\ref{eq:comp:prod:ebm}): a Gaussian mixture with 8 modes in a circle and a uniform distribution covering two of the modes. For a visual representation
of the two individual distributions and their resulting product distribution together with samples from the reverse diffusion and HMC corrected samples, see Figure \ref{fig:exp:2d_toy}. The baseline reverse diffusion process uses $T=100$ steps. In the MCMC variants, following \cite{du2023reduce}, we omit the optional reverse step for a fair comparison. MCMC sampling runs for $\mcSteps = 10$ at each $t$, with (U-)HMC using 3 leapfrog steps per MCMC step.
% Figure environment removed
%
We evaluate performance using three metrics. The first is negative log-likelihood (NLL), which assesses the likelihood of generated samples under the true data distribution. To address potential samples outside the true distribution’s support, we extend it by adding a small uniform probability.
The second metric is a Gaussian mixture model (GMM), where we fit bi-modal GMMs to samples from both the true and model distributions and compute the Frobenius norm of the variance mean difference.
Finally, we use the Wasserstein-2 distance ($W_2$) to measure the discrepancy between the data and model distributions by computing the optimal assignment between sampled sets~\citep{villani2009opttransp}.

We present quantitative results for the 2D composition in Table~\ref{tab:exp:2d_combined}\hyperlink{tab:2d_perf}{(a)}, averaged over 10 independent trials. In each trial, we train the diffusion models from scratch and sample 2000 points using different MCMC methods. The results show that the corrected sampling methods outperform the unadjusted ones. HMC variants yield better results than Langevin, while the reverse process performs worse. Score and energy parameterizations exhibit similar NLL and GMM performance within their respective methods. However, with HMC, the score parameterization significantly outperforms the energy parameterization in $W_2$. Performance also saturates with as few as three points in the trapezoidal rule.

Additionally, we measured memory usage and runtime during this experiment, see Table~\ref{tab:exp:2d_combined}\hyperlink{tab:2d_runtime}{(b)}. Score-based parameterization was more than twice as memory-efficient as energy-based parameterization and, with the exception of LA with 8 extra trapezoidal evaluations, faster for the corresponding MCMC methods. Notably, HMC curve was nearly three times faster. While our approach requires more model evaluations, this does not necessarily make it slower or more memory-intensive than using an energy-based model. However, these results are implementation-dependent, and further investigation is needed to confirm whether these trends generalize to other setups. 
\begin{table}[ht]
\centering
\caption{
Quantitative results for different samplers in the 2D composition experiment. 
(a) shows performance metrics (NLL, GMM, and $W_2$) based on 10 independent trials, with lower values indicating better performance. 
(b) reports average runtime (in seconds) and peak memory consumption (in MiB). 
For the score parameterization, we include variants with different numbers of additional points in the trapezoidal rule (e.g., 1L, 3L, 8L) and different integration paths (“L” for a straight line and “C” for the HMC trajectory).
}
\label{tab:exp:2d_combined}
\vspace{0.5em}

\begin{minipage}[t]{0.62\linewidth}
\centering
\textbf{(a) Performance metrics}\hypertarget{tab:2d_perf}{}\\[0.5ex]
\begin{tabular}{|c|c|c|c|c|}
\hline
 & Sampler & NLL\textdownarrow & GMM\textdownarrow & $W_2$\textdownarrow \\
\hline
\multirow{5}{*}{\rotatebox[origin=c]{90}{Energy}} 
& Reverse   & $8.22 \pm 0.21$ & $27.01 \pm 1.34$ & $5.81 \pm 0.19$ \\
& U-LA      & $7.52 \pm 0.22$ & $14.61 \pm 1.35$ & $4.19 \pm 0.45$ \\
& LA        & $6.50 \pm 0.30$ & $14.66 \pm 1.46$ & $4.24 \pm 0.55$ \\
& U-HMC     & $5.72 \pm 0.18$ & $6.53 \pm 0.91$  & $4.19 \pm 1.25$ \\
& HMC       & $\pmb{4.09 \pm 0.14}$ & $\pmb{3.33 \pm 0.65}$ & $\pmb{4.12 \pm 1.44}$ \\
\hline
\multirow{10}{*}{\rotatebox[origin=c]{90}{Score}} 
& Reverse   & $8.15 \pm 0.24$ & $26.88 \pm 1.20$ & $5.80 \pm 0.20$ \\
& U-LA      & $7.57 \pm 0.12$ & $14.99 \pm 0.62$ & $4.44 \pm 0.63$ \\
& LA-1L     & $6.45 \pm 0.20$ & $14.28 \pm 1.07$ & $4.03 \pm 0.52$ \\
& LA-3L     & $6.61 \pm 0.17$ & $15.19 \pm 0.92$ & $4.22 \pm 0.46$ \\
& LA-8L     & $6.53 \pm 0.17$ & $14.75 \pm 0.91$ & $4.20 \pm 0.51$ \\
& U-HMC     & $5.77 \pm 0.12$ & $6.90 \pm 0.71$  & $3.39 \pm 0.77$ \\
& HMC-1L    & $4.29 \pm 0.13$ & $3.72 \pm 0.61$  & $2.92 \pm 1.02$ \\
& HMC-3L    & $\pmb{4.07 \pm 0.13}$ & $3.08 \pm 0.69$ & $\pmb{2.68 \pm 1.20}$ \\
& HMC-8L    & $\pmb{4.07 \pm 0.14}$ & $3.17 \pm 0.56$ & $2.87 \pm 0.89$ \\
& HMC-C     & $\pmb{4.07 \pm 0.12}$ & $\pmb{3.06 \pm 0.54}$ & $2.94 \pm 0.90$ \\
\hline
\end{tabular}
\end{minipage}
\hfill
\begin{minipage}[t]{0.35\linewidth}
\centering
\textbf{(b) Runtime and memory usage}\hypertarget{tab:2d_runtime}{}\\[0.5ex]
\begin{tabular}{|c|c|c|c|}
\hline
 & Sampler & Time & Memory \\
\hline
\multirow{5}{*}{\rotatebox[origin=c]{90}{Energy}} 
& Reverse   & $2.1$ & $5252$ \\
& U-LA      & $2.7$ & $5252$ \\
& LA        & $10.4$ & $5252$ \\
& U-HMC     & $19.3$ & $5254$ \\
& HMC       & $22.8$ & $5256$ \\
\hline
\multirow{10}{*}{\rotatebox[origin=c]{90}{Score}} 
& Reverse   & $1.6$ & $2178$ \\
& U-LA      & $2.2$ & $2180$ \\
& LA-1L     & $6.1$ & $2180$ \\
& LA-3L     & $8.8$ & $2180$ \\
& LA-8L     & $13.6$ & $2180$ \\
& U-HMC     & $9.5$ & $2180$ \\
& HMC-1L    & $8.8$ & $2180$ \\
& HMC-3L    & $11.6$ & $2180$ \\
& HMC-8L    & $16.4$ & $2180$ \\
& HMC-C     & $7.1$ & $2180$ \\
\hline
\end{tabular}
\end{minipage}
\end{table}

\subsection{Guided Diffusion} \label{section:guideddiffusion}

We evaluate our proposed sampling methods for guided diffusion on the CIFAR-100~\citep{krizhevsky2009cifar} and ImageNet~\citep{deng2009imagenet} datasets. The sampling process is based on a score function defined in (\ref{eq:guidance:score}). For both datasets, the marginal score, $\gradX \log q(\x_t)$, is estimated using an unconditional diffusion model parameterized by a UNet architecture. For the guidance model, we use classifier-full guidance, training a time-dependent classifier to predict class labels across all diffusion steps, $\pCfull(y \mid \x_t, t)$. This classifier shares its architecture with the encoder part of the UNet used for the diffusion model and is extended with a dense output layer. The guidance scale is set to $\lambda = 20.0$ across all experiments. Sampling is based on the standard reverse process with $T=1000$, and additional MCMC steps are incorporated to refine the generated samples.

To quantify generation quality, we use three evaluation metrics: the Fréchet Inception Distance (FID)~\citep{heusel2017gans}, which compares the distribution of generated and real images; classification accuracy, based on a separate pre-trained classifier applied to generated samples; and, for ImageNet, an additional top-5 accuracy metric.

\textbf{CIFAR-100:}  
For CIFAR-100, we trained the diffusion models from scratch using the same UNet architecture and training settings as in \cite{ho2020denoising}, which were originally designed for CIFAR-10~\citep{krizhevsky2009cifar}. The MCMC samplers add $\mcSteps = 2$ or $6$ extra MCMC steps at each diffusion step $t$ for (U-)HMC and (U-)LA, respectively, with (U-)HMC using three leapfrog steps per MCMC step. 

For this experiment, more points are needed in the trapezoidal rule’s mesh than in the 2D experiment. Based on previous insights, for HMC we integrate only along the curve from the leapfrog steps, with an additional midpoint evaluation, resulting in three extra model evaluations per HMC step. For LA, we use ten points along the line, resulting in eight extra evaluations per step.

Recognizing the impact of the step length on MCMC methods in general, we parameterize the step length as a function of the beta-schedule $\langStep_t = a \beta_t^b$. We conducted a simple parameter search for parameters $a$ and $b$, to determine a suitable step length for each MCMC variant.

The results are shown in Table~\ref{tab:exp:cifar100}. Average accuracy is obtained using a separate classifier trained exclusively on noise-free pairs $(\x_0, y)$, following the VGG-13-BN architecture~\citep{simonyan2014very}. The table shows a general trend of improvement over the baseline reverse process when additional MCMC steps are added. In particular, the MH-corrected samplers LA and HMC show significant improvements in FID scores, which are arguably the more important metric for image generation.

Comparing the score and energy parameterizations, their performances share similar characteristics. Interestingly, the reverse process favors the score parameterization, supporting the claim that this less restricted approach better models the score function. However, the energy parameterization sees larger improvements from the added MCMC steps. This indicates, perhaps, that direct energy estimation provides a better correction step compared to our method of approximating the pseudo-energy difference from $\epsP$. Although the energy-based method performs slightly better in this setting, our MH-corrected sampling methods achieve comparable improvements without requiring an energy model.

\begin{table}
\centering
\caption{Accuracy and FID score for classifier-full guidance on CIFAR-100. The metrics are based on 50k generated samples for each sampling method with both energy and score models.}
\begin{tabular}{|c|c|c|c|}
\hline
 & Sampler & Accuracy [\%]\textuparrow & FID\textdownarrow \\ \hline
\multirow{5}{*}{Energy} 
& Reverse   & 72.6 & 33.4 \\
& U-LA      & $\pmb{87.3}$ & 24.6 \\
& LA        & 80.0 & 12.7 \\
& U-HMC     & 87.2 & 25.4 \\
& HMC       & 84.9 & $\pmb{12.4}$ \\
\hline
\multirow{5}{*}{Score}
& Reverse   & $74.2$ & $31.8$ \\
& U-LA      & $\pmb{82.9}$ & $25.9$\\
& LA-8L     & $75.2$ & $15.5$ \\
& U-HMC     & $79.0$ & $28.6$ \\
& HMC-3C    & $75.8$ & $\pmb{13.3}$ \\
\hline
\end{tabular}
\label{tab:exp:cifar100}
\end{table}

\textbf{ImageNet:}  
For ImageNet, training diffusion models from scratch is computationally expensive, so we rely on pre-trained models. Score-based models are publicly available through the OpenAI GitHub repository\footnote{\url{https://github.com/openai/guided-diffusion}}, as provided by \cite{dhariwal2021diffbeatgan}. Unfortunately, no equivalent pre-trained energy-based models are available. Given the high computational demands of large-scale diffusion models, we focus solely on evaluating HMC and compare it to the reverse process. The HMC sampler adds $\mcSteps=2$ MCMC steps per diffusion step $t$, with each step consisting of three leapfrog steps. For the trapezoidal rule, we incorporate the points from the leapfrog steps and add two additional points between each leapfrog step. The step length parameterization and tuning follow the same procedure as in CIFAR-100.

The results can be seen in Table~\ref{tab:exp:imagenet}. Accuracy metrics are computed using a pre-trained RegNetX-8.0GF \cite{radosavovic2020designing} classifier. The reverse process and HMC perform very similarly in average accuracy, but our method shows a slight improvement in top-5 average accuracy. HMC obtains a significantly better FID score.

\begin{table}
\centering
\caption{Average accuracy, top-5 accuracy, and FID score for classifier-full guidance on ImageNet. The metrics are based on 50k generated samples for both sampling methods with score parameterizations.}
\begin{tabular}{|c|c|c|c|c|}
\hline
 & Sampler & Acc [\%]\textuparrow & Acc-5 [\%]\textuparrow & FID\textdownarrow \\ \hline
\multirow{2}{*}{Score}
& Reverse   & $\pmb{50.0}$ & $83.9$ & $14.5$ \\
& HMC-6C    & $49.9$ & $\pmb{85.1}$ & $\pmb{11.6}$ \\
\hline
\end{tabular}
\label{tab:exp:imagenet}
\end{table}

\subsection{Image tapestry}
We conduct a so-called image tapestry experiment, similar to the one in \cite{du2023reduce} and based on their code\footnote{https://github.com/yilundu/reduce\_reuse\_recycle}, as our final experiment. This experiment involves not only the composition of guidance---in this case, classifier-free guidance---but also the composition of combining multiple overlapping text-to-image models. This approach allows us to construct an image with specified content at different spatial locations. Here, we use a pre-trained DeepFloyd-IF\footnote{Available at https://huggingface.co/DeepFloyd/IF-I-XL-v1.0} model. For each diffusion step ($T=100$), 15 extra LA steps were added, with three additional evaluation points for line integration for each step. The guidance scale $\lambda = 20.0$. For more details, see the \cref{sec:expdetails:image_tapestry}. In \cref{fig:tapestry}, we can see a generated tapestry image, and in \cref{fig:tapestry_content}, we can see the specified content at the corresponding spatial locations. There are, in total, nine overlapping content boxes: four are positioned in each corner with different content, while the remaining five are arranged to create a unified image using the same content prompt.

% Figure environment removed