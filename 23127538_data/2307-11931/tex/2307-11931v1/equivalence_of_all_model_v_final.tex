%\documentclass[12pt]{book}
%\usepackage{index}
%\makeindex
%\renewcommand\indexname{Index of notions}
%\newindex{notation}{adx}{and}{Index of symbols}
%\newindex{notion}{bdx}{bnd}{Index of notions}
%\usepackage{tikz}
\usepackage{xcolor,xspace}
\usepackage{url}
\usepackage{epsfig,graphicx,endnotes,kotex,subfigure,multirow,amsmath,algorithm,algpseudocode}
\newcommand\StateX{\Statex\hspace{\algorithmicindent}}%
%\usepackage{breakurl}
%\usepackage[sort,space]{cite}
\usepackage{balance}
%\usepackage{tabularx}
\usepackage{enumitem}
\usepackage{flushend}
\usepackage[T1]{fontenc}
\usepackage{color,soul}
\hyphenation{op-tical net-works semi-conduc-tor}
%\usepackage{filecontents}
%\usepackage{booktabs} % For formal tables
\usepackage{amsthm}
\newtheorem{theorem}{Theorem}
\newtheorem{corollary}{Corollary}
\newtheorem{lemma}{Lemma}
\renewcommand{\qedsymbol}{\rule{0.7em}{0.7em}}

%\newcommand\notion[1]{\textit{#1}\index[notion]{#1}}
\newcommand\wcnotion[2]{\textit{#1}\index[notion]{#2}}
\newcommand\wcnotionsym[3]{\textit{#1}\index[notation]{#2}\index[notion]{#3}}
\newcommand\wcsnotion[3]{\textit{#1}\index[notion]{#2!\textit{#3}}}
\newcommand\snotion[2]{\textit{#1}\index[notion]{#1!\textit{#2}}}
\newcommand\snotionsym[3]{\textit{#1}\index[notion]{#1!\textit{#3}}\index[notation]{#2!\textit{#3}}}
\newcommand\wcsnotionsym[4]{\textit{#1}\index[notation]{#2!\textit{#4}}\index[notion]{#3!\textit{#4}}}

\newcommand\wcnotation[2]{\textit{#1}\index[notation]{#2}}
\newcommand\wcsnotation[3]{\textit{#1}\index[notation]{#2!\textit{#3}}}

\newcommand\sym[1]{\index[notation]{#1}}
\newcommand\ssym[2]{\index[notation]{#1!\textit{#2}}}

\newcommand{\exclam}{!}





\newcommand{\Ab}{\mathbb{A}} 
\newcommand{\Zb}{\mathbb{Z}} 
\newcommand{\Eb}{\mathbb{E}} 
\newcommand{\Nb}{\mathbb{N}}
\newcommand{\Tb}{\mathbf{T}} 
\newcommand{\Yb}{\mathbb{Y}} 
\newcommand{\Ib}{\mathbb{I}} 
\newcommand{\Ob}{\mathbb{O}} 
\newcommand{\Pb}{\mathbb{P}} 
\newcommand{\Qb}{\mathbb{Q}} 
\newcommand{\Sb}{\mathbb{S}} 
\newcommand{\Hb}{\mathbb{H}} 
\newcommand{\Jb}{\mathbf{J}} 
\newcommand{\Kb}{\mathbb{K}} 
\newcommand{\Mb}{\mathbb{M}} 
\newcommand{\Wb}{\mathbf{W}} 
\newcommand{\Xb}{\mathbb{X}} 
\newcommand{\Cb}{\mathbf{C}}
\newcommand{\Vb}{\mathbb{V}}
\newcommand{\Bb}{\mathbb{B}}


\newcommand{\Acal}{\mathcal{A}} 
\newcommand{\Zcal}{\mathcal{Z}} 
\newcommand{\Ecal}{\mathcal{E}} 
\newcommand{\Rcal}{\mathcal{R}} 
\newcommand{\Tcal}{\mathcal{T}} 
\newcommand{\Ycal}{\mathcal{Y}} 
\newcommand{\Ucal}{\mathcal{U}} 
\newcommand{\Ical}{\mathcal{I}} 
\newcommand{\Ocal}{\mathcal{O}} 
\newcommand{\Pcal}{\mathcal{P}} 
\newcommand{\Qcal}{\mathcal{Q}} 
\newcommand{\Scal}{\mathcal{S}} 
\newcommand{\Dcal}{\mathcal{D}} 
\newcommand{\Fcal}{\mathcal{F}} 
\newcommand{\Gcal}{\mathcal{G}} 
\newcommand{\Hcal}{\mathcal{H}} 
\newcommand{\Jcal}{\mathcal{J}} 
\newcommand{\Kcal}{\mathcal{K}} 
\newcommand{\Lcal}{\mathcal{L}} 
\newcommand{\Mcal}{\mathcal{M}} 
\newcommand{\Wcal}{\mathcal{W}} 
\newcommand{\Xcal}{\mathcal{X}} 
\newcommand{\Ccal}{\mathcal{C}} 
\newcommand{\Vcal}{\mathcal{V}} 
\newcommand{\Bcal}{\mathcal{B}} 
\newcommand{\Ncal}{\mathcal{N}} 


\newcommand{\Ago}{\mathfrak{A}} 
\newcommand{\Zgo}{\mathfrak{Z}} 
\newcommand{\Ego}{\mathfrak{E}} 
\newcommand{\Rgo}{\mathfrak{R}} 
\newcommand{\Tgo}{\mathfrak{T}} 
\newcommand{\Ygo}{\mathfrak{Y}} 
\newcommand{\Ugo}{\mathfrak{U}} 
\newcommand{\Igo}{\mathfrak{I}} 
\newcommand{\Ogo}{\mathfrak{O}} 
\newcommand{\Pgo}{\mathfrak{P}} 
\newcommand{\Qgo}{\mathfrak{Q}} 
\newcommand{\Sgo}{\mathfrak{S}} 
\newcommand{\Dgo}{\mathfrak{D}} 
\newcommand{\Fgo}{\mathfrak{F}} 
\newcommand{\Ggo}{\mathfrak{G}} 
\newcommand{\Hgo}{\mathfrak{H}} 
\newcommand{\Jgo}{\mathfrak{J}} 
\newcommand{\Kgo}{\mathfrak{K}} 
\newcommand{\Lgo}{\mathfrak{L}} 
\newcommand{\Mgo}{\mathfrak{M}} 
\newcommand{\Wgo}{\mathfrak{W}} 
\newcommand{\Xgo}{\mathfrak{X}} 
\newcommand{\Cgo}{\mathfrak{C}} 
\newcommand{\Vgo}{\mathfrak{V}} 
\newcommand{\Bgo}{\mathfrak{B}} 
\newcommand{\Ngo}{\mathfrak{N}}



\newcommand{\sslash}{\mathbin{/\mkern-6mu/}}

\newcommand{\note}[1]{{\color{red}#1}}

\def\-{\raisebox{.75pt}{-}}


\newcommand{\uvar}{\_}


%basic notation
\newcommand{\id}{\text{Id}}
\newcommand{\Db}{\mathbf{D}} 
\DeclareMathOperator*{\dom}{dom}
\DeclareMathOperator*{\codom}{codom}
\DeclareMathOperator{\tw}{tw}


%derived notation
\newcommand{\Rb}{\mathbf{R}} 
\newcommand{\Lb}{\mathbf{L}} 
\newcommand{\Fb}{\mathbf{F}} 
\DeclareMathOperator{\Gb}{G} 
  
%ambiguous notation 
\DeclareMathOperator{\N}{N}
\DeclareMathOperator{\T}{T}
\DeclareMathOperator{\J}{J}


%set of maps
\DeclareMathOperator*{\W}{W}
\DeclareMathOperator*{\Wm}{tW}
\DeclareMathOperator*{\Wseg}{W_{Seg}}
\DeclareMathOperator*{\Wsat}{W_{Sat}}

\DeclareMathOperator*{\M}{M}
\DeclareMathOperator*{\Mm}{tM}
\DeclareMathOperator*{\Mseg}{M_{Seg}}
\DeclareMathOperator*{\Msat}{M_{Sat}}

\DeclareMathOperator*{\I}{I}
\DeclareMathOperator*{\F}{F}

%augmented directed complexes
\DeclareMathOperator*{\CDA}{ADC}
\DeclareMathOperator*{\CDAB}{ADC_B}

%categories
\newcommand\omegacat{\omega\mbox{-$\cat$}}
\DeclareMathOperator\Set{Set}
\DeclareMathOperator\Sp{Sp}

%infini groupoids
\DeclareMathOperator*{\Sq}{Sq}
\DeclareMathOperator*{\Li}{Li}
\DeclareMathOperator{\Hom}{Hom}


%infini 1 categories
\DeclareMathOperator*{\Lfib}{LFib}
\DeclareMathOperator*{\Rfib}{RFib}

\DeclareMathOperator*{\LCartoperator}{LCart}
\DeclareMathOperator*{\core}{core}
\newcommand{\LCart}{\mbox{$\LCartoperator$}}

\newcommand{\LCartc}{\mbox{$\LCartoperator$}^c}
\DeclareMathOperator*{\RCart}{RCart}
\DeclareMathOperator*{\RCartc}{RCart^c}




%infini omega categories
\newcommand{\uLCart}{\underline{\LCartoperator}}
\newcommand{\uLCartc}{\underline{\LCartoperator}^c}
\newcommand{\uRCart}{\underline{RCart}}
\newcommand{\uRCartc}{\underline{RCart}^c}

\DeclareMathOperator{\uHom}{\underline{Hom}}
\DeclareMathOperator{\gHom}{\underline{Hom}_{\ominus}}
\DeclareMathOperator{\Map}{Map}
\DeclareMathOperator{\im}{Im}

\newcommand{\uni}{\underline{\omega}}
\newcommand\w[1]{\widehat{#1}}

%functors
\DeclareMathOperator*{\ev}{ev}
\DeclareMathOperator*{\Arr}{Arr}
\newcommand{\Noiun}{\N_{\tiny{(\omega,1)}}}


\newcommand{\colim}{\operatornamewithlimits{colim}}
\newcommand{\laxcolim}{\operatornamewithlimits{laxcolim}}
\newcommand{\laxlim}{\operatornamewithlimits{laxlim}}


%prefixes
\DeclareMathOperator{\Lan}{Lan}
\DeclareMathOperator{\Ran}{Ran}
\newcommand\iun{(\infty,1)}
\newcommand\io{(\infty,\omega)}
\newcommand\ioun{(\infty,\omega,1)}
\newcommand\zoun{(0,\omega,1)}
\newcommand\zo{(0,\omega)}

%leibnitz products
\DeclareMathOperator{\hstar}{\hat{\star}}
\DeclareMathOperator{\htimes}{\hat{\times}}
\DeclareMathOperator{\hotimes}{\hat{\otimes}}


%Gray operations
\DeclareMathOperator{\costarindex}{f}
\newcommand{\costar}{\mathbin{\overset{co}{\star}}}
\newcommand{\fwedge}{\mathbin{\rotatebox[origin=c]{270}{$\gtrdot$}}}


%inclassable
\newcommand{\invamalg}{\mathbin{\rotatebox[origin=c]{180}{$\amalg$}}}
\DeclareMathOperator{\botimes}{\bar{\otimes}}
\DeclareMathOperator\cst{cst}
\DeclareMathOperator\Operatormark{mk}
\newcommand{\mk}{\Operatormark}

%category theory
\DeclareMathOperator\Fun{Fun}
\DeclareMathOperator\Nat{Nat}
\DeclareMathOperator\End{End}



%fundamental notation
\DeclareMathOperator\mcat{cat_m}
\DeclareMathOperator\cat{cat}
\DeclareMathOperator\grd{grd}
\DeclareMathOperator\R{R}

\newcommand\ocat{(\infty,\omega)\mbox{-$\cat$}}
\newcommand\ouncat{(\infty,\omega,1)\mbox{-$\cat$}}
\newcommand\ocatm{{(\infty,\omega)\mbox{-$\mcat$}}}
\newcommand\zocatm{(0,\omega)\mbox{-$\mcat$}}
\newcommand\zocat{(0,\omega)\mbox{-$\cat$}}
\DeclareMathOperator\zocatB{\zocat_B}
\newcommand\icat{(\infty,1)\mbox{-$\cat$}}
\newcommand\qcat{\mbox{Q$\cat$}}
\newcommand\ncat[1]{(\infty, #1)\mbox{-$\cat$}}
\newcommand\zncat[1]{(0, #1)\mbox{-$\cat$}}
\newcommand\igrd{\infty\mbox{-$\grd$}}



\DeclareMathOperator{\OperatorinfiniPsh}{Psh^\infty}
\DeclareMathOperator{\OperatorinfinitPsh}{tPsh^\infty}
\DeclareMathOperator{\OperatorPsh}{Psh}
\DeclareMathOperator{\OperatormPsh}{mPsh}
\DeclareMathOperator{\OperatortPsh}{tPsh}
\newcommand\iPsh[1]{\OperatorinfiniPsh({#1})}
\newcommand\tiPsh[1]{\OperatorinfinitPsh({#1})}
\newcommand\Psh[1]{\OperatorPsh({#1})}
\newcommand\ssetPsh[1]{\OperatorPsh_\Delta({#1})}
\newcommand\tPsh[1]{\OperatortPsh({#1})}
\newcommand\tPshM[1]{{\OperatortPsh}_M({#1})}
\newcommand\mPsh[1]{\OperatormPsh({#1})}
\newcommand\mPshM[1]{{\OperatormPsh}_M({#1})}

%segal stuff
\DeclareMathOperator{\OperatorSeg}{Seg}
\DeclareMathOperator{\OperatortSeg}{tSeg}
\DeclareMathOperator{\OperatormSeg}{mSeg}
\newcommand\Seg{\OperatorSeg}
\newcommand\mSeg{\OperatormSeg}
\newcommand\stratSeg{\OperatortSeg}

%simplicial variations
\DeclareMathOperator{\Sset}{\Psh{\Delta}}
\newcommand{\mSset}{\mPsh{\Delta}}
\newcommand{\stratSset}{\tPsh{\Delta}}


%univers
\DeclareMathOperator{\U}{\mathbf{U}}
\DeclareMathOperator{\V}{\mathbf{V}}
\DeclareMathOperator{\Wcard}{\mathbf{W}}
\DeclareMathOperator{\Z}{\mathbf{Z}}



%Grothendieck constructions
\newcommand{\ringpartial}{\mathring{\partial}}
%
%
%\usepackage[inline]{showlabels}
%
%\usepackage{fancyhdr}
%\usepackage{titlesec}
%\usepackage{textcase}
%
%\pagestyle{fancy}
%
%
%\fancyfoot[C]{\thepage}
%
%
%\title{\Huge{Theory and models of $(\infty,\omega)$-categories}}
%\author{Félix Loubaton}
%\date{}
%\linespread{1.2}	
%\geometry{a4paper,top=3cm,bottom=4cm,left=1.5cm,right=3cm, heightrounded,bindingoffset=5mm}	
%
%
%\fancyhf{}
%\fancyhfoffset[RO,LE]{0.5cm}
%\fancyhfoffset[LE,RO]{0.5cm}
%
%\fancyhead[RO]{\rmfamily\nouppercase{\rightmark}}
%\fancyhead[LE]{\rmfamily\nouppercase{\leftmark}}
%\fancyfoot[C]{\thepage}
%\begin{document}
%

Let $n\in \Nb\cup\{\omega\}$.
Following the terminology of Barwick and Schommer-Pries (\cite{Barwick_on_the_unicity_of_the_theory_of_higher_categories}), we call \textit{model of $(\infty,n)$-categories} any model category whose corresponding $(\infty, 1)$-category is $\ncat{n}$.

With the definition of $(\infty,n)$-categories given in the introduction, we have a natural model for the $\iun$-category $\ncat{n}$, given by Rezk's complete Segal $\Theta_n$-spaces, i.e. space valued presheaves on $\Theta_n$ satisfying the (homotopical) Segal conditions and (homotopical) completeness conditions. However, there are many other models, see for instance \cite{Ara_Higher_quasi_cat}, \cite{Bergner_Comparison_of_model_of_infini_n_categories}, \cite{Bergner_Comparison_of_model_for_infini_n_categories_II}, \cite{Bergner_reedy_category_and_the_theta_construction} (we refer to  \cite{Barwick_on_the_unicity_of_the_theory_of_higher_categories}
for a comprehensive presentation of these models and their equivalence). For example, one can mention $n$-fold Segal spaces and Simpson's and Tamsamani's Segal $n$-categories among others.

It was conjectured (\cite{Street_algebra_of_orianted_simplexes}, \cite{Verity_a_complicial_compendium}, \cite{Barwick_on_the_unicity_of_the_theory_of_higher_categories}) that Verity's $n$-complicial sets were also a model of $(\infty,n)$-categories. This would imply that Campion-Kapulkin-Maehara's $n$-comical sets also are, as they are shown to be Quillen equivalent to $n$-complicial sets in \cite{Doherty_Equivalence_of_cubical_and_simplicial_approaches}.

Results of Bergner, Gagna, Harpaz, Joyal, Lanari, Lurie, Rezk and Tierney (\cite{Bergner_Comparison_of_model_of_infini_n_categories},\cite{Bergner_Comparison_of_model_for_infini_n_categories_II}, \cite{Rezk_a_cartesian_of_weak_n_categories}, \cite{Lurie_Htt},\cite{Lurie_goodwillie_calculus}, \cite{Gagna_on_the_equivallence_of_all_model_for_infini2_cat}, \cite{Joyal_Quasi-categories_vs_Segal_spaces}) imply that $2$-complicial sets are a model of $(\infty,2)$-categories (see \cite{Gagna_on_the_equivallence_of_all_model_for_infini2_cat} to understand how to use all this source to obtained the desired result and \cite{Bergner_explicit_comparaison_bt_theta_2_space_and_2_complicial_set} for a direct comparison between complete Segal $\Theta_2$-spaces and $2$-complicial sets).
The purpose of this chapter is to generalize this result to any $n\in \Nb\cup\{\omega\}$.


To this extend, we first address the more general problem of finding sufficient conditions on a model category $A$ to build a \textit{Gray cylinder} $C\mapsto I\otimes C$ and a \textit{Gray cone} $C\mapsto e\star C$ on Segal precategories enriched in $A$. These two operations should be linked by the following homotopy cocartesian square
% https://q.uiver.app/?q=WzAsNCxbMSwwLCJJXFxvdGltZXMgQyJdLFsxLDEsImVcXHN0YXIgQyJdLFswLDEsImUiXSxbMCwwLCJcXHswXFx9XFxvdGltZXMgQyJdLFswLDFdLFszLDJdLFsyLDFdLFszLDBdXQ==
\[\begin{tikzcd}
	{\{0\}\otimes C} & {I\otimes C} \\
	e & {e\star C}
	\arrow[from=1-2, to=2-2]
	\arrow[from=1-1, to=2-1]
	\arrow[from=2-1, to=2-2]
	\arrow[from=1-1, to=1-2]
\end{tikzcd}\]
where $e$ is the terminal object. The conditions that $A$ has to	 fulfill are encapsulated in the notion of \textit{Gray module} (paragraph \ref{para:Gray module}). Thanks to the Gray cylinder and cone, we can show the following theorem:

\begin{itheorem}[\ref{theo:Quillen adjunction}]
If $A$ is a Gray module, there is a Quillen adjunction between the Ozornova-Rovelli model structure for $\omega$-complicial sets on stratified simplicial sets and stratified Segal precategories enriched in $A$ where the left adjoint sends $[n]$ to $e\star e\star ... \star e\star \emptyset$
\end{itheorem} 

We will apply this theorem to the case where $A$ is the category of stratified simplicial sets endowed with the model structure for $\omega$-complicial sets, and after tedious work, we get
\begin{itheorem}[\ref{theo:letheo}]
Let $n\in \Nb$.
The model structure for $n$-complicial sets is a model of $(\infty,n)$-categories.
\end{itheorem}
As a corollary we have
\begin{itheorem}[\ref{theo:lecorozo}]
The adjunction between the model structure for complete Segal $\Theta$-spaces and $\omega$-complicial set constructed in \cite{Ozornova_a_quillen_adjunction_between_globular_and_complicial} is a Quillen equivalence.
\end{itheorem} 




\section{Preliminaries}

\subsection{Segal $A$-precategories}
Let $A$ be a category of stratified presheaves on a elegant Reedy category (as defined in paragraph \ref{para:reedy} and section \ref{section:Marked and stratified presheaves}), endowed with a nice model structure (as defined in paragraph \ref{para:nice model structure}). We suppose furthermore that the terminal element of $A$, denoted by $e$, is representable. We then have an adjunction 
% https://q.uiver.app/?q=WzAsMixbMCwwLCJcXGlvdGE6XFxTZXQiXSxbMSwwLCJBOm9iIl0sWzAsMSwiIiwwLHsib2Zmc2V0IjotMn1dLFsxLDAsIiIsMCx7Im9mZnNldCI6LTJ9XSxbMiwzLCIiLDAseyJsZXZlbCI6MSwic3R5bGUiOnsibmFtZSI6ImFkanVuY3Rpb24ifX1dXQ==
\begin{equation}
\label{eq:ob adj}
\begin{tikzcd}
	{\iota:\Set} & {A:ob}
	\arrow[""{name=0, anchor=center, inner sep=0}, shift left=2, from=1-1, to=1-2]
	\arrow[""{name=1, anchor=center, inner sep=0}, shift left=2, from=1-2, to=1-1]
	\arrow["\dashv"{anchor=center, rotate=-90}, draw=none, from=0, to=1]
\end{tikzcd}
\end{equation}
where the left adjoint sends a set $S$ onto $\coprod_S e$ and the right adjoint is the evaluation at $e$.
The objects lying in the image of $\iota$ are called \notion{discrete objects}.

An object $C$ of $\Fun( \Delta^{op},A)$ is a \notion{Segal $A$-precatagory} if $C_0$ is discrete. We denote by \wcnotation{$\Seg(A)$}{(seg@$\Seg(A)$} the full subcategory of $\Fun(\Delta^{op},A)$ spanned by the Segal $A$-precategories. 



\p We consider the functor $A\times \Delta \to \Fun(\Delta^{op},A)$ defined by the assignation $a\times [n]\to |[a,n]|$ where $|[a,n]|([m]):=a\times \iota (\Hom_\Delta([m],[n]))$. We define the Segal $A$-precategory \wcsnotation{$[a,n]$}{((g10@$[a,n]$}{for $A$-Segal precategories} as the pushout: 
% https://q.uiver.app/?q=WzAsNCxbMSwwLCJ8W2Esbl18Il0sWzEsMSwiW2Esbl0iXSxbMCwwLCJcXHVuZGVyc2V0e2tcXGxlcSBufXtcXGN1cH17fFthLFxce2tcXH1dfH0iXSxbMCwxLCJ8W2UsMF18Il0sWzIsMF0sWzIsM10sWzMsMV0sWzAsMV0sWzEsMiwiIiwxLHsic3R5bGUiOnsibmFtZSI6ImNvcm5lciJ9fV1d
\[\begin{tikzcd}
	{\underset{k\leq n}{\cup}{|[a,\{k\}]|}} & {|[a,n]|} \\
	{|[e,0]|} & {[a,n]}
	\arrow[from=1-1, to=1-2]
	\arrow[from=1-1, to=2-1]
	\arrow[from=2-1, to=2-2]
	\arrow[from=1-2, to=2-2]
	\arrow["\lrcorner"{anchor=center, pos=0.125, rotate=180}, draw=none, from=2-2, to=1-1]
\end{tikzcd}\]
The object $[e,0]$ is simply denoted by $[0]$. Remark that this object is the terminal Segal $A$-precategory.

The assignation $(a,n)\mapsto [a,n]$ induces by left Kan extension a colimit preserving functor 
$$[\uvar,\uvar]:A\times \Sset \to \Seg(A).$$
The image of this functor is dense in $\Seg(A)$.

For $\{n_i\}_{i\leq k}$ and $\{a\to a_i\}_{i\leq k}$ two finite sequences, we denote by \wcsnotation{$[a_0,n_0]\vee[a_1,n_1]\vee...\vee [a_k,n_k]$}{((g20@$[a_0,n_0]\vee[a_1,n_1]\vee...\vee [a_k,n_k]$}{for Segal $A$-precategories} the Segal $A$-precategory fitting in the following pushout:
% https://q.uiver.app/?q=WzAsNCxbMSwxLCJbYV8wLG5fMF1cXHZlZVthXzEsbl8xXVxcdmVlLi4uW2FfayxuX2tdIl0sWzEsMCwiW2EsXFxTaWdtYV97aVxcbGVxIGt9bl9pXSJdLFswLDAsIlxcYW1hbGdfe2lcXGxlcSBrfVthLG5faV0iXSxbMCwxLCJcXGFtYWxnX3tpXFxsZXEga31bYV9pLG5faV0iXSxbMiwxXSxbMSwwXSxbMiwzXSxbMywwXSxbMCw0LCIiLDIseyJsZXZlbCI6MSwic3R5bGUiOnsibmFtZSI6ImNvcm5lciJ9fV1d
\[\begin{tikzcd}
	{\amalg_{i\leq k}[a,n_i]} & {[a,\Sigma_{i\leq k}n_i]} \\
	{\amalg_{i\leq k}[a_i,n_i]} & {[a_0,n_0]\vee[a_1,n_1]\vee...[a_k,n_k]}
	\arrow[""{name=0, anchor=center, inner sep=0}, from=1-1, to=1-2]
	\arrow[from=1-2, to=2-2]
	\arrow[from=1-1, to=2-1]
	\arrow[from=2-1, to=2-2]
	\arrow["\lrcorner"{anchor=center, pos=0.125, rotate=180}, draw=none, from=2-2, to=0]
\end{tikzcd}\]

The case we will use the most is the one of the Segal $A$-precategories $[e,1]\vee[a,n]$ and $[a,n]\vee[e,1]$ corresponding to the sequence $((1,n),(a\to e,a\to a))$ and $((n,1),(a\to a,a\to e))$.

\p
\label{para:defi of delta[B]}
Let $B$ be the Reedy category and $M$ the subset of objects of $B$ such that $A$ is the category of $M$-stratified presheaves on $B$. We define the category $\Delta[B]$ as the fully faithful subcategory of $\Seg(A)$ whose objects are of shape $[b,n]$ for $b\in B$ and $n$ an integer. Eventually, we define $\Delta[M]$ as the set of objects of shape $[b,n]$ for $b\in M$ and $n>0$. 
We can easily check that the category $\Seg(A)$ is the category of $\Delta[M]$-stratified presheaves on $\Delta[B]$.

A cellular model for $\stratSeg(A)$ is given by the set of morphisms $[b,\partial n]\cup [a,n]\to [b,n]$ for $n$ an integer, and $a\to b$ a generating cofibration of $A$.

Eventually, for any Segal $A$-precategory $C$, we have an isomorphism $$C\cong \colim_{\Delta[tB]_{/C}}[b,n].$$

Following the definition of section \ref{section:Marked and stratified presheaves}, a morphism between Segal precategories is \textit{entire} if it is the identity on the underlying $\Delta[B]$-presheaves.

\begin{prop}
\label{prop:delta[B] is reedy}
The category $\Delta[B]$ as a structure of elegant Reedy category.
\end{prop}
\begin{proof}
Remark first that $\Hom_{\Delta[B]}([a,n],[b,m])$ fits in the following cocartesian square:
% https://q.uiver.app/#q=WzAsNCxbMSwxLCJcXEhvbV97XFxEZWx0YVtCXX0oW2Esbl0sW2IsbV0pIl0sWzEsMCwiXFxIb21fe0J9KGEsYilcXHRpbWVzIFxcSG9tX3tcXERlbHRhfShbbl0sW21dKSJdLFswLDAsIlxcY29wcm9kX3trXFxsZXEgbX1cXEhvbV97Qn0oYSxiKVxcdGltZXMgXFxIb21fe1xcRGVsdGF9KFtuXSxcXHtrXFx9KSJdLFswLDEsIlxcY29wcm9kX3trXFxsZXEgbX0gXFxIb21fe1xcRGVsdGF9KFtuXSxcXHtrXFx9KSJdLFsxLDBdLFsyLDNdLFszLDBdLFsyLDFdXQ==
\[\begin{tikzcd}
	{\coprod_{k\leq m}\Hom_{B}(a,b)\times \Hom_{\Delta}([n],\{k\})} & {\Hom_{B}(a,b)\times \Hom_{\Delta}([n],[m])} \\
	{\coprod_{k\leq m} \Hom_{\Delta}([n],\{k\})} & {\Hom_{\Delta[B]}([a,n],[b,m])}
	\arrow[from=1-2, to=2-2]
	\arrow[from=1-1, to=2-1]
	\arrow[from=2-1, to=2-2]
	\arrow[from=1-1, to=1-2]
\end{tikzcd}\]
We then define the degree functor $ob(\Delta[B])\to \Nb$ by the formula $d([b,n])=d(b)d(n)$. The subcategory $(\Delta[B])_{+}$ is the image of $\Delta_+\times B_+$, and the subcategory $(\Delta[B])_{-}$ is the image of $\Delta_-\times B_-$. 


We recall that we suppose that the Reedy category $B$ is elegant. 
Let $X$ be a presheaf on $\Delta[B]$, $[a,n]$ an element of $\Delta[A]$, $[f,g]:[a,n]\to [a',n']$ and $[h,i]:[a,n]\to [a',n']$ two negative morphisms, an element $x$ of $X([a,n])$, two non degenerate elements $y\in X([a',n'])$ and $z\in X([a'',n''])$ such that $[f,g]^*y=x$, $[h,i]^*z=x$.

We suppose first that $n\neq 0$.
 We denote $\pi:B\times \Delta\to \Delta[B]$ the canonical projection and 
$$\pi^*:\Psh{\Delta[B]}\to \Psh{\Delta\times B}$$ the functor obtained by precomposing.
Remark that for any $a,n$, $(\pi^*X)(a,n)=X([a,n])$.
Furthermore, we  have again equalities $(f,g)^*y=x$, $(h,i)^*z=x$.
As $\Delta\times B$ is Reedy elegant, this implies that $f=h$, $g=i$ and $y=z$. 

If $n=0$, then $[f,g]$ and $[h,i]$ are the identity, and we directly have $y=z$. 
The Reedy category $\Delta[B]$ is then elegant.
\end{proof}





\begin{definition}
\label{defi:generating_acyclic_cofibration}
We define the simplicial set \wcnotation{$E^{\cong}$}{(econg@$E^{\cong}$} as the colimit of the diagram:
$$[e,0]\leftarrow [e,1]\xrightarrow{[e,d^1d^3]} [e,3]\xleftarrow{[e,d^0d^2]} [e,1]\to [e,0].$$
An \snotion{elementary anodyne extension}{for Segal $A$-precategory} is one of the following:
\begin{enumerate}
 \item The \notion{generating Reedy cofibrations}:
 $$[a,n]\cup [b,\partial[n]]\to [b,n],~~\mbox{for $a\to b$ a generating acyclic cofibration of A.}$$
\item The \notion{Segal extensions}: 
$$[a,1]\cup[a,1]\cup ...\cup [a,1]\to [a,n],~~\mbox{for $a$ an object of $A$ and $n>0$.}$$
\item The \notion{completeness extensions}: 
$$\{0\}\to E^{\cong}.$$
\end{enumerate}
\end{definition}

\p 
\label{para:def sega a cat}
A \notion{Segal $A$-category} is a Segal $A$-precategory having the right lifting property against all elementary anodyne extensions.

Let $C$ be a Segal $A$-categories. We define the presheaf $ho(C):\Delta^{op}\to \textbf{Set}$ sending $[n]$ to $\Hom_{ho(A)}(e,C_n)$. As explained in \cite[$\S $ 14.5]{Simpson_Homotopy_theory_of_higher_categories}, this simplicial set has the unique right lifting property against Segal's maps, and is then the nerve of a category that we also note by $ho(C)$.
An arrow $x:[e,1]\to C$ is an \wcnotion{isomorphism}{isomorphism for an arrow $x:[e,1]\to C$} if its image in $ho(C)$ is. 

We can give an other characterization of isomorphisms in Segal $A$-categories. An arrow $x:[e,1]\to C$ is an isomorphism if and only if there exists a lifting in the following diagram: 
% https://q.uiver.app/#q=WzAsMyxbMCwxLCJFXntcXGNvbmd9Il0sWzAsMCwiW2UsMV0iXSxbMSwwLCJDIl0sWzEsMiwieCJdLFsxLDBdLFswLDIsIiIsMCx7InN0eWxlIjp7ImJvZHkiOnsibmFtZSI6ImRhc2hlZCJ9fX1dXQ==
\[\begin{tikzcd}
	{[e,1]} & C \\
	{E^{\cong}}
	\arrow["x", from=1-1, to=1-2]
	\arrow[from=1-1, to=2-1]
	\arrow[dashed, from=2-1, to=1-2]
\end{tikzcd}\]

 A morphism $f:C\to D$ between Segal $A$-categories is an \textit{equivalence of Segal $A$-categories} if $C_1\to D_1$ is a weak equivalence in $A$, and for any element $x\in ob(D)$, there exists $y\in ob(C)$ and an isomorphism in $D$ between $f(y)$ and $x$.



\begin{theorem}[{\cite[21.2.1]{Simpson_Homotopy_theory_of_higher_categories}}]
\label{theo:carlos}
There exists a nice model structure on $\Seg(A)$ where fibrant objects are Segal $A$-categories and weak equivalences between Segal $A$-categories are equivalences of Segal $A$-categories. 

A left adjoint from $\Seg(A)$ to a model category $C$ is a left Quillen functor if it preserves cofibrations, and sends elementary anodyne extensions to weak equivalences.
\end{theorem}

\begin{prop}
Any Segal $A$-precategory is a homotopy colimit of objects of shape $[a,n]$.
\end{prop}
\begin{proof}
Let $C$ be a Segal $A$-precategory. We have $C\cong \colim_{\Delta[tB]_{/C}}\uvar.$
The result then follows from propositions \ref{prop:elelangat stable by slice}, \ref{prop:reedy structure on tB} and \ref{prop:delta[B] is reedy}.
\end{proof}



\subsection{Stratified Segal $A$-precategories}




\p A \notion{stratified Segal $A$-precatagory} is a pair $(C,tC)$ where $tC$ is a subset of $ob(C_1)$ that factors $s^0: C_0\to ob(C_1)$. A \textit{morphism of stratified Segal $A$-precatagory} $(C,tC)\to (D,tD)$ is the data of a morphism $f:C\to D$ such that $f(tC)\subset tD$.
The category of stratified Segal $A$-precategories is denoted by \wcnotation{$\stratSeg(A)$}{(tseg@$\stratSeg(A)$}.

We have an adjunction \ssym{((b30@$(\uvar)^\natural$}{for stratified Segal $A$-precategories}
% https://q.uiver.app/#q=WzAsMixbMCwwLCIoXFx1dmFyKV5cXGZsYXQ6XFxTZWcoQSkiXSxbMSwwLCJcXHN0cmF0U2VnKEEpOihcXHV2YXIpXlxcbmF0dXJhbCJdLFsxLDAsIiIsMCx7Im9mZnNldCI6LTJ9XSxbMCwxLCIiLDAseyJvZmZzZXQiOi0yfV0sWzMsMiwiIiwwLHsibGV2ZWwiOjEsInN0eWxlIjp7Im5hbWUiOiJhZGp1bmN0aW9uIn19XV0=
\begin{equation}
\label{eq:adj u flat}
\begin{tikzcd}
	{(\uvar)^\flat:\Seg(A)} & {\stratSeg(A):(\uvar)^\natural}
	\arrow[""{name=0, anchor=center, inner sep=0}, shift left=2, from=1-2, to=1-1]
	\arrow[""{name=1, anchor=center, inner sep=0}, shift left=2, from=1-1, to=1-2]
	\arrow["\dashv"{anchor=center, rotate=-90}, draw=none, from=1, to=0]
\end{tikzcd}
\end{equation}
where the left adjoint is a fully faithful inclusion that sends $C$ to $C^\flat:=(C,Im(s^0))$. The right adjoint is the obvious forgetful functor.
We will identify Segal $A$-precategories with their images in stratified Segal $A$-precategories under the left adjoint.


\p
We define $[e,1]_t:= ([e,1],[e,1]_1)$. The subcategory of objects of shape $[a,n]$ or $[e,1]_t$ is then dense in $\stratSeg(A)$.


Let $B$ be the Reedy category and $M$ the subset of objects of $B$ such that $A$ is the category of $M$-stratified presheaves on $B$. We recall that we defined the category $\Delta[B]$ and the set of morphism $\Delta[M]$ in paragraph \ref{para:defi of delta[B]}. We set $t\Delta[M]$ as the reunion of $\Delta[M]$ and the singleton $\{[e,1]_t\}$.
We can easily check that the category $\stratSeg(A)$ is the category of $t\Delta[M]$-stratified presheaves on $\Delta[B]$. The set of generating cofibrations for $\stratSeg(A)$ then consists of morphisms of shape $[e,1]\to [e,1]_t$ or $[a,n]\cup[b,\partial n]\to [b,n]$ where $a\to b $ is a generating cofibration of $A$. \sym{((g21@$[e,1]_t$}	
For any stratified Segal $A$-precategory $C$, we then have an isomorphism 
$$C\cong \colim_{t\Delta[tB]_{/C}}\uvar.$$
where $t\Delta[tB]$ is the full subcategory of $\stratSeg(A)$ whose objects are of in $\Delta[B]$ or $t\Delta[M]$.

Following the definition of section \ref{section:Marked and stratified presheaves}, a morphism between stratified Segal precategories is \textit{entire} if it is the identity on the underlying $\Delta[B]$-presheaves.




\p A \notion{marked Segal $A$-category} is a pair $(C,C^{\cong})$ where $C$ is a Segal $A$-category and $C^{\cong}$ is the subset of $ob(C_1)$ consisting of all isomorphisms. A morphism $f:(C,C^{\cong})\to (D,D^{\cong})$ between marked Segal $A$-categories is an \notion{equivalence of marked Segal $A$-categories} if $C_1\to D_1$ is a weak equivalence in $A$, and for any element $x\in ob(D)$, there exists $y\in ob(C)$ and $v:f(y)\to x\in D^{\cong}$.



\p We are now willing to endow $\stratSeg(A)$ with a nice model structure whose fibrant objects are marked Segal $A$-category and weak equivalences between fibrant objects are equivalences of marked Segal $A$-categories. 
We define the stratified Segal $A$-precategories $(E^{\cong})'$ as the following pushout: 
% https://q.uiver.app/?q=WzAsNCxbMSwwLCJFXntcXGNvbmd9Il0sWzEsMSwiKEVee1xcY29uZ30pJyJdLFswLDAsIltlLDFdIl0sWzAsMSwiW2UsMV1fdCJdLFsyLDAsImReMGReMyJdLFszLDFdLFsyLDNdLFswLDFdLFsxLDIsIiIsMCx7InN0eWxlIjp7Im5hbWUiOiJjb3JuZXIifX1dXQ==
\[\begin{tikzcd}
	{[e,1]} & {E^{\cong}} \\
	{[e,1]_t} & {(E^{\cong})'}
	\arrow["{d^0d^3}", from=1-1, to=1-2]
	\arrow[from=2-1, to=2-2]
	\arrow[from=1-1, to=2-1]
	\arrow[from=1-2, to=2-2]
	\arrow["\lrcorner"{anchor=center, pos=0.125, rotate=180}, draw=none, from=2-2, to=1-1]
\end{tikzcd}\]
We define the set of map $J$ as the reunion of the set of generating acyclic cofibration of $\Seg(A)$ and of $\{[e,1]_t\to (E^{\cong})'\}$ and $\{E^{\cong}\to (E^{\cong})'\}$. We suppose furthermore that $J$ includes the acyclic cofibrations $\{0\}\to E^{\cong}$ and $\{1\}\to E^{\cong}$. 





\begin{lemma}
\label{lemma:fib a marked segal cat}
A morphism $f$ has the right lifting property against $J$ if and only if $f^\natural$ is a fibration and  $f$ has the right lifting property against $[e,1]_t\to (E^{\cong})'$ and $E^{\cong}\to (E^{\cong})'$. An object $X$ has the right lifting property against $J$ if and only if it is a marked Segal $A$-category.
\end{lemma}
\begin{proof}
Straightforward.
\end{proof}


\begin{lemma}
\label{lemma:invertivble natural trans are detect pointwise}
Let $i:K\to L$ be a cofibration that induces an isomorphism on objects. 
The morphism 
$$K\times E^{\cong}\coprod_{K\times [e,1]}L\times [e,1]\to L\times E^{\cong}$$
is an acyclic cofibration of the model strucure on $\Seg(A)$.
\end{lemma}
\begin{proof}
By two out of three, and some diagram chasing, is it sufficent to demonstrate the result for $K$ being $L_0$. We then have to show that the square
% https://q.uiver.app/#q=WzAsNCxbMSwwLCJMXFx0aW1lcyBbZSwxXSJdLFsxLDEsIkxcXHRpbWVzIEVee1xcY29uZ30iXSxbMCwxLCJMXzBcXHRpbWVzIEVee1xcY29uZ30iXSxbMCwwLCJMXzBcXHRpbWVzIFtlLDFdIl0sWzIsMV0sWzMsMl0sWzAsMV0sWzMsMF1d
\[\begin{tikzcd}
	{L_0\times [e,1]} & {L\times [e,1]} \\
	{L_0\times E^{\cong}} & {L\times E^{\cong}}
	\arrow[from=2-1, to=2-2]
	\arrow[from=1-1, to=2-1]
	\arrow[from=1-2, to=2-2]
	\arrow[from=1-1, to=1-2]
\end{tikzcd}\]
is homotopy cocoartesian. As the model structure is cartesian, and as $E^{\cong}\to 1$ is a weak equivalence, this is suffisent to show that the following square is homotopy cocartesian:
% https://q.uiver.app/#q=WzAsNCxbMSwwLCJMXFx0aW1lcyBbZSwxXSJdLFsxLDEsIkwiXSxbMCwxLCJMXzAiXSxbMCwwLCJMXzBcXHRpbWVzIFtlLDFdIl0sWzIsMV0sWzMsMl0sWzAsMV0sWzMsMF1d
\[\begin{tikzcd}
	{L_0\times [e,1]} & {L\times [e,1]} \\
	{L_0} & L
	\arrow[from=2-1, to=2-2]
	\arrow[from=1-1, to=2-1]
	\arrow[from=1-2, to=2-2]
	\arrow[from=1-1, to=1-2]
\end{tikzcd}\]
 As $\uvar\times [e,1]$ and $\uvar\times E^{\cong}$ are left Quillen functors, we can reduce to the case where $L$ is $[a,n]$ and using Segal extension, to the case where $L$ is $[a,1]$. We then have to show that the following square is homotopy cocartesian
 % https://q.uiver.app/#q=WzAsNCxbMSwwLCJbYSwxXVxcdGltZXMgW2UsMV0iXSxbMSwxLCJbYSwxXSJdLFswLDEsIlxcezBcXH1cXGN1cFxcezFcXH0iXSxbMCwwLCIoXFx7MFxcfVxcY3VwXFx7MVxcfSlcXHRpbWVzIFtlLDFdIl0sWzIsMV0sWzMsMl0sWzAsMV0sWzMsMF1d
\begin{equation}
\label{eq:cartesian square9870}
\begin{tikzcd}
	{(\{0\}\cup\{1\})\times [e,1]} & {[a,1]\times [e,1]} \\
	{\{0\}\cup\{1\}} & {[a,1]}
	\arrow[from=2-1, to=2-2]
	\arrow[from=1-1, to=2-1]
	\arrow[from=1-2, to=2-2]
	\arrow[from=1-1, to=1-2]
\end{tikzcd}
\end{equation}
 Remark then that $[a,1]\times[e,1]$ is the colimit of the following span:
% https://q.uiver.app/#q=WzAsMyxbMCwwLCJbZSwxXVxcdmVlW2EsMV0iXSxbMSwwLCJbYSwxXSJdLFsyLDAsIlthLDFdXFx2ZWVbZSwxXSJdLFsxLDAsIlthLGReMV0iLDJdLFsxLDIsIlthLGReMV0iXV0=
\[\begin{tikzcd}
	{[e,1]\vee[a,1]} & {[a,1]} & {[a,1]\vee[e,1]}
	\arrow["{[a,d^1]}"', from=1-2, to=1-1]
	\arrow["{[a,d^1]}", from=1-2, to=1-3]
\end{tikzcd}\]
The pushout of the span of \eqref{eq:cartesian square9870} is then the (homotopy) colimit of 
% https://q.uiver.app/#q=WzAsMyxbMCwwLCJbMF1cXGNvcHJvZFxcbGltaXRzX3tbZSwxXX1bZSwxXVxcdmVlW2EsMV0iXSxbMSwwLCJbYSwxXSJdLFsyLDAsIlthLDFdXFx2ZWVbZSwxXVxcY29wcm9kXFxsaW1pdHNfe1tlLDFdfVswXSJdLFsxLDAsIlthLGReMV0iLDJdLFsxLDIsIlthLGReMV0iXV0=
\[\begin{tikzcd}
	{[0]\coprod\limits_{[e,1]}[e,1]\vee[a,1]} & {[a,1]} & {[a,1]\vee[e,1]\coprod\limits_{[e,1]}[0]}
	\arrow["{[a,d^1]}"', from=1-2, to=1-1]
	\arrow["{[a,d^1]}", from=1-2, to=1-3]
\end{tikzcd}\]
By two out of three, and using Segal extensions, the two morphisms 
$$[0]\coprod_{[e,1]}[e,1]\vee[a,1]\to [a,1] ~~~~~\mbox{ and }~~~~~[a,1]\vee[e,1]\coprod_{[e,1]}[0]\to [a,1]$$ induced by $[a,d^0]$ and $[a,d^2]$ are weak equivalences.
In particular, this implies that the canonical morphism from the pushout of the span of \eqref{eq:cartesian square9870} to $[a,1]$ is a weak equivalence. As the upper horizontal vertical morphisms of \eqref{eq:cartesian square9870} is a cofibration, this implies that this square is homotopy cocartesian which concludes the proof.
\end{proof}

\begin{lemma}
\label{lemma:j stable by leibtnis}
Let $i:K\to L$ be a monomorphism and $f:X\to Y$ a morphism having the right lifting property against $J$. The induced morphism 
$$f^i:X^{L}\to X^{K}\times_{Y^{K}} Y^{L}$$ has the right lifting property against $J$.
\end{lemma}
\begin{proof}
As the model structure on $\Seg(A)$ is cartesian, $(f^i)^\natural$ is a fibration. We then have to show that this morphism has the right lifting property against $[e,1]_t\to (E^{\cong})'$ and $E^{\cong}\to (E^{\cong})'$. We can reduce to the case where $i$ is a generating acyclic cofibration. If $i$ is $\emptyset\to [0]$, this is obvious. We then suppose that $i$ is $[e,1]\to [e,1]_t$ or $[a,\partial n]\cup [b,n]\to [b,n]$ for $a\to b$ a generating acyclic cofibration of $A$. In both case, $i$ induces an equivalence on objects. The morphism $i\hat{\times}(E^{\cong}\to (E^{\cong})')$ is then the identity. Moreover, $i\hat{\times}([e,1]_t\to (E^{\cong})')$ fits in the following cocartesian square
% https://q.uiver.app/#q=WzAsNCxbMSwwLCJMXFx0aW1lcyBbZSwxXV90XFxjb3Byb2Rfe0tcXHRpbWVzIFtlLDFdX3R9S1xcdGltZXMgKEVee1xcY29uZ30pJyJdLFsxLDEsIkxcXHRpbWVzIChFXntcXGNvbmd9KSciXSxbMCwwLCJMXlxcbmF0dXJhbFxcdGltZXMgW2UsMV1cXGNvcHJvZF97S15cXG5hdHVyYWxcXHRpbWVzIFtlLDFdfUteXFxuYXR1cmFsXFx0aW1lcyAoRV57XFxjb25nfSkiXSxbMCwxLCJMXlxcbmF0dXJhbFxcdGltZXMgRV57XFxjb25nfSJdLFsyLDNdLFsyLDBdLFszLDFdLFswLDFdXQ==
\[\begin{tikzcd}
	{L^\natural\times [e,1]\coprod_{K^\natural\times [e,1]}K^\natural\times (E^{\cong})} & {L\times [e,1]_t\coprod_{K\times [e,1]_t}K\times (E^{\cong})'} \\
	{L^\natural\times E^{\cong}} & {L\times (E^{\cong})'}
	\arrow[from=1-1, to=2-1]
	\arrow[from=1-1, to=1-2]
	\arrow[from=2-1, to=2-2]
	\arrow[from=1-2, to=2-2]
\end{tikzcd}\]
The lemma \ref{lemma:invertivble natural trans are detect pointwise} implies $f$ has the right lifting property against the left vertical morphism, and so also against the right vertical one.
 By adjunction, this implies that $f^i$ has the desired lifting property.
\end{proof}


\begin{prop}
\label{prop:model structure on stratified Segal category}
There exists a nice model structure on $\stratSeg(A)$ where fibrant objects are stratified Segal $A$-categories and weak equivalences between marked Segal $A$-categories are stratified equivalences. The adjunction 
% https://q.uiver.app/#q=WzAsMixbMCwwLCIoXFx1dmFyKV5cXGZsYXQ6XFxTZWcoQSkiXSxbMSwwLCJcXHN0cmF0U2VnKEEpOihcXHV2YXIpXlxcbmF0dXJhbCJdLFsxLDAsIiIsMCx7Im9mZnNldCI6LTJ9XSxbMCwxLCIiLDAseyJvZmZzZXQiOi0yfV0sWzMsMiwiIiwwLHsibGV2ZWwiOjEsInN0eWxlIjp7Im5hbWUiOiJhZGp1bmN0aW9uIn19XV0=
\[\begin{tikzcd}
	{(\uvar)^\flat:\Seg(A)} & {\stratSeg(A):(\uvar)^\natural}
	\arrow[""{name=0, anchor=center, inner sep=0}, shift left=2, from=1-2, to=1-1]
	\arrow[""{name=1, anchor=center, inner sep=0}, shift left=2, from=1-1, to=1-2]
	\arrow["\dashv"{anchor=center, rotate=-90}, draw=none, from=1, to=0]
\end{tikzcd}\]
induces a Quillen equivalence.


 A left adjoint from $\stratSeg(A)$ to a model category $C$ is a left Quillen functor if it preserves cofibrations, and sends elementary anodyne extensions and morphisms $[e,1]_t\to 1$, $E^{\cong}\to (E^{\cong})'$ to weak equivalences.
\end{prop}
\begin{proof}
We recall that we define $J$ as the reunion of the set of generating acyclic cofibrations of $\Seg(A)$ and of $\{[e,1]_t\to (E^{\cong})'\}$ and $\{E^{\cong}\to (E^{\cong})'\}$ and we suppose that it includes the trivial cofibrations $\{0\}\to E^{\cong}$ and $\{1\}\to E^{\cong}$. We denote $I$ a cellular model for $\Psh{t\Delta[tB]}$.


As $\stratSeg(A)$ is the category of $t\Delta[M]$ stratified presheaves on $\Delta[B]$, we have an adjunction
% https://q.uiver.app/#q=WzAsMixbMSwwLCJcXHN0cmF0U2VnKEEpOlxcaW90YSJdLFswLDAsIlxccGk6XFxQc2h7dFxcRGVsdGFbdEJdfSJdLFsxLDAsIiIsMCx7Im9mZnNldCI6LTJ9XSxbMCwxLCIiLDAseyJvZmZzZXQiOi0yfV0sWzIsMywiIiwwLHsibGV2ZWwiOjEsInN0eWxlIjp7Im5hbWUiOiJhZGp1bmN0aW9uIn19XV0=
\[\begin{tikzcd}
	{\pi:\Psh{t\Delta[tB]}} & {\stratSeg(A):\iota}
	\arrow[""{name=0, anchor=center, inner sep=0}, shift left=2, from=1-1, to=1-2]
	\arrow[""{name=1, anchor=center, inner sep=0}, shift left=2, from=1-2, to=1-1]
	\arrow["\dashv"{anchor=center, rotate=-90}, draw=none, from=0, to=1]
\end{tikzcd}\]
where the right adjoint is fully faithfull. 


The set $l(r(\iota(\J)\hat{\times}I))$ is a class of anodyne extension relative to the interval $\uvar\times E^{\cong}$ as defined in \cite[paragraph 1.3.12]{cisinski_prefaisceaux_comme_modele}. We then consider $\Psh{t\Delta[tB]}$ endowed with the model structure induced by \cite[théorème 1.3.22]{cisinski_prefaisceaux_comme_modele}. An object is fibrant if and only if it has the right lifting property against $\iota(\J)\hat{\times}I$. A morphism between fibrant objects is a fibration if and only if it has the right lifting property against $\iota(\J)\hat{\times}I$.


According to proposition \ref{prop:transfert from presheaves on tB to stratified presheaves}, this induces a model structure on $\stratSeg(A)$. By adjunction and using lemma \ref{lemma:j stable by leibtnis}, an object is fibrant if and only if it has the right lifting property against $J$ and a morphism between fibrant objects is a fibration if and only if it has the right lifting property against $J$. According to lemma \ref{lemma:fib a marked segal cat}, the fibrant objects correspond to marked Segal $A$-categories.

The theorem \ref{theo:carlos} implies that the adjunction \eqref{eq:adj u flat} is a Quillen adjunction.
It's unit is the identity, and lemma \ref{lemma:fib a marked segal cat} implies that the counit, computed on a fibrant object $(C,C^{\cong})$, is the canonical inclusion $(C,C^\flat)\to (C,C^{\cong})$. As this morphism is a transfinite composite of $E^{\cong}\to (E^{\cong})'$, it is a weak equivalence. The Quillen pair \ref{lemma:fib a marked segal cat} is then a Quillen equivalence.
As a consequence, the model structure on $\stratSeg(A)$ is cartesian and simplicial, and weak equivalences between fibrant objects are stratified equivalences. 


It then remains to prove the last assertion. Suppose given a left adjoint $F:\stratSeg(A)\to C$ that preserves cofibrations, and sends elementary anodyne extensions and morphisms $[e,1]_t\to 1$, $E^{\cong}\to (E^{\cong})'$ to weak equivalences. The theorem \ref{theo:carlos} implies that the restriction of $F$ to $\Seg(A)$ is a left Quillen functor, and this functors then sends any acyclic cofibration of $\Seg(A)$ to a weak equivalence. 
As we have a commutative diagram,
% https://q.uiver.app/#q=WzAsNCxbMSwwLCJbMV1fdCJdLFswLDAsIihFXntcXGNvbmd9KSciXSxbMCwxLCJFXntcXGNvbmd9Il0sWzEsMSwiWzBdIl0sWzIsMV0sWzAsMV0sWzIsM10sWzAsM10sWzEsM11d
\[\begin{tikzcd}
	{(E^{\cong})'} & {[1]_t} \\
	{E^{\cong}} & {[0]}
	\arrow[from=2-1, to=1-1]
	\arrow[from=1-2, to=1-1]
	\arrow[from=2-1, to=2-2]
	\arrow[from=1-2, to=2-2]
	\arrow[from=1-1, to=2-2]
\end{tikzcd}\]
we deduce by two out of three that $F$ sends $[1]_t\to (E^{\cong})'$ to a weak equivalence. The functor $F$ then sends any morphism of $J$ to a weak equivalence. 

As fibrant objects and fibrations between fibrant objects are detected by right lifting property against $J$, the right adjoint of $F$ preserves them. The corollary A.2 of \cite{Dugger_Replacing_model_categories_with_simplicial_ones} implies that $F$ is a left Quillen functor.
\end{proof}




\begin{prop}
Any stratified Segal $A$-precategory is a homotopy colimit of objects of shape $[a,n]$ or $[e,1]_t$.
\end{prop}
\begin{proof}
Let $C$ be a stratified Segal $A$-precategory. We have 
$C\cong \colim_{t\Delta[tB]_{/C}}\uvar.$
The result then follows from propositions \ref{prop:elelangat stable by slice}, \ref{prop:reedy structure on tB} and \ref{prop:delta[B] is reedy}.
\end{proof}

\p
We now present the main way of constructing functors whose codomain is $\stratSeg(A)$. 
\begin{construction}
\label{cons:lifting of a functor from A times Delta}
Suppose given a colimit preserving functor $G:A\times \Delta\to D$ in a complete category, an object $G(e,1)'$ and a morphism $p:G(e,1)\to G(e,1)'$ such that for any object $d$ of $D$, $\Hom(p,d)$ is a monomorphism. We define the functor 
$\overline{G}:\stratSeg(A)\to D$
as the unique colimit preserving functor such that $\overline{G}([e,1]_t):= G(e,1)'$ and for any $a,n$, $\overline{G}([a,n])$ fits in the following cocartesian square: 
% https://q.uiver.app/?q=WzAsNCxbMSwxLCJcXG92ZXJsaW5le0d9KFthLG5dKSJdLFswLDAsIlxcY29wcm9kX3tpXFxpbltuXX0gRyhhLFxce2lcXH0pIl0sWzEsMCwiRyhhLFtuXSkiXSxbMCwxLCJcXGNvcHJvZF97aVxcaW5bbl19IEcoZSxcXHtpXFx9KSJdLFsxLDJdLFsxLDNdLFszLDBdLFsyLDBdXQ==
\[\begin{tikzcd}
	{\coprod_{i\in[n]} G(a,\{i\})} & {G(a,[n])} \\
	{\coprod_{i\in[n]} G(e,\{i\})} & {\overline{G}([a,n])}
	\arrow[from=1-1, to=1-2]
	\arrow[from=1-1, to=2-1]
	\arrow[from=2-1, to=2-2]
	\arrow[from=1-2, to=2-2]
\end{tikzcd}\]
Remark that if the top horizontal morphism is a cofibration, the previous square is homotopy cocartesian.
\end{construction}



\p In this model structure, the morphism $[e,1]_t\to 1$ is a weak equivalence. For any $a\in A$ and $n\in \Nb$, we define $[e,1]_t\vee[a,n]$ as the pushout:
% https://q.uiver.app/#q=WzAsNCxbMSwwLCJbZSwxXVxcdmVlW2Esbl0iXSxbMCwwLCJbZSwxXSJdLFswLDEsIltlLDFdX3QiXSxbMSwxLCJbZSwxXV90XFx2ZWVbYSxuXSJdLFsxLDJdLFsyLDNdLFsxLDBdLFswLDNdXQ==
\[\begin{tikzcd}
	{[e,1]} & {[e,1]\vee[a,n]} \\
	{[e,1]_t} & {[e,1]_t\vee[a,n]}
	\arrow[from=1-1, to=2-1]
	\arrow[from=2-1, to=2-2]
	\arrow[from=1-1, to=1-2]
	\arrow[from=1-2, to=2-2]
\end{tikzcd}\]
The canonical morphism $[e,1]_t\cup [a,1]\cup...\cup [a,1]\to [e,1]_t\vee[a,n]$ is then a weak equivalence. By two out of three, and using the weak equivalence $[e,1]_t\to 1$, this implies that $[e,1]_t\vee[a,n]\to [a,n]$ is a weak equivalence. \sym{((g22@$[e,1]_t\vee[a,n]$}	


We define similarly the object $[a,n]\vee[e,1]_t$ that comes along with a weak equivalence $[a,n]\vee[e,1]_t\to [a,n]$.


\subsection{Gray module}
\p
Let $A$ be a category of stratified presheaves on an elegant Reedy category (as defined in paragraph \ref{para:reedy} and section \ref{section:Marked and stratified presheaves}), endowed with a nice model structure (as defined in paragraph \ref{para:nice model structure}). We suppose furthermore that the terminal element of $A$, denoted by $e$, is representable. We also suppose that $A$ is endowed with \textit{intelligent $n$-truncation} for any $n\in \mathbb{N}\cup\{\omega\}$, i.e a family of left Quillen functors $\tau^i_{\uvar}:(\mathbb{N}\cup\{\omega\})^{op}\to \End(A)$ such that
\begin{enumerate}[itemsep=0mm]
\item[$-$] $\tau^i_\omega = id$,
\item[$-$] for any $n\leq m$, $\tau^i_n\tau^i_m=\tau^i_n$,
\item[$-$] for any $n\leq m$, the natural transformation $\tau^i_m\to \tau^i_n$ is an entire monomorphism,
\end{enumerate}
and a left Quillen bifunctor $\uvar\otimes \uvar: \stratSset^1\times A \to A$ such that 
\begin{enumerate}
\item[$-$] for $K$ and $L$ two stratified simplicial sets, and $a\in A$, there is a morphism $K\otimes (L\otimes a)\to (K\times L)\otimes a$ natural in $K,L$ and $a$, such that the following square commutes
% https://q.uiver.app/#q=WzAsNCxbMSwwLCIoS1xcdGltZXMgTClcXG90aW1lcyAoTVxcb3RpbWVzIGEpIl0sWzAsMCwiS1xcb3RpbWVzIChMXFxvdGltZXMgKE1cXG90aW1lcyBhKSkiXSxbMSwxLCIoS1xcdGltZXMgTFxcdGltZXMgTSkgXFxvdGltZXMgYSJdLFswLDEsIktcXG90aW1lcyAoKExcXHRpbWVzIE0pXFxvdGltZXMgYSkiXSxbMSwwXSxbMCwyXSxbMywyXSxbMSwzXV0=
\[\begin{tikzcd}
	{K\otimes (L\otimes (M\otimes a))} & {(K\times L)\otimes (M\otimes a)} \\
	{K\otimes ((L\times M)\otimes a)} & {(K\times L\times M) \otimes a}
	\arrow[from=1-1, to=1-2]
	\arrow[from=1-2, to=2-2]
	\arrow[from=2-1, to=2-2]
	\arrow[from=1-1, to=2-1]
\end{tikzcd}\]
for any stratified simplicial sets $M$.
\item[$-$] The functor $[0]\otimes \uvar: A\to A$ is the identity.
\item[$-$] For any integer $n$, for any object $a$ invariant under $\tau^i_n$, and for any stratified simplicial set $K$, the object $K\otimes a$ is invariant under $\tau^i_{n+1}$.
\end{enumerate}
Here, the model category $\stratSset^1$ corresponds to the model structure for $1$-complicial sets on stratified simplicial sets given in theorem \ref{theo:model structure on complicial set}.


\p
We define $e\star a$ as the pushout:
% https://q.uiver.app/?q=WzAsNCxbMSwwLCJbMV1cXG90aW1lcyBhIl0sWzAsMCwiXFx7MFxcfVxcb3RpbWVzIGEiXSxbMCwxLCJlIl0sWzEsMSwiZVxcc3RhciBhIl0sWzEsMl0sWzEsMF0sWzAsM10sWzIsM10sWzMsMSwiIiwxLHsic3R5bGUiOnsibmFtZSI6ImNvcm5lciJ9fV1d
\[\begin{tikzcd}
	{\{0\}\times a} & {[1]\otimes a} \\
	e & {e\star a}
	\arrow[from=1-1, to=2-1]
	\arrow[from=1-1, to=1-2]
	\arrow[from=1-2, to=2-2]
	\arrow[from=2-1, to=2-2]
	\arrow["\lrcorner"{anchor=center, pos=0.125, rotate=180}, draw=none, from=2-2, to=1-1]
\end{tikzcd}\]
We consider the natural transformations $s^0\star a:e\star e\star a\to e\star a$ and $d^0\star a:a\to e\star a$,
induced respectively by the morphism
$$
\begin{array}{cclcccccccc}
[1]\otimes [1]\otimes a&\to & ([1]\times [1])\otimes a &\to & [1]\otimes a\\
&&(\{i\}\times \{j\})\otimes a&\mapsto & \{i\wedge j\}\otimes a.
\end{array}$$
and the morphism 
$$\{1\}\otimes a \to [1]\otimes a.$$
These natural transformations induce commutative diagrams:
% https://q.uiver.app/#q=WzAsOCxbMCwwLCJlXFxzdGFyIGVcXHN0YXIgZVxcc3RhciBhICJdLFswLDEsIiBlXFxzdGFyIGVcXHN0YXIgYSAiXSxbMSwwLCIgZVxcc3RhciBlXFxzdGFyIGEgIl0sWzEsMSwiZVxcc3RhciBhIl0sWzMsMCwiZVxcc3RhciBhIl0sWzQsMCwiIGVcXHN0YXIgZVxcc3RhciBhICJdLFs0LDEsImVcXHN0YXIgYSJdLFs1LDAsImVcXHN0YXIgYSJdLFsyLDMsInNeMFxcc3RhciBhIl0sWzEsMywic14wXFxzdGFyIGEiLDJdLFswLDIsInNeMFxcc3RhciAoZVxcc3RhciBhKSJdLFswLDEsImVcXHN0YXIgKHNeMFxcc3RhciBhKSIsMl0sWzQsNSwiZVxcc3RhciBkXjAiXSxbNSw3LCJkXjBcXHN0YXIgKGVcXHN0YXIgYSkiXSxbNSw2LCJzXjBcXHN0YXIgYSJdLFs3LDYsImlkIiwwLHsiY3VydmUiOi0xfV0sWzQsNiwiaWQiLDIseyJjdXJ2ZSI6MX1dXQ==
\[\begin{tikzcd}
	{e\star e\star e\star a } & { e\star e\star a } && {e\star a} & { e\star e\star a } & {e\star a} \\
	{ e\star e\star a } & {e\star a} &&& {e\star a}
	\arrow["{s^0\star a}", from=1-2, to=2-2]
	\arrow["{s^0\star a}"', from=2-1, to=2-2]
	\arrow["{s^0\star (e\star a)}", from=1-1, to=1-2]
	\arrow["{e\star (s^0\star a)}"', from=1-1, to=2-1]
	\arrow["{e\star d^0}", from=1-4, to=1-5]
	\arrow["{d^0\star (e\star a)}", from=1-5, to=1-6]
	\arrow["{s^0\star a}", from=1-5, to=2-5]
	\arrow["id", curve={height=-6pt}, from=1-6, to=2-5]
	\arrow["id"', curve={height=6pt}, from=1-4, to=2-5]
\end{tikzcd}\]
The (inverted) composition $g,f\mapsto g\circ f$ is a monoidal structure on the category of endomorphisms of $A$ and the natural transformation $s^0:e\star e \star\uvar \to e\star \uvar$ defines a structure of monoid for $e\star\uvar$.
This induces a functor $\Delta\times A\to A$ sending $([n],a)$ to $e\star e\star ....\star a$. We extend this to a functor $\Delta_t\times A\to A$ in defining $[n]_t\star a$ as the pushout:
% https://q.uiver.app/?q=WzAsNCxbMSwxLCJbbl1fdFxcc3RhciBhIl0sWzEsMCwiW25dXFxzdGFyIGEiXSxbMCwwLCJcXHVuZGVyc2V0e2tcXGdlcSAtMX17XFxjb3Byb2R9fn5cXHVuZGVyc2V0e2IsflxcdGF1XmlfayhiKT1ifXtcXGNvcHJvZH1+flxcdW5kZXJzZXR7YlxcdG8gYX17XFxjb3Byb2R9W25dXFxzdGFyICBiIl0sWzAsMSwiXFx1bmRlcnNldHtrXFxnZXEgLTF9e1xcY29wcm9kfX5+XFx1bmRlcnNldHtiLH5cXHRhdV5pX2soYik9Yn17XFxjb3Byb2R9fn5cXHVuZGVyc2V0e2JcXHRvIGF9e1xcY29wcm9kfVxcdGF1Xmlfe24ra30oW25dXFxzdGFyIGIpIl0sWzMsMF0sWzIsMV0sWzEsMF0sWzIsM10sWzAsNSwiIiwxLHsibGV2ZWwiOjEsInN0eWxlIjp7Im5hbWUiOiJjb3JuZXIifX1dXQ==
\[\begin{tikzcd}
	{\underset{k\geq -1}{\coprod}~~\underset{b,~\tau^i_k(b)=b}{\coprod}~~\underset{b\to a}{\coprod}[n]\star b} & {[n]\star a} \\
	{\underset{k\geq -1}{\coprod}~~\underset{b,~\tau^i_k(b)=b}{\coprod}~~\underset{b\to a}{\coprod}\tau^i_{n+k}([n]\star b)} & {[n]_t\star a}
	\arrow[from=2-1, to=2-2]
	\arrow[""{name=0, anchor=center, inner sep=0}, from=1-1, to=1-2]
	\arrow[from=1-2, to=2-2]
	\arrow[from=1-1, to=2-1]
	\arrow["\lrcorner"{anchor=center, pos=0.125, rotate=180}, draw=none, from=2-2, to=0]
\end{tikzcd}\]
where $\tau^i_{-1}$ is the constant functor with value $\emptyset$.

\p 
\label{para:Gray module}
Such model category $A$ is a \notion{Gray module} if for any $a$, the induced functor $\uvar\star a:\Delta_t\to A_{a/}$ lifts to a left Quillen functor $\uvar\star a:\stratSset^\omega\to A_{a/}$.

We recall that $\stratSset^\omega$ denotes the model structure for $\omega$-complicial sets given in theorem \ref{theo:model structure on complicial set}.

For the rest of this chapter, we fix a Gray module $A$. For a stratified simplicial set $K\in \stratSset$, the object $K\star \emptyset\in A$ is simply noted by $K$.

 

\begin{remark}
In general, $[n]\otimes e$ and $[n]\star \emptyset$ are two very different objects. Indeed $[n]\otimes e$ has to be invariant up to homotopy under $\tau^i_1$ which is not the case for $[n]\star \emptyset$. Analogously $[k]\otimes ([l]\otimes [a])$ and $([k]\otimes [l])\otimes [a]$ have \textit{a priori} no links. When we write $[n_0]\otimes[n_1]\otimes..[n_k]\otimes a$, we will always mean $[n_0]\otimes([n_1]\otimes..([n_k]\otimes a))$.
 \end{remark}



\begin{example}
\label{example:stratsset is gray module}
For any $d\in \mathbb{N}\cup \{\omega\}$,
the model category $\stratSset^d$, corresponding to the model structure for $d$-complicial sets on stratified simplicial sets, and where $K\otimes L := \tau^i_1(K)\boxtimes L$, is an example of Gray module.

Indeed, if $n$ is any integer, we define $[n]^{\diamond}:=[0]\diamond [0]\diamond...\diamond [0]$ and $[n]_t^{\diamond}:= \tau^i_n([n]^{\diamond})$. This induces a colimit preserving functor $K\mapsto K^{\diamond}$. The join coming from $\tau^i_1(\uvar)\boxtimes \uvar$ then corresponds to the functor $(K,L)\mapsto K^\diamond\diamond L$. The proposition \ref{prop:equivalence between diamond and join product} provides a natural transformation $K^{\diamond}\diamond L\to K\star L$, wich implies that the first functor is left Quillen. 
\end{example}





\section{Gray constructions for stratified Segal $A$-categories}
 We now construct a Gray cylinder and a Gray cone on $\stratSeg(A)$, using the structure of Gray module that $A$ has. We denote by $\Delta_+$ the augmented simplex category and $d^0$ the unique morphism $\emptyset\to [0]$.
\subsection{Gray cylinder}
\p
We define the functor 
$$
\begin{array}{ccl}
\Delta^3\times A &\to& \Seg(A)\\
~[n_0],[n_1],[n_2],a&\mapsto &[a,n_0]\vee[[n_1]\otimes a,1]\vee[a,n_2]
\end{array}$$
where $[a,n_0]\vee[[n_1]\otimes a,1]\vee[a,n_2]$ fits in the following pushout:
% https://q.uiver.app/?q=WzAsNCxbMSwxLCJbYSxuXzBdXFx2ZWVbW25fMV1cXG90aW1lcyBhLDFdXFx2ZWVbYSxuXzJdIl0sWzAsMCwiW1tuXzFdXFxvdGltZXMgYSxuXzBdXFxhbWFsZyBbW25fMV1cXG90aW1lcyBhLG5fMl0iXSxbMSwwLCJbW25fMV1cXG90aW1lcyBhLG5fMCsxK25fMl0iXSxbMCwxLCJbWzBdXFxvdGltZXMgYSxuXzBdXFxhbWFsZyBbWzBdXFxvdGltZXMgYSxuXzJdIl0sWzEsM10sWzIsMF0sWzMsMF0sWzEsMl0sWzAsNywiIiwxLHsibGV2ZWwiOjEsInN0eWxlIjp7Im5hbWUiOiJjb3JuZXIifX1dXQ==
\[\begin{tikzcd}
	{[[n_1]\otimes a,n_0]\amalg [[n_1]\otimes a,n_2]} & {[[n_1]\otimes a,n_0+1+n_2]} \\
	{[[0]\otimes a,n_0]\amalg [[0]\otimes a,n_2]} & {[a,n_0]\vee[[n_1]\otimes a,1]\vee[a,n_2]}
	\arrow[from=1-1, to=2-1]
	\arrow[from=1-2, to=2-2]
	\arrow[from=2-1, to=2-2]
	\arrow[""{name=0, anchor=center, inner sep=0}, from=1-1, to=1-2]
	\arrow["\lrcorner"{anchor=center, pos=0.125, rotate=180}, draw=none, from=2-2, to=0]
\end{tikzcd}\]
If $n$ is an integer, \wcnotation{$\Delta^3_{/[n]}$}{(delta3@$\Delta^3_{/[n]}$} is the pullback: 
% https://q.uiver.app/?q=WzAsNCxbMSwwLCJcXERlbHRhXjMiXSxbMSwxLCJcXERlbHRhIl0sWzAsMSwiXFxEZWx0YV97L1tuXX0iXSxbMCwwLCJcXERlbHRhXjNfey9bbl19Il0sWzMsMF0sWzAsMV0sWzIsMV0sWzMsMl0sWzMsMSwiIiwxLHsic3R5bGUiOnsibmFtZSI6ImNvcm5lciJ9fV1d
\[\begin{tikzcd}
	{\Delta^3_{/[n]}} & {\Delta^3} \\
	{\Delta_{/[n]}} & \Delta
	\arrow[from=1-1, to=1-2]
	\arrow[from=1-2, to=2-2]
	\arrow[from=2-1, to=2-2]
	\arrow[from=1-1, to=2-1]
	\arrow["\lrcorner"{anchor=center, pos=0.125}, draw=none, from=1-1, to=2-2]
\end{tikzcd}\]
where the right hand functor sends $([n_0],[n_1],[n_2])$ to $[n_0]\star [n_1]^{op}\star [n_2]$.
\begin{prop}
\label{prop:delta 3 n is reedy elegant}
The category $\Delta^3_{/[n]}$ is an elegant Reedy category. 
\end{prop}
\begin{proof}
We denote $X$ the trisimplicial set whose value on $[n_0],[n_1],[n_2]$ is $\Hom_{\Delta}([n_0]\star [n_1]^{op}\star [n_2],[n])$. The category $\Delta^3_{/[n]}$ fits in the pullback
% https://q.uiver.app/#q=WzAsNCxbMCwwLCJcXERlbHRhXjNfey9bbl19Il0sWzEsMCwiXFxQc2h7XFxEZWx0YV4zfV97L1h9Il0sWzEsMSwiXFxQc2h7XFxEZWx0YV4zfSJdLFswLDEsIlxcRGVsdGFeMyJdLFsxLDJdLFswLDNdLFszLDJdLFswLDFdXQ==
\[\begin{tikzcd}
	{\Delta^3_{/[n]}} & {\Psh{\Delta^3}_{/X}} \\
	{\Delta^3} & {\Psh{\Delta^3}}
	\arrow[from=1-2, to=2-2]
	\arrow[from=1-1, to=2-1]
	\arrow[from=2-1, to=2-2]
	\arrow[from=1-1, to=1-2]
\end{tikzcd}\]
and is then an elegant Reedy category according to proposition \ref{prop:elelangat stable by slice}.
\end{proof}


\p
We define the functor 
$$
\begin{array}{rcl}
A\times \Delta&\to& \Seg(A)\\
~[n],a&\mapsto &F(a,n) 
\end{array}$$
by the formula 
$ F(a,n) :=\underset{\Delta^3_{/[n]}}{\colim}~[a,n_0]\vee[[n_1]\otimes a,1]\vee [a,n_2].$

In order to extend this functor to stratified Segal $A$-precategories with construction \ref{cons:lifting of a functor from A times Delta}, we will need to define the value on $[e,1]_t$, i.e. to choose an object $F(e,1)'$ and an entire cofibration $F(e,1)\to F(e,1)'$. It will be useful to have a more explicit description of this object.
\begin{example}
\label{exe:explicit Gray cycinder 1}
The sub-category of $\Delta^3_{/[1]}$ composed of non degenerate objects can be pictured by the graph:
% https://q.uiver.app/?q=WzAsOCxbMiwxLCJbMV1cXHN0YXJbMF1ee29wfVxcc3RhclswXSJdLFsyLDIsIlswXVxcc3RhclswXV57b3B9XFxzdGFyWzBdIl0sWzAsMiwiWzBdXFxzdGFyWzBdXntvcH1cXHN0YXJbMF0iXSxbMCwxLCJbMF1cXHN0YXJbMF1ee29wfVxcc3RhclsxXSJdLFswLDAsIlswXVxcc3RhclswXV57b3B9XFxzdGFyWzBdIl0sWzIsMCwiWzBdXFxzdGFyWzBdXntvcH1cXHN0YXJbMF0iXSxbMSwwLCJbMF1cXHN0YXJbMV1ee29wfVxcc3RhclswXSJdLFsxLDEsIlsxXSJdLFsxLDAsImReMCIsMl0sWzIsMywiZF4zIl0sWzQsMywiZF4yIiwyXSxbNSw2LCJkXjEiLDJdLFs0LDYsImReMiJdLFs2LDcsInNeMHNeMiIsMV0sWzUsMCwiZF4xIl0sWzUsNywic14xIiwxXSxbMSw3LCJkXjBzXjBzXjEiXSxbMCw3LCJzXjFzXjIiLDFdLFs0LDcsInNeMCIsMV0sWzMsNywic14wc14wIiwxXSxbMiw3LCJkXjFzXjBzXjEiLDJdXQ==
\[\begin{tikzcd}
	{[0]\star[0]^{op}\star[0]} & {[0]\star[1]^{op}\star[0]} & {[0]\star[0]^{op}\star[0]} \\
	{[0]\star[0]^{op}\star[1]} & {[1]} & {[1]\star[0]^{op}\star[0]} \\
	{[0]\star[0]^{op}\star[0]} && {[0]\star[0]^{op}\star[0]}
	\arrow["{d^0}"', from=3-3, to=2-3]
	\arrow["{d^3}", from=3-1, to=2-1]
	\arrow["{d^2}"', from=1-1, to=2-1]
	\arrow["{d^1}"', from=1-3, to=1-2]
	\arrow["{d^2}", from=1-1, to=1-2]
	\arrow["{s^0s^2}"{description}, from=1-2, to=2-2]
	\arrow["{d^1}", from=1-3, to=2-3]
	\arrow["{s^1}"{description}, from=1-3, to=2-2]
	\arrow["{d^0s^0s^1}", from=3-3, to=2-2]
	\arrow["{s^1s^2}"{description}, from=2-3, to=2-2]
	\arrow["{s^0}"{description}, from=1-1, to=2-2]
	\arrow["{s^0s^0}"{description}, from=2-1, to=2-2]
	\arrow["{d^1s^0s^1}"', from=3-1, to=2-2]
\end{tikzcd}\]
The Segal $A$-precategory $F(e,1)$ is then the colimit of the following diagram:
% https://q.uiver.app/#q=WzAsNSxbNCwwLCJbZSwyXSJdLFswLDAsIltlLDJdIl0sWzEsMCwiW2UsMV0iXSxbMywwLCJbZSwxXSJdLFsyLDAsIltbMV0sMV0iXSxbMiwxLCJbZSxkXjFdIiwyXSxbMyw0LCJbZF4xLDFdIiwyXSxbMiw0LCJbZF4wLDFdIl0sWzMsMCwiW2UsZF4xXSJdXQ==
\[\begin{tikzcd}
	{[e,2]} & {[e,1]} & {[[1],1]} & {[e,1]} & {[e,2]}
	\arrow["{[e,d^1]}"', from=1-2, to=1-1]
	\arrow["{[d^1,1]}"', from=1-4, to=1-3]
	\arrow["{[d^0,1]}", from=1-2, to=1-3]
	\arrow["{[e,d^1]}", from=1-4, to=1-5]
\end{tikzcd}\]
\end{example}
\p \sym{(iotimes@$I\otimes\uvar$}
We define the functor $$ I \otimes\uvar: \stratSeg(A)\to \stratSeg(A)$$ induced, as in the construction \ref{cons:lifting of a functor from A times Delta}, by $F$ and with $F(e,1)'$ as the colimit of the following diagram:
% https://q.uiver.app/?q=WzAsOSxbNiwwLCJbZSwyXSJdLFs3LDAsIltlLDFdIl0sWzEsMCwiW2UsMV0iXSxbMiwwLCJbZSwyXSJdLFszLDAsIltlLDFdIl0sWzUsMCwiW2UsMV0iXSxbNCwwLCJbWzFdX3QsMV0iXSxbMCwwLCJbZSwxXV90Il0sWzgsMCwiW2UsMV1fdCJdLFsxLDAsIltlLGReMF0iLDJdLFsyLDMsIltlLGReMl0iXSxbNCwzLCJbZSxkXjFdIiwyXSxbNSw2LCJbZF4xLDFdIiwyXSxbNCw2LCJbZF4wLDFdIl0sWzUsMCwiW2UsZF4xXSJdLFsyLDddLFsxLDhdXQ==
\[\begin{tikzcd}[sep = small]
	{[e,1]_t} & {[e,1]} & {[e,2]} & {[e,1]} & {[[1]_t,1]} & {[e,1]} & {[e,2]} & {[e,1]} & {[e,1]_t}
	\arrow["{[e,d^0]}"', from=1-8, to=1-7]
	\arrow["{[e,d^2]}", from=1-2, to=1-3]
	\arrow["{[e,d^1]}"', from=1-4, to=1-3]
	\arrow["{[d^1,1]}"', from=1-6, to=1-5]
	\arrow["{[d^0,1]}", from=1-4, to=1-5]
	\arrow["{[e,d^1]}", from=1-6, to=1-7]
	\arrow[from=1-2, to=1-1]
	\arrow[from=1-8, to=1-9]
\end{tikzcd}\]


The two objects of $\Delta^3_{[n]}$, $s^ns^{n+1}:[n]\star [0]^{op}\star[0]\to [n]$ and $s^0s^0:[0]\star[0]^{op}\star[n]\to [n]$, induce two morphisms:
$d^1\otimes [a,n]:\{0\}\otimes[a,n]:= [a,n]\hookrightarrow [a,n]\vee[e,1]\to I\otimes [a,n]$ and $d^0\otimes [a,n]:\{1\}\otimes[a,n]:= [a,n]\hookrightarrow [e,1]\vee[a,n]\to I\otimes [a,n]$. By extending them by colimits we get two maps 
$$d^1\otimes C:\{0\}\otimes C := C\to I\otimes C~~~\mbox{and}~~~
d^0\otimes C: \{1\}\otimes C := C\to I\otimes C.$$



\begin{prop}
The Segal $A$-precategory $I \otimes[a,1]$ is the colimit and the homotopy colimit of the diagram:
% https://q.uiver.app/#q=WzAsNSxbNCwwLCJbYSwxXVxcdmVlW2UsMV0iXSxbMCwwLCJbZSwxXVxcdmVlW2EsMV0iXSxbMSwwLCJbYSwxXSJdLFszLDAsIlthLDFdIl0sWzIsMCwiW1sxXVxcb3RpbWVzIGEsMV0iXSxbMiwxLCJbZSxkXjFdIiwyXSxbMyw0LCJbZF4xLDFdIiwyXSxbMiw0LCJbZF4wLDFdIl0sWzMsMCwiW2UsZF4xXSJdXQ==
\[\begin{tikzcd}
	{[e,1]\vee[a,1]} & {[a,1]} & {[[1]\otimes a,1]} & {[a,1]} & {[a,1]\vee[e,1]}
	\arrow["{[e,d^1]}"', from=1-2, to=1-1]
	\arrow["{[d^1,1]}"', from=1-4, to=1-3]
	\arrow["{[d^0,1]}", from=1-2, to=1-3]
	\arrow["{[e,d^1]}", from=1-4, to=1-5]
\end{tikzcd}\]
\end{prop}
\begin{proof}
 The description of $\Delta^{3}_{/[1]}$ is given in the example \ref{exe:explicit Gray cycinder 1}.
The stratified Segal $A$-precategory $F(a,1)$ is then the colimit of the following diagram:
% https://q.uiver.app/#q=WzAsNSxbNCwwLCJbYSwyXSJdLFswLDAsIlthLDJdIl0sWzEsMCwiW2EsMV0iXSxbMywwLCJbYSwxXSJdLFsyLDAsIltbMV1cXG90aW1lcyBhLDFdIl0sWzIsMSwiW2UsZF4xXSIsMl0sWzMsNCwiW2ReMSwxXSIsMl0sWzIsNCwiW2ReMCwxXSJdLFszLDAsIltlLGReMV0iXV0=
\[\begin{tikzcd}
	{[a,2]} & {[a,1]} & {[[1]\otimes a,1]} & {[a,1]} & {[a,2]}
	\arrow["{[e,d^1]}"', from=1-2, to=1-1]
	\arrow["{[d^1,1]}"', from=1-4, to=1-3]
	\arrow["{[d^0,1]}", from=1-2, to=1-3]
	\arrow["{[e,d^1]}", from=1-4, to=1-5]
\end{tikzcd}\]
and the Segal $A$-precategory $I \otimes[a,1]$ is the colimit of the given diagram. As all the morphisms are cofibrations, this colimit is a homotopy colimit.
\end{proof}

\begin{remark}
\label{rem:link with thestrict case cyinder}
To justify why this definition of the Gray interval is the good one, let's study the case of $\zo$-categories.
We denote by $I$ the $\zo$-category generated by the graph $0\to 1$.
If $C$ is an $\zo$-category, we denote by $[C,1]$ the $\zo$-category with two objects - denoted by $0$ and $1$ - and verifying: 
$$\Hom_{[C,1]}(0,1) :=C,~~~\Hom_{[C,1]}(1,0) := \emptyset,~~~\Hom_{[C,1]}(0,0)=\Hom_{[C,1]}(1,1):=\{id\}.$$
We denote by $e$ the terminal $\zo$-category. For example $[e,1] = I$. 
Applying the duality $(\uvar)^{op}$ to the formula given in theorem \ref{theo:appendice formula for otimes}, the $\zo$-category $I\otimes [C,1]$ is the colimit of the following diagram:
% https://q.uiver.app/#q=WzAsNSxbNCwwLCJbQywxXVxcdmVlW2UsMV0iXSxbMCwwLCJbZSwxXVxcdmVlW0MsMV0iXSxbMSwwLCJbQywxXSJdLFszLDAsIltDLDFdIl0sWzIsMCwiW1sxXVxcb3RpbWVzIEMsMV0iXSxbMiwxLCJcXHRyaWFuZ2xlZG93biIsMl0sWzMsNCwiIFtkXjFcXG90aW1lcyBDLDFdIiwyXSxbMiw0LCIgW2ReMFxcb3RpbWVzIEMsMV0iXSxbMywwLCJcXHRyaWFuZ2xlZG93biJdXQ==
\[\begin{tikzcd}
	{[e,1]\vee[C,1]} & {[C,1]} & {[[1]\otimes C,1]} & {[C,1]} & {[C,1]\vee[e,1]}
	\arrow["\triangledown"', from=1-2, to=1-1]
	\arrow["{ [d^1\otimes C,1]}"', from=1-4, to=1-3]
	\arrow["{ [d^0\otimes C,1]}", from=1-2, to=1-3]
	\arrow["\triangledown", from=1-4, to=1-5]
\end{tikzcd}\]
where $\triangledown$ denotes the whiskerings.
\end{remark}
\subsection{Gray cone}
\p 
We define the functor 
$$
\begin{array}{ccl}
\Delta^2\times A &\to& \Seg(A)\\
~[n_0],[n_1] ,a&\mapsto &[[n_0]\otimes a,1]\vee[a,n_1]
\end{array}$$
where $[[n_0]\otimes a,1]\vee[a,n_1]$ fits in the following pushouts:
% https://q.uiver.app/#q=WzAsNCxbMSwxLCJbW25fMF1cXG90aW1lcyBhLDFdXFx2ZWVbYSxuXzFdIl0sWzAsMCwiIFtbbl8wXVxcb3RpbWVzIGEsbl8xXSJdLFsxLDAsIltbbl8wXVxcb3RpbWVzIGEsMStuXzFdIl0sWzAsMSwiW2Esbl8xXSJdLFsxLDNdLFsyLDBdLFszLDBdLFsxLDJdLFswLDcsIiIsMSx7ImxldmVsIjoxLCJzdHlsZSI6eyJuYW1lIjoiY29ybmVyIn19XV0=
\[\begin{tikzcd}
	{ [[n_0]\otimes a,n_1]} & {[[n_0]\otimes a,1+n_1]} \\
	{[a,n_1]} & {[[n_0]\otimes a,1]\vee[a,n_1]}
	\arrow[from=1-1, to=2-1]
	\arrow[from=1-2, to=2-2]
	\arrow[from=2-1, to=2-2]
	\arrow[""{name=0, anchor=center, inner sep=0}, from=1-1, to=1-2]
	\arrow["\lrcorner"{anchor=center, pos=0.125, rotate=180}, draw=none, from=2-2, to=0]
\end{tikzcd}\]

If $n$ is an integer, \wcnotation{$\Delta^2_{/[n]}$}{(delta2@$\Delta^2_{/[n]}$} is the pullback: 
% https://q.uiver.app/?q=WzAsNCxbMSwwLCJcXERlbHRhXjIiXSxbMSwxLCJcXERlbHRhIl0sWzAsMSwiXFxEZWx0YV97L1tuXX0iXSxbMCwwLCJcXERlbHRhXjJfey9bbl19Il0sWzMsMF0sWzAsMV0sWzIsMV0sWzMsMl0sWzMsMSwiIiwxLHsic3R5bGUiOnsibmFtZSI6ImNvcm5lciJ9fV1d
\[\begin{tikzcd}
	{\Delta^2_{/[n]}} & {\Delta^2} \\
	{\Delta_{/[n]}} & \Delta
	\arrow[from=1-1, to=1-2]
	\arrow[from=1-2, to=2-2]
	\arrow[from=2-1, to=2-2]
	\arrow[from=1-1, to=2-1]
	\arrow["\lrcorner"{anchor=center, pos=0.125}, draw=none, from=1-1, to=2-2]
\end{tikzcd}\]
where the right hand functor sends $([n_0],[n_1])$ to $[n_0]^{op}\star [n_1]$.
\begin{prop}
\label{prop:delta 2 n is reedy elegant}
The category $\Delta^2_{/[n]}$ is an elegant Reedy category. 
\end{prop}
\begin{proof}
The proof is analogue to the one of proposition \ref{prop:delta 3 n is reedy elegant}.
\end{proof}


\p 
We define the functor 
$$
\begin{array}{rcl}
A\times \Delta&\to& \Seg(A)\\
~[n],a&\mapsto &H(a,n)
\end{array}$$
by the formula 
$H(a,n):=\colim_{\Delta^2_{/[n]}} [[n_0]\otimes a,1]\vee[a,n_1]$.
 
In order to extend this functor to stratified Segal $A$-precategories with construction \ref{cons:lifting of a functor from A times Delta}, we will need to define the value on $[e,1]_t$, i.e. to choose an object $H(e,1)'$ and an entire cofibration $H(e,1)\to H(e,1)'$. It will be useful to have a more explicit description of this object. 
\begin{example}
\label{exe:explicit Gray cone 1}
The sub-category of $\Delta^2_{/[1]}$ composed of non degenerate objects can be pictured by the graph:
% https://q.uiver.app/?q=WzAsNixbMCwyLCJbMF1ee29wfVxcc3RhclswXSJdLFswLDEsIlswXV57b3B9XFxzdGFyWzFdIl0sWzAsMCwiWzBdXntvcH1cXHN0YXJbMF0iXSxbMiwwLCJbMF1ee29wfVxcc3RhclswXSJdLFsxLDAsIlsxXV57b3B9XFxzdGFyWzBdIl0sWzEsMSwiWzFdIl0sWzAsMSwiZF4yIl0sWzIsMSwiZF4xIiwyXSxbMyw0LCJkXjEiLDJdLFsyLDQsImReMiJdLFs0LDUsInNeMSIsMV0sWzMsNSwiZF4wc14wIl0sWzIsNSwiaWQiLDFdLFsxLDUsInNeMCIsMV0sWzAsNSwiZF4xc14wIiwyXV0=
\[\begin{tikzcd}
	{[0]^{op}\star[0]} & {[1]^{op}\star[0]} & {[0]^{op}\star[0]} \\
	{[0]^{op}\star[1]} & {[1]} \\
	{[0]^{op}\star[0]}
	\arrow["{d^2}", from=3-1, to=2-1]
	\arrow["{d^1}"', from=1-1, to=2-1]
	\arrow["{d^1}"', from=1-3, to=1-2]
	\arrow["{d^2}", from=1-1, to=1-2]
	\arrow["{s^1}"{description}, from=1-2, to=2-2]
	\arrow["{d^0s^0}", from=1-3, to=2-2]
	\arrow["id"{description}, from=1-1, to=2-2]
	\arrow["{s^0}"{description}, from=2-1, to=2-2]
	\arrow["{d^1s^0}"', from=3-1, to=2-2]
\end{tikzcd}\]
The Segal $A$-precategory $ H(e,1)$ is then the colimit of the following diagram:
% https://q.uiver.app/#q=WzAsMyxbMCwwLCJbZSwyXSJdLFsxLDAsIltlLDFdIl0sWzIsMCwiW1sxXSwxXSJdLFsxLDAsIltlLGReMV0iLDJdLFsxLDIsIltkXjAsMV0iXV0=
\[\begin{tikzcd}
	{[e,2]} & {[e,1]} & {[[1],1]}
	\arrow["{[e,d^1]}"', from=1-2, to=1-1]
	\arrow["{[d^0,1]}", from=1-2, to=1-3]
\end{tikzcd}\]
\end{example}


\p \sym{(estar@$e\star\uvar$}
We define the functor $$ e \star\uvar: \stratSeg(A)\to \stratSeg(A)$$ induced, as in the construction \ref{cons:lifting of a functor from A times Delta} by $ H$ and with $H(e,1)'$ as the colimit of the following diagram:
% https://q.uiver.app/#q=WzAsNSxbMSwwLCJbZSwxXSJdLFsyLDAsIltlLDJdIl0sWzMsMCwiW2UsMV0iXSxbNCwwLCJbWzFdX3QsMV0iXSxbMCwwLCJbZSwxXV90Il0sWzAsMSwiW2UsZF4wXSJdLFsyLDEsIltlLGReMV0iLDJdLFsyLDMsIltkXjAsMV0iXSxbMCw0XV0=
\[\begin{tikzcd}
	{[e,1]_t} & {[e,1]} & {[e,2]} & {[e,1]} & {[[1]_t,1]}
	\arrow["{[e,d^0]}", from=1-2, to=1-3]
	\arrow["{[e,d^1]}"', from=1-4, to=1-3]
	\arrow["{[d^0,1]}", from=1-4, to=1-5]
	\arrow[from=1-2, to=1-1]
\end{tikzcd}\]

The object $s^0:[0]^{op}\star[n]\to [n]$ induces a composite morphism $d^0\star[a,n]:\emptyset\star[a,n]:= [a,n]\hookrightarrow [1,1]\vee[a,n]\to e\star [a,n]$, which induces by extension by colimit a natural transformation $$d^0\star C:\emptyset\star C:= C\to e\star C.$$

\begin{prop}
\label{prop:explicit expression of e star a,1}
The Segal $A$-precategory
 $e\star [a,1]$ is the colimit and the homotopy colimit of the following diagram:
% https://q.uiver.app/#q=WzAsMyxbMCwwLCJbZSwxXVxcdmVlW2EsMV0iXSxbMSwwLCJbYSwxXSJdLFsyLDAsIltlXFxzdGFyIGEsMV0iXSxbMSwwLCJbZSxkXjFdIiwyXSxbMSwyLCJbZF57MH1cXHN0YXIgYSwxXSJdXQ==
$$\begin{tikzcd}
	{[e,1]\vee[a,1]} & {[a,1]} & {[e\star a,1]}
	\arrow["{[e,d^1]}"', from=1-2, to=1-1]
	\arrow["{[d^{0}\star a,1]}", from=1-2, to=1-3]
\end{tikzcd}
$$
\end{prop}
\begin{proof}
We have already given the description of $\Delta^2_{/[1]}$ in
the example \ref{exe:explicit Gray cone 1}.
The Segal $A$-precategory $H(a,1)$ is the colimit of the following diagram:
% https://q.uiver.app/#q=WzAsMyxbMCwwLCJbYSwyXSJdLFsxLDAsIlthLDFdIl0sWzIsMCwiW1sxXVxcb3RpbWVzIGEsMV0iXSxbMSwwLCJbZSxkXjFdIiwyXSxbMSwyLCJbZF4wXFxvdGltZXMgYSwxXSJdXQ==
\[\begin{tikzcd}
	{[a,2]} & {[a,1]} & {[[1]\otimes a,1]}
	\arrow["{[e,d^1]}"', from=1-2, to=1-1]
	\arrow["{[d^0\otimes a,1]}", from=1-2, to=1-3]
\end{tikzcd}\]
and $e\star [a,1]$ is the colimit of the given diagram.
As all the morphisms are cofibrations, this colimit is a homotopy colimit.
\end{proof}

\begin{remark}
Using again notations of remark \ref{rem:link with thestrict case cyinder}, if $C$ is an $\zo$-category, the $\zo$-category $e\star C$ is the colimit of the following diagram:
% https://q.uiver.app/#q=WzAsMyxbMCwwLCJbZSwxXVxcdmVlW0MsMV0iXSxbMSwwLCJbQywxXSJdLFsyLDAsIltlXFxzdGFyIEMsMV0iXSxbMSwwLCJcXHRyaWFuZ2xlZG93biIsMl0sWzEsMiwiW2ReezB9XFxzdGFyIEMsMV0iXV0=
\[\begin{tikzcd}
	{[e,1]\vee[C,1]} & {[C,1]} & {[e\star C,1]}
	\arrow["\triangledown"', from=1-2, to=1-1]
	\arrow["{[d^{0}\star C,1]}", from=1-2, to=1-3]
\end{tikzcd}\]
where $\triangledown$ is the whiskering. Our definition of the join is therefore analogous to that of the strict world.
\end{remark}


\begin{prop}
\label{prop:explicit expression of e star e star a,1}
The Segal $A$-precategory
 $[1]\star [a,1]$ is the colimit of the following diagram:
% https://q.uiver.app/?q=WzAsMTEsWzAsMCwiW1syXVxcYm90aW1lcyBhLDFdIl0sWzEsMCwiW1sxXVxcb3RpbWVzIGEsMV0iXSxbMiwwLCJbWzFdLDFdXFx2ZWVbYSwxXSJdLFszLDAsIltlLDFdXFx2ZWVbYSwxXSJdLFs0LDAsIltlLDJdXFx2ZWVbYSwxXSJdLFswLDEsIltlXFxzdGFyIGEsMV0iXSxbMywxLCJbYSwxXSJdLFs0LDEsIltlLDFdXFx2ZWVbYSwxXSJdLFswLDIsIltbMV1cXHN0YXIgYSwxXSJdLFs0LDIsIltlLDFdXFx2ZWVbZVxcc3RhciBhLDFdIl0sWzMsMiwiW2VcXHN0YXIgYSwxXSJdLFsxLDAsIltkXjBcXG90aW1lcyBhLDFdIiwyXSxbMSwyLCJbWzFdXFxvdGltZXMgYSxkXjFdIl0sWzMsMiwiW2ReMFxcb3RpbWVzIGEsMl0iLDJdLFszLDQsIlthLGReMV0iXSxbNiw3LCJbYSxkXjFdIl0sWzUsMCwiW2ReMVxcYm90aW1lcyBhLDFdIl0sWzYsMywiW2EsZF4xXSJdLFs3LDQsIlthLGReMl0iLDJdLFs1LDgsIltkXjFcXHN0YXIgYSwxXSIsMl0sWzcsOSwiW2ReezB9XFxzdGFyIGEsMl0iXSxbNiw1LCJbZF57MH1cXHN0YXIgYSwxXSIsMl0sWzEwLDgsIltkXjBcXHN0YXIgYSwxXSJdLFs2LDEwLCJbZF57MH1cXHN0YXIgYSwxXSIsMl0sWzEwLDksIltlXFxzdGFyIGEsZF4xXSIsMl1d
$$
\begin{tikzcd}[column sep=0.7 cm]
	{[[2]\botimes a,1]} & {[[1]\otimes a,1]} & {[[1],1]\vee[a,1]} & {[e,1]\vee[a,1]} & {[e,2]\vee[a,1]} \\
	{[e\star a,1]} &&& {[a,1]} & {[e,1]\vee[a,1]} \\
	{[[1]\star a,1]} &&& {[e\star a,1]} & {[e,1]\vee[e\star a,1]}
	\arrow["{[d^0\otimes a,1]}"', from=1-2, to=1-1]
	\arrow["{[[1]\otimes a,d^1]}", from=1-2, to=1-3]
	\arrow["{[d^0\otimes a,2]}"', from=1-4, to=1-3]
	\arrow["{[a,d^1]}", from=1-4, to=1-5]
	\arrow["{[a,d^1]}", from=2-4, to=2-5]
	\arrow["{[d^1\botimes a,1]}", from=2-1, to=1-1]
	\arrow["{[a,d^1]}", from=2-4, to=1-4]
	\arrow["{[a,d^2]}"', from=2-5, to=1-5]
	\arrow["{[d^1\star a,1]}"', from=2-1, to=3-1]
	\arrow["{[d^{0}\star a,2]}", from=2-5, to=3-5]
	\arrow["{[d^{0}\star a,1]}"', from=2-4, to=2-1]
	\arrow["{[d^0\star a,1]}", from=3-4, to=3-1]
	\arrow["{[d^{0}\star a,1]}"', from=2-4, to=3-4]
	\arrow["{[e\star a,d^1]}"', from=3-4, to=3-5]
\end{tikzcd}$$
where $[2]\botimes a$ and $[[1],1]\vee[a,1]$ are the pushouts:
% https://q.uiver.app/?q=WzAsOCxbMiwwLCJbMl1cXG90aW1lcyBhIl0sWzAsMCwiWzFdXFxvdGltZXMgYVxcYW1hbGcgWzFdXFxvdGltZXMgYSJdLFswLDEsImVcXHN0YXIgYVxcYW1hbGcgZVxcc3RhciBhIl0sWzIsMSwiWzJdXFxib3RpbWVzIGEiXSxbMywwLCJbWzFdXFxvdGltZXMgYSwxXVxcYW1hbGcgW1sxXVxcb3RpbWVzIGEsMl0iXSxbNCwxLCJbWzFdLDFdXFx2ZWVbYSwxXSJdLFs0LDAsIltbMV1cXG90aW1lcyBhLDJdIl0sWzMsMSwiW1sxXSwxXVxcYW1hbGdbYSwxXSJdLFsxLDAsImReMVxcb3RpbWVzIGFcXGFtYWxnIGReMlxcb3RpbWVzIGEiXSxbMiwzLCJkXjFcXGJvdGltZXMgYVxcYW1hbGcgZF4yXFxib3RpbWVzIGEiLDJdLFsxLDJdLFswLDNdLFs0LDYsIltbMV1cXG90aW1lcyBhLGReMlxcYW1hbGcgZF4xXSJdLFs0LDddLFs3LDVdLFs2LDVdLFs1LDQsIiIsMSx7InN0eWxlIjp7Im5hbWUiOiJjb3JuZXIifX1dLFszLDgsIiIsMSx7ImxldmVsIjoxLCJzdHlsZSI6eyJuYW1lIjoiY29ybmVyIn19XV0=
\[\begin{tikzcd}
	{[1]\otimes a\amalg [1]\otimes a} && {[2]\otimes a} & {[[1]\otimes a,1]\amalg [[1]\otimes a,2]} & {[[1]\otimes a,2]} \\
	{e\star a\amalg e\star a} && {[2]\botimes a} & {[[1],1]\amalg[a,1]} & {[[1],1]\vee[a,1]}
	\arrow[""{name=0, anchor=center, inner sep=0}, "{d^1\otimes a\amalg d^2\otimes a}", from=1-1, to=1-3]
	\arrow["{d^1\botimes a\amalg d^2\botimes a}"', from=2-1, to=2-3]
	\arrow[from=1-1, to=2-1]
	\arrow[from=1-3, to=2-3]
	\arrow["{[[1]\otimes a,d^2\amalg d^1]}", from=1-4, to=1-5]
	\arrow[from=1-4, to=2-4]
	\arrow[from=2-4, to=2-5]
	\arrow[from=1-5, to=2-5]
	\arrow["\lrcorner"{anchor=center, pos=0.125, rotate=180}, draw=none, from=2-5, to=1-4]
	\arrow["\lrcorner"{anchor=center, pos=0.125, rotate=180}, draw=none, from=2-3, to=0]
\end{tikzcd}\]
\end{prop}
\begin{proof}
Let's start by studying the object $H(a,2)$. Here is a final subcategory of $\Delta^2_{/[2]}$:
% https://q.uiver.app/?q=WzAsNixbMSwxLCJbMl0iXSxbMCwxLCJbMl1ee29wfVxcc3RhclswXSJdLFswLDAsIlsxXV57b3B9XFxzdGFyWzBdIl0sWzEsMCwiWzFdXntvcH1cXHN0YXJbMV0iXSxbMiwwLCJbMF1ee29wfVxcc3RhclsxXSJdLFsyLDEsIlswXV57b3B9XFxzdGFyWzJdIl0sWzEsMCwic14yIiwyXSxbMiwxLCJkXjIiLDJdLFsyLDMsImReMiJdLFszLDAsInNeMSJdLFs0LDMsImReMSIsMl0sWzQsNSwiZF4xIl0sWzUsMCwic14wIl1d
\[\begin{tikzcd}
	{[1]^{op}\star[0]} & {[1]^{op}\star[1]} & {[0]^{op}\star[1]} \\
	{[2]^{op}\star[0]} & {[2]} & {[0]^{op}\star[2]}
	\arrow["{s^2}"', from=2-1, to=2-2]
	\arrow["{d^2}"', from=1-1, to=2-1]
	\arrow["{d^2}", from=1-1, to=1-2]
	\arrow["{s^1}", from=1-2, to=2-2]
	\arrow["{d^1}"', from=1-3, to=1-2]
	\arrow["{d^1}", from=1-3, to=2-3]
	\arrow["{s^0}", from=2-3, to=2-2]
\end{tikzcd}\]
The Segal $A$-precategory $H(a,2)$ is then the colimit of the following diagram:
% https://q.uiver.app/?q=WzAsNSxbMCwwLCJbWzJdXFxvdGltZXMgYSwxXSJdLFsxLDAsIltbMV1cXG90aW1lcyBhLDFdIl0sWzIsMCwiW1sxXVxcb3RpbWVzIGEsMV1cXHZlZVthLDFdIl0sWzMsMCwiW2EsMl0iXSxbNCwwLCJbYSwzXSJdLFsxLDAsIltkXjBcXG90aW1lcyBhLDFdIiwyXSxbMSwyLCJbWzFdXFxvdGltZXMgYSxkXjFdIl0sWzMsMiwiW2ReMFxcb3RpbWVzIGEsMl0iLDJdLFszLDQsIlthLGReMV0iXV0=
\[\begin{tikzcd}
	{[[2]\otimes a,1]} & {[[1]\otimes a,1]} & {[[1]\otimes a,1]\vee[a,1]} & {[a,2]} & {[a,3]}
	\arrow["{[d^0\otimes a,1]}"', from=1-2, to=1-1]
	\arrow["{[[1]\otimes a,d^1]}", from=1-2, to=1-3]
	\arrow["{[d^0\otimes a,2]}"', from=1-4, to=1-3]
	\arrow["{[a,d^1]}", from=1-4, to=1-5]
\end{tikzcd}\]
The Segal $A$-precategory $e\star([e,1]\vee[a,1])$ is then the colimit of the following diagram:
% https://q.uiver.app/#q=WzAsNSxbMCwwLCJbWzJdXFxib3RpbWVzIGEsMV0iXSxbMSwwLCJbWzFdXFxvdGltZXMgYSwxXSJdLFsyLDAsIltbMV0sMV1cXHZlZVthLDFdIl0sWzMsMCwiW2UsMV1cXHZlZVthLDFdIl0sWzQsMCwiW2UsMl1cXHZlZVthLDFdIl0sWzEsMCwiW2ReMFxcb3RpbWVzIGEsMV0iLDJdLFsxLDIsIltbMV1cXG90aW1lcyBhLGReMV0iXSxbMywyLCJbZF4wXFxvdGltZXMgYSwyXSIsMl0sWzMsNCwiW2EsZF4xXSJdXQ==
\[\begin{tikzcd}[column sep=0.7 cm]
	{[[2]\botimes a,1]} & {[[1]\otimes a,1]} & {[[1],1]\vee[a,1]} & {[e,1]\vee[a,1]} & {[e,2]\vee[a,1]}
	\arrow["{[d^0\otimes a,1]}"', from=1-2, to=1-1]
	\arrow["{[[1]\otimes a,d^1]}", from=1-2, to=1-3]
	\arrow["{[d^0\otimes a,2]}"', from=1-4, to=1-3]
	\arrow["{[a,d^1]}", from=1-4, to=1-5]
\end{tikzcd}\]
The fact that $[1]\star[a,1]$ is the colimit of the given diagram then follows from the equality $[1]\star[a,1]=e\star(e\star[a,1])$ and from the explicit expression of 
$e\star [a,1]$ given in proposition \ref{prop:explicit expression of e star a,1}.
\end{proof}


\subsection{Link between the Gray cylinder and Gray cone}
\p There is a canonical morphism $I\otimes [a,n]\to e\star [a,n]$ sending $[a,n_0]\vee[[n_1]\otimes a,1]\vee[a,n_2]$ to $[[n_1]\otimes a,1]\vee[a,n_2]$.
Note that the induced morphism $I\otimes [e,1]\to e\star [e,1]\to e\star [e,1]_t$ factors through $I\otimes [e,1]_t$. We can then extend it by colimit to a natural transformation $I\otimes C\to e\star C$.


We now define $(I\otimes [a,n])_{/\{0\}\otimes [a,n]}$ and $[a,n_0]\vee[[n_1]\otimes a,1]\vee[a,n_2]_{/[a,n_0]}$ as the pushouts:
% https://q.uiver.app/?q=WzAsOCxbMSwxLCIoSVxcb3RpbWVzIFthLG5dKV97L1xcezBcXH1cXG90aW1lcyBbYSxuXX0iXSxbMSwwLCJJXFxvdGltZXNbYSxuXSJdLFswLDAsIlthLG5dXFxvdGltZXNcXHswXFx9Il0sWzAsMSwiZSJdLFszLDEsIiBbYSxuXzBdXFx2ZWVbW25fMV1cXG90aW1lcyBhLDFdXFx2ZWVbYSxuXzJdX3svW2Esbl8wXX0iXSxbMywwLCIgW2Esbl8wXVxcdmVlW1tuXzFdXFxvdGltZXMgYSwxXVxcdmVlW2Esbl8yXSJdLFsyLDAsIiBbYSxuXzBdIl0sWzIsMSwiZSJdLFsyLDFdLFszLDBdLFsyLDNdLFsxLDBdLFs2LDddLFs1LDRdLFs3LDRdLFs2LDVdLFs0LDE1LCIiLDEseyJsZXZlbCI6MSwic3R5bGUiOnsibmFtZSI6ImNvcm5lciJ9fV0sWzAsOCwiIiwxLHsibGV2ZWwiOjEsInN0eWxlIjp7Im5hbWUiOiJjb3JuZXIifX1dXQ==
\[\begin{tikzcd}[column sep = 0.5cm]
	{[a,n]\otimes\{0\}} & {I\otimes[a,n]} & { [a,n_0]} & { [a,n_0]\vee[[n_1]\otimes a,1]\vee[a,n_2]} \\
	e & {(I\otimes [a,n])_{/\{0\}\otimes [a,n]}} & e & { [a,n_0]\vee[[n_1]\otimes a,1]\vee[a,n_2]_{/[a,n_0]}}
	\arrow[""{name=0, anchor=center, inner sep=0}, from=1-1, to=1-2]
	\arrow[from=2-1, to=2-2]
	\arrow[from=1-1, to=2-1]
	\arrow[from=1-2, to=2-2]
	\arrow[from=1-3, to=2-3]
	\arrow[from=1-4, to=2-4]
	\arrow[from=2-3, to=2-4]
	\arrow[""{name=1, anchor=center, inner sep=0}, from=1-3, to=1-4]
	\arrow["\lrcorner"{anchor=center, pos=0.125, rotate=180}, draw=none, from=2-4, to=1]
	\arrow["\lrcorner"{anchor=center, pos=0.125, rotate=180}, draw=none, from=2-2, to=0]
\end{tikzcd}\]
By Segal extensions and by two out of three, the following canonical morphism 
$$ [a,n_0]\vee[[n_1]\otimes a,1]\vee[a,n_2]_{/[a,n_0]}\to [[n_1]\otimes a,1]\vee[a,n_2]$$
is a weak equivalence. As $\Delta^3_{/[n]}$ is Reedy elegant, this induces a weak equivalence
$$\colim_{\Delta^3_{/[n]}} [a,n_0]\vee[[n_1]\otimes a,1]\vee[a,n_2]_{/[a,n_0]}\to \colim_{\Delta^3_{/[n]}} [[n_1]\otimes a,1]\vee[a,n_2].$$
Remark furthermore that the left hand object is equivalent to $(I\otimes [a,n])_{/\{0\}\otimes [a,n]}$ and the right one to 
 $H(a,n)$. As the construction \ref{cons:lifting of a functor from A times Delta} preserves weakly invertible natural transformations between functors that preserve cofibration, this induces a weakly invertible natural transformation $(I\otimes [a,n])_{/\{0\}\otimes [a,n]}\to e \star[a,n]$. This directly implies that squares
% https://q.uiver.app/?q=WzAsOCxbMSwwLCJJXFxvdGltZXNbYSxuXSJdLFswLDAsIlxcezBcXH1cXG90aW1lcyBbYSxuXSJdLFswLDEsImUiXSxbMSwxLCJlXFxzdGFyW2Esbl0iXSxbNCwwLCJJXFxvdGltZXNbZSwxXV90Il0sWzMsMCwiXFx7MFxcfVxcb3RpbWVzIFtlLDFdX3QiXSxbMywxLCJlIl0sWzQsMSwiZVxcc3RhcltlLDFdX3QiXSxbMSwyXSxbMiwzXSxbMSwwXSxbMCwzXSxbNSw2XSxbNiw3XSxbNCw3XSxbNSw0XV0=
\[\begin{tikzcd}
	{\{0\}\otimes [a,n]} & {I\otimes[a,n]} && {\{0\}\otimes [e,1]_t} & {I\otimes[e,1]_t} \\
	e & {e\star[a,n]} && e & {e\star[e,1]_t}
	\arrow[from=1-1, to=2-1]
	\arrow[from=2-1, to=2-2]
	\arrow[from=1-1, to=1-2]
	\arrow[from=1-2, to=2-2]
	\arrow[from=1-4, to=2-4]
	\arrow[from=2-4, to=2-5]
	\arrow[from=1-5, to=2-5]
	\arrow[from=1-4, to=1-5]
\end{tikzcd}\]
are homotopy cocartesian. As every stratified Segal $A$-precategory is a homotopy colimit of objects of shape $[a,n]$ and $[e,1]_t$, and as $I\otimes \uvar$ and $e\star \uvar$ preserves monomorphisms, this implies the following proposition:

\begin{prop}
\label{labe:Link between the Gray cylinder and cone}
For any stratified Segal $A$-precategory $C$, the natural transformation $I\otimes \uvar\to e\star \uvar$ fits into a homotopy cocartesian square:
% https://q.uiver.app/?q=WzAsNCxbMSwwLCJJXFxvdGltZXMgQyJdLFswLDAsIlxcezBcXH1cXG90aW1lcyBDIl0sWzEsMSwiZVxcc3RhciBDIl0sWzAsMSwiZSJdLFsxLDBdLFszLDJdLFsxLDNdLFswLDJdXQ==
\[\begin{tikzcd}
	{\{0\}\otimes C} & {I\otimes C} \\
	e & {e\star C}
	\arrow[from=1-1, to=1-2]
	\arrow[from=2-1, to=2-2]
	\arrow[from=1-1, to=2-1]
	\arrow[from=1-2, to=2-2]
\end{tikzcd}\]
\end{prop}

\p
We define the functor 
$$
\begin{array}{rcl}
A\times \Delta&\to& \Seg(A)\\
~[n],a&\mapsto &T(a,n)
\end{array}$$
by the formula 
$T(a,n) := [[n]\otimes a,1]$.

Eventually
we define the functor $ \Sigma^{\circ}[a,n]: \stratSeg(A)\to \stratSeg(A)$ \sym{(sigmacirc@$\Sigma^{\circ}$} induced, as in the construction \ref{cons:lifting of a functor from A times Delta}, by $ T$ and with $T(e,1):= [[1]_t\otimes e,1]$. This functor is called the \wcnotion{$\circ$-suspension}{suspensions@$\circ$-suspension}.
With a proof similar to the on of proposition \ref{labe:Link between the Gray cylinder and cone}, one can show:
\begin{prop}
\label{labe:Link between the Gray cylinder and cosuspension}
There exists a natural transformation $e\star \uvar\to\Sigma^{\circ}(\uvar)$ such that
for any marked Segal $A$-precategory $C$, $e\star C\to \Sigma^{\circ}C$ induces a homotopy cocartesian square:
% https://q.uiver.app/?q=WzAsNCxbMSwwLCJlXFxzdGFyIEMiXSxbMCwwLCIgQyJdLFsxLDEsIlxcU2lnbWFee1xcY2lyY31DIl0sWzAsMSwiIGUiXSxbMSwwXSxbMywyXSxbMSwzXSxbMCwyXV0=
\[\begin{tikzcd}
	{ C} & {e\star C} \\
	{ e} & {\Sigma^{\circ}C}
	\arrow[from=1-1, to=1-2]
	\arrow[from=2-1, to=2-2]
	\arrow[from=1-1, to=2-1]
	\arrow[from=1-2, to=2-2]
\end{tikzcd}\]
\end{prop}


\subsection{Gray constructions are left Quillen}
In this section, we show that the Gray cylinder is a Quillen functor. Combined with the proposition \ref{labe:Link between the Gray cylinder and cone}, this will imply that the Gray cone is Quillen.
\p
Let $x:[k_0]\star[k_1]^{op}\star [k_2] \to[n]$ be an element of $\Delta^3_{/[n]}$. The \wcnotion{degree}{degree of an element of $\Delta^3_{/[n]}$} of $x$, is $f(0)-f(k_1)$ where $f$ is the composite morphism:
$$f:[k_1]^{op}\to [k_0]\star[k_1]^{op}\star [k_2] \to [n]$$
We will denote by $K_{\leq i}$ the full subcategory of $\Delta^3_{/[n]}$ whose objects are of degree inferior or equal to $i$.


An element $x:[k_0]\star[k_1]^{op}\star [k_2] \to[n]$ of degree $d$ is \wcnotion{regular}{regular elements of $\Delta^3_{/[n]}$} if $k_1 = d$, $k_0+k_1+k_2 = n$ and 
$$x(l):=
\left\{
\begin{array}{ll}
l &\mbox{if $l\leq k_0$}\\
l-1 &\mbox{if $k_0<l\leq k_0+k_1$}\\
l-2 &\mbox{if $k_0+k_1<l$}\\
\end{array} \right.$$
Remark that the regular object $x$ is characterized by the triple $(k_0,k_1,k_2)$.

\p
Let $x:[k_0]\star[k_1]^{op}\star [k_2] \to[n]$ be an element $\Delta^3_{/[n]}$, and $i:[0]\to [k_0]\star[k_1]^{op}\star [k_2]$ a morphism. We denote by $d^ix :=[k_0']\star[k_1']^{op}\star [k_2'] \xrightarrow{d} [k_0]\star[k_1]^{op}\star [k_2] \to[n]$ the morphism that avoids $i$, and where $k_j' := k_j-1$ if $i$ factors through $[k_j]$ and $k_j' := k_j$ if not. We then define
$(\Delta^3_{/[n]})_{/\Lambda^ix}$ as the full subcategory of $(\Delta^3_{/[n]})_{/x}$ that includes any non negative object $x'\to x$ that are different of $d^ix\to x$ and $id:x\to x$.

\begin{lemma}
\label{lem:Gray cinder is Quillen 2}
For any regular object $x:[k_0]\star[k_1]^{op}\star [k_2] \to[n]$ and for any $i:[0]\to [k_0]\star[k_1]^{op}\star [k_2]$ which is neither $k_0+1$ nor $k_0+k_1+1$, the morphism
$$\underset{(\Delta^3_{/[n]})_{\Lambda^ix}}{\colim} [a,\uvar]\vee[\uvar \otimes a,1]\vee[a,\uvar] 
\to [a,k_0]\vee[[k_1]\otimes a,1]\vee[a,k_2]$$
is an acyclic cofibration. 
\end{lemma} 
\begin{proof}
Suppose first that the image of $i$ is in $[k_0]$. There is a cocartesian square:
% https://q.uiver.app/#q=WzAsNCxbMSwxLCJbYSxrXzBdXFx2ZWVbW2tfMV1cXG90aW1lcyBhLDFdXFx2ZWVbYSxrXzJdIl0sWzEsMCwiXFx1bmRlcnNldHsoXFxEZWx0YV4zX3svW25dfSlfey9cXExhbWJkYV5peH19e1xcY29saW19IFthLFxcdXZhcl1cXHZlZVtcXHV2YXIgXFxvdGltZXMgYSwxXVxcdmVlW2EsXFx1dmFyXSAiXSxbMCwwLCJbW2tfMV1cXG90aW1lcyBhLCBcXExhbWJkYV57aX1ba18wKzEra18yXV1cXGN1cCBbXFxwYXJ0aWFsW2tfMV1cXG90aW1lcyBhLCAgW2tfMCsxK2tfMl1dIl0sWzAsMSwiW1trXzFdXFxvdGltZXMgYSwgW2tfMCsxK2tfMl1dIl0sWzIsMV0sWzMsMF0sWzIsM10sWzEsMF1d
\[\begin{tikzcd}[column sep=0.38cm]
	{[[k_1]\otimes a, \Lambda^{i}[k_0+1+k_2]]\cup [\partial[k_1]\otimes a,  [k_0+1+k_2]]} & {\underset{(\Delta^3_{/[n]})_{/\Lambda^ix}}{\colim} [a,\uvar]\vee[\uvar \otimes a,1]\vee[a,\uvar] } \\
	{[[k_1]\otimes a, [k_0+1+k_2]]} & {[a,k_0]\vee[[k_1]\otimes a,1]\vee[a,k_2]}
	\arrow[from=1-1, to=1-2]
	\arrow[from=2-1, to=2-2]
	\arrow[from=1-1, to=2-1]
	\arrow[from=1-2, to=2-2]
\end{tikzcd}\]
where the left-hand morphism is an acyclic cofibration. The case where the image of $i$ is in $[k_2]$ is similar. 
Suppose now that $i$ lands in $[k_1]$. We then define $i':=i-k_0-1$, and there is a cocartesian square:
% https://q.uiver.app/#q=WzAsNCxbMSwxLCJbYSxrXzBdXFx2ZWVbW2tfMV1cXG90aW1lcyBhLDFdXFx2ZWVbYSxrXzJdIl0sWzEsMCwiXFx1bmRlcnNldHsoXFxEZWx0YV4zX3svW25dfSlfey9cXExhbWJkYV5peH19e1xcY29saW19IFthLFxcdXZhcl1cXHZlZVtcXHV2YXIgXFxvdGltZXMgYSwxXVxcdmVlW2EsXFx1dmFyXSAiXSxbMCwwLCJbXFxMYW1iZGFee2knfVtrXzFdXFxvdGltZXMgYSwgW2tfMCsxK2tfMl1dXFxjdXBbIFtrXzFdXFxvdGltZXMgYSxcXHBhcnRpYWwgW2tfMCsxK2tfMl1dIl0sWzAsMSwiWyBba18xXVxcb3RpbWVzIGEsIFtrXzArMStrXzJdXSJdLFsyLDFdLFszLDBdLFsyLDNdLFsxLDBdXQ==
\[\begin{tikzcd}[column sep=0.33cm]
	{[\Lambda^{i'}[k_1]\otimes a, [k_0+1+k_2]]\cup[ [k_1]\otimes a,\partial [k_0+1+k_2]]} & {\underset{(\Delta^3_{/[n]})_{/\Lambda^ix}}{\colim} [a,\uvar]\vee[\uvar \otimes a,1]\vee[a,\uvar] } \\
	{[ [k_1]\otimes a, [k_0+1+k_2]]} & {[a,k_0]\vee[[k_1]\otimes a,1]\vee[a,k_2]}
	\arrow[from=1-1, to=1-2]
	\arrow[from=2-1, to=2-2]
	\arrow[from=1-1, to=2-1]
	\arrow[from=1-2, to=2-2]
\end{tikzcd}\]
where the left-hand morphism is an acyclic cofibration.
\end{proof}

\begin{lemma}
\label{lem:Gray cinder is Quillen 3}
Let $0<k<n$ be two integers. The morphism
$$\underset{\Delta^3_{/\Lambda^k[n]} \cup K_{\leq d}}{\colim} [a,\uvar]\vee[\uvar \otimes a,1]\vee[a,\uvar] \to \underset{\Delta^3_{/\Lambda^k[n]} \cup K_{\leq d+1}}{\colim} [a,\uvar]\vee[\uvar \otimes a,1]\vee[a,\uvar] $$
is an acyclic cofibration
\end{lemma}
\begin{proof}
For $x:=[k_0]\star [k_1]^{op}\star[k_2]\to [n]$ a regular element of degree $d+1$, we denote by $s_x$ the section of $x$ that avoids $k_0+1$ and $k_0+k_1+1$. We denote $R_{d+1}$ the set of regular elements of degree $d+1$.
We claim that we have a cocartesian square
% https://q.uiver.app/#q=WzAsNCxbMSwxLCJcXERlbHRhXjNfey9cXExhbWJkYV5rW25dfVxcY3VwIEtfe1xcbGVxIGQrMX0iXSxbMSwwLCJcXERlbHRhXjNfey9cXExhbWJkYV5rW25dfSBcXGN1cCBLX3tcXGxlcSBkfSJdLFswLDEsIlxcY29wcm9kX3t4XFxpbiBSX3tkKzF9fShcXERlbHRhXjNfey9bbl19KV97L3h9Il0sWzAsMCwiXFxjb3Byb2Rfe3hcXGluIFJfe2QrMX19KFxcRGVsdGFeM197L1tuXX0pX3svXFxMYW1iZGFee3Nfayh4KX0geH0iXSxbMywyXSxbMywxXSxbMiwwXSxbMSwwXV0=
\begin{equation}
\label{eq:Gray cinder is Quillen 3}
\begin{tikzcd}
	{\coprod_{x\in R_{d+1}}(\Delta^3_{/[n]})_{/\Lambda^{s_k(x)} x}} & {\Delta^3_{/\Lambda^k[n]} \cup K_{\leq d}} \\
	{\coprod_{x\in R_{d+1}}(\Delta^3_{/[n]})_{/x}} & {\Delta^3_{/\Lambda^k[n]}\cup K_{\leq d+1}}
	\arrow[from=1-1, to=2-1]
	\arrow[from=1-1, to=1-2]
	\arrow[from=2-1, to=2-2]
	\arrow[from=1-2, to=2-2]
\end{tikzcd}
\end{equation}
This will induce a cocartesian square:
% https://q.uiver.app/#q=WzAsNCxbMSwxLCJcXHVuZGVyc2V0e1xcRGVsdGFeM197L1xcTGFtYmRhXmtbbl19IFxcY3VwIEtfe1xcbGVxIGQgKzF9fXtcXGNvbGltfSBbYSxcXHV2YXJdXFx2ZWVbXFx1dmFyIFxcb3RpbWVzIGEsMV1cXHZlZVthLFxcdXZhcl0gIl0sWzAsMCwiXFxjb3Byb2Rfe3hcXGluIFJfe2QrMX19XFx1bmRlcnNldHsoXFxEZWx0YV4zX3svW25dfSlfey9cXExhbWJkYV57c194KGspfXh9fXtcXGNvbGltfSBbYSxcXHV2YXJdXFx2ZWVbXFx1dmFyIFxcb3RpbWVzIGEsMV1cXHZlZVthLFxcdXZhcl0gIl0sWzEsMCwiXFx1bmRlcnNldHtcXERlbHRhXjNfey9cXExhbWJkYV5rW25dfSBcXGN1cCBLX3tcXGxlcSBkIH19e1xcY29saW19IFthLFxcdXZhcl1cXHZlZVtcXHV2YXIgXFxvdGltZXMgYSwxXVxcdmVlW2EsXFx1dmFyXSAiXSxbMCwxLCJcXGNvcHJvZF97eFxcaW4gUl97ZCsxfX1cXCBbYSxrXzBdXFx2ZWVbW2tfMV1cXG90aW1lcyBhLDFdXFx2ZWVbYSxrXzJdICJdLFsxLDNdLFsyLDBdLFszLDBdLFsxLDJdXQ==
\[\begin{tikzcd}[column sep =0.3cm]
	{\coprod_{x\in R_{d+1}}\underset{(\Delta^3_{/[n]})_{/\Lambda^{s_x(k)}x}}{\colim} [a,\uvar]\vee[\uvar \otimes a,1]\vee[a,\uvar] } & {\underset{\Delta^3_{/\Lambda^k[n]} \cup K_{\leq d }}{\colim} [a,\uvar]\vee[\uvar \otimes a,1]\vee[a,\uvar] } \\
	{\coprod_{x\in R_{d+1}}\ [a,k_0]\vee[[k_1]\otimes a,1]\vee[a,k_2] } & {\underset{\Delta^3_{/\Lambda^k[n]} \cup K_{\leq d +1}}{\colim} [a,\uvar]\vee[\uvar \otimes a,1]\vee[a,\uvar] }
	\arrow[from=1-1, to=2-1]
	\arrow[from=1-2, to=2-2]
	\arrow[from=2-1, to=2-2]
	\arrow[from=1-1, to=1-2]
\end{tikzcd}\]
where the left vertical morphism is an acyclic cofibration according to lemma \ref{lem:Gray cinder is Quillen 2}, which will conclude the proof.

We then have to justify the cocartesianess of the square \eqref{eq:Gray cinder is Quillen 3}. We denote by $D$ the colimit of the underlying span of this square and $\psi:D\to \Delta^3_{/\Lambda^k[n]}\cup K_{\leq d+1}$ the induced morphism. We will construct an inverse $\phi$ of this functor.

Let $x:[k_0]\star[k_1]^{op}\star [k_2] \to[n]$ be an element of $\Delta^3_{/[n]}$ of degree $(d+1)$. We denote by $x_r$ the regular element characterized by the triple $(x(k_1),d+1,n-x(k_0+k_1+1))$. There is a unique morphism $x\to x_r$. Furthermore, for any other regular element $x'$, $\Hom(x,x')=\emptyset$. We then set $$\phi(x):= x\to x_r\in (\Delta^3_{/[n]})_{/x_r}.$$ If $x:[k_0]\star[k_1]^{op}\star [k_2] \to[n]$ is an element of $\Delta^3_{/\Lambda^k[n]}$, we set $$\phi(x):=x\in \Delta^3_{/\Lambda^k[n]}\cup K_{\leq d}.$$ To justify that this is well defined, remark that for any object $x$ of $\Delta^3_{\Lambda^k[n]}$ of degree $d+1$, the morphism $x\to x_r$ factors through $\Lambda^{s_k(x_r)}x_r$. This assignation lifts to a functor $\phi:\Delta^3_{/\Lambda^k[n]}\cup K_{\leq d+1}\to D$ that is an inverse of $\psi$.
\end{proof}



\begin{prop}
\label{prop:Gray cinder is Quillen 5}
The morphism 
$I\otimes([a,1]\cup[a,1]\cup... \cup [a,1])\to I\otimes [a,n]$ is an acyclic cofibration.
\end{prop}
\begin{proof}
Let $0<k< n$ be two integers.
Let's demonstrate first that morphisms $I\otimes[a,\Lambda^k[n]]\to I\otimes [a,n]$ are acyclic cofibrations.
We set
$$P_d:=\underset{\Delta^3_{/\Lambda^k[n]} \cup K_{\leq d}}{\colim} [a,\uvar]\vee[\uvar \otimes a,1]\vee[a,\uvar].$$
According to lemma \ref{lem:Gray cinder is Quillen 3} , we have a sequence of acyclic cofibrations 
$I\otimes[a,\Lambda^k[n]]=P_0\to P_1...\to P_n= I\otimes [a,n] $. This implies that the functor $I\otimes [a,\uvar]:\Sset\to \stratSeg(A)$ sends inner anodyne extensions to weak equivalences. 

Eventually, proposition 3.7.4 of \cite{Cisinski_Higher_categories_and_homotopical_algebra} states that the inclusion $[1]\cup...\cup[1]\cup[1]\to[n]$ is an inner anodyne extension, which concludes the proof.
\end{proof}





\begin{lemma}
\label{lem:Gray cinder is Quillen 6}
Let $a\to b$ be a generating acyclic cofibration. The morphism 
$I\otimes ([a,n]\cup [b,\partial[n]])\to I\otimes [b,n]$ is an acyclic cofibration.
\end{lemma}
\begin{proof}
It is obvious that $I\otimes [a,n]\to I\otimes [b,n]$ is an acyclic cofibration. As $I\otimes [\uvar,\partial[n]]$ is the homotopy colimit of element of shape $I\otimes [\uvar,[k]]$, the morphism $I\otimes [a,\partial[n]]\to I\otimes [b,\partial[n]]$ also is an acyclic cofibration. Now, we consider the diagram: 
% https://q.uiver.app/?q=WzAsNSxbMSwxLCJJXFxvdGltZXMgW2Esbl1cXGN1cCBbYixcXHBhcnRpYWxbbl1dIl0sWzIsMiwiSVxcb3RpbWVzICBbYixuXSJdLFsxLDAsIklcXG90aW1lcyAgW2Esbl0iXSxbMCwwLCJJXFxvdGltZXMgIFthLFxccGFydGlhbFtuXV0iXSxbMCwxLCJJXFxvdGltZXMgIFtiLFxccGFydGlhbFtuXV0iXSxbMyw0XSxbNCwwXSxbMCwxXSxbMiwwXSxbMywyXSxbMCwzLCIiLDEseyJzdHlsZSI6eyJuYW1lIjoiY29ybmVyIn19XSxbMiwxLCIiLDEseyJjdXJ2ZSI6LTJ9XSxbNCwxLCIiLDEseyJjdXJ2ZSI6Mn1dXQ==
\[\begin{tikzcd}
	{I\otimes [a,\partial[n]]} & {I\otimes [a,n]} \\
	{I\otimes [b,\partial[n]]} & {I\otimes [a,n]\cup [b,\partial[n]]} \\
	&& {I\otimes [b,n]}
	\arrow[from=1-1, to=2-1]
	\arrow[from=2-1, to=2-2]
	\arrow[from=2-2, to=3-3]
	\arrow[from=1-2, to=2-2]
	\arrow[from=1-1, to=1-2]
	\arrow["\lrcorner"{anchor=center, pos=0.125, rotate=180}, draw=none, from=2-2, to=1-1]
	\arrow[curve={height=-12pt}, from=1-2, to=3-3]
	\arrow[curve={height=12pt}, from=2-1, to=3-3]
\end{tikzcd}\]
By stability of acyclic cofibration by pushouts and by two out of three, this implies the result.
\end{proof}

\begin{lemma}
\label{lem:Gray cinder is Quillen 7}
The morphism $I\otimes E^{\cong}\to I\otimes (E^{\cong})'$ is an acyclic cofibration. 
\end{lemma}
\begin{proof}
First of all, remark that $E^{\cong}\to [0]$ is a weak equivalence in $\stratSeg(A)$. 
According to the proposition \ref{labe:Link between the Gray cylinder and cone}, we then have a commutative square:
% https://q.uiver.app/#q=WzAsNCxbMCwwLCJJXFxvdGltZXMgRV57XFxjb25nfSJdLFsxLDAsIklcXG90aW1lcyAoRV57XFxjb25nfSknIl0sWzAsMSwiW0Vee1xcY29uZ31cXG90aW1lcyBlLDFdIl0sWzEsMSwiWyhFXntcXGNvbmd9KSdcXG90aW1lcyBlLDFdIl0sWzAsMiwiXFxzaW0iLDJdLFsxLDMsIlxcc2ltIl0sWzIsMywiXFxzaW0iXSxbMCwxXV0=
\[\begin{tikzcd}
	{I\otimes E^{\cong}} & {I\otimes (E^{\cong})'} \\
	{[E^{\cong}\otimes e,1]} & {[(E^{\cong})'\otimes e,1]}
	\arrow["\sim"', from=1-1, to=2-1]
	\arrow["\sim", from=1-2, to=2-2]
	\arrow["\sim", from=2-1, to=2-2]
	\arrow[from=1-1, to=1-2]
\end{tikzcd}\]
where all arrows labelled by $\sim$ are weak equivalences. By two out of three, this implies the result.
\end{proof}

\begin{lemma}
\label{lem:Gray cinder is Quillen 8}
The morphism $ I\otimes [e,1]_t\to I\otimes e$ is a weak equivalence.
\end{lemma}
\begin{proof}
This morphism is the horizontal colimit of the diagram
% https://q.uiver.app/#q=WzAsMTAsWzAsMCwiW2UsMV1fdFxcdmVlW2UsMV0iXSxbMSwwLCJbZSwxXSJdLFsyLDAsIltbMV1fdCwxXSJdLFszLDAsIltlLDFdIl0sWzQsMCwiW2UsMV1cXHZlZVtlLDFdX3QiXSxbMCwxLCJbZSwxXSJdLFsxLDEsIltlLDFdIl0sWzIsMSwiW2UsMV0iXSxbMywxLCJbZSwxXSJdLFs0LDEsIltlLDFdIl0sWzYsNV0sWzMsNF0sWzgsOV0sWzgsN10sWzYsN10sWzEsMl0sWzMsMl0sWzAsNV0sWzEsMF0sWzEsNl0sWzIsN10sWzQsOV0sWzMsOF1d
\[\begin{tikzcd}
	{[e,1]_t\vee[e,1]} & {[e,1]} & {[[1]_t,1]} & {[e,1]} & {[e,1]\vee[e,1]_t} \\
	{[e,1]} & {[e,1]} & {[e,1]} & {[e,1]} & {[e,1]}
	\arrow[from=2-2, to=2-1]
	\arrow[from=1-4, to=1-5]
	\arrow[from=2-4, to=2-5]
	\arrow[from=2-4, to=2-3]
	\arrow[from=2-2, to=2-3]
	\arrow[from=1-2, to=1-3]
	\arrow[from=1-4, to=1-3]
	\arrow[from=1-1, to=2-1]
	\arrow[from=1-2, to=1-1]
	\arrow[from=1-2, to=2-2]
	\arrow[from=1-3, to=2-3]
	\arrow[from=1-5, to=2-5]
	\arrow[from=1-4, to=2-4]
\end{tikzcd}\]
As all the vertical morphisms are weak equivalences, and as these colimits are homotopy colimits, this concludes the proof.
\end{proof}

\begin{prop}
\label{prop:cylinder is Quillen}
The functor $I\otimes\uvar:\stratSeg(A)\to \stratSeg(A)$ is a left Quillen functor. 
\end{prop}
\begin{proof}
It is obvious that this functor preserves cofibrations. Proposition \ref{prop:Gray cinder is Quillen 5} and lemmas \ref{lem:Gray cinder is Quillen 6}, \ref{lem:Gray cinder is Quillen 7} and \ref{lem:Gray cinder is Quillen 8} imply that it sends elementary anodyne extensions, and morphisms $E^{\cong}\to (E^{\cong})'$, $[e,1]_t\to 1$ to weak equivalences. According to proposition \ref{prop:model structure on stratified Segal category}, this implies the result.
\end{proof}


\begin{cor}
\label{cor:cone is Quillen}
The functor $e\star \uvar:\stratSeg(A)\to \stratSeg(A)_{e/}$ is a left Quillen functor.
\end{cor}
\begin{proof}
First of all, it is obvious that this functor preserves cofibrations. It is then enough to show that it preserves weak equivalences.
Proposition \ref{labe:Link between the Gray cylinder and cone} implies that $e\star \uvar$ is the homotopy colimit of the diagram of functors
$e\leftarrow id \xrightarrow{i_0} I\otimes \uvar.$
Each of these functors preserves weak equivalences, and so does $e\star\uvar$.
\end{proof}
 

\section{Quillen Adjunction with $\stratSset$}
The purpose of this section is to construct a Quillen adjunction
% https://q.uiver.app/?q=WzAsMixbMCwwLCJcXHN0cmF0U3NldCJdLFsxLDAsIlxcc3RyYXRTZWcoQSkiXSxbMCwxLCIiLDEseyJvZmZzZXQiOi0yfV0sWzEsMCwiIiwxLHsib2Zmc2V0IjotMn1dLFsyLDMsIiIsMSx7ImxldmVsIjoxLCJzdHlsZSI6eyJuYW1lIjoiYWRqdW5jdGlvbiJ9fV1d
\[\begin{tikzcd}
	\stratSset & {\stratSeg(A)}
	\arrow[""{name=0, anchor=center, inner sep=0}, shift left=2, from=1-1, to=1-2]
	\arrow[""{name=1, anchor=center, inner sep=0}, shift left=2, from=1-2, to=1-1]
	\arrow["\dashv"{anchor=center, rotate=-90}, draw=none, from=0, to=1]
\end{tikzcd}\]
where the left adjoint sends $[n]$ to $e\star e\star ...\star e$. 


In section \ref{section:Cosimplicial object}, we show that this assignment extends to a left adjoint. In sections \ref{section:Complicial horn inclusion}, \ref{section:Complicial thinness extensions}, and \ref{section:Saturation extensions}, we show that this left adjoint sends complicial horn inclusions, complicial thinness extensions, and saturation extensions to weak equivalences.
\subsection{Cosimplicial object}
\label{section:Cosimplicial object}
\p 
We consider the following span:
% https://q.uiver.app/?q=WzAsMyxbMiwwLCJcXHVuZGVyc2V0e1xcRGVsdGFeMl97L1tuXX19e1xcY29saW19flxcRGVsdGFeMl97L1sxK25fMV19Il0sWzEsMCwiXFx1bmRlcnNldHtcXERlbHRhXjJfey9bbl19fXtcXGNvbGltfX5cXERlbHRhXjJfey9bbl8xXX0iXSxbMCwwLCJcXERlbHRhXjJfey9bbl19Il0sWzEsMF0sWzEsMl1d
\[\begin{tikzcd}
	{\Delta^2_{/[n]}} & {\underset{\Delta^2_{/[n]}}{\colim}~\Delta^2_{/[n_1]}} & {\underset{\Delta^2_{/[n]}}{\colim}~\Delta^2_{/[1+n_1]}}
	\arrow[from=1-2, to=1-3]
	\arrow[from=1-2, to=1-1]
\end{tikzcd}\]
where the right functor is induced by $1+\uvar:[n_1]\to[1+n_1]$ and where the left one sends an element 
$([n_0]^{op}\star [n_1]\to [n], [n_2]^{op}\star[n_3]\to [n_1])$ to the composite:
$h:[n_2]^{op}\star[n_3]\to [n_1]\to [n].$	
We define $H^2(a,n)$ as the pushout: 
% https://q.uiver.app/?q=WzAsNCxbMCwxLCJcXHVuZGVyc2V0e1xcRGVsdGFeMl97L1tuXX19e1xcY29saW19flxcdW5kZXJzZXR7XFxEZWx0YV4yX3svWzErbl8xXX19e1xcY29saW19fltbbl8yXVxcb3RpbWVzW25fMF1cXG90aW1lcyBhLDFdXFx2ZWVbW25fMF1cXG90aW1lcyBhLG5fM10iXSxbMCwwLCJcXHVuZGVyc2V0e1xcRGVsdGFeMl97L1tuXX19e1xcY29saW19flxcdW5kZXJzZXR7XFxEZWx0YV4yX3svW25fMV19fXtcXGNvbGltfX5bW25fMl1cXG90aW1lc1tuXzBdXFxvdGltZXMgYSwxXVxcdmVlW1tuXzBdXFxvdGltZXMgYSxuXzNdIl0sWzEsMCwiXFx1bmRlcnNldHtcXERlbHRhXjJfey9bbl19fXtcXGNvbGltfX5bW25fMl1cXG90aW1lcyBhLDFdXFx2ZWVbIGEsbl8zXSJdLFsxLDEsIkheMihhLG4pIl0sWzEsMF0sWzEsMl0sWzIsM10sWzAsM10sWzMsNSwiIiwyLHsibGV2ZWwiOjEsInN0eWxlIjp7Im5hbWUiOiJjb3JuZXIifX1dXQ==
\[\begin{tikzcd}
	{\underset{\Delta^2_{/[n]}}{\colim}~\underset{\Delta^2_{/[n_1]}}{\colim}~[[n_2]\otimes[n_0]\otimes a,1]\vee[[n_0]\otimes a,n_3]} & {\underset{\Delta^2_{/[n]}}{\colim}~[[n_2]\otimes a,1]\vee[ a,n_3]} \\
	{\underset{\Delta^2_{/[n]}}{\colim}~\underset{\Delta^2_{/[1+n_1]}}{\colim}~[[n_2]\otimes[n_0]\otimes a,1]\vee[[n_0]\otimes a,n_3]} & {H^2(a,n)}
	\arrow[from=1-1, to=2-1]
	\arrow[""{name=0, anchor=center, inner sep=0}, from=1-1, to=1-2]
	\arrow[from=1-2, to=2-2]
	\arrow[from=2-1, to=2-2]
	\arrow["\lrcorner"{anchor=center, pos=0.125, rotate=180}, draw=none, from=2-2, to=0]
\end{tikzcd}\]
By construction, we have a cocartesian square
% https://q.uiver.app/#q=WzAsNCxbMCwwLCJcXGNvcHJvZFxcbGltaXRzX3tsXFxsZXEgMStuXzF9XFx1bmRlcnNldHtcXERlbHRhXjJfey9bbl19fXtcXGNvbGltfX5cXHVuZGVyc2V0e1xcRGVsdGFeMl97L1xce2xcXH19fXtcXGNvbGltfX5bW25fMl1cXG90aW1lc1tuXzBdXFxvdGltZXMgYSwxXVxcdmVlW1tuXzBdXFxvdGltZXMgYSxuXzNdIl0sWzAsMSwiXFxjb3Byb2RcXGxpbWl0c197bFxcbGVxIDErbl8xfVxcdW5kZXJzZXR7XFxEZWx0YV4yX3svW25dfX17XFxjb2xpbX1+XFx1bmRlcnNldHtcXERlbHRhXjJfey9cXHtsXFx9fX17XFxjb2xpbX1+W1tuXzJdXFxvdGltZXMgZSwxXVxcdmVlW2Usbl8zXSJdLFsxLDAsIkheMihhLG4pXFxjb3Byb2RcXGxpbWl0c197SF4yKGEsXFxhbWFsZ197cFxcbGVxIG59XFx7cFxcfSl9SF4yKGUsXFxhbWFsZ197cFxcbGVxIG59XFx7cFxcfSkiXSxbMSwxLCJlXFxzdGFyIGVcXHN0YXJbYSxuXSJdLFswLDFdLFswLDJdLFsxLDNdLFsyLDNdLFszLDUsIiIsMix7ImxldmVsIjoxLCJzdHlsZSI6eyJuYW1lIjoiY29ybmVyIn19XV0=
\begin{equation}
\label{eq:lin H2 avec ee an}
\begin{tikzcd}[column sep = 0.3cm]                                                                                                                                                                                                                                                                                                                                                                                                                                                                                   
	{\coprod\limits_{l\leq 1+n_1}\underset{\Delta^2_{/[n]}}{\colim}~\underset{\Delta^2_{/\{l\}}}{\colim}~[[n_2]\otimes[n_0]\otimes a,1]\vee[[n_0]\otimes a,n_3]} & {H^2(a,n)\coprod\limits_{H^2(a,\amalg_{p\leq n}\{p\})}H^2(e,\amalg_{p\leq n}\{p\})} \\
	{\coprod\limits_{l\leq 1+n_1}\underset{\Delta^2_{/[n]}}{\colim}~\underset{\Delta^2_{/\{l\}}}{\colim}~[[n_2]\otimes e,1]\vee[e,n_3]} & {e\star e\star[a,n]}
	\arrow[from=1-1, to=2-1]
	\arrow[""{name=0, anchor=center, inner sep=0}, from=1-1, to=1-2]
	\arrow[from=2-1, to=2-2]
	\arrow[from=1-2, to=2-2]
	\arrow["\lrcorner"{anchor=center, pos=0.125, rotate=180}, draw=none, from=2-2, to=0]
\end{tikzcd}
\end{equation}
Let $x:= ([n_0]^{op}\star [n_1]\to [n], [n_2]^{op}\star[n_3]\to [1+n_1])$ be an element of $\underset{\Delta^2_{/[n]}}{\colim}~\Delta^2_{/[1+n_1]}$. We define two integers $-1\leq \tilde{n}_2\leq n_2$ and $-1\leq \tilde{n}_3\leq n_3$ as the ones fitting in the following pullbacks in $\Delta_+$
% https://q.uiver.app/?q=WzAsOCxbMiwwLCJbbl8xXSJdLFsyLDEsIlsxK25fMV0iXSxbMywxLCIgW25fMl1ee29wfVxcc3RhcltuXzNdIl0sWzEsMSwiIFtuXzJdXntvcH1cXHN0YXJbbl8zXSJdLFswLDEsIltuXzJdXntvcH0iXSxbNCwxLCJbbl8zXSJdLFswLDAsIiBbXFx0aWxkZXtufV8yXV57b3B9Il0sWzQsMCwiW1xcdGlsZGV7bn1fM10iXSxbNiwwXSxbMCwxXSxbNCwzXSxbMywxXSxbMiwxXSxbNSwyXSxbNyw1XSxbNywwXSxbNiw0XSxbNiwzLCIiLDEseyJzdHlsZSI6eyJuYW1lIjoiY29ybmVyIn19XSxbNywyLCIiLDEseyJzdHlsZSI6eyJuYW1lIjoiY29ybmVyIn19XV0=
\[\begin{tikzcd}
	{ [\tilde{n}_2]^{op}} && {[n_1]} && {[\tilde{n}_3]} \\
	{[n_2]^{op}} & { [n_2]^{op}\star[n_3]} & {[1+n_1]} & { [n_2]^{op}\star[n_3]} & {[n_3]}
	\arrow[from=1-1, to=1-3]
	\arrow[from=1-3, to=2-3]
	\arrow[from=2-1, to=2-2]
	\arrow[from=2-2, to=2-3]
	\arrow[from=2-4, to=2-3]
	\arrow[from=2-5, to=2-4]
	\arrow[from=1-5, to=2-5]
	\arrow[from=1-5, to=1-3]
	\arrow[from=1-1, to=2-1]
	\arrow["\lrcorner"{anchor=center, pos=0.125}, draw=none, from=1-1, to=2-2]
	\arrow["\lrcorner"{anchor=center, pos=0.125, rotate=-90}, draw=none, from=1-5, to=2-4]
\end{tikzcd}\]
where we set the convention $[-1]=\emptyset$. This induces a cartesian square
% https://q.uiver.app/#q=WzAsNSxbMCwxLCJbbl8wXV57b3B9XFxzdGFyW25fMV0iXSxbMSwxLCJbbl8wXV57b3B9XFxzdGFyWzErbl8xXSJdLFswLDAsIltuXzBdXntvcH1cXHN0YXIgW1xcdGlsZGV7bn1fMl1ee29wfVxcc3RhcltcXHRpbGRle259XzNdIl0sWzEsMCwiW25fMF1ee29wfVxcc3RhciBbbl8yXV57b3B9XFxzdGFyW25fM10iXSxbMCwyLCJbbl0iXSxbMiwwXSxbMCw0XSxbMCwxXSxbMiwzXSxbMywxXSxbMiw3LCIiLDEseyJsZXZlbCI6MSwic3R5bGUiOnsibmFtZSI6ImNvcm5lciJ9fV1d
\[\begin{tikzcd}
	{[n_0]^{op}\star [\tilde{n}_2]^{op}\star[\tilde{n}_3]} & {[n_0]^{op}\star [n_2]^{op}\star[n_3]} \\
	{[n_0]^{op}\star[n_1]} & {[n_0]^{op}\star[1+n_1]} \\
	{[n]}
	\arrow[from=1-1, to=2-1]
	\arrow[from=2-1, to=3-1]
	\arrow[""{name=0, anchor=center, inner sep=0}, from=2-1, to=2-2]
	\arrow[from=1-1, to=1-2]
	\arrow[from=1-2, to=2-2]
	\arrow["\lrcorner"{anchor=center, pos=0.125}, draw=none, from=1-1, to=0]
\end{tikzcd}\]

We consider the morphism $j:[n_2]\otimes [n_0]\otimes a \to ([n_2]\times [n_0])\otimes a\to ([\tilde{n}_2]\star [n_0])\otimes a$ where the right-hand morphism sends $\{(k,l)\}\otimes a$ to $(\{k\}\star \emptyset)\otimes a$ if $k\leq \tilde{n_2}$ and to $(\emptyset\star \{l\})\otimes a$ if not.
The inclusion $[\tilde{n}_3]\to [n_3]$ induces an inclusion $i:[1+\tilde{n}_3]\to [1+n_3]$. We denote $r$ the unique retraction of this inclusion that verifies $r(k) = 0$ if $k\notin Im(i)$.
Put together, $j$ and $r$ induce a morphism:
$$\psi_x:[[n_2]\otimes [n_0]\otimes a,1]\vee[[n_0]\otimes a,n_3] \to [([\tilde{n}_2]\star [n_0])\otimes a,1]\vee[a,\tilde{n}_3]$$
where we set the convention $[([\tilde{n}_2]\star [n_0])\otimes a,1]\vee[a,-1] := [0]$.

Remark that if $[n_2]^{op}\star [n_3]\to [1+n_1]$ factors through $[n_1]\to [1+n_1]$, we have $\tilde{n}_2=n_2$ and $\tilde{n}_3=n_3$, and a unique arrow fitting in a commutative triangle
% https://q.uiver.app/#q=WzAsMyxbMCwxLCJbW25fMl1cXG90aW1lcyBbbl8wXVxcb3RpbWVzIGEsMV1cXHZlZVtbbl8wXVxcb3RpbWVzIGEsbl8zXSAiXSxbMSwxLCJbKFtcXHRpbGRle259XzJdXFxzdGFyIFtuXzBdKVxcb3RpbWVzIGEsMV1cXHZlZVthLFxcdGlsZGV7bn1fM10iXSxbMSwwLCJbKFtcXHRpbGRle259XzJdXFxzdGFyIFxcZW1wdHlzZXQpXFxvdGltZXMgYSwxXVxcdmVlW2EsXFx0aWxkZXtufV8zXSJdLFswLDEsIlxccHNpX3giLDJdLFsyLDFdLFswLDIsIiIsMix7InN0eWxlIjp7ImJvZHkiOnsibmFtZSI6ImRhc2hlZCJ9fX1dXQ==
\[\begin{tikzcd}
	& {[([\tilde{n}_2]\star \emptyset)\otimes a,1]\vee[a,\tilde{n}_3]} \\
	{[[n_2]\otimes [n_0]\otimes a,1]\vee[[n_0]\otimes a,n_3] } & {[([\tilde{n}_2]\star [n_0])\otimes a,1]\vee[a,\tilde{n}_3]}
	\arrow["{\psi_x}"', from=2-1, to=2-2]
	\arrow[from=1-2, to=2-2]
	\arrow[dashed, from=2-1, to=1-2]
\end{tikzcd}\]


Considering the canonical morphism
$$[([\tilde{n}_2]\star [n_0])\otimes a,1]\vee[a,\tilde{n}_3]\to e\star[a,n]$$
if $\tilde{n}_3\geq 0$ (coming from the fact that $([n_0]^{op}\star [\tilde{n}_2]^{op})\star[\tilde{n}_3]\to [n]$ is an element of $\Delta^2_{[n]}$),
and the morphism 
$$[([\tilde{n}_2]\star [n_0])\otimes a,1]\vee[a,\tilde{n}_3]\to e\star \emptyset\to e\star[a,n]$$
if $\tilde{n}_3=-1$, this induces a natural transformation 
$$H^{s^0}(a,n):H^2(a,n)\to e\star[a,n]$$
induced by $\psi_{\uvar}$ on $\underset{\Delta^2_{/[n]}}{\colim}~\underset{\Delta^2_{/[1+n_1]}}{\colim}~[[n_2]\otimes[n_0]\otimes a,1]\vee[[n_0]\otimes a,n_3]$ 
and by the identity on $\underset{\Delta^2_{/[n]}}{\colim}~[[n_2]\otimes a,1]\vee[ a,n_3]$.

By construction, if $[n_0]^{op}\star [n_1]\to [n]$ factor through $\{p\}$ for $p\leq n$ we have a commutative diagram
% https://q.uiver.app/#q=WzAsNSxbMCwwLCJbW25fMl1cXG90aW1lc1tuXzBdXFxvdGltZXMgYSwxXVxcdmVlW1tuXzBdXFxvdGltZXMgYSxuXzNdIl0sWzAsMSwiW1tuXzJdXFxvdGltZXNbbl8wXVxcb3RpbWVzIGUsMV1cXHZlZVtbbl8wXVxcb3RpbWVzIGUsbl8zXSJdLFsyLDAsIkheMihhLG4pIl0sWzIsMSwiIGVcXHN0YXJbYSxuXSJdLFsxLDEsImVcXHN0YXJcXHtwXFx9Il0sWzAsMV0sWzAsMl0sWzIsM10sWzEsNF0sWzQsM11d
\[\begin{tikzcd}
	{[[n_2]\otimes[n_0]\otimes a,1]\vee[[n_0]\otimes a,n_3]} && {H^2(a,n)} \\
	{[[n_2]\otimes[n_0]\otimes e,1]\vee[[n_0]\otimes e,n_3]} & {e\star\{p\}} & { e\star[a,n]}
	\arrow[from=1-1, to=2-1]
	\arrow[from=1-1, to=1-3]
	\arrow[from=1-3, to=2-3]
	\arrow[from=2-1, to=2-2]
	\arrow[from=2-2, to=2-3]
\end{tikzcd}\]
If $[n_2]^{op}\star[n_3]\to [1+n_1]$ factors through $\{0\}$, $\tilde{n}_3$ is equal to $-1$, and we have a commutative diagram
% https://q.uiver.app/#q=WzAsNSxbMCwwLCJbW25fMl1cXG90aW1lc1tuXzBdXFxvdGltZXMgYSwxXVxcdmVlW1tuXzBdXFxvdGltZXMgYSxuXzNdIl0sWzAsMSwiW1tuXzJdXFxvdGltZXMgZSwxXVxcdmVlW2Usbl8zXSJdLFsyLDAsIkheMihhLG4pIl0sWzEsMSwiZVxcc3RhclxcZW1wdHlzZXQiXSxbMiwxLCIgZVxcc3RhclthLG5dIl0sWzAsMV0sWzAsMl0sWzEsM10sWzIsNF0sWzMsNF1d
\[\begin{tikzcd}
	{[[n_2]\otimes[n_0]\otimes a,1]\vee[[n_0]\otimes a,n_3]} && {H^2(a,n)} \\
	{[[n_2]\otimes e,1]\vee[e,n_3]} & e\star\emptyset & { e\star[a,n]}
	\arrow[from=1-1, to=2-1]
	\arrow[from=1-1, to=1-3]
	\arrow[from=2-1, to=2-2]
	\arrow[from=1-3, to=2-3]
	\arrow[from=2-2, to=2-3]
\end{tikzcd}\]
and if  $[n_2]^{op}\star[n_3]\to [1+n_1]$ factors through any other point, $\tilde{n}_3$ is equal to $0$, and we have a commutative diagram
% https://q.uiver.app/#q=WzAsNSxbMCwwLCJbW25fMl1cXG90aW1lc1tuXzBdXFxvdGltZXMgYSwxXVxcdmVlW1tuXzBdXFxvdGltZXMgYSxuXzNdIl0sWzAsMSwiW1tuXzJdXFxvdGltZXMgZSwxXVxcdmVlW2Usbl8zXSJdLFsyLDAsIkheMihhLG4pIl0sWzEsMSwiZVxcc3Rhclxce2tcXH0iXSxbMiwxLCIgZVxcc3RhclthLG5dIl0sWzAsMV0sWzAsMl0sWzEsM10sWzIsNF0sWzMsNF1d
\[\begin{tikzcd}
	{[[n_2]\otimes[n_0]\otimes a,1]\vee[[n_0]\otimes a,n_3]} && {H^2(a,n)} \\
	{[[n_2]\otimes e,1]\vee[e,n_3]} & {e\star\{k\}} & { e\star[a,n]}
	\arrow[from=1-1, to=2-1]
	\arrow[from=1-1, to=1-3]
	\arrow[from=2-1, to=2-2]
	\arrow[from=1-3, to=2-3]
	\arrow[from=2-2, to=2-3]
\end{tikzcd}\]
where $k$ is the image of the composite morphism $[\tilde{n}_2]^{op}\star[\tilde{n}_3]\to [n_1]\to[n]$.
The cocartesian square \eqref{eq:lin H2 avec ee an} then implies that $H^2(a,n)$ 
lifts to a natural transformation 
$$s^0\star [a,n]: e\star e\star [a,n]\to e \star [a,n].$$
By extension by colimits, this induces a natural transformation
$$C\mapsto \big(s^0\star C: e\star e\star C\to e\star C\big).$$

To define the cosimplicial object, we will need to show the commutativity of several diagrams whose initial objects are of shape $e\star..\star e\star [a,n]$. To this extend, it is enough to find coverings of these objects by easier one, and to show that the induced diagrams commute. 
\begin{lemma}
\label{lem:cover of n star a}
We set $\Pi^0_{/[n]} := \Delta^{2}_{/[n]}$ and 
$$\Pi^{k+1}_{/[n]}:= \colim_{\Delta^2_{/[n]}} \colim_{\Delta^2_{/[n_1+1]}}...\colim_{\Delta^2_{/[n_{2k-1}+1]}}\Delta^2_{/[n_{2k+1}+1]}$$
There is an epimorphism:
$$\colim_{\Pi^k_{/[n]}\times A}[[n_{2k}]\otimes [n_{2k-2}]\otimes...\otimes [n_0]\otimes a,1+ n_{2k-1}]\to \underbrace{e\star e\star ... \star e}_{k+1}\star [a,n]$$
\end{lemma}
\begin{proof}
This is an easy proof by induction, after remarking that $${[[n_0]\otimes a, 1+n_1]\to [[n_0]\otimes a, 1]\vee[a,n_1]}$$
 is an epimorphism.
\end{proof}

\begin{lemma}
\label{lemma:monoid 1}
The following triangles commute:
% https://q.uiver.app/?q=WzAsNCxbMSwwLCJlXFxzdGFyIGVcXHN0YXJbYSxuXSJdLFsyLDAsImVcXHN0YXIgW2Esbl0iXSxbMCwwLCJlXFxzdGFyW2Esbl0iXSxbMSwxLCJlXFxzdGFyW2Esbl0iXSxbMiwwLCJkXnswfVxcc3RhcntlXFxzdGFyW2Esbl19Il0sWzAsMywic14wXFxzdGFye1thLG5dfSJdLFsyLDMsImlkIiwyXSxbMSwzLCJpZCJdLFsxLDAsImVcXHN0YXIgZF57MH1cXHN0YXJ7W2Esbl19IiwyXV0=
\[\begin{tikzcd}
	{e\star[a,n]} & {e\star e\star[a,n]} & {e\star [a,n]} \\
	& {e\star[a,n]}
	\arrow["{d^{0}\star{e\star[a,n]}}", from=1-1, to=1-2]
	\arrow["{s^0\star{[a,n]}}", from=1-2, to=2-2]
	\arrow["id"', from=1-1, to=2-2]
	\arrow["id", from=1-3, to=2-2]
	\arrow["{e\star d^{0}\star{[a,n]}}"', from=1-3, to=1-2]
\end{tikzcd}\]
\end{lemma}
\begin{proof}
We will prove only the left triangle and we leave the other to the reader. Let $x:= ([n_0]^{op}\star [n_1]\to [n], [n_2]^{op}\star[n_3]\to [1+n_1])$ be an element of $\colim_{\Delta^2_{/[n]}}~\Delta^2_{/[1+n_1]}$. We have a diagram:
% https://q.uiver.app/?q=WzAsNixbMiwwLCJlXFxzdGFyIGVcXHN0YXJbYSxuXSJdLFsxLDAsImVcXHN0YXJbYSxuXSJdLFsyLDEsImVcXHN0YXJbYSxuXSJdLFswLDEsIltbbl8wXVxcb3RpbWVzIGEsIDErbl8xXSJdLFsxLDEsIltbMF1cXG90aW1lc1tuXzBdXFxvdGltZXMgYSwgMV1cXHZlZVtbbl8wXVxcb3RpbWVzIGEsIDErbl8xXSJdLFsxLDIsIltbbl8wXVxcb3RpbWVzIGEsIDErbl8xXSJdLFsxLDAsImReMFxcc3RhcntlXFxzdGFyW2Esbl19Il0sWzAsMiwic14wXFxzdGFye1thLG5dfSJdLFszLDFdLFszLDQsIltbbl8wXVxcb3RpbWVzIGEsIGReMF0iXSxbNCwwXSxbNCw1LCJcXHBzaV94Il0sWzUsMl0sWzMsNSwiaWQiLDJdLFsxLDIsIiIsMCx7InN0eWxlIjp7ImJvZHkiOnsibmFtZSI6ImRvdHRlZCJ9fX1dXQ==
\[\begin{tikzcd}
	& {e\star[a,n]} & {e\star e\star[a,n]} \\
	{[[n_0]\otimes a, 1+n_1]} & {[[0]\otimes[n_0]\otimes a, 1]\vee[[n_0]\otimes a, 1+n_1]} & {e\star[a,n]} \\
	& {[[n_0]\otimes a, 1+n_1]}
	\arrow["{d^0\star{e\star[a,n]}}", from=1-2, to=1-3]
	\arrow["{s^0\star{[a,n]}}", from=1-3, to=2-3]
	\arrow[from=2-1, to=1-2]
	\arrow["{[[n_0]\otimes a, d^0]}", from=2-1, to=2-2]
	\arrow[from=2-2, to=1-3]
	\arrow["{\psi_x}", from=2-2, to=3-2]
	\arrow[from=3-2, to=2-3]
	\arrow["id"', from=2-1, to=3-2]
	\arrow[dotted, from=1-2, to=2-3]
\end{tikzcd}\]
where we know that everything except the right triangle commutes. As this is true for any $x$, lemma \ref{lem:cover of n star a} implies the desired commutativity.
\end{proof}





\begin{lemma}
\label{lemma:monoid 2}
The following square commutes 
% https://q.uiver.app/?q=WzAsNCxbMCwwLCJlXFxzdGFyIGVcXHN0YXIgZVxcc3RhclthLG5dIl0sWzEsMCwiZVxcc3RhciBlXFxzdGFyW2Esbl0iXSxbMCwxLCJlXFxzdGFyIGVcXHN0YXJbYSxuXSJdLFsxLDEsImVcXHN0YXJbYSxuXSJdLFswLDIsImVcXHN0YXIgc14xXFxzdGFye1thLG5dfSIsMl0sWzIsMywic14xXFxzdGFye1thLG5dfSIsMl0sWzAsMSwic14xXFxzdGFye2VcXHN0YXJbYSxuXX0iXSxbMSwzLCJzXjFcXHN0YXJ7W2Esbl19Il1d
\[\begin{tikzcd}
	{e\star e\star e\star[a,n]} & {e\star e\star[a,n]} \\
	{e\star e\star[a,n]} & {e\star[a,n]}
	\arrow["{e\star s^1\star{[a,n]}}"', from=1-1, to=2-1]
	\arrow["{s^1\star{[a,n]}}"', from=2-1, to=2-2]
	\arrow["{s^1\star{e\star[a,n]}}", from=1-1, to=1-2]
	\arrow["{s^1\star{[a,n]}}", from=1-2, to=2-2]
\end{tikzcd}\]
\end{lemma}
\begin{proof}
Let 
$x = (f:[n_0]^{op}\star[n_1]\to[n],g:[n_2]^{op}\star[n_3]\to[1+n_1],h:[n_4]^{op}\star[n_5]\to[n_3+1]) $ be an object of $\Pi^2_k$. We define integers $-1\leq \bar{n}_4\leq n_4$ and $-1\leq \bar{n}_5\leq n_5$ as the one fitting in the following pullbacks in $\Delta_+$.
% https://q.uiver.app/?q=WzAsNixbMSwwLCJbMStcXHRpbGRle259XzNdIl0sWzAsMSwiW25fNF1ee29wfSJdLFsxLDEsIiBbMStuXzNdIl0sWzAsMCwiW1xcYmFye259XzRdXntvcH0iXSxbMiwwLCJbXFxiYXJ7bn1fNV0iXSxbMiwxLCJbbl81XSJdLFswLDJdLFsxLDJdLFszLDBdLFszLDFdLFs0LDVdLFs1LDJdLFs0LDBdLFs0LDIsIiIsMSx7InN0eWxlIjp7Im5hbWUiOiJjb3JuZXIifX1dLFszLDYsIiIsMSx7ImxldmVsIjoxLCJzdHlsZSI6eyJuYW1lIjoiY29ybmVyIn19XV0=
\[\begin{tikzcd}
	{[\bar{n}_4]^{op}} & {[1+\tilde{n}_3]} & {[\bar{n}_5]} \\
	{[n_4]^{op}} & { [1+n_3]} & {[n_5]}
	\arrow[""{name=0, anchor=center, inner sep=0}, from=1-2, to=2-2]
	\arrow[from=2-1, to=2-2]
	\arrow[from=1-1, to=1-2]
	\arrow[from=1-1, to=2-1]
	\arrow[from=1-3, to=2-3]
	\arrow[from=2-3, to=2-2]
	\arrow[from=1-3, to=1-2]
	\arrow["\lrcorner"{anchor=center, pos=0.125, rotate=-90}, draw=none, from=1-3, to=2-2]
	\arrow["\lrcorner"{anchor=center, pos=0.125}, draw=none, from=1-1, to=0]
\end{tikzcd}\]
This induces cartesian squares
% https://q.uiver.app/#q=WzAsNixbMCwxLCJbbl8wXV57b3B9XFxzdGFyW1xcdGlsZGV7bn1fMl1ee29wfVxcc3RhcltcXHRpbGRle259XzNdIl0sWzIsMSwiW25fMF1ee29wfVxcc3RhcltuXzJdXntvcH1cXHN0YXJbMSsgbl8zXSJdLFsxLDEsIltuXzBdXntvcH1cXHN0YXJbXFx0aWxkZXtufV8yXV57b3B9XFxzdGFyWzErIFxcdGlsZGV7bn1fM10iXSxbMiwwLCJbbl8wXV57b3B9XFxzdGFyW25fMl1ee29wfVxcc3RhcltuXzRdXntvcH1cXHN0YXJbbl81XSJdLFsxLDAsIltuXzBdXntvcH1cXHN0YXJbXFx0aWxkZXtufV8yXV57b3B9XFxzdGFyW1xcYmFye259XzRdXntvcH1cXHN0YXJbXFxiYXJ7bn1fNV0iXSxbMCwwLCJbbl8wXV57b3B9XFxzdGFyW1xcdGlsZGV7bn1fMl1ee29wfVxcc3RhcltcXHRpbGRle1xcYmFye259fV80XV57b3B9XFxzdGFyW1xcdGlsZGV7XFxiYXJ7bn19XzVdIl0sWzIsMV0sWzAsMl0sWzUsMF0sWzUsNF0sWzQsMl0sWzMsMV0sWzQsM11d
\[\begin{tikzcd}[column sep=0.5cm]
	{[n_0]^{op}\star[\tilde{n}_2]^{op}\star[\tilde{\bar{n}}_4]^{op}\star[\tilde{\bar{n}}_5]} & {[n_0]^{op}\star[\tilde{n}_2]^{op}\star[\bar{n}_4]^{op}\star[\bar{n}_5]} & {[n_0]^{op}\star[n_2]^{op}\star[n_4]^{op}\star[n_5]} \\
	{[n_0]^{op}\star[\tilde{n}_2]^{op}\star[\tilde{n}_3]} & {[n_0]^{op}\star[\tilde{n}_2]^{op}\star[1+ \tilde{n}_3]} & {[n_0]^{op}\star[n_2]^{op}\star[1+ n_3]}
	\arrow[from=2-2, to=2-3]
	\arrow[from=2-1, to=2-2]
	\arrow[from=1-1, to=2-1]
	\arrow[from=1-1, to=1-2]
	\arrow[from=1-2, to=2-2]
	\arrow[from=1-3, to=2-3]
	\arrow[from=1-2, to=1-3]
\end{tikzcd}\]
The outer squares fits in the following cartesian squares:
% https://q.uiver.app/#q=WzAsOSxbMCwyLCJbbl8wXV57b3B9XFxzdGFyW25fMV0iXSxbMCwzLCJbbl0iXSxbMSwyLCJbbl8wXV57b3B9XFxzdGFyWzErbl8xXSJdLFsxLDEsIltuXzBdXntvcH1cXHN0YXJbbl8yXV57b3B9XFxzdGFyW25fM10iXSxbMiwxLCJbbl8wXV57b3B9XFxzdGFyW25fMl1ee29wfVxcc3RhclsxKyBuXzNdIl0sWzIsMCwiW25fMF1ee29wfVxcc3RhcltuXzJdXntvcH1cXHN0YXJbbl80XV57b3B9XFxzdGFyW25fNV0iXSxbMCwxLCJbbl8wXV57b3B9XFxzdGFyW1xcdGlsZGV7bn1fMl1ee29wfVxcc3RhcltcXHRpbGRle259XzNdIl0sWzEsMCwiW25fMF1ee29wfVxcc3RhcltuXzJdXntvcH1cXHN0YXJbXFx0aWxkZXtufV80XV57b3B9XFxzdGFyW1xcdGlsZGV7bn1fNV0iXSxbMCwwLCJbbl8wXV57b3B9XFxzdGFyW1xcdGlsZGV7bn1fMl1ee29wfVxcc3RhcltcXHRpbGRle1xcYmFye259fV80XV57b3B9XFxzdGFyW1xcdGlsZGV7XFxiYXJ7bn19XzVdIl0sWzMsNF0sWzYsM10sWzAsMl0sWzAsMV0sWzYsMF0sWzMsMl0sWzUsNF0sWzgsNl0sWzcsM10sWzgsN10sWzcsNV1d
\[\begin{tikzcd}[column sep=0.5cm]
	{[n_0]^{op}\star[\tilde{n}_2]^{op}\star[\tilde{\bar{n}}_4]^{op}\star[\tilde{\bar{n}}_5]} & {[n_0]^{op}\star[n_2]^{op}\star[\tilde{n}_4]^{op}\star[\tilde{n}_5]} & {[n_0]^{op}\star[n_2]^{op}\star[n_4]^{op}\star[n_5]} \\
	{[n_0]^{op}\star[\tilde{n}_2]^{op}\star[\tilde{n}_3]} & {[n_0]^{op}\star[n_2]^{op}\star[n_3]} & {[n_0]^{op}\star[n_2]^{op}\star[1+ n_3]} \\
	{[n_0]^{op}\star[n_1]} & {[n_0]^{op}\star[1+n_1]} \\
	{[n]}
	\arrow[from=2-2, to=2-3]
	\arrow[from=2-1, to=2-2]
	\arrow[from=3-1, to=3-2]
	\arrow[from=3-1, to=4-1]
	\arrow[from=2-1, to=3-1]
	\arrow[from=2-2, to=3-2]
	\arrow[from=1-3, to=2-3]
	\arrow[from=1-1, to=2-1]
	\arrow[from=1-2, to=2-2]
	\arrow[from=1-1, to=1-2]
	\arrow[from=1-2, to=1-3]
\end{tikzcd}\]
This induces a diagram:
% https://q.uiver.app/#q=WzAsOCxbMCwyLCJbW25fezR9XVxcb3RpbWVzICBbbl97Mn1dXFxvdGltZXMgW25fMF1cXG90aW1lcyBhLDErIG5fezV9XSJdLFsxLDIsIlsoW1xcdGlsZGV7bn1fezR9XVxcc3RhciAgW25fezJ9XSlcXG90aW1lcyBbbl8wXVxcb3RpbWVzIGEsMSsgXFx0aWxkZXtufV97NX1dIl0sWzEsMywiW1tcXHRpbGRle1xcYmFye259fV80XVxcc3RhciAgW1xcdGlsZGV7bn1fezJ9XVxcc3RhciBbbl8wXVxcb3RpbWVzIGEsMSsgXFx0aWxkZXtcXGJhcntufX1fNV0iXSxbMCwzLCJbW1xcYmFye259X3s0fV1cXG90aW1lcyAoIFtcXHRpbGRle259X3syfV1cXHN0YXIgW25fMF0pXFxvdGltZXMgYSwxKyBcXGJhcntufV97NX1dIl0sWzEsMCwiZVxcc3RhciBlXFxzdGFyIGVcXHN0YXJbYSxuXSJdLFsyLDAsImVcXHN0YXIgZVxcc3RhclthLG5dIl0sWzEsMSwiZVxcc3RhciBlXFxzdGFyW2Esbl0iXSxbMiwxLCJlXFxzdGFyW2Esbl0iXSxbMywyXSxbMCwzXSxbMSwyXSxbMCwxXSxbNSw3LCJzXjFcXHN0YXJ7W2Esbl19Il0sWzYsNywic14xXFxzdGFye1thLG5dfSIsMix7InN0eWxlIjp7ImJvZHkiOnsibmFtZSI6ImRvdHRlZCJ9fX1dLFs0LDUsInNeMVxcc3RhcntlXFxzdGFyW2Esbl19Il0sWzQsNiwiZVxcc3RhciBzXjFcXHN0YXJ7W2Esbl19Il0sWzAsNF0sWzEsNV0sWzIsN10sWzMsNiwiIiwxLHsic3R5bGUiOnsiYm9keSI6eyJuYW1lIjoiZG90dGVkIn19fV1d
\[\begin{tikzcd}[column sep=0.5cm]
	& {e\star e\star e\star[a,n]} & {e\star e\star[a,n]} \\
	& {e\star e\star[a,n]} & {e\star[a,n]} \\
	{[[n_{4}]\otimes [n_{2}]\otimes [n_0]\otimes a,1+ n_{5}]} & {[([\tilde{n}_{4}]\star [n_{2}])\otimes [n_0]\otimes a,1+ \tilde{n}_{5}]} \\
	{[[\bar{n}_{4}]\otimes ( [\tilde{n}_{2}]\star [n_0])\otimes a,1+ \bar{n}_{5}]} & {[[\tilde{\bar{n}}_4]\star [\tilde{n}_{2}]\star [n_0]\otimes a,1+ \tilde{\bar{n}}_5]}
	\arrow[from=4-1, to=4-2]
	\arrow[from=3-1, to=4-1]
	\arrow[from=3-2, to=4-2]
	\arrow[from=3-1, to=3-2]
	\arrow["{s^1\star{[a,n]}}", from=1-3, to=2-3]
	\arrow["{s^1\star{[a,n]}}"', dotted, from=2-2, to=2-3]
	\arrow["{s^1\star{e\star[a,n]}}", from=1-2, to=1-3]
	\arrow["{e\star s^1\star{[a,n]}}", from=1-2, to=2-2]
	\arrow[from=3-1, to=1-2]
	\arrow[from=3-2, to=1-3]
	\arrow[from=4-2, to=2-3]
	\arrow[dotted, from=4-1, to=2-2]
\end{tikzcd}\]
where we know that everything except the behind square commutes. As this is true for any $x$, lemma \ref{lem:cover of n star a} implies the desired commutativity.

\end{proof}

\begin{definition}
For $k\leq 1$, the \wcsnotionsym{intelligent $k$-truncation functor}{(taui@$\tau^i_n$}{truncation@$n$-truncation}{for stratified Segal $A$-precategories}, noted by $\tau^i_k$, is the colimit preserving functor such that $\tau^i_k([a,n]) = [\tau^i_{k-1}(a),n]$ and $\tau^i_k[e,1]_t = [e,1]_t$. The intelligent \textit{$0$-truncation functor}, denoted by $\tau^i_0$, is the colimit preserving functor such that $\tau^i_0([a,n])$ fits in the following pushout 
% https://q.uiver.app/?q==
\[\begin{tikzcd}
	{\underset{ob(a)\times Hom([1],[n])}{\coprod}[e,1]} & {[\tau^i_0(a),n]} \\
	{\underset{ob(a)\times Hom([1],[n])}{\coprod}[e,1]_t} & {\tau^i_0([a,n])}
	\arrow[from=1-1, to=1-2]
	\arrow[from=1-2, to=2-2]
	\arrow[from=1-1, to=2-1]
	\arrow[from=2-1, to=2-2]
\end{tikzcd}\]
and such that $\tau^i_0[e,1]_t = [e,1]_t$.
As the intelligent $k$-truncations on $A$ are left Quillen, the intelligent $k$-truncations on $\stratSeg(A)$ preserve generating Reedy cofibrations and Segal extensions. It is staightforward that they also send $[e,1]_t\to [0]$ and $E^{\cong}\to (E^{\cong})'$ to weak equivalences. According to theorem \ref{prop:model structure on stratified Segal category}, they are left Quillen functors. 
\end{definition}

\p 
\label{para:definition of the cosimplicial object}
The (inverted) composition $g,f\mapsto g\circ f$ is a monoidal structure on the category of endomorphisms of $\stratSeg(A)$. Lemmas \ref{lemma:monoid 1} and \ref{lemma:monoid 2} show that $e\star \uvar$ is a monoid for this monoidal structure. This induces a cosimplicial object: 
$$\begin{array}{rcl}
\Delta &\to & \End(\stratSeg(A))\\
~[n] &\mapsto & [n]\star\uvar :=\underbrace{e\star e\star...\star e}_{n+1}\star \uvar
\end{array}$$
We extend this functor to $\Delta_t$ in setting for a stratified Segal $A$-precategory $C$ and an integer $n>0$:
% https://q.uiver.app/?q=WzAsNCxbMSwxLCJbbl1fdFxcc3RhciBDIl0sWzEsMCwiW25dXFxzdGFyIEMiXSxbMCwwLCJcXHVuZGVyc2V0e2tcXGdlcSAtMX17XFxjb3Byb2R9fn5cXHVuZGVyc2V0e0QsflxcdGF1XmlfayhEKT1EfXtcXGNvcHJvZH1+flxcdW5kZXJzZXR7RFxcdG8gQ317XFxjb3Byb2R9W25dXFxzdGFyICBEIl0sWzAsMSwiXFx1bmRlcnNldHtrXFxnZXEgLTF9e1xcY29wcm9kfX5+XFx1bmRlcnNldHtELH5cXHRhdV5pX2soRCk9RH17XFxjb3Byb2R9fn5cXHVuZGVyc2V0e0RcXHRvIEN9e1xcY29wcm9kfVxcdGF1Xmlfe24ra30oW25dXFxzdGFyIEQpIl0sWzMsMF0sWzIsMV0sWzEsMF0sWzIsM10sWzAsNSwiIiwxLHsibGV2ZWwiOjEsInN0eWxlIjp7Im5hbWUiOiJjb3JuZXIifX1dXQ==
\[\begin{tikzcd}
	{\underset{k\geq -1}{\coprod}~~\underset{D,~\tau^i_k(D)=D}{\coprod}~~\underset{D\to C}{\coprod}[n]\star D} & {[n]\star C} \\
	{\underset{k\geq -1}{\coprod}~~\underset{D,~\tau^i_k(D)=D}{\coprod}~~\underset{D\to C}{\coprod}\tau^i_{n+k}([n]\star D)} & {[n]_t\star C}
	\arrow[from=2-1, to=2-2]
	\arrow[""{name=0, anchor=center, inner sep=0}, from=1-1, to=1-2]
	\arrow[from=1-2, to=2-2]
	\arrow[from=1-1, to=2-1]
	\arrow["\lrcorner"{anchor=center, pos=0.125, rotate=180}, draw=none, from=2-2, to=0]
\end{tikzcd}\]
where $\tau^i_{-1}$ is the constant functor with value $\emptyset$.
Evaluated on the empty Segal $A$-category, and by extension under colimits, this gives a functor 
\begin{equation}
\stratSset\to \stratSeg(A).
\end{equation}
The image of $[n]$ (resp. $[n]_t$) is also noted by $[n]$ (resp. $[n]_t$).


By construction, for $K,L$ two stratified sets and $D$ a stratified Segal $A$-precategory, we have $K\star (L\star C)\cong (K\star L)\star C$.



\begin{lemma}
\label{lemma:leibnizt joint is Quillen}
Let $K$ be a stratified simplicial set. The morphism $K\star\uvar$ is a left Quillen functor. 
Moreover, if
 $i$ is a cofibration of stratified simplicial sets and $g$ an acyclic cofibration of stratified Segal $A$-precategories, the morphism $i\hstar g$ is an acyclic cofibration. 
\end{lemma}
\begin{proof}
As every simplicial set is a homotopy colimit of representables and as $\star$ preserves monomorphisms, it is enough to show the first assertion for $K= [n]$. In this case, this is a repeated application of the corollary \ref{cor:cone is Quillen}.
By diagram chasing and the use of two out of three, this implies the second assertion.
\end{proof}


\subsection{Complicial horn inclusions}
\label{section:Complicial horn inclusion}

\begin{notation*}
In this section, we will often consider morphisms $\tilde{a}\to \tilde{b}$ that fit into cocartesian squares:
% https://q.uiver.app/#q=WzAsNCxbMCwwLCJhIl0sWzEsMCwiYiJdLFswLDEsIlxcdGlsZGV7YX0iXSxbMSwxLCJcXHRpbGRle2J9Il0sWzAsMl0sWzAsMSwiaSJdLFsyLDNdLFsxLDNdLFswLDMsIiIsMix7InN0eWxlIjp7Im5hbWUiOiJjb3JuZXIifX1dXQ==
\[\begin{tikzcd}
	a & b \\
	{\tilde{a}} & {\tilde{b}}
	\arrow[from=1-1, to=2-1]
	\arrow["i", from=1-1, to=1-2]
	\arrow[from=2-1, to=2-2]
	\arrow[from=1-2, to=2-2]
	\arrow["\lrcorner"{anchor=center, pos=0.125}, draw=none, from=1-1, to=2-2]
\end{tikzcd}\]
where $a\to \tilde{a}$ and $b\to \tilde{b}$ are epimorphisms.
To avoid complicating the notations unnecessarily, the induced morphism $\tilde{a}\to \tilde{b}$ will just be denoted $i$.
\end{notation*}

\p 
\label{para:marked segal}
A \notion{marked Segal $A$-precategory} is a stratified Segal $A$-precategory having the right lifting property against all entire acyclic cofibrations. We denote by \wcnotation{$\mSeg(A)$}{(mseg@$\mSeg(A)$} the full subcategory of marked Segal $A$-precategory. We then have an adjunction: 
% https://q.uiver.app/#q=WzAsMixbMSwwLCJcXG1TZWcoQSk6XFxpb3RhIl0sWzAsMCwiKFxcdXZhcilfe1xcbWt9Olxcc3RyYXRTZWcoQSkiXSxbMCwxLCIiLDAseyJvZmZzZXQiOi0yfV0sWzEsMCwiIiwwLHsib2Zmc2V0IjotMn1dLFszLDIsIiIsMCx7ImxldmVsIjoxLCJzdHlsZSI6eyJuYW1lIjoiYWRqdW5jdGlvbiJ9fV1d
\[\begin{tikzcd}
	{(\uvar)_{\mk}:\stratSeg(A)} & {\mSeg(A):\iota}
	\arrow[""{name=0, anchor=center, inner sep=0}, shift left=2, from=1-2, to=1-1]
	\arrow[""{name=1, anchor=center, inner sep=0}, shift left=2, from=1-1, to=1-2]
	\arrow["\dashv"{anchor=center, rotate=-90}, draw=none, from=1, to=0]
\end{tikzcd}\]
where the left adjoint $(\uvar)_{\mk}$ sends a stratified Segal $A$-precategory $(C,tC)$ to the marked Segal $A$-precategory $(C,\overline{tC})$, where $\overline{tC}$ is the smaller stratification that includes $tC$ and makes $(C,\overline{tC})$ a marked Segal $A$-precategory, and where the right adjoint is a fully faithful inclusion.
Remark furthermore that at the level of preshaves, these two adjoints are the identity. We denote \wcnotation{$r_C:C\to C_{\mk}$}{(rc@$r_C:C\to C_{\mk}$} the canonical inclusion. The proposition \ref{prop:X to Xmk is acycli cof} states that $r_C$ is an entire acyclic cofibration.

There is an isomorphism $(e\star C_{\mk})_{\mk}\cong (e\star C)_{\mk}$. Indeed $e\star\uvar$ preserves both entire cofibrations and weak equivalences, we have two entire acyclic cofibration $e\star C\to (e\star C)_{\mk}$ and $e \star C\to (e\star C_{\mk})_{\mk}$.  As the two codomain are marked, they are isomorphic. 


The fact that will be used the most with the marked Segal $A$-precategory is their right lifting property with respect to morphisms of shape $[\tau^i_n(a),\Lambda^1[2]]\cup [a,2]\to [\tau^i_n(a),2]$. This fact will  be used freely.

\p We recall that $[2]\botimes a$ is the following pushout:
% https://q.uiver.app/#q=WzAsNCxbMiwwLCJbMl1cXG90aW1lcyBhIl0sWzAsMCwiWzFdXFxvdGltZXMgYVxcYW1hbGcgWzFdXFxvdGltZXMgYSJdLFswLDEsImVcXHN0YXIgYVxcYW1hbGcgZVxcc3RhciBhIl0sWzIsMSwiWzJdXFxib3RpbWVzIGEiXSxbMSwwLCJkXjFcXG90aW1lcyBhXFxhbWFsZyBkXjJcXG90aW1lcyBhIl0sWzIsMywiZF4xXFxib3RpbWVzIGFcXGFtYWxnIGReMlxcYm90aW1lcyBhIiwyXSxbMSwyXSxbMCwzXSxbMyw0LCIiLDEseyJsZXZlbCI6MSwic3R5bGUiOnsibmFtZSI6ImNvcm5lciJ9fV1d
\[\begin{tikzcd}
	{[1]\otimes a\amalg [1]\otimes a} && {[2]\otimes a} \\
	{e\star a\amalg e\star a} && {[2]\botimes a}
	\arrow[""{name=0, anchor=center, inner sep=0}, "{d^1\otimes a\amalg d^2\otimes a}", from=1-1, to=1-3]
	\arrow["{d^1\botimes a\amalg d^2\botimes a}"', from=2-1, to=2-3]
	\arrow[from=1-1, to=2-1]
	\arrow[from=1-3, to=2-3]
	\arrow["\lrcorner"{anchor=center, pos=0.125, rotate=180}, draw=none, from=2-3, to=0]
\end{tikzcd}\]
We define $[e,1]\vee(e\star[a,1])$ as the colimit of the following diagram
% https://q.uiver.app/#q=WzAsMyxbMCwwLCJbZSwxXVxcdmVlW2VcXHN0YXIgYSwxXSJdLFsxLDAsIltlLDFdXFx2ZWVbYSwxXSJdLFsyLDAsIltlLDJdXFx2ZWVbYSwxXSJdLFsxLDAsIltkXjBcXHN0YXIgYSwyXSIsMl0sWzEsMiwiW2EsZF4yXSJdXQ==
\[\begin{tikzcd}
	{[e,1]\vee[e\star a,1]} & {[e,1]\vee[a,1]} & {[e,2]\vee[a,1]}
	\arrow["{[d^0\star a,2]}"', from=1-2, to=1-1]
	\arrow["{[a,d^2]}", from=1-2, to=1-3]
\end{tikzcd}\]
The canonical composite morphism 
$$[e\star a,1]\xrightarrow{[e\star a,d^1]}[e,1]\vee[e\star a,1]\to [e,1]\vee(e\star[a,1])$$ 
is also denoted by $[e\star a,d^1]$. Eventually, we define $\overline{[1]\star[a,1]}$ as the following pushout
% https://q.uiver.app/#q=WzAsNCxbMSwxLCJcXG92ZXJsaW5le1sxXVxcc3RhclthLDFdfSJdLFswLDAsIlsxXVxcc3RhclxcezBcXH0iXSxbMCwxLCJbMl1fdCJdLFsxLDAsIntbMV1cXHN0YXJbYSwxXX0iXSxbMSwyXSxbMiwwXSxbMywwXSxbMSwzXSxbMCwxLCIiLDEseyJzdHlsZSI6eyJuYW1lIjoiY29ybmVyIn19XV0=
\[\begin{tikzcd}
	{[1]\star\{0\}} & {{[1]\star[a,1]}} \\
	{[2]_t} & {\overline{[1]\star[a,1]}}
	\arrow[from=1-1, to=2-1]
	\arrow[from=2-1, to=2-2]
	\arrow[from=1-2, to=2-2]
	\arrow[from=1-1, to=1-2]
	\arrow["\lrcorner"{anchor=center, pos=0.125, rotate=180}, draw=none, from=2-2, to=1-1]
\end{tikzcd}\] 

\begin{lemma}
\label{lemma:le lemme quon voulais pas faire}
There is a weak equivalence from $\overline{[1]\star[a,1]}$ to the colimit of the diagram
% https://q.uiver.app/#q=WzAsMyxbMiwwLCJbZSwxXVxcdmVlKGVcXHN0YXJbIGEsMV0pIl0sWzEsMCwiW2VcXHN0YXIgYSwxXSJdLFswLDAsIltbMV1cXHN0YXIgYSwxXSJdLFsxLDAsIltlXFxzdGFyIGEsZF4xXSJdLFsxLDIsIltkXjBcXHN0YXIgYSwxXSIsMl1d
\[\begin{tikzcd}
	{[[1]\star a,1]} & {[e\star a,1]} & {[e,1]\vee(e\star[ a,1])}
	\arrow["{[e\star a,d^1]}", from=1-2, to=1-3]
	\arrow["{[d^0\star a,1]}"', from=1-2, to=1-1]
\end{tikzcd}\]
making $\overline{[1]\star[a,1]}$ the homotopy colimit of the previous diagram.
\end{lemma}
\begin{proof}
The proposition \ref{prop:explicit expression of e star e star a,1} implies that $(\overline{[1]\star [a,1]})_{\mk}$ is the colimit of the diagram
% https://q.uiver.app/#q=WzAsMTEsWzAsMCwiW1syXV4yXFxib3RpbWVzIGEsMV0iXSxbMSwwLCJbWzFdX3RcXG90aW1lcyBhLDFdIl0sWzIsMCwiW1sxXV90LDFdXFx2ZWVbYSwxXSJdLFszLDAsIltlLDFdXFx2ZWVbYSwxXSJdLFs0LDAsIltlLDJdXFx2ZWVbYSwxXSJdLFswLDEsIltlXFxzdGFyIGEsMV0iXSxbMywxLCJbYSwxXSJdLFs0LDEsIltlLDFdXFx2ZWVbYSwxXSJdLFs0LDIsIltlLDFdXFx2ZWVbZVxcc3RhciBhLDFdIl0sWzMsMiwiW2VcXHN0YXIgYSwxXSJdLFswLDIsIltbMV1cXHN0YXIgYSwxXSJdLFsxLDAsIltkXjBcXG90aW1lcyBhLDFdIiwyXSxbMSwyLCJbWzFdXFxvdGltZXMgYSxkXjFdIl0sWzMsMiwiW2ReMFxcb3RpbWVzIGEsMl0iLDJdLFszLDQsIlthLGReMV0iXSxbNiw3LCJbYSxkXjFdIl0sWzUsMCwiW2ReMVxcYm90aW1lcyBhLDFdIl0sWzYsMywiW2EsZF4xXSJdLFs3LDQsIlthLGReMl0iLDJdLFs3LDgsIltkXnswfVxcc3RhciBhLDJdIl0sWzYsNSwiW2ReezB9XFxzdGFyIGEsMV0iLDJdLFs2LDksIltkXnswfVxcc3RhciBhLDFdIiwyXSxbOSw4LCJbZVxcc3RhciBhLGReMV0iLDJdLFs1LDEwLCJbZF4xXFxzdGFyIGEsMV0iLDJdLFs5LDEwLCJbZF4wXFxzdGFyIGEsMV0iXV0=
\begin{equation}
\label{eq:changemeet markage2}
\begin{tikzcd}[column sep=0.7cm]
	{[[2]^2\botimes a,1]} & {[[1]_t\otimes a,1]} & {[[1]_t,1]\vee[a,1]} & {[e,1]\vee[a,1]} & {[e,2]\vee[a,1]} \\
	{[e\star a,1]} &&& {[a,1]} & {[e,1]\vee[a,1]} \\
	{[[1]\star a,1]} &&& {[e\star a,1]} & {[e,1]\vee[e\star a,1]}
	\arrow["{[d^0\otimes a,1]}"', from=1-2, to=1-1]
	\arrow["{[[1]\otimes a,d^1]}", from=1-2, to=1-3]
	\arrow["{[d^0\otimes a,2]}"', from=1-4, to=1-3]
	\arrow["{[a,d^1]}", from=1-4, to=1-5]
	\arrow["{[a,d^1]}", from=2-4, to=2-5]
	\arrow["{[d^1\botimes a,1]}", from=2-1, to=1-1]
	\arrow["{[a,d^1]}", from=2-4, to=1-4]
	\arrow["{[a,d^2]}"', from=2-5, to=1-5]
	\arrow["{[d^{0}\star a,2]}", from=2-5, to=3-5]
	\arrow["{[d^{0}\star a,1]}"', from=2-4, to=2-1]
	\arrow["{[d^{0}\star a,1]}"', from=2-4, to=3-4]
	\arrow["{[e\star a,d^1]}"', from=3-4, to=3-5]
	\arrow["{[d^1\star a,1]}"', from=2-1, to=3-1]
	\arrow["{[d^0\star a,1]}", from=3-4, to=3-1]
\end{tikzcd}
\end{equation}
In the previous diagram, the fact that we have $[[1]_t\otimes a,1]$  instead of $[[1]\otimes a,1]$ comes from the fact that we have considered $(\overline{[1]\star [a,1]})_{\mk}$ instead of $\overline{[1]\star [a,1]}$.


Consider now the morphism
\begin{equation}
\label{eq:changemeet markage}
[[2]^2\botimes a,1]\coprod_{[[1]_t\otimes a,1]}   [[1]_t,1]\vee[a,1]\to e\star[a,1]
\end{equation}
induces by the vertical colimit of the diagram
% https://q.uiver.app/#q=WzAsNixbMCwwLCJbWzJdXjJcXGJvdGltZXMgYSwxXSJdLFsxLDAsIltbMV1fdFxcb3RpbWVzIGEsMV0iXSxbMiwwLCJbWzFdX3QsMV1cXHZlZVthLDFdIl0sWzIsMSwiW2UsMV1cXHZlZVthLDFdIl0sWzAsMSwiW2VcXHN0YXIgYSwxXSJdLFsxLDEsIlthLDFdIl0sWzEsMCwiW2ReMFxcb3RpbWVzIGEsMV0iLDJdLFsxLDIsIltbMV1cXG90aW1lcyBhLGReMV0iXSxbMiwzLCJbc14wLDFdXFx2ZWVbYSwxXSJdLFs1LDNdLFs1LDRdLFsxLDUsIltzXjBcXG90aW1lcyBhLDFdIl0sWzAsNCwiW3NeMFxcYm90aW1lcyBhLDFdIiwyXV0=
\begin{equation}
\label{eq:changemeet markage5}
\begin{tikzcd}
	{[[2]^2\botimes a,1]} & {[[1]_t\otimes a,1]} & {[[1]_t,1]\vee[a,1]} \\
	{[e\star a,1]} & {[a,1]} & {[e,1]\vee[a,1]}
	\arrow["{[d^0\otimes a,1]}"', from=1-2, to=1-1]
	\arrow["{[[1]\otimes a,d^1]}", from=1-2, to=1-3]
	\arrow["{[s^0,1]\vee[a,1]}", from=1-3, to=2-3]
	\arrow[from=2-2, to=2-3]
	\arrow[from=2-2, to=2-1]
	\arrow["{[s^0\otimes a,1]}", from=1-2, to=2-2]
	\arrow["{[s^0\botimes a,1]}"', from=1-1, to=2-1]
\end{tikzcd}
\end{equation}
As all the vertical morphisms of \eqref{eq:changemeet markage5} are cofibrations, the colimit of each line is a homotopy colimit. As all the horizontal morphisms of \eqref{eq:changemeet markage5} are weak equivalences, the morphism \eqref{eq:changemeet markage} also is a weak equivalence.  

Consider now the span
\begin{equation}
\label{eq:deojfoizejfgorezj}
 e\star[a,1]\xleftarrow{\eqref{eq:changemeet markage}} [[2]^2\botimes a,1]\coprod_{[[1]_t\otimes a,1]}   [[1]_t,1]\vee[a,1]\to (\overline{[1]\star [a,1]})_{\mk}
 \end{equation}
As the right hand morphism is a cofibration, and as  \eqref{eq:changemeet markage} is a weak equivalence, the canonical morphism from 
$(\overline{[1]\star [a,1]})_{\mk}$ to the colimit of \eqref{eq:deojfoizejfgorezj} is a weak equivalence.
Using the diagram \eqref{eq:changemeet markage2}, the colimit of \eqref{eq:deojfoizejfgorezj} is also the colimit of the following diagram
% https://q.uiver.app/#q=WzAsOSxbMCwwLCJlXFxzdGFyW2EsMV0iXSxbMSwwLCJbZSwxXVxcdmVlW2EsMV0iXSxbMiwwLCJbZSwyXVxcdmVlW2EsMV0iXSxbMCwxLCJbZVxcc3RhciBhLDFdIl0sWzEsMSwiW2EsMV0iXSxbMiwxLCJbZSwxXVxcdmVlW2EsMV0iXSxbMiwyLCJbZSwxXVxcdmVlW2VcXHN0YXIgYSwxXSJdLFsxLDIsIltlXFxzdGFyIGEsMV0iXSxbMCwyLCJbWzFdXFxzdGFyIGEsMV0iXSxbMSwyLCJbYSxkXjFdIl0sWzQsNSwiW2EsZF4xXSJdLFs0LDEsIlthLGReMV0iXSxbNSwyLCJbYSxkXjJdIiwyXSxbNSw2LCJbZF57MH1cXHN0YXIgYSwyXSJdLFs0LDMsIltkXnswfVxcc3RhciBhLDFdIiwyXSxbNCw3LCJbZF57MH1cXHN0YXIgYSwxXSIsMl0sWzcsNiwiW2VcXHN0YXIgYSxkXjFdIiwyXSxbMyw4LCJbZF4xXFxzdGFyIGEsMV0iLDJdLFs3LDgsIltkXjBcXHN0YXIgYSwxXSJdLFsxLDBdLFszLDBdLFswLDQsIiIsMCx7InN0eWxlIjp7Im5hbWUiOiJjb3JuZXIifX1dXQ==
\[\begin{tikzcd}
	{e\star[a,1]} & {[e,1]\vee[a,1]} & {[e,2]\vee[a,1]} \\
	{[e\star a,1]} & {[a,1]} & {[e,1]\vee[a,1]} \\
	{[[1]\star a,1]} & {[e\star a,1]} & {[e,1]\vee[e\star a,1]}
	\arrow["{[a,d^1]}", from=1-2, to=1-3]
	\arrow["{[a,d^1]}", from=2-2, to=2-3]
	\arrow["{[a,d^1]}", from=2-2, to=1-2]
	\arrow["{[a,d^2]}"', from=2-3, to=1-3]
	\arrow["{[d^{0}\star a,2]}", from=2-3, to=3-3]
	\arrow["{[d^{0}\star a,1]}"', from=2-2, to=2-1]
	\arrow["{[d^{0}\star a,1]}"', from=2-2, to=3-2]
	\arrow["{[e\star a,d^1]}"', from=3-2, to=3-3]
	\arrow["{[d^1\star a,1]}"', from=2-1, to=3-1]
	\arrow["{[d^0\star a,1]}", from=3-2, to=3-1]
	\arrow[from=1-2, to=1-1]
	\arrow[from=2-1, to=1-1]
	\arrow["\lrcorner"{anchor=center, pos=0.125}, draw=none, from=1-1, to=2-2]
\end{tikzcd}\]
As the  upper left square is cocartesian, the colimit of the previous diagram is equivalent to the colimit of the  given diagram.
All put together, we have demonstrated the assertion.
\end{proof}


\begin{lemma}
\label{lemma:le lemme quon voulais pas faire2}
The morphism 
$$[e,1]\vee(e\star[a,1])\cup \{1\}\star [e\star a,1]\to[e,1]\vee(e\star[e\star a,1])$$
is a weak equivalence. 
\end{lemma}
\begin{proof}
We have a cocartesian square
% https://q.uiver.app/#q=WzAsNCxbMSwxLCJbZSwxXVxcdmVlKGVcXHN0YXJbYSwxXSlcXGN1cCBcXHsxXFx9XFxzdGFyIFtlXFxzdGFyIGEsMV0iXSxbMSwwLCJbZSwxXVxcY3VwIGVcXHN0YXJbZVxcc3RhciBhLDFdIl0sWzAsMCwiW2UsMV1cXGN1cCBlXFxzdGFyWyBhLDFdIl0sWzAsMSwiW2UsMV1cXHZlZShlXFxzdGFyW2EsMV0pIl0sWzIsM10sWzMsMF0sWzIsMSwiW2UsMV1cXGN1cCBlXFxzdGFyW2ReMFxcc3RhciBhLDFdIl0sWzEsMF1d
\begin{equation}
\label{eq:lemma:le lemme quon voulais pas faire2}
\begin{tikzcd}
	{[e,1]\cup e\star[ a,1]} & {[e,1]\cup e\star[e\star a,1]} \\
	{[e,1]\vee(e\star[a,1])} & {[e,1]\vee(e\star[a,1])\cup \{1\}\star [e\star a,1]}
	\arrow[from=1-1, to=2-1]
	\arrow[from=2-1, to=2-2]
	\arrow["{[e,1]\cup e\star[d^0\star a,1]}", from=1-1, to=1-2]
	\arrow[from=1-2, to=2-2]
\end{tikzcd}
\end{equation}
Remark that the left vertical morphism is the vertical colimit and homotopy colimit of the diagram
% https://q.uiver.app/#q=WzAsNixbMCwxLCJbZSwxXVxcdmVlW2VcXHN0YXIgYSwxXSJdLFsxLDEsIltlLDFdXFx2ZWVbYSwxXSJdLFsyLDEsIltlLDJdXFx2ZWVbYSwxXSJdLFswLDAsIltlLDFdXFxjdXBbZVxcc3RhciBhLDFdIl0sWzEsMCwiW2UsMV1cXGN1cFthLDFdIl0sWzIsMCwiW2UsMV1cXGN1cFtlLDFdXFx2ZWVbYSwxXSJdLFsxLDBdLFsxLDJdLFs0LDNdLFs0LDVdLFs0LDFdLFs1LDJdLFszLDBdXQ==
\[\begin{tikzcd}
	{[e,1]\cup[e\star a,1]} & {[e,1]\cup[a,1]} & {[e,1]\cup[e,1]\vee[a,1]} \\
	{[e,1]\vee[e\star a,1]} & {[e,1]\vee[a,1]} & {[e,2]\vee[a,1]}
	\arrow[from=2-2, to=2-1]
	\arrow[from=2-2, to=2-3]
	\arrow[from=1-2, to=1-1]
	\arrow[from=1-2, to=1-3]
	\arrow[from=1-2, to=2-2]
	\arrow[from=1-3, to=2-3]
	\arrow[from=1-1, to=2-1]
\end{tikzcd}\]
and is then a weak equivalence. Similarly, $[e,1]\cup e\star[e\star a,1]\to [e,1]\vee(e\star[e\star a,1])$ is a weak equivalence. 
This implies that the right vertical morphism of \eqref{eq:lemma:le lemme quon voulais pas faire2} is a weak equivalence. By two out of three this concludes the proof.
\end{proof}



\begin{lemma}
\label{lem:outer horn inclusion2}
The morphism $\{1\}\star [0]\to [1]_t\star [0]$ is an acyclic cofibration.
\end{lemma}
\begin{proof}
Using proposition \ref{prop:explicit expression of e star a,1} we deduce that $[1]_t\star [0]$ is the colimit of the diagram 
% https://q.uiver.app/#q=WzAsMyxbMiwwLCJbZSwxXV90XFx2ZWVbZSwxXSJdLFsxLDAsIltlLDFdIl0sWzAsMCwiW1sxXV90LDFdIl0sWzEsMF0sWzEsMl1d
\[\begin{tikzcd}
	{[[1]_t,1]} & {[e,1]} & {[e,1]_t\vee[e,1]}
	\arrow[from=1-2, to=1-3]
	\arrow[from=1-2, to=1-1]
\end{tikzcd}\]
The inclusion $\{1\}\star [0]\to [1]_t\star [0]$ is then the composite of the following sequence
% https://q.uiver.app/#q=WzAsNSxbMCwxLCJbZSwxXSJdLFsxLDEsIltlLDFdX3RcXHZlZVtlLDFdIl0sWzEsMCwiW2UsMV0iXSxbMiwwLCJbWzFdX3QsMV0iXSxbMiwxLCJbMV1fdFxcc3RhciBbMF0iXSxbMCwxLCJbZSxkXjBdIl0sWzIsMV0sWzMsNF0sWzIsMywiW2ReMCwxXSJdLFsxLDRdLFs0LDIsIiIsMSx7InN0eWxlIjp7Im5hbWUiOiJjb3JuZXIifX1dXQ==
\[\begin{tikzcd}
	& {[e,1]} & {[[1]_t,1]} \\
	{[e,1]} & {[e,1]_t\vee[e,1]} & {[1]_t\star [0]}
	\arrow["{[e,d^0]}", from=2-1, to=2-2]
	\arrow[from=1-2, to=2-2]
	\arrow[from=1-3, to=2-3]
	\arrow["{[d^0,1]}", from=1-2, to=1-3]
	\arrow[from=2-2, to=2-3]
	\arrow["\lrcorner"{anchor=center, pos=0.125, rotate=180}, draw=none, from=2-3, to=1-2]
\end{tikzcd}\]
As the morphism $[e,d^0]$ and $[d^0,1]$ are acyclic cofibrations, this concludes the proof.
\end{proof}



\begin{lemma}
\label{lem:outer horn inclusion1}
The morphism $\{1\}\star[a,1]\to [1]_t\star[a,1]$ is an acyclic cofibration.
\end{lemma}
\begin{proof}
The Segal $A$-precategory $[1]_t\star [a,1]$ is the colimit  and the homotopy colimit of the diagram
% https://q.uiver.app/#q=WzAsNSxbMSwxLCJcXG92ZXJsaW5le1sxXVxcc3RhclthLDFdfSJdLFswLDAsIlsxXVxcc3RhclxcZW1wdHlzZXQiXSxbMCwxLCJbMV1fdFxcc3RhclxcZW1wdHlzZXQiXSxbMiwwLCJbYVxcc3RhclsxXSwxXSJdLFsyLDEsIlthXFxzdGFyWzFdX3QsMV0iXSxbMywwXSxbMyw0XSxbMSwyXSxbMSwwXV0=
\[\begin{tikzcd}
	{[1]\star\emptyset} && {[a\star[1],1]} \\
	{[1]_t\star\emptyset} & {\overline{[1]\star[a,1]}} & {[a\star[1]_t,1]}
	\arrow[from=1-3, to=2-2]
	\arrow[from=1-3, to=2-3]
	\arrow[from=1-1, to=2-1]
	\arrow[from=1-1, to=2-2]
\end{tikzcd}\]
The lemma \ref{lemma:le lemme quon voulais pas faire} then implies that we have a weak equivalence from $[1]_t\star [a,1]$ to the colimit, denoted by $K$, of the diagram 
% https://q.uiver.app/#q=WzAsMyxbMiwwLCJbZSwxXV90XFx2ZWUoZVxcc3RhciBbIGEsMV0pIl0sWzEsMCwiW2VcXHN0YXIgYSwxXSJdLFswLDAsIltbMV1fdFxcc3RhciBhLDFdIl0sWzEsMCwiW2VcXHN0YXIgYSxkXjFdIl0sWzEsMiwiW2ReMFxcc3RhciBhLDFdIiwyXV0=
\[\begin{tikzcd}
	{[[1]_t\star a,1]} & {[e\star a,1]} & {[e,1]_t\vee(e\star [ a,1])}
	\arrow["{[e\star a,d^1]}", from=1-2, to=1-3]
	\arrow["{[d^0\star a,1]}"', from=1-2, to=1-1]
\end{tikzcd}\]
As all the morphisms are cofibrations, $K$ is also the homotopy colimit of the previous diagram.

The morphism $[e,1]_t\vee(e\star [ a,1])\to e\star [ a,1]$ is a  weak equivalence as it is a homotopy colimit of weak equivalences. 
Moreover, the morphism $[[1]_t\star a,1]\to [e\star a,1]$ is also a weak equivalence. This implies that the composite $s^0\star[a,1]:[1]_t\star [a,1]\to K\to [0]\star[a,1]$ is a weak equivalence.
The morphism $\{1\}\star[a,1]\to [1]_t\star[a,1]$ is a section of $s^0\star[a,1]$ and is then also a weak equivalence.
\end{proof}


\begin{lemma}
\label{lem:horn_inclusion_2}
The morphism 
$\Lambda^1[2]\star [0]\to [2]_t\star [0]$
is an acyclic cofibration.
\end{lemma}
\begin{proof}
The Segal $A$-precategory $[2]_t\star [0]$ is the colimit of the following diagram
% https://q.uiver.app/#q=WzAsMyxbMiwwLCJcXG92ZXJsaW5le1sxXVxcc3RhclsxXX0iXSxbMCwwLCJbWzJdX3QsMV0iXSxbMSwwLCJbWzJdLDFdIl0sWzIsMV0sWzIsMF1d
\[\begin{tikzcd}
	{[[2]_t,1]} & {[[2],1]} & {\overline{[1]\star[1]}}
	\arrow[from=1-2, to=1-1]
	\arrow[from=1-2, to=1-3]
\end{tikzcd}\]
The lemma \ref{lemma:le lemme quon voulais pas faire} then implies that we have a weak equivalence from $[2]_t\star [0]$ to the colimit, denoted by $K$, of the diagram 
% https://q.uiver.app/#q=WzAsMyxbMiwwLCJbZSwxXVxcdmVlKGVcXHN0YXJbZSwxXSkiXSxbMSwwLCJbWzFdLDFdIl0sWzAsMCwiW1syXV90LDFdIl0sWzEsMCwiW1sxXSxkXjFdIl0sWzEsMiwiW2ReezB9LDFdIiwyXV0=
\[\begin{tikzcd}
	{[[2]_t,1]} & {[[1],1]} & {[e,1]\vee(e\star[e,1])}
	\arrow["{[[1],d^1]}", from=1-2, to=1-3]
	\arrow["{[d^{0},1]}"', from=1-2, to=1-1]
\end{tikzcd}\]
On the other side, $\Lambda^1[2]\star [0]$ is the colimit of the diagram 
% https://q.uiver.app/#q=WzAsNyxbNCwxLCJbWzFdLDFdIl0sWzUsMSwiW2UsMV0iXSxbNiwxLCJbZSwyXSJdLFswLDEsIltbMV0sMV0iXSxbMSwxLCJbZSwxXSJdLFsyLDEsIltlLDJdIl0sWzMsMCwiW2UsMV0iXSxbMSwwLCJbZF4wLDFdIiwyXSxbNiwwLCJbZF4xLDFdIl0sWzUsNCwiW2UsZF4xXSJdLFs0LDMsIltkXjAsMV0iXSxbNiw1LCJbZSxkXjBdIiwyXSxbMSwyLCJbZSxkXjFdIl1d
\[\begin{tikzcd}
	&&& {[e,1]} \\
	{[[1],1]} & {[e,1]} & {[e,2]} && {[[1],1]} & {[e,1]} & {[e,2]}
	\arrow["{[d^0,1]}"', from=2-6, to=2-5]
	\arrow["{[d^1,1]}", from=1-4, to=2-5]
	\arrow["{[e,d^1]}", from=2-3, to=2-2]
	\arrow["{[d^0,1]}", from=2-2, to=2-1]
	\arrow["{[e,d^0]}"', from=1-4, to=2-3]
	\arrow["{[e,d^1]}", from=2-6, to=2-7]
\end{tikzcd}\]
The composite $\Lambda^1[2]\star [0]\to [2]_t\star [0]\to K$ fits in the sequence of acyclic cofibrations
% https://q.uiver.app/#q=WzAsMTAsWzAsMSwiXFxMYW1iZGFeMVsyXVxcc3RhciBbMF0iXSxbMCwwLCJbZSxkXjBdXFxjdXBbZSxkXjJdIl0sWzEsMCwiW2UsM10iXSxbMSwxLCJcXGJ1bGxldCJdLFsyLDAsIltcXExhbWJkYV4xWzJdLDFdIl0sWzMsMCwiW1syXV90LDFdIl0sWzMsMSwiSyJdLFsyLDEsIlxcYnVsbGV0Il0sWzEsMiwiKFtlLDFdXFxjdXBbWzFdLDFdKVxcY3VwW2UsMV1cXHZlZVtcXHBhcnRpYWxbMV0sMV0iXSxbMiwyLCJbZSwxXVxcdmVlW1sxXSwxXSJdLFsxLDBdLFsxLDJdLFsyLDNdLFswLDNdLFs0LDVdLFs2LDQsIiIsMCx7InN0eWxlIjp7Im5hbWUiOiJjb3JuZXIifX1dLFs3LDZdLFszLDddLFs1LDZdLFs0LDddLFs4LDNdLFs4LDldLFs5LDddLFszLDExLCIiLDAseyJsZXZlbCI6MSwic3R5bGUiOnsibmFtZSI6ImNvcm5lciJ9fV0sWzksMTcsIiIsMCx7ImxldmVsIjoxLCJzdHlsZSI6eyJuYW1lIjoiY29ybmVyIn19XV0=
\[\begin{tikzcd}[column sep =0.3cm]
	{[e,d^0]\cup[e,d^2]} & {[e,3]} & {[\Lambda^1[2],1]} & {[[2]_t,1]} \\
	{\Lambda^1[2]\star [0]} & \bullet & \bullet & K \\
	& {([e,1]\cup[[1],1])\cup[e,1]\vee[\partial[1],1]} & {[e,1]\vee[[1],1]}
	\arrow[from=1-1, to=2-1]
	\arrow[""{name=0, anchor=center, inner sep=0}, from=1-1, to=1-2]
	\arrow[from=1-2, to=2-2]
	\arrow[from=2-1, to=2-2]
	\arrow[from=1-3, to=1-4]
	\arrow["\lrcorner"{anchor=center, pos=0.125, rotate=180}, draw=none, from=2-4, to=1-3]
	\arrow[from=2-3, to=2-4]
	\arrow[""{name=1, anchor=center, inner sep=0}, from=2-2, to=2-3]
	\arrow[from=1-4, to=2-4]
	\arrow[from=1-3, to=2-3]
	\arrow[from=3-2, to=2-2]
	\arrow[from=3-2, to=3-3]
	\arrow[from=3-3, to=2-3]
	\arrow["\lrcorner"{anchor=center, pos=0.125, rotate=180}, draw=none, from=2-2, to=0]
	\arrow["\lrcorner"{anchor=center, pos=0.125, rotate=180}, draw=none, from=3-3, to=1]
\end{tikzcd}\]
and is then a weak equivalence.
By two out of three, this concludes the proof.
 \end{proof}
 
 
 





\begin{lemma}
\label{lem:horn_inclusion_3}
The morphism 
$\Lambda^1[2]\star [a,1]\to [2]_t\star [a,1]$
is an acyclic cofibration.
\end{lemma}
\begin{proof}
The lemma \ref{lem:horn_inclusion_2} implies that the inclusion $\Lambda^1[2]\star [a,1]\to \Lambda^1[2]\star [a,1]\cup [2]_t\star \{0\}$ is an acyclic cofibration.
Using proposition \ref{prop:explicit expression of e star a,1}, we deduce that the
 Segal $A$-precategory $[2]_t\star[a,1]$ is the colimit of the diagram
% https://q.uiver.app/#q=WzAsNSxbMSwxLCJbMV1cXHN0YXIoW2UsMV1cXHZlZVthLDFdKSJdLFsyLDAsIlsxXVxcc3RhclthLDFdIl0sWzIsMSwiXFxvdmVybGluZXtcXG92ZXJsaW5le1sxXVxcc3RhcltlXFxzdGFyIGEsMV19fSJdLFswLDAsIlsxXVxcc3RhcltlLDFdIl0sWzAsMSwiXFxvdmVybGluZXtcXG92ZXJsaW5le1sxXVxcc3RhcltlLDFdfX0iXSxbMSwyLCJbMV1cXHN0YXJbZF4wXFxzdGFyIGEsMV0iXSxbMSwwLCJbMV1cXHN0YXJbYSxkXjFdIiwyXSxbMyw0XSxbMywwXV0=
\[\begin{tikzcd}
	{[1]\star[e,1]} && {[1]\star[a,1]} \\
	{\overline{\overline{[1]\star[e,1]}}} & {[1]\star([e,1]\vee[a,1])} & {\overline{\overline{[1]\star[e\star a,1]}}}
	\arrow["{[1]\star[d^0\star a,1]}", from=1-3, to=2-3]
	\arrow["{[1]\star[a,d^1]}"', from=1-3, to=2-2]
	\arrow[from=1-1, to=2-1]
	\arrow[from=1-1, to=2-2]
\end{tikzcd}\]
while $ \Lambda^1[2]\star [a,1]\cup [2]_t\star \{0\}$ is the colimit  of the diagram
% https://q.uiver.app/#q=WzAsNSxbMCwxLCJcXG92ZXJsaW5le1xcb3ZlcmxpbmV7WzFdXFxzdGFyW2UsMV19fSJdLFsxLDEsIlxcezFcXH1cXHN0YXIoW2UsMV1cXHZlZVthLDFdKVxcY3VwWzFdXFxzdGFyW2EsMV0iXSxbMiwwLCJcXHsxXFx9XFxzdGFyW2EsMV0iXSxbMiwxLCJcXHsxXFx9XFxzdGFyW2VcXHN0YXIgYSwxXSJdLFswLDAsIlxcezFcXH1cXHN0YXJbZSwxXSJdLFsyLDEsIlxcezFcXH1cXHN0YXJbYSxkXjFdIiwyXSxbMiwzLCJcXHsxXFx9XFxzdGFyW2ReMFxcc3RhciBhLDFdIl0sWzQsMF0sWzQsMV1d
\[\begin{tikzcd}
	{\{1\}\star[e,1]} && {\{1\}\star[a,1]} \\
	{\overline{\overline{[1]\star[e,1]}}} & {\{1\}\star([e,1]\vee[a,1])\cup[1]\star[a,1]} & {\{1\}\star[e\star a,1]}
	\arrow["{\{1\}\star[a,d^1]}"', from=1-3, to=2-2]
	\arrow["{\{1\}\star[d^0\star a,1]}", from=1-3, to=2-3]
	\arrow[from=1-1, to=2-1]
	\arrow[from=1-1, to=2-2]
\end{tikzcd}\]
where $\overline{\overline{[1]\star[e,1]}}:=[2]_t\star [0]$ and where $\overline{\overline{[1]\star[e\star a,1]}}$ is the following pushout:
% https://q.uiver.app/#q=WzAsNSxbMiwxLCJcXG92ZXJsaW5le1xcb3ZlcmxpbmV7WzFdXFxzdGFyW2VcXHN0YXIgYSwxXX19Il0sWzIsMCwiXFxvdmVybGluZXtbMV1cXHN0YXIgW2VcXHN0YXIgYSwxXX0iXSxbMSwwLCJlXFxzdGFyIFtbMV1cXHN0YXIgYSwxXSJdLFswLDAsIiBbWzJdXFxzdGFyIGEsMV0iXSxbMCwxLCIgW1syXV90XFxzdGFyIGEsMV0iXSxbMywyXSxbMiwxXSxbMyw0XSxbNCwwXSxbMSwwXSxbMCwyLCIiLDEseyJzdHlsZSI6eyJuYW1lIjoiY29ybmVyIn19XV0=
\[\begin{tikzcd}
	{ [[2]\star a,1]} & {e\star [[1]\star a,1]} & {\overline{[1]\star [e\star a,1]}} \\
	{ [[2]_t\star a,1]} && {\overline{\overline{[1]\star[e\star a,1]}}}
	\arrow[from=1-1, to=1-2]
	\arrow[from=1-2, to=1-3]
	\arrow[from=1-1, to=2-1]
	\arrow[from=2-1, to=2-3]
	\arrow[from=1-3, to=2-3]
	\arrow["\lrcorner"{anchor=center, pos=0.125, rotate=180}, draw=none, from=2-3, to=1-2]
\end{tikzcd}\]
Let $K_1$ be the following pushout:
% https://q.uiver.app/#q=WzAsNCxbMCwwLCJcXHsxXFx9XFxzdGFyIChbZSwxXVxcdmVlIFthLDFdKSBcXGN1cCBbMV1cXHN0YXIgKFtlLDFdXFxjdXAgW2EsMV0pICJdLFswLDEsIlsxXVxcc3RhciAoW2UsMV1cXHZlZVthLDFdKSJdLFsxLDAsIlxcTGFtYmRhXjFbMl1cXHN0YXJbYSwxXVxcY3VwIFsyXV90XFxzdGFyXFx7MFxcfSJdLFsxLDEsIktfMSJdLFswLDFdLFswLDJdLFsxLDNdLFsyLDNdLFszLDUsIiIsMix7ImxldmVsIjoxLCJzdHlsZSI6eyJuYW1lIjoiY29ybmVyIn19XV0=
\[\begin{tikzcd}
	{\{1\}\star ([e,1]\vee [a,1]) \cup [1]\star ([e,1]\cup [a,1]) } & {\Lambda^1[2]\star[a,1]\cup [2]_t\star\{0\}} \\
	{[1]\star ([e,1]\vee[a,1])} & {K_1}
	\arrow[from=1-1, to=2-1]
	\arrow[""{name=0, anchor=center, inner sep=0}, from=1-1, to=1-2]
	\arrow[from=2-1, to=2-2]
	\arrow[from=1-2, to=2-2]
	\arrow["\lrcorner"{anchor=center, pos=0.125, rotate=180}, draw=none, from=2-2, to=0]
\end{tikzcd}\]
The left-hand morphism is equal to $(d^0:[0]\to [1])\hstar ([e,1]\cup[a,1]\to [e,1]\vee[a,1])$ which is an acyclic cofibration according to lemma \ref{lemma:leibnizt joint is Quillen}.
Furthermore, the morphism $K_1\to [2]_t\star [a,1]$ fits in the following pushout:
% https://q.uiver.app/#q=WzAsNCxbMSwwLCJLXzEiXSxbMSwxLCJbMl1fdFxcc3RhclthLDFdIl0sWzAsMCwiXFxvdmVybGluZXtbMV1cXHN0YXJbIGEsMV19XFxjdXBcXHsxXFx9XFxzdGFyW2VcXHN0YXIgYSwxXSJdLFswLDEsIlxcb3ZlcmxpbmV7XFxvdmVybGluZXtbMV1cXHN0YXJbZVxcc3RhciBhLDFdfX0iXSxbMCwxXSxbMiwwXSxbMiwzXSxbMywxXSxbMSw1LCIiLDEseyJsZXZlbCI6MSwic3R5bGUiOnsibmFtZSI6ImNvcm5lciJ9fV1d
\[\begin{tikzcd}
	{\overline{[1]\star[ a,1]}\cup\{1\}\star[e\star a,1]} & {K_1} \\
	{\overline{\overline{[1]\star[e\star a,1]}}} & {[2]_t\star[a,1]}
	\arrow[from=1-2, to=2-2]
	\arrow[""{name=0, anchor=center, inner sep=0}, from=1-1, to=1-2]
	\arrow[from=1-1, to=2-1]
	\arrow[from=2-1, to=2-2]
	\arrow["\lrcorner"{anchor=center, pos=0.125, rotate=180}, draw=none, from=2-2, to=0]
\end{tikzcd}\]
The lemma \ref{lemma:le lemme quon voulais pas faire} implies that we have a weak equivalence from $\overline{[1]\star[ a,1]}\cup\{1\}\star[e\star a,1]$ to the colimit, denoted by $K_2$, of the diagram 
% https://q.uiver.app/#q=WzAsMyxbMiwwLCJbZSwxXVxcdmVlKGVcXHN0YXJbYSwxXSlcXGN1cCBcXHsxXFx9XFxzdGFyW2VcXHN0YXIgYSwxXSJdLFsxLDAsIltlXFxzdGFyIGEsMV0iXSxbMCwwLCJbWzFdXFxzdGFyIGEsMV0iXSxbMSwwLCJbZVxcc3RhciBhLGReMV0iXSxbMSwyLCJbZF4wXFxzdGFyIGEsMV0iLDJdXQ==
\[\begin{tikzcd}
	{[[1]\star a,1]} & {[e\star a,1]} & {[e,1]\vee(e\star[a,1])\cup \{1\}\star[e\star a,1]}
	\arrow["{[e\star a,d^1]}", from=1-2, to=1-3]
	\arrow["{[d^0\star a,1]}"', from=1-2, to=1-1]
\end{tikzcd}\]
As all the morphisms are cofibrations, $K_2$ is also the homotopy colimit of the previous diagram. We now define $K_3$ as the colimit of the diagram
% https://q.uiver.app/#q=WzAsMyxbMCwwLCJbXFxMYW1iZGFeMVsyXVxcc3RhciAgYSwxXSJdLFsyLDAsIltlLDFdXFx2ZWUoZVxcc3RhcltlXFxzdGFyIGEsMV0pIl0sWzEsMCwiW1sxXVxcc3RhciBhLDFdIl0sWzIsMCwiW2ReMFxcc3RhciBhLDFdIiwyXSxbMiwxLCJbWzFdXFxzdGFyIGEsZF4xXSJdXQ==
\[\begin{tikzcd}
	{[\Lambda^1[2]\star  a,1]} & {[[1]\star a,1]} & {[e,1]\vee(e\star[e\star a,1])}
	\arrow["{[d^0\star a,1]}"', from=1-2, to=1-1]
	\arrow["{[[1]\star a,d^1]}", from=1-2, to=1-3]
\end{tikzcd}\]
The canonical morphism $K_2\to K_3$ fits in the cocartesian square
% https://q.uiver.app/#q=WzAsNCxbMCwxLCJbZSwxXVxcdmVlKGVcXHN0YXJbZVxcc3RhciBhLDFdKSJdLFswLDAsIltlLDFdXFx2ZWUoZVxcc3RhclthLDFdKVxcY3VwIFxcezFcXH1cXHN0YXJbZVxcc3RhciBhLDFdIl0sWzEsMCwiS18yIl0sWzEsMSwiS18zIl0sWzEsMF0sWzAsM10sWzEsMl0sWzIsM10sWzMsNiwiIiwxLHsibGV2ZWwiOjEsInN0eWxlIjp7Im5hbWUiOiJjb3JuZXIifX1dXQ==
\[\begin{tikzcd}
	{[e,1]\vee(e\star[a,1])\cup \{1\}\star[e\star a,1]} & {K_2} \\
	{[e,1]\vee(e\star[e\star a,1])} & {K_3}
	\arrow[from=1-1, to=2-1]
	\arrow[from=2-1, to=2-2]
	\arrow[""{name=0, anchor=center, inner sep=0}, from=1-1, to=1-2]
	\arrow[from=1-2, to=2-2]
	\arrow["\lrcorner"{anchor=center, pos=0.125, rotate=180}, draw=none, from=2-2, to=0]
\end{tikzcd}\]
and is then a weak equivalence according to the lemma \ref{lemma:le lemme quon voulais pas faire2}.

On the other side, the lemma \ref{lemma:le lemme quon voulais pas faire} also implies that we have a weak equivalence from $\overline{\overline{[1]\star[e\star a,1]}}$ to the colimit, denoted by $K_4$, of the diagram
% https://q.uiver.app/#q=WzAsMyxbMCwwLCJbWzJdX3RcXHN0YXIgIGEsMV0iXSxbMiwwLCJbZSwxXVxcdmVlKGVcXHN0YXJbZVxcc3RhciBhLDFdKSJdLFsxLDAsIltbMV1cXHN0YXIgYSwxXSJdLFsyLDAsIltkXjBcXHN0YXIgYSwxXSIsMl0sWzIsMSwiW1sxXVxcc3RhciBhLGReMV0iXV0=
\[\begin{tikzcd}
	{[[2]_t\star  a,1]} & {[[1]\star a,1]} & {[e,1]\vee(e\star[e\star a,1])}
	\arrow["{[d^0\star a,1]}"', from=1-2, to=1-1]
	\arrow["{[[1]\star a,d^1]}", from=1-2, to=1-3]
\end{tikzcd}\]
As all the morphisms are cofibrations, $K_4$ is also the homotopy colimit of the previous diagram.
As $\Lambda^1[2]\star a \to [2]_t\star a$ is a weak equivalence in $A$, this implies that the canonical morphism $K_3\to K_4$ is also a weak equivalence. We then have  commutative diagram:
% https://q.uiver.app/#q=WzAsNSxbMCwwLCJbMV1cXHN0YXJbYSwxXVxcY3VwIFxcezFcXH1cXHN0YXJbZVxcc3RhciBhLDFdIl0sWzIsMCwiXFxvdmVybGluZXtbMV1cXHN0YXJbZVxcc3RhciBhLDFdfSJdLFsxLDEsIktfMyJdLFsyLDEsIktfNCJdLFswLDEsIktfMiJdLFsyLDMsIlxcc2ltIiwyXSxbMCwxXSxbMSwzLCJcXHNpbSJdLFswLDQsIlxcc2ltIiwyXSxbNCwyLCJcXHNpbSIsMl1d
\[\begin{tikzcd}
	{[1]\star[a,1]\cup \{1\}\star[e\star a,1]} && {\overline{[1]\star[e\star a,1]}} \\
	{K_2} & {K_3} & {K_4}
	\arrow["\sim"', from=2-2, to=2-3]
	\arrow[from=1-1, to=1-3]
	\arrow["\sim", from=1-3, to=2-3]
	\arrow["\sim"', from=1-1, to=2-1]
	\arrow["\sim"', from=2-1, to=2-2]
\end{tikzcd}\]
where all arrows labelled by $\sim$ are weak equivalences. By two out of three, this implies the result.
\end{proof}






\begin{lemma}
\label{lem:horn_inclusion_4}
For any stratified Segal $A$-precategory $C$, the morphisms $\Lambda^1[2]\star C\to [2]_t\star C$ and $\{1\}\star C\to [1]_t\star C$ are
 acyclic cofibrations.
Moreover, for any cofibration of stratified Segal $A$-precategory $i$, and $j$ being either $\{1\}\to [1]_t$ or $\Lambda^1[2]\to [2]_t$, the morphism $j\hstar i$ is an acyclic cofibration.
\end{lemma}
\begin{proof}
We begin with the first assertion.
The lemma \ref{lemma:leibnizt joint is Quillen} implies that  $\Lambda^1[2]\star\uvar$ and $[2]_t\star\uvar$ are left Quillen functors.  As every object is a homotopy colimits of objects of shape $[a,n]$ or $[e,1]_t$, we can reduce to the case where $C$ is of this shape.
Using Segal extensions, we can reduce to the case where $C$ is $[a,1]$, $[0]$ or $[e,1]_t$. 


If $C$ is $[a,1]$ or $[0]$, the result follows from lemmas \ref{lem:outer horn inclusion2}, \ref{lem:outer horn inclusion1}, \ref{lem:horn_inclusion_2} and \ref{lem:horn_inclusion_3}.


Eventually, for $C := [e,1]_t$, we have a diagram:
% https://q.uiver.app/#q=WzAsOCxbMywxLCJbMl1fdFxcc3RhcltlLDFdX3QiXSxbMywwLCJcXExhbWJkYV4xWzJdXFxzdGFyIFtlLDFdX3QiXSxbNCwwLCJcXExhbWJkYV4xWzJdXFxzdGFyIFswXSJdLFs0LDEsIlsyXV90XFxzdGFyIFswXSJdLFswLDAsIlxcezFcXH1cXHN0YXJbZSwxXV90Il0sWzAsMSwiWzFdX3RcXHN0YXJbZSwxXV90Il0sWzEsMCwiXFx7MFxcfVxcc3RhciBbMF0iXSxbMSwxLCJbMV1fdFxcc3RhciBbMF0iXSxbMSwwXSxbMSwyXSxbMCwzXSxbMiwzXSxbNSw3XSxbNCw1XSxbNiw3XSxbNCw2XV0=
\[\begin{tikzcd}
	{\{1\}\star[e,1]_t} & {\{0\}\star [0]} && {\Lambda^1[2]\star [e,1]_t} & {\Lambda^1[2]\star [0]} \\
	{[1]_t\star[e,1]_t} & {[1]_t\star [0]} && {[2]_t\star[e,1]_t} & {[2]_t\star [0]}
	\arrow[from=1-4, to=2-4]
	\arrow[from=1-4, to=1-5]
	\arrow[from=2-4, to=2-5]
	\arrow[from=1-5, to=2-5]
	\arrow[from=2-1, to=2-2]
	\arrow[from=1-1, to=2-1]
	\arrow[from=1-2, to=2-2]
	\arrow[from=1-1, to=1-2]
\end{tikzcd}\]
Lemmas \ref{lemma:leibnizt joint is Quillen}, \ref{lem:outer horn inclusion2} and \ref{lem:horn_inclusion_2} imply that all
 horizontal morphisms and right vertical morphisms are weak equivalences. By two out of three, this implies that the left vertical morphisms are weak equivalences.
 
 
This concludes the proof of the first assertion. The second one is obtained with some diagram chasing.
\end{proof}



\begin{prop}
\label{prop:horn_inclusion}
The functor $\stratSset\to \stratSeg(A)$ sends complicial horn inclusions to weak equivalences.
\end{prop}
\begin{proof}
Let $k\leq n$ be two integers. First, we suppose that $0<k<n$. We then have an equality $$(\Lambda^k[n]\to [n]^k)= (\partial[k-2]\to [k-2])\hstar (\Lambda^1[2]\to [2]_t)\hstar (\partial [n-k-2]\to [n-k-2]).$$ This is an acyclic cofibration according to lemmas \ref{lemma:leibnizt joint is Quillen} and \ref{lem:horn_inclusion_4}. If $k=0$, we have an equality $$(\Lambda^0[n]\to [n]^0) = (\{1\}\to [e,1]_t)\hstar (\partial[n-2]\to [n-2])$$
and the right hand morphism is an acyclic cofibration again thanks to lemma \ref{lem:horn_inclusion_4}. Eventually, for $k=n$, note that $$(\Lambda^n[n]\to [n]^n) = (\partial[n-2]\to [n-2])\hstar (\{0\}\to [e,1]_t).$$ This morphism is an acyclic cofibration according to lemma \ref{lemma:leibnizt joint is Quillen}.
\end{proof}



\subsection{Complicial thinness extensions}
\label{section:Complicial thinness extensions}
\begin{notation*}
In this section, we will often consider morphisms $\tilde{a}\to \tilde{b}$ that fit into cocartesian squares:
% https://q.uiver.app/#q=WzAsNCxbMCwwLCJhIl0sWzEsMCwiYiJdLFswLDEsIlxcdGlsZGV7YX0iXSxbMSwxLCJcXHRpbGRle2J9Il0sWzAsMl0sWzAsMSwiaSJdLFsyLDNdLFsxLDNdLFswLDMsIiIsMix7InN0eWxlIjp7Im5hbWUiOiJjb3JuZXIifX1dXQ==
\[\begin{tikzcd}
	a & b \\
	{\tilde{a}} & {\tilde{b}}
	\arrow[from=1-1, to=2-1]
	\arrow["i", from=1-1, to=1-2]
	\arrow[from=2-1, to=2-2]
	\arrow[from=1-2, to=2-2]
	\arrow["\lrcorner"{anchor=center, pos=0.125}, draw=none, from=1-1, to=2-2]
\end{tikzcd}\]
where $a\to \tilde{a}$ and $b\to \tilde{b}$ are epimorphisms.
To avoid complicating the notations unnecessarily, the induced morphism $\tilde{a}\to \tilde{b}$ will just be denoted $i$.
\end{notation*}

\begin{lemma}
\label{lemma:thinnes extension case 0 and n}
Morphisms $([n]^0)'\to ([n]^0)''$ and $([n]^n)' \to ([n]^n)''$ are acyclic cofibrations.
\end{lemma}
\begin{proof}
For $k$ equal to $0$ or $n$, we have pushout diagrams: 
% https://q.uiver.app/?q=WzAsNixbMCwwLCJbbl1eayJdLFswLDEsIltuLTFdIl0sWzEsMCwiKFtuXV5rKSciXSxbMiwwLCIoW25dXmspJyciXSxbMSwxLCJbbi0xXV90Il0sWzIsMSwiW24tMV1fdCJdLFswLDFdLFswLDJdLFsyLDRdLFsxLDRdLFsyLDNdLFszLDVdLFs0LDUsImlkIiwyXSxbNCwwLCIiLDAseyJzdHlsZSI6eyJuYW1lIjoiY29ybmVyIn19XSxbNSwyLCIiLDAseyJzdHlsZSI6eyJuYW1lIjoiY29ybmVyIn19XV0=
\[\begin{tikzcd}
	{[n]^k} & {([n]^k)'} & {([n]^k)''} \\
	{[n-1]} & {[n-1]_t} & {[n-1]_t}
	\arrow[from=1-1, to=2-1]
	\arrow[from=1-1, to=1-2]
	\arrow[from=1-2, to=2-2]
	\arrow[from=2-1, to=2-2]
	\arrow[from=1-2, to=1-3]
	\arrow[from=1-3, to=2-3]
	\arrow["id"', from=2-2, to=2-3]
	\arrow["\lrcorner"{anchor=center, pos=0.125, rotate=180}, draw=none, from=2-2, to=1-1]
	\arrow["\lrcorner"{anchor=center, pos=0.125, rotate=180}, draw=none, from=2-3, to=1-2]
\end{tikzcd}\]
Lemmas \ref{lemma:leibnizt joint is Quillen} and \ref{lem:horn_inclusion_4} imply that both $s^0:[n]^0\to [n-1]$ and $s^{n-1}:[n]^{n-1}\to [n-1]$ are weak equivalences. As horizontal morphisms are cofibrations, the left properness imply that all the vertical morphisms are weak equivalences.
 By two out of three, this shows that $([n]^k)' \to ([n]^k)''$ is a weak equivalence. 
\end{proof}



\begin{construction}
\label{cons:the big construction}
We consider these objects of $\Delta^2_{/[1]}$ and $\Delta^2_{/[2]}$:
$$\begin{array}{cc}
s^1:[1]^{op}\star [0]\to[1]&s^0:[0]^{op}\star [1]\to[1]
\\s^1:[1]^{op}\star[1]\to[2]&s^2:[2]^{op}\star[0]\to [2].
\end{array}$$
They induce morphisms:
$$\begin{array}{cc}
\alpha_a:[e\star a,1]\to e\star [a,1]&\beta_a:[e,1]\vee [ a,1]\to e\star [a,1] \\
 \delta_a:[e\star a,1]\vee[a,1]\to e\star([a,2])& \epsilon_a:[[2]\botimes a,1]\to e\star([a,2])\end{array}$$
where $[2]\botimes a$ and $[e\star a,1]\vee[a,1]$ are the following pushouts:
% https://q.uiver.app/?q=WzAsOCxbMiwwLCJbMl1cXG90aW1lcyBhIl0sWzAsMCwiWzFdXFxvdGltZXMgYVxcYW1hbGcgWzFdXFxvdGltZXMgYSJdLFswLDEsImVcXHN0YXIgYVxcYW1hbGcgZVxcc3RhciBhIl0sWzIsMSwiWzJdXFxib3RpbWVzIGEiXSxbNSwwLCJbWzFdXFxvdGltZXMgYSwyXSJdLFs1LDEsIltlXFxzdGFyIGEsMV1cXHZlZVthLDFdIl0sWzQsMCwiW1sxXVxcb3RpbWVzIGEsMV1cXGFtYWxnW1sxXVxcb3RpbWVzIGEsMV0iXSxbNCwxLCJbZVxcc3RhciBhLDFdXFxhbWFsZyBbYSwxXSJdLFsxLDAsImReMVxcb3RpbWVzIGFcXGFtYWxnIGReMlxcb3RpbWVzIGEiXSxbMiwzLCJkXjFcXGJvdGltZXMgYVxcYW1hbGcgZF4yXFxib3RpbWVzIGEiLDJdLFsxLDJdLFswLDNdLFs2LDddLFs2LDQsIltbMV1cXG90aW1lcyBhLGReMlxcYW1hbGcgZF4wXSJdLFs3LDVdLFs0LDVdLFszLDgsIiIsMSx7ImxldmVsIjoxLCJzdHlsZSI6eyJuYW1lIjoiY29ybmVyIn19XSxbNSwxMywiIiwwLHsibGV2ZWwiOjEsInN0eWxlIjp7Im5hbWUiOiJjb3JuZXIifX1dXQ==
\[\begin{tikzcd}
	{[1]\otimes a\amalg [1]\otimes a} && {[2]\otimes a} && {[[1]\otimes a,1]\amalg[[1]\otimes a,1]} & {[[1]\otimes a,2]} \\
	{e\star a\amalg e\star a} && {[2]\botimes a} && {[e\star a,1]\amalg [a,1]} & {[e\star a,1]\vee[a,1]}
	\arrow[""{name=0, anchor=center, inner sep=0}, "{d^1\otimes a\amalg d^2\otimes a}", from=1-1, to=1-3]
	\arrow["{d^1\botimes a\amalg d^2\botimes a}"', from=2-1, to=2-3]
	\arrow[from=1-1, to=2-1]
	\arrow[from=1-3, to=2-3]
	\arrow[from=1-5, to=2-5]
	\arrow[""{name=1, anchor=center, inner sep=0}, "{[[1]\otimes a,d^2\amalg d^0]}", from=1-5, to=1-6]
	\arrow[from=2-5, to=2-6]
	\arrow[from=1-6, to=2-6]
	\arrow["\lrcorner"{anchor=center, pos=0.125, rotate=180}, draw=none, from=2-3, to=0]
	\arrow["\lrcorner"{anchor=center, pos=0.125, rotate=180}, draw=none, from=2-6, to=1]
\end{tikzcd}\]
Moreover there are commutative diagrams:
% https://q.uiver.app/?q=WzAsMjMsWzEsMiwiWzFdXntvcH1cXHN0YXJbMV0iXSxbMSwzLCJbMl0iXSxbMywyLCJbMl1ee29wfVxcc3RhclswXSJdLFszLDMsIlsyXSJdLFsyLDIsIlsxXV57b3B9XFxzdGFyWzBdIl0sWzIsMywiWzFdIl0sWzAsMiwiWzFdXntvcH1cXHN0YXJbMF0iXSxbMCwzLCJbMV0iXSxbMiwwLCJbMF1ee29wfVxcc3RhclswXSJdLFsyLDEsIlswXV57b3B9XFxzdGFyWzFdIl0sWzMsMSwiWzFdICJdLFszLDAsIlsxXV57b3B9XFxzdGFyWzBdIl0sWzEsMCwiWzBdXntvcH1cXHN0YXJbMV0iXSxbMSwxLCIgWzFdIl0sWzAsMCwiWzFdIl0sWzAsNCwiWzFdXntvcH1cXHN0YXJbMF0iXSxbMSw0LCJbMl1ee29wfVxcc3RhclswXSJdLFswLDUsIlsxXV57b3B9XFxzdGFyWzFdIl0sWzEsNSwiWzJdIl0sWzMsNCwiWzJdXntvcH1cXHN0YXJbMF0iXSxbMyw1LCJbMl0iXSxbMiw0LCJbMV1ee29wfVxcc3RhclswXSJdLFsyLDUsIlsxXSJdLFswLDEsInNeMSJdLFsyLDMsInNeMiJdLFs0LDIsImReMSJdLFs0LDUsInNeMSIsMl0sWzUsMywiZF4xIiwyXSxbNiw3LCJzXjEiLDJdLFs2LDAsImReMiJdLFs3LDEsImReMiIsMl0sWzksMTAsInNeMCIsMl0sWzgsOSwiZF4xIiwyXSxbOCwxMSwiZF4xIl0sWzExLDEwLCJzXjEiXSxbMTIsMTMsInNeMCJdLFsxNCwxMiwiZF4wIl0sWzE3LDE4LCJzXjEiLDJdLFsxNiwxOCwic14yIl0sWzE1LDE3LCJkXjIiLDJdLFsxNSwxNiwiZF4yIl0sWzE0LDEzLCJpZCIsMl0sWzIxLDE5LCJkXjAiXSxbMjEsMjIsInNeMSIsMl0sWzIyLDIwLCJkXjAiLDJdLFsxOSwyMF1d
\[\begin{tikzcd}
	{[1]} & {[0]^{op}\star[1]} & {[0]^{op}\star[0]} & {[1]^{op}\star[0]} \\
	& { [1]} & {[0]^{op}\star[1]} & {[1] } \\
	{[1]^{op}\star[0]} & {[1]^{op}\star[1]} & {[1]^{op}\star[0]} & {[2]^{op}\star[0]} \\
	{[1]} & {[2]} & {[1]} & {[2]} \\
	{[1]^{op}\star[0]} & {[2]^{op}\star[0]} & {[1]^{op}\star[0]} & {[2]^{op}\star[0]} \\
	{[1]^{op}\star[1]} & {[2]} & {[1]} & {[2]}
	\arrow["{s^1}", from=3-2, to=4-2]
	\arrow["{s^2}", from=3-4, to=4-4]
	\arrow["{d^1}", from=3-3, to=3-4]
	\arrow["{s^1}"', from=3-3, to=4-3]
	\arrow["{d^1}"', from=4-3, to=4-4]
	\arrow["{s^1}"', from=3-1, to=4-1]
	\arrow["{d^2}", from=3-1, to=3-2]
	\arrow["{d^2}"', from=4-1, to=4-2]
	\arrow["{s^0}"', from=2-3, to=2-4]
	\arrow["{d^1}"', from=1-3, to=2-3]
	\arrow["{d^1}", from=1-3, to=1-4]
	\arrow["{s^1}", from=1-4, to=2-4]
	\arrow["{s^0}", from=1-2, to=2-2]
	\arrow["{d^0}", from=1-1, to=1-2]
	\arrow["{s^1}"', from=6-1, to=6-2]
	\arrow["{s^2}", from=5-2, to=6-2]
	\arrow["{d^2}"', from=5-1, to=6-1]
	\arrow["{d^2}", from=5-1, to=5-2]
	\arrow["id"', from=1-1, to=2-2]
	\arrow["{d^0}", from=5-3, to=5-4]
	\arrow["{s^1}"', from=5-3, to=6-3]
	\arrow["{d^0}"', from=6-3, to=6-4]
	\arrow[from=5-4, to=6-4]
\end{tikzcd}\]
which induce commutative diagrams:
% https://q.uiver.app/?q=WzAsMjQsWzEsMywiZVxcc3RhclthLDJdIl0sWzEsMiwiW2VcXHN0YXIgYSwxXVxcdmVlW2EsMV0iXSxbMiwyLCJbZVxcc3RhciBhLDFdIl0sWzIsMywiZVxcc3RhclsgYSwxXSJdLFszLDMsImVcXHN0YXJbYSwyXSJdLFszLDIsIltbMl1cXGJvdGltZXMgYSwxXSJdLFswLDIsIltlXFxzdGFyIGEsMV0iXSxbMCwzLCJlXFxzdGFyWyBhLDFdIl0sWzIsMCwiW2EsMV0iXSxbMiwxLCJbZSwxXVxcdmVlW2EsMV0iXSxbMywwLCJbZVxcc3RhciBhLDFdIl0sWzMsMSwiZVxcc3RhciBbYSwxXSJdLFswLDAsIlthLDFdIl0sWzEsMCwiW2UsMV1cXHZlZVthLDFdIl0sWzEsMSwiZVxcc3RhciBbYSwxXSJdLFswLDQsIltbMV1cXG90aW1lcyBhLDFdIl0sWzAsNSwiW2VcXHN0YXIgYSwxXVxcdmVlW2EsMV0iXSxbMSw1LCJlXFxzdGFyW2EsMl0iXSxbMSw0LCJbWzJdXFxib3RpbWVzIGEsMV0iXSxbMCwxLCJlXFxzdGFyIFthLDFdIixbMCwwLDEwMCwxXV0sWzIsNCwiW2VcXHN0YXIgICBhLDFdIl0sWzMsNCwiW1syXVxcYm90aW1lcyBhLDFdIl0sWzMsNSwiZVxcc3RhclthLDJdIl0sWzIsNSwiZVxcc3RhclsgYSwxXSJdLFsxLDAsIlxcZGVsdGFfYSJdLFsyLDMsIlxcYWxwaGFfYSIsMl0sWzIsNSwiW2ReMVxcYm90aW1lcyBhLDFdIl0sWzUsNCwiXFxlcHNpbG9uX2EiXSxbMyw0LCJlXFxzdGFyIFthLGReMV0iLDJdLFs2LDcsIlxcYWxwaGFfYSIsMl0sWzcsMCwiZVxcc3RhclthLGReMl0iLDJdLFs2LDEsIltlXFxzdGFyIGEsZF4yXSJdLFs4LDksIlthLGReMV0iLDJdLFs4LDEwLCJbZF57MH1cXHN0YXIgYSwxXSJdLFsxMCwxMSwiXFxhbHBoYV9hIl0sWzksMTEsIlxcYmV0YV9hIl0sWzEzLDE0LCJcXGJldGFfYSJdLFsxMiwxMywiW2EsZF4wXSJdLFs2LDcsIigzKToiLDEseyJvZmZzZXQiOjUsImN1cnZlIjo1LCJzdHlsZSI6eyJib2R5Ijp7Im5hbWUiOiJub25lIn0sImhlYWQiOnsibmFtZSI6Im5vbmUifX19XSxbMTUsMTYsIltbMV1cXG90aW1lcyBhLGReMV0iLDJdLFsxOCwxNywiXFxlcHNpbG9uX2EiXSxbMTUsMTgsIltkXjBcXG90aW1lcyBhLDFdIl0sWzE1LDE2LCIoNSk6IiwxLHsib2Zmc2V0Ijo1LCJjdXJ2ZSI6NSwic3R5bGUiOnsiYm9keSI6eyJuYW1lIjoibm9uZSJ9LCJoZWFkIjp7Im5hbWUiOiJub25lIn19fV0sWzE2LDE3LCJcXGRlbHRhX2EiLDJdLFs1LDQsIjooNCkiLDEseyJvZmZzZXQiOi01LCJjdXJ2ZSI6LTUsInN0eWxlIjp7ImJvZHkiOnsibmFtZSI6Im5vbmUifSwiaGVhZCI6eyJuYW1lIjoibm9uZSJ9fX1dLFsxMiwxNCwiZF57MH1cXHN0YXIge1thLDFdfSIsMl0sWzEyLDE5LCIoMSk6IiwxLHsib2Zmc2V0Ijo1LCJjdXJ2ZSI6NSwic3R5bGUiOnsiYm9keSI6eyJuYW1lIjoibm9uZSJ9LCJoZWFkIjp7Im5hbWUiOiJub25lIn19fV0sWzIwLDIxLCJbZF4yXFxib3RpbWVzIGEsMV0iXSxbMjMsMjIsImVcXHN0YXJbYSxkXjBdIiwyXSxbMjAsMjMsIlxcYWxwaGFfYSIsMl0sWzIxLDIyLCJcXGVwc2lsb25fYSJdLFsyMSwyMiwiOig2KSIsMSx7Im9mZnNldCI6LTUsImN1cnZlIjotNSwic3R5bGUiOnsiYm9keSI6eyJuYW1lIjoibm9uZSJ9LCJoZWFkIjp7Im5hbWUiOiJub25lIn19fV0sWzEwLDExLCI6KDIpIiwxLHsib2Zmc2V0IjotNSwiY3VydmUiOi01LCJzdHlsZSI6eyJib2R5Ijp7Im5hbWUiOiJub25lIn0sImhlYWQiOnsibmFtZSI6Im5vbmUifX19XV0=
\[\begin{tikzcd}
	{[a,1]} & {[e,1]\vee[a,1]} & {[a,1]} & {[e\star a,1]} \\
	\textcolor{white}{e\star [a,1]} & {e\star [a,1]} & {[e,1]\vee[a,1]} & {e\star [a,1]} \\
	{[e\star a,1]} & {[e\star a,1]\vee[a,1]} & {[e\star a,1]} & {[[2]\botimes a,1]} \\
	{e\star[ a,1]} & {e\star[a,2]} & {e\star[ a,1]} & {e\star[a,2]} \\
	{[[1]\otimes a,1]} & {[[2]\botimes a,1]} & {[e\star a,1]} & {[[2]\botimes a,1]} \\
	{[e\star a,1]\vee[a,1]} & {e\star[a,2]} & {e\star[ a,1]} & {e\star[a,2]}
	\arrow["{\delta_a}", from=3-2, to=4-2]
	\arrow["{\alpha_a}"', from=3-3, to=4-3]
	\arrow["{[d^1\botimes a,1]}", from=3-3, to=3-4]
	\arrow["{\epsilon_a}", from=3-4, to=4-4]
	\arrow["{e\star [a,d^1]}"', from=4-3, to=4-4]
	\arrow["{\alpha_a}"', from=3-1, to=4-1]
	\arrow["{e\star[a,d^2]}"', from=4-1, to=4-2]
	\arrow["{[e\star a,d^2]}", from=3-1, to=3-2]
	\arrow["{[a,d^1]}"', from=1-3, to=2-3]
	\arrow["{[d^{0}\star a,1]}", from=1-3, to=1-4]
	\arrow["{\alpha_a}", from=1-4, to=2-4]
	\arrow["{\beta_a}", from=2-3, to=2-4]
	\arrow["{\beta_a}", from=1-2, to=2-2]
	\arrow["{[a,d^0]}", from=1-1, to=1-2]
	\arrow["{(3):}"{description}, shift right=5, curve={height=30pt}, draw=none, from=3-1, to=4-1]
	\arrow["{[[1]\otimes a,d^1]}"', from=5-1, to=6-1]
	\arrow["{\epsilon_a}", from=5-2, to=6-2]
	\arrow["{[d^0\otimes a,1]}", from=5-1, to=5-2]
	\arrow["{(5):}"{description}, shift right=5, curve={height=30pt}, draw=none, from=5-1, to=6-1]
	\arrow["{\delta_a}"', from=6-1, to=6-2]
	\arrow["{:(4)}"{description}, shift left=5, curve={height=-30pt}, draw=none, from=3-4, to=4-4]
	\arrow["{d^{0}\star {[a,1]}}"', from=1-1, to=2-2]
	\arrow["{(1):}"{description}, shift right=5, curve={height=30pt}, draw=none, from=1-1, to=2-1]
	\arrow["{[d^2\botimes a,1]}", from=5-3, to=5-4]
	\arrow["{e\star[a,d^0]}"', from=6-3, to=6-4]
	\arrow["{\alpha_a}"', from=5-3, to=6-3]
	\arrow["{\epsilon_a}", from=5-4, to=6-4]
	\arrow["{:(6)}"{description}, shift left=5, curve={height=-30pt}, draw=none, from=5-4, to=6-4]
	\arrow["{:(2)}"{description}, shift left=5, curve={height=-30pt}, draw=none, from=1-4, to=2-4]
\end{tikzcd}\]
\end{construction}
\begin{definition}
Let $b$ be an object of $A$ and $x:a\to b,~x':a'\to b$ two morphisms. The element $b$ is \wcnotion{$n$-relying on $x$}{relying on $x$@$n$-relying on $x$} if for any $k\geq -1$, the following square is homotopy cocartesian:
% https://q.uiver.app/?q=WzAsNCxbMCwxLCJcXHRhdV5pX3tuK2srMX0oW2tdXFxzdGFyIGEpIl0sWzEsMSwiXFx0YXVeaV97bitrKzF9KFtrXVxcc3RhciBiKSJdLFswLDAsIltrXVxcc3RhciBhIl0sWzEsMCwiW2tdXFxzdGFyIGIiXSxbMiwzXSxbMCwxXSxbMiwwXSxbMywxXV0=
\[\begin{tikzcd}
	{[k]\star a} & {[k]\star b} \\
	{\tau^i_{n+k+1}([k]\star a)} & {\tau^i_{n+k+1}([k]\star b)}
	\arrow[from=1-1, to=1-2]
	\arrow[from=2-1, to=2-2]
	\arrow[from=1-1, to=2-1]
	\arrow[from=1-2, to=2-2]
\end{tikzcd}\]
The element $b$ is \wcnotion{$n$-relying on $x$ and $x'$}{relying on $x$ and $x'$@$n$-relying on $x$ and $x'$} if for any $k\geq -1$, the following square is homotopy cocartesian:
% https://q.uiver.app/?q=WzAsNCxbMCwxLCJcXHRhdV5pX3tuK2srMX0oW2tdXFxzdGFyIGEpXFxhbWFsZyBcXHRhdV5pX3tuK2srMX0oW2tdXFxzdGFyIGEnKSJdLFsxLDEsIlxcdGF1Xmlfe24raysxfShba11cXHN0YXIgYikiXSxbMCwwLCJba11cXHN0YXIgYVxcYW1hbGcgW2tdXFxzdGFyIGEnIl0sWzEsMCwiW2tdXFxzdGFyIGIiXSxbMCwyLCIiLDAseyJzdHlsZSI6eyJ0YWlsIjp7Im5hbWUiOiJhcnJvd2hlYWQifSwiaGVhZCI6eyJuYW1lIjoibm9uZSJ9fX1dLFsyLDNdLFswLDFdLFsxLDMsIiIsMCx7InN0eWxlIjp7InRhaWwiOnsibmFtZSI6ImFycm93aGVhZCJ9LCJoZWFkIjp7Im5hbWUiOiJub25lIn19fV1d
\[\begin{tikzcd}
	{[k]\star a\amalg [k]\star a'} & {[k]\star b} \\
	{\tau^i_{n+k+1}([k]\star a)\amalg \tau^i_{n+k+1}([k]\star a')} & {\tau^i_{n+k+1}([k]\star b)}
	\arrow[tail reversed, no head, from=2-1, to=1-1]
	\arrow[from=1-1, to=1-2]
	\arrow[from=2-1, to=2-2]
	\arrow[tail reversed, no head, from=2-2, to=1-2]
\end{tikzcd}\]
\end{definition}

\p We recall that we denote by $C_{\mk}$ the marked Segal $A$-precategory associated to a stratified Segal $A$-precategory $C$. The canonical inclusion $C\to C_{\mk}$ is denoted $r_C$ and is an acyclic cofibration according to he proposition \ref{prop:X to Xmk is acycli cof}. These notions and notations are defined in paragraph \ref{para:marked segal}.
The fact that will be used the most with the marked Segal $A$-precategory is their right lifting property with respect to morphisms of shape $[\tau^i_n(a),\Lambda^1[2]]\cup [a,2]\to [\tau^i_n(a),2]$. This fact will  be used freely.

\begin{definition}
\label{def:order relation case 1}
Let $C$ be a Segal $A$-precategory. We define the relation \wcnotation{$\geq_{n}$}{((g37@$\geq_{n}$} on morphisms of shape $[a,1]\to C$ for $a$ verifying $\tau^i_{n}a = a$, as the smallest reflexive and transitive relation such that $(x:[a,1]\to C) \geq_n (x':[a',1]\to C)$ whenever one of the three following conditions is verified:
\begin{enumerate}
\item The elements $a$ and $a'$ are equal and there exists a lifting the following diagram:
% https://q.uiver.app/?q=WzAsNCxbMCwwLCJbYSwxXSJdLFsxLDEsIlthLDFdXFx2ZWVbXFx0YXVeaV97bi0xfWEsMV0iXSxbMCwyLCJbYSwxXSJdLFsyLDEsIkMiXSxbMCwxLCJbYSxkXjJdIiwyXSxbMCwzLCJ4IiwwLHsiY3VydmUiOi0yfV0sWzIsMSwiW2EsZF4xXSJdLFsxLDMsIiIsMSx7InN0eWxlIjp7ImJvZHkiOnsibmFtZSI6ImRvdHRlZCJ9fX1dLFsyLDMsIngnIiwyLHsiY3VydmUiOjJ9XV0=
\[\begin{tikzcd}
	{[a,1]} \\
	& {[a,1]\vee[\tau^i_{n-1}a,1]} & C \\
	{[a,1]}
	\arrow["{[a,d^2]}"', from=1-1, to=2-2]
	\arrow["x", curve={height=-12pt}, from=1-1, to=2-3]
	\arrow["{[a,d^1]}", from=3-1, to=2-2]
	\arrow[dotted, from=2-2, to=2-3]
	\arrow["{x'}"', curve={height=12pt}, from=3-1, to=2-3]
\end{tikzcd}\]
\item The elements $a$ and $a'$ are equal and there exists a lifting in the following diagram:
% https://q.uiver.app/?q=WzAsNCxbMCwwLCJbYSwxXSJdLFsxLDEsIltcXHRhdV5pX3tuLTF9YSwxXVxcdmVlW2EsMV0iXSxbMCwyLCJbYSwxXSJdLFsyLDEsIkMiXSxbMCwxLCJbYSxkXjBdIiwyXSxbMCwzLCJ4IiwwLHsiY3VydmUiOi0yfV0sWzIsMSwiW2EsZF4xXSJdLFsxLDMsIiIsMSx7InN0eWxlIjp7ImJvZHkiOnsibmFtZSI6ImRvdHRlZCJ9fX1dLFsyLDMsIngnIiwyLHsiY3VydmUiOjJ9XV0=
\[\begin{tikzcd}
	{[a,1]} \\
	& {[\tau^i_{n-1}a,1]\vee[a,1]} & C \\
	{[a,1]}
	\arrow["{[a,d^0]}"', from=1-1, to=2-2]
	\arrow["x", curve={height=-12pt}, from=1-1, to=2-3]
	\arrow["{[a,d^1]}", from=3-1, to=2-2]
	\arrow[dotted, from=2-2, to=2-3]
	\arrow["{x'}"', curve={height=12pt}, from=3-1, to=2-3]
\end{tikzcd}\]
\item There exists an element $b$ which is $(n-1)$-relying on $a\to b$ and dotted arrows in the following diagram:
% https://q.uiver.app/#q=WzAsNCxbMCwwLCJbYSwxXSJdLFswLDIsIlthJywxXSJdLFsyLDEsIkNfe1xcbWt9Il0sWzEsMSwiW2IsMV0iXSxbMCwzLCIgIiwyXSxbMSwzLCIiLDAseyJzdHlsZSI6eyJib2R5Ijp7Im5hbWUiOiJkb3R0ZWQifX19XSxbMCwyLCJyX0NcXGNpcmMgeCIsMCx7ImN1cnZlIjotMn1dLFsxLDIsInJfQ1xcY2lyYyB4JyIsMix7ImN1cnZlIjoyfV0sWzMsMiwiIiwxLHsic3R5bGUiOnsiYm9keSI6eyJuYW1lIjoiZG90dGVkIn19fV1d
\[\begin{tikzcd}
	{[a,1]} \\
	& {[b,1]} & {C_{\mk}} \\
	{[a',1]}
	\arrow["{ }"', from=1-1, to=2-2]
	\arrow[dotted, from=3-1, to=2-2]
	\arrow["{r_C\circ x}", curve={height=-12pt}, from=1-1, to=2-3]
	\arrow["{r_C\circ x'}"', curve={height=12pt}, from=3-1, to=2-3]
	\arrow[dotted, from=2-2, to=2-3]
\end{tikzcd}\]
\end{enumerate}
\end{definition}

\begin{definition}
\label{def:order relation case 2}
We also set $(\bar{x}:[\bar{a},1]\to C,\bar{x}':[\bar{a}',1]\to C)\geq_n \bar{x}'':[\bar{a}'',1]\to C$ if there exists three elements $x:[a,1]\to C$, $x':[a',1]\to C$ and $x'':[a'',1]\to C$ such that $\bar{x}\geq_n x$, $\bar{x}'\geq_n x'$, $x''\geq_n \bar{x}''$ and one of the two following conditions is verified:
\begin{enumerate}
\item The elements $a$, $a'$ and $a''$ are equal and there exists a dotted arrow:
% https://q.uiver.app/?q=WzAsNCxbMCwwLCJbYSwxXVxcY3VwW2EsMV0iXSxbMCwyLCJbYSwxXSJdLFsyLDEsIkMiXSxbMSwxLCJbYSwyXSJdLFswLDMsIlthLGReMlxcY3VwIGReMF0iLDIseyJsYWJlbF9wb3NpdGlvbiI6MTB9XSxbMSwzLCJbYSxkXjFdIiwwLHsibGFiZWxfcG9zaXRpb24iOjQwfV0sWzAsMiwieFxcY3VwIHgnIiwwLHsiY3VydmUiOi0yfV0sWzEsMiwieCcnIiwyLHsiY3VydmUiOjJ9XSxbMywyLCIiLDEseyJzdHlsZSI6eyJib2R5Ijp7Im5hbWUiOiJkb3R0ZWQifX19XV0=
\[\begin{tikzcd}
	{[a,1]\cup[a,1]} \\
	& {[a,2]} & C \\
	{[a,1]}
	\arrow["{[a,d^2\cup d^0]}"'{pos=0.1}, from=1-1, to=2-2]
	\arrow["{[a,d^1]}"{pos=0.4}, from=3-1, to=2-2]
	\arrow["{x\cup x'}", curve={height=-12pt}, from=1-1, to=2-3]
	\arrow["{x''}"', curve={height=12pt}, from=3-1, to=2-3]
	\arrow[dotted, from=2-2, to=2-3]
\end{tikzcd}\]
\item There exists an element $b$ which is $(n-1)$-relying on $a\to b$ and $a'\to b$ and dotted arrows in the following diagram:
% https://q.uiver.app/#q=WzAsNCxbMCwwLCJbYSwxXVxcYW1hbGdbYScsMV0iXSxbMCwyLCJbYScnLDFdIl0sWzIsMSwiQ197bWt9Il0sWzEsMSwiW2IsMV0iXSxbMCwzXSxbMSwzLCIiLDAseyJsYWJlbF9wb3NpdGlvbiI6NDAsInN0eWxlIjp7ImJvZHkiOnsibmFtZSI6ImRvdHRlZCJ9fX1dLFswLDIsInJfQ1xcY2lyYyB4XFxhbWFsZyByX0NcXGNpcmMgeCciLDAseyJjdXJ2ZSI6LTJ9XSxbMSwyLCJyX0NcXGNpcmMgeCcnIiwyLHsiY3VydmUiOjJ9XSxbMywyLCIiLDEseyJzdHlsZSI6eyJib2R5Ijp7Im5hbWUiOiJkb3R0ZWQifX19XV0=
\[\begin{tikzcd}
	{[a,1]\amalg[a',1]} \\
	& {[b,1]} & {C_{mk}} \\
	{[a'',1]}
	\arrow[from=1-1, to=2-2]
	\arrow[dotted, from=3-1, to=2-2]
	\arrow["{r_C\circ x\amalg r_C\circ x'}", curve={height=-12pt}, from=1-1, to=2-3]
	\arrow["{r_C\circ x''}"', curve={height=12pt}, from=3-1, to=2-3]
	\arrow[dotted, from=2-2, to=2-3]
\end{tikzcd}\]
\end{enumerate}
\end{definition}


\begin{prop}
\label{prop:meaning of geq case 1}
Let $C$ be a stratified Segal $A$-precategory and $x:[a,1]\to C$, $y:[a',1]\to C$ two morphisms such that $x\geq_n y$. The morphism 
$$ C\coprod_{[a,1]} \tau^i_n( [a,1])\to \tau^i_n( [a',1])\coprod_{[a',1]}C\coprod_{[a,1]} \tau^i_n( [a,1])$$
is an acyclic cofibration. 
\end{prop}
\begin{proof}
By two out of three, we can suppose without loss of generality that $C$ is already a marked Segal $A$-precategory. We suppose first that $x$ and $y$ fulfill one of the three cases of definition \ref{def:order relation case 1}. The following square is then homotopy cartesian: 
% https://q.uiver.app/?q=WzAsNCxbMSwwLCJDIl0sWzAsMCwiW2EsMV0iXSxbMCwxLCJcXHRhdV5pX25bYSwxXSJdLFsxLDEsIlxcdGF1XmlfblthLDFdXFxhbWFsZ197W2EsMV19IENcXGNvcHJvZF97W2EnLDFdfSBcXHRhdV5pX25bYScsMV0iXSxbMSwwLCJ4Il0sWzEsMl0sWzAsM10sWzIsM11d
\[\begin{tikzcd}
	{[a,1]} & C \\
	{\tau^i_n[a,1]} & {\tau^i_n[a,1]\amalg_{[a,1]} C\coprod_{[a',1]} \tau^i_n[a',1]}
	\arrow["x", from=1-1, to=1-2]
	\arrow[from=1-1, to=2-1]
	\arrow[from=1-2, to=2-2]
	\arrow[from=2-1, to=2-2]
\end{tikzcd}\]
As the cocartesian square:
% https://q.uiver.app/?q=WzAsNCxbMSwwLCJDIl0sWzAsMCwiW2EsMV0iXSxbMCwxLCJcXHRhdV5pX25bYSwxXSJdLFsxLDEsIlxcdGF1XmlfblthLDFdXFxhbWFsZ197W2EsMV19IEMiXSxbMSwwLCJ4Il0sWzEsMl0sWzAsM10sWzIsM10sWzMsMSwiIiwwLHsic3R5bGUiOnsibmFtZSI6ImNvcm5lciJ9fV1d
\[\begin{tikzcd}
	{[a,1]} & C \\
	{\tau^i_n[a,1]} & {\tau^i_n[a,1]\amalg_{[a,1]} C}
	\arrow["x", from=1-1, to=1-2]
	\arrow[from=1-1, to=2-1]
	\arrow[from=1-2, to=2-2]
	\arrow[from=2-1, to=2-2]
	\arrow["\lrcorner"{anchor=center, pos=0.125, rotate=180}, draw=none, from=2-2, to=1-1]
\end{tikzcd}\]
is also homotopy cocartesian, this
 implies that $$C\coprod_{[a,1]} \tau^i_n( [a,1])\to \tau^i_n( [a',1])\coprod_{[a',1]}C\coprod_{[a,1]} \tau^i_n( [a,1])$$ is an acyclic cofibration. Suppose now that there exists a familly of morphisms $(x_k:[a_k,1])_{k\leq m}\to C$ such that $x_0=x$, $x_m=y$ and for any $k$, $x_k$ and $x_{k+1}$ fullfill one of the three cases of definition \ref{def:order relation case 1}. We then have two homotopy cocartesian squares:
% https://q.uiver.app/?q=WzAsNixbMiwwLCJDIl0sWzEsMCwiW2EsMV0iXSxbMSwxLCJcXHRhdV5pX25bYSwxXSJdLFsyLDEsIkNcXGNvcHJvZF97XFxjb3Byb2Rfe2tcXGxlcSBtfVthX2ssMV19XFxjb3Byb2Rfe2tcXGxlcSBtfVxcdGF1XmlfblthX2ssMV0iXSxbMCwwLCJDXFxjb3Byb2Rfe1thJywxXX0gXFx0YXVeaV9uW2EnLDFdIl0sWzAsMSwiQ1xcY29wcm9kX3tcXGNvcHJvZF97a1xcbGVxIG19W2FfaywxXX1cXGNvcHJvZF97a1xcbGVxIG19XFx0YXVeaV9uW2FfaywxXSJdLFsxLDBdLFsxLDJdLFswLDNdLFsyLDNdLFsxLDRdLFsyLDVdLFs0LDVdXQ==
\[\begin{tikzcd}
	{C\coprod_{[a',1]} \tau^i_n[a',1]} & {[a,1]} & C \\
	{C\coprod_{\coprod_{k\leq m}[a_k,1]}\coprod_{k\leq m}\tau^i_n[a_k,1]} & {\tau^i_n[a,1]} & {C\coprod_{\coprod_{k\leq m}[a_k,1]}\coprod_{k\leq m}\tau^i_n[a_k,1]}
	\arrow[from=1-2, to=1-3]
	\arrow[from=1-2, to=2-2]
	\arrow[from=1-3, to=2-3]
	\arrow[from=2-2, to=2-3]
	\arrow[from=1-2, to=1-1]
	\arrow[from=2-2, to=2-1]
	\arrow[from=1-1, to=2-1]
\end{tikzcd}\]
As before, this implies that $$C\coprod_{[a,1]} \tau^i_n( [a,1])\to C\coprod_{\coprod_{k\leq m}[a_k,1]}\coprod_{k\leq m}\tau^i_n[a_k,1]$$ and $$\tau^i_n( [a',1])\coprod_{[a',1]}C\coprod_{[a,1]} \tau^i_n( [a,1])\to C\coprod_{\coprod_{k\leq m}[a_k,1]}\coprod_{k\leq m}\tau^i_n[a_k,1]$$ are acyclic cofibrations. 
By two out of three, this implies the result.
\end{proof}

One can show similarly:

\begin{prop}
\label{prop:meaning of geq case 2}
Let $C$ be a stratified Segal $A$-precategory, and $x:[a,1]\to C$, $y:[a',1]\to C$ and $z:[a'',1]\to C$ three morphisms such that $(x,y)\geq_n z$. The morphism 
$$ \tau^i_n( [a',1])\coprod_{[a',1]}C\coprod_{[a,1]} \tau^i_n( [a,1])\to \tau^i_n( [a',1])\coprod_{[a',1]}C\coprod_{[a,1]} \tau^i_n( [a,1]) \coprod_{[a'',1]} \tau^i_n( [a'',1])$$
is an acyclic cofibration. 
\end{prop}


\begin{lemma}
\label{lem:abstract thinness 0}
Let $n$ be a non null integer and $a$ an element such that $\tau^i_{n}(a)=a$. The object $[2]^2\otimes a$ is $n$-relying on $d^1\botimes a:e\star a\to [2]^2\botimes a$.
\end{lemma}
\begin{proof}
As the morphism $d^1\botimes a:e\star a\to [2]^2\botimes a$ is a weak equivalence, so are the horizontal morphisms of the following diagram:
% https://q.uiver.app/?q=WzAsNCxbMCwwLCJba11cXHN0YXIgZVxcc3RhciBhIl0sWzEsMCwiW2tdXFxzdGFyKFsyXV4yXFxib3RpbWVzIGEpIl0sWzAsMSwiXFx0YXVeaV97bitrKzF9KFtrXVxcc3RhciBlXFxzdGFyIGEpIl0sWzEsMSwiXFx0YXVeaV97bitrKzF9KFtrXVxcc3RhcihbMl1eMlxcYm90aW1lcyBhKSkiXSxbMCwxLCJcXHNpbSJdLFswLDJdLFsxLDNdLFsyLDMsIlxcc2ltIiwyXV0=
\[\begin{tikzcd}
	{[k]\star e\star a} & {[k]\star([2]^2\botimes a)} \\
	{\tau^i_{n+k+1}([k]\star e\star a)} & {\tau^i_{n+k+1}([k]\star([2]^2\botimes a))}
	\arrow["\sim", from=1-1, to=1-2]
	\arrow[from=1-1, to=2-1]
	\arrow[from=1-2, to=2-2]
	\arrow["\sim"', from=2-1, to=2-2]
\end{tikzcd}\]
As the vertical morphisms are cofibrations, this implies that this square is homotopy cocartesian.
\end{proof}

\begin{lemma}
\label{lem:abstract thinness 1}
Let $n$ be a non null integer and $a$ an element such that $\tau^i_{n}(a)=a$. The object $[2]\botimes a$ is $n$-relying on $d^0\otimes a:[1]\otimes a\to [2]\botimes a$ and $d^2\otimes a:e\star a\to [2]\otimes a$. Moreover, $[2]\botimes a\coprod_{d^0\otimes a} \tau^i_{n}([1]\otimes a)$ (resp. $[2]\botimes a\coprod_{d^2\botimes a} \tau^i_{n}(e\star a)$) is $n$-relying on $d^{2}\otimes a$ (resp. $d^0\botimes a$).
\end{lemma}
\begin{proof}
Consider the following diagram:
% https://q.uiver.app/?q=WzAsNixbMiwwLCJba11cXHN0YXIoWzJdXFxvdGltZXMgYSkiXSxbMCwwLCJba11cXHN0YXIoWzFdXFxvdGltZXMgYSlcXGFtYWxnIFtrXVxcc3RhcihbMV1cXG90aW1lcyBhKSJdLFsxLDAsIltrXVxcc3RhcihcXExhbWJkYV4xWzJdXFxvdGltZXMgYSkiXSxbMCwxLCJcXHRhdV5pX3tuK2srMX0oW2tdXFxzdGFyKFsxXVxcb3RpbWVzIGEpKVxcYW1hbGcgXFx0YXVeaV97bitrKzF9KFtrXVxcc3RhcihbMV1cXG90aW1lcyBhKSkiXSxbMSwxLCJcXHRhdV5pX3tuK2srMX0oW2tdXFxzdGFyKFxcTGFtYmRhXjFbMl1cXG90aW1lcyBhKSkiXSxbMiwxLCJcXHRhdV5pX3tuK2srMX0oW2tdXFxzdGFyKFsyXVxcb3RpbWVzIGEpKSJdLFsxLDNdLFsxLDJdLFsyLDRdLFszLDRdLFsyLDAsIlxcc2ltIl0sWzQsNSwiXFxzaW0iLDJdLFswLDVdLFs0LDcsIiIsMSx7ImxldmVsIjoxLCJzdHlsZSI6eyJuYW1lIjoiY29ybmVyIn19XV0=
\[\begin{tikzcd}[column sep=0.3cm]
	{[k]\star([1]\otimes a)\amalg [k]\star([1]\otimes a)} & {[k]\star(\Lambda^1[2]\otimes a)} & {[k]\star([2]\otimes a)} \\
	{\tau^i_{n+k+1}([k]\star([1]\otimes a))\amalg \tau^i_{n+k+1}([k]\star([1]\otimes a))} & {\tau^i_{n+k+1}([k]\star(\Lambda^1[2]\otimes a))} & {\tau^i_{n+k+1}([k]\star([2]\otimes a))}
	\arrow[from=1-1, to=2-1]
	\arrow[""{name=0, anchor=center, inner sep=0}, from=1-1, to=1-2]
	\arrow[from=1-2, to=2-2]
	\arrow[from=2-1, to=2-2]
	\arrow["\sim", from=1-2, to=1-3]
	\arrow["\sim"', from=2-2, to=2-3]
	\arrow[from=1-3, to=2-3]
	\arrow["\lrcorner"{anchor=center, pos=0.125, rotate=180}, draw=none, from=2-2, to=0]
\end{tikzcd}\]
The left square is cocartesian and so homotopy cocartesian. Horizontal morphisms of the right square are weak equivalences, so this square is also homotopy cocartesian.
The outer square is then homotopy cocartesian and this implies that 
$[[2]\otimes a,1]$ is $n$-relying on $d^0\otimes a$ and $d^2\otimes a$. We then have a diagram:
% https://q.uiver.app/?q=WzAsNixbMSwwLCJba11cXHN0YXIoWzJdXFxvdGltZXMgYSkiXSxbMSwxLCJcXHRhdV5pX3tuK2srMX0oW2tdXFxzdGFyKFsyXVxcb3RpbWVzIGEpKSJdLFsyLDAsIltrXVxcc3RhcihbMl1cXGJvdGltZXMgYSkiXSxbMiwxLCJcXHRhdV5pX3tuK2srMX0oW2tdXFxzdGFyKFsyXVxcYm90aW1lcyBhKSkiXSxbMCwwLCJba11cXHN0YXIoWzFdXFxvdGltZXMgYSlcXGFtYWxnIFtrXVxcc3RhcihbMV1cXG90aW1lcyBhKSJdLFswLDEsIlxcdGF1Xmlfe24raysxfShba11cXHN0YXIoWzFdXFxvdGltZXMgYSkpXFxhbWFsZyBcXHRhdV5pX3tuK2srMX0oW2tdXFxzdGFyKFsxXVxcb3RpbWVzIGEpKSJdLFswLDJdLFswLDFdLFsxLDNdLFsyLDNdLFs1LDFdLFs0LDBdLFs0LDVdLFszLDYsIiIsMix7ImxldmVsIjoxLCJzdHlsZSI6eyJuYW1lIjoiY29ybmVyIn19XV0=
\[\begin{tikzcd}[column sep=0.3cm]
	{[k]\star([1]\otimes a)\amalg [k]\star([1]\otimes a)} & {[k]\star([2]\otimes a)} & {[k]\star([2]\botimes a)} \\
	{\tau^i_{n+k+1}([k]\star([1]\otimes a))\amalg \tau^i_{n+k+1}([k]\star([1]\otimes a))} & {\tau^i_{n+k+1}([k]\star([2]\otimes a))} & {\tau^i_{n+k+1}([k]\star([2]\botimes a))}
	\arrow[""{name=0, anchor=center, inner sep=0}, from=1-2, to=1-3]
	\arrow[from=1-2, to=2-2]
	\arrow[from=2-2, to=2-3]
	\arrow[from=1-3, to=2-3]
	\arrow[from=2-1, to=2-2]
	\arrow[from=1-1, to=1-2]
	\arrow[from=1-1, to=2-1]
	\arrow["\lrcorner"{anchor=center, pos=0.125, rotate=180}, draw=none, from=2-3, to=0]
\end{tikzcd}\]
where the two squares are homotopy cocartesian and so is the outer one. This implies the first assertion and the two others follow easily.
\end{proof}

\begin{lemma}
\label{lem:abstract thinness 2}
Let $n$ be an integer strictly superior to $1$ and $a$ such that $\tau^i_{n}(a) = a$.
We consider the projection $\pi:[a,2]\to [a,1]\vee[\tau^i_{n-1}(a),1]$ and $\pi':[a,2]\to [\tau^i_{n-1}(a),1]\vee[a,1]$. We then have inequalities
$$ e\star\pi\circ \epsilon_a\circ [d^0\otimes a, 1]\geq_{n+1} e\star\pi\circ \epsilon_a\circ [d^1\botimes a, 1]$$ and 
$$ e\star\pi'\circ \epsilon_a\circ [d^2\botimes a, 1]\geq_{n+1}e\star \pi\circ \epsilon_a\circ [d^1\botimes a, 1].$$
\end{lemma}
\begin{proof}
Using the diagram $(6).\ref{cons:the big construction}$ we get a diagram
% https://q.uiver.app/#q=WzAsNyxbMCwwLCJbZVxcc3RhciBhLDFdIl0sWzEsMCwiW1syXVxcYm90aW1lcyBhLDFdIl0sWzIsMSwiZVxcc3RhclthLDJdIl0sWzEsMSwiZVxcc3RhclsgYSwxXSJdLFsyLDIsImVcXHN0YXIoW2EsMV1cXHZlZVtcXHRhdV5pX3tuLTF9KGEpLDFdKSJdLFsxLDIsImVcXHN0YXJbXFx0YXVeaV97bi0xfShhKSwxXSJdLFswLDEsIltcXHRhdV5pX3tufShlXFxzdGFyIGEpLDFdIl0sWzAsMSwiW2ReMlxcYm90aW1lcyBhLDFdIl0sWzMsMiwiZVxcc3RhclthLGReMF0iLDJdLFswLDMsIlxcYWxwaGFfYSJdLFsxLDIsIlxcZXBzaWxvbl9hIl0sWzIsNCwiZVxcc3RhclxccGkiXSxbMyw1XSxbNSw0XSxbMCw2XSxbNiw1XV0=
\[\begin{tikzcd}
	{[e\star a,1]} & {[[2]\botimes a,1]} \\
	{[\tau^i_{n}(e\star a),1]} & {e\star[ a,1]} & {e\star[a,2]} \\
	& {e\star[\tau^i_{n-1}(a),1]} & {e\star([a,1]\vee[\tau^i_{n-1}(a),1])}
	\arrow["{[d^2\botimes a,1]}", from=1-1, to=1-2]
	\arrow["{e\star[a,d^0]}"', from=2-2, to=2-3]
	\arrow["{\alpha_a}", from=1-1, to=2-2]
	\arrow["{\epsilon_a}", from=1-2, to=2-3]
	\arrow["e\star\pi", from=2-3, to=3-3]
	\arrow[from=2-2, to=3-2]
	\arrow[from=3-2, to=3-3]
	\arrow[from=1-1, to=2-1]
	\arrow[from=2-1, to=3-2]
\end{tikzcd}\]
The morphism $r_{e\star([a,1]\vee[\tau^i_{n-1}(a),1])}\circ e\star\pi\circ \epsilon_a$ then factors through $[[2]\botimes a\coprod_{d^2\botimes a} \tau^i_{n}(e\star a),1]$. According to lemma \ref{lem:abstract thinness 1}, we then get the first inequalities.

For the second inequality, using the diagrams $(3).\ref{cons:the big construction}$ and $(5).\ref{cons:the big construction}$, we have a diagram:
% https://q.uiver.app/#q=WzAsOSxbMCwyLCJbZVxcc3RhciBhLDFdIl0sWzEsMiwiZVxcc3RhclsgYSwxXSJdLFsxLDAsIltbMV1cXG90aW1lcyBhLDFdIl0sWzEsMSwiW2VcXHN0YXIgYSwxXVxcdmVlW2EsMV0iXSxbMiwxLCJlXFxzdGFyW2EsMl0iXSxbMiwwLCJbWzJdXFxib3RpbWVzIGEsMV0iXSxbMiwyLCJlXFxzdGFyKFtcXHRhdV5pX3tuLTF9KGEpLDFdXFx2ZWVbYSwxXSkiXSxbMSwzLCJlXFxzdGFyWyBcXHRhdV5pX3tuLTF9KGEpLDFdIl0sWzAsMywiW1xcdGF1Xmlfe259KGVcXHN0YXIgYSksMV0iXSxbMCwxLCJcXGFscGhhX2EiLDJdLFsyLDMsIltbMV1cXG90aW1lcyBhLGReMV0iLDJdLFs1LDQsIlxcZXBzaWxvbl9hIl0sWzIsNSwiW2ReMFxcb3RpbWVzIGEsMV0iXSxbMyw0LCJcXGRlbHRhX2EiXSxbNCw2LCJlXFxzdGFyIFxccGknIl0sWzAsMywiW2VcXHN0YXIgYSxkXjJdIl0sWzEsNCwiZVxcc3RhclthLGReMl0iLDFdLFsxLDddLFs3LDZdLFs4LDcsIlxcYWxwaGFfe1xcdGF1Xmlfe24tMX0oYSl9IiwyXSxbMCw4XV0=
\[\begin{tikzcd}
	& {[[1]\otimes a,1]} & {[[2]\botimes a,1]} \\
	& {[e\star a,1]\vee[a,1]} & {e\star[a,2]} \\
	{[e\star a,1]} & {e\star[ a,1]} & {e\star([\tau^i_{n-1}(a),1]\vee[a,1])} \\
	{[\tau^i_{n}(e\star a),1]} & {e\star[ \tau^i_{n-1}(a),1]}
	\arrow["{\alpha_a}"', from=3-1, to=3-2]
	\arrow["{[[1]\otimes a,d^1]}"', from=1-2, to=2-2]
	\arrow["{\epsilon_a}", from=1-3, to=2-3]
	\arrow["{[d^0\otimes a,1]}", from=1-2, to=1-3]
	\arrow["{\delta_a}", from=2-2, to=2-3]
	\arrow["{e\star \pi'}", from=2-3, to=3-3]
	\arrow["{[e\star a,d^2]}", from=3-1, to=2-2]
	\arrow["{e\star[a,d^2]}"{description}, from=3-2, to=2-3]
	\arrow[from=3-2, to=4-2]
	\arrow[from=4-2, to=3-3]
	\arrow["{\alpha_{\tau^i_{n-1}(a)}}"', from=4-1, to=4-2]
	\arrow[from=3-1, to=4-1]
\end{tikzcd}\]
This implies that $r_{e\star([\tau^i_{n-1}(a),1]\vee[a,1])}\circ e\star\pi'\circ e\star[a,d^2]\circ \alpha_a$ factors through $[\tau^i_n(e\star a),1]$. The morphism $r_{e\star([\tau^i_{n-1}(a),1]\vee[a,1])}\circ e\star\pi\circ \epsilon_a$ then factors through $[[2]\otimes a\coprod_{d^0\otimes a} \tau^i_{n}([1]\otimes a),1]$. According to lemma \ref{lem:abstract thinness 1}, we then get the second inequality.

\end{proof}


\begin{lemma}
\label{lem:abstract thinness 3}
Let $n$ be an integer strictly superior to $1$ and $a$ such that $\tau^i_{n}(a) = a$.
We then have $\delta_a\circ [e\star a,d^2]\geq_{n+1} \delta_a\circ [[1]\otimes a,d^1]$. 
\end{lemma}
\begin{proof}
There is a diagram:
% https://q.uiver.app/?q=WzAsOCxbMCwxLCJlXFxzdGFyW2EsMl0iXSxbMSwxLCJbZVxcc3RhciBhLDFdXFx2ZWVbYSwxXSJdLFsxLDAsIltlXFxzdGFyIGEsMV0iXSxbMSwyLCJbWzFdXFxvdGltZXMgYSwxXSJdLFsyLDAsIltbMV1cXG90aW1lcyBhLDFdIl0sWzIsMSwiW1sxXVxcb3RpbWVzIGEsMV1cXHZlZVthLDFdIl0sWzIsMiwiW1sxXVxcb3RpbWVzIGEsMV0iXSxbMCwwLCJbZVxcc3RhciBhLDFdIl0sWzEsMCwiXFxkZWx0YV9hIl0sWzIsMSwiW2VcXHN0YXIgYSxkXjJdIl0sWzMsMSwiW1sxXVxcb3RpbWVzIGEsZF4xXSIsMl0sWzQsNSwiW1sxXVxcb3RpbWVzIGEsZF4yXSJdLFs2LDMsImlkIl0sWzYsNSwiW1sxXVxcb3RpbWVzIGEsZF4xXSIsMl0sWzQsMl0sWzUsMV0sWzcsMiwiaWQiXV0=
\[\begin{tikzcd}
	{[e\star a,1]} & {[e\star a,1]} & {[[1]\otimes a,1]} \\
	{e\star[a,2]} & {[e\star a,1]\vee[a,1]} & {[[1]\otimes a,1]\vee[a,1]} \\
	& {[[1]\otimes a,1]} & {[[1]\otimes a,1]}
	\arrow["{\delta_a}", from=2-2, to=2-1]
	\arrow["{[e\star a,d^2]}", from=1-2, to=2-2]
	\arrow["{[[1]\otimes a,d^1]}"', from=3-2, to=2-2]
	\arrow["{[[1]\otimes a,d^2]}", from=1-3, to=2-3]
	\arrow["id", from=3-3, to=3-2]
	\arrow["{[[1]\otimes a,d^1]}"', from=3-3, to=2-3]
	\arrow[from=1-3, to=1-2]
	\arrow[from=2-3, to=2-2]
	\arrow["id", from=1-1, to=1-2]
\end{tikzcd}\]
As the morphism $[[1]\otimes a,1]\vee[a,1]\to [e\star a,1]\vee[a,1]$ factors through $[[1]\otimes a,1]\vee[\tau^i_{n}([1]\otimes a),1]$, we get the desired inequality.
\end{proof}





\begin{prop}
\label{prop:geqn stable by star}
Let $a$ be an object such that $\tau^i_{n}(a)=a$.
Let $x:[a,1]\to C,y:[a',1]\to C$ be two morphisms, such that $x\geq_ny$, then if we denote by $\bar{x} := e\star x\circ \alpha_a$ and $\bar{y} :=e\star y\circ \alpha_{a'} $, we have $\bar{x}\geq_{n+1}\bar{y}$.
\end{prop}
\begin{proof}
First, we suppose that we are in the first case of the definition \ref{def:order relation case 1}. We can then suppose without loss of generality that $C= [a,1]\vee[\tau^i_{n-1}(a),1]$. We denote by $\pi$ the projection of $[a,2]$ on $[a,1]\vee[\tau^i_{n-1}(a),1]$.
 Using the diagrams $(3).\ref{cons:the big construction}$, $(4).\ref{cons:the big construction}$ and $(5).\ref{cons:the big construction}$, we have a diagram:
% https://q.uiver.app/#q=WzAsOSxbMSwxLCJlXFxzdGFyW2EsMl0iXSxbMCwxLCJbZVxcc3RhciBhLDFdXFx2ZWVbYSwxXSJdLFsyLDAsIltlXFxzdGFyIGEsMV0iXSxbMiwxLCJlXFxzdGFyWyBhLDFdIl0sWzAsMiwiW2VcXHN0YXIgYSwxXSJdLFsxLDIsImVcXHN0YXJbIGEsMV0iXSxbMCwwLCJbWzFdXFxvdGltZXMgYSwxXSJdLFsxLDAsIltbMl1cXGJvdGltZXMgYSwxXSJdLFsyLDIsImVcXHN0YXIgKCBbYSwxXVxcdmVlW1xcdGF1Xmlfe24tMX0oYSksMV0pIl0sWzEsMCwiXFxkZWx0YV9hIl0sWzIsMywiXFxhbHBoYV9hIl0sWzQsNSwiXFxhbHBoYV9hIiwyXSxbNSwwLCJlXFxzdGFyW2EsZF4yXSJdLFs0LDEsIltlXFxzdGFyIGEsZF4yXSJdLFs2LDcsIltkXjBcXG90aW1lcyBhLDFdIl0sWzAsOCwiZVxcc3RhclxccGkiXSxbNiwxLCJbWzFdXFxvdGltZXMgYSxkXjFdIiwyXSxbNywwLCJcXGVwc2lsb25fYSJdLFsyLDcsIltkXjFcXGJvdGltZXMgYSwxXSIsMl0sWzMsMCwiZVxcc3RhciBbYSxkXjFdIiwyXV0=
\[\begin{tikzcd}
	{[[1]\otimes a,1]} & {[[2]\botimes a,1]} & {[e\star a,1]} \\
	{[e\star a,1]\vee[a,1]} & {e\star[a,2]} & {e\star[ a,1]} \\
	{[e\star a,1]} & {e\star[ a,1]} & {e\star ( [a,1]\vee[\tau^i_{n-1}(a),1])}
	\arrow["{\delta_a}", from=2-1, to=2-2]
	\arrow["{\alpha_a}", from=1-3, to=2-3]
	\arrow["{\alpha_a}"', from=3-1, to=3-2]
	\arrow["{e\star[a,d^2]}", from=3-2, to=2-2]
	\arrow["{[e\star a,d^2]}", from=3-1, to=2-1]
	\arrow["{[d^0\otimes a,1]}", from=1-1, to=1-2]
	\arrow["e\star\pi", from=2-2, to=3-3]
	\arrow["{[[1]\otimes a,d^1]}"', from=1-1, to=2-1]
	\arrow["{\epsilon_a}", from=1-2, to=2-2]
	\arrow["{[d^1\botimes a,1]}"', from=1-3, to=1-2]
	\arrow["{e\star [a,d^1]}"', from=2-3, to=2-2]
\end{tikzcd}\]
Thanks to lemmas \ref{lem:abstract thinness 2} and \ref{lem:abstract thinness 3}, this implies the result.

If we are in the second case of \ref{def:order relation case 1}, we can suppose that $C = [\tau^i_{n-1}(a),1]\vee[a,1]$, and we note by $\pi'$ the projection from $[a,2]\to[\tau^i_{n-1}(a),1]\vee[a,1]$ . Using the diagrams $(4).\ref{cons:the big construction}$ and $(6).\ref{cons:the big construction}$, we have a diagram:
% https://q.uiver.app/#q=WzAsNyxbMiwwLCJbZVxcc3RhciBhLDFdIl0sWzIsMSwiZVxcc3RhclsgYSwxXSJdLFsxLDEsImVcXHN0YXJbYSwyXSJdLFsxLDAsIltbMl1cXGJvdGltZXMgYSwxXSJdLFswLDAsIltlXFxzdGFyICAgYSwxXSJdLFswLDEsImVcXHN0YXJbIGEsMV0iXSxbMSwyLCJlXFxzdGFyKFtcXHRhdV5pX3tuLTF9KGEpLDFdXFx2ZWVbYSwxXSkiXSxbMCwxLCJcXGFscGhhX2EiXSxbMywyLCJcXGVwc2lsb25fYSJdLFs0LDMsIltkXjJcXGJvdGltZXMgYSwxXSJdLFs1LDIsImVcXHN0YXJbYSxkXjBdIiwyXSxbNCw1LCJcXGFscGhhX2EiLDJdLFsyLDYsImVcXHN0YXJcXHBpJyJdLFsxLDIsImVcXHN0YXIgW2EsZF4xXSJdLFswLDMsIltkXjFcXGJvdGltZXMgYSwxXSIsMl1d
\[\begin{tikzcd}
	{[e\star a,1]} & {[[2]\botimes a,1]} & {[e\star a,1]} \\
	{e\star[ a,1]} & {e\star[a,2]} & {e\star[ a,1]} \\
	& {e\star([\tau^i_{n-1}(a),1]\vee[a,1])}
	\arrow["{\alpha_a}", from=1-3, to=2-3]
	\arrow["{\epsilon_a}", from=1-2, to=2-2]
	\arrow["{[d^2\botimes a,1]}", from=1-1, to=1-2]
	\arrow["{e\star[a,d^0]}"', from=2-1, to=2-2]
	\arrow["{\alpha_a}"', from=1-1, to=2-1]
	\arrow["{e\star\pi'}", from=2-2, to=3-2]
	\arrow["{e\star [a,d^1]}", from=2-3, to=2-2]
	\arrow["{[d^1\botimes a,1]}"', from=1-3, to=1-2]
\end{tikzcd}\]
Thanks to lemmas \ref{lem:abstract thinness 2}, this implies the result.

If we are in the third case, it is a direct consequence of the naturality of $\alpha$, of the definition of $n$-reliability and of the fact that $(e\star C)_{\mk}\cong (e\star C_{\mk})_{\mk}$ as remarked in \ref{para:marked segal}.
\end{proof}


\begin{prop}
\label{prop:2geqn stable by star}
Let $x:[a,1]\to C$, $y:[a',1]\to C$ and $z:[a'',1]$ be three morphisms, such that $(x,y)\geq_nz$, then if we denote by $\bar{x} := e\star x\circ \alpha_a$, $\bar{y} :=e\star y\circ \alpha_{a'} $ and $\bar{z} :=e\star z\circ \alpha_{a''} $, we have $(\bar{x},\bar{y})\geq_{n+1}\bar{z}$.
\end{prop}
\begin{proof}
Suppose first that we are in the first case of the definition \ref{def:order relation case 2}. We can then suppose without loss of generality that $C= [a,2]$.
We define $\tilde{x}:=\epsilon_a\circ [d^0\otimes a,1]$. Diagram $(6).\ref{cons:the big construction}$ and
lemma \ref{lem:abstract thinness 2} imply that $(\tilde{x},\bar{y})\geq_{n+1} \bar{z}$. Eventually, diagrams $(3).\ref{cons:the big construction}$ and $(5).\ref{cons:the big construction}$ induce a diagram:
% https://q.uiver.app/?q=WzAsNixbMSwxLCJlXFxzdGFyW2EsMl0iXSxbMSwwLCJbZVxcc3RhciBhLDFdXFx2ZWVbYSwxXSJdLFswLDAsIltlXFxzdGFyIGEsMV0iXSxbMCwxLCJlXFxzdGFyWyBhLDFdIl0sWzIsMCwiW1sxXVxcb3RpbWVzIGEsMV0iXSxbMiwxLCJbWzJdXFxib3RpbWVzIGEsMV0iXSxbMSwwLCJcXGRlbHRhX2EiXSxbMiwzLCJcXGFscGhhX2EiLDJdLFszLDAsImVcXHN0YXJbYSxkXjJdIiwyXSxbMiwxLCJbZVxcc3RhciBhLGReMl0iXSxbNCw1LCJbZF4wXFxvdGltZXMgYSwxXSJdLFs0LDEsIltbMV1cXG90aW1lcyBhLGReMV0iLDJdLFs1LDAsIlxcZXBzaWxvbl9hIl1d
\[\begin{tikzcd}
	{[e\star a,1]} & {[e\star a,1]\vee[a,1]} & {[[1]\otimes a,1]} \\
	{e\star[ a,1]} & {e\star[a,2]} & {[[2]\botimes a,1]}
	\arrow["{\delta_a}", from=1-2, to=2-2]
	\arrow["{\alpha_a}"', from=1-1, to=2-1]
	\arrow["{e\star[a,d^2]}"', from=2-1, to=2-2]
	\arrow["{[e\star a,d^2]}", from=1-1, to=1-2]
	\arrow["{[d^0\otimes a,1]}", from=1-3, to=2-3]
	\arrow["{[[1]\otimes a,d^1]}"', from=1-3, to=1-2]
	\arrow["{\epsilon_a}", from=2-3, to=2-2]
\end{tikzcd}\]
wich implies that $\bar{x}\geq_{n+1}\tilde{x}$.


If we are in the second case of the definition, it is a direct consequence of the naturality of $\alpha$, of the definition of $n$-reliability and of the fact that $(e\star C)_{\mk}\cong (e\star C_{\mk})_{\mk}$ as remarked in paragraph \ref{para:marked segal}.
\end{proof}



\begin{lemma}
\label{lemma:case k 0}
For any $a$ such that $\tau^i_na=a$ and $x:[a,1]\to C$, if we denote by $\bar{x} := e\star x \circ d^0\star[a,1]$ and $\tilde{x}:= e\star x \circ \alpha_a\circ [d^0\star a,1]$, then $\bar{x}\geq_{n+1}\tilde{x}$.
\end{lemma}
\begin{proof}
Using the diagrams $(1).\ref{cons:the big construction}$ and $(2).\ref{cons:the big construction}$, we have a diagram:
% https://q.uiver.app/?q=WzAsNixbMCwyLCJbYSwxXSJdLFswLDEsIltlLDFdXFx2ZWVbYSwxXSJdLFsxLDEsImVcXHN0YXIgW2EsMV0iXSxbMCwwLCJbYSwxXSJdLFsxLDAsIltlXFxzdGFyIGEsMV0iXSxbMiwxLCJDIl0sWzAsMSwiW2EsZF4wXSJdLFswLDIsImReezB9XFxzdGFyIHtbYSwxXX0iLDJdLFsxLDIsIlxcYmV0YV9hIl0sWzMsNCwiW2ReezB9XFxzdGFyIGEsMV0iXSxbMywxLCJbYSxkXjFdIiwyXSxbNCwyLCJcXGFscGhhX2EiXSxbMiw1LCJlXFxzdGFyIHgiXV0=
\[\begin{tikzcd}
	{[a,1]} & {[e\star a,1]} \\
	{[e,1]\vee[a,1]} & {e\star [a,1]} & C \\
	{[a,1]}
	\arrow["{[a,d^0]}", from=3-1, to=2-1]
	\arrow["{d^{0}\star {[a,1]}}"', from=3-1, to=2-2]
	\arrow["{\beta_a}", from=2-1, to=2-2]
	\arrow["{[d^{0}\star a,1]}", from=1-1, to=1-2]
	\arrow["{[a,d^1]}"', from=1-1, to=2-1]
	\arrow["{\alpha_a}", from=1-2, to=2-2]
	\arrow["{e\star x}", from=2-2, to=2-3]
\end{tikzcd}\]
which implies the desired inequality.
\end{proof}


\p
\label{para:a cocartesian square for intelingent truncation} 
We now use these results to show that the thinness extensions are weak equivalences. We define by induction the morphism $\iota_n:[[n-1],1]\to[n]$ where $\iota_2:= \alpha_{[0]}$ and $\iota_{n+1} := e\star\iota_n\circ \alpha_{[n-1]}$.


 We can easily show by induction that $[n]$ is a colimit of terms which are all invariant under $\tau^i_{n-1}$ except the one corresponding to $\iota_n$.
 For any $n$ we then have a pushout square:
% https://q.uiver.app/?q=WzAsNCxbMCwwLCJbW24tMV0sMV0iXSxbMSwwLCJbbl0iXSxbMSwxLCJbbl1fdCJdLFswLDEsIltbbi0xXV90LDFdIl0sWzAsMSwiXFxpb3RhX24iXSxbMCwzXSxbMSwyXSxbMywyXSxbMiwwLCIiLDEseyJzdHlsZSI6eyJuYW1lIjoiY29ybmVyIn19XV0=
\[\begin{tikzcd}
	{[[n-1],1]} & {[n]} \\
	{[[n-1]_t,1]} & {[n]_t}
	\arrow["{\iota_n}", from=1-1, to=1-2]
	\arrow[from=1-1, to=2-1]
	\arrow[from=1-2, to=2-2]
	\arrow[from=2-1, to=2-2]
	\arrow["\lrcorner"{anchor=center, pos=0.125, rotate=180}, draw=none, from=2-2, to=1-1]
\end{tikzcd}\]

\begin{lemma}
\label{lem:thinness extension last step0}
For any $n$ and for any $k<n$, such that $k\neq n-2$, we have inequalities
$d^k\circ\iota_{n-1}\geq_{n-1}\iota_n \circ[d^k,1]$ and $(d^n\circ\iota_{n-1},d^{n-2}\circ \iota_{n-1})\geq_{n-1}\iota_n\circ [d^{n-2},1]$
\end{lemma}
\begin{proof}
We start by showing the first inequality by induction on $n$. If $n=2$, the only case is $k=1$, and the two morphisms are equal.

Suppose now the result true at the stage $n$. 
If $k>0$, we have
$$
\begin{array}{rllc}
d^{k}\circ\iota_{n}&=& e\star d^{k-1}\circ e\star \iota_{n-1}\circ \alpha_{[n-2]}\\
&\geq_{n}&e\star \iota_n \circ e\star [d^{k-1},1] \circ \alpha_{[n-2]}&\mbox{(induction hypothesis and \ref{prop:geqn stable by star})}\\
&=& e\star \iota_n \circ \alpha_{[n-1]} \circ [e\star d^{k-1},1]\\
&=& \iota_{n+1} \circ \alpha_{[n-1]} \circ [d^{k},1]
\end{array}
$$
We still have to deal with the case $k=0$. As $d^0:[n]\to [n+1]$ (resp $[d^0,1]:[[n-1],1]\to [[n],1]$) is equal to $d^0\star [n]$ (resp. $[d^0\star [n-1],1]$), this is exactly the content of lemma \ref{lemma:case k 0}.

For the second inequality, we proceed again by induction. We remark that this is true for $n=2$. Suppose now the result true at the stage $n$. We have
$$
\begin{array}{rllc}
(d^{n+1}\circ\iota_{n},d^{n-1}\iota_{n})&=& (e\star d^{n}\circ e\star \iota_{n-1}\circ\alpha_{[n-2]},e\star d^{n-2}\circ e\star \iota_{n-1}\circ\alpha_{[n-2]})\\
&\geq_{n-1}& e\star \iota_n\circ e\star[d^{n-2},1]\circ \alpha_{[n-2]}~~~~~\mbox{(induction hypothesis and \ref{prop:2geqn stable by star})}\\
&=& e\star \iota_n\circ e\star \alpha_{[n-1]}\circ[e\star d^{n-2},1]\\ 
&=& \iota_{n+1}\circ [d^{n-1},1]
\end{array}
$$
\end{proof}


\begin{lemma}
\label{lem:thinness extension last step1}
Let $0<k<n$ be two integers.
We denote by $\tau^k$ the projection $[n]\to [n]^k$. We then have $$\tau^k\circ \iota_n\circ [d^k,1]\geq_{n-1}\tau^k\circ d^k\circ\iota_{n-1}.$$
\end{lemma}
\begin{proof}
We demonstrate the result by induction on $n$. For the initialization, the only case is $n=2$ and $k=1$, and is obvious. 
Suppose now the result true at the stage $n$, and let $k>1$. 
We have inequalities:
$$\begin{array}{rlll}
\tau^k\circ \iota_{n+1}\circ [d^k,1] &=& e\star \tau^k\circ e\star \iota_n \circ \alpha_{[n-1]}\circ [d^k,1]\\
&=&\star \tau^k\circ e\star \iota_n \circ e\star [d^{k-1},1]\circ \alpha_{[n-2]}\\
&\geq_n & e\star \tau_k\circ e\star d^{k-1}\circ e\star \iota_{n-1}\circ \alpha_{[n-2]}& \mbox{(induction hypothesis and \ref{prop:geqn stable by star})}\\
&=& \tau_k\circ d^k\circ \iota_n
\end{array}$$
We still have to deal with the case $k=1$. Using diagrams $(1)$, $(2)$, $(4)$ and $(5)$, of construction \ref{cons:the big construction}, we get a diagram:
% https://q.uiver.app/?q=WzAsMTAsWzIsMiwiZVxcc3Rhcltbbi0xXSwxXSJdLFsxLDIsImVcXHN0YXJbW24tMl0sMV0iXSxbMSwxLCJlXFxzdGFyKFtlLDFdXFx2ZWVbW24tMl0sMV0pIl0sWzIsMSwiW24rMV0iXSxbMSwwLCJlXFxzdGFyW1tuLTJdLDFdIl0sWzIsMCwiW25dIl0sWzAsMiwiW1tuLTFdLDFdIl0sWzAsMCwiW1tuLTFdLDFdIl0sWzAsMSwiW1syXVxcYm90aW1lcyBbbi0yXSwxXSJdLFszLDEsIltuKzFdXjEiXSxbNSwzLCJkXjEiXSxbMCwzLCJlXFxzdGFyXFxpb3RhX24iLDJdLFsxLDAsImVcXHN0YXJbZF4wLDFdIiwyXSxbMSwyLCJlXFxzdGFyIFtbbi0xXSxkXjFdIiwyXSxbNCwyLCJlXFxzdGFyIFtbbi0xXSxkXjBdIl0sWzQsNSwiZVxcc3RhciBcXGlvdGFfe24tMX0iXSxbMiwzLCJlXFxzdGFyXFxiZXRhX3tbbi0xXX0iXSxbNiwxLCJcXGFscGhhX3tbbi0yXX0iLDJdLFs3LDQsIlxcYWxwaGFfe1tuLTJdfSJdLFs4LDIsImVcXHN0YXJcXHBpXFxjaXJjIFxcZXBzaWxvbl97W24tMl19Il0sWzcsOCwiW2ReMlxcYm90aW1lcyBbbi0yXSwxXSIsMl0sWzYsOCwiW2ReMVxcYm90aW1lcyBbbi0yXSwxXSJdLFszLDksIlxcdGF1XjEiXV0=
\[\begin{tikzcd}
	{[[n-1],1]} & {e\star[[n-2],1]} & {[n]} \\
	{[[2]\botimes [n-2],1]} & {e\star([e,1]\vee[[n-2],1])} & {[n+1]} & {[n+1]^1} \\
	{[[n-1],1]} & {e\star[[n-2],1]} & {e\star[[n-1],1]}
	\arrow["{d^1}", from=1-3, to=2-3]
	\arrow["{e\star\iota_n}"', from=3-3, to=2-3]
	\arrow["{e\star[d^0,1]}"', from=3-2, to=3-3]
	\arrow["{e\star [[n-1],d^1]}"', from=3-2, to=2-2]
	\arrow["{e\star [[n-1],d^0]}", from=1-2, to=2-2]
	\arrow["{e\star \iota_{n-1}}", from=1-2, to=1-3]
	\arrow["{e\star\beta_{[n-1]}}", from=2-2, to=2-3]
	\arrow["{\alpha_{[n-2]}}"', from=3-1, to=3-2]
	\arrow["{\alpha_{[n-2]}}", from=1-1, to=1-2]
	\arrow["{e\star\pi\circ \epsilon_{[n-2]}}", from=2-1, to=2-2]
	\arrow["{[d^2\botimes [n-2],1]}"', from=1-1, to=2-1]
	\arrow["{[d^1\botimes [n-2],1]}", from=3-1, to=2-1]
	\arrow["{\tau^1}", from=2-3, to=2-4]
\end{tikzcd}\]
where $\pi$ is the projection $[[n-2],2]\to [e,1]\vee[[n-2],1]$.
However, according to the diagrams $(5)$ and $(3)$ of \ref{cons:the big construction}, there is a diagram:
% https://q.uiver.app/?q=WzAsMTAsWzEsMiwiZVxcc3RhcihbZSwxXVxcdmVlW1tuLTJdLDFdKSJdLFswLDEsIltbMl1cXGJvdGltZXMgW24tMl0sMV0iXSxbMSwzLCJbbisxXV4xIl0sWzAsMCwiW1sxXVxcb3RpbWVzIFtuLTJdLDFdIl0sWzEsMCwiW2VcXHN0YXIgW24tMl0sMV1cXHZlZVtbbi0yXSwxXSJdLFsyLDMsIlsyXV90Il0sWzEsMSwiW1tuLTJdLDJdIl0sWzIsMCwiW2VcXHN0YXJbbi0yXSwxXSJdLFsyLDEsImVcXHN0YXJbW24tMl0sMV0iXSxbMiwyLCJlXFxzdGFyW2UsMV0iXSxbMywxLCJbZF4wXFxvdGltZXMgW24tMl0sMV0iLDJdLFszLDQsIltbMV1cXG90aW1lc1tuLTJdLGReMV0iXSxbNSwyLCJkXjNcXGNpcmMuLi5cXGNpcmMgZF57bisxfSJdLFs0LDYsIlxcZGVsdGFfe1tuLTJdfSJdLFsxLDYsIiBcXGVwc2lsb25fe1tuLTJdfSIsMl0sWzYsMCwiZVxcc3RhciBcXHBpIl0sWzcsNCwiW2VcXHN0YXJbbi0yXSxkXjJdIiwyXSxbNyw4LCJcXGFscGhhX3tbbi0yXX0iXSxbOCw5XSxbOSwwXSxbOCw2XSxbMCwyLCJcXHRhdV8xXFxjaXJjIGVcXHN0YXJcXGJldGFfe1tuLTFdfSIsMl0sWzksNV1d
\[\begin{tikzcd}
	{[[1]\otimes [n-2],1]} & {[e\star [n-2],1]\vee[[n-2],1]} & {[e\star[n-2],1]} \\
	{[[2]\botimes [n-2],1]} & {[[n-2],2]} & {e\star[[n-2],1]} \\
	& {e\star([e,1]\vee[[n-2],1])} & {e\star[e,1]} \\
	& {[n+1]^1} & {[2]_t}
	\arrow["{[d^0\otimes [n-2],1]}"', from=1-1, to=2-1]
	\arrow["{[[1]\otimes[n-2],d^1]}", from=1-1, to=1-2]
	\arrow["{d^3\circ...\circ d^{n+1}}", from=4-3, to=4-2]
	\arrow["{\delta_{[n-2]}}", from=1-2, to=2-2]
	\arrow["{ \epsilon_{[n-2]}}"', from=2-1, to=2-2]
	\arrow["{e\star \pi}", from=2-2, to=3-2]
	\arrow["{[e\star[n-2],d^2]}"', from=1-3, to=1-2]
	\arrow["{\alpha_{[n-2]}}", from=1-3, to=2-3]
	\arrow[from=2-3, to=3-3]
	\arrow[from=3-3, to=3-2]
	\arrow[from=2-3, to=2-2]
	\arrow["{\tau_1\circ e\star\beta_{[n-1]}}"', from=3-2, to=4-2]
	\arrow[from=3-3, to=4-3]
\end{tikzcd}\]
This implies that $[[2]\botimes [n-2],1]\to [n+1]^{k}\to ([n+1]^{k})_{\mk} $ factors through $[[2]\botimes[n-2]\coprod_{d^0\otimes a}\tau^i_{n-1}([1]\otimes [n-2]),1]$. We can then apply lemma \ref{lem:abstract thinness 1}. 	
\end{proof}

\begin{lemma}
\label{lem:thinness extension last step2}
Let $0<k<n-1$ be two integers.
We denote by $\tau^k$ the projection $[n]\to [n]^k$. We then have $$(\tau^k\circ \iota_n\circ [d^{k-1},1], \tau^k\circ \iota_n\circ [d^{k+1},1])\geq_{n-1}\tau^k\circ \iota_n\circ [d^{k},1]$$ and $$\tau^{n-1}\circ \iota_n\circ [d^{n-2},1]\geq_{n-1}\tau^k\circ \iota_n\circ [d^{n-1},1].$$
\end{lemma}
\begin{proof}
By construction, for any $a$, the morphism $[[2]\star a,1]\to [2]\star[a,1]\to [2]_t\star[a,1]$ factors through $[[2]_t\star a,1]$. By induction, this implies that the composite morphism $[[n-1],1]\xrightarrow{\iota_n}[n]\to [n]^k$ factors through $[[n-1]^k,1]$ for any $k<n-1$. This implies the first assertion. 

For the second one, note that $[[1],e]\to [2]\to [2]_t$ factors through $[[1]_t,e]$. By induction, this implies that the composite morphism $[[n-1],1]\xrightarrow{\iota_n}[n]\to [n]^{n-1}$ factors through $[[n-1]^{n-2},1]$ which gives the second one.
\end{proof}



\begin{prop}
\label{prop:thinness extension}
For any $0\leq k\leq n$, the morphism $([n]^k)' \to ([n]^k)''$ is a weak equivalence.
\end{prop}
\begin{proof}
The case $k=0$ and $k=n$ are demonstrated in lemma \ref{lemma:thinnes extension case 0 and n}. For the case $0<k<n$, lemmas \ref{lem:thinness extension last step0}, \ref{lem:thinness extension last step1} and \ref{lem:thinness extension last step2} imply that if we denote by $\tau_k$ the projection $[n]\to [n]^k$, we have an inequality: $(\tau_k\circ d^{k-1}\circ \iota_{n-1},\tau_k\circ d^{k+1}\circ \iota_{n-1})\geq_{n-1}\tau_k\circ d^k\circ \iota_{n-1}$. Together with the proposition \ref{prop:meaning of geq case 2}, this implies that the following square is homotopy cartesian:
% https://q.uiver.app/?q=WzAsNCxbMSwwLCJbbl1eayJdLFswLDAsIltuLTFdXFxjdXBbbi0xXSJdLFswLDEsIltuLTFdX3RcXGN1cFtuLTFdX3QiXSxbMSwxLCIoW25dXmspJyciXSxbMSwwLCJkXntrKzF9XFxjdXAgZF57ay0xfSJdLFsxLDJdLFsyLDNdLFswLDNdXQ==
\[\begin{tikzcd}
	{[n-1]\cup[n-1]} & {[n]^k} \\
	{[n-1]_t\cup[n-1]_t} & {([n]^k)''}
	\arrow["{d^{k+1}\cup d^{k-1}}", from=1-1, to=1-2]
	\arrow[from=1-1, to=2-1]
	\arrow[from=2-1, to=2-2]
	\arrow[from=1-2, to=2-2]
\end{tikzcd}\]
The morphism $([n]^k)' \to ([n]^k)''$ is then a weak equivalence.
\end{proof}

\subsection{Saturation extensions}
\label{section:Saturation extensions}

Let $\Lambda[3]^{eq}\to [3]^{eq}$ be the entire inclusion generated by $Im(d^3)\cup Im(d^0)\subset [3]$. This inclusion fits in the following sequence:
% https://q.uiver.app/?q=WzAsMTAsWzAsMSwiXFxMYW1iZGFbM11ee2VxfSJdLFsxLDEsIlxcYnVsbGV0Il0sWzIsMSwiXFxidWxsZXQiXSxbMywxLCJbM11ee2VxfSJdLFswLDAsIlxcTGFtYmRhXjFbMl0iXSxbMSwwLCJbMl1fdCJdLFsxLDIsIlxcTGFtYmRhXjFbM10iXSxbMiwyLCJbM10iXSxbMiwwLCIoWzNdXjEpJyJdLFszLDAsIihbM11eMSknJyJdLFs0LDBdLFs1LDEsImReMiJdLFs0LDVdLFswLDFdLFsxLDQsIiIsMSx7InN0eWxlIjp7Im5hbWUiOiJjb3JuZXIifX1dLFs2LDFdLFs3LDJdLFs2LDddLFsxLDJdLFsyLDYsIiIsMSx7InN0eWxlIjp7Im5hbWUiOiJjb3JuZXIifX1dLFs4LDJdLFs5LDNdLFs4LDldLFsyLDNdLFszLDgsIiIsMSx7InN0eWxlIjp7Im5hbWUiOiJjb3JuZXIifX1dXQ==
\[\begin{tikzcd}
	{\Lambda^1[2]} & {[2]_t} & {([3]^1)'} & {([3]^1)''} \\
	{\Lambda[3]^{eq}} & \bullet & \bullet & {[3]^{eq}} \\
	& {\Lambda^1[3]} & {[3]}
	\arrow[from=1-1, to=2-1]
	\arrow["{d^2}", from=1-2, to=2-2]
	\arrow[from=1-1, to=1-2]
	\arrow[from=2-1, to=2-2]
	\arrow["\lrcorner"{anchor=center, pos=0.125, rotate=180}, draw=none, from=2-2, to=1-1]
	\arrow[from=3-2, to=2-2]
	\arrow[from=3-3, to=2-3]
	\arrow[from=3-2, to=3-3]
	\arrow[from=2-2, to=2-3]
	\arrow["\lrcorner"{anchor=center, pos=0.125, rotate=-90}, draw=none, from=2-3, to=3-2]
	\arrow[from=1-3, to=2-3]
	\arrow[from=1-4, to=2-4]
	\arrow[from=1-3, to=1-4]
	\arrow[from=2-3, to=2-4]
	\arrow["\lrcorner"{anchor=center, pos=0.125, rotate=180}, draw=none, from=2-4, to=1-3]
\end{tikzcd}\]
This inclusion is then a weak equivalence according to propositions \ref{prop:horn_inclusion} and \ref{prop:thinness extension}.
Now, note that we have a pushout:
% https://q.uiver.app/?q=WzAsNCxbMSwwLCJcXExhbWJkYVszXV57ZXF9Il0sWzAsMCwiWzJdX3RcXGFtYWxnIFsyXV90Il0sWzAsMSwiW2UsMl1cXGNvcHJvZCBbZSwyXSJdLFsxLDEsIltlLFszXV57ZXF9XSJdLFsxLDJdLFsxLDBdLFswLDNdLFsyLDNdLFszLDEsIiIsMSx7InN0eWxlIjp7Im5hbWUiOiJjb3JuZXIifX1dXQ==
\[\begin{tikzcd}
	{[2]_t\amalg [2]_t} & {\Lambda[3]^{eq}} \\
	{[e,2]\coprod [e,2]} & {[e,[3]^{eq}]}
	\arrow[from=1-1, to=2-1]
	\arrow[from=1-1, to=1-2]
	\arrow[from=1-2, to=2-2]
	\arrow[from=2-1, to=2-2]
	\arrow["\lrcorner"{anchor=center, pos=0.125, rotate=180}, draw=none, from=2-2, to=1-1]
\end{tikzcd}\]
As the left vertical morphism is a weak equivalence, so is the right one. 
Let $\Lambda[3]^{\sharp}\to [3]^{\sharp}$ be the entire inclusion generated by $Im(d^3)\cup Im(d^0)\subset [3]$.
Using the same reasoning, we show that this cofibration is acyclic and that there is a weak equivalence $\Lambda[3]^\sharp \to [e,[3]^\sharp]$. We then have a commutative square:
% https://q.uiver.app/?q=WzAsNixbMSwwLCJcXExhbWJkYVszXV57ZXF9Il0sWzAsMCwiW2UsWzNdXntlcX1dIl0sWzEsMSwiXFxMYW1iZGFbM11ee3tcXHNoYXJwfX0iXSxbMCwxLCJbZSxbM11ee1xcc2hhcnB9XSJdLFsyLDAsIlszXV57ZXF9Il0sWzIsMSwiWzNdXnt7XFxzaGFycH19Il0sWzAsMSwiXFxzaW0iLDJdLFswLDQsIlxcc2ltIl0sWzQsNV0sWzIsMywiXFxzaW0iXSxbMiw1LCJcXHNpbSIsMl0sWzEsMywiXFxzaW0iLDJdLFswLDJdXQ==
\[\begin{tikzcd}
	{[e,[3]^{eq}]} & {\Lambda[3]^{eq}} & {[3]^{eq}} \\
	{[e,[3]^{\sharp}]} & {\Lambda[3]^{{\sharp}}} & {[3]^{{\sharp}}}
	\arrow["\sim"', from=1-2, to=1-1]
	\arrow["\sim", from=1-2, to=1-3]
	\arrow[from=1-3, to=2-3]
	\arrow["\sim", from=2-2, to=2-1]
	\arrow["\sim"', from=2-2, to=2-3]
	\arrow["\sim"', from=1-1, to=2-1]
	\arrow[from=1-2, to=2-2]
\end{tikzcd}\]
where all arrows labelled by $\sim$ are weak equivalences. By two out of three, this implies that $[3]^{eq}\to [3]^\sharp$ is a weak equivalence. Combined with the lemma \ref{lemma:leibnizt joint is Quillen}, this implies the following proposition:

\begin{prop}
\label{prop:saturation extension}
For any $n\geq -1$, the morphism $[n]\star [3]^{eq}\to [n]\star [3]^{\sharp}$ is an acyclic cofibration.
\end{prop}




\begin{theorem}
\label{theo:Quillen adjunction}
The stratified cosimplicial object constructed in paragraph
\ref{para:definition of the cosimplicial object} induces a Quillen adjunction $\stratSset^\omega\to \stratSeg(A)$.
\end{theorem}
\begin{proof}
It is a direct consequence of theorem \ref{theo:model structure on complicial set} and propositions \ref{prop:horn_inclusion}, \ref{prop:thinness extension}, and \ref{prop:saturation extension}.
\end{proof}




\section{The case $A:=\stratSset^n$}
For $n\in \Nb\cup\{\omega\}$, we denote by $\stratSset^n$ the category of stratified simplicial set endowed with the model structure for $n$-complicial set given in theorem \ref{theo:model structure on complicial set}. As remarked in example \ref{example:stratsset is gray module}, these model categories are Gray modules. The functor $\stratSset\to\stratSeg(\stratSset^n)$ defined in \ref{para:definition of the cosimplicial object} is left Quillen according to theorem \ref{theo:Quillen adjunction}.
 It was noted in paragraph \ref{para:a cocartesian square for intelingent truncation} that for $k>0$, $[k]\to [k]_t$ fits in the following cocartesian square: 
 % https://q.uiver.app/?q=WzAsNCxbMSwxLCJba11fdCJdLFsxLDAsIltrXSJdLFswLDAsIltbay0xXSwxXSJdLFswLDEsIltbay0xXV90LDFdIl0sWzIsMSwiXFxpb3RhX2siXSxbMiwzXSxbMywwXSxbMSwwXV0=
\[\begin{tikzcd}
	{[[k-1],1]} & {[k]} \\
	{[[k-1]_t,1]} & {[k]_t}
	\arrow["{\iota_k}", from=1-1, to=1-2]
	\arrow[from=1-1, to=2-1]
	\arrow[from=2-1, to=2-2]
	\arrow[from=1-2, to=2-2]
\end{tikzcd}\]
The functor $\stratSset\to\stratSeg(\stratSset^n)$ then sends $[k]\to [k]_t$ to an acyclic cofibration for $k>n+1$, and then induces a left Quillen functor 
\begin{equation}
\label{eq:defi of in}
i^{n+1}:\stratSset^{n+1}\to\stratSeg(\stratSset^n)
\end{equation}

\subsection{Comparison with $\zocat$}
We denote by 
% https://q.uiver.app/#q=WzAsMixbMCwwLCJcXFI6XFxzdHJhdFNzZXReXFxvbWVnYSJdLFsxLDAsIlxcem9jYXQ6XFxOIl0sWzAsMSwiIiwwLHsib2Zmc2V0IjotMn1dLFsxLDAsIiIsMCx7Im9mZnNldCI6LTJ9XSxbMiwzLCIiLDAseyJsZXZlbCI6MSwic3R5bGUiOnsibmFtZSI6ImFkanVuY3Rpb24ifX1dXQ==
\[\begin{tikzcd}
	{\R:\stratSset^\omega} & {\zocat:\N}
	\arrow[""{name=0, anchor=center, inner sep=0}, shift left=2, from=1-1, to=1-2]
	\arrow[""{name=1, anchor=center, inner sep=0}, shift left=2, from=1-2, to=1-1]
	\arrow["\dashv"{anchor=center, rotate=-90}, draw=none, from=0, to=1]
\end{tikzcd}\]
the adjunction between stratified simplicial sets and $\zo$-categories described in section 
\ref{section:Street nerve}. For an $\zo$-category $C$ and an integer $n$, the $\zo$-category $[C,n]$ is defined as the colimit of the following diagram
% https://q.uiver.app/?q=WzAsNyxbMCwxLCJcXFNpZ21hIEMiXSxbMiwxLCJcXFNpZ21hIEMiXSxbNCwxLCIuLi4iXSxbNiwxLCJcXFNpZ21hIEMiXSxbMSwwLCJbMF0iXSxbMywwLCJbMF0iXSxbNSwwLCJbMF0iXSxbNCwwLCJpXzBeKyIsMl0sWzQsMSwiaV8wXi0iXSxbNiwyLCJpXzBeKyIsMl0sWzUsMiwiaV8wXi0iXSxbNSwxLCJpXzBeKyIsMl0sWzYsMywiaV8wXi0iXV0=
\[\begin{tikzcd}
	& {[0]} && {[0]} && {[0]} \\
	{\Sigma C} && {\Sigma C} && {...} && {\Sigma C}
	\arrow["{i_0^+}"', from=1-2, to=2-1]
	\arrow["{i_0^-}", from=1-2, to=2-3]
	\arrow["{i_0^+}"', from=1-6, to=2-5]
	\arrow["{i_0^-}", from=1-4, to=2-5]
	\arrow["{i_0^+}"', from=1-4, to=2-3]
	\arrow["{i_0^-}", from=1-6, to=2-7]
\end{tikzcd}\]
This induces an adjunction
% https://q.uiver.app/#q=WzAsMixbMCwwLCJcXFI6XFxzdHJhdFNlZyhcXHN0cmF0U3NldCkiXSxbMSwwLCJcXHpvY2F0OlxcTiJdLFswLDEsIiIsMCx7Im9mZnNldCI6LTJ9XSxbMSwwLCIiLDAseyJvZmZzZXQiOi0yfV0sWzIsMywiIiwwLHsibGV2ZWwiOjEsInN0eWxlIjp7Im5hbWUiOiJhZGp1bmN0aW9uIn19XV0=
\[\begin{tikzcd}
	{\R:\stratSeg(\stratSset)} & {\zocat:\N}
	\arrow[""{name=0, anchor=center, inner sep=0}, shift left=2, from=1-1, to=1-2]
	\arrow[""{name=1, anchor=center, inner sep=0}, shift left=2, from=1-2, to=1-1]
	\arrow["\dashv"{anchor=center, rotate=-90}, draw=none, from=0, to=1]
\end{tikzcd}\]
where the left adjoint sends $[K,n]$ to $[\R(K),n]$ and $[e,1]_t$ on $[0]$.
\sym{(r@$\R:\stratSeg(\stratSset)\to \zocat$}\sym{(n@$\N:\zocat\to \stratSeg(\stratSset)$}
\begin{lemma}
\label{lemma: nerve commute is suspension}
For any $\zo$-category $C$, the canonical morphism 
$$[\N C,1]\to \N[C,1]$$ is an isomorphism. 
\end{lemma}

\begin{proof}
Let $K$ be a stratified simplicial set, $n$ an integer. By construction, we have two cartesian squares
% https://q.uiver.app/?q=WzAsNCxbMSwwLCJcXEhvbV97XFxEZWx0YX0oW25dLFsxXSlcXHRpbWVzIFxcSG9tX3tcXHN0cmF0U3NldH0oSyxcXE4gQykiXSxbMCwwLCJcXGNvcHJvZFxcbGltaXRzX3tcXGVwc2lsb25cXGluXFx7MCwxXFx9fVxcSG9tX3tcXERlbHRhfShbbl0sXFx7XFxlcHNpbG9uXFx9KVxcdGltZXMgXFxIb21fe1xcc3RyYXRTc2V0fShLLFxcTiBDKSJdLFswLDEsIlxcY29wcm9kXFxsaW1pdHNfe1xcZXBzaWxvblxcaW5cXHswLDFcXH19XFxIb21fe1xcRGVsdGF9KFtuXSxcXHtcXGVwc2lsb25cXH0pIl0sWzEsMSwiXFxIb21fe1xcc3RyYXRTZWcoXFxzdHJhdFNzZXQpfShbSyxuXSxbXFxOIEMsMV0pIl0sWzEsMl0sWzEsMF0sWzIsM10sWzAsM11d
\[\begin{tikzcd}
	{\coprod\limits_{\epsilon\in\{0,1\}}\Hom_{\Delta}([n],\{\epsilon\})\times \Hom_{\stratSset}(K,\N C)} & {\Hom_{\Delta}([n],[1])\times \Hom_{\stratSset}(K,\N C)} \\
	{\coprod\limits_{\epsilon\in\{0,1\}}\Hom_{\Delta}([n],\{\epsilon\})} & {\Hom_{\stratSeg(\stratSset)}([K,n],[\N C,1])}
	\arrow[from=1-1, to=2-1]
	\arrow[from=1-1, to=1-2]
	\arrow[from=2-1, to=2-2]
	\arrow[from=1-2, to=2-2]
\end{tikzcd}\]
% https://q.uiver.app/#q=WzAsNCxbMSwwLCJcXEhvbV97XFxEZWx0YX0oW25dLFsxXSlcXHRpbWVzIFxcSG9tX3tcXHpvY2F0fShcXFIoSyksIEMpIl0sWzAsMCwiXFxjb3Byb2RcXGxpbWl0c197XFxlcHNpbG9uXFxpblxcezAsMVxcfX1cXEhvbV97XFxEZWx0YX0oW25dLFxce1xcZXBzaWxvblxcfSlcXHRpbWVzICBcXEhvbV97XFx6b2NhdH0oXFxSKEspLCBDKSJdLFswLDEsIlxcY29wcm9kXFxsaW1pdHNfe1xcZXBzaWxvblxcaW5cXHswLDFcXH19XFxIb21fe1xcRGVsdGF9KFtuXSxcXHtcXGVwc2lsb25cXH0pIl0sWzEsMSwiXFxIb21fe1xcem9jYXR9KFxcUihbSyxuXSksW0MsMV0pIl0sWzEsMl0sWzEsMF0sWzIsM10sWzAsM11d
\[\begin{tikzcd}
	{\coprod\limits_{\epsilon\in\{0,1\}}\Hom_{\Delta}([n],\{\epsilon\})\times \Hom_{\zocat}(\R(K), C)} & {\Hom_{\Delta}([n],[1])\times \Hom_{\zocat}(\R(K), C)} \\
	{\coprod\limits_{\epsilon\in\{0,1\}}\Hom_{\Delta}([n],\{\epsilon\})} & {\Hom_{\zocat}(\R([K,n]),[C,1])}
	\arrow[from=1-1, to=2-1]
	\arrow[from=1-1, to=1-2]
	\arrow[from=2-1, to=2-2]
	\arrow[from=1-2, to=2-2]
\end{tikzcd}\]
which directly concludes the proof.
\end{proof} 





\begin{lemma}
\label{lemma:compairaon beetwen join and the formula of chap 2}
Let $C$ be an $\zo$-category and $n$ an integer. There is a canonical commutative square in $\zocat$: 
% https://q.uiver.app/?q=WzAsNCxbMSwwLCJcXGNvbGltX3tcXERlbHRhXjJfey9bbl19fVtbbl8wXVxcb3RpbWVzIEMsMV1cXHZlZVtDLG5fMV0iXSxbMSwxLCIxXFxzdGFyIFtDLG5dIl0sWzAsMCwiXFxjb3Byb2RcXGxpbWl0c197a1xcbGVxIG59XFxjb2xpbV97XFxEZWx0YV4yX3svXFx7a1xcfX19W1tuXzBdXFxvdGltZXMgQywxXVxcdmVlW0Msbl8xXSJdLFswLDEsIlxcY29wcm9kXFxsaW1pdHNfe2tcXGxlcSBufVxcY29saW1fe1xcRGVsdGFeMl97L1xce2tcXH19fVtbbl8wXSwxXVxcdmVlW25fMV0iXSxbMCwxXSxbMiwzXSxbMywxXSxbMiwwXV0=
\[\begin{tikzcd}
	{\coprod\limits_{k\leq n}\colim_{\Delta^2_{/\{k\}}}[[n_0]\otimes C,1]\vee[C,n_1]} & {\colim_{\Delta^2_{/[n]}}[[n_0]\otimes C,1]\vee[C,n_1]} \\
	{\coprod\limits_{k\leq n}\colim_{\Delta^2_{/\{k\}}}[[n_0],1]\vee[n_1]} & {1\star [C,n]}
	\arrow[from=1-2, to=2-2]
	\arrow[from=1-1, to=2-1]
	\arrow[from=2-1, to=2-2]
	\arrow[from=1-1, to=1-2]
\end{tikzcd}\]
natural in $C:\zocat$ and $[n]:\Delta$.
\end{lemma}
\begin{proof}
In this proof, we use the Steiner theory recalled in section \ref{section:Steiner thery}.
It is sufficient to show the assertion when $C$ is a globular form, and then \textit{a fortiori}, an $\zo$-category with an atomic and loop free basis. Using the equivalence between $\zocatB$ and $\CDAB$ given in \ref{theorem:steiner} and the equivalences
$$(K\otimes L)^{op} \sim L^{op}\otimes K^{op}~~~(K\otimes L)^{co} \sim L^{co}\otimes K^{co} ~~~ (1\star K)^{op}\sim K^{op}\star 1$$
provided by propositions A.20 and 6.10 of \cite{Ara_Maltsiniotis_joint_et_tranche}, it is sufficient to construct for every augmented direct complex $K$ a natural commutative square: 
% https://q.uiver.app/?q=WzAsNCxbMSwxLCJbSyxuXVxcc3RhciAxIl0sWzEsMCwiXFxjb2xpbV97W25fMV1cXHN0YXIgW25fMF1cXHRvIFtuXX1bSyxuXzFdXFx2ZWVbS1xcb3RpbWVzIFxcbGFtYmRhW25fMF0sMV0iXSxbMCwwLCJcXGNvcHJvZF97a1xcbGVxIG59XFxjb2xpbV97W25fMV1cXHN0YXIgW25fMF1cXHRvIFxce2tcXH19W0ssbl8xXVxcdmVlW0tcXG90aW1lc1xcbGFtYmRhW25fMF0sMV0iXSxbMCwxLCJcXGNvcHJvZF97a1xcbGVxIG59XFxjb2xpbV97W25fMV1cXHN0YXIgW25fMF1cXHRvIFxce2tcXH19XFxsYW1iZGFbbl8xXVxcdmVlW1xcbGFtYmRhW25fMF0sMV0iXSxbMywwXSxbMiwzXSxbMSwwXSxbMiwxXV0=
\[\begin{tikzcd}
	{\coprod_{k\leq n}\colim_{[n_1]\star [n_0]\to \{k\}}[K,n_1]\vee[K\otimes\lambda[n_0],1]} & {\colim_{[n_1]\star [n_0]\to [n]}[K,n_1]\vee[K\otimes \lambda[n_0],1]} \\
	{\coprod_{k\leq n}\colim_{[n_1]\star [n_0]\to \{k\}}\lambda[n_1]\vee[\lambda[n_0],1]} & {[K,n]\star 1}
	\arrow[from=2-1, to=2-2]
	\arrow[from=1-1, to=2-1]
	\arrow[from=1-2, to=2-2]
	\arrow[from=1-1, to=1-2]
\end{tikzcd}\]


For an element $f:[n_0]\star[n_1]\to [n]$ of $\Delta^2_{/[n]}$, we considere the morphism 
$\phi_f:[K,n_1]\vee[K\otimes \lambda[n_0],1]\to [K,n]\star 1$ as the unique morphism fulfilling 
$$\phi_f(	[x,v_{i,i+1}]):= [x,v_{f_0(i),f_0(i)+1}]\star \emptyset+...+ [x,v_{f_0(i)-1,f_0(i+1)}]\star \emptyset$$
$$\phi_f(	[x\otimes v_i,1]):= 0$$
$$\phi_f(	[x\otimes v_{i,i+1},1]):= [x,v_{f_1(i),f_1(i)+1}]\star 1+...+ [x,v_{f_1(i)-1,f_1(i+1)}]\star 1$$
for $x$ an element of $K$ and where we denote by $f_0$ and $f_1$ the induced morphisms $[n_0]\to [n_0]\star [n_1]\to [n]$ and $[n_1]\to [n_0]\star [n_1]\to [n]$.

Peforming this for any such $f:[n_0]\star[n_1]\to [n]$ of $\Delta^2_{/[n]}$, this induces a morphism 
$$\psi:\colim_{\Delta^2_{/[n]}}[[n_0]\otimes a,1]\vee[a,n_1]\to 1\star [a,n]$$
whose restriction to $\coprod\limits_{k\leq n}\colim_{\Delta^2_{/\{k\}}}[[n_0]\otimes a,1]\vee[a,n_1]$ factors through $\coprod\limits_{k\leq n}\colim_{\Delta^2_{/\{k\}}}[[n_0],1]\vee[1,n_1]$ and this concludes the proof.
\end{proof}






\begin{lemma}
\label{lemma:joi commutes with realization}
There is an invertible natural transformation $\R(e\star\uvar)\to 1\star \R(\uvar)$ that firs in a commutative square
% https://q.uiver.app/#q=WzAsNCxbMSwxLCIgMVxcc3RhciBcXFIoXFx1dmFyKSJdLFsxLDAsIlxcUihlXFxzdGFyXFx1dmFyKSJdLFswLDEsIlxcZW1wdHlzZXRcXHN0YXIgXFxSKFxcdXZhcikiXSxbMCwwLCJcXFIoXFxlbXB0eXNldFxcc3RhclxcdXZhcikiXSxbMiwwXSxbMywxXSxbMSwwXSxbMywyLCJpZCIsMl1d
\[\begin{tikzcd}
	{\R(\emptyset\star\uvar)} & {\R(e\star\uvar)} \\
	{\emptyset\star \R(\uvar)} & { 1\star \R(\uvar)}
	\arrow[from=2-1, to=2-2]
	\arrow[from=1-1, to=1-2]
	\arrow[from=1-2, to=2-2]
	\arrow["id"', from=1-1, to=2-1]
\end{tikzcd}\]
\end{lemma}
\begin{proof}
The lemma \ref{lemma:compairaon beetwen join and the formula of chap 2} provides such natural transformation. As $\R$ sends weak equivalences to isomorphisms, it is sufficient to show that $\R(e\star [K,1])\to 1\star [\R(K),1]$ is an equivalence, which directly follows from the explicit description of these two objects provided by proposition \ref{prop:explicit expression of e star a,1} and by the example \ref{exe:explicit Gray cone 1}.
\end{proof}

\begin{prop}
\label{prop:first triangle}
The following triangle commutes:
% https://q.uiver.app/#q=WzAsMyxbMCwxLCJcXHN0cmF0U3NldF57bisxfSJdLFsxLDAsIlxcc3RyYXRTZWcoXFxzdHJhdFNzZXRebikiXSxbMSwxLCJcXHpvY2F0Il0sWzEsMiwiXFxSIl0sWzAsMiwiXFxSIiwyXSxbMCwxLCJpXntuKzF9Il1d
\[\begin{tikzcd}
	& {\stratSeg(\stratSset^n)} \\
	{\stratSset^{n+1}} & \zocat
	\arrow["\R", from=1-2, to=2-2]
	\arrow["\R"', from=2-1, to=2-2]
	\arrow["{i^{n+1}}", from=2-1, to=1-2]
\end{tikzcd}\]
For any integer $k\leq n+1$, the induced morphism $i^{n+1}(\N \Db_k)\to \N(\Db_k)$ is a weak equivalence.
\end{prop}
\begin{proof}
It is sufficient to show the result for $n:=\omega$.
Let $C$ be the subcategory of $\stratSset$ such that this triangle commutes. This subcategory is closed under colimits, and we then want to show that it contains $\Delta$. As the orientals have no non trivial automorphism, lemma \ref{lemma:joi commutes with realization} implies that $C$ contains the subcategory of $t\Delta$ whose morphisms are monomorphisms and as $i^\omega$ commutes with the intelligent truncation, $C$ also includes $[n]\to[n]_t$. We still have to show that $C$ contains the degeneracy, and for this, we proceed by induction. 

We then suppose that for any $k<n$, any degeneracy $[k+1]\to [k]$ is in $C$. As $C$ contains monomorphisms of $\Delta$, it contains any morphism $[k]\to [n]$ with $k\leq n$. 
 Let $j:[n+1]\to [n]$ be a degeneracy. We have a \textit{a priori} non commutative diagram:
% https://q.uiver.app/#q=WzAsNixbMSwxLCJ7XFxSfShpXlxcb21lZ2FbbisxXSkiXSxbMSwyLCJ7XFxSfShpXlxcb21lZ2Fbbl0pIl0sWzAsMSwie1xcUn0oW24rMV0pIl0sWzAsMiwie1xcUn0oW25dKSJdLFsxLDAsIntcXFJ9KGleXFxvbWVnYVxccGFydGlhbCBbbisxXSkiXSxbMCwwLCJ7XFxSfShcXHBhcnRpYWxbbisxXSkiXSxbMCwyLCIiLDIseyJsZXZlbCI6Miwic3R5bGUiOnsiaGVhZCI6eyJuYW1lIjoibm9uZSJ9fX1dLFs0LDBdLFs1LDJdLFsyLDNdLFsxLDMsIiIsMSx7ImxldmVsIjoyLCJzdHlsZSI6eyJoZWFkIjp7Im5hbWUiOiJub25lIn19fV0sWzQsNSwiIiwwLHsibGV2ZWwiOjIsInN0eWxlIjp7ImhlYWQiOnsibmFtZSI6Im5vbmUifX19XSxbMCwxXV0=
\[\begin{tikzcd}
	{{\R}(\partial[n+1])} & {{\R}(i^\omega\partial [n+1])} \\
	{{\R}([n+1])} & {{\R}(i^\omega[n+1])} \\
	{{\R}([n])} & {{\R}(i^\omega[n])}
	\arrow[Rightarrow, no head, from=2-2, to=2-1]
	\arrow[from=1-2, to=2-2]
	\arrow[from=1-1, to=2-1]
	\arrow[from=2-1, to=3-1]
	\arrow[Rightarrow, no head, from=3-2, to=3-1]
	\arrow[Rightarrow, no head, from=1-2, to=1-1]
	\arrow[from=2-2, to=3-2]
\end{tikzcd}\]
As $C$ is closed under colimits, the induction hypothesis implies that the outer and the upper square commute. For the lower diagram to commute, we have to check that the top cell of $\R([n+1])$ is sent on the same element on ${\R}(i^\omega[n])$. This is the case because the two paths send it to an unity. 


We now turn our attention to the second assertion. We define the functor $\Sigma^{\circ}:\stratSset\to \stratSset$ that sends a stratified simplicial set $K$ onto the following pushout:
% https://q.uiver.app/#q=WzAsNCxbMSwwLCIxXFxzdGFyIEsiXSxbMSwxLCJcXFNpZ21hXntcXGNpcmN9SyJdLFswLDEsIjEiXSxbMCwwLCJLIl0sWzMsMl0sWzIsMV0sWzMsMF0sWzAsMV0sWzEsMywiIiwxLHsic3R5bGUiOnsibmFtZSI6ImNvcm5lciJ9fV1d
\[\begin{tikzcd}
	K & {1\star K} \\
	1 & {\Sigma^{\circ}K}
	\arrow[from=1-1, to=2-1]
	\arrow[from=2-1, to=2-2]
	\arrow[from=1-1, to=1-2]
	\arrow[from=1-2, to=2-2]
	\arrow["\lrcorner"{anchor=center, pos=0.125, rotate=180}, draw=none, from=2-2, to=1-1]
\end{tikzcd}\]
Remark that we have a canonical equivalence
$$(\Sigma^{\circ}X)^{op} \sim \Sigma^\star X^{op}$$
where $\Sigma^\star$ is the functor defined in paragraph \ref{para:sigma star}.
As the nerve commutes with the op-dualities, and as globes are invariant under it, a repeated application of \cite[theorem 3.22]{Ozornova_a_quillen_adjunction_between_globular_and_complicial} imply that the following canonical morphism between stratified simplicial sets
$$(\Sigma^\circ)^k[0]\to \N(\Db_k)$$
is an acylic cofibration.
Furthermore, proposition \ref{labe:Link between the Gray cylinder and cosuspension} provides a weak equivalence
$$i^{n+1}(\Sigma^{\circ} K)\to \Sigma^{\circ} K.$$
A direct induction then induces a weak equivalence 
$$i^{n+1}((\Sigma^\circ)^k[0])\to (\Sigma^\circ)^k[0]$$


Otherwise, remark that by construction, $\Sigma^{\circ}[K,1]:=[[0]\diamond K\coprod_{K}[0],1]$. The weak equivalence $[0]\diamond K\to [0]\star K$ provided by proposition \ref{prop:equivalence between diamond and join product} induces a weak equivalence
$$\Sigma^{\circ}[K,1]\to [\Sigma^{\circ}K,1].$$
As $\Sigma^{\circ}[0] = [[0],1]$, a direct induction induces a weak equivalence 
$$(\Sigma^\circ)^k[0]\to [(\Sigma^\circ)^{k-1}([0]),1].$$




All put together, and using lemma \ref{lemma: nerve commute is suspension} and the fact that $\R$ sends weak equivalences to isomorphisms, we get a commutative diagram
% https://q.uiver.app/#q=WzAsNixbMCwwLCJpXntuKzF9KChcXFNpZ21hXlxcY2lyYylea1swXSkiXSxbMywwLCJpXntuKzF9KFxcTiBcXERiX3trfSApIl0sWzEsMSwiWyhcXFNpZ21hXlxcY2lyYylee2stMX1bMF0sMV0iXSxbMywxLCJcXE4gXFxEYl9rICJdLFsyLDEsIltcXE5cXERiX3trLTF9LDFdIl0sWzAsMSwiKFxcU2lnbWFeXFxjaXJjKV5rWzBdIl0sWzAsMV0sWzEsM10sWzIsNCwiXFxzaW0iLDJdLFs0LDMsImlkIiwyXSxbMCw1LCJcXHNpbSIsMl0sWzUsMiwiXFxzaW0iLDJdXQ==
\[\begin{tikzcd}
	{i^{n+1}((\Sigma^\circ)^k[0])} &&& {i^{n+1}(\N \Db_{k} )} \\
	{(\Sigma^\circ)^k[0]} & {[(\Sigma^\circ)^{k-1}[0],1]} & {[\N\Db_{k-1},1]} & {\N \Db_k }
	\arrow[from=1-1, to=1-4]
	\arrow[from=1-4, to=2-4]
	\arrow["\sim"', from=2-2, to=2-3]
	\arrow["id"', from=2-3, to=2-4]
	\arrow["\sim"', from=1-1, to=2-1]
	\arrow["\sim"', from=2-1, to=2-2]
\end{tikzcd}\]
where all arrows labelled by $\sim$ are weak equivalences. By two out of three, this concludes the proof.
\end{proof}



\subsection{The other adjunction}
We define the colimit preserving functor 
\begin{equation}
\label{eq:defi of jn}
j:\stratSeg(\stratSset)\to \stratSset
\end{equation}
 sending $[K,n]$ to the pushout:
% https://q.uiver.app/?q=WzAsNCxbMSwwLCJLXFxib3h0aW1lc1tuXSJdLFswLDAsIlxcY3VwX3tpXFxsZXEgbn1LXFxib3h0aW1lc1xce2lcXH0iXSxbMCwxLCJcXGN1cF97aVxcbGVxIG59WzBdIl0sWzEsMSwiaihbSyxuXSkiXSxbMSwyXSxbMiwzXSxbMSwwXSxbMCwzXSxbMywxLCIiLDEseyJzdHlsZSI6eyJuYW1lIjoiY29ybmVyIn19XV0=
\[\begin{tikzcd}
	{\cup_{i\leq n}K\boxtimes\{i\}} & {K\boxtimes[n]} \\
	{\cup_{i\leq n}[0]} & {j([K,n])}
	\arrow[from=1-1, to=2-1]
	\arrow[from=2-1, to=2-2]
	\arrow[from=1-1, to=1-2]
	\arrow[from=1-2, to=2-2]
	\arrow["\lrcorner"{anchor=center, pos=0.125, rotate=180}, draw=none, from=2-2, to=1-1]
\end{tikzcd}\]
 and $[[0],1]_t$ to $[1]_t$. As $\uvar\boxtimes\uvar$ is a left Quillen bifunctor, and as $j([[0],1]_t\to [0])=[1]_t\to [0]$ and $j([[0],E^{\cong}]\to [[0],(E^{\cong})'])= E^{\cong}\to (E^{\cong})'$ are weak equivalences, 
the proposition \ref{prop:model structure on stratified Segal category} implies that the functor 
$$j^{\omega}:\stratSeg(\stratSset^{\omega})\to \stratSset^{\omega}$$ is a left Quillen functor. By definition of the Gray pre-tensor given in \cite[Definition 128]{Verity_weak_complicial_sets_I}, we remark that $j([[k],n]\to [[k]_t,n])$ is a pushout of a disjoint union of $[k+1]\to [k+1]_t$. This implies that for any $n\in \mathbb{N}$, 
$$j^{n+1}:\stratSeg(\stratSset^{n})\to \stratSset^{n+1}$$ 
 is a left Quillen functor.
\begin{prop}
\label{prop:second triangle}
The following triangle commutes:
% https://q.uiver.app/#q=WzAsMyxbMSwwLCJcXHN0cmF0U3NldF57bisxfSJdLFswLDEsIlxcc3RyYXRTZWcoXFxzdHJhdFNzZXRebikiXSxbMSwxLCJcXHpvY2F0Il0sWzEsMiwiXFxSIiwyXSxbMCwyLCJcXFIiXSxbMSwwLCJqXntuKzF9Il1d
\[\begin{tikzcd}
	& {\stratSset^{n+1}} \\
	{\stratSeg(\stratSset^n)} & \zocat
	\arrow["\R"', from=2-1, to=2-2]
	\arrow["\R", from=1-2, to=2-2]
	\arrow["{j^{n+1}}", from=2-1, to=1-2]
\end{tikzcd}\]
For any integer $k\leq n+1$, the induced morphism $j^{n+1}(\N \Db_k)\to \N(\Db_k)$ is a weak equivalence.
\end{prop}
\begin{proof}
The first assertion is a direct consequence of the definition of $\R:\stratSeg(\stratSset^n)\to \zocat$ and the corrolary \ref{cor:crushing of Gray tensor is identitye strict case}.

For the second one, remark that the case $k=0$ is trivial, and for $k>0$, lemma \ref{lemma: nerve commute is suspension}, theorem \ref{theo:strict susension} and the definition of $j^{n+1}$ induce weak equivalences 
$$j^{n+1}(\N \Db_k)=j^{n+1}([\N\Db_{k-1},1])= \Sigma \N \Db_{k-1}\to \N [\Db_{k-1},1] = \N \Db_k$$
\end{proof}






\subsection{Complicial sets as a model of $\io$-categories}


\begin{prop}
\label{prop:first equivalence}
For any $n\in \Nb\cup\{\omega\}$, the composite 
$$j^{n+1}\circ i^{n+1}:\stratSset^{n+1}\to \stratSset^{n+1}$$
is a Quillen equivalence.
\end{prop}
\begin{proof}
Using theorem \ref{theo:strict susension}, and propositions \ref{prop:first triangle} and \ref{prop:second triangle},
we have a zigzag of weak equivalences
$$j^{\omega}\circ i^{\omega}(\Db_n)\to j^{\omega}\circ i^{\omega}(\N(\Db_n))\to \N(\Db_n)\leftarrow \Db_n$$
natural in $n$.
The corollary \ref{cor:criterion_to_be_linked_to_identity_case stratified} then provides a zigzag of weakly invertible natural transformations
$$j^{\omega}\circ i^{\omega}\leftrightsquigarrow id_{\stratSset^{\omega}}.$$
This also induces for any integer $n$ a zigzag of weakly invertible natural transformations
$$j^{n+1}\circ i^{n+1}\leftrightsquigarrow id_{\stratSset^{n+1}}.$$
\end{proof}

\begin{theorem}
\label{theo:letheo}
For $n<\omega$, the model category $\stratSset^{n}$ is a model of $(\infty,n)$-categories.
\end{theorem}
\begin{proof}
To demonstrate the theorem, we will proceed by induction. The initialization is exactly the theorem 2.14 of \cite{Bergner_explicit_comparaison_bt_theta_2_space_and_2_complicial_set}. Suppose now the result is true at the stage $n$. We can apply \cite[example 15.8]{Barwick_on_the_unicity_of_the_theory_of_higher_categories} which implies that the $(\infty,1)$-category represented by $\Seg(\stratSset^n)$ is a model of $(\infty,n+1)$-categories, and according to \ref{prop:model structure on stratified Segal category}, so is $\stratSeg(\stratSset^n)$. Eventually, the proposition \ref{prop:first triangle} and \ref{prop:second triangle} induce a functor
$$i^{n+1}\circ j^{n+1}:\stratSeg(\stratSset^n)\to \stratSeg(\stratSset^n)$$
that preserves globes up to homotopy.
Proposition 15.10 of \cite{Barwick_on_the_unicity_of_the_theory_of_higher_categories} states that $i^{n+1}\circ j^{n+1}$ is a Quillen equivalence, and proposition \ref{prop:first equivalence} implies that $j^{n+1}\circ i^{n+1}$ is a Quillen equivalence. The functor $i^{n+1}$ is then a Quillen equivalence, and $\stratSset^{n+1}$ is a model of $(\infty,n+1)$-categories.
\end{proof}

\p For an integer $n$, we consider the model structure on $\ssetPsh{\Theta_n}$ (resp. $\ssetPsh{\Theta}$) obtained as the left Bousfield localization of the projective model structure along the set of map $\W_n$ (resp. $\W$ ) defined in paragraph \ref{para:definition of W}.
For any $n<\omega$, the inclusion $\Theta_n\to \Theta$ induces a Quillen adjunction
% https://q.uiver.app/?q=WzAsMixbMCwwLCJcXGlvdGFebjpcXHNzZXRQc2h7XFxUaGV0YV9ufSJdLFsxLDAsIlxcc3NldFBzaHtcXFRoZXRhfTpcXHRhdV9uIl0sWzAsMSwiIiwwLHsib2Zmc2V0IjotMn1dLFsxLDAsIiIsMCx7Im9mZnNldCI6LTJ9XSxbMiwzLCIiLDAseyJsZXZlbCI6MSwic3R5bGUiOnsibmFtZSI6ImFkanVuY3Rpb24ifX1dXQ==
\begin{equation}
\label{eq:adjunction beetwen theta n and theta}
\begin{tikzcd}
	{\iota^n:\ssetPsh{\Theta_n}} & {\ssetPsh{\Theta}:\tau_n}
	\arrow[""{name=0, anchor=center, inner sep=0}, shift left=2, from=1-1, to=1-2]
	\arrow[""{name=1, anchor=center, inner sep=0}, shift left=2, from=1-2, to=1-1]
	\arrow["\dashv"{anchor=center, rotate=-90}, draw=none, from=0, to=1]
\end{tikzcd}
\end{equation}


 
 
\p Let $n\in\Nb\cup\{\omega\}$.
We consider the functor
$$\Theta_n\times \Delta\to \stratSset$$
sending a pair $(a,[n])$ onto $\N(a)\times \tau^i_0([n])$. By left Kan extension, this induces an adjunction
% https://q.uiver.app/#q=WzAsMixbMCwwLCJMX246XFxzc2V0UHNoe1xcVGhldGFfbn0iXSxbMSwwLCJcXHN0cmF0U3NldDpOX3tMX259Il0sWzAsMSwiIiwwLHsib2Zmc2V0IjotMn1dLFsxLDAsIiIsMCx7Im9mZnNldCI6LTJ9XSxbMiwzLCIiLDAseyJsZXZlbCI6MSwic3R5bGUiOnsibmFtZSI6ImFkanVuY3Rpb24ifX1dXQ==
\begin{equation}
\label{eq:adjunction betwen theta and complicial}
\begin{tikzcd}
	{L_n:\ssetPsh{\Theta_n}} & {\stratSset:N_{L_n}}
	\arrow[""{name=0, anchor=center, inner sep=0}, shift left=2, from=1-1, to=1-2]
	\arrow[""{name=1, anchor=center, inner sep=0}, shift left=2, from=1-2, to=1-1]
	\arrow["\dashv"{anchor=center, rotate=-90}, draw=none, from=0, to=1]
\end{tikzcd}
\end{equation}





\begin{theorem}[Ozornova-Rovelli]
\label{theo:fondamental adj}
The adjunction 
% https://q.uiver.app/#q=WzAsMixbMCwwLCJMX246XFxzc2V0UHNoe1xcVGhldGFfbn0iXSxbMSwwLCJcXHN0cmF0U3NldF5uOk5fe0xfbn0iXSxbMCwxLCIiLDAseyJvZmZzZXQiOi0yfV0sWzEsMCwiIiwwLHsib2Zmc2V0IjotMn1dLFsyLDMsIiIsMCx7ImxldmVsIjoxLCJzdHlsZSI6eyJuYW1lIjoiYWRqdW5jdGlvbiJ9fV1d
\[\begin{tikzcd}
	{L_n:\ssetPsh{\Theta_n}} & {\stratSset^n:N_{L_n}}
	\arrow[""{name=0, anchor=center, inner sep=0}, shift left=2, from=1-1, to=1-2]
	\arrow[""{name=1, anchor=center, inner sep=0}, shift left=2, from=1-2, to=1-1]
	\arrow["\dashv"{anchor=center, rotate=-90}, draw=none, from=0, to=1]
\end{tikzcd}\]
is a Quillen adjunction.
\end{theorem}
\begin{proof}
This is \cite[theorem 4.16]{Ozornova_a_quillen_adjunction_between_globular_and_complicial}.
\end{proof}

\begin{remark}
The two authors demonstrate this result when $\stratSset$ is endowed with the model structure for $n$-complicial sets with $n<\omega$. However, their argument generalizes directly to the case $n=\omega$.
\end{remark}

A direct induction using \cite[theorem 3.22]{Ozornova_a_quillen_adjunction_between_globular_and_complicial} implies that the left adjoint preserves globes.


\begin{prop}
\label{prop:intermedeire}
For any $n\in \Nb$, the adjunction given in theorem \ref{theo:fondamental adj}
is a Quillen equivalence. 
\end{prop}
\begin{proof}
This is an adjunction between two models of $(\infty,n)$-categories. As the left adjoint preserves globes up to homotopy, the result follows from \cite[proposition 15.10]{Barwick_on_the_unicity_of_the_theory_of_higher_categories}.
\end{proof}
 
 \p If $C$ is a model category, we denote by $C^{\iun}$ the corresponding $\iun$-category. 


\begin{lemma}
\label{lemma:iota homotpically fully faitfhfull}
For any integer $n$, the $\iun$-functor 
$$\iota^n:(\ssetPsh{\Theta_n})^{\iun}\to (\ssetPsh{\Theta})^{\iun}$$
is fully faithful.
\end{lemma}
\begin{proof}
This is proposition \ref{ref:infini n a full sub cat}.
\end{proof}

\begin{lemma}
\label{lemma:the composite is ff}
For any integer $n$, the $\iun$-functor $$\tau^i_n:(\stratSset^n)^{\iun}\to (\stratSset^\omega)^{\iun}$$
is fully faithful.
\end{lemma}
\begin{proof}
This is a direct consequence of the fact that  $\stratSset^n$ is the left Bousfield localization of $\stratSset^\omega$ along morphisms $[m]\to [m]_t$ for $m>n$.
\end{proof}


\begin{lemma}
\label{lemma:L ff}
The $\iun$-functor $L_\omega: (\ssetPsh{\Theta})^{\iun}\to (\stratSset^\omega)^{\iun}$ is fully faithful.
\end{lemma}
\begin{proof}
We have to show that for any pair of $\Theta$-spaces $X$ and $Y$, the induced morphism of $\infty$-groupoids
$$\Hom_{(\ssetPsh{\Theta})^{\iun}}(X,Y)\to \Hom_{(\stratSset^\omega)^{\iun}}(L_\omega(X),L_\omega(Y))$$
is an equivalence. 
As every  $\Theta$-space is a $\iun$-colimit of globular sums, which are themself $\iun$-colimits of globes, we can suppose that $X$ is of shape $\Db_n$. In this case $\Db_n$ is $\omega$-small. As $L(\Db_n)$ has a finite presentation, given by the $n$-times interated suspension of $[0]$, it is also $\omega$-small.

Eventually, proposition \ref{prop:infini omega a limit of infini n} implies that every $\Theta$-spaces is a directed colimit of objects that are in the image of  $\iota_n$ for an integer $n$. We can then restrict ourselves to the case where $Y$ is in the image of $\iota_n$. As we have an equivalences $L_{\omega}\circ \iota_n\sim \tau^i_n  \circ L_n$, the results follows from
proposition \ref{prop:intermedeire}, and lemmas \ref{lemma:iota homotpically fully faitfhfull} and \ref{lemma:the composite is ff}.
\end{proof}


\begin{theorem}
\label{theo:lecorozo}
For any $n\in \Nb\cup\{\omega\}$, the adjunction
% https://q.uiver.app/#q=WzAsMixbMCwwLCJMX246XFxzc2V0UHNoe1xcVGhldGFfbn0iXSxbMSwwLCJcXHN0cmF0U3NldF5cXG9tZWdhOk5fe0xfbn0iXSxbMCwxLCIiLDAseyJvZmZzZXQiOi0yfV0sWzEsMCwiIiwwLHsib2Zmc2V0IjotMn1dLFsyLDMsIiIsMCx7ImxldmVsIjoxLCJzdHlsZSI6eyJuYW1lIjoiYWRqdW5jdGlvbiJ9fV1d
\[\begin{tikzcd}
	{L_n:\ssetPsh{\Theta_n}} & {\stratSset^\omega:N_{L_n}}
	\arrow[""{name=0, anchor=center, inner sep=0}, shift left=2, from=1-1, to=1-2]
	\arrow[""{name=1, anchor=center, inner sep=0}, shift left=2, from=1-2, to=1-1]
	\arrow["\dashv"{anchor=center, rotate=-90}, draw=none, from=0, to=1]
\end{tikzcd}\]
is a Quillen equivalence. 
The two induced diagrams
% https://q.uiver.app/#q=WzAsOCxbMCwxLCJcXHNzZXRQc2h7XFxUaGV0YX0iXSxbMSwwLCJcXHN0cmF0U3NldF5cXG9tZWdhIl0sWzEsMSwiXFx6b2NhdCJdLFswLDAsIlxcem9jYXQiXSxbMiwxLCJcXHN0cmF0U3NldF5cXG9tZWdhIl0sWzMsMCwiXFxzc2V0UHNoe1xcVGhldGF9Il0sWzIsMCwiXFx6b2NhdCJdLFszLDEsIlxcem9jYXQiXSxbMCwyLCJcXHBpXzAiLDJdLFswLDEsIkxfXFxvbWVnYSIsMV0sWzEsMiwiXFxSIl0sWzMsMCwiXFxpb3RhIiwyXSxbMywxLCJcXE4iXSxbNCw1LCJOX3tMX1xcb21lZ2F9IiwxXSxbNiw1LCJcXGlvdGEiXSxbNiw0LCJcXE4iLDJdLFs0LDcsIlxcUiIsMl0sWzUsNywiXFxwaV8wIl1d
\[\begin{tikzcd}
	\zocat & {\stratSset^\omega} & \zocat & {\ssetPsh{\Theta}} \\
	{\ssetPsh{\Theta}} & \zocat & {\stratSset^\omega} & \zocat
	\arrow["{\pi_0}"', from=2-1, to=2-2]
	\arrow["{L_\omega}"{description}, from=2-1, to=1-2]
	\arrow["\R", from=1-2, to=2-2]
	\arrow["\iota"', from=1-1, to=2-1]
	\arrow["\N", from=1-1, to=1-2]
	\arrow["{N_{L_\omega}}"{description}, from=2-3, to=1-4]
	\arrow["\iota", from=1-3, to=1-4]
	\arrow["\N"', from=1-3, to=2-3]
	\arrow["\R"', from=2-3, to=2-4]
	\arrow["{\pi_0}", from=1-4, to=2-4]
\end{tikzcd}\]
commute up to homotopy.
\end{theorem}
\begin{proof}
If $n<\omega$, the first assertion is a consequence of proposition  \ref{prop:intermedeire}. Suppose now that $n=\omega$.
The lemma \ref{lemma:L ff} implies that the left adjoint is homotopically fully faithful. It then remains to show that the right adjoint is conservative. This is a direct consequence of the preservation of globes by $L_{\omega}$ up to homotopy and theorem \ref{theo:f_weak_equivalence_ssi_f_G_equivalence}.

For the second assertion, it it sufficient to demonstrate that the restriction to $\Theta$ of the canonical natural transformation $\R \circ L \to \pi_0$ is the identity. As these two functors send Segal extensions on identities, it it sufficient to show the result on globes where it directly follows from the preservation of globes by $L$ up to homotopy.
\end{proof}



%
%%%%%\bibliographystyle{alpha}
%\bibliography{../../header/biblio}
%\end{document}


