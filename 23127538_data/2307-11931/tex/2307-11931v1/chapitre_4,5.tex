%\documentclass[12pt]{book}
%\usepackage{index}
%\makeindex
%\renewcommand\indexname{Index of notions}
%\newindex{notation}{adx}{and}{Index of symbols}
%\newindex{notion}{bdx}{bnd}{Index of notions}
%\usepackage{tikz}
\usepackage{xcolor,xspace}
\usepackage{url}
\usepackage{epsfig,graphicx,endnotes,kotex,subfigure,multirow,amsmath,algorithm,algpseudocode}
\newcommand\StateX{\Statex\hspace{\algorithmicindent}}%
%\usepackage{breakurl}
%\usepackage[sort,space]{cite}
\usepackage{balance}
%\usepackage{tabularx}
\usepackage{enumitem}
\usepackage{flushend}
\usepackage[T1]{fontenc}
\usepackage{color,soul}
\hyphenation{op-tical net-works semi-conduc-tor}
%\usepackage{filecontents}
%\usepackage{booktabs} % For formal tables
\usepackage{amsthm}
\newtheorem{theorem}{Theorem}
\newtheorem{corollary}{Corollary}
\newtheorem{lemma}{Lemma}
\renewcommand{\qedsymbol}{\rule{0.7em}{0.7em}}

%\newcommand\notion[1]{\textit{#1}\index[notion]{#1}}
\newcommand\wcnotion[2]{\textit{#1}\index[notion]{#2}}
\newcommand\wcnotionsym[3]{\textit{#1}\index[notation]{#2}\index[notion]{#3}}
\newcommand\wcsnotion[3]{\textit{#1}\index[notion]{#2!\textit{#3}}}
\newcommand\snotion[2]{\textit{#1}\index[notion]{#1!\textit{#2}}}
\newcommand\snotionsym[3]{\textit{#1}\index[notion]{#1!\textit{#3}}\index[notation]{#2!\textit{#3}}}
\newcommand\wcsnotionsym[4]{\textit{#1}\index[notation]{#2!\textit{#4}}\index[notion]{#3!\textit{#4}}}

\newcommand\wcnotation[2]{\textit{#1}\index[notation]{#2}}
\newcommand\wcsnotation[3]{\textit{#1}\index[notation]{#2!\textit{#3}}}

\newcommand\sym[1]{\index[notation]{#1}}
\newcommand\ssym[2]{\index[notation]{#1!\textit{#2}}}

\newcommand{\exclam}{!}





\newcommand{\Ab}{\mathbb{A}} 
\newcommand{\Zb}{\mathbb{Z}} 
\newcommand{\Eb}{\mathbb{E}} 
\newcommand{\Nb}{\mathbb{N}}
\newcommand{\Tb}{\mathbf{T}} 
\newcommand{\Yb}{\mathbb{Y}} 
\newcommand{\Ib}{\mathbb{I}} 
\newcommand{\Ob}{\mathbb{O}} 
\newcommand{\Pb}{\mathbb{P}} 
\newcommand{\Qb}{\mathbb{Q}} 
\newcommand{\Sb}{\mathbb{S}} 
\newcommand{\Hb}{\mathbb{H}} 
\newcommand{\Jb}{\mathbf{J}} 
\newcommand{\Kb}{\mathbb{K}} 
\newcommand{\Mb}{\mathbb{M}} 
\newcommand{\Wb}{\mathbf{W}} 
\newcommand{\Xb}{\mathbb{X}} 
\newcommand{\Cb}{\mathbf{C}}
\newcommand{\Vb}{\mathbb{V}}
\newcommand{\Bb}{\mathbb{B}}


\newcommand{\Acal}{\mathcal{A}} 
\newcommand{\Zcal}{\mathcal{Z}} 
\newcommand{\Ecal}{\mathcal{E}} 
\newcommand{\Rcal}{\mathcal{R}} 
\newcommand{\Tcal}{\mathcal{T}} 
\newcommand{\Ycal}{\mathcal{Y}} 
\newcommand{\Ucal}{\mathcal{U}} 
\newcommand{\Ical}{\mathcal{I}} 
\newcommand{\Ocal}{\mathcal{O}} 
\newcommand{\Pcal}{\mathcal{P}} 
\newcommand{\Qcal}{\mathcal{Q}} 
\newcommand{\Scal}{\mathcal{S}} 
\newcommand{\Dcal}{\mathcal{D}} 
\newcommand{\Fcal}{\mathcal{F}} 
\newcommand{\Gcal}{\mathcal{G}} 
\newcommand{\Hcal}{\mathcal{H}} 
\newcommand{\Jcal}{\mathcal{J}} 
\newcommand{\Kcal}{\mathcal{K}} 
\newcommand{\Lcal}{\mathcal{L}} 
\newcommand{\Mcal}{\mathcal{M}} 
\newcommand{\Wcal}{\mathcal{W}} 
\newcommand{\Xcal}{\mathcal{X}} 
\newcommand{\Ccal}{\mathcal{C}} 
\newcommand{\Vcal}{\mathcal{V}} 
\newcommand{\Bcal}{\mathcal{B}} 
\newcommand{\Ncal}{\mathcal{N}} 


\newcommand{\Ago}{\mathfrak{A}} 
\newcommand{\Zgo}{\mathfrak{Z}} 
\newcommand{\Ego}{\mathfrak{E}} 
\newcommand{\Rgo}{\mathfrak{R}} 
\newcommand{\Tgo}{\mathfrak{T}} 
\newcommand{\Ygo}{\mathfrak{Y}} 
\newcommand{\Ugo}{\mathfrak{U}} 
\newcommand{\Igo}{\mathfrak{I}} 
\newcommand{\Ogo}{\mathfrak{O}} 
\newcommand{\Pgo}{\mathfrak{P}} 
\newcommand{\Qgo}{\mathfrak{Q}} 
\newcommand{\Sgo}{\mathfrak{S}} 
\newcommand{\Dgo}{\mathfrak{D}} 
\newcommand{\Fgo}{\mathfrak{F}} 
\newcommand{\Ggo}{\mathfrak{G}} 
\newcommand{\Hgo}{\mathfrak{H}} 
\newcommand{\Jgo}{\mathfrak{J}} 
\newcommand{\Kgo}{\mathfrak{K}} 
\newcommand{\Lgo}{\mathfrak{L}} 
\newcommand{\Mgo}{\mathfrak{M}} 
\newcommand{\Wgo}{\mathfrak{W}} 
\newcommand{\Xgo}{\mathfrak{X}} 
\newcommand{\Cgo}{\mathfrak{C}} 
\newcommand{\Vgo}{\mathfrak{V}} 
\newcommand{\Bgo}{\mathfrak{B}} 
\newcommand{\Ngo}{\mathfrak{N}}



\newcommand{\sslash}{\mathbin{/\mkern-6mu/}}

\newcommand{\note}[1]{{\color{red}#1}}

\def\-{\raisebox{.75pt}{-}}


\newcommand{\uvar}{\_}


%basic notation
\newcommand{\id}{\text{Id}}
\newcommand{\Db}{\mathbf{D}} 
\DeclareMathOperator*{\dom}{dom}
\DeclareMathOperator*{\codom}{codom}
\DeclareMathOperator{\tw}{tw}


%derived notation
\newcommand{\Rb}{\mathbf{R}} 
\newcommand{\Lb}{\mathbf{L}} 
\newcommand{\Fb}{\mathbf{F}} 
\DeclareMathOperator{\Gb}{G} 
  
%ambiguous notation 
\DeclareMathOperator{\N}{N}
\DeclareMathOperator{\T}{T}
\DeclareMathOperator{\J}{J}


%set of maps
\DeclareMathOperator*{\W}{W}
\DeclareMathOperator*{\Wm}{tW}
\DeclareMathOperator*{\Wseg}{W_{Seg}}
\DeclareMathOperator*{\Wsat}{W_{Sat}}

\DeclareMathOperator*{\M}{M}
\DeclareMathOperator*{\Mm}{tM}
\DeclareMathOperator*{\Mseg}{M_{Seg}}
\DeclareMathOperator*{\Msat}{M_{Sat}}

\DeclareMathOperator*{\I}{I}
\DeclareMathOperator*{\F}{F}

%augmented directed complexes
\DeclareMathOperator*{\CDA}{ADC}
\DeclareMathOperator*{\CDAB}{ADC_B}

%categories
\newcommand\omegacat{\omega\mbox{-$\cat$}}
\DeclareMathOperator\Set{Set}
\DeclareMathOperator\Sp{Sp}

%infini groupoids
\DeclareMathOperator*{\Sq}{Sq}
\DeclareMathOperator*{\Li}{Li}
\DeclareMathOperator{\Hom}{Hom}


%infini 1 categories
\DeclareMathOperator*{\Lfib}{LFib}
\DeclareMathOperator*{\Rfib}{RFib}

\DeclareMathOperator*{\LCartoperator}{LCart}
\DeclareMathOperator*{\core}{core}
\newcommand{\LCart}{\mbox{$\LCartoperator$}}

\newcommand{\LCartc}{\mbox{$\LCartoperator$}^c}
\DeclareMathOperator*{\RCart}{RCart}
\DeclareMathOperator*{\RCartc}{RCart^c}




%infini omega categories
\newcommand{\uLCart}{\underline{\LCartoperator}}
\newcommand{\uLCartc}{\underline{\LCartoperator}^c}
\newcommand{\uRCart}{\underline{RCart}}
\newcommand{\uRCartc}{\underline{RCart}^c}

\DeclareMathOperator{\uHom}{\underline{Hom}}
\DeclareMathOperator{\gHom}{\underline{Hom}_{\ominus}}
\DeclareMathOperator{\Map}{Map}
\DeclareMathOperator{\im}{Im}

\newcommand{\uni}{\underline{\omega}}
\newcommand\w[1]{\widehat{#1}}

%functors
\DeclareMathOperator*{\ev}{ev}
\DeclareMathOperator*{\Arr}{Arr}
\newcommand{\Noiun}{\N_{\tiny{(\omega,1)}}}


\newcommand{\colim}{\operatornamewithlimits{colim}}
\newcommand{\laxcolim}{\operatornamewithlimits{laxcolim}}
\newcommand{\laxlim}{\operatornamewithlimits{laxlim}}


%prefixes
\DeclareMathOperator{\Lan}{Lan}
\DeclareMathOperator{\Ran}{Ran}
\newcommand\iun{(\infty,1)}
\newcommand\io{(\infty,\omega)}
\newcommand\ioun{(\infty,\omega,1)}
\newcommand\zoun{(0,\omega,1)}
\newcommand\zo{(0,\omega)}

%leibnitz products
\DeclareMathOperator{\hstar}{\hat{\star}}
\DeclareMathOperator{\htimes}{\hat{\times}}
\DeclareMathOperator{\hotimes}{\hat{\otimes}}


%Gray operations
\DeclareMathOperator{\costarindex}{f}
\newcommand{\costar}{\mathbin{\overset{co}{\star}}}
\newcommand{\fwedge}{\mathbin{\rotatebox[origin=c]{270}{$\gtrdot$}}}


%inclassable
\newcommand{\invamalg}{\mathbin{\rotatebox[origin=c]{180}{$\amalg$}}}
\DeclareMathOperator{\botimes}{\bar{\otimes}}
\DeclareMathOperator\cst{cst}
\DeclareMathOperator\Operatormark{mk}
\newcommand{\mk}{\Operatormark}

%category theory
\DeclareMathOperator\Fun{Fun}
\DeclareMathOperator\Nat{Nat}
\DeclareMathOperator\End{End}



%fundamental notation
\DeclareMathOperator\mcat{cat_m}
\DeclareMathOperator\cat{cat}
\DeclareMathOperator\grd{grd}
\DeclareMathOperator\R{R}

\newcommand\ocat{(\infty,\omega)\mbox{-$\cat$}}
\newcommand\ouncat{(\infty,\omega,1)\mbox{-$\cat$}}
\newcommand\ocatm{{(\infty,\omega)\mbox{-$\mcat$}}}
\newcommand\zocatm{(0,\omega)\mbox{-$\mcat$}}
\newcommand\zocat{(0,\omega)\mbox{-$\cat$}}
\DeclareMathOperator\zocatB{\zocat_B}
\newcommand\icat{(\infty,1)\mbox{-$\cat$}}
\newcommand\qcat{\mbox{Q$\cat$}}
\newcommand\ncat[1]{(\infty, #1)\mbox{-$\cat$}}
\newcommand\zncat[1]{(0, #1)\mbox{-$\cat$}}
\newcommand\igrd{\infty\mbox{-$\grd$}}



\DeclareMathOperator{\OperatorinfiniPsh}{Psh^\infty}
\DeclareMathOperator{\OperatorinfinitPsh}{tPsh^\infty}
\DeclareMathOperator{\OperatorPsh}{Psh}
\DeclareMathOperator{\OperatormPsh}{mPsh}
\DeclareMathOperator{\OperatortPsh}{tPsh}
\newcommand\iPsh[1]{\OperatorinfiniPsh({#1})}
\newcommand\tiPsh[1]{\OperatorinfinitPsh({#1})}
\newcommand\Psh[1]{\OperatorPsh({#1})}
\newcommand\ssetPsh[1]{\OperatorPsh_\Delta({#1})}
\newcommand\tPsh[1]{\OperatortPsh({#1})}
\newcommand\tPshM[1]{{\OperatortPsh}_M({#1})}
\newcommand\mPsh[1]{\OperatormPsh({#1})}
\newcommand\mPshM[1]{{\OperatormPsh}_M({#1})}

%segal stuff
\DeclareMathOperator{\OperatorSeg}{Seg}
\DeclareMathOperator{\OperatortSeg}{tSeg}
\DeclareMathOperator{\OperatormSeg}{mSeg}
\newcommand\Seg{\OperatorSeg}
\newcommand\mSeg{\OperatormSeg}
\newcommand\stratSeg{\OperatortSeg}

%simplicial variations
\DeclareMathOperator{\Sset}{\Psh{\Delta}}
\newcommand{\mSset}{\mPsh{\Delta}}
\newcommand{\stratSset}{\tPsh{\Delta}}


%univers
\DeclareMathOperator{\U}{\mathbf{U}}
\DeclareMathOperator{\V}{\mathbf{V}}
\DeclareMathOperator{\Wcard}{\mathbf{W}}
\DeclareMathOperator{\Z}{\mathbf{Z}}



%Grothendieck constructions
\newcommand{\ringpartial}{\mathring{\partial}}
%
%
%\usepackage[inline]{showlabels}
%
%\usepackage{fancyhdr}
%\usepackage{titlesec}
%\usepackage{textcase}
%
%\pagestyle{fancy}
%
%
%\title{\Huge{Theory and models of $(\infty,\omega)$-categories}}
%\author{Félix Loubaton}
%\date{}
%\linespread{1.2}	
%\geometry{a4paper,top=3cm,bottom=4cm,left=1.5cm,right=3cm, heightrounded,bindingoffset=5mm}	
%
%
%\fancyhf{}
%\fancyhfoffset[RO,LE]{0.5cm}
%\fancyhfoffset[LE,RO]{0.5cm}
%
%\fancyhead[RO]{\rmfamily\nouppercase{\rightmark}}
%\fancyhead[LE]{\rmfamily\nouppercase{\leftmark}}
%\fancyfoot[C]{\thepage}
%
%\begin{document}



\chapter{The $\iun$-category of marked $\io$-categories}
\label{chapter:chapter the 1 category of marked categories}

\minitoc
\vspace{2cm}

This chapter is dedicated to the study of \textit{marked $\io$-categories}, which are pairs $(C,tC)$, where $C$ is an $\io$-category and $tC:=(tC_n)_{n>0}$ is a sequence of full sub $\infty$-groupoids of $C_n$ that include identities and are stable under composition and whiskering with (possibly unmarked) cells of lower dimensions. There are two canonical ways to mark an $\io$-category $C$. In the first, denoted by $C^\flat$, we mark as little as possible. In the second, denoted by $C^\sharp$, we mark everything.

The first section of the chapter defines these objects and establishes analogs of many results from section \ref{chapter:Basica construciton} to this new framework. In particular, the \textit{marked Gray cylinder} $\uvar\otimes [1]^\sharp$ is defined. If $A$ is an $\io$-category, the underlying $\io$-category of $A^\sharp\otimes[1]^\sharp$ is $A\times [1]$, and the underlying $\io$-category of $A^\flat\otimes[1]^\sharp$ is $A\otimes[1]$. By varying the marking, and at the level of underlying $\io$-categories, we "continuously" move from the cartesian product with the directed interval to the Gray tensor product with the directed interval.

The motivation for introducing markings comes from the notion of \textit{left (and right) cartesian fibrations}. A left cartesian fibration is a morphism between marked $\io$-categories such that only the marked cells of the codomain have cartesian lifting, and the marked cells of the domain correspond exactly to such cartesian lifting. For example, a left cartesian fibration $X\to A^\sharp$ is just a "usual" left cartesian fibration where we have marked the cartesian lifts of the domain, and every morphism $C^\flat \to D^\flat$ is a left cartesian fibration. This shows that marking plays a very different role here than in the case of marked simplicial sets, where it was there to represent (weak) invertibility. For example, if we had wanted to carry out this work in a complicial-like model category, we would have had to consider bimarked simplicial sets.



After defining and enumerating the stability properties enjoyed by this class of left (and right) cartesian fibration, we give several characterizations of this notion in theorem \ref{theo:other characterisation of left caresian fibration}. 

The more general subclass of left cartesian fibrations that still behaves well is the class of \textit{classified left cartesian fibrations}. 
This corresponds to left cartesian fibrations $X\to A$ such that there exists a cartesian square:
\[\begin{tikzcd}
	X & Y \\
	A & {A^\sharp}
	\arrow[from=1-1, to=2-1]
	\arrow[from=2-1, to=2-2]
	\arrow[from=1-1, to=1-2]
	\arrow[from=1-2, to=2-2]
	\arrow["\lrcorner"{anchor=center, pos=0.125}, draw=none, from=1-1, to=2-2]
\end{tikzcd}\]
 where the right vertical morphism is a left cartesian fibration and $A^\sharp$ is obtained from $A$ by marking all cells. In the second section, we prove the following fundamental result:

\begin{itheorem}[\ref{theo:pullback along un marked cartesian fibration}]
Let $p:X\to A$ be a classified left cartesian fibration. Then the functor $p^*:\ocatm_{/A}\to \ocatm_{/X}$ preserves colimits.
\end{itheorem}

The third subsection is devoted to the proof of the following theorem
\begin{itheorem}[\ref{theo:left cart stable by colimit}]
Let $A$ be an $\io$-category and $F:I\to \ocatm_{/A^\sharp}$ be a diagram that is pointwise a left cartesian fibration. The induced morphism 
$\colim_IF$ is a left cartesian fibration over $A^\sharp$.
\end{itheorem}



In the fourth subsection we study \textit{smooth} and \textit{proper} morphisms and we obtain the following expected result:
\begin{iprop}[\ref{prop:quillent theorem A}]
For a morphism $X\to A^\sharp$, and an object $a$ of $A$, we denote by $X_{/a}$ the marked $\io$-category fitting in the following cartesian squares. 
% https://q.uiver.app/#q=WzAsNCxbMSwxLCJBXlxcc2hhcnAiXSxbMCwxLCJBXlxcc2hhcnBfe2EvfSJdLFsxLDAsIlgiXSxbMCwwLCJYX3thL30iXSxbMSwwXSxbMiwwXSxbMywyXSxbMywxXSxbMywwLCIiLDEseyJzdHlsZSI6eyJuYW1lIjoiY29ybmVyIn19XV0=
\[\begin{tikzcd}
	{X_{a/}} & X \\
	{A^\sharp_{a/}} & {A^\sharp}
	\arrow[from=2-1, to=2-2]
	\arrow[from=1-2, to=2-2]
	\arrow[from=1-1, to=1-2]
	\arrow[from=1-1, to=2-1]
	\arrow["\lrcorner"{anchor=center, pos=0.125}, draw=none, from=1-1, to=2-2]
\end{tikzcd}\]
We denote by $\bot:\ocatm\to \ocat$ the functor sending a marked $\io$-category to its localization by marked cells.
\begin{enumerate}
\item Let $E$, $F$ be two elements of $\ocatm_{/A^\sharp}$ corresponding to morphisms $X\to A^\sharp$, $Y\to A^\sharp$, and
 $\phi:E\to F$ a morphism between them. We denote by $\Fb E$ and $\Fb F$ the left cartesian fiborant replacement of $E$ and $F$. 
 
The induced morphism $\Fb\phi:\Fb E\to \Fb F$ is an equivalence if and only if for any object $a$ of $A$, the induced morphism 
$$\bot X_{/a}\to \bot Y_{/a}$$ 
is an equivalence of $\io$-categories.
\item A morphism $X\to A^\sharp$ is initial if and only if for any object $a$ of $A$, $\bot X_{/a}$ is the terminal $\io$-category.
\end{enumerate}
\end{iprop}



Finally, in the last subsection, for a marked $\io$-category $I$, we define and study a (huge) $\io$-category $\uLCartc(I)$ that has classified left cartesian fibrations as objects and morphisms between classified left cartesian fibrations as arrows.

\paragraph{Cardinality hypothesis.}
We fix during this chapter two Grothendieck universes $\V\in\Wcard$, such that $\omega\in \U$. When nothing is specified, this corresponds to the implicit choice of the cardinality $\V$.
We then denote by $\Set$ the $\Wcard$-small $1$-category of $\V$-small sets, $\igrd$ the $\Wcard$-small $\iun$-category of $\V$-small $\infty$-groupoids and $\icat$ the $\Wcard$-small $\iun$-category of $\V$-small $\iun$-categories.


\section{Marked $\io$-categories}
\subsection{Definition of marked $\io$-categories}
\p
A \notion{marked $\zo$-category} is a pair $(C,tC)$ where $C$ is an $\zo$-category and $tC:=(tC_n)_{n>0}$ is a sequence of subsets of $C_n$, containing identities, stable by composition and by whiskering with (possibly unmarked) cells of lower dimension.
A $n$-cell $a:\Db_n\to (C,tC)$ is \wcsnotion{marked}{marked $n$-cell}{for marked $\zo$-categories} if it belongs to $tC_n$.

A \notion{marked morphism} $f:(C,tC)\to (D,tT)$ is the data of a morphism on the underlying $\zo$-categories such that $f(tC_n)\subset tD_n$.
The category of marked $\zo$-categories is denoted by \wcnotation{$\zocatm$}{((a40@$\zocatm$}. 



\p There are two canonical ways to mark an $\zo$-category. For $C\in \zocat$, we define $C^\sharp := (C,(C_n)_{n>0})$ and $C^\flat := (C,(\Ib(C_{n-1})_{n>0}))$. The first one corresponds to the case where all cells are marked, and the second one where only the identities are marked. These two functors fit in the following adjoint triple:
\ssym{((b10@$(\uvar)^\sharp$}{for (marked) $\zo$-categories}\ssym{((b20@$(\uvar)^\flat$}{for (marked) $\zo$-categories}\ssym{((b30@$(\uvar)^\natural$}{for (marked) $\zo$-categories}
\[\begin{tikzcd}
	{(\uvar)^{\flat}: \zocat} & {\zocatm:(\uvar)^\natural} & {(\uvar)^\natural:\zocatm} & { \zocat	:(\uvar)^\sharp}
	\arrow[""{name=0, anchor=center, inner sep=0}, shift left=2, from=1-1, to=1-2]
	\arrow[""{name=1, anchor=center, inner sep=0}, shift left=2, from=1-2, to=1-1]
	\arrow[""{name=2, anchor=center, inner sep=0}, shift left=2, from=1-3, to=1-4]
	\arrow[""{name=3, anchor=center, inner sep=0}, shift left=2, from=1-4, to=1-3]
	\arrow["\dashv"{anchor=center, rotate=-90}, draw=none, from=0, to=1]
	\arrow["\dashv"{anchor=center, rotate=-90}, draw=none, from=2, to=3]
\end{tikzcd}\]
where $(\uvar)^\natural$ is the obvious forgetfull functor.
To simplify notations, for a marked $\zo$-category $C$, the marked $\io$-categories $(C^\natural)^\flat$ and $(C^\natural)^\sharp$ will be simply denoted by $C^\flat$ and $C^\sharp$.

\begin{example}
For $n$ an integer, we denote by $(\Db_n)_t$ the marked $\zo$-category whose underlying $\zo$-category is $\Db_n$ and whose only non-trivial marked cell is the top dimensional one.
\end{example}

\begin{definition}
We define the category \wcnotation{$t\Theta$}{(tTheta@$t\Theta$} as the full subcategory of $\zocatm$ whose objects are of shape $a^\flat$ for $a$ a globular sum, or \wcnotation{$(\Db_n)_t$}{(dn@$(\Db_n)_t$} for an integer $n\in\Nb$. Remark that this subcategory is dense in $\zocatm$.
\end{definition}




\p We define the $\iun$-category of \wcnotion{stratified $\infty$-presheaves on $\Theta$}{stratified $\infty$-presheaf on $\Theta$}, noted by \wcnotation{$\tiPsh{\Theta}$}{(tPsh@$\tiPsh{\Theta}$}, as the full sub $\iun$-category of $\iPsh{t\Theta}$ whose objects correspond to $\infty$-presheaves $X$ such that the induced morphism
$X((\Db_n)_t)\to X(\Db_n)$
is a monomorphism. 

\begin{prop}
\label{prop:marked presheaves are locally cartesian closed}
The $\iun$-category $\tiPsh{\Theta}$ is locally cartesian closed. 
\end{prop}
\begin{proof}
The $\iun$-category $\tiPsh{\Theta}$ is the localization of the $\iun$-category $\iPsh{t\Theta}$ along the set of map $\widehat{I}$ with $$I:=\{(\Db_n)_t\coprod_{\Db_n}(\Db_n)_t\to(\Db_n)_t\}_n.$$
As $\iPsh{t\Theta}$ is locally cartesian closed, we have to show that for any integer $n>0$ and any cartesian square in $\iPsh{t\Theta}$:
% https://q.uiver.app/#q=WzAsNCxbMSwxLCIoXFxEYl9uKV90Il0sWzAsMSwiKFxcRGJfbilfdFxcY29wcm9kX3tcXERiX259KFxcRGJfbilfdCJdLFsxLDAsIlgiXSxbMCwwLCJYJyJdLFsyLDAsIiAiXSxbMywxXSxbMywyLCIgIl0sWzEsMF0sWzMsMCwiIiwyLHsic3R5bGUiOnsibmFtZSI6ImNvcm5lciJ9fV1d
\[\begin{tikzcd}
	{X'} & X \\
	{(\Db_n)_t\coprod_{\Db_n}(\Db_n)_t} & {(\Db_n)_t}
	\arrow["{ }", from=1-2, to=2-2]
	\arrow[from=1-1, to=2-1]
	\arrow["{ }", from=1-1, to=1-2]
	\arrow[from=2-1, to=2-2]
	\arrow["\lrcorner"{anchor=center, pos=0.125}, draw=none, from=1-1, to=2-2]
\end{tikzcd}\]
the top horizontal morphism is in $\widehat{I}$. Using once again the locally cartesian closeness of $\iPsh{t\Theta}$, it is sufficient to show that for any integer $n>0$ and for any morphism 
$j:b\to (\Db_n)_t$ between elements of $t\Theta$, the morphism $i$ appearing in the following cartesian square of $\iPsh{t\Theta}$ is an equivalence or is in $I$:
 % q.uiver.app/#q=WzAsNCxbMSwxLCIoXFxEYl9uKV90Il0sWzAsMSwiKFxcRGJfbilfdFxcY29wcm9kX3tcXERiX259KFxcRGJfbilfdCJdLFsxLDAsImIiXSxbMCwwLCJCIl0sWzIsMCwiaiJdLFszLDFdLFszLDIsImkiXSxbMSwwXV0=
\[\begin{tikzcd}
	B & b \\
	{(\Db_n)_t\coprod_{\Db_n}(\Db_n)_t} & {(\Db_n)_t}
	\arrow["j", from=1-2, to=2-2]
	\arrow[from=1-1, to=2-1]
	\arrow["i", from=1-1, to=1-2]
	\arrow[from=2-1, to=2-2]
\end{tikzcd}\]
Two cases have to be considered. If $j$ is the identity this is trivially true. If $j$ is any other morphism, it factors through $\Db_n\to (\Db_n)_t$, 	and the following square is cartesian
% q.uiver.app/#q=WzAsNCxbMSwxLCIoXFxEYl9uKV90Il0sWzAsMSwiXFxEYl9uIl0sWzEsMCwiYiJdLFswLDAsImIiXSxbMiwwLCJqIl0sWzMsMV0sWzMsMiwiaWQiXSxbMSwwXV0=
\[\begin{tikzcd}
	b & b \\
	{\Db_n} & {(\Db_n)_t}
	\arrow["j", from=1-2, to=2-2]
	\arrow[from=1-1, to=2-1]
	\arrow["id", from=1-1, to=1-2]
	\arrow[from=2-1, to=2-2]
\end{tikzcd}\]
 This implies that $B$ is equivalent to $b\coprod_bb\sim b$, and $i$ is then the identity.
 \end{proof}

\p \label{para:defi of music}
For a stratified $\infty$-presheaf $X$ on $\Theta$, we denote by $tX_n$ the $\infty$-groupoid $X((\Db_n)_t)$.
A stratified $\infty$-presheaves on $\Theta$ is then the data of a pair $(X,tX)$ such that $X\in \iPsh{\Theta}$ and $tX:=(tX_n)_{n>0}$ is a sequence of $\infty$-groupoid such that for any $n>0$, $tX_n$ is a full sub $\infty$-groupoid of $X_n$ including all units.


For $X\in \iPsh{\Theta}$, we define $X^\sharp := (X,(X_n)_{n>0})$ and $X^\flat := (X,(\Ib (X_{n-1})_{n>0})$ and we have an adjoint triple 
\ssym{((b10@$(\uvar)^\sharp$}{for (marked) $\io$-categories}\ssym{((b20@$(\uvar)^\flat$}{for (marked) $\io$-categories}\ssym{((b30@$(\uvar)^\natural$}{for (marked) $\io$-categories}
\[\begin{tikzcd}
	{(\uvar)^{\flat}: \iPsh{\Theta}} & {\tiPsh{\Theta}:(\uvar)^\natural} & {(\uvar)^\natural:\tiPsh{\Theta}} & { \Psh{\Theta}:(\uvar)^\sharp}
	\arrow[""{name=0, anchor=center, inner sep=0}, shift left=2, from=1-1, to=1-2]
	\arrow[""{name=1, anchor=center, inner sep=0}, shift left=2, from=1-2, to=1-1]
	\arrow[""{name=2, anchor=center, inner sep=0}, shift left=2, from=1-3, to=1-4]
	\arrow[""{name=3, anchor=center, inner sep=0}, shift left=2, from=1-4, to=1-3]
	\arrow["\dashv"{anchor=center, rotate=-90}, draw=none, from=0, to=1]
	\arrow["\dashv"{anchor=center, rotate=-90}, draw=none, from=2, to=3]
\end{tikzcd}\]
where $(\uvar)^\natural$ is the obvious forgetful functor.



\p We define the category \wcnotation{$t\Delta[t\Theta]$}{(tDeltaTheta@$t\Delta[t\Theta]$} as the pullback
% https://q.uiver.app/#q=WzAsNCxbMCwwLCJ0XFxEZWx0YVt0XFxUaGV0YV0iXSxbMSwwLCJ0XFxUaGV0YSJdLFsxLDEsIlxcVGhldGEiXSxbMCwxLCJcXERlbHRhW1xcVGhldGFdIl0sWzEsMiwiKFxcdXZhcileXFxuYXR1cmFsIl0sWzMsMl0sWzAsM10sWzAsMiwiIiwyLHsic3R5bGUiOnsibmFtZSI6ImNvcm5lciJ9fV0sWzAsMV1d
\[\begin{tikzcd}
	{t\Delta[t\Theta]} & t\Theta \\
	{\Delta[\Theta]} & \Theta
	\arrow["{(\uvar)^\natural}", from=1-2, to=2-2]
	\arrow[from=2-1, to=2-2]
	\arrow[from=1-1, to=2-1]
	\arrow["\lrcorner"{anchor=center, pos=0.125}, draw=none, from=1-1, to=2-2]
	\arrow[from=1-1, to=1-2]
\end{tikzcd}\]
The objects of $t\Delta[t\Theta]$ then are of shape $[1]^\sharp$ or $[a,n]$ with $a\in t\Theta$ and $n\in \Delta$.
The $(\infty,1)$-category of \wcnotion{stratified presheaves on $\Delta[\Theta]$}{stratified $\infty$-presheaf on $\Delta[\Theta]$}, denoted by $\tiPsh{\Delta[\Theta]}$, is the full sub $\iun$-category of $\iPsh{t\Delta[t\Theta]}$ whose objects correspond to $\infty$-presheaves $X$ such that the induced morphism
$X((\Db_n)_t)\to X(\Db_n)$
is a monomorphism. 

\begin{prop}
\label{prop:marked presheaves are locally cartesian closed2}
The $\iun$-category $\tiPsh{\Delta[\Theta]}$ is locally cartesian closed. 
\end{prop}
\begin{proof}
The proof is almost identical to the one of proposition \ref{prop:marked presheaves are locally cartesian closed}
\end{proof}


\p For a stratified $\infty$-presheaf $X$ on $\Delta[\Theta]$, we denote by $tX_1$ the   $\infty$-groupoid $X([1]^\sharp)$, and for any $n>1$, we denote by $tX_n$ the $\infty$-groupoid $X((\Db_n)_t)$.

A stratified $\infty$-presheaf on $\Delta[\Theta]$ is then the data of a pair $(X,tX)$ such that $X\in \iPsh{\Delta[\Theta]}$
and $tX:=(tX_n)_{n>0}$ is a sequence of $\infty$-groupoid such that for any $n>0$, $tX_n$ is a full sub $\infty$-groupoid of $X_n$ including all units.


 For $X\in \iPsh{\Delta[\Theta]}$, we define once again $X^\sharp := (X,( X_n)_{n>0})$ and $X^\flat := (X, (\Ib (X_{n-1}))_{n>0})$ and we still have an adjoint triple
% q.uiver.app/#q=WzAsNCxbMCwwLCIoXFx1dmFyKV5cXG5hdHVyYWxcXGlQc2h7XFxEZWx0YVtcXFRoZXRhXX0iXSxbMSwwLCJcXHRpUHNoe1xcRGVsdGFbXFxUaGV0YV19OihcXHV2YXIpXlxcbmF0dXJhbCJdLFsyLDAsIihcXHV2YXIpXlxcbmF0dXJhbDpcXHRpUHNoe1xcRGVsdGFbXFxUaGV0YV19Il0sWzMsMCwiXFxpUHNoe1xcRGVsdGFbXFxUaGV0YV19OihcXHV2YXIpXlxcc2hhcnAiXSxbMCwxLCIiLDAseyJvZmZzZXQiOi0yfV0sWzEsMCwiIiwwLHsib2Zmc2V0IjotMn1dLFsyLDMsIiIsMCx7Im9mZnNldCI6LTJ9XSxbMywyLCIiLDAseyJvZmZzZXQiOi0yfV0sWzQsNSwiIiwwLHsibGV2ZWwiOjEsInN0eWxlIjp7Im5hbWUiOiJhZGp1bmN0aW9uIn19XSxbNiw3LCIiLDAseyJsZXZlbCI6MSwic3R5bGUiOnsibmFtZSI6ImFkanVuY3Rpb24ifX1dXQ==
\[\begin{tikzcd}[column sep=0.5cm]
	{(\uvar)^\natural\iPsh{\Delta[\Theta]}} & {\tiPsh{\Delta[\Theta]}:(\uvar)^\natural} & {(\uvar)^\natural:\tiPsh{\Delta[\Theta]}} & {\iPsh{\Delta[\Theta]}:(\uvar)^\sharp}
	\arrow[""{name=0, anchor=center, inner sep=0}, shift left=2, from=1-1, to=1-2]
	\arrow[""{name=1, anchor=center, inner sep=0}, shift left=2, from=1-2, to=1-1]
	\arrow[""{name=2, anchor=center, inner sep=0}, shift left=2, from=1-3, to=1-4]
	\arrow[""{name=3, anchor=center, inner sep=0}, shift left=2, from=1-4, to=1-3]
	\arrow["\dashv"{anchor=center, rotate=-90}, draw=none, from=0, to=1]
	\arrow["\dashv"{anchor=center, rotate=-90}, draw=none, from=2, to=3]
\end{tikzcd}\]
where $(\uvar)^\natural$ is the obvious forgetfull functor.

\p
We once again have an adjunction:
% q.uiver.app/#q=WzAsMixbMCwwLCJpXyE6XFx0aVBzaHtcXERlbHRhW1xcVGhldGFdfSJdLFsxLDAsIlxcdGlQc2h7XFxUaGV0YX06aV4qIl0sWzAsMSwiIiwwLHsib2Zmc2V0IjotMn1dLFsxLDAsIiIsMCx7Im9mZnNldCI6LTJ9XV0=
\[\begin{tikzcd}
	{i_!:\tiPsh{\Delta[\Theta]}} & {\tiPsh{\Theta}:i^*}
	\arrow[shift left=2, from=1-1, to=1-2]
	\arrow[shift left=2, from=1-2, to=1-1]
\end{tikzcd}\]
induced by the canonical inclusion $t\Delta[t\Theta]\to t\Theta$.
For an integer $n$, we define the functor \sym{((b40@$(\uvar)^{\sharp_n}$}$(\uvar)^{\sharp_n}:\iPsh{\Theta}\to \tiPsh{\Theta}$ and $(\uvar)^{\sharp_n}:\iPsh{\Delta[\Theta]}\to \tiPsh{\Delta[\Theta]}$ sending a $\infty$-presheaf $X$ onto $ (X, (X^n_k)_{k>0})$ where $X^n_k:= \Ib(X_{k-1})$ if $k<n$, and $X^n_k:=X_k$ if not. We eventually set \sym{(tw@$\Wm$}\sym{(tm@$\Mm$}
$$\Wm:= \coprod_{n}(\Wseg)^{\sharp_n}\coprod (\Wsat)^\flat~~~~~\Mm:= \coprod_{n}(\Mseg)^{\sharp_n}\coprod(\Msat)^\flat$$
As $i_!(\Mm)$ is contained in $\Wm$, the previous adjunction induces a derived one:
% q.uiver.app/#q=WzAsMixbMCwwLCJcXExiIGlfITpcXHRpUHNoe1xcRGVsdGFbXFxUaGV0YV19X3tcXE1tfSJdLFsxLDAsIlxcdGlQc2h7XFxUaGV0YX1fe1xcV219OmleKlxcUmIiXSxbMCwxLCIiLDAseyJvZmZzZXQiOi0yfV0sWzEsMCwiIiwwLHsib2Zmc2V0IjotMn1dLFsyLDMsIiIsMCx7ImxldmVsIjoxLCJzdHlsZSI6eyJuYW1lIjoiYWRqdW5jdGlvbiJ9fV1d
\begin{equation}
\label{eq:derived marked adjunction theta and delta theta}
\begin{tikzcd}
	{\Lb i_!:\tiPsh{\Delta[\Theta]}_{\Mm}} & {\tiPsh{\Theta}_{\Wm}:i^*\Rb}
	\arrow[""{name=0, anchor=center, inner sep=0}, shift left=2, from=1-1, to=1-2]
	\arrow[""{name=1, anchor=center, inner sep=0}, shift left=2, from=1-2, to=1-1]
	\arrow["\dashv"{anchor=center, rotate=-90}, draw=none, from=0, to=1]
\end{tikzcd}
\end{equation}

\begin{prop}
\label{prop:derived marked adjunction theta and delta theta}
The derived adjunction \eqref{eq:derived marked adjunction theta and delta theta}
is an adjoint equivalence.
\end{prop}
\begin{proof}
It is enough to show that for any element $a:t\Delta[t\Theta]$ and any $b:t\Theta$, $a\to i^*i_!a$ and $i_!i^*b\to b$ are respectively in $\widehat{\Mm}$ and $\widehat{\Wm}$. If $a$ is of shape $[b,n]^\flat$, this is a direct consequence of proposition \ref{prop:infini changing theta}, and if $a$ is $(\Db_n)_t$ the unit is the identity. We proceed similarly for $i_!i^*b\to b$.
\end{proof}
 The inclusion $t\Theta\to \zocatm$ induces an adjunction
% q.uiver.app/#q=WzAsMixbMCwwLCJcXHRQc2h7XFxUaGV0YX0iXSxbMSwwLCJcXHpvY2F0bSJdLFswLDEsIiIsMCx7Im9mZnNldCI6LTJ9XSxbMSwwLCIiLDAseyJvZmZzZXQiOi0yfV0sWzIsMywiIiwwLHsibGV2ZWwiOjEsInN0eWxlIjp7Im5hbWUiOiJhZGp1bmN0aW9uIn19XV0=
\[\begin{tikzcd}
	{\tPsh{\Theta}} & \zocatm
	\arrow[""{name=0, anchor=center, inner sep=0}, shift left=2, from=1-1, to=1-2]
	\arrow[""{name=1, anchor=center, inner sep=0}, shift left=2, from=1-2, to=1-1]
	\arrow["\dashv"{anchor=center, rotate=-90}, draw=none, from=0, to=1]
\end{tikzcd}\]
and we can easily check that this induces an equivalence between $\zocatm$ and the sub-category of $\tPsh{\Theta}$ of $\Wm$-local objects.
Together with proposition \ref{prop:derived marked adjunction theta and delta theta}, this induces equivalences
$$\tPsh{\Theta}_{\Mm} \cong \tPsh{\Delta[\Theta]}_{\Wm}\cong \zocatm$$


\p A \notion{marked $\io$-category} is a $\Wm$-local stratified $\infty$-presheaves on $\Theta$. We denote by \wcnotation{$\ocatm$}{((a70@$\ocatm$} the $\iun$-category of marked $\io$-categories.
Unfolding the definition, a marked $\io$-category is a pair $(C,tC)$ where $C$ is an $\io$-category and $tC:=(tC_n)_{n>0}$ is a sequence of full sub $\infty$-groupoids of $C_n$, containing identities, stable by composition and by whiskering with cells of lower dimension.
A $n$-cell $a:\Db_n\to (C,tC)$ is \wcsnotion{marked}{marked $n$-cell}{for marked $\io$-categories} if it belongs to the image of $tC_n$.


There are two obvious ways to mark a $\io$-category. For $C\in \ocat$, we define $C^\sharp := (C,(C_n)_{n>0})$ and $C^\flat := (C,(\Ib(C_{n-1})_{n>0}))$. The first one corresponds to the case where all cells are marked, and the second one where only the identities are marked. These two functors fit in the following adjoint triple:
\[\begin{tikzcd}[column sep=0.7cm]
	{(\uvar)^{\flat}: \ocat} & {\ocatm:(\uvar)^\natural} & {(\uvar)^\natural:\ocatm} & { \ocat	:(\uvar)^\sharp}
	\arrow[""{name=0, anchor=center, inner sep=0}, shift left=2, from=1-1, to=1-2]
	\arrow[""{name=1, anchor=center, inner sep=0}, shift left=2, from=1-2, to=1-1]
	\arrow[""{name=2, anchor=center, inner sep=0}, shift left=2, from=1-3, to=1-4]
	\arrow[""{name=3, anchor=center, inner sep=0}, shift left=2, from=1-4, to=1-3]
	\arrow["\dashv"{anchor=center, rotate=-90}, draw=none, from=0, to=1]
	\arrow["\dashv"{anchor=center, rotate=-90}, draw=none, from=2, to=3]
\end{tikzcd}\]
where $(\uvar)^\natural$ is the obvious forgetful functor.	
To simplify notations, for a marked $\io$-category $C$, the marked $\io$-categories $(C^\natural)^\flat$ and $(C^\natural)^\sharp$ will be simply denoted by $C^\flat$ and $C^\sharp$.


\p Following paragraph \ref{para:dualities non strict case}, for any subset $S$ of $\Nb^*$, we define the duality\ssym{((b49@$(\uvar)^S$}{for marked $\io$-categories}
$$(\uvar)^S:\ocatm\to \ocatm$$
whose value on $(C,tC)$ is $(C^S,tC)$.
In particular, we have the \snotionsym{odd duality}{((b60@$(\uvar)^{op}$}{for marked $\io$-categories} $(\uvar)^{op}$, corresponding to the set of odd integer, the \snotionsym{even duality}{((b50@$(\uvar)^{co}$}{for marked $\io$-categories} $(\uvar)^{co}$, corresponding to the subset of non negative even integer, the \snotionsym{full duality}{((b80@$(\uvar)^{\circ}$}{for marked $\io$-categories} $(\uvar)^{\circ}$, corresponding to $\Nb^*$ and the \snotionsym{transposition}{((b70@$(\uvar)^t$}{for marked $\io$-categories} $(\uvar)^t$, corresponding to the singleton $\{1\}$. Eventually, we have equivalences
$$((\uvar)^{co})^{op}\sim (\uvar)^{\circ} \sim ((\uvar)^{op})^{co}.$$


\p Given a functor $F:I\to \ocatm$, the colimit of $F$ is given by the marked $\io$-category $(C,tC)$ with 
$$C:=\colim_{I}F^\natural$$
and for any $n$, $(tC)_n$ is the image of the morphism 
$$\colim_I tF_n\to (\colim_{I}F)^\natural_n.$$
The case of the limit is easier as we have 
$$\lim_{I}F := (\lim_{I}F^\natural,(\lim_{I}(tF_n)_{n>0}).$$
In particular, if $(C,tC)$ and $(D,tD)$ are two marked $\io$-categories, we have
$$(C,tC)\times (D,tD):= (C\times D, (tC_n\times tD_n)_{n>0}).$$
\begin{prop}
\label{prop:cartesian product preserves W marked version}
The cartesian product in $\ocatm$ preserves colimits in both variables.
\end{prop}
\begin{proof}
Let $F:I\to \ocatm$ be a diagram and $C$ a marked $\io$-category. The underlying $\io$-categories of $\colim_I (F\times C)$ and $(\colim_IF)\times C$ are the same as the cartesian product preserves colimits in $\ocat$. The equivalence of the two markings 
 is a direct consequence of the fact that the cartesian product in $\igrd$ preserves both colimits and the formation of image.
\end{proof}
This demonstrates the existence of an internal hom functor that we denote once again by $\uHom(\uvar,\uvar)$.



\p We denote again $\pi_0:\tiPsh{\Theta}\to \tPsh{\Theta}$ colimit preserving sending a stratified $\infty$-presheaf $X$ to the stratified presheaf $a\mapsto \pi_0(X_a)$. As this functor preserves $\Wm$, it induces an adjoint pair:
\sym{(pi@$\pi_0:\ocatm\to \zocatm$}\sym{n@$\N:\zocatm\to \ocatm$}
% https://q.uiver.app/#q=WzAsMixbMCwwLCJcXHBpXzA6XFxvY2F0Il0sWzEsMCwiXFx6b2NhdDpcXE4iXSxbMCwxLCIiLDAseyJvZmZzZXQiOi0yfV0sWzEsMCwiIiwwLHsib2Zmc2V0IjotMn1dLFsyLDMsIiIsMCx7ImxldmVsIjoxLCJzdHlsZSI6eyJuYW1lIjoiYWRqdW5jdGlvbiJ9fV1d
\[\begin{tikzcd}
	{\pi_0:\ocat} & {\zocat:\N}
	\arrow[""{name=0, anchor=center, inner sep=0}, shift left=2, from=1-1, to=1-2]
	\arrow[""{name=1, anchor=center, inner sep=0}, shift left=2, from=1-2, to=1-1]
	\arrow["\dashv"{anchor=center, rotate=-90}, draw=none, from=0, to=1]
\end{tikzcd}\]
where the right adjoint $\N$ is fully faithful.
A marked $\io$-category lying in the image of the nerve is called \wcnotion{strict}{strict marked $\io$-category}.
Remark eventually that the following square is cartesian
% q.uiver.app/#q=WzAsNCxbMCwwLCJcXHpvY2F0bSJdLFsxLDAsIlxcb2NhdG0iXSxbMSwxLCJcXG9jYXQiXSxbMCwxLCJcXHpvY2F0Il0sWzAsMSwiXFxOIl0sWzEsMiwiKFxcdXZhcileXFxuYXR1cmFsIl0sWzAsMywiKFxcdXZhcileXFxuYXR1cmFsIiwyXSxbMywyLCJcXE4iLDJdXQ==
\[\begin{tikzcd}
	\zocatm & \ocatm \\
	\zocat & \ocat
	\arrow["\N", from=1-1, to=1-2]
	\arrow["{(\uvar)^\natural}", from=1-2, to=2-2]
	\arrow["{(\uvar)^\natural}"', from=1-1, to=2-1]
	\arrow["\N"', from=2-1, to=2-2]
\end{tikzcd}\]
A marked $\io$-category is then strict if and only if it's underlying $\io$-category is.

\p The \wcsnotionsym{marked suspension}{((d60@$[\uvar,1]$}{suspension}{for marked $\io$-categories} is the colimit preserving functor $$[\uvar,1]:\ocatm\to \ocatm_{\bullet,\bullet}$$ sending $a^\flat$ onto $[a,1]^\flat$ and $(\Db_n)_t$ to $([\Db_n,1])_t$. 
It then admits a right adjoint: \ssym{(hom@$\hom_{\uvar}(\uvar,\uvar)$}{for marked $\io$-categories}
$$\begin{array}{lll}
\ocatm_{\bullet,\bullet}&\to& \ocatm\\
(C,a,b)&\mapsto &\hom_C(a,b)
\end{array}
$$

With the same computation than the one of paragraph \ref{para:wiskering}, we show that for a marked $\io$-category $C$, any $1$-cell $f:x\to x'$ induces for any object $y$, a morphism
$$f_!:\hom_C(x',y)\to \hom_C(x,y).$$
Conversely, a $1$-cell $g:y\to y'$ induces for any object $x$ a morphism
$$g_!:\hom_C(x,y)\to \hom_C(x,y')$$

\p In section \ref{section:iocategories}, we define the notion of fully faithful morphism of $\io$-categories. There is an equivalent notion for marked $\io$-categories:
\begin{definition}
A morphism $f:C\to D$ is \snotion{fully faithful}{for marked $\io$-categories} if for any pair of objects $x,y$, the morphism of marked $\io$-categories $\hom_C(x,y)\to \hom_D(fx,fy)$ is an equivalence, and if a $1$-cell $v$ is marked whenever $f(v)$ is.
\end{definition}

 We now give some adaptation of the result on fully faithful functors to the case of marked $\io$-categories without proofs, as they are obvious modifications to this new framework.

\begin{prop}
\label{prop:ff 1 marked case}
A morphism is fully faithful if and only if it has the unique right lifting property against $\emptyset\to \Db_n$ and $\Db_n\to (\Db_n)_t$ for $n>0$.
\end{prop}

\begin{prop}
\label{prop:ff 2 marked case}
Fully faithful morphisms are stable under limits.
\end{prop}

\begin{prop}
\label{prop:fully faithful plus surjective on objet marked case}
A morphism $f:C\to D$ is an equivalence if and only if it is fully faithful and surjective on objects.
\end{prop}

\p A morphism $f:C\to D$ between marked $\io$-categories is a \snotion{discrete Conduché functor}{for marked $\io$-categories} if for any triplet of integers $k< n\leq m$, $f$ has the unique right lifting property against $$\Ib_{m+1}:\Db_{m+1}^\flat\to \Db_{m}^\flat ~~\mbox{ and }~~\triangledown^{\sharp_n}_{k,m}:\Db_{m}^{\sharp_n}\to \Db_{m}^{\sharp_n}\coprod_{ \Db_{k}^\flat} \Db_{m}^{\sharp_n}.$$

\begin{example}
If $f$ is a discrete Conduché functor between marked $\io$-categories, $f^\sharp$ is a discrete Conduché functor. Conversely, if $g$ is a discrete Conduché functor between $\io$-categories, so are $g^\sharp$, $g^\flat$ and $g^{\sharp_n}$ for any integer $n$.
\end{example}


\p
A \notion{marked globular sum} is a marked $\io$-category whose underlying $\io$-category is a globular sum and such that for any pair of integers $k\leq n$, and any pair of $k$-composable $n$-cells $(x,y)$, $x\circ_k y$ is marked if and only if $x$ and $y$ are marked.



A morphism $i:a\to b$ between marked globular sum is \wcsnotion{globular}{globular morphism}{for marked $\zo$-categories} if the morphism $i^\natural$ is globular.


The proposition \ref{prop:algebraic ortho to globular} implies that a morphism $a\to b$ between marked globular sums is a discrete Conduché functor if and only if it is globular.

\begin{lemma}
\label{lemma:pullback by conduch marked preserves colimitpre}
Let $p:C\to D^\flat$ be a discrete Conduché functor between marked $\io$-categories. The canonical morphism $(C^\natural)^\flat\to C$ is an equivalence. 
\end{lemma}
\begin{proof}
Suppose given a marked $n$-cell $v:\Db_n\to C^\natural$. As the marking on $C$ is trivial, this induces a commutative square
% https://q.uiver.app/#q=WzAsNCxbMSwwLCIgQ15cXG5hdHVyYWwiXSxbMCwwLCJcXERiX24iXSxbMSwxLCJEIl0sWzAsMSwiXFxEYl97bi0xfSJdLFsxLDNdLFsxLDAsInYiXSxbMCwyLCJwXlxcbmF0dXJhbCJdLFszLDJdLFszLDAsImwiLDEseyJzdHlsZSI6eyJib2R5Ijp7Im5hbWUiOiJkYXNoZWQifX19XV0=
\[\begin{tikzcd}
	{\Db_n} & { C^\natural} \\
	{\Db_{n-1}} & D
	\arrow[from=1-1, to=2-1]
	\arrow["v", from=1-1, to=1-2]
	\arrow["{p^\natural}", from=1-2, to=2-2]
	\arrow[from=2-1, to=2-2]
	\arrow["l"{description}, dashed, from=2-1, to=1-2]
\end{tikzcd}\]
that admits a lift $l$ as $p^\natural$ is a discrete Conduché functor, which concludes the proof.
\end{proof}


\begin{prop}
\label{prop:pullback by conduch marked preserves colimit}
Let $p:C\to D$ be a discrete Conduché functor between marked $\io$-categories. The pullback functor $p^*$ preserves colimits.
\end{prop}
\begin{proof}
As $\tiPsh{\Theta}$ is locally cartesian closed, one has to show that for any pair of cartesian squares
% q.uiver.app/#q=WzAsNixbMSwxLCJEJyJdLFsyLDAsIkMiXSxbMiwxLCJEIl0sWzAsMSwiRCcnIl0sWzAsMCwiQycnIl0sWzEsMCwiQyciXSxbMSwyLCJwIl0sWzMsMCwiaSIsMl0sWzQsNSwiaiJdLFs1LDFdLFswLDJdLFs1LDBdLFs0LDNdLFs0LDAsIiIsMSx7InN0eWxlIjp7Im5hbWUiOiJjb3JuZXIifX1dLFs1LDIsIiIsMSx7InN0eWxlIjp7Im5hbWUiOiJjb3JuZXIifX1dXQ==
\[\begin{tikzcd}
	{C''} & {C'} & C \\
	{D''} & {D'} & D
	\arrow["p", from=1-3, to=2-3]
	\arrow["i"', from=2-1, to=2-2]
	\arrow["j", from=1-1, to=1-2]
	\arrow[from=1-2, to=1-3]
	\arrow[from=2-2, to=2-3]
	\arrow[from=1-2, to=2-2]
	\arrow[from=1-1, to=2-1]
	\arrow["\lrcorner"{anchor=center, pos=0.125}, draw=none, from=1-1, to=2-2]
	\arrow["\lrcorner"{anchor=center, pos=0.125}, draw=none, from=1-2, to=2-3]
\end{tikzcd}\]
if $i$ is $\Wm$, then $j$ is in $\widehat{\Wm}$. Suppose first that $i$ is in $\Wsat^\flat$. According of the lemma \ref{lemma:pullback by conduch marked preserves colimitpre} the $\io$-categories $C'$ and $C''$ are  of shape $(E)^\flat$ and $(E')^\flat$ for $E$ and $E'$ two $\io$-categories. The proposition \ref{prop:pulback of Wsat} then implies that $i$ is in $\widehat{\W^\flat}\subset \widehat{\Wm}$. If $i$ is in $(\Wseg)^{\sharp_n}$ the proof is an easy adaptation of the one of lemma \ref{lemma:conduche preserves W}.
\end{proof}





\p 
 We now give some adaptation of the result on special colimits stated in paragraph \ref{para: spetial colimits} to the case of marked $\io$-categories without proofs, as they are easy modifications.

We denote by $\iota$ the inclusion of $\ocatm$ into $\tiPsh{\Theta}$.
A functor $F:I\to \ocatm$ has a \snotion{special colimit}{for marked $\io$-categories} if the canonical morphism 
\begin{equation}
\label{eq:special colimit marked case}
\colim_{i:I}\iota F(i)\to \iota(\colim_{i:I}F(i))
\end{equation}
is an equivalence of stratified presheaves. 

Similarly, we say that a functor $\psi: I\to \Arr(\ocatm)$ has a \textit{special colimit} if the canonical morphism 
$$\colim_{i:I}\iota \psi(i)\to \iota(\colim_{i:I}\psi(i))$$
is an equivalence in the arrow $\iun$-category of $\tiPsh{\Theta}$.

\begin{example}
Let $C$ be a marked $\io$-category. The canonical diagram $t\Theta_{/C}\to \ocat$ has a special colimit, given by $C$.
\end{example}
\begin{prop}
\label{prop:special colimit marked case}
Let $F,G:I\to \ocatm$ be two functors, and $\psi:F\to G$ a natural transformation. If $\psi$ is cartesian, and $G$ has a special colimit, then $\psi$ and $F$ have special colimits. 
\end{prop}

\begin{prop}
\label{prop:example of a special colimit marked case}
For any integer $n$, and element $a\in t\Theta$ and $b\in \Theta$, the equalizer diagram 
% https://q.uiver.app/#q=WzAsMixbMCwwLCJcXGNvcHJvZF97aytsPW4tMX1bYSxrXVxcdmVlW2FcXHRpbWVzIGJeXFxzaGFycCwxXVxcdmVlW2EsbF0iXSxbMSwwLCJcXGNvcHJvZF97aytsPW59W2Esa11cXHZlZVsgYiwxXV5cXHNoYXJwXFx2ZWVbYSxsXSJdLFswLDEsIiIsMix7Im9mZnNldCI6LTJ9XSxbMCwxLCIiLDAseyJvZmZzZXQiOjJ9XV0=
\[\begin{tikzcd}
	{\coprod_{k+l=n-1}[a,k]\vee[a\times b^\sharp,1]\vee[a,l]} & {\coprod_{k+l=n}[a,k]\vee[ b,1]^\sharp\vee[a,l]}
	\arrow[shift left=2, from=1-1, to=1-2]
	\arrow[shift right=2, from=1-1, to=1-2]
\end{tikzcd}\]
where the top diagram is induced by $[a\times b^\sharp,1]\to [a,1]\vee[b,1]^\sharp$ and to bottom one by $[a\times b^\sharp,1]\to [b,1]^\sharp\vee[a,1]$,
has a special colimit, which is $[a,n]\times [b,1]^\sharp$.
\end{prop}



\begin{prop}
\label{prop:example of a special colimit 2 marked case}
Any sequence of marked $\io$-categories has a special colimit. 
\end{prop}


\begin{prop}
\label{prop:example of a special colimit4 marked case}
Suppose given a cartesian square
% https://q.uiver.app/#q=WzAsNCxbMCwwLCIgQiJdLFsxLDAsIkMiXSxbMSwxLCJbMV1eXFxzaGFycCJdLFswLDEsIlxcezBcXH0iXSxbMSwyXSxbMCwxXSxbMCwzXSxbMywyXSxbMCwyLCIiLDAseyJzdHlsZSI6eyJuYW1lIjoiY29ybmVyIn19XV0=
\[\begin{tikzcd}
	{ B} & C \\
	{\{0\}} & {[1]^\sharp}
	\arrow[from=1-2, to=2-2]
	\arrow[from=1-1, to=1-2]
	\arrow[from=1-1, to=2-1]
	\arrow[from=2-1, to=2-2]
	\arrow["\lrcorner"{anchor=center, pos=0.125}, draw=none, from=1-1, to=2-2]
\end{tikzcd}\]
The diagram 
% https://q.uiver.app/#q=WzAsMyxbMSwwLCJbQiwxXSJdLFswLDAsIlsxXV5cXHNoYXJwXFx2ZWVbQiwxXSJdLFsyLDAsIltDLDFdIl0sWzAsMSwiXFx0cmlhbmdsZWRvd24iLDJdLFswLDJdXQ==
\[\begin{tikzcd}
	{[1]^\sharp\vee[B,1]} & {[B,1]} & {[C,1]}
	\arrow["\triangledown"', from=1-2, to=1-1]
	\arrow[from=1-2, to=1-3]
\end{tikzcd}\]
has a special colimit.
\end{prop}

\begin{prop}
\label{prop:example of a special colimit3 marked case}
Suppose given two cartesian squares
% https://q.uiver.app/#q=WzAsNixbMCwwLCIgQiJdLFsxLDAsIkMiXSxbMiwwLCJEIl0sWzEsMSwiWzFdXlxcc2hhcnAiXSxbMCwxLCJcXHswXFx9Il0sWzIsMSwiXFx7MVxcfSJdLFsxLDNdLFswLDFdLFsyLDVdLFswLDRdLFsyLDFdLFs0LDNdLFs1LDNdLFswLDMsIiIsMCx7InN0eWxlIjp7Im5hbWUiOiJjb3JuZXIifX1dLFsyLDMsIiIsMCx7InN0eWxlIjp7Im5hbWUiOiJjb3JuZXIifX1dXQ==
\[\begin{tikzcd}
	{ B} & C & D \\
	{\{0\}} & {[1]^\sharp} & {\{1\}}
	\arrow[from=1-2, to=2-2]
	\arrow[from=1-1, to=1-2]
	\arrow[from=1-3, to=2-3]
	\arrow[from=1-1, to=2-1]
	\arrow[from=1-3, to=1-2]
	\arrow[from=2-1, to=2-2]
	\arrow[from=2-3, to=2-2]
	\arrow["\lrcorner"{anchor=center, pos=0.125}, draw=none, from=1-1, to=2-2]
	\arrow["\lrcorner"{anchor=center, pos=0.125, rotate=-90}, draw=none, from=1-3, to=2-2]
\end{tikzcd}\]
The diagram 
% https://q.uiver.app/#q=WzAsNSxbMSwwLCJbQiwxXSJdLFszLDAsIltELDFdIl0sWzAsMCwiWzFdXlxcc2hhcnBcXHZlZVtCLDFdIl0sWzQsMCwiW0QsMV1cXHZlZVsxXV5cXHNoYXJwIl0sWzIsMCwiW0MsMV0iXSxbMSwzLCJcXHRyaWFuZ2xlZG93biJdLFswLDIsIlxcdHJpYW5nbGVkb3duIiwyXSxbMCw0XSxbMSw0XV0=
\[\begin{tikzcd}
	{[1]^\sharp\vee[B,1]} & {[B,1]} & {[C,1]} & {[D,1]} & {[D,1]\vee[1]^\sharp}
	\arrow["\triangledown", from=1-4, to=1-5]
	\arrow["\triangledown"', from=1-2, to=1-1]
	\arrow[from=1-2, to=1-3]
	\arrow[from=1-4, to=1-3]
\end{tikzcd}\]
has a special colimit.
\end{prop}








\subsection{Gray tensor product of marked $\io$-categories}
We define the \wcsnotion{marked Gray tensor product}{Gray tensor product}{for marked $\io$-categories}
$$\uvar\otimes (\uvar)^\sharp:\ocatm\times \icat \to \ocatm$$
sending a marked $\io$-category $C$ and a $\iun$-category $K$ to the marked $\io$-category $C\otimes K^\sharp$, such that $(C\otimes K^\sharp)^\natural$ fits in the cocartesian square
% q.uiver.app/#q=WzAsNCxbMSwwLCJDXlxcbmF0dXJhbFxcb3RpbWVzIEsiXSxbMCwwLCJcXGNvcHJvZF97IHRDfVxcRGJfblxcb3RpbWVzIEsiXSxbMCwxLCJcXGNvcHJvZF97IHRDfVxcdGF1XmlfbihcXERiX25cXG90aW1lcyBLKSJdLFsxLDEsIihDXFxvdGltZXMgS15cXHNoYXJwKV5cXG5hdHVyYWwiXSxbMSwwXSxbMSwyXSxbMiwzXSxbMCwzXSxbMywxLCIiLDEseyJzdHlsZSI6eyJuYW1lIjoiY29ybmVyIn19XV0=
\[\begin{tikzcd}
	{\coprod_{ tC}\Db_n\otimes K} & {C^\natural\otimes K} \\
	{\coprod_{ tC}\tau^i_n(\Db_n\otimes K)} & {(C\otimes K^\sharp)^\natural}
	\arrow[from=1-1, to=1-2]
	\arrow[from=1-1, to=2-1]
	\arrow[from=2-1, to=2-2]
	\arrow[from=1-2, to=2-2]
	\arrow["\lrcorner"{anchor=center, pos=0.125, rotate=180}, draw=none, from=2-2, to=1-1]
\end{tikzcd}\]
and such that $t(C\otimes K^\sharp)_n$ consists of $n$-cells lying in the image of the morphism 
$$\tau_{n-1}C\otimes K\coprod (tC)_n\otimes K_0\to (C\otimes K^\sharp)^\natural.$$

\begin{prop}
\label{prop:otimes marked preserves colimits}
The functor $\uvar\otimes(\uvar)^\sharp:\ocatm\times \icat\to \ocatm$ preserves colimits. 
\end{prop}
\begin{proof}
By construction, we have two cocartesian squares:
% https://q.uiver.app/#q=WzAsOCxbMCwyLCJcXGNvcHJvZF97dChcXGNvbGltIEYpfVxcRGJfblxcb3RpbWVzIEsiXSxbMCwzLCJcXGNvcHJvZF97IHQoXFxjb2xpbSBGKX1cXHRhdV5pX3tufShcXERiX25cXG90aW1lcyBLKSJdLFswLDAsIlxcY29wcm9kX3tcXGNvbGltIHRGIH1cXERiX25cXG90aW1lcyBLIl0sWzAsMSwiXFxjb3Byb2Rfe1xcY29saW0gIHRGIH1cXHRhdV5pX3tufShcXERiX25cXG90aW1lcyBLKSJdLFsxLDAsIlxcY29saW0gKEZeXFxuYXR1cmFsXFxvdGltZXMgSykiXSxbMSwxLCJcXGNvbGltIChGXFxvdGltZXMgS15cXHNoYXJwKV5cXG5hdHVyYWwiXSxbMSwyLCIoXFxjb2xpbSBGXlxcbmF0dXJhbClcXG90aW1lcyBLIl0sWzEsMywiKChcXGNvbGltIEYpXFxvdGltZXMgS15cXHNoYXJwKV5cXG5hdHVyYWwiXSxbMiwzXSxbMCwxXSxbNCw1XSxbNiw3XSxbMCw2XSxbMiw0XSxbMyw1XSxbMSw3XSxbNSwxMywiIiwxLHsibGV2ZWwiOjEsInN0eWxlIjp7Im5hbWUiOiJjb3JuZXIifX1dLFs3LDEyLCIiLDEseyJsZXZlbCI6MSwic3R5bGUiOnsibmFtZSI6ImNvcm5lciJ9fV1d
\[\begin{tikzcd}
	{\coprod_{\colim tF }\Db_n\otimes K} & {\colim (F^\natural\otimes K)} \\
	{\coprod_{\colim tF }\tau^i_{n}(\Db_n\otimes K)} & {\colim (F\otimes K^\sharp)^\natural} \\
	{\coprod_{t(\colim F)}\Db_n\otimes K} & {(\colim F^\natural)\otimes K} \\
	{\coprod_{ t(\colim F)}\tau^i_{n}(\Db_n\otimes K)} & {((\colim F)\otimes K^\sharp)^\natural}
	\arrow[from=1-1, to=2-1]
	\arrow[from=3-1, to=4-1]
	\arrow[from=1-2, to=2-2]
	\arrow[from=3-2, to=4-2]
	\arrow[""{name=0, anchor=center, inner sep=0}, from=3-1, to=3-2]
	\arrow[""{name=1, anchor=center, inner sep=0}, from=1-1, to=1-2]
	\arrow[from=2-1, to=2-2]
	\arrow[from=4-1, to=4-2]
	\arrow["\lrcorner"{anchor=center, pos=0.125, rotate=180}, draw=none, from=2-2, to=1]
	\arrow["\lrcorner"{anchor=center, pos=0.125, rotate=180}, draw=none, from=4-2, to=0]
\end{tikzcd}\]
By the preservation of colimit by the Gray tensor product for $\io$-categories and by the functor $(\uvar)^\natural$, we have an equivalence 
$$\colim (F^\natural\otimes K)\sim (\colim F^\natural)\otimes K$$
However, the canonical morphism $\colim tF \to t(\colim F)$ is an epimorphism, and according to proposition \ref{prop:truncation of epimorphism is pushout}, the following canonical square
is cocartesian
% q.uiver.app/#q=WzAsNCxbMSwwLCJcXGNvcHJvZF97dChcXGNvbGltIEYpfVxcRGJfblxcb3RpbWVzIEsiXSxbMSwxLCJcXGNvcHJvZF97IHQoXFxjb2xpbSBGKX1cXHRhdV5pX3tufShcXERiX25cXG90aW1lcyBLKSJdLFswLDAsIlxcY29wcm9kX3tcXGNvbGltIHRGIH1cXERiX25cXG90aW1lcyBLIl0sWzAsMSwiXFxjb3Byb2Rfe1xcY29saW0gIHRGIH1cXHRhdV5pX3tufShcXERiX25cXG90aW1lcyBLKSJdLFsyLDNdLFsyLDBdLFszLDFdLFswLDFdXQ==
\[\begin{tikzcd}
	{\coprod_{\colim tF }\Db_n\otimes K} & {\coprod_{t(\colim F)}\Db_n\otimes K} \\
	{\coprod_{\colim tF }\tau^i_{n}(\Db_n\otimes K)} & {\coprod_{ t(\colim F)}\tau^i_{n}(\Db_n\otimes K)}
	\arrow[from=1-1, to=2-1]
	\arrow[from=1-1, to=1-2]
	\arrow[from=2-1, to=2-2]
	\arrow[from=1-2, to=2-2]
\end{tikzcd}\]
Combined with the first two cocartesian squares, this implies that that $\colim (F\otimes K^\sharp)^\natural$ and $((\colim F)\otimes K^\sharp)^\natural$ are equivalent.


According to proposition \ref{prop:intelignet truncation is poitwise an epi} and by construction, the morphisms
 $$\colim (\tau^i_n F^\natural\otimes K)\to \tau^i_n(\colim F^\natural\otimes K)~~\mbox{ and }~~\colim (tF\otimes K_0) \to t(\colim F\otimes K_0)$$ are epimorphisms. The marked $\io$-categories $\colim (F\otimes K^\sharp)$ and $(\colim F)\otimes K^\sharp$
 then have the same marked cells.
\end{proof}




\begin{prop}
\label{prop:associativity of Gray amput}
Let $C$ be a $\io$-category, $D$ a marked $\io$-category and $K,L$ two $(\infty,1)$-categories. 
\begin{enumerate}
\item The underlying $\io$-category of $C^\flat\otimes K^\sharp$ is $C\otimes K$.
\item The canonical morphism $C^\sharp\otimes K^\sharp\to C^\sharp\times K^\sharp$ is an equivalence.
\end{enumerate}
\end{prop}
\begin{proof}
The first assertion is obvious. 


Let $a$ be a globular sum and $[k]$ an object of $\Delta$.
We claim that the following two squares are cocartesian:
% https://q.uiver.app/#q=WzAsNixbMCwwLCJcXGNvcHJvZF9uXFxjb3Byb2RcXGxpbWl0c197XFxEYl9uXFx0byAgYX1cXERiX25cXG90aW1lcyBba10iXSxbMCwxLCJcXGNvcHJvZF9uXFxjb3Byb2RcXGxpbWl0c197XFxEYl9uXFx0byAgYX1cXHRhdV5pX3tufShcXERiX25cXG90aW1lcyBba10pIl0sWzEsMCwiXFxjb3Byb2Rfe259XFx0YXVfbiBhXFxvdGltZXMgW2tdIl0sWzEsMSwiXFxjb3Byb2Rfe259XFx0YXVeaV97bn0oXFx0YXVfbiBhXFxvdGltZXMgW2tdKSJdLFsyLDAsImFcXG90aW1lc1trXSJdLFsyLDEsIihhXlxcc2hhcnBcXHRpbWVzIFtrXV5cXHNoYXJwKV5cXG5hdHVyYWwiXSxbMCwxXSxbMiwzXSxbMSwzXSxbMCwyXSxbNCw1XSxbMyw1XSxbMiw0XV0=
\[\begin{tikzcd}
	{\coprod_n\coprod\limits_{\Db_n\to a}\Db_n\otimes [k]} & {\coprod_{n}\tau_n a\otimes [k]} & {a\otimes[k]} \\
	{\coprod_n\coprod\limits_{\Db_n\to a}\tau^i_{n}(\Db_n\otimes [k])} & {\coprod_{n}\tau^i_{n}(\tau_n a\otimes [k])} & {(a^\sharp\times [k]^\sharp)^\natural}
	\arrow[from=1-1, to=2-1]
	\arrow[from=1-2, to=2-2]
	\arrow[from=2-1, to=2-2]
	\arrow[from=1-1, to=1-2]
	\arrow[from=1-3, to=2-3]
	\arrow[from=2-2, to=2-3]
	\arrow[from=1-2, to=1-3]
\end{tikzcd}\]
The cocartesianess of the left square is a consequence of propositions \ref{prop:truncation of epimorphism is pushout} and \ref{prop:canonical epi}. The outer square is cocartesian by definition, and by left cancellation, this implies the cocartesianess of the right square.
The lemma \ref{lemma:technique marked oicategoros} then implies that the underlying category of $a^\sharp\otimes [k]^\sharp$ is $a\times[k]$. As every cell of $a^\sharp\otimes[k]^\sharp$ is marked, this concludes the proof of the second assertion.
\end{proof}


\begin{prop}
\label{prop:associativity of Gray2}
Let $D$ be an $\io$-category, $C$ a marked $\io$-category and $K$ an $\iun$-category.
The canonical morphism
$(D^\sharp\times C)\otimes K^\sharp\to D^\sharp\times (C\otimes K^\sharp)$ is an equivalence.
\end{prop}
\begin{proof}
As $\times$ and $\otimes$ preserve colimits, we can reduce to the case where $D$ is an element of $\Theta$, $C$ of $t\Theta$ and $K$ of $\Delta$, and we proceed by induction on the dimension of $D$. Remark first that if $D$ is $[0]$, the result is obvious, and if it is $(\Db_1)_t$, the result follows from the second assertion of proposition \ref{prop:associativity of Gray amput}. Suppose then the result is true at the stage $n$. Using once again the fact that $\times$ and $\otimes$ preserve colimits, we can reduce to the case where
 $D^\sharp$ is $[a,1]^\sharp$, $C$ is $[b,1]$ with $b$ an element of $\Theta_t$ of dimension $n$, and $K^\sharp$ is $[1]^\sharp$.
 
 The formula given in proposition \ref{prop:example of a special colimit marked case} implies that $([a,1]^\sharp\times[b,1])\otimes[1]^\sharp$ is the colimit of the sequence: 
 % https://q.uiver.app/#q=WzAsMyxbMCwwLCIoW2EsMV1eXFxzaGFycFxcdmVlIFtiLDFdKVxcb3RpbWVzWzFdXlxcc2hhcnAiXSxbMSwwLCJbYV5cXHNoYXJwXFx0aW1lcyBiLDFdXFxvdGltZXNbMV1eXFxzaGFycCJdLFsyLDAsIihbYiwxXVxcdmVlIFthLDFdXlxcc2hhcnApXFxvdGltZXNbMV1eXFxzaGFycCJdLFsxLDBdLFsxLDJdXQ==
\begin{equation}
\label{eq:prop:associativity of Gray2}
\begin{tikzcd}
	{([a,1]^\sharp\vee [b,1])\otimes[1]^\sharp} & {[a^\sharp\times b,1]\otimes[1]^\sharp} & {([b,1]\vee [a,1]^\sharp)\otimes[1]^\sharp}
	\arrow[from=1-2, to=1-1]
	\arrow[from=1-2, to=1-3]
\end{tikzcd}
\end{equation}
The marked $\io$-category $([a,1]^\sharp\vee [b,1])\otimes[1]^\sharp$ is then the colimit of the diagram 
% https://q.uiver.app/#q=WzAsMyxbMCwwLCJbYSwxXV5cXHNoYXJwXFx0aW1lcyBbMV1eXFxzaGFycCJdLFsxLDAsIlsxXV5cXHNoYXJwIl0sWzIsMCwiW2IsMV1cXG90aW1lc1sxXV5cXHNoYXJwIl0sWzEsMF0sWzEsMl1d
\[\begin{tikzcd}
	{[a,1]^\sharp\times [1]^\sharp} & {[1]^\sharp} & {[b,1]\otimes[1]^\sharp}
	\arrow[from=1-2, to=1-1]
	\arrow[from=1-2, to=1-3]
\end{tikzcd}\]
and using the formulas \eqref{eq:eq for cylinder marked version} and \ref{prop:example of a special colimit marked case}, $([a,1]^\sharp\vee [b,1])\otimes[1]^\sharp$ is the colimit of the diagram
% https://q.uiver.app/#q=WzAsNyxbMiw0LCJbYSwxXV5cXHNoYXJwXFx2ZWVbYiwxXVxcdmVlWzFdXlxcc2hhcnAiXSxbMiwyLCJbYSwxXV5cXHNoYXJwXFx2ZWVbYlxcb3RpbWVzWzFdXlxcc2hhcnAsMV0iXSxbMiwwLCJbYSwxXV5cXHNoYXJwXFx2ZWVbMV1eXFxzaGFycFxcdmVlW2IsMV0iXSxbMiwzLCJbYSwxXV5cXHNoYXJwXFx2ZWVbYlxcb3RpbWVzXFx7MVxcfSwxXSJdLFsyLDEsIlthLDFdXlxcc2hhcnBcXHZlZVtiXFxvdGltZXNcXHswXFx9LDFdIl0sWzEsMCwiW2EsMV1eXFxzaGFycFxcdmVlW2IsMV0iXSxbMCwwLCJbMV1eXFxzaGFycFxcdmVlW2EsMV1eXFxzaGFycFxcdmVlW2IsMV0iXSxbMywwXSxbMywxXSxbNCwxXSxbNCwyXSxbNSwyXSxbNSw2XV0=
\[\begin{tikzcd}
	{[1]^\sharp\vee[a,1]^\sharp\vee[b,1]} & {[a,1]^\sharp\vee[b,1]} & {[a,1]^\sharp\vee[1]^\sharp\vee[b,1]} \\
	&& {[a,1]^\sharp\vee[b\otimes\{0\},1]} \\
	&& {[a,1]^\sharp\vee[b\otimes[1]^\sharp,1]} \\
	&& {[a,1]^\sharp\vee[b\otimes\{1\},1]} \\
	&& {[a,1]^\sharp\vee[b,1]\vee[1]^\sharp}
	\arrow[from=4-3, to=5-3]
	\arrow[from=4-3, to=3-3]
	\arrow[from=2-3, to=3-3]
	\arrow[from=2-3, to=1-3]
	\arrow[from=1-2, to=1-3]
	\arrow[from=1-2, to=1-1]
\end{tikzcd}\]
Similarly, $([b,1]\vee [a,1]^\sharp)\otimes[1]^\sharp$ is the colimit of the diagram
% https://q.uiver.app/#q=WzAsNyxbMCwwLCJbMV1eXFxzaGFycFxcdmVlW2IsMV1cXHZlZVthLDFdXlxcc2hhcnAiXSxbMCwxLCJbYlxcb3RpbWVzXFx7MFxcfSwxXVxcdmVlW2EsMV1eXFxzaGFycCJdLFswLDIsIltiXFxvdGltZXNbMV1eXFxzaGFycCwxXVxcdmVlW2EsMV1eXFxzaGFycCJdLFswLDMsIltiXFxvdGltZXNcXHsxXFx9LDFdXFx2ZWVbYSwxXV5cXHNoYXJwIl0sWzAsNCwiW2IsMV1cXHZlZVsxXV5cXHNoYXJwXFx2ZWVbYSwxXV5cXHNoYXJwIl0sWzEsNCwiW2IsMV1cXHZlZVthLDFdIl0sWzIsNCwiW2IsMV1cXHZlZVthLDFdXlxcc2hhcnBcXHZlZVsxXV5cXHNoYXJwIl0sWzEsMF0sWzEsMl0sWzMsMl0sWzMsNF0sWzUsNF0sWzUsNl1d
\[\begin{tikzcd}
	{[1]^\sharp\vee[b,1]\vee[a,1]^\sharp} \\
	{[b\otimes\{0\},1]\vee[a,1]^\sharp} \\
	{[b\otimes[1]^\sharp,1]\vee[a,1]^\sharp} \\
	{[b\otimes\{1\},1]\vee[a,1]^\sharp} \\
	{[b,1]\vee[1]^\sharp\vee[a,1]^\sharp} & {[b,1]\vee[a,1]} & {[b,1]\vee[a,1]^\sharp\vee[1]^\sharp}
	\arrow[from=2-1, to=1-1]
	\arrow[from=2-1, to=3-1]
	\arrow[from=4-1, to=3-1]
	\arrow[from=4-1, to=5-1]
	\arrow[from=5-2, to=5-1]
	\arrow[from=5-2, to=5-3]
\end{tikzcd}\]
Eventually, the formulas \eqref{eq:eq for cylinder marked version} and the induction hypothesis imply that $[a^\sharp\times b,1]\otimes[1]^\sharp$ is the colimit of the diagram
% https://q.uiver.app/#q=WzAsNSxbMCwwLCJbMV1eXFxzaGFycFxcdmVlW2FeXFxzaGFycFxcdGltZXMgYiwxXSJdLFsxLDEsIlthXlxcc2hhcnBcXHRpbWVzIGJcXG90aW1lc1xcezBcXH0sMV0iXSxbMSwyLCJbYV5cXHNoYXJwXFx0aW1lcyAoYlxcb3RpbWVzWzFdXlxcc2hhcnApLDFdIl0sWzEsMywiW2FeXFxzaGFycFxcdGltZXMgYlxcb3RpbWVzXFx7MVxcfSwxXSJdLFsyLDQsIlthXlxcc2hhcnBcXHRpbWVzIGIsMV1cXHZlZVsxXV5cXHNoYXJwIl0sWzEsMF0sWzEsMl0sWzMsMl0sWzMsNF1d
\[\begin{tikzcd}
	{[1]^\sharp\vee[a^\sharp\times b,1]} \\
	& {[a^\sharp\times b\otimes\{0\},1]} \\
	& {[a^\sharp\times (b\otimes[1]^\sharp),1]} \\
	& {[a^\sharp\times b\otimes\{1\},1]} \\
	&& {[a^\sharp\times b,1]\vee[1]^\sharp}
	\arrow[from=2-2, to=1-1]
	\arrow[from=2-2, to=3-2]
	\arrow[from=4-2, to=3-2]
	\arrow[from=4-2, to=5-3]
\end{tikzcd}\]
As all these colimits are special and composed of monomorphisms, the objects $([a,1]^\sharp\vee [b,1])\otimes[1]^\sharp$, $([b,1]^\sharp\vee [a,1]^\sharp)\otimes[1]^\sharp$ and $[a^\sharp\times b,1]\otimes[1]^\sharp$ are strict. As the colimit \eqref{eq:prop:associativity of Gray2} is also special, $([a,1]^\sharp\times[b,1])\otimes[1]^\sharp$ is strict.

All put together, $([a,1]^\sharp\times[b,1])\otimes[1]^\sharp$ is the colimit of the diagram
% https://q.uiver.app/#q=WzAsMTksWzQsNCwiW2EsMV1eXFxzaGFycFxcdmVlW2IsMV1cXHZlZVsxXV5cXHNoYXJwIl0sWzQsMiwiW2EsMV1eXFxzaGFycFxcdmVlW2JcXG90aW1lc1sxXV5cXHNoYXJwLDFdIl0sWzQsMCwiW2EsMV1eXFxzaGFycFxcdmVlWzFdXlxcc2hhcnBcXHZlZVtiLDFdIl0sWzQsMywiW2EsMV1eXFxzaGFycFxcdmVlW2JcXG90aW1lc1xcezFcXH0sMV0iXSxbNCwxLCJbYSwxXV5cXHNoYXJwXFx2ZWVbYlxcb3RpbWVzXFx7MFxcfSwxXSJdLFsyLDQsIltiLDFdXFx2ZWVbYSwxXV5cXHNoYXJwXFx2ZWVbMV1eXFxzaGFycCJdLFszLDQsIlthXlxcc2hhcnBcXHRpbWVzIGIsMV1cXHZlZVsxXV5cXHNoYXJwIl0sWzIsMywiW2FeXFxzaGFycFxcdGltZXMgYlxcb3RpbWVzXFx7MVxcfSwxXSJdLFsyLDIsIlthXlxcc2hhcnBcXHRpbWVzIChiXFxvdGltZXNbMV1eXFxzaGFycCksMV0iXSxbMiwxLCJbYV5cXHNoYXJwXFx0aW1lcyBiXFxvdGltZXNcXHswXFx9LDFdIl0sWzMsMCwiW2EsMV1eXFxzaGFycFxcdmVlW2IsMV0iXSxbMiwwLCJbMV1eXFxzaGFycFxcdmVlW2EsMV1eXFxzaGFycFxcdmVlW2IsMV0iXSxbMSw0LCJbYiwxXVxcdmVlW2EsMV0iXSxbMCwzLCJbYlxcb3RpbWVzXFx7MVxcfSwxXVxcdmVlW2EsMV1eXFxzaGFycCJdLFswLDIsIltiXFxvdGltZXNbMV1eXFxzaGFycCwxXVxcdmVlW2EsMV1eXFxzaGFycCJdLFswLDEsIltiXFxvdGltZXNcXHswXFx9LDFdXFx2ZWVbYSwxXV5cXHNoYXJwIl0sWzEsMCwiWzFdXlxcc2hhcnBcXHZlZVthXlxcc2hhcnBcXHRpbWVzIGIsMV0iXSxbMCwwLCJbMV1eXFxzaGFycFxcdmVlW2IsMV1cXHZlZVthLDFdXlxcc2hhcnAiXSxbMCw0LCJbYiwxXVxcdmVlWzFdXlxcc2hhcnBcXHZlZVthLDFdXlxcc2hhcnAiXSxbMywwXSxbMywxXSxbNCwxXSxbNCwyXSxbNiwwXSxbNiw1XSxbOSw0XSxbOSwxMF0sWzEwLDJdLFs3LDZdLFs3LDhdLFs5LDhdLFs4LDFdLFs3LDNdLFsxMCwxMV0sWzksMTVdLFs4LDE0XSxbNywxM10sWzEzLDE0XSxbMTUsMTRdLFsxNiwxMV0sWzEyLDVdLFsxMiwxOF0sWzE2LDE3XSxbMTMsMThdLFsxNSwxN10sWzksMTZdLFs3LDEyXV0=
\[\begin{tikzcd}[column sep =0.1cm]
	{[1]^\sharp\vee[b,1]\vee[a,1]^\sharp} & {[1]^\sharp\vee[a^\sharp\times b,1]} & {[1]^\sharp\vee[a,1]^\sharp\vee[b,1]} & {[a,1]^\sharp\vee[b,1]} & {[a,1]^\sharp\vee[1]^\sharp\vee[b,1]} \\
	{[b\otimes\{0\},1]\vee[a,1]^\sharp} && {[a^\sharp\times b\otimes\{0\},1]} && {[a,1]^\sharp\vee[b\otimes\{0\},1]} \\
	{[b\otimes[1]^\sharp,1]\vee[a,1]^\sharp} && {[a^\sharp\times (b\otimes[1]^\sharp),1]} && {[a,1]^\sharp\vee[b\otimes[1]^\sharp,1]} \\
	{[b\otimes\{1\},1]\vee[a,1]^\sharp} && {[a^\sharp\times b\otimes\{1\},1]} && {[a,1]^\sharp\vee[b\otimes\{1\},1]} \\
	{[b,1]\vee[1]^\sharp\vee[a,1]^\sharp} & {[b,1]\vee[a,1]} & {[b,1]\vee[a,1]^\sharp\vee[1]^\sharp} & {[a^\sharp\times b,1]\vee[1]^\sharp} & {[a,1]^\sharp\vee[b,1]\vee[1]^\sharp}
	\arrow[from=4-5, to=5-5]
	\arrow[from=4-5, to=3-5]
	\arrow[from=2-5, to=3-5]
	\arrow[from=2-5, to=1-5]
	\arrow[from=5-4, to=5-5]
	\arrow[from=5-4, to=5-3]
	\arrow[from=2-3, to=2-5]
	\arrow[from=2-3, to=1-4]
	\arrow[from=1-4, to=1-5]
	\arrow[from=4-3, to=5-4]
	\arrow[from=4-3, to=3-3]
	\arrow[from=2-3, to=3-3]
	\arrow[from=3-3, to=3-5]
	\arrow[from=4-3, to=4-5]
	\arrow[from=1-4, to=1-3]
	\arrow[from=2-3, to=2-1]
	\arrow[from=3-3, to=3-1]
	\arrow[from=4-3, to=4-1]
	\arrow[from=4-1, to=3-1]
	\arrow[from=2-1, to=3-1]
	\arrow[from=1-2, to=1-3]
	\arrow[from=5-2, to=5-3]
	\arrow[from=5-2, to=5-1]
	\arrow[from=1-2, to=1-1]
	\arrow[from=4-1, to=5-1]
	\arrow[from=2-1, to=1-1]
	\arrow[from=2-3, to=1-2]
	\arrow[from=4-3, to=5-2]
\end{tikzcd}\]

Now, using the formula given in proposition \ref{prop:example of a special colimit marked case}, and taking the colimit line by line of the previous diagram, $([a,1]^\sharp\times[b,1])\otimes[1]^\sharp$ is the colimit of the diagram 
% https://q.uiver.app/#q=WzAsNSxbMCwwLCJbYSwxXV5cXHNoYXJwXFx0aW1lcyhbMV1eXFxzaGFycFxcdmVlW2IsMV0pIl0sWzAsMSwiW2EsMV1eXFxzaGFycFxcdGltZXNbYlxcb3RpbWVzXFx7MFxcfSwxXSJdLFswLDIsIlthLDFdXlxcc2hhcnBcXHRpbWVzW2JcXG90aW1lc1sxXV5cXHNoYXJwLDFdIl0sWzAsMywiW2EsMV1eXFxzaGFycFxcdGltZXNbYlxcb3RpbWVzXFx7MVxcfSwxXSJdLFswLDQsIlthLDFdXlxcc2hhcnBcXHRpbWVzKFtiLDFdXFx2ZWVbMV1eXFxzaGFycCkiXSxbMSwwXSxbMSwyXSxbMywyXSxbMyw0XV0=
\[\begin{tikzcd}
	{[a,1]^\sharp\times([1]^\sharp\vee[b,1])} \\
	{[a,1]^\sharp\times[b\otimes\{0\},1]} \\
	{[a,1]^\sharp\times[b\otimes[1]^\sharp,1]} \\
	{[a,1]^\sharp\times[b\otimes\{1\},1]} \\
	{[a,1]^\sharp\times([b,1]\vee[1]^\sharp)}
	\arrow[from=2-1, to=1-1]
	\arrow[from=2-1, to=3-1]
	\arrow[from=4-1, to=3-1]
	\arrow[from=4-1, to=5-1]
\end{tikzcd}\]
Using for the last times formula \eqref{eq:eq for cylinder marked version}, $([a,1]^\sharp\times[b,1])\otimes[1]^\sharp$ is equivalent to $[a,1]^\sharp\times([b,1]\otimes[1]^\sharp)$.
\end{proof}

\begin{prop}
 \label{prop:associativity of Gray amput2}
Let $D$ be a marked $\io$-category and $K,L$ two $(\infty,1)$-categories. 
There is a natural equivalence
$(D\otimes K^\sharp)\otimes L^\sharp\to D\otimes(K\times L)^\sharp$.
\end{prop}
\begin{proof}
Suppose first that $D$ is of shape $C^\flat$.
The proposition \ref{prop:canonical epi} implies that $\coprod_{t(C^\flat\otimes K^\sharp)^\natural}\Db_n\to (C^\flat\otimes K^\sharp)^\natural$ and $(\coprod_n \tau_{n-1}C\otimes K) \to (C^\flat\otimes K^\sharp)^\natural$ have the same image. The proposition \ref{prop:truncation of epimorphism is pushout}, then implies that
the underlying $\io$-category of $(C^\flat\otimes K^\sharp)\otimes L^\sharp$ fits in the cocartesian square
% https://q.uiver.app/#q=WzAsNCxbMCwwLCJcXGNvcHJvZF9uIFxcdGF1X3tuLTF9Q1xcb3RpbWVzIEtcXG90aW1lcyBMIl0sWzEsMCwiQ1xcb3RpbWVzIEtcXG90aW1lcyBMIl0sWzAsMSwiXFxjb3Byb2RfbiBcXHRhdV5pX24oXFx0YXVfe24tMX1DXFxvdGltZXMgS1xcb3RpbWVzIEwpIl0sWzEsMSwiKChDXlxcZmxhdFxcb3RpbWVzIEteXFxzaGFycClcXG90aW1lcyBMXlxcc2hhcnApXlxcbmF0dXJhbCJdLFswLDJdLFswLDFdLFsxLDNdLFsyLDNdXQ==
\[\begin{tikzcd}
	{\coprod_n \tau_{n-1}C\otimes K\otimes L} & {C\otimes K\otimes L} \\
	{\coprod_n \tau^i_n(\tau_{n-1}C\otimes K\otimes L)} & {((C^\flat\otimes K^\sharp)\otimes L^\sharp)^\natural}
	\arrow[from=1-1, to=2-1]
	\arrow[from=1-1, to=1-2]
	\arrow[from=1-2, to=2-2]
	\arrow[from=2-1, to=2-2]
\end{tikzcd}\]
The second assertion of lemma \ref{lemma:technique marked oicategoros} then implies that $((C^\flat\otimes K^\sharp)\otimes L^\sharp)^\natural$ is equivalent to $C^\flat\otimes (K\times L)$.
For a general marked $\io$-category $D$, the underlying $\io$-category of $(D^\flat\otimes K^\sharp)\otimes L^\sharp$ then fits by construction in the cocartesian square
% https://q.uiver.app/#q=WzAsNCxbMSwwLCJEXlxcZmxhdFxcb3RpbWVzIChLXFx0aW1lcyBMKSJdLFswLDAsIlxcY29wcm9kX25cXGNvcHJvZF97dERfbn1cXERiX25cXG90aW1lcyBLXzBcXG90aW1lcyBMXFxhbWFsZ1xcRGJfblxcb3RpbWVzIEtcXG90aW1lcyBMIl0sWzAsMSwiXFxjb3Byb2RfblxcY29wcm9kX3t0RF9ufVxcdGF1XmlfbihcXERiX25cXG90aW1lcyBLXzBcXG90aW1lcyBMKVxcYW1hbGdcXHRhdV5pX24oXFxEYl9uXFxvdGltZXMgSylcXG90aW1lcyBMIl0sWzEsMSwiKChEXFxvdGltZXMgS15cXHNoYXJwKVxcb3RpbWVzIExeXFxzaGFycCleXFxuYXR1cmFsIl0sWzEsMF0sWzEsMl0sWzIsM10sWzAsM11d
\[\begin{tikzcd}
	{\coprod_n\coprod_{tD_n}\Db_n\otimes K_0\otimes L\amalg\Db_n\otimes K\otimes L} & {D^\flat\otimes (K\times L)} \\
	{\coprod_n\coprod_{tD_n}\tau^i_n(\Db_n\otimes K_0\otimes L)\amalg\tau^i_n(\Db_n\otimes K)\otimes L} & {((D\otimes K^\sharp)\otimes L^\sharp)^\natural}
	\arrow[from=1-1, to=1-2]
	\arrow[from=1-1, to=2-1]
	\arrow[from=2-1, to=2-2]
	\arrow[from=1-2, to=2-2]
\end{tikzcd}\]

Furthermore, the underlying $\io$-category of $D\otimes(K\times L)^\sharp$ fits in the cocartesian square
% https://q.uiver.app/#q=WzAsNCxbMCwwLCJcXGNvcHJvZF9uXFxjb3Byb2Rfe3RDX259XFxEYl9uXFxvdGltZXMgKEtcXHRpbWVzIEwpIl0sWzAsMSwiXFxjb3Byb2RfblxcY29wcm9kX3t0Q19ufVxcdGF1XmlfbihcXERiX25cXG90aW1lcyAoS1xcdGltZXMgTCkpIl0sWzEsMCwiRF5cXGZsYXRcXG90aW1lcyAoS1xcdGltZXMgTCkiXSxbMSwxLCIoRFxcb3RpbWVzKEtcXHRpbWVzIEwpXlxcc2hhcnApXlxcbmF0dXJhbCJdLFswLDJdLFsxLDNdLFsyLDNdLFswLDFdXQ==
\[\begin{tikzcd}
	{\coprod_n\coprod_{tC_n}\Db_n\otimes (K\times L)} & {D^\flat\otimes (K\times L)} \\
	{\coprod_n\coprod_{tC_n}\tau^i_n(\Db_n\otimes (K\times L))} & {(D\otimes(K\times L)^\sharp)^\natural}
	\arrow[from=1-1, to=1-2]
	\arrow[from=2-1, to=2-2]
	\arrow[from=1-2, to=2-2]
	\arrow[from=1-1, to=2-1]
\end{tikzcd}\]
Using the canonical morphism $\tau^i_n(\Db_n\otimes K)\otimes L\to  \tau^i_n (\Db_n\otimes K \otimes L)\to \tau^i_n(\Db_n\otimes (K\times L))$, we have a canonical morphism 
$$((D^\flat\otimes K^\sharp)\otimes L^\sharp)^\natural \to (D\otimes(K\times L)^\sharp)^\natural.$$
As all these functors preserves colimits, the full sub $\infty$-groupoid of elements $(D,K,L)$ of $\ocatm\times \icat\times \icat$ such that this comparison is an equivalence and preserves and detects marking is closed by colimits. It is then sufficient to show that it includes $([1]^\sharp,[1],[1])$ and $([a,1],[1],[1])$ for $a\in t\Theta$.
We can then proceed as in the proof of proposition \ref{prop:associativity of Gray2}, making these two objects explicit thanks to the equations given in paragraph \ref{paragrap: equation fullfill by cylinder and join marked version}. As the proof takes up a lot of space and is very similar to that of proposition \textit{op cit}, we leave it to the reader.
\end{proof}


\subsection{Gray operations on marked $\io$-categories}
\p
The Gray tensor product for marked $\io$-category restricts to a functor
$$\uvar\otimes[1]^\sharp:\ocatm \to\ocatm$$ called the \wcsnotion{marked Gray cylinder}{Gray cylinder}{for marked $\io$-categories}\sym{((d30@$\uvar\otimes[1]^\sharp$}.
We will denote by 
$$\begin{array}{rcl}
\ocatm&\to&\ocatm\\
C&\mapsto &C^{[1]^\sharp}
\end{array}$$
its right adjoint.\sym{(c@$C^{[1]^\sharp}$}
The equation \eqref{eq:liens entre Gray cylindre et suspension}, establishing a link between the suspension and the Gray cylinder implies that the following diagram is cocartesian for any $C:\ocat$:
\begin{equation}
\label{eq:liens entre Gray cylindre et suspension version marque}
\begin{tikzcd}
	{C^\flat\otimes\{0,1\}} & {C^\flat\otimes [1]^\sharp} \\
	{1\amalg 1} & {[C,1]^\sharp}
	\arrow[from=1-1, to=2-1]
	\arrow[from=1-1, to=1-2]
	\arrow["\lrcorner"{anchor=center, pos=0.125, rotate=180}, draw=none, from=2-2, to=1-1]
	\arrow[from=2-1, to=2-2]
	\arrow[from=1-2, to=2-2]
\end{tikzcd}
\end{equation}

\begin{prop}
\label{prop:otimes et op marked version}
There is diagram
% q.uiver.app/#q=WzAsNixbMSwwLCIoQ1xcb3RpbWVzWzFdXlxcc2hhcnApXlxcY2lyYyJdLFsyLDAsIihDXFxvdGltZXNcXHswXFx9KV5cXGNpcmMiXSxbMCwwLCIoQ1xcb3RpbWVzXFx7MVxcfSleXFxjaXJjIl0sWzEsMSwiQ15cXGNpcmNcXG90aW1lc1sxXV5cXHNoYXJwIl0sWzIsMSwiQ15cXGNpcmNcXG90aW1lc1xcezFcXH0iXSxbMCwxLCJDXlxcY2lyY1xcb3RpbWVzXFx7MFxcfSJdLFswLDMsIlxcc2ltIl0sWzEsNCwiXFxzaW0iXSxbMiw1LCJcXHNpbSJdLFsxLDBdLFsyLDBdLFs1LDNdLFs0LDNdXQ==
\[\begin{tikzcd}
	{(C\otimes\{1\})^\circ} & {(C\otimes[1]^\sharp)^\circ} & {(C\otimes\{0\})^\circ} \\
	{C^\circ\otimes\{0\}} & {C^\circ\otimes[1]^\sharp} & {C^\circ\otimes\{1\}}
	\arrow["\sim", from=1-2, to=2-2]
	\arrow["\sim", from=1-3, to=2-3]
	\arrow["\sim", from=1-1, to=2-1]
	\arrow[from=1-3, to=1-2]
	\arrow[from=1-1, to=1-2]
	\arrow[from=2-1, to=2-2]
	\arrow[from=2-3, to=2-2]
\end{tikzcd}\]
natural in $C:\ocatm$,
where all vertical arrows are equivalences. There is an invertible natural transformation
$$ C\star 1\sim (1\costar C^{\circ})^\circ.$$
\end{prop}
\begin{proof}
The corollary \ref{cor:otimes et op} provides an invertible transformation 
$$(C^\natural\otimes [1])^\circ\sim (C^\natural)^\circ\otimes[1]$$
The first assertion then follows from the definition of the Gray tensor product for marked $\io$-categories.
The second assertion is a consequence of the definition of the marked Gray cone and $\circ$-cone.
\end{proof}


\begin{example}
In all the following diagrams, marked cells are represented by crossed-out arrows.


The object $\Db_1^\flat\otimes[1]^\sharp$ corresponds to the diagram
% https://q.uiver.app/#q=WzAsNCxbMCwwLCIwMCJdLFswLDEsIjEwIl0sWzEsMSwiMTEiXSxbMSwwLCIwMSJdLFswLDFdLFsxLDIsIi8iLDNdLFswLDMsIi8iLDNdLFszLDJdLFszLDEsIi8iLDMseyJzaG9ydGVuIjp7InNvdXJjZSI6MjAsInRhcmdldCI6MjB9LCJsZXZlbCI6Mn1dXQ==
\[\begin{tikzcd}
	00 & 01 \\
	10 & 11
	\arrow[from=1-1, to=2-1]
	\arrow["{/}"{marking}, from=2-1, to=2-2]
	\arrow["{/}"{marking}, from=1-1, to=1-2]
	\arrow[from=1-2, to=2-2]
	\arrow["{/}"{marking}, shorten <=4pt, shorten >=4pt, Rightarrow, from=1-2, to=2-1]
\end{tikzcd}\]
the object $(\Db_1)^\sharp\otimes[1]^\sharp$ corresponds to the diagram
% https://q.uiver.app/#q=WzAsNCxbMCwwLCIwMCJdLFswLDEsIjEwIl0sWzEsMSwiMTEiXSxbMSwwLCIwMSJdLFswLDEsIi8iLDNdLFsxLDIsIi8iLDNdLFswLDMsIi8iLDNdLFszLDIsIi8iLDNdXQ==
\[\begin{tikzcd}
	00 & 01 \\
	10 & 11
	\arrow["{/}"{marking}, from=1-1, to=2-1]
	\arrow["{/}"{marking}, from=2-1, to=2-2]
	\arrow["{/}"{marking}, from=1-1, to=1-2]
	\arrow["{/}"{marking}, from=1-2, to=2-2]
\end{tikzcd}\]
the object $\Db_2^\flat\otimes[1]^\sharp$ corresponds to the diagram
% https://q.uiver.app/#q=WzAsOCxbMSwwLCIwMSJdLFswLDAsIjAwIl0sWzAsMSwiMTAiXSxbMSwxLCIxMSJdLFsyLDAsIjAwIl0sWzMsMCwiMDEiXSxbMywxLCIxMSJdLFsyLDEsIjEwIl0sWzEsMCwiLyIsM10sWzEsMl0sWzIsMywiLyIsM10sWzAsM10sWzAsMiwiLyIsMyx7InNob3J0ZW4iOnsic291cmNlIjoyMCwidGFyZ2V0IjoyMH0sImxldmVsIjoyfV0sWzQsN10sWzQsNSwiLyIsM10sWzUsNl0sWzUsNywiLyIsMyx7InNob3J0ZW4iOnsic291cmNlIjoyMCwidGFyZ2V0IjoyMH0sImxldmVsIjoyfV0sWzEsMiwiIiwyLHsiY3VydmUiOjV9XSxbNyw2LCIvIiwzXSxbNSw2LCIiLDEseyJjdXJ2ZSI6LTV9XSxbOSwxNywiICIsMix7InNob3J0ZW4iOnsic291cmNlIjoyMCwidGFyZ2V0IjoyMH19XSxbMTksMTUsIiAiLDIseyJzaG9ydGVuIjp7InNvdXJjZSI6MjAsInRhcmdldCI6MjB9fV0sWzExLDEzLCIiLDAseyJvZmZzZXQiOi0xLCJzaG9ydGVuIjp7InNvdXJjZSI6MjAsInRhcmdldCI6MjB9LCJsZXZlbCI6MSwic3R5bGUiOnsiaGVhZCI6eyJuYW1lIjoibm9uZSJ9fX1dLFsxMSwxMywiIiwyLHsib2Zmc2V0IjoxLCJzaG9ydGVuIjp7InNvdXJjZSI6MjAsInRhcmdldCI6MjB9LCJsZXZlbCI6MSwic3R5bGUiOnsiaGVhZCI6eyJuYW1lIjoibm9uZSJ9fX1dLFsxMSwxMywiLyIsMyx7InNob3J0ZW4iOnsic291cmNlIjoyMCwidGFyZ2V0IjoyMH0sImxldmVsIjoxfV1d
\[\begin{tikzcd}
	00 & 01 & 00 & 01 \\
	10 & 11 & 10 & 11
	\arrow["{/}"{marking}, from=1-1, to=1-2]
	\arrow[""{name=0, anchor=center, inner sep=0}, from=1-1, to=2-1]
	\arrow["{/}"{marking}, from=2-1, to=2-2]
	\arrow[""{name=1, anchor=center, inner sep=0}, from=1-2, to=2-2]
	\arrow["{/}"{marking}, shorten <=4pt, shorten >=4pt, Rightarrow, from=1-2, to=2-1]
	\arrow[""{name=2, anchor=center, inner sep=0}, from=1-3, to=2-3]
	\arrow["{/}"{marking}, from=1-3, to=1-4]
	\arrow[""{name=3, anchor=center, inner sep=0}, from=1-4, to=2-4]
	\arrow["{/}"{marking}, shorten <=4pt, shorten >=4pt, Rightarrow, from=1-4, to=2-3]
	\arrow[""{name=4, anchor=center, inner sep=0}, curve={height=30pt}, from=1-1, to=2-1]
	\arrow["{/}"{marking}, from=2-3, to=2-4]
	\arrow[""{name=5, anchor=center, inner sep=0}, curve={height=-30pt}, from=1-4, to=2-4]
	\arrow["{ }"', shorten <=6pt, shorten >=6pt, Rightarrow, from=0, to=4]
	\arrow["{ }"', shorten <=6pt, shorten >=6pt, Rightarrow, from=5, to=3]
	\arrow[shift left=0.7, shorten <=6pt, shorten >=8pt, no head, from=1, to=2]
	\arrow[shift right=0.7, shorten <=6pt, shorten >=8pt, no head, from=1, to=2]
	\arrow["{/}"{marking}, shorten <=6pt, shorten >=6pt, from=1, to=2]
\end{tikzcd}\]
and the object $(\Db_2)_t\otimes[1]^\sharp$ corresponds to the diagram
% https://q.uiver.app/#q=WzAsOCxbMSwwLCIwMSJdLFswLDAsIjAwIl0sWzAsMSwiMTAiXSxbMSwxLCIxMSJdLFsyLDAsIjAwIl0sWzMsMCwiMDEiXSxbMywxLCIxMSJdLFsyLDEsIjEwIl0sWzEsMCwiLyIsM10sWzEsMl0sWzIsMywiLyIsM10sWzAsM10sWzAsMiwiLyIsMyx7InNob3J0ZW4iOnsic291cmNlIjoyMCwidGFyZ2V0IjoyMH0sImxldmVsIjoyfV0sWzQsN10sWzQsNSwiLyIsM10sWzUsNl0sWzUsNywiLyIsMyx7InNob3J0ZW4iOnsic291cmNlIjoyMCwidGFyZ2V0IjoyMH0sImxldmVsIjoyfV0sWzEsMiwiIiwyLHsiY3VydmUiOjV9XSxbNyw2LCIvIiwzXSxbNSw2LCIiLDEseyJjdXJ2ZSI6LTV9XSxbMTEsMTMsIj0iLDMseyJzaG9ydGVuIjp7InNvdXJjZSI6NDAsInRhcmdldCI6NDB9LCJzdHlsZSI6eyJib2R5Ijp7Im5hbWUiOiJub25lIn0sImhlYWQiOnsibmFtZSI6Im5vbmUifX19XSxbMTksMTUsIiAvIiwzLHsic2hvcnRlbiI6eyJzb3VyY2UiOjIwLCJ0YXJnZXQiOjIwfX1dLFs5LDE3LCIgLyIsMyx7InNob3J0ZW4iOnsic291cmNlIjoyMCwidGFyZ2V0IjoyMH19XV0=
\[\begin{tikzcd}
	00 & 01 & 00 & 01 \\
	10 & 11 & 10 & 11
	\arrow["{/}"{marking}, from=1-1, to=1-2]
	\arrow[""{name=0, anchor=center, inner sep=0}, from=1-1, to=2-1]
	\arrow["{/}"{marking}, from=2-1, to=2-2]
	\arrow[""{name=1, anchor=center, inner sep=0}, from=1-2, to=2-2]
	\arrow["{/}"{marking}, shorten <=4pt, shorten >=4pt, Rightarrow, from=1-2, to=2-1]
	\arrow[""{name=2, anchor=center, inner sep=0}, from=1-3, to=2-3]
	\arrow["{/}"{marking}, from=1-3, to=1-4]
	\arrow[""{name=3, anchor=center, inner sep=0}, from=1-4, to=2-4]
	\arrow["{/}"{marking}, shorten <=4pt, shorten >=4pt, Rightarrow, from=1-4, to=2-3]
	\arrow[""{name=4, anchor=center, inner sep=0}, curve={height=30pt}, from=1-1, to=2-1]
	\arrow["{/}"{marking}, from=2-3, to=2-4]
	\arrow[""{name=5, anchor=center, inner sep=0}, curve={height=-30pt}, from=1-4, to=2-4]
	\arrow["{=}"{marking}, draw=none, from=1, to=2]
	\arrow["{ /}"{marking}, shorten <=6pt, shorten >=6pt, Rightarrow, from=5, to=3]
	\arrow["{ /}"{marking}, shorten <=6pt, shorten >=6pt, Rightarrow, from=0, to=4]
\end{tikzcd}\]
\end{example}


\p 
\label{para:slice and joint}
We also define the functors
$$\uvar\star 1:\ocatm\to\ocatm~~~~~~1\costar \uvar:\ocatm\to \ocatm,$$
respectively called the \wcsnotionsym{marked Gray cone}{((d40@$\uvar\star 1$}{Gray cone}{for marked $\io$-categories} and the \wcsnotion{marked Gray $\circ$-cone}{Gray $\circ$-cone}{for marked $\io$-categories}\index[notation]{((d50@$1\overset{co}{\star}\_$!\textit{for marked $\io$-categories}}, where for any marked $\io$-category $C$, $C\star 1$ and $1\costar C$, fit in the following cocartesian square
% q.uiver.app/#q=WzAsOCxbMywwLCJDXFxvdGltZXMgWzFdXlxcc2hhcnAiXSxbMiwwLCJDXFxvdGltZXNcXHswXFx9Il0sWzIsMSwiMSJdLFszLDEsIjFcXGNvc3RhcnQgQyJdLFsxLDAsIkNcXG90aW1lcyBbMV1eXFxzaGFycCJdLFswLDAsIkNcXG90aW1lc1xcezFcXH0iXSxbMCwxLCIxIl0sWzEsMSwiQ1xcc3RhciAxIl0sWzUsNF0sWzEsMF0sWzAsM10sWzEsMl0sWzIsM10sWzQsN10sWzUsNl0sWzYsN10sWzcsNSwiIiwxLHsic3R5bGUiOnsibmFtZSI6ImNvcm5lciJ9fV0sWzMsMSwiIiwxLHsic3R5bGUiOnsibmFtZSI6ImNvcm5lciJ9fV1d
\[\begin{tikzcd}
	{C\otimes\{1\}} & {C\otimes [1]^\sharp} & {C\otimes\{0\}} & {C\otimes [1]^\sharp} \\
	1 & {C\star 1} & 1 & {1\costar C}
	\arrow[from=1-1, to=1-2]
	\arrow[from=1-3, to=1-4]
	\arrow[from=1-4, to=2-4]
	\arrow[from=1-3, to=2-3]
	\arrow[from=2-3, to=2-4]
	\arrow[from=1-2, to=2-2]
	\arrow[from=1-1, to=2-1]
	\arrow[from=2-1, to=2-2]
	\arrow["\lrcorner"{anchor=center, pos=0.125, rotate=180}, draw=none, from=2-2, to=1-1]
	\arrow["\lrcorner"{anchor=center, pos=0.125, rotate=180}, draw=none, from=2-4, to=1-3]
\end{tikzcd}\]
These two functors preserve colimit.
The proposition \ref{prop:otimes et op marked version} induces an
invertible natural transformation
$$ C\star 1\sim (1\costar C^{\circ})^\circ.$$
\begin{example}
In all the following diagrams, marked cells are represented by crossed-out arrows.

The objects $\Db_1^\flat\star 1$ and $1\costar \Db_1^\flat$ correspond respectively the diagrams
% https://q.uiver.app/#q=WzAsNixbMCwwLCIwIl0sWzAsMSwiMSJdLFsxLDEsIlxcc3RhciJdLFszLDEsIlxcc3RhciJdLFs0LDAsIjAiXSxbNCwxLCIxIl0sWzAsMV0sWzEsMiwiLyIsM10sWzAsMiwiLyIsM10sWzQsNV0sWzMsNCwiLyIsM10sWzMsNSwiLyIsM10sWzksMTEsIi8iLDMseyJvZmZzZXQiOjIsInNob3J0ZW4iOnsic291cmNlIjoyMCwidGFyZ2V0IjoyMH19XSxbOCwxLCIvIiwzLHsic2hvcnRlbiI6eyJzb3VyY2UiOjIwfX1dXQ==
\[\begin{tikzcd}
	0 &&&& 0 \\
	1 & \star && \star & 1
	\arrow[from=1-1, to=2-1]
	\arrow["{/}"{marking}, from=2-1, to=2-2]
	\arrow[""{name=0, anchor=center, inner sep=0}, "{/}"{marking}, from=1-1, to=2-2]
	\arrow[""{name=1, anchor=center, inner sep=0}, from=1-5, to=2-5]
	\arrow["{/}"{marking}, from=2-4, to=1-5]
	\arrow[""{name=2, anchor=center, inner sep=0}, "{/}"{marking}, from=2-4, to=2-5]
	\arrow["{/}"{marking}, shift right=2, shorten <=4pt, shorten >=4pt, Rightarrow, from=1, to=2]
	\arrow["{/}"{marking}, shorten <=2pt, Rightarrow, from=0, to=2-1]
\end{tikzcd}\]
the objects $(\Db_1)_t\star 1$ and $1\costar (\Db_1)_t$ correspond respectively the diagrams
% https://q.uiver.app/#q=WzAsNixbMCwwLCIwIl0sWzAsMSwiMSJdLFsxLDEsIlxcc3RhciJdLFszLDEsIlxcc3RhciJdLFs0LDAsIjAiXSxbNCwxLCIxIl0sWzAsMSwiLyIsM10sWzEsMiwiLyIsM10sWzAsMiwiLyIsM10sWzQsNSwiLyIsM10sWzMsNCwiLyIsM10sWzMsNSwiLyIsM11d
\[\begin{tikzcd}
	0 &&&& 0 \\
	1 & \star && \star & 1
	\arrow["{/}"{marking}, from=1-1, to=2-1]
	\arrow["{/}"{marking}, from=2-1, to=2-2]
	\arrow["{/}"{marking}, from=1-1, to=2-2]
	\arrow["{/}"{marking}, from=1-5, to=2-5]
	\arrow["{/}"{marking}, from=2-4, to=1-5]
	\arrow["{/}"{marking}, from=2-4, to=2-5]
\end{tikzcd}\]
the objects $\Db_2^\flat\star 1$ and $1\costar \Db_2^\flat$ correspond respectively the diagrams
% https://q.uiver.app/#q=WzAsMTQsWzAsMCwiMCJdLFswLDEsIjEiXSxbMSwxLCJcXHN0YXIiXSxbMiwwLCIwIl0sWzMsMSwiXFxzdGFyIl0sWzIsMSwiMSJdLFsxLDBdLFs1LDAsIjAiXSxbNCwxLCJcXHN0YXIiXSxbNSwxLCIxIl0sWzYsMSwiXFxzdGFyIl0sWzcsMCwiMCJdLFs3LDEsIjEiXSxbNiwwXSxbMCwxXSxbMSwyLCIvIiwzXSxbMyw1XSxbMCwxLCIiLDIseyJjdXJ2ZSI6NX1dLFs1LDQsIi8iLDNdLFswLDIsIi8iLDNdLFs2LDIsIiIsMCx7InN0eWxlIjp7ImJvZHkiOnsibmFtZSI6Im5vbmUifSwiaGVhZCI6eyJuYW1lIjoibm9uZSJ9fX1dLFszLDQsIi8iLDNdLFs3LDgsIi8iLDNdLFs3LDldLFs4LDksIi8iLDNdLFsxMSwxMCwiLyIsM10sWzExLDEyXSxbMTIsMTAsIi8iLDNdLFsxMSwxMiwiIiwxLHsiY3VydmUiOi01fV0sWzEzLDEwLCIiLDIseyJzdHlsZSI6eyJib2R5Ijp7Im5hbWUiOiJub25lIn0sImhlYWQiOnsibmFtZSI6Im5vbmUifX19XSxbMTQsMTcsIiAiLDIseyJzaG9ydGVuIjp7InNvdXJjZSI6MjAsInRhcmdldCI6MjB9fV0sWzE5LDEsIi8iLDMseyJzaG9ydGVuIjp7InNvdXJjZSI6MjAsInRhcmdldCI6MjB9fV0sWzIwLDE2LCIiLDAseyJvZmZzZXQiOi0xLCJzaG9ydGVuIjp7InNvdXJjZSI6MjAsInRhcmdldCI6MjB9LCJsZXZlbCI6MSwic3R5bGUiOnsiaGVhZCI6eyJuYW1lIjoibm9uZSJ9fX1dLFsyMCwxNiwiIiwyLHsib2Zmc2V0IjoxLCJzaG9ydGVuIjp7InNvdXJjZSI6MjAsInRhcmdldCI6MjB9LCJsZXZlbCI6MSwic3R5bGUiOnsiaGVhZCI6eyJuYW1lIjoibm9uZSJ9fX1dLFsyMCwxNiwiLyIsMyx7InNob3J0ZW4iOnsic291cmNlIjoyMCwidGFyZ2V0IjoyMH0sImxldmVsIjoxfV0sWzIxLDUsIi8iLDMseyJzaG9ydGVuIjp7InNvdXJjZSI6MjB9fV0sWzI4LDI2LCIiLDEseyJzaG9ydGVuIjp7InNvdXJjZSI6MjAsInRhcmdldCI6MjB9fV0sWzI2LDI3LCIvIiwzLHsib2Zmc2V0IjoyLCJzaG9ydGVuIjp7InNvdXJjZSI6MjAsInRhcmdldCI6MjB9fV0sWzIzLDI0LCIvIiwzLHsib2Zmc2V0IjoyLCJzaG9ydGVuIjp7InNvdXJjZSI6MjAsInRhcmdldCI6MjB9fV0sWzIzLDI5LCIiLDIseyJvZmZzZXQiOjEsInNob3J0ZW4iOnsic291cmNlIjoyMCwidGFyZ2V0IjoyMH0sImxldmVsIjoxLCJzdHlsZSI6eyJoZWFkIjp7Im5hbWUiOiJub25lIn19fV0sWzIzLDI5LCIvIiwzLHsic2hvcnRlbiI6eyJzb3VyY2UiOjIwLCJ0YXJnZXQiOjIwfSwibGV2ZWwiOjF9XSxbMjMsMjksIiIsMix7Im9mZnNldCI6LTEsInNob3J0ZW4iOnsic291cmNlIjoyMCwidGFyZ2V0IjoyMH0sImxldmVsIjoxLCJzdHlsZSI6eyJoZWFkIjp7Im5hbWUiOiJub25lIn19fV1d
\[\begin{tikzcd}
	0 & {} & 0 &&& 0 & {} & 0 \\
	1 & \star & 1 & \star & \star & 1 & \star & 1
	\arrow[""{name=0, anchor=center, inner sep=0}, from=1-1, to=2-1]
	\arrow["{/}"{marking}, from=2-1, to=2-2]
	\arrow[""{name=1, anchor=center, inner sep=0}, from=1-3, to=2-3]
	\arrow[""{name=2, anchor=center, inner sep=0}, curve={height=30pt}, from=1-1, to=2-1]
	\arrow["{/}"{marking}, from=2-3, to=2-4]
	\arrow[""{name=3, anchor=center, inner sep=0}, "{/}"{marking}, from=1-1, to=2-2]
	\arrow[""{name=4, anchor=center, inner sep=0}, draw=none, from=1-2, to=2-2]
	\arrow[""{name=5, anchor=center, inner sep=0}, "{/}"{marking}, from=1-3, to=2-4]
	\arrow["{/}"{marking}, from=1-6, to=2-5]
	\arrow[""{name=6, anchor=center, inner sep=0}, from=1-6, to=2-6]
	\arrow[""{name=7, anchor=center, inner sep=0}, "{/}"{marking}, from=2-5, to=2-6]
	\arrow["{/}"{marking}, from=1-8, to=2-7]
	\arrow[""{name=8, anchor=center, inner sep=0}, from=1-8, to=2-8]
	\arrow[""{name=9, anchor=center, inner sep=0}, "{/}"{marking}, from=2-8, to=2-7]
	\arrow[""{name=10, anchor=center, inner sep=0}, curve={height=-30pt}, from=1-8, to=2-8]
	\arrow[""{name=11, anchor=center, inner sep=0}, draw=none, from=1-7, to=2-7]
	\arrow["{ }"', shorten <=6pt, shorten >=6pt, Rightarrow, from=0, to=2]
	\arrow["{/}"{marking}, shorten <=2pt, shorten >=2pt, Rightarrow, from=3, to=2-1]
	\arrow[shift left=0.7, shorten <=6pt, shorten >=8pt, no head, from=4, to=1]
	\arrow[shift right=0.7, shorten <=6pt, shorten >=8pt, no head, from=4, to=1]
	\arrow["{/}"{marking}, shorten <=6pt, shorten >=6pt, from=4, to=1]
	\arrow["{/}"{marking}, shorten <=2pt, Rightarrow, from=5, to=2-3]
	\arrow[shorten <=6pt, shorten >=6pt, Rightarrow, from=10, to=8]
	\arrow["{/}"{marking}, shift right=2, shorten <=4pt, shorten >=4pt, Rightarrow, from=8, to=9]
	\arrow["{/}"{marking}, shift right=2, shorten <=4pt, shorten >=4pt, Rightarrow, from=6, to=7]
	\arrow[shift right=0.7, shorten <=6pt, shorten >=8pt, no head, from=6, to=11]
	\arrow["{/}"{marking}, shorten <=6pt, shorten >=6pt, from=6, to=11]
	\arrow[shift left=0.7, shorten <=6pt, shorten >=8pt, no head, from=6, to=11]
\end{tikzcd}\]
and the objects $(\Db_2)_t\star 1$ and $1\costar (\Db_2)_t$ correspond respectively the diagrams
% https://q.uiver.app/#q=WzAsMTQsWzAsMCwiMCJdLFswLDEsIjEiXSxbMSwxLCJcXHN0YXIiXSxbMiwwLCIwIl0sWzMsMSwiXFxzdGFyIl0sWzIsMSwiMSJdLFsxLDBdLFs1LDAsIjAiXSxbNCwxLCJcXHN0YXIiXSxbNSwxLCIxIl0sWzYsMSwiXFxzdGFyIl0sWzcsMCwiMCJdLFs3LDEsIjEiXSxbNiwwXSxbMCwxXSxbMSwyLCIvIiwzXSxbMyw1XSxbMCwxLCIiLDIseyJjdXJ2ZSI6NX1dLFs1LDQsIi8iLDNdLFswLDIsIi8iLDNdLFs2LDIsIiIsMCx7InN0eWxlIjp7ImJvZHkiOnsibmFtZSI6Im5vbmUifSwiaGVhZCI6eyJuYW1lIjoibm9uZSJ9fX1dLFszLDQsIi8iLDNdLFs3LDgsIi8iLDNdLFs3LDldLFs4LDksIi8iLDNdLFsxMSwxMCwiLyIsM10sWzExLDEyXSxbMTIsMTAsIi8iLDNdLFsxMSwxMiwiIiwxLHsiY3VydmUiOi01fV0sWzEzLDEwLCIiLDIseyJzdHlsZSI6eyJib2R5Ijp7Im5hbWUiOiJub25lIn0sImhlYWQiOnsibmFtZSI6Im5vbmUifX19XSxbMTQsMTcsIi8iLDMseyJzaG9ydGVuIjp7InNvdXJjZSI6MjAsInRhcmdldCI6MjB9fV0sWzE5LDEsIi8iLDMseyJzaG9ydGVuIjp7InNvdXJjZSI6MjAsInRhcmdldCI6MjB9fV0sWzIxLDUsIi8iLDMseyJzaG9ydGVuIjp7InNvdXJjZSI6MjB9fV0sWzI4LDI2LCIvIiwzLHsic2hvcnRlbiI6eyJzb3VyY2UiOjIwLCJ0YXJnZXQiOjIwfX1dLFsyNiwyNywiLyIsMyx7Im9mZnNldCI6Miwic2hvcnRlbiI6eyJzb3VyY2UiOjIwLCJ0YXJnZXQiOjIwfX1dLFsyMywyNCwiLyIsMyx7Im9mZnNldCI6Miwic2hvcnRlbiI6eyJzb3VyY2UiOjIwLCJ0YXJnZXQiOjIwfX1dLFsyMCwxNiwiPSIsMSx7InNob3J0ZW4iOnsic291cmNlIjoyMCwidGFyZ2V0IjoyMH0sInN0eWxlIjp7ImJvZHkiOnsibmFtZSI6Im5vbmUifSwiaGVhZCI6eyJuYW1lIjoibm9uZSJ9fX1dLFsyMywyOSwiPSIsMyx7InNob3J0ZW4iOnsic291cmNlIjoyMCwidGFyZ2V0IjoyMH0sInN0eWxlIjp7ImJvZHkiOnsibmFtZSI6Im5vbmUifSwiaGVhZCI6eyJuYW1lIjoibm9uZSJ9fX1dXQ==
\[\begin{tikzcd}
	0 & {} & 0 &&& 0 & {} & 0 \\
	1 & \star & 1 & \star & \star & 1 & \star & 1
	\arrow[""{name=0, anchor=center, inner sep=0}, from=1-1, to=2-1]
	\arrow["{/}"{marking}, from=2-1, to=2-2]
	\arrow[""{name=1, anchor=center, inner sep=0}, from=1-3, to=2-3]
	\arrow[""{name=2, anchor=center, inner sep=0}, curve={height=30pt}, from=1-1, to=2-1]
	\arrow["{/}"{marking}, from=2-3, to=2-4]
	\arrow[""{name=3, anchor=center, inner sep=0}, "{/}"{marking}, from=1-1, to=2-2]
	\arrow[""{name=4, anchor=center, inner sep=0}, draw=none, from=1-2, to=2-2]
	\arrow[""{name=5, anchor=center, inner sep=0}, "{/}"{marking}, from=1-3, to=2-4]
	\arrow["{/}"{marking}, from=1-6, to=2-5]
	\arrow[""{name=6, anchor=center, inner sep=0}, from=1-6, to=2-6]
	\arrow[""{name=7, anchor=center, inner sep=0}, "{/}"{marking}, from=2-5, to=2-6]
	\arrow["{/}"{marking}, from=1-8, to=2-7]
	\arrow[""{name=8, anchor=center, inner sep=0}, from=1-8, to=2-8]
	\arrow[""{name=9, anchor=center, inner sep=0}, "{/}"{marking}, from=2-8, to=2-7]
	\arrow[""{name=10, anchor=center, inner sep=0}, curve={height=-30pt}, from=1-8, to=2-8]
	\arrow[""{name=11, anchor=center, inner sep=0}, draw=none, from=1-7, to=2-7]
	\arrow["{/}"{marking}, shorten <=6pt, shorten >=6pt, Rightarrow, from=0, to=2]
	\arrow["{/}"{marking}, shorten <=2pt, shorten >=2pt, Rightarrow, from=3, to=2-1]
	\arrow["{/}"{marking}, shorten <=2pt, Rightarrow, from=5, to=2-3]
	\arrow["{/}"{marking}, shorten <=6pt, shorten >=6pt, Rightarrow, from=10, to=8]
	\arrow["{/}"{marking}, shift right=2, shorten <=4pt, shorten >=4pt, Rightarrow, from=8, to=9]
	\arrow["{/}"{marking}, shift right=2, shorten <=4pt, shorten >=4pt, Rightarrow, from=6, to=7]
	\arrow["{=}"{description}, draw=none, from=4, to=1]
	\arrow["{=}"{marking}, draw=none, from=6, to=11]
\end{tikzcd}\]
\end{example}





We will also denote by 
$$\begin{array}{ccccccc}
\ocatm_{\bullet} &\to&\ocatm&&\ocatm_{\bullet} &\to&\ocatm\\
(C,c)&\mapsto &C_{/c} & &(C,c) &\mapsto &C_{c/}
\end{array}
$$
the right adjoints of Gray cone and of the Gray $\circ$-cone, respectively called the \wcsnotionsym{slice of $C$ over $c$}{(cc@$C_{c/}$}{slice over}{for marked $\io$-categories} and the \wcsnotionsym{slice of $C$ under $c$}{(cc@$C_{/c}$}{slice under}{for marked $\io$-categories}.
The proposition \ref{prop:otimes et op marked version} induces an  invertible natural transformation:
$$C_{/c}\sim (C^{\circ}_{c/})^\circ.$$
Given an $\io$-category $C$, and $c,d$ two objects, the cocartesian square \eqref{eq:liens entre Gray cylindre et suspension version marque} induces two cartesian squares:
% https://q.uiver.app/#q=WzAsOCxbMSwwLCJDXlxcc2hhcnBfey9kfSJdLFszLDAsIkNeXFxzaGFycF97Yy99Il0sWzMsMSwiQ15cXHNoYXJwIl0sWzIsMSwiXFx7ZFxcfSJdLFswLDEsIlxce2NcXH0iXSxbMSwxLCJDXlxcc2hhcnAiXSxbMCwwLCJcXGhvbV9DKGMsZCleXFxmbGF0Il0sWzIsMCwiXFxob21fQyhjLGQpXlxcZmxhdCJdLFs2LDRdLFszLDJdLFsxLDJdLFs3LDNdLFswLDVdLFs0LDVdLFs2LDBdLFs2LDUsIiIsMSx7InN0eWxlIjp7Im5hbWUiOiJjb3JuZXIifX1dLFs3LDIsIiIsMSx7InN0eWxlIjp7Im5hbWUiOiJjb3JuZXIifX1dLFs3LDFdXQ==
\begin{equation}
\label{eq:fiber of marked splices}
\begin{tikzcd}
	{\hom_C(c,d)^\flat} & {C^\sharp_{/d}} & {\hom_C(c,d)^\flat} & {C^\sharp_{c/}} \\
	{\{c\}} & {C^\sharp} & {\{d\}} & {C^\sharp}
	\arrow[from=1-1, to=2-1]
	\arrow[from=2-3, to=2-4]
	\arrow[from=1-4, to=2-4]
	\arrow[from=1-3, to=2-3]
	\arrow[from=1-2, to=2-2]
	\arrow[from=2-1, to=2-2]
	\arrow[from=1-1, to=1-2]
	\arrow["\lrcorner"{anchor=center, pos=0.125}, draw=none, from=1-1, to=2-2]
	\arrow["\lrcorner"{anchor=center, pos=0.125}, draw=none, from=1-3, to=2-4]
	\arrow[from=1-3, to=1-4]
\end{tikzcd}
\end{equation}

\p 
\label{paragrap: equation fullfill by cylinder and join marked version}
The equation given in paragraph \ref{paragrap: equation fullfill by cylinder and join}
induces similar ones for the marked version of these operations. For every marked $\io$-category $C$, there are a natural identification between $[C,1]\otimes [1]^\sharp$ and the colimit of the following diagram
\begin{equation}
\label{eq:eq for cylinder marked version}
\begin{tikzcd}
	{[1]^\sharp\vee [ C,1]} & {[C\otimes\{0\},1]} & {[C\otimes [1]^\sharp,1]} & {[C\otimes\{1\},1]} & {[C,1]\vee[1]^\sharp}
	\arrow[from=1-2, to=1-1]
	\arrow[from=1-2, to=1-3]
	\arrow[from=1-4, to=1-3]
	\arrow[from=1-4, to=1-5]
\end{tikzcd}
\end{equation}
There is also a natural identification between
 $1\costar [C,1]$ and the colimit of the diagram
\begin{equation}
\label{eq:eq for Gray cone marked version}
\begin{tikzcd}
	 {[1]^\sharp\vee [C,1]} & {[C,1]} & {[C\star 1,1]} 
	\arrow[from=1-2, to=1-1]
	\arrow[from=1-2, to=1-3]
\end{tikzcd}
\end{equation}
and between $[C,1] \star 1$ and the colimit of the diagram
\begin{equation}
\label{eq:eq for cojoin marked version}
\begin{tikzcd}
	 {[1\costar C,1]}& {[C,1]} & {[C,1]\vee[1]^\sharp} 
	\arrow[from=1-2, to=1-1]
	\arrow[from=1-2, to=1-3]
\end{tikzcd}
\end{equation}








\p
 For any $C:\ocat$, we denote by \wcnotation{$m_{C^\sharp}$}{(mc@$m_{C^\sharp}$} the colimit preserving functor 
$\ocatm\to\ocatm$ whose value on $[a,n]^\flat$ is $[a\times C^\sharp,n]$, on $[1]^\sharp$ is $[C,1]^\sharp$, and on $[(\Db_n)_t,1]$ is $
[(\Db_n)_t\times C^\sharp,1]$.
Remark that the assignation $C\mapsto m_{C^\sharp}$ is natural in $C$ and that $m_1$ is the identity.
We define the colimit preserving functor:
\ssym{((d20@$\ominus$}{for marked $\io$-categories}
$$\begin{array}{ccc}
\ocatm\times\ocatm &\to& \ocatm\\
(X,Y)&\mapsto &X\ominus Y^\sharp
\end{array}
$$
where for any marked $\io$-category $C$ and element $[b,n]$ of $\Delta[\Theta]$, $C\ominus [b,n]^\sharp$ is the following pushout: 
% q.uiver.app/#q=WzAsNCxbMSwwLCJtX3tiXlxcc2hhcnB9KENcXG90aW1lc1tuXV5cXHNoYXJwKSJdLFswLDAsIlxcY29wcm9kXFxsaW1pdHNfe2tcXGxlcSBufW1fe2JeXFxzaGFycH0oQ1xcb3RpbWVzXFx7a1xcfSkiXSxbMCwxLCJcXGNvcHJvZFxcbGltaXRzX3trXFxsZXEgbn1tXzEoQ1xcb3RpbWVzXFx7a1xcfSkiXSxbMSwxLCJDXFxvbWludXNbYixuXV5cXHNoYXJwIl0sWzEsMl0sWzEsMF0sWzAsM10sWzIsM10sWzMsNSwiIiwxLHsibGV2ZWwiOjEsInN0eWxlIjp7Im5hbWUiOiJjb3JuZXIifX1dXQ==
\begin{equation}
\label{eq: def of ominus marked}
\begin{tikzcd}
	{\coprod\limits_{k\leq n}m_{b^\sharp}(C\otimes\{k\})} & {m_{b^\sharp}(C\otimes[n]^\sharp)} \\
	{\coprod\limits_{k\leq n}m_1(C\otimes\{k\})} & {C\ominus[b,n]^\sharp}
	\arrow[from=1-1, to=2-1]
	\arrow[""{name=0, anchor=center, inner sep=0}, from=1-1, to=1-2]
	\arrow[from=1-2, to=2-2]
	\arrow[from=2-1, to=2-2]
	\arrow["\lrcorner"{anchor=center, pos=0.125, rotate=180}, draw=none, from=2-2, to=0]
\end{tikzcd}
\end{equation}
By construction, we then have $C\ominus [1]^\sharp:=C\otimes [1]^\sharp$.
The equation \eqref{eq:formula for the ominus} implies that for every marked $\io$-category $C$, 
there is a natural identification between $[C,1]\ominus[b,1]^\sharp$ and the colimit of the following diagram
% https://q.uiver.app/#q=WzAsNSxbMCwwLCJbYiwxXV5cXHNoYXJwXFx2ZWVbQywxXSJdLFs0LDAsIltDLDFdXFx2ZWVbYiwxXV5cXHNoYXJwIl0sWzEsMCwiW0NcXG90aW1lc1xcezBcXH1cXHRpbWVzIGJeXFxzaGFycCwxXSJdLFszLDAsIltDXFxvdGltZXNcXHsxXFx9XFx0aW1lcyBiXlxcc2hhcnAsMV0iXSxbMiwwLCJbKENcXG90aW1lc1sxXV5cXHNoYXJwKVxcdGltZXMgYl5cXHNoYXJwKSwxXSJdLFsyLDRdLFszLDRdLFszLDFdLFsyLDBdXQ==
\begin{equation}
\label{eq:formula for the ominus marked case}
\begin{tikzcd}[column sep = 0.3cm]
	{[b,1]^\sharp\vee[C,1]} & {[C\otimes\{0\}\times b^\sharp,1]} & {[(C\otimes[1]^\sharp)\times b^\sharp),1]} & {[C\otimes\{1\}\times b^\sharp,1]} & {[C,1]\vee[b,1]^\sharp}
	\arrow[from=1-2, to=1-3]
	\arrow[from=1-4, to=1-3]
	\arrow[from=1-4, to=1-5]
	\arrow[from=1-2, to=1-1]
\end{tikzcd}
\end{equation}

\begin{prop}
\label{prop:ominus and opmarked}
There is an equivalence 
$$(C\ominus B^\sharp)^\circ\sim C^\circ\ominus (B^\circ)^\sharp$$
natural in $C$ and $B$.
\end{prop}
\begin{proof}
It it sufficient to construct this equivalence when $B$ is of shape $[b,n]$.
The corollary \ref{cor:ominus et op} induces an equivalence
$$(C^\natural\otimes[n])^\circ\sim (C^\circ)^\natural\otimes [n]^\circ.$$
By the construction of the Gray tensor product of marked $\io$-categories, we have an equivalence 
$$(C\otimes[n]^\sharp)^\circ\sim C^\circ\otimes ([n]^\circ)^\sharp.$$
 The results then directly follows from the definition of the operation $\ominus$ and from the equivalence $(m_{b^\sharp}(\uvar))^\circ\sim m_{(b^\sharp)^\circ}((\uvar)^\circ)$.
\end{proof}



\begin{prop}
\label{prop:associativity of ominus}
Let $C$ be a $\io$-category, $D$ a marked $\io$-category and $[b,n]$ a globular sum.
\begin{enumerate}
\item The underlying $\io$-category of $C^\flat\ominus [b,n]^\sharp$ is $C\ominus [b,n]$.
\item The canonical morphism $C^\sharp\ominus [b,n]^\sharp\to C^\sharp\times [b,n]^\sharp$ is an equivalence.
\item The canonical morphism
$(C^\sharp\times D)\ominus [b,n]^\sharp\to C^\sharp\times (D\ominus K^\sharp)$ is an equivalence.
\end{enumerate}
\end{prop}
\begin{proof}
This is a consequence of propositions \ref{prop:cartesian square and times}, \ref{prop:associativity of Gray amput}, \ref{prop:associativity of Gray2} and \ref{prop:associativity of Gray amput2} and of the construction of $\ominus$.
\end{proof}



\p We now give some strictification results.

\begin{lemma}
\label{lemma:a otimes 1 is strict}
Let $C$ be a marked $\io$-category.
The canonical squares
% https://q.uiver.app/#q=WzAsNixbMSwwLCJDXFxvdGltZXNbMV1eXFxzaGFycCJdLFsxLDEsIlsxXV5cXHNoYXJwIl0sWzAsMSwiXFx7MFxcfSJdLFswLDAsIkMiXSxbMiwxLCJcXHsxXFx9Il0sWzIsMCwiQyJdLFszLDJdLFsyLDFdLFs0LDFdLFswLDFdLFs1LDBdLFszLDBdLFs1LDRdXQ==
\[\begin{tikzcd}
	C & {C\otimes[1]^\sharp} & C \\
	{\{0\}} & {[1]^\sharp} & {\{1\}}
	\arrow[from=1-1, to=2-1]
	\arrow[from=2-1, to=2-2]
	\arrow[from=2-3, to=2-2]
	\arrow[from=1-2, to=2-2]
	\arrow[from=1-3, to=1-2]
	\arrow[from=1-1, to=1-2]
	\arrow[from=1-3, to=2-3]
\end{tikzcd}\]
are cartesian. 
\end{lemma}
\begin{proof}
As the morphisms $\{\epsilon\}\to [1]$ for $\epsilon\leq 1$ are discrete Conduché functors, pullback along them preserves colimits, and we can then reduce to the case where $C$ is of the shape $[1]^\sharp$ or $[a,1]$ with  $a$ is an element of $t\Theta$. 
The case $C:=[1]^\sharp$ is obvious as we have $[1]^\sharp\otimes[1]^\sharp\sim [1]^\sharp\times[1]^\sharp$ according to  the first assertion of proposition \ref{prop:associativity of Gray amput}. We then focus on the case $C:=[a,1]$.


We claim that for any marked $\io$-category $D$, the square
% https://q.uiver.app/#q=WzAsNCxbMSwwLCJbRCwxXSJdLFsxLDEsIlsxXV5cXHNoYXJwIl0sWzAsMSwiXFx7XFxlcHNpbG9uXFx9Il0sWzAsMCwiXFx7XFxlcHNpbG9uXFx9Il0sWzMsMl0sWzIsMV0sWzMsMF0sWzAsMV1d
\begin{equation}
\label{eq:lemma:a otimes 1 is strict}
\begin{tikzcd}
	{\{\epsilon\}} & {[D,1]} \\
	{\{\epsilon\}} & {[1]^\sharp}
	\arrow[from=1-1, to=2-1]
	\arrow[from=2-1, to=2-2]
	\arrow[from=1-1, to=1-2]
	\arrow[from=1-2, to=2-2]
\end{tikzcd}
\end{equation}
is cartesian. To show this, as  the morphisms $\{\epsilon\}\to [1]$, are discrete Conduché functors one can reduce to the case where $D$ is a globular sum, where it is obvious.

We now return to the proof of the assertion.  
Using the equation \eqref{eq:eq for cylinder marked version}, the morphism $[a,1]\otimes[1]^\sharp$ is the horizontal colimit of the following diagram:
% https://q.uiver.app/#q=WzAsMTAsWzAsMCwiWzFdXlxcc2hhcnBcXHZlZVthLDFdIl0sWzEsMCwiW2FcXG90aW1lc1xcezBcXH0sMV0iXSxbMywwLCJbYVxcb3RpbWVzXFx7MVxcfSwxXSJdLFs0LDAsIlthLDFdXFx2ZWVbMV1eXFxzaGFycCJdLFsyLDAsIlthXFxvdGltZXNbMV1eXFxzaGFycCwxXSJdLFs0LDEsIlsxXV5cXHNoYXJwIl0sWzEsMSwiWzFdXlxcc2hhcnAiXSxbMCwxLCJbMV1eXFxzaGFycCJdLFsyLDEsIlsxXV5cXHNoYXJwIl0sWzMsMSwiWzFdXlxcc2hhcnAiXSxbMSwwXSxbMiwzXSxbMiw0XSxbMSw0XSxbNiw3XSxbNiw4XSxbOSw4XSxbOSw1XSxbMyw1LCJzXjAiXSxbMiw5XSxbMSw2XSxbNCw4XSxbMCw3LCJzXjEiLDJdXQ==
\[\begin{tikzcd}
	{[1]^\sharp\vee[a,1]} & {[a\otimes\{0\},1]} & {[a\otimes[1]^\sharp,1]} & {[a\otimes\{1\},1]} & {[a,1]\vee[1]^\sharp} \\
	{[1]^\sharp} & {[1]^\sharp} & {[1]^\sharp} & {[1]^\sharp} & {[1]^\sharp}
	\arrow[from=1-2, to=1-1]
	\arrow[from=1-4, to=1-5]
	\arrow[from=1-4, to=1-3]
	\arrow[from=1-2, to=1-3]
	\arrow[from=2-2, to=2-1]
	\arrow[from=2-2, to=2-3]
	\arrow[from=2-4, to=2-3]
	\arrow[from=2-4, to=2-5]
	\arrow["{s^0}", from=1-5, to=2-5]
	\arrow[from=1-4, to=2-4]
	\arrow[from=1-2, to=2-2]
	\arrow[from=1-3, to=2-3]
	\arrow["{s^1}"', from=1-1, to=2-1]
\end{tikzcd}\]


The results is then a direct application of the cartesian square \eqref{eq:lemma:a otimes 1 is strict} and of the fact  that pullbacks along morphisms $\{\epsilon\}\to [1]$ for $\epsilon\leq 1$ preserves colimits.
\end{proof}


\begin{prop}
\label{prop:tensor of glboer are strics}
For any object $a$ of $t\Theta$, the marked $\io$-categories $a\otimes [1]^\sharp$, $a\star 1$ and $1\costar a$ are strict. 
\end{prop}
\begin{proof}
We will show only the the strictness of the object $a\otimes[1]^\sharp$, as the proofs for $a\star 1$ and $1\costar a$ are similar.


Suppose first that $a$ is of shape $b^\flat$.
The first assertion of proposition \ref{prop:associativity of Gray amput} implies that the underlying $\io$-categories of $b^\flat\otimes [1]^\sharp$ is $b\otimes [1]$ which is strict according to proposition \ref{prop:strict stuff are pushout}. 


To conclude, we have to show that for any integer $n$, $(\Db_n)_t\otimes[1]^\sharp$ is strict. We proceed by induction.
Suppose first that $a$ is $(\Db_1)_t$. The second assertion of proposition \ref{prop:associativity of Gray amput} implies that 
$(\Db_1)_t \otimes[1]^\sharp$ is $([1]\times[1])^\sharp$ which is a strict object. 

Suppose now that $(\Db_n)_t\otimes[1]^\sharp$ is strict. 
The equation \eqref{eq:eq for cylinder marked version} stipulates that $(\Db_{n+1})_t\otimes[1]^\sharp$ is the colimit of the diagram.
$$\begin{tikzcd}[column sep = 0.2cm]
	{[1]^\sharp\vee [(\Db_n)_t,1]} & {[(\Db_n)_t\otimes\{0\},1]} & {[(\Db_n)_t\otimes [1]^\sharp,1]} & {[(\Db_n)_t\otimes\{1\},1]} & {[\Db_n)_t,1]\vee[1]^\sharp}
	\arrow[from=1-2, to=1-1]
	\arrow[from=1-2, to=1-3]
	\arrow[from=1-4, to=1-3]
	\arrow[from=1-4, to=1-5]
\end{tikzcd}$$
The induction hypothesis and the proposition \ref{prop:suspension preserves stricte} implies that all the objects are strict. According to proposition \ref{prop:example of a special colimit3 marked case}, whose hypotheses are provided by lemma \ref{lemma:a otimes 1 is strict}, this diagram admits a special colimit. As all the morphisms are monomorphism, this implies that $(\Db_{n+1})_t\otimes[1]^\sharp$ is strict, which concludes the proof.
\end{proof}



\begin{prop}
\label{prop:some strict marked io categories2}
If $C$ is a marked $\io$-category, $a$ a globular sum and $a^\flat\to C$ any morphism, the $\io$-categories $C\coprod_{a^\flat} a^\flat\otimes [1]^\sharp$, $C\coprod_{a^\flat} \star 1$ and $1\costar a^\flat \coprod_aC$ are strict.
\end{prop}
\begin{proof}
Using the first assertion of proposition \ref{prop:associativity of Gray amput}, the underlying $\io$-categories of $C\coprod_{a^\flat} a^\flat\otimes [1]^\sharp$, $C\coprod_{a^\flat}a^\flat \star 1$ and $1\costar a^\flat \coprod_aC$ are respectively $C^\natural\coprod_{a} a\otimes [1]$, $C^\natural\coprod_{a} a\star 1$ and $1\costar a \coprod_aC^\natural$, which are strict objects according to propositions \ref{prop:strict stuff are stable under Gray cone} and \ref{prop:strict stuff are stable under coproduc with cylinder}.
\end{proof}

\begin{theorem}
\label{theo:strictness marked}
If $C$ is strict $\io$-category, the marked $\io$-categories $C^\flat\star 1$, $1\costar C^\flat$ and $C^\flat\otimes [1]^\sharp$ are strict.
\end{theorem}
\begin{proof}
The first assertion of proposition \ref{prop:associativity of Gray amput} implies that the underlying $\io$-categories of these marked $\io$-categories respectively are $C\star 1$, $1\costar C$ and $C\otimes [1]$. As these objects are strict according to theorem \ref{theo:strictness}, this concludes the proof.
\end{proof}


\begin{prop}
\label{prop:crushing of Gray tensor is identitye marked case}
The colimit preserving endofunctor $F:\ocat\to \ocatm$, sending $[a,n]$ to the colimit of the span
$$\coprod_{k\leq n}\{k\}\leftarrow \coprod_{k\leq n}a^\flat\otimes\{k\}\to a^\flat\otimes[n]^\sharp$$
is equivalent to the functor $(\uvar)^\sharp:\ocat\to \ocatm$.
\end{prop}
\begin{proof}
This is a direct consequence of the first assertion of proposition \ref{prop:associativity of Gray amput}, of corollary \ref{cor:crushing of Gray tensor is identitye} and of the definition of the marking of the Gray tensor product for marked $\io$-categories.
\end{proof}
The last proposition implies that for any marked $\io$-category $C$ and any globular sum $a$, the simplicial $\infty$-groupoid
$$\begin{array}{rcl}
\Delta^{op}&\to &\igrd\\
~[n]~&\mapsto &\Hom([a,n]^\sharp,C)
\end{array} $$
is a $\iun$-category.


\begin{theorem}
\label{theo:formula between pullback of slice and tensor marked case}
Let $C$ be an $\io$-category. The two following canonical squares are cartesian:
% https://q.uiver.app/#q=WzAsOCxbMSwxLCJbQywxXV5cXHNoYXJwIl0sWzEsMCwiMVxcY29zdGFyIENeXFxmbGF0Il0sWzAsMSwiXFx7MFxcfSJdLFswLDAsIjEiXSxbMywxLCJbQywxXV5cXHNoYXJwIl0sWzMsMCwiQ15cXGZsYXRcXHN0YXIgMSJdLFsyLDEsIlxcezFcXH0iXSxbMiwwLCIxIl0sWzMsMV0sWzIsMF0sWzMsMl0sWzEsMF0sWzcsNV0sWzYsNF0sWzcsNl0sWzUsNF1d
\[\begin{tikzcd}
	1 & {1\costar C^\flat} & 1 & {C^\flat\star 1} \\
	{\{0\}} & {[C,1]^\sharp} & {\{1\}} & {[C,1]^\sharp}
	\arrow[from=1-1, to=1-2]
	\arrow[from=2-1, to=2-2]
	\arrow[from=1-1, to=2-1]
	\arrow[from=1-2, to=2-2]
	\arrow[from=1-3, to=1-4]
	\arrow[from=2-3, to=2-4]
	\arrow[from=1-3, to=2-3]
	\arrow[from=1-4, to=2-4]
\end{tikzcd}\]
The five squares appearing in the following canonical diagram are both cartesian and cocartesian:
% https://q.uiver.app/#q=WzAsOCxbMSwyLCIxXFxjb3N0YXIgQ15cXGZsYXQiXSxbMiwyLCJbQywxXV5cXHNoYXJwIl0sWzIsMSwiQ15cXGZsYXRcXHN0YXIgMSJdLFsxLDEsIkNeXFxmbGF0XFxvdGltZXNbMV1eXFxzaGFycCJdLFsyLDAsIjEiXSxbMSwwLCJDXlxcZmxhdFxcb3RpbWVzXFx7MFxcfSJdLFswLDEsIkNeXFxmbGF0XFxvdGltZXNcXHsxXFx9Il0sWzAsMiwiMSJdLFsyLDFdLFswLDFdLFszLDBdLFszLDJdLFs1LDRdLFs0LDJdLFs1LDNdLFs2LDNdLFs3LDBdLFs2LDddXQ==
\[\begin{tikzcd}
	& {C^\flat\otimes\{0\}} & 1 \\
	{C^\flat\otimes\{1\}} & {C^\flat\otimes[1]^\sharp} & {C^\flat\star 1} \\
	1 & {1\costar C^\flat} & {[C,1]^\sharp}
	\arrow[from=2-3, to=3-3]
	\arrow[from=3-2, to=3-3]
	\arrow[from=2-2, to=3-2]
	\arrow[from=2-2, to=2-3]
	\arrow[from=1-2, to=1-3]
	\arrow[from=1-3, to=2-3]
	\arrow[from=1-2, to=2-2]
	\arrow[from=2-1, to=2-2]
	\arrow[from=3-1, to=3-2]
	\arrow[from=2-1, to=3-1]
\end{tikzcd}\]
\end{theorem}
\begin{proof}
This is a direct consequence of the first assertion of proposition \ref{prop:associativity of Gray amput}, of theorem \ref{theo:formula between pullback of slice and tensor} and of the definition of the marking of the Gray tensor product for marked $\io$-categories.
\end{proof}




\subsection{Marked Gray deformation retract}
We provide analogous results for section \ref{subsection:Gray deformation retract}, with proofs that are entirely similar and, therefore, omitted.

\p A \wcnotion{left Gray deformation retract structure}{left or right Gray deformation retract structure} for a morphism $i:C\to D$ between marked $\io$-categories is the data of a \textit{retract}
 $r:D\to C$, a \textit{deformation} $\psi:D\otimes [1]^\sharp\to D$, and equivalences
$$ri\sim id_C~~~~~\psi_{|D\otimes\{0\}}\sim ir~~~~~\psi_{|D\otimes\{1\}}\sim id_D~~~~~ \psi_{|C\otimes[1]^\sharp}\sim i\cst_C
$$ 
A morphism $i:C\to D$ between marked $\io$-categories is a \wcnotion{left Gray deformation retract}{left or right Gray deformation retract} if it admits a left deformation retract structure. By abuse of language, such data will just be denoted by $(i,r,\psi)$.

We define dually the notion of \textit{right Gray deformation retract structure} and of \textit{right Gray deformation retract} in exchanging $0$ and $1$ in the previous definition.

We define similarly the notion of \notion{left or right deformation retract} by replacing $\otimes$ by $\times$.

\p
 A \textit{left Gray deformation retract structure for a morphism $i:f\to g$} in the $\iun$-category of arrows of $\ocatm$ is the data of a \textit{retract}
 $r:g\to f$, a \textit{deformation} $\psi:g\otimes [1]^\sharp\to g$ and equivalences
$$ri\sim id_f~~~~~\psi_{|g\otimes\{0\}}\sim ir~~~~~\psi_{|g\otimes\{1\}}\sim id_D~~~~~ \psi_{|f\otimes[1]^\sharp}\sim i\cst_C
$$ 
A morphism $i:C\to D$ between two arrows of $\ocatm$ is a \textit{left Gray deformation retract} if it admits a left deformation retract structure. By abuse of language, such data will just be denoted by $(i,r,\psi)$.

We define dually the notion of \textit{right Gray deformation retract structure} and of \textit{right Gray deformation retract} in exchanging $0$ and $1$ in the previous definition.

We define similarly the notion of \notion{left and right deformation retract} by replacing $\otimes$ by $\times$.


\begin{example}
\label{example:canonical example of left deformation retract}
Let $C$ be a marked $\io$-category. The morphism $C\otimes\{0\}\to C\otimes[1]^\sharp$ is a left Gray deformation retract.
Indeed, the retract is given by $C\otimes\Ib:C\otimes[1]^\sharp\to C\otimes\{0\}$, and the natural transformation is induced by
$$(C\otimes[1]^\sharp)\otimes[1]^\sharp\sim C\otimes([1]\times [1])^\sharp\xrightarrow{C\otimes\psi^\sharp} C\otimes[1]^\sharp$$
where the first equivalence is the one of proposition \ref{prop:associativity of Gray amput2}, and $\psi:[1]\times[1]\to [1]$ is the unique morphism sending $(\epsilon,\epsilon')$ to $\epsilon\wedge \epsilon'$.


Similarly, the morphism $C\otimes\{1\}\to C\otimes[1]^\sharp$ is a right deformation retract.
\end{example}

\p Left and right Gray retracts enjoy many stability properties: 
\begin{prop}
\label{prop:left Gray deformation retract stable under pushout}
Let $(i_a,r_a,\psi_a)$ be a natural family of left (resp. right) Gray deformation retract structures indexed by an $(\infty,1)$-category $A$.
The triple $(\colim_{A}i_a,\colim_{A}r_a,\colim_{A}\psi_a)$ is a left (resp. right) $k$-Gray deformation retract structure.
\end{prop}

\begin{prop}
\label{prop:stability under pullback}
Suppose given a diagram
% q.uiver.app/#q=WzAsNixbMCwwLCJYIl0sWzAsMSwiWCJdLFsxLDAsIlkiXSxbMSwxLCJZJyJdLFsyLDAsIloiXSxbMiwxLCJaJyJdLFswLDFdLFsyLDNdLFs0LDVdLFswLDIsInAiXSxbNCwyLCJxIiwyXSxbMSwzLCJwJyIsMl0sWzUsMywicSciXV0=
\[\begin{tikzcd}
	X & Y & Z \\
	X & {Y'} & {Z'}
	\arrow[from=1-1, to=2-1]
	\arrow[from=1-2, to=2-2]
	\arrow[from=1-3, to=2-3]
	\arrow["p", from=1-1, to=1-2]
	\arrow["q"', from=1-3, to=1-2]
	\arrow["{p'}"', from=2-1, to=2-2]
	\arrow["{q'}", from=2-3, to=2-2]
\end{tikzcd}\]
such that $p\to p'$ and $q\to q'$ are left (resp. right) Gray deformation retract. The induced square $q^*p\to (q')^*p'$ is a left (resp. right) $k$-Gray deformation retract.
\end{prop}

\begin{prop}
\label{prop:stability by composition }
If $p\to p'$ and $p'\to p''$ are two left (resp. right) Gray deformation retracts, so is $p\to p''$.
\end{prop}

\begin{prop}
\label{prop:Gray deformation retract and passage to hom}
Let $(i:C\to D,r,\psi)$ be a left (resp. right) Gray deformation structure. For any $x: C$ and $y:D$ (resp. $x: D$ and $y:C$), the morphism
$$\begin{array}{cc}
&\hom_C(x,ry)\xrightarrow{i} \hom_D(ix,iry)\xrightarrow{{\psi_y}_!} \hom_D(ix,y)\\
(resp. &\hom_C(rx,y)\xrightarrow{i} \hom_D(irx,iy)\xrightarrow{{\psi_x}_!} \hom_D(x,iy))
\end{array}
$$
is a right (resp. left) Gray deformation retract, whose retract is given by 
$$\begin{array}{cc}
&\hom_D(ix,y)\xrightarrow{r}\hom_C(x,ry)\\
(resp. &\hom_D(x,iy)\xrightarrow{r}\hom_C(rx,y))
\end{array}$$

If $(i:C\to D,r,\psi)$ is a left (resp. right) deformation structure, for any $x: C$ and $y:D$ (resp. $x: D$ and $y:C$), the two morphisms above are inverses one of each other.
\end{prop}

\begin{prop}
\label{prop:Gray deformation retract and passage to hom v2}
For any left (resp. right) Gray deformation retracts between $p$ and $p'$:
% q.uiver.app/#q=WzAsNCxbMCwwLCJDIl0sWzAsMSwiQyciXSxbMSwwLCJEIl0sWzEsMSwiRCciXSxbMCwxLCJwIiwyXSxbMCwyLCJpIl0sWzIsMywicCciXSxbMSwzLCJpJyIsMl1d
\[\begin{tikzcd}
	C & D \\
	{C'} & {D'}
	\arrow["p"', from=1-1, to=2-1]
	\arrow["i", from=1-1, to=1-2]
	\arrow["{p'}", from=1-2, to=2-2]
	\arrow["{i'}"', from=2-1, to=2-2]
\end{tikzcd}\]
and for any pair of objects $x: C$ and $y:D$ (resp. $x: D$ and $y:C$), the outer square of the following diagram
% q.uiver.app/#q=WzAsNixbMCwwLCJcXGhvbV97Q30oeCxyeSkiXSxbMiwwLCJcXGhvbV97RH0oaXgseSkiXSxbMCwxLCJcXGhvbV97Qyd9KHB4LHByJ3kpIl0sWzEsMSwiXFxob21fe0QnfShwJ2kneCxwJ2kncid5KSJdLFsyLDEsIlxcaG9tX3tEJ30ocCdpJ3gscCd5KSJdLFsxLDAsIlxcaG9tX3tEfShpeCxpcnkpIl0sWzIsMywiaSciLDJdLFszLDQsIntcXHBzaSdfe3AneX19XyEiLDJdLFswLDJdLFsxLDRdLFs1LDEsIntcXHBzaV95fV8hIl0sWzAsNSwiaSJdLFs1LDNdXQ==
\[\begin{tikzcd}
	{\hom_{C}(x,ry)} & {\hom_{D}(ix,iry)} & {\hom_{D}(ix,y)} \\
	{\hom_{C'}(px,pr'y)} & {\hom_{D'}(p'i'x,p'i'r'y)} & {\hom_{D'}(p'i'x,p'y)}
	\arrow["{i'}"', from=2-1, to=2-2]
	\arrow["{{\psi'_{p'y}}_!}"', from=2-2, to=2-3]
	\arrow[from=1-1, to=2-1]
	\arrow[from=1-3, to=2-3]
	\arrow["{{\psi_y}_!}", from=1-2, to=1-3]
	\arrow["i", from=1-1, to=1-2]
	\arrow[from=1-2, to=2-2]
\end{tikzcd}\]
(resp.% q.uiver.app/#q=WzAsNixbMCwwLCJcXGhvbV97Q30ocngseSkiXSxbMiwwLCJcXGhvbV97RH0oeCxpeSkiXSxbMCwxLCJcXGhvbV97Qyd9KHByJ3gscHkpIl0sWzEsMSwiXFxob21fe0QnfShwJ2kncid4LHAnaSd5KSJdLFsyLDEsIlxcaG9tX3tEJ30ocCd4LHAnaSd5KVxcYmlnKSJdLFsxLDAsIlxcaG9tX3tEfShpcngsaXkpIl0sWzIsMywiaSciLDJdLFszLDQsIntcXHBzaSdfe3AneH19XyEiLDJdLFswLDJdLFsxLDRdLFs1LDEsIntcXHBzaV94fV8hIl0sWzAsNSwiaSJdLFs1LDNdXQ==
\[\begin{tikzcd}
	{\hom_{C}(rx,y)} & {\hom_{D}(irx,iy)} & {\hom_{D}(x,iy)} \\
	{\hom_{C'}(pr'x,py)} & {\hom_{D'}(p'i'r'x,p'i'y)} & {\hom_{D'}(p'x,p'i'y)\big)}
	\arrow["{i'}"', from=2-1, to=2-2]
	\arrow["{{\psi'_{p'x}}_!}"', from=2-2, to=2-3]
	\arrow[from=1-1, to=2-1]
	\arrow[from=1-3, to=2-3]
	\arrow["{{\psi_x}_!}", from=1-2, to=1-3]
	\arrow["i", from=1-1, to=1-2]
	\arrow[from=1-2, to=2-2]
\end{tikzcd}\]
is a left (resp. right) Gray deformation retract, whose retract is given by
% q.uiver.app/#q=WzAsOCxbMiwwLCIocmVzcC5cXGhvbV97RH0oeCxpeSkiXSxbMiwxLCJcXGhvbV97RCd9KHAneCxwJ2kneSkiXSxbMywwLCJcXGhvbV97Q30ocngseSkiXSxbMywxLCJcXGhvbV97Qyd9KHByJ3gscHkpXFxiaWcpIl0sWzAsMCwiXFxob21fe0R9KGl4LHkpIl0sWzEsMCwiXFxob21fe0N9KHgscnkpIl0sWzAsMSwiXFxob21fe0QnfShwJ2kneCxwJ3kpXFxiaWcpIl0sWzEsMSwiXFxob21fe0MnfShweCxwcid5KSJdLFswLDFdLFswLDIsInIiXSxbMSwzLCJyJyIsMl0sWzIsM10sWzYsNywiciciLDJdLFs0LDUsInIiXSxbNSw3XSxbNCw2XV0=
\[\begin{tikzcd}
	{\hom_{D}(ix,y)} & {\hom_{C}(x,ry)} & {(resp.\hom_{D}(x,iy)} & {\hom_{C}(rx,y)} \\
	{\hom_{D'}(p'i'x,p'y)\big)} & {\hom_{C'}(px,pr'y)} & {\hom_{D'}(p'x,p'i'y)} & {\hom_{C'}(pr'x,py)\big)}
	\arrow[from=1-3, to=2-3]
	\arrow["r", from=1-3, to=1-4]
	\arrow["{r'}"', from=2-3, to=2-4]
	\arrow[from=1-4, to=2-4]
	\arrow["{r'}"', from=2-1, to=2-2]
	\arrow["r", from=1-1, to=1-2]
	\arrow[from=1-2, to=2-2]
	\arrow[from=1-1, to=2-1]
\end{tikzcd}\]
If $p\to p'$ is a left (resp. right) deformation structure, for any $x: C$ and $y:D$ (resp. $x: D$ and $y:C$), the two morphisms above are inverses one of each other.
\end{prop}





\begin{prop}
\label{prop:suspension of left Gray deformation retract}
If $i$ is a left Gray deformation retract, $[i,1]$ is a right Gray deformation retract. Conversely, if $i$ is a right Gray deformation retract, $[i,1]$ is a left Gray deformation retract morphism.
\end{prop}


\begin{prop}
\label{prop:when glob inclusion are left Gray deformation}
Let $a$ be a globular sum of dimension $(n+1)$. We denote by $s_n(a)$ and $t_n(a)$ the globular sum defined in \ref{para:definition of source et but}. If $n$ is even, $s_n(a)^\flat\to a^{\sharp_n}$ is a left Gray deformation retract, and $t_n(a)^\flat\to a^{\sharp_n}$ is a right Gray deformation retract. Dually, if $n$ is odd, $t_n(a)^\flat\to a^{\sharp_n}$ is a left Gray deformation retract, and $s_n(a)^\flat\to a^{\sharp_n}$ is a right Gray deformation retract.
\end{prop}


\begin{prop}
\label{prop:exemple of right deformation retract}
Let $i:C\to D$ be a left Gray deformation retract and $A$ a marked $\io$-category.
The morphism $A\times i$ is a left Gray deformation retract. 
\end{prop}
\begin{proof}
Let $r$ and $\psi$ be retracts and deformation of $i$.
We define $\psi_A$ as the composite
$$(A\times D)\otimes[1]^\sharp\to A\times (D\otimes[1]^\sharp)\xrightarrow{A\times \psi} A\times D$$
Remark that the triple $(A\times i,A\times r,\psi_A)$ is a left Gray deformation retract structure.
\end{proof}




\begin{prop}
\label{prop:retraction criter}
Let $(i:[C,1]\to D,r,\phi)$ be a left deformation retract structure. The following natural square is cartesian:
% q.uiver.app/#q=WzAsNCxbMSwwLCJcXHVIb20oWzFdXlxcc2hhcnAsRCkiXSxbMSwxLCJEIl0sWzAsMSwiW0MsMV0iXSxbMCwwLCJEIl0sWzIsMSwiaSIsMl0sWzMsMiwiciIsMl0sWzAsMSwiKGleLV8wKV8hIl0sWzMsMCwiXFxwaGkiXV0=
\[\begin{tikzcd}
	D & {\uHom([1]^\sharp,D)} \\
	{[C,1]} & D
	\arrow["i"', from=2-1, to=2-2]
	\arrow["r"', from=1-1, to=2-1]
	\arrow["{(i^-_0)_!}", from=1-2, to=2-2]
	\arrow["\phi", from=1-1, to=1-2]
\end{tikzcd}\]
\end{prop}
\begin{proof}
We set $P:=[C,1]\times_{D}\uHom([1]^\sharp,D)$ and $\psi:D\to P$ the induced morphism.
The proposition \ref{prop:example of a special colimit marked case} implies that $\hom_{ P}(\psi(x),\psi(y))$ is the limit of the diagram:
% https://q.uiver.app/#q=WzAsNCxbMCwwLCJcXGhvbV97W0MsMV19KHJ4LHJ5KSJdLFsxLDAsIlxcaG9tX3tEfShpcngsaXJ5KSJdLFsyLDAsIlxcaG9tX3tEfShpcngseSkiXSxbMywwLCJcXGhvbV97RH0oeCx5KSJdLFswLDEsImkiXSxbMSwyLCJ7XFxwaGlfeX1fISJdLFszLDIsIntcXHBoaV94fV8hIiwyXV0=
\[\begin{tikzcd}
	{\hom_{[C,1]}(rx,ry)} & {\hom_{D}(irx,iry)} & {\hom_{D}(irx,y)} & {\hom_{D}(x,y)}
	\arrow["i", from=1-1, to=1-2]
	\arrow["{{\phi_y}_!}", from=1-2, to=1-3]
	\arrow["{{\phi_x}_!}"', from=1-4, to=1-3]
\end{tikzcd}\]
The proposition \ref{prop:Gray deformation retract and passage to hom} then implies that the canonical morphism
$$\hom_D( x,y)\to \hom_{ P}(\psi(x),\psi(y))$$
is an equivalence.


The morphism $\psi$ is then fully faithful. According to proposition \ref{prop:fully faithful plus surjective on objet marked case}, it remains to show that it induces a surjection on objects. For this, let $v:x\to y$ be an element of $P$. As the only marked $1$-cells in $[C,1]$ are equivalences, $r(v)$ is an equivalence. The morphism 
$$[1]^\sharp\times[1]^\sharp\xrightarrow{v\times [1]^\sharp} D\times [1]^\sharp\xrightarrow{\phi} D$$
 induces a square in $D$ of shape
% https://q.uiver.app/#q=WzAsNCxbMSwwLCJ4Il0sWzEsMSwieSJdLFswLDAsImlyeCJdLFswLDEsImlyeSJdLFszLDEsIlxccGhpKHkpIiwyXSxbMCwxLCJ2Il0sWzIsMCwiXFxzaW0iXSxbMiwzLCJcXHNpbSJdLFsyLDMsImlyKHYpIiwyLHsic3R5bGUiOnsiYm9keSI6eyJuYW1lIjoibm9uZSJ9LCJoZWFkIjp7Im5hbWUiOiJub25lIn19fV1d
\[\begin{tikzcd}
	irx & x \\
	iry & y
	\arrow["{\phi(y)}"', from=2-1, to=2-2]
	\arrow["v", from=1-2, to=2-2]
	\arrow["\sim", from=1-1, to=1-2]
	\arrow["\sim", from=1-1, to=2-1]
	\arrow["{ir(v)}"', draw=none, from=1-1, to=2-1]
\end{tikzcd}\]
where all the arrows labeled by $\sim$ are equivalences. This implies that $v\sim \phi(y)$ and the morphism $\psi$ is then surjective on objects. This concludes the proof.
\end{proof}







\section{Cartesian fibrations}
\subsection{Left and right cartesian fibrations}
\label{subsection Left and right cartesian fibration}

\p We denote by \wcnotation{$\I$}{(i@$\I$} the set of morphisms of shape $X\otimes \{0\}\to X\otimes [1]^\sharp$ for $X$ being either $\Db_n^\flat$ or $(\Db_n)_t$. A morphism is \wcnotion{initial}{initial morphism} if it is in $\widehat{\I}$. Conversely, we denote by \wcnotation{$\F$}{(f@$\F$} the set of morphisms of shape $X\otimes \{1\}\to X\otimes [1]^\sharp$ for $X$ being either $\Db_n^\flat$ or $(\Db_n)_t$. A morphism is \wcnotion{final}{final morphism} if it is in $\widehat{\F}$.

Initial and final morphisms are stable under colimits, retract, composition and  left cancellation according to the result of section \ref{section:Factorization system}.  

The proposition \ref{prop:otimes et op marked version} implies that the full duality $(\uvar)^\circ$ sends final (resp. initial) morphisms to initial (resp. final) morphisms.

\begin{example}
\label{exe:the easiest example of initial and finla morphism}
By stability of initial and final morphisms by colimits, for any marked $\io$-category $C$, $C\otimes\{0\}\to C\otimes[1]^\sharp$ is initial, and $C\otimes\{1\}\to C\otimes[1]^\sharp$ is final.
\end{example}





\begin{prop}
\label{prop:left Gray deformation retract are initial}
Left Gray deformation retracts (resp. left deformation retract) are initial and right Gray deformation retracts (resp. right deformation retract) are final. 
\end{prop}
\begin{proof} 
Let $i:C\to D$ be a left Gray deformation retract. The diagram
% q.uiver.app/#q=WzAsNixbMCwwLCJDIl0sWzAsMSwiRFxcb3RpbWVzXFx7MVxcfSJdLFsxLDAsIkRcXG90aW1lc1xcezBcXH0iXSxbMSwxLCJEXFxvdGltZXMgWzFdXlxcc2hhcnAiXSxbMiwwLCJDIl0sWzIsMSwiRCJdLFswLDEsImkiLDJdLFsxLDNdLFsyLDNdLFszLDUsIlxccHNpIiwyXSxbMCwyLCJpIl0sWzIsNCwiciJdLFs0LDUsImkiXV0=
\[\begin{tikzcd}
	C & {D\otimes\{0\}} & C \\
	{D\otimes\{1\}} & {D\otimes [1]^\sharp} & D
	\arrow["i"', from=1-1, to=2-1]
	\arrow[from=2-1, to=2-2]
	\arrow[from=1-2, to=2-2]
	\arrow["\psi"', from=2-2, to=2-3]
	\arrow["i", from=1-1, to=1-2]
	\arrow["r", from=1-2, to=1-3]
	\arrow["i", from=1-3, to=2-3]
\end{tikzcd}\]
expresses $i$ as a retract of $D\otimes \{0\}\to D\otimes [1]^\sharp$, which is an initial morphism according to example \ref{exe:the easiest example of initial and finla morphism}. The morphism  $i$ is then initial. 

As left deformation retracts are left Gray deformation retracts, they are initial.
The case of right (Gray) deformation retracts follows by duality.
\end{proof}

\begin{cor}
\label{cor:when glob inclusion are final and initial}
Let $a$ be a globular sum of dimension $(n+1)$. We denote by $s_n(a)$ and $t_n(a)$ the globular sum defined in \ref{para:definition of source et but}. If $n$ is even, $s_n(a)^\flat\to a^{\sharp_n}$ is initial, and $t_n(a)^\flat\to a^{\sharp_n}$ is final. Dually, if $n$ is odd, $t_n(a)^\flat\to a^{\sharp_n}$ is initial, and $s_n(a)^\flat\to a^{\sharp_n}$ is final
\end{cor}
\begin{proof}
This is a direct consequence of propositions \ref{prop:when glob inclusion are left Gray deformation} and \ref{prop:left Gray deformation retract are initial}.
\end{proof}

\begin{prop}
\label{prop:trivialization are initial}
For any $n$, the morphism $\Ib_n:(\Db_{n+1})_t\to \Db_n^\flat$ is both initial and final.
\end{prop}
\begin{proof}
According to lemma \ref{cor:when glob inclusion are final and initial} there exists $\alpha\in\{-,+\}$ such that 
 $i_{n}^\alpha:(\Db_n)^\flat\to (\Db_{n+1})_t$ is initial.
 As $\Ib_n$ is a retraction of this morphism, and as initial morphisms are closed under left cancellation according to proposition \ref{prop:closed under colimit imply saturated}, $\Ib_n$ is initial. The second case follows by duality.
\end{proof}
These morphisms will be called the \wcnotion{marked trivializations}{marked trivialization}.


\begin{prop}
\label{prop:cotimes 1 to c is a trivialization}
Let $C$ be a marked $\io$-category.
The morphism $C\otimes[1]^\sharp\to C$ is in the smallest cocomplete $\infty$-groupoid of morphism containing the marked trivialization. In particular, this morphism is both initial and final. 
\end{prop}
\begin{proof}
We denote $K$ the smallest cocomplete $\infty$-groupoid of morphisms containing the marked trivializations.
As the $\infty$-groupoid of objects $C$ fulfilling the wanted property is closed by colimits, it is sufficient to demonstrate the result for $C$ being either $\Db_n^\flat$ or $(\Db_{n+1})_t$ for $n$ an integer. We will then proceed by induction. Suppose first that $C$ is $\Db_0^\flat$ or $(\Db_1)_t$. The first case is trivial, for the second one, remark that $(\Db_1)_t\otimes[1]^\sharp\sim [1]^\sharp\times[1]^\sharp\to [1]^\sharp$ is the horizontal colimit of the diagram
% https://q.uiver.app/#q=WzAsNixbMCwwLCJbMl1eXFxzaGFycCJdLFsxLDAsIlsxXV5cXHNoYXJwIl0sWzIsMCwiWzJdXlxcc2hhcnAiXSxbMCwxLCJbMV1eXFxzaGFycCJdLFsxLDEsIlsxXV5cXHNoYXJwIl0sWzIsMSwiWzFdXlxcc2hhcnAiXSxbMCwzLCJzXjAiLDJdLFsxLDRdLFsyLDUsInNeMSJdLFsxLDBdLFsxLDJdLFs0LDNdLFs0LDVdXQ==
\[\begin{tikzcd}
	{[2]^\sharp} & {[1]^\sharp} & {[2]^\sharp} \\
	{[1]^\sharp} & {[1]^\sharp} & {[1]^\sharp}
	\arrow["{s^0}"', from=1-1, to=2-1]
	\arrow[from=1-2, to=2-2]
	\arrow["{s^1}", from=1-3, to=2-3]
	\arrow[from=1-2, to=1-1]
	\arrow[from=1-2, to=1-3]
	\arrow[from=2-2, to=2-1]
	\arrow[from=2-2, to=2-3]
\end{tikzcd}\]
and is then in $K$. Suppose now the result is true at the stage $(n-1)$. Let $C$ be $\Db_n^\flat$ (resp.$(\Db_{n+1})_t$). We set $D:=\Db_{n-1}^\flat$ (resp. $D:=(\Db_{n})_t$). We then have $C\sim [D,1]$. The equation \eqref{eq:eq for cylinder marked version} implies that $C\otimes[1]^\sharp\to C$ is the horizontal colimit of the diagram:
% https://q.uiver.app/#q=WzAsMTAsWzAsMCwiWzFdXlxcc2hhcnBcXHZlZVtELDFdIl0sWzEsMCwiW0RcXG90aW1lc1xcezBcXH0sMV0iXSxbMywwLCJbRFxcb3RpbWVzXFx7MVxcfSwxXSJdLFs0LDAsIltELDFdXFx2ZWVbMV1eXFxzaGFycCJdLFsyLDAsIltEXFxvdGltZXNbMV1eXFxzaGFycCwxXSJdLFswLDEsIltELDFdIl0sWzEsMSwiW0QsMV0iXSxbMiwxLCJbRCwxXSJdLFszLDEsIltELDFdIl0sWzQsMSwiW0QsMV0iXSxbNiw1XSxbNiw3XSxbOCw3XSxbOCw5XSxbMCw1XSxbNCw3XSxbMiw4XSxbMyw5XSxbMSwwXSxbMSw0XSxbMiw0XSxbMiwzXSxbMSw2XV0=
\[\begin{tikzcd}
	{[1]^\sharp\vee[D,1]} & {[D\otimes\{0\},1]} & {[D\otimes[1]^\sharp,1]} & {[D\otimes\{1\},1]} & {[D,1]\vee[1]^\sharp} \\
	{[D,1]} & {[D,1]} & {[D,1]} & {[D,1]} & {[D,1]}
	\arrow[from=2-2, to=2-1]
	\arrow[from=2-2, to=2-3]
	\arrow[from=2-4, to=2-3]
	\arrow[from=2-4, to=2-5]
	\arrow[from=1-1, to=2-1]
	\arrow[from=1-3, to=2-3]
	\arrow[from=1-4, to=2-4]
	\arrow[from=1-5, to=2-5]
	\arrow[from=1-2, to=1-1]
	\arrow[from=1-2, to=1-3]
	\arrow[from=1-4, to=1-3]
	\arrow[from=1-4, to=1-5]
	\arrow[from=1-2, to=2-2]
\end{tikzcd}\]
The leftest and rightest morphisms obviously are in $K$.
As marked trivializations are stable by suspension, the induction hypothesis implies that the middle vertical morphisms of the previous diagram are in $K$, which concludes the proof.
\end{proof}
\begin{prop}
\label{prop:cotimes 1 to ctimes 1 is a trivialization}
Let $C$ be a marked $\io$-category.
The morphism $C\otimes[1]^\sharp\to C\times [1]^\sharp$ is in the smallest cocomplete $\infty$-groupoid of morphism containing the marked trivializations. In particular, this morphism is both initial and final. 
\end{prop}
\begin{proof}
We denote $K$ the smallest cocomplete $\infty$-groupoid of morphisms containing the marked trivializations.
As the $\infty$-groupoid of objects $C$ fulfilling the wanted property is closed by colimits, it is sufficient to demonstrate the result for $C$ being either $\Db_n^\flat$ or $(\Db_{n+1})_t$ for $n$ an integer. If $C$ is either $(\Db_0)^\flat$ or $(\Db_1)_t$ the considered morphism is the identity. We then suppose that $n>0$. Let $C$ be $\Db_n^\flat$ (resp.$(\Db_{n+1})_t$). We set $D:=\Db_{n-1}^\flat$ (resp. $D:=(\Db_{n})_t$). We then have $C\sim [D,1]$. The equation \eqref{eq:eq for cylinder marked version} and the equation given in \ref{prop:example of a special colimit marked case} imply that $C\otimes[1]^\sharp\to C\times[1]^\sharp$ is the horizontal colimit of the diagram:
% https://q.uiver.app/#q=WzAsMTAsWzAsMCwiWzFdXlxcc2hhcnBcXHZlZVtELDFdIl0sWzEsMCwiW0RcXG90aW1lc1xcezBcXH0sMV0iXSxbMywwLCJbRFxcb3RpbWVzXFx7MVxcfSwxXSJdLFs0LDAsIltELDFdXFx2ZWVbMV1eXFxzaGFycCJdLFsyLDAsIltEXFxvdGltZXNbMV1eXFxzaGFycCwxXSJdLFswLDEsIlsxXV5cXHNoYXJwXFx2ZWVbRCwxXSJdLFsxLDEsIltELDFdIl0sWzIsMSwiW0QsMV0iXSxbMywxLCJbRCwxXSJdLFs0LDEsIltELDFdXFx2ZWVbMV1eXFxzaGFycCJdLFs2LDVdLFs2LDddLFs4LDddLFs4LDldLFswLDVdLFs0LDddLFsyLDhdLFszLDldLFsxLDBdLFsxLDRdLFsyLDRdLFsyLDNdLFsxLDZdXQ==
\[\begin{tikzcd}
	{[1]^\sharp\vee[D,1]} & {[D\otimes\{0\},1]} & {[D\otimes[1]^\sharp,1]} & {[D\otimes\{1\},1]} & {[D,1]\vee[1]^\sharp} \\
	{[1]^\sharp\vee[D,1]} & {[D,1]} & {[D,1]} & {[D,1]} & {[D,1]\vee[1]^\sharp}
	\arrow[from=2-2, to=2-1]
	\arrow[from=2-2, to=2-3]
	\arrow[from=2-4, to=2-3]
	\arrow[from=2-4, to=2-5]
	\arrow[from=1-1, to=2-1]
	\arrow[from=1-3, to=2-3]
	\arrow[from=1-4, to=2-4]
	\arrow[from=1-5, to=2-5]
	\arrow[from=1-2, to=1-1]
	\arrow[from=1-2, to=1-3]
	\arrow[from=1-4, to=1-3]
	\arrow[from=1-4, to=1-5]
	\arrow[from=1-2, to=2-2]
\end{tikzcd}\]
The proposition \ref{prop:cotimes 1 to c is a trivialization} then states that the middle vertical morphisms of the previous diagram are in $K$, which concludes the proof.
\end{proof}



\begin{prop}
\label{prop:suspension of initial}
If $i$ is an initial morphism, $[i,1]$ is a final morphism. Conversely, if $i$ is a final morphism, $[i,1]$ is an initial morphism.
\end{prop}
\begin{proof}
As the suspension preserves colimits, we can restrict to the case where $i$ is of shape $C\otimes\{0\}\to C\otimes[1]^\sharp$, and this is then a consequence of propositions \ref{prop:suspension of left Gray deformation retract} and \ref{prop:left Gray deformation retract are initial}.
\end{proof}



\begin{prop}
\label{prop:initial stable under product}
For any marked $\io$-category $K$, the functor $K\times\uvar:\ocatm\to \ocatm$ preserves initial and final morphisms. 
\end{prop}
\begin{proof}
The functor $K\times\uvar$ preserves colimits and this is then enough to show that 
it preserves left and right Gray deformation retracts, which is a consequence of proposition \ref{prop:exemple of right deformation retract}.
\end{proof}

\p
A \wcnotion{left cartesian fibration}{left or right cartesian fibration} is a morphism $f:C\to D$ between marked $\io$-categories having the unique right lifting property against initial morphisms.
A \textit{right cartesian fibration} is a morphism $f:C\to D$ between marked $\io$-categories having the unique right lifting property against final morphisms.

Left and right cartesian fibrations are stable under limits, retract, composition and  right cancellation according to the result of section \ref{section:Factorization system}.  



The proposition \ref{prop:otimes et op marked version} implies that the full duality $(\uvar)^\circ$ sends left (resp. right) cartesian fibrations to right (resp. left) cartesian fibrations.


The construction \ref{cons:small object argument} produces a unique factorization system between initial (resp final) morphisms and left (resp. right) cartesian fibrations. If $f:A\to B$ is any morphism, we will denote by $\Fb f: A'\to B$ the left cartesian fibration obtained via this factorization system. 

\begin{prop}
\label{prop:left fib over flat}
If $f:C\to D^\flat$ is a left cartesian fibration, then the canonical morphism $(C^\natural)^\flat\to C$ is an equivalence. Conversely, any morphism $C^\flat \to D^\flat$ is a left cartesian fibration.
\end{prop}
\begin{proof}
The first assertion is a consequence of the fact that marked trivializations are initial. The second assertion is a direct consequence of proposition \ref{prop:cotimes 1 to c is a trivialization}.
\end{proof}

\begin{prop}
\label{prop:cartesian fibration between arrow}
Let $p:X\to C$ be a morphism, and $x,y$ two objects of $X$. Then, if $p$ is a right (resp. left) cartesian fibration, the induced morphism $p:\hom_X(x,y)\to \hom_C(x,y)$ is a left (resp. right) cartesian fibration.
\end{prop}
\begin{proof}
This is a direct consequence of proposition \ref{prop:suspension of initial}.
\end{proof}






\begin{prop}
\label{prop:left Gray transfomration stable under pullback along cartesian fibration}
Consider a cocartesian square
% q.uiver.app/#q=WzAsNixbMSwxLCJZJyJdLFsyLDAsIlgiXSxbMiwxLCJZIl0sWzAsMCwiWCcnIl0sWzEsMCwiWCciXSxbMCwxLCJZJyciXSxbMSwyLCJwIiwyXSxbMyw1LCJwJyciLDJdLFs0LDAsInAnIiwyXSxbMyw0LCJqIl0sWzQsMV0sWzUsMCwiaSIsMl0sWzAsMl0sWzMsMCwiIiwyLHsic3R5bGUiOnsibmFtZSI6ImNvcm5lciJ9fV0sWzQsMiwiIiwyLHsic3R5bGUiOnsibmFtZSI6ImNvcm5lciJ9fV1d
\[\begin{tikzcd}
	{X''} & {X'} & X \\
	{Y''} & {Y'} & Y
	\arrow["p"', from=1-3, to=2-3]
	\arrow["{p''}"', from=1-1, to=2-1]
	\arrow["{p'}"', from=1-2, to=2-2]
	\arrow["j", from=1-1, to=1-2]
	\arrow[from=1-2, to=1-3]
	\arrow["i"', from=2-1, to=2-2]
	\arrow[from=2-2, to=2-3]
	\arrow["\lrcorner"{anchor=center, pos=0.125}, draw=none, from=1-1, to=2-2]
	\arrow["\lrcorner"{anchor=center, pos=0.125}, draw=none, from=1-2, to=2-3]
\end{tikzcd}\]
If $p$ is a left (resp. right) cartesian fibration and $i$ is a right (resp. left) Gray deformation retract, then $p''\to p'$ is a right (resp. left) Gray deformation retract. 
Similarly, if $p$ is a left (resp. right) cartesian fibration and $i$ is a right (resp. left) deformation retract, then $p''\to p'$ is a right (resp. left) deformation retract. 
\end{prop}
\begin{proof}
We suppose that $p$ is a right cartesian fibration. By stability under pullbacks, so is $p'$.
Let $(i:C\to D,r,\phi)$ be a left Gray deformation retract structure.
We define the morphism $\psi$ as the lift of the following commutative square:
% q.uiver.app/#q=WzAsNSxbMiwxLCJZJyJdLFswLDAsIlgnJ1xcb3RpbWVzWzFdXlxcc2hhcnBcXGN1cCBYJ1xcb3RpbWVzXFx7MFxcfSJdLFsyLDAsIlgnIl0sWzAsMSwiWCdcXG90aW1lc1sxXV5cXHNoYXJwIl0sWzEsMSwiWScnXFxvdGltZXNbMV1eXFxzaGFycCJdLFsxLDNdLFsyLDAsInAnIl0sWzEsMiwiKFgnJ1xcb3RpbWVzXFxJYilcXGN1cCBpZCJdLFszLDRdLFs0LDBdLFszLDIsIlxccHNpIiwxLHsic3R5bGUiOnsiYm9keSI6eyJuYW1lIjoiZG90dGVkIn19fV1d
\[\begin{tikzcd}
	{X''\otimes[1]^\sharp\cup X'\otimes\{0\}} && {X'} \\
	{X'\otimes[1]^\sharp} & {Y''\otimes[1]^\sharp} & {Y'}
	\arrow[from=1-1, to=2-1]
	\arrow["{p'}", from=1-3, to=2-3]
	\arrow["{(X''\otimes\Ib)\cup id}", from=1-1, to=1-3]
	\arrow[from=2-1, to=2-2]
	\arrow[from=2-2, to=2-3]
	\arrow["\psi"{description}, dotted, from=2-1, to=1-3]
\end{tikzcd}\]
Remark that the restriction of $\psi$ to $X'\otimes\{1\}$ factors through $X''$ and then defines a retract $s:Y\to X$ of $j$. This provides a right Gray deformation structure for $p\to p''$. We proceed similarly for the dual case. To verify the second claim, one may utilize the same proof, exchanging $\otimes$ with $\times$.
\end{proof}

\begin{cor}
\label{cor:morphism between is an equivalence when equivalence on fiber}
Let $p:X\to B^\sharp$ and $q:Y\to B^\sharp$ be two left cartesian fibrations and $\phi:p\to q$ a morphism over $ B^\sharp$. The morphism $\phi$ is an equivalence if and only if, for any object $b$ of $B$, the induced morphism $\{b\}^*\phi :\{b\}^*X\to \{b\}^*Y$ is an equivalence.
\end{cor}
\begin{proof}
As $\tiPsh{\Theta}$ is locally cartesian closed, pullback commutes with special colimits, and as every $\io$-category is the special colimit of its $k$-truncation for $k\in \Nb$ according to proposition \ref{prop:example of a special colimit 2 marked case} , one can suppose that $B$ is a marked $(\infty,k)$-category for $k<\omega$, and we then proceed by induction on $k$. 
Suppose then the result is true for $(\infty,k)$-categories and that $B$ is an $(\infty,k+1)$-category. Remark first that $\phi$ induces an equivalence between $\tau_0(X)$ and $\tau_0(Y)$. 

Let $x$ and $y$ be two objects of $X$ and
 $v:[1]^\sharp\to B^\sharp$ be a cell whose source is $px$ and target $py$. This induces cartesian squares
% q.uiver.app/#q=WzAsOSxbMiwwLCJYIl0sWzIsMSwiWSJdLFsyLDIsIkJeXFxzaGFycCJdLFswLDIsIlxcezFcXH0iXSxbMSwyLCJbMV1eXFxzaGFycCJdLFswLDEsIllfMSJdLFswLDAsIlhfMSJdLFsxLDAsIlhfdiJdLFsxLDEsIllfdiJdLFs2LDUsIlxccGhpXzEiXSxbMCwxLCJcXHBoaSJdLFs3LDgsIlxccGhpX3YiXSxbMSwyXSxbOCw0XSxbNSwzXSxbMyw0XSxbNCwyLCJ2IiwyXSxbNSw4XSxbNiw3XSxbOCwxXSxbNywwXSxbNiw4LCIiLDEseyJzdHlsZSI6eyJuYW1lIjoiY29ybmVyIn19XSxbNywxLCIiLDEseyJzdHlsZSI6eyJuYW1lIjoiY29ybmVyIn19XSxbOCwyLCIiLDEseyJzdHlsZSI6eyJuYW1lIjoiY29ybmVyIn19XSxbNSw0LCIiLDEseyJzdHlsZSI6eyJuYW1lIjoiY29ybmVyIn19XV0=
\[\begin{tikzcd}
	{X_1} & {X_v} & X \\
	{Y_1} & {Y_v} & Y \\
	{\{1\}} & {[1]^\sharp} & {B^\sharp}
	\arrow["{\phi_1}", from=1-1, to=2-1]
	\arrow["\phi", from=1-3, to=2-3]
	\arrow["{\phi_v}", from=1-2, to=2-2]
	\arrow[from=2-3, to=3-3]
	\arrow[from=2-2, to=3-2]
	\arrow[from=2-1, to=3-1]
	\arrow[from=3-1, to=3-2]
	\arrow["v"', from=3-2, to=3-3]
	\arrow[from=2-1, to=2-2]
	\arrow[from=1-1, to=1-2]
	\arrow[from=2-2, to=2-3]
	\arrow[from=1-2, to=1-3]
	\arrow["\lrcorner"{anchor=center, pos=0.125}, draw=none, from=1-1, to=2-2]
	\arrow["\lrcorner"{anchor=center, pos=0.125}, draw=none, from=1-2, to=2-3]
	\arrow["\lrcorner"{anchor=center, pos=0.125}, draw=none, from=2-2, to=3-3]
	\arrow["\lrcorner"{anchor=center, pos=0.125}, draw=none, from=2-1, to=3-2]
\end{tikzcd}\]
By hypothesis, $\phi_1$ is an equivalence.
According to proposition \ref{prop:left Gray transfomration stable under pullback along cartesian fibration}, $\phi_1\to \phi_v$ is a right deformation retract, and according to proposition \ref{prop:Gray deformation retract and passage to hom}, this induces a cartesian square
% q.uiver.app/#q=WzAsNCxbMSwwLCJcXGhvbV97WF92fSh4LHkpIl0sWzEsMSwiXFxob21fe1lfdn0oXFxwc2kgeCxcXHBzaSB5KSJdLFswLDAsIlxcaG9tX3tYXzF9KHgscnkpIl0sWzAsMSwiXFxob21fe1lfMX0oXFxwc2kgeCxyXFxwc2kgeSkiXSxbMCwxXSxbMiwzXSxbMiwwXSxbMywxXSxbMiwxLCIiLDEseyJzdHlsZSI6eyJuYW1lIjoiY29ybmVyIn19XV0=
\[\begin{tikzcd}
	{\hom_{X_1}(x,ry)} & {\hom_{X_v}(x,y)} \\
	{\hom_{Y_1}(\psi x,r\psi y)} & {\hom_{Y_v}(\psi x,\psi y)}
	\arrow[from=1-2, to=2-2]
	\arrow[from=1-1, to=2-1]
	\arrow[from=1-1, to=1-2]
	\arrow[from=2-1, to=2-2]
	\arrow["\lrcorner"{anchor=center, pos=0.125}, draw=none, from=1-1, to=2-2]
\end{tikzcd}\]
where horizontal morphisms are equivalences. By hypothesis, the left vertical one is an equivalence, and then, by two out of three, so is the right vertical one. 

We then have, for any $1$-cell $v$, the following cartesian squares
% https://q.uiver.app/#q=WzAsNixbMCwwLCJcXGhvbV97WF92fSh4LHkpIl0sWzAsMSwiXFxob21fe1lfdn0oXFxwc2kgeCxcXHBzaSB5KSJdLFsxLDAsIlxcaG9tX3tYfSh4LHkpIl0sWzEsMSwiXFxob21fe1l9KFxccHNpIHgsXFxwc2kgeSkiXSxbMSwyLCJcXGhvbV97Qn0ocHgscHkpXlxcc2hhcnAiXSxbMCwyLCJcXHt2XFx9Il0sWzAsMSwiXFxzaW0iLDJdLFs1LDRdLFszLDRdLFsyLDNdLFsxLDNdLFsxLDVdLFswLDJdXQ==
\[\begin{tikzcd}
	{\hom_{X_v}(x,y)} & {\hom_{X}(x,y)} \\
	{\hom_{Y_v}(\psi x,\psi y)} & {\hom_{Y}(\psi x,\psi y)} \\
	{\{v\}} & {\hom_{B}(px,py)^\sharp}
	\arrow["\sim"', from=1-1, to=2-1]
	\arrow[from=3-1, to=3-2]
	\arrow[from=2-2, to=3-2]
	\arrow[from=1-2, to=2-2]
	\arrow[from=2-1, to=2-2]
	\arrow[from=2-1, to=3-1]
	\arrow[from=1-1, to=1-2]
\end{tikzcd}\]
where the arrow labeled  by $\sim$ is an equivalence. As $\hom_{B}(px,py)^\sharp$ is an $(\infty,k)$-category, the induction hypothesis implies that $\hom_{X}(x,y)\to \hom_{Y}(\psi x,\psi y)$  is an equivalence. The morphism $\phi$ is then fully faithful, and as we already know that it is essentially surjective, this concludes the proof.
\end{proof}





\p We have by construction a factorization system in initial morphism followed by left cartesian fibration, and another one in final morphism followed by right cartesian fibration. We are willing to find an explicit expression for such factorization in some easy cases. We then fix $i:C^\flat \to D$ with $D$ being any marked $\io$-category.

If $C^\flat\to D$ is a functor between marked $\io$-categories, we define $D_{/C^{\flat}}$ and $D_{C^{\flat}/}$ as the following pullbacks 
% q.uiver.app/#q=WzAsOCxbMCwwLCJEX3tDXntcXGZsYXR9L30iXSxbMSwwLCJEXntbMV1eXFxzaGFycH0iXSxbMSwxLCJEIl0sWzAsMSwiQ157XFxmbGF0fSJdLFs0LDAsIkRee1sxXV5cXHNoYXJwfSJdLFs0LDEsIkQiXSxbMywxLCJDXntcXGZsYXR9Il0sWzMsMCwiRF97L0Nee1xcZmxhdH19Il0sWzEsMiwiKGlfMF4tKV8hIl0sWzMsMl0sWzAsM10sWzAsMV0sWzAsMiwiIiwxLHsic3R5bGUiOnsibmFtZSI6ImNvcm5lciJ9fV0sWzQsNSwiKGlfMV4tKV8hIl0sWzcsNl0sWzYsNV0sWzcsNF0sWzcsNSwiIiwxLHsic3R5bGUiOnsibmFtZSI6ImNvcm5lciJ9fV1d
\[\begin{tikzcd}
	{D_{C^{\flat}/}} & {D^{[1]^\sharp}} && {D_{/C^{\flat}}} & {D^{[1]^\sharp}} \\
	{C^{\flat}} & D && {C^{\flat}} & D
	\arrow["{(i_0^-)_!}", from=1-2, to=2-2]
	\arrow[from=2-1, to=2-2]
	\arrow[from=1-1, to=2-1]
	\arrow[from=1-1, to=1-2]
	\arrow["\lrcorner"{anchor=center, pos=0.125}, draw=none, from=1-1, to=2-2]
	\arrow["{(i_1^-)_!}", from=1-5, to=2-5]
	\arrow[from=1-4, to=2-4]
	\arrow[from=2-4, to=2-5]
	\arrow[from=1-4, to=1-5]
	\arrow["\lrcorner"{anchor=center, pos=0.125}, draw=none, from=1-4, to=2-5]
\end{tikzcd}\]
If $C$ is the terminal $\io$-category, this notation is compatible with the one of the slice over and under introduced in paragraph \ref{para:slice and joint}.


\begin{lemma}
\label{lemma:explicit factoryzation 1}
The morphism $i:C^\flat\to D_{/C^{\flat}}$ appearing in the following diagram
% q.uiver.app/#q=WzAsNixbMSwxLCJEX3tDXntcXGZsYXR9L30iXSxbMiwxLCJEXntbMV1eXFxzaGFycH0iXSxbMiwyLCJEIl0sWzEsMiwiQ157XFxmbGF0fSJdLFswLDEsIkNee1xcZmxhdH0iXSxbMSwwLCJEIl0sWzEsMiwiKGlfMF4tKV8hIl0sWzMsMl0sWzAsM10sWzAsMV0sWzAsMiwiIiwxLHsic3R5bGUiOnsibmFtZSI6ImNvcm5lciJ9fV0sWzQsNSwiIiwwLHsiY3VydmUiOi0yfV0sWzUsMSwiIiwwLHsiY3VydmUiOi0yfV0sWzQsMywiaWQiLDJdLFs0LDAsImkiLDAseyJzdHlsZSI6eyJib2R5Ijp7Im5hbWUiOiJkYXNoZWQifX19XV0=
\[\begin{tikzcd}
	& D \\
	{C^{\flat}} & {D_{C^{\flat}/}} & {D^{[1]^\sharp}} \\
	& {C^{\flat}} & D
	\arrow["{(i_0^-)_!}", from=2-3, to=3-3]
	\arrow[from=3-2, to=3-3]
	\arrow[from=2-2, to=3-2]
	\arrow[from=2-2, to=2-3]
	\arrow["\lrcorner"{anchor=center, pos=0.125}, draw=none, from=2-2, to=3-3]
	\arrow[curve={height=-12pt}, from=2-1, to=1-2]
	\arrow[curve={height=-12pt}, from=1-2, to=2-3]
	\arrow["id"', from=2-1, to=3-2]
	\arrow["i", dashed, from=2-1, to=2-2]
\end{tikzcd}\]
is initial.
\end{lemma}
\begin{proof}
Using proposition \ref{prop:associativity of Gray amput2}, we have a natural transformation
$$(\uvar\otimes[1]^\sharp)\otimes[1]^\sharp \sim \uvar\otimes([1]^\sharp\times[1]^\sharp)
\xrightarrow{\uvar\otimes\psi} \uvar\otimes[1]^\sharp$$
where $\psi$ sends $(\epsilon,\epsilon')$ on $\max(\epsilon,\epsilon')$. This induces a natural transformation $D^{[1]^\sharp}\to (D^{[1]^\sharp})^{[1]^\sharp}$, corresponding by adjunction to transformation $\phi:D^{[1]^\sharp}\otimes[1]^\sharp\to D^{[1]^\sharp}$. We set $r:D_{C^{\flat}/}\to C^\flat$ as the canonical projection. Eventually, remark that $(i,r,\phi)$ is a left Gray deformation retract. According to proposition \ref{prop:left Gray deformation retract are initial}, this concludes the proof.
\end{proof}

\begin{lemma}
\label{lemma:explicit factoryzation 2}
The composite $q:D_{C^{\flat}/}\to D^{[1]^\sharp}\xrightarrow{(i_0^+)_!} D$ is a left cartesian fibration.
\end{lemma}
\begin{proof}
Consider a commutative diagram
% https://q.uiver.app/#q=WzAsNCxbMSwwLCJEX3tDXntcXGZsYXR9L30iXSxbMSwxLCJEIl0sWzAsMCwiS1xcb3RpbWVzXFx7MFxcfSJdLFswLDEsIktcXG90aW1lc1sxXV5cXHNoYXJwIl0sWzIsM10sWzIsMF0sWzMsMV0sWzAsMV1d
\begin{equation}
\label{eq:lemma:explicit factoryzation 2}
\begin{tikzcd}
	{K\otimes\{0\}} & {D_{C^{\flat}/}} \\
	{K\otimes[1]^\sharp} & D
	\arrow[from=1-1, to=2-1]
	\arrow[from=1-1, to=1-2]
	\arrow[from=2-1, to=2-2]
	\arrow[from=1-2, to=2-2]
\end{tikzcd}
\end{equation}
The $\infty$-groupoid of lifts of this previous diagram is equivalent to the $\infty$-groupoid of pairs consisting of a commutative triangle 
% https://q.uiver.app/#q=WzAsMyxbMCwxLCJLXFxvdGltZXNbMV1eXFxzaGFycFxcb3RpbWVzXFx7MFxcfSJdLFsxLDEsIkQiXSxbMCwwLCJLXFxvdGltZXNcXHswXFx9XFxvdGltZXNcXHswXFx9Il0sWzIsMF0sWzIsMSwiZiJdLFswLDEsIiIsMCx7InN0eWxlIjp7ImJvZHkiOnsibmFtZSI6ImRhc2hlZCJ9fX1dXQ==
\[\begin{tikzcd}
	{K\otimes\{0\}\otimes\{0\}} \\
	{K\otimes[1]^\sharp\otimes\{0\}} & D
	\arrow[from=1-1, to=2-1]
	\arrow["f", from=1-1, to=2-2]
	\arrow[dashed, from=2-1, to=2-2]
\end{tikzcd}\]
where $f$ is induced by $K\otimes\{0\}\to D_{C^{\flat}/}$,
 and a lift in the induced diagram
% https://q.uiver.app/#q=WzAsNCxbMCwwLCJLXFxvdGltZXNcXHswXFx9XFxvdGltZXNbMV1eXFxzaGFycFxcY3VwIEtcXG90aW1lc1sxXV5cXHNoYXJwXFxvdGltZXNcXHsxXFx9IFxcY3VwIEtcXG90aW1lc1sxXV5cXHNoYXJwXFxvdGltZXNcXHswXFx9Il0sWzAsMSwiS1xcb3RpbWVzWzFdXlxcc2hhcnBcXG90aW1lc1sxXV5cXHNoYXJwIl0sWzEsMCwiRCJdLFsxLDEsIjEiXSxbMCwyXSxbMCwxXSxbMSwzXSxbMiwzXSxbMSwyLCIiLDEseyJzdHlsZSI6eyJib2R5Ijp7Im5hbWUiOiJkYXNoZWQifX19XV0=
\[\begin{tikzcd}
	{K\otimes\{0\}\otimes[1]^\sharp\cup K\otimes[1]^\sharp\otimes\{1\} \cup K\otimes[1]^\sharp\otimes\{0\}} & D \\
	{K\otimes[1]^\sharp\otimes[1]^\sharp} & 1
	\arrow[from=1-1, to=1-2]
	\arrow[from=1-1, to=2-1]
	\arrow[from=2-1, to=2-2]
	\arrow[from=1-2, to=2-2]
	\arrow[dashed, from=2-1, to=1-2]
\end{tikzcd}\]
According to proposition \ref{prop:cotimes 1 to c is a trivialization}, the morphism $K\otimes[1]^\sharp\otimes\{0\}\to C^\flat$ factors through a morphism $K\to C^\flat$, and is then uniquely determined by $f:K\otimes\{0\}\otimes\{0\}\to C^\flat$,  and proposition \ref{prop:associativity of Gray amput2} provides a natural equivalence between $(K\otimes[1]^\sharp)\otimes[1]^\sharp$ and $K\otimes([1]^\sharp\times [1]^\sharp)$. The $\infty$-groupoid of lifts of the diagram \eqref{eq:lemma:explicit factoryzation 2} is then equivalent  to the  $\infty$-groupoid of lifts of  the left square of the following diagram
% https://q.uiver.app/#q=WzAsNixbMCwwLCJLXFxvdGltZXNcXHswXFx9XFxvdGltZXNbMV1eXFxzaGFycFxcY3VwIEtcXG90aW1lc1sxXV5cXHNoYXJwXFxvdGltZXNcXHsxXFx9IFxcY3VwIEtcXG90aW1lc1sxXV5cXHNoYXJwXFxvdGltZXNcXHswXFx9Il0sWzAsMSwiS1xcb3RpbWVzWzFdXlxcc2hhcnBcXG90aW1lc1sxXV5cXHNoYXJwIl0sWzIsMCwiRCJdLFsyLDEsIjEiXSxbMSwxLCJLXFxvdGltZXNbMl1eXFxzaGFycCJdLFsxLDAsIktcXG90aW1lc1sxXV5cXHNoYXJwXFxjdXAgS1xcb3RpbWVzWzFdXlxcc2hhcnAiXSxbMCwxXSxbMiwzXSxbMSw0XSxbMCw1XSxbNSw0XSxbNSwyXSxbNCwzXSxbNCwyLCIiLDAseyJzdHlsZSI6eyJib2R5Ijp7Im5hbWUiOiJkYXNoZWQifX19XSxbNCw5LCIiLDEseyJsZXZlbCI6MSwic3R5bGUiOnsibmFtZSI6ImNvcm5lciJ9fV1d
\[\begin{tikzcd}[column sep =0.7cm]
	{K\otimes\{0\}\otimes[1]^\sharp\cup K\otimes[1]^\sharp\otimes\{1\} \cup K\otimes[1]^\sharp\otimes\{0\}} & {K\otimes[1]^\sharp\cup K\otimes[1]^\sharp} & D \\
	{K\otimes[1]^\sharp\otimes[1]^\sharp} & {K\otimes[2]^\sharp} & 1
	\arrow[from=1-1, to=2-1]
	\arrow[from=1-3, to=2-3]
	\arrow[from=2-1, to=2-2]
	\arrow[""{name=0, anchor=center, inner sep=0}, from=1-1, to=1-2]
	\arrow[from=1-2, to=2-2]
	\arrow[from=1-2, to=1-3]
	\arrow[from=2-2, to=2-3]
	\arrow[dashed, from=2-2, to=1-3]
	\arrow["\lrcorner"{anchor=center, pos=0.125, rotate=180}, draw=none, from=2-2, to=0]
\end{tikzcd}\]
As $K\otimes[1]^\sharp\coprod_{K\otimes[0]}K\otimes[1]^\sharp\to K\otimes[2]^\sharp$ is an equivalence, this   $\infty$-groupoid is contractible.
\end{proof}





\begin{prop}
\label{prop:explicit factoryzation}
The factorisation of $p:C^\flat \to D$
in an initial morphism followed by a left cartesian fibration is
$$C^\flat \xrightarrow{i} D_{C^\flat/}\xrightarrow{q} D,$$
and its factorization in a final morphism and a right cartesian fibration is 
$$C^\flat \xrightarrow{i} D_{/C^\flat}\xrightarrow{q} D.$$
\end{prop}
\begin{proof}
This is a direct application of lemma \ref{lemma:explicit factoryzation 1} and \ref{lemma:explicit factoryzation 1} and of their dual version.
\end{proof}
The more important example of the previous proposition is the case $C:=\{a\}$. In this case, the corresponding left cartesian fibration is the slice of $D$ under $a$
$$D_{a/}\to D$$
 and the corresponding right cartesian fibration is the slice of $D$ over $a$
$$D_{/a}\to D.$$
\p 
Let $p:X\to Y$ be a morphism between $\io$-categories. A marked $1$-cell $v:x\to x'$ is \wcnotion{left cancellable}{left or right cancellable $1$-cell} if for any $y$, the following natural square is cartesian:
% q.uiver.app/#q=WzAsNCxbMCwwLCJcXGhvbV9YKHgnLHkpIl0sWzEsMCwiXFxob21fWCh4LHkpIl0sWzEsMSwiXFxob21fWShweCxweSkiXSxbMCwxLCJcXGhvbV9ZKHB4JyxweSkiXSxbMCwxLCJ2XyEiXSxbMywyLCJwKHYpXyEiLDJdLFsxLDJdLFswLDNdXQ==
\[\begin{tikzcd}
	{\hom_X(x',y)} & {\hom_X(x,y)} \\
	{\hom_Y(px',py)} & {\hom_Y(px,py)}
	\arrow["{v_!}", from=1-1, to=1-2]
	\arrow["{p(v)_!}"', from=2-1, to=2-2]
	\arrow[from=1-2, to=2-2]
	\arrow[from=1-1, to=2-1]
\end{tikzcd}\]

Conversely, a $1$-cell $v:y\to y'$ is \textit{right cancellable} if for any $x$, the following natural square is cartesian:
% q.uiver.app/#q=WzAsNCxbMCwwLCJcXGhvbV9YKHgseSkiXSxbMSwwLCJcXGhvbV9YKHgseScpIl0sWzEsMSwiXFxob21fWShweCxweScpIl0sWzAsMSwiXFxob21fWShweCxweScpIl0sWzAsMSwidl8hIl0sWzMsMiwicCh2KV8hIiwyXSxbMSwyXSxbMCwzXV0=
\[\begin{tikzcd}
	{\hom_X(x,y)} & {\hom_X(x,y')} \\
	{\hom_Y(px,py)} & {\hom_Y(px,py')}
	\arrow["{v_!}", from=1-1, to=1-2]
	\arrow["{p(v)_!}"', from=2-1, to=2-2]
	\arrow[from=1-2, to=2-2]
	\arrow[from=1-1, to=2-1]
\end{tikzcd}\]


\begin{lemma}
\label{lemma:technical lemma on cancellable cell 1}
Let $p$ be a morphism. 
The following conditions are equivalent:
\begin{enumerate}
\item $p$ has the unique right lifting property against $\{0\}\to [1]^\sharp$ and marked $1$-cells are left cancellable.
\item $p$ has the unique right lifting property against $[a,1]\xrightarrow{\triangledown} [1]^\sharp\vee[a,1]$ for any object $a$ of $t\Theta$.
\item $p$ has the unique right lifting property against $[a,1]\xrightarrow{\triangledown} [1]^\sharp\vee[a,1]$ and $[1]^\sharp\xrightarrow{\triangledown}[1]^\sharp\vee[1]^\sharp$ for any object $a$ of $t\Theta$.
\end{enumerate}
Conversely, the following are equivalent:
\begin{enumerate}
\item[(1)'] $p$ has the unique right lifting property against $\{1\}\to [1]^\sharp$ and marked $1$-cells are right cancellable.
\item[(2)'] $p$ has the unique right lifting property against $[a,1]\xrightarrow{\triangledown} [a,1]\vee[1]^\sharp$ for any object $a$ of $t\Theta$.
\item[(3)'] $p$ has the unique right lifting property against $[a,1]\xrightarrow{\triangledown}[a,1]\vee[1]^\sharp$ and $[1]^\sharp\xrightarrow{\triangledown}[1]^\sharp\vee[1]^\sharp$ for any object $a$ of $t\Theta$.
\end{enumerate}
\end{lemma}
\begin{proof}
 The fact that $1$-cells are left cancellable is equivalent to asking that $i$ has the unique right lifting property against 
$$[a,1]\amalg_{\{0\}} [1]^\sharp\to [1]^\sharp\vee[a,1]$$
 for any object $a$ of $t\Theta$.
 Suppose that $p$ fulfills $(1)$.
As the class of morphisms having the unique right lifting property against $p$ are closed under composition and by left cancellation according to \ref{prop:closed under colimit imply saturated}, this implies that $p$ has the unique right lifting property against 
$$[a,1]\xrightarrow{\triangledown} [1]^\sharp\vee[a,1]$$
and then that $(1)\Rightarrow (2)$. 

Suppose now that $p$ fulfills $(2)$. Remark that we have a retract
% q.uiver.app/#q=WzAsNixbMCwwLCJcXHswXFx9Il0sWzEsMCwiWzFdIl0sWzIsMCwiXFx7MFxcfSJdLFswLDEsIlsxXV5cXHNoYXJwIl0sWzEsMSwiWzFdXlxcc2hhcnBcXHZlZVsxXSJdLFsyLDEsIlsxXV5cXHNoYXJwIl0sWzAsM10sWzMsNCwiIiwwLHsic3R5bGUiOnsidGFpbCI6eyJuYW1lIjoiaG9vayIsInNpZGUiOiJ0b3AifX19XSxbNCw1LCJpZFxcdmVlXFxJYiIsMl0sWzAsMV0sWzEsNCwiXFx0cmlhbmdsZWRvd24iLDJdLFsxLDIsIlxcSWIiXSxbMiw1XV0=
\[\begin{tikzcd}
	{\{0\}} & {[1]} & {\{0\}} \\
	{[1]^\sharp} & {[1]^\sharp\vee[1]} & {[1]^\sharp}
	\arrow[from=1-1, to=2-1]
	\arrow[hook, from=2-1, to=2-2]
	\arrow["id\vee\Ib"', from=2-2, to=2-3]
	\arrow[from=1-1, to=1-2]
	\arrow["\triangledown"', from=1-2, to=2-2]
	\arrow["\Ib", from=1-2, to=1-3]
	\arrow[from=1-3, to=2-3]
\end{tikzcd}\]
and as the class of morphisms having the unique right lifting property against $p$ is closed under retracts, this implies that $p$ has the unique right lifting property against $\{0\}\to [1]^\sharp$. By stability by left cancellation, $p$ has the unique right lifting property against 
$$[a,1]\amalg_{\{0\}} [1]^\sharp\to [1]^\sharp\vee[a,1].$$
As remarked above, this implies that $1$-cells are left cancellable. We then have $(1)\Leftrightarrow (2)$.

There is an obvious implication $(3)\Rightarrow (2)$. For the converse, remark that the
class of morphisms having the unique right lifting property against $p$ is closed under colimits and then contains $\{0\}\to [1]^\sharp\vee[1]^\sharp$, and so by left cancellation, it includes $ [1]^\sharp\xrightarrow{\triangledown}[1]^\sharp\vee[1]^\sharp$.
The proof of the equivalence of $(1)'$, $(2)'$ and $(3)'$ is symetrical.
\end{proof} 



\begin{lemma}
\label{lemma:technical lemma on cancellable cell 3}
Let $p:X\to Y$ be a morphism having the unique right lifting property against marked trivializations, such that for any element $a$ of $t\Theta$, and any cartesian squares: 	
% https://q.uiver.app/#q=WzAsNixbMSwxLCJbMV1eXFxzaGFycFxcdmVlW2EsMV0iXSxbMiwwLCJYIl0sWzIsMSwiWSJdLFswLDAsIlgnJyJdLFsxLDAsIlgnIl0sWzAsMSwiW2EsMV0iXSxbMSwyLCJwIiwyXSxbMyw1LCJwJyciLDJdLFs0LDAsInAnIiwyXSxbMyw0LCJrIl0sWzQsMV0sWzUsMCwiIiwyLHsic3R5bGUiOnsidGFpbCI6eyJuYW1lIjoiaG9vayIsInNpZGUiOiJ0b3AifX19XSxbMCwyXSxbMywwLCIiLDIseyJzdHlsZSI6eyJuYW1lIjoiY29ybmVyIn19XSxbNCwyLCIiLDIseyJzdHlsZSI6eyJuYW1lIjoiY29ybmVyIn19XV0=
\[\begin{tikzcd}
	{X''} & {X'} & X \\
	{[a,1]} & {[1]^\sharp\vee[a,1]} & Y
	\arrow["p"', from=1-3, to=2-3]
	\arrow["{p''}"', from=1-1, to=2-1]
	\arrow["{p'}"', from=1-2, to=2-2]
	\arrow["k", from=1-1, to=1-2]
	\arrow[from=1-2, to=1-3]
	\arrow[hook, from=2-1, to=2-2]
	\arrow[from=2-2, to=2-3]
	\arrow["\lrcorner"{anchor=center, pos=0.125}, draw=none, from=1-1, to=2-2]
	\arrow["\lrcorner"{anchor=center, pos=0.125}, draw=none, from=1-2, to=2-3]
\end{tikzcd}\]
the square $p''\to p'$ is a right deformation retract.
Then, $p$ has the unique right lifting property against $[a,1]\xrightarrow{\triangledown} [1]^\sharp\vee[a,1]$ for any object $a$ of $t\Theta$. 
\end{lemma}
\begin{proof}
Suppose given a square
% https://q.uiver.app/#q=WzAsNCxbMCwxLCJbMV1eXFxzaGFycFxcdmVlW2EsMV0iXSxbMSwwLCJYIl0sWzEsMSwiWSJdLFswLDAsIlthLDFdIl0sWzEsMiwicCIsMl0sWzMsMCwiXFx0cmlhbmdsZWRvd24iLDJdLFszLDFdLFswLDIsImciLDJdXQ==
\[\begin{tikzcd}
	{[a,1]} & X \\
	{[1]^\sharp\vee[a,1]} & Y
	\arrow["p"', from=1-2, to=2-2]
	\arrow["\triangledown"', from=1-1, to=2-1]
	\arrow[from=1-1, to=1-2]
	\arrow["g"', from=2-1, to=2-2]
\end{tikzcd}\]
and let $p'$ and $p''$ be the morphisms appearing in the following cartesian squares:
% https://q.uiver.app/#q=WzAsNixbMSwxLCJbMV1eXFxzaGFycFxcdmVlW2EsMV0iXSxbMiwwLCJYIl0sWzIsMSwiWSJdLFswLDAsIlgnJyJdLFsxLDAsIlgnIl0sWzAsMSwiW2EsMV0iXSxbMSwyLCJwIiwyXSxbMyw1LCJwJyciLDJdLFs0LDAsInAnIiwyXSxbMyw0LCJrIl0sWzQsMV0sWzUsMCwiIiwyLHsic3R5bGUiOnsidGFpbCI6eyJuYW1lIjoiaG9vayIsInNpZGUiOiJ0b3AifX19XSxbMCwyLCJnIiwyXSxbMywwLCIiLDIseyJzdHlsZSI6eyJuYW1lIjoiY29ybmVyIn19XSxbNCwyLCIiLDIseyJzdHlsZSI6eyJuYW1lIjoiY29ybmVyIn19XV0=
\[\begin{tikzcd}
	{X''} & {X'} & X \\
	{[a,1]} & {[1]^\sharp\vee[a,1]} & Y
	\arrow["p"', from=1-3, to=2-3]
	\arrow["{p''}"', from=1-1, to=2-1]
	\arrow["{p'}"', from=1-2, to=2-2]
	\arrow["k", from=1-1, to=1-2]
	\arrow[from=1-2, to=1-3]
	\arrow[hook, from=2-1, to=2-2]
	\arrow["g"', from=2-2, to=2-3]
	\arrow["\lrcorner"{anchor=center, pos=0.125}, draw=none, from=1-1, to=2-2]
	\arrow["\lrcorner"{anchor=center, pos=0.125}, draw=none, from=1-2, to=2-3]
\end{tikzcd}\]
To show the proposition, one has to demonstrate that the induced diagram
% q.uiver.app/#q=WzAsNCxbMSwxLCJbMV1eXFxzaGFycFxcdmVlW2EsMV0iXSxbMCwwLCJbYSwxXSJdLFsxLDAsIlgnIl0sWzAsMSwiWzFdXlxcc2hhcnBcXHZlZVthLDFdIl0sWzEsMywiXFx0cmlhbmdsZWRvd24iLDJdLFsyLDAsInAnIiwyXSxbMSwyLCJqIl0sWzMsMCwiaWQiLDJdLFsxLDAsIiIsMix7InN0eWxlIjp7Im5hbWUiOiJjb3JuZXIifX1dXQ==
\[\begin{tikzcd}
	{[a,1]} & {X'} \\
	{[1]^\sharp\vee[a,1]} & {[1]^\sharp\vee[a,1]}
	\arrow["\triangledown"', from=1-1, to=2-1]
	\arrow["{p'}"', from=1-2, to=2-2]
	\arrow["j", from=1-1, to=1-2]
	\arrow["id"', from=2-1, to=2-2]
	\arrow["\lrcorner"{anchor=center, pos=0.125}, draw=none, from=1-1, to=2-2]
\end{tikzcd}\]
admits a unique lifting. We denote by $x_0$ and $x_2$ the image of the object of $[a,1]$ via the morphism $j$, and $(k:X''\to X',r,\phi)$ the left deformation retract existing by hypothesis. According to the dual version of proposition \ref{prop:retraction criter}, the unique marked $1$-cell in $X'$ over $[1]^\sharp\hookrightarrow [1]^\sharp\vee[a,1]$ with $x_0$ for source is $\phi(x_0):x_0\to r(x_0)$.
The $\infty$-groupoid of lifts of this diagram is then equivalent to the $\infty$-groupoid of lifts of the following diagram
% q.uiver.app/#q=WzAsNCxbMSwwLCJcXGhvbV97WCd9KHJ4XzAseF8yKSJdLFsxLDEsIlxcaG9tX3tYJ30oeF8wLHhfMikiXSxbMCwxLCJhIl0sWzAsMCwiXFxlbXB0eXNldCJdLFswLDEsIntcXHBoaV97eF8wfX1fISJdLFsyLDFdLFszLDJdLFszLDBdXQ==
\[\begin{tikzcd}
	\emptyset & {\hom_{X'}(rx_0,x_2)} \\
	a & {\hom_{X'}(x_0,x_2)}
	\arrow["{{\phi_{x_0}}_!}", from=1-2, to=2-2]
	\arrow[from=2-1, to=2-2]
	\arrow[from=1-1, to=2-1]
	\arrow[from=1-1, to=1-2]
\end{tikzcd}\]
However, the right vertical morphism is an isomorphism according to proposition \ref{prop:Gray deformation retract and passage to hom} which concludes the proof.
\end{proof}
\p Keeping in mind the last lemma, we define \wcnotation{$\I_{g}$}{(ig@$\I_g$} and \wcnotation{$\F_{g}$}{(fg@$\F_g$} as the smallest sets of morphisms of $\zocatm$ fullfilling these conditions:
\begin{enumerate}
\item for any $a\in \Theta^t$, $[a,1]\hookrightarrow[1]^\sharp\vee[a,1]$ is in $\F_g$ and $[a,1]\hookrightarrow[a,1]\vee[1]^\sharp$ is in $\I_g$
\item for any $i$ in $\F_g$, $[i,1]$ is in $\I_g$, for any $j$ in $\I_g$, $[i,1]$ is in $\F_g$,
\end{enumerate}
Propositions \ref{prop:left Gray deformation retract stable under pushout} and \ref{prop:suspension of left Gray deformation retract} then imply that morphisms of $\I_g$ are left Gray deformation retracts and morphisms of $\F_g$ are right Gray deformation retracts.

\p 
We extend by induction the definition of right and left cancellable to cells of any dimension as follows: a $n$-cell $v$ is \wcnotion{left or right cancellable}{left cancellable $n$-cell} (resp. \textit{right cancellable}) if the corresponding $(n-1)$-cell of $\hom_X(x,y)$ is left cancellable (resp. right cancellable) for the morphism $\hom_X(x,y)\to \hom_Y(px,py)$, where $x$ and $y$ denote the $0$-sources and $0$-but of $v$.

\begin{lemma}
\label{lemma:technical lemma on cancellable cell 2}
Let $p':X'\to Y'$ be a morphism such that $p$ has the unique right lifting property against marked trivializations and suppose that we have a left Gray deformation retract $p'\to p$. We denote by $(r:Y'\to Y,i,\phi)$ the left deformation retract structure induced on the codomain, and suppose that the deformation $\phi:Y\otimes[1]^\sharp\to Y$ factors through $\psi:Y\times[1]^\sharp\to Y$. Then, the square $p'\to p$ is a left deformation retract.
\end{lemma}
\begin{proof}
Proposition \ref{prop:cotimes 1 to ctimes 1 is a trivialization} states that $Y\otimes[1]^\sharp\to Y\times [1]^\sharp$ is a colimit of marked trivializations. There is then a lift in the following diagram:
% https://q.uiver.app/#q=WzAsNSxbMCwwLCJYXFxvdGltZXNbMV1eXFxzaGFycCJdLFswLDEsIlhcXHRpbWVzWzFdXlxcc2hhcnAiXSxbMiwwLCJYIl0sWzEsMSwiWVxcdGltZXNbMV1eXFxzaGFycCJdLFsyLDEsIlkiXSxbMCwxXSxbMSwzXSxbMiw0XSxbMCwyLCJcXHBoaSciXSxbMyw0LCJcXHBzaSIsMl0sWzEsMiwiXFxwc2knIiwxLHsic3R5bGUiOnsiYm9keSI6eyJuYW1lIjoiZG90dGVkIn19fV1d
\[\begin{tikzcd}
	{X\otimes[1]^\sharp} && X \\
	{X\times[1]^\sharp} & {Y\times[1]^\sharp} & Y
	\arrow[from=1-1, to=2-1]
	\arrow[from=2-1, to=2-2]
	\arrow[from=1-3, to=2-3]
	\arrow["{\phi'}", from=1-1, to=1-3]
	\arrow["\psi"', from=2-2, to=2-3]
	\arrow["{\psi'}"{description}, dotted, from=2-1, to=1-3]
\end{tikzcd}\]
where $\phi'$ is the deformation induced on domains.
This endows $p'\to p$ with a structure of left deformation retract, where the retraction is the same, and the deformation is given by $(\psi',\psi)$.
\end{proof}


\begin{theorem}
\label{theo:other characterisation of left caresian fibration}
Consider the following shape of diagram
% q.uiver.app/#q=WzAsNixbMSwxLCJZJyJdLFsyLDAsIlgiXSxbMiwxLCJZIl0sWzAsMCwiWCcnIl0sWzEsMCwiWCciXSxbMCwxLCJZJyciXSxbMSwyLCJwIiwyXSxbMyw1LCJwJyciLDJdLFs0LDAsInAnIiwyXSxbMyw0XSxbNCwxXSxbNSwwLCJpIiwyXSxbMCwyXSxbMywwLCIiLDIseyJzdHlsZSI6eyJuYW1lIjoiY29ybmVyIn19XSxbNCwyLCIiLDIseyJzdHlsZSI6eyJuYW1lIjoiY29ybmVyIn19XV0=
\begin{equation}
\label{eq:prop:other characterisation of left caresian fibration}
\begin{tikzcd}
	{X''} & {X'} & X \\
	{Y''} & {Y'} & Y
	\arrow["p"', from=1-3, to=2-3]
	\arrow["{p''}"', from=1-1, to=2-1]
	\arrow["{p'}"', from=1-2, to=2-2]
	\arrow[from=1-1, to=1-2]
	\arrow[from=1-2, to=1-3]
	\arrow["i"', from=2-1, to=2-2]
	\arrow[from=2-2, to=2-3]
	\arrow["\lrcorner"{anchor=center, pos=0.125}, draw=none, from=1-1, to=2-2]
	\arrow["\lrcorner"{anchor=center, pos=0.125}, draw=none, from=1-2, to=2-3]
\end{tikzcd}
\end{equation}
The following are equivalent:
\begin{enumerate}
\item The morphism $p$ is a left cartesian fibration.
\item $p$ has the unique right lifting property against marked trivialization, and for any diagram of shape \eqref{eq:prop:other characterisation of left caresian fibration},
if $i$ is a right Gray deformation retract, so is $p''\to p'$. 
\item $p$ has the unique right lifting property against marked trivialization and, for any diagram of shape \eqref{eq:prop:other characterisation of left caresian fibration},
if $i$ is in $\F_g$, the square $p''\to p'$ is a right Gray deformation retract.
\item For any even integer $n$, $p$ has the unique right lifting property against $i_n^+:\Db_{n}\to (\Db_{n+1})_t$ and marked $n$-cells are right cancellable; for any odd integer $p$ has the unique right lifting property against $i_n^-:\Db_{n}\to (\Db_{n+1})_t$ and marked $n$-cells are left cancellable.
\item $p$ as the unique right lifting property against $\{0\}\to [1]^\sharp$, marked $1$-cells are left cancellable, and
for any pair of objects $(x,y)$ of $X$, $\hom_X(x,y)\to \hom_Y(px,py)$ is a right cartesian fibration.
\end{enumerate} 
Conversely, the following are equivalent:
\begin{enumerate}
\item[(1)'] The morphism $p$ is a right cartesian fibration.
\item[(2)'] $p$ has the unique right lifting property against marked trivialization and for any diagram of shape \eqref{eq:prop:other characterisation of left caresian fibration},
if $i$ is a left Gray deformation retract, so is $p''\to p'$.
\item[(3)'] $p$ has the unique right lifting property against marked trivialization, and for any diagram of shape \eqref{eq:prop:other characterisation of left caresian fibration},
if $i$ is in $\I_g$, the square $p''\to p'$ is a left Gray deformation retract.
\item[(4)'] For any even integer $n$, $p$ has the unique right lifting property against $i_n^-:\Db_{n}\to (\Db_{n+1})_t$ and marked $n$-cells are left cancellable; for any odd integer $p$ has the unique right lifting property against $i_n^+:\Db_{n}\to (\Db_{n+1})_t$ and marked $n$-cells are right cancellable.
\item[(5)'] $p$ as the unique right lifting property against $\{1\}\to [1]^\sharp$, marked $1$-cells are right cancellable, and
for any pair of objects $(x,y)$ of $X$, $\hom_X(x,y)\to \hom_Y(px,py)$ is a left cartesian fibration.
\end{enumerate} 

\end{theorem}
\begin{proof}
The implication from $(1)$ to $(2)$ and $(1)'$ to $(2)'$ is the content of proposition \ref{prop:left Gray transfomration stable under pullback along cartesian fibration}.

The implication from $(2)$ to $(3)$ and $(2)'$ to $(3)'$ comes from the fact that $\I_g$ (resp. $\F_g$) consists of right (resp. left) Gray deformation retracts.

Suppose now that $p$ fulfills condition $(3)$. Lemma \ref{lemma:technical lemma on cancellable cell 2} implies that if $i$ is of shape $[a,1]\hookrightarrow [1]^\sharp\vee[a,1]$ for $a:t\Theta$, $p''\to p'$ is a right deformation retract. Lemma 
\ref{lemma:technical lemma on cancellable cell 3} and \ref{lemma:technical lemma on cancellable cell 1} then imply that $p$ has the unique right lifting property against $\{0\}\to [1]^\sharp$ and marked $1$-cells are left cancellable.

We are now willing to show that for any pair of objects $(x,y)$, $\hom_X(x,y)\to \hom_Y(px,py)$ fulfills condition $(3)'$, and an obvious induction will complete the proof of $(3)\Rightarrow (4)$.
We then consider $x,y$ two objects of $X$, $i:b\to a$ in $\I_g$ and any morphism $a\to \hom_Y(px,py)$. The previous data induces a pullback square
% q.uiver.app/#q=WzAsNixbMSwxLCJbYSwxXSJdLFsyLDAsIlgiXSxbMiwxLCJZIl0sWzAsMCwiWCcnIl0sWzEsMCwiWCciXSxbMCwxLCJbYiwxXSJdLFsxLDIsInAiLDJdLFszLDUsInAnJyIsMl0sWzQsMCwicCciLDJdLFszLDRdLFs0LDFdLFs1LDAsIltpLDFdIiwyXSxbMCwyXSxbMywwLCIiLDIseyJzdHlsZSI6eyJuYW1lIjoiY29ybmVyIn19XSxbNCwyLCIiLDIseyJzdHlsZSI6eyJuYW1lIjoiY29ybmVyIn19XV0=
\[\begin{tikzcd}
	{X''} & {X'} & X \\
	{[b,1]} & {[a,1]} & Y
	\arrow["p"', from=1-3, to=2-3]
	\arrow["{p''}"', from=1-1, to=2-1]
	\arrow["{p'}"', from=1-2, to=2-2]
	\arrow[from=1-1, to=1-2]
	\arrow[from=1-2, to=1-3]
	\arrow["{[i,1]}"', from=2-1, to=2-2]
	\arrow[from=2-2, to=2-3]
	\arrow["\lrcorner"{anchor=center, pos=0.125}, draw=none, from=1-1, to=2-2]
	\arrow["\lrcorner"{anchor=center, pos=0.125}, draw=none, from=1-2, to=2-3]
\end{tikzcd}\]
where the bottom right morphism sends $\{0\}$ to $px$ and $\{1\}$ to $py$.
By construction, $[i,1]$ is in $\F_g$, and so by assumption, the morphism $p'\to p''$ is a right Gray deformation retract. Applying the functor $\hom_{\uvar}(\uvar,\uvar)$ we get the following pullback diagram:
% q.uiver.app/#q=WzAsNixbMSwxLCJhIl0sWzIsMCwiXFxob21fe1h9KHgseSkiXSxbMiwxLCJcXGhvbV97WX0ocHgscHkpIl0sWzAsMCwiXFxob21fe1gnJ30oeCx5KSJdLFsxLDAsIlxcaG9tX3tYJ30oeCx5KSJdLFswLDEsImIiXSxbMSwyLCJcXHRpbGRle3B9IiwyXSxbMyw1LCJcXHRpbGRle3B9JyciLDJdLFs0LDAsIlxcdGlsZGV7cH0nIiwyXSxbMyw0XSxbNCwxXSxbNSwwLCJpIiwyXSxbMCwyXSxbMywwLCIiLDIseyJzdHlsZSI6eyJuYW1lIjoiY29ybmVyIn19XSxbNCwyLCIiLDIseyJzdHlsZSI6eyJuYW1lIjoiY29ybmVyIn19XV0=
\[\begin{tikzcd}
	{\hom_{X''}(x,y)} & {\hom_{X'}(x,y)} & {\hom_{X}(x,y)} \\
	b & a & {\hom_{Y}(px,py)}
	\arrow["{\tilde{p}}"', from=1-3, to=2-3]
	\arrow["{\tilde{p}''}"', from=1-1, to=2-1]
	\arrow["{\tilde{p}'}"', from=1-2, to=2-2]
	\arrow[from=1-1, to=1-2]
	\arrow[from=1-2, to=1-3]
	\arrow["i"', from=2-1, to=2-2]
	\arrow[from=2-2, to=2-3]
	\arrow["\lrcorner"{anchor=center, pos=0.125}, draw=none, from=1-1, to=2-2]
	\arrow["\lrcorner"{anchor=center, pos=0.125}, draw=none, from=1-2, to=2-3]
\end{tikzcd}\]
and the dual version of proposition \ref{prop:Gray deformation retract and passage to hom v2} implies that $\tilde{p}''\to \tilde{p}'$ is a left Gray deformation retract. As this is true for any $i:b\to a$ in $\I_g$, for any object of $X$, and any $a\to \hom_Y(px,py)$, this implies that $\hom_X(x,y)\to \hom_Y(px,py)$ fulfills condition $(3)'$. As mentioned above, an obvious induction induces $(3)\Rightarrow (4)$. We show similarly $(3)'\Rightarrow (4)'$.


Now let's show $(4)\Rightarrow (1)$ and $(4)'\Rightarrow (1)'$. We show by induction on $n$ that for any 
element $a$ of $t\Gb_n:=\{\Db_k\}_{0\leq k\leq n}\cup \{(\Db_k)_t\}_{1\leq k\leq n}$, if $p$ fulfills $(4)$ (resp. $(4)'$) $p$ has the unique right lifting property against
 $a\otimes\{0\}\to a\otimes[1]^\sharp$ (against $a\otimes\{1\}\to a\otimes[1]^\sharp$).
 
Suppose then that this is true at the stage $n$, and suppose that $p$ fulfills $(4)$. 
Let $a$ be an object of $t\Gb_n$. 	Remark that according to the equation \eqref{eq:eq for cylinder marked version}, $[a,1]\otimes\{0\}\to [a,1]\otimes[1]^\sharp$ fits in the sequence of pushouts
% q.uiver.app/#q=WzAsMTAsWzEsMCwiW2EsMV1cXG90aW1lcyBcXHswXFx9Il0sWzEsMSwiW2EsMV1cXHZlZVsxXV5cXHNoYXJwIl0sWzEsMiwiW2EsMV1cXHZlZVsxXV5cXHNoYXJwXFxjdXBbYVxcb3RpbWVzWzFdXlxcc2hhcnAsMV0iXSxbMSwzLCJbYSwxXVxcb3RpbWVzWzFdXlxcc2hhcnAiXSxbMCwwLCJbMF0iXSxbMCwxLCJbMV1eXFxzaGFycCJdLFsyLDIsIlthXFxvdGltZXNbMV1eXFxzaGFycCwxXSJdLFsyLDEsIlthXFxvdGltZXNcXHsxXFx9LDFdIl0sWzAsMiwiW2EsMV0iXSxbMCwzLCJbMV1eXFxzaGFycFxcdmVlW2EsMV0iXSxbNCw1LCJpXzBeLSIsMl0sWzQsMCwiaV8wXisiXSxbNSwxXSxbMCwxLCIiLDAseyJzdHlsZSI6eyJ0YWlsIjp7Im5hbWUiOiJob29rIiwic2lkZSI6InRvcCJ9fX1dLFs3LDFdLFs2LDJdLFsxLDJdLFs3LDZdLFs4LDJdLFs5LDNdLFsyLDNdLFs4LDksIlxcdHJpYW5nbGVkb3duIiwyXSxbMSwxMSwiIiwxLHsibGV2ZWwiOjEsInN0eWxlIjp7Im5hbWUiOiJjb3JuZXIifX1dLFsyLDE0LCIiLDEseyJsZXZlbCI6MSwic3R5bGUiOnsibmFtZSI6ImNvcm5lciJ9fV0sWzMsMTgsIiIsMSx7ImxldmVsIjoxLCJzdHlsZSI6eyJuYW1lIjoiY29ybmVyIn19XV0=
\[\begin{tikzcd}
	{[0]} & {[a,1]\otimes \{0\}} \\
	{[1]^\sharp} & {[a,1]\vee[1]^\sharp} & {[a\otimes\{1\},1]} \\
	{[a,1]} & {[a,1]\vee[1]^\sharp\cup[a\otimes[1]^\sharp,1]} & {[a\otimes[1]^\sharp,1]} \\
	{[1]^\sharp\vee[a,1]} & {[a,1]\otimes[1]^\sharp}
	\arrow["{i_0^-}"', from=1-1, to=2-1]
	\arrow[""{name=0, anchor=center, inner sep=0}, "{i_0^+}", from=1-1, to=1-2]
	\arrow[from=2-1, to=2-2]
	\arrow[hook, from=1-2, to=2-2]
	\arrow[""{name=1, anchor=center, inner sep=0}, from=2-3, to=2-2]
	\arrow[from=3-3, to=3-2]
	\arrow[from=2-2, to=3-2]
	\arrow[from=2-3, to=3-3]
	\arrow[""{name=2, anchor=center, inner sep=0}, from=3-1, to=3-2]
	\arrow[from=4-1, to=4-2]
	\arrow[from=3-2, to=4-2]
	\arrow["\triangledown"', from=3-1, to=4-1]
	\arrow["\lrcorner"{anchor=center, pos=0.125, rotate=180}, draw=none, from=2-2, to=0]
	\arrow["\lrcorner"{anchor=center, pos=0.125, rotate=90}, draw=none, from=3-2, to=1]
	\arrow["\lrcorner"{anchor=center, pos=0.125, rotate=180}, draw=none, from=4-2, to=2]
\end{tikzcd}\]
By induction hypothesis, for any pair of objects $(x,y)$ of $X$, $\hom_X(x,y)\to \hom_Y(px,py)$ has the unique right lifting property against $a\otimes\{1\}\to a\otimes[1]^\sharp$ for $a\in t\Gb_n$. Furthermore, lemma \ref{lemma:technical lemma on cancellable cell 1} implies that $p$ has the unique right lifting property against $\triangledown:[a,1]\to [1]^\sharp\vee[a,1]$. The morphism $p$ then has the unique right lifting property against $[a\otimes\{1\},1]\to [a\otimes[1]^\sharp,1]$ for $a\in t\Gb_n$. The class of morphisms having the unique right lifting property against $p$ being closed under colimits, this implies that it includes $[a,1]\otimes\{0\}\to [a,1]\otimes[1]^\sharp$. To conclude, one has to show that $p$ has the unique right lifting property against $[1]^\sharp\times\{0\}\to [1]^\sharp\times [1]^\sharp$. Remark that according to proposition \ref{prop:example of a special colimit marked case}, $[1]^\sharp\times\{0\}\to [1]^\sharp\times[1]^\sharp$ fits in the sequence of pushouts:
% q.uiver.app/#q=WzAsNyxbMSwwLCJbMV1eXFxzaGFycFxcdGltZXNcXHswXFx9Il0sWzAsMCwiWzBdIl0sWzAsMSwiWzFdXlxcc2hhcnAiXSxbMSwxLCJbMV1eXFxzaGFycFxcdmVlWzFdXlxcc2hhcnAiXSxbMiwxLCJbMV1eXFxzaGFycCJdLFsyLDIsIlsxXV5cXHNoYXJwXFx2ZWVbMV1eXFxzaGFycCJdLFsxLDIsIlsxXV5cXHNoYXJwXFx0aW1lc1sxXV5cXHNoYXJwIl0sWzEsMiwiaV8wXi0iLDJdLFsxLDAsImlfMF4rIl0sWzQsMywiXFx0cmlhbmdsZWRvd24iLDJdLFs0LDUsIlxcdHJpYW5nbGVkb3duIl0sWzIsMywiIiwyLHsic3R5bGUiOnsidGFpbCI6eyJuYW1lIjoiaG9vayIsInNpZGUiOiJ0b3AifX19XSxbMyw2XSxbNSw2XSxbMCwzLCIiLDAseyJzdHlsZSI6eyJ0YWlsIjp7Im5hbWUiOiJob29rIiwic2lkZSI6InRvcCJ9fX1dLFszLDEsIiIsMCx7InN0eWxlIjp7Im5hbWUiOiJjb3JuZXIifX1dLFs2LDQsIiIsMCx7InN0eWxlIjp7Im5hbWUiOiJjb3JuZXIifX1dXQ==
\[\begin{tikzcd}
	{[0]} & {[1]^\sharp\times\{0\}} \\
	{[1]^\sharp} & {[1]^\sharp\vee[1]^\sharp} & {[1]^\sharp} \\
	& {[1]^\sharp\times[1]^\sharp} & {[1]^\sharp\vee[1]^\sharp}
	\arrow["{i_0^-}"', from=1-1, to=2-1]
	\arrow["{i_0^+}", from=1-1, to=1-2]
	\arrow["\triangledown"', from=2-3, to=2-2]
	\arrow["\triangledown", from=2-3, to=3-3]
	\arrow[hook, from=2-1, to=2-2]
	\arrow[from=2-2, to=3-2]
	\arrow[from=3-3, to=3-2]
	\arrow[hook, from=1-2, to=2-2]
	\arrow["\lrcorner"{anchor=center, pos=0.125, rotate=180}, draw=none, from=2-2, to=1-1]
	\arrow["\lrcorner"{anchor=center, pos=0.125, rotate=90}, draw=none, from=3-2, to=2-3]
\end{tikzcd}\]
According to lemma \ref{lemma:technical lemma on cancellable cell 1}, $p$ has the unique right lifting property against $\triangledown:[1]^\sharp\to [1]^\sharp\vee[1]^\sharp$ and so also against $[1]^\sharp\times\{0\}\to [1]^\sharp\times [1]^\sharp$. This concludes the proof of the implication $(4)\Rightarrow (1)$. We show similarly $(4)'\Rightarrow (1)'$.

Eventually, the equivalences $(1)\Rightarrow (5)$ and $(1)'\Rightarrow (5)'$ are a consequence of proposition \ref{prop:cartesian fibration between arrow} and of the implications $(1)\Rightarrow (4)$ and $(1)'\Rightarrow (4)'$. The implications $(5)\Rightarrow (4)$ and $(5)'\Rightarrow (4)'$ are a consequence of the implications  $(1)'\Rightarrow (4)'$ and $(1)\Rightarrow (4)$ applied to the morphisms $\hom_X(x,y)\to \hom_Y(px,py)$ for all objects $x,y$.
\end{proof}


\begin{cor}
\label{cor:on the fact that fib are define against representable}
A morphism $p:X\to A^\sharp$ is a left cartesian fibration if and only if for any globular sum $b$ and morphism $j:b\to A$, $j^*p$ is a left cartesian fibration over $b^\sharp$.
\end{cor} 
\begin{proof}
This is a direct consequence of the equivalence between conditions $(1)$ and $(3)$ of theorem \ref{theo:other characterisation of left caresian fibration}, and the fact that the codomains of marked trivializations and the codomains of morphisms of $\F_g$ are marked globular sums. 
\end{proof}



\subsection{Cartesian fibration are exponentiable}
\label{subsection:A criterion to be a left cartesian fibration}

We recall that a {marked globular sum} is a marked $\io$-category whose underlying $\io$-category is a globular sum and such that for any pair of integers $k\leq n$, and any pair of $k$-composable $n$-cells $(x,y)$, $x\circ_k y$ is marked if and only if $x$ and $y$ are marked.
 
 
 A morphism $i:a\to b$ between marked globular sums is {globular} if the morphism $i^\natural$ is {globular}.
 
 
 A globular morphism $i$ between marked globular sums is then a discrete Conduché functor, which implies according to proposition \ref{prop:pullback by conduch marked preserves colimit} that the functor $i^*:\ocatm_{/b}\to \ocatm_{/a}$ preserves colimits.


\p Let $b$ be a globular sum and $f:X\to b^\sharp$ a morphism. We say that $f$ is \wcnotion{$b$-exponentiable}{expo@$b$-exponentiable} if 
the canonical morphism $$\colim_{i:\Sp_b^\sharp} {i}^*f\to f$$ is an equivalence. 

\begin{prop}
\label{prop:exponantiable stable under colim}
Let $F:I\to \ocatm_{/b^\sharp}$ be a functor which is pointwise $b$-exponentiable. The morphism $\colim_I F$ is $b$-exponentiable 
\end{prop}
\begin{proof}
Remark that all morphisms $\Db_n^\sharp\to b^\sharp$ in $\Sp^\sharp_b$ are globular, and so are discrete Conduché functors. We then have a sequence of equivalences
$$\colim_{i:\Sp_b^\sharp} {i}^*\colim_I F\sim \colim_I\colim_{i:\Sp_b^\sharp} {i}^*F\sim \colim_I F.$$
\end{proof}


\begin{prop}
\label{prop:how to create exponentiable}
Let $a$ be a globular sum, and $f:X\to a^\sharp$ be a morphism. The induced morphism $\colim_{i:\Sp_{a}^\sharp}i^*f$ is $a$-exponentiable.
\end{prop}
\begin{proof}
As marked globular morphisms are marked discrete Conduché functors, for any $j:\Db_n^\sharp\to a^\sharp\in \Sp_a$, $j^*\colim_{i:\Sp_{a}^\sharp}i^*f$ is equivalent to $j^*f$. We then have a sequence of equivalences
$$ \colim_{j:\Sp_{a}^\sharp}j^* \colim_{i:\Sp_{a}^\sharp}i^*f \sim \colim_{j:\Sp_{a}^\sharp}j^*f .$$
\end{proof}



\begin{prop}
\label{prop:exponantiable stable under pullback}
Let $f:X\to b^\sharp$ be exponentiable in $b$ and $j:a^\sharp\to b^\sharp$ a globular morphism. The morphism $j^*f:X\to a^\sharp$ is exponentiable in $a$.
\end{prop}
\begin{proof}
The morphism $j:a^\sharp\to b^\sharp$ is a marked discrete Conduché functor, so $j^*$ preserves colimits according to proposition \ref{prop:pullback by conduch marked preserves colimit}. 
We then have a sequence of equivalences 
$$j^*f\sim j^* \colim_{i:\Sp_b} i^*f \sim \colim_{i:\Sp_b} (ji)^*f\sim \colim_{k:\Sp_a} k^*f .$$
\end{proof}



\begin{lemma}
\label{lemma:technical lemma exponentiability}
Let $i:c\to d$ be in $\F_g$, $b$ a globular sum, and $f:d\to b^\sharp$ any morphism. 
Then, there exists a commutative square
% q.uiver.app/#q=WzAsNSxbMCwxLCJjIl0sWzEsMSwiZCJdLFsyLDAsImJeXFxzaGFycCJdLFsxLDAsImQnIl0sWzAsMCwiYyciXSxbMCw0LCJoIl0sWzEsMywiZyJdLFs0LDMsImknIl0sWzAsMSwiaSIsMl0sWzMsMl0sWzEsMiwiZiIsMl1d
\[\begin{tikzcd}
	{c'} & {d'} & {b^\sharp} \\
	c & d
	\arrow["h", from=2-1, to=1-1]
	\arrow["g", from=2-2, to=1-2]
	\arrow["{i'}", from=1-1, to=1-2]
	\arrow["i"', from=2-1, to=2-2]
	\arrow[from=1-2, to=1-3]
	\arrow["f"', from=2-2, to=1-3]
\end{tikzcd}\]
\begin{enumerate}
\item $d\to d'$ is a finite composition of pushouts of morphism of shape $i_n^\alpha:\Db_n\to (\Db_{n+1})_t$ with $n$ an integer and $\alpha:=+$ if $n$ is even, and $-$ if not.
\item $d'\to b^\sharp$ is globular.
\item $h\to g$ is a right Gray deformation retract.
\end{enumerate}
\end{lemma}
\begin{proof}
We obtain $(d')^\natural$ by factorizing $f^\natural$ into an algebraic morphism $g^\natural$ followed by a globular morphism. The marking $d'$ is the smaller one that makes $g$ a morphism of marked $\zo$-categories. By construction, $c\to d$ fits in a cocartesian square
% https://q.uiver.app/#q=WzAsNCxbMSwwLCJjIl0sWzEsMSwiZCJdLFswLDEsIiAoXFxEYl97bisxfSlfdCJdLFswLDAsIlxcRGJfbl5cXGZsYXQiXSxbMCwxXSxbMywyLCJpXlxcYWxwaGFfbiIsMl0sWzIsMV0sWzMsMF0sWzEsMywiIiwxLHsic3R5bGUiOnsibmFtZSI6ImNvcm5lciJ9fV1d
\[\begin{tikzcd}
	{\Db_n^\flat} & c \\
	{ (\Db_{n+1})_t} & d
	\arrow[from=1-2, to=2-2]
	\arrow["{i^\alpha_n}"', from=1-1, to=2-1]
	\arrow[from=2-1, to=2-2]
	\arrow[from=1-1, to=1-2]
	\arrow["\lrcorner"{anchor=center, pos=0.125, rotate=180}, draw=none, from=2-2, to=1-1]
\end{tikzcd}\]
where all morphisms are globular, and where $\alpha$ is $+$ if $n$ is even, and $-$ if not. As the procedure is similar for any $n$, we will suppose that $n=0$, and $d$ is then equivalent to $[1]^\sharp\vee[a,1]$ for $a\in t\Theta$. The fact that $g$ is algebraic implies that there exists a marked globular sum $c'$ and an integer $k$, such that $d'$ is of shape $[k]^\sharp\vee c'$ and such that $gi$ factors through $c'$. These data verify the desired condition.
\end{proof}


\begin{prop}
\label{prop:criterion to be left cartesian firbation}
Let $p:X\to b^\sharp$ be a morphism exponentiable in $b$. Consider also the following shape of diagram
% q.uiver.app/#q=WzAsNixbMiwwLCJYIl0sWzIsMSwiYl5cXHNoYXJwIl0sWzAsMCwiWCcnIl0sWzEsMCwiWCciXSxbMCwxLCJDIl0sWzEsMSwiQyciXSxbMCwxLCJwIiwyXSxbNCw1LCJpIiwyXSxbNSwxLCJqIiwyXSxbMyw1LCJwJyIsMl0sWzIsNCwicCcnIiwyXSxbMywwXSxbMiwzXSxbMiw1LCIiLDIseyJzdHlsZSI6eyJuYW1lIjoiY29ybmVyIn19XSxbMywxLCIiLDIseyJzdHlsZSI6eyJuYW1lIjoiY29ybmVyIn19XV0=
\begin{equation}
\label{eq:prop:criterion to be left cartesian firbation}
\begin{tikzcd}
	{X''} & {X'} & X \\
	C & {C'} & {b^\sharp}
	\arrow["p"', from=1-3, to=2-3]
	\arrow["i"', from=2-1, to=2-2]
	\arrow["j"', from=2-2, to=2-3]
	\arrow["{p'}"', from=1-2, to=2-2]
	\arrow["{p''}"', from=1-1, to=2-1]
	\arrow[from=1-2, to=1-3]
	\arrow[from=1-1, to=1-2]
	\arrow["\lrcorner"{anchor=center, pos=0.125}, draw=none, from=1-1, to=2-2]
	\arrow["\lrcorner"{anchor=center, pos=0.125}, draw=none, from=1-2, to=2-3]
\end{tikzcd}
\end{equation}
The following are equivalent.
\begin{enumerate}
\item For any globular morphism $i:[a,1]^\sharp\to b^\sharp$, $i^*p$ is a left cartesian fibration.
\item For any diagram of shape \eqref{eq:prop:criterion to be left cartesian firbation}, if $i$ is  $i_n^\alpha:\Db_n\to (\Db_{n+1})_t$ with $n$ an integer and $\alpha:=+$ if $n$ is even and $-$ if not, and $j$ is globular, then $p''\to p'$ is a right Gray deformation retract.
\item For any diagram of shape \eqref{eq:prop:criterion to be left cartesian firbation}, if $i$ is a finite composition of pushouts of morphism of shape $i_n^\alpha:\Db_n\to (\Db_{n+1})_t$ with $n$ an integer and $\alpha:=+$ if $n$ is even and $-$ if not, and $j$ is globular, then $p''\to p'$ is a right Gray deformation retract.
\item For any diagram of shape \eqref{eq:prop:criterion to be left cartesian firbation}, if $i$ is in $ \F_g$, then $p''\to p'$ is a right Gray deformation retract.
\item The morphism $p$ is a left cartesian fibration.
\end{enumerate}
\end{prop}
\begin{proof}
The implication $(1)\Rightarrow (2)$
comes from theorem \ref{theo:other characterisation of left caresian fibration} as morphisms of shape $ i_n^\alpha$ are right Gray deformation retracts according to proposition \ref{prop:when glob inclusion are left Gray deformation}, and as every globular morphism $\Db_{n+1}\to b$ factors through a globular morphism $[a,1]\to b$.

We suppose that the second condition is fulfilled. As left Gray deformation retracts are stable under composition according to proposition \ref{prop:stability by composition }, we can restrict to the case where $i':c\to d$ fits in a cocartesian square
% https://q.uiver.app/#q=WzAsNCxbMSwwLCJjIl0sWzEsMSwiZCJdLFswLDEsIiAoXFxEYl97bisxfSlfdCJdLFswLDAsIlxcRGJfbl5cXGZsYXQiXSxbMCwxXSxbMywyLCJpXlxcYWxwaGFfbiIsMl0sWzIsMV0sWzMsMF0sWzEsMywiIiwxLHsic3R5bGUiOnsibmFtZSI6ImNvcm5lciJ9fV1d
\[\begin{tikzcd}
	{\Db_n^\flat} & c \\
	{ (\Db_{n+1})_t} & d
	\arrow[from=1-2, to=2-2]
	\arrow["{i^\alpha_n}"', from=1-1, to=2-1]
	\arrow[from=2-1, to=2-2]
	\arrow[from=1-1, to=1-2]
	\arrow["\lrcorner"{anchor=center, pos=0.125, rotate=180}, draw=none, from=2-2, to=1-1]
\end{tikzcd}\]
where all morphisms are globular, and where $\alpha$ is $+$ if $n$ is even, and $-$ if not.
Let $p_0$ and $p_1$ be the morphism fitting in cocartesian squares
% q.uiver.app/#q=WzAsNixbMiwwLCJYIl0sWzIsMSwiIGJeXFxzaGFycCJdLFsxLDAsIlhfMSJdLFsxLDEsIihcXERiX3tuKzF9KV90Il0sWzAsMSwiXFxEYl97bn1eXFxmbGF0Il0sWzAsMCwiWF8wIl0sWzAsMSwicCIsMl0sWzMsMV0sWzIsMywicF8xIiwyXSxbMiwwXSxbMiwxLCIiLDIseyJzdHlsZSI6eyJuYW1lIjoiY29ybmVyIn19XSxbNCwzLCIgaV9uXlxcYWxwaGEiLDJdLFs1LDQsInBfMCIsMl0sWzUsMl1d
\[\begin{tikzcd}
	{X_0} & {X_1} & X \\
	{\Db_{n}^\flat} & {(\Db_{n+1})_t} & { b^\sharp}
	\arrow["p"', from=1-3, to=2-3]
	\arrow[from=2-2, to=2-3]
	\arrow["{p_1}"', from=1-2, to=2-2]
	\arrow[from=1-2, to=1-3]
	\arrow["\lrcorner"{anchor=center, pos=0.125}, draw=none, from=1-2, to=2-3]
	\arrow["{ i_n^\alpha}"', from=2-1, to=2-2]
	\arrow["{p_0}"', from=1-1, to=2-1]
	\arrow[from=1-1, to=1-2]
\end{tikzcd}\]
This defines a diagram in the $(\infty,1)$-category of arrows of $\ocatm$:
% q.uiver.app/#q=WzAsNixbMCwxLCJwXzEiXSxbMCwwLCJwXzAiXSxbMSwwLCJwXzAiXSxbMSwxLCJwXzAiXSxbMiwxLCJwJyciXSxbMiwwLCJwJyciXSxbMywwXSxbMSwwXSxbMiwxXSxbMiw1XSxbMiwzXSxbMyw0XSxbNSw0XV0=
\[\begin{tikzcd}
	{p_0} & {p_0} & {p''} \\
	{p_1} & {p_0} & {p''}
	\arrow[from=2-2, to=2-1]
	\arrow[from=1-1, to=2-1]
	\arrow[from=1-2, to=1-1]
	\arrow[from=1-2, to=1-3]
	\arrow[from=1-2, to=2-2]
	\arrow[from=2-2, to=2-3]
	\arrow[from=1-3, to=2-3]
\end{tikzcd}\]
As the proposition \ref{prop:exponantiable stable under pullback} implies that $p'$ is $d$-exponentiable, the morphism $p''\to p'$ is the horizontal colimit of the previous diagram.  According to proposition \ref{prop:left Gray transfomration stable under pullback along cartesian fibration}, $p_0\to p_1$ is a left Gray deformation retract, and proposition \ref{prop:left Gray deformation retract stable under pushout} implies that $p''\to p'$ also is a left Gray deformation retract. This proves $(2)\Rightarrow (3)$.


Suppose now that condition $(3)$ is fulfilled and let $i$ be in $\F_g$. Consider the diagram
% q.uiver.app/#q=WzAsNSxbMCwxLCJjIl0sWzEsMSwiZCJdLFsyLDAsImJeXFxzaGFycCJdLFsxLDAsImQnIl0sWzAsMCwiYyciXSxbMCw0LCJoIl0sWzEsMywiZyJdLFs0LDMsImknIl0sWzAsMSwiaSIsMl0sWzMsMl0sWzEsMiwiZiIsMl1d
\[\begin{tikzcd}
	{c'} & {d'} & {b^\sharp} \\
	c & d
	\arrow["h", from=2-1, to=1-1]
	\arrow["g", from=2-2, to=1-2]
	\arrow["{i'}", from=1-1, to=1-2]
	\arrow["i"', from=2-1, to=2-2]
	\arrow[from=1-2, to=1-3]
	\arrow["f"', from=2-2, to=1-3]
\end{tikzcd}\]
induced by lemma \ref{lemma:technical lemma exponentiability}. 
We denote by $\tilde{p}''$ and $\tilde{p}'$ the morphisms fitting in the following cartesian squares.
% q.uiver.app/#q=WzAsNixbMiwxLCIgYl5cXHNoYXJwIl0sWzEsMSwiIGQnIl0sWzAsMSwiIGMnIl0sWzIsMCwiWCJdLFsxLDAsIlxcdGlsZGV7WH0nIl0sWzAsMCwiXFx0aWxkZXtYfScnIl0sWzIsMSwiIGknIiwyXSxbMSwwXSxbMywwLCJwIiwyXSxbNCwxLCJcXHRpbGRle3B9JyIsMl0sWzUsMiwiXFx0aWxkZXtwfScnIiwyXSxbNSw0XSxbNCwzXSxbNCwwLCIiLDEseyJzdHlsZSI6eyJuYW1lIjoiY29ybmVyIn19XSxbNSwxLCIiLDEseyJzdHlsZSI6eyJuYW1lIjoiY29ybmVyIn19XV0=
\[\begin{tikzcd}
	{\tilde{X}''} & {\tilde{X}'} & X \\
	{ c'} & { d'} & { b^\sharp}
	\arrow["{ i'}"', from=2-1, to=2-2]
	\arrow[from=2-2, to=2-3]
	\arrow["p"', from=1-3, to=2-3]
	\arrow["{\tilde{p}'}"', from=1-2, to=2-2]
	\arrow["{\tilde{p}''}"', from=1-1, to=2-1]
	\arrow[from=1-1, to=1-2]
	\arrow[from=1-2, to=1-3]
	\arrow["\lrcorner"{anchor=center, pos=0.125}, draw=none, from=1-2, to=2-3]
	\arrow["\lrcorner"{anchor=center, pos=0.125}, draw=none, from=1-1, to=2-2]
\end{tikzcd}\]
As $p$ fulfills $(3)$, $\tilde{p}''\to \tilde{p}'$ is a right Gray deformation retract. By construction, 
the square $h\to g$ also is a right Gray deformation retract. 
As $p''$ and $p'$ are respectively the pullback of $\tilde{p}''$ along $h$ and the pullback of $\tilde{p}'$ along $g$, the dual version of \ref{prop:stability under pullback} implies that $p''\to p'$ is a right Gray deformation retract.

The implication $(4)\Rightarrow (5)$ is induced by theorem \ref{theo:other characterisation of left caresian fibration}. Eventually, the implication $(5)\Rightarrow (1)$ is a consequence of the preservation of left cartesian fibration under pullback.
\end{proof}



\begin{cor}
\label{cor:fibration over representable are expenitalbe}
A fibration $p$ over $a^\sharp$ is $a$-exponentiable. 
\end{cor}
\begin{proof}
We define $q:=\colim_{i:\Sp_a^\sharp}i^*p$. This morphism comes with a canonical comparison $q\to p$. According to proposition \ref{prop:how to create exponentiable}, $q$ is $a$-exponentiable.  For any globular morphism $j:[b,1]^\sharp\to a$, we have $j^*q\sim j^*p$ as $j$ is a discrete Conduché functor. In particular, $j^*q$ is a left cartesian fibration and  $q$ then verifies the first condition of proposition \ref{prop:criterion to be left cartesian firbation}. This implies that $q$ is a left cartesian fibration. 


As all morphisms $j:1\to a^\sharp$ are marked globular, and so are discrete Conduché functors, 
there are equivalences
$$j^*\colim_{i:\Sp_a^\sharp}i^*p\sim j^*p$$
and the morphism $q\to p$ induces an equivalence on fiber. This morphisms is then an equivalence according to corollary \ref{cor:morphism between is an equivalence when equivalence on fiber}.
\end{proof}




\begin{lemma}
\label{lemma:pulback of Wsat}
Let $f:A\to B^\sharp$ be a left cartesian fibration, $n$ an integer, and consider a diagram of $\tiPsh{\Theta}$ of shape
% q.uiver.app/#q=WzAsNixbMCwwLCJBJyciXSxbMCwxLCIoXFxTaWdtYV5uRV57ZXF9KV5cXGZsYXQiXSxbMSwxLCJcXERiX25eXFxmbGF0Il0sWzIsMSwiQl5cXHNoYXJwIl0sWzEsMCwiQSciXSxbMiwwLCJBIl0sWzUsMywiZiJdLFs0LDIsImYnIl0sWzAsMSwiZicnIl0sWzEsMiwiaSIsMl0sWzAsNCwiaiJdLFs0LDVdLFsyLDNdLFs0LDMsIiIsMSx7InN0eWxlIjp7Im5hbWUiOiJjb3JuZXIifX1dLFswLDIsIiIsMSx7InN0eWxlIjp7Im5hbWUiOiJjb3JuZXIifX1dXQ==
\[\begin{tikzcd}
	{A''} & {A'} & A \\
	{(\Sigma^nE^{eq})^\flat} & {\Db_n^\flat} & {B^\sharp}
	\arrow["f", from=1-3, to=2-3]
	\arrow["{f'}", from=1-2, to=2-2]
	\arrow["{f''}", from=1-1, to=2-1]
	\arrow["i"', from=2-1, to=2-2]
	\arrow["j", from=1-1, to=1-2]
	\arrow[from=1-2, to=1-3]
	\arrow[from=2-2, to=2-3]
	\arrow["\lrcorner"{anchor=center, pos=0.125}, draw=none, from=1-2, to=2-3]
	\arrow["\lrcorner"{anchor=center, pos=0.125}, draw=none, from=1-1, to=2-2]
\end{tikzcd}\]
Then $j$ is in $\widehat{\Wm}$.
\end{lemma}
\begin{proof}
As $f'$ and $f''$ are left cartesian fibrations, the only marked cell in $A'$ and $A''$ are the identities according to proposition \ref{prop:left fib over flat}. We can then suppose that the left square lies in $\ocat$, and then apply proposition \ref{prop:pulback of Wsat}.
\end{proof}

\begin{lemma}
\label{lemma:pullback along markkin}
Let $b$ be a globular sum, and $n$ an integer. For any cartesian squares in $\iPsh{\Theta}$,
% q.uiver.app/#q=WzAsNixbMCwwLCJBJyciXSxbMCwxLCJCJyciXSxbMSwxLCJCJyJdLFsyLDEsImJee1xcc2hhcnB9Il0sWzEsMCwiQSciXSxbMiwwLCJiXntcXHNoYXJwX259Il0sWzUsM10sWzQsMiwiICJdLFswLDFdLFsxLDIsImkiLDJdLFswLDQsImoiXSxbNCw1XSxbMiwzXSxbNCwzLCIiLDEseyJzdHlsZSI6eyJuYW1lIjoiY29ybmVyIn19XSxbMCwyLCIiLDEseyJzdHlsZSI6eyJuYW1lIjoiY29ybmVyIn19XV0=
\[\begin{tikzcd}
	{A''} & {A'} & {b^{\sharp_n}} \\
	{B''} & {B'} & {b^{\sharp}}
	\arrow[from=1-3, to=2-3]
	\arrow["{ }", from=1-2, to=2-2]
	\arrow[from=1-1, to=2-1]
	\arrow["i"', from=2-1, to=2-2]
	\arrow["j", from=1-1, to=1-2]
	\arrow[from=1-2, to=1-3]
	\arrow[from=2-2, to=2-3]
	\arrow["\lrcorner"{anchor=center, pos=0.125}, draw=none, from=1-2, to=2-3]
	\arrow["\lrcorner"{anchor=center, pos=0.125}, draw=none, from=1-1, to=2-2]
\end{tikzcd}\]
if $i$ is in $\widehat{\Wm}$, so is $j$.
\end{lemma}
\begin{proof}
As $\tiPsh{\Theta}$ is cartesian closed, one can suppose that $i$ is in $\W$. In this case the diagram can be seen as a diagram in $\Psh{\Theta}$. The proof is an easy verification of all the possible cases.
\end{proof}

\begin{prop}
\label{prop:W stable under pullback}
For any cartesian square of $\tiPsh{\Theta}$,
% q.uiver.app/#q=WzAsNixbMCwwLCJBJyciXSxbMCwxLCJCJyciXSxbMSwxLCJCJyJdLFsyLDEsIkJeXFxzaGFycCJdLFsxLDAsIkEnIl0sWzIsMCwiQSJdLFs1LDMsImYiXSxbNCwyXSxbMCwxXSxbMSwyLCJpIiwyXSxbMCw0LCJqIl0sWzQsNV0sWzIsM10sWzQsMywiIiwxLHsic3R5bGUiOnsibmFtZSI6ImNvcm5lciJ9fV0sWzAsMiwiIiwxLHsic3R5bGUiOnsibmFtZSI6ImNvcm5lciJ9fV1d
\[\begin{tikzcd}
	{A''} & {A'} & A \\
	{B''} & {B'} & {B^\sharp}
	\arrow["f", from=1-3, to=2-3]
	\arrow[from=1-2, to=2-2]
	\arrow[from=1-1, to=2-1]
	\arrow["i"', from=2-1, to=2-2]
	\arrow["j", from=1-1, to=1-2]
	\arrow[from=1-2, to=1-3]
	\arrow[from=2-2, to=2-3]
	\arrow["\lrcorner"{anchor=center, pos=0.125}, draw=none, from=1-2, to=2-3]
	\arrow["\lrcorner"{anchor=center, pos=0.125}, draw=none, from=1-1, to=2-2]
\end{tikzcd}\]
where $f$ is a left cartesian fibration, if $i$ is in $\widehat{\Wm}$, so is $j$.
\end{prop}
\begin{proof}
As $\tiPsh{\Theta}$ is cartesian closed, one can suppose that $i$ is in $\W$. Several cases have to be considered.
If $i$ is of shape $(\Sigma^nE^{eq})^\flat\to \Db_n^\flat$, this is lemma \ref{lemma:pulback of Wsat}. 
Suppose now that $i$ is of shape $\Sp_b^{\sharp_n}\to b^{\sharp_n}$. This induces a diagram
% q.uiver.app/#q=WzAsMTAsWzAsMCwiQScnIl0sWzAsMiwiXFxTcF9iXntcXHNoYXJwX259Il0sWzIsMiwiYl57XFxzaGFycF9ufSJdLFs0LDIsIkJeXFxzaGFycCJdLFsyLDAsIkEnIl0sWzQsMCwiQSJdLFszLDMsImJee1xcc2hhcnB9Il0sWzEsMywiXFxTcF9iXntcXHNoYXJwfSJdLFszLDEsIkEnJyciXSxbMSwxLCJBJycnJyJdLFs1LDMsImYiXSxbNCwyXSxbMCwxXSxbMSwyLCJpIiwyLHsibGFiZWxfcG9zaXRpb24iOjYwfV0sWzAsNCwiaiJdLFs0LDVdLFsyLDNdLFswLDIsIiIsMSx7InN0eWxlIjp7Im5hbWUiOiJjb3JuZXIifX1dLFsxLDddLFsyLDZdLFs3LDYsImknIiwyXSxbNiwzXSxbOSw3XSxbOSw4LCJqJyIsMCx7ImxhYmVsX3Bvc2l0aW9uIjo0MH1dLFs4LDZdLFs4LDVdLFswLDldLFs0LDhdLFs4LDMsIiIsMSx7InN0eWxlIjp7Im5hbWUiOiJjb3JuZXIifX1dLFs0LDEwLCIiLDEseyJsZXZlbCI6MSwic3R5bGUiOnsibmFtZSI6ImNvcm5lciJ9fV1d
\[\begin{tikzcd}
	{A''} && {A'} && A \\
	& {A''''} && {A'''} \\
	{\Sp_b^{\sharp_n}} && {b^{\sharp_n}} && {B^\sharp} \\
	& {\Sp_b^{\sharp}} && {b^{\sharp}}
	\arrow[""{name=0, anchor=center, inner sep=0}, "f", from=1-5, to=3-5]
	\arrow[from=1-3, to=3-3]
	\arrow[from=1-1, to=3-1]
	\arrow["i"'{pos=0.6}, from=3-1, to=3-3]
	\arrow["j", from=1-1, to=1-3]
	\arrow[from=1-3, to=1-5]
	\arrow[from=3-3, to=3-5]
	\arrow["\lrcorner"{anchor=center, pos=0.125}, draw=none, from=1-1, to=3-3]
	\arrow[from=3-1, to=4-2]
	\arrow[from=3-3, to=4-4]
	\arrow["{i'}"', from=4-2, to=4-4]
	\arrow[from=4-4, to=3-5]
	\arrow[from=2-2, to=4-2]
	\arrow["{j'}"{pos=0.4}, from=2-2, to=2-4]
	\arrow[from=2-4, to=4-4]
	\arrow[from=2-4, to=1-5]
	\arrow[from=1-1, to=2-2]
	\arrow[from=1-3, to=2-4]
	\arrow["\lrcorner"{anchor=center, pos=0.125}, draw=none, from=2-4, to=3-5]
	\arrow["\lrcorner"{anchor=center, pos=0.125}, draw=none, from=1-3, to=0]
\end{tikzcd}\]
where all squares are cartesian. Corollary \ref{cor:fibration over representable are expenitalbe} implies that $j'$ is in $\widehat{\W}$, and according to lemma \ref{lemma:pullback along markkin}, so is $j$.
\end{proof}

 A left cartesian fibration $A\to B$ is \wcnotion{classified}{classified left cartesian fibration} if there exists a cocartesian square: 
% q.uiver.app/#q=WzAsNCxbMCwwLCJBIl0sWzAsMSwiQiJdLFsxLDEsIkJeXFxzaGFycCJdLFsxLDAsIkEnIl0sWzAsM10sWzMsMl0sWzAsMV0sWzEsMl0sWzAsMiwiIiwxLHsic3R5bGUiOnsibmFtZSI6ImNvcm5lciJ9fV1d
\[\begin{tikzcd}
	A & {A'} \\
	B & {B^\sharp}
	\arrow[from=1-1, to=1-2]
	\arrow[from=1-2, to=2-2]
	\arrow[from=1-1, to=2-1]
	\arrow[from=2-1, to=2-2]
	\arrow["\lrcorner"{anchor=center, pos=0.125}, draw=none, from=1-1, to=2-2]
\end{tikzcd}\]


\begin{theorem}
\label{theo:pullback along un marked cartesian fibration}
Let $p:A\to B$ be a classified left cartesian fibration. The functor $p^*:\ocatm_{/B}\to \ocatm_{/A}$ preserves colimits. 
\end{theorem}
\begin{proof}
As $\tPsh{\Theta}$ is locally cartesian closed, it is enough to show that the functor $p^*:\tiPsh{\Theta}_{/B}\to \tiPsh{\Delta[\Theta]}_{/A}$ sends $\Wm$ onto $\widehat{\Wm}$.
As morphisms fulfilling this property are stable under pullback, one can suppose that $p$ is of shape $B\to A^\sharp$, then applies proposition \ref{prop:W stable under pullback}.
\end{proof}

\begin{cor}
\label{cor:fib over a colimit}
Let $B$ be the colimit of a diagram $F:I\to \ocat$, and
 $p:X\to \colim_i B_i$ a left cartesian fibration. The canonical morphism
 $$ \colim_{i:B_i\to B}i^*p\to p$$
 is an equivalence.
\end{cor}
\begin{proof}
This morphism corresponds to the square
% q.uiver.app/#q=WzAsNCxbMSwwLCJYIl0sWzEsMSwiQl5cXHNoYXJwIl0sWzAsMSwiXFxjb2xpbV97aTpJfUJfaSJdLFswLDAsIlxcY29saW1fe2k6SX1wXipCX2kiXSxbMiwxXSxbMywyXSxbMywwXSxbMCwxLCJwIl1d
\[\begin{tikzcd}
	{\colim_{i:I}p^*B_i} & X \\
	{\colim_{i:I}B_i} & {B^\sharp}
	\arrow[from=2-1, to=2-2]
	\arrow[from=1-1, to=2-1]
	\arrow[from=1-1, to=1-2]
	\arrow["p", from=1-2, to=2-2]
\end{tikzcd}\]
The lower horizontal morphism is an equivalence by hypothesis, and the upper one is an equivalence as $p^*$ preserves colimits.
\end{proof}


\subsection{Colimits of cartesian fibrations}
\label{section:Colimit of left cartesian fibrations}
Through this section, we will identify any marked $\io$-category $C$ with the canonical induced morphism $C\to1$. If $f:X\to Y$ is a morphism, $f\times C$ then corresponds to the canonical morphism $X\times C\to Y$.


\begin{lemma}
\label{lemma: colimit of fib over b}
Let $b$ be a globular sum and $F:I\to \ocatm_{/b^\sharp}$ be a diagram that is pointwise a left cartesian fibration. The induced morphism 
$\colim_IF$ is a left cartesian fibration over $b^\sharp$. 
\end{lemma}
\begin{proof}
We denote $G:I\to \ocatm$ the diagram induced by $F$ by taking the domain.
Remark first that proposition \ref{prop:exponantiable stable under colim} implies that $\colim_IF$ is $b$-exponentiable. 
Let $n$ be an integer. Suppose given cartesian squares
% https://q.uiver.app/#q=WzAsNixbMSwxLCIoXFxEYl97bisxfSlfdCJdLFsyLDAsIlxcY29saW1fSVgiXSxbMiwxLCJiXlxcc2hhcnAiXSxbMCwxLCJcXERiX25eXFxmbGF0Il0sWzEsMCwiWSJdLFswLDAsIlknIl0sWzEsMiwiXFxjb2xpbV9JRiJdLFszLDAsImlfbl5cXGFscGhhIiwyXSxbNCwwXSxbNCwxXSxbNSw0LCJmIl0sWzUsM10sWzAsMiwiaiIsMl0sWzUsMCwiIiwyLHsic3R5bGUiOnsibmFtZSI6ImNvcm5lciJ9fV0sWzQsMiwiIiwyLHsic3R5bGUiOnsibmFtZSI6ImNvcm5lciJ9fV1d
\[\begin{tikzcd}
	{Y'} & Y & {\colim_IX} \\
	{\Db_n^\flat} & {(\Db_{n+1})_t} & {b^\sharp}
	\arrow["{\colim_IF}", from=1-3, to=2-3]
	\arrow["{i_n^\alpha}"', from=2-1, to=2-2]
	\arrow[from=1-2, to=2-2]
	\arrow[from=1-2, to=1-3]
	\arrow["f", from=1-1, to=1-2]
	\arrow[from=1-1, to=2-1]
	\arrow["j"', from=2-2, to=2-3]
	\arrow["\lrcorner"{anchor=center, pos=0.125}, draw=none, from=1-1, to=2-2]
	\arrow["\lrcorner"{anchor=center, pos=0.125}, draw=none, from=1-2, to=2-3]
\end{tikzcd}\]
where $\alpha$ is $+$ is $n$ is even and $-$ if not and with $j$ globular. According to proposition \ref{prop:criterion to be left cartesian firbation}, we have to show that $f$ is a right Gray deformation retract to conclude. As $F$ is pointwise a left cartesian fibration, proposition \textit{op cit} implies that for any $i:I$, the morphism $f(i)$ appearing in the cartesian squares:
% https://q.uiver.app/#q=WzAsNixbMSwxLCIoXFxEYl97bisxfSlfdCJdLFsyLDAsIlgoaSkiXSxbMiwxLCJiXlxcc2hhcnAiXSxbMCwxLCJcXERiX25eXFxmbGF0Il0sWzEsMCwiWSJdLFswLDAsIlknIl0sWzEsMiwiRihpKSJdLFszLDAsImlfbl5cXGFscGhhIiwyXSxbNCwwXSxbNCwxXSxbNSw0LCJmKGkpIl0sWzUsM10sWzAsMiwiaiIsMl0sWzUsMCwiIiwyLHsic3R5bGUiOnsibmFtZSI6ImNvcm5lciJ9fV0sWzQsMiwiIiwyLHsic3R5bGUiOnsibmFtZSI6ImNvcm5lciJ9fV1d
\[\begin{tikzcd}
	{Y'} & Y & {X(i)} \\
	{\Db_n^\flat} & {(\Db_{n+1})_t} & {b^\sharp}
	\arrow["{F(i)}", from=1-3, to=2-3]
	\arrow["{i_n^\alpha}"', from=2-1, to=2-2]
	\arrow[from=1-2, to=2-2]
	\arrow[from=1-2, to=1-3]
	\arrow["{f(i)}", from=1-1, to=1-2]
	\arrow[from=1-1, to=2-1]
	\arrow["j"', from=2-2, to=2-3]
	\arrow["\lrcorner"{anchor=center, pos=0.125}, draw=none, from=1-1, to=2-2]
	\arrow["\lrcorner"{anchor=center, pos=0.125}, draw=none, from=1-2, to=2-3]
\end{tikzcd}\]
is a right Gray deformation retract. 
 As $j$ and $ji_n^\alpha$ are marked globular, they are discrete Conduché functors, and so exponentiable according to proposition \ref{prop:pullback by conduch marked preserves colimit}. The following canonical morphism
 $$\colim_I f(i)\to f$$
 is then an equivalence. As right Gray deformation retracts are stable by colimits, this concludes the proof.
\end{proof}

\begin{lemma}
\label{lemma: colimit of fib over b2}
Let $A$ be an $\io$-category and $F:I\to \ocatm_{/A^\sharp}$ be a diagram that is pointwise a left cartesian fibration. Let $i:a^\sharp\to b^\sharp$ be a morphism between globular sums and $i:b^\sharp\to A^\sharp$ any morphism.
The canonical comparison $$\colim_I (ji)^*F\to i^*\colim_I j^*F$$
is an equivalence.
\end{lemma}
\begin{proof}
Lemma \ref{lemma: colimit of fib over b} implies that the two morphisms are left cartesian fibrations. As equivalences between these morphisms are detected on fibers, we can suppose that $a$ is $[0]$. In this case, the morphism $i$ is a discrete Conduché functor, and is then exponentiable according to proposition \ref{prop:pullback by conduch marked preserves colimit}. This directly concludes the proof.
\end{proof}


\begin{theorem}
\label{theo:left cart stable by colimit}
Let $A$ be an $\io$-category and $F:I\to \ocatm_{/A^\sharp}$ be a diagram that is pointwise a left cartesian fibration. The induced morphism 
$\colim_IF$ is a left cartesian fibration over $A^\sharp$.
\end{theorem}
\begin{proof}
Consider the functor $\psi:\Theta_{/A}\to \Arr(\ocatm)$ whose value on $j:b\to A$ is $\colim_I j^*F$.
As $F$ is pointwise a left cartesian fibration, the corollary \ref{cor:fib over a colimit} induces equivalences
$$\colim_{\Theta_{/A}}\psi:= \colim_{j:b\to A}\colim_I j^*F\sim \colim_I \colim_{j:b\to A}j^*F\sim \colim_I F$$


 The functor $\psi$  is cartesian according to lemma \ref{lemma: colimit of fib over b2}, and as $\codom \psi$ as a special colimit (given by $A^\sharp$), so has $\psi$ according to proposition \ref{prop:special colimit marked case}. In particular, this implies that for any $j:b\to A$, the following canonical morphism
$$\colim_I j^* F=: \psi(j)\to j^*\colim_{\Theta_{/A}}\psi\sim j^* \colim_I F$$
is an equivalence. As the left object is a left cartesian fibration according to lemma \ref{lemma: colimit of fib over b}, so is the right one.
As this is true for any $j:b\to A$, the corollary \ref{cor:on the fact that fib are define against representable} implies that $ \colim_I F$ is a left cartesian fibration.
\end{proof}

\begin{cor}
\label{cor:inclusion of lcatt into the slice preserves colimits}
Let $A$ be an $\io$-category. The inclusion $\LCart(A^\sharp)\to \ocatm_{/A^\sharp}$ preserves both colimits and limits.
\end{cor}
\begin{proof}
The preservation of limits is a consequence of the fact that that this inclusion is a right adjoint. The preservation of colimits is a direct consequence of the theorem \ref{theo:left cart stable by colimit}.
\end{proof}

\p We now use the last theorem to provide an alternative explicit expression of the left cartesian fibration $\Fb h^0_{[C,1]}$. We obtain this in the theorem \ref{theo:equivalence betwen slice and join}.





\begin{prop}
\label{prop:appendice version equivalence betwen slice and join strict word}
Let $C$ be an $\zo$-category with an atomic and loop free basis. The canonical projection $\gamma:1\costar C^\flat \to [C,1]^\sharp$ is a left cartesian fibration.
\end{prop}
\begin{proof}
Let $C$ be such $\zo$-category.
The corollary \ref{cor:otimes et op}, the theorem \ref{theo:join preserves stict VMG version} and the proposition \ref{prop:suspension preserves stricte} imply that both the domain and the codomain of $\gamma$ are strict. We can then show the result in $\zocatm$.
By construction, the basis of $1\costar \lambda C$ is given by the graduated set: 
$$(B_{1\costar \lambda C})_n:=
\left\{
\begin{array}{ll}
\{\emptyset \costar c,c\in (B_{C})_0\}\cup \{\emptyset \costar c,c \in (B_C)_0\}&\mbox{if $n=0$}\\
\{1 \costar c,c\in (B_{C})_{n-1}\}\cup \{\emptyset \costar c,c\in (B_C)_n\} &\mbox{if $n>0$}\\
\end{array}\right.
$$
where $B_C$ is the basis of $C$. The derivative is induced by: 
$$\partial (1\costar c):= 1\costar \partial c + (-1)^{|c|}\emptyset\otimes c~~~~~~~~~~\partial(\emptyset\star c):= \emptyset\costar \partial c$$
where we set the convention $\partial c:=0$ if $|c|=0$.
Let $n$ be an integer and $x$ an element of $(1\costar \lambda C)_n$. The induced morphism $\Db_n\to 1\costar C^\flat$ is marked if and only if there is no element of shape $\emptyset\star c$ in the support of $x$. 


For an integer $n>0$, we define $s_n: (\Sigma \lambda C)_n\to (1\costar \lambda C)_n$ as the unique group morphism fulfilling $$s_n(\Sigma c):= 1\costar c$$ for $c$ any element of $\lambda C_{n-1}$. Remark that for any non negative integer $n$, and any element $d$ of $(1\costar \lambda C)_n$, $s_n(d)$ is contained in $d$. However, the family of morphism $\{s_n\}_{n\in \Nb}$ does not commute with the derivative. Let $n$ be an integer and $x$ an element of $(1\costar \lambda C)_n$. The induced morphism $\Db_n\to 1\costar C^\flat$ is therefore marked if and only if $x$ is equal to $s_n\gamma_n(x)$.




Eventually,
we recall that $(\Db_n)_t\otimes[1]^\sharp$ is the colimit of the diagram:
% q.uiver.app/#q=WzAsMyxbMiwwLCJcXHRhdV4gaV8gbihcXERiX25eXFxmbGF0XFxvdGltZXNbMV1eXFxzaGFycCkiXSxbMSwwLCJcXERiX25eXFxmbGF0XFxvdGltZXNcXHswXFx9XFxjb3Byb2QgXFxEYl9uXlxcZmxhdFxcb3RpbWVzXFx7MVxcfSJdLFswLDAsIihcXERiX24pX3RcXG90aW1lc1xcezBcXH1cXGNvcHJvZCAoXFxEYl9uKV90XFxvdGltZXNcXHsxXFx9Il0sWzEsMl0sWzEsMF1d
\[\begin{tikzcd}
	{(\Db_n)_t\otimes\{0\}\coprod (\Db_n)_t\otimes\{1\}} & {\Db_n^\flat\otimes\{0\}\coprod \Db_n^\flat\otimes\{1\}} & {\tau^ i_ n(\Db_n^\flat\otimes[1]^\sharp)}
	\arrow[from=1-2, to=1-1]
	\arrow[from=1-2, to=1-3]
\end{tikzcd}\]
We then have to show that for any integer $n$, any diagram of shape % q.uiver.app/#q=WzAsNCxbMSwwLCIxXFxjb3N0YXIgXFxsYW1iZGEgQyJdLFsxLDEsIlxcU2lnbWEgXFxsYW1iZGEgQyJdLFswLDEsIlxcbGFtYmRhXFxEYl9uXFxvdGltZXNbMV0iXSxbMCwwLCJcXGxhbWJkYVxcRGJfblxcb3RpbWVzXFx7MFxcfVxcY3VwIFxcbGFtYmRhXFxwYXJ0aWFsXFxEYl9uXFxvdGltZXNbMV0iXSxbMywyXSxbMiwxLCJmIiwyXSxbMywwLCJnIl0sWzAsMV1d
\[\begin{tikzcd}
	{\lambda\Db_n\otimes\{0\}\cup \lambda\partial\Db_n\otimes[1]} & {1\costar \lambda C} \\
	{\lambda\Db_n\otimes[1]} & {\Sigma \lambda C}
	\arrow[from=1-1, to=2-1]
	\arrow["f"', from=2-1, to=2-2]
	\arrow["g", from=1-1, to=1-2]
	\arrow[from=1-2, to=2-2]
\end{tikzcd}\]
with $f(e_n\otimes[1])$ and $f(e^\alpha_k\otimes[1])$ for $\alpha\in\{-,+\}$ and $k<n$ correponding to a marked cell, admits a unique lifting $l$ with the following extra condition: if $n>0$, if $f(e_n\otimes[1])$ is null and if $g(e_n\otimes\{0\})$ corresponds to a marked cell, then $l(e_n\otimes[1])$ is null and $l(e_n\otimes\{1\})$ corresponds to a marked cell. 


Suppose first that $n=0$. We set $l_0:\lambda (\Db_0\otimes[1])_0\to (1\costar \lambda C)_0$ as the unique group morphism extending $g_0$ and such that 
$$l_0(e_0\otimes\{1\}):= \partial s_1(f_1(e_0\otimes[1])+ g_0(e_0\otimes\{1\}).$$
We also define $l_1:\lambda (\Db_0\otimes[1])_1\to (1\costar \lambda C)_1$ as the group morphism characterized by: 
$$l_1(e_0\otimes[1]):= s_1(f_1(e_0\otimes[1])).$$
For $k>1$, we set $l_k:\lambda (\Db_0\otimes[1])_k\to (1\costar \lambda C)_k$ as the constant morphism on $0$.
We directly deduce the equality $\partial l= l \partial$.
We then have defined the desired lifting, which is obviously the unique one possible.


Suppose now that $n>0$. We set $l_k:=g_k:\lambda (\Db_n\otimes[1])_k\to (1\costar \lambda C)_k$ for $k<n$ and $l_n:\lambda (\Db_n\otimes[1])_n\to (1\costar \lambda C)_n$ as the unique group morphism extending $g_n$ and such that 
$$l_n(e_n\otimes\{1\}) := (-1)^\alpha \partial s_{n+1}( f(e_n\otimes[1])) - (-1)^\alpha s_{n}( f((\partial e_n)\otimes[1])) + g_n(e_n\otimes\{0\})$$
where $\alpha$ is $+$ if $n$ is even and $-$ if not.
We define $l_{n+1}:\lambda (\Db_n\otimes[1])_{n+1}\to (1\costar \lambda C)_{n+1}$ as the group morphism characterized by: 
$$l_{n+1}(e_n\otimes[1]):= s_{n+1}(f_{n+1}(e_n\otimes[1])).$$
Eventually, for $k>n$, we set $l_k:\lambda (\Db_n\otimes[1])_k\to (1\costar \lambda C)_k$ as the constant morphism on $0$.


For an integer $k<n$ and $\alpha\in\{-,+\}$, as the $(k+1)$-cell corresponding to $g_{k+1}(e_k^\alpha\otimes [1])$ is marked, we have an equality
$$g_{k+1}(e_k^\alpha\otimes [1]) = s_{k+1}f_{k+1}(e_k^\alpha\otimes [1]).$$
This then implies the equalities
$$\begin{array}{rcl}
\partial(l_{n+1}(e_n\otimes[1])) &=& l_{n+1}(\partial (e_n\otimes[1]))\\
\partial(l_{n}(e_n\otimes \{1\}))&=& g_{n-1}(\partial e_n\otimes \{1\})
\end{array}$$
As it was the only non trivial case, we have $l\partial = \partial l.$
We then have defined the desired lifting, which is obviously the unique one possible. Moreover, if we suppose that $f(e_n\otimes[1])$ is null and $g(e_n\otimes\{0\})$ corresponds to a marked cell, this implies that 
$ s_{n+1}( f(e_n\otimes[1])) =0$ and that the $g_n(e_n\otimes\{0\})$ is in the image of $s_n$. The object $f(e_n\otimes[1])$ also is in the image of $s_n$ and so corresponds to a marked cell.
\end{proof}

\begin{lemma}
\label{lemma:equivalence betwen slice and join strict word}
There is a unique morphism $1\costar C^\flat\to [C,1]^\sharp_{0/}$ fitting in a square
% q.uiver.app/#q=WzAsNCxbMSwwLCJbQywxXV5cXHNoYXJwX3swL30iXSxbMCwwLCIxIl0sWzAsMSwiMVxcY29zdGFydCBDXlxcZmxhdCJdLFsxLDEsIltDLDFdXlxcc2hhcnAiXSxbMSwyXSxbMSwwXSxbMCwzXSxbMiwzXSxbMiwwLCIiLDEseyJzdHlsZSI6eyJib2R5Ijp7Im5hbWUiOiJkb3R0ZWQifX19XV0=
\[\begin{tikzcd}
	1 & {[C,1]^\sharp_{0/}} \\
	{1\costar C^\flat} & {[C,1]^\sharp}
	\arrow[from=1-1, to=2-1]
	\arrow[from=1-1, to=1-2]
	\arrow[from=1-2, to=2-2]
	\arrow[from=2-1, to=2-2]
	\arrow[dotted, from=2-1, to=1-2]
\end{tikzcd}\]
This morphism is an equivalence whenever $C$ is a globular sum.
\end{lemma}
\begin{proof}
We have by construction a cocartesian square
% https://q.uiver.app/#q=WzAsNCxbMSwwLCJDXlxcZmxhdFxcb3RpbWVzWzFdXlxcc2hhcnAiXSxbMCwwLCJDXlxcZmxhdFxcb3RpbWVzXFx7MFxcfSJdLFswLDEsIjEiXSxbMSwxLCIxXFxjb3N0YXJ0IENeXFxmbGF0Il0sWzEsMl0sWzEsMF0sWzIsM10sWzAsM10sWzMsMSwiIiwxLHsic3R5bGUiOnsibmFtZSI6ImNvcm5lciJ9fV1d
\[\begin{tikzcd}
	{C^\flat\otimes\{0\}} & {C^\flat\otimes[1]^\sharp} \\
	1 & {1\costar C^\flat}
	\arrow[from=1-1, to=2-1]
	\arrow[from=1-1, to=1-2]
	\arrow[from=2-1, to=2-2]
	\arrow[from=1-2, to=2-2]
	\arrow["\lrcorner"{anchor=center, pos=0.125, rotate=180}, draw=none, from=2-2, to=1-1]
\end{tikzcd}\]
which implies that $1\to 1\costar C^\flat$ is initial. This directly implies the first assertion.
We now prove the second assertion. We suppose that $C$ is a globular sum $a$. The $\io$-categories $1\costar a^\flat$ is strict according to proposition \ref{prop:tensor of glboer are strics}. Proposition \ref{prop:appendice version equivalence betwen slice and join strict word} states that the canonical morphism $1\costar a^\flat \to [a,1]^\sharp$ is a left cartesian fibration. As the comparison map is initial by left cancellation, this concludes the proof.	 
\end{proof}

\begin{prop}
\label{prop:equivalence betwen slice and join strict word2}
Let $b$ be a globular form and $j:b\to C$ a morphism between $\io$-categories. The following diagram is cartesian
% https://q.uiver.app/#q=WzAsNCxbMSwxLCJbQywxXV5cXHNoYXJwIl0sWzEsMCwiW0MsMV1eXFxzaGFycF97MC99Il0sWzAsMSwiW2IsMV1eXFxzaGFycCJdLFswLDAsIjFcXGNvc3RhciBiXlxcZmxhdFxcY29wcm9kX3tiXlxcZmxhdH1DXlxcZmxhdCJdLFsyLDAsIltqLDFdXlxcc2hhcnAiLDJdLFszLDFdLFszLDJdLFsxLDBdXQ==
\[\begin{tikzcd}
	{1\costar b^\flat\coprod_{b^\flat}C^\flat} & {[C,1]^\sharp_{0/}} \\
	{[b,1]^\sharp} & {[C,1]^\sharp}
	\arrow["{[j,1]^\sharp}"', from=2-1, to=2-2]
	\arrow[from=1-1, to=1-2]
	\arrow[from=1-1, to=2-1]
	\arrow[from=1-2, to=2-2]
\end{tikzcd}\]
\end{prop}
\begin{proof}
The lemma \ref{lemma:equivalence betwen slice and join strict word} implies that the morphism $1\costar b^\flat\to [b,1]^\sharp$ is equivalent to $\Fb h_0^{[b,1]}$.
We then have to check that the canonical morphism 
\begin{equation}
\label{eq:in a tecnical lemma}
\Fb h_0^{[b,1]}\coprod_{b^\flat}C^\flat\to [j,1]^*\Fb h_0^{[C,1]}
\end{equation}
is an equivalence. According to theorem \ref{theo:left cart stable by colimit}, the two objects are left cartesian fibrations, and we then have to check that this morphism induce equivalences on fibers. Remark furthermore that the two morphisms $\{0\}\to [b,1]^\sharp$ and $\{1\}\to [b,1]^\sharp$ are discrete Conduché functors and then exponentiable according to proposition \ref{prop:pullback by conduch marked preserves colimit}. The fibers on $0$ and $1$ of the morphism \eqref{eq:in a tecnical lemma} then corresponds to the equivalences
$$1\coprod_{\emptyset}\emptyset\sim 1~~~~\mbox{ and }~~~~ b\coprod_bC\sim C.$$
\end{proof}




\begin{theorem}
\label{theo:equivalence betwen slice and join}
Let $C$ be a $\io$-category. The left cartesian fibration $\Fb h^0_{[C,1]}$ is equivalent to the projection 
$1\costar C^\flat\to [C,1]^\sharp$.
\end{theorem}
\begin{proof}
Let $i:[b,1]^\sharp\to [C,1]^\sharp$ be any morphism. The proposition
\ref{prop:equivalence betwen slice and join strict word2} states that the following square is cartesian:
 % q.uiver.app/#q=WzAsNCxbMSwwLCJbQywxXV5cXHNoYXJwX3swL30iXSxbMSwxLCJbQywxXV5cXHNoYXJwIl0sWzAsMSwiW2IsMV1eXFxzaGFycCJdLFswLDAsIjFcXGNvc3RhcnQgYl5cXGZsYXRcXGNvcHJvZF97Yl5cXGZsYXR9Q15cXGZsYXQiXSxbMiwxXSxbMywyXSxbMywwXSxbMCwxXV0=
\[\begin{tikzcd}
	{1\costar b^\flat\coprod_{b^\flat}C^\flat} & {[C,1]^\sharp_{0/}} \\
	{[b,1]^\sharp} & {[C,1]^\sharp}
	\arrow[from=2-1, to=2-2]
	\arrow[from=1-1, to=2-1]
	\arrow[from=1-1, to=1-2]
	\arrow[from=1-2, to=2-2]
\end{tikzcd}\]
Eventually, remark that we have an equivalence 
$$\colim_{b\to C}[b,1]\sim [C,1].$$
The theorem \ref{theo:pullback along un marked cartesian fibration} then induces equivalences
$$[C,1]^\sharp_{0/}\sim \colim_{i:b\to C}1\costar b^\flat\coprod_{b^\flat}C^\flat \sim 1\costar C^\flat\coprod_{C^\flat}C^\flat\sim 1\costar C^\flat$$
over $[C,1]^\sharp$. This concludes the proof.
\end{proof}

\begin{cor}
\label{cor:cor of the past10}
Let $b$ be a globular form and $j:b\to C$ any morphism. The following square is cartesian:
% https://q.uiver.app/#q=WzAsNCxbMSwxLCJbQywxXSJdLFsxLDAsIjFcXGNvc3RhciBDIl0sWzAsMSwiW2IsMV0iXSxbMCwwLCIxXFxjb3N0YXIgYlxcY29wcm9kX2IgQyJdLFsxLDBdLFszLDJdLFsyLDBdLFszLDFdXQ==
\[\begin{tikzcd}
	{1\costar b\coprod_b C} & {1\costar C} \\
	{[b,1]} & {[C,1]}
	\arrow[from=1-2, to=2-2]
	\arrow[from=1-1, to=2-1]
	\arrow[from=2-1, to=2-2]
	\arrow[from=1-1, to=1-2]
\end{tikzcd}\]
\end{cor}
\begin{proof}
We apply the functor $(\uvar)^\natural$ to the cartesian square given in proposition \ref{prop:equivalence betwen slice and join strict word2} and the equivalence given in theorem \ref{theo:equivalence betwen slice and join}.
\end{proof}



\begin{cor}
\label{cor:cor of the past3}
Let $C$ be an $\io$-category. We denote by $\gamma:C\star 1\to [C,1]$ and $\gamma':1\costar C\to [C,1]$ the two canonical projections. The functors $\gamma^*:\ocat_{/[C,1]}\to \ocat_{/C\star 1}$ and $\gamma^*:\ocat_{/[C,1]}\to \ocat_{/1\costar C}$ preserve colimits. 
\end{cor}
\begin{proof}
We have a cocartesian square
% https://q.uiver.app/#q=WzAsNCxbMCwxLCJbQywxXV5cXGZsYXQiXSxbMCwwLCIoMVxcY29zdGFyIEMpXlxcZmxhdCJdLFsxLDEsIltDLDFdXlxcc2hhcnAiXSxbMSwwLCIxXFxjb3N0YXIgQ15cXGZsYXQiXSxbMSwwLCJcXGdhbW1hXlxcZmxhdCIsMl0sWzMsMl0sWzAsMl0sWzEsM10sWzEsMiwiIiwxLHsic3R5bGUiOnsibmFtZSI6ImNvcm5lciJ9fV1d
\[\begin{tikzcd}
	{(1\costar C)^\flat} & {1\costar C^\flat} \\
	{[C,1]^\flat} & {[C,1]^\sharp}
	\arrow["{\gamma^\flat}"', from=1-1, to=2-1]
	\arrow[from=1-2, to=2-2]
	\arrow[from=2-1, to=2-2]
	\arrow[from=1-1, to=1-2]
	\arrow["\lrcorner"{anchor=center, pos=0.125}, draw=none, from=1-1, to=2-2]
\end{tikzcd}\]
The theorem \ref{theo:equivalence betwen slice and join} implies that the right hand morphism is a left cartesian fibration, and $\gamma^\flat$ is then a classified left cartesian fibration. The result is then a direct consequence of theorem \ref{theo:pullback along un marked cartesian fibration}.
The other assertion follows by duality.
\end{proof}


\subsection{Smooth and proper morphisms}
\p 
For a marked $\io$-category $C$, we denote by \textit{$\LCart(C)$} \sym{(lcart@$\LCart(\uvar)$}\sym{(rcart@$\RCart(\uvar)$} (resp. $\RCart(C)$) the full sub $\iun$-category of $\ocatm_{/C}$ whose objects are left cartesian fibrations. We can equivalently define $\LCart(C)$ as the localization of $\ocatm_{/C}$ along $\widehat{\I_{/C}}$. For $E$, $F$ two objects of $\LCart(C)$ corresponding respectively to two left cartesian fibrations
$p:X\to C$ and $q:X\to C$, we denote by \wcnotation{$\Map(E,F)$}{(map@$\Map(\uvar,\uvar)$} the $\io$-category fitting in the cocartesian square:
% https://q.uiver.app/#q=WzAsNCxbMCwwLCJcXE1hcChFLEYpIl0sWzEsMCwiXFx1SG9tKFgsWSkiXSxbMSwxLCJcXHVIb20oWCxDKSJdLFswLDEsIlxce3BcXH0iXSxbMSwyLCJxXyEiXSxbMywyXSxbMCwxXSxbMCwzXSxbMCwyLCIiLDEseyJzdHlsZSI6eyJuYW1lIjoiY29ybmVyIn19XV0=
\[\begin{tikzcd}
	{\Map(E,F)} & {\uHom(X,Y)} \\
	{\{p\}} & {\uHom(X,C)}
	\arrow["{q_!}", from=1-2, to=2-2]
	\arrow[from=2-1, to=2-2]
	\arrow[from=1-1, to=1-2]
	\arrow[from=1-1, to=2-1]
	\arrow["\lrcorner"{anchor=center, pos=0.125}, draw=none, from=1-1, to=2-2]
\end{tikzcd}\]

\p We recall that a left cartesian fibration $X\to C$ is \textit{classified} when there exists a cartesian square: 
% q.uiver.app/#q=WzAsNCxbMCwwLCJYIl0sWzAsMSwiQyJdLFsxLDEsIkNeXFxzaGFycCJdLFsxLDAsIlgnIl0sWzAsM10sWzMsMl0sWzAsMV0sWzEsMl0sWzAsMiwiIiwxLHsic3R5bGUiOnsibmFtZSI6ImNvcm5lciJ9fV1d
\[\begin{tikzcd}
	X & {X'} \\
	C & {C^\sharp}
	\arrow[from=1-1, to=1-2]
	\arrow[from=1-2, to=2-2]
	\arrow[from=1-1, to=2-1]
	\arrow[from=2-1, to=2-2]
	\arrow["\lrcorner"{anchor=center, pos=0.125}, draw=none, from=1-1, to=2-2]
\end{tikzcd}\]
We denote by \wcnotation{$\LCartc(C)$}{(lcart@$\LCartc(\uvar)$} the full sub $\iun$-category of $\LCart(C)$ whose objects are classified left cartesian fibrations.





\p Remark that every morphism $f:C\to D$ induces an adjunction
\[\begin{tikzcd}
	{f_!:\ocat_{/C}} & {\ocat_{/D}:f^*}
	\arrow[shift left=2, from=1-1, to=1-2]
	\arrow[shift left=2, from=1-2, to=1-1]
\end{tikzcd}\]
where the left adjoint $f_!$ is the composition and the right one is the pullback.
This induces an adjunction at the level of localized $\iun$-category:
\[\begin{tikzcd}
	{\Lb f_!:\LCart(C)} & {\LCart(D):\Rb f^*=f^*}
	\arrow[shift left=2, from=1-1, to=1-2]
	\arrow[shift left=2, from=1-2, to=1-1]
\end{tikzcd}\]
 
\p A morphism $f:C\to D$ is \wcnotion{smooth}{smooth morphism} if $f^*:\ocatm_{/D}\to \ocatm_{/C}$ preserves colimits, and for every cartesian square of the form
% q.uiver.app/#q=WzAsNixbMCwwLCJDJyciXSxbMiwwLCJDIl0sWzIsMSwiRCJdLFsxLDEsIkQnIl0sWzAsMSwiRCcnIl0sWzEsMCwiQyciXSxbMCw1LCJ2JyJdLFs1LDFdLFsxLDIsImYiXSxbMywyXSxbNCwzLCJ2IiwyXSxbNSwzXSxbMCw0XSxbMCwzLCIiLDIseyJzdHlsZSI6eyJuYW1lIjoiY29ybmVyIn19XSxbNSwyLCIiLDIseyJzdHlsZSI6eyJuYW1lIjoiY29ybmVyIn19XV0=
\begin{equation}
\label{eq:smooth diagram}
\begin{tikzcd}
	{C''} & {C'} & C \\
	{D''} & {D'} & D
	\arrow["{v'}", from=1-1, to=1-2]
	\arrow[from=1-2, to=1-3]
	\arrow["f", from=1-3, to=2-3]
	\arrow[from=2-2, to=2-3]
	\arrow["v"', from=2-1, to=2-2]
	\arrow[from=1-2, to=2-2]
	\arrow[from=1-1, to=2-1]
	\arrow["\lrcorner"{anchor=center, pos=0.125}, draw=none, from=1-1, to=2-2]
	\arrow["\lrcorner"{anchor=center, pos=0.125}, draw=none, from=1-2, to=2-3]
\end{tikzcd}
\end{equation}
if $v$ is inital, so is $v'$.
When $f$ is smooth, the functor $f^*$ admits a left adjoint
% q.uiver.app/#q=WzAsMixbMCwwLCJmXio6XFxvY2F0bV97L0R9Il0sWzEsMCwiXFxvY2F0bV97L0N9OmZfKiJdLFsxLDAsIiIsMix7Im9mZnNldCI6LTJ9XSxbMCwxLCIiLDIseyJvZmZzZXQiOi0yfV0sWzMsMiwiIiwyLHsibGV2ZWwiOjEsInN0eWxlIjp7Im5hbWUiOiJhZGp1bmN0aW9uIn19XV0=
\[\begin{tikzcd}
	{f^*:\ocatm_{/D}} & {\ocatm_{/C}:f_*}
	\arrow[""{name=0, anchor=center, inner sep=0}, shift left=2, from=1-2, to=1-1]
	\arrow[""{name=1, anchor=center, inner sep=0}, shift left=2, from=1-1, to=1-2]
	\arrow["\dashv"{anchor=center, rotate=-90}, draw=none, from=1, to=0]
\end{tikzcd}\]
and as $f^*$ preserves initial morphisms, this induces a derived adjunction:
% q.uiver.app/#q=WzAsMixbMCwwLCJcXExiIGZeKjpcXExDYXJ0KEQpIl0sWzEsMCwiXFxMQ2FydChDKTpcXFJiIGZfKiJdLFsxLDAsIiIsMix7Im9mZnNldCI6LTJ9XSxbMCwxLCIiLDIseyJvZmZzZXQiOi0yfV0sWzMsMiwiIiwyLHsibGV2ZWwiOjEsInN0eWxlIjp7Im5hbWUiOiJhZGp1bmN0aW9uIn19XV0=
\[\begin{tikzcd}
	{\Lb f^*:\LCart(D)} & {\LCart(C):\Rb f_*}
	\arrow[""{name=0, anchor=center, inner sep=0}, shift left=2, from=1-2, to=1-1]
	\arrow[""{name=1, anchor=center, inner sep=0}, shift left=2, from=1-1, to=1-2]
	\arrow["\dashv"{anchor=center, rotate=-90}, draw=none, from=1, to=0]
\end{tikzcd}\]
where $\Rb f_*$ is just the restriction of $f_*$.

\begin{prop}
\label{prop:projection are smooth}
Let $I, J$ be two marked $\io$-categories. The projection $I\times J\to I$ is smooth. 
\end{prop}
\begin{proof}
This is a direct consequence of the fact that cartesian product preserves colimits and initial morphisms.
\end{proof}
\begin{prop}
\label{prop:left cartesian fibration are smooth}
Classified right cartesian fibrations are smooth.
\end{prop}
\begin{proof}
The theorem \ref{theo:pullback along un marked cartesian fibration} states that $f^*$ preserves colimits. Suppose given a diagram of shape \eqref{eq:smooth diagram}. As initial morphisms are the smallest cocomplete class containing morphism $I$, and as $f^*$ preserves colimits, one can suppose that $v$ belongs to $I$, and then is a left Gray deformation retract. To conclude, one applies proposition
\ref{prop:left Gray transfomration stable under pullback along cartesian fibration}.
\end{proof}


\p A morphism $f:C\to D$ is \wcnotion{proper}{proper morphism} if $f^*:\ocatm_{/D}\to \ocatm_{/C}$ preserves colimits and for every cartesian square of the form
% q.uiver.app/#q=WzAsNixbMCwwLCJDJyciXSxbMiwwLCJDIl0sWzIsMSwiRCJdLFsxLDEsIkQnIl0sWzAsMSwiRCcnIl0sWzEsMCwiQyciXSxbMCw1LCJ2JyJdLFs1LDFdLFsxLDIsImYiXSxbMywyXSxbNCwzLCJ2IiwyXSxbNSwzXSxbMCw0XSxbMCwzLCIiLDIseyJzdHlsZSI6eyJuYW1lIjoiY29ybmVyIn19XSxbNSwyLCIiLDIseyJzdHlsZSI6eyJuYW1lIjoiY29ybmVyIn19XV0=
\begin{equation}
\label{eq:proper diagram}
\begin{tikzcd}
	{C''} & {C'} & C \\
	{D''} & {D'} & D
	\arrow["{v'}", from=1-1, to=1-2]
	\arrow[from=1-2, to=1-3]
	\arrow["f", from=1-3, to=2-3]
	\arrow[from=2-2, to=2-3]
	\arrow["v"', from=2-1, to=2-2]
	\arrow[from=1-2, to=2-2]
	\arrow[from=1-1, to=2-1]
	\arrow["\lrcorner"{anchor=center, pos=0.125}, draw=none, from=1-1, to=2-2]
	\arrow["\lrcorner"{anchor=center, pos=0.125}, draw=none, from=1-2, to=2-3]
\end{tikzcd}
\end{equation}
if $v$ is final, so is $v'$.
A morphism $f$ is then proper if and only if $f^{\circ}$ is smooth. Propositions \ref{prop:projection are smooth} and \ref{prop:left cartesian fibration are smooth} then imply that projections and classified right cartesian fibrations are proper.



\p We denote by $\bot:\ocatm\to \ocat$ the left Kan extension of the functor $t\Theta\to \ocat$ that sends $a^\flat$ on $a$ and $(\Db_{n+1})_t$ on $\Db_n$. Roughly speaking, $\bot$ sends a marked $\io$-category to it's localization by marked cells. By abuse of notation, we also denote\sym{((g3@$\bot$} $\bot: 
\Arr(\ocatm)\to \ocat$, the composite functor 
$$\Arr(\ocatm)\xrightarrow{\dom}\ocatm\xrightarrow{\bot} \ocat$$
This functor preserves colimits and sends initial and final morphisms to equivalences. For any object $E$ of $\LCart(A)$ and for any morphism $i:A\to B$, we then have a canonical equivalence 
\begin{equation}
\label{eq:bot kill pull}
\bot \Lb i_! E\sim \bot E.
\end{equation}


Let $A$ be an $\io$-category and $a:1\to A^\sharp$ an object of $A$. 
According to proposition \ref{prop:explicit factoryzation}, the factorisation of $a:1\to A^\sharp$ in a final morphism followed by a right cartesian fibration is given by the canonical inclusion $\{a\}\to A^\sharp_{a/}$ and the canonical projection $\pi_a:A^\sharp_{a/}\to A^\sharp$.
Let $E$ be an object of $\LCart(A^\sharp)$ corresponding to a left cartesian fibration $p:X\to A^\sharp$.
We then have a diagram
% q.uiver.app/#q=WzAsNixbMiwwLCJYIl0sWzIsMSwiQV5cXHNoYXJwIl0sWzAsMCwiWF9hIl0sWzEsMCwiWF97L2F9Il0sWzAsMSwiXFx7YVxcfSJdLFsxLDEsIkFeXFxzaGFycF97YS99Il0sWzIsNF0sWzQsNV0sWzMsMF0sWzAsMSwicCJdLFszLDVdLFs1LDEsIlxccGlfYSIsMl0sWzIsMywiaSJdLFsyLDUsIiIsMCx7InN0eWxlIjp7Im5hbWUiOiJjb3JuZXIifX1dLFszLDEsIiIsMCx7InN0eWxlIjp7Im5hbWUiOiJjb3JuZXIifX1dXQ==
\[\begin{tikzcd}
	{X_a} & {X_{/a}} & X \\
	{\{a\}} & {A^\sharp_{a/}} & {A^\sharp}
	\arrow[from=1-1, to=2-1]
	\arrow[from=2-1, to=2-2]
	\arrow[from=1-2, to=1-3]
	\arrow["p", from=1-3, to=2-3]
	\arrow[from=1-2, to=2-2]
	\arrow["{\pi_a}"', from=2-2, to=2-3]
	\arrow["i", from=1-1, to=1-2]
	\arrow["\lrcorner"{anchor=center, pos=0.125}, draw=none, from=1-1, to=2-2]
	\arrow["\lrcorner"{anchor=center, pos=0.125}, draw=none, from=1-2, to=2-3]
\end{tikzcd}\]
and the morphism $i$ is final as $p$ is proper. As $\bot$ sends final morphisms to equivalences, we then have an invertible natural transformation: 
\begin{equation}
\label{eq:explicit derived fiber}
\Rb a^*E\sim \bot \Rb a^*E\sim \bot \Rb \pi_a^*E
\end{equation}

\begin{prop}
\label{prop:fiber preserves colimits}
The functor $\Rb a^*:\LCart(A^\sharp)\to \LCart(1)\sim \ocat$ preserves colimits. 
\end{prop}
\begin{proof}
As $\pi_a$ is a right cartesian fibration, it is smooth and $\Rb \pi_a^*$ then preserves colimits. The functor $\bot$ also preserves them. The result then follows from the equivalence \eqref{eq:explicit derived fiber}.
\end{proof}

\p Let $E$ be an object of $\ocatm_{/A^\sharp}$ corresponding to a morphism $X\to A^\sharp$. We denote $\tilde{X}\to A^\sharp$ the left fibrant replacement of $E$. We then have a diagram
% https://q.uiver.app/#q=WzAsNixbMCwxLCJYIl0sWzEsMSwiXFx0aWxkZXtYfSJdLFsyLDEsIkFeXFxzaGFycCJdLFsyLDAsIkFeXFxzaGFycF97YS99Il0sWzEsMCwiXFx0aWxkZXtYfV97YS99Il0sWzAsMCwiWF97YS99Il0sWzUsNF0sWzQsM10sWzEsMiwiXFxGYiBFIiwyXSxbMCwxXSxbMywyLCJcXHBpX2EiXSxbNCwxXSxbNSwwXSxbNSwxLCIiLDIseyJzdHlsZSI6eyJuYW1lIjoiY29ybmVyIn19XSxbNCwyLCIiLDIseyJzdHlsZSI6eyJuYW1lIjoiY29ybmVyIn19XV0=
\[\begin{tikzcd}
	{X_{a/}} & {\tilde{X}_{a/}} & {A^\sharp_{a/}} \\
	X & {\tilde{X}} & {A^\sharp}
	\arrow[from=1-1, to=1-2]
	\arrow[from=1-2, to=1-3]
	\arrow["{\Fb E}"', from=2-2, to=2-3]
	\arrow[from=2-1, to=2-2]
	\arrow["{\pi_a}", from=1-3, to=2-3]
	\arrow[from=1-2, to=2-2]
	\arrow[from=1-1, to=2-1]
	\arrow["\lrcorner"{anchor=center, pos=0.125}, draw=none, from=1-1, to=2-2]
	\arrow["\lrcorner"{anchor=center, pos=0.125}, draw=none, from=1-2, to=2-3]
\end{tikzcd}\] 
 As $\pi_a$ is smooth, the canonical morphism 
$X_{a/}\to \tilde{X}_{a/}$ is initial. Combined with \eqref{eq:explicit derived fiber}, this induces an equivalence:
\begin{equation}
\label{eq:explicit derived fiber2}
\Rb a^*(\Fb E)\sim \bot X_{/a}
\end{equation}
\begin{prop}
\label{prop:quillent theorem A}
For a morphism $X\to A^\sharp$, and an object $a$ of $A$, we denote by $X_{/a}$ the marked $\io$-category fitting in the following cartesian square: 
% https://q.uiver.app/#q=WzAsNCxbMSwxLCJBXlxcc2hhcnAiXSxbMCwxLCJBXlxcc2hhcnBfe2EvfSJdLFsxLDAsIlgiXSxbMCwwLCJYX3thL30iXSxbMSwwXSxbMiwwXSxbMywyXSxbMywxXSxbMywwLCIiLDEseyJzdHlsZSI6eyJuYW1lIjoiY29ybmVyIn19XV0=
\[\begin{tikzcd}
	{X_{a/}} & X \\
	{A^\sharp_{a/}} & {A^\sharp}
	\arrow[from=2-1, to=2-2]
	\arrow[from=1-2, to=2-2]
	\arrow[from=1-1, to=1-2]
	\arrow[from=1-1, to=2-1]
	\arrow["\lrcorner"{anchor=center, pos=0.125}, draw=none, from=1-1, to=2-2]
\end{tikzcd}\]
We denote by $\bot:\ocatm\to \ocat$ the functor sending a marked $\io$-category to its localization by marked cells.
\begin{enumerate}
\item Let $E$, $F$ be two elements of $\ocatm_{/A^\sharp}$ corresponding to morphisms $X\to A^\sharp$, $Y\to A^\sharp$, and
 $\phi:E\to F$ a morphism between them. The induced morphism $\Fb\phi:\Fb E\to \Fb F$ is an equivalence if and only if for any object $a$ of $A$, the induced morphism 
$$\bot X_{/a}\to \bot Y_{/a}$$ 
is an equivalence of $\io$-categories.
\item A morphism $X\to A^\sharp$ is initial if and only if for any object $a$ of $A$, $\bot X_{/a}$ is the terminal $\io$-category.
\end{enumerate}
\end{prop}
\begin{proof}
The first assertion is a direct consequence of the equation \eqref{eq:explicit derived fiber2} and of the fact that equivalences between left cartesian fibrations are detected on fibers. 


A morphism $p:X\to A$ is initial if and only if $\Fb p$ is equivalent to the identity of $A^\sharp$, and according to the first assertion, if and only if for any object $a$ of $A$, the canonical morphism $\bot X_{a/}\to \bot A^\sharp_{a/}$ is an equivalence. However, the canonical morphism $\{a\}\to A_{/a}^\sharp$ is final, and $\bot A^\sharp_{a/}$ is then the terminal $\io$-category. This concludes the proof of the second assertion.
\end{proof}

\p 
Suppose given a commutative square of marked $\io$-categories: 
% https://q.uiver.app/#q=WzAsNCxbMCwwLCJBIl0sWzAsMSwiQl5cXHNoYXJwIl0sWzEsMCwiQyJdLFsxLDEsIkReXFxzaGFycCJdLFswLDIsImoiXSxbMiwzLCJ1Il0sWzAsMSwidiIsMl0sWzEsMywiaSIsMl1d
\begin{equation}
\label{eq:BC data}
\begin{tikzcd}
	A & C \\
	{B^\sharp} & {D^\sharp}
	\arrow["j", from=1-1, to=1-2]
	\arrow["u", from=1-2, to=2-2]
	\arrow["v"', from=1-1, to=2-1]
	\arrow["i"', from=2-1, to=2-2]
\end{tikzcd}
\end{equation}
This induces a square 
% https://q.uiver.app/#q=WzAsNCxbMSwwLCJcXExDYXJ0YyhBKSJdLFsxLDEsIlxcTENhcnQoQl5cXHNoYXJwKSJdLFswLDAsIlxcTENhcnRjKEMpIl0sWzAsMSwiXFxMQ2FydChEXlxcc2hhcnApIl0sWzIsMCwiXFxSYiBqXioiXSxbMiwzLCJcXExiICB1XyEiLDJdLFswLDEsIlxcTGIgdl8hIl0sWzMsMSwiXFxSYiBpXioiLDJdLFswLDMsIiIsMSx7InNob3J0ZW4iOnsic291cmNlIjozMCwidGFyZ2V0IjozMH0sImxldmVsIjoyfV1d
\begin{equation}
\label{eq:BC lax commutative square}
\begin{tikzcd}
	{\LCartc(C)} & {\LCartc(A)} \\
	{\LCart(D^\sharp)} & {\LCart(B^\sharp)}
	\arrow["{\Rb j^*}", from=1-1, to=1-2]
	\arrow["{\Lb  u_!}"', from=1-1, to=2-1]
	\arrow["{\Lb v_!}", from=1-2, to=2-2]
	\arrow["{\Rb i^*}"', from=2-1, to=2-2]
	\arrow[shorten <=8pt, shorten >=8pt, Rightarrow, from=1-2, to=2-1]
\end{tikzcd}
\end{equation}
that commutes up to a natural transformation 
\begin{equation}
\label{eq:BC nat}
\begin{array}{rcl}
\Lb v_!\circ \Rb j^*&\to &\Lb v_!\circ \Rb j^* \circ \Rb u^* \circ \Lb u_!\\
&\sim & \Lb v_!\circ \Rb v^* \circ \Rb i^* \circ \Lb u_!\\
&\to& \Rb i^* \circ \Lb u_!
\end{array}
\end{equation}
A square \eqref{eq:BC data} verifies the \notion{Beck-Chevaley condition} if this natural transformation \eqref{eq:BC nat} is an equivalence. This square verifies the \notion{weak Beck-Chevaley condition} if the natural transformation once composed with $\bot$ becomes an equivalence.

\begin{prop}
\label{prop:base change}
If the square \eqref{eq:BC data} is cartesian and $i$ is smooth, then it verifies the Beck-Chevaley condition.
\end{prop}
\begin{proof}
By construction, $\Lb v_!\circ \Rb j^*$ sends an object $E$ of $\LCartc(C)$ onto the fibrant replacement of $ v_!j^* E$. 
As $i$ is smooth, $\Rb i^* \circ \Lb u_!$ sends an object $E$ of $\LCart(C)$ onto the fibrant replacement of $i^*u_! E$. As pullbacks are stable under composition, we have $i^*u_!\sim v_!j^*$.
\end{proof}

\begin{lemma}
\label{lemma:smoth technical 1}
A square \eqref{eq:BC data} where both $j$ and $i$ are final verifies the weak Beck-Chevaley condition.
\end{lemma}
\begin{proof}
As  $\bot$ sends initial and final morphisms to equivalences, for any $E: \LCartc(A)$ and any $F:\LCartc(C)$, we have equivalences
$$\bot \Lb v_! E \sim \bot E~~~\mbox{ and }~~~\bot \Lb v_! F \sim \bot F.$$
Moreover, as classified left cartesian fibrations are proper, for any $G:\LCartc(C)$ and $H:\LCart(D^\sharp)$,  we have equivalences
$$\bot \Lb j^*G \sim \bot G~~~\mbox{ and }~~~\bot \Lb i^* H \sim \bot H.$$
This implies  the result.
\end{proof}

\begin{lemma}
\label{lemma:smoth technical 2}
Suppose given a cartesian square 
% https://q.uiver.app/#q=WzAsNCxbMCwwLCJBIl0sWzAsMSwiQl5cXHNoYXJwIl0sWzEsMCwiQyJdLFsxLDEsIkReXFxzaGFycCJdLFswLDIsImoiXSxbMiwzLCJ1Il0sWzAsMSwidiIsMl0sWzEsMywiaSIsMl1d
% https://q.uiver.app/#q=WzAsNCxbMCwwLCJBIl0sWzAsMSwiQl5cXHNoYXJwIl0sWzEsMCwiQyJdLFsxLDEsIkReXFxzaGFycCJdLFswLDIsImoiXSxbMiwzLCJ1Il0sWzAsMSwidiIsMl0sWzEsMywiaSIsMl1d
\[\begin{tikzcd}
	A & C \\
	{B^\sharp} & {D^\sharp}
	\arrow["j", from=1-1, to=1-2]
	\arrow["u", from=1-2, to=2-2]
	\arrow["v"', from=1-1, to=2-1]
	\arrow["i"', from=2-1, to=2-2]
\end{tikzcd}\]
such that for any object $b$ of $B^\sharp$, the outer square of the induced diagram
% https://q.uiver.app/#q=WzAsNixbMSwwLCJBIl0sWzEsMSwiQl5cXHNoYXJwIl0sWzIsMCwiQyJdLFsyLDEsIkReXFxzaGFycCJdLFswLDEsIkJeXFxzaGFycF97L2J9Il0sWzAsMCwiQV97Yi99Il0sWzAsMiwiaiJdLFsyLDMsInUiXSxbMCwxLCJ2IiwyXSxbMSwzLCJpIiwyXSxbNSw0LCJ2JyIsMl0sWzQsMSwiXFxwaV9iIiwyXSxbNSwwLCJcXHBpX2InIl0sWzUsMSwiIiwwLHsic3R5bGUiOnsibmFtZSI6ImNvcm5lciJ9fV0sWzAsMywiIiwwLHsic3R5bGUiOnsibmFtZSI6ImNvcm5lciJ9fV1d
\[\begin{tikzcd}
	{A_{b/}} & A & C \\
	{B^\sharp_{/b}} & {B^\sharp} & {D^\sharp}
	\arrow["j", from=1-2, to=1-3]
	\arrow["u", from=1-3, to=2-3]
	\arrow["v"', from=1-2, to=2-2]
	\arrow["i"', from=2-2, to=2-3]
	\arrow["{v'}"', from=1-1, to=2-1]
	\arrow["{\pi_b}"', from=2-1, to=2-2]
	\arrow["{\pi_b'}", from=1-1, to=1-2]
	\arrow["\lrcorner"{anchor=center, pos=0.125}, draw=none, from=1-1, to=2-2]
	\arrow["\lrcorner"{anchor=center, pos=0.125}, draw=none, from=1-2, to=2-3]
\end{tikzcd}\]
verifies the weak Beck Chevaley condition. Then the right hand square verifies the Beck Chevaley condition.
\end{lemma}
\begin{proof}
Let $E$ be an element of $\LCart(C)$. Using the hypothesis, the fact that $\pi_a$ is a right cartesian fibration, and so smooth,, we have a sequence of equivalences: 
$$\begin{array}{rcll}
\bot \Rb \pi_b^*  \Lb v_! \Rb j^*E&\sim &\bot \Lb v'_! \Rb {\pi'_b}^*  \Rb j^*E&(\ref{prop:base change})\\
&\sim & \bot \Rb \pi_b^*  \Rb i  \Lb u_! E&\mbox{(hypothesis)}
\end{array}$$
Using the equivalence \eqref{eq:explicit derived fiber}, this implies that for any element $b$ of $B$, we have an equivalence 
$$ \Rb b^*  \Lb v_! \Rb j^*E\to \Rb b^*  \Rb i  \Lb u_!E$$
which concludes the proof as equivalences between left cartesian fibrations are detected fiberwise.
\end{proof}






\begin{prop}
\label{prop:BC condition}
Let $i:I\to A^\sharp$ and $j:C^\sharp\to D^\sharp$ be two morphisms. The square 
% q.uiver.app/#q=WzAsNCxbMCwwLCJDXlxcc2hhcnBcXHRpbWVzIEkiXSxbMCwxLCJDXlxcc2hhcnBcXHRpbWVzIEFeXFxzaGFycCJdLFsxLDAsIkReXFxzaGFycFxcdGltZXMgSSJdLFsxLDEsIkReXFxzaGFycFxcdGltZXMgQV5cXHNoYXJwIl0sWzAsMV0sWzEsM10sWzAsMl0sWzIsM11d
\[\begin{tikzcd}
	{C^\sharp\times I} & {D^\sharp\times I} \\
	{C^\sharp\times A^\sharp} & {D^\sharp\times A^\sharp}
	\arrow[from=1-1, to=2-1]
	\arrow[from=2-1, to=2-2]
	\arrow[from=1-1, to=1-2]
	\arrow[from=1-2, to=2-2]
\end{tikzcd}\]
verifies the Beck-Chevaley condition.
\end{prop}
\begin{proof}
According to lemma \ref{lemma:smoth technical 2}, one has to show that for any pair $(a,c)$ where $a$ is an object of $A^\sharp$ and $c$ of $C^\sharp$, the induced cartesian square
% q.uiver.app/#q=WzAsNCxbMSwwLCJEXlxcc2hhcnBcXHRpbWVzIEkiXSxbMSwxLCJEXlxcc2hhcnBcXHRpbWVzIEFeXFxzaGFycCJdLFswLDEsIkNeXFxzaGFycF97Yy99XFx0aW1lcyBBX3thL31eXFxzaGFycCJdLFswLDAsIkNeXFxzaGFycF97Yy99XFx0aW1lcyBJX3thL30iXSxbMCwxXSxbMywyXSxbMywwXSxbMiwxXV0=
\[\begin{tikzcd}
	{C^\sharp_{c/}\times I_{a/}} & {D^\sharp\times I} \\
	{C^\sharp_{c/}\times A_{a/}^\sharp} & {D^\sharp\times A^\sharp}
	\arrow[from=1-2, to=2-2]
	\arrow[from=1-1, to=2-1]
	\arrow[from=1-1, to=1-2]
	\arrow[from=2-1, to=2-2]
\end{tikzcd}\]
verifies the weak Beck-Chevaley condition. Remark that this square factors as two cartesian squares:
% q.uiver.app/#q=WzAsNixbMiwwLCJEXlxcc2hhcnBcXHRpbWVzIEkiXSxbMiwxLCJEXlxcc2hhcnBcXHRpbWVzIEFeXFxzaGFycCJdLFswLDEsIkNeXFxzaGFycF97Yy99XFx0aW1lcyBBX3thL31eXFxzaGFycCJdLFswLDAsIkNeXFxzaGFycF97Yy99XFx0aW1lcyBJX3thL30iXSxbMSwxLCJEXlxcc2hhcnBfe2ooYykvfVxcdGltZXMgQV97YS99Xlxcc2hhcnAiXSxbMSwwLCJEXlxcc2hhcnBfe2ooYykvfVxcdGltZXMgSV97YS99Il0sWzAsMV0sWzMsMl0sWzIsNF0sWzQsMV0sWzMsNV0sWzUsMF0sWzUsNF1d
\[\begin{tikzcd}
	{C^\sharp_{c/}\times I_{a/}} & {D^\sharp_{j(c)/}\times I_{a/}} & {D^\sharp\times I} \\
	{C^\sharp_{c/}\times A_{a/}^\sharp} & {D^\sharp_{j(c)/}\times A_{a/}^\sharp} & {D^\sharp\times A^\sharp}
	\arrow[from=1-3, to=2-3]
	\arrow[from=1-1, to=2-1]
	\arrow[from=2-1, to=2-2]
	\arrow[from=2-2, to=2-3]
	\arrow[from=1-1, to=1-2]
	\arrow[from=1-2, to=1-3]
	\arrow[from=1-2, to=2-2]
\end{tikzcd}\]
The two morphisms $\{c\}\to C^\sharp_{c/}$ and $\{c\}\to D^\sharp_{j(c)/}$ are initial, and by stability by left cancellation, so is $C^\sharp_{c/}\to D^\sharp_{j(c)/}$. By stability by cartesian product, the two horizontal morphisms of the left square are initial. Lemma \ref{lemma:smoth technical 1} then implies that the left square verifies the weak Beck-Chevaley condition. According to proposition \ref{prop:base change}, the right square fulfills the Beck-Chevaley condition, and so \textit{a fortiori}, the weak one. The outer square then verified the weak Beck-Chevaley condition, which concludes the proof.
\end{proof}


\p 
Suppose given a commutative square of marked $\io$-categories:
% https://q.uiver.app/#q=WzAsNCxbMCwwLCJBIl0sWzAsMSwiQiJdLFsxLDAsIkNeXFxzaGFycCJdLFsxLDEsIkReXFxzaGFycCJdLFswLDIsImoiXSxbMiwzLCJ1Il0sWzAsMSwidiIsMl0sWzEsMywiaSIsMl1d
\begin{equation}
\label{eq:BC data}
\begin{tikzcd}
	A & {C^\sharp} \\
	B & {D^\sharp}
	\arrow["j", from=1-1, to=1-2]
	\arrow["u", from=1-2, to=2-2]
	\arrow["v"', from=1-1, to=2-1]
	\arrow["i"', from=2-1, to=2-2]
\end{tikzcd}
\end{equation}
where  $j$ and $i$ are smooth. This induces a square
% https://q.uiver.app/#q=WzAsNCxbMCwxLCJcXExDYXJ0YyhBKSJdLFswLDAsIlxcTENhcnRjKEIpIl0sWzEsMSwiXFxMQ2FydChDXlxcc2hhcnApIl0sWzEsMCwiXFxMQ2FydChEXlxcc2hhcnApIl0sWzAsMiwiXFxSYiBqXyoiLDJdLFszLDIsIlxcTGIgIHVeKiJdLFsxLDAsIlxcTGIgdl4qIiwyXSxbMSwzLCJcXFJiIGlfKiJdLFszLDAsIiIsMSx7InNob3J0ZW4iOnsic291cmNlIjozMCwidGFyZ2V0IjozMH0sImxldmVsIjoyfV1d
\begin{equation}
\label{eq:BC lax commutative square2}
\begin{tikzcd}
	{\LCartc(B)} & {\LCart(D^\sharp)} \\
	{\LCartc(A)} & {\LCart(C^\sharp)}
	\arrow["{\Rb j_*}"', from=2-1, to=2-2]
	\arrow["{\Lb  u^*}", from=1-2, to=2-2]
	\arrow["{\Lb v^*}"', from=1-1, to=2-1]
	\arrow["{\Rb i_*}", from=1-1, to=1-2]
	\arrow[shorten <=8pt, shorten >=8pt, Rightarrow, from=1-2, to=2-1]
\end{tikzcd}
\end{equation}
that commutes up to a natural transformation 
\begin{equation}
\label{eq:BC nat2}
\begin{array}{rcl}
\Lb u^*\circ \Rb i_*&\to & \Rb j_*\circ\Lb j^* \circ \Lb u^*\circ \Rb i_*\\
&\sim &\Rb j_*\circ\Lb v^*\circ\Lb i^*  \circ \Rb i_*\\
&\to &\Rb j_*\circ\Lb v^*
\end{array}
\end{equation}
A square \eqref{eq:BC data} verifies the \notion{opposed Beck-Chevaley condition} if $i$ and $j$ are smooth and  the natural transformation \eqref{eq:BC nat2} is an equivalence.

\begin{prop}
\label{prop:base change2}
If the square \eqref{eq:BC nat2} is cartesian,  and $i$ and $j$ are smooth, then it verifies the opposed Beck-Chevaley condition.
\end{prop}
\begin{proof}
By adjunction, it is sufficient to show that the induced natural transformation
$$\Lb v_!\circ \Rb j^*\to \Rb i^* \circ \Lb u_!:\LCart(C^\sharp)\to \LCart(B)$$
is an equivalence. By construction, $\Lb v_!\circ \Rb j^*$ sends an object $E$ of $\LCart(C^\sharp)$ onto the fibrant replacement of $ v_!j^* E$. 
As $i$ is smooth, $\Rb i^* \circ \Lb u_!$ sends an object $E$ of $\LCart(C^\sharp)$ onto the fibrant replacement of $i^*u_! E$. As pullbacks are stable under composition, we have $i^*u_!\sim v_!j^*$.
\end{proof}

\begin{prop}
\label{prop:BC condition 2}
Let $i:I\to A^\sharp$ be a smooth morphism and $j:C^\sharp\to D^\sharp$ any morphism. The square
% q.uiver.app/#q=WzAsNCxbMCwwLCIgQ15cXHNoYXJwXFx0aW1lcyBJIl0sWzEsMCwiIENeXFxzaGFycFxcdGltZXMgQV5cXHNoYXJwIl0sWzAsMSwiRF5cXHNoYXJwXFx0aW1lcyBJIl0sWzEsMSwiIEReXFxzaGFycFxcdGltZXMgQV5cXHNoYXJwIl0sWzAsMl0sWzEsM10sWzAsMV0sWzIsM11d
\[\begin{tikzcd}
	{ C^\sharp\times I} & { C^\sharp\times A^\sharp} \\
	{D^\sharp\times I} & { D^\sharp\times A^\sharp}
	\arrow[from=1-1, to=2-1]
	\arrow[from=1-2, to=2-2]
	\arrow[from=1-1, to=1-2]
	\arrow[from=2-1, to=2-2]
\end{tikzcd}\]
verifies the opposed Beck-Chevaley condition.
\end{prop}
\begin{proof}
As $id_{C^\sharp}\times i$ and $id_{D^\sharp}\times i$ are pullbacks of $i$, they are smooth. The result  is then follows from proposition \ref{prop:base change2}.
\end{proof}

\subsection{The $\Wcard$-small $\io$-category of $\V$-small left cartesian fibrations}

\p Let $I$ be a marked $\io$-category, and $a$ a globular sum. We recall that the pullback along the canonical projection $\pi_a:I\times a^\flat\to I$ induces an adjunction
% https://q.uiver.app/#q=WzAsMixbMSwwLCJcXG9jYXRtX3svSX06e1xccGlfYX1eKiJdLFswLDAsIntcXHBpX2F9XyE6XFxvY2F0X3svSVxcdGltZXMgYV5cXGZsYXR9Il0sWzEsMCwiIiwyLHsib2Zmc2V0IjotMn1dLFswLDEsIiIsMix7Im9mZnNldCI6LTJ9XSxbMiwzLCIiLDIseyJsZXZlbCI6MSwic3R5bGUiOnsibmFtZSI6ImFkanVuY3Rpb24ifX1dXQ==
\[\begin{tikzcd}
	{{\pi_a}_!:\ocat_{/I\times a^\flat}} & {\ocatm_{/I}:{\pi_a}^*}
	\arrow[""{name=0, anchor=center, inner sep=0}, shift left=2, from=1-1, to=1-2]
	\arrow[""{name=1, anchor=center, inner sep=0}, shift left=2, from=1-2, to=1-1]
	\arrow["\dashv"{anchor=center, rotate=-90}, draw=none, from=0, to=1]
\end{tikzcd}\]
\begin{lemma}
\label{lemma:to show fully faithfullness1}
Let $E$ and $F$ be two objects of $\ocatm_{/I}$ and $\psi:\pi_{[a,1]}^*E\to \pi_{[a,1]}^*F$ an equivalence.
The exists a unique commutative diagram of shape
% https://q.uiver.app/#q=WzAsNCxbMSwwLCIoXFxwaV97W2EsMV19KV8hXFxwaV97W2EsMV19XipGIl0sWzAsMCwiKFxccGlfe1thLDFdfSlfIVxccGlfe1thLDFdfV4qRSJdLFsxLDEsIkYiXSxbMCwxLCJFIl0sWzEsMCwiKFxccGlfe1thLDFdfSlfIVxccHNpIl0sWzAsMiwiXFxlcHNpbG9uIl0sWzEsMywiXFxlcHNpbG9uIiwyXSxbMywyLCJcXHBoaSIsMix7InN0eWxlIjp7ImJvZHkiOnsibmFtZSI6ImRhc2hlZCJ9fX1dXQ==
\[\begin{tikzcd}
	{(\pi_{[a,1]})_!\pi_{[a,1]}^*E} & {(\pi_{[a,1]})_!\pi_{[a,1]}^*F} \\
	E & F
	\arrow["{(\pi_{[a,1]})_!\psi}", from=1-1, to=1-2]
	\arrow["\epsilon", from=1-2, to=2-2]
	\arrow["\epsilon"', from=1-1, to=2-1]
	\arrow["\phi"', dashed, from=2-1, to=2-2]
\end{tikzcd}\]
Moreover, the arrow $\phi$ is an equivalence.
\end{lemma}
\begin{proof}
Unfolding the definition, we have to show the existence and unicity of commutative diagrams of shape
% https://q.uiver.app/#q=WzAsNCxbMSwwLCJZXFx0aW1lcyBbYSwxXV5cXGZsYXQiXSxbMCwwLCJYXFx0aW1lcyBbYSwxXV5cXGZsYXQiXSxbMSwxLCJZIl0sWzAsMSwiWCJdLFsxLDAsIlxcZG9tIFxccHNpIl0sWzAsMl0sWzEsM10sWzMsMiwiXFxwaGkiLDIseyJzdHlsZSI6eyJib2R5Ijp7Im5hbWUiOiJkYXNoZWQifX19XV0=
\begin{equation}
\label{eq:square to show fully faithfullness}
\begin{tikzcd}
	{X\times [a,1]^\flat} & {Y\times [a,1]^\flat} \\
	X & Y
	\arrow["{\dom \psi}", from=1-1, to=1-2]
	\arrow[from=1-2, to=2-2]
	\arrow[from=1-1, to=2-1]
	\arrow["\phi"', dashed, from=2-1, to=2-2]
\end{tikzcd}
\end{equation}
where the two vertical morphisms are the projection and 
where $X$ and $Y$ correspond respectively to the domain of $E$ and $F$. As $\dom\psi$ is a morphism over $I\times [a,1]^\flat$, we already have a commutative diagram of shape:
% https://q.uiver.app/#q=WzAsNCxbMSwwLCJZXFx0aW1lcyBbYSwxXV5cXGZsYXQiXSxbMCwwLCJYXFx0aW1lcyBbYSwxXV5cXGZsYXQiXSxbMSwxLCJbYSwxXV5cXGZsYXQiXSxbMCwxLCJbYSwxXV5cXGZsYXQiXSxbMSwwLCJcXGRvbSBcXHBzaSJdLFswLDJdLFsxLDNdLFszLDIsImlkIiwyXV0=
\[\begin{tikzcd}
	{X\times [a,1]^\flat} & {Y\times [a,1]^\flat} \\
	{[a,1]^\flat} & {[a,1]^\flat}
	\arrow["{\dom \psi}", from=1-1, to=1-2]
	\arrow[from=1-2, to=2-2]
	\arrow[from=1-1, to=2-1]
	\arrow["id"', from=2-1, to=2-2]
\end{tikzcd}\]
By the universal property of cartesian product, this directly implies that if a square of shape \eqref{eq:square to show fully faithfullness} exists, it has to be unique, and that the morphism $\phi$ will be an equivalence. It then remains to show the existence.




Let $\psi'$ be an inverse of $\psi$. We denote $\tilde{\psi}:X\times [a,1]^\flat\to Y$ and $\tilde{\psi}':Y\times [a,1]^\flat\to X$ the morphisms induce by the adjunction from $\psi$ and $\psi'$. For $\epsilon\in\{0,1\}$, we denote by $\psi_{\epsilon}:X\times \{\epsilon\} \to Y$ and $\psi'_{\epsilon}:Y\times \{\epsilon\} \to X$ the induced morphisms. In particular $\psi_{\epsilon}$ and $\psi'_{\epsilon}$ are inverse one of the other.

 By construction, we have a commutative diagram
% https://q.uiver.app/#q=WzAsNCxbMCwwLCJYXFx0aW1lcyBbYSwxXV5cXGZsYXRcXHRpbWVzIFthLDFdXlxcZmxhdCJdLFsxLDAsIllcXHRpbWVzW2EsMV1eXFxmbGF0Il0sWzEsMSwiWCJdLFswLDEsIlhcXHRpbWVzW2EsMV1eXFxmbGF0Il0sWzAsMSwiXFx0aWxkZXtcXHBzaX1cXHRpbWVzW2EsMV1eXFxmbGF0Il0sWzEsMiwiXFx0aWxkZXtcXHBzaX0nIl0sWzMsMCwiWFxcdGltZXMgXFx0cmlhbmdsZWRvd24iXSxbMywyLCJcXHBpIiwyXV0=
\[\begin{tikzcd}
	{X\times [a,1]^\flat\times [a,1]^\flat} & {Y\times[a,1]^\flat} \\
	{X\times[a,1]^\flat} & X
	\arrow["{\tilde{\psi}\times[a,1]^\flat}", from=1-1, to=1-2]
	\arrow["{\tilde{\psi}'}", from=1-2, to=2-2]
	\arrow["{X\times \triangledown}", from=2-1, to=1-1]
	\arrow["\pi"', from=2-1, to=2-2]
\end{tikzcd}\]
where $\triangledown$ is the diagonal and $\psi$ the canonical projection. This corresponds to a commutative diagram in the $\iun$-category $[n]\mapsto \Hom(X\times [a,n]^\flat,X)$:
% https://q.uiver.app/#q=WzAsNCxbMCwwLCJpZF9YIl0sWzEsMCwiaWRfWCJdLFsxLDEsImlkX1giXSxbMCwxLCJpZF9YIl0sWzAsMywiXFx0aWxkZVxccHNpJypcXHBzaV8wIiwyXSxbMSwyLCJcXHRpbGRlXFxwc2knKlxccHNpXzEiXSxbMywyLCJcXHBzaV8xJypcXHRpbGRlXFxwc2kiLDJdLFswLDEsIlxccHNpJ18wKlxcdGlsZGVcXHBzaSJdLFswLDIsImlkX3tpZF9YfSIsMV1d
\[\begin{tikzcd}
	{id_X} & {id_X} \\
	{id_X} & {id_X}
	\arrow["{\tilde\psi'*\psi_0}"', from=1-1, to=2-1]
	\arrow["{\tilde\psi'*\psi_1}", from=1-2, to=2-2]
	\arrow["{\psi_1'*\tilde\psi}"', from=2-1, to=2-2]
	\arrow["{\psi'_0*\tilde\psi}", from=1-1, to=1-2]
	\arrow["{id_{id_X}}"{description}, from=1-1, to=2-2]
\end{tikzcd}\]
Remark that in the $\iun$-category $[n]\mapsto \Hom(X\times [a,n]^\flat,Y)$, we have equivalences
$$\tilde\psi\sim\psi_0' *\psi_0*\psi~~~\mbox{ and }\tilde\psi\sim\psi_1' *\psi_1*\psi$$
and the previous diagram then induces two commutative triangles
% https://q.uiver.app/#q=WzAsNixbMywwLCJcXHBzaV8wIl0sWzQsMCwiXFxwc2lfMSJdLFs0LDEsIlxccHNpXzAiXSxbMCwxLCJcXHBzaV8wIl0sWzEsMSwiXFxwc2lfMSJdLFswLDAsIlxccHNpXzEiXSxbMCwyLCJpZF97XFxwc2lfMH0iLDJdLFswLDEsIlxcdGlsZGVcXHBzaSJdLFsxLDIsIlxccHNpXzAqXFx0aWxkZVxccHNpJypcXHBzaV8xIl0sWzMsNCwiXFx0aWxkZVxccHNpIiwyXSxbNSwzLCJcXHBzaV8xKlxcdGlsZGVcXHBzaScqXFxwc2lfMCIsMl0sWzUsNCwiaWRfe1xccHNpXzF9Il1d
\[\begin{tikzcd}
	{\psi_1} &&& {\psi_0} & {\psi_1} \\
	{\psi_0} & {\psi_1} &&& {\psi_0}
	\arrow["{id_{\psi_0}}"', from=1-4, to=2-5]
	\arrow["\tilde\psi", from=1-4, to=1-5]
	\arrow["{\psi_0*\tilde\psi'*\psi_1}", from=1-5, to=2-5]
	\arrow["\tilde\psi"', from=2-1, to=2-2]
	\arrow["{\psi_1*\tilde\psi'*\psi_0}"', from=1-1, to=2-1]
	\arrow["{id_{\psi_1}}", from=1-1, to=2-2]
\end{tikzcd}\]
View as a $1$-cell of $[n]\mapsto \Hom(X\times [a,n]^\flat,Y)$, $\tilde{\psi}$ is then an equivalence. This implies the existence of a lifts in the following diagram
% https://q.uiver.app/#q=WzAsMyxbMCwwLCJbYSwxXV5cXGZsYXQiXSxbMSwwLCJcXHVIb20oWCxZKSJdLFswLDEsIjEiXSxbMCwxLCJcXHRpbGRlXFxwc2kiXSxbMCwyXSxbMiwxLCJcXHBoaSIsMix7InN0eWxlIjp7ImJvZHkiOnsibmFtZSI6ImRhc2hlZCJ9fX1dXQ==
\[\begin{tikzcd}
	{[a,1]^\flat} & {\uHom(X,Y)} \\
	1
	\arrow["\tilde\psi", from=1-1, to=1-2]
	\arrow[from=1-1, to=2-1]
	\arrow["\phi"', dashed, from=2-1, to=1-2]
\end{tikzcd}\]
which induces the wanted square:
% https://q.uiver.app/#q=WzAsNCxbMCwwLCJYXFx0aW1lc1thLDFdXlxcZmxhdCJdLFsxLDAsIllcXHRpbWVzW2EsMV1eXFxmbGF0Il0sWzAsMSwiWCJdLFsxLDEsIlgiXSxbMCwyXSxbMiwzLCJcXHBoaSIsMix7InN0eWxlIjp7ImJvZHkiOnsibmFtZSI6ImRhc2hlZCJ9fX1dLFsxLDNdLFswLDMsIlxcdGlsZGV7XFxwc2l9Il0sWzAsMV1d
\[\begin{tikzcd}
	{X\times[a,1]^\flat} & {Y\times[a,1]^\flat} \\
	X & X
	\arrow[from=1-1, to=2-1]
	\arrow["\phi"', dashed, from=2-1, to=2-2]
	\arrow[from=1-2, to=2-2]
	\arrow["{\tilde{\psi}}", from=1-1, to=2-2]
	\arrow[from=1-1, to=1-2]
\end{tikzcd}\]
\end{proof}


\begin{lemma}
\label{lemma:to show fully faithfullness2}
Let $I$ be a marked $\io$-category and $a$ a globular form. 
The canonical morphisms of $\infty$-groupoids:
$$\pi_{[a,1]}^*:\tau_0\ocatm_{/I}\to \tau_0\ocatm_{/I\times [a,1]^\flat}$$
$$\pi_{[a,1]}^*:\tau_0 \Arr(\ocatm_{/I})\to \tau_0\Arr(\ocatm_{/I\times [a,1]^\flat})$$
are fully faithful.
\end{lemma}
\begin{proof}
Let $E$ and $F$ be two objects of $\ocatm_{/I}$. The morphism 
$$\Hom_{\tau_0\ocatm_{/I}}(E,F) \to \Hom_{\tau_0\ocatm_{/I\times[a,1]^\flat}}(\pi_{[a,1]}^*E,\pi_{[a,1]}^*F) $$ has an inverse that sends $\psi:\pi_{[a,1]}^*E\to \pi_{[a,1]}^*F$ onto the morphism $\phi:E\to F$ appearing in the commutative square provided by lemma \ref{lemma:to show fully faithfullness1}.

The second assertion is demonstrated similarly.
\end{proof}

\begin{prop}
\label{prp:to show fully faithfullness3}
Let $I$ be a marked $\io$-category and $a$ a globular form. We denote by $\pi_a:I\times a^\flat\to I$ the canonical projection.
The canonical morphisms of $\infty$-groupoids:
$$\Rb{\pi_a}^*:\tau_0\LCartc(I)\to \tau_0\LCartc(I\times a^\flat)$$
$$\Rb{\pi_a}^*:\tau_0 \Arr(\LCartc(I))\to \tau_0\Arr(\LCartc(I\times a^\flat))$$
are fully faithful.
\end{prop}
\begin{proof}
Let $[\textbf{b},n]:= a$. Considere first the adjunction:
% https://q.uiver.app/#q=WzAsMixbMCwxLCJcXExDYXJ0YyhJXlxcZmxhdFxcdGltZXMgW1xcdGV4dGJme2J9LG5dKSJdLFswLDAsIlxcTENhcnRjKElcXHRpbWVzIFtiXzAsMV1eXFxmbGF0KVxcdGltZXNfe1xcTENhcnRjKEkpfS4uLlxcdGltZXNfe1xcTENhcnRjKEkpfVxcTENhcnRjKElcXHRpbWVzIFtiX3tuLTF9LDFdXlxcZmxhdCkiXSxbMCwxLCIiLDAseyJvZmZzZXQiOi0yfV0sWzEsMCwiXFxjb2xpbV9JIiwwLHsib2Zmc2V0IjotMn1dLFszLDIsIiIsMCx7ImxldmVsIjoxLCJzdHlsZSI6eyJuYW1lIjoiYWRqdW5jdGlvbiJ9fV1d
\[\begin{tikzcd}
	{\LCartc(I\times [b_0,1]^\flat)\times_{\LCartc(I)}...\times_{\LCartc(I)}\LCartc(I\times [b_{n-1},1]^\flat)} \\
	{\LCartc(I^\flat\times [\textbf{b},n])}
	\arrow[""{name=0, anchor=center, inner sep=0}, shift left=2, from=2-1, to=1-1]
	\arrow[""{name=1, anchor=center, inner sep=0}, "{\colim_I}", shift left=2, from=1-1, to=2-1]
	\arrow["\dashv"{anchor=center, rotate=-180}, draw=none, from=1, to=0]
\end{tikzcd}\]
The corollary \ref{cor:fib over a colimit} implies that the counit of this adjunction is an equivalence.
This implies that the right adjoint
$$\LCartc(I^\flat\times [\textbf{b},n])\to \LCartc(I\times [b_0,1]^\flat)\times_{\LCartc(I)}...\times_{\LCartc(I)}\LCartc(I\times [b_{n-1},1]^\flat)$$ is fully faithful. 
By right cancellation and using the fact that fully faithful functors are stable by limits, it is sufficient to show that for any $k<n$, 
$$\Rb{\pi_{[b_i,1]}}^*:\tau_0\LCartc(I)\to \tau_0\LCartc(I\times [b_k,1]^\flat)$$
is fully faithful. 
Moreover, for any such $k$, we have a commutative square
% https://q.uiver.app/#q=WzAsNCxbMSwwLCJcXHRhdV8wXFxMQ2FydGMoSVxcdGltZXMgW2JfaywxXV5cXGZsYXQpIl0sWzAsMCwiXFx0YXVfMFxcTENhcnRjKEkpIl0sWzAsMSwiXFx0YXVfMFxcb2NhdG1fey9JfSJdLFsxLDEsIlxcdGF1XzBcXG9jYXRtX3svSVxcdGltZXMgW2JfaywxXV5cXGZsYXR9Il0sWzEsMCwiXFxSYntcXHBpX3tbYl9rLDFdfX1eKiJdLFsxLDJdLFswLDNdLFsyLDMsIntcXHBpX3tbYl9rLDFdfX1eKiIsMl1d
\[\begin{tikzcd}
	{\tau_0\LCartc(I)} & {\tau_0\LCartc(I\times [b_k,1]^\flat)} \\
	{\tau_0\ocatm_{/I}} & {\tau_0\ocatm_{/I\times [b_k,1]^\flat}}
	\arrow["{\Rb{\pi_{[b_k,1]}}^*}", from=1-1, to=1-2]
	\arrow[from=1-1, to=2-1]
	\arrow[from=1-2, to=2-2]
	\arrow["{{\pi_{[b_k,1]}}^*}"', from=2-1, to=2-2]
\end{tikzcd}\]
whose vertical morphisms are fully faithful by construction. The results the follows from lemma \ref{lemma:to show fully faithfullness2} by right cancellation.

The second assertion is demonstrated similarly.

\end{proof}




\p For an $\io$-category $A$ and a globular sum $a$, we define $\LCart(A^\sharp;a)$ as the full sub $\iun$-category of $\LCartc(A^\sharp\times a^\flat)$ whose objects are of shape $E\times id_a^\flat$ for $E$ an object of $\LCart(A^\sharp)$. The proposition \ref{prp:to show fully faithfullness3} implies that the canonical morphism 
$$\tau_0\LCart(A^\sharp)\to \tau_0\LCart(A^\sharp;a)$$
is an equivalence of $\infty$-groupoid. We define \wcnotation{$\uLCart(A^\sharp)$}{(lcart@$\uLCart(\uvar)$} as the $\Wcard$-small $\io$-category whose value on $[a,n]$ is given by:
$$\uLCart(A^\sharp)([a,n]):=\Hom([n],\LCart(A^\sharp;a)).$$
For a marked $\io$-category $I$ and a globular sum $a$, we define similarly $\LCartc(I;a)$ as the full sub $\iun$-category of $\LCartc(I\times a^\flat)$ whose objects are of shape $E\times id_a^\flat$ for $E$ an object of $\LCartc(I)$. The proposition \ref{prp:to show fully faithfullness3} implies that the canonical morphism 
$$\tau_0\LCartc(I)\to \tau_0\LCartc(I;a)$$
is an equivalence of $\infty$-groupoid. We define \wcnotation{$\uLCartc(I)$}{(lcartc@$\uLCartc(\uvar)$} as the $\Wcard$-small $\io$-category whose value on $[a,n]$ is given by:
$$\uLCartc(I)([a,n]):=\Hom([n],\LCartc(I;a)).$$
These two definitions are compatible as we have an equivalence between $\uLCartc(A^\sharp)$ and $\uLCart(A^\sharp)$.

\p Let $E$ and $F$ be two objects of $\uLCartc(I)$, and $a$ a globular sum. Remark that a morphism $[a,1]\to \uLCartc(I)$ corresponds to a morphism $E\times id_a\to F\times id_a$, and so to a morphism $X\times a\to Y$ over $I$ where $X$ and $Y$ are respectively the domain of $E$ and $F$. We then have an equivalence: 
\begin{equation}
\hom_{\uLCart(I)}(E,F)\sim \Map_I(E,F). 
\end{equation}
This then implies that $\LCartc(I)$ is locally $\V$-small.



\p Let $i:I\to J$ be a morphism between marked $\io$-category, $a$ a globular sum, and $p$ a classified left cartesian fibration over 
$a^\flat\times J$. Remark that we have a canonical equivalence $$\Rb (i\times id_{a^\flat})^*(p\times id_{a^\flat})\sim (\Rb i^*p)\times id_{a^\flat}$$ natural in $a:\Theta^{op}$. The functor $\Rb (i\times id_{a^\flat})^*$ then restricts to a functor 
$$(i_a)^*:\LCartc(J;a)\to \LCartc(I;a)$$
natural in $a:\Theta^{op}$, and then to a morphism of $\io$-categories:
\begin{equation}
\label{eq:i pullback}
i^*:\uLCartc(J)\to \uLCartc(I)
\end{equation}
\index[notation]{(f5@$f^*:	\uLCartc(J)\to \uLCartc(I)$}

\p 
Let $i:I\to A^\sharp$ be a morphism between marked $\io$-categories. We are now willing to construct a morphism $i_!:\uLCartc(I)\to \uLCart(A^\sharp)$ which corresponds to $\Lb i_!:\LCartc(I)\to \LCart(A^\sharp)$ on the maximal sub $\iun$-category. 


We denote by $E_0$ and $E_1$ the $\iun$-categories fitting in the cartesian square: 
% https://q.uiver.app/#q=WzAsOCxbMSwxLCJcXG9jYXRtIl0sWzAsMCwiRV8wIl0sWzAsMSwiXFxBcnJee2ZpYn0oXFxvY2F0bSkiXSxbMSwwLCJcXFRoZXRhIl0sWzIsMCwiRV8xIl0sWzIsMSwiXFxBcnJee2ZpYn0oXFxvY2F0bSkiXSxbMywxLCJcXG9jYXRtIl0sWzMsMCwiXFxUaGV0YSJdLFsyLDAsIlxcY29kb20iLDJdLFszLDAsIlxccHNpXzAiXSxbMSwyXSxbMSwzXSxbMSwwLCIiLDEseyJzdHlsZSI6eyJuYW1lIjoiY29ybmVyIn19XSxbNyw2LCJcXHBzaV8xIl0sWzQsNV0sWzUsNiwiXFxjb2RvbSIsMl0sWzQsN10sWzQsNiwiIiwxLHsic3R5bGUiOnsibmFtZSI6ImNvcm5lciJ9fV1d
\[\begin{tikzcd}
	{E_0} & \Theta & {E_1} & \Theta \\
	{\Arr^{fib}(\ocatm)} & \ocatm & {\Arr^{fib}(\ocatm)} & \ocatm
	\arrow["\codom"', from=2-1, to=2-2]
	\arrow["{\psi_0}", from=1-2, to=2-2]
	\arrow[from=1-1, to=2-1]
	\arrow[from=1-1, to=1-2]
	\arrow["\lrcorner"{anchor=center, pos=0.125}, draw=none, from=1-1, to=2-2]
	\arrow["{\psi_1}", from=1-4, to=2-4]
	\arrow[from=1-3, to=2-3]
	\arrow["\codom"', from=2-3, to=2-4]
	\arrow[from=1-3, to=1-4]
	\arrow["\lrcorner"{anchor=center, pos=0.125}, draw=none, from=1-3, to=2-4]
\end{tikzcd}\]
where $\Arr^{fib}(\ocatm)$ is the full sub $\iun$-category of $\Arr(\ocatm)$ whose objects are classified left cartesian fibrations, and where $\psi_0$ and $\psi_1$ send respectively $a$ on $I\times a^\flat$ and $A^\sharp\times a^\flat$. 
The morphism $i$ induces an adjunction
% q.uiver.app/#q=WzAsMixbMCwwLCJpXyE6RV8wIl0sWzEsMCwiRV8xOmleKiJdLFswLDEsIiIsMix7Im9mZnNldCI6LTJ9XSxbMSwwLCIiLDIseyJvZmZzZXQiOi0yfV0sWzIsMywiIiwyLHsibGV2ZWwiOjEsInN0eWxlIjp7Im5hbWUiOiJhZGp1bmN0aW9uIn19XV0=
\begin{equation}
\label{eq:adj i pull}
\begin{tikzcd}
	{i_!:E_0} & {E_1:i^*}
	\arrow[""{name=0, anchor=center, inner sep=0}, shift left=2, from=1-1, to=1-2]
	\arrow[""{name=1, anchor=center, inner sep=0}, shift left=2, from=1-2, to=1-1]
	\arrow["\dashv"{anchor=center, rotate=-90}, draw=none, from=0, to=1]
\end{tikzcd}
\end{equation}
where the left adjoint sends a left cartesian fibration $p$ over $I\times a^\flat$ to $\Lb (i\times id_a)_!p$ and the right adjoint sends a left cartesian fibration $q$ over $A^\sharp\times a^\flat$ to $\Rb (i\times id_a)^* q$.
\begin{lemma}
\label{lemma:technical lemma i pull}
Let $p$ be a left cartesian fibration over $I^\sharp$. We have an equivalence $$\Lb (i\times id_{a^\flat})_!(p\times id_{a^\flat})\sim (\Lb i_! p)\times id_{a^\flat}.$$
Let $q$ be a left cartesian fibration over $A^\sharp$. We have an equivalence $$\Rb (i\times id_{a^\flat})^*(q\times id_{a^\flat})\sim (\Rb i^* q)\times id_{a^\flat}.$$
\end{lemma}
\begin{proof}
The first assertion is straightforward as the cartesian product with $a^\flat$ preserves initial morphisms and left cartesian fibrations. The second assertion is obvious.
\end{proof}
We define $\tilde{E_0}$ and $\tilde{E_1}$ as the full sub $\iun$-categories of $E_0$ and $E_1$ whose objects are respectively of shape $p\times id_a$ and $q\times id_a$ for $p$ and $q$ classified left cartesian fibrations over $I$ and $A^\sharp$.
The last lemma implies that \eqref{eq:adj i pull} restricts to an adjunction
% q.uiver.app/#q=WzAsMixbMCwwLCJpXyE6XFx0aWxkZXtFXzB9Il0sWzEsMCwiXFx0aWxkZXtFXzF9OmleKiJdLFswLDEsIiIsMix7Im9mZnNldCI6LTJ9XSxbMSwwLCIiLDIseyJvZmZzZXQiOi0yfV0sWzIsMywiIiwyLHsibGV2ZWwiOjEsInN0eWxlIjp7Im5hbWUiOiJhZGp1bmN0aW9uIn19XV0=
\begin{equation}
\label{eq:adj i pull2}
\begin{tikzcd}
	{i_!:\tilde{E_0}} & {\tilde{E_1}:i^*}
	\arrow[""{name=0, anchor=center, inner sep=0}, shift left=2, from=1-1, to=1-2]
	\arrow[""{name=1, anchor=center, inner sep=0}, shift left=2, from=1-2, to=1-1]
	\arrow["\dashv"{anchor=center, rotate=-90}, draw=none, from=0, to=1]
\end{tikzcd}
\end{equation}

\begin{lemma}
\label{lemma:technical lemma i pull2} $~$
\begin{enumerate}
\item
Let $q\to q'$ be a morphism in $\tilde{E_0}$ corresponding to a cartesian square. The induced morphism $i_!(q)\to i_!(q')$ also corresponds to a cartesian square. 
\item
Let $q\to q'$ be a morphism in $\tilde{E_1}$ corresponding to a cartesian square. The induced morphism $i^*(q)\to i^*(q')$ also corresponds to a cartesian square. 
\end{enumerate}
\end{lemma}
\begin{proof}
Cartesian morphisms in $\tilde{E_0}$ corresponds to cartesian squares
% https://q.uiver.app/#q=WzAsNCxbMSwwLCJYXFx0aW1lcyBiXntcXGZsYXR9Il0sWzEsMSwiSVxcdGltZXMgYl57XFxmbGF0fSJdLFswLDAsIlhcXHRpbWVzIGFee1xcZmxhdH0iXSxbMCwxLCJJXFx0aW1lcyBhXntcXGZsYXR9Il0sWzIsMywicFxcdGltZXMgaWRfYSIsMl0sWzMsMV0sWzAsMSwicFxcdGltZXMgaWRfYiJdLFsyLDBdXQ==
\[\begin{tikzcd}
	{X\times a^{\flat}} & {X\times b^{\flat}} \\
	{I\times a^{\flat}} & {I\times b^{\flat}}
	\arrow["{p\times id_a}"', from=1-1, to=2-1]
	\arrow[from=2-1, to=2-2]
	\arrow["{p\times id_b}", from=1-2, to=2-2]
	\arrow[from=1-1, to=1-2]
\end{tikzcd}\]
and cartesian morphisms in $\tilde{E_1}$ corresponds to cartesian squares
% https://q.uiver.app/#q=WzAsNCxbMSwwLCJZXFx0aW1lcyBiXntcXGZsYXR9Il0sWzEsMSwiQV5cXHNoYXJwXFx0aW1lcyBiXntcXGZsYXR9Il0sWzAsMCwiWVxcdGltZXMgYV57XFxmbGF0fSJdLFswLDEsIkFeXFxzaGFycFxcdGltZXMgYV57XFxmbGF0fSJdLFsyLDMsInFcXHRpbWVzIGlkX2EiLDJdLFszLDFdLFswLDEsInFcXHRpbWVzIGlkX2IiXSxbMiwwXV0=
\[\begin{tikzcd}
	{Y\times a^{\flat}} & {Y\times b^{\flat}} \\
	{A^\sharp\times a^{\flat}} & {A^\sharp\times b^{\flat}}
	\arrow["{q\times id_a}"', from=1-1, to=2-1]
	\arrow[from=2-1, to=2-2]
	\arrow["{q\times id_b}", from=1-2, to=2-2]
	\arrow[from=1-1, to=1-2]
\end{tikzcd}\]
The results directly follows from lemma \ref{lemma:technical lemma i pull}.
\end{proof}
The canonical projection $\tilde{E_0}\to \Theta$ and $\tilde{E_1}\to \Theta$ are Grothendieck fibrations in $\iun$-categories. The cartesian lifting is given by cartesian squares. Moreover, their Grothendieck deconstructions correspond respectively to 
$a\mapsto \LCartc(I;a)$ and $a\mapsto \LCart(A^\sharp;b)$. As both $i_!$ and $i^*$ preserve cartesian lifting according to lemma \ref{lemma:technical lemma i pull2}, they induce by Grothendieck deconstruction a family of adjunction
% q.uiver.app/#q=WzAsMixbMCwwLCIoaV9hKV8hOlxcTENhcnRjKEk7YSkiXSxbMSwwLCJcXExDYXJ0KEFeXFxzaGFycDthKTooaV9hKV4qIl0sWzAsMSwiIiwyLHsib2Zmc2V0IjotMn1dLFsxLDAsIiIsMix7Im9mZnNldCI6LTJ9XSxbMiwzLCIiLDIseyJsZXZlbCI6MSwic3R5bGUiOnsibmFtZSI6ImFkanVuY3Rpb24ifX1dXQ==
\begin{equation}
\label{eq:adj i pull3}
\begin{tikzcd}
	{(i_a)_!:\LCartc(I;a)} & {\LCart(A^\sharp;a):(i_a)^*}
	\arrow[""{name=0, anchor=center, inner sep=0}, shift left=2, from=1-1, to=1-2]
	\arrow[""{name=1, anchor=center, inner sep=0}, shift left=2, from=1-2, to=1-1]
	\arrow["\dashv"{anchor=center, rotate=-90}, draw=none, from=0, to=1]
\end{tikzcd}
\end{equation}
natural in $a:\Theta^{op}$. The family of functors $(i_a)_!$ then induces a morphism of $\io$-category\index[notation]{(f4@$f_{\mbox{$\exclam$}}:\uLCartc(I)\to \uLCart(A^\sharp)$}
\begin{equation}
\label{eq:i pull}
i_!:\uLCartc(I)\to \uLCart(A^\sharp)
\end{equation}
which corresponds to $\Lb i_!:\LCartc(I)\to \LCart(A^\sharp)$ on the maximal sub $\iun$-category.
The unit and counit of adjunction \eqref{eq:adj i pull3} induce morphisms
\begin{equation}
\label{eq:i pull unit an counit}
\mu:id\to i^*i_!~~~~ \epsilon:i_!i^*\to id
\end{equation}
and equivalences
$(\epsilon\circ_0 i_!)\circ_1(i_!\circ_0 \mu) \sim id_{i_!}$ and $(i^*\circ_0 \epsilon)\circ_1 (\mu \circ_0 i^* )\sim id_{i^*}$.




\p Let $j:C^\sharp\to D^\sharp$ be a morphism between $\io$-categories. We claim that the commutative square 
% q.uiver.app/#q=WzAsNCxbMSwxLCJcXHVMQ2FydGMoQ15cXHNoYXJwXFx0aW1lcyBJKSJdLFsxLDAsIlxcdUxDYXJ0YyhEXlxcc2hhcnBcXHRpbWVzIEkpIl0sWzAsMCwiXFx1TENhcnQoRF5cXHNoYXJwXFx0aW1lcyBBXlxcc2hhcnApIl0sWzAsMSwiXFx1TENhcnQoQ15cXHNoYXJwXFx0aW1lcyBBXlxcc2hhcnApIl0sWzEsMCwiKGpcXHRpbWVzIGlkX3tJfSleKiJdLFsyLDEsIiggaWRfe0ReXFxzaGFycH1cXHRpbWVzIGkpXioiXSxbMywwLCIoIGlkX3tDXlxcc2hhcnB9XFx0aW1lcyBpKV4qIiwyXSxbMiwzLCIoalxcdGltZXMgaWRfe0FeXFxzaGFycH0pXioiLDJdXQ==
\[\begin{tikzcd}
	{\uLCart(D^\sharp\times A^\sharp)} & {\uLCartc(D^\sharp\times I)} \\
	{\uLCart(C^\sharp\times A^\sharp)} & {\uLCartc(C^\sharp\times I)}
	\arrow["{(j\times id_{I})^*}", from=1-2, to=2-2]
	\arrow["{( id_{D^\sharp}\times i)^*}", from=1-1, to=1-2]
	\arrow["{( id_{C^\sharp}\times i)^*}"', from=2-1, to=2-2]
	\arrow["{(j\times id_{A^\sharp})^*}"', from=1-1, to=2-1]
\end{tikzcd}\]
induces a commutative square
% https://q.uiver.app/#q=WzAsNCxbMCwwLCJcXHVMQ2FydGMoRF5cXHNoYXJwXFx0aW1lcyBJKSJdLFsxLDAsIlxcdUxDYXJ0YyhDXlxcc2hhcnBcXHRpbWVzIEkpIl0sWzEsMSwiXFx1TENhcnQoQ15cXHNoYXJwXFx0aW1lcyBBXlxcc2hhcnApIl0sWzAsMSwiXFx1TENhcnQoRF5cXHNoYXJwXFx0aW1lcyBBXlxcc2hhcnApIl0sWzAsMywiKCBpZF97RF5cXHNoYXJwfVxcdGltZXMgaSlfISIsMl0sWzAsMSwiKGpcXHRpbWVzIGlkX3tJfSleKiJdLFsxLDIsIiggaWRfe0NeXFxzaGFycH1cXHRpbWVzIGkpXyEiXSxbMywyLCIoalxcdGltZXMgaWRfe0FeXFxzaGFycH0pXioiLDJdXQ==
\begin{equation}
\label{eq:commutative pull push}
\begin{tikzcd}
	{\uLCartc(D^\sharp\times I)} & {\uLCartc(C^\sharp\times I)} \\
	{\uLCart(D^\sharp\times A^\sharp)} & {\uLCart(C^\sharp\times A^\sharp)}
	\arrow["{( id_{D^\sharp}\times i)_!}"', from=1-1, to=2-1]
	\arrow["{(j\times id_{I})^*}", from=1-1, to=1-2]
	\arrow["{( id_{C^\sharp}\times i)_!}", from=1-2, to=2-2]
	\arrow["{(j\times id_{A^\sharp})^*}"', from=2-1, to=2-2]
\end{tikzcd}
\end{equation}
\textit{A priori}, the natural transformations \eqref{eq:i pull unit an counit} implies that this square commutes up the natural transformation:
$$
\begin{array}{rcl}
( id_{C^\sharp}\times i)_!\circ (j\times id_{I})^*&\to &( id_{C^\sharp}\times i)_! \circ (j\times id_{I})^* \circ ( id_{D^\sharp}\times i)^*\circ ( id_{D^\sharp}\times i)_!\\
&\sim &( id_{C^\sharp}\times i)_!\circ ( id_{C^\sharp}\times i)^*\circ (j\times id_{A^\sharp})^*\circ ( id_{D^\sharp}\times i)_!\\
&\to&(j\times id_{A^\sharp})^*\circ ( id_{D^\sharp}\times i)_!
\end{array}
$$
Proposition \ref{prop:BC condition} implies that this natural transformation is pointwise an equivalence, and so is globally an equivalence.

\p We now suppose that the morphism $i:I\to A^\sharp$ is smooth, and we are willing to construct a morphism $i_*:\uLCart(A^\sharp)\to \uLCart(I)$ which corresponds to $\Rb i_*:\LCartc(I)\to \LCart(A^\sharp)$ on the sub maximal $\iun$-categories.. 

As smooth morphisms are stable by pullback, the maps $i\times id_b^\flat$ are smooth for any $b:\Theta$. The morphism $i^*:E_0\to E_1$ then preserves colimits and fits into an adjunction
% https://q.uiver.app/#q=WzAsMixbMCwwLCJpXio6RV8xIl0sWzEsMCwiRV8wOmlfKiJdLFswLDEsIiIsMix7Im9mZnNldCI6LTJ9XSxbMSwwLCIiLDIseyJvZmZzZXQiOi0yfV0sWzIsMywiIiwyLHsibGV2ZWwiOjEsInN0eWxlIjp7Im5hbWUiOiJhZGp1bmN0aW9uIn19XV0=
\begin{equation}
\label{eq:adj i pullstar}
\begin{tikzcd}
	{i^*:E_1} & {E_0:i_*}
	\arrow[""{name=0, anchor=center, inner sep=0}, shift left=2, from=1-1, to=1-2]
	\arrow[""{name=1, anchor=center, inner sep=0}, shift left=2, from=1-2, to=1-1]
	\arrow["\dashv"{anchor=center, rotate=-90}, draw=none, from=0, to=1]
\end{tikzcd}
\end{equation}
where the left adjoint sends a left cartesian fibration $p$ over $A^\sharp\times a^\flat$ to $ (i\times id_a)^*p$ and the right adjoint sends a left cartesian fibration $q$ over $I\times a^\flat$ to $\Rb (i\times id_a)_* q$.
\begin{lemma}
\label{lemma:technical lemma i pullstar}
Let $p$ be a left cartesian fibration over $I$. We have an equivalence $$\Rb (i\times id_{a^\flat})_*(p\times id_{a^\flat})\sim (\Rb i_* p)\times id_{a^\flat}.$$
\end{lemma}
\begin{proof}
The morphism $p\times id_{a^\flat}$ is the limit of the cospan
$$p\to id_I\leftarrow id_I\times id_{a^\flat}$$
The result is then a direct consequence of the fact that $\Rb i_*$ preserves limits as it is a right adjoint.
\end{proof}



We recall that $\tilde{E_0}$ and $\tilde{E_1}$ are defined as the full sub $\iun$-categories of $E_0$ and $E_1$ whose objects are respectively of shape $p\times id_a$ and $q\times id_a$ for $p$ and $q$ classified left cartesian fibrations over $I$ and $A^\sharp$.
The lemma \ref{lemma:technical lemma i pullstar} and the second assertion of lemma \ref{lemma:technical lemma i pull} imply that \eqref{eq:adj i pullstar} restricts to an adjunction
% https://q.uiver.app/#q=WzAsMixbMCwwLCJpXio6XFx0aWxkZXtFXzF9Il0sWzEsMCwiXFx0aWxkZXtFXzB9OmlfKiJdLFswLDEsIiIsMix7Im9mZnNldCI6LTJ9XSxbMSwwLCIiLDIseyJvZmZzZXQiOi0yfV0sWzIsMywiIiwyLHsibGV2ZWwiOjEsInN0eWxlIjp7Im5hbWUiOiJhZGp1bmN0aW9uIn19XV0=
\begin{equation}
\label{eq:adj i pull2star}
\begin{tikzcd}
	{i^*:\tilde{E_1}} & {\tilde{E_0}:i_*}
	\arrow[""{name=0, anchor=center, inner sep=0}, shift left=2, from=1-1, to=1-2]
	\arrow[""{name=1, anchor=center, inner sep=0}, shift left=2, from=1-2, to=1-1]
	\arrow["\dashv"{anchor=center, rotate=-90}, draw=none, from=0, to=1]
\end{tikzcd}
\end{equation}


\begin{lemma}
\label{lemma:technical lemma i pull2star} 
Let $q\to q'$ be a morphism in $\tilde{E_0}$ corresponding to a cartesian square. The induced morphism $i_*(q)\to i_*(q')$ also corresponds to a cartesian square. 
\end{lemma}
\begin{proof}
The proof is similar to that of the lemma \ref{lemma:technical lemma i pull2}, using lemma \ref{lemma:technical lemma i pullstar} instead of  lemma \ref{lemma:technical lemma i pull}.
\end{proof}

The lemmas \ref{lemma:technical lemma i pull2} and \ref{lemma:technical lemma i pull2star} imply that the two adjoints of \eqref{eq:adj i pull2star} preserve the cartesian cells of the Grothendieck fibrations $\tilde{E_0}\to \Theta$ and $\tilde{E_1}\to \Theta$. These two adjoints then induce by
 Grothendieck deconstruction a family of adjunction
% https://q.uiver.app/#q=WzAsMixbMCwwLCIoaV9hKV4qOlxcTENhcnQoQV5cXHNoYXJwO2EpIl0sWzEsMCwiXFxMQ2FydGMoSTthKTooaV9hKV8qIl0sWzAsMSwiIiwyLHsib2Zmc2V0IjotMn1dLFsxLDAsIiIsMix7Im9mZnNldCI6LTJ9XSxbMiwzLCIiLDIseyJsZXZlbCI6MSwic3R5bGUiOnsibmFtZSI6ImFkanVuY3Rpb24ifX1dXQ==
\begin{equation}
\label{eq:adj i pull3star}
\begin{tikzcd}
	{(i_a)^*:\LCart(A^\sharp;a)} & {\LCartc(I;a):(i_a)_*}
	\arrow[""{name=0, anchor=center, inner sep=0}, shift left=2, from=1-1, to=1-2]
	\arrow[""{name=1, anchor=center, inner sep=0}, shift left=2, from=1-2, to=1-1]
	\arrow["\dashv"{anchor=center, rotate=-90}, draw=none, from=0, to=1]
\end{tikzcd}
\end{equation}
natural in $a:\Theta^{op}$. The family of functors $(i_a)_*$ then induces a morphism of $\io$-categories\index[notation]{(f6@$f_*:\uLCartc(I)\to \uLCart(A^\sharp)$}
\begin{equation}
\label{eq:i push op}
i_*:\uLCartc(I)\to \uLCart(A^\sharp)
\end{equation}
which is equivalent to $\Rb i_*:\LCartc(I)\to \LCart(A^\sharp)$ on the sub maximal $\iun$-categories.
The unit and counit of adjunction \eqref{eq:adj i pull3star} induce natural transformation
\begin{equation}
\label{eq:i pull unit an counit op}
\mu: id\to i_*i^*~~~~ \epsilon:i^*i_*\to id
\end{equation}
and equivalences
$(\epsilon\circ_0 i^*)\circ_1(i^*\circ_0 \mu) \sim id_{i^*}$ and $(i_*\circ_0 \epsilon)\circ_1 (\mu \circ_0 i_* )\sim id_{i_*}$.



\p Let $j:C^\sharp\to D^\sharp$ be a morphism between $\io$-categories. We claim that the commutative square 
% q.uiver.app/#q=WzAsNCxbMSwxLCJcXHVMQ2FydGMoQ15cXHNoYXJwXFx0aW1lcyBJKSJdLFsxLDAsIlxcdUxDYXJ0YyhEXlxcc2hhcnBcXHRpbWVzIEkpIl0sWzAsMCwiXFx1TENhcnQoRF5cXHNoYXJwXFx0aW1lcyBBXlxcc2hhcnApIl0sWzAsMSwiXFx1TENhcnQoQ15cXHNoYXJwXFx0aW1lcyBBXlxcc2hhcnApIl0sWzEsMCwiKGpcXHRpbWVzIGlkX3tJfSleKiJdLFsyLDEsIiggaWRfe0ReXFxzaGFycH1cXHRpbWVzIGkpXioiXSxbMywwLCIoIGlkX3tDXlxcc2hhcnB9XFx0aW1lcyBpKV4qIiwyXSxbMiwzLCIoalxcdGltZXMgaWRfe0FeXFxzaGFycH0pXioiLDJdXQ==
\[\begin{tikzcd}
	{\uLCart(D^\sharp\times A^\sharp)} & {\uLCartc(D^\sharp\times I)} \\
	{\uLCart(C^\sharp\times A^\sharp)} & {\uLCartc(C^\sharp\times I)}
	\arrow["{(j\times id_{I})^*}", from=1-2, to=2-2]
	\arrow["{( id_{D^\sharp}\times i)^*}", from=1-1, to=1-2]
	\arrow["{( id_{C^\sharp}\times i)^*}"', from=2-1, to=2-2]
	\arrow["{(j\times id_{A^\sharp})^*}"', from=1-1, to=2-1]
\end{tikzcd}\]
induces a commutative square
% q.uiver.app/#q=WzAsNCxbMCwwLCJcXHVMQ2FydGMoRF5cXHNoYXJwXFx0aW1lcyBJKSJdLFswLDEsIlxcdUxDYXJ0YyhDXlxcc2hhcnBcXHRpbWVzIEkpIl0sWzEsMSwiXFx1TENhcnQoQ15cXHNoYXJwXFx0aW1lcyBBXlxcc2hhcnApIl0sWzEsMCwiXFx1TENhcnQoRF5cXHNoYXJwXFx0aW1lcyBBXlxcc2hhcnApIl0sWzAsMywiKCBpZF97RF5cXHNoYXJwfVxcdGltZXMgaSlfKiJdLFswLDEsIihqXFx0aW1lcyBpZF97SX0pXioiLDJdLFsxLDIsIiggaWRfe0NeXFxzaGFycH1cXHRpbWVzIGkpXyoiLDJdLFszLDIsIihqXFx0aW1lcyBpZF97QV5cXHNoYXJwfSleKiJdXQ==
\begin{equation}
\label{eq:commutative pull push op}
\begin{tikzcd}
	{\uLCartc(D^\sharp\times I)} & {\uLCart(D^\sharp\times A^\sharp)} \\
	{\uLCartc(C^\sharp\times I)} & {\uLCart(C^\sharp\times A^\sharp)}
	\arrow["{( id_{D^\sharp}\times i)_*}", from=1-1, to=1-2]
	\arrow["{(j\times id_{I})^*}"', from=1-1, to=2-1]
	\arrow["{( id_{C^\sharp}\times i)_*}"', from=2-1, to=2-2]
	\arrow["{(j\times id_{A^\sharp})^*}", from=1-2, to=2-2]
\end{tikzcd}
\end{equation}

\textit{A priori}, the natural transformations \eqref{eq:i pull unit an counit op} implies that this square commutes up the natural transformation:
$$
\begin{array}{rcl}
(j\times id_{A^\sharp})^*\circ (id_{D^\sharp}\times i)_*&\to & (id_{C^\sharp}\times i)_*\circ(id_{C^\sharp}\times i)^*\circ(j\times id_{A^\sharp})^*\circ (id_{D^\sharp}\times i)_*\\
&\sim &(id_{C^\sharp}\times i)_*\circ(j\times id_I)^*\circ(id_{D^\sharp}\times i)^*\circ (id_{D^\sharp}\times i)_*\\
&\to &(id_{C^\sharp}\times i)_*\circ(j\times id_I)^*
\end{array}
$$
Proposition \ref{prop:BC condition 2} implies that this natural transformation is pointwise an equivalence, and so is globally an equivalence.



\chapter{The $\io$-category of small $\io$-categories}
\label{chapter:The io-category of small io-categories}

\minitoc
\vspace{2cm}
This chapter aims to establish analogs of the fundamental categorical constructions to the $\io$ case. In the first section, we define the $\io$-category of small $\io$-categories $\uni$ (paragraph \ref{para:defi of uni}), and we prove a first incarnation of the Grothendieck construction:
\begin{icor}[\ref{cor: Grt equivalence}]
Let $\uni$ be the $\io$-category of small $\io$-categories, and $A$ an $\io$-category. There is an equivalence
$$\int_A:\Hom(A,\uni)\to \tau_0 \LCart(A^\sharp).$$
where $\tau_0 \LCart(A^\sharp)$ is the $\infty$-groupoid of left cartesian fibrations over $A^\sharp$ with small fibers.
\end{icor}
Given a functor $f:A\to \uni$, the left cartesian fibration $\int_Af$ is a colimit (computed in $\ocatm_{/A^\sharp}$) of
a simplicial object whose value on $n$ is of shape
$$\coprod_{x_0,...,x_n:A_0}X(x_0)^\flat\times\hom_A(x_0,...,x_n)^\flat\times A^\sharp_{x_n/}\to A^\sharp$$
This formula is similar to the one given in \cite{Gepner_Lax_colimits_and_free_fibration}
 for $\iun$-categories, and to the one given in \cite{Warren_the_strict_omega_groupoid_interpretation_of_type_theory} for strict $\omega$-categories.

We also prove a univalence result:

\begin{icor}[\ref{cor:univalence}]
Let $I$ be a marked $\io$-category. We denote by $I^\sharp$ the marked $\io$-category obtained from $I$	 by marking all cells and $\iota:I\to I^\sharp$ the induced morphism. There is a natural correspondence between \begin{enumerate}
\item functors
$f:I\otimes [1]^\sharp\to \uni^\sharp,$

\item pairs of small left cartesian fibration $X\to I$, $Y\to I$ together with diagrams 
% https://q.uiver.app/?q=WzAsNixbMSwyLCJJIl0sWzAsMSwiXFxpb3RhXipZIl0sWzMsMiwiSV5cXHNoYXJwIl0sWzIsMSwiWSJdLFsxLDAsIlxcaW90YV4qWCJdLFszLDAsIlgiXSxbMCwyLCJcXGlvdGEiLDJdLFsxLDBdLFszLDJdLFsxLDNdLFs0LDBdLFs1LDJdLFs0LDVdLFs0LDEsIlxccGhpIiwxXSxbNCw2LCIiLDAseyJsZXZlbCI6MSwic3R5bGUiOnsibmFtZSI6ImNvcm5lciJ9fV0sWzEsNiwiIiwwLHsibGV2ZWwiOjEsInN0eWxlIjp7Im5hbWUiOiJjb3JuZXIifX1dXQ==
\[\begin{tikzcd}
	& {\iota^*X} && X \\
	{\iota^*Y} && Y \\
	& I && {I^\sharp}
	\arrow[""{name=0, anchor=center, inner sep=0}, "\iota"', from=3-2, to=3-4]
	\arrow[from=2-1, to=3-2]
	\arrow[from=2-3, to=3-4]
	\arrow[from=2-1, to=2-3]
	\arrow[from=1-2, to=3-2]
	\arrow[from=1-4, to=3-4]
	\arrow[from=1-2, to=1-4]
	\arrow["\phi"{description}, from=1-2, to=2-1]
	\arrow["\lrcorner"{anchor=center, pos=0.125}, draw=none, from=1-2, to=0]
	\arrow["\lrcorner"{anchor=center, pos=0.125}, draw=none, from=2-1, to=0]
\end{tikzcd}\]
\end{enumerate}
\end{icor}

Recall that if $I$ is of shape $B^\sharp$, then the underlying $\io$-category of $B^\sharp\otimes[1]^\sharp$ is $B\times [1]$, and if $I$ is of shape $B^\flat$, the underlying $\io$-category of $B^\flat\otimes[1]^\sharp$ is $B\otimes[1]$. On the other hand, if $I$ is $B^\sharp$, $\iota$ is the identity, and $\phi$ then preserves all cartesian liftings, and if $I$ is $B^\flat$, $\phi$ doesn't need to preserve cartesian liftings.

By varying the marking, we can continuously move from the cartesian product with the interval to the Gray product with the interval on one side, and on the other side, we can continuously move from morphisms between left cartesian fibrations that preserve the marking to the ones that do not preserve it \textit{a priori}.

Eventually, we also get an $\io$-functorial Grothendieck construction, expressed by the following corollary:

\begin{icor}[\ref{cor:lcar et hom}]
Let $A$ be a $\U$-small $\io$-category.
Let $\uLCart(A^\sharp)$ be the $\io$-category of small left cartesian fibrations over $A^\sharp$. 
There is an equivalence
$$\uHom(A,\uni)\sim \uLCart(A^\sharp)$$
natural in $A$.
\end{icor}


In the second section of this chapter, for a locally small $\io$-category $C$, we construct the Yoneda embedding, which is a functor $y:C\to \widehat{C}$ where $\widehat{C}:=\uHom(C^t,\uni)$. We prove the Yoneda lemma:
\begin{itheorem}[\ref{theo:Yoneda ff}]
The Yoneda embedding is fully faithful.
\end{itheorem}
\begin{itheorem}[\ref{theo:Yoneda lemma}]
Let $C$ be an $\io$-category. There is an equivalence between the functor
$$\hom_{\w{C}}(y_{\uvar},\uvar):C^t\times \w{C}\to \uni$$ and
the functor 
$$ev:C^t\times \w{C}\to \uni .$$
\end{itheorem}
In the last three sections, we use these results to study and demonstrate the basic properties of adjunctions, lax (co)limits, and left Kan extensions.

We begin by studying adjunctions, and we establish the following expected result.
\begin{itheorem}[\ref{theo:two adjunction definition}]
Let $u:C\to D$ and $v:D\to C$ be two functors between locally $\U$-small $\io$-categories. 
The two following are equivalent. 
\begin{enumerate}
\item The pair $(u,v)$ admits an adjoint structure.
\item Their exists a pair of natural transformations $\mu: id_C \to vu$ and $\epsilon:uv\to id_D$ together with equivalences $(\epsilon\circ_0 u)\circ_1(u\circ_0 \mu) \sim id_{u}$ and $(v\circ_0 \epsilon)\circ_1 (\mu \circ_0 v )\sim id_{v}$.
\end{enumerate}
\end{itheorem}

In the next subsection, given a morphism $f:I\to C^\sharp$ between marked $\io$-categories, we define the notion of lax colimit and lax limit for the functor $f$. If $f$ admits such a lax colimit, for any $1$-cell $i:a\to b$ in $I$, we have a triangle
% https://q.uiver.app/#q=WzAsNCxbMCwwXSxbMCwxLCJGKGEpIl0sWzEsMSwiXFxsYXhjb2xpbV9JRiJdLFsxLDAsIkYoYikiXSxbMSwzLCJGKGkpIiwwLHsiY3VydmUiOi01fV0sWzEsMl0sWzMsMSwiIiwxLHsic2hvcnRlbiI6eyJzb3VyY2UiOjMwLCJ0YXJnZXQiOjMwfSwibGV2ZWwiOjJ9XSxbMCwxLCIiLDAseyJzdHlsZSI6eyJib2R5Ijp7Im5hbWUiOiJub25lIn0sImhlYWQiOnsibmFtZSI6Im5vbmUifX19XSxbMywyXV0=
\[\begin{tikzcd}
	{} & {F(b)} \\
	{F(a)} & {\laxcolim_IF}
	\arrow["{F(i)}", curve={height=-30pt}, from=2-1, to=1-2]
	\arrow[from=2-1, to=2-2]
	\arrow[shorten <=8pt, shorten >=8pt, Rightarrow, from=1-2, to=2-1]
	\arrow[draw=none, from=1-1, to=2-1]
	\arrow[from=1-2, to=2-2]
\end{tikzcd}\]
If $i$ is marked, the preceding $2$-cell is an equivalence. 
For any $2$-cell $u:i\to j$, we have a diagram
% https://q.uiver.app/#q=WzAsNyxbMCwxLCJGKGEpIl0sWzEsMCwiRihiKSJdLFsxLDEsIlxcbGF4Y29saW1fSUYiXSxbMiwxLCJGKGEpIl0sWzMsMCwiRihiKSJdLFszLDEsIlxcbGF4Y29saW1fSUYiXSxbMiwwXSxbMCwxLCJGKGkpIiwxXSxbMCwyXSxbMSwyXSxbMSwyXSxbMCwxLCJGKGopIiwwLHsiY3VydmUiOi01fV0sWzMsNCwiRihqKSIsMCx7ImN1cnZlIjotNX1dLFszLDVdLFs0LDVdLFs0LDMsIiIsMSx7InNob3J0ZW4iOnsic291cmNlIjozMCwidGFyZ2V0IjozMH0sImxldmVsIjoyfV0sWzYsMywiIiwwLHsic3R5bGUiOnsiYm9keSI6eyJuYW1lIjoibm9uZSJ9LCJoZWFkIjp7Im5hbWUiOiJub25lIn19fV0sWzEwLDgsIiIsMSx7Im9mZnNldCI6Miwic2hvcnRlbiI6eyJzb3VyY2UiOjQwLCJ0YXJnZXQiOjQwfX1dLFsxMSw3LCIiLDEseyJzaG9ydGVuIjp7InNvdXJjZSI6MjAsInRhcmdldCI6MjB9fV0sWzEwLDE2LCIiLDEseyJvZmZzZXQiOi0xLCJzaG9ydGVuIjp7InNvdXJjZSI6MzAsInRhcmdldCI6MzB9LCJsZXZlbCI6MSwic3R5bGUiOnsiaGVhZCI6eyJuYW1lIjoibm9uZSJ9fX1dLFsxMCwxNiwiIiwxLHsic2hvcnRlbiI6eyJzb3VyY2UiOjMwLCJ0YXJnZXQiOjMwfSwibGV2ZWwiOjF9XSxbMTAsMTYsIiIsMSx7Im9mZnNldCI6MSwic2hvcnRlbiI6eyJzb3VyY2UiOjMwLCJ0YXJnZXQiOjMwfSwibGV2ZWwiOjEsInN0eWxlIjp7ImhlYWQiOnsibmFtZSI6Im5vbmUifX19XV0=
\[\begin{tikzcd}
	& {F(b)} & {} & {F(b)} \\
	{F(a)} & {\laxcolim_IF} & {F(a)} & {\laxcolim_IF}
	\arrow[""{name=0, anchor=center, inner sep=0}, "{F(i)}"{description}, from=2-1, to=1-2]
	\arrow[""{name=1, anchor=center, inner sep=0}, from=2-1, to=2-2]
	\arrow[from=1-2, to=2-2]
	\arrow[""{name=2, anchor=center, inner sep=0}, from=1-2, to=2-2]
	\arrow[""{name=3, anchor=center, inner sep=0}, "{F(j)}", curve={height=-30pt}, from=2-1, to=1-2]
	\arrow["{F(j)}", curve={height=-30pt}, from=2-3, to=1-4]
	\arrow[from=2-3, to=2-4]
	\arrow[from=1-4, to=2-4]
	\arrow[shorten <=8pt, shorten >=8pt, Rightarrow, from=1-4, to=2-3]
	\arrow[""{name=4, anchor=center, inner sep=0}, draw=none, from=1-3, to=2-3]
	\arrow[shift right=2, shorten <=12pt, shorten >=12pt, Rightarrow, from=2, to=1]
	\arrow[shorten <=4pt, shorten >=4pt, Rightarrow, from=3, to=0]
	\arrow[shift left=0.7, shorten <=14pt, shorten >=16pt, no head, from=2, to=4]
	\arrow[shorten <=14pt, shorten >=14pt, from=2, to=4]
	\arrow[shift right=0.7, shorten <=14pt, shorten >=16pt, no head, from=2, to=4]
\end{tikzcd}\]
If $u$ is marked, the $3$-cell is an equivalence. We can continue these diagrams in higher dimensions and we have
similar assertions for lax limits.
The marking therefore allows us to play on the "lax character" of the universal property that the lax colimit must verify.


After providing several characterizations of lax colimits and limits, we prove the following result:
\begin{itheorem}[\ref{theo:presheaevs colimi of representable}]
Let $C$ be a $\U$-small $\io$-category. Let $f$ be an object of $\w{C}$. We define $C^\sharp_{/f}$ as the following pullback
% https://q.uiver.app/#q=WzAsNCxbMSwwLCJcXHd7Q31eXFxzaGFycF97L2Z9Il0sWzEsMSwiXFx3e0N9Xlxcc2hhcnAiXSxbMCwxLCJDXlxcc2hhcnAiXSxbMCwwLCJDXlxcc2hhcnBfey9mfSJdLFszLDJdLFszLDBdLFswLDFdLFsyLDEsInleXFxzaGFycCIsMl1d
\[\begin{tikzcd}
	{C^\sharp_{/f}} & {\w{C}^\sharp_{/f}} \\
	{C^\sharp} & {\w{C}^\sharp}
	\arrow[from=1-1, to=2-1]
	\arrow[from=1-1, to=1-2]
	\arrow[from=1-2, to=2-2]
	\arrow["{y^\sharp}"', from=2-1, to=2-2]
\end{tikzcd}\]
The colimit of the functor 
$\pi:C^\sharp_{/f}\to C^\sharp\xrightarrow{y^\sharp} \w{C}^\sharp$ is $f$.
\end{itheorem}

We conclude this chapter by studying Kan extensions.



\paragraph{Cardinality hypothesis.}
We fix during this chapter three Grothendieck universes $\U \in \V\in\Wcard$, such that $\omega\in \U$. 
All defined notions depend on a choice of cardinality. When nothing is specified, this corresponds to the implicit choice of the cardinality $\V$.
We denote by $\Set$ the $\Wcard$-small $1$-category of $\V$-small sets, $\igrd$ the $\Wcard$-small $\iun$-category of $\V$-small $\infty$-groupoids and $\icat$ the $\Wcard$-small $\iun$-category of $\V$-small $\iun$-categories. 

\section{Univalence}
\label{section:Univalence}
\subsection{Internal category}
\p For $X$ an object of $\iPsh{\Theta}$ and $K$ a simplicial $\infty$-groupoid, we define the simplicial object $\langle X, K\rangle$ of $\ocat$ whose value on $n$ is given by \index[notation]{((g20@$\langle a,n\rangle$}
$$\langle X,K\rangle_n := X\times K_n$$
If $K$ is the representable $[n]$, this object is simply denoted by $\langle X,n\rangle$.
We also define the following set of morphism of $\iPsh{\Delta\times \Theta}$:\sym{(t@$\T$}
$$\T:= \{\langle a,f\rangle,~ a\in \Theta, f\in \mbox{$\W_1$}\} \cup \{\langle g,n\rangle,~ g\in \W, [n]\in \Delta\}$$

\p A \wcnotion{$\ioun$-category}{category5@$\ioun$-category} is a $\T$-local $\infty$-presheaf $C\in \iPsh{\Theta\times \Delta}$. We then naturally define \sym{((a80@$\ouncat$}
$$\ouncat := \iPsh{\Theta\times \Delta}_{\T}.$$
Unfolding the definition, an $\ioun$-category is a simplicial object $C:\Delta^{op}\to \ocat$
such that the induced morphisms
$$C_0\to\lim_{[k]\to E^{eq}}C_k~~~~\mbox{ and }~~~C_n\to C_1\times_{C_0}\times...\times_{C_0}C_1~n\in \Nb$$
are equivalences. 
Remark that we have a cartesian square
% https://q.uiver.app/#q=WzAsNCxbMCwwLCJcXG91bmNhdCJdLFsxLDAsIlxcRnVuKFxcVGhldGFee29wfSxcXGljYXQpIl0sWzEsMSwiXFxGdW4oXFxUaGV0YV57b3B9LFxcaWdyZClcXHRpbWVzIFxcRnVuKFxcVGhldGFee29wfSxcXGlncmQpIl0sWzAsMSwiXFxvY2F0XFx0aW1lcyBcXG9jYXQiXSxbMSwyXSxbMywyXSxbMCwzXSxbMCwxXSxbMCw1LCIiLDEseyJsZXZlbCI6MSwic3R5bGUiOnsibmFtZSI6ImNvcm5lciJ9fV1d
\[\begin{tikzcd}
	\ouncat & {\Fun(\Theta^{op},\icat)} \\
	{\ocat\times \ocat} & {\Fun(\Theta^{op},\igrd)\times \Fun(\Theta^{op},\igrd)}
	\arrow[from=1-2, to=2-2]
	\arrow[""{name=0, anchor=center, inner sep=0}, from=2-1, to=2-2]
	\arrow[from=1-1, to=2-1]
	\arrow[from=1-1, to=1-2]
	\arrow["\lrcorner"{anchor=center, pos=0.125}, draw=none, from=1-1, to=0]
\end{tikzcd}\]
where the lower horizontal morphism is induced by the canonical inclusion of $\io$-category onto $\infty$-presheaves on $\Theta$, and the right vertical one is induced by the functor that maps an $\iun$-category to the pair consisting of the $\infty$-groupoid of objects and the $\infty$-groupoid of arrows.




\p
A morphism $p:X\to A$ between two $\infty$-presheaves on $\Theta\times \Delta$ is a \notion{left fibration} if it has the unique right lifting property against the set of morphism \sym{(j@$\J$}
$$\J:=\{\langle a,\{0\}\rangle \to \langle a,n\rangle~,a\in\Theta, [n]\in\Delta\}\cup \{\langle g,0\rangle,~ g\in\W\}$$
Unfolding the notation, this is equivalent to asking that $X_0\to A_0$ is $\W$-local, and that the natural square 
% q.uiver.app/#q=WzAsNCxbMCwwLCJYX24iXSxbMSwwLCJYX3tcXHswXFx9fSJdLFswLDEsIkFfbiJdLFsxLDEsIkFfe1xcezBcXH19Il0sWzAsMV0sWzAsMl0sWzIsM10sWzEsM11d
\[\begin{tikzcd}
	{X_n} & {X_{\{0\}}} \\
	{A_n} & {A_{\{0\}}}
	\arrow[from=1-1, to=1-2]
	\arrow[from=1-1, to=2-1]
	\arrow[from=2-1, to=2-2]
	\arrow[from=1-2, to=2-2]
\end{tikzcd}\]
is cartesian. 
\begin{prop}
\label{prop:if left fib the fib}
We have an inclusion $T\subset \widehat{J}$.
\end{prop}
\begin{proof}
Let $a$ be an object of $\Theta$.
The  $\infty$-groupoid of morphisms $i$ of $\iPsh{\Delta}$ such that $\langle a,i\rangle$ is in $\widehat{J}$ contains by definition $\{0\}\to [n]$, and is closed by colimits and left cancelation. This $\infty$-groupoid then contains all initial morphism between $\infty$-presheaves on $\Delta$. As morphisms of $\W_1$ are initial, $\widehat{J}$ includes morphisms of shape $\langle a, f\rangle$ for  $a\in \Theta$ and $f\in \W_1$.

Let $g:a\to b$ be a morphism of $\W$ and $n$ an integer. We have a commutative square
% https://q.uiver.app/#q=WzAsNCxbMCwwLCJcXGxhbmdsZSBhLFxcezBcXH1cXHJhbmdsZSJdLFswLDEsIlxcbGFuZ2xlIGIsXFx7MFxcfVxccmFuZ2xlIl0sWzEsMCwiXFxsYW5nbGUgYSxuXFxyYW5nbGUiXSxbMSwxLCJcXGxhbmdsZSBiLG5cXHJhbmdsZSJdLFswLDEsIlxcbGFuZ2xlIGcsXFx7MFxcfVxccmFuZ2xlIiwyXSxbMiwzLCJcXGxhbmdsZSBnLG5cXHJhbmdsZSJdLFswLDJdLFsxLDNdXQ==
\[\begin{tikzcd}
	{\langle a,\{0\}\rangle} & {\langle a,n\rangle} \\
	{\langle b,\{0\}\rangle} & {\langle b,n\rangle}
	\arrow["{\langle g,\{0\}\rangle}"', from=1-1, to=2-1]
	\arrow["{\langle g,n\rangle}", from=1-2, to=2-2]
	\arrow[from=1-1, to=1-2]
	\arrow[from=2-1, to=2-2]
\end{tikzcd}\]
The two horizontal morphisms are in $\widehat{J}$. By left cancellation, this implies that  $\langle g,n\rangle$ is in $ \widehat{J}$ which concludes the proof.
\end{proof}
If $X\to A$ is a left fibration, with $A$ a $\ioun$-category, the last proposition implies that $X$ is also a $\ioun$-category. We denote by \wcnotation{$\Lfib(A)$}{(lfib@$\Lfib(\uvar)$} the full sub $\iun$-category of $\ouncat_{/A}$ whose objects are left fibrations.


\begin{prop}
\label{prop:lfib and W}
There is a canonical equivalence: 
$$\Lfib(\langle a,C \rangle)\sim \Fun(C,\ocat_{/a})$$
natural in $a:\Theta^{op}$ and $C:\icat^{op}$.
\end{prop}
\begin{proof}
Let $a$ be an object of $\Theta^{op}$ and $C$ an $\iun$-category. We have a canonical equivalence 
$$\iPsh{\Theta\times \Delta}_{/\langle a , C\rangle}\sim \iPsh{\Theta_{/a}\times \Delta_{/C}}\sim \Fun(\Theta_{/a}^{op},\iPsh{\Delta}_{/C})$$
The previous equivalence induces an equivalence
$$(\iPsh{\Theta\times \Delta}_{/\langle a,C\rangle})_{\{\langle b,\{0\}\rangle \to \langle b,[n]\rangle\}_{/\langle a , C\rangle}} \sim \Fun(\Theta_{/a}^{op}, (\iPsh{\Delta}_{/C})_{\I^0_{/C}})$$
where $\I^0_{/C}$ corresponds to the $\infty$-groupoid of morphisms of $\iPsh{\Delta}_{/C}$ of shape
% https://q.uiver.app/#q=WzAsMyxbMCwxLCJcXHswXFx9Il0sWzEsMCwiW25dIl0sWzEsMSwiQyJdLFswLDJdLFswLDFdLFsxLDJdXQ==
\[\begin{tikzcd}
	& {[n]} \\
	{\{0\}} & C
	\arrow[from=2-1, to=2-2]
	\arrow[from=2-1, to=1-2]
	\arrow[from=1-2, to=2-2]
\end{tikzcd}\]
for $n$ any integer.
The $\iun$-category $(\iPsh{\Delta}_{/C})_{\I^0_{/C}}$ is equivalent to the $\iun$-category of Grothendieck $\V$-small opfibrations fibered in $\infty$-groupoid over $C$, which is itself equivalent to $\Fun(C,\igrd)$ according to the Grothendieck construction. 
We then have an equivalence 
\begin{equation}
\label{eq:lfib and W}
(\iPsh{\Theta\times \Delta}_{/\langle a,C\rangle})_{\{\langle b,\{0\}\rangle \to \langle b,[n]\rangle\}_{/\langle a , C\rangle}} \sim \Fun (\Theta_{/a}^{op}, \Fun(C,\igrd))\sim \Fun (C, \iPsh{\Theta}_{/a})
\end{equation}
By definition, $\Lfib(\langle a,C\rangle)$ is the fully faithful sub $\iun$-category of the left hand $\iun$-category corresponding to objects that are local with respect to the image of set of morphism
 $\{\langle g,0\rangle, g\in \W\}_{/\langle a,C\rangle}$ by the localization functor 
 $$(\iPsh{\Theta\times \Delta}_{/\langle a,C\rangle})\to (\iPsh{\Theta\times \Delta}_{/\langle a,C\rangle})_{\{\langle b,\{0\}\rangle \to \langle b,[n]\rangle\}_{/\langle a , C\rangle}}.$$
Such $\infty$-presheaves corresponds via the equivalence \eqref{eq:lfib and W} to functors $C\to \iPsh{\Theta}_{/a}$ that are pointwise $\W_{/a}$-local. As $\W_{/a}$-local $\infty$-presheaves on $\Theta_{/a}$ corresponds to elements of $\ocat_{/a}$, we have an equivalence
$$\Lfib(\langle a,C\rangle)\sim \Fun(C,\ocat_{/a}).$$
\end{proof}

\p
A morphism $f:A\to B$ between two $\infty$-presheaves on $\Theta\times \Delta$ induces an adjunction
\begin{equation}
\label{eq:adj between left fibration}
\begin{tikzcd}
	{f_!:\ouncat{/A}} & {\ouncat_{/B}:f^*}
	\arrow[shift left=2, from=1-1, to=1-2]
	\arrow[shift left=2, from=1-2, to=1-1]
\end{tikzcd}
\end{equation}
where $f_!$ is the composition and $f^*$ is the pullback.
As $\Lfib(A)$ is the localization of $\ouncat_{/A}$ along the class of morphisms $\widehat{\J_{/A}}$,
the previous adjunction induces a derived adjunction:
\begin{equation}
\label{eq:derived adj between left fibration}
\begin{tikzcd}
	{\Lb f_!:\Lfib(A)} & {\Lfib(B):\Rb f^*}
	\arrow[shift left=2, from=1-1, to=1-2]
	\arrow[shift left=2, from=1-2, to=1-1]
\end{tikzcd}
\end{equation}
where $\Lb f_!$ sends $E$ onto $\Fb f_!E$ and $\Rb f^*$ is just the restriction of $f^*$ to $\Lfib(B)$.


\p
We denote by $\pi_!:\Fun(\Delta^{op},\iPsh{\Theta})\to \iPsh{\Delta[\Theta]}$ the functor induced by extention by colimits by the canonical morphism $\pi:\Delta\times \Theta\to \Delta[\Theta]$. We also define $\Noiun:\iPsh{\Delta[\Theta]}\to \Fun(\Delta^{op},\iPsh{\Theta})$ as the right adjoint of $\pi_!$. As $\pi_!$ preserves representable, \wcnotation{$\Noiun$}{(noiun@$\Noiun$} preserves colimits. Remark that the image of $T$ by $\pi_!$ is contained in $\widehat{\M}$, and $\Noiun$ induces then by restriction a functor
$$\Noiun:\ocat\to \ouncat.$$
If $C$ is an $\io$-category, $\Noiun C$ corresponds to the simplicial object in $\ocat$:
% q.uiver.app/#q=WzAsNCxbMywwLCJcXGNvcHJvZF97eF8wOlxcdGF1XzBDfTEiXSxbMiwwLCJcXGNvcHJvZF97eF8wLHhfMTpcXHRhdV8wQ31cXGhvbV9DKHhfMCx4XzEpIl0sWzEsMCwiXFxjb3Byb2Rfe3hfMCx4XzEseF8yOlxcdGF1XzBDfVxcaG9tX0MoeF8wLHhfMSx4XzIpIl0sWzAsMCwiXFxjZG90cyJdLFsyLDFdLFsxLDAsIiIsMSx7Im9mZnNldCI6LTJ9XSxbMiwxLCIiLDEseyJvZmZzZXQiOi00fV0sWzEsMiwiIiwxLHsib2Zmc2V0IjoyfV0sWzIsMSwiIiwxLHsib2Zmc2V0Ijo0fV0sWzEsMiwiIiwxLHsib2Zmc2V0IjotMn1dLFswLDFdLFsxLDAsIiIsMSx7Im9mZnNldCI6Mn1dXQ==
\[\begin{tikzcd}
	\cdots & {\coprod_{x_0,x_1,x_2:\tau_0C}\hom_C(x_0,x_1,x_2)} & {\coprod_{x_0,x_1:\tau_0C}\hom_C(x_0,x_1)} & {\coprod_{x_0:\tau_0C}1}
	\arrow[from=1-2, to=1-3]
	\arrow[shift left=2, from=1-3, to=1-4]
	\arrow[shift left=4, from=1-2, to=1-3]
	\arrow[shift right=2, from=1-3, to=1-2]
	\arrow[shift right=4, from=1-2, to=1-3]
	\arrow[shift left=2, from=1-3, to=1-2]
	\arrow[from=1-4, to=1-3]
	\arrow[shift right=2, from=1-3, to=1-4]
\end{tikzcd}\]
If $p:X\to \Noiun C$ is a left fibration, and $x$ an object of $C$, we will denote by $X(x)$ the fiber of $p_0:X_0\to \Noiun C$ on $x$, and $E(x)$ the canonical morphism $X(x)\to 1$. Unfolding the definitions, and using corollary \ref{cor:if codomain a groupoid, then f is exponentiable}, we then have for any integer $n$ a canonical equivalence:
$$X_n \sim \coprod_{x_0,...,x_n}X(x_0)\times \hom_C(x_0,...,x_n)$$

\begin{prop}
\label{prop:equivalence beetwen left fibration}
Let $C$ be an $\io$-category, and $E$, $F$ two objects of $\Lfib(\Noiun C)$ corresponding to morphisms $X\to \Noiun C$, $Y\to \Noiun C$. Let $\phi:E\to F$ be a morphism.
The following are equivalent:
\begin{enumerate}
\item $\phi$ is an equivalence,
\item for any object $x$ of $C$, the induced morphism $\Rb x^*\phi:\Rb x^*E\to \Rb x^*E$ is an equivalence,
\item for any object $x$ of $C$, the induced morphism $\phi(x):X(x)\to Y(x)$ is an equivalence,
\end{enumerate}
\end{prop}
\begin{proof}
The implication $(1)\Rightarrow (2)$ is direct.
The implication $(2)\Rightarrow (3)$ comes from the fact that for any object $x$ of $C$, the value on $0$ of the simplicial object $\Rb x^*E$ (resp. $\Rb x^*F$) is $X(x)\to 1$ (resp. $Y(x)\to 1$). 

Suppose now that $\phi$ fulfills the last condition. As $\Noiun C$ is $C_0\sim \coprod_{C_0}1$, we have equivalences 
$$X_0\sim \coprod_{x:C_0} X(x)~~~~~Y_0\sim \coprod_{x:C_0} Y(x).$$
The morphism $\phi_0:X_0\to Y_0$ is then an equivalence. Eventually, as $E$ and $F$ are left fibrations, we have 
$$X_n\sim X_{\{0\}}\times_{ (\Noiun C)_{\{0\}} }(\Noiun C)_n\sim Y_{\{0\}}\times_{ (\Noiun C)_{\{0\}} }(\Noiun C)_n\sim Y_n.$$
This implies $(3)\Rightarrow (1)$, which concludes the proof. 
\end{proof}


\begin{prop}
\label{prop:lfib and W 2}
There is an equivalence natural in $C:\ocatm^{op}$ 
between $\Lfib(\Noiun [C,1])$ and the $\iun$-category whose objects are arrows  of shape
$$X(0)\times C\to X(1)$$
and morphisms are natural transformations such that the induced morphism
$X(0)\times C\to Y(0)\times C$
is of shape $f\times id_C$.

For a left fibration $E$ corresponding to a morphism $X\to [C,1]$, this arrow is the one appearing in the diagram:
% https://q.uiver.app/#q=WzAsOCxbMSwwLCJYXzEiXSxbMywwLCJYXzAiXSxbMywyLCJcXE5vaXVuKFtDLDFdKV97XFx7MVxcfX0iXSxbMSwyLCJcXE5vaXVuKFtDLDFdKV8xIl0sWzAsMywiKENeXFxmbGF0LDAsMSkiXSxbMCwxLCJYKDApXlxcZmxhdFxcdGltZXMgQ15cXGZsYXQiXSxbMiwzLCJcXHsxXFx9Il0sWzIsMSwiWCgxKV5cXGZsYXQiXSxbNCwzXSxbMywyXSxbNSwwXSxbMCwxXSxbNSw0XSxbMCwzXSxbMSwyXSxbNCw2XSxbNyw2XSxbNiwyXSxbNywxXSxbNSw3XV0=
\[\begin{tikzcd}[sep =0.3cm]
	& {X_1} && {X_0} \\
	{X(0)^\flat\times C^\flat} && {X(1)^\flat} \\
	& {\Noiun([C,1])_1} && {\Noiun([C,1])_{\{1\}}} \\
	{(C^\flat,0,1)} && {\{1\}}
	\arrow[from=4-1, to=3-2]
	\arrow[from=3-2, to=3-4]
	\arrow[from=2-1, to=1-2]
	\arrow[from=1-2, to=1-4]
	\arrow[from=2-1, to=4-1]
	\arrow[from=1-2, to=3-2]
	\arrow[from=1-4, to=3-4]
	\arrow[from=4-1, to=4-3]
	\arrow[from=2-3, to=4-3]
	\arrow[from=4-3, to=3-4]
	\arrow[from=2-3, to=1-4]
	\arrow[from=2-1, to=2-3]
\end{tikzcd}\]
where the left and the right squares are cartesian.
\end{prop}
\begin{proof}
Left fibrations are detected on pullback along representable. The functor $\Lfib(\uvar)$ then sends colimits of $\iPsh{\Theta\times \Delta}$ to limits.
Remark that we have a cocartesian square
% https://q.uiver.app/#q=WzAsNCxbMSwxLCJcXE5vaXVuW0MsMV0iXSxbMSwwLCJcXGxhbmdsZSBDLDFcXHJhbmdsZSJdLFswLDAsIlxcY29wcm9kX3trXFxsZXExfVxcbGFuZ2xlIEMsXFx7a1xcfVxccmFuZ2xlIl0sWzAsMSwiXFxjb3Byb2Rfe2tcXGxlcTF9XFxsYW5nbGUgWzBdLFxce2tcXH1cXHJhbmdsZSJdLFsyLDFdLFsxLDBdLFsyLDNdLFszLDBdLFswLDIsIiIsMSx7InN0eWxlIjp7Im5hbWUiOiJjb3JuZXIifX1dXQ==
\[\begin{tikzcd}
	{\coprod_{k\leq1}\langle C,\{k\}\rangle} & {\langle C,1\rangle} \\
	{\coprod_{k\leq1}\langle [0],\{k\}\rangle} & {\Noiun[C,1]}
	\arrow[from=1-1, to=1-2]
	\arrow[from=1-2, to=2-2]
	\arrow[from=1-1, to=2-1]
	\arrow[from=2-1, to=2-2]
	\arrow["\lrcorner"{anchor=center, pos=0.125, rotate=180}, draw=none, from=2-2, to=1-1]
\end{tikzcd}\]
According to proposition \ref{prop:lfib and W}, and as $\Lfib(\uvar)$ send colimits to limits, $\Lfib(\Noiun [C,1])$ fits in the cartesian square
% https://q.uiver.app/#q=WzAsNCxbMCwwLCJcXExmaWIoXFxOb2l1biBbQywxXSkiXSxbMSwwLCJcXEZ1bihbMV0sXFxvY2F0X3svQ30pIl0sWzEsMSwiXFxvY2F0X3svQ31cXHRpbWVzIFxcb2NhdF97L0N9Il0sWzAsMSwiXFxvY2F0XFx0aW1lcyBcXG9jYXQiXSxbMywyXSxbMCwzXSxbMSwyXSxbMCwxXSxbMCw0LCIiLDIseyJsZXZlbCI6MSwic3R5bGUiOnsibmFtZSI6ImNvcm5lciJ9fV1d
\[\begin{tikzcd}
	{\Lfib(\Noiun [C,1])} & {\Fun([1],\ocat_{/C})} \\
	{\ocat\times \ocat} & {\ocat_{/C}\times \ocat_{/C}}
	\arrow[""{name=0, anchor=center, inner sep=0}, from=2-1, to=2-2]
	\arrow[from=1-1, to=2-1]
	\arrow[from=1-2, to=2-2]
	\arrow[from=1-1, to=1-2]
	\arrow["\lrcorner"{anchor=center, pos=0.125}, draw=none, from=1-1, to=0]
\end{tikzcd}\]
Using the adjunction 
% https://q.uiver.app/#q=WzAsMixbMSwwLCJcXG9jYXQ6XFx1dmFyXFx0aW1lcyBYIl0sWzAsMCwiXFxkb206XFxvY2F0X3svWH0iXSxbMSwwLCIiLDIseyJvZmZzZXQiOi0yfV0sWzAsMSwiIiwyLHsib2Zmc2V0IjotMn1dLFsyLDMsIiIsMix7ImxldmVsIjoxLCJzdHlsZSI6eyJuYW1lIjoiYWRqdW5jdGlvbiJ9fV1d
\[\begin{tikzcd}
	{\dom:\ocat_{/C}} & {\ocat:\uvar\times C}
	\arrow[""{name=0, anchor=center, inner sep=0}, shift left=2, from=1-1, to=1-2]
	\arrow[""{name=1, anchor=center, inner sep=0}, shift left=2, from=1-2, to=1-1]
	\arrow["\dashv"{anchor=center, rotate=-90}, draw=none, from=0, to=1]
\end{tikzcd}\]
the $\iun$-category $\Lfib(\Noiun [C,1])$ fits in the cartesian square
% https://q.uiver.app/#q=WzAsNCxbMCwwLCJcXExmaWIoXFxOb2l1biBbQywxXSkiXSxbMSwwLCJcXEZ1bihbMV0sXFxvY2F0KSJdLFsxLDEsIlxcb2NhdFxcdGltZXMgXFxvY2F0Il0sWzAsMSwiXFxvY2F0XFx0aW1lcyBcXG9jYXQiXSxbMywyLCIoXFx1dmFyXFx0aW1lcyBDLGlkKSIsMl0sWzAsM10sWzEsMl0sWzAsMV0sWzAsMiwiIiwyLHsic3R5bGUiOnsibmFtZSI6ImNvcm5lciJ9fV1d
\[\begin{tikzcd}
	{\Lfib(\Noiun [C,1])} & {\Fun([1],\ocat)} \\
	{\ocat\times \ocat} & {\ocat\times \ocat}
	\arrow["{(\uvar\times C,id)}"', from=2-1, to=2-2]
	\arrow[from=1-1, to=2-1]
	\arrow[from=1-2, to=2-2]
	\arrow[from=1-1, to=1-2]
	\arrow["\lrcorner"{anchor=center, pos=0.125}, draw=none, from=1-1, to=2-2]
\end{tikzcd}\]
The first assertion then follows from the last cartesian square and the proposition \ref{prp:to show fully faithfullness3} applied to $I:=1.$
 The second is obtained by walking through the equivalences used in the proof of proposition \ref{prop:lfib and W}.
\end{proof}


\begin{prop}
\label{prop:lfib and W 3}
There is an equivalence natural in $C:\ocatm^{op}$ between $\Lfib(([C,1]\otimes[1]^\sharp)^\natural)$ and the $\iun$-category whose objects are diagrams of shape
% https://q.uiver.app/#q=WzAsNixbMCwwLCJYKDAsMClcXHRpbWVzIENeXFxuYXR1cmFsXFxvdGltZXNcXHswXFx9Il0sWzEsMiwiWCgxLDApIl0sWzEsMCwiWCgwLDEpXFx0aW1lcyBDXlxcbmF0dXJhbCJdLFsyLDEsIlgoMSwxKSJdLFswLDIsIlgoMCwwKVxcdGltZXMgQ15cXG5hdHVyYWxcXG90aW1lc1xcezFcXH0iXSxbMSwxLCJYKDAsMClcXHRpbWVzIChDXFxvdGltZXNbMV1eXFxzaGFycCleXFxuYXR1cmFsIl0sWzAsMl0sWzIsM10sWzEsM10sWzAsNV0sWzUsM10sWzQsMV0sWzQsNV1d
\[\begin{tikzcd}
	{X(0,0)\times C^\natural\otimes\{0\}} & {X(0,1)\times C^\natural} \\
	& {X(0,0)\times (C\otimes[1]^\sharp)^\natural} & {X(1,1)} \\
	{X(0,0)\times C^\natural\otimes\{1\}} & {X(1,0)}
	\arrow[from=1-1, to=1-2]
	\arrow[from=1-2, to=2-3]
	\arrow[from=3-2, to=2-3]
	\arrow[from=1-1, to=2-2]
	\arrow[from=2-2, to=2-3]
	\arrow[from=3-1, to=3-2]
	\arrow[from=3-1, to=2-2]
\end{tikzcd}\]
such that $X(0,0)\times C^\natural\otimes\{0\}\to X(0,1)\times C^\natural$ is of shape $f\times id_{C^\natural}$. Morphisms are natural transformations such that the induced morphisms 
$X(0,1)\times C^\natural\to Y(0,1)\times C^\natural$ and $X(0,0)\times (C\otimes[1]^\sharp)^\natural\to Y(0,0)\times (C\otimes[1]^\sharp)^\natural$
are of shape $g\times C^\natural$ and $h\times (C\otimes[1]^\sharp)^\natural$.
\end{prop}
\begin{proof}
The equation \eqref{eq:eq for cylinder marked version} implies that $([C,1]\otimes[1]^\sharp)^\natural$ is the colimit of the diagram
% https://q.uiver.app/#q=WzAsNSxbMCwwLCJbMV1cXHZlZVtDLDFdXlxcbmF0dXJhbCJdLFsxLDAsIltDXFxvdGltZXNeXFxuYXR1cmFsXFx7MFxcfSwxXSJdLFsyLDAsIltDXFxvdGltZXNbMV1eXFxzaGFycCwxXV5cXG5hdHVyYWwiXSxbMywwLCJbQ15cXG5hdHVyYWxcXG90aW1lc1xcezFcXH0sMV0iXSxbNCwwLCJbQywxXV5cXG5hdHVyYWxcXHZlZVsxXSJdLFszLDJdLFszLDRdLFsyLDFdLFswLDFdXQ==
\[\begin{tikzcd}
	{[1]\vee[C,1]^\natural} & {[C\otimes^\natural\{0\},1]} & {[C\otimes[1]^\sharp,1]^\natural} & {[C^\natural\otimes\{1\},1]} & {[C,1]^\natural\vee[1]}
	\arrow[from=1-4, to=1-3]
	\arrow[from=1-4, to=1-5]
	\arrow[from=1-3, to=1-2]
	\arrow[from=1-1, to=1-2]
\end{tikzcd}\]
According to proposition \ref{prop:example of a special colimit3 marked case} and lemma \ref{lemma:a otimes 1 is strict}, this colimit is special, and the  $\iun$-category $\Noiun ([C,1]\otimes[1]^\sharp)^\natural$ is then colimit, computed in $\Psh{\Theta\times\Delta}$, of the diagram
% https://q.uiver.app/#q=WzAsOSxbMiwxLCJcXGxhbmdsZSAoQ1xcb3RpbWVzWzFdXlxcc2hhcnApXlxcbmF0dXJhbCwxXFxyYW5nbGUiXSxbMywwLCJcXGxhbmdsZSBDXlxcbmF0dXJhbFxcb3RpbWVzXFx7MVxcfSwxXFxyYW5nbGUiXSxbMSwwLCJcXGxhbmdsZSBDXlxcbmF0dXJhbFxcb3RpbWVzXFx7MFxcfSwxXFxyYW5nbGUiXSxbMywxLCJcXGxhbmdsZSBDXlxcbmF0dXJhbCwyXFxyYW5nbGUiXSxbMSwxLCJcXGxhbmdsZSBDXlxcbmF0dXJhbCwyXFxyYW5nbGUiXSxbNCwwLCJcXGxhbmdsZSBDXlxcbmF0dXJhbCxcXHswXFx9XFxyYW5nbGVcXGNvcHJvZCBcXGxhbmdsZSBDXlxcbmF0dXJhbCwxXFxyYW5nbGUiXSxbNCwxLCJcXGxhbmdsZSBbMF0sMVxccmFuZ2xlXFxjb3Byb2QgXFxsYW5nbGUgWzBdLDFcXHJhbmdsZSJdLFswLDAsIlxcbGFuZ2xlIENeXFxuYXR1cmFsLDFcXHJhbmdsZVxcY29wcm9kXFxsYW5nbGUgQ15cXG5hdHVyYWwsXFx7MlxcfVxccmFuZ2xlIl0sWzAsMSwiXFxsYW5nbGUgWzBdLDFcXHJhbmdsZVxcY29wcm9kIFxcbGFuZ2xlIFswXSwxXFxyYW5nbGUiXSxbNSwzXSxbNyw0XSxbMiw0XSxbMiwwXSxbMSwwXSxbMSwzXSxbNyw4XSxbNSw2XV0=
\[\begin{tikzcd}[column sep =0.2cm]
	{\langle C^\natural,1\rangle\coprod\langle C^\natural,\{2\}\rangle} & {\langle C^\natural\otimes\{0\},1\rangle} && {\langle C^\natural\otimes\{1\},1\rangle} & {\langle C^\natural,\{0\}\rangle\coprod \langle C^\natural,1\rangle} \\
	{\langle [0],1\rangle\coprod \langle [0],1\rangle} & {\langle C^\natural,2\rangle} & {\langle (C\otimes[1]^\sharp)^\natural,1\rangle} & {\langle C^\natural,2\rangle} & {\langle [0],1\rangle\coprod \langle [0],1\rangle}
	\arrow[from=1-5, to=2-4]
	\arrow[from=1-1, to=2-2]
	\arrow[from=1-2, to=2-2]
	\arrow[from=1-2, to=2-3]
	\arrow[from=1-4, to=2-3]
	\arrow[from=1-4, to=2-4]
	\arrow[from=1-1, to=2-1]
	\arrow[from=1-5, to=2-5]
\end{tikzcd}\]
We then deduce the result from the proposition \ref{prop:lfib and W} in the same way as in the previous proof.
\end{proof}

\begin{prop}
\label{prop:Lfib commue with colimit}
Let $F:I\to \ocat$ be a $\Wcard$-small diagram. The canonical functor
$$\Lfib(\Noiun \colim_IF)\to \lim_I\Lfib(\Noiun F)$$
is an equivalence, where $\colim_IF$ denotes the colimit taken in $\ocat$.
\end{prop}
\begin{proof}
Let $C$ be an object of $\iPsh{\Theta}$.
As left fibrations are detected by unique right lifting property against morphisms whose codomains are of shape $\langle a,n\rangle$, a morphism $p:X\to \Noiun C$ is a left fibration if and only if for any $i:[a,n]\to C$, $(\Noiun i)^*p$ is a left fibration. 
The functor 
$$\begin{array}{ccl}
\Psh{\Delta[\Theta]}^{op}&\to &\icat_{\Wcard}\\
X&\mapsto & \Lfib(\Noiun X)
\end{array}$$
then sends colimits to limits, where $\icat_{\Wcard}$ denotes the (huge) $\iun$-category of $\Wcard$-small $\iun$-categories. To conclude the proof, we then have to show that it sends any morphism $f\in\M$ to an equivalence. If $f$ is of shape $[g,1]$ for $g\in\W$, this directly follows from proposition \ref{prop:lfib and W 2}. Suppose now that $f$ is $[a,\Sp_n]\to [a,n]$. Remark that we have a cocartesian square:
% https://q.uiver.app/#q=WzAsNCxbMSwxLCJcXE5vaXVuIChbYSxuXSkiXSxbMSwwLCJcXE5vaXVuIChbYSxcXFNwX25dKSJdLFswLDAsIlxcbGFuZ2xlIGEsIFxcU3BfblxccmFuZ2xlIl0sWzAsMSwiXFxsYW5nbGUgYSxuXFxyYW5nbGUiXSxbMiwzXSxbMywwXSxbMiwxXSxbMSwwXSxbMCwyLCIiLDEseyJzdHlsZSI6eyJuYW1lIjoiY29ybmVyIn19XV0=
\[\begin{tikzcd}
	{\langle a, \Sp_n\rangle} & {\Noiun ([a,\Sp_n])} \\
	{\langle a,n\rangle} & {\Noiun ([a,n])}
	\arrow[from=1-1, to=2-1]
	\arrow[from=2-1, to=2-2]
	\arrow[from=1-1, to=1-2]
	\arrow[from=1-2, to=2-2]
	\arrow["\lrcorner"{anchor=center, pos=0.125, rotate=180}, draw=none, from=2-2, to=1-1]
\end{tikzcd}\]
The morphism $\Lfib(\Noiun [a,\Sp_n])\to \Lfib(\Noiun [a,n])$ then fits in the cartesian square: 
% https://q.uiver.app/#q=WzAsNCxbMCwwLCJcXExmaWIoXFxOb2l1biBbYSxuXSkiXSxbMCwxLCJcXExmaWIoXFxOb2l1biBbYSxcXFNwX25dKSJdLFsxLDEsIlxcTGZpYihcXGxhbmdsZSBhLCBcXFNwX25cXHJhbmdsZSkiXSxbMSwwLCJcXExmaWIoXFxsYW5nbGUgYSxuXFxyYW5nbGUpIl0sWzMsMl0sWzAsM10sWzEsMl0sWzAsMV0sWzAsNiwiIiwxLHsibGV2ZWwiOjEsInN0eWxlIjp7Im5hbWUiOiJjb3JuZXIifX1dXQ==
\[\begin{tikzcd}
	{\Lfib(\Noiun [a,n])} & {\Lfib(\langle a,n\rangle)} \\
	{\Lfib(\Noiun [a,\Sp_n])} & {\Lfib(\langle a, \Sp_n\rangle)}
	\arrow[from=1-2, to=2-2]
	\arrow[from=1-1, to=1-2]
	\arrow[""{name=0, anchor=center, inner sep=0}, from=2-1, to=2-2]
	\arrow[from=1-1, to=2-1]
	\arrow["\lrcorner"{anchor=center, pos=0.125}, draw=none, from=1-1, to=0]
\end{tikzcd}\]
According to proposition \ref{prop:lfib and W}, we have equivalences
$$\Lfib(\langle a ,\Sp_n\rangle)\sim \lim_{[k]\to\Sp_n}\Fun([k],\ocat_{/a})\sim \Fun([n],\ocat_{/a})\sim \Lfib(\langle a ,n\rangle)$$
It remains the case $f:=E^{eq}\to 1$. We have equivalences $\Noiun E^{eq}\sim \langle [0],E^{eq}\rangle$ and $\Noiun 1\sim 1$ .
The proposition \ref{prop:lfib and W} induces equivalences
$$ \Lfib( \langle [0],E^{eq}\rangle) \sim \lim_{[k]\to E^{eq}}\Fun([k],\ocat)\sim \Fun(1,\ocat)$$
which concludes the proof.
\end{proof}


\p 
\label{para:defi of uni}
Let $A$ be an $\ioun$-category. An object $E:\ouncat_{/A}$ is \wcnotion{$\U$-small}{small object@$\U$-small object of $\ouncat_{/A}$} if for any morphism $i:\langle b,n\rangle\to A$, the space of morphism between $i$ and $E$ is $\U$-small. Remark that an object $F$ of $\Lfib(\Noiun A)$ corresponding to a left fibration $X\to \Noiun A$ is $\U$-small if an only if for any object $a$ of $A$, $X(a)$ is $\U$-small .
Eventually, we define $\Lfib_{\U}( A)$ as the full sub $\iun$-category of $\Lfib( A)$ whose objects correspond to $\U$-small left fibrations. In particular, $\Lfib_{\U}( A)$ is a $\V$-small $\iun$-category unlike $\Lfib( A)$ which is a $\Wcard$-small $\iun$-category. Moreover, the proposition \ref{prop:Lfib commue with colimit} implies that the functor 
$$C:\ocat\mapsto \tau_0\mbox{$\Lfib_{\U}$}(\Noiun C)$$
sends colimits to limits. We then define $\uni$ as the $\io$-category that represents this object:
\begin{equation}
\label{eq:defi of uni}
\begin{array}{rcll}
\uni:&\Theta^{op} &\to &\igrd\\
& a&\mapsto & \tau_0\mbox{$\Lfib_{\U}$}(\Noiun a)
\end{array}
\end{equation}


 We then have by definition an equivalence 
\begin{equation}
\Hom(C,\uni)\sim \tau_0 \mbox{$\Lfib_{\U}$}(\Noiun C).
\end{equation}
As the functor $\Noiun$ preserves product, for any $\io$-category $D$, 
we also have a canonical equivalence
\begin{equation}
\Hom(C,\uHom(D,\uni))\sim \tau_0(\mbox{$\Lfib_{\U}$}(\Noiun C\times \Noiun D)).
\end{equation}
Eventually, by construction, the $\infty$-groupoid of objects of $\uni$ corresponds to the $\infty$-groupoid of $\U$-small $\io$-categories, and according to proposition \ref{prop:lfib and W 2}, we have an equivalence 
\begin{equation}
\label{eq:hom of uni}
\hom_{\uni}(C,D)\sim \uHom(C,D).
\end{equation}
The $\io$-category $\uni$ seems to be a decent candidate for the $\io$-category of $\U$-small $\io$-categories.


\p \label{par:dualities fo omega}
Let $S$ be a subset of $\Nb^*$. We define the subset $\Sigma S=\{i+1,i\in S\}$. 
Remark that for any $n$, we have \ssym{((b49@$(\uvar)^S$}{for $\uni$}
$$(\Noiun C)_n^S\sim (\Noiun C^{\Sigma S})_n$$
We then set the functor 
$$(\uvar)^S:\uni\to (\uni)^{\Sigma S}$$
sending a $\U$-small left fibration $X\to \Noiun C$ to the left fibration $n\mapsto (X_n^S\to (\Noiun C^{\Sigma S})_n^S)$. These functors are called \snotion{dualities}{for $\uni$}.
In particular, we have the \snotionsym{odd duality}{((b60@$(\uvar)^{op}$}{for $\uni$} $(\uvar)^{op}:\uni\to \uni^{co}$, corresponding to the set of odd integer, the \snotionsym{even duality}{((b50@$(\uvar)^{co}$}{for $\uni$} $(\uvar)^{co}:\uni\to (\uni^{t })^{op}$, corresponding to the subset of non negative even integer, the \snotionsym{full duality}{((b80@$(\uvar)^{\circ}$}{for $\uni$} $(\uvar)^{\circ}:\uni \to \uni^{t\circ}$, corresponding to $\Nb^*$ and the \snotionsym{transposition}{((b70@$(\uvar)^t$}{for $\uni$} $(\uvar)^t:\uni \to \uni^{\Sigma t}$, corresponding to the singleton $\{1\}$. Eventually, we have equivalences
$$((\uvar)^{co})^{op}\sim (\uvar)^{\circ} \sim ((\uvar)^{op})^{co}.$$ 


\subsection{Grothendieck construction}
\begin{notation*}
Through this section, we will identify any marked $\io$-categories $C$ with the canonical induced morphism $C\to1$. If $f:X\to Y$ is a morphism, $f\times C$ then corresponds to the canonical morphism $X\times C\to Y$.
\end{notation*}
\p Let $A$ be an $\io$-category and $a$ an object of $A$, we denote by \wcnotation{$h_a^A$}{(h@$h_{a}^{A}$} the morphism $1\to A^\sharp$ induces by $a$. At the end of section \ref{subsection Left and right cartesian fibration}, we have remarked that the left fibrant replacement of $h_a^A$, that we denoted by \wcnotation{$\Fb h^A_a$}{(fh@$\Fb h_{a}^{A}$}, is the fibration $A^\sharp_{a/}\to A^\sharp$. Equation \eqref{eq:fiber of marked splices} induces, for any object $b$ of $A^\sharp$, a cartesian square
% https://q.uiver.app/#q=WzAsNCxbMSwwLCJBXntcXHNoYXJwfV97YS99Il0sWzEsMSwiQV57XFxzaGFycH0iXSxbMCwxLCJcXHtiXFx9Il0sWzAsMCwiXFxob21fQShhLGIpXlxcZmxhdCJdLFsyLDFdLFszLDJdLFszLDBdLFswLDEsIlxcRmIgaF9hXkEiXV0=
\begin{equation}
\label{eq:fiber of slice}
\begin{tikzcd}
	{\hom_A(a,b)^\flat} & {A^{\sharp}_{a/}} \\
	{\{b\}} & {A^{\sharp}}
	\arrow[from=2-1, to=2-2]
	\arrow[from=1-1, to=2-1]
	\arrow[from=1-1, to=1-2]
	\arrow["{\Fb h_a^A}", from=1-2, to=2-2]
\end{tikzcd}
\end{equation}
which induces a canonical morphism $h^A_b\times \hom_A(a,b)^\flat\to \Fb h^A_a$, and consequently, a morphism $\Fb h^A_b\times \hom_A(a,b)^\flat\to \Fb h^A_a$.

The case of $A:=[C,1]$ will be of particular interest. The morphism $\Fb h^{[C,1]}_{1}$ is just $ h^{[C,1]}_{1}$ and theorem \ref{theo:equivalence betwen slice and join} implies that $\Fb h^{[C,1]}_{0}$ is the canonical morphism $1\costar C^\flat\to [C,1]^\sharp$. In this last case, the square \eqref{eq:fiber of slice} corresponds to the square
% https://q.uiver.app/#q=WzAsNCxbMSwwLCIxXFxjb3N0YXIgQ15cXGZsYXQiXSxbMSwxLCJbQywxXV5cXHNoYXJwIl0sWzAsMSwiXFx7MVxcfSJdLFswLDAsIkNeXFxmbGF0Il0sWzIsMV0sWzMsMl0sWzMsMF0sWzAsMSwiXFxGYiBoXzBee1tDLDFdfSJdXQ==
\[\begin{tikzcd}
	{C^\flat} & {1\costar C^\flat} \\
	{\{1\}} & {[C,1]^\sharp}
	\arrow[from=2-1, to=2-2]
	\arrow[from=1-1, to=2-1]
	\arrow[from=1-1, to=1-2]
	\arrow["{\Fb h_0^{[C,1]}}", from=1-2, to=2-2]
\end{tikzcd}\]
induces by the one of theorem \ref{theo:formula between pullback of slice and tensor marked case}.
When nothing is specified, the morphism $C^\flat \to \Fb h_0^{[C,1]}$ will always corresponds to this square.


\p\sym{(fh@$\Fb h^C_{\cdot}$}\sym{(cpoint@$C_{\cdot/}$}
Let $C$ be an $\io$-category. We define the simplicial marked $\io$-category $C_{\cdot/}$ and the simplicial arrow of marked $\io$-categories
 $\Fb h^C_{\cdot}$ whose value on an integer $n$ is given by the following pullback 
% https://q.uiver.app/#q=WzAsNCxbMSwwLCIoQ157XFxzaGFycH0pXntbbisxXV5cXHNoYXJwfSJdLFsxLDEsIihDXlxcc2hhcnApXntbbl1eXFxzaGFycH1cXHRpbWVzIChDXlxcc2hhcnApXntcXHtuKzFcXH19Il0sWzAsMSwiKFxcTm9pdW4gQylfbl5cXGZsYXRcXHRpbWVzIENee1xcc2hhcnB9Il0sWzAsMCwiKENfe1xcY2RvdC99KV9uIl0sWzIsMV0sWzAsMV0sWzMsMiwiKFxcRmIgaF97XFxjZG90fSlfbiIsMl0sWzMsMF0sWzMsNCwiIiwwLHsibGV2ZWwiOjEsInN0eWxlIjp7Im5hbWUiOiJjb3JuZXIifX1dXQ==
\[\begin{tikzcd}
	{(C_{\cdot/})_n} & {(C^{\sharp})^{[n+1]^\sharp}} \\
	{(\Noiun C)_n^\flat\times C^{\sharp}} & {(C^\sharp)^{[n]^\sharp}\times (C^\sharp)^{\{n+1\}}}
	\arrow[""{name=0, anchor=center, inner sep=0}, from=2-1, to=2-2]
	\arrow[from=1-2, to=2-2]
	\arrow["{(\Fb h_{\cdot})_n}"', from=1-1, to=2-1]
	\arrow[from=1-1, to=1-2]
	\arrow["\lrcorner"{anchor=center, pos=0.125}, draw=none, from=1-1, to=0]
\end{tikzcd}\]
 and where the functoriality in $n$ is induced by the universal property of pullback.
Unfolding the definition, on all integer $n$, the canonical morphism $(C_{\cdot/})_n\to C^\sharp$ corresponds to the morphism 
$$ \coprod\limits_{x_0,...,x_n:C_0} \hom_C^\flat(x_0,...,x_n)\times \Fb h_{x_n}^C$$
and is then a left cartesian fibration according to theorem \ref{theo:left cart stable by colimit}.
 
 
\p 
\label{para:definition of integral de grot}
Let $E$ be an object of $\ouncat_{/\Noiun C}$ corresponding to an arrow $X \to\Noiun C$. The \wcnotion{Grothendieck construction}{grothendieck construction@Grothendieck construction} of $E$, is the object of $\ocatm_{/C^\sharp}$ defined by the formula
$$\int_CE:=\colim_n (X^\flat \times_{(\Noiun C)^\flat } \Fb h_{\cdot})_n.$$
As the Grothendieck construction is by definition a colimit of left cartesian fibrations, the theorem \ref{theo:left cart stable by colimit} implies that it is also a left cartesian fibration. The Grothendieck construction then defines a functor
$$\int_{C}: \ouncat_{/\Noiun C}\to \LCart(C^\sharp).$$
Unfolding the definition, if $E$ is a left fibration, $\int_C E$ is the colimit of a simplicial diagram whose value on $n$ is:
$$
\coprod\limits_{x_0,...,x_n:C_0}X(x_0)\times \hom_C^\flat(x_0,...,x_n)\times \Fb h_{x_n}^C$$

\begin{example}
\label{exe:of int}
Let $E$ be an object of $\Lfib(\Noiun [a,1])$ corresponding to a morphism $X\to \Noiun ([a,1])$. According to proposition \ref{prop:lfib and W 2}, this object corresponds to a morphism $X(0)\times a\to X(1)$. The arrow $\int_{[a,1]}E$ corresponds to the colimit of the following diagram:
% https://q.uiver.app/#q=WzAsMyxbMCwwLCJFKDApXlxcZmxhdFxcdGltZXNcXEZiIGhee1thLDFdfV97MH0iXSxbMiwwLCJFKDEpXlxcZmxhdCJdLFsxLDAsIkUoMCleXFxmbGF0XFx0aW1lcyBhXlxcZmxhdCJdLFsyLDBdLFsyLDFdXQ==
\[\begin{tikzcd}
	{E(0)^\flat\times\Fb h^{[a,1]}_{0}} & {E(0)^\flat\times a^\flat} & {E(1)^\flat}
	\arrow[from=1-2, to=1-1]
	\arrow[from=1-2, to=1-3]
\end{tikzcd}\]
The domain of this arrow is then the colimit of the following diagram:
% https://q.uiver.app/#q=WzAsMyxbMCwwLCJYKDApXlxcZmxhdFxcdGltZXNbYSwxXV5cXHNoYXJwX3swL30iXSxbMiwwLCJYKDEpXlxcZmxhdCJdLFsxLDAsIlgoMCleXFxmbGF0XFx0aW1lcyBhXlxcZmxhdCJdLFsyLDBdLFsyLDFdXQ==
\[\begin{tikzcd}
	{X(0)^\flat\times[a,1]^\sharp_{0/}} & {X(0)^\flat\times a^\flat} & {X(1)^\flat}
	\arrow[from=1-2, to=1-1]
	\arrow[from=1-2, to=1-3]
\end{tikzcd}\]
\end{example}

\begin{lemma}
\label{lemma:intpreserces initial}
The functor $\int_C:\ouncat_{/\Noiun C}\to \LCart(C^\sharp)$ preserves colimits. Moreover, it sends morphisms of $\J$ to equivalences. 
\end{lemma}
\begin{proof}
According to corollary \ref{cor:inclusion of lcatt into the slice preserves colimits}, it is sufficient to show that the composite 
$$\ouncat_{/\Noiun C}\xrightarrow{ \int_C} \LCart(C^\sharp)\xrightarrow{\dom}\ocatm$$
preserves colimits.

To this extend, we consider the functor
$$\alpha: \iPsh{\Theta\times \Delta}_{/\Noiun C}\to \iPsh{t\Theta\times \Delta}$$
sending an object $E$ of $\Lfib(\Noiun C)$ corresponding to a morphism $X\to (\Noiun C)$ to 
$X\times_{(\Noiun C)^\flat } C_{\cdot/}$, 
and the functor 
$$\beta:\iPsh{t\Theta\times \Delta}\to \ocatm$$
that is the left Kan extension of the functor $t\Theta\times \Delta\to t\Theta\to \mPsh{\Theta}$. As $\iPsh{\Theta\times \Delta}$ is locally cartesian closed, $\alpha$ preserves colimits.
The composite $\beta\circ\alpha$ then preserves colimits. Moreover, we have a commutative diagram
% https://q.uiver.app/#q=WzAsNCxbMSwwLCJcXG9jYXRtIl0sWzAsMCwiXFxpUHNoe1xcVGhldGFcXHRpbWVzIFxcRGVsdGF9X3svXFxOb2l1biBDfSJdLFswLDEsIlxcb3VuY2F0X3svXFxOb2l1biBDfSJdLFsxLDEsIlxcTENhcnQoQ15cXHNoYXJwKSJdLFsxLDAsIlxcYmV0YVxcY2lyY1xcYWxwaGEiXSxbMSwyLCJcXEZiIiwyXSxbMiwzLCJcXGludF9DIiwyXSxbMywwLCJcXGRvbSIsMl1d
\[\begin{tikzcd}
	{\iPsh{\Theta\times \Delta}_{/\Noiun C}} & \ocatm \\
	{\ouncat_{/\Noiun C}} & {\LCart(C^\sharp)}
	\arrow["\beta\circ\alpha", from=1-1, to=1-2]
	\arrow["\Fb"', from=1-1, to=2-1]
	\arrow["{\int_C}"', from=2-1, to=2-2]
	\arrow["\dom"', from=2-2, to=1-2]
\end{tikzcd}\]
According to proposition \ref{prop:if left fib the fib}, one then has to show that $\beta\circ\alpha$ sends any morphism of $\J$ to an equivalence to conclude. Indeed, it will implies that $\beta\circ \alpha$ lifts to a colimit preserving functor $$\Db(\beta\circ \alpha):\ouncat_{/\Noiun C}\to \ocatm,$$ and the previous square implies that this morphism is equivalent to $\dom \int_C$.

Suppose given two cartesian squares
% https://q.uiver.app/#q=WzAsNixbMCwxLCJcXGxhbmdsZSBhLCBcXHswXFx9XFxyYW5nbGUiXSxbMSwxLCJcXGxhbmdsZSBhLCBbbl1cXHJhbmdsZSJdLFsyLDEsIihcXE5vaXVuIEMpXlxcZmxhdCJdLFsyLDAsIiBDX3tcXGNkb3QvfSJdLFswLDAsIlgiXSxbMSwwLCJYJyJdLFswLDEsImYiLDJdLFszLDJdLFs0LDBdLFs1LDFdLFsxLDJdLFs1LDNdLFs0LDUsImciXSxbNCwxLCIiLDAseyJzdHlsZSI6eyJuYW1lIjoiY29ybmVyIn19XSxbNSwyLCIiLDAseyJzdHlsZSI6eyJuYW1lIjoiY29ybmVyIn19XV0=
\[\begin{tikzcd}
	X & {X'} & { C_{\cdot/}} \\
	{\langle a, \{0\}\rangle} & {\langle a, [n]\rangle} & {(\Noiun C)^\flat}
	\arrow["f"', from=2-1, to=2-2]
	\arrow[from=1-3, to=2-3]
	\arrow[from=1-1, to=2-1]
	\arrow[from=1-2, to=2-2]
	\arrow[from=2-2, to=2-3]
	\arrow[from=1-2, to=1-3]
	\arrow["g", from=1-1, to=1-2]
	\arrow["\lrcorner"{anchor=center, pos=0.125}, draw=none, from=1-1, to=2-2]
	\arrow["\lrcorner"{anchor=center, pos=0.125}, draw=none, from=1-2, to=2-3]
\end{tikzcd}\]
By currying, we see these objects as functors $t\Theta^{op}\to \iPsh{\Delta}$. The right vertical morphism is then pointwise a right fibration of $\iun$-categories fibered in $\infty$-groupoids, as it corresponds, for a fixed $a:t\Theta$ and $n:\Delta$, to the morphism of $\infty$-groupoid:
$$\coprod_{x_0,...,x_n:C_0}\Hom(a,\hom_C(x_0,...,x_n)^\flat)\times\Hom(a,C^\sharp_{x_n/})\to \coprod_{x_0,...,x_n:C_0}\Hom(a,\hom_C(x_0,...,x_n)^\flat).$$

As the morphism $f$ is pointwise initial, so is $g$. 
As $\beta$ sends pointwise initial morphisms to equivalence, this implies that $\beta\alpha (f):= \beta(g)$ is an equivalence. 

Suppose now given two cartesian squares
% https://q.uiver.app/#q=WzAsNixbMCwxLCJcXGxhbmdsZSBhLCAwXFxyYW5nbGUiXSxbMSwxLCJcXGxhbmdsZSBiLCAwXFxyYW5nbGUiXSxbMiwxLCIoXFxOb2l1biBDKV5cXGZsYXQiXSxbMiwwLCIgQ197XFxjZG90L30iXSxbMCwwLCJYIl0sWzEsMCwiWCciXSxbMCwxLCJcXGxhbmdsZSBmLDBcXHJhbmdsZSIsMl0sWzMsMl0sWzQsMF0sWzUsMV0sWzEsMl0sWzUsM10sWzQsNSwiZyJdLFs0LDEsIiIsMCx7InN0eWxlIjp7Im5hbWUiOiJjb3JuZXIifX1dLFs1LDIsIiIsMCx7InN0eWxlIjp7Im5hbWUiOiJjb3JuZXIifX1dXQ==
\[\begin{tikzcd}
	X & {X'} & { C_{\cdot/}} \\
	{\langle a, 0\rangle} & {\langle b, 0\rangle} & {(\Noiun C)^\flat}
	\arrow["{\langle f,0\rangle}"', from=2-1, to=2-2]
	\arrow[from=1-3, to=2-3]
	\arrow[from=1-1, to=2-1]
	\arrow[from=1-2, to=2-2]
	\arrow[from=2-2, to=2-3]
	\arrow[from=1-2, to=1-3]
	\arrow["g", from=1-1, to=1-2]
	\arrow["\lrcorner"{anchor=center, pos=0.125}, draw=none, from=1-1, to=2-2]
	\arrow["\lrcorner"{anchor=center, pos=0.125}, draw=none, from=1-2, to=2-3]
\end{tikzcd}\]
with $f\in \W$. By currying, we see these objects as functors $\Delta\to\iPsh{t\Theta}$. The right vertical morphism is then pointwise a right cartesian fibration. As the morphism $\langle f,0\rangle$ is pointwise in $\widehat{\Wm}$, so is $g$. The morphism $\colim_n g_n$ is then in $\widehat{\Wm}$ and $\beta\alpha (f):= \beta(g)$ is an equivalence.
\end{proof}




\p We will denote also by 
$$\int_C:\Lfib(\Noiun C)\to \LCart(C^\sharp)$$ 
the restriction of the Grothendieck construction. 
This will not cause any confusion as from now on we will only consider the 
Grothendieck construction of left fibration.
 The lemma \ref{lemma:intpreserces initial} then implies that this functor is colimit preserving, and it is then part of an adjunction \index[notation]{(partial@$\partial_C$}
\begin{equation}
\label{eq:underived GR constuction}
\begin{tikzcd}
	{\int_C:\Lfib(\Noiun C)} & { \LCart(C^\sharp):\partial_C}
	\arrow[""{name=0, anchor=center, inner sep=0}, shift left=2, from=1-1, to=1-2]
	\arrow[""{name=1, anchor=center, inner sep=0}, shift left=2, from=1-2, to=1-1]
	\arrow["\dashv"{anchor=center, rotate=-90}, draw=none, from=0, to=1]
\end{tikzcd}
\end{equation}

%
\begin{lemma}
\label{lemma:partial fiber}
Let $i:C^\sharp\to D^\sharp$ be a morphism. The natural transformation $$\partial_{C}\circ\Rb i^*\to \Rb (\Noiun{i})^*\circ\partial_D$$ is an equivalence.
\end{lemma}
\begin{proof}
As equivalences between left fibrations are detected on fibers, one can suppose that $C$ is the terminal $\io$-category. Let $c$ denote the object of $D$ corresponding to $i$.
Let $E$ be an object of $\Lfib(\Noiun1)$, corresponding to a morphism $A\to 1$. According to lemma \ref{lemma:intpreserces initial}, we then have equivalences
$$\begin{array}{rclr}
\Lb i_! \int_1 E &\sim & \Lb i_! ( A^\flat\times h_1^1)\\
&\sim & A^\flat\times \Fb h_c^D\\
	&=: &\int_D {\Noiun{i}}_!E\\
	&\sim &\int_D \Lb(\Noiun{i})_!E& (\ref{lemma:intpreserces initial})\\
\end{array}$$
The canonical morphism $\Lb i_!\circ \int_1 \to \int_D \circ \Lb{(\Noiun{i})}_!$ is then an equivalence, which implies by adjunction that $\partial_{1}\circ\Rb_i^*\to \Rb (\Noiun{i})^*\circ\partial_D$ also is.
\end{proof}
\p Let $C$ be an $\io$-category and $c$ an object of $C^\sharp$.
We define $(\Noiun C)_{/c}$ as the simplicial object in $\ocat$ whose value on $(a,n)$ fits in the cocartesian square 
% https://q.uiver.app/#q=WzAsNCxbMCwwLCIoKFxcTm9pdW4gQylfey9jfSlfeyhhLG4pfSJdLFsxLDAsIihcXE5vaXVuIEMpX3soYSxuKzEpfSJdLFsxLDEsIihcXE5vaXVuIEMpX3soYSxcXHtuKzFcXH0pfSJdLFswLDEsIlxce2NcXH0iXSxbMywyXSxbMCwzXSxbMSwyXSxbMCwxXSxbMCwyLCIiLDEseyJzdHlsZSI6eyJuYW1lIjoiY29ybmVyIn19XV0=
\[\begin{tikzcd}
	{((\Noiun C)_{/c})_{(a,n)}} & {(\Noiun C)_{(a,n+1)}} \\
	{\{c\}} & {(\Noiun C)_{(a,\{n+1\})}}
	\arrow[from=2-1, to=2-2]
	\arrow[from=1-1, to=2-1]
	\arrow[from=1-2, to=2-2]
	\arrow[from=1-1, to=1-2]
	\arrow["\lrcorner"{anchor=center, pos=0.125, rotate=45}, draw=none, from=1-1, to=2-2]
\end{tikzcd}\]
Unfolding the definition, $(\Noiun C)_{/c}$ is the simplicial diagram whose value on $n$ is
$$\coprod_{x_0,...,x_n}\hom_C(x_0,...,x_n,c)$$
\begin{lemma}
\label{lemma:fiber of F h .}
There is an equivalence
$$((\Noiun C)_{/c})^\flat \sim c^* \Fb h_{\cdot}.$$
\end{lemma}
\begin{proof}
A morphism $\langle a,n\rangle\to (c^* \Fb h_{\cdot})^\natural$ is the data of a commutative square
% https://q.uiver.app/#q=WzAsNCxbMSwwLCJhXlxcZmxhdFxcb3RpbWVzW24rMV1eXFxzaGFycCJdLFsxLDEsIkNeXFxzaGFycCJdLFswLDAsIlxcY29wcm9kX3trXFxsZXEgbisxfSBhXlxcZmxhdFxcb3RpbWVzXFx7a1xcfSJdLFswLDEsIlxcY29wcm9kX3trXFxsZXEgbisxfSBcXHtrXFx9Il0sWzAsMV0sWzMsMV0sWzIsM10sWzIsMF1d
\[\begin{tikzcd}
	{\coprod_{k\leq n+1} a^\flat\otimes\{k\}} & {a^\flat\otimes[n+1]^\sharp} \\
	{\coprod_{k\leq n+1} \{k\}} & {C^\sharp}
	\arrow[from=1-2, to=2-2]
	\arrow[from=2-1, to=2-2]
	\arrow[from=1-1, to=2-1]
	\arrow[from=1-1, to=1-2]
\end{tikzcd}\]
which is, according to proposition \ref{prop:crushing of Gray tensor is identitye marked case}, equivalent to a morphism
$$[a,n+1]^\sharp\to C^\sharp$$
and so to a morphism $\langle a,n\rangle\to (\Noiun C)_{c/}$. As $c^*\Fb h_{\cdot}$ has a trivial marking, this shows the desired equivalence.
\end{proof}
\begin{lemma}
\label{lemma:int fiber 1}
Let $p:X\to \Noiun C$ be a left fibration, and $c$ an object of $C$. 
The canonical morphism 
$$X(c)\to \colim_n (X\times_{\Noiun C} (\Noiun C)_{/c})_n$$
is an equivalence.
\end{lemma}
\begin{proof}
We will show a slightly stronger statement, which is that the morphism
$$X(c)\to \colim_n (X\times_{(\Noiun C)} (\Noiun C)_{/c})_n$$
is an equivalence when the colimit is taken in $\infty$-presheaves on $\Theta$.
As the colimit in presheaves commutes with evaluation, one has to show that for any globular sum $a$, the canonical morphism of $\infty$-groupoids
$$\Hom(a,X(c))\to \colim_n (\Hom(a,X_n)\times_{\Hom(a,(\Noiun C)_n)}\Hom(a, (\Noiun C)_{/c})_n)$$
is an equivalence. Remark that the simplicial $\infty$-groupoid $ \Hom(a, ((\Noiun C)_{/c})_\bullet)$ is equivalent to the simplicial $\infty$-groupoid $(\Hom(a,\Noiun C)_\bullet)_{/c}$.
If we denote also by $\Hom(a,X(c))$ the constant simplicial $\infty$-groupoid $n\mapsto \Hom(a,X(c))$, we have a cartesian square
% https://q.uiver.app/#q=WzAsNixbMiwwLCIgXFxIb20oYSxYX1xcYnVsbGV0KSJdLFsyLDEsIlxcSG9tKGEsIChcXE5vaXVuIEMpX1xcYnVsbGV0KSJdLFswLDEsIlxce2NcXH0iXSxbMSwxLCJcXEhvbShhLCAoXFxOb2l1biBDKV9cXGJ1bGxldClfey9jfSJdLFswLDAsIlxcSG9tKGEsWChjKSkiXSxbMSwwLCIgXFxIb20oYSxYX1xcYnVsbGV0KVxcdGltZXNfe1xcSG9tKGEsKFxcTm9pdW4gQylfXFxidWxsZXQpfVxcSG9tKGEsIChcXE5vaXVuIEMpX1xcYnVsbGV0KV97L2N9Il0sWzAsMV0sWzQsNV0sWzUsM10sWzQsMl0sWzIsM10sWzMsMV0sWzUsMF0sWzUsMTEsIiIsMix7ImxldmVsIjoxLCJzdHlsZSI6eyJuYW1lIjoiY29ybmVyIn19XSxbNCwxMCwiIiwyLHsibGV2ZWwiOjEsInN0eWxlIjp7Im5hbWUiOiJjb3JuZXIifX1dXQ==
\[\begin{tikzcd}[column sep =0.3cm]
	{\Hom(a,X(c))} & { \Hom(a,X_\bullet)\times_{\Hom(a,(\Noiun C)_\bullet)}\Hom(a, (\Noiun C)_\bullet)_{/c}} & { \Hom(a,X_\bullet)} \\
	{\{c\}} & {\Hom(a, (\Noiun C)_\bullet)_{/c}} & {\Hom(a, (\Noiun C)_\bullet)}
	\arrow[from=1-3, to=2-3]
	\arrow[from=1-1, to=1-2]
	\arrow[from=1-2, to=2-2]
	\arrow[from=1-1, to=2-1]
	\arrow[""{name=0, anchor=center, inner sep=0}, from=2-1, to=2-2]
	\arrow[""{name=1, anchor=center, inner sep=0}, from=2-2, to=2-3]
	\arrow[from=1-2, to=1-3]
	\arrow["\lrcorner"{anchor=center, pos=0.125}, draw=none, from=1-2, to=1]
	\arrow["\lrcorner"{anchor=center, pos=0.125}, draw=none, from=1-1, to=0]
\end{tikzcd}\]
Moreover, the left vertical morphism is a left fibration of $\iun$-category fibered in $\infty$-groupoid.
As pullbacks along left fibrations preserve final morphisms,
the morphism
$$\Hom(a,X(c))\to \Hom(a,X_\bullet)\times_{\Hom(a,(\Noiun C)_\bullet)}\Hom(a, (\Noiun C)_\bullet)_{/c}$$
is final. Taking the colimit, this implies the result.
\end{proof}


\begin{lemma}
\label{lemma:int fiber 2}
Let $i:C^\sharp\to D^\sharp$ be a morphism. The natural transformation 
$$\int_D\circ \Rb(\Noiun i)^*\to \Rb i^* \circ\int_C$$
is an equivalence.
\end{lemma}
\begin{proof}
 As equivalences between left cartesian fibrations are detected on fibers, one can suppose that $C$ is the terminal $\io$-category. Let $c$ denote the object of $D$ corresponding to $i$ and let $E$ be an object of $\Lfib(\Noiun C)$, corresponding to a left fibration $X\to \Noiun C$. 
 By construction, $\int_CE$ is a colimit of left cartesian fibrations. However, as proposition \ref{prop:fiber preserves colimits} states that $\Rb i^*$ commutes with colimit, we have 
$$\begin{array}{rclc}
\Rb i^*\int_CE&\sim &\colim_n X_n^\flat\times_{(\Noiun C)_n^\flat}\Rb i^*\Fb h^C_{\cdot}\\
&\sim &\colim_{n}(X\times_{\Noiun C} (\Noiun C)_{/c})^\flat_n&(\ref{lemma:fiber of F h .})
\end{array}$$
Moreover, remark that $\int_1 \Rb (\Noiun i)^* E$ is equivalent to $X(c)$, and the canonical morphism
$\int_D \Rb(\Noiun i)^*E\to \Rb i^* \int_CE$ is then the image by $(\uvar)^\flat$ of the equivalence given by lemma \ref{lemma:int fiber 1}.
\end{proof}

\begin{prop}
\label{prop: derived int and partial are natural}
The functors $\int_C$ and $\partial_C$ are natural in $C:\ocat^{op}$.
\end{prop}
\begin{proof}
We denote by $\Arr^{fib}(\ocatm)$ (resp. $\Arr^{fib}(\ouncat)$) the full sub $\iun$-category of $\Arr(\ocatm)$ (resp. $\Arr(\ouncat)$) whose objects are $\U$-small left cartesian fibrations (resp. $\U$-small left fibrations). 
We also set $\ocat\times_{\ocatm}\Arr^{fib}(\ocatm)$ and $\ocat\times_{\ouncat}\Arr^{fib}(\ouncat)$ as the pullbacks:
% q.uiver.app/#q=WzAsOCxbMCwwLCJcXG9jYXRcXHRpbWVzX3tcXG9jYXRtfVxcQXJyXntmaWJ9KFxcb2NhdG0pIl0sWzAsMSwiXFxvY2F0Il0sWzEsMCwiXFxBcnJee2ZpYn0oXFxvY2F0bSkiXSxbMSwxLCJcXG9jYXRtIl0sWzAsMiwiXFxvY2F0XFx0aW1lc197XFxvdW5jYXR9XFxBcnJee2ZpYn0oXFxvdW5jYXQpIl0sWzEsMiwiXFxBcnJee2ZpYn0oXFxvdW5jYXQpIl0sWzEsMywiXFxvdW5jYXQiXSxbMCwzLCJcXG9jYXQiXSxbMSwzLCIoXFx1dmFyKV57XFxzaGFycH0iLDJdLFsyLDMsIlxcY29kb20iXSxbMCwxXSxbMCwyXSxbNyw2LCJcXE5vaXVuIiwyXSxbNCw3XSxbNSw2LCJcXGNvZG9tIl0sWzQsNV0sWzAsOCwiIiwwLHsibGV2ZWwiOjEsInN0eWxlIjp7Im5hbWUiOiJjb3JuZXIifX1dLFs0LDEyLCIiLDAseyJsZXZlbCI6MSwic3R5bGUiOnsibmFtZSI6ImNvcm5lciJ9fV1d
\[\begin{tikzcd}
	{\ocat\times_{\ocatm}\Arr^{fib}(\ocatm)} & {\Arr^{fib}(\ocatm)} \\
	\ocat & \ocatm \\
	{\ocat\times_{\ouncat}\Arr^{fib}(\ouncat)} & {\Arr^{fib}(\ouncat)} \\
	\ocat & \ouncat
	\arrow[""{name=0, anchor=center, inner sep=0}, "{(\uvar)^{\sharp}}"', from=2-1, to=2-2]
	\arrow["\codom", from=1-2, to=2-2]
	\arrow[from=1-1, to=2-1]
	\arrow[from=1-1, to=1-2]
	\arrow[""{name=1, anchor=center, inner sep=0}, "\Noiun"', from=4-1, to=4-2]
	\arrow[from=3-1, to=4-1]
	\arrow["\codom", from=3-2, to=4-2]
	\arrow[from=3-1, to=3-2]
	\arrow["\lrcorner"{anchor=center, pos=0.125}, draw=none, from=1-1, to=0]
	\arrow["\lrcorner"{anchor=center, pos=0.125}, draw=none, from=3-1, to=1]
\end{tikzcd}\]
The two left vertical morphism inherit from the right vertical morphisms of a structure of Grothendieck fibrations fibered in $\iun$-categories, where cartesian liftings are given by morphisms between arrows corresponding to cartesian squares.


As the assignation $C\mapsto \Fb h_{\cdot}^C$ can be promoted in a functor $\ocat\to \Arr(\Fun(\Delta,\ocatm))$
the functors $\int_C$ and $\partial_C$ are the restrictions of two functors $\int$ and $\partial$ fitting in commutative triangles:
% https://q.uiver.app/#q=WzAsNixbMSwwLCJcXG9jYXRcXHRpbWVzX3tcXG9jYXRtfVxcQXJyXntmaWJ9KFxcb2NhdG0pIl0sWzEsMSwiXFxvY2F0Il0sWzEsMiwiXFxvY2F0XFx0aW1lc197XFxvdW5jYXR9XFxBcnJee2ZpYn0oXFxvdW5jYXQpIl0sWzEsMywiXFxvY2F0Il0sWzAsMSwiXFxvY2F0XFx0aW1lc197XFxvdW5jYXR9XFxBcnJee2ZpYn0oXFxvdW5jYXQpIl0sWzAsMywiXFxvY2F0XFx0aW1lc197XFxvY2F0bX1cXEFycl57ZmlifShcXG9jYXRtKSJdLFswLDFdLFsyLDNdLFs0LDFdLFs0LDAsIlxcaW50Il0sWzUsMiwiXFxwYXJ0aWFsIl0sWzUsM11d
\[\begin{tikzcd}
	& {\ocat\times_{\ocatm}\Arr^{fib}(\ocatm)} \\
	{\ocat\times_{\ouncat}\Arr^{fib}(\ouncat)} & \ocat \\
	& {\ocat\times_{\ouncat}\Arr^{fib}(\ouncat)} \\
	{\ocat\times_{\ocatm}\Arr^{fib}(\ocatm)} & \ocat
	\arrow[from=1-2, to=2-2]
	\arrow[from=3-2, to=4-2]
	\arrow[from=2-1, to=2-2]
	\arrow["\int", from=2-1, to=1-2]
	\arrow["\partial", from=4-1, to=3-2]
	\arrow[from=4-1, to=4-2]
\end{tikzcd}\]
Lemmas \ref{lemma:partial fiber} and \ref{lemma:int fiber 2} imply that these two functors preserve cartesian arrows, and the Grothendieck deconstruction then implies the desired result.
\end{proof}

\begin{theorem}
\label{theo:gr construction}
For any $\io$-category $C$, the adjunction 
$$\begin{tikzcd}
	{\int_C:\Lfib(\Noiun C)} & { \LCart(C^\sharp):\partial_C}
	\arrow[""{name=0, anchor=center, inner sep=0}, shift left=2, from=1-1, to=1-2]
	\arrow[""{name=1, anchor=center, inner sep=0}, shift left=2, from=1-2, to=1-1]
	\arrow["\dashv"{anchor=center, rotate=-90}, draw=none, from=0, to=1]
\end{tikzcd}$$
defined in \eqref{eq:underived GR constuction}, is an adjoint equivalence.
\end{theorem}
\begin{proof}
As equivalences between left fibrations and between left cartesian fibrations are detected on fibers, and as the two functors are natural in $C$, it is sufficient to show the result for $C$ being the terminal $\io$-category. In this case remark that $\Lfib(\Noiun1)\sim \LCart(1)$ and that both $\int_1$ and $\partial_1$ are the identities. 
\end{proof}

\begin{cor}
\label{cor:fib over a colimit2}
Let $F:I\to \ocatm$ be a $\Wcard$-small diagram. The canonical functor
$$\LCartc(\colim_IF) \to \lim_I \LCartc(F)$$
is an equivalence.
\end{cor}
\begin{proof}
This functor fits in an adjunction:
% q.uiver.app/#q=WzAsMixbMSwwLCJcXExDYXJ0YyhcXGNvbGltX0kgRikiXSxbMCwwLCJcXGNvbGltX0k6XFxsaW1fSVxcTENhcnRjKEYpIl0sWzAsMSwiIiwwLHsib2Zmc2V0IjotMn1dLFsxLDAsIiIsMCx7Im9mZnNldCI6LTJ9XSxbMywyLCIiLDAseyJsZXZlbCI6MSwic3R5bGUiOnsibmFtZSI6ImFkanVuY3Rpb24ifX1dXQ==
\[\begin{tikzcd}
	{\colim_I:\lim_I\LCartc(F)} & {\LCartc(\colim_I F)}
	\arrow[""{name=0, anchor=center, inner sep=0}, shift left=2, from=1-2, to=1-1]
	\arrow[""{name=1, anchor=center, inner sep=0}, shift left=2, from=1-1, to=1-2]
	\arrow["\dashv"{anchor=center, rotate=-90}, draw=none, from=1, to=0]
\end{tikzcd}\]
The corollary \ref{cor:fib over a colimit} implies that the counit of this adjunction is an equivalence. To conclude, we have to show that the right adjoint is essentially surjective.
By definition, the morphism $\tau_0\LCart(I^\sharp)\to \tau_0\LCartc(I)$ is an equivalence.
According to theorem \ref{theo:gr construction}, on the $\infty$-groupoid of objects, the right adjoint corresponds to the equivalence
$$\tau_0\Lfib(\Noiun\colim_I F^\sharp) \to \lim_I \tau_0\Lfib(\Noiun F^\sharp)$$
given in proposition \ref{prop:Lfib commue with colimit}.
\end{proof}

%
\begin{cor}
\label{cor:antecedant of slice}
Let $C$ be an $\io$-category and $c$ be an object of $c$. The left fibration $\partial_C \Fb h_c$ is the morphism of simplicial objects:
% q.uiver.app/#q=WzAsOCxbMywwLCJcXGNvcHJvZF97eF8wOkNfMH1cXGhvbV9DKHkseF8wKSJdLFszLDEsIlxcY29wcm9kX3t4XzA6Q18wfTEiXSxbMiwxLCJcXGNvcHJvZF97eF8wLHhfMTpDXzB9XFxob21fQyh4XzAseF8xKSJdLFsxLDEsIlxcY29wcm9kX3t4XzAseF8xLHhfMjpDXzB9XFxob21fQyh4XzAseF8xLHhfMikiXSxbMCwxLCJcXGNkb3RzIl0sWzIsMCwiXFxjb3Byb2Rfe3hfMCx4XzE6Q18wfVxcaG9tX0MoeSx4XzAseF8xKSJdLFsxLDAsIlxcY29wcm9kX3t4XzAseF8xLHhfMjpDXzB9XFxob21fQyh5LHhfMCx4XzEseF8yKSJdLFswLDAsIlxcY2RvdHMiXSxbMywyLCIiLDIseyJvZmZzZXQiOjR9XSxbMywyLCIiLDAseyJvZmZzZXQiOi00fV0sWzMsMl0sWzIsMywiIiwxLHsib2Zmc2V0IjotMn1dLFsyLDMsIiIsMSx7Im9mZnNldCI6Mn1dLFsyLDEsIiIsMSx7Im9mZnNldCI6LTJ9XSxbMiwxLCIiLDEseyJvZmZzZXQiOjJ9XSxbMSwyXSxbNSwyXSxbMCwxXSxbNSwwLCIiLDEseyJvZmZzZXQiOi0yfV0sWzAsNV0sWzUsMCwiIiwxLHsib2Zmc2V0IjoyfV0sWzYsNSwiIiwxLHsib2Zmc2V0Ijo0fV0sWzYsNV0sWzYsNSwiIiwxLHsib2Zmc2V0IjotNH1dLFs1LDYsIiIsMSx7Im9mZnNldCI6Mn1dLFs1LDYsIiIsMSx7Im9mZnNldCI6LTJ9XSxbNiwzXV0=
\[\begin{tikzcd}[column sep =0.5cm]
	\cdots & {\coprod_{x_0,x_1,x_2:C_0}\hom_C(y,x_0,x_1,x_2)} & {\coprod_{x_0,x_1:C_0}\hom_C(y,x_0,x_1)} & {\coprod_{x_0:C_0}\hom_C(y,x_0)} \\
	\cdots & {\coprod_{x_0,x_1,x_2:C_0}\hom_C(x_0,x_1,x_2)} & {\coprod_{x_0,x_1:C_0}\hom_C(x_0,x_1)} & {\coprod_{x_0:C_0}1}
	\arrow[shift right=4, from=2-2, to=2-3]
	\arrow[shift left=4, from=2-2, to=2-3]
	\arrow[from=2-2, to=2-3]
	\arrow[shift left=2, from=2-3, to=2-2]
	\arrow[shift right=2, from=2-3, to=2-2]
	\arrow[shift left=2, from=2-3, to=2-4]
	\arrow[shift right=2, from=2-3, to=2-4]
	\arrow[from=2-4, to=2-3]
	\arrow[from=1-3, to=2-3]
	\arrow[from=1-4, to=2-4]
	\arrow[shift left=2, from=1-3, to=1-4]
	\arrow[from=1-4, to=1-3]
	\arrow[shift right=2, from=1-3, to=1-4]
	\arrow[shift right=4, from=1-2, to=1-3]
	\arrow[from=1-2, to=1-3]
	\arrow[shift left=4, from=1-2, to=1-3]
	\arrow[shift right=2, from=1-3, to=1-2]
	\arrow[shift left=2, from=1-3, to=1-2]
	\arrow[from=1-2, to=2-2]
\end{tikzcd}\]
\end{cor}
\begin{proof}
We denote by $E:=X\to \Noiun C$ this left fibration.
According to theorem \ref{theo:gr construction}, we can equivalently show that the Grothendieck integral of $E$ is the morphism $C^{\sharp}_{c/}\to C$. 
Remark that we have by construction a family of cartesian squares
% https://q.uiver.app/#q=WzAsNixbMCwwLCJYX25cXHRpbWVzX3soXFxOb2l1biBDKV9ufSAoQ197XFxjZG90L30pX24iXSxbMCwxLCJcXHtjXFx9XFx0aW1lcyAoXFxOb2l1biBDKV9uXFx0aW1lcyBDXlxcc2hhcnAiXSxbMSwxLCJDXlxcc2hhcnAgXFx0aW1lcyAoQ15cXHNoYXJwKV57W25dXlxcc2hhcnB9XFx0aW1lcyBDXlxcc2hhcnAiXSxbMSwwLCIoQ15cXHNoYXJwKV57WzErbisxXV5cXHNoYXJwfSJdLFsyLDAsIihDXlxcc2hhcnApXntbMV1eXFxzaGFycH0iXSxbMiwxLCJDXlxcc2hhcnBcXHRpbWVzIENeXFxzaGFycCJdLFswLDFdLFswLDNdLFsxLDJdLFszLDJdLFszLDQsIihDXntcXHNoYXJwfSlee2hfbn0iXSxbMiw1XSxbNCw1XSxbMyw1LCIiLDAseyJzdHlsZSI6eyJuYW1lIjoiY29ybmVyIn19XSxbMCw4LCIiLDEseyJsZXZlbCI6MSwic3R5bGUiOnsibmFtZSI6ImNvcm5lciJ9fV1d
\[\begin{tikzcd}
	{X_n\times_{(\Noiun C)_n} (C_{\cdot/})_n} & {(C^\sharp)^{[1+n+1]^\sharp}} & {(C^\sharp)^{[1]^\sharp}} \\
	{\{c\}\times (\Noiun C)_n\times C^\sharp} & {C^\sharp \times (C^\sharp)^{[n]^\sharp}\times C^\sharp} & {C^\sharp\times C^\sharp}
	\arrow[from=1-1, to=2-1]
	\arrow[from=1-1, to=1-2]
	\arrow[""{name=0, anchor=center, inner sep=0}, from=2-1, to=2-2]
	\arrow[from=1-2, to=2-2]
	\arrow["{(C^{\sharp})^{h_n}}", from=1-2, to=1-3]
	\arrow[from=2-2, to=2-3]
	\arrow[from=1-3, to=2-3]
	\arrow["\lrcorner"{anchor=center, pos=0.125}, draw=none, from=1-2, to=2-3]
	\arrow["\lrcorner"{anchor=center, pos=0.125}, draw=none, from=1-1, to=0]
\end{tikzcd}\]
natural in $n$, where $h_n$ is the simplicial morphism preserving the extremal points. The outer square factors in two cartesian squares:
% https://q.uiver.app/#q=WzAsNixbMCwwLCJYX25cXHRpbWVzX3soXFxOb2l1biBDKV9ufSAoQ197XFxjZG90L30pX24iXSxbMCwxLCJcXHtjXFx9XFx0aW1lcyAoXFxOb2l1biBDKV9uXFx0aW1lcyBDXlxcc2hhcnAiXSxbMSwxLCJcXHtjXFx9XFx0aW1lcyAgQ15cXHNoYXJwIl0sWzEsMCwiQ157XFxzaGFycH1fe2MvfSJdLFsyLDAsIihDXlxcc2hhcnApXntbMV1eXFxzaGFycH0iXSxbMiwxLCJDXlxcc2hhcnBcXHRpbWVzIENeXFxzaGFycCJdLFswLDFdLFswLDNdLFsxLDJdLFszLDJdLFswLDIsIiIsMSx7InN0eWxlIjp7Im5hbWUiOiJjb3JuZXIifX1dLFsyLDVdLFs0LDVdLFszLDRdLFszLDUsIiIsMCx7InN0eWxlIjp7Im5hbWUiOiJjb3JuZXIifX1dLFswLDIsIiIsMCx7InN0eWxlIjp7Im5hbWUiOiJjb3JuZXIifX1dXQ==
\[\begin{tikzcd}
	{X_n\times_{(\Noiun C)_n} (C_{\cdot/})_n} & {C^{\sharp}_{c/}} & {(C^\sharp)^{[1]^\sharp}} \\
	{\{c\}\times (\Noiun C)_n\times C^\sharp} & {\{c\}\times C^\sharp} & {C^\sharp\times C^\sharp}
	\arrow[from=1-1, to=2-1]
	\arrow[from=1-1, to=1-2]
	\arrow[from=2-1, to=2-2]
	\arrow[from=1-2, to=2-2]
	\arrow["\lrcorner"{anchor=center, pos=0.125}, draw=none, from=1-1, to=2-2]
	\arrow[from=2-2, to=2-3]
	\arrow[from=1-3, to=2-3]
	\arrow[from=1-2, to=1-3]
	\arrow["\lrcorner"{anchor=center, pos=0.125}, draw=none, from=1-2, to=2-3]
	\arrow["\lrcorner"{anchor=center, pos=0.125}, draw=none, from=1-1, to=2-2]
\end{tikzcd}\]
This provides a canonical morphism 
$$\int_{C} E := \colim_n ( X_n\times_{(\Noiun C)_n} (\Fb h_{\cdot})_n)\to \Fb h_c^C$$
To conclude, one has to show that it is an equivalence, and for this, to check that this is the case on fibers, where it directly follows from the naturality of the integral given in proposition \ref{prop: derived int and partial are natural}.
\end{proof}

\begin{cor}
\label{cor:explicit partial}
Let $E$ be an object of $\ocat_{/[b,1]^\sharp}$ corresponding to a morphism $p:X\to [b,1]^\sharp$. 
Consider the induced cartesian squares:
% https://q.uiver.app/#q=WzAsOCxbMSwxLCIgWF97MH0iXSxbMCwwLCIgWF97MH1cXHRpbWVzIGJeXFxmbGF0Il0sWzIsMCwiIFhfey8xfSJdLFszLDEsIlgiXSxbMywzLCJbYiwxXV5cXHNoYXJwIl0sWzIsMiwiW2IsMV1eXFxzaGFycF97LzF9Il0sWzEsMywiXFx7MFxcfSJdLFswLDIsImJeXFxmbGF0Il0sWzEsMF0sWzEsMiwiZyJdLFsyLDMsImYiXSxbNSw0XSxbNiw0XSxbNyw1XSxbNyw2XSxbMiw1XSxbMyw0XSxbMSw3XSxbMCwzXSxbMCw2XV0=
\[\begin{tikzcd}
	{ X_{0}\times b^\flat} && { X_{/1}} \\
	& { X_{0}} && X \\
	{b^\flat} && {[b,1]^\sharp_{/1}} \\
	& {\{0\}} && {[b,1]^\sharp}
	\arrow[from=1-1, to=2-2]
	\arrow["g", from=1-1, to=1-3]
	\arrow["f", from=1-3, to=2-4]
	\arrow[from=3-3, to=4-4]
	\arrow[from=4-2, to=4-4]
	\arrow[from=3-1, to=3-3]
	\arrow[from=3-1, to=4-2]
	\arrow[from=1-3, to=3-3]
	\arrow[from=2-4, to=4-4]
	\arrow[from=1-1, to=3-1]
	\arrow[from=2-2, to=2-4]
	\arrow[from=2-2, to=4-2]
\end{tikzcd}\]
The span associated to $\partial_{[b,1]}\Fb E$ via the equivalence of proposition \ref{prop:lfib and W 2} is 
\begin{equation}
\label{eq:cor:explicit parital}
\bot X_{0}\leftarrow(\bot X_{0})\times b\xrightarrow{\bot g} \bot X_{/1}.
\end{equation}
\end{cor}
\begin{proof}
We denote $\tilde{X}\to [b,1]^\sharp$ the morphism associated to $\Fb E$. As, 
As $[b,1]^\sharp_{/1}\to [b,1]^\sharp$ and $\{0\}\to [b,1]^\sharp$ are right cartesian fibrations, they are smooth, and the canonical morphisms
$$X_{/1}\to \tilde{X}_{/1}~~~~~~~\mbox{ and }~~~~~~~X_{0}\to \tilde{X}_{0}$$
are initial.
As $\bot$ sends initial morphisms to equivalences, the induced morphisms
$$\bot X_{/1}\to \bot \tilde{X}_{/1}~~~~~~~\mbox{ and }~~~~~~~\bot X_{0}\to \bot \tilde{X}_{0}$$
are equivalences. We can then suppose that $E$ corresponds to a left cartesian fibration.


As $\{1\} \to [b,1]^{\sharp}$ is a right Gray deformation retract, so is the inclusion $X_1\to X_{/1}$ according to proposition \ref{prop:left Gray transfomration stable under pullback along cartesian fibration}. The right Gray deformation retract structure induces a diagram:
% https://q.uiver.app/#q=WzAsNSxbMiwxLCJYX3svMX0iXSxbMSwxLCJYX3svMX1cXG90aW1lc1sxXV5cXHNoYXJwIl0sWzAsMCwiWF97LzF9XFxvdGltZXNcXHswXFx9Il0sWzEsMiwiWF8xXFxvdGltZXNcXHsxXFx9Il0sWzAsMiwiWF97LzF9XFxvdGltZXNcXHsxXFx9Il0sWzIsMCwiaWQiLDAseyJjdXJ2ZSI6LTN9XSxbNCwxXSxbMywwXSxbNCwzLCJyIiwyXSxbMiwxXSxbMSwwLCJcXHBoaSIsMV1d
\[\begin{tikzcd}
	{X_{/1}\otimes\{0\}} \\
	& {X_{/1}\otimes[1]^\sharp} & {X_{/1}} \\
	{X_{/1}\otimes\{1\}} & {X_1\otimes\{1\}}
	\arrow["id", curve={height=-18pt}, from=1-1, to=2-3]
	\arrow[from=3-1, to=2-2]
	\arrow[from=3-2, to=2-3]
	\arrow["r"', from=3-1, to=3-2]
	\arrow[from=1-1, to=2-2]
	\arrow["\phi"{description}, from=2-2, to=2-3]
\end{tikzcd}\]
By post composing with $g:X_0\otimes b^\flat \to X_{/1}$ and post composing $f:X_{/1}\to X$, we get a diagram:
% https://q.uiver.app/#q=WzAsNixbMSwxLCIoIFhfMFxcdGltZXMgYl5cXGZsYXQpXFxvdGltZXNbMV1eXFxzaGFycCJdLFswLDAsIiggWF8wXFx0aW1lcyBiXlxcZmxhdClcXG90aW1lc1xcezBcXH0iXSxbMCwyLCIoIFhfMFxcdGltZXMgYl5cXGZsYXQpXFxvdGltZXNcXHsxXFx9Il0sWzEsMiwiWF8xXFxvdGltZXNcXHsxXFx9Il0sWzIsMSwiWCJdLFsxLDAsIlhfMCJdLFsxLDBdLFsyLDBdLFswLDRdLFsxLDVdLFsyLDMsInJnIiwyXSxbMyw0XSxbNSw0XV0=
\[\begin{tikzcd}
	{( X_0\times b^\flat)\otimes\{0\}} & {X_0} \\
	& {( X_0\times b^\flat)\otimes[1]^\sharp} & X \\
	{( X_0\times b^\flat)\otimes\{1\}} & {X_1\otimes\{1\}}
	\arrow[from=1-1, to=2-2]
	\arrow[from=3-1, to=2-2]
	\arrow[from=2-2, to=2-3]
	\arrow[from=1-1, to=1-2]
	\arrow["rg"', from=3-1, to=3-2]
	\arrow[from=3-2, to=2-3]
	\arrow[from=1-2, to=2-3]
\end{tikzcd}\]
Remark furthermore that the following diagram:
% https://q.uiver.app/#q=WzAsNSxbMiwwLCJYXzBcXHRpbWVzIFxcRmIgaF57W2IsMV19X3swL30iXSxbMiwxLCJbYiwxXV5cXHNoYXJwIl0sWzAsMCwiKCBYXzBcXHRpbWVzIGJeXFxmbGF0KVxcb3RpbWVzXFx7MFxcfSJdLFswLDEsIiggWF8wXFx0aW1lcyBiXlxcZmxhdClcXG90aW1lc1sxXV5cXHNoYXJwIl0sWzEsMCwiWF8wXFx0aW1lc1xcezBcXH0iXSxbMywwLCJsIiwxLHsic3R5bGUiOnsiYm9keSI6eyJuYW1lIjoiZGFzaGVkIn19fV0sWzIsM10sWzAsMV0sWzMsMV0sWzIsNF0sWzQsMF1d
\[\begin{tikzcd}
	{( X_0\times b^\flat)\otimes\{0\}} & {X_0\times\{0\}} & {X_0\times \Fb h^{[b,1]}_{0/}} \\
	{( X_0\times b^\flat)\otimes[1]^\sharp} && {[b,1]^\sharp}
	\arrow["l"{description}, dashed, from=2-1, to=1-3]
	\arrow[from=1-1, to=2-1]
	\arrow[from=1-3, to=2-3]
	\arrow[from=2-1, to=2-3]
	\arrow[from=1-1, to=1-2]
	\arrow[from=1-2, to=1-3]
\end{tikzcd}\]
admits a lift $l$. Indeed, the left vertical morphism is initial, and the right vertical one is a left cartesian fibration. 
All put together, we get a diagram 
% https://q.uiver.app/#q=WzAsNCxbMSwwLCJYXzBcXHRpbWVzIFxcRmIgaF57W2IsMV19X3swL30iXSxbMSwxLCJFIl0sWzAsMCwiWF8wXFx0aW1lcyBiXlxcZmxhdCJdLFswLDEsIlhfezF9Il0sWzAsMV0sWzMsMV0sWzIsMywicmciLDJdLFsyLDBdXQ==
\[\begin{tikzcd}
	{X_0\times b^\flat} & {X_0\times \Fb h^{[b,1]}_{0/}} \\
	{X_{1}} & E
	\arrow[from=1-2, to=2-2]
	\arrow[from=2-1, to=2-2]
	\arrow["rg"', from=1-1, to=2-1]
	\arrow[from=1-1, to=1-2]
\end{tikzcd}\]
where the upper horizontal morphism is induced by the restriction of $l$ to $(X_0\times b^\flat)\otimes\{1\}$.
As $X_1\to X_{/1}$ is initial, we have $\bot X_{/1}\sim \bot X_1$ and $\bot r$ is an equivalence. We denote by $F$ the left fibration associated to \eqref{eq:cor:explicit parital}. The previous square then corresponds to a morphism 
$$\int_{[b,1]}F\to E$$
Using the naturality of $\int_{[b,1]}$, one can see that this morphism induces an equivalence on fibers, and is then an equivalence. Applying $\partial_{[b,1]}$ and using theorem \ref{theo:gr construction}, this concludes the proof.
\end{proof}



\p A left cartesian fibration is \wcnotion{$\U$-small}{small left@$\U$-small left cartesian fibration} if its fibers are $\U$-small $\io$-categories. For an $\io$-category $A$, we denote by $\LCart_{\U}(A^\sharp)$ the full sub $\iun$-category of $\LCart(A^\sharp)$ whose objects correspond to $\U$-small left cartesian fibrations over $A^\sharp$.
\begin{cor}
\label{cor: Grt equivalence}
Let $\uni$ be the $\V$-small $\io$-category of $\U$-small $\io$-categories and $A$ a $\V$-small $\io$-category. There is an equivalence
$$\int_A:\Hom(A,\uni)\to \tau_0 \LCart_{\U}(A^\sharp)$$
natural in $A:\ocat^{op}$.
\end{cor}
\begin{proof}
This is a direct consequence of the theorem \ref{theo:gr construction} and the definition of $\uni$.
\end{proof}
\begin{cor}
\label{cor: universal fibration}
The left cartesian fibration $\int_{\omega}id$ is a the universal left cartesian fibration with $\U$-small fibers, i.e for any left cartesian fibration $X\to A^\sharp$ with $\U$-small fibers, there exists a unique morphism $X\to \uni$ and a unique cartesian square:
% https://q.uiver.app/#q=WzAsNCxbMCwxLCJBXlxcc2hhcnAiXSxbMCwwLCJYIl0sWzEsMSwiXFx1bmleXFxzaGFycCJdLFsxLDAsIlxcZG9tXFxpbnRfe1xcdW5pfWlkIl0sWzEsMF0sWzAsMl0sWzMsMiwiXFxpbnRfe1xcdW5pfWlkIl0sWzEsM10sWzEsMiwiIiwxLHsic3R5bGUiOnsibmFtZSI6ImNvcm5lciJ9fV1d
\[\begin{tikzcd}
	X & {\dom\int_{\uni}id} \\
	{A^\sharp} & {\uni^\sharp}
	\arrow[from=1-1, to=2-1]
	\arrow[from=2-1, to=2-2]
	\arrow["{\int_{\uni}id}", from=1-2, to=2-2]
	\arrow[from=1-1, to=1-2]
	\arrow["\lrcorner"{anchor=center, pos=0.125}, draw=none, from=1-1, to=2-2]
\end{tikzcd}\]
\end{cor}
\begin{proof}
This is a direct consequence of the corollary \ref{cor: Grt equivalence} and the functoriality of the Grothendieck construction given in proposition \ref{prop: derived int and partial are natural}.
\end{proof}



\subsection{Univalence}
\begin{notation*}
Through this section, we will identify any marked $\io$-category $C$ with the canonical induced morphism $C\to1$. If $f:X\to Y$ is a morphism, $f\times C$ then corresponds to the canonical morphism $X\times C\to Y$.
\end{notation*}
\p For the remaining of this section, we fix a marked $\io$-category $I$.
Remark that $\Fb h^{[n]}_k$ corresponds to the inclusion $(d_{0}^\sharp)^k:[n-k]^\sharp\to [n]^\sharp$.
 We define the functor \index[notation]{(intt@${{\oint}_{n,I}}$}
$$\oint_{n,I}: \Fun([n],\ocatm_{/I})\to \ocatm_{/I\otimes[n]^\sharp}$$
whose value on a morphism $E:[n]\to \ocatm_{/I}$ corresponding to a sequence $E_0\to ....\to E_n$, is
$$\oint_{n,I}E:=\colim_{m} \coprod_{i_0\leq... \leq i_m\leq n} E_{i_0}\otimes \Fb h_{i_m}^{[n]}.$$
As this functor is colimit preserving, it induces an adjunction \index[notation]{(partiall@$\ringpartial_{n,I}$}
\begin{equation}
\label{eq:Gr adj lax 1}
\begin{tikzcd}
	{\oint_{n,I}:\Fun([n],\ocatm_{/I})} & {\ocatm_{/I\otimes[n]^\sharp}:\ringpartial_{n,I}}
	\arrow[""{name=0, anchor=center, inner sep=0}, shift left=2, from=1-1, to=1-2]
	\arrow[""{name=1, anchor=center, inner sep=0}, shift left=2, from=1-2, to=1-1]
	\arrow["\dashv"{anchor=center, rotate=-90}, draw=none, from=0, to=1]
\end{tikzcd}
\end{equation}


\begin{lemma}
\label{lemma:oint preserves init}
The functor $\oint_{n,I}$ sends a natural transformation that is pointwise initial to an initial morphism.
\end{lemma}
\begin{proof}
As initial morphisms are closed under colimits, we have to show that for any integer $k$, and any morphism $E\to F$ of $\ocatm_{/I}$ corresponding to a sequence $X\xrightarrow{i} Y\to I$, the induced morphism $X\otimes [n-k]^\sharp\to Y\otimes [n-k]^\sharp$ over $I\otimes[n]^\sharp$ is initial whenever $i$ is. For this, remark that there is a square
% q.uiver.app/#q=WzAsNCxbMCwwLCJYXFxvdGltZXNcXHswXFx9Il0sWzAsMSwiWVxcb3RpbWVzXFx7MFxcfSJdLFsxLDAsIlhcXG90aW1lc1tuLWtdXlxcc2hhcnAiXSxbMSwxLCJZXFxvdGltZXNbbi1rXV5cXHNoYXJwIl0sWzAsMSwiaSJdLFsxLDNdLFswLDJdLFsyLDNdXQ==
\[\begin{tikzcd}
	{X\otimes\{0\}} & {X\otimes[n-k]^\sharp} \\
	{Y\otimes\{0\}} & {Y\otimes[n-k]^\sharp}
	\arrow["i", from=1-1, to=2-1]
	\arrow[from=2-1, to=2-2]
	\arrow[from=1-1, to=1-2]
	\arrow[from=1-2, to=2-2]
\end{tikzcd}\]
where the two horizontal morphisms are initial.
By stability by composition and left cancellation of initial morphism, this implies the result.
\end{proof}

\p According to the last lemma, the adjunction \eqref{eq:Gr adj lax 1} induces a derived adjunction
% q.uiver.app/#q=WzAsMixbMCwwLCJcXExiIFxcb2ludF97bixJfTpcXEZ1bihbbl0sXFxMQ2FydChJKSkiXSxbMSwwLCJcXExDYXJ0KElcXG90aW1lc1tuXV5cXHNoYXJwKTpcXFJiIFxccmluZ3BhcnRpYWxfe24sSX0iXSxbMCwxLCIiLDAseyJvZmZzZXQiOi0yfV0sWzEsMCwiIiwwLHsib2Zmc2V0IjotMn1dLFsyLDMsIiIsMCx7ImxldmVsIjoxLCJzdHlsZSI6eyJuYW1lIjoiYWRqdW5jdGlvbiJ9fV1d
\begin{equation}
\label{eq:Gr adj lax 2}
\begin{tikzcd}
	{\Lb \oint_{n,I}:\Fun([n],\LCart(I))} & {\LCart(I\otimes[n]^\sharp):\Rb \ringpartial_{n,I}}
	\arrow[""{name=0, anchor=center, inner sep=0}, shift left=2, from=1-1, to=1-2]
	\arrow[""{name=1, anchor=center, inner sep=0}, shift left=2, from=1-2, to=1-1]
	\arrow["\dashv"{anchor=center, rotate=-90}, draw=none, from=0, to=1]
\end{tikzcd}
\end{equation}
where $\Rb \ringpartial_{n,I}$ is just the restriction of $\ringpartial_{n,I}$ to $\LCart(I\otimes[n]^\sharp)$.
\begin{lemma}
\label{lemma:ringpartial fiber}
Let $i:[n]^\sharp \to [m]^\sharp$ and $j:I\to J$ be two morphisms. Let $E$ be an object of $\LCart(I\otimes [m]^\sharp)$. The natural transformation
$$\ringpartial_{n,I} (j\otimes i)^*E\to j^*\circ \ringpartial_{m,J} E\circ i^\natural $$
 is an equivalence.
\end{lemma}
\begin{proof}
As invertible natural transformations are detected pointwise, one can suppose that $n=0$, and let $k$ be the image of $[0]$ by $i$.
Let $E_0\to E_1\to.. \to E_m $ be the sequence of morphisms of $\LCart(J)$ corresponding to $\ringpartial_{m,J} E$. 


The object $ j^*\circ \ringpartial_{m,J} E\circ i^\natural$ is then equivalent to $ j^* E_k$ by definition. 
As $\ringpartial_{0,I}$ is the identity, we have to show that the canonical morphism $(j\otimes \{k\})^*E\to j^*E_k$ is an equivalence. Remark that for any
 $F$ of $\ocatm_{/I}$, we have by adjunction a commutative square: 
% https://q.uiver.app/#q=WzAsNCxbMCwwLCJcXEhvbShGLChqXFxvdGltZXMgIFxce2tcXH0pXipFKSJdLFswLDEsIlxcSG9tKChqXFxvdGltZXMgIFxce2tcXH0pXyFGLEUpIl0sWzEsMSwiXFxIb20oKGpfIUYpXFxvdGltZXMgaF9rXntbbl19LEUpIl0sWzEsMCwiXFxIb20oRixcXExiIGpeKiBFX2spIl0sWzAsMSwiXFxzaW0iXSxbMSwyXSxbMCwzXSxbMywyLCJcXHNpbSJdXQ==
\[\begin{tikzcd}
	{\Hom(F,(j\otimes \{k\})^*E)} & {\Hom(F,\Lb j^* E_k)} \\
	{\Hom((j\otimes \{k\})_!F,E)} & {\Hom((j_!F)\otimes h_k^{[n]},E)}
	\arrow["\sim", from=1-1, to=2-1]
	\arrow[from=2-1, to=2-2]
	\arrow[from=1-1, to=1-2]
	\arrow["\sim", from=1-2, to=2-2]
\end{tikzcd}\]
where the two vertical morphisms are equivalences. As $((j\otimes \{k\})_!F\sim (j_!F)\otimes h_k^{[n]}$, the lower morphism is an equivalence, and so is the top one. This implies the desired result.
\end{proof}


\p In the following lemmas and proposition, we focus on the case where $I$ is of the form $A^\sharp$, where everything happens more simply.
\begin{lemma}
\label{lemme:oint a sharp is natural1}
Let $j:A\to B$ be a morphism between $\io$-categories and $i:[n]\to [m]$ a morphism of $\Delta$. Let $E$ be an object of $\Fun([n],\LCart(A^\sharp))$.
The canonical morphism 
$$\Lb\oint_{n,A^\sharp}( \Rb j^*\circ E\circ i)\to \Rb(j\times i^\sharp)^* \Lb \oint_{m,B^\sharp} E$$
is an equivalence.
\end{lemma}
\begin{proof}
As equivalences in $\Fun([m],\LCart(B^\sharp))$ are detected on points, an equivalences on $\LCart(B^\sharp\times [m]^\sharp)$ are detected on fibers, we can suppose that $n=0$, $A=1$, and we denote by $k$ the image of $i$ and $a$ the image of $B$. As $\Lb\oint_{0,1}$ is the identity, one has to show that the canonical morphism 
\begin{equation}
\label{eq:equationoint a sharp is natural1}
\Rb a^* E_k\to \Rb(a\times \{k\})^* \Lb \oint_{m,B^\sharp} E
\end{equation}
 is an equivalence.

Moreover, for any $l\leq n$, the proposition \ref{prop:cotimes 1 to ctimes 1 is a trivialization} implies that the canonical morphism $\Fb(E_l\otimes \Fb h_l^{[n]})\to E_l\times \Fb h_l^{[n]}$ is an equivalence, as this two left cartesian fibrations are replacement of $E_l\otimes h_l^{[n]}\sim E_l\times h_l^{[n]}$. According to proposition \ref{prop:fiber preserves colimits}, $\Rb(a\times \{k\}^\sharp)^*$ preserves colimits, we then have 
$$ \Rb(a\times \{k\})^* \Lb \oint_{m,B^\sharp} E\sim \colim_m\coprod_{i_0\leq ...\leq i_m\leq k}\Rb a^*E_{i_0}\sim \colim_{i:[k]}\Rb a^*E_i\sim \Rb a^*E_k.$$
The morphism \eqref{eq:equationoint a sharp is natural1} is then an equivalence, which concludes the proof.
\end{proof}


\begin{prop}
\label{prop:ring partial is natural}
The functor $\Rb\ringpartial_{n,I}$ is natural in $n:\Delta^{op}$ and $I:\ocatm^{op}$. The functor $\oint_{n,A^\sharp}$ is natural in $n:\Delta^{op}$ and $A:\ocat^{op}$.
\end{prop}
\begin{proof}
The proof is similar to the one of proposition \ref{prop: derived int and partial are natural}, using lemma \ref{lemma:ringpartial fiber} and lemma \ref{lemme:oint a sharp is natural1} instead of lemma \ref{lemma:partial fiber} and lemma \ref{lemma:int fiber 2}. 
 \end{proof}
 

\begin{prop}
\label{prop:lax gr construction particular case}
For any $\io$-category $A$ and any integer $n$, the adjunction 
% https://q.uiver.app/#q=WzAsMixbMCwwLCJcXExiIFxcb2ludF97bixBXlxcc2hhcnB9OlxcRnVuKFtuXSxcXExDYXJ0KEFeXFxzaGFycCkpIl0sWzEsMCwiXFxMQ2FydCgoQVxcdGltZXNbbl0pXlxcc2hhcnApOlxcUmIgXFxyaW5ncGFydGlhbF97bixBXlxcc2hhcnB9Il0sWzAsMSwiIiwwLHsib2Zmc2V0IjotMn1dLFsxLDAsIiIsMCx7Im9mZnNldCI6LTJ9XSxbMiwzLCIiLDAseyJsZXZlbCI6MSwic3R5bGUiOnsibmFtZSI6ImFkanVuY3Rpb24ifX1dXQ==
\[\begin{tikzcd}
	{\Lb \oint_{n,A^\sharp}:\Fun([n],\LCart(A^\sharp))} & {\LCart((A\times[n])^\sharp):\Rb \ringpartial_{n,A^\sharp}}
	\arrow[""{name=0, anchor=center, inner sep=0}, shift left=2, from=1-1, to=1-2]
	\arrow[""{name=1, anchor=center, inner sep=0}, shift left=2, from=1-2, to=1-1]
	\arrow["\dashv"{anchor=center, rotate=-90}, draw=none, from=0, to=1]
\end{tikzcd}\]
 is an adjoint equivalence.
\end{prop}
\begin{proof}
As in both case equivalences are detected on fibers, and as these functors are natural in $A$ and $n$, one can show the result for $A$ being the terminal $\io$-category and $n=0$. In this case remark that these two functors are the identities. 
\end{proof}


\p
We set \wcnotation{$\Fun^c([n],\LCart(I))$}{(func@$\Fun^c([\uvar],\uvar)$} as the pullback
% https://q.uiver.app/#q=WzAsNCxbMCwwLCJcXEZ1bl5jKFtuXSxcXExDYXJ0KEkpKSJdLFsxLDAsIlxcRnVuKFtuXSxcXExDYXJ0KEkpKSJdLFsxLDEsIlxccHJvZF97a1xcbGVxIG59XFxGdW4oXFx7a1xcfSxcXExDYXJ0KEkpKSJdLFswLDEsIlxccHJvZF97a1xcbGVxIG59XFxMQ2FydChJXlxcc2hhcnApIl0sWzAsM10sWzMsMl0sWzEsMl0sWzAsMV0sWzAsNSwiIiwyLHsibGV2ZWwiOjEsInN0eWxlIjp7Im5hbWUiOiJjb3JuZXIifX1dXQ==
\[\begin{tikzcd}
	{\Fun^c([n],\LCart(I))} & {\Fun([n],\LCart(I))} \\
	{\prod_{k\leq n}\LCart(I^\sharp)} & {\prod_{k\leq n}\Fun(\{k\},\LCart(I))}
	\arrow[from=1-1, to=2-1]
	\arrow[""{name=0, anchor=center, inner sep=0}, from=2-1, to=2-2]
	\arrow[from=1-2, to=2-2]
	\arrow[from=1-1, to=1-2]
	\arrow["\lrcorner"{anchor=center, pos=0.125}, draw=none, from=1-1, to=0]
\end{tikzcd}\]
where $I^\sharp$ stand for $(I^\natural)^\sharp$. An object of this $\iun$-category is then a sequence in $\LCart(I)$:
% https://q.uiver.app/#q=WzAsMyxbMSwwLCIuLi4iXSxbMiwwLCJGXyBuIl0sWzAsMCwiRl8wIl0sWzIsMF0sWzAsMV1d
\[\begin{tikzcd}
	{F_0} & {...} & {F_ n}
	\arrow[from=1-1, to=1-2]
	\arrow[from=1-2, to=1-3]
\end{tikzcd}\]
such that for any integer $i\leq n$, $F_i$ is classified. 
A $1$-cell of this $\iun$-category is a sequence of square in $\LCart(I)$:
% https://q.uiver.app/#q=WzAsNixbMSwwLCIuLi4iXSxbMiwwLCJGXyBuIl0sWzAsMCwiRl8wIl0sWzAsMSwiR18wIl0sWzIsMSwiR19uIl0sWzEsMSwiLi4uIl0sWzIsMF0sWzAsMV0sWzAsNV0sWzIsM10sWzEsNF0sWzMsNV0sWzUsNF1d
\[\begin{tikzcd}
	{F_0} & {...} & {F_ n} \\
	{G_0} & {...} & {G_n}
	\arrow[from=1-1, to=1-2]
	\arrow[from=1-2, to=1-3]
	\arrow[from=1-2, to=2-2]
	\arrow[from=1-1, to=2-1]
	\arrow[from=1-3, to=2-3]
	\arrow[from=2-1, to=2-2]
	\arrow[from=2-2, to=2-3]
\end{tikzcd}\]
such that for any $k\leq n$, the morphism $F_k\to G_k$ comes from a morphism beetwen the corresponding objects of $\LCart(I^\sharp)$.


\begin{prop}
\label{prop:Fun preserve colimies}
Let $F:I\to \ocatm$ be a $\Wcard$-small diagram. The canonical functor
$$\Fun^c([n],\LCart(\colim_I F))\to \lim_I\Fun^c([n],\LCart(F))$$
is an equivalence.
\end{prop}
\begin{proof}
This morphism fits in an adjunction:
% q.uiver.app/#q=WzAsMixbMCwwLCJcXGNvbGltX0k6XFxsaW1fSVxcRnVuXmMoW25dLFxcTENhcnQoRikpIl0sWzEsMCwiXFxGdW5eYyhbbl0sXFxMQ2FydChcXGNvbGltX0kgRikpIl0sWzEsMCwiIiwwLHsib2Zmc2V0IjotMn1dLFswLDEsIiIsMCx7Im9mZnNldCI6LTJ9XSxbMywyLCIiLDAseyJsZXZlbCI6MSwic3R5bGUiOnsibmFtZSI6ImFkanVuY3Rpb24ifX1dXQ==
\[\begin{tikzcd}
	{\colim_I:\lim_I\Fun^c([n],\LCart(F))} & {\Fun^c([n],\LCart(\colim_I F))}
	\arrow[""{name=0, anchor=center, inner sep=0}, shift left=2, from=1-2, to=1-1]
	\arrow[""{name=1, anchor=center, inner sep=0}, shift left=2, from=1-1, to=1-2]
	\arrow["\dashv"{anchor=center, rotate=-90}, draw=none, from=1, to=0]
\end{tikzcd}\]
The corollary \ref{cor:fib over a colimit} implies that the counit of this adjunction is an equivalence. To conclude, we have to show that the right adjoint is essentially surjective. On objects, this adjunction corresponds to the canonical equivalence 
$$\lim_I\Hom([n],\LCartc(F))\sim \Hom([n],\LCartc(\colim_IF))$$
induced by corollary \ref{cor:fib over a colimit2}
\end{proof}


\p As $\Rb\ringpartial_{0,I}$ is the identity, lemma \ref{lemma:ringpartial fiber} implies that the functor
$$\LCart((I\otimes[n]^\sharp)^\sharp)\to \LCart(I\otimes[n]^\sharp) \xrightarrow{\Rb\ringpartial_{n,I}} \Fun([n],\LCart(I))$$
 factors through a functor \index[notation]{(partialll@$\ringpartial_{n,I}^c$}
 \begin{equation}
 \label{eq:def of right partial classified}
\ringpartial_{n,I}^c:\LCart((I\otimes[n]^\sharp)^\sharp)\to \Fun^c([n],\LCart(I))
\end{equation}
We are now willing to show that this functor is an equivalence, and to this extent, we will construct an inverse.



\p We fix an object $a$ of $t\Theta$. We define $[a,1]^\sharp:=([a,1]^\natural)^\sharp$
 and $\iota$ the canonical inclusion $[a,1]\to [a,1]^\sharp$.
 
We directly have an equivalence 
$$\Lb\iota_!\Rb\iota^*\Fb h_{1}^{[a^\natural,1]}\sim \Fb h_{1}^{[a^\natural,1]}$$
The next lemma provides an explicit expression for $\Lb\iota_!\Rb\iota^*\Fb h_{0}^{[a^\natural,1]}$.

\begin{lemma}
\label{lemma:replacement of unmarked slice}
Let $a$ be an object of $t\Theta$. We have an equivalence
$$\Lb\iota_!\Rb\iota^*\Fb h_{0}^{[a^\natural,1]}\sim \Fb h_{0}^{[a^\natural,1]}\coprod_{a^\flat\otimes\{0\}}(a\otimes[1]^\sharp)^\flat.$$
Moreover the morphism 
$\Lb\iota_!(a^\flat \to \Fb h_{0}^{[a^\natural,1]})$ corresponds to the inclusion 
$$(a\otimes\{0\})^\flat \to (a\otimes[1]^\sharp)^\flat \to \Fb h_{0}^{[a^\natural,1]}\coprod_{a^\flat\otimes\{0\}}(a\otimes[1]^\sharp)^\flat.$$
\end{lemma}
\begin{proof}
The theorem \ref{theo:equivalence betwen slice and join} implies that 
$\iota_!\Rb\iota^*\Fb h_{0}^{[b,1]}$ and $\iota_!\Rb\iota^*\Fb h_{0}^{[(\Db_n)_t,1]}$ are respectively equivalent to 
$$(1\costar b)^\flat\to [b,1]^\sharp~~~~ \mbox{and}~~~~ (1\costar \Db_n)^{\sharp_{n+1}}\to [\Db_n,1]^\sharp$$
The theorem \ref{theo:formula between pullback of slice and tensor marked case} induces cartesian diagrams
% https://q.uiver.app/#q=WzAsMTYsWzMsMywiW2IsMV1eXFxzaGFycCJdLFszLDEsIigxXFxjb3N0YXIgYileXFxmbGF0Il0sWzIsMiwiYl5cXGZsYXRcXHN0YXIgMSJdLFsyLDAsIihiXFxvdGltZXNbMV0pXlxcZmxhdCJdLFsxLDMsIlxcezBcXH0iXSxbMSwxLCIxIl0sWzAsMiwiYl5cXGZsYXQiXSxbMCwwLCJiXlxcZmxhdFxcb3RpbWVzXFx7MFxcfSJdLFs0LDAsIlxcRGJfbl5cXGZsYXRcXG90aW1lc1xcezBcXH0iXSxbNCwyLCJcXERiX25eXFxmbGF0Il0sWzYsMiwiXFxEYl9uXlxcZmxhdFxcc3RhciAxIl0sWzUsMywiXFx7MFxcfSJdLFs3LDMsIltcXERiX24sMV1eXFxzaGFycCJdLFs3LDEsIigxXFxjb3N0YXIgXFxEYl9uKV57XFxzaGFycF97bisxfX0iXSxbNSwxLCIxIl0sWzYsMCwiKFxcRGJfblxcb3RpbWVzWzFdKV57XFxzaGFycF97bisxfX0iXSxbNCwwXSxbMSwwXSxbMywyXSxbNyw2XSxbNSw0XSxbNiw0XSxbMiwwXSxbMywxXSxbNyw1XSxbNywzXSxbNSwxXSxbNiwyXSxbOCwxNF0sWzgsMTVdLFsxNSwxM10sWzEwLDEyXSxbOSwxMV0sWzgsOV0sWzE0LDExXSxbMTUsMTBdLFsxMywxMl0sWzksMTBdLFsxMSwxMl0sWzE0LDEzXV0=
\[\begin{tikzcd}[sep =0.1cm]
	{b^\flat\otimes\{0\}} && {(b\otimes[1])^\flat} && {\Db_n^\flat\otimes\{0\}} && {(\Db_n\otimes[1])^{\sharp_{n+1}}} \\
	& 1 && {(1\costar b)^\flat} && 1 && {(1\costar \Db_n)^{\sharp_{n+1}}} \\
	{b^\flat} && {b^\flat\star 1} && {\Db_n^\flat} && {\Db_n^\flat\star 1} \\
	& {\{0\}} && {[b,1]^\sharp} && {\{0\}} && {[\Db_n,1]^\sharp}
	\arrow[from=4-2, to=4-4]
	\arrow[from=2-4, to=4-4]
	\arrow[from=1-3, to=3-3]
	\arrow[from=1-1, to=3-1]
	\arrow[from=2-2, to=4-2]
	\arrow[from=3-1, to=4-2]
	\arrow[from=3-3, to=4-4]
	\arrow[from=1-3, to=2-4]
	\arrow[from=1-1, to=2-2]
	\arrow[from=1-1, to=1-3]
	\arrow[from=2-2, to=2-4]
	\arrow[from=3-1, to=3-3]
	\arrow[from=1-5, to=2-6]
	\arrow[from=1-5, to=1-7]
	\arrow[from=1-7, to=2-8]
	\arrow[from=3-7, to=4-8]
	\arrow[from=3-5, to=4-6]
	\arrow[from=1-5, to=3-5]
	\arrow[from=2-6, to=4-6]
	\arrow[from=1-7, to=3-7]
	\arrow[from=2-8, to=4-8]
	\arrow[from=3-5, to=3-7]
	\arrow[from=4-6, to=4-8]
	\arrow[from=2-6, to=2-8]
\end{tikzcd}\]
Remark furthermore that we have an equivalence
$$\bot(\Db_n\otimes[1])^{\sharp_{n+1}}\sim \tau^i_{n}(\Db_n\otimes[1])=: ((\Db_n)_t\otimes[1]^\sharp)^\natural.$$
Applying the full duality to theorem \ref{theo:equivalence betwen slice and join} and using the corollary \ref{cor:explicit partial}, this proves the first assertion.


The second assertion follows from the naturality in $E$ of the construction given in corollary \ref{cor:explicit partial} and from the squares
% https://q.uiver.app/#q=WzAsMTIsWzEsMiwiW2IsMV1eXFxzaGFycCJdLFsxLDEsIigxXFxjb3N0YXIgYileXFxmbGF0Il0sWzAsMiwiYl5cXGZsYXRcXHN0YXIgMSJdLFswLDEsIihiXFxvdGltZXNbMV0pXlxcZmxhdCJdLFszLDIsIltcXERiX24sMV1eXFxzaGFycCJdLFszLDEsIigxXFxjb3N0YXIgXFxEYl9uKV57XFxzaGFycF97bisxfX0iXSxbMiwyLCJcXERiX25eXFxmbGF0XFxzdGFyIDEiXSxbMiwxLCIoXFxEYl9uXFxvdGltZXNbMV0pXntcXHNoYXJwX3tuKzF9fSJdLFsxLDAsImJeXFxmbGF0Il0sWzAsMCwiYl5cXGZsYXRcXG90aW1lc1xcezFcXH0iXSxbMywwLCJcXERiX25eXFxmbGF0Il0sWzIsMCwiXFxEYl9uXlxcZmxhdFxcb3RpbWVzXFx7MVxcfSJdLFszLDJdLFszLDFdLFsyLDBdLFsxLDBdLFs2LDRdLFs1LDRdLFs3LDZdLFs3LDVdLFs5LDNdLFs5LDhdLFs4LDFdLFsxMCw1XSxbMTEsN10sWzExLDEwXV0=
\[\begin{tikzcd}
	{b^\flat\otimes\{1\}} & {b^\flat} & {\Db_n^\flat\otimes\{1\}} & {\Db_n^\flat} \\
	{(b\otimes[1])^\flat} & {(1\costar b)^\flat} & {(\Db_n\otimes[1])^{\sharp_{n+1}}} & {(1\costar \Db_n)^{\sharp_{n+1}}} \\
	{b^\flat\star 1} & {[b,1]^\sharp} & {\Db_n^\flat\star 1} & {[\Db_n,1]^\sharp}
	\arrow[from=2-1, to=3-1]
	\arrow[from=2-1, to=2-2]
	\arrow[from=3-1, to=3-2]
	\arrow[from=2-2, to=3-2]
	\arrow[from=3-3, to=3-4]
	\arrow[from=2-4, to=3-4]
	\arrow[from=2-3, to=3-3]
	\arrow[from=2-3, to=2-4]
	\arrow[from=1-1, to=2-1]
	\arrow[from=1-1, to=1-2]
	\arrow[from=1-2, to=2-2]
	\arrow[from=1-4, to=2-4]
	\arrow[from=1-3, to=2-3]
	\arrow[from=1-3, to=1-4]
\end{tikzcd}\]
that are cartesian according to theorem \ref{theo:formula between pullback of slice and tensor marked case}.
\end{proof}

\p We fix an object $a$ of $t\Theta$. Let $E$ be an object of $\LCart([a,1]^\sharp)$. According to theorem \ref{theo:gr construction}, there exists a morphism $X(0)\times a^\natural \to X(1)$ such that $E$ corresponds to the colimit
$$X(0)^\flat\times \Fb h_0^{[a^\natural,1]}\coprod_{X(0)^\flat \times a^\flat}X(1)^\flat $$
We claim that $\Lb\iota_!\Rb \iota^*E$ is the left cartesian fibration
\begin{equation}
X(0)^\flat\times (\Fb h_0^{[a^\natural,1]}\coprod_{ a^\flat} (a\otimes[1]^\sharp)^\flat) \coprod_{X(0)^\flat\times (a\otimes\{1\})^\flat}X(1)^\flat 
\label{eq:explicit iota excalmation}
\end{equation}
Indeed, the lemma \ref{lemma:replacement of unmarked slice} provides an initial morphism from $\iota_!\Rb \iota^*E$ to this object, and the theorem \ref{theo:left cart stable by colimit} implies that this object is a left cartesian fibration.

\begin{lemma}
\label{lemma:characterisation of natural transoformation}
Let $\psi: \iota_!\Rb \iota^*\to \Lb\iota_!\Rb \iota^*$ be a natural transformation, endowed with a family of natural commutative squares:
% q.uiver.app/#q=WzAsNCxbMSwwLCJcXExiXFxpb3RhXyFcXFJiIFxcaW90YV4qKEJeXFxmbGF0XFx0aW1lcyBFKSJdLFswLDAsIiBcXGlvdGFfIVxcUmIgXFxpb3RhXiooQl5cXGZsYXRcXHRpbWVzIEUpIl0sWzAsMSwiQl5cXGZsYXQgXFx0aW1lc1xcaW90YV8hXFxSYiBcXGlvdGFeKkUiXSxbMSwxLCJCXlxcZmxhdCBcXHRpbWVzXFxpb3RhXyFcXFJiIFxcaW90YV4qRSJdLFsxLDAsIlxccHNpX3tCXlxcZmxhdFxcdGltZXMgRX0iXSxbMSwyXSxbMiwzLCJCXlxcZmxhdFxcdGltZXNcXHBzaV97RX0iLDJdLFswLDNdXQ==
\[\begin{tikzcd}
	{ \iota_!\Rb \iota^*(B^\flat\times E)} & {\Lb\iota_!\Rb \iota^*(B^\flat\times E)} \\
	{B^\flat \times\iota_!\Rb \iota^*E} & {B^\flat \times\iota_!\Rb \iota^*E}
	\arrow["{\psi_{B^\flat\times E}}", from=1-1, to=1-2]
	\arrow[from=1-1, to=2-1]
	\arrow["{B^\flat\times\psi_{E}}"', from=2-1, to=2-2]
	\arrow[from=1-2, to=2-2]
\end{tikzcd}\]
where we identify marked $\io$-categories with their canonical morphims to the terminal marked $\io$-category. The natural transformation $\psi$ is then the one obtained by the functorial factorization in initial morphisms followed by left cartesian fibrations. 
\end{lemma}
\begin{proof}
The natural transformation $\psi$ induces a natural transformation $\Db \psi:\Lb\iota_!\Rb \iota^*\to \Lb\iota_!\Rb \iota^*$ and we have to check that this last natural transformation is the identity. The explicit Grothendieck construction states that $E$ is a colimit of left cartesian fibration of shape $B^\flat \times \Fb h^{[a^\natural,1]}_{\epsilon}$ for $\epsilon\in \{0,1\}$. The hypothesis implies that we just have to show that $\Db \psi_{\Fb h^{[a^\natural,1]}_{0}}$ and $\Db \psi_{\Fb h^{[a^\natural,1]}_{1}}$ are equivalences, and we will check this on fibers.

Using the explicit expression of $\Lb\iota_!\Rb \iota$ given in \eqref{eq:explicit iota excalmation}, we have equivalences
$$\{0\}^*\Lb\iota_!\Rb \iota \Fb h_0^{[a^\natural,1]} \sim 1~~~~~~~~
\{0\}^*\Lb\iota_!\Rb \iota \Fb h_1^{[a^\natural,1]} \sim \emptyset~~~~~~~~
\{1\}^*\Lb\iota_!\Rb \iota \Fb h_0^{[a^\natural,1]} \sim 1$$
which directly implies that $\{0\}^*\Db \psi_{\Fb h_0^{[a^\natural,1]}}$, $\{0\}^*\Db \psi_{\Fb h_1^{[a^\natural,1]}}$ and $\{1\}^*\Db \psi_{\Fb h_1^{[a^\natural,1]}}$ are equivalences. The only case remaining is $\{1\}^*\Db \psi_{\Fb h_0^{[a^\natural,1]}}$. This morphism corresponds to an endomorphism of $(a\otimes[1]^\sharp)^\natural$, which is a strict object according to \ref{prop:tensor of glboer are strics}. By right cancellation, the morphism induced by the domain of $\Db\psi_{\Fb h_0^{[a^\natural,1]}}$ is a left cartesian fibration. There exists then a lift in the following diagram
% https://q.uiver.app/#q=WzAsNCxbMSwwLCJbYSwxXV57XFxzaGFycH1fezAvfVxcY29wcm9kX3thXlxcZmxhdFxcb3RpbWVzXFx7MFxcfX0oYVxcb3RpbWVzWzFdXlxcc2hhcnApXlxcZmxhdCJdLFsxLDEsIlthLDFdXntcXHNoYXJwfV97MC99XFxjb3Byb2Rfe2FeXFxmbGF0XFxvdGltZXNcXHswXFx9fShhXFxvdGltZXNbMV1eXFxzaGFycCleXFxmbGF0Il0sWzAsMCwiXFx7MFxcfSJdLFswLDEsIlthLDFdXntcXHNoYXJwfV97MC99Il0sWzAsMSwiXFxkb21cXERiXFxwc2lfe1xcRmIgaF8wXntbYV5cXG5hdHVyYWwsMV19fSJdLFsyLDNdLFszLDEsIlxcaW90YSIsMl0sWzIsMF0sWzMsMCwibCIsMV1d
\begin{equation}
\label{eq:square in proof of replement ofneofoeijfoepaj}
\begin{tikzcd}
	{\{0\}} & {[a,1]^{\sharp}_{0/}\coprod_{a^\flat\otimes\{0\}}(a\otimes[1]^\sharp)^\flat} \\
	{[a,1]^{\sharp}_{0/}} & {[a,1]^{\sharp}_{0/}\coprod_{a^\flat\otimes\{0\}}(a\otimes[1]^\sharp)^\flat}
	\arrow["{\dom\Db\psi_{\Fb h_0^{[a^\natural,1]}}}", from=1-2, to=2-2]
	\arrow[from=1-1, to=2-1]
	\arrow["\iota"', from=2-1, to=2-2]
	\arrow[from=1-1, to=1-2]
	\arrow["l"{description}, from=2-1, to=1-2]
\end{tikzcd}
\end{equation}
where $\iota$ is the canonical inclusion. As $l$ and $\iota$ are lifts in the following diagram:
% https://q.uiver.app/#q=WzAsNCxbMSwwLCJbYSwxXV57XFxzaGFycH1fezAvfVxcY29wcm9kX3thXlxcZmxhdFxcb3RpbWVzXFx7MFxcfX0oYVxcb3RpbWVzWzFdXlxcc2hhcnApXlxcZmxhdCJdLFsxLDEsIlthLDFdXlxcc2hhcnAiXSxbMCwwLCJcXHswXFx9Il0sWzAsMSwiW2EsMV1ee1xcc2hhcnB9X3swL30iXSxbMCwxXSxbMiwzXSxbMywxXSxbMiwwXSxbMywwXV0=
\[\begin{tikzcd}
	{\{0\}} & {[a,1]^{\sharp}_{0/}\coprod_{a^\flat\otimes\{0\}}(a\otimes[1]^\sharp)^\flat} \\
	{[a,1]^{\sharp}_{0/}} & {[a,1]^\sharp}
	\arrow[from=1-2, to=2-2]
	\arrow[from=1-1, to=2-1]
	\arrow[from=2-1, to=2-2]
	\arrow[from=1-1, to=1-2]
	\arrow[from=2-1, to=1-2]
\end{tikzcd}\]
they are equivalent. Taking the fiber on $\{1\}$ of the cartesian square \eqref{eq:square in proof of replement ofneofoeijfoepaj}, this induces a commutative triangle:
% https://q.uiver.app/#q=WzAsMyxbMSwwLCIoYVxcb3RpbWVzWzFdXlxcc2hhcnApXlxcbmF0dXJhbCJdLFsxLDEsIihhXFxvdGltZXNbMV1eXFxzaGFycCleXFxuYXR1cmFsIl0sWzAsMCwiKGFcXG90aW1lc1xcezBcXH0pXlxcbmF0dXJhbCJdLFsyLDBdLFswLDEsIlxcezFcXH1eKlxcRGIgXFxwc2lfe1xcRmIgaF8wXntbYV5cXG5hdHVyYWwsMV19fSJdLFsyLDFdXQ==
\[\begin{tikzcd}
	{(a\otimes\{0\})^\natural} & {(a\otimes[1]^\sharp)^\natural} \\
	& {(a\otimes[1]^\sharp)^\natural}
	\arrow[from=1-1, to=1-2]
	\arrow["{\{1\}^*\Db \psi_{\Fb h_0^{[a^\natural,1]}}}", from=1-2, to=2-2]
	\arrow[from=1-1, to=2-2]
\end{tikzcd}\]
Eventually, the naturality induces a commutative squares. 
% https://q.uiver.app/#q=WzAsNCxbMCwwLCIoYVxcb3RpbWVzWzFdXlxcc2hhcnApXlxcbmF0dXJhbCJdLFswLDEsIihhXFxvdGltZXNbMV1eXFxzaGFycCleXFxuYXR1cmFsIl0sWzEsMCwiKGFcXG90aW1lc1xcezFcXH0pXlxcbmF0dXJhbCJdLFsxLDEsIihhXFxvdGltZXNcXHsxXFx9KV5cXG5hdHVyYWwiXSxbMCwxLCJcXHsxXFx9XipcXERiXFxwc2lfe1xcRmIgaF8wXntbYV5cXG5hdHVyYWwsMV19fSIsMl0sWzAsMl0sWzEsM10sWzIsMywiXFx7MVxcfV4qXFxEYlxccHNpX3thXlxcZmxhdFxcdGltZXMgXFxGYiBoXzFee1thXlxcbmF0dXJhbCwxXX19XFxzaW0gaWQiXV0=
\[\begin{tikzcd}
	{(a\otimes[1]^\sharp)^\natural} & {(a\otimes\{1\})^\natural} \\
	{(a\otimes[1]^\sharp)^\natural} & {(a\otimes\{1\})^\natural}
	\arrow["{\{1\}^*\Db\psi_{\Fb h_0^{[a^\natural,1]}}}"', from=1-1, to=2-1]
	\arrow[from=1-1, to=1-2]
	\arrow[from=2-1, to=2-2]
	\arrow["{\{1\}^*\Db\psi_{a^\flat\times \Fb h_1^{[a^\natural,1]}}\sim id}", from=1-2, to=2-2]
\end{tikzcd}\]
The restriction of the morphism $\Db\psi_{\Fb h_0^{[a^\natural,1]}}$ to $a\otimes\{0\}$ and $a\otimes\{1\}$ is therefore the identity.
Using Steiner theory, we can easily show that it forces $\Db\psi_{\Fb h_0^{[a^\natural,1]}}$ to also be the identity.
\end{proof}

\p We fix an object $F$ of $\LCart([a,1]^\sharp)$, and a morphism $\phi:\Rb \iota^* E\to \Rb \iota^* F$. By adjunction, this corresponds to a morphism $\tilde{\phi}: \iota_! \Rb\iota^*E\to F$, and as $F$ corresponds to a left cartesian fibration, this induces a morphism $\Db\tilde{\phi}:\Lb\iota_! \Rb\iota^*E\to F$. Using once again theorem \ref{theo:gr construction}, this induces a morphism 
$\partial_{[a^\natural,1]} \Lb\iota_! \Rb\iota^*E\to \partial_{[a^\natural,1]} F$, that corresponds, according to the explicit expression of $\Lb\iota_!\Rb \iota$ given in \eqref{eq:explicit iota excalmation}, to a commutative square
% q.uiver.app/#q=WzAsNCxbMCwxLCJYKDApXFx0aW1lcyAoYVxcb3RpbWVzWzFdXlxcc2hhcnApXlxcbmF0dXJhbCBcXGNvcHJvZF97WCgwKVxcdGltZXMgYV5cXG5hdHVyYWx9WCgxKSJdLFsxLDEsIlkoMSkiXSxbMSwwLCJZKDApXFx0aW1lcyBhXlxcbmF0dXJhbCJdLFswLDAsIlgoMClcXHRpbWVzIGFeXFxuYXR1cmFsIl0sWzAsMSwiXFxEYiBcXHRpbGRle1xccGhpfSgxKSIsMl0sWzMsMiwiXFxEYiBcXHRpbGRle1xccGhpfSgwKVxcdGltZXMgYV5cXG5hdHVyYWwiXSxbMiwxXSxbMywwXV0=
\[\begin{tikzcd}
	{X(0)\times a^\natural} & {Y(0)\times a^\natural} \\
	{X(0)\times (a\otimes[1]^\sharp)^\natural \coprod_{X(0)\times a^\natural}X(1)} & {Y(1)}
	\arrow["{\Db \tilde{\phi}(1)}"', from=2-1, to=2-2]
	\arrow["{\Db \tilde{\phi}(0)\times a^\natural}", from=1-1, to=1-2]
	\arrow[from=1-2, to=2-2]
	\arrow[from=1-1, to=2-1]
\end{tikzcd}\]
where $Y(0)\times a^\flat \to Y(1)$ corresponds to $\partial_{[a^\natural,1]} F$.
This is equivalent to a diagram
% q.uiver.app/#q=WzAsNixbMiwyLCJYKDEpIl0sWzMsMSwiWSgxKSJdLFsyLDAsIlkoMClcXHRpbWVzIGFeXFxuYXR1cmFsIl0sWzAsMiwiWCgwKVxcdGltZXMgYV5cXG5hdHVyYWwiXSxbMCwwLCJYKDApXFx0aW1lcyBhXlxcbmF0dXJhbCJdLFsxLDEsIlgoMClcXHRpbWVzIChhXFxvdGltZXMgWzFdXlxcc2hhcnApXlxcbmF0dXJhbCJdLFswLDFdLFsyLDFdLFszLDBdLFs0LDIsIlxcRGIgXFx0aWxkZXtcXHBoaX0oMClcXHRpbWVzIGFeXFxuYXR1cmFsIl0sWzQsNV0sWzMsNV0sWzUsMV1d
\begin{equation}
\label{eq:lax technical big diagram}
\begin{tikzcd}
	{X(0)\times a^\natural} && {Y(0)\times a^\natural} \\
	& {X(0)\times (a\otimes [1]^\sharp)^\natural} && {Y(1)} \\
	{X(0)\times a^\natural} && {X(1)}
	\arrow[from=3-3, to=2-4]
	\arrow[from=1-3, to=2-4]
	\arrow[from=3-1, to=3-3]
	\arrow["{\Db \tilde{\phi}(0)\times a^\natural}", from=1-1, to=1-3]
	\arrow[from=1-1, to=2-2]
	\arrow[from=3-1, to=2-2]
	\arrow[from=2-2, to=2-4]
\end{tikzcd}
\end{equation}
According to proposition \ref{prop:lfib and W 3}, this corresponds to an object $\xi(\phi)$ of $\Lfib(\Noiun([a,1]\otimes[1]^\sharp)^\natural))$ endowed with two equivalences:
$$\partial_{[a^\natural,1]}E\sim \Noiun([a,1]\otimes\{0\})^*\xi(\phi)~~~~~~~
\partial_{[a^\natural,1]} F\sim \Noiun([a,1]\otimes\{1\})^*\xi(\phi)$$
Using the naturality of $\int_C$ demonstrated in proposition \ref{prop: derived int and partial are natural}, these equivalences induce equivalences:
\begin{equation}
\label{eq:fiber of xi}
E\sim ([a,1]\otimes\{0\})^*\int_{([a,1]\otimes[1]^\sharp)^\natural}\xi(\phi)~~~~~~~
 F\sim ([a,1]\otimes\{1\})^*\int_{([a,1]\otimes[1]^\sharp)^\natural}\xi(\phi)
\end{equation}
All the operations we performed were functorial and admitted inverses. 
We then have constructed an equivalence
\begin{equation}
\label{eq:inverse of ring partial}
\int_{([a,1]\otimes[1]^\sharp)^\natural}\xi:\Fun^c([1],\LCart([a,1]))\to \LCart(([a,1]\otimes[1]^\sharp)^\sharp)
\end{equation}
 
\begin{lemma}
\label{lemma:lax Gr construction technical}
There is a unique commutative square of shape
% https://q.uiver.app/#q=WzAsNixbMSwwLCJFXFxvdGltZXNcXHswXFx9Il0sWzEsMiwiRlxcb3RpbWVzIFxcezFcXH0iXSxbMSwxLCJcXGludF97KFthLDFdXFxvdGltZXNbMV1eXFxzaGFycCleXFxuYXR1cmFsfVxceGkoXFxwaGkpIl0sWzAsMSwiXFxpb3RhXyFcXGlvdGFeKkVcXG90aW1lcyBpZF97WzFdXlxcc2hhcnB9Il0sWzAsMCwiXFxpb3RhXyFcXGlvdGFeKkVcXG90aW1lcyBcXHswXFx9Il0sWzAsMiwiXFxpb3RhXyFcXGlvdGFeKkVcXG90aW1lc1xcezFcXH0iXSxbMCwyXSxbMSwyXSxbNCwzXSxbNSwzXSxbNSwxXSxbMywyXSxbNCwwXV0=
\begin{equation}
\label{eq:lemma:lax Gr construction technical}
\begin{tikzcd}
	{\iota_!\iota^*E\otimes \{0\}} & {E\otimes\{0\}} \\
	{\iota_!\iota^*E\otimes id_{[1]^\sharp}} & {\int_{([a,1]\otimes[1]^\sharp)^\natural}\xi(\phi)} \\
	{\iota_!\iota^*E\otimes\{1\}} & {F\otimes \{1\}}
	\arrow[from=1-2, to=2-2]
	\arrow[from=3-2, to=2-2]
	\arrow[from=1-1, to=2-1]
	\arrow[from=3-1, to=2-1]
	\arrow[from=3-1, to=3-2]
	\arrow[from=2-1, to=2-2]
	\arrow[from=1-1, to=1-2]
\end{tikzcd}
\end{equation}
where the upper horizontal morphism is induced by the unit of the adjunction $(\iota_!,\iota^*)$. Moreover, the bottom horizontal morphism is $\tilde\phi$.
\end{lemma}
\begin{proof}
The unicity and existence of the middle horizontal morphism come from the initiality of the morphism $\iota_!\iota^*E\otimes \{0\}\to \iota_!\iota^*E\otimes [1]^\sharp$. The unicity and existence of the lower horizontal morphism is a consequence of the equation \eqref{eq:fiber of xi}. 
As the diagram \eqref{eq:lax technical big diagram} factors as
% https://q.uiver.app/#q=WzAsOCxbMSwzLCJYKDEpIl0sWzIsMSwiWSgxKSJdLFsxLDAsIlkoMClcXHRpbWVzIGFeXFxuYXR1cmFsIl0sWzAsMywiWCgwKVxcdGltZXMgYV5cXG5hdHVyYWwiXSxbMCwxLCJYKDApXFx0aW1lcyBhXlxcbmF0dXJhbCJdLFsxLDIsIlgoMClcXHRpbWVzIChhXFxvdGltZXMgWzFdXlxcc2hhcnApXlxcbmF0dXJhbCJdLFsxLDEsIlgoMClcXHRpbWVzIGFeXFxuYXR1cmFsIl0sWzIsMiwiWCgwKVxcdGltZXMgKGFcXG90aW1lcyBbMV1eXFxzaGFycCleXFxuYXR1cmFsXFxjb3Byb2Rfe1goMClcXHRpbWVzIGFeXFxuYXR1cmFsfVgoMSkiXSxbMiwxXSxbMywwXSxbNCw1XSxbMyw1XSxbNCw2LCJcXERiIFxcdGlsZGV7XFxwaGl9KDApXFx0aW1lcyBhXlxcbmF0dXJhbCJdLFs1LDddLFswLDddLFs2LDJdLFs2LDddLFs3LDFdXQ==
\[\begin{tikzcd}
	& {Y(0)\times a^\natural} \\
	{X(0)\times a^\natural} & {X(0)\times a^\natural} & {Y(1)} \\
	& {X(0)\times (a\otimes [1]^\sharp)^\natural} & {X(0)\times (a\otimes [1]^\sharp)^\natural\coprod_{X(0)\times a^\natural}X(1)} \\
	{X(0)\times a^\natural} & {X(1)}
	\arrow[from=1-2, to=2-3]
	\arrow[from=4-1, to=4-2]
	\arrow[from=2-1, to=3-2]
	\arrow[from=4-1, to=3-2]
	\arrow["{\Db \tilde{\phi}(0)\times a^\natural}", from=2-1, to=2-2]
	\arrow[from=3-2, to=3-3]
	\arrow[from=4-2, to=3-3]
	\arrow[from=2-2, to=1-2]
	\arrow[from=2-2, to=3-3]
	\arrow[from=3-3, to=2-3]
\end{tikzcd}\]
the downer square of the diagram of \eqref{eq:lemma:lax Gr construction technical} factors as
% q.uiver.app/#q=WzAsNixbMiwxLCJGXFxvdGltZXMgXFx7MVxcfSJdLFsyLDAsIlxcaW50X3soW2EsMV1cXG90aW1lc1sxXV5cXHNoYXJwKV5cXG5hdHVyYWx9XFx4aShcXHBoaSkiXSxbMCwwLCJcXGlvdGFfIVxcaW90YV4qRVxcb3RpbWVzIFsxXV5cXHNoYXJwIl0sWzAsMSwiXFxpb3RhXyFcXGlvdGFeKkVcXG90aW1lc1xcezFcXH0iXSxbMSwxLCJcXExiIFxcaW90YV8hIFxcaW90YV4qRVxcb3RpbWVzXFx7MVxcfSJdLFsxLDAsIlxcaW50X3soW2EsMV1cXG90aW1lc1sxXV5cXHNoYXJwKV5cXG5hdHVyYWx9XFx4aShcXG11X0UpIl0sWzAsMV0sWzMsMl0sWzMsNF0sWzIsNV0sWzQsNV0sWzUsMV0sWzQsMCwiXFxEYiBcXHRpbGRle1xccGhpfSIsMl1d
\[\begin{tikzcd}
	{\iota_!\iota^*E\otimes [1]^\sharp} & {\int_{([a,1]\otimes[1]^\sharp)^\natural}\xi(\mu_E)} & {\int_{([a,1]\otimes[1]^\sharp)^\natural}\xi(\phi)} \\
	{\iota_!\iota^*E\otimes\{1\}} & {\Lb \iota_! \iota^*E\otimes\{1\}} & {F\otimes \{1\}}
	\arrow[from=2-3, to=1-3]
	\arrow[from=2-1, to=1-1]
	\arrow[from=2-1, to=2-2]
	\arrow[from=1-1, to=1-2]
	\arrow[from=2-2, to=1-2]
	\arrow[from=1-2, to=1-3]
	\arrow["{\Db \tilde{\phi}}"', from=2-2, to=2-3]
\end{tikzcd}\]
where $\mu_E$ denotes the canonical morphism $\iota_!\iota^*E\to \Lb \iota_! \iota^*E$. To conclude, one has to show that the lower left horizontal morphism is $\mu_E$. As these constructions are natural, and commute with the cartesian product with $B^\flat\to 1$ for $B$ an $\io$-category, the lemma \ref{lemma:characterisation of natural transoformation} implies the desired result.
\end{proof}

\begin{lemma}
\label{lemma:ring partial eq for a 1}
The functor $\ringpartial^c_{1,[a,1]}$ defined in \eqref{eq:def of right partial classified} in is an equivalence.
\end{lemma}
\begin{proof}
The lemma \ref{lemma:lax Gr construction technical} induces a diagram
% https://q.uiver.app/#q=WzAsNCxbMSwxLCIoXFxpb3RhXFxvdGltZXMgaWRfe1sxXX0pXipcXGludF97KFthLDFdXFxvdGltZXNbMV1eXFxzaGFycCleXFxuYXR1cmFsfVxceGkoXFxwaGkpIl0sWzAsMSwiXFxpb3RhXipGXFxvdGltZXNcXEZiIGhee1sxXX1fMSJdLFsxLDAsIlxcaW90YV4qRVxcb3RpbWVzXFxGYiBoXntbMV19XzAiXSxbMCwwLCJcXGlvdGFeKkVcXG90aW1lc1xcRmIgaF57WzFdfV8xIl0sWzEsMF0sWzMsMV0sWzMsMl0sWzIsMF1d
\[\begin{tikzcd}
	{\iota^*E\otimes\Fb h^{[1]}_1} & {\iota^*E\otimes\Fb h^{[1]}_0} \\
	{\iota^*F\otimes\Fb h^{[1]}_1} & {(\iota\otimes id_{[1]})^*\int_{([a,1]\otimes[1]^\sharp)^\natural}\xi(\phi)}
	\arrow[from=2-1, to=2-2]
	\arrow[from=1-1, to=2-1]
	\arrow[from=1-1, to=1-2]
	\arrow[from=1-2, to=2-2]
\end{tikzcd}\]
which corresponds to a natural transformation 
$$\oint_{1,[a,1]} \phi \to (\iota\otimes id_{[1]})^* \int_{([a,1]\otimes[1]^\sharp)^\natural}\xi(\phi)~~~\leftrightsquigarrow~~~ \phi\to \ringpartial^c_{1,[a,1]}\int_{([a,1]\otimes[1]^\sharp)^\natural}\xi(\phi)$$
Eventually, remark that proposition \ref{prop:ring partial is natural} and the equivalences \eqref{eq:fiber of xi} imply that this natural transformation is pointwise an equivalence. 
The functor \eqref{eq:inverse of ring partial} is then a left inverse of $\ringpartial^c_{1,[a,1]}$. As it is an equivalence, so is $\ringpartial^c_{1,[a,1]}$.
\end{proof}


\begin{prop}
\label{prop:ring partial eq for I n}
For any marked $\io$-category $I$, and integer $n$, the morphism 
$$\ringpartial^c_{n,I}:\LCart((I\otimes[n]^\sharp)^\sharp) \to \Fun^c([n],\LCart(I))$$
defined in \eqref{eq:def of right partial classified} is an equivalence. 
\end{prop}
\begin{proof}
Corollary \ref{cor:fib over a colimit2}, and propositions \ref{prop:otimes marked preserves colimits} and \ref{prop:Fun preserve colimies} imply that the two functors on $\Delta^{op}\times \ocatm^{op}$:
$$\begin{array}{rcl}
(n,I)&\mapsto & \LCartc(I\otimes[n]^\sharp)\\
(n,I)&\mapsto &\Fun^c([n],\LCartc(I))
\end{array}$$
send colimits to limits. We can then reduce to the case where $I$ is an element of $t\Theta$ and $n=1$. 
If $I$ is $[1]^\sharp$, remark that $\ringpartial^c_{n,[1]^\sharp}$ is equivalent to $\ringpartial_{n,[1]^\sharp}$ which is an equivalence according to proposition \ref{prop:lax gr construction particular case}.
If $I$ is of shape $[a,1]$ for $a$ in $t\Theta$, this is the content of lemma \ref{lemma:ring partial eq for a 1}.
\end{proof}



\p We recall that a left cartesian fibration is $\U$-small if its fibers are $\U$-small $\io$-categories. For an $\io$-category $A$, we denote by $\LCart_{\U}(A^\sharp)$ the full sub $\iun$-category of $\LCart_{\U}(A^\sharp)$ whose objects correspond to $\U$-small left cartesian fibrations over $A^\sharp$. For a marked $\io$-category $I$, we define similarly $\LCartc_{\U}(I)$ as the full sub $\iun$-category of $\LCartc_{\U}(I)$ whose objects correspond to $\U$-small classified left cartesian fibrations over $I$.

\begin{cor}
\label{cor:univalence}
Let $\uni$ be the $\V$-small $\io$-category of $\U$-small $\io$-categories.
Let $n$ be an integer and $I$ be a $\V$-small marked $\io$-category. We denote by $I^\sharp$ the marked $\io$-category obtained from $I$ by marking all cells, and $\iota:I\to I^\sharp$ the induced morphism. There is an equivalence, natural in $[n]:\Delta^{op}$ and $I:\ocatm^{op}$, between functors
$$f:I\otimes[n]^\sharp\to \uni^\sharp$$
and sequences
$$\iota^*\int_{I^\natural}f_0\to ... \to \iota^*\int_{I^\natural}f_n$$
where for any $k\leq n$, $f_k$ is the functor $I^\natural\to \uni$ induced by $I\otimes\{k\}\to I\otimes[n]^\sharp\to \uni^\sharp$.
\end{cor}
\begin{proof}
This is a direct application of the equivalence 
$$\tau_0\LCart((I\otimes[n]^\sharp)^\sharp) \to \Hom([n],\LCartc(I))$$
induced by proposition \ref{prop:ring partial eq for I n}.
\end{proof}

\begin{cor}
\label{cor:univalence tranche}
Let $I$ be a $\V$-small marked $\io$-category and $c$ an object of $\uni$. We denote by $I^\sharp$ the marked $\io$-category obtained from $I$ by marking all cells, and $\iota:I\to I^\sharp$ the induced morphism. There is an equivalence, natural in $I:\ocatm^{op}$, between functors
$$f:I\to \uni^\sharp_{c/}$$
and arrows:
$$I\times \int_1c\to \iota^* \int_{I^\natural}\tilde{f}$$
where $\tilde{f}$ is the induced functor $I^\natural\to \uni_{c/}\to\uni $.
\end{cor}
\begin{proof}
By construction, we have a cocartesian square.
% https://q.uiver.app/#q=WzAsNCxbMSwwLCJJXFxvdGltZXNbMV1eXFxzaGFycCJdLFswLDAsIklcXG90aW1lc1xcezBcXH0iXSxbMCwxLCIxIl0sWzEsMSwiMVxcY29zdGFyICBJIl0sWzEsMl0sWzIsM10sWzEsMF0sWzAsM10sWzMsMSwiIiwxLHsic3R5bGUiOnsibmFtZSI6ImNvcm5lciJ9fV1d
\[\begin{tikzcd}
	{I\otimes\{0\}} & {I\otimes[1]^\sharp} \\
	1 & {1\costar I}
	\arrow[from=1-1, to=2-1]
	\arrow[from=2-1, to=2-2]
	\arrow[from=1-1, to=1-2]
	\arrow[from=1-2, to=2-2]
	\arrow["\lrcorner"{anchor=center, pos=0.125, rotate=180}, draw=none, from=2-2, to=1-1]
\end{tikzcd}\]
As $\tau_0\LCart(\uvar)$ sends colimits to limits, this is a consequence of the last corollary.
\end{proof}


\begin{cor}
\label{cor:parametric univalence}
Let $n$ be an integer, $I$ a $\V$-small marked $\io$-category, and $A$ an $\io$-category. We denote by $I^\sharp$ the marked $\io$-category obtained from $I$ by marking all cells, and $\iota:I\to I^\sharp$ the induced morphism. There is an equivalence, natural in $[n]:\Delta^{op}$ and $I:\ocatm^{op}$, between functors
$$f:I\otimes[n]^\sharp\to \uHom(A,\uni)$$
and sequences
$$(\iota\times A^\sharp)^*\int_{I^\natural\times A}f_0\to ... \to (\iota\times A^\sharp)^*\int_{I^\natural\times A}f_n$$
where for any $k\leq n$, $f_k$ is the functor $I^\natural\times A\to \uni$ induced by $(I\otimes\{k\})\times A^\sharp\to (I\otimes[n]^\sharp)\times A^\sharp\to \uni^\sharp$.
\end{cor}
\begin{proof}
This is a direct application of the last corollary and the equivalence $(I\otimes[n]^\sharp)\times A^\sharp\sim (I\times A^\sharp)\otimes[n]^\sharp$ given in proposition \ref{prop:associativity of Gray2}.
\end{proof}



\begin{cor}
\label{cor:parametric univalence tranche}
Let $I$ be a $\V$-small marked $\io$-category, $A$ an $\io$-category, and $g$ an object of $\uHom(A,\uni)$. We denote by $I^\sharp$ the marked $\io$-category obtained from $I$ by marking all cells, and $\iota:I\to I^\sharp$ the induced morphism. There is an equivalence, natural in $I:\ocatm^{op}$, between functors
$$f:I\to \uHom(A,\uni)^\sharp_{g/}$$
and arrows:
$$I\times \int_Ag\to (\iota\times A^\sharp)^* \int_{I^\natural\times A}\tilde{f}$$
where $\tilde{f}:I^\natural\times A\to \uni$ is the functor corresponding to $I^\natural\to\uHom(A,\uni)_{g/}\to \uHom(A,\uni)$.
\end{cor}
\begin{proof}
We once again have a cocartesian square
% https://q.uiver.app/#q=WzAsNCxbMSwwLCJJXFxvdGltZXNbMV1eXFxzaGFycCJdLFswLDAsIklcXG90aW1lc1xcezBcXH0iXSxbMCwxLCIxIl0sWzEsMSwiMVxcY29zdGFyICBJIl0sWzEsMl0sWzIsM10sWzEsMF0sWzAsM10sWzMsMSwiIiwxLHsic3R5bGUiOnsibmFtZSI6ImNvcm5lciJ9fV1d
\[\begin{tikzcd}
	{I\otimes\{0\}} & {I\otimes[1]^\sharp} \\
	1 & {1\costar I}
	\arrow[from=1-1, to=2-1]
	\arrow[from=2-1, to=2-2]
	\arrow[from=1-1, to=1-2]
	\arrow[from=1-2, to=2-2]
	\arrow["\lrcorner"{anchor=center, pos=0.125, rotate=180}, draw=none, from=2-2, to=1-1]
\end{tikzcd}\]
As $\tau_0\LCart(\uvar)$ sends colimits to limits, this is a consequence of the last corollary and the equivalence $(I\otimes[1]^\sharp)\times A^\sharp\sim (I\times A^\sharp)\otimes[1]^\sharp$ given in proposition \ref{prop:associativity of Gray2}.
\end{proof}

\subsection{$\io$-Functorial Grothendieck construction}

\p For $I$ a marked $\io$-category and $A$ an $\io$-category, we define the $\io$-category \wcnotation{$\gHom(I,A)$}{(hom@$\gHom(\uvar,\uvar)$}, whose value on a globular sum $a$, is given by 
$$\Hom(a,\gHom(I,A)):=\Hom(I\ominus a^\sharp,A^\sharp)$$

The section is devoted to the proof of the following theorem:
\begin{theorem}
\label{theo:lcartc et ghom}
Let $I$ be a $\U$-small marked $\io$-category.
Let $\uni$ be the $\V$-small $\io$-category of $\U$-small $\io$-categories, and $\uLCartc_{\U}(I)$ the $\V$-small $\io$-category of $\U$-small left cartesian fibrations. 
There is an equivalence
$$\gHom(I,\uni)\sim \uLCartc_{\U}(I)$$
natural in $I$.
On the maximal sub $\infty$-groupoid, this equivalence corresponds to the Grothendieck construction of theorem \ref{theo:gr construction}.
\end{theorem}

\begin{cor}
\label{cor:lcar et hom}
Let $A$ be a $\U$-small $\io$-category.
Let $\uLCartc_{\U}(A^\sharp)$ be the $\V$-small $\io$-category of $\U$-small left cartesian fibrations. 
There is an equivalence
$$\uHom(A,\uni)\sim \uLCart_{\U}(A^\sharp)$$
natural in $A$.
On the maximal sub $\infty$-groupoid, this equivalence corresponds to the Grothendieck construction of theorem \ref{theo:gr construction}.
\end{cor}
\begin{proof}
This is a consequence of the equivalences $\uLCart(A^\sharp)\sim \uLCartc(A^\sharp)$, of the previous theorem and of the equivalence 
$\uHom(A,\uni)\sim \gHom(A^\sharp,\uni)$ induced by the second assertion of proposition \ref{prop:associativity of ominus}.
\end{proof}






\p 
\label{par: i pull and push beetwe io category of morphism}
The previous results provide equivalences \index[notation]{(f8@$f^*:\gHom(I,\uni)\to \gHom(J,\uni)$}
$$ \gHom(I,\uni)\sim \uLCartc(I) ~~~~\mbox{and}~~~~\uHom(A,\omega)\sim \uLCart(A^\sharp)$$
By construction, for any morphism $f:I\to J$ between marked $\omega$-categories, we have a morphism 
$$f^*:\gHom(J,\uni)\to \uHom(I,\uni)$$
Suppose now that the codomain of $f$ is of shape $A^\sharp$.
 The morphism \eqref{eq:i pull} induces a morphism \index[notation]{(f7@$f_{\mbox{$\exclam$}}:\gHom(I,\uni)\to \uHom(A,\uni)$}
$$f_!:\gHom(I,\uni)\to \uHom(A,\uni)$$ and \eqref{eq:i pull unit an counit} induces natural transformations:
$$
\mu:id\to f^*f_!~~~~ \epsilon:f_!f^*\to id
$$
coming along with equivalences:
$(\epsilon\circ_0 f_!)\circ_1(f_!\circ_0 \mu) \sim id_{f_!}$ and $(f^*\circ_0 \epsilon)\circ_1 (\mu \circ_0 f^* )\sim id_{f^*}$.
When $f$ is proper, the morphism \eqref{eq:i push op} induces a morphism \index[notation]{(f9@$f_*:\gHom(I,\uni)\to \uHom(A,\uni)$}
$$f_*:\gHom(I,\uni)\to \uHom(A,\uni)$$
 and \eqref{eq:i pull unit an counit op} induces natural transformations:
 $$
\mu: id\to f_*f^*~~~~ \epsilon:f^*f_*\to id
$$
coming along with equivalences:
$(\epsilon\circ_0 f^*)\circ_1(f^*\circ_0 \mu) \sim id_{f^*}$ and $(f_*\circ_0 \epsilon)\circ_1 (\mu \circ_0 f_* )\sim id_{f_*}$.
Moreover, for every morphism $j:C\to D^\sharp$, \eqref{eq:commutative pull push} 
induces a canonical commutative square
% https://q.uiver.app/#q=WzAsNCxbMCwwLCJcXGdIb20oRF5cXHNoYXJwXFx0aW1lcyBJLFxcdW5pKSJdLFswLDEsIlxcZ0hvbShDXlxcc2hhcnBcXHRpbWVzIEksXFx1bmkpIl0sWzEsMSwiXFx1SG9tKENcXHRpbWVzIEEsXFx1bmkpIl0sWzEsMCwiXFx1SG9tKERcXHRpbWVzIEEgLFxcdW5pKSJdLFswLDMsIiggaWRfe0ReXFxzaGFycH1cXHRpbWVzIGYpXyEiXSxbMCwxLCIoalxcdGltZXMgaWRfe0l9KV4qIiwyXSxbMSwyLCIoIGlkX3tDXlxcc2hhcnB9XFx0aW1lcyBmKV8hIiwyXSxbMywyLCIoalxcdGltZXMgaWRfe0FeXFxzaGFycH0pXioiXV0=
\[\begin{tikzcd}
	{\gHom(D^\sharp\times I,\uni)} & {\uHom(D\times A ,\uni)} \\
	{\gHom(C^\sharp\times I,\uni)} & {\uHom(C\times A,\uni)}
	\arrow["{( id_{D^\sharp}\times f)_!}", from=1-1, to=1-2]
	\arrow["{(j\times id_{I})^*}"', from=1-1, to=2-1]
	\arrow["{( id_{C^\sharp}\times f)_!}"', from=2-1, to=2-2]
	\arrow["{(j\times id_{A^\sharp})^*}", from=1-2, to=2-2]
\end{tikzcd}\]
and when $f$ is proper, \eqref{eq:commutative pull push op} induces a canonical commutative square
% https://q.uiver.app/#q=WzAsNCxbMCwwLCJcXGdIb20oRF5cXHNoYXJwXFx0aW1lcyBJLFxcdW5pKSJdLFswLDEsIlxcZ0hvbShDXlxcc2hhcnBcXHRpbWVzIEksXFx1bmkpIl0sWzEsMSwiXFx1SG9tKENcXHRpbWVzIEEsXFx1bmkpIl0sWzEsMCwiXFx1SG9tKERcXHRpbWVzIEEgLFxcdW5pKSJdLFswLDMsIiggaWRfe0ReXFxzaGFycH1cXHRpbWVzIGYpXyoiXSxbMCwxLCIoalxcdGltZXMgaWRfe0l9KV4qIiwyXSxbMSwyLCIoIGlkX3tDXlxcc2hhcnB9XFx0aW1lcyBmKV8qIiwyXSxbMywyLCIoalxcdGltZXMgaWRfe0FeXFxzaGFycH0pXioiXV0=
\[\begin{tikzcd}
	{\gHom(D^\sharp\times I,\uni)} & {\uHom(D\times A ,\uni)} \\
	{\gHom(C^\sharp\times I,\uni)} & {\uHom(C\times A,\uni)}
	\arrow["{( id_{D^\sharp}\times f)_*}", from=1-1, to=1-2]
	\arrow["{(j\times id_{I})^*}"', from=1-1, to=2-1]
	\arrow["{( id_{C^\sharp}\times f)_*}"', from=2-1, to=2-2]
	\arrow["{(j\times id_{A^\sharp})^*}", from=1-2, to=2-2]
\end{tikzcd}\]

\p 	
We now turn our attention back to the proof of the theorem \ref{theo:lcartc et ghom}.
\begin{lemma}
\label{lemma:lax univalence 0}
Let $I$ be a marked $\io$-category and $b^\flat$ a globular sum. We denote by $\pi_b:I\times b^\flat\to I$ the canonical projection.
There is an equivalence of $\iun$-categories:
$$\LCart(I\times b^\flat)\sim \LCart(I)_{/\pi_b}$$
\end{lemma}
\begin{proof}
Remark first that we have an equivalence 
$$(\ocatm_{/I})_{/\pi_b}\sim \ocatm_{/I\times b}$$
Now suppose given a triangle 
% https://q.uiver.app/#q=WzAsMyxbMCwxLCJYIl0sWzEsMCwiSVxcdGltZXMgYl5cXGZsYXQiXSxbMSwxLCJJIl0sWzAsMV0sWzEsMiwiXFxwaV9iIl0sWzAsMl1d
\[\begin{tikzcd}
	& {I\times b^\flat} \\
	X & I
	\arrow[from=2-1, to=1-2]
	\arrow["{\pi_b}", from=1-2, to=2-2]
	\arrow[from=2-1, to=2-2]
\end{tikzcd}\]
As left cartesian fibrations are stable by composition and right cancellation, and as $\pi_b$ is a left cartesian fibration,  the diagonal morphism is a left cartesian fibration if and only if the horizontal morphism is. 

The $\iun$-categories $\LCart(I)_{/\pi_b}$ and  $\LCart(I\times b^\flat)$ then identity with the same full sub $\iun$-category of $(\ocatm_{/I})_{/\pi_b}\sim \ocatm_{/I\times b}$.
\end{proof}



\begin{lemma}
\label{lemma:lax univalence 1}
There is a family of cartesian squares
% https://q.uiver.app/#q=WzAsNCxbMCwwLCJcXHRhdV8wXFxMQ2FydChbYVxcdGltZXMgYixuXV5cXHNoYXJwKSJdLFsxLDAsIlxcdGF1XzBcXExDYXJ0KFthLG5dXlxcc2hhcnBcXHRpbWVzIGJeXFxmbGF0KSJdLFsxLDEsIlxccHJvZF97a1xcbGVxIG59XFx0YXVfMFxcTENhcnQoXFx7a1xcfVxcdGltZXMgYl5cXGZsYXQpIl0sWzAsMSwiXFxwcm9kX3trXFxsZXEgbn1cXHRhdV8wXFxMQ2FydChcXHtrXFx9KSJdLFszLDJdLFsxLDJdLFswLDNdLFswLDFdXQ==
\[\begin{tikzcd}
	{\tau_0\LCart([a\times b,n]^\sharp)} & {\tau_0\LCart([a,n]^\sharp\times b^\flat)} \\
	{\prod_{k\leq n}\tau_0\LCart(\{k\})} & {\prod_{k\leq n}\tau_0\LCart(\{k\}\times b^\flat)}
	\arrow[from=2-1, to=2-2]
	\arrow[from=1-2, to=2-2]
	\arrow[from=1-1, to=2-1]
	\arrow[from=1-1, to=1-2]
\end{tikzcd}\]
natural in $a,b$ and $n$.
\end{lemma}
\begin{proof}
Remark first that the proposition
\ref{prop:crushing of Gray tensor is identitye marked case} provides cocartesian squares:
% https://q.uiver.app/#q=WzAsOCxbMSwxLCJbYVxcdGltZXMgYixuXV5cXHNoYXJwIl0sWzEsMCwiKGFeXFxmbGF0XFx0aW1lcyBiXlxcZmxhdClcXG90aW1lc1tuXV5cXHNoYXJwIl0sWzAsMCwiXFxjb3Byb2Rfe2tcXGxlcSBufShhXlxcZmxhdFxcdGltZXMgYl5cXGZsYXQpXFxvdGltZXNcXHtrXFx9Il0sWzAsMSwiXFxjb3Byb2Rfe2tcXGxlcSBufVxce2tcXH0iXSxbMywxLCJbYSxuXV5cXHNoYXJwIl0sWzMsMCwiYV5cXGZsYXRcXG90aW1lc1tuXV5cXHNoYXJwIl0sWzIsMCwiXFxjb3Byb2Rfe2tcXGxlcSBufWFeXFxmbGF0XFxvdGltZXNcXHtrXFx9Il0sWzIsMSwiXFxjb3Byb2Rfe2tcXGxlcSBufVxce2tcXH0iXSxbMiwzXSxbMywwXSxbMiwxXSxbMSwwXSxbMCwyLCIiLDAseyJzdHlsZSI6eyJuYW1lIjoiY29ybmVyIn19XSxbNyw0XSxbNSw0XSxbNiw3XSxbNiw1XSxbNCw2LCIiLDEseyJzdHlsZSI6eyJuYW1lIjoiY29ybmVyIn19XV0=
\[\begin{tikzcd}
	{\coprod_{k\leq n}(a^\flat\times b^\flat)\otimes\{k\}} & {(a^\flat\times b^\flat)\otimes[n]^\sharp} & {\coprod_{k\leq n}a^\flat\otimes\{k\}} & {a^\flat\otimes[n]^\sharp} \\
	{\coprod_{k\leq n}\{k\}} & {[a\times b,n]^\sharp} & {\coprod_{k\leq n}\{k\}} & {[a,n]^\sharp}
	\arrow[from=1-1, to=2-1]
	\arrow[from=2-1, to=2-2]
	\arrow[from=1-1, to=1-2]
	\arrow[from=1-2, to=2-2]
	\arrow["\lrcorner"{anchor=center, pos=0.125, rotate=180}, draw=none, from=2-2, to=1-1]
	\arrow[from=2-3, to=2-4]
	\arrow[from=1-4, to=2-4]
	\arrow[from=1-3, to=2-3]
	\arrow[from=1-3, to=1-4]
	\arrow["\lrcorner"{anchor=center, pos=0.125, rotate=180}, draw=none, from=2-4, to=1-3]
\end{tikzcd}\]
According to the corollary \ref{cor:fib over a colimit2}, and proposition \ref{prop:ring partial eq for I n}, and as $\Rb (\pi_{\uvar})_!:\LCart(1)\to \LCart(\uvar^\flat)$ factors through $\LCartc(\uvar^\flat)$, 
this induces cartesian squares:
% https://q.uiver.app/#q=WzAsOCxbMCwwLCJcXExDYXJ0KFthXFx0aW1lcyBiLG5dXlxcc2hhcnApIl0sWzEsMCwiXFxGdW4oW25dLFxcTENhcnQoYV5cXGZsYXRcXHRpbWVzIGJee1xcZmxhdH0pKSJdLFsxLDEsIlxccHJvZF97a1xcbGVxIG59XFxMQ2FydChhXlxcZmxhdFxcdGltZXMgYl57XFxmbGF0fSkiXSxbMCwxLCJcXHByb2Rfe2tcXGxlcSBufVxcTENhcnQoMSkiXSxbMiwwLCJcXExDYXJ0KFthLG5dXlxcc2hhcnApIl0sWzMsMCwiXFxGdW4oW25dLFxcTENhcnQoIGFee1xcZmxhdH0pKSJdLFszLDEsIlxccHJvZF97a1xcbGVxIG59XFxMQ2FydChhXlxcZmxhdCkiXSxbMiwxLCJcXHByb2Rfe2tcXGxlcSBufVxcTENhcnQoMSkiXSxbNCw3XSxbNyw2XSxbNCw1XSxbNSw2XSxbMCwxXSxbMCwzXSxbMywyXSxbMSwyXSxbMCwxNCwiIiwxLHsibGV2ZWwiOjEsInN0eWxlIjp7Im5hbWUiOiJjb3JuZXIifX1dLFs0LDksIiIsMSx7ImxldmVsIjoxLCJzdHlsZSI6eyJuYW1lIjoiY29ybmVyIn19XV0=
\begin{equation}
\label{eq:lemma:lax univalence 2}
\begin{tikzcd}[column sep = 0.3cm]
	{\LCart([a\times b,n]^\sharp)} & {\Fun([n],\LCart(a^\flat\times b^{\flat}))} & {\LCart([a,n]^\sharp)} & {\Fun([n],\LCart( a^{\flat}))} \\
	{\prod_{k\leq n}\LCart(1)} & {\prod_{k\leq n}\LCart(a^\flat\times b^{\flat})} & {\prod_{k\leq n}\LCart(1)} & {\prod_{k\leq n}\LCart(a^\flat)}
	\arrow[from=1-3, to=2-3]
	\arrow[""{name=0, anchor=center, inner sep=0}, from=2-3, to=2-4]
	\arrow[from=1-3, to=1-4]
	\arrow[from=1-4, to=2-4]
	\arrow[from=1-1, to=1-2]
	\arrow[from=1-1, to=2-1]
	\arrow[""{name=1, anchor=center, inner sep=0}, from=2-1, to=2-2]
	\arrow[from=1-2, to=2-2]
	\arrow["\lrcorner"{anchor=center, pos=0.125}, draw=none, from=1-1, to=1]
	\arrow["\lrcorner"{anchor=center, pos=0.125}, draw=none, from=1-3, to=0]
\end{tikzcd}
\end{equation}
For a marked $\io$-category $I$, we denote $\pi_b:I\times b\to I $ the canonical projection. As the $\iun$-categorical slice and the maximal full sub $\infty$-groupoid preserve cartesian squares,
the second cartesian square induces a cartesian square
% https://q.uiver.app/#q=WzAsNCxbMCwwLCJcXHRhdV8wKFxcTENhcnQoIFthLG5dXlxcc2hhcnApX3svXFxwaV9ifSkiXSxbMSwwLCJcXEhvbShbbl0sXFxMQ2FydCggYV57XFxmbGF0fSlfey9cXHBpX2J9KSJdLFsxLDEsIlxccHJvZF97a1xcbGVxIG59XFx0YXVfMChcXExDYXJ0KCBhXntcXGZsYXR9KV97L1xccGlfYn0pIl0sWzAsMSwiXFxwcm9kX3trXFxsZXEgbn1cXHRhdV8wKFxcTENhcnQoMSlfey9iXlxcZmxhdH0pIl0sWzAsM10sWzMsMl0sWzAsMV0sWzEsMl0sWzAsNSwiIiwxLHsibGV2ZWwiOjEsInN0eWxlIjp7Im5hbWUiOiJjb3JuZXIifX1dXQ==
\[\begin{tikzcd}
	{\tau_0(\LCart( [a,n]^\sharp)_{/\pi_b})} & {\Hom([n],\LCart( a^{\flat})_{/\pi_b})} \\
	{\prod_{k\leq n}\tau_0(\LCart(1)_{/b^\flat})} & {\prod_{k\leq n}\tau_0(\LCart( a^{\flat})_{/\pi_b})}
	\arrow[from=1-1, to=2-1]
	\arrow[""{name=0, anchor=center, inner sep=0}, from=2-1, to=2-2]
	\arrow[from=1-1, to=1-2]
	\arrow[from=1-2, to=2-2]
	\arrow["\lrcorner"{anchor=center, pos=0.125}, draw=none, from=1-1, to=0]
\end{tikzcd}\]
and according to  lemma \ref{lemma:lax univalence 0}, this corresponds to  a cartesian square
% https://q.uiver.app/#q=WzAsNCxbMCwwLCJcXHRhdV8wXFxMQ2FydChbYSxuXV5cXHNoYXJwXFx0aW1lcyBiXlxcZmxhdCkiXSxbMSwwLCJcXEhvbShbbl0sXFxMQ2FydCggYV57XFxmbGF0fVxcdGltZXMgYl5cXGZsYXQpKSJdLFsxLDEsIlxccHJvZF97a1xcbGVxIG59XFx0YXVfMFxcTENhcnQoIGFee1xcZmxhdH1cXHRpbWVzIGJeXFxmbGF0KSJdLFswLDEsIlxccHJvZF97a1xcbGVxIG59XFx0YXVfMFxcTENhcnQoICBiXlxcZmxhdCkiXSxbMCwzXSxbMywyXSxbMCwxXSxbMSwyXSxbMCw1LCIiLDEseyJsZXZlbCI6MSwic3R5bGUiOnsibmFtZSI6ImNvcm5lciJ9fV1d
\[\begin{tikzcd}
	{\tau_0\LCart([a,n]^\sharp\times b^\flat)} & {\Hom([n],\LCart( a^{\flat}\times b^\flat))} \\
	{\prod_{k\leq n}\tau_0\LCart(  b^\flat)} & {\prod_{k\leq n}\tau_0\LCart( a^{\flat}\times b^\flat)}
	\arrow[from=1-1, to=2-1]
	\arrow[""{name=0, anchor=center, inner sep=0}, from=2-1, to=2-2]
	\arrow[from=1-1, to=1-2]
	\arrow[from=1-2, to=2-2]
	\arrow["\lrcorner"{anchor=center, pos=0.125}, draw=none, from=1-1, to=0]
\end{tikzcd}\]
Combined with the first cartesian square of \eqref{eq:lemma:lax univalence 2}, this induces a commutative diagram
% https://q.uiver.app/#q=WzAsNixbMSwwLCJcXHRhdV8wXFxMQ2FydChbYSxuXV5cXHNoYXJwXFx0aW1lcyBiXlxcZmxhdCkiXSxbMiwwLCJcXEhvbShbbl0sXFxMQ2FydCggYV57XFxmbGF0fVxcdGltZXMgYl5cXGZsYXQpKSJdLFsyLDEsIlxccHJvZF97a1xcbGVxIG59XFx0YXVfMFxcTENhcnQoIGFee1xcZmxhdH1cXHRpbWVzIGJeXFxmbGF0KSJdLFsxLDEsIlxccHJvZF97a1xcbGVxIG59XFx0YXVfMFxcTENhcnQoICBiXlxcZmxhdCkiXSxbMCwxLCJcXHByb2Rfe2tcXGxlcSBufVxcdGF1XzBcXExDYXJ0KCAgMSkiXSxbMCwwLCJcXHRhdV8wXFxMQ2FydChbYVxcdGltZXMgYiwgbl1eXFxzaGFycCkiXSxbMCwzXSxbMywyXSxbMCwxXSxbMSwyXSxbNCwzXSxbNSw0XSxbNSwwXSxbMCw3LCIiLDEseyJsZXZlbCI6MSwic3R5bGUiOnsibmFtZSI6ImNvcm5lciJ9fV1d
\[\begin{tikzcd}
	{\tau_0\LCart([a\times b, n]^\sharp)} & {\tau_0\LCart([a,n]^\sharp\times b^\flat)} & {\Hom([n],\LCart( a^{\flat}\times b^\flat))} \\
	{\prod_{k\leq n}\tau_0\LCart(  1)} & {\prod_{k\leq n}\tau_0\LCart(  b^\flat)} & {\prod_{k\leq n}\tau_0\LCart( a^{\flat}\times b^\flat)}
	\arrow[from=1-2, to=2-2]
	\arrow[""{name=0, anchor=center, inner sep=0}, from=2-2, to=2-3]
	\arrow[from=1-2, to=1-3]
	\arrow[from=1-3, to=2-3]
	\arrow[from=2-1, to=2-2]
	\arrow[from=1-1, to=2-1]
	\arrow[from=1-1, to=1-2]
	\arrow["\lrcorner"{anchor=center, pos=0.125}, draw=none, from=1-2, to=0]
\end{tikzcd}\]
where the right and the outer square are cartesian. By right cancellation, the left square is cartesian which concludes the proof.

\end{proof}


\begin{lemma}
\label{lemma:lax univalence 2.5}
Let $b$ be a globular sum and let  $F:I\to \ocat$ be a $\Wcard$-small diagram. 
The canonical morphism
$$\LCart(\colim_IF^\sharp\times b^\flat) \to \lim_I \LCart(F^\sharp\times b^\flat)$$
is an equivalence.
\end{lemma}
\begin{proof}
The corollary \ref{cor:fib over a colimit2} implies that the canonical morphism
$$\LCart(\colim_IF^\sharp) \to \lim_I \LCart(F^\sharp)$$
is an equivalence.
We recall that for any $\io$-category $A$, we denote by $\pi_b:A^\sharp\times b^\flat\to A^\sharp$ the canonical projection. As the $\iun$-categorical slice preserves limits, the previous equivalence induces an equivalence
$$\LCart(\colim_IF^\sharp)_{/\pi_b} \to \lim_I \LCart(F^\sharp)_{/\pi_b}.$$
The results then follows from lemma \ref{lemma:lax univalence 0}.
\end{proof}


\begin{lemma}
\label{lemma:lax univalence 2}
There is a family of cartesian squares
% https://q.uiver.app/#q=WzAsNCxbMCwwLCJcXHRhdV8wXFxMQ2FydCgoSVxcb21pbnVzW2Isbl1eXFxzaGFycCleXFxzaGFycCkiXSxbMSwwLCJcXHRhdV8wXFxMQ2FydCgoSVxcb3RpbWVzW25dXlxcc2hhcnApXlxcc2hhcnBcXHRpbWVzIGJeXFxmbGF0KSJdLFsxLDEsIlxccHJvZF97a1xcbGVxIG59XFx0YXVfMFxcTENhcnQoKEleXFxzaGFycFxcb3RpbWVzXFx7a1xcfSlcXHRpbWVzIGJeXFxmbGF0KSJdLFswLDEsIlxccHJvZF97a1xcbGVxIG59XFx0YXVfMFxcTENhcnQoSV5cXHNoYXJwXFxvdGltZXNcXHtrXFx9KSJdLFszLDJdLFsxLDJdLFswLDNdLFswLDFdXQ==
\[\begin{tikzcd}
	{\tau_0\LCart((I\ominus[b,n]^\sharp)^\sharp)} & {\tau_0\LCart((I\otimes[n]^\sharp)^\sharp\times b^\flat)} \\
	{\prod_{k\leq n}\tau_0\LCart(I^\sharp\otimes\{k\})} & {\prod_{k\leq n}\tau_0\LCart((I^\sharp\otimes\{k\})\times b^\flat)}
	\arrow[from=2-1, to=2-2]
	\arrow[from=1-2, to=2-2]
	\arrow[from=1-1, to=2-1]
	\arrow[from=1-1, to=1-2]
\end{tikzcd}\]
natural in $I,b$ and $n$.
\end{lemma}
\begin{proof}
By definition, $(I\ominus [b,n]^\sharp)^\sharp$ fits in the following cartesian square:
% https://q.uiver.app/#q=WzAsNCxbMSwxLCIoSVxcb21pbnVzIFtiLG5dXlxcc2hhcnApXlxcc2hhcnAiXSxbMCwxLCJcXGNvbGltX3tbYSxtXVxcdG8gKElcXG90aW1lc1tuXV5cXHNoYXJwKV5cXG5hdHVyYWx9W2FcXHRpbWVzIGIsbV1eXFxzaGFycCJdLFswLDAsIlxcY29saW1fe1thLG1dXFx0byBcXGFtYWxnX2tJXlxcbmF0dXJhbFxcb3RpbWVzIFxce2tcXH19W2FcXHRpbWVzIGIsbV1eXFxzaGFycCJdLFsxLDAsIlxcY29saW1fe1thLG1dXFx0byBcXGFtYWxnX2tJXlxcbmF0dXJhbFxcb3RpbWVzIFxce2tcXH19W2EsbV1eXFxzaGFycCJdLFsyLDFdLFsxLDBdLFszLDBdLFsyLDNdLFswLDcsIiIsMCx7ImxldmVsIjoxLCJzdHlsZSI6eyJuYW1lIjoiY29ybmVyIn19XV0=
\[\begin{tikzcd}
	{\colim_{[a,m]\to \amalg_kI^\natural\otimes \{k\}}[a\times b,m]^\sharp} & {\colim_{[a,m]\to \amalg_kI^\natural\otimes \{k\}}[a,m]^\sharp} \\
	{\colim_{[a,m]\to (I\otimes[n]^\sharp)^\natural}[a\times b,m]^\sharp} & {(I\ominus [b,n]^\sharp)^\sharp}
	\arrow[from=1-1, to=2-1]
	\arrow[from=2-1, to=2-2]
	\arrow[from=1-2, to=2-2]
	\arrow[""{name=0, anchor=center, inner sep=0}, from=1-1, to=1-2]
	\arrow["\lrcorner"{anchor=center, pos=0.125, rotate=180}, draw=none, from=2-2, to=0]
\end{tikzcd}\]
Combined with corollary \ref{cor:fib over a colimit2}, this implies that the $\infty$-groupoid $\tau_0\LCart((I\ominus [b,n]^\sharp)^\sharp)$ fits in the cartesian square:
% https://q.uiver.app/#q=WzAsNCxbMCwwLCJcXHRhdV8wXFxMQ2FydGMoKElcXG9taW51cyBbYixuXV5cXHNoYXJwKV5cXHNoYXJwKSJdLFswLDEsIlxcbGltX3tbYSxtXVxcdG8gKElcXG90aW1lc1tuXV5cXHNoYXJwKV5cXG5hdHVyYWx9XFx0YXVfMFxcTENhcnQoW2FcXHRpbWVzIGIsbV1eXFxzaGFycCkiXSxbMSwxLCJcXGxpbV97W2EsbV1cXHRvIFxcYW1hbGdfa0leXFxuYXR1cmFsXFxvdGltZXMgXFx7a1xcfX1cXHRhdV8wXFxMQ2FydChbYVxcdGltZXMgYixtXV5cXHNoYXJwKSJdLFsxLDAsIlxcbGltX3tbYSxtXVxcdG8gXFxhbWFsZ19rSV5cXG5hdHVyYWxcXG90aW1lcyBcXHtrXFx9fVxcdGF1XzBcXExDYXJ0KFthLG1dXlxcc2hhcnApIl0sWzEsMl0sWzAsMV0sWzAsM10sWzMsMl0sWzAsNCwiIiwwLHsibGV2ZWwiOjEsInN0eWxlIjp7Im5hbWUiOiJjb3JuZXIifX1dXQ==
\[\begin{tikzcd}
	{\tau_0\LCartc((I\ominus [b,n]^\sharp)^\sharp)} & {\lim_{[a,m]\to \amalg_kI^\natural\otimes \{k\}}\tau_0\LCart([a,m]^\sharp)} \\
	{\lim_{[a,m]\to (I\otimes[n]^\sharp)^\natural}\tau_0\LCart([a\times b,m]^\sharp)} & {\lim_{[a,m]\to \amalg_kI^\natural\otimes \{k\}}\tau_0\LCart([a\times b,m]^\sharp)}
	\arrow[""{name=0, anchor=center, inner sep=0}, from=2-1, to=2-2]
	\arrow[from=1-1, to=2-1]
	\arrow[from=1-1, to=1-2]
	\arrow[from=1-2, to=2-2]
	\arrow["\lrcorner"{anchor=center, pos=0.125}, draw=none, from=1-1, to=0]
\end{tikzcd}\]
Applying lemma \ref{lemma:lax univalence 1}, and the fact that any morphism $\{l\}\to [a,m]\to (I\otimes[n]^\sharp)^\natural$ uniquely factors through $\coprod_{k}I^\natural \otimes\{k\}$,
we get a cartesian square
% https://q.uiver.app/#q=WzAsNCxbMCwwLCJcXHRhdV8wXFxMQ2FydGMoKElcXG9taW51cyBbYixuXV5cXHNoYXJwKV5cXHNoYXJwKSJdLFswLDEsIlxcbGltX3tbYSxtXVxcdG8gKElcXG90aW1lc1tuXV5cXHNoYXJwKV5cXG5hdHVyYWx9XFx0YXVfMFxcTENhcnQoW2EsbV1eXFxzaGFycFxcdGltZXMgYl5cXGZsYXQpIl0sWzEsMSwiXFxsaW1fe1thLG1dXFx0byBcXGFtYWxnX2tJXlxcbmF0dXJhbFxcb3RpbWVzIFxce2tcXH19XFx0YXVfMFxcTENhcnQoW2EsbV1eXFxzaGFycFxcdGltZXMgYl5cXGZsYXQpIl0sWzEsMCwiXFxsaW1fe1thLG1dXFx0byBcXGFtYWxnX2tJXlxcbmF0dXJhbFxcb3RpbWVzIFxce2tcXH19XFx0YXVfMFxcTENhcnQoW2EsbV1eXFxzaGFycCkiXSxbMSwyXSxbMCwxXSxbMCwzXSxbMywyXSxbMCw0LCIiLDAseyJsZXZlbCI6MSwic3R5bGUiOnsibmFtZSI6ImNvcm5lciJ9fV1d
\[\begin{tikzcd}
	{\tau_0\LCartc((I\ominus [b,n]^\sharp)^\sharp)} & {\lim_{[a,m]\to \amalg_kI^\natural\otimes \{k\}}\tau_0\LCart([a,m]^\sharp)} \\
	{\lim_{[a,m]\to (I\otimes[n]^\sharp)^\natural}\tau_0\LCart([a,m]^\sharp\times b^\flat)} & {\lim_{[a,m]\to \amalg_kI^\natural\otimes \{k\}}\tau_0\LCart([a,m]^\sharp\times b^\flat)}
	\arrow[""{name=0, anchor=center, inner sep=0}, from=2-1, to=2-2]
	\arrow[from=1-1, to=2-1]
	\arrow[from=1-1, to=1-2]
	\arrow[from=1-2, to=2-2]
	\arrow["\lrcorner"{anchor=center, pos=0.125}, draw=none, from=1-1, to=0]
\end{tikzcd}\]
Eventually, the lemma \ref{lemma:lax univalence 2.5} induces equivalences
$$\begin{array}{rll}
\lim_{[a,m]\to (I\otimes[n]^\sharp)^\natural}\tau_0\LCart([a,m]^\sharp\times b^\flat)&\sim&\tau_0 \LCart((I\otimes[n]^\sharp)^\sharp\times b^\flat)\\
\lim_{[a,m]\to I^\natural}\tau_0\LCart([a,m]^\sharp\times b^\flat)&\sim&\tau_0 \LCart(I^\sharp\times b^\flat)\\
\lim_{[a,m]\to I^\natural}\tau_0\LCart([a,m]^\sharp)&\sim &\tau_0\LCart(I^\sharp)
\end{array}
$$
This concludes the proof.
\end{proof}

\begin{lemma}
\label{lemma:lax univalence 3}
There is a family of cartesian squares
% https://q.uiver.app/#q=WzAsNCxbMCwwLCJcXEhvbShbbl0sIFxcTENhcnRjKEk7YikpIl0sWzEsMCwiXFx0YXVfMFxcTENhcnQoKElcXG90aW1lc1tuXV5cXHNoYXJwKV5cXHNoYXJwXFx0aW1lcyBiXlxcZmxhdCkiXSxbMSwxLCJcXHByb2Rfe2tcXGxlcSBufVxcdGF1XzBcXExDYXJ0KChJXlxcc2hhcnBcXG90aW1lc1xce2tcXH0pXFx0aW1lcyBiXlxcZmxhdCkiXSxbMCwxLCJcXHByb2Rfe2tcXGxlcSBufVxcdGF1XzBcXExDYXJ0KEleXFxzaGFycFxcb3RpbWVzXFx7a1xcfSkiXSxbMywyXSxbMSwyXSxbMCwzXSxbMCwxXV0=
\[\begin{tikzcd}
	{\Hom([n], \LCartc(I;b))} & {\tau_0\LCart((I\otimes[n]^\sharp)^\sharp\times b^\flat)} \\
	{\prod_{k\leq n}\tau_0\LCart(I^\sharp\otimes\{k\})} & {\prod_{k\leq n}\tau_0\LCart((I^\sharp\otimes\{k\})\times b^\flat)}
	\arrow[from=2-1, to=2-2]
	\arrow[from=1-2, to=2-2]
	\arrow[from=1-1, to=2-1]
	\arrow[from=1-1, to=1-2]
\end{tikzcd}\]
natural in $I,b$ and $n$.
\end{lemma}
\begin{proof}
By the construction of $\LCartc(I;b)$, we have a cartesian square 
% https://q.uiver.app/#q=WzAsNCxbMCwwLCJcXEhvbShbbl0sIFxcTENhcnRjKEk7YikpIl0sWzEsMCwiXFxIb20oW25dLCBcXExDYXJ0KElcXHRpbWVzIGJeXFxmbGF0KSkiXSxbMSwxLCJcXHByb2Rfe2tcXGxlcSBufSBcXHRhdV8wXFxMQ2FydChJXFx0aW1lcyBiXlxcZmxhdCkiXSxbMCwxLCJcXHByb2Rfe2tcXGxlcSBufSBcXHRhdV8wXFxMQ2FydGMoSSkiXSxbMywyXSxbMSwyXSxbMCwzXSxbMCwxXSxbMCw0LCIiLDIseyJsZXZlbCI6MSwic3R5bGUiOnsibmFtZSI6ImNvcm5lciJ9fV1d
\[\begin{tikzcd}
	{\Hom([n], \LCartc(I;b))} & {\Hom([n], \LCart(I\times b^\flat))} \\
	{\prod_{k\leq n} \tau_0\LCartc(I)} & {\prod_{k\leq n} \tau_0\LCart(I\times b^\flat)}
	\arrow[""{name=0, anchor=center, inner sep=0}, from=2-1, to=2-2]
	\arrow[from=1-2, to=2-2]
	\arrow[from=1-1, to=2-1]
	\arrow[from=1-1, to=1-2]
	\arrow["\lrcorner"{anchor=center, pos=0.125}, draw=none, from=1-1, to=0]
\end{tikzcd}\]
According to lemma \ref{lemma:lax univalence 0}, this induces a cartesian square 
% https://q.uiver.app/#q=WzAsNCxbMCwwLCJcXEhvbShbbl0sIFxcTENhcnRjKEk7YikpIl0sWzAsMSwiXFxwcm9kX3trXFxsZXEgbn0gXFx0YXVfMFxcTENhcnRjKEkpIl0sWzEsMSwiXFxwcm9kX3trXFxsZXEgbn1cXHRhdV8wIChcXExDYXJ0KEkpX3svXFxwaV9ifSkiXSxbMSwwLCJcXEhvbShbbl0sIFxcTENhcnQoSSlfey9cXHBpX2J9KSJdLFswLDFdLFswLDNdLFsxLDJdLFszLDJdLFswLDYsIiIsMix7ImxldmVsIjoxLCJzdHlsZSI6eyJuYW1lIjoiY29ybmVyIn19XV0=
\[\begin{tikzcd}
	{\Hom([n], \LCartc(I;b))} & {\Hom([n], \LCart(I)_{/\pi_b})} \\
	{\prod_{k\leq n} \tau_0\LCartc(I)} & {\prod_{k\leq n}\tau_0 (\LCart(I)_{/\pi_b})}
	\arrow[from=1-1, to=2-1]
	\arrow[from=1-1, to=1-2]
	\arrow[""{name=0, anchor=center, inner sep=0}, from=2-1, to=2-2]
	\arrow[from=1-2, to=2-2]
	\arrow["\lrcorner"{anchor=center, pos=0.125}, draw=none, from=1-1, to=0]
\end{tikzcd}\]
As the functor $\LCartc(I)\to \LCart(I)_{/\pi_b}$ factors through $\LCartc(I)_{/\pi_b}$, the proposition 
\ref{prop:ring partial eq for I n} induces a cartesian square
% https://q.uiver.app/#q=WzAsNCxbMCwwLCJcXEhvbShbbl0sIFxcTENhcnRjKEk7YikpIl0sWzEsMCwiXFx0YXVfMChcXExDYXJ0KChJXFxvdGltZXNbbl1eXFxzaGFycCleXFxzaGFycClfey9cXHBpX2J9KSJdLFsxLDEsIlxccHJvZF97a1xcbGVxIG59XFx0YXVfMChcXExDYXJ0KEleXFxzaGFycFxcb3RpbWVzXFx7a1xcfSlfey9cXHBpX2J9KSJdLFswLDEsIlxccHJvZF97a1xcbGVxIG59XFx0YXVfMFxcTENhcnQoSV5cXHNoYXJwXFxvdGltZXNcXHtrXFx9KSJdLFszLDJdLFsxLDJdLFswLDNdLFswLDFdLFswLDQsIiIsMix7ImxldmVsIjoxLCJzdHlsZSI6eyJuYW1lIjoiY29ybmVyIn19XV0=
\[\begin{tikzcd}
	{\Hom([n], \LCartc(I;b))} & {\tau_0(\LCart((I\otimes[n]^\sharp)^\sharp)_{/\pi_b})} \\
	{\prod_{k\leq n}\tau_0\LCart(I^\sharp\otimes\{k\})} & {\prod_{k\leq n}\tau_0(\LCart(I^\sharp\otimes\{k\})_{/\pi_b})}
	\arrow[""{name=0, anchor=center, inner sep=0}, from=2-1, to=2-2]
	\arrow[from=1-2, to=2-2]
	\arrow[from=1-1, to=2-1]
	\arrow[from=1-1, to=1-2]
	\arrow["\lrcorner"{anchor=center, pos=0.125}, draw=none, from=1-1, to=0]
\end{tikzcd}\]
Eventually, a last application of lemma \ref{lemma:lax univalence 0} concludes the proof.
\end{proof}


\begin{lemma}
\label{lemma:lax univalence 4}
There is an equivalence 
$$\tau_0(\LCart((I\ominus [b,n]^\sharp)^\sharp) \sim \Hom([n],\LCartc(I;b))$$
natural in $I:\ocatm^{op}$, $b:\Theta^{op}$ and $[n]:\Delta^{op}$.
\end{lemma}
\begin{proof}
This is a direct consequence of lemmas \ref{lemma:lax univalence 2} and \ref{lemma:lax univalence 3}.
\end{proof}


\begin{proof}[Proof of theorem \ref{theo:lcartc et ghom}]
Lemma \ref{lemma:lax univalence 4} provides an natural equivalence 
$$\tau_0(\LCart((I\ominus [b,n]^\sharp)^\sharp) \sim \Hom([n],\LCartc(I;b))$$
that preserves smallness.
\end{proof}



\section{Yoneda lemma and applications}
\subsection{Yoneda lemma}
\p An $\io$-category $C$ is \wcnotion{locally $\U$-small}{locally $\U$-small $\io$-category} if for any pair of objects $x$ and $y$, $\hom_C(x,y)$ is $\U$-small. 

\begin{example}
\label{exe:Hom uni is lsm}
For all $\U$-small $\io$-category $A$, the corollary \ref{cor:lcar et hom} provides an equivalence
$$\hom_{\uHom(A,\uni)}(f,g)\sim \Map(\int_Af,\int_Ag)$$
As $\int_Af$ and $\int_Ag$ are $\U$-small left cartesian fibrations over a $\U$-small basis, their codomains are $\U$-small and $\Map(\int_Af,\int_Ag)$ is then $\U$-small. The $\io$-category $\uHom(A,\uni)$ is then locally $\U$-small.
\end{example}
We can generalize this example as follow:
\begin{prop}
\label{prop:when Hom A B is locally small}
Let $A$ be a $\U$-small $\io$-category, and $C$ is a locally $\U$-small $\io$-category. The $\io$-category $\uHom(A,C)$ is locally $\U$-small. 
\end{prop}
\begin{proof}
We have to check that for any globular sum $b$, the morphism 
$$\Hom(A\times [b,1],C)\to \Hom(A\times (\{0\}\amalg\{1\}),C)$$
has $\U$-small fibers. As $A$, seen as an $\infty$-presheaves on $\Theta$, is a $\U$-small colimit of representables, we can reduce to the case where $A\in \Theta$. As $C$ is local with respect to Segal extensions, and as the cartesian product conserves them, we can reduce to the case where $A$ is of shape $[a,1]$ for $a$ a globular sum. We now fix a morphism $f:[a,1]\times (\{0\}\amalg\{1\})\to C$. Eventually, using the canonical equivalence between $[a,1]\times [b,1]$ and the colimit of the span
$$[a,1]\vee[b,1]\leftarrow [a\times b,1]\to [b,1]\vee[a,1],$$
the $\infty$-groupoid $\Hom([a,1]\times [b,1],C)_f$ fits in a cartesian square:
% https://q.uiver.app/#q=WzAsNCxbMCwwLCJcXEhvbShbYSwxXVxcdGltZXMgW2IsMV0sQylfZiJdLFsxLDAsIlxcSG9tKGIsXFxob20oZigwLDApLGYoMCwxKSkpIl0sWzEsMSwiXFxIb20oYVxcdGltZXMgYixcXGhvbShmKDAsMCksZigxLDEpKSkiXSxbMCwxLCJcXEhvbShiLFxcaG9tKGYoMSwwKSxmKDEsMSkpKSJdLFszLDJdLFswLDNdLFswLDFdLFsxLDJdXQ==
\[\begin{tikzcd}
	{\Hom([a,1]\times [b,1],C)_f} & {\Hom(b,\hom(f(0,0),f(0,1)))} \\
	{\Hom(b,\hom(f(1,0),f(1,1)))} & {\Hom(a\times b,\hom(f(0,0),f(1,1)))}
	\arrow[from=2-1, to=2-2]
	\arrow[from=1-1, to=2-1]
	\arrow[from=1-1, to=1-2]
	\arrow[from=1-2, to=2-2]
\end{tikzcd}\]
As all these objects are $\U$-small by assumption, this concludes the proof.
\end{proof}

\p Let $C$ be an $\io$-category $C$. We define the simplicial object $S(\Noiun C)$ by the formula
$$S(\Noiun C)_n:= \coprod_{x_0,...,x_n:A_0} \coprod_{y_0,...,y_n:A_0}\hom_C(x_n,...,x_0,y_0,...,y_n)$$
This object comes along with a canonical projection 
\begin{equation}
\label{eq:definition of the hom0}
S(\Noiun C)\to \Noiun{C^t}\times \Noiun C.
\end{equation}
which obviously is a left fibration. As this construction if functorial, it induces a functor:
$$\begin{array}{rcl}
\ocat &\to &\Arr(\ouncat)\\
C&\mapsto & (S(\Noiun C)\to \Noiun{C^t}\times \Noiun C)
\end{array}$$

\p Through this section, we fix a locally $\U$-small $\io$-category $C$. 
The left fibration \eqref{eq:definition of the hom0} is then $\U$-small, and by definition of $\uni$, this induces a morphism
\begin{equation}
\label{eq:definition of the hom}
\hom_C(\uvar,\uvar):C^t\times C\to \uni
\end{equation}
Using the canonical equivalence
$$\Fb h^{C^t\times C}_{(x,y)}\sim \Fb h^{C^t}_{x}\times \Fb h^{C}_{y}$$
the corresponding left cartesian fibration is then the colimit of a simplicial object whose value on $n$ is given by:
$$\coprod_{x_0,...,x_n}\coprod_{y_0,...,y_n} \Fb h^{C^t}_{x_n}\times \hom_{C}(x_n,...,x_0,y_0,...,y_n)^\flat\times \Fb h^{C}_{y_n}$$
\p We define the \wcnotion{$\io$-category of $\io$-presheaves on $C$}{presheaves@$\io$-presheaves} \sym{(c@$\w{C}$}:$$\w{C}:=\uHom(C^t,\uni ).$$ This $\io$-category is locally $\U$-small according to proposition \ref{prop:when Hom A B is locally small}. The \notion{Yoneda embedding}\sym{(y@$y_{\uvar}$} $y: C\to \w{C}$ is the functor induced by the hom functor \eqref{eq:definition of the hom} by currying.

An $\io$-presheaves is \wcnotion{representable}{representable $\io$-presheaves} if it is in the image of $y$.

\p We recall that for a subset $S$ of $\Nb^*$, and an object $X$ of $\ouncat$, we denote by $X^S$ the simplicial object $n\mapsto X_n^S$. We also set $\Sigma S:=\{i+1,i\in S\}$. We then have equivalences
$$(\Noiun C)^S\sim \Noiun (C^{\Sigma C}) ~~~\mbox{and}~~~ S(\Noiun C))^S\sim S(\Noiun (C^{\Sigma C}))$$
For an object $X$ of $\ouncat$, we denote by $X_{op}$ the simplicial object $n\mapsto X_{n^{op}}$. We then have equivalences 
$$(\Noiun C)_{op}\sim \Noiun (C^{t}) ~~~\mbox{and}~~~ S(\Noiun C))_{op}\sim S(\Noiun (C^t))$$
Using the dualities defined in paragraph \ref{par:dualities fo omega}, we then have commutative diagrams
% https://q.uiver.app/#q=WzAsNyxbMCwwLCIoQ157dFxcU2lnbWEgU31cXHRpbWVzIENee1xcU2lnbWEgU30pXntcXFNpZ21hIFN9Il0sWzIsMCwiXFx1bmlee1xcU2lnbWEgU30iXSxbMCwxLCJDXnRcXHRpbWVzIEMiXSxbMiwxLCJcXHVuaSJdLFs0LDAsIkNcXHRpbWVzIENedCJdLFs0LDEsIkNedFxcdGltZXMgQyJdLFs1LDEsIlxcdW5pIl0sWzIsMywiXFxob21fe0N9IiwyXSxbMCwxLCJcXGhvbV57XFxTaWdtYSBTfV97Q157XFxTaWdtYSBTfX0iXSxbMCwyLCJcXHNpbSIsMl0sWzEsMywiKFxcdXZhcileUyJdLFs1LDYsIlxcaG9tX0MiLDJdLFs0LDUsIlxcdHciLDJdLFs0LDYsIlxcaG9tX3tDXnR9Il1d
\[\begin{tikzcd}
	{(C^{t\Sigma S}\times C^{\Sigma S})^{\Sigma S}} && {\uni^{\Sigma S}} && {C\times C^t} \\
	{C^t\times C} && \uni && {C^t\times C} & \uni
	\arrow["{\hom_{C}}"', from=2-1, to=2-3]
	\arrow["{\hom^{\Sigma S}_{C^{\Sigma S}}}", from=1-1, to=1-3]
	\arrow["\sim"', from=1-1, to=2-1]
	\arrow["{(\uvar)^S}", from=1-3, to=2-3]
	\arrow["{\hom_C}"', from=2-5, to=2-6]
	\arrow["\tw"', from=1-5, to=2-5]
	\arrow["{\hom_{C^t}}", from=1-5, to=2-6]
\end{tikzcd}\]
where $\tw$ is the functor exchanging the argument. This two diagram corresponds to the natural transformations
$$\hom_{C^{\Sigma S}}(x,y)\sim \hom_{C}(x,y)^S~~~\mbox{and}~~~\hom_{C^t}(x,y)\sim \hom_{C}(y,x).$$

In combining the two previous diagrams, we get a commutative square:
% https://q.uiver.app/#q=WzAsNCxbMCwwLCIoQ157XFxjaXJjIHR9XFx0aW1lcyBDXntcXGNpcmN9KV57XFxjaXJjIHR9Il0sWzIsMCwiXFx1bmlee3tcXGNpcmMgdH19Il0sWzAsMSwiQ150XFx0aW1lcyBDIl0sWzIsMSwiXFx1bmkiXSxbMiwzLCJcXGhvbV97Q30iLDJdLFswLDEsIlxcaG9tXntcXGNpcmMgdH1fe0NeXFxjaXJjfSJdLFswLDIsIlxcdHciLDJdLFsxLDMsIihcXHV2YXIpXlxcY2lyYyJdXQ==
\[\begin{tikzcd}
	{(C^{\circ t}\times C^{\circ})^{\circ t}} && {\uni^{{\circ t}}} \\
	{C^t\times C} && \uni
	\arrow["{\hom_{C}}"', from=2-1, to=2-3]
	\arrow["{\hom^{\circ t}_{C^\circ}}", from=1-1, to=1-3]
	\arrow["\tw"', from=1-1, to=2-1]
	\arrow["{(\uvar)^\circ}", from=1-3, to=2-3]
\end{tikzcd}\]
corresponding to the natural transformation
$$\hom_{C^\circ}(x,y)\sim \hom_{C}(y,x)^\circ.$$
\begin{prop}
\label{prop:Yoneda is Fb} 
Let $A$ be an locally $\U$-small $\io$-category.
Let $a$ be an object of $A$.
There is an equivalence
$$\int_{A}\hom_A(a,\uvar)\to \Fb h^{A}_a$$
Taking the fibers on $a$, the induced morphism $\hom_A(a,a)\to \hom_A(a,a)$ preserves the identity.
 In particular, for any object $c$ of $C$, this induces an equivalence
$$\int_{C^t}y_c\to \Fb h^{C^t}_c$$
\end{prop}
\begin{proof}
By construction, $\int_{A}\hom_A(a,\uvar)$ is the Grothendieck construction of the left fibration:
% https://q.uiver.app/#q=WzAsOCxbMywwLCJcXGNvcHJvZF97eF8wOkFfMH1cXGhvbV97QX0oYSx4XzApIl0sWzMsMSwiXFxjb3Byb2Rfe3hfMDpBXzB9MSJdLFsyLDEsIlxcY29wcm9kX3t4XzAseF8xOkFfMH1cXGhvbV97QX0oeF8wLHhfMSkiXSxbMSwxLCJcXGNvcHJvZF97eF8wLHhfMSx4XzI6QV8wfVxcaG9tX3tBfSh4XzAseF8xLHhfMikiXSxbMCwxLCJcXGNkb3RzIl0sWzIsMCwiXFxjb3Byb2Rfe3hfMCx4XzE6QV8wfVxcaG9tX3tBfShhLHhfMCx4XzEpIl0sWzEsMCwiXFxjb3Byb2Rfe3hfMCx4XzEseF8yOkFfMH1cXGhvbV97QX0oYSx4XzAseF8xLHhfMikiXSxbMCwwLCJcXGNkb3RzIl0sWzMsMiwiIiwyLHsib2Zmc2V0Ijo0fV0sWzMsMiwiIiwwLHsib2Zmc2V0IjotNH1dLFszLDJdLFsyLDMsIiIsMSx7Im9mZnNldCI6LTJ9XSxbMiwzLCIiLDEseyJvZmZzZXQiOjJ9XSxbMiwxLCIiLDEseyJvZmZzZXQiOi0yfV0sWzIsMSwiIiwxLHsib2Zmc2V0IjoyfV0sWzEsMl0sWzUsMl0sWzAsMV0sWzUsMCwiIiwxLHsib2Zmc2V0IjotMn1dLFswLDVdLFs1LDAsIiIsMSx7Im9mZnNldCI6Mn1dLFs2LDUsIiIsMSx7Im9mZnNldCI6NH1dLFs2LDVdLFs2LDUsIiIsMSx7Im9mZnNldCI6LTR9XSxbNSw2LCIiLDEseyJvZmZzZXQiOjJ9XSxbNSw2LCIiLDEseyJvZmZzZXQiOi0yfV0sWzYsM11d
\[\begin{tikzcd}[column sep =0.4cm]
	\cdots & {\coprod_{x_0,x_1,x_2:A_0}\hom_{A}(a,x_0,x_1,x_2)} & {\coprod_{x_0,x_1:A_0}\hom_{A}(a,x_0,x_1)} & {\coprod_{x_0:A_0}\hom_{A}(a,x_0)} \\
	\cdots & {\coprod_{x_0,x_1,x_2:A_0}\hom_{A}(x_0,x_1,x_2)} & {\coprod_{x_0,x_1:A_0}\hom_{A}(x_0,x_1)} & {\coprod_{x_0:A_0}1}
	\arrow[shift right=4, from=2-2, to=2-3]
	\arrow[shift left=4, from=2-2, to=2-3]
	\arrow[from=2-2, to=2-3]
	\arrow[shift left=2, from=2-3, to=2-2]
	\arrow[shift right=2, from=2-3, to=2-2]
	\arrow[shift left=2, from=2-3, to=2-4]
	\arrow[shift right=2, from=2-3, to=2-4]
	\arrow[from=2-4, to=2-3]
	\arrow[from=1-3, to=2-3]
	\arrow[from=1-4, to=2-4]
	\arrow[shift left=2, from=1-3, to=1-4]
	\arrow[from=1-4, to=1-3]
	\arrow[shift right=2, from=1-3, to=1-4]
	\arrow[shift right=4, from=1-2, to=1-3]
	\arrow[from=1-2, to=1-3]
	\arrow[shift left=4, from=1-2, to=1-3]
	\arrow[shift right=2, from=1-3, to=1-2]
	\arrow[shift left=2, from=1-3, to=1-2]
	\arrow[from=1-2, to=2-2]
\end{tikzcd}\]
The results then follow from the corollary \ref{cor:antecedant of slice}.
\end{proof}





\p The identity $\w{C}\to \w{C}$ induces by currying a canonical morphism \sym{(ev@$\ev$}
$$\ev: C^t\times \w{C}\to \uni$$
called the \textit{evaluation functor}. Given an object $c$ of $C$ and $f$ of $\widehat{C}$, we then have $\ev(c,f)\sim f(c)$ and so 
$$(c,\{f\})^*\int_{C\times \w{C}}\ev\sim c^*\int_{C^t}f$$

Let $E$ be an object of $\ocatm_{/\w{C}^\sharp}$ corresponding to a morphism $g:X\to \w{C}^\sharp$.
We denote $\iota:X\to X^\sharp$ the canonical inclusion.
 A morphism 
 $$E\to \int_{\w{C}}\ev(c,\uvar)$$
  corresponds by adjunction to a morphism 
\begin{equation}
\label{eq:evaluation 1}
id_X\to g^*\int_{\w{C}}\ev(c,\uvar)
\end{equation}
However, we have a canonical commutative square
% https://q.uiver.app/#q=WzAsNCxbMCwwLCJYXlxcbmF0dXJhbCJdLFsxLDAsIlxcd3tDfSJdLFsxLDEsIlxcdW5pIl0sWzAsMSwiWF5cXG5hdHVyYWxcXHRpbWVzIEMiXSxbMSwyLCJcXGV2KGMsXFx1dmFyKSJdLFswLDEsImdeXFxuYXR1cmFsIl0sWzMsMiwiXFx0aWxkZXtnfSIsMl0sWzAsMywiWF5cXG5hdHVyYWxcXHRpbWVzIFxce2NcXH0iLDJdXQ==
\[\begin{tikzcd}
	{X^\natural} & {\w{C}} \\
	{X^\natural\times C^t} & \uni
	\arrow["{\ev(c,\uvar)}", from=1-2, to=2-2]
	\arrow["{g^\natural}", from=1-1, to=1-2]
	\arrow["{\tilde{g}}"', from=2-1, to=2-2]
	\arrow["{X^\natural\times \{c\}}"', from=1-1, to=2-1]
\end{tikzcd}\]
where $\tilde{g}$ is the morphism defined by currying from $g^\natural:X^\natural\to \w{C}$. Using the naturality of the Grothendieck construction, the previous commutative square implies that the data of \eqref{eq:evaluation 1} corresponds to a morphism
$$id_X\to (\iota\times \{c\})^* \int_{X^\natural\times C^t}\tilde{g}$$
an by adjunction, to a morphism 
$$
X\times \Fb h_c^{C^t}\to (\iota\times (C^t)^\sharp)^*\int_{X^\natural\times C^t}\tilde{g}
$$
We then have constructed an equivalence 
\begin{equation}
\label{eq:evaluation 3}
\Hom(E, \int_{\w{C}}\ev(c,\uvar))\sim \Hom(X\times \Fb h_c^{C^t}, (\iota\times (C^t)^\sharp)^*\int_{X^\natural\times C^t}\tilde{g})
\end{equation}
natural in $E$.

Remark furthermore that if $E$ is $h^{\w{C}}_f$ for $f$ an object of $\w{C}$, the equivalence corresponds to the canonical equivalences
$$\begin{array}{rcl}
\Hom_{\ocatm_{/\w{C}^\sharp}}(h^{\w{C}}_f, \int_{\w{C}}\ev(c,\uvar))&\sim &\Hom_{\ocatm}(1,\{f\}^*\int_{\w{C}}\ev(c,\uvar))\\
&\sim& \Hom_{\ocatm}(1,c^*\int_{C^t}f)\\
&\sim& \Hom_{\ocatm_{/(C^t)^\sharp}}( \Fb h_c^{C^t},\int_{C^t}f)
\end{array}$$

\begin{prop}
\label{prop:un fonctorial Yoneda}
For any object $c$ of $C$, there exists a unique pair consisting of a morphism
$$\int_{\w{C}} \hom_{\w{C}}(y_c,\uvar)\to \int_{\w{C}}\ev(c,\uvar)$$
and a commutative square of shape
% https://q.uiver.app/#q=WzAsNCxbMSwxLCJcXGhvbV9DKGMsYylcXHNpbSBcXHt5X2NcXH1eKiBcXGludF97XFx3e0N9fVxcZXYoYyxcXHV2YXIpIl0sWzEsMCwiXFxob21fe1xcd2lkZWhhdHtDfX0oeV9jLHlfYylcXHNpbSBcXHt5X2NcXH1eKiBcXGludF97XFx3e0N9fVxcaG9tX3tcXHd7Q319KHlfYyxcXHV2YXIpIl0sWzAsMCwiXFx7aWRfe3lfY31cXH0iXSxbMCwxLCJcXHtpZF9jXFx9Il0sWzIsMywiIiwxLHsibGV2ZWwiOjIsInN0eWxlIjp7ImhlYWQiOnsibmFtZSI6Im5vbmUifX19XSxbMiwxXSxbMywwXSxbMSwwXV0=
\begin{equation}
\label{eq:prop:un fonctorial Yoneda}
\begin{tikzcd}
	{\{id_{y_c}\}} & {\hom_{\widehat{C}}(y_c,y_c)\sim \{y_c\}^* \int_{\w{C}}\hom_{\w{C}}(y_c,\uvar)} \\
	{\{id_c\}} & {\hom_C(c,c)\sim \{y_c\}^* \int_{\w{C}}\ev(c,\uvar)}
	\arrow[Rightarrow, no head, from=1-1, to=2-1]
	\arrow[from=1-1, to=1-2]
	\arrow[from=2-1, to=2-2]
	\arrow[from=1-2, to=2-2]
\end{tikzcd}
\end{equation}
Moreover, this comparison morphism is an equivalence.
\end{prop}
\begin{proof}
The proposition \ref{prop:Yoneda is Fb} implies that $\int_{\w{C}}\hom_{\w{C}}(y_c,\uvar)$ is equivalent to $\Fb h^{\w{C}}_{y_c}$. A natural transformation $\int_{\w{C}}\hom_{\widehat{C}}(y_c,\uvar)\to g$ then corresponds to a morphism 
$\Fb h^{\w{C}}_{y_c}\to \int_{\w{C}}g$ and is then uniquely characterized by the value on $\{id_{y_c}\}$, which proves the uniqueness.

It remains to show the existence.
Let $E$ be an object of $\ocatm_{/\w{C}^\sharp}$ corresponding to a morphism $g:X\to \w{C}^\sharp$ . We denote $\iota:X\to X^\sharp$ the canonical inclusion. According to proposition \ref{prop:Yoneda is Fb}, a morphism $E\to \int_{\w{C}}\hom_{\widehat{C}}(y_c,\uvar)$ corresponds to a morphism $E\to \Fb h^{\w{C}}_{y_c}$, and so to a triangle
% https://q.uiver.app/#q=WzAsMyxbMCwxLCJYIl0sWzEsMSwiXFx3e0N9Xlxcc2hhcnAiXSxbMSwwLCJcXHd7Q31eXFxzaGFycF97eV9jL30iXSxbMCwyXSxbMiwxXSxbMCwxXV0=
\[\begin{tikzcd}
	& {\w{C}^\sharp_{y_c/}} \\
	X & {\w{C}^\sharp}
	\arrow[from=2-1, to=1-2]
	\arrow[from=1-2, to=2-2]
	\arrow[from=2-1, to=2-2]
\end{tikzcd}\]
According to corollary \ref{cor:parametric univalence tranche}, this data is equivalent to the one of 
$$X\times \int_{C^t}y_c\to (\iota\times (C^t)^\sharp)^*\int_{X^\natural\times C^t}\tilde{g}$$
where $\tilde{g}$ is the morphism defined by currying from $g^\natural:X^\natural\to \w{C}$. The proposition \ref{prop:Yoneda is Fb}, and 
 the equivalence \eqref{eq:evaluation 3} then induce an equivalence:
 $$\Hom_{\ocatm_{/\w{C}^\sharp}}(E, \int_{\w{C}}\hom_{\w{C}}(y_c,\uvar))\sim \Hom_{\ocatm_{/\w{C}^\sharp}}(E,\int_{\w{C}}\ev(c,\uvar))$$
Walking through all the equivalences, we can easily see that when $E$ is $h^{\w{C}}_{y_c}$, this equivalence sends the upper horizontal morphism of \eqref{eq:prop:un fonctorial Yoneda} to the lower horizontal one.
We then have an equivalence
$$\int_{\w{C}} \hom_{\w{C}}(y_c,\uvar)\sim \int_{\w{C}}\ev(c,\uvar).$$
that comes along with the desired commutative square.
\end{proof}
	
	

\begin{theorem}
\label{theo:Yoneda ff}
The Yoneda embedding is fully faithful. As a consequence, every morphism $A\to \w{C}$ that is pointwise representable uniquely factors through the Yoneda embedding.
\end{theorem}
\begin{proof}
We fix an object $c$ of $C$.
By construction of the Yoneda embedding and the evaluation, we have an equivalence $\ev(c,y_d)\sim \hom_C(c,d)$ natural in $d:C$.  Applying the Grothendieck deconstruction to the equivalence given in proposition \ref{prop:un fonctorial Yoneda}, we then get an equivalence 
$$\eta_d:\hom_{\widehat{C}}(y_c,y_d)\sim \hom_C(c,d)$$
natural in $d:C$ and that preserves the identity. 

We also have a transformation 
$$\hom_{y}(c,d):\hom_{C}(c,d)\to \hom_{\w{C}}(y_c,y_d)$$
natural in $d:C$, that also preserves the identity. 
We then have constructed a natural transformation
$$\psi_{c,d}:\hom_C(c,d)\xrightarrow{\hom_{y}(c,d)} \hom_{\w{C}}(y_{c},y_{d})\xrightarrow{\eta_d}\hom_C(c,d)$$
natural in $d:C$, and which preserves the identity. As the Grothendieck construction of $\hom_{C}(c,\uvar)$ is $\Fb h^C_c$ according to proposition \ref{prop:Yoneda is Fb}, the morphism
$$\int_C\psi_{c}:\Fb h^C_c\to \Fb h^C_c$$ 
is characterized by its value on $\{id_c\}$ and is then the identity. This implies that $\psi_c$ is the identity.
By two out of three, this implies that $\hom_{y}(c,\uvar)$ also is an equivalence, which concludes the proof.
\end{proof}






\begin{lemma}
\label{lemma:a particular Kan extension}
Let $i:C\to D$ be a morphism between locally $\U$-small $\io$-categories.
The canonical morphism of $\LCart((C^t)^\sharp\times D^\sharp)$:
$$\Lb(id\times i)_!\int_{C^t\times C}\hom_{C} \to \int_{C^t\times D}\hom_D(i(\uvar),\uvar)$$
is an equivalence.
\end{lemma}
\begin{proof}
Let $c,d$ be any objects of respectively $C$ and $D$.  We then have equivalences
$$\begin{array}{rcll}
\Rb (c,d)^* \Lb (id\times i)_!\int_{ C^t\times C^t}\hom_{ C}&\sim&\Rb \{d\}^*  \Lb i_! \Rb (id\times \{c\})^*\int_{ C^t\times C}\hom_{C}& (\ref{prop:BC condition})\\
&\sim& \Rb \{d\}^* \Lb i_!\Fb h^{C}_{c} &(\ref{prop:Yoneda is Fb})\\
&\sim & \Rb \{d\}^* \Fb h^{D}_{i(c)}\\
&\sim & \hom_D(i(c),d)^\flat
\end{array}$$
Remark that we also have an equivalence 
$$\Rb (c,d)^*\int_{C^t\times D}\hom_D(i(\uvar),\uvar)\sim \hom_D(i(c),d)^\flat$$
and that the induced endomorphism of $ \hom_D(i(c),d)^\flat$ is the identity. As equivalences are detected pointwise, this concludes the proof.
\end{proof}




\begin{theorem}
\label{theo:Yoneda lemma}
Let $C$ be a locally $\U$-small $\io$-category. There is an equivalence between the functor
$$\hom_{\w{C}}(y_{\uvar},\uvar):C^t\times \w{C}\to \uni$$ and
the functor 
$$\ev:C^t\times \w{C}\to \uni.$$
Restricted to $\w{C}\times \{c\}$ for $c$ an object of $C$, this equivalence is the one of proposition \ref{prop:un fonctorial Yoneda}.
\end{theorem}
\begin{proof}
The triangle
% https://q.uiver.app/#q=WzAsMyxbMSwxLCJcXHd7Q30iXSxbMCwxLCJcXHd7Q30iXSxbMCwwLCJDIl0sWzIsMCwieSJdLFsyLDEsInkiLDJdLFsxLDAsImlkIiwyXV0=
\[\begin{tikzcd}
	C \\
	{\w{C}} & {\w{C}}
	\arrow["y", from=1-1, to=2-2]
	\arrow["y"', from=1-1, to=2-1]
	\arrow["id"', from=2-1, to=2-2]
\end{tikzcd}\] 
induces by adjunction a triangle
% https://q.uiver.app/#q=WzAsMyxbMSwxLCJcXHd7Q30iXSxbMCwxLCJDXnRcXHRpbWVzXFx3e0N9Il0sWzAsMCwiQ150XFx0aW1lcyBDIl0sWzIsMCwiXFxob20iXSxbMiwxLCJpZFxcdGltZXMgeSIsMl0sWzEsMCwiXFxldiIsMl1d
\[\begin{tikzcd}
	{C^t\times C} \\
	{C^t\times\w{C}} & {\w{C}}
	\arrow["\hom", from=1-1, to=2-2]
	\arrow["{id\times y}"', from=1-1, to=2-1]
	\arrow["\ev"', from=2-1, to=2-2]
\end{tikzcd}\]
This corresponds to an equivalence
$$\int_{C^t\times C}\hom_{C}(\uvar,\uvar)\to (id\times y)^*\int_{C^t\times \w{C}}\ev.$$
By naturality, for any object $c$ of $C$, the pullback of the previous equivalence along $C^t\times\{c\}$ is the identity. In particular, the induced morphism $\hom(c,c)\to \hom(c,c)$ between the fibers over $(c,c)$ preserves the object $\{id_c\}$. According to lemma \ref{lemma:a particular Kan extension}, the previous equivalence induces a morphism
\begin{equation}
\label{eq:proof of yoneda}
\int_{C^t\times \w{C}}\hom_{\w{C}}(y_{\uvar},\uvar)\to \int_{C^t\times \w{C}}\ev.
\end{equation}
that comes along, by construction, with a commutative square
% https://q.uiver.app/#q=WzAsNCxbMSwxLCJcXGhvbV9DKGMsYylcXHNpbSBcXHt5X2NcXH1eKiBcXGludF97XFx3e0N9fVxcZXYoYyxcXHV2YXIpIl0sWzEsMCwiXFxob21fe1xcd2lkZWhhdHtDfX0oeV9jLHlfYylcXHNpbSBcXHt5X2NcXH1eKiBcXGludF97XFx3e0N9fVxcaG9tX3tcXHd7Q319KHlfYyxcXHV2YXIpIl0sWzAsMCwiXFx7aWRfe3lfY31cXH0iXSxbMCwxLCJcXHtpZF9jXFx9Il0sWzIsMywiIiwxLHsibGV2ZWwiOjIsInN0eWxlIjp7ImhlYWQiOnsibmFtZSI6Im5vbmUifX19XSxbMiwxXSxbMywwXSxbMSwwXV0=
\[\begin{tikzcd}
	{\{id_{y_c}\}} & {\hom_{\widehat{C}}(y_c,y_c)\sim \{y_c\}^* \int_{\w{C}}\hom_{\w{C}}(y_c,\uvar)} \\
	{\{id_c\}} & {\hom_C(c,c)\sim \{y_c\}^* \int_{\w{C}}\ev(c,\uvar)}
	\arrow[Rightarrow, no head, from=1-1, to=2-1]
	\arrow[from=1-1, to=1-2]
	\arrow[from=2-1, to=2-2]
	\arrow[from=1-2, to=2-2]
\end{tikzcd}\]
for any object $c$ of $C$. The restriction of the morphism \eqref{eq:proof of yoneda} to $\w{C}\times \{c\}$ is then equivalent to the natural transformation given in proposition \ref{prop:un fonctorial Yoneda}, and is an equivalence. As equivalences between left cartesian fibrations are detected on fibers, this concludes the proof.
\end{proof}


\begin{cor}
\label{cor: universal fibration 2}
The universal left cartesian fibration with $\U$-small fibers is the canonical projection 
$\uni^\sharp_{1/}\to \uni^\sharp$.
\end{cor}
\begin{proof}
The corollary \ref{cor: universal fibration 2} implies that universal left cartesian fibration with $\U$-small fibers is $\int_{\uni}id$. The Yoneda lemma implies that this left cartesian fibration is equivalent to $\int_{\uni}\hom_{\uni}(1,\uvar)$. Eventually, the proposition \ref{prop:Yoneda is Fb} states that this left cartesian fibration is equivalent to $\uni^\sharp_{1/}\to \uni^\sharp$.
\end{proof}






\subsection{Adjoint functors}

\begin{definition}
Let $C$ and $D$ be two locally $\U$-small $\io$-categories and $u:C\to D,$ $v:D\to C$ two functors. An \notion{adjoint structure} for the pair $(u,v)$ is the data of a invertible natural transformation
$$\phi: \hom_D(u(\uvar),\uvar)\sim \hom_C(\uvar,v(\uvar))$$
In this case, $u$ is a \wcnotion{left adjoint}{left or right adjoint} of $v$ and $v$ is a \textit{right adjoint} of $u$.
\end{definition}


\begin{prop}
\label{prop:adj if slice as terminal}
Let $u:C\to D$ be a functor between locally $\U$-small $\io$-categories. 
For $b$ an object of $D$, we define $(C^t)^\sharp_{b/}$ and $C^\sharp_{b/}$ as the marked $\io$-categories fitting in the cartesian squares:
% https://q.uiver.app/#q=WzAsOCxbMiwxLCJDXlxcc2hhcnAiXSxbMywxLCJEXlxcc2hhcnAiXSxbMywwLCJEXlxcc2hhcnBfe2IvfSJdLFsyLDAsIkNeXFxzaGFycF97Yi99Il0sWzEsMCwiKERedCleXFxzaGFycF97Yi99Il0sWzEsMSwiKERedCleXFxzaGFycCJdLFswLDEsIihDXnQpXlxcc2hhcnAiXSxbMCwwLCIoQ150KV5cXHNoYXJwX3svYn0iXSxbMywwXSxbMCwxLCJ1IiwyXSxbMiwxXSxbMywxLCIiLDEseyJzdHlsZSI6eyJuYW1lIjoiY29ybmVyIn19XSxbMywyXSxbNyw2XSxbNyw0XSxbNCw1XSxbNiw1LCJ1XnQiLDJdLFs3LDUsIiIsMSx7InN0eWxlIjp7Im5hbWUiOiJjb3JuZXIifX1dXQ==
\[\begin{tikzcd}
	{(C^t)^\sharp_{/b}} & {(D^t)^\sharp_{b/}} & {C^\sharp_{b/}} & {D^\sharp_{b/}} \\
	{(C^t)^\sharp} & {(D^t)^\sharp} & {C^\sharp} & {D^\sharp}
	\arrow[from=1-3, to=2-3]
	\arrow["u"', from=2-3, to=2-4]
	\arrow[from=1-4, to=2-4]
	\arrow["\lrcorner"{anchor=center, pos=0.125}, draw=none, from=1-3, to=2-4]
	\arrow[from=1-3, to=1-4]
	\arrow[from=1-1, to=2-1]
	\arrow[from=1-1, to=1-2]
	\arrow[from=1-2, to=2-2]
	\arrow["{u^t}"', from=2-1, to=2-2]
	\arrow["\lrcorner"{anchor=center, pos=0.125}, draw=none, from=1-1, to=2-2]
\end{tikzcd}\]
The following are equivalent.
\begin{enumerate}
\item The functor $u$ admits a right adjoint.
\item For any element $b$ of $D$, the marked $\io$-category $(C^t)^\sharp_{b/}$ 
admits an initial element.
\end{enumerate}
Similarly, the following are equivalent.
\begin{enumerate}
\item[(1)'] The functor $u$ admits a left adjoint.
\item[(2)'] For any element $b$ of $D$, $C^\sharp_{b/}$ admits an initial element.
\end{enumerate}
\end{prop}
\begin{proof}
Suppose first that $(1)$ is fulfilled, and let $v:D\to C$ be a functor and $\phi:\hom(u(a),b)\sim\hom(a,v(b))$ be an invertible natural transformation. In particular, this implies that we have an equivalence
$$\int_{C^t\times D}\hom_D(u(a),b)\sim \int_{C^t\times D}\hom_C(a,v(b))$$
Pulling back along $C^t\times \{b\}$ where $b$ is any object of $D$, we get an equivalence between 
$(C^t)^\sharp_{b/}$ and $(C^t)^\sharp_{v(b)/}$. As this last marked $\io$-category admits an initial element, given by the image $id_{v(b)}$, this shows the implication $(1)\Rightarrow (2)$.

For the converse, suppose that $u$ fulfills condition $(2)$. The functor $\hom_D(u(\uvar),\uvar)):C^t\times D\to \uni$ corresponds by adjonction to a functor $v':D\to \w{C}$. By assumption, for any $b\in B$, $v'(b)$ is a representable $\io$-presheaf. The Yoneda lemma then implies that $v$ factors through a functor $v:D\to C$. Using once again Yoneda lemma, we have a sequence of equivalences
$$\hom_D(u(a),b)\sim v'(b)(a)\sim \hom_C(b,v(a)).$$

The equivalence between $(1)'$ and $(2)'$ is proved similarly.
\end{proof}



\p
 Let $(u,v,\phi)$ be an adjoint structure. There is a transformation 
$$\hom_C(a,a')\to \hom_D(u(a),u(a'))\to \hom_C(a,vu(a'))$$
natural in $a:C^t$, $a':C$. According to the Yoneda lemma, this corresponds to a natural transformation $\mu: id_C \to vu$, called the \wcnotion{unit of the adjunction}{unit and counit of an adjunction}. Similarly, the natural transformation:
$$\hom_D(b,b')\to \hom_C(v(b),v(b'))\to \hom_C(uv(b),b')$$
induces a natural transformation $\epsilon:uv\to id_D$, called \textit{the counit of the adjunction.}

\begin{lemma}
\label{lemma:naturality of hom apply to natural transformation}
Suppose we have two morphisms $f:C\to D$ and $g:C\to D$ between locally $\U$-small $\io$-categories as well as a natural transformation $\nu:f\to g$. This induces a commutative diagram 
% q.uiver.app/#q=WzAsNCxbMCwwLCJcXGhvbV9DKGEsYikiXSxbMCwxLCJcXGhvbV9EKGYoYSksZihiKSkiXSxbMSwwLCJcXGhvbV9EKGcoYSksZyhiKSkiXSxbMSwxLCJcXGhvbV9EKGYoYSksZyhiKSkiXSxbMCwxXSxbMCwyXSxbMiwzLCIoXFxudV97YX0pXyEiXSxbMSwzLCIoXFxudV97Yn0pXyEiLDJdXQ==
\[\begin{tikzcd}
	{\hom_C(a,b)} & {\hom_D(g(a),g(b))} \\
	{\hom_D(f(a),f(b))} & {\hom_D(f(a),g(b))}
	\arrow[from=1-1, to=2-1]
	\arrow[from=1-1, to=1-2]
	\arrow["{(\nu_{a})_!}", from=1-2, to=2-2]
	\arrow["{(\nu_{b})_!}"', from=2-1, to=2-2]
\end{tikzcd}\]
natural in $a:C^t, b:C$.
\end{lemma}
\begin{proof}
Remark that $\hom_{[1]}(0,1)\sim \hom_{[1]}(1,1)\sim \hom_{[1]}(0,0)=1$.
Using the naturality of the hom functor, we have a commutative diagram 
% q.uiver.app/#q=WzAsNixbMCwyLCJcXGhvbV9DKGEsYilcXHRpbWVzIFxcaG9tX3tbMV19KDEsMSkiXSxbMSwwLCJcXGhvbV9EKGYoYSksZihiKSkiXSxbMSwyLCJcXGhvbV9EKGcoYSksZyhiKSkiXSxbMSwxLCJcXGhvbV9EKGYoYSksZyhiKSkiXSxbMCwwLCJcXGhvbV9DKGEsYilcXHRpbWVzIFxcaG9tX3tbMV19KDAsMCkiXSxbMCwxLCJcXGhvbV9DKGEsYilcXHRpbWVzIFxcaG9tX3tbMV19KDAsMSkiXSxbMiwzLCIoXFxudV97YX0pXyEiLDJdLFsxLDMsIihcXG51X3tifSlfISJdLFs0LDUsIlxcc2ltIiwyXSxbMCw1LCJcXHNpbSJdLFs1LDNdLFswLDJdLFs0LDFdXQ==
\[\begin{tikzcd}
	{\hom_C(a,b)\times \hom_{[1]}(0,0)} & {\hom_D(f(a),f(b))} \\
	{\hom_C(a,b)\times \hom_{[1]}(0,1)} & {\hom_D(f(a),g(b))} \\
	{\hom_C(a,b)\times \hom_{[1]}(1,1)} & {\hom_D(g(a),g(b))}
	\arrow["{(\nu_{a})_!}"', from=3-2, to=2-2]
	\arrow["{(\nu_{b})_!}", from=1-2, to=2-2]
	\arrow["\sim"', from=1-1, to=2-1]
	\arrow["\sim", from=3-1, to=2-1]
	\arrow[from=2-1, to=2-2]
	\arrow[from=3-1, to=3-2]
	\arrow[from=1-1, to=1-2]
\end{tikzcd}\]
where the left-hand vertical morphisms are equivalences.
\end{proof}

\begin{prop}
\label{prop:If unit and counit so adjunction}
Let $u:C\to D$ and $v:D\to C$ be two functors between locally $\U$-small $\io$-categories, $\mu:id_C\to vu$, $\epsilon:uv\to id_D$ be two natural transformations coming along with equivalences 
$$(\epsilon\circ_0 u)~\circ_1~(u\circ_0 \mu) \sim id_{u}~~~~ (v\circ_0 \epsilon)~\circ_1(\mu \circ_0 v )\sim id_{v}.$$
If we set $\phi$ as the composite 
$$\hom_D(u(a),b)\to \hom_C(vu(a),v(b))\xrightarrow{(\mu_a)_!} \hom_C(a,v(b)),$$
the triple $(u,v,\phi)$ is an adjoint structure.
Moreover, the unit of the adjunction is $\mu$ and its counit is $\epsilon$.
\end{prop}
\begin{proof}
Suppose we have such data. We define $\psi$ as the composite
$$\hom_C(a,v(b))\to \hom_D(u(a),uv(b))\xrightarrow{(\epsilon_a)_!} \hom_D(u(a),b)
$$
natural in $a:C^t$ and $b:D$. We then have to show that these two morphisms are inverse of each other. For this consider the diagram
% q.uiver.app/#q=WzAsOCxbMSwwLCJcXGhvbV9DKHZ1KGEpLHYoYikpIl0sWzIsMCwiXFxob21fQyhhLHYoYikpIl0sWzIsMiwiXFxob21fRCh1KGEpLGIpIl0sWzEsMiwiXFxob21fRCh1dnUoYSksYikiXSxbMCwwLCJcXGhvbV9EKHUoYSksYikiXSxbMSwxLCJcXGhvbV9EKHV2dShhKSx1dihiKSkiXSxbMiwxLCJcXGhvbV9EKHUoYSksdXYoYikpIl0sWzAsMiwiXFxob21fRCh1KGEpLGIpIl0sWzAsMSwiKFxcbXVfYSlfISJdLFszLDIsIih1KFxcbXVfe2F9KSlfISIsMl0sWzQsMF0sWzUsNiwiKHUoXFxtdV97YX0pKV8hIl0sWzAsNV0sWzEsNl0sWzUsMywiKFxcZXBzaWxvbl9iKV8hIl0sWzYsMiwiKFxcZXBzaWxvbl9iKV8hIl0sWzcsMywiKFxcZXBzaWxvbl97dShhKX0pXyEiLDJdLFs0LDcsIiIsMSx7ImxldmVsIjoyLCJzdHlsZSI6eyJoZWFkIjp7Im5hbWUiOiJub25lIn19fV1d
\[\begin{tikzcd}
	{\hom_D(u(a),b)} & {\hom_C(vu(a),v(b))} & {\hom_C(a,v(b))} \\
	& {\hom_D(uvu(a),uv(b))} & {\hom_D(u(a),uv(b))} \\
	{\hom_D(u(a),b)} & {\hom_D(uvu(a),b)} & {\hom_D(u(a),b)}
	\arrow["{(\mu_a)_!}", from=1-2, to=1-3]
	\arrow["{(u(\mu_{a}))_!}"', from=3-2, to=3-3]
	\arrow[from=1-1, to=1-2]
	\arrow["{(u(\mu_{a}))_!}", from=2-2, to=2-3]
	\arrow[from=1-2, to=2-2]
	\arrow[from=1-3, to=2-3]
	\arrow["{(\epsilon_b)_!}", from=2-2, to=3-2]
	\arrow["{(\epsilon_b)_!}", from=2-3, to=3-3]
	\arrow["{(\epsilon_{u(a)})_!}"', from=3-1, to=3-2]
	\arrow[Rightarrow, no head, from=1-1, to=3-1]
\end{tikzcd}\]
which is commutative thanks to lemma \ref{lemma:naturality of hom apply to natural transformation} and the naturality of the $\hom$.
By hypothesis, the left lower horizontal morphism is equivalent to the identity.
The outer square then defines an equivalence between $\psi\circ \phi$ and the identity. We show similarly $\phi\circ \psi\sim id$.

For the second assertion, remark that the composition 
$$\hom_C(a,a')\to \hom_D(u(a),u(a'))\xrightarrow{\phi(a,u(a')} \hom_C(a,vu(a'))$$
is by definition equivalent to 
$$\hom_C(a,a')\to \hom_D(vu(a),vu(a'))\xrightarrow{(\mu_a)_!} \hom_C(a,vu(a'))$$
and according to the lemma \ref{lemma:naturality of hom apply to natural transformation}, to
$$\hom_C(a,a')\xrightarrow{(\mu_{a'})_!} \hom_C(a,vu(a'))$$
The Yoneda lemma then implies that the unit of the adjunction is $\mu$. We proceed similarly for the counit.
\end{proof}

\p In paragraph \ref{par: i pull and push beetwe io category of morphism}, for a morphism $i:I\to A^\sharp$ between marked $\io$-categories, we define the morphism $i_!:\gHom(I,\uni)\to \uHom(A,\uni)$ and when $i$ is proper, a morphism $i_*:\gHom(I,\uni)\to \uHom(A,\uni)$.


\begin{cor}
\label{cor:naive kan extension}
Let $i:I\to A^\sharp$ be a morphism between $\U$-small $\io$-category. The functor $i^*:\uHom(A,\uni)\to \gHom(I,\uni)$ has a left adjoint given by the functor $i_!:\gHom(I,\uni)\to \uHom(A,\uni)$. If $i$ is proper, the functor $i^*$ has a right adjoint $i_*:\gHom(I,\uni)\to \uHom(A,\uni)$.
\end{cor} 
\begin{proof}
With the characterization of adjunction given in proposition \ref{prop:If unit and counit so adjunction}, this is a direct consequence of natural transformations given in paragraph \ref{par: i pull and push beetwe io category of morphism}.
\end{proof}





\p
We conclude this section with the proof of the following theorem.
\begin{theorem}
\label{theo:two adjunction definition}
Let $u:C\to D$ and $v:D\to C$ be two functors between locally $\U$-small $\io$-categories. 
The two following are equivalent. 
\begin{enumerate}
\item The pair $(u,v)$ admits an adjoint structure.
\item Their exists a pair of natural transformations $\mu: id_C \to vu$ and $\epsilon:uv\to id_D$ together with equivalences $(\epsilon\circ_0 u)\circ_1(u\circ_0 \mu) \sim id_{u}$ and $(v\circ_0 \epsilon)\circ_1 (\mu \circ_0 v )\sim id_{v}$.
\end{enumerate}
\end{theorem}
We directly give a corollary:

\begin{cor}
\label{cor:adjonction induced adjunction by post composition}
Let $(u:B\to C,v:C\to B)$ be an adjoint pair between locally $\U$-small $\io$-categories and $D$ a locally $\U$-small $\io$-category.
If $C$ and $B$ are $\U$-small, this induces an adjunction
% q.uiver.app/#q=WzAsMixbMSwwLCJcXHVIb20oQixEKTpcXHV2YXJcXGNpcmMgdiJdLFswLDAsIlxcdXZhclxcY2lyYyB1OlxcdUhvbShDLEQpIl0sWzEsMCwiIiwwLHsib2Zmc2V0IjotMn1dLFswLDEsIiIsMCx7Im9mZnNldCI6LTJ9XSxbMiwzLCIiLDAseyJsZXZlbCI6MSwic3R5bGUiOnsibmFtZSI6ImFkanVuY3Rpb24ifX1dXQ==
\[\begin{tikzcd}
	{\uvar\circ u:\uHom(C,D)} & {\uHom(B,D):\uvar\circ v}
	\arrow[""{name=0, anchor=center, inner sep=0}, shift left=2, from=1-1, to=1-2]
	\arrow[""{name=1, anchor=center, inner sep=0}, shift left=2, from=1-2, to=1-1]
	\arrow["\dashv"{anchor=center, rotate=-90}, draw=none, from=0, to=1]
\end{tikzcd}\]
and if $D$ is $\U$-small an adjunction
% q.uiver.app/#q=WzAsMixbMCwwLCJ1XFxjaXJjIFxcdXZhcjpcXHVIb20oRCxDKSJdLFsxLDAsIlxcdUhvbShELEIpOnZcXGNpcmNcXHV2YXIiXSxbMCwxLCIiLDAseyJvZmZzZXQiOi0yfV0sWzEsMCwiIiwwLHsib2Zmc2V0IjotMn1dLFsyLDMsIiIsMCx7ImxldmVsIjoxLCJzdHlsZSI6eyJuYW1lIjoiYWRqdW5jdGlvbiJ9fV1d
\[\begin{tikzcd}
	{u\circ \uvar:\uHom(D,C)} & {\uHom(D,B):v\circ\uvar}
	\arrow[""{name=0, anchor=center, inner sep=0}, shift left=2, from=1-1, to=1-2]
	\arrow[""{name=1, anchor=center, inner sep=0}, shift left=2, from=1-2, to=1-1]
	\arrow["\dashv"{anchor=center, rotate=-90}, draw=none, from=0, to=1]
\end{tikzcd}\]
\end{cor}
\begin{proof}
Let $\mu$ and $\epsilon$ be the unit and the counit of the adjunction. We define $\mu': \uHom(C,D)\times [1]\to \uHom(C,D)$, induced by currying the morphism 
$$ \uHom(C,D)\times [1]\times C\xrightarrow{id\times \mu} \uHom(C,D)\times C\xrightarrow{\ev} D$$
and $\epsilon':\uHom(B,D)\times [1]\to \uHom(B,D)$ by currying the morphism 
$$ \uHom(B,D)\times [1]\times B\xrightarrow{id\times \epsilon} \uHom(B,D)\times B\xrightarrow{\ev} B$$
We can easily check that $\mu'$ and $\epsilon'$ fulfill the triangle identities, and theorem \ref{theo:two adjunction definition} then implies that the pair $(\uvar\circ u,\uvar\circ v)$ admits an adjunction structure. We proceed similarly for the second assertion. 
\end{proof}

\p For the remaining, we fix two functors $u:C\to D$ and $v:D\to C$ between $\io$-categories as well as an equivalence
$$\phi:\hom_{D}(u(a),b)\sim \hom_C(a,v(b))$$
natural in $a:C^t$ and $b:D$.



\begin{lemma}
\label{lemma: if ajdunction then unit 1}
The natural transformation
$$\hom_D(u(a),b)\to \hom_C(vu(a),v(b))\xrightarrow{(\mu_a)_!}\hom_C(a,v(b))$$
is equivalent to $\phi:\hom_D(u(a),b)\to \hom_D(a,v(b))$.
Similarly, the natural transformation

$$\hom_C(a,v(b))\to \hom_D(u(a),uv(b))\xrightarrow{(\epsilon_b)_!}\hom_D(u(a),b)$$
is equivalent to $\phi^{-1}:\hom_D(a,v(b))\to \hom_D(u(a),b)$.
\end{lemma}
\begin{proof}
Remark that we have a commutative diagram
% q.uiver.app/#q=WzAsNSxbMiwwLCJcXGhvbV9DKHZ1KGEpLHZ1KGIpKSJdLFsyLDEsIlxcaG9tX0QoYSx2dShiKSkiXSxbMSwwLCJcXGhvbV9EKHUoYSksdShiKSkiXSxbMCwwLCJcXGhvbV9DKGEsYikiXSxbMCwxLCJcXGhvbV9EKHUoYSksdShiKSkiXSxbMCwxLCIoXFxtdV9hKV8hIl0sWzIsMF0sWzMsMl0sWzMsMSwiKFxcbXVfYilfISIsMV0sWzQsMSwiXFxwaGkiLDFdLFszLDRdXQ==
\[\begin{tikzcd}
	{\hom_C(a,b)} & {\hom_D(u(a),u(b))} & {\hom_C(vu(a),vu(b))} \\
	{\hom_D(u(a),u(b))} && {\hom_D(a,vu(b))}
	\arrow["{(\mu_a)_!}", from=1-3, to=2-3]
	\arrow[from=1-2, to=1-3]
	\arrow[from=1-1, to=1-2]
	\arrow["{(\mu_b)_!}"{description}, from=1-1, to=2-3]
	\arrow["\phi"{description}, from=2-1, to=2-3]
	\arrow[from=1-1, to=2-1]
\end{tikzcd}\]
The commutativity of the left triangle comes from the definition of $\mu$, and the second one, from the lemma \ref{lemma:naturality of hom apply to natural transformation}, applied to $\mu$.
This then induces a commutative square
% https://q.uiver.app/#q=WzAsNCxbMCwwLCJcXGludF97Q150XFx0aW1lcyBDfVxcaG9tX0MiXSxbMiwxLCJcXFJiKGlkXFx0aW1lcyB1KV4qXFxpbnRfe0NedFxcdGltZXMgRH1cXGhvbV9DKFxcdXZhcix2KFxcdXZhcikpIl0sWzIsMCwiXFxSYihpZFxcdGltZXMgdSleKlxcaW50X3tDXnRcXHRpbWVzIER9XFxob21fRCh1KFxcdXZhciksXFx1dmFyKSJdLFswLDEsIlxcUmIoaWRcXHRpbWVzIHUpXipcXGludF97Q150XFx0aW1lcyBEfVxcaG9tX0QodShcXHV2YXIpLFxcdXZhcikiXSxbMCwyXSxbMiwxLCJcXFJiIChpZFxcdGltZXMgdSleKlxcaW50X3tDXnRcXHRpbWVzIER9KFxcbXVfYSlfIVxcY2lyYyBcXGhvbV92Il0sWzAsM10sWzMsMSwiXFxSYihpZFxcdGltZXMgdSleKlxcaW50X3tDXnRcXHRpbWVzIER9XFxwaGkiLDJdXQ==
\[\begin{tikzcd}
	{\int_{C^t\times C}\hom_C} && {\Rb(id\times u)^*\int_{C^t\times D}\hom_D(u(\uvar),\uvar)} \\
	{\Rb(id\times u)^*\int_{C^t\times D}\hom_D(u(\uvar),\uvar)} && {\Rb(id\times u)^*\int_{C^t\times D}\hom_C(\uvar,v(\uvar))}
	\arrow[from=1-1, to=1-3]
	\arrow["{\Rb (id\times u)^*\int_{C^t\times D}(\mu_a)_!\circ \hom_v}", from=1-3, to=2-3]
	\arrow[from=1-1, to=2-1]
	\arrow["{\Rb(id\times u)^*\int_{C^t\times D}\phi}"', from=2-1, to=2-3]
\end{tikzcd}\]
By adjunction, this corresponds to a commutative square 
% q.uiver.app/#q=WzAsNCxbMCwwLCJcXExiKGlkXFx0aW1lcyB1KV8hXFxpbnRfe0NedFxcdGltZXMgQ31cXGhvbV9DIl0sWzIsMSwiXFxpbnRfe0NedFxcdGltZXMgRH1cXGhvbV9DKFxcdXZhcix2KFxcdXZhcikpIl0sWzIsMCwiXFxpbnRfe0NedFxcdGltZXMgRH1cXGhvbV9EKHUoXFx1dmFyKSxcXHV2YXIpIl0sWzAsMSwiXFxpbnRfe0NedFxcdGltZXMgRH1cXGhvbV9EKHUoXFx1dmFyKSxcXHV2YXIpIl0sWzAsMl0sWzIsMSwiXFxpbnRfe0NedFxcdGltZXMgRH0oXFxtdV9hKV8hXFxjaXJjIFxcaG9tX3YiXSxbMywxLCJcXGludF97Q150XFx0aW1lcyBEfVxccGhpIiwyXSxbMCwzXV0=
\[\begin{tikzcd}
	{\Lb(id\times u)_!\int_{C^t\times C}\hom_C} && {\int_{C^t\times D}\hom_D(u(\uvar),\uvar)} \\
	{\int_{C^t\times D}\hom_D(u(\uvar),\uvar)} && {\int_{C^t\times D}\hom_C(\uvar,v(\uvar))}
	\arrow[from=1-1, to=1-3]
	\arrow["{\int_{C^t\times D}(\mu_a)_!\circ \hom_v}", from=1-3, to=2-3]
	\arrow["{\int_{C^t\times D}\phi}"', from=2-1, to=2-3]
	\arrow[from=1-1, to=2-1]
\end{tikzcd}\]
However, the top horizontal and left vertical morphisms are equivalences according to lemma \ref{lemma:a particular Kan extension}.
We then have an equivalence $$ \int_{C^t\times D}(\mu_a)_!\circ \hom_v\sim \int_{C^t\times D}\phi$$
which implies the result.
The other assertion is shown similarly.
\end{proof}



\begin{lemma}
\label{lemma:if ajdunction then unit 2}
There are equivalences
$(\epsilon\circ_0 u)\circ_1(u\circ_0 \mu) \sim id_{u}$ and $(v\circ_0 \epsilon)\circ_1 (\mu \circ_0 v )\sim id_{v}$.
\end{lemma}
\begin{proof}
As the proof of the two assertions are similar, we will only show the second one.
To demonstrate this, it is enough to show that the induced natural transformation 
\begin{equation}
\label{eq:sequence ajdunction}
\hom_C(a,v(b))\xrightarrow{(\mu_{v(b)})_!} \hom_C(a,vuv(b)) \xrightarrow{(v(\epsilon_{(b)}))_!} \hom_C(a,v(b))\xrightarrow{\phi^{-1}}\hom_D(u(a),b)
\end{equation}
is equivalent to $\phi^{-1}$.
By definition, the first morphism is equivalent to the composition
$$\hom_C(a,v(b))\to \hom_D(u(a),uv(b))\xrightarrow{\phi} \hom_C(a,vuv(b))$$
and as $\phi^{-1}$ is a natural transformation, we have a commutative square
% q.uiver.app/#q=WzAsNCxbMCwwLCJcXGhvbV9DKGEsdnV2KGIpKSJdLFsxLDAsIlxcaG9tX0MoYSx2KGIpKSJdLFsxLDEsIlxcaG9tX0QodShhKSxiKSJdLFswLDEsIlxcaG9tX0ModShhKSx1dihiKSkiXSxbMCwxLCIodihcXGVwc2lsb25fe2J9KSlfISJdLFsxLDIsIlxccGhpXnstMX0iXSxbMCwzLCJcXHBoaV57LTF9IiwyXSxbMywyLCIoXFxlcHNpbG9uX3tifSlfeyF9IiwyXV0=
\[\begin{tikzcd}
	{\hom_C(a,vuv(b))} & {\hom_C(a,v(b))} \\
	{\hom_C(u(a),uv(b))} & {\hom_D(u(a),b)}
	\arrow["{(v(\epsilon_{b}))_!}", from=1-1, to=1-2]
	\arrow["{\phi^{-1}}", from=1-2, to=2-2]
	\arrow["{\phi^{-1}}"', from=1-1, to=2-1]
	\arrow["{(\epsilon_{b})_{!}}"', from=2-1, to=2-2]
\end{tikzcd}\]
The composite of the sequence \eqref{eq:sequence ajdunction} is then equivalent to 
$$\hom_C(a,v(b))\to \hom_D(u(a),uv(b))\xrightarrow{(\epsilon_{b})_{!}} \hom_D(u(a),b)$$
which is itself equivalent to $\phi^{-1}$ according to lemma \ref{lemma: if ajdunction then unit 1}.
\end{proof}


\begin{proof}[Proof of theorem \ref{theo:two adjunction definition}]
The implication $(1)\Rightarrow (2)$ is given by proposition \ref{prop:If unit and counit so adjunction} and the contraposed by the lemma \ref{lemma:if ajdunction then unit 2}.
\end{proof}


\subsection{Lax colimits}
\p According to corollary \ref{cor:naive kan extension}, a morphism $f:A\to B$ between $\U$-small $\io$-categories induces an adjoint pair:
% q.uiver.app/#q=WzAsMixbMCwwLCJmXyE6XFx3e0F9Il0sWzEsMCwiXFx3e0J9OmZeKiJdLFswLDEsIiIsMCx7Im9mZnNldCI6LTJ9XSxbMSwwLCIiLDAseyJvZmZzZXQiOi0yfV0sWzIsMywiIiwwLHsibGV2ZWwiOjEsInN0eWxlIjp7Im5hbWUiOiJhZGp1bmN0aW9uIn19XV0=
\begin{equation}
\label{eq:adjoint presheaves}
\begin{tikzcd}
	{f_!:\w{A}} & {\w{B}:f^*}
	\arrow[""{name=0, anchor=center, inner sep=0}, shift left=2, from=1-1, to=1-2]
	\arrow[""{name=1, anchor=center, inner sep=0}, shift left=2, from=1-2, to=1-1]
	\arrow["\dashv"{anchor=center, rotate=-90}, draw=none, from=0, to=1]
\end{tikzcd}
\end{equation}

\begin{prop}
\label{prop:left extension commutes with Yoneda}
Let $f:A\to B$ be a morphism between $\U$-small $\io$-categories.
There is an equivalence 
$$f_!(y_a)\sim y_{f(a)}$$
natural in $a:A$.
\end{prop}
\begin{proof}
Consider the sequence of equivalences
$$\begin{array}{rcll}
\hom_{\w{B}}(f_!(y_a), g)&\sim &\hom_{\w{A}}(y_a, f^*(g))& \eqref{eq:adjoint presheaves}\\
&\sim &\ev(a,f^*(g))&(\mbox{Yoneda lemma})\\
&\sim &\ev(f(a),g)&(\mbox{naturality of $\ev$})\\
&\sim & \hom_{\w{B}}(y_{f(a)}, g)&(\mbox{Yoneda lemma})\\
\end{array}$$
Eventually, the Yoneda lemma applied to $(\w{B})^t$ concludes the proof.
\end{proof}
\p For $I$ a marked $\io$-category and $A$ an $\io$-category,
we recall that $\gHom(I,A)$ is the $\io$-category whose value on a globular sum $a$ is given by:
$$\Hom(a,\gHom(I,A)):=\Hom(I\ominus a^\sharp, A^\sharp)$$
\begin{remark}
Let $B$ be an $\io$-category.
We want to give an intuition of the object $\gHom(B^\flat,\omega)$. The objects of this $\io$-category are the functors $I\to \omega$. The $1$-cells are the lax transformations $F\Rightarrow G$. For $n>1$, the $n$-cells are the lax transformations $F^{\times \Db_{n-1}}\Rightarrow G$ where $F^{\times \Db_{n-1}}:I\to \omega$ is the functor that sends $i$ onto $F(i)\times \Db_{n-1}$.
This last assertion is a consequence of the equivalence 
$$\tau_0(\LCart((I\ominus [b,n]^\sharp)^\sharp) \sim \Hom([n],\LCartc(I;b))$$
provided by the lemma \ref{lemma:lax univalence 4}.
\end{remark}
\begin{prop}
If $I$ is $\U$-small and $A$ is locally $\U$-small, the $\io$-category $\gHom(I,A)$ is locally $\U$-small.
\end{prop}
\begin{proof}
We have to check that for any globular sum $b$, the morphism 
$$\Hom(I\ominus [b,1]^\sharp,A^\sharp)\to \Hom(I\ominus (\{0\}\amalg\{1\}),A^\sharp)$$
has $\U$-small fibers. As $I$, seen as an $\infty$-presheaves on $t\Theta$, is a $\U$-small colimit of representables, we can reduce to the case where $I\in t\Theta$. As $A$ is local with respect to Segal extensions, and as $\ominus$ conserves them, we can reduce to the case where $I$ is of shape $[1]^\sharp$ or $[a,1]$ for $a$ a in $t\Theta$. If $I$ is $[1]^\sharp$, according to the second assertion of proposition \ref{prop:associativity of ominus}, $[1]^\sharp\ominus [b,1]^\sharp$ is equivalent to $([1]\times [b,1])^\sharp$ and the result follows from proposition \ref{prop:when Hom A B is locally small}.

For the second case, we fix a morphism $f:[a,1]\times (\{0\}\amalg\{1\})\to A$. 
Using the canonical equivalence between $[a,1]\ominus [b,1]^\sharp$ and the colimit of the diagram \eqref{eq:formula for the ominus marked case},
the $\infty$-groupoid $\Hom(I\ominus [b,1]^\sharp,A^\sharp)_f$ is the limit of the diagram:
% https://q.uiver.app/#q=WzAsNSxbMSwwLCJcXEhvbSgoYVxcb3RpbWVzXFx7MFxcfV5cXHNoYXJwKV5cXG5hdHVyYWxcXHRpbWVzIGIsXFxob20oZigwLDApLGYoMSwxKSkiXSxbMCwxLCJcXEhvbSgoYVxcb3RpbWVzWzFdXlxcc2hhcnApXlxcbmF0dXJhbFxcdGltZXMgYixcXGhvbShmKDAsMCksZigxLDEpKSkiXSxbMSwyLCJcXEhvbSgoYVxcb3RpbWVzXFx7MFxcfV5cXHNoYXJwKV5cXG5hdHVyYWxcXHRpbWVzIGIsXFxob20oZigwLDApLGYoMSwxKSkpKSJdLFswLDIsIlxcSG9tKGFeXFxuYXR1cmFsLFxcaG9tKGYoMCwwKSxmKDEsMCkpKSJdLFswLDAsIlxcSG9tKGFeXFxuYXR1cmFsLFxcaG9tKGYoMSwwKSxmKDEsMSkpKSJdLFsxLDJdLFsxLDBdLFszLDJdLFs0LDBdXQ==
\[\begin{tikzcd}[column sep=0.1cm]
	{\Hom(a^\natural,\hom(f(1,0),f(1,1)))} & {\Hom((a\otimes\{0\}^\sharp)^\natural\times b,\hom(f(0,0),f(1,1))} \\
	{\Hom((a\otimes[1]^\sharp)^\natural\times b,\hom(f(0,0),f(1,1)))} \\
	{\Hom(a^\natural,\hom(f(0,0),f(1,0)))} & {\Hom((a\otimes\{0\}^\sharp)^\natural\times b,\hom(f(0,0),f(1,1))))}
	\arrow[from=2-1, to=3-2]
	\arrow[from=2-1, to=1-2]
	\arrow[from=3-1, to=3-2]
	\arrow[from=1-1, to=1-2]
\end{tikzcd}\]
As all these objects are $\U$-small by assumption, this concludes the proof.
\end{proof}

\p
Let $I$ be a $\U$-small marked $\io$-category, $A$ a locally $\U$-small $\io$-category $A$ and $F:I\to A^\sharp$ a functor. 
A \notion{lax colimit} of $F$ is an object \wcnotation{$\laxcolim_IF$}{(laxcolim@$\laxcolim$} of $A$ together with an equivalence
$$\hom_{A}(\laxcolim_IF, b)\sim \hom_{\gHom(I,A)}(F,\cst b)$$
natural in $b:A$. 
Conversely, a \notion{lax limit} of $F$ is an object \wcnotation{$\laxlim_IF$}{(laxlim@$\laxlim$} of $A$ together with an equivalence
$$\hom_{A}(b,\laxlim_IF)\sim \hom_{\gHom(I,A)}(\cst b,F)$$
natural in $b:A$. 
We say that a locally $\U$-small $\io$-category $C$ is \notion{lax $\U$-complete} (resp. \notion{lax $\U$-cocomplete}), if for any $\U$-small marked $\io$-category $I$ and any functor $F:I\to C$, $F$ admits limits (resp. colimits).


Using proposition \ref{prop:adj if slice as terminal}, $C$ is lax $\U$-complete (resp. lax $\U$-cocomplete) if and only if for any $\U$-small marked $\io$-category $I$, the functor $\cst:C\to \gHom(I,C)$ admits a right adjoint (resp. a left adjoint).



The proposition \ref{prop:ominus and opmarked}  induces an equivalence
$$\gHom(I,A)^{\circ}\sim\gHom(I^{\circ},A^{\circ})$$
As a consequence, a functor $F:I\to A^\sharp$ admits a lax colimit if and only if $F^\circ:I^\circ\to (A^\circ)^\sharp$ admits a lax limit. If $F$ admits such lax colimit, the lax limit of $F^\circ$ is the image by the canonical equivalence $A_0\sim A^\circ_0$ of the lax colimit of $F$.


\begin{remark}
We want to give an intuition of the lax colimits.
Let $I$ be a $\U$-small marked $\io$-category, $A$ a locally $\U$-small $\io$-category $A$ and $F:I\to A^\sharp$ a functor admitting a lax colimit $\laxcolim_IF$. For any $1$-cell $i:a\to b$ in $I$, we have a triangle
% https://q.uiver.app/#q=WzAsNCxbMCwwXSxbMCwxLCJGKGEpIl0sWzEsMSwiXFxsYXhjb2xpbV9JRiJdLFsxLDAsIkYoYikiXSxbMSwzLCJGKGkpIiwwLHsiY3VydmUiOi01fV0sWzEsMl0sWzMsMSwiIiwxLHsic2hvcnRlbiI6eyJzb3VyY2UiOjMwLCJ0YXJnZXQiOjMwfSwibGV2ZWwiOjJ9XSxbMCwxLCIiLDAseyJzdHlsZSI6eyJib2R5Ijp7Im5hbWUiOiJub25lIn0sImhlYWQiOnsibmFtZSI6Im5vbmUifX19XSxbMywyXV0=
\[\begin{tikzcd}
	{} & {F(b)} \\
	{F(a)} & {\laxcolim_IF}
	\arrow["{F(i)}", curve={height=-30pt}, from=2-1, to=1-2]
	\arrow[from=2-1, to=2-2]
	\arrow[shorten <=8pt, shorten >=8pt, Rightarrow, from=1-2, to=2-1]
	\arrow[draw=none, from=1-1, to=2-1]
	\arrow[from=1-2, to=2-2]
\end{tikzcd}\]
If $i$ is marked, the preceding $2$-cell is an equivalence. 
For any $2$-cell $u:i\to j$, we have a diagram
% https://q.uiver.app/#q=WzAsNyxbMCwxLCJGKGEpIl0sWzEsMCwiRihiKSJdLFsxLDEsIlxcbGF4Y29saW1fSUYiXSxbMiwxLCJGKGEpIl0sWzMsMCwiRihiKSJdLFszLDEsIlxcbGF4Y29saW1fSUYiXSxbMiwwXSxbMCwxLCJGKGkpIiwxXSxbMCwyXSxbMSwyXSxbMSwyXSxbMCwxLCJGKGopIiwwLHsiY3VydmUiOi01fV0sWzMsNCwiRihqKSIsMCx7ImN1cnZlIjotNX1dLFszLDVdLFs0LDVdLFs0LDMsIiIsMSx7InNob3J0ZW4iOnsic291cmNlIjozMCwidGFyZ2V0IjozMH0sImxldmVsIjoyfV0sWzYsMywiIiwwLHsic3R5bGUiOnsiYm9keSI6eyJuYW1lIjoibm9uZSJ9LCJoZWFkIjp7Im5hbWUiOiJub25lIn19fV0sWzEwLDgsIiIsMSx7Im9mZnNldCI6Miwic2hvcnRlbiI6eyJzb3VyY2UiOjQwLCJ0YXJnZXQiOjQwfX1dLFsxMSw3LCIiLDEseyJzaG9ydGVuIjp7InNvdXJjZSI6MjAsInRhcmdldCI6MjB9fV0sWzEwLDE2LCIiLDEseyJvZmZzZXQiOi0xLCJzaG9ydGVuIjp7InNvdXJjZSI6MzAsInRhcmdldCI6MzB9LCJsZXZlbCI6MSwic3R5bGUiOnsiaGVhZCI6eyJuYW1lIjoibm9uZSJ9fX1dLFsxMCwxNiwiIiwxLHsic2hvcnRlbiI6eyJzb3VyY2UiOjMwLCJ0YXJnZXQiOjMwfSwibGV2ZWwiOjF9XSxbMTAsMTYsIiIsMSx7Im9mZnNldCI6MSwic2hvcnRlbiI6eyJzb3VyY2UiOjMwLCJ0YXJnZXQiOjMwfSwibGV2ZWwiOjEsInN0eWxlIjp7ImhlYWQiOnsibmFtZSI6Im5vbmUifX19XV0=
\[\begin{tikzcd}
	& {F(b)} & {} & {F(b)} \\
	{F(a)} & {\laxcolim_IF} & {F(a)} & {\laxcolim_IF}
	\arrow[""{name=0, anchor=center, inner sep=0}, "{F(i)}"{description}, from=2-1, to=1-2]
	\arrow[""{name=1, anchor=center, inner sep=0}, from=2-1, to=2-2]
	\arrow[from=1-2, to=2-2]
	\arrow[""{name=2, anchor=center, inner sep=0}, from=1-2, to=2-2]
	\arrow[""{name=3, anchor=center, inner sep=0}, "{F(j)}", curve={height=-30pt}, from=2-1, to=1-2]
	\arrow["{F(j)}", curve={height=-30pt}, from=2-3, to=1-4]
	\arrow[from=2-3, to=2-4]
	\arrow[from=1-4, to=2-4]
	\arrow[shorten <=8pt, shorten >=8pt, Rightarrow, from=1-4, to=2-3]
	\arrow[""{name=4, anchor=center, inner sep=0}, draw=none, from=1-3, to=2-3]
	\arrow[shift right=2, shorten <=12pt, shorten >=12pt, Rightarrow, from=2, to=1]
	\arrow[shorten <=4pt, shorten >=4pt, Rightarrow, from=3, to=0]
	\arrow[shift left=0.7, shorten <=14pt, shorten >=16pt, no head, from=2, to=4]
	\arrow[shorten <=14pt, shorten >=14pt, from=2, to=4]
	\arrow[shift right=0.7, shorten <=14pt, shorten >=16pt, no head, from=2, to=4]
\end{tikzcd}\]
If $u$ is marked, the $3$-cell is an equivalence. We can continue these diagrams in higher dimensions and we have
similar assertions for lax limits.

The marking therefore allows us to play on the "lax character" of the universal property that the lax colimit must verify.
\end{remark}



\p Let $A$ be a $\U$-small $\io$-category and $I$ a $\U$-small marked $\io$-category.
Recall that $\gHom(I,\w{A})$ is equivalent to $\gHom(I\times (A^t)^\sharp,\uni)$. Let $t$ be the canonical morphism $I\to 1$. 
As $t$ is smooth, corollary \ref{cor:naive kan extension} induces adjunctions
% https://q.uiver.app/#q=WzAsMixbMCwwLCJcXGdIb20oSSxcXHd7QX0pIl0sWzIsMCwiXFx3e0F9Il0sWzEsMCwiKHRcXHRpbWVzIGlkX0EpXioiLDFdLFswLDEsIih0XFx0aW1lcyBpZF9BKV8hIiwwLHsib2Zmc2V0IjotNX1dLFswLDEsIih0XFx0aW1lcyBpZF9BKV8qIiwyLHsib2Zmc2V0Ijo1fV0sWzIsNCwiIiwxLHsibGV2ZWwiOjEsInN0eWxlIjp7Im5hbWUiOiJhZGp1bmN0aW9uIn19XSxbMywyLCIiLDEseyJsZXZlbCI6MSwic3R5bGUiOnsibmFtZSI6ImFkanVuY3Rpb24ifX1dXQ==
\begin{equation}
\label{eq:expliciti colimit for presheaves}
\begin{tikzcd}
	{\gHom(I,\w{A})} && {\w{A}}
	\arrow[""{name=0, anchor=center, inner sep=0}, "{(t\times id_A)^*}"{description}, from=1-3, to=1-1]
	\arrow[""{name=1, anchor=center, inner sep=0}, "{(t\times id_A)_!}", shift left=5, from=1-1, to=1-3]
	\arrow[""{name=2, anchor=center, inner sep=0}, "{(t\times id_A)_*}"', shift right=5, from=1-1, to=1-3]
	\arrow["\dashv"{anchor=center, rotate=-90}, draw=none, from=0, to=2]
	\arrow["\dashv"{anchor=center, rotate=-90}, draw=none, from=1, to=0]
\end{tikzcd}
\end{equation}
and $\w{A}$ is then lax $\U$-complete and lax $\U$-cocomplete. For a morphism $g:I\to \w{A}^\sharp$ corresponding to an object $E$ of $\LCartc(I\times (A^t)^\sharp)$, we then have 
\begin{equation}
\label{eq:expliciti colimit for presheaves2}
\int_{A^t}\laxcolim_I g \sim \Lb (t\times id_{(A^t)^\sharp})_!E~~~\int_{A^t}\laxlim_I g \sim \Rb (t\times id_{(A^t)^\sharp})_*E
\end{equation}
Let $i:B^\sharp\to A^\sharp$ be any morphism. The squares given in paragraph \ref{par: i pull and push beetwe io category of morphism} induce the commutative squares
% https://q.uiver.app/#q=WzAsNixbMCwwLCJcXGdIb20oSSxcXHd7QX0pIl0sWzEsMCwiXFx3e0F9Il0sWzEsMSwiXFx3e0J9Il0sWzAsMSwiXFxnSG9tKEksXFx3e0J9KSJdLFsyLDAsIlxcZ0hvbShJLFxcd3tBfSkiXSxbMiwxLCJcXGdIb20oSSxcXHd7Qn0pIl0sWzAsMSwiXFxsYXhjb2xpbV9JIl0sWzMsMiwiXFxsYXhjb2xpbV9JIiwyXSxbMCwzLCIoaWRfSVxcdGltZXMgaV50KV4qIiwyXSxbMSwyLCJpXioiXSxbNCw1LCIoaWRfSVxcdGltZXMgaV50KV4qIl0sWzQsMSwiXFxsYXhsaW1fSSIsMl0sWzUsMiwiXFxsYXhsaW1fSSJdXQ==
\[\begin{tikzcd}
	{\gHom(I,\w{A})} & {\w{A}} & {\gHom(I,\w{A})} \\
	{\gHom(I,\w{B})} & {\w{B}} & {\gHom(I,\w{B})}
	\arrow["{\laxcolim_I}", from=1-1, to=1-2]
	\arrow["{\laxcolim_I}"', from=2-1, to=2-2]
	\arrow["{(id_I\times i^t)^*}"', from=1-1, to=2-1]
	\arrow["{i^*}", from=1-2, to=2-2]
	\arrow["{(id_I\times i^t)^*}", from=1-3, to=2-3]
	\arrow["{\laxlim_I}"', from=1-3, to=1-2]
	\arrow["{\laxlim_I}", from=2-3, to=2-2]
\end{tikzcd}\]
In particular, choosing $B:=1$, this implies that the lax colimits and limits in $\io$-presheaves commute with evaluation.


The next proposition implies that limits and colimits in $\io$-presheaves can be detected as the level of the sub maximal $\iun$-categories of $\gHom(I,\w{A})$ and $\w{A}$. We recall that the sub maximal $\iun$-categories of $\gHom(I,\w{A})$, denoted by $\tau_1\gHom(I,\w{A})$, is the adjoint of the functor $[n]\mapsto I\otimes[n]^\sharp$.
\begin{prop}
Let $I$ be a $\U$-small marked $\io$-category, and $g:I\to A^\sharp$ a functor. An object $f$ of $\w{A}$ has a structure of colimit of the functor $g$ if and only if there exists an equivalence
$$\Hom_{\tau_1\w{A}}(f,h)\sim \Hom_{\tau_1\gHom(I,\w{A})}(F,\cst h)$$
natural in $h:(\tau^1 \w{A})^{op}$.
Similarly, the object $f$ has a structure of limit of the functor $F$ if and only if there exists an equivalence
$$\Hom_{\tau_1\w{A}}(h,f)\sim \Hom_{\tau_1\gHom(I,\w{A})}(\cst h,F)$$
natural in $h:(\tau^1 \w{A})^{op}$.
\end{prop}
\begin{proof}
We recall that theorem \ref{theo:lcartc et ghom} and corollary \ref{cor:lcar et hom} induces equivalences
$$\tau_1\w{A}\sim \LCart_{\U}((A^t)^\sharp)~~~ \tau_1\gHom(I,A)\sim \LCartc_{\U}(I\otimes (A^t)^\sharp)$$
and that we have a triplet of adjoints
% https://q.uiver.app/#q=WzAsMixbMCwwLCJcXExDYXJ0Y197XFxVfShJXFxvdGltZXMgKEFedCleXFxzaGFycCkiXSxbMiwwLCJcXExDYXJ0X3tcXFV9KChBXnQpXlxcc2hhcnApIl0sWzEsMCwiKHRcXHRpbWVzIGlkX3tBXnR9KV4qIiwxXSxbMCwxLCJcXExiICh0XFx0aW1lcyBpZF97QV50fSlfISIsMCx7Im9mZnNldCI6LTV9XSxbMCwxLCJcXFJiKHRcXHRpbWVzIGlkX3tBXnR9KV8qIiwyLHsib2Zmc2V0Ijo1fV0sWzMsMiwiIiwxLHsibGV2ZWwiOjEsInN0eWxlIjp7Im5hbWUiOiJhZGp1bmN0aW9uIn19XSxbMiw0LCIiLDEseyJsZXZlbCI6MSwic3R5bGUiOnsibmFtZSI6ImFkanVuY3Rpb24ifX1dXQ==
\[\begin{tikzcd}
	{\LCartc_{\U}(I\otimes (A^t)^\sharp)} && {\LCart_{\U}((A^t)^\sharp)}
	\arrow[""{name=0, anchor=center, inner sep=0}, "{(t\times id_{A^t})^*}"{description}, from=1-3, to=1-1]
	\arrow[""{name=1, anchor=center, inner sep=0}, "{\Lb (t\times id_{A^t})_!}", shift left=5, from=1-1, to=1-3]
	\arrow[""{name=2, anchor=center, inner sep=0}, "{\Rb(t\times id_{A^t})_*}"', shift right=5, from=1-1, to=1-3]
	\arrow["\dashv"{anchor=center, rotate=-90}, draw=none, from=1, to=0]
	\arrow["\dashv"{anchor=center, rotate=-90}, draw=none, from=0, to=2]
\end{tikzcd}\]
which is the image by $\tau_1$ of the triplet of adjoints \eqref{eq:expliciti colimit for presheaves}.
The first hypothesis induces an equivalence $$\int_{A^t}f \sim \Lb (t\times id_{(A^t)^\sharp})_!E$$ and the second one an equivalence 
$$\int_{A^t}f \sim \Rb (t\times id_{(A^t)^\sharp})_*E$$ where $E$ denote the object of $\LCartc(I\times (A^t)^\sharp)$ corresponding to $g$. The assertions then follow from the equivalences \eqref{eq:expliciti colimit for presheaves2}.
\end{proof}


\begin{example}
We recall that we denote by $\bot:\Arr(\ocatm)\to \ocat$ the functor sending a left fibration $Y\to A$ to the localization of $Y$ by marked cells. This functors sends initial and final morphisms to equivalences. If $E$ is a left cartesian fibration over a marked $\io$-category $I$, we then have $\bot E\sim \Lb t_! E$ where $t$ denotes the morphism $I\to 1$.

 Let $g:I\to \uni$ be a diagram. We denote $\iota:I\to I^\sharp$ the canonical inclusion.
By the explicit expression of lax colimit given above, we then have an equivalence 
$$\laxcolim_I g \sim \bot \iota^*\int_{I^{\natural}}g^\natural.$$
If $I$ is equivalent to $I^\flat$, we then have
$$\laxcolim_I g \sim \dom(\int_{I^{\natural}}g^\natural)^\natural.$$
 \begin{enumerate}
 \item[$-$]
Let $c:1\to \uni$ be a morphism corresponding to an $\io$-category $C$. For any $\io$-category $A$, we then have 
$$\laxcolim_{A^\sharp} \cst_c\sim (\tau_0 A)\times C~~~~~\laxcolim_{A^\flat} \cst_c\sim A\times C$$

 \item[$-$] Let $f:[b,1]\to \uni$ be a morphism corresponding to a morphism $A\times b\to B$. We then have 
 $$\laxcolim_{[b,1]^\flat} f\sim A\times (1\costar b)\coprod_{A\times b}B$$
 \end{enumerate}
\end{example}

\begin{example}
 
Using the explicit expression of lax limit given above, we have an equivalence
$$\laxlim_I g \sim \Map(id_I,\iota^*\int_{I^{\natural}}g^\natural)$$
 \begin{enumerate}
 \item[$-$]
Let $c:1\to \uni$ be a morphism corresponding to an $\io$-category $C$. For any $\io$-category $A$, we then have 
$$\laxlim_{A^\sharp} \cst_c\sim \uHom(\tau_0 A, C)~~~~~\laxlim_{A^\flat} \cst_c\sim \uHom(A,C)$$

 \item[$-$] Let $f:[b,1]\to \uni$ be a morphism corresponding to a morphism $A\times b\to B$. Let $c$ be a globular sum. According to corollary \ref{cor:univalence tranche}, a morphism $id_{[b,1]^\flat}\times c^\flat\to \iota^*\int_{[b,1]^{\flat}}g^\natural$ corresponds to a diagram 
% https://q.uiver.app/#q=WzAsNCxbMSwxLCIxXFxjb3N0YXIgW2IsMV0iXSxbMCwyLCJbYiwxXSJdLFsyLDEsIlxcdW5pIl0sWzAsMCwiMSJdLFsxLDIsImYiLDIseyJjdXJ2ZSI6Mn1dLFszLDIsIlxce2JcXH0iLDAseyJjdXJ2ZSI6LTJ9XSxbMywwXSxbMSwwXSxbMCwyXV0=
\[\begin{tikzcd}
	1 \\
	& {1\costar [b,1]} & \uni \\
	{[b,1]}
	\arrow["f"', curve={height=12pt}, from=3-1, to=2-3]
	\arrow["{\{b\}}", curve={height=-12pt}, from=1-1, to=2-3]
	\arrow[from=1-1, to=2-2]
	\arrow[from=3-1, to=2-2]
	\arrow[from=2-2, to=2-3]
\end{tikzcd}\]
and according to proposition \ref{prop:lfib and W 3}, to a diagram	
% https://q.uiver.app/#q=WzAsNixbMiwwLCJBXFx0aW1lcyBiIl0sWzMsMSwiQiJdLFsxLDEsImNcXHRpbWVzKCBiXFxvdGltZXNbMV0pIl0sWzAsMCwiY1xcdGltZXMgYlxcb3RpbWVzXFx7MFxcfSJdLFswLDIsImNcXHRpbWVzIGJcXG90aW1lc1xcezFcXH0iXSxbMiwyLCJjIl0sWzAsMV0sWzIsMV0sWzMsMl0sWzQsNV0sWzQsMl0sWzUsMV0sWzMsMF1d
\[\begin{tikzcd}
	{c\times b\otimes\{0\}} && {A\times b} \\
	& {c\times( b\otimes[1])} && B \\
	{c\times b\otimes\{1\}} && c
	\arrow[from=1-3, to=2-4]
	\arrow[from=2-2, to=2-4]
	\arrow[from=1-1, to=2-2]
	\arrow[from=3-1, to=3-3]
	\arrow[from=3-1, to=2-2]
	\arrow[from=3-3, to=2-4]
	\arrow[from=1-1, to=1-3]
\end{tikzcd}\]
where the upper horizontal morphism is of shape $g\times b$. We then have 
 $$\laxlim_{[b,1]^\flat} f\sim A\prod_{\Hom(b,B)} \Hom(b\star 1,B).
 $$
 \end{enumerate}
 \end{example}




\begin{prop}
\label{prop:colimit restricted to final}
Let $i:I\to J$ be a morphism between $\U$-small marked $\io$-categories, $A$ a $\U$-small $\io$-category and $f:J\to \w{A}^\sharp$ a morphism.  If $i$ is final, then the canonical morphism
$$\laxcolim_{I}f\circ i\to \laxcolim_{J}f$$
is an equivalence.


If $i$ is initial, then the canonical morphism
$$\laxlim_{J}f\to \laxlim_{I}f\circ i$$
is an equivalence. 
\end{prop}
\begin{proof}
We only show the first assertion as the second follows by duality.
As equivalences are detected pointwise and as the lax colimit commutes with evaluation, one can suppose that $A:=1$, and so $\w{A}:=\uni$. 
We denote by $E$ (resp. $H$) the object of $\LCart(J)$ (resp.$\LCart(I)$) corresponding to $f$ (resp.$f\circ i$) and $X\to I$ (resp. $Y\to J$) the corresponding left cartesian fibration. We then have a cartesian square
% https://q.uiver.app/#q=WzAsNCxbMCwwLCJZIl0sWzEsMCwiWCJdLFswLDEsIkoiXSxbMSwxLCJJIl0sWzIsMywiaSIsMl0sWzAsMSwiaSciXSxbMSwzLCJIIl0sWzAsMiwiRSIsMl1d
\[\begin{tikzcd}
	Y & X \\
	J & I
	\arrow["i"', from=2-1, to=2-2]
	\arrow["{i'}", from=1-1, to=1-2]
	\arrow["H", from=1-2, to=2-2]
	\arrow["E"', from=1-1, to=2-1]
\end{tikzcd}\]
As classified left cartesian fibrations are proper, $i'$ is final. We recall that we denote by $\bot:\ocatm\to \ocat$ the functor sending a marked $\io$-category to its localization by marked cells, and that $\bot$ sends final morphism to equivalences. If we denote by $t$ the two morphisms $I\to 1$ and $J\to 1$, we then have a sequence of equivalences:
$$\laxcolim_{I}f\circ i\sim \Lb t_! H \sim \bot Y\sim \bot X\sim \Lb t_! E \sim \laxcolim_{J}f$$
\end{proof}



\begin{lemma}
\label{lemma:tehcnical colimit}
Let $F:I\to A^\sharp$ be a morphism between $\U$-small marked $\io$-categories. 
There is an equivalence 
$$ \hom_{\gHom(I,A)}(\cst_a,F)\sim \laxlim_I\hom_A(a,F)$$
natural in $F: \gHom(I,A)$ and $a:A^t$.
\end{lemma}
\begin{proof}
Remark that there is a commutative square:
% q.uiver.app/#q=WzAsNCxbMSwwLCJcXGdIb20oSSxBKSJdLFsxLDEsIlxcZ0hvbShJLFxcdUhvbShBXnQsXFxvbWVnYSkpIl0sWzAsMCwiQSJdLFswLDEsIlxcdUhvbShBXnQsXFxvbWVnYSkiXSxbMiwwLCJcXGNzdCJdLFszLDEsIlxcY3N0IiwyXSxbMiwzLCJ5IiwyXSxbMCwxLCJcXGdIb20oSSx5KSJdXQ==
\[\begin{tikzcd}
	A & {\gHom(I,A)} \\
	{\w{A}} & {\gHom(I,\w{A})}
	\arrow["\cst", from=1-1, to=1-2]
	\arrow["\cst"', from=2-1, to=2-2]
	\arrow["y"', from=1-1, to=2-1]
	\arrow["{\gHom(I,y)}", from=1-2, to=2-2]
\end{tikzcd}\]
and that the right vertical morphism is fully faithful as $y$ is.
We then have a sequence of equivalences
$$
\begin{array}{rcll}
\hom_{\gHom(I,A)}(\cst_a,F)&\sim& \hom_{\gHom(I,\w{A})}(\cst_{y_a},\gHom(I,y)(F))\\
&\sim& \hom_{\w{A}}(y_a,\laxlim_{I}\gHom(I,y)(F)))\\
&\sim& (\laxlim_{I}\gHom(I,y)(F))(a)&\mbox{(Yoneda lemma)}\\
&\sim& \laxlim_I\hom_A(a,F(i))
\end{array}$$
where the last one comes from the fact that evaluations commute with lax limits.
\end{proof}

\begin{prop}
\label{prop:other characthereisation of limits}
Consider a functor $F:I\to A^\sharp$ between $\U$-small marked $\io$-categories. Then $F$ admits a lax limit if and only if there exists an object $l$ and an equivalence
$$\hom_A(a,l)\sim \laxlim_I\hom_A(a,F(i))$$
natural in $a:A^t$. If such an object exists, then $l$ is the lax limit of $F$.
Dually, $F$ admits a lax colimit if and only if there exists an object $c$ and an equivalence
$$\hom_A(c,a)\sim \laxlim_I\hom_A(F(i),a)$$
natural in $a:A$. If such an object exists, then $c$ is the lax colimit of $F$.
\end{prop}
\begin{proof}
The first assertion  is a direct application of lemma \ref{lemma:tehcnical colimit}. The second one follows by duality, using the fact that the functor 
$$(\uvar)^\circ:\uni\to \uni^{t\circ}$$
preserves limits as it is an equivalence.
\end{proof}



\begin{cor}
\label{cor:left adjoint preserves limits}
Left adjoints between $\U$-small $\io$-categories preserve colimits and right adjoints preserve limits.
\end{cor}
\begin{proof}
Let $u:C\to D$ and $v:D\to C$ be two adjoint functors. Let $F:I\to C^\sharp$ be a functor admitting a colimit.
We then have a sequence of equivalences
$$
\begin{array}{rclc}
\hom_C(u(\laxcolim_IF),b)&\sim &\hom_D(\laxcolim_IF,v(b))\\
&\sim & \laxlim_I\hom_D(F,v(b))&(\ref{prop:other characthereisation of limits})\\
&\sim &\laxlim_I\hom_C(u(F),b)\\
&\sim &\hom_C(\laxlim_Iu(F),b)&(\ref{prop:other characthereisation of limits})
\end{array}
$$
natural in $b:D$. The result then follows from the Yoneda lemma applied to $C^t$. The other assertion is proved similarly.
\end{proof}


\begin{cor}
Consider a functor $F:I\to A^\sharp$ between $\U$-small marked $\io$-categories. Then $F$ admits a limit if and only if there exists an object $l$ and an equivalence
$$\hom_A(a,l)\sim \hom_{\gHom(I,\uni)}(\cst 1,\hom_A(a,F(\uvar))$$
natural in $a:A^t$. If such an object exists, then $l$ is a limit of $F$.
Dually, $F$ admits a colimit if and only if there exists an object $c$ and an equivalence
$$\hom_A(c,a)\sim \hom_{\gHom(I,\uni)}( \cst 1,\hom_A(F(\uvar),a))$$
natural in $a:A$. If such an object exists, then $c$ is the colimit of $F$.
\end{cor}
\begin{proof}
Remark that we have an equivalence
$$\hom_{\gHom(I,\uni)}(\cst 1,\hom_A(a,F(\uvar)))\sim \hom_{\uni}(1,\laxlim_I\hom_A(a,F(\uvar))$$
Eventually, the Yoneda lemma implies that 
$$\hom_{\uni}(1,\laxlim_I\hom_A(a,F(\uvar))\sim\laxlim_I\hom_A(a,F(\uvar))$$
The result then follows from proposition \ref{prop:other characthereisation of limits}.
\end{proof}

\begin{remark}
The characterization of the lax colimit and limit given in previous corollary is the generalization to the case $\io$ of the characterization of lax colimit and limit for $(\infty,2)$-categories given in \cite[corollary 5.1.7]{Gagna_fibrations_and_lax_limit_infini_2_categories}.
\end{remark}

\begin{prop}
\label{prop:limit and final}
Let $i:I\to J$ and $F:J\to A^\sharp$ be two morphisms between $\U$-small marked $\io$-categories. If $i$ is initial, and $F$ admits a lax limit, the functor $F\circ i$ also admits a lax limit, and the canonical morphism:
$$\laxlim_{I}F\to \laxlim_{J}F\circ i$$
is an equivalence.
Dually, if $i$ is final, and $F$ admits a lax colimit, the functor $F\circ i$ also admits a lax colimit, and the canonical morphism:
$$ \laxcolim_{J}F\circ i \to \laxlim_{I}F$$
is an equivalence.
\end{prop}
\begin{proof}
The first assertion is a direct application of the characterization of limits given in proposition \ref{prop:other characthereisation of limits} and of proposition \ref{prop:colimit restricted to final}.
The second assertion follows by duality.
\end{proof}






The proof of the following lemma is a direct adaptation of the one of proposition 5.1 of \cite{Gepner_Lax_colimits_and_free_fibration}.
\begin{prop}
\label{prop:explicit hom between morphism}
Let $f:A\to B$ be any morphism between $\U$-small $\io$-categories..
There is an equivalence
$$\hom_{\uHom(A,B)}(f,g)\sim \laxlim_{a\to b: S(A)}\hom_{B}(f(a),g(a))$$
natural in $f$ and $g$.
\end{prop}
\begin{proof}
Remark first that the left term is in fact equivalent to 
$$\laxlim_{a\to b: S(A)}h^*\hom_{B}(\uvar,\uvar)$$
where $h$ is the left cartesian fibration $S(A)\to A^t\times A$ corresponding to $\hom_A: A^t\times A\to \uni$. We then have 
$$
\begin{array}{rcll}
\laxlim_{a\to b: S(A)}\hom_{B}(f(a),g(a)) &\sim &\hom_{\uni}(1,\laxlim_{a\to b: S(A)}h^*\hom_{B}(\uvar,\uvar)) &(\ref{theo:Yoneda lemma})\\
&\sim &\hom_{\gHom(S(A),\uni)}(\cst 1,h^*\hom_{B}(\uvar,\uvar))\\
&\sim &\hom_{\uHom(A^t\times A,\uni)}(h_! \cst 1,\hom_{B}(\uvar,\uvar))&(\ref{cor:naive kan extension})\\
\end{array}$$
By construction, $h_! \cst 1$ is the Grothendieck deconstruction of the left cartesian fibration $\Lb h_!id \sim h$, and so is equivalent to $\hom_A$. 
We then have 
$$\laxlim_{a\to b: S(A)}\hom_{B}(f(a),g(a))\sim \hom_{\uHom(A^t\times A,\uni)}(\hom_A(\uvar,\uvar),\hom_B(f(\uvar),g(\uvar)))$$
We have a canonical equivalence $\uHom(A^t\times A,\uni)\sim \uHom(A,\w{A})$ sending the functor $\hom_A$ to the Yoneda embedding $y^A$, and $\hom_B(f(\uvar),g(\uvar))$ is $f^*(y^B\circ g)$. 
We then have 
$$
\begin{array}{rcll}
\hom_{}(\hom_A(\uvar,\uvar),\hom_B(f(\uvar),g(\uvar)))&\sim &\hom_{\uHom(A,\w{A})}(y^A,f^*(y^B\circ g))\\
&\sim &\hom_{\uHom(A,\w{B})}(f_!\circ y^A,y^B\circ g)&(\ref{cor:naive kan extension})\\
&\sim &\hom_{\uHom(A,\w{B})}(y^B\circ f,y^B\circ g)&(\ref{prop:left extension commutes with Yoneda})\\
&\sim &\hom_{\uHom(A,B)}(f, g)&(\mbox{Yoneda lemma})\\
\end{array}$$
\end{proof}





\p We suppose the existence of a Grothendieck universe $\Z$ containing $\Wcard$. As a consequence, we can use all the results of the last three subsections to respectively $\V$-small and locally $\V$-small objects.

Let $A$ be a $\U$-small $\io$-category. Let $f$ be an object of $\w{A}$. We define $A^\sharp_{/f}$ as the following pullback
% q.uiver.app/#q=WzAsNCxbMSwwLCJcXHVIb20oQV50LFxcdWNhdCleXFxzaGFycF97L2Z9Il0sWzEsMSwiXFx1SG9tKEFedCxcXHVjYXQpXlxcc2hhcnAiXSxbMCwxLCJBXlxcc2hhcnAiXSxbMCwwLCJBXlxcc2hhcnBfey9mfSJdLFszLDJdLFszLDBdLFswLDFdLFsyLDFdXQ==
\[\begin{tikzcd}
	{A^\sharp_{/f}} & {\w{A}^\sharp_{/f}} \\
	{A^\sharp} & {\w{A}^\sharp}
	\arrow[from=1-1, to=2-1]
	\arrow[from=1-1, to=1-2]
	\arrow[from=1-2, to=2-2]
	\arrow[from=2-1, to=2-2]
\end{tikzcd}\]

\begin{theorem}
\label{theo:presheaevs colimi of representable}
The colimit of the functor 
$\pi:A^\sharp_{/f}\to A^\sharp\to \w{A}^\sharp$ is $f$.
\end{theorem}
\begin{proof}
We denote by $\pi'$ the canonical projection $\w{A}^\sharp_{/f}\to \w{A}^\sharp$, and 
$t_{A^\sharp_{/f}}:A^\sharp_{/f}\to 1$, $t_{ \w{A}^\sharp_{/f}}:\w{A}^\sharp_{/f}\to 1$ the canonical morphisms.
By the explicit construction of colimits in $\io$-presheaves, we have equivalences 
$$\int_{A^t}\colim_{A^\sharp_{/f}}\pi \sim (id_{(A^t)^\sharp}\times t_{A^\sharp_{/f}})_!E
~~~~~~~~
\int_{A^t}\colim_{\w{A}^\sharp_{/f}}\pi' \sim (id_{(A^t)^\sharp}\times t_{ \w{A}^\sharp_{/f}})_!F$$
where $E$ is the object of $\LCart(A^\sharp\times A^\sharp_{/f})$ induced by currying $\pi$, 
and $F$ is the object of $\LCart(A^\sharp\times \w{A}^\sharp_{/f})$ induced by currying $\pi'$.
We denote by $X\to A^\sharp\times A^\sharp_{/f}$ the left cartesian fibration corresponding to $E$,
and by $Y\to (A^t)^\sharp\times \w{A}^\sharp_{/f}$ the left fibration corresponding to $F$. All this data fits in the diagram
% https://q.uiver.app/#q=WzAsOCxbMCwyLCIoQV50KV5cXHNoYXJwXFx0aW1lcyBBXntcXHNoYXJwfV97L2Z9Il0sWzIsMiwiKEFedCleXFxzaGFycFxcdGltZXMgQV57XFxzaGFycH0iXSxbMiwwLCJTKEEpIl0sWzAsMCwiWCJdLFszLDMsIihBXnQpXlxcc2hhcnBcXHRpbWVzIFxcd3tBfV5cXHNoYXJwIl0sWzEsMywiKEFedCleXFxzaGFycFxcdGltZXMgKFxcd3tBfSlee1xcc2hhcnB9X3svZn0iXSxbMywxLCJcXGRvbShcXGludF97QV50XFx0aW1lcyBcXHd7QX19XFxldikiXSxbMSwxLCJZIl0sWzMsMCwiRSJdLFswLDFdLFszLDJdLFsyLDFdLFsxLDRdLFswLDVdLFsyLDYsImkiXSxbNiw0XSxbNSw0XSxbMyw3LCJqIl0sWzcsNl0sWzcsNSwiRiIsMCx7ImxhYmVsX3Bvc2l0aW9uIjo0MH1dXQ==
\[\begin{tikzcd}
	X && {S(A)} \\
	& Y && {\dom(\int_{A^t\times \w{A}}\ev)} \\
	{(A^t)^\sharp\times A^{\sharp}_{/f}} && {(A^t)^\sharp\times A^{\sharp}} \\
	& {(A^t)^\sharp\times (\w{A})^{\sharp}_{/f}} && {(A^t)^\sharp\times \w{A}^\sharp}
	\arrow["E", from=1-1, to=3-1]
	\arrow[from=3-1, to=3-3]
	\arrow[from=1-1, to=1-3]
	\arrow[from=1-3, to=3-3]
	\arrow[from=3-3, to=4-4]
	\arrow[from=3-1, to=4-2]
	\arrow["i", from=1-3, to=2-4]
	\arrow[from=2-4, to=4-4]
	\arrow[from=4-2, to=4-4]
	\arrow["j", from=1-1, to=2-2]
	\arrow[from=2-2, to=2-4]
	\arrow["F"{pos=0.4}, from=2-2, to=4-2]
\end{tikzcd}\]
where all squares are cartesian. 
Furthermore, according to the Yoneda lemma, $\dom(\int_{A^t\times \w{A}}\ev))$ is equivalent to $\dom(\int_{A^t\times \w{A}}\hom_{\w{A}}(y_{\uvar},\uvar))$, and 
lemma \ref{lemma:a particular Kan extension} implies that $i$ is initial. As the lower horizontal morphism is a right cartesian fibration, and the dual version of proposition \ref{prop:left cartesian fibration are smooth} induces that $j$ is initial. 
This implies that the canonical morphism
$$(id_{(A^t)^\sharp}\times \bot_{A^\sharp_{/f}})_!E\to (id_{(A^t)^\sharp}\times \bot_{ \w{A}^\sharp_{/f}})_!F$$
is an equivalence, and we then have
$$\colim_{A^\sharp_{/f}}\pi\sim \colim_{\w{A}^\sharp_{/f}}\pi'$$
However, 
$A^\sharp_{/f}$ admits a terminal element, given by $id_f$, and according to proposition \ref{prop:other characthereisation of limits}, we have 
$$\colim_{A^\sharp_{/f}}\pi\sim f.$$
\end{proof}

\begin{cor}
\label{cor:if cocomplete then Yoneda right adjoint}
A $\U$-small $\io$-category $A$ is lax $\U$-cocomplete if and only if the Yoneda embedding has a left adjoint, which we will also note by \wcnotation{$\laxcolim$}{(laxcolim@$\laxcolim:\widehat{C}\to C$}.
\end{cor}
\begin{proof}
If such a left adjoint exists, as $\w{A}$ is lax $\U$-cocomplete, corollary 
\ref{cor:left adjoint preserves limits} implies that $A$ is lax $\U$-cocomplete. Suppose now that $A$ is lax $\U$-cocomplete and let $f:A^t\to \uni$ be a functor. Let $c$ be the colimit of the functor $A^\sharp_{/f}\to A^\sharp$.
According to theorem \ref{theo:presheaevs colimi of representable}, we have a sequence of equivalences
$$\begin{array}{rcl}
 \hom_{\w{A}}(f,y(a))&\sim &\hom_{\w{A}}(\laxcolim_{A^\sharp_{/f}}y(\uvar),y(a))\\
 &\sim & \laxlim_{A^\sharp_{/f}}\hom_{\w{A}}(y(\uvar),y(a))\\
 &\sim & \laxlim_{A^\sharp_{/f}} \hom_A(\uvar,a)\\
 &\sim &\hom_{A} \hom(c,a)\\
\end{array}$$
natural in $a:A^t$. The functor 
$$a:A\mapsto \hom_{\w{A}}(f,y(a))$$
is then representable, which concludes the proof according to proposition \ref{prop:adj if slice as terminal}.
\end{proof}

\p Let $i:A\to B$ be a functor between two $\U$-small $\io$-categories. We define $N_i:B\to \w{A}$ as
$$a:A^t, b:B\mapsto \hom_B(i(a),b)$$
\begin{cor}
\label{cor:adjonction with prehseaves}
Let $i:A\to B$ be a functor between two $\U$-small $\io$-categories with $B$ lax $\U$-cocomplete. 
The morphism $N_i:B\to \w{A}$
admits a left adjoint that sends an $\io$-presheaf $f$ to $\laxcolim_{A^\sharp_{/f}} i(\uvar)$
\end{cor}
\begin{proof}
The proof is similar to the one of corollary \ref{cor:if cocomplete then Yoneda right adjoint}.
\end{proof}

\subsection{Kan extentions}

We suppose the existence of a Grothendieck universe $\Z$ containing $\Wcard$. As a consequence, we can use all the results of the last three subsections to respectively $\V$-small and locally $\V$-small objects.


\p 
Let $f:A\to B^\sharp$ be a morphism between marked $\U$-small $\io$-categories. This induces for any $\io$-category $C$ a morphism 
$$\uvar\circ f:\gHom(B,C)\to \uHom(A,C).$$
Let $g:A\to C$ be a morphism.
A \notion{left Kan extension} of $g$ along $f$ is a functor \sym{(lanf@$\Lan_fg$}$\Lan_fg:B\to C$ and an equivalence
$$\hom_{\uHom(B,C)}(\Lan_fg,h)\sim \hom_{\gHom(A,C)}(g, h\circ f).$$
Remark that if the left Kan extension along $f$ exists for any $g$, the proposition \ref{prop:adj if slice as terminal} implies that
the assignation $g\mapsto \Lan_fg$ can be promoted to a left adjoint, which is called the \notion{global left Kan extension} of $f$. 


\begin{prop}
\label{prop:Kan extension an naive kan extension}
Let $C$ be a $\U$-small $\io$-category, $f:I\to B^\sharp$ a functor between $\U$-small $\io$-categories and $g:I\to \uHom(C,\uni)$ a functor. The functor $g$ then corresponds to a morphism $\tilde{g}:\gHom( C^\sharp\times I,\uni)$.
The left Kan extension of $f$ along $g$ corresponds to the morphism $(id_{C^\sharp}\times f)_!\tilde{g}$.
\end{prop}
\begin{proof}
This is a direct consequence of corollary \ref{cor:naive kan extension}.
\end{proof}
	
\begin{cor}
\label{cor:Kan extension of Yonedal along i}
Let $i:A\to B$ be a morphism between $\U$-small $\io$-categories. The left Kan extension of the Yoneda embedding $y:A\to \w{A}$ along $i$ is $N_i:B\to \widehat{A}$.
\end{cor}
\begin{proof}
According to proposition \ref{prop:Kan extension an naive kan extension}, the desired left Kan extension is given by 
$$(B^t\times i)_!\hom_B$$
which is $N_i$ according to lemma \ref{lemma:a particular Kan extension}.
\end{proof}

\begin{prop}
Let $i:A\to B$ a functor between $\U$-small $\io$-categories. The left Kan extension of $y^B\circ i$ along $y^A$ is given by $i_!$.
\end{prop}
\begin{proof}
Let $i:A\to B$ be any functor. Remark first that the Yoneda lemma and the corollary \ref{cor:Kan extension of Yonedal along i} imply that the left Kan extension of $y:A\to \w{A}$ along $y:A\to \w{A}$ is the identity of $\w{A}$.
We then have a sequence of equivalences
$$
\begin{array}{rcll}
\hom_{\uHom(\w{A},\w{A})}(i_!,f)&\sim &\hom_{\uHom(\w{A},\w{A})}(id,i^*\circ f)&(\ref{cor:naive kan extension}) \\
&\sim & \hom_{\uHom(A,\w{A})}(y_A,i^*\circ f\circ y^A)&(\mbox{Yoneda lemma}) \\
&\sim & \hom_{\uHom(A,\w{B})}(i_! \circ y^A, f\circ y^A)&(\ref{cor:naive kan extension}) \\
&\sim & \hom_{\uHom(A,\w{B})}( y_B\circ i, f\circ y^A)&(\ref{prop:left extension commutes with Yoneda})\\
\end{array}
$$
natural in $f:\uHom(\w{A},\w{B})$.
\end{proof}







\begin{cor}
For any morphism $A\to B$ between $\U$-small $\io$-categories with $B$ lax $\U$-cocomplete, there exists a unique colimit preserving functor $\w{A}\to B$ extending $i$.
\end{cor}
\begin{proof}
Let $|\uvar|_i: \w{A}\to B$ be the functor defined in corollary \ref{cor:adjonction with prehseaves}. 
As this functor is an extension of $A$, it fulfills the desired condition, that shows the existence.
The $\io$-category of functors verifying the desired property is given by the pullback 
% https://q.uiver.ap/#q=WzAsNCxbMSwwLCJcXHVIb21fIShcXHd7QX0sQikiXSxbMSwxLCJcXHVIb20oQSxCKSJdLFswLDEsIlxce2lcXH0iXSxbMCwwLCJcXHVIb21fIShcXHd7QX0sQilfe2l9Il0sWzIsMV0sWzAsMV0sWzMsMl0sWzMsMF1d
\[\begin{tikzcd}
	{\uHom_!(\w{A},B)_{i}} & {\uHom_!(\w{A},B)} \\
	{\{i\}} & {\uHom(A,B)}
	\arrow[from=2-1, to=2-2]
	\arrow[from=1-2, to=2-2]
	\arrow[from=1-1, to=2-1]
	\arrow[from=1-1, to=1-2]
\end{tikzcd}\]
where $\uHom_!(\w{A},B)$ is the full sub $\io$-category of $\uHom(\w{A},B)$ whose objects are colimit preserving functors.
As $|\uvar|_i$ is the left Kan extension of $i$ along the Yoneda embedding, there is a transformation 
$$|\uvar|_i\to h$$ natural in $h:\uHom(\w{A},B))_{i}$. To conclude, one has to show that for any object $h$ of $\uHom(\w{A},B))_{i}$ , $|\uvar|_i\to h$ is an equivalence, and so that for any object $f$ of $\w{A}$, $|f|_i\to h(f)$ is an equivalence. As $f$ is a lax colimit of representables as shown in theorem \ref{theo:presheaevs colimi of representable} and as both $|\uvar|_i$ and $h$ preserve lax colimits, this is immediate.
\end{proof}

\begin{cor}
Let $A,B$ and $C$ be three $\U$-small $\io$-categories with $B$ lax $\U$-cocomplete, and
$i:A\to C$ and $f:A\to B$ two functors. The left Kan extension of $i$ along $f$ is given by the composite functor.
$$B\xrightarrow{N_f}\w{A}\xrightarrow{i_!}\w{C}\xrightarrow{\laxcolim_{}} C$$
\end{cor}
\begin{proof}
We have a sequence of equivalences
$$\begin{array}{rcll}
\hom_{\uHom(C,B)}(\laxcolim_{}\circ i_!\circ N_f,h)&\sim & \hom_{\uHom(C,\w{A})}( N_f,i^*\circ y^B\circ h)\\
&\sim & \hom_{\uHom(A,\w{A})}( y^A,i^*\circ y^B\circ h\circ f)&(\ref{cor:Kan extension of Yonedal along i})\\
&\sim & \hom_{\uHom(A,\w{B})}(i_!\circ y^A, y^B\circ h\circ f)&(\ref{cor:naive kan extension})\\
&\sim & \hom_{\uHom(A,\w{B})}(y^B\circ i, y^B\circ h\circ f)&(\ref{prop:left extension commutes with Yoneda})\\
&\sim & \hom_{\uHom(A,B)}( i, h\circ f)&(\ref{theo:Yoneda ff})
\end{array}$$
natural in $h:\uHom(C,B)$.
\end{proof}

%%
%%%\bibliography{../../header/biblio}{}
%\bibliographystyle{alpha}
%\printindex[notation]
%\printindex
%\printindex[notion]
%\end{document}

