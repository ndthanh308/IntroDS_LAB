%\documentclass[12pt]{book}
%\usepackage{index}
%\makeindex
%\renewcommand\indexname{Index of notions}
%\newindex{notation}{adx}{and}{Index of symbols}
%\newindex{notion}{bdx}{bnd}{Index of notions}
%\usepackage{tikz}
\usepackage{xcolor,xspace}
\usepackage{url}
\usepackage{epsfig,graphicx,endnotes,kotex,subfigure,multirow,amsmath,algorithm,algpseudocode}
\newcommand\StateX{\Statex\hspace{\algorithmicindent}}%
%\usepackage{breakurl}
%\usepackage[sort,space]{cite}
\usepackage{balance}
%\usepackage{tabularx}
\usepackage{enumitem}
\usepackage{flushend}
\usepackage[T1]{fontenc}
\usepackage{color,soul}
\hyphenation{op-tical net-works semi-conduc-tor}
%\usepackage{filecontents}
%\usepackage{booktabs} % For formal tables
\usepackage{amsthm}
\newtheorem{theorem}{Theorem}
\newtheorem{corollary}{Corollary}
\newtheorem{lemma}{Lemma}
\renewcommand{\qedsymbol}{\rule{0.7em}{0.7em}}

%\newcommand\notion[1]{\textit{#1}\index[notion]{#1}}
\newcommand\wcnotion[2]{\textit{#1}\index[notion]{#2}}
\newcommand\wcnotionsym[3]{\textit{#1}\index[notation]{#2}\index[notion]{#3}}
\newcommand\wcsnotion[3]{\textit{#1}\index[notion]{#2!\textit{#3}}}
\newcommand\snotion[2]{\textit{#1}\index[notion]{#1!\textit{#2}}}
\newcommand\snotionsym[3]{\textit{#1}\index[notion]{#1!\textit{#3}}\index[notation]{#2!\textit{#3}}}
\newcommand\wcsnotionsym[4]{\textit{#1}\index[notation]{#2!\textit{#4}}\index[notion]{#3!\textit{#4}}}

\newcommand\wcnotation[2]{\textit{#1}\index[notation]{#2}}
\newcommand\wcsnotation[3]{\textit{#1}\index[notation]{#2!\textit{#3}}}

\newcommand\sym[1]{\index[notation]{#1}}
\newcommand\ssym[2]{\index[notation]{#1!\textit{#2}}}

\newcommand{\exclam}{!}





\newcommand{\Ab}{\mathbb{A}} 
\newcommand{\Zb}{\mathbb{Z}} 
\newcommand{\Eb}{\mathbb{E}} 
\newcommand{\Nb}{\mathbb{N}}
\newcommand{\Tb}{\mathbf{T}} 
\newcommand{\Yb}{\mathbb{Y}} 
\newcommand{\Ib}{\mathbb{I}} 
\newcommand{\Ob}{\mathbb{O}} 
\newcommand{\Pb}{\mathbb{P}} 
\newcommand{\Qb}{\mathbb{Q}} 
\newcommand{\Sb}{\mathbb{S}} 
\newcommand{\Hb}{\mathbb{H}} 
\newcommand{\Jb}{\mathbf{J}} 
\newcommand{\Kb}{\mathbb{K}} 
\newcommand{\Mb}{\mathbb{M}} 
\newcommand{\Wb}{\mathbf{W}} 
\newcommand{\Xb}{\mathbb{X}} 
\newcommand{\Cb}{\mathbf{C}}
\newcommand{\Vb}{\mathbb{V}}
\newcommand{\Bb}{\mathbb{B}}


\newcommand{\Acal}{\mathcal{A}} 
\newcommand{\Zcal}{\mathcal{Z}} 
\newcommand{\Ecal}{\mathcal{E}} 
\newcommand{\Rcal}{\mathcal{R}} 
\newcommand{\Tcal}{\mathcal{T}} 
\newcommand{\Ycal}{\mathcal{Y}} 
\newcommand{\Ucal}{\mathcal{U}} 
\newcommand{\Ical}{\mathcal{I}} 
\newcommand{\Ocal}{\mathcal{O}} 
\newcommand{\Pcal}{\mathcal{P}} 
\newcommand{\Qcal}{\mathcal{Q}} 
\newcommand{\Scal}{\mathcal{S}} 
\newcommand{\Dcal}{\mathcal{D}} 
\newcommand{\Fcal}{\mathcal{F}} 
\newcommand{\Gcal}{\mathcal{G}} 
\newcommand{\Hcal}{\mathcal{H}} 
\newcommand{\Jcal}{\mathcal{J}} 
\newcommand{\Kcal}{\mathcal{K}} 
\newcommand{\Lcal}{\mathcal{L}} 
\newcommand{\Mcal}{\mathcal{M}} 
\newcommand{\Wcal}{\mathcal{W}} 
\newcommand{\Xcal}{\mathcal{X}} 
\newcommand{\Ccal}{\mathcal{C}} 
\newcommand{\Vcal}{\mathcal{V}} 
\newcommand{\Bcal}{\mathcal{B}} 
\newcommand{\Ncal}{\mathcal{N}} 


\newcommand{\Ago}{\mathfrak{A}} 
\newcommand{\Zgo}{\mathfrak{Z}} 
\newcommand{\Ego}{\mathfrak{E}} 
\newcommand{\Rgo}{\mathfrak{R}} 
\newcommand{\Tgo}{\mathfrak{T}} 
\newcommand{\Ygo}{\mathfrak{Y}} 
\newcommand{\Ugo}{\mathfrak{U}} 
\newcommand{\Igo}{\mathfrak{I}} 
\newcommand{\Ogo}{\mathfrak{O}} 
\newcommand{\Pgo}{\mathfrak{P}} 
\newcommand{\Qgo}{\mathfrak{Q}} 
\newcommand{\Sgo}{\mathfrak{S}} 
\newcommand{\Dgo}{\mathfrak{D}} 
\newcommand{\Fgo}{\mathfrak{F}} 
\newcommand{\Ggo}{\mathfrak{G}} 
\newcommand{\Hgo}{\mathfrak{H}} 
\newcommand{\Jgo}{\mathfrak{J}} 
\newcommand{\Kgo}{\mathfrak{K}} 
\newcommand{\Lgo}{\mathfrak{L}} 
\newcommand{\Mgo}{\mathfrak{M}} 
\newcommand{\Wgo}{\mathfrak{W}} 
\newcommand{\Xgo}{\mathfrak{X}} 
\newcommand{\Cgo}{\mathfrak{C}} 
\newcommand{\Vgo}{\mathfrak{V}} 
\newcommand{\Bgo}{\mathfrak{B}} 
\newcommand{\Ngo}{\mathfrak{N}}



\newcommand{\sslash}{\mathbin{/\mkern-6mu/}}

\newcommand{\note}[1]{{\color{red}#1}}

\def\-{\raisebox{.75pt}{-}}


\newcommand{\uvar}{\_}


%basic notation
\newcommand{\id}{\text{Id}}
\newcommand{\Db}{\mathbf{D}} 
\DeclareMathOperator*{\dom}{dom}
\DeclareMathOperator*{\codom}{codom}
\DeclareMathOperator{\tw}{tw}


%derived notation
\newcommand{\Rb}{\mathbf{R}} 
\newcommand{\Lb}{\mathbf{L}} 
\newcommand{\Fb}{\mathbf{F}} 
\DeclareMathOperator{\Gb}{G} 
  
%ambiguous notation 
\DeclareMathOperator{\N}{N}
\DeclareMathOperator{\T}{T}
\DeclareMathOperator{\J}{J}


%set of maps
\DeclareMathOperator*{\W}{W}
\DeclareMathOperator*{\Wm}{tW}
\DeclareMathOperator*{\Wseg}{W_{Seg}}
\DeclareMathOperator*{\Wsat}{W_{Sat}}

\DeclareMathOperator*{\M}{M}
\DeclareMathOperator*{\Mm}{tM}
\DeclareMathOperator*{\Mseg}{M_{Seg}}
\DeclareMathOperator*{\Msat}{M_{Sat}}

\DeclareMathOperator*{\I}{I}
\DeclareMathOperator*{\F}{F}

%augmented directed complexes
\DeclareMathOperator*{\CDA}{ADC}
\DeclareMathOperator*{\CDAB}{ADC_B}

%categories
\newcommand\omegacat{\omega\mbox{-$\cat$}}
\DeclareMathOperator\Set{Set}
\DeclareMathOperator\Sp{Sp}

%infini groupoids
\DeclareMathOperator*{\Sq}{Sq}
\DeclareMathOperator*{\Li}{Li}
\DeclareMathOperator{\Hom}{Hom}


%infini 1 categories
\DeclareMathOperator*{\Lfib}{LFib}
\DeclareMathOperator*{\Rfib}{RFib}

\DeclareMathOperator*{\LCartoperator}{LCart}
\DeclareMathOperator*{\core}{core}
\newcommand{\LCart}{\mbox{$\LCartoperator$}}

\newcommand{\LCartc}{\mbox{$\LCartoperator$}^c}
\DeclareMathOperator*{\RCart}{RCart}
\DeclareMathOperator*{\RCartc}{RCart^c}




%infini omega categories
\newcommand{\uLCart}{\underline{\LCartoperator}}
\newcommand{\uLCartc}{\underline{\LCartoperator}^c}
\newcommand{\uRCart}{\underline{RCart}}
\newcommand{\uRCartc}{\underline{RCart}^c}

\DeclareMathOperator{\uHom}{\underline{Hom}}
\DeclareMathOperator{\gHom}{\underline{Hom}_{\ominus}}
\DeclareMathOperator{\Map}{Map}
\DeclareMathOperator{\im}{Im}

\newcommand{\uni}{\underline{\omega}}
\newcommand\w[1]{\widehat{#1}}

%functors
\DeclareMathOperator*{\ev}{ev}
\DeclareMathOperator*{\Arr}{Arr}
\newcommand{\Noiun}{\N_{\tiny{(\omega,1)}}}


\newcommand{\colim}{\operatornamewithlimits{colim}}
\newcommand{\laxcolim}{\operatornamewithlimits{laxcolim}}
\newcommand{\laxlim}{\operatornamewithlimits{laxlim}}


%prefixes
\DeclareMathOperator{\Lan}{Lan}
\DeclareMathOperator{\Ran}{Ran}
\newcommand\iun{(\infty,1)}
\newcommand\io{(\infty,\omega)}
\newcommand\ioun{(\infty,\omega,1)}
\newcommand\zoun{(0,\omega,1)}
\newcommand\zo{(0,\omega)}

%leibnitz products
\DeclareMathOperator{\hstar}{\hat{\star}}
\DeclareMathOperator{\htimes}{\hat{\times}}
\DeclareMathOperator{\hotimes}{\hat{\otimes}}


%Gray operations
\DeclareMathOperator{\costarindex}{f}
\newcommand{\costar}{\mathbin{\overset{co}{\star}}}
\newcommand{\fwedge}{\mathbin{\rotatebox[origin=c]{270}{$\gtrdot$}}}


%inclassable
\newcommand{\invamalg}{\mathbin{\rotatebox[origin=c]{180}{$\amalg$}}}
\DeclareMathOperator{\botimes}{\bar{\otimes}}
\DeclareMathOperator\cst{cst}
\DeclareMathOperator\Operatormark{mk}
\newcommand{\mk}{\Operatormark}

%category theory
\DeclareMathOperator\Fun{Fun}
\DeclareMathOperator\Nat{Nat}
\DeclareMathOperator\End{End}



%fundamental notation
\DeclareMathOperator\mcat{cat_m}
\DeclareMathOperator\cat{cat}
\DeclareMathOperator\grd{grd}
\DeclareMathOperator\R{R}

\newcommand\ocat{(\infty,\omega)\mbox{-$\cat$}}
\newcommand\ouncat{(\infty,\omega,1)\mbox{-$\cat$}}
\newcommand\ocatm{{(\infty,\omega)\mbox{-$\mcat$}}}
\newcommand\zocatm{(0,\omega)\mbox{-$\mcat$}}
\newcommand\zocat{(0,\omega)\mbox{-$\cat$}}
\DeclareMathOperator\zocatB{\zocat_B}
\newcommand\icat{(\infty,1)\mbox{-$\cat$}}
\newcommand\qcat{\mbox{Q$\cat$}}
\newcommand\ncat[1]{(\infty, #1)\mbox{-$\cat$}}
\newcommand\zncat[1]{(0, #1)\mbox{-$\cat$}}
\newcommand\igrd{\infty\mbox{-$\grd$}}



\DeclareMathOperator{\OperatorinfiniPsh}{Psh^\infty}
\DeclareMathOperator{\OperatorinfinitPsh}{tPsh^\infty}
\DeclareMathOperator{\OperatorPsh}{Psh}
\DeclareMathOperator{\OperatormPsh}{mPsh}
\DeclareMathOperator{\OperatortPsh}{tPsh}
\newcommand\iPsh[1]{\OperatorinfiniPsh({#1})}
\newcommand\tiPsh[1]{\OperatorinfinitPsh({#1})}
\newcommand\Psh[1]{\OperatorPsh({#1})}
\newcommand\ssetPsh[1]{\OperatorPsh_\Delta({#1})}
\newcommand\tPsh[1]{\OperatortPsh({#1})}
\newcommand\tPshM[1]{{\OperatortPsh}_M({#1})}
\newcommand\mPsh[1]{\OperatormPsh({#1})}
\newcommand\mPshM[1]{{\OperatormPsh}_M({#1})}

%segal stuff
\DeclareMathOperator{\OperatorSeg}{Seg}
\DeclareMathOperator{\OperatortSeg}{tSeg}
\DeclareMathOperator{\OperatormSeg}{mSeg}
\newcommand\Seg{\OperatorSeg}
\newcommand\mSeg{\OperatormSeg}
\newcommand\stratSeg{\OperatortSeg}

%simplicial variations
\DeclareMathOperator{\Sset}{\Psh{\Delta}}
\newcommand{\mSset}{\mPsh{\Delta}}
\newcommand{\stratSset}{\tPsh{\Delta}}


%univers
\DeclareMathOperator{\U}{\mathbf{U}}
\DeclareMathOperator{\V}{\mathbf{V}}
\DeclareMathOperator{\Wcard}{\mathbf{W}}
\DeclareMathOperator{\Z}{\mathbf{Z}}



%Grothendieck constructions
\newcommand{\ringpartial}{\mathring{\partial}}
%
%
%\usepackage[inline]{showlabels}
%
%\usepackage{fancyhdr}
%\usepackage{titlesec}
%\usepackage{textcase}
%
%\pagestyle{fancy}
%
%
%\fancyfoot[C]{\thepage}
%
%
%\title{\Huge{Theory and models of $(\infty,\omega)$-categories}}
%\author{Félix Loubaton}
%\date{}
%\linespread{1.2}	
%\geometry{a4paper,top=3cm,bottom=4cm,left=1.5cm,right=3cm, heightrounded,bindingoffset=5mm}	
%
%
%\fancyhf{}
%\fancyhfoffset[RO,LE]{0.5cm}
%\fancyhfoffset[LE,RO]{0.5cm}
%
%\fancyhead[RO]{\rmfamily\nouppercase{\rightmark}}
%\fancyhead[LE]{\rmfamily\nouppercase{\leftmark}}
%\fancyfoot[C]{\thepage}
%
%
%\begin{document}
%\maketitle 


This chapter is devoted to the study of Verity's complicial sets (\cite{Verity_complicial_set_characterising_the_simplicial_nerve}).
One of the benefits of complicial sets is that they admit a simple definition of the Gray tensor product. Being strongly linked to $\zo$-categories by the Street nerve, they are also a privileged framework for stating and proving strictification results, as done in \cite{Ozornova_Fundamental_pushouts_of_n_complical_set}, \cite{Gagna_Nerves_and_cones_of_free_loop_free_omega_categories}, \cite{Ozornova_a_quillen_adjunction_between_globular_and_complicial} and \cite{Maehara_oriental_as_free_weak_omega_categories}. 
However, they do not interact \textit{a priori} well with the globular language. The goal of this chapter is to show that, with some computation, it is possible to have a globular point of view in this model. 

The first section is a recollection of usual results and definitions about complicial sets. 
In the second section, we aim to prove an analogue of the formula given in \ref{theo:appendice formula for otimes} to the complicial setting.
We also have a suspension in this category, which is denoted by $X\mapsto \Sigma X$. Objects $[1]\fwedge \Sigma X$ and $\Sigma X\fwedge [1]$ are defined in \ref{subsection:wedge}, but for now, we can suppose that they are fibrant replacements of respectively $[1]\coprod_{[0]}\Sigma X$ and $\Sigma X\coprod_{[0]}[1]$.
They come along with morphisms that are analogue to whiskerings, and that we also note by $\triangledown$: 
$$\triangledown:\Sigma X\to [1]\fwedge\Sigma X ~~~~\mbox{and}~~~~ \triangledown:\Sigma X\to\Sigma X\fwedge [1].$$ 
We then show the following theorem:
\begin{itheorem}[\ref{theo:interval_first_formula}]
There exists a zigzag of acyclic cofibrations, natural in $X$, between $(\Sigma X)\otimes [1]$ and the colimit of the following diagram:
 $$\Sigma X\fwedge [1]\xleftarrow{\triangledown} \Sigma (X\otimes\{0\}) \hookrightarrow \Sigma (X\otimes[1])\hookleftarrow \Sigma (X\otimes\{1\})\xrightarrow{\triangledown} [1]\fwedge \Sigma X.$$
\end{itheorem}
We also provide similar formulas for the \textit{Gray cone} and Gray \textit{$\circ$-cone}:
\begin{itheorem}[\ref{theo:cyl_formula}]
There exists a zigzag of acyclic cofibrations, natural in $X$, between $\Sigma X \star[0]$ and the colimit of the following diagram: 
$$ \Sigma X\fwedge [1]\leftarrow \Sigma X\to \Sigma([0]\costar X).$$
There exists a zigzag of acyclic cofibrations, natural in $X$, between  $[0]\costar \Sigma X$ and the colimit of the following diagram: 
$$\Sigma(X\star[0]) \leftarrow \Sigma X\to [1]\fwedge\Sigma X.$$
\end{itheorem}
The third section uses this formula and the strictification result of Gagna, Ozornova and	 Rovelli (\cite{Gagna_Nerves_and_cones_of_free_loop_free_omega_categories}) to demonstrate a criterion for detecting autoequivalences of complicial sets by their behavior on globes.
Indeed, in section \ref{section:Globular equivalences}, by iterating the suspension, we construct a globular object: 
% https://q.uiver.app/?q=WzAsNCxbMCwwLCJcXERiXzAiXSxbMSwwLCJcXERiXzEiXSxbMiwwLCJcXERiXzIiXSxbMywwLCIuLi4iXSxbMCwxLCJpXzBeKyIsMCx7Im9mZnNldCI6LTJ9XSxbMSwyLCJpXzFeKyIsMCx7Im9mZnNldCI6LTJ9XSxbMiwzLCJpXzNeKyIsMCx7Im9mZnNldCI6LTJ9XSxbMCwxLCJpXzBeLSIsMix7Im9mZnNldCI6Mn1dLFsxLDIsImlfMV4tIiwyLHsib2Zmc2V0IjoyfV0sWzIsMywiaV8zXi0iLDIseyJvZmZzZXQiOjJ9XV0=
\[\begin{tikzcd}
	{\Db_0} & {\Db_1} & {\Db_2} & {...}
	\arrow["{i_0^+}", shift left=2, from=1-1, to=1-2]
	\arrow["{i_1^+}", shift left=2, from=1-2, to=1-3]
	\arrow["{i_3^+}", shift left=2, from=1-3, to=1-4]
	\arrow["{i_0^-}"', shift right=2, from=1-1, to=1-2]
	\arrow["{i_1^-}"', shift right=2, from=1-2, to=1-3]
	\arrow["{i_3^-}"', shift right=2, from=1-3, to=1-4]
\end{tikzcd}\]
\begin{itheorem}[\ref{theo:criterion_to_be_linked_to_identity}]
Let $i$ be a left Quillen endofunctor for the model category for complicial sets. Suppose that there exists a zigzag of weakly invertible natural transformations:
$$i(\Db_{\uvar}) \leftrightsquigarrow \Db_{\uvar}.$$
Then, there exists a zigzag of weakly invertible natural transformations between $i$ and $id$.
\end{itheorem} 
Proposition 15.10 of \cite{Barwick_on_the_unicity_of_the_theory_of_higher_categories} provides a similar result for models of $(\infty,n)$-categories.

\section{Preliminaries}


\subsection{Generalities on model categories}
\label{chapter:Generalities on model categories}
For this chapter, we fix a model category $C$ whose cofibrations are monomorphisms.




\p We give first some results on homotopy colimits. These results will be used freely throughout these first two chapters.



\begin{prop}
\label{prop:hom colimit 1}
Suppose given a square
% https://q.uiver.app/#q=WzAsNCxbMCwwLCJhIl0sWzAsMSwiYyJdLFsxLDAsImIiXSxbMSwxLCJkIl0sWzAsMV0sWzIsM10sWzEsM10sWzAsMl1d
\[\begin{tikzcd}
	a & b \\
	c & d
	\arrow[from=1-1, to=2-1]
	\arrow[from=1-2, to=2-2]
	\arrow[from=2-1, to=2-2]
	\arrow[from=1-1, to=1-2]
\end{tikzcd}\]
such that the two horizontal morphisms are weak equivalences. Then this square is homotopy cocartesian. 
\end{prop}
\begin{proof}
This is \cite[proposition 2.3.26]{Cisinski_Higher_categories_and_homotopical_algebra}.
\end{proof}
\begin{prop}
\label{prop:hom colimit 2}
Suppose given a cocartesian square
% https://q.uiver.app/#q=WzAsNCxbMCwwLCJhIl0sWzAsMSwiYyJdLFsxLDAsImIiXSxbMSwxLCJkIl0sWzAsMV0sWzIsM10sWzEsM10sWzAsMl0sWzMsMCwiIiwxLHsic3R5bGUiOnsibmFtZSI6ImNvcm5lciJ9fV1d
\[\begin{tikzcd}
	a & b \\
	c & d
	\arrow[from=1-1, to=2-1]
	\arrow[from=1-2, to=2-2]
	\arrow[from=2-1, to=2-2]
	\arrow[from=1-1, to=1-2]
	\arrow["\lrcorner"{anchor=center, pos=0.125, rotate=180}, draw=none, from=2-2, to=1-1]
\end{tikzcd}\]
where the left vertical morphism is a cofibration. Then this square is homotopy cocartesian. 
\end{prop}
\begin{proof}
This is \cite[corollary 2.3.28]{Cisinski_Higher_categories_and_homotopical_algebra}.
\end{proof}

\begin{prop}
\label{prop:hom colimit 3}
Let $F:\alpha\to C$ be a diagram indexed by an ordinal. The transfinite composition $\colim_\alpha F$ is the homotopy colimit of the diagram $F$.
\end{prop}
\begin{proof}
This is \cite[proposition 2.3.13]{Cisinski_Higher_categories_and_homotopical_algebra}.
\end{proof}


\begin{prop}
\label{prop:hom colimit 4}
Suppose given a diagram 
% https://q.uiver.app/#q=WzAsNyxbMSwwLCJiXzAiXSxbMCwxLCJhXzAiXSxbMiwxLCJhXzEiXSxbMywwLCIuLi4iXSxbNCwxLCJhX3tuLTF9Il0sWzYsMSwiYV97bn0iXSxbNSwwLCJiX3tuLTF9Il0sWzAsMV0sWzAsMiwiIiwyLHsic3R5bGUiOnsidGFpbCI6eyJuYW1lIjoiaG9vayIsInNpZGUiOiJ0b3AifX19XSxbMywyXSxbMyw0LCIiLDIseyJzdHlsZSI6eyJ0YWlsIjp7Im5hbWUiOiJob29rIiwic2lkZSI6InRvcCJ9fX1dLFs2LDRdLFs2LDUsIiIsMix7InN0eWxlIjp7InRhaWwiOnsibmFtZSI6Imhvb2siLCJzaWRlIjoidG9wIn19fV1d
\[\begin{tikzcd}
	& {b_0} && {...} && {b_{n-1}} \\
	{a_0} && {a_1} && {a_{n-1}} && {a_{n}}
	\arrow[from=1-2, to=2-1]
	\arrow[hook, from=1-2, to=2-3]
	\arrow[from=1-4, to=2-3]
	\arrow[hook, from=1-4, to=2-5]
	\arrow[from=1-6, to=2-5]
	\arrow[hook, from=1-6, to=2-7]
\end{tikzcd}\]
where all morphisms labelled by $\hookrightarrow$ are cofibrations.
The colimit of this diagram is also the homotopy colimit of this diagram.
\end{prop}
\begin{proof}
Let $I_n$ be the category indexing the previous diagram. We denote $i_0$, $j_0$,..., $i_{n-1}$, $j_{n-1}$, $i_{n}$ it's objects.
The projective model structure on $\Fun(I_n,C)$ is given by functor $G$ such that for any $k<n$, $F(j_k)\to F(i_k)$, $F(j_k)\to F(i_{k+1})$ are monomorphisms, and such that for any $0<k<n$, $F(j_k)\coprod F(j_{k+1}) \to F(i_k)$ is a monomorphism. Remark that such presheaves verify the condition given in the statement of the proposition.

 We will show on induction on $n$ that a natural transformation $\psi$ between two diagrams $F,G:I_n\to C$ that fulfills the desired condition induces a weak equivalence between their colimits. As we can always chose $F$ to be the cofibrant replacement of $G$ in the projective model structure on $\Fun(I_n,C)$, it will imply the desired result. 

The case $n=1$ is proposition \ref{prop:hom colimit 2}. Suppose now the result is true at the stage $(n-1)$ and let $\psi$ be a weakly invertible natural transformation between two diagram $F,G:I_n\to C$ that fulfills the desired condition. We denote by $\iota:I_{n-1}\to I_n$ the canonical inclusion that sends $i_k$(resp. $j_k$) on $i_k$(resp. $j_k$) for $k<n$ (resp. $k<n-1$).
We then have a diagram 
% https://q.uiver.app/#q=WzAsNixbMiwwLCJGKGlfbikiXSxbMSwwLCJGKGpfe24tMX0pIl0sWzAsMCwiXFxjb2xpbV97SV97bi0xfX1GXFxjaXJjXFxpb3RhIl0sWzAsMSwiXFxjb2xpbV97SV97bi0xfX1HXFxjaXJjXFxpb3RhIl0sWzEsMSwiRyhqX3tuLTF9KSJdLFsyLDEsIkcoaV9uKSJdLFsxLDAsIiIsMix7InN0eWxlIjp7InRhaWwiOnsibmFtZSI6Imhvb2siLCJzaWRlIjoidG9wIn19fV0sWzEsMl0sWzIsMywiXFxzaW0iLDJdLFswLDUsIlxcc2ltIiwyXSxbMSw0LCJcXHNpbSIsMl0sWzQsM10sWzQsNSwiIiwyLHsic3R5bGUiOnsidGFpbCI6eyJuYW1lIjoiaG9vayIsInNpZGUiOiJ0b3AifX19XV0=
\[\begin{tikzcd}
	{\colim_{I_{n-1}}F\circ\iota} & {F(j_{n-1})} & {F(i_n)} \\
	{\colim_{I_{n-1}}G\circ\iota} & {G(j_{n-1})} & {G(i_n)}
	\arrow[hook, from=1-2, to=1-3]
	\arrow[from=1-2, to=1-1]
	\arrow["\sim"', from=1-1, to=2-1]
	\arrow["\sim"', from=1-3, to=2-3]
	\arrow["\sim"', from=1-2, to=2-2]
	\arrow[from=2-2, to=2-1]
	\arrow[hook, from=2-2, to=2-3]
\end{tikzcd}\]
where all arrows labeled by $\sim$ are weak equivalences. Remark furthermore that the limit of the two lines are respectively $\colim_{I_{n}}F$ and $\colim_{I_{n}}G$. A last application of proposition \ref{prop:hom colimit 2} concludes the proof.
\end{proof}




\p 
The definition of elegant Reedy category is given in paragraph \ref{para:reedy}.
As all the presheaves categories that we will encounter through this text are presheaves on elegant Reedy categories, we will use freely the following theorem:
\begin{theorem}[Hirschhorn]
\label{theo:hom colimi}
We suppose that $C$ is a simplicial model category.
Let $A$ be a elegant Reedy category, and $F:A\to C$ a functor such that the induced morphism $\colim_{\partial a}F\to F(a)$ is a monomorphism for any object $a$. The object $\colim_A F$ is the homotopy colimit of $F$. In particular, if $C$ is $\Psh{A}$, every object $X$ is the homotopy colimit of the diagram $A_{/X}\to A\to \Psh{A}$.
\end{theorem}
\begin{proof}
Using the characterization of elegant Reedy category given by proposition 3.8 of \cite{Bergner_reedy_category_and_the_theta_construction}, and \cite[proposition 15.10.2]{Hirschhorn_Model_categories_and_their_localizations},
it's easy to see that they have fibrant constant in the sens of \cite[definition 15.10.1]{Hirschhorn_Model_categories_and_their_localizations}. We can then apply the theorem 19.9.1	 of \cite{Hirschhorn_Model_categories_and_their_localizations}.
\end{proof}


\p
\label{para:nice model structure}
A model structure is \wcnotion{nice}{nice model structure} if it is simplicial, combinatorial, cartesian and its cofibrations are monomorphisms. 

\begin{notation}
Let $\uvar\square\uvar:C\times D\to E$ be a bifunctor. 
If $f:a\to b$ and $g:x\to y$ are respectively morphisms of $C$ and $D$, we will note by $f~\hat{\square}~g$ the induced morphism $a\square y\coprod_{a\square x} b\square x\to b\square y$.\sym{((g38@$\hat{\square}$}
\end{notation}
\begin{prop}[{\cite[proposition A.3.7.3]{Lurie_Htt}}]
\label{prop:left_boosfiled_localization}
Let $A$ be a nice model structure and $S$ a set of cofibrations. There exists a model structure $A_S$ on the same category, and a left Quillen adjoint $L:A\to A_S$, such that an object is fibrant in $A_S$ if and only if it is fibrant in $A$ and has the right lifting property against all morphisms of shape $i\htimes f$ where $i$ is a cofibration and $f$ in $S$. Moreover, a left Quillen functor $F:A\to C$ lifts to $A_{S}$ if and only if for any cofibration $i$ and morphism $f\in S$, $F(i\htimes f)$ is a weak equivalence.
\end{prop}

\begin{cor}
\label{cor:left_boosfiled_localization}
Let $A$, $C$ be two nice model categories, $F:A\to C$ a left Quillen functor, $S$ a set of cofibrations and $T$ a set of morphisms such that for any cofibrations $i$ and morphisms $f\in S$, the morphism $i\htimes f$ is included in the smallest saturated class stable by two out of three, containing weak equivalences and $T$. Then a left Quillen functor $F:A\to C$ lifts to $A$ if and only if it sends morphisms of $T$ to weak equivalences.
\end{cor}
\begin{proof}
Let $U$ be the class of morphisms in $A$ that are sent to weak equivalences by $F$. This class is obviously stable by two out of three, retracts and contains weak equivalences. As the model structure on $C$ is combinatorial and left proper, it is saturated. The class $U$ then includes all morphisms of shape $i\htimes f$ for $i$ a cofibration and $f\in S$, which implies that $F$ can be lifted to $A_S$.
\end{proof}

\p Let $i:A\to B$ and $i':A'\to B'$ be two cofibrations. A \notion{zigzag of acyclic cofibration} between $i$ and $i'$, denoted $i\leftrightsquigarrow i'$ is a zigzag in the category of arrows such that all the horizontal maps are acyclic cofibrations, and all the vertical maps are cofibrations. 


\begin{lemma}
Let $i$ and $j$ be two cofibrations, and $f:X\to Y$ a fibration between fibrant objects. Suppose that we have a morphism in the category of arrows $i\to j$ which is  pointwise an acyclic cofibration. Then, if $j$ has the left lifting property against $f$, so has $i$.
\end{lemma}
\begin{proof}
We consider a diagram of the following shape:
% https://q.uiver.app/?q=WzAsNixbMCwwLCJBIl0sWzEsMCwiQSciXSxbMCwxLCJCIl0sWzEsMSwiQiciXSxbMiwwLCJYIl0sWzIsMSwiWSJdLFswLDIsImkiXSxbMCwxLCJcXHNpbSJdLFsyLDMsIlxcc2ltIiwyXSxbMSwzLCJqIiwyXSxbNCw1XSxbMCw0LCIiLDIseyJjdXJ2ZSI6LTN9XSxbMiw1LCIiLDIseyJjdXJ2ZSI6M31dLFszLDUsImxfMCIsMix7InN0eWxlIjp7ImJvZHkiOnsibmFtZSI6ImRvdHRlZCJ9fX1dLFsxLDQsImxfMSIsMSx7InN0eWxlIjp7ImJvZHkiOnsibmFtZSI6ImRvdHRlZCJ9fX1dLFszLDQsImxfMiIsMSx7InN0eWxlIjp7ImJvZHkiOnsibmFtZSI6ImRvdHRlZCJ9fX1dXQ==
\[\begin{tikzcd}
	A & {A'} & X \\
	B & {B'} & Y.
	\arrow["i", from=1-1, to=2-1]
	\arrow["\sim", from=1-1, to=1-2]
	\arrow["\sim"', from=2-1, to=2-2]
	\arrow["j"', from=1-2, to=2-2]
	\arrow[from=1-3, to=2-3]
	\arrow[curve={height=-18pt}, from=1-1, to=1-3]
	\arrow[curve={height=18pt}, from=2-1, to=2-3]
	\arrow["{l_0}"', dotted, from=2-2, to=2-3]
	\arrow["{l_1}"{description}, dotted, from=1-2, to=1-3]
	\arrow["{l_2}"{description}, dotted, from=2-2, to=1-3]
\end{tikzcd}\]
We construct, one after the other, the lifting $l_0$, $l_1$ and $l_2$.
\end{proof}

\begin{lemma}
Let $i$ and $j$ be two cofibrations, and $f:X\to Y$ a fibration between fibrant objects. Suppose that we have a morphism in the category of arrows $i\to j$ which is  pointwise an acyclic cofibration. Then, if $i$ has the right lifting property against $f$, so has $j$.
\end{lemma}
\begin{proof}
We consider a diagram of the following shape:
% https://q.uiver.app/?q=WzAsNyxbMCwwLCJBIl0sWzAsMSwiQiJdLFsxLDAsIkEnIl0sWzIsMiwiQiciXSxbMSwxLCJCXFxjb3Byb2RfQSBBJyJdLFs0LDAsIlgiXSxbNCwyLCJZIl0sWzAsMV0sWzAsMiwiXFxzaW0iLDJdLFsxLDQsIlxcc2ltIl0sWzQsMywiXFxzaW0iLDFdLFsxLDMsIlxcc2ltIiwyLHsiY3VydmUiOjF9XSxbMiwzLCIiLDIseyJjdXJ2ZSI6LTF9XSxbMiw0XSxbMiw1XSxbMyw2XSxbNSw2XSxbNCw1LCJsXzAiLDEseyJzdHlsZSI6eyJib2R5Ijp7Im5hbWUiOiJkb3R0ZWQifX19XSxbMyw1LCJsXzEiLDEseyJzdHlsZSI6eyJib2R5Ijp7Im5hbWUiOiJkb3R0ZWQifX19XV0=
\[\begin{tikzcd}
	A & {A'} &&& X \\
	B & {B\coprod_A A'} \\
	&& {B'} && Y.
	\arrow[from=1-1, to=2-1]
	\arrow["\sim"', from=1-1, to=1-2]
	\arrow["\sim", from=2-1, to=2-2]
	\arrow["\sim"{description}, from=2-2, to=3-3]
	\arrow["\sim"', curve={height=6pt}, from=2-1, to=3-3]
	\arrow[curve={height=-6pt}, from=1-2, to=3-3]
	\arrow[from=1-2, to=2-2]
	\arrow[from=1-2, to=1-5]
	\arrow[from=3-3, to=3-5]
	\arrow[from=1-5, to=3-5]
	\arrow["{l_0}"{description}, dotted, from=2-2, to=1-5]
	\arrow["{l_1}"{description}, dotted, from=3-3, to=1-5]
\end{tikzcd}\]
We construct, one after the other, the lifting $l_0$, $l_1$.
\end{proof}

\begin{prop}
\label{prop:lifting_property_zigzag_of_acyclic_cofibration}
Let $f$ be a fibration between fibrant objects and $i$ and $j$ two cofibrations such that there exists a zigzag of acyclic cofibrations $i\leftrightsquigarrow j$. Then $f$ has the right lifting property against $i$ if and only if it has the right lifting property against $j$. 
\end{prop} 
\begin{proof}
This is a direct consequence of the last two lemmas.
\end{proof}







\subsection{Marked and stratified presheaves}
\label{section:Marked and stratified presheaves}
\p 
Let $B$ be an elegant Reedy category and $M$ a subset of the set of objects of $B$. A \wcnotion{$M$-stratified presheaf on $B$}{stratified presheaf on $B$}, or just a \textit{stratified prehsheaf on $B$} when the subset $M$ will be non-ambiguous, is a pair $(X,tX)$ where $X$ is a presheaf on $B$ and $tX:=\coprod_{a\in M} tX_a$ is the disjoint union of sets, such that for any $a\in M$, $tX_a$ is a subset of $X_a$ including degeneracies, i.e the image of morphisms $X_p:X_b\to X_a$ for $p:b\to a$ in $B_-$.


A \notion{stratified morphism} $f:(X,tX)\to (Y,tY)$ is the data of a morphism on the underlying presheaf such that $f(tX_n)\subset tY_n$.
The category of stratified presheaves is denoted by \wcnotation{$\tPshM{B}$}{(tpsh@$\tPshM{B}$}. 

A morphism between two stratified presheaves is \wcnotion{entire}{entire morphism} if it is the identity on the underlying presheaves.

We then have an adjunction
% https://q.uiver.app/#q=WzAsMixbMCwwLCIoXFx1dmFyKV5cXGZsYXQ6XFxQc2h7Qn0iXSxbMSwwLCJcXHRQc2hNe0J9OihcXHV2YXIpXlxcbmF0dXJhbCJdLFswLDEsIiIsMCx7Im9mZnNldCI6LTJ9XSxbMSwwLCIiLDAseyJvZmZzZXQiOi0yfV0sWzIsMywiIiwwLHsibGV2ZWwiOjEsInN0eWxlIjp7Im5hbWUiOiJhZGp1bmN0aW9uIn19XV0=
\[\begin{tikzcd}
	{(\uvar)^\flat:\Psh{B}} & {\tPshM{B}:(\uvar)^\natural}
	\arrow[""{name=0, anchor=center, inner sep=0}, shift left=2, from=1-1, to=1-2]
	\arrow[""{name=1, anchor=center, inner sep=0}, shift left=2, from=1-2, to=1-1]
	\arrow["\dashv"{anchor=center, rotate=-90}, draw=none, from=0, to=1]
\end{tikzcd}\]
where the left adjoint is a fully faithful inclusion that sends a presheaf $X$ onto $(X,S)$ where $S$ is the smaller stratification on $X$, and where the right adjoint is the obvious forgetful functor. We will identify presheaves on $B$ with their image by the functor $(\uvar)^\flat$.

\p If $b$ is an object of $M$, we denote by $b_t$ the stratifed presheaf $(b,S)$, where $S$ is the smaller stratification that includes $id:b\to b$.



We then define $t_MB$ as the full subcategory of $\tPshM{B}$ spanned by the objects of shape $a$ or $b_t$ with $a\in B$ and $b\in M$. We then have equalities:
$$\begin{array}{rcl}
\Hom_{t_MB}(a,b)&:=& \Hom_B(a,b),\\
\Hom_{t_MB}(a,b_t)&:=& \Hom_B(a,b),\\
\Hom_{t_MB}(a_t,b)&:=& \Hom_B(a,b)\cap B_- \diagdown \{id_a\}, \\
\Hom_{t_MB}(a_t,b_t)&:=&\Hom_B(a,b)\cap B_-.\\
\end{array}$$
The canonical functor $B\to t_MB$ is then fully faithful and we will identify object of $B$ with their image through this functor.
\begin{prop}
\label{prop:reedy structure on tB}
The category $t_MB$ admits a structure of elegant Reedy category, that makes the inclusion $B\to t_MB$ a morphism of Reedy category. There is no non trivial negative morphism whose codomain is of shape $b_t$ for $b\in M$. There is no non trivial positive morphism whose domain is of shape $b_t$ for $b\in M$. 
\end{prop}
\begin{proof}
We define the degree degree function $ob(t_MB)\to \Nb$ by the assignment 
$$d'(b):= 2 d(b)~~~~~d'(b_t):= 2d(b)+1$$
The category $(t_MB)_+$ is the smallest that includes $B_+$ and morphisms of shape $a\to a_t$. The category $(t_MB)_-$ is the smallest that includes $B_-$ and morphisms of shape $b_t\to a$.

To prove the axioms of Reedy category, we can replicate the strategy used in proposition C.2 of \cite{Ozornova_model_structure_for_infini_n_categories} with obvious modification to this more general framework.

We still have to show that $tB$ is elegant. Let $X$ be a presheaf on $t_MB$, $a$ an element of $t_MB$, $f:a\to a'$ and $g:a\to a'$ two negative morphisms, an element $x$ of $X(a)$, two non degenerate elements $y\in X(a')$ and $z\in X(a'')$ such that $f^*y=x$, $g^*z=x$. 

Suppose first that $a$ is in $B$. In this case, $f$ and $g$ are also in $B$, and as this Reedy category is elegant by assumption, this implies $f=g$ and $y=z$. Suppose now that $a$ is of shape $b_t$ for $b\in B$. We denote $\alpha$ the canonical morphism $\alpha:b\to b_t$. By definition of negative morphism, the codomain of $f$ and $g$ are in $B$. The morphisms $\alpha f $ and $\alpha g$ then are in $B$. Moreover, these two morphisms are negative, and we have $(\alpha f)^*y=\alpha^* x$, $(\alpha g)^*z=\alpha^* x$. As $B$ is elegant, $\alpha f=\alpha g$ and $y=z$. Eventually, remark that the first equality implies that $f$ is equal to $g$. 
\end{proof}
A cellular model for $t_MB$ is given by $C\cup\{b\to b_t,b\in M\}$ where $C$ is a cellular model for $B$.


\p The category of $M$-stratified presheaves is then equivalent to the fully faithful subcategory of presheaves $X$ on $t_MB$ such that for any $b\in M$, $X(b_t)\to X(b)$ is a monomorphism. 
In particular, we have an adjunction % https://q.uiver.app/#q=WzAsMixbMCwwLCJcXHBpOlxcUHNoe3RfTUJ9Il0sWzEsMCwiXFx0UHNoTXtCfTpcXGlvdGEiXSxbMCwxLCIiLDEseyJvZmZzZXQiOi0yfV0sWzEsMCwiIiwxLHsib2Zmc2V0IjotMn1dLFsyLDMsIiIsMSx7ImxldmVsIjoxLCJzdHlsZSI6eyJuYW1lIjoiYWRqdW5jdGlvbiJ9fV1d
\begin{equation}
\label{eq:entre presheaveds on tB et stratified presheages}
\begin{tikzcd}
	{\pi:\Psh{t_MB}} & {\tPshM{B}:\iota}
	\arrow[""{name=0, anchor=center, inner sep=0}, shift left=2, from=1-1, to=1-2]
	\arrow[""{name=1, anchor=center, inner sep=0}, shift left=2, from=1-2, to=1-1]
	\arrow["\dashv"{anchor=center, rotate=-90}, draw=none, from=0, to=1]
\end{tikzcd}
\end{equation}
Remark furthermore that the unit $X\to \iota \pi X$ is a trivial fibration. Indeed, the cellular model is given $C\cup\{b\to b_t,b\in M\}$, where  $C$ is a cellular model for $B$, and the unit obviously has the right lifting property against it.

\begin{prop}
\label{prop:transfert from presheaves on tB to stratified presheaves}
Suppose given a combinatorial on $\Psh{t_MB}$ whose cofibrations are monomorphisms. Then there exists a combinatorial model structure on $\tPshM{B}$ making the adjunction \ref{eq:entre presheaveds on tB et stratified presheages} a Quillen equivalence. 


A morphism of $\tPshM{B}$ is a cofibration if and only if it is a monomorphism. A morphism is a fibration (resp. a weak equivalence) if and only if its image by $\iota$ is.
\end{prop}
\begin{proof}
We are willing to apply \cite[theorem 11.3.2]{Hirschhorn_Model_categories_and_their_localizations}. As two adjoints of \eqref{eq:entre presheaveds on tB et stratified presheages} preserve smallness, the first condition is obviously fulfilled. Using the fact that $\iota$ is fully faithful, the second condition of theorem \textit{op cit} is equivalent to asking that for any acyclic cofibration $i$ of $\Psh{t_MB}$, the morphism $\iota\pi i$ is a weak equivalence. As the unit $id\to \iota \pi $ is pointwise a trivial fibration, this directly follows from the stability of weak equivalences by two out of three.

This provides the model structure. As the unit is pointwise a trivial fibration and the counit is the identity, the adjunction
\eqref{eq:entre presheaveds on tB et stratified presheages} induces a Quillen equivalence. 
\end{proof}


\p We now fix a Reedy category $B$, a subset $M$ of objects of $B$, and we suppose given a nice model structure on $\tPshM{B}$ (as defined in paragraph \ref{para:nice model structure}).
A \wcnotion{$M$-marked presheaf on $B$}{marked presheaf on $B$} is a stratified presheaf having the unique right lifting property against all entire acyclic cofibrations. In particular, any fibrant objects is marked. 

We denote by \wcnotation{$\mPshM{B}$}{(mpsh@$\mPshM{\uvar}$} the full subcategory of marked presheaves on $B$. We then have an adjunction: \sym{((b91@$(\uvar)_{\mk}$}
% https://q.uiver.app/#q=WzAsMixbMSwwLCJcXG1Qc2hNe0J9OlxcaW90YSJdLFswLDAsIihcXHV2YXIpX3tcXG1rfTpcXHRQc2hNe0J9Il0sWzAsMSwiIiwwLHsib2Zmc2V0IjotMn1dLFsxLDAsIiIsMCx7Im9mZnNldCI6LTJ9XSxbMywyLCIiLDAseyJsZXZlbCI6MSwic3R5bGUiOnsibmFtZSI6ImFkanVuY3Rpb24ifX1dXQ==
\begin{equation}
\label{eq:adj beetwen stratified and marked}
\begin{tikzcd}
	{(\uvar)_{\mk}:\tPshM{B}} & {\mPshM{B}:\iota}
	\arrow[""{name=0, anchor=center, inner sep=0}, shift left=2, from=1-2, to=1-1]
	\arrow[""{name=1, anchor=center, inner sep=0}, shift left=2, from=1-1, to=1-2]
	\arrow["\dashv"{anchor=center, rotate=-90}, draw=none, from=1, to=0]
\end{tikzcd}
\end{equation}
where the left adjoint $(\uvar)_{\mk}$ sends a stratified presheaf $(X,tX)$ to the marked presheaf $(X,\overline{tX})$, where $\overline{tX}$ is the smaller stratification that includes $tX$ and makes $(X,\overline{tX})$ a marked presheaf, and where the right adjoint is a fully faithful inclusion.
Remark furthermore that at the level of presheaves, these two adjoints are the identity. 

\begin{prop}
\label{prop:X to Xmk is acycli cof}
Let $X$ be a $M$-stratified presheaf on $B$.
The canonical morphism $X\to \iota (X_{\mk})$ is an entire acyclic cofibration.
\end{prop}
\begin{proof}
Let $\kappa$ be a regular cardinal such that $X$ is $\kappa$-small. Remark first the domain of a entire monomorphism is $\kappa$-small if and only if its codomain is.


Let $I$ be the set of entire acyclic cofibrations with $\kappa$-small codomains and domains. This set generates via the small object argument a weak factorization system, and we denote by $X\to X'\to 1$ the factorization of $X\to 1$. We are willing to show that $X'$ is $M$-marked. As $X\to X'$ is an entire acyclic cofibration by construction, this will directly imply that $X'$ is equal to $\iota (X_{\mk})$ and so demonstrate the desired result.

Suppose then given a diagram
% https://q.uiver.app/#q=WzAsNCxbMCwwLCJLIl0sWzEsMCwiWCciXSxbMSwxLCIxIl0sWzAsMSwiTCJdLFszLDJdLFswLDFdLFsxLDJdLFswLDMsImkiLDJdXQ==
\[\begin{tikzcd}
	K & {X'} \\
	L & 1
	\arrow[from=2-1, to=2-2]
	\arrow[from=1-1, to=1-2]
	\arrow[from=1-2, to=2-2]
	\arrow["i"', from=1-1, to=2-1]
\end{tikzcd}\]
with $i$ an entire acyclic cofibration. We have to show that it admits a lift.
Remark that this square factors as:
% https://q.uiver.app/#q=WzAsNixbMCwwLCJLIl0sWzEsMCwiWCciXSxbMiwxLCIxIl0sWzAsMSwiTCJdLFsyLDAsIlgnIl0sWzEsMSwiWCdcXGNvcHJvZF9LTCJdLFswLDFdLFswLDMsImkiLDJdLFs0LDJdLFsxLDQsIiIsMCx7ImxldmVsIjoyLCJzdHlsZSI6eyJoZWFkIjp7Im5hbWUiOiJub25lIn19fV0sWzMsNV0sWzUsMl0sWzEsNSwiaSciXSxbNSwwLCIiLDEseyJzdHlsZSI6eyJuYW1lIjoiY29ybmVyIn19XV0=
\[\begin{tikzcd}
	K & {X'} & {X'} \\
	L & {X'\coprod_KL} & 1
	\arrow[from=1-1, to=1-2]
	\arrow["i"', from=1-1, to=2-1]
	\arrow[from=1-3, to=2-3]
	\arrow[Rightarrow, no head, from=1-2, to=1-3]
	\arrow[from=2-1, to=2-2]
	\arrow[from=2-2, to=2-3]
	\arrow["{i'}", from=1-2, to=2-2]
	\arrow["\lrcorner"{anchor=center, pos=0.125, rotate=180}, draw=none, from=2-2, to=1-1]
\end{tikzcd}\]
The morphism $i'$ is an entire acyclic cofibration with $\kappa$-small codomain and domain and then belongs to $i$. The right square of the previous diagram then admits a lift. This induces a lift in the in the original square, and this concludes the proof.
\end{proof}
\begin{prop} 
\label{prop:model structure on marked presheaves}
Suppose given a nice model structure on $\tPshM{B}$.
This induces a nice model structure on $\mPshM{B}$, making the adjunction \eqref{eq:adj beetwen stratified and marked} a Quillen equivalence. A morphism between two marked presheaves is a cofibration (resp. a fibration) (resp. a weak equivalence) if it is a cofibration (resp. a fibration) (resp. a weak equivalence) when seen as a morphism of $\tPshM{B}$. 
\end{prop} 
\begin{proof} Let $f:X\to Y$ be a fibration between stratified presheaves. If $Y$ is marked, so is $X$. The two weak factorization systems on $\mPshM{B}$ are then induced by the one of $\tPshM{B}$. We leave it to the reader to check that this model structure is nice. 

The unit is pointwise a weak equivalence according to proposition \ref{prop:X to Xmk is acycli cof} and the counit is the identity. 
The adjunction \eqref{eq:adj beetwen stratified and marked} is then a Quillen equivalence.
\end{proof}



\section{The complicial model}

\subsection{Model structure on marked simplicial sets}
This section is a recollection of  the principal results of \cite{Ozornova_model_structure_for_infini_n_categories}. We refer to \cite{Rhiel_Complicial_sets_an_ouverture} for an introduction to complicial sets.

\p
A \notion{stratified simplicial set} is a pair $(X,tX)$ where $X$ is a simplicial set and $tX := \cup_{n>0}tX_n$
 a graded set such that for any $n\geq 1$, $tX_n$ is a subset of $X_n$ that includes all degenerate simplices. A simplex in $tX$ is called \wcnotion{thin}{thin simplex}. 
 
 
A \textit{stratified morphism} $f:(X,tX)\to (Y,tY)$ is the data of a morphism on the underlying simplicial set such that $f(tX_n)\subset tY_n$.
The category of stratified simplicial sets is denoted by \wcnotation{$\stratSset$}{(tpshdelta@$\stratSset$}. 


Given a functor $i:I\mapsto (F(i),tF(i))$ with value in stratified simplicial sets, its colimit is given by $(\colim F(i),M)$ where $M$ is the smaller stratification that includes the image of $tF(i)\to \colim F(i)$ for any $i:I$.

We can extend the join to stratified simplicial sets as follows: 
If $(X,tX)$ and $(Y,tY)$ are two stratified simplicial sets, we define $tX\star tY$ as the set of simplices of $X\star Y$ of shape $x\star y$ where either $x$ or $y$ are thin. We then define 
$$(X,tX)\star (Y,tY) := (X\star Y, tX\star tY).$$

\begin{definition}
A stratified monomorphism $f:X\to Y$ is 
\begin{enumerate}
\item \textit{entire} if it is an identity on underlying simplicial sets.
\item \wcnotion{regular}{regular morphism} if for every $n\geq 1$ the following diagram is a pullback:
% https://q.uiver.app/?q=WzAsNCxbMCwwLCJ0WF9uIl0sWzAsMSwidFlfbiJdLFsxLDAsIlhfbiJdLFsxLDEsIllfbiJdLFsxLDNdLFsyLDNdLFswLDJdLFswLDFdLFswLDMsIiIsMSx7InN0eWxlIjp7Im5hbWUiOiJjb3JuZXIifX1dXQ==
\[\begin{tikzcd}
	{tX_n} & {X_n} \\
	{tY_n} & {Y_n}.
	\arrow[from=2-1, to=2-2]
	\arrow[from=1-2, to=2-2]
	\arrow[from=1-1, to=1-2]
	\arrow[from=1-1, to=2-1]
	\arrow["\lrcorner"{anchor=center, pos=0.125}, draw=none, from=1-1, to=2-2]
\end{tikzcd}\]
\end{enumerate}
\end{definition}

\begin{definition}
We define several stratified structures on $[n]$. 
\begin{enumerate}
\item \wcnotation{$[n]_t$}{((g31@$[n]_t$}. The top $n$-simplex is thin. All degeneracies are thin.
\item \wcnotation{$[n]^k$}{((g32@$[n]^k$}. All simplices that include $\{k-1,k,k+1\}\cap[n]$ are thin. All degeneracies are thin.
\item \wcnotation{$([n]^k)'$}{((g33@$([n]^k)'$}. All simplices that include $\{k-1,k,k+1\}\cap[n]$, together with the $(k-1)$-face and the $(k+1)$ face are thin. All degeneracies are thin.
\item \wcnotation{$([n]^k)''$}{((g34@$([n]^k)''$}. All simplices that include $\{k-1,k,k+1\}\cap[n]$, together with the $(k-1)$-face, the $k$-face and the $(k+1)$ face are thin. All degeneracies are thin.
\item \wcnotation{$[3]^{eq}$}{((g35@$[3]^{eq}$}. All simplices of dimension strictly higher than $2$, together with $[0,2]$ and $[1,3]$ are thin. All degeneracies are thin.
\item \wcnotation{$[n]^\sharp$}{((g36@$[n]^\sharp$}. All simplices are thin.
\end{enumerate}
\end{definition}
\begin{definition}	
An \snotion{elementary anodyne extension}{for stratified simplicial sets} is one of the following:
\begin{enumerate}
\item The \notion{complicial horn inclusions} are the regular extensions:
$$\Lambda^k[n]\to [n]^k,~n\geq 1,~ n\geq k\geq 0.$$
\item The \notion{complicial thinness extensions}:
$$([n]^k)'\to ([n]^k)'',~n\geq 2,~ n\geq k\geq 0.$$
\item The \notion{saturation extensions}:
$$[n]\star[3]^{eq}\star[m]\to [n]\star[3]^{\sharp}\star[m],~ n,m\geq -1.$$
\end{enumerate}
The set of complicial horn inclusions is $\Lambda$ and the reunion of \textit{complicial thinness extensions} and of \textit{saturation extensions} is $S$.
\end{definition}


\begin{definition}
\label{defi:complicial set}
Let $n\in \Nb\cup\{\omega\}$.	
A \wcnotion{$n$-complicial set}{complicial set} is a stratified set having the right lifting property against all elementary anodyne extensions and against all morphisms $[k]\to [k]_t$ for $k>n$. 
\end{definition}

\begin{theorem}[Ozornova, Rovelli, Verity]
\label{theo:model structure on complicial set}
Let $n\in \Nb\cup\{\omega\}$.	
There exists a nice model structure on stratified simplicial sets, denoted by $\stratSset^n$, whose fibrant objects are $n$-complicial sets. 

A left adjoint $F:\stratSset\to D$ to a model category is a left Quillen functor if it preserves cofibrations and sends all elementary anodyne extensions and morphisms $[k]\to [k]_t$, for $k>n$, to weak equivalences. \sym{(tpshdeltan@$\stratSset^n$}
\end{theorem}
\begin{proof}
This is \cite[theorem 1.25]{Ozornova_model_structure_for_infini_n_categories}.
\end{proof}
During this chapter, we will only be interested in the model structure for $\omega$-complicial sets, and we will therefore drop the index $\omega$. The $\omega$-complicial sets will then just be called \textit{complicial sets} and we will denote by $\stratSset$ the model category $\stratSset^{\omega}$.

\p A \notion{marked simplicial set} is a stratified simplicial set that has the right lifting property against entire acyclic cofibrations. In particular, all complicial sets are marked. The category of marked simplicial sets is denoted by \wcnotation{$\mSset$}{(mpsh@$\mSset$}. There is an adjunction:
% https://q.uiver.app/#q=WzAsMixbMSwwLCJcXG1Tc2V0OlxcaW90YSJdLFswLDAsIihcXHV2YXIpX3tcXG1rfTpcXHN0cmF0U3NldCJdLFsxLDAsIiIsMCx7Im9mZnNldCI6LTJ9XSxbMCwxLCJpIiwwLHsib2Zmc2V0IjotMn1dLFsyLDMsIiIsMCx7ImxldmVsIjoxLCJzdHlsZSI6eyJuYW1lIjoiYWRqdW5jdGlvbiJ9fV1d
\begin{equation}
\label{adj:beetwen marked an stratified}
\begin{tikzcd}
	{(\uvar)_{\mk}:\stratSset} & {\mSset:\iota}
	\arrow[""{name=0, anchor=center, inner sep=0}, shift left=2, from=1-1, to=1-2]
	\arrow[""{name=1, anchor=center, inner sep=0}, "i", shift left=2, from=1-2, to=1-1]
	\arrow["\dashv"{anchor=center, rotate=-90}, draw=none, from=0, to=1]
\end{tikzcd}
\end{equation}
The left adjoint $(\uvar)_{\mk}$ sends a stratified simplicial set $(X,tX)$ to the marked simplicial set $(X,\overline{tX})$, where $\overline{tX}$ is the smaller stratification that includes $tX$ and makes $(X,\overline{tX})$ a marked simplicial set. Moreover, the proposition \ref{prop:X to Xmk is acycli cof} implies that the canonical morphism $X\to \iota (X)_{\mk}$ is an entire acyclic cofibration.



Given a functor $i:I\mapsto (F(i),tF(i))$ with value in marked simplicial sets, its colimit is given by $(\colim F(i),\overline{M})$ where $M$ is the smaller stratification that includes the image of $tF(i)\to \colim F(i)$ for any $i:I$.

\begin{prop}
\label{prop:model structure on marked simplicial set}
The category $\mSset$ admits a nice model structure that makes the adjunction \ref{adj:beetwen marked an stratified} a Quillen equivalence.
\end{prop}
\begin{proof}
This is a direct consequence of proposition \ref{prop:model structure on marked presheaves} and theorem \ref{theo:model structure on complicial set}.
\end{proof}

\p
\label{para:inteligentr trucation for simplicial set}
Let $n$ be an integer, and $(X,tX)$ a marked simplicial set. We define $\tau^i_n(tX)$ as the reunion of $tX$ and all simplices of dimension strictly superior to $n$. This induces a functor, called the \snotionsym{intelligent $n$-truncation}{(taui@$\tau^i_n$}{for marked simplicial sets}:
$$\begin{array}{rcll}
\tau^i_n :& \mSset&\mapsto &\mSset\\
 &(X,tX)&\mapsto &(X, \overline{\tau^i_n(tX)}).
\end{array}$$
This functor preserves cofibrations.	
Given the explicit description of colimits in marked simplicial sets, it is easy to see that $\tau^i_n$ preserves colimits. 
For every elementary anodyne extension $i:K\to L$, we have a pushout 
% https://q.uiver.app/?q=WzAsNCxbMCwwLCJLIl0sWzEsMCwiTCJdLFswLDEsIlxcdGF1XmlfbihLKSJdLFsxLDEsIlxcdGF1XmlfbihMKS4iXSxbMCwyXSxbMiwzXSxbMSwzXSxbMCwxXSxbMywwLCIiLDAseyJzdHlsZSI6eyJuYW1lIjoiY29ybmVyIn19XV0=
\[\begin{tikzcd}
	K & L \\
	{\tau^i_n(K)} & {\tau^i_n(L).}
	\arrow[from=1-1, to=2-1]
	\arrow[from=2-1, to=2-2]
	\arrow[from=1-2, to=2-2]
	\arrow[from=1-1, to=1-2]
	\arrow["\lrcorner"{anchor=center, pos=0.125, rotate=180}, draw=none, from=2-2, to=1-1]
\end{tikzcd}\]
The intelligent $n$-truncation is then a left Quillen functor.


It's associated right adjoint is called the \wcsnotionsym{$n$-truncation}{(tau@$\tau_n$}{truncation@$n$-truncation}{for marked simplicial sets} and is denoted by 
$$\tau_n:\mSset\to \mSset.$$

\subsection{Gray tensor product}

\begin{construction}[{\cite[Notation 5]{Verity_weak_complicial_sets_I}}] For any $n,p,q\geq 0$ such that $n=p+q$, we define:
\begin{itemize}
\item the \notion{degeneration partition operator}:
$$
\begin{array}{rclllrrclll}
\invamalg^1_{p,q}:&[n]&\to&[p]&&~~~~~~&\invamalg^2_{p,q}:&[n]&\to&[q]&\\
&k&\mapsto &k &\mbox{if $k\leq p$} &&&k&\mapsto &0& \mbox{if $k\leq p$}\\
&k&\mapsto &p 	&\mbox{if $k>p$} &&&k&\mapsto &k-p& \mbox{if $k> p$}.
\end{array}
$$
\item the \notion{face partition operator}:
$$
\begin{array}{rcllrrcll}
\amalg^1_{p,q}:&[p]&\to&[n]&~~~~~~&\amalg^2_{p,q}:&[q]&\to&[n]\\
&k&\mapsto &k &&&k&\mapsto &k+p.
\end{array}
$$
\end{itemize}
\end{construction}


\begin{definition}[{\cite[Definition 128]{Verity_weak_complicial_sets_I}}]
Let $(X,tX)$ and $(Y,tY)$ be two stratified simplicial sets. 
We define the \snotionsym{Gray tensor product}{((d00@$\otimes$}{for stratified simplicial sets} of $(X,tX)$ and $(Y,tY)$ as the stratified simplicial set 
$$(X,tX)\otimes (Y,tY):=(X\times Y,tX\otimes tY)$$ where $tX\otimes tY$ is the set of pairs $(x,y)$ such that for any partitions $(p,q)$ of $n$ either $\amalg^1_{p,q}x$ or $\amalg^2_{p,q}y$ is thin. 
\end{definition}

\begin{remark}
Let $X,Y$ be two stratified simplicial sets such that all simplices of $X$ are thin. The morphism 
$X\otimes Y\to X\times Y$ is then an isomorphism.
\end{remark}




\p In \cite{Verity_weak_complicial_sets_I}, it is shown that the Gray tensor is associative. The problem of this operation comes from the fact that it doesn't commute with colimits. Verity then defines an other binary operation, which is cocontinuous, the \textit{Gray pretensor} (\cite[definition 135]{Verity_weak_complicial_sets_I}) $(X,tX)\boxtimes(Y,tY):=(X\times Y, tX\boxtimes tY)$, together with a natural transformation: 
$$\uvar\boxtimes\uvar\to \uvar\otimes\uvar$$
that is pointwise an entire acyclic cofibration (\cite[lemma 149]{Verity_complicial_set_characterising_the_simplicial_nerve}). Moreover, in \cite{Ozornova_Gray_tensor_product_and_saturated_n_complicia}, it is shown that this pretensor is a Quillen bifunctor for the model structure on $\stratSset$. 

\begin{definition}[Gray tensor product for marked simplicial sets]
Let $X$ and $Y$ be two marked simplicial sets. We define the \snotionsym{Gray tensor product}{((d00@$\otimes$}{for marked simplicial sets} of $X$ and $Y$ as the marked simplicial set 
$$X\otimes Y:= (\iota(X)\otimes \iota(Y))_{\mk}$$
 where $((\uvar)_{\mk},\iota)$ is the adjunction \ref{adj:beetwen marked an stratified}.
As $\uvar\boxtimes\uvar\to \uvar\otimes\uvar$ is pointwise a entire acyclic cofibration, we have an equality: 
$$X\otimes Y:= (\iota(X)\boxtimes \iota(Y))_{\mk}.$$
\end{definition}

\begin{prop}
\label{prop:R_commutes_with_gray_tensor}
We have equalities
$$(\uvar\boxtimes \uvar)_{\mk}=(\uvar\otimes \uvar)_{\mk}= (\uvar)_{\mk}\otimes (\uvar)_{\mk}.$$
\end{prop}
\begin{proof}
The first equality is a consequence of the fact that $\uvar\boxtimes\uvar\to \uvar\otimes\uvar$ is pointwise a entire acyclic cofibration.

 For the second one, we have to show that $(X\otimes Y)_{\mk}=(\iota(X_{\mk})\otimes \iota(Y_{\mk}))_{\mk}$.
The unit of the adjunction $(\iota,(\uvar)_{\mk})$ induces a morphism $h:(X\otimes Y)_{\mk}\to (\iota(X_{\mk})\otimes \iota(Y_{\mk}))_{\mk}$. This morphism is an entire acyclic cofibration according to proposition \ref{prop:X to Xmk is acycli cof}, 
and the corollary 2.2 of \cite{Ozornova_Gray_tensor_product_and_saturated_n_complicia} and the fact that $(\uvar)_{\mk}$ is a left Quillen functor.

 We then have lifts in the following diagram:
% https://q.uiver.app/#q=WzAsMyxbMCwxLCIoXFxpb3RhKFhfe1xcbWt9KVxcb3RpbWVzIFxcaW90YShZX3tcXG1rfSkpX3tcXG1rfSJdLFsxLDAsIihYXFxvdGltZXMgWSlfe1xcbWt9Il0sWzAsMCwiKFhcXG90aW1lcyBZKV97XFxta30iXSxbMiwxLCJpZCJdLFsyLDAsImgiLDJdLFswLDEsImsiLDJdXQ==
\[\begin{tikzcd}
	{(X\otimes Y)_{\mk}} & {(X\otimes Y)_{\mk}} \\
	{(\iota(X_{\mk})\otimes \iota(Y_{\mk}))_{\mk}}
	\arrow["id", from=1-1, to=1-2]
	\arrow["h"', from=1-1, to=2-1]
	\arrow["k"', from=2-1, to=1-2]
\end{tikzcd}\]
As both $k$ and $h$ are the identity on the underlying simplicial sets, this implies that the stratifications of $(X\otimes Y)_{\mk}$ and $(X\otimes Y)_{\mk}$ coincide, and this two objects are then equal. 
\end{proof}

We can then deduce the following proposition:
\begin{prop}
\label{prop:gray_product_is_a_left_Quillen_bifunctor}
The Gray tensor product is associative, and is a left Quillen bifunctor in $\mSset$.
\end{prop}
\begin{proof}
The first assertion is a consequence of proposition \ref{prop:R_commutes_with_gray_tensor} and the fact that the binary operation $\otimes$ on $\stratSset$ is associative. The second one is a consequence of proposition \ref{prop:R_commutes_with_gray_tensor} and \cite[Theorem 2.1]{Ozornova_Gray_tensor_product_and_saturated_n_complicia}.
\end{proof}

We now give a lemma investigating the interaction between the truncation, the intelligent truncation and the Gray tensor product.
\begin{lemma}
\label{lemma:technique marked oicategoros}
Let $C$ and $D$ be two stratified simplicial sets.
\begin{enumerate}
\item
The following canonical square is cocartesian
% https://q.uiver.app/?q=WzAsNCxbMSwwLCJDXFxvdGltZXMgRCJdLFswLDAsIlxcY29wcm9kX3tufSBcXHRhdV9uQ1xcb3RpbWVzIFxcdGF1X25EIl0sWzAsMSwiXFxjb3Byb2Rfe259IFxcdGF1XmlfbihcXHRhdV9uQ1xcb3RpbWVzIFxcdGF1X25EKSJdLFsxLDEsIkNcXHRpbWVzIEQiXSxbMSwwXSxbMSwyXSxbMiwzXSxbMCwzXSxbMywxLCIiLDEseyJzdHlsZSI6eyJuYW1lIjoiY29ybmVyIn19XV0=
\[\begin{tikzcd}
	{\coprod_{n} \tau_nC\otimes \tau_nD} & {C\otimes D} \\
	{\coprod_{n} \tau^i_n(\tau_nC\otimes \tau_nD)} & {C\times D}
	\arrow[from=1-1, to=1-2]
	\arrow[from=1-1, to=2-1]
	\arrow[from=2-1, to=2-2]
	\arrow[from=1-2, to=2-2]
	\arrow["\lrcorner"{anchor=center, pos=0.125, rotate=180}, draw=none, from=2-2, to=1-1]
\end{tikzcd}\]

\item 
If $D$ is invariant under $\tau^i_2$, the following canonical square is cocartesian
% https://q.uiver.app/?q=WzAsNCxbMSwwLCJDXFxvdGltZXMgRCJdLFswLDAsIlxcY29wcm9kX3sgbn0gXFx0YXVfe259Q1xcb3RpbWVzIEQiXSxbMCwxLCJcXGNvcHJvZF97bn0gXFx0YXVeaV97bisxfShcXHRhdV97bn1DXFxvdGltZXMgRCkiXSxbMSwxLCJDXFxvdGltZXMgXFx0YXVeaV8xRCJdLFsxLDJdLFsyLDNdLFswLDNdLFszLDEsIiIsMSx7InN0eWxlIjp7Im5hbWUiOiJjb3JuZXIifX1dLFsxLDBdXQ==
\[\begin{tikzcd}
	{\coprod_{ n} \tau_{n}C\otimes D} & {C\otimes D} \\
	{\coprod_{n} \tau^i_{n+1}(\tau_{n}C\otimes D)} & {C\otimes \tau^i_1D}
	\arrow[from=1-1, to=2-1]
	\arrow[from=2-1, to=2-2]
	\arrow[from=1-2, to=2-2]
	\arrow["\lrcorner"{anchor=center, pos=0.125, rotate=180}, draw=none, from=2-2, to=1-1]
	\arrow[from=1-1, to=1-2]
\end{tikzcd}\]
\end{enumerate}
\end{lemma}
\begin{proof}
Let $C^\natural$ and $D^\natural$ be the underlying simplicial sets of $C$ and $D$.
Remark first that the two vertical morphisms of the first square are the identity. 
The induced morphism 
\begin{equation}
\label{eq:felkjzfezoifjezoi}
\coprod_{n} \tau^i_n(\tau_nC\otimes \tau_nD)\coprod_{\coprod_{n} \tau_nC\otimes \tau_nD}C\otimes D\to C\times D
\end{equation}
is then the identity of $C^\natural\times D^\natural$ at the level of underlying simplicial sets. To conclude, one has to show that every simplex $C^\natural\times D^\natural$ that is marked in the right term of \eqref{eq:felkjzfezoifjezoi} is also marked in the left term.
For this, let $n$ be a non negative integer, $x\in C^\natural_k$ and $y\in D^\natural_k$, such that $x$ is marked in $C$ and $y$ is marked in $D$.
The $k$-simplex $(x,y)$ then is in the image of $\tau^i_{k-1}(\tau_{k-1}C\otimes \tau_{k-1}D)$ and is then marked in the left term of \eqref{eq:felkjzfezoifjezoi}. This concludes the proof of the first assertion. 

The two vertical morphisms of the second square also are the identity and the induced morphism 
\begin{equation}
\label{eq:felkjzfezoifjezoibfdbd}
\coprod_{n} \tau^i_{n+ 1}(\tau_{n}C\otimes D)\coprod_{\coprod_{ n} \tau_{n}C\otimes D}C\otimes D\to C\otimes \tau^i_1	D
\end{equation}
is then once again the identity of $C^\natural\times D^\natural$ at the level of underlying simplicial sets.
Unfolding the definition, the marking of the left term is the smaller one that includes the one of $C\otimes D$ and every 
$k$-simplex $(x,y)$ such that both $x$ and $d^kx$ are marked in $C$.

Let $(x,y)$ be a $k$-simplex of $C^\natural\times D^\natural$. Suppose first that it is marked in $C\otimes D$. Remark that $(x,y)$ is then marked in $\tau_{k}C\otimes D$, and so is in the left term of \eqref{eq:felkjzfezoifjezoibfdbd}. Suppose now that both $x$ and $d^kx$ are marked in $C$. This implies that $s^{k-1}d^kx$ is in the image of $\tau_{k-1} C$. The simplex $(s^{k-1}d^kx,y)$ is then in the image of $\tau^i_{k}(\tau_{k-1}C\otimes D)$ and is then marked in the left term of \eqref{eq:felkjzfezoifjezoibfdbd}.


Now remark that we have 
$$ d^{k-1}(s^{k-1}x,s^ky)=(x,s^{k-1}d^{k-1}y) ~~~~~~~~~~ d^{k}(s^{k-1}x,s^ky)= (x,y)$$
$$d^{k+1}(s^{k-1}x,s^ky)= (s^{k-1}d^kx,y)$$
and both the $(k-1)$ and $(k+1)$ faces of $(s^{k-1}x,s^ky)$ are marked.
We leave it to the reader to check that by definition every sub $l$-simplex $z$ of $(s^{k-1}x,s^ky)$ containing the points $k-1$, $k$ and $k+1$ is marked in $C\otimes D$, and so in $\tau_{k}C\otimes D$, and, therefore, in the left term of \eqref{eq:felkjzfezoifjezoibfdbd}. As the marking is stable by complicial thinness extension, this implies that $(x,y)$ is also marked in the left term of \eqref{eq:felkjzfezoifjezoibfdbd}.

The marking of the right term of \eqref{eq:felkjzfezoifjezoibfdbd} is then included in the marking of the left term. They then coincide, which concludes the proof.
\end{proof}

\begin{remark}

The reason for including the assumption that $D$ is invariant under $\tau^i_2$ is solely because it will be the only relevant case. If we remove this assumption, the statement remains true, but the proof becomes a little bit more technical.
\end{remark}




\p
Let $X$ be a marked simplicial set. We define the \snotion{suspension}{for marked simplicial sets} of $X$, noted by \wcnotation{$\Sigma X$}{(sigma@$\Sigma\uvar$}, as the following pushout:
% https://q.uiver.app/?q=WzAsNCxbMSwwLCJYXFxvdGltZXMgWzFdIl0sWzAsMCwiWFxcb3RpbWVzXFxwYXJ0aWFsIFsxXSJdLFswLDEsIlxccGFydGlhbFsxXSJdLFsxLDEsIlxcU2lnbWEgWCJdLFswLDNdLFszLDEsIiIsMCx7InN0eWxlIjp7Im5hbWUiOiJjb3JuZXIifX1dLFsxLDJdLFsyLDNdLFsxLDBdXQ==
\[\begin{tikzcd}
	{X\otimes\partial [1]} & {X\otimes [1]} \\
	{\partial[1]} & {\Sigma X}
	\arrow[from=1-2, to=2-2]
	\arrow["\lrcorner"{anchor=center, pos=0.125, rotate=180}, draw=none, from=2-2, to=1-1]
	\arrow[from=1-1, to=2-1]
	\arrow[from=2-1, to=2-2]
	\arrow[from=1-1, to=1-2]
\end{tikzcd}\]
This assignation defines a cocontinuous functor $\Sigma:\mSset\to \mSset_{\partial[1]/}.$ For every acyclic cofibration $K\to L$, we have cartesian squares
% https://q.uiver.app/?q=WzAsNixbMSwwLCJLXFxvdGltZXNbMV1cXGN1cCBMXFxvdGltZXNcXHBhcnRpYWxbMV0iXSxbMiwwLCJMXFxvdGltZXNbMV0iXSxbMCwwLCJMXFxvdGltZXNcXHBhcnRpYWxbMV0iXSxbMCwxLCJcXHBhcnRpYWxbMV0iXSxbMSwxLCJcXFNpZ21hIEsiXSxbMiwxLCJcXFNpZ21hIEwiXSxbMiwzXSxbMiwwXSxbMyw0XSxbMCw0XSxbMSw1XSxbNCw1XSxbMCwxXSxbNCw3LCIiLDIseyJsZXZlbCI6MSwic3R5bGUiOnsibmFtZSI6ImNvcm5lciJ9fV0sWzUsMTIsIiIsMix7ImxldmVsIjoxLCJzdHlsZSI6eyJuYW1lIjoiY29ybmVyIn19XV0=
\[\begin{tikzcd}
	{L\otimes\partial[1]} & {K\otimes[1]\cup L\otimes\partial[1]} & {L\otimes[1]} \\
	{\partial[1]} & {\Sigma K} & {\Sigma L}
	\arrow[from=1-1, to=2-1]
	\arrow[""{name=0, anchor=center, inner sep=0}, from=1-1, to=1-2]
	\arrow[from=2-1, to=2-2]
	\arrow[from=1-2, to=2-2]
	\arrow[from=1-3, to=2-3]
	\arrow[from=2-2, to=2-3]
	\arrow[""{name=1, anchor=center, inner sep=0}, from=1-2, to=1-3]
	\arrow["\lrcorner"{anchor=center, pos=0.125, rotate=180}, draw=none, from=2-2, to=0]
	\arrow["\lrcorner"{anchor=center, pos=0.125, rotate=180}, draw=none, from=2-3, to=1]
\end{tikzcd}\]
The suspension then preserves acyclic cofibration and is then a left Quillen functor.

This functor admits a right adjoint, that sends a pair $(a,b,C)$ to \wcnotation{$C(a,b)$}{(cab@$C(a,b)$} where $a,b$ are two $0$-simplices of $C$. If $p:C\to D$ is a morphism between complicial sets, and $a,b$ two $0$-simplices of $C$, we denote by 
$$p(a,b):C(a,b)\to D(pa,pb)$$
the induced morphism.


\p
We introduce an other operation, the \notion{diamond product}, that makes the link between the Gray tensor product and the join. 
Let $X$ and $Y$ be two marked simplicial sets. We define \sym{((d21@$\diamond$}$X\diamond Y$ as the colimit of the diagram:
% https://q.uiver.app/?q=WzAsNSxbMiwwLCJYXFxvdGltZXNbMV1cXG90aW1lcyBZIl0sWzMsMCwiWFxcb3RpbWVzIFxcezFcXH1cXG90aW1lcyBZIl0sWzAsMCwiWCJdLFs0LDAsIlkiXSxbMSwwLCJYXFxvdGltZXMgXFx7MFxcfVxcb3RpbWVzIFkiXSxbMSwwXSxbMSwzXSxbNCwyXSxbNCwwXV0=
\[\begin{tikzcd}
	X & {X\otimes \{0\}\otimes Y} & {X\otimes[1]\otimes Y} & {X\otimes \{1\}\otimes Y} & Y
	\arrow[from=1-4, to=1-3]
	\arrow[from=1-4, to=1-5]
	\arrow[from=1-2, to=1-1]
	\arrow[from=1-2, to=1-3]
\end{tikzcd}\]
The functors 
$$\uvar\diamond X:\mSset\to \mSset_{/X} ~~~~\mbox{and}~~~~ X\diamond \uvar:\mSset\to \mSset_{/X}$$
are colimit preserving. Furthermore, for every acyclic cofibration $K\to L$, the morphism $K\diamond X\to L\diamond X$ is the horizontal colimit of the diagram: 
% https://q.uiver.app/?q=WzAsNixbMiwwLCJLXFxvdGltZXMgWzFdXFxvdGltZXMgWCJdLFsyLDEsIkxcXG90aW1lcyBbMV0gXFxvdGltZXMgWCJdLFsxLDAsIktcXG90aW1lcyBcXHBhcnRpYWxbMV1cXG90aW1lcyBYIl0sWzEsMSwiTFxcb3RpbWVzIFxccGFydGlhbFsxXVxcb3RpbWVzIFgiXSxbMCwwLCJLXFxhbWFsZyBYIl0sWzAsMSwiTFxcYW1hbGcgWCJdLFsyLDRdLFsyLDBdLFszLDFdLFszLDVdLFsyLDNdLFs0LDVdLFswLDFdXQ==
\[\begin{tikzcd}
	{K\amalg X} & {K\otimes \partial[1]\otimes X} & {K\otimes [1]\otimes X} \\
	{L\amalg X} & {L\otimes \partial[1]\otimes X} & {L\otimes [1] \otimes X}
	\arrow[from=1-2, to=1-1]
	\arrow[from=1-2, to=1-3]
	\arrow[from=2-2, to=2-3]
	\arrow[from=2-2, to=2-1]
	\arrow[from=1-2, to=2-2]
	\arrow[from=1-1, to=2-1]
	\arrow[from=1-3, to=2-3]
\end{tikzcd}\]
However, these two horizontal colimits are homotopy colimits, and all the horizontal maps of the previous diagram are weak equivalences. This morphism is then an acyclic cofibration. This shows that 
 $\uvar\diamond X$ is a left Quillen functor. We show analogously that $X\diamond \uvar$ is a left Quillen functor.


\begin{lemma}
There exists a unique natural transformation $\gamma_{X,Y}:X\diamond Y\to X\star Y$ that fits in the following diagram: 
% https://q.uiver.app/?q=WzAsNCxbMCwxLCJYXFxkaWFtb25kIFkiXSxbMCwwLCJYXFxjb3Byb2QgWSJdLFsxLDAsIlhcXHN0YXIgWSJdLFsxLDEsIlsxXSJdLFsxLDBdLFsxLDJdLFsyLDNdLFswLDNdLFswLDIsIlxcZ2FtbWFfe1gsWX0iXV0=
\[\begin{tikzcd}
	{X\coprod Y} & {X\star Y} \\
	{X\diamond Y} & {[1]}
	\arrow[from=1-1, to=2-1]
	\arrow[from=1-1, to=1-2]
	\arrow[from=1-2, to=2-2]
	\arrow[from=2-1, to=2-2]
	\arrow["{\gamma_{X,Y}}", from=2-1, to=1-2]
\end{tikzcd}\]
\end{lemma}
\begin{proof}
We begin by defining this morphism on simplicial sets, and for this we can suppose that both $X$ and $Y$ are representables, ie $X:=[n]$, $Y:=[m]$.
On object, this morphism is induced by the assignation:
$$p(k,0,l) := k~~~p(k,1,l) := l.$$ 

We need to verify that this morphism preserves thin cells. Suppose now that $(x,v,y)$ is a thin $n$-simplex of $X\diamond Y$. There are several cases to consider. \textbf{Case $v_n=0$.} The simplex $x$ is then thin, and is sent to $x\star \emptyset$ which is also thin. \textbf{Case $v_0=1$.} Similar. \textbf{Case $v_0=0$ and $v_n=1$.} Let $p$ be the smaller integer such that $v_p=1$. Either $\amalg_{p-1,n-p+1}^1(x)$ or $\amalg_{p,n-p}^2(y)$ is thin. This implies that $\phi_{X,Y}(x,v,y)= \amalg_{p-1,n-p+1}^1(x)\star \amalg_{p,n-p}^2(y)$ is thin. 
\end{proof}

\begin{prop}
\label{prop:equivalence between diamond and join product}
For any $X,Y$, the morphism $\gamma_{X,Y}$ is a weak equivalence. 
\end{prop}
\begin{proof}
The set of couples $(X,Y)$ such that $\gamma_{X,Y}$ is a weak equivalence is saturated by monomorphisms. It is then enough to show the result for any couples of representables. 

Let's start by the case $(X,Y)=([n],[m])$. Let $s:X\star Y\to X\diamond Y$ be the morphism defined on objects by the formula: 
$$s(k\star \emptyset) := (k,0,0)~~~s(\emptyset \star l) := (n,1,l)$$
We have
$$\gamma_{X,Y}s = id ~~~~s\gamma_{X,Y} (k,\epsilon,l) =(k + \epsilon (n-k), \epsilon,\epsilon l).$$

Let $\eta:[n]\diamond [m]\to [n]\diamond [m]$ be induced by the application
$$(k,\epsilon,l)\mapsto (k,\epsilon,\epsilon l).$$
We are now going to construct two morphisms
$$\epsilon_0: ([n]\diamond[m])\times [1]_t\to [n]\diamond[m]~~~~\mbox{ and }~~~~\epsilon_1: ([n]\diamond[m])\times [1]_t\to [n]\diamond[m]$$
such that $$
\begin{array}{rrl}
\epsilon_0(\uvar,0)=\eta&&\epsilon_0(\uvar,1)=s\gamma_{X,Y}\\
\epsilon_1(\uvar,0)=\eta&~~~~&\epsilon_1(\uvar,1)=id\\
\end{array}$$
The first one is induced on the level of simplicial sets by
$$(k,\epsilon,l,\alpha)\mapsto (k + \alpha\epsilon (n-k),\epsilon,\epsilon l ),$$
and the second one by
$$(k,\epsilon,l,\alpha)\mapsto (k,\epsilon,(\epsilon\vee\alpha)l),$$
where $\epsilon\vee\alpha := \epsilon+\alpha - \epsilon\alpha.$
These two morphisms extend to marked simplicial sets. 

We proceed in a similar way with cases $(X,Y) = ([n]_t,[m]), ([n],[m]_t)$ or $([n]_t,[m]_t)$. 
\end{proof}

As we already now that functors $\uvar\diamond X$ and $X\diamond \uvar$ preserve weak equivalences, the previous proposition implies that for any marked simplicial sets $X$, functors $\uvar\star X$ and $X\star \uvar$ preserves weak equivalences and are then
 left Quillen functors. 


\p 
\label{para:sigma star}
Let $X$ be a marked simplicial set. We now describe an variation on the suspension. We define \wcnotation{$\Sigma^\star X$}{(sigmastar@$\Sigma^\star\uvar$}, as the following pushout:
% https://q.uiver.app/#q=WzAsNCxbMSwwLCJYXFxzdGFyIFswXSJdLFswLDAsIlgiXSxbMCwxLCIxIl0sWzEsMSwiXFxTaWdtYV5cXHN0YXIgWCJdLFswLDNdLFszLDEsIiIsMCx7InN0eWxlIjp7Im5hbWUiOiJjb3JuZXIifX1dLFsxLDJdLFsxLDBdLFsyLDNdXQ==
\[\begin{tikzcd}
	X & {X\star [0]} \\
	1 & {\Sigma^\star X}
	\arrow[from=1-2, to=2-2]
	\arrow["\lrcorner"{anchor=center, pos=0.125, rotate=180}, draw=none, from=2-2, to=1-1]
	\arrow[from=1-1, to=2-1]
	\arrow[from=1-1, to=1-2]
	\arrow[from=2-1, to=2-2]
\end{tikzcd}\]
This assignation defines a cocontinuous functor $\Sigma^\star:\mSset\to \mSset_{\partial[1]/}.$ Using proposition \ref{prop:equivalence between diamond and join product}, all the vertical morphisms of the following diagram are weak equivalences:
% https://q.uiver.app/#q=WzAsNixbMiwwLCJYXFxkaWFtb25kIDEiXSxbMSwwLCJYIl0sWzAsMCwiMSJdLFsxLDEsIlgiXSxbMiwxLCJYXFxzdGFyMSJdLFswLDEsIjEiXSxbMSwwXSxbMSwyXSxbMyw1XSxbMCw0XSxbMyw0XSxbMSwzXSxbMiw1XV0=
\[\begin{tikzcd}
	1 & X & {X\diamond 1} \\
	1 & X & X\star1
	\arrow[from=1-2, to=1-3]
	\arrow[from=1-2, to=1-1]
	\arrow[from=2-2, to=2-1]
	\arrow[from=1-3, to=2-3]
	\arrow[from=2-2, to=2-3]
	\arrow[from=1-2, to=2-2]
	\arrow[from=1-1, to=2-1]
\end{tikzcd}\]
Remark furthermore that the colimits of these lines are also homotopy colimits. Taking the horizontal colimit, this induces a weak equivalence
\begin{equation}
\label{eq:sigma et sigam star}
\Sigma X\to \Sigma^{\star}X
\end{equation}
natural in $X$.




\p
We define the \notion{co-join} of $X$ and $Y$, denoted by \index[notation]{((d22@$\overset{co}{\star}$}$X\costar Y$, as the colimit of the following diagram:
% https://q.uiver.app/?q=WzAsNSxbMiwwLCJZXFxvdGltZXMgWzFdXFxvdGltZXMgWCJdLFszLDAsIllcXG90aW1lcyBcXHswXFx9XFxvdGltZXMgWCJdLFsxLDAsIllcXG90aW1lcyBcXHsxXFx9XFxvdGltZXMgWCJdLFswLDAsIlkiXSxbNCwwLCJYIl0sWzIsM10sWzIsMF0sWzEsNF0sWzEsMF1d
\[\begin{tikzcd}
	Y & {Y\otimes \{1\}\otimes X} & {Y\otimes [1]\otimes X} & {Y\otimes \{0\}\otimes X} & X
	\arrow[from=1-2, to=1-1]
	\arrow[from=1-2, to=1-3]
	\arrow[from=1-4, to=1-5]
	\arrow[from=1-4, to=1-3]
\end{tikzcd}\]
The functors 
$$\uvar\costar X:\mSset\to \mSset_{/X} ~~\mbox{and}~~ X\costar \uvar:\mSset\to \mSset_{/X}$$
are colimit preserving. Furthermore, for every acyclic cofibration $K\to L$, the morphism $K\costar X\to L\costar X$ is the horizontal colimit of the diagram:
% https://q.uiver.app/?q=WzAsNixbMiwwLCJYXFxvdGltZXMgWzFdXFxvdGltZXMgSyJdLFsyLDEsIlhcXG90aW1lcyBbMV0gXFxvdGltZXMgSyJdLFsxLDAsIlhcXG90aW1lcyBcXHBhcnRpYWxbMV1cXG90aW1lcyBLIl0sWzEsMSwiWFxcb3RpbWVzIFxccGFydGlhbFsxXVxcb3RpbWVzIEwiXSxbMCwwLCJLXFxhbWFsZyBYIl0sWzAsMSwiTFxcYW1hbGcgWCJdLFsyLDRdLFsyLDBdLFszLDFdLFszLDVdLFsyLDNdLFswLDFdLFs0LDVdXQ==
\[\begin{tikzcd}
	{K\amalg X} & {X\otimes \partial[1]\otimes K} & {X\otimes [1]\otimes K} \\
	{L\amalg X} & {X\otimes \partial[1]\otimes L} & {X\otimes [1] \otimes K}
	\arrow[from=1-2, to=1-1]
	\arrow[from=1-2, to=1-3]
	\arrow[from=2-2, to=2-3]
	\arrow[from=2-2, to=2-1]
	\arrow[from=1-2, to=2-2]
	\arrow[from=1-3, to=2-3]
	\arrow[from=1-1, to=2-1]
\end{tikzcd}\]
However, these two horizontal colimits are homotopy colimits, and all the horizontal maps of the previous diagram are weak equivalences. This morphism is then an acyclic cofibration.
This shows that $\uvar\costar X$ is a left Quillen functor. We show analogously that $X\costar \uvar$ is a left Quillen functor.

\p
\label{subsection:wedge} Let $X$ be a simplicial set. We define the \textit{wedge} of $\Sigma X$ and $[1]$, noted by \sym{(sigmavee@$\Sigma X~\rotatebox[origin=c]{270}{$\gtrdot$}~[1]$}\sym{(sigmave@$[1]~\rotatebox[origin=c]{270}{$\gtrdot$}\Sigma X$}$\Sigma X\fwedge [1]$, as the colimit of the following diagram:
% https://q.uiver.app/?q=WzAsNixbMSwwLCJYXFxvdGltZXNbMl1fdCJdLFswLDAsIlhcXG90aW1lc1swLDFdIl0sWzAsMSwiXFxTaWdtYSBYIl0sWzIsMCwiWFxcb3RpbWVzWzEsMl0iXSxbMiwxLCJbMSwyXSJdLFsxLDEsIlhcXGZ3ZWRnZVsxXSJdLFsxLDJdLFszLDRdLFsxLDBdLFszLDBdLFswLDVdLFs0LDVdLFsyLDVdXQ==
\[\begin{tikzcd}
	{X\otimes[0,1]} & {X\otimes[2]_t} & {X\otimes[1,2]} \\
	{\Sigma X} & {X\fwedge[1]} & {[1,2]}
	\arrow[from=1-1, to=2-1]
	\arrow[from=1-3, to=2-3]
	\arrow[from=1-1, to=1-2]
	\arrow[from=1-3, to=1-2]
	\arrow[from=1-2, to=2-2]
	\arrow[from=2-3, to=2-2]
	\arrow[from=2-1, to=2-2]
\end{tikzcd}\]
This assignation defines a cocontinuous functor $\uvar\fwedge [1]:\mSset\to \mSset_{[0]\amalg [1]/}.$ For every acyclic cofibration $K\to L$, the morphism $K\fwedge [1]\to L\fwedge [1]$ is the horizontal colimit of the diagram:
% https://q.uiver.app/?q=WzAsNixbMiwwLCJLXFxvdGltZXNbMl1fdCJdLFsxLDAsIktcXG90aW1lcyhbMF1cXGNvcHJvZFsxLDJdKSJdLFswLDAsIlswXVxcY29wcm9kWzFdIl0sWzIsMSwiTFxcb3RpbWVzWzJdX3QiXSxbMSwxLCJMXFxvdGltZXNbMl1fdCJdLFswLDEsIktcXG90aW1lc1syXV90Il0sWzEsMl0sWzEsMF0sWzIsNV0sWzQsNV0sWzQsM10sWzAsM10sWzEsNF1d
\[\begin{tikzcd}
	{[0]\coprod[1]} & {K\otimes([0]\coprod[1,2])} & {K\otimes[2]_t} \\
	{K\otimes[2]_t} & {L\otimes[2]_t} & {L\otimes[2]_t}
	\arrow[from=1-2, to=1-1]
	\arrow[from=1-2, to=1-3]
	\arrow[from=1-1, to=2-1]
	\arrow[from=2-2, to=2-1]
	\arrow[from=2-2, to=2-3]
	\arrow[from=1-3, to=2-3]
	\arrow[from=1-2, to=2-2]
\end{tikzcd}\]
However, these two horizontal colimits are homotopy colimits, and all the horizontal maps of the previous diagram are weak equivalences. This morphism is then an acyclic cofibration.
This shows that this functor is a left Quillen functor. We denote by $$\triangledown:\Sigma X\to \Sigma X\fwedge [1]$$ the morphism induced by the inclusion $X\otimes [0,2]\subset X\otimes [2]_t$ and 
$$\Sigma X\hookrightarrow \Sigma X\fwedge [1]$$
the morphism induced by the inclusion $X\otimes [1,2]\subset X\otimes [2]_t$.
We define similarly the left Quillen functor $$[1]\fwedge\uvar:\mSset\to \mSset_{[1]\amalg [0]/}$$ and the morphisms
$$\triangledown:\Sigma X\to [1]\fwedge\Sigma X~~~\mbox{and}~~~\Sigma X\hookrightarrow [1]\fwedge\Sigma X .$$

\begin{prop}
Morphisms 
$$ \Sigma X\coprod_{[0]}[1]\to \Sigma X\fwedge [1]~~~~\mbox{and}~~~~ [1]\coprod_{[0]}\Sigma X\to [1]\fwedge \Sigma X$$
are acyclic cofibrations. 
\end{prop}
\begin{proof}
We have cartesian squares:
% https://q.uiver.app/?q=WzAsNixbMSwwLCJYXFxvdGltZXMgXFxMYW1iZGFeezF9WzJdIl0sWzIsMCwiWFxcb3RpbWVzWzJdX3QiXSxbMCwwLCJYXFxvdGltZXMoWzBdXFxjb3Byb2RbMSwyXSkiXSxbMCwxLCJbMF1cXGNvcHJvZFsxXSJdLFsxLDEsIlxcU2lnbWEgWFxcY29wcm9kX3tbMF19IFsxXSJdLFsyLDEsIlxcU2lnbWEgWFxcZndlZGdlIFsxXS4iXSxbMiwzXSxbMyw0XSxbMiwwXSxbMCwxXSxbMSw1XSxbMCw0XSxbNSwwLCIiLDEseyJzdHlsZSI6eyJuYW1lIjoiY29ybmVyIn19XSxbNCwyLCIiLDEseyJzdHlsZSI6eyJuYW1lIjoiY29ybmVyIn19XSxbNCw1XV0=
\[\begin{tikzcd}
	{X\otimes([0]\coprod[1,2])} & {X\otimes \Lambda^{1}[2]} & {X\otimes[2]_t} \\
	{[0]\coprod[1]} & {\Sigma X\coprod_{[0]} [1]} & {\Sigma X\fwedge [1].}
	\arrow[from=1-1, to=2-1]
	\arrow[from=2-1, to=2-2]
	\arrow[from=1-1, to=1-2]
	\arrow[from=1-2, to=1-3]
	\arrow[from=1-3, to=2-3]
	\arrow[from=1-2, to=2-2]
	\arrow["\lrcorner"{anchor=center, pos=0.125, rotate=180}, draw=none, from=2-3, to=1-2]
	\arrow["\lrcorner"{anchor=center, pos=0.125, rotate=180}, draw=none, from=2-2, to=1-1]
	\arrow[from=2-2, to=2-3]
\end{tikzcd}\]
The upper right horizontal morphism is an acyclic cofibration, and so is the downer right horizontal one. We proceed similarly for the other morphism.
\end{proof}

\subsection{Gray cylinder, Gray cone and Gray $\circ$-cone}
\p The Gray tensor product induced a left Quillen functor 
$$\uvar\otimes[1]:\mSset\to \mSset$$
called the \snotionsym{Gray cylinder}{((d30@$\uvar\otimes[1]$}{for marked simplicial sets}. 
The join and the co-join also incuce two left Quillen functors
$$\uvar\star [0]:\mSset\to \mSset_{[0]/}~~~~~[0]\costar \uvar:\mSset\to \mSset_{[0]/}$$
called the \snotionsym{Gray cone}{((d40@$\uvar\star 1$}{for marked simplicial sets} and the \snotion{Gray $\circ$-cone}{for marked simplicial sets}\index[notation]{((d50@$1\overset{co}{\star}\_$!\textit{for marked simplicial sets}}. We denote by 
$$
\begin{array}{rclcrcl}
 \mSset_{\cdot} &\to &\mSset & & \mSset_{\cdot}&\to & \mSset\\
(X,x)&\mapsto & X_{/x} &~~~~~ & (X,x)&\mapsto & X_{x/}\\
\end{array}
$$
respectively called the \wcsnotionsym{slice of $X$ over $x$}{(cc@$C_{c/}$}{slice over}{for marked simplicial sets} and the \wcsnotionsym{slice of $X$ under $x$}{(cc@$C_{/c}$}{slice under}{for marked simplicial sets}, the right adjoints of the Gray cone and the Gray $\circ$-cone.

Remark furthermore that we have canonical natural transformation $X_{x/}\to X$ and $X_{/x}\to X$, induced by the natural transformation $X\to X\star [0]$ and $X\to [0]\costar X$.
\p 
The category of endomorphisms of marked simplicial sets has a monoidal structure given by the composition. The endomorphism $[0]\costar \uvar$ admits a monoid structure, where the multiplication is the natural transformation:
$[0]\costar ([0]\costar X)\to [0]\costar X$, induced by the pairing: 
$$
\begin{array}{rcl}
X\otimes[1]\otimes[1]&\to& X\otimes[1]\\
(x,i,j)&\mapsto& (x,i\wedge j).
\end{array}$$

This defines a cosimplicial object in $\End(\mSset)$, which evaluated on $\emptyset$, provides a cosimplicial object in $\mSset$: 
$$\begin{array}{rcl}
\Delta &\to & \mSset\\
n&\mapsto&[n]_{\circ}:=[0]\costar (... ([0]\costar[0])).
\end{array}$$
Eventually, we set $([n]_t)_{\circ} := \tau^i_{n-1}([n]_{\circ})$. We then have defined a functor:
$$(\uvar)_{\circ}:t\Delta \to \mSset.$$ 
\sym{((b90@$(\uvar)_{\circ}$}





\subsection{Street nerve}
\label{section:Street nerve}

We recall that $\zo$-categories are defined in section \ref{section:zocategories}. The Gray operations on $\zo$-categories - 
$\uvar\otimes[1]$, $\uvar\star 1$, $1\costar \uvar$ -
are defined in section \ref{section:definition of Gray operations}.


In \cite{Street_algebra_of_orianted_simplexes}, Street defines a cosimplicial object in $\zocat$, that associates to $n$, the $n^{th}$ \notion{oriental} $O_n$. 
The original construction of this object is complicated, but Ara and Maltsiniotis have shown that it can be easily defined using Gray operations. Indeed, in \cite[Corollaire 7.10]{Ara_Maltsiniotis_joint_et_tranche}, these authors construct an isomorphism
$$O_n\cong \overbrace{1\star...\star 1}^{n+1}$$
natural in $n$.

We can extend the functor $O_{\uvar}:\Delta\to \zocat$ to $t\Delta$ by defining
$$(O_n)_t:=\tau^i_{n-1}(O_n).$$
By extention by colimit, this induces a functor 
$$\R:\stratSset\to \zocat.$$
As explained in example 11 of \cite{Verity_weak_complicial_set_part2_nerve_of_complicial_Gray_categories}, $\R$ preserves the Gray tensor product, and so also the suspension, the wedge, the Gray cone and the Gray $\circ$-cone. 
 Moreover, \cite[Theorem 249]{Verity_complicial_set} states that this functor sends complicial horn inclusions and complicial thinness extensions to isomorphisms. It obviously also sends saturation extensions to isomorphisms. This functor then sends every weak equivalences to isomorphisms, and then lifts to a colimit preserving functor $\R:\mSset\to \zocat$ and induces an adjoint pair: \sym{(r@$\R:\mSset\to \zocat$}\sym{(n@$\N:\zocat\to \mSset$}
 % https://q.uiver.app/#q=WzAsMixbMCwwLCJcXFI6XFxtU3NldCJdLFsxLDAsIlxcem9jYXQ6XFxOIl0sWzAsMSwiIiwwLHsib2Zmc2V0IjotMn1dLFsxLDAsIiIsMCx7Im9mZnNldCI6LTJ9XSxbMiwzLCIiLDAseyJsZXZlbCI6MSwic3R5bGUiOnsibmFtZSI6ImFkanVuY3Rpb24ifX1dXQ==
\[\begin{tikzcd}
	{\R:\mSset} & {\zocat:\N}
	\arrow[""{name=0, anchor=center, inner sep=0}, shift left=2, from=1-1, to=1-2]
	\arrow[""{name=1, anchor=center, inner sep=0}, shift left=2, from=1-2, to=1-1]
	\arrow["\dashv"{anchor=center, rotate=-90}, draw=none, from=0, to=1]
\end{tikzcd}\]


We now recall two fundamental results of strictification:
\begin{theorem}[Gagna, Ozornova, Rovelli]
\label{theo:strict representable}
Let $n$ be an integer. The canonical morphism
$$[n]\to \N(\R([n]))$$
is an acyclic cofibration.
\end{theorem}
\begin{proof}
This is \cite[corollary 5.4]{Gagna_Nerves_and_cones_of_free_loop_free_omega_categories}.
\end{proof}
\begin{theorem}[Ozornova, Rovelli]
\label{theo:strict susension}
Let $C$ be an $\zo$-category.
The canonical morphism
$$\Sigma \N C \to \N([C,1])$$
is an acyclic cofibration.
\end{theorem}
\begin{proof}
The morphism \eqref{eq:sigma et sigam star} provides a weak equivalence
$\Sigma \N C\to \Sigma^{\star} \N C$.
As this morphism is sent to an isomorphism by $R$, it induces a commutative triangle
% https://q.uiver.app/#q=WzAsMyxbMSwwLCJcXFNpZ21hXntcXHN0YXJ9ICBcXE4gQyJdLFswLDEsIlxcU2lnbWEgIFxcTiBDIl0sWzIsMSwiXFxOKFtDLDFdKSJdLFsxLDJdLFsxLDAsIlxcc2ltIl0sWzAsMl1d
\[\begin{tikzcd}
	& {\Sigma^{\star} \N C} \\
	{\Sigma \N C} && {\N([C,1])}
	\arrow[from=2-1, to=2-3]
	\arrow["\sim", from=2-1, to=1-2]
	\arrow[from=1-2, to=2-3]
\end{tikzcd}\]
The theorem 3.22 of \cite{Ozornova_a_quillen_adjunction_between_globular_and_complicial} stipulates that $\Sigma^{\star} \N C\to \N([C,1])$ is a weak equivalence, which concludes the proof.
\end{proof}


\begin{definition}
We define the \notion{Street endofunctor} \wcnotation{$i_{str}$}{(istr@$i_{str}$} to be the colimit preserving functor defined on representables by: 
$$i_{str}([n]) := \N(\R([n]))~~~\mbox{ and }~~~i_{str}([n]_t) :=\tau^i_{n-1} (i_{str}([n]))$$
\end{definition}

\begin{prop}
\label{prop:i_str_is_Quillen}
 The functor $i_{srt}$ is left Quillen and 
the natural transformation 
$$id \to i_{srt}$$ 
is weakly invertible.
\end{prop}
\begin{proof}
As noticed earlier, for any integer $n$, the map $[n]\to i_{srt}([n])$ is a weak equivalence.
We recall that the intelligent truncation functor $\tau^i_{n-1}:\mSset\to \mSset$ is a left Quillen functor, and so preserves weak equivalences between cofibrant objects. The morphism $[n]_t\to i_{str}([n]_t)$ is then a weak equivalence.
The set of objects $X$ such that the morphism $X\to i_{srt}X$ is a weak equivalence is closed by homotopy colimits and includes all representables. As $i_{srt}$ preserves monomorphisms, it then consists of all marked simplicial sets. Now let $K\to L$ be an acyclic cofibration. We have a commutative square:
% https://q.uiver.app/?q=WzAsNCxbMSwwLCJpX3tzdHJ9KEspIl0sWzEsMSwiaV97c3RyfShMKSJdLFswLDAsIksiXSxbMCwxLCJMIl0sWzIsMywiXFxzaW0iLDJdLFsyLDAsIlxcc2ltIl0sWzMsMSwiXFxzaW0iLDJdLFswLDFdXQ==
\[\begin{tikzcd}
	K & {i_{str}(K)} \\
	L & {i_{str}(L)}
	\arrow["\sim"', from=1-1, to=2-1]
	\arrow["\sim", from=1-1, to=1-2]
	\arrow["\sim"', from=2-1, to=2-2]
	\arrow[from=1-2, to=2-2]
\end{tikzcd}\]
By two out of three, $i_{str}(K)\to i_{str}(L)$ is then an acyclic cofibration. The functor $i_{srt}$ is then left Quillen. 
\end{proof}


\section{Suspension and Gray operations}
\label{section:Suspension and Gray operation}
\subsection{Formula for the Gray cylinder}
The aim of this subsection is to demonstrate the following theorem:
\begin{theorem}
\label{theo:interval_first_formula}
There is a zigzag of acyclic cofibrations, natural in $X$, between the colimit of the diagram
$$[1]\fwedge\Sigma X\xleftarrow{\triangledown} \Sigma (X\otimes\{0\})\hookrightarrow \Sigma(X\otimes[1])\hookleftarrow \Sigma (X\otimes\{1\})\xrightarrow{\triangledown} \Sigma X\fwedge[1]$$
and $(\Sigma X)\otimes [1]$.
\end{theorem}

\begin{construction}

Let $C$ be the following colimit:
% https://q.uiver.app/?q=WzAsNCxbMSwwLCJbM11cXHRpbWVzWzFdIl0sWzAsMCwiWzNdXFx0aW1lc1xcezBcXH1cXGNvcHJvZCBbM11cXHRpbWVzXFx7MVxcfSJdLFswLDEsIlsxXVxcY29wcm9kWzFdIl0sWzEsMSwiQy4iXSxbMSwyLCJzXjBzXjBcXGNvcHJvZCBzXjJzXjMiLDJdLFsxLDBdLFswLDNdLFsyLDNdLFszLDUsIiIsMix7ImxldmVsIjoxLCJzdHlsZSI6eyJuYW1lIjoiY29ybmVyIn19XV0=
\[\begin{tikzcd}
	{[3]\times\{0\}\coprod [3]\times\{1\}} & {[3]\times[1]} \\
	{[1]\coprod[1]} & {C.}
	\arrow["{s^0s^0\coprod s^2s^3}"', from=1-1, to=2-1]
	\arrow[""{name=0, anchor=center, inner sep=0}, from=1-1, to=1-2]
	\arrow[from=1-2, to=2-2]
	\arrow[from=2-1, to=2-2]
	\arrow["\lrcorner"{anchor=center, pos=0.125, rotate=180}, draw=none, from=2-2, to=0]
\end{tikzcd}\]

We define several marked simplicial sets whose underlying simplicial sets are sub objects of C: 
% https://q.uiver.app/?q=WzAsMTgsWzEsMCwiMDAiXSxbMiwwLCIwMSJdLFsyLDEsIjExIl0sWzEsMSwiMTAiXSxbMSwyLCIyMCJdLFsyLDIsIjIxIl0sWzAsMF0sWzQsMCwiMDAiXSxbNSwwLCIwMSJdLFs0LDEsIjIwIl0sWzUsMSwiMjEiXSxbMSwzLCIzMCJdLFsyLDMsIjMxIl0sWzQsMiwiMDAiXSxbMywzXSxbNSwyLCIwMSJdLFs0LDMsIjMwIl0sWzUsMywiMzEiXSxbMSwyXSxbMyw0LCJcXGxhcmdle0FfMTo9fn5+fn5+fSIsMix7ImxldmVsIjoyLCJzdHlsZSI6eyJoZWFkIjp7Im5hbWUiOiJub25lIn19fV0sWzIsNSwiIiwwLHsibGV2ZWwiOjIsInN0eWxlIjp7ImhlYWQiOnsibmFtZSI6Im5vbmUifX19XSxbMCwzLCJcXGxhcmdle0FfMDo9fn5+fn5+fSIsMix7ImxldmVsIjoyLCJzdHlsZSI6eyJoZWFkIjp7Im5hbWUiOiJub25lIn19fV0sWzAsMV0sWzQsNV0sWzMsMl0sWzAsMl0sWzMsNV0sWzcsOSwiXFxsYXJnZXtBXzM6PX5+fn5+fn0iLDIseyJsZXZlbCI6Miwic3R5bGUiOnsiaGVhZCI6eyJuYW1lIjoibm9uZSJ9fX1dLFs4LDEwXSxbOSwxMF0sWzcsMTBdLFsxMSwxMl0sWzEzLDE2LCJcXGxhcmdle0FfNDo9fn5+fn5+fSIsMl0sWzE2LDE3XSxbMTUsMTddLFsxMywxNV0sWzEzLDE3XSxbNyw4XSxbNCwxMSwiXFxsYXJnZXtBXzI6PX5+fn5+fn0iLDJdLFs0LDEyXSxbNSwxMiwiIiwxLHsibGV2ZWwiOjIsInN0eWxlIjp7ImhlYWQiOnsibmFtZSI6Im5vbmUifX19XSxbMjUsMSwiXFxzaW0iLDEseyJzaG9ydGVuIjp7InNvdXJjZSI6MjB9LCJzdHlsZSI6eyJib2R5Ijp7Im5hbWUiOiJub25lIn0sImhlYWQiOnsibmFtZSI6Im5vbmUifX19XSxbMjYsMiwiXFxzaW0iLDEseyJzaG9ydGVuIjp7InNvdXJjZSI6MjB9LCJzdHlsZSI6eyJib2R5Ijp7Im5hbWUiOiJub25lIn0sImhlYWQiOnsibmFtZSI6Im5vbmUifX19XSxbMjUsMywiXFxzaW0iLDEseyJzaG9ydGVuIjp7InNvdXJjZSI6MjB9LCJzdHlsZSI6eyJib2R5Ijp7Im5hbWUiOiJub25lIn0sImhlYWQiOnsibmFtZSI6Im5vbmUifX19XSxbMzAsOCwiXFxzaW0iLDEseyJzaG9ydGVuIjp7InNvdXJjZSI6MjB9LCJzdHlsZSI6eyJib2R5Ijp7Im5hbWUiOiJub25lIn0sImhlYWQiOnsibmFtZSI6Im5vbmUifX19XSxbMzAsOSwiIiwxLHsic2hvcnRlbiI6eyJzb3VyY2UiOjIwfX1dLFszNiwxNSwiXFxzaW0iLDEseyJzaG9ydGVuIjp7InNvdXJjZSI6MjB9LCJzdHlsZSI6eyJib2R5Ijp7Im5hbWUiOiJub25lIn0sImhlYWQiOnsibmFtZSI6Im5vbmUifX19XSxbMzYsMTYsIiIsMSx7InNob3J0ZW4iOnsic291cmNlIjoyMH19XSxbMjYsNCwiIiwwLHsic2hvcnRlbiI6eyJzb3VyY2UiOjIwfX1dLFszOSwxMSwiXFxzaW0iLDEseyJzaG9ydGVuIjp7InNvdXJjZSI6MjB9LCJzdHlsZSI6eyJib2R5Ijp7Im5hbWUiOiJub25lIn0sImhlYWQiOnsibmFtZSI6Im5vbmUifX19XSxbMzksNSwiXFxzaW0iLDEseyJzaG9ydGVuIjp7InNvdXJjZSI6MjB9LCJzdHlsZSI6eyJib2R5Ijp7Im5hbWUiOiJub25lIn0sImhlYWQiOnsibmFtZSI6Im5vbmUifX19XV0=
\[\begin{tikzcd}
	{} & 00 & 01 && 00 & 01 \\
	& 10 & 11 && 20 & 21 \\
	& 20 & 21 && 00 & 01 \\
	& 30 & 31 & {} & 30 & 31
	\arrow[from=1-3, to=2-3]
	\arrow["{\large{A_1:=~~~~~~}}"', Rightarrow, no head, from=2-2, to=3-2]
	\arrow[Rightarrow, no head, from=2-3, to=3-3]
	\arrow["{\large{A_0:=~~~~~~}}"', Rightarrow, no head, from=1-2, to=2-2]
	\arrow[from=1-2, to=1-3]
	\arrow[from=3-2, to=3-3]
	\arrow[from=2-2, to=2-3]
	\arrow[""{name=0, anchor=center, inner sep=0}, from=1-2, to=2-3]
	\arrow[""{name=1, anchor=center, inner sep=0}, from=2-2, to=3-3]
	\arrow["{\large{A_3:=~~~~~~}}"', Rightarrow, no head, from=1-5, to=2-5]
	\arrow[from=1-6, to=2-6]
	\arrow[from=2-5, to=2-6]
	\arrow[""{name=2, anchor=center, inner sep=0}, from=1-5, to=2-6]
	\arrow[from=4-2, to=4-3]
	\arrow["{\large{A_4:=~~~~~~}}"', from=3-5, to=4-5]
	\arrow[from=4-5, to=4-6]
	\arrow[from=3-6, to=4-6]
	\arrow[from=3-5, to=3-6]
	\arrow[""{name=3, anchor=center, inner sep=0}, from=3-5, to=4-6]
	\arrow[from=1-5, to=1-6]
	\arrow["{\large{A_2:=~~~~~~}}"', from=3-2, to=4-2]
	\arrow[""{name=4, anchor=center, inner sep=0}, from=3-2, to=4-3]
	\arrow[Rightarrow, no head, from=3-3, to=4-3]
	\arrow["\sim"{description}, Rightarrow, draw=none, from=0, to=1-3]
	\arrow["\sim"{description}, Rightarrow, draw=none, from=1, to=2-3]
	\arrow["\sim"{description}, Rightarrow, draw=none, from=0, to=2-2]
	\arrow["\sim"{description}, Rightarrow, draw=none, from=2, to=1-6]
	\arrow[shorten <=2pt, Rightarrow, from=2, to=2-5]
	\arrow["\sim"{description}, Rightarrow, draw=none, from=3, to=3-6]
	\arrow[shorten <=2pt, Rightarrow, from=3, to=4-5]
	\arrow[shorten <=2pt, Rightarrow, from=1, to=3-2]
	\arrow["\sim"{description}, Rightarrow, draw=none, from=4, to=4-2]
	\arrow["\sim"{description}, Rightarrow, draw=none, from=4, to=3-3]
\end{tikzcd}\]
where arrows labeled by $=$ are degenerate and simplicies labeled by $\sim$ are thin.


Let $B_0$ be the sub object corresponding to the image of $[0,1,2]\times[0,1]$ where the marking includes all cells of dimension $\leq 2$, except $[10,20,21]$ and $[00,20,21]$.

Let $B_1$ be the sub object corresponding to the image of $[0,2,3]\times[0,1]$ where the marking includes all cells of dimension $\leq 2$, except $[00,20,21]$, $[00,30,31]$ and $[00,20,31]$.

Let $B$ be the reunion of $[0,1,2]\times[0,1]$ and $[0,2,3]\times[0,1]$ where the marking is the reunion of $B_0$ and $B_1$.
\end{construction}


\begin{lemma}
Morphisms $A_0\cup A_1\to B_0$ and $A_3\to B_0$ are acyclic cofibrations. 
\end{lemma}
\begin{proof}
The cofibration $A_0\cup A_1\to B_0$ fits in the following pushout square:
% https://q.uiver.app/?q=WzAsNCxbMSwwLCJBXzFcXGN1cCBBXzIiXSxbMSwxLCJCXzAiXSxbMCwwLCJcXExhbWJkYV57MX1bMl1cXG90aW1lcyBbMV1cXGN1cFsyXV90XFxvdGltZXMgXFxwYXJ0aWFsWzFdIl0sWzAsMSwiWzJdX3RcXG90aW1lcyBbMV0iXSxbMiwwXSxbMiwzXSxbMCwxXSxbMywxLCJbMCwxLDJdXFx0aW1lc1swLDFdIiwyXSxbMSw0LCIiLDEseyJsZXZlbCI6MSwic3R5bGUiOnsibmFtZSI6ImNvcm5lciJ9fV1d
\[\begin{tikzcd}
	{\Lambda^{1}[2]\otimes [1]\cup[2]_t\otimes \partial[1]} & {A_1\cup A_2} \\
	{[2]_t\otimes [1]} & {B_0}
	\arrow[""{name=0, anchor=center, inner sep=0}, from=1-1, to=1-2]
	\arrow[from=1-1, to=2-1]
	\arrow[from=1-2, to=2-2]
	\arrow["{[0,1,2]\times[0,1]}"', from=2-1, to=2-2]
	\arrow["\lrcorner"{anchor=center, pos=0.125, rotate=180}, draw=none, from=2-2, to=0]
\end{tikzcd}\]

The cofibration $A_3\to B_0$ is a sequence of inclusions:
$$A_3=:(D_0,M_0)\subset (D_1,M_1)\subset (D_2,M_2)\subset(D_3,M_3)\subset(D_4,M_4)\subset (D_5,M_5)\subset (D_6,M_6):= B_0,$$ where 

\begin{itemize}[leftmargin=* ,parsep=0cm,itemsep=0cm,topsep=0cm]
\item $D_1 = D_0\cup [00,{01},11]$ ;
\item $D_2 = D_1\cup [ {00},10,11]$ ;
\item $D_2 = D_1\cup [ {00},10,21]$ ;
\item $D_4 = D_3\cup [00, {01},11,21]$; 
\item $D_5 = D_4\cup [ {00},10,11,21]$; 
\item $D_6 = D_5\cup [ {00},10,20,21]$; 
\end{itemize} and
\begin{itemize}[leftmargin=* ,parsep=0cm,itemsep=0cm,topsep=0cm]
\item $(D_0,M_0)\to (D_1,M_1)$ is a pushout of $\Lambda^1[2]\to [2]^1$;
\item $(D_1,M_1)\to (D_2,M_2)$ is a pushout of $\Lambda^0[2]\to [2]^0$;
\item $(D_2,M_2)\to (D_3,M_3)$ is a pushout of $\Lambda^0[2]\to [2]^0$;
\item $(D_3,M_3)\to (D_4,M_4)$ is a pushout of $\Lambda^1[3]\to [3]^1$;
\item $(D_4,M_4)\to (D_5,M_5)$ is a pushout of $\Lambda^0[3]\to [3]^0$;
\item $(D_5,M_5)\to (D_6,M_6)$ is a pushout of $\Lambda^0[3]\to [3]^0$.
\end{itemize}
\end{proof}


\begin{lemma}
Morphisms $A_2\cup A_3\to B_1$ and $A_4\to B_1$ are acyclic cofibrations. 
\end{lemma}
\begin{proof}
The cofibration $A_2\cup A_3\to B_1$ fits in the pushout square:
% https://q.uiver.app/?q=WzAsNCxbMSwwLCJBXzJcXGN1cCBBXzMiXSxbMSwxLCJCXzEiXSxbMCwwLCJcXExhbWJkYV57MX1bMl1cXG90aW1lcyBbMV1cXGN1cCBbMl1fdFxcb3RpbWVzIFxccGFydGlhbFsxXSJdLFswLDEsIlsyXV90XFxvdGltZXMgWzFdIl0sWzIsMF0sWzMsMSwiWzAsMiwzXVxcdGltZXNbMCwxXSIsMl0sWzIsM10sWzAsMV1d
\[\begin{tikzcd}
	{\Lambda^{1}[2]\otimes [1]\cup [2]_t\otimes \partial[1]} & {A_2\cup A_3} \\
	{[2]_t\otimes [1]} & {B_1}
	\arrow[from=1-1, to=1-2]
	\arrow["{[0,2,3]\times[0,1]}"', from=2-1, to=2-2]
	\arrow[from=1-1, to=2-1]
	\arrow[from=1-2, to=2-2]
\end{tikzcd}\]
The cofibration $A_4\to B_1$ is a sequence of inclusions:
$$A_4=:(D_0,M_0)\subset (D_1,M_1)\subset (D_2,M_2)\subset(D_3,M_3)\subset(D_4,M_4)\subset (D_5,M_5)\subset (D_6,M_6):= B_1$$ where 
\begin{itemize}[leftmargin=* ,parsep=0cm,itemsep=0cm,topsep=0cm]
\item $D_1 = D_0\cup [00,21, {31}]$ ;
\item $D_2 = D_1\cup [20, {30},31]$ ;
\item $D_3 = D_2\cup [20,21, {31}]$;
\item $D_4 = D_3\cup [00,01,21, {31}]$;
\item $D_5 = D_4\cup [00,20, {30},31]$ ;
\item $D_6 = D_5\cup [00,20,21, {31}]$ ;
\end{itemize}
and
\begin{itemize}[leftmargin=* ,parsep=0cm,itemsep=0cm,topsep=0cm]
\item $(D_0,M_0)\to (D_1,M_1)$ is a pushout of $\Lambda^2[2]\to [2]^2$;
\item $(D_1,M_1)\to (D_2,M_2)$ is a pushout of $\Lambda^1[2]\to [2]^1$;
\item $(D_2,M_2)\to (D_3,M_3)$ is a pushout of $\Lambda^2[2]\to [2]^2$;
\item $(D_3,M_3)\to (D_4,M_4)$ is a pushout of $\Lambda^3[3]\to [3]^3$;
\item $(D_4,M_4)\to (D_5,M_5)$ is a pushout of $\Lambda^2[3]\to [3]^2$;
\item $(D_5,M_5)\to (D_6,M_6)$ is a pushout of $\Lambda^3[3]\to [3]^3$.
\end{itemize}
\end{proof}




\begin{lemma}
\label{lemma:formula for gray 0}
The maps $A_0\cup A_1\cup A_2\to B$ and $A_4\to B$ are acyclic cofibrations. 
\end{lemma}
\begin{proof}
This is a direct consequence of the last two lemmas.
\end{proof}




\begin{construction}
The marked simplicial set
$\overline{X\otimes B}$ is the pushout:% https://q.uiver.app/?q=WzAsNCxbMSwxLCJcXG92ZXJsaW5le1hcXG90aW1lcyBCfSJdLFsxLDAsIlhcXG90aW1lcyBCIl0sWzAsMCwiWFxcb3RpbWVzKFswMCwwMV1cXGNvcHJvZCBbMzAsMzFdKSJdLFswLDEsIlswMCwwMV1cXGNvcHJvZCBbMzAsMzFdIl0sWzIsM10sWzMsMF0sWzEsMF0sWzIsMV0sWzAsNywiIiwyLHsibGV2ZWwiOjEsInN0eWxlIjp7Im5hbWUiOiJjb3JuZXIifX1dXQ==
\[\begin{tikzcd}
	{X\otimes([00,01]\coprod [30,31])} & {X\otimes B} \\
	{[00,01]\coprod [30,31]} & {\overline{X\otimes B}.}
	\arrow[from=1-1, to=2-1]
	\arrow[from=2-1, to=2-2]
	\arrow[from=1-2, to=2-2]
	\arrow[""{name=0, anchor=center, inner sep=0}, from=1-1, to=1-2]
	\arrow["\lrcorner"{anchor=center, pos=0.125, rotate=180}, draw=none, from=2-2, to=0]
\end{tikzcd}\]

Let $\overline{X\otimes A_i}$ and $\overline{X\otimes B_i}$ be the sub-objects of $\overline{X\otimes B}$ corresponding to image of ${X\otimes A_i}$ and $	{X\otimes B_i}$.
\end{construction}

\begin{lemma}
\label{lemma:formula for gray 1}
The inclusion 
$	\overline{X\otimes A_{0}}\cup \overline{X\otimes A_{1}}\cup \overline{X\otimes A_{2}}\to	\overline{X\otimes B}$ and 
$\overline{X\otimes A_{4}}\to \overline{X\otimes B}$ are acyclic cofibrations.
\end{lemma}
\begin{proof}
Remark that we have cocartesian squares
% https://q.uiver.app/#q=WzAsNixbMSwwLCJ7WFxcb3RpbWVzIEFfezB9fVxcY3VwIHtYXFxvdGltZXMgQV97MX19XFxjdXAge1hcXG90aW1lcyBBX3syfX0iXSxbMiwwLCJ7WFxcb3RpbWVzIEJ9Il0sWzAsMCwiWFxcb3RpbWVzKFswMCwwMV1cXGNvcHJvZCBbMzAsMzFdKSJdLFswLDEsIlswMCwwMV1cXGNvcHJvZCBbMzAsMzFdIl0sWzEsMSwiXFxvdmVybGluZXtYXFxvdGltZXMgQV97MH19XFxjdXAgXFxvdmVybGluZXtYXFxvdGltZXMgQV97MX19XFxjdXAgXFxvdmVybGluZXtYXFxvdGltZXMgQV97Mn19Il0sWzIsMSwiXFxvdmVybGluZXtYXFxvdGltZXMgQn0iXSxbMCwxXSxbMiwzXSxbMyw0XSxbMiwwXSxbMCw0XSxbNCw1XSxbMSw1XSxbNCw5LCIiLDEseyJsZXZlbCI6MSwic3R5bGUiOnsibmFtZSI6ImNvcm5lciJ9fV0sWzUsNiwiIiwxLHsibGV2ZWwiOjEsInN0eWxlIjp7Im5hbWUiOiJjb3JuZXIifX1dXQ==
\[\begin{tikzcd}
	{X\otimes([00,01]\coprod [30,31])} & {{X\otimes A_{0}}\cup {X\otimes A_{1}}\cup {X\otimes A_{2}}} & {{X\otimes B}} \\
	{[00,01]\coprod [30,31]} & {\overline{X\otimes A_{0}}\cup \overline{X\otimes A_{1}}\cup \overline{X\otimes A_{2}}} & {\overline{X\otimes B}}
	\arrow[""{name=0, anchor=center, inner sep=0}, from=1-2, to=1-3]
	\arrow[from=1-1, to=2-1]
	\arrow[from=2-1, to=2-2]
	\arrow[""{name=1, anchor=center, inner sep=0}, from=1-1, to=1-2]
	\arrow[from=1-2, to=2-2]
	\arrow[from=2-2, to=2-3]
	\arrow[from=1-3, to=2-3]
	\arrow["\lrcorner"{anchor=center, pos=0.125, rotate=180}, draw=none, from=2-2, to=1]
	\arrow["\lrcorner"{anchor=center, pos=0.125, rotate=180}, draw=none, from=2-3, to=0]
\end{tikzcd}\]
and 
% https://q.uiver.app/#q=WzAsNixbMSwwLCJ7WFxcb3RpbWVzIEFfezR9fSJdLFsyLDAsIntYXFxvdGltZXMgQn0iXSxbMCwwLCJYXFxvdGltZXMoWzAwLDAxXVxcY29wcm9kIFszMCwzMV0pIl0sWzAsMSwiWzAwLDAxXVxcY29wcm9kIFszMCwzMV0iXSxbMSwxLCJcXG92ZXJsaW5le1hcXG90aW1lcyBBX3s0fX0iXSxbMiwxLCJcXG92ZXJsaW5le1hcXG90aW1lcyBCfSJdLFswLDFdLFsyLDNdLFszLDRdLFsyLDBdLFswLDRdLFs0LDVdLFsxLDVdLFs0LDksIiIsMSx7ImxldmVsIjoxLCJzdHlsZSI6eyJuYW1lIjoiY29ybmVyIn19XSxbNSw2LCIiLDEseyJsZXZlbCI6MSwic3R5bGUiOnsibmFtZSI6ImNvcm5lciJ9fV1d
\[\begin{tikzcd}
	{X\otimes([00,01]\coprod [30,31])} & {{X\otimes A_{4}}} & {{X\otimes B}} \\
	{[00,01]\coprod [30,31]} & {\overline{X\otimes A_{4}}} & {\overline{X\otimes B}}
	\arrow[""{name=0, anchor=center, inner sep=0}, from=1-2, to=1-3]
	\arrow[from=1-1, to=2-1]
	\arrow[from=2-1, to=2-2]
	\arrow[""{name=1, anchor=center, inner sep=0}, from=1-1, to=1-2]
	\arrow[from=1-2, to=2-2]
	\arrow[from=2-2, to=2-3]
	\arrow[from=1-3, to=2-3]
	\arrow["\lrcorner"{anchor=center, pos=0.125, rotate=180}, draw=none, from=2-2, to=1]
	\arrow["\lrcorner"{anchor=center, pos=0.125, rotate=180}, draw=none, from=2-3, to=0]
\end{tikzcd}\]
The result then follows from lemma \ref{lemma:formula for gray 0}.
\end{proof}
\begin{lemma}
\label{lemma:formula for gray 2}
The morphisms 
$\overline{X\otimes A_0} \to [1]\fwedge \Sigma X$ and $\overline{X\otimes A_2} \to \Sigma X\fwedge [1],$
induced by the morphism $A_0\to [00,01,11]_t$ and $A_2\to [20,30,31]_t$, are acyclic cofibrations. 
\end{lemma}
\begin{proof}
We have cocartesian squares
% https://q.uiver.app/?q=WzAsNixbMiwwLCJYXFxvdGltZXMgQV8wIl0sWzAsMCwiWFxcb3RpbWVzIChbMDAsMDFdXFxjb3Byb2QgXFx7MTFcXH0pIl0sWzAsMSwiWzAwLDAxXVxcY29wcm9kIFxcezExXFx9Il0sWzEsMCwiWFxcb3RpbWVzIFswMCwwMV1cXGNvcHJvZF97WFxcb3RpbWVzWzAxXX0gWFxcb3RpbWVzWzAxLDExXSJdLFsxLDEsIlsxXVxcY29wcm9kX3tcXERlbHRhWzBdfVxcU2lnbWEgWCJdLFsyLDEsIlxcb3ZlcmxpbmV7WFxcb3RpbWVzIEFfMH0iXSxbMSwyXSxbMywwLCJcXHNpbSJdLFsxLDNdLFs0LDUsIlxcc2ltIl0sWzMsNF0sWzAsNV0sWzIsNF0sWzQsOCwiIiwxLHsibGV2ZWwiOjEsInN0eWxlIjp7Im5hbWUiOiJjb3JuZXIifX1dLFs1LDcsIiIsMSx7ImxldmVsIjoxLCJzdHlsZSI6eyJuYW1lIjoiY29ybmVyIn19XV0=
\[\begin{tikzcd}
	{X\otimes ([00,01]\coprod \{11\})} & {X\otimes [00,01]\coprod_{X\otimes[01]} X\otimes[01,11]} & {X\otimes A_0} \\
	{[00,01]\coprod \{11\}} & {[1]\coprod_{[0]}\Sigma X} & {\overline{X\otimes A_0}}
	\arrow[from=1-1, to=2-1]
	\arrow[""{name=0, anchor=center, inner sep=0}, "\sim", from=1-2, to=1-3]
	\arrow[""{name=1, anchor=center, inner sep=0}, from=1-1, to=1-2]
	\arrow["\sim", from=2-2, to=2-3]
	\arrow[from=1-2, to=2-2]
	\arrow[from=1-3, to=2-3]
	\arrow[from=2-1, to=2-2]
	\arrow["\lrcorner"{anchor=center, pos=0.125, rotate=180}, draw=none, from=2-2, to=1]
	\arrow["\lrcorner"{anchor=center, pos=0.125, rotate=180}, draw=none, from=2-3, to=0]
\end{tikzcd}\]
That shows that $[1]\coprod_{[0]} \Sigma X \to \overline{X\otimes A_0}$ is an acyclic cofibration. We then have a commutative diagram: 
% https://q.uiver.app/#q=WzAsMyxbMCwwLCJbMV1cXGNvcHJvZF97WzBdfVxcU2lnbWEgWCJdLFsxLDAsIlxcb3ZlcmxpbmV7WFxcb3RpbWVzIEFfMH0iXSxbMiwwLCJbMV1cXGZ3ZWRnZVxcU2lnbWEgWCJdLFswLDEsIlxcc2ltIl0sWzEsMl0sWzAsMiwiXFxzaW0iLDAseyJjdXJ2ZSI6LTR9XV0=
\[\begin{tikzcd}
	{[1]\coprod_{[0]}\Sigma X} & {\overline{X\otimes A_0}} & {[1]\fwedge\Sigma X}
	\arrow["\sim", from=1-1, to=1-2]
	\arrow[from=1-2, to=1-3]
	\arrow["\sim", curve={height=-24pt}, from=1-1, to=1-3]
\end{tikzcd}\]
and by two out of three, this shows that $\overline{X\otimes A_0} \to [1]\fwedge \Sigma X$ is an acyclic cofibration. 
We proceed similarly for the second morphism. 
\end{proof}



\begin{lemma}
\label{lemma:formula for gray 3}
Marked simplicial sets $\overline{X\otimes A_1}$ and $\overline{X\otimes A_4}$ are respectively equal to $\Sigma (X\otimes [1])$ and $(\Sigma X)\otimes [1]$.
\end{lemma}
\begin{proof}
This is true by the definition of these objects.
\end{proof} 

\begin{proof}[Proof of theorem \ref{theo:interval_first_formula}]
According to lemma \ref{lemma:formula for gray 3} we have a cocartesian square
% https://q.uiver.app/#q=WzAsNCxbMCwwLCJcXG92ZXJsaW5le1hcXG90aW1lcyBBX3swfX1cXGNvcHJvZFxcb3ZlcmxpbmV7WFxcb3RpbWVzIEFfezJ9fSJdLFsxLDAsIlxcb3ZlcmxpbmV7WFxcb3RpbWVzIEFfezB9fVxcY3VwIFxcb3ZlcmxpbmV7WFxcb3RpbWVzIEFfezF9fVxcY3VwIFxcb3ZlcmxpbmV7WFxcb3RpbWVzIEFfezJ9fSJdLFsxLDEsIlsxXVxcZndlZGdlXFxTaWdtYSBYXFxjb3Byb2Rfe1xcU2lnbWEgKFhcXG90aW1lc1xcezBcXH0pfSBcXFNpZ21hKFhcXG90aW1lc1sxXSlcXGNvcHJvZF97XFxTaWdtYSAoWFxcb3RpbWVzXFx7MVxcfSl9IFxcU2lnbWEgWFxcZndlZGdlWzFdIl0sWzAsMSwiWzFdXFxmd2VkZ2VcXFNpZ21hIFhcXGNvcHJvZCBcXFNpZ21hIFhcXGZ3ZWRnZVsxXSJdLFswLDNdLFswLDFdLFsxLDJdLFszLDJdXQ==
\[\begin{tikzcd}
	{\overline{X\otimes A_{0}}\coprod\overline{X\otimes A_{2}}} & {\overline{X\otimes A_{0}}\cup \overline{X\otimes A_{1}}\cup \overline{X\otimes A_{2}}} \\
	{[1]\fwedge\Sigma X\coprod \Sigma X\fwedge[1]} & {[1]\fwedge\Sigma X\coprod_{\Sigma (X\otimes\{0\})} \Sigma(X\otimes[1])\coprod_{\Sigma (X\otimes\{1\})} \Sigma X\fwedge[1]}
	\arrow[from=1-1, to=2-1]
	\arrow[from=1-1, to=1-2]
	\arrow[from=1-2, to=2-2]
	\arrow[from=2-1, to=2-2]
\end{tikzcd}\]
The left vertical morphism is a weak equivalence according to lemma \ref{lemma:formula for gray 2}, and the horizontal morphisms are cofibrations. By left properness, the right vertical morphism is a weak equivalence. Combined with lemmas \ref{lemma:formula for gray 1} and \ref{lemma:formula for gray 3}, this provides a zigzag of weak equivalences between 
$
[1]\fwedge\Sigma X\coprod_{\Sigma (X\otimes\{0\})} \Sigma(X\otimes[1])\coprod_{\Sigma (X\otimes\{1\})} \Sigma X\fwedge[1]$
 and $(\Sigma X)\otimes[1].$
\end{proof}




\subsection{Formulas for the Gray cone and the Gray $\circ$-cone}
\begin{theorem}
\label{theo:cyl_formula}
There is a zigzag of acyclic cofibrations, natural in $X$, between the colimit of the diagram 
$$ \Sigma X\fwedge [1]\leftarrow \Sigma X\to \Sigma([0]\costar X)$$
and $\Sigma X \star[0]$.

There is a zigzag of acyclic cofibrations, natural in $X$, between the colimit of the diagram 
$$\Sigma(X\star[0]) \leftarrow \Sigma X\to [1]\fwedge\Sigma X$$
and $[0]\costar \Sigma X$.
\end{theorem}

\begin{proof}
We consider the diagram:
% https://q.uiver.app/?q=WzAsNixbMiwwLCJcXFNpZ21hIFhcXGZ3ZWRnZVsxXVxcY29wcm9kX3tcXFNpZ21hIFh9IFxcU2lnbWEoICBYXFxvdGltZXNbMV0pIFxcY29wcm9kX3tcXFNpZ21hIFh9WzFdXFxmd2VkZ2VcXFNpZ21hIFgiXSxbMSwxLCJbMV1cXGZ3ZWRnZSBcXFNpZ21hIFgiXSxbMiwxLCJcXFNpZ21hIFhcXGZ3ZWRnZVsxXVxcY29wcm9kX3tcXFNpZ21hIFh9IFxcU2lnbWEoICBYXFxvdGltZXNbMV0pIFxcY29wcm9kX3tcXFNpZ21hIFh9WzFdXFxmd2VkZ2VcXFNpZ21hIFgiXSxbMCwxLCJbMV0iXSxbMSwwLCJbMV1cXGNvcHJvZF97WzBdfVxcU2lnbWEgWCJdLFswLDAsIlsxXSJdLFswLDIsImlkIl0sWzQsNV0sWzEsM10sWzQsMF0sWzEsMl0sWzQsMSwiXFxzaW0iLDJdLFs1LDMsImlkIiwyXV0=
\[\begin{tikzcd}
	{[1]} & {[1]\coprod_{[0]}\Sigma X} & {\Sigma X\fwedge[1]\coprod_{\Sigma X} \Sigma( X\otimes[1]) \coprod_{\Sigma X}[1]\fwedge\Sigma X} \\
	{[1]} & {[1]\fwedge \Sigma X} & {\Sigma X\fwedge[1]\coprod_{\Sigma X} \Sigma( X\otimes[1]) \coprod_{\Sigma X}[1]\fwedge\Sigma X}
	\arrow["id", from=1-3, to=2-3]
	\arrow[from=1-2, to=1-1]
	\arrow[from=2-2, to=2-1]
	\arrow[from=1-2, to=1-3]
	\arrow[from=2-2, to=2-3]
	\arrow["\sim"', from=1-2, to=2-2]
	\arrow["id"', from=1-1, to=2-1]
\end{tikzcd}\]
All vertical morphisms are weak equivalences.
We denote by $A$ the colimit of the first line. The theorem \ref{theo:interval_first_formula} implies that there is a zigzag of acyclic cofibrations between $A$ and $X\diamond [0]$. Colimits of the two lines are homotopy colimits, and the comparison morphism is then an acyclic cofibration. 
We then have a zigzag of acyclic cofibrations: 
$$
X\star [0]\leftarrow X\diamond[0] \leftrightsquigarrow A\to \Sigma X\fwedge [1]\coprod_{\Sigma X} \Sigma([0]\costar X)
$$


The second assertion is demonstrated similarly.
\end{proof}



\begin{cor}
\label{cor:star and zigzag}
Let $f:C\to D$ be a fibration between complicial sets, and $K\to L$ a cofibration. It $f$
has the right lifting property against $$\Sigma( [0]\costar K\cup \emptyset \star L )\to \Sigma([0]\costar L),$$ then $f$
 has the right lifting property against $$(\Sigma K)\star [0]\cup (\Sigma L)\star \emptyset \to \Sigma K\star [0].$$
 
If $f$ has the right lifting property against $\Sigma [1]\to \Sigma[1]_t$, then $f$ has the right lifting property against
$$[1]_t\star\emptyset \cup[1]\star [0] \to [1]_t\star [0]$$
\end{cor}
\begin{proof}
Suppose that $f$ fulfills the condition. The class of cofibration having the right lifting property against $f$ is closed by pushouts and, according to \ref{prop:lifting_property_zigzag_of_acyclic_cofibration}, by zigzag of acyclic cofibration. The morphism 
$$\alpha:\Sigma L\fwedge [1]\coprod\limits_{\Sigma L} \Sigma([0]\costar K\coprod\limits_{\emptyset \star K}\emptyset\star L )\to
 \Sigma L\fwedge [1]\coprod\limits_{\Sigma L} \Sigma([0]\costar L)$$ is then in this class. 
Remark that we have a cocartesian square
% https://q.uiver.app/#q=WzAsNCxbMSwwLCJcXFNpZ21hIEwgXFxjdXBbMV1cXGNvcHJvZFxcbGltaXRzX3tcXFNpZ21hIEsgXFxjdXBbMV19XFxTaWdtYSBLXFxmd2VkZ2UgWzFdXFxjb3Byb2RcXGxpbWl0c197XFxTaWdtYSBMfSBcXFNpZ21hKFswXVxcY29zdGFyIEspIl0sWzAsMCwiXFxTaWdtYSBMIFxcY3VwWzFdXFxjb3Byb2RcXGxpbWl0c197XFxTaWdtYSBLIFxcY3VwWzFdfVxcU2lnbWEgS1xcZndlZGdlIFsxXSJdLFsxLDEsIlxcU2lnbWEgTFxcZndlZGdlIFsxXVxcY29wcm9kXFxsaW1pdHNfe1xcU2lnbWEgTH0gXFxTaWdtYShbMF1cXGNvc3RhciBLXFxjb3Byb2RcXGxpbWl0c197XFxlbXB0eXNldCBcXHN0YXIgS31cXGVtcHR5c2V0XFxzdGFyIEwgKSJdLFswLDEsIlxcU2lnbWEgTFxcZndlZGdlIFsxXSJdLFsxLDBdLFsxLDNdLFswLDJdLFszLDJdXQ==
\[\begin{tikzcd}
	{\Sigma L \cup[1]\coprod\limits_{\Sigma K \cup[1]}\Sigma K\fwedge [1]} & {\Sigma L \cup[1]\coprod\limits_{\Sigma K \cup[1]}\Sigma K\fwedge [1]\coprod\limits_{\Sigma L} \Sigma([0]\costar K)} \\
	{\Sigma L\fwedge [1]} & {\Sigma L\fwedge [1]\coprod\limits_{\Sigma L} \Sigma([0]\costar K\coprod\limits_{\emptyset \star K}\emptyset\star L )}
	\arrow[from=1-1, to=1-2]
	\arrow[from=1-1, to=2-1]
	\arrow[from=1-2, to=2-2]
	\arrow[from=2-1, to=2-2]
\end{tikzcd}\]
where the left vertical morphism, and so also the right vertical morphism, is an acyclic cofibration. This induces a zigzag of acyclic cofibration between $\alpha$ and $\beta$ where $\beta$ is 
$$\Sigma L \cup[1]\coprod\limits_{\Sigma K \cup[1]}\Sigma K\fwedge [1]\coprod\limits_{\Sigma L} \Sigma([0]\costar K)\to 
 \Sigma L\fwedge [1]\coprod\limits_{\Sigma L} \Sigma([0]\costar L)$$
Eventually, the theorem \ref{theo:cyl_formula} induces a zigzag of acyclic cofibration between $\beta$ and 
$(\Sigma K)\star [0]\cup (\Sigma L)\star \emptyset \to \Sigma K\star [0]$ which concludes the proof of the first assertion. 


For the second assertion, remark that $[1]_t\star [0]$ is $\tau^i_1([1]_t\star\emptyset \cup [1]\star [0])$. As $\tau^i_1$ is a left Quillen functor, 
the theorem \ref{theo:cyl_formula} induces a zigzag of acyclic cofibration between $[1]_t\star\emptyset \cup[1]\star [0] \to [1]_t\star [0]$ and 
$$[1]_t\fwedge [1]\coprod_{[1]}\Sigma [1]\to [1]_t\fwedge [1]\coprod_{[1]}\Sigma [1]_t.$$
As this cofibration is a pushout of $\Sigma [1]\to \Sigma [1]_t$, this concludes the proof.
\end{proof}

\begin{cor}
\label{cor:costar and zigzag}
Let $f:C\to D$ be a fibration between complicial sets, and $K\to L$ a cofibration. It $f$
has the right lifting property against
$$\Sigma (L\star \emptyset \cup K \star [0])\to \Sigma (L\star [0]),$$
then $f$ has the right lifting property against 
$$ [0]\costar \Sigma K \cup \emptyset \star \Sigma L \to [0]\costar \Sigma L.$$ 

 
If $f$ has the right lifting property against $\Sigma [1]\to \Sigma[1]_t$, then $f$ has the right lifting property against
$$[0]\costar [1]\cup \emptyset \star [1]_t\to [0]\costar [1]_t$$
\end{cor}
\begin{proof}
The proof is similar to the one of corollary \ref{cor:star and zigzag}.

\end{proof}



\section{Globular equivalences}
\label{section:Globular equivalences}

\subsection{Homotopy categories}


\p
The \wcsnotionsym{$n$-globe}{(da@$\Db_n$}{globe@$n$-globe}{for marked simplicial sets} is the marked simplicial set $\Db_n:=\Sigma^n [0]$. We then have 
$\Db_0:=[0]$ and $\Db_{n+1}:= \Sigma \Db_n$.
This defines a globular object in $\mSset$:
% https://q.uiver.app/?q=WzAsNCxbMCwwLCJcXERiXzAiXSxbMSwwLCJcXERiXzEiXSxbMiwwLCJcXERiXzIiXSxbMywwLCIuLi4iXSxbMCwxLCJpXzBeKyIsMCx7Im9mZnNldCI6LTJ9XSxbMSwyLCJpXzFeKyIsMCx7Im9mZnNldCI6LTJ9XSxbMiwzLCJpXzNeKyIsMCx7Im9mZnNldCI6LTJ9XSxbMCwxLCJpXzBeLSIsMix7Im9mZnNldCI6Mn1dLFsxLDIsImlfMV4tIiwyLHsib2Zmc2V0IjoyfV0sWzIsMywiaV8zXi0iLDIseyJvZmZzZXQiOjJ9XV0=
\[\begin{tikzcd}
	{\Db_0} & {\Db_1} & {\Db_2} & {...}
	\arrow["{i_0^+}", shift left=2, from=1-1, to=1-2]
	\arrow["{i_1^+}", shift left=2, from=1-2, to=1-3]
	\arrow["{i_3^+}", shift left=2, from=1-3, to=1-4]
	\arrow["{i_0^-}"', shift right=2, from=1-1, to=1-2]
	\arrow["{i_1^-}"', shift right=2, from=1-2, to=1-3]
	\arrow["{i_3^-}"', shift right=2, from=1-3, to=1-4]
\end{tikzcd}\]
and we have equalities:
$$i_{n+1}^- i^+_n=i^+_{n+1} i^-_n~~~~i^+_{n+1} i^-_n=i^+_{n+1} i^+_n.$$
 We also set $(\Db_n)_t:= \tau^i_{n-1}(\Db_n)$ for $n>0$ and $\partial\Db_n:=\Sigma^n \emptyset$. 
 We then have a canonical inclusions 
 $$\partial \Db_0\to \Db_0$$ and
 for any $n>0$, we have canonical inclusions 
 $$\partial \Db_n\to \Db_n\to (\Db_n)_t.$$




Let $C$ be a complicial set. A \wcsnotion{$n$-cell}{cell@$n$-cell}{for marked simplicial sets} $a$ of $C$ is a morphism $a:\Db_n\to C$. If $n$ is non null, the \textit{source} of $a$ (resp. the \textit{target} of $a$)
is the $(n-1)$-cell $a\circ i^-_{n-1}$ (resp. $a\circ i^+_{n-1}$). The cell $a$ is thin if the corresponding morphism $\Db_n\to C$ factorizes via $(\Db_n)_t$.


\p From now on, and until the end of this section, we fix a complicial set $C$. All considered cells are cells of $C$.
	

Let $n$ be a non null integer, and $a,b$ two $n$-cells.
Cells $a$ and $b$ are \textit{parallel} if they share the same source and the same target. They are \textit{composable} if the source of $a$ is the target of $b$.


Let $a$ and $b$ be two parallel cells. The cell $a$ is \wcnotion{equivalent}{equivalent $n$-cells} to the cell $b$ if there exists a thin $(n+1)$-cell $d:a\to b$, or equivalently, if there exists a homotopy $\Db_n\times [1]_t$ between $a$ and $b$, and constant on $\partial \Db_n\times [1]_t$. This relation is denoted by $\sim$.



\begin{lemma}
The relation $\sim$ is reflexive, symmetric and transitive.
\end{lemma}
\begin{proof}
This comes from usual properties of fibrant objects.
\end{proof}



\begin{lemma}
Let $a$, $b$ be two equivalent cells. If $a$ is thin, so is $b$.
\end{lemma}
\begin{proof}
As $\{0\}\to [1]_t$ is a weak equivalence, so is $\Db_n\times [1]_t\cup (\Db_n)_t\times \{0\}\to (\Db_n)_t\times [1]_t$. As $C$ is fibrant, this directly implies the result. 
\end{proof}

\begin{construction}
\label{cons:composition_homotopy_category}
Let $a,b$ be two composable $n$-cells . A composition of ${a}$ and ${b}$ is a $n$-cell $a\circ b$ that fits in a diagram:
% https://q.uiver.app/?q=WzAsNCxbMCwwLCJcXERiX25cXGNvcHJvZF97XFxEYl97bi0xfX1cXERiX24iXSxbMCwxLCJcXFNpZ21hXntuLTF9KFsyXV90KSJdLFsxLDEsIkMiXSxbMCwyLCJcXERiX3tufSJdLFswLDFdLFswLDIsImFcXGNvcHJvZCBiIl0sWzMsMiwiYVxcY2lyYyBiIiwyXSxbMywxXSxbMSwyXV0=
\[\begin{tikzcd}
	{\Db_n\coprod_{\Db_{n-1}}\Db_n} \\
	{\Sigma^{n-1}([2]_t)} & C \\
	{\Db_{n}}
	\arrow[from=1-1, to=2-1]
	\arrow["{a\coprod b}", from=1-1, to=2-2]
	\arrow["{a\circ b}"', from=3-1, to=2-2]
	\arrow[from=3-1, to=2-1]
	\arrow[from=2-1, to=2-2]
\end{tikzcd}\]
As $C$ is a fibrant object, if $(a\circ b)'$ is any other composition, $(a\circ b)'\sim a\circ b$.
\end{construction}

\begin{lemma}
\label{lemma:associativity_of_composition_in_homotopy_category}
Let $a,b,c$ be three composable cells. There exists compositions such that $(a\circ b)\circ c = a\circ (b\circ c)$.
\end{lemma}
\begin{proof}
Let $M$ be the marking on $[3]$ that includes all simplices of dimension superior or equal to $2$. We define $\Sp_{[3]}$ as the simplicial set $[1]\coprod_{[0]} [1]\coprod_{[0]} [1]$. Remark that the cofibration $\Sp_{[3]}\to ([3],M)$ is acyclic. We then have a lift $f$ in the following diagram
% https://q.uiver.app/?q=WzAsMyxbMCwwLCJcXFNpZ21hXntuLTF9XFxTcF97WzNdfSJdLFswLDEsIlxcU2lnbWFee24tMX0oWzNdLE0pIl0sWzEsMCwiQyJdLFswLDFdLFswLDIsImFcXGNvcHJvZCBiXFxjb3Byb2QgYyJdLFsxLDIsImYiLDIseyJzdHlsZSI6eyJib2R5Ijp7Im5hbWUiOiJkYXNoZWQifX19XV0=
\[\begin{tikzcd}
	{\Sigma^{n-1}\Sp_{[3]}} & C \\
	{\Sigma^{n-1}([3],M)}
	\arrow[from=1-1, to=2-1]
	\arrow["{a\coprod b\coprod c}", from=1-1, to=1-2]
	\arrow["f"', dashed, from=2-1, to=1-2]
\end{tikzcd}\]
The morphism $f$ provides all the desired compositions.
\end{proof}


\begin{definition}
We define the category $\pi_0(C)$ whose objects are $0$-cells $x:s\to t$, and edges between $x,y:s\to t$ are equivalence classes of the set of $1$-cells $f:x\to y$ quotiented by the relation $\sim$. The composition is given by construction \ref{cons:composition_homotopy_category} which is associative according to lemma \ref{lemma:associativity_of_composition_in_homotopy_category}.

 Let $n>0$ be an integer, and $s,t$ two parallel $(n-1)$-cells. We define the category $\pi_n(s,t,C)$ whose objects are $n$-cells $x:s\to t$, and edges between $x,y:s\to t$ are equivalence classes of the set of $(n+1)$-cells $f:x\to y$ quotiented by the relation $\sim$. 
The composition is given by construction \ref{cons:composition_homotopy_category} which is associative according to lemma \ref{lemma:associativity_of_composition_in_homotopy_category}.
\end{definition}


\begin{prop}
\label{prop:in_the_homotopy_category_thin_is_iso}
Let $x,y:s\to t$ be two parallel $n$-cells, and $f:x\to y$ a $n+1$-cell. The cell $f$ is thin if and only if $[f]:x\to y$ is an isomorphism in $\pi_n(s,t,C)$.
\end{prop}
\begin{proof}
Suppose first that $f$ is thin. There are liftings in the following diagrams:
% https://q.uiver.app/#q=WzAsNixbMCwwLCJcXFNpZ21hXntufVxcTGFtYmRhXjBbMl0iXSxbMSwwLCJDIl0sWzAsMSwiXFxTaWdtYV57bn1bMl1eMCJdLFsyLDAsIlxcU2lnbWFee259XFxMYW1iZGFeMlsyXSJdLFszLDAsIkMiXSxbMiwxLCJcXFNpZ21hXntufVsyXV4wIl0sWzAsMSwiZlxcYW1hbGcgaWQiXSxbMCwyXSxbMiwxLCJoIiwyLHsic3R5bGUiOnsiYm9keSI6eyJuYW1lIjoiZG90dGVkIn19fV0sWzMsNCwiaWRcXGFtYWxnIGYiXSxbNSw0LCJrIiwyLHsic3R5bGUiOnsiYm9keSI6eyJuYW1lIjoiZG90dGVkIn19fV0sWzMsNV1d
\[\begin{tikzcd}
	{\Sigma^{n}\Lambda^0[2]} & C & {\Sigma^{n}\Lambda^2[2]} & C \\
	{\Sigma^{n}[2]^0} && {\Sigma^{n}[2]^0}
	\arrow["{f\amalg id}", from=1-1, to=1-2]
	\arrow[from=1-1, to=2-1]
	\arrow["h"', dotted, from=2-1, to=1-2]
	\arrow["{id\amalg f}", from=1-3, to=1-4]
	\arrow["k"', dotted, from=2-3, to=1-4]
	\arrow[from=1-3, to=2-3]
\end{tikzcd}\]
Let $g:y\to z$ be the restriction of $h$ to $\Sigma^{n}[1,2]$ and $l:y\to z$ be the restriction of $k$ to $\Sigma^{n}[0,1]$. We then have $[f][g]= id$, and $[h][f]=id$, and $[f]$ is then an isomorphism. 

For the other direction, suppose that $[f]$ is an isomorphism. Let $M$ be the marking on $[3]$ that includes all simplices of dimension superior or equal to $2$. As $\Sp_{[3]}\to ([3],M)$ is a weak equivalence, there is a lifting in the following diagram:
% https://q.uiver.app/?q=WzAsMyxbMCwwLCJcXFNpZ21hXm4oWzAsMV1cXGNvcHJvZF97XFx7MVxcfX1bMSwyXVxcY29wcm9kX3tcXHsyXFx9fVsyLDNdKSJdLFsyLDAsIkMiXSxbMCwxLCJcXFNpZ21hXm4oWzNdLE0pIl0sWzAsMSwiZl57LTF9XFxhbWFsZyBmXFxhbWFsZyBmXnstMX0iXSxbMCwyXSxbMiwxLCJoIiwyLHsic3R5bGUiOnsiYm9keSI6eyJuYW1lIjoiZG90dGVkIn19fV1d
\[\begin{tikzcd}
	{\Sigma^n([0,1]\coprod_{\{1\}}[1,2]\coprod_{\{2\}}[2,3])} && C \\
	{\Sigma^n([3],M)}
	\arrow["{f^{-1}\amalg f\amalg f^{-1}}", from=1-1, to=1-3]
	\arrow[from=1-1, to=2-1]
	\arrow["h"', dotted, from=2-1, to=1-3]
\end{tikzcd}\]
Now $h(\Sigma^n[0,3])$ and $h(\Sigma^n[0,2])$ are respectively compositions of $(f,f^{-1})$ and $(f^{-1},f)$. Hypotheses imply that these compositions are equivalent to identities, and so are thin. The morphism then lifts to $\Sigma^n [3]^{eq}$. The object $C$ being fibrant, $h$ lifts to $\Sigma^n [3]^\sharp$, and $f$ is then thin.
\end{proof}




\begin{lemma}
\label{lemma:homotopycategory_are_idenpendant_of}
Let $s,t$ and $s',t'$ be two pairs of parallel cells, and $\psi:\partial \Db_n\times [1]_t\to C$ a homotopy between $s\cup t:\partial \Db_n\to C$ and $s'\cup t':\partial \Db_n\to C$. Then 
$$\pi_n(s,t,C)\cong \pi_n(s',t',C)$$
\end{lemma}
\begin{proof}
For each $x:s\to t$, there exists a lifting $h_x$ in the following diagram:
% https://q.uiver.app/?q=WzAsMyxbMCwwLCJcXERiX25cXHRpbWVzXFx7MFxcfVxcY3VwXFxwYXJ0aWFsXFxEYl9uXFx0aW1lcyBbMV1fdCJdLFswLDEsIlxcRGJfblxcdGltZXMgWzFdX3QiXSxbMSwwLCJDIl0sWzAsMV0sWzAsMiwieFxcY3VwXFxwc2kiXSxbMSwyLCJoIiwyLHsic3R5bGUiOnsiYm9keSI6eyJuYW1lIjoiZG90dGVkIn19fV1d
\[\begin{tikzcd}
	{\Db_n\times\{0\}\cup\partial\Db_n\times [1]_t} & C \\
	{\Db_n\times [1]_t}
	\arrow[from=1-1, to=2-1]
	\arrow["x\cup\psi", from=1-1, to=1-2]
	\arrow["h"', dotted, from=2-1, to=1-2]
\end{tikzcd}\]
and we define $F(x)$ as the restriction of $h_x$ to $\Db_n\times \{1\}$. For a $(n+1)$-cell $f:x\to y$, there exists a lifting $h_f$ in the following diagram:
% https://q.uiver.app/?q=WzAsMyxbMCwwLCJcXERiX3tuKzF9XFx0aW1lcyBcXHswXFx9XFxjdXAgXFxwYXJ0aWFsXFxEYl97bisxfVxcdGltZXMgWzFdX3QiXSxbMSwwLCJDIl0sWzAsMSwiXFxEYl97bisxfVxcdGltZXMgWzFdX3QiXSxbMCwxLCJmXFxjdXAgaF94XFxjdXAgaF95Il0sWzAsMl0sWzIsMSwiaF9mIiwyXV0=
\[\begin{tikzcd}
	{\Db_{n+1}\times \{0\}\cup \partial\Db_{n+1}\times [1]_t} & C \\
	{\Db_{n+1}\times [1]_t}
	\arrow["{f\cup h_x\cup h_y}", from=1-1, to=1-2]
	\arrow[from=1-1, to=2-1]
	\arrow["{h_f}"', from=2-1, to=1-2]
\end{tikzcd}\]
and we define $F(f)$ as the restriction of $h_f$ to $\Db_{n+1}\times \{1\}$. Furthermore, the unicity up to homotopy of lifting implies that $[F(f)]$ is independent of the choice of the lifting, and that $f\sim g$ implies $[F(f)]=[F(g)]$.
If $g:y\to z$ is an other morphism, and $\psi:\Sigma^n[2]_t \to C$ corresponds to the composition of $f$ and $g$, 
there is a lift in the following diagram:
% https://q.uiver.app/?q=WzAsMyxbMiwwLCJDIl0sWzAsMCwiXFxTaWdtYV5uIFsyXV90XFxjdXAgKFxcU2lnbWFeblxccGFydGlhbFsyXSlcXHRpbWVzIFsxXV90Il0sWzAsMSwiXFxTaWdtYV5uIFsyXV90XFx0aW1lcyBbMV1fdCJdLFsxLDAsIiBcXHBoaSBcXGN1cCBoX2ZcXGN1cCBoX2dcXGN1cCBoX3tmXFxjaXJjIGd9Il0sWzEsMl0sWzIsMCwiIiwxLHsic3R5bGUiOnsiYm9keSI6eyJuYW1lIjoiZG90dGVkIn0sImhlYWQiOnsibmFtZSI6Im5vbmUifX19XV0=
\[\begin{tikzcd}
	{\Sigma^n [2]_t\cup (\Sigma^n\partial[2])\times [1]_t} && C \\
	{\Sigma^n [2]_t\times [1]_t}
	\arrow["{ \phi \cup h_f\cup h_g\cup h_{f\circ g}}", from=1-1, to=1-3]
	\arrow[from=1-1, to=2-1]
	\arrow[dotted, no head, from=2-1, to=1-3]
\end{tikzcd}\]
Restricted to $\Sigma^n [2]_t\times \{1\}$ this shows that $F$ commutes with compositions. We then have defined a functor 
$$F:\pi_n(s,t,C)\to \pi_n(s',t',C).$$

Using exactly the same procedure, where we just invert $0$ and $1$, we define a functor:
$$G:\pi_n(s',t',C)\to \pi_n(s,t,C).$$
Now, we have a lift in the following diagram:
% https://q.uiver.app/?q=WzAsMyxbMCwwLCJcXERiX3tufVxcdGltZXMgXFxMYW1iZGFeezJ9WzJdXlxcc2hhcnBcXGN1cFxccGFydGlhbFxcRGJfblxcdGltZXNbMl1eXFxzaGFycCJdLFswLDEsIlxcRGJfblxcdGltZXNbMl1eXFxzaGFycCJdLFszLDAsIkMiXSxbMCwyLCJoX3hcXGN1cCBoX3tGKHgpfVxcY3VwXFxwc2koaWRcXHRpbWVzIHNeMCkiXSxbMCwxXSxbMSwyLCJrX3giLDIseyJzdHlsZSI6eyJib2R5Ijp7Im5hbWUiOiJkb3R0ZWQifX19XV0=
\[\begin{tikzcd}
	{\Db_{n}\times \Lambda^{2}[2]^\sharp\cup\partial\Db_n\times[2]^\sharp} &&& C \\
	{\Db_n\times[2]^\sharp}
	\arrow["{h_x\cup h_{F(x)}\cup\psi(id\times s^0)}", from=1-1, to=1-4]
	\arrow[from=1-1, to=2-1]
	\arrow["{k_x}"', dotted, from=2-1, to=1-4]
\end{tikzcd}\]
The restriction of $k_x$ to $\Db_n\times [0,1]_t$ provides a thin cell $x\to G(F(x))$, which corresponds to an isomorphism in $\pi_n(s,t,C)$ according to proposition \ref{prop:in_the_homotopy_category_thin_is_iso}. If $f:x\to y$ is a $(n+1)$-cell, there is a lifting in the following diagram:
% https://q.uiver.app/?q=WzAsMyxbMCwwLCJcXERiX3tuKzF9XFx0aW1lcyBcXExhbWJkYV57Mn1bMl1eXFxzaGFycFxcY3VwXFxwYXJ0aWFsXFxEYl97bisxfVxcdGltZXNbMl1eXFxzaGFycCJdLFswLDEsIlxcRGJfe24rMX1cXHRpbWVzWzJdXlxcc2hhcnAiXSxbMywwLCJDIl0sWzAsMiwiaF9mXFxjdXAgaF97RihmKX1cXGN1cCBrX3hcXGN1cCBrX3kiXSxbMSwyLCJrX2YiLDIseyJzdHlsZSI6eyJib2R5Ijp7Im5hbWUiOiJkb3R0ZWQifX19XSxbMCwxXV0=
\[\begin{tikzcd}
	{\Db_{n+1}\times \Lambda^{2}[2]^\sharp\cup\partial\Db_{n+1}\times[2]^\sharp} &&& C \\
	{\Db_{n+1}\times[2]^\sharp}
	\arrow["{h_f\cup h_{F(f)}\cup k_x\cup k_y}", from=1-1, to=1-4]
	\arrow["{k_f}"', dotted, from=2-1, to=1-4]
	\arrow[from=1-1, to=2-1]
\end{tikzcd}\]
The restriction of $k_f$ to $\Db_{n+1}\times[0,1]_t$ induces in $\pi_n(s,t,C)$ a commutative diagram:
% https://q.uiver.app/?q=WzAsNCxbMCwwLCJ4Il0sWzEsMCwiR0Z4Il0sWzAsMSwieSJdLFsxLDEsIkdGeSJdLFsxLDMsIltHRmZdIl0sWzAsMiwiW2ZdIiwyXSxbMiwzXSxbMCwxXV0=
\[\begin{tikzcd}
	x & GFx \\
	y & GFy.
	\arrow["{[GFf]}", from=1-2, to=2-2]
	\arrow["{[f]}"', from=1-1, to=2-1]
	\arrow[from=2-1, to=2-2]
	\arrow[from=1-1, to=1-2]
\end{tikzcd}\]
We then have an invertible natural transformation $\psi: id\to GF$. Similarly we can construct an other natural transformation $id\to GF$, which shows the desired equivalence of categories.
\end{proof}
\p
Let $a$ be an element of  $\Hom_{ho(\mSset)}(\partial \Db_n,C)$. We define 
\begin{equation}
\label{eq:def of pi a}
\pi_n(a, C) := \pi_n(s,t,C)
\end{equation}
where $s,t$ is a pair of parallel arrows such that $s\cup t$ represents $a$.
The previous proposition shows that this is well defined.


\subsection{A criterion to be a weak equivalence}
\p
A morphism $p:C\to D$ between complicial sets is a \wcnotion{$\Db$-equivalence}{Dequivalence@$\Db$-equivalence} if 
$$\pi_0(C)\to \pi_0(D)$$
is an equivalence of categories, and for any $n>0$ and pair of parallel arrow $s,t$, the induced functor
$$\pi_n(s,t,C)\to \pi_n(ps,pt,D)$$
is an equivalence of categories. 

A \wcnotion{$\Db$-trivial fibration}{Dtrivial fibration@$\Db$-trivial fibration} is a fibration having the right lifting property against $\partial\Db_n\to \Db_n$ and $\Db_n\to (\Db_{n})_t$.


\begin{lemma}
\label{lemma:fibration_are_isofibration}
Let $\alpha\in\{-,+\}$.
The morphism $i^\alpha_{n+1}:\Db_n\to (\Db_{n+1})_t$ is an acyclic cofibration. 
\end{lemma}
\begin{proof}
We have a pushout diagram
% https://q.uiver.app/?q=WzAsNCxbMSwwLCJcXERiX25cXHRpbWVzIFxce1xcYWxwaGFcXH0iXSxbMCwwLCJcXERiX25cXHRpbWVzXFx7XFxhbHBoYVxcfVxcY3VwXFxwYXJ0aWFsXFxEYl9uXFx0aW1lcyBbMV1fdCJdLFswLDEsIlxcRGJfblxcdGltZXMgWzFdX3QiXSxbMSwxLCIoXFxEYl9uKV90Il0sWzEsMCwiaWRcXGN1cCBcXHBhcnRpYWxcXHRpbWVzIHNeMCJdLFsxLDJdLFsyLDNdLFswLDMsImleXFxhbHBoYV97bisxfSJdLFszLDQsIiIsMCx7ImxldmVsIjoxLCJzdHlsZSI6eyJuYW1lIjoiY29ybmVyIn19XV0=
\[\begin{tikzcd}
	{\Db_n\times\{\alpha\}\cup\partial\Db_n\times [1]_t} & {\Db_n\times \{\alpha\}} \\
	{\Db_n\times [1]_t} & {(\Db_n)_t}
	\arrow[""{name=0, anchor=center, inner sep=0}, "{id\cup \partial\times s^0}", from=1-1, to=1-2]
	\arrow[from=1-1, to=2-1]
	\arrow[from=2-1, to=2-2]
	\arrow["{i^\alpha_{n+1}}", from=1-2, to=2-2]
	\arrow["\lrcorner"{anchor=center, pos=0.125, rotate=180}, draw=none, from=2-2, to=0]
\end{tikzcd}\]
The left hand morphism being an acyclic cofibration, this concludes the proof.
\end{proof}

\begin{lemma}
\label{lemma:acyclic_cofibration_are_G_equivalence}
Acyclic cofibrations between complicial sets are $\Db$-equivalences.
\end{lemma}
\begin{proof}
Let $i:A\to B$ be an acyclic cofibration. The morphism $i$ admits a retraction $r:B\to A$: 
% https://q.uiver.app/?q=WzAsMyxbMCwwLCJBIl0sWzAsMSwiQiJdLFsxLDAsIkEiXSxbMCwyLCJpZCJdLFswLDEsImkiLDJdLFsxLDIsInIiLDJdXQ==
\[\begin{tikzcd}
	A & A \\
	B.
	\arrow["id", from=1-1, to=1-2]
	\arrow["i"', from=1-1, to=2-1]
	\arrow["r"', from=2-1, to=1-2]
\end{tikzcd}\]
and a homotopy $\psi$ between $id_B$ and $ir$ which is constant on the image of $i$, obtained as the lift in the following diagram:
% https://q.uiver.app/#q=WzAsMyxbMCwwLCJCXFx0aW1lc1xcezBcXH1cXGNvcHJvZCBfe0FcXHRpbWVzXFx7MFxcfX1BXFx0aW1lc1sxXV90Il0sWzAsMSwiQlxcdGltZXNbMV1fdCJdLFsxLDAsIkIiXSxbMCwxXSxbMSwyLCJcXHBoaSIsMix7InN0eWxlIjp7ImJvZHkiOnsibmFtZSI6ImRhc2hlZCJ9fX1dLFswLDJdXQ==
\[\begin{tikzcd}
	{B\times\{0\}\coprod _{A\times\{0\}}A\times[1]_t} & B \\
	{B\times[1]_t}
	\arrow[from=1-1, to=2-1]
	\arrow["\phi"', dashed, from=2-1, to=1-2]
	\arrow[from=1-1, to=1-2]
\end{tikzcd}\]
Let $n>0$ be an integer, and $s$, $t$ be two $(n-1)$-cells of $C$. The retraction implies that $i_!$ is an injection on morphisms. For any $n$-cell $y:i(s)\to i(t)$ in $B$, the homotopy $\psi$ induces a thin cell $y\to ir(y)$ which corresponds to an isomorphism in $\pi_n(is,it,B)$ according to proposition \ref{prop:in_the_homotopy_category_thin_is_iso}. The functor $i_!$ is then essentially surjective. For any $(n+1)$-cell $f:i(x)\to i(y)$, the homotopy $\psi$ induces an equivalence $[ir(f)]\sim [f]$. The morphism $i_!$ is a surjection on morphisms. All put together, $i_!$ is fully faithfull and essentially surjective, and is then an equivalence. We proceed similarly to show that $i_!:\pi_0(A)\to \pi_0(B)$ is an equivalence.
\end{proof}



\begin{lemma}
\label{lemma:2_out_of_3_for_G_equivalence}
Suppose given a commutative triangle between complicial sets
% https://q.uiver.app/?q=WzAsMyxbMCwxLCJBIl0sWzEsMCwiQiJdLFsyLDEsIkMiXSxbMCwyLCJnIiwyXSxbMSwyLCJmIl0sWzAsMSwiaSJdXQ==
\[\begin{tikzcd}
	& B \\
	A && C
	\arrow["g"', from=2-1, to=2-3]
	\arrow["f", from=1-2, to=2-3]
	\arrow["i", from=2-1, to=1-2]
\end{tikzcd}\]
If $i$ is an acyclic cofibration, and $g$ is a $\Db$-equivalence, then $f$ is a $\Db$-equivalence.
\end{lemma}
\begin{proof}
Let $s,t$ be any pair of parallel arrows in $B$. There exists a pair of parallel arrows $s',t'$ in $A$ such that $s\cup t$ and $is'\cup it'$ correspond to the same element in $[\partial\Db_n,B]$. We then have a diagram:
% https://q.uiver.app/?q=WzAsNSxbMSwwLCIgXFxwaShzLHQsQikiXSxbMSwxLCIgXFxwaShpcyxpdCxCKSJdLFsyLDAsIlxccGkoZnMsZnQsQykiXSxbMiwxLCJcXHBpKGdzLGd0LEMpIl0sWzAsMSwiXFxwaShzLHQsQikiXSxbNCwxLCJcXHNpbSJdLFsxLDNdLFsyLDMsIlxcc2ltIl0sWzAsMSwiXFxzaW0iXSxbMCwyXSxbNCwzLCJcXHNpbSIsMix7ImN1cnZlIjozfV1d
\[\begin{tikzcd}
	& { \pi(s,t,B)} & {\pi(fs,ft,C)} \\
	{\pi(s,t,B)} & { \pi(is,it,B)} & {\pi(gs,gt,C).}
	\arrow["\sim", from=2-1, to=2-2]
	\arrow[from=2-2, to=2-3]
	\arrow["\sim", from=1-3, to=2-3]
	\arrow["\sim", from=1-2, to=2-2]
	\arrow[from=1-2, to=1-3]
	\arrow["\sim"', curve={height=18pt}, from=2-1, to=2-3]
\end{tikzcd}\]
where arrows labeled by $\sim$ are isomorphisms according to lemmas \ref{lemma:homotopycategory_are_idenpendant_of} and \ref{lemma:acyclic_cofibration_are_G_equivalence}. By two out of three, this shows that $ \pi(s,t,B)\to \pi(fs,ft,C)$ is an isomorphism, and $f$ is then a $\Db$ equivalence.
\end{proof}



\begin{prop}
\label{prop:caracterisation_of_G_fibration}
Let $p:C\to D$ be a fibration between complicial sets.
The morphism $p$ is a $\Db$-trivial fibration if and only if it is a $\Db$-equivalence.
\end{prop}
\begin{proof}
If $p$ is a $\Db$-trivial fibration, it is obvious that it is a $\Db$-equivalence. For the converse, suppose $p$ is a fibration and a $\Db$-equivalence, and consider a diagram 
% https://q.uiver.app/#q=WzAsNCxbMSwwLCJDIl0sWzEsMSwiRCJdLFswLDAsIlxccGFydGlhbFxcRGJfbiJdLFswLDEsIlxcRGJfbiJdLFsyLDNdLFsyLDBdLFszLDEsIngiLDJdLFswLDEsInAiXV0=
\[\begin{tikzcd}
	{\partial\Db_n} & C \\
	{\Db_n} & D
	\arrow[from=1-1, to=2-1]
	\arrow[from=1-1, to=1-2]
	\arrow["x"', from=2-1, to=2-2]
	\arrow["p", from=1-2, to=2-2]
\end{tikzcd}\]
As $p$ is a $\Db$-equivalence this implies that there exists a cell $\overline{x}:\Db_n\to C$ together with a thin $(n+1)$-cell $y:p(\overline{x})\to y$. All this data corresponds to a diagram:
% https://q.uiver.app/#q=WzAsNCxbMSwwLCJDIl0sWzEsMSwiRCJdLFswLDAsIlxcRGJfbiJdLFswLDEsIihcXERiX3tuKzF9KV90Il0sWzAsMSwicCJdLFsyLDMsIlxcZGVsdGFeMF97bisxfSIsMl0sWzIsMCwiXFxiYXJ7eH0iXSxbMywxLCJ5IiwyXV0=
\[\begin{tikzcd}
	{\Db_n} & C \\
	{(\Db_{n+1})_t} & D
	\arrow["p", from=1-2, to=2-2]
	\arrow["{\delta^0_{n+1}}"', from=1-1, to=2-1]
	\arrow["{\bar{x}}", from=1-1, to=1-2]
	\arrow["y"', from=2-1, to=2-2]
\end{tikzcd}\]
The left hand morphism being an acyclic cofibration according to \ref{lemma:fibration_are_isofibration}, this diagram admits a lift $h:(\Db_{n+1})_t\to C$. The restriction of $h$ to $i^+_{n+1}$ provides a lift in the first diagram. Now, we consider a diagram of shape: 
% https://q.uiver.app/#q=WzAsNCxbMSwwLCJDIl0sWzEsMSwiRCJdLFswLDAsIlxcRGJfbiJdLFswLDEsIihcXERiX24pX3QiXSxbMiwzXSxbMiwwLCJnIl0sWzMsMV0sWzAsMSwicCJdXQ==
\[\begin{tikzcd}
	{\Db_n} & C \\
	{(\Db_n)_t} & D
	\arrow[from=1-1, to=2-1]
	\arrow["g", from=1-1, to=1-2]
	\arrow[from=2-1, to=2-2]
	\arrow["p", from=1-2, to=2-2]
\end{tikzcd}\]
with $n>1$.
Let $s,t$ be respectively the $(n-1)$-source and the $(n-1)$-target of $g$. Hypotheses imply that $[p(g)]$ is an isomorphism in $\pi_n(s,t,D)$ and because $p$ is a $\Db$-equivalence, so is $[g]$. According to lemma \ref{prop:in_the_homotopy_category_thin_is_iso}, this implies that $g$ is thin. There exists then a lifting in the previous diagram. The case $n=1$ is similar.
 The morphism $f$ is then a $\Db$-trivial fibration.
\end{proof}



\begin{lemma}
\label{lemma:slice_G_fibrations}
Let $p:X\to Y$ be a $\Db$-trivial fibration between complicial sets. Then for any $x\in X_0$, the induced fibrations
$$X_{/x}\to X\times_Y Y_{/p(x)} ~~\mbox{and}~~ X_{x/}\to X\times_Y Y_{p(x)/}$$
are $\Db$-trivial fibrations.
\end{lemma}
\begin{proof}
We define $\mathbb{P}(p,n)$ to be the statement that $p$ has the right lifting property against 
$$ \Db_n\cup \partial\Db_n\star [0]\to \Db_{n+1}\star[0] \mbox{ and }(\Db_n)_t\cup \Db_n\star [0]\to (\Db_{n})_t\star[0]$$
and against
$$[0]\costar \partial \Db_n\cup \Db_n\to [0]\costar \Db_{n+1} \mbox{ and }[0]\star \Db_n\cup (\Db_n)_t\to [0]\costar (\Db_n)_t$$
We then have to show that for any $n$, $\mathbb{P}(p,n)$ holds.

First, it is obvious that each $\Db$-equivalence $p$ satisfies $\mathbb{P}(p,0)$. As $p$ is a fibration, the corollaries \ref{cor:star and zigzag} and \ref{cor:costar and zigzag} then imply that $\mathbb{P}(p,n+1)$ is equivalent to $\mathbb{P}(p(a,b),n)$ for any $a,b\in X_0$, where $p(a,b)$ is the induced morphism: $X(a,b)\to Y(p(a),p(b))$. 

Using the fact that $p(a,b)$ is a $\Db$-trivial fibration as soon as $p$ is, this shows the desired result.
\end{proof}


\begin{lemma}
\label{lemma:G_fibration_right lifting property_against_partial}
 $\Db$-Trivial fibrations between complicial sets have the right lifting property against $\partial[n]\to [n]$.
\end{lemma}
\begin{proof}
Let $C$ be the class of cofibrations having the right lifting property against $\Db$-equivalences. The lemma \ref{lemma:slice_G_fibrations} implies that for any 
 $K\to L$ in $C$, the induced morphism:
$$L\cup K\star[0]\to L\star[0]$$
is in $C$. 
The class $C$ is then closed under Leibniz join. Furthermore, it includes $\partial[1]\to [1]$, and then, by induction, it includes $\partial[n]\to[n]$ for any integer $n$.
\end{proof}


\begin{lemma}
\label{lemma:G_fibration_right lifting property_against_sat}
$\Db$-Trivial fibrations between complicial sets have the right lifting property against $[n]\to [n]_t$.
\end{lemma}
\begin{proof}
Let $p$ be $\Db$-trivial fibrations between complicial sets, and
 $C_{n,p}$ be the set of objects $A$ such that $p$ has the right lifting property against:
$$A\to \tau^i_{n-1}(A).$$
This set is then closed under colimits, and by zigzags of acyclic cofibrations.
Let $k\leq n$ be two integers. We define $\mathbb{P}(k,n,p)$ to be the statement that 
$$ \Sigma [n-k]_{\circ}\star[k-1]~~~\mbox{and}~~~ [k-1]_{\circ}\costar\Sigma [n-k] $$
are in $C_{n+1,p}$.
The statement $\mathbb{P}(0,0,f)$ corresponds to the belonging of $\Db_1$ to $C_{1,p}$, which is obviously true. Suppose that $0<k$ and $\mathbb{P}(k-1,n,p)$. 
According to theorem \ref{theo:cyl_formula}, the object $\Sigma [n-k]_{\circ}\star[k-1]$ is linked by a zigzag of acyclic cofibrations to the colimit of 
$$
(\Sigma [n-k]_{\circ} \fwedge [1])\star [k-2] \leftarrow (\Sigma [n-k]_{\circ})\star [k-2] \to (\Sigma [n-k+1]_{\circ})\star [k-2]
$$
The center object and the left hand object are in $C_{n+1,p}$ because there are invariant under $\tau^i_{n}$, and the 
 right hand object is in $C_{n+1,p}$ by induction hypothesis. The object $\Sigma [n-k]_{\circ}\star[k-1]$ is then in $C_{n+1,p}$.
We demonstrate similarly that $[k-1]_{\circ}\costar\Sigma [n-k]$ is in $C_{n+1,p}$.

This then implies $\mathbb{P}(k,n,p)$. Eventually, $\mathbb{P}(0,n+1,p)$ is equivalent to $\mathbb{P}(n,n,p(a,b))$ for any pair of objects $(a,b)\in X_0$.
The statement $\mathbb{P}(k,n,p)$ is then true for any $k,n$ and $\Db$-trivial fibrations between complicial sets $p$. This implies that $p$ has the right lifting property against $[n]\to [n]_t$.
\end{proof}



\begin{theorem}
\label{theo:f_weak_equivalence_ssi_f_G_equivalence}
Let $p$ be a map between complicial sets. Then $p$ is a weak equivalence if and only if it is a $\Db$-equivalence.
\end{theorem}
\begin{proof}
According to lemmas \ref{lemma:acyclic_cofibration_are_G_equivalence} and \ref{lemma:2_out_of_3_for_G_equivalence} we can restrict ourselves to the case where $p$ is a fibration. If it is a weak equivalence, $p$ is then a trivial fibration and is then a $\Db$-equivalence. Suppose now that $p$ is a $\Db$-equivalence. According to proposition \ref{prop:caracterisation_of_G_fibration}, $p$ is then a $\Db$-trivial fibration. Lemmas \ref{lemma:G_fibration_right lifting property_against_partial} and \ref{lemma:G_fibration_right lifting property_against_sat} imply that $p$ is a trivial fibration.
\end{proof}

\begin{definition}
Let $p:X\to Y$ be a morphism between complicial sets. The morphism $p$ is \snotion{essentially surjective}{for marked simplicial sets} if for any $x\in Y_0$, there exists $\bar{x}\in X_0$ together with a thin cell $\bar{x}\to x$.
The morphism $f$ is \snotion{fully faithful}{for marked simplicial sets} if the induced morphisms: 
$$X(a,b)\to Y(pa,pb)$$
are weak equivalences for any $a,b\in X_0$.
\end{definition}



\begin{cor}
Let $p$ be a map between complicial sets. Then $p$ is a weak equivalence if and only if it is fully faithfull and essentially surjective.
\end{cor}
\begin{proof}
If $p$ is a weak equivalence, it is then fully faithfull and essentially surjective. Conversely, suppose $p$ is fully faithfull and essentially surjective. 
The morphism $\pi_0(X)\to \pi_0(Y)$ is fully faithfull and essentially surjective, and then an equivalence of category. For $(a,b)$ a pair of $0$-cells, we have equalities:
% https://q.uiver.app/?q=WzAsNCxbMCwwLCJcXHBpXzEoYSxiLFgpIl0sWzAsMSwiXFxwaV8xKHBhLHBiLFkpIl0sWzEsMCwiXFxwaV8wKFgoYSxiKSkiXSxbMSwxLCJcXHBpXzAoWShwYSxwYikpIl0sWzIsMywiXFxwaV8wcChhLGIpIl0sWzAsMSwiXFxwaV8xcCIsMl0sWzEsMywiIiwxLHsibGV2ZWwiOjIsInN0eWxlIjp7ImhlYWQiOnsibmFtZSI6Im5vbmUifX19XSxbMCwyLCIiLDAseyJsZXZlbCI6Miwic3R5bGUiOnsiaGVhZCI6eyJuYW1lIjoibm9uZSJ9fX1dXQ==
\[\begin{tikzcd}
	{\pi_1(a,b,X)} & {\pi_0(X(a,b))} \\
	{\pi_1(pa,pb,Y)} & {\pi_0(Y(pa,pb)).}
	\arrow["{\pi_0p(a,b)}", from=1-2, to=2-2]
	\arrow["{\pi_1p}"', from=1-1, to=2-1]
	\arrow[Rightarrow, no head, from=2-1, to=2-2]
	\arrow[Rightarrow, no head, from=1-1, to=1-2]
\end{tikzcd}\]
The morphism $\pi_1(a,b,p)$ is then an equivalence of categories. For $(s,t)$ a pair of parallel arrows of dimension $>1$, if we denote by $a$ and $b$ the $0$-source and the $0$-target of $s$ and $t$, we have a diagram:
% https://q.uiver.app/?q=WzAsNCxbMCwwLCJcXHBpX24ocyx0LFgpIl0sWzAsMSwiXFxwaV9uKHBhLHBiLFkpIl0sWzEsMCwiXFxwaV97bi0xfShzLHQsWChhLGIpKSJdLFsxLDEsIlxccGlfe24tMX0ocyx0LFkocGEscGIpKSJdLFsyLDMsIlxccGlfe24tMX0ocyx0ICxwKGEsYikpIl0sWzAsMSwiXFxwaV9ucCIsMl0sWzEsMywiIiwxLHsibGV2ZWwiOjIsInN0eWxlIjp7ImhlYWQiOnsibmFtZSI6Im5vbmUifX19XSxbMCwyLCIiLDAseyJsZXZlbCI6Miwic3R5bGUiOnsiaGVhZCI6eyJuYW1lIjoibm9uZSJ9fX1dXQ==
\[\begin{tikzcd}
	{\pi_n(s,t,X)} & {\pi_{n-1}(s,t,X(a,b))} \\
	{\pi_n(pa,pb,Y)} & {\pi_{n-1}(s,t,Y(pa,pb)).}
	\arrow["{\pi_{n-1}(s,t ,p(a,b))}", from=1-2, to=2-2]
	\arrow["{\pi_np}"', from=1-1, to=2-1]
	\arrow[Rightarrow, no head, from=2-1, to=2-2]
	\arrow[Rightarrow, no head, from=1-1, to=1-2]
\end{tikzcd}\]
The morphism $\pi_n(a,b,p)$ is then an equivalence of categories.
The morphism $p$ is then a $\Db$-equivalence, and according to \ref{theo:f_weak_equivalence_ssi_f_G_equivalence}, a weak equivalence.
\end{proof}

\subsection{A criterion to be a weakly invertible transformation}
\label{section:A criterion to be a weakly invertible transformation}
The purpose of this section is to show the following proposition:
\begin{prop}
\label{prop:criterimu_to_be_an_weak_equivalence}
Let $i:\mSset\to \mSset$ and $j:\mSset\to \mSset$ be two left Quillen functors and $\psi:i\to j$ a natural transformation. If 
$\psi(\Db_n):i (\Db_n) \to j (\Db_n)$ is a weak equivalence for any $n$, then $\psi(X):i(X)\to j(X)$ is a weak equivalence for any $X$.
\end{prop}
For the remaining of this section, we fix two left Quillen functors $i$, $j$ and a natural transformation $\psi:i\to j$ satisfying the previous hypothesis. We denote by $N_i$ and $N_j$ the right adjoints of $i$ and $j$.

\begin{lemma}
\label{lemma:psipartial_is_a_weak_equivalence}
Morphisms $\psi(\partial\Db_n):i(\partial\Db_n)\to j(\partial\Db_n)$ are weak equivalences. 
\end{lemma}
\begin{proof}
We proceed by induction on $n$. The case $n=0$ is trivial. Suppose then the result true at the stage $n-1$. Remark then that $\partial \Db_n$ is the colimit and the homotopy colimit of the span
$$\Db_{n-1}\leftarrow \partial\Db_{n-1}\to \Db_{n-1}$$
As $i$ and $j$ are left Quillen functors, the induction hypothesis implies that $\psi(\partial\Db_n):i(\partial\Db_n)\to j(\partial\Db_n)$ is a weak equivalence.
\end{proof}

\begin{lemma}
\label{lemma:psisat_is_a_weak_equivalence}
Morphisms $\psi((\Db_n)_t):i((\Db_n)_t)\to j((\Db_n)_t)$ are weak equivalences. 
\end{lemma}
\begin{proof}
There is a diagram:
% https://q.uiver.app/?q=WzAsNCxbMCwxLCJpXyEoXFxEYl9uKV90Il0sWzIsMSwial8hKFxcRGJfbilfdCJdLFswLDAsImlfIVxcRGJfe24tMX0iXSxbMiwwLCJqXyFcXERiX3tuLTF9Il0sWzIsMywiXFxwc2koXFxEYl9uKSJdLFsyLDAsImlfIShpXi1fbikiLDJdLFszLDEsImpfIShpXi1fbikiXSxbMCwxLCJcXHBzaSgoXFxEYl9uKV90KSIsMl0sWzIsMywiXFxzaW0iLDIseyJzdHlsZSI6eyJib2R5Ijp7Im5hbWUiOiJub25lIn0sImhlYWQiOnsibmFtZSI6Im5vbmUifX19XSxbMiwwLCJcXHNpbSIsMCx7InN0eWxlIjp7ImJvZHkiOnsibmFtZSI6Im5vbmUifSwiaGVhZCI6eyJuYW1lIjoibm9uZSJ9fX1dLFszLDEsIlxcc2ltIiwyLHsic3R5bGUiOnsiYm9keSI6eyJuYW1lIjoibm9uZSJ9LCJoZWFkIjp7Im5hbWUiOiJub25lIn19fV1d
\[\begin{tikzcd}
	{i_!\Db_{n-1}} && {j_!\Db_{n-1}} \\
	{i_!(\Db_n)_t} && {j_!(\Db_n)_t}
	\arrow["{\psi(\Db_n)}", from=1-1, to=1-3]
	\arrow["{i_!(i^-_n)}"', from=1-1, to=2-1]
	\arrow["{j_!(i^-_n)}", from=1-3, to=2-3]
	\arrow["{\psi((\Db_n)_t)}"', from=2-1, to=2-3]
	\arrow["\sim"', draw=none, from=1-1, to=1-3]
	\arrow["\sim", draw=none, from=1-1, to=2-1]
	\arrow["\sim"', draw=none, from=1-3, to=2-3]
\end{tikzcd}\]
By two out of three, this shows that $\psi((\Db_n)_t)$ is a weak equivalence.
\end{proof}




\begin{lemma}
\label{lemma:j*is_a_trivial_fibration}
For any complicial set $Y$, the canonical morphism $N_jY\to N_i Y$ is a weak equivalence.
\end{lemma}
\begin{proof}
Let $Y$ be a complicial set. For any integer $n$, we have by adjunction a bijection
$$\Hom_{ho(\mSset)}(\Db_n, N_jY)\cong \Hom_{ho(\mSset)}(\Db_n, N_iY)$$
and according to lemmas \ref{lemma:psipartial_is_a_weak_equivalence} and \ref{lemma:psisat_is_a_weak_equivalence}, we have bijections
$$\Hom_{ho(\mSset)}(\partial \Db_n, N_jY)\cong \Hom_{ho(\mSset)}(\partial\Db_n, N_iY)$$
$$\Hom_{ho(\mSset)}((\Db_n)_t, N_jY)\cong \Hom_{ho(\mSset)}((\Db_n)_t, N_iY).$$
Let $a$ be an element of $\Hom_{ho(\mSset)}(\partial \Db_n, N_jY)$. We recall that the category $\pi_n(a,N_jY)$ is defined in \ref{eq:def of pi a}. The previous equivalences implies that we have an isomorphism of category
$$\pi_n(a,N_jY)\cong \pi_n(a,N_jY).$$
which concludes the proof according to theorem \ref{theo:f_weak_equivalence_ssi_f_G_equivalence}.
\end{proof}

\begin{proof}[Proof of the proposition \ref{prop:criterimu_to_be_an_weak_equivalence}]
Let $X$ be any marked simplicial set and $Y$ a complicial set. We have equalities:
% https://q.uiver.app/#q=WzAsNCxbMCwwLCJcXEhvbV97aG8oXFxtU3NldCl9KGpfIVgsWSkiXSxbMCwxLCJcXEhvbV97aG8oXFxtU3NldCl9KGlfIVgsWSkiXSxbMSwwLCJcXEhvbV97aG8oXFxtU3NldCl9KFgsal4qWSkiXSxbMSwxLCJcXEhvbV97aG8oXFxtU3NldCl9KFgsaV4qWSkiXSxbMiwzXSxbMCwxXSxbMCwyLCIiLDEseyJsZXZlbCI6Miwic3R5bGUiOnsiaGVhZCI6eyJuYW1lIjoibm9uZSJ9fX1dLFsxLDMsIiIsMSx7ImxldmVsIjoyLCJzdHlsZSI6eyJoZWFkIjp7Im5hbWUiOiJub25lIn19fV1d
\[\begin{tikzcd}
	{\Hom_{ho(\mSset)}(j_!X,Y)} & {\Hom_{ho(\mSset)}(X,j^*Y)} \\
	{\Hom_{ho(\mSset)}(i_!X,Y)} & {\Hom_{ho(\mSset)}(X,i^*Y)}
	\arrow[from=1-2, to=2-2]
	\arrow[from=1-1, to=2-1]
	\arrow[Rightarrow, no head, from=1-1, to=1-2]
	\arrow[Rightarrow, no head, from=2-1, to=2-2]
\end{tikzcd}\]
Lemma \ref{lemma:j*is_a_trivial_fibration} implies that the right hand morphism is a bijection, and so is the left hand morphism. 
For any $X$, $\psi(X)$ is then a weak equivalence.
\end{proof}






\subsection{Weak characterization of the identity}
 For the rest of this section, we fix a left Quillen functor $i:\mSset\to \mSset$ such that there exists a zigzag of weakly invertible natural transformations:
$$i(\Db_{\uvar}) \leftrightsquigarrow \Db_{\uvar}.$$

\begin{lemma}
\label{lemma:weak characteroeiation 1}
Let $n$ be any integer, the following natural transformations are pointwise acyclic cofibrations:
$$i\tau^i_n\to \tau^i_{n}i\tau^i_n \leftarrow \tau^i_{n}i.$$
\end{lemma}
\begin{proof}
These are natural transformations between left Quillen functors. The hypothesis implies that they induce weak equivalences on globes of dimension inferior or equal to $n$. Remark that for any $k> n$, as $i_{k-1}^-:\Db_{k-1}\to (\Db_{k})_t$ is an acyclic cofibration and $\tau^i_n$ preserves them, 
$\tau^i_n\Db_{k-1}\to \tau^i_n\Db_{k}$ is an acyclic cofibration. A direct induction implies that $\Db_{n}= \tau^i_n\Db_{n}\to \tau^i_n\Db_{k}$ is an acyclic cofibration.
We then have a commutative diagram: 
% https://q.uiver.app/#q=WzAsNCxbMCwwLCJpe1xcdGF1Xmlfbn0oXFxEYl9rKSJdLFsxLDAsIntcXHRhdV5pX259aSB7XFx0YXVeaV9ufShcXERiX2spIl0sWzIsMCwie1xcdGF1Xmlfbn1pKFxcRGJfaykiXSxbMSwxLCJpKFxcRGJfe259KSJdLFszLDAsIlxcc2ltIiwyXSxbMywxLCJcXHNpbSIsMl0sWzMsMiwiXFxzaW0iXSxbMCwxXSxbMiwxXSxbMywwLCIiLDAseyJzdHlsZSI6eyJib2R5Ijp7Im5hbWUiOiJub25lIn0sImhlYWQiOnsibmFtZSI6Im5vbmUifX19XSxbMywxLCIiLDAseyJzdHlsZSI6eyJib2R5Ijp7Im5hbWUiOiJub25lIn0sImhlYWQiOnsibmFtZSI6Im5vbmUifX19XSxbMywyLCIiLDIseyJzdHlsZSI6eyJib2R5Ijp7Im5hbWUiOiJub25lIn0sImhlYWQiOnsibmFtZSI6Im5vbmUifX19XV0=
\[\begin{tikzcd}
	{i{\tau^i_n}(\Db_k)} & {{\tau^i_n}i {\tau^i_n}(\Db_k)} & {{\tau^i_n}i(\Db_k)} \\
	& {i(\Db_{n})}
	\arrow["\sim"', from=2-2, to=1-1]
	\arrow["\sim"', from=2-2, to=1-2]
	\arrow["\sim", from=2-2, to=1-3]
	\arrow[from=1-1, to=1-2]
	\arrow[from=1-3, to=1-2]
	\arrow[draw=none, from=2-2, to=1-1]
	\arrow[draw=none, from=2-2, to=1-2]
	\arrow[draw=none, from=2-2, to=1-3]
\end{tikzcd}\]
where all morphisms labelled by $\sim$ are weak equivalences.

By two out of three, this implies that theses natural transformations induce weak equivalences on all globes, and proposition \ref{prop:criterimu_to_be_an_weak_equivalence} concludes the proof.
\end{proof}

\begin{prop}
\label{prop:modification_of_the_value_on_thin_representables}
 There exists a zigzag of weakly invertible natural transformations 
$$i\leftrightsquigarrow j$$
where $j$ is a left Quillen functor such that $j([n])=i([n])$ and $j([n]_t)=\tau^i_{n-1}i([n])$, and such that the image of $[n]\to [n]_t$ by $j$ is induced by the canonical morphism $id\to \tau^i_{n-1}(id)$.
\end{prop} 
\begin{proof}
We define $\tilde{i}$ (resp. $j$) to be the colimit preserving functor defined on representables by $\tilde{i}([n]):=i([n])$ and $\tilde{i}:=([n]_t)=\tau^i_{n-1}i([n]_t)$ (resp. $j([n]):=i([n])$ and $j([n]_t):=\tau^i_{n-1}i([n])$). We then have a zigzag of natural transformations 
$$i\xrightarrow{\sim} \tilde{i}\xleftarrow{\sim} j.$$
that are pointwise acyclic cofibrations according to \ref{lemma:weak characteroeiation 1}.
This implies that both $\tilde{i}$ and $j$ are left Quillen functors.
\end{proof}


\p
In the following lemma, we use the Steiner theory recalled in section \ref{section:Steiner thery}. 
\begin{lemma}
\label{lemma:unicity_of_composition}
Let $m$ be an integer and $X$ and $Y$ be two $\zo$-categories admitting a loop free and atomic basis. We denote by $0$, $1$ and $t$ the three points of $\Sigma X\vee[1]$.
 Let $$f: \Sigma^m ([X,1]\star Y)\to \Sigma^m( ([X,1]\vee[1])\star Y)$$ be a morphism fitting in the following diagram:
% https://q.uiver.app/#q=WzAsNCxbMiwwLCJcXFNpZ21hXm0oIChbWCwxXVxcdmVlWzFdKVxcc3RhciBZKSJdLFswLDEsIlxcU2lnbWFebShbWCwxXVxcc3RhciBZKSJdLFswLDAsIlxcU2lnbWFebSgoXFx7MFxcfVxcY29wcm9kXFx7MVxcfSlcXHN0YXIgWSkiXSxbMiwxLCJcXFNpZ21hXm0oW1gsMV1cXHN0YXIgWSkiXSxbMSwwLCJmIl0sWzIsMCwiXFxTaWdtYV5tKGdcXHN0YXIgWSkiXSxbMCwzXSxbMSwzLCJpZCIsMl0sWzIsMV1d
\[\begin{tikzcd}
	{\Sigma^m((\{0\}\coprod\{1\})\star Y)} && {\Sigma^m( ([X,1]\vee[1])\star Y)} \\
	{\Sigma^m([X,1]\star Y)} && {\Sigma^m([X,1]\star Y)}
	\arrow["f", from=2-1, to=1-3]
	\arrow["{\Sigma^m(g\star Y)}", from=1-1, to=1-3]
	\arrow[from=1-3, to=2-3]
	\arrow["id"', from=2-1, to=2-3]
	\arrow[from=1-1, to=2-1]
\end{tikzcd}\]
where $g$ sends $0$ on $0$, and sends $1$ on $t$ and the right vertical morphism is the unique retraction of the inclusion $\Sigma^m([X,1]\star Y)\hookrightarrow \Sigma^m( ([X,1]\vee[1])\star Y)$.


Then $f$ is $\Sigma^m(\triangledown\star Y)$. 
\end{lemma}
\begin{proof}
All these categories admit loop free and atomic basis. We can then show this lemma in the category of augmented directed complexes. Furthermore, in this category, the suspension only makes an index shift, so we can assume without loss of generality that $m=0$.


The commutativity of the diagram implies that 
$$
\begin{array}{rclr}
f(0\star x)&=& 0\star x\\
f(1\star x)&=&t\star x\\
f([x,1] \star y)&=& [x,1] \star y +r_{x ,y} &\\
\end{array}
$$
where $r_{x,y}$ is a positive sum of elements of $(B_{[1]\star Y})_{|x|+|y|+1}$.
We show by induction on $|x|+|y|$ that: 
$$\begin{array}{rcll}
r_{x,y}&=& [1]\star y&\mbox{ if $|x|= 0$}\\
&=&0&\mbox{ if $|x|> 0$} .\\
\end{array}$$

Suppose the result true when the sum of dimensions of $x$ and $y$ is $(k-1)$. Let $x, y$ be two cells such that $|x|+|y|=k$.
\textbf{Case $|x|=0$.} The commutativity of $f$ with $\partial$ and the induction hypothesis imply that 
$$\begin{array}{rcl}
\partial r_{x,y} &=& f(\partial ([x,1]\star y)) - \partial ([x,1]\star y)\\
 &=& \{t\}\star y - \{0\}\star y + f([x,1]\star \partial y) - \{1\}\star y + \{0\}\star y - [x,1]\star \partial y\\
 &=& \{t\}\star y - \{1\}\star y + [1]\star \partial y\\
\end{array}$$
and $r_{x,y}$ is then equal to $[1]\star y$. \textbf{Case $|x|>0$.} The commutativity of $f$ with $\partial$ implies that 
$$ \partial r_{x,y} = 0$$
and $r_{x,y}$ is then equal to $0$.
\end{proof}


\p Let $C$ be the subcategory of marked simplicial sets such that $\R(iX) =\R(X)$.
We define the statement $\mathbb{P}(k,n,m)$ as the conjunction of the following assertions:
\begin{enumerate}
\item $\Sigma^m( \Sigma [n-k]_{\circ}\star[k-1])$ and $\Sigma^m( [k-1]_{\circ}\costar \Sigma [n-k])$ are in $C$.
\item For any $-1\leq l\leq k-1$ and $-1\leq p\leq n-k$, and all monomorphisms $[l]\to [k-1]$ and $[p]\to [n-k]$, the morphisms
$$\Sigma^m( \Sigma [p]_{\circ}\star[l]) \to \Sigma^m( \Sigma [n-k]_{\circ}\star[k-1])~~\mbox{and}~~
\Sigma^m( [l]_{\circ} \costar \Sigma[p])\to \Sigma^m( [k-1]_{\circ} \costar \Sigma[n-k])$$
are in $C$.
\end{enumerate}
where we set the convention $[-1]:=\emptyset$.

\begin{lemma}
\label{lemma:about_P}
If for any $k,n,m$ $\mathbb{P}(k,n,m)$ is satisfied, then $C$ includes $t\Delta$. 
\end{lemma}
\begin{proof}
Let $n$ be an integer. 
As $\Sigma [0]_\circ =[1]$, the predicate $\forall_n\mathbb{P}(n,n,0)$ implies that $C$ includes $[n]$. As $C$ is closed by zigzags of acyclic cofibrations, the proposition \ref{prop:modification_of_the_value_on_thin_representables} implies that it contains $[n]_t$ and the morphisms $[n]\to [n]_t$ for any integer $[n]$.

We now show by induction on $n$ that $C$ includes all monomorphisms $[k]\to [n]$. Suppose that the result is true at the stage $(n-1)$.
 As $\Sigma [0]_\circ =[1]$, the predicate $\forall_n\mathbb{P}(n,n,0)$ implies that $C$ contains morphisms $d^0\cup d^1:	[n-1]\cup [n-1]\to [n]$ and $d^k:[n-1]\to [n]$ for $k>1$. Remark that the two inclusions $[n-1]\to [n-1]\cup [n-1]$ fit in cocartesian squares of shape
% https://q.uiver.app/#q=WzAsNCxbMSwxLCJbbi0xXVxcY3VwIFtuLTFdIl0sWzAsMSwiW24tMV0iXSxbMSwwLCJbbi0xXSJdLFswLDAsIltuLTJdIl0sWzMsMV0sWzEsMF0sWzMsMl0sWzIsMF0sWzAsMywiIiwxLHsic3R5bGUiOnsibmFtZSI6ImNvcm5lciJ9fV1d
\[\begin{tikzcd}
	{[n-2]} & {[n-1]} \\
	{[n-1]} & {[n-1]\cup [n-1]}
	\arrow[from=1-1, to=2-1]
	\arrow[from=2-1, to=2-2]
	\arrow[from=1-1, to=1-2]
	\arrow[from=1-2, to=2-2]
	\arrow["\lrcorner"{anchor=center, pos=0.125, rotate=180}, draw=none, from=2-2, to=1-1]
\end{tikzcd}\]
where all morphisms are inclusions. As $C$ is closed by colimits, the induction hypothesis implies they are in $C$. As $C$ is closed by composition, it includes $d^0:[n-1]\to [n]$ and $d^1:[n-1]\to [n]$. Any monomorphism $[n-1]\to [n]$ then belongs to $[n]$.
As every monomorphisms $[k]\to [n]$ factors as two monomorphisms $[k]\to [n-1]\to [n]$, they are included in $C$. This concludes this induction.

We still have to show that $C$ contains the degeneracy, and for this, we proceed by induction. 
We then suppose that for any $k<n$, any degeneracy $[k+1]\to [k]$ is in $C$. As $C$ contains monomorphisms of $\Delta$, it contains any morphism $[k]\to [n]$ with $k\leq n$. 
 Let $j:[n+1]\to [n]$ be a degeneracy. We have a \textit{a priori} non commutative diagram:
% https://q.uiver.app/#q=WzAsNixbMCwxLCJ7XFxSfShpW24rMV0pIl0sWzAsMiwie1xcUn0oW24rMV0pIl0sWzEsMSwie1xcUn0oW24rMV0pIl0sWzEsMiwie1xcUn0oW25dKSJdLFswLDAsIntcXFJ9KGlcXHBhcnRpYWwgW24rMV0pIl0sWzEsMCwie1xcUn0oXFxwYXJ0aWFsW24rMV0pIl0sWzAsMV0sWzAsMiwiIiwyLHsibGV2ZWwiOjIsInN0eWxlIjp7ImhlYWQiOnsibmFtZSI6Im5vbmUifX19XSxbNCwwXSxbNSwyXSxbMiwzXSxbMSwzLCIiLDEseyJsZXZlbCI6Miwic3R5bGUiOnsiaGVhZCI6eyJuYW1lIjoibm9uZSJ9fX1dLFs0LDUsIiIsMCx7ImxldmVsIjoyLCJzdHlsZSI6eyJoZWFkIjp7Im5hbWUiOiJub25lIn19fV1d
\[\begin{tikzcd}
	{{\R}(i\partial [n+1])} & {{\R}(\partial[n+1])} \\
	{{\R}(i[n+1])} & {{\R}([n+1])} \\
	{{\R}([n+1])} & {{\R}([n])}
	\arrow[from=2-1, to=3-1]
	\arrow[Rightarrow, no head, from=2-1, to=2-2]
	\arrow[from=1-1, to=2-1]
	\arrow[from=1-2, to=2-2]
	\arrow[from=2-2, to=3-2]
	\arrow[Rightarrow, no head, from=3-1, to=3-2]
	\arrow[Rightarrow, no head, from=1-1, to=1-2]
\end{tikzcd}\]
As $C$ is closed under colimits, the induction hypothesis implies that the outer and the upper square commute. For the lower diagram to commutes, we have to check that the top cell of $\R([n+1])$ is sent on the same element on ${\R}([n])$. That is the case because the two paths send it to an unity. 
\end{proof}




\begin{prop}
\label{prop:existence_of_comparaison_with_street}
We have an equality
 $Ri = R $.
\end{prop} 
\begin{proof}
The sub category $C$ is closed under colimits and by zigzags of acyclic cofibrations. According to lemma \ref{lemma:about_P}, it is enough to show $\mathbb{P}(k,n,m)$ for any $k,n,m$. We will proceed by induction.
The property $\forall_m\mathbb{P}(0,1,m)$ corresponds to the belonging of globes to $C$, which is true.

The property $\forall_m\mathbb{P}(0,n,m)$ is equivalent to $\forall_m\mathbb{P}(n-1,n-1,m+1)$. Suppose $\forall_m\mathbb{P}(k-1,n,m)$ and $\forall_m\mathbb{P}(k-1,n-1,m)$ for an integer $k$ strictly superior to $0$ and inferior or equal to $n$. We are willing to show $\forall_m\mathbb{P}(k,n,m)$.


For any integer $l$, the $\zo$-categories $\R([l]_{\circ})$ and $\R([l])$ admit a loop free and atomic basis as $\zo$-categories having this property are stable by $1\costar \uvar$ and as $\R$ commutes with these operations. 
As $C$ is closed under colimits, the induction hypothesis implies that $C$ includes the canonical morphism
$$ \Sigma^m ( (\Sigma[n-k]_{\circ})\star [k-2])\to \Sigma^m ( (\Sigma[n-k]_{\circ}\coprod_{[0]}[1])\star [k-2]).$$
As the codomain is weakly equivalent to $\Sigma^m ( (\Sigma[n-k]_{\circ}\fwedge[1])\star [k-2])$, this implies that $C$ includes
\begin{equation}
\label{eq:eqlemmaunicity_of_composition1}
\Sigma^m ( (\Sigma[n-k]_{\circ})\star [k-2]) \hookrightarrow \Sigma^m ( (\Sigma[n-k]_{\circ}\fwedge[1])\star [k-2]).
\end{equation}
Moreover the image by $\R$ of the canonical morphism 
$$\Sigma^m ( (\Sigma[n-k]_{\circ}\fwedge[1])\star [k-2])\to \Sigma^m ( (\Sigma[n-k]_{\circ})\star [k-2]) $$
is the unique retraction of the image by $\R$ of the morphism \eqref{eq:eqlemmaunicity_of_composition1}, and then belongs to $C$. 
The lemma \ref{lemma:unicity_of_composition} then implies that the morphism 
\begin{equation}
\label{eq:eqlemmaunicity_of_composition2}
\Sigma^m ( \triangledown\star [k-2]):\Sigma^m ( (\Sigma[n-k]_{\circ})\star [k-2])\to \Sigma^m ( (\Sigma[n-k]_{\circ}\fwedge[1])\star [k-2])
\end{equation}
is in $C$. 
We will use freely in the rest of the proof that morphisms \eqref{eq:eqlemmaunicity_of_composition1} and \eqref{eq:eqlemmaunicity_of_composition2} are in $C$.
 
Theorem \ref{theo:cyl_formula} implies that the object $\Sigma^m( \Sigma [n-k]_{\circ}\star[k-1])$ is linked by a zigzag of acyclic cofibrations to the colimit of 
$$
\Sigma^m ( (\Sigma[n-k]_{\circ}\fwedge[1])\star [k-2]) \leftarrow
\Sigma^m ( \Sigma[n-k]_{\circ}\star [k-2]) \to
\Sigma^m ( \Sigma[n-k+1]_{\circ}\star [k-2])
$$
and is then in $C$.
We proceed similarly to show that $\Sigma^m( [k-1]_{\circ}\costar \Sigma [n-k])$ belongs to $C$. 


Let $-1\leq l\leq k-1$ and $-1\leq p\leq n-k$ be two integers, and $f:[l]\to [k-1]$ and $g:[p]\to [n-k]$ be two monomorphisms. Suppose first that $f$ is of shape $[0]\star f'$ for $f':[l-1]\to [k-2]$.
In this case, $\Sigma^m( \Sigma [p]_{\circ}\star[l]) \to \Sigma^m( \Sigma [n-k]_{\circ}\star[k-1])$ is the vertical colimit of the diagram
% https://q.uiver.app/?q=WzAsNixbMSwxLCIgXFxTaWdtYV5tICggXFxTaWdtYVtuLWtdX3tcXGNpcmN9XFxzdGFyIFtrLTJdKSJdLFsxLDIsIiBcXFNpZ21hXm0gKCBcXFNpZ21hW24taysxXV97XFxjaXJjfVxcc3RhciBbay0yXSkiXSxbMSwwLCJcXFNpZ21hXm0gKCAoXFxTaWdtYVtuLWtdX3tcXGNpcmN9XFxmd2VkZ2VbMV0pXFxzdGFyIFtrLTJdKSJdLFswLDEsIiBcXFNpZ21hXm0gKCBcXFNpZ21hW3BdX3tcXGNpcmN9XFxzdGFyIFtsLTFdKSJdLFswLDAsIiBcXFNpZ21hXm0gKCAoXFxTaWdtYVtwXV97XFxjaXJjfVxcZndlZGdlWzFdKVxcc3RhciBbbC0xXSkiXSxbMCwyLCIgXFxTaWdtYV5tICggXFxTaWdtYVtwKzFdX3tcXGNpcmN9XFxzdGFyIFtsLTFdKSJdLFswLDJdLFswLDFdLFszLDBdLFs0LDJdLFszLDRdLFszLDVdLFs1LDFdXQ==
\[\begin{tikzcd}
	{ \Sigma^m ( (\Sigma[p]_{\circ}\fwedge[1])\star [l-1])} & {\Sigma^m ( (\Sigma[n-k]_{\circ}\fwedge[1])\star [k-2])} \\
	{ \Sigma^m ( \Sigma[p]_{\circ}\star [l-1])} & { \Sigma^m ( \Sigma[n-k]_{\circ}\star [k-2])} \\
	{ \Sigma^m ( \Sigma[p+1]_{\circ}\star [l-1])} & { \Sigma^m ( \Sigma[n-k+1]_{\circ}\star [k-2])}
	\arrow[from=2-2, to=1-2]
	\arrow[from=2-2, to=3-2]
	\arrow[from=2-1, to=2-2]
	\arrow[from=1-1, to=1-2]
	\arrow[from=2-1, to=1-1]
	\arrow[from=2-1, to=3-1]
	\arrow[from=3-1, to=3-2]
\end{tikzcd}\]
and the induction hypothesis imply that it belongs to $C$. 
Suppose now that $f'$ avoids the initial object of $[k-1]$. In this case, the morphism $\Sigma^m( \Sigma [p]_{\circ}\star[l]) \to \Sigma^m( \Sigma [n-k]_{\circ}\star[k-1])$
 is the vertical colimit of the diagram
 % https://q.uiver.app/?q=WzAsNSxbMiwxLCIgXFxTaWdtYV5tICggXFxTaWdtYVtuLWtdX3tcXGNpcmN9XFxzdGFyIFtrLTJdKSJdLFsyLDIsIiBcXFNpZ21hXm0gKCBcXFNpZ21hW24taysxXV97XFxjaXJjfVxcc3RhciBbay0yXSkiXSxbMiwwLCJcXFNpZ21hXm0gKCAoXFxTaWdtYVtuLWtdX3tcXGNpcmN9XFxmd2VkZ2VbMV0pXFxzdGFyIFtrLTJdKSJdLFsxLDAsIlxcU2lnbWFebSAoIChcXFNpZ21hW24ta11fe1xcY2lyY30pXFxzdGFyIFtrLTJdKSJdLFswLDAsIlxcU2lnbWFebSggXFxTaWdtYSBbcF1fe1xcY2lyY31cXHN0YXJbbF0pIl0sWzAsMl0sWzAsMV0sWzMsMiwiIiwyLHsic3R5bGUiOnsidGFpbCI6eyJuYW1lIjoiaG9vayIsInNpZGUiOiJ0b3AifX19XSxbNCwzXV0=
\[\begin{tikzcd}
	{\Sigma^m( \Sigma [p]_{\circ}\star[l])} & {\Sigma^m ( (\Sigma[n-k]_{\circ})\star [k-2])} & {\Sigma^m ( (\Sigma[n-k]_{\circ}\fwedge[1])\star [k-2])} \\
	&& { \Sigma^m ( \Sigma[n-k]_{\circ}\star [k-2])} \\
	&& { \Sigma^m ( \Sigma[n-k+1]_{\circ}\star [k-2])}
	\arrow[from=2-3, to=1-3]
	\arrow[from=2-3, to=3-3]
	\arrow[hook, from=1-2, to=1-3]
	\arrow[from=1-1, to=1-2]
\end{tikzcd}\]
and the induction hypothesis imply that it belongs to $C$.
We prove similarly that $\Sigma^m( [l]_{\circ} \costar \Sigma[p])\to \Sigma^m( [k-1]_{\circ} \costar \Sigma[n-k])$ belongs to $C$.

We have then proven $\mathbb{P}(k,n,m)$, and this concludes the proof.
\end{proof}






\begin{theorem}
\label{theo:criterion_to_be_linked_to_identity}
Let $i: \mSset\to \mSset$ be a left Quillen functor. Suppose that there exists a zigzag of weakly invertible natural transformations:
$$i(\Db_{\uvar}) \leftrightsquigarrow \Db_{\uvar}.$$
Then, there exists a zigzag of weakly invertible natural transformations between $i$ and $id$. In particular, $i$ is a left Quillen equivalence.
\end{theorem} 
\begin{proof} 
The proposition \ref{prop:existence_of_comparaison_with_street} implies that we have a natural transformation $\psi:i\to i_{str}$. 
Furthermore, hypotheses imply that this natural transformation is a weak equivalence on globes. According to proposition \ref{prop:criterimu_to_be_an_weak_equivalence}, $\psi$ is then a weakly invertible natural transformation.
We then have a zigzag of weakly invertible natural transformations: 
$$i\xrightarrow{\sim} i_{str}\xleftarrow{\sim} id.$$
\end{proof}

\begin{cor}
\label{cor:criterion_to_be_linked_to_identity_case stratified}
Let $i: \stratSset\to \stratSset$ be a left Quillen functor. Suppose that there exists a zigzag of weakly invertible natural transformations:
$$i(\Db_{\uvar}) \leftrightsquigarrow \Db_{\uvar}.$$
Then, there exists a zigzag of weakly invertible natural transformations between $i$ and $id$. In particular, $i$ is a left Quillen equivalence.
\end{cor} 
\begin{proof} 
We recall that the adjunction between stratified and marked simplicial sets is denoted by:
% https://q.uiver.app/#q=WzAsMixbMSwwLCJcXG1Tc2V0OlxcaW90YSJdLFswLDAsIihcXHV2YXIpX3tcXG1rfTpcXHN0cmF0U3NldCJdLFsxLDAsIiIsMCx7Im9mZnNldCI6LTJ9XSxbMCwxLCIiLDAseyJvZmZzZXQiOi0yfV0sWzIsMywiIiwwLHsibGV2ZWwiOjEsInN0eWxlIjp7Im5hbWUiOiJhZGp1bmN0aW9uIn19XV0=
\[\begin{tikzcd}
	{(\uvar)_{\mk}:\stratSset} & {\mSset:\iota}
	\arrow[""{name=0, anchor=center, inner sep=0}, shift left=2, from=1-1, to=1-2]
	\arrow[""{name=1, anchor=center, inner sep=0}, shift left=2, from=1-2, to=1-1]
	\arrow["\dashv"{anchor=center, rotate=-90}, draw=none, from=0, to=1]
\end{tikzcd}\]
The proposition \ref{prop:transfert from presheaves on tB to stratified presheaves} states that this adjonction is a Quillen equivalence and that the functor $\iota$ preserves acyclic cofibrations. 


Remark now that the functor $(\uvar)_{\mk}\circ i\circ \iota:\mSset\to \mSset$ verifies the hypothesis of theorem \ref{theo:criterion_to_be_linked_to_identity} and we then have a zigzag of of weakly invertible natural transformations:
$$(\uvar)_{\mk}\circ i\circ \iota  \leftrightsquigarrow id$$
This induces a zigzag of of weakly invertible natural transformations:
$$i\to \iota\circ(\uvar)_{\mk}\circ i\circ \iota \circ (\uvar)_{\mk} \leftrightsquigarrow \iota\circ(\uvar)_{\mk}\leftarrow id$$
\end{proof}

%\bibliography{../../header/biblio}{}
%\bibliographystyle{alpha}
%\end{document}
%%
%
%
%


