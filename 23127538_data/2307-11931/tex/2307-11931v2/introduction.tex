%\documentclass[12pt]{book}
%\usepackage{tikz}
\usepackage{xcolor,xspace}
\usepackage{url}
\usepackage{epsfig,graphicx,endnotes,kotex,subfigure,multirow,amsmath,algorithm,algpseudocode}
\newcommand\StateX{\Statex\hspace{\algorithmicindent}}%
%\usepackage{breakurl}
%\usepackage[sort,space]{cite}
\usepackage{balance}
%\usepackage{tabularx}
\usepackage{enumitem}
\usepackage{flushend}
\usepackage[T1]{fontenc}
\usepackage{color,soul}
\hyphenation{op-tical net-works semi-conduc-tor}
%\usepackage{filecontents}
%\usepackage{booktabs} % For formal tables
\usepackage{amsthm}
\newtheorem{theorem}{Theorem}
\newtheorem{corollary}{Corollary}
\newtheorem{lemma}{Lemma}
\renewcommand{\qedsymbol}{\rule{0.7em}{0.7em}}

%	\newcommand\notion[1]{\textit{#1}\index[notion]{#1}}
\newcommand\wcnotion[2]{\textit{#1}\index[notion]{#2}}
\newcommand\wcnotionsym[3]{\textit{#1}\index[notation]{#2}\index[notion]{#3}}
\newcommand\wcsnotion[3]{\textit{#1}\index[notion]{#2!\textit{#3}}}
\newcommand\snotion[2]{\textit{#1}\index[notion]{#1!\textit{#2}}}
\newcommand\snotionsym[3]{\textit{#1}\index[notion]{#1!\textit{#3}}\index[notation]{#2!\textit{#3}}}
\newcommand\wcsnotionsym[4]{\textit{#1}\index[notation]{#2!\textit{#4}}\index[notion]{#3!\textit{#4}}}

\newcommand\wcnotation[2]{\textit{#1}\index[notation]{#2}}
\newcommand\wcsnotation[3]{\textit{#1}\index[notation]{#2!\textit{#3}}}

\newcommand\sym[1]{\index[notation]{#1}}
\newcommand\ssym[2]{\index[notation]{#1!\textit{#2}}}

\newcommand{\exclam}{!}





\newcommand{\Ab}{\mathbb{A}} 
\newcommand{\Zb}{\mathbb{Z}} 
\newcommand{\Eb}{\mathbb{E}} 
\newcommand{\Nb}{\mathbb{N}}
\newcommand{\Tb}{\mathbf{T}} 
\newcommand{\Yb}{\mathbb{Y}} 
\newcommand{\Ib}{\mathbb{I}} 
\newcommand{\Ob}{\mathbb{O}} 
\newcommand{\Pb}{\mathbb{P}} 
\newcommand{\Qb}{\mathbb{Q}} 
\newcommand{\Sb}{\mathbb{S}} 
\newcommand{\Hb}{\mathbb{H}} 
\newcommand{\Jb}{\mathbf{J}} 
\newcommand{\Kb}{\mathbb{K}} 
\newcommand{\Mb}{\mathbb{M}} 
\newcommand{\Wb}{\mathbf{W}} 
\newcommand{\Xb}{\mathbb{X}} 
\newcommand{\Cb}{\mathbf{C}}
\newcommand{\Vb}{\mathbb{V}}
\newcommand{\Bb}{\mathbb{B}}


\newcommand{\Acal}{\mathcal{A}} 
\newcommand{\Zcal}{\mathcal{Z}} 
\newcommand{\Ecal}{\mathcal{E}} 
\newcommand{\Rcal}{\mathcal{R}} 
\newcommand{\Tcal}{\mathcal{T}} 
\newcommand{\Ycal}{\mathcal{Y}} 
\newcommand{\Ucal}{\mathcal{U}} 
\newcommand{\Ical}{\mathcal{I}} 
\newcommand{\Ocal}{\mathcal{O}} 
\newcommand{\Pcal}{\mathcal{P}} 
\newcommand{\Qcal}{\mathcal{Q}} 
\newcommand{\Scal}{\mathcal{S}} 
\newcommand{\Dcal}{\mathcal{D}} 
\newcommand{\Fcal}{\mathcal{F}} 
\newcommand{\Gcal}{\mathcal{G}} 
\newcommand{\Hcal}{\mathcal{H}} 
\newcommand{\Jcal}{\mathcal{J}} 
\newcommand{\Kcal}{\mathcal{K}} 
\newcommand{\Lcal}{\mathcal{L}} 
\newcommand{\Mcal}{\mathcal{M}} 
\newcommand{\Wcal}{\mathcal{W}} 
\newcommand{\Xcal}{\mathcal{X}} 
\newcommand{\Ccal}{\mathcal{C}} 
\newcommand{\Vcal}{\mathcal{V}} 
\newcommand{\Bcal}{\mathcal{B}} 
\newcommand{\Ncal}{\mathcal{N}} 


\newcommand{\Ago}{\mathfrak{A}} 
\newcommand{\Zgo}{\mathfrak{Z}} 
\newcommand{\Ego}{\mathfrak{E}} 
\newcommand{\Rgo}{\mathfrak{R}} 
\newcommand{\Tgo}{\mathfrak{T}} 
\newcommand{\Ygo}{\mathfrak{Y}} 
\newcommand{\Ugo}{\mathfrak{U}} 
\newcommand{\Igo}{\mathfrak{I}} 
\newcommand{\Ogo}{\mathfrak{O}} 
\newcommand{\Pgo}{\mathfrak{P}} 
\newcommand{\Qgo}{\mathfrak{Q}} 
\newcommand{\Sgo}{\mathfrak{S}} 
\newcommand{\Dgo}{\mathfrak{D}} 
\newcommand{\Fgo}{\mathfrak{F}} 
\newcommand{\Ggo}{\mathfrak{G}} 
\newcommand{\Hgo}{\mathfrak{H}} 
\newcommand{\Jgo}{\mathfrak{J}} 
\newcommand{\Kgo}{\mathfrak{K}} 
\newcommand{\Lgo}{\mathfrak{L}} 
\newcommand{\Mgo}{\mathfrak{M}} 
\newcommand{\Wgo}{\mathfrak{W}} 
\newcommand{\Xgo}{\mathfrak{X}} 
\newcommand{\Cgo}{\mathfrak{C}} 
\newcommand{\Vgo}{\mathfrak{V}} 
\newcommand{\Bgo}{\mathfrak{B}} 
\newcommand{\Ngo}{\mathfrak{N}}



\newcommand{\sslash}{\mathbin{/\mkern-6mu/}}

\newcommand{\note}[1]{{\color{red}#1}}

\def\-{\raisebox{.75pt}{-}}


\newcommand{\uvar}{\_}


%basic notation
\newcommand{\id}{\text{Id}}
\newcommand{\Db}{\mathbf{D}} 
\DeclareMathOperator*{\dom}{dom}
\DeclareMathOperator*{\codom}{codom}
\DeclareMathOperator{\tw}{tw}


%derived notation
\newcommand{\Rb}{\mathbf{R}} 
\newcommand{\Lb}{\mathbf{L}} 
\newcommand{\Fb}{\mathbf{F}} 
\DeclareMathOperator{\Gb}{G} 
  
%ambiguous notation 
\DeclareMathOperator{\N}{N}
\DeclareMathOperator{\T}{T}
\DeclareMathOperator{\J}{J}


%set of maps
\DeclareMathOperator*{\W}{W}
\DeclareMathOperator*{\Wm}{tW}
\DeclareMathOperator*{\Wseg}{W_{Seg}}
\DeclareMathOperator*{\Wsat}{W_{Sat}}

\DeclareMathOperator*{\M}{M}
\DeclareMathOperator*{\Mm}{tM}
\DeclareMathOperator*{\Mseg}{M_{Seg}}
\DeclareMathOperator*{\Msat}{M_{Sat}}

\DeclareMathOperator*{\I}{I}
\DeclareMathOperator*{\F}{F}

%augmented directed complexes
\DeclareMathOperator*{\CDA}{ADC}
\DeclareMathOperator*{\CDAB}{ADC_B}

%categories
\newcommand\omegacat{\omega\mbox{-$\cat$}}
\DeclareMathOperator\Set{Set}
\DeclareMathOperator\Sp{Sp}

%infini groupoids
\DeclareMathOperator*{\Sq}{Sq}
\DeclareMathOperator*{\Li}{Li}
\DeclareMathOperator{\Hom}{Hom}


%infini 1 categories
\DeclareMathOperator*{\Lfib}{LFib}
\DeclareMathOperator*{\Rfib}{RFib}

\DeclareMathOperator*{\LCartoperator}{LCart}
\DeclareMathOperator*{\core}{core}
\newcommand{\LCart}{\mbox{$\LCartoperator$}}

\newcommand{\LCartc}{\mbox{$\LCartoperator$}^c}
\DeclareMathOperator*{\RCart}{RCart}
\DeclareMathOperator*{\RCartc}{RCart^c}




%infini omega categories
\newcommand{\uLCart}{\underline{\LCartoperator}}
\newcommand{\uLCartc}{\underline{\LCartoperator}^c}
\newcommand{\uRCart}{\underline{RCart}}
\newcommand{\uRCartc}{\underline{RCart}^c}

\DeclareMathOperator{\uHom}{\underline{Hom}}
\DeclareMathOperator{\gHom}{\underline{Hom}_{\ominus}}
\DeclareMathOperator{\Map}{Map}
\DeclareMathOperator{\im}{Im}

\newcommand{\uni}{\underline{\omega}}
\newcommand\w[1]{\widehat{#1}}

%functors
\DeclareMathOperator*{\ev}{ev}
\DeclareMathOperator*{\Arr}{Arr}
\newcommand{\Noiun}{\N_{\tiny{(\omega,1)}}}


\newcommand{\colim}{\operatornamewithlimits{colim}}
\newcommand{\laxcolim}{\operatornamewithlimits{laxcolim}}
\newcommand{\laxlim}{\operatornamewithlimits{laxlim}}


%prefixes
\DeclareMathOperator{\Lan}{Lan}
\DeclareMathOperator{\Ran}{Ran}
\newcommand\iun{(\infty,1)}
\newcommand\io{(\infty,\omega)}
\newcommand\ioun{(\infty,\omega,1)}
\newcommand\zoun{(0,\omega,1)}
\newcommand\zo{(0,\omega)}

%leibnitz products
\DeclareMathOperator{\hstar}{\hat{\star}}
\DeclareMathOperator{\htimes}{\hat{\times}}
\DeclareMathOperator{\hotimes}{\hat{\otimes}}


%Gray operations
\DeclareMathOperator{\costarindex}{f}
\newcommand{\costar}{\mathbin{\overset{co}{\star}}}
\newcommand{\fwedge}{\mathbin{\rotatebox[origin=c]{270}{$\gtrdot$}}}


%inclassable
\newcommand{\invamalg}{\mathbin{\rotatebox[origin=c]{180}{$\amalg$}}}
\DeclareMathOperator{\botimes}{\bar{\otimes}}
\DeclareMathOperator\cst{cst}
\DeclareMathOperator\Operatormark{mk}
\newcommand{\mk}{\Operatormark}

%category theory
\DeclareMathOperator\Fun{Fun}
\DeclareMathOperator\Nat{Nat}
\DeclareMathOperator\End{End}



%fundamental notation
\DeclareMathOperator\mcat{cat_m}
\DeclareMathOperator\cat{cat}
\DeclareMathOperator\grd{grd}
\DeclareMathOperator\R{R}

\newcommand\ocat{(\infty,\omega)\mbox{-$\cat$}}
\newcommand\ouncat{(\infty,\omega,1)\mbox{-$\cat$}}
\newcommand\ocatm{{(\infty,\omega)\mbox{-$\mcat$}}}
\newcommand\zocatm{(0,\omega)\mbox{-$\mcat$}}
\newcommand\zocat{(0,\omega)\mbox{-$\cat$}}
\DeclareMathOperator\zocatB{\zocat_B}
\newcommand\icat{(\infty,1)\mbox{-$\cat$}}
\newcommand\qcat{\mbox{Q$\cat$}}
\newcommand\ncat[1]{(\infty, #1)\mbox{-$\cat$}}
\newcommand\zncat[1]{(0, #1)\mbox{-$\cat$}}
\newcommand\igrd{\infty\mbox{-$\grd$}}



\DeclareMathOperator{\OperatorinfiniPsh}{Psh^\infty}
\DeclareMathOperator{\OperatorinfinitPsh}{tPsh^\infty}
\DeclareMathOperator{\OperatorPsh}{Psh}
\DeclareMathOperator{\OperatormPsh}{mPsh}
\DeclareMathOperator{\OperatortPsh}{tPsh}
\newcommand\iPsh[1]{\OperatorinfiniPsh({#1})}
\newcommand\tiPsh[1]{\OperatorinfinitPsh({#1})}
\newcommand\Psh[1]{\OperatorPsh({#1})}
\newcommand\ssetPsh[1]{\OperatorPsh_\Delta({#1})}
\newcommand\tPsh[1]{\OperatortPsh({#1})}
\newcommand\tPshM[1]{{\OperatortPsh}_M({#1})}
\newcommand\mPsh[1]{\OperatormPsh({#1})}
\newcommand\mPshM[1]{{\OperatormPsh}_M({#1})}

%segal stuff
\DeclareMathOperator{\OperatorSeg}{Seg}
\DeclareMathOperator{\OperatortSeg}{tSeg}
\DeclareMathOperator{\OperatormSeg}{mSeg}
\newcommand\Seg{\OperatorSeg}
\newcommand\mSeg{\OperatormSeg}
\newcommand\stratSeg{\OperatortSeg}

%simplicial variations
\DeclareMathOperator{\Sset}{\Psh{\Delta}}
\newcommand{\mSset}{\mPsh{\Delta}}
\newcommand{\stratSset}{\tPsh{\Delta}}


%univers
\DeclareMathOperator{\U}{\mathbf{U}}
\DeclareMathOperator{\V}{\mathbf{V}}
\DeclareMathOperator{\Wcard}{\mathbf{W}}
\DeclareMathOperator{\Z}{\mathbf{Z}}



%Grothendieck constructions
\newcommand{\ringpartial}{\mathring{\partial}}
%
%\usepackage[inline]{showlabels}
%
%\usepackage{fancyhdr}
%\usepackage{titlesec}
%\usepackage{textcase}
%
%\pagestyle{fancy}
%\fancyhf{}
%\fancyhead[RO]{\rmfamily\nouppercase{\rightmark}}
%\fancyhead[LE]{\rmfamily\nouppercase{\leftmark}}
%\fancyfoot[C]{\thepage}
%
%
%\title{\Huge{Theory and models of $(\infty,\omega)$-categories}}
%\author{Félix Loubaton}
%\date{}
%\linespread{1.2}
%\geometry{a4paper,top=3cm,bottom=4cm,left=1.5cm,right=3cm, heightrounded,bindingoffset=5mm}	
%\fancyhfoffset[RO,LE]{0.5cm}
%\fancyhfoffset[LE,RO]{0.5cm}
%
%\begin{document}
%\tableofcontents





The theory of $(\infty,1)$-categories is now a prolific field of research with applications in various domains. The past years have also witnessed the rise of $(\infty,2)$-categories. We will provide two reasons motivating the study of $(\infty,2)$-categories.

A first motivation comes from their applications in other domains. We think in particular of the work of Gaitsgory and Rozenblyum (\cite{Gaitsgory_A_study_on_DAG}) in derived algebraic geometry, where $(\infty,2)$-categories are an essential tool for encoding the six functor formalism.


A second motivation for considering $(\infty,2)$-categories arises from the theory of $\iun$-categories itself. Just as $1$-categories organize into a $2$-category, $\iun$-categories organize into an $(\infty,2)$-category. Working with this richer structure provides a powerful framework for developing formal category theory, as performed in \cite{Gray_Formal_category_theory} for the strict case and \cite{Riehl_element_of_infini_categories} for $\iun$-categories.


However, there is no reason to stop at dimension $2$. Let us once again mention two reasons for exploring $(\infty,n)$-categories for $n\in \Nb\cup\{\omega\}$.

Firstly, $(\infty,n)$-categories are already being used in other research fields, such as topological quantum field theory, where this notion is essential to the formalization and proof of the cobordism hypothesis (\cite{Baez_Higher-dimensional_algebra_and_topological_quantum_field_theory}, \cite{lurie_on_the_classification_of_topological_field_theories}, \cite{Grady_the_geometric_cobordism_hypothesis}, \cite{Calaque_a_note_on_the_category_of_cobordism}).

Secondly, even to understand the theory of $(\infty,n)$-categories, it is useful to manipulate $(\infty,k)$-categories for $k \geq n$. A first example is given by the fact that $(\infty,n)$-categories organize into an $(\infty,n+1)$-category, and this richer structure plays an important role in the theory of $(\infty,n)$-categories. For instance, the Grothendieck construction, which is always essential when working with any flavor of categories, is a lax colimit in the ambient $(\infty,n+1)$-category of $(\infty,n)$-categories. A second example arises from the Gray tensor product, which is a fundamental operation that arises when $n>1$. This operation is necessary to encode the notion of lax transformation, which leads to the concepts of lax colimits and limits. It is also worth noticing that it plays a crucial role in \cite{Gaitsgory_A_study_on_DAG}.
\begin{example*}[examples of some Gray tensor products]
We denote by $\Db_1$ the $1$-category generated by the $1$-graph
% https://q.uiver.app/#q=WzAsMixbMCwwLCIwIl0sWzEsMCwiMSJdLFswLDFdXQ==
\[\begin{tikzcd}
	0 & 1
	\arrow[from=1-1, to=1-2]
\end{tikzcd}\]
and  by $\Db_2$ the $2$-category generated by the $2$-graph
% https://q.uiver.app/#q=WzAsMixbMCwwLCIwIl0sWzEsMCwiMSJdLFswLDEsIiIsMCx7ImN1cnZlIjotMn1dLFswLDEsIiIsMix7ImN1cnZlIjoyfV0sWzIsMywiIiwwLHsic2hvcnRlbiI6eyJzb3VyY2UiOjIwLCJ0YXJnZXQiOjIwfX1dXQ==
\[\begin{tikzcd}
	0 & 1
	\arrow[""{name=0, anchor=center, inner sep=0}, curve={height=-12pt}, from=1-1, to=1-2]
	\arrow[""{name=1, anchor=center, inner sep=0}, curve={height=12pt}, from=1-1, to=1-2]
	\arrow[shorten <=3pt, shorten >=3pt, Rightarrow, from=0, to=1]
\end{tikzcd}\]
The Gray tensor product of $\Db_1$ with itself, denoted by $\Db_1\otimes\Db_1$, is the $2$-category generated by the diagram
% https://q.uiver.app/?q=WzAsNCxbMCwwLCIwMCJdLFswLDEsIjEwIl0sWzEsMSwiMTEiXSxbMSwwLCIwMSJdLFswLDFdLFsxLDJdLFswLDNdLFszLDJdLFszLDEsIiIsMSx7InNob3J0ZW4iOnsic291cmNlIjoyMCwidGFyZ2V0IjoyMH0sImxldmVsIjoyfV1d
\[\begin{tikzcd}
	00 & 01 \\
	10 & 11
	\arrow[from=1-1, to=2-1]
	\arrow[from=2-1, to=2-2]
	\arrow[from=1-1, to=1-2]
	\arrow[from=1-2, to=2-2]
	\arrow[shorten <=4pt, shorten >=4pt, Rightarrow, from=1-2, to=2-1]
\end{tikzcd}\]
The Gray tensor product of $\Db_2$ with $\Db_1$, denoted by $\Db_2\otimes\Db_1$, is the $3$-category generated by the diagram
% https://q.uiver.app/?q=WzAsOCxbMSwwLCIwMSJdLFswLDAsIjAwIl0sWzAsMSwiMTAiXSxbMSwxLCIxMSJdLFsyLDAsIjAwIl0sWzMsMCwiMDEiXSxbMywxLCIxMSJdLFsyLDEsIjEwIl0sWzEsMF0sWzEsMl0sWzIsM10sWzAsM10sWzAsMiwiIiwxLHsic2hvcnRlbiI6eyJzb3VyY2UiOjIwLCJ0YXJnZXQiOjIwfSwibGV2ZWwiOjJ9XSxbNCw3XSxbNCw1XSxbNSw2XSxbNSw3LCIiLDEseyJzaG9ydGVuIjp7InNvdXJjZSI6MjAsInRhcmdldCI6MjB9LCJsZXZlbCI6Mn1dLFsxLDIsIiIsMix7ImN1cnZlIjo1fV0sWzcsNl0sWzUsNiwiIiwxLHsiY3VydmUiOi01fV0sWzksMTcsIiAiLDIseyJzaG9ydGVuIjp7InNvdXJjZSI6MjAsInRhcmdldCI6MjB9fV0sWzE5LDE1LCIgIiwyLHsic2hvcnRlbiI6eyJzb3VyY2UiOjIwLCJ0YXJnZXQiOjIwfX1dLFsxMSwxMywiIiwwLHsib2Zmc2V0IjotMSwic2hvcnRlbiI6eyJzb3VyY2UiOjIwLCJ0YXJnZXQiOjIwfSwibGV2ZWwiOjEsInN0eWxlIjp7ImhlYWQiOnsibmFtZSI6Im5vbmUifX19XSxbMTEsMTMsIiIsMix7Im9mZnNldCI6MSwic2hvcnRlbiI6eyJzb3VyY2UiOjIwLCJ0YXJnZXQiOjIwfSwibGV2ZWwiOjEsInN0eWxlIjp7ImhlYWQiOnsibmFtZSI6Im5vbmUifX19XSxbMTEsMTMsIiIsMSx7InNob3J0ZW4iOnsic291cmNlIjoyMCwidGFyZ2V0IjoyMH0sImxldmVsIjoxfV1d
\[\begin{tikzcd}
	00 & 01 & 00 & 01 \\
	10 & 11 & 10 & 11
	\arrow[from=1-1, to=1-2]
	\arrow[""{name=0, anchor=center, inner sep=0}, from=1-1, to=2-1]
	\arrow[from=2-1, to=2-2]
	\arrow[""{name=1, anchor=center, inner sep=0}, from=1-2, to=2-2]
	\arrow[shorten <=4pt, shorten >=4pt, Rightarrow, from=1-2, to=2-1]
	\arrow[""{name=2, anchor=center, inner sep=0}, from=1-3, to=2-3]
	\arrow[from=1-3, to=1-4]
	\arrow[""{name=3, anchor=center, inner sep=0}, from=1-4, to=2-4]
	\arrow[shorten <=4pt, shorten >=4pt, Rightarrow, from=1-4, to=2-3]
	\arrow[""{name=4, anchor=center, inner sep=0}, curve={height=30pt}, from=1-1, to=2-1]
	\arrow[from=2-3, to=2-4]
	\arrow[""{name=5, anchor=center, inner sep=0}, curve={height=-30pt}, from=1-4, to=2-4]
	\arrow["{ }"', shorten <=6pt, shorten >=6pt, Rightarrow, from=0, to=4]
	\arrow["{ }"', shorten <=6pt, shorten >=6pt, Rightarrow, from=5, to=3]
	\arrow[shift left=0.7, shorten <=6pt, shorten >=8pt, no head, from=1, to=2]
	\arrow[shift right=0.7, shorten <=6pt, shorten >=8pt, no head, from=1, to=2]
	\arrow[shorten <=6pt, shorten >=6pt, from=1, to=2]
\end{tikzcd}\]
\end{example*}
As we can see from these examples, the Gray tensor product adds the dimension of the inputs (in contrast to the cartesian product, which takes the maximum). Thus, $(\infty,n)$-categories are not stable under this operation. One can handle this by considering a truncated version of the Gray tensor product, but we believe that avoiding such violent operation will lead to a more natural understanding of the complex combinatorics it encodes.

One way to avoid all these issues related to the increasing of dimension is to directly focus on $(\infty,\omega)$-categories, which will be the standpoint of this thesis.


\phantomsection
\addcontentsline{toc}{section}{A brief definition of $(\gamma,n)$-categories for $n\in \Nb\cup\{\omega\}$}
\section*{A brief definition of $(\gamma,n)$-categories for $n\in \Nb\cup\{\omega\}$}


A \textit{globular set} is the data of a diagram of sets
% https://q.uiver.app/#q=WzAsNCxbMCwwLCJYXzAiXSxbMSwwLCJYXzEiXSxbMiwwLCJYXzIiXSxbMywwLCIuLi4iXSxbMSwwLCJcXHBpXzBeKyIsMix7Im9mZnNldCI6Mn1dLFsyLDEsIlxccGlfMV4rIiwyLHsib2Zmc2V0IjoyfV0sWzMsMiwiXFxwaV8zXisiLDIseyJvZmZzZXQiOjJ9XSxbMSwwLCJcXHBpXzBeLSIsMCx7Im9mZnNldCI6LTJ9XSxbMiwxLCJcXHBpXzFeLSIsMCx7Im9mZnNldCI6LTJ9XSxbMywyLCJcXHBpXzNeLSIsMCx7Im9mZnNldCI6LTJ9XV0=
\[\begin{tikzcd}
	{X_0} & {X_1} & {X_2} & {...}
	\arrow["{\pi_0^+}"', shift right=2, from=1-2, to=1-1]
	\arrow["{\pi_1^+}"', shift right=2, from=1-3, to=1-2]
	\arrow["{\pi_3^+}"', shift right=2, from=1-4, to=1-3]
	\arrow["{\pi_0^-}", shift left=2, from=1-2, to=1-1]
	\arrow["{\pi_1^-}", shift left=2, from=1-3, to=1-2]
	\arrow["{\pi_3^-}", shift left=2, from=1-4, to=1-3]
\end{tikzcd}\]
with the relations $\pi_{n-1}^{\epsilon}\pi_{n}^+ = \pi_n^{\epsilon} \pi_{n}^-$ for any $n>0$ and $\epsilon \in \{+,-\}$. We also denote by $\pi^{\epsilon}_k$ the map $X_n \to X_k$ for $k< n$ obtained by composing any string of arrows starting with $\pi^\epsilon_{k}$. An \textit{$\omega$-category} is a globular set $X$ together with
\begin{enumerate}
\item operations of \textit{compositions}
\[ X_n\times_{X_k} X_n\to X_n ~~~(0\leq k<n) \]
which associate to two $n$-cells $(x,y)$ verifying $\pi_k^+(x) = \pi_k^-(y)$, an $n$-cell $x\circ_ky$,
\item as well as \textit{units}
\[X_n\to X_{n+1}\]
which associate to an $n$-cell $x$, an $(n+1)$-cell $\Ib_x$, 
\end{enumerate}
and satisfying some associativity and unitaly axioms which will be expected by any reader familiar with $2$-categories (see \ref{para:def of omega cat} for the precise formulation of these axioms).
A \textit{morphism of $\omega$-categories} is a map of globular sets commuting with both operations. The category of $\omega$-categories is denoted by \textit{$\omegacat$}.

The category $\Theta$ of Joyal is the full subcategory of $\omegacat$ spanned by the \textit{globular sums}. These objects are precisely defined in paragraph \ref{para:les sommes glob}. Roughly speaking, globular sums are the $\omega$-categories obtained by "directed" gluing of \textit{globes}. In particular, globes are the easiest example of globular sums. Here are a few examples of globes and globular sums, where we identify the pasting diagrams with the $\omega$-categories they generate.

\begin{example*}[some examples of globes]
\label{exe:exemple 0}
% https://q.uiver.app/#q=WzAsMTIsWzEsMCwiXFxidWxsZXQiXSxbMiwwXSxbMywwLCJcXGJ1bGxldCJdLFs0LDAsIlxcYnVsbGV0Il0sWzUsMF0sWzYsMCwiXFxidWxsZXQiXSxbNywwLCJcXGJ1bGxldCJdLFswLDBdLFs5LDAsIlxcYnVsbGV0Il0sWzEwLDAsIlxcYnVsbGV0Il0sWzgsMF0sWzAsMl0sWzIsM10sWzUsNiwiIiwwLHsiY3VydmUiOi00fV0sWzUsNiwiIiwyLHsiY3VydmUiOjR9XSxbNywwLCJcXERiXzA6PSIsMSx7InN0eWxlIjp7ImJvZHkiOnsibmFtZSI6Im5vbmUifSwiaGVhZCI6eyJuYW1lIjoibm9uZSJ9fX1dLFsxLDIsIlxcRGJfMTo9IiwxLHsic3R5bGUiOnsiYm9keSI6eyJuYW1lIjoibm9uZSJ9LCJoZWFkIjp7Im5hbWUiOiJub25lIn19fV0sWzQsNSwiXFxEYl8yOj0iLDEseyJzdHlsZSI6eyJib2R5Ijp7Im5hbWUiOiJub25lIn0sImhlYWQiOnsibmFtZSI6Im5vbmUifX19XSxbOCw5LCIiLDEseyJjdXJ2ZSI6LTR9XSxbOCw5LCIiLDEseyJjdXJ2ZSI6NH1dLFsxMCw4LCJcXERiXzM6PSIsMSx7InN0eWxlIjp7ImJvZHkiOnsibmFtZSI6Im5vbmUifSwiaGVhZCI6eyJuYW1lIjoibm9uZSJ9fX1dLFsxMywxNCwiIiwwLHsic2hvcnRlbiI6eyJzb3VyY2UiOjIwLCJ0YXJnZXQiOjIwfX1dLFsxOCwxOSwiIiwxLHsib2Zmc2V0IjotMywic2hvcnRlbiI6eyJzb3VyY2UiOjIwLCJ0YXJnZXQiOjIwfX1dLFsxOCwxOSwiIiwxLHsib2Zmc2V0IjozLCJzaG9ydGVuIjp7InNvdXJjZSI6MjAsInRhcmdldCI6MjB9fV0sWzIzLDIyLCJcXFJyaWdodGFycm93IiwxLHsic2hvcnRlbiI6eyJzb3VyY2UiOjIwLCJ0YXJnZXQiOjIwfSwic3R5bGUiOnsiYm9keSI6eyJuYW1lIjoibm9uZSJ9LCJoZWFkIjp7Im5hbWUiOiJub25lIn19fV1d
\[\begin{tikzcd}
	{} & \bullet & {} & \bullet & \bullet & {} & \bullet & \bullet & {} & \bullet & \bullet \\
	\\
	{}
	\arrow[from=1-4, to=1-5]
	\arrow[""{name=0, anchor=center, inner sep=0}, curve={height=-24pt}, from=1-7, to=1-8]
	\arrow[""{name=1, anchor=center, inner sep=0}, curve={height=24pt}, from=1-7, to=1-8]
	\arrow["{\Db_0:=}"{description}, draw=none, from=1-1, to=1-2]
	\arrow["{\Db_1:=}"{description}, draw=none, from=1-3, to=1-4]
	\arrow["{\Db_2:=}"{description}, draw=none, from=1-6, to=1-7]
	\arrow[""{name=2, anchor=center, inner sep=0}, curve={height=-24pt}, from=1-10, to=1-11]
	\arrow[""{name=3, anchor=center, inner sep=0}, curve={height=24pt}, from=1-10, to=1-11]
	\arrow["{\Db_3:=}"{description}, draw=none, from=1-9, to=1-10]
	\arrow[shorten <=6pt, shorten >=6pt, Rightarrow, from=0, to=1]
	\arrow[""{name=4, anchor=center, inner sep=0}, shift left=3, shorten <=6pt, shorten >=6pt, Rightarrow, from=2, to=3]
	\arrow[""{name=5, anchor=center, inner sep=0}, shift right=3, shorten <=6pt, shorten >=6pt, Rightarrow, from=2, to=3]
	\arrow["\Rrightarrow"{description}, draw=none, from=5, to=4]
\end{tikzcd}\]
\end{example*}

\begin{example*}[some examples of globular sums]
\label{exe:exemple 1}
% https://q.uiver.app/#q=WzAsMTEsWzIsMCwiXFxidWxsZXQiXSxbMywwLCJcXGJ1bGxldCJdLFsxLDAsIlxcYnVsbGV0Il0sWzUsMCwiXFxidWxsZXQiXSxbNiwwLCJcXGJ1bGxldCJdLFs4LDAsIlxcYnVsbGV0Il0sWzksMCwiXFxidWxsZXQiXSxbMTAsMCwiXFxidWxsZXQiXSxbMCwwXSxbNCwwXSxbNywwXSxbMyw0LCIiLDAseyJjdXJ2ZSI6LTR9XSxbMyw0LCIiLDIseyJjdXJ2ZSI6NH1dLFszLDRdLFsyLDBdLFswLDFdLFs1LDZdLFs1LDYsIiIsMix7ImN1cnZlIjo0fV0sWzUsNiwiIiwxLHsiY3VydmUiOi00fV0sWzYsNywiIiwxLHsiY3VydmUiOi00fV0sWzYsNywiIiwxLHsiY3VydmUiOjR9XSxbOCwyLCJhXzA6PSIsMSx7InN0eWxlIjp7ImJvZHkiOnsibmFtZSI6Im5vbmUifSwiaGVhZCI6eyJuYW1lIjoibm9uZSJ9fX1dLFs5LDMsImFfMTo9IiwxLHsic3R5bGUiOnsiYm9keSI6eyJuYW1lIjoibm9uZSJ9LCJoZWFkIjp7Im5hbWUiOiJub25lIn19fV0sWzEwLDUsImFfMjo9IiwxLHsic3R5bGUiOnsiYm9keSI6eyJuYW1lIjoibm9uZSJ9LCJoZWFkIjp7Im5hbWUiOiJub25lIn19fV0sWzExLDEzLCIiLDAseyJzaG9ydGVuIjp7InNvdXJjZSI6MjAsInRhcmdldCI6MjB9fV0sWzEzLDEyLCIiLDAseyJzaG9ydGVuIjp7InNvdXJjZSI6MjAsInRhcmdldCI6MjB9fV0sWzE2LDE3LCIiLDAseyJvZmZzZXQiOi0zLCJzaG9ydGVuIjp7InNvdXJjZSI6MjAsInRhcmdldCI6MzB9fV0sWzE2LDE3LCIiLDIseyJvZmZzZXQiOjMsInNob3J0ZW4iOnsic291cmNlIjoyMCwidGFyZ2V0IjozMH19XSxbMTgsMTYsIiIsMSx7InNob3J0ZW4iOnsic291cmNlIjoyMCwidGFyZ2V0IjoyMH19XSxbMTksMjAsIiIsMSx7InNob3J0ZW4iOnsic291cmNlIjoyMCwidGFyZ2V0IjoyMH19XSxbMjcsMjYsIlxcUnJpZ2h0YXJyb3ciLDEseyJvZmZzZXQiOi0xLCJzaG9ydGVuIjp7InNvdXJjZSI6MjAsInRhcmdldCI6MjB9LCJzdHlsZSI6eyJib2R5Ijp7Im5hbWUiOiJub25lIn0sImhlYWQiOnsibmFtZSI6Im5vbmUifX19XV0=
\[\begin{tikzcd}
	{} & \bullet & \bullet & \bullet & {} & \bullet & \bullet & {} & \bullet & \bullet & \bullet
	\arrow[""{name=0, anchor=center, inner sep=0}, curve={height=-24pt}, from=1-6, to=1-7]
	\arrow[""{name=1, anchor=center, inner sep=0}, curve={height=24pt}, from=1-6, to=1-7]
	\arrow[""{name=2, anchor=center, inner sep=0}, from=1-6, to=1-7]
	\arrow[from=1-2, to=1-3]
	\arrow[from=1-3, to=1-4]
	\arrow[""{name=3, anchor=center, inner sep=0}, from=1-9, to=1-10]
	\arrow[""{name=4, anchor=center, inner sep=0}, curve={height=24pt}, from=1-9, to=1-10]
	\arrow[""{name=5, anchor=center, inner sep=0}, curve={height=-24pt}, from=1-9, to=1-10]
	\arrow[""{name=6, anchor=center, inner sep=0}, curve={height=-24pt}, from=1-10, to=1-11]
	\arrow[""{name=7, anchor=center, inner sep=0}, curve={height=24pt}, from=1-10, to=1-11]
	\arrow["{a_0:=}"{description}, draw=none, from=1-1, to=1-2]
	\arrow["{a_1:=}"{description}, draw=none, from=1-5, to=1-6]
	\arrow["{a_2:=}"{description}, draw=none, from=1-8, to=1-9]
	\arrow[shorten <=3pt, shorten >=3pt, Rightarrow, from=0, to=2]
	\arrow[shorten <=3pt, shorten >=3pt, Rightarrow, from=2, to=1]
	\arrow[""{name=8, anchor=center, inner sep=0}, shift left=3, shorten <=3pt, shorten >=5pt, Rightarrow, from=3, to=4]
	\arrow[""{name=9, anchor=center, inner sep=0}, shift right=3, shorten <=3pt, shorten >=5pt, Rightarrow, from=3, to=4]
	\arrow[shorten <=3pt, shorten >=3pt, Rightarrow, from=5, to=3]
	\arrow[shorten <=6pt, shorten >=6pt, Rightarrow, from=6, to=7]
	\arrow["\Rrightarrow"{description}, shift left=1, draw=none, from=9, to=8]
\end{tikzcd}\]
\end{example*}
\begin{example*}[some examples of morphisms between globular sums]
%[column sep=0.367in]
% https://q.uiver.app/?q=WzAsMTksWzEwLDMsIlxcYnVsbGV0Il0sWzExLDMsIlxcYnVsbGV0Il0sWzQsMywiXFxidWxsZXQiXSxbNSwzLCJcXGJ1bGxldCJdLFs2LDMsIlxcYnVsbGV0Il0sWzcsMywiXFxidWxsZXQiXSxbOCwzLCJcXGJ1bGxldCJdLFswLDMsIlxcYnVsbGV0Il0sWzIsMywiXFxidWxsZXQiXSxbNywwLCJcXGJ1bGxldCJdLFs4LDAsIlxcYnVsbGV0Il0sWzEwLDAsIlxcYnVsbGV0Il0sWzExLDAsIlxcYnVsbGV0Il0sWzEsMywiXFxidWxsZXQiXSxbMCwwLCJcXGJ1bGxldCJdLFsyLDAsIlxcYnVsbGV0Il0sWzQsMCwiXFxidWxsZXQiXSxbNiwwLCJcXGJ1bGxldCJdLFs1LDAsIlxcYnVsbGV0Il0sWzAsMV0sWzMsNCwiIiwwLHsiY3VydmUiOi00fV0sWzMsNCwiIiwwLHsiY3VydmUiOjR9XSxbNSw2LCIiLDAseyJjdXJ2ZSI6NH1dLFs1LDYsIiIsMSx7ImN1cnZlIjotNH1dLFs2LDAsImZfMyIsMCx7InNob3J0ZW4iOnsic291cmNlIjo0MCwidGFyZ2V0Ijo0MH0sInN0eWxlIjp7InRhaWwiOnsibmFtZSI6Im1hcHMgdG8ifX19XSxbMiwzXSxbOCwyLCJmXzIiLDAseyJzaG9ydGVuIjp7InNvdXJjZSI6NDAsInRhcmdldCI6NDB9LCJzdHlsZSI6eyJ0YWlsIjp7Im5hbWUiOiJtYXBzIHRvIn19fV0sWzksMTAsIiIsMSx7ImN1cnZlIjotNH1dLFs5LDEwLCIiLDEseyJjdXJ2ZSI6NH1dLFsxMSwxMiwiIiwxLHsiY3VydmUiOi00fV0sWzExLDEyLCIiLDEseyJjdXJ2ZSI6NH1dLFsxMSwxMl0sWzEwLDExLCJmXzEiLDAseyJzaG9ydGVuIjp7InNvdXJjZSI6NDAsInRhcmdldCI6NDB9LCJzdHlsZSI6eyJ0YWlsIjp7Im5hbWUiOiJtYXBzIHRvIn19fV0sWzEzLDgsIiIsMSx7ImN1cnZlIjotNH1dLFs3LDEzLCIiLDAseyJjdXJ2ZSI6NH1dLFsxNCwxNV0sWzE1LDE2LCJmXzAiLDAseyJzaG9ydGVuIjp7InNvdXJjZSI6NDAsInRhcmdldCI6NDB9LCJzdHlsZSI6eyJ0YWlsIjp7Im5hbWUiOiJtYXBzIHRvIn19fV0sWzE2LDE4XSxbMTgsMTddLFsyLDMsIiIsMCx7ImN1cnZlIjotNH1dLFsyLDMsIiIsMCx7ImN1cnZlIjo0fV0sWzIwLDIxLCIiLDAseyJzaG9ydGVuIjp7InNvdXJjZSI6MjAsInRhcmdldCI6MjB9fV0sWzIzLDIyLCIiLDEseyJzaG9ydGVuIjp7InNvdXJjZSI6MjAsInRhcmdldCI6MjB9fV0sWzI5LDMxLCIiLDEseyJzaG9ydGVuIjp7InNvdXJjZSI6MjAsInRhcmdldCI6MjB9fV0sWzMxLDMwLCIiLDEseyJzaG9ydGVuIjp7InNvdXJjZSI6MjAsInRhcmdldCI6MjB9fV0sWzI3LDI4LCIiLDEseyJzaG9ydGVuIjp7InNvdXJjZSI6MjAsInRhcmdldCI6MjB9fV0sWzM5LDI1LCIiLDAseyJzaG9ydGVuIjp7InNvdXJjZSI6MjAsInRhcmdldCI6MjB9fV0sWzI1LDQwLCIiLDAseyJvZmZzZXQiOi0zLCJzaG9ydGVuIjp7InNvdXJjZSI6MjAsInRhcmdldCI6MzB9fV0sWzI1LDQwLCIiLDAseyJvZmZzZXQiOjMsInNob3J0ZW4iOnsic291cmNlIjoyMCwidGFyZ2V0IjozMH19XSxbNDgsNDcsIlxcUnJpZ2h0YXJyb3ciLDEseyJvZmZzZXQiOi0xLCJzaG9ydGVuIjp7InNvdXJjZSI6MjAsInRhcmdldCI6MjB9fV1d
\[\begin{tikzcd}[column sep=0.367in]
	\bullet && \bullet && \bullet & \bullet & \bullet & \bullet & \bullet && \bullet & \bullet \\
	\\
	\\
	\bullet & \bullet & \bullet && \bullet & \bullet & \bullet & \bullet & \bullet && \bullet & \bullet
	\arrow[from=4-11, to=4-12]
	\arrow[""{name=0, anchor=center, inner sep=0}, curve={height=-24pt}, from=4-6, to=4-7]
	\arrow[""{name=1, anchor=center, inner sep=0}, curve={height=24pt}, from=4-6, to=4-7]
	\arrow[""{name=2, anchor=center, inner sep=0}, curve={height=24pt}, from=4-8, to=4-9]
	\arrow[""{name=3, anchor=center, inner sep=0}, curve={height=-24pt}, from=4-8, to=4-9]
	\arrow["{f_3}", shorten <=19pt, shorten >=19pt, maps to, from=4-9, to=4-11]
	\arrow[""{name=4, anchor=center, inner sep=0}, from=4-5, to=4-6]
	\arrow["{f_2}", shorten <=19pt, shorten >=19pt, maps to, from=4-3, to=4-5]
	\arrow[""{name=5, anchor=center, inner sep=0}, curve={height=-24pt}, from=1-8, to=1-9]
	\arrow[""{name=6, anchor=center, inner sep=0}, curve={height=24pt}, from=1-8, to=1-9]
	\arrow[""{name=7, anchor=center, inner sep=0}, curve={height=-24pt}, from=1-11, to=1-12]
	\arrow[""{name=8, anchor=center, inner sep=0}, curve={height=24pt}, from=1-11, to=1-12]
	\arrow[""{name=9, anchor=center, inner sep=0}, from=1-11, to=1-12]
	\arrow["{f_1}", shorten <=19pt, shorten >=19pt, maps to, from=1-9, to=1-11]
	\arrow[curve={height=-24pt}, from=4-2, to=4-3]
	\arrow[curve={height=24pt}, from=4-1, to=4-2]
	\arrow[from=1-1, to=1-3]
	\arrow["{f_0}", shorten <=19pt, shorten >=19pt, maps to, from=1-3, to=1-5]
	\arrow[from=1-5, to=1-6]
	\arrow[from=1-6, to=1-7]
	\arrow[""{name=10, anchor=center, inner sep=0}, curve={height=-24pt}, from=4-5, to=4-6]
	\arrow[""{name=11, anchor=center, inner sep=0}, curve={height=24pt}, from=4-5, to=4-6]
	\arrow[shorten <=6pt, shorten >=6pt, Rightarrow, from=0, to=1]
	\arrow[shorten <=6pt, shorten >=6pt, Rightarrow, from=3, to=2]
	\arrow[shorten <=3pt, shorten >=3pt, Rightarrow, from=7, to=9]
	\arrow[shorten <=3pt, shorten >=3pt, Rightarrow, from=9, to=8]
	\arrow[shorten <=6pt, shorten >=6pt, Rightarrow, from=5, to=6]
	\arrow[shorten <=3pt, shorten >=3pt, Rightarrow, from=10, to=4]
	\arrow[""{name=12, anchor=center, inner sep=0}, shift left=3, shorten <=3pt, shorten >=5pt, Rightarrow, from=4, to=11]
	\arrow[""{name=13, anchor=center, inner sep=0}, shift right=3, shorten <=3pt, shorten >=5pt, Rightarrow, from=4, to=11]
	\arrow["\Rrightarrow"{description}, shift left=1, shorten <=2pt, shorten >=2pt, from=13, to=12]
\end{tikzcd}\]
\end{example*}

 For $n\in \Nb\cup \{\omega\}$, we define $\Theta_n$ as the full subcategory of $\Theta$ whose objects correspond to $n$-categories. In particular, $\Theta_0$ is the terminal category, $\Theta_1$ is $\Delta$, and $\Theta_\omega$ is $\Theta$.


Let $\gamma$ be a complete $\iun$-category and $n\in \Nb\cup \{\omega\}$. A \textit{$(\gamma,n)$-category} is a functor $\Theta_n^{op}\to \gamma$ that satisfies the \textit{Segal conditions} and \textit{completeness conditions}. We denote by $(\gamma,n)\mbox{-$\cat$}$ the $\iun$-category of $(\gamma,n)$-categories. Since we have not given a precise definition of $\Theta$, we cannot explicitly state these conditions, but we will try to explain their essence.



\textbf{Segal conditions.} As the diagrams given in the examples suggest, every globular sum is a colimit of globes. For instance, $a_2$ is the colimit of the following diagram
% https://q.uiver.app/#q=WzAsNSxbMSwxLCJcXERiXzAiXSxbMiwxLCJcXERiXzIiXSxbMCwxLCJcXERiXzEiXSxbMCwwLCJcXERiXzIiXSxbMCwyLCJcXERiXzMiXSxbMiwzLCJpXzFeKyJdLFsyLDQsImlfMV4tIiwyXSxbMCwyLCJpXzBeKyIsMl0sWzAsMSwiaV8wXi0iXV0=
\[\begin{tikzcd}
	{\Db_2} \\
	{\Db_1} & {\Db_0} & {\Db_2} \\
	{\Db_3}
	\arrow["{i_1^+}", from=2-1, to=1-1]
	\arrow["{i_1^-}"', from=2-1, to=3-1]
	\arrow["{i_0^+}"', from=2-2, to=2-1]
	\arrow["{i_0^-}", from=2-2, to=2-3]
\end{tikzcd}\]
A functor $X:\Theta_n^{op}\to \gamma$ satisfies the \textit{Segal conditions} if it sends these colimits to limits. For instance, the presheaf $X$ must send $a_2$ to the limit of the diagram 
% https://q.uiver.app/#q=WzAsNSxbMSwxLCJYKFxcRGJfMCkiXSxbMiwxLCJYKFxcRGJfMikiXSxbMCwxLCJYKFxcRGJfMSkiXSxbMCwwLCJYKFxcRGJfMikiXSxbMCwyLCJYKFxcRGJfMykiXSxbMywyLCJcXHBpXzFeKyIsMl0sWzQsMiwiXFxwaV8xXi0iXSxbMiwwLCJcXHBpXzBeKyJdLFsxLDAsIlxccGlfMF4tIiwyXV0=
\[\begin{tikzcd}
	{X(\Db_2)} \\
	{X(\Db_1)} & {X(\Db_0)} & {X(\Db_2)} \\
	{X(\Db_3)}
	\arrow["{\pi_1^+}"', from=1-1, to=2-1]
	\arrow["{\pi_1^-}", from=3-1, to=2-1]
	\arrow["{\pi_0^+}", from=2-1, to=2-2]
	\arrow["{\pi_0^-}"', from=2-3, to=2-2]
\end{tikzcd}\]
The morphisms $X(f_0)$ and $X(f_1)$ can then be interpreted as compositions and the morphism $X(f_3)$ as a unit.

\textbf{Completeness conditions.} Let $X:\Theta_n^{op}\to \gamma$ be a functor satisfying the Segal conditions. Given an integer $k\leq n$, we have two notions of equivalence on the $k$-cells of $X$, i.e. the morphisms $1\to X(\Db_k)$. The first comes from the canonical equivalence provided by the $\infty$-groupoid $\Hom(1, X(\Db_k))$, and the second is more categorical and identifies \textit{isomorphic} elements, i.e. $k$-cells $a,b$ such that there exists $(k+1)$-cells $f:a\to b$, $g:b\to a$ and equivalences
$$g\circ_k f\sim id_a~~~~~~~~~\mbox{and}~~~~~~~~~f\circ_k g\sim id_b.$$
 The presheaf $X$ satisfies the completeness condition if these two notions of equivalence coincide. Thus, \textit{groupoids}, i.e., $(\gamma,n)$-categories in which all $k$-cells are equivalent to the identity of their source (or target), correspond to constant functors $\Theta^{op}\to \gamma$. The datum of the $(\infty,1)$-category $\gamma$ can be understood as a \textit{choice of a notion of groupoid}.


\paragraph{}
When $\gamma$ is the category of sets, the $(\gamma,n)$-categories will simply be denoted as $(0,n)$-categories, and when $\gamma$ is the $\iun$-category of spaces, they will be denoted as $(\infty,n)$-categories.

 For instance, $(0,\omega)$-categories correspond to $\Theta$-sets satisfying the Segal and completeness conditions. The first one induce an inclusion of $(0,\omega)$-categories into $\omega$-categories and the latter forces isomorphisms to be identities. The $(0,\omega)$-categories then correspond to \textit{Gaunt $\omega$-categories}.

Although this concept is not studied in the present thesis, it is worth noticing that one could define $(k,n)$-categories for any $k\in \Nb$. In this case, we would consider the $(\gamma,n)$-categories with $\gamma$ being the $\iun$-category of $k$-truncated $\infty$-groupoids. This notation is compatible with the one given in \cite{Rezk_a_cartesian_of_weak_n_categories} when $k\geq n$ but it also allows to  give meaning to $(k,n)$-categories for $k<n$.

\paragraph{}
As stated earlier, this work is devoted to the concept of $\io$-categories, which corresponds to the case where $\gamma$ is the category of spaces. This notion is sometimes considered ambiguous. Indeed, Schommer-Pries and Rezk have independently argued (\cite{134099}) that there should be more than one notion of $(\infty,\omega)$-categories. The one we use here is commonly referred to as \textit{the inductive one}, in the sense that $\ocat$ is identified with the limit of the sequence:
$$\ncat{0}\xleftarrow{\tau_0} \ncat{1}\xleftarrow{}... \leftarrow\ncat{n} \xleftarrow{\tau_{n}}\ncat{n+1}\xleftarrow{}...$$
where the functors $\tau_n$ "forget" the cells of dimension $n$. For a more detailed discussion in the (semi-)strict case, we refer to \cite{Henry_an_inductive_model_structure_for_infini_categories}.




\vspace{1cm}
\phantomsection
\addcontentsline{toc}{section}{Overview of the thesis}
\section*{Overview of the thesis}

This thesis is divided into two parts which can be read independently. However, each of them uses results from the preliminary section.

\phantomsection
\addcontentsline{toc}{subsection}{Preliminaries}
\subsection*{Preliminaries}


\paragraph{Chapter \ref{chapter:The category of zocategories}.}
The first section is devoted to the definition of $\zo$-categories and of the category $\Theta$ of Joyal. We also show that the category $\Theta$ presents the category of $\zo$-categories, and we also exhibit an other presentation of this category (corollary \ref{cor:changing theta}).


The second section begins with a review of Steiner theory, which is an extremely useful tool for providing concise and computational descriptions of $\zo$-categories. Following Ara and Maltsiniotis, we employ this theory to define the Gray tensor product, denoted by $\otimes$, in $\zo$-categories. We then introduce the Gray operations, starting with the Gray cylinder $\uvar\otimes[1]$ which is the Gray tensor product with the directed interval $[1]:=0\to 1$. Then, we have the Gray cone and Gray $\circ$-cone, denoted by $\uvar\star 1$ and $1\costar \uvar$, that send an $\zo$-category $C$ onto the following pushouts:
% https://q.uiver.app/#q=WzAsOCxbNCwwLCJDXFxvdGltZXNbMV0iXSxbMywwLCJDXFxvdGltZXNcXHswXFx9Il0sWzMsMSwiMSJdLFs0LDEsIjFcXGNvc3RhciBDIl0sWzAsMCwiQ1xcb3RpbWVzXFx7MVxcfSJdLFsxLDAsIkNcXG90aW1lc1sxXSJdLFswLDEsIjEiXSxbMSwxLCJDXFxzdGFyIDEiXSxbMCwzXSxbMSwyXSxbMiwzXSxbMSwwXSxbNSw3XSxbNiw3XSxbNCw2XSxbNCw1XSxbNyw0LCIiLDEseyJzdHlsZSI6eyJuYW1lIjoiY29ybmVyIn19XSxbMywxLCIiLDEseyJzdHlsZSI6eyJuYW1lIjoiY29ybmVyIn19XV0=
\[\begin{tikzcd}
	{C\otimes\{1\}} & {C\otimes[1]} && {C\otimes\{0\}} & {C\otimes[1]} \\
	1 & {C\star 1} && 1 & {1\costar C}
	\arrow[from=1-5, to=2-5]
	\arrow[from=1-4, to=2-4]
	\arrow[from=2-4, to=2-5]
	\arrow[from=1-4, to=1-5]
	\arrow[from=1-2, to=2-2]
	\arrow[from=2-1, to=2-2]
	\arrow[from=1-1, to=2-1]
	\arrow[from=1-1, to=1-2]
	\arrow["\lrcorner"{anchor=center, pos=0.125, rotate=180}, draw=none, from=2-2, to=1-1]
	\arrow["\lrcorner"{anchor=center, pos=0.125, rotate=180}, draw=none, from=2-5, to=1-4]
\end{tikzcd}\]


We also present a formula that illustrates the interaction between the suspension and the Gray cylinder. As this formula plays a crucial role in both Part I and Part II, we provide its intuition at this stage.

 If $A$ is any $\zo$-category, the suspension of $A$, denoted by $[A,1]$, is the $\zo$-category having two objects - denoted by $0$ and $1$- and such that 
$$\Hom_{[A,1]}(0,1) := A,~~~\Hom_{[A,1]}(1,0) := \emptyset,~~~\Hom_{[A,1]}(0,0)=\Hom_{[A,1]}(1,1):=\{id\}.$$
We also define $[1]\vee[A,1]$ as the gluing of $[1]$ and $[A,1]$ along the $0$-target of $[1]$ and the $0$-source of $[A,1]$. We define similarly $[A,1]\vee[1]$.
These two objects come along with \textit{whiskerings}:
$$\triangledown:[A,1]\to [1]\vee [A,1] ~~~~\mbox{and}~~~~ \triangledown:[A,1] \to [A,1]\vee [1]$$ 
that preserve the extremal points.


The $\zo$-category $[1]\otimes [1]$ is induced by the diagram:
% https://q.uiver.app/?q=WzAsNCxbMCwwLCIwMCJdLFswLDEsIjEwIl0sWzEsMSwiMTEiXSxbMSwwLCIwMSJdLFswLDFdLFsxLDJdLFswLDNdLFszLDJdLFszLDEsIiIsMSx7InNob3J0ZW4iOnsic291cmNlIjoyMCwidGFyZ2V0IjoyMH0sImxldmVsIjoyfV1d
\[\begin{tikzcd}
	00 & 01 \\
	10 & 11
	\arrow[from=1-1, to=2-1]
	\arrow[from=2-1, to=2-2]
	\arrow[from=1-1, to=1-2]
	\arrow[from=1-2, to=2-2]
	\arrow[shorten <=4pt, shorten >=4pt, Rightarrow, from=1-2, to=2-1]
\end{tikzcd}\]
and is then equal to the colimit of the following diagram: 
$$[1]\vee [1]\xleftarrow{\triangledown} [1]\hookrightarrow [[1],1]\hookleftarrow[1]\xrightarrow{\triangledown } [1]\vee [1].$$
The $\zo$-category $ [[1],1]\otimes [1]$ is induced by the diagram:
% https://q.uiver.app/?q=WzAsOCxbMSwwLCIwMSJdLFswLDAsIjAwIl0sWzAsMSwiMTAiXSxbMSwxLCIxMSJdLFsyLDAsIjAwIl0sWzMsMCwiMDEiXSxbMywxLCIxMSJdLFsyLDEsIjEwIl0sWzEsMF0sWzEsMl0sWzIsM10sWzAsM10sWzAsMiwiIiwxLHsic2hvcnRlbiI6eyJzb3VyY2UiOjIwLCJ0YXJnZXQiOjIwfSwibGV2ZWwiOjJ9XSxbNCw3XSxbNCw1XSxbNSw2XSxbNSw3LCIiLDEseyJzaG9ydGVuIjp7InNvdXJjZSI6MjAsInRhcmdldCI6MjB9LCJsZXZlbCI6Mn1dLFsxLDIsIiIsMix7ImN1cnZlIjo1fV0sWzcsNl0sWzUsNiwiIiwxLHsiY3VydmUiOi01fV0sWzksMTcsIiAiLDIseyJzaG9ydGVuIjp7InNvdXJjZSI6MjAsInRhcmdldCI6MjB9fV0sWzE5LDE1LCIgIiwyLHsic2hvcnRlbiI6eyJzb3VyY2UiOjIwLCJ0YXJnZXQiOjIwfX1dLFsxMSwxMywiIiwwLHsib2Zmc2V0IjotMSwic2hvcnRlbiI6eyJzb3VyY2UiOjIwLCJ0YXJnZXQiOjIwfSwibGV2ZWwiOjEsInN0eWxlIjp7ImhlYWQiOnsibmFtZSI6Im5vbmUifX19XSxbMTEsMTMsIiIsMix7Im9mZnNldCI6MSwic2hvcnRlbiI6eyJzb3VyY2UiOjIwLCJ0YXJnZXQiOjIwfSwibGV2ZWwiOjEsInN0eWxlIjp7ImhlYWQiOnsibmFtZSI6Im5vbmUifX19XSxbMTEsMTMsIiIsMSx7InNob3J0ZW4iOnsic291cmNlIjoyMCwidGFyZ2V0IjoyMH0sImxldmVsIjoxfV1d
\[\begin{tikzcd}
	00 & 01 & 00 & 01 \\
	10 & 11 & 10 & 11
	\arrow[from=1-1, to=1-2]
	\arrow[""{name=0, anchor=center, inner sep=0}, from=1-1, to=2-1]
	\arrow[from=2-1, to=2-2]
	\arrow[""{name=1, anchor=center, inner sep=0}, from=1-2, to=2-2]
	\arrow[shorten <=4pt, shorten >=4pt, Rightarrow, from=1-2, to=2-1]
	\arrow[""{name=2, anchor=center, inner sep=0}, from=1-3, to=2-3]
	\arrow[from=1-3, to=1-4]
	\arrow[""{name=3, anchor=center, inner sep=0}, from=1-4, to=2-4]
	\arrow[shorten <=4pt, shorten >=4pt, Rightarrow, from=1-4, to=2-3]
	\arrow[""{name=4, anchor=center, inner sep=0}, curve={height=30pt}, from=1-1, to=2-1]
	\arrow[from=2-3, to=2-4]
	\arrow[""{name=5, anchor=center, inner sep=0}, curve={height=-30pt}, from=1-4, to=2-4]
	\arrow["{ }"', shorten <=6pt, shorten >=6pt, Rightarrow, from=0, to=4]
	\arrow["{ }"', shorten <=6pt, shorten >=6pt, Rightarrow, from=5, to=3]
	\arrow[shift left=0.7, shorten <=6pt, shorten >=8pt, no head, from=1, to=2]
	\arrow[shift right=0.7, shorten <=6pt, shorten >=8pt, no head, from=1, to=2]
	\arrow[shorten <=6pt, shorten >=6pt, from=1, to=2]
\end{tikzcd}\]
and is then equal to the colimit of the following diagram: 
 $$[1]\vee[[1],1]\xleftarrow{\triangledown} [[1]\otimes\{0\},1]\hookrightarrow[[1]\otimes[1],1]\hookleftarrow [[1]\otimes\{1\},1]\xrightarrow{\triangledown}[[1],1]\vee[1]$$
We prove a formula that combines these two examples:

\begin{itheorem}[\ref{theo:appendice formula for otimes}]
In the category of $\zo$-categories, there exists an isomorphism, natural in $A$, between $[A,1]\otimes[1]$ and the colimit of the following diagram
% https://q.uiver.app/#q=WzAsNSxbMCwwLCJbMV1cXHZlZVtBLDFdIl0sWzEsMCwiW0FcXG90aW1lc1xcezBcXH0sMV0iXSxbMiwwLCIgW0FcXG90aW1lc1sxXSwxXSJdLFszLDAsIltBXFxvdGltZXNcXHsxXFx9LDFdIl0sWzQsMCwiW0EsMV1cXHZlZVsxXSJdLFsxLDAsIlxcdHJpYW5nbGVkb3duIiwyXSxbMywyXSxbMyw0LCJcXHRyaWFuZ2xlZG93biJdLFsxLDJdXQ==
\[\begin{tikzcd}
	{[1]\vee[A,1]} & {[A\otimes\{0\},1]} & { [A\otimes[1],1]} & {[A\otimes\{1\},1]} & {[A,1]\vee[1]}
	\arrow["\triangledown"', from=1-2, to=1-1]
	\arrow[from=1-4, to=1-3]
	\arrow["\triangledown", from=1-4, to=1-5]
	\arrow[from=1-2, to=1-3]
\end{tikzcd}\]
\end{itheorem} 

We also provide similar formulas for the \textit{Gray cone} and the \textit{Gray $\circ$-cone}.
\begin{itheorem}[\ref{theo:appendice formula for star}]
There is a natural identification between $1\costar [A,1]$ and the colimit of the following diagram
% https://q.uiver.app/#q=WzAsMyxbMCwwLCJbMV1cXHZlZVtBLDFdIl0sWzEsMCwiW0EsMV0iXSxbMiwwLCIgW0FcXHN0YXIgMSwxXSJdLFsxLDAsIlxcdHJpYW5nbGVkb3duIiwyXSxbMSwyXV0=
\[\begin{tikzcd}
	{[1]\vee[A,1]} & {[A,1]} & { [A\star 1,1]}
	\arrow["\triangledown"', from=1-2, to=1-1]
	\arrow[from=1-2, to=1-3]
\end{tikzcd}\]
There is a natural identification between $[A,1]\star 1$ and the colimit of the following diagram
% https://q.uiver.app/#q=WzAsMyxbMCwwLCIgWzFcXGNvc3RhciBBLDFdIl0sWzEsMCwiW0EsMV0iXSxbMiwwLCJbQSwxXVxcdmVlWzFdIl0sWzEsMF0sWzEsMiwiXFx0cmlhbmdsZWRvd24iXV0=
\[\begin{tikzcd}
	{ [1\costar A,1]} & {[A,1]} & {[A,1]\vee[1]}
	\arrow[from=1-2, to=1-1]
	\arrow["\triangledown", from=1-2, to=1-3]
\end{tikzcd}\]
\end{itheorem}


\phantomsection
\addcontentsline{toc}{subsection}{On the side of models}
\subsection*{On the side of models}


Following the terminology of Barwick and Schommer-Pries (\cite{Barwick_on_the_unicity_of_the_theory_of_higher_categories}), we call \textit{model of $(\infty,n)$-categories} any model category whose corresponding $(\infty, 1)$-category is $\ncat{n}$.

With the definition of $(\infty,n)$-categories given above, we have a natural model for the $\iun$-category $\ncat{n}$, given by Rezk's complete Segal $\Theta_n$-spaces, i.e. space valued presheaves on $\Theta_n$ satisfying the (homotopical) Segal conditions and (homotopical) completeness conditions. However, there are many other models, see for instance \cite{Ara_Higher_quasi_cat}, \cite{Bergner_Comparison_of_model_of_infini_n_categories}, \cite{Bergner_Comparison_of_model_for_infini_n_categories_II}, \cite{Bergner_reedy_category_and_the_theta_construction} (we refer to \cite{Barwick_on_the_unicity_of_the_theory_of_higher_categories}
for a comprehensive presentation of these models and their equivalences). For example, one can mention $n$-fold Segal spaces and Simpson's and Tamsamani's Segal $n$-categories among others.

It was conjectured (\cite{Street_algebra_of_orianted_simplexes}, \cite{Verity_a_complicial_compendium}, \cite{Barwick_on_the_unicity_of_the_theory_of_higher_categories}) that Verity's $n$-complicial sets were also a model of $(\infty,n)$-categories. This would imply that Campion-Kapulkin-Maehara's $n$-comical sets also are, as they are shown to be Quillen equivalent to $n$-complicial sets in \cite{Doherty_Equivalence_of_cubical_and_simplicial_approaches}. In the second chapter, we will give a positive answer to this conjecture (theorem \ref{theo:letheo}).

One of the major consequences of this result is to endow $\ocat$ with a monoidal product called the \textit{Gray tensor product}. This operation will play a crucial role in the second part of this thesis, which is dedicated to the theory of $\io$-categories.

\vspace{1cm}
The two main models we work with are Verity's complical sets (definition \ref{defi:complicial set}) and (a slight modification of) Segal $A$-precategories (defined in paragraph \ref{para:def sega a cat}) as developed by Simpson (\cite{Simpson_Homotopy_theory_of_higher_categories}). In the complical model, we will make crucial use of the strictification results of Ozornova and Rovelli (\cite{Ozornova_Fundamental_pushouts_of_n_complical_set}, \cite{Ozornova_a_quillen_adjunction_between_globular_and_complicial}). 

\paragraph{Chapter \ref{chapter:Studies of the complicial model}.}
One of the benefits of complicial sets is that they admit a simple definition of the Gray tensor product. Being strongly linked to $\zo$-categories by the Street nerve, they are also a privileged framework for stating and proving strictification results, as done in \cite{Ozornova_Fundamental_pushouts_of_n_complical_set}, \cite{Gagna_Nerves_and_cones_of_free_loop_free_omega_categories}, \cite{Ozornova_a_quillen_adjunction_between_globular_and_complicial} and \cite{Maehara_oriental_as_free_weak_omega_categories}. 
However, they do not interact \textit{a priori} well with the globular language. The goal of this chapter is to show that, with some computation, it is possible to have a globular point of view in this model. 

The first section is a recollection of usual results and definitions about complicial sets. 
In the second section, we aim to prove an analogue of the formula given in \ref{theo:appendice formula for otimes} to the complicial setting.
We also have a suspension in this category, which is denoted by $X\mapsto \Sigma X$. Objects $[1]\fwedge \Sigma X$ and $\Sigma X\fwedge [1]$ are defined in \ref{subsection:wedge}, but for now, we can suppose that they are fibrant replacements of respectively $[1]\coprod_{[0]}\Sigma X$ and $\Sigma X\coprod_{[0]}[1]$.
They come along with morphisms that are analogue to whiskerings, and that we also note by $\triangledown$: 
$$\triangledown:\Sigma X\to [1]\fwedge\Sigma X ~~~~\mbox{and}~~~~ \triangledown:\Sigma X\to\Sigma X\fwedge [1].$$ 
We then show the following theorem:
\begin{itheorem}[\ref{theo:interval_first_formula}]
There exists a zigzag of acyclic cofibrations, natural in $X$, between $(\Sigma X)\otimes [1]$ and the colimit of the following diagram:
 $$\Sigma X\fwedge [1]\xleftarrow{\triangledown} \Sigma (X\otimes\{0\}) \hookrightarrow \Sigma (X\otimes[1])\hookleftarrow \Sigma (X\otimes\{1\})\xrightarrow{\triangledown} [1]\fwedge \Sigma X.$$
\end{itheorem}
We also provide similar formulas for the \textit{Gray cone} and Gray \textit{$\circ$-cone}:
\begin{itheorem}[\ref{theo:cyl_formula}]
There exists a zigzag of acyclic cofibrations, natural in $X$, between $\Sigma X \star[0]$ and the colimit of the following diagram: 
$$ \Sigma X\fwedge [1]\leftarrow \Sigma X\to \Sigma([0]\costar X).$$
There exists a zigzag of acyclic cofibrations, natural in $X$, between  $[0]\costar \Sigma X$ and the colimit of the following diagram: 
$$\Sigma(X\star[0]) \leftarrow \Sigma X\to [1]\fwedge\Sigma X.$$
\end{itheorem}

The third section uses this formula and the strictification result of Gagna, Ozornova and	 Rovelli (\cite{Gagna_Nerves_and_cones_of_free_loop_free_omega_categories}) to demonstrate a criterion for detecting autoequivalences of complicial sets by their behavior on globes.
Indeed, in section \ref{section:Globular equivalences}, by iterating the suspension, we construct a globular object: 
% https://q.uiver.app/?q=WzAsNCxbMCwwLCJcXERiXzAiXSxbMSwwLCJcXERiXzEiXSxbMiwwLCJcXERiXzIiXSxbMywwLCIuLi4iXSxbMCwxLCJpXzBeKyIsMCx7Im9mZnNldCI6LTJ9XSxbMSwyLCJpXzFeKyIsMCx7Im9mZnNldCI6LTJ9XSxbMiwzLCJpXzNeKyIsMCx7Im9mZnNldCI6LTJ9XSxbMCwxLCJpXzBeLSIsMix7Im9mZnNldCI6Mn1dLFsxLDIsImlfMV4tIiwyLHsib2Zmc2V0IjoyfV0sWzIsMywiaV8zXi0iLDIseyJvZmZzZXQiOjJ9XV0=
\[\begin{tikzcd}
	{\Db_0} & {\Db_1} & {\Db_2} & {...}
	\arrow["{i_0^+}", shift left=2, from=1-1, to=1-2]
	\arrow["{i_1^+}", shift left=2, from=1-2, to=1-3]
	\arrow["{i_3^+}", shift left=2, from=1-3, to=1-4]
	\arrow["{i_0^-}"', shift right=2, from=1-1, to=1-2]
	\arrow["{i_1^-}"', shift right=2, from=1-2, to=1-3]
	\arrow["{i_3^-}"', shift right=2, from=1-3, to=1-4]
\end{tikzcd}\]
\begin{itheorem}[\ref{theo:criterion_to_be_linked_to_identity}]
Let $i$ be a left Quillen endofunctor for the model category for complicial sets. Suppose that there exists a zigzag of weakly invertible natural transformations:
$$i(\Db_{\uvar}) \leftrightsquigarrow \Db_{\uvar}.$$
Then, there exists a zigzag of weakly invertible natural transformations between $i$ and $id$.
\end{itheorem} 
Proposition 15.10 of \cite{Barwick_on_the_unicity_of_the_theory_of_higher_categories} provides a similar result for models of $(\infty,n)$-categories.

\paragraph{Chapter \ref{chapter:complicial set as a model of io categories}.}
Results of Bergner, Gagna, Harpaz, Lanari, Lurie and Rezk (\cite{Bergner_Comparison_of_model_of_infini_n_categories},\cite{Bergner_Comparison_of_model_for_infini_n_categories_II}, \cite{Rezk_a_cartesian_of_weak_n_categories}, \cite{Lurie_Htt},\cite{Lurie_goodwillie_calculus}, \cite{Gagna_on_the_equivallence_of_all_model_for_infini2_cat}) imply that $2$-complicial sets are a model of $(\infty,2)$-categories (see \cite{Gagna_on_the_equivallence_of_all_model_for_infini2_cat} to understand how to use all this source to obtained the desired result and \cite{Bergner_explicit_comparaison_bt_theta_2_space_and_2_complicial_set} for a direct comparison between complete Segal $\Theta_2$-spaces and $2$-complicial sets).
The purpose of this chapter is to generalize this result to any $n\in \Nb\cup\{\omega\}$.

To this extend, we first address the more general problem of finding sufficient conditions on a model category $A$ to build a \textit{Gray cylinder} $C\mapsto I\otimes C$ and a \textit{Gray cone} $C\mapsto e\star C$ on Segal precategories enriched in $A$. These two operations should be linked by the following homotopy cocartesian square
% https://q.uiver.app/?q=WzAsNCxbMSwwLCJJXFxvdGltZXMgQyJdLFsxLDEsImVcXHN0YXIgQyJdLFswLDEsImUiXSxbMCwwLCJcXHswXFx9XFxvdGltZXMgQyJdLFswLDFdLFszLDJdLFsyLDFdLFszLDBdXQ==
\[\begin{tikzcd}
	{\{0\}\otimes C} & {I\otimes C} \\
	e & {e\star C}
	\arrow[from=1-2, to=2-2]
	\arrow[from=1-1, to=2-1]
	\arrow[from=2-1, to=2-2]
	\arrow[from=1-1, to=1-2]
\end{tikzcd}\]
where $e$ is the terminal object. The conditions that $A$ has to	 fulfill are encapsulated in the notion of \textit{Gray module} (paragraph \ref{para:Gray module}). Thanks to the Gray cylinder and cone, we can show the following theorem:

\begin{itheorem}[\ref{theo:Quillen adjunction}]
If $A$ is a Gray module, there is a Quillen adjunction between the Ozornova-Rovelli model structure for $\omega$-complicial sets on stratified simplicial sets and stratified Segal precategories enriched in $A$ where the left adjoint sends $[n]$ to $e\star e\star ... \star e\star \emptyset$
\end{itheorem} 

We will apply this theorem to the case where $A$ is the category of stratified simplicial sets endowed with the model structure for $\omega$-complicial sets, and after tedious work, we get
\begin{itheorem}[\ref{theo:letheo}]
Let $n\in \Nb$.
The model structure for $n$-complicial sets is a model of $(\infty,n)$-categories.
\end{itheorem}
As a corollary we have
\begin{itheorem}[\ref{theo:lecorozo}]
The adjunction between the model structure for complete Segal $\Theta$-spaces and $\omega$-complicial set constructed in \cite{Ozornova_a_quillen_adjunction_between_globular_and_complicial} is a Quillen equivalence.
\end{itheorem} 


\phantomsection
\addcontentsline{toc}{subsection}{On the side of theory} 
\subsection*{On the side of theory}

In the second part of this thesis, we will adapt the constructions of classical category theory to the case $\io$. 
In this part, we will freely use the language of $\iun$-categories\footnote{ As there are currently several directions for the formalization of the language of $\iun$-categories (\cite{Riehl_element_of_infini_categories}, \cite{Riehl_A_type_theory_for_synthetic_-categories}, \cite{North_Towards_a_directed_homotopy_type_theory}, \cite{Cisinski-Univalent-Directed-Type-theory}), talking about "the" language of $\iun$-categories may be confusing.

In such case, the reader may consider that we are working within the quasi-category $\qcat$ of $\Tb$-small quasi-categories for $\Tb$ a Grothendieck universe. This quasi-category may be obtained either using the coherent nerve as described in \cite[chapter 3]{Lurie_Htt}, or by considering it as the codomain of the universal cocartesian fibration with $\Tb$-small fibers as done in \cite{Cisinski_The_universal_coCartesian_fibration}. In both cases, the straightening/unstraightening correspondence provides a morphism
$$\N(\Sset_{\Tb})\to \qcat$$
that exhibits $\qcat$ as the quasi-categorical localization of $\N(\Sset_{\Tb})$ with respect to the weak equivalences of the Joyal's model structure (\cite[theorem 8.13]{Cisinski_The_universal_coCartesian_fibration}). 

The constructions we use to build new objects - (co)limits of functor between quasi-categories, quasi-categories of functor, localization of quasi-categories, sub maximal Kan complex, full sub quasi-category, adjunction, left and right Kan extension, Yoneda lemma - are well documented in the Joyal model structure (see \cite{Lurie_Htt} or \cite{Cisinski_Higher_categories_and_homotopical_algebra})
, and therefore have direct incarnation in the quasi-category $\qcat$. }.



Chapter \ref{chapter:the infini 1 categorory of io categories} is devoted to the basic theory of $\io$-categories. Chapter \ref{chapter:chapter the 1 category of marked categories} introduces the notion of \textit{marked $\io$-categories} and studies \textit{left Cartesian fibrations}. Chapter \ref{chapter:The io-category of small io-categories} is dedicated to the \textit{Grothendieck construction}, \textit{univalence}, the \textit{Yoneda lemma}, and other standard categorical constructions.

Several of these results, or their analogues in the $(\infty,n)$ setting for some integer $n$, are already present in the literature. The case $n=1$, i.e. that of $(\infty,1)$-category theory, is now a prolific research field, and it would be impossible to list all the authors who have contributed to it. Nonetheless, we would like to mention Joyal for his pioneering work (\cite{Joyal_Quasi-categories_and_Kan_complexes}), Lurie for his major contribution (\cite{Lurie_Htt}), and Cisinski (\cite{Cisinski_Higher_categories_and_homotopical_algebra}) because his approach has deeply inspired the present work.

For the case $n=2$, the Grothendieck construction as well as lax limits and colimits have been extensively studied by Gagna, Lanari and Harpaz in \cite{Gagna_fibrations_and_lax_limit_infini_2_categories} and \cite{Gagna_Cartesian_Fibrations_of_infini_2_categories}, as well as by García and Stern in \cite{Garcia_2_cartesian_fibration_I} and \cite{Garcia_2_cartesian_fibration_II}.

For arbitrary $n$, Grothendieck construction has been described in \cite{Nuiten_on_straightening_for_segal_spaces} and \cite{Rasekh_yoneda_lemma_for_simplicial_spaces}. A partial version of the Yoneda lemma is also present in \cite{Rasekh_yoneda_lemma_for_simplicial_spaces}, \cite{Hinich_colimit_in_enriched_infini_categories}, and \cite{Heine_an_equivalence_between_enricherd_infini_categorories_and_categories_with_weak_action}.


\vspace{1cm}



\paragraph{Chapter \ref{chapter:the infini 1 categorory of io categories}.} 
This chapter is dedicated to the basic definition of $\io$-categories. In the first section, we recall some results on factorization systems in presentable $\iun$-categories. In the second section, we define $\io$-categories and give some basic properties. 
We also define and study \textit{discrete Conduché functor}, which are morphisms having the unique right lifting property against 
units $\Ib_{n+1}:\Db_{n+1}\to \Db_n$ for any integer $n$, and against compositions $\triangledown_{k,n}:\Db_n\to \Db_n\coprod_{\Db_k}\Db_n$ for any pair of integers $k\leq n$. This notion was originally defined and studied in the context of strict $\omega$-category by Guetta in \cite{Guetta_conduche}.
\begin{itheorem}[\ref{theo:pullback along conduche preserves colimits}]
Let $f:C\to D$ be a discrete Conduché functor. The pullback functor $f^*:\ocat_{/D}\to \ocat_{/C}$ preserves colimits.
\end{itheorem}


 In the third section, we study Gray operations for $\io$-categories. We conclude this chapter by proving results of strictification. In particular, we demonstrate the following theorem:
\begin{itheorem}[\ref{prop:strict stuff are pushout}]
Let $C$ be an $\io$-category, $b$ a globular sum, and $f:b\to C$ any morphism. The $\io$-categories $$1\costar b\coprod_b C,~C\coprod_b b\otimes[1]~\mbox{and}~C\coprod_b b\star 1$$
are strict whenever $C$ is.
\end{itheorem}
We will also prove the following theorem:
\begin{itheorem}[\ref{theo:strictness}]
If $C$ is strict, so are $C\star 1$, $1\costar C$ and $C\otimes [1]$.
\end{itheorem}
In the process, we will demonstrate another fundamental equation combining $C\otimes[1]$, $1\costar C$, $C\star 1$, and $[C,1]$.
\begin{itheorem}[\ref{theo:formula between pullback of slice and tensor}]
Let $C$ be an $\io$-category. The five squares appearing in the following canonical diagram are both cartesian and cocartesian:
% https://q.uiver.app/#q=WzAsOCxbMSwyLCIxXFxjb3N0YXIgQyJdLFsyLDIsIltDLDFdIl0sWzIsMSwiQ1xcc3RhciAxIl0sWzEsMSwiQ1xcb3RpbWVzWzFdIl0sWzIsMCwiMSJdLFsxLDAsIkNcXG90aW1lc1xcezBcXH0iXSxbMCwxLCJDXFxvdGltZXNcXHsxXFx9Il0sWzAsMiwiMSJdLFsyLDFdLFswLDFdLFszLDBdLFszLDJdLFs1LDRdLFs0LDJdLFs1LDNdLFs2LDNdLFs3LDBdLFs2LDddXQ==
\[\begin{tikzcd}
	& {C\otimes\{0\}} & 1 \\
	{C\otimes\{1\}} & {C\otimes[1]} & {C\star 1} \\
	1 & {1\costar C} & {[C,1]}
	\arrow[from=2-3, to=3-3]
	\arrow[from=3-2, to=3-3]
	\arrow[from=2-2, to=3-2]
	\arrow[from=2-2, to=2-3]
	\arrow[from=1-2, to=1-3]
	\arrow[from=1-3, to=2-3]
	\arrow[from=1-2, to=2-2]
	\arrow[from=2-1, to=2-2]
	\arrow[from=3-1, to=3-2]
	\arrow[from=2-1, to=3-1]
\end{tikzcd}\]
where $[C,1]$ is the \textit{suspension of $C$}.
\end{itheorem}






\paragraph{Chapter \ref{chapter:chapter the 1 category of marked categories}.}
This chapter is dedicated to the study of \textit{marked $\io$-categories}, which are pairs $(C,tC)$, where $C$ is an $\io$-category and $tC:=(tC_n)_{n>0}$ is a sequence of full sub $\infty$-groupoids of $C_n$ that include identities and are stable under composition and whiskering with (possibly unmarked) cells of lower dimensions. There are two canonical ways to mark an $\io$-category $C$. In the first, denoted by $C^\flat$, we mark as little as possible. In the second, denoted by $C^\sharp$, we mark everything.

The first section of the chapter defines these objects and establishes analogs of many results from section \ref{chapter:Basica construciton} to this new framework. In particular, the \textit{marked Gray cylinder} $\uvar\otimes [1]^\sharp$ is defined. If $A$ is an $\io$-category, the underlying $\io$-category of $A^\sharp\otimes[1]^\sharp$ is $A\times [1]$, and the underlying $\io$-category of $A^\flat\otimes[1]^\sharp$ is $A\otimes[1]$. By varying the marking, and at the level of underlying $\io$-categories, we "continuously" move from the cartesian product with the directed interval to the Gray tensor product with the directed interval.

The motivation for introducing markings comes from the notion of \textit{left (and right) cartesian fibrations}. A left cartesian fibration is a morphism between marked $\io$-categories such that only the marked cells of the codomain have cartesian lifting, and the marked cells of the domain correspond exactly to such cartesian lifting. For example, a left cartesian fibration $X\to A^\sharp$ is just a "usual" left cartesian fibration where we have marked the cartesian lifts of the domain, and every morphism $C^\flat \to D^\flat$ is a left cartesian fibration. This shows that marking plays a very different role here than in the case of marked simplicial sets, where it was there to represent (weak) invertibility. For example, if we had wanted to carry out this work in a complicial-like model category, we would have had to consider bimarked simplicial sets.



After defining and enumerating the stability properties enjoyed by this class of left (and right) cartesian fibration, we give several characterizations of this notion in theorem \ref{theo:other characterisation of left caresian fibration}. 

The more general subclass of left cartesian fibrations that still behaves well is the class of \textit{classified left cartesian fibrations}. 
This corresponds to left cartesian fibrations $X\to A$ such that there exists a cartesian square:
\[\begin{tikzcd}
	X & Y \\
	A & {A^\sharp}
	\arrow[from=1-1, to=2-1]
	\arrow[from=2-1, to=2-2]
	\arrow[from=1-1, to=1-2]
	\arrow[from=1-2, to=2-2]
	\arrow["\lrcorner"{anchor=center, pos=0.125}, draw=none, from=1-1, to=2-2]
\end{tikzcd}\]
 where the right vertical morphism is a left cartesian fibration and $A^\sharp$ is obtained from $A$ by marking all cells. In the second section, we prove the following fundamental result:

\begin{itheorem}[\ref{theo:pullback along un marked cartesian fibration}]
Let $p:X\to A$ be a classified left cartesian fibration. Then the functor $p^*:\ocatm_{/A}\to \ocatm_{/X}$ preserves colimits.
\end{itheorem}

The third subsection is devoted to the proof of the following theorem
\begin{itheorem}[\ref{theo:left cart stable by colimit}]
Let $A$ be an $\io$-category and $F:I\to \ocatm_{/A^\sharp}$ be a diagram that is pointwise a left cartesian fibration. The induced morphism 
$\colim_IF$ is a left cartesian fibration over $A^\sharp$.
\end{itheorem}



In the fourth subsection we study \textit{smooth} and \textit{proper} morphisms and we obtain the following expected result:
\begin{iprop}[\ref{prop:quillent theorem A}]
For a morphism $X\to A^\sharp$, and an object $a$ of $A$, we denote by $X_{/a}$ the marked $\io$-category fitting in the following cartesian squares. 
% https://q.uiver.app/#q=WzAsNCxbMSwxLCJBXlxcc2hhcnAiXSxbMCwxLCJBXlxcc2hhcnBfe2EvfSJdLFsxLDAsIlgiXSxbMCwwLCJYX3thL30iXSxbMSwwXSxbMiwwXSxbMywyXSxbMywxXSxbMywwLCIiLDEseyJzdHlsZSI6eyJuYW1lIjoiY29ybmVyIn19XV0=
\[\begin{tikzcd}
	{X_{a/}} & X \\
	{A^\sharp_{a/}} & {A^\sharp}
	\arrow[from=2-1, to=2-2]
	\arrow[from=1-2, to=2-2]
	\arrow[from=1-1, to=1-2]
	\arrow[from=1-1, to=2-1]
	\arrow["\lrcorner"{anchor=center, pos=0.125}, draw=none, from=1-1, to=2-2]
\end{tikzcd}\]
We denote by $\bot:\ocatm\to \ocat$ the functor sending a marked $\io$-category to its localization by marked cells.
\begin{enumerate}
\item Let $E$, $F$ be two elements of $\ocatm_{/A^\sharp}$ corresponding to morphisms $X\to A^\sharp$, $Y\to A^\sharp$, and
 $\phi:E\to F$ a morphism between them. We denote by $\Fb E$ and $\Fb F$ the left cartesian fiborant replacement of $E$ and $F$. 
 
The induced morphism $\Fb\phi:\Fb E\to \Fb F$ is an equivalence if and only if for any object $a$ of $A$, the induced morphism 
$$\bot X_{/a}\to \bot Y_{/a}$$ 
is an equivalence of $\io$-categories.
\item A morphism $X\to A^\sharp$ is initial if and only if for any object $a$ of $A$, $\bot X_{/a}$ is the terminal $\io$-category.
\end{enumerate}
\end{iprop}



Finally, in the last subsection, for a marked $\io$-category $I$, we define and study a (huge) $\io$-category $\uLCartc(I)$ that has classified left cartesian fibrations as objects and morphisms between classified left cartesian fibrations as arrows.



\paragraph{Chapter \ref{chapter:The io-category of small io-categories}.}
This chapter aims to establish analogs of the fundamental categorical constructions to the $\io$ case. In the first section, we define the $\io$-category of small $\io$-categories $\uni$ (paragraph \ref{para:defi of uni}), and we prove a first incarnation of the Grothendieck construction:
\begin{icor}[\ref{cor: Grt equivalence}]
Let $\uni$ be the $\io$-category of small $\io$-categories, and $A$ an $\io$-category. There is an equivalence
$$\int_A:\Hom(A,\uni)\to \tau_0 \LCart(A^\sharp).$$
where $\tau_0 \LCart(A^\sharp)$ is the $\infty$-groupoid of left cartesian fibrations over $A^\sharp$ with small fibers.
\end{icor}
Given a functor $f:A\to \uni$, the left cartesian fibration $\int_Af$ is a colimit (computed in $\ocatm_{/A^\sharp}$) of
a simplicial object whose value on $n$ is of shape
$$\coprod_{x_0,...,x_n:A_0}X(x_0)^\flat\times\hom_A(x_0,...,x_n)^\flat\times A^\sharp_{x_n/}\to A^\sharp$$
This formula is similar to the one given in \cite{Gepner_Lax_colimits_and_free_fibration}
 for $\iun$-categories, and to the one given in \cite{Warren_the_strict_omega_groupoid_interpretation_of_type_theory} for strict $\omega$-categories.

We also prove a univalence result:

\begin{icor}[\ref{cor:univalence}]
Let $I$ be a marked $\io$-category. We denote by $I^\sharp$ the marked $\io$-category obtained from $I$	 by marking all cells and $\iota:I\to I^\sharp$ the induced morphism. There is a natural correspondence between \begin{enumerate}
\item functors
$f:I\otimes [1]^\sharp\to \uni^\sharp,$

\item pairs of small left cartesian fibration $X\to I^\sharp$, $Y\to I^\sharp$ together with diagrams 
% https://q.uiver.app/?q=WzAsNixbMSwyLCJJIl0sWzAsMSwiXFxpb3RhXipZIl0sWzMsMiwiSV5cXHNoYXJwIl0sWzIsMSwiWSJdLFsxLDAsIlxcaW90YV4qWCJdLFszLDAsIlgiXSxbMCwyLCJcXGlvdGEiLDJdLFsxLDBdLFszLDJdLFsxLDNdLFs0LDBdLFs1LDJdLFs0LDVdLFs0LDEsIlxccGhpIiwxXSxbNCw2LCIiLDAseyJsZXZlbCI6MSwic3R5bGUiOnsibmFtZSI6ImNvcm5lciJ9fV0sWzEsNiwiIiwwLHsibGV2ZWwiOjEsInN0eWxlIjp7Im5hbWUiOiJjb3JuZXIifX1dXQ==
\[\begin{tikzcd}
	& {\iota^*X} && X \\
	{\iota^*Y} && Y \\
	& I && {I^\sharp}
	\arrow[""{name=0, anchor=center, inner sep=0}, "\iota"', from=3-2, to=3-4]
	\arrow[from=2-1, to=3-2]
	\arrow[from=2-3, to=3-4]
	\arrow[from=2-1, to=2-3]
	\arrow[from=1-2, to=3-2]
	\arrow[from=1-4, to=3-4]
	\arrow[from=1-2, to=1-4]
	\arrow["\phi"{description}, from=1-2, to=2-1]
	\arrow["\lrcorner"{anchor=center, pos=0.125}, draw=none, from=1-2, to=0]
	\arrow["\lrcorner"{anchor=center, pos=0.125}, draw=none, from=2-1, to=0]
\end{tikzcd}\]
\end{enumerate}
\end{icor}

Recall that if $I$ is of shape $B^\sharp$, then the underlying $\io$-category of $B^\sharp\otimes[1]^\sharp$ is $B\times [1]$, and if $I$ is of shape $B^\flat$, the underlying $\io$-category of $B^\flat\otimes[1]^\sharp$ is $B\otimes[1]$. On the other hand, if $I$ is $B^\sharp$, $\iota$ is the identity, and $\phi$ then preserves all cartesian liftings, and if $I$ is $B^\flat$, $\phi$ doesn't need to preserve cartesian liftings.

By varying the marking, we can continuously move from the cartesian product with the interval to the Gray product with the interval on one side, and on the other side, we can continuously move from morphisms between left cartesian fibrations that preserve the marking to the ones that do not preserve it \textit{a priori}.

Eventually, we also get an $\io$-functorial Grothendieck construction, expressed by the following corollary:

\begin{icor}[\ref{cor:lcar et hom}]
Let $A$ be a $\U$-small $\io$-category.
Let $\uLCart(A^\sharp)$ be the $\io$-category of small left cartesian fibrations over $A^\sharp$. 
There is an equivalence
$$\uHom(A,\uni)\sim \uLCart(A^\sharp)$$
natural in $A$.
\end{icor}


In the second section of this chapter, for a locally small $\io$-category $C$, we construct the Yoneda embedding, which is a functor $y:C\to \widehat{C}$ where $\widehat{C}:=\uHom(C^t,\uni)$. We prove the Yoneda lemma:
\begin{itheorem}[\ref{theo:Yoneda ff}]
The Yoneda embedding is fully faithful.
\end{itheorem}
\begin{itheorem}[\ref{theo:Yoneda lemma}]
Let $C$ be an $\io$-category. There is an equivalence between the functor
$$\hom_{\w{C}}(y_{\uvar},\uvar):C^t\times \w{C}\to \uni$$ and
the functor 
$$ev:C^t\times \w{C}\to \uni .$$
\end{itheorem}
In the last three sections, we use these results to study and demonstrate the basic properties of adjunctions, lax (co)limits, and left Kan extensions.

We begin by studying adjunctions, and we establish the following expected result.
\begin{itheorem}[\ref{theo:two adjunction definition}]
Let $u:C\to D$ and $v:D\to C$ be two functors between locally $\U$-small $\io$-categories. 
The two following are equivalent. 
\begin{enumerate}
\item The pair $(u,v)$ admits an adjoint structure.
\item Their exists a pair of natural transformations $\mu: id_C \to vu$ and $\epsilon:uv\to id_D$ together with equivalences $(\epsilon\circ_0 u)\circ_1(u\circ_0 \mu) \sim id_{u}$ and $(v\circ_0 \epsilon)\circ_1 (\mu \circ_0 v )\sim id_{v}$.
\end{enumerate}
\end{itheorem}

In the next subsection, given a morphism $f:I\to C^\sharp$ between marked $\io$-categories, we define the notion of lax colimit and lax limit for the functor $f$. If $f$ admits such a lax colimit, for any $1$-cell $i:a\to b$ in $I$, we have a triangle
% https://q.uiver.app/#q=WzAsNCxbMCwwXSxbMCwxLCJGKGEpIl0sWzEsMSwiXFxsYXhjb2xpbV9JRiJdLFsxLDAsIkYoYikiXSxbMSwzLCJGKGkpIiwwLHsiY3VydmUiOi01fV0sWzEsMl0sWzMsMSwiIiwxLHsic2hvcnRlbiI6eyJzb3VyY2UiOjMwLCJ0YXJnZXQiOjMwfSwibGV2ZWwiOjJ9XSxbMCwxLCIiLDAseyJzdHlsZSI6eyJib2R5Ijp7Im5hbWUiOiJub25lIn0sImhlYWQiOnsibmFtZSI6Im5vbmUifX19XSxbMywyXV0=
\[\begin{tikzcd}
	{} & {F(b)} \\
	{F(a)} & {\laxcolim_IF}
	\arrow["{F(i)}", curve={height=-30pt}, from=2-1, to=1-2]
	\arrow[from=2-1, to=2-2]
	\arrow[shorten <=8pt, shorten >=8pt, Rightarrow, from=1-2, to=2-1]
	\arrow[draw=none, from=1-1, to=2-1]
	\arrow[from=1-2, to=2-2]
\end{tikzcd}\]
If $i$ is marked, the preceding $2$-cell is an equivalence. 
For any $2$-cell $u:i\to j$, we have a diagram
% https://q.uiver.app/#q=WzAsNyxbMCwxLCJGKGEpIl0sWzEsMCwiRihiKSJdLFsxLDEsIlxcbGF4Y29saW1fSUYiXSxbMiwxLCJGKGEpIl0sWzMsMCwiRihiKSJdLFszLDEsIlxcbGF4Y29saW1fSUYiXSxbMiwwXSxbMCwxLCJGKGkpIiwxXSxbMCwyXSxbMSwyXSxbMSwyXSxbMCwxLCJGKGopIiwwLHsiY3VydmUiOi01fV0sWzMsNCwiRihqKSIsMCx7ImN1cnZlIjotNX1dLFszLDVdLFs0LDVdLFs0LDMsIiIsMSx7InNob3J0ZW4iOnsic291cmNlIjozMCwidGFyZ2V0IjozMH0sImxldmVsIjoyfV0sWzYsMywiIiwwLHsic3R5bGUiOnsiYm9keSI6eyJuYW1lIjoibm9uZSJ9LCJoZWFkIjp7Im5hbWUiOiJub25lIn19fV0sWzEwLDgsIiIsMSx7Im9mZnNldCI6Miwic2hvcnRlbiI6eyJzb3VyY2UiOjQwLCJ0YXJnZXQiOjQwfX1dLFsxMSw3LCIiLDEseyJzaG9ydGVuIjp7InNvdXJjZSI6MjAsInRhcmdldCI6MjB9fV0sWzEwLDE2LCIiLDEseyJvZmZzZXQiOi0xLCJzaG9ydGVuIjp7InNvdXJjZSI6MzAsInRhcmdldCI6MzB9LCJsZXZlbCI6MSwic3R5bGUiOnsiaGVhZCI6eyJuYW1lIjoibm9uZSJ9fX1dLFsxMCwxNiwiIiwxLHsic2hvcnRlbiI6eyJzb3VyY2UiOjMwLCJ0YXJnZXQiOjMwfSwibGV2ZWwiOjF9XSxbMTAsMTYsIiIsMSx7Im9mZnNldCI6MSwic2hvcnRlbiI6eyJzb3VyY2UiOjMwLCJ0YXJnZXQiOjMwfSwibGV2ZWwiOjEsInN0eWxlIjp7ImhlYWQiOnsibmFtZSI6Im5vbmUifX19XV0=
\[\begin{tikzcd}
	& {F(b)} & {} & {F(b)} \\
	{F(a)} & {\laxcolim_IF} & {F(a)} & {\laxcolim_IF}
	\arrow[""{name=0, anchor=center, inner sep=0}, "{F(i)}"{description}, from=2-1, to=1-2]
	\arrow[""{name=1, anchor=center, inner sep=0}, from=2-1, to=2-2]
	\arrow[from=1-2, to=2-2]
	\arrow[""{name=2, anchor=center, inner sep=0}, from=1-2, to=2-2]
	\arrow[""{name=3, anchor=center, inner sep=0}, "{F(j)}", curve={height=-30pt}, from=2-1, to=1-2]
	\arrow["{F(j)}", curve={height=-30pt}, from=2-3, to=1-4]
	\arrow[from=2-3, to=2-4]
	\arrow[from=1-4, to=2-4]
	\arrow[shorten <=8pt, shorten >=8pt, Rightarrow, from=1-4, to=2-3]
	\arrow[""{name=4, anchor=center, inner sep=0}, draw=none, from=1-3, to=2-3]
	\arrow[shift right=2, shorten <=12pt, shorten >=12pt, Rightarrow, from=2, to=1]
	\arrow[shorten <=4pt, shorten >=4pt, Rightarrow, from=3, to=0]
	\arrow[shift left=0.7, shorten <=14pt, shorten >=16pt, no head, from=2, to=4]
	\arrow[shorten <=14pt, shorten >=14pt, from=2, to=4]
	\arrow[shift right=0.7, shorten <=14pt, shorten >=16pt, no head, from=2, to=4]
\end{tikzcd}\]
If $u$ is marked, the $3$-cell is an equivalence. We can continue these diagrams in higher dimensions and we have
similar assertions for lax limits.
The marking therefore allows us to play on the "lax character" of the universal property that the lax colimit must verify.


After providing several characterizations of lax colimits and limits, we prove the following result:
\begin{itheorem}[\ref{theo:presheaevs colimi of representable}]
Let $C$ be a $\U$-small $\io$-category. Let $f$ be an object of $\w{C}$. We define $C^\sharp_{/f}$ as the following pullback
% https://q.uiver.app/#q=WzAsNCxbMSwwLCJcXHd7Q31eXFxzaGFycF97L2Z9Il0sWzEsMSwiXFx3e0N9Xlxcc2hhcnAiXSxbMCwxLCJDXlxcc2hhcnAiXSxbMCwwLCJDXlxcc2hhcnBfey9mfSJdLFszLDJdLFszLDBdLFswLDFdLFsyLDEsInleXFxzaGFycCIsMl1d
\[\begin{tikzcd}
	{C^\sharp_{/f}} & {\w{C}^\sharp_{/f}} \\
	{C^\sharp} & {\w{C}^\sharp}
	\arrow[from=1-1, to=2-1]
	\arrow[from=1-1, to=1-2]
	\arrow[from=1-2, to=2-2]
	\arrow["{y^\sharp}"', from=2-1, to=2-2]
\end{tikzcd}\]
The colimit of the functor 
$\pi:C^\sharp_{/f}\to C^\sharp\xrightarrow{y^\sharp} \w{C}^\sharp$ is $f$.
\end{itheorem}

We conclude this chapter by studying Kan extensions.


\phantomsection
\addcontentsline{toc}{section}{Notice of authority}
\section*{Notice of authority}

The chapter \ref{chapter:Studies of the complicial model} is a shorter version of the preprint \cite{Loubaton_dualities_in_the_complicial_model}. Chapter \ref{chapter:complicial set as a model of io categories} is almost identical to the preprint \cite{Loubaton_complicial_sets_as_a_model_of_infini_n_categories}. During this thesis, two other papers were written: \cite{Loubaton_condition_de_kan} (in progress of publication at the SMF) and \cite{Henry_an_inductive_model_structure_for_infini_categories} (in collaboration with Simon Henry). Although the topics are similar, the questions addressed are quite different, and these papers are thus not included in the present text.

%\bibliographystyle{alpha}
%\bibliography{../../header/biblio}
%\end{document}


