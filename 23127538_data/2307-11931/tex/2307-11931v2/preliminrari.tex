%\documentclass[12pt]{book}
%\usepackage{index}
%\makeindex
%\renewcommand\indexname{Index of notions}
%\newindex{notation}{adx}{and}{Index of symbols}
%\newindex{notion}{bdx}{bnd}{Index of notions}
%\usepackage{tikz}
\usepackage{xcolor,xspace}
\usepackage{url}
\usepackage{epsfig,graphicx,endnotes,kotex,subfigure,multirow,amsmath,algorithm,algpseudocode}
\newcommand\StateX{\Statex\hspace{\algorithmicindent}}%
%\usepackage{breakurl}
%\usepackage[sort,space]{cite}
\usepackage{balance}
%\usepackage{tabularx}
\usepackage{enumitem}
\usepackage{flushend}
\usepackage[T1]{fontenc}
\usepackage{color,soul}
\hyphenation{op-tical net-works semi-conduc-tor}
%\usepackage{filecontents}
%\usepackage{booktabs} % For formal tables
\usepackage{amsthm}
\newtheorem{theorem}{Theorem}
\newtheorem{corollary}{Corollary}
\newtheorem{lemma}{Lemma}
\renewcommand{\qedsymbol}{\rule{0.7em}{0.7em}}

%\newcommand\notion[1]{\textit{#1}\index[notion]{#1}}
\newcommand\wcnotion[2]{\textit{#1}\index[notion]{#2}}
\newcommand\wcnotionsym[3]{\textit{#1}\index[notation]{#2}\index[notion]{#3}}
\newcommand\wcsnotion[3]{\textit{#1}\index[notion]{#2!\textit{#3}}}
\newcommand\snotion[2]{\textit{#1}\index[notion]{#1!\textit{#2}}}
\newcommand\snotionsym[3]{\textit{#1}\index[notion]{#1!\textit{#3}}\index[notation]{#2!\textit{#3}}}
\newcommand\wcsnotionsym[4]{\textit{#1}\index[notation]{#2!\textit{#4}}\index[notion]{#3!\textit{#4}}}

\newcommand\wcnotation[2]{\textit{#1}\index[notation]{#2}}
\newcommand\wcsnotation[3]{\textit{#1}\index[notation]{#2!\textit{#3}}}

\newcommand\sym[1]{\index[notation]{#1}}
\newcommand\ssym[2]{\index[notation]{#1!\textit{#2}}}

\newcommand{\exclam}{!}





\newcommand{\Ab}{\mathbb{A}} 
\newcommand{\Zb}{\mathbb{Z}} 
\newcommand{\Eb}{\mathbb{E}} 
\newcommand{\Nb}{\mathbb{N}}
\newcommand{\Tb}{\mathbf{T}} 
\newcommand{\Yb}{\mathbb{Y}} 
\newcommand{\Ib}{\mathbb{I}} 
\newcommand{\Ob}{\mathbb{O}} 
\newcommand{\Pb}{\mathbb{P}} 
\newcommand{\Qb}{\mathbb{Q}} 
\newcommand{\Sb}{\mathbb{S}} 
\newcommand{\Hb}{\mathbb{H}} 
\newcommand{\Jb}{\mathbf{J}} 
\newcommand{\Kb}{\mathbb{K}} 
\newcommand{\Mb}{\mathbb{M}} 
\newcommand{\Wb}{\mathbf{W}} 
\newcommand{\Xb}{\mathbb{X}} 
\newcommand{\Cb}{\mathbf{C}}
\newcommand{\Vb}{\mathbb{V}}
\newcommand{\Bb}{\mathbb{B}}


\newcommand{\Acal}{\mathcal{A}} 
\newcommand{\Zcal}{\mathcal{Z}} 
\newcommand{\Ecal}{\mathcal{E}} 
\newcommand{\Rcal}{\mathcal{R}} 
\newcommand{\Tcal}{\mathcal{T}} 
\newcommand{\Ycal}{\mathcal{Y}} 
\newcommand{\Ucal}{\mathcal{U}} 
\newcommand{\Ical}{\mathcal{I}} 
\newcommand{\Ocal}{\mathcal{O}} 
\newcommand{\Pcal}{\mathcal{P}} 
\newcommand{\Qcal}{\mathcal{Q}} 
\newcommand{\Scal}{\mathcal{S}} 
\newcommand{\Dcal}{\mathcal{D}} 
\newcommand{\Fcal}{\mathcal{F}} 
\newcommand{\Gcal}{\mathcal{G}} 
\newcommand{\Hcal}{\mathcal{H}} 
\newcommand{\Jcal}{\mathcal{J}} 
\newcommand{\Kcal}{\mathcal{K}} 
\newcommand{\Lcal}{\mathcal{L}} 
\newcommand{\Mcal}{\mathcal{M}} 
\newcommand{\Wcal}{\mathcal{W}} 
\newcommand{\Xcal}{\mathcal{X}} 
\newcommand{\Ccal}{\mathcal{C}} 
\newcommand{\Vcal}{\mathcal{V}} 
\newcommand{\Bcal}{\mathcal{B}} 
\newcommand{\Ncal}{\mathcal{N}} 


\newcommand{\Ago}{\mathfrak{A}} 
\newcommand{\Zgo}{\mathfrak{Z}} 
\newcommand{\Ego}{\mathfrak{E}} 
\newcommand{\Rgo}{\mathfrak{R}} 
\newcommand{\Tgo}{\mathfrak{T}} 
\newcommand{\Ygo}{\mathfrak{Y}} 
\newcommand{\Ugo}{\mathfrak{U}} 
\newcommand{\Igo}{\mathfrak{I}} 
\newcommand{\Ogo}{\mathfrak{O}} 
\newcommand{\Pgo}{\mathfrak{P}} 
\newcommand{\Qgo}{\mathfrak{Q}} 
\newcommand{\Sgo}{\mathfrak{S}} 
\newcommand{\Dgo}{\mathfrak{D}} 
\newcommand{\Fgo}{\mathfrak{F}} 
\newcommand{\Ggo}{\mathfrak{G}} 
\newcommand{\Hgo}{\mathfrak{H}} 
\newcommand{\Jgo}{\mathfrak{J}} 
\newcommand{\Kgo}{\mathfrak{K}} 
\newcommand{\Lgo}{\mathfrak{L}} 
\newcommand{\Mgo}{\mathfrak{M}} 
\newcommand{\Wgo}{\mathfrak{W}} 
\newcommand{\Xgo}{\mathfrak{X}} 
\newcommand{\Cgo}{\mathfrak{C}} 
\newcommand{\Vgo}{\mathfrak{V}} 
\newcommand{\Bgo}{\mathfrak{B}} 
\newcommand{\Ngo}{\mathfrak{N}}



\newcommand{\sslash}{\mathbin{/\mkern-6mu/}}

\newcommand{\note}[1]{{\color{red}#1}}

\def\-{\raisebox{.75pt}{-}}


\newcommand{\uvar}{\_}


%basic notation
\newcommand{\id}{\text{Id}}
\newcommand{\Db}{\mathbf{D}} 
\DeclareMathOperator*{\dom}{dom}
\DeclareMathOperator*{\codom}{codom}
\DeclareMathOperator{\tw}{tw}


%derived notation
\newcommand{\Rb}{\mathbf{R}} 
\newcommand{\Lb}{\mathbf{L}} 
\newcommand{\Fb}{\mathbf{F}} 
\DeclareMathOperator{\Gb}{G} 
  
%ambiguous notation 
\DeclareMathOperator{\N}{N}
\DeclareMathOperator{\T}{T}
\DeclareMathOperator{\J}{J}


%set of maps
\DeclareMathOperator*{\W}{W}
\DeclareMathOperator*{\Wm}{tW}
\DeclareMathOperator*{\Wseg}{W_{Seg}}
\DeclareMathOperator*{\Wsat}{W_{Sat}}

\DeclareMathOperator*{\M}{M}
\DeclareMathOperator*{\Mm}{tM}
\DeclareMathOperator*{\Mseg}{M_{Seg}}
\DeclareMathOperator*{\Msat}{M_{Sat}}

\DeclareMathOperator*{\I}{I}
\DeclareMathOperator*{\F}{F}

%augmented directed complexes
\DeclareMathOperator*{\CDA}{ADC}
\DeclareMathOperator*{\CDAB}{ADC_B}

%categories
\newcommand\omegacat{\omega\mbox{-$\cat$}}
\DeclareMathOperator\Set{Set}
\DeclareMathOperator\Sp{Sp}

%infini groupoids
\DeclareMathOperator*{\Sq}{Sq}
\DeclareMathOperator*{\Li}{Li}
\DeclareMathOperator{\Hom}{Hom}


%infini 1 categories
\DeclareMathOperator*{\Lfib}{LFib}
\DeclareMathOperator*{\Rfib}{RFib}

\DeclareMathOperator*{\LCartoperator}{LCart}
\DeclareMathOperator*{\core}{core}
\newcommand{\LCart}{\mbox{$\LCartoperator$}}

\newcommand{\LCartc}{\mbox{$\LCartoperator$}^c}
\DeclareMathOperator*{\RCart}{RCart}
\DeclareMathOperator*{\RCartc}{RCart^c}




%infini omega categories
\newcommand{\uLCart}{\underline{\LCartoperator}}
\newcommand{\uLCartc}{\underline{\LCartoperator}^c}
\newcommand{\uRCart}{\underline{RCart}}
\newcommand{\uRCartc}{\underline{RCart}^c}

\DeclareMathOperator{\uHom}{\underline{Hom}}
\DeclareMathOperator{\gHom}{\underline{Hom}_{\ominus}}
\DeclareMathOperator{\Map}{Map}
\DeclareMathOperator{\im}{Im}

\newcommand{\uni}{\underline{\omega}}
\newcommand\w[1]{\widehat{#1}}

%functors
\DeclareMathOperator*{\ev}{ev}
\DeclareMathOperator*{\Arr}{Arr}
\newcommand{\Noiun}{\N_{\tiny{(\omega,1)}}}


\newcommand{\colim}{\operatornamewithlimits{colim}}
\newcommand{\laxcolim}{\operatornamewithlimits{laxcolim}}
\newcommand{\laxlim}{\operatornamewithlimits{laxlim}}


%prefixes
\DeclareMathOperator{\Lan}{Lan}
\DeclareMathOperator{\Ran}{Ran}
\newcommand\iun{(\infty,1)}
\newcommand\io{(\infty,\omega)}
\newcommand\ioun{(\infty,\omega,1)}
\newcommand\zoun{(0,\omega,1)}
\newcommand\zo{(0,\omega)}

%leibnitz products
\DeclareMathOperator{\hstar}{\hat{\star}}
\DeclareMathOperator{\htimes}{\hat{\times}}
\DeclareMathOperator{\hotimes}{\hat{\otimes}}


%Gray operations
\DeclareMathOperator{\costarindex}{f}
\newcommand{\costar}{\mathbin{\overset{co}{\star}}}
\newcommand{\fwedge}{\mathbin{\rotatebox[origin=c]{270}{$\gtrdot$}}}


%inclassable
\newcommand{\invamalg}{\mathbin{\rotatebox[origin=c]{180}{$\amalg$}}}
\DeclareMathOperator{\botimes}{\bar{\otimes}}
\DeclareMathOperator\cst{cst}
\DeclareMathOperator\Operatormark{mk}
\newcommand{\mk}{\Operatormark}

%category theory
\DeclareMathOperator\Fun{Fun}
\DeclareMathOperator\Nat{Nat}
\DeclareMathOperator\End{End}



%fundamental notation
\DeclareMathOperator\mcat{cat_m}
\DeclareMathOperator\cat{cat}
\DeclareMathOperator\grd{grd}
\DeclareMathOperator\R{R}

\newcommand\ocat{(\infty,\omega)\mbox{-$\cat$}}
\newcommand\ouncat{(\infty,\omega,1)\mbox{-$\cat$}}
\newcommand\ocatm{{(\infty,\omega)\mbox{-$\mcat$}}}
\newcommand\zocatm{(0,\omega)\mbox{-$\mcat$}}
\newcommand\zocat{(0,\omega)\mbox{-$\cat$}}
\DeclareMathOperator\zocatB{\zocat_B}
\newcommand\icat{(\infty,1)\mbox{-$\cat$}}
\newcommand\qcat{\mbox{Q$\cat$}}
\newcommand\ncat[1]{(\infty, #1)\mbox{-$\cat$}}
\newcommand\zncat[1]{(0, #1)\mbox{-$\cat$}}
\newcommand\igrd{\infty\mbox{-$\grd$}}



\DeclareMathOperator{\OperatorinfiniPsh}{Psh^\infty}
\DeclareMathOperator{\OperatorinfinitPsh}{tPsh^\infty}
\DeclareMathOperator{\OperatorPsh}{Psh}
\DeclareMathOperator{\OperatormPsh}{mPsh}
\DeclareMathOperator{\OperatortPsh}{tPsh}
\newcommand\iPsh[1]{\OperatorinfiniPsh({#1})}
\newcommand\tiPsh[1]{\OperatorinfinitPsh({#1})}
\newcommand\Psh[1]{\OperatorPsh({#1})}
\newcommand\ssetPsh[1]{\OperatorPsh_\Delta({#1})}
\newcommand\tPsh[1]{\OperatortPsh({#1})}
\newcommand\tPshM[1]{{\OperatortPsh}_M({#1})}
\newcommand\mPsh[1]{\OperatormPsh({#1})}
\newcommand\mPshM[1]{{\OperatormPsh}_M({#1})}

%segal stuff
\DeclareMathOperator{\OperatorSeg}{Seg}
\DeclareMathOperator{\OperatortSeg}{tSeg}
\DeclareMathOperator{\OperatormSeg}{mSeg}
\newcommand\Seg{\OperatorSeg}
\newcommand\mSeg{\OperatormSeg}
\newcommand\stratSeg{\OperatortSeg}

%simplicial variations
\DeclareMathOperator{\Sset}{\Psh{\Delta}}
\newcommand{\mSset}{\mPsh{\Delta}}
\newcommand{\stratSset}{\tPsh{\Delta}}


%univers
\DeclareMathOperator{\U}{\mathbf{U}}
\DeclareMathOperator{\V}{\mathbf{V}}
\DeclareMathOperator{\Wcard}{\mathbf{W}}
\DeclareMathOperator{\Z}{\mathbf{Z}}



%Grothendieck constructions
\newcommand{\ringpartial}{\mathring{\partial}}
%
%
%\usepackage[inline]{showlabels}
%
%\usepackage{fancyhdr}
%\usepackage{titlesec}
%\usepackage{textcase}
%
%\pagestyle{fancy}
%
%
%\title{\Huge{Theory and models of $(\infty,\omega)$-categories}}
%\author{Félix Loubaton}
%\date{}
%\linespread{1.2}	
%\geometry{a4paper,top=3cm,bottom=4cm,left=1.5cm,right=3cm, heightrounded,bindingoffset=5mm}	
%
%
%\fancyhf{}
%\fancyhfoffset[RO,LE]{0.5cm}
%\fancyhfoffset[LE,RO]{0.5cm}
%
%\fancyhead[RO]{\rmfamily\nouppercase{\rightmark}}
%\fancyhead[LE]{\rmfamily\nouppercase{\leftmark}}
%\fancyfoot[C]{\thepage}
%
%\begin{document}


\chapter{The category of $\zo$-categories}
\label{chapter:The category of zocategories}

\minitoc
\vspace{1cm}
The first section is devoted to the definition of $\zo$-categories and of the category $\Theta$ of Joyal. We also show that the category $\Theta$ presents the category of $\zo$-categories, and we also exhibit an other presentation of this category (corollary \ref{cor:changing theta}).


The second section begins with a review of Steiner theory, which is an extremely useful tool for providing concise and computational descriptions of $\zo$-categories. Following Ara and Maltsiniotis, we employ this theory to define the Gray tensor product, denoted by $\otimes$, in $\zo$-categories. We then introduce the Gray operations, starting with the Gray cylinder $\uvar\otimes[1]$ which is the Gray tensor product with the directed interval $[1]:=0\to 1$. Then, we have the Gray cone and Gray $\circ$-cone, denoted by $\uvar\star 1$ and $1\costar \uvar$, that send an $\zo$-category $C$ onto the following pushouts:
% https://q.uiver.app/#q=WzAsOCxbNCwwLCJDXFxvdGltZXNbMV0iXSxbMywwLCJDXFxvdGltZXNcXHswXFx9Il0sWzMsMSwiMSJdLFs0LDEsIjFcXGNvc3RhciBDIl0sWzAsMCwiQ1xcb3RpbWVzXFx7MVxcfSJdLFsxLDAsIkNcXG90aW1lc1sxXSJdLFswLDEsIjEiXSxbMSwxLCJDXFxzdGFyIDEiXSxbMCwzXSxbMSwyXSxbMiwzXSxbMSwwXSxbNSw3XSxbNiw3XSxbNCw2XSxbNCw1XSxbNyw0LCIiLDEseyJzdHlsZSI6eyJuYW1lIjoiY29ybmVyIn19XSxbMywxLCIiLDEseyJzdHlsZSI6eyJuYW1lIjoiY29ybmVyIn19XV0=
\[\begin{tikzcd}
	{C\otimes\{1\}} & {C\otimes[1]} && {C\otimes\{0\}} & {C\otimes[1]} \\
	1 & {C\star 1} && 1 & {1\costar C}
	\arrow[from=1-5, to=2-5]
	\arrow[from=1-4, to=2-4]
	\arrow[from=2-4, to=2-5]
	\arrow[from=1-4, to=1-5]
	\arrow[from=1-2, to=2-2]
	\arrow[from=2-1, to=2-2]
	\arrow[from=1-1, to=2-1]
	\arrow[from=1-1, to=1-2]
	\arrow["\lrcorner"{anchor=center, pos=0.125, rotate=180}, draw=none, from=2-2, to=1-1]
	\arrow["\lrcorner"{anchor=center, pos=0.125, rotate=180}, draw=none, from=2-5, to=1-4]
\end{tikzcd}\]


We also present a formula that illustrates the interaction between the suspension and the Gray cylinder. As this formula plays a crucial role in both Part I and Part II, we provide its intuition at this stage.

 If $A$ is any $\zo$-category, the suspension of $A$, denoted by $[A,1]$, is the $\zo$-category having two objects - denoted by $0$ and $1$- and such that 
$$\Hom_{[A,1]}(0,1) := A,~~~\Hom_{[A,1]}(1,0) := \emptyset,~~~\Hom_{[A,1]}(0,0)=\Hom_{[A,1]}(1,1):=\{id\}.$$
We also define $[1]\vee[A,1]$ as the gluing of $[1]$ and $[A,1]$ along the $0$-target of $[1]$ and the $0$-source of $[A,1]$. We define similarly $[A,1]\vee[1]$.
These two objects come along with \textit{whiskerings}:
$$\triangledown:[A,1]\to [1]\vee [A,1] ~~~~\mbox{and}~~~~ \triangledown:[A,1] \to [A,1]\vee [1]$$ 
that preserve the extremal objects.


The $\zo$-category $[1]\otimes [1]$ is induced by the diagram:
% https://q.uiver.app/?q=WzAsNCxbMCwwLCIwMCJdLFswLDEsIjEwIl0sWzEsMSwiMTEiXSxbMSwwLCIwMSJdLFswLDFdLFsxLDJdLFswLDNdLFszLDJdLFszLDEsIiIsMSx7InNob3J0ZW4iOnsic291cmNlIjoyMCwidGFyZ2V0IjoyMH0sImxldmVsIjoyfV1d
\[\begin{tikzcd}
	00 & 01 \\
	10 & 11
	\arrow[from=1-1, to=2-1]
	\arrow[from=2-1, to=2-2]
	\arrow[from=1-1, to=1-2]
	\arrow[from=1-2, to=2-2]
	\arrow[shorten <=4pt, shorten >=4pt, Rightarrow, from=1-2, to=2-1]
\end{tikzcd}\]
and is then equal to the colimit of the following diagram: 
$$[1]\vee [1]\xleftarrow{\triangledown} [1]\hookrightarrow [[1],1]\hookleftarrow[1]\xrightarrow{\triangledown } [1]\vee [1].$$
The $\zo$-category $ [[1],1]\otimes [1]$ is induced by the diagram:
% https://q.uiver.app/?q=WzAsOCxbMSwwLCIwMSJdLFswLDAsIjAwIl0sWzAsMSwiMTAiXSxbMSwxLCIxMSJdLFsyLDAsIjAwIl0sWzMsMCwiMDEiXSxbMywxLCIxMSJdLFsyLDEsIjEwIl0sWzEsMF0sWzEsMl0sWzIsM10sWzAsM10sWzAsMiwiIiwxLHsic2hvcnRlbiI6eyJzb3VyY2UiOjIwLCJ0YXJnZXQiOjIwfSwibGV2ZWwiOjJ9XSxbNCw3XSxbNCw1XSxbNSw2XSxbNSw3LCIiLDEseyJzaG9ydGVuIjp7InNvdXJjZSI6MjAsInRhcmdldCI6MjB9LCJsZXZlbCI6Mn1dLFsxLDIsIiIsMix7ImN1cnZlIjo1fV0sWzcsNl0sWzUsNiwiIiwxLHsiY3VydmUiOi01fV0sWzksMTcsIiAiLDIseyJzaG9ydGVuIjp7InNvdXJjZSI6MjAsInRhcmdldCI6MjB9fV0sWzE5LDE1LCIgIiwyLHsic2hvcnRlbiI6eyJzb3VyY2UiOjIwLCJ0YXJnZXQiOjIwfX1dLFsxMSwxMywiIiwwLHsib2Zmc2V0IjotMSwic2hvcnRlbiI6eyJzb3VyY2UiOjIwLCJ0YXJnZXQiOjIwfSwibGV2ZWwiOjEsInN0eWxlIjp7ImhlYWQiOnsibmFtZSI6Im5vbmUifX19XSxbMTEsMTMsIiIsMix7Im9mZnNldCI6MSwic2hvcnRlbiI6eyJzb3VyY2UiOjIwLCJ0YXJnZXQiOjIwfSwibGV2ZWwiOjEsInN0eWxlIjp7ImhlYWQiOnsibmFtZSI6Im5vbmUifX19XSxbMTEsMTMsIiIsMSx7InNob3J0ZW4iOnsic291cmNlIjoyMCwidGFyZ2V0IjoyMH0sImxldmVsIjoxfV1d
\[\begin{tikzcd}
	00 & 01 & 00 & 01 \\
	10 & 11 & 10 & 11
	\arrow[from=1-1, to=1-2]
	\arrow[""{name=0, anchor=center, inner sep=0}, from=1-1, to=2-1]
	\arrow[from=2-1, to=2-2]
	\arrow[""{name=1, anchor=center, inner sep=0}, from=1-2, to=2-2]
	\arrow[shorten <=4pt, shorten >=4pt, Rightarrow, from=1-2, to=2-1]
	\arrow[""{name=2, anchor=center, inner sep=0}, from=1-3, to=2-3]
	\arrow[from=1-3, to=1-4]
	\arrow[""{name=3, anchor=center, inner sep=0}, from=1-4, to=2-4]
	\arrow[shorten <=4pt, shorten >=4pt, Rightarrow, from=1-4, to=2-3]
	\arrow[""{name=4, anchor=center, inner sep=0}, curve={height=30pt}, from=1-1, to=2-1]
	\arrow[from=2-3, to=2-4]
	\arrow[""{name=5, anchor=center, inner sep=0}, curve={height=-30pt}, from=1-4, to=2-4]
	\arrow["{ }"', shorten <=6pt, shorten >=6pt, Rightarrow, from=0, to=4]
	\arrow["{ }"', shorten <=6pt, shorten >=6pt, Rightarrow, from=5, to=3]
	\arrow[shift left=0.7, shorten <=6pt, shorten >=8pt, no head, from=1, to=2]
	\arrow[shift right=0.7, shorten <=6pt, shorten >=8pt, no head, from=1, to=2]
	\arrow[shorten <=6pt, shorten >=6pt, from=1, to=2]
\end{tikzcd}\]
and is then equal to the colimit of the following diagram: 
 $$[1]\vee[[1],1]\xleftarrow{\triangledown} [[1]\otimes\{0\},1]\hookrightarrow[[1]\otimes[1],1]\hookleftarrow [[1]\otimes\{1\},1]\xrightarrow{\triangledown}[[1],1]\vee[1]$$
We prove a formula that combines these two examples:

\begin{itheorem}[\ref{theo:appendice formula for otimes}]
In the category of $\zo$-categories, there exists an isomorphism, natural in $A$, between $[A,1]\otimes[1]$ and the colimit of the following diagram
% https://q.uiver.app/#q=WzAsNSxbMCwwLCJbMV1cXHZlZVtBLDFdIl0sWzEsMCwiW0FcXG90aW1lc1xcezBcXH0sMV0iXSxbMiwwLCIgW0FcXG90aW1lc1sxXSwxXSJdLFszLDAsIltBXFxvdGltZXNcXHsxXFx9LDFdIl0sWzQsMCwiW0EsMV1cXHZlZVsxXSJdLFsxLDAsIlxcdHJpYW5nbGVkb3duIiwyXSxbMywyXSxbMyw0LCJcXHRyaWFuZ2xlZG93biJdLFsxLDJdXQ==
\[\begin{tikzcd}
	{[1]\vee[A,1]} & {[A\otimes\{0\},1]} & { [A\otimes[1],1]} & {[A\otimes\{1\},1]} & {[A,1]\vee[1]}
	\arrow["\triangledown"', from=1-2, to=1-1]
	\arrow[from=1-4, to=1-3]
	\arrow["\triangledown", from=1-4, to=1-5]
	\arrow[from=1-2, to=1-3]
\end{tikzcd}\]
\end{itheorem} 

We also provide similar formulas for the \textit{Gray cone} and the \textit{Gray $\circ$-cone}.
\begin{itheorem}[\ref{theo:appendice formula for star}]
There is a natural identification between $1\costar [A,1]$ and the colimit of the following diagram
% https://q.uiver.app/#q=WzAsMyxbMCwwLCJbMV1cXHZlZVtBLDFdIl0sWzEsMCwiW0EsMV0iXSxbMiwwLCIgW0FcXHN0YXIgMSwxXSJdLFsxLDAsIlxcdHJpYW5nbGVkb3duIiwyXSxbMSwyXV0=
\[\begin{tikzcd}
	{[1]\vee[A,1]} & {[A,1]} & { [A\star 1,1]}
	\arrow["\triangledown"', from=1-2, to=1-1]
	\arrow[from=1-2, to=1-3]
\end{tikzcd}\]
There is a natural identification between $[A,1]\star 1$ and the colimit of the following diagram
% https://q.uiver.app/#q=WzAsMyxbMCwwLCIgWzFcXGNvc3RhciBBLDFdIl0sWzEsMCwiW0EsMV0iXSxbMiwwLCJbQSwxXVxcdmVlWzFdIl0sWzEsMF0sWzEsMiwiXFx0cmlhbmdsZWRvd24iXV0=
\[\begin{tikzcd}
	{ [1\costar A,1]} & {[A,1]} & {[A,1]\vee[1]}
	\arrow[from=1-2, to=1-1]
	\arrow["\triangledown", from=1-2, to=1-3]
\end{tikzcd}\]
\end{itheorem}




\section{Basic constructions}
\label{chapter:Basica construciton preliminaire}
\subsection{$\zo$-Categories}
\label{section:zocategories}
\p A \notion{globular set} is a presheaf on the \textit{category of globes} $\Gb$, which is the category induces by the diagram
% q.uiver.app/#q=WzAsNCxbMCwwLCJcXERiXzAiXSxbMSwwLCJcXERiXzEiXSxbMiwwLCJcXERiXzIiXSxbMywwLCIuLi4iXSxbMCwxLCJpXzBeKyIsMCx7Im9mZnNldCI6LTJ9XSxbMSwyLCJpXzFeKyIsMCx7Im9mZnNldCI6LTJ9XSxbMiwzLCJpXzNeKyIsMCx7Im9mZnNldCI6LTJ9XSxbMCwxLCJpXzBeLSIsMix7Im9mZnNldCI6Mn1dLFsxLDIsImlfMV4tIiwyLHsib2Zmc2V0IjoyfV0sWzIsMywiaV8zXi0iLDIseyJvZmZzZXQiOjJ9XV0=
\[\begin{tikzcd}
	{\Db_0} & {\Db_1} & {\Db_2} & {...}
	\arrow["{i_0^+}", shift left=2, from=1-1, to=1-2]
	\arrow["{i_1^+}", shift left=2, from=1-2, to=1-3]
	\arrow["{i_3^+}", shift left=2, from=1-3, to=1-4]
	\arrow["{i_0^-}"', shift right=2, from=1-1, to=1-2]
	\arrow["{i_1^-}"', shift right=2, from=1-2, to=1-3]
	\arrow["{i_3^-}"', shift right=2, from=1-3, to=1-4]
\end{tikzcd}\]
with the relations $i_n^{+} i_{n-1}^\epsilon = i_n^{-} i_{n-1}^\epsilon $ for any $n>0$ and $\epsilon \in \{+,-\}$. We also denote by $i^{\epsilon}_k$ the map $\Db_{k} \to \Db_n$ for $k< n$ obtained by composing any string of arrows ending with $i^\epsilon_{k}$. These and the identity arrows are the only maps in the category $\Gb$.

If $X$ is a globular set, one denotes by $X_n$ the set $X(\Db_n)$. Its elements are called \wcsnotion{$n$-cells}{cell@$n$-cell}{for $\zo$-categories}. The $0$-cells are sometimes called \textit{objects}. The maps $X_n \to X_k$ induced by $i^\epsilon_k : \Db_k \to \Db_n$ is denoted by $\pi^\epsilon_k$.


\p
\label{para:def of omega cat}
An \wcnotion{$\omega$-category}{category@$\omega$-category} is a globular set $X$ together with
\begin{enumerate}
\item operations of \textit{compositions}
\[ X_n\times_{X_k} X_n\to X_n ~~~(0\leq k<n) \]
which associate to two $n$-cells $(x,y)$ verifying $\pi_k^-(x) = \pi_k^+(y)$, a $n$-cells $x\circ_ky$,
\item as well as \textit{units}
\[X_n\to X_{n+1}\]
which associate to an $n$-cell $x$, a $(n+1)$-cell $\Ib_x$, 
\end{enumerate}
and satisfying the following axioms:
\begin{enumerate}

\item $\forall x \in X_n, \pi^\epsilon_n(\Ib_x) = x $.

\item $\pi^+_k (x \circ_n y) = \pi_k^{+}(x)$ and $\pi^-_k(x \circ_n y) = \pi_k^-(y)$ whenever the composition is defined and $k \leqslant n$.

\item $\pi^\epsilon_k (x \circ_n y) = \pi_k^{\epsilon}(x) \circ_n \pi^\epsilon_k(y)$ whenever the composition is defined and $k > n$.

\item $ x \circ_n \Ib_{\pi^-_n x} = x$ and $ \Ib_{\pi^+_n x} \circ_n x = x$.

\item $(x \circ_n y) \circ_n z = x \circ_n (y \circ_n z) $ as soon as one of these is defined.

\item If $k <n$

\[ (x \circ_n y) \circ_k ( z \circ_n w) = (x \circ_k z) \circ_n (y \circ_k w) \]
when the left-hand side is defined.

\end{enumerate}
A $n$-cell $a$ is \textit{non trivial} if is not in the image of the application $\Ib:X_{n-1}\to X_n$.

A \textit{morphism of $\omega$-categories} is a map of globular sets commuting with both operations. The category of $\omega$-categories is denoted by \textit{$\omegacat$}.



\p
\index[notion]{globe@$n$-globe!for $\zo$-categories}
By abuse of notation, we also denote by \wcsnotation{$\Db_n$}{(da@$\Db_n$}{for $\zo$-categories} the $\omega$-category that admits for any $k<n$ only two $k$-non-trivial cells, denoted by $e_k^-$ and $e_k^+$, and a single $n$-non-trivial cell, denoted by $e_n$ verifying :
\[
\begin{array}{rcl}
\pi_l^{-}(e_k^\epsilon)= e_l^{-}&\pi_l^{+}(e_k^\epsilon)= e_l^{+}& \mbox{ for $l\leq k<n$}\\
\pi_l^{-}(e_n)= e_l^{-}&\pi_l^{+}(e_n)= e_l^{+}& \mbox{ for $l\leq n$}\\
\end{array}
\]

Remark furthermore that the $\omega$-category $\Db_n$ represents $n$-cells, in the sense that $\Hom(\Db_n,C)\cong C_n$. We will not make the difference between $n$-cells and the corresponding morphism of $\Db_n\to C$. 

The $\omega$-category $\partial\Db_n$ is obtained from $\Db_n$ by removing the $n$-cell $e_n$. We thus have a morphism
\[i_n: \partial\Db_n\to \Db_n.\]
Note that $\partial \Db_0 = \emptyset$.


\p 
We say that an $\zo$-category $X$ is a \notion{polygraph} if it can be constructed from the empty $\zo$-category by freely adding arrows with specified source and target. That is if $X$ can be obtained as a transfinite composition $\emptyset = X_0 \to X_1 \to \dots \to X_i \to \colim X_i = X$ where for each $i$, the map $X_i \to X_{i+1}$ is a pushout of $\coprod_S \partial \Db_n \to \coprod_S \Db_{n+1}$.

 An arrow of a polygraph is said to be a \emph{generator} if it is one of the arrows that has been freely added at some stage.
 

Each cell in a polygraph can be written as an iterated composite of generators or iterated unit of generators (not necessarily in a unique way). For a $n$-cell $f$, the set of generators of dimension $n$ that appear in such an expression (and even the number of times they appear) is the same for all such expressions. As a consequence, a iterated composition of non trivial cells is always non trivial.


 


\p \label{para:dualities strict case}
 For any subset $S$ of $\Nb^*$, we define the functor $(\uvar)^S:\omegacat\to \omegacat$ \ssym{((b49@$(\uvar)^S$}{for $\zo$-categories} sending a $\omega$-category $C$ to the category $C^S$ such that for any $n$, there is an isomorphism $C_n\to C_{n}^S$ that sends every $n$-cell $f$ to a cell $\overline{f}$ fulfilling
$$\pi_{n-1}^-(\overline{f})=\overline{\pi^+_{n-1}(f)}~~~~\pi_{n-1}^+(\overline{f})=\overline{\pi^-_{n-1}(f)}$$
if $i\in S$ and 
$$\pi_{n-1}^-(\overline{f})=\overline{\pi^-_{n-1}(f)}~~~~\pi_{n-1}^+(\overline{f})=\overline{\pi^+_{n-1}(f)}$$
if $i\notin S$.
These functors are called \snotion{dualities}{for $\zo$-categories} as they are inverse of themselves. Even if there are plenty of them, we will be interested in only a few of them. In particular, we have the \snotionsym{odd duality}{((b60@$(\uvar)^{op}$}{for $\zo$-categories} $(\uvar)^{op}$, corresponding to the set of odd integer, the \snotionsym{even duality}{((b50@$(\uvar)^{co}$}{for $\zo$-categories} $(\uvar)^{co}$, corresponding to the subset of non negative even integer, the \snotionsym{full duality}{((b80@$(\uvar)^{\circ}$}{for $\zo$-categories} $(\uvar)^{\circ}$, corresponding to $\Nb^*$ and the \snotionsym{transposition}{((b70@$(\uvar)^t$}{for $\zo$-categories} $(\uvar)^t$, corresponding to the singleton $\{1\}$. Eventually, we have equivalences
$$((\uvar)^{co})^{op}\sim (\uvar)^{\circ} \sim ((\uvar)^{op})^{co}.$$



\p Let $\Psh{\Gb}_{\bullet,\bullet}$ be the category of globular set with two distinguished points, i.e. of triples $(X,a,b)$ where $a$ and $b$ are elements of $X_0$.
Let $[\uvar,1]:\Gb\to \Psh{\Gb}_{\bullet,\bullet}$ be the functor sending $\Db_n$ on $(\Db_{n+1},\{0\},\{1\})$ and $i_n^{\epsilon}$ on $i_{n+1}^{\epsilon}$. This induces a functor $[\uvar,1]:\Psh{\Gb}\to \Psh{\Gb}$ that we call the \textit{suspension}. We leave it to the reader to check that whenever $C$ has a structure of $\omega$-category, $[C,1]$ inherits one from it. This functor then induces a functor 
$$[\uvar,1]:\omegacat\to \omegacat$$
that we calls again the \snotionsym{suspension}{((d60@$[\uvar,1]$}{for $\zo$-categories}. Eventually, we denote by $i_0^-:\{0\}\to [C,1]$ (resp. $i_0^+:\{1\}\to [C,1]$) the morphism corresponding to the left point (resp. to the right point). For an integer $n$, we define by induction the functor $\Sigma^n:\Psh{\Gb}\to \Psh{\Gb}$\ssym{(sigma@$\Sigma^n$}{for $\zo$-categories} with the formula:
$$\Sigma^0:= id ~~~~~\Sigma^{n+1}:=\Sigma^n[\uvar,1].$$

\p Let $n$ be a non null integer.
A $n$-cells $f:s\to t$ is an \notion{equivalence} if there exists $n$-cells $g:t\to s$ and $g':t\to s$ such that 
$$f\circ_{n-1} g=\Ib_t~~~~~~g\circ_{n-1} f=\Ib_s$$
A \wcnotion{$\zo$-category}{category1@$\zo$-category} is an $\omega$-category whose only equivalences are the identities.
These objects are called \textit{Gaunt $\omega$-categories} in \cite{Barwick_on_the_unicity_of_the_theory_of_higher_categories} and \textit{rigid $\omega$-categories} in \cite{Rezk_a_cartesian_of_weak_n_categories}. Remark that $\zo$-categories are stable under suspensions and dualities.
We then define \wcnotation{$\zocat$}{((a20@$\zocat$} as the full subcategory of $\omegacat$ whose objects are the $\zo$-categories. 


\p
Let $n$ be an integer. An \wcnotion{$(0,n)$-category}{category2@$(0,n)$-category} is an $\zo$-category whose cell of dimension strictly higher than $n$ are units. The category of $n$-categories is denoted by \wcnotation{$\zncat{n}$}{((a10@$\zncat{n}$} and is the full subcategory of $\zocat$ whose objects are $(0,n)$-categories.

 Remark that the category $\zncat{n}$ is the localization of $\zocat$ along morphisms $\Db_{k}\to \Db_{n}$ for $k\geq n$. We then have for any $n$ an adjunction 
% q.uiver.app/#q=WzAsMixbMCwwLCJpX246XFx6bmNhdHtufSJdLFsxLDAsIlxcem9jYXQ6XFx0YXVfbiJdLFsxLDAsIiIsMCx7Im9mZnNldCI6LTJ9XSxbMCwxLCIiLDAseyJvZmZzZXQiOi0yfV0sWzMsMiwiIiwwLHsibGV2ZWwiOjEsInN0eWxlIjp7Im5hbWUiOiJhZGp1bmN0aW9uIn19XV0=
\[\begin{tikzcd}
	{i_n:\zncat{n}} & {\zocat:\tau_n}
	\arrow[""{name=0, anchor=center, inner sep=0}, shift left=2, from=1-2, to=1-1]
	\arrow[""{name=1, anchor=center, inner sep=0}, shift left=2, from=1-1, to=1-2]
	\arrow["\dashv"{anchor=center, rotate=-90}, draw=none, from=1, to=0]
\end{tikzcd}\]
The right adjoint is called the \wcsnotionsym{$n$-truncation}{(tau@$\tau_n$}{truncation@$n$-truncation}{for $\zo$-categories}.
For any $n$, we define the colimit preserving functor $\tau^i_n:\zocat\to \zncat{n}$, called the \snotionsym{intelligent $n$-truncation}{(taui@$\tau^i_n$}{for $\zo$-categories}, sending $\Db_k$ on $\Db_{\min(n,k)}$. The functor $\tau^i_n$ fits in an adjunction
% q.uiver.app/#q=WzAsMixbMSwwLCJcXHpuY2F0e259OmlfbiJdLFswLDAsIlxcdGF1XmlfbjpcXHpvY2F0Il0sWzEsMCwiIiwwLHsib2Zmc2V0IjotMn1dLFswLDEsIiIsMCx7Im9mZnNldCI6LTJ9XSxbMiwzLCIiLDAseyJsZXZlbCI6MSwic3R5bGUiOnsibmFtZSI6ImFkanVuY3Rpb24ifX1dXQ==
\[\begin{tikzcd}
	{\tau^i_n:\zocat} & {\zncat{n}:i_n}
	\arrow[""{name=0, anchor=center, inner sep=0}, shift left=2, from=1-1, to=1-2]
	\arrow[""{name=1, anchor=center, inner sep=0}, shift left=2, from=1-2, to=1-1]
	\arrow["\dashv"{anchor=center, rotate=-90}, draw=none, from=0, to=1]
\end{tikzcd}\]
We will identify objects of $\zncat{n}$ with their image in $\zocat$ and we will then also note by $\tau_n$ and $\tau^i_n$ the composites $i_n\tau^i_n$ and $i_n\tau^i_n$.

\p The family of truncation functor induces a sequence 
$$...\to \zncat{n+1}\xrightarrow{\tau_{n}} \zncat{n}\to...\to \zncat{1}\xrightarrow{\tau_{0}}\zncat{0}.$$
The canonical morphism
$$\zocat\to \lim_{n:\Nb}\zncat{n},$$
that sends an $\zo$-category $C$ to the sequence $(\tau_n C,\tau_n\tau_{n+1}C\cong \tau_n C)$, has an inverse given by the functor
$$\colim_{\Nb}:\lim_{n:\Nb}\zncat{n}\to \zocat$$
that sends a sequence $(C_n, \tau_{n}C_{n+1}\cong C_n) $ to the colimit of the induced sequence:
$$i_0C_0\to i_1C_1\to...\to i_nC_n\to...$$	
 We then have an equivalence 
$$\zocat\cong \lim_{n:\Nb}\zncat{n}.$$


\subsection{The category $\Theta$}
\label{subsection:the categoru theta}

\p Let $n$ be a non negative integer and $\textbf{a}:=\{a_0,a_1,...,a_{n-1}\}$ a sequence of $\zo$-categories. We denote \wcnotation{$[\textbf{a},n]$}{((g00@$[\textbf{a},n]$} the colimit of the following diagram:
% q.uiver.app/#q=WzAsNyxbMCwxLCJbYV8wLDFdIl0sWzIsMSwiW2FfMSwxXSJdLFs2LDEsIlthX3tuLTF9LDFdIl0sWzQsMSwiLi4uIl0sWzEsMCwiMSJdLFszLDAsIjEiXSxbNSwwLCIxIl0sWzQsMCwiaV8wXisiLDJdLFs0LDEsImlfMF4tIl0sWzUsMSwiaV8wXisiLDJdLFs1LDMsImlfMF4tIl0sWzYsMywiaV8wXisiLDJdLFs2LDIsImlfMF4tIl1d
\[\begin{tikzcd}
	& 1 && 1 && 1 \\
	{[a_0,1]} && {[a_1,1]} && {...} && {[a_{n-1},1]}
	\arrow["{i_0^+}"', from=1-2, to=2-1]
	\arrow["{i_0^-}", from=1-2, to=2-3]
	\arrow["{i_0^+}"', from=1-4, to=2-3]
	\arrow["{i_0^-}", from=1-4, to=2-5]
	\arrow["{i_0^+}"', from=1-6, to=2-5]
	\arrow["{i_0^-}", from=1-6, to=2-7]
\end{tikzcd}\]

\p \label{para:les sommes glob}
We define \wcnotation{$\Theta$}{(theta@$\Theta$} as the smallest full subcategory of $\zocat$ that includes the terminal $\zo$-category $[0]$, and such that
for any non negative integer $n$, and any finite sequence $\textbf{a}:=\{a_0,a_1,...,a_{n-1}\}$ of objects of $\Theta$, it includes the $\zo$-category $[\textbf{a},n]$.
Objects of $\Theta$ are called \notion{globular sum}.


Remark that a morphism $g:[\textbf{a},n]\to [\textbf{b},m]$ is exactly the data of a morphism $f:[n]\to [m]$, and for any integer $i$, a morphism
$$a_i\to \prod_{f(i)\leq k< f(i+1)}b_k.$$


\begin{example}
\label{exemple:of globular sum}
For any $n$, $\Db_n$ is a globular sum. The $\zo$-category induced by the $\omega$-graph 
% q.uiver.app/#q=WzAsMyxbMCwwLCJcXGJ1bGxldCJdLFsxLDAsIlxcYnVsbGV0Il0sWzIsMCwiXFxidWxsZXQiXSxbMCwxXSxbMSwyLCIiLDAseyJjdXJ2ZSI6LTR9XSxbMSwyLCIiLDAseyJjdXJ2ZSI6NH1dLFsxLDJdLFs0LDYsIiIsMCx7InNob3J0ZW4iOnsic291cmNlIjoyMCwidGFyZ2V0IjoyMH19XSxbNiw1LCIiLDAseyJzaG9ydGVuIjp7InNvdXJjZSI6MjAsInRhcmdldCI6MjB9fV1d
\[\begin{tikzcd}
	\bullet & \bullet & \bullet
	\arrow[from=1-1, to=1-2]
	\arrow[""{name=0, anchor=center, inner sep=0}, curve={height=-24pt}, from=1-2, to=1-3]
	\arrow[""{name=1, anchor=center, inner sep=0}, curve={height=24pt}, from=1-2, to=1-3]
	\arrow[""{name=2, anchor=center, inner sep=0}, from=1-2, to=1-3]
	\arrow[shorten <=3pt, shorten >=3pt, Rightarrow, from=0, to=2]
	\arrow[shorten <=3pt, shorten >=3pt, Rightarrow, from=2, to=1]
\end{tikzcd}\]
is a globular sum.
\end{example}



\p For a globular sum $a$ and an integer $n$, we define $[a,n]:=[\{a,a,...,a\},n]$.\ssym{((g10@$[a,n]$}{for $\zo$-categories}
For a sequence of integer $\{n_0,..,n_k\}$ and a sequence of globular sum $\{a_0,..,a_k\}$, we define \wcsnotation{$[a_0,n_0]\vee[a_1,n_1]\vee...\vee [a_k,n_k]$}{((g20@$[a_0,n_0]\vee[a_1,n_1]\vee...\vee [a_k,n_k]$}{for $\Theta$} as the globular sum $[\{a_0,..,a_1,...,a_k,...\},n_0+n_1+...+n_k]$.

We denote by $[0]$ the terminal $\io$-category, and $[n]$ the globular sum $[[0],n]$.
We have a fully faithful functor $\Delta\to \Theta$ sending $[n]$ onto $[n]$.. 



\p \label{para:reedy}
 A \notion{Reedy category} is a small category $A$ equipped with two subcategories $A_+$, $A_-$ and a \textit{degree} function $d:ob(A)\to \Nb$ such that: 
\begin{enumerate}
\item for every non identity morphism $f:a\to b$, if $f$ belongs to $A_-$, $d(a)>d(b)$, and if $f$ belongs to $A_+$, $d(a)<d(b)$.
\item every morphism of $A$ uniquely factors as a morphism of $A_-$ followed by a morphism of $A_+$.
\end{enumerate}

A Reedy category $A$ is \wcnotion{elegant}{elegant Reedy category} if for any presheaf $X$ on $A$, for any $a\in A$ and any $c\in X(a)$, there exists a unique morphism $f:a\to a'\in A_{-}$ and a unique non degenerate object $c'\in X(a')$ such that $c=X(f)(c')$. 

\begin{prop}
\label{prop:elelangat stable by slice}
Let $X$ be a presheaf on an elegant Reedy category $A$. The category $A_{/X}$ is an elegant Reedy category.
\end{prop}
\begin{proof}
We have a canonical projection $\pi:A_{/X}\to A$. A morphism is positive (resp. negative) if it's image by $\pi$ is. The degree of an element $c$ of $A_{/X}$ is the degree of $\pi(c)$. We leave it to the reader to check that this endows $A_{/X}$ with a structure of Reedy category. 

The fact that $A_{/X}$ is elegant is a direct consequence of the isomorphism $\Psh{A_{/X}}\cong \Psh{A}_{/X}$.
\end{proof}




\p We define by induction the \wcnotion{dimension}{dimension of a globular sum} of a globular sum $a$, denoted by $|a|$. The dimension of $[0]$ is $0$, and the dimension of $[\textbf{a},n]$ is the maximum of the set $\{|a_k|+1\}_{k< n}$. We denote by \wcnotation{$\Theta_n$}{(thetan@$\Theta_n$} the full subcategory of $\Theta$ whose objects are the globular sum of dimension inferior or equal to $n$.


\begin{prop}[Berger, Bergner-Rezk]
\label{prop:theta is elegan reedy}
The category $\Theta$ and, for any $n\in \Nb$, the category $\Theta_n$ are elegant Reedy category.

A morphism $g:[\textbf{a},n]\to [\textbf{b},m]$ is \wcnotion{degenerate}{degenerate morphism of $\Theta$} (i.e a morphism of $\Theta_{-}$) if the corresponding morphism $f:[n]\to [m]$ is a degenerate morphism of $\Delta$, and for any $i<n$ and any $f(i)\leq k<f(k+1)$, the corresponding morphism $a_i\to b_k$ is degenerate. Furthermore, a morphism is degenerate  if and only if it is a epimorphism in $\Psh{\Theta}$.

A morphism is in $\Theta^+$ if and only if it is a monomorphism in $\Psh{\Theta}$. 
\end{prop}
\begin{proof}
The Reedy structure is a consequence of lemma 2.4 of \cite{Berger_a_cellular_nerve}. The fact that for any $n<\omega$, $\Theta_n$ is elegant is 
\cite[corollary 4.5.]{Bergner_reedy_category_and_the_theta_construction}. As for any $n<\omega$, the inclusion $\Theta_n\to \Theta$ preserves strong pushout, the characterization of elegant Reedy category given by \cite[proposition 3.8.]{Bergner_reedy_category_and_the_theta_construction} implies that $\Theta$ is also elegant.
\end{proof}

\p 
\label{para:algebraic and globular}
We recall that a morphism $g:[\textbf{a},n]\to [\textbf{b},m]$ is exactly the data of a morphism $f:[n]\to [m]$, and for any integer $i$, a morphism
$$a_i\to \prod_{f(i)\leq k< f(i+1)}b_k.$$
The morphism $g$ is \wcsnotion{globular}{globular morphism}{for $\zo$-categories} if for any $k<n$, $f(k+1)=f(k)+1$ and the morphism $a_k\to b_k$ is globular. The morphism $g$ is \wcnotion{algebraic}{algebraic morphism of $\Theta$} if it cannot be written as a composite $ig'$ where $i$ is a globular morphism.


\begin{example}
The morphism % q.uiver.app/#q=WzAsNixbNCwwLCJcXGJ1bGxldCJdLFs1LDAsIlxcYnVsbGV0Il0sWzYsMCwiXFxidWxsZXQiXSxbMSwwLCJcXGJ1bGxldCJdLFsyLDAsIlxcYnVsbGV0Il0sWzAsMCwiXFxidWxsZXQiXSxbMCwxXSxbMSwyLCIiLDAseyJjdXJ2ZSI6LTR9XSxbMSwyLCIiLDAseyJjdXJ2ZSI6NH1dLFsxLDJdLFszLDRdLFszLDQsIiIsMix7ImN1cnZlIjotNH1dLFs1LDNdLFs0LDAsIiIsMix7InNob3J0ZW4iOnsic291cmNlIjozMCwidGFyZ2V0IjozMH0sInN0eWxlIjp7InRhaWwiOnsibmFtZSI6Im1hcHMgdG8ifX19XSxbNyw5LCIiLDAseyJzaG9ydGVuIjp7InNvdXJjZSI6MjAsInRhcmdldCI6MjB9fV0sWzksOCwiIiwwLHsic2hvcnRlbiI6eyJzb3VyY2UiOjIwLCJ0YXJnZXQiOjIwfX1dLFsxMSwxMCwiIiwyLHsic2hvcnRlbiI6eyJzb3VyY2UiOjIwLCJ0YXJnZXQiOjIwfX1dXQ==
\[\begin{tikzcd}
	\bullet & \bullet & \bullet && \bullet & \bullet & \bullet
	\arrow[from=1-5, to=1-6]
	\arrow[""{name=0, anchor=center, inner sep=0}, curve={height=-24pt}, from=1-6, to=1-7]
	\arrow[""{name=1, anchor=center, inner sep=0}, curve={height=24pt}, from=1-6, to=1-7]
	\arrow[""{name=2, anchor=center, inner sep=0}, from=1-6, to=1-7]
	\arrow[""{name=3, anchor=center, inner sep=0}, from=1-2, to=1-3]
	\arrow[""{name=4, anchor=center, inner sep=0}, curve={height=-24pt}, from=1-2, to=1-3]
	\arrow[from=1-1, to=1-2]
	\arrow[shorten <=14pt, shorten >=14pt, maps to, from=1-3, to=1-5]
	\arrow[shorten <=3pt, shorten >=3pt, Rightarrow, from=0, to=2]
	\arrow[shorten <=3pt, shorten >=3pt, Rightarrow, from=2, to=1]
	\arrow[shorten <=3pt, shorten >=3pt, Rightarrow, from=4, to=3]
\end{tikzcd}\]
is globular. This is not the case for the morphism
% q.uiver.app/#q=WzAsNixbNCwwLCJcXGJ1bGxldCJdLFs1LDAsIlxcYnVsbGV0Il0sWzYsMCwiXFxidWxsZXQiXSxbMSwwLCJcXGJ1bGxldCJdLFsyLDAsIlxcYnVsbGV0Il0sWzAsMCwiXFxidWxsZXQiXSxbMCwxXSxbMSwyLCIiLDAseyJjdXJ2ZSI6LTR9XSxbMSwyLCIiLDAseyJjdXJ2ZSI6NH1dLFsxLDJdLFszLDQsIiIsMCx7ImN1cnZlIjo0fV0sWzMsNCwiIiwyLHsiY3VydmUiOi00fV0sWzUsM10sWzQsMCwiIiwyLHsic2hvcnRlbiI6eyJzb3VyY2UiOjMwLCJ0YXJnZXQiOjMwfSwic3R5bGUiOnsidGFpbCI6eyJuYW1lIjoibWFwcyB0byJ9fX1dLFs3LDksIiIsMCx7InNob3J0ZW4iOnsic291cmNlIjoyMCwidGFyZ2V0IjoyMH19XSxbOSw4LCIiLDAseyJzaG9ydGVuIjp7InNvdXJjZSI6MjAsInRhcmdldCI6MjB9fV0sWzExLDEwLCIiLDIseyJzaG9ydGVuIjp7InNvdXJjZSI6MjAsInRhcmdldCI6MjB9fV1d
\[\begin{tikzcd}
	\bullet & \bullet & \bullet && \bullet & \bullet & \bullet
	\arrow[from=1-5, to=1-6]
	\arrow[""{name=0, anchor=center, inner sep=0}, curve={height=-24pt}, from=1-6, to=1-7]
	\arrow[""{name=1, anchor=center, inner sep=0}, curve={height=24pt}, from=1-6, to=1-7]
	\arrow[""{name=2, anchor=center, inner sep=0}, from=1-6, to=1-7]
	\arrow[""{name=3, anchor=center, inner sep=0}, curve={height=24pt}, from=1-2, to=1-3]
	\arrow[""{name=4, anchor=center, inner sep=0}, curve={height=-24pt}, from=1-2, to=1-3]
	\arrow[from=1-1, to=1-2]
	\arrow[shorten <=14pt, shorten >=14pt, maps to, from=1-3, to=1-5]
	\arrow[shorten <=3pt, shorten >=3pt, Rightarrow, from=0, to=2]
	\arrow[shorten <=3pt, shorten >=3pt, Rightarrow, from=2, to=1]
	\arrow[shorten <=6pt, shorten >=6pt, Rightarrow, from=4, to=3]
\end{tikzcd}\]
that sends the $2$-cell of the left globular sum on the $1$-composite of the two $2$-cells of the right globular sum.
\end{example}

\begin{prop}[{\cite[Proposition 3.3.10]{Ara_thesis}}]
\label{prop:algebraic ortho to globular}
Every morphism in $\Theta$ can be factored uniquely in an algebraic morphism followed by a globular morphism.
\end{prop}


\p 
\label{para:definition of source et but}
We define for any globular sum $a$ and any integer $n$ a globular sum $s_n(a):=:t_n(a)$ and two morphisms
$$s_n(a)\to a\leftarrow t_n(a).$$
We first set $s_0(a):=:t_0(a) :=[0]$. The inclusion $s_0(a)\to a$ corresponds to the initial point and $t_0(a)\to a$ to the terminal point.
 For $n>0$, we define $s_n([\textbf{a},n]):=:t_n([\textbf{a},n]) :=[s_{n-1}(\textbf{a}),n]$ where $s_{n-1}(\textbf{a})$ is the sequence 
 $\{s_{n-1}(a_i)\}_{i<n}$.

\begin{example}
If $a$ is the globular sum of example \ref{exemple:of globular sum}, we have:
% q.uiver.app/#q=WzAsMTIsWzEsMiwiXFxidWxsZXQiXSxbMiwyLCJcXGJ1bGxldCJdLFszLDIsIlxcYnVsbGV0Il0sWzIsMCwiXFxidWxsZXQiXSxbMywwLCJcXGJ1bGxldCJdLFsxLDAsIlxcYnVsbGV0Il0sWzEsNCwiXFxidWxsZXQiXSxbMiw0LCJcXGJ1bGxldCJdLFszLDQsIlxcYnVsbGV0Il0sWzAsMiwiYTo9Il0sWzAsMCwic18xKGEpOj0iXSxbMCw0LCJ0XzEoYSk6PSJdLFswLDFdLFsxLDIsIiIsMCx7ImN1cnZlIjotNH1dLFsxLDIsIiIsMCx7ImN1cnZlIjo0fV0sWzEsMl0sWzMsNCwiIiwwLHsiY3VydmUiOi00fV0sWzUsM10sWzYsN10sWzcsOCwiIiwwLHsiY3VydmUiOjR9XSxbMywxLCIiLDAseyJzaG9ydGVuIjp7InNvdXJjZSI6NDAsInRhcmdldCI6NDB9LCJzdHlsZSI6eyJ0YWlsIjp7Im5hbWUiOiJtYXBzIHRvIn19fV0sWzcsMSwiIiwyLHsic2hvcnRlbiI6eyJzb3VyY2UiOjQwLCJ0YXJnZXQiOjQwfSwic3R5bGUiOnsidGFpbCI6eyJuYW1lIjoibWFwcyB0byJ9fX1dLFsxMywxNSwiIiwwLHsic2hvcnRlbiI6eyJzb3VyY2UiOjIwLCJ0YXJnZXQiOjIwfX1dLFsxNSwxNCwiIiwwLHsic2hvcnRlbiI6eyJzb3VyY2UiOjIwLCJ0YXJnZXQiOjIwfX1dXQ==
\[\begin{tikzcd}
	{s_1(a):=} & \bullet & \bullet & \bullet \\
	\\
	{a:=} & \bullet & \bullet & \bullet \\
	\\
	{t_1(a):=} & \bullet & \bullet & \bullet
	\arrow[from=3-2, to=3-3]
	\arrow[""{name=0, anchor=center, inner sep=0}, curve={height=-24pt}, from=3-3, to=3-4]
	\arrow[""{name=1, anchor=center, inner sep=0}, curve={height=24pt}, from=3-3, to=3-4]
	\arrow[""{name=2, anchor=center, inner sep=0}, from=3-3, to=3-4]
	\arrow[curve={height=-24pt}, from=1-3, to=1-4]
	\arrow[from=1-2, to=1-3]
	\arrow[from=5-2, to=5-3]
	\arrow[curve={height=24pt}, from=5-3, to=5-4]
	\arrow[shorten <=13pt, shorten >=13pt, maps to, from=1-3, to=3-3]
	\arrow[shorten <=13pt, shorten >=13pt, maps to, from=5-3, to=3-3]
	\arrow[shorten <=3pt, shorten >=3pt, Rightarrow, from=0, to=2]
	\arrow[shorten <=3pt, shorten >=3pt, Rightarrow, from=2, to=1]
\end{tikzcd}\]
\end{example}




\p
\label{para:definition of W}
The morphism $[\uvar,1]:\Theta\to \Theta$ induces by extension by colimit a functor 
$$[\uvar,1]:\Psh{\Theta}\to \Psh{\Theta}.$$
We define by induction on $a$ a $\Theta$-presheaf \wcnotation{$\Sp_a$}{(sp@$\Sp_{a}$} and a morphism $\Sp_a\to a$. 
If $a$ is $[0]$, we set $\Sp_{[0]}:=[0]$. For $n>0$, we define 
$\Sp_{[\textbf{a},n]}$ as the set valued presheaf on $\Theta$ obtained as the colimit of the diagram
% q.uiver.app/#q=WzAsNyxbMCwxLCJbIFxcU3Bfe2FfMH0sMV0iXSxbNiwxLCJbXFxTcF97YV97bi0xfX0sMV0iXSxbMSwwLCIxIl0sWzIsMSwiW1xcU3Bfe2FfMX0sMV0iXSxbMywwLCIxIl0sWzUsMCwiMSJdLFs0LDEsIlxcY2RvdHMiXSxbMiwzLCJpXzBeLSJdLFs0LDMsImlfMF4rIiwyXSxbMiwwLCJpXzBeKyIsMl0sWzUsMSwiaV8wXi0iXSxbNCw2LCJpXzBeLSJdLFs1LDYsImlfMF4rIiwyXV0=
\[\begin{tikzcd}
	& 1 && 1 && 1 \\
	{[ \Sp_{a_0},1]} && {[\Sp_{a_1},1]} && \cdots && {[\Sp_{a_{n-1}},1]}
	\arrow["{i_0^-}", from=1-2, to=2-3]
	\arrow["{i_0^+}"', from=1-4, to=2-3]
	\arrow["{i_0^+}"', from=1-2, to=2-1]
	\arrow["{i_0^-}", from=1-6, to=2-7]
	\arrow["{i_0^-}", from=1-4, to=2-5]
	\arrow["{i_0^+}"', from=1-6, to=2-5]
\end{tikzcd}\]
We define \wcnotation{$E^{eq}$}{(eeq@$E^{eq}$} as the set valued preheaves on $\Delta$ obtained as the colimit of the diagram
% q.uiver.app/#q=WzAsNyxbMiwxLCJbMl0iXSxbMywwLCJbMV0iXSxbMSwwLCJbMV0iXSxbNCwxLCJbMl0iXSxbNSwwLCJbMV0iXSxbMCwxLCJbMF0iXSxbNiwxLCJbMF0iXSxbMiw1XSxbMiwwLCJkXjEiXSxbNCwzLCJkXjEiLDJdLFsxLDAsImReMCIsMl0sWzEsMywiZF4yIl0sWzQsNl1d
\[\begin{tikzcd}
	& {[1]} && {[1]} && {[1]} \\
	{[0]} && {[2]} && {[2]} && {[0]}
	\arrow[from=1-2, to=2-1]
	\arrow["{d^1}", from=1-2, to=2-3]
	\arrow["{d^1}"', from=1-6, to=2-5]
	\arrow["{d^0}"', from=1-4, to=2-3]
	\arrow["{d^2}", from=1-4, to=2-5]
	\arrow[from=1-6, to=2-7]
\end{tikzcd}\]
For any integer $n$, the morphism $\Sigma^n:\Theta\to \Theta$, which is the $n$-iteration of $[\uvar,1]$, induces by colimit a functor \ssym{(sigma@$\Sigma^n$}{for $\io$-categories}
$$\Sigma^n:\Psh{\Theta}\to \Psh{\Theta}.$$
 We define two sets of morphisms of $\Psh{\Theta}$:\sym{(w@$\W$}\sym{(wseg@$\Wseg$}\sym{(wsat@$\Wsat$}
$$\Wseg := \{\Sp_a\to a,~a\in\Theta\}~~~~\Wsat:= \{\Sigma^n E^{eq}\to \Db_n\}$$
and we set $$\W:=\Wseg\cup \Wsat.$$
For any $n$, we also define $$\mbox{$\W_n$}:= \W\cap \Theta_n.$$


\p
\label{para:defi of delta theta}
We recall that for an integer $n$ and a globular sum $a$, we defined $[a,n]:=[\{a,a,...,a\},n]$.
 This defines a functor $i:\Delta[\Theta] \to \Theta$
sending $(n,a)$ on $[a,n]$ where 
 \wcnotation{$\Delta[\Theta]$}{(deltaTheta@$\Delta[\Theta]$} is the following pushout of category: 
% q.uiver.app/#q=WzAsNCxbMSwxLCJcXERlbHRhW1xcVGhldGFdIl0sWzEsMCwiXFxEZWx0YVxcdGltZXNcXFRoZXRhIl0sWzAsMCwiXFx7WzBdXFx9XFx0aW1lcyBcXFRoZXRhIl0sWzAsMSwiMSJdLFsyLDFdLFsyLDNdLFsxLDBdLFszLDBdLFswLDIsIiIsMSx7InN0eWxlIjp7Im5hbWUiOiJjb3JuZXIifX1dXQ==
\[\begin{tikzcd}
	{\{[0]\}\times \Theta} & \Delta\times\Theta \\
	1 & {\Delta[\Theta]}
	\arrow[from=1-1, to=1-2]
	\arrow[from=1-1, to=2-1]
	\arrow[from=1-2, to=2-2]
	\arrow[from=2-1, to=2-2]
	\arrow["\lrcorner"{anchor=center, pos=0.125, rotate=180}, draw=none, from=2-2, to=1-1]
\end{tikzcd}\]
For the sake of simplicity, we will also denote by $[a,n]$ (resp. $[n]$) the object of $\Delta[\Theta]$ corresponding to $(n,a)$ (resp. to $(n,[0])$).
 We define two sets of morphisms:\sym{(m@$\M$} \sym{(mseg@$\Mseg$}\sym{(msat@$\Msat$}
$$
\begin{array}{c}
\Mseg := \{[a,\Sp_n]\to [a,n],~a:\Theta\}\cup\{[f,1],~f\in \Wseg\}\\
\Msat:= \{E^{eq}\to [0]\} \cup\{[f,1],~f\in \Wsat\}
\end{array}$$
and we set $$\M := \Mseg \cup \Msat.$$

For an integer $n$, we define \wcnotation{$\Delta[\Theta_n]$}{(deltaThetan@$\Delta[\Theta_n]$} as the following pushout of category: 
% q.uiver.app/#q=WzAsNCxbMSwxLCJcXERlbHRhW1xcVGhldGFfbl0iXSxbMSwwLCJcXERlbHRhXFx0aW1lc1xcVGhldGFfbiJdLFswLDAsIlxce1swXVxcfVxcdGltZXNcXFRoZXRhX24iXSxbMCwxLCIxIl0sWzIsMV0sWzIsM10sWzEsMF0sWzMsMF0sWzAsMiwiIiwxLHsic3R5bGUiOnsibmFtZSI6ImNvcm5lciJ9fV1d
\[\begin{tikzcd}
	{\{[0]\}\times\Theta_n} & {\Delta\times\Theta_n} \\
	1 & {\Delta[\Theta_n]}
	\arrow[from=1-1, to=1-2]
	\arrow[from=1-1, to=2-1]
	\arrow[from=1-2, to=2-2]
	\arrow[from=2-1, to=2-2]
	\arrow["\lrcorner"{anchor=center, pos=0.125, rotate=180}, draw=none, from=2-2, to=1-1]
\end{tikzcd}\]
and the functor $i$ induces a functor $\Delta[\Theta_n]\to \Theta_{n+1}$.
For any $n$, we define $$\mbox{$\M_n$}:= \M\cap \Delta[\Theta_n].$$ 


\p Let $C$ be a presentable category and $S$ a set of monomorphisms with small codomains. An object $x$ is \textit{$S$-local} if for any $i:a\to b\in S$, the induced functor $\Hom(i,x):\Hom(b,x)\to \Hom(a,x)$ is an isomorphism. 
We define \textit{$C_{S}$} as the full subcategory of $C$ composed of $S$-local objects.
According to theorem \ref{theo:adjunction between presheaves and local presheaves}, the inclusion $\iota:C_S\to C$ is part of an adjunction
% q.uiver.app/#q=WzAsMixbMSwwLCJDX1M6XFxpb3RhIl0sWzAsMCwiXFxMYl9TOkMiXSxbMSwwLCIiLDAseyJvZmZzZXQiOi0yfV0sWzAsMSwiIiwwLHsib2Zmc2V0IjotMn1dLFsyLDMsIiIsMCx7ImxldmVsIjoxLCJzdHlsZSI6eyJuYW1lIjoiYWRqdW5jdGlvbiJ9fV1d
\[\begin{tikzcd}
	{\Fb_S:C} & {C_S:\iota}
	\arrow[""{name=0, anchor=center, inner sep=0}, shift left=2, from=1-1, to=1-2]
	\arrow[""{name=1, anchor=center, inner sep=0}, shift left=2, from=1-2, to=1-1]
	\arrow["\dashv"{anchor=center, rotate=-90}, draw=none, from=0, to=1]
\end{tikzcd}\]
Moreover, the theorem \textit{op cit} also states that $\Fb_S:C\to C_S$ is the localization of $C$ by the smallest class of morphisms containing $S$ and stable under composition and colimit.

Suppose given an other category $D$ fitting in an adjunction
% q.uiver.app/#q=WzAsMixbMCwwLCJmOkMiXSxbMSwwLCJEOmciXSxbMCwxLCIiLDAseyJvZmZzZXQiOi0yfV0sWzEsMCwiIiwwLHsib2Zmc2V0IjotMn1dLFsyLDMsIiIsMCx7ImxldmVsIjoxLCJzdHlsZSI6eyJuYW1lIjoiYWRqdW5jdGlvbiJ9fV1d
\[\begin{tikzcd}
	{F:C} & {D:G}
	\arrow[""{name=0, anchor=center, inner sep=0}, shift left=2, from=1-1, to=1-2]
	\arrow[""{name=1, anchor=center, inner sep=0}, shift left=2, from=1-2, to=1-1]
	\arrow["\dashv"{anchor=center, rotate=-90}, draw=none, from=0, to=1]
\end{tikzcd}\]
with unit $\nu$ and counit $\epsilon$,
as well as a set of morphisms $T$ of $D$ such that $F(S)\subset T$. 
By adjunction property, it implies that for any $T$-local object $d\in D$, $G(d)$ is $S$-local.
The previous adjunction induces a derived adjunction
\[\begin{tikzcd}
	{\Lb F:C_S} & {D_T:\Rb G}
	\arrow[""{name=0, anchor=center, inner sep=0}, shift left=2, from=1-1, to=1-2]
	\arrow[""{name=1, anchor=center, inner sep=0}, shift left=2, from=1-2, to=1-1]
	\arrow["\dashv"{anchor=center, rotate=-90}, draw=none, from=0, to=1]
\end{tikzcd}\]
where $\Lb F$ is defined by the formula $c\mapsto \Fb_T F(c)$ and $\Rb G$ is the restriction of $G$ to $D_T$. The unit is given by $\nu\circ \Fb_S$ and the counit by the restriction of $\epsilon$ to $D_T$.



\p The functor $i:\Delta[\Theta]\to \Theta$ defined in paragraph \ref{para:defi of delta theta} induces an adjunction:
$$
\begin{tikzcd}
	{ i_!:\Psh{\Delta[\Theta]}} & {\Psh{\Theta}:i^*}
	\arrow[shift left=2, from=1-1, to=1-2]
	\arrow[shift left=2, from=1-2, to=1-1]
\end{tikzcd}
$$
where the left adjoint is the left Kan extension of the functor $\Delta[\Theta]\to \Theta\to \Psh{\Theta}$.
Remark that there is an obvious inclusion $i_!(\M)\subset \W$. In virtue of the last paragraph, this induces an adjunction between derived categories:
\begin{equation}
\label{eq:derived adjunction strict}
\begin{tikzcd}
	{\Lb i_!:\Psh{\Delta[\Theta]}_{\M}} & {\Psh{\Theta}_{\W}:\Rb i^*}
	\arrow[shift left=2, from=1-1, to=1-2]
	\arrow[shift left=2, from=1-2, to=1-1]
\end{tikzcd}
\end{equation}
The corollary 12.3 of \cite{Barwick_on_the_unicity_of_the_theory_of_higher_categories} and  the corollary \ref{cor:changing theta} (which is proved in the next section) induce equivalences
$$\zocat\cong \Psh{\Theta}_{\W}\cong \Psh{\Delta[\Theta]}_{\M}.$$



Similarly, for any integer $n$, the inclusion $i:\Delta[\Theta_n]\to \Theta_{n+1}$ induces an adjunction between derived categories:
\begin{equation}
\label{eq:derived adjunction case n strict}
\begin{tikzcd}
	{\Lb i_!:\Psh{\Delta[\Theta]_n}_{\M_n}} & {\Psh{\Theta_{n+1}}_{\W_n}:\Rb i^*}
	\arrow[shift left=2, from=1-1, to=1-2]
	\arrow[shift left=2, from=1-2, to=1-1]
\end{tikzcd}
\end{equation}
and corollary 12.3 of \cite{Barwick_on_the_unicity_of_the_theory_of_higher_categories} and corollary \ref{cor:changing theta} induce equivalences
$$\zncat{n+1}\cong \Psh{\Theta_{n+1}}_{\W_{n+1}}\cong \Psh{\Delta[\Theta_n]}_{\M_{n+1}}.$$




\subsection{The link between presheaves on $\Theta$ and on $\Delta[\Theta]$}
\p 
\label{para:precomplet}
A class of monomorphism $T$ is \wcnotionsym{precocomplete}{(ss@$\overline{S}$}{precocomplete set of arrows} if
\begin{enumerate}
\item[$-$] It is closed by transfinite compositions and pushouts.
\item[$-$] It is closed by left \textit{cancellation}, i.e for any pair of composable morphisms $f$ and $g$, if $gf$ and $f$ are in $S$ , so is $g$.
\item[$-$] For any elegant Reedy category $A$, and any functor $F:A\to \Arr(C)$ such that the induced morphism $\colim_{\partial a}F\to F(a)$ is a monomorphism for any object $a$, and such that  $F$ is pointwise in $S$, then $\colim_AF$ is in $S$.
\end{enumerate} 
For a set of morphisms $S$, we denote $\overline{S}$ the smallest precocomplete class of morphisms containing $S$.

\p
The aim of this subsection is to demonstrate the following proposition:

\begin{theorem}
\label{theo:unit and counit are in W}
For any $a\in \Theta$ and $b\in \Delta[\Theta]$, morphisms $i_!i^*a\to a$ and $b\to i^*i_! b$ are respectively in $\overline{\W}$ and $\overline{\M}$.
\end{theorem}

As a corollary, we directly have:
\begin{cor}
\label{cor:changing theta}
The adjunction 
$$\begin{tikzcd}
	{\Lb i_!:\Psh{\Delta[\Theta]}_{\M}} & {\Psh{\Theta}_{\W}:\Rb i^*}
	\arrow[shift left=2, from=1-1, to=1-2]
	\arrow[shift left=2, from=1-2, to=1-1]
\end{tikzcd}$$
given in \eqref{eq:derived adjunction strict} is an adjoint equivalence. For any integer $n$, the adjunction 
$$\begin{tikzcd}
	{\Lb i_!:\Psh{\Delta[\Theta]_n}_{\M_n}} & {\Psh{\Theta_{n+1}}_{\W_n}:\Rb i^*}
	\arrow[shift left=2, from=1-1, to=1-2]
	\arrow[shift left=2, from=1-2, to=1-1]
\end{tikzcd}$$
given in \eqref{eq:derived adjunction case n strict} is an adjoint equivalence. 
\end{cor}
\begin{proof}
The first assertion is a consequence of theorem \ref{theo:unit and counit are in W} and of the fact that $\overline{\W}$ (resp. $\overline{\M}$) is a included in the smallest class containing $\W$ (resp. $\M$) and stable by two out of three and colimits.
We prove the second assertion similarly.
\end{proof}

\p 
We denote by 
$$[\uvar,\uvar]: \Psh{\Theta}\times \Psh{\Delta}\to \Psh{\Delta[\Theta]}$$
the extension by colimit of the functor $\Theta\times \Delta\to \Psh{\Delta[\Theta]}$ sending $(a,n)$ onto $[a,n]$.
For an integer $n$, we denote
$$[\uvar,n]:\Psh{\Theta}^n\to \Psh{\Theta}$$ 
the extension by colimit of the functor 
$\Theta^n\to\Psh{\Theta}$ sending $\textbf{a}:=\{a_1,...,a_n\}$ onto $[\textbf{a},n]$. Eventually, we define 
$$[\uvar,d^0\cup d^n]:\Psh{\Theta}^n\to \Psh{\Theta}$$ 
the extension by colimit of the functor 
$\Theta^n\to\Psh{\Theta}$ sending $\textbf{a}:=\{a_1,...,a_n\}$ onto the colimit of the span.
$$[\{a_0,...,a_{n-2}\},{n-1}]\leftarrow [\{a_1,...,a_{n-2}\},{n-2}]\to [\{a_1,...,a_{n-1}\},{n-1}]$$


\begin{lemma}
\label{lemma:the functor [] preserves classes}
The image of $\overline{\W}\times \overline{\W_1}$ by the functor $[\uvar,\uvar]:\Psh{\Theta}\times \Psh{\Delta}\to \Psh{\Delta[\Theta]}$ is included in $\overline{\W}$.
\end{lemma}
\begin{proof}
As $[\uvar,\uvar]$ preserves colimits and monomorphisms, it is enough to show that for any pair $f,g\in \W\times \W_1$, $[f,g]$ is in $\W$ which is obvious.
\end{proof}



\begin{lemma}
\label{lemma:i etoile of W is in M 0}
For any globular sum $v$, and any integer $n$,
the morphism $[v,d^0\cup d^n]\cup[\partial v,n]\to [v,n]$ appearing in the diagram
% q.uiver.app/#q=WzAsNSxbMCwwLCJbXFxwYXJ0aWFsIHYsZF4wXFxjdXAgZF5uXSJdLFswLDEsIltcXHBhcnRpYWwgdixuXSJdLFsxLDAsIlt2LGReMFxcY3VwIGRebl0iXSxbMiwyLCJbIHYsbl0iXSxbMSwxLCJbXFxwYXJ0aWFsIHYsbl1cXGN1cFt2LGReMFxcY3VwIGRebl0iXSxbMCwxXSxbMiw0XSxbNCwzXSxbMSwzLCIiLDEseyJjdXJ2ZSI6M31dLFsyLDMsIiIsMyx7ImN1cnZlIjotM31dLFswLDJdLFsxLDRdXQ==
\[\begin{tikzcd}
	{[\partial v,d^0\cup d^n]} & {[v,d^0\cup d^n]} \\
	{[\partial v,n]} & {[\partial v,n]\cup[v,d^0\cup d^n]} \\
	&& {[ v,n]}
	\arrow[from=1-1, to=2-1]
	\arrow[from=1-2, to=2-2]
	\arrow[from=2-2, to=3-3]
	\arrow[curve={height=18pt}, from=2-1, to=3-3]
	\arrow[curve={height=-18pt}, from=1-2, to=3-3]
	\arrow[from=1-1, to=1-2]
	\arrow[from=2-1, to=2-2]
\end{tikzcd}\]
is in $\overline{\M}$.
\end{lemma}
\begin{proof}
Let $a$ be a globular sum.
Remark that the morphism $[a,\Sp_n]\to [a,d^0\cup d^n]$ is in $\overline{\M}$. By left cancellation, this implies that $[a,d^0\cup d^n]\to [a,n]$ is in $\overline{\M}$. For any presheaf $X$ on $\Theta$, $\Theta_{/X}$ is an elegant Reedy category, and $[X,d^0\cup d^n]\to [X,n]$ is then in $\overline{\M}$. In particular, $[\partial v,d^0\cup d^n]\to [\partial v,n]$ is in $\overline{\M}$, and so is $[v,d^0\cup d^n]\to [\partial v,n]\cup[v,d^0\cup d^n]$ by stability by coproduct. A last use of the stability by left cancellation then concludes the proof.
\end{proof}


\p 
Let  $[b,m]$ be an element of $\Delta[\Theta]$. We denote $\Hom^*(i([b,m]),[\textbf{a},n])$ the subset of $\Hom(i([b,m]),[\textbf{a},n])$ that consists of morphisms that preserve extremal objects. The explicit expression of morphism in $\Theta$ implies the bijection:
\begin{equation}
\label{eq:hom in theta}
\Hom_{\Theta}^*(i([b,m]),[\textbf{a},n])\cong\Hom_{\Delta}([n],[m])^*\times \prod_{i<n}\Hom_{\Theta}(b,a_i)
\end{equation}
where $\Hom_{\Delta}^*([n],[m])$ is the subset of $\Hom_{\Delta}([n],[m])$ consisting of morphisms that preserve extremal objects.



Let $\textbf{a}:=\{a_0,a_1,...,a_{n-1}\}$ be a finite sequence of globular sums. We define $\Theta^{\hookrightarrow}_{/\textbf{a}}$ as the category whose objects are collections of maps $\{b\to a_i\}_{ i< n}$ such that there exists no degenerate morphism $b\to b'$ factorizing all $b\to a_i$. Morphisms are monomorphisms $b\to b'$ making all induced triangles commute. 


The bijection \eqref{eq:hom in theta}  induces a bijection between the objects of $\Theta^{\hookrightarrow}_{/\textbf{a}}$ and the morphisms $[b,n]\to i^*[\textbf{a},n]$ that are the identity on objects and that can not be factored through any 	degenerate morphism $[b,n]\to [\tilde{b},n]$.  	


\begin{lemma}
\label{lemma:i etoile of W is in M 1}
For any morphism $p:[b,m]\to i^*[\textbf{a},n]$ in $\Psh{\Delta[\Theta]}$ that preserves extremal objects, there exists a unique pair $(\{b'\to a_i\}_{i<n},[f,i]:[b,m]\to [b',n])$ where $\{b'\to a_i\}_{i<n}$ is an element of $\Theta^{\hookrightarrow}_{/\textbf{a}}$, $f$ is a degenerate morphism, and such that the induced triangle
% https://q.uiver.app/#q=WzAsMyxbMCwwLCJbYixtXSJdLFsxLDAsIltiJyxuXSJdLFsxLDEsImleKltcXHRleHRiZnthfSxuXSJdLFswLDEsIltmLGldIl0sWzEsMiwicCciXSxbMCwyLCJwIiwyXV0=
\[\begin{tikzcd}
	{[b,m]} & {[b',n]} \\
	& {i^*[\textbf{a},n]}
	\arrow["{[f,i]}", from=1-1, to=1-2]
	\arrow["{p'}", from=1-2, to=2-2]
	\arrow["p"', from=1-1, to=2-2]
\end{tikzcd}\]
commutes.
\end{lemma}
\begin{proof}
By adjunction and thanks to the bijection \eqref{eq:hom in theta}, $p$ corresponds to a pair $(j:[m]\to [n], \{b\to a_i\}_{i<n})$, and $i$ has to be equal to $j$.

Using once again this bijection, and the fact that degeneracies are epimorphisms, we have to show that there exists a unique degenerate morphism $g:b\to b'$ that factors the morphisms $b\to a_i$ for all $i<n$, and such that the induced family of morphisms $\{b'\to a_i\}_{i<n}$ is an element of $\Theta^{\hookrightarrow}_{/\textbf{a}}$.


As any infinite sequence of degenerate morphisms is constant at some point, the existence is immediate.

Suppose given two morphisms $b\to b'$, $b\to b''$ fulfilling the previous condition.
The proposition 3.8 of \cite{Bergner_reedy_category_and_the_theta_construction} implies that there exists a globular sum $\tilde{b}$ and two degenerate morphisms $b'\to \tilde{b}$ and $b''\to \tilde{b}$  such that the induced square
% q.uiver.app/#q=WzAsNCxbMCwwLCJiIl0sWzEsMCwiYiciXSxbMSwxLCJcXHRpbGRle2J9Il0sWzAsMSwiYicnIl0sWzAsM10sWzMsMl0sWzAsMV0sWzEsMl1d
\[\begin{tikzcd}
	b & {b'} \\
	{b''} & {\tilde{b}}
	\arrow[from=1-1, to=2-1]
	\arrow[from=2-1, to=2-2]
	\arrow[from=1-1, to=1-2]
	\arrow[from=1-2, to=2-2]
\end{tikzcd}\]
is cartesian. The universal property of pushout implies that $b\to \tilde{b}$ also fulfills the previous condition. By definition of $b'$ and $b''$, this implies that they are equal to $\tilde{b}$, and this shows the uniqueness.
\end{proof}



\begin{lemma}
\label{lemma:i etoile of W is in M 0.5}
Let $\{b\to a_i\}_{ i< n}$ be an element of  $\Theta^{\hookrightarrow}_{/\textbf{a}}$  and $i:b'\to b$ a monomorphism of $\Theta$. The induced family $\{b'\to b\to a_i\}_{ i< n}$ is an object of $\Theta^{\hookrightarrow}_{/\textbf{a}}$. 
\end{lemma}
\begin{proof}
The lemma \ref{lemma:i etoile of W is in M 1} implies that there exists  a  unique degenerate morphism $j:b'\to \tilde{b}$ that factors all the morphism $b'\to b\to  a_i$ for $i<n$, and such the induced family of morphisms $\{\tilde{b}\to a_i\}_{i<n}$ is an element of $\Theta^{\hookrightarrow}_{/\textbf{a}}$.  We proceed by contradiction, and we then suppose that $j$ is different from the identity.

We then have, for any $i<n$, a commutative square
% https://q.uiver.app/#q=WzAsNCxbMSwwLCJiIl0sWzEsMSwiYV9pIl0sWzAsMSwiXFx0aWxkZXtifSJdLFswLDAsImInIl0sWzMsMCwiaSJdLFswLDFdLFszLDIsImoiLDJdLFsyLDFdXQ==
\[\begin{tikzcd}
	{b'} & b \\
	{\tilde{b}} & {a_i}
	\arrow["i", from=1-1, to=1-2]
	\arrow[from=1-2, to=2-2]
	\arrow["j"', from=1-1, to=2-1]
	\arrow[from=2-1, to=2-2]
\end{tikzcd}\]
As the morphism $j$ is degenerate and different of the identity, there exists an integer $k$ and a non trivial $k$-cell $d$ of $b'$ that is sent to an identity by $j$. Now, let $d'$ be a $k$-generator  of the polygraph $b$ that appears in the decomposition of $i(d)$. The commutativity of the previous square and the fact that the $\zo$-categories $a_i$ are polygraphs implies that for any $i$, the $k$-cell $a'$ is sent to an identity by the morphism $b\to a_i$.  As for any $i< n$ and any $l\geq k$,  there is no non trivial $l$-cell in $a_i$ whose $(k-1)$-source and $(k-1)$-target are the same, this implies that every $l$-cell of $b$ that is $(k-1)$-parallel with $d'$ is send to the identity by the morphism $b\to a_i$.

We denote $\bar{b}$ the globular sum obtained by crushing all $l$-cells of $b$ that are $(k-1)$-parallel with $d'$. The induced degenerate morphism $b\to \bar{b}$ factors all the morphisms $b\to a_i$ which is in contradiction with the fact that  $\{{b}\to a_i\}_{i<n}$ is an element of $\Theta^{\hookrightarrow}_{/\textbf{a}}$.
\end{proof}


\p 
We say that an element $\{v\to a_i\}_{i<n}$ in the category $\Theta^{\hookrightarrow}_{/\textbf{a}}$ is \textit{of height $0$} if $v\to a_0$ factors through $\partial a_0$ or $v\to a_{n-1}$ factors through $\partial a_{n-1}$. The \textit{height of an element $w$} is the maximal integer $m$ such that there exists a sequence 
$v_0\to v_1\to...\to v_m=w$ in $\Theta^{\hookrightarrow}_{/\textbf{a}}$ with $v_i\neq v_{i+1}$ for any $i<m$ and such that $v_0$ is of height $0$ and $v_1$ is not. As $\Theta$ is a Reedy category, all elements have finite height.


\begin{lemma}
\label{lemma:i etoile of W is in M 1.5}
For any morphism $p:[b,m]\to i^*[\textbf{a},n]$ that preserves extremal objects, there exists a unique integer $k$, a unique element $\{b'\to a_i\}_{i<n}$ of height $k$, and a unique morphism $[f,i]:[b,m]\to [b',n]$ that doesn't factors through $[\partial b',n]$, and such that the induced triangle
% https://q.uiver.app/#q=WzAsMyxbMCwwLCJbYixtXSJdLFsxLDAsIltiJyxuXSJdLFsxLDEsImleKltcXHRleHRiZnthfSxuXSJdLFsxLDIsInAnIl0sWzAsMSwiW2YsaV0iXSxbMCwyXV0=
\[\begin{tikzcd}
	{[b,m]} & {[b',n]} \\
	& {i^*[\textbf{a},n]}
	\arrow["{p'}", from=1-2, to=2-2]
	\arrow["{[f,i]}", from=1-1, to=1-2]
	\arrow[from=1-1, to=2-2]
\end{tikzcd}\]
commutes.

If $\{\tilde{b}\to a_i\}_{i<n}$ is any other object of non negative height, and $[\tilde{f},j]:[b,m]\to [\tilde{b},n]$ is a morphism that make the induced triangle
% https://q.uiver.app/#q=WzAsMyxbMCwwLCJbYixtXSJdLFsxLDAsIltcXHRpbGRle2J9LG5dIl0sWzEsMSwiaV4qW1xcdGV4dGJme2F9LG5dIl0sWzEsMiwiXFx0aWxkZXtwfSJdLFswLDEsIltcXHRpbGRle2Z9LGpdIl0sWzAsMl1d
\[\begin{tikzcd}
	{[b,m]} & {[\tilde{b},n]} \\
	& {i^*[\textbf{a},n]}
	\arrow["{\tilde{p}}", from=1-2, to=2-2]
	\arrow["{[\tilde{f},j]}", from=1-1, to=1-2]
	\arrow[from=1-1, to=2-2]
\end{tikzcd}\]
commutative, then $\{\tilde{b}\to a_i\}_{i<n}$ is of height strictly superior to $k$ and $[\tilde{f},j]$ factors through $[\partial\tilde{b},n]$.
\end{lemma}
\begin{proof}
The lemma \ref{lemma:i etoile of W is in M 1} implies the first assertion. For the second one, suppose given an object $\{\tilde{b}\to a_i\}_{i<n}$ of non negative height and a morphism  $[\tilde{f},j]:[b,m]\to [\tilde{b},n]$ fulfilling the desired condition. The bijection \eqref{eq:hom in theta} directly implies that $j$ is equal to $i$, and the first assertion  implies that $\tilde{f}$ is non degenerate.

We can then factor $\tilde{f}:b\to \tilde{b}$ in a degenerate morphism $b\to \bar{b}$ followed by a monomorphism $ \bar{b}\to \tilde{b}$ which is not the identity. The lemma \ref{lemma:i etoile of W is in M 0.5} then implies that $\{\bar{b}\to \tilde{b}\to a_i\}_{i<n}$ is an element of  $\Theta^{\hookrightarrow}_{/\textbf{a}}$. The first assertion then implies that the two morphisms $[b,m]\to [b',n]$ and $[b,m]\to {[\bar{b},n]}$ are equals. As the  monomorphism $  {[{b}',n]}={[\bar{b},n]}\to [\tilde{b},n]$ is not the identity, this concludes the proof.
\end{proof}

\begin{lemma}
\label{lemma:i etoile of W is in M 2}
The morphism $i^*[\partial^0\textbf{a},n]\cup i^*[\partial^{n-1}\textbf{a},n]\to i^*[\textbf{a},n]$ is in $\overline{\M}$, where $\partial^j\textbf{a}$ corresponds to the sequence $\{a_1,..,\partial a_j,..,a_n\}$. 
\end{lemma}
\begin{proof}
For $k\in\Nb\cup\{\infty\}$, we define $x_k$ as the smallest sub object of $i^*[\textbf{a},n]$ such that for any element 
of height inferior or equal to $k$ of $\Theta^{\hookrightarrow}_{/\textbf{a}}$, the corresponding morphism $[b,n]\to i^*[\textbf{a},n]$ factors through $x_k$. In particular we have $x_0= i^*[\partial^0\textbf{a},n]\cup i^*[\partial^{n-1}\textbf{a},n]$, and the lemma \ref{lemma:i etoile of W is in M 1} implies that $x_{\infty} =i^*[\textbf{a},n]$. 

Every morphism $[b,m]\to i^*[\textbf{a},n]$ that does not preserve extremal points then factors through $x_0$. 
The lemma \ref{lemma:i etoile of W is in M 1.5} implies that for any integer $k$, the canonical square 
% https://q.uiver.app/#q=WzAsNCxbMCwwLCJcXGNvcHJvZF97KFxcVGhldGFee1xcaG9va3JpZ2h0YXJyb3d9X3svXFx0ZXh0YmZ7YX19KV97aysxfX1bYixkXjBcXGN1cCBkXm5dXFxjdXBbXFxwYXJ0aWFsIGIsbl0iXSxbMSwwLCJ4X2siXSxbMSwxLCJ4X3trKzF9Il0sWzAsMSwiXFxjb3Byb2RfeyhcXFRoZXRhXntcXGhvb2tyaWdodGFycm93fV97L1xcdGV4dGJme2F9fSlfe2srMX19W2Isbl0iXSxbMCwzXSxbMywyXSxbMCwxXSxbMSwyXV0=
\begin{equation}
\label{eq:lemma:i etoile of W is in M 2}
\begin{tikzcd}
	{\coprod_{(\Theta^{\hookrightarrow}_{/\textbf{a}})_{k+1}}[b,d^0\cup d^n]\cup[\partial b,n]} & {x_k} \\
	{\coprod_{(\Theta^{\hookrightarrow}_{/\textbf{a}})_{k+1}}[b,n]} & {x_{k+1}}
	\arrow[from=1-1, to=2-1]
	\arrow[from=2-1, to=2-2]
	\arrow[from=1-1, to=1-2]
	\arrow[from=1-2, to=2-2]
\end{tikzcd}
\end{equation}
is cocartesian.  The lemma \ref{lemma:i etoile of W is in M 0} and the stability under pushout of $\overline{\M}$  imply that $x_k\to x_{k+1}$ is in $\overline{\M}$.
 As $i^*[\textbf{a},n]$ is the transfinite composition of the sequence $x_0\to x_1\to...$, this implies that $x_0\to i^*[\textbf{a},n]$ is in $\overline{\M}$ which conclude the proof.
\end{proof}

\begin{lemma}
The morphism $i^*\Sp_a\to i^*a$ is in $\overline{\M}$ for any globular sum $a$.
\end{lemma}
\begin{proof}
Let $[\textbf{a},n]:= a$. As $\overline{\M}$ is closed under pushouts and composition, lemma \ref{lemma:i etoile of W is in M 2} implies that the morphism 
$$i^*[\{a_0,...,a_{n-2}\},n-1]\cup i^*[\{a_1,...,a_{n-1}\},n-1]\to i^*[\textbf{a},n]$$
is in $\widehat{\M}$. 
An easy induction on $n$ shows that this is also the case for the morphism 
$$[a_0,1]\cup... \cup [a_{n-1},1]= i^*[a_0,1]\cup... \cup i^*[a_{n-1},1]\to i^*[\textbf{a},n].$$
Now remark that $i^*\Sp_{[\textbf{a},n]}$ is equivalent to 
$$[\Sp_{a_0},1]\cup... \cup [\Sp_{a_{n-1}},1].$$
As the morphisms $[\Sp_i,1]\to [a_i,1]$ are by definition in $\M$, this concludes the proof.
\end{proof}

\begin{prop}
\label{prop:i etoile of W is in M}
There is an inclusion $i^*\W\subset \overline{\M}$.
\end{prop}
\begin{proof}
For Segal extensions, this is precisely the content of the last lemma. For saturation extensions, remark that $i^*\Wsat = \Msat$.
\end{proof}



\begin{proof}[Proof of theorem \ref{theo:unit and counit are in W}]
Let $a$ be a globe. We then have $i_!i^*a = a$. Suppose now that $a$ is any globular sum. We then have a commutative diagram
% q.uiver.app/#q=WzAsNCxbMCwxLCJpXyFpXiphIl0sWzAsMCwiaV8haV4qXFxTcF9hIl0sWzEsMCwiXFxTcF9hIl0sWzEsMSwiYSJdLFsxLDBdLFsxLDIsIiIsMix7ImxldmVsIjoyLCJzdHlsZSI6eyJoZWFkIjp7Im5hbWUiOiJub25lIn19fV0sWzIsM10sWzAsM11d
\[\begin{tikzcd}
	{i_!i^*\Sp_a} & {\Sp_a} \\
	{i_!i^*a} & a
	\arrow[from=1-1, to=2-1]
	\arrow[Rightarrow, no head, from=1-1, to=1-2]
	\arrow[from=1-2, to=2-2]
	\arrow[from=2-1, to=2-2]
\end{tikzcd}\]
where the upper horizontal morphism is an identity.
The proposition \ref{prop:i etoile of W is in M} and the fact that $i_!(\M)\subset \W$ implies that the vertical morphisms of the previous diagram are in $\overline{\W}$. By left cancellation, this implies that $i_!i^*a\to a$ belongs to $\overline{\W}$ for any globular sum. We proceed analogously to show that for any $b\in \Delta[\Theta]$, $b\to i^*i_! b$ is in $\overline{\M}$.
\end{proof}









\section{Gray Operations}
\subsection{Recollection on Steiner theory}
\label{section:Steiner thery} 

We present here the Steiner theory developed in \cite{Steiner_omega_categories_and_chain_complexes}.


\p
An augmented directed complex $(K,K^*,e)$ is given by a complex of abelian groups $K$, with an augmentation $e$: $$\Zb \xleftarrow{e} K_0 \xleftarrow{\partial_0} K_1 \xleftarrow{\partial_1} K_2 \xleftarrow{\partial_2} K_3 \xleftarrow{\partial_3}. .. $$
and a graded set $K^* = (K^*_n)_{n\in\Nb}$ such that for any $n$, $K_n^*$ is a submonoid of $K_n$. A morphism of directed complexes between $(K,K^*,e)$ and $(L,L^*,e')$ is given by a morphism of augmented complexes of abelian groups $f : (K,e)\to (L,e')$ such that $f(K^*_n)\subset L^*_n$ for any $n$. We note by \wcnotation{$\CDA$}{(adc@$\CDA$} the category of augmented directed complexes. 

%
Steiner then constructs an adjunction
\[\begin{tikzcd}
	{\lambda:\omegacat} & {\CDA:\nu}
	\arrow[""{name=0, anchor=center, inner sep=0}, shift left=2, from=1-1, to=1-2]
	\arrow[""{name=1, anchor=center, inner sep=0}, shift left=2, from=1-2, to=1-1]
	\arrow["\dashv"{anchor=center, rotate=-90}, draw=none, from=0, to=1]
\end{tikzcd}\]
The functor $\lambda$ is the simplest to define: \sym{(lambda@$\lambda:\omegacat\to \CDA$}

\begin{definition}
Let $C$ be a $\omega$-category.
We denote by $(\lambda C)_n$ the abelian group generated by the set $\{[x]_n: x\in C_n\}$ and the relations
$$[x*_m y]_n \sim [x]_n + [y]_n \mbox{ for $m<n$ }.$$
We define the morphism $\partial_n: (\lambda C)_{n+1}\to (\lambda C)_n$ on generators by the formula:
$$\partial_n([x]_{n+1}) := [d_n^+ x]_{n} - [d_n^- x]_{n}.$$
\end{definition}
We can easily check that the morphism $\partial$ is a differential. We define an augmentation $e:(\lambda C)_{0}\to \Zb$ by setting $e([x]_0) = 1$ on generators. 
We denote by $(\lambda C)_n^*$ the additive submonoid generated by the elements $[x]_n$. We then set:
$$\lambda C := (\{(\lambda C)_n \}_{n\in \Nb},\{(\lambda C)^*_n \}_{n\in \Nb},e ).$$ This assignation lifts to a functor:
$$\begin{array}{ccccc}
\lambda &:& \omegacat&\to&\CDA\\
&&C&\mapsto&\lambda C.
\end{array}$$
\begin{example}~
\begin{enumerate}
\item
For any integer $n$, $\lambda\Db_n$ is the augmented directed complex whose underlying chain complex is given by:
$$
\Zb\xleftarrow{e}
\Zb[e_0^-,e_0^+] \xleftarrow{\partial_0}
... \xleftarrow{\partial_{n-2}}
\Zb[e_{n-1}^-,e_{n-1}^+] \xleftarrow{\partial_{n-1}}
\Zb[e_{n}] \xleftarrow{\partial_{n}}
0\leftarrow ...$$
where for any $0<k<n$ and $\alpha\in\{-,+\}$
$$e(e_0^\alpha)=1~~~\partial_{k-1}(e_k^\alpha)= e_{k-1}^+-e_{k-1}^-~~~\partial_{n-1}(e_n)= e_{n-1}^+-e_{n-1}^-.$$
\item
The augmented directed complex $\lambda[n]$ has for underlying chain complex:
$$
\Zb\xleftarrow{e}
\Zb[v_0,v_1,...,v_n] \xleftarrow{\partial_0}
\Zb[v_{0,1},v_{1,2}...,v_{n-1,n}] \xleftarrow{\partial_{1}}
0\leftarrow ...$$
where for any $k<n$ and $\alpha\in\{-,+\}$
$$e(v_k)=e(v_n)=1~~~ \partial_{1}(v_{k,k+1})=v_{k+1}-v_k.$$
\end{enumerate}
\end{example}


%

\p We now define the functor $\nu:\CDA\to \omegacat$. Throughout, we fix an augmented directed complex $(K,K^*,e)$.
A \textit{Steiner array} (or simply a \notion{array}) of dimension $n$ is the data of a finite double sequence: \sym{(nu@$\nu:\CDA\to \omegacat$}
$$\left(\begin{matrix}
x^-_0 &x^-_1&x^-_2&x^-_3 &...&x_n^-\\
x^+_0 &x^+_1&x^+_2&x^+_3 &...&x_n^+
\end{matrix}\right)$$
such that
\begin{enumerate}
\item $x^-_n=x^+_n$;
\item For any $i\leq n$ and $\alpha\in\{-,+\}$, $x_i^\alpha$ is an element of $K^*_i$;
\item For any $0<i\leq n$, $\partial_{i-1}(x_i^\alpha)= x_{i-1}^+ - x_{i-1}^-$;
\end{enumerate}
An array is said to be \wcnotion{coherent}{coherent array} if $e(x^+_0) = e(x^-_0) = 1$.
\begin{definition}
We define the globular set $\nu K$, whose $n$-cells are the coherent arrays of dimension $n$. The source and target maps are defined for $k<n$ by the formula: 

$$d^\alpha_k\begin{pmatrix}
x^-_0 &x^-_1&x^-_2&...&x^-_n\\
x^+_0 &x^+_1&x^+_2&...& x^+_n
\end{pmatrix} = \begin{pmatrix}
x^-_0 &x^-_1&x^-_2&...& x^-_{k-1}&x^\alpha_k\\
x^+_0 &x^+_1&x^+_2&...& x^+_{k-1}&x^\alpha_k\end{pmatrix}$$

There is an obvious group structure on the arrays:
$$\begin{pmatrix}
x^-_0 &x^-_1&...& x^-_n\\
x^+_0 &x^+_1&...& x^+_n
\end{pmatrix}
+
\begin{pmatrix}
y^-_0 &y^-_1&...& y^-_n\\
y^+_0 &y^+_1&...& y^+_n
\end{pmatrix}
=
\begin{pmatrix}
x^-_0+y^-_0 &x^-_1+ y^-_1&...&x^-_n+ y^-_n \\
x^+_0+y^+_0 &x^+_1+ y^+_1&...&x^+_n +y^+_n 
\end{pmatrix}
$$
\label{defi:definition of composition and units of nu k}

\begin{itemize}
\item[$-$]
For two coherent arrays $x$ and $y$ such that $d^-_k(x) =d^+_k(y) = z$, we define their $k$-composition by the following formula: 
$$x*_k y := x- z + y .$$ More explicitly:
$$\begin{pmatrix}
x^-_0 &...& x^-_n\\
x^+_0 &...& x^+_n
\end{pmatrix}
*_k
\begin{pmatrix}
y^-_0 &...& y^-_n\\
y^+_0 &...& y^+_n
\end{pmatrix}
 := 
\begin{pmatrix}
y^-_0&...&y_k^-& y_{k+1}^- + x_{k+1}^- & ...& y_{n}^- + x_{n}^-\\
x^+_0 &...&x_k^+& y_{k+1}^+ + x_{k+1}^+ & ...& y_{n}^+ + x_{n}^+ 
\end{pmatrix}
$$
\item[$-$]
For an integer $m>n$, we define the $m$-sized array $1^m_x$ as follows:
$$1^m_x :=
\begin{pmatrix}
x^-_0 &...& x^-_n& 0 &...&0\\
x^+_0 &...& x^+_n& 0 &...&0	
\end{pmatrix}$$
\end{itemize}
The globular set $\nu K$, equipped with these compositions and units is an $\omega$-category.
\end{definition}

\begin{definition}
We define the functor $\nu: \CDA \to \omegacat$ which associates to an augmented directed complex $K$, the $\omega$-category $\nu K$, and to a morphism of augmented directed complexes $f: K \to L$, the morphism of $\omega$-categories.
$$
\begin{array}{rccc}
\nu f : &\nu K &\to& \nu L\\
& \left(\begin{matrix}
x^-_0 &...&x_n^-\\
x^+_0&...&x_n^+
\end{matrix}\right) 
&\mapsto&
\left(\begin{matrix}
f_0(x^-_0) &...&f_n(x_n^-)\\
f_0(x^+_0)&...&f_n(x_n^+)
\end{matrix}\right) 
\end{array}
$$
\end{definition}


\begin{theorem}[Steiner]
\label{theo:ajdonction de steiner avec unite et counite explicite}
The functors $\lambda$ and $\nu$ form an adjoint pair 
\[\begin{tikzcd}
	{\lambda:\omegacat} & {\CDA:\nu}
	\arrow[""{name=0, anchor=center, inner sep=0}, shift left=2, from=1-1, to=1-2]
	\arrow[""{name=1, anchor=center, inner sep=0}, shift left=2, from=1-2, to=1-1]
	\arrow["\dashv"{anchor=center, rotate=-90}, draw=none, from=0, to=1]
\end{tikzcd}\]
For a $\omega$-category $C$, the unit of the adjunction is given by:
$$\begin{array}{rrcl}
~~~~~\eta :& C &\to & \nu \lambda C \\
& x\in C_n &\mapsto & 
\begin{pmatrix}
[d^-_0(x)]_0&...&[d^-_{n-1}(x)]_{n-1}&[x]_n\\
[d^+_0(x)]_0&...& [d^+_{n-1}(x)]_{n-1}&[x]_n
\end{pmatrix}
\end{array}
$$
For an augmented directed complex $K$, the counit is given by:
$$\begin{array}{rrcl}
\pi :& \lambda \nu K &\to & K~~~~~~~~~~~~~~~~ \\
& [x ]_n \in (\lambda \nu K)_n&\mapsto & x_n^+ = x_n^-
\end{array}
$$
\end{theorem}
\begin{proof}
This is \cite[theorem 2.11]{Steiner_omega_categories_and_chain_complexes}.
\end{proof}



%
\p 
A \snotion{basis}{for augmented directed complexes} for an augmented directed complex $(K,K^*,e)$ is a graded set $B = (B_n)_{n\in\Nb}$ such that for every $n$, $B_n$ is both a basis for the monoid $K_n^*$ and for the group $K_n$.
\begin{remark}
The elements of $B_n$ can be characterized as the minimal elements of $K_n^*\backslash{0}$ for the following order relation:
	$$x\leq y \mbox{ iff } y-x \in K_n^*$$
This shows that if a basis exists, it is unique.
\end{remark}
\p Any element of $K_n$ can then be written uniquely as a sum $\sum_{b\in B_n} \lambda_b b$. This leads us to define new operations:
For an element $x := \sum_{b\in B_n} \lambda_b b$ of $K_n$, we define the \textit{positive part} and the \textit{negative part}:
$$
\begin{array}{rcl}
(x)_+ &:=& \sum_{b\in B_n, \lambda_b> 0} ~\lambda_bb\\
(x)_- &:=& \sum_{b\in B_n, \lambda_b< 0} -\lambda_bb
\end{array}
$$
We then have $x = (x)_+ - (x)_-$. An element $x$ is \textit{positive} (resp. \textit{negative}) when $x =(x)_+$ (resp. when $x =-(x)_-$).
Let $y = \sum_{b\in B_n} \mu_b b$, we set : 
$$
\begin{array}{rcl}
x\wedge y &:=& \sum_{b\in B_n} \mbox{ min}(\lambda_b, \mu_b)~ b \\
\end{array}
$$
Eventually, we set \sym{(partialna@$\partial_n^+(\uvar)$}\sym{(partialnb@$\partial_n^-(\uvar)$}
$$
\begin{array}{rcl}
\partial_n^+(\uvar) &:=& (\partial_n(\uvar))_+ : K_{n+1}\to K^*_n\\
\partial_n^-(\uvar) &:= &(\partial_n(\uvar))_- : K_{n+1}\to K^*_n
\end{array}
$$


When an element $b$ of the basis is in the support of $x$, i.e $\lambda_b\neq 0$, we say that \textit{$b$ belongs to $x$}, which is denoted by $b\in x$.
\begin{example}
For any integer $n$, $\lambda\Db_n$ admits a basis, given by the graded set $B_{\lambda\Db_n}$ fulfilling:
$$(B_{\lambda\Db_n})_k:= \left\{ 
\begin{array}{ll}
\{e_k^-,e_k^+\}&\mbox{ if $k<n$}\\
\{e_n\}&\mbox{ if $k=n$}\\
\emptyset&\mbox{ if $k>n$}\\
\end{array}\right.$$ 
The augmented directed complex $\lambda[n]$ also admits a basis, given by the graded set $B_{\lambda\Db_n}$ fulfilling:
$$(B_{\lambda\Db_n})_k:= \left\{ 
\begin{array}{ll}
\{v_0,v_1,...,v_n\}&\mbox{ if $k=0$}\\
\{v_{0,1},v_{1,2}...,v_{n-1,n}\}&\mbox{ if $k=1$}\\
\emptyset&\mbox{ if k>1}\\
\end{array} \right.$$ 
\end{example}

%
\p
Let $a\in K^*_n$. We set by a decreasing induction on $k\leq n$ : 
 $$ \begin{array}{rclc}
 \langle a\rangle_k^\alpha &:= & a & \mbox{if $k = n$}\\
 &:= & \partial_k^\alpha\langle a\rangle^\alpha_{k+1} & \mbox{if not}
\end{array} 
$$
The array associated to $a$ is then: 
$$\langle a\rangle := \begin{pmatrix}
\langle a\rangle^-_0 &...&\langle a\rangle^-_{n-1}&a\\
\langle a\rangle^+_0 &...&\langle a\rangle^+_{n-1}&a
\end{pmatrix}$$
The basis is said to be \wcnotion{unitary}{unitary basis} when for any $b\in B$, the array $\langle b\rangle$ is coherent.




\p We define the relation $\odot$ on $B$ as being the smallest transitive and reflexive relation such that for any pair of elements of the basis $a,b$, 
$$a\odot_n b \mbox{ if } \mbox{($|a|>0$ and $b\in\langle a\rangle_{|a|-1}^-$)}~~\mbox{or}~~\mbox{($|b|>0$ and $a\in \langle b\rangle_{|b|-1}^+$)}$$
A basis is said to be \wcsnotion{loop free}{loop free basis}{for augmented directed complexes} when for any $n$, the relation $\odot_n$ is a (partial) order on $B$.
\begin{remark}
In \cite{Ara_Maltsiniotis_joint_et_tranche}, this notion is called \textit{strongly loop free}.
\end{remark}

\begin{example}
For any integer $n$, $\lambda\Db_n$ and $\lambda[n]$ admit a loop free and unitary basis.
\end{example}





\p We now define the subcategory \wcnotation{$\CDAB$}{(adcb@$\CDAB$} of $\CDA$ composed of augmented directed complexes which admit a unitary and loop free basis. 
We will now describe the analog of the notion of basis for $\omega$-categories. 

\begin{definition}
A $\omega$-category $C$ is \wcnotion{generated by composition}{generated by composition} by a set $E\subset C$ when any cell can be written as a composition of elements of $E$ and iterated units of elements of $E$. This set is a \snotion{basis}{for $\io$-categories} if $\{[e]_{d(e)}\}_{e\in E}$ is a basis of the augmented directed complex $\lambda C$. 
\end{definition}


\begin{prop}
An $\omega$-category $C$ that admits a basis is an $\zo$-category.
\end{prop}
\begin{proof}
Let $C$ be an $\omega$-category that admits a basis $E$. Suppose that there exists a non trivial $n$-cell $\alpha$ that admits an inverse $\beta$. We then have $[\alpha]_n+ [\beta]_n=[\alpha \circ_{n-1} \beta]_n =0$. As $\lambda C$ is free, we have $[\alpha]_n=0$. This implies the equality $[e]_n=0$ for any element $e\in E$ of dimension $n$ that appears in a decomposition of $\alpha$. This is obviously in contradiction with the fact that $\{[e]_{d(e)}\}_{e\in E}$ is a basis of the augmented directed complex $\lambda C$. 
\end{proof}


\begin{definition}
A basis $E$ of an $\zo$-category is : 
\begin{enumerate}
\item \wcsnotion{Loop free}{loop free basis}{for $\zo$-categories} when $\{[e]_{d(e)}\}_{e\in E}$ is.
\item \wcnotion{Atomic}{atomic basis} when $[d_n^+ e]_n \wedge [d_n^- e]_n = 0$ for any $e\in E$ and any natural number $n$ strictly smaller than the dimension of $e$. 
\end{enumerate}
\end{definition}



\begin{prop}
 If a loop free basis $E$ is atomic then $\{[e]\}_{e\in E}$ is unitary.
 \end{prop}
\begin{proof}
 This is \cite[proposition 4.6]{Steiner_omega_categories_and_chain_complexes}.
 \end{proof}

 
\begin{example}
For any integer $n$, $\Db_n$ and $[n]$ admit a loop free and atomic basis.
More generally, \cite[proposition 4.13]{Ara_Maltsiniotis_joint_et_tranche} states that 
any globular sum admits a loop free and atomic basis. 
\end{example}

\p Proposition $1.23$ of \cite{Ara_a_categorical_characterization_of_strong_Steiner_omega_categories} states that if an $\zo$-category admits a loop-free and atomic basis, it is unique.
We then define the category \wcnotation{$\zocatB$}{((a30@$\zocatB$} as the full subcategory of $\omegacat$ composed of $\zo$-categories admitting an atomic and loop-free basis.

 \begin{theorem}[Steiner]
 \label{theorem:steiner}
 Once restricted to $\zocat_B$ and $\CDAB$, the adjunction 
\[\begin{tikzcd}
	{\lambda:\omegacat} & {\CDA:\nu}
	\arrow[""{name=0, anchor=center, inner sep=0}, shift left=2, from=1-1, to=1-2]
	\arrow[""{name=1, anchor=center, inner sep=0}, shift left=2, from=1-2, to=1-1]
	\arrow["\dashv"{anchor=center, rotate=-90}, draw=none, from=0, to=1]
\end{tikzcd}\]
becomes an adjoint equivalence, i.e. :
$$ \lambda_{|\zocatB } \circ \nu_{|\CDAB} \cong id_{|\CDAB}~~~~~~~ id_{|\zocatB }\cong \nu_{|\CDAB} \circ \lambda_{|\zocatB }$$
\end{theorem}
\begin{proof}
See \cite[theorem 5.11]{Steiner_omega_categories_and_chain_complexes}.
\end{proof}

If $K$ is an augmented directed complex admitting a unitary and loop-free basis $B$, then the $\zo$-category $\nu K$ admits an atomic and loop-free basis given by the set $\langle B\rangle := \{\langle b\rangle,b\in B\}$. Conversely if an $\zo$-category $C$ admits an atomic and loop-free basis $E$, then the augmented directed complex $\lambda C$ admits a unitary and loop-free basis given by the family of sets $[E_n] := \{[e]_{d(e)}, e\in E_n\}$. 
The isomorphisms
$$\lambda \nu K\cong K \mbox{~~~ and ~~~} C\cong \nu\lambda C$$
induce isomorphisms:
$$[\langle B\rangle ]\cong B \mbox{~~~ and ~~~} E \cong \langle [E]\rangle.$$
\p We define the \snotion{full duality}{for augmented directed complexes} \ssym{((b80@$(\uvar)^{\circ}$}{for augmented directed complexes}
$$(\uvar)^\circ:\CDA\to \CDA$$
that sends a augmented directed complex $((K,\partial),K^*,e)$ to $((K, - \partial),K^*,e)$. We left the reader to check that $K^\circ$ admits a loop free and atomic basis when this is the case for $K$. This functor then induces a functor:
$$(\uvar)^\circ:\CDAB\to \CDAB.$$
Morever, we have a canonical equivalence: 
$$\lambda (C^\circ)\cong (\lambda C)^\circ$$
natural in $C$.

\p 
Let $f:M\to N$ be a morphism between two augmented directed complexes admitting unitary and loop-free bases $B_M$ and $B_N$. The morphism $f$ is \wcnotion{quasi-rigid}{quasi-rigid morphism} if for any $n$, and any $b\in (B_M)_n$,
$$f_n(b)\neq 0 ~\Rightarrow ~ f_n(b)\in B_N\mbox{ and }\nu(f)\langle b\rangle = \langle f_n(b)\rangle.$$


\begin{theorem}
\label{theo:Kan condition}
Suppose given a commutative square in $\CDAB$
% https://q.uiver.app/#q=WzAsNCxbMCwwLCJLIl0sWzEsMCwiTV8xIl0sWzAsMSwiTV8wIl0sWzEsMSwiTSJdLFswLDEsImteMCJdLFsxLDMsImxeMSJdLFswLDIsImteMCIsMl0sWzIsMywibF4wIiwyXV0=
\[\begin{tikzcd}
	K & {M_1} \\
	{M_0} & M
	\arrow["{k^0}", from=1-1, to=1-2]
	\arrow["{l^1}", from=1-2, to=2-2]
	\arrow["{k^0}"', from=1-1, to=2-1]
	\arrow["{l^0}"', from=2-1, to=2-2]
\end{tikzcd}\]
and such that all morphisms are quasi-rigid. Let $B_K,~B_{M_0},~B_{M_1},~B_{M}$ be the bases of $K,~M_0,~M_1,~ M$.

Then, this square is cocartesian if and only if for any $n$, the induced diagram of sets
% https://q.uiver.app/#q=WzAsNCxbMCwwLCIoQl97S30pX25cXGN1cFxcezBcXH0iXSxbMSwwLCIoQl97TV8xfSlfblxcY3VwXFx7MFxcfSJdLFswLDEsIihCX3tNXzB9KV9uXFxjdXBcXHswXFx9Il0sWzEsMSwiKEJfe019KV9uXFxjdXBcXHswXFx9Il0sWzAsMSwia14wX24iXSxbMSwzLCJsXjFfbiJdLFswLDIsImteMF9uIiwyXSxbMiwzLCJsXjBfbiIsMl1d
\[\begin{tikzcd}
	{(B_{K})_n\cup\{0\}} & {(B_{M_1})_n\cup\{0\}} \\
	{(B_{M_0})_n\cup\{0\}} & {(B_{M})_n\cup\{0\}}
	\arrow["{k^0_n}", from=1-1, to=1-2]
	\arrow["{l^1_n}", from=1-2, to=2-2]
	\arrow["{k^0_n}"', from=1-1, to=2-1]
	\arrow["{l^0_n}"', from=2-1, to=2-2]
\end{tikzcd}\]
is cocartesian. Furthermore, the induced square in $\zocat$
% https://q.uiver.app/#q=WzAsNCxbMCwwLCJcXG51IEsiXSxbMSwwLCJcXG51IE1fMSJdLFswLDEsIlxcbnUgTV8wIl0sWzEsMSwiXFxudSBNIl0sWzAsMSwiXFxudSBrXjAiXSxbMSwzLCJcXG51IGxeMSJdLFswLDIsIlxcbnUga14wIiwyXSxbMiwzLCJcXG51IGxeMCIsMl1d
\[\begin{tikzcd}
	{\nu K} & {\nu M_1} \\
	{\nu M_0} & {\nu M}
	\arrow["{\nu k^0}", from=1-1, to=1-2]
	\arrow["{\nu l^1}", from=1-2, to=2-2]
	\arrow["{\nu k^0}"', from=1-1, to=2-1]
	\arrow["{\nu l^0}"', from=2-1, to=2-2]
\end{tikzcd}\]
is cocartesian.
\end{theorem}
\begin{proof}
This is a combination of theorems 3.1.2 and 3.2.7 of \cite{Loubaton_condition_de_kan}.
\end{proof}



\subsection{Gray operations on augmented directed complexes}
We follow Steiner (\cite{Steiner_omega_categories_and_chain_complexes}) and Ara-Maltsiniotis (\cite{Ara_Maltsiniotis_joint_et_tranche}) for the definitions and first properties of Gray operations on augmented directed complexes.

\p 
Let $(K,K^*,e)$ and $(L,L^*,f)$ be two augmented directed complexes. We define the \snotion{Gray tensor product}{for augmented directed complexes} of $(K,K^*,e)$ and $(L,L^*,f)$ as the augmented directed complex
$$(K,K^*,e)\otimes (L,L^*,f):= (K\otimes L,(K\otimes L)^*,e\otimes f)$$
where 
\begin{enumerate}
\item[$-$] $K\otimes L$ is the chain complex whose value on $n$ is:
$$(K\otimes L)_n:= \oplus_{k+l=n}K_k\otimes L_l$$
and the differential is the unique graded group morphism fulfilling: 
$$\partial (x\otimes y):= \partial x\otimes y + (-1)^{|x|}x\otimes \partial y$$
where we set the convention $\partial x:=0$ if $|x|=0$.
\item[$-$] $(K\otimes L)^*$ is given on all integer $n$ by :
$$(K\otimes L)^*_n:= \oplus_{k+l=n}K_k^*\otimes L_l^*.$$
\item[$-$] $e\otimes f:K_0\otimes L_0\to \Zb$ is the unique morphism fulfilling 
$$(e\otimes f)(x\otimes y)= e(x)f(y).$$
\end{enumerate}
\p The Gray tensor product induces a monoidal structure on $\CDA$. Its unit is given by $\lambda \Db_0$. Furthermore, Steiner shows that if $K$ and $L$ admit loop free and unitary bases, so does $K\otimes L$. The monoidal structure then restricts to a monoidal structure on $\CDAB$. 
Eventually \cite[proposition A.20]{Ara_Maltsiniotis_joint_et_tranche} provides an equivalence 
\begin{equation}
\label{eq:dualities and otimes}
(K\otimes L)^\circ \cong K^\circ\otimes L^\circ
\end{equation}

\p To simplify notion, the augmented directed complex $\lambda[1]$ will simply be denoted by $[1]$. The induced functor 
$$\uvar\otimes [1]:\CDA\to \CDA$$
is called the \snotionsym{Gray cylinder}{((d30@$\uvar\otimes[1]$}{for augmented directed complexes}. 
For $(K,K^*,e)$ an augmented directed complex, we then have
$$(K,K^*,e)\otimes [1]:=(K\otimes [1] ,(K\otimes [1])^*,e)$$
where
\begin{enumerate}
\item[$-$] $K\otimes [1]$ is the chain complex whose value on $n$ is:
$$(K\otimes [1])_n:=\left\{
\begin{array}{ll}
\{x\otimes \{\epsilon\},x\in K_0,\epsilon=0,1\}&\mbox{if $n=0$}\\
\{x\otimes \{\epsilon\},x\in K_n,\epsilon=0,1\}\oplus \{x\otimes[1],x\in K_{n-1}\} &\mbox{if $n>0$}
\end{array}\right.$$
and the differential is the unique graded group morphism fulfilling: 
$$\partial (x\otimes [1]):= \partial x\otimes [1] + (-1)^{|x|}(x\otimes \{1\}-x\otimes \{0\} )~~~~~\partial (x\otimes\{\epsilon\}) = (\partial x)\otimes\{\epsilon\}$$
for $\epsilon\in\{0,1\}$, and
where we set the convention $\partial x:=0$ if $|x|=0$.
\item[$-$] $(K\otimes [1])^*$ is given on all integer $n$ by :
$$(K\otimes [1])^*_n:=\left\{
\begin{array}{ll}
\{x\otimes \{\epsilon\},x\in K^*_0,\epsilon=0,1\}&\mbox{if $n=0$}\\
\{x\otimes\{ \epsilon\},x\in K^*_n,\epsilon=0,1\}\oplus \{x\otimes[1],x\in K^*_{n-1}\} &\mbox{if $n>0$}
\end{array}\right.$$
\item[$-$] $e:(K\otimes [1])_0\to \Zb$ is the unique morphism fulfilling 
$$e(x\otimes \{0\})=e(x\otimes \{1\})= e(x).$$
\end{enumerate}



\p We define the \snotionsym{Gray cone}{((d40@$\uvar\star 1$}{for augmented directed complexes} and the \snotion{Gray $\circ$-cone}{for augmented directed complexes}\index[notation]{((d50@$1\overset{co}{\star}\_$!\textit{for augmented directed complexes}}:
$$\begin{array}{ccccccc}
\CDA &\to&\CDA&&\CDA &\to&\CDA\\
K&\mapsto &K\star 1 & &K &\mapsto &1\costar K
\end{array}
$$
where $K\star 1$ and $1\costar K$ are defined as the following pushout: 
% q.uiver.app/#q=WzAsOCxbMSwwLCJLXFxvdGltZXMgWzFdIl0sWzAsMCwiS1xcb3RpbWVzXFx7MVxcfSJdLFswLDEsIjEiXSxbMSwxLCJLXFxzdGFyIDEiXSxbMiwwLCJLXFxvdGltZXNcXHswXFx9Il0sWzMsMCwiS1xcb3RpbWVzIFsxXSJdLFszLDEsIjFcXGNvc3RhciBLIl0sWzIsMSwiMSJdLFsxLDJdLFsxLDBdLFsyLDNdLFswLDNdLFszLDEsIiIsMSx7InN0eWxlIjp7Im5hbWUiOiJjb3JuZXIifX1dLFs0LDddLFs3LDZdLFs0LDVdLFs1LDZdLFs2LDQsIiIsMSx7InN0eWxlIjp7Im5hbWUiOiJjb3JuZXIifX1dXQ==
\begin{equation}
\label{eq:defin of cstar costar CDA}
\begin{tikzcd}
	{K\otimes\{1\}} & {K\otimes [1]} & {K\otimes\{0\}} & {K\otimes [1]} \\
	1 & {K\star 1} & 1 & {1\costar K}
	\arrow[from=1-1, to=2-1]
	\arrow[from=1-1, to=1-2]
	\arrow[from=2-1, to=2-2]
	\arrow[from=1-2, to=2-2]
	\arrow["\lrcorner"{anchor=center, pos=0.125, rotate=180}, draw=none, from=2-2, to=1-1]
	\arrow[from=1-3, to=2-3]
	\arrow[from=2-3, to=2-4]
	\arrow[from=1-3, to=1-4]
	\arrow[from=1-4, to=2-4]
	\arrow["\lrcorner"{anchor=center, pos=0.125, rotate=180}, draw=none, from=2-4, to=1-3]
\end{tikzcd}
\end{equation}
The equation \eqref{eq:dualities and otimes} provides an equivalence
$$(C\star 1)^\circ\cong 1\costar C^\circ.$$
According to \cite[corollary 6.21]{Ara_Maltsiniotis_joint_et_tranche} and to the previous equivalence, if $K$ admits a loop free and unitary basis, this is also the case for $K\star 1$ and $1\costar K$. The {Gray cone} and the {Gray $\circ$-cone} then induce functors:
$$\begin{array}{ccccccc}
\CDAB&\to&\CDAB&&\CDAB &\to&\CDAB\\
K&\mapsto &K\star 1 & &K &\mapsto &1\costar K
\end{array}
$$

\p Unfolding the definition, we have
$$(K,K',e)\star 1:=(K\star 1, (K\star 1)^*,e)~~~~~1\costar(K,K',e):=(1\costar K, (1\costar K)^*,e)$$
where
\begin{enumerate}
\item[$-$] $K\star 1$ and $1\costar K$ are the chain complex whose value on $n$ are:
$$(K\star 1)_n:=\left\{
\begin{array}{ll}
\Zb[\emptyset\star 1]\oplus \{x\star \emptyset,x\in K_0\}&\mbox{if $n=0$}\\
\{\emptyset\star x,x\in K_n\}\oplus \{x\star 1,x\in K_{n-1}\} &\mbox{if $n>0$}
\end{array}\right.$$
$$(1\costar K)^n:=\left\{
\begin{array}{ll}
\Zb[1\costar\emptyset]\oplus \{\emptyset\costar x,x\in K_0\}&\mbox{if $n=0$}\\
\{\emptyset\costar x,x\in K_n\}\oplus \{1\costar x,x\in K_{n-1}\} &\mbox{if $n>0$}
\end{array}\right.$$
and the differentials are the unique graded group morphisms fulfilling: 
$$\begin{array}{rr}
\partial (x\star 1)= \partial x\star 1 + (-1)^{|x|} x\star \emptyset&\partial( x \star \emptyset )=\partial x\star \emptyset \\
 \partial (1\costar x)= 1\costar \partial x + (-1)^{|x|} \emptyset\costar x& \partial( \emptyset \costar x )= \emptyset \costar x\\
\end{array}$$
where we set the convention $\partial x:=0$ if $|x|=0$.
\item[$-$] The graded monoids $(K\star 1)^*$ and $(1\costar K)^*$ are given on all integer $n$ by :
$$(K\star 1)^*:=\left\{
\begin{array}{ll}
\Nb[\emptyset\star 1]\oplus \{x\star \emptyset,x\in K^*_0\}&\mbox{if $n=0$}\\
\{\emptyset\star x,x\in K^*_n\}\oplus \{x\star 1,x\in K^*_{n-1}\} &\mbox{if $n>0$}
\end{array}\right.$$
$$(1\costar K)^*:=\left\{
\begin{array}{ll}
\Nb[1\costar\emptyset]\oplus \{\emptyset\costar x,x\in K^*_0\}&\mbox{if $n=0$}\\
\{\emptyset\costar x,x\in K^*_n\}\oplus \{1\costar x,x\in K^*_{n-1}\} &\mbox{if $n>0$}
\end{array}\right. .$$
\item[$-$] The augmentations $e:(K\star 1)_0\to \Zb$ and $e:(1\costar K)_0\to\Zb$ are the unique ones fulfilling 
$$
\begin{array}{cc}
e( \emptyset \star 1) =1 & e(x\star \emptyset)=e(x)\\
e( 1\costar \emptyset ) =1 & e( \emptyset\costar x)=e(x).
\end{array}$$
\end{enumerate}



\begin{prop}
\label{prop:non trivial automorphisme 1}
Let $A$ be an augmented directed complex admitting no non-trivial automorphisms. Then the augmented directed complexes $A\star 1$ and $1\costar A$ have no non-trivial automorphisms.
\end{prop}
\begin{proof}
Let $\phi:A\star 1\to A\star 1$ be an automorphism. The morphism $\phi$ then induces a bijection on the elements of the basis of $A\star 1$.

 As the element $\emptyset\star 1\in (A\star 1)_0$ is the only element of the basis such that for all $v\in (A\star 1)_1$  $\partial_0^-(v)\neq \emptyset\star 1$, it is preserved by $\phi$. As a consequence, for any element $x$ of the basis of $A_0$, $\phi(x\star \emptyset)$ is of shape $x'\star \emptyset$. The morphism $\phi$ then preserves $(A\star \emptyset)_0$.

Now, remark that for any element $e\in (A\star 1)^*_{n+1}$, there exists $x\in (A\star 1)^*_n$ such that $x\star 1\leq e$ if and only if there exists $y\in (A\star 1)^*_{n-1}$ such that $y\star 1\leq \partial^+(e)$. By a direct induction, this implies that there exists $x\in (A\star 1)^*_n$ such that $x\star 1\leq e$ if and only if $\partial^+_0(e)\in \Zb[\emptyset\star 1]$.

Combined with the previous observation, this implies that for any element $x$ of the basis of $A_{n+1}$, $\phi(x\star \emptyset)$ is of shape $x'\star \emptyset$.
The automorphism $\phi$ then induces by restriction an automorphism $\phi_{|A\star\emptyset}:A\to A$, and the hypothesis implies that it is the identity.

We now show by induction on $n$ that $\phi_n:(A\star 1)_n\to (A\star 1)_n$ is the identity. Suppose the result true at the stage $n$. For any element $x$ of the basis of $A_{n}$, we then have 
$$\partial \phi(x\star 1) = \phi(\partial (x\star 1)) = \partial (x\star 1).$$
By the definition of the derivative of $A\star 1$, and as $\phi$ preserves the basis, this forces the equality $\phi(x\star 1)=x\star 1$. As we already know that for any element $x$ of the basis of $A_{n+1}$ we have $\phi(x\star \emptyset)=x\star \emptyset$, this concludes the induction.

We then have $\phi=id$ and $A\star 1$ has no non trivial automorphisms.
The case $1\costar A$ follows directly by using the fact that dualities preserve  augmented directed complexes admitting no non-trivial automorphisms. 

\end{proof}


\p 
We define the \snotionsym{suspension}{((d60@$[\uvar,1]$}{for augmented directed complexes} as the functor 
$$[\uvar,1]:\CDA\to \CDA$$
where $[K,1]$ is defined as the following pushout:
% q.uiver.app/#q=WzAsNCxbMSwxLCJbSywxXSJdLFsxLDAsIktcXG90aW1lcyBbMV0iXSxbMCwwLCJLXFxvdGltZXMgXFx7MCwxXFx9Il0sWzAsMSwiMVxcY29wcm9kIDEiXSxbMiwzXSxbMywwXSxbMiwxXSxbMSwwXSxbMCwyLCIiLDAseyJzdHlsZSI6eyJuYW1lIjoiY29ybmVyIn19XV0=
\begin{equation}
\label{eq:def of suspension cda}
\begin{tikzcd}
	{K\otimes \{0,1\}} & {K\otimes [1]} \\
	{1\coprod 1} & {[K,1]}
	\arrow[from=1-1, to=2-1]
	\arrow[from=2-1, to=2-2]
	\arrow[from=1-1, to=1-2]
	\arrow[from=1-2, to=2-2]
	\arrow["\lrcorner"{anchor=center, pos=0.125, rotate=180}, draw=none, from=2-2, to=1-1]
\end{tikzcd}
\end{equation}
We leave to the reader to check that $[K,1]$ admits a loop free and unitary basis when this is the case for $K$. This functor then induces a functor:
$$[\uvar,1]:\CDAB\to \CDAB$$

\p Unfolding the definition, we have
$$[(K,K',e),1]:=([K,1] ,([K,1])^*,e)$$
where
\begin{enumerate}
\item[$-$] $[K,1]$ is the chain complex whose value on $n$ is:
$$[K,1]:=\left\{
\begin{array}{ll}
 \Zb[\{0\},\{1\}]&\mbox{if $n=0$}\\
 \{[x,1],x\in K_{n-1}\} &\mbox{if $n>0$}
\end{array}\right.$$
and the differential is the unique graded group morphism fulfilling: 
$$\partial([x,1]):= \left\{
 \begin{array}{lll} 
 \{1\}-\{0\}&\mbox{if $|x|=0$}\\ 
 ~[\partial x,1]&\mbox{if $|x|>0$}
 \end{array}\right.
$$
\item[$-$] $([K,1])^*$ is given on all integer $n$ by:
$$([K,1])^*_n:=\left\{
\begin{array}{ll}
\Nb[0,1]&\mbox{if $n=0$}\\
 \{[x,1],x\in K^*_{n-1}\} &\mbox{if $n>0$}
\end{array}\right.$$
\item[$-$] $e:([K,1])_0\to \Zb$ is the unique morphism	 fulfilling 
$$e( 0)=e( 1)= e(x).$$
\end{enumerate}


\begin{prop}
\label{prop:non trivial automorphisme 2}
Let $A$ be a non null augmented directed complex admitting no non-trivial automorphisms. Then the augmented directed complex $[A,1]$ has no non-trivial automorphisms.
\end{prop}
\begin{proof}
Let $\phi:[A,1]\to [A,1]$ be an automorphism. As the element $\{1\}\in ([A,1])_0$ is the only element of the basis such that for all $v\in [A,1]_1$  $\partial_0^-(v)\neq \{1\}$, it is preserved by $\phi$. As a consequence, $\phi$ also preserves $\{0\}$. The induced morphism $\phi_0:[A,1]_0\to [A,1]_0$ is then the identity. 

Now, remark that $(\phi_{n+1})_{n\in \Nb}:A\to A$ is an automorphism and is then the identity. This implies that for all $n>0$, $\phi_n:[A,1]_n\to [A,1]_n$ is then identity, which concludes the proof.
\end{proof}

\p 
We define the \textit{wedges} as the functors
$$[\uvar,1]\vee[1]:\CDA\to \CDA~~~~~~ [1]\vee[\uvar,1]:\CDA\to \CDA$$
where $[K,1]\vee [1]$ and $[1]\vee[K,1]$ are defined as the following pushouts:
% q.uiver.app/#q=WzAsOCxbMSwxLCIgW0ssMV1cXHZlZVxcbGFtYmRhWzFdIl0sWzEsMCwiXFxsYW1iZGFbMV0iXSxbMCwwLCJcXGxhbWJkYSBbMF0iXSxbMCwxLCJbSywxXSJdLFs0LDEsIlxcbGFtYmRhWzFdXFx2ZWVbSywxXSJdLFs0LDAsIltLLDFdIl0sWzMsMCwiXFxsYW1iZGEgWzBdIl0sWzMsMSwiXFxsYW1iZGFbMV0iXSxbMiwzLCJcXHsxXFx9IiwyXSxbMywwXSxbMiwxLCJcXHswXFx9Il0sWzEsMF0sWzAsMiwiIiwwLHsic3R5bGUiOnsibmFtZSI6ImNvcm5lciJ9fV0sWzYsNSwiXFx7MFxcfSJdLFs2LDcsIlxcezFcXH0iLDJdLFs1LDRdLFs3LDRdLFs0LDYsIiIsMSx7InN0eWxlIjp7Im5hbWUiOiJjb3JuZXIifX1dXQ==
\[\begin{tikzcd}
	{\lambda [0]} & {[1]} && {\lambda [0]} & {[K,1]} \\
	{[K,1]} & { [K,1]\vee[1]} && {[1]} & {[1]\vee[K,1]}
	\arrow["{\{1\}}"', from=1-1, to=2-1]
	\arrow[from=2-1, to=2-2]
	\arrow["{\{0\}}", from=1-1, to=1-2]
	\arrow[from=1-2, to=2-2]
	\arrow["\lrcorner"{anchor=center, pos=0.125, rotate=180}, draw=none, from=2-2, to=1-1]
	\arrow["{\{0\}}", from=1-4, to=1-5]
	\arrow["{\{1\}}"', from=1-4, to=2-4]
	\arrow[from=1-5, to=2-5]
	\arrow[from=2-4, to=2-5]
	\arrow["\lrcorner"{anchor=center, pos=0.125, rotate=180}, draw=none, from=2-5, to=1-4]
\end{tikzcd}\]
Once again, we can easily check that $[K,1]\vee[1]$ and $[1]\vee[K,1]$ have a loop free and unitary basis when this is the case for $K$. These functors then induce functors
$$[\uvar,1]\vee[1]:\CDAB\to \CDAB~~~~~~ [1]\vee[\uvar,1]:\CDAB\to \CDAB$$
%
\p Unfolding the definition, we have
$$[(K,K',e),1]\vee [1]:=([K,1]\vee [1] ,([K,1]\vee [1])^*,e)$$ $$
[1]\vee(K,K',e),1]:=([1]\vee[K,1] ,([1]\vee[K,1])^*,e)$$
where
\begin{enumerate}
\item[$-$] $[K,1]\vee [1]$ and $[1]\vee[K,1]$ are the chain complexes whose value on $n$ are:
$$[K,1]\vee[1]:=\left\{
\begin{array}{ll}
\Zb[\{0\},\{1\},\{2\}]&\mbox{if $n=0$}\\
 \{[x,1],x\in K_{0}\}\oplus \Zb[e_1] &\mbox{if $n=1$}\\
 \{[x,1],x\in K_{n-1}\} &\mbox{if $n>1$}
\end{array}\right.$$
$$[1]\vee[K,1]:=\left\{
\begin{array}{ll}
\Zb[\{0\},\{1\},\{2\}]&\mbox{if $n=0$}\\
\Zb[e_1] \oplus \{[x,1],x\in K_{0}\} &\mbox{if $n=1$}\\
 \{[x,1],x\in K_{n-1}\} &\mbox{if $n>1$}
\end{array}\right.$$
and the differentials are the unique graded group morphism fulfilling: 
$$\partial_{[K,1]\vee[1]} (e_1):= \{2\}-\{1\}
~~~
\partial_{[K,1]\vee[1]} ([x,1]):=
\left\{
\begin{array}{ll}
 \{1\}-\{0\}&\mbox{if $|x|=0$}\\
 ~[\partial x,1]&\mbox{if $|x|>0$}\\
\end{array}\right.
$$
$$
\partial_{[1]\vee[K,1]} (e_1):= \{1\}-\{0\}
~~~
\partial_{[1]\vee[K,1]} ([x,1]):=
\left\{
\begin{array}{ll}
 \{2\}-\{1\}&\mbox{if $|x|=0$}\\
~ [\partial x,1]&\mbox{if $|x|>0$}\\
\end{array}\right.
$$
\item[$-$] $([K,1]\vee [1])^*$ and $([1]\vee[K,1])^*$ are given on all integer $n$ by:
$$([K,1]\vee[1])^*:=\left\{
\begin{array}{ll}
\{\{0\},\{1\},\{2\}\}&\mbox{if $n=0$}\\
 \{[x,1],x\in K_0^*\}\oplus \Nb[e_1] &\mbox{if $n=1$}\\
 \{[x,1],x\in K_{n-1}\} &\mbox{if $n>1$}
\end{array}\right.$$
$$([1]\vee[K,1])^*:=\left\{
\begin{array}{ll}
\{\{0\},\{1\},\{2\}\}&\mbox{if $n=0$}\\
\Nb[e_1]\oplus\cup \{[x,1],x\in K^*_{0}\} &\mbox{if $n=1$}\\
 \{[x,1],x\in K^*_{n-1}\} &\mbox{if $n>1$}
\end{array}\right.$$
\item[$-$] The augmentations $e$ are the unique morphism fulfilling 
$$e( \{0\})=e(\{ 1\})= e(\{2\})=1.$$
\end{enumerate}




\begin{prop}
\label{prop:non trivial automorphisme 3}
Let $A$ be a non null augmented directed complex admitting no non-trivial automorphisms. Then the augmented directed complexes $[A,1]\vee[1]$ and $[1]\vee[A,1]$ have no non-trivial automorphisms.
\end{prop}
\begin{proof}
The proof is similar to the one of proposition \ref{prop:non trivial automorphisme 2} and we leave it to the reader.
\end{proof}

\p 
There are two canonical morphisms 
$$\triangledown:\Sigma K\to \Sigma K \vee [1]
~~~~~~~ \triangledown:\Sigma K\to [1]\vee \Sigma K $$
that are the unique ones fulfilling
$$\triangledown(\{0\}):= \{0\}~~~\triangledown(\{1\}):= \{2\}~~~
\triangledown([x,1]):=\left\{ 
\begin{array}{ll}
~[x,1]+e_1&\mbox{if $|x|=0$}\\
~[x,1]&\mbox{if $|x|>0$}\\
\end{array}\right.$$
When we write $ \Sigma K\to \Sigma K \vee [1]$ and $\Sigma K\to [1]\vee \Sigma K$ and nothing more is specified, it will always mean that we considered the morphisms $\triangledown$.

\begin{prop}
 \label{prop:appendice formula for otimes cda}
 Let $K$ be an augmented directed complex. 
 There is a natural transformation between the colimit of the following diagram
$$
\begin{tikzcd}
	{[1]\vee [K,1]} & {[K\otimes\{0\},1]} & {[K\otimes [1],1]} & {[K\otimes\{1\},1]} & {[K,1]\vee [1]}
	\arrow[from=1-2, to=1-1]
	\arrow[from=1-2, to=1-3]
	\arrow[from=1-4, to=1-3]
	\arrow[from=1-4, to=1-5]
\end{tikzcd}$$
and $[K,1]\otimes [1]$.
\end{prop}
\begin{proof}
The cone is induced by morphisms
$$
\begin{array}{rl}
&[1]\vee [K,1]\to [K,1]\otimes [1]\\
(\mbox{resp}.&[ K,1]\vee[1]\to [ K,1] \otimes [1])
\end{array}
$$ sending an element $x$ in the basis of $[1]$ to $\{0\}\otimes x$ (resp. $\{1\}\otimes x$), an element $y$ in the basis of $[ K,1]$ to $y\otimes\{1\}$ (resp. $y\otimes\{0\}$), 
and by the morphism 
$$f:[K\otimes [1],1]\to [K,1]\otimes [1]$$
defined by the formula 
$$f([x\otimes y,1]):= [ x,1]\otimes y$$ 
for $x$ in the basis of $K$ and $y$ in the basis of $[1]$.
We leave it to the reader to check the compatibilities of this three morphisms.
\end{proof}




\subsection{Gray operations on $\zo$-categories}
\label{section:definition of Gray operations}
We follow Ara-Maltsiniotis \cite{Ara_Maltsiniotis_joint_et_tranche} for the definitions and first properties of Gray operations on $\zo$-categories. Originally, these authors work with $\omega$-categories, and not with $\zo$-categories. However, this modification does not affect proof, and we then allow ourselves to use their results in our framework.


\begin{theorem}[Steiner, Ara-Maltsiniotis]
There is a unique colimit preserving monoidal structure on $\zocat$,
up to a unique monoidal isomorphism, making the functor
$\nu_{|\CDAB}:\CDAB\to \zocat$
a monoidal functor, when $\CDAB$ is endowed with the monoidal structure given by the Gray tensor product.
\end{theorem}
\begin{proof}
This is \cite[theorem A.15]{Ara_Maltsiniotis_joint_et_tranche}.
\end{proof}

\p The monoidal product on $\zocat$ induced by the previous theorem is called the \snotionsym{Gray tensor product}{((d00@$\otimes$}{for $\zo$-categories} and is denoted by $\otimes$. It's unit is $ \Db_0$. If $C$ and $D$ are $\zo$-categories with an atomic and loop free basis, we have by construction
$$C\otimes D := \nu(\lambda C\otimes \lambda D).$$
The induced functor 
$$\uvar\otimes[1]:\zocat\to \zocat$$
is called the \snotionsym{Gray cylinder}{((d30@$\uvar\otimes[1]$}{for $\zo$-categories}.


\begin{prop}
Let $C$ be an $\io$-category.
The following canonical square 
% q.uiver.app/#q=WzAsNCxbMSwwLCJDXFxvdGltZXNbMV0iXSxbMCwwLCJDXFxvdGltZXNcXHswLDFcXH0iXSxbMCwxLCIxXFxjb3Byb2QgMSJdLFsxLDEsIltDLDFdIl0sWzEsMl0sWzEsMF0sWzIsM10sWzAsM10sWzMsMSwiIiwxLHsic3R5bGUiOnsibmFtZSI6ImNvcm5lciJ9fV1d
\[\begin{tikzcd}
	{C\otimes\{0,1\}} & {C\otimes[1]} \\
	{1\coprod 1} & {[C,1]}
	\arrow[from=1-1, to=2-1]
	\arrow[from=1-1, to=1-2]
	\arrow[from=2-1, to=2-2]
	\arrow[from=1-2, to=2-2]
	\arrow["\lrcorner"{anchor=center, pos=0.125, rotate=180}, draw=none, from=2-2, to=1-1]
\end{tikzcd}\]
is cocartesian
\end{prop}
\begin{proof}
As all these functors commute with colimits, it is sufficient to demonstrate this assertion when $C$ is a globular sum, and \textit{a fortiori} when $C$ admits a loop free and atomic basis. In this case, remark that all the morphisms appearing in canonical cartesian square
% https://q.uiver.app/#q=WzAsNCxbMSwwLCJcXGxhbWJkYSBDXFxvdGltZXNbMV0iXSxbMCwwLCJcXGxhbWJkYSBDXFxvdGltZXNcXHswLDFcXH0iXSxbMCwxLCIxXFxjb3Byb2QgMSJdLFsxLDEsIltcXGxhbWJkYSBDLDFdIl0sWzEsMl0sWzEsMF0sWzIsM10sWzAsM10sWzMsMSwiIiwxLHsic3R5bGUiOnsibmFtZSI6ImNvcm5lciJ9fV1d
\[\begin{tikzcd}
	{\lambda C\otimes\{0,1\}} & {\lambda C\otimes[1]} \\
	{1\coprod 1} & {[\lambda C,1]}
	\arrow[from=1-1, to=2-1]
	\arrow[from=1-1, to=1-2]
	\arrow[from=2-1, to=2-2]
	\arrow[from=1-2, to=2-2]
	\arrow["\lrcorner"{anchor=center, pos=0.125, rotate=180}, draw=none, from=2-2, to=1-1]
\end{tikzcd}\]
 are quasi-rigid. 
The results then follow from an application of theorem \ref{theo:Kan condition}.
\end{proof}



\p 
\label{para:explicit Dbn otiomes [1]}
Applying the duality $(\uvar)^{op}$ to the computation achieved in appendix B.1 of \cite{Ara_Maltsiniotis_joint_et_tranche}, we can give an explicit expression of $\Db_n\otimes [ 1]$. As a polygraph, the generating arrows of $\Db_n\otimes [1]$ are:
$$ e^\epsilon_k\otimes\{0\}~~~~~e^\epsilon_k\otimes\{1\}~~~~~e^\epsilon_k\otimes[1]$$
 \[ a^-_0 \otimes e^\epsilon_k \qquad a^+_0 \otimes e^\epsilon_k \qquad a \otimes e^\epsilon_k \]
 where $\epsilon$ is either $+$ or $-$, $k \leqslant n$ and $e^+_n = e^-_n$. Their source and target are given as follows:
 \[ \pi^-( e^\epsilon_k \otimes\{0\}) = e^-_{k-1} \otimes\{0\} \qquad\qquad\qquad \pi^+(e^\epsilon_k \otimes\{0\}) = e^+_{k-1}\otimes\{0\} \]
%
 \[ \pi^-(e^\epsilon_k \otimes\{1\} ) = e^-_{k-1} \otimes\{1\}\qquad\qquad\qquad \pi^+(e^\epsilon_k\otimes\{1\} ) = e^+_{k-1}\otimes\{1\} \]
%
$$\pi^{-}(e^\epsilon_{2k}\otimes[1]) =...\circ_2(e^+_0\otimes[1])\circ_0(e^\epsilon_{2k}\otimes\{0\})\circ_1 (e^-_1\otimes[1])\circ_3... \circ_{2k-1}(e_{2k-1}^-\otimes[1])$$
%
$$\pi^{+}(e^\epsilon_{2k}\otimes[1]) = (e_{2k-1}^+\otimes[1])\circ_{2k-1}...\circ_3(e^+_1\otimes[1])\circ_1(e^\epsilon_{2k}\otimes\{1\})\circ_0 (e^-_0\otimes[1])\circ_2...$$
%
$$\pi^{-}(e^\epsilon_{2k+1}\otimes[1]) = ...\circ_3(e^+_1\otimes[1])\circ_1(e^\epsilon_{2k+1}\otimes\{1\})\circ_0 (e^-_0\otimes[1])\circ_2...\circ_{2k}(e_{2k}^-\otimes[1])$$
%
$$\pi^{+}(e^\epsilon_{2k+1}\otimes[1]) = (e_{2k}^+\otimes[1])\circ_{2k}...\circ_2(e^+_0\otimes[1])\circ_0(e^\epsilon_{2k+1}\otimes\{0\})\circ_1 (e^-_1\otimes[1])\circ_3...$$
 We did not put parenthesis in the expression above, to keep them shorter, the default convention is to do the composition $\circ_i$ in order of increasing values of $i$.
 
\begin{example}
The $\zo$-category $\Db_1\otimes[1]$ is the polygraph: 
% https://q.uiver.app/?q=WzAsNCxbMCwwLCIwMCJdLFswLDEsIjEwIl0sWzEsMSwiMTEiXSxbMSwwLCIwMSJdLFswLDFdLFsxLDJdLFswLDNdLFszLDJdLFszLDEsIiIsMSx7InNob3J0ZW4iOnsic291cmNlIjoyMCwidGFyZ2V0IjoyMH0sImxldmVsIjoyfV1d
\[\begin{tikzcd}
	00 & 01 \\
	10 & 11
	\arrow[from=1-1, to=2-1]
	\arrow[from=2-1, to=2-2]
	\arrow[from=1-1, to=1-2]
	\arrow[from=1-2, to=2-2]
	\arrow[shorten <=4pt, shorten >=4pt, Rightarrow, from=1-2, to=2-1]
\end{tikzcd}\]
The $\zo$-category $\Db_2\otimes[1]$ is the polygraph: 
% https://q.uiver.app/?q=WzAsOCxbMSwwLCIwMSJdLFswLDAsIjAwIl0sWzAsMSwiMTAiXSxbMSwxLCIxMSJdLFsyLDAsIjAwIl0sWzMsMCwiMDEiXSxbMywxLCIxMSJdLFsyLDEsIjEwIl0sWzEsMF0sWzEsMl0sWzIsM10sWzAsM10sWzAsMiwiIiwxLHsic2hvcnRlbiI6eyJzb3VyY2UiOjIwLCJ0YXJnZXQiOjIwfSwibGV2ZWwiOjJ9XSxbNCw3XSxbNCw1XSxbNSw2XSxbNSw3LCIiLDEseyJzaG9ydGVuIjp7InNvdXJjZSI6MjAsInRhcmdldCI6MjB9LCJsZXZlbCI6Mn1dLFsxLDIsIiIsMix7ImN1cnZlIjo1fV0sWzcsNl0sWzUsNiwiIiwxLHsiY3VydmUiOi01fV0sWzksMTcsIiAiLDIseyJzaG9ydGVuIjp7InNvdXJjZSI6MjAsInRhcmdldCI6MjB9fV0sWzE5LDE1LCIgIiwyLHsic2hvcnRlbiI6eyJzb3VyY2UiOjIwLCJ0YXJnZXQiOjIwfX1dLFsxMSwxMywiIiwwLHsib2Zmc2V0IjotMSwic2hvcnRlbiI6eyJzb3VyY2UiOjIwLCJ0YXJnZXQiOjIwfSwibGV2ZWwiOjEsInN0eWxlIjp7ImhlYWQiOnsibmFtZSI6Im5vbmUifX19XSxbMTEsMTMsIiIsMix7Im9mZnNldCI6MSwic2hvcnRlbiI6eyJzb3VyY2UiOjIwLCJ0YXJnZXQiOjIwfSwibGV2ZWwiOjEsInN0eWxlIjp7ImhlYWQiOnsibmFtZSI6Im5vbmUifX19XSxbMTEsMTMsIiIsMSx7InNob3J0ZW4iOnsic291cmNlIjoyMCwidGFyZ2V0IjoyMH0sImxldmVsIjoxfV1d
\[\begin{tikzcd}
	00 & 01 & 00 & 01 \\
	10 & 11 & 10 & 11
	\arrow[from=1-1, to=1-2]
	\arrow[""{name=0, anchor=center, inner sep=0}, from=1-1, to=2-1]
	\arrow[from=2-1, to=2-2]
	\arrow[""{name=1, anchor=center, inner sep=0}, from=1-2, to=2-2]
	\arrow[shorten <=4pt, shorten >=4pt, Rightarrow, from=1-2, to=2-1]
	\arrow[""{name=2, anchor=center, inner sep=0}, from=1-3, to=2-3]
	\arrow[from=1-3, to=1-4]
	\arrow[""{name=3, anchor=center, inner sep=0}, from=1-4, to=2-4]
	\arrow[shorten <=4pt, shorten >=4pt, Rightarrow, from=1-4, to=2-3]
	\arrow[""{name=4, anchor=center, inner sep=0}, curve={height=30pt}, from=1-1, to=2-1]
	\arrow[from=2-3, to=2-4]
	\arrow[""{name=5, anchor=center, inner sep=0}, curve={height=-30pt}, from=1-4, to=2-4]
	\arrow["{ }"', shorten <=6pt, shorten >=6pt, Rightarrow, from=0, to=4]
	\arrow["{ }"', shorten <=6pt, shorten >=6pt, Rightarrow, from=5, to=3]
	\arrow[shift left=0.7, shorten <=6pt, shorten >=8pt, no head, from=1, to=2]
	\arrow[shift right=0.7, shorten <=6pt, shorten >=8pt, no head, from=1, to=2]
	\arrow[shorten <=6pt, shorten >=6pt, from=1, to=2]
\end{tikzcd}\]
\end{example}


\p We define the \snotionsym{Gray cone}{((d40@$\uvar\star 1$}{for $\zo$-categories} and the \snotion{Gray $\circ$-cone}{for $\zo$-categories}\index[notation]{((d50@$1\overset{co}{\star}\_$!\textit{for $\zo$-categories}}:
$$\begin{array}{ccccccc}
\zocat &\to&\zocat_{\cdot}&&\zocat &\to&\zocat_{\cdot}\\
C&\mapsto &C\star 1 & &C &\mapsto &1\costar C
\end{array}
$$
where $C\star 1$ and $1\costar C$ are defined as the following pushout: 
% q.uiver.app/#q=WzAsOCxbMSwwLCJDXFxvdGltZXMgWzFdIl0sWzAsMCwiQ1xcb3RpbWVzXFx7MVxcfSJdLFswLDEsIjEiXSxbMSwxLCJDXFxzdGFyIDEiXSxbMiwwLCJDXFxvdGltZXNcXHswXFx9Il0sWzMsMCwiQ1xcb3RpbWVzIFsxXSJdLFszLDEsIjFcXGNvc3RhciBDIl0sWzIsMSwiMSJdLFsxLDJdLFsxLDBdLFsyLDNdLFswLDNdLFszLDEsIiIsMSx7InN0eWxlIjp7Im5hbWUiOiJjb3JuZXIifX1dLFs0LDddLFs3LDZdLFs0LDVdLFs1LDZdLFs2LDQsIiIsMSx7InN0eWxlIjp7Im5hbWUiOiJjb3JuZXIifX1dXQ==
\[\begin{tikzcd}
	{C\otimes\{1\}} & {C\otimes [1]} & {C\otimes\{0\}} & {C\otimes [1]} \\
	1 & {C\star 1} & 1 & {1\costar C}
	\arrow[from=1-1, to=2-1]
	\arrow[from=1-1, to=1-2]
	\arrow[from=2-1, to=2-2]
	\arrow[from=1-2, to=2-2]
	\arrow["\lrcorner"{anchor=center, pos=0.125, rotate=180}, draw=none, from=2-2, to=1-1]
	\arrow[from=1-3, to=2-3]
	\arrow[from=2-3, to=2-4]
	\arrow[from=1-3, to=1-4]
	\arrow[from=1-4, to=2-4]
	\arrow["\lrcorner"{anchor=center, pos=0.125, rotate=180}, draw=none, from=2-4, to=1-3]
\end{tikzcd}\]


\begin{example}
The $\zo$-categories $\Db_1\star 1$ and $1\costar \Db_1$ correspond respectively to the polygraphs: 
% https://q.uiver.app/#q=WzAsNixbMCwwLCIwIl0sWzAsMSwiMSJdLFsxLDEsIlxcc3RhciJdLFszLDEsIlxcc3RhciJdLFs0LDAsIjAiXSxbNCwxLCIxIl0sWzAsMV0sWzEsMl0sWzAsMl0sWzQsNV0sWzMsNF0sWzMsNV0sWzgsMSwiIiwwLHsic2hvcnRlbiI6eyJzb3VyY2UiOjIwfX1dLFs5LDExLCIiLDAseyJvZmZzZXQiOjIsInNob3J0ZW4iOnsic291cmNlIjoyMCwidGFyZ2V0IjoyMH19XV0=
\[\begin{tikzcd}
	0 &&&& 0 \\
	1 & \star && \star & 1
	\arrow[from=1-1, to=2-1]
	\arrow[from=2-1, to=2-2]
	\arrow[""{name=0, anchor=center, inner sep=0}, from=1-1, to=2-2]
	\arrow[""{name=1, anchor=center, inner sep=0}, from=1-5, to=2-5]
	\arrow[from=2-4, to=1-5]
	\arrow[""{name=2, anchor=center, inner sep=0}, from=2-4, to=2-5]
	\arrow[shorten <=2pt, Rightarrow, from=0, to=2-1]
	\arrow[shift right=2, shorten <=4pt, shorten >=4pt, Rightarrow, from=1, to=2]
\end{tikzcd}\]
The $\zo$-categories $\Db_2\star 1$ and $1\costar \Db_2$ correspond respectively to the polygraphs: 
% https://q.uiver.app/#q=WzAsMTQsWzAsMCwiMCJdLFswLDEsIjEiXSxbMSwxLCJcXHN0YXIiXSxbMiwwLCIwIl0sWzMsMSwiXFxzdGFyIl0sWzIsMSwiMSJdLFsxLDBdLFs1LDAsIjAiXSxbNCwxLCJcXHN0YXIiXSxbNSwxLCIxIl0sWzYsMSwiXFxzdGFyIl0sWzcsMCwiMCJdLFs3LDEsIjEiXSxbNiwwXSxbMCwxXSxbMSwyXSxbMyw1XSxbMCwxLCIiLDIseyJjdXJ2ZSI6NX1dLFs1LDRdLFswLDJdLFs2LDIsIiIsMCx7InN0eWxlIjp7ImJvZHkiOnsibmFtZSI6Im5vbmUifSwiaGVhZCI6eyJuYW1lIjoibm9uZSJ9fX1dLFszLDRdLFs3LDhdLFs3LDldLFs4LDldLFsxMSwxMF0sWzExLDEyXSxbMTIsMTBdLFsxMSwxMiwiIiwxLHsiY3VydmUiOi01fV0sWzEzLDEwLCIiLDIseyJzdHlsZSI6eyJib2R5Ijp7Im5hbWUiOiJub25lIn0sImhlYWQiOnsibmFtZSI6Im5vbmUifX19XSxbMTQsMTcsIiAiLDIseyJzaG9ydGVuIjp7InNvdXJjZSI6MjAsInRhcmdldCI6MjB9fV0sWzE5LDEsIiIsMSx7InNob3J0ZW4iOnsic291cmNlIjoyMCwidGFyZ2V0IjoyMH19XSxbMjAsMTYsIiIsMCx7Im9mZnNldCI6LTEsInNob3J0ZW4iOnsic291cmNlIjoyMCwidGFyZ2V0IjoyMH0sImxldmVsIjoxLCJzdHlsZSI6eyJoZWFkIjp7Im5hbWUiOiJub25lIn19fV0sWzIwLDE2LCIiLDIseyJvZmZzZXQiOjEsInNob3J0ZW4iOnsic291cmNlIjoyMCwidGFyZ2V0IjoyMH0sImxldmVsIjoxLCJzdHlsZSI6eyJoZWFkIjp7Im5hbWUiOiJub25lIn19fV0sWzIwLDE2LCIiLDEseyJzaG9ydGVuIjp7InNvdXJjZSI6MjAsInRhcmdldCI6MjB9LCJsZXZlbCI6MX1dLFsyMSw1LCIiLDAseyJzaG9ydGVuIjp7InNvdXJjZSI6MjB9fV0sWzI4LDI2LCIiLDEseyJzaG9ydGVuIjp7InNvdXJjZSI6MjAsInRhcmdldCI6MjB9fV0sWzI2LDI3LCIiLDEseyJvZmZzZXQiOjIsInNob3J0ZW4iOnsic291cmNlIjoyMCwidGFyZ2V0IjoyMH19XSxbMjMsMjQsIiIsMix7Im9mZnNldCI6Miwic2hvcnRlbiI6eyJzb3VyY2UiOjIwLCJ0YXJnZXQiOjIwfX1dLFsyMywyOSwiIiwyLHsib2Zmc2V0IjoxLCJzaG9ydGVuIjp7InNvdXJjZSI6MjAsInRhcmdldCI6MjB9LCJsZXZlbCI6MSwic3R5bGUiOnsiaGVhZCI6eyJuYW1lIjoibm9uZSJ9fX1dLFsyMywyOSwiIiwwLHsic2hvcnRlbiI6eyJzb3VyY2UiOjIwLCJ0YXJnZXQiOjIwfSwibGV2ZWwiOjF9XSxbMjMsMjksIiIsMix7Im9mZnNldCI6LTEsInNob3J0ZW4iOnsic291cmNlIjoyMCwidGFyZ2V0IjoyMH0sImxldmVsIjoxLCJzdHlsZSI6eyJoZWFkIjp7Im5hbWUiOiJub25lIn19fV1d
\[\begin{tikzcd}
	0 & {~} & 0 &&& 0 & {~} & 0 \\
	1 & \star & 1 & \star & \star & 1 & \star & 1
	\arrow[""{name=0, anchor=center, inner sep=0}, from=1-1, to=2-1]
	\arrow[from=2-1, to=2-2]
	\arrow[""{name=1, anchor=center, inner sep=0}, from=1-3, to=2-3]
	\arrow[""{name=2, anchor=center, inner sep=0}, curve={height=30pt}, from=1-1, to=2-1]
	\arrow[from=2-3, to=2-4]
	\arrow[""{name=3, anchor=center, inner sep=0}, from=1-1, to=2-2]
	\arrow[""{name=4, anchor=center, inner sep=0}, draw=none, from=1-2, to=2-2]
	\arrow[""{name=5, anchor=center, inner sep=0}, from=1-3, to=2-4]
	\arrow[from=1-6, to=2-5]
	\arrow[""{name=6, anchor=center, inner sep=0}, from=1-6, to=2-6]
	\arrow[""{name=7, anchor=center, inner sep=0}, from=2-5, to=2-6]
	\arrow[from=1-8, to=2-7]
	\arrow[""{name=8, anchor=center, inner sep=0}, from=1-8, to=2-8]
	\arrow[""{name=9, anchor=center, inner sep=0}, from=2-8, to=2-7]
	\arrow[""{name=10, anchor=center, inner sep=0}, curve={height=-30pt}, from=1-8, to=2-8]
	\arrow[""{name=11, anchor=center, inner sep=0}, draw=none, from=1-7, to=2-7]
	\arrow["{ }"', shorten <=6pt, shorten >=6pt, Rightarrow, from=0, to=2]
	\arrow[shorten <=2pt, shorten >=2pt, Rightarrow, from=3, to=2-1]
	\arrow[shift left=0.7, shorten <=6pt, shorten >=8pt, no head, from=4, to=1]
	\arrow[shift right=0.7, shorten <=6pt, shorten >=8pt, no head, from=4, to=1]
	\arrow[shorten <=6pt, shorten >=6pt, from=4, to=1]
	\arrow[shorten <=2pt, Rightarrow, from=5, to=2-3]
	\arrow[shorten <=6pt, shorten >=6pt, Rightarrow, from=10, to=8]
	\arrow[shift right=2, shorten <=4pt, shorten >=4pt, Rightarrow, from=8, to=9]
	\arrow[shift right=2, shorten <=4pt, shorten >=4pt, Rightarrow, from=6, to=7]
	\arrow[shift right=0.7, shorten <=6pt, shorten >=8pt, no head, from=6, to=11]
	\arrow[shorten <=6pt, shorten >=6pt, from=6, to=11]
	\arrow[shift left=0.7, shorten <=6pt, shorten >=8pt, no head, from=6, to=11]
\end{tikzcd}\]
\end{example}



\begin{prop}
Let $C$ be an $\zo$-category with an unitary and loop free basis. The canonical comparaisons
$$ (\lambda C)\star 1\to \lambda (C\star 1) ~~~~~~~ 1\costar (\lambda C)\to \lambda (1\costar C)$$
are equivalences.

Let $K$ be an augmented directed complex with a loop free and unitary basis. The canonical comparaisons
$$ (\nu K)\star 1\to \nu (K\star 1) ~~~~~~~ 1\costar (\nu K)\to \nu (1\costar K)$$
are equivalences. 
\end{prop}
\begin{proof}
The first assertion directly follows from the fact $\lambda$ commutes with colimits. For the second one,
we can easily check that all the morphisms appearing in the squares \eqref{eq:defin of cstar costar CDA} are quasi-rigid.
The results then follow from an application of theorem \ref{theo:Kan condition}.
\end{proof}



\p We now give some technical results that we will use later.
\begin{lemma}
\label{lemma:non trivial automorphisme 4}
Let $S$ be the smallest set of $\zo$-categories such that
\begin{enumerate}
\item $S$ is stable by isomorphisms,
\item the terminal $\zo$-category belong to $S$,
\item $S$ is stable by $\uvar\star 1$, $1\costar \uvar$, $[\uvar,1]$, $[\uvar,1]\vee[1]$ and $[1]\vee[\uvar,1]$.
\end{enumerate}
Then, the $\zo$-categories belonging to $S$ have non non-trivial automorphisms.
\end{lemma}
\begin{proof}
The set of $\zo$-categories admitting an atomic and loop free basis fulfills the three condition. As a consequence, every $\zo$-category in $S$ has an atomic and loop free basis. Using theorem \ref{theorem:steiner}, it is then sufficient to show that any augmented directed complex in $\lambda(S)$ has no non-trivial automorphisms. The result then follows from propositions \ref{prop:non trivial automorphisme 1}, \ref{prop:non trivial automorphisme 2} and \ref{prop:non trivial automorphisme 3}.
\end{proof}


\begin{prop}
\label{prop:the globes a non non trivial automorphisms}
Let $n$ be an integer $n$. The $\zo$-categories $\Db_n$ and $\underbrace{1\star 1\star ... \star 1}_{n}$ have no non-trivial automorphisms.
\end{prop}
\begin{proof}
This is a direct consequence of lemma \ref{lemma:non trivial automorphisme 4} as these two $\zo$-categories belong to $S$.
\end{proof}

\p The following propositions express the link between the Gray operations and the suspension. They will play a fundamental role in the rest of this work.
\begin{theorem}
 \label{theo:appendice formula for otimes} 
 Let $C$ be an $\zo$-category.
There is a natural identification between $[ C,1]\otimes [1]$ and the colimit of the following diagram
$$
\begin{tikzcd}
	{[1]\vee [ C,1]} & {[C\otimes\{0\},1]} & {[C\otimes [1],1]} & {[C\otimes\{1\},1]} & {[C,1]\vee[1]}
	\arrow[from=1-2, to=1-1]
	\arrow[from=1-2, to=1-3]
	\arrow[from=1-4, to=1-3]
	\arrow[from=1-4, to=1-5]
\end{tikzcd}$$
\end{theorem}
\begin{proof}
As all these functors preserve colimits, it is sufficient to construct the comparison when $C$ is a globular sum, and to show that it is an equivalence when $C$ is a globe. 
As globular sums have atomic and loop free bases, the comparison is induced by proposition \ref{prop:appendice formula for otimes cda}. Using the explicit description of the $\zo$-category $\Db_n\otimes[1]$ given in paragraph \ref{para:explicit Dbn otiomes [1]}, it is straightforward to see that it induces an equivalence on globes.
\end{proof}

The definitional cocartesian squares
% q.uiver.app/#q=WzAsOCxbMSwwLCJDXFxvdGltZXMgWzFdIl0sWzAsMCwiQ1xcb3RpbWVzXFx7MVxcfSJdLFswLDEsIjEiXSxbMSwxLCJDXFxzdGFyIDEiXSxbMiwwLCJDXFxvdGltZXNcXHswXFx9Il0sWzMsMCwiQ1xcb3RpbWVzIFsxXSJdLFszLDEsIjFcXGNvc3RhciBDIl0sWzIsMSwiMSJdLFsxLDJdLFsxLDBdLFsyLDNdLFswLDNdLFszLDEsIiIsMSx7InN0eWxlIjp7Im5hbWUiOiJjb3JuZXIifX1dLFs0LDddLFs3LDZdLFs0LDVdLFs1LDZdLFs2LDQsIiIsMSx7InN0eWxlIjp7Im5hbWUiOiJjb3JuZXIifX1dXQ==
\[\begin{tikzcd}
	{C\otimes\{1\}} & {C\otimes [1]} & {C\otimes\{0\}} & {C\otimes [1]} \\
	1 & {C\star 1} & 1 & {1\costar C}
	\arrow[from=1-1, to=2-1]
	\arrow[from=1-1, to=1-2]
	\arrow[from=2-1, to=2-2]
	\arrow[from=1-2, to=2-2]
	\arrow["\lrcorner"{anchor=center, pos=0.125, rotate=180}, draw=none, from=2-2, to=1-1]
	\arrow[from=1-3, to=2-3]
	\arrow[from=2-3, to=2-4]
	\arrow[from=1-3, to=1-4]
	\arrow[from=1-4, to=2-4]
	\arrow["\lrcorner"{anchor=center, pos=0.125, rotate=180}, draw=none, from=2-4, to=1-3]
\end{tikzcd}\]
 imply the following proposition:
\begin{theorem}
 \label{theo:appendice formula for star} 
There is a natural identification between $1\costar [C,1]$ and the colimit of the following diagram
% q.uiver.app/#q=WzAsMyxbMiwwLCJbQ1xcc3RhciAxLDFdIl0sWzEsMCwiW0MsMV0iXSxbMCwwLCJbMV1cXHZlZSBbQywxXSJdLFsxLDBdLFsxLDJdXQ==
\[\begin{tikzcd}
	{[1]\vee [C,1]} & {[C,1]} & {[C\star 1,1]}
	\arrow[from=1-2, to=1-3]
	\arrow[from=1-2, to=1-1]
\end{tikzcd}\]
There is a natural identification between $[C,1]\star 1$ and the colimit of the following diagram
% q.uiver.app/#q=WzAsMyxbMiwwLCJbQywxXVxcdmVlWzFdIl0sWzEsMCwiW0MsMV0iXSxbMCwwLCJbMVxcY29zdGFyIEMsMV0iXSxbMSwwXSxbMSwyXV0=
\[\begin{tikzcd}
	{[1\costar C,1]} & {[C,1]} & {[C,1]\vee[1]}
	\arrow[from=1-2, to=1-3]
	\arrow[from=1-2, to=1-1]
\end{tikzcd}\]
\end{theorem}


\begin{prop}
\label{prop:cartesian squares}
Let $C$ be an $\zo$-category with an atomic and loop free basis. The two following canonical squares are cartesian:
% https://q.uiver.app/#q=WzAsOCxbMSwxLCJbQywxXSJdLFsxLDAsIjFcXGNvc3RhciBDIl0sWzAsMSwiXFx7MFxcfSJdLFswLDAsIjEiXSxbMywxLCJbQywxXSJdLFszLDAsIkNcXHN0YXIgMSJdLFsyLDEsIlxcezFcXH0iXSxbMiwwLCIxIl0sWzMsMV0sWzIsMF0sWzMsMl0sWzEsMF0sWzcsNV0sWzYsNF0sWzcsNl0sWzUsNF1d
\[\begin{tikzcd}
	1 & {1\costar C} & 1 & {C\star 1} \\
	{\{0\}} & {[C,1]} & {\{1\}} & {[C,1]}
	\arrow[from=1-1, to=1-2]
	\arrow[from=2-1, to=2-2]
	\arrow[from=1-1, to=2-1]
	\arrow[from=1-2, to=2-2]
	\arrow[from=1-3, to=1-4]
	\arrow[from=2-3, to=2-4]
	\arrow[from=1-3, to=2-3]
	\arrow[from=1-4, to=2-4]
\end{tikzcd}\]
The five squares appearing in the following canonical diagram are both cartesian and cocartesian:
% https://q.uiver.app/#q=WzAsOCxbMSwyLCIxXFxjb3N0YXIgQyJdLFsyLDIsIltDLDFdIl0sWzIsMSwiQ1xcc3RhciAxIl0sWzEsMSwiQ1xcb3RpbWVzWzFdIl0sWzIsMCwiMSJdLFsxLDAsIkNcXG90aW1lc1xcezBcXH0iXSxbMCwxLCJDXFxvdGltZXNcXHsxXFx9Il0sWzAsMiwiMSJdLFsyLDFdLFswLDFdLFszLDBdLFszLDJdLFs1LDRdLFs0LDJdLFs1LDNdLFs2LDNdLFs3LDBdLFs2LDddXQ==
\[\begin{tikzcd}
	& {C\otimes\{0\}} & 1 \\
	{C\otimes\{1\}} & {C\otimes[1]} & {C\star 1} \\
	1 & {1\costar C} & {[C,1]}
	\arrow[from=2-3, to=3-3]
	\arrow[from=3-2, to=3-3]
	\arrow[from=2-2, to=3-2]
	\arrow[from=2-2, to=2-3]
	\arrow[from=1-2, to=1-3]
	\arrow[from=1-3, to=2-3]
	\arrow[from=1-2, to=2-2]
	\arrow[from=2-1, to=2-2]
	\arrow[from=3-1, to=3-2]
	\arrow[from=2-1, to=3-1]
\end{tikzcd}\]
\end{prop}
\begin{proof}
The five squares are cocartesian by construction. 
Since the proofs of the cartesianess of all squares are identical, we will only show the proof for the square
% https://q.uiver.app/#q=WzAsNCxbMCwxLCIxXFxjb3N0YXIgQyJdLFsxLDEsIltDLDFdIl0sWzEsMCwiQ1xcc3RhciAxIl0sWzAsMCwiQ1xcb3RpbWVzWzFdIl0sWzIsMV0sWzAsMV0sWzMsMF0sWzMsMl1d
\[\begin{tikzcd}
	{C\otimes[1]} & {C\star 1} \\
	{1\costar C} & {[C,1]}
	\arrow[from=1-2, to=2-2]
	\arrow[from=2-1, to=2-2]
	\arrow[from=1-1, to=2-1]
	\arrow[from=1-1, to=1-2]
\end{tikzcd}\]
To this extend, remark that for any integer $n$, the  following square is cartesian. 
% q.uiver.app/#q=WzAsNCxbMSwxLCIoQl97W1xcbGFtYmRhIEMsMV19KV9uXFxjdXAgXFx7MFxcfSJdLFswLDEsIihCX3tcXGxhbWJkYSBDXFxzdGFyIDF9KV9uXFxjdXAgXFx7MFxcfSJdLFsxLDAsIihCX3sxXFxjb3N0YXIgXFxsYW1iZGEgQ30pX25cXGN1cCBcXHswXFx9Il0sWzAsMCwiKEJfe1xcbGFtYmRhIENcXG90aW1lc1sxXX0pX25cXGN1cCBcXHswXFx9Il0sWzMsMV0sWzIsMF0sWzMsMl0sWzEsMF1d
\[\begin{tikzcd}
	{(B_{\lambda C\otimes[1]})_n\cup \{0\}} & {(B_{1\costar \lambda C})_n\cup \{0\}} \\
	{(B_{\lambda C\star 1})_n\cup \{0\}} & {(B_{[\lambda C,1]})_n\cup \{0\}}
	\arrow[from=1-1, to=2-1]
	\arrow[from=1-2, to=2-2]
	\arrow[from=1-1, to=1-2]
	\arrow[from=2-1, to=2-2]
\end{tikzcd}\]
This then implies that the following square in the category $\CDA$ is cartesian. 
% q.uiver.app/#q=WzAsNCxbMSwxLCJbXFxsYW1iZGEgQywxXSJdLFswLDEsIlxcbGFtYmRhIENcXHN0YXIgMSJdLFsxLDAsIjFcXGNvc3RhciBcXGxhbWJkYSBDIl0sWzAsMCwiXFxsYW1iZGEgQ1xcb3RpbWVzWzFdIl0sWzMsMV0sWzIsMF0sWzMsMl0sWzEsMF1d
\[\begin{tikzcd}
	{\lambda C\otimes[1]} & {1\costar \lambda C} \\
	{\lambda C\star 1} & {[\lambda C,1]}
	\arrow[from=1-1, to=2-1]
	\arrow[from=1-2, to=2-2]
	\arrow[from=1-1, to=1-2]
	\arrow[from=2-1, to=2-2]
\end{tikzcd}\]
As $\nu$ is a right adjoint, it preserves limits,  and as it commutes with Gray operation, this concludes the proof.
\end{proof}

\begin{lemma}
\label{lemma: pullback and sum}
Let $a$, $b$, $c$ and $d$ be four globular sums.
Suppose given a cartesian square:
% q.uiver.app/#q=WzAsNCxbMCwwLCJhIl0sWzEsMCwiYiJdLFswLDEsImMiXSxbMSwxLCJkIl0sWzAsMl0sWzIsM10sWzAsMV0sWzEsM10sWzAsMywiIiwxLHsic3R5bGUiOnsibmFtZSI6ImNvcm5lciJ9fV1d
\[\begin{tikzcd}
	a & b \\
	c & d
	\arrow[from=1-1, to=2-1]
	\arrow[from=2-1, to=2-2]
	\arrow[from=1-1, to=1-2]
	\arrow[from=1-2, to=2-2]
	\arrow["\lrcorner"{anchor=center, pos=0.125}, draw=none, from=1-1, to=2-2]
\end{tikzcd}\]
where the two horizontal morphisms are globular.
The two following squares are cartesian 
% q.uiver.app/#q=WzAsOCxbMCwwLCJiXFxjb3Byb2RfYWFcXHN0YXIgMSJdLFsxLDAsImJcXHN0YXIgMSJdLFswLDEsIlxcU2lnbWEgYyJdLFsxLDEsIlxcU2lnbWEgZCJdLFsyLDEsIlxcU2lnbWEgYyJdLFszLDEsIlxcU2lnbWEgZCJdLFszLDAsIjFcXGNvc3RhciBiIl0sWzIsMCwiMVxcY29zdGFyIGFcXGNvcHJvZF9hIGIiXSxbMCwyXSxbMiwzXSxbMCwxXSxbMSwzXSxbMCwzLCIiLDEseyJzdHlsZSI6eyJuYW1lIjoiY29ybmVyIn19XSxbNyw0XSxbNiw1XSxbNCw1XSxbNyw2XSxbNyw1LCIiLDEseyJzdHlsZSI6eyJuYW1lIjoiY29ybmVyIn19XV0=
\[\begin{tikzcd}
	{b\coprod_aa\star 1} & {b\star 1} & {1\costar a\coprod_a b} & {1\costar b} \\
	{\Sigma c} & {\Sigma d} & {\Sigma c} & {\Sigma d}
	\arrow[from=1-1, to=2-1]
	\arrow[from=2-1, to=2-2]
	\arrow[from=1-1, to=1-2]
	\arrow[from=1-2, to=2-2]
	\arrow["\lrcorner"{anchor=center, pos=0.125}, draw=none, from=1-1, to=2-2]
	\arrow[from=1-3, to=2-3]
	\arrow[from=1-4, to=2-4]
	\arrow[from=2-3, to=2-4]
	\arrow[from=1-3, to=1-4]
	\arrow["\lrcorner"{anchor=center, pos=0.125}, draw=none, from=1-3, to=2-4]
\end{tikzcd}\]
\end{lemma}
\begin{proof}
We show only the cartesianess of the first square, as the cartesianess of the second one follows by applying the duality $(\uvar)^\circ$. A direct computation shows that for any integer $n$, the following square is cartesian
% q.uiver.app/#q=WzAsNCxbMCwwLCJcXGxhbWJkYSBiXFxjb3Byb2Rfe1xcbGFtYmRhIGF9XFxsYW1iZGEgYVxcc3RhciAxIl0sWzEsMCwiXFxsYW1iZGEgYlxcc3RhciAxIl0sWzAsMSwiXFxTaWdtYVxcbGFtYmRhICBjIl0sWzEsMSwiXFxTaWdtYSBcXGxhbWJkYSBkIl0sWzAsMl0sWzIsM10sWzAsMV0sWzEsM10sWzAsMywiIiwxLHsic3R5bGUiOnsibmFtZSI6ImNvcm5lciJ9fV1d
\[\begin{tikzcd}
	{\lambda b\coprod_{\lambda a}\lambda a\star 1} & {\lambda b\star 1} \\
	{\Sigma\lambda c} & {\Sigma \lambda d}
	\arrow[from=1-1, to=2-1]
	\arrow[from=2-1, to=2-2]
	\arrow[from=1-1, to=1-2]
	\arrow[from=1-2, to=2-2]
	\arrow["\lrcorner"{anchor=center, pos=0.125}, draw=none, from=1-1, to=2-2]
\end{tikzcd}\]
To conclude, one has to show that the canonical morphism
$$ \nu(\lambda b)\coprod_{\nu(\lambda a)}\nu(\lambda a\star 1)\to \nu (\lambda b\coprod_{\lambda a}\lambda a\star 1) $$
is an equivalence. 	
As $a\to b$ is globular, all the morphisms of the following cocartesian square are quasi-rigid. 
% q.uiver.app/#q=WzAsNCxbMCwxLCJcXGxhbWJkYSBhXFxzdGFyIDEgIl0sWzEsMSwiXFxsYW1iZGEgYlxcY29wcm9kX3tcXGxhbWJkYSBifVxcbGFtYmRhIGFcXHN0YXIgMSAiXSxbMSwwLCJcXGxhbWJkYSBiIl0sWzAsMCwiXFxsYW1iZGEgYSJdLFszLDBdLFszLDJdLFsyLDFdLFswLDFdXQ==
\[\begin{tikzcd}
	{\lambda a} & {\lambda b} \\
	{\lambda a\star 1 } & {\lambda b\coprod_{\lambda b}\lambda a\star 1 }
	\arrow[from=1-1, to=2-1]
	\arrow[from=1-1, to=1-2]
	\arrow[from=1-2, to=2-2]
	\arrow[from=2-1, to=2-2]
\end{tikzcd}\]
The results then follow from an application of theorem \ref{theo:Kan condition}.
\end{proof}


\p The end of this section is devoted to proving the following theorem: 
\begin{theorem}
\label{theo:appendince unicity of operation}
Let $F$ be an endofunctor of $\zocat$ such that the induced functor $\zocat\to \zocat_{F(\emptyset)/}$ is colimit preserving and $\psi$ an invertible natural transformation between $\Gb\cup \{\emptyset\}\to \zocat\xrightarrow{F}\zocat$ and $\Gb\cup \{\emptyset\}\to \zocat\xrightarrow{G}\zocat$ where $G$ is either the Gray cylinder, the Gray cone, the Gray $\circ$-cone or an iterated suspension.

Then, the natural transformation $\psi$ can be extended to an invertible natural transformation between $F$ and $G$.
\end{theorem}
The previous theorem implies that the equations given in theorem \ref{theo:appendice formula for otimes} and \ref{theo:appendice formula for star} characterize respectively the Gray cylinder, the Gray cone, and the Gray $\circ$-cone.
We also have the following corollary:
\begin{cor}
\label{cor:crushing of Gray tensor is identitye strict case}
The colimit preserving endofunctor $F:\zocat\to \zocat$, sending $[a,n]$ to the colimit of the span
$$\coprod_{k\leq n}\{k\}\leftarrow \coprod_{k\leq n}a\otimes\{k\}\to a\otimes[n]$$
is equivalent to the identity.
\end{cor}
\begin{proof}
The theorem \ref{theo:appendice formula for otimes} implies that the restriction of $F$ to globes is equivalent to the restriction of the identity to globes. As the identity is the $0$-iterated suspension, we can apply theorem \ref{theo:appendince unicity of operation}.
\end{proof}



\begin{lemma}
\label{lemma:sub categgory of Theta}
A sub category $\Theta'$ of $\Theta$, stable by colimit and containing globular morphisms is equal to $\Theta$ iff
\begin{enumerate}
\item for any integer $n$, $i_n^{-}:\Db_n\to \Db_{n+1}$ belongs to $\Theta'$.
\item For any integer $n$, the unit $\Ib_n:\Db_{n+1}\to \Db_n$ belongs to $\Theta'$.
\item For any pair of integers $k<n$, the composition $\triangledown_{k,n}:\Db_n\to \Db_n\coprod_{k}\Db_n$ belongs to $\Theta'$.
\end{enumerate}
\end{lemma}
\begin{proof}
Suppose that $\Theta'$ fulfills these conditions.
As globular morphisms are compositions of pushouts along morphisms of shape $i_n^{-}$, they belong to $\Theta'$.
 As algebraic morphisms are compositions of colimits of morphism of shape $\triangledown_{k,n}$ or $\Ib_n$, they belong to $\Theta'$.
The result then follows from \cite[proposition 3.3.10]{Ara_thesis} that states that every morphism factors as an algebraic morphism followed by a globular morphism.
\end{proof}


\begin{lemma}
\label{lemma:unit forced}
Let $n$ be an integer, and $G$ be either the Gray cylinder, the Gray cone, the Gray $\circ$-cone or an iterated suspension, and suppose
given a square 
% q.uiver.app/#q=WzAsNCxbMSwxLCJHKFxcRGJfe24rMX0pIl0sWzIsMSwiRyhcXERiX24pIl0sWzAsMCwiRyhcXERiX24pIl0sWzAsMiwiRyhcXERiX24pIl0sWzAsMSwiZiJdLFsyLDAsIkcoaV9uXi0pIl0sWzMsMCwiRyhpX25eKykiLDJdLFsyLDEsImlkIiwwLHsiY3VydmUiOi0zfV0sWzMsMSwiaWQiLDIseyJjdXJ2ZSI6M31dXQ==
\[\begin{tikzcd}
	{G(\Db_n)} \\
	& {G(\Db_{n+1})} & {G(\Db_n)} \\
	{G(\Db_n)}
	\arrow["f", from=2-2, to=2-3]
	\arrow["{G(i_n^-)}", from=1-1, to=2-2]
	\arrow["{G(i_n^+)}"', from=3-1, to=2-2]
	\arrow["id", curve={height=-18pt}, from=1-1, to=2-3]
	\arrow["id"', curve={height=18pt}, from=3-1, to=2-3]
\end{tikzcd}\]
Then, the morphism $f$ is $G(\Ib_n)$.
\end{lemma}
\begin{proof}
As the proof for any possibilities of $G$ are similar, we will show only the case $G:=\uvar\otimes [1]$.
As for any integer $n$, $\Db_n\otimes[1]$ admits a loop free and atomic basis, we can then show the desired assertion after applying the functor $\lambda$.
Remark first that the assumption implies that $\partial f((e_{n+1}\otimes \{\alpha\})=0$, and so $f((e_{n+1}\otimes \{\alpha\}) =0$. We also have $f(e_{n+1}\otimes[1])=0$ as $(\lambda (\Db_n\otimes[1])_{n+2} =0$. This implies that $f$ is equal to $\lambda(G(\Ib_n))$.
\end{proof} 
\begin{lemma}
\label{lemma:comp forced}
Let $k<n$ be two integers, and $G$ be either the Gray cylinder, the Gray cone, the Gray $\circ$-cone or an iterated suspension, and suppose
given a square 
% q.uiver.app/#q=WzAsNixbMSwxLCJHKFxcRGJfbikiXSxbMywxLCIgRyhcXERiX3tufVxcY29wcm9kX2tcXERiX24pIl0sWzAsMCwiRyhcXERiX3tuLTF9KSJdLFswLDIsIkcoXFxEYl97bi0xfSkiXSxbMiwwLCIgRyhcXERiX3tuLTF9XFxjb3Byb2Rfa1xcRGJfe24tMX0pIl0sWzIsMiwiIEcoXFxEYl97bi0xfVxcY29wcm9kX2tcXERiX3tuLTF9KSJdLFswLDEsImYiXSxbMiwwLCJHKGlfbl4tKSIsMV0sWzMsMCwiRyhpX25eKykiLDFdLFs1LDEsIkcoaV9uXispXFxjb3Byb2RfayBHKGlfbl4rKSIsMV0sWzQsMSwiRyhpX25eLSlcXGNvcHJvZF9rIEcoaV9uXi0pIiwxXSxbMyw1LCJcXHRyaWFuZ2xlZG93bl97bi0xLGt9IiwyXSxbMiw0LCJcXHRyaWFuZ2xlZG93bl97bi0xLGt9Il1d
\[\begin{tikzcd}
	{G(\Db_{n-1})} && { G(\Db_{n-1}\coprod_k\Db_{n-1})} \\
	& {G(\Db_n)} && { G(\Db_{n}\coprod_k\Db_n)} \\
	{G(\Db_{n-1})} && { G(\Db_{n-1}\coprod_k\Db_{n-1})}
	\arrow["f", from=2-2, to=2-4]
	\arrow["{G(i_n^-)}"{description}, from=1-1, to=2-2]
	\arrow["{G(i_n^+)}"{description}, from=3-1, to=2-2]
	\arrow["{G(i_n^+)\coprod_k G(i_n^+)}"{description}, from=3-3, to=2-4]
	\arrow["{G(i_n^-)\coprod_k G(i_n^-)}"{description}, from=1-3, to=2-4]
	\arrow["{\triangledown_{n-1,k}}"', from=3-1, to=3-3]
	\arrow["{\triangledown_{n-1,k}}", from=1-1, to=1-3]
\end{tikzcd}\]
where we set $\triangledown_{n,n}:=id$.
Then, the morphism $f$ is $G(\triangledown_{n,k})$.
\end{lemma}
\begin{proof}
As the proof for any possibilities of $G$ are similar, we will show only the case $G:=\uvar\otimes [1]$.
As for any integer $n$, $\Db_n\otimes[1]$ admits a loop free and atomic basis, we can then show the desired assertion after applying the functor $\lambda$. Suppose first that $k<n-1$.
By assumption, we have 
$$
\begin{array}{rcl}
\partial f(e_n\otimes \{\alpha\})&=& \partial (e_n^0\otimes \{\alpha\} +e_n^1\otimes \{\alpha\})\\
\partial f(e_n\otimes [1])&=& \partial (e_n^0\otimes [1]) + \partial (e_n^1\otimes [1]) \\
\end{array}
$$
This forces the equalities
$$
\begin{array}{rcl}
 f(e_n\otimes \{\alpha\})&=& e_n^0\otimes \{\alpha\} +e_n^1\otimes \{\alpha\}\\
 f(e_n\otimes [1])&=& e_n^0\otimes [1] + e_n^1\otimes [1] \\
\end{array}
$$
and $f$ is then equal to $\triangledown_{n,k}\otimes[1]$. The case $k=n-1$ is similar.
\end{proof}

\begin{proof}[Proof of theorem \ref{theo:appendince unicity of operation}]
As every globular sum is a colimit of globes, we can extend $\psi$ to a (\textit{a priori} non natural) transformation, 
$\psi:F_{|\Theta}\to G_{|\Theta}$.
Let $\Theta'$ be the maximal sub category of $\Theta$ such that $\psi_{\Theta'}$ is an equality.
As $G(\Db_n)$ does not have non trivial automorphisms, the assumption implies that	 $\Theta'$ fulfills the first condition of lemma \ref{lemma:sub categgory of Theta}. The lemma \ref{lemma:unit forced} implies that it fulfills the second condition, and an easy induction on $(n-k)$ using lemma \ref{lemma:comp forced} implies that it fulfills the last condition. Applying the lemma \ref{lemma:sub categgory of Theta}, this concludes the proof.
\end{proof}

%\cleardoublepage
%\phantomsection
%\addcontentsline{toc}{part}{Index of symbols} 
%\printindex[notation]
%\clearpage
%\phantomsection
%\addcontentsline{toc}{part}{Index of notions} 
%\printindex[notion]
%
%\cleardoublepage
%\phantomsection
%\addcontentsline{toc}{part}{Bibliography} 
%\bibliography{../../header/biblio}{}
%\bibliographystyle{alpha}
%\end{document}