\usepackage{geometry}
\usepackage{lscape}
\usepackage{amsmath,amsthm}
\usepackage{amsfonts,amssymb}

\usepackage{chngpage}
\usepackage[integrals]{wasysym}
\usepackage[colorlinks=true,linktocpage=true,linkcolor=black,citecolor=black]{hyperref}
\usepackage[capitalise]{cleveref} 
\usepackage{enumitem}
\setenumerate[1]{label=(\arabic*)} 
\setlist[itemize]{noitemsep}
\usepackage{footmisc} 
\usepackage{mathtools}
\RequirePackage{tikz-cd}
\RequirePackage{amssymb}
\usetikzlibrary{calc}
\usetikzlibrary{decorations.pathmorphing}
 \sloppy
\setlist[enumerate]{itemsep=0mm}

%\renewcommand{\qedsymbol}{} % remove the square at the end of proof

% *** quiver ***
% A package for drawing commutative diagrams exported from https://q.uiver.app.
%
% This package is currently a wrapper around the `tikz-cd` package, importing necessary TikZ
% libraries, and defining a new TikZ style for curves of a fixed height.
%
% Version: 1.2.3
% Authors:
% - varkor (https://github.com/varkor)
% - AndréC (https://tex.stackexchange.com/users/138900/andr%C3%A9c)

\NeedsTeXFormat{LaTeX2e}
\ProvidesPackage{quiver}[2021/01/11 quiver]

% `tikz-cd` is necessary to draw commutative diagrams.
\RequirePackage{tikz-cd}
% `amssymb` is necessary for `\lrcorner` and `\ulcorner`.
\RequirePackage{amssymb}
% `calc` is necessary to draw curved arrows.
\usetikzlibrary{calc}
% `pathmorphing` is necessary to draw squiggly arrows.
\usetikzlibrary{decorations.pathmorphing}

% A TikZ style for curved arrows of a fixed height, due to AndréC.
\tikzset{curve/.style={settings={#1},to path={(\tikztostart)
    .. controls ($(\tikztostart)!\pv{pos}!(\tikztotarget)!\pv{height}!270:(\tikztotarget)$)
    and ($(\tikztostart)!1-\pv{pos}!(\tikztotarget)!\pv{height}!270:(\tikztotarget)$)
    .. (\tikztotarget)\tikztonodes}},
    settings/.code={\tikzset{quiver/.cd,#1}
        \def\pv##1{\pgfkeysvalueof{/tikz/quiver/##1}}},
    quiver/.cd,pos/.initial=0.35,height/.initial=0}

% TikZ arrowhead/tail styles.
\tikzset{tail reversed/.code={\pgfsetarrowsstart{tikzcd to}}}
\tikzset{2tail/.code={\pgfsetarrowsstart{Implies[reversed]}}}
\tikzset{2tail reversed/.code={\pgfsetarrowsstart{Implies}}}
% TikZ arrow styles.
\tikzset{no body/.style={/tikz/dash pattern=on 0 off 1mm}}

%\usepackage{amsbsy,amscd,latexsym,amstext,delarray}

\usepackage{tikz,tikz-cd} %Grapĥique
\usetikzlibrary{calc}%\usetikzlibrary{3d} % allows [canvas is yz plane at x=0] as an option for scope.
\tikzset{curve/.style={settings={#1},to path={(\tikztostart)
    .. controls ($(\tikztostart)!\pv{pos}!(\tikztotarget)!\pv{height}!270:(\tikztotarget)$)
    and ($(\tikztostart)!1-\pv{pos}!(\tikztotarget)!\pv{height}!270:(\tikztotarget)$)
    .. (\tikztotarget)\tikztonodes}},
    settings/.code={\tikzset{quiver/.cd,#1}
        \def\pv##1{\pgfkeysvalueof{/tikz/quiver/##1}}},
    quiver/.cd,pos/.initial=0.35,height/.initial=0}
    	

\setcounter{tocdepth}{2}

\counterwithin{equation}{subsection}
\newtheorem{lemma}[equation]{Lemma}
\newtheorem{theorem}[equation]{Theorem}
\newtheorem{prop}[equation]{Proposition}
\newtheorem*{prop*}{Proposition}
\newtheorem{cor}[equation]{Corollary}
\newtheorem{conj}[equation]{Conjecture}

\theoremstyle{definition}
\newtheorem{definition}[equation]{Definition}

%\theoremstyle{remark}
\newtheorem{remark}[equation]{Remark}
\newtheorem{draw}[equation]{Draw}
\newtheorem{example}[equation]{Example}
\newtheorem*{example*}{Example}
\newtheorem{warning}[equation]{Warning}
\newtheorem{assumption}[equation]{Assumption}
\newtheorem{construction}[equation]{Construction}
\newtheorem{notation}[equation]{Notation}
\newtheorem*{notation*}{Notation}
\newtheorem{hyprec}[equation]{Hypothèse de récurrence}
\newtheorem{rec}[equation]{Récurrence}
\newtheorem{ini}[equation]{Initialisation}
\usepackage[T1]{fontenc}
\newtheorem{npar}[equation]{\unskip}

\newcommand*{\p}{
\refstepcounter{equation}
\paragraph{}
\textbf{\theequation .} 
}

\usepackage{sectsty}
\usepackage{minitoc}
  
\newtheoremstyle{ipropstyle}% name
  {}% space above
  {}% space below
  {\itshape}% body font
  {}% indent amount
  {\bfseries}% theorem head font
  {.}% punctuation after theorem head
  {0.43em}% space after theorem head
  {\thmname{#1}\hspace{0.32em}\thmnote{#3}}% theorem head spec (can be left empty, meaning `normal')

\theoremstyle{ipropstyle}
\newtheorem*{iprop}{Proposition}
\newtheorem*{itheorem}{Theorem}
\newtheorem*{icor}{Corollary}
\newtheorem*{iexample}{Exemple}

