%\documentclass[12pt]{book}
%\usepackage{index}
%\makeindex
%\renewcommand\indexname{Index of notions}
%\newindex{notation}{adx}{and}{Index of symbols}
%\newindex{notion}{bdx}{bnd}{Index of notions}
%\usepackage{tikz}
\usepackage{xcolor,xspace}
\usepackage{url}
\usepackage{epsfig,graphicx,endnotes,kotex,subfigure,multirow,amsmath,algorithm,algpseudocode}
\newcommand\StateX{\Statex\hspace{\algorithmicindent}}%
%\usepackage{breakurl}
%\usepackage[sort,space]{cite}
\usepackage{balance}
%\usepackage{tabularx}
\usepackage{enumitem}
\usepackage{flushend}
\usepackage[T1]{fontenc}
\usepackage{color,soul}
\hyphenation{op-tical net-works semi-conduc-tor}
%\usepackage{filecontents}
%\usepackage{booktabs} % For formal tables
\usepackage{amsthm}
\newtheorem{theorem}{Theorem}
\newtheorem{corollary}{Corollary}
\newtheorem{lemma}{Lemma}
\renewcommand{\qedsymbol}{\rule{0.7em}{0.7em}}

%\newcommand\notion[1]{\textit{#1}\index[notion]{#1}}
\newcommand\wcnotion[2]{\textit{#1}\index[notion]{#2}}
\newcommand\wcnotionsym[3]{\textit{#1}\index[notation]{#2}\index[notion]{#3}}
\newcommand\wcsnotion[3]{\textit{#1}\index[notion]{#2!\textit{#3}}}
\newcommand\snotion[2]{\textit{#1}\index[notion]{#1!\textit{#2}}}
\newcommand\snotionsym[3]{\textit{#1}\index[notion]{#1!\textit{#3}}\index[notation]{#2!\textit{#3}}}
\newcommand\wcsnotionsym[4]{\textit{#1}\index[notation]{#2!\textit{#4}}\index[notion]{#3!\textit{#4}}}

\newcommand\wcnotation[2]{\textit{#1}\index[notation]{#2}}
\newcommand\wcsnotation[3]{\textit{#1}\index[notation]{#2!\textit{#3}}}

\newcommand\sym[1]{\index[notation]{#1}}
\newcommand\ssym[2]{\index[notation]{#1!\textit{#2}}}

\newcommand{\exclam}{!}





\newcommand{\Ab}{\mathbb{A}} 
\newcommand{\Zb}{\mathbb{Z}} 
\newcommand{\Eb}{\mathbb{E}} 
\newcommand{\Nb}{\mathbb{N}}
\newcommand{\Tb}{\mathbf{T}} 
\newcommand{\Yb}{\mathbb{Y}} 
\newcommand{\Ib}{\mathbb{I}} 
\newcommand{\Ob}{\mathbb{O}} 
\newcommand{\Pb}{\mathbb{P}} 
\newcommand{\Qb}{\mathbb{Q}} 
\newcommand{\Sb}{\mathbb{S}} 
\newcommand{\Hb}{\mathbb{H}} 
\newcommand{\Jb}{\mathbf{J}} 
\newcommand{\Kb}{\mathbb{K}} 
\newcommand{\Mb}{\mathbb{M}} 
\newcommand{\Wb}{\mathbf{W}} 
\newcommand{\Xb}{\mathbb{X}} 
\newcommand{\Cb}{\mathbf{C}}
\newcommand{\Vb}{\mathbb{V}}
\newcommand{\Bb}{\mathbb{B}}


\newcommand{\Acal}{\mathcal{A}} 
\newcommand{\Zcal}{\mathcal{Z}} 
\newcommand{\Ecal}{\mathcal{E}} 
\newcommand{\Rcal}{\mathcal{R}} 
\newcommand{\Tcal}{\mathcal{T}} 
\newcommand{\Ycal}{\mathcal{Y}} 
\newcommand{\Ucal}{\mathcal{U}} 
\newcommand{\Ical}{\mathcal{I}} 
\newcommand{\Ocal}{\mathcal{O}} 
\newcommand{\Pcal}{\mathcal{P}} 
\newcommand{\Qcal}{\mathcal{Q}} 
\newcommand{\Scal}{\mathcal{S}} 
\newcommand{\Dcal}{\mathcal{D}} 
\newcommand{\Fcal}{\mathcal{F}} 
\newcommand{\Gcal}{\mathcal{G}} 
\newcommand{\Hcal}{\mathcal{H}} 
\newcommand{\Jcal}{\mathcal{J}} 
\newcommand{\Kcal}{\mathcal{K}} 
\newcommand{\Lcal}{\mathcal{L}} 
\newcommand{\Mcal}{\mathcal{M}} 
\newcommand{\Wcal}{\mathcal{W}} 
\newcommand{\Xcal}{\mathcal{X}} 
\newcommand{\Ccal}{\mathcal{C}} 
\newcommand{\Vcal}{\mathcal{V}} 
\newcommand{\Bcal}{\mathcal{B}} 
\newcommand{\Ncal}{\mathcal{N}} 


\newcommand{\Ago}{\mathfrak{A}} 
\newcommand{\Zgo}{\mathfrak{Z}} 
\newcommand{\Ego}{\mathfrak{E}} 
\newcommand{\Rgo}{\mathfrak{R}} 
\newcommand{\Tgo}{\mathfrak{T}} 
\newcommand{\Ygo}{\mathfrak{Y}} 
\newcommand{\Ugo}{\mathfrak{U}} 
\newcommand{\Igo}{\mathfrak{I}} 
\newcommand{\Ogo}{\mathfrak{O}} 
\newcommand{\Pgo}{\mathfrak{P}} 
\newcommand{\Qgo}{\mathfrak{Q}} 
\newcommand{\Sgo}{\mathfrak{S}} 
\newcommand{\Dgo}{\mathfrak{D}} 
\newcommand{\Fgo}{\mathfrak{F}} 
\newcommand{\Ggo}{\mathfrak{G}} 
\newcommand{\Hgo}{\mathfrak{H}} 
\newcommand{\Jgo}{\mathfrak{J}} 
\newcommand{\Kgo}{\mathfrak{K}} 
\newcommand{\Lgo}{\mathfrak{L}} 
\newcommand{\Mgo}{\mathfrak{M}} 
\newcommand{\Wgo}{\mathfrak{W}} 
\newcommand{\Xgo}{\mathfrak{X}} 
\newcommand{\Cgo}{\mathfrak{C}} 
\newcommand{\Vgo}{\mathfrak{V}} 
\newcommand{\Bgo}{\mathfrak{B}} 
\newcommand{\Ngo}{\mathfrak{N}}



\newcommand{\sslash}{\mathbin{/\mkern-6mu/}}

\newcommand{\note}[1]{{\color{red}#1}}

\def\-{\raisebox{.75pt}{-}}


\newcommand{\uvar}{\_}


%basic notation
\newcommand{\id}{\text{Id}}
\newcommand{\Db}{\mathbf{D}} 
\DeclareMathOperator*{\dom}{dom}
\DeclareMathOperator*{\codom}{codom}
\DeclareMathOperator{\tw}{tw}


%derived notation
\newcommand{\Rb}{\mathbf{R}} 
\newcommand{\Lb}{\mathbf{L}} 
\newcommand{\Fb}{\mathbf{F}} 
\DeclareMathOperator{\Gb}{G} 
  
%ambiguous notation 
\DeclareMathOperator{\N}{N}
\DeclareMathOperator{\T}{T}
\DeclareMathOperator{\J}{J}


%set of maps
\DeclareMathOperator*{\W}{W}
\DeclareMathOperator*{\Wm}{tW}
\DeclareMathOperator*{\Wseg}{W_{Seg}}
\DeclareMathOperator*{\Wsat}{W_{Sat}}

\DeclareMathOperator*{\M}{M}
\DeclareMathOperator*{\Mm}{tM}
\DeclareMathOperator*{\Mseg}{M_{Seg}}
\DeclareMathOperator*{\Msat}{M_{Sat}}

\DeclareMathOperator*{\I}{I}
\DeclareMathOperator*{\F}{F}

%augmented directed complexes
\DeclareMathOperator*{\CDA}{ADC}
\DeclareMathOperator*{\CDAB}{ADC_B}

%categories
\newcommand\omegacat{\omega\mbox{-$\cat$}}
\DeclareMathOperator\Set{Set}
\DeclareMathOperator\Sp{Sp}

%infini groupoids
\DeclareMathOperator*{\Sq}{Sq}
\DeclareMathOperator*{\Li}{Li}
\DeclareMathOperator{\Hom}{Hom}


%infini 1 categories
\DeclareMathOperator*{\Lfib}{LFib}
\DeclareMathOperator*{\Rfib}{RFib}

\DeclareMathOperator*{\LCartoperator}{LCart}
\DeclareMathOperator*{\core}{core}
\newcommand{\LCart}{\mbox{$\LCartoperator$}}

\newcommand{\LCartc}{\mbox{$\LCartoperator$}^c}
\DeclareMathOperator*{\RCart}{RCart}
\DeclareMathOperator*{\RCartc}{RCart^c}




%infini omega categories
\newcommand{\uLCart}{\underline{\LCartoperator}}
\newcommand{\uLCartc}{\underline{\LCartoperator}^c}
\newcommand{\uRCart}{\underline{RCart}}
\newcommand{\uRCartc}{\underline{RCart}^c}

\DeclareMathOperator{\uHom}{\underline{Hom}}
\DeclareMathOperator{\gHom}{\underline{Hom}_{\ominus}}
\DeclareMathOperator{\Map}{Map}
\DeclareMathOperator{\im}{Im}

\newcommand{\uni}{\underline{\omega}}
\newcommand\w[1]{\widehat{#1}}

%functors
\DeclareMathOperator*{\ev}{ev}
\DeclareMathOperator*{\Arr}{Arr}
\newcommand{\Noiun}{\N_{\tiny{(\omega,1)}}}


\newcommand{\colim}{\operatornamewithlimits{colim}}
\newcommand{\laxcolim}{\operatornamewithlimits{laxcolim}}
\newcommand{\laxlim}{\operatornamewithlimits{laxlim}}


%prefixes
\DeclareMathOperator{\Lan}{Lan}
\DeclareMathOperator{\Ran}{Ran}
\newcommand\iun{(\infty,1)}
\newcommand\io{(\infty,\omega)}
\newcommand\ioun{(\infty,\omega,1)}
\newcommand\zoun{(0,\omega,1)}
\newcommand\zo{(0,\omega)}

%leibnitz products
\DeclareMathOperator{\hstar}{\hat{\star}}
\DeclareMathOperator{\htimes}{\hat{\times}}
\DeclareMathOperator{\hotimes}{\hat{\otimes}}


%Gray operations
\DeclareMathOperator{\costarindex}{f}
\newcommand{\costar}{\mathbin{\overset{co}{\star}}}
\newcommand{\fwedge}{\mathbin{\rotatebox[origin=c]{270}{$\gtrdot$}}}


%inclassable
\newcommand{\invamalg}{\mathbin{\rotatebox[origin=c]{180}{$\amalg$}}}
\DeclareMathOperator{\botimes}{\bar{\otimes}}
\DeclareMathOperator\cst{cst}
\DeclareMathOperator\Operatormark{mk}
\newcommand{\mk}{\Operatormark}

%category theory
\DeclareMathOperator\Fun{Fun}
\DeclareMathOperator\Nat{Nat}
\DeclareMathOperator\End{End}



%fundamental notation
\DeclareMathOperator\mcat{cat_m}
\DeclareMathOperator\cat{cat}
\DeclareMathOperator\grd{grd}
\DeclareMathOperator\R{R}

\newcommand\ocat{(\infty,\omega)\mbox{-$\cat$}}
\newcommand\ouncat{(\infty,\omega,1)\mbox{-$\cat$}}
\newcommand\ocatm{{(\infty,\omega)\mbox{-$\mcat$}}}
\newcommand\zocatm{(0,\omega)\mbox{-$\mcat$}}
\newcommand\zocat{(0,\omega)\mbox{-$\cat$}}
\DeclareMathOperator\zocatB{\zocat_B}
\newcommand\icat{(\infty,1)\mbox{-$\cat$}}
\newcommand\qcat{\mbox{Q$\cat$}}
\newcommand\ncat[1]{(\infty, #1)\mbox{-$\cat$}}
\newcommand\zncat[1]{(0, #1)\mbox{-$\cat$}}
\newcommand\igrd{\infty\mbox{-$\grd$}}



\DeclareMathOperator{\OperatorinfiniPsh}{Psh^\infty}
\DeclareMathOperator{\OperatorinfinitPsh}{tPsh^\infty}
\DeclareMathOperator{\OperatorPsh}{Psh}
\DeclareMathOperator{\OperatormPsh}{mPsh}
\DeclareMathOperator{\OperatortPsh}{tPsh}
\newcommand\iPsh[1]{\OperatorinfiniPsh({#1})}
\newcommand\tiPsh[1]{\OperatorinfinitPsh({#1})}
\newcommand\Psh[1]{\OperatorPsh({#1})}
\newcommand\ssetPsh[1]{\OperatorPsh_\Delta({#1})}
\newcommand\tPsh[1]{\OperatortPsh({#1})}
\newcommand\tPshM[1]{{\OperatortPsh}_M({#1})}
\newcommand\mPsh[1]{\OperatormPsh({#1})}
\newcommand\mPshM[1]{{\OperatormPsh}_M({#1})}

%segal stuff
\DeclareMathOperator{\OperatorSeg}{Seg}
\DeclareMathOperator{\OperatortSeg}{tSeg}
\DeclareMathOperator{\OperatormSeg}{mSeg}
\newcommand\Seg{\OperatorSeg}
\newcommand\mSeg{\OperatormSeg}
\newcommand\stratSeg{\OperatortSeg}

%simplicial variations
\DeclareMathOperator{\Sset}{\Psh{\Delta}}
\newcommand{\mSset}{\mPsh{\Delta}}
\newcommand{\stratSset}{\tPsh{\Delta}}


%univers
\DeclareMathOperator{\U}{\mathbf{U}}
\DeclareMathOperator{\V}{\mathbf{V}}
\DeclareMathOperator{\Wcard}{\mathbf{W}}
\DeclareMathOperator{\Z}{\mathbf{Z}}



%Grothendieck constructions
\newcommand{\ringpartial}{\mathring{\partial}}
%
%
%\usepackage[inline]{showlabels}
%
%\usepackage{fancyhdr}
%\usepackage{titlesec}
%\usepackage{textcase}
%
%\pagestyle{fancy}
%
%
%\title{\Huge{Theory and models of  $(\infty,\omega)$-categories}}
%\author{Félix Loubaton}
%\date{}
%\linespread{1.2}	
%\geometry{a4paper,top=3cm,bottom=4cm,left=1.5cm,right=3cm, heightrounded,bindingoffset=5mm}	
%
%
%\fancyhf{}
%\fancyhfoffset[RO,LE]{0.5cm}
%\fancyhfoffset[LE,RO]{0.5cm}
%
%\fancyhead[RO]{\rmfamily\nouppercase{\rightmark}}
%\fancyhead[LE]{\rmfamily\nouppercase{\leftmark}}
%\fancyfoot[C]{\thepage}
%
%\begin{document}

\subsection*{Résumé}

La théorie des $(\infty,1)$-catégories est aujourd'hui un domaine de recherche prolifique avec des applications dans divers domaines. Ces dernières années ont également vu l'essor des $(\infty,n)$-catégories. Par exemple, les travaux de Gaitsgory et Rozenblyum (\cite{Gaitsgory_A_study_on_DAG}) en géométrie algébrique dérivée utilisent les $(\infty,2)$-catégories pour encoder le formalisme des six foncteurs. On peut aussi citer la théorie topologique des champs quantiques, qui utilise la notion de $(\infty,n)$-catégories dans la formalisation et la preuve de l'hypothèse du cobordisme (\cite{Baez_Higher-dimensional_algebra_and_topological_quantum_field_theory}, \cite{lurie_on_the_classification_of_topological_field_theories}, \cite{Grady_the_geometric_cobordism_hypothesis}, \cite{Calaque_a_note_on_the_category_of_cobordism}).

Il convient donc de développer une théorie des $(\infty,n)$-catégories. Cependant, pour réaliser une telle tâche, il est utile de manipuler les catégories $(\infty,k)$ pour $k \geq n$. Par exemple, la construction de Grothendieck, qui est toujours essentielle lorsque l'on travaille avec n'importe quel type de catégories, est une colimite lax dans la $(\infty,n+1)$-catégorie ambiante des $(\infty,n)$-catégories. Un deuxième exemple provient du produit tensoriel de Gray, qui est nécessaire pour encoder la notion de transformation lax, et donc pour définir la notion de colimite et limite lax. Le produit tensoriel de Gray ajoute la dimension des entrées, et les $(\infty,n)$-catégories ne sont donc pas stables sous ce bifoncteur. Une façon d'éviter tous ces problèmes liés à l'augmentation de la dimension est de se concentrer directement sur les $\io$-catégories, ce qui sera le parti pris de ce travail.

Dans la première partie de cette thèse, nous étudions les modèles des $\io$-catégories. Le résultat principal consiste à établir une équivalence de Quillen entre les $\Theta$-espaces complets de Segal et les ensembles compliciaux de Verity. Une des conséquences majeures de ce résultat est que lorsque l'on travaillera dans la $\iun$-catégorie correspondant à ces structures modèles, son lien avec les $\Theta$-espaces complets de Segal de Rezk nous permettra d'utiliser le langage globulaire, tandis que son lien avec les ensembles compliciaux nous donnera accès au produit tensoriel de Gray.


Dans la seconde partie de cette thèse, nous adapterons les constructions de la théorie classique des catégories au cas $\io$. Le chapitre \ref{chapter:the infini 1 categorory of io categories} est consacré à la théorie de base des $\io$-catégories. Le chapitre \ref{chapter:chapter the 1 category of marked categories} introduit la notion de {$\io$-catégories marquées} et étudie les {fibrations cartésiennes}. Le chapitre \ref{chapter:The io-category of small io-categories} est consacré à {la construction de Grothendieck}, à l'{univalence}, au {lemme de Yoneda}, et à d'autres constructions catégoriques standard.

\paragraph{Mots clés.} Ensembles compliciaux, $\io$-catégories, $(\infty,n)$-catégories, fibrations cartésiennes, construction de Grothendieck, univalence, lemme de Yoneda, (co)limites lax.
\newpage
\subsection*{Abstract}
The theory of $(\infty,1)$-categories is today a prolific area of research with applications in a variety of fields. Recent years have also seen the rise of $(\infty,n)$-categories. For example, the work of Gaitsgory and Rozenblyum (\cite{Gaitsgory_A_study_on_DAG}) in derived algebraic geometry uses $(\infty,2)$-categories to encode the formalism of six functors. Another example is topological quantum field theory, which uses the notion of $(\infty,n)$-categories in the formalization and proof of the cobordism hypothesis (\cite{Baez_Higher-dimensional_algebra_and_topological_quantum_field_theory}, \cite{lurie_on_the_classification_of_topological_field_theories}, \cite{Grady_the_geometric_cobordism_hypothesis}, \cite{Calaque_a_note_on_the_category_of_cobordism}).

A theory of $(\infty,n)$-categories therefore needs to be developed. However, to accomplish such a task, it is useful to manipulate $(\infty,k)$-categories for $k \geq n$. For example, the Grothendieck construction, which is always essential when working with any type of category, is a lax colimit in the ambient $(\infty,n+1)$-category of $(\infty,n)$-categories. A second example comes from the Gray tensor product, which is needed to encode the notion of lax transformation, and thus to define the notion of lax colimit and limit. The Gray tensor product adds the dimension of the inputs, and the $(\infty,n)$-categories are not stable under this bifonctor. One way of avoiding all these problems associated with the increasing of the dimension is to focus directly on $\io$-categories.

In the first part of this thesis, we study models of $\io$-categories. The main result is to establish a Quillen equivalence between Segal $\Theta$-complete spaces and Verity complicial sets. A major consequence of this result is that when working in the $\iun$-category corresponding to these model structures, its link with Rezk's $\Theta$-complete Segal spaces will allow us to use the globular language, while its link with complicial sets will give us access to Gray's tensor product.

In the second part of this thesis, we will adapt the constructions of classical category theory to the $\io$ case. Chapter \ref{chapter:the infini 1 categorory of io categories} is devoted to the basic theory of $\io$-categories. The chapter \ref{chapter:chapter the 1 category of marked categories} introduces the notion of {$\io$-marked categories} and studies {cartesian fibrations}. The chapter \ref{chapter:The io-category of small io-categories} is devoted to the {Grothendieck construction}, the {univalence}, the {Yoneda lemma}, and other standard categorical constructions.

\paragraph{Keywords.} Complicial sets, $\io$-categories, $(\infty,n)$-categories, cartesian fibrations,  Grothendieck construction, univalence, lemme de Yoneda, lax (co)limits.

%\end{document}
