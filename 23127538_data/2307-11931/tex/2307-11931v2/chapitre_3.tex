%\documentclass[12pt]{book}
%\usepackage{index}
%\makeindex
%\renewcommand\indexname{Index of notions}
%\newindex{notation}{adx}{and}{Index of symbols}
%\newindex{notion}{bdx}{bnd}{Index of notions}
%\usepackage{tikz}
\usepackage{xcolor,xspace}
\usepackage{url}
\usepackage{epsfig,graphicx,endnotes,kotex,subfigure,multirow,amsmath,algorithm,algpseudocode}
\newcommand\StateX{\Statex\hspace{\algorithmicindent}}%
%\usepackage{breakurl}
%\usepackage[sort,space]{cite}
\usepackage{balance}
%\usepackage{tabularx}
\usepackage{enumitem}
\usepackage{flushend}
\usepackage[T1]{fontenc}
\usepackage{color,soul}
\hyphenation{op-tical net-works semi-conduc-tor}
%\usepackage{filecontents}
%\usepackage{booktabs} % For formal tables
\usepackage{amsthm}
\newtheorem{theorem}{Theorem}
\newtheorem{corollary}{Corollary}
\newtheorem{lemma}{Lemma}
\renewcommand{\qedsymbol}{\rule{0.7em}{0.7em}}

%\newcommand\notion[1]{\textit{#1}\index[notion]{#1}}
\newcommand\wcnotion[2]{\textit{#1}\index[notion]{#2}}
\newcommand\wcnotionsym[3]{\textit{#1}\index[notation]{#2}\index[notion]{#3}}
\newcommand\wcsnotion[3]{\textit{#1}\index[notion]{#2!\textit{#3}}}
\newcommand\snotion[2]{\textit{#1}\index[notion]{#1!\textit{#2}}}
\newcommand\snotionsym[3]{\textit{#1}\index[notion]{#1!\textit{#3}}\index[notation]{#2!\textit{#3}}}
\newcommand\wcsnotionsym[4]{\textit{#1}\index[notation]{#2!\textit{#4}}\index[notion]{#3!\textit{#4}}}

\newcommand\wcnotation[2]{\textit{#1}\index[notation]{#2}}
\newcommand\wcsnotation[3]{\textit{#1}\index[notation]{#2!\textit{#3}}}

\newcommand\sym[1]{\index[notation]{#1}}
\newcommand\ssym[2]{\index[notation]{#1!\textit{#2}}}

\newcommand{\exclam}{!}





\newcommand{\Ab}{\mathbb{A}} 
\newcommand{\Zb}{\mathbb{Z}} 
\newcommand{\Eb}{\mathbb{E}} 
\newcommand{\Nb}{\mathbb{N}}
\newcommand{\Tb}{\mathbf{T}} 
\newcommand{\Yb}{\mathbb{Y}} 
\newcommand{\Ib}{\mathbb{I}} 
\newcommand{\Ob}{\mathbb{O}} 
\newcommand{\Pb}{\mathbb{P}} 
\newcommand{\Qb}{\mathbb{Q}} 
\newcommand{\Sb}{\mathbb{S}} 
\newcommand{\Hb}{\mathbb{H}} 
\newcommand{\Jb}{\mathbf{J}} 
\newcommand{\Kb}{\mathbb{K}} 
\newcommand{\Mb}{\mathbb{M}} 
\newcommand{\Wb}{\mathbf{W}} 
\newcommand{\Xb}{\mathbb{X}} 
\newcommand{\Cb}{\mathbf{C}}
\newcommand{\Vb}{\mathbb{V}}
\newcommand{\Bb}{\mathbb{B}}


\newcommand{\Acal}{\mathcal{A}} 
\newcommand{\Zcal}{\mathcal{Z}} 
\newcommand{\Ecal}{\mathcal{E}} 
\newcommand{\Rcal}{\mathcal{R}} 
\newcommand{\Tcal}{\mathcal{T}} 
\newcommand{\Ycal}{\mathcal{Y}} 
\newcommand{\Ucal}{\mathcal{U}} 
\newcommand{\Ical}{\mathcal{I}} 
\newcommand{\Ocal}{\mathcal{O}} 
\newcommand{\Pcal}{\mathcal{P}} 
\newcommand{\Qcal}{\mathcal{Q}} 
\newcommand{\Scal}{\mathcal{S}} 
\newcommand{\Dcal}{\mathcal{D}} 
\newcommand{\Fcal}{\mathcal{F}} 
\newcommand{\Gcal}{\mathcal{G}} 
\newcommand{\Hcal}{\mathcal{H}} 
\newcommand{\Jcal}{\mathcal{J}} 
\newcommand{\Kcal}{\mathcal{K}} 
\newcommand{\Lcal}{\mathcal{L}} 
\newcommand{\Mcal}{\mathcal{M}} 
\newcommand{\Wcal}{\mathcal{W}} 
\newcommand{\Xcal}{\mathcal{X}} 
\newcommand{\Ccal}{\mathcal{C}} 
\newcommand{\Vcal}{\mathcal{V}} 
\newcommand{\Bcal}{\mathcal{B}} 
\newcommand{\Ncal}{\mathcal{N}} 


\newcommand{\Ago}{\mathfrak{A}} 
\newcommand{\Zgo}{\mathfrak{Z}} 
\newcommand{\Ego}{\mathfrak{E}} 
\newcommand{\Rgo}{\mathfrak{R}} 
\newcommand{\Tgo}{\mathfrak{T}} 
\newcommand{\Ygo}{\mathfrak{Y}} 
\newcommand{\Ugo}{\mathfrak{U}} 
\newcommand{\Igo}{\mathfrak{I}} 
\newcommand{\Ogo}{\mathfrak{O}} 
\newcommand{\Pgo}{\mathfrak{P}} 
\newcommand{\Qgo}{\mathfrak{Q}} 
\newcommand{\Sgo}{\mathfrak{S}} 
\newcommand{\Dgo}{\mathfrak{D}} 
\newcommand{\Fgo}{\mathfrak{F}} 
\newcommand{\Ggo}{\mathfrak{G}} 
\newcommand{\Hgo}{\mathfrak{H}} 
\newcommand{\Jgo}{\mathfrak{J}} 
\newcommand{\Kgo}{\mathfrak{K}} 
\newcommand{\Lgo}{\mathfrak{L}} 
\newcommand{\Mgo}{\mathfrak{M}} 
\newcommand{\Wgo}{\mathfrak{W}} 
\newcommand{\Xgo}{\mathfrak{X}} 
\newcommand{\Cgo}{\mathfrak{C}} 
\newcommand{\Vgo}{\mathfrak{V}} 
\newcommand{\Bgo}{\mathfrak{B}} 
\newcommand{\Ngo}{\mathfrak{N}}



\newcommand{\sslash}{\mathbin{/\mkern-6mu/}}

\newcommand{\note}[1]{{\color{red}#1}}

\def\-{\raisebox{.75pt}{-}}


\newcommand{\uvar}{\_}


%basic notation
\newcommand{\id}{\text{Id}}
\newcommand{\Db}{\mathbf{D}} 
\DeclareMathOperator*{\dom}{dom}
\DeclareMathOperator*{\codom}{codom}
\DeclareMathOperator{\tw}{tw}


%derived notation
\newcommand{\Rb}{\mathbf{R}} 
\newcommand{\Lb}{\mathbf{L}} 
\newcommand{\Fb}{\mathbf{F}} 
\DeclareMathOperator{\Gb}{G} 
  
%ambiguous notation 
\DeclareMathOperator{\N}{N}
\DeclareMathOperator{\T}{T}
\DeclareMathOperator{\J}{J}


%set of maps
\DeclareMathOperator*{\W}{W}
\DeclareMathOperator*{\Wm}{tW}
\DeclareMathOperator*{\Wseg}{W_{Seg}}
\DeclareMathOperator*{\Wsat}{W_{Sat}}

\DeclareMathOperator*{\M}{M}
\DeclareMathOperator*{\Mm}{tM}
\DeclareMathOperator*{\Mseg}{M_{Seg}}
\DeclareMathOperator*{\Msat}{M_{Sat}}

\DeclareMathOperator*{\I}{I}
\DeclareMathOperator*{\F}{F}

%augmented directed complexes
\DeclareMathOperator*{\CDA}{ADC}
\DeclareMathOperator*{\CDAB}{ADC_B}

%categories
\newcommand\omegacat{\omega\mbox{-$\cat$}}
\DeclareMathOperator\Set{Set}
\DeclareMathOperator\Sp{Sp}

%infini groupoids
\DeclareMathOperator*{\Sq}{Sq}
\DeclareMathOperator*{\Li}{Li}
\DeclareMathOperator{\Hom}{Hom}


%infini 1 categories
\DeclareMathOperator*{\Lfib}{LFib}
\DeclareMathOperator*{\Rfib}{RFib}

\DeclareMathOperator*{\LCartoperator}{LCart}
\DeclareMathOperator*{\core}{core}
\newcommand{\LCart}{\mbox{$\LCartoperator$}}

\newcommand{\LCartc}{\mbox{$\LCartoperator$}^c}
\DeclareMathOperator*{\RCart}{RCart}
\DeclareMathOperator*{\RCartc}{RCart^c}




%infini omega categories
\newcommand{\uLCart}{\underline{\LCartoperator}}
\newcommand{\uLCartc}{\underline{\LCartoperator}^c}
\newcommand{\uRCart}{\underline{RCart}}
\newcommand{\uRCartc}{\underline{RCart}^c}

\DeclareMathOperator{\uHom}{\underline{Hom}}
\DeclareMathOperator{\gHom}{\underline{Hom}_{\ominus}}
\DeclareMathOperator{\Map}{Map}
\DeclareMathOperator{\im}{Im}

\newcommand{\uni}{\underline{\omega}}
\newcommand\w[1]{\widehat{#1}}

%functors
\DeclareMathOperator*{\ev}{ev}
\DeclareMathOperator*{\Arr}{Arr}
\newcommand{\Noiun}{\N_{\tiny{(\omega,1)}}}


\newcommand{\colim}{\operatornamewithlimits{colim}}
\newcommand{\laxcolim}{\operatornamewithlimits{laxcolim}}
\newcommand{\laxlim}{\operatornamewithlimits{laxlim}}


%prefixes
\DeclareMathOperator{\Lan}{Lan}
\DeclareMathOperator{\Ran}{Ran}
\newcommand\iun{(\infty,1)}
\newcommand\io{(\infty,\omega)}
\newcommand\ioun{(\infty,\omega,1)}
\newcommand\zoun{(0,\omega,1)}
\newcommand\zo{(0,\omega)}

%leibnitz products
\DeclareMathOperator{\hstar}{\hat{\star}}
\DeclareMathOperator{\htimes}{\hat{\times}}
\DeclareMathOperator{\hotimes}{\hat{\otimes}}


%Gray operations
\DeclareMathOperator{\costarindex}{f}
\newcommand{\costar}{\mathbin{\overset{co}{\star}}}
\newcommand{\fwedge}{\mathbin{\rotatebox[origin=c]{270}{$\gtrdot$}}}


%inclassable
\newcommand{\invamalg}{\mathbin{\rotatebox[origin=c]{180}{$\amalg$}}}
\DeclareMathOperator{\botimes}{\bar{\otimes}}
\DeclareMathOperator\cst{cst}
\DeclareMathOperator\Operatormark{mk}
\newcommand{\mk}{\Operatormark}

%category theory
\DeclareMathOperator\Fun{Fun}
\DeclareMathOperator\Nat{Nat}
\DeclareMathOperator\End{End}



%fundamental notation
\DeclareMathOperator\mcat{cat_m}
\DeclareMathOperator\cat{cat}
\DeclareMathOperator\grd{grd}
\DeclareMathOperator\R{R}

\newcommand\ocat{(\infty,\omega)\mbox{-$\cat$}}
\newcommand\ouncat{(\infty,\omega,1)\mbox{-$\cat$}}
\newcommand\ocatm{{(\infty,\omega)\mbox{-$\mcat$}}}
\newcommand\zocatm{(0,\omega)\mbox{-$\mcat$}}
\newcommand\zocat{(0,\omega)\mbox{-$\cat$}}
\DeclareMathOperator\zocatB{\zocat_B}
\newcommand\icat{(\infty,1)\mbox{-$\cat$}}
\newcommand\qcat{\mbox{Q$\cat$}}
\newcommand\ncat[1]{(\infty, #1)\mbox{-$\cat$}}
\newcommand\zncat[1]{(0, #1)\mbox{-$\cat$}}
\newcommand\igrd{\infty\mbox{-$\grd$}}



\DeclareMathOperator{\OperatorinfiniPsh}{Psh^\infty}
\DeclareMathOperator{\OperatorinfinitPsh}{tPsh^\infty}
\DeclareMathOperator{\OperatorPsh}{Psh}
\DeclareMathOperator{\OperatormPsh}{mPsh}
\DeclareMathOperator{\OperatortPsh}{tPsh}
\newcommand\iPsh[1]{\OperatorinfiniPsh({#1})}
\newcommand\tiPsh[1]{\OperatorinfinitPsh({#1})}
\newcommand\Psh[1]{\OperatorPsh({#1})}
\newcommand\ssetPsh[1]{\OperatorPsh_\Delta({#1})}
\newcommand\tPsh[1]{\OperatortPsh({#1})}
\newcommand\tPshM[1]{{\OperatortPsh}_M({#1})}
\newcommand\mPsh[1]{\OperatormPsh({#1})}
\newcommand\mPshM[1]{{\OperatormPsh}_M({#1})}

%segal stuff
\DeclareMathOperator{\OperatorSeg}{Seg}
\DeclareMathOperator{\OperatortSeg}{tSeg}
\DeclareMathOperator{\OperatormSeg}{mSeg}
\newcommand\Seg{\OperatorSeg}
\newcommand\mSeg{\OperatormSeg}
\newcommand\stratSeg{\OperatortSeg}

%simplicial variations
\DeclareMathOperator{\Sset}{\Psh{\Delta}}
\newcommand{\mSset}{\mPsh{\Delta}}
\newcommand{\stratSset}{\tPsh{\Delta}}


%univers
\DeclareMathOperator{\U}{\mathbf{U}}
\DeclareMathOperator{\V}{\mathbf{V}}
\DeclareMathOperator{\Wcard}{\mathbf{W}}
\DeclareMathOperator{\Z}{\mathbf{Z}}



%Grothendieck constructions
\newcommand{\ringpartial}{\mathring{\partial}}
%
%
%\usepackage[inline]{showlabels}
%
%\usepackage{fancyhdr}
%\usepackage{titlesec}
%\usepackage{textcase}
%
%\pagestyle{fancy}
%
%
%\title{\Huge{Theory and models of $(\infty,\omega)$-categories}}
%\author{Félix Loubaton}
%\date{}
%\linespread{1.2}	
%\geometry{a4paper,top=3cm,bottom=4cm,left=1.5cm,right=3cm, heightrounded,bindingoffset=5mm}	
%
%
%\fancyhf{}
%\fancyhfoffset[RO,LE]{0.5cm}
%\fancyhfoffset[LE,RO]{0.5cm}
%
%\fancyhead[RO]{\rmfamily\nouppercase{\rightmark}}
%\fancyhead[LE]{\rmfamily\nouppercase{\leftmark}}
%\fancyfoot[C]{\thepage}
%
%\begin{document}

\part{On the side of theory}


\chapter{The $(\infty,1)$-category of $\io$-categories}
\label{chapter:the infini 1 categorory of io categories}

\minitoc
\vspace{2cm}
This chapter is dedicated to the basic definition of $\io$-categories. In the first section, we recall some results on factorization systems in presentable $\iun$-categories. In the second section, we define $\io$-categories and give some basic properties. 
We also define and study \textit{discrete Conduché functor}, which are morphisms having the unique right lifting property against 
units $\Ib_{n+1}:\Db_{n+1}\to \Db_n$ for any integer $n$, and against compositions $\triangledown_{k,n}:\Db_n\to \Db_n\coprod_{\Db_k}\Db_n$ for any pair of integers $k\leq n$. This notion was originally defined and studied in the context of strict $\omega$-category by Guetta in \cite{Guetta_conduche}.
\begin{itheorem}[\ref{theo:pullback along conduche preserves colimits}]
Let $f:C\to D$ be a discrete Conduché functor. The pullback functor $f^*:\ocat_{/D}\to \ocat_{/C}$ preserves colimits.
\end{itheorem}


 In the third section, we study Gray operations for $\io$-categories. We conclude this chapter by proving results of strictification. In particular, we demonstrate the following theorem:
\begin{itheorem}[\ref{prop:strict stuff are pushout}]
Let $C$ be an $\io$-category, $b$ a globular sum, and $f:b\to C$ any morphism. The $\io$-categories $$1\costar b\coprod_b C,~C\coprod_b b\otimes[1]~\mbox{and}~C\coprod_b b\star 1$$
are strict whenever $C$ is.
\end{itheorem}
We will also prove the following theorem:
\begin{itheorem}[\ref{theo:strictness}]
If $C$ is strict, so are $C\star 1$, $1\costar C$ and $C\otimes [1]$.
\end{itheorem}
In the process, we will demonstrate another fundamental equation combining $C\otimes[1]$, $1\costar C$, $C\star 1$, and $[C,1]$.
\begin{itheorem}[\ref{theo:formula between pullback of slice and tensor}]
Let $C$ be an $\io$-category. The five squares appearing in the following canonical diagram are both cartesian and cocartesian:
% https://q.uiver.app/#q=WzAsOCxbMSwyLCIxXFxjb3N0YXIgQyJdLFsyLDIsIltDLDFdIl0sWzIsMSwiQ1xcc3RhciAxIl0sWzEsMSwiQ1xcb3RpbWVzWzFdIl0sWzIsMCwiMSJdLFsxLDAsIkNcXG90aW1lc1xcezBcXH0iXSxbMCwxLCJDXFxvdGltZXNcXHsxXFx9Il0sWzAsMiwiMSJdLFsyLDFdLFswLDFdLFszLDBdLFszLDJdLFs1LDRdLFs0LDJdLFs1LDNdLFs2LDNdLFs3LDBdLFs2LDddXQ==
\[\begin{tikzcd}
	& {C\otimes\{0\}} & 1 \\
	{C\otimes\{1\}} & {C\otimes[1]} & {C\star 1} \\
	1 & {1\costar C} & {[C,1]}
	\arrow[from=2-3, to=3-3]
	\arrow[from=3-2, to=3-3]
	\arrow[from=2-2, to=3-2]
	\arrow[from=2-2, to=2-3]
	\arrow[from=1-2, to=1-3]
	\arrow[from=1-3, to=2-3]
	\arrow[from=1-2, to=2-2]
	\arrow[from=2-1, to=2-2]
	\arrow[from=3-1, to=3-2]
	\arrow[from=2-1, to=3-1]
\end{tikzcd}\]
where $[C,1]$ is the \textit{suspension of $C$}.
\end{itheorem}

\paragraph{About the use of the language of $(\infty,1)$-categories.}
In this chapter and the two following, we will freely use the language of $\iun$-categories\footnote{  As there are currently several directions for the formalization of the language of $\iun$-categories (\cite{Riehl_element_of_infini_categories}, \cite{Riehl_A_type_theory_for_synthetic_-categories}, \cite{North_Towards_a_directed_homotopy_type_theory}, \cite{Cisinski-Univalent-Directed-Type-theory}), talking about "the" language of (infinite,1)-categories may be confusing.


In such case, the reader may consider that we are working within the quasi-category $\qcat$ of $\Tb$-small quasi-categories for $\Tb$ a Grothendieck universe. This quasi-category may be obtained either using the coherent nerve as described in \cite[chapter 3]{Lurie_Htt}, or by considering it as the codomain of the universal cocartesian fibration with $\Tb$-small fibers as done in \cite{Cisinski_The_universal_coCartesian_fibration}. In both cases, the straightening/unstraightening correspondence provides a morphism
$$\N(\Sset_{\Tb})\to \qcat$$
that exhibits $\qcat$ as the quasi-categorical localization of $\N(\Sset_{\Tb})$ with respect to the weak equivalences of the Joyal's model structure (\cite[theorem 8.13]{Cisinski_The_universal_coCartesian_fibration}). 

The constructions we use to build new objects - (co)limits of functor between quasi-categories, quasi-categories of functor, localization of quasi-categories, sub maximal Kan complex, full sub quasi-category, adjunction, left and right Kan extension, Yoneda lemma - are well documented in the Joyal model structure (see \cite{Lurie_Htt} or \cite{Cisinski_Higher_categories_and_homotopical_algebra})
, and therefore have direct incarnation in the quasi-category $\qcat$. }.



We allow ourselves the following abuse of language: when a $\infty$-groupoid $X$ is contractible, we will use the expression \textit{the element of $X$} to refer to any element of $X$. For example, we'll talk about \textit{the} composition of two functors, or \textit{the} colimit/limit of a functor. The adjective \textit{unique} should be understood as \textit{the $\infty$-groupoid of choice is contractible}. 


An equivalence $v$ in a $\iun$-category $C$ between an object $a$ and an object $b$ is denoted by $v:a\sim b$.


The maximal sub $\infty$-groupoid of an $\iun$-category $C$ is denoted by $\tau_0(C)$.

Eventually, we will identify (strict) categories with the $\iun$-categories obtained by applying the simplicial nerve.

\paragraph{Cardinality hypothesis.}
We fix during this chapter three Grothendieck universes $\U \in \V\in\Wcard$, such that $\omega\in \U$. 
All defined notions depend on a choice of cardinality. When nothing is specified, this corresponds to the implicit choice of the cardinality $\V$.
With this convention in mind, we denote by {$\Set$} the $\Wcard$-small $1$-category of $\V$-small sets, {$\igrd$} the $\Wcard$-small $\iun$-category of $\V$-small $\infty$-groupoids and {$\icat$} the $\Wcard$-small $\iun$-category of $\V$-small $\iun$-categories. 


\section{Preliminaries}
\subsection{Explicit computation of some colimits}


\p
We have an adjunction:
% q.uiver.app/#q=WzAsMixbMSwwLCJcXFNldDpcXGlvdGEiXSxbMCwwLCJcXHBpXzA6XFxpZ3JkIl0sWzAsMSwiIiwwLHsib2Zmc2V0IjotMn1dLFsxLDAsIiIsMCx7Im9mZnNldCI6LTJ9XSxbMywyLCIiLDAseyJsZXZlbCI6MSwic3R5bGUiOnsibmFtZSI6ImFkanVuY3Rpb24ifX1dXQ==
\begin{equation}
\label{eq:adj betwen set and space}
\begin{tikzcd}
	{\pi_0:\igrd} & {\Set:\iota}
	\arrow[""{name=0, anchor=center, inner sep=0}, shift left=2, from=1-2, to=1-1]
	\arrow[""{name=1, anchor=center, inner sep=0}, shift left=2, from=1-1, to=1-2]
	\arrow["\dashv"{anchor=center, rotate=-90}, draw=none, from=1, to=0]
\end{tikzcd}
\end{equation}
For a category $B$, we denote by {$\Psh{B}$} the category of functors $B^{op}\to \Set$.
For a $\iun$-category $A$, we denote by \wcnotation{$\iPsh{A}$}{(psh@$\iPsh{\uvar}$} the $\iun$-category of functors $A^{op}\to \igrd$. A presheaf on $B$, (resp. a $\infty$-presheaves on $A$) is \textit{$\U$-small} if it is pointwise a $\U$-small set (resp. a $\U$-small $\infty$-groupoid).


\p If $A$ is a $1$-category, the adjunction \eqref{eq:adj betwen set and space} induces an adjunction:
% q.uiver.app/#q=WzAsMixbMSwwLCJcXFBzaHtBfTpcXGlvdGEiXSxbMCwwLCJcXHBpXzA6XFxpUHNoe0F9Il0sWzAsMSwiIiwwLHsib2Zmc2V0IjotMn1dLFsxLDAsIiIsMCx7Im9mZnNldCI6LTJ9XSxbMywyLCIiLDAseyJsZXZlbCI6MSwic3R5bGUiOnsibmFtZSI6ImFkanVuY3Rpb24ifX1dXQ==
\begin{equation}
\label{eq:adj betwen A set and A space}
\begin{tikzcd}
	{\pi_0:\iPsh{A}} & {\Psh{A}:\iota}
	\arrow[""{name=0, anchor=center, inner sep=0}, shift left=2, from=1-2, to=1-1]
	\arrow[""{name=1, anchor=center, inner sep=0}, shift left=2, from=1-1, to=1-2]
	\arrow["\dashv"{anchor=center, rotate=-90}, draw=none, from=1, to=0]
\end{tikzcd}
\end{equation}

\p We recall that the notion of {elegant Reedy category} is defined in paragraph \ref{para:reedy}.
The following lemma provides a powerful way to compute simple colimits in $\iun$-categories by reducing to computations in (stricts) categories. These techniques will be used freely in the rest of this text.


\begin{lemma}
\label{lemma:colimit computed in set presheaves}
Let $A$ be a $\V$-small category. We denote $\iota:\Psh{A}\to \iPsh{A}$ the canonical inclusion.
\begin{enumerate}
\item 
The functor $\iota$ preserves cocartesian square 
% https://q.uiver.app/#q=WzAsNCxbMCwwLCJhIl0sWzAsMSwiYyJdLFsxLDAsImIiXSxbMSwxLCJkIl0sWzAsMV0sWzIsM10sWzAsMl0sWzEsM10sWzMsMCwiIiwxLHsic3R5bGUiOnsibmFtZSI6ImNvcm5lciJ9fV1d
\[\begin{tikzcd}
	a & b \\
	c & d
	\arrow[from=1-1, to=2-1]
	\arrow[from=1-2, to=2-2]
	\arrow[from=1-1, to=1-2]
	\arrow[from=2-1, to=2-2]
	\arrow["\lrcorner"{anchor=center, pos=0.125, rotate=180}, draw=none, from=2-2, to=1-1]
\end{tikzcd}\]
where the left vertical morphism is a monomorphism.
\item 
The functor $\iota$ preserves colimit of finite diagrams of shape: 
% q.uiver.app/#q=WzAsNyxbMSwwLCJcXGJ1bGxldCJdLFswLDEsIlxcYnVsbGV0Il0sWzIsMSwiXFxidWxsZXQiXSxbMywwLCIuLi4iXSxbNCwxLCJcXGJ1bGxldCJdLFs2LDEsIlxcYnVsbGV0Il0sWzUsMCwiXFxidWxsZXQiXSxbMCwxXSxbMCwyLCIiLDIseyJzdHlsZSI6eyJ0YWlsIjp7Im5hbWUiOiJob29rIiwic2lkZSI6InRvcCJ9fX1dLFszLDJdLFszLDQsIiIsMix7InN0eWxlIjp7InRhaWwiOnsibmFtZSI6Imhvb2siLCJzaWRlIjoidG9wIn19fV0sWzYsNF0sWzYsNSwiIiwyLHsic3R5bGUiOnsidGFpbCI6eyJuYW1lIjoiaG9vayIsInNpZGUiOiJ0b3AifX19XV0=
\[\begin{tikzcd}
	& \bullet && {...} && \bullet \\
	\bullet && \bullet && \bullet && \bullet
	\arrow[from=1-2, to=2-1]
	\arrow[hook, from=1-2, to=2-3]
	\arrow[from=1-4, to=2-3]
	\arrow[hook, from=1-4, to=2-5]
	\arrow[from=1-6, to=2-5]
	\arrow[hook, from=1-6, to=2-7]
\end{tikzcd}\]
where morphisms labeled $\hookrightarrow$ are monomorphisms.
\item The functor $\iota$ preserves transfinite composition. 
\item For any $\V$-small elegant Reedy category, and any functor $F:I\to \Psh{A}$ that is Reedy cofibrant, i.e such that for any $i\in I$, $\colim_{\partial i}F\to F(i)$ is a monomorphism,
the canonical comparison 
$$\iota \colim F\to \colim \iota F$$
is an isomorphism. In particular, if $A$ is itself an elegant Reedy category, for any set-valued presheaf $X$ on $A$, there is an equivalence 
$$\iota(X)\sim \colim_{A_{/X}}a.$$ 
\end{enumerate}
\end{lemma}
\begin{proof}
For this result, we use model categories. We consider the interval induces by the constant functor $I:A\to \Psh{\Delta}$ with value $[1]$. We then consider the model structure on $\Psh{A\times \Delta}$ produced by \cite[theorem 1.3.22]{cisinski_prefaisceaux_comme_modele} and induces by the homotopical data $(I\times \uvar,\emptyset)$. This model structure represent $\iPsh{A}$.
To conclude, we then have to show that all the given colimits, seen as (simplicialy constant) presheaves on $\Delta\times A$ are also homotopy colimits of the same diagrams. This then follows from proposition \ref{prop:hom colimit 2}, \ref{prop:hom colimit 3}, \ref{prop:hom colimit 4} and theorem \ref{theo:hom colimi}.
\end{proof}



\subsection{Factorization sytems}
\label{section:Factorization system}
\p For the rest of the section, we fix a \textit{presentable $\iun$-category} $C$, i.e a $\iun$-category $C$ that is a reflexive and $\V$-accessible localization of a $\iun$-category of $\infty$-presheaves on a $\V$-small $\iun$-category.

A full sub $\infty$-groupoid of the $\infty$-groupoid of arrows of $C$ is \wcnotionsym{cocomplete}{(s@$\widehat{S}$}{cocomplete $\infty$-groupoid of arrows} if it is closed under colimit and composition and contains the equivalences. For a $\infty$-groupoid $S$, we define $\widehat{S}$ as the smallest cocomplete full sub $\infty$-groupoid of the $\infty$-groupoid of arrows containing $S$. 

\begin{remark}
A cocomplete full sub $\infty$-groupoid $U$ is closed by pushouts along any morphism. Indeed, suppose given a cocartesian square
% q.uiver.app/#q=WzAsNCxbMCwwLCJhIl0sWzAsMSwiYyJdLFsxLDAsImIiXSxbMSwxLCJkIl0sWzAsMSwiZiIsMl0sWzAsMl0sWzIsMywiZiciXSxbMSwzXSxbMywwLCIiLDEseyJzdHlsZSI6eyJuYW1lIjoiY29ybmVyIn19XV0=
\[\begin{tikzcd}[row sep=scriptsize]
	a & b \\
	c & d
	\arrow["f"', from=1-1, to=2-1]
	\arrow[from=1-1, to=1-2]
	\arrow["{f'}", from=1-2, to=2-2]
	\arrow[from=2-1, to=2-2]
	\arrow["\lrcorner"{anchor=center, pos=0.125, rotate=180}, draw=none, from=2-2, to=1-1]
\end{tikzcd}\]
with $f$ in $U$. Remark that $f'$ is the horizontal colimit of the diagram
% q.uiver.app/#q=WzAsNixbMSwwLCJhIl0sWzIsMCwiYiJdLFswLDAsImEiXSxbMSwxLCJhIl0sWzIsMSwiYiJdLFswLDEsImMiXSxbMyw1XSxbMyw0XSxbMSw0LCJpZCJdLFswLDMsImlkIl0sWzAsMV0sWzAsMiwiaWQiLDJdLFsyLDUsImYiLDJdXQ==
\[\begin{tikzcd}[row sep=scriptsize]
	a & a & b \\
	c & a & b
	\arrow[from=2-2, to=2-1]
	\arrow[from=2-2, to=2-3]
	\arrow["id", from=1-3, to=2-3]
	\arrow["id", from=1-2, to=2-2]
	\arrow[from=1-2, to=1-3]
	\arrow["id"', from=1-2, to=1-1]
	\arrow["f"', from=1-1, to=2-1]
\end{tikzcd}\]
and then is in $U$.
\end{remark}

We say that an $\infty$-groupoid of morphisms $T$ is \wcnotion{closed under left cancellation}{closed under left or right cancellation} (resp. \textit{closed under right cancellation}), if for any pair of composable morphisms $f$ and $g$, if $gf$ and $f$ are in $T$, so is $g$ (resp. if $gf$ and $g$ are in $T$, so is $f$).


\begin{prop}
\label{prop:closed under colimit imply saturated}
Let $U$ be a cocomplete $\infty$-groupoid of arrows of $C$. The $\infty$-groupoid $U$ is closed under left cancellation.
\end{prop}
\begin{proof}
Suppose given $f:a\to b$, $g:b\to c$ such that $gf$ and $f$ are in $S$. As $g$ is the horizontal colimit of the following diagram
% q.uiver.app/#q=WzAsNixbMCwwLCJiIl0sWzAsMSwiYiJdLFsxLDAsImEiXSxbMiwwLCJhIl0sWzEsMSwiYiJdLFsyLDEsImMiXSxbMyw1LCJnIl0sWzIsNCwiZiJdLFs0LDFdLFs0LDVdLFsyLDBdLFsyLDNdLFswLDEsImlkX2IiXV0=
\[\begin{tikzcd}
	b & a & a \\
	b & b & c
	\arrow["g", from=1-3, to=2-3]
	\arrow["f", from=1-2, to=2-2]
	\arrow[from=2-2, to=2-1]
	\arrow[from=2-2, to=2-3]
	\arrow[from=1-2, to=1-1]
	\arrow[from=1-2, to=1-3]
	\arrow["{id_b}", from=1-1, to=2-1]
\end{tikzcd}\]
it is in $U$.
\end{proof}



\p We recall some standard results on factorization systems, which appear in many places in the literature, such as in section 5.5.5 of \cite{Lurie_Htt} for the $\iun$-case and  \cite{Joyal_factorisation} for the strict case.


Let $S$ be a $\V$-small $\infty$-groupoid of maps of $C$. We denote by $\Arr_S(C)$ the full sub $\iun$-category of $\Arr(C)$ whose objects correspond to arrows of $S$.


A \notion{weak factorization system in $(L,R)$} is the data of two full sub $\infty$-groupoids $L$ and $R$ of the $\infty$-groupoid of arrows of $C$, stable under composition and containing equivalences, and of section 
$\Arr_R(C)\to \Arr_L(C)\times_C \Arr_R(C)$ of the functor $ \Arr_L(C)\times_C \Arr_R(C)\to \Arr(C)$ sending two arrows onto their composite.
This is a \wcnotion{factorization system}{factorization system in $(L,R)$} if the functor $\Arr(C)\to \Arr_L(C)\times_C \Arr_R(C)$ is an equivalence. 


Until the end of this section, we suppose given such factorization system in $(L,R)$.




\begin{definition}
Let $i$ and $p$ be two morphisms, and consider a square of shape:
\[\begin{tikzcd}
	a & b \\
	c & d
	\arrow["i"', from=1-1, to=2-1]
	\arrow["p", from=1-2, to=2-2]
	\arrow[from=2-1, to=2-2]
	\arrow[from=1-1, to=1-2]
\end{tikzcd}\]
A \wcnotion{lift}{lift in a square} in such square is the data of a morphism $h:c\to b$ and of two commutative triangles
% q.uiver.app/#q=WzAsNixbMCwwLCJhIl0sWzAsMSwiYyJdLFsxLDAsImIiXSxbMiwxLCJjIl0sWzMsMCwiYiJdLFszLDEsImQiXSxbMCwyXSxbMCwxLCJpIiwyXSxbMSwyLCJoIiwyXSxbMyw0LCJoIl0sWzMsNV0sWzQsNSwicCJdXQ==
\[\begin{tikzcd}
	a & b && b \\
	c && c & d
	\arrow[from=1-1, to=1-2]
	\arrow["i"', from=1-1, to=2-1]
	\arrow["h"', from=2-1, to=1-2]
	\arrow["h", from=2-3, to=1-4]
	\arrow[from=2-3, to=2-4]
	\arrow["p", from=1-4, to=2-4]
\end{tikzcd}\]

Equivalently, we can see a square of the previous shape as a morphism $s:1\to \Sq({i,p}):=\Hom(a,b)\times_{\Hom(a,d)}\Hom(c,d)$\sym{(sq@$\Sq(i,p)$} and a lift as the data of a morphism $h:1\to \Hom(c,d)$ and of a commutative triangle
\[\begin{tikzcd}
	& {\Hom(c,b)} \\
	1 & {\Sq(i,p)}
	\arrow["s"', from=2-1, to=2-2]
	\arrow["h", from=2-1, to=1-2]
	\arrow[from=1-2, to=2-2]
\end{tikzcd}\]

The \textit{$\infty$-groupoid of lift of $s$} is the fibers of $\Hom(c,b)\to \Sq(i,p)$ at $s$.
\end{definition}

\begin{definition}
Let $i$ and $p$ be two morphisms. The morphism \wcnotion{$i$ has the unique left lifting property against $p$}{unique left or right lifting property}, or equivalently, \textit{$p$ has the unique right lifting property against $i$}, if for any square $s\in \Sq(i,p)$, the $\infty$-groupoid of lift of $s$ is contractible. This is equivalent to asking for the morphism $\Hom(c,d)\to \Sq(i,p)$ to be an equivalence.
\end{definition}

\begin{lemma}
\label{lemma:when weak factorization system are factoryzation system}
Suppose that we have a weak factorization system in $(L',R')$ such that morphisms in $R'$ have the unique right lifting property against morphisms of $L'$. The weak factorization system is a factorization system.
\end{lemma}
\begin{proof}
Our goal is to demonstrate that the fibers of $\Arr_{L'}(C)\times_C\Arr_{R'}(C)\to \Arr(C)$ are contractible. Let $f$ be a morphism of $C$. As we have a weak factorization system, there exists an element in the fiber at $f$. Suppose given two elements in this fiber. This corresponds to a square
% q.uiver.app/#q=WzAsNCxbMCwwLCJcXGNkb3QiXSxbMCwxLCJcXGNkb3QiXSxbMSwxLCJcXGNkb3QiXSxbMSwwLCJcXGNkb3QiXSxbMCwxLCJpIiwyXSxbMSwyLCJwIiwyXSxbMCwzLCJpJyJdLFszLDIsInAnIl1d
\[\begin{tikzcd}
	\cdot & \cdot \\
	\cdot & \cdot
	\arrow["i"', from=1-1, to=2-1]
	\arrow["p"', from=2-1, to=2-2]
	\arrow["{i'}", from=1-1, to=1-2]
	\arrow["{p'}", from=1-2, to=2-2]
\end{tikzcd}\]
Morphisms between these two factorizations correspond to lifts in the previous square, which are contractible by assumption, and the fiber is then contractible. 
\end{proof}
We recall that in this section, we suppose that we have a factorization system in $(L,R)$.
\begin{lemma}
\label{lemma:caracterisation of L and R with lifting property 1}
Morphisms in $L$ have the unique left lifting property with respect to morphisms in $R$. 
\end{lemma}
\begin{proof}
Let $i:a\to c$ be a morphim of $L$ and $p:b\to d$ a morphism of $R$.
The factorization functor induces an equivalence between squares $s\in \Sq(i,p)$ and diagrams of shape
% q.uiver.app/#q=WzAsNSxbMCwwLCJhIl0sWzAsMiwiYyJdLFsyLDAsImIiXSxbMiwyLCJkIl0sWzEsMSwiZSJdLFswLDFdLFsyLDNdLFswLDRdLFsxLDRdLFs0LDNdLFs0LDJdXQ==
\[\begin{tikzcd}[row sep=tiny]
	a && b \\
	& e \\
	c && d
	\arrow[from=1-1, to=3-1]
	\arrow[from=1-3, to=3-3]
	\arrow[from=1-1, to=2-2]
	\arrow[from=3-1, to=2-2]
	\arrow[from=2-2, to=3-3]
	\arrow[from=2-2, to=1-3]
\end{tikzcd}\]
where all the morphisms of the left triangle are in $L$ and the ones of the right triangle are in $R$.
Such diagrams are then in equivalence between composite $c\to e\to b$ where the first morphism is in $S$ and the second in $R$. Using once again the factorization functor, we can see that this data is exactly equivalent to a lift in the square $s$.
\end{proof}

We now show the converse of the previous lemma.

\begin{lemma}
\label{lemma:caracterisation of L and R with lifting property 2}
A morphism having the unique left lifting property against morphisms of $R$ is in $L$. Analogously, a morphism having the unique right lifting property against morphisms of $L$ is in $R$.
\end{lemma}
\begin{proof}
Let $f$ be a morphism having the unique left lifting property against morphisms in $R$. We factorize the morphism $f$ in $i\in L$ followed by $p\in R$ and we want to produce an equivalence $f\sim i$. The previous data induces by construction a square
% q.uiver.app/#q=WzAsNCxbMCwwLCJcXGJ1bGxldCJdLFswLDEsIlxcYnVsbGV0Il0sWzEsMCwiXFxidWxsZXQiXSxbMSwxLCJcXGJ1bGxldCJdLFswLDEsImYiLDJdLFswLDIsImkiXSxbMiwzLCJwIl0sWzEsMywiIiwyLHsibGV2ZWwiOjIsInN0eWxlIjp7ImhlYWQiOnsibmFtZSI6Im5vbmUifX19XSxbMSwyLCIiLDEseyJzdHlsZSI6eyJib2R5Ijp7Im5hbWUiOiJkYXNoZWQifX19XV0=
\[\begin{tikzcd}
	a & b \\
	c & d
	\arrow["f"', from=1-1, to=2-1]
	\arrow["i", from=1-1, to=1-2]
	\arrow["p", from=1-2, to=2-2]
	\arrow[Rightarrow, no head, from=2-1, to=2-2]
	\arrow[dashed, from=2-1, to=1-2]
\end{tikzcd}\]
By hypothesis, this square admits a lift $l:c\to b$, that we factorize in a morphism $r'\in L$ followed by a morphism $p'\in R$. The commutativity of the lower triangle implies equivalences $pl'\sim pp'r'\sim id$, and by unicity, $r'\sim id$ and $pp'\sim id$. The lift $l$ is equivalent to $p'$ and is then in $R$. The commutativity of the upper triangle implies $lf\sim lpi \sim i$ and by unicity again, $p'p\sim id$. The morphism $p$ is then an isomorphism, this implies that $f\sim i$, and $f$ is then in $L$. We proceed similarly for the dual assertion.
\end{proof}

\begin{prop}
\label{prop:caracterisation of L and R with lifting property}
A morphism is in $L$ (resp. in $R$) if and only if it has the unique left lifting property against morphisms of $R$ (resp. the unique right lifting property against the morphisms of $R$).
\end{prop}
\begin{proof}
This is the content of lemma \ref{lemma:caracterisation of L and R with lifting property 1} and \ref{lemma:caracterisation of L and R with lifting property 2}.
\end{proof}


\begin{prop}
\label{prop:fonctorialite des relevement}
The forgetful functor from the $\iun$-category of squares with lifts, and whose left (resp. right) vertical morphism is in $L$ (resp. in $R$), to the $\iun$-category of squares whose left (resp. right) vertical morphism is in $L$ (resp. in $R$), is an equivalence.

Roughly speaking, the formation of the lift in squares whose left (resp. right) vertical morphism is in $L$ (resp. in $R$) is functorial.
\end{prop}
\begin{proof}
The $\iun$-category of squares with lifts, and whose left (resp. right) vertical morphism is in $L$ (resp. in $R$), is the $\iun$-category
$$ \mbox{$\Arr_L(C)$}\times_C  \mbox{$\Arr(C)$}\times_C  \mbox{$\Arr_R(C)$}$$
and the  $\iun$-category   whose left (resp. right) vertical morphism is in $L$ (resp. in $R$) of squares is the limit of the diagram
% https://q.uiver.app/#q=WzAsMyxbMCwwLCJcXEFycl9MKEMpXFx0aW1lc19DIFxcQXJyKEMpIl0sWzIsMCwiXFxBcnIoQylcXHRpbWVzX0NcXEFycl9SKEMpIl0sWzEsMCwiXFxBcnIoQykiXSxbMSwyLCJcXHRyaWFuZ2xlZG93biIsMl0sWzAsMiwiXFx0cmlhbmdsZWRvd24iXV0=
\[\begin{tikzcd}
	{\Arr_L(C)\times_C \Arr(C)} & {\Arr(C)} & {\Arr(C)\times_C\Arr_R(C)}
	\arrow["\triangledown"', from=1-3, to=1-2]
	\arrow["\triangledown", from=1-1, to=1-2]
\end{tikzcd}\]
The forgetful functor is induced by the commutative diagram
% https://q.uiver.app/#q=WzAsNCxbMCwxLCJcXEFycl9MKEMpXFx0aW1lc19DIFxcQXJyKEMpIl0sWzIsMCwiXFxBcnIoQylcXHRpbWVzX0NcXEFycl9SKEMpIl0sWzIsMSwiXFxBcnIoQykiXSxbMCwwLCIgXFxBcnJfTChDKVxcdGltZXNfQyAgXFxBcnIoQylcXHRpbWVzX0MgIFxcQXJyX1IoQykiXSxbMSwyLCJcXHRyaWFuZ2xlZG93biJdLFszLDAsIiBcXEFycl9MKEMpXFx0aW1lc19DIFxcdHJpYW5nbGVkb3duIiwyXSxbMywxLCIgXFx0cmlhbmdsZWRvd25cXHRpbWVzX0MgIFxcQXJyX1IoQykiXSxbMCwyLCJcXHRyaWFuZ2xlZG93biIsMl1d
\[\begin{tikzcd}
	{ \Arr_L(C)\times_C  \Arr(C)\times_C  \Arr_R(C)} && {\Arr(C)\times_C\Arr_R(C)} \\
	{\Arr_L(C)\times_C \Arr(C)} && {\Arr(C)}
	\arrow["\triangledown", from=1-3, to=2-3]
	\arrow["{ \Arr_L(C)\times_C \triangledown}"', from=1-1, to=2-1]
	\arrow["{ \triangledown\times_C  \Arr_R(C)}", from=1-1, to=1-3]
	\arrow["\triangledown"', from=2-1, to=2-3]
\end{tikzcd}\]
and we then have to show that it is cartesian. 


By definition of factorization system, the morphism 
$$\triangledown:  \mbox{$\Arr_L(C)$}\times_C  \mbox{$\Arr_R(C)$}\to \Arr(C)$$ is an equivalence. The previous square is then equivalent to the square
% https://q.uiver.app/#q=WzAsNCxbMCwxLCJcXEFycl9MKEMpXFx0aW1lc19DIFxcQXJyX0woQylcXHRpbWVzX0MgIFxcQXJyX1IoQykiXSxbMiwwLCJcXEFycihDKV9MXFx0aW1lc19DXFxBcnJfUihDKVxcdGltZXNfQ1xcQXJyX1IoQykiXSxbMiwxLCJcXEFycl9MKEMpXFx0aW1lc19DICBcXEFycl9SKEMpIl0sWzAsMCwiIFxcQXJyX0woQylcXHRpbWVzX0MgXFxBcnIoQylfTFxcdGltZXNfQ1xcQXJyX1IoQylcXHRpbWVzX0MgIFxcQXJyX1IoQykiXSxbMSwyLCJcXEFycl9MKEMpXFx0aW1lc19DXFx0cmlhbmdsZWRvd24iXSxbMywwLCIgXFxBcnJfTChDKVxcdGltZXNfQyAgXFxBcnJfTChDKVxcdGltZXNfQ1xcdHJpYW5nbGVkb3duIiwyXSxbMywxLCIgXFx0cmlhbmdsZWRvd25cXHRpbWVzX0MgIFxcQXJyX1IoQylcXHRpbWVzX0MgIFxcQXJyX1IoQykiLDAseyJvZmZzZXQiOi0yLCJzdHlsZSI6eyJib2R5Ijp7Im5hbWUiOiJub25lIn0sImhlYWQiOnsibmFtZSI6Im5vbmUifX19XSxbMCwyLCIgXFx0cmlhbmdsZWRvd25cXHRpbWVzX0MgIFxcQXJyX1IoQykiLDJdLFszLDFdXQ==
\[\begin{tikzcd}
	{ \Arr_L(C)\times_C \Arr(C)_L\times_C\Arr_R(C)\times_C  \Arr_R(C)} && {\Arr(C)_L\times_C\Arr_R(C)\times_C\Arr_R(C)} \\
	{\Arr_L(C)\times_C \Arr_L(C)\times_C  \Arr_R(C)} && {\Arr_L(C)\times_C  \Arr_R(C)}
	\arrow["{\Arr_L(C)\times_C\triangledown}", from=1-3, to=2-3]
	\arrow["{ \Arr_L(C)\times_C  \Arr_L(C)\times_C\triangledown}"', from=1-1, to=2-1]
	\arrow["{ \triangledown\times_C  \Arr_R(C)\times_C  \Arr_R(C)}", shift left=2, draw=none, from=1-1, to=1-3]
	\arrow["{ \triangledown\times_C  \Arr_R(C)}"', from=2-1, to=2-3]
	\arrow[from=1-1, to=1-3]
\end{tikzcd}\]
which is obviously cartesian.
\end{proof}





\begin{prop}
\label{prop:cloture of L recap}
The $\infty$-groupoid $L$ is stable under colimit, retract, composition, and left cancellation. The $\infty$-groupoid $R$ is stable under limit, retract, composition, and right cancellation. 
\end{prop}
\begin{proof}
Let $p:b\to d$ be a morphism of $R$ and $\{i_j:a_j\to c_j\}_{j:J}$ a family of morphisms of $L$ indexed by a functor $J\to \Arr_L(C)$, admitting a colimit $\bar{i}:\bar{a}\to \bar{c}$. Both functors $r\mapsto \Sq(r,p)$ and $c\mapsto\Hom(c,b)$ send colimits on limits. This implies that the morphism \[\Hom(\bar{c},b)\to\Sq(\bar{i},p)\] is the limit in $\Arr(\Sp)$ of the family of morphisms 
$$\Hom(c_j,b)\to\Sq({i_j,p}).$$
Each of these morphisms is an equivalence by assumption, so that implies that $\Hom(\bar{c},b)\to\Sq({\bar{i},p})$ is an equivalence. As this is true for any $p$ in $R$, proposition \ref{prop:caracterisation of L and R with lifting property} implies that $\bar{i}$ is in $L$.

Consider now a retract diagram:
% q.uiver.app/#q=WzAsNixbMSwwLCJhJyJdLFsxLDEsImMnIl0sWzAsMCwiYSJdLFswLDEsImMiXSxbMiwwLCJhIl0sWzIsMSwiYyJdLFsyLDMsImkiXSxbNCw1LCJpIl0sWzMsMV0sWzAsMSwiaSciXSxbMiwwXSxbMCw0XSxbMSw1XSxbMyw1LCJpZCIsMix7ImN1cnZlIjoyfV0sWzIsNCwiaWQiLDAseyJjdXJ2ZSI6LTJ9XV0=
\[\begin{tikzcd}
	a & {a'} & a \\
	c & {c'} & c
	\arrow["i", from=1-1, to=2-1]
	\arrow["i", from=1-3, to=2-3]
	\arrow[from=2-1, to=2-2]
	\arrow["{i'}", from=1-2, to=2-2]
	\arrow[from=1-1, to=1-2]
	\arrow[from=1-2, to=1-3]
	\arrow[from=2-2, to=2-3]
	\arrow["id"', curve={height=12pt}, from=2-1, to=2-3]
	\arrow["id", curve={height=-12pt}, from=1-1, to=1-3]
\end{tikzcd}\]
such that $i'$ is in $L$. For any morphism $p:b\to d$ of $R$, this induces a retract diagram
% q.uiver.app/#q=WzAsNixbMSwwLCJcXEhvbShjJyxiKSJdLFsxLDEsIlxcU3EoaScscCkiXSxbMCwwLCJcXEhvbShjLGIpIl0sWzAsMSwiXFxTcShpLHApIl0sWzIsMCwiXFxIb20oYyxiKSJdLFsyLDEsIlxcU3EoaSxwKSJdLFsyLDMsIiAiXSxbNCw1LCIgIl0sWzMsMV0sWzAsMSwiICJdLFsyLDBdLFswLDRdLFsxLDVdLFszLDUsImlkIiwyLHsiY3VydmUiOjJ9XSxbMiw0LCJpZCIsMCx7ImN1cnZlIjotMn1dXQ==
\[\begin{tikzcd}
	{\Hom(c,b)} & {\Hom(c',b)} & {\Hom(c,b)} \\
	{\Sq(i,p)} & {\Sq(i',p)} & {\Sq(i,p)}
	\arrow["{ }", from=1-1, to=2-1]
	\arrow["{ }", from=1-3, to=2-3]
	\arrow[from=2-1, to=2-2]
	\arrow["{ }", from=1-2, to=2-2]
	\arrow[from=1-1, to=1-2]
	\arrow[from=1-2, to=1-3]
	\arrow[from=2-2, to=2-3]
	\arrow["id"', curve={height=12pt}, from=2-1, to=2-3]
	\arrow["id", curve={height=-12pt}, from=1-1, to=1-3]
\end{tikzcd}\]
As equivalences are stable under retract, $\Hom(c,b)\to \Sq(i,p)$ is an equivalence, and as it is true for any $p$ in $R$, $i$ is in $L$.

For the cloture under left cancellation, this is proposition \ref{prop:closed under colimit imply saturated}.

We proceed similarly for the dual assertion.
\end{proof}


\p We fix an $\infty$-groupoid $S$ of arrows of $C$ with $\U$-small domain and codomain. We define \sym{(ls@$L_S$}$L_S := \widehat{S}$, i.e as the smallest full sub $\infty$-groupoid of arrows of $C$ stable under colimits, composition and including $S$, and \wcnotation{$R_S$}{(rs@$R_S$} as the full sub $\infty$-groupoid of arrows of $C$ having the unique right lifting property against morphisms of $S$. 
\begin{construction}[Small object Argument]
\label{cons:small object argument}
Let $f:x\to y$ be an arrow. We define by induction on $\U$ a sequence $\{x_\alpha\}_{\alpha<\U}$ sending $\emptyset$ on $x$.
For a limit ordinal $\alpha<\U$, we set $x_{\alpha}:= \colim_{\alpha'<\alpha}{x_{\alpha'}}$. For a successor ordinal, we define $x_{\alpha+1}$ as the pushout:
% q.uiver.app/#q=WzAsNSxbMSwwLCJ4X1xcYWxwaGEiXSxbMCwwLCJcXGNvbGltX3thXFx0byBiXFxpbiBTfVxcYmlnKFxcY29saW1fe1xcU3EoYVxcdG8gYix4X1xcYWxwaGFcXHRvIHkpfWFcXHVuZGVyc2V0e1xcY29saW1fe1xcSG9tKGIseF9cXGFscGhhKX1hfXtcXGNvcHJvZH0gXFxjb2xpbV97XFxIb20oYix4X1xcYWxwaGEpfWJcXGJpZykiXSxbMSwxLCJ4X3tcXGFscGhhKzF9Il0sWzAsMSwiXFxjb2xpbV97YVxcdG8gYlxcaW4gU31cXGJpZyhcXGNvbGltX3tcXFNxKGFcXHRvIGIseF9cXGFscGhhXFx0byB5KX0gYlxcYmlnKSJdLFsyLDIsInkiXSxbMSwwXSxbMywyXSxbMCwyXSxbMSwzXSxbMyw0LCIiLDEseyJjdXJ2ZSI6Mn1dLFswLDQsIiIsMSx7ImN1cnZlIjotMn1dLFsyLDQsIiIsMSx7InN0eWxlIjp7ImJvZHkiOnsibmFtZSI6ImRhc2hlZCJ9fX1dLFsyLDUsIiIsMSx7ImxldmVsIjoxLCJzdHlsZSI6eyJuYW1lIjoiY29ybmVyIn19XV0=
\[\begin{tikzcd}
	{\colim_{a\to b\in S}\big(\colim_{\Sq(a\to b,x_\alpha\to y)}a\underset{\colim_{\Hom(b,x_\alpha)}a}{\coprod} \colim_{\Hom(b,x_\alpha)}b\big)} & {x_\alpha} \\
	{\colim_{a\to b\in S}\big(\colim_{\Sq(a\to b,x_\alpha\to y)} b\big)} & {x_{\alpha+1}} \\
	&& y
	\arrow[""{name=0, anchor=center, inner sep=0}, from=1-1, to=1-2]
	\arrow[from=2-1, to=2-2]
	\arrow[from=1-2, to=2-2]
	\arrow[from=1-1, to=2-1]
	\arrow[curve={height=12pt}, from=2-1, to=3-3]
	\arrow[curve={height=-12pt}, from=1-2, to=3-3]
	\arrow[dashed, from=2-2, to=3-3]
	\arrow["\lrcorner"{anchor=center, pos=0.125, rotate=180}, draw=none, from=2-2, to=0]
\end{tikzcd}\]
Let $i:x\to\tilde{x}$ be the transfinite composition of this sequence. There is an induced morphism $p:\tilde{x}\to y$, and an equivalence $f\sim pi$. 
\end{construction}

\begin{prop}
\label{prop:factorization system from S}
The previous construction defines a factorization system between $L_S$ and $R_S$. 
\end{prop}
\begin{proof}
Let $f:x\to y $ be any morphism. The previous construction produces functorially morphisms $i:x\to \tilde{x}$ and $p:\tilde{x}\to y $ whose composite is $f$. The morphism $i$ is obviously in $L_S$. We then need to show that $p$ has the unique right lifting property against any morphism of $L_S$. Let $j:a\to b$ be any morphism in $L_S$, $n$ an integer and consider a commutative square
% https://q.uiver.app/#q=WzAsNCxbMCwwLCJhXFxjb3Byb2Rfe1xcY29saW1fe1xcU2Jfbn0gYX0gXFxjb2xpbV97XFxTYl9ufSBiIl0sWzAsMSwiYiJdLFsxLDAsIlxcdGlsZGV7eH0iXSxbMSwxLCJ5Il0sWzAsMSwiaiIsMl0sWzIsMywicCJdLFsxLDNdLFswLDJdXQ==
\[\begin{tikzcd}
	{a\coprod_{\colim_{\Sb_n} a} \colim_{\Sb_n} b} & {\tilde{x}} \\
	b & y
	\arrow["j"', from=1-1, to=2-1]
	\arrow["p", from=1-2, to=2-2]
	\arrow[from=2-1, to=2-2]
	\arrow[from=1-1, to=1-2]
\end{tikzcd}\]
By stability by $\omega$-small colimits, the object $a\coprod_{\colim_{\Sb_n} a} \colim_{\Sb_n} b$ is $\U$-small. There exists then $\alpha<\U$ such that the top morphism factors through $x_\alpha$, and by construction there exists a morphism $l:b\to x_{\alpha+1}$ and a comutative square
% https://q.uiver.app/#q=WzAsNixbMCwwLCJhXFxjb3Byb2Rfe1xcY29saW1fe1xcU2Jfbn0gYX0gXFxjb2xpbV97XFxTYl9ufSBiIl0sWzAsMiwiYiJdLFsyLDAsIlxcdGlsZGV7eH0iXSxbMiwyLCJ5Il0sWzEsMCwieF9cXGFscGhhIl0sWzEsMSwieF97XFxhbHBoYSsxfSJdLFswLDEsImoiLDJdLFsyLDMsInAiXSxbMSwzXSxbMCw0XSxbNCwyXSxbMSw1LCJsIiwwLHsic3R5bGUiOnsiYm9keSI6eyJuYW1lIjoiZG90dGVkIn19fV0sWzQsNV0sWzUsM10sWzUsMl1d
\[\begin{tikzcd}
	{a\coprod_{\colim_{\Sb_n} a} \colim_{\Sb_n} b} & {x_\alpha} & {\tilde{x}} \\
	& {x_{\alpha+1}} \\
	b && y
	\arrow["j"', from=1-1, to=3-1]
	\arrow["p", from=1-3, to=3-3]
	\arrow[from=3-1, to=3-3]
	\arrow[from=1-1, to=1-2]
	\arrow[from=1-2, to=1-3]
	\arrow["l", dotted, from=3-1, to=2-2]
	\arrow[from=1-2, to=2-2]
	\arrow[from=2-2, to=3-3]
	\arrow[from=2-2, to=1-3]
\end{tikzcd}\]
The induced diagonal is a lift in the first square. This implies that $\Hom(b,x)\to \Sq(j,p)$ has the right lifting property against $\Sb_n\to 1$. Eventually, this implies that $\Hom(b,x)\to \Sq(j,p)$ is an equivalence of $\infty$-groupoid, and $p$ then has the unique right lifting property against $i$. We then have a weak factorization system, which is a factorization system according to lemma \ref{lemma:when weak factorization system are factoryzation system}. 
\end{proof}








\subsection{Reflexive localization}

\p An object $x$ is \wcnotion{$S$-local}{local@$S$-local} if for any $i:a\to b\in S$, the induced functor $\Hom(i,x):\Hom(b,x)\to \Hom(a,x)$ is an equivalence. 
We define \wcnotation{$C_{S}$}{(cs@$C_S$} as the full sub $\iun$-category of $C$ composed of $S$-local objects.

\begin{lemma}
\label{lemma:object is $S$ local if fibrant}
An object is $S$-local if and only if $x\to 1$ is in $R_S$.
\end{lemma}
\begin{proof}
Let $i\in S$.
Remark that the functor $\Hom(b,x)\to \Sq(i,x\to 1)\sim\Hom(a,x)$ is $\Hom(i,f)$. The proposition \ref{prop:caracterisation of L and R with lifting property} then implies the desired result.
\end{proof}


\begin{theorem}
\label{theo:adjunction between presheaves and local presheaves}
The inclusion $\iota:C_S\to C$ is part of an adjunction
% q.uiver.app/#q=WzAsMixbMSwwLCJDX1M6XFxpb3RhIl0sWzAsMCwiXFxMYl9TOkMiXSxbMSwwLCIiLDAseyJvZmZzZXQiOi0yfV0sWzAsMSwiIiwwLHsib2Zmc2V0IjotMn1dLFsyLDMsIiIsMCx7ImxldmVsIjoxLCJzdHlsZSI6eyJuYW1lIjoiYWRqdW5jdGlvbiJ9fV1d
\[\begin{tikzcd}
	{\Fb_S:C} & {C_S:\iota}
	\arrow[""{name=0, anchor=center, inner sep=0}, shift left=2, from=1-1, to=1-2]
	\arrow[""{name=1, anchor=center, inner sep=0}, shift left=2, from=1-2, to=1-1]
	\arrow["\dashv"{anchor=center, rotate=-90}, draw=none, from=0, to=1]
\end{tikzcd}\]
Moreover, $\Fb_S:C\to C_S$ is the localization of $C$ by $\widehat{S}$.\sym{(f@$\Fb$}
\end{theorem}
\begin{proof}
For an object $x$, the small object argument provides a factorization of $x\to 1$ into a morphism $x\to \Fb_S x$ of $L_S$ followed by a morphism $\Fb_S x\to 1$ in $R_S$. According to lemma \ref{lemma:object is $S$ local if fibrant}, $\Fb_Sx$ is in $C_S$. As the factorization is functorial, this defines a functor $\Fb_S:C\to C_S$, and a natural transformation $\nu:id\to \Fb_S$ constant on $S$-local objects. As $\Fb_S\iota$ is equivalent to the identity, this induces the claimed adjunction. 

For the second proposition, let $F:C\to D$ be a functor sending morphisms of $L_S$ on equivalences. We define $\Db(F):= F\circ \iota$, and we have a diagram
% q.uiver.app/#q=WzAsMyxbMCwwLCJDIl0sWzIsMCwiRCJdLFsxLDEsIkNfUyJdLFswLDIsIlxcRmJfUyIsMl0sWzAsMSwiRiJdLFsyLDEsIlxcRGIoRikiLDJdLFs0LDIsIiIsMSx7InNob3J0ZW4iOnsic291cmNlIjozMCwidGFyZ2V0IjozMH19XV0=
\[\begin{tikzcd}
	C && D \\
	& {C_S}
	\arrow["{\Fb_S}"', from=1-1, to=2-2]
	\arrow[""{name=0, anchor=center, inner sep=0}, "F", from=1-1, to=1-3]
	\arrow["{\Db(F)}"', from=2-2, to=1-3]
	\arrow[shorten <=5pt, shorten >=5pt, Rightarrow, from=0, to=2-2]
\end{tikzcd}\]
that commutes up to the natural transformation $F\circ_0 \nu:F\to D(F)\circ \Fb_S$. However, the natural transformation $\nu$ is pointwise in $L_S$, which implies that $F\circ \nu$ is pointwise an equivalence, and the previous diagram then commutes. Now, let $G:C_S\to D$ be any other functor such that $G\circ\Fb_S \sim F$. By precomposing with iota, this implies that $G\sim F\circ \iota$.
\end{proof}

\begin{cor}
 \label{cor:derived colimit preserving functor}
The $\iun$-category $C_S$ is cocomplete. Moreover, if $F:C\to D$ is a colimit preserving functor sending $S$ onto equivalences, the induced functor $\Db F:C_S\to D$ preserves colimits.
\end{cor}
\begin{proof}
The first assertion is a direct consequence of the adjunction given in theorem \ref{theo:adjunction between presheaves and local presheaves}.

This adjunction also implies that the colimit of a functor $G:A\to C_S$ is given by $\Fb_S(\colim_{a:A} \iota G(a))$.
As the canonical morphism $\colim_{a:A} \iota G(a)\to \Fb_S(\colim_{a:A} \iota G(a))$ is  by construction in $\widehat{S}$ this proves the second assertion.
\end{proof}






\p Suppose given an adjunction between two $\iun$-categories
% q.uiver.app/#q=WzAsMixbMCwwLCJmOkMiXSxbMSwwLCJEOmciXSxbMCwxLCIiLDAseyJvZmZzZXQiOi0yfV0sWzEsMCwiIiwwLHsib2Zmc2V0IjotMn1dLFsyLDMsIiIsMCx7ImxldmVsIjoxLCJzdHlsZSI6eyJuYW1lIjoiYWRqdW5jdGlvbiJ9fV1d
\[\begin{tikzcd}
	{F:C} & {D:G}
	\arrow[""{name=0, anchor=center, inner sep=0}, shift left=2, from=1-1, to=1-2]
	\arrow[""{name=1, anchor=center, inner sep=0}, shift left=2, from=1-2, to=1-1]
	\arrow["\dashv"{anchor=center, rotate=-90}, draw=none, from=0, to=1]
\end{tikzcd}\]
with unit $\nu$ and counit $\epsilon$,
as well as an $\infty$-groupoid of morphisms $S$ of $C$ and $T$ of $D$ such that $F(S)\subset \widehat{T}$. 
By adjunction property, it implies that for any $T$-local object $d\in D$, $G(d)$ is $S$-local.
The previous adjunction induces a derived adjunction\sym{(lf@$\Lb F$} \sym{(rg@$\Rb G$}
\[\begin{tikzcd}
	{\Lb F:C_S} & {D_T:\Rb G}
	\arrow[""{name=0, anchor=center, inner sep=0}, shift left=2, from=1-1, to=1-2]
	\arrow[""{name=1, anchor=center, inner sep=0}, shift left=2, from=1-2, to=1-1]
	\arrow["\dashv"{anchor=center, rotate=-90}, draw=none, from=0, to=1]
\end{tikzcd}\]
where $\Lb F$ is defined by the formula $c\mapsto \Fb_T F(c)$ and $\Rb G$ is the restriction of $G$ to $D_T$. The unit is given by $\nu\circ \Fb_S$ and the counit by the restriction of $\epsilon$ to $D_T$.

\begin{example}
\label{exe:exe localization}
\index[notation]{(f0@$f_{\mbox{$\exclam$}}:C_{/c}\to C_{d/}$}
\index[notation]{(f1@$f^*:C_{/d}\to C_{c/}$}
\index[notation]{(f2@$f_*:C_{/c}\to C_{d/}$}
\index[notation]{(lf0@$\Lb f_{\mbox{$\exclam$}}:(C_{/c})_{S_{c/}}\to (C_{d/})_{S_{d/}}$}
\index[notation]{(rf1@$\Rb f^*:(C_{/d})_{S_{d/}}\to (C_{c/})_{S_{c/}}$}
\index[notation]{(lf2@$\Lb f^*:(C_{/d})_{S_{d/}}\to (C_{c/})_{S_{c/}}$}
\index[notation]{(rf3@$\Rb f_*:(C_{/c})_{S_{c/}}\to (C_{d/})_{S_{d/}}$}
Let $C$ be a presentable $\iun$-category, $S$ a full sub $\infty$-groupoid of morphisms of $\iPsh{A}$ with $\U$-small codomain and domain. 
Eventually, we set \wcnotation{$S_{/c}$}{(sc@$S_{/c}$} as the $\infty$-groupoid of morphisms of shape
\[\begin{tikzcd}
	& b \\
	a & c
	\arrow[from=2-1, to=2-2]
	\arrow[from=1-2, to=2-2]
	\arrow["s", from=2-1, to=1-2]
\end{tikzcd}\]
where $s:S$.

 A morphism $f:c\to d$ induces an adjunction
\[\begin{tikzcd}
	{f_!:C_{/c}} & {C_{/d}:f^*}
	\arrow[""{name=0, anchor=center, inner sep=0}, shift left=2, from=1-1, to=1-2]
	\arrow[""{name=1, anchor=center, inner sep=0}, shift left=2, from=1-2, to=1-1]
	\arrow["\dashv"{anchor=center, rotate=-90}, draw=none, from=0, to=1]
\end{tikzcd}\] 
where the left adjoint is the composition and the right adjoint is the pullback. By construction, $f_!(S_{/c})\subset S_{/d}$. The previous adjunction can then be derived, and induced an adjunction:
\[\begin{tikzcd}
	{\Lb f_!:(C_{/c})_{S_{/c}}} & {(C_{/d})_{S_{/d}}:\Rb f^*}
	\arrow[""{name=0, anchor=center, inner sep=0}, shift left=2, from=1-1, to=1-2]
	\arrow[""{name=1, anchor=center, inner sep=0}, shift left=2, from=1-2, to=1-1]
	\arrow["\dashv"{anchor=center, rotate=-90}, draw=none, from=0, to=1]
\end{tikzcd}\]
where the right adjoint is just the restriction of $f^*$ to $S_{/d}$-local objects.

If the functor $f^*:C_{/d}\to C_{/c}$ preserves colimits and $f^*(S_{/c})\subset S_{/d}$, the adjunction
% q.uiver.app/#q=WzAsMixbMCwwLCJmXio6Q197L2R9Il0sWzEsMCwiQ197L2N9OmZfKiJdLFswLDEsIiIsMCx7Im9mZnNldCI6LTJ9XSxbMSwwLCIiLDAseyJvZmZzZXQiOi0yfV0sWzIsMywiIiwwLHsibGV2ZWwiOjEsInN0eWxlIjp7Im5hbWUiOiJhZGp1bmN0aW9uIn19XV0=
\[\begin{tikzcd}
	{f^*:C_{/d}} & {C_{/c}:f_*}
	\arrow[""{name=0, anchor=center, inner sep=0}, shift left=2, from=1-1, to=1-2]
	\arrow[""{name=1, anchor=center, inner sep=0}, shift left=2, from=1-2, to=1-1]
	\arrow["\dashv"{anchor=center, rotate=-90}, draw=none, from=0, to=1]
\end{tikzcd}\]
induces an adjunction 
% q.uiver.app/#q=WzAsMixbMCwwLCJcXExiIGZeKjooQ197L2R9KV97U197L2R9fSJdLFsxLDAsIihDX3svY30pX3tTX3svY319OlxcUmIgZl8qIl0sWzAsMSwiIiwwLHsib2Zmc2V0IjotMn1dLFsxLDAsIiIsMCx7Im9mZnNldCI6LTJ9XSxbMiwzLCIiLDAseyJsZXZlbCI6MSwic3R5bGUiOnsibmFtZSI6ImFkanVuY3Rpb24ifX1dXQ==
\[\begin{tikzcd}
	{\Lb f^*:(C_{/d})_{S_{/d}}} & {(C_{/c})_{S_{/c}}:\Rb f_*}
	\arrow[""{name=0, anchor=center, inner sep=0}, shift left=2, from=1-1, to=1-2]
	\arrow[""{name=1, anchor=center, inner sep=0}, shift left=2, from=1-2, to=1-1]
	\arrow["\dashv"{anchor=center, rotate=-90}, draw=none, from=0, to=1]
\end{tikzcd}\]

\end{example}
 


\section{Basic constructions}
\label{chapter:Basica construciton}
\subsection{$\io$-Categories}
\label{section:iocategories}
The definitions of section \ref{subsection:the categoru theta} will be used freely here.
\p
We denote by 
$$[\uvar,\uvar]: \iPsh{\Theta}\times \iPsh{\Delta}\to \iPsh{\Delta[\Theta]}$$
the extension by colimit of the functor $\Theta\times \Delta\to \iPsh{\Delta[\Theta]}$ sending $(a,n)$ onto $[a,n]$.
For an integer $n$, we denote
$$[\uvar,n]:\iPsh{\Theta}^n\to \iPsh{\Theta}$$ 
the extension by colimit of the functor 
$\Theta^n\to\iPsh{\Theta}$ sending $\textbf{a}:=\{a_1,...,a_n\}$ onto $[\textbf{a},n]$.

\p  We have an adjunction 
\begin{equation}
\label{eq:underived adjunction part}
\begin{tikzcd}
	{ i_!:\iPsh{\Delta[\Theta]}} & {\iPsh{\Theta}:i^*}
	\arrow[shift left=2, from=1-1, to=1-2]
	\arrow[shift left=2, from=1-2, to=1-1]
\end{tikzcd}
\end{equation}
where the left adjoint is the left Kan extension of the functor $\Delta[\Theta]\xrightarrow{i} \Theta\to \iPsh{\Theta}$. The sets of morphisms $\W$ and $\M$ are respectively defined in paragraphs \ref{para:definition of W} and \ref{para:defi of delta theta}.
There is an obvious inclusion $i_!(\M)\subset \W$. The previous adjunction then induced a derived adjunction
\begin{equation}
\label{eq:derived adjunction}
\begin{tikzcd}
	{\Lb i_!:\Psh{\Delta[\Theta]}_{\M}} & {\Psh{\Theta}_{\W}:\Rb i^*}
	\arrow[shift left=2, from=1-1, to=1-2]
	\arrow[shift left=2, from=1-2, to=1-1]
\end{tikzcd}
\end{equation}

\begin{prop}
\label{prop:infini changing theta}
The unit and counit of the adjunction \eqref{eq:underived adjunction part} are respectively in $\widehat{\M}$ and $\widehat{\W}$. As a consequence, the adjunction \eqref{eq:derived adjunction} is an adjoint equivalence.
\end{prop}
\begin{proof}
We denote by $\iota:\Psh{\Theta}\to \iPsh{\Theta}$ and $\iota:\Psh{\Delta[\Theta]}\to \iPsh{\Delta[\Theta]}$ the two canonical inclusions. By the definition of the smallest precocomplete class (paragraph \ref{para:precomplet}) and according to lemma \ref{lemma:colimit computed in set presheaves}, we have inclusions $\iota(\overline{\W})\subset \widehat{\W}$ and $\iota(\overline{\M})\subset \widehat{\M}$. The result then directly follows from theorem \ref{theo:unit and counit are in W}.  
\end{proof}





\p A \wcnotion{$\io$-category}{category4@$\io$-category} is a $\W$-local $\infty$-presheaf $C\in \iPsh{\Theta}$. We then define \index[notation]{((a60@$\ocat$}
$$\ocat := \iPsh{\Theta}_{\W}.$$
Proposition \ref{prop:infini changing theta} implies that $\ocat$ identifies itself with the full sub $\iun$-category of $\iPsh{\Delta[\Theta]}$ of $\M$-local objects:
$$\ocat \sim \iPsh{\Delta[\Theta]}_{\M}.$$
We recall that the sets of morphisms $\W$ and $\M$ are respectively defined in paragraphs \ref{para:definition of W} and \ref{para:defi of delta theta}.

\p
We denote by $\pi_0:\iPsh{\Theta}\to \Psh{\Theta}$ the functor sending an $\infty$-presheaf $X$ onto the presheaf
$$\pi_0X:a\mapsto\pi_0(X_a)$$
This functor admits a fully faithful right adjoint: $\N:\Psh{\Theta}\to \iPsh{\Theta}$. 
As $\pi_0$ preserves $\W$, it induces an adjoint pair: \sym{(pi@$\pi_0:\ocat\to \zocat$}\sym{n@$\N:\zocat\to \ocat$}
% https://q.uiver.app/#q=WzAsMixbMCwwLCJcXHBpXzA6XFxvY2F0Il0sWzEsMCwiXFx6b2NhdDpcXE4iXSxbMCwxLCIiLDAseyJvZmZzZXQiOi0yfV0sWzEsMCwiIiwwLHsib2Zmc2V0IjotMn1dLFsyLDMsIiIsMCx7ImxldmVsIjoxLCJzdHlsZSI6eyJuYW1lIjoiYWRqdW5jdGlvbiJ9fV1d
\[\begin{tikzcd}
	{\pi_0:\ocat} & {\zocat:\N}
	\arrow[""{name=0, anchor=center, inner sep=0}, shift left=2, from=1-1, to=1-2]
	\arrow[""{name=1, anchor=center, inner sep=0}, shift left=2, from=1-2, to=1-1]
	\arrow["\dashv"{anchor=center, rotate=-90}, draw=none, from=0, to=1]
\end{tikzcd}\]
where the right adjoint $\N$ is fully faithful.
Every $\zo$-category can then be seen as an $\io$-category and we will call \wcnotion{strict}{strict $\io$-category} the $\io$-categories lying in the image of this functor.

The inclusion $\Delta\to \Theta$ induces by extention by colimit a functor $\iPsh{\Delta}\to \iPsh{\Theta}$. Passing to the localization, this induces a fully faithful inclusion $\ncat{1}\to \ocat$.

The inclusion $\{[0]\}\to \Theta$ induces by extention by colimit a functor $\igrd \to \iPsh{\Theta}$. Passing to the localization, this induces a fully faithful inclusion $\igrd \to \ocat$. The $\io$-categories lying in the image of this functors will be also called \textit{$\infty$-groupoids}. 


\p A \wcsnotion{$n$-cell}{cell@$n$-cell}{for $\io$-categories} of an $\io$-category is a morphism $\Db_n\to C$.
If $C$ is an $\io$-category, we denote by $C_n$ the value of $C$ on $\Db_n$. 

\begin{prop}
\label{prop:equivalences detected on globes}
Let $C,D$ be two $\io$-categories, and $f:C\to D$ any map. The morphism $f$ is an equivalence if and only if for any $n$, the induced morphism $f_n: C_n\to D_n$ is an equivalence. 
\end{prop}
\begin{proof}
This is a necessary condition. For the converse, let $f$ be a morphism fulfilling this condition. To show that $f$ is an equivalence, we have to show that for any globular sum $a$, $f_a: C_a\to D_a$ is an equivalence. This is true as 
$$f_a:C_a\to D_a~\sim~ \lim_{n\in\Sp_a}{f_n:C_n\to D_n}.$$	
\end{proof}

\begin{lemma}
\label{lemma:equivalence if unique right lifting property against globes.}
A functor is an equivalence if it has the unique right lifting property against $\emptyset\to \Db_n$ for any $n\geq 0$. 
\end{lemma}
\begin{proof}
This is a necessary condition. For the converse, let $f:C\to D$ be a morphism fulfilling this condition. By definition of left unique lifting property, it implies that the induced morphism
$f_n:C_n\to D_n$ is an equivalence for any $n\geq 0$. Using proposition \ref{prop:equivalences detected on globes}, $f$ is an equivalence.
 \end{proof}
 
 

\p Let $\iPsh{\Theta}_{\bullet,\bullet}$ be the $(\infty,1)$-category of $\infty$-presheaves on $\Theta$ with two distinguished points, i.e. of triples $(C,a,b)$ where $a$ and $b$ are elements of $C_0$.
The functor $[\uvar,1]:\Theta\to \iPsh{\Theta}_{\bullet,\bullet}$ that sends $a$ onto $([a,1],\{0\},\{1\})$ induces by extension by colimit an adjunction
% https://q.uiver.app/#q=WzAsMixbMSwwLCJcXGlQc2h7XFxUaGV0YX1fe1xcYnVsbGV0LFxcYnVsbGV0fTpcXGhvbV97XFx1dmFyfShcXHV2YXIsXFx1dmFyKSJdLFswLDAsIltcXHV2YXIsMV06XFxpUHNoe1xcVGhldGF9Il0sWzEsMCwiIiwyLHsib2Zmc2V0IjotMn1dLFswLDEsIiIsMix7Im9mZnNldCI6LTJ9XSxbMiwzLCIiLDIseyJsZXZlbCI6MSwic3R5bGUiOnsibmFtZSI6ImFkanVuY3Rpb24ifX1dXQ==
\begin{equation}
\label{eq:suspesnion betweenpresheaves}
\begin{tikzcd}
	{[\uvar,1]:\iPsh{\Theta}} & {\iPsh{\Theta}_{\bullet,\bullet}:\hom_{\uvar}(\uvar,\uvar)}
	\arrow[""{name=0, anchor=center, inner sep=0}, shift left=2, from=1-1, to=1-2]
	\arrow[""{name=1, anchor=center, inner sep=0}, shift left=2, from=1-2, to=1-1]
	\arrow["\dashv"{anchor=center, rotate=-90}, draw=none, from=0, to=1]
\end{tikzcd}
\end{equation}
As the left adjoint preserves representables, the right adjoint commutes with colimit. It is then easy to check on representables that the unit of this adjunction is an equivalence. As a consequence, the left adjoint is fully faithful.


\begin{lemma}
\label{lemma:hom of the suspension prequel}
Let $C$ be an $\infty$-presheaves on $\Theta$. The canonical morphisms
$$C\to \hom_{[C,1]}(0,1)~~\hom_{[C,1]}(0,0)\to 1~~~ \hom_{[C,1]}(1,1) \sim 1~~~~\emptyset\to \hom_{[C,1]}(1,0)$$
are equivalences.
\end{lemma}
\begin{proof}
As both $\hom$ and $[\uvar,1]$ preserve colimits, it is sufficient to check this property on representables, where it is an easy computation.
\end{proof}

\begin{prop}
\label{prop:supspension preserves cat}
The functor $[\uvar,1]:\iPsh{\Theta}\to \iPsh{\Theta}$ preserves $\io$-categories.
\end{prop}
\begin{proof}
By construction, for any pair of integers $k<n$, and any pair of globular sums $([\textbf{a},n],b)$, we have cartesian squares
% https://q.uiver.app/#q=WzAsOCxbMSwwLCJcXEhvbV97XFxUaGV0YX0oW1xcdGV4dGJme2F9LG5dLFtiLDFdKSJdLFswLDAsIjEiXSxbMSwxLCJcXEhvbV97XFxEZWx0YX0oW25dLFsxXSkiXSxbMCwxLCJcXHtcXGVwc2lsb25cXH0iXSxbMywxLCJcXEhvbV97XFxEZWx0YX0oW25dLFsxXSkiXSxbMiwxLCJcXHtcXGFscGhhX2tcXH0iXSxbMywwLCJcXEhvbV97XFxUaGV0YX0oW1xcdGV4dGJme2F9LG5dLFtiLDFdKSJdLFsyLDAsIlxcSG9tX3tcXFRoZXRhfShhX2ssYikiXSxbMywyXSxbMSwwXSxbMCwyXSxbMSwzXSxbNiw0XSxbNSw0XSxbNyw1XSxbNyw2XSxbMSw4LCIiLDEseyJsZXZlbCI6MSwic3R5bGUiOnsibmFtZSI6ImNvcm5lciJ9fV0sWzcsMTMsIiIsMSx7ImxldmVsIjoxLCJzdHlsZSI6eyJuYW1lIjoiY29ybmVyIn19XV0=
\[\begin{tikzcd}
	1 & {\Hom_{\Theta}([\textbf{a},n],[b,1])} & {\Hom_{\Theta}(a_k,b)} & {\Hom_{\Theta}([\textbf{a},n],[b,1])} \\
	{\{\epsilon\}} & {\Hom_{\Delta}([n],[1])} & {\{\alpha_k\}} & {\Hom_{\Delta}([n],[1])}
	\arrow[""{name=0, anchor=center, inner sep=0}, from=2-1, to=2-2]
	\arrow[from=1-1, to=1-2]
	\arrow[from=1-2, to=2-2]
	\arrow[from=1-1, to=2-1]
	\arrow[from=1-4, to=2-4]
	\arrow[""{name=1, anchor=center, inner sep=0}, from=2-3, to=2-4]
	\arrow[from=1-3, to=2-3]
	\arrow[from=1-3, to=1-4]
	\arrow["\lrcorner"{anchor=center, pos=0.125}, draw=none, from=1-1, to=0]
	\arrow["\lrcorner"{anchor=center, pos=0.125}, draw=none, from=1-3, to=1]
\end{tikzcd}\]
where $\epsilon$ denote any constant functor with value $0$ or $1$, and $\alpha_k$ the morphism that sends $k$ on $0$ and $k+1$ on $1$.
Let $C$ be an $\io$-category.
As the $\iun$-category $\igrd$ is locally cartesian closed,  we have  cartesian squares
% https://q.uiver.app/#q=WzAsOCxbMSwwLCJcXEhvbV97XFxpUHNoe1xcVGhldGF9fShbXFx0ZXh0YmZ7YX0sbl0sW0MsMV0pIl0sWzAsMCwiMSJdLFsxLDEsIlxcSG9tX3tcXERlbHRhfShbbl0sWzFdKSJdLFswLDEsIlxce1xcZXBzaWxvblxcfSJdLFszLDEsIlxcSG9tX3tcXERlbHRhfShbbl0sWzFdKSJdLFsyLDEsIlxce1xcYWxwaGFfa1xcfSJdLFszLDAsIlxcSG9tX3tcXGlQc2h7XFxUaGV0YX19KFtcXHRleHRiZnthfSxuXSxbQywxXSkiXSxbMiwwLCJcXEhvbV97XFxpUHNoe1xcVGhldGF9fShhX2ssQykiXSxbMywyXSxbMSwwXSxbMCwyXSxbMSwzXSxbNiw0XSxbNSw0XSxbNyw1XSxbNyw2XSxbMSw4LCIiLDEseyJsZXZlbCI6MSwic3R5bGUiOnsibmFtZSI6ImNvcm5lciJ9fV0sWzcsMTMsIiIsMSx7ImxldmVsIjoxLCJzdHlsZSI6eyJuYW1lIjoiY29ybmVyIn19XV0=
\begin{equation}
\label{eq:prop:supspension preserves cat}
\begin{tikzcd}[column sep =0.3cm]
	1 & {\Hom_{\iPsh{\Theta}}([\textbf{a},n],[C,1])} & {\Hom_{\iPsh{\Theta}}(a_k,C)} & {\Hom_{\iPsh{\Theta}}([\textbf{a},n],[C,1])} \\
	{\{\epsilon\}} & {\Hom_{\Delta}([n],[1])} & {\{\alpha_k\}} & {\Hom_{\Delta}([n],[1])}
	\arrow[""{name=0, anchor=center, inner sep=0}, from=2-1, to=2-2]
	\arrow[from=1-1, to=1-2]
	\arrow[from=1-2, to=2-2]
	\arrow[from=1-1, to=2-1]
	\arrow[from=1-4, to=2-4]
	\arrow[""{name=1, anchor=center, inner sep=0}, from=2-3, to=2-4]
	\arrow[from=1-3, to=2-3]
	\arrow[from=1-3, to=1-4]
	\arrow["\lrcorner"{anchor=center, pos=0.125}, draw=none, from=1-1, to=0]
	\arrow["\lrcorner"{anchor=center, pos=0.125}, draw=none, from=1-3, to=1]
\end{tikzcd}
\end{equation}
which induces cartesian squares
% https://q.uiver.app/#q=WzAsOCxbMSwwLCJcXEhvbV97XFxpUHNoe1xcVGhldGF9fShcXFNwX3tbXFx0ZXh0YmZ7YX0sbl19LFtDLDFdKSJdLFswLDAsIjEiXSxbMSwxLCJcXEhvbV97XFxEZWx0YX0oW25dLFsxXSkiXSxbMCwxLCJcXHtcXGVwc2lsb25cXH0iXSxbMywxLCJcXEhvbV97XFxEZWx0YX0oW25dLFsxXSkiXSxbMiwxLCJcXHtcXGFscGhhX2tcXH0iXSxbMywwLCJcXEhvbV97XFxpUHNoe1xcVGhldGF9fShcXFNwX3tbXFx0ZXh0YmZ7YX0sbl19LFtDLDFdKSJdLFsyLDAsIlxcSG9tX3tcXGlQc2h7XFxUaGV0YX19KFxcU3Bfe2Ffa30sQykiXSxbMywyXSxbMSwwXSxbMCwyXSxbMSwzXSxbNiw0XSxbNSw0XSxbNyw1XSxbNyw2XSxbMSw4LCIiLDEseyJsZXZlbCI6MSwic3R5bGUiOnsibmFtZSI6ImNvcm5lciJ9fV0sWzcsMTMsIiIsMSx7ImxldmVsIjoxLCJzdHlsZSI6eyJuYW1lIjoiY29ybmVyIn19XV0=
\[\begin{tikzcd}[column sep =0.3cm]
	1 & {\Hom_{\iPsh{\Theta}}(\Sp_{[\textbf{a},n]},[C,1])} & {\Hom_{\iPsh{\Theta}}(\Sp_{a_k},C)} & {\Hom_{\iPsh{\Theta}}(\Sp_{[\textbf{a},n]},[C,1])} \\
	{\{\epsilon\}} & {\Hom_{\Delta}([n],[1])} & {\{\alpha_k\}} & {\Hom_{\Delta}([n],[1])}
	\arrow[""{name=0, anchor=center, inner sep=0}, from=2-1, to=2-2]
	\arrow[from=1-1, to=1-2]
	\arrow[from=1-2, to=2-2]
	\arrow[from=1-1, to=2-1]
	\arrow[from=1-4, to=2-4]
	\arrow[""{name=1, anchor=center, inner sep=0}, from=2-3, to=2-4]
	\arrow[from=1-3, to=2-3]
	\arrow[from=1-3, to=1-4]
	\arrow["\lrcorner"{anchor=center, pos=0.125}, draw=none, from=1-1, to=0]
	\arrow["\lrcorner"{anchor=center, pos=0.125}, draw=none, from=1-3, to=1]
\end{tikzcd}\]
This directly implies that $[C,1]$ is $\Wseg$-local.

Furthermore, for any integer $n>0$, the cartesian squares \eqref{eq:prop:supspension preserves cat} induces cartesian squares
% https://q.uiver.app/#q=WzAsOCxbMSwwLCJcXEhvbV97XFxpUHNoe1xcVGhldGF9fShcXFNpZ21hXm5FXntlcX0sW0MsMV0pIl0sWzAsMCwiMSJdLFsxLDEsIlxcSG9tX3tcXERlbHRhfShbMV0sWzFdKSJdLFswLDEsIlxce1xcZXBzaWxvblxcfSJdLFszLDEsIlxcSG9tX3tcXERlbHRhfShbMV0sWzFdKSJdLFsyLDEsIlxce1xcYWxwaGFfa1xcfSJdLFszLDAsIlxcSG9tX3tcXGlQc2h7XFxUaGV0YX19KFxcU2lnbWFebkVee2VxfSxbQywxXSkiXSxbMiwwLCJcXEhvbV97XFxpUHNoe1xcVGhldGF9fShcXFNpZ21hXntuLTF9RV57ZXF9LEMpIl0sWzMsMl0sWzEsMF0sWzAsMl0sWzEsM10sWzYsNF0sWzUsNF0sWzcsNV0sWzcsNl0sWzEsOCwiIiwxLHsibGV2ZWwiOjEsInN0eWxlIjp7Im5hbWUiOiJjb3JuZXIifX1dLFs3LDEzLCIiLDEseyJsZXZlbCI6MSwic3R5bGUiOnsibmFtZSI6ImNvcm5lciJ9fV1d
\[\begin{tikzcd}[column sep =0.2cm]
	1 & {\Hom_{\iPsh{\Theta}}(\Sigma^nE^{eq},[C,1])} & {\Hom_{\iPsh{\Theta}}(\Sigma^{n-1}E^{eq},C)} & {\Hom_{\iPsh{\Theta}}(\Sigma^nE^{eq},[C,1])} \\
	{\{\epsilon\}} & {\Hom_{\Delta}([1],[1])} & {\{\alpha_k\}} & {\Hom_{\Delta}([1],[1])}
	\arrow[""{name=0, anchor=center, inner sep=0}, from=2-1, to=2-2]
	\arrow[from=1-1, to=1-2]
	\arrow[from=1-2, to=2-2]
	\arrow[from=1-1, to=2-1]
	\arrow[from=1-4, to=2-4]
	\arrow[""{name=1, anchor=center, inner sep=0}, from=2-3, to=2-4]
	\arrow[from=1-3, to=2-3]
	\arrow[from=1-3, to=1-4]
	\arrow["\lrcorner"{anchor=center, pos=0.125}, draw=none, from=1-1, to=0]
	\arrow["\lrcorner"{anchor=center, pos=0.125}, draw=none, from=1-3, to=1]
\end{tikzcd}\]
which implies that $[C,1]$  is local with respect to $\Sigma^nE^{eq}\to \Sigma^{n}1$.

Eventually, suppose given a diagram of shape
\begin{equation}
\label{eq:proof of sigma preserves omega cat 3}
\begin{tikzcd}
	{E^{eq}} & {[C,1]} \\
	{1}
	\arrow[from=1-1, to=2-1]
	\arrow[from=1-1, to=1-2]
\end{tikzcd}
\end{equation}
The canonical morphism $E^{eq}\to [C,1]\xrightarrow{\pi} [1]$ then factors through $0$ or $1$. As the two fibers of $\pi$ are trivial,  the diagram \eqref{eq:proof of sigma preserves omega cat 3} admits a unique lift, which concludes the proof.
\end{proof}



\p
\label{para:wiskering} As $[\uvar,1]$ sends $\W$ to a subset of $\M$, the functor $\hom_{\uvar,\uvar}(\uvar)$ preserves $\io$-categories. Combined with the last proposition, this implies that 
the adjunction \eqref{eq:suspesnion betweenpresheaves} restricts to an adjunction:
% https://q.uiver.app/#q=WzAsMixbMSwwLCJcXG9jYXRfe1xcYnVsbGV0LFxcYnVsbGV0fTpcXGhvbV97XFx1dmFyfShcXHV2YXIsXFx1dmFyKSJdLFswLDAsIltcXHV2YXIsMV06XFxvY2F0Il0sWzEsMCwiIiwyLHsib2Zmc2V0IjotMn1dLFswLDEsIiIsMix7Im9mZnNldCI6LTJ9XSxbMiwzLCIiLDIseyJsZXZlbCI6MSwic3R5bGUiOnsibmFtZSI6ImFkanVuY3Rpb24ifX1dXQ==
\begin{equation}
\label{eq:suspesnion between category}
\begin{tikzcd}
	{[\uvar,1]:\ocat} & {\ocat_{\bullet,\bullet}:\hom_{\uvar}(\uvar,\uvar)}
	\arrow[""{name=0, anchor=center, inner sep=0}, shift left=2, from=1-1, to=1-2]
	\arrow[""{name=1, anchor=center, inner sep=0}, shift left=2, from=1-2, to=1-1]
	\arrow["\dashv"{anchor=center, rotate=-90}, draw=none, from=0, to=1]
\end{tikzcd}
\end{equation}
The left adjoint is the \wcsnotionsym{suspension functor}{((d60@$[\uvar,1]$}{suspension}{for $\io$-categories}\ssym{(hom@$\hom_{\uvar}(\uvar,\uvar)$}{for $\io$-categories}.
\begin{prop}
\label{prop:hom of the suspension}
Let $C$ be an $\io$-categories. We have natural equivalences
$$\hom_{[C,1]}(0,1)\sim C~~~~\hom_{[C,1]}(0,0)\sim \hom_{[C,1]}(1,1) \sim 1~~~~\hom_{[C,1]}(1,0)\sim \emptyset.$$
\end{prop}
\begin{proof}
This is a direct consequence of lemma \ref{lemma:hom of the suspension prequel}.
\end{proof}



\p
Suppose given an $\io$-category $C$ and a $1$-cells $f:x'\to x$. 
As $C$ is an $\io$-category, for any globular sum $a$, the morphism
$$\Hom([1]\vee[a,1],C)\to \Hom([1], C)\times_{\Hom([0],C)}\Hom([a,1],C)$$
is an equivalence. 
This induces a morphism
$$\Hom(a,\hom_C(x,y))\to \Hom([1]\vee[a,1],(C,x',y))\to \Hom(a,\hom_{C}(x',y))$$	
where the two distinguished points of $[1]\vee[a,1]$ are the extremal ones, and where the left-hand morphism is the restriction of the inverse of the previous morphism. By the Yoneda lemma, this corresponds to a morphism
$$f_!:\hom_C(x',y)\to \hom_C(x,y).$$
Conversely, a $1$-cell $g:y\to y'$ induces a morphism
$$g_!:\hom_C(x,y)\to \hom_C(x,y').$$

 
\p \label{para: spetial colimits}
We denote by $\iota$ the inclusion of $\ocat$ into $\iPsh{\Theta}$.
A functor $F:I\to \ocat$ has a \snotion{special colimit}{for $\io$-categories} if the canonical morphism 
\begin{equation}
\label{eq:special colimit}
\colim_{i:I}\iota F(i)\to \iota(\colim_{i:I}F(i))
\end{equation}
is an equivalence of presheaves. 

Similarly, we say that a functor $\psi: I\to \Arr(\ocat)$ has a \textit{special colimit} if the canonical morphism 
$$\colim_{i:I}\iota \psi(i)\to \iota(\colim_{i:I}\psi(i))$$
is an equivalence in the arrow $\iun$-category of $\iPsh{\Theta}$.


\begin{example}
Let $C$ be an $\io$-category. The canonical diagram $\Theta_{/C}\to \ocat$ has a special colimit, given by $C$.
\end{example}
\begin{prop}
\label{prop:special colimit}
Let $F,G:I\to \ocat$ be two functors, and $\psi:F\to G$ a natural transformation. If $\psi$ is cartesian, and $G$ has a special colimit, then $\psi$ and $F$ have special colimits. 
\end{prop}
\begin{proof}
We have to show that $F$ has a special colimit, it will directly imply that $\psi$ also has one. The morphism \eqref{eq:special colimit} is always in $\widehat{\W}$. To conclude, one then has to show that $\colim_{i:I}\iota \psi(i)$ is $\W$-local. To this extend, it is enough to demonstrate that the canonical morphism 
$$\colim_{i:I}\iota \psi(i): \colim_{i:I}\iota F(i)\to \colim_{i:I}\iota G(i)$$ 
has the unique right lifting property against $\W$. We then consider a square
\begin{equation}
\label{eq:proof special colimit}
\begin{tikzcd}
	a & {\colim_{i:I}\iota F(i)} \\
	b & { \colim_{i:I}\iota G(i)}
	\arrow["{\colim_{i:I}\iota \psi(i)}", from=1-2, to=2-2]
	\arrow[from=2-1, to=2-2]
	\arrow[from=1-1, to=2-1]
	\arrow[from=1-1, to=1-2]
\end{tikzcd}
\end{equation}
where $f\in W$. As the domain of $f$ is representable, there always exists $j:I$, such that the bottom horizontal morphism factors through $G(j)$. As $\psi$ is cartesian, the square \eqref{eq:proof special colimit} factors in two squares, where the right one is cartesian. 
% q.uiver.app/#q=WzAsNixbMCwwLCJhIl0sWzAsMSwiYiJdLFsyLDAsIlxcY29saW1fe2k6SX1cXGlvdGEgRihpKSJdLFsyLDEsIiBcXGNvbGltX3tpOkl9XFxpb3RhIEcoaSkiXSxbMSwxLCJHKGkpIl0sWzEsMCwiRihpKSJdLFsyLDMsIlxcY29saW1fe2k6SX1cXGlvdGEgXFxwc2koaSkiXSxbMCwxXSxbNSw0LCJcXHBzaShpKSJdLFs1LDJdLFs0LDNdLFs1LDMsIiIsMSx7InN0eWxlIjp7Im5hbWUiOiJjb3JuZXIifX1dLFswLDVdLFsxLDRdXQ==
\[\begin{tikzcd}
	a & {F(i)} & {\colim_{i:I}\iota F(i)} \\
	b & {G(i)} & { \colim_{i:I}\iota G(i)}
	\arrow["{\colim_{i:I}\iota \psi(i)}", from=1-3, to=2-3]
	\arrow[from=1-1, to=2-1]
	\arrow["{\psi(i)}", from=1-2, to=2-2]
	\arrow[from=1-2, to=1-3]
	\arrow[from=2-2, to=2-3]
	\arrow["\lrcorner"{anchor=center, pos=0.125}, draw=none, from=1-2, to=2-3]
	\arrow[from=1-1, to=1-2]
	\arrow[from=2-1, to=2-2]
\end{tikzcd}\]
Lifts in the square \eqref{eq:proof special colimit} are then equivalent to lifts in the left square, which exist and are unique as $F(i)\to G(i)$ has the unique right lifting property against $\W$.
\end{proof}

\begin{prop}
\label{prop:example of a special colimit}
For any integer $n$, and globular sums $a$ and $b$, the equalizer diagram 
% q.uiver.app/#q=WzAsMixbMCwwLCJcXGNvcHJvZF97aytsPW4tMX1bYSxrXVxcdmVlW2FcXHRpbWVzIGIsMV1cXHZlZVthLGxdIl0sWzEsMCwiXFxjb3Byb2Rfe2srbD1ufVthLGtdXFx2ZWVbIGIsMV1cXHZlZVthLGxdIl0sWzAsMSwiIiwyLHsib2Zmc2V0IjotMn1dLFswLDEsIiIsMCx7Im9mZnNldCI6Mn1dXQ==
\[\begin{tikzcd}
	{\coprod_{k+l=n-1}[a,k]\vee[a\times b,1]\vee[a,l]} & {\coprod_{k+l=n}[a,k]\vee[ b,1]\vee[a,l]}
	\arrow[shift left=2, from=1-1, to=1-2]
	\arrow[shift right=2, from=1-1, to=1-2]
\end{tikzcd}\]
where the top diagram is induced by $[a\times b,1]\to [a,1]\vee[b,1]$ and to bottom one by $[a\times b,1]\to [b,1]\vee[a,1]$,
has a special colimit, which is $[a,n]\times [b,1]$.
\end{prop}
\begin{proof}
The lemma \ref{lemma:colimit computed in set presheaves} implies that the colimit of the previous diagram, computed in $\iPsh{\Theta}$ is strict. It is then enough to show that this colimit, computed in $\Psh{\Theta}$, is equivalent to $[a,n]\times [b,1]$. As this last object is $\W$-local, this will concludes the proof. The remaining combinatorial exercise is left to the reader. 
\end{proof}

\begin{prop}
\label{prop:example of a special colimit 2}
Any sequence of $\io$-categories has a special colimit. 
\end{prop}
\begin{proof}
Suppose given such sequence.
If the sequence is finite, this is obviously true. Suppose now that the sequence is non finite. As codomains and domains of morphism of $\W$ are $\omega$-small, the colimit of the sequence, computed in $\iPsh{\Theta}$ is $\W$-local, which concludes the proof.
\end{proof}


\begin{lemma}
\label{lemma:[ ,1] preserves spcial limits}
The functor $[\uvar,1]:\ocat\to \ocat_{\bullet,\bullet}$
preserves special colimits.
\end{lemma}
\begin{proof}
This is a direct consequence of proposition \ref{prop:supspension preserves cat}.
\end{proof}

\begin{lemma}
\label{lemma:[ ,1] vee [ ,1]preserves spcial limits}
We denote by $$
\begin{array}{cl}
 & [\uvar,1]\vee[1]:\ocat\to \ocat_{[0]\amalg[1]/}\\
 \mbox{(resp.} & [1]\vee[\uvar,1]:\ocat\to \ocat_{[1]\amalg[0]/})
\end{array}$$ the colimit preserving functor that sends an element $a$ of $\Theta$ onto the globular sum $[a,1]\vee[1]$ (resp. $[1]\vee[a,1]$).

The functors $[\uvar,1]\vee[\uvar,1]$  and $[1]\vee[\uvar,1]$ preserve special colimits.
\end{lemma}
\begin{proof}
To prove this, we establish a result analogous to the one given in the proposition \ref{prop:supspension preserves cat}. We omit its proof because it is long but essentially identical.
\end{proof}

\begin{prop}
\label{prop:example of a special colimit3}
Suppose given two cartesian squares
% https://q.uiver.app/#q=WzAsNixbMCwwLCIgQiJdLFsxLDAsIkMiXSxbMiwwLCJEIl0sWzEsMSwiWzFdIl0sWzAsMSwiXFx7MFxcfSJdLFsyLDEsIlxcezFcXH0iXSxbMSwzXSxbMCwxXSxbMiw1XSxbMCw0XSxbMiwxXSxbNCwzXSxbNSwzXSxbMCwzLCIiLDAseyJzdHlsZSI6eyJuYW1lIjoiY29ybmVyIn19XSxbMiwzLCIiLDAseyJzdHlsZSI6eyJuYW1lIjoiY29ybmVyIn19XV0=
\[\begin{tikzcd}
	{ B} & C & D \\
	{\{0\}} & {[1]} & {\{1\}}
	\arrow[from=1-2, to=2-2]
	\arrow[from=1-1, to=1-2]
	\arrow[from=1-3, to=2-3]
	\arrow[from=1-1, to=2-1]
	\arrow[from=1-3, to=1-2]
	\arrow[from=2-1, to=2-2]
	\arrow[from=2-3, to=2-2]
	\arrow["\lrcorner"{anchor=center, pos=0.125}, draw=none, from=1-1, to=2-2]
	\arrow["\lrcorner"{anchor=center, pos=0.125, rotate=-90}, draw=none, from=1-3, to=2-2]
\end{tikzcd}\]
The diagram 
% https://q.uiver.app/#q=WzAsNSxbMSwwLCJbIEIsMV0iXSxbMywwLCJbRCwxXSJdLFswLDAsIlsxXVxcdmVlW0IsMV0iXSxbNCwwLCJbRCwxXVxcdmVlWzFdIl0sWzIsMCwiW0MsMV0iXSxbMSwzLCJcXHRyaWFuZ2xlZG93biJdLFswLDIsIlxcdHJpYW5nbGVkb3duIiwyXSxbMCw0XSxbMSw0XV0=
\[\begin{tikzcd}
	{[1]\vee[B,1]} & {[ B,1]} & {[C,1]} & {[D,1]} & {[D,1]\vee[1]}
	\arrow["\triangledown", from=1-4, to=1-5]
	\arrow["\triangledown"', from=1-2, to=1-1]
	\arrow[from=1-2, to=1-3]
	\arrow[from=1-4, to=1-3]
\end{tikzcd}\]
has a special colimit.
\end{prop}
\begin{proof}
Remark firsts that the colimit, computed in $\iPsh{\Theta}$, of the diagram
% https://q.uiver.app/#q=WzAsNSxbMCwwLCJbMV1cXHZlZVsxXSJdLFs0LDAsIlsxXVxcdmVlWzFdIl0sWzMsMCwiWzFdIl0sWzEsMCwiWzFdIl0sWzIsMCwiW1sxXSwxXSJdLFszLDAsIlxcdHJpYW5nbGVkb3duIiwyXSxbMiwxLCJcXHRyaWFuZ2xlZG93biJdLFszLDQsIltcXHswXFx9LDFdIl0sWzIsNCwiW1xcezFcXH0sMV0iLDJdXQ==
\[\begin{tikzcd}
	{[1]\vee[1]} & {[1]} & {[[1],1]} & {[1]} & {[1]\vee[1]}
	\arrow["\triangledown"', from=1-2, to=1-1]
	\arrow["\triangledown", from=1-4, to=1-5]
	\arrow["{[\{0\},1]}", from=1-2, to=1-3]
	\arrow["{[\{1\},1]}"', from=1-4, to=1-3]
\end{tikzcd}\]
is strict. We leave it to the reader to check that the previous diagram has a special colimit. 

Remark now that $\Theta$ is stable by pullback and $[\uvar,1]$ preserves cartesian squares in $\Theta$. 
The lemma \ref{lemma:[ ,1] preserves spcial limits} states that $[\uvar,1]$ preserves  special colimit, and as $\iPsh{\Theta}$ is locally cartesian closed, pullbacks also preserve them.
As every $\io$-category is a special colimit of representables, this implies that the squares
% https://q.uiver.app/#q=WzAsNixbMCwwLCJbQiwxXSJdLFsxLDAsIltDLDFdIl0sWzIsMCwiW0QsMV0iXSxbMSwxLCJbWzFdLDFdIl0sWzAsMSwiWzFdIl0sWzIsMSwiWzFdIl0sWzAsMV0sWzIsMV0sWzQsMywiW1xcezBcXH0sMV0iLDJdLFs1LDMsIltcXHsxXFx9LDFdIl0sWzAsNF0sWzEsM10sWzAsMywiIiwwLHsic3R5bGUiOnsibmFtZSI6ImNvcm5lciJ9fV0sWzIsNV0sWzIsMywiIiwwLHsic3R5bGUiOnsibmFtZSI6ImNvcm5lciJ9fV1d
\[\begin{tikzcd}
	{[B,1]} & {[C,1]} & {[D,1]} \\
	{[1]} & {[[1],1]} & {[1]}
	\arrow[from=1-1, to=1-2]
	\arrow[from=1-3, to=1-2]
	\arrow["{[\{0\},1]}"', from=2-1, to=2-2]
	\arrow["{[\{1\},1]}", from=2-3, to=2-2]
	\arrow[from=1-1, to=2-1]
	\arrow[from=1-2, to=2-2]
	\arrow["\lrcorner"{anchor=center, pos=0.125}, draw=none, from=1-1, to=2-2]
	\arrow[from=1-3, to=2-3]
	\arrow["\lrcorner"{anchor=center, pos=0.125, rotate=-90}, draw=none, from=1-3, to=2-2]
\end{tikzcd}\]
are cartesian.
Furthermore, for any globular sum $b$, we have cartesian squares
% https://q.uiver.app/#q=WzAsOCxbMiwwLCJbYiwxXSJdLFszLDAsIlsxXVxcdmVlW2IsMV0iXSxbMywxLCJbMV1cXHZlZVsxXSJdLFsyLDEsIlsxXSJdLFsxLDAsIltiLDFdXFx2ZWVbMV0iXSxbMSwxLCJbMV1cXHZlZVsxXSJdLFswLDAsIltiLDFdIl0sWzAsMSwiWzFdIl0sWzMsMiwiXFx0cmlhbmdsZWRvd24iLDJdLFsxLDJdLFswLDNdLFswLDFdLFs0LDVdLFsxLDMsIiIsMix7InN0eWxlIjp7Im5hbWUiOiJjb3JuZXIifX1dLFs3LDUsIlxcdHJpYW5nbGVkb3duIiwyXSxbNiw3XSxbNCw3LCIiLDIseyJzdHlsZSI6eyJuYW1lIjoiY29ybmVyIn19XSxbNiw0XV0=
\[\begin{tikzcd}
	{[b,1]} & {[b,1]\vee[1]} & {[b,1]} & {[1]\vee[b,1]} \\
	{[1]} & {[1]\vee[1]} & {[1]} & {[1]\vee[1]}
	\arrow["\triangledown"', from=2-3, to=2-4]
	\arrow[from=1-4, to=2-4]
	\arrow[from=1-3, to=2-3]
	\arrow[from=1-3, to=1-4]
	\arrow[from=1-2, to=2-2]
	\arrow["\lrcorner"{anchor=center, pos=0.125, rotate=-90}, draw=none, from=1-4, to=2-3]
	\arrow["\triangledown"', from=2-1, to=2-2]
	\arrow[from=1-1, to=2-1]
	\arrow["\lrcorner"{anchor=center, pos=0.125, rotate=-90}, draw=none, from=1-2, to=2-1]
	\arrow[from=1-1, to=1-2]
\end{tikzcd}\]
According to lemma \ref{lemma:[ ,1] vee [ ,1]preserves spcial limits}, $[\uvar,1]\vee[1]$ and $[1]\vee[\uvar,1]$ preserve special colimits. As every $\io$-category is a colimit of representables, this implies that the squares
% https://q.uiver.app/#q=WzAsOCxbMCwwLCJbQiwxXSJdLFsyLDAsIltELDFdIl0sWzEsMCwiWzFdXFx2ZWVbQiwxXSJdLFszLDAsIltELDFdXFx2ZWVbMV0iXSxbMSwxLCJbMV1cXHZlZVsxXSJdLFszLDEsIlsxXVxcdmVlWzFdIl0sWzIsMSwiWzFdIl0sWzAsMSwiWzFdIl0sWzEsM10sWzAsMl0sWzcsNCwiXFx0cmlhbmdsZWRvd24iLDJdLFs2LDUsIlxcdHJpYW5nbGVkb3duIiwyXSxbMyw1XSxbMSw2XSxbMCw3XSxbMiw0XSxbMCw0LCIiLDAseyJzdHlsZSI6eyJuYW1lIjoiY29ybmVyIn19XSxbMSw1LCIiLDAseyJzdHlsZSI6eyJuYW1lIjoiY29ybmVyIn19XV0=
\[\begin{tikzcd}
	{[B,1]} & {[1]\vee[B,1]} & {[D,1]} & {[D,1]\vee[1]} \\
	{[1]} & {[1]\vee[1]} & {[1]} & {[1]\vee[1]}
	\arrow[from=1-3, to=1-4]
	\arrow[from=1-1, to=1-2]
	\arrow["\triangledown"', from=2-1, to=2-2]
	\arrow["\triangledown"', from=2-3, to=2-4]
	\arrow[from=1-4, to=2-4]
	\arrow[from=1-3, to=2-3]
	\arrow[from=1-1, to=2-1]
	\arrow[from=1-2, to=2-2]
	\arrow["\lrcorner"{anchor=center, pos=0.125}, draw=none, from=1-1, to=2-2]
	\arrow["\lrcorner"{anchor=center, pos=0.125}, draw=none, from=1-3, to=2-4]
\end{tikzcd}\]
are cartesian.
The result then follows from proposition \ref{prop:special colimit}.
\end{proof}
\begin{prop}
\label{prop:example of a special colimit4}
Suppose given a cartesian square
% https://q.uiver.app/#q=WzAsNCxbMCwwLCIgQiJdLFsxLDAsIkMiXSxbMSwxLCJbMV0iXSxbMCwxLCJcXHswXFx9Il0sWzEsMl0sWzAsMV0sWzAsM10sWzMsMl0sWzAsMiwiIiwwLHsic3R5bGUiOnsibmFtZSI6ImNvcm5lciJ9fV1d
\[\begin{tikzcd}
	{ B} & C \\
	{\{0\}} & {[1]}
	\arrow[from=1-2, to=2-2]
	\arrow[from=1-1, to=1-2]
	\arrow[from=1-1, to=2-1]
	\arrow[from=2-1, to=2-2]
	\arrow["\lrcorner"{anchor=center, pos=0.125}, draw=none, from=1-1, to=2-2]
\end{tikzcd}\]
The diagram 
% https://q.uiver.app/#q=WzAsMyxbMSwwLCJbIEIsMV0iXSxbMCwwLCJbMV1cXHZlZVtCLDFdIl0sWzIsMCwiW0MsMV0iXSxbMCwxLCJcXHRyaWFuZ2xlZG93biIsMl0sWzAsMl1d
\[\begin{tikzcd}
	{[1]\vee[B,1]} & {[ B,1]} & {[C,1]}
	\arrow["\triangledown"', from=1-2, to=1-1]
	\arrow[from=1-2, to=1-3]
\end{tikzcd}\]
has a special colimit.
\end{prop}
\begin{proof}
The proof is similar to the previous one. 
\end{proof}



\p  We have an adjunction 
\begin{equation}
\label{eq:underived adjunction case n}
\begin{tikzcd}
	{ i_!:\iPsh{\Delta[\Theta_{n-1}]}} & {\iPsh{\Theta_n}:i^*}
	\arrow[shift left=2, from=1-1, to=1-2]
	\arrow[shift left=2, from=1-2, to=1-1]
\end{tikzcd}
\end{equation}
where the left adjoint is the left Kan extension of the functor $\Delta[\Theta_{n-1}]\xrightarrow{i} \Theta_{n}\to \iPsh{\Theta_{n}}$. We recall that the sets of morphisms $\W_n$ and $\M_n$ are respectively defined in paragraphs \ref{para:definition of W} and \ref{para:defi of delta theta}.
Remark that there is an obvious inclusion $i_!(\M_n)\subset \W_n$. The previous adjunction then induced a derived adjunction
\begin{equation}
\label{eq:derived adjunction case n}
\begin{tikzcd}
	{\Lb i_!:\Psh{\Delta[\Theta_{n-1}]}_{\M}} & {\Psh{\Theta_{n}}_{\W}:\Rb i^*}
	\arrow[shift left=2, from=1-1, to=1-2]
	\arrow[shift left=2, from=1-2, to=1-1]
\end{tikzcd}
\end{equation}

\begin{prop}
\label{prop:infini changing theta n}
The unit and counit of the adjunction \eqref{eq:underived adjunction case n} are respectively in $\widehat{\M}_n$ and $\widehat{\W}_n$. As a consequence, the adjunction \eqref{eq:derived adjunction case n} is an adjoint equivalence.
\end{prop}
\begin{proof}
We denote by $\iota:\Psh{\Theta_n}\to \iPsh{\Theta_n}$ and $\iota:\Psh{\Delta[\Theta_{n-1}]}\to \iPsh{\Delta[\Theta_{n-1}]}$ the two canonical inclusions. By the definition of the smallest precocomplete class (paragraph \ref{para:precomplet}) and according to lemma \ref{lemma:colimit computed in set presheaves}, we have inclusions $\iota(\overline{\W_{n}})\subset \widehat{\W_{n}}$ and $\iota(\overline{\M_{n}})\subset \widehat{\M_{n}}$. The result then directly follows from theorem \ref{theo:unit and counit are in W}.  
\end{proof}


\p \label{para:truncation and inteligent trucation}
 Let $n>0$ be an integer. An \wcnotion{$(\infty,n)$-category}{category3@$(\infty,n)$-category} is a $\W_n$-local $\infty$-presheaf $C\in \iPsh{\Theta_n}$. We then define \sym{((a50@$\ncat{n}$}
$$\ncat{n} := \iPsh{\Theta_n}_{\W_n}.$$
Remark that the $\iun$-category $\ncat{0}$ is equivalent to $\igrd$.
Proposition \ref{prop:infini changing theta n} implies that $\ncat{n}$ identifies itself with the full sub $\iun$-category of $\iPsh{\Delta[\Theta_{n-1}]}$ of $\M_n$-local objects:
$$\ncat{n} \sim \iPsh{\Delta[\Theta_{n-1}]}_{\M_n}.$$
The inclusion $i_n:\Theta_n\to \Theta$ fits in an adjunction
% q.uiver.app/#q=WzAsMixbMSwwLCJcXFRoZXRhX246aV9uIl0sWzAsMCwiXFx0YXVeaV9uOlxcVGhldGEiXSxbMSwwLCIiLDAseyJvZmZzZXQiOi0yfV0sWzAsMSwiIiwwLHsib2Zmc2V0IjotMn1dLFsyLDMsIiIsMCx7ImxldmVsIjoxLCJzdHlsZSI6eyJuYW1lIjoiYWRqdW5jdGlvbiJ9fV1d
\[\begin{tikzcd}
	{\tau^i_n:\Theta} & {\Theta_n:i_n}
	\arrow[""{name=0, anchor=center, inner sep=0}, shift left=2, from=1-1, to=1-2]
	\arrow[""{name=1, anchor=center, inner sep=0}, shift left=2, from=1-2, to=1-1]
	\arrow["\dashv"{anchor=center, rotate=-90}, draw=none, from=0, to=1]
\end{tikzcd}\]
where the left adjoint sends $\Db_k$ on $\Db_{\min{(n,k)}}$.
By extension by colimits, this induces an adjoint pair 
% https://q.uiver.app/#q=WzAsMixbMSwwLCJcXGlQc2h7XFxUaGV0YV9ufTppX24iXSxbMCwwLCJcXHRhdV5pX246XFxpUHNoe1xcVGhldGF9Il0sWzEsMCwiIiwwLHsib2Zmc2V0IjotMn1dLFswLDEsIiIsMCx7Im9mZnNldCI6LTJ9XSxbMiwzLCIiLDAseyJsZXZlbCI6MSwic3R5bGUiOnsibmFtZSI6ImFkanVuY3Rpb24ifX1dXQ==
\begin{equation}
\label{eq:inclusion of n cat pre}
\begin{tikzcd}
	{\tau^i_n:\iPsh{\Theta}} & {\iPsh{\Theta_n}:i_n.}
	\arrow[""{name=0, anchor=center, inner sep=0}, shift left=2, from=1-1, to=1-2]
	\arrow[""{name=1, anchor=center, inner sep=0}, shift left=2, from=1-2, to=1-1]
	\arrow["\dashv"{anchor=center, rotate=-90}, draw=none, from=0, to=1]
\end{tikzcd}
\end{equation}
where the two functors are colimit preserving.
As the image of every morphism of $\W$ by $\tau^i_n$ is in $\W_n$ or is an equivalence, and as the image of $\W_n$ by $i_n$ is included in $\W$, the previous adjunction induces by localization an adjunction
% q.uiver.app/#q=WzAsMixbMSwwLCJcXG5jYXR7bn06aV9uIl0sWzAsMCwiXFx0YXVeaV9uOlxcb2NhdCJdLFsxLDAsIiIsMCx7Im9mZnNldCI6LTJ9XSxbMCwxLCIiLDAseyJvZmZzZXQiOi0yfV0sWzIsMywiIiwwLHsibGV2ZWwiOjEsInN0eWxlIjp7Im5hbWUiOiJhZGp1bmN0aW9uIn19XV0=
\begin{equation}
\label{eq:inclusion of n cat}
\begin{tikzcd}
	{\tau^i_n:\ocat} & {\ncat{n}:i_n}
	\arrow[""{name=0, anchor=center, inner sep=0}, shift left=2, from=1-1, to=1-2]
	\arrow[""{name=1, anchor=center, inner sep=0}, shift left=2, from=1-2, to=1-1]
	\arrow["\dashv"{anchor=center, rotate=-90}, draw=none, from=0, to=1]
\end{tikzcd}
\end{equation}
where the two adjoints are colimit preserving.
The left adjoint is called the \snotionsym{intelligent $n$-truncation}{(taui@$\tau^i_n$}{for $\io$-categories}.
\begin{prop}
\label{ref:infini n a full sub cat}
The functor $i_n: \ncat{n}\to \ocat$ is fully faithful.
\end{prop}
\begin{proof}
We have to check that the unit of the adjunction \eqref{eq:inclusion of n cat} is an equivalence. As the two functors preserve colimits, we have to show that the restriction to $\Theta$ of the unit is an equivalence which is obvious.
\end{proof}
Being colimit preserving, the functor $i_n$ is also part of an adjunction
% q.uiver.app/#q=WzAsMixbMCwwLCJpX246XFxuY2F0e259Il0sWzEsMCwiXFxvY2F0OlxcdGF1X24iXSxbMSwwLCIiLDAseyJvZmZzZXQiOi0yfV0sWzAsMSwiIiwwLHsib2Zmc2V0IjotMn1dLFszLDIsIiIsMCx7ImxldmVsIjoxLCJzdHlsZSI6eyJuYW1lIjoiYWRqdW5jdGlvbiJ9fV1d
\begin{equation}
\begin{tikzcd}
	{i_n:\ncat{n}} & {\ocat:\tau_n}
	\arrow[""{name=0, anchor=center, inner sep=0}, shift left=2, from=1-2, to=1-1]
	\arrow[""{name=1, anchor=center, inner sep=0}, shift left=2, from=1-1, to=1-2]
	\arrow["\dashv"{anchor=center, rotate=-90}, draw=none, from=1, to=0]
\end{tikzcd}
\end{equation}
The right adjoint is called the \wcsnotionsym{$n$-truncation}{(tau@$\tau_n$}{truncation@$n$-truncation}{for $\io$-category}. 

We will identify objects of $\ncat{n}$ with their image in $\ocat$ and we will then also note by $\tau_n$ and $\tau^i_n$ the composites $i_n\tau^i_n$ and $i_n\tau^i_n$.


\begin{prop}
\label{prop:taun preserves special colimits}
The functor $\tau_n:\ocat\to \ocat$ preserves special colimits.
\end{prop}
\begin{proof}
As $i_n$ preserves representable objects, the functor $\tau_n:\ocat\to \ncat{n}$ preserves special colimits. As $i_n:\iPsh{\Theta_n}\to \iPsh{\Theta}$ preserves colimits and $\W$-local objects, this concludes the proof.
\end{proof}

\begin{prop}
\label{prop:inteligent trucatio and a particular colimit}
Let $C$ be an $\io$-category and $n$ an integer. The following canonical square is cartesian
% https://q.uiver.app/#q=WzAsNCxbMCwwLCJDIl0sWzEsMCwiXFx0YXVfbl5pQyJdLFswLDEsIlxcdGF1X25eaUMiXSxbMSwxLCJcXHRhdV9uXmlDIl0sWzAsMl0sWzIsM10sWzEsM10sWzAsMV1d
\[\begin{tikzcd}
	C & {\tau_n^iC} \\
	{\tau_n^iC} & {\tau_n^iC}
	\arrow[from=1-1, to=2-1]
	\arrow[from=2-1, to=2-2]
	\arrow[from=1-2, to=2-2]
	\arrow[from=1-1, to=1-2]
\end{tikzcd}\]
\end{prop}
\begin{proof}
For this results we use model categories. The theorem \ref{theo:lecorozo} implies that the $\iun$-category $\ocat$ is presented by the category of marked simplicial sets $\mSset$ endowed with the model structure for $\omega$-complicial sets given by proposition \ref{prop:model structure on marked simplicial set}, and the functor $\tau^i_n:\ocat\to \ocat$ corresponds to the left Quillen functor $\tau^i_n:\mSset\to \mSset$ given in paragraph \ref{para:inteligentr trucation for simplicial set}. Remark that in this model category, for any marked simplicial set $X$, the following square is cocartesian
% https://q.uiver.app/#q=WzAsNCxbMCwwLCJYIl0sWzEsMCwiXFx0YXVfbl5pWCJdLFswLDEsIlxcdGF1X25eaVgiXSxbMSwxLCJcXHRhdV9uXmlYIl0sWzAsMl0sWzIsM10sWzEsM10sWzAsMV1d
\[\begin{tikzcd}
	X & {\tau_n^iX} \\
	{\tau_n^iX} & {\tau_n^iX}
	\arrow[from=1-1, to=2-1]
	\arrow[from=2-1, to=2-2]
	\arrow[from=1-2, to=2-2]
	\arrow[from=1-1, to=1-2]
\end{tikzcd}\]
As all the morphisms are cofibrations, this square is also homotopy cocartesian which concludes the proof.
\end{proof}






\p The family of truncation functor induces a sequence 
$$...\to \ncat{n+1}\xrightarrow{\tau_{n}} \ncat{n}\to...\to \ncat{1}\xrightarrow{\tau_{0}}\ncat{0}$$
which induces an adjunction
\begin{equation}
\label{eq:inductivity}
\begin{tikzcd}
	{\colim_{n:\Nb}:\lim_{n:\Nb}\ncat{n}} & {\ocat:(\tau_n)_{n:\Nb}}
	\arrow[""{name=0, anchor=center, inner sep=0}, shift left=2, from=1-1, to=1-2]
	\arrow[""{name=1, anchor=center, inner sep=0}, shift left=2, from=1-2, to=1-1]
	\arrow["\dashv"{anchor=center, rotate=-90}, draw=none, from=0, to=1]
\end{tikzcd}
\end{equation}
where the left adjoint sends a sequence $(C_n, C_n\sim \tau_nC_{n+1})_{n:\Nb}$ to the colimit of the induced sequence
$$i_0C_0\to i_1C_1\to ... \to i_nC_n\to ..., $$
and the right adjoint sends an $\io$-category $C$ to the sequence $(\tau_nC,\tau_nC\sim \tau_{n}\tau_{n+1}C)_{n:\Nb}$. Indeed, we have equivalence
$$
\begin{array}{rcl}
\Hom(\colim_{n:\Nb}i_n C_n,D)&\sim& \lim_{n:\Nb}\Hom(C_n,\tau_n D)
\\&\sim& \Hom( (C_n, C_n\sim \tau_nC_{n+1})_{n:\Nb},(\tau_n D,\tau_n D\sim \tau_{n}\tau_{n+1}D)_{n:\Nb})
\end{array}$$
natural in $(C_n, C_n\sim \tau_nC_{n+1})_{n:\Nb}$ and $D$.

\begin{prop}
\label{prop:infini omega a limit of infini n}
The adjunction \eqref{eq:inductivity} is an adjoint equivalence. As a consequence, we have an equivalence
$$\ocat\sim \lim_{n:\Nb}\ncat{n}.$$
\end{prop}
\begin{proof}
According to proposition \ref{prop:example of a special colimit 2}, any sequence $(C_n)_{n:\Nb}:\lim_{n:\Nb}\ncat{n}$ has a special colimit.
Let $k$ be an integer. According to proposition \ref{prop:taun preserves special colimits}, this implies the equivalence
$$\tau_k(\colim_{n:\Nb}C_n) \sim \colim_{n:\Nb}(\tau_kC_n).$$
Furthermore, the sequence $(\tau_kC_n)_{n:\Nb}$ is constant after the rank $k$. We then have 
$$\tau_k\colim_{n:\Nb}C_n \sim \tau_k C_n.$$
This directly implies that the unit of the adjunction \eqref{eq:inductivity} is an equivalence. 

To conclude, one has to show that the right adjoint is conservative, i.e that a morphism $f$ is an equivalence if and only if for any $n$, $\tau_n f$ is an equivalence. This last statement is a direct consequence of proposition \ref{prop:equivalences detected on globes}.
\end{proof} 
\p
The following proposition states that the cartesian product preserves colimits in both variables. There exists then an internal hom functor that we denote by \wcnotation{$\uHom(\uvar,\uvar)$}{(hom@$\uHom(\uvar,\uvar)$}.

\begin{prop}
\label{prop:cartesian product preserves W}
The cartesian product in $\ocat$ preserves colimits in both variables.
\end{prop}
We first need several lemmas:

\begin{lemma}
\label{lemma:product of representable in preshaves on Delta Theta}
Let $a$, $b$ be two globular sums, and $n,m$ two integer. The colimit in $\iPsh{\Delta[\Theta]}$ of the diagram 
% https://q.uiver.app/#q=WzAsNSxbMSwxLCJbYVxcdGltZXMgYixbbl1cXHRpbWVzIFttXV0iXSxbMiwwLCJcXGNvcHJvZF97bFxcbGVxIG19W2FcXHRpbWVzIGIsW25dXFx0aW1lcyBcXHtsXFx9XSJdLFswLDAsIlxcY29wcm9kX3trXFxsZXEgbn1bYVxcdGltZXMgYixcXHtrXFx9XFx0aW1lcyBbbV1dIl0sWzAsMSwiXFxjb3Byb2Rfe2tcXGxlcSBufVtiLG1dIl0sWzIsMSwiXFxjb3Byb2Rfe2xcXGxlcSBtfVthLG5dIl0sWzIsM10sWzIsMF0sWzEsMF0sWzEsNF1d
\[\begin{tikzcd}
	{\coprod_{k\leq n}[a\times b,\{k\}\times [m]]} && {\coprod_{l\leq m}[a\times b,[n]\times \{l\}]} \\
	{\coprod_{k\leq n}[b,m]} & {[a\times b,[n]\times [m]]} & {\coprod_{l\leq m}[a,n]}
	\arrow[from=1-1, to=2-1]
	\arrow[from=1-1, to=2-2]
	\arrow[from=1-3, to=2-2]
	\arrow[from=1-3, to=2-3]
\end{tikzcd}\]
is $[a,n]\times [b,m]$.
\end{lemma}
\begin{proof}
The lemma \ref{lemma:colimit computed in set presheaves} implies that the object 
$$K:=\coprod_{k\leq n}[b,m]\coprod_{\coprod_{k\leq n}[a\times b,\{k\}\times [m]]}[a\times b,[n]\times [m]]$$
is strict. As the induced morphism 
$\coprod_{l\leq m}[a\times b,[n]\times \{l\}]\to K$, is a monomorphism, the lemma \textit{op cit} implies that the colimit of the diagram given in the statement is strict. We can then show the result in the category of set valued presheaves on $ \Delta[\Theta]$ and we leave this combinatorial exercise to the reader.
\end{proof}

\begin{lemma}
\label{lemma:technical cartesian product preserves W}
Let $f$ be a morphism of $\W_1$ and $n$ an integer. The morphism $f\times [n]$ is in $\widehat{\W_1}$.
\end{lemma}
\begin{proof}
Suppose first that $f$ is of shape $\Sp_m\to [m]$. Remark first that for any $k$, $[k]\times [m]$ is $\W_1$-local as both $[k]$ and $[m]$ are. We then have $\Fb_{\W_1}([k]\times[m])\sim [k]\times [m]$.
As the fibrant replacement preserves colimits and as the cartesian product in $\iun$-categories preserves colimits, we have a sequence of equivalences in $\icat$:
$$
\begin{array}{rcl}
\Fb_{\W_1}(\Sp_m\times [n])&\sim& \Fb_{\W_1}([1]\times [n])\coprod_{ \Fb_{\W_1}([0]\times [n])}...\coprod_{ \Fb_{\W_1}([0]\times [n])}\Fb_{\W_1}([1]\times [n])\\
&\sim& [1]\times [n]\coprod_{ [0]\times [n]}...\coprod_{ [0]\times [n]} [1]\times [n]\\
&\sim & [m]\times [n]
\end{array}
$$
By construction, the morphism $\Sp_m\times [n]\to \Fb_{\W_1}(\Sp_m\times [n])$ is in $\widehat{\W_1}$. We proceed similarly for the case $f:=E^{eq}\to [0]$.
\end{proof}

\begin{proof}[Proof of proposition \ref{prop:cartesian product preserves W}]
As the cartesian product on $\iPsh{\Theta}$ preserves colimits in both variables, according to corollary \ref{cor:derived colimit preserving functor}, we then have to show that for any globular sum $a$, and any $f\in\W$, $f\times a$ is in $\widehat{\W}$.

We demonstrate by induction on $k$ that for any $f\in\W_k$ and any globular sum $a$, $f\times a$ is in $\W_k$. The case $k=0$ is trivial as $\W_0$ is the singleton $\{id_{[0]}\}$.

Suppose then the statement is true at this stage $k$. We recall that we denote $(i_!,i^*)$ the left and right adjoints between $\iPsh{\Delta[\Theta]}$ and $\iPsh{\Theta}$. As $i^*$ preserves cartesian product, proposition \ref{prop:infini changing theta} implies that it is enough to show that for any $f\in\M_{k+1}$ and any object $[b,n]$, $f\times [b,n]$ is in $\widehat{\M}$. 

Suppose first that $f$ is of shape $[a,1]\to [c,1]$ for $a\to c \in \W_k$. According to lemma \ref{lemma:product of representable in preshaves on Delta Theta}, the morphism $f\times[b,m]$ is the colimit in depth of the diagram 
% https://q.uiver.app/#q=WzAsMTAsWzEsMSwiW2FcXHRpbWVzIGIsWzFdXFx0aW1lcyBbbV1dIl0sWzIsMCwiXFxjb3Byb2Rfe2xcXGxlcSBtfVthXFx0aW1lcyBiLFsxXVxcdGltZXMgXFx7bFxcfV0iXSxbMCwwLCJcXGNvcHJvZF97a1xcbGVxIDF9W2FcXHRpbWVzIGIsXFx7a1xcfVxcdGltZXMgW21dXSJdLFswLDEsIlxcY29wcm9kX3trXFxsZXEgMX1bYixtXSJdLFsyLDEsIlxcY29wcm9kX3tsXFxsZXEgbX1bYSwxXSJdLFsxLDMsIlxcY29wcm9kX3trXFxsZXEgMX1bYixtXSJdLFsxLDIsIlxcY29wcm9kX3trXFxsZXEgMX1bY1xcdGltZXMgYixcXHtrXFx9XFx0aW1lcyBbbV1dIl0sWzIsMywiW2NcXHRpbWVzIGIsWzFdXFx0aW1lcyBbbV1dIl0sWzMsMiwiXFxjb3Byb2Rfe2xcXGxlcSBtfVtjXFx0aW1lcyBiLFsxXVxcdGltZXMgXFx7bFxcfV0iXSxbMywzLCJcXGNvcHJvZF97bFxcbGVxIG19W2MsMV0iXSxbMiwzXSxbMiwwXSxbMSwwXSxbMSw0XSxbMyw1XSxbMiw2XSxbMCw3XSxbNCw5XSxbMSw4XSxbNiw3XSxbOCw3XSxbOCw5XSxbNiw1XV0=
\[\begin{tikzcd}[column sep =0.1cm]
	{\coprod_{k\leq 1}[a\times b,\{k\}\times [m]]} && {\coprod_{l\leq m}[a\times b,[1]\times \{l\}]} \\
	{\coprod_{k\leq 1}[b,m]} & {[a\times b,[1]\times [m]]} & {\coprod_{l\leq m}[a,1]} \\
	& {\coprod_{k\leq 1}[c\times b,\{k\}\times [m]]} && {\coprod_{l\leq m}[c\times b,[1]\times \{l\}]} \\
	& {\coprod_{k\leq 1}[b,m]} & {[c\times b,[1]\times [m]]} & {\coprod_{l\leq m}[c,1]}
	\arrow[from=1-1, to=2-1]
	\arrow[from=1-1, to=2-2]
	\arrow[from=1-3, to=2-2]
	\arrow[from=1-3, to=2-3]
	\arrow[from=2-1, to=4-2]
	\arrow[from=1-1, to=3-2]
	\arrow[from=2-2, to=4-3]
	\arrow[from=2-3, to=4-4]
	\arrow[from=1-3, to=3-4]
	\arrow[from=3-2, to=4-3]
	\arrow[from=3-4, to=4-3]
	\arrow[from=3-4, to=4-4]
	\arrow[from=3-2, to=4-2]
\end{tikzcd}\]
The lemma \ref{lemma:the functor [] preserves classes} and the induction hypothesis implies that all the depth morphisms are in $\widehat{M}$.
By stability by colimit, this implies that $f\times[b,m]$ belongs to $\widehat{\M}$.

Suppose now that $f$ is of shape $[a,\Sp_n]\to [a,n]$. According to lemma \ref{lemma:product of representable in preshaves on Delta Theta}, the morphism $f\times[b,m]$ is the colimit in depth of the diagram 
% https://q.uiver.app/#q=WzAsMTAsWzEsMSwiW2FcXHRpbWVzIGIsXFxTcF9uXFx0aW1lcyBbbV1dIl0sWzIsMCwiXFxjb3Byb2Rfe2xcXGxlcSBtfVthXFx0aW1lcyBiLFxcU3BfblxcdGltZXMgXFx7bFxcfV0iXSxbMCwwLCJcXGNvcHJvZF97a1xcbGVxIG59W2FcXHRpbWVzIGIsXFx7a1xcfVxcdGltZXMgW21dXSJdLFswLDEsIlxcY29wcm9kX3trXFxsZXEgbn1bYixtXSJdLFsyLDEsIlxcY29wcm9kX3tsXFxsZXEgbX1bYSxcXFNwX25dIl0sWzEsMywiXFxjb3Byb2Rfe2tcXGxlcSBufVtiLG1dIl0sWzEsMiwiXFxjb3Byb2Rfe2tcXGxlcSBufVthXFx0aW1lcyBiLFxce2tcXH1cXHRpbWVzIFttXV0iXSxbMiwzLCJbYVxcdGltZXMgYixbbl1cXHRpbWVzIFttXV0iXSxbMywyLCJcXGNvcHJvZF97bFxcbGVxIG19W2FcXHRpbWVzIGIsW25dXFx0aW1lcyBcXHtsXFx9XSJdLFszLDMsIlxcY29wcm9kX3tsXFxsZXEgbX1bYSxuXSJdLFsyLDNdLFsyLDBdLFsxLDBdLFsxLDRdLFs2LDddLFs0LDldLFsxLDhdLFs4LDldLFs4LDddLFs2LDVdLFszLDVdLFsyLDZdLFswLDddXQ==
\[\begin{tikzcd}[column sep = 0.1cm]
	{\coprod_{k\leq n}[a\times b,\{k\}\times [m]]} && {\coprod_{l\leq m}[a\times b,\Sp_n\times \{l\}]} \\
	{\coprod_{k\leq n}[b,m]} & {[a\times b,\Sp_n\times [m]]} & {\coprod_{l\leq m}[a,\Sp_n]} \\
	& {\coprod_{k\leq n}[a\times b,\{k\}\times [m]]} && {\coprod_{l\leq m}[a\times b,[n]\times \{l\}]} \\
	& {\coprod_{k\leq n}[b,m]} & {[a\times b,[n]\times [m]]} & {\coprod_{l\leq m}[a,n]}
	\arrow[from=1-1, to=2-1]
	\arrow[from=1-1, to=2-2]
	\arrow[from=1-3, to=2-2]
	\arrow[from=1-3, to=2-3]
	\arrow[from=3-2, to=4-3]
	\arrow[from=2-3, to=4-4]
	\arrow[from=1-3, to=3-4]
	\arrow[from=3-4, to=4-4]
	\arrow[from=3-4, to=4-3]
	\arrow[from=3-2, to=4-2]
	\arrow[from=2-1, to=4-2]
	\arrow[from=1-1, to=3-2]
	\arrow[from=2-2, to=4-3]
\end{tikzcd}\]
The lemma \ref{lemma:technical cartesian product preserves W} implies that $\Sp_n\times [m]\to [n]\times [m]$ is in $\widehat{\W_1}$. Combined with lemma \ref{lemma:the functor [] preserves classes}, this implies that all the morphisms in depth are in $\widehat{\M}$. By stability by colimit, so is $f\times[b,m]$.


It remains to show the case $f= E^{eq}\to [0]$. According to lemma \ref{lemma:product of representable in preshaves on Delta Theta}, the morphism $f\times[b,m]$ is the horizontal colimit of the diagram
% https://q.uiver.app/#q=WzAsNixbMiwxLCJbYixtXSJdLFsyLDAsIiBbYixFXntlcX1cXHRpbWVzIFttXV0iXSxbMSwwLCJcXGNvcHJvZF97a1xcbGVxIG19IFtiLEVee2VxfVxcdGltZXMgXFx7a1xcfV0iXSxbMCwwLCJcXGNvcHJvZF97a1xcbGVxIG19IEVee2VxfSJdLFsxLDEsIlxcY29wcm9kX3trXFxsZXEgbX1bMF0iXSxbMCwxLCJcXGNvcHJvZF97a1xcbGVxIG19WzBdIl0sWzIsMV0sWzIsM10sWzQsNV0sWzQsMF0sWzEsMF0sWzIsNF0sWzMsNV1d
\[\begin{tikzcd}
	{\coprod_{k\leq m} E^{eq}} & {\coprod_{k\leq m} [b,E^{eq}\times \{k\}]} & { [b,E^{eq}\times [m]]} \\
	{\coprod_{k\leq m}[0]} & {\coprod_{k\leq m}[0]} & {[b,m]}
	\arrow[from=1-2, to=1-3]
	\arrow[from=1-2, to=1-1]
	\arrow[from=2-2, to=2-1]
	\arrow[from=2-2, to=2-3]
	\arrow[from=1-3, to=2-3]
	\arrow[from=1-2, to=2-2]
	\arrow[from=1-1, to=2-1]
\end{tikzcd}\]
The lemma \ref{lemma:technical cartesian product preserves W} implies that $ E^{eq}\times [m]\to [m]$ is in $\widehat{\W_1}$. Combined with lemma \ref{lemma:the functor [] preserves classes}, this implies that all the vertical morphisms are in $\widehat{\M}$. By stability by colimit, so is $f\times[b,m]$.
\end{proof}

\begin{cor}
\label{cor:if codomain a groupoid, then f is exponentiable}
Let $C$ be an $\io$-category, $S$ an $\infty$-groupoid, and $f:C\to S$ any morphism.
The functor $f^*:\ocat_{/S}\to \ocat_{/C}$ preserves colimits. 
\end{cor}
\begin{proof}
As $\iPsh{\Theta}$ is locally cartesian closed, we just have to verify that for any cartesian squares:
% https://q.uiver.app/#q=WzAsNixbMiwwLCJDIl0sWzIsMSwiUyJdLFsxLDEsImIiXSxbMCwxLCJhIl0sWzAsMCwiQycnIl0sWzEsMCwiQyciXSxbMywyLCJpIiwyXSxbMCwxXSxbNCw1LCJqIl0sWzUsMF0sWzIsMV0sWzQsM10sWzUsMl0sWzQsMiwiIiwyLHsic3R5bGUiOnsibmFtZSI6ImNvcm5lciJ9fV0sWzUsMSwiIiwyLHsic3R5bGUiOnsibmFtZSI6ImNvcm5lciJ9fV1d
\[\begin{tikzcd}
	{C''} & {C'} & C \\
	a & b & S
	\arrow["i"', from=2-1, to=2-2]
	\arrow[from=1-3, to=2-3]
	\arrow["j", from=1-1, to=1-2]
	\arrow[from=1-2, to=1-3]
	\arrow[from=2-2, to=2-3]
	\arrow[from=1-1, to=2-1]
	\arrow[from=1-2, to=2-2]
	\arrow["\lrcorner"{anchor=center, pos=0.125}, draw=none, from=1-1, to=2-2]
	\arrow["\lrcorner"{anchor=center, pos=0.125}, draw=none, from=1-2, to=2-3]
\end{tikzcd}\]
if $i$ is in $\W$, then $j$ is in $\widehat{\W}$. Suppose given such cartesian squares. As $b$ is a globular form, $\tau^i_0(b)\sim 1$ and 
as $S$ is an $\infty$-groupoid, there exists an object $s$ of $S$ such that the morphism $b\to S$ factor through $\{s\}\to S$. If we denote by $C_s$ the fiber of $f$ in $\{s\}$, the morphisms $i$ and $j$ then fit in the following cartesian squares:
% https://q.uiver.app/#q=WzAsOCxbMiwwLCJDX3MiXSxbMiwxLCJcXHtzXFx9Il0sWzEsMSwiYiJdLFswLDEsImEiXSxbMCwwLCJDX3NcXHRpbWVzIGEiXSxbMSwwLCJDX3NcXHRpbWVzIGIiXSxbMywwLCJDIl0sWzMsMSwiUyJdLFszLDIsImkiLDJdLFswLDFdLFs0LDUsImoiXSxbNSwwXSxbMiwxXSxbNCwzXSxbNSwyXSxbNCwyLCIiLDIseyJzdHlsZSI6eyJuYW1lIjoiY29ybmVyIn19XSxbNSwxLCIiLDIseyJzdHlsZSI6eyJuYW1lIjoiY29ybmVyIn19XSxbMSw3XSxbMCw2XSxbNiw3XSxbMCw3LCIiLDEseyJzdHlsZSI6eyJuYW1lIjoiY29ybmVyIn19XV0=
\[\begin{tikzcd}
	{C_s\times a} & {C_s\times b} & {C_s} & C \\
	a & b & {\{s\}} & S
	\arrow["i"', from=2-1, to=2-2]
	\arrow[from=1-3, to=2-3]
	\arrow["j", from=1-1, to=1-2]
	\arrow[from=1-2, to=1-3]
	\arrow[from=2-2, to=2-3]
	\arrow[from=1-1, to=2-1]
	\arrow[from=1-2, to=2-2]
	\arrow["\lrcorner"{anchor=center, pos=0.125}, draw=none, from=1-1, to=2-2]
	\arrow["\lrcorner"{anchor=center, pos=0.125}, draw=none, from=1-2, to=2-3]
	\arrow[from=2-3, to=2-4]
	\arrow[from=1-3, to=1-4]
	\arrow[from=1-4, to=2-4]
	\arrow["\lrcorner"{anchor=center, pos=0.125}, draw=none, from=1-3, to=2-4]
\end{tikzcd}\]
The proposition \ref{prop:cartesian product preserves W} implies that $j$ verifies the desired property, which concludes the proof.
\end{proof}

The following proposition implies that a natural transformation is an equivalence if and only if it is pointwise one. 
\begin{prop}
\label{prop:cartesian square and times}
For any $\io$-categories $X$ and $C$, the following natural square is cartesian:
% q.uiver.app/#q=WzAsNCxbMCwwLCJcXHRhdV8wXFx1SG9tKFgsQykiXSxbMSwwLCJcXHVIb20oWCxDKSJdLFsxLDEsIlxcdUhvbShcXHRhdV8wWCxDKSJdLFswLDEsIlxcdUhvbShcXHRhdV8wWCxcXHRhdV8wQykiXSxbMCwzXSxbMSwyXSxbMywyXSxbMCwxXV0=
\[\begin{tikzcd}
	{\tau_0\uHom(X,C)} & {\uHom(X,C)} \\
	{\uHom(\tau_0X,\tau_0C)} & {\uHom(\tau_0X,C)}
	\arrow[from=1-1, to=2-1]
	\arrow[from=1-2, to=2-2]
	\arrow[from=2-1, to=2-2]
	\arrow[from=1-1, to=1-2]
\end{tikzcd}\]
\end{prop}
\begin{proof}
As $\uHom(\uvar,C)$ sends colimits to limits, we can suppose that $X$ is of shape $\Db_n$ for $n\geq 0$. Eventually, proposition \ref{prop:equivalences detected on globes} implies that pullbacks are detected on globes. We then have to show that for any integer $m$, the following square is cartesian:
% q.uiver.app/#q=WzAsNCxbMCwwLCJcXHRhdV8wXFx1SG9tKFxcRGJfbixDKSJdLFsxLDAsIlxcSG9tKFxcRGJfblxcdGltZXNcXERiX20sQykiXSxbMSwxLCJcXEhvbSgoXFx0YXVfMFxcRGJfbilcXHRpbWVzXFxEYl9tLEMpIl0sWzAsMSwiXFxIb20oXFx0YXVfMFxcRGJfblxcdGltZXNcXERiX20sXFx0YXVfMEMpIl0sWzAsM10sWzEsMl0sWzMsMl0sWzAsMV1d
$$
\begin{tikzcd}
	{\tau_0\uHom(\Db_n,C)} & {\Hom(\Db_n\times\Db_m,C)} \\
	{\Hom(\tau_0\Db_n\times\Db_m,\tau_0C)} & {\Hom((\tau_0\Db_n)\times\Db_m,C)}
	\arrow[from=1-1, to=2-1]
	\arrow[from=1-2, to=2-2]
	\arrow[from=2-1, to=2-2]
	\arrow[from=1-1, to=1-2]
\end{tikzcd}$$
To this extent, we claim that the following square is cocartesian in $\ocat$:
% q.uiver.app/#q=WzAsNCxbMSwwLCJcXERiX25cXHRpbWVzXFxEYl9tIl0sWzAsMCwiKFxcdGF1XzBcXERiX24pXFx0aW1lc1xcRGJfbSJdLFswLDEsIlxcdGF1XzBcXERiX24iXSxbMSwxLCJcXERiX24iXSxbMSwyXSxbMiwzXSxbMSwwXSxbMCwzXV0=
\begin{equation}
\label{eq:proof of cartesian}
\begin{tikzcd}
	{(\tau_0\Db_n)\times\Db_m} & {\Db_n\times\Db_m} \\
	{\tau_0\Db_n} & {\Db_n}
	\arrow[from=1-1, to=2-1]
	\arrow[from=2-1, to=2-2]
	\arrow[from=1-1, to=1-2]
	\arrow[from=1-2, to=2-2]
\end{tikzcd}
\end{equation}
Applying the functor $\uHom(\uvar,C)$ it will prove the desired property.
To show the cocartesianess of \eqref{eq:proof of cartesian}, remark that if either $n$ or $m$ is null, this is trivial. If not, proposition \ref{prop:example of a special colimit} states that $\Db_n\times\Db_m$ is the colimit of the span:
$$[\Db_{n-1},1]\vee[\Db_{m-1},1]\leftarrow [\Db_{n-1}\times \Db_{m-1},1]\to [\Db_{m-1},1]\vee[\Db_{n-1},1]$$
Using the two cartesian squares
% q.uiver.app/#q=WzAsOCxbMywwLCJbXFxEYl97bi0xfSwxXVxcdmVlW1xcRGJfe20tMX0sMV0iXSxbMiwwLCJbXFxEYl97bS0xfSwxXSJdLFsyLDEsIlswXSJdLFszLDEsIltcXERiX3tuLTF9LDFdIl0sWzEsMCwiW1xcRGJfe20tMX0sMV1cXHZlZVtcXERiX3tuLTF9LDFdIl0sWzEsMSwiW1xcRGJfe24tMX0sMV0iXSxbMCwwLCJbXFxEYl97bS0xfSwxXSJdLFswLDEsIlswXSJdLFsxLDJdLFsxLDBdLFsyLDNdLFswLDNdLFs0LDVdLFszLDEsIiIsMix7InN0eWxlIjp7Im5hbWUiOiJjb3JuZXIifX1dLFs2LDddLFs3LDVdLFs2LDRdLFs1LDYsIiIsMCx7InN0eWxlIjp7Im5hbWUiOiJjb3JuZXIifX1dXQ==
\[\begin{tikzcd}
	{[\Db_{m-1},1]} & {[\Db_{m-1},1]\vee[\Db_{n-1},1]} & {[\Db_{m-1},1]} & {[\Db_{n-1},1]\vee[\Db_{m-1},1]} \\
	{[0]} & {[\Db_{n-1},1]} & {[0]} & {[\Db_{n-1},1]}
	\arrow[from=1-3, to=2-3]
	\arrow[from=1-3, to=1-4]
	\arrow[from=2-3, to=2-4]
	\arrow[from=1-4, to=2-4]
	\arrow[from=1-2, to=2-2]
	\arrow["\lrcorner"{anchor=center, pos=0.125, rotate=180}, draw=none, from=2-4, to=1-3]
	\arrow[from=1-1, to=2-1]
	\arrow[from=2-1, to=2-2]
	\arrow[from=1-1, to=1-2]
	\arrow["\lrcorner"{anchor=center, pos=0.125, rotate=180}, draw=none, from=2-2, to=1-1]
\end{tikzcd}\]
this implies that the pushout of the upper span of \eqref{eq:proof of cartesian} is then the colimit of the diagram:
\begin{equation}
\label{eq:proof of cartesian2}
[\Db_{n-1},1]\leftarrow [\Db_{n-1}\times \Db_{m-1},1]\to [\Db_{n-1},1]
\end{equation}
The proposition \ref{prop:inteligent trucatio and a particular colimit} states that the square
% https://q.uiver.app/#q=WzAsNCxbMCwwLCIgXFxEYl97bS0xfSJdLFswLDEsIjEiXSxbMSwwLCIxIl0sWzEsMSwiMSJdLFsxLDNdLFswLDJdLFswLDFdLFsyLDNdXQ==
\[\begin{tikzcd}
	{ \Db_{m-1}} & 1 \\
	1 & 1
	\arrow[from=2-1, to=2-2]
	\arrow[from=1-1, to=1-2]
	\arrow[from=1-1, to=2-1]
	\arrow[from=1-2, to=2-2]
\end{tikzcd}\]
is cocartesian. Combined with proposition \ref{prop:cartesian product preserves W}, this implies that the square
% https://q.uiver.app/#q=WzAsNCxbMCwwLCJcXERiX3tuLTF9XFx0aW1lcyBcXERiX3ttLTF9Il0sWzAsMSwiXFxEYl97bi0xfSJdLFsxLDAsIlxcRGJfe24tMX0iXSxbMSwxLCJcXERiX3tuLTF9Il0sWzEsM10sWzAsMl0sWzAsMV0sWzIsM11d
\[\begin{tikzcd}
	{\Db_{n-1}\times \Db_{m-1}} & {\Db_{n-1}} \\
	{\Db_{n-1}} & {\Db_{n-1}}
	\arrow[from=2-1, to=2-2]
	\arrow[from=1-1, to=1-2]
	\arrow[from=1-1, to=2-1]
	\arrow[from=1-2, to=2-2]
\end{tikzcd}\]
is cocartesian. 
As a consequence, the colimit of the span \eqref{eq:proof of cartesian2}, and so of the upper span of \eqref{eq:proof of cartesian}, is $ [\Db_{n-1},1]\sim \Db_n$, which concludes the proof. 
\end{proof}



\p
\label{para:dualities non strict case}
In paragraph \ref{para:dualities strict case}, for any subset $S$ of $\Nb^*$, we have defined the duality $(\uvar)^S:\zocat\to \zocat$.
 These functors restrict to functors $\Theta\to \Theta$ that induce by extension by colimit functors \ssym{((b49@$(\uvar)^S$}{for $\io$-categories}
$$(\uvar)^S:\iPsh{\Theta}\to \iPsh{\Theta}$$
which are once again called \snotion{dualities}{for $\io$-categories}.
It is easy to see that this functor preserves $\io$-categories and then induces functors
$$(\uvar)^S:\ocat\to \ocat.$$

In particular, we have the \snotionsym{odd duality}{((b60@$(\uvar)^{op}$}{for $\io$-categories} $(\uvar)^{op}$, corresponding to the set of odd integer, the \snotionsym{even duality}{((b50@$(\uvar)^{co}$}{for $\io$-categories} $(\uvar)^{co}$, corresponding to the subset of non negative even integer, the \snotionsym{full duality}{((b80@$(\uvar)^{\circ}$}{for $\io$-categories} $(\uvar)^{\circ}$, corresponding to $\Nb^*$ and the \snotionsym{transposition}{((b70@$(\uvar)^t$}{for $\io$-categories} $(\uvar)^t$, corresponding to the singleton $\{1\}$. Eventually, we have equivalences
$$((\uvar)^{co})^{op}\sim (\uvar)^{\circ} \sim ((\uvar)^{op})^{co}.$$


\p A morphism $f:C\to D$ is an \notion{epimorphism} if it is in the smallest cocomplete $\infty$-groupoid of arrows of $\ocat$ that includes the codiagonal $\Db_n\coprod\Db_n\to \Db_n$ for any $n\geq 0$. A morphism is a \notion{monomorphism} if it has the unique right lifting property against epimorphisms.


 A morphism $i:C\to D$ is then a monomorphism if and only if for any $n$, $C_n\to D_n$ is a monomorphism.
The small object argument induces a factorization system:
\begin{equation}
\label{eq:epimonomorphism factorization}
C\to \im i\to D
\end{equation}
of any morphism $i:C\to D$, where the left map is an epimorphism, and the right one is a monomorphism. The object \wcnotation{$\im i$}{(im@$\im$} is called the \wcnotion{image of $i$}{image of a morphism}. We then have by construction the following result:

\begin{prop}
A morphism is an equivalence if and only if it is both a monomorphism and a epimorphism.
\end{prop}

\begin{prop}
\label{prop:the image is stable under cartesian product}
The image is stable under the cartesian product.
\end{prop}
\begin{proof}
One has to show that both epimorphisms and monomorphisms are stable under the functor $\uvar\times A$ for $A$ any $\io$-category. For monomorphisms, it is a direct consequence of the fact that this notion has been defined with a right lifting property. For epimorphisms, as $\uvar\times A$ commutes with colimit, we can reduce to show that for any $n$, 
$$(\Db_n\coprod\Db_n)\times A \sim \Db_n\times A\coprod\Db_n\times A\to \Db_n \times A$$
is an epimorphism. 
However, the $\infty$-groupoid of object $B$ such that 
$B\coprod B\to B$ is an epimorphism is closed by colimits and contains globes. This $\infty$-groupoid then contains all the object and so in particular $\Db_n \times A$.
\end{proof}

\begin{lemma}
\label{lemma:id is an epi}
For any integer $n$, the projection $\Ib:\Db_{n+1}\to \Db_n$ is an epimorphism. 
\end{lemma}
\begin{proof}
Remark first that we have a cocartesian square:
% q.uiver.app/#q=WzAsNCxbMCwxLCJcXHBhcnRpYWxcXERiX24iXSxbMCwwLCJcXHBhcnRpYWxcXERiX25cXGNvcHJvZCBcXHBhcnRpYWxcXERiX24iXSxbMSwwLCJcXERiX25cXGNvcHJvZFxcRGJfbiJdLFsxLDEsIlxccGFydGlhbFxcRGJfe24rMX0iXSxbMSwyXSxbMSwwXSxbMiwzXSxbMCwzXSxbMyw0LCIiLDEseyJsZXZlbCI6MSwic3R5bGUiOnsibmFtZSI6ImNvcm5lciJ9fV1d
\[\begin{tikzcd}
	{\partial\Db_n\coprod \partial\Db_n} & {\Db_n\coprod\Db_n} \\
	{\partial\Db_n} & {\partial\Db_{n+1}}
	\arrow[""{name=0, anchor=center, inner sep=0}, from=1-1, to=1-2]
	\arrow[from=1-1, to=2-1]
	\arrow[from=1-2, to=2-2]
	\arrow[from=2-1, to=2-2]
	\arrow["\lrcorner"{anchor=center, pos=0.125, rotate=180}, draw=none, from=2-2, to=0]
\end{tikzcd}\]
As the left hand morphism is an epimorphism, so is the right one. By stability by left cancellation, this implies that $\partial\Db_{n+1}\to \Db_n$ is an epimorphism.
Now, the map $\Ib$ can be factored as: 
% q.uiver.app/#q=WzAsNyxbMCwwLCJcXHBhcnRpYWxcXERiX3tuKzF9Il0sWzAsMSwiXFxEYl97bisxfSJdLFsxLDAsIlxcRGJfbiJdLFsxLDEsIlxcRGJfe24rMX1cXGNvcHJvZF97XFxwYXJ0aWFsXFxEYl97bisxfX1cXERiX24iXSxbMSwyLCJcXHBhcnRpYWxcXERiX3tuKzJ9Il0sWzIsMiwiXFxEYl97bisxfSJdLFsyLDEsIlxcRGJfbiJdLFswLDFdLFswLDJdLFsxLDNdLFsyLDNdLFs0LDNdLFszLDZdLFs0LDVdLFs2LDQsIiIsMSx7InN0eWxlIjp7Im5hbWUiOiJjb3JuZXIifX1dLFs1LDZdLFszLDgsIiIsMSx7ImxldmVsIjoxLCJzdHlsZSI6eyJuYW1lIjoiY29ybmVyIn19XV0=
\[\begin{tikzcd}
	{\partial\Db_{n+1}} & {\Db_n} \\
	{\Db_{n+1}} & {\Db_{n+1}\coprod_{\partial\Db_{n+1}}\Db_n} & {\Db_n} \\
	& {\partial\Db_{n+2}} & {\Db_{n+1}}
	\arrow[from=1-1, to=2-1]
	\arrow[""{name=0, anchor=center, inner sep=0}, from=1-1, to=1-2]
	\arrow[from=2-1, to=2-2]
	\arrow[from=1-2, to=2-2]
	\arrow[from=3-2, to=2-2]
	\arrow[from=2-2, to=2-3]
	\arrow[from=3-2, to=3-3]
	\arrow["\lrcorner"{anchor=center, pos=0.125, rotate=-90}, draw=none, from=2-3, to=3-2]
	\arrow[from=3-3, to=2-3]
	\arrow["\lrcorner"{anchor=center, pos=0.125, rotate=180}, draw=none, from=2-2, to=0]
\end{tikzcd}\]
which directly implies that $\Ib$ is an epimorphism. 
\end{proof}

\begin{prop}
\label{prop:intelignet truncation is poitwise an epi}
For any integer $n$, the canonical natural transformation $id\to \tau^i_n$ is pointwise an epimorphism. 
\end{prop}
\begin{proof}
This is a direct consequence of lemma \ref{lemma:id is an epi}.
\end{proof}

\begin{prop}
\label{prop:canonical epi}
For any integer $n$, any $(\infty,n)$-category $C$, and any $\io$-category $D$, the canonical morphisms
$$\alpha:\coprod_{C_n}\Db_n\to C~~~~~\beta:\coprod_{(n,D_n)}\Db_n\to D$$
are epimorphisms.
\end{prop}
\begin{proof}
Let $I$ be the image of $\alpha$. We are willing to show that the canonical morphism $j:I\to C$ is an equivalence.
According to lemma \ref{lemma:equivalence if unique right lifting property against globes.}, and as $j$ is a monomorphism, we have to show that $j$ has the (non unique) right lifting property against $\emptyset\to \Db_k$ for any $k\leq n$. It is sufficient to show that $\alpha$ has the (non unique) right lifting property against $\emptyset\to \Db_k$ for any $k\leq n$, which is obviously true. 
We proceed similarly for $\beta$.
\end{proof}

\begin{prop}
\label{prop:truncation of epimorphism is pushout}
Let $i:A\to B$ be an epimorphism and $n$ an integer. The canonical square 
% https://q.uiver.app/#q=WzAsNCxbMCwwLCJBIl0sWzEsMCwiQiJdLFswLDEsIlxcdGF1XmlfbihBKSJdLFsxLDEsIlxcdGF1XmlfbihCKSJdLFswLDJdLFsyLDMsIlxcdGF1XmlfbihpKSIsMl0sWzAsMSwiaSJdLFsxLDNdXQ==
\[\begin{tikzcd}
	A & B \\
	{\tau^i_n(A)} & {\tau^i_n(B)}
	\arrow[from=1-1, to=2-1]
	\arrow["{\tau^i_n(i)}"', from=2-1, to=2-2]
	\arrow["i", from=1-1, to=1-2]
	\arrow[from=1-2, to=2-2]
\end{tikzcd}\]
is cocartesian. 
\end{prop}
\begin{proof}
We can reduce to the case where $i$ is $\Db_k\coprod\Db_k\to \Db_k$. If $n\geq k$, it is directly true, and we then suppose $n<k$. In this case, the colimit of the span:
$$\Db_n\coprod \Db_n \leftarrow \Db_k\coprod\Db_k\to \Db_k$$
is $\Db_n\coprod_{\Db_k}\Db_n$. The proposition \ref{prop:inteligent trucatio and a particular colimit} implies that this pushout is $\Db_n$, which concludes the proof.
\end{proof}



\p
A functor $f:C\to D$ is \snotion{fully faithful}{for $\io$-categories} if for any pair of objects $a,b\in C$, the induced morphism 
$\hom_C(a,b)\to \hom_D(fa,fb)$ is an equivalence. 



\begin{prop}
\label{prop:ff 1}
A functor is fully faithful if and only if it has the unique right lifting property against $\{0\}\coprod \{1\}\to \Db_n$ for $n>0$.
\end{prop}
\begin{proof}
Let $f$ be a functor having the unique right lifting property against $\{0\}\coprod \{1\}\to \Db_n$ for $n>0$. As $[\emptyset,1] =\{0\}\coprod \{1\}$ and $[\Db_n,1] = \Db_{n+1}$, 
this is equivalent to asking for any pair of objects $c,d$ and for any integer $n$, that $f(c,d)$ has the unique right lifting property against $\emptyset\to \Db_n$, which in turn is equivalent to $f$ being fully faithful according to lemma \ref{lemma:equivalence if unique right lifting property against globes.}.
\end{proof}

\begin{prop}
\label{prop:ff 2}
Fully faithful functors are stable under limits.
\end{prop}
\begin{proof}
This is a consequence of the fact that fully faithful functors are characterized by unique right lifting properties.
\end{proof}

\begin{lemma}
\label{lemma:ff 2}
Let $p:C\to D$ be a fully faithful functor. The induced morphism $C_0\to D_0$ is a monomorphism.
\end{lemma}
\begin{proof}
To this extent, we have to show that $p:C\to D$ has the unique right lifting property against $1\coprod 1\to 1$. This is equivalent to show that $p$ has the unique right lifting property against $\iota: 1\coprod 1 \to E^{eq}$.

The proposition \ref{prop:ff 1} implies that $p$ as the unique right lifting property against $1\coprod 1\to \Db_1$ and $1\coprod 1\to \Db_2$
By left cancellation, this implies that  $p$ has the unique right lifting property against $\Db_2\to \Db_1$. As $\iota$ is a composition of pushouts along  $1\coprod 1\to \Db_1$ and  $\Db_2\to \Db_1$, this directly concludes the proof.
\end{proof}



\begin{prop}
\label{prop:fully faithful plus surjective on objet}
A morphism $f:C\to D$ is an equivalence if and only if it is fully faithful and induces a surjection on objects.
\end{prop}
\begin{proof}
This is necessary. Suppose that $f$ is fully faithful. According to 	\ref{prop:ff 1}, for any $n>0$, $f_n:C_n\to D_n$ is an equivalence. If $f$ induces a surjection on objects, lemma \ref{lemma:ff 2} implies that $f_0:C_0\to D_0$ is an equivalence. We can then apply proposition \ref{prop:equivalences detected on globes}.
\end{proof}






\subsection{Discrete Conduché functors}
\label{section:conduche}
\p
We denote \wcnotation{$\triangledown_{k,n}$}{(nabla@$\triangledown_{k,n}$} the unique globular morphism between $\Db_n$ and $\Db_n\coprod_{\Db_k}\Db_n$.
A morphism $f:C\to D$ between $\io$-categories is a \snotion{discrete Conduché functor}{for $\io$-categories} if it has the unique right lifting property against 
units $\Ib_{n+1}:\Db_{n+1}\to \Db_n$ for any integer $n$, and against compositions $\triangledown_{k,n}:\Db_n\to \Db_n\coprod_{\Db_k}\Db_n$ for any pair of integers $k\leq n$.
\begin{lemma}
\label{lemma:technicalconduche have the rlp aginst alebraic morphism}
The two following full sub $\infty$-groupoids of morphisms of $\ocat$ are equivalent: 
\begin{enumerate}
\item The smallest cocomplete full sub $\infty$-groupoid of morphisms containing the family of morphism $\{\Ib_{n+1}:\Db_{n+1}\to \Db_n,\}$ and the family $\{\triangledown_{k,n}:\Db_n\to \Db_n\coprod_{\Db_k}\Db_n\, ~k\leq n\}$.
\item The smallest cocomplete full sub $\infty$-groupoid of morphisms containing algebraic morphisms of $\Theta$ (this notion is defined in paragraph \ref{para:algebraic and globular}). 
\end{enumerate}
\end{lemma}
\begin{proof}
For any pair of integers $k\leq n$, $\Ib_{n+1}$ and $\triangledown_{k,n}$ are algebraic morphisms. This directly induces the inclusion of the fist $\infty$-groupoid in the second one. To conclude, one has to show that every algebraic morphism $i:a\to b$ is contained in the first $\infty$-groupoid.


We proceed by induction on $|a|+|b|$. Suppose first that there exists $n$ such that $a=\Db_n$. In this case two cases have to be considered. Either $n>0$ and $i$ factors as $\Db_n\xrightarrow{\Ib_n} \Db_{n-1}\xrightarrow{j} b$. The result then follows by the induction hypothesis. Suppose now that $i$ does not factor though $\Ib_n$. In this case, there exists $k$ such that $i$ factors as $\Db_n\xrightarrow{\triangledown_{k,n}} \Db_n\coprod_{\Db_k}\Db_n\xrightarrow{j} b$. The unique factorization system between algebraic and globular morphisms given in proposition \ref{prop:algebraic ortho to globular} produces a diagram 
% q.uiver.app/#q=WzAsOCxbMiwzLCIgXFxEYl9uXFxjb3Byb2Rfe1xcRGJfa31cXERiX24gIl0sWzAsMiwiXFxEYl9uIl0sWzIsMCwiXFxEYl9uIl0sWzUsMywiYiJdLFs1LDAsImJfMiJdLFszLDIsImJfMCJdLFsxLDEsIlxcRGJfayJdLFs0LDEsImJfMSJdLFswLDMsImoiLDFdLFsxLDAsIiIsMCx7InN0eWxlIjp7InRhaWwiOnsibmFtZSI6Imhvb2siLCJzaWRlIjoidG9wIn19fV0sWzIsMCwiIiwwLHsic3R5bGUiOnsidGFpbCI6eyJuYW1lIjoiaG9vayIsInNpZGUiOiJ0b3AifX19XSxbMiw0LCJqXzIiLDJdLFsxLDUsImpfMCJdLFs1LDMsIiIsMCx7InN0eWxlIjp7InRhaWwiOnsibmFtZSI6Imhvb2siLCJzaWRlIjoidG9wIn19fV0sWzQsMywiIiwwLHsic3R5bGUiOnsidGFpbCI6eyJuYW1lIjoiaG9vayIsInNpZGUiOiJ0b3AifX19XSxbNiwwLCIiLDAseyJzdHlsZSI6eyJ0YWlsIjp7Im5hbWUiOiJob29rIiwic2lkZSI6InRvcCJ9fX1dLFs2LDIsIiIsMSx7InN0eWxlIjp7InRhaWwiOnsibmFtZSI6Imhvb2siLCJzaWRlIjoidG9wIn19fV0sWzYsMSwiIiwxLHsic3R5bGUiOnsidGFpbCI6eyJuYW1lIjoiaG9vayIsInNpZGUiOiJib3R0b20ifX19XSxbNyw1LCIiLDEseyJzdHlsZSI6eyJ0YWlsIjp7Im5hbWUiOiJob29rIiwic2lkZSI6ImJvdHRvbSJ9fX1dLFs3LDQsIiIsMSx7InN0eWxlIjp7InRhaWwiOnsibmFtZSI6Imhvb2siLCJzaWRlIjoidG9wIn19fV0sWzcsMywiIiwwLHsic3R5bGUiOnsidGFpbCI6eyJuYW1lIjoiaG9vayIsInNpZGUiOiJ0b3AifX19XSxbNiw3LCJqXzEiLDFdXQ==
\[\begin{tikzcd}
	&& {\Db_n} &&& {b_2} \\
	& {\Db_k} &&& {b_1} \\
	{\Db_n} &&& {b_0} \\
	&& { \Db_n\coprod_{\Db_k}\Db_n } &&& b
	\arrow["j"{description}, from=4-3, to=4-6]
	\arrow[hook, from=3-1, to=4-3]
	\arrow[hook, from=1-3, to=4-3]
	\arrow["{j_2}"', from=1-3, to=1-6]
	\arrow["{j_0}", from=3-1, to=3-4]
	\arrow[hook, from=3-4, to=4-6]
	\arrow[hook, from=1-6, to=4-6]
	\arrow[hook, from=2-2, to=4-3]
	\arrow[hook, from=2-2, to=1-3]
	\arrow[hook', from=2-2, to=3-1]
	\arrow[hook', from=2-5, to=3-4]
	\arrow[hook, from=2-5, to=1-6]
	\arrow[hook, from=2-5, to=4-6]
	\arrow["{j_1}"{description}, from=2-2, to=2-5]
\end{tikzcd}\]
where arrows labeled by $\hookrightarrow$ are globular and the other ones are algebraic. Remark that we have a cocartesian square in $\iun$-category of arrows of $\ocat$:
% q.uiver.app/#q=WzAsNCxbMCwwLCJqXzEiXSxbMCwxLCJqXzAiXSxbMSwwLCJqXzIiXSxbMSwxLCJqIl0sWzAsMV0sWzAsMl0sWzEsM10sWzIsM11d
\[\begin{tikzcd}
	{j_1} & {j_2} \\
	{j_0} & j
	\arrow[from=1-1, to=2-1]
	\arrow[from=1-1, to=1-2]
	\arrow[from=2-1, to=2-2]
	\arrow[from=1-2, to=2-2]
\end{tikzcd}\]
is cocartesian. As $j_0$, $j_1$ and $j_2$ are in the first $\infty$-groupoid by induction hypothesis, so is $j$. By stability by composition, the morphism $i$ is then in the first $\infty$-groupoid.


Suppose now that the domain of $i:a\to b$ is not a globe. Using once again the unique factorization system between algebraic and globular, we can construct a functor $\Sp_a\to \Arr(\Theta)$ whose value on $\Db_n\hookrightarrow a$ is given by the unique algebraic morphism $j$ fitting in a commutative square
% q.uiver.app/#q=WzAsNCxbMCwxLCJhIl0sWzEsMSwiYiJdLFswLDAsIlxcRGJfbiJdLFsxLDAsImInIl0sWzIsMCwiIiwwLHsic3R5bGUiOnsidGFpbCI6eyJuYW1lIjoiaG9vayIsInNpZGUiOiJ0b3AifX19XSxbMywxLCIiLDAseyJzdHlsZSI6eyJ0YWlsIjp7Im5hbWUiOiJob29rIiwic2lkZSI6InRvcCJ9fX1dLFswLDEsImkiLDJdLFsyLDMsImoiXV0=
\[\begin{tikzcd}
	{\Db_n} & {b'} \\
	a & b
	\arrow[hook, from=1-1, to=2-1]
	\arrow[hook, from=1-2, to=2-2]
	\arrow["i"', from=2-1, to=2-2]
	\arrow["j", from=1-1, to=1-2]
\end{tikzcd}\]
where arrows labeled by $\hookrightarrow$ are globular. By induction hypothesis, $j$ is in the first $\infty$-groupoid. The colimit of 
$\Sp_a\to \Arr(\Theta)\to \Arr(\ocat)$ is then in the first $\infty$-groupoid. As this colimit is $i$, this concludes the proof.
\end{proof}

\begin{prop}
\label{prop:conduche have the rlp aginst alebraic morphism}
A morphism $f:X\to Y$ is a discrete Conduché functor if and only if it as the unique right lifting property against algebraic morphism of $\Theta$ (this notion is defined in paragraph \ref{para:algebraic and globular}). 
\end{prop}
\begin{proof}
Given a morphism $f$, the full sub $\infty$-groupoid of morphisms having the unique left lifting property against $f$ is cocomplete. The result is then a direct implication of lemma 
\ref{lemma:technicalconduche have the rlp aginst alebraic morphism}.
\end{proof}

\begin{example}
The proposition \ref{prop:algebraic ortho to globular} implies that a morphism $a\to b$ between globular sums is a discrete Conduché functor if and only if it is globular.
\end{example}

\begin{lemma}
\label{lemma:conduche technical}
Let $p:C\to a$ be discrete Conduché functor with $a$ a globular sum. We denote by $(\Theta_{/p})^{Cd}$ the full sub $\iun$-category of $\Theta_{/p}$ whose objects are triangles 
% q.uiver.app/#q=WzAsMyxbMCwwLCJiIl0sWzEsMCwiQyJdLFsxLDEsImEiXSxbMCwxXSxbMSwyLCJwIl0sWzAsMl1d
\[\begin{tikzcd}
	b & C \\
	& a
	\arrow[from=1-1, to=1-2]
	\arrow["p", from=1-2, to=2-2]
	\arrow[from=1-1, to=2-2]
\end{tikzcd}\]
where every arrow is a discrete Conduché functor.
The canonical inclusion of $\iun$-category $\iota:(\Theta_{/p})^{Cd}\to \Theta_{/p}$ is final.
\end{lemma}
\begin{proof}
To prove this statement, we will endow $\iota$ with a structure of right deformation retract. We then first build a right inverse of $\iota$.
Any triangle 
% q.uiver.app/#q=WzAsMyxbMCwwLCJiIl0sWzEsMCwiQyJdLFsxLDEsImEiXSxbMCwxXSxbMSwyLCJwIl0sWzAsMl1d
\[\begin{tikzcd}
	b & C \\
	& a
	\arrow[from=1-1, to=1-2]
	\arrow["p", from=1-2, to=2-2]
	\arrow[from=1-1, to=2-2]
\end{tikzcd}\]
induces a diagram of shape
% q.uiver.app/#q=WzAsNCxbMCwwLCJiIl0sWzEsMCwiQyJdLFsxLDEsImEiXSxbMCwxLCJiJyJdLFswLDFdLFswLDNdLFszLDJdLFszLDEsImwiXSxbMSwyLCJwIl1d
\[\begin{tikzcd}
	b & C \\
	{b'} & a
	\arrow[from=1-1, to=1-2]
	\arrow[from=1-1, to=2-1]
	\arrow[from=2-1, to=2-2]
	\arrow["l", from=2-1, to=1-2]
	\arrow["p", from=1-2, to=2-2]
\end{tikzcd}\]
where $b'$ is obtained in factorizing $b\to a$ in a algebraic morphism followed by a globular morphism, and $l$ comes from the unique right lifting property of $p$ against algebraic morphisms. By right cancellation, this implies that $l$ is a discrete Conduché functor.

 As these two operations are functorial, this defines a retraction $r: \Theta_{/p}\to (\Theta_{/p})^{Cd}$ sending the triangle spotted by $b,C$ and $a$ to the triangle spotted by $b',C$ and $a$. Moreover, this retraction comes along with a natural transformation $id\to r\iota$. As right deformation retracts are final, this concludes the proof.
\end{proof}


\begin{lemma}
\label{lemma:conduche preserves W}
Let $p:C\to D$ be a discrete Conduché functor. Then for any globular sums $a$, and any cartesian squares in $\iPsh{\Theta}$:
% q.uiver.app/#q=WzAsNixbMCwwLCJDJyciXSxbMiwwLCJDIl0sWzIsMSwiRCJdLFswLDEsIlxcU3BfYSJdLFsxLDEsImEiXSxbMSwwLCJDJyJdLFswLDMsInAnJyJdLFs1LDQsInAnIl0sWzEsMiwicCJdLFswLDUsImoiXSxbNSwxXSxbMyw0XSxbNCwyXSxbNSwyLCIiLDEseyJzdHlsZSI6eyJuYW1lIjoiY29ybmVyIn19XSxbMCw0LCIiLDEseyJzdHlsZSI6eyJuYW1lIjoiY29ybmVyIn19XV0=
\[\begin{tikzcd}
	{C''} & {C'} & C \\
	{\Sp_a} & a & D
	\arrow["{p''}", from=1-1, to=2-1]
	\arrow["{p'}", from=1-2, to=2-2]
	\arrow["p", from=1-3, to=2-3]
	\arrow["j", from=1-1, to=1-2]
	\arrow[from=1-2, to=1-3]
	\arrow[from=2-1, to=2-2]
	\arrow[from=2-2, to=2-3]
	\arrow["\lrcorner"{anchor=center, pos=0.125}, draw=none, from=1-2, to=2-3]
	\arrow["\lrcorner"{anchor=center, pos=0.125}, draw=none, from=1-1, to=2-2]
\end{tikzcd}\]
the morphism $j$ is in $\widehat{\Wseg}$.
\end{lemma}
\begin{proof}
By stability under pullback,
the morphism $p'$ is a discrete Conduché functor. 
Taking the notations of lemma \ref{lemma:conduche technical}, $p'$ is equivalent to $\colim_{(\Theta_{/p})^{Cd}}b\to a$ where this colimit is taken in $\iPsh{\Theta}_{/a}$. As $\iPsh{\Theta}$ is locally cartesian closed and as $\widehat{\W}$ is by definition closed by colimits, we can then reduce to the case where $p'$ is a discrete Conduché functor between globular sums, i.e a globular morphism $b\to a$.
In this case, the following canonical square is a pullback
% q.uiver.app/#q=WzAsNCxbMSwxLCJhIl0sWzEsMCwiYiJdLFswLDAsIlxcU3BfYiJdLFswLDEsIlxcU3BfYSJdLFszLDBdLFsxLDAsInAnIl0sWzIsM10sWzIsMV0sWzIsMCwiIiwxLHsic3R5bGUiOnsibmFtZSI6ImNvcm5lciJ9fV1d
\[\begin{tikzcd}
	{\Sp_b} & b \\
	{\Sp_a} & a
	\arrow[from=2-1, to=2-2]
	\arrow["{p'}", from=1-2, to=2-2]
	\arrow[from=1-1, to=2-1]
	\arrow[from=1-1, to=1-2]
	\arrow["\lrcorner"{anchor=center, pos=0.125}, draw=none, from=1-1, to=2-2]
\end{tikzcd}\]
and this concludes the proof.
\end{proof}



\begin{lemma}
\label{lemma:pulback of Wsat preresult}
Consider a cartesian square 
% q.uiver.app/#q=WzAsNCxbMCwwLCJYIl0sWzAsMSwiXFxTaWdtYV57bn0gRV57ZXF9Il0sWzEsMCwiWSJdLFsxLDEsIlxcRGJfe259Il0sWzAsMV0sWzAsMiwiaiJdLFsxLDNdLFsyLDNdLFswLDMsIiIsMix7InN0eWxlIjp7Im5hbWUiOiJjb3JuZXIifX1dXQ==
\[\begin{tikzcd}
	X & Y \\
	{\Sigma^{n} E^{eq}} & {\Db_{n}}
	\arrow[from=1-1, to=2-1]
	\arrow["j", from=1-1, to=1-2]
	\arrow[from=2-1, to=2-2]
	\arrow[from=1-2, to=2-2]
	\arrow["\lrcorner"{anchor=center, pos=0.125}, draw=none, from=1-1, to=2-2]
\end{tikzcd}\]
in $\iPsh{\Theta}$. The morphism $j$ is in $\widehat{\W}$.
\end{lemma}
\begin{proof}
 If we are in the case $n=0$, this directly follows from the preservation of $\W$ by cartesian product, demonstrated in the proof of proposition \ref{prop:cartesian product preserves W}.
We now suppose the result is true at stage $n$, and we first show that for any square
% https://q.uiver.app/#q=WzAsNCxbMCwwLCJYIl0sWzAsMSwiW1xcU2lnbWFee259IEVee2VxfSwxXSJdLFsxLDAsIlkiXSxbMSwxLCJbXFxEYl97bisxfSwxXSJdLFswLDFdLFswLDIsImoiXSxbMSwzXSxbMiwzLCJwIl0sWzAsMywiIiwyLHsic3R5bGUiOnsibmFtZSI6ImNvcm5lciJ9fV1d
\[\begin{tikzcd}
	X & Y \\
	{[\Sigma^{n} E^{eq},1]} & {[\Db_{n+1},1]}
	\arrow[from=1-1, to=2-1]
	\arrow["j", from=1-1, to=1-2]
	\arrow[from=2-1, to=2-2]
	\arrow["p", from=1-2, to=2-2]
	\arrow["\lrcorner"{anchor=center, pos=0.125}, draw=none, from=1-1, to=2-2]
\end{tikzcd}\]
in $\iPsh{\Delta[\Theta]}$, $j$ is in $\widehat{\M}$.
As $\iPsh{\Delta[\Theta]}$ is locally cartesian closed and $\widehat{\M}$ closed under colimits, one can suppose that $Y$ is of shape $[a,k]$ and we denote $f:[k]\to [1]$ the morphism induced by $p$. By stability under pullback, $X$ is then set-valued. Furthermore, we can then check in $\Psh{\Delta[\Theta]}$ that this presheaf fits in a cocartesian square:
% q.uiver.app/#q=WzAsNCxbMSwwLCJbXFxTaWdtYV5uRV57ZXF9XFx0aW1lc197XFxEYl9ufSBhLGtdIl0sWzAsMCwiW1xcU2lnbWFebkVee2VxfVxcdGltZXNfe1xcRGJfbn0gYSxmXnstMX0oMCldXFxjb3Byb2QgW1xcU2lnbWFebkVee2VxfVxcdGltZXNfe1xcRGJfbn1hLGZeey0xfSgxKV0iXSxbMCwxLCJbYSxmXnstMX0oMCldXFxjb3Byb2QgW2EsZl57LTF9KDEpXSJdLFsxLDEsIlgiXSxbMSwyXSxbMSwwXSxbMiwzXSxbMCwzXV0=
\[\begin{tikzcd}
	{[\Sigma^nE^{eq}\times_{\Db_n} a,f^{-1}(0)]\coprod [\Sigma^nE^{eq}\times_{\Db_n}a,f^{-1}(1)]} & {[\Sigma^nE^{eq}\times_{\Db_n} a,k]} \\
	{[a,f^{-1}(0)]\coprod [a,f^{-1}(1)]} & X
	\arrow[from=1-1, to=2-1]
	\arrow[from=1-1, to=1-2]
	\arrow[from=2-1, to=2-2]
	\arrow[from=1-2, to=2-2]
\end{tikzcd}\]
By induction hypothesis $[\Sigma^nE^{eq}\times_{\Db_n} a,l]\to [a,l]$ is in $\widehat{\M}$ for any integer $l$. As $X\to [a,k]$ is the colimit in depth of the diagram
% https://q.uiver.app/#q=WzAsMTAsWzEsMSwiW1xcU2lnbWFebkVee2VxfVxcdGltZXNfe1xcRGJfbn0gYSxrXSJdLFswLDAsIltcXFNpZ21hXm5FXntlcX1cXHRpbWVzX3tcXERiX259IGEsZl57LTF9KDApXSJdLFswLDEsIlthLGZeey0xfSgwKV0iXSxbMiwwLCIgW1xcU2lnbWFebkVee2VxfVxcdGltZXNfe1xcRGJfbn1hLGZeey0xfSgxKV0iXSxbMiwxLCJbYSxmXnstMX0oMSldIl0sWzEsMywiW2EsZl57LTF9KDApXSJdLFsxLDIsIlthLGZeey0xfSgwKV0iXSxbMiwzLCJbYSxrXSJdLFszLDIsIlthLGZeey0xfSgxKV0iXSxbMywzLCJbYSxmXnstMX0oMSldIl0sWzcsNl0sWzYsNV0sWzEsMl0sWzEsMF0sWzMsMF0sWzMsNF0sWzgsN10sWzgsOV0sWzIsNV0sWzEsNl0sWzAsN10sWzQsOV0sWzMsOF1d
\[\begin{tikzcd}[column sep = 0.7cm]
	{[\Sigma^nE^{eq}\times_{\Db_n} a,f^{-1}(0)]} && { [\Sigma^nE^{eq}\times_{\Db_n}a,f^{-1}(1)]} \\
	{[a,f^{-1}(0)]} & {[\Sigma^nE^{eq}\times_{\Db_n} a,k]} & {[a,f^{-1}(1)]} \\
	& {[a,f^{-1}(0)]} && {[a,f^{-1}(1)]} \\
	& {[a,f^{-1}(0)]} & {[a,k]} & {[a,f^{-1}(1)]}
	\arrow[from=4-3, to=3-2]
	\arrow[from=3-2, to=4-2]
	\arrow[from=1-1, to=2-1]
	\arrow[from=1-1, to=2-2]
	\arrow[from=1-3, to=2-2]
	\arrow[from=1-3, to=2-3]
	\arrow[from=3-4, to=4-3]
	\arrow[from=3-4, to=4-4]
	\arrow[from=2-1, to=4-2]
	\arrow[from=1-1, to=3-2]
	\arrow[from=2-2, to=4-3]
	\arrow[from=2-3, to=4-4]
	\arrow[from=1-3, to=3-4]
\end{tikzcd}\]
this implies that this morphism is in $\widehat{\M}$.

We now return to $\infty$-presheaves on $\Theta$. We recall that we denote by $(i_!,i^*)$ the adjunction between $\iPsh{\Delta[\Theta]}$ and $\iPsh{\Theta}$. Suppose given a cartesian square:
% q.uiver.app/#q=WzAsNCxbMCwwLCJYIl0sWzAsMSwiXFxTaWdtYV57bisxfSBFXntlcX0iXSxbMSwwLCJZIl0sWzEsMSwiXFxEYl97bisxfSJdLFswLDFdLFswLDIsImoiXSxbMSwzXSxbMiwzXSxbMCwzLCIiLDIseyJzdHlsZSI6eyJuYW1lIjoiY29ybmVyIn19XV0=
\[\begin{tikzcd}
	X & Y \\
	{\Sigma^{n+1} E^{eq}} & {\Db_{n+1}}
	\arrow[from=1-1, to=2-1]
	\arrow["j", from=1-1, to=1-2]
	\arrow[from=2-1, to=2-2]
	\arrow[from=1-2, to=2-2]
	\arrow["\lrcorner"{anchor=center, pos=0.125}, draw=none, from=1-1, to=2-2]
\end{tikzcd}\]
This induces two squares
% https://q.uiver.app/#q=WzAsOCxbMiwxLCJYIl0sWzMsMSwiWSJdLFsyLDAsImlfIWleKlgiXSxbMywwLCJpXyFpXipZIl0sWzAsMCwiaV4qWCJdLFsxLDAsImleKlkiXSxbMCwxLCJbXFxTaWdtYV57bn1FXntlcX0sMV0iXSxbMSwxLCJbXFxEYl9uLDFdIl0sWzAsMV0sWzIsMF0sWzMsMV0sWzIsMywiaV8haV4qaiJdLFs0LDZdLFs1LDddLFs2LDddLFs0LDUsImleKmoiXSxbNCw3LCIiLDEseyJzdHlsZSI6eyJuYW1lIjoiY29ybmVyIn19XV0=
\[\begin{tikzcd}
	{i^*X} & {i^*Y} & {i_!i^*X} & {i_!i^*Y} \\
	{[\Sigma^{n}E^{eq},1]} & {[\Db_n,1]} & X & Y
	\arrow[from=2-3, to=2-4]
	\arrow[from=1-3, to=2-3]
	\arrow[from=1-4, to=2-4]
	\arrow["{i_!i^*j}", from=1-3, to=1-4]
	\arrow[from=1-1, to=2-1]
	\arrow[from=1-2, to=2-2]
	\arrow[from=2-1, to=2-2]
	\arrow["{i^*j}", from=1-1, to=1-2]
	\arrow["\lrcorner"{anchor=center, pos=0.125}, draw=none, from=1-1, to=2-2]
\end{tikzcd}\]
Where the cartesianess of the left square comes from the fact that $i^*$ preserves cartesian squares as it is a right adjoint. We just have demonstrated that $i^*j$ is in $\widehat{\M}$. Using proposition \ref{prop:infini changing theta}, and by left cancellation, the right square implies that $j$ is in $\widehat{W}$, which concludes the proof.
\end{proof}

\begin{prop}
\label{prop:pulback of Wsat}
Let $p:C\to D$ be a functor between $\io$-categories. Then for any globular sums $a$, and any cartesian squares in $\iPsh{\Theta}$:
% https://q.uiver.app/#q=WzAsNixbMCwwLCJDJyciXSxbMiwwLCJDIl0sWzIsMSwiRCJdLFswLDEsIlxcU2lnbWFebkVee2VxfSJdLFsxLDEsIlxcRGJfbiJdLFsxLDAsIkMnIl0sWzEsMiwicCJdLFswLDUsImoiXSxbNSwxXSxbMyw0XSxbNCwyXSxbNSwyLCIiLDEseyJzdHlsZSI6eyJuYW1lIjoiY29ybmVyIn19XSxbMCw0LCIiLDEseyJzdHlsZSI6eyJuYW1lIjoiY29ybmVyIn19XSxbMCwzXSxbNSw0XV0=
\[\begin{tikzcd}
	{C''} & {C'} & C \\
	{\Sigma^nE^{eq}} & {\Db_n} & D
	\arrow["p", from=1-3, to=2-3]
	\arrow["j", from=1-1, to=1-2]
	\arrow[from=1-2, to=1-3]
	\arrow[from=2-1, to=2-2]
	\arrow[from=2-2, to=2-3]
	\arrow["\lrcorner"{anchor=center, pos=0.125}, draw=none, from=1-2, to=2-3]
	\arrow["\lrcorner"{anchor=center, pos=0.125}, draw=none, from=1-1, to=2-2]
	\arrow[from=1-1, to=2-1]
	\arrow[from=1-2, to=2-2]
\end{tikzcd}\]
the morphism $j$ is in $\widehat{\W}$.
\end{prop}
\begin{proof}
This is a direct consequence of lemma \ref{lemma:pulback of Wsat preresult}.
\end{proof}
\begin{theorem}
\label{theo:pullback along conduche preserves colimits}
Let $f:C\to D$ be a discrete Conduché functor. The pullback functor $f^*:\ocat_{/D}\to \ocat_{/C}$ preserves colimits.
\end{theorem}
\begin{proof}
As $\iPsh{\Theta}$ is locally cartesian closed, we can use the corollary \ref{cor:derived colimit preserving functor}. The hypotheses are provided by lemmas \ref{lemma:conduche preserves W} and proposition \ref{prop:pulback of Wsat}.
\end{proof}


\section{Gray Operations}
\subsection{Gray operations on $\io$-categories}

Theorem \ref{theo:lecorozo} states that the $(\infty,1)$-category $\ocat$ is represented by the model category of marked simplicial sets given in proposition \ref{prop:model structure on marked simplicial set} and the functor $\N:\zocat\to \ocat$ corresponds to the Street nerve $\N:\ocat\to \mSset$.



An important feature of this model category is that it admits a monoidal structure $\otimes$ given by the \snotionsym{Gray tensor product}{((d00@$\otimes$}{for $\io$-categories}. Furthermore, proposition \ref{prop:gray_product_is_a_left_Quillen_bifunctor} ensures that this operation commutes with colimits in both variables. 
The induced functor 
$$\uvar\otimes[1]:\ocat\to \ocat$$
is called the \snotionsym{Gray cylinder}{((d30@$\uvar\otimes[1]$}{for $\io$-categories}.
We will show later, in corollary \ref{cor:otimes et op}, that we have a natural diagram
% q.uiver.app/#q=WzAsNixbMSwwLCIoQ1xcb3RpbWVzWzFdKV5cXGNpcmMiXSxbMiwwLCIoQ1xcb3RpbWVzXFx7MFxcfSleXFxjaXJjIl0sWzAsMCwiKENcXG90aW1lc1xcezFcXH0pXlxcY2lyYyJdLFsxLDEsIkNeXFxjaXJjXFxvdGltZXNbMV0iXSxbMiwxLCJDXlxcY2lyY1xcb3RpbWVzXFx7MVxcfSJdLFswLDEsIkNeXFxjaXJjXFxvdGltZXNcXHswXFx9Il0sWzAsMywiXFxzaW0iXSxbMSw0LCJcXHNpbSJdLFsyLDUsIlxcc2ltIl0sWzEsMF0sWzIsMF0sWzUsM10sWzQsM11d
\[\begin{tikzcd}
	{(C\otimes\{1\})^\circ} & {(C\otimes[1])^\circ} & {(C\otimes\{0\})^\circ} \\
	{C^\circ\otimes\{0\}} & {C^\circ\otimes[1]} & {C^\circ\otimes\{1\}}
	\arrow["\sim", from=1-2, to=2-2]
	\arrow["\sim", from=1-3, to=2-3]
	\arrow["\sim", from=1-1, to=2-1]
	\arrow[from=1-3, to=1-2]
	\arrow[from=1-1, to=1-2]
	\arrow[from=2-1, to=2-2]
	\arrow[from=2-3, to=2-2]
\end{tikzcd}\]

We denote by 
$$\begin{array}{rcl}
\ocat&\to&\ocat\\
C&\mapsto &C^{[1]}
\end{array}$$
the right adjoint of the Gray cylinder.\sym{(c@$C^{[1]}$}


Eventually, recall that we have a natural transformation $C\otimes [1]\to [C,1]$ whose restriction to $C\otimes\{0\}$ (resp. to $C\otimes\{1\}$) is constant on $\{0\}$ (resp. on $\{1\}$), and such that the following induced square is cocartesian:
\begin{equation}
\label{eq:liens entre Gray cylindre et suspension}
\begin{tikzcd}
	{C\otimes\{0,1\}} & {C\otimes [1]} \\
	{1\amalg 1} & {[C,1]}
	\arrow[from=1-1, to=2-1]
	\arrow[from=1-1, to=1-2]
	\arrow["\lrcorner"{anchor=center, pos=0.125, rotate=180}, draw=none, from=2-2, to=1-1]
	\arrow[from=2-1, to=2-2]
	\arrow[from=1-2, to=2-2]
\end{tikzcd}
\end{equation}


\p We define the \snotionsym{Gray cone}{((d40@$\uvar\star 1$}{for $\io$-categories} and the \snotion{Gray $\circ$-cone}{for $\io$-categories}\index[notation]{((d50@$1\overset{co}{\star}\_$!\textit{for $\io$-categories}}:
$$\begin{array}{ccccccc}
\ocat &\to&\ocat_{\bullet}&&\ocat &\to&\ocat_{\bullet}\\
C&\mapsto &C\star 1 & &C &\mapsto &1\costar C
\end{array}
$$
where $C\star 1$ and $1\costar C$ are defined as the following pushout: 
% q.uiver.app/#q=WzAsOCxbMSwwLCJDXFxvdGltZXMgWzFdIl0sWzAsMCwiQ1xcb3RpbWVzXFx7MVxcfSJdLFswLDEsIjEiXSxbMSwxLCJDXFxzdGFyIDEiXSxbMiwwLCJDXFxvdGltZXNcXHswXFx9Il0sWzMsMCwiQ1xcb3RpbWVzIFsxXSJdLFszLDEsIjFcXGNvc3RhciBDIl0sWzIsMSwiMSJdLFsxLDJdLFsxLDBdLFsyLDNdLFswLDNdLFszLDEsIiIsMSx7InN0eWxlIjp7Im5hbWUiOiJjb3JuZXIifX1dLFs0LDddLFs3LDZdLFs0LDVdLFs1LDZdLFs2LDQsIiIsMSx7InN0eWxlIjp7Im5hbWUiOiJjb3JuZXIifX1dXQ==
\[\begin{tikzcd}
	{C\otimes\{1\}} & {C\otimes [1]} & {C\otimes\{0\}} & {C\otimes [1]} \\
	1 & {C\star 1} & 1 & {1\costar C}
	\arrow[from=1-1, to=2-1]
	\arrow[from=1-1, to=1-2]
	\arrow[from=2-1, to=2-2]
	\arrow[from=1-2, to=2-2]
	\arrow["\lrcorner"{anchor=center, pos=0.125, rotate=180}, draw=none, from=2-2, to=1-1]
	\arrow[from=1-3, to=2-3]
	\arrow[from=2-3, to=2-4]
	\arrow[from=1-3, to=1-4]
	\arrow[from=1-4, to=2-4]
	\arrow["\lrcorner"{anchor=center, pos=0.125, rotate=180}, draw=none, from=2-4, to=1-3]
\end{tikzcd}\]
The corollary \ref{cor:otimes et op} will imply an
invertible natural transformation
$$ C\star 1\sim (1\costar C^{\circ})^\circ.$$

We will denote by 
$$\begin{array}{ccccccc}
\ocat_{\bullet} &\to& \ocat&&\ocat_{\bullet} &\to& \ocat\\
(C,c)&\mapsto &C_{/c} & &(C,c) &\mapsto &C_{c/}
\end{array}
$$
the right adjoints of the Gray cone and the Gray $\circ$-cone, respectively called the \wcsnotionsym{slice of $C$ over $c$}{(cc@$C_{/c}$}{slice over}{for $\io$-categories} and the \wcsnotionsym{slice of $C$ under $c$}{(cc@$C_{c/}$}{slice under}{for $\io$-categories}. 
The corollary \ref{cor:otimes et op} will imply an
invertible natural transformation
$$C_{/c}\sim (C^{\circ}_{c/})^\circ.$$
Given an $\io$-category $C$, and two objects $c,d$, we have by construction two cartesian squares:
% https://q.uiver.app/#q=WzAsOCxbMSwwLCJDX3svZH0iXSxbMCwxLCJcXHtjXFx9Il0sWzEsMSwiQyJdLFswLDAsIlxcaG9tX0MoYyxkKSJdLFszLDAsIkNfe2MvfSJdLFszLDEsIkMiXSxbMiwxLCJcXHtkXFx9Il0sWzIsMCwiXFxob21fQyhjLGQpIl0sWzEsMl0sWzMsMV0sWzMsMF0sWzAsMl0sWzcsNl0sWzcsNF0sWzYsNV0sWzQsNV1d
\[\begin{tikzcd}
	{\hom_C(c,d)} & {C_{/d}} & {\hom_C(c,d)} & {C_{c/}} \\
	{\{c\}} & C & {\{d\}} & C
	\arrow[from=2-1, to=2-2]
	\arrow[from=1-1, to=2-1]
	\arrow[from=1-1, to=1-2]
	\arrow[from=1-2, to=2-2]
	\arrow[from=1-3, to=2-3]
	\arrow[from=1-3, to=1-4]
	\arrow[from=2-3, to=2-4]
	\arrow[from=1-4, to=2-4]
\end{tikzcd}\]


\p 
As explained in section \ref{section:Street nerve}, the functor $\pi_0$ induces canonical equivalences
$$\pi_0(C\otimes[1])\cong \pi_0(C)\otimes[1]~~~\pi_0(C\star 1)\cong \pi_0(C)\star 1~~~\pi_0(1\costar C)\cong 1\costar \pi_0(C)$$
natural in $C$.
We will show in theorem \ref{theo:strictness} that the nerve $\N:\zocat\to \ocat$ also preserves the Gray operations.
As a consequence, we obtain the following examples of Gray operations:

\begin{example}
The $\io$-category $\Db_1\otimes[1]$ corresponds to the polygraph
% https://q.uiver.app/?q=WzAsNCxbMCwwLCIwMCJdLFswLDEsIjEwIl0sWzEsMSwiMTEiXSxbMSwwLCIwMSJdLFswLDFdLFsxLDJdLFswLDNdLFszLDJdLFszLDEsIiIsMSx7InNob3J0ZW4iOnsic291cmNlIjoyMCwidGFyZ2V0IjoyMH0sImxldmVsIjoyfV1d
\[\begin{tikzcd}
	00 & 01 \\
	10 & 11
	\arrow[from=1-1, to=2-1]
	\arrow[from=2-1, to=2-2]
	\arrow[from=1-1, to=1-2]
	\arrow[from=1-2, to=2-2]
	\arrow[shorten <=4pt, shorten >=4pt, Rightarrow, from=1-2, to=2-1]
\end{tikzcd}\]
The $\io$-category $\Db_2\otimes[1]$ corresponds to the polygraph
% https://q.uiver.app/?q=WzAsOCxbMSwwLCIwMSJdLFswLDAsIjAwIl0sWzAsMSwiMTAiXSxbMSwxLCIxMSJdLFsyLDAsIjAwIl0sWzMsMCwiMDEiXSxbMywxLCIxMSJdLFsyLDEsIjEwIl0sWzEsMF0sWzEsMl0sWzIsM10sWzAsM10sWzAsMiwiIiwxLHsic2hvcnRlbiI6eyJzb3VyY2UiOjIwLCJ0YXJnZXQiOjIwfSwibGV2ZWwiOjJ9XSxbNCw3XSxbNCw1XSxbNSw2XSxbNSw3LCIiLDEseyJzaG9ydGVuIjp7InNvdXJjZSI6MjAsInRhcmdldCI6MjB9LCJsZXZlbCI6Mn1dLFsxLDIsIiIsMix7ImN1cnZlIjo1fV0sWzcsNl0sWzUsNiwiIiwxLHsiY3VydmUiOi01fV0sWzksMTcsIiAiLDIseyJzaG9ydGVuIjp7InNvdXJjZSI6MjAsInRhcmdldCI6MjB9fV0sWzE5LDE1LCIgIiwyLHsic2hvcnRlbiI6eyJzb3VyY2UiOjIwLCJ0YXJnZXQiOjIwfX1dLFsxMSwxMywiIiwwLHsib2Zmc2V0IjotMSwic2hvcnRlbiI6eyJzb3VyY2UiOjIwLCJ0YXJnZXQiOjIwfSwibGV2ZWwiOjEsInN0eWxlIjp7ImhlYWQiOnsibmFtZSI6Im5vbmUifX19XSxbMTEsMTMsIiIsMix7Im9mZnNldCI6MSwic2hvcnRlbiI6eyJzb3VyY2UiOjIwLCJ0YXJnZXQiOjIwfSwibGV2ZWwiOjEsInN0eWxlIjp7ImhlYWQiOnsibmFtZSI6Im5vbmUifX19XSxbMTEsMTMsIiIsMSx7InNob3J0ZW4iOnsic291cmNlIjoyMCwidGFyZ2V0IjoyMH0sImxldmVsIjoxfV1d
\[\begin{tikzcd}
	00 & 01 & 00 & 01 \\
	10 & 11 & 10 & 11
	\arrow[from=1-1, to=1-2]
	\arrow[""{name=0, anchor=center, inner sep=0}, from=1-1, to=2-1]
	\arrow[from=2-1, to=2-2]
	\arrow[""{name=1, anchor=center, inner sep=0}, from=1-2, to=2-2]
	\arrow[shorten <=4pt, shorten >=4pt, Rightarrow, from=1-2, to=2-1]
	\arrow[""{name=2, anchor=center, inner sep=0}, from=1-3, to=2-3]
	\arrow[from=1-3, to=1-4]
	\arrow[""{name=3, anchor=center, inner sep=0}, from=1-4, to=2-4]
	\arrow[shorten <=4pt, shorten >=4pt, Rightarrow, from=1-4, to=2-3]
	\arrow[""{name=4, anchor=center, inner sep=0}, curve={height=30pt}, from=1-1, to=2-1]
	\arrow[from=2-3, to=2-4]
	\arrow[""{name=5, anchor=center, inner sep=0}, curve={height=-30pt}, from=1-4, to=2-4]
	\arrow["{ }"', shorten <=6pt, shorten >=6pt, Rightarrow, from=0, to=4]
	\arrow["{ }"', shorten <=6pt, shorten >=6pt, Rightarrow, from=5, to=3]
	\arrow[shift left=0.7, shorten <=6pt, shorten >=8pt, no head, from=1, to=2]
	\arrow[shift right=0.7, shorten <=6pt, shorten >=8pt, no head, from=1, to=2]
	\arrow[shorten <=6pt, shorten >=6pt, from=1, to=2]
\end{tikzcd}\]
\end{example}
\begin{example}
The $\io$-categories $\Db_1\star 1$ and $1\costar \Db_1$ correspond respectively to the polygraphs: 
% https://q.uiver.app/#q=WzAsNixbMCwwLCIwIl0sWzAsMSwiMSJdLFsxLDEsIlxcc3RhciJdLFszLDEsIlxcc3RhciJdLFs0LDAsIjAiXSxbNCwxLCIxIl0sWzAsMV0sWzEsMl0sWzAsMl0sWzQsNV0sWzMsNF0sWzMsNV0sWzgsMSwiIiwwLHsic2hvcnRlbiI6eyJzb3VyY2UiOjIwfX1dLFs5LDExLCIiLDAseyJvZmZzZXQiOjIsInNob3J0ZW4iOnsic291cmNlIjoyMCwidGFyZ2V0IjoyMH19XV0=
\[\begin{tikzcd}
	0 &&&& 0 \\
	1 & \star && \star & 1
	\arrow[from=1-1, to=2-1]
	\arrow[from=2-1, to=2-2]
	\arrow[""{name=0, anchor=center, inner sep=0}, from=1-1, to=2-2]
	\arrow[""{name=1, anchor=center, inner sep=0}, from=1-5, to=2-5]
	\arrow[from=2-4, to=1-5]
	\arrow[""{name=2, anchor=center, inner sep=0}, from=2-4, to=2-5]
	\arrow[shorten <=2pt, Rightarrow, from=0, to=2-1]
	\arrow[shift right=2, shorten <=4pt, shorten >=4pt, Rightarrow, from=1, to=2]
\end{tikzcd}\]
The $\io$-categories $\Db_2\star 1$ and $1\costar \Db_2$ correspond respectively to the polygraphs: 
% https://q.uiver.app/#q=WzAsMTQsWzAsMCwiMCJdLFswLDEsIjEiXSxbMSwxLCJcXHN0YXIiXSxbMiwwLCIwIl0sWzMsMSwiXFxzdGFyIl0sWzIsMSwiMSJdLFsxLDBdLFs1LDAsIjAiXSxbNCwxLCJcXHN0YXIiXSxbNSwxLCIxIl0sWzYsMSwiXFxzdGFyIl0sWzcsMCwiMCJdLFs3LDEsIjEiXSxbNiwwXSxbMCwxXSxbMSwyXSxbMyw1XSxbMCwxLCIiLDIseyJjdXJ2ZSI6NX1dLFs1LDRdLFswLDJdLFs2LDIsIiIsMCx7InN0eWxlIjp7ImJvZHkiOnsibmFtZSI6Im5vbmUifSwiaGVhZCI6eyJuYW1lIjoibm9uZSJ9fX1dLFszLDRdLFs3LDhdLFs3LDldLFs4LDldLFsxMSwxMF0sWzExLDEyXSxbMTIsMTBdLFsxMSwxMiwiIiwxLHsiY3VydmUiOi01fV0sWzEzLDEwLCIiLDIseyJzdHlsZSI6eyJib2R5Ijp7Im5hbWUiOiJub25lIn0sImhlYWQiOnsibmFtZSI6Im5vbmUifX19XSxbMTQsMTcsIiAiLDIseyJzaG9ydGVuIjp7InNvdXJjZSI6MjAsInRhcmdldCI6MjB9fV0sWzE5LDEsIiIsMSx7InNob3J0ZW4iOnsic291cmNlIjoyMCwidGFyZ2V0IjoyMH19XSxbMjAsMTYsIiIsMCx7Im9mZnNldCI6LTEsInNob3J0ZW4iOnsic291cmNlIjoyMCwidGFyZ2V0IjoyMH0sImxldmVsIjoxLCJzdHlsZSI6eyJoZWFkIjp7Im5hbWUiOiJub25lIn19fV0sWzIwLDE2LCIiLDIseyJvZmZzZXQiOjEsInNob3J0ZW4iOnsic291cmNlIjoyMCwidGFyZ2V0IjoyMH0sImxldmVsIjoxLCJzdHlsZSI6eyJoZWFkIjp7Im5hbWUiOiJub25lIn19fV0sWzIwLDE2LCIiLDEseyJzaG9ydGVuIjp7InNvdXJjZSI6MjAsInRhcmdldCI6MjB9LCJsZXZlbCI6MX1dLFsyMSw1LCIiLDAseyJzaG9ydGVuIjp7InNvdXJjZSI6MjB9fV0sWzI4LDI2LCIiLDEseyJzaG9ydGVuIjp7InNvdXJjZSI6MjAsInRhcmdldCI6MjB9fV0sWzI2LDI3LCIiLDEseyJvZmZzZXQiOjIsInNob3J0ZW4iOnsic291cmNlIjoyMCwidGFyZ2V0IjoyMH19XSxbMjMsMjQsIiIsMix7Im9mZnNldCI6Miwic2hvcnRlbiI6eyJzb3VyY2UiOjIwLCJ0YXJnZXQiOjIwfX1dLFsyMywyOSwiIiwyLHsib2Zmc2V0IjoxLCJzaG9ydGVuIjp7InNvdXJjZSI6MjAsInRhcmdldCI6MjB9LCJsZXZlbCI6MSwic3R5bGUiOnsiaGVhZCI6eyJuYW1lIjoibm9uZSJ9fX1dLFsyMywyOSwiIiwwLHsic2hvcnRlbiI6eyJzb3VyY2UiOjIwLCJ0YXJnZXQiOjIwfSwibGV2ZWwiOjF9XSxbMjMsMjksIiIsMix7Im9mZnNldCI6LTEsInNob3J0ZW4iOnsic291cmNlIjoyMCwidGFyZ2V0IjoyMH0sImxldmVsIjoxLCJzdHlsZSI6eyJoZWFkIjp7Im5hbWUiOiJub25lIn19fV1d
\[\begin{tikzcd}
	0 & {~} & 0 &&& 0 & {~} & 0 \\
	1 & \star & 1 & \star & \star & 1 & \star & 1
	\arrow[""{name=0, anchor=center, inner sep=0}, from=1-1, to=2-1]
	\arrow[from=2-1, to=2-2]
	\arrow[""{name=1, anchor=center, inner sep=0}, from=1-3, to=2-3]
	\arrow[""{name=2, anchor=center, inner sep=0}, curve={height=30pt}, from=1-1, to=2-1]
	\arrow[from=2-3, to=2-4]
	\arrow[""{name=3, anchor=center, inner sep=0}, from=1-1, to=2-2]
	\arrow[""{name=4, anchor=center, inner sep=0}, draw=none, from=1-2, to=2-2]
	\arrow[""{name=5, anchor=center, inner sep=0}, from=1-3, to=2-4]
	\arrow[from=1-6, to=2-5]
	\arrow[""{name=6, anchor=center, inner sep=0}, from=1-6, to=2-6]
	\arrow[""{name=7, anchor=center, inner sep=0}, from=2-5, to=2-6]
	\arrow[from=1-8, to=2-7]
	\arrow[""{name=8, anchor=center, inner sep=0}, from=1-8, to=2-8]
	\arrow[""{name=9, anchor=center, inner sep=0}, from=2-8, to=2-7]
	\arrow[""{name=10, anchor=center, inner sep=0}, curve={height=-30pt}, from=1-8, to=2-8]
	\arrow[""{name=11, anchor=center, inner sep=0}, draw=none, from=1-7, to=2-7]
	\arrow["{ }"', shorten <=6pt, shorten >=6pt, Rightarrow, from=0, to=2]
	\arrow[shorten <=2pt, shorten >=2pt, Rightarrow, from=3, to=2-1]
	\arrow[shift left=0.7, shorten <=6pt, shorten >=8pt, no head, from=4, to=1]
	\arrow[shift right=0.7, shorten <=6pt, shorten >=8pt, no head, from=4, to=1]
	\arrow[shorten <=6pt, shorten >=6pt, from=4, to=1]
	\arrow[shorten <=2pt, Rightarrow, from=5, to=2-3]
	\arrow[shorten <=6pt, shorten >=6pt, Rightarrow, from=10, to=8]
	\arrow[shift right=2, shorten <=4pt, shorten >=4pt, Rightarrow, from=8, to=9]
	\arrow[shift right=2, shorten <=4pt, shorten >=4pt, Rightarrow, from=6, to=7]
	\arrow[shift right=0.7, shorten <=6pt, shorten >=8pt, no head, from=6, to=11]
	\arrow[shorten <=6pt, shorten >=6pt, from=6, to=11]
	\arrow[shift left=0.7, shorten <=6pt, shorten >=8pt, no head, from=6, to=11]
\end{tikzcd}\]
\end{example}

\p 
\label{paragrap: equation fullfill by cylinder and join}
In section \ref{section:Suspension and Gray operation} are shown several equations fulfilled by the Gray cylinder, the Gray cone, and the Gray $\circ$-cone, that we recall here. For every $\io$-category $C$, there is a natural identification between $[C,1]\otimes [1]$ and the colimit of the following diagram
% q.uiver.app/#q=WzAsNSxbMCwwLCJbMV1cXHZlZSBbIEMsMV0iXSxbMSwwLCJbQ1xcb3RpbWVzXFx7MFxcfSwxXSJdLFsyLDAsIltDXFxvdGltZXMgWzFdLDFdIl0sWzQsMCwiW0MsMV1cXHZlZVsxXSJdLFszLDAsIltDXFxvdGltZXNcXHsxXFx9LDFdIl0sWzEsMF0sWzEsMl0sWzQsMl0sWzQsM11d
\begin{equation}
\label{eq:eq for cylinder}
\begin{tikzcd}
	{[1]\vee [ C,1]} & {[C\otimes\{0\},1]} & {[C\otimes [1],1]} & {[C\otimes\{1\},1]} & {[C,1]\vee[1]}
	\arrow[from=1-2, to=1-1]
	\arrow[from=1-2, to=1-3]
	\arrow[from=1-4, to=1-3]
	\arrow[from=1-4, to=1-5]
\end{tikzcd}
\end{equation}
 There is also a natural identification between
 $1\costar [C,1]$ and the colimit of the diagram
% q.uiver.app/#q=WzAsMyxbMiwwLCJbQ1xcc3RhciAxLDFdIl0sWzEsMCwiW0MsMV0iXSxbMCwwLCJbMV1cXHZlZSBbQywxXSJdLFsxLDBdLFsxLDJdXQ==
\begin{equation}
\label{eq:eq for Gray cone}
\begin{tikzcd}
	{[1]\vee [C,1]} & {[C,1]} & {[C\star 1,1]}
	\arrow[from=1-2, to=1-3]
	\arrow[from=1-2, to=1-1]
\end{tikzcd}
\end{equation}
and $[C,1] \star 1$ and the colimit of the diagram
% q.uiver.app/#q=WzAsMyxbMiwwLCJbQywxXVxcdmVlWzFdIl0sWzEsMCwiW0MsMV0iXSxbMCwwLCJbMVxcY29zdGFyIEMsMV0iXSxbMSwwXSxbMSwyXV0=
\begin{equation}
\label{eq:eq for cojoin}
\begin{tikzcd}
	{[1\costar C,1]} & {[C,1]} & {[C,1]\vee[1]}
	\arrow[from=1-2, to=1-3]
	\arrow[from=1-2, to=1-1]
\end{tikzcd}
\end{equation}
In each of the three previous diagrams, morphisms $[C,1]\to [1]\vee[C,1]$ and $[C,1]\to [C,1]\vee[1]$ are the unique ones preserving extremal points.

\begin{remark}
It is worth noticing the great similarity of these equations with the one given in theorems \ref{theo:appendice formula for otimes} and \ref{theo:appendice formula for star}
\end{remark}

\p
Let $C$ be an $\io$-category and $K$ a $(\infty,1)$-category.
There is a canonical morphism $C\otimes K\to C\times K$. In a way, one can see $C\times K$ as an intelligent truncated version of the Gray tensor product $C\otimes K$. We will make this intuition precise by constructing a hierarchy of Gray tensor products with $(\infty,1)$-categories. 
For $k\in \Nb\cup\{\omega\}$, we define the functor
$$\begin{array}{ccl}
\ocat \times \ncat{1}&\to &\ocat\\
(C,K)&\mapsto &C\otimes_k K
\end{array}$$
where $C\otimes_kK$ fits in the cocartesian square
% q.uiver.app/#q=WzAsNCxbMSwxLCJDXFxvdGltZXNfe2t9IEsiXSxbMCwwLCJcXGNvbGltX3tuXFxnZXEga30oXFx0YXVfbkMpXFxvdGltZXMgSyJdLFswLDEsIlxcY29saW1fe25cXGdlcSBrfVxcdGF1XmlfbigoXFx0YXVfbkMpXFxvdGltZXMgSykiXSxbMSwwLCJDXFxvdGltZXMgSyJdLFsxLDJdLFsxLDNdLFszLDBdLFsyLDBdLFswLDEsIiIsMSx7InN0eWxlIjp7Im5hbWUiOiJjb3JuZXIifX1dXQ==
\[\begin{tikzcd}
	{\colim_{n\geq k}(\tau_nC)\otimes K} & {C\otimes K} \\
	{\colim_{n\geq k}\tau^i_n((\tau_nC)\otimes K)} & {C\otimes_{k} K}
	\arrow[from=1-1, to=2-1]
	\arrow[from=1-1, to=1-2]
	\arrow[from=1-2, to=2-2]
	\arrow[from=2-1, to=2-2]
	\arrow["\lrcorner"{anchor=center, pos=0.125, rotate=180}, draw=none, from=2-2, to=1-1]
\end{tikzcd}\]

The induced functors $\uvar\otimes_k[1]:\ocat\to\ocat$ are called the \wcnotionsym{$k$-Gray cylinder}{((d10@$\otimes_n$}{Gray cylindera@$n$-Gray cylinder}.
Formula \eqref{eq:eq for cylinder} implies that for every $\io$-category $C$,
there is a natural identification between $[C,1]\otimes_{k+1} [1]$ and the colimit of the following diagram
% q.uiver.app/#q=WzAsNSxbMCwwLCJbMV1cXHZlZSBbIEMsMV0iXSxbMSwwLCJbQ1xcb3RpbWVzX2tcXHswXFx9LDFdIl0sWzIsMCwiW0NcXG90aW1lc19rIFsxXSwxXSJdLFs0LDAsIltDLDFdXFx2ZWVbMV0iXSxbMywwLCJbQ1xcb3Rpb
\begin{equation}
\label{eq:eq for k cylinder}
\begin{tikzcd}
	{[1]\vee [ C,1]} & {[C\otimes_k\{0\},1]} & {[C\otimes_k [1],1]} & {[C\otimes_k\{1\},1]} & {[C,1]\vee[1]}
	\arrow[from=1-2, to=1-1]
	\arrow[from=1-2, to=1-3]
	\arrow[from=1-4, to=1-3]
	\arrow[from=1-4, to=1-5]
\end{tikzcd}
\end{equation}
Remark that the endofunctor $\uvar\otimes_0[1]$ is the identity, 
the first assertion of lemma \ref{lemma:technique marked oicategoros} implies that the endofunctor $\uvar\otimes_1[1]$ is equivalent to $\uvar\times [1]$, and the endofunctor $\otimes_{\omega}[1]$ is just the normal Gray cylinder.

\begin{prop}
\label{prop:otimesk preserves colimits}
For any integer $k>0$, $\uvar\otimes_k[1]$ preserves colimits. 
\end{prop}
\begin{proof}
In order to simplify the notation, for a functor $F:\ocat\to \ocat$, the $\infty$-presheaves $\colim_{\Theta_{/\Sigma^nE^{eq}}}\iota F$, where $\iota$ in the inclusion $\ocat\to \iPsh{\Theta}$, will just be denoted by $F(\Sigma^nE^{eq})$.


As $\tau$ and $\tau^i$ preserves colimits in $\iPsh{\Theta}$ and $\widehat{\Wseg}$, and as $\uvar\otimes[1]$ preserves colimits, we just have to show that for any $n$, $(\Sigma^nE^{eq})\otimes_k[1]\to (\Sigma^n1)\otimes_k[1]$ is in $\widehat{\W}$. 

We then proceed by induction on $k$. The cases $k=0$ and $k=1$ are trivial as $\uvar\otimes_0[1]$ is the identity and $\uvar\otimes_1[1]$ is the tensor product with $[1]$.

Suppose the result is true at the stage $k$ for $k>1$. If $n=0$, remark that $E^{eq}\otimes_k[1]$ (resp. $ 1\otimes_k[1]$) is equivalent to $E^{eq}\otimes[1]$ (resp. $ 1\otimes[1]$) and the morphism is then in $\widehat{\W}$. Now, if $n>0$, formula \eqref{eq:eq for k cylinder} implies that $(\Sigma^nE^{eq})\otimes_k[1]\to (\Sigma^n1)\otimes_k[1]$ is the colimit in depth of the following diagram:
% q.uiver.app/#q=WzAsMTAsWzAsMSwiWzFdXFx2ZWUgWyBcXFNpZ21hXntuLTF9RV57ZXF9LDFdIl0sWzAsMCwiWyBcXFNpZ21hXntuLTF9RV57ZXF9XFxvdGltZXNfe2stMX1cXHswXFx9LDFdIl0sWzEsMSwiWyBcXFNpZ21hXntuLTF9RV57ZXF9XFxvdGltZXNfe2stMX1bMV0sMV0iXSxbMiwxLCJbIFxcU2lnbWFee24tMX1FXntlcX0sMV1cXHZlZVsxXSJdLFsyLDAsIlsgXFxTaWdtYV57bi0xfUVee2VxfVxcb3RpbWVzX3trLTF9XFx7MVxcfSwxXSJdLFsxLDMsIlsxXVxcdmVlIFsgXFxTaWdtYV57bi0xfTEsMV0iXSxbMSwyLCJbIFxcU2lnbWFee24tMX0xXFxvdGltZXNfe2stMX1cXHswXFx9LDFdIl0sWzIsMywiWyBcXFNpZ21hXntuLTF9MVxcb3RpbWVzX3trLTF9WzFdLDFdIl0sWzMsMiwiWyBcXFNpZ21hXntuLTF9MVxcb3RpbWVzX3trLTF9XFx7MVxcfSwxXSJdLFszLDMsIlsgXFxTaWdtYV57bi0xfTEsMV1cXHZlZVsxXSJdLFsxLDBdLFsxLDJdLFs0LDJdLFs0LDNdLFs2LDVdLFs2LDddLFs4LDddLFs4LDldLFsxLDZdLFsyLDddLFs0LDhdLFswLDVdLFszLDldXQ==
\[\begin{tikzcd}[column sep=0.2cm]
	{[ \Sigma^{n-1}E^{eq}\otimes_{k-1}\{0\},1]} && {[ \Sigma^{n-1}E^{eq}\otimes_{k-1}\{1\},1]} \\
	{[1]\vee [ \Sigma^{n-1}E^{eq},1]} & {[ \Sigma^{n-1}E^{eq}\otimes_{k-1}[1],1]} & {[ \Sigma^{n-1}E^{eq},1]\vee[1]} \\
	& {[ \Sigma^{n-1}1\otimes_{k-1}\{0\},1]} && {[ \Sigma^{n-1}1\otimes_{k-1}\{1\},1]} \\
	& {[1]\vee [ \Sigma^{n-1}1,1]} & {[ \Sigma^{n-1}1\otimes_{k-1}[1],1]} & {[ \Sigma^{n-1}1,1]\vee[1]}
	\arrow[from=1-1, to=2-1]
	\arrow[from=1-1, to=2-2]
	\arrow[from=1-3, to=2-2]
	\arrow[from=1-3, to=2-3]
	\arrow[from=3-2, to=4-2]
	\arrow[from=3-2, to=4-3]
	\arrow[from=3-4, to=4-3]
	\arrow[from=3-4, to=4-4]
	\arrow[from=1-1, to=3-2]
	\arrow[from=2-2, to=4-3]
	\arrow[from=1-3, to=3-4]
	\arrow[from=2-1, to=4-2]
	\arrow[from=2-3, to=4-4]
\end{tikzcd}\]
by induction hypothesis, and using lemma \ref{lemma:the functor [] preserves classes}, all the morphisms in depth are in $\widehat{\W}$, and so is their colimit.
\end{proof}
The functor $\uvar\otimes[1]_k$ then admits a right adjoint
$$(\uvar)^{[1]_k}:\ocat\to \ocat.$$





\p
We now describe a last operation that will play an essential role in the definition of lax colimit and lax limit. For any $C:\ocat$, we denote by \wcnotation{$m_C$}{(mc@$m_C$} the colimit preserving functor 
$\ocat\to\ocat$ whose value on a representable $[a,n]$ is $[a\times C,n]$. Remark that the assignation $C\mapsto m_C$ is natural in $C$ and that $m_1$ is the identity.
We define the colimit preserving functor: \ssym{((d20@$\ominus$}{for $\io$-categories}
$$\begin{array}{ccc}
\ocat\times\ocat &\to&\ocat\\
(X,Y)&\mapsto &X\ominus Y
\end{array}
$$
where for any $\io$-category $C$ and any element $[b,n]$ of $\Delta[\Theta]$, $X\ominus [b,n]$ is the following pushout: 
% q.uiver.app/#q=WzAsNCxbMSwwLCJtX2IoQ1xcb3RpbWVzW25dKSJdLFswLDAsIlxcY29wcm9kXFxsaW1pdHNfe2tcXGxlcSBufW1fYihDXFxvdGltZXNcXHtrXFx9KSJdLFswLDEsIlxcY29wcm9kXFxsaW1pdHNfe2tcXGxlcSBufW1fMShDXFxvdGltZXNcXHtrXFx9KSJdLFsxLDEsIkNcXG9taW51c1tiLG5dIl0sWzEsMl0sWzEsMF0sWzAsM10sWzIsM10sWzMsNSwiIiwxLHsibGV2ZWwiOjEsInN0eWxlIjp7Im5hbWUiOiJjb3JuZXIifX1dXQ==
\[\begin{tikzcd}
	{\coprod\limits_{k\leq n}m_b(C\otimes\{k\})} & {m_b(C\otimes[n])} \\
	{\coprod\limits_{k\leq n}m_1(C\otimes\{k\})} & {C\ominus[b,n]}
	\arrow[from=1-1, to=2-1]
	\arrow[""{name=0, anchor=center, inner sep=0}, from=1-1, to=1-2]
	\arrow[from=1-2, to=2-2]
	\arrow[from=2-1, to=2-2]
	\arrow["\lrcorner"{anchor=center, pos=0.125, rotate=180}, draw=none, from=2-2, to=0]
\end{tikzcd}\]
By construction, the functor $\uvar\ominus \uvar$ commutes with colimits in both variables.
We also have the identification $C\ominus [1]:=C\otimes [1]$.

Eventually, formula \eqref{eq:eq for cylinder} induces a natural identification between $[C,1]\ominus[b,1]$ and the colimit of the following diagram
% https://q.uiver.app/#q=WzAsNSxbMSwwLCJbQ1xcb3RpbWVzXFx7MFxcfVxcdGltZXMgYiwxXSJdLFs0LDAsIltDLDFdXFx2ZWVbYiwxXSJdLFswLDAsIltiLDFdXFx2ZWVbQywxXSJdLFszLDAsIltDXFxvdGltZXNcXHsxXFx9XFx0aW1lcyBiLDFdIl0sWzIsMCwiWyhDXFxvdGltZXNbMV0pXFx0aW1lcyBiKSwxXSJdLFswLDRdLFszLDRdLFszLDFdLFswLDJdXQ==
\begin{equation}
\label{eq:formula for the ominus}
\begin{tikzcd}[column sep = 0.3cm]
	{[b,1]\vee[C,1]} & {[C\otimes\{0\}\times b,1]} & {[(C\otimes[1])\times b),1]} & {[C\otimes\{1\}\times b,1]} & {[C,1]\vee[b,1]}
	\arrow[from=1-2, to=1-3]
	\arrow[from=1-4, to=1-3]
	\arrow[from=1-4, to=1-5]
	\arrow[from=1-2, to=1-1]
\end{tikzcd}
\end{equation}



\subsection{Gray deformation retract}
\label{subsection:Gray deformation retract}

\p
 A \wcnotion{left $k$-Gray deformation retract structure}{left or right $k$-Gray deformation retract structure} for a morphism $i:C\to D$ is the data of a \textit{retract}
 $r:D\to C$, a \textit{deformation} $\psi:D\otimes_k [1]\to D$, and equivalences
$$ri\sim id_C~~~~~\psi_{|D\otimes_k\{0\}}\sim ir~~~~~\psi_{|D\otimes_k\{1\}}\sim id_D~~~~~ \psi_{|C\otimes_k[1]}\sim i\cst_C
$$ 
A morphism $i:C\to D$ between $\io$-categories is a \wcnotion{left $k$-Gray deformation retract}{left or right $k$-Gray deformation retract} if it admits a left deformation retract structure. By abuse of language, such data will just be denoted by $(i,r,\psi)$.


We define dually the notion of \textit{right $k$-Gray deformation retract structure} and of \textit{right $k$-Gray deformation retract} in exchanging $0$ and $1$ in the previous definition.

\p
 A \textit{left $k$-Gray deformation retract structure} for a morphism $i:f\to g$ in the $\iun$-category of arrows of $\ocat$ is the data of a \textit{retract}
 $r:g\to f$, a \textit{deformation} $\psi:g\otimes_k [1]\to g$ and equivalences
$$ri\sim id_f~~~~~\psi_{|g\otimes_k\{0\}}\sim ir~~~~~\psi_{|g\otimes_k\{1\}}\sim id_D~~~~~ \psi_{|f\otimes_k[1]}\sim i\cst_C
$$ 
A morphism $i:C\to D$ between arrows of $\ocat$ is a \textit{left $k$-Gray deformation retract} if it admits a left deformation retract structure. By abuse of language, such data will just be denoted by $(i,r,\psi)$.


We define dually the notion of \textit{right $k$-Gray deformation retract structure} and of \textit{right $k$-Gray deformation retract} in exchanging $0$ and $1$ in the previous definition.




\begin{example}
\label{example:canonical example of left deformation retract unmarked}
Let $k\in \Nb\cup\{\omega\}$ and 
let $C$ be an $(\infty,k)$-category. We consider the morphism $i:C\otimes\{0\}\to C\otimes[1]$. We define $r:C\otimes[1]\xrightarrow{C\otimes\Ib} C\otimes\{0\}$. Eventually, we set 
$$\psi:C\otimes[1]\otimes[1]\to C\otimes([1]\times [1])\xrightarrow{C\otimes \phi}C\otimes[1]$$
where $\phi:[1]\times[1]$ is the morphism sending $(i,j)$ on the minimum of $i$ and $j$. 

As $C$ is an $(\infty,k)$-category, $\psi$ factors through $C\otimes[1]\to \tau^i_k(C\otimes[1])\sim C\otimes_k[1]$. We denote by $\phi:C\otimes_k[1]\to C\otimes\{0\}$ the induced morphism.
The triple $(i,r,\phi)$ is a left $k$-Gray deformation retract structure. Conversely, 
 $C\otimes\{1\}\to C\otimes[1]$ is a right deformation retract.
 
 
One can show similarly that $1\to 1\costar C$ is a left $k$-Gray deformation retract, and $1\to C\star 1$ is a right $k$-Gray deformation retract.
\end{example}




\p The $\infty$-groupoid of left and right Gray retracts enjoys many stability properties: 
\begin{prop}
\label{prop:left Gray deformation retract stable under pushout unmarked}
Let $(i_a,r_a,\psi_a)$ be a natural familly of left (resp. right) $k$-Gray deformation retract structures indexed by an $(\infty,1)$-category $A$.
The triple $(\colim_{A}i_a,\colim_{A}r_a,\colim_{A}\psi_a)$ is a left (resp. right) $k$-Gray deformation retract structure.
\end{prop}
\begin{proof}
This is an immediate consequence of the fact that $\uvar\otimes_k[1]$ preserves colimits.
\end{proof}
\begin{prop}
\label{prop:stability under pullback unmarked}
Suppose that we have a diagram 
% q.uiver.app/#q=WzAsNixbMCwwLCJYIl0sWzAsMSwiWCJdLFsxLDAsIlkiXSxbMSwxLCJZJyJdLFsyLDAsIloiXSxbMiwxLCJaJyJdLFswLDFdLFsyLDNdLFs0LDVdLFswLDIsInAiXSxbNCwyLCJxIiwyXSxbMSwzLCJwJyIsMl0sWzUsMywicSciXV0=
\[\begin{tikzcd}
	X & Y & Z \\
	X & {Y'} & {Z'}
	\arrow[from=1-1, to=2-1]
	\arrow[from=1-2, to=2-2]
	\arrow[from=1-3, to=2-3]
	\arrow["p", from=1-1, to=1-2]
	\arrow["q"', from=1-3, to=1-2]
	\arrow["{p'}"', from=2-1, to=2-2]
	\arrow["{q'}", from=2-3, to=2-2]
\end{tikzcd}\]
such that $p\to p'$ and $q\to q'$ are left (resp. right) $k$-Gray deformation retract. The induced square $q^*p\to (q')^*p'$ is a left (resp. right) $k$-Gray deformation retract.
\end{prop}
\begin{proof}
The proof is an easy diagram chasing.
\end{proof}
\begin{prop}
\label{prop:stability by composition unmarked}
If $p\to p'$ and $p'\to p''$ are two left (resp. right) $k$-Gray deformation retracts, so is $p\to p''$.
\end{prop}
\begin{proof}
The proof is an easy diagram chasing.
\end{proof}

\p The two following propositions show how the shifting of dimension preserves Gray transformation retract.

\begin{prop}
\label{prop:Gray deformation retract and passage to hom unmarked}
Let $(i:C\to D,r,\psi)$ be a left (resp. right) $(k+1)$-Gray deformation structure. For any $x: C$ and $y:D$ (resp. $x: D$ and $y:C$), the morphism
$$\begin{array}{cc}
&\hom_C(x,ry)\xrightarrow{i} \hom_D(ix,iry)\xrightarrow{{\psi_y}_!} \hom_D(ix,y)\\
(resp. &\hom_C(rx,y)\xrightarrow{i} \hom_D(irx,iy)\xrightarrow{{\psi_x}_!} \hom_D(x,iy))
\end{array}
$$
is a right (resp. left) $k$-Gray deformation retract, whose retract is given by 
$$\begin{array}{cc}
&\hom_D(ix,y)\xrightarrow{r}\hom_C(x,ry)\\
(resp. &\hom_D(x,iy)\xrightarrow{r}\hom_C(rx,y))
\end{array}$$
\end{prop}
\begin{proof}
By currying $\psi$, this induces a diagram
% q.uiver.app/#q=WzAsNSxbMCwxLCJEIl0sWzEsMCwiQyJdLFsyLDAsIkQiXSxbMiwyLCJEIl0sWzEsMSwiRF57WzFdX3trKzF9fSJdLFswLDEsInIiLDAseyJjdXJ2ZSI6LTF9XSxbMSwyLCJpIl0sWzAsMywiaWQiLDIseyJjdXJ2ZSI6Mn1dLFs0LDNdLFs0LDJdLFswLDQsIlxccHNpIiwxXV0=
\[\begin{tikzcd}
	& C & D \\
	D & {D^{[1]_{k+1}}} \\
	&& D
	\arrow["r", curve={height=-6pt}, from=2-1, to=1-2]
	\arrow["i", from=1-2, to=1-3]
	\arrow["id"', curve={height=12pt}, from=2-1, to=3-3]
	\arrow[from=2-2, to=3-3]
	\arrow[from=2-2, to=1-3]
	\arrow["\psi"{description}, from=2-1, to=2-2]
\end{tikzcd}\]
For any pair of objects $(z,y)$ of $D$, according to formula \eqref{eq:eq for k cylinder}, this induces a diagram
% q.uiver.app/#q=WzAsNSxbMCwxLCJcXGhvbV9EKHoseSkiXSxbMCwyLCJcXGhvbV9DKHJ6LHJ5KSJdLFsxLDIsIlxcaG9tX0QoaXJ6LGlyeSkiXSxbMSwwLCJcXGhvbV9EKHoseSkiXSxbMSwxLCJcXGhvbV9EKGlyeixpcnkpXFx0aW1lc197XFxob21fRChpcnoseSl9XFxob21fRChpcnoseSlee1sxXV9rfVxcdGltZXNfe1xcaG9tX0QoaXJ6LHkpfSBcXGhvbV9EKHoseSkiXSxbMCwxLCJyIl0sWzEsMiwiaSJdLFswLDMsImlkIiwwLHsiY3VydmUiOi0yfV0sWzQsM10sWzQsMl0sWzAsNCwiXFxwc2kiLDFdXQ==
\[\begin{tikzcd}
	& {\hom_D(z,y)} \\
	{\hom_D(z,y)} & {\hom_D(irz,iry)\times_{\hom_D(irz,y)}\hom_D(irz,y)^{[1]_k}\times_{\hom_D(irz,y)} \hom_D(z,y)} \\
	{\hom_C(rz,ry)} & {\hom_D(irz,iry)}
	\arrow["r", from=2-1, to=3-1]
	\arrow["i", from=3-1, to=3-2]
	\arrow["id", curve={height=-12pt}, from=2-1, to=1-2]
	\arrow[from=2-2, to=1-2]
	\arrow[from=2-2, to=3-2]
	\arrow["\psi"{description}, from=2-1, to=2-2]
\end{tikzcd}\]
If $z$ is of shape $ix$, the diagram becomes
% q.uiver.app/#q=WzAsOCxbMCwxLCJcXGhvbV9EKGl4LHkpIl0sWzAsMiwiXFxob21fQyh4LHJ5KSJdLFsxLDAsIlxcaG9tX0QoaXgseSkiXSxbMSwxLCJcXGhvbV9EKGl4LGlyeSlcXHRpbWVzX3tcXGhvbV9EKGl4LHkpfVxcaG9tX0QoaXgseSlee1sxXV9rfSJdLFsxLDIsIlxcaG9tX0QoaXgsaXJ5KSJdLFsyLDAsIlxcaG9tX0QoaXgseSkiXSxbMiwxLCJcXGhvbV9EKGl4LHkpXntbMV1fa30iXSxbMiwyLCJcXGhvbV9EKGl4LHkpIl0sWzAsMSwiciIsMl0sWzAsMiwiaWQiLDAseyJjdXJ2ZSI6LTJ9XSxbMywyXSxbMCwzLCJcXHBzaSIsMV0sWzEsNCwiaSIsMl0sWzMsNF0sWzQsNywie1xccHNpX3l9XyEiLDJdLFszLDZdLFsyLDUsImlkIl0sWzYsN10sWzYsNV0sWzMsMTQsIiIsMSx7ImxldmVsIjoxLCJzdHlsZSI6eyJuYW1lIjoiY29ybmVyIn19XV0=
\[\begin{tikzcd}
	& {\hom_D(ix,y)} & {\hom_D(ix,y)} \\
	{\hom_D(ix,y)} & {\hom_D(ix,iry)\times_{\hom_D(ix,y)}\hom_D(ix,y)^{[1]_k}} & {\hom_D(ix,y)^{[1]_k}} \\
	{\hom_C(x,ry)} & {\hom_D(ix,iry)} & {\hom_D(ix,y)}
	\arrow["r"', from=2-1, to=3-1]
	\arrow["id", curve={height=-12pt}, from=2-1, to=1-2]
	\arrow[from=2-2, to=1-2]
	\arrow["\psi"{description}, from=2-1, to=2-2]
	\arrow["i"', from=3-1, to=3-2]
	\arrow[from=2-2, to=3-2]
	\arrow[""{name=0, anchor=center, inner sep=0}, "{{\psi_y}_!}"', from=3-2, to=3-3]
	\arrow[from=2-2, to=2-3]
	\arrow["id", from=1-2, to=1-3]
	\arrow[from=2-3, to=3-3]
	\arrow[from=2-3, to=1-3]
	\arrow["\lrcorner"{anchor=center, pos=0.125}, draw=none, from=2-2, to=0]
\end{tikzcd}\]
By decurrying, this induces a morphism $\phi:\hom_D(ix,y)\otimes_k[1]\to \hom_D(ix,y)$. We leave the reader verify that the triple $({\psi_y}_!i,r,\phi)$ is a right $k$-Gray deformation retract structure.
We proceed similarly for the other case.
\end{proof}

\begin{prop}
\label{prop:Gray deformation retract and passage to hom v2 unmarked}
For any left (resp. right) $(k+1)$-Gray deformation retract between $p$ and $p'$:
% q.uiver.app/#q=WzAsNCxbMCwwLCJDIl0sWzAsMSwiQyciXSxbMSwwLCJEIl0sWzEsMSwiRCciXSxbMCwxLCJwIiwyXSxbMCwyLCJpIl0sWzIsMywicCciXSxbMSwzLCJpJyIsMl1d
\[\begin{tikzcd}
	C & D \\
	{C'} & {D'}
	\arrow["p"', from=1-1, to=2-1]
	\arrow["i", from=1-1, to=1-2]
	\arrow["{p'}", from=1-2, to=2-2]
	\arrow["{i'}"', from=2-1, to=2-2]
\end{tikzcd}\]
and for any pair of objects $x: C$ and $y:D$ (resp. $x: D$ and $y:C$), the outer square of the following diagram
% q.uiver.app/#q=WzAsNixbMCwwLCJcXGhvbV97Q30oeCxyeSkiXSxbMiwwLCJcXGhvbV97RH0oaXgseSkiXSxbMCwxLCJcXGhvbV97Qyd9KHB4LHByJ3kpIl0sWzEsMSwiXFxob21fe0QnfShwJ2kneCxwJ2kncid5KSJdLFsyLDEsIlxcaG9tX3tEJ30ocCdpJ3gscCd5KSJdLFsxLDAsIlxcaG9tX3tEfShpeCxpcnkpIl0sWzIsMywiaSciLDJdLFszLDQsIntcXHBzaSdfe3AneX19XyEiLDJdLFswLDJdLFsxLDRdLFs1LDEsIntcXHBzaV95fV8hIl0sWzAsNSwiaSJdLFs1LDNdXQ==
\[\begin{tikzcd}
	{\hom_{C}(x,ry)} & {\hom_{D}(ix,iry)} & {\hom_{D}(ix,y)} \\
	{\hom_{C'}(px,pr'y)} & {\hom_{D'}(p'i'x,p'i'r'y)} & {\hom_{D'}(p'i'x,p'y)}
	\arrow["{i'}"', from=2-1, to=2-2]
	\arrow["{{\psi'_{p'y}}_!}"', from=2-2, to=2-3]
	\arrow[from=1-1, to=2-1]
	\arrow[from=1-3, to=2-3]
	\arrow["{{\psi_y}_!}", from=1-2, to=1-3]
	\arrow["i", from=1-1, to=1-2]
	\arrow[from=1-2, to=2-2]
\end{tikzcd}\]
(resp.% q.uiver.app/#q=WzAsNixbMCwwLCJcXGhvbV97Q30ocngseSkiXSxbMiwwLCJcXGhvbV97RH0oeCxpeSkiXSxbMCwxLCJcXGhvbV97Qyd9KHByJ3gscHkpIl0sWzEsMSwiXFxob21fe0QnfShwJ2kncid4LHAnaSd5KSJdLFsyLDEsIlxcaG9tX3tEJ30ocCd4LHAnaSd5KVxcYmlnKSJdLFsxLDAsIlxcaG9tX3tEfShpcngsaXkpIl0sWzIsMywiaSciLDJdLFszLDQsIntcXHBzaSdfe3AneH19XyEiLDJdLFswLDJdLFsxLDRdLFs1LDEsIntcXHBzaV94fV8hIl0sWzAsNSwiaSJdLFs1LDNdXQ==
\[\begin{tikzcd}
	{\hom_{C}(rx,y)} & {\hom_{D}(irx,iy)} & {\hom_{D}(x,iy)} \\
	{\hom_{C'}(pr'x,py)} & {\hom_{D'}(p'i'r'x,p'i'y)} & {\hom_{D'}(p'x,p'i'y)\big)}
	\arrow["{i'}"', from=2-1, to=2-2]
	\arrow["{{\psi'_{p'x}}_!}"', from=2-2, to=2-3]
	\arrow[from=1-1, to=2-1]
	\arrow[from=1-3, to=2-3]
	\arrow["{{\psi_x}_!}", from=1-2, to=1-3]
	\arrow["i", from=1-1, to=1-2]
	\arrow[from=1-2, to=2-2]
\end{tikzcd}\]
is a left (resp. right) $(k+1)$-Gray deformation retract, whose retract is given by
% q.uiver.app/#q=WzAsOCxbMiwwLCIocmVzcC5cXGhvbV97RH0oeCxpeSkiXSxbMiwxLCJcXGhvbV97RCd9KHAneCxwJ2kneSkiXSxbMywwLCJcXGhvbV97Q30ocngseSkiXSxbMywxLCJcXGhvbV97Qyd9KHByJ3gscHkpXFxiaWcpIl0sWzAsMCwiXFxob21fe0R9KGl4LHkpIl0sWzEsMCwiXFxob21fe0N9KHgscnkpIl0sWzAsMSwiXFxob21fe0QnfShwJ2kneCxwJ3kpXFxiaWcpIl0sWzEsMSwiXFxob21fe0MnfShweCxwcid5KSJdLFswLDFdLFswLDIsInIiXSxbMSwzLCJyJyIsMl0sWzIsM10sWzYsNywiciciLDJdLFs0LDUsInIiXSxbNSw3XSxbNCw2XV0=
\[\begin{tikzcd}
	{\hom_{D}(ix,y)} & {\hom_{C}(x,ry)} & {(resp.\hom_{D}(x,iy)} & {\hom_{C}(rx,y)} \\
	{\hom_{D'}(p'i'x,p'y)\big)} & {\hom_{C'}(px,pr'y)} & {\hom_{D'}(p'x,p'i'y)} & {\hom_{C'}(pr'x,py)\big)}
	\arrow[from=1-3, to=2-3]
	\arrow["r", from=1-3, to=1-4]
	\arrow["{r'}"', from=2-3, to=2-4]
	\arrow[from=1-4, to=2-4]
	\arrow["{r'}"', from=2-1, to=2-2]
	\arrow["r", from=1-1, to=1-2]
	\arrow[from=1-2, to=2-2]
	\arrow[from=1-1, to=2-1]
\end{tikzcd}\]
\end{prop}
\begin{proof}
This comes from the fact that the construction of the retraction and the deformation in the previous proposition was functorial.
\end{proof}


\begin{prop}
\label{prop:suspension of left Gray deformation retract unmarked}
If $i$ is a left $k$-Gray deformation retract, $[i,1]$ is a right $(k+1)$-Gray deformation retract. Conversely, if $i$ is a right $k$-Gray deformation retract, $[i,1]$ is a left $(k+1)$-Gray deformation retract morphism.
\end{prop}
\begin{proof}
Let $(i:C\to D,r,\phi)$ be a left $k$-Gray deformation retract structure. We define the morphism $\psi:[D,1]\otimes_{k+1}[1]\to [D,1]$ as the horizontal colimit of the following diagram:
% q.uiver.app/#q=WzAsOCxbMCwwLCJbMV1ee31cXHZlZVtELDFdIl0sWzEsMCwiW0RcXG90aW1lc19rXFx7MFxcfSwxXSJdLFsyLDAsIltEXFxvdGltZXNfa1sxXSwxXSJdLFszLDAsIltEXFxvdGltZXNfa1xcezFcXH0sMV0iXSxbNCwwLCJbRCwxXVxcdmVlWzFdXnt9Il0sWzIsMSwiW0QsMV0iXSxbMSwxLCJbQywxXSJdLFszLDEsIltELDFdIl0sWzMsNF0sWzIsNSwiW1xccGhpLDFdIiwyXSxbMSwwXSxbMSwyXSxbMywyXSxbNiw1LCJbaSwxXSIsMl0sWzEsNiwiW3IsMV0iLDJdLFs3LDUsIltpZCwxXSJdLFszLDcsIltpZCwxXSJdLFswLDZdLFs0LDddXQ==
\[\begin{tikzcd}
	{[1]^{}\vee[D,1]} & {[D\otimes_k\{0\},1]} & {[D\otimes_k[1],1]} & {[D\otimes_k\{1\},1]} & {[D,1]\vee[1]^{}} \\
	& {[C,1]} & {[D,1]} & {[D,1]}
	\arrow[from=1-4, to=1-5]
	\arrow["{[\phi,1]}"', from=1-3, to=2-3]
	\arrow[from=1-2, to=1-1]
	\arrow[from=1-2, to=1-3]
	\arrow[from=1-4, to=1-3]
	\arrow["{[i,1]}"', from=2-2, to=2-3]
	\arrow["{[r,1]}"', from=1-2, to=2-2]
	\arrow["{[id,1]}", from=2-4, to=2-3]
	\arrow["{[id,1]}", from=1-4, to=2-4]
	\arrow[from=1-1, to=2-2]
	\arrow[from=1-5, to=2-4]
\end{tikzcd}\]
Eventually, remark that the triple $([i,1],[r,1],\psi)$ is a right $(k+1)$-Gray deformation retract. The other assertion is demonstrated similarly.
\end{proof}


\begin{prop}
\label{prop:of left Gray deformation retract unmarked}
For any integer $n$,
if $n$ is even, $i_{n}^-:\Db_{n}\to \Db_{n+1}$ is a left $n$-Gray deformation retract and $i_{n}^+:\Db_{n}\to \Db_{n+1}$ is a right $n$-Gray deformation retract, and if $n$ is odd, $i_{n}^-$ is a right $n$-Gray deformation retract and $i_{n}^+$ is a left $n$-Gray deformation retract.
\end{prop}
\begin{proof}
It is obvious that $\{0\}\to [1]$ is a left $1$-Gray deformation retract and $\{1\}\to [1]$ is a right $1$-Gray deformation retract. A repeated application of \ref{prop:suspension of left Gray deformation retract unmarked} proves the assertion.
\end{proof}



\begin{prop}
\label{prop:when glob inclusion are left Gray deformation unmarked}
Let $a$ be a globular sum of dimension $(n+1)$. We denote by $s_n(a)$ and $t_n(a)$ the globular sum defined in paragraph \ref{para:definition of source et but}. 

If $n$ is even, $s_n(a)\to a$ is a left $n$-Gray deformation retract and $t_n(a)\to a$ is a right $n$-Gray deformation retract, and if $n$ is odd, $s_n(a)\to a$ is a right $n$-Gray deformation retract and $t_n(a)\to a$ is a left $n$-Gray deformation retract.
\end{prop}
\begin{proof}
This is a direct consequence of proposition \ref{prop:of left Gray deformation retract unmarked} and \ref{prop:left Gray deformation retract stable under pushout unmarked} as $s_n(a)\to a$ is a composition of pushouts of $i_{n}^-:\Db_{n}\to (\Db_{n+1})_t$. The other assertion is proved similarly.
\end{proof}







\subsection{Gray operations and strict objects}

\label{section:on preservation of strict}
Recall that we have an adjunction
% q.uiver.app/#q=WzAsMixbMSwwLCJcXHpvY2F0OlxcTiJdLFswLDAsIlxccGlfMDpcXG9jYXQiXSxbMSwwLCIiLDAseyJvZmZzZXQiOi0yfV0sWzAsMSwiIiwwLHsib2Zmc2V0IjotMn1dLFsyLDMsIiIsMCx7ImxldmVsIjoxLCJzdHlsZSI6eyJuYW1lIjoiYWRqdW5jdGlvbiJ9fV1d
\[\begin{tikzcd}
	{\pi_0:\ocat} & {\zocat:\N}
	\arrow[""{name=0, anchor=center, inner sep=0}, shift left=2, from=1-1, to=1-2]
	\arrow[""{name=1, anchor=center, inner sep=0}, shift left=2, from=1-2, to=1-1]
	\arrow["\dashv"{anchor=center, rotate=-90}, draw=none, from=0, to=1]
\end{tikzcd}\]
An $\io$-category lying in the image of the nerve functor $\N$ is called strict. As explained in example 11 of \cite{Verity_weak_complicial_set_part2_nerve_of_complicial_Gray_categories}, $\pi_0$ preserves Gray tensor product, and so also the suspension, the Gray cone, and the Gray $\circ$-cone.

The strict categories play an important role as they allow us to make explicit calculations. In particular, it will be very useful to know which cocontinuous functors preserve them.
\begin{prop}
\label{prop:criter stricte easy}
An $\io$-category $C$ is strict if and only if $C_0$ is a set and for any pair of objects $x,y$, $\hom_C(x,y)$ is strict.
\end{prop}
\begin{proof}
By definition, an $\io$-category is strict if and only if, for any globular sum $[\textbf{b},n]$, $\Hom([\textbf{b},n],C)$ is a set. However, as $C$ is $\W$-local, we have an equivalence between $\Hom([\textbf{b},n],C)$ and 
$$\coprod_{x_0,x_1,...,x_n\in C_0}\Hom(b_1, \hom_C(x_0,x_1))\times...\times \Hom(b_n, \hom_C(x_{n-1},x_n))$$
As all the objects of the previous expression are set by hypothesis, and as the inclusion of set into $\infty$-groupoid is stable under coproduct and product, $\Hom([b,n],C)$ is a set.
\end{proof}

\begin{prop}
\label{prop:suspension preserves stricte}
If $C$ is a strict $\io$-category, so is $[C,1]$. 
\end{prop}
\begin{proof}
There is an obvious equivalence $[\N\uvar,1]\sim \N[\uvar,1]$ which directly implies the result.
\end{proof}





 
\begin{lemma}
\label{lemma:gray operation on globes are strict}
For any $n$, $\Db_n\otimes[1]$, $\Db_n\star 1$ and $1\costar \Db_n$ are strict.
\end{lemma}
\begin{proof}
We proceed by induction on $n$. The result is obviously true for $n=0$. Suppose it is true as the stage $n$. According to equation \eqref{eq:eq for cylinder}, $\Db_n\otimes[1]$ is the colimit of the following diagram
% q.uiver.app/#q=WzAsNSxbMCwwLCJbMV1cXHZlZSBcXERiX24iXSxbMSwwLCJcXERiX24iXSxbMiwwLCJbXFxEYl97bi0xfVxcb3RpbWVzIFsxXSwxXSJdLFs0LDAsIlxcRGJfblxcdmVlWzFdIl0sWzMsMCwiXFxEYl9uIl0sWzEsMF0sWzEsMl0sWzQsMl0sWzQsM11d
\begin{equation}
\label{eq:tensor of globuees is sitrict}
\begin{tikzcd}
	{[1]\vee \Db_n} & {\Db_n} & {[\Db_{n-1}\otimes [1],1]} & {\Db_n} & {\Db_n\vee[1]}
	\arrow[from=1-2, to=1-1]
	\arrow[from=1-2, to=1-3]
	\arrow[from=1-4, to=1-3]
	\arrow[from=1-4, to=1-5]
\end{tikzcd}
\end{equation}
The induction hypothesis and proposition \ref{prop:suspension preserves stricte} implies that all the objects are strict.
The proposition \ref{prop:cartesian squares} then implies that the diagram
% https://q.uiver.app/#q=WzAsNixbMSwwLCJcXERiX3tuLTF9XFxvdGltZXNbMV0iXSxbMSwxLCJbMV0iXSxbMCwxLCJcXHswXFx9Il0sWzAsMCwiXFxEYl97bi0xfSJdLFsyLDAsIlxcRGJfe24tMX0iXSxbMiwxLCJcXHsxXFx9Il0sWzMsMl0sWzAsMV0sWzQsNV0sWzIsMV0sWzMsMF0sWzQsMF0sWzUsMV1d
\[\begin{tikzcd}
	{\Db_{n-1}} & {\Db_{n-1}\otimes[1]} & {\Db_{n-1}} \\
	{\{0\}} & {[1]} & {\{1\}}
	\arrow[from=1-1, to=2-1]
	\arrow[from=1-2, to=2-2]
	\arrow[from=1-3, to=2-3]
	\arrow[from=2-1, to=2-2]
	\arrow[from=1-1, to=1-2]
	\arrow[from=1-3, to=1-2]
	\arrow[from=2-3, to=2-2]
\end{tikzcd}\]
verifies the hypothesis of proposition \ref{prop:example of a special colimit3}.
The proposition \textit{op. cit.} then
states that the colimit of \eqref{eq:tensor of globuees is sitrict} is special, which implies, according to lemma \ref{lemma:colimit computed in set presheaves}, that  its colimit, which is $\Db_n\otimes[1]$,  is also strict.

We proceed similarly for the Gray cone and the Gray $\circ$-cone.
\end{proof}

We now recall the following fundamental result of strictification:
\begin{theorem}[Gagna, Ozornova, Rovelli]
\label{theo:join preserves stict VMG version}
For any globular sum $a$, $a\star 1$ and $1\costar a$ are stricts.
\end{theorem}
\begin{proof}
The fact that $a\star 1$ is strict is a particular case of theorem 5.2 of \cite{Gagna_Nerves_and_cones_of_free_loop_free_omega_categories}. For the second assertion, remark that we have a canonical comparison, natural in $a:\Theta$: 
$$1\costar a\to \N\pi_0(1\costar a)\sim \N \pi_0 (a^\circ \star 1)^\circ \sim (\N \pi_0 (a^\circ \star 1))^\circ\sim (a^\circ\star 1)^{\circ}$$
where the first equivalence is a consequence of \cite[proposition A.22]{Ara_Maltsiniotis_joint_et_tranche}, the second comes from the commutativity of $\pi_0$ and $\N$ with dualities, and the last one is the (already demonstrated) first assertion.
The subset of object of $\Theta$ making this comparison an equivalence is closed by colimits and, according to lemma \ref{lemma:gray operation on globes are strict}, contains globes. This subset then contains all the globular sums. As strict objects are stable by dualities, this concludes the proof of the second assertion.
\end{proof}



\begin{lemma}
\label{lemma:strictification2}
Let $\alpha$ be $-$ if $n$ is even (resp. odd) and $+$ if $n$ is odd (resp.even).
Consider a cartesian square
% q.uiver.app/#q=WzAsNCxbMCwwLCJDXzAiXSxbMSwwLCJEIl0sWzAsMSwiXFxEYl97bn0iXSxbMSwxLCJcXERiX3tuKzF9Il0sWzAsMiwicCIsMl0sWzEsMywicCciXSxbMCwxXSxbMCwzLCIiLDEseyJzdHlsZSI6eyJuYW1lIjoiY29ybmVyIn19XSxbMiwzLCJpX3tufV5cXGFscGhhIiwyXV0=
\begin{equation}
\label{eq:lemma:strictification2}
\begin{tikzcd}
	{C_0} & D \\
	{\Db_{n}} & {\Db_{n+1}}
	\arrow["p"', from=1-1, to=2-1]
	\arrow["{p'}", from=1-2, to=2-2]
	\arrow[from=1-1, to=1-2]
	\arrow["\lrcorner"{anchor=center, pos=0.125}, draw=none, from=1-1, to=2-2]
	\arrow["{i_{n}^\alpha}"', from=2-1, to=2-2]
\end{tikzcd}
\end{equation}
such that $p\to p'$ is a left $(n+1)$-Gray deformation retract (resp. a right $(n+1)$-Gray deformation retract). Let $C_1$ be the $\io$-category fitting in the pullback 
% q.uiver.app/#q=WzAsNCxbMCwwLCJDXzEiXSxbMSwwLCJEIl0sWzAsMSwiXFxEYl97bn0iXSxbMSwxLCJcXERiX3tuKzF9Il0sWzAsMiwicCIsMl0sWzEsMywicCciXSxbMCwxXSxbMCwzLCIiLDEseyJzdHlsZSI6eyJuYW1lIjoiY29ybmVyIn19XSxbMiwzLCJpX3tufV57MS1cXGFscGhhfSIsMl1d
\begin{equation}
\label{eq:lemma:strictification3}
\begin{tikzcd}
	{C_1} & D \\
	{\Db_{n}} & {\Db_{n+1}}
	\arrow["p"', from=1-1, to=2-1]
	\arrow["{p'}", from=1-2, to=2-2]
	\arrow[from=1-1, to=1-2]
	\arrow["\lrcorner"{anchor=center, pos=0.125}, draw=none, from=1-1, to=2-2]
	\arrow["{i_{n}^{1-\alpha}}"', from=2-1, to=2-2]
\end{tikzcd}
\end{equation}
Then if $C_0$ and $C_1$ are strict, so is $D$.
\end{lemma}
\begin{proof}
We denote by $(i,r,\phi)$ the deformation retract structure corresponding to $C_0\to D$. 
We show this result by induction, and let's start with the case $n=0$. 
This corresponds to the case where $C_0\to D$ fits in a pullback diagram.
% q.uiver.app/#q=WzAsNCxbMCwwLCJDXzAiXSxbMSwwLCJEIl0sWzEsMSwiWzFdIl0sWzAsMSwiXFx7MFxcfSJdLFswLDFdLFszLDJdLFswLDNdLFsxLDJdXQ==
\[\begin{tikzcd}
	{C_0} & D \\
	{\{0\}} & {[1]}
	\arrow[from=1-1, to=1-2]
	\arrow[from=2-1, to=2-2]
	\arrow[from=1-1, to=2-1]
	\arrow[from=1-2, to=2-2]
\end{tikzcd}\]
 Let $x,y$ be two objects of $D$. Suppose first that $x$ and $y$ are over the same object of $[1]$. In this case, $\hom_D(x,y)$ is equivalent to either $\hom_{C_0}(x,y)$ or $\hom_{C_1}(x,y)$ and is then strict. If $x$ is over $1$ and $y$ over $0$, the $\infty$-groupoid $\hom_D(x,y)$ is empty. If $x$ is over $0$ and $y$ is over $1$, $\hom_D(x,y)$ is equivalent to $\hom_{C_0}(x,ry)$ according to \ref{prop:Gray deformation retract and passage to hom unmarked} and is then strict by hypothesis. Eventually, $\tau_0(D)$ is equivalent to $\tau_0(C_1)$ and is then a set. According to \ref{prop:criter stricte easy}, this implies that $D$ is strict.

Suppose now the result is true at stage $(n-1)$.
Let $p'\to p$ be a square verifying the condition.  Remark that, at the level of objects, the inclusion $C_0\to D$, its retract, and its deformation, are the identity.


Let $x$ and $y$ be two objects of $D$. 
As before, the only interesting case is when $x$ is over $0$ and $y$ is over $1$. In this case,
applying $\hom(\uvar,\uvar)$ to the square \eqref{eq:lemma:strictification2}, we get a cartesian square
% https://q.uiver.app/#q=WzAsNCxbMCwwLCJcXGhvbV97Q18wfSh4LHkpIl0sWzEsMCwiXFxob21fRCh4LHkpIl0sWzEsMSwiXFxEYl97bn0iXSxbMCwxLCJcXERiX3tuLTF9Il0sWzAsMV0sWzMsMiwiaV97bi0xfV5cXGFscGhhIiwyXSxbMCwzXSxbMSwyXV0=
\[\begin{tikzcd}
	{\hom_{C_0}(x,y)} & {\hom_D(x,y)} \\
	{\Db_{n-1}} & {\Db_{n}}
	\arrow[from=1-1, to=1-2]
	\arrow["{i_{n-1}^\alpha}"', from=2-1, to=2-2]
	\arrow[from=1-1, to=2-1]
	\arrow[from=1-2, to=2-2]
\end{tikzcd}\]
which is a right $n$-Gray deformation retract according to proposition \ref{prop:Gray deformation retract and passage to hom unmarked}.
Applying $\hom(\uvar,\uvar)$ to the square \eqref{eq:lemma:strictification3}, we get a cartesian square
% https://q.uiver.app/#q=WzAsNCxbMCwwLCJcXGhvbV97Q18xfSh4LHkpIl0sWzEsMCwiXFxob21fRCh4LHkpIl0sWzEsMSwiXFxEYl97bn0iXSxbMCwxLCJcXERiX3tuLTF9Il0sWzAsMV0sWzMsMiwiaV97bi0xfV57MS1cXGFscGhhfSIsMl0sWzAsM10sWzEsMl1d
\[\begin{tikzcd}
	{\hom_{C_1}(x,y)} & {\hom_D(x,y)} \\
	{\Db_{n-1}} & {\Db_{n}}
	\arrow[from=1-1, to=1-2]
	\arrow["{i_{n-1}^{1-\alpha}}"', from=2-1, to=2-2]
	\arrow[from=1-1, to=2-1]
	\arrow[from=1-2, to=2-2]
\end{tikzcd}\]
As $C_1$ is strict, so is $\hom_{C_1}(x,y)$.
 We can then apply the induction hypothesis, which implies that $\hom_D(x,y)$ is strict. As $\tau_0 D$ is equivalent to $\tau_0 C_{0}$, it is a set. We can apply proposition \ref{prop:criter stricte easy} which implies that $D$ is strict. 
\end{proof}



\p 
For an integer $n>0$,
we define by induction 
\begin{enumerate}
\item[$-$]
a left $(n+1)$-Gray retract structure for the inclusion 
\begin{equation}
\label{eq:Gray retract structurure for Gray cone}
\Db_n\star\emptyset \cup \Db_{n-1}\star 1 \to \Db_n\star 1
\end{equation}
where the gluing is performed along $i_n^\alpha:\Db_{n-1}\star\emptyset\to \Db_n\star\emptyset$ with $\alpha$ being $+$ if $n$ is odd and $-$ if not, 
\item[$-$]
a right $(n+1)$-Gray retract structure for the inclusion
\begin{equation}
\label{eq:Gray retract structurure for circ Gray cone}
1\costar\Db_{n-1} \cup \emptyset\costar\Db_{n}\to 1 \costar \Db_n
\end{equation}
where the gluing is performed along $i_n^\alpha:\emptyset\costar \Db_{n-1}\to \emptyset\costar \Db_n$ with $\alpha$ being $-$ if $n$ is odd and $+$ if not. 
\end{enumerate}
If $n=1$, the first morphism corresponds to the inclusion
% q.uiver.app/#q=WzAsNyxbMiwwLCJcXGJ1bGxldCJdLFsyLDEsIlxcYnVsbGV0Il0sWzMsMSwiXFxidWxsZXQiXSxbMCwwLCJcXGJ1bGxldCJdLFswLDEsIlxcYnVsbGV0Il0sWzEsMSwiXFxidWxsZXQiXSxbMSwwXSxbMCwxXSxbMSwyXSxbMCwyXSxbMyw0XSxbNCw1XSxbNiw1LCIiLDIseyJzdHlsZSI6eyJib2R5Ijp7Im5hbWUiOiJub25lIn0sImhlYWQiOnsibmFtZSI6Im5vbmUifX19XSxbOSw4LCIiLDAseyJvZmZzZXQiOjIsInNob3J0ZW4iOnsic291cmNlIjoyMCwidGFyZ2V0IjoyMH19XSxbMTIsNywiIiwyLHsic2hvcnRlbiI6eyJzb3VyY2UiOjIwLCJ0YXJnZXQiOjIwfSwibGV2ZWwiOjEsInN0eWxlIjp7InRhaWwiOnsibmFtZSI6Im1hcHMgdG8ifX19XV0=
\[\begin{tikzcd}
	\bullet & {} & \bullet \\
	\bullet & \bullet & \bullet & \bullet
	\arrow[""{name=0, anchor=center, inner sep=0}, from=1-3, to=2-3]
	\arrow[""{name=1, anchor=center, inner sep=0}, from=2-3, to=2-4]
	\arrow[""{name=2, anchor=center, inner sep=0}, from=1-3, to=2-4]
	\arrow[from=1-1, to=2-1]
	\arrow[from=2-1, to=2-2]
	\arrow[""{name=3, anchor=center, inner sep=0}, draw=none, from=1-2, to=2-2]
	\arrow[shift right=2, shorten <=2pt, shorten >=2pt, Rightarrow, from=2, to=1]
	\arrow[shorten <=6pt, shorten >=6pt, maps to, from=3, to=0]
\end{tikzcd}\]
and the second one to the inclusion:
% q.uiver.app/#q=WzAsNyxbMCwxLCJcXGJ1bGxldCJdLFsxLDEsIlxcYnVsbGV0Il0sWzEsMCwiXFxidWxsZXQiXSxbMiwxLCJcXGJ1bGxldCJdLFszLDAsIlxcYnVsbGV0Il0sWzMsMSwiXFxidWxsZXQiXSxbMiwwXSxbMiwxXSxbMyw1XSxbMyw0XSxbNCw1XSxbMCwyXSxbNiwzLCIiLDIseyJzdHlsZSI6eyJib2R5Ijp7Im5hbWUiOiJub25lIn0sImhlYWQiOnsibmFtZSI6Im5vbmUifX19XSxbOSw4LCIiLDAseyJvZmZzZXQiOi0yLCJzaG9ydGVuIjp7InNvdXJjZSI6MjAsInRhcmdldCI6MjB9fV0sWzcsMTIsIiIsMix7InNob3J0ZW4iOnsic291cmNlIjoyMCwidGFyZ2V0IjoyMH0sImxldmVsIjoxLCJzdHlsZSI6eyJ0YWlsIjp7Im5hbWUiOiJtYXBzIHRvIn19fV1d
\[\begin{tikzcd}
	& \bullet & {} & \bullet \\
	\bullet & \bullet & \bullet & \bullet
	\arrow[""{name=0, anchor=center, inner sep=0}, from=1-2, to=2-2]
	\arrow[""{name=1, anchor=center, inner sep=0}, from=2-3, to=2-4]
	\arrow[""{name=2, anchor=center, inner sep=0}, from=2-3, to=1-4]
	\arrow[from=1-4, to=2-4]
	\arrow[from=2-1, to=1-2]
	\arrow[""{name=3, anchor=center, inner sep=0}, draw=none, from=1-3, to=2-3]
	\arrow[shift left=2, shorten <=2pt, shorten >=2pt, Rightarrow, from=2, to=1]
	\arrow[shorten <=6pt, shorten >=6pt, maps to, from=0, to=3]
\end{tikzcd}\]
The propositions \ref{prop:of left Gray deformation retract unmarked} and \ref{prop:left Gray deformation retract stable under pushout unmarked} imply that the first morphism is a left $2$-Gray deformation retract and the second one a right $2$-Gray deformation retract. 
Suppose now that these two morphisms are constructed at stage $n$.
The formula \eqref{eq:eq for Gray cone} implies that $\Db_{n+1}\star\emptyset \cup \Db_{n}\star 1 \to \Db_{n+1}\star 1$ fits in the cocartesian square
% https://q.uiver.app/#q=WzAsNCxbMCwxLCJbMVxcY29zdGFyXFxEYl9uLDFdIl0sWzEsMSwiIFxcRGJfblxcc3RhciAxIl0sWzAsMCwiWzFcXGNvc3RhclxcRGJfe24tMX1cXGN1cCBcXGVtcHR5c2V0XFxzdGFyXFxEYl9uLDFdIl0sWzEsMCwiXFxEYl9uXFxzdGFyXFxlbXB0eXNldFxcY3VwIFxcRGJfe24tMX1cXHN0YXIgMSJdLFswLDFdLFsyLDNdLFsyLDBdLFszLDFdXQ==
\[\begin{tikzcd}
	{[1\costar\Db_{n-1}\cup \emptyset\star\Db_n,1]} & {\Db_n\star\emptyset\cup \Db_{n-1}\star 1} \\
	{[1\costar\Db_n,1]} & { \Db_n\star 1}
	\arrow[from=2-1, to=2-2]
	\arrow[from=1-1, to=1-2]
	\arrow[from=1-1, to=2-1]
	\arrow[from=1-2, to=2-2]
\end{tikzcd}\]
The induction hypothesis and the propositions \ref{prop:suspension of left Gray deformation retract unmarked} and \ref{prop:left Gray deformation retract stable under pushout unmarked} endow this morphism with a left $(n+2)$-Gray retract structure. We constructs similarly the right $(n+2)$-Gray retract structure for the inclusion $1\costar\Db_{n-1} \cup \emptyset\costar\Db_{n}\to 1 \costar \Db_n$.
\begin{prop}
\label{prop:strict stuff are stable under Gray cone}
Let $C$ be a strict $\io$-category, $a$ a globular sum, and $f:a\to C$ any morphism. The $\io$-categories $C\coprod_a a\star 1$ and $1\costar a\coprod_a C$ are strict.
\end{prop}
\begin{proof}
We will prove the result by induction on the number of non-identity cells of $a$. Remark that for any globular sum $b$, there exists a globular sum $a$, an integer $n$, and a cartesian square composed of globular morphism
% q.uiver.app/#q=WzAsNCxbMSwwLCJhIl0sWzEsMSwiYiJdLFswLDAsIlxcRGJfe24tMX0iXSxbMCwxLCJcXERiX3tufSJdLFsyLDMsImleXFxhbHBoYV97bi0xfSIsMl0sWzMsMSwibCIsMl0sWzEsMiwiIiwxLHsic3R5bGUiOnsibmFtZSI6ImNvcm5lciJ9fV0sWzIsMF0sWzAsMV1d
\[\begin{tikzcd}
	{\Db_{n-1}} & a \\
	{\Db_{n}} & b
	\arrow["{i^\alpha_{n-1}}"', from=1-1, to=2-1]
	\arrow["l"', from=2-1, to=2-2]
	\arrow["\lrcorner"{anchor=center, pos=0.125, rotate=180}, draw=none, from=2-2, to=1-1]
	\arrow[from=1-1, to=1-2]
	\arrow[from=1-2, to=2-2]
\end{tikzcd}\]
with $\alpha=+$ if $n$ is odd, and $\alpha=-$ if $n$ is even, and such that $l$ admits a retract $r$. As $i^\alpha_{n-1}$ is globular, the pullback along this morphism preserves colimits according to theorem \ref{theo:pullback along conduche preserves colimits}. We then have a cartesian square:
% q.uiver.app/#q=WzAsNCxbMCwwLCJhIl0sWzEsMCwiYiJdLFsxLDEsIlxcRGJfe259Il0sWzAsMSwiXFxEYl97bi0xfSJdLFsxLDIsInIiXSxbMywyLCJpXlxcYWxwaGFfe24tMX0iLDJdLFswLDNdLFswLDFdXQ==
\[\begin{tikzcd}
	a & b \\
	{\Db_{n-1}} & {\Db_{n}}
	\arrow["r", from=1-2, to=2-2]
	\arrow["{i^\alpha_{n-1}}"', from=2-1, to=2-2]
	\arrow[from=1-1, to=2-1]
	\arrow[from=1-1, to=1-2]
\end{tikzcd}\]
 We also define $a'$ as the pullback:
% q.uiver.app/#q=WzAsNCxbMCwwLCJhJyJdLFsxLDAsImIiXSxbMSwxLCJcXERiX3tufSJdLFswLDEsIlxcRGJfe24tMX0iXSxbMSwyLCJyIl0sWzMsMiwiaV57LVxcYWxwaGF9X3tuLTF9IiwyXSxbMCwzXSxbMCwxXV0=
\[\begin{tikzcd}
	{a'} & b \\
	{\Db_{n-1}} & {\Db_{n}}
	\arrow["r", from=1-2, to=2-2]
	\arrow["{i^{-\alpha}_{n-1}}"', from=2-1, to=2-2]
	\arrow[from=1-1, to=2-1]
	\arrow[from=1-1, to=1-2]
\end{tikzcd}\]
and remark that $a'$ is a globular sum. Eventually, we fix a morphism $b\to C$. As $a$ and $a'$ are sub globular sum of $b$, the number of non-identity cells in each of them is strictly less than the one of $b$.
 We then suppose that for any strict $\io$-category $C$, and any morphism $b\to C$, the two induced $\io$-category 
 $C\coprod_a a\star 1$ and $C\coprod_{a'}a'\star 1$ are strict, and we are willing to show that $C\coprod_bb\star 1$ also is. 
We claim that the two following squares are cartesian
% q.uiver.app/#q=WzAsOCxbMCwwLCJiXFxjb3Byb2Rfe2F9YVxcc3RhcjEiXSxbMCwxLCJbXFxEYl97bi0xfSwxXSJdLFsxLDEsIltcXERiX3tufSwxXSJdLFsxLDAsImJcXHN0YXIgMSJdLFsyLDAsImJcXGNvcHJvZF97YSd9YSdcXHN0YXIxIl0sWzIsMSwiW1xcRGJfe24tMX0sMV0iXSxbMywxLCJbXFxEYl97bn0sMV0iXSxbMywwLCJiXFxzdGFyIDEiXSxbMCwxXSxbMCwzXSxbMSwyLCJbaV5cXGFscGhhX3tuLTF9LDFdIiwyXSxbMywyXSxbNCw1XSxbNCw3XSxbNSw2LCJbaV57LVxcYWxwaGF9X3tuLTF9LDFdIiwyXSxbNyw2XV0=
\[\begin{tikzcd}
	{b\coprod_{a}a\star1} & {b\star 1} & {b\coprod_{a'}a'\star1} & {b\star 1} \\
	{[\Db_{n-1},1]} & {[\Db_{n},1]} & {[\Db_{n-1},1]} & {[\Db_{n},1]}
	\arrow[from=1-1, to=2-1]
	\arrow[from=1-1, to=1-2]
	\arrow["{[i^\alpha_{n-1},1]}"', from=2-1, to=2-2]
	\arrow[from=1-2, to=2-2]
	\arrow[from=1-3, to=2-3]
	\arrow[from=1-3, to=1-4]
	\arrow["{[i^{-\alpha}_{n-1},1]}"', from=2-3, to=2-4]
	\arrow[from=1-4, to=2-4]
\end{tikzcd}\]
According to theorem \ref{theo:join preserves stict VMG version}, proposition \ref{prop:suspension preserves stricte}, and the induction hypothesis, all the objects of these squares are strict. We can show the cartesianess in $\zocat$, where it follows from lemma \ref{lemma: pullback and sum}. 
As morphism $[i^-_{n-1},1],[i^+_{n-1},1]$ are globular, the pullback functors $[i^-_{n-1},1]^*$, $[i^+_{n-1},1]^*$ preserve colimits according to theorem \ref{theo:pullback along conduche preserves colimits}. We then have two cartesian squares:
% q.uiver.app/#q=WzAsOCxbMCwwLCJDXFxjb3Byb2Rfe2F9YVxcc3RhcjEiXSxbMCwxLCJbXFxEYl97bi0xfSwxXSJdLFsxLDEsIltcXERiX3tufSwxXSJdLFsxLDAsIkNcXGNvcHJvZF9iYlxcc3RhciAxIl0sWzIsMCwiQ1xcY29wcm9kX3thJ31hJ1xcc3RhcjEiXSxbMiwxLCJbXFxEYl97bi0xfSwxXSJdLFszLDEsIltcXERiX3tufSwxXSJdLFszLDAsIkNcXGNvcHJvZF9iYlxcc3RhciAxIl0sWzAsMV0sWzAsM10sWzEsMiwiW2leXFxhbHBoYV97bi0xfSwxXSIsMl0sWzMsMl0sWzQsNV0sWzQsN10sWzUsNiwiW2leey1cXGFscGhhfV97bi0xfSwxXSIsMl0sWzcsNl1d
\begin{equation}
\label{eq:two square in the proof of strict}
\begin{tikzcd}
	{C\coprod_{a}a\star1} & {C\coprod_bb\star 1} & {C\coprod_{a'}a'\star1} & {C\coprod_bb\star 1} \\
	{[\Db_{n-1},1]} & {[\Db_{n},1]} & {[\Db_{n-1},1]} & {[\Db_{n},1]}
	\arrow[from=1-1, to=2-1]
	\arrow[from=1-1, to=1-2]
	\arrow["{[i^\alpha_{n-1},1]}"', from=2-1, to=2-2]
	\arrow[from=1-2, to=2-2]
	\arrow[from=1-3, to=2-3]
	\arrow[from=1-3, to=1-4]
	\arrow["{[i^{-\alpha}_{n-1},1]}"', from=2-3, to=2-4]
	\arrow[from=1-4, to=2-4]
\end{tikzcd}
\end{equation}
and by the induction hypothesis, the two top left objects are strict. 
Eventually, remark that we have a cocartesian square
% q.uiver.app/#q=WzAsNCxbMCwxLCJDXFxjb3Byb2Rfe2F9YVxcc3RhcjEiXSxbMSwxLCJDXFxjb3Byb2RfYmJcXHN0YXIgMSJdLFswLDAsIlxcRGJfblxcY29wcm9kX3tcXERiX3tuLTF9fVxcRGJfe24tMX1cXHN0YXIgMSJdLFsxLDAsIlxcRGJfe259XFxzdGFyIDEiXSxbMCwxXSxbMiwwXSxbMiwzXSxbMywxXSxbMSwyLCIiLDEseyJzdHlsZSI6eyJuYW1lIjoiY29ybmVyIn19XV0=
\[\begin{tikzcd}
	{\Db_n\coprod_{\Db_{n-1}}\Db_{n-1}\star 1} & {\Db_{n}\star 1} \\
	{C\coprod_{a}a\star1} & {C\coprod_bb\star 1}
	\arrow[from=2-1, to=2-2]
	\arrow[from=1-1, to=2-1]
	\arrow[from=1-1, to=1-2]
	\arrow[from=1-2, to=2-2]
	\arrow["\lrcorner"{anchor=center, pos=0.125, rotate=180}, draw=none, from=2-2, to=1-1]
\end{tikzcd}\]
and the proposition \ref{prop:left Gray deformation retract stable under pushout unmarked} then implies that the left square of 
\eqref{eq:two square in the proof of strict} is a left $(n+1)$-Gray retract, and the lemma \ref{lemma:strictification2} implies that $C\coprod_bb\star 1$ is strict. This proves the first assertion. The second one is proved similarly.
\end{proof}


\p We now want to give an analogue of proposition \ref{prop:strict stuff are stable under Gray cone} for the Gray cylinder. In what follows, we will use the results of sections \ref{section:Colimit of left cartesian fibrations} and \ref{subsection:A criterion to be a left cartesian fibration} (more precisely the proposition \ref{prop:equivalence betwen slice and join strict word2}, the theorem \ref{theo:equivalence betwen slice and join} and the corollaries \ref{cor:cor of the past10}, \ref{cor:cor of the past3}).
We assure the reader that this is not a tautology, as the proofs of these results are not based on the following propositions and theorems




\begin{prop}
\label{prop:fibers of 1 star a}
Let $a$ be a globular sum. The two following canonical squares are cartesian
% https://q.uiver.app/#q=WzAsOCxbMSwxLCJbYSwxXSJdLFsxLDAsIjFcXGNvc3RhciBhIl0sWzAsMSwiXFx7MFxcfSJdLFswLDAsIjEiXSxbMywxLCJbYSwxXSJdLFszLDAsImFcXHN0YXIgMSJdLFsyLDEsIlxcezFcXH0iXSxbMiwwLCIxIl0sWzMsMV0sWzIsMF0sWzMsMl0sWzEsMF0sWzcsNV0sWzYsNF0sWzcsNl0sWzUsNF1d
\[\begin{tikzcd}
	1 & {1\costar a} & 1 & {a\star 1} \\
	{\{0\}} & {[a,1]} & {\{1\}} & {[a,1]}
	\arrow[from=1-1, to=1-2]
	\arrow[from=2-1, to=2-2]
	\arrow[from=1-1, to=2-1]
	\arrow[from=1-2, to=2-2]
	\arrow[from=1-3, to=1-4]
	\arrow[from=2-3, to=2-4]
	\arrow[from=1-3, to=2-3]
	\arrow[from=1-4, to=2-4]
\end{tikzcd}\]
The five squares appearing in the following canonical diagram are both cartesian and cocartesian:
% https://q.uiver.app/#q=WzAsOCxbMSwyLCIxXFxjb3N0YXIgYSJdLFsyLDIsIlthLDFdIl0sWzIsMSwiYVxcc3RhciAxIl0sWzEsMSwiYVxcb3RpbWVzWzFdIl0sWzIsMCwiMSJdLFsxLDAsImFcXG90aW1lc1xcezBcXH0iXSxbMCwxLCJhXFxvdGltZXNcXHsxXFx9Il0sWzAsMiwiMSJdLFsyLDFdLFswLDFdLFszLDBdLFszLDJdLFs1LDRdLFs0LDJdLFs1LDNdLFs2LDNdLFs3LDBdLFs2LDddXQ==
\[\begin{tikzcd}
	& {a\otimes\{0\}} & 1 \\
	{a\otimes\{1\}} & {a\otimes[1]} & {a\star 1} \\
	1 & {1\costar a} & {[a,1]}
	\arrow[from=2-3, to=3-3]
	\arrow[from=3-2, to=3-3]
	\arrow[from=2-2, to=3-2]
	\arrow[from=2-2, to=2-3]
	\arrow[from=1-2, to=1-3]
	\arrow[from=1-3, to=2-3]
	\arrow[from=1-2, to=2-2]
	\arrow[from=2-1, to=2-2]
	\arrow[from=3-1, to=3-2]
	\arrow[from=2-1, to=3-1]
\end{tikzcd}\]
\end{prop}
\begin{proof}
The five squares of the second diagram are cocartesian by construction. Furthermore, remark that all the objects appearing in the squares
% https://q.uiver.app/#q=WzAsMTYsWzEsMywiW2EsMV0iXSxbMSwyLCJhXFxzdGFyIDEiXSxbMCwyLCJhXFxvdGltZXNcXHsxXFx9Il0sWzAsMywiXFx7MVxcfSJdLFszLDIsIjFcXGNvc3RhciBhIl0sWzMsMywiW2EsMV0iXSxbMiwzLCJcXHswXFx9Il0sWzIsMiwiYVxcb3RpbWVzXFx7MFxcfSJdLFsxLDAsImFcXHN0YXIgMSJdLFsxLDEsIlthLDFdIl0sWzMsMCwiMVxcY29zdGFyIGEiXSxbMywxLCJbYSwxXSJdLFswLDEsIlxcezBcXH0iXSxbMiwxLCJcXHsxXFx9Il0sWzIsMCwiYSJdLFswLDAsImEiXSxbMSwwXSxbMiwzXSxbMiwxXSxbMywwXSxbMTUsMTJdLFs4LDldLFsxNCwxM10sWzEwLDExXSxbMTIsOV0sWzEzLDExXSxbMTQsMTBdLFsxNSw4XSxbNyw2XSxbNCw1XSxbNiw1XSxbNyw0XV0=
\[\begin{tikzcd}
	a & {a\star 1} & a & {1\costar a} \\
	{\{0\}} & {[a,1]} & {\{1\}} & {[a,1]} \\
	{a\otimes\{1\}} & {a\star 1} & {a\otimes\{0\}} & {1\costar a} \\
	{\{1\}} & {[a,1]} & {\{0\}} & {[a,1]}
	\arrow[from=3-2, to=4-2]
	\arrow[from=3-1, to=4-1]
	\arrow[from=3-1, to=3-2]
	\arrow[from=4-1, to=4-2]
	\arrow[from=1-1, to=2-1]
	\arrow[from=1-2, to=2-2]
	\arrow[from=1-3, to=2-3]
	\arrow[from=1-4, to=2-4]
	\arrow[from=2-1, to=2-2]
	\arrow[from=2-3, to=2-4]
	\arrow[from=1-3, to=1-4]
	\arrow[from=1-1, to=1-2]
	\arrow[from=3-3, to=4-3]
	\arrow[from=3-4, to=4-4]
	\arrow[from=4-3, to=4-4]
	\arrow[from=3-3, to=3-4]
\end{tikzcd}\]
are strict according to theorem \ref{theo:join preserves stict VMG version} and proposition \ref{prop:suspension preserves stricte}. One can the show their cartesianess in $\zocat$, where it follows from proposition \ref{prop:cartesian squares}.

By stability by right cancellation of cartesian square, it remains to show that the square
% https://q.uiver.app/#q=WzAsNCxbMSwwLCIxXFxjb3N0YXIgYSJdLFsxLDEsIlthLDFdIl0sWzAsMSwiYVxcc3RhciAxIl0sWzAsMCwiYVxcb3RpbWVzWzFdIl0sWzMsMl0sWzAsMV0sWzIsMV0sWzMsMF1d
\[\begin{tikzcd}
	{a\otimes[1]} & {1\costar a} \\
	{a\star 1} & {[a,1]}
	\arrow[from=1-1, to=2-1]
	\arrow[from=1-2, to=2-2]
	\arrow[from=2-1, to=2-2]
	\arrow[from=1-1, to=1-2]
\end{tikzcd}\]
is cartesian.
Using the fact that pullback along $1\costar a\to [a,1]$ preserves colimits as stated by corollary \ref{cor:cor of the past3}, it is sufficient to show that for any globular morphism $\Db_n\to a$, the outer square of the diagram
% q.uiver.app/#q=WzAsNixbMSwxLCJhXFxzdGFyIDEiXSxbMCwxLCJcXERiX25cXHN0YXIgMSJdLFswLDAsIlxcRGJfblxcb3RpbWVzWzFdXFxjb3Byb2Rfe1xcRGJfbn0gYSJdLFsxLDAsImFcXG90aW1lc1sxXSJdLFsyLDAsIjFcXGNvc3RhciBhIl0sWzIsMSwiW2EsMV0iXSxbMSwwXSxbMywwXSxbMCw1XSxbMiwxXSxbMiwzXSxbMyw0XSxbNCw1XV0=
\[\begin{tikzcd}
	{\Db_n\otimes[1]\coprod_{\Db_n} a} & {a\otimes[1]} & {1\costar a} \\
	{\Db_n\star 1} & {a\star 1} & {[a,1]}
	\arrow[from=2-1, to=2-2]
	\arrow[from=1-2, to=2-2]
	\arrow[from=2-2, to=2-3]
	\arrow[from=1-1, to=2-1]
	\arrow[from=1-1, to=1-2]
	\arrow[from=1-2, to=1-3]
	\arrow[from=1-3, to=2-3]
\end{tikzcd}\]
is cartesian.
Remark that this outer square also factors as:
% q.uiver.app/#q=WzAsNixbMCwwLCJcXERiX25cXG90aW1lc1sxXVxcY29wcm9kX3tcXERiX259IGEiXSxbMCwxLCJcXERiX25cXHN0YXIgMSJdLFsxLDEsIltcXERiX24sMV0iXSxbMSwwLCIxXFxjb3N0YXIgXFxEYl9uXFxjb3Byb2QgX3tcXERiX259YSJdLFsyLDAsIjFcXGNvc3RhciBhIl0sWzIsMSwiW2EsMV0iXSxbMCwzXSxbMCwxXSxbMSwyXSxbMywyXSxbMiw1XSxbMyw0XSxbNCw1XV0=
\[\begin{tikzcd}
	{\Db_n\otimes[1]\coprod_{\Db_n} a} & {1\costar \Db_n\coprod _{\Db_n}a} & {1\costar a} \\
	{\Db_n\star 1} & {[\Db_n,1]} & {[a,1]}
	\arrow[from=1-1, to=1-2]
	\arrow[from=1-1, to=2-1]
	\arrow[from=2-1, to=2-2]
	\arrow[from=1-2, to=2-2]
	\arrow[from=2-2, to=2-3]
	\arrow[from=1-2, to=1-3]
	\arrow[from=1-3, to=2-3]
\end{tikzcd}\]
The cartesianess of the left square is a consequence of the preservation of colimit of the pullback along the morphism $\Db_n\star 1\to [\Db_n,1]$, and of the cartesian square provided by proposition \ref{prop:cartesian squares}. We recall that we can indeed use the last proposition, as we already show in lemma \ref{lemma:gray operation on globes are strict} that $\Db_n\otimes[1]$, $1\costar \Db_n$ and $\Db_n\star 1$ are strict.

For the right hand square, all the objects are strict according to proposition \ref{prop:strict stuff are stable under Gray cone}. We can then show the cartesianess in $\zocat$, where it follows from lemma \ref{lemma: pullback and sum}.
\end{proof}


\begin{lemma}
\label{lemma:an other canonical square}
Let $C$ be an $\io$-category, $a$ a globular sum, and $a\to C$ any morphism.
The following canonical square is cartesian: 
% https://q.uiver.app/#q=WzAsNCxbMSwxLCJbYSwxXSJdLFswLDEsIjFcXGNvc3RhciBhIl0sWzEsMCwiQ1xcY29wcm9kX2EgYVxcc3RhciAxIl0sWzAsMCwiQ1xcY29wcm9kX2FhXFxvdGltZXNbMV0iXSxbMSwwXSxbMywxXSxbMywyXSxbMiwwXV0=
\[\begin{tikzcd}
	{C\coprod_aa\otimes[1]} & {C\coprod_a a\star 1} \\
	{1\costar a} & {[a,1]}
	\arrow[from=2-1, to=2-2]
	\arrow[from=1-1, to=2-1]
	\arrow[from=1-1, to=1-2]
	\arrow[from=1-2, to=2-2]
\end{tikzcd}\]
\end{lemma}
\begin{proof}
For any $\io$-category $D$, the first square of proposition \ref{prop:fibers of 1 star a} implies that the following square is cartesian
% https://q.uiver.app/#q=WzAsNCxbMSwwLCJEXFxvdGltZXNcXHswXFx9Il0sWzEsMSwiW2EsMV0iXSxbMCwxLCIxXFxjb3N0YXIgYSJdLFswLDAsIkRcXG90aW1lc1xcezBcXH0iXSxbMywyXSxbMCwxXSxbMiwxXSxbMywwXV0=
\[\begin{tikzcd}
	{D\otimes\{0\}} & {D\otimes\{0\}} \\
	{1\costar a} & {[a,1]}
	\arrow[from=1-1, to=2-1]
	\arrow[from=1-2, to=2-2]
	\arrow[from=2-1, to=2-2]
	\arrow[from=1-1, to=1-2]
\end{tikzcd}\]
The statement then follows from proposition \textit{op cit} and the preservation of colimit of the pullback along the morphism $1\costar a\to [a,1]$ stated by corollary \ref{cor:cor of the past3}.
\end{proof}


\begin{prop}
\label{prop:strict stuff are stable under coproduc with cylinder}
Let $C$ be a strict $\io$-category, $a$ a globular sum, and $a\to C$ any morphism. The $\io$-category $C\coprod_a a\otimes [1]$ is strict. In particular $a\otimes[1]$ is strict.
\end{prop}
\begin{proof}
According to propositions \ref{prop:suspension preserves stricte} and \ref{prop:strict stuff are stable under Gray cone}, the two lower objects and the upper right one of the cartesian square of lemma \ref{lemma:an other canonical square} are strict whenever $C$ is. As strict object are stable under pullback, this concludes the proof.
\end{proof}

\p We combine the proposition \ref{prop:strict stuff are stable under Gray cone} and \ref{prop:strict stuff are stable under coproduc with cylinder} in the following theorem:

\begin{theorem}
\label{prop:strict stuff are pushout}
Let $C$ be an $\io$-category, $a$ a globular sum, and $f:a\to C$ any morphism. The $\io$-categories $$1\costar a\coprod_a C~~~~C\coprod_a a\otimes[1]~~~~C\coprod_a a\star 1$$
are strict whenever $C$ is. In particular, $a\otimes[1]$, $a\star 1$ and $1\costar a$ are strict.
\end{theorem}


\begin{cor}
\label{cor: a otimes n is strict}
Let $a$ be a globular sum, and $K$ an order set, viewed as an $\iun$-category. The $\io$-category $a\otimes K$ is strict.
\end{cor}
\begin{proof}
If $K$ is $[n]$, an easy induction using proposition \ref{prop:strict stuff are stable under coproduc with cylinder} shows the result.
In the general case, remark that $K$ is the special colimit of the diagram $\pi:\Delta^{\hookrightarrow}_{/K}\to \iPsh{\Delta}$ where $\Delta^{\hookrightarrow}_{/K}$ is the category whose objects are monomorphisms $[n]\to K$ and arrows are monomorphisms between domains making the induced triangle commutative, while $\pi$ sends $[n]\to K$ to $[n]$.
We claim that the natural transformation 
$$a\otimes \pi\to \pi$$
is cartesian. Proposition \ref{prop:special colimit} then implies that $a\otimes\pi$ has a special colimit. Moreover, $a\otimes \pi$
fulfills the hypotheses of the third assertion of lemma \ref{lemma:colimit computed in set presheaves}. Its colimit is then strict, and this concludes the proof of the first assertion. 

To demonstrate the cartesianess of the natural transformation $a\otimes \pi\to \pi$, one has to show that for any monomorphism $i:[k]\to [l]$,
the induced square 
% q.uiver.app/#q=WzAsNCxbMCwwLCJhXFxvdGltZXMgW2tdIl0sWzAsMSwiW2tdIl0sWzEsMSwiW2xdIl0sWzEsMCwiYVxcb3RpbWVzW2xdIl0sWzAsMV0sWzMsMl0sWzEsMl0sWzAsM11d
\[\begin{tikzcd}
	{a\otimes [k]} & {a\otimes[l]} \\
	{[k]} & {[l]}
	\arrow[from=1-1, to=2-1]
	\arrow[from=1-2, to=2-2]
	\arrow[from=2-1, to=2-2]
	\arrow[from=1-1, to=1-2]
\end{tikzcd}\]
is cartesian.


As $[k]\to [l]$ is fully faithful, so is $[k]\times_{[l]}a\otimes[l]\to a\otimes[l]$. If we manage to show that $a\otimes[k]\to a\otimes[l]$ is fully faithful, it will imply by right cancelation that $a\otimes [k]\to [l]\coprod_{[k]}a\otimes[l]$ is also fully faithful, and as this morphism is obviously surjective on objects it will conclude the proof.


We then have to show that for any integer $n>0$, any square of shape
% q.uiver.app/#q=WzAsNCxbMSwwLCIgYVxcb3RpbWVzICBba10iXSxbMSwxLCIgYVxcb3RpbWVzIFtsXSJdLFswLDAsIiBcXHBhcnRpYWxcXERiX24iXSxbMCwxLCIgXFxEYl9uIl0sWzAsMV0sWzMsMSwiZiIsMl0sWzIsMCwiZyJdLFsyLDNdXQ==
\[\begin{tikzcd}
	{ \partial\Db_n} & { a\otimes [k]} \\
	{ \Db_n} & { a\otimes [l]}
	\arrow[from=1-2, to=2-2]
	\arrow["f"', from=2-1, to=2-2]
	\arrow["g", from=1-1, to=1-2]
	\arrow[from=1-1, to=2-1]
\end{tikzcd}\]
admits a unique lifting.
Suppose given such square.
Using the Steiner theory recalled in \ref{section:Steiner thery}, it is equivalent show that the induced square of augmented directed complexes:
% q.uiver.app/#q=WzAsNCxbMSwwLCJcXGxhbWJkYSBhXFxvdGltZXMgXFxsYW1iZGEgW2tdIl0sWzEsMSwiXFxsYW1iZGEgYVxcb3RpbWVzXFxsYW1iZGEgW2xdIl0sWzAsMCwiXFxsYW1iZGEgXFxwYXJ0aWFsXFxEYl9uIl0sWzAsMSwiXFxsYW1iZGEgXFxEYl9uIl0sWzAsMV0sWzMsMSwiZiIsMl0sWzIsMCwiZyJdLFsyLDNdXQ==
\[\begin{tikzcd}
	{\lambda \partial\Db_n} & {\lambda a\otimes \lambda [k]} \\
	{\lambda \Db_n} & {\lambda a\otimes\lambda [l]}
	\arrow[from=1-2, to=2-2]
	\arrow["f"', from=2-1, to=2-2]
	\arrow["g", from=1-1, to=1-2]
	\arrow[from=1-1, to=2-1]
\end{tikzcd}\]
admits a unique lifting. 
We recall that the basis of $\lambda\Db_n$ is given by the graded set:
$$(B_{\lambda\Db_n})_k:= \left\{ 
\begin{array}{ll}
\{e_k^-,e_k^+\}&\mbox{ if $k<n$}\\
\{e_n\}&\mbox{ if $k=n$}\\
\emptyset&\mbox{ if $k>n$}\\
\end{array}\right.$$ 
and that the basis of $\lambda[n]$ also admits is given by the graded set
$$(B_{\lambda\Db_n})_k:= \left\{ 
\begin{array}{ll}
\{v_0,v_1,...,v_n\}&\mbox{ if $k=0$}\\
\{v_{0,1},v_{1,2}...,v_{n-1,n}\}&\mbox{ if $k=1$}\\
\emptyset&\mbox{ if k>1}\\
\end{array} \right.$$ 

 We will suppose that $n$ is odd as the proof for $n$ even is similar. As the right vertical morphism is an injection, we just have to show the existence of the lifting.


There exists a unique sequence $\{b_0,...,b_{l-1}\}$ of element of $(\lambda b)_{n-1}$
and a unique sequence $\{c_0,...,c_{l}\}$ of element of $(\lambda b)_{n}$ such that 
$$f(e_n)= b_0\otimes v_{0,1}+...+ b_{l-1}\otimes v_{l-1,l}+c_0\otimes v_0+...+ c_l\otimes v_l$$
The commutativity of the square then implies that the cell
$$ \partial b_0\otimes v_{0,1}+...+ \partial b_{l-1}\otimes v_{l-1,l}+ (\partial c_0-b_0)\otimes v_0+(\partial c_1+b_0-b_1)\otimes v_1...+ (\partial c_l + b_l)\otimes v_l$$
is in the image of $\lambda a\otimes\lambda i$. 
As a consequence, for any $j<k$, we have
$$ \left\{
\begin{array}{ll}
\partial b_0=\partial b_1=...= \partial b_{i(0)-1}\\
\partial b_{i(j)}=\partial b_{i(j)+1}... =\partial b_{i(j+1)-1} &\mbox{for $j<k$}\\
\partial b_{i(k)}=\partial b_{i(k)+1}=...=\partial b_{l-1}\\
\end{array}\right.$$
and 
$$
\left\{
\begin{array}{ll}
\partial c_0-b_0=0 &\mbox{if $0$ is not in the image of $i$}\\
\partial c_p+b_{p-1}-b_p=0& \mbox{if $p>0$ is not in the image of $i$}\\
\partial c_l+b_{l-1}=0& \mbox{if $l$ is not in the image of $i$}\\
\end{array}\right.$$
The first set of equations forces the equalities: 
$$ \left\{
\begin{array}{ll}
 b_0= b_1=...= b_{i(0)-1}\\
 b_{i(j)}= b_{i(j)+1}... = b_{i(j+1)-1} &\mbox{for $j<k$}\\
 b_{i(k)}= b_{i(k)+1}=...= b_{l-1}\\
\end{array}\right.$$
Combined with the second set of equations this implies that $c_p$ is null whenever $p$ is not in the image of $i$.
We then have 
$$f(e_n)=b_{i(0)}\otimes \lambda i(v_{0,1})+...+b_{i(k)}\otimes \lambda i(v_{k-1,k})
+c_{i(0)}\otimes \lambda i (v_0)+...+ c_i(k)\otimes \lambda i (v_k)$$

We then define the morphism $l$ as the unique morphism extending $g$ and that fulfills 
$$l_n(e_n):= b_{i(0)}\otimes v_{0,1}+...+b_{i(k)}\otimes v_{k-1,k}
+c_{i(0)}\otimes v_0+...+ c_i(k)\otimes v_k$$ 
This morphism is the wanted lift.
\end{proof}







\begin{cor}
\label{cor:otimes et op}
There is a natural diagram
% q.uiver.app/#q=WzAsNixbMSwwLCIoQ1xcb3RpbWVzWzFdKV5cXGNpcmMiXSxbMiwwLCIoQ1xcb3RpbWVzXFx7MFxcfSleXFxjaXJjIl0sWzAsMCwiKENcXG90aW1lc1xcezFcXH0pXlxcY2lyYyJdLFsxLDEsIkNeXFxjaXJjXFxvdGltZXNbMV0iXSxbMiwxLCJDXlxcY2lyY1xcb3RpbWVzXFx7MVxcfSJdLFswLDEsIkNeXFxjaXJjXFxvdGltZXNcXHswXFx9Il0sWzAsMywiXFxzaW0iXSxbMSw0LCJcXHNpbSJdLFsyLDUsIlxcc2ltIl0sWzEsMF0sWzIsMF0sWzUsM10sWzQsM11d
\[\begin{tikzcd}
	{(C\otimes\{1\})^\circ} & {(C\otimes[1])^\circ} & {(C\otimes\{0\})^\circ} \\
	{C^\circ\otimes\{0\}} & {C^\circ\otimes[1]} & {C^\circ\otimes\{1\}}
	\arrow["\sim", from=1-2, to=2-2]
	\arrow["\sim", from=1-3, to=2-3]
	\arrow["\sim", from=1-1, to=2-1]
	\arrow[from=1-3, to=1-2]
	\arrow[from=1-1, to=1-2]
	\arrow[from=2-1, to=2-2]
	\arrow[from=2-3, to=2-2]
\end{tikzcd}\]
where all vertical arrows are equivalences. There is an invertible natural transformation
$$ C\star 1\sim (1\costar C^{\circ})^\circ.$$
\end{cor}
\begin{proof}
As these functors preserve colimits, we can define this equivalence on representables. As cylinders (resp. cone) (resp. $\circ$-cone) of representables are strict according to theorem \ref{prop:strict stuff are pushout}, and as $(\uvar)^\circ$ preserves strict objects, it is enough to show these equivalences in $\zocat$, where it follows from \cite[proposition A.22]{Ara_Maltsiniotis_joint_et_tranche}.
\end{proof}


\begin{cor}
\label{cor:ominus et op}
Let $A$ and $B$ two $\io$-categories. There is an equivalence  
$$(A\ominus B)^\circ \sim A^\circ\ominus B^\circ$$
natural in $A$ and $B$.
\end{cor}
\begin{proof}
It is sufficient to construct the equivalence when $A$ is a globular sum $a$ and $B$ is of shape $[b,n]$. 
Remark first that the corollary \ref{cor: a otimes n is strict} implies that $(a\otimes[n])^\circ$ and $a^\circ\otimes[n]^\circ$ are strict objects. The proposition A.22 of \cite{Ara_Maltsiniotis_joint_et_tranche}  then implies that these two objects are isomorphic.  The results then directly follows from the definition of the operation $\ominus$ and from the equivalence $(m_b(\uvar))^\circ\sim m_{b^\circ}((\uvar)^\circ)$.
\end{proof}




\begin{cor}
\label{cor:characterisaiont of Gray operation}
Let $F$ be an endofunctor of $\ocat$ such that the induced functor $\ocat\to \ocat_{F(\emptyset)/}$ is colimit preserving, and $\psi$ is an invertible natural transformation between $\Gb^{+}\to \ocat\xrightarrow{F}\ocat$ and $\Gb^{+}\to \ocat\xrightarrow{H}\ocat$ where $\Gb^{+}$ is obtained from $\Gb$ by adding an initial element $\{\emptyset\}$, and $H$ is either the Gray cylinder, the Gray cone, the Gray $\circ$-cone or an iterated suspension.

Then, the natural transformation $\psi$ can be extended to an invertible natural transformation between $F$ and $H$.
\end{cor}
\begin{proof}
We denote by $\Theta^+$ the category obtained from $\Theta$ by adding an initial element $\emptyset$. 
Remark first that the theorem \ref{theo:appendince unicity of operation} implies that we have an invertible natural transformation 
$$\pi_0 \circ F_{|\Theta^+}\to \pi_0 \circ H_{|\Theta^+}.$$
The propositions \ref{prop:strict stuff are stable under Gray cone}, \ref{prop:strict stuff are stable under coproduc with cylinder} and \ref{prop:suspension preserves stricte} imply that the canonical morphism 
$$H_{|\Theta^+}\to \N\circ \pi_0\circ 	H_{|\Theta^+}$$ is an equivalence. The two previous morphisms then induce a comparison:
$$F_{|\Theta^+}\to \N\circ \pi_0 \circ F_{|\Theta^+}\to H_{|\Theta^+}$$
By extension by colimits, this produces a natural transformation $\phi:F\to H$ extending $\psi$. The full sub $\infty$-groupoid of objects $C$ such that $\phi_C:F(C)\to H(C)$ is an equivalence is closed by colimits, contains globes, and so is the maximal sub 	$\infty$-groupoid.
\end{proof}
The previous corollary implies that the equations \eqref{eq:eq for cylinder}, \eqref{eq:eq for Gray cone} and \eqref{eq:eq for cojoin} characterize respectively the Gray cylinder, the Gray cone, and the Gray $\circ$-cone. 

\begin{cor}
\label{cor:crushing of Gray tensor is identitye}
The colimit preserving endofunctor $F:\ocat\to \ocat$, sending $[a,n]$ to the colimit of the span
$$\coprod_{k\leq n}\{k\}\leftarrow \coprod_{k\leq n}a\otimes\{k\}\to a\otimes[n]$$
is equivalent to the identity.
\end{cor}
\begin{proof}
The proposition \ref{prop:fibers of 1 star a} implies that the restriction of $F$ to globes is equivalent to the restriction of the identity to globes. As the identity is the $0$-iterated suspension, we can apply corollary \ref{cor:characterisaiont of Gray operation}.
\end{proof}

The last corollary implies that for any $\io$-category $C$ and any globular sum $a$, the simplicial $\infty$-groupoid
$$\begin{array}{rcl}
\Delta^{op}&\to &\igrd\\
~[n]~&\mapsto &\Hom([a,n],C)
\end{array} $$
is a $\iun$-category.



\begin{theorem}
\label{theo:formula between pullback of slice and tensor}
Let $C$ be an $\io$-category. The two following canonical squares are cartesian:
% https://q.uiver.app/#q=WzAsOCxbMSwxLCJbQywxXSJdLFsxLDAsIjFcXGNvc3RhciBDIl0sWzAsMSwiXFx7MFxcfSJdLFswLDAsIjEiXSxbMywxLCJbQywxXSJdLFszLDAsIkNcXHN0YXIgMSJdLFsyLDEsIlxcezFcXH0iXSxbMiwwLCIxIl0sWzMsMV0sWzIsMF0sWzMsMl0sWzEsMF0sWzcsNV0sWzYsNF0sWzcsNl0sWzUsNF1d
\[\begin{tikzcd}
	1 & {1\costar C} & 1 & {C\star 1} \\
	{\{0\}} & {[C,1]} & {\{1\}} & {[C,1]}
	\arrow[from=1-1, to=1-2]
	\arrow[from=2-1, to=2-2]
	\arrow[from=1-1, to=2-1]
	\arrow[from=1-2, to=2-2]
	\arrow[from=1-3, to=1-4]
	\arrow[from=2-3, to=2-4]
	\arrow[from=1-3, to=2-3]
	\arrow[from=1-4, to=2-4]
\end{tikzcd}\]
The five squares appearing in the following canonical diagram are both cartesian and cocartesian:
% https://q.uiver.app/#q=WzAsOCxbMSwyLCIxXFxjb3N0YXIgQyJdLFsyLDIsIltDLDFdIl0sWzIsMSwiQ1xcc3RhciAxIl0sWzEsMSwiQ1xcb3RpbWVzWzFdIl0sWzIsMCwiMSJdLFsxLDAsIkNcXG90aW1lc1xcezBcXH0iXSxbMCwxLCJDXFxvdGltZXNcXHsxXFx9Il0sWzAsMiwiMSJdLFsyLDFdLFswLDFdLFszLDBdLFszLDJdLFs1LDRdLFs0LDJdLFs1LDNdLFs2LDNdLFs3LDBdLFs2LDddXQ==
\[\begin{tikzcd}
	& {C\otimes\{0\}} & 1 \\
	{C\otimes\{1\}} & {C\otimes[1]} & {C\star 1} \\
	1 & {1\costar C} & {[C,1]}
	\arrow[from=2-3, to=3-3]
	\arrow[from=3-2, to=3-3]
	\arrow[from=2-2, to=3-2]
	\arrow[from=2-2, to=2-3]
	\arrow[from=1-2, to=1-3]
	\arrow[from=1-3, to=2-3]
	\arrow[from=1-2, to=2-2]
	\arrow[from=2-1, to=2-2]
	\arrow[from=3-1, to=3-2]
	\arrow[from=2-1, to=3-1]
\end{tikzcd}\]
\end{theorem}
\begin{proof}
The five squares of the second diagram are cocartesian by construction.

If $C$ is empty, all the considered squares are cartesian. We can then suppose that there exists a globular sum $a$, and a morphism $a\to C$. We claim that the two following squares are cartesian.
% https://q.uiver.app/#q=WzAsNixbMiwxLCJbQywxXSJdLFsyLDAsIkNcXHN0YXIgMSJdLFswLDEsIlxcezBcXH1cXGNvcHJvZCBcXHsxXFx9Il0sWzAsMCwiQ1xcY29wcm9kIDEiXSxbMSwwLCJDXFxjb3Byb2RfYSBhXFxzdGFyIDEiXSxbMSwxLCJbYSwxXSJdLFszLDJdLFsxLDBdLFszLDRdLFs0LDFdLFsyLDVdLFs1LDBdLFs0LDVdXQ==
\[\begin{tikzcd}
	{C\coprod 1} & {C\coprod_a a\star 1} & {C\star 1} \\
	{\{0\}\coprod \{1\}} & {[a,1]} & {[C,1]}
	\arrow[from=1-1, to=2-1]
	\arrow[from=1-3, to=2-3]
	\arrow[from=1-1, to=1-2]
	\arrow[from=1-2, to=1-3]
	\arrow[from=2-1, to=2-2]
	\arrow[from=2-2, to=2-3]
	\arrow[from=1-2, to=2-2]
\end{tikzcd}\]
The cartesianess of the left square is a consequence of proposition \ref{prop:fibers of 1 star a} and of the fact that 
$\{0\}\to [a,1]$ and $\{1\}\to [a,1]$ are discrete Conduché functors and so pullback along them preserves colimits. The cartesianess of the right square is a consequence of the preservation of Gray operations by the full duality stated in corollary \ref{cor:otimes et op}, and of the cartesian square provided by corollary \ref{cor:cor of the past10}.
The two following squares are then cartesian:
% https://q.uiver.app/#q=WzAsOCxbMSwxLCJbQywxXSJdLFsxLDAsIkNcXHN0YXIxIl0sWzAsMCwiMSJdLFswLDEsIlxcezFcXH0iXSxbMiwxLCJcXHswXFx9Il0sWzMsMCwiQ1xcc3RhcjEiXSxbMywxLCJbQywxXSJdLFsyLDAsIkMiXSxbMSwwXSxbNyw0XSxbNCw2XSxbNSw2XSxbNyw1XSxbMiwxXSxbMiwzXSxbMywwXV0=
\[\begin{tikzcd}
	1 & C\star1 & C & C\star1 \\
	{\{1\}} & {[C,1]} & {\{0\}} & {[C,1]}
	\arrow[from=1-2, to=2-2]
	\arrow[from=1-3, to=2-3]
	\arrow[from=2-3, to=2-4]
	\arrow[from=1-4, to=2-4]
	\arrow[from=1-3, to=1-4]
	\arrow[from=1-1, to=1-2]
	\arrow[from=1-1, to=2-1]
	\arrow[from=2-1, to=2-2]
\end{tikzcd}\]
As the duality $(\uvar)^{\circ}$ preserves limits, and combined with corollary \ref{cor:otimes et op}, this implies that the two following squares are also cartesian:
% https://q.uiver.app/#q=WzAsOCxbMSwxLCJbQywxXSJdLFsxLDAsIjFcXGNvc3RhciBDIl0sWzAsMCwiMSJdLFswLDEsIlxcezBcXH0iXSxbMiwxLCJcXHsxXFx9Il0sWzMsMCwiMVxcY29zdGFyIEMiXSxbMywxLCJbQywxXSJdLFsyLDAsIkMiXSxbMSwwXSxbNyw0XSxbNCw2XSxbNSw2XSxbNyw1XSxbMiwxXSxbMiwzXSxbMywwXV0=
\[\begin{tikzcd}
	1 & {1\costar C} & C & {1\costar C} \\
	{\{0\}} & {[C,1]} & {\{1\}} & {[C,1]}
	\arrow[from=1-2, to=2-2]
	\arrow[from=1-3, to=2-3]
	\arrow[from=2-3, to=2-4]
	\arrow[from=1-4, to=2-4]
	\arrow[from=1-3, to=1-4]
	\arrow[from=1-1, to=1-2]
	\arrow[from=1-1, to=2-1]
	\arrow[from=2-1, to=2-2]
\end{tikzcd}\]


By stability by right cancellation of cartesian square, it remains to show that the square
% https://q.uiver.app/#q=WzAsNCxbMCwxLCIxXFxjb3N0YXIgQyJdLFsxLDEsIltDLDFdIl0sWzEsMCwiQ1xcc3RhciAxIl0sWzAsMCwiQ1xcb3RpbWVzWzFdIl0sWzMsMl0sWzAsMV0sWzIsMV0sWzMsMF1d
\[\begin{tikzcd}
	{C\otimes[1]} & {C\star 1} \\
	{1\costar C} & {[C,1]}
	\arrow[from=1-1, to=1-2]
	\arrow[from=2-1, to=2-2]
	\arrow[from=1-2, to=2-2]
	\arrow[from=1-1, to=2-1]
\end{tikzcd}\]
is cartesian. Consider the two following squares
% https://q.uiver.app/#q=WzAsNixbMCwxLCIxXFxjb3N0YXIgYSJdLFswLDAsIkNcXGNvcHJvZF9hIGFcXG90aW1lc1sxXSJdLFsyLDEsIltDLDFdIl0sWzIsMCwiQ1xcc3RhciAxIl0sWzEsMSwiW2EsMV0iXSxbMSwwLCJDXFxjb3Byb2RfYWFcXHN0YXIgMSJdLFsxLDBdLFszLDJdLFs1LDRdLFs0LDJdLFswLDRdLFsxLDVdLFs1LDNdXQ==
\[\begin{tikzcd}
	{C\coprod_a a\otimes[1]} & {C\coprod_aa\star 1} & {C\star 1} \\
	{1\costar a} & {[a,1]} & {[C,1]}
	\arrow[from=1-1, to=2-1]
	\arrow[from=1-3, to=2-3]
	\arrow[from=1-2, to=2-2]
	\arrow[from=2-2, to=2-3]
	\arrow[from=2-1, to=2-2]
	\arrow[from=1-1, to=1-2]
	\arrow[from=1-2, to=1-3]
\end{tikzcd}\]	
We already demonstrate that the right one is cartesian and the lemma \ref{lemma:an other canonical square} states that the left one is also cartesian. The outer square is then cartesian.


Using that pulling back along $C\star 1\to [C,1]$ preserves colimits as shown in corollary \ref{cor:cor of the past3}, and the fact that $1\costar C$ (resp. $C\otimes[1]$) is the colimit of all the $1\costar a$ (resp. $a\otimes[1]$) for $a$ ranging over the morphisms $a\to C$, this concludes the proof.
\end{proof}

\begin{theorem}
\label{theo:strictness}
If $C$ is strict, so are $C\star 1$, $1\costar C$ and $C\otimes [1]$.
\end{theorem}
\begin{proof}
Forgetting the marking, the theorem \ref{theo:equivalence betwen slice and join} implies that $1\costar C$ is equivalent to $[C,1]_{0/}$ which is strict as $[C,1]$ is according to proposition \ref{prop:suspension preserves stricte}. The second assertion comes from the fact that the full duality preserves $\zo$-categories and that $1\costar C^{\circ} \sim (C\star 1)^{\circ}$.


The theorem \ref{theo:formula between pullback of slice and tensor} implies that we have a cartesian square
% q.uiver.app/#q=WzAsNCxbMSwwLCIxXFxjb3N0YXIgQyJdLFsxLDEsIltDLDFdIl0sWzAsMSwiQ1xcc3RhciAxIl0sWzAsMCwiQ1xcb3RpbWVzIFsxXSJdLFszLDJdLFszLDBdLFswLDFdLFsyLDFdXQ==
\[\begin{tikzcd}
	{C\otimes [1]} & {1\costar C} \\
	{C\star 1} & {[C,1]}
	\arrow[from=1-1, to=2-1]
	\arrow[from=1-1, to=1-2]
	\arrow[from=1-2, to=2-2]
	\arrow[from=2-1, to=2-2]
\end{tikzcd}\]
 As strict objects are stable under pullbacks, this concludes the proof.
\end{proof}

%
%
%%\bibliography{../../header/biblio}{}
%\bibliographystyle{alpha}
%\printindex[notation]
%\printindex
%\printindex[notion]
%\end{document}
