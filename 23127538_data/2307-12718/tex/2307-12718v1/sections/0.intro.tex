\section{Introduction}
\label{sec:introduction}

Recent advances in Neural Radiance Fields (NeRFs)~\cite{mildenhall2021nerf} strongly improved the fidelity of generated novel views by fitting a neural network to predict the volume density and the emitted radiance of each 3D point in a scene. The differentiable volume rendering step allows having a set of images, with known camera poses, as the only input for model fitting. Moreover, the limited amount of data, \ie~(image, camera pose) pairs, needed to train a NeRF model, facilitates its adoption and drives the increasing range of its possible applications. Among these, view synthesis recently emerged for street view reconstruction~\cite{muller2022autorf,xie2023snerf} in the context of AR/VR applications, robotics, and autonomous driving, with considerable efforts towards vehicle novel view generation. However, these attempts focus on images representing large-scale unbounded scenes, such as those from KITTI~\cite{geiger2012we}, and usually fail to achieve high-quality 3D vehicle reconstruction.

In this paper, we introduce an additional use case for neural radiance fields, \ie~\textit{vehicle inspection}, where the goal is to represent an individual high-quality instance of a given car. The availability of a high-fidelity 3D vehicle representation could be beneficial whenever the car body has to be analyzed in detail. For instance, insurance companies or body shops could rely on NeRF-generated views to assess possible external damages after a road accident and estimate their repair cost. Moreover, rental companies could compare two NeRF models trained before and after each rental, respectively, to assign responsibility for any new damages. This would avoid expert on-site inspection or a rough evaluation based on a limited number of captures.

For this purpose, we provide an experimental overview of the state-of-the-art NeRF methods, suitable for vehicle reconstruction. To make the experimental setting reproducible and to provide a basis for new experimentation, we propose \textit{\ours}, a new benchmark to assess neural radiance field methods on the~\textit{vehicle inspection} task. Specifically, we generate a novel dataset consisting of 8 different synthetic scenes, corresponding to as many high-quality 3D car meshes with realistic details and challenging light conditions. As depicted in Fig.~\ref{fig:first_page}, we provide not only RGB images with camera poses, but also binary masks of different car components to validate the reconstruction quality of specific vehicle parts (\eg~wheels or windows). Moreover, for each camera position, we generate the ground truth depth map with the double goal of examining the ability of NeRF architectures to correctly predict volume density and, at the same time, enable future works based on RGB-D inputs. We evaluate the novel view generation and depth estimation performance of several methods under diverse settings (both global and component-level). Finally, since the process of image collection for fitting neural radiance fields could be time consuming in real scenarios, we provide the same scenes by varying the number of training images, in order to determine the robustness to the amount of training data.

After an overview of the main related works in Sec.~\ref{sec:related}, we thoroughly describe the process of 3D mesh gathering, scene setup, and dataset generation in Sec.~\ref{sec:dataset}. The evaluation of existing NeRF architectures on \textit{\ours} is presented in Sec.~\ref{sec:benchmark}.