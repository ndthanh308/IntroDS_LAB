\documentclass[oneside,english,12pt]{amsart}
\usepackage{lmodern}
\usepackage[T1]{fontenc}       			 % Text encoding, output
\usepackage[latin9]{inputenc}  			 % Input accented characters
\usepackage[english]{babel}			 % u.a. correct hyphenation -, --
\usepackage[top=3cm, bottom=3cm, left=3cm, right=3cm, heightrounded, marginparwidth=2.5cm, marginparsep=6mm]{geometry}

\usepackage{setspace} % Spacing
	\setstretch{1.2} 	 
	\setlength{\parskip}{4pt}
	\setlength{\parindent}{0pt}
\usepackage{marginnote}
\usepackage{verbatim}
\usepackage{float}
\usepackage{mathrsfs}	
\usepackage{amstext}
\usepackage{amsthm}
\usepackage{amssymb}
\usepackage{amsopn} 
\usepackage{bbold}
\usepackage{stix}
\usepackage{enumitem}				% Enumeration
\usepackage[all]{xy}    				% Commutative diagrams
\usepackage{mathdots}   				% Variously oriented dots
\usepackage{aliascnt}   				% Single enumeration of all theorem environments
\usepackage{esint}					% Contains \fint
\usepackage{tikz}  \usetikzlibrary{matrix}	% Vector images
\usepackage[backref=page]{hyperref}
\renewcommand*{\backref}[1]{}
\renewcommand*{\backrefalt}[4]{%
    \ifcase #1 (Not cited.)%
    \or        (Cited on page~#2.)%
    \else      (Cited on pages~#2.)%
    \fi}
\usepackage{mathdots}
\hypersetup{
    colorlinks,
    linkcolor={red!50!black},
    citecolor={green!50!black},
    urlcolor={blue!80!black},
    linktocpage
}
\usepackage{soul}  % use \hl to highlight
\usepackage{longtable} % page break for long tables
\usepackage{lipsum}
\usepackage{footnote}
\makesavenoteenv{tabular}

\usepackage{multirow}
\usepackage{longtable} 				% page break for long tables
\allowdisplaybreaks


\renewcommand{\descriptionlabel}[1]{%
  \hspace\labelsep \upshape\bfseries #1%
}

\setcounter{tocdepth}{1}

% Theorems
\numberwithin{equation}{section}
\numberwithin{figure}{section}

\theoremstyle{theorem}
  \newtheorem*{cor*}{Corollary}
  \newtheorem*{thm*}{Theorem}
  \newtheorem*{lem*}{Lemma}
  \newtheorem*{claim*}{Claim}
  \newtheorem*{mclaim*}{Main Claim}
  
  \newtheorem{thmx}{Theorem}
  \renewcommand{\thethmx}{\Alph{thmx}}
  \providecommand{\thmxautorefname}{Theorem}
  
  \newaliascnt{corx}{thmx}
  \newtheorem{corx}[corx]{Corollary}
  \aliascntresetthe{corx}
  \providecommand{\corxautorefname}{Corollary}
  
  \newtheorem{thm}{Theorem}[section]
  \providecommand{\thmautorefname}{Theorem}
  
  \newaliascnt{lem}{thm}
  \newtheorem{lem}[lem]{Lemma}
  \aliascntresetthe{lem}
  \providecommand{\lemautorefname}{Lemma}
  
  \newaliascnt{klem}{thm}
  \newtheorem{klem}[klem]{Key Lemma}
  \aliascntresetthe{klem}
  \providecommand{\klemautorefname}{Key Lemma}

  \newaliascnt{cor}{thm}
  \newtheorem{cor}[cor]{Corollary}
  \aliascntresetthe{cor}
  \providecommand{\corautorefname}{Corollary}

  \newaliascnt{prop}{thm}  
  \newtheorem{prop}[prop]{Proposition}
  \aliascntresetthe{prop}
  \providecommand{\propautorefname}{Proposition}

\theoremstyle{definition}
  \newaliascnt{defn}{thm}
  \newtheorem{defn}[defn]{Definition}
  \aliascntresetthe{defn}
  \providecommand{\defnautorefname}{Definition}
  
  \newaliascnt{exmpl}{thm}
  \newtheorem{exmpl}[exmpl]{Example}
  \aliascntresetthe{exmpl}
  \providecommand{\exmplautorefname}{Example}
  
\theoremstyle{remark}
  \newaliascnt{rem}{thm}
  \newtheorem{rem}[rem]{Remark}
  \aliascntresetthe{rem}
  \providecommand{\remautorefname}{Remark}
  
  \theoremstyle{remark}
  \newaliascnt{set}{thm}
  \newtheorem{set}[set]{Setting}
  \aliascntresetthe{set}
  \providecommand{\setautorefname}{Setting}
  
  \theoremstyle{remark}
  \newaliascnt{con}{thm}
  \newtheorem{con}[con]{Construction}
  \aliascntresetthe{con}
  \providecommand{\conautorefname}{Construction}
  
% Blackboard letters
\newcommand{\bbA}{\mathbb{A}}
\newcommand{\bbB}{\mathbb{B}}
\newcommand{\bbC}{\mathbb{C}}
\newcommand{\bbD}{\mathbb{D}}
\newcommand{\bbE}{\mathbb{E}}
\newcommand{\bbF}{\mathbb{F}}
\newcommand{\bbG}{\mathbb{G}}
\newcommand{\bbH}{\mathbb{H}}
\newcommand{\bbI}{\mathbb{I}}
\newcommand{\bbJ}{\mathbb{J}}
\newcommand{\bbK}{\mathbb{K}}
\newcommand{\bbL}{\mathbb{L}}
\newcommand{\bbM}{\mathbb{M}}
\newcommand{\bbN}{\mathbb{N}}
\newcommand{\bbO}{\mathbb{O}}
\newcommand{\bbP}{\mathbb{P}}
\newcommand{\bbQ}{\mathbb{Q}}
\newcommand{\bbR}{\mathbb{R}}             
\newcommand{\bbS}{\mathbb{S}}
\newcommand{\bbT}{\mathbb{T}}
\newcommand{\bbU}{\mathbb{U}}
\newcommand{\bbV}{\mathbb{V}}
\newcommand{\bbW}{\mathbb{W}}
\newcommand{\bbX}{\mathbb{X}}
\newcommand{\bbY}{\mathbb{Y}}
\newcommand{\bbZ}{\mathbb{Z}}      
\newcommand{\bbone}{\mathbb{1}} 

% Calligraphic letters
\newcommand{\clA}{\mathcal{A}}
\newcommand{\clB}{\mathcal{B}}
\newcommand{\clC}{\mathcal{C}}
\newcommand{\clD}{\mathcal{D}}
\newcommand{\clE}{\mathcal{E}}
\newcommand{\clF}{\mathcal{F}}
\newcommand{\clG}{\mathcal{G}}
\newcommand{\clH}{\mathcal{H}}
\newcommand{\clI}{\mathcal{I}}
\newcommand{\clJ}{\mathcal{J}}
\newcommand{\clK}{\mathcal{K}}
\newcommand{\clL}{\mathcal{L}}
\newcommand{\clM}{\mathcal{M}}
\newcommand{\clN}{\mathcal{N}}
\newcommand{\clO}{\mathcal{O}}
\newcommand{\clP}{\mathcal{P}}
\newcommand{\clQ}{\mathcal{Q}}
\newcommand{\clR}{\mathcal{R}}             
\newcommand{\clS}{\mathcal{S}}
\newcommand{\clT}{\mathcal{T}}
\newcommand{\clU}{\mathcal{U}}
\newcommand{\clV}{\mathcal{V}}
\newcommand{\clW}{\mathcal{W}}
\newcommand{\clX}{\mathcal{X}}
\newcommand{\clY}{\mathcal{Y}}
\newcommand{\clZ}{\mathcal{Z}}  

% Fraktur letters
\newcommand{\frA}{\mathfrak{A}}
\newcommand{\frB}{\mathfrak{B}}
\newcommand{\frC}{\mathfrak{C}}
\newcommand{\frD}{\mathfrak{D}}
\newcommand{\frE}{\mathfrak{E}}
\newcommand{\frF}{\mathfrak{F}}
\newcommand{\frG}{\mathfrak{G}}
\newcommand{\frH}{\mathfrak{H}}
\newcommand{\frI}{\mathfrak{I}}
\newcommand{\frJ}{\mathfrak{J}}
\newcommand{\frK}{\mathfrak{K}}
\newcommand{\frL}{\mathfrak{L}}
\newcommand{\frM}{\mathfrak{M}}
\newcommand{\frN}{\mathfrak{N}}
\newcommand{\frO}{\mathfrak{O}}
\newcommand{\frP}{\mathfrak{P}}
\newcommand{\frQ}{\mathfrak{Q}}
\newcommand{\frR}{\mathfrak{R}}             
\newcommand{\frS}{\mathfrak{S}}
\newcommand{\frT}{\mathfrak{T}}
\newcommand{\frU}{\mathfrak{U}}
\newcommand{\frV}{\mathfrak{V}}
\newcommand{\frW}{\mathfrak{W}}
\newcommand{\frX}{\mathfrak{X}}
\newcommand{\frY}{\mathfrak{Y}}
\newcommand{\frZ}{\mathfrak{Z}}

% Script letters
\newcommand{\scA}{\mathscr{A}}
\newcommand{\scB}{\mathscr{B}}
\newcommand{\scC}{\mathscr{C}}
\newcommand{\scD}{\mathscr{D}}
\newcommand{\scE}{\mathscr{E}}
\newcommand{\scF}{\mathscr{F}}
\newcommand{\scG}{\mathscr{G}}
\newcommand{\scH}{\mathscr{H}}
\newcommand{\scI}{\mathscr{I}}
\newcommand{\scJ}{\mathscr{J}}
\newcommand{\scK}{\mathscr{K}}
\newcommand{\scL}{\mathscr{L}}
\newcommand{\scM}{\mathscr{M}}
\newcommand{\scN}{\mathscr{N}}
\newcommand{\scO}{\mathscr{O}}
\newcommand{\scP}{\mathscr{P}}
\newcommand{\scQ}{\mathscr{Q}}
\newcommand{\scR}{\mathscr{R}}             
\newcommand{\scS}{\mathscr{S}}
\newcommand{\scT}{\mathscr{T}}
\newcommand{\scU}{\mathscr{U}}
\newcommand{\scV}{\mathscr{V}}
\newcommand{\scW}{\mathscr{W}}
\newcommand{\scX}{\mathscr{X}}
\newcommand{\scY}{\mathscr{Y}}
\newcommand{\scZ}{\mathscr{Z}}   
\newcommand{\kay}{\mathscr{k}}

% Operators
\DeclareMathOperator{\Aut}{Aut}
\DeclareMathOperator{\Alt}{Alt}
\DeclareMathOperator{\Char}{char}
\DeclareMathOperator{\dd}{d}
\DeclareMathOperator{\EE}{E}
\DeclareMathOperator{\ev}{ev}
\DeclareMathOperator{\coker}{coker}
\DeclareMathOperator{\Gr}{Gr}    
\DeclareMathOperator{\id}{id}          
\DeclareMathOperator{\HH}{H}
\DeclareMathOperator{\Hom}{Hom}
\DeclareMathOperator{\im}{im}
\DeclareMathOperator{\Ind}{Ind}
\DeclareMathOperator{\Int}{Int}
\DeclareMathOperator{\lk}{lk}
\DeclareMathOperator{\LL}{{\rm L}}
\DeclareMathOperator{\Prob}{Prob}
\DeclareMathOperator{\res}{res}
\DeclareMathOperator{\sgn}{sgn}
\DeclareMathOperator{\Span}{span} 
\DeclareMathOperator{\stab}{Stab}
\DeclareMathOperator{\St}{st}
\DeclareMathOperator{\Sub}{Sub}
\DeclareMathOperator{\supp}{supp}
\DeclareMathOperator{\Sym}{Sym}
\DeclareMathOperator{\tr}{tr}

\newcommand{\Hb}{{\rm H}_{\rm b}}
\newcommand{\tHb}{\tilde{\rm H}_{\rm b}}
\newcommand{\Hc}{{\rm H}_{\rm c}}
\newcommand{\Linfty}{L^\infty}
\newcommand{\Linftya}{L^\infty_{\rm alt}}
\newcommand{\Linftyw}{L^\infty_{\rm w^\ast}}
\newcommand{\Linftywa}{L^\infty_{\rm w^\ast,alt}}
\newcommand{\Binfty}{\scL^\infty}
\newcommand{\Binftya}{\scL^\infty_{\rm alt}}
\newcommand{\bcdot}{{\scriptscriptstyle \bullet}}
\newcommand{\bind}{{\bf ind}}
\newcommand{\IE}{{}^{\rm I}{\rm E}}
\newcommand{\Id}{{}^{\rm I}{\rm d}}
\newcommand{\IIE}{{}^{\rm II}{\rm E}}
\newcommand{\IId}{{}^{\rm II}{\rm d}}
\newcommand{\bbslash}{\backslash \!\! \backslash}
\newcommand{\Xnk}{X_n^{(k)}}
\newcommand{\BX}{\bar{X}}
\newcommand{\BXnk}{\bar{X}_n^{(k)}}
\newcommand{\BXrk}{\bar{X}_r^{(k)}}
\renewcommand{\o}{\overline}
\newcommand{\C}{\mathbb C}
\newcommand{\mres}{\, \mathbin{\vrule height 1.6ex depth 0pt width
0.13ex\vrule height 0.13ex depth 0pt width 1.3ex}}

% Matrix groups
\DeclareMathOperator{\GL}{GL}         % general linear group
\DeclareMathOperator{\SL}{SL}         % special linear group
\DeclareMathOperator{\PSL}{PSL}       % projective special linear group
\DeclareMathOperator{\OO}{O}               % orthogonal group
\DeclareMathOperator{\SO}{SO}             % special orthogonal group
\DeclareMathOperator{\UU}{U}               % unitary group
\DeclareMathOperator{\SU}{SU}             % special unitary group
\DeclareMathOperator{\Sp}{Sp}              % symplectic group

% And-Or
\newcommand{\qand}{\quad \mathrm{and} \quad}
\newcommand{\qqand}{\qquad \mathrm{and} \qquad}

\AtBeginDocument{\addtocontents{toc}{\protect\setlength{\parskip}{0pt}}}

\begin{document}

\title[]{A Quillen stability criterion for bounded cohomology}
% Author I information
\author{Carlos De la Cruz Mengual}
\address{Faculty of Electrical and Computer Engineering \\ Technion, Haifa, Israel}
\email{c.delacruz@technion.ac.il}

% Author II information
\author{Tobias Hartnick}
\address{Institute of Algebra and Geometry \\
KIT, Karlsruhe, Germany}
\email{tobias.hartnick@kit.edu}

\begin{abstract}
We provide a version of Quillen's homological stability criterion for continuous bounded cohomology. This criterion is exploited in the companion paper \cite{DM+Hartnick} in order to derive new bounded cohomological stability results for various families of classical groups.
\end{abstract}

\maketitle


\section{Introduction}
%\subsection{Quillen's homological stability criterion}
An infinite chain
$
	(G_r)_{r\geq 0} = (G_0 < G_1 < G_2 < \cdots)
$ 
of groups is called \emph{homologically stable} if there exists a function\footnote{As usual in homological algebra, we enumerate starting from $0$. Accordingly, we
convene that $0 \in \bbN$, and for any $k \in \bbN$, we denote by $[k]$ the $(k+1)$-element set $\{0,\ldots,k\} \subset \bbN$. We also set $[\infty] := \bbN$.} $r: \bbN \to \bbN$ such that the respective inclusions induce isomorphisms
\[
\HH_q(G_{r(q)}) \ \cong \ \HH_q(G_{r(q)+1}) \ \cong \ \HH_q(G_{r(q)+2}) \ \cong \ \cdots   
\]
in group homology for all $q \in \bbN$. Any such function $r$ is then called a \emph{stability range} for the family. Homological stability (with a linear stability range) has been established for many families of classical linear algebraic groups, but also for various families of non-linear groups of interest such as mapping class groups or automorphism groups of free groups. 

A particularly succesful method to establish homological stability, employed for example in \cite{vdK, Harer, HatVogt, Essert,Sprehn-Wahl2}, is based on \emph{Quillen's stability criterion}, which can be stated as follows: Suppose that, for every $r \in \bbN$, we are given a $\Delta$-complex $X(r)$ (also known as semi-simplicial set) endowed with a simplicial $G_r$-action,
%\emph{semi-simplicial}\footnote{A set endowed with a structure that enables the definition of its simplicial (co)homology; see \autoref{def_ssobjects}. Semi-simplicial sets are also known as $\Delta$-complexes, though we will refrain from the use of this terminology.} $G_r$-set $X(r)$ 
and two natural numbers $\gamma(r), \tau(r)$ with the following properties:
\begin{enumerate}[label=(Q\arabic*),leftmargin=2.3pc]
	\item $X(r)$ is \emph{$\gamma(r)$-acyclic}, i.e. the reduced homology $\tilde{\HH}_\bcdot(X(r))$ vanishes up to degree $\gamma(r)$;
	\item $X(r)$ is \emph{$\tau(r)$-transitive}, i.e. there is only one $G_r$-orbit of $l$-simplices for $l \in \{0,\ldots,\tau(r)\}$; 
	\item the complexes are \emph{$\tau(r)$-compatible}, i.e.\ the stabilizer of an $l$-simplex in $X(r)$ is isomorphic to $G_{r-l-1}$ for every $l < \tau(r)$, and these isomorphisms are compatible with inclusions of stabilizers in the obvious way.
\end{enumerate} 
Then $(G_r)_{r \geq 0}$ is homologically stable provided $\min\{\gamma(r), \tau(r)\} \to \infty$ as $r \to \infty$. In addition, a stability range can be computed explicitly from the functions $\gamma$ and $\tau$. See Quillen's unpublished notes \cite{Quillen}, or the more recent treatments \cite{Bestvina,Sprehn-Wahl2} for details and further references. 
%\subsection{Bc-stability}

The purpose of the present article is to establish a similar stability criterion for \emph{continuous bounded cohomology} of \emph{locally compact second countable} (lcsc) \emph{groups}. Here, in analogy with the classical situation, we say that an infinite family $G_0<G_1< G_2 < \cdots$ of lcsc groups is \emph{bc-stable} (short for \emph{bounded-cohomologically stable}) with \emph{stability range} $r: \bbN \to \bbN$ if, for every $q \geq 0$, the respective inclusions induce isomorphisms
\[
\Hb^q(G_{r(q)}) \ \cong \ \Hb^q(G_{r(q)+1}) \ \cong \ \Hb^q(G_{r(q)+2}) \ \cong \ \cdots   
\]
in continuous bounded cohomology. We will give conditions analogous to (Q1) -- (Q3) and show that these imply bc-stability.

The theory of bounded cohomology for discrete groups goes back to Johnson, Trauber, and Gromov \cite{Gromov}, and was later extended to the realm of topological groups by Burger and Monod \cite{Burger-Monod1}; we refer to the monographs \cite{Frigerio, Monod-Book} for background and numerous applications. 
Compared to classical (co-)homology, (continuous) bounded cohomology has a more functional analytic flavour---in fact, it can be defined as a derived functor in some suitable exact category of Banach modules \cite{Buehler}. While it is sometimes possible to extend classical cohomological results to this setting, this usually requires additional efforts in order to take certain functional analytic peculiarities into account. 
\subsection{A spectral sequence in bounded cohomology}
From now on, let $G$ denote a lcsc group; we will refer to a standard Borel space with a Borel $G$-action and a $G$-invariant probability measure class as a \emph{Lebesgue $G$-space}. To formulate versions of Quillen's conditions in the functional analytic setting, we replace $\Delta$-complexes with a simplicial $G$-action (i.e.\ semi-simplicial $G$-sets) by \emph{Lebesgue $G$-complexes} (i.e.\ semi-simplicial objects in the category of Lebesgue $G$-spaces). To a Lebesgue $G$-complex $X = (X_q)_{q \geq 0}$, we can associate a complex
\[
0 \to \mathbb R \to \Linfty(X_0)^G \to \Linfty(X_1)^G \to \Linfty(X_2)^G \to \dots
\]
of Banach spaces, where the codifferentials are given by alternating sums of dual face maps. We then say that $X$ is \emph{boundedly $\gamma_0$-acyclic} if the cohomology of this complex vanishes up to degree $\gamma_0$. We also say that $X$ is \emph{essentially $\tau_0$-transitive} if $G$ acts essentially transitively (i.e.\ with a co-null orbit) on $q$-simplices for $q \in \{0, 1, \dots, \tau_0\}$. In this case, we can choose simplices $o_0 \in X_0, \dots, o_{\tau_0} \in X_{\tau_0}$ such that $o_i$ is a face of $o_{i+1}$ for all $i \in \{0, \dots, \tau_0-1\}$, and such that the orbits of $o_0, \dots, o_{\tau_0}$ are co-null in the respective spaces of simplices (see \autoref{ConGenericFlag}). We then refer to $(o_0, \dots, o_{\tau_0})$ as a \emph{generic flag} in $X$. Our main tool will be the following spectral sequence connecting the continuous bounded cohomology rings of the stabilizers of the $o_i$ to the bounded cohomology of the ambient group:
\begin{thmx}\label{BasicSS} Let $X$ be a Lebesgue $G$-complex, which is boundedly $\gamma_0$-acyclic and essentially $\tau_0$-transitive for some $\gamma_0, \tau_0 \in \mathbb N \cup \{\infty\}$. Moreover, let  $(o_{0}, \dots, o_{\tau_0})$ be a generic flag in $X$ with stabilizers $H_i := \stab_G(o_i)$ for $i \in \{0, \dots \tau_0\}$, set $H_{-1} := G$ and let $\iota_j: H_j \to H_{j-1}$ denote the inclusion maps. Then there exists a first-quadrant spectral sequence $\EE_\bcdot^{\bcdot,\bcdot}$ such that:
\begin{enumerate}[label = \emph{(\roman*)},leftmargin=25pt]
\item $\EE_\bcdot^{\bcdot,\bcdot}$ converges and $\EE_\infty^t = 0$ for every $t \in \{0, \dots, \gamma_0 + 1\}$.
\item $\EE_1^{p,q} = \Hb^q(H_{p-1})$ for all $p \in \{0, \dots \tau_0+1\}$ and $q\geq 0$.
\item $\EE_2^{p,0} = 0 \; \text{for all }p \in \{0, \dots \tau_0+1\}$.
\item The first page differential $\dd_1^{p,q}:  \Hb^q(H_{p-1}) \to \Hb^q(H_{p})$ is given by
\[
\dd_1^{p,q} =  \left\{\!\!\begin{array}{ll}
	\Hb^q(\iota_p)& \mbox{if } p \mbox{ is even,} \\
	0 & \mbox{if } p \mbox{ is odd,}
\end{array}\right.
\]
for all $p \in \{0, \dots, \tau_0\}$ and $q\geq 0$.
\end{enumerate}
\end{thmx}
\subsection{Measurable Quillen families}
Assume that we are given an infinite family $(G_r)_{r\geq 0}=(G_0 < G_1 < G_2 < \cdots)$ of lcsc groups and a Lebesgue $G_r$-complex $X(r)$ for every $r \geq 0$. We can then define measurable counterparts to the Quillen conditions (Q1) and (Q2) as follows. We assume that we are given functions $\gamma, \tau: \mathbb N \to \mathbb N \cup \{\pm \infty\}$ such that
\begin{enumerate}[leftmargin=3.6pc, label=(MQ\arabic*)]
	\item $X(r)$ is boundedly $\gamma(r)$-acyclic.
	\item $X(r)$ is essentially $\tau(r)$-transitive.
\end{enumerate} 
These assumptions imply by \autoref{BasicSS} that, for each $r \geq 0$, we obtain a spectral sequence which relates the continuous bounded cohomology of $G_r$ to the continuous bounded cohomology of the stabilizers $G_r=: H_{r,-1} > H_{r,0} > \dots >H_{r,\tau(r)}$ of a generic flag $(o_{r,0}, \dots, o_{r, \tau(r)})$ in $X(r)$. To obtain a stability result in the vein of Quillen's, we need to relate the bounded cohomology of these stabilizers to the one of the previous groups in the sequence. We offer three different compatibility conditions which are useful in different situations and are all sufficient to obtain bc-stability:
\begin{enumerate}[leftmargin=3.7pc, label=(MQ3\alph*)]
\item $H_{r,q} = G_{r-q-1}$ for all $r \geq 0$ and $q \in \{-1,0, \dots, \tau(r)\}$ (where by convention $G_{k}$ denotes the trivial group for $k<0$).
\item For every $r \geq 0$ and $q \in \{-1,0, \dots, \tau(r)\}$, there exists an epimorphism
$
	\pi_{r,q}: H_{r,q} \twoheadrightarrow G_{r-q-1} 
$
with amenable kernel such that for all $r \geq 0$ and $q \in \{-1,0, \dots, \tau(r)-1\}$, the following diagram commutes:
\begin{equation*}
		\begin{gathered}	
		\xymatrixcolsep{3.5pc}
		\xymatrix@R=13pt{H_{r,q+1} \ar@{^{(}->}[r]^{\iota_{r,q+1}}  \ar@{->>}[d]^{\pi_{r,q+1}} & H_{r,q} \ar@{->>}[d]^{\pi_{r,q}} \\
		G_{r-q-2} \ar@{^{(}->}[r]^{\iota_{r-q-2}} & G_{r-q-1}} \vspace{-5pt}
		\end{gathered}
\end{equation*}
\item For every $r \geq 0$ and $q \in \{-1,0, \dots, \tau(r)\}$, there exist an epimorphism
$
	\pi_{r,q}: H_{r,q} \twoheadrightarrow G_{r-q-1} 
$
with amenable kernel and a continuous homomorphic section $\sigma_{r,q}$ of $\pi_{r,q}$ such that for all $r \geq 0$ and $q \in \{-1,0, \dots \tau(r)-1\}$ the following diagram commutes:
\begin{equation*}
		\begin{gathered}	
		\xymatrixcolsep{3.5pc}
		\xymatrix@R=13pt{H_{r,q+1} \ar@{^{(}->}[r]^{\iota_{r,q+1}} & H_{r,q} \ar@{->>}[d]^{\pi_{r,q}} \\
		G_{r-q-2} \ar[u]^{\sigma_{r,q+1}} \ar@{^{(}->}[r]^{\iota_{r-q-2}} & G_{r-q-1}} 
		\end{gathered}
\end{equation*}
\end{enumerate}
Here, (MQ3a) is Quillen's original condition; (MQ3b) and (MQ3c) are two different relaxations, which take advantage of the fact that continuous surjections with amenable kernel induce isomorphisms in continuous bounded cohomology. The latter is the most technical, but also most useful condition, as witnessed by \autoref{GLnExample} below.
\begin{defn}\label{DefInfiniteQuillen} We say that $(G_r,X(r))_{r \geq 0}$ is a \emph{measurable $(\gamma, \tau)$-Quillen family} if conditions (MQ1) and (MQ2), and either (MQ3a), (MQ3b) or (MQ3c) hold.
\end{defn}
The following theorem is the main qualitative result of this article:
\begin{thmx}[Quillen stability for continuous bounded cohomology]\label{MainThmQual} Assume that  $(G_r,X(r))_{r \geq 0}$ is a measurable $(\gamma, \tau)$-Quillen family. If both $\gamma$ and $\tau$ are proper, then $(G_r)_{r \geq 0}$ is bc-stable.
\end{thmx}
\autoref{MainThmQual} will be established by comparing the various spectral sequences associated to the Lebesgue complexes $X(r)$ by means of \autoref{BasicSS}. This will actually provide a quantitative version of \autoref{MainThmQual}: we will obtain an explicit stability range which depends only on the functions $\gamma$ and $\tau$. This quantitative version actually also makes sense for finite Quillen families. We will state the most general version of our main theorem as \autoref{MainTheorem} below.

For \emph{product complexes}, i.e. complexes of the form $X(r)_q = X(r)_0^{q+1}$ with the forgetful face maps, \autoref{MainThmQual} and its quantitative version were essentially established by Monod in \cite{Monod-Stab}, and our proof is an extension of Monod's original proof. As we will explain after \autoref{GLnExample}, our more general version is crucial if one want to generalize Monod's results concerning bc-stability of general linear groups to other classes of classical groups.

\subsection{The discrete case}
\autoref{MainThmQual} applies in particular to the case in which $(G_r)_{r \geq 0}$ are countable discrete groups. In this case, the statement simplifies considerably, and we record this special case for ease of reference. The sequence $(X_r)_{r \geq 0}$ is just a family of $\Delta$-complexes, and the $G_r$-invariant measure classes on simplices of $X_r$ are all represented by the respective counting measures. To prove (rational) homological stability of $(G_r)_{r \geq 0}$ using Quillen's criterion, it suffices to show that $G_r$ acts $\tau(r)$-transitively on $X_r$ with stabilizers $H_{r,q} = G_{r-q-1}$ for some proper function $\tau: \bbN \to \bbN \cup \{\pm \infty\}$ and that $X_r$ is (rationally) $\gamma(r)$-acyclic for some proper function $\gamma: \bbN \to \bbN \cup \{\pm \infty\}$. One way to do this is by exhibiting a (partial) contracting chain homotopy $h_*$ for the complex
\[
\dots \to \bbQ X_1 \xrightarrow{d_0} \bbQ X_0 \xrightarrow{d_{-1}} \bbQ \to 0
\]
so that $d_k \circ h_k + h_{k-1} \circ d_{k-1}= 1$. This chain homotopy will then induce a (partial) dual contracting chain homotopy of the complex
\[
0 \to \bbR \xrightarrow{d^{-1}} \ell^\infty(X_0) \xrightarrow{d^0} \ell^\infty(X_1) \to \dots
\]
provided that for every $k \in \bbN$ there is a constant $C_k > 0$ such that for
every simplex $\sigma \in X_k$ the $\ell^1$-norm $\|h_k(\sigma)\|_1$ is bounded by $C_k$, i.e.
\[
h_k(\sigma) = \sum_{i=1}^{n_\sigma} \alpha_i \tau_i \implies  \sum_{i=1}^{n_\sigma} |\alpha_i| \leq C_k. 
\]
In this case we refer to $h_*$ as a \emph{rational $\ell^1$-homotopy}.
\begin{corx}\label{Discrete} Let $(G_r)_{r\geq 0}$ be a family of countable groups. For every $r \geq 0$, let $X_r$ be a $\gamma(r)$-acyclic $\Delta$-complex with a $\tau(r)$-transitive $G_r$-action and stabilizers $H_{r,q} = G_{r-q-1}$. 
\begin{enumerate}[label = \emph{(\roman*)},leftmargin=25pt]
\item If $\gamma, \, \tau: \bbN \to \bbN \cup \{\pm \infty\}$ are proper, then $(G_r)_{r \geq 0}$ is homologically stable.
\item If moreover rational $\gamma(r)$-acyclicity of $X_r$ is witnessed by a rational $\ell^1$-homotopy, then $(G_r)_{r \geq 0}$ is also bc-stable.
\end{enumerate}
\end{corx}
Again, the stability range can be computed explicitly in terms of $\gamma$ and $\tau$; see \autoref{MainTheorem}. \autoref{Discrete} provides a clear strategy to establish bc-stability for all families of countable groups for which homological stability has been established using Quillen's method: One just has to check whether the underlying homotopy is rationally $\ell^1$. Unfortunately, the homotopies in question are usually only given implicitly. We thus leave it to future work to determine for which countable groups an explicit rational $\ell^1$-homotopy can be constructed.

\subsection{The case of reductive Lie groups}
Our main interest in establishing \autoref{MainThmQual} in its present generality was to provide a framework for establishing bc-stability of the classical families of reductive Lie groups. For all of these families, bc-stability is predicted by the (notoriously open) isomorphism conjecture in continuous bounded cohomology. However, prior to this work, this predicted stability had only been established for general and special linear groups over $\bbR$ and $\bbC$ by Monod \cite{Monod-Stab} using a special case of \autoref{MainThmQual}. Let us briefly explain how the results for general linear groups\footnote{As explained by Monod in the Note before Lemma 10 in \cite{Monod-Vanish}, the stability range given in \cite{Monod-Stab} is not correct due to an inaccuracy in the induction step. We use this opportunity to state the correct range; see \cite[Sect. 6.3 and 6.4]{DM-Thesis} for details. Also, the quaternionic case is absent from \cite{Monod-Stab}, but it does not require any new ideas.} fit into our present context:
\begin{exmpl}[General linear groups]\label{GLnExample} Let $G_r := \mathrm{GL}_r(\mathbb K)$ with $\bbK \in \{\bbR, \bbC, \bbH\}$. These groups constitute an infinite family $(G_0 < G_1 < G_2 < \cdots )$ of lcsc groups with the block inclusions
\begin{equation} \label{GLn_inclusion}
	\iota_r: G_r \hookrightarrow G_{r+1}, \quad A \mapsto 1 \times A := \begin{pmatrix} 1 & 0 \\ 0 & A \end{pmatrix}.
\end{equation}
Let $X(r)_q := \mathbb P(\mathbb K^r)^{q+1}$, where $\mathbb P(\mathbb K^r)$ denotes the projective space of $\mathbb K^r$ and the face maps are the usual forgetful maps. If we equip each of these spaces with its canonical Lebesgue measure class, then each $X(r)$ becomes a boundedly acyclic Lebesgue $G(r)$-complex. %Indeed, after choosing a probability measure $\mu$ in the Lebesgue measure class of $\bbP(\bbK^r)$, a contracting homotopy for the augmented $\Linfty$-complex is given by integration over the first variable with respect to $\mu$.
Moreover, the action of $G_r$ on $X(r)$ is essentially $r$-transitive. If $e_1, \dots, e_r$ denotes the standard basis of $\bbK^r$ and $e_{r+1} = \sum_{i=1}^r e_i$, then a generic flag in $X(r)$ is given by $(o_0, \dots,o_{r})$, where $o_q := ([e_1], \dots, [e_{q+1}])$. The corresponding stabilizers $H_{r,q} = \stab_{G_r}(o_q)$ are 
\[
H_{r,q} = \left\{\begin{pmatrix} D & V \\ 0 & A \end{pmatrix} \ \Bigg| \ D \in G_{q+1} \mbox{ diagonal}, \ A \in G_{r-q-1}, \ V \in \mathrm{M}_{(q+1) \times (r-q-1)}(\bbK) \right\}
\]
for $q \in [r-1]$, and $H_{r,r} = (\bbK^\times) \cdot I_r$; we also set $H_{r,-1} = G_r$. Conditions (MQ3a) is violated, and while the obvious projections $\pi_{r,q}: H_{r,q} \to G_{r-q-1}$ do have amenable kernel, they do not satisfy Condition (MQ3b) as the corresponding diagram does not commute. However, Condition (MQ3c) is satisfied with the homomorphic sections \vspace{-3pt}
\[
	\sigma_{r,q}:G_{r-q-1} \to H_{r,q}, \quad A \mapsto I_{q+1} \times A, \vspace{-3pt}
\]
of $\pi_{r,q}$; this is the reason why we insist on this slightly technical compatibility condition. 

We may now deduce that $(G_r, X(r))_{r\geq 0}$ is a measurable $(\infty, r)$-Quillen family, hence $(G_r)$ is bc-stable. An explicit stability range can be read off from  \autoref{MainTheorem} below. In the notation of that theorem, we have $q_0 = 2$, $\gamma(r) = \infty$ and $\tau(r) = r$ (since $\Hb^2(G_r) = 0$ for all $r \geq 0$; this follows from \cite{Burger-Monod3} and \cite[Cor. 8.8.6, Ex. 9.9.3]{Monod-Book}), and hence\vspace{-3pt}
\[
	\min\{\widetilde{\gamma}(q,r), \widetilde{\tau}(q,r)-1\} = \min_{j=q_0}^{q} \big\{\tau\big(r+1-2(q-j)\big) - j\big\}-1= r-2q+q_0 = r-(2q-2).\vspace{-3pt}
\]
We deduce that for every $q \geq q_0+1 = 3$, the inclusions $\iota_r$ induce isomorphisms and injections
\[
\dots \cong \Hb^q(\GL_{2q-1}(\bbK)) \cong \Hb^{q}(\GL_{2q-2}(\bbK)) \hookrightarrow \Hb^{q}(\GL_{2q-3}(\bbK)) \hookrightarrow \Hb^{q}(\GL_{2q-4}(\bbK)).
\]
For the general linear groups, we thus obtain bc-stability with slope two.
\end{exmpl}
In the $\mathrm{GL}_r$-case, the underlying complex $X(r)$ can be chosen to be a product complex, since $\GL_r$ acts essentially $r$-transitively on projective space. It is known that reductive Lie groups of type other than $A_n$ do not admit highly transitive actions on generalized flag manifolds \cite{Popov}. Thus the product version of \autoref{MainThmQual} is insufficient to establish bc-stability for other classical groups. This was our main motivation in stating \autoref{MainThmQual} in its present form.

\begin{exmpl}[Symplectic, orthogonal, and unitary groups] \label{Sp2nExample}
For $\bbK \in \{\bbR,\bbC\}$ and $d \in \bbN$, let $(G_r)_{r\geq 0}$ denote one of the following classical families of Lie groups 
\begin{equation*}
\begin{array}{rrrrrr}
	\{1\} <& \!\! \Sp_2(\bbK) <& \!\!\Sp_4(\bbK) <& \!\! \cdots <& \!\! \Sp_{2r}(\bbK) <& \!\! \cdots \\[2pt]
	\OO_{d,\,0}(\bbK) <& \!\! \OO_{d+1,\,1}(\bbK) <& \!\! \OO_{d+2,\,2}(\bbK) <& \!\! \cdots <& \!\! \OO_{d+r,\,r}(\bbK) <& \!\! \cdots \\[2pt]
	\UU_{d,\,0}(\bbC) <& \!\! \UU_{d+1,\,1}(\bbC) <& \!\! \UU_{d+2,\,2}(\bbC) <& \!\! \cdots <& \!\! \UU_{d+r,\,r}(\bbC) <& \!\! \cdots
\end{array}
\end{equation*}
In \cite{DM+Hartnick}, we are going to apply \autoref{MainThmQual} to establish bc-stability for all of these families $(G_r)$. Our proof is based on the fact that all of the groups above arise a 
 automorphism groups of sesquilinear forms. With each of the groups $G_r$ above, one can thus associate a corresponding \emph{Stiefel complex} in the sense of \cite{Vogtmann}, and as in the case of the general linear groups, it is easy to see that the actions are $(r-1)$-transitive and compatible (in the sense of (MQ3c)). In order to establish bc-stability, one thus only has to show that the Stiefel complex $X_r$ is mesurably $\gamma(r)$-acyclic for some proper function $\gamma$. This is established in \cite{DM+Hartnick} using probabilistic methods, leading to an acyclicity range of the form $\gamma(r) \sim \log_2(r)$. 
\end{exmpl}
\subsection{The need for finite measurable Quillen families}
\autoref{MainThmQual} does not apply directly to the families of special linear (or special orthogonal or special unitary) groups. The problem is that while $G_r= \SL_r(\bbK)$ acts highly transitively on the product complex $X(r)$ defined as in \autoref{GLnExample}, the semisimple factor of the stabilizer $H_{r,q}$ is isomorphic to $\GL_{r-q-1}(\bbK)$ instead of $\SL_{r-q-1}(\bbK)$. 

To deal with this problem, we introduce in Section \ref{SecQuillenFamily} below the notion of a finite measurable Quillen family. It then turns out that (for $\bbK \in \{\bbR, \bbC, \bbH\}$), the family
\[
\{1\} < \GL_1(\bbK) < \GL_2(\bbK) < \dots < \GL_{R-1}(\bbK) < \SL_R(\bbK) 
\]
is a finite measurable Quillen family in this sense, and our main theorem (\autoref{MainTheorem}) also applies. Assuming the theorem and its notation, we recover Monod's results for $\SL_n$.

\begin{exmpl}[Special linear groups] \label{SLnExample}
Let $\bbK \in \{\bbR,\bbC,\bbH\}$ and $R \geq 2$. We set $G_r := \GL_r(\bbK)$ for any $r \in [R-1]$ and $G_R := \SL_R(\bbK)$. The inclusions $\iota_r:G_r \hookrightarrow G_{r+1}$ are defined as in \eqref{GLn_inclusion} for $r < R-1$, and as
\[
	A \mapsto (\det A)^{-1} \times A
\]
for $r = R-1$. We consider the same acyclic product complexes $X(r)$ as in \autoref{GLnExample}; then the action of $G_r$ on $X(r)$ is $\tau(r)$-transitive, where $\tau(r) = r$ if $r < R$ and $\tau(R)= R-1$. In the notation of \autoref{MainTheorem} the initial condition is again given by $q_0 = 2$ if $R \geq 3$ and $q_0 = 1$ if $R = 2$. Assume first that $R \geq 3$. Then
\begin{align*}
	\min\{\widetilde{\gamma}(q,R-1), \widetilde{\tau}(q,R-1)-1\} &= \min_{j=q_0}^{q} \big\{\tau\big(R-2(q-j)\big) - j\big\}-1 \\
	&= \min\left\{\min_{j=q_0}^{q-1}\big\{R-2q+j\big\},R-1-q\right\} -1 \\
	&= \min\{R-2q+2,R-q-1\}-1 = R-(2q-2),
\end{align*}
and, hence, the map $\iota_R$ induces
\begin{enumerate}[label=(\roman*)]
\item an isomorphism $\Hb^q(\SL_R(\bbK)) \cong \Hb^q(\GL_{R-1}(\bbK))$ if $R \geq 2q-2$;
\item an injection $\Hb^q(\SL_R(\bbK)) \hookrightarrow \Hb^q(\GL_{R-1}(\bbK))$ if $R \geq 2q-4$.
\end{enumerate}
Combining this information with \autoref{GLnExample}, we see that for $q \geq 3$ the natural inclusions induce isomorphisms/injections\footnote{Note that the stability range given in \cite{Monod-Stab} is not correct, since it uses the inaccurate stability range for $\GL_r(\bbK)$.}
\[
 \dots \cong \Hb^q(\SL_{2q}(\bbK)) \cong \Hb^q(\SL_{2q-1}(\bbK)) \cong \Hb^q(\SL_{2q-2}(\bbK)) \hookrightarrow \Hb^q(\GL_{2q-3}(\bbK)).\vspace{-2pt}
\]
For $q = 2$, not covered by the discussion above, is well-known (see \cite{Burger-Monod3} and \cite[Ex. 9.9.3]{Monod-Book}):\vspace{-2pt}
\[
 \dots \cong \Hb^2(\SL_{4}(\bbK)) \cong \Hb^2(\SL_{3}(\bbK)) \cong \{0\} \hookrightarrow   \Hb^q(\SL_{2}(\bbK));\vspace{-2pt}
\]
here the latter inclusion is an isomorphism for $\bbK = \bbC$ or $\bbH$, but not for $\bbK = \bbR$. \vspace{-3pt}
\end{exmpl}
%For similar results concerning special unitary and special orthogonal groups, see \cite{dlCMH2}.
%We write $\Sym_k$ for the group of symmetries of the set $[k]$
{\bf Acknowledgements.} We thank Marc Burger and Nicolas Monod for insightful conversations. This work was completed during a postdoctoral fellowship of the first author at the Weizmann Institute of Science, Israel. He was supported by the Swiss National Science Foundation, Grants No. 169106 and 188010. The second author was supported by the Deutsche Forschungsgemeinschaft, Grant No. HA 8094/1-1 within the Schwerpunktprogramm SPP 2026 (Geometry at Infinity). \vspace{-2pt}


\section{Spectral sequences associated to Lebesgue complexes} \label{sec:monodcoeff}
Throughout this section, $G$ denotes an lcsc group. The goal of this section is to associate a first-quadrant spectral sequence in continuous bounded cohomology with every partially boundedly acyclic Lebesgue $G$-complex. We explain the terminology used in this statement and recall the necessary background on bounded cohomology; for further details, see \cite{Monod-Book}. 

\subsection{Continuous bounded cohomology and coefficient $G$-modules}
A \emph{Banach $G$-module} is a Banach space over the field of real numbers equipped with a $G$-action by linear isometries, and a \emph{$G$-morphism} is a $G$-equivariant, bounded operator between Banach $G$-modules. A Banach $G$-module is said to be \emph{separable} if the underlying Banach space is separable, and \emph{continuous} if the $G$-action is jointly continuous. 

For a general Banach $G$-module $E$ (not assumed separable or continuous), the \emph{continuous bounded cohomology} $\Hb^\bcdot(G; E)$ of $G$ with coefficients in $E$ is defined as the cohomology of the complex $(C_{\mathrm{b}}^n(G; E)^G)_{n \in \mathbb N}$ of $G$-invariants, where\vspace{-3pt}
\[
C_{\mathrm{b}}^0(G; E) := C_{\mathrm{b}}(G; E) \qand C_{\mathrm{b}}^n(G; E) := C_{\mathrm{b}}(G, C_{\mathrm{b}}^{n-1}(G; E)) \quad (n \geq 1). 
\]
Here $C_{\mathrm{b}}(G; E)$ denotes the Banach space of continuous bounded functions $f: G \to V$ with the supremum norm. Since $G$ is assumed to be lcsc, the bounded cohomology can also be computed as (\cite[Remark 7.4.9]{Monod-Book}) \vspace{-2pt}
\[
\Hb^\bullet(G; E) = H^\bullet(0 \to C_{\mathrm{b}}(G; E)^G \to C_{\mathrm{b}}(G^2; E)^G \to C_{\mathrm{b}}(G^3; E)^G \to \cdots).
\]
If $E = \mathbb R$ with trivial $G$-action, we will drop $E$ from notation and write $\Hb^\bullet(G) := \Hb^\bullet(G; \mathbb R)$.

In \cite{Monod-Book}, Monod introduced the category of \emph{coefficient $G$-modules}, for which the continuous bounded cohomology can be computed via more accessible measurable resolutions (see \autoref{MeasRes} \and \autoref{MeasRes2} below). Since this category is central to the theory of continuous bounded cohomology of locally compact groups, we briefly recall the definition. 

Given a Banach $G$-module $E$, we denote by $E^\sharp$ the topological dual of $E$, considered as a Banach $G$-module with respect to the contragredient $G$-action. Note that $E^\sharp$ needs not be separable nor continuous, even if $E$ has either of those properties. 

\begin{defn}\label{DefCoefficientModule}
A \emph{coefficient $G$-module} is a pair $(E,E^\#)$, where $E$ is a separable, continuous Banach $G$-module. A morphism $(E, E^\#) \to (F, F^\#)$ of coefficient $G$-modules, or a \emph{dual $G$-morphism}, is a pair $(\Phi,\Phi^\#)$, where $\Phi: E \to F$ is a bounded operator with a $G$-equivariant dual $\Phi^\#: F^\# \to E^\#$. We write ${\bf Coef}_G$ for the category of coefficient $G$-modules. 
\end{defn}

\begin{rem} If $(\Phi,\Phi^\#):  (E, E^\#) \to (F, F^\#)$ is a dual $G$-morphism, then $\Phi^\#: F^\# \to E^\#$ will in general not be norm continuous, but only weak-$*$-continuous with respect to the specified pre-duals. On the other hand, $G$-equivariance of $\Phi^\#$ is actually equivalent to $G$-equivariance of $\Phi$. Indeed, it is immediate from the definitions that the latter implies the former, and conversely $\Phi$ is the restriction of $(\Phi^\#)^\#$ to $E \subset (E^\#)^\#$, hence the same argument applies.

If we consider a coefficient $G$-module $(E^\flat,E)$, then the only role of the specified pre-dual $E^\flat$  is to define a weak-$\ast$ topology on $E$. We will thus usually drop the pre-dual $E^\flat$ from notation and simply refer to $E$ as a coefficient $G$-module. With this abuse of notation, a morphism between coefficient $G$-modules $E$ and $F$ is simply a $G$-equivariant map $E \to F$, which is continuous with respect to the weak-$*$-topologies defined by the omitted pre-duals.
\end{rem}
If $E$ is a coefficient $G$-module, then we equip $E$ with the Borel $\sigma$-algebra associated with the weak-$*$-topology on $E$. Given $n \in \mathbb N$, we then write
\begin{equation} \label{def:LinftyG}
 \Linfty(G^n;E):=\{\phi: G^n \to E \: : \: \phi \mbox{ is measurable and essentially bounded}\}/\sim,
\end{equation}
where $\sim$ denotes almost everywhere equality with respect to the Haar measure on $G^n$.
\begin{prop}[{\cite[Prop.\ 7.5.1]{Monod-Book}}]\label{MeasRes} If $E$ is a coefficient $G$-module, then
\[
\Hb^\bullet(G; E) = \HH^\bullet(0 \to \Linfty(G; E)^G \to  \Linfty(G^2; E)^G \to  \Linfty(G^3; E)^G \to \dots).
\]
\end{prop}
The following lemma is an essential technical tool for us. Here, ${\bf Vect}$ denotes the category of real vector spaces, and we say that a sequence $0 \to  E^0 \to E^1 \to E^2 \to \cdots$ of coefficient $G$-modules is a \emph{cochain complex}, resp. \emph{exact}, if the underlying sequence of vector spaces has the corresponding property. 
\begin{lem}[{\cite[Lemma 8.2.5]{Monod-Book}}] \label{thm:exactftr}
For any $n \in \bbN$, the functor ${\bf Coef}_G \to {\bf Vect}$ given by $E \mapsto \Linfty(G^n, E)^G$ is exact. \qed
\end{lem}
More concretely, this means that if $0\to A \to B \to C \to 0$ is a short exact sequence of coefficient $G$-modules, then 
\[
	0 \to \Linfty(G^n; A)^G \to \Linfty(G^n; B)^G \to \Linfty(G^n; C)^G \to 0
\]
is an exact sequence of vector spaces.

\subsection{Lebesgue $G$-complexes}\label{sec:GObjects} 
The next notion allows us to produce examples of coefficient $G$-modules.
\begin{defn} \label{defn:regspace}
We say that $X$ is a \emph{Lebesgue space} it is a standard Borel space, endowed with a Borel probability measure class. If in addition $X$ admits a Borel $G$-action such that the measure class is $G$-invariant, we will say it is a \emph{Lebesgue $G$-space}. For economy, we will say that the pair $(X,\mu)$ is a Lebesgue ($G$-)space whenever $X$ is a Lebesgue ($G$-)space, and $\mu$ is a fixed ($G$-quasi-invariant) probability measure $\mu$ within the measure class. 

We denote by ${\bf Leb}$ the category whose objects are Lebesgue spaces and whose morphisms are Borel maps which are \emph{measure-class-preserving}. Similarly we denote by ${\bf Leb}_G$ the category of Lebesgue $G$-spaces and $G$-equivariant morphisms of Lebesgue spaces. 
\end{defn}

\begin{con} \label{LInfty}
If $(X, \mu)$ is a Lebesgue $G$-space, then the pair $(L^1(X,\mu), \Linfty(X))$ is a coefficient $G$-module for any choice of probability measure $\mu$ in the specified measure class on $X$. The action on $L^1(X,\mu)$ is given by
\[
g \cdot \phi(x) := \rho_\mu(g^{-1}, x) \cdot \phi(g^{-1}x),
\]
where $\rho_\mu(g, x) := \dd(g_*\mu)/\dd\!\mu(x)$ denotes the Radon-Nikodym cocycle of $\mu$. The contragredient action on $\Linfty(X)$ is given explicitly by the formula $g \cdot f(x) := f(g^{-1}x)$. See \cite[Appendix D]{Buehler} for a proof of the continuity of $L^1(X,\mu)$. Observe that any other choice of the probability measure $\mu$ in the measure class of $X$ produces an isomorphic coefficient $G$-module. 
\end{con}

\begin{rem}\label{LinftyFunctor}
The assignment $\Linfty(-): {\bf Leb}_G^{\rm op} \to {\bf Coef}_G$ is a functor. Indeed, if $T: Y \to X$ is a morphism of Lebesgue $G$-spaces, then the induced operator
\[
	\Linfty(T): \Linfty(X) \to \Linfty(Y), \quad \Linfty(T)(\phi) := \phi \, \circ \,  T
\]
is a dual $G$-morphism. As a matter of fact, choosing a $G$-quasi-invariant probability $\nu \in \Prob(Y)$ and letting $\mu:= T_\ast \nu$, the pre-dual operator to $\Linfty(T)$ is
\[
	L^1(T): L^1(Y,\nu) \to L^1(X,\mu), \quad \big( L^1(T)(\psi) \big)(x) := \dd(T_\ast( \psi\!\cdot\!\nu))/\dd\!\mu. 
\]
\end{rem}

We will need the following generalization of \autoref{LInfty}.
\begin{con}\label{LInftyGeneral}
For any Lebesgue $G$-space $(X, \mu)$ and coefficient $G$-module $E$, we define as in \eqref{def:LinftyG}
\[
 \Linfty(X;E):=\{\phi: X \to E \: : \: \phi \mbox{ is weak-$*$ Borel and essentially bounded}\}/\sim,
\]
where $\sim$ denotes $\mu$-almost everywhere equality. This defines a Banach $G$-module when
equipped with the essential supremum norm and the $G$-action given by the formula $g.f(x) = g.f(g^{-1}x)$. It is also a coefficient $G$-module, being the dual of the Bochner space $L^1(X,\mu;E)$. We thus have a functor $\Linfty(-; E): {\bf Leb}_G^{\rm op} \to {\bf Coef}_G$ for every coefficient $G$-module $E$.
\end{con}
We have the following generalization of \autoref{MeasRes}; see \cite[Sec.\ 5.3]{Monod-Book} for the notion of an amenable Lebesgue $G$-space (in the sense of Zimmer).
\begin{prop}[{\cite[Thm.\ 7.5.3]{Monod-Book}}]\label{MeasRes2} If $E$ is a coefficient $G$-module and $S$ is an amenable Lebesgue $G$-space, then
\[
\Hb^\bullet(G; E) = \HH^\bullet(0 \to \Linfty(S; E)^G \to  \Linfty(S^2; E)^G \to  \Linfty(S^3; E)^G \to \dots).
\]
\end{prop}

From Lebesgue $G$-spaces, we can build associated complexes using the following notion:
\begin{defn}[{\cite[Def. 8.1.9]{Weibel}}] \label{def_ssobjects}
A \emph{semi-simplicial object} $X$ over a category $\clC$ is a sequence of objects $(X_k)_{k \in \bbN}$, together with morphisms $\delta_{i,k}:\, X_{k+1} \to X_k$  for all $k$ and $i \in [k]$, called \emph{face maps}, such that 
\begin{equation} \label{eq:facecondition}
	\delta_{i,k-1} \, \circ \, \delta_{j,k} = \delta_{j-1,k-1} \, \circ \, \delta_{i,k} \qquad \mbox{whenever } i<j\text{.}
\end{equation}
If $X$ and $Y$ are semi-simplicial objects over a category $\clC$, then a \emph{semi-simplicial morphism} $f: X \to Y$ is a collection $(f_k: X_k \to Y_k)_{k \in \bbN}$ of morphisms such that $f_{k} \circ \delta_{i, k-1} = \delta_{i,k} \circ f_{k+1}$ for all $k \in \bbN$ and all $i\in [k+1]$.  
\end{defn}

We will usually omit the index $k$ in $\delta_{i,k}$, writing simply $\delta_{i}$ for all $i$-th face maps. 

\begin{defn}\label{DefGObject} A semi-simplicial object in ${\bf Leb}_G$ will be called a \emph{Lebesgue $G$-complex}.
\end{defn}

\begin{rem} \label{stabilizer_cont}  If $X$ is a Lebesgue $G$-space, then we refer to points in $\bigsqcup_{k \in \bbN} X_k$ as \emph{simplices} and to points of $X_k$ as \emph{$k$-simplices} of $X$. Given two simplices $x, y \in X$, we say that $x$ is a \emph{face} of $y$, denoted $x \prec y$, if $x$ can be obtained from applying finitely many face maps to $y$. A (finite or infinite) sequence $x_0 \prec x_1 \prec x_2 \prec \cdots$ of simplices is called a \emph{flag}.

Since standard Borel spaces are countably separated, it follows from \cite[Cor.\ 2.1.20]{Zimmer} that the stabilizer of any simplex $x$ in a Lebesgue $G$-complex $X$ is a closed subgroup of $G$. If $x,y \in X$ are such that $x \prec y$, then  the reversed inclusion $ \stab_G (y) < \stab_G (x)$ of stabilizer subgroups holds, hence every flag $x_0 \prec x_1\prec x_2 \prec \cdots$ gives rise to a chain of closed subgroups
\[
G > \stab_G(x_0) > \stab_G(x_1) > \stab_G(x_2) > \cdots 
\]
\end{rem}
\begin{con}
Let $X$ be a Lebegue $G$-complex.  We define $X_{-1}$ to be a singleton, so that $\Linfty(X_{-1}) \cong \bbR$. There is then a unique ``face map''
$\delta_{0}: X_0 \to X_{-1}$, which induces the inclusion of constants $\delta^0:\bbR \to \Linfty(X_0)$. 

Applying the $\Linfty$-functor from \autoref{LinftyFunctor} to the face maps $\delta_i: X_{l+1} \to X_l$ provides a family of morphisms $\delta^i: \Linfty(X_l) \to \Linfty(X_{l+1})$ of coefficient $G$-modules. If we define
\[
\dd^l: \Linfty(X_l) \to \Linfty(X_{l+1}), \quad \dd^l := {\textstyle \sum_{i=0}^{l+1}} (-1)^i \delta^i \quad (l \geq -1),
\]
then $\dd^{l} \circ \dd^{l-1} = 0$ for all $l \geq 0$, and hence we obtain a cochain complex
\begin{eqnarray}\label{LInftyComplexStiefel}
0 \rightarrow \bbR \xrightarrow{\dd^{-1}} \Linfty(X_0) \xrightarrow{\dd^{0}} \Linfty(X_1) \xrightarrow{\dd^{1}} \Linfty(X_2) \xrightarrow{\dd^{2}} \Linfty(X_3) \, \xrightarrow{} \cdots
\end{eqnarray}
of coefficient $G$-modules.
\end{con}
\begin{defn} \label{def:augmented}
Given a Lebesgue $G$-complex $X$, the cochain complex \eqref{LInftyComplexStiefel} of coefficient $G$-modules is called the \emph{augmented $\Linfty$-complex} associated to $X$.

We will say that $X$ is \emph{boundedly acyclic} if its augmented $\Linfty$-complex is exact, and \emph{boundedly $\gamma_0$-acyclic} for some $\gamma_0 \in \bbN \cup \{\infty\}$ if $\ker \dd^l = \im \dd^{l-1}$ for every $l \in [\gamma_0]$. 

If a Lebesgue $G$-complex is boundedly $\gamma_0$-acyclic for some $\gamma_0 \in \mathbb N$ that we do not want to specify, then we will say that $X$ is \emph{partially boundedly acyclic}. We will also convene that every Lebesgue $G$-complex is boundedly $(-\infty)$-acyclic. 
\end{defn} %The corresponding \emph{unaugmented $\Linfty$-complex} is then defined as
%\begin{eqnarray}\label{LInftyComplexStiefelUnaug}
%0 \to \Linfty(X_0) \xrightarrow{\dd^{0}} \Linfty(X_1) \xrightarrow{\dd^{1}} \Linfty(X_2) \xrightarrow{\dd^{2}} \Linfty(X_3) \xrightarrow{} \cdots
%\end{eqnarray}

\subsection{The spectral sequence of a partially boundedly acyclic Lebesgue $G$-complex} \label{sec:specseq_admissible}
With every partially boundedly acyclic Lebesgue $G$-complex, we can now associate a spectral sequence; our notation concerning spectral sequences follows \cite{McCleary}. 
\begin{prop} \label{thm:E_semisimplicial}
Let $X = (X_q)_{q \in \bbN}$ be a Lebesgue $G$-complex which is boundedly $\gamma_0$-acyclic for some $\gamma_0 \in \mathbb N \cup \{\infty\}$. Then there exists a first-quadrant spectral sequence $\EE_\bcdot^{\bcdot,\bcdot}$ with first page terms and differentials
\begin{equation} \label{eq:firstpage}
	\EE_1^{p,q} = \Hb^q(G;\Linfty(X_{p-1})) \qand \dd_1^{p,q} = \Hb^q(G; \dd^{p-1}) \quad \mbox{for all } p, q \geq 0
\end{equation}
that converges to $\EE_\infty^t = 0$ for every $t \in [\gamma_0+1]$. 
\end{prop}

\begin{proof}
Consider the first-quadrant double complex $(L^{\bcdot,\bcdot},\dd_{\rm H}, \dd_{\rm V})$, whose terms and differentials for $p,q \geq 0$ are given by
\begin{equation*}
\begin{array}{c}
	L^{p,q}:=\Linfty\big(G^{p+1} \times X_{q-1}\big)^{G} \cong \Linfty\big(G^{p+1}; \Linfty(X_{q-1})\big)^{G}, \vspace{4pt} \\
\begin{array}{ll}
	\dd^{p,q}_{\rm H}:\,L^{p,q} \to L^{p+1,q}, \qquad &  \dd^{p,q}_{\rm H}\!\! f(g_0,\ldots,g_p):= \sum_{i=0}^p (-1)^i f(g_0,\ldots,\hat{g_i},\ldots,g_p), \vspace{4pt} \\
	\dd_{\rm V}^{p,q}:\,L^{p,q} \rightarrow L^{p,q+1}, \qquad & \dd^{p,q}_{\rm V}\!\! f(g_0,\ldots,g_{p-1}):= \dd^{q-1}\!\big(f(g_0,\ldots,g_{p-1})\big). 
\end{array}
\end{array}
\end{equation*}
\noindent Here $\dd^{q-1}$ denotes the coboundary operator of the augmented $\Linfty$-complex of $X$, as in \eqref{LInftyComplexStiefel}. We let $\IE_\bcdot^{\bcdot,\bcdot}$ and $\IIE_\bcdot^{\bcdot,\bcdot}$ be the two spectral sequences associated with the horizontal and vertical filtrations of $L^{\bcdot,\bcdot}$, respectively, both of which converge to the cohomology of the total complex of $L^{\bcdot,\bcdot}$ and whose first-page terms and differentials are given by 
\begin{align*}
\IE_{1}^{p,q} & =\HH^q (L^{p,\bcdot},\,\dd_{\rm V}^{p,\bcdot}),\quad\Id_{1}^{p,q}=\HH^q(\dd_{\rm H}^{p,\bcdot}): \,\IE_{1}^{p,q}\rightarrow\IE_{1}^{p+1,q},\\[-3pt]
\IIE_{1}^{p,q} & =\HH^q(L^{\bcdot,p},\,\dd^{\bcdot,p}_{\rm H}),\quad\IId_{1}^{p,q}= \HH^q(\dd_{\rm V}^{\bcdot,p}):\,\IIE_{1}^{p,q}\rightarrow\IIE_{1}^{p+1,q}. 
\end{align*}
For a proof of the existence and convergence properties of these two spectral sequences, see \cite[Theorem 2.15]{McCleary}. By the $\gamma_0$-acyclicity of $X$ and exactness of the functor $\Linfty(G^{p+1},-)^G$ for every $p \geq 0$ (see \autoref{thm:exactftr}), the complex
\[
0 \to L^{p,0} \to L^{p,1} \to \cdots \to L^{p,\gamma_0} \to L^{p,\gamma_0+1} \to L^{p,\gamma_0+2} \to \cdots,
\]
is exact up to degree $\gamma_0+1$. Thus, $\IE_1^{p,q} = 0$ for all $p \geq 0$ and $q \in [\gamma_0+1]$, and therefore $\IE_\infty^{t} \cong \bigoplus_{p+q = t} \IE_\infty^{p,q} = 0$  for every $t \in [\gamma_0+1]$. We deduce that also the spectral sequence $\EE_\bcdot^{\bcdot,\bcdot} := \IIE_\bcdot^{\bcdot,\bcdot}$ converges to $0$ in degrees $t \in [\gamma_0 + 1]$. Finally, it follows immediately from \autoref{MeasRes} that the first page of $\EE_\bcdot^{\bcdot, \bcdot}$ is given by \eqref{eq:firstpage}.
\end{proof}

\section{The highly essentially transitive case}
In this section, $G$ continues to denote a lcsc group and $X$ denotes a partially boundedly acyclic Lebesgue $G$-complex. We are going to discuss how certain transitivity properties of the $G$-action on $X$ affect the spectral sequence constructed in the previous section. This will lead us to a proof of \autoref{BasicSS} from the introduction.
\begin{defn} \label{def:conn+trans}
Let $X$ be an Lebesgue $G$-complex and $\tau_0 \in \bbN \cup \{\infty\}$. We say that $X$ is \emph{essentially $\tau_0$-transitive} if $X_l$ admits a conull $G$-orbit for all $l \in [\tau_0]$. We convene that any Lebesgue $G$-complex is essentially $(-\infty)$-transitive. 
%\begin{enumerate}[label=(\roman*),leftmargin=20pt]
%	\item \emph{$\tau_0$-transitive} if the $G$-action on $X_l$ is transitive for all $l \in [\tau_0]$; 
%	\item \emph{essentially $\tau_0$-transitive} if there exists a subcomplex $Y \subset X$ such that $Y_l \subset X_l$ is co-null and  the $G$-action on $Y_l$ is transitive for all $l \in [\tau_0]$.
%\end{enumerate}
\end{defn}  

For the remainder of this section we assume that $X$ is a Lebesgue $G$-complex, which is boundedly $\gamma_0$-acyclic and essentially $\tau_0$-transitive for some $\gamma_0, \tau_0 \in \mathbb N \cup \{\infty\}$. We denote by $\EE_{\bcdot}^{\bcdot, \bcdot}$ the associated spectral sequence constructed in \autoref{thm:E_semisimplicial}.
\begin{rem} The bottom row of the first page of the spectral sequence $\EE_{\bcdot}^{\bcdot, \bcdot}$ is given by
\begin{align*}
\EE_1^{p,0} &= \Hb^0(G;\Linfty(X_{p-1})) = \Linfty(X_{p-1})^G  \qand \\[-3pt]
\dd_1^{p,0} &= \Hb^0(G; \dd^{p-1}) \quad \mbox{for all } p, q \geq 0.
\end{align*}
Since the constant functions are always contained in $\Linfty(X_{p-1})^G$ we have an embedding of constants $c_p: \mathbb R \hookrightarrow \EE_1^{p,0}$ for every $p \geq 0$, such that all of the diagrams
\begin{equation*}
\xymatrixcolsep{2pc}
\xymatrix@R=5pt{\EE_1^{p,0}\ar[rr]^{\delta^i} && \EE_1^{p+1,0}  \\
& \ar@{_{(}->}[ul]^{c_p} \bbR \ar@{^{(}->}[ur]_{c_{p+1}}}
\end{equation*}
commute. Since $G$ acts essentially $\tau_0$-transitive, we have
\[
\EE_1^{p,0} = \Linfty(X_{p-1})^G = c_p(\mathbb R) \cong \bbR \quad \text{for all } p \in [\tau_0+1].
\]
It thus follows from the commuting diagram above that each of the dual face maps $\delta^i: \EE_1^{p,0} \to  \EE_1^{p+1,0}$ equals the identity if 
$p \in [\tau_0]$ and is an injection if $p = \tau_0+1$. We deduce that the face map $\dd_1^{p,0}$ is the zero map if $p \in [\tau_0+1]$ is odd, an isomorphism if $p \in [\tau_0]$ is even, and injective if $p = \tau_0+1$ is even. Consequently we have
\begin{equation}\label{T3}
\EE_2^{p,0} = 0 \; \text{for all }p \in [\tau_0+1].
\end{equation}
\end{rem}
\begin{con}\label{ConGenericFlag} 
By assumption, we can find $o_{\tau_0} \in X_{\tau_0}$ such that the orbit $G.o_{\tau_0} \subset X_{\tau_0}$ is conull. Since the face maps are $G$-equivariant and measure-class-preserving, any codimension-1 face $o_{\tau_0-1} \in X_{\tau_0-1}$ has a conull orbit in $X_{\tau_0-1}$. Thus, we construct a flag $o_{-1} \prec o_0 \prec \cdots \prec o_{\tau_0}$ inductively with $o_q \in X_q$ such that $o_q$ has a conull orbit in $X_q$ for all $q \in \{-1\} \cup [\tau_0]$ (recall our convention that $X_{-1}$ is a singleton, and thus $o_{-1}$ denotes its only element). We refer to such a flag as a \emph{generic flag} in $X$. Given a generic flag, we define 
\[
H_q := \stab_G(o_q) \quad (q \in \{-1\} \cup [\tau_0]);
\]
note that according to our convention, $G$ acts trivially on $X_{-1}$, so $H_{-1} = G$.

By \autoref{stabilizer_cont}, we have a chain of closed subgroups
\[
G = H_{-1}>H_0 > H_1 > \dots > H_{\tau_0},
\]
and we denote by $j_q: H_q \to H_{q-1}$ the corresponding inclusion maps. Now let $q \in \{-1\} \cup [\tau_0-1]$ and $i \in [q+1]$, so that $\delta_i(o_{q+1}) \in X_{q}$. By transitivity, there exists $w_{q,i} \in G$ such that
\begin{equation*}\label{wqi}
\delta_i(o_{q+1}) = w_{q,i}^{-1}.o_{q},
\end{equation*}
and we fix a choice of such elements of $G$ once and for all. 
\end{con}

\begin{lem}\label{wLemma}
For all $q \in \{-1\} \cup [\tau_0-1]$ and $i \in [q+1]$, we have the inclusion $w_{q,i} H_{q+1} w_{q,i}^{-1} < H_q$. In particular, there is a well-defined map
\[
\Int(w_{q, i}): H_{q+1} \to H_q, \quad h \mapsto  w_{q,i}hw_{q,i}^{-1}.
\]
\end{lem}
\begin{proof} If $h \in H_{q+1}$, then
\[
w_{q,i}^{-1}.o_q = \delta_i(o_{q+1}) = \delta_i(h.o_{q+1}) = h.\delta_i(o_{q+1}) = h.w_{q,i}^{-1}.o_q \implies w_{q,i}hw_{q,i}^{-1} \in H_q.\qedhere
\]
\end{proof}
%For later use we observe that with the convention $G_{-1} := H$ the lemma holds for $q = -1$ and $w_{q, i} \in G$ arbitrary.
\begin{con}
For $p \in [\tau_0+1]$, we define the $(p-1)$-th \emph{induction module} $\Ind^{p-1} := \Linfty(G)^{H_{p-1}}$, where the $H_{p-1}$-invariance is taken with respect to the restriction of the $G$-action by left-translation. Equipped with the right-translation $G$-action $\rho$, the space $\Ind^{p-1}$ is a Banach $G$-module. We now define a family of induction maps
 \begin{equation*} \label{induction_bla}
	\Ind\!\!: \Linfty(G^{q+1})^{H_{p-1}} \to \Linfty(G^{q+1},\Ind^{p-1})^{G}, \quad (\Ind\varphi)(\vec{g})(x)\!:= \varphi(x\vec{g})  \qquad (q \geq 0),
\end{equation*}
which produces a  morphism of cochain complexes (see \cite[Prop. 10.1.3]{Monod-Book}). We also define
\begin{equation*} 
	\Psi_0: \Ind^{p-1} \to \Linfty(X_{p-1}), \quad (\Psi_0\varphi)(g\cdot o_{p-1}) := \varphi(g^{-1}).
\end{equation*}
and define a morphism of cochain complexes $\Psi := \Linfty(G^{q+1};\Psi_0)^{G}$, so that
\begin{equation*} \label{PSI}
	\Psi: \Linfty(G^{q+1};\Ind^{p-1})^{G} \ \to \Linfty(G^{q+1};\Linfty(X_{p-1}))^{G} \quad (q \geq 0).
\end{equation*}
Note that, by \autoref{MeasRes2}, the cohomology of $\Linfty(G^{q+1})^{H_{p-1}}$  is $\Hb^q(H_{p-1})$, and by \autoref{MeasRes}, the cohomology of $ \Linfty(G^{q+1};\Linfty(X_{p-1}))^{G}$ is $\Hb^q(G;\Linfty(X_{p-1})) = \EE_1^{p,q}$.
\end{con}
\begin{prop}\label{propisomstabil} The composition $\Psi \circ \Ind: \Linfty(G^{q+1})^{H_{p-1}} \to \Linfty(G^{q+1};\Linfty(X_{p-1}))^{G}$ induces an isomorphism
\begin{equation*} \label{isom_stabil}
	I_{p,q}:\;\Hb^q(H_{p-1}) \to \EE_1^{p,q} \quad \mbox{for every } p \in [\tau_0+1] \text{ and every } \ q\geq 0. \vspace{-5pt}
\end{equation*}\end{prop}
\begin{proof} It suffices to show that $\Ind$ and $\Psi$ induce isomorphisms in cohomology. The former is just the Eckmann--Shapiro lemma for $\Linfty$-modules (see \cite[Prop. 10.1.3]{Monod-Book}) and the latter is immediate from the fact $\Psi_0$ is an isomorphism for any fixed $q\geq 0$ by essential transitivity.
\end{proof}
%Note that (with the convention $H_{-1} := G$) we have $\EE_1^{0,q}= \Hb^q(G)  = \Hb^q(H_{-1})$ for all $q \geq 0$. We thus define $I_{0,q}$ to be the identity, then $I_{p,q}:\;\Hb^q(H_{p-1}) \to \EE_1^{p,q}$ is an isomorphism for all $p \in [\tau_0 + 1]$ and $q \geq 0$, including for $p = 0$. 
Having computed the terms on the first page of $\EE_\bcdot^{\bcdot,\bcdot}$, we now compute the differentials.
\begin{lem} \label{thm:eckmann_facemap}
For every $p \in [\tau_0]$, $q \geq 0$, and $i \in [p]$, we have a commuting diagram \vspace{-5pt}
\begin{equation*} 
		\xymatrixcolsep{4.5pc}
		\xymatrixrowsep{1pc}
		\xymatrix{\EE_1^{p,q} \ar[r]^{\Hb^q(G;\delta^i)} & \EE_1^{p+1,q} \\
		\Hb^q(H_{p-1}) \ar[u]^{I_{p,q}} \ar[r]^{\Hb^q(j_p)} & \Hb^q(H_{p}) \ar[u]_{I_{p+1,q}}} 
\end{equation*}
\end{lem}
\begin{proof} 
We claim that for $i, p, q$ as in the lemma, the map 
\[
	\lambda_{w_{p-1, i}^{-1}} : \Linfty(G^{q+1})^{H_{p-1}} \to \Linfty(G^{q+1})^{H_{p}},\quad \lambda_{w_{p-1, i}^{-1}} f(g_0, \dots, g_q) \mapsto f(w_{p-1, i}g_0, \dots, w_{p-1, i}g_q)
\]
is well-defined. Indeed, by \autoref{wLemma}, we find for every $h_p \in H_p$ an element $h_{p-1} \in H_{p-1}$ such that $w_{p-1, i}\, h_p \, w_{p-1,i}^{-1} = h_{p-1}$. For all $f \in  \Linfty(G^{q+1})^{H_{p-1}}$ and $\vec{g} \in G^{q+1}$, we then have
\[
\lambda_{w_{p-1, i}^{-1}}f(h_p^{-1}.\vec{g}) = f(w_{p-1, i}h_p^{-1}.\vec{g}) = f(h_{p-1}^{-1}w_{p-1, i}.\vec{g})= f(w_{p-1, i}.\vec{g}) = \lambda_{w_{p-1, i}^{-1}}f(\vec{g}),
\]
hence $\lambda_{w_{p-1, i}^{-1}}f \in \Linfty(G^{q+1})^{H_{p}}$. We then obtain a commuting diagram
\[
\xymatrixrowsep{1.2pc}
\begin{xy}\xymatrix{
0 \ar[r] & \mathbb R \ar[r] \ar[d]^{\mathrm{Id}}&  \Linfty(G)^{H_{p-1}} \ar[r] \ar[d]^{\lambda_{w_{p-1, i}^{-1}}} & \Linfty(G^2)^{H_{p-1}} \ar[r]  \ar[d]^{\lambda_{w_{p-1, i}^{-1}}}& \Linfty(G^3)^{H_{p-1}} \ar[r]  \ar[d]^{\lambda_{w_{p-1, i}^{-1}}} & \dots\\
0 \ar[r] & \mathbb R \ar[r] &  \Linfty(G)^{H_{p}} \ar[r] & \Linfty(G^2)^{H_{p}} \ar[r] & \Linfty(G^3)^{H_{p}} \ar[r] & \dots\\
}\end{xy}
\]
It then follows from \cite[Prop.\ 8.4.2]{Monod-Book} that the maps $\lambda_{w_{p-1, i}^{-1}}$ induce the map $\Hb^q(j_p)$ in bounded cohomology. It thus remains to show only that the diagram
\begin{equation*} 
		\xymatrixcolsep{3pc}	\xymatrix@R=18pt{\Linfty(G^{q+1};\Linfty(X_{p-1}))^{G} \ar[r]^{\delta^i} & \Linfty(G^{q+1};\Linfty(X_p))^{G} \\
		\Linfty(G^{q+1})^{H_{p-1}} \ar[u]^{\Psi \, \circ \Ind} \ar[r]^{\lambda_{w_{p-1,i}^{-1}}} & \Linfty(G^{q+1})^{H_{p}} \ar[u]_{\Psi \, \circ \Ind}} 
\end{equation*}
commutes. Now if $\varphi \in \Linfty(G^{q+1})^{H_{p-1}}$, $\vec{g} \in G^{q+1}$, and $x \in G$, then
\begin{align*}
	(\delta^i &\circ \Psi \circ \Ind)(\varphi)(\vec{g})(x\cdot o_{p}) = (\Psi \circ \Ind)(\varphi)(\vec{g})(xw_{p-1,i}^{-1} \cdot o_{p-1}) = \varphi(w_{p-1, i} x^{-1} \vec{g})  \\
	&= \lambda_{w_{p-1, i}^{-1}} \varphi(x^{-1}\vec{g}) =  (\Psi \circ \Ind \circ \lambda_{w_{p-1,i}^{-1}})(\varphi)(\vec{g})(x \cdot o_{p}).
\end{align*}
This establishes the lemma.% in the case $p \geq 1$ and the case $p = 0$ is proved similarly.
\end{proof}
Since $\dd^p =  \sum_{i=0}^{p+1} (-1)^i\, \delta^i$, the isomorphisms $I_{p,q}$ intertwine the map $\dd_1^{p,q} = \Hb^q(G; \dd^{p-1})$ with
\[
 \sum_{i=0}^{p} (-1)^i \, \Hb^q(j_p) = \left\{\!\!\begin{array}{ll}
	\Hb^q(j_p)& \mbox{if } p \mbox{ is even,} \\
	0 & \mbox{if } p \mbox{ is odd.}
\end{array}\right.
\]
Combining this fact with Propositions \ref{thm:E_semisimplicial} and \ref{propisomstabil} and with \eqref{T3}, we arrive at \autoref{BasicSS}. 

\section{Measurable Quillen families}\label{SecQuillenFamily}
In \autoref{DefInfiniteQuillen} we have introduced the notion of an infinite measurable Quillen family. Since the quantitative version of \autoref{MainThmQual} (see \autoref{MainTheorem}
below) also applies to certain finite families of groups, we modify the definitions accordingly.
\begin{set}\label{MainSetting} Let $R \in \bbN \cup \{\infty\}$ be a \emph{length parameter} and let \vspace{-3pt}
\[
	\gamma, \ \tau : [R] \to \bbN \cup \{\pm\infty\} \vspace{-3pt}
\]
be two functions, called respectively the \emph{acyclicity range} and \emph{transitivity range}. We assume that we are given
\begin{itemize}
\item a lcsc group $G_r$ for every $r \in [R]$; 
\item an embedding $\iota_r: G_r \hookrightarrow G_{r+1}$ for every $r < R$;
\item  a Lebesgue $G_r$-complex $X(r)$ for every $r \in [R]$.
\end{itemize}
We set $G_r := \{1\}$ for all $r < 0$. Furthermore, we keep our convention from last section that $X(r)_{-1}$ is a singleton, and we denote by $o_{r,-1}$ its only element. 
\end{set}
Generalizing \autoref{DefInfiniteQuillen}, we declare:
\begin{defn} We say that $(G_r,X(r))_{r \in [R]}$ is a \emph{measurable $(R,\gamma, \tau)$-Quillen family}  provided 
 \begin{enumerate}[leftmargin=37pt, label=(MQ\arabic*)]
	\item $X(r)$ is a boundedly $\gamma(r)$-acyclic Lebesgue $G_r$-complex for every $r \in [R]$;
	\item $X(r)$ is essentially $\tau(r)$-transitive Lebesgue $G_r$-complex for every $r \in [R]$;
	\item for every $r\in [R]$, there exists a generic flag  $o_{r,-1} \prec o_{r,0} \prec o_{r,1} \prec \cdots \prec o_{r,\tau(r)}$ in $X(r)$ such that the corresponding stabilizers $H_{r,q} := \stab_{G_r}(o_{r,q}) < G_r$ are compatible with $G_{r-\tau(r)}, \dots, G_{r}$ either in the sense of Condition (MQ3b) or in the sense of Condition (MQ3c) from the introduction.
\end{enumerate} 
In particular, a measurable $(\infty, \gamma, \tau)$-Quillen family is the same as a measurable $(\gamma, \tau)$-Quillen in the sense of \autoref{DefInfiniteQuillen}. We write $j_{r,q}: H_{r,q} \to H_{r,q-1}$ for the inclusions of stabilizers. 
%By convention, if $\gamma(r) = -\infty$, then the condition (MQ1) is empty. 
\end{defn}
Applying \autoref{BasicSS} to each of the complexes $X(r)$ in a measurable Quillen family, we obtain:
\begin{cor}\label{SpectralSequencesMain} Assume that $(G_r,X(r))_{r \in [R]}$ is a measurable  $(R,\gamma, \tau)$-Quillen family. Then for every $r \in [R]$, there exists a spectral sequence ${}^r\!\EE_\bcdot^{\bcdot,\bcdot}$ such that
\begin{enumerate}[label = \emph{(\roman*)},leftmargin=25pt]
\item ${}^r\!\EE_\bcdot^{\bcdot,\bcdot}$ converges and ${}^r\!\EE_\infty^t = 0$ for every $t \in [\gamma(r)+1]$.
\item ${}^r\!\EE_1^{p,q} = \Hb^q(G_{r-p})$ for all $p \in [\tau(r)+1]$ and $q\geq 0$.
\item ${}^r\!\EE_2^{p,0} = 0$ for all $p \in [\tau(r)+1]$ and $q \geq 0$. 
\item For all $p \in [\tau(r)]$ and $q\geq 0$, the map $\dd_1^{p,q}:  \Hb^q(G_{r-p}) \to \Hb^q(G_{r-p-1})$ is given by
\[
\dd_1^{p,q} = \left\{\!\!\begin{array}{ll}
	\Hb^q(\iota_{r-p-1}) & \mbox{if } p \mbox{ is even,} \\
	0 & \mbox{if } p \mbox{ is odd.}
\end{array}\right.
\]
\end{enumerate}
\end{cor}
\begin{proof}
Since the inflation maps in bounded cohomology whose inducing epimorphisms have amenable kernels are isomorphisms \cite[Cor. 8.5.2]{Monod-Book}, Condition (MQ3) induces the following commutative diagram for all $r \in [R]$, $p \in [r]$ and $q \in [\tau(r)-1]$: \vspace{-3pt}
\begin{equation*} \label{projection_cohom}
		\begin{gathered}	
		\xymatrixcolsep{5.5pc}
		\xymatrixrowsep{1.5pc}
		\xymatrix{\Hb^q(H_{r,p-1}) \ar[r]^{\Hb^q(j_{r,p})} & \Hb^q(H_{r,p})  \\
		\Hb^q(G_{r-p}) \ar[u]_{\cong}^{\Hb^q(\pi_{r,p})} \ar[r]^{\Hb^q(\iota_{r-p-1})} & \Hb^q(G_{r-p-1}) \ar[u]_{\Hb^q(\pi_{r,p-1})}^{\cong}} 
		\end{gathered} \vspace{-3pt}
\end{equation*} 
The corollary follows now from \autoref{BasicSS}.
\end{proof}

If for any $s \in [R-1]$ and any $q \geq 0$ we abbreviate \vspace{-1.5pt}
\[
\HH^{q}_{s} := \Hb^{q}(G_{s}) \qand \iota_{s}^q:= \Hb^q(\iota_{s}), \vspace{-1.5pt}
\]
then the first page of ${}^r\!\EE^{\bcdot, \bcdot}_1$ is given by \autoref{fig:sampleE}. 
% Figure environment removed
The terms to the left of the dotted line are explained by \autoref{SpectralSequencesMain} (ii), while the behavior of the terms to the right of the dotted line is \emph{a priori} not understood. According to \autoref{SpectralSequencesMain} (iii), the parity of $\tau(r)$ determines the pattern of the arrows just to the left of the dotted line: the blue labels correspond to the case in which $\tau(r)$ is odd, and the red ones underneath correspond to the even case. 
\begin{rem}  Let $(G_r,X(r))_{r \in [R]}$ be a measurable  $(R, \gamma, \tau)$-Quillen family. We say that $q_0 \in \mathbb N$ is an \emph{initial parameter} for this measurable Quillen family if
\[
 \Hb^q(\iota_r):  \Hb^q(G_{r+1}) \to \Hb^q(G_r) \text{ is an isomorphism for all }q\leq q_0 \text{ and }r<R. 
\]
In theory, one would always like to work with the optimal initial parameter
\[
q_0 := \sup\{q \mid \Hb^q(\iota_r): \Hb^q(G_{r+1}) \to \Hb^q(G_r) \text{ is an isomorphism for all }r<R\},
\]
but in practice, the optimal initial parameter is not known. If no \emph{a priori} infomation about the bounded cohomology of the group $G_r$ is available, then one can always work with $q_0 := 1$, since $\Hb^1(G)$ vanishes for lcsc groups $G$. In some cases of interest, better initial parameters are available. Notably, if all $G_r$ are connected semisimple Lie groups with finite center and either all of the groups $G_r$ are of Hermitian type or all are of non-Hermitian type, then $q_0 \geq 2$. 
\end{rem}
To state the quantitative version of our bc-stability criterion, we introduce the following
\begin{defn} Let $(G_r,X(r))_{r \in [R]}$ be a measurable  $(R, \gamma, \tau)$-Quillen family with initial parameter $q_0 \in \mathbb N$. We define the associated \emph{dual acyclicity range} and \emph{dual transitivity function} to be functions $\widetilde{\gamma},\widetilde{\tau}:\,\bbN\times [R-1] \to \bbN \cup \{\pm\infty\}$ defined by the formulae 
\begin{equation*}
	\widetilde{\gamma}(q, r) := \min_{j=q_0}^q \big\{\gamma\big(r+1-2(q-j)\big) - j\big\} \qand \widetilde{\tau}(q,r) :=  \min_{j=q_0}^q \big\{\tau\big(r+1-2(q-j)\big) - j\big\}.\vspace{3pt}
\end{equation*}
for $r + 1 - 2(q-q_0) \geq 0$, and $\widetilde\gamma(q,r) = \widetilde\tau(q,r) = -\infty$ otherwise. 
\end{defn}
We state now our main result:
\begin{thm}\label{MainTheorem} Let $(G_r,X(r))_{r \in [R]}$ be a measurable  $(R, \gamma, \tau)$-Quillen family with initial parameter $q_0 \geq 1$ and associated dual acyclicity range $\widetilde{\gamma}$ and dual transitivity function $\widetilde{\tau}$. Then the inclusion $\iota_r$ induces an isomorphism (resp. an injection) 
\[\Hb^q(\iota_r): \Hb^q(G_{r+1}) \, \xrightarrow{\cong \ } \, \Hb^q(G_r) \quad (\text{resp. }\Hb^{q+1}(\iota_r): \Hb^{q+1}(G_{r+1}) \hookrightarrow \Hb^{q+1}(G_r)).
\] 
provided $r \in [R-1]$ and $q \geq 0$ satisfy the condition
\begin{equation}
\min\{\widetilde{\gamma}(q,r), \widetilde{\tau}(q,r)-1\} \geq 0.
\end{equation}
% $\min\{\widetilde{\gamma}(q,r), \widetilde{\tau}(q,r)-1\} \geq 0$ the inclusion $\iota_r$ induces the isomorphism (resp. the injection) 
%\[\Hb^q(\iota_r): \Hb^q(G_{r+1}) \, \xrightarrow{\cong \ } \, \Hb^q(G_r) \quad (\text{resp. }\Hb^{q+1}(\iota_r): \Hb^{q+1}(G_{r+1}) \hookrightarrow \Hb^{q+1}(G_r)).
%\] 
\end{thm}

An immediate corollary of this quantitative bc-stability theorem is \autoref{MainThmQual}, as stated in the introduction.
\begin{proof}[Proof of \autoref{MainThmQual} from \autoref{MainTheorem}] We may assume without loss of generality that $q_0 = 1$. Since $\gamma(r) \to \infty$ and $\tau(r) \to \infty$, we find for every $q \geq q_0$ some $r(q) \in \mathbb N$ such that for all $j \in \{q_0, \ldots, q\}$ and all $r \geq r(q)$, we have
\[
 \gamma\big(r+1-2(q-j)\big) \geq  j \qand  \tau\big(r+1-2(q-j)\big) \geq j+1.
\]
Then \autoref{MainTheorem} implies the chain of isomorphisms \vspace{-3pt}
\[
	\Hb^q(G_{r(q)}) \xleftarrow{\ \cong} \, \Hb^q(G_{r(q)+1}) \xleftarrow{\ \cong} \, \Hb^q(G_{r(q)+2}) \xleftarrow{\ \cong} \, \cdots,
\] 
which is the desired bc-stability.
\end{proof}
The functions $\widetilde{\gamma}(q,r)$ and $\widetilde{\tau}(q,r)$ have been defined so that they satisfy the following lemma, which will allow us to make inductive arguments:
\begin{lem}\label{Combinatorics} Let $r,\, q \in \bbN$, $q \geq q_0$, and assume that $\min\{\widetilde{\gamma}(q,r), \widetilde{\tau}(q,r)-1\} \geq 0$. Then the following hold:
\begin{enumerate}[label = \emph{(\roman*)},leftmargin=25pt]
\item $\gamma(r+1) \geq q$ and $\tau(r+1) \geq q+1$. %FormerD
\item $\min\{\widetilde{\gamma}(q-1,r),\widetilde{\tau}(q-1,r)-1\} \geq 0$. %FormerA
\item If $p \in [q-q_0]$ is odd, then $\min\{\widetilde{\gamma}(q-p, r-p-1),\widetilde{\tau}(q-p, r-p-1)-1\} \geq 0$. %FormerB
\item If $p \in [q-q_0]$ is even, then $\min\{\widetilde{\gamma}(q-p-1,r-p-2),\widetilde{\tau}(q-p-1,r-p-2)-1\} \geq 0$. %FormerC
\end{enumerate}
\end{lem}
\begin{proof} The assumption means that for all $j \in \{q_0, \dots, q\}$ we have
\begin{align} 
	r+1-2(q-j) &\geq 0, \qand \label{eq:cond_isom_welldef} \\[-3pt]
	\gamma\big(r+1-2(q-j)\big) &\geq j, \quad \tau\big(r+1-2(q-j)\big) \geq j + 1.\label{eq:cond_isom_bis} \vspace{-5pt}
\end{align}
\begin{enumerate}[label=(\roman*),leftmargin=25pt]
\item Choose $j=q$ in \eqref{eq:cond_isom_bis}.
\item If $q=q_0$, then $r+1-2(q-1-q_0) \geq 0$ and $\widetilde\gamma(q-1,r) = \widetilde\tau(q-1,r) = \infty$ for any $r$. Assume now that $q > q_0$, and let $j \in \{q_0,\ldots,q-1\}$. Observe that 
\[
	r+1-2(q-1-j) = r+1-2\big(q-(j+1)\big). \vspace{-1pt}
\]
Since $j+1 \in \{q_0,\ldots,q\}$, from this equality combined with \eqref{eq:cond_isom_welldef} and \eqref{eq:cond_isom_bis} we deduce  \vspace{-1pt}
\begin{align*}
	\qquad r+1-2(q-1-j) &\geq 0, \qand \\[-3pt]
	\qquad \gamma\big(r+1-2(q-1-j)\big) &\geq j+1 > j, \quad \tau\big(r+1-2(q-1-j)\big) \geq j+2 > j + 1,
\end{align*} 
which establishes the claim. 

\item Observe that for any odd $p \in [q-q_0]$ and any $j \in \{q_0,\ldots,q-p\}$, we have 
\[
	r-p - 2(q-p-j) = r+1 -2 \left(q-\left(j+\frac{p-1}2\right)\right), 
\]
and $j+(p-1)/2 \in \{q_0 + (p-1)/2,\ldots,q-(p+1)/2\} \subset \{q_0,\ldots,q\}$. Just as in (ii), the claim follows now from \eqref{eq:cond_isom_welldef} and \eqref{eq:cond_isom_bis}. 


\item 
For any even $p \in [q-q_0]$ and any $j \in \{q,\ldots,q-p\}$, we have
\[
	r-p-1-2(q-p-1-j) = r+1-2\left(q-\left(j+\frac{p}2\right)\right),
\]
and $j + p/2 \!\in\! \{q_0 + p/2,\ldots,q-p/2\} \!\subset\! \{q_0,\ldots,q\}$, and conclude by \eqref{eq:cond_isom_welldef} and \eqref{eq:cond_isom_bis}. \qedhere
\end{enumerate}
\end{proof}
\section{Proof of the main theorem}
The remainder of this article is devoted to deriving \autoref{MainTheorem} from \autoref{SpectralSequencesMain}. We consider a measurable  $(R, \gamma, \tau)$-Quillen family $(G_r,X(r))_{r \in [R]}$ with initial parameter $q_0 \geq 1$, dual acyclicity range $\widetilde{\gamma}$ and dual transitivity function $\widetilde{\tau}$ and abbreviate
\[
\HH^{q}_{s} := \Hb^{q}(G_{s}) \qand \iota_{s}^q:= \Hb^q(\iota_{s})
\]
as before. We consider the following statement for $q \in \mathbb N$ and $r < R$:
\begin{itemize}
\item[{($S_{q,r}$)}] $\iota^q_r: \HH^q_{r+1} \to \HH^q_r$ is an isomorphism and $\iota^{q+1}_{r}: \HH^{q+1}_{r+1} \to \HH^{q+1}_r$ is an injection.
\end{itemize}
The conclusion of \autoref{MainTheorem} then says that for all $q \in \bbN$, the following statement ($S_q$) holds true:
\begin{itemize}
\item[{($S_q$)}] If $r<R$ satisfies $\min\{\widetilde{\gamma}(q,r), \widetilde{\tau}(q,r)-1\} \geq 0$, then ($S_{q,r}$) holds.
\end{itemize}
If $q < q_0$, then $r + 1 - 2(q-q_0) \geq 0$ and hence by definition $\widetilde{\gamma}(q,r) = \widetilde{\tau}(q,r) = \infty$. Since $q_0$ is an initial parameter, both $\iota^q_r$ and $\iota^{q+1}_r$ are isomorphisms for all $r < R$. Thus ($S_q$) holds for all $q < q_0$. We now proceed to prove ($S_q$) for $q \geq q_0$ by induction on $q$. For this purpose, assume that $q \geq q_0$ and that ($S_{q'}$) holds for all $q' < q$. In view of \autoref{Combinatorics}, we have:
\begin{lem}\label{ABC} For $r < R$ with $\min\{\widetilde{\gamma}(q,r), \widetilde{\tau}(q,r)-1\} \geq 0$, the following hold:
\begin{enumerate}[label=\emph{(\Alph*)}]
	\item The map $\iota_{r}^q$ is injective. %and an isomorphism for $p \in \{1,\ldots,q\}$ even,
	\item The map $\iota^{q-p}_{r-p-1}$ is an isomorphism for all odd numbers $p \in [q]$.
	\item The map $\iota_{r-p-2}^{q-p}$ is an injection for all even numbers $p \in [q]$.
\end{enumerate} 
\end{lem}
\begin{proof} (A) By \autoref{Combinatorics} (ii), the inequality $\min\{\widetilde{\gamma}(q-1,r),\widetilde{\tau}(q-1,r)-1\} \geq 0$ holds. Thus, we may apply ($S_{q-1}$) to deduce the claim.

(B) If $p > q - q_0$, then $q-p < q_0$ and hence $\iota^{q-p}_{r-p-1}$ is an isomorphism by the initial condition. If $p \leq q-q_0$, then by \autoref{Combinatorics} (iii) we have $\min\{\widetilde{\gamma}(q-p, r-p-1),\widetilde{\tau}(q-p, r-p-1)-1\} \geq 0$. We may thus use ($S_{q-p}$) to show that ($S_{q-p, r-p-1}$) is true, implying the claim.

(C) The case $p > q-q_0$ is again covered by the initial condition, and if $p \leq q-q_0$, then by \autoref{Combinatorics} (iv) we have $\min\{\widetilde{\gamma}(q-p-1,r-p-2),\widetilde{\tau}(q-p-1,r-p-2)-1\} \geq 0$. Thus, by ($S_{q-p-1}$), the statement ($S_{q-p-1, r-p-2}$) applies.
\end{proof}
To establish ($S_q$), we fix a natural number $r < R$ such that $\min\{\widetilde{\gamma}(q,r), \widetilde{\tau}(q,r)-1\} \geq 0$ and consider the spectral sequence $\EE_\bcdot^{\bcdot,\bcdot} := {}^{r+1}\!\EE_\bcdot^{\bcdot,\bcdot}$  from \autoref{SpectralSequencesMain}. By \autoref{Combinatorics} (i), we have
\begin{equation}\label{D}
	\gamma(r+1) \geq q \qand \tau(r+1) \geq q+1,
\end{equation} 
and thus \autoref{SpectralSequencesMain} gives
\begin{equation} \label{eq:Gr+1}
\begin{gathered}
\begin{array}{lll}
	\EE_1^{p',q'} \!\!\!&= \HH^{q'}_{r+1-p'} &  \mbox{ for } p' \in [q+2],\, q' \geq 0, \\[2pt]
	\EE_2^{p', 0} \!\!\!&= 0 & \mbox{ for } p' \in [q+2],\\
	\dd_1^{p',q'} \!\!\!&= \left\{\!\!\begin{array}{ll}
	\iota^{q'}_{r-p'} & \mbox{if } p' \mbox{ is even,} \\
	0 & \mbox{if } p' \mbox{ is odd,}
\end{array}\right.  &  \mbox{ for } p' \in [q+1],\, q' \geq 0,
\\
	\EE_\infty^{t} \!\!\!&= 0 & \mbox{ for } t \in [q + 1].
\end{array}
\end{gathered}
\end{equation}
The second equality is \eqref{T3}. We are going to show now that
\begin{equation}\label{T1}
\EE_2^{0, q+1} \cong \ker(\iota^{q+1}_r) \qand 
\EE_2^{1,q} \cong \coker(\iota^{q}_{r}),
\end{equation}
and that
\begin{equation}\label{T2}
\EE_2^{p', q'} = 0 \text{ for all }(p',q') \in \{0,\dots, q\}^2 \text{ with }p'+q' = q+2.
\end{equation}

% Figure environment removed
These computations will be sufficient to establish ($S_q$): As indicated in \autoref{fig:IIE_2}, the differentials emanating from $\EE_2^{0, q+1}$ and $\EE_2^{1,q}$ as well as all those emanating from the positions $(0,q+1)$ and $(1,q)$ on any of the following pages end up on the diagonal $p'+q' = q+2$. Since this diagonal contains only zeros up to row $q$, we conclude that 
 \[
\EE_\infty^{0, q+1} \cong \EE_2^{0, q+1} \cong \ker(\iota^{q+1}_r)%, \quad E_2^{0,q} = \ker(\iota_r^q) 
\qand \EE_\infty^{1,q} \cong \EE_2^{1,q} \cong \coker(\iota^{q}_{r}).
\]
In particular, it follows that 
$
\ker(\iota^{q+1}_r) \oplus \coker(\iota^{q}_{r}) \hookrightarrow E_\infty^{q+1} = 0.
$
This gives the injectivity of $\iota_r^{q+1}$ and the surjectivity of $\iota_r^q$. Since $\iota_r^q$ is injective by (A), the claim ($S_q$) is proven. We have thus reduced the proof of \autoref{MainTheorem} to \eqref{T1} and \eqref{T2}.

To prove \eqref{T1} and \eqref{T2}, we only need to consider the few arrows on the first page of the spectral sequence depicted in \autoref{fig:IIE_1}. 
The statement \eqref{T1} is immediate from the fact that the leftmost maps in the $q$-th and $(q+1)$-th row of $\EE_1^{\bcdot, \bcdot}$ are respectively given by
\[
\HH^q_{r+1} \xrightarrow{\iota_r^q} \HH^q_{r} %\xrightarrow{0} \HH^{q}_{r-1} 
\qand \HH^{q+1}_{r+1} \xrightarrow{\iota^{q+1}_r} \HH^{q+1}_{r} %\xrightarrow{0} \HH^{q+1}_{r-1}.
\]

As for \eqref{T2}, we first consider $q' \in \{1, \ldots, q\}$ and define $p' := q-q' \in [q-1]$ so that $p'+q' = q$. By \eqref{eq:Gr+1}, the two maps
$% \begin{equation*}
\EE_1^{p'+1,q'} \to \EE_1^{p'+2,q'}  \to \EE_1^{p'+3,q'}  
$%\end{equation*}
are given by
\[
\left\{ \begin{array}{ll}
	 \HH^{q-p'}_{r-p'} \xrightarrow{0} \HH^{q-p'}_{r-p'-1} \xrightarrow{\iota^{q-p'}_{r-p'-2}} \HH^{q-p'}_{r-p'-2}, & \text{if }p' \text{ is even,} \\
	 \HH^{q-p'}_{r-p'} \xrightarrow{\iota^{q-p'}_{r-p'-1}} \HH^{q-p'}_{r-p'-1} \xrightarrow{0} \HH^{q-p'}_{r-p'-2}, & \text{if }p' \text{ is odd.}  
\end{array} \right.
\]
From (B) and (C) of \autoref{ABC}, we deduce that 
\[
\EE_2^{p'+2, q'} = 0 \quad \text{for all } (p', q') \in [q]^2  \text{ with }p'+q' = q,
\]
where the case $(p',q')=(q,0)$ is covered in \eqref{eq:Gr+1}. This concludes the proof of \eqref{T2}. \qed
% Figure environment removed

\begin{thebibliography}{123}
\bibitem{Bestvina}
	M. Bestvina,
	\emph{Homological stability of Aut($F_n$) revisited}. 
	Hyperbolic geometry and geometric group theory, 1-11,
	Adv. Stud. Pure Math. 73, 
	Math. Soc. Japan, 
	Tokyo, 2017. 
	
\bibitem{Buehler}
	T. B\"uhler,
	\emph{On the algebraic foundations of bounded cohomology}.  
	Mem. Amer. Math. Soc. 214 (2011), no. 1006, xxii+97 pp.
	
\bibitem{Burger-Monod1}
	M. Burger, N. Monod, 
	\emph{Bounded cohomology of lattices in higher rank Lie groups.} 
	J. Eur. Math. Soc. (JEMS) 1 (1999), no. 2, 199-235. 
	
%\bibitem{Burger-Monod2}
%	M. Burger, N. Monod,
%	\emph{Continuous bounded cohomology and applications to rigidity theory}. 
%	Geom. Funct. Anal. 12 (2002), no. 2, 219-280. 
	
\bibitem{Burger-Monod3}
	M. Burger, N. Monod, 
	\emph{On and Around the Bounded Cohomology of $\SL_2$}. In: M. Burger, A. Iozzi, A. (eds) Rigidity in Dynamics and Geometry. Springer, Berlin, Heidelberg, 2002. 

\bibitem{DM-Thesis}
	C. De la Cruz Mengual, 
	\emph{On Bounded-Cohomological Stability for Classical Groups.} 
	ETH Zurich (2019), Zurich.
	
\bibitem{DM+Hartnick}
	C. De la Cruz Mengual, T. Hartnick,
	\emph{Stabilization of Bounded Cohomology for Classical Groups.}
	arXiv preprint (2022), \href{https://arxiv.org/abs/2201.03879}{arXiv:2201.03879}.

%\bibitem{Devyatov}
%	R. Devyatov,
%	\emph{Generically transitive actions on multiple flag varieties}.
%	Int. Math. Res. Not. IMRN (2014), no. 11, 2972-2989. 
	
%\bibitem{Dupont}
%    	J.L. Dupont,
%    	\emph{Bounds for characteristic numbers of flat bundles},
%  	Algebraic topology, Aarhus 1978 (Proc. Sympos., Univ. Aarhus, Aarhus, 1978), 	Lecture Notes in Math., vol. 763, Springer, Berlin, 1979,
%  	pp. 109--119.
%
%\bibitem{EW}
%	M. Einsiedler; T. Ward,
%	\emph{Ergodic theory with a view towards number theory}.
%	Graduate Texts in Mathematics, 259. Springer-Verlag London, Ltd., London, 2011.
	
\bibitem{Essert}
	J. Essert,
	\emph{Homological stability for classical groups}. 
	Israel J. Math. 198 (2013), no. 1, 169-204. 

%\bibitem{Federer}
%	H. Federer, 
%	\emph{Geometric measure theory}.
%	Die Grundlehren der mathematischen Wissenschaften, Band 153, 
%	Springer-Verlag, New York Inc., New York 1969 xiv+676 pp. 
	
\bibitem{Frigerio}
	R. Frigerio, 
	\emph{Bounded cohomology of discrete groups}.
	Mathematical Surveys and Monographs, 227. 
	American Mathematical Society, 
	Providence, RI, 2017. 
	
%\bibitem{GHV} 
%	W. Greub, S. Halperin, R. Vanstone,
%	\emph{Connections, curvature and cohomology, Volume III: Cohomology of principal bundles and homogeneous spaces}. 
%	Pure and Applied Mathematics, Vol. 47-III. 
%	Academic Press [Harcourt Brace Jovanovich, Publishers], New York--London, 1976.

\bibitem{Gromov}
	M. Gromov,
	\emph{Volume and bounded cohomology}.
	Inst. Hautes \'Etudes Sci. Publ. Math. No. 56 (1982), 5-99 (1983). 

\bibitem{Harer}
	J.L. Harer, 
	\emph{Stability of the homology of the mapping class groups of orientable surfaces}.
	Ann. of Math. (2) 121 (1985), no. 2, 215-249.
	
\bibitem{HatVogt}
	A. Hatcher, K. Vogtmann, 
	\emph{Homology stability for outer automorphism groups of free groups}. 
	Algebr. Geom. Topol. 4 (2004), 1253-1272. 
	
%\bibitem{Hel} S. Helgason.
%\newblock {\em Differential Geometry, Lie Groups and Symmetric Spaces}. Corrected reprint of the 1978 original. 
%\newblock Graduate Studies in Mathematics, 34. American Mathematical Society, Providence, RI (2001).
	
\bibitem{vdK}
	W. van der Kallen, 
	\emph{Homology stability for linear groups}.
	Invent. Math. 60 (1980), no. 3, 269-295. 

%\bibitem{LeBoudec-MatteBon}
%	A. Le Boudec, N. Matte Bon,
%	\emph{Locally compact groups whose ergodic or minimal actions are all free.}
%	Int. Math. Res. Not. IMRN 2020, no. 11, 3318-3340.
	
\bibitem{Quillen}
	D. Quillen,
	\emph{Notebook 1974-1}. 
	Available at \href{http://www.claymath.org/publications/quillen-notebooks}{http://www.claymath.org/publications/quillen-notebooks}.
	
%\bibitem{Mattila}
%	P. Mattila, R.D. Mauldin, R. Daniel,
%	\emph{Measure and dimension functions: measurability and densities.}
%	Math. Proc. Cambridge Philos. Soc. 121 (1997), no. 1, 81-100.

\bibitem{McCleary}
	J. McCleary,
	\emph{A user's guide to spectral sequences}. 
	Second edition. Cambridge Studies in Advanced Mathematics, 58.
	Cambridge University Press, Cambridge, 2001. 
		
%\bibitem{Toda-Mimura}
%	M. Mimura, H. Toda,
%	\emph{Topology of Lie groups. I, II}. 
%	Translated from the 1978 Japanese edition by the authors. Translations of Mathematical Monographs, 91. 
%	American Mathematical Society, Providence, RI, 1991.
	
%\bibitem{Monod-Survey}
%	N. Monod,
%	\emph{An invitation to bounded cohomology}. 
%	International Congress of Mathematicians. Vol. II, 1183-1211, 
%	Eur. Math. Soc., 
%	Z\"urich, 2006. 

\bibitem{Monod-Book} 
	N. Monod,  
	\emph{Continuous bounded cohomology of locally compact groups}. 
	Lecture Notes in Mathematics, 1758. 
	Springer-Verlag, 
	Berlin, 2001. 
	
%\bibitem{Monod-sot}
%	N. Monod, 
%	\emph{On the bounded cohomology of semi-simple groups, S-arithmetic groups and products}.
%	J. Reine Angew. Math. 640 (2010), 167-202. 
	
\bibitem{Monod-Stab} 
	N. Monod, 
	\emph{Stabilization for $\SL_n$ in bounded cohomology}. 
	Discrete geometric analysis, 191-202, Contemp. Math., 347, 
	Amer. Math. Soc., 
	Providence, RI, 2004. 
	
\bibitem{Monod-Vanish} 
	Monod, N.  
	\emph{Vanishing up to the rank in bounded cohomology}. 
	Math. Res. Lett. 14 (2007), 
	no. 4, 681-687. 

%\bibitem{Pieters}
%	H. Pieters,
%	\emph{The boundary model for the continuous cohomology of ${\rm Isom}^+(\bbH^n)$}.
%	Groups, Geometry and Dynamics 12(4):1239-1263
	
\bibitem{Popov}
	V.L. Popov, 	
	\emph{Generically multiple transitive algebraic group actions}. 
	Algebraic groups and homogeneous spaces, 481-523,
	Tata Inst. Fund. Res. Stud. Math., 19, Tata Inst. Fund. Res., Mumbai, 2007. 

%\bibitem{Sprehn-Wahl1}
%	D. Sprehn, N. Wahl,
%	\emph{Forms over fields and Witt's lemma.}
%	Math. Scand. 126 (2020), no. 3, 401-423.
	
\bibitem{Sprehn-Wahl2}
	D. Sprehn, N. Wahl,
	\emph{Homological stability for classical groups.}
	Trans. Amer. Math. Soc. 373 (2020), no. 7, 4807-4861.
	
%\bibitem{Stas} 
%    	J.D. Stasheff, 
%    	\emph{Continuous cohomology of groups and classifying spaces}. 
%    	Bull. Amer. Math. Soc. 84 (1978), no. 4, 513-530.

%\bibitem{Taylor}
%	D.E. Taylor, 
%	\emph{The geometry of the classical groups.} 
%	Sigma Series in Pure Mathematics, 9. Heldermann Verlag, Berlin, 1992. 

\bibitem{Vogtmann}
	K. Vogtmann,
	\emph{A Stiefel complex for the orthogonal group of a field.}
	Comment. Math. Helv. 57 (1982), no. 1, 11-21.

\bibitem{Weibel}
	C.A. Weibel, 
	\emph{An introduction to homological algebra}. 
	Cambridge Studies in Advanced Mathematics, 38. Cambridge University Press, Cambridge, 1994.
	
\bibitem{Zimmer}
	R. J. Zimmer, 
	\emph{Ergodic theory and semisimple groups}.
	Monographs in Mathematics, Vol. 81. Boston-Basel-Stuttgart: Birkh\"auser, 1984.

\end{thebibliography}
\end{document}