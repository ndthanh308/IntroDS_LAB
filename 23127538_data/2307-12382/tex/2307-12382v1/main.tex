% \documentclass[journal]{vgtc}                     % final (journal style)
% \documentclass[journal,hideappendix]{vgtc}        % final (journal style) without appendices
% \documentclass[review,journal]{vgtc}              % review (journal style)
% \documentclass[review,journal,hideappendix]{vgtc} % review (journal style)
%\documentclass[widereview]{vgtc}                  % wide-spaced review
\documentclass[preprint,journal]{vgtc}            % preprint (journal style)


%% Uncomment one of the lines above depending on where your paper is
%% in the conference process. ``review'' and ``widereview'' are for review
%% submission, ``preprint'' is for pre-publication in an open access repository,
%% and the final version doesn't use a specific qualifier.

%% If you are submitting a paper to a conference for review with a double
%% blind reviewing process, please use one of the ``review'' options and replace the value ``0'' below with your
%% OnlineID. Otherwise, you may safely leave it at ``0''.
\onlineid{1197}

%% In preprint mode you may define your own headline. If not, the default IEEE copyright message will appear in preprint mode.
\preprinttext{To appear in IEEE Transactions on Visualization and Computer Graphics.}

%% In preprint mode, this adds a link to the version of the paper on IEEEXplore
%% Uncomment this line when you produce a preprint version of the article 
%% after the article receives a DOI for the paper from IEEE
\ieeedoi{xx.xxxx/TVCG.201x.xxxxxxx}

%% declare the category of your paper, only shown in review mode
\vgtccategory{Research}

%% please declare the paper type of your paper to help reviewers, only shown in review mode
%% choices:
%% * algorithm/technique
%% * application/design study
%% * evaluation
%% * system
%% * theory/model
\vgtcpapertype{Analytics \& Decisions}

%% Paper title.
\title{\name{}: Visualizing and Understanding Commonsense Reasoning Capabilities of Natural Language Models}

%% Author ORCID IDs should be specified using \authororcid like below inside
%% of the \author command. ORCID IDs can be registered at https://orcid.org/.
%% Include only the 16-digit dashed ID.
\author{%
  Xingbo Wang,
  Renfei Huang, 
  Zhihua Jin,
  Tianqing Fang,
  and Huamin Qu
}

\authorfooter{
  %% insert punctuation at end of each item
  \item
  	Xingbo Wang is with Weill Cornell Medical College, Cornell University. This work was done at the Hong Kong University of Science and Technology.
  	E-mail: xingbo.wang@\{connect.ust.hk, med.cornell.edu\}.
  \item
  	Renfei Huang, Zhihua Jin, Tianqing Fang, and Huamin Qu are with the Hong Kong University of Science and Technology. E-mail: \{rhuangan, zjinak, tfangaa, huamin\}@ust.hk.

}

%% Abstract section.
\abstract{%
Recently, large pretrained language models have achieved compelling performance on commonsense benchmarks. 
% Motivation of explainability
Nevertheless, it is unclear what commonsense knowledge the models learn and whether they solely exploit spurious patterns.
% Limitations in XAI techniques
Feature attributions are popular explainability techniques that identify
% methods are popular for explaining models by identifying 
important input concepts for model outputs. However, commonsense knowledge tends to be implicit and rarely explicitly presented in inputs. These methods cannot infer models' implicit reasoning over mentioned concepts.
% Therefore, these methods cannot help infer how models perform implicit reasoning on mentioned concepts.
% since commonsense is rarely stated in the input, these methods cannot help infer how models build latent connections between mentioned concepts.
We present \name{}, a visual explanatory system that utilizes external commonsense knowledge bases to contextualize model behavior 
% on different concepts and their underlying relations 
for commonsense question-answering.
Specifically, we extract relevant commonsense knowledge in inputs as references to align model behavior with human knowledge.
Our system features multi-level visualization 
and interactive model probing and editing for different concepts and their underlying relations. 
Through a user study,
we show that \name{} helps NLP experts conduct a systematic and scalable visual analysis of models' relational reasoning over concepts in different situations.
}

%% Keywords that describe your work. Will show as 'Index Terms' in journal
%% please capitalize first letter and insert punctuation after last keyword
\keywords{Commonsense reasoning, visual analytics, XAI, natural language processing}

%% A teaser figure can be included as follows

%%%%%%%%%%%%%%%%%%%%%%%%%%%%%%%%%%%%%%%%%%%%%%%%%%%%%%
%%  OLD version teaser %%  
% \teaser{
%   % Figure removed
%   \caption{
%   \renfeivis{The user interface of \name{} consists of three views. The \gv{} (A) summarizes the model performance by the projection plots of question stems and target concepts and the relations between them. The \sv{} (B) summarizes the context alignment between model behavior and \cpn knowledge over different subsets. The \iv{} (C) provides the statistics and detailed local explanations of instances selected using \gv{} or \sv{}. The selected instance is reflected in the \gv{} as larger points. It also enables users to probe the model behavior by editing the questions directly in the interface. }
%     }
%   \label{fig:teaser}
% }
%%%%%%%%%%%%%%%%%%%%%%%%%%%%%%%%%%%%%%%%%%%%%%%%%%%%%%

\teaser{
  % Figure removed
  \vspace{-3mm}
  \caption{\vis{Since commonsense knowledge is not explicitly stated, it is challenging to conduct a scalable analysis of what commonsense knowledge NLP models do (not) learn. We employ a knowledge graph to derive implicit commonsense in the model input as context information. Then, we use it to align model behavior with human reasoning through multi-level interactive visualizations. Thereafter, users can understand, diagnose, and edit specific knowledge areas where models do not perform well.}}
  \label{fig:workflow_teaser}
}

%% Uncomment below to disable the manuscript note
%\renewcommand{\manuscriptnotetxt}{}

%% Copyright space is enabled by default as required by guidelines.
%% It is disabled by the 'review' option or via the following command:
%\nocopyrightspace


%%%%%%%%%%%%%%%%%%%%%%%%%%%%%%%%%%%%%%%%%%%%%%%%%%%%%%%%%%%%%%%%
%%%%%%%%%%%%%%%%%%%%%% LOAD PACKAGES %%%%%%%%%%%%%%%%%%%%%%%%%%%
%%%%%%%%%%%%%%%%%%%%%%%%%%%%%%%%%%%%%%%%%%%%%%%%%%%%%%%%%%%%%%%%

%% Tell graphicx where to find files for figures when calling \includegraphics.
%% Note that due to the \DeclareGraphicsExtensions{} call it is no longer necessary
%% to provide the the path and extension of a graphics file:
%% % Figure removed is completely sufficient.
\graphicspath{{figs/}{figures/}{pictures/}{images/}{./}} % where to search for the images

%% Only used in the template examples. You can remove these lines.
\usepackage{tabu}                      % only used for the table example
\usepackage{booktabs}                  % only used for the table example
\usepackage{lipsum}                    % used to generate placeholder text
\usepackage{mwe}                       % used to generate placeholder figures

%%%%%%%%% Watermark %%%%%%%%%
% \usepackage{background}
% \backgroundsetup{scale=4.4, opacity=1, contents={Submitted Draft for VIS'23}}

%% We encourage the use of mathptmx for consistent usage of times font
%% throughout the proceedings. However, if you encounter conflicts
%% with other math-related packages, you may want to disable it.
\usepackage{mathptmx}                  % use matching math font
% \usepackage{enumitem}
\usepackage{paralist}
\usepackage{comment}
\usepackage{amsmath}
\usepackage{amssymb}
\usepackage{amsfonts}
% \usepackage[dvipsnames]{xcolor}



%%%%%%%%%%%%%%%%%%%%%%%%%%%%%%%%%%%%%%%%%%%%%%%%%%%%%%%%%%%%%%%%
%%%%%%%%%%%%%%%%%%%%%% command definition %%%%%%%%%%%%%%%%%%%%%%
%%%%%%%%%%%%%%%%%%%%%%%%%%%%%%%%%%%%%%%%%%%%%%%%%%%%%%%%%%%%%%%%
\newcommand{\xingbo}[1]{{\color{black} #1}}
\newcommand{\renfei}[1]{{\color{black} #1}}
\newcommand{\renfeivis}[1]{{\color{black} #1}}
\newcommand{\rev}[1]{{\color{black} #1}}
\newcommand{\vis}[1]{{\color{black} #1}}
\newcommand{\todo}[1]{{\color{black} #1}}
\newcommand{\referappendix}[1]{{\color{black} #1}}
%%%%%%%%%%%%%%%%%%%%%%%%%%%%%%%%%
% conditional acceptance revision
%%%%%%%%%%%%%%%%%%%%%%%%%%%%%%%%%
\newcommand{\revv}[1]{{\color{black} #1}}
\newcommand{\revvv}[1]{{\color{black} #1}}

% global commands for the writing
\newcommand{\ie}{i.e.}
\newcommand{\eg}{e.g.}
\newcommand{\esp}{esp.}
\newcommand{\etal}{et al.}
\newcommand{\hl}[1]{\emph{``{#1}''}}
\newcommand{\imp}[1]{\textbf{\textit{{#1}}}}
\newcommand{\systemname}{{CommonsenseVIS}}
\newcommand{\name}{{\textit{CommonsenseVIS}}}
% system views
\newcommand{\gv}{Global View}
\newcommand{\sv}{Subset View}
\newcommand{\iv}{Instance View}
\newcommand{\cpn}{ConceptNet}

\def\subsectionautorefname{Section}
\def\subsubsectionautorefname{Section}
\def\appendixautorefname{Suppl.}

% background set up


\begin{document}

%%%%%%%%%%%%%%%%%%%%%%%%%%%%%%%%%%%%%%%%%%%%%%%%%%%%%%%%%%%%%%%%
%%%%%%%%%%%%%%%%%%%%%% START OF THE PAPER %%%%%%%%%%%%%%%%%%%%%%
%%%%%%%%%%%%%%%%%%%%%%%%%%%%%%%%%%%%%%%%%%%%%%%%%%%%%%%%%%%%%%%%
\firstsection{Introduction}
\maketitle
\section{Introduction}
Current quantum hardware is unable to carry out universal quantum computations due to the buildup of errors that occur during the computation. 
The magnitude of the individual error is currently above the value that the Threshold Theorem requires in order to kick-start quantum error correction and fault-tolerant quantum computation~\cite[Section 10.6]{nielsen_chuang_2010}. 
Although the experimentally achieved fidelity rates are promising and the error bounds are inching closer to the required threshold, we will have to work for the foreseeable future with quantum hardware with errors that build-up during the computation.  This implies that we can only do a limited number of steps before the output of the computation has become completely uncorrelated with the intended one.

For fault-tolerant quantum computing, we repeat four steps: 
1) We apply a number of single and two-qubit quantum gates, in parallel whenever possible; 
2) We perform a syndrome measurement on a subset of the qubits; 
3) We perform fast classical computations to determine which errors have occurred and how to correct them; 
and, 4) We apply correction terms based on the classical computations.
We then repeat these four steps with a next sequence of gates. 
These four steps are essential to fault-tolerant quantum computing. 


The starting point of this work is to use the four steps outlined above, not to carry out error correction and fault-tolerant computation, but to enhance short, constant-depth, {\em uncorrected} quantum circuits that perform single qubit gates and {\em nearest-neighbor} two qubit gates. 
Since in the long run we will have to implement error-correction and fault-tolerant computation anyhow, and this is done by such a four-step process, why not make other use of this architecture? Moreover, on some of the quantum hardware platforms, these operations are already in place.
Embracing this idea we naturally arrive at the question: what is the computational power of \textit{low-depth} quantum-classical circuits organized as in the four steps outlined above? 
We thus investigate circuits that execute a small, ideally constant, number of stages, where at each stage we may apply, in parallel, single qubit gates and {\em nearest-neighbor} two qubit gates, followed by measurements, followed by low-depth classical computations of which the outcome can control quantum gates in later stages. 
It is not clear, at first, whether such circuits, especially with constant depth, can do anything remotely useful. 
But we will see that this is indeed the case: many quantum computations can be done by such circuits in constant depth. 
By parallelizing quantum computations in this way, we improve the overall computational capabilities of these circuits, as we do not incur errors on qubits that are idle, simply because qubits are not idle for a very long time. 
Furthermore, reducing the depth of quantum circuits, at the cost of increasing width, allows the circuit to be run faster even if errors occur.

The first usage of such a four-step layout, not to do error correction, but to perform computations, can be found in the paradigm of measurement-based quantum computing~\cite{gottesman1999demonstrating,raussendorf2001one,jozsa2006introduction,clark2007generalised}: 
A universal form of quantum computing where a quantum state is prepared and operations are performed by measuring qubits in different bases, depending on previous measurements and intermediate measurements.

\citeauthor{PhamSvore2013} were the first to formalize the four-step protocol for performing computations~\cite{PhamSvore2013}. They included specific hardware topologies by considering two-dimensional graphs for imposing constraints on qubit interactions. In their model, they develop circuits for particularly useful multi-qubit gates, including specifying costs in the width, number of qubits, depth, number of concurrent time steps, size, and total number of non-Identity operations.
As a result, they find an algorithm that factors integers in polylogarithmic depth.
\citeauthor{Browne:2011} showed that the main tool in the work by \citeauthor{PhamSvore2013}, the fan-out gate, can also be replaced by additional log-depth classical computations in the measurement-based quantum computing setting~\cite{Browne:2011}.

More recently, \citeauthor{Cirac:2021} introduced a scheme to implement unitary operations involving quantum circuits combined with Local Operations and Classical Communication ($\mathsf{LOCC}$) channels: $\mathsf{LOCC}$-assisted quantum circuits~\cite{Cirac:2021}. Similarly to the four-step scheme we just described, they allow for a short depth circuit to be run on the qubits, followed by one round of $\mathsf{LOCC}$, in which ancilla qubits are measured and local unitaries are applied based on the measurement outcomes. They show that in this model any 1D transitionally invariant matrix-product state (MPS) with fixed bond dimension is in the same phase of matter as the trivial state. Similar ideas can be found in~\cite{TVV_NonAbelianTopologicalOrder_2022, tantivasadakarn2021long}.

In this work, we introduce a new model, called \textit{Local Alternating Quantum-Classical Computations} ($\LAQCC$). In this model we alternate between running quantum circuits (constrained by locality), ending in the measurement of a subset of qubits, and fast classical computations based on the measurement results. The outcome of the classical computations are then used to control future quantum circuits. We allow for flexibility in this model, by giving different constraints to the power of both the quantum circuits and the classical circuits as well as the number of alternations between them. 
Most attention will be given to $\LAQCC$ containing quantum circuits of constant depth, classical circuits of logarithmic depth and at most a constant number of alternations between them. 
Any circuit constructed in this model is considered to be of constant depth. 
We restrict ourselves to logarithmic depth classical computations, as this is the first natural and non-trivial extension beyond constant-depth classical computations. 
Constant-depth classical computations do however also have an equivalent constant-depth quantum implementation.

The definition of $\LAQCC$ sharpens the original definition of \citeauthor{PhamSvore2013} by adding constraints to the intermediate classical computations. This allows us to bound the power of $\LAQCC$ from above. 

The main result of \citeauthor{Cirac:2021}, that 1D translational invariant MPS with fixed bond dimension can be prepared by $\mathsf{LOCC}$-assisted circuits, relies on local symmetries of the MPS. These symmetries allow them to prepare local states (on a constant number of qubits) and glue them together by doing one round of the appropriate entangling measurement and corrections, after which they run a round of local unitaries to get the desired result. This general scheme for preparing states that exhibit an MPS description with the appropriate local symmetries requires only geometrically local unitaries and one round of measurement and corrections an therefore is accessible in $\LAQCC$. Studying different local symmetries, known as Symmetry Protected Topological (SPT) phases of matter, to find measurement-based constant depth circuits for states is a broad ongoing field of research~\cite{TVV_NonAbelianTopologicalOrder_2022, tantivasadakarn2021long, smith2023deterministic}. 
All these schemes have a $\LAQCC$ implementation.

%$\LAQCC$-circuits also exist for general schemes of preparing local states, based on the local tensors, and gluing them together using one round of entangled measurement and corrections, based on the local symmetry. 
%The main result of \citeauthor{Cirac:2021}, that 1D translational invariant MPS with fixed bond dimension can be prepared by $\mathsf{LOCC}$-assisted circuits, relies heavily on local symmetries of the MPS and as a result also has an equivalent $\LAQCC$ implementation. 
%The corrections applied after the measurement round are local unitaries depending on the local symmetries of the MPS. 

 

%This general scheme of preparing local states, based on the local tensors, and gluing it together by doing one round of entangled measurement and corrections, based on the local symmetry, is accessible in $\LAQCC$.
Note however that \citeauthor{Cirac:2021} also suggest a circuit for the $W$-state.
This circuit uses sequentially and dependent measurement-based corrections of the ancilla qubits. 
These dependent measurements translate to sequential alternations between the quantum and classical circuits and therefore increase the total depth to linear depth, exceeding the constant-depth constraints imposed by $\LAQCC$-circuits. 

We study the power of the $\LAQCC$ model with respect to state preparation, showing that even with only constant quantum-depth and logarithmic classical depth it remains possible to prepare states with long-range entanglement.
Another surprising result is that it is unlikely that $\LAQCC$ circuits are classically simulatable. We show that any instantaneous quantum polynomial-time (IQP) circuit~\cite{Bremner2010,Shepherd2009} has an $\LAQCC$ implementation.
Classical simulation of IQP circuits implies the collapse of the polynomial hierarchy to the third level, which is not believed to be true~\cite{Bremner2017}. Therefore, we expect that $\LAQCC$ circuits are unlikely to be classically simulatable. We bound the power of $\LAQCC$ by showing that it is contained in $\QNC^1$, the class of polynomial-size, log-depth circuits.

Next, we also study the power that intermediate classical calculations can add to quantum computations, by considering a new model that alternates between polynomially many polynomial-depth quantum circuits and unbounded classical computations
We study this model by doing a complexity theoretical analysis, where we draw inspiration from the notions of complexity given by \citeauthor{RosenthalYuen:2022}, \citeauthor{MetgerYuen:2023}, and \citeauthor{Aaronson:2004}.
All three complexity notions are based on the notion of state preparation, instead of more traditional definition of complexity such as the decidability of a computational problem. 
The first two consider classes based on sequences of quantum states preparable by a polynomial-sized quantum circuit, where the circuits are uniformly generated by a computational class, for instance, the class $\mathsf{PSPACE}$, which results in the complexity class $\mathsf{StatePSPACE}$~\cite{RosenthalYuen:2022,MetgerYuen:2023}.
The third notion considers a relative complexity, where the complexity is measured between two given states, and is measured by the number of gates, from a given gate-set, required to transform one state in another state~\cite{Aaronson:2004}. 
For our definition of state preparation complexity, we drop the uniformity constraint from~\cite{RosenthalYuen:2022,MetgerYuen:2023} and define a class as $\mathsf{StateX}$, which refers to states preparable by circuits of type $\mathsf{X}$. 
As an example, if $\mathsf{X} = \QNC^0$, this results in the class $\mathsf{StateQNC^0}$, which is the set of states preparable from the $\ket{0}^n$ state by poly-size constant-depth circuits. 
This notion is similar to the relative complexity from~\cite{Aaronson:2004}, where one state is the  $\ket{0}^n$ state and instead of counting the number of gates we consider the set of states preparable by a fixed number of gates. Using this notion of complexity we show that any state preparable by an $\LAQCC^*$ circuit is also preparable by a $\mathsf{PostQPoly}$ circuit, the class of circuits of polynomial depth with an additional post-selection gate. 

All Clifford circuits have a constant-depth $\LAQCC$ implementation, implying that any stabilizer state can be implemented by a constant-depth $\LAQCC$ circuit, see Section~\ref{sec:clifford_circuits} for a proof of this statement. 
Efficient circuits for stabilizer states have been known already through measurement-based quantum computing. Therefore this paper focuses on the preparation of non-stabilizer states, and as a surprising result we find novel constant-depth protocols for four very natural classes of non-stabilizer states.
Despite the extensive research into these four classes of non-stabilizer states and the many applications of them, no efficient constant- or low-depth state preparation protocols are known yet. We specifically consider these four classes as they are all often used as initial states in other algorithms.

The first state is a uniform superposition over an arbitrary number of states. 
This state finds applications in many quantum algorithms, as they often start with a uniform superposition over multiple states. 
This superposition is often achieved by applying Hadamard gates to every qubit due to its simplicity to prepare. 
Yet, the analysis of many algorithms, such as Shor's algorithm~\cite{Shor:1997}, would benefit from a different initial superposition. 
The circuit to prepare the uniform superposition over an arbitrary number of states uses an exact version of Grover search as a subroutine, that turns a probabilistic circuit, with a known constant probability of success, into a deterministic circuit. 
We use the circuit for preparing a uniform superposition over an arbitrary number of states as a subroutine in the next two quantum state preparation protocols. 

The second state is the $W$-state, the uniform superposition over all computational basis states of Hamming-weight~$1$, a natural long-ranged entangled state that displays a fundamentally nonequivalent type of entanglement from the Greenberger–Horne–Zeilinger state~\cite{WState:2000}, for which $\LAQCC$-type constant-depth circuits were previously known~\cite{PhamSvore2013, Cirac:2021}. 
The $W$-state is often used as benchmark for new quantum hardware~\cite{Haffner2005,Neeley2010,GarciaPerez:2021}. 
A novel way to prepare the $W$-state therefore gives a new way to benchmark different quantum devices with each other. 
A circuit for preparing the $W$-state was given in~\cite{Cirac:2021}, but this implementation requires sequentially alternating measurements followed by local unitaries, which in the $\LAQCC$ model is not considered to be of constant depth. 
We improve this protocol by giving an $\LAQCC$ implementation of the $W$-state, based on a compress-uncompress method that links the one-hot and binary encoding of integers.

The third state considered is the Dicke state, a generalization of the $W$-state, a superposition over all computational basis states with Hamming-weight $k$~\cite{Dicke:1954}. 
Dicke states have relevance in various practical settings.
For instance, for quantum game theory~\cite{zdemir2007}, quantum storage~\cite{Bacon_Compress:2006,Plesch:2010}, quantum error correction~\cite{ouyang2014permutation}, quantum metrology~\cite{toth2012multipartite}, and quantum networking~\cite{prevedel2009experimental}. 
Dicke states have been used as a starting state for variational optimization algorithms, most notably Quantum Alternating Operator Ansatz (QAOA)~\cite{Hadfield2019}, to find solutions to problems such as Maximum k-vertex Cover~\cite{Brandhofer2022,cook2020quantum}.
The ground states of physical Hamiltonians describing one-dimensional chains tend to show a resemblance to Dicke states such as states resulting from the Bethe ansatz, making them an ideal starting state when investigating the ground state behavior of these Hamiltonians~\cite{TDL_BetheAnsatzDerivation:2010,B_ExcitedStateQuantumPhaseTransitions:2013,DickeTransitions:2021}. 
For instance, the algorithm by \citeauthor{van2021preparing}, who give an algorithm to prepare the Bethe ansatz eigenstates of the spin-1/2 XXZ spin chain, starts by first preparing a Dicke state~\cite{van2021preparing}. 
A Dicke-state preparation protocol based on the compress-uncompress methodology used in the $W$-state furthermore finds applications in entanglement distillation, where the entanglement of a large state is concentrated on only a few qubits. 
Efficient deterministic circuits for preparing Dicke states have been proposed by \citeauthor{bartschi2019deterministic}~\cite{bartschi2019deterministic, bartschi2022deterministic_short_depth}. 
They provide a quantum circuit of depth $\mathO(k \log(\frac{n}{k}))$, allowing arbitrary connectivity, to prepare a Dicke state, which they conjecture to be optimal when $k$ is constant. 
In this work, we provide a constant-depth $\LAQCC$ circuit below their conjectured bound already for constant $k$. 
However, this does not directly disprove their conjecture, as we allow for intermediate measurements and classical computations. 
More significantly, we even construct constant-depth $\LAQCC$ circuits for $k = \mathO(\sqrt{n})$ greatly improving their bound.
This construction extends the compress-uncompress method for the $W$-state combined with additional subroutines. 

We continue with a log-depth state preparation protocol for the Dicke-state for arbitrary $k$. 
This protocol implements an efficient transformation between the factoradic number representation and the combinatorial number representation of a positive integer. 
The combinatorial number representation relates directly to the Dicke state. 
The provided efficient transformation between number representation systems might be of independent interest. 

We conclude by modifying our protocol for preparing a Dicke-state to a protocol that prepares quantum many-body scar states in constant-depth. 
These states have low entanglement and longer coherence times than states with similar energy density.
These characteristics make many-body scar states interesting to analyze and relevant within physics.
Many-body scar states appear for instance in the AKLT model~\cite{AKLT:1987,MRBAR:2018,MRB:2018} and different spin models~\cite{SI:2019,MOBFR:2020}.
Known methods for preparing these states have polynomial-depth~\cite{Gustafson:2023}, whereas our circuit has constant depth. 

% We conclude by studying the power that intermediate classical calculations can add to quantum computations. 
% In this study, we define a new model that relaxes constant-depth quantum circuits to polynomial depth quantum circuits, log-depth classical calculations to unbounded classical computations and a constant number of alternations to a polynomial number of alternations. 
% We call this model $\LAQCC^*$. 
% We study this model by doing a complexity theoretical analysis, where we draw inspiration from the notions of complexity given by \citeauthor{RosenthalYuen:2022}, \citeauthor{MetgerYuen:2023}, and \citeauthor{Aaronson:2004}.
% All three complexity notions are based on the notion of state preparation, instead of more traditional definition of complexity such as the decidability of a computational problem. 
% The first two consider classes based on sequences of quantum states preparable by a polynomial-sized quantum circuit, where the circuits are uniformly generated by a computational class, for instance, the class $\mathsf{PSPACE}$, which results in the complexity class $\mathsf{StatePSPACE}$~\cite{RosenthalYuen:2022,MetgerYuen:2023}.
% The third notion considers a relative complexity, where the complexity is measured between two given states, and is measured by the number of gates, from a given gate-set, required to transform one state in another state~\cite{Aaronson:2004}. 
% For our definition of state preparation complexity, we drop the uniformity constraint from~\cite{RosenthalYuen:2022,MetgerYuen:2023} and define a class as $\mathsf{StateX}$, which refers to states preparable by circuits of type $\mathsf{X}$. 
% As an example, if $\mathsf{X} = \QNC^0$, this results in the class $\mathsf{StateQNC^0}$, which is the set of states preparable from the $\ket{0}^n$ state by poly-size constant-depth circuits. 
% This notion is similar to the relative complexity from~\cite{Aaronson:2004}, where one state is the  $\ket{0}^n$ state and instead of counting the number of gates we consider the set of states preparable by a fixed number of gates. Using this notion of complexity we show that any state preparable by an $\LAQCC^*$ circuit is also preparable by a $\mathsf{PostQPoly}$ circuit, the class of circuits of polynomial depth with an additional post-selection gate. 

\paragraph{Summary of results}
\begin{itemize}
    \item We give a new definition of a computational model that captures the power of the four step process: applying a constant number of layers of one- and two-qubit gates; performing a syndrome measurement; perform a fast classical computation determining corrections; apply corrections. We call this model \emph{Local Alternating Quantum Classical Computations}, or $\LAQCC$ for short. In this model we bound the allowed quantum operations, intermediate classical calculations, and number of rounds separately. In Section~\ref{sec:LAQCC_model} we define this model and give a list of operations based on results from literature contained in this computational model. In some of these operations we explicitly use that we allow for multiple, but at most constant, rounds  of corrections.
    \item  We show show that there exist $\LAQCC$ circuits that can not be weakly simulated in Section~\ref{sec:IQP_in_LAQCC}. We further show that for every $\LAQCC$ circuit there exists a $\QNC^1$ circuit simulating it perfectly, in Section~\ref{sec:LAQCC_in_QNC1}.
    \item We introduce a new type computational complexity for preparing states and show that the extension of $\LAQCC$ where we allow a polynomial number of rounds and unbounded classical computation, is contained in $\mathsf{PostQPoly}$, the class of polynomial circuits with post-selection, in Section~\ref{sec:Complexity results}.
    \item We show a protocol to prepare the uniform superposition state of size $q$ in $\LAQCC$ using $\mathO(\ceil{\log_2(q)}^2)$ qubits in Section~\ref{sec:superposition_modulo_q}. 
    \item We show a protocol to prepare the $W_n$ state in $\LAQCC$ using $\mathO(n\log(n))$ qubits in Section~\ref{sec:W_state_in_LAQCC}.
    \item We show two ways of preparing the Dicke-$(n,k)$ state. The first method is in $\LAQCC$, works up to $k = \mathO(\sqrt{n})$, uses $\mathO(n^2\log(n))$ qubits, and is found in Section~\ref{sec:dicke:small_k}. The second method is in $\LAQCC\text{-}\mathsf{LOG}$ (an extension of $\LAQCC$ allowing for logarithmic number of alterations instead of constant), works for any $k$, uses $\mathO(\text{poly}(n))$ qubits, and is found in Section~\ref{sec:Dicke_in_LAQCC_LOG}. 
    \item We extend on our $\LAQCC$ method of generating Dicke-$(n,k)$ states for $k = \mathO(\sqrt{n})$ and show a protocol to generate many-body scar states for a particular Hamiltonian in $\LAQCC$ (Section~\ref{sec:many_body_scar}). 
\end{itemize}
Summarized in a table, we provide the following state generation protocols:
\begin{table}[htb]
\centering
\begin{tabular}{l|l|l|l}
\textbf{State description} & \textbf{Width} & \textbf{Depth} & \textbf{Implementation}\\
\hline 
Uniform superposition mod $q$: $\frac{1}{\sqrt{q}} \sum_{i = 0}^{q-1}\ket{i}$ & $\mathO(\ceil{\log^2 q})$ & $\mathO(1)$ & Section~\ref{sec:superposition_modulo_q}\\

$W$-state: $\frac{1}{\sqrt{n}}\sum_{i = 0}^{n-1}\ket{e_i}$ & $\mathO(n \log n)$ & $\mathO(1)$ & Section~\ref{sec:W_state_in_LAQCC}\\

Dicke-$(n,k)$, $k = \mathO(\sqrt{n})$: $\binom{n}{k}^{-1/2}\sum_{x \in \{0,1\}^n: |x| = k} \ket{x}$ &  $\mathO(n^2\log n)$ & $\mathO(1)$ 
&Section~\ref{sec:dicke:small_k}\\

Dicke-$(n,k)$: $\binom{n}{k}^{-1/2}\sum_{x \in \{0,1\}^n: |x| = k} \ket{x}$ & $\mathO(\text{poly}(n))$ & $\mathO(\log n)$ &Section~\ref{sec:Dicke_in_LAQCC_LOG}\\

QMBS: $\ket{S_k} = \frac{1}{k! \sqrt{\mathcal N(n,k)}}(Q^\dagger)^k \ket{\Omega}$ &  $\mathO(n^2\log n)$ & $\mathO(1)$  &  Section~\ref{sec:many_body_scar}
\end{tabular}
\caption{Summary of state preparation protocols given in this paper.}
\label{tab:sate_prep}
\end{table}
In the entry for the quantum many-body scar state $Q$ denotes the raising operator and $\mathcal N(n,k)=\binom{n-k-1}{k}$. 
Section~\ref{sec:many_body_scar} will provide more details on the variables and the implementation. 

\paragraph{Organization of the paper}
\noindent We first introduce relevant preliminaries in Section~\ref{sec:preliminaries}. 
In Section~\ref{sec:LAQCC_model} we formally define the class of Local Alternating Quantum-Classical Computations ($\LAQCC$). We also show that any Clifford circuit can be implemented in constant depth $\LAQCC$ (a result based on a result from measurement-based quantum computing~\cite{jozsa2006introduction}). 
This result allows us to give many useful multi-qubit gates and routines in Section~\ref{sec:gates_created_in_LAQCC}. 
Beyond that we show that constant depth $\LAQCC$ circuits are contained in $\QNC^1$ and that any $\mathsf{IQP}$ circuit has an $\LAQCC$ implementation.
We conclude this section with an analysis of a more powerful instantiation of $\LAQCC$ and show an inclusion with respect to the class $\mathsf{PostQPoly}$, which is the class of circuits of polynomial depth with one additional post-selection gate. 
In Section~\ref{sec:state_prep_in_LAQCC} we give $\LAQCC$ circuit implementations for preparing the uniform superposition over an arbitrary number of states, the $W$-state and the Dicke state up to $k = \mathO(\sqrt{n})$. We furthermore give a log-depth circuit implementation for preparing the Dicke state for any $k$. We conclude by showing a $\LAQCC$ circuit for generating many body scar states of a particular type of Hamiltonian.


\section{Related Work}
%\subsection{Cost Volume based Deep Stereo Matching}
%Stereo matching is a typical problem that has been studied for decades and a well-known four-step pipeline \cite{scharstein2002taxonomy} has been established, where cost volume construction is an indispensable step. Current state-of-the-art stereo matching methods are all cost volume based methods and they can be categorized into two types. Typically, a cost volume is a 4D tensor of height, width, disparity, and features. The first category just uses a full correlation to generate a single-feature cost volume. Such methods are usually efficient but lose much information because of the decimation of feature channels. Many previous work, including Dispnet \cite{dispnet}, MADNet \cite{madnet}, IResNet \cite{iresnet} and AANet \cite{aanet}, belong to this category. The second category usually uses concatenation \cite{gcnet} or group-wise correlation \cite{gwcnet} to generate a multi-feature 4D cost volume. Such a method can achieve better performance while requiring higher computational complexity and memory consumption. Actually, a majority of the top-performing networks in public leaderboards belong to this category, such as GANet \cite{ganet}, CSPN \cite{cspn} and ACFNet \cite{acfnet}. These methods generally employ multiple 3D convolution layers to constantly regularize the 4D cost volume and then apply softmax over the disparity dimension to produce a discrete disparity probability distribution. The final predicted disparity is obtained by softly weighting indices according to their probability, which is also called soft argmin in GCNet \cite{gcnet}. However, soft argmin leaves the output susceptible to multi-modal disparity probability distributions. ACFNet \cite{acfnet} observes this problem and proposes to directly supervise the cost volume with unimodal ground truth distributions. In contrast, we define an uncertainty estimation to quantify the degree to which the cost volume tends to be multi-modal distribution, higher implies the higher possibility of estimation error.

\subsection{Multi-scale Cost Volume based Stereo Matching}
Cost volume construction is an indispensable step in the well-known four-step pipeline for stereo matching \cite{scharstein2002taxonomy, pamisurvey1, pamisurvey2}. Typically, current state-of-the-art stereo matching methods can be categorized into two types of cost volume-based methods, where the cost volume is a 4D tensor of height, width, disparity, and features. The first category usually uses the single-feature 3D cost volume generated by full correlation, which is efficient while losing much information due to the decimation of feature channels. Many real-time methods, such as Dispnet \cite{dispnet}, MADNet \cite{madnet, madnet_pami} and AANet \cite{aanet}, belongs to the category. Moreover, two-stage refinement \cite{mcvmfc} and pyramidal towers \cite{madnet} are commonly applied in the single-feature cost volume based network to construct multi-scale cost volume. The second category usually uses the multi-feature 4D cost volume generated by concatenation \cite{gcnet} or group-wise correlation \cite{gwcnet}, which can achieve better performance with higher computational complexity and memory consumption. Most top-performing networks, including GANet \cite{ganet}, CSPN \cite{cspn} and ACFNet \cite{acfnet} belong to this category. 
% In these methods, the 4D cost volume is constantly regularized by multiple 3D convolution layers and then a discrete disparity probability distribution can be produced by softmax. Next, the final predicted disparity can be obtained by softly weighting indices according to their probability \cite{gcnet}. However, such output is susceptible to multimodal disparity probability distributions and ACFNet \cite{acfnet} gives a solution by directly supervising the cost volume with unimodal ground truth distributions to alleviate this problem. 
Recently, to alleviate the high computational complexity and memory consumption when employing multi-feature 4D cost volumes, \cite{cvpmvsnet, cascade, uscnet} propose to use cascade cost volume representation in multi-view stereo. These methods usually first predict an initial disparity at the coarsest resolution of the image and then gradually refine the disparity by narrowing down the disparity search space. More closely related to our approach is Casstereo \cite{cascade}, which first extended such representation to stereo matching. It selected to uniform sample a pre-defined range to generate the next stage’s disparity search range. Instead, we employ pixel-level uncertainty estimation to adaptively adjust the next stage disparity searching range and generate pseudo-labels for subsequent domain adaptation. Our method also shares similarities with UCSNet \cite{uscnet}, which constructs uncertainty-aware cost volume in multi-view stereo while it doesn’t employ uncertainty estimation to generate pseudo-labels.

%\subsection{Multi-scale Cost Volume based Deep Stereo Matching} 
% \subsection{Multi-scale Cost Volume based Stereo Matching} 
%Multi-scale cost volume firstly was applied in the single-feature cost volume based network with the form of two-stage refinement \cite{mcvmfc} and pyramidal towers \cite{madnet}. Recently, cascade cost volume representation \cite{cvpmvsnet, cascade, uscnet} was proposed in multi-view stereo to alleviate the high computational complexity and memory consumption when employing multi-feature 4D cost volumes. These methods generally predict an initial disparity at the coarsest resolution of the image. Then, they will narrow down the disparity search space and gradually refine the disparity. More closely related to our approach is Casstereo \cite{cascade}, which first extended such representation to stereo matching. It selected to uniform sample a pre-defined range to generate the next stage’s disparity search range. Instead, we employ uncertainty estimation to adaptively adjust the next stage pixel-level disparity searching range and push the next stage's cost volume to be predominantly unimodal.

% The single-feature cost volume based network with the form of two-stage refinement \cite{mcvmfc} and pyramidal towers \cite{madnet} first employ multi-scale cost volume for stereo matching. Recently, to alleviate the high computational complexity and memory consumption when employing multi-feature 4D cost volumes, \cite{cvpmvsnet, cascade, uscnet} propose to use cascade cost volume representation in multi-view stereo, which generally predict an initial disparity at the coarsest resolution of the image. Then, the disparity search space is narrowed down and the disparity is gradually refined. More closely related to our approach is Casstereo \cite{cascade}, which first extended such representation to stereo matching. It selected to uniform sample a pre-defined range to generate the next stage’s disparity search range. Instead, we employ uncertainty estimation to adaptively adjust the next stage pixel-level disparity searching range and push the next stage's cost volume to be predominantly unimodal.

% Figure environment removed

\subsection{Robust Stereo Matching} 
There exist three categories of generalization definitions for robust stereo matching. 1) Cross-domain Generalization: the network’s ability to perform well on unseen scenes (cannot see the image pairs of the target domain in advance). Towards this end, Jia et al \cite{sungeneralizaiton} propose to incorporate scene geometry priors into an end-to-end network. Zhang et al \cite{dsmnet} introduce a domain normalization and a trainable non-local graph-based filter to construct a domain-invariant stereo matching network. 2) Adapt Generalization: the network’s ability to adapt pre-trained models to the new domain with unlabeled target data. Previous work usually pre-trains the models on synthetic data and then adapts it to new target domains with Graph Laplacian regularization \cite{zoom}, non-adversarial progressive color transfer \cite{adastereo}, and Knowledge Reverse Distillation \cite{aohnet}. More closely related to our approach are \cite{aohnet, unsuperviseddomainadaptation} in stereo matching and Monoresmatch \cite{monoresmatch} in monocular depth estimation, which also proposes to generate a pseudo-label for domain adaptation. However, these methods all select to employ classical stereo matching methods \cite{sgm} alongside with confidence estimators, e.g., left-right consistency check to generate pseudo-labels. That is all these methods need an independent method to generate corresponding pseudo-labels. Instead, the proposed method is an end-to-end network that can generate the predicted disparity map, corresponding uncertainty map and pseudo-labels jointly, which is a more simple, yet efficient way. 
% Instead, our proposed method can employ pixel-level and area-level uncertainty estimation to self-distill the predicted disparity maps of our pre-training model and generate sparse while reliable pseudo-labels to align the domain gap, which is a more simple, yet efficient way. 
3) Joint Generalization: the network’s ability to perform well on a variety of datasets with the same model parameters. MCV-MFC \cite{mcvmfc} introduces a two-stage finetuning scheme to achieve a good trade-off between generalization and fitting capability on multiple datasets. However, it doesn’t touch the inner difference between diverse datasets, e.g, the unbalanced disparity distribution. To further address this problem, we propose a cascade cost volume to adaptively the next stage disparity searching space, where the pixel-level uncertainty estimation is at the core.

% \subsection{Monocular Depth Estimation}
% Monocular depth estimation aims to estimate depth values from a single image, instead of stereo images or multiple frames in a video. This problem is ill-posed because of the ambiguity of object sizes. However, humans could estimate the depth from a single image with prior knowledge of the scenes. Recently, learning based methods were explored to learn depth values by supervised or unsupervised learning. Eigen et al. first employed Convolutional Neural Networks (CNN) to predict depth in a coarse-to-fine manner and further improved its performance by multi-task learning. Liu et al. presented deep convolutional neural fields model by combining deep model with continuous CRF. Li et al. [22] refined deep CNN outputs with a hierarchical CRF. Multi-scale continuous CRF was formulated into a deep sequential network by Xu et al. [45] to refine depth estimation. Unsupervised methods tried to train monocular depth estimation with stereo
% image pairs or image sequences and test on single images. Garg et al. [9] used novel image view synthesis loss to train a depth estimation network in an unsupervised way. Godard et al. [11] introduced left-right consistency regularization to improve the performance of view synthesis loss. Recently, some work also propose to use the stereo matching network as a proxy to learn depth from synthetic data or directly employ traditional stereo matching methods to distill proxies labels from the target domain, which proves the feasibility of distilling stereo matching networks to learn monocular depth estimation.



\section{Design Requirements}
\label{sec.design_requirements}
% background, (goal and target users)
We aim to develop a visual analytics system to help NLP experts understand and diagnose commonsense reasoning capabilities of NLP models in a systematic and scalable manner.
% our tool benefits
Explaining such model abilities helps users determine whether models are suitable and trustworthy for downstream applications and enhance specific knowledge that models do not learn well.
However, it is challenging to depict and summarize the vast and complex space of commonsense knowledge that models learn, as it is not directly presented in the input, and concepts are entangled with various relations and contexts.
% recognize what types of concepts and relations models learn well or poorly and whether they rely on some superficial word correlations for reasoning.
% \vis{After that, our users can determine whether models are suitable and trustworthy for downstream applications and enhance specific knowledge that models do not learn well.}
%%%%%%%%%%%%%%%%%%%%%%%%% OLD %%%%%%%%%%%%%%%%%%%%%%%%%
% Furthermore, they can enhance models by encoding required commonsense knowledge from knowledge bases as additional inputs~\cite{gkp, wang2020connecting, yao2022nlp, dalvi2022towards} or into the model parameters~\cite{de2021editing, meng2022locating}.
%%%%%%%%%%%%%%%%%%%%%%%%% OLD %%%%%%%%%%%%%%%%%%%%%%%%%
% challenges - our tool improvement over prior work (consider commonsense knowledge which may not explicitly be specified in the statement)
% However, it is challenging to interpret what commonsense models know, since it is not directly mentioned in statements and does not rely much on linguistic contexts for inference tasks.
% \rev{However, it is challenging to reveal and summarize what commonsense models know since it is not directly mentioned in the input. Moreover, the space of commonsense knowledge is vast and complex, where concepts are entangled with different relations and contexts.}

% how we formulate the design requirements
% To develop a concrete understanding of commonsense reasoning and methods of explaining NLP models, 
We first conducted a literature review on explainability techniques \renfei{\cite{ribeiro2016should, lundberg2017unified, kaushik2019learning, wu2021polyjuice}} and visual analytics~\renfei{\cite{sharedinterest, whatiftool, feng2023xnli,jin2023shortcutlens}} for NLP, and commonsense reasoning~\renfei{\cite{conceptnet, cui2020commonsense}}. 
To further characterize users' common practices and needs, we collaborated with three NLP experts (\imp{E1-E3}, \imp{E1} is the coauthor) through regular weekly meetings for about six months.
\imp{E1} is a Ph.D. candidate who investigates commonsense knowledge acquisition and reasoning. 
\imp{E2} has obtained a Ph.D. degree in HCI and has rich experience in building human-centered interactive NLP models.
And \imp{E3} is a research scientist from an international media company whose expertise is in explainable AI and visualization for NLP. 
% All of them have several publications in their research fields.
%% How we collaborate
During the meetings, we asked them about 
% \rev{1) current practice of evaluating NLP models;}
1) the general methods of NLP model evaluation; 
2) what types of explanations for models' commonsense reasoning capabilities;
and 3) the desired system task support.
% functions for systematic and scalable model analysis for commonsense reasoning.
Meanwhile, we developed our system prototypes iteratively and collected their feedback for further improvement.

% general methods of evaluating models and what do they care about during evaluation

\rev{\textbf{Current practice and limitations.}}
Our users usually start with performance metrics (\eg, accuracy) to locate data instances (\esp, wrong predictions) and manually summarize what commonsense knowledge is needed for inference. 
Specifically, users identify important relations and concepts for commonsense reasoning and combine performance metrics with feature attribution methods to determine whether models capture important concepts or superficial word associations. Moreover, they can probe the models by modifying the data instances to verify their hypotheses.
However, this analysis process is tedious, mentally demanding, and difficult to generalize to larger data subsets. They desire a visual analytics tool to analyze what commonsense knowledge is contained in data instances and (not) learned by NLP
models.
% To support the systematic and scalable exploration of commonsense knowledge, a visual analytics tool is necessary.

% To understand models' commonsense reasoning abilities, our users usually start with performance metrics (\eg, accuracy).
% Then, they sample some data instances, especially those with wrong predictions, for further examination.
% For each instance, they manually summarize what commonsense knowledge is needed for inference. 
% \rev{Specifically, our users often build the mental model about \textit{relations} between mentioned \textit{concepts} in data instances. And they identify important relations and concepts for commonsense reasoning.}
% Afterward, they combine performance metrics with feature attribution methods (\eg, SHAP) to see whether models capture those important concepts or superficial word associations for commonsense reasoning. Furthermore, they generate hypotheses about models' reasoning over relations between concepts.
% And they probe the models by modifying the data instances to verify their hypotheses.

%% Limitations
% However, the analysis method above focuses on individual instances. It is tedious and mentally demanding to go through all of them, identify the important commonsense knowledge in the instances, and reason about model behavior on concepts and relations.
% Besides, it is hard to generalize their findings to larger data subsets, 
% \rev{for example, how a LM performs on the instances containing different groups of concepts connected by the same latent relations.}
% They value our effort in building such a visual analytics tool to support the systematic and scalable exploration of what commonsense knowledge \rev{(\esp, concepts and relations) is contained in data instances and (not) learned by NLP models.}
% % summarize design requirements
% % Finally, we summarize our derived design requirements as follows.
% Our derived design requirements are as follows.

% Expert background 
%  - expert one: PhD candidate whose research interests are commonsense knowledge acquisition and reasoning.
%  - expert two: a research scientist from a media company whose expertise is in explainable AI and visualization for NLP
%  - expert three: a graduated PhD who builds human-centered interactive NLP models

\textbf{R1. Reveal commonsense knowledge in data instances.}
Our users need to distill the external commonsense knowledge from data instances, which helps verify if model behavior aligns well with human knowledge~\cite{sharedinterest,jin2023shortcutlens}. Since concepts and relations are critical components of commonsense knowledge~\cite{conceptnet, cui2020commonsense}, the system should extract relevant concepts and their relations in questions as references to understand data itself and model behavior:

\begin{compactdesc}
\setlength\itemindent{-1em}
  \item[Q1:] What concepts (\eg, entities) are mentioned in the instances?
  \item[Q2:] What are the latent relations between the mentioned concepts?
\end{compactdesc}

% \hspace{\parindent}\textit{Q1. What concepts (\eg, entities) are mentioned in the instances?}
% \hspace{\parindent}\textit{Q2. What are the latent relations between the mentioned concepts?}

\textbf{R2. Summarize model performance on different concepts and relations.}
% Performance metrics are crucial quantitative indicators of model evaluation, which help users gain a concrete understanding of model behavior.
Our users usually depend on accuracy scores to pinpoint cases where models fail and prioritize exploring them.
% summarization importance
To scale up the analysis of individual instances to large datasets, it is necessary to summarize model performance from multiple aspects~\cite{whatiftool, sharedinterest, feng2023xnli}.

% In commonsense reasoning, concepts, relations, and contexts are essential for model understanding and evaluation. 
\imp{E3} said that a concept-driven summary can reveal what topics models perform well. \imp{E1} mentioned that compared to the vast concept space, relations have more summative power and connect concepts meaningfully. \imp{E3} suggested relating model performance to linguistic contexts to assess their ability to use commonsense knowledge in different situations. For instance, testing models on instances where adults and children use staplers helps understand whether models can distinguish between them.
Therefore, the system should answer:
% Therefore, the system should help answer the following questions:

% In commonsense reasoning, concepts, relations, and their contexts are three essential for model understanding and evaluation.
% \imp{E3} said that a concept-driven summary of model performance can help reveal what topics (\eg, sports, music) models perform well.
% And \imp{E1} mentioned that compared to enormous concept space, relations are more compact and have more summative power that connects concepts in a meaningful way.
% Besides, \imp{E3} added that relating the model performance to linguistic contexts 
% provides evidence of whether models can correctly utilize commonsense knowledge in different situations.
% % how models capture important cues to find latent relations between concepts in different situations.
% For example, both adults and children are capable of using staplers. However, adults usually associate with office and children are likely to use staplers in school.
% And testing models on the cases aid the understanding of whether models can distinguish between adults and children.
% To summarize, the system should help users answer the following questions about model performance:

\begin{compactdesc}
\setlength\itemindent{-1em}
  \item[Q3:] What concepts, relations, and their combinations are predicted right or wrong by the models?
  \item[Q4:] What are the contexts of the relations and concepts? What is the model performance?
\end{compactdesc}

% \hspace{\parindent}\textit{Q3. What concepts, relations, and their combinations are predicted right or wrong by the models?}

% \hspace{\parindent}\textit{Q4. What are the contexts of the relations and concepts? What is the model performance?}

\textbf{R3. Infer model relational reasoning over concepts based on relevant commonsense knowledge.}
% Besides model performance, our users usually need to identify important input features (\eg, words, phrases) for model predictions and then judge the rationality of the model reasoning by aligning these features with their prior knowledge~\cite{ribeiro2016should, lundberg2017unified, sharedinterest}.
To develop a mental model about models' commonsense knowledge and reasoning, users need to first use their own prior knowledge to build the relevant reasoning paths that connect important words in statements. Then, they need to check whether models capture these meaningful concepts in statements based on their importance to model predictions. Although sometimes models are correct, they may rely on task-unrelated linguistic features (\eg, stop words) to make decisions. 
Moreover, \imp{E2} and \imp{E3} thought that to better surface the patterns of how models regard unmentioned relations, it is necessary to show whether models attach importance to the mentioned words in statements connected by those relations.
By concept-driven comparison between important concepts recognized by models and humans, users can generate hypotheses about models' relational reasoning over concepts:
% since some relations between concepts are not explicit in statements, it is necessary to reveal whether 
% Moreover, to develop a mental model about models' commonsense, users need to 
% To generate hypothesis about models' commonsense reasoning abilities, our users are interested in 

\begin{compactdesc}
\setlength\itemindent{-1em}
  \item[Q5:] What concepts are important for model predictions? Are they reasonable?
  \item[Q6:] What unexpressed relations are necessary for inference? Do models cover the concepts connected by these relations?
  \item[Q7:] What are the differences between the important concepts for commonsense reasoning and for model predictions?
\end{compactdesc}


% \hspace{\parindent}\textit{Q5. What concepts are important for model predictions? Are they reasonable?}

% \hspace{\parindent}\textit{Q6. What unexpressed relations are necessary for inference? Do models cover the concepts connected by these relations?}

% \hspace{\parindent}\textit{Q7. What are the differences between the important concepts for commonsense reasoning and for model predictions?}


\textbf{R4. Allow interactive probing and editing of NLP models.}
One straightforward and useful way to understand and debug models is interactively interrogating them~\cite{whatiftool, kaushik2019learning, wu2021polyjuice}. To generate and verify the what-if hypothesis about model behavior, users can conduct counterfactual analysis by manipulating specific input components and seeing how models react to these changes.
% Through interactions with models, users can inject their own knowledge and probe models based on their interests.
% Specifically, what-if analysis 
This helps disentangle influences of individual concepts in statements for model predictions and check whether models are biased towards some concepts. Moreover, modifying the input components can test the robustness of models against noisy concepts and probe the underlying relations of interest that link the mentioned concepts in the input.
% - manipulate and modify data instances based on human knowledge and model behavior to probe relations and concepts
% - conduct counterfactual analysis through interactions with models
\vis{After model probing, users may desire to conduct posthoc editing of model behavior to inject their desired knowledge and make a flexible localized update about specific knowledge areas that models do not learn well~\cite{de2021editing,mitchell2022fast}.}

\revvv{Given the requirements, we consolidated a series of system tasks that guides the systematic exploration of models' commonsense reasoning capabilities: Initially, commonsense knowledge from data is extracted as concept-relation triplets (\imp{R1}). 
Then, the system summarizes model performance across these concepts and relations
% Then, the system summarizes model performance on different relations and concepts 
(\imp{R2}), and further assesses the overall relation learning (\imp{R3}). 
Next, it enables users to pinpoint error instances related to specific concepts or relations (\imp{R2}). 
For these instances, the system summarizes the concepts considered important by models in varying contexts, aligning them with those in the extracted triplets (\imp{R3}). 
Moreover, visualization is utilized in conjunction with model probing (\imp{R4}) to comprehensively explain models' input-output behavior on these instances (\imp{R3}).
% Furthermore, visualizations of models' input-output behavior on these instances (\imp{R3}) and probe models (\imp{R4}) for a comprehensive model understanding.
Finally, users can bookmark instances for targeted model refinement (\imp{R4}).
}


\section{System \& Methods}
\label{sec.system}
\begin{comment}
% system introduction
Based on the design requirements, we design and implement \name{}, 
\rev{a visual analytics system that can contextualize and visualize the commonsense reasoning capabilities of NLP models in a systematic and scalable manner.
% Particularly, \name{} adopts an external commonsense knowledge base to derive and summarize the implicit commonsense knowledge in input data. 
% Then, based on the extracted knowledge, the system contextualizes and visualizes the model behaviors on different concepts and their implicit relations to guide multi-faceted and multi-level exploration of model.
% aligns the model behavior with human commonsense knowledge
Particularly, \name{} leverages an external \textbf{knowledge graph} to 1) derive and summarize commonsense knowledge in input data with concepts and their relations, and  2) facilitate contextualized multi-level exploration and diagnosis of model behaviors on different concepts and their implicit relations.}
% based on feature attribution methods and external commonsense knowledge bases.

Question answering (QA) is a common way to evaluate model understanding and reasoning over natural language~\cite{sap2020commonsense}. And most commonsense reasoning benchmark datasets adopt the QA form~\cite{socialiqa, bisk2020piqa, sakaguchi2021winogrande, 2020unifiedqa, CSQA1, CSQA2}.
% , which is designed so that the answers require commonsense knowledge and reasoning and cannot be directly derived from the question contexts. 
Therefore, \rev{without loss of generality, in this paper, we target commonsense QA tasks to showcase our system.}
% , in this paper, we target commonsense QA and choose one representative benchmark, CSQA~\cite{CSQA1}, to showcase our system.

In this section, we describe the system framework, input data and model, methods of extracting commonsense knowledge and  contextualizing model behaviors, and the interface designs of \name{}.
\end{comment}

\vis{
% Motivated by the design requirements, 
% We design \name{}, a visual analytics system that enables scalable and systematic analysis of NLP models' commonsense reasoning capabilities. 
Our system, \name{}, leverages an external knowledge graph to summarize and derive commonsense knowledge and facilitate multi-level exploration and diagnosis of model behaviors in commonsense question-answering tasks. We focus on question-answering tasks because they are a common evaluation method for natural language understanding, and most commonsense reasoning benchmarks adopt the QA format~\cite{socialiqa, bisk2020piqa, sakaguchi2021winogrande, 2020unifiedqa, CSQA1, CSQA2}.}
% We describe our system framework, input data and model, methods for extracting commonsense knowledge and contextualizing model behaviors, and the interface design of \name{}.}

% \vis{Based on the design requirements, we design and implement \name{}, a visual analytics system that enables systematic and scalable analysis of NLP models' commonsense reasoning capabilities. Our system leverages an external knowledge graph to 1) derive and summarize commonsense knowledge, and 2) facilitate contextualized multi-level exploration and diagnosis of model behaviors. Specifically, we focus on commonsense question-answering (QA) tasks to showcase our system's capabilities, given that QA is a common evaluation method for natural language understanding~\cite{sap2020commonsense} and most commonsense reasoning benchmarks adopt the QA form~\cite{socialiqa, bisk2020piqa, sakaguchi2021winogrande, 2020unifiedqa, CSQA1, CSQA2}. In this section, we describe our system framework, input data and model, methods for extracting commonsense knowledge and contextualizing model behaviors, and the interface design of CommonsenseVIS.}

% In this section, we first describe the system framework (\autoref{subsec.system_framework}). 
% \rev{Then, we describe the system input data and model (\autoref{subsec.system_data}), followed by the methods of extracting commonsense knowledge from data (\autoref{subsec.extract_commonsense}) and contextualizing model behaviors (\autoref{subsec.align_model}). Finally, we introduce the interface designs of \name{} (\autoref{sec.user_interface}).}

% Then, we describe the methods of model behavior contextualization. After that, we introduce the visualization and interaction designs. 

% % Figure environment removed 

% Figure environment removed 

\revv{\subsection{System Overview}}
\label{subsec.system_framework}
% \vis{\autoref{fig:system_framework}} summarizes the system framework. 
\revv{\autoref{fig:system_framework}} provides an overview of our system. \name{} takes in QA instances and an NLP model to compute model answers. Then, it identifies important concepts (\ie, words) in questions using feature attribution methods and extracts relevant commonsense knowledge from input data using an external knowledge base. This knowledge helps align the model behavior with \cpn{}. The user interface enables multi-level exploration, interactive probing, and editing.
% This knowledge serves as a reference for aligning model behavior with ConceptNet knowledge. The user interface facilitates multi-level exploration of model behavior, interactive model probing and editing.

%%%%%%%%%%%%%%%%%%%%%%%%%%%%%%%%%% OLD %%%%%%%%%%%%%%%%%%%%%%%%%%%%%%%%%%%%%%%
% \autoref{fig:system_framework} summarizes the system framework. Given the QA instances and an NLP model, the system computes the model answers to the questions.
% % (in \autoref{fig:system_framework}A). 
% Then, the system identifies the concepts (\ie, words) in the questions that are important to the model decisions based on feature attribution methods. Meanwhile, the system extracts relevant commonsense knowledge contained in input data  (represented as graphs with words in questions and answers being nodes and their relations being links) based on a large external commonsense knowledge base.
% % ---ConceptNet~\cite{conceptnet}. 
% This knowledge is set as contextual references for 
% % the model behavior.
% % Then, the system generates model explanations by 
% aligning the model behavior with ConceptNet knowledge about different concepts and relations.
% % with the extracted knowledge regarding the input question concepts and their underlying relations.
% % (in \autoref{fig:system_framework}C).
% % as well as the concepts in the questions that are important to the model decisions based on feature attribution methods (\ie, SHAP in our case).
% % Meanwhile, the system extracts relevant commonsense knowledge contained in data instances (represented as graphs with question concepts being nodes and their relations being links) based on a large external commonsense knowledge base---ConceptNet~\cite{conceptnet}.
% % This knowledge is set as contextual references for the model behavior.
% % Then, the system generated model explanations by aligning the model behavior with the extracted knowledge regarding the question concepts and their underlying relations between each other.
% Finally, the user interface 
% % (\autoref{fig:teaser}) 
% guides the multi-level and multi-faceted exploration of the model explanations and enables interactive model probing and editing. 
% % probing of models.

%%%%%%%%%%%%%%%%%%%%%%%%%%%%%%%%%% OLD %%%%%%%%%%%%%%%%%%%%%%%%%%%%%%%%%%%%%%%

\subsection{System Data \& Model}\label{subsec.system_data}
\rev{Here, we introduce the system input, including the QA data, model, and external knowledge base for contextualizing model behavior.
% Before diving into the contextualization and visualization of the model's commonsense reasoning capabilities, we describe the system input, including the QA data, model, and external commonsense knowledge base.

\textbf{QA data.} Each QA instance contains a \emph{question concept}, a \emph{target concept (\ie, answer)}, \emph{alternative answers} (if any), and a \emph{question stem}.
Following the previous commonsense QA benchmarks~\cite{CSQA1,CSQA2,atomic2019}, \revv{\textbf{concepts} are defined as words, and question stems provides \textbf{contexts} for the commonsense relations between the question and target concepts.}
\revv{As shown in \autoref{fig:system_framework}A, the question concept is air conditioning, the target concept is house, and air conditioning is located at the house. And the context in the question stem is: ``A man...watches the game on Saturday...''.}
% the question concept (QC)/prompt, the target concept (TC)/answer, alternative answers (if any), and the question stem (QS). 
% The QS probes the commonsense relations between QC and TC.
% The QS suggests the context and commonsense relation between QC and TC, while the alternative choices are set as distractors.
% \\\textbf{clarify the relations between QC, TC, and QS.}\\
% This data format is adopted by many commonsense QA benchmarks~\cite{CSQA1,CSQA2,atomic2019}.
% And if a question concept is not presented, 
If a question concept is absent,
knowledge graph embedding methods~\cite{bordes2013translating} can be used to determine the relation strength between the target concept and words in the question stem. The word with the highest score becomes the question concept~\cite{lin2019kagnet}.
% If a question concept is absent,
% users can apply knowledge graph embedding methods~\cite{bordes2013translating} to score the relation strength between the target concept and words in the question stem and find the one with the highest score as the question concept~\cite{lin2019kagnet}.
% and the concepts in the question stem and find the one with the highest score as the QC~\cite{lin2019kagnet}. 
% Then, users can prepare the desired QA input structure.

% For the demonstration purpose, we choose one representative commonsense QA benchmark---CSQA~\cite{CSQA1}.
We utilize one representative commonsense QA benchmark, CSQA~\cite{CSQA1}, for demonstration.
The dataset has 12,102 multiple-choice questions, covering diverse topics and various forms of commonsense.
Each human-authored question contextualizes relations between a question concept and a target concept (\ie, the correct answer among five candidates). 
Triplets of these concepts and relations are drawn from \cpn{}~\cite{conceptnet}.
% incorporating a wide range of commonsense, such as spatial, social, and casual. 
% Each question has one correct answer and four distractors and is authored by humans to reflect the context and commonsense relation between a question concept and a target concept. The triplets of question concepts, target concepts and their relation are drawn from \cpn{}~\cite{conceptnet}.
The most frequent question concepts are about people, water, and animals, probing various relations such as spatial (41\%) and causal (23\%).
Questions are formulated in diverse forms (\eg, wh-questions, statements, and hypotheses) with 13 words on average.

\textbf{QA model.}
Our system is designed to accommodate various NLP models that select answers to given questions, as it focuses on the input-output model behavior and we can adopt model-agnostic feature attribution methods to quantify this behavior.

\revv{For the purpose of system demonstration, we have chosen UnifiedQA\footnote{\small \url{https://huggingface.co/allenai/unifiedqa-v2-t5-large-1363200}}~\cite{2020unifiedqa, khashabi2022unifiedqa} as an example for model analysis. It is an open-source, general QA model that has been pre-trained across various QA datasets, showing great generalization capabilities.
We use SHAP to compute the importance scores of model inputs because of its strong theoretical foundation and widespread adoption in various domains.}

% The system can accept various NLP models that select answers to the given questions, as it focuses on the input-output behavior of the models and can adopt model-agnostic feature attribution methods (e.g., SHAP, LIME) to quantify such behavior. In this paper, for the system demonstration, 
% we use UnifiedQA~\cite{2020unifiedqa, khashabi2022unifiedqa}--- an open-sourced general QA model pretrained across different QA datasets---as an example for model analysis. 
% \revv{We choose SHAP to compute the importance scores of model inputs because of its widespread recognition and adoption across various domain applications.}


% Given the QA data, the system can accept various QA NLP models since the system focuses on the input-output behavior of the models and can adopt model-agnostic feature attribution methods (\eg, SHAP, LIME) to quantify such behavior.

% In this paper, we use UnifiedQA~\cite{2020unifiedqa}, a state-of-the-art QA model, as an example for model analysis. 
% Similar to other modern natural language models, it is built upon a large base model~\cite{t5model} and trained across different QA datasets. 
% Given a question with multiple choices from CSQA, the model selects one as the answer.
\textbf{Commonsense knowledge base.}
We utilize an external knowledge base to capture the commonsense knowledge in the QA data, which provides context for inferring the model's implicit reasoning.  To ensure meaningful and helpful context for model analysis, the knowledge base must \textit{sufficiently cover relevant commonsense }reflected in the QA data. 

We adopt \cpn{}~\cite{conceptnet} as an external resource, a large-scale commonsense knowledge graph connecting \textbf{concepts} (\ie, words) with \textbf{relations}.
% (\eg, \hl{be capable of} links words \hl{people} to \hl{driving}). 
The graph integrates diverse knowledge sources with over 8 million nodes and over 21 million links.
Particularly, it uses 36 general relations (\eg, \hl{IsA}, \hl{UsedFor}) to connect words, mostly covering taxonomic, lexical knowledge, and physical commonsense knowledge.
% (e.g., \hl{IsA})
% (\eg, \hl{Synonym})
% (\eg, \hl{MadeOf}, \hl{PartOf}).
ConceptNet is widely used to enhance NLP models with commonsense capabilities~\cite{lin2019kagnet, feng2020scalable} and build reasoning benchmarks~\cite{CSQA1, CSQA2, lin2021riddlesense}. 
For example, \revv{the questions and answers in CSQA are based on word-relation triplets (\texttt{A}, \texttt{Relation}, \texttt{B}) from ConceptNet.
The prevalent relations include \texttt{AtLocation} (A is typically located at B), \texttt{Causes} (A is the typical cause for B), and \texttt{CapableOf} (A can typically do B).}
% \texttt{AtLocation} means that B is typically the location for A.
% \texttt{Causes} means that A is typically the cause for B.
% \texttt{CapableOf} means that something that A can typically do is B.
Moreover, over 98\% of words in CSQA questions are covered in ConceptNet. Therefore, ConceptNet is a suitable resource for contextualizing model behaviors on CSQA and other commonsense QA datasets~\cite{feng2020scalable}.

% Here, we adopt \cpn{}~\cite{conceptnet} as the external resource. It is a large-scale commonsense knowledge graph, where concepts (\ie, \textbf{words}) are connected with \textbf{relations} (\eg, relation \hl{be capable of} presents the directed link from \hl{people} to \hl{driving}). 
% It has over 8 million nodes (1.5 English nodes) and over 21 million links in the graph, integrating diverse knowledge sources.
% Particularly, it uses 36 general relations to connect words, mostly covering taxonomic (e.g., \hl{IsA}), lexical knowledge (e.g., \hl{RelatedTo}, \hl{Synonym}), and physical commonsense knowledge (e.g., \hl{MadeOf}, \hl{PartOf}).
% It has good generality and is broadly used to inject commonsense capabilities into NLP models~\cite{lin2019kagnet, feng2020scalable} and build reasoning benchmarks~\cite{CSQA1, CSQA2, lin2021riddlesense}.
% For example, all questions and answers in CSQA dataset are authored based on word-relation triplets drawn from \cpn{}. And over 98\% of words in CSQA questions are covered in \cpn{}.
% Thus, it is a suitable resource for contextualizing model behaviors on CSQA and many other commonsense QA datasets~\cite{feng2020scalable}. 

% \textbf{Commonsense knowledge base.} We resort to an external knowledge base to capture the commonsense knowledge in the QA data, which provides the context for inferring the model's implicit commonsense reasoning.
% To ensure such context is meaningful and helpful for the model analysis, the knowledge base should have \textit{sufficient coverage of the relevant commonsense} reflected in the corresponding QA data.

% Here, we adopt \cpn{}~\cite{conceptnet} as the external resource. It is a large-scale commonsense knowledge graph, where concepts (\ie, \textbf{words}) are connected with \textbf{relations} (\eg, relation \hl{be capable of} presents the directed link from \hl{people} to \hl{driving}). 
% It has over 8 million nodes (1.5 English nodes) and over 21 million links in the graph, integrating diverse knowledge sources.
% Particularly, it uses 36 general relations to connect words, mostly covering taxonomic (e.g., \hl{IsA}), lexical knowledge (e.g., \hl{RelatedTo}, \hl{Synonym}), and physical commonsense knowledge (e.g., \hl{MadeOf}, \hl{PartOf}).
% It has good generality and is broadly used to inject commonsense capabilities into NLP models~\cite{lin2019kagnet, feng2020scalable} and build reasoning benchmarks~\cite{CSQA1, CSQA2, lin2021riddlesense}.
% For example, all questions and answers in CSQA dataset are authored based on word-relation triplets drawn from \cpn{}. And over 98\% of words in CSQA questions are covered in \cpn{}.
% Thus, it is a suitable resource for contextualizing model behaviors on CSQA and many other commonsense QA datasets~\cite{feng2020scalable}. 
}


\subsection{Extract Relevant Commonsense Knowledge}\label{subsec.extract_commonsense}
% introduce motivation (why we do it and what commonsense knowledge base we use)
To help users build a concrete understanding of commonsense questions and their connections with model behavior, we distill relevant commonsense knowledge in data instances based on \cpn{} (\imp{R1}).
% In our work, 
% we introduce ConceptNet~\cite{conceptnet} as an external resource to find underlying relations between the relevant concepts of the questions. ConceptNet is one of the most comprehensive commonsense knowledge graph, covering millions of concepts and relations (\eg, \hl{antonym}, \hl{at location}, \hl{related to}).
% Besides, many existing commonsense reasoning benchmarks (including the CSQA dataset used in our paper)~\cite{CSQA1, CSQA2, lin2021riddlesense} are built upon or extended from ConceptNet.

% introduce extraction methods
The commonsense knowledge extraction consists of two major steps \renfei{(\autoref{fig:system_framework}B)}, including recognizing mentioned concepts in the questions and constructing sub-graphs on the concepts.
% First, each question instance of CSQA contains a question concept (QC), a question stem (QS), and a target concept (TC, also the ground truth). 
% Readers can refer to \autoref{subsec.datasets} for more details on the CSQA dataset.
\rev{To reflect the reasoning paths from the question concept to the target concept/answer, we perform tokenization of the question stem by n-gram (\rev{$n=1,2,3$\footnote{\rev{To balance the coverage of meaningful phrases with varying lengths and computational complexity, we limit maximum gram size to be three~\cite{manning1999foundations}.}}}) and match the tokens (\ie, words of length $n$) with the concepts in ConceptNet to identify a set of candidate concepts for commonsense reasoning. 
Since the matched concepts (with different lengths) may have overlaps, we reduce the redundancy by keeping the longest matched concepts in \cpn{}.}
Moreover, to enhance the robustness of matching, we conduct soft matching by lemmatization and removal of stop words and punctuations.
For example, after token matching, the candidate concepts in a question \hl{A man wants air conditioning, ...} will be \{man, want, air conditioning, ...\}.
Next, those tokens are used to construct a knowledge graph that contains the question concept and the target concept
% QC and TC 
to describe the reasoning process.
By leveraging the connections among the candidate concepts, question concept, and target concept in \cpn{}, we establish relational paths, employing a two-hop relation search. 
% Specifically, we employ two-hop relation search to find the relation connections among the candidate concepts, question concept, and target concept in \cpn{}.
% QC, and TC in ConceptNet. 
% If concept A is within a two-hop relation search of concept B using \cpn{}, then A and B will be connected and their relational paths will be stored.
We set the hop size to two to balance the computation scalability and coverage of reasoning paths, following the prior work~\cite{lin2019kagnet,yasunaga2021qagnn}.
% because the number of the possible paths between two concepts is \textit{exponential} to the hop size. And we keep the relevant reasoning paths by compromising between computation scalability and coverage of reasoning paths, following the prior work~\cite{lin2019kagnet,yasunaga2021qagnn}.
% \renfei{For example, since the relation connects ``ship'' with ``metal'' and ``metal'' with ``corrosion'', ``ship'' and ``corrosion'' will be connected in the extracted graph. }
% \xingbo{TODO: consider adding an example here if necessary}
% And two concepts are connected if they are within N-hop search. 
% Here, to balance the scalability and relevance of the knowledge extraction, we set the hop to two by following the prior work~\cite{lin2019kagnet}. 
Thereafter, the resulting graph of concepts and relations (in~\renfei{\autoref{fig:system_framework}B}) describes the relevant commonsense knowledge for the question. This graph is referred as ConceptNet knowledge.


\subsection{Align Model Behavior with ConceptNet Knowledge}\label{subsec.align_model}
\rev{To help users build mental models about the model's relational reasoning over concepts,
% the model's commonsense knowledge and implicit reasoning, 
we align the model input-output behavior with ConceptNet knowledge regarding different concepts and relations (\imp{R3}).}
% The alignment provides contexts for users to decide whether the model captures proper words for reasoning and learns the implicit commonsense relations among them.
% concept alignment
For concept alignment \renfei{(\autoref{fig:system_framework}C)}, SHAP is used to calculate the importance scores of the input concepts to the model outputs.
And we call those with large positive influences on the model predictions as model concepts.
\rev{Then, we compute the differences between the set of model concepts and the set of \cpn{} concepts (\ie, question concepts and concepts in question stems derived in \autoref{subsec.extract_commonsense}).}
% relation alignment
% \rev{Since commonsense relations are not explicit in the input, we introduce}
\rev{For relation alignment, 
% For relation alignment , we contextualize model behavior with commonsense relations in \cpn{}. Particularly, 
we mainly consider the key relations (\ie, the relations between question concepts and target concepts) for correctly answering the questions (\autoref{fig:system_framework}C). Noticing that \textit{question concepts} are included in question stems as \textit{model inputs} and \textit{target concepts are ground truths for model outputs}, we surface the model learning of 
% the \textit{QC-TC relations} 
their relations
by investigating the relationships of model inputs and outputs.}
% To surface the model learning of the underlying QC-TC relations,
% relations between QCs and TCs, 
% we investigate input-output relationships 
\rev{Specifically, the inputs and outputs are high-dimensional embeddings that the model operates on. And we compute the linear transformation matrix $W \in \mathbb{R}^{d \times d'}$ between model input embeddings $X \in \mathbb{R}^{N \times d}$ and output embeddings $Y \in \mathbb{R}^{N \times d'}$.
Particularly, to reflect relations between question concepts and target concepts encoded in $W$, we use those correctly-predicted instances (\ie, model predictions $P$ are equal to target concepts) as the anchor points for the transformation. And we adopt a least-square error objective to compute the linear matrix $W$: $\mathop{\mathrm{argmin}}_{W \in \mathbb{R}^{d \times d'}}~||XW - Y||_2$, where $(X, Y) = \{(x_i, y_i)~|~TC_i = P_i\},~i=1,...,N $.
% given the ground truth answers $TC$ and model predictions $P$, we consider the correctly-predicted instances as the anchor points for alignment and a least-squares error objective:
}

\begin{comment}
\begin{equation}
\mathop{\mathrm{argmin}}_{W \in \mathbb{R}^{d \times d'}}~||XW - Y||_2
\end{equation}
\begin{equation}
(X, Y) = \{(x_i, y_i)~|~TC_i = P_i\},~i=1,...,N 
\end{equation}
\end{comment}


% Where $X = \{x_i | x_i \in \}$, $Y = \{\}$, and $W$ is the linear mapping between $X$ and $Y$.
% Where $X$ and $Y$ correspond to the pairs of input-output model embeddings in high-dimensional vector spaces $\{ (x_i, y_i) | \}$.

\rev{The general idea is that the input-output relationships can be modeled by translations in the model embedding space~\cite{bordes2013translating, dinu2014improving}: if a model can capture the relations between question concepts and target concepts,  then question concept embeddings transformed with the matrix $W$ should be close to target concept embeddings.}
% each other after applying the transformation matrix $W$ to QC embeddings.


\subsection{Model Editing}\label{subsec.model_editing}

\vis{After identifying model deficits in specific commonsense knowledge, we present \textit{editor networks} to modify model parameters that can correct problematic model answers (\imp{``reliability''}), as well as other semantically-equivalent questions (\imp{``generality''}) without affecting unrelated knowledge much (\imp{``locality''}).
Particularly, editor networks are neural networks trained to modify model parameters (from $\theta$ to $\theta'$) with the objectives that maximize editing accuracy on both editing targets ($x_e, y_e$) and their equivalence ($x_e', y_e'$) while minimizing differences (KL divergence) in model predictions on locality examples ($x_{loc}, y_{loc}$) before and after the edits: $L_e = -log p_{\theta}'(y_{e}'|x_{e}'), L_{loc} = \texttt{KL}(p_{\theta}(\cdot | x_{loc})||p_{\theta}'(\cdot | x_{loc}))$.
% \begin{equation}
% L_e = -log p_{\theta}'(y_{e}'|x_{e}'), L_{loc} = \texttt{KL}(p_{\theta}(\cdot | x_{loc})||p_{\theta}'(\cdot | x_{loc}))
% \end{equation}
The total loss is $L_{total} = - w_{e} \cdot L_{e} + L_{loc}$, where $w_e$ is a weight factor. The editing examples come from QA pairs in CSQA train/val set, 
where their equivalences are generated by popular 
data augmentation techniques, i.e., back-translation and EDA~\cite{DBLP:conf/emnlp/WeiZ19}.
% backtranslation and \todo{data augmentation} techniques, 
Locality examples are independently sampled. We adopt gradient decomposition techniques~\cite{mitchell2022fast} to train editor networks on the last two transformer layers of the model. More technical details are included in 
\referappendix{Suppl. A.}
% \referappendix{Suppl. \ref{sec.model_editing}}.
}
% 


\begin{comment}
The model behavior is described by input feature importance and output accuracy. 
The feature importance is measured by SHAP values of associated pieces. We consider positive SHAP values and connect the pieces into tokens (concepts). 
Then we align model behavior with ConceptNet knowledge. 
To align concepts, we compare the differences between concepts extracted from SHAP values and from ConceptNet. The concepts that are covered by both indicate the concepts that are considered by the model. The concepts that are only considered by SHAP values are those stop words or concepts that are rarely seen. The ConceptNet-only concepts indicate the contexts that the model ignored and may lead to wrong answers. 

For relation alignment, we mainly consider the relations from QCs to TCs which are the commonsense knowledge explicitly tested. 
To surface the patterns of the relation learning, we learn a translation matrix between QS embeddings and TC embeddings from the model. After the translation, the sufficiently learned questions within a relation should form a cluster. By inspecting the instances away from the clusters, we are able to identify relations not learned instances.  

In summary, we use the data and computation metrics mentioned in this section for visualizations. 
\end{comment}

% Compute Feature Importance and model performance. 
% - We consider data with positive SHAP values.

% Align model behavior with ConceptNet Knowledge (concept and relations):
% - Concept-driven alignment (compare SHAP with relevant concepts from ConceptNet)
% - Surface patterns of relation learning. (compute translation matrix between question stem embeddings and target concept/answer embeddings)

% Summarize our data and computation metrics for visualization
% - data: csqa instances, model (and embeddings)
% - computation metrics. (accuracy, SHAP coverage on QC/TC/QS)


\subsection{User Interface of \systemname}
\label{sec.user_interface}

% Figure environment removed 


The user interface (\vis{\autoref{fig:teaser}}) enables a multi-level exploration of model behavior following an \textit{overview-to-detail} flow, contextualized by \cpn{}. 
\revv{The exploration process starts with the \gv{}, which summarizes model performance on different concepts and relations and assesses overall relation learning. 
Users then can pinpoint error cases, and the system summarizes the contexts of alignment between model behavior and \cpn{} on different subsets.
Upon selecting instance subsets in the \gv{} or \sv{}, \iv{} shows statistics and visual explanations for these instances. It facilitates interactive model probing for comprehensive understanding and enables users to bookmark particular instances for targeted model refinement.}
% The  \gv{} summarizes the model performance regarding QSs, TCs, and QC-TC relations extracted from \cpn{}. The Subset View further visualizes the alignment between model behavior and \cpn{} knowledge regarding different groups of QCs, QSs, and TCs (\imp{R1, R2, R3}). 
% Users can select a group of instances in the \gv{} or \sv{}, and \iv{} displays the statistics and local explanations. 
% It also allows users to interactively probe and edit the model (\imp{R1, R4}). 
% All views use a consistent color scheme, with green indicating accuracy, red indicating error, and categorical color palettes showing various relations and statistics.
The system uses red to encode the model error, green to indicate accuracy, categorical colors to encode different relations and statistics.

%%%%%%%%%%%%%%%%%%%%%%%%%%%% OLD %%%%%%%%%%%%%%%%%%%%%%%%%%%%
\begin{comment}
% According to the generated model explanations based on ConceptNet, 
\rev{After contextualizing model behavior using an external knowledge graph,}
the user interface of \name{} (\autoref{fig:teaser}) enables a multi-level exploration of model behavior following an \textit{overview-to-detail} flow.
% regarding different concepts and their implicit relations for commonsense reasoning (\imp{R2, R3}), as well as interactive probing of models (\imp{R4}).
After users load the QA dataset and the language model, 
the \gv{} summarizes the model performance regarding QSs, TCs, and QC-TC relations extracted from \cpn{}.
% regarding different concepts and commensense relations in QA instances.
% on QSs, TCs, and the latent relations in ConceptNet .
% Then, they can examine the \sv{} to explore the connections between 
The \sv{} further visualizes the alignment between model behavior and ConceptNet knowledge on different groups of QCs, QSs, and TCs
% concepts in instances.
% groups of QCs, QSs, and TCs 
(\imp{R1, R2, R3}).
After selecting a group of instances in the \gv{} and/or \sv{}, \iv{} displays the statistics and local explanations. In addition, it allows users to modify the input and output, and interact with the model (\imp{R1, R4}).

All the views adopt the same color scheme where green encodes accuracy, red denotes error, and categorical color palettes show varied types of relations and statistics.
\end{comment}
%%%%%%%%%%%%%%%%%%%%%%%%%%%% OLD %%%%%%%%%%%%%%%%%%%%%%%%%%%%

% Figure environment removed

\subsubsection{\gv{}}\label{subsec.global_view}
\rev{Initially, users can refer to the \gv{} to gain an overview of the model performance regarding different concepts and commonsense relations contained in QA data (\imp{R1, R2}).}
Specifically, the \gv{} (\vis{\autoref{fig:teaser}A}) adopts different projection strategies to group question stems and target concepts (\ie, answers) and visualize them as two separate scatter plots. 
% It further relates them to model performance and surfaces the model learning of QS-TC relation types (\imp{R1, R2}). 
% We use two scatter plots to visualize the projection of QSs and TCs. 
\revv{For projection, we choose UMAP~\cite{mcinnes2018umap} with cosine similarity measures to cluster model embeddings for question stems and target concepts}
% of QSs and TCs 
because of its good processing speed and preservation of embeddings' global structure. 
\revv{After projection, similar question stems (\ie, similar contextualizations of question concepts) or target concepts are close to each other.}
% Moreover, users can select 
% Then, users can switch the dot color schemes (``Correctness'' or ``Relation'') at the header to explore the distributions of prediction errors or QC-TC relations in the scatter plots.
\revv{To further analyze error and relation distributions among these instances, users can adjust the dot color schemes at the header. When the ``Correctness'' scheme is selected, dots are colored in red and green to show distributions of incorrect and correct instances. Alternatively, selecting the ``Relation'' scheme applies categorical colors to the dots, highlighting instances with different relations.}
Meanwhile, users can change projection mode into ``Correctness'' or ``Relation'' at the header to accentuate the differences between instances with high and low errors or instances with varied relations.
% QC-TC relations.
To achieve this, we utilize instance correctness and relations between question concepts and target concepts as additional labels for UMAP to perform supervised dimension reduction for clear cluster separation in the scatter plots.
\revv{To mitigate the overplotting in the scatter plots, the system supports semantic zooming that allows users to navigate specific areas of interest (\eg, error instances) within dense data points.
% within dense data points. 
Moreover, users can filter out the instances with particular relations by clicking the rectangles between the two scatter plots, where each rectangle encodes relation frequency and accuracy.}
% different \rev{instances}
% clusters of instances 
% in the context of prediction errors and QC-TC relations.
% Moreover, users can look at the rectangles between the two scatter plots to explore model accuracies \revv{(between 0 and 1)} for various 
% QC-TC  relations\footnote{\rev{Multiple relations can exist between the question concepts and target concepts in an instance. And each will be counted in the frequency calculation.}}.
% Each rectangle corresponds to a relationship. 
\revv{For each rectangle, we use green (instead of categorical colors) to emphasize model accuracy for that relation, where the width of the green bar denotes accuracy, and its height corresponds to relation frequency.}
The system sorts these rectangles by relation frequency, allowing users to prioritize model performance exploration of more prevalent relations.
% Within each rectangle, the height of green bars correlates with relation frequency\footnote{\rev{Multiple relations can exist between the question concepts and target concepts in an instance. And each will be counted in the frequency calculation.}} and its width represents accuracy. 
% We use green color to encode rectangles (\ie, relations) instead of categorical colors because we want to emphasize the model accuracy information.
% \rev{These rectangles are sorted by relation frequency such that users can prioritize exploring model performance on prevalent relations.}

\rev{Besides, the \gv{} assesses how the model regards latent relations between questions and answers (\imp{R3}).} In the ``Relation X Transformed'' projection mode (in \autoref{fig:teaser}), the \gv{} separates instances with different relations in the scatter plots and supports the alignment and comparison of transformed question stems with target concepts.
% (details are in \autoref{subsec.align_model}).
If there is a good correspondence between transformed clusters of question stems and target concepts in the scatter plots, then the relations between question and target concepts could possibly be learned.
\rev{Finally, users may lasso a group of instances or click specific relation bars to inspect the context summary in the \sv{}.}
% Besides, to enable a multi-faceted analysis of clusters of data instances, we use relation types or correctness as labels to perform supervised dimension reduction for clear cluster separation in the scatter plots. Users can switch between projection and transformation strategies at the header to explore model behavior from different perspectives. 

% To summarize model performance regarding QSs and TCs (\imp{R2}), we first apply UMAP to project their high-dimensional model embeddings onto a 2D plane, respectively. 
% UMAP is used because of its good processing speed and presevation of embeddings' global structure. After projection, instances are presented as dots and those having similar model embeddings are close to each other. 
% Users can color the dots by correctness to relate model performance to different clusters of instances. 
% To summarize model performance with QC-TC relations, 


\textbf{Alternative design}. We have considered an alternative---grouped bar charts---to visualize the relations between question and target concepts (\renfei{\autoref{fig:alternative designs}A}). 
For each relation, green bars show accuracy while blue bars encode frequency. The longer the bars, the larger the encoded values.
% It uses a parallel bar chart with two bars for each QC-TC relation. The height of the vertically placed green bar encodes the accuracy and the blue bar encodes the number of questions including the specific relation between their QCs to TCs. 
We collected experts' feedback on this alternative.
% and collected their feedback. 
\imp{E1} said that our final design using a single color looks simpler and cleaner. 
\rev{\imp{E3} commented that horizontally aligning green bars next to blue bars in the grouped bar charts could be confusing since the frequency and accuracy have different units of measurement.
\imp{E2} reported that our final design can reflect the proportion of different relations in the whole dataset more clearly.
% it is easier to compare different proportions of QC-TC relations in the dataset.
In addition, it sorts the frequent relations in descending order, helping prioritize the exploration.}
% makes the smaller proportion relation bar looks tinier which intuitively indicates its importance. 
% Moreover, our final design is more space-efficient than the grouped bar charts when the number of relations increases.
% scalability of space usage of the final design is also better than the parallel bar chart. 

% The \gv{} consists of two scatter plots that display the projection of QSs and TCs. The users can choose between six types of different projection strategies and two color schemes for different analytical purposes. 
% There is a stack of rectangles connecting the two scatter plots showing information on the relations between the QCs and TCs. The height of each rectangle encodes the proportion of questions with the specific relation between QC and TC. The horizontal green bar encodes the model accuracy within that group.  


\subsubsection{\sv{}}\label{subsec.subset_view}
% Besides the global summary of model performance and relation learning in the \gv{},
\revv{After selecting a group of instances with specific concepts or relations in the \gv{},}
users can utilize the \sv{} to explore the
% the \textit{context} about the 
concept alignment between the model behavior and \cpn{} knowledge across different subsets (\imp{R2}).
% The \sv{} \renfei{(\autoref{fig:teaser}B)} summarizes the context about the alignment between the model behavior and ConceptNet knowledge regarding QCs, QSs, and TCs (\imp{R1, R2, R3}).
This view employs cluster glyphs to analyze model behavior across instances with semantically similar question concepts, question stems, and target concepts. Hierarchical clustering of ConceptNet Numberbatch embeddings~\cite{conceptnet} is performed for question stems, question concepts, and target concepts.
% Specifically, the \sv{} employs three groups of cluster glyphs to summarize how the model behaves on groups of instances having semantically similar question concepts, question stems, and target concepts. 
% We perform hierarchical clustering of ConceptNet Numberbatch~\cite{conceptnet} embeddings of question stems, question concepts, and target concepts, respectively.
% of QCs, QSs, and TCs, respectively. 
We use ConceptNet Numberbatch embeddings because they encode word meanings based on \cpn{}'s semantic network and perform well on \revv{word-relatedness benchmarks~\cite{conceptnet}.} Then, users can scan through the cluster glyphs 
% for question stems, question and target concepts 
and sneak peek into the corresponding model performance, the important words for model decisions, and how they are aligned with \cpn{} concepts (\renfei{\autoref{fig:alternative designs}B}).
% For each cluster, we design a glyph to help users sneak peek into the statistics and model behavior (\renfei{\autoref{fig:alternative designs}B}).
For each cluster glyph, two bars are presented at the top \revv{showing the average accuracy (between 0 and 1) and overlap ratio (between 0 and 1)} between model concepts and \cpn{} concepts.
The lower parts display the differences between the model and \cpn{} concepts.
% covered by the model (\ie, considered important) or not.
The first row displays the top ConceptNet concepts frequently missed by the model. And an orange bar is put to the left, revealing the frequency. Then, the second row shows the frequent model concepts and their frequency (with blue bars). 
To further explore concept associations across different questions, and question and target concepts, their cluster glyphs are connected with links if their data instances overlap---the wider the link, the greater the shared data instances.
\revv{To reduce the visual clutter of links, the system allows users to adjust the cluster numbers at the header. When users hover over a specific cluster glyph, the system highlights only the connections relevant to that cluster, while keeping other links hidden.}
% \rev{Users can click the interested cluster glyph to see the details in the \iv{}.}

% \rev{Users may also want to know what are the question contexts (\ie, QSs) for a group of QCs or TCs of interest.}
% To show the relationships between QCs and QSs and TCs, their cluster glyphs are arranged from top to bottom and connected by links. Two clusters will be linked together if their data instances overlap. And the link width is proportional to the number of shared data instances.
% \revv{To reduce the visual clutter of links, the system enables to adjust cluster numbers at the header and highlights only the connections relevant to a specific cluster glyph when users hover over it and keep other links hidden.}
% of QCs, QSs, and TCs are arranged from top to bottom and connected by links. 
% Two clusters will be linked together if their data instances overlap with each other. And the width of the links is proportional to the number of shared data instances.

% for each cluster, we design a glyph to sneak peek into the overall differences between the model concepts and ConceptNet concepts.
% we seperate concepts covered only by models and 
% The \sv{} displays more details of lassoed data in the \gv{} with cluster glyphs connected by links. 
% We propose a novel glyph design to help users sneak peek into the overall information of each cluster. 
% Each cluster glyph is presented with two bars on top showing the accuracy and average SHAP coverage within the cluster. The lower parts consist of two groups of information. The above group are the concepts that have SHAP coverage scores below 0.5 and the group below are the concepts that have SHAP coverage scores above 0.5. 
% The two rows of texts show the top 5 words with the highest frequency in each group. 
% There are two rectangles on the left showing the count of the words in the upper texts and the lower texts colored with yellow and blue respectively. 

% The cluster glyphs are arranged in three rows with links connecting them. The top row shows the QCs within the selected questions. The middle row shows the question stems and the bottom row shows the TCs of the questions. 
% The links between cluster glyphs show the clusters share the same question IDs between each row. The width of the links shows the proportion of data between clusters. 


\textbf{Alternative design}. 
Initially, we considered using a word cloud (\renfei{\autoref{fig:alternative designs}B}) to summarize the most frequent concepts (not) covered by the model. And the word size relates to frequency. 
However, the word cloud is not space-efficient and mixes the model concepts with ConceptNet concepts and increases the visual complexity, making the system less user-friendly.
More importantly, our users prioritize reading the concept words in plain style. Therefore, we chose our current design.

% the most frequently appeared words (contexts) that model attached as important or the model did not treat as important. 
% E1 reported that 
% although word cloud is fancy as an individual visualization. When arranged in grids, it does not fit into the space very well. Besides, 
% the word clouds increase the visual complexity of the \sv{}, making the system less user-friendly. 
% E2 added that word clouds are not consistent with the overall design styling of our system. 
% E3 said that if we want to add more statistical information such as accuracy around a word cloud, it would be difficult and unnatural to the user. 


\subsubsection{\iv{}}
After selecting instances in the \gv{} or \sv{}, the Instance View \renfei{(\autoref{fig:teaser}C)} provides statistics and local explanations about the model. It enables probing of the model with different inputs and outputs to test its learning of concepts and commonsense relations (\imp{R1, R4}). 
The top stacked bars show model accuracy and average question concept (QC) hit ratio \revv{(between 0 and 1)}. Users can click the green (or gray) segments with the stacked bars to filter the data instances correctly (or wrongly) predicted. 
The histogram below displays the top frequent concepts considered important to the model. 
Users can explore individual instances and model explanations with pagination. The question stems that strongly contribute to model outputs are highlighted with green backgrounds, and question concept is underlined. Model choices colored red indicate a wrong answer, and the ground truth is colored green. 
Users can verify and generalize their findings by searching for linguistic patterns in data instances that contain certain words or structures (\eg, \hl{many NOUN}) at the top. 
% This allows them to seek out specific linguistic patterns of interest, which may encompass certain words or structures (e.g., "many NOUN"). 
% The patterns can contain specific words or structures (\eg, \hl{many NOUN}). 
% After searching, users may examine the statistics and/or model behavior on individual instances.
\revv{For instance, after searching a question concept of interest, users can review the model performance on different contextualizations (\ie, question stems) of that concept and associated relations in the \gv{}. Then, further detail can be investigated, including statistics and model explanations for individual instances, in the \iv{}.}
 
% After selecting instances of interest in the \gv{} or \sv{}, the \iv{} \renfei{(\autoref{fig:teaser}C)} provides the statistics and local explanations about them.
% Moreover, it enables users to probe the model with different inputs and outputs to test the model learning of concepts and commonsense relations (\imp{R1, R4}).
% \rev{Users can first inspect the two stacked bars at the top to know the model accuracy and average QC hit ratio of the model. 
% They can click the green (or gray) segments with the stacked bars to filter the data instances correctly (or wrongly) predicted.
% Moreover, users can skim through the top frequent concepts considered important by the model in the histogram below.}
% % In the instance view, users can observe the accuracy and the average QC hit ratio of the selected questions on the top. 
% % The SHAP attached important words are counted and the top 20 words are displayed in a bar chart. The text of the words is shown below. 
% \rev{Then, users can explore the individual instances and model explanations with pagination at the bottom of the \iv{}.}
% % are displayed with pagination at the bottom.
% % Users can navigate through the instances to 
% \rev{For each instance, users can check the QCs, TCs, and their relationships (retrieved from \cpn{}), question-answer pairs, and local explanations.
% Next, they may generate hypotheses about the model's learning of concepts and relations.}
% Particularly, to explain model learning of concepts, the words in the QSs that strongly contribute to the model outputs will be highlighted with green background. Also, QC is underlined for quick identification.
% If the model answer is wrong, the model choice is colored red while the ground truth (\ie, TC) is colored green. If the model is correct, only TC will be colored green. \rev{Based on the model accuracy and highlighted words in the QSs, users can decide whether the model relies on reasonable words to make decisions.
% Furthermore, users can generalize their findings about the model learning of concepts by searching linguistic patterns of interest at the header of the \iv{}. The patterns can be instances containing specific words or structures (\eg, \hl{many NOUN}). Users may examine the statistics and/or model behavior on individual instances after searching.
% }

\rev{For individual instances, users can edit them to form and validate hypotheses about the model learning of relations.
For example, 
% if the model attaches importance to QCs and answers the questions correctly, then it is possible the model learns the QC-TC relations.
if the model is wrong, users may hover over different answer choices to see their relations with \cpn{} concepts in question stems.
If both model answers and target concepts share the same relations with question concepts, the model potentially does not understand the contexts.
Then, users can edit the text content of question stems and individual answer choices (\eg, remove some words in questions and change answer choices), followed by re-running the model on the edited QA pairs. The new model answers will be highlighted in blue. By examining the new results, users can validate whether the relations between the question and target concepts are learned. 
% For example, if QSs are rephrased to question the QC-TC relations directly, and the model is wrong, then the model possibly does not learn the relations.
}
% To help users test their hypotheses about model relational learning over concepts, users can edit the QS and choices and re-run the model on edited content. The new model answers will be highlighted in blue. 
% When hovering over each answer choice, its relations with the QS concepts that are within one-hop distance in ConceptNet will be displayed. 
% This helps users quickly identify which words in question stem may strongly influence the model result. 
% Furthermore, users can search linguistic patterns of interest, for example, instances containing specific words or structures (\eg, \hl{many NOUN}), in the query input box to generalize their findings about the instances.
% The detailed questions and model reasoning information are displayed with pagination, each time the user is displayed with the details of a single question. 
% The users can navigate through different pages to check instance-level results. Each page displays one question. 
% The SHAP values of tokens in the question stem are encoded with color opacity. We underline the QC in the stem to make users easier to locate where the QC is. 
% If the model's result is correct, only one choice will be colored green. If the model's result is wrong, the choice of model result is colored red and the ground truth (TC) is colored green. 

\vis{Users can bookmark instances about specific knowledge that the model does not learn well. Then, they can conduct model editing in the Model Editing Panel \todo{(\autoref{fig:teaser}D)}, where information about questions, relations, ground truths, and model results are summarized in a table. Users can apply editing to instances of interest and inspect the editing performance. Moreover, they can load the edited model for exploration.}

\begin{comment}

\subsubsection{User Interactions}\label{subsec.ui}
\name{} offers various interactions to support multi-level analysis of model behavior with details on demand.

% Besides the interface components introduced earlier, \name{} supports a rich set of interactions that facilitate multi-level analysis of model behavior with details on demand.

\textbf{Lasso and pan-and-zoom}. 
In the \gv{}, users can lasso a group of data instances in the scatter plots to examine the details in the \sv{} and \iv{}. 
And users can use pan-and-zoom in the scatter plots to navigate local clusters more easily. 
% the scatter plots provide users with a pan-and-zoom to navigate the local clusters easier. Users can reset the plots to the default scale by clicking the reset button. 

% % Interactions
% The scatter plot supports users with zoom and pan to navigate the local area easier. When the mouse hovers over the points, the information of each question and its corresponding point in the other scatter plot is displayed at the same time. 
% The user can lasso the data points in scatter plots to perform further interactions with the lassoed data points in the subset view and instance view. 
% When hovering over the relations rectangles, detailed information about the relation is shown. The users can click on each rectangle to filter questions with certain QC-TC relations in scatter plots. The data is also updated into the subset view and the instance view. 


\textbf{Hovering and clicking}. 
To make the interface cleaner and less overwhelming, we hide lots of details, and users can hover or click to see the details on demand. For example, in the \gv{}, users can hover over
the dots in the scatter plots and the relation bars in the middle to see the pairs of QSs and TCs and relation accuracy, respectively.
When hovering the cluster glyphs in the \sv{}, detailed concepts and statistics of the clusters, together with their connections with other clusters, will be displayed. 
In the \iv{}, hovering over the charts will display the detailed numbers. Also, users can hover over the answer choices to query their relations with the QS concepts.

Moreover, users can filter or highlight the information by clicking.
For example, relation bars in the \gv{}, and stacked bars in the \sv{} can be clicked to filter data instances.
In addition, users can navigate through instances by clicking the pagination buttons. Meanwhile, its corresponding dots and clusters will be highlighted in the \gv{} and \sv, respectively.


% Instance view:
% When hovering on the statistics information bars in the Instance View, detailed information is shown. 
% The users can click the accuracy and average QC hit bars to filter and display the questions within that group. 

% \textbf{Linked highlighting}

% \textbf{In-situ Instance Editting}
% The users can edit the question stem and choices and rerun the model on edited content to test their hypotheses. 
% When hovering on each choice, the concepts in question stem that is within one-hop distance is displayed with the relations between them. This supports users to quickly identifying which words in question stem may strongly influence the model result. 
 
\end{comment}

\section{Case Study}
\label{sec:case_study}
In this section, we carry out a qualitative analysis to highlight the merits of our model by demonstrating the generated samples using the WMT14 dataset. Several representative samples are presented in Table 3.





Through comparison, we observe that the conventional RDM model struggles to grasp the mapping relationship between different language domains and is inclined to generate text with a blend of domains. Conversely, under the guidance of language embedding, our XDLM model proficiently masters the mapping relations between different languages. The sample also illustrates the enhanced speed of convergence exhibited by XDLM under the pretraining provided by TDLM.

%\begin{table}[]
%\centering
%\begin{tabular}{|c|l|l|}
%\hline
%\multicolumn{3}{|c|}{Source: Yesterday, Gutacht's Mayor gave a clear answer to this question.} \\
%\multicolumn{3}{|c|}{Target: Diese Frage hat Gutachs Bürgermeister gestern klar beantwortet.}  \\
%\hline
%Name & Iteration & Decodes \\
%\hline
%\multirow{RDM-Multinomial} & 0 & Content 1 \\
% & 1 & Content 2 \\
% & 2 & Content 3 \\
% & 3 & Content 4 \\
% & 4 & Content 5 \\
%\hline
%\multirow{XDLM} & 0 & Content 6 \\
% & 1 & Content 7 \\
% & 2 & Content 8 \\
% & 3 & Content 9 \\
% & 4 & Content 10 \\
%\hline
%\end{tabular}
%\caption{Qualitative samples of test paraphrases generated from different diffusion models. Words are in lower case}
%\label{table:your_label}
%\end{table}


\begin{table}[!t]
\begin{center}
\setlength{\tabcolsep}{1.0pt}
\begin{tabular}{lccc}
\toprule
 & PT & SNP & SNP+ \\
 & \cite{liu2021painttransformer} & \cite{zou2021stylized} & \cite{zou2021stylized} \\
\midrule
Preferences & 97.9\% & 97.1\% & 95.0\%  \\
\bottomrule
\end{tabular}
\end{center}
\caption{User study comparing the preferences between \methodname~and the respective baseline.}
\label{tab:user_study}
\end{table}


\section{Discussion}
\label{sec: discussion}
\kmsdelete{In this work} We study \kmsreplace{Fairness-Aware PAC learning}{Fair-ERM} in the malicious noise model, and  in some cases allow 
the learner to maintain optimal overall accuracy despite the signal in Group $B$ being almost entirely washed out.
%when we allow learners to use the
%$\PQ$ randomized expansion of the hypothesis class $\mathcal{H}$
In particular we show that different fairness constraints have fundamentally different behavior in the presence of Malicious Noise, in terms of the amount of accuracy loss that a given level of Malicious Noise could cause a fairness-constrained learner to incur. 
The key to achieving our results, which are more optimistic than those in \cite{lampert}, is allowing for improper learners using the (P,Q)-randomized expansions of the given class $\mathcal{H}$.
%We \kmsreplace{present a picture of the}{prove upper and lower bounds on}
%accuracy loss for a range of fairness notions, given \kmsreplace{this simple randomization step.}{learning over $\PQ$.
%In general our results indicate Fair-ERM (given learning over $\PQ$) is more robust than claimed in \cite{lampert}.
The type of smoothness we create by using $\PQ$ seems to be a natural property that is likely shared by many natural hypothesis classes.

Fairness notions are motivated as a response to learned disparities when there is \kmsdelete{data corruption or} systemic error affecting \kmsdelete{the data for}
one group. 
Fairness notions are supposed to mitigate this by ruling out classifiers that have worse performance on a sub-group. 
This can peg both classifiers at a lower level of performance \kmsdelete{(e.g that the lower subgroup)} in order to \emph{motivate} \cite{hardt16} improving the data collection or labelling process to obtain more reliable performance. 
%So in \kmsreplace{some}{a} sense, sensitivity of the fairness notion to poor sub-group performance caused by malicious noise is the \textit{point} of fairness constraints! 
However, it also desirable that fairness constraints perform gracefully when subject to Malicious Noise because fairness constraints will be used in contexts where the data is unreliable and noisy and this might not be known to the learner.
This tension, exposed by our work, motivates 
%a revisiting of fairness notions from first principles approach and trying to axiomatize the 
%desired properties of a fairness intervention a la cryptography and privacy. \footnote{Work in multi-calibration \cite{multicalib} is a viable direction for this problem but it is unclear how 
%that and related notions behave with unreliable data. }
on going work studying the sensitivity level of fairness constraints. 
%If we we are to take a view, if a classifier is deployed 

\section{Conclusion and Future Work}
In this work, I design corruption-robust algorithms for the Lipschitz contextual search problem. I present the \emph{agnostic checking} technique and demonstrate its effectiveness in designing corruption-robust algorithms. There are several open problems for future research. First, in the algorithm I propose for pricing loss, the schedule for agnostic checks is fixed upfront. Can the learner design an adaptive checking schedule for the pricing loss? Second, this work assumes the learner has knowledge of the Lipschitz constant $L$. Can the learner design efficient no-regret algorithms without knowledge of $L$? 



%% if specified like this the section will be ommitted in review mode
\acknowledgments{%
The authors wish to thank anonymous reviewers for their valuable feedback. 
This research was supported in part by Hong Kong Theme-based Research Scheme grant T41-709/17N and a grant from MSRA. 
}


\bibliographystyle{abbrv-doi-hyperref}
%\bibliographystyle{abbrv-doi-hyperref-narrow}
%\bibliographystyle{abbrv-doi}
%\bibliographystyle{abbrv-doi-narrow}

\bibliography{main}


%% ^^^^^   FOR IEEE VIS, EVERYTHING HERE MAY BE INCLUDED IN THE    ^^^^^ %%
%% 2-PAGE ALLOTMENT FOR REFERENCES, FIGURE CREDITS, AND ACKNOWLEDGEMENTS %%

\newpage

\appendix % You can use the `hideappendix` class option to skip everything after \appendix
\begin{comment}
\section{System Architecture}
\label{appendix:architecture}
\system has a novel modularized system architecture with three key components: 
\emph{StreamManager}, 
\emph{TxnManager} and \emph{TxnScheduler}. 
These components are instantiated in each thread locally.
The execution outline of \system is presented in Algorithm~\ref{alg:algo}.
Transactional stream processing is continuous and potentially never ends (Line 1$\sim$8).
The dependency resolution and execution of state transactions are separated into two non-overlapping phases by punctuations~\cite{Tucker:2003:EPS:776752.776780} (Line 2 and 5), which guarantees that no subsequent input event will have a smaller timestamp. 
Effectively, a batch of state transactions is collected during the first phase, and processed during the second phase.

In the first phase (i.e., stream processing phase), 
the \emph{StreamManager} conducts preprocessing for every input event ($e$). Similar to some prior works~\cite{tstream}, state transactions may be issued but not immediately processed during preprocessing (Line 3).
The \emph{pre\_processing} and \emph{post\_processing} functions are exposed as APIs to users.
The \emph{TxnManager} handles dependency resolution (Line 4) among state transactions and insert decomposed operations to construct a \tpg. We discuss the detailed two-phase \tpg construction process in Section~\ref{subsec:construction}.

In the second phase  (i.e., transaction processing phase), 
the \emph{TxnManager} is first involved again to refine (Line 6) the constructed \tpg with further dependency resolution.
The \emph{TxnScheduler} 
schedules operations for concurrent execution based on the constructed \tpg according to the three dimensions of scheduling decisions (Line 7). 
In particular, a scheduling decision model $M$ is instantiated based on the constructed \tpg (Line 14).
\textbf{\circled{1}} Guided by $M$, execution threads adopt an exploration strategy (Section~\ref{subsec:explore}) to explore the constructed \tpg for operations available to be scheduled constrained by dependencies. 
\textbf{\circled{2}} 
During exploration, one or multiple operations may be treated as the 
% basic 
unit of scheduling (Section~\ref{subsec:granularity}). 
Subsequently, \textbf{\circled{3}} every thread executes operation(s) in the unit of scheduling with various abort handling mechanisms (Section~\ref{subsec:abort_handling}).
Only when state transactions are processed (i.e., committed or aborted) can the associated input events be postprocessed (Line 8) by the \emph{StreamManager} based on transaction processing results.
\end{comment}

\begin{comment}
\begin{algorithm}
\footnotesize
    \KwData{$e$ \tcp{Input event}}
    \KwData{$txn_{ts}$ \tcp{State transaction}}
    \KwData{$G$ \tcp{The currently constructed TPG}}
    \While{!finish processing of input streams}{
        \eIf(\tcp*[h]{Phase 1}){\text{$e$ is not a $punctuation$}}{
                $txn_{ts}$ $\gets$ PRE\_Processing($e$)\;
                \textbf{TPG\_Construction}($G$, $txn_{ts}$)\; 
          }(\tcp*[h]{Phase 2}){
                \textbf{TPG\_Refinement}($G$)\; 
                \textbf{TXN\_Scheduling}($G$)\; 
                POST\_Processing()\;
          }
    }
    
    \SetKwFunction{FMain}{TPG\_Construction}
    \SetKwProg{Fn}{Function}{:}{}
    \Fn{\FMain{$G$, $txn_{ts}$}}{
        $O_{1..k}$ $\gets$ \textbf{Partition} $txn_{ts}$\;
        \ForEach{\text{operation $O_{i}$ $\in$ $O_{1..k}$}}{
            \textbf{Identify} its \ld\;
            $G$ $\gets$ $G$ + $O_{i}$ \;
        }
    }
    \SetKwFunction{FMain}{TPG\_Refinement}
    \SetKwProg{Fn}{Function}{:}{}
    \Fn{\FMain{$G$}}{
        \ForEach{\text{vertex $e_{i}$ $\in$ $G$}}{
            \textbf{Identify} its \td, \pd\;
        }
    }
    
    \SetKwFunction{FMain}{TXN\_Scheduling}
    \SetKwProg{Fn}{Function}{:}{}
    \Fn{\FMain{$G$}}{
        $M$ $\gets$ Instantiated with $G$;\tcp{A decision model}
        \While{!finish scheduling of $G$
        }{
          \textbf{\circled{2}} $Scheduling Unit$ $\gets$ \textbf{\circled{1}} \emph{Explore}($G$, $M$)\; 
            \textbf{\circled{3}} \emph{Execute with Abort Handling} ($Scheduling Unit$)\; 
        }
    }
  \caption{Execution Outline of \system}
  \label{alg:algo}
\end{algorithm}
\end{comment}
\end{document}

