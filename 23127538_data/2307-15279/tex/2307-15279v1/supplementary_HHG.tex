\documentclass[%superscriptaddress,
%groupedaddress,
%unsortedaddress,
%runinaddress,
%frontmatterverbose, 
%twocolumn,
preprint,
showpacs,preprintnumbers,
%nofootinbib,
%nobibnotes,
bibnotes,
 amsmath,amssymb,
 aps,
 pra,
superscriptaddress,
%pra,
%prb,
%rmp,
%prstab,
%prstper,
%floatfix
longbibliography,
]{revtex4-2}

\usepackage[english]{babel}
\usepackage[utf8]{inputenc}
\usepackage{geometry}
\usepackage{amsmath,amsthm,amsfonts,amssymb,amscd,mathtools}
\usepackage{hyperref}
\usepackage{lipsum}
\usepackage{dcolumn}% Align table columns on decimal point
\usepackage{bm}
\usepackage{enumerate}
\usepackage{makecell,tabularx}
%\usepackage[version=3]{mhchem} % Formula subscripts using \ce{}
\usepackage{textcomp}
\usepackage{graphicx}
\usepackage{fixmath}
\usepackage{xcolor}

\bibliographystyle{unsrt}

\usepackage{color, colortbl}
\definecolor{Gray}{gray}{0.9}

%%%%%%%%%%%%%%%%%%%%%%%%%%%%%%%%%%%%%%%%%%%%%%%%%%%%%%%%%%%%%%%%%%%%%%%%%%%%%%%%%%%%%%%%%%%%%%%%%%%%%%%%%%%%%%%%%%
% misc
\def\half{\frac{1}{2}}
\def\quarter{\frac{1}{4}}
\def\noi{\noindent}
\def\beq{\begin{equation}}
\def\eeq{\end{equation}}
\def\nnl{\\[0.15cm] \nonumber}
\def\nl{\\[0.15cm] }
\def\CR{\nonumber\\[0.15cm]}
\def \gap{\:\:\:\:}
\newcommand{\id}{\mathds{1}}
\newcommand{\sub}[2]{{#1}_{\mbox{\!\! \scriptsize #2}}}
\newcommand{\mv}[1]{\mathbf{#1 }}
\newcommand{\bv}[1]{\mathbf{ #1 }}
\newcommand{\tbf}[1]{\textbf{#1}}
%derivatives
 \newcommand{\pdiff}[2]{\frac{\partial #1}{\partial #2}}
 \newcommand{\pdiffn}[3]{\frac{\partial^{#3} #1}{\partial #2^{#3}}}
%integrals
 \newcommand{\intas}[1]{\int d^3 \mathbf{#1}\:}
 \newcommand{\intasod}[1]{\int d{#1}\:}
 %bold greek
\newcommand{\balpha}{\mathbf{\alpha}}
\newcommand{\bbeta}{\mathbf{\beta}}
\newcommand{\bgamma}{\mathbf{\gamma}}
% references
\newcommand{\rref}[1]{Ref.~\cite{#1}}
\newcommand{\fref}[1]{Fig.~\ref{#1}}
\newcommand{\ffref}[2]{Fig.~\ref{#1}, Fig.~\ref{#2}}
\newcommand{\frefp}[2]{Fig.~\ref{#1}(#2)}
\newcommand{\eref}[1]{Eq.~(\ref{#1})}
\newcommand{\esref}[2]{Eqs.~(\ref{#1}) and (\ref{#2})}
\newcommand{\sref}[1]{section~\ref{#1}}
\newcommand{\srefc}[2]{section~\ref{#1}~(chapter~\ref{#2})}
\newcommand{\cref}[1]{chapter~\ref{#1}}
\newcommand{\Sref}[1]{Section~\ref{#1}}
\newcommand{\Cref}[1]{Chapter~\ref{#1}}
\newcommand{\tref}[1]{table~\ref{#1}}
\newcommand{\Tref}[1]{Table~\ref{#1}}
\newcommand{\aref}[1]{appendix~\ref{#1}}
\newcommand{\bref}[1]{(\ref{#1})}
\newcommand{\brefs}[2]{(\ref{#1})~(section~\ref{#2})}
% optics
\def\3rd{3$^{\text{rd}}$}
\def\5th{5$^{\text{th}}$}


% Editing commands
\renewcommand{\thefigure}{S\arabic{figure}}

%%%%%%%%%%%%%%%%%%%%%%%%%%%%%%%%%%%%%%%%%%%%%%%%%%%%%%%%%%%%%%%%%%%%%%%%%%%%%%%%%%%%%%%%%%%%%%%%%%%%%%%%%%%%%%%%%%

\begin{document}
\title{Supplementary Information: \\Spectral tuning of high-harmonic generation with resonance-gradient dielectric metasurfaces}


\author{Piyush Jangid}
\altaffiliation{Contributed equally}
\affiliation{Nonlinear Physics Center, Research School of Physics, Australian National University, Canberra ACT 2601, Australia}
\author{Felix Ulrich Richter}
\altaffiliation{Contributed equally}
\affiliation{Institute of Bioengineering, \'Ecole Polytechnique F\'ed\'erale de Lausanne (EPFL), Lausanne 1015, Switzerland}
\author{Ming Lun Tseng}
\affiliation{Institute of Electronics, National Yang Ming Chiao Tung University, Hsinchu 300, Taiwan}
\author{Ivan Sinev}
\affiliation{Institute of Bioengineering, \'Ecole Polytechnique F\'ed\'erale de Lausanne (EPFL), Lausanne 1015, Switzerland}
\author{Sergey Kruk}
\affiliation{Nonlinear Physics Center, Research School of Physics, Australian National University, Canberra ACT 2601, Australia}
\author{Hatice Altug}
\email{Corresponding author: hatice.altug@epfl.ch}
\affiliation{Institute of Bioengineering, \'Ecole Polytechnique F\'ed\'erale de Lausanne (EPFL), Lausanne 1015, Switzerland}
\author{Yuri Kivshar}
\email{Corresponding author: yuri.kivshar@anu.edu.au}
\affiliation{Nonlinear Physics Center, Research School of Physics, Australian National University, Canberra ACT 2601, Australia}

%\email{yuri.kivshar@anu.edu.au}

%\date{\today}% It is always \today, today,
             %  but any date may be explicitly specified

\maketitle  

\section{\label{sup_sec:mode_analysis} Resonator mode analysis}
%
% Figure environment removed

As highlighted in the main manuscript, the gradient metasurface exhibits two resonant modes that contribute to the spectrally tunable enhancement of the HHG signal. Fig.~\ref{sup_fig:analysis}a shows the calculated reflectivity map for different scaling factors of the metasurface unit cell for normally incident light polarized along the short axis of the resonator cutout. We choose the scaling factor 1.7 to illustrate the nature of the observed modes. Figs.~\ref{sup_fig:analysis}b and c show the distribution of the electric field enhancement in the unit cell for the wavelengths of 4.245~um and 2.8~um, respectively. The overlayed arrows show the local direction of the in-plane electric field. The field distribution of the main mode demonstrates two characteristic electric field loops that indicate the excitation of two counter-parallel magnetic dipoles in the unit cell at this frequency (illustrated schematically above each of the maps). Similar field distribution is observed for the secondary mode (Fig.~\ref{sup_fig:analysis})c, however, additional e-field minima between the loops indicate that this mode has a higher order than the main one. Lastly, increased reflection in the spectral region between the two modes correspond to the diffraction cut-off.

% Figure environment removed


%%%%%%%%%%%%%%%%%%%%%%%%%%%%%%%%%%%%%%%%%%%%%%%%%%%%%%%%%%%%%%
%\newpage
\section{\label{sup_sec:nanofabrication} Metasurface Fabrication}
The metasurfaces are fabricated on calcium difluoride ($\text{CaF}_2$) substrates featuring minimal absorption losses and low refractive index within the mid-IR spectral range. To enhance the stability of the germanium layer on the $\text{CaF}_2$ substrates, a 5~nm thick layer of silicon oxide ($\text{SiO}_2$) was sputtered as an adhesion layer. Subsequently, a 700~nm layer of germanium was deposited using electron beam evaporation. The desired resonator pattern was defined by utilizing electron beam lithography on spin-coated single-layer PMMA (PMMA 495k A8) films. To facilitate this process, a 25~nm layer of aluminium oxide ($\text{Al}_2\text{O}_3$) serving as hard-mask was deposited through electron beam evaporation and defined by a wet chemical lift-off using commercially available \textit{Remover 1165}. The resonator pattern was then transferred to the underlying germanium layer by fluorine-based dry plasma etching.


%%%%%%%%%%%%%%%%%%%%%%%%%%%%%%%%%%%%%%%%%%%%%%%%%%%%%%%%%%%%%%e
%\newpage
\section{\label{sup_sec:efficiency_cal} Efficiency of harmonics generation}

To get the efficiency of generated harmonics, we obtain the average harmonics power by multiplying average intensity counts per spectrometer integration time with the photon energy. We further multiply the average power by a factor $\gamma$, which accounts for the losses through various channels, e.g., via coupling with optical fibre, via spectrometer's diffraction grating, etc. To estimate $\gamma$, we select the pump at the wavelengths of the \3rd and \5th harmonics, and then take the ratio of the average power obtained via powermeter and via the spectrometer.\\

\noindent Below, we calculate the average power of \3rd ($P_{3}$) and \5th ($P_{5}$) harmonics for distinct incident pump powers $P_\omega$ and pump wavelengths $\lambda_{\omega}$.

\begin{table}[h]
\centering
\setlength\extrarowheight{1pt}
\setlength{\tabcolsep}{10pt} % Default value: 6pt
\begin{tabular}{||c|c||c|c|c|c|c||}
     \hline\hline
     %\rowcolor{Gray}     
     $P_{\omega}$ (mW) &$\lambda_{\omega}$ (nm) &\multicolumn{2}{c|}{$\bar{n}_{3}$ (counts/s)} &$\gamma_3$ &$P_{3}$ (nW) \\[1pt]
     % 
     %\hline
     & &\makecell{(Si detector)} &\makecell{(InGaAs detector)} & & \\[1pt]
     %
     \hline\hline
     %
     46 &2526 &$1.794 \times 10^{6}$ & &$3.76 \times 10^4$ &14.59\\[1pt]
     \hline
     %
     46 &2830 &$1.108 \times 10^{6}$ & &$14.84 \times 10^4$ &34.62\\[1pt]
     \hline
     %
     46 &4034 & &$1.855 \times 10^{3}$ &$10.06 \times 10^7$ &27.57\\[1pt]
     \hline\hline
\end{tabular}
\caption{\textbf{\3rd harmonics calculations.} Efficiency = $P_3/P_{\omega}$. $\bar{n}_{3}$ = average \3rd harmonics intensity counts per integration time.}
\label{sup_table:3rd_harmonics}

\vspace{0.8cm}

\begin{tabular}{||c|c||c|c|c|c|c||}
     \hline\hline
     %\rowcolor{Gray}
     $P_{\omega}$ (mW) &$\lambda_{\omega}$ (nm) &\makecell{$\bar{n}_{5}$ (counts/s) \\(Si detector)} &$\gamma_5$ &$P_{5}$ (pW) \\[1pt]
     %
     \hline\hline
     46 &2830 &$3.620 \times 10^{2}$ &$2.51 \times 10^5$ &31.90\\[1pt]
     \hline
     %
     %
     46 &3250 &$7.703 \times 10^{2}$ &$1.60 \times 10^5$ &37.66\\[1pt]
     \hline
     %
     95.4 &3250 &$3.043 \times 10^{3}$ &$1.60 \times 10^5$ &148.79\\[1pt]
     \hline
     %
     46 &4034 &$1.636 \times 10^{3}$ &$9.37 \times 10^5$ & 37.75\\[1pt]
     \hline\hline
\end{tabular}
\caption{\textbf{\5th harmonics calculations.} Efficiency = $P_5/P_{\omega}$. $\bar{n}_{5}$ = average \5th harmonics intensity counts per integration time.}
\label{sup_table:5th_harmonics}
\end{table}

\end{document}