\documentclass[review]{elsarticle}

\usepackage{lineno,hyperref,mathtools}
\usepackage[letterpaper,top=2cm,bottom=2cm,left=3cm,right=3cm,marginparwidth=1.75cm]{geometry}

\journal{Journal of \LaTeX\ Templates}

\usepackage{amssymb}
\usepackage{bm}
\usepackage{amssymb}
\usepackage{amsthm}
\usepackage{multirow,booktabs}
\usepackage{cases}
\usepackage{threeparttable}


\newcommand{\F}{\mathbb{F}}
\newcommand{\Z}{\mathbb{Z}}
\newcommand{\C}{\mathcal{C}}
\newcommand{\lcm}{\mathrm{lcm}}
\newcommand{\ord}{\mathrm{ord}}
\newcommand{\MinRep}{\mathrm{MinRep}}

\bibliographystyle{elsarticle-num}
%%%%%%%%%%%%%%%%%%%%%%%
\newtheorem*{question}{{Question}}
\newtheorem{theorem}{Theorem}
\newtheorem{proposition}[theorem]{Proposition}
\newtheorem{remark}{Remark}
\newtheorem{definition}[theorem]{Definition}
\newtheorem{lemma}[theorem]{Lemma}
\newtheorem{corollary}[theorem]{Corollary}
\newtheorem{example}[theorem]{Example}
\begin{document}
	
	\begin{frontmatter}
		
\title{Two types of negacyclic BCH codes}
\tnotetext[mytitlenote]{}

\author[mymainaddress]{Yanhui Zhang\corref{mycorrespondingauthor}}
\cortext[mycorrespondingauthor]{Corresponding author}
\ead{zhangyanhui0327@163.com}
\address[mymainaddress]{School of Mathematics, Hefei University of Technology, Hefei 230009, China}
		\begin{abstract}
			Negacyclic BCH codes are an important subclass of negacyclic codes, which have efficient encoding and decoding algorithms, but their parameters are difficult to determine. In this paper, we mainly study two types of negacyclic BCH codes of length $n=\frac{q^{m}-1}{4},\frac{q^{m}+1}{4}$. As byproducts, we investigate the first three largest odd coset leaders modulo $n$. The parameters of two types of negacyclic BCH codes are analysed with small and large dimensions, and the weight distribution of neagcyclic BCH codes of length $n=\frac{q^m-1}{4}$ are determined for designed distance in some ranges.
		\end{abstract}
		
		\begin{keyword}
			Negacyclic codes \sep BCH codes \sep Cyclotomic cosets \sep Coset leaders		
		\end{keyword}
	\end{frontmatter}
	\section{Introduction}
	Throughout this paper, let $q$ be an odd prime power, we denote a finite field with $q$ elements as $\F_{q}$. An $[n,k]$ linear code over $\F_q$ is a linear subspace of $\F_q^{n}$ with dimension $k$. If $(c_{0}, c_{1},\ldots, c_{n-1}) \in \C$ implies $(\lambda c_{n-1}, c_{0}, c_{1},\ldots, c_{n-2})\in \C$, then $\C$ is called a $\lambda-$constacyclic code, where $\lambda\in \F_{q}^{*}$. Let $R=\F_{q}/(x^n-\lambda)$ be a residue class ring, and identify any vector $(c_{0}, c_{1},\ldots, c_{n-1})\in \F_q^{n}$ with
	\begin{center}
		$c_{0}+c_{1}x+c_{2}x^{2}+\cdots+c_{n-1}x^{n-1}\in \F_q[x]/ ( x^{n}-\lambda)$,
	\end{center}
	then a $\lambda-$constacyclic code $\C$ over $\F_q$ is an ideal of $R$. If $\lambda=1$, the code $\C$ is called a cyclic code, and the code $\C$ is called a negacyclic code for $\lambda=-1$. Note that $R$ is a principal ideal ring, this implies that there is the smallest degree monic polynomial $g(x)$ such that $\C$=$\langle g(x) \rangle$. Furthermore, $g(x)|(x^{n}-\lambda)$, $g(x)$ and $h(x) = (x^{n}-\lambda)/g(x)$ are called the generator polynomial and check polynomial of $\C$, respectively. 
	
	Suppose $n$ is a positive integer, and $m=ord_{2n}(q)$ is the order of $q$ modulo $2n$, where $gcd(n,q)=1$. Put $\F_{q^m}^*=\langle\alpha\rangle$ and $\beta= \alpha^{\frac{q^{m}-1}{2n}}$, then $\beta$ is a primitive $2n$-th root of unity in $\F_{q^m}$. $M_{\beta^{{1+2i}}}(x)$ is denoted as the minimal polynomial of $\beta ^{1+2i}$ over $\F_q$, where $0\leq i\leq n-1$. For positive integers $\delta$ and $b$, define	
	\begin{center}
		$g_{(n,q,\delta,b)}(x):=\lcm \left(M_{\beta^{{1+2b}}}(x),M_{\beta^{{1+2(b+1)}}}(x),\ldots,M_{\beta^{{1+2(b+\delta-2)}}}(x)\right)$,
	\end{center}
	where $2\leq \delta \leq n$, $\lcm$ denotes the least common multiple of these $M_{\beta^{{1+2i}}}(x)$, $b\leq i\leq b+\delta-2$. 
	Denote $\C_{(n,q,\delta,b)}$ is the negacyclic BCH code with generator polynomial $g_{(n,q,\delta,b)}$. If $b=0$, we denote it by $\C_{(n,q,\delta)}$. Furthermore, $d(\C_{(n,q,\delta)})$ and $dim(\C_{(n,q,\delta)})$ are denoted as the minimum distance and dimension of $\C_{(n,q,\delta)}$, respectively.
	
	BCH codes were first introduced by Hocquenghem in \cite{RefJ10}, and were extended to general finite fields in \cite{RefJ12} by Gorenstein and Zierler. In the last few decades, cyclic BCH codes have been widely investigated in \cite{RefJ1}, \cite{RefJ2}, \cite{RefJ14}, \cite{RefJ5}-\cite{RefJ9} and \cite{RefJ18}-\cite{RefJ20}. Recently, many researchers have developed a strong interest in the study of negacyclic BCH codes.
	
	Negacyclic codes were proposed by Berkelamp in \cite{RefJ3}, \cite{RefJ4}, Krishna and Sarwate found that negacyclic codes over finite fields can product optimal linear codes in \cite{RefJ15}. Since then, many quantum codes from negacyclic BCH codes have been constructed by Kai et al. in \cite{RefJ16}, \cite{RefJ17}. Recently, Pang et al. studied three classes of negacyclic BCH codes with large dimensions in \cite{RefJ21}. At the same time, Zhu et al. constructed quantum codes with good parameters from the negacyclic BCH codes of length $n=\frac{q^{2m}-1}{q-1}$ in \cite{RefJ28}. In \cite{RefJ13}, Guo et al. investigated a family of $q^2$-ary narrow-sense and non-narrow-sense negacyclic BCH codes with length $n=\frac{q^{2m}-1}{2}$. In \cite{RefJ23}, Wang et al. investigated two families of negacyclic BCH codes with length $n=\frac{q^{m}-1}{2}$, $\frac{q^{m}+1}{2}$ and presented their parameters. 
	
	Inspired by the work above, we will study two types of negacyclic BCH codes of length $n=\frac{q^{m}-1}{4}$, $\frac{q^{m}+1}{4}$ in this paper. Several types of negacyclic BCH with good parameters are presented, and many optimal linear codes are settled. The structure of this article is as follows: In section 2, we give some preliminaries. In section 3, we study the negacyclic BCH codes $\C_{(\frac{q^m-1}{4},q,\delta)}$ with small and large dimensions, and the minimum distance of several negacyclic BCH codes are settled. In addition, the weight distribution of $\C_{(\frac{q^m-1}{4},q,\delta)}$ is determined for designed distance in some ranges. In section 4, we study the codes $\C_{(\frac{q^m+1}{4},q,\delta)}$ with small and large dimensions, and investigate the minimum distance for $\delta=2$. In section 5, we make a conclusion for this paper.
	\section{Preliminaries}
	In this section, we present some basic concepts and known results that will be used later in this paper.
	\subsection{Basic notations}
	For any $s$ with $0\leq s\leq n-1$, the $q$-cyclotomic coset of representative $s$ modulo $n$ is referred to as
	\[C_{s}^{n}=\{sq^i\pmod{n}: \ 0\leq i\leq l_s-1\},\]
	where $l_s$ is the smallest integer such that $s\equiv sq^{l_s}\pmod n$, and is denoted by $|C^{n}_{s}|$. We assume $CL(s):=\min\{i:\ i\in C_{s}^{n}\}$, and $CL(s)$ is called the coset leader of $C_{s}^{n}$, define $\MinRep_{n}:=\{CL(s): \ 0\leq s\leq n-1\}$. In addition, we denote $sq^i\pmod{n}$ by $[sq^{i}]_{n}$ for integer $0\leq i\leq l_s-1$.
	
	For a positive integer $n$ such that $gcd(n,q)=1$, let $\alpha$ be a generator of $\F_{q^m}^{*}$ and $m=ord_n(q)$. Denote $\gamma= \alpha^{\frac{q^{m}-1}{n}}$, then $\gamma$ is a primitive $n$-th root of unity in $\F_{q^m}$. Define	
	\begin{center}
		$g^{'}(x):=\lcm \left(m_{1}(x),m_{2}(x),\ldots,m_{\delta -1}(x)\right)$,
	\end{center}where $2\leq \delta \leq n$ and $m_{i}(x)$ denote the minimal polynomial of $\gamma ^{i}$ over $\F_q$. Denote $\C^{'}_{(n,q,\delta,1)}=\langle g^{'}(x)\rangle$,
	then $\C^{'}_{(n,q,\delta,1)}$ is called a cyclic BCH code.	Furthermore, we denote $\underbrace{A,A,\ldots,A}_{i\ times}$ as $\underbrace{A,\ldots}_{i}$ and $[u,v]:=\{u,u+1,\ldots,v\}$, where $u,v$ are integers with $u\leq v$.
	\subsection{Known results on BCH codes}
	We list some known conclusions, the following Lemmas will need in Section 3, 4.
	\begin{lemma}(\cite{RefJ15})\label{l4}
		Let $\C$ be a negacyclic code of length $n$ with generator polynomial $g(x)$ and $\beta$ be a primitive $2n$-th root of unity. If there are integers $k,b,\delta$ with $gcd(k,n)=1$ and $2\leq \delta\leq n$ such that
		$$g(\beta^{1+2kb})=g(\beta^{1+2k(b+1)})=\cdots=g(\beta^{1+2k(b+\delta-2)})=0,$$
		then $d(\C)\geq \delta$.
	\end{lemma}
	\begin{lemma}(\cite{RefJ9})\label{l5}
		Let $\C$ be a $q$-ary $[n,k,d]$ code, where $d$ is an even integer. Then
		$$\sum_{i=0}^{\frac{d-2}{2}}\begin{pmatrix}
			n-1\\
			i
		\end{pmatrix}(q-1)^i\leq q^{n-1-k}.$$
	\end{lemma}
	\begin{lemma}(\cite{RefJ27})\label{l1}
		Let $n=\frac{q^{m}-1}{\lambda}$, where $m=2h\geq 4$ and $\lambda|(q-1)$, then
		\begin{itemize}
			\item[(1)] If $1\leq i\leq \frac{q^{h+1}-1}{\lambda}$ with $i\not\equiv0\pmod{q}$. Let $f(a,b,c)=aq^{h}+\frac{b(q^{h}-1)}{\lambda}+c$, where $a,b,c$ are integers. Then $i\notin \MinRep_{n}$ if and only if $i\in \Delta_1\cup \Delta_2\cup \Delta_3$, where \begin{center}
				$\Delta_1=\{f(a,0,c):1\leq c<a\leq \frac{q-1}{\lambda}\},$\\
				$\Delta_2=\{f(a,b,c):1\leq c\leq a<\frac{q-1}{\lambda},1\leq b<\lambda\},$\\
				$\Delta_3=\{f(a,b,a+1):0\leq a<\frac{q-1}{\lambda},\frac{\lambda}{2}<b<\lambda\}.$	\end{center}
			\item[(2)] If $2\mid\lambda$, let $\Delta=\{\frac{c(q^h+1)}{2}:1\leq c\leq \frac{2(q-1)}{\lambda}\}$ and $1\leq i\leq \frac{q^{h+1}-1}{\lambda}$. Then
			$$|C_{i}^{n}|=\begin{cases}
				h,& \ \text{if}\ i\in \Delta;\\
				m,&\ \text{otherwise}.
			\end{cases}$$ 
		\end{itemize}
	\end{lemma}
	\begin{lemma}(\cite{RefJ22})\label{le6}
		Let $n=\frac{q^m-1}{2}$ and $1\leq i\leq \lfloor\frac{m+6}{4}\rfloor$, where $q$ is an odd prime power. Then the $i-$th largest coset leader modulo $n$ is
		$$\delta^{'}_{i}=\frac{q^m-q^{m-1}-q^{\lfloor\frac{m-3}{2}+i\rfloor}-1}{2}.$$
		Moreover, \begin{center}
			$|C_{\delta_{i}^{'}}^{n}|=$$\begin{cases}
				\frac{m}{2},&\ \text{if}\ 2\mid m\ and\ i=1;\\
				m,&\ \text{otherwise}.
			\end{cases}$
		\end{center}
	\end{lemma}
	\begin{lemma}(\cite{RefJ25},\cite{RefJ26})\label{le8}
		Let $n=\frac{q^{m}+1}{2}$, where $q$ is an odd prime power and $m=2h+1\geq 5$. 
		\begin{itemize}
			\item[(1)]Let $f(a,b,c)=\frac{q^{h+1}+1}{2}+a\frac{q^{h}-1}{2}+bq^{h}+c$ and $g(a,b,c,d)=\frac{q^{h+1}-1}{2}+a\frac{q^{h}+1}{2}+bq^{h}+c$, where $a,b,c,d$ are integers.
			If $1\leq i< \frac{2q^{h+1}-2q+1}{2}$ with $i\not\equiv0\pmod{q}$, then $i$ is a coset leader modulo $\frac{q^{m}+1}{2}$ except $i\in X_{1}\cup X_{2}\cup X_{3}$ and $|C_{i}^{n}| =2m$, where
			\begin{center}
				If $2\mid h$,
				$\begin{cases}
					X_{1}=\lbrace f(0,0,c): -\frac{q}{2} \leq c\leq \frac{q-2}{2}  \rbrace;\\
					X_{2}=\lbrace f(0,b,0):1\leq b\leq \frac{q-1}{2} \rbrace;\\	
					X_{3}=\lbrace f(2,b,0):0 \leq b\leq \frac{q-3}{2}\rbrace.
				\end{cases}$If $2\nmid h$, $\begin{cases}
					X_{1}=\lbrace g(0,0,c):-\frac{q-2}{2}\leq c\leq \frac{q}{2}  \rbrace;\\		
					X_{2}=\lbrace g(0,b,0):1\leq b\leq \frac{q-1}{2}  \rbrace;\\
					X_{3}=\lbrace g(2,b,0):0 \leq b\leq \frac{q-3}{2} \rbrace.
				\end{cases}$
			\end{center}
			\item[(2)]If $q\equiv3\pmod{4}$, then the first three largest coset leaders modulo $n$ are: $$\delta_{1}^{'}=\frac{q^{m}+1}{4},\ \delta_{2}^{'}=\frac{q^{m}-1}{4}-\frac{q^{m-1}}{2},\ 
			\delta_{3}^{'}=\frac{q^{m}+1}{4}-\frac{q^{m-1}+q}{2}.$$ 
			Moreover, $|C_{\delta_{1}^{'}}^{n}|=1$ and $|C_{\delta_{2}^{'}}^{n}|=|C_{\delta_{3}^{'}}^{n}|=2m$.
		\end{itemize}
		
	\end{lemma}
	\begin{lemma}(\cite{RefJ24},\cite{RefJ29})\label{l20}
		Let $n=\frac{q^{m}+1}{\lambda}$, where $2\nmid q$ and $\lambda\mid (q+1)$.
		\begin{itemize}
			\item[(1)]If $\lambda=1$, let $i$, $h$ and $l$ be integers such that $1\leq i< m$, 	$-\frac{l(q^{m-i}-1)}{q^{i}+1}<h<\frac{l(q^{m-i}+1)}{q^{i}-1}$, $1\leq l\leq \frac{q^{i}-1}{2}$. Then $a$ is a coset leader modulo $n$ with $0\leq a\leq q^{m}$ if and only if $a\leq \frac{q^{m}+1}{2}$ and $a\neq lq^{m-i}+h$.
			\item[(2)]If $1\leq i\leq n-1$, then $i$ is a coset leader of $C_{s}^{n}$ modulo $n$ if and only if $\lambda i$ is a coset leader of $C_{\lambda i}^{\lambda n}$ modulo $q^{m}+1$. Moreover, $|C_{i}^{n}|=|C_{\lambda i}^{\lambda n}|$.
			
		\end{itemize}
	\end{lemma}
	\section{The case of $n=\frac{q^m-1}{4}$}
	In this section, we always assume $n=\frac{q^m-1}{4}$ is a positive integer. Then we can divide it into two cases: $q\equiv 1\pmod4$ with $m$ is an integer or $q\equiv 3\pmod4$ with $m$ is an even integer. 
	
	In the following, we first investigate the parameters of negacyclic BCH codes with large dimensions. 
	\begin{lemma}\label{l2}
		Let $n=\frac{q^m-1}{2}$ and $1\leq i\leq \frac{q^{\frac{m+2}{2}}-1}{2}$ is an odd integer with $i\not\equiv0\pmod{q}$, where $m=2h\geq 4$. Then $i\notin \MinRep_{n}$ if and only if $i\in T_1\cup T_2\cup T_3$, where
		\begin{itemize} 
			\item[(1)]If $q\equiv1\pmod4$ or $q\equiv3\pmod4 \ \text{with} \ m\equiv0 \pmod4$, then
			$$\begin{aligned}
				&T_1=\{(2u+1)q^h+2v,(2u+1)q^h+2v+\frac{q^h-1}{2}:1\leq v\leq u<\frac{q-3}{4}\},\\
				&T_2=\{2uq^h+2v+1,2uq^h+2v+\frac{q^h+1}{2}:0\leq v< u<\frac{q-1}{4}\},\\
				&T_3=\begin{cases}
					\{\frac{(q-1)q^h}{2}+2v+1:0\leq v< \frac{q-1}{4}\},&\ \text{if}\ q\equiv1\pmod4;\\
					\{\frac{(q-1)q^h}{2}+2v:1\leq v\leq \frac{q-3}{4}\},&\ \text{if}\ q\equiv3\pmod4 \ \text{with}\ m\equiv0 \pmod4.
				\end{cases}
			\end{aligned}$$
			\item[(2)]If $q\equiv3\pmod4$ with $m\equiv2\pmod 4$, then
			$$\begin{aligned}
				&T_1=\{(2u+1)q^h+2v+\frac{q^h+1}{2}:0\leq v\leq u<\frac{q-3}{4}\},\\
				&T_2=\{2uq^h+2v+1:0\leq v< u\leq \frac{q-3}{4}\},\\
				&T_3=\{(2u+1)q^h+2v,2uq^h+2v+\frac{q^h-1}{2}:1\leq v\leq u\leq\frac{q-3}{4}\}.
			\end{aligned}$$
		\end{itemize}
		\begin{proof}
			Let $1\leq i\leq \frac{q^{h+1}-1}{2}$ with $i\not\equiv0\pmod{q}$, we have $i\notin \MinRep_{n}$ if and only if 
			\begin{equation}\label{e1}
				\begin{cases}
					i=aq^h+c,&\ 1\leq c<a\leq \frac{q-1}{2};\\
					i=aq^h+c+\frac{q^h-1}{2},&\ 1\leq c\leq a<\frac{q-1}{2}.
				\end{cases}
			\end{equation}by Lemma \ref{l1}.
			\\$\bf{Case 1.}$ If $q\equiv1\pmod 4$ or $q\equiv3\pmod4$ with $m\equiv0\pmod 4$, then $2\mid \frac{q^{h}-1}{2}$, i.e., we only need to consider the case of $2\nmid (aq^h+c)$.
			\begin{itemize}
				\item[1)]
				If $2\nmid a$, then $2\mid c$. We can assume that $a=2u+1$ and $c=2v$, then we have 
				\begin{center}
					$\begin{cases}
						1\leq 2v<2u+1\leq \frac{q-1}{2};\\
						1\leq 2v\leq 2u+1<\frac{q-1}{2},
					\end{cases}$ $\Rightarrow$
					$\begin{cases}
						1\leq v\leq u\leq \frac{q-3}{4};\\
						1\leq v\leq u<\frac{q-3}{4}
					\end{cases}$
				\end{center} by (\ref{e1}), where $u$, $v$ are integers. Thus we have $i\notin \MinRep_{n}$ is odd with  $q\equiv3\pmod 4$ when $i\in T_1\cup T_3$, $i\notin \MinRep_{n}$ is odd with  $q\equiv1\pmod 4$ when $i\in T_1$, where $$\begin{cases}
					T_1=\{(2u+1)q^h+2v,(2u+1)q^h+2v+\frac{q^h-1}{2}:1\leq v\leq u<
					\frac{q-3}{4}\};\\
					T_3=\{\frac{(q-1)q^h}{2}+2v:1\leq v\leq \frac{q-3}{4}\}.
				\end{cases}$$
				\item[2)]
				If $2\mid a$, then $2\nmid c$. We can assume that $a=2u$ and $c=2v+1$, then we have 
				\begin{center}
					$\begin{cases}
						1\leq 2v+1<2u\leq \frac{q-1}{2};\\
						1\leq 2v+1\leq 2u<\frac{q-1}{2},
					\end{cases}$ $\Rightarrow$
					$\begin{cases}
						0\leq v< u\leq \frac{q-1}{4};\\
						0\leq v< u<\frac{q-1}{4}
					\end{cases}$
				\end{center} by (\ref{e1}), where $u$, $v$ are integers. Thus we have $i\notin \MinRep_{n}$ is odd with  $q\equiv3\pmod 4$ when $i\in T_2$, $i\notin \MinRep_{n}$ is odd with  $q\equiv1\pmod 4$ when $i\in T_2\cup T_3$, where $$\begin{cases}
					T_2=\{2uq^h+2v+1,2uq^h+2v+\frac{q^h+1}{2}:0\leq v< u<
					\frac{q-1}{4}\};\\
					T_3=\{\frac{(q-1)q^h}{2}+2v+1:0\leq v< \frac{q-1}{4}\}.
				\end{cases}$$
			\end{itemize}
			$\bf{Case 2.}$ If $q\equiv3$ with $m\equiv2\pmod 4$, then $2\nmid \frac{q^{h}-1}{2}$. We can get the results by the same way of Case 1.
			Thus this completes the proof.
		\end{proof}
	\end{lemma}
	Next we will investigate the parameters of negacyclic BCH codes $\C_{(n,q,\delta)}$ with $2\leq \delta\leq \frac{q^{\frac{m+2}{2}}+5}{4}$, where $m=2h\geq 4$.
	\begin{theorem}\label{th8}
		Let $n=\frac{q^m-1}{4}$ and $m=2h\geq 4$, where $q\equiv1\pmod4$ or $q\equiv3\pmod4$ with $m\equiv0\pmod4$. Then we have the negacyclic BCH code $\C_{(n,q,\delta)}$ has parameters $[n,k,d]$, where \begin{center}
			$\begin{cases}
				d\geq\delta+1,& \ \text{if}\ \delta\equiv\frac{q+1}{2}\pmod q;\\
				d\geq\delta,&\  \text{otherwise}
			\end{cases}$ 
		\end{center}and the dimension $k$ is provided as follows:
		\begin{itemize}
			\item[(1)]If $q\geq 5$ and $2\leq\delta\leq q^{h}+1$, then 
			$$k=\begin{cases}
				n-m\big\lceil \frac{(2\delta-3)(q-1)}{2q}\big\rceil,& \text{if}\ 2\leq \delta\leq \frac{q^h-1}{4}+1;\\
				n-m\big\lceil \frac{(2\delta-3)(q-1)}{2q}\big\rceil+h,& \text{if}\ \frac{q^h-1}{4}+2\leq \delta\leq\frac{3(q^h+3)}{4}-1;\\
				n-m\big\lceil \frac{(2\delta-3)(q-1)}{2q}\big\rceil+m,& \text{if}\ \frac{3(q^h+3)}{4}\leq \delta\leq q^h+1.
			\end{cases}$$
			\item[(2)]If $q=5$ and $q^h+2\leq \delta\leq\frac{q^{h+1}+5}{4}$, then $$k=n-m\big\lceil \frac{(2\delta-3)(q-1)}{2q}\big\rceil+2m.$$
			\item[(3)]If $q>5$, define $\tau$ is an integer with $1\leq \tau<\frac{q-3}{4}$.
			\begin{itemize}	
				\item[3.1)]If $q^h+2\leq \delta\leq(\tau+1) q^h+1$, then
				$$k=\begin{cases}
					n-m\big(\big\lceil \frac{(2\delta-3)(q-1)}{2q}\big\rceil-2\tau^2\big),& \text{if}\ \tau (q^h+1)+1\leq \delta\leq \tau q^h+\frac{q^h+3}{4};\\
					n-m\big(\big\lceil \frac{(2\delta-3)(q-1)}{2q}\big\rceil-(2\tau ^2+\tau)\big)+h,& \text{if}\ \tau (q^h+1)+\frac{q^h+7}{4}\leq \delta\leq \tau q^h+\frac{q^h+3}{2};\\
					n-m\big(\big\lceil \frac{(2\delta-3)(q-1)}{2q}\big\rceil-2(\tau ^2+\tau)\big)+h,& \text{if}\ \tau (q^h+1)+\frac{q^h+3}{2}\leq \delta\leq \tau q^h+\frac{3q^h+5}{4};\\
					n-m\big(\big\lceil \frac{(2\delta-3)(q-1)}{2q}\big\rceil-(2\tau ^2+3\tau +1)\big),& \text{if}\ \tau (q^h+1)+\frac{3q^h+1}{2}+2\leq \delta\leq (\tau+1) q^h+1.
				\end{cases}$$	
				\item[3.2)]If $q\equiv1\pmod4$ and $\frac{q-1}{4}q^h+2\leq\delta\leq \frac{q^{h+1}+5}{4}$, then
				$$k=\begin{cases}
					n-m\big(\big\lceil \frac{(2\delta-3)(q-1)}{2q}\big\rceil-(\delta-\frac{q-1}{4}q^h+\frac{q^2-4q-5}{8})\big),& \text{if}\ \frac{q-1}{4}q^h+2\leq \delta\leq\frac{q-1}{4}(q^h+1)+1;\\
					n-m\big(\big\lceil \frac{(2\delta-3)(q-1)}{2q}\big\rceil-\frac{(q-1)^2}{8}\big),& \text{if}\ \frac{q-1}{4}(q^h+1)+2\leq \delta\leq \frac{q^{h+1}+5}{4}.
				\end{cases}$$
				\item[3.3)]If $q\equiv3\pmod4$ with $m\equiv0\pmod 4$ and $ \frac{(q-3)q^h+q+1}{4}\leq \delta\leq \frac{q^{h+1}+5}{4}$, then
				$$k=\begin{cases}
					n-m\big(\big\lceil \frac{(2\delta-3)(q-1)}{2q}\big\rceil-\frac{(q-3)^2}{8}\big),& \text{if}\ \frac{q^{h+1}-3q^h+q+1}{4}\leq \delta\leq\frac{q^{h+1}-2q^h+3}{4};\\
					n-m\big(\big\lceil \frac{(2\delta-3)(q-1)}{2q}\big\rceil-\frac{(q-1)(q-3)}{8}\big)+h,& \text{if}\ \frac{q^{h+1}-2q^h+q}{4}+1\leq \delta\leq \frac{q^{h+1}-q^h+2}{4}+1;\\
					n-m\big(\big\lceil \frac{(2\delta-3)(q-1)}{2q}\big\rceil-\frac{(q-1)^2}{8}\big),& \text{if}\ \frac{q^{h+1}-q^h+q+3}{4}\leq \delta\leq \frac{q^{h+1}+5}{4}.
				\end{cases}$$	
			\end{itemize}
		\end{itemize}
		\begin{proof}
			It is clear that the generator polynomal of $\C_{(n,q,\delta)}$ is $g(x)=lcm(M_{\beta^{1}}(x),M_{\beta^{3}}(x),\ldots,M_{\beta^{1+2(\delta-2)}}(x))$,  where $\beta$ is a primitive $2n$-th root of unity. Let $\Gamma=\lbrace 1+2i:0\leq i\leq \delta-2,\ \ 1+2i\not\equiv0\pmod q\rbrace$, where $2\leq \delta\leq\frac{q^{h+1}+5}{4}$. Then $1+2i\in \Gamma$ is a coset leader modulo $2n$  except $1+2i\in\cup^{3}_{i=1} T_{i}$ by Lemma \ref{l2}, where $T_i$ is defined in Lemma \ref{l2}. Define $T_0=\{\frac{(2t+1)(q^h+1)}{2}:0\leq t\leq \frac{q-2}{2}\}$, we have
			$$|C_{1+2i}^{2n}|=\begin{cases}
				h,& \text{if}\ 1+2i\in T_0;\\
				m,& \text{if}\ 1+2i\in \Gamma\textbackslash T_0,
			\end{cases}$$ by Lemma \ref{l1}.
			Note that $T_i\cap T_j=\emptyset$ for $0\leq i\neq j\leq 3$, the dimension of the negacyclic BCH code $\C_{(n,q,\delta)}$ is 
			\begin{equation}\label{eq2}
				k=n-m|\Gamma|+m\sum_{i=1}^{3}|\Gamma\cap T_i|+\frac{m}{2}|\Gamma\cap T_0|.
			\end{equation}
			Note that $min\{T_0\}=\frac{q^h+1}{2}$, $min\{T_1,T_2,T_3\}=2q^h+1$ and
			$$\begin{aligned}
				|\Gamma|&=\delta-1-|\{1+2i:0\leq i\leq \delta-2,\ \ 1+2i\equiv0\pmod q\}|\\
				&=\delta-1-(\big\lceil \frac{2\delta-3-q}{2q}\big\rceil+1)=\big\lceil \frac{(2\delta-3)(q-1)}{2q}\big\rceil.
			\end{aligned}$$ 
			
			For $q\geq 5$ and $2\leq\delta\leq q^{h}+1$, we have the following. If $2\leq \delta\leq \frac{q^h-1}{4}+1$, then $2\delta-3\leq \frac{q^h-1}{2}$. We have $\Gamma\cap T_i=\emptyset$ for any $i\in [0,3]$, it follows from (\ref{eq2}) that $$k=n-m\big\lceil \frac{(2\delta-3)(q-1)}{2q}\big\rceil.$$
			By the same way we have $\Gamma\cap T_0=\{\frac{q^h+1}{2}\}$ and $\Gamma\cap (\cup _{i=1}^{3}T_i)=\emptyset$ when $\frac{q^h-1}{4}+2\leq \delta\leq \frac{3(q^h+3)}{4}-1$, $\Gamma\cap T_0=\{\frac{q^h+1}{2},\frac{3(q^h+1)}{2}\}$ and $\Gamma\cap (\cup _{i=1}^{3}T_i)=\emptyset$ when $\frac{3(q^h+3)}{4}\leq \delta\leq q^h+1$, then the value of $k$ can be got by (\ref{eq2}).
			
			For $q=5$ and $q^h+2\leq \delta\leq\frac{q^{h+1}+5}{4}$, we have $\Gamma\cap T_0=\{\frac{5^{h}+1}{2},\frac{3(5^h+1)}{2}\}$ and $\Gamma\cap (\cup _{i=1}^{3}T_i)=\{2\cdot5^h+1\}$. It follows from (\ref{eq2}) that  $$k=n-m\big\lceil \frac{(2\delta-3)(q-1)}{2q}\big\rceil+2m.$$
			
			For $q> 5$ and $q^h+2\leq \delta\leq \frac{q^{h+1}+5}{5}$, we only prove the case of $\tau (q^h+1)+1\leq \delta\leq \tau q^h+\frac{q^h+3}{4}$, as the proof of other cases are similar. Note that $2\tau q^h+2\tau -1\leq 2\delta-3\leq 2\tau q^h+\frac{q^h-3}{2}$, then 
			\begin{center}
				$\Gamma\cap T_{3}=\emptyset$, $\Gamma\cap T_{0}=\lbrace \frac{(4t+1)(q^{h}+1)}{2},\  \frac{(4t+3)(q^{h}+1)}{2} : 0\leq t<\tau \rbrace$,\\
				$\Gamma\cap T_{1}=\{(2u+1)q^h+2v,(2u+1)q^h+2v+\frac{q^h-1}{2}:1\leq v\leq u<\tau\}$,\\
				$\Gamma\cap T_{2}=\{2uq^h+2v+1,2uq^h+2v+\frac{q^h+1}{2}:0\leq v< u<\tau\}\cup \{2\tau q^h+2v+1:0\leq v<\tau\}$.	
			\end{center}
			It is clear that $| \Gamma\cap T_{3}|=0$, $|\Gamma\cap T_{0}|=2\tau$, $|\Gamma\cap T_{1}|=\tau(\tau-1)$ and $|\Gamma\cap T_{2}|=\tau^2$, then $k$ can be got by (\ref{eq2}). Thus this completes the proof.
		\end{proof}
	\end{theorem}
	\begin{theorem}\label{th9}
		Let $n=\frac{q^m-1}{4}$ and $m=2h\geq 4$, where $q\equiv3\pmod4$. Then we have the negacyclic BCH codes $C_{(n,q,\delta)}$ has parameters $[n,k,d]$, where \begin{center}
			$\begin{cases}
				d\geq \delta+1, \ &\text{if}\ \delta\equiv\frac{q+1}{2}\pmod q;\\
				d\geq \delta,\  &\text{otherwise}
			\end{cases}$ 
		\end{center}and the dimension $k$ is provided as follows:
		\begin{itemize}
			\item[(1)]If $q=3$ and $2\leq\delta\leq \frac{q^{h+1}+5}{4}$, then 
			\begin{itemize}
				\item[1.1)]If $m\equiv2\pmod4$, then $$k=n-m\big\lceil \frac{(2\delta-3)(q-1)}{2q}\big\rceil.$$
				\item[1.2)]If $m\equiv0\pmod4$, then
				$$k=\begin{cases}
					n-m\big\lceil \frac{(2\delta-3)(q-1)}{2q}\big\rceil,&\ \text{if}\ 2\leq \delta\leq \frac{q^h+3}{4};\\
					n-m\big\lceil \frac{(2\delta-3)(q-1)}{2q}\big\rceil+h,&\ \text{if}\ \frac{q^h+3}{4}+1\leq \delta\leq\frac{q^{h+1}+5}{4}.
				\end{cases}$$	
			\end{itemize}
			
			\item[(2)]If $q>3$ and $m\equiv2\pmod4$, then 
			\begin{itemize}
				\item[2.1)]If $2\leq\delta\leq\frac{3(q^h+1)}{4}$, then $$k=n-m\big\lceil \frac{(2\delta-3)(q-1)}{2q}\big\rceil.$$
				\item[2.2)]If $\frac{3(q^h+1)}{4}+1\leq \delta\leq\frac{q^{h+1}+5}{4}$, define $\tau$ is an integer with $1\leq \tau\leq\frac{q-3}{4}$. Then
				$$k=\begin{cases}
					n-m\big(\big\lceil \frac{(2\delta-3)(q-1)}{2q}\big\rceil-(2\tau^2-\tau)\big),&\ \text{if}\ \tau (q^h+1)+\frac{3-q^h}{4}\leq \delta\leq \tau q^h+1;\\
					n-m\big(\big\lceil \frac{(2\delta-3)(q-1)}{2q}\big\rceil-2\tau ^2\big),&\ \text{if}\ \tau (q^h+1)+1\leq \delta\leq \tau q^h+\frac{q^h+5}{4};\\
					n-m\big(\big\lceil \frac{(2\delta-3)(q-1)}{2q}\big\rceil-(2\tau ^2+\tau )\big),&\ \text{if}\ \tau (q^h+1)+\frac{q^h+5}{4}\leq \delta\leq \tau q^h+\frac{q^h+3}{2};\\
					n-m\big(\big\lceil \frac{(2\delta-3)(q-1)}{2q}\big\rceil-2(\tau ^2+\tau )\big),&\ \text{if}\ \tau (q^h+1)+\frac{q^h+3}{2}\leq \delta\leq \tau q^h+\frac{3(q^h+1)}{4}\ \text{and $\tau\neq \frac{q-3}{4}$};\\
					n-m\big(\big\lceil \frac{(2\delta-3)(q-1)}{2q}\big\rceil-\frac{(q-3)(q+1)}{8}\big),&\ \text{if}\ \frac{(q-1)q^h}{4}+\frac{q+3}{4}\leq \delta\leq \frac{q^{h+1}+5}{4}.
				\end{cases}$$	
			\end{itemize}
		\end{itemize}
		\begin{proof}
			From Lemmas \ref{l1} and \ref{l2}, we can obtain the results by the same way of Theorem \ref{th8}.
		\end{proof}
	\end{theorem}
	Next we will investigate the parameters of negacyclic BCH codes $\C_{(n,q,\delta)}$ with $2\leq \delta\leq \frac{3q^{\frac{m+1}{2}}-2q^{\frac{m-1}{2}}+3}{4}$, where $m=2h+1\geq 5$
	\begin{lemma}\label{l3}
		Let $n=\frac{q^m-1}{2}$ and $1\leq i\leq \frac{q^\frac{m+3}{2}-1}{2}$ with $i\not\equiv0\pmod q$, where $m=2h+1\geq 5$. Then $i\in \MinRep_{n}$ and $|C_{i}^{n}|=m$ except that $i\in A_1\cup A_2\cup A_3\cup A_4\cup A_5\cup A_6$, where
		\begin{center}
			$A_1=\{i_{h+1}q^{h+1}+i_1q+i_0:0\leq i_1<i_{h+1}\leq \frac{q-1}{2}, \ 1\leq i_0\leq q-1\},$\\
			$A_2=\{i_{h+1}q^{h+1}+i_1q+i_0+\frac{q-1}{2}\sum_{l=2}^{h}q^l:1\leq i_1-\frac{q-1}{2}\leq i_{h+1}< \frac{q-1}{2}, \ 1\leq i_0\leq q-1\},$\\
			$A_3=\{i_{h+1}q^{h+1}+i_0+\frac{q-1}{2}\sum_{l=1}^{h}q^l:0\leq i_{h+1}< \frac{q-1}{2} , \ \frac{q-1}{2}< i_0\leq q-1\},$\\
			$A_4=\{i_{h+1}q^{h+1}+i_hq^h+i_0:1\leq i_0\leq i_{h+1}<\frac{q-1}{2}, \ 1\leq i_h\leq q-1\},$\\
			$A_5=\{\frac{q-1}{2}q^{h+1}+i_hq^h+i_0:1\leq i_0,i_h\leq\frac{q-1}{2} \},$\\
			$A_6=\{i_{h+1}q^{h+1}+i_hq^h+i_0+\frac{q-1}{2}\sum_{l=1}^{h-1}q^l:1\leq i_0-\frac{q-1}{2}\leq i_{h+1}\leq \frac{q-1}{2}\ \text{with}\ 0\leq i_h<\frac{q-1}{2},$\\ or$\ 0\leq i_0-\frac{q-1}{2}-1\leq i_{h+1}<\frac{q-1}{2}\ \text{with}\ \frac{q-1}{2}< i_h\leq q-1\}.$
		\end{center}
		\begin{proof}We first to discuss the value of $|C_{i}^{n}|$, where $1\leq i\leq \frac{q^\frac{m+3}{2}-1}{2}$. For $m=5$ or 7, we have $|C_{i}^{n}|=1$ or $m$. Note that $i<iq<n$, then $|C_{i}^{n}|\neq1$, i.e., $|C_{i}^{n}|=m$. For $m\geq9$, then $ \frac{m}{3}\geq |C_{i}^{n}|$ by $|C_{i}^{n}|\mid m$. Note that $i<iq^k<n$ for all $0<k\leq \frac{m}{3}$, then $|C_{i}^{n}|=m$.
			
			To investigate all coset leaders $i$ such that $1\leq i\leq \frac{q^\frac{m+3}{2}-1}{2}$ with $i\not\equiv0\pmod q$, we need to solve the equation \begin{equation}\label{eq3}
				iq^k\equiv j\pmod n,
			\end{equation} where $1\leq j<i\leq \frac{q^\frac{m+3}{2}-1}{2}$, $1\leq k\leq 2h$ and $q\nmid i,j$. We denote the $q-$expansions of $i,j$ as  $i=\sum_{l=0}^{h+1}i_{l}q^l$, $j=\sum_{l=0}^{h+1}j_{l}q^l$, where $1\leq i_{0},j_0\leq q-1$, $0\leq i_{h+1},j_{h+1}\leq \frac{q-1}{2}$ and $0\leq i_l,j_l\leq q-1$ for all $l$ with $1\leq l\leq h$.
			\\$\bf{Case 1.}$ For $1\leq k\leq h-1$, then $iq^k\leq \frac{q^{h+2}-1}{2}q^{h-1}<n$, i.e., the Equation (\ref{eq3}) has no solution.
			\\$\bf{Case 2.}$ For $h+2\leq k\leq 2h$, it is easy to see that $$[iq^k]_{2n}=(i_{2h-k},i_{2h-k-1},\ldots,i_0,\underbrace{0,\ldots}_{h-1},i_{h+1},i_{h},\ldots,i_{2h-k+1}).$$
			If $i_{2h-k}>\frac{q-1}{2}$, then
			$$[iq^k]_{n}=[iq^k]_{2n}-n>(0,\underbrace{\frac{q-1}{2},\ldots}_{m-1})\geq \frac{q^{h+2}-1}{2},$$ i.e., the Equation (\ref{eq3}) has no solution.
			\\If $i_{2h-k}<\frac{q-1}{2}$, then $[iq^k]_{n}=[iq^k]_{2n}>\frac{q^{h+2}-1}{2}$, i.e., the Equation (\ref{eq3}) has no solution.
			\\For $i_{2h-k}=\frac{q-1}{2}$, we have the following cases.
			\begin{itemize}
				\item[1)]If $i_l= \frac{q-1}{2}$ for all $0\leq l\leq 2h-k$, then we have $[iq^k]_{n}=[iq^k]_{2n}>\frac{q^{h+2}-1}{2}$, i.e., the Equation (\ref{eq3}) has no solution.
				\item[2)]If there exists an integer $l_1$ such that $i_{l_{1}}\neq\frac{q-1}{2}$ and $i_l=\frac{q-1}{2}$ for all $l_1+1\leq l\leq 2h-k$. We have 
				\begin{center}
					$[iq^k]_{n}=$
					$\begin{cases}
						[iq^k]_{2n}>\frac{q^{h+2}-1}{2},&\ \text{if}\ i_{l_{1}}< \frac{q-1}{2};\\
						[iq^k]_{2n}-n>\frac{q^{h+2}-1}{2},&\ \text{if}\ i_{l_{1}}> \frac{q-1}{2},
					\end{cases}$
				\end{center}i.e., the Equation (\ref{eq3}) has no solution.
			\end{itemize}
			$\bf{Case 3.}$ For $k=h$, we have
			\begin{equation}\label{eq4}
				[iq^k]_{2n}=(i_{h},i_{h-1},\ldots,i_0,\underbrace{0,\ldots}_{h-1},i_{h+1}),
			\end{equation} 
			then we have the following cases.
			\begin{itemize}
				\item[1)] If $i_h>\frac{q-1}{2}$, then $[iq^k]_{n}=[iq^k]_{2n}-n>\frac{q^{h+2}-1}{2}$, i.e., the Equation (\ref{eq3}) has no solution.
				\item[2)] If $i_h<\frac{q-1}{2}$, then $[iq^k]_{n}=[iq^k]_{2n}$. From (\ref{eq3}) and (\ref{eq4}), we have 
				$$i=i_{h+1}q^{h+1}+i_1q+i_0,\ j=i_1q^{h+1}+i_0q^h+i_{h+1}.$$
				Note that $j<i$, then Equation (\ref{eq3}) has a solution $i=i_{h+1}q^h+i_1q+i_0$, where $1\leq i_0\leq q-1$ and $0\leq i_1<i_{h+1}\leq \frac{q-1}{2}$.
				\item[3)] For $i_h=i_{h-1}=\cdots=i_0=\frac{q-1}{2}$ or there exists an integer $l_1$ such that $i_{l_{1}}<\frac{q-1}{2}$ with $i_l=\frac{q-1}{2}$ for all $l_1+1\leq l\leq h$, then $[iq^k]_{n}=[iq^k]_{2n}>\frac{q^{h+2}-1}{2}$, i.e., the Equation (\ref{eq3}) has no solution.
				\item[4)] For there exists an integer $l_1$ such that $i_{l_{1}}>\frac{q-1}{2}$ and $i_l=\frac{q-1}{2}$ for all $l_1+1\leq l\leq h$, we have $i_{h+1}<\frac{q-1}{2}$ and $[iq^k]_{n}=[iq^k]_{2n}-n$.
				\\If $l_{1}\geq 2$, then $$[iq^h]_n>(0,\ldots,0,i_{1}+\frac{q-1}{2},i_0+\frac{q-1}{2},\underbrace{\frac{q-1}{2},\ldots}_{h-1},i_{h+1}+\frac{q-1}{2})>\frac{q^{h+2}-1}{2},$$ i.e., the Equation (\ref{eq3}) has no solution.
				\\If $l_{1}= 1$, we have  $$i=(0,\ldots,0,i_{h+1},\underbrace{\frac{q-1}{2},\ldots}_{h-1},i_1,i_0)$$
				\begin{center}
					$\left\{\begin{aligned}
						&j=[iq^h]_n=(0,\ldots,0,i_{1}-\frac{q-1}{2},i_0-\frac{q-1}{2}-1,\underbrace{\frac{q-1}{2},\ldots}_{h-1},i_{h+1}+\frac{q-1}{2}+1),&\ \text{if} \ i_0> \frac{q-1}{2};\\
						&j=[iq^h]_n=(0,\ldots,0,i_{1}-\frac{q-1}{2}-1,i_0+\frac{q-1}{2},\underbrace{\frac{q-1}{2},\ldots}_{h-1},i_{h+1}+\frac{q-1}{2}+1),&\ \text{if} \ i_0\leq \frac{q-1}{2}.
					\end{aligned} 
					\right.$
				\end{center}
				Note that $j<i$, then Equation (\ref{eq3}) has a solution $i=i_{h+1}q^{h+1}+\frac{q-1}{2}\sum_{l=2}^{h}q^l+i_1q+i_0$, where 
				\begin{center}
					$\begin{cases}
						1\leq i_1-\frac{q-1}{2}\leq i_{h+1}<\frac{q-1}{2},&\ \text{if}\ \frac{q-1}{2}< i_0\leq q-1;\\
						0\leq i_1-\frac{q-1}{2}-1< i_{h+1}<\frac{q-1}{2},&\ \text{if}\ 1\leq i_0\leq\frac{q-1}{2},
					\end{cases}$ $\Rightarrow$ 
					$\begin{cases}
						1\leq i_1-\frac{q-1}{2}\leq i_{h+1}<\frac{q-1}{2},\\ and\ 1\leq i_0\leq q-1.
					\end{cases}$
				\end{center}
				If $l_{1}= 0$, we have
				\begin{center}
					$\left\{\begin{aligned}	&i=(0,\ldots,0,i_{h+1},\frac{q-1}{2},\underbrace{\frac{q-1}{2},\ldots}_{h-1},i_0);\\
						&[iq^h]_n=(0,\ldots,0,0,i_0-\frac{q-1}{2}-1,\underbrace{\frac{q-1}{2},\ldots}_{h-1},i_{h+1}+\frac{q-1}{2}+1).
					\end{aligned} 
					\right.$
				\end{center}
				Note that $j<i$, then Equation (\ref{eq3}) has a solution $i=i_{h+1}q^{h+1}+\frac{q-1}{2}\sum_{l=1}^{h}q^l+i_0$, where $0\leq i_{h+1}<\frac{q-1}{2}$ and $\frac{q-1}{2}< i_0\leq q-1$.
			\end{itemize}
			$\bf{Case 4.}$ If $k=h+1$, then
			\begin{equation}\label{eq5}
				[iq^k]_{2n}=(i_{h-1},i_{h-2},\cdots,i_0,\underbrace{0,0,\cdots,0}_{h-1},i_{h+1},i_{h}).
			\end{equation} By the same way of Case 3., we have the Equation (\ref{eq3}) has solutions $i=i_{h+1}q^{h+1}+i_{h}q^h+i_0$, where $$\begin{cases}
				1\leq i_0\leq i_{h+1}<\frac{q-1}{2},&\ \text{if} \ 1\leq i_h\leq q-1;\\
				1\leq i_0\leq i_{h+1}=\frac{q-1}{2},&\ \text{if}\ 1\leq i_h\leq \frac{q-1}{2}
			\end{cases}$$ and $i=i_{h+1}q^{h+1}+i_hq^h+\frac{q-1}{2}\sum_{l=1}^{h-1}q^l+i_0$, where
			$$\begin{cases}
				1\leq i_0-\frac{q-1}{2}\leq i_{h+1}\leq \frac{q-1}{2},\ &\ \text{if}\ 0\leq i_h<\frac{q-1}{2};\\ \ 0\leq i_0-\frac{q-1}{2}-1\leq i_{h+1}<\frac{q-1}{2},\ &\ \text{if}\ \frac{q-1}{2}< i_h\leq q-1.
			\end{cases}$$ Thus this completes the proof.
		\end{proof}
	\end{lemma}
	\begin{theorem}\label{th11}
		Let $n=\frac{q^m-1}{2}$ and $\epsilon=\lceil (\delta-1)(1-q^{-1})\rceil$, where $q$ is an odd prime power and $m=2h+1\geq 5$. Then the cyclic BCH code $\C^{'}_{(q,n,\delta,1)}$ have parameters $[\frac{q^m-1}{2},k,d\geq \delta]$, where the dimension $k$ is provided as follows:
		\begin{itemize}
			\item [(1)] If $2\leq \delta\leq q^{h+1}+\frac{q-1}{2}q^h+1$, define $\tau$ is an integer with $0\leq \tau\leq \frac{q-3}{2}$. Then 
			\begin{center}
				$k=$
				$\begin{cases}
					n-m\epsilon,& \text{if}\ 2\leq \delta\leq\frac{q^{h+1}+1}{2};\\
					n-m(\epsilon-\frac{q-1}{2}),& \text{if}\ \frac{q^{h+1}+q}{2}\leq \delta\leq\frac{q^{h+1}+1}{2}+q^h;\\
					n-m(\epsilon-\frac{q-1}{2}-\tau),& \text{if}\ \frac{q^{h+1}+1}{2}+\tau q^h+1\leq \delta\leq\frac{q^{h+1}+1}{2}+(\tau+1)q^h,\tau\neq 0;\\
					n-m(\epsilon-q+1),& \text{if}\ q^{h+1}-\frac{q^h-1}{2}+1\leq \delta\leq q^{h+1}+1;\\
					n-m(\epsilon-2(q-1)),& \text{if}\ q^{h+1}+q\leq \delta\leq q^{h+1}+\frac{q^h+1}{2};\\
					n-m(\epsilon-2(q-1)-(2\tau+1)),& \text{if}\ q^{h+1}+\frac{q^h+1}{2}+\tau q^h+1\leq \delta\leq q^{h+1}+(\tau+1)q^h+1;\\
					n-m(\epsilon-2(q-1)-2\tau),& \text{if}\ q^{h+1}+\tau q^h+2\leq \delta\leq q^{h+1}+\frac{q^h+1}{2}+\tau q^h,\tau\neq 0.
				\end{cases}$
			\end{center}
			\item [(2)] If $q=3$ and $q^{h+1}+\frac{q-1}{2}q^h+2\leq \delta\leq \frac{q^{h+2}+1}{2}$, then $k=n-m(\epsilon-6)$.
			\item [(3)] If $q\geq 5$ and $q^{h+1}+\frac{q-1}{2}q^h+2\leq \delta\leq2q^{h+1}-q^h+1$, let $\iota $ is an integer with $\frac{q+1}{2}\leq \iota \leq q-2$. Then \begin{center}
				$k=$
				$\begin{cases}
					n-m(\epsilon-3(q-1)),& \text{if}\ q^{h+1}+\frac{q-1}{2}q^h+2\leq \delta\leq q^{h+1}+\frac{q^{h+1}+1}{2};\\
					n-m(\epsilon-\frac{7(q-1)}{2}),& \text{if}\ q^{h+1}+\frac{q^{h+1}+q}{2}\leq \delta\leq q^{h+1}+\frac{(q+1)q^h}{2}+1;\\
					n-m(\epsilon-3(q-1)-\iota),& \text{if}\ q^{h+1}+\iota q^h+2\leq \delta\leq q^{h+1}+(\iota+1)q^h+1.
				\end{cases}$\end{center}
		\end{itemize}
		\begin{proof}
			It is clear that the generator polynomal of $\C^{'}_{(n,q,\delta,1)}$ is $g(x)=lcm(M_{\gamma^{1}}(x),M_{\gamma^{2}}(x),\ldots,M_{\gamma^{\delta-1}}(x))$,  where $\gamma$ is a primitive $n$-th root of unity. Let $\Gamma=\lbrace i:1\leq i\leq \delta-1,\ \ i\not\equiv 0\pmod q\rbrace$, then $|\Gamma|=\lceil (\delta-1)(1-q^{-1})\rceil$, where $2\leq \delta\leq \frac{q^{h+2}+3}{2}$. From Lemma \ref{l3}, we have $i\in \Gamma$ is a coset leader modulo $n$ except $i\in \cup_{l=1}^{6} A_{l}$ and $|C_{i}^{n}|=m$,  where $A_{l}$ are given by Lemma \ref{l3}. Note that $A_{i}\cap A_{j}=\emptyset$ for $1\leq i\neq j\leq 6$, then
			\begin{center}
				$k=n-m\mid \Gamma \mid+m\sum_{l=1}^{6}\mid \Gamma\cap A_{l}\mid.$
			\end{center}
			We can prove the results by the same way of Theorem \ref{th8}.
		\end{proof}
	\end{theorem}
	\begin{theorem}\label{th12}
		Let $n=\frac{q^m-1}{4}$ and $m=2h+1\geq 5$, where $q\equiv1\pmod4$. Then the negacyclic BCH code $\C_{(n,q,\delta)}$ has parameters $[n,k,d]$, where \begin{center}
			$\begin{cases}
				d\geq\delta+1,& \ \text{if}\ \delta\equiv\frac{q+1}{2}\pmod q;\\
				d\geq\delta,&\  \text{otherwise}
			\end{cases}$ 
		\end{center}and the dimension $k$ is provided as follows:
		\begin{center}
			$k=$
			$\begin{cases}
				n-m\big\lceil \frac{(2\delta-3)(q-1)}{2q}\big\rceil,& \text{if}\ 2\leq \delta\leq\frac{q^{h+1}+3}{4};\\
				n-m\big(\big\lceil \frac{(2\delta-3)(q-1)}{2q}\big\rceil-\frac{q-1}{4}\big),& \text{if}\ \frac{q^{h+1}+q+2}{4}\leq \delta\leq\frac{q^{h+1}+3}{4}+q^h;\\
				n-m\big(\big\lceil \frac{(2\delta-3)(q-1)}{2q}\big\rceil-\frac{q-1}{4}-\tau\big),& \text{if}\ \frac{q^{h+1}+7}{4}+\tau q^h\leq \delta\leq\frac{q^{h+1}+3}{4}+(\tau+1)q^h,\tau \neq\frac{q-1}{4};\\
				n-m\big(\big\lceil \frac{(2\delta-3)(q-1)}{2q}\big\rceil-\frac{q-1}{2}\big),& \text{if}\ \frac{2q^{h+1}-q^h+3}{4}+1\leq \delta\leq\frac{q^{h+1}+3}{2};\\
				n-m\big(\big\lceil \frac{(2\delta-3)(q-1)}{2q}\big\rceil-(q-1)\big),& \text{if}\ \frac{q^{h+1}+q}{2}+1\leq \delta\leq\frac{q^{h+1}+q^h}{2}+1;\\
				n-m\big(\big\lceil \frac{(2\delta-3)(q-1)}{2q}\big\rceil-(q+2\tau -2)\big),& \text{if}\ \frac{q^{h+1}-q^h}{2}+\tau q^h+2\leq \delta\leq\frac{2q^{h+1}-q^h+3}{4}+\tau q^h;\\
				n-m\big(\big\lceil \frac{(2\delta-3)(q-1)}{2q}\big\rceil-(q+2\tau -1)\big),& \text{if}\ \frac{2q^{h+1}-q^h+7}{4}+\tau q^h\leq \delta\leq\frac{q^{h+1}+q^h}{2}+\tau q^h+1,\tau \neq\frac{q-1}{4},
			\end{cases}$
		\end{center} where $1\leq \tau \leq \frac{q-1}{4}$.
		\begin{proof}
			From Lemma \ref{l3}, we can obtain the results following by the same way of Theorem \ref{th8}.
		\end{proof}
	\end{theorem}
	Next we will investigate the parameters of negacyclic BCH codes $\C_{(n,q,\delta)}$ with one or two zeros.
	\begin{theorem}\label{th13}
		Let $n=\frac{q^m-1}{4}$ and $m\geq 3$, where $q\equiv1\pmod4$ or $q\equiv3\pmod4$ with $2\mid m$. Then the negacyclic BCH code $\C_{(n,q,2)}$ has parameters $[n,n-m,d]$, where
		\begin{center}
			$\begin{cases}
				d=3,\ \text{if}\ q=3\ or\ q=5\ \text{with}\ 2\nmid m;\\
				d=2,\ \text{if}\ q>5\ or\ q=5\ \text{with}\ 2\mid m.
			\end{cases}$
		\end{center}
		\begin{proof}
			Note that the generator polynomal of $\C_{(n,q,2)}$ is $g(x)=M_{\beta}(x)$, where $\beta$ be a primitive $2n$-th root of unity. Since $|C_{1}^{2n}|=m$ for any $m\geq3$, then $dim(\C_{(n,q,2)})=n-m$. From Lemma \ref{l4} and the Sphere Packing Bound, we have $4\geq d(\C_{(n,q,2)})\geq 2$. If $d=4$, this implies $1+\frac{(q^m-5)(q-1)}{4}\leq q^{m-1}$ by Lemma \ref{l5}, which is impossible by $m\geq 3$. Hence, $3\geq d(\C_{(n,q,2)})\geq 2$.
			
			If $q=3$ with $2\mid m$, then $3\in C_{1}^{2n}\Rightarrow\C_{(n,q,2)}=\C_{(n,q,3)}$. From Lemma \ref{l4}, we have $d(\C_{(n,q,2)})\geq 3$, i.e., $d(\C_{(n,q,2)})=3$. 
			
			If $q=5$, then $gcd(n,2)=gcd(m,2)$. For $2\mid m$, we have $\beta^\frac{n}{2}\in \F_5^*$ and $x^{\frac{n}{2}}-\beta^\frac{n}{2}\in \C_{(n,q,2)}$, then $d(\C_{(n,q,2)})=2$. For $2\nmid m$, suppose there exists a codeword $a_0+a_1x^s\in \C_{(n,q,2)}$, where $a_0,a_1 \in \F_5^*$ and $0<s<n$. We have $x^s+a_1^{-1}a_0\in\C_{(n,q,2)}$, which implies $\beta^{4s}=1$. It follows that $2n\mid 4s$, then $n\mid s$ by $gcd(n,2)=1$. This contradicts the fact that $0<s<n$, then $d(\C_{(n,q,2)})\neq2$, then $d(\C_{(n,q,2)})=3$ for $2\nmid m$.
			
			If $q>5$, it is clear that $\beta^{\frac{q^m-1}{q-1}}\in \F_q^{*}$, then there exist $a_0$, $a_1\in \F_q^*$ such that $a_0+a_1\beta^{\frac{q^m-1}{q-1}}=0$. Note that $\frac{q^m-1}{q-1}<\frac{q^m-1}{4}$, this implies that $a_0+a_1x^{\frac{q^m-1}{q-1}}\in \C_{(n,q,2)}$, then $d(\C_{(n,q,2)})=2$. Thus this completes the proof.
		\end{proof}
	\end{theorem}
	
	\begin{theorem}\label{th14}
		Let $n=\frac{q^m-1}{4}$ and $m\geq 3$, where $q\equiv1\pmod4$ or $q\equiv3\pmod4$ with $2\mid m$. Then the negacyclic BCH code $\C_{(n,q,3)}$ has parameters $[n,n-2m,d]$, where
		$$\begin{cases}
			d=3,&\ \text{if}\ q>5 \ \text{and}\ q\neq 9\ when\ 2\nmid m;\\
			d=4,&\ \text{if}\ q=5\ \text{with}\ 2\mid m;\\
			5\geq d\geq 4,&\ \text{if}\ q=5\ \text{with}\ 2\nmid m;\\
			5\geq d\geq 3,&\ \text{if}\ q=9\ \text{with}\ 2\nmid m.
		\end{cases}$$
		\begin{proof}
			Note that the generator polynomal of $\C_{(n,q,3)}$ is $g(x)=M_{\beta}(x)M_{\beta^3}(x)$, where $\beta$ be a primitive $2n$-th root of unity. Since $|C_{1}^{2n}|=|C_{3}^{2n}|=m$ for any $m\geq3$, then $dim(\C_{(n,q,3)})=n-2m$. From Lemmas \ref{l4} and \ref{l5}, we have $5\geq d(\C_{(n,q,3)})\geq 3$ for $q\geq 5$.
			
			If $q=5$, then $5\in C_{1}^{2n}$, then $5\geq d(\C_{(n,q,3)})\geq4$ by Lemma \ref{l4}. For $2\mid m$, we have $6\mid n$, then we assume  $c(x)=1+\beta^{\frac{n}{2}}x^{\frac{n}{6}}+\beta^{\frac{n}{2}}x^{\frac{n}{2}}-x^{\frac{2n}{3}}$. It is clear that $c(\beta)=c(\beta^3)=0$ and $c(x)\in \F_{5}[x]$, then $c(x)\in\C_{(n,q,3)}$ and $d(c(x))=4$, then $d(\C_{(n,q,3)})=4$.
			
			For $q>5$, we have the following cases.
			\begin{itemize}
				\item[1)] If $q>9 $ and $2\nmid m$, we assume that $s_1=\frac{4n}{q-1}$ and $s_2=\frac{8n}{q-1}$. It is clear that $s_1,s_2$ are integers with $0<s_{1},s_2<n$, suppose the system of equations:
				$$\begin{cases}
					1+a_0\beta^{s_1}+a_1\beta^{s_2}=0;\\
					1+a_0\beta^{3s_1}+a_1\beta^{3s_2}=0.
				\end{cases}$$
				Note that $\beta^{s_1},\beta^{s_2}\in \F_{q}^*$, then 
				\begin{center}
					$\begin{vmatrix}
						\beta^{s_1} & \beta^{s_2}\\
						\beta^{3s_1} & \beta^{3s_2}\\
					\end{vmatrix}$=$\beta^{\frac{20n}{q-1}}(\beta^{\frac{8n}{q-1}}-1)\in \F_{q}^*.$
				\end{center}
				Then the equations above have a unique solution $(a_0,a_1)\in (\F_q^*)^2$. It is clear that $1+a_0x^{s_1}+a_1x^{s_2}\in \C_{(n,q,3)}$, then $d(\C_{(n,q,3)})=3$.
				\item[2)] If $q\geq 7$ and $2\mid m$, we can assume $s_1=\frac{2n}{q-1}$ and $s_2=\frac{4n}{q-1}$. By the same way as above, there exists $b_0,b_1\in \F_q^*$ such that $1+b_0x^{s_1}+b_1x^{s_2}\in \C_{(n,q,3)}$, then $d(\C_{(n,q,3)})=3$.
			\end{itemize}
			Thus this completes the proof.
		\end{proof}
	\end{theorem}
	In the following, we will investigate the parameters of negacyclic BCH codes with small dimensions.
	\begin{lemma}\label{le7}
		Let $n=\frac{q^m-1}{2}$ and $q$ be an odd prime power. Then
		\begin{itemize}
			\item[(1)]If $q\equiv1\pmod4$, then the first three largest odd coset leaders modulo $n$ are:
			\begin{center}
				$\delta_{1}=\frac{q^m-q^{m-1}-q^{\lfloor\frac{m-1}{2}\rfloor}-1}{2},\ 
				\delta_{2}=\frac{q^m-q^{m-1}-q^{\lfloor\frac{m+1}{2}\rfloor}-1}{2},\ 
				\delta_{3}=\frac{q^m-q^{m-1}-q^{\lfloor\frac{m+1}{2}\rfloor}-1}{2}\  (m\geq 6).$
			\end{center}
			Moreover, $$\begin{cases}
				|C_{\delta_{1}}^{n}|=\frac{m}{2},\ |C_{\delta_{2}}^{n}|=|C_{\delta_{3}}^{n}|=m,&\ \text{if}\  2\mid m;\\
				|C_{\delta_{1}}^{n}|=|C_{\delta_{2}}^{n}|=|C_{\delta_{3}}^{n}|=m,&\ \text{if}\  2\nmid m.
			\end{cases}$$	
			\item[(2)]If $q\equiv3\pmod4$ and $2\mid m$, then the first two largest odd coset leaders modulo $n$ are:\begin{center}
				$\delta_{1}=$
				$\begin{cases}
					\frac{q^{m}-q^{m-1}-q^{\frac{m}{2}}-1}{2},&\ \text{if} \ m\equiv2 \pmod4;\\
					\frac{q^{m}-q^{m-1}-q^{\frac{m-2}{2}}-1}{2},&\ \text{if} \ m\equiv0 \pmod4.
				\end{cases}$\\
				$\delta_{2}=$
				$\begin{cases}
					\frac{q^{m}-q^{m-1}-q^{\frac{m+4}{2}}-1}{2},&\ \text{if} \ m\equiv2 \pmod4\ with \ m\geq 10;\\
					\frac{q^{m}-q^{m-1}-q^{\frac{m+2}{2}}-1}{2},&\ \text{if} \ m\equiv0 \pmod4\ with \ m\geq 6.
				\end{cases}$
			\end{center}
			Moreover, 
			$$\begin{cases}
				|C_{\delta_{1}}^{n}|=|C_{\delta_{2}}^{n}|=m,&\ \text{if}\  m\equiv2\pmod 4;\\
				|C_{\delta_{1}}^{n}|=\frac{m}{2},\ |C_{\delta_{2}}^{n}|=m,&\ \text{if}\  m\equiv0\pmod 4.
			\end{cases}$$	
		\end{itemize}
		\begin{proof}
			We only proof the case of $q\equiv1\pmod4$, since other cases are similar. It is easy to get that the $i-$th largest coset leader modulo $n$ is
			$\delta_{i}^{'}=\frac{q^m-q^{m-1}}{2}-\frac{q^{\lfloor\frac{m-3}{2}+i\rfloor}+1}{2}$ by Lemma \ref{le6}. Since $q\equiv1\pmod4$, we deduce $q^m\equiv1\pmod4$ for any $m$, then $\delta_{i}^{'}$ is odd for any $1\leq i\leq \lfloor\frac{m+6}{4}\rfloor$. 
			\\Thus this completes the proof.
		\end{proof}
	\end{lemma}
	\begin{theorem}\label{th16}
		Let $n=\frac{q^m-1}{4}$ and $q\equiv1\pmod4$, $\delta_1,\delta_2,\delta_3$ are given in Lemma \ref{le7}. Then the negacyclic BCH code $\C_{(n,q,\delta)}$ has parameters $[n,m(i-1)+\kappa,d\geq \frac{\delta_{i}+1}{2}]$ with  $\frac{\delta_{i+1}+3}{2}\leq \delta\leq \frac{\delta_{i}+1}{2}$ $(i=1,2)$, where \begin{center}
			$\kappa=$$\begin{cases}
				\frac{m}{2},&\ \text{if}\ 2\mid m;\\
				m,&\ \text{if}\ 2\nmid m .
			\end{cases}$	
		\end{center}
		\begin{proof}
			If $\frac{\delta_{2}+3}{2}\leq \delta\leq \frac{\delta_{1}+1}{2},$ then the check polynomial of $\C_{(n,q,\delta)}$ is $M_{\beta^{\delta_{1}}}(x)$, where $\beta$ is a primitive $2n$-th root of unity in $\F_{q^m}$. From Lemmas \ref{l4} and \ref{le7}, we can get the expected conclusion. The proof of $\frac{\delta_{3}+3}{2}\leq \delta\leq \frac{\delta_{2}+1}{2}$ is similar, so the proof is omitted.
		\end{proof}
	\end{theorem}
	\begin{theorem}\label{th17}
		Let $n=\frac{q^m-1}{4}$ and $q\equiv3\pmod4$, where $ m\equiv2 \pmod4\ with \ m\geq 10$. If $\frac{\delta_{2}+3}{2}\leq \delta\leq \frac{\delta_{1}+1}{2}$ and $\delta_{1},\delta_2$ are given as Lemma \ref{le7}, then 
		\begin{itemize}
			\item[(1)]If $q>3$, then the negacyclic BCH code $\C_{(n,q,\delta)}$ has parameters $[n,m,d\geq \frac{\delta_{1}+1}{2}]$.
			\item[(2)]If $q=3$, then $dim(\C_{(n,3,\delta)})=m$ and $\C_{(n,3,\delta)}$ is a two-weight code.
			Moreover, the weight distribution of $\C_{(n,3,\delta)}$ is \begin{table}[h]
				\centering
				\footnotesize
				\renewcommand{\arraystretch}{1.5}
				\setlength{\tabcolsep}{6pt}
				\caption{Weight distribution of $\C_{(n,3,\delta)}$}
				\label{tab:my_label}
				\scalebox{1.0}{
					\begin{tabular}{lll}
						\toprule
						\textbf{Weight} & \textbf{Frequency} \\ \midrule
						0 & 1 \\
						$\frac{3^{m-1}-3^\frac{m-2}{2}}{2}$ & $\frac{3^m-1}{2}$ \\
						$\frac{3^{m-1}+3^\frac{m-2}{2}}{2}$ & $\frac{3^m-1}{2}$ \\
						\bottomrule
					\end{tabular}
				}
			\end{table}
		\end{itemize}
		\begin{proof}
			If $\frac{\delta_{2}+3}{2}\leq \delta\leq \frac{\delta_{1}+1}{2}$, we can get $dim(\C_{(n,q,\delta)})=m$ and $d(\C_{(n,q,\delta)})\geq \frac{\delta_{1}+1}{2}$ by Lemmas \ref{l4} and \ref{le7}. Note that
			the code $\C_{(n,3,\delta)}$ has one nonzero $\beta^{\delta_{1}}$, where $\beta$ is a primitive $2n$-th root of unity in $\F_{3^m}$. For $q=3$, we denote $\alpha$ is a primitive element of $\F_{3^m}$, then $\beta=\alpha^{2}$. Note that $-2\delta_{1}\equiv 3^{m-1}+3^{\frac{m}{2}} \pmod {3^{m}-1}$, then the trace expression of $\C_{(n,3,\delta)}$ is
			$$\begin{aligned}
				\C_{(n,3,\delta)}&=\{(Tr_{3^{m}/3}(a\beta^{-\delta_{1}i}))_{i=0}^{n-1}: a\in \F_{3^m}\}\\
				&=\{(Tr_{3^{m}/3}(a\alpha^{(3^{m-1}+3^{\frac{m}{2}})i}))_{i=0}^{n-1}: a\in \F_{3^m}\}\end{aligned} $$  
			Note that $Tr_{3^{m}/3}(a\alpha^{(3^{m-1}+3^{\frac{m}{2}})i})=Tr_{3^{m}/3}(a^3\alpha^{(3^{\frac{m+2}{2}}+1)i})$ and $gcd(3^{m}-1,3^{\frac{m+2}{2}}+1)=2$, then the code $\C_{(n,3,\delta)}$ is permutation equivalent to the
			following code $$\{c(a)=(Tr_{3^{m}/3}(a\alpha^{2i}))_{i=0}^{n-1}: a\in \F_{3^m}\}.$$
			
			Clearly, $c(0)$ is the zero codeword. If $a\neq0$, we have 
			$$\begin{aligned}\label{eq1}
				wt(c(a))&=n-|\{i:Tr_{3^{m}/3}(a\alpha^{2i})=0,0\leq i\leq n-1\}|\\
				&=n-\frac{1}{3}\sum_{i=0}^{n-1}\sum_{x\in \F_{3}}\zeta_{p}^{xTr_{3^{m}/3}(a\alpha^{2i})}\\
				&=\frac{2n}{3}-\frac{1}{3}\sum_{i=0}^{n-1}\sum_{x\in \F_{3}^*}\zeta_{p}^{xTr_{3^{m}/3}(a\alpha^{2i})}.
			\end{aligned}$$
			Note that $q=3$, then 
			$$\begin{aligned}
				\sum_{i=0}^{n-1}\sum_{x\in \F_{3}^*}\zeta_{p}^{xTr_{3^{m}/3}(a\alpha^{2i})}&=\sum_{i=0}^{n-1}\zeta_{p}^{Tr_{3^{m}/3}(a\alpha^{2i})}+\sum_{i=0}^{n-1}\zeta_{p}^{-Tr_{3^{m}/3}(a\alpha^{2i})}\\
				&=\sum_{i=0}^{n-1}\zeta_{p}^{Tr_{3^{m}/3}(a\alpha^{2i})}+\sum_{i=0}^{n-1}\zeta_{p}^{Tr_{3^{m}/3}(a\alpha^{2(n+i)})}\\
				&=\sum_{i=0}^{2n-1}\zeta_{p}^{Tr_{3^{m}/3}(a\alpha^{2i})}.
			\end{aligned}$$
			Note that $\sum_{i=0}^{2n-1}\zeta_{p}^{Tr_{3^{m}/3}(a\alpha^{2i})}=\sum_{i=2n}^{4n-1}\zeta_{p}^{Tr_{3^{m}/3}(a\alpha^{2i})}$, then
			$$\begin{aligned}
				wt(c(a))&=\frac{2n}{3}-\frac{1}{3}\sum_{i=0}^{n-1}\sum_{x\in \F_{3}^*}\zeta_{p}^{xTr_{3^{m}/3}(a\alpha^{2i})}\\
				&=\frac{2n}{3}-\frac{1}{3}\sum_{i=0}^{2n-1}\zeta_{p}^{Tr_{3^{m}/3}(a\alpha^{2i})}\\
				&=\frac{2n}{3}-\frac{1}{6}\sum_{i=0}^{4n-1}\zeta_{p}^{Tr_{3^{m}/3}(a\alpha^{2i})}\\
				&=\frac{2n}{3}-\frac{1}{6}\sum_{y\in \F_{3^{m}}^{*}}\zeta_{p}^{Tr_{3^{m}/3}(ay^{2})}\\
				&=\frac{3^m-1}{6}-\frac{1}{6}(\eta(a)G(\eta)-1),
			\end{aligned}$$
			where $\eta$ is the quadratic character of $\F_{3^m}$ and $G(\eta)$ is the quadratic Gauss sum. For all $a\in  \F_{3^{m}}$, we have $\eta(a)=1$ for $\frac{3^m-1}{2}$ times, and $\eta(a)=-1$ for $\frac{3^m-1}{2}$ times.
			Using the values of $G(\eta)=(-1)^{m-1}(\sqrt{-1})^mq^{\frac{m}{2}}$, we obtain the weight distribution is Table $1$. Thus this completes the proof.
		\end{proof}
	\end{theorem}
	\begin{theorem}\label{th18}
		Let $n=\frac{q^m-1}{4}$ and $q\equiv3\pmod4$, where $ m\equiv0 \pmod4\ with \ m\geq 6$. If $\frac{\delta_{2}+3}{2}\leq \delta\leq \frac{\delta_{1}+1}{2}$, then the negacyclic BCH code $\C_{(n,q,\delta)}$ is an  $[n,\frac{m}{2},\frac{(q-1)(q^{m-1}+q^{\frac{m-2}{2}})}{4}]$ one-weight code, where $\delta_1,\delta_2$ are given in Lemma \ref{le7}.
		\begin{proof}
			By the same way of Theorem \ref{th16}, we have $dim(\C_{(n,q,\delta)})=\frac{m}{2}$ by Lemmas \ref{l4} and \ref{le7}. Note that the code $\C_{(n,q,\delta)}$ has only one nonzeroes $\beta^{\delta_{1}}$ and $|C_{\delta_1}^{2n}|=\frac{m}{2}$, where $\beta$ is a primitive $2n$-th root of unity in $\F_{q^m}$. Denote $\alpha$ is a primitive element of $\F_{q^m}$, then $\beta=\alpha^{2}$. Let $h=\frac{m}{2}$, then $2\mid h$. Note that $-2\delta_{1}\equiv q^{m-1}+q^{\frac{m-2}{2}} \pmod {q^{m}-1}$, then the trace expression of $\C_{(n,q,\delta)}$ is
			$$\begin{aligned}
				\C_{(n,q,\delta)}&=\{(Tr_{q^{h}/q}(a\beta^{-\delta_{1}i}))_{i=0}^{n-1}: a\in \F_{q^h}\}\\
				&=\{(Tr_{q^{h}/q}(a\alpha^{(q^{m-1}+q^{\frac{m-2}{2}})i}))_{i=0}^{n-1}: a\in \F_{q^h}\}.\end{aligned} $$ 
			Note that $Tr_{q^{h}/q}(a\alpha^{(q^{m-1}+q^{\frac{m-2}{2}})i})=Tr_{q^{h}/q}(a^q\alpha^{(q^{\frac{m}{2}}+1)i})$, then the code $\C_{(q,n,\delta,0)}$ has the same weight distribution with the following code
			$$\{c(a)=(Tr_{q^{h}/q}(a\alpha^{(q^{h}+1)i}))_{i=0}^{n-1}: a\in \F_{q^h}\}.$$
			Note that $4\mid q^h-1$, then $\frac{q^{m}-1}{4}=\frac{q^h-1}{4}(q^h-1)+\frac{q^h-1}{2}$. Let $n^{'}=q^{h}-1$, we have $$c(a)=\underbrace{c_{1}(a)||\cdots||c_1(a)}_{\frac{q^h-1}{4}}||c_{2}(a),$$
			where $||$ denotes the concatenation of vectors, $\C_{1}=\{c_1(a)=(Tr_{q^{h}/q}(a\alpha^{(q^{h}+1)i}))_{i=0}^{n^{'}-1}: a\in \F_{q^h}\}$ and $\C_{2}=\{c_2(a)=(Tr_{q^{h}/q}(a\alpha^{(q^{h}+1)i}))_{i=0}^{\frac{n^{'}}{2}-1}: a\in \F_{q^h}\}$.
			Denote $\gamma=\alpha^{(q^{h}+1)}$, then $\gamma$ is a primitive element of $\F_{q^{h}}$. Then 
			$$\begin{aligned}
				wt(c_{1}(a))&=n^{'}-|\{i:Tr_{q^{h}/q}(a\gamma ^i)=0,0\leq i\leq n^{'}-1\}|\\
				&=n^{'}-|\{x\in \F_{q^{h}}^{*}:Tr_{q^{h}/q}(ax)=0\}|.
			\end{aligned}$$
			Hence, $\C_{1}$ is an one-weight code over $\F_{q}$ with $wt(c_{1}(a))=q^{h}-q^{h-1}$. For $\C_{2}$, we have $$
			\begin{aligned}
				wt(c_{2}(a))&=\frac{n^{'}}{2}-|\{i:Tr_{q^{h}/q}(a\gamma ^i)=0,0\leq i\leq \frac{n^{'}}{2}-1\}|\\
				&=\frac{n^{'}}{2}-\frac{1}{q}\sum_{i=0}^{\frac{n^{'}}{2}-1}\sum_{x\in \F_{q}}\zeta_{p}^{xTr_{q^{h}/q}(a\gamma^{i})}\\
				&=\frac{n^{'}}{2}-\frac{1}{2q}\sum_{i=0}^{n^{'}-1}\sum_{x\in \F_{q}}\zeta_{p}^{xTr_{q^{h}/q}(a\gamma^{i})}\\
				&=\frac{n^{'}}{2}-\frac{1}{2q}\sum_{x\in \F_{q}}\sum_{y\in \F_{q^h}^{*}}\zeta_{p}^{Tr_{q^{h}/q}(xay)}\\
				&=\frac{n^{'}}{2}-\frac{n^{'}}{2q}+\frac{q-1}{2q}-\frac{1}{2q}\sum_{x\in \F_{q}^{*}}\sum_{y\in \F_{q^h}}\zeta_{p}^{Tr_{q^{h}/q}(xay)}\\
				&=\frac{(q-1)q^{h-1}}{2}.
			\end{aligned}$$
			Summarize the above discussion, we have $$wt(c(a))=\frac{q^{h}-1}{4}wt(c_{1}(a))+wt(c_{2}(a))=\frac{(q^{2h-1}+q^{h-1})(q-1)}{4}.$$ Thus this completes the proof.
		\end{proof}	
	\end{theorem}
	\begin{theorem}\label{th19}
		Let $n=\frac{q^m-1}{4}$ and $q\equiv1\pmod4$, where $ m=2h\geq 2$. If $\frac{\delta_{2}+3}{2}\leq \delta\leq \frac{\delta_{1}+1}{2}$, then the negacyclic BCH code $\C_{(n,q,\delta)}$ is an  $[n,\frac{m}{2},\frac{(q-1)(q^{m-1}+q^{\frac{m-2}{2}})}{4}]$ one-weight code, where $\delta_1,\delta_2$ are given in Lemma \ref{le7}.
		\begin{proof}
			From Lemma \ref{le7}, we can obtain the results following by the same way of Theorem \ref{th18}.
		\end{proof}
	\end{theorem}
	\section{The case of $n=\frac{q^m+1}{4}$}
	In this section, we assume $n=\frac{q^m+1}{4}$, where $q\equiv3\pmod 4$ and $m$ is an odd integer. We first study the parameters of the negacyclic BCH codes with small dimensions.
	
	\begin{lemma}\label{l17}
		Let $n=\frac{q^m+1}{2}$, where $ q\equiv3\pmod 4$ and $m=2h+1\geq 5$. If $1\leq i\leq q^{h+1}-q$ is an odd integer with $i\not\equiv0\pmod{q}$, then $i\notin \MinRep_{n}$ if and only if $i\in B_1\cup B_2$, where
		\begin{itemize} 
			\item[(1)]If $2\mid h$, then
			$$\begin{cases}
				B_1=\{\frac{q^{h+1}-1}{2}+2u:-\frac{q-3}{4}\leq u\leq \frac{q-3}{4}\};\\
				B_2=\{\frac{q^{h+1}+1}{2}+(2v-1)q^h,\frac{q^{h+1}-1}{2}+2vq^h,\frac{q^{h+1}+1}{2}+\frac{q-1}{2}q^h:1\leq v\leq\frac{q-3}{4}\}.
			\end{cases}$$
			\item[(2)]If $2\nmid h$, then
			$$\begin{cases}
				B_1=\{\frac{q^{h+1}+1}{2}+2u:-\frac{q-3}{4}\leq u\leq \frac{q-3}{4}\};\\
				B_2=\{\frac{q^{h+1}-1}{2}+(2v-1)q^h,\frac{q^{h+1}+1}{2}+2vq^h,\frac{q^{h+1}-1}{2}+\frac{q-1}{2}q^h:1\leq v\leq\frac{q-3}{4}\}.
			\end{cases}$$
		\end{itemize}
		\begin{proof}
			From Lemma \ref{le8}, we can obtain the results by the same way of Lemma \ref{l2}.
		\end{proof}
	\end{lemma}
	\begin{theorem}\label{th20}
		Let $n=\frac{q^m+1}{4}$, where $ q\equiv3\pmod 4$ and $m=2h+1\geq 5$. Then we have the negacyclic BCH code $C_{(n,q,\delta)}$ has parameters $[n,k,d]$, where \begin{center}
			$\begin{cases}
				d\geq 2\delta+1,& \ \text{if}\ \delta\equiv\frac{q+1}{2}\pmod q;\\
				d\geq 2\delta-1,&\  \text{otherwise}.
			\end{cases}$ 
		\end{center}and the dimension $k$ is provided as follows:
		\begin{itemize}
			\item[(1)]If $2\mid h$, define $\tau$ is an integer with $1\leq \tau\leq \frac{q-3}{4}$. Then 
			$$k=\begin{cases}
				n-2m\big\lceil \frac{(2\delta-3)(q-1)}{2q}\big\rceil;&\text{if}\ 2\leq \delta\leq \frac{q^{h+1}-q}{4}+1;\\
				n-2m\big(\big\lceil \frac{(2\delta-3)(q-1)}{2q}\big\rceil-\frac{q-1}{2}\big), &\text{if}\ \frac{q^{h+1}+q+2}{4}\leq \delta\leq\frac{q^{h+1}+2q^h+3}{4};\\
				n-2m\big(\big\lceil \frac{(2\delta-3)(q-1)}{2q}\big\rceil-\frac{q-1}{2}-(2\tau-1)\big),&\text{if}\ \frac{q^{h+1}-2q^h+3}{4}+\tau q^h+1\leq \delta\leq \frac{q^{h+1}+1}{4}+\tau q^h;\\
				n-2m\big(\big\lceil \frac{(2\delta-3)(q-1)}{2q}\big\rceil-\frac{q-1}{2}-2\tau\big), &\text{if}\ \frac{q^{h+1}+5}{4}+\tau q^h\leq \delta\leq \frac{q^{h+1}+2q^h+3}{4}+\tau q^h;\\
				n-2m\big(\big\lceil \frac{(2\delta-3)(q-1)}{2q}\big\rceil-(q-1)\big), &\text{if}\ \frac{2q^{h+1}-q^h+3}{4}+1\leq \delta\leq \frac{q^{h+1}-q}{2}+1.
			\end{cases}$$
			\item[(2)]If $2\nmid h$, define $\tau$ is an integer with $1\leq \tau\leq \frac{q-3}{4}$. Then 
			$$k=\begin{cases}
				n-2m\big\lceil \frac{(2\delta-3)(q-1)}{2q}\big\rceil,&\text{if}\ 2\leq \delta\leq \frac{q^{h+1}-q+2}{4}+1;\\
				n-2m\big(\big\lceil \frac{(2\delta-3)(q-1)}{2q}\big\rceil-\frac{q-1}{2}\big), &\text{if}\ \frac{q^{h+1}+q}{4}+1\leq \delta\leq\frac{q^{h+1}+2q^h+1}{4};\\
				n-2m\big(\big\lceil \frac{(2\delta-3)(q-1)}{2q}\big\rceil-\frac{q-1}{2}-(2\tau-1)\big),&\text{if}\ \frac{q^{h+1}-2q^h+1}{4}+\tau q^h+1\leq \delta\leq \frac{q^{h+1}+3}{4}+\tau q^h;\\
				n-m\big(\big\lceil \frac{(2\delta-3)(q-1)}{2q}\big\rceil-\frac{q-1}{2}-2\tau\big), &\text{if}\ \frac{q^{h+1}-1}{4}+\tau q^h+2\leq \delta\leq \frac{q^{h+1}+2q^h+1}{4}+\tau q^h;\\
				n-2m\big(\big\lceil \frac{(2\delta-3)(q-1)}{2q}\big\rceil-(q-1)\big), &\text{if}\ \frac{2q^{h+1}-q^h+1}{4}+1\leq \delta\leq \frac{q^{h+1}-q}{2}+1.
			\end{cases}$$	
		\end{itemize}
		\begin{proof}
			We only prove the case of $2\mid h$, since the case of $2\nmid h$ is similar. It is clear that the generator polynomal of $\C_{(n,q,\delta)}$ is $g(x)=lcm(M_{\beta^{1}}(x),M_{\beta^{3}}(x),\ldots,M_{\beta^{1+2(\delta-2)}}(x))$,  where $\beta$ is a primitive $2n$-th root of unity. Let $\Gamma=\lbrace 1+2i:0\leq i\leq \delta-2,\ \ 1+2i\not\equiv0\pmod q\rbrace$ for $2\leq \delta\leq \frac{q^{h+1}-q}{2}+1$, then
			$|\Gamma|=\big\lceil\frac{(2\delta-3)(q-1)}{2q}\big\rceil.$ From Lemmas \ref{le8} and \ref{l17}, we have $1+2i\in \Gamma$ is a coset leader except
			$1+2i\in B_{1}\cup B_{2}$ and $|C_{1+2i}^{2n}|=2m$. Then  
			\begin{equation}\label{eq6}
				k=n-2m|\Gamma|+2m(|\Gamma\cap B_1|+|\Gamma\cap B_2|).
			\end{equation}
			
			If $2\leq \delta\leq\frac{q^{h+1}-q}{4}+1$, note that $max\{\Gamma\}<min\{B_1\cup B_2\}=\frac{q^{h+1}-q}{2}+1$, then $\Gamma\cap B_1=\Gamma\cap B_2=\emptyset$. It follows from (\ref{eq6}) that $	k=n-2m\big\lceil \frac{(2\delta-3)(q-1)}{2q}\big\rceil$. By the same way the results can be established for $\frac{q^{h+1}+q+2}{4}\leq \delta\leq \frac{q^{h+1}-q}{2}+1$.
		\end{proof}
	\end{theorem}
	\begin{theorem}\label{th21}
		Let $n=\frac{q^m+1}{4}$, where $q\equiv3\pmod4$ and $m=2h+1\geq 3$. Then the negacyclic BCH code $\C_{(n,q,2)}$ has parameters $[n,n-2m,d]$, where
		$$\begin{cases}
			d=3,&\ \text{if}\ q>7;\\
			6\geq d\geq5,&\ \text{if}\ q=3;\\
			6\geq d\geq3,&\ \text{if}\ q=7.
		\end{cases}$$
		\begin{proof}
			Note that the generator polynomal of $\C_{(n,q,2)}$ is $g(x)=M_{\beta}(x)$, where $\beta$ be a primitive $2n$-th root of unity. Since $|C_{1}^{2n}|=2m$ for any $m\geq3$, then $dim(\C_{(n,q,2)})=n-2m$. From Lemma \ref{l4} and the Sphere Packing Bound, we have $6\geq d(\C_{(n,q,2)})\geq 3$.
			
			If $q=3$, we have $-3,-1,1,3\in C_1^{2n}$, then we have $d\geq 5$ by Lemma \ref{l4}, i.e., $6\geq d\geq 5$.
			
			If $q>7$, let $\nu=\beta^{\frac{q^m+1}{q+1}}$, then $\nu\in \F_{q^{2}}^{*}$ and $2\frac{q^m+1}{q+1}<n$. Then there exist $a_0,a_1,a_2\in \F_{q}^{*}$ such that $a_{0}+a_1\nu+a_2\nu^2=0$. It is clear that $\nu^2\neq \nu$ and $\nu^2\neq 1$, then there exist $a_0,a_{1},a_{2}\in \F^{*}_{q}$ such that $a_{0}+a_1\nu+a_2\nu^2 \in\C_{(n,q,2)}$. Hence, $d=3$ for $q>7$.
		\end{proof}
	\end{theorem}
	In the following, we will investigate the parameters of negacyclic BCH codes with small dimensions.
	\begin{lemma}\label{le21}
		Let $n=q^m+1$ and $aq^i=u_i(q^m+1)+[aq^{i}]_{n}$, where $q$ is odd, $0\leq a\leq \frac{q^m+1}{2}$ and $u_i$ is a positive integer. For any $1\leq i\leq m-1$, there is no integer $l$ such that $$	u_{i}+\frac{[aq^{i}]_{n}-a}{q^{m}+1}<l<u_{i}+\frac{[aq^{i}]_{n}+a}{q^{m}+1},$$ then we have $a\in \MinRep_{n}$.
		\begin{proof}
			From Lemma \ref{l20}, we have $a\in \MinRep_{n}$ and $0\leq a\leq  \frac{q^{m}+1}{2}$ which implies that there are no integers $1\leq i\leq m-1$, $1\leq l\leq \frac{q^{i}-1}{2}$ and $-\frac{l(q^{m-i}-1)}{q^{i}+1}<h<\frac{l(q^{m-i}+1)}{q^{i}-1}$ such that 
			\begin{equation}\label{equ1}
				a=lq^{m-i}+h.
			\end{equation}
			If (\ref{equ1}) holds, we have
			$\frac{l(q^{m}+1)}{q^{i}+1}<a<\frac{l(q^{m}+1)}{q^{i}-1}$ $\Rightarrow$ $\frac{a(q^{i}-1)}{q^{m}+1}<l<\frac{a(q^{i}+1)}{q^{m}+1}.$
			Note that $aq^{i}=u_{i}(q^{m}+1)+[aq^{i}]_{n}$, then 
			\begin{equation}\label{equ3}
				u_{i}+\frac{[aq^{i}]_{n}-a}{q^{m}+1}<l<u_{i}+\frac{[aq^{i}]_{n}+a}{q^{m}+1}.
			\end{equation}
			Therefore, if there is no $l$ such that (\ref{equ3}) holds for any $1\leq i\leq m-1$, then this implies that $a\in \MinRep_{n}$.
		\end{proof}
	\end{lemma}
	\begin{lemma}\label{le22}
		Let $n=\frac{q^m+1}{2}$, where $q\equiv3\pmod{8}$ and $m=2h+1\geq 5$. Then the first three largest odd coset leaders modulo $n$ are:
		$$\delta_{1}=\frac{q^{m}+1}{4},\ \delta_{2}=\frac{q^{m}+1}{4}-\frac{q^{m-1}+q}{2},\ 
		\delta_{3}=\frac{q^{m}-1}{4}-\frac{q^{m-1}+q^2-q}{2}.$$ 
		Moreover, $|C_{\delta_{1}}^{n}|=1$ and $|C_{\delta_{2}}^{n}|=|C_{\delta_{3}}^{n}|=2m$.
		\begin{proof}
			Note that $q\equiv3\pmod{8}$ and $m=2h+1\geq 5$, then $q^m\equiv3\pmod8$ and $q^{m-1}\equiv1\pmod8$. Therefore we have $\delta_{1}^{'} $ and $\delta_{3}^{'}$ are odd integers, $\delta_{2}^{'}$ is even integer, where $\delta_{i}^{'}$ is given in Lemma \ref{le8}. Hence, we have $\delta_1,\delta_2$ are the first two largest odd coset leaders modulo $n$. 
			
			We claim that $\delta_{3} \in \MinRep_{n}$, i.e., $2\delta_{3}\in \MinRep_{2n}$ by Lemma \ref{l20}. It is easy to get that
			\begin{center}
				$[2\delta_{3}q^{i}]_{2n}$=$\begin{cases}
					\frac{q^m-1}{2}+q^{i+1}+q^{i-1}-q^{i+2}-q^i+1,& \text{if}\ 1\leq i\leq m-3;\\
					\frac{q^m-1}{2}+q^{m-1}+q^{m-3}-q^{m-2}+2,& \text{if}\ i= m-2;\\
					\frac{q^m-1}{2}+q^{m-2}-q^{m-1}+q,& \text{if}\ i=m-1.
				\end{cases}$
			\end{center}
			On one hand, \begin{center}
				$[2\delta_{3}q^{i}]_{2n}-2\delta_3$=$\begin{cases}
					q^{m-1}+q^{i+1}+q^{i-1}-q^{i+2}-q^i+q^2-q+1>0,& \text{if}\ 1\leq i\leq m-3;\\
					2q^{m-1}+q^{m-3}-q^{m-2}+q^2-q+2>0,& \text{if}\ i= m-2;\\
					q^{m-2}+q^{2}>0,& \text{if}\ i=m-1.
				\end{cases}$
			\end{center}
			On the other hand, \begin{center}
				$[2\delta_{3}q^{i}]_{2n}+2\delta_3$=$\begin{cases}
					q^{m}+q^{i+1}+q^{i-1}-(q^{m-1}+q^{i+2}+q^i)-q^2+q<q^m+1 ,& \text{if}\ 1\leq i\leq m-3;\\
					q^{m}+1-q^{m-2}-q^{m-3}-q^2+q<q^m+1,& \text{if}\ i= m-2;\\
					q^m+q^{m-2}-2q^{m-1}-q^{2}+2q-1<q^m+1,& \text{if}\ i=m-1.
				\end{cases}$
			\end{center}
			Hence, there is no integer $l$ such that (\ref{equ3}) holds for any $1\leq i\leq m-1$, i.e., $2\delta_3\in \MinRep_{2n}$ by Lemma \ref{le21}. Note that  $[2\delta_{3}q^{m}]_{2n}=\frac{q^m+3}{2}+q^{m-1}+q^2-q>2\delta_3$ and $[2\delta_{3}q^{i}]_{2n}-2\delta_3>0$ for all $1\leq i\leq m-1$, then we have $|C_{2\delta_3}^{2n}|=2m$. Furthermore, $\delta_3$ is odd, then $\delta_3$ is odd coset leader modulo $n$ and $|C_{\delta_3}^{n}|=2m$.
			
			We next prove that there is no odd integer $a$ such that $a\in \MinRep_{n}$ for any  $\delta_3+1\leq a\leq\delta_2-1$. 
			Suppose there is an odd integer $a$ such that $a\in \MinRep_{n}$ and $\delta_3+1\leq a\leq\delta_2-1$. From Lemma \ref{l20}, we have $2a\in\MinRep_{2n}$ and $2\delta_3+2\leq 2a\leq2\delta_2-2$. Note that 
			\begin{center}
				$\left\{\begin{aligned}
					2\delta_{2}-2=(\frac{q-3}{2},\underbrace{\frac{q-1}{2},\ldots}_{m-4},\frac{q-1}{2},\frac{q-3}{2},\frac{q-3}{2});\\
					2\delta_{3}+2=(\frac{q-3}{2},\underbrace{\frac{q-1}{2},\ldots}_{m-4},\frac{q-3}{2},\frac{q-1}{2},\frac{q+5}{2}).
				\end{aligned} 
				\right.$
			\end{center}
			Let the $q-$adic expansion of $2a$ is $2a=\sum_{i=0}^{m-1}a_{i}q^{i}$, then $a_{m-1}=\frac{q-3}{2},a_{m-2}=\cdots=a_{3}=\frac{q-1}{2}$ and $0\leq a_0,a_1,a_2\leq q-1$. Next we will prove $a_0,a_1,a_2\in\{\frac{q-3}{2},\frac{q-1}{2},\frac{q+1}{2}\}$, it is easy to verify for $q=3$.
			If $q>3$, we have the following two cases.
			\begin{itemize}
				\item[1)]If $a_{i}\leq \frac{q-5}{2}$, then $[2aq^{m-i-1}]_{2n}<(\frac{q-5}{2},\underbrace{q-1,\ldots}_{m-2},q-1)<2\delta_{3}<2a$, a contradiction to the assumption that $2a\in \MinRep_{2n}$.
				\item[2)]If $a_{i}\geq \frac{q+3}{2}$, then
				\begin{center}
					$\begin{cases}
						[2aq^{m-i-1}]_{2n}> (\frac{q+3}{2},0,0,\ldots,0),    &\text{if}\ i\neq0;\\
						[2aq^{m-k-1}]_{2n}>(\frac{q+1}{2},\frac{q+1}{2},0,\ldots,0),  &\text{if}\ i=0.
					\end{cases}$
				\end{center}  Note that $2n-[2aq^{m-i-1}]_{n}\in C_{2a}^{2n}$, but $$2n-[2aq^{m-i-1}]_{2n}<(\frac{q-3}{2},\frac{q-3}{2},q-1,\ldots,q-1)+2<2\delta_{3}<2a,$$ which gives a contradiction to $2a\in \MinRep_{2n}$.
			\end{itemize}
			Note that $2\mid 2a$, $2\delta_3+2\leq 2a\leq2\delta_2-2$ and $a_0,a_1,a_2\in\{\frac{q-3}{2},\frac{q-1}{2},\frac{q+1}{2}\}$, then $(a_2,a_1,a_0)$ = $(\frac{q-1}{2},\frac{q-3}{2},\frac{q-3}{2})$ or $(\frac{q-3}{2},\frac{q+1}{2},\frac{q-1}{2})$, i.e., $2a=2\delta_2-2$ or $2\delta_3+(q-1)$. But $\delta_2-1$ and $\delta_3+\frac{q-1}{2}$ are even integers,  a contradiction to the assumption that $a$ is an odd integer.
			
			Collecting all discussions above, we have there is no odd integer $a$ such that $a\in \MinRep_{n}$ for all $\delta_3+1\leq a\leq\delta_2-1$. Thus this completes the proof.
		\end{proof}
	\end{lemma}
	\begin{lemma}\label{le23}
		Let $n=\frac{q^m+1}{2}$, where $q\equiv7\pmod{8}$ and $m=2h+1\geq 5$. Then the first three largest odd coset leaders modulo $n$ are:
		$$\delta_{1}=\frac{q^{m}-1}{4}-\frac{q^{m-1}}{2},\ \delta_{2}=\frac{q^{m}+1}{4}-\frac{q^{m-1}+q^2}{2},\ 
		\delta_{3}=\frac{q^{m}-1}{4}-\frac{q^{m-1}+q^3-q^2+q}{2}.$$ 
		Moreover, $|C_{\delta_{1}}^{n}|=|C_{\delta_{2}}^{n}|=|C_{\delta_{3}}^{n}|=2m$.
		\begin{proof}
			Note that $q\equiv7\pmod{8}$ and $m=2h+1\geq 5$, then $q^m\equiv7\pmod8$ and $q^{m-1}\equiv1\pmod8$. Therefore we have $\delta_{2}^{'} $ is odd integer, $\delta_{1}^{'}$ and $\delta_{3}^{'}$ are even integers, where $\delta_{i}^{'}$ is given in Lemma \ref{le8}. Hence, we have $\delta_1$ is the first largest odd coset leaders modulo $n$. 
			
			We claim that $\delta_{2},\delta_{3} \in \MinRep_{n}$, i.e., $2\delta_{2},2\delta_3\in \MinRep_{2n}$ by Lemma \ref{l20}. It is easy to get that
			\begin{center}
				$[2\delta_{2}q^{i}]_{2n}$=$\begin{cases}
					\frac{q^m+1}{2}+q^{i-1}+q^{i+1},& \text{if}\ 1\leq i\leq m-3;\\
					\frac{q^m+1}{2}+q^{m-3}+1,& \text{if}\ i= m-2;\\
					\frac{q^m+1}{2}+q^{m-2}+q,& \text{if}\ i=m-1.
				\end{cases}$\end{center}\begin{center}
				$[2\delta_{3}q^{i}]_{2n}$=$\begin{cases}
					\frac{q^m+1}{2}+q^{i+2}+q^{i-1}-q^{i+3}-2q^i,& \text{if}\ 1\leq i\leq m-4;\\
					\frac{q^m+1}{2}+q^{m-1}+(q-2)q^{m-4}+1,& \text{if}\ i= m-3;\\
					\frac{q^m-1}{2}+(q-2)q^{m-3}+q,& \text{if}\ i=m-2;\\
					\frac{q^m+1}{2}+(q-2)q^{m-2}+q^2-q,& \text{if}\ i=m-1.
				\end{cases}$
			\end{center}
			Therefore, we can get $[2\delta_{2}q^{i}]_{2n}-2\delta_2,[2\delta_{3}q^{i}]_{2n}-2\delta_3>0$ and $[2\delta_{2}q^{i}]_{2n}+2\delta_2,[2\delta_{3}q^{i}]_{2n}+2\delta_3<q^m+1$ for all $1\leq i\leq m-1$.
			By the same way as Lemma \ref{le22}, we have $\delta_2,\delta_3\in \MinRep_{n}$,  $|C_{\delta_2}^{n}|=|C_{\delta_3}^{n}|=2m$ and $\delta_2$ is the second largest odd coset leader.
			
			We claim that $\delta_3$ is the third largest odd coset leader.
			Suppose there exists an odd coset leader $a$ such that $\delta_3+1\leq a\leq\delta_2-1$, we have $2a\in\MinRep_{2n}$ with $2\delta_3+2\leq 2a\leq2\delta_2-2$ by Lemma \ref{l20}. Denote the $q-$adic expansion of $2a$ is $2a=\sum_{i=0}^{m-1}a_{i}q^{i}$, then $a_{m-1}=\frac{q-3}{2},a_{m-2}=\cdots=a_{4}=\frac{q-1}{2}$ and $0\leq a_0,a_1,a_2,a_3\leq q-1$. With a way simalar as in Lemma \ref{le22}, we get $a_0,a_1,a_2\in\{\frac{q-3}{2},\frac{q-1}{2},\frac{q+1}{2}\}$. Furthermore, $2\mid2a$ and $2\nmid a$, then 
			$$2a=(\frac{q-3}{2},\underbrace{\frac{q-1}{2},\ldots}_{m-5},\frac{q-3}{2},\frac{q+1}{2},\frac{q+1}{2},\frac{q-3}{2}) \ or \ (\frac{q-3}{2},\underbrace{\frac{q-1}{2},\ldots}_{m-5},\frac{q-1}{2},\frac{q-3}{2},\frac{q-3}{2},\frac{q-1}{2})$$
			If  $2a=(\frac{q-3}{2},\frac{q-1}{2},\ldots,\frac{q-3}{2},\frac{q+1}{2},\frac{q+1}{2},\frac{q-3}{2})$, note that $2n-[2aq^{m-i-1}]_{n}\in C_{2a}^{2n}$ for all $1\leq i\leq m-1$ and $$2n-[2aq^{m-3}]_{2n}=(\frac{q-3}{2},\frac{q-3}{2},\frac{q+3}{2},\frac{q-3}{2},\underbrace{\frac{q-1}{2},\ldots}_{m-5},\frac{q+1}{2})<2a,$$ a contradiction to the assumption that $2a\in \MinRep_{2n}$.
			\\If $2a=(\frac{q-3}{2},\frac{q-1}{2},\ldots,\frac{q-1}{2},\frac{q-3}{2},\frac{q-3}{2},\frac{q-1}{2})$, note that $$[2aq^{m-3}]_{2n}=(\frac{q-3}{2},\underbrace{\frac{q-1}{2},\ldots}_{m-5},\frac{q-1}{2},\frac{q-3}{2},\frac{q-3}{2},\frac{q-1}{2})<2a,$$ which gives a contradiction to $2a\in \MinRep_{2n}$. 
			
			Collecting all discussions above, we have there is no odd integer $a$ such that $a\in \MinRep_{n}$ for all $\delta_3+1\leq a\leq\delta_2-1$. This completes the proof.
		\end{proof}
	\end{lemma}
	\begin{theorem}\label{th25}
		Let $n=\frac{q^m+1}{4}$, where $m=2h+1\geq 5$. Then
		\begin{itemize}
			\item[(1)]If $q\equiv3\pmod8$, then the negacyclic BCH code $\C_{(n,q,\delta)}$ has parameters
			$$\begin{cases}
				[n,1,d\geq \delta_{1}],&\ \text{if}\ \ \frac{\delta_{2}+3}{2}\leq \delta\leq \frac{\delta_{1}+1}{2};\\
				[n,2m+1,d\geq \delta_{2}],&\ \text{if}\ \frac{\delta_{3}+3}{2}\leq \delta\leq \frac{\delta_{2}+1}{2},
			\end{cases}$$where $\delta_1$, $\delta_2$ and $\delta_3$ are given in Lemma \ref{le22}.	
			\item[(2)]If $q\equiv7\pmod8$, then the negacyclic BCH code $\C_{(n,q,\delta)}$ has parameters $[n,2mi,d\geq \delta_{i}]$ for $\frac{\delta_{i+1}+3}{2}\leq \delta\leq \frac{\delta_{i}+1}{2}$ $(i = 1,2)$, where $\delta_i$ is given in Lemma \ref{le23}.
		\end{itemize}
		\begin{proof}
			From Lemmas \ref{le22} and \ref{le23}, we can obtain the results by the same way of Theorem \ref{th16}.
		\end{proof}
	\end{theorem}
	\section{Conclusions}\label{set6} The main contributions of this paper are as follows:
	\begin{itemize}
		\item Based on the discussion of odd coset leaders in some ranges, we get the parameters of negacyclic BCH code $\C_{(n,q,\delta)}$ with large dimension, where $n=\frac{q^m-1}{4},\frac{q^m+1}{4}$(see Theorems \ref{th8}, \ref{th9}, \ref{th12}, \ref{th13}, \ref{th14}, \ref{th20}, \ref{th21}).
		
		\item By determining the first three largest odd coset leaders modulo $\frac{q^m-1}{2}$ and $\frac{q^m+1}{2}$, we present the parameters of negacyclic BCH code $\C_{(n,q,\delta)}$ with small dimension, where $n=\frac{q^m-1}{4},\frac{q^m+1}{4}$(see Theorems \ref{th16}, \ref{th17}, \ref{th18}, \ref{th19}, \ref{th25}).
	\end{itemize}
	%\section*{References}
	\begin{thebibliography}{}
		\bibitem{RefJ1}Augot, D., Charpin, P., Sendrier, N.: Studying the locator polynomials of minimum weight codewords of BCH codes. IEEE Trans. Inf. Theory 38(3), 960-973 (1992)
		\bibitem{RefJ2}Aly, S.A., Klappenecker, A., Sarvepalli, P.K.: On quantum and classical BCH codes. IEEE Trans. Inf. Theory 53(3), 1183-1188 (2007) 
		\bibitem{RefJ3}Berlekamp E.R.: Algebraic Coding Theory. McGraw–Hill Book, New York (1968)
		\bibitem{RefJ4}Berlekamp E.R.: Negacyclic codes for the Lee metric. In: Proceedings of the Conference on Combinatorial Mathematics and Its Applications, pp. 298–316. University of North Carolina Press, Chapel Hill (1968)
		\bibitem{RefJ5}Ding, C.: Parameters of several classes of BCH codes. IEEE Trans. Inf. Theory 61(10), 5322-5330 (2015)
		\bibitem{RefJ6}Ding, C., Du, X., Zhou, Z.: The Bose and minimum distance of a class of BCH codes. IEEE Trans. Inf. Theory 61(5), 2351-2356 (2015)
		\bibitem{RefJ7}Ding, C., Fan, C., Zhou, Z.: The dimension and minimum distance of two classes of primitive BCH codes. Finite Fields Appl. 45, 237-263 (2017)
		\bibitem{RefJ8}Ding, C., Li, C., Li, N., Zhou, Z.: Three-weight cyclic codes and their weight distributions. Discrete Math. 339(2), 415-427 (2016)
		\bibitem{RefJ9}Fang, W., Wen, J., Fu, F.: A q-polynomial approach to constacyclic codes. Finite Fields Appl. 47, 161–182 (2017)
		\bibitem{RefJ10}Hocquenghem, A.: Codes correcteurs d’erreurs, Chiffres (Paris) 2, 147-156 (1959)
		\bibitem{RefJ11}Grassl, M.: Bounds on the minimum distance of linear codes and quantum codes. Online available at http://www.codetables.de
		\bibitem{RefJ12}Gorenstein, D.C., Zierler, N.: A class of error-correcting codes in $p^m$ symbols. J. SIAM 9, 207-214 (1961)
		\bibitem{RefJ13}Guo, G., Li, R., Liu, Y., Wang, J.: A family of negacyclic BCH codes of length $n=\frac{q^{2m}-1}{2}$. Crypto. Commun. 12, 187-203 (2020)
		\bibitem{RefJ14}Gong, B., Ding, C., Li, C.: The dual codes of several classes of BCH codes. IEEE Trans. Inf. Theory 68(2), 953-964 (2021)
		\bibitem{RefJ15}Krishna, A., Sarwate, D.V.,: Pseudocyclic maximum-distance-separable codes. IEEE Trans. Inf. Theory 36(4), 880–884 (1990)
		\bibitem{RefJ16} Kai, X., Zhu, S.: New quantum MDS codes from negacyclic codes. IEEE Trans. Inf. Theory 59(2), 1193–1197 (2013)
		\bibitem{RefJ17} Kai, X., Zhu, S., Tang Y.: Quantum negacyclic codes. Phys. Rev. A 88(1), 012326 (2013)
		\bibitem{RefJ18}Li, S.: The minimum distance of some narrow-sense primitive BCH codes. SIAM J. Discrete Math. 31(4), 2530-2569 (2017)
		\bibitem{RefJ19}Liu, H., Ding, C., Li, C.: Dimensions of three types of BCH codes over ${\mathrm {GF}}(q)$. Discrete Math. 340(8), 1910-1927 (2017)
		\bibitem{RefJ20}Liu, Y., Li, R., Guo, L., Song, H.: Dimensions of nonbinary antiprimitive BCH codes and some conjectures. Discrete Math. (2023). https://doi.org/10.1016/j.disc.2023.113496
		\bibitem{RefJ21}Pang, B., Zhu, S., Sun, Z.: On LCD negacyclic codes over finite fields. J. Syst. Sci. Complex 31, 1065–1077 (2018)
		\bibitem{RefJ22}Wang, L., Lu, D., Zhu S.: Parameters of some BCH codes over $\F_q$ of length $\frac{q^m-1}{2}$. Adv. Math. Commun. (2023). doi: 10.3934/amc.2023007 
		\bibitem{RefJ23}Wang, X., Sun, Z., Ding, C.: Two families of negacyclic BCH codes. Des. Codes Cryptogr. (2023). https://doi.org/10.1007/s10623-023-01208-6
		\bibitem{RefJ24}Zhang, H., Cao, X.: Dimensions of some LCD BCH codes. arXiv:2305.06508 [cs.IT]
		\bibitem{RefJ25}Zhang, J., Li, P., Kai, X., Sun, Z.: Some results on BCH codes of length $\frac{q^m+1}{2}$. Adv. Math. Commun. (2023). doi: 10.3934/amc.2023010
		\bibitem{RefJ26}Zhang, Y., Liu, L., Li, L., Wu, T.: A class of LCD BCH codes of length $n=\frac{q^{m}+1}{\lambda}$. (2023). https://doi.org/10.21203/rs.3.rs-3023222/v1
		\bibitem{RefJ27}Zhu, S., Sun, Z., Kai, X.: A class of narrow-sense BCH codes. IEEE Trans. Inf. Theory 65(8) 4699-4714 (2019)
		\bibitem{RefJ28}Zhu, S., Z. Sun, P. Li, A class of negacyclic BCH codes and its application to quantum codes. Des. Codes. Crptogr. 86, 2139–2165 (2018)
		\bibitem{RefJ29}Zhu, H., Shi, M., Wang, X., Helleseth, T.: The q-ary antiprimitive BCH codes. IEEE Trans. Inf. Theory 68(3), 1683-1695 (2021)
	\end{thebibliography}
\end{document} 