\documentclass[12pt]{amsart}

\usepackage{amssymb, amsmath, xcolor}
\usepackage{import}
\usepackage{tikz}
\tikzset{filled/.style={minimum width=5pt,inner sep=0pt,circle,fill=black}}
\usetikzlibrary {arrows.meta}
\usepackage{ytableau}
\usepackage{subcaption}
\usepackage{mathtools}
\usepackage{todonotes}
\usepackage[utf8]{inputenc}
\usepackage{float}


\newtheorem{theorem}{Theorem}[section]
\newtheorem{lemma}[theorem]{Lemma}
\newtheorem{corollary}[theorem]{Corollary}
\newtheorem{conjecture}[theorem]{Conjecture}

\theoremstyle{definition}
\newtheorem{definition}[theorem]{Definition}
\newtheorem{example}[theorem]{Example}
\newtheorem{procedure}[theorem]{Procedure}

\theoremstyle{remark}
\newtheorem{remark}[theorem]{Remark}

\numberwithin{equation}{section}
\numberwithin{figure}{section}



\renewcommand{\mod}{\operatorname{mod}}
\newcommand{\YT}{\operatorname{YT}}
\newcommand{\M}{\operatorname{M}}
\newcommand{\R}{\operatorname{R}}
\newcommand{\mM}{\operatorname{mM}}
\newcommand{\Out}{\operatorname{Out}}
\newcommand{\In}{\operatorname{In}}
\newcommand{\N}{\mathbb{N}}
\DeclarePairedDelimiter\floor{\lfloor}{\rfloor}

%-----%-----%-----%-----%
%-----%-----%-----%-----%
%-----%-----%-----%-----%
%-----%-----%-----%-----%
%-----%-----%-----%-----%
%-----%-----%-----%-----%
%-----%-----%-----%-----%
%-----%-----%-----%-----%



%-----%-----%-----%-----%
%-----%-----%-----%-----%
\title[Young Tableaux Via Minors]{Young Tableau Reconstruction Via Minors}
%-----%-----%-----%-----%
%-----%-----%-----%-----%
\author[Erickson, Herden, Meddaugh, Sepanski, $\ldots$]{William Q. Erickson, Daniel Herden, Jonathan Meddaugh, Mark R. Sepanski, Cordell Hammon, Jasmin Mohn, Indalecio Ruiz-Bolanos}
%-----%-----%-----%-----%
%-----%-----%-----%-----%
\address{
All authors:
Department of Mathematics,
Baylor University,
Sid Richardson Building,
1410 S.~4th Street,
Waco, TX 76706, USA}
%-----%-----%-----%-----%
%-----%-----%-----%-----%
\email{will\_erickson@baylor.edu, daniel\_herden@baylor.edu,  jonathan\_meddaugh@baylor.edu, mark\_sepanski@baylor.edu}
%-----%-----%-----%-----%
%-----%-----%-----%-----%
\date{\today}
%-----%-----%-----%-----%
%-----%-----%-----%-----%



%-----%-----%-----%-----%
%-----%-----%-----%-----%
\begin{document}
%-----%-----%-----%-----%
%-----%-----%-----%-----%
%----- Keywords Subjclass -----%
\keywords{Young tableau, reconstruction, jeu de taquin, minor}
\subjclass[2020]{Primary: 05E10; Secondary: }
%-----%-----%-----%-----%
%-----%-----%-----%-----%


%-----%-----%-----%-----%
%-----%-----%-----%-----%
%----- Abstract -----%
\begin{abstract}
    The tableau reconstruction problem, posed by Monks (2009), asks the following.  Starting with a standard Young tableau $T$, a 1-minor of $T$ is a tableau obtained by first deleting any cell of $T$, and then performing jeu de taquin slides to fill the resulting gap. This can be iterated to arrive at the set of $k$-minors of $T$.  The problem is this: given $k$, what are the values of $n$ such that every tableau of size $n$ can be reconstructed from its set of $k$-minors?  For $k=1$, the problem was recently solved by Cain and Lehtonen.  In this paper, we solve the problem for $k=2$, proving the sharp lower bound $n \geq 8$.  In the case of multisets of $k$-minors, we also give a lower bound for arbitrary $k$, as a first step toward a sharp bound in the general multiset case.  \end{abstract}
%-----%-----%-----%-----%
%-----%-----%-----%-----%


%-----%-----%-----%-----%
%-----%-----%-----%-----%
\maketitle
\tableofcontents
\ytableausetup{centertableaux}
%-----%-----%-----%-----%
%-----%-----%-----%-----%



%-----%-----%-----%
%-----%-----%-----%
\section{Introduction}
%-----%-----%-----%
%-----%-----%-----%

In \cite{Monks2009}, Monks solved the \emph{partition reconstruction problem} posed by Pretzel and Siemons in \cite{Pretzel}: given a positive integer $k$, determine the values of $n$ such that each partition of $n$ can be reconstructed from its set of $k$-minors (i.e., the partitions obtained by deleting $k$ cells from the corresponding Young diagram).  This problem was originally motivated by the representation theory of the symmetric group $S_n$: Monks's solution thus determines when the character of an $S_n$-module can be recovered from its restriction to stabilizer subgroups of $S_n$.

In turn, Monks posed an analogue of the problem by asking the same question for standard Young tableaux \cite[Section 4.3]{Monks2009}.  These are Young diagrams whose cells are given a total ordering such that the cells increase along rows and columns.  With this added structure, Monks defined a natural analogue for $k$-minors in terms of Sch\"utzenberger's \emph{jeu de taquin} procedure.

Specifically, the \emph{tableau reconstruction problem} is the following.  Starting with a tableau $T$, we choose any cell to delete, and then perform the necessary sequence of jeu de taquin ``slides'' to fill the resulting gap until we obtain a new Young tableau, one cell smaller than the original; see Section \ref{sec:prelim} for details. We call this new tableau a \emph{$1$-minor} of $T$.  The \emph{set of $1$-minors} of $T$ collects the resulting $1$-minors for all possible choices of initial cells to delete.  The \emph{set of $k$-minors} of $T$ is then defined recursively to be the union of the sets of $1$-minors of the $(k-1)$-minors of $T$.  The problem is this: given $k$, what are the values of $n$ such that every tableau of size $n$ can be reconstructed from its set of $k$-minors?  For $k=1$, the problem was recently solved by Cain and Lehtonen \cite{CainLehtonen2022}, who proved reconstructibility for the sharp bound $n \geq 5$.  As the authors of the paper observe, their methods are quite specific to the $k=1$ case.

It is worth observing that for $k=1$, the process proposed by Monks for deleting cells can be interpreted as a generalization of Schüt\-zen\-ber\-ger's \emph{dual promotion operator} on tableaux \cite{Schutzenberger1972}.  In particular, the dual promotion  operator yields the $1$-minor obtained by deleting the upper-left cell in the tableau (assuming we ignore the largest cell in the result; see Remark~\ref{rem:promotion}).  Incidentally, by invoking the inverse of the dual promotion operator, we are able to give a highly efficient algorithm to reconstruct $T$ directly from a certain distinguished $1$-minor, as opposed to the inductive procedure in~\cite{CainLehtonen2022}; see Remark~\ref{rem:alternate proof M1}.

Our main result in this paper is a solution to the tableau reconstruction problem for $k = 2$.  In this case, we prove that $n \geq 8$ is the sharp lower bound for reconstructibility from $2$-minors, see Theorem~\ref{thm: k = 2 sharp}. For general $k$, we have found some evidence to suggest that the lower bound is $n \geq k^2 + 2k$, which arises as the size of a $(k+1) \times (k+1)$ tableau with one cell removed (see Figure~\ref{figure: k+1 square minus lower right corner} and Conjecture~\ref{conj:k^2 + 2k}).  We also consider the problem in terms of \emph{multisets} of $k$-minors; in this case, we are able to formulate a lower cubic bound for all $k$ (see Theorem~\ref{thm: exp lower bound}).  The software calculations we performed to verify our results have also led to a conjecture for multisets of minors; see Conjecture~\ref{conj:k+4}, where the bound on $n$ appears to be linear in $k$.

\section*{Acknowledgement}
The authors thank Daniel Bossaller for helpful suggestions on this project.

%-----%-----%-----%
%-----%-----%-----%
\section{Preliminaries}
\label{sec:prelim}
%-----%-----%-----%
%-----%-----%-----%


In this paper, we write $\N$ for $\mathbb{Z}_{\geq 1}$ and will use interval notation restricted to $\N$. For example, we will write $[1, n]$ for $\{1, 2, \ldots, n\}$.

For $n \in \N$, a \emph{partition} of $n$ is a weakly decreasing finite sequence $\lambda = (\lambda_1, \ldots, \lambda_m)$ of positive integers such that $\sum_i \lambda_i = n$.
 We call $n$ the \emph{size} of $\lambda$.
  The \emph{Young diagram} of shape $\lambda$ is a left-aligned array of cells with $\lambda_h$ boxes in the $h$th row, counting from the top. As an example, the Young diagram of shape $(5, 5, 4, 2, 1, 1)$ is given in Figure~\ref{figure: yd 554211}.

% Figure environment removed

Let $\lambda$ be a partition of $n$, and suppose $N\geq n$. A \emph{standard Young tableau} of shape $\lambda$, over the alphabet $[1, N]$, is obtained from the Young diagram of $\lambda$ by filling the cells with distinct elements of $[1, N]$ so that entries in each row and column are strictly increasing. In this paper, we will simply use the word \emph{tableau} to mean a standard Young tableau. We write $\YT(n,N)$ for the set of tableaux of size $n$ over $[1,N]$.  In this paper, usually $N = n$; in this case, we simply write $\YT(n)$
for the set of tableaux of size $n$ over $[1,n]$.
For example, Figure \ref{figure: yt 554211} gives an example of a tableau of shape $(5, 5, 4, 2, 1, 1)$ over $[1,18]$. We often identify a cell with its entry; we also say that a cell is \emph{labeled} by its entry.

% Figure environment removed

Given a tableau $T\in\YT(n)$, an \emph{outer corner} (\emph{OC} for short) is a cell of $T$ which is both the right end of a row and the bottom end of a column of $T$. For example, in Figure \ref{figure: yt 554211}, the OCs are $15$, $16$, $17$, and~$18$.

If $R$ is a collection of $k$ cells of $T$ we write $|R| = k$. Thus $|T| = n$.

If $m$ is an entry of a cell $c$ of $T$, we define its
\emph{outer area}, denoted by either $\Out(m)$ or $\Out(c)$, to be the collection of all cells of $T$ that are to the right of $m$ or below $m$. Figure \ref{figure: out example} gives an example of $\Out(m)$.

% Figure environment removed

We also define the \emph{inner area} of $m$, written as either $\In(m)$ or $\In(c)$, to be the set of cells that are both weakly left of $m$ and weakly above~$m$.
 Figure \ref{figure: in example} gives an example of $\In(m)$.  Note that all cells in $\In(m)$ have entries that are less than or equal to $m$.

% Figure environment removed


There is a natural way of deleting cells from a tableau using the process known as \emph{jeu de taquin} (introduced in  \cite{Schützenberger1977} in the context of rectifying semistandard skew tableaux). To describe the process, begin with $T \in \YT(n)$ and a cell filled with the entry $m$. Begin by deleting this cell, leaving an empty space. Next, iterate the following procedure until it terminates: if there exists a cell either directly to the right or directly below the empty space, slide the cell with the smaller entry into the position of the empty space. This terminates when there are no cells directly to the right or below the current empty space; in other words, the process necessarily terminates at an OC. Finally, subtract $1$ from each entry larger than $m$. The resulting tableau is an element of $\YT(n-1)$ and is denoted by $T - m$. The jeu de taquin works the same way for $T \in \YT(n,N)$ and produces an element of $\YT(n-1,N-1)$.
An example is given in Figure \ref{figure: cell deletion example}.

% Figure environment removed

For $T \in \YT(n)$ and $k \in \N$, a $k$\emph{-minor} of $T$ is a tableau formed by iteratively deleting $k$ cells from $T$ via jeu de taquin. We write
$\M_k(T)$ for the \emph{set of all $k$-minors} of $T$ and $\mM_k(T)$ for the \emph{multiset of all $k$-minors} of $T$.  Clearly if $k \geq n$, then $\M_k(T) = \varnothing$, so we assume $k < n$.
If $T'\in \M_k(T)$ contains an OC which is also an OC of $T$ (possibly with a different entry), then we call the cell a \emph{surviving outer corner} (\emph{SOC} for short).

It is known
\cite[Thm.~2.5]{Monks2009} that the shape of $T\in\YT(n)$ can be recovered from the shapes of the tableaux in $\M_k(T)$ when
\begin{equation} \label{equation: Monks shape determination bound}
    n \geq k^2 + 2k.
\end{equation}
It is known that this bound is sharp, although in general, for a given $k$, there exist some sporadic tableaux $T$ with $n<k^2 + 2k$ for which the shape of $T$ can still be recovered from $\M_k(T)$.

\begin{remark}
\label{rem:promotion}
As mentioned in the introduction, when the initial deleted entry $m$ is $1$, the process described above is essentially the dual promotion operator $\partial^*$ introduced in~\cite{Schutzenberger1972}.
(Elsewhere in the literature, confusingly, this operator is also called the promotion operator, or even the demotion operator.)
To be precise, $T-1 = \partial^*(T) - n$ for all $T \in \YT(n)$. There is also an inverse, denoted by $\partial$.
The operators $\partial$ and $\partial^*$ have been the subject of more recent research in \cite{Shimozono, Stanley, Rhoades, Ahlbach}, among others.
\end{remark}

\begin{remark}
\label{rem:alternate proof M1}
The invertibility of $\partial^*$ allows for a highly efficient algorithm to reconstruct $T \in \YT(n)$ from $\M_1(T)$ for all $n \geq 5$.
(This differs from the inductive method described in ~\cite{CainLehtonen2022}.)
Since $\M_1(T)$ determines the shape of $T$ for $n \geq 5$, one can determine the non-surviving OC in any minor; then since $\partial^*$ is invertible, in order to recover $T$, it suffices to identify $T-1 \in \M_1(T)$.
Define the \emph{initial string} of a tableau to be the maximal set of cells labeled $1, \ldots, \ell$ which form an unbroken row or column. We have $\ell \ge 2$. One easily checks the following:
\begin{itemize}
\item If $\M_1(T)$ contains a minor whose initial string is properly contained in the initial strings of all other minors, then this minor is $T-1 =T-2=\ldots =T-\ell$, where $\ell$ denotes the length of the initial string of $T$, and we are done. This case occurs whenever $\ell \geq 3$.

\item If there is no minimal initial string, then $\ell = 2$ and so the location of the entry $2$ in $T-1$ $( = T-2)$ differs from all other minors.
Hence as long as $|\M_1(T)\,| \geq 3$, we can immediately identify $T-1$ and we are done.
If $|\M_1(T)\,| = 2$, then $T$ must have the shape $(n-1,1)$ or its transpose, with $T-1$ ${( = T-2)}$ being the unique minor consisting of one single row or one single column, respectively.
\end{itemize}
\end{remark}

%-----%-----%-----%
%-----%-----%-----%
\section{General Identification Lemmas}
%-----%-----%-----%
%-----%-----%-----%

In this section we give lemmas that can help reconstruct parts of tableaux from their minors.

\begin{lemma} \label{lemmma: construct In(m) when m in same place}
    Let $n,k\in\N$ and $T\in\YT(n)$. Suppose $m$ is a known entry of a cell $c$ in $T$ that is an OC. If there exists $T'\in\M_k(T)$ in which the cell $c$ exists and is labeled by $m$, then all cells of $T$ with entries less than $m$ are reconstructible from $\M_k(T)$. In particular, $\In(m)$ is reconstructible from $\M_k(T)$.
\end{lemma}

 \begin{proof}
    First note that, since $c$ is an OC, no cell may slide into its position during cell deletion.
    Now, in the construction of $T'$, if any cell had been deleted that was less than $m$, then the value of $m$ would have been reduced in $T'$. Since this did not happen, only cells larger than $m$ were deleted.
    Since every entry of $\In(m)$ in $T$ is less than $m$, only cells in $\Out(m)$ that were larger than $m$ were deleted. Any change resulting from deleting such cells is confined to cells in $\Out(m)$ that are greater than $m$. In particular, this means that the entries of $T$ that started out less than $m$ are all unchanged in $T'$ and therefore recoverable.
 \end{proof}

Slightly more generally, the same proof gives the following result.

\begin{lemma} \label{lemmma: general construct In(m) when m in same place and nothing could slide}
    Let $n,k\in\N$ and $T\in\YT(n)$. Suppose $m$ is a known entry of a cell $c$ in $T$ and it is known that no other cell can slide to $c$ and retain the same entry $m$ after $k$ deletions. If there exists $T'\in\M_k(T)$ in which the cell $c$ exists and is labeled by $m$, then all cells of $T$ with entries less than $m$ are reconstructible from $\M_k(T)$. In particular, $\In(m)$ is reconstructible from $M_k(T)$.
\end{lemma}





% %-----%-----%-----%
% %-----%-----%-----%
% \section{Location of  $[(k+1)^2, n]$}
% %-----%-----%-----%
% %-----%-----%-----%


\begin{definition}
    Let $n \in \N$ and $T \in \YT(n)$. Write $$\R_n T$$ for the tableau in $\YT(n-1)$ obtained by removing the cell with label $n$ from $T$. (Throughout the paper, ``remove'' carries this obvious sense, while ``delete'' refers to the jeu de taquin process. Note, however, $R_n T = T - n$.)
 More generally for $d \in [1, n]$, write $\R_{[d, n]} T$ for the tableau $\R_{d}(\R_{d + 1}(\ldots (\R_n T)\ldots))$. We also allow $\R_{[d, n]}$ to be applied to a set of tableaux by applying $\R_{[d, n]}$ to each tableau in the set.
\end{definition}

In \cite[Lemma 3.3]{CainLehtonen2022}, it is shown that
\begin{equation} \label{equation: removal of n and M1}
    \M_1(\R_n T) = \R_{n-1} \M_1(T)
\end{equation}
for $T \in \YT(n)$. This result generalizes.

\begin{lemma} \label{lemma: removal of top elements and Mk}
    Let $n,k \in \N$, let $d \in [k+1,n]$, and let $T \in \YT(n)$. Then
        \[ \M_k (\R_{[d, n]} T) = \R_{[d-k, n-k]} \M_k (T). \]
\end{lemma}

\begin{proof}
    By iterating, it suffices to show that
    \[ \M_k (\R_{n} T) = \R_{n-k} \M_k (T). \]
    We show this equality via induction on $k$.
    The base case of $k = 1$ is already done in \eqref{equation: removal of n and M1}. Assuming the result is known up to $k - 1$, we simply calculate that
    \begin{align*}
        \M_k(\R_n T) &= \bigcup_{T' \in \M_{k-1}(\R_n T)} \M_1( T' ) \\
                    &= \bigcup_{T' \in \R_{n-k+1}\M_{k-1}(T)} \M_1( T' ) \\
                    &= \bigcup_{T'' \in \M_{k-1}(T)} \M_1( \R_{n-k+1} T'' ) \\
                    &= \bigcup_{T'' \in \M_{k-1}(T)} \R_{n-k} \M_1( T'' ) \\
                    &= \R_{n-k} \left(\bigcup_{T'' \in \M_{k-1}(T)} \M_1( T'' )\right)
                    = \R_{n-k} \M_k(T).\qedhere
    \end{align*}
\end{proof}



\begin{lemma}
\label{lemma:location of k2 + 2k through n}
    Let $n,k \in \N$ and $T \in \YT(n)$. If $n \geq k^2 + 2k + 1$, then the location in $T$ of each entry in $[(k+1)^2, n]$ is determined by $\M_k(T)$.
\end{lemma}

\begin{proof}
    Recall from \eqref{equation: Monks shape determination bound} that the shape of $T$ is determined by (the shapes in) $\M_k(T)$ when $n \geq k^2 + 2k$.
    When $n - 1 \geq k^2 + 2k$, the shape of $\R_n T$ is determined by $\M_k (\R_n T)$ which, in turn, is the same as $\R_{n-k} \M_k(T)$,
    by Lemma \ref{lemma: removal of top elements and Mk}. Since taking the complement of the shape of $\R_n T$ in the shape of $T$ gives us the location of $n$ in $T$, it follows that $\M_k(T)$ determines the location of $n$ when $n \geq k^2 + 2k + 1$.

    Next, when $n - 2 \geq k^2 + 2k$, the shape of $\R_{[n-1, \: n]} T$ is determined by $\M_k (\R_{[n-1, n]} T)$ which, in turn, is the same as $\R_{[n-k-1, \: n-k]} \M_k(T)$. Again by comparing the shapes of $\R_n T$ and $\R_{[n-1,\: n]} T$, it follows that $\M_k(T)$ determines the location of $n-1$ when $n \geq k^2 + 2k + 2$.

    Straightforward induction demonstrates that the location of $[m, n]$ is determined by $\M_k(T)$ when $n \geq k^2 + 2k + 1 + n - m.$ In particular, this is minimally satisfied by $m = k^2 + 2k + 1$ when $n \geq k^2 + 2k + 1.$
\end{proof}






%-----%-----%-----%
%-----%-----%-----%
\section{General Cubic Lower Bound}
%-----%-----%-----%
%-----%-----%-----%



\begin{theorem} \label{thm: exp lower bound}
    Let $n,k \in \N$ and $T \in \YT(n)$. The multiset $\mM_k(T)$ determines $T$ when
    \begin{equation} \label{eq:4.1}
    2 \binom{n - k^2 -2k}{k} > \binom{n}{k}.
    \end{equation}
    In particular, this is satisfied when
    \[ n > k^2 + 3k - 1+\frac{k^2 + 2k}{2^{\frac{1}{k}} - 1}, \]
    which gives
    \[ n > \frac{k^3+2k^2}{\ln 2}+\frac {k^2}2 +2k-1 + \frac{\ln 2}{12} (k+2) \]
    as a cubic lower bound.
\end{theorem}

\begin{proof}
    From Theorem \ref{lemma:location of k2 + 2k through n}, we know that the location in $T$ of each entry in $[(k+1)^2, n]$ is determined by $\M_k(T)$ when $n \geq k^2 + 2k + 1$. It remains to find the location of $1, 2, \ldots, k^2 + 2k$.

    For $m \in [1, k^2 + 2k]$, the location of $m$ remains fixed in any $k$-minor of $T$ when the deleted cells are chosen from $[m + 1, n]$. Thus there are at least $(n - m) (n - m - 1) \ldots (n - m - k + 1)$
    minors in $\mM_k(T)$ with $m$ fixed. As there are $n (n-1) \ldots (n-k+1)$ minors in total, the location of $m$ may be identified as the most frequent location of $m$ in the multiset of $k$-minors when
    \[ \binom{n - m}{k} > \frac{1}{2} \binom{n}{k}. \]
    In particular, if the above inequality is satisfied for $m = k^2 + 2k$, it is satisfied for all $m\le k^2 + 2k$.

    To give an explicit, non-sharp lower bound, observe that
    \[\frac{n-k^2-2k-j}{n-j} = 1-\frac{k^2+2k}{n-j}\geq  1-\frac{k^2+2k}{n-k+1}\]
    for $0\leq j \leq k-1$. So, to find the location of $1, 2, \ldots, k^2 + 2k$, it suffices to have $( 1-\frac{k^2+2k}{n-k+1})^k > \frac{1}{2}$. It is straightforward to rewrite this as
    \begin{align*}
    n & > k-1+\frac{k^2+2k}{1 - 2^{-\frac{1}{k}}}\\ & = k-1 + \frac{2^{\frac{1}{k}}(k^2 + 2k)}{2^{\frac{1}{k}} - 1}= k^2 + 3k - 1+\frac{k^2 + 2k}{2^{\frac{1}{k}} - 1}.
    \end{align*}
    From the Taylor series of the function $f(x)=2^x$ one easily checks
    \[\frac{x}{2^x-1}\le \frac1{\ln 2}-\frac x2 + \frac{\ln 2}{12} x^2\]
    for $x\ge 0$, which gives
    \[k^2 + 3k - 1+\frac{k^2 + 2k}{2^{\frac{1}{k}} - 1}\le \frac{k^3+2k^2}{\ln 2}+\frac {k^2}2 +2k-1 + \frac{\ln 2}{12} (k+2).\qedhere\]
\end{proof}

It is easy to check with Equation \eqref{eq:4.1} that the multiset $\mM_2(T)$ determines $T$ when $n \geq 28$. In fact, we will see below in Theorem \ref{thm: k = 2 sharp} that $n \geq 8$ is the sharp lower bound for determining $T$ by $\M_2(T)$. As a result, the above cubic bound is far from sharp.






%-----%-----%-----%
%-----%-----%-----%
\section{Improved Location of $n$}
%-----%-----%-----%
%-----%-----%-----%





For $T \in \YT(n)$, Lemma \ref{lemma:location of k2 + 2k through n} shows that $\M_k(T)$ determines the location of $n$ when $n \geq k^2 + 2k + 1$. This section shows the same is true when $n = k^2 + 2k$.


\begin{lemma} \label{lemma: outer corner counting}
    Let $n,k\in\N$ and $T\in\YT(n)$ with OCs $c_1, \ldots, c_\ell$.
    Suppose $|\Out(c_j)| \leq k $ for at most one OC. Then the location of $n$ is recoverable from $\M_k(T)$.
\end{lemma}

\begin{proof}
    Let $x_1, \ldots, x_\ell$ denote the entries of the cells $c_1, \ldots, c_\ell$, respectively.

    Suppose first that $|\Out(c_j)| \geq k + 1$ for all $j$, $1\leq j \leq \ell$. Then by deleting $k$ cells in $\Out(c_j)$, there exists $T'\in \M_k(T)$ in which the cell $c_j$ is a SOC. Find the minimum entry $m_j$ in $c_j$ among all $T'\in\M_k(T)$ in which $c_j$ survives. Since nothing can slide into the cell $c_j$, the entry there in $T'$ equals
    $x_j$ minus the number of deleted cells which were less than $x_j$.

    If $x_j = n$, then $m_j = n - k$ as every other cell is less than $n$.
    If $x_j < n - k$, then certainly $m_j < n - k$.

    If $n - k \leq x_j < n$, only the entries $x_j + 1, x_j + 2, \ldots, n$ are larger than $x_j$ in $\Out(c_j)$. Thus there are at least
    $d_j = k + 1 - (n - x_j)$ entries in $\Out(c_j)$ that are less than $x_j$. Choose any $k$ entries of $\Out(c_j)$ that include at least $d_j$ entries that are less than $x_j$. This results in a $T'$ for which $c_j$ survives and has an entry of at most
    $x_j - d_j = n - k - 1$. Thus $m_j < n - k$. As a result, $x_j$ is $n$ if and only if $m_j = n - k$.

    This allows the location of $n$ to be determined when $|\Out(c_j)| \geq k + 1$ for all $j$. Moreover, if all but one of the $|\Out(c_j)| \geq k + 1$, then the same analysis applies to all but one of the $x_j$. As a result, $n$ may be located if it is one of these $x_j$. If it is not one of these, as $n$ must lie in an OC, it must be the only remaining $x_j$.
\end{proof}







If at least two OCs $c_j$ have $|\Out(c_j)| \leq k$, it turns out that the configuration for $T$ is very limited, at least when $n \geq k^2 +2k$, which is the lower bound for recovering the shape of $T$ from the shapes of $\M_k(T)$, by Equation \eqref{equation: Monks shape determination bound}.



\begin{lemma} \label{lemma: bad shape with small outer area}
    Let $n,k\in\N$ and $T\in\YT(n)$ with OCs $c_1, \ldots, c_\ell$. Suppose $|\Out(c_j)| \leq k$ for at least two OCs. Then $|T| \leq k^2 +2k$. Equality holds if and only if $T$ has shape $((k+1)^k, k)$, \emph{i.e.}, the shape of a $(k+1)\times (k+1)$ square with the lower-right cell removed; see Figure~\ref{figure: k+1 square minus lower right corner}.
\end{lemma}
% Figure environment removed

\begin{proof}
    Let cells $c, d$ be two OCs of $T$ with outer area at most $k$. We will give an algorithm that modifies $T$ by moving cells and possibly increasing $|T|$ while retaining at least two OCs with outer area at most $k$. The algorithm will end with the shape $((k+1)^k, k)$ to finish the proof.

    We may assume that $c$ is below $d$. Begin by moving all cells of $T$ that are not in either $\In(c)$ or $\In(d)$ below $\In(c)$ in the shape of a rectangle of width $1$, see Figure \ref{figure: move cells pic 1}. Write $T_1$ for the new Young diagram.
    % Figure environment removed

    Note that $|T_1|=|T|$. If $c$ is not in the first column, then $c,d$ remain OCs with $|\Out(c)|$ and $|\Out(d)|$ unchanged.
    If $c$ is in the first column, then $d$ remains an OC with $|\Out(d)|$ unchanged. However, the other OC shifts to the bottom of the first column. Its outer area may decrease. In this case, relabel this new OC as $c$.

    Next, let $p$ be the number of cells below $d$ and $q$ be the number of cells to the right of $c$. Form the Young diagram $T_2$ to be of shape
    $((p+1)^q,p)$, see Figure \ref{figure: move cells pic 2}.
    % Figure environment removed

    In $T_2$, label the bottom OC $c$ and the top OC $d$. Observe that $|T_2|=pq+p+q \geq |T_1|$,  and that
    $|\Out(c)|$ and $|\Out(d)|$ do not change.

    Finally, look at all possible Young diagrams consisting of a rectangle minus its lower-right cell, whose two outer areas are at most $k$. The maximum area is achieved for $p=q=k$ by the shape $((k+1)^k, k)$.
\end{proof}





We are left with discussing the YT with shape $((k+1)^k, k)$ that arises in Lemma \ref{lemma: bad shape with small outer area}, see Figure \ref{figure: k+1 square minus lower right corner}. We start with an auxiliary result.

\begin{lemma} \label{lemma: movement of n-k in column/row}
    Let $k\in\N$, $n = (k+1)^2 -1$, and $T\in\YT(n)$ with shape $((k+1)^k, k)$. Then the value of each OC is at least $n-k$. The set $\M_k(T)$ determines whether each OC sits in $[n-k,n-2]$, or $[n-1,n]$. Moreover, in the first case, the exact value of the OC is determined.
\end{lemma}

        % Figure environment removed

\begin{proof}
    $T$ has the shape of a $(k+1) \times (k+1)$ square minus its lower-right cell; see Figure \ref{figure: T where k=4}.
    Label the lower OC as $x$ and the upper OC as $y$. Observe that $n$ must be one of them. Write $X$ for the set of cells of $T$ that are in the same row as $x$ and $Y$ for the set of cells of $T$ in the same column as $y$. As the analyses for $x$ and $y$ are similar, we will work only with~$y$.

    The fact that each OC is at least $n-k$ follows from the fact that elements larger than $y$ can only appear in the outer area $\Out(y) =X$, where $|X|=k$.

    First observe that when $y$ is either $n$ or $n-1$, it is possible to find some $T'\in\M_k(T)$ with an $n-k$ in any cell of $Y$ by first deleting $n$ if $y= n-1$ and then deleting upper cells in $Y$ followed by cells in $X$.

    If $y=n-2$, then finding some $T'\in\M_k(T)$ with an $n-k$ in a cell of $Y$ would require some cell of $Y$ to be labeled $n-i$ with $2\leq i \leq k$ at the start of the deletion process, where the label becomes an $n-k$ in $Y$ by the end of the process.
    This requires $k-i$ cells to be deleted which are less than $n-i$ and all $i$ cells to be deleted which are greater than $n-i$, where some cells above $n-i$ may slide out of $Y$ or be deleted. Since the cells larger than $n-i$ will not cause $n-i$ to slide, the farthest $n-i$ can slide and become an $n-k$ is $k-i$ cells up.
    When $i=2$, $n-2$ sits at the bottom of $Y$ and, at most, could result in a label $n-k$ up to cell number $1+(k-2)=k-1$ of $Y$, counting from the bottom. When $i>2$ and $n-i$ sits in the $j$th cell of $Y$, counting from the bottom, then $i\geq 1+j$. In this case, at most, the label $n-k$ could also only reach cell number $j + (k-i) \leq k-1$ of $Y$.
    Moreover, by deleting the appropriate number of cells in $Y$ and $X$, an $n-k$ may be achieved in any of the cells of $Y$ from the bottom up to position $k-1$.

    A similar analysis holds for $y=n-i_0$, $2\leq i_0 \leq k$ except that a label $n-k$ will only appear in any of the cells of $Y$ from the bottom up to position $k-i_0 +1$.
\end{proof}

In addition, for $k\ge 2$, we can show that the location of $n$ in $T$ is determined by the set of $k$-minors $\M_k(T)$.

\begin{theorem} \label{theorem: n position}
    Let $k\in\N_{\ge 2}$, $n = (k+1)^2 -1$, and $T\in\YT(n)$ with shape $((k+1)^k, k)$. Then $\M_k(T)$ determines the location of $n$.
\end{theorem}

\begin{proof}
    By Lemma \ref{lemma: movement of n-k in column/row}, we are done if any of the OCs of $T$ sits in $[n-k,n-2]$.
    Thus, it remains to determine the location of $n$ when $n-1$ and $n$ are the OCs of $T$. Let $\mathcal{L}$ be the set of cells, $c$, satisfying:
    \begin{itemize}
    \item[$(1)$] For all $T'\in\M_k(T)$ for which $c$ exists, its value is at least $n-2k$
    % (2) there exists $T'\in\M_k(T)$ for which $c$ has the value $n-k$,
    and
    \item[$(2)$] there exist $T'\in\M_k(T)$ for which $c$ has the value $n-2k$.
    \end{itemize}

    First of all, we claim that the cell $c^*$ of $T$ holding $n-k$ lies in $\mathcal{L}$. Condition (1) is automatic here.
    % Condition (2) is achieved by deleting everything in $[n-k+1, n]$.
    For Condition (2), observe that $c^*$ fails to be an OC when $k\ge 2$. Thus, the size of the complement of $\In(c^*)$ is at least $2k$. Since there are exactly $k$ elements greater than $n-k$, this leaves at least $k$ elements smaller than $n-k$ in the complement of $\In(c^*)$ that can be deleted without moving the cell $c^*$.

        % Figure environment removed

    Next, we claim that if $c$ is not the cell adjacent to both OCs (see Figure \ref{figure: critical cell}), then $c \in \mathcal{L}$ if and only if its value is $n-k$. Suppose $c \in \mathcal{L}$ and $c$ is not adjacent to both OCs. Write $i_0$ for its value.
    If $i_0 > n-k$, then property (2) would be violated. Thus $i_0 \le n-k$. In particular, $c$ is not an OC.
    Next, observe that the size of the complement of $\In(c)$ is at least $2k+1$.
    If $i_0 < n-2k$, deleting any $k$ of the cells in the complement of $\In(c)$ would violate number (1).
    If $i_0 = n-k-i$ for $1\leq i \leq k$, then the complement of $\In(c)$ contains at least $(2k+1)-(k+i) = k + 1 - i$ elements smaller than $i_0$. Deleting $k$ elements from the complement of $\In(c)$, including at least $k+1-i$ that are smaller than $i_0$, reduces the cell's value to at most $n-2k-1$, which violates (1). Therefore, $i_0 = n - k$.

    As a result, it is possible to tell the location $c^*$ of $n-k$ in $T$. If there is a cell in $\mathcal{L}$ that is not adjacent to both OCs, $n-k$ is there. Otherwise, it is in the cell adjacent to both OCs.

    Now pick some $T'\in\M_k(T)$ for which $c^*$ has the value $n-2k$. Any such $T'$ results from $T$ by deleting $k$ cells less than $n-k$ without moving the cell $c^*$. During this deletion process, the entries $n-1$ and $n$ may be moved and are reduced by $k$. However, their relative position is preserved in the following sense.
    If $n-1$ begins in the lower OC of $T$, then, after $i$ deletions, $n-1-i$ will always be in a row weakly below the row of $n-i$. If $n-1$ begins in the upper OC of $T$, then $n-1-i$ will always be in a row strictly above the row of $n-i$. These statements are easily verified by induction.
    For example, if $n-1-i$ is in the same row as $n-i$ after $i$ deletions, then $n-1-i$ must be in the cell immediately to the left of $n-i$. After the next deletions, $n-1-(i+1)$ can only be in a row above $n-(i+1)$ if the jeu de taquin process moves the cell with entry $n-1-i$ up.
    However, in that case, the box above $n-i$ is smaller than $n-1-i$ and would slide over to the left, a contradiction. The other cases are similar.

    As a result, the relative positioning of $n-1-k$ and $n-k$ in $T'$ determines the original positioning of $n-1$ and $n$ in $T$. In particular, $n$ is in the lower OC of $T$ if and only if $n-k$ is in a row strictly below $n-1-k$ in $T'$.
\end{proof}


Combining these results gives the following.


\begin{corollary} \label{cor: location of n}
    Let $n,k\in\N$, $k\ge 2$, and $T\in\YT(n)$.
    Then $\M_k(T)$ determines the location of $n$ when $n \geq k^2 + 2k$.
\end{corollary}













%-----%-----%-----%
%-----%-----%-----%
\section{Sharp Bound for $k=2$}
\label{sec:sharp bound k=2}
%-----%-----%-----%
	%-----%-----%-----%
	
	In this section, we prove that $\M_2(T)$ determines $T$ when $n \geq 8$.
	This result is sharp for $k = 2$ as the mapping $T \longmapsto \M_2(T)$ is not injective when $n = 7$.
	For example, the $2$-minors of the tableaux in Figure \ref{figure: same 2-minors} are identical.
	% Figure environment removed
	Of note, the mapping $T \longmapsto \mM_2(T)$ is injective for $n$ equal to $6$ and $7$. This can be verified by a straightforward computer calculation as multisets are easy to distinguish. Injectivity for this mapping fails at $n = 5$.
	
	
	
	\begin{theorem} \label{thm: k = 2 sharp}
		Let $n \in \N$ and $T \in \YT(n)$. Then $\M_2(T)$ determines $T$ when $n \geq 8$.
	\end{theorem}
	
	By Lemma \ref{lemma:location of k2 + 2k through n}, $\M_2(T)$ determines the location of $[9, n]$. Thus, only the location of $[1, 8]$ remains to be determined.
	Lemma \ref{lemma: removal of top elements and Mk} shows that $\M_2(R_{[9, n]} T)$ equals $R_{[7, n-2]} \M_2(T)$. Thus Theorem \ref{thm: k = 2 sharp} follows immediately from the following.
	
	\begin{lemma} \label{lemma: n = 8 lemma for k = 2}
		Let $T \in \YT(8)$. Then $\M_2(T)$ determines $T$.
	\end{lemma}

The proof of Lemma \ref{lemma: n = 8 lemma for k = 2} (which will occupy the rest of this section) breaks naturally into a number of cases according to the shape of $T$, which is known from $M_2(T)$. However, we can significantly reduce the number of cases to check by \cite[Lemma 3.6]{Monks2009}.

\begin{lemma}[\cite{Monks2009}] \label{Monks 3.6}
	Let $n,k\in\N$ and $T\in\YT(n)$. Then $M_2(T)$ determines the shape of $T$ if $n$ cannot be expressed as $n=(a+1)b+c-1$ for $a,b,c\in\N$ satisfying $a\le c\le k$ and $b+ (c\, \mod a) \le k$.
\end{lemma}

Applying Lemma~\ref{Monks 3.6} to the choice $n=6, k=2$ gives that $M_2(R_{[7,8]}T)$ determines the shape of $R_{[7,8]}T$ (hereafter referred to as $T'$). Since 7 and 8 are in the complement of $T'$ and the location of 8 is determined from $M_2(T)$ by Corollary \ref{cor: location of n},  the location of 7 in $T$ is also determined. Thus it only remains to show that $M_2(T)$ determines $T'$. Recall that Lemma  \ref{lemma: removal of top elements and Mk} gives the 2-minors of $T'$ as $M_2(T')=R_{[5,6]}M_2(T)$.
	
	Up to symmetry, there are 6 shapes of $T'$ to consider:
	$(6)$, $(5,1)$, $(4,2)$, $(4,1,1)$, $(3,3)$, and $(3,2,1)$.  %As there is only one tableaux of shape $(6)$,  if $T'$ is of shape $(6)$, $T'$ is determined by shape alone, which is known from $M_2(T')$.
	
	
	
	
\begin{lemma} \label{hanging shapes}
	Let $T'$ be of shape $(6)$, $(5,1)$, or  $(4,1,1)$. Then $\M_2(T')$ determines $T'$.
\end{lemma}

\begin{proof}
Let $T'$ have top row of length $r$. If $r=6$, there is nothing to prove, as there is only one tableau of shape $(6)$. Otherwise, label the top row entries of $T'$ as $\alpha_1, \ldots, \alpha_{r}$ and look at any $S \in \M_2(T')$ with a top row of length $r-2$. Since nothing can slide up during deletion, the only possible values that could appear in $S$ in the box originally holding $\alpha_i$, for $1 \leq i \leq r - 2$, are $\alpha_i, \alpha_{i+1} -1$, and $\alpha_{i+2} -2$, where $\alpha_i\le \alpha_{i+1} -1 \le \alpha_{i+2} -2$. By deleting the entries $\alpha_{r-1}$ and $\alpha_r$ from $T'$, the minimal value of $\alpha_i$ is always obtained. Therefore $\alpha_i$ is determined by $\M_2(T')$ for all $1 \leq i \leq r - 2$.

If $2 \leq i \leq r - 2$, then deleting $\alpha_{2}$ and $\alpha_{3}$ from $T'$ yields a 2-minor which achieves the maximal value of $\alpha_{i + 2} - 2$ in position $i$ of the first row, determining $\alpha_{i+2}$ to be 2 more than this maximal entry. Therefore, $\alpha_i$ is determined by $\M_2(T')$ also for all $4 \leq i \leq r$.

When $r=5$, i.e., $T'$ is of shape $(5,1)$, the entire top row of $T'$ is now determined. The single remaining entry is also determined, and therefore all of $T'$ is determined from $\M_2(T')$.

When $r=4$, it remains to determine $\alpha_3$. Once $\alpha_3$ is found, the remaining values are known and their positions are determined by strict monotonicity of the first column. If $\alpha_4=\alpha_2+2$,  then $\alpha_3=\alpha_2+1$ is uniquely determined by strict monotonicity across the first row, and therefore $T'$ is determined from $\M_2(T')$.

It remains to consider three cases: $\alpha_2=3$ and $\alpha_4=6$; $\alpha_2=2$ and $\alpha_4=5$; and $\alpha_2=2$ and $\alpha_4=6$.

Let us first consider $\alpha_2=3$ and $\alpha_4=6$. In this case $\alpha_1=1$, the middle row entry is 2, and the bottom row entry is either 4 or 5 (whichever $\alpha_3$ is not). If the bottom row entry is 5, then the 2-minor arising from deleting 6 and 2 will have its 4 in the bottom row. If the bottom row entry of $T'$ is 4, there is no such 2-minor -- such a minor would require deleting the first entry of the top row or the unique entry of the middle row, which would result in shifting the 4 up, but it would then be relabeled as something less than 4. Thus, $\M_2(T')$ determines $T'$ in this case.

Now, let us consider $\alpha_2=2$ and $\alpha_4=5$. Note that if $\alpha_3=3$, the only entry that can occur as the second entry of the top row of a 2-minor of shape $(3,1)$ is a 2, as either the 2 has not moved, or either the 1 or 2 has been deleted, and the 3 has slid over and been relabeled 2. However, if $\alpha_3=4$, the value 3 can and does appear as the entry in this position of the 2-minor arising from deleting $6$ and $2$. Thus $\M_2(T')$ determines all of $T'$.

Finally, we consider $\alpha_2=2$ and $\alpha_4=6$. First recall that the only possible values that can appear as the second entry in the top row of some 2-minor of $T'$ of shape $(2,1,1)$ are $2$, $\alpha_3-1$, and $4$. By deleting various combinations of the top row entries $\alpha_2,\alpha_3,$ and $\alpha_4$, we see that each of these values is achievable. Thus, if there are three distinct entries that occur in this position of a 2-minor, we have $2<\alpha_3-1<4$, so $\alpha_3-1=3$, i.e., $\alpha_3=4$. In this case, $T'$ is determined. Otherwise, there are only two distinct values in that position, and thus either $\alpha_3=3$ or $\alpha_3=5$. Then, by examining possible bottom row entries of a 2-minor of shape $(3,1)$ (as in the case $\alpha_2=3$ and $\alpha_4=6$), we can determine $\alpha_3$. In particular, $4$ appears among the possible entry values in this position if and only if $\alpha_3=3$. Thus again $T'$ is determined.
\end{proof}



\begin{lemma}
	Let $T'$ be of shape  $(3,2,1)$. Then $M_2(T')$ determines $T'$.
\end{lemma}

\begin{proof}
	Note that $T'$ has three OCs, and that the entries in these OCs are $5$ and $6$, and one of $3$ and $4$. Note that only one OC has outer area less than or equal to 2. As such, by Lemma \ref{lemma: outer corner counting}, the location of the 6 is known.
	
	Fix an OC and consider the set of entries that occur in that SOC in a 2-minor. It is easy to see that 4 is a possibility if and only if the OC was originally occupied by $4,5$ or 6.
	
	Thus, if there is an SOC in which no 2-minor has a 4, this cell must be occupied in $T'$ by 3. By strict monotonicity, this cannot be the OC in the middle row, and thus either the first column or top row (depending on which OC is under consideration) is determined, as the other entries in this column/row must be 1 and 2. Since the location of 6 is known, 5 must be in the remaining OC, 4 in the sole remaining cell, and $T'$ is fully determined.

	Otherwise, the OCs of $T'$ are occupied by $4,5,$ and 6, and the entries 2 and 3 must be located in the two cells of $T'$ right next to the 1 in the upper left corner of $T'$. In order to proceed, note that any 2-minor of $T'$ in which 4 remains labeled 4 must preserve every label less than 4 by Lemma \ref{lemmma: construct In(m) when m in same place}. Hence there is exactly one such 2-minor of $T'$ (obtained by deleting the 5 and 6), and all its entries agree with those of $T'$. On the other hand, for every OC of $T'$, deleting the other two OCs results in a 2-minor where this OC is an SOC labeled 4 with every label less than 4 preserved. Thus, there exists at least one OC of $T'$ with a unique 2-minor where this OC is labeled 4, and any such 2-minor will provide us with the correct location of every label less than 4 in $T'$. By symmetry, we may focus now on the case that the label 2 is located in the cell below 1, see Figure \ref{figure: (3,2,1) part 1}. If 2 is located in the cell to the right of 1 instead, just flip the tableau over.

    % Figure environment removed

    Since the location of 6 is known, it remains to locate the entries 4 and 5 in $T'$. We are left with considering three pairs of tableaux as depicted in Figure \ref{figure: (3,2,1) part 2}. Each pair corresponds to a different location of the entry 6. In each case we need to distinguish between the left and right tableau on basis of their 2-minors.

    % Figure environment removed

    We start with the top pair in Figure \ref{figure: (3,2,1) part 2}. Note that the left tableau has a 2-minor of shape $(3,1)$ with 3 as the bottom row entry (obtained by deleting the 2 and 6). The right tableau has no such 2-minor. In particular, note that 4 would be an SOC for such a 2-minor. Thus, Lemma \ref{lemmma: construct In(m) when m in same place} applies and the 2-minor must preserve every label less than~4, a contradiction.

    For the other two pairs, similar arguments apply. For the middle pair, the left
    tableau has again a 2-minor of shape $(3,1)$ with 3 as the bottom row entry (obtained by deleting the 2 and 6). For the bottom pair, the left tableau has a 2-minor of shape $(2,1,1)$ with 2 as the right column entry (obtained by deleting the 2 and 6).
    % On the contrary, there is a 2-minor in which 6 has been reduced to 4 and has not slid in which the inner area is preserved (obtained by deleting 4 and 5) and a 2-minor in which it is not. This is obtained by deleting the entry 1, whose jeu de taquin process results in the 2 sliding into the top left position and either 4 or 5 sliding into the position vacated by 2. These are then reduced to 1 and 3, respectively, while the 3 in $T'$ has been reduced to 2. In particular, $T'$ has the positions of 2 and 3 reversed.
	%
	% With this observation in hand, we can complete the determination of $T'$ as follows. There is exactly one OC with the property that there is a unique 2-minor with a 4 in that location. This 2-minor is equal to $R_{[5,6]}T'$. Since the location of 6 is known, the sole remaining cell must contain 5, and $T'$ is fully determined.
 \end{proof}

It is worth pointing out that, in fact, $M_2(T')$ alone is enough to determine $T'$ in the above two lemmas, establishing that there are many elements of $YT(6)$ which are recoverable from their 2-minors. In the remaining lemmas, information about $M_2(T')$ is not sufficient, and we will make use of the additional information provided by $M_2(T)$.

\begin{lemma} \label{3-3}
	Let $T'$ be of shape $(3,3)$. Then $M_2(T)$ determines $T'$.
\end{lemma}

\begin{proof}
	
	First, consider the case that $T$ is of shape $(5,3)$. Label the first three entries of the top row as $\alpha_1=1$, $\alpha_2$, and $\alpha_3$. By the argument presented in the proof of Lemma \ref{hanging shapes}, $\alpha_2$ and $\alpha_3$ are determined to be the minimal values occurring as the corresponding entries of the top row of a 2-minor of $T$ of shape $(3,3)$. Thus, the top row of $T'$ is known, and since $T'$ has exactly two rows, $T'$ is completely determined by strict monotonicity.
	
	Now, suppose that $T$ is \emph{not} of shape $(5,3)$, i.e., the top row of $T$ has at most 4 entries.	This leaves us with the options $(4,4)$, $(4,3,1)$, $(3,3,2)$, and $(3,3,1,1)$ for the possible shape of $T$, see Figure \ref{figure: 4 tableaux}.

	% Figure environment removed
	
	Let $\mathcal L$ be the set of all $S\in M_2(T)$ in which the position of $7$ in $T$ is now occupied by $6$. We will show that  $\{R_{6}S:S\in \mathcal L\}=M_1(T')$. Since $T'\in YT(6)$, this will determine $T'$ by \cite{Monks2009}.
	
	First, note that the deletion of 8 followed by the deletion of some $i$ less than $7$ cannot result in 7 sliding, and therefore results in $7$ being relabeled $6$, i.e., results in a 2-minor $S$ belonging to $\mathcal L$. It is immediate that $R_{6}S$ is precisely the $1$-minor of $T'$ resulting from deletion of~$i$. Thus $\{R_{6}S:S\in \mathcal L\}\supseteq M_1(T')$.
	
	To see the reverse inclusion, let us first observe that there are three possible ways in which a 2-minor of $T$ can belong to $\mathcal L$:
 \begin{itemize}
 \item[$(a)$] 7 does not slide, and is relabeled 6;
 \item[$(b)$] 7 is deleted, leading 8 to slide into its position and be relabeled 6; or
 \item[$(c)$] 7 slides, leading 8 to slide into its position and be relabeled 6.
 \end{itemize}
	
	It is clear that any 2-minor arising in fashion $(b)$ can be realized by deleting 8 rather than 7, and thus arises in fashion $(a)$ as well. Any 2-minor $S$ arising in fashion $(a)$ is the result of deleting 8 and some $i$ less than 7. In this case $R_{6}S$ is precisely the 1-minor of $T'$ given by deletion of $i$, i.e., $R_{6}S\in M_1(T')$.
	
	We complete the proof by demonstrating that there are no members of $\mathcal L$ arising in fashion $(c)$. Suppose that $S$ were such a 2-minor.
	
	Since 7 slides in the creation of $S$, it cannot be in the third row of $T$ from the top, as it would require the deletion of at least three entries for it to slide. Thus 7 is the fourth entry of the top row of $T$. Since 8 slides into the position of 7 in the creation of $S$, $T$ must be of shape $(4,4)$ with 8 as the fourth bottom row entry.
	
	Now, fix $i<7$ and consider the result of deleting $i$ from $T$. It is evident that this culminates in either 6 sliding or being deleted, leading 8 to slide left. It is clear that 8 cannot then slide into the position occupied by 7, a contradiction.	
\end{proof}

Using arguments similar to those above, we are able to prove the following, which will complete the proof of  Lemma  \ref{lemma: n = 8 lemma for k = 2}.


\begin{lemma}
	Let $T'$ be of shape $(4,2)$. Then $M_2(T)$ determines $T'$.
\end{lemma}

\begin{proof}
	Label the entries in the top row of $T'$ as $\alpha_1=1,\alpha_2,\alpha_3,$ and $\alpha_4$. As in the proof of Lemma \ref{hanging shapes}, $\alpha_2$ is determined by investigating 2-minors in $\M_2(T')$ of shape $(2,2)$. Note that if we are able to determine $\alpha_3$ and $\alpha_4$, strict monotonicity along the bottom row will determine the remainder of $T'$.
	
	If $\alpha_2=4$, the entries $\alpha_3=5$ and $\alpha_4=6$ are determined by strict monotonicity, and therefore all of $T'$ is determined.
	
	If $\alpha_2=3$, then we know that $\{\alpha_3,\alpha_4\}\subseteq\{4,5,6\}$. If $4\notin\{\alpha_3,\alpha_4\}$, the bottom row of $T'$ has entries $2$ and $4$. If $S\in M_2(T')$ is of shape $(3,1)$, then the bottom row entry of $S$ is at most 3 (as 4 will be relabeled if it has slid into this position). If $4\in\{\alpha_3,\alpha_4\}$, then the bottom row of $T'$ has entries 2 and either 5 or 6. Either way, there is some $S\in M_2(T')$ of shape $(3,1)$ with 4 as bottom row entry (achieved by deleting both 2 and $\alpha_4$). Thus, $M_2(T')$ determines whether $4\in \{\alpha_3,\alpha_4\}$. If $4\notin\{\alpha_3,\alpha_4\}$, then $\alpha_3=5$, $\alpha_4=6$, and $T'$ is thus fully determined. Otherwise $4\in\{\alpha_3,\alpha_4\}$, and in particular, $\alpha_3=4$. We determine $\alpha_4$ as follows. If $\alpha_4=5$, then the top row of $T'$ is $(1,3,4,5)$. If $S\in M_2(T')$ is of shape $(3,1)$, it must arise from deletion of at most one of $3,4,$ or 5 (and at least one of $1,2,$ or 6), which can only result in a top row of $S$ equal to $(1,2,3)$ or $(1,3,4)$. However, if $\alpha_4=6$, there is a 2-minor of $T'$ of shape $(3,1)$ with top row $(1,2,4)$, which is achieved by deleting $4$ and $2$. Thus, $M_2(T')$ distinguishes these cases as well, and $T'$ is determined.
	
	Finally, we consider the case when $\alpha_2=2$. It is here that we will need the extra information given by $M_2(T)$. For the rest of the proof, we will label the top row entries of $T'$ as $\alpha_1 =1, \alpha_2 =2, \alpha_3,$ and $\alpha_4$.
	
	We begin by considering the case that 7 is the 5th entry in the top row of $R_8T$; see Figure \ref{figure: 7 in T part 1}. Applying an argument similar to that we used in Lemma \ref{hanging shapes} for shape $(4,1,1)$, we can determine $R_8T$ from $M_2(R_8T)$. Specifically, we can determine $\alpha_3$ as the minimal entry that occurs in this position in the top row of 2-minors of $R_8T$ of shape $(3,2)$. Once $\alpha_4$ is known, the second row of $R_8T$ is determined by strict monotonicity. The entry $\alpha_4$ can be found as follows: If there is no 2-minor of $R_8T$ of shape $(4,1)$ with
    5 as the second row entry, then $\alpha_4 =6$. If $\alpha_4 \ne 6$ and there exists some 2-minor of $R_8T$ of shape $(3,2)$ with 4 as third top row entry, then $\alpha_4 =5$. Otherwise $\alpha_4 =4$.

 	% Figure environment removed

	Now, suppose that $7$ is in the third row of $R_8T$ from the top; see Figure~\ref{figure: 7 in T part 2}. In this case, looking at the third top row entry of 2-minors of $R_8T$ of shape $(3,2)$, the minimal available value gives $\alpha_3$ and the maximal value gives $\alpha_4 -1$. In particular, note that 7 cannot slide up with only one single deletion of something less than 7 and must itself be deleted instead.

   	% Figure environment removed
	
	Finally, suppose that $7$ is in the second row of $R_8T$; see Figure \ref{figure: 7 in T part 3}. We start with some basic observations on $\alpha_4$. By strict monotonicity along the top row, $\alpha_4$ is either $4,5,$ or~6.

    % Figure environment removed

	 Consider the set of 2-minors of $R_8T$ of shape $(4,1)$. If $\alpha_4$ is 5 or 6, there is at least one such 2-minor with 5 as the last entry of its top row (obtained by deleting 7 and the second entry of the bottom row). If $\alpha_4=4$, then no such 2-minor is present. In this case, monotonicity along rows forces $\alpha_3=3$, completing the determination of $T'$.

  Thus, the case $\alpha_4 \in \{4,5\}$ remains. This leaves us with the five tableaux in Figure \ref{figure: 5 tableaux}.

  % If $\alpha_4=6$, then there will be a 2-minor of $R_8T$ of shape $(3,2)$ with a 5 as the last top row entry (obtained by deleting 7 and $\alpha_3$). If $\alpha_4=5$, then there is no 2-minor of $R_8T$ of shape $(3,2)$ with 5 as the last entry in its top row (such a 2-minor requires deleting at least one of $\alpha_1,\alpha_2,$ or $\alpha_3$, which would then slide and reduce the 5). Thus, $\alpha_4$ is determined.

	% Figure environment removed

 The three tableaux on the top row of Figure \ref{figure: 5 tableaux} can now be identified by investigating the possible values of the bottom row entry of their 2-minors of shape $(4,1)$. First, note that for the top left tableau the value 2 is possible in this position (obtained by deleting the 7 and 2 of $R_8T$). For the remaining four tableaux, cells can only slide up after a top row cell has been deleted first during an earlier jeu de taquin process, and the shape $(4,1)$ must result from deleting two of the bottom row entries of $R_8T$. In particular, none of these tableaux will have a 2-minor of shape $(4,1)$ with bottom row entry 2. This distinguishes the top left tableau in Figure \ref{figure: 5 tableaux} from the other four tableaux. For the remaining four tableaux, one easily verifies that all of the values $3,4,$ and $5$ are possible as bottom row entries of 2-minors of shape $(4,1)$ if and only if the top row of $R_8T$ is $(1,2,4,6)$. If only values 3 and 5 appear in this position, then the top row of $R_8T$ is $(1,2,4,5)$. And if only values 4 and 5 appear, then $R_8T$ is one of the two tableaux on the bottom row of Figure \ref{figure: 5 tableaux}.

 It remains to distinguish between the two tableaux on the bottom row of Figure \ref{figure: 5 tableaux}. Note that the bottom left tableau has a 2-minor of shape $(3,2)$ with top row $(1,3,4)$ (obtained by deleting the 1 and 2 of $R_8T$) while direct inspection shows that no such 2-minor exists for the bottom right tableau. In particular, it would be necessary to delete two of the entries $1,2,$ and $3$ of $R_8T$ to achieve a 2-minor of shape $(3,2)$ without the value 2 as a top row entry. However, every 2-minor resulting in this fashion will have top row $(1,3,5)$.
	% If $\alpha_4=4$, monotonicity along rows forces $\alpha_3=3$, completing the determination of $T'$. If $\alpha_4=5$, then no entry of $R_8T$ can slide into this position. Since there is a 2-minor of $R_8T$ with a 5 in this position, this 2-minor has first row equal to the first row of $T'$ by Lemma \ref{lemmma: general construct In(m) when m in same place and nothing could slide}. Again, this completes the determination of $T'$. Finally, suppose that $\alpha_4=6$. Once again, consider the 2-minors of $R_8T$  of shape $(4,1)$ with 5 as the last entry of row 1---such a minor requires deletion of 7 and either another entry from row 2 or one of the entries $\alpha_1=1$ or $\alpha_2=2$ (only if the jeu de taquin process results in an entry sliding up).
    %
	% In particular, consider the entries that occur third in row 1.  If $\alpha_3=3$, then deleting 1 or 2 does not result in an entry sliding up, and thus the only 2-minors of this form are formed by deleting 7 and something in row 2 (which is necessarily greater than 3). Thus 3 is the only such entry. If $\alpha_5=5$, then 4 is the only such entry (as something less than 5 was deleted, thus reducing 5). If $\alpha_3=4$, then both 3 and 4 appear in this position (in this case, the second row has entries 3 and 5. Deleting the former produces a 3, the latter produces a 4). Thus, the set of entries occurring in this position determines $\alpha_3$, which completes the determination of the first row of $T'$.
\end{proof}



%%%%%%%%%% OLD SECTION 6 %%%%%%%%%%%%%%%%
%In this section, we prove that $\M_2(T)$ determines $T$ when $n \geq 8$.
%This result is sharp result for $k = 2$ as the mapping $T \longrightarrow \M_2(T)$ is no longer injective when $n = 7$.
%For example, the $2$-minors of the tableaux in Figure \ref{figure: same 2-minors}
%% Figure environment removed
%are identical.
%Of note, the mapping $T \longrightarrow \mM_2(T)$ is injective for $n$ equal to $7$ and $6$. Injectivity fails at $n = 5$. These can be verified by a straightforward computer calculation.
%
%
%
%\begin{theorem} \label{thm: k = 2 sharp}
%	Let $n \in \N$ and $T \in \YT(n)$. Then $\M_2(T)$ determines $T$ when $n \geq 8$.
%\end{theorem}
%
%By Lemma \ref{lemma:location of k2 + 2k through n}, $\M_2(T)$ determines the location of $[9, n]$. Thus, only the location of $[1, 8]$ remain to be determined.
%However, Lemma \ref{lemma: removal of top elements and Mk} shows that $\M_2(T)$ determines $\M_2(R_{[9, n]} T)$ as $R_{[7, n-2]} \M_2(T)$.
%
%This reduces the proof of \ref{thm: k = 2 sharp} to demonstrating the following.
%
%\begin{lemma} \label{lemma: n = 8 lemma for k = 2}
%	Let $T \in \YT(8)$. Then $\M_2(T)$ determines $T$.
%\end{lemma}
%
%Note that Equation \ref{equation: Monks shape determination bound} and Corollary \ref{cor: location of n} show that $\M_2(T)$ determines the shape of $T$ and the location of $8$.
%
%The proof of Lemma \ref{lemma: n = 8 lemma for k = 2} will occupy the rest of this section and consists of some utility lemmas and a case by case analysis for the possible shapes of $T \in \YT(8)$. Up to symmetry, there are 12 shapes to consider:
%$(8)$, $(7,1)$, $(6,2)$, $(6,1^2)$, $(5,3)$, $(5,2,1)$, $(5, 1^3)$, $(4^2)$, $(4,3,1)$, $(4,2^2)$, $(4,2,1^2)$, and $(3^2,2)$.
%
%We begin with a few lemmas.
%
%% \begin{lemma} \label{lem: 2+ overhang for k = 2}
%	%     Let $n, k \in \N$ with $n \geq k^2 + k$. Let $T \in \YT(n)$ with shape $(r_1, r_2, \ldots, r_c)$ satisfying $r_1 \geq r_2 + 2$ and $c \geq 2$. Then $\M_2(T)$ determines the rightmost element of the top row of $T$, denoted $\beta$. Moreover, if $\beta$ is not $n$ or $n-1$, then the location of all elements less than $\beta$ are determined, including the whole top row.
%	% \end{lemma}
%
%% \begin{proof}
%	%     By Corollary \ref{cor: location of n}, the location of $n$ is known. If it is the rightmost element of the top row of $T$, we are done.
%	
%	%     Otherwise, suppose $n$ is not in the top row. Label the last two entries of the top row as $\alpha, \beta$. Look at $T' \in \M_2(T)$ with exactly one box removed from the top row of $T$ and let $\gamma$ be the maximal entry, over all such $T'$, obtained in the top row. As the only values that could possibly appear in last box of the top row of a $T'$ are $\alpha -2, \alpha -1, \alpha, \beta - 2, \beta  -1$ and since $\alpha \leq \beta -1$, $\gamma$ would be $\beta -1$ if it could be obtained. In fact, by deleting $\alpha, n$, this is possible. Therefore, $\gamma = \beta -1$ and so $\beta$ is determined.
%	
%	% Finally, if $\beta$ is also not $n - 1$, then both $n$ and $n-1$ are not in the top row. Thus deleting $n$ and $n-1$ leaves the top row fixed.
%	% Lemma \ref{lemmma: general construct In(m) when m in same place and nothing could slide} finishes the result.
%	% \end{proof}
%
%
%\begin{lemma} \label{lem: 2+ overhang for k = 2}
%	Let $n \in \N$ with $n \geq 8$. Let $T \in \YT(n)$ with shape $(r_1, r_2, \ldots, r_c)$ satisfying $r_1 \geq r_2 + 2$. Then $\M_2(T)$ determines the top row of $T$.
%	% The second row, first column is also determined. Moreover, if the last entry is neither $n$ nor $n-1$, then the location of all elements less than the last entry are determined.
%\end{lemma}
%
%\begin{proof}
%	Label the entries in the top row as $\alpha_1, \ldots, \alpha_{r_1}$. Let $1 \leq i \leq r_1 - 2$. Look at $T' \in \M_2(T)$ with two boxes removed from the top row of $T$. Since nothing can slide up, the only possible values that could appear in $T'$ in the box originally holding $\alpha_i$, for $1 \leq i \leq r_1 - 2$, are
%	$\alpha_i,
%	\alpha_{i+1} -1, \alpha_{i+2} -2$. By deleting the last two boxes, the minimal value of $\alpha_i$ is always obtained. Therefore $\alpha_i$ is determined by $\M_2(T)$ for $1 \leq i \leq r_1 - 2$.
%	
%	If $r_1 \geq r_2 + 3$ and $r_2 + 1 \leq i \leq r_1 - 2$, then deleting $\alpha_{r_2 + 1}, \alpha_{r_2 + 2}$ achieves the maximal value of $\alpha_{i + 2} - 2$. Therefore, $\alpha_i$ is determined by $\M_2(T)$ for $r_2 + 3 \leq i \leq r_1$. Thus the whole top row is determined when $r_1 \geq r_2 + 4$.
%	
%	When $r_1 = r_2 + 3$, it remains to determine $\alpha_i$ for $i = r_1 - 1$.
%	By deleting the various combinations of two of the last three boxes,
%	each value of $\alpha_{r_1 - 2}, \alpha_{r_1 - 1} -1, \alpha_{r_1} -2$ appear
%	in some $T'$ in the box originally holding $\alpha_{r_1 - 2}$. However, the values may not be distinct.
%	If either exactly one or three distinct values appear, then $\alpha_{r_1 - 1}$ is determined by the middle value.
%	Otherwise, exactly two values appear. Then either $\alpha_{r_1 - 1} - 1$ is is the minimal or the maximal value. In the first case, $\alpha_{r_1 - 1} = \alpha_{r_1 - 2} + 1$ with $\alpha_{r_1 - 1} + 2 \leq \alpha_{r_1}$. In the second case, $\alpha_{r_1 - 2} \leq \alpha_{r_1 - 1} - 2$ with $\alpha_{r_1 - 1} +1 = \alpha_{r_1}$. By looking at cases, we will determine $\alpha_{r_1 - 1}$ by showing how to differentiate between these two possibilities.
%	
%	Consider first $a_{r_1} = n$.
%	We seek to tell the difference between final top row entries of the form $\alpha_{r_1 - 2}, \alpha_{r_1 - 2} + 1, n$ and $\alpha_{r_1 - 2}, n-1, n$ where $n - \alpha_{r_1 - 2} \geq 3$.
%	Look at all $T' \in \M_2(T)$ with no boxes removed from the top row of $T$.
%	
%	If there are two entries of $T$ below the top row that are less than $\alpha_{r_1 - 2}$, then there is a $T'$ with top row final entries of $\alpha_{r_1 - 2} - 2, \alpha_{r_1 - 2} - 1, n - 2$ or $\alpha_{r_1 - 2} - 2, n - 3, n -2$, respectively. The gaps between these numbers are $1$ and $n - \alpha_{r_1 - 2} - 1 \geq 2$, but in different locations. Therefore they are distinguishable.
%	
%	If there is only one entry of $T$ below the top row that is less than $\alpha_{r_1 - 2}$, then using it and the entry below garaunteed by $n - \alpha_{r_1 - 2} \geq 3$, there is a $T'$ where $\alpha_{r_1 - 2} - 1, \alpha_{r_1 - 2}, n - 2$ or $\alpha_{r_1 - 2} - 1, n - 3, n -2$ are the final top row entries, respectively. The gaps between these numbers are $1$ and $n - \alpha_{r_1 - 2} - 2$, but in different locations. The second gap is at least $2$ when $n - \alpha_{r_1 - 2} \geq 4$ so that the cases are distinguishable. If $n - \alpha_{r_1 - 2} = 3$, then there is exactly one entry of $T$ bigger than $\alpha_{r_1 - 2}$ that appears below the top row. As there is only one entry less than $\alpha_{r_1 - 2}$, this means the number of boxes below the top row is $2$. This means that $r_1$ is $3$ or $5$ and $n$ is $6$ or $7$, which violates $n \geq 8$.
%	
%	The only other possibility is that there are no entries of $T$ below the top row that are less than $\alpha_{r_1 - 2}$. Then there is a $T'$ with final top row entries of $\alpha_{r_1 - 2}, \alpha_{r_1 - 2} + 1, n - 2$ or $\alpha_{r_1 - 2}, n - 3, n -2$, respectively. Similar to above, these are distinguishable when $n - \alpha_{r_1 - 2} \geq 5$. If $n - \alpha_{r_1 - 2} = 4$, there are only two entries that can appear below. This means that $r_1$ is $3$ or $5$ and $n$ is $6$ or $7$, which violates $n \geq 8$.
%	% If $n - \alpha_{r_1 - 2} = 5$, Then there are exactly two appropriate entries that can appear below. As above, this violates $n \geq 8$.
%	The valuation of $n - \alpha_{r_1 - 2} = 3$ is also not possible since then there is only one entry that can appear below which would require $n = 5$.
%	
%	This finishes the case of $a_{r_1} = n$ when $r_1 = r_2 + 3$. We now examine the situation where $n$ does not appear on the top row. It then appears in a known lower OC.
%	
%	Consider the case where $a_{r_1} = n - 1$. The cases to distinguish are
%	$\alpha_{r_1 - 2}, \alpha_{r_1 - 2} + 1, n - 1$ and $\alpha_{r_1 - 2}, n-2, n - 1$ where $n - 1 - \alpha_{r_1 - 2} \geq 3$. If there are two entries or exactly one entry of $T$ below the top row that are less than $\alpha_{r_1 - 2}$, then, similar to the argument above (making use of the $n$ below in the later case), we are done since the larger gap is $n - 2 -\alpha_{r_1} \geq 2$.
%	Otherwise, there are no entries of $T$ below the top row that are less than $\alpha_{r_1 - 2}$. Similar to the argument above and using $n$ below, we are done when when $n - 1 - \alpha_{r_1 - 2} \geq 4$. But if $n - 1 - \alpha_{r_1 - 2} = 3$, this leads to a violation of $n \geq 8$.
%	
%	At last, consider the case $r_1 = r_2 + 3$ and $a_{r_1} \leq n - 2$ so that $n, n-1$ both appear below the top row in $T$. Here, simply look for the maximal entry in $T'$ appearing in the box originally occupied by $\alpha_{r_1 - 1}$.
%	
%	This finishes the case of $r_1 = r_2 + 3$.
%	We now consider the remaining case of $r_1 = r_2 + 2$.
%	
%	Here, we need to determine $\alpha_i$ for $i = r_1 - 1, r_1$. First, we show that $\alpha_{r_1}$ is determined by $\M_2(T)$.
%	By Corollary \ref{cor: location of n}, the location of $n$ is known. If it is the rightmost element of the top row of $T$, we are done.
%	
%	Otherwise, suppose $n$ is not in the top row. Look at $T' \in \M_2(T)$ with exactly one box removed from the top row of $T$. Consider the maximal entry appearing in the top row of some $T'$. The only values that could possibly appear in last box of the top row of $T'$ are $\alpha_{r_1 - 1} -1, \alpha_{r_1 - 1}, \alpha_{r_1} -2, \alpha_{r_1} -1$. Since the maximum, $\alpha_{r_1} -1$, is achievable, $\alpha_{r_1}$ is determined.
%	
%	We now must detrmine $\alpha_{r_1 - 1}$. If $\alpha_{r_1} \leq n - 2$, look at the maximal entry in the original location of $\alpha_{r_1 - 1}$ among $T' \in \M_2(T)$ with no boxes removed from the top row of $T$. The maximal value is $\alpha_{r_1 - 1}$.
%	
%	% If $\alpha_{r_1} = n - 1$, look at the entry in the original location of $\alpha_{r_1 - 1}$ among $T' \in \M_2(T)$ with exactly one box removed from the top row of $T$. The entries $n - 2, n-3, \alpha_{r_1 - 1}$ are achievable and sometimes $\alpha_{r_1 - 1} -1$. If $\alpha_{r_1 - 1} \leq n - 5$, then via the gap, $\alpha_{r_1 - 1}$ is identifiable. Otherwise, $n - 4 \leq \alpha_{r_1 - 1} \leq n - 2$.
%	% When $\alpha_{r_1 - 1} = n - 2$, only $n -2, n-3$ appear. In the other two cases of $\alpha_{r_1 - 1} = n - 3, n - 4$, at least $n - 4$ appears.     It remains to distinguish between $\alpha_{r_1 - 1} = n - 3, n - 4$.
%	
%	If $\alpha_{r_1} = n - 1$, look at the entry in the original location of $\alpha_{r_1 - 1}$ among $T' \in \M_2(T)$ with no boxes removed from the top row of $T$ and with an $n - 2$ in the top row. This forces $n$ to be deleted.
%	If $\alpha_{r_1 - 1} \leq n - 3$, then $\alpha_{r_1 - 1}$ appears and possibly $\alpha_{r_1 - 1} - 1$. If $\alpha_{r_1 - 1} = n - 3$, using $n \geq 8$ to get at least three boxes below the top row, $n - 3, n - 4$ appear. If $\alpha_{r_1 - 1} = n - 2$, then only $n - 3$ appears. As these cases are distinguishable, $\alpha_{r_1 - 1}$ is determined.
%	
%	Now suppose $\alpha_{r_1} = n$. Look at $T' \in \M_2(T)$ with exactly one box removed from the top row of $T$ and the possible locations of $n - 2$.
%	If $\alpha_{r_1 - 1} = n - 1$, then $n - 2$ only appears in the top row. If $\alpha_{r_1 - 1} \not= n - 1$, then $n - 2$ will appear in a row underneath the top row by deleting $n$ and a box adjacent to $n - 1$ under the top row. Therefore, we can determine when $\alpha_{r_1 - 1} = n - 1$ if $\alpha_{r_1} = n$.
%	
%	Finally, if $\alpha_{r_1 - 1} \leq n - 2$ and $\alpha_{r_1} = n$, look at $T' \in \M_2(T)$ with exactly one box removed from the top row of $T$ and the entry in the box originally occupied by $\alpha_{r_1 - 1}$. The entries $n - 2, \alpha_{r_1 - 1}$ appear and sometimes $\alpha_{r_1 - 1} - 1$.
%	If $\alpha_{r_1 - 1} \leq n - 4$ the gap formed at $n - 3$ identifies $\alpha_{r_1 - 1}$.
%	If $\alpha_{r_1 - 1}$ is $n - 2$ or $n - 3$, there is no gap. However, if $\alpha_{r_1 - 1} = n - 2$, then $n - 3$ appears as there are at least two boxes below.
%	If $\alpha_{r_1 - 1} = n - 3$, then $n - 4$ appears as there are at least three boxes below since $n \geq 8$. These are therefore distinguishable.
%	As a result, $\alpha_{r_1 - 1}$ is determined.
%	
%	
%	% ???? NO. Arrrrg. (Might require bottom row slide up).In fact, all these values can be achieved, though they may not be distinct, by deleting two elements of $\alpha_i, \alpha_{i+1}, \alpha_{i+2}$.
%	% As $\alpha_i \leq \alpha_{i+1} -1 \leq \alpha_{i+2} -2$,
%	% the minimum and maximum values determine $\alpha_i$ and $\alpha_{i+2}$.
%	% %(If either just one or if three distinct values are obtained, then $\alpha_{i+1}$ is also determined. Otherwise, $\alpha_{i+1}$ is either $\alpha_i + 1$ or $\alpha_{i+2} - 1$.)
%	
%	% By varying $i$ when $r_1 \geq 4$, the whole top row is determined. When $r_1 = 3$, the middle entry still needs identification. In this case, the constraint on $n$ means that $T$ has shape $(3,1^q)$ with $q \geq 5$. In this case, apply the symmetrical transpose argument to determine the first row. This, in turn, determines the first column.
%	
%	% ADD 2nd row first column... if not doing next thm.
%	
%	% For the last comment of the Lemma, if the last entry is also not $n$ or $n - 1$, then both are not in the top row. Thus deleting $n$ and $n-1$ leaves the top row fixed.
%	% Lemma \ref{lemmma: general construct In(m) when m in same place and nothing could slide} finishes the result.
%\end{proof}
%
%
%
%
%
%
%\begin{theorem} \label{thm: 2+ overhang, k = 2, done}
%	Let $n \in \N$ with $n \geq 8$.  Let $T \in \YT(n)$ with shape $(r_1, r_2, \ldots, r_c)$ satisfying $r_1 \geq r_2 + 2$. Then $\M_2(T)$ determines $T$.
%\end{theorem}
%
%\begin{proof}
%	By Lemma \ref{lem: 2+ overhang for k = 2}, the top row of $T$ is determined by $\M_2(T)$.
%	Let $\gamma$ be the entry of a fixed box of $T$ not in the top row. Let $\gamma_m$ be the maximal entry found in that box over all $T' \in \M_2(T)$ with two boxes removed from the top row.
%	
%	The mapping from $\gamma$ to $\gamma_m$ preserves the relative ordering of the $\gamma$ since there is no sliding. As the set of all $\gamma$ is known from the complement of the top row, the relative ordering admits a unique filling of the tableau.
%	% Consider the case of $\beta = n$. If $\alpha = n - 1$, $\gamma = \gamma_m$ and we are done. If $\alpha = n - 2$, then $\gamma = \gamma_m$ for $\gamma \leq n - 3$ and $n - 2$ for $\gamma = n - 1$. As this is invertable, we are done.  .....
%\end{proof}
%
%
%
%
%
%
%% {\color{red} False.}
%% \begin{lemma} \label{lem: one overhang, most of top row}
%	%     Let $n, k \in \N$ with $n \geq 8$. Let $T \in \YT(n)$ with shape $(r_1, r_2, \ldots, r_c)$. If $r_1 = r_2 + 1$, then $\M_2(T)$ determines the first $r_1 - 2$ entries of the top row. If $r_3 = r_2$, the $r_1 - 1$ entry is also determined.
%	% \end{lemma}
%
%% \begin{proof}
%	%     Look at $T' \in \M_2(T)$ with no boxes removed from the top row of $T$. One possibility is that $n$ is on the top row, which forces $n - 1$ to be not in the top row, or $n - 1$ is at the end of the top row, which forces $n$ to be not in the top row, or both $n, n - 1$ are not in the top row. In each scenario, the maximum value obtained in the first $r_1 - 2$ entries of the top row is the original entry. Additionally, if $r_2 = r_3$, then the $r_1 - 1$ entry can be obtained in the same way.
%	% \end{proof}
%
%
%
%
%
%
%\begin{lemma} \label{lem: bottom}
%	Let $n, k \in \N$ with $n \geq 8$. Let $T \in \YT(n)$ with shape $(r_1, r_2, \ldots, r_c)$.
%	Then $\M_2(T)$ determines the rightmost $r_c - 2$ entries of the bottom row. Moreover, if the last entry is neither $n$ nor $n-1$, then the location of all elements less than the last entry are determined, including the whole bottom row and its inner area.
%\end{lemma}
%
%\begin{proof}
%	When $c = 1$, there is nothing to do, so we will assume $c \geq 2$.
%	Label the bottom row entries as $\alpha_1, \ldots, \alpha_{r_c}$.
%	Look at $T' \in \M_2(T)$ with exactly two boxes removed from the bottom row of $T$.
%	For $1\leq i \leq r_c -2$, the maximal value that appears in the box originally holding $\alpha_i$ is $\alpha_{i+2} - 2$ by deleting the first two boxes. Therefore $\alpha_i$ is determined for $3 \leq i \leq r_c$.
%	The last statement follows from Lemma \ref{lemmma: general construct In(m) when m in same place and nothing could slide}.
%\end{proof}
%
%
%
%
%
%
%
%We now turn to the proof of Lemma \ref{lemma: n = 8 lemma for k = 2} and the case by case analysis for the possible shapes of $T \in \YT(8)$. For the shape $(8)$, there is nothing to do.
%
%\begin{center}
%	$(7,1)$, $(6,2)$, $(6,1^2)$, $(5,3)$, $(5,2,1)$, $(5, 1^3)$, $(4,2^2)$, $(4,2,1^2)$ Entry Identification
%\end{center}
%These all follow from Theorem \ref{thm: 2+ overhang, k = 2, done}.
%
%
%% Figure environment removed
%\begin{center}
%	$(4^2)$ Entry Identification
%\end{center}
%Reference Figure \ref{figure: (4^2)} for the shape.
%By Lemma \ref{lem: bottom}, the rightmost two elements of the bottom row are determined.
%
%Consider first the case when $7$ lies on the bottom row. Write $\alpha$ for the top rightmost entry. Look at $T' \in \M_2(T)$ with two boxes removed from the bottom row of $T$ and the value of the top rightmost entry. The values $6$ and $\alpha$ can be achieved and possibly $\alpha - 1$ and $\alpha - 2$.
%If $\alpha$ is $5$ or $6$, the achieved values are exactly $6$, $\alpha$, and $\alpha -1$ so that $\alpha$ is known.
%If $\alpha \leq 5$, Lemma \ref{lemmma: general construct In(m) when m in same place and nothing could slide} applied to the $\alpha$ identifies the top row and the remaining numbers uniquely fill the bottom row.
%
%When $\alpha = 6$ and $7$ lies on the bottom row, write $\beta$ for the element to the left of $7$. It is $4$ or $5$. Look at $T' \in \M_2(T)$ with two boxes removed from the bottom row of $T$ and the value of the second entry on the bottom. A $4$ can be achieved if and only if $\beta = 4$. In that case, Lemma \ref{lemmma: general construct In(m) when m in same place and nothing could slide} applied to $\beta$ identifies the bottom row and the remaining numbers uniquely fill the top row.
%
%When $\beta = 5$, write $\gamma$ for the penultimate top row entry. It is $3$ or $4$. Look at $T' \in \M_2(T)$ with one box removed from each row of $T$ and the value of the top rightmost entry. A $3$ can be achieved if and only if $\gamma = 3$. In that case, Lemma \ref{lemmma: general construct In(m) when m in same place and nothing could slide} applied to the $\gamma$ identifies the top row and the remaining numbers uniquely fill the bottom row.
%
%If $\gamma = 4$, the remaining two entries are $2$ and $3$. It is straightforward to calculate $\M_2(T)$ in each case to see they give different results. For instance, a $T'$ with top row $(1,2,4)$ appears only when the $2$ is on top.
%
%Turn now to the case that $7$ is the last element in the top row. This forces $6$ to be the penultimate element of the bottom row. There are two possibilities. The first is that $5$ appears to the left of $6$,  above $6$ is either $3$ or $4$, and above $5$ is either $2$ or $3$. The second is that $4$ appears to the left of $6$, above $6$ is $5$, and above $4$ is either $2$ or $3$.
%%Write $\alpha = 5$ in the first case and $\alpha = 4$ in the second.
%
%In the first case, there are only three possibilities for the top row of $T$: $(1,3,4,7), (1,2,4,7), (1,2,3,7)$. The first can be identified as the only one with an element of $M_2(T)$ with top row $(1,3,4)$. The second can be identified as the only one with an element of $M_2(T)$ with top row of $(1,2,4)$. Otherwise, it is the third possibility.
%
%In the second case, look at $T' \in \M_2(T)$ with two boxes removed from the bottom row of $T$ containing a $6$ at the end of the first row and whose bottom entry in the second column is $4$. Such $T'$ exist by deleting $6, 8$. Such $T'$ require either deleting $8$ or first deleting/sliding $7$. In the later option, the initial location of $6, 7$ actually requires deleting $7$. As a result, either $7$ or $8$ must be deleted so that $6$ cannot slide to become the $4$. Lemma \ref{lemmma: general construct In(m) when m in same place and nothing could slide} applied to the $4$ finishes things.
%
%
%
%% Figure environment removed
%\begin{center}
%	$(4,3,1)$ Entry Identification
%\end{center}
%Reference Figure \ref{figure: (4,3,1)} for the shape.
%First look at the case where $8$ appears in the bottom OC. By Lemma \ref{lemma: removal of top elements and Mk}, $\M_2(T)$ determines $\M_2(R_{8}T)$. The proof of Lemma \ref{lem: bottom} applies to $R_{8}T$ and so the middle OC is determined. Moreover, either $7$ appears as the middle OC or it appears as the top OC and $6$ appears in the middle OC.
%
%Suppose the middle OC is $7$. In this case, the top OC is also determined by its maximal value in a $2$-minor. Moreover, $6$ appears to the left of $7$ or in the upper OC so that its location is known.
%
%Suppose the middle row ends in $6,7$.
%% look at $T' \in \M_2(T)$ with two boxes removed from the middle row of $T$. This identifies the last entry of the first row by the maximal value achieved there in $\M_2(T)$.
%If the top OC is $5$, then Lemma \ref{lemmma: general construct In(m) when m in same place and nothing could slide} applied to the $5$ finishes things. Otherwise, the entry is $4$ and $5$ appears to the left of $6$. Lemma \ref{lemmma: general construct In(m) when m in same place and nothing could slide} applied to the $4$ finishes things.
%
%If $7$ is the middle OC and $6$ is the top OC or if $6$ is the middle OC and therefore $7$ is the top OC, Lemma \ref{lemmma: general construct In(m) when m in same place and nothing could slide} applied to the $6$ finishes things.
%
%The next case is where $8$ appears in the middle OC. The number $7$ must appear to the left of $8$ or in one of the two OCs. These two cases may be distinguished by looking at $T' \in \M_2(T)$ with the middle OC and one other OC removed from $T$. In the first case, $6$ always appears in the middle row. In the later case, it can appear in a SOC.
%
%When $7$ is to the left of $8$, look at $T' \in \M_2(T)$ with two boxes removed from the middle row of $T$. As the $7, 8$ cannot slide up or the $7$ to the left, the entry in each remaining box determines its original value.
%
%If $7$ appears in an OC, the precise OC holding $7$ can by determined by looking at $T' \in \M_2(T)$ with two boxes removed from the middle row of $T$ and $6$ not appearing in the middle row. In this case, $6$ can be made to appear only in the SOC of the original position of $7$.
%
%When $7$ is in the bottom OC, look at $T' \in \M_2(T)$ with no boxes removed from the top row of $T$. The rightmost entry of the top row is determined by the maximal value obtained. Applying Lemma \ref{lemmma: general construct In(m) when m in same place and nothing could slide} to it, determines the top row. As the remaining numbers only fit in one way, we are done.
%
%Otherwise, consider the case of $7$ appearing in the top OC. The bottom entry is determined by the maximal entry among $T' \in \M_2(T)$ with the bottom OC as a SOC. If it is $5$ or $6$, apply Lemma \ref{lemmma: general construct In(m) when m in same place and nothing could slide} to it and we are done. If it is $3$ or $4$, the bottom two numbers of the first row are at most $4$ and $5$ and $6$ are adjacent to $8$. Looking at the minimal value achieved in $T' \in \M_2(T)$ with no boxes removed from the top row of $T$ will determine the value of the entry above $8$ by adding two. Apply Lemma \ref{lemmma: general construct In(m) when m in same place and nothing could slide} to it, and we are done.
%
%Turn to the case where $8$ appears in the top OC. Then $7$ appears in an OC too. By looking at
%$T' \in \M_2(T)$
%with two OCs removed. If $6$ never appears in one of the remaining SOCs,
%then the maximum value there is the original value and it is at most $5$.
%Call this value $\beta$. In this case, $\beta$ must occur in the lower OC and $7$ appears in the middle OC. If $\beta = 5$, apply Lemma \ref{lemmma: general construct In(m) when m in same place and nothing could slide} to it, and we are done.
%Otherwise, $5$ and $6$ are adjacent to $7$. Looking at the minimal value achieved in $T' \in \M_2(T)$ with no boxes removed from the top row of $T$, adding two will determine the value of the entry above $7$.
%Apply Lemma \ref{lemmma: general construct In(m) when m in same place and nothing could slide} to it, and we are done.
%
%Finally, consider the case where $6$ and $7$ appear in the lower OCs. Then $4$ and $5$ are adjacent to the middle OC. Look at $T' \in \M_2(T)$ with the lower OC as a SOC. This SOC can be made to hold $4$ if and only if $6$ was the original occupant. Therefore $6$ and $7$ are determined.
%Applying Lemma \ref{lemmma: general construct In(m) when m in same place and nothing could slide} to $6$ finishes things up.
%
%% By looking at the maximal value of $T' \in \M_2(T)$ with no boxes removed from the middle row of $T$, the value to the left of the middle OC is determined. Therefore, the location of $4$ and $5$ are known. Applying Lemma \ref{lemmma: general construct In(m) when m in same place and nothing could slide} to the entry above the middle OC identifies the location of $2$ and $3$.
%
%
%
%
%
%% Figure environment removed
%\begin{center}
%	$(3^2,2)$ Entry Identification
%\end{center}
%Reference Figure \ref{figure: (3^2,2)} for the shape.
%As the arguments are transpose symmetric, suppose $8$ appears in the upper OC, Corollary \ref{cor: location of n}.
%
%Look at $T' \in \M_2(T)$ with no boxes removed from the top row and possible entries at the end of the top row. If it started as $7$, then $6$ and $5$ appear. If it started as $6$, then $6$, $5$, $4$ appear. If it started as $5$, then $5$, $4$, and $3$ appear. If it started as a $4$, then $4$ and $3$ appear. If it started as a $3$, then just $3$ appears. As a result, the top right entry is determined.
%% By Lemma \ref{lemmma: general construct In(m) when m in same place and nothing could slide}, the top row, and every entry smaller, is determined if the value is at most $5$.
%
%If the top right entry is $7$, then the lower OC is $6$. Lemma \ref{lemmma: general construct In(m) when m in same place and nothing could slide} applied to $6$ finishes things.
%
%If the top right entry is $6$, then the lower OC is $7$ and $5$ is adjacent to $7$. The first entry of the last row can be determined by looking at $T' \in \M_2(T)$ with a box removed from each of the first two rows and the maximum value there. If the value is $5$, then using such a $T'$ and Lemma \ref{lemmma: general construct In(m) when m in same place and nothing could slide} applied to $5$ finishes things. Otherwise, the value is $4$ and $5$ is above the $7$. Using such a $T'$ and Lemma \ref{lemmma: general construct In(m) when m in same place and nothing could slide} applied to $4$ wraps things up.
%
%If the top right entry is $5$, then Lemma \ref{lemmma: general construct In(m) when m in same place and nothing could slide} applied to $5$ identifies everything except $6$ and $7$. However, $7$ appears in the lower OC so we are done.
%
%If the top right entry is $4$, then a similar argument places $7$ in the lower OC and identifies everything except $5$ and $6$. Now $6$ and $5$ are adjacent to $7$.
%Look at $T' \in \M_2(T)$ with a box removed from each of the first two rows.
%If the first entry of the bottom row was a $6$, then there are $T'$ where it becomes a $5$. However, if it started as a $5$, the highest it can become is $4$.
%
%If the top right entry is $3$, then the top row is known, the first entry of the second row is $4$, and $5$ and $6$ are adjacent to $7$. A similar argument as above identifies the first entry of the bottom row to finish the argument.

%-----%-----%-----%
%-----%-----%-----%
\section{Concluding Remarks}
%-----%-----%-----%
%-----%-----%-----%

For reconstructibility from $\M_k(T)$ for general $k$, it seems that the most one can hope for is the following quadratic bound, which is suggested by the few cases we are able to check via software:

\begin{conjecture}
    \label{conj:k^2 + 2k}
    Let $k \geq 2$.  Let $T \in \YT(n)$, where $n \geq k^2 + 2k$.  Then $\M_k(T)$ determines $T$.
\end{conjecture}

As for reconstructibility from multisets $\mM_k(T)$, there is some evidence that the lower bound is linear in $k$.  We have been able to verify the following conjecture by computer for $k \leq 5$:

\begin{conjecture}
\label{conj:k+4}
    Let $k \geq 1$.  Let $T \in \YT(n)$, where $n \geq k + 4$.  Then $\mM_k(T)$ determines $T$.
\end{conjecture}

As a final remark, we point out that the lower bound on $n$ for reconstructibility depends very much on the shape of tableaux: upon restriction to certain shapes, $\M_k(T)$ determines $T$ for values of $n$ that are much smaller than the general lower bound.  This can be used to streamline proofs in the course of solving the problem for higher values of $k$.

%-----%-----%-----%
%-----%-----%-----%
%%%%%%%%%% Bibliography %%%%%%%%%%
%-----%-----%-----%
%-----%-----%-----%

% \newpage

%-----%-----%
\bibliographystyle{alpha}
\bibliography{main}
%-----%-----%

\end{document}
















% $(7,1)$, $(6,2)$, $(6,1^2)$, $(5,3)$, and $(5, 1^3)$:\\
% Lemma \ref{lem: 2+ overhang for k = 2} shows that the top row is determined by $\M_2(T)$. As the remaining numbers fit uniquely in the bottom row (first column, respectively), we are done.


% $(5,2,1)$:\\
% Lemma \ref{lem: 2+ overhang for k = 2} shows that the top row is determined by $\M_2(T)$. In addition, the minimum of the three remaining numbers must appear in the second row, first column. Thus, only the two lower OCs are unknown. As the location of $8$ is known, Corollary \ref{cor: location of n}, if it resides in one of the lower OCs, we are done.

% Therefore we may reduce to the case where $8$ is the last entry of the top row. If $7$ also lies on the top row, then look at $T' \in \M_2(T)$ with two boxes removed from the top row of $T$. The maximal value in each SOC is the original value and we are done. Otherwise, $7$ is in one of the lower OCs. That location is determined by the location of $6$ in a lower SOC of any $T'$ for which $6$ fails to appear in the top row. Such $T'$ exist by deleting the last two elements of the top row of $T$.


% $(4^2)$:\\
% {\color{red} 8 adj above or to right?}
% By Lemma \ref{lem: bottom}, the rightmost three elements of the bottom row are determined. If $7$ lies on the bottom row, then by Lemma \ref{lemma: removal of top elements and Mk}, $\M_2(T)$ determines $\M_2(R_{[7,8]}T)$. The proof of Lemma \ref{lem: 2+ overhang for k = 2} applies to $R_{[7,8]}T$ and so the top row of $T$ is determined. As there is only one remaining number, we are done.

% Turn now to the case that $7$ is the last element in the top row. This forces $6$ to be the penultimate element of the bottom row. There are two possibilities. The first is that $5$ appears to the left of $6$,  above $6$ is either $3$ or $4$, and above $5$ is either $2$ or $3$. The second is that $4$ appears to the left of $6$, above $6$ is $5$, and above $4$ is either $2$ or $3$.
% %Write $\alpha = 5$ in the first case and $\alpha = 4$ in the second.

% In the first case, there are only three possibilities for the top row of $T$: $(1,3,4,7), (1,2,4,7), (1,2,3,7)$. The first can be identified as the only one with an element of $M_2(T)$ with top row $(1,3,4)$. The second can be identified as the only one with an element of $M_2(T)$ with top row of $(1,2,4)$. Otherwise, it is the third possiblity.

% In the second case, look at $T' \in \M_2(T)$ with two boxes removed from the bottom row of $T$ containing a $6$ at the end of the first row and whose bottom entry in the second column is $4$. Such $T'$ exist by deleting $6, 8$. Such $T'$ require either deleting $8$ or first deleting/sliding $7$. In the later option, the initial location of $6, 7$ actually requires deleting $7$. As a result, either $7$ or $8$ must be deleted so that $6$ cannot slide to become the $4$. Lemma \ref{lemmma: general construct In(m) when m in same place and nothing could slide} applied to the $4$ finishes things.


% $(4,3,1)$:\\
% First look at the case where $8$ appears in the bottom OC. By Lemma \ref{lemma: removal of top elements and Mk}, $\M_2(T)$ determines $\M_2(R_{8}T)$. The proof of Lemma \ref{lem: bottom} applies to $R_{8}T$ and so the rightmost two elements of the middle row are determined.
% {color{red} check didn't use location of $n$.}

% One possibility is that the middle row ends in $6,7$. Looking at $T' \in \M_2(T)$ with two boxes removed from the middle row of $T$ identifies the last entry of the first row by the maximal value achieved there in $\M_2(T)$. If that entry is $5$. Then Lemma \ref{lemmma: general construct In(m) when m in same place and nothing could slide} applied to the $5$ finishes things. Otherwise, the entry is $4$ and $5$ appears to the left of $6$. Lemma \ref{lemmma: general construct In(m) when m in same place and nothing could slide} applied to the $4$ finishes things.

% The other possibility is that the middle row ends in $7$ with $6$ not appearing to the left or the middle row ends in $6$. Then $6$ appears in the last box of the top row or, respectively, $7$ appears in the last box. Lemma \ref{lemmma: general construct In(m) when m in same place and nothing could slide} applied to the $6$ finishes things.

% The next case is where $8$ appears in the middle OC. The number $7$ must appear to the left of $7$ or in one of the two OCs. These two cases may be distinguished by looking at $T' \in \M_2(T)$ with the middle OC and one other OC removed from $T$. In the first case, $6$ always appears in the middle row. In the later case, it can appear in a SOC.

% When $7$ is to the left of $8$, look at $T' \in \M_2(T)$ with two boxes removed from the middle row of $T$ and $6$ appearing in the middle row. As the $7, 8$ cannot slide, the maximum entry in each remaining box determines its original value.

% If $7$ appears in an OC, the precise OC holding $7$ can by determined by looking at $T' \in \M_2(T)$ with two boxes removed from the middle row of $T$ and $6$ not appearing in the middle row. In this case, $6$ can be made to appear only in the SOC of the original position of $7$.

% If $7$ is in the bottom OC, look at $T' \in \M_2(T)$ with no boxes removed from the top row of $T$. The top row is determined by the maximal value obtained. As the remaining numbers only fit in one way, we are done.

% Turn to the case of $7$ appearing in the top OC. The bottom entry is determined by the maximal entry among $T' \in \M_2(T)$ with the bottom OC as a SOC. If it is $5$ or $6$, apply Lemma \ref{lemmma: general construct In(m) when m in same place and nothing could slide} to it and we are done. If it is $3$ or $4$, the bottom two numbers of the first row are at most $4$ and $5$ and $6$ are adjacent to $8$. Looking at the minimal value achieved in $T' \in \M_2(T)$ with no boxes removed from the top row of $T$ will determine the value of the entry above $8$. Apply Lemma \ref{lemmma: general construct In(m) when m in same place and nothing could slide} to it, and we are done.

% Turn to the case where $8$ appears in the top OC. Then $7$ appears in an OC too. By looking at $T' \in \M_2(T)$ with two OCs removed. If $6$ never appears in the remaining SOC, we can determine if one of the OCs has a starting value of at most $5$. Call this value $\beta$. In this case, $\betta$ happens in the lower OC and $7$ appears in the middle OC. If $\beta = 5$, apply Lemma \ref{lemmma: general construct In(m) when m in same place and nothing could slide} to it, and we are done. Otherwise, $5$ and $6$ are adjacent to $7$. Looking at the minimal value achieved in $T' \in \M_2(T)$ with no boxes removed from the top row of $T$ will determine the value of the entry above $7$. Apply Lemma \ref{lemmma: general construct In(m) when m in same place and nothing could slide} to it, and we are done.

% Finally, consider the case where $6$ and $7$ appear in the two lower OCs. Then $4$ and $5$ are adjacent to the middle OC. Look at $T' \in \M_2(T)$ with the lower OC as a SOC. This SOC can be made to hold $4$ if and only if $6$ was the original occupant. Therefore $6$ and $7$ are determined.
% Applying Lemma \ref{lemmma: general construct In(m) when m in same place and nothing could slide} to $6$ finishes things up.

% % By looking at the maximal value of $T' \in \M_2(T)$ with no boxes removed from the middle row of $T$, the value to the left of the middle OC is determined. Therefore, the location of $4$ and $5$ are known. Applying Lemma \ref{lemmma: general construct In(m) when m in same place and nothing could slide} to the entry above the middle OC identifies the location of $2$ and $3$.



% % Figure environment removed
% \begin{center}
%     $(4,2^2)$ Entry Identification
% \end{center}
% Reference Figure \ref{figure: (4,2^2)} for the shape.
% By Lemmas \ref{lem: 2+ overhang for k = 2} and \ref{lem: bottom}, the top row and bottom OC are known. This also determines the entry in the second row, first column, by the minimum of the remaining numbers.

% If $7, 8$ appear in the top row, then $6$ is in the bottom OC and we are done by Lemma \ref{lemmma: general construct In(m) when m in same place and nothing could slide}. On the opposite extreme, if $7, 8$ are not in the top row, then the final entry of the top row is in $4, 5, 6$. If it $6$, then Lemma \ref{lemmma: general construct In(m) when m in same place and nothing could slide} does the trick. If it is $4$ or $5$, look at $T' \in \M_2(T)$ with two boxes removed from the top row. The minimum value appearing in the two unknown boxes is the original value minus two.

% If $8$ appears in the top OC and $6$ is adjacent to $8$, then $7$ is in the lower OC. Look at $T' \in \M_2(T)$ with two boxes removed from the top row. The maximum value appearing in the two unknown boxes is the original value. A similar argument works when $7$ appears in the top OC and $6$ is adjacent to $7$.

% If $8$ appears in the top OC, but $6$ is not adjacent to $8$, then $7$ is in the lower OC and $6$ is adjacent to it. To find the location of $6$ and finish this case, look at $T' \in \M_2(T)$ with two boxes removed from the top row. The unknown box that has a $5$ locates the original position of $6$. A similar argument works when $7$ appears in the top OC, but $6$ is not adjacent to $7$.




% % Figure environment removed
% \begin{center}
%     $(4,2,1^2)$ Entry Identification
% \end{center}
% Reference Figure \ref{figure: (4,2,1^2)} for the shape.
% By Lemma \ref{lem: 2+ overhang for k = 2}, the top row is known. This also determines the entry in the second row, first column, by the minimum of the remaining numbers.

% If one of the lower OCs ha


% $(3^2,2)$.










% \newpage

% %-----%-----%-----%
% %-----%-----%-----%
% \section{AFTER THIS IS BAD/OLD}
% %-----%-----%-----%
% %-----%-----%-----%





% %-----%-----%-----%
% %-----%-----%-----%
% \section{Determination of $T$ when $n \geq k^2 + 2k + 2$}
% %-----%-----%-----%
% %-----%-----%-----%

% In the proof below, we write $\R_{> m}T$ to denote the result of removing all entries in $T$ that are greater than $m$, if any exist.


% \begin{lemma}
% \label{lemma: Mk-1 from Mk}
%     Suppose $n \geq k^2 + 2k + 2$.  Let $T \in \YT(n)$ and $U=\R_{[k^2+2k+1,\:n]}T \in \YT(k^2+2k)$.  Then $\M_{k-1}(U)$ is determined by $\M_k(T)$.
% \end{lemma}

% \begin{proof}
%     First we show that every element of $\M_{k-1}(U)$ is a subtableau of an element of $\M_k(T)$. Let $U' \in \M_{k-1}(U)$.  Then we have $U' \in \YT(k^2 + k + 1)$. By definition, $U'$ is the result of deleting a (not necessarily unique) sequence $(c_1, \ldots, c_{k-1})$ of cells in $U$.  Let $\mu$ be the shape of $U$.  Now consider the element $T' \in \M_k(T)$ obtained by deleting the sequence $(c_1, \ldots, c_{k-1})$ from $\R_n T$.  Since $n \geq k^2 + 2k +2$, the cell $n$ of $T$ lies outside $\mu$; therefore, $U'$ can be obtained from $T'$ by first removing all cells of $T'$ lying outside $\mu$, and then removing all remaining entries greater than $|U'| = k^2 + k +1$.
%     Since $|U|-|U'| = k-1$, we observe that
%     \begin{equation}
%     \label{size of mu minus small T}
%    \Big|\mu \setminus \text{shape of } \R_{>k^2+k+1}T' \Big| = k-1.
%     \end{equation}
%   Now we use this observation to reverse the procedure: consider all those elements $T' \in \M_k(T)$ such that~\eqref{size of mu minus small T} is satisfied.
%  We know that there is at least one such tableau corresponding to each element of $\M_{k-1}(U)$.  Hence, upon removing the cells of each such $T'$ lying outside $\mu$, and then removing any entries greater than $k^2 + k +1$, we obtain precisely $\M_{k-1}(U)$.
% \end{proof}

% \begin{theorem}
%     \label{theorem:k-minors}
%     Let $T \in \YT(n)$, with $n \geq k^2 + 2k + 2$.  Then $T$ is determined by $\M_k(T)$.
% \end{theorem}

% \begin{proof}
%     We prove this by induction on $k$.  In the base case  $k=1$, we have $k^2 + 2k + 2 = 5$, and it is known from \cite[Thm.~3.7]{CainLehtonen2022} that $T \in \YT(n)$ is determined by $\M_1(T)$ when $n \geq 5$.

%     Now assume that each $T \in \YT(n)$ is determined by $\M_{k-1}(T)$ whenever $n \geq (k-1)^2 + 2(k-1) + 2 = k^2 + 1$.  Let $T \in \YT(n)$.  By Lemma~\ref{lemma:location of k2 + 2k through n}, the locations in $T$ of the entries in $(k^2 +2k,\:n]$ are determined by $\M_k(T)$.  It remains to determine the entries in $U := \R_{[k^2 + 2k +1, \: n]}T$.  By Lemma~\ref{lemma: Mk-1 from Mk} we know that $\M_{k-1}(U)$ is determined by $\M_k(T)$; furthermore, since $U \in \YT(k^2 + 2k)$, and since $k^2 + 2k \geq k^2 + 1$, it follows from the induction hypothesis that $U$ is determined by $\M_{k-1}(U)$.  Hence the entire tableau $T$ is determined by $\M_k(T)$.
% \end{proof}



% %-----%-----%-----%
% %-----%-----%-----%

% \section{$k = 2$}

% %-----%-----%-----%
% %-----%-----%-----%








% %-----%-----%-----%
% %-----%-----%-----%
% \section{Determination of $T$ when $n \geq k^2 + 2k + 1$}
% %-----%-----%-----%
% %-----%-----%-----%







% %-----%-----%-----%
% %-----%-----%-----%
% \section{Closing Remarks}
% %-----%-----%-----%
% %-----%-----%-----%

% Actually $k=k^2 + 2k$ unless 3 shapes...

% Sharpness??



















% {\color{red} REDOING EVERYTHING... after this is old stuff...}



% %-----%-----%-----%
% %-----%-----%-----%
% \section{Finding the Location of $n$}
% %-----%-----%-----%
% %-----%-----%-----%


% In this section we give results for finding the location of $n$ in $T$ from $\M_k(T)$.

% % \begin{lemma}

% %     Let $n,k\in\N$ and $T\in\YT(n)$ with OCs $x_1, \ldots, x_\ell$. Suppose $|\Out(x_j)| \leq k $ for at most one OC. Then the location of $n$ is recoverable from $M_k(n)$.
% % \end{lemma}

% % \begin{proof}
% %     Suppose first that $|\Out(x_j)| \geq k + 1$ for all $j$, $1\leq j \leq \ell$. Then by deleting $k$ cells in $\Out(x_j)$, there exists $T'\in \M_k(T)$ in which the cell, $C_j$, of $x_j$ is a SOC. Find the minimum entry in $C_j$, $m_j$, among all $T'\in\Out(x_j)$ in which $C_j$ survives. Since nothing can slide into the cell $C_j$, the entry there in $T'$ is
% %     $x_j$ minus the number of deleted cells who were less than $x_j$.
% %     If $x_j = n$, $m_j = n - k$ as every other cell is less than $n$.
% %     If $x_j < n - k$, then certainly $m_j < n - k$.
% %     If $n - k \leq x_j < n$, only the entries $x_j + 1, x_j + 2, \ldots, n$ are larger than $x_j$ in $\Out(x_j)$. Thus there are at least
% %     $d_j = k + 1 - (n - x_j)$ entries in $\Out(x_j)$ that are less than $x_j$. Choose any $k$ entries of $\Out(x_j)$ that include at least $d_j$ entries that are less than $x_j$. This results in a $T'$ for which $C_j$ survives and has an entry of at most
% %     $x_j - d_j = n - k - 1$. Thus $m_j < n - k$. As a result, $x_j$ is $n$ if and only if $m_j = n - k$.

% %     This allows the location of $n$ to be determined when $|\Out(x_j)| \geq k + 1$ for all $j$. Moreover, if all but one of the $|\Out(x_j)| \geq k + 1$, then the same analysis applies to all but one of the $x_j$. As a result, $n$ may be located if it is one of these $x_j$. If it is not one of these, as $n$ must lie in an OC, it must be located in the only remaining $x_j$.
% % \end{proof}







% % If at least two OCs, $x_j$, have $|\Out(x_j)| \leq k$, it turns out that the configuration for $T$ is very limited, at least when $n \geq k^2 +2k$, the lower bound for recovering the shape of $T$ from the shapes of $\M_k(T)$, Equation \ref{equation: Monks shape determination bound}.



% % \begin{lemma} \label{lemma: bad shape with small outer area}
% %     Let $n,k\in\N$ and $T\in\YT(n)$ with OCs $x_1, \ldots, x_\ell$. Suppose $|\Out(x_j)| \leq k$ for at least two OCs. Then $|T| \leq k^2 +2k$. Equality holds if and only if $T$ has shape $((k+1)^k, k)$, \emph{i.e.}, the shape of a $(k+1)\times (k+1)$ square with the lower right cell removed, see Figure \ref{figure: k+1 square minus lower right corner}.
% % \end{lemma}
% % % Figure environment removed

% % \begin{proof}
% %     Let $x, y$ be two OCs of $T$ with outer area at most $k$. We will give an algorithm that modifies $T$ by moving cells and possibly increasing $|T|$ while retaining at least two OCs with outer area at most $k$. The algorithm will end with the shape $((k+1)^k, k)$ to finish the proof.

% %     We may assume that $x$ is below $y$. Begin by moving all cells of $T$ that are not in either $\In(x)$ or $\In(y)$ below $\In(x)$ in the shape of a rectangle of width $1$. See Figure \ref{figure: move cells pic 1}. Write $T_1$ for the new Young diagram.
% %     % Figure environment removed

% %     Note that $|T_1|=|T|$. If $x$ is not in the first column, then $x,y$ remain OCs with $|\Out(x)|, |\Out(y)|$ unchanged.
% %     If $x$ is in the first column, then $y$ remains an OC with $|\Out(y)|$ unchanged. However, the other OC shifts to the bottom of the first column. Its outer area may decrease. In this case, relabel the OC as $x$.

% %     Next, let $a$ be the number of cells below $y$ and $b$ be the number of cells to the right of $x$. Form the Young Diagram $T_2$ to be of shape
% %     $((b+1)^a,b)$. See Figure \ref{figure: move cells pic 2}.
% %     % Figure environment removed

% %     In $T_2$, label the bottom OC $x$ and the top OC $y$. Observe that $|T_2| \geq |T_1|$  and that
% %     $|\Out(x)|, |\Out(y)|$ do not change.

% %     Finally, look at all possible Young diagrams consisting of a rectangle minus the lower right cell whose two outer areas are at most $k$. The maximum area is achieved by the shape $((k+1)^k, k)$.
% % \end{proof}







% % For the YT with shape $((k+1)^k, k)$ that arises in Lemma \ref{lemma: bad shape with small outer area},
% % see Figure \ref{figure: k+1 square minus lower right corner}, we will show the location of $n$ is determined by the $k$-minors.

% % \begin{lemma} \label{lemma: movement of n-k in column/row}
% %     Let $k\in\N$, $n = (k+1)^2 -1$, and $T\in\YT(n)$ with shape $((k+1)^k, k)$. Then the value of each OC is at least $n-k$. The set $\M_k(T)$ determines whether each OC sits in $[n-k,n-2]$, or $[n-1,n]$. Moreover, in the first case, the exact value of the OC is determined.
% % \end{lemma}

% % \begin{proof}
% %     $T$ has the shape of a $(k+1) \times (k+1)$ rectangle minus its lower right cell, see Figure \ref{figure: k+1 square minus lower right corner}.
% %     Label the lower OC as $x$ and the upper OC as $y$. Observe that $n$ must be one of them. Write $X$ for the set of cells of $T$ that are in the same row as $x$ and $Y$ for the cells in the same column as $y$. As the analysis for $X$ and $Y$ are similar, we will work only with $Y$.

% %     The fact that each OC is at least $n-k$ follows from the fact that the outer area of an OC contains larger elements than the OC, but only has area $k$.

% %     First observe that when $y$ is either $n$ or $n-1$, it is possible to find some $T'\in\M_k(T)$ with an $n-k$ in any cell of $Y$ by first deleting $n$ if $y= n-1$ and then deleting upper cells in $Y$.

% %     If $y=n-2$, then to find some $T'\in\M_k(T)$ with an $n-k$ in a cell of $Y$ would need some cell of $Y$ to be $n-i$ with $2\leq i < n -1$ that becomes an $n-k$ in $Y$.
% %     This requires $k-i$ cells to be deleted less than $n-i$, $i$ cells to be deleted greater than $n-i$, and cells above $n-i$ to slide out of $Y$ or be deleted. Since the cells larger than $n-i$ will not cause $n-i$ to slide, the farthest $n-i$ can slide and become an $n-k$ is $k-i$ cells up.
% %     When $i=2$, $n-2$ sits at the bottom of $Y$ and, at most, could result in an $n-k$ up to the $1+k-2=k-1$ cell of $Y$, counting from the bottom. When $i>2$ and $n-2$ sits in the $j$th cell of $Y$, counting from the bottom, then $i\geq 1+j$. In this case, at most, the $n-k$ could also only reach the $j + k-i \leq k-1$ cell of $Y$.
% %     Moreover, by deleting the appropriate number of cells in $Y$ and $X$, an $n-k$ may be achieved in any of the cells of $Y$ from the bottom to position $k-1$.

% %     A similar analysis hold for $y=n-i_0$, $2\leq i_0 \leq k$ except that an $n-k$ will only appear in any of the cells of $Y$ from the bottom to position $k+i_0 -1$.
% % \end{proof}







% \begin{lemma} \label{lemma: location of n-k in bad shape}
%     Let $k\in\N$, $n = (k+1)^2 -1$, and $T\in\YT(n)$ with shape $((k+1)^k, k)$. Then $\M_k(T)$ determines the location of $n-k$.
% \end{lemma}

% \begin{proof}
%     $T$ has the shape of a $(k+1) \times (k+1)$ rectangle minus its lower right cell, see Figure \ref{figure: k+1 square minus lower right corner}.
%     Label the lower OC as $x$ and the upper OC as $y$. Observe that $n$ must be one of them. Write $X$ for the set of cells of $T$ that are in the same row as $x$, $Y$ for the cells in the same column as $y$, and $Z$ for $\In(x)\cap \In(y)$.

%     By Lemma \ref{lemma: movement of n-k in column/row}, we know if $n-k$ is an OC and its location in that case. Assume now that $n-k$ is not an OC.

%     Write $\mathcal{C}$ for the set of cells of $T$ that satisfy: (1) there is a minor in $\M_k(t)$ in which the cell holds an $n-k$, (2) there is a minor in $\M_k(t)$ in which the cell holds an $n-2k$, and (3) for any minor in $\M_k(t)$ for which the cell exists, its entry is at least $n-2k$.

%     We begin by showing that the original location of $n-k$ resides in $\mathcal{C}$. For (1), delete all cells in $(n-k, n]$. Next, as $n-k$ is not an OC, $|\Out(n-k)| \geq 2k$. Since there are only $k$ elements larger than $n-k$ and deleting elements of $\Out(n-k)$ do not move $n-k$, we may delete $k$ elements from $\Out(n-k)$ that are smaller than $n-k$ to achieve (2). Finally, (3) follows since a cell, at most, can only be decreased by $1$ for every cell deleted.

%     If $i_0 \in \mathcal{C}$, then we show that $n-2k \leq i_0 \leq n-k$. The right inequality follows from (2) and the fact that a cell can, at most, be reduced by $k$. The left inequality follows from (3) and deleting $(n-k,n]$. Moreover, if $i_0  \not= n-k$, we show that $|\Out(i_0)| \leq 2k$. Write $i_0 = n - k -i$ for $1\leq i \leq k$ so that there are $k+i$ elements larger than $i_0$.
%     If $|\Out(i_0)| \geq 2k+1$, this would leave at least $k-i+1$ elements of $\Out(i_0)$ smaller than $i_0$. Deleting these elements and $i-1$ elements from $(n-k, n]$ will leave the cell unmoved and have an entry of $n-k-1$. This violates (3).

%     If $i_0 \in \mathcal{C}$, but $i_0  \not= n-k$, we call it an imposter. If no imposters exist, we know the location of $n-k$. They cannot occur in an OC as the outer area would be too small to accommodate all larger elements. In addition, their outer area must be at most $2k$. This is highly constraining and forces their possible locations to be adjacent to an OC. See Figure \ref{figure: possible n-k imposter locations}.
%     % Figure environment removed




%     Suppose there exist a minor in $\M_k(T)$ whose cell holds $n-k$ and another minor in which the same cell holds $n-2k$. As the only cells that can become an $n-k$ lie in $[n-k,n]$????????????????

%     If we determine the location of $n-k$ and there exists a $T'\in \M_k(T)$ in which the $n-k$ cell becomes $n-2k$, then we claim that we know the location of all cells in $[1,n-k]$. This is because the only cells that could slide into the cell of $n-k$ are elements of $[n-k,n]$. Of those, the only one that could become $n-2k$ is $n-k$ So, for this $T'$, $n-k$ does not slide and every element of $[n-k+1,n]$ was deleted. This means that all the cells initially labeled with $[1,n-k]$ did not move or change value. Thus they are known.?????

%     By Lemma \ref{lemma: movement of n-k in column/row}, we know that each OC sits in either $[n-k,n-2]$, or $[n-1,n]$. Moreover, in the first case, the exact value of the OC is also determined. One of the OCs is necessarily $n$.

%     Look first at the the case where one OC is $n-k$. Then Lemma \ref{lemmma: construct In(m) when m in same place} does as needed.



% \end{proof}






% \begin{lemma} \label{lemma: determination of T in bad shape}
%     Let $k\in\N$, $n = (k+1)^2 -1$, and $T\in\YT(n)$ with shape $((k+1)^k, k)$. Then $T$ is reconstructable from $M_k(T)$.
% \end{lemma}

% \begin{proof}
%     $T$ has the shape of a $(k+1) \times (k+1)$ rectangle minus its lower right cell, see Figure \ref{figure: k+1 square minus lower right corner}.
%     Label the lower OC as $x$ and the upper OC as $y$. Observe that $n$ must be one of them. Write $X$ for the set of cells of $T$ that are in the same row as $x$ and $Y$ for the cells in the same column as $y$.










%     By definition of a ST, note that a label $m$ in $T$ must be an OC, or the cells to the right and below (if they exist) must be labeled with elements of $[m+1, \ldots, n]$. As a consequence, the only possibly OC values lie in $[n-k, \ldots, n]$. Also, be definition of the jeu de taquin, the only labels of $T$ that can result in an eventual label of $n-k$ in any $T'\in \M_k(T)$ are $[n-k,\ldots,n]$.

%     First we show that $M_k(T)$ determines whether $n-k$ is an OC or not. By symmetry, it is enough to consider the case of determining whether $y$ is $n-k$ or not.

%     If $y=n-k$, then $[n-k+1,n]$ sits in $X$. Therefore, the only way to achieve an $n-k$ in some $T'\in\M_k(T)$ that sits in a subset of $Y$ is by deleting everything in $[n-k+1,\ldots,n]$. In particular, there exist a $T'$ with $n-k$ appearing in $y$'s cell and this is the only cell of $Y$ that can hold an $n-k$.

%     If $y=n-i$ with $0\leq i < k$, then $[n-i+1,\ldots,n]$ sits in $X$. By deleting $[n-i+1,\ldots,n]$ and $k-i$ elements of $Y$ above $y$, we obtain a $T'$ with $n-k$ sitting in $Y$, but not in $y$'s cell. As a result, we can tell if $y$ is $n-k$ or not.

%     Next we show that $M_k(T)$ determines whether $n-k$ is in $X$ or $Y$ and, if so, its location.




% \end{proof}













% %-----%-----%-----%
% %-----%-----%-----%
% \section{Reduction of Problem}
% %-----%-----%-----%
% %-----%-----%-----%

% Set $M_k$ essentially commutes with $n$ removal as sets.

% Enough to do certain base cases (location of everything) and know, in general, location of $n$ from $M_k$.


% %-----%-----%-----%
% %-----%-----%-----%
% \section{Determination of Base Case--may change with new ideas}
% %-----%-----%-----%
% %-----%-----%-----%


% Outer reduction. So $n$ location known except some base cases.

% Only case to consider is $(k + 1)^2 -1$.



% %-----%-----%-----%
% %-----%-----%-----%
% \section{Location of $n$}
% %-----%-----%-----%
% %-----%-----%-----%
% \begin{center}
% \scalebox{.75}{
% \begin{tikzpicture}
% \draw (0,5) -- (5,5);
% \draw (0,0) -- (0,5);
% \draw (0,0) -- (4,0);
% \draw (5,5) -- (5,1);
% \fill[blue!45] (2,0) -- (2,1) -- (3,1) -- (3,0);
% \fill[blue!45] (3,1) -- (3,2) -- (4,2) -- (4,1);
% \fill[blue!45] (4,2) -- (4,3) -- (5,3) -- (5,2);
% \fill[green!45] (3,0) -- (3,1) -- (4,1) -- (4,0);
% \fill[green!45] (4,1) -- (4,2) -- (5,2) -- (5,1);
% \draw (2,0) rectangle (3,1) node[pos=.5] {$A_1$};
% \draw (3,1) rectangle (4,2) node[pos=.5] {$A_2$};
% \draw (4,2) rectangle (5,3) node[pos=.5] {$A_3$};
% \draw (4,1) rectangle (5,2) node[pos=.5] {$n-1$};
% \draw (3,0) rectangle (4,1) node[pos=.5] {$n$};
% \end{tikzpicture}
% \qquad
% \begin{tikzpicture}
% \draw (0,5) -- (5,5);
% \draw (0,0) -- (0,5);
% \draw (0,0) -- (4,0);
% \draw (5,5) -- (5,1);
% \fill[blue!45] (2,0) -- (2,1) -- (3,1) -- (3,0);
% \fill[blue!45] (3,1) -- (3,2) -- (4,2) -- (4,1);
% \fill[blue!45] (4,2) -- (4,3) -- (5,3) -- (5,2);
% \fill[green!45] (3,0) -- (3,1) -- (4,1) -- (4,0);
% \fill[green!45] (4,1) -- (4,2) -- (5,2) -- (5,1);
% \draw (2,0) rectangle (3,1) node[pos=.5] {$A_1$};
% \draw (3,1) rectangle (4,2) node[pos=.5] {$A_2$};
% \draw (4,2) rectangle (5,3) node[pos=.5] {$A_3$};
% \draw (4,1) rectangle (5,2) node[pos=.5] {$n$};
% \draw (3,0) rectangle (4,1) node[pos=.5] {$n-1$};
% \end{tikzpicture}}
% \end{center}
% \begin{lemma}
% Suppose $T$ appears as in Figure. Then we can determine the location of $n-2$ in $T$ using the $k-$minors of $T$.
% \end{lemma}
% \begin{proof}
% Suppose $T$ appears as above.
% Then $n-2$ must appear in the blue blocks.
% Without loss of generality suppose $n-2$ appears in $A_i$.

% Claim: $n-(k+3)$ never appears in $A_i'$ for any $T' \in M_k(T)$.\\
% Suppose toward contradiction $n-(k+3)$ appears in $A_i'$ for some $T' \in M_k(T)$.
% Then, either $n-2, n-1,$ or $n$ in $T$ maps to $n-(k+3)$ in $T'$. This is impossible as we are only taking $k$ minors.
% Thus , $n-(k+3)$ never appears in $A_i'$ for any $T' \in M_k(T)$.

% Similarly, $n-(k+s)$ never appears in $A_i'$ for any $T' \in M_k(T)$ with $s \geq 3$.

% Claim: For $j \neq i$, $A_j' \leq n-(k+3)$ for some $T' \in M_k(T)$.\\
% Suppose  $A_j' > n-(k+3)$ for all $T' \in M_k(T)$. Then, $A_j > n-3$. This is contradiction as $A_j < n-2$ for $j \neq i$.
% \end{proof}

% %-----%-----%-----%
% %-----%-----%-----%
% \section{More Ideas...}
% %-----%-----%-----%
% %-----%-----%-----%
% \begin{definition}
%     Let $T$ be a Young tableau. We use ${Sh(T)}$ to denote the shape of $T$.
% \end{definition}

% \begin{lemma}\label{removing one cell from a sub tableau}
%     Let $n<m$ and $L\in YT(n), T\in YT(m)$ with $L\subset T$.
%     Then, for any $w\leq n$, $L-(w) = (T-(w))|_{Sh(L)} \setminus (m, \dots, n).$
% \end{lemma}
% \begin{proof}
%     After removing $w$, the cells in $L$ and the corresponding cells in $T$ will be slid the same way.
%     To put it another way, $T|_{Sh(L)} - (w) = L-(w)$ since $T|_{Sh(L)} = L.$
%     As for the cells in $T\setminus L$, each of these has an entry [before the removal of $w$] that is greater than $n$.
%     After the jeu de taquin, these cells have an entry larger than $n-1$, so they are removed from $(T-(w))|_{Sh(L)} \setminus (m, \dots, n).$
%     So, every cell in $T$ either corresponds to a cell in $L$, in which case it's location in $T-(w)$ and $L-(w)$ are the same, or does not correspond to a cell in $L$ in which case it gets ``ignored".
% \end{proof}



% For the rest section, let fix $k$ and let $n\geq k^2+2k+2.$
% Let $T\in YT(n)$ and let $L = T - (n, n-1,\dots, k^2+2k+2)\in YT(k^2+2k+1)).$ In this way, $T$ is an extension of $L$.


% \begin{lemma}
%     Let $Z\in M_{k-1}(L).$
%     Then there exists $S\in M_k(T)$ with $S|_{Sh(L)} \setminus (n,\dots, k^2+2k+3)=Z.$
% \end{lemma}
% \begin{proof}
%     Since $Z\in M_{k-1}(L),$ there exists an ordered list $(j_1,\dots, j_{k-1})$ of entries in $L$ such that $L-(j_1, \dots, j_{k-1}) = Z$.
%     Notice that $n\notin L$ so $L\subset T-(n).$
%     The result follows by repeatedly applying Lemma \ref{removing one cell from a sub tableau} to $T-(n), T-(n, j_1), T-(n, j_1, j_2),\dots, T-(n, j_1, j_2\dots, j_{k-1}).$
% \end{proof}

% \begin{lemma}
%     If $S\in M_k(T)$ and $S|_{Sh(L)}$ has exactly $k-1$ cells that are either empty or larger than $k^2+k+2$,  then $S|_{Sh(L)} \setminus (n,\dots, k^2+k+3)\in M_{k-1}(L)$
% \end{lemma}
% \begin{proof}
%     Since $S\in M_k(T),$ there is an ordered set of elements such that $T-(j_1, j_2,\dots, j_k) = S.$
%     Let $Z = S|_{Sh(L)}.$

%     Since $Z$ has exactly $k-1$ cells that are either empty or contain elements larger than any element found in any $Y\in M_{k-1}(L),$ we must have that $k-1$ many elements of $(j_1, j_2,\dots, j_{k})$ are also elements in $L.$
%     That is, there is only one element from $(j_1,\dots j_k)$ that is greater than $k^2+2k+1,$ the largest element in $L.$

%     Suppose this element is $j_m.$ Then every element from $$(j_1, j_2,\dots, j_{m-1}, j_{m+1}, j_{m+2},\dots, j_k)$$ is in $L$ and $Z = L-(j_1, j_2,\dots, j_{m-1}, j_{m+1}, j_{m+2},\dots, j_k)$.
% \end{proof}






