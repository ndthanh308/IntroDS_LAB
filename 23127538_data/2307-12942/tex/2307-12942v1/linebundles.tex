\documentclass[a4paper,11pt]{amsart}
\usepackage{amsthm, amsfonts,amsmath,amscd,amssymb, amsrefs}
\usepackage{mathrsfs}
\usepackage{mathtools}
\usepackage{tikz-cd}
\usepackage[hidelinks]{hyperref}
\usepackage{setspace}
%\doublespacing
\usepackage[margin=1.03in]{geometry}
\usepackage{stmaryrd}

\title{Line Bundles on The First Drinfeld Covering
}
\author{James Taylor}
\email{james.taylor@maths.ox.ac.uk}
\address{Mathematical Institute, University of Oxford, Oxford, OX2 6GG, UK}
\date{\today}
\subjclass[2020]{11S37, 14G22, 14C22.}
\keywords{Picard group, Drinfeld tower}
%-----------------------

\newcommand\cHH{\check{\mathrm{H}}}
\newcommand\HH{\mathrm{H}}


\DeclareMathOperator{\Compl}{\textbf{Compl}}
\DeclareMathOperator{\Set}{\textbf{Set}}
\DeclareMathOperator{\FSch}{\textbf{FSch}}
\DeclareMathOperator{\FGL}{\textbf{FGL}}
\DeclareMathOperator{\Nilp}{\textbf{Nilp}}
\DeclareMathOperator{\SFD}{SFD}
\DeclareMathOperator{\Cart}{Cart}

\newcommand{\OO}{\mathcal{O}}
\newcommand{\Om}{\Omega}
\newcommand{\hOm}{\widehat{\Omega}}
\DeclareMathOperator{\et}{\acute{e}t}
\DeclareMathOperator{\Ab}{Ab}
\DeclareMathOperator{\Lie}{Lie}
\DeclareMathOperator{\Mod}{Mod}
\DeclareMathOperator{\id}{id}
\DeclareMathOperator{\pr}{pr}
\DeclareMathOperator{\PSh}{PSh}
\DeclareMathOperator{\supp}{supp}
\DeclareMathOperator{\height}{ht}
\DeclareMathOperator{\Grass}{Grass}
\DeclareMathOperator{\Flag}{Flag}
\DeclareMathOperator{\Sh}{Sh}
\DeclareMathOperator{\coker}{coker}
\DeclareMathOperator{\Frac}{Frac}
\DeclareMathOperator{\PAb}{PAb}
\DeclareMathOperator{\Obj}{Obj}
\DeclareMathOperator{\Hom}{Hom}
\DeclareMathOperator{\Mor}{Mor}
\DeclareMathOperator{\Ext}{Ext}
\DeclareMathOperator{\Gal}{Gal}
\DeclareMathOperator{\End}{End}
\DeclareMathOperator{\Cov}{Cov}
\DeclareMathOperator{\GL}{GL}
\DeclareMathOperator{\PGL}{PGL}
\DeclareMathOperator{\SL}{SL}
\DeclareMathOperator{\Aut}{Aut}
\DeclareMathOperator{\Stab}{Stab}
\DeclareMathOperator{\Nrd}{Nrd}
\DeclareMathOperator{\Sp}{Sp}
\DeclareMathOperator{\ad}{ad}
\DeclareMathOperator{\Fr}{Fr}
\DeclareMathOperator{\Spec}{Spec}
\DeclareMathOperator{\Proj}{Proj}
\DeclareMathOperator{\Spf}{Spf}
\DeclareMathOperator{\Spv}{Spv}
\DeclareMathOperator{\Spa}{Spa}
\DeclareMathOperator{\rank}{rank}
\DeclareMathOperator{\Cont}{Cont}
\DeclareMathOperator{\Rat}{Rat}
\DeclareMathOperator{\Nm}{Nm}
\DeclareMathOperator{\Br}{Br}
\DeclareMathOperator{\gr}{gr}
\DeclareMathOperator{\im}{im}
\DeclareMathOperator{\cd}{cd}
\DeclareMathOperator{\lcm}{lcm}
\DeclareMathOperator{\Pic}{Pic}
\DeclareMathOperator{\PicCon}{PicCon}
\DeclareMathOperator{\Con}{Con}
\DeclareMathOperator{\Der}{Der}
\DeclareMathOperator{\dlog}{dlog}
\DeclareMathOperator{\texttors}{tors}
\DeclareMathOperator{\bsp}{\mathbf{Sp}}
\DeclareMathOperator{\px}{\partial_x}
\newcommand{\fn}[3]{#1 \colon #2 \rightarrow #3}
\newcommand{\tors}[1]{#1_{\texttors}}
\newcommand{\twopartdef}[4]
{
	\left\{
		\begin{array}{ll}
			#1 & \mbox{if } #2 \\
			#3 & \mbox{if } #4
		\end{array}
	\right.
}

\newcommand{\bA}{{\mathbb A}}
\newcommand{\bB}{{\mathbb B}}
\newcommand{\bC}{{\mathbb C}}
\newcommand{\bD}{{\mathbb D}}
\newcommand{\bF}{{\mathbb F}}
\newcommand{\bE}{{\mathbb E}}
\newcommand{\bG}{{\mathbb G}}
\newcommand{\bH}{{\mathbb H}}
\newcommand{\bI}{{\mathbb I}}
\newcommand{\bK}{{\mathbb K}}
\newcommand{\bN}{{\mathbb N}}
\newcommand{\bP}{{\mathbb P}}
\newcommand{\bQ}{{\mathbb Q}}
\newcommand{\bR}{{\mathbb R}}

\newcommand{\sA}{{\mathcal A}}
\newcommand{\sB}{{\mathcal B}}
\newcommand{\sC}{{\mathcal C}}
\newcommand{\sD}{{\mathcal D}}
\newcommand{\sE}{{\mathcal E}}
\newcommand{\sF}{{\mathcal F}}
\newcommand{\sG}{{\mathcal G}}
\newcommand{\sH}{{\mathcal H}}
\newcommand{\sI}{{\mathcal I}}
\newcommand{\sJ}{{\mathcal J}}
\newcommand{\sL}{{\mathcal L}}
\newcommand{\sM}{{\mathcal M}}
\newcommand{\sN}{{\mathcal N}}
\newcommand{\sO}{{\mathcal O}}
\newcommand{\sP}{{\mathcal P}}
\newcommand{\sQ}{{\mathcal Q}}
\newcommand{\sR}{{\mathcal R}}
\newcommand{\sU}{{\mathcal U}}
\newcommand{\sS}{{\mathcal S}}
\newcommand{\sT}{{\mathcal T}}
\newcommand{\sV}{{\mathcal V}}
\newcommand{\sX}{{\mathcal X}}
\newcommand{\sY}{{\mathcal Y}}


\newcommand{\fA}{{\mathfrak A}}
\newcommand{\fa}{{\mathfrak a}}
\newcommand{\fb}{{\mathfrak b}}
\newcommand{\fg}{{\mathfrak g}}
\newcommand{\fm}{{\mathfrak m}}
\newcommand{\fp}{{\mathfrak p}}
\newcommand{\fq}{{\mathfrak q}}
\newcommand{\fX}{{\mathfrak X}}

\newcommand{\Id}{{\ensuremath{\mathrm{Id}}}}

\newcommand{\vc}{\ensuremath{\mathbf{c}}\xspace}
\newcommand{\ve}{\ensuremath{\mathbf{e}}\xspace}
\newcommand{\vd}{\ensuremath{\mathbf{d}}\xspace}
\newcommand{\vE}{\ensuremath{\mathbf{E}}\xspace}
\newcommand{\vf}{\ensuremath{\mathbf{f}}\xspace}
\newcommand{\vg}{\ensuremath{\mathbf{g}}\xspace}
\newcommand{\vh}{\ensuremath{\mathbf{h}}\xspace}
\newcommand{\vi}{\ensuremath{\mathbf{i}}\xspace}
\newcommand{\vj}{\ensuremath{\mathbf{j}}\xspace}
\newcommand{\vk}{\ensuremath{\mathbf{k}}\xspace}
\newcommand{\vm}{\ensuremath{\mathbf{m}}\xspace}
\newcommand{\vv}{\ensuremath{\mathbf{v}}\xspace}
\newcommand{\vw}{\ensuremath{\mathbf{w}}\xspace}
\newcommand{\vx}{\ensuremath{\mathbf{x}}\xspace}
\newcommand{\vy}{\ensuremath{\mathbf{y}}\xspace}
\newcommand{\vz}{\ensuremath{\mathbf{z}}\xspace}

\newcommand{\bU}{\ensuremath{\mathbf{U}}}
\newcommand{\bV}{\ensuremath{\mathbf{V}}}
\newcommand{\bX}{\ensuremath{\mathbf{X}}}
\newcommand{\bY}{\ensuremath{\mathbf{Y}}}
\newcommand{\bZ}{\ensuremath{\mathbb{Z}}}
\newcommand{\bi}{\ensuremath{\mathbf{i}}}
\newcommand{\bj}{\ensuremath{\mathbf{j}}}
\newcommand{\bk}{\ensuremath{\mathbf{k}}}
\newcommand{\bn}{\ensuremath{\mathbf{n}}}
\newcommand{\bS}{\ensuremath{\mathbf{S}}}


%TheoremEnvironments

\newtheorem{thm}{Theorem}
\newtheorem*{thm*}{Theorem}
\numberwithin{thm}{section}
\newtheorem{lemma}[thm]{Lemma} 
\newtheorem{prop}[thm]{Proposition} 
\newtheorem{cor}[thm]{Corollary} 
\newtheorem{conj}[thm]{Conjecture} 

\theoremstyle{definition}
\newtheorem{defn}[thm]{Definition}
\newtheorem{eg}[thm]{Example}
\newtheorem{remark}[thm]{Remark}

\theoremstyle{remark} 
\newtheorem*{note}{Note}

\newenvironment{Proof}[1][Proof]
  {\proof[#1]\leftskip=1cm\rightskip=1cm}
  {\endproof}

%\doublespace

\begin{document}
\begin{abstract}
	Let $\Omega^d$ be the \mbox{$d$-dimensional} Drinfeld symmetric space for a finite extension $F$ of~$\bQ_p$. Let $\Sigma^1$ be a geometrically connected component of the first Drinfeld covering of $\Omega^d$ and let $\bF$ be the residue field of the unique degree $d+1$ unramified extension of $F$. We show that the natural homomorphism
	\[
	\widehat{(\bF, +)}\rightarrow \Pic(\Sigma^1)[p]
	\]
	determined by the second Drinfeld covering is injective.
	Here $\widehat{(\bF, +)}$ is the group of characters of $(\bF, +)$.
	In particular, $\Pic(\Sigma^1)[p] \neq 0$. We also show that all vector bundles on $\Omega^1$ are trivial, which extends the classical result that $\Pic(\Omega^1) = 0$.
\end{abstract}
\maketitle

\section{Introduction}

Let $p$ be a prime, $F$ a finite extension of $\bQ_p$, and $L$ the completion of the maximal unramified extension of $F$. The \emph{Drinfeld tower} is a system of $d$-dimensional rigid analytic spaces over $L$,
\[
	\sM_0 \leftarrow \sM_1 \leftarrow \sM_2 \leftarrow \cdots,
\]
for which the spaces $\sM_n$ are equipped with compatible actions of $D^\times \times \GL_{d+1}(F)$, where $D$ is the division algebra with invariant $1/(d+1)$ over $F$~\cite{DRI, BC, RZ}. For $d = 1$, this tower has been shown to realise both the local Langlands and the Jacquet-Langlands correspondence for $\GL_2(F)$ in its \'{e}tale cohomology~\cite{CAR, HARR, BOY, HARTAY}, and when $F = \bQ_p$ encode part of the $p$-adic local Langlands correspondence for $\GL_2(\bQ_p)$~\cite{CDN1}.

The base space $\sM_0$ is non-canonically a disjoint union over $\bZ$ of copies of $\Omega^d$, the $d$-dimensional Drinfeld symmetric space. This is the admissible open subset of $\bP^{d, \text{an}}$ defined by removing all $F$-rational hyperplanes. The space $\sM_1$ has been studied by many authors \cite{TEIT, WANG, LP, JUNEQN}, and recently Junger has shown that, after a finite extension of $L$, the preimage of each copy of $\Omega^d$ in $\sM_1$ is a disjoint union of $(q-1)$ copies of $\Sigma^1$, a particular geometrically connected Kummer-type cyclic Galois covering of $\Omega^d$ \cite{JUNEQN}. Little is known about the geometry of the higher covering spaces $(\sM_n)_{n \geq 2}$.

% When considered as an open subset of $\sM_0$, $\Omega$ is naturally considered as a moduli space of $p$-divisible groups, and is a fundamental example of a Rapoport-Zink space \cite{RZ}. The space $\Omega$ is also of importance in other areas of number theory, for example $\Omega$ can be interpreted as a uniformisation space for Shimura curves, and is used to define invariants of higher weight modular forms \cite{DT}.

It is a classical result that $\Pic(\Omega^1) = 0$ \cite{FVDP}, and recently this has been generalised to higher dimensions and more general hyperplane arrangements by Junger \cite{JUNCOH}. Understanding the Picard groups of the covering spaces $(\sM_n)_{n \geq 1}$ is much more difficult, and almost nothing is known in this context. Previously, we showed that for $d = 1$ there is no $p$-torsion in the Picard group of any open subset of $\Sigma^1$ which lies above a vertex of the Bruhat-Tits tree \cite{JT}.

In this paper, we show that in any dimension $\Pic(\Sigma^1)[p] \neq 0$. Furthermore, assuming the ground field of $\Sigma^1$ contains a primitive $p$th root of $1$, setting $G \coloneqq \SL_{d+1}(\OO_F)$, and writing $\bF$ for the residue field of the unique degree $d+1$ unramified extension of $F$, we prove the following.

\begin{thm*}[Theorem \ref{mainthm}]
There is an injective homomorphism,
\[
	\widehat{(\bF, +)} \hookrightarrow \Pic(\Sigma^1)[p]^G.
\]
\end{thm*}
Here $\widehat{(\bF, +)}$ is the group of characters of $(\bF, +)$. The homomorphism is constructed as follows. For any Galois covering $f \colon X \rightarrow Y$ with abelian Galois group $H$ of exponent $e$, there is a decomposition
\[
	f_* \OO_X = \bigoplus_{\chi \in \widehat{H}} \sL_{\chi}, \qquad \sL_{\chi} \coloneqq e_{\chi} \cdot f_* \OO_X.
\]
Furthermore, $\sL_{\chi} \in \Pic(Y)[e]$ for any $\chi \in \widehat{H}$, and the association
\[
	\widehat{H} \rightarrow \Pic(Y)[e], \qquad \chi \mapsto \sL_{\chi},
\]
is a group homomorphism (cf.\ Proposition \ref{abeliangaloisprop}). For any $n \geq 1$, $\sigma_n \colon \sM_{n+1} \rightarrow \sM_{n}$ is a Galois covering with Galois group $(1 + \Pi^n \OO_D) / (1 + \Pi^{n+1}\OO_D)$, which is canonically identified with $(\bF, +)$. The homomorphism of Theorem \ref{mainthm} is then the homomorphism associated to the abelian Galois covering $\sigma_1 \colon \Sigma^2 \rightarrow \Sigma^1$, where $\Sigma^2$ is the preimage of $\Sigma^1$ in $\sM_2$.

Now for the remainder of the introduction let $d = 1$, and set $\Omega \coloneqq \Omega^1$. Our main interest in Theorem \ref{mainthm} is in the following. A precise statement of the $p$-adic local Langlands correspondence is formulated when $F = \bQ_p$ \cite{COL}, and Dospinescu and Le Bras \cite{DLB} have used this to show that for $F = \bQ_p$ and all $n \geq 1$, the $D(G)$-module $\OO(\sM_n)$ is coadmissible, where $D(G)$ is the distribution algebra of $G$. 

The extension $\sigma_0 \colon \Sigma^1 \rightarrow \Omega$ is an abelian Galois covering with Galois group $\Gamma \coloneqq (\bF_{q^2}^\times)^{q-1} \subset \bF_{q^2}^\times$, and using the above formalism $\OO(\Sigma^1)$ can be decomposed as 
\[
	\OO(\Sigma^1) = \bigoplus_{\chi \in \widehat{\Gamma}} \sL_{\chi}(\Omega).
\]
In their forthcoming work \cite{AW} Ardakov and Wadsley show, without any restriction on $F$, that for any $\chi \neq 1$, $\sL_{\chi}(\Omega)$ is a topologically irreducible coadmissible $D(G)$-module. If $\chi = 1$ then $\sL_{\chi}(\Omega) = \OO(\Omega)$, which is understood as a $D(G)$-module \cite{ORL}. The main idea of \cite{AW} is to understand the $G$-equivariant line bundle with connection $\sL_{\chi}$ on $\Omega$, and then push this to $\bP^{1,\text{an}}$ and use the $G$-equivariant Beilinson-Bernstein correspondence \cite{ARD} to deduce properties of the corresponding $D(G)$-module $\sL_{\chi}(\Omega)$. 

We would like to use similar techniques to understand the global sections of the higher Drinfeld coverings as $D(G)$-modules. For the second covering, we consider the extension $\sigma_1 \colon \Sigma^2 \rightarrow \Sigma^1$, and want to understand the $G$-equivariant line bundles with connection $\sL_{\chi} = e_{\chi} \cdot \sigma_{1,*}\OO_{\Sigma^2}$ for any $\chi \neq 1$. In this context, Theorem \ref{mainthm} says that the underlying line bundle of each $\sL_{\chi}$ is non-trivial. This is in contrast to what happens for $\sigma_0 \colon \Sigma^1 \rightarrow \Omega$ because $\Pic(\Omega) = 0$. The analysis developed in \cite{AW} of $G$-equivariant line bundles with connection works under the assumption that the underlying line bundle is trivial, and Theorem \ref{mainthm} shows that none of the line bundles $\sL_{\chi}$ fit into this formalism.

% Therefore, rather than understanding $\sL_{\chi}$ as a $G$-equivariant line bundle with connection on $\Sigma^1$, one could instead push $\sL_{\chi}$ to $\Omega$ and try and understand $\sigma_{0,*} \sL_{\chi}$ directly as a $G$-equivariant vector bundle with connection on $\Omega$. We show that all vector bundles on $\Omega$ are trivial (Corollary \ref{cor2}), which suggests the above approach is more feasible. This extends and uses the classical result that $\Pic(\Omega) = 0$. The key property we use is that $\OO(\Omega)$ is a Pr\"{u}fer domain, which is unknown to hold in higher dimensions.

In this paper we also show that all vector bundles on $\Omega$ are trivial (Corollary \ref{cor2}), which extends and uses the classical result that $\Pic(\Omega) = 0$. The key property we use is that $\OO(\Omega)$ is a Pr\"{u}fer domain, which is unknown to hold in higher dimensions. In the context of the above discussion, this suggests that rather than first understanding $\sL_{\chi} = e_{\chi} \cdot \sigma_{1,*}\OO_{\Sigma^2}$ as a $G$-equivariant line bundle with connection on $\Sigma^1$ and then pushing to $\Omega$, a potentially more feasible approach is to try and analyse $\sigma_{0,*} \sL_{\chi}$ directly as a $G$-equivariant vector bundle with connection on $\Omega$.

% Therefore, rather than the above approach of first understanding $\sigma_{1,*}\OO_{\Sigma^2}$ and then pushing to $\Omega$, a reasonable alternative is to try and analyse $(\sigma_{0} \circ \sigma_{1})_* \OO_{\Sigma^2}$ directly as a $G$-equivariant vector bundle with connection on $\Omega$. This approach is potentially more feasible, as all vector bundles on $\Omega$ are trivial (Corollary \ref{cor2}). We provide a proof of this fact, which extends and uses the result that $\Pic(\Omega) = 0$. The key property we use is that $\OO(\Omega)$ is a Pr\"{u}fer domain, which is unknown to hold in higher dimensions.

\subsubsection*{Acknowledgements} The author would like to thank Konstantin Ardakov for many useful discussions and both Konstantin Ardakov and Simon Wadsley for making available their preprint \cite{AW}. This research was financially supported by the EPSRC.

\subsection*{Notation} 

Throughout we shall use the following notation. $F$ is a finite extension of $\bQ_p$, with ring of integers $\OO_F$, uniformiser $\pi$, and residue field $\bF_q$. $K$ is a complete field extension of $F$, $L$ is the completion of the maximal unramified extension of $F$, and $\bC_p$ is the completion of $\overline{F}$. The integer $d \geq 1$ will denote the dimension of the spaces we consider. We set $G = \SL_{d+1}(\OO_F)$ and write $D$ for the division over over $F$ of invariant $1/(d+1)$.

\section{Abelian Galois Coverings}\label{section1}

In this section, we describe how the pushforward of the structure sheaf of an abelian Galois covering decomposes into line bundles. The approach we take here is influenced by the work  of Borevi\v{c} for Kummer extensions of rings \cite{BOR}.

Let $\Gamma$ be an abstract group. We write $\underline{\Gamma}$ for the corresponding constant rigid analytic group over $K$. Recall that that a (right) action of $\underline{\Gamma}$ on a rigid space $X$ over $K$ is a morphism $a \colon X \times \underline{\Gamma} \rightarrow X$ of rigid spaces over $K$ such that the diagrams
% https://q.uiver.app/#q=WzAsNyxbMSwwLCJYIFxcdGltZXMgXFx1bmRlcmxpbmV7XFxHYW1tYX0iXSxbMCwwLCJYIFxcdGltZXMgXFx1bmRlcmxpbmV7XFxHYW1tYX0gXFx0aW1lcyBcXHVuZGVybGluZXtcXEdhbW1hfSJdLFswLDEsIlggXFx0aW1lcyBcXHVuZGVybGluZXtcXEdhbW1hfSJdLFsxLDEsIlgiXSxbMywwLCJYIFxcdGltZXMgXFx1bmRlcmxpbmV7MX0iXSxbNSwwLCJYIFxcdGltZXMgXFx1bmRlcmxpbmV7XFxHYW1tYX0iXSxbNCwxLCJYIl0sWzEsMCwiYSBcXHRpbWVzIHBfMiJdLFsxLDIsInBfWCBcXHRpbWVzIG0iLDJdLFsyLDMsImEiLDJdLFswLDMsImEiXSxbNCw2LCJwX1giLDJdLFs1LDYsImEiXSxbNCw1LCIiLDAseyJzdHlsZSI6eyJ0YWlsIjp7Im5hbWUiOiJob29rIiwic2lkZSI6InRvcCJ9fX1dXQ==
\[\begin{tikzcd}
	{X \times \underline{\Gamma} \times \underline{\Gamma}} & {X \times \underline{\Gamma}} && {X \times \underline{1}} && {X \times \underline{\Gamma}} \\
	{X \times \underline{\Gamma}} & X &&& X
	\arrow["{a \times p_2}", from=1-1, to=1-2]
	\arrow["{p_X \times m}"', from=1-1, to=2-1]
	\arrow["a"', from=2-1, to=2-2]
	\arrow["a", from=1-2, to=2-2]
	\arrow["{p_X}"', from=1-4, to=2-5]
	\arrow["a", from=1-6, to=2-5]
	\arrow[hook, from=1-4, to=1-6]
\end{tikzcd}\]
commute. If in addition $f \colon X \rightarrow Y$ is a morphism of rigid spaces over $K$, then the action is called equivariant with respect to the trivial action of $\underline{\Gamma}$ on $Y$ if
% https://q.uiver.app/#q=WzAsNCxbMCwwLCJYIFxcdGltZXMgXFx1bmRlcmxpbmV7XFxHYW1tYX0iXSxbMCwxLCJYIl0sWzEsMCwiWCJdLFsxLDEsIlkiXSxbMCwxLCJwX1giLDJdLFswLDIsImEiXSxbMSwzLCJmIiwyXSxbMiwzLCJmIl1d
\[\begin{tikzcd}
	{X \times \underline{\Gamma}} & X \\
	X & Y
	\arrow["{p_X}"', from=1-1, to=2-1]
	\arrow["a", from=1-1, to=1-2]
	\arrow["f"', from=2-1, to=2-2]
	\arrow["f", from=1-2, to=2-2]
\end{tikzcd}\]
commutes.
As for schemes, an action of $\underline{\Gamma}$ on $X$ which is equivariant with respect to the trivial action on $Y$ is equivalent to the data of a group homomorphism $\rho \colon \Gamma^{\text{op}} \rightarrow \Aut_Y(X)$, where $\Aut_Y(X)$ is the group of automorphisms of $X$ which respect the morphism $f \colon X \rightarrow Y$.
In this situation the sheaf of $\OO_Y$-modules $f_* \OO_X$ has an (left) action of $\Gamma$,
\[
	\Gamma \rightarrow \Aut_k(f_*\OO_X), \qquad g \mapsto (\rho(g)^\sharp_{f^{-1}(U)} \colon f_*\OO_X(U) \rightarrow f_*\OO_X(U))_{U \subset Y},
\]
which is well-defined as $\rho(g)^\sharp \colon \OO_X \rightarrow \rho(g)_* \OO_X$, and
\[
\rho(g)^{-1}(f^{-1}(U)) = f^{-1}(U)
\]
for any open $U \subset Y$. Therefore we can consider the sheaf of $\OO_Y$-modules $(f_* \OO_X)^{\Gamma}$ defined by
\[
	(f_* \OO_X)^{\Gamma} = \OO_X(f^{-1}(U))^{\Gamma}
\]
for any admissible open subset $U$ of $Y$, which is a sheaf because $(-)^{\Gamma}$ preserves products and equalisers. 

\begin{defn}
Suppose that $\Gamma$ is a finite group, $a \colon X \times \underline{\Gamma} \rightarrow X$ is an action of $\underline{\Gamma}$ on $X$, and $f \colon X \rightarrow Y$ is a finite \'{e}tale morphism of rigid spaces over $K$ which is equivariant with respect to the trivial action of $\underline{\Gamma}$ on $Y$. Then $f \colon X \rightarrow Y$ is a \emph{Galois covering with Galois group $\Gamma$} if the natural map $\OO_Y \rightarrow (f_* \OO_X)^{\Gamma}$ is an isomorphism of $\OO_Y$-modules and
\[
	p_X \times a \colon X \times \underline{\Gamma} \rightarrow X \times_Y X
\]
is an isomorphism of rigid spaces over $K$.
\end{defn}

For the remainder of this section we assume that $\Gamma$ is a finite abelian group, $f \colon X \rightarrow Y$ is a Galois covering with Galois group $\Gamma$, and $K$ contains a primitive $e(\Gamma)$th root of $1$, where $e(\Gamma)$ is the exponent of $\Gamma$. For each $\chi \in \widehat{\Gamma}$ we write $e_{\chi}$ for the corresponding central primitive idempotent
\[
	e_{\chi} = \frac{1}{|\Gamma|} \sum_{\gamma \in \Gamma} \chi(\gamma^{-1}) \gamma \in K[\Gamma].
\]

\begin{defn}
We define the $\OO_Y$-module
\[
	\sL_{\chi} \coloneqq e_{\chi} \cdot f_* \OO_X.
\]
\end{defn}

\begin{prop}\label{abeliangaloisprop}
There is a direct sum decomposition of $\OO_Y$-modules
\[
	f_* \OO_X = \bigoplus_{\chi \in \widehat{\Gamma}} \sL_{\chi},
\]
and multiplication in $f_* \OO_X$ induces an isomorphism
\[
	\sL_{\chi} \otimes_{\OO_Y} \sL_{\psi} \xrightarrow{\sim} \sL_{\chi \psi}.
\]
In particular, each $\sL_{\chi}$ is an $e(\Gamma)$-torsion invertible $\OO_Y$-module, and the association
\[
	\widehat{\Gamma} \rightarrow \Pic(Y)[e(\Gamma)], \qquad \chi \mapsto \sL_{\chi},
\]
is a group homomorphism.
\end{prop}

\begin{proof}
The sheaf $f_*\OO_X$ is an $\OO_Y[\Gamma]$-module, and the direct sum decomposition of $f_*\OO_X$ follows from the fact that the $e_{\chi}$ are central orthogonal idempotents. Suppose now that $U$ is an affinoid open subset of $Y$, and let
\[
V \coloneqq f^{-1}(U) = U \times_Y X \hookrightarrow X.
\]
Then $\underline{\Gamma}$ acts on $V$, $f \colon V \rightarrow U$ is equivariant with respect to the trivial action of $\underline{\Gamma}$ on $U$, and we have a commutative diagram of isomorphisms,
% https://q.uiver.app/#q=WzAsNCxbMCwwLCJVIFxcdGltZXNfWSAoWCBcXHRpbWVzIFxcdW5kZXJsaW5le1xcR2FtbWF9KSJdLFsxLDAsIlUgXFx0aW1lc19ZKFggXFx0aW1lc19ZIFgpIl0sWzEsMSwiKFUgXFx0aW1lc19ZIFgpIFxcdGltZXNfVSAoIFUgXFx0aW1lc19ZIFgpIl0sWzAsMSwiViBcXHRpbWVzIFxcdW5kZXJsaW5le1xcR2FtbWF9Il0sWzEsMl0sWzMsMl0sWzAsM10sWzAsMV1d
\[\begin{tikzcd}
	{U \times_Y (X \times \underline{\Gamma})} & {U \times_Y(X \times_Y X)} \\
	{V \times \underline{\Gamma}} & {(U \times_Y X) \times_U ( U \times_Y X)}
	\arrow[from=1-2, to=2-2]
	\arrow[from=2-1, to=2-2]
	\arrow[from=1-1, to=2-1]
	\arrow[from=1-1, to=1-2]
\end{tikzcd}\]
Write $A \coloneqq \OO(U)$ and $B \coloneqq \OO(V)$. Since $f \colon X \rightarrow Y$ is finite, $V$ is affinoid, and because $\OO_Y \rightarrow (f_* \OO_X)^\Gamma$ is an isomorphism, $A \rightarrow B$ is injective and has image $B^\Gamma$.
Furthermore, because $B$ is finitely generated over $A$, the natural inclusion
\[
	B \otimes_A B \rightarrow B \widehat{\otimes}_A B,
\]
is an isomorphism \cite[3.7.3(6)]{BGR}. Therefore, the composition of this inclusion with the global sections of $p_X \times a$ induces an isomorphism,
\[
	B \otimes_A B \xrightarrow{\sim} B \otimes_K \OO(\Gamma), \qquad x \otimes y \mapsto \sum_{\gamma \in \Gamma} x(\gamma \cdot y) \otimes \delta_{\gamma}.
\]
Now $B$ is a right $\OO(\Gamma)$-comodule algebra for the Hopf algebra $\OO(\Gamma)$ via
\[
	\rho \colon B \rightarrow B \otimes_K \OO(\Gamma), \qquad \rho \colon b \mapsto \sum_{\gamma \in \Gamma} \gamma(b) \otimes \delta_\gamma,
\]
and the above isomorphism says exactly that $A \rightarrow B$ is a $\OO(\Gamma)$-Galois extension in the sense of \cite[Def. 8.1.1]{MONT}. Because $K$ contains a primitive $e(\Gamma)$th root of $1$, the natural map $K[\widehat{\Gamma}] \rightarrow \OO(\Gamma)$ is an isomorphism of Hopf algebras over $K$. Therefore, using this identification we can view $B$ is a $K[\widehat{\Gamma}]$-comodule algebra. We have that for $b \in B$,
\begin{equation}\label{comodulemap}
	\rho(b) = \sum_{\gamma \in \Gamma} \gamma(b) \otimes \delta_{\gamma} = \sum_{\chi \in \widehat{\Gamma}} b_{\chi} \otimes \chi,
\end{equation}
for some unique $b_{\chi} \in B$, and we define
\[
B_{\chi} \coloneqq \{b_{\chi} \mid b \in B, \chi \in \widehat{\Gamma}\}.
\]
Because $A \rightarrow B$ is $K[\widehat{\Gamma}]$-Galois, these $B_{\chi}$ make $B$ a strongly graded $\widehat{\Gamma}$-algebra by a result of Ulbrich \cite[Thm. 8.1.7]{MONT}, meaning that 
\[
	B = \bigoplus_{\chi \in \widehat{\Gamma}} B_{\chi}, \qquad \text{and} \qquad B_{\chi} \cdot B_{\psi} = B_{\chi \psi} \qquad \text{for all} \qquad \chi, \psi \in \widehat{\Gamma}.
\]
In fact, $e_{\chi} \cdot B = B_{\chi}$. Indeed, by column orthogonality
\[
	\delta_{\gamma} = \frac{1}{|\Gamma|} \sum_{\chi \in \widehat{\Gamma}} \chi(\gamma^{-1}) \chi,
\]
and therefore substituting this into equation (\ref{comodulemap}) and comparing the coefficient of $\chi$ shows that,
\[
	e_{\chi} \cdot b = b_{\chi}.
\]
There is a natural surjective morphism of $A$-modules,
\[
	m_{\chi, \psi} \colon B_{\chi} \otimes_A B_{\psi} \rightarrow B_{\chi} \cdot B_{\psi} = B_{\chi \psi}.
\]
In order to show that this is injective, first note that $m_{\chi, \psi}$ forms part of the commutative square
% https://q.uiver.app/#q=WzAsNCxbMCwwLCJCX3tcXGNoaX0gXFxvdGltZXNfQSBCX3tcXHBzaX0iXSxbMSwwLCJCX3tcXGNoaX0gXFxvdGltZXNfQSBCIl0sWzEsMSwiQiJdLFswLDEsIkJfe1xcY2hpIFxccHNpfSJdLFswLDEsIiIsMCx7InN0eWxlIjp7InRhaWwiOnsibmFtZSI6Imhvb2siLCJzaWRlIjoidG9wIn19fV0sWzEsMiwiXFxzaW0iXSxbMCwzLCJtX3tcXGNoaSwgXFxwc2l9IiwyXSxbMywyLCIiLDIseyJzdHlsZSI6eyJ0YWlsIjp7Im5hbWUiOiJob29rIiwic2lkZSI6InRvcCJ9fX1dXQ==
\[\begin{tikzcd}
	{B_{\chi} \otimes_A B_{\psi}} & {B_{\chi} \otimes_A B} \\
	{B_{\chi \psi}} & B
	\arrow[hook, from=1-1, to=1-2]
	\arrow[ from=1-2, to=2-2]
	\arrow["{m_{\chi, \psi}}"', from=1-1, to=2-1]
	\arrow[hook, from=2-1, to=2-2]
\end{tikzcd}\]
and the top arrow is injective because $B_{\chi}$ is flat over $A$. Therefore it is sufficient to show that 
\begin{equation}\label{wtsinjective}
	B_{\chi} \otimes_A B \rightarrow B_{\chi} \cdot B \hookrightarrow B,
\end{equation}
is injective. This homomorphism is surjective, because $B_{\chi}$ generates $B$ as a $B$-module, which follows from the fact that
\[
	B_{\chi} \cdot B_{\chi^{-1}} = B_{1} = e_{1} \cdot B = A.
\]
The homomorphism (\ref{wtsinjective}) is therefore a surjective homomorphism between finitely generated rank $1$ $B$-modules, and as such it is injective.
% Indeed, such a homomorphism is injective if and only it is injective locally, and then the claim follows because surjective maps between finitely generated free modules of the same rank are necessarily injective \cite[Thm. 2.4]{MATSU}.

Now, returning to the global situation, multiplication induces a morphism of $\OO_Y$-modules,
\begin{equation}\label{OYmodhom}
	\sL_{\chi} \otimes_{\OO_Y} \sL_{\psi} \rightarrow f_*\OO_Y.
\end{equation}
Because $f_*\OO_Y$ is coherent, then locally over an affinoid open subset $U$ as above this is identified with the morphism sheaves associated under the associated sheaf construction to the $A$-module homomorphism
\[
	e_{\chi} \cdot B \otimes_{A} e_{\psi} \cdot B \rightarrow B.
\]
We have shown above that the this has image $e_{\chi \psi} \cdot B$, and therefore the morphism of $\OO_Y$-modules (\ref{OYmodhom}) above induces an isomorphism,
\[
	\sL_{\chi} \otimes_{\OO_Y} \sL_{\psi} \xrightarrow{\sim} \sL_{\chi \psi}. \qedhere
\]
\end{proof}

\begin{remark}
In fact one can show that if $Y$ is connected and $\Gamma_0$ is the stabiliser of any connected component $X_0$ of $X$, that $f \colon X_0 \rightarrow Y$ is a Galois extension with Galois group $\Gamma_0$, and the homomorphism
	\[
		\widehat{\Gamma} \rightarrow \Pic(Y)[e(\Gamma)]
	\]
	factors as the composition
	\[
		\widehat{\Gamma} \twoheadrightarrow  \widehat{\Gamma_0} \rightarrow \Pic(Y)[e(\Gamma_0)] \hookrightarrow \Pic(Y)[e(\Gamma)].
	\]
\end{remark}
\section{Drinfeld Symmetric Spaces}

Let $F$ be a finite extension of $\bQ_p$, $L$ the completion of the maximal unramified extension of $F$, and $K$ a complete field extension of $F$. Set $G \coloneqq \SL_{d+1}(\OO_F)$, and let $D$ be the division over $F$ of invariant $1/(d+1)$ with ring of integers $\OO_D$ and uniformiser $\Pi$. Let $\Omega^d$ be the Drinfeld symmetric space of dimension $d$ over $K$, which is the admissible open subset of $\bP^{d, \text{an}}_K$ defined by removing all $F$-rational hyperplanes. This is stable under the natural action of $\GL_{d+1}(F)$ on $\bP^{d,\text{an}}_K$.

The Drinfeld tower is a system of rigid analytic spaces over $L$,
\[
\sM_0 \leftarrow \sM_1 \leftarrow \sM_2 \leftarrow \cdots,	
\]
and each space has an action of $\GL_{d+1}(F) \times D^\times$ for which the transition morphisms are equivariant. 
For background material on these spaces see \cite{DRI, BC, RZ} , or \cite[\S 2]{JUNEQN} for an overview. Setting $\GL_{d+1}^0(F) \coloneqq \{g \in \GL_{d+1}(F) \mid \nu(\det(g)) = 0\}$, there is a non-canonical $\GL_{d+1}^0(F) \times \OO_D^\times$-equivariant isomorphism,
\[
	\sM_0 \xrightarrow{\sim} \bigsqcup_{h \in \bZ} \Omega_{L}^d,
\]
where on the disjoint union $\OO_D^\times$ acts trivially and $\GL_{d+1}^0(F)$ acts on each copy of $\Omega_{L}^d$. Fixing the copy of $\Omega_{L}^d$ in index zero and considering the preimage of $\Omega_{L}^d$ in each covering space $(\sM_n)_{n \geq 1}$, we can consider the tower,
\[
	\Omega_{L}^d \leftarrow \sM_{1}^0 \leftarrow \sM_{2}^0 \leftarrow \cdots.
\]
Because $\Omega_{L}^d$ is stable under the action of $\GL_{d+1}^0(F) \times \OO_D^\times$, for each $n \geq 1$, $\sM_n^0 \subset \sM_n$ is $\GL_{d+1}^0(F) \times \OO_D^\times$-stable. The subgroup $1 + \Pi^n \OO_D \leq \OO_D^{\times}$ acts trivially on $\sM_{n,0}$, and the morphism $\sM_{n,0} \rightarrow \Omega_{L}^d$ is Galois with Galois group $\OO_D^{\times} / (1 + \Pi^n \OO_D)$ \cite[Thm. 2.2]{KOH}.
Each of the spaces $(\sM_n^0)_{n \geq 1}$ is connected over $L$ \cite[Thm. 2.5]{KOH}, but not geometrically connected. The following result is due to Boutot and Zink, and describes the connected components of $(\sM_n^0)_{n \geq 1}$ over $\bC_p$.
Write $\Nrd \colon D^\times \rightarrow F^\times$ for the reduced norm of $D$.
\begin{prop}\label{conncomp}
	There is a family of $\GL_{d+1}^0(F) \times \OO_D^\times$-equivariant bijections
	\[
		\pi_0\left(\sM^0_{n, \bC_p}\right) \xrightarrow{\sim} \frac{\OO_F^\times}{ 1 + \pi^{\lceil\frac{n}{d+1} \rceil}\OO_F}
	\]
	for any $n \geq 1$, compatible with the natural restriction maps on both sides. Here $(g,x) \in \GL_{d+1}^0(F) \times \OO_D^\times$ acts on the right by multiplication by $\det(g)\Nrd(x) \in \OO_F^\times$.
	\end{prop}
	
	\begin{proof}
		This is \cite[Thm. 0.20]{BZ}, noting that $\Nrd(1 + \Pi^n \OO_D) = 1 + \pi^{\lceil\frac{n}{d+1} \rceil}\OO_F$ \cite[Lem. 5]{RIEHM}.
\end{proof}

In this section, we want to give a description of the $G$-invariant mod-$p$ global units of $\Omega^d$, which will use in the next section. Recall that if $R$ is a commutative ring, $\bP^d(R)$ is the set of tuples $(r_0, ... ,r_d) \in R^{d+1}$ such that $R = Rr_0 + \cdots + Rr_{d}$, up to the scaling action $u\cdot(r_0, ... ,r_d) = (ur_0, ... ,ur_d)$ of $R^\times$.

\begin{defn}
	For each $m \geq 1$, let $\sH_m \coloneqq \bP^d(\OO_F / \pi^m \OO_F)$.
\end{defn}

\begin{lemma}\label{actioncor}
For all $m \geq 1$, the action of $G$ on $\sH_m$ is transitive.
\end{lemma}

\begin{proof}
	For notational simplicity, set $R \coloneqq \OO_F / \pi^m \OO_F$. The natural map $G \rightarrow \SL_{d+1}(R)$ is surjective because $\OO_F$ and $R$ are local rings so both groups are generated by elementary matrices \cite[Thm. 4.3.9]{HOL}. The action of $\GL_{d+1}(R)$ on $\sH_m$ is transitive because any element $\mathbf{r} = (r_0, ... ,r_d)$ with $[\mathbf{r}] \in \sH_m$ can be extended to a basis of $R^{d+1}$, which can be seen by reducing mod-$\pi$. Then the action of $\SL_{d+1}(R)$ on $\sH_m$ is transitive, as the stabiliser subgroup of the element $x = [(1 \colon 0 \colon \cdots \colon 0)]$,
	\[
		\Stab_{\SL_{d+1}(R)}(x) \leq \Stab_{\GL_{d+1}(R)}(x)
	\]
	is of index $|R^\times|$, the same as the index of $\SL_{d+1}(R)$ in $\GL_{d+1}(R)$.
\end{proof}

\begin{defn}
For an abelian group $A$ and $m \geq 1$, we write $A[\sH_m]$ for the abelian group
\[
	A[\sH_m] \coloneqq \left\{f \colon \sH_m \rightarrow A \right\},
\]
of all functions from $\sH_m$ to $A$, and
\[
	A[\sH_m]^0 \coloneqq \left\{f \colon \sH_m \rightarrow A \: \middle| \: \sum_{x \in \sH_m} f(x) = 0 \right\} \subset A[\sH].
\]
\end{defn}
For any $m \geq 1$, there is a natural map
\[
	\rho_{m} \colon \sH_{m+1} \rightarrow \sH_m,
\]
which induces
\[
	\rho_{m, *} \colon A[\sH_{m+1}] \rightarrow A[\sH_m],
\]
defined by
\[
	\rho_{m, *}(f)(x) = \sum_{y \in \rho_m^{-1}(x)} f(y),
\]
for all $x \in \sH_m$. This restricts to $\rho_{m, *} \colon A[\sH_{m+1}]^0 \rightarrow A[\sH_m]^0$.
\begin{defn}
	We set
\[
	A[[\sH]]^0 \coloneqq \varprojlim_{m \geq 1} A[\sH_m]^0.
\]
\end{defn}

Because each $A[\sH_m]^0$ is a $\bZ[G]$-module in compatible way, so is $A[[\sH]]^0$. 

\begin{prop}[{\cite[Thm. 4.5(2)]{JUNEQN}}]\label{globunitsomega}
There is an isomorphism of $\bZ[G]$-modules,
	\[
		\OO(\Omega^d_K)^\times / K^\times \xrightarrow{\sim} \bZ[[\sH]]^0.
	\]
\end{prop}

For any $m \geq 1$,
\[
|\sH_m| = q^{(m-1)d} (q^{d+1} - 1) / (q-1),
\]
and the restriction map
\[
	\rho_{m} \colon \sH_{m+1} \rightarrow \sH_m,
\]
is surjective with each fibre of size $q^d$.

In the proof of the next lemma, we will make use of the following element.

\begin{defn}
For each $m \geq 1$, let $\Sigma_m \in \bZ / p \bZ \, [\sH_m]$ be defined by,
\[
	\Sigma_m(x) = 1,
\]
for all $x \in \sH_m$.
\end{defn}

\begin{lemma}\label{nomodpGinvariants}
$(\bZ / p \bZ \, [[\sH]]^{0})^G = 0$.
\end{lemma}

\begin{proof}
	For any $m \geq 1$, we have projection maps,
	\[
		\phi_{m} \colon (\bZ / p \bZ \, [[\sH]]^0)^G \rightarrow (\bZ / p \bZ  \, [\sH_m]^0)^G.
	\]
	Suppose that we have some $G$-invariant function, $f \in (\bZ / p \bZ \, [[\sH]]^0)^G$.
	Then for any $m \geq 1$, because $\sH_{m+1}$ is a finite set with a transitive action of $G$ (by Lemma \ref{actioncor}),
	\[
	\phi_{m+1}(f) = \lambda \Sigma_{m+1},
	\]
	for some $\lambda \in \bZ / p \bZ$. Now,
	\[
		\rho_{m} \colon \sH_{m+1} \rightarrow \sH_m,
	\]
	is surjective with each fibre of size $q^d$, hence $\phi_{m}(f) = q^d \lambda \Sigma_{m} = 0$, as $p \mid q$. Therefore, $\phi_{m}(f) = 0$ for all $m \geq 1$, and hence $f = 0$.
\end{proof}

\begin{cor}\label{globalinvunitsomega}
The inclusion $K^\times \rightarrow \OO(\Omega)^\times$ induces an isomorphism,
\[
	K^\times / K^{ \times p} \xrightarrow{\sim} \left(\OO(\Omega^d)^\times / \OO(\Omega^d)^{ \times p} \right)^G.
\]
\end{cor}

\begin{proof}
	We have a short exact sequence of $\bZ[G]$-modules,
	\[
		0 \rightarrow K^\times \rightarrow \OO(\Omega^d)^\times \rightarrow \bZ[[\sH]]^0 \rightarrow 0,
	\]
	and applying $- \otimes \bZ / p \bZ$, we obtain an exact sequence of abelian groups,
	\begin{equation}\label{exactseq1}
		\bZ[[\sH]]^0[p] \rightarrow K^\times / K^{\times p} \rightarrow \OO(\Omega^d)^\times / \OO(\Omega^d)^{ \times p} \rightarrow \frac{\bZ[[\sH]]^0}{p \bZ [[\sH]]^0} \rightarrow 0.
	\end{equation}
	Because $p$-torsion commutes with taking the inverse limit,
	\[
		\bZ[[\sH]]^0[p] = \varprojlim_{m \geq 1} \bZ[\sH_m]^0[p] = 0.
	\]
	Furthermore, we have an exact sequence of inverse systems
	\[
		0 \rightarrow (\bZ[\sH_m]^0)_{m \geq 1} \xrightarrow{\times p} (\bZ[\sH_m]^0)_{m \geq 1} \rightarrow (\bZ / p \bZ \,[\sH_m]^0)_{m \geq 1} \rightarrow 0,
	\]
	and,
	\[
		\varprojlim_{m \geq 1}{}^1 \bZ[\sH_m]^0 = 0,
	\]
	because each transition map is surjective, thus the natural map
	\[
		\frac{\bZ[[\sH]]^0}{p \bZ [[\sH]]^0} \xrightarrow{\sim} \bZ / p \bZ \, [[\sH]]^0,
	\]
	is an isomorphism. Therefore, taking the $G$-invariants of the exact sequence (\ref{exactseq1}) above,
	\[
		0 \rightarrow K^\times / K^{ \times p} \rightarrow \left(\OO(\Omega^d)^\times / \OO(\Omega^d)^{ \times p} \right)^G \rightarrow (\bZ / p \bZ \, [[\sH]]^0)^G.
	\]
	Then the conclusion follows by Lemma \ref{nomodpGinvariants}.
\end{proof}

\section{Line Bundles on the First Drinfeld Covering}

Recall that we write $L$ for the completion of the maximal unramified extension of $F$, and that $K$ is a complete field extension of $F$. Let $\varpi \in \overline{F}$ be a primitive $(q-1)$st root of $-\pi$. In this section we assume that $K$ contains $L(\varpi)$. The extension $L(\varpi)$ is the first Lubin-Tate extension of $L$, and as such is independent of the choice of $\pi$ \cite[Thm. 3]{LT}. 

We are interested in the space $\sM_1$, which admits the following explicit description due to Junger.
\begin{defn}
	If $X$ is a rigid space over $K$, then for any $n\geq 1$, the \emph{Kummer map},
	\[
	\kappa \colon \OO(X)^\times \rightarrow \HH^1_{\text{\'{e}t}}(X, \mu_n)
	\]
	sends $u \in \OO(X)^\times$ to
	\[
		X(u^{\frac{1}{n}}) \coloneqq \underline{\text{Sp}}_{X}(\OO_X[z] / z^n - u).
	\]
\end{defn}
Let $N \coloneqq q^{d+1} -1$ and $N' \coloneqq N / (q-1)$. In \cite[Thm. 4.9]{JUNEQN} it is shown that
\[
	\sM_1^0 \cong \Omega_L \left(\left(\pi u^{q-1} \right)^{\frac{1}{N}}\right),
\]
for some particular $u \in \OO(\Omega)^\times$. Note that because $L$ contains all coprime to $p$ roots of $1$, $L(\varpi)$ contains a primitive $(q-1)$st root $\tau$ of $\pi$. Therefore, over $L(\varpi)$,
\[
	\sM_{1, L(\varpi)}^0 \cong \Omega_{L(\varpi)} \left(\left((\tau u)^{q-1} \right)^{\frac{1}{N}}\right) \cong 
	\bigsqcup_{\zeta^{q-1} = 1} \Sigma^1_{\zeta},
\]
where
\[
	\Sigma^1_{\zeta} \coloneqq \Omega_{L(\varpi)}\left(\left( \zeta \tau u \right)^{\frac{1}{N'}}\right).
\]	

\begin{defn}
We let $\Sigma^1 \coloneqq \Sigma^1_{1, K}$, and let $\Sigma^2$ be the preimage of $\Sigma^1$ in $\sM_{2, K}^0$.
\end{defn}

As a consequence of Proposition \ref{conncomp}, because $\lceil\frac{1}{d+1} \rceil = 1 = \lceil\frac{2}{d+1} \rceil$, we have the following.

\begin{cor}\label{geomconn}
$\Sigma^1$ and $\Sigma^2$ are geometrically connected.
\end{cor}

\begin{remark}
	By Proposition \ref{conncomp} the $\Sigma^1_{\zeta}$ are the geometrically connected components of $\sM_{1, L(\varpi)}^0$. Furthermore, these components are all isomorphic, as 
	\[
	\Nrd \colon \OO_D / (1 + \Pi \OO_D) \rightarrow \OO_F / (1 + \pi \OO_F)
	\]
	is surjective so the Galois group of $\sM_{1, L(\varpi)}^0 \rightarrow \Omega^d_{L(\varpi)}$ acts transitively on these components.
\end{remark}

The extension
\[
	\sM_{2,K} \rightarrow \sM_{1, K}
\]
is Galois with Galois group
\[
H \coloneqq (1 + \Pi \OO_D) / (1 + \Pi^2 \OO_D).
\]
The extension $\Sigma^2 \rightarrow \Sigma^1$ is the restriction of this Galois covering to the open subset $\Sigma^1$ of $\sM_{1, K}$, and therefore is also Galois with Galois group $H$. From Proposition \ref{conncomp} we note that $\GL_{d+1}^0(F)$ acts through the determinant on the geometrically connected components of the tower and thus $G$ stabilises both $\Sigma^1$ and $\Sigma^2$. Furthermore, the action of $G$ on both $\Sigma^1$ and $\Sigma^2$ commutes with the Galois action.

\begin{prop}\label{globalinvunitssigma}
The inclusion $K^\times \rightarrow \OO(\Omega)^\times$ induces an isomorphism,
	\[
		K^\times / K^{ \times p} \xrightarrow{\sim} \left(\OO(\Sigma^1)^\times / \OO(\Sigma^1)^{ \times p} \right)^G.
	\]
\end{prop}

\begin{proof}
	Let $\sigma$ be a primitive $N$th root of $\pi$. Then by \cite[Thm. 5.1]{JUNEQN}, there is a short exact sequence of abelian groups
	\[
		0 \rightarrow \OO(\Omega_{K(\sigma)})^\times \rightarrow \OO(\Sigma^1_{K(\sigma)})^\times \rightarrow \bZ / (q+1) \bZ \rightarrow 0.
	\]
	Taking $\Gal(K(\sigma) / K(\varpi))$-invariants and applying $- \otimes \bZ / p \bZ$, we are left with an isomorphism
	\[
		\OO(\Omega^d)^\times / \OO(\Omega^d)^{ \times p} \xrightarrow{\sim} \OO(\Sigma^1)^\times / \OO(\Sigma^1)^{ \times p}.
	\]
	The result then follows from Corollary \ref{globalinvunitsomega}.
\end{proof}

We now want to show that the homomorphism
\[
\widehat{H} \rightarrow \Pic(\Sigma^1)[p]
\]
associated to the Galois covering $f \colon \Sigma^2 \rightarrow \Sigma^1$ is injective. In order to prove this, we will make use of the following explicit description of the Kummer exact sequence.

Recall that if $X$ is a rigid space over $K$, the for any $n \geq 1$ the Kummer exact sequence is the short exact sequence
\[
	0 \rightarrow \OO(X)^\times / \OO(X)^{\times n} \rightarrow \HH^1_{\et}(X, \mu_n) \rightarrow \Pic(X)[n] \rightarrow 0,
\]
arising from the long exact sequence of the functor $\Gamma(X_{\et}, -)$ applied to the sequence
\[
	0 \rightarrow \mu_n \rightarrow \bG_m \xrightarrow{\times n} \bG_m \rightarrow 0,
\]
of sheaves of $X_{\et}$, which is exact because $n$ is invertible in $K$ \cite[\S 3.2]{PJ}. There is a more explicit description of this sequence, which we summarise now. References in the case of schemes are \cite[Tag 03PK]{STACK}, \cite[\S III.4]{MIL}, from which the case for rigid spaces can be deduced mutatis mutandis.

Let $\{(\sL, \alpha)\} / \! \cong$ be the set of pairs $(\sL, \alpha)$, where $\sL \in \Pic(X)$ and $\alpha \colon \sL^{\otimes n} \xrightarrow{\sim} \OO_X$ is an $\OO_X$-linear isomorphism, considered up to the natural notion of isomorphism. The set $\{(\sL, \alpha)\} /\! \cong$ forms an abelian group and there is an isomorphism of short exact sequences,
% https://q.uiver.app/#q=WzAsMTAsWzAsMCwiMCJdLFs0LDAsIjAiXSxbNCwxLCIwIl0sWzAsMSwiMCJdLFsxLDAsIlxcT08oWCleXFx0aW1lcyAvIFxcT08oWClee1xcdGltZXMgbn0iXSxbMSwxLCJcXE9PKFgpXlxcdGltZXMgLyBcXE9PKFgpXntcXHRpbWVzIG59Il0sWzIsMCwiXFxISF4xX3tcXGV0fShYX3tcXGV0fSwgXFxtdV9uKSJdLFsyLDEsIlxceyhcXHNMLCBcXGFscGhhKVxcfSAvIFxcY29uZyJdLFszLDAsIlxcUGljKFgpW25dIl0sWzMsMSwiXFxQaWMoWClbbl0iXSxbMyw1XSxbNSw3XSxbNyw5XSxbOSwyXSxbOCwxXSxbNiw4XSxbNCw2XSxbMCw0XSxbNCw1LCI9IiwyXSxbNiw3LCJcXHNpbSIsMl0sWzgsOSwiPSIsMl1d
\[\begin{tikzcd}
	0 & {\OO(X)^\times / \OO(X)^{\times n}} & {\{(\sL, \alpha)\} / \! \cong} & {\Pic(X)[n]} & 0 \\
	0 & {\OO(X)^\times / \OO(X)^{\times n}} & {\HH^1_{\et}(X_{\et}, \mu_n)} & {\Pic(X)[n]} & 0
	\arrow[from=2-1, to=2-2]
	\arrow[from=2-2, to=2-3]
	\arrow[from=2-3, to=2-4]
	\arrow[from=2-4, to=2-5]
	\arrow[from=1-4, to=1-5]
	\arrow[from=1-3, to=1-4]
	\arrow[from=1-2, to=1-3]
	\arrow[from=1-1, to=1-2]
	\arrow["{=}"', from=1-2, to=2-2]
	\arrow["\sim"', from=1-3, to=2-3]
	\arrow["{=}"', from=1-4, to=2-4]
\end{tikzcd}\]
The homomorphism $\{(\sL, \alpha)\} /\!  \cong \: \rightarrow \Pic(X)[n]$ is simply $[(\sL, \alpha)] \mapsto [\sL]$.
Given a pair $[(\sL, \alpha)]$, then the associated $\mu_n$-torsor in $\HH^1_{\et}(X, \mu_n)$ is $Z \coloneqq \underline{\Sp}(\sA)$, where $\sA$ is the coherent sheaf of $\OO_X$-algebras
\[
	\sA = \bigoplus_{i = 0}^{n-1} \sL^{\otimes i},
\]
with multiplication the natural maps
\[
\begin{array}{ll}
	\sL^{\otimes i} \otimes \sL^{\otimes j} \rightarrow \sL^{i + j} & \qquad \mbox{if } i + j \leq n-1, \\
	\sL^{\otimes i} \otimes \sL^{\otimes j} \rightarrow \sL^{i + j} \xrightarrow{\alpha} \sL^{i + j - n} & \qquad \mbox{if } i+j \geq n,
\end{array}
\]
for $0 \leq i,j \leq n$. In order to describe the structure of $Z$ as a $\mu_n$-torsor, we first consider this construction locally.

Suppose that $\sL = \OO_X$. In this case, $\alpha \colon \OO_X^{\otimes n} \rightarrow \OO_X$, and we can use the canonical morphism $\psi \colon \OO_X \rightarrow \OO_X^{\otimes n}$ to define $a \coloneqq \alpha(\psi(1)) \in \OO_X(X)^\times$. Then under the construction above,
\[
Z = \underline{\Sp}(\OO_X[z] / (z^n - a)).
\]
For any rigid space $Y$ over $X$, $Z(Y) = \{s \in \OO_Y(Y) \mid s^n = a\}$, which has the structure of a $\mu_n$-torsor via
\[
	\mu_n(Y) \times Z(Y) \rightarrow Z(Y), \qquad (\zeta, s) \mapsto \zeta s.
\]
Now for a general pair $[(\sL, \alpha)]$, the associated space $Z$ is locally in the rigid topology of the above form, and these structures patch to give $Z$ the structure of a $\mu_n$-torsor.

Suppose now that $K$ contains a primitive $n$th root of $1$. In this case the group scheme $\mu_n$ is naturally identified with the constant group scheme $\underline{\mu_n(K)}$, and under this identification there is a correspondence between $\mu_n$-torsors and Galois covering $Z \rightarrow X$ with Galois group $\underline{\mu_n(K)}$ (in the language of Section \ref{section1}).

We are interested in the homomorphism $\HH^1_{\et}(X,\mu_n) \rightarrow \Pic(X)[n]$. From the description of the $\mu_n$-action above, we see that if a Galois covering $f \colon Z \rightarrow X$ corresponds to the pair $[(\sL, \alpha)]$, we can recover $\sL$ as the line bundle
\[
	\sL \cong e_{\iota} \cdot f_*\OO_Z,
\]
where $\iota$ is the natural inclusion $\iota \colon \mu_n(K) \rightarrow K^\times$. More generally, $f \colon Z \rightarrow X$ is a Galois covering with Galois group $\Gamma$, and $\chi \colon \Gamma \xrightarrow{\sim} \mu_n(K)$ is an isomorphism, then in the exact sequence
\[
	0 \rightarrow \OO(X)^\times / \OO(X)^{\times n} \rightarrow \HH^1_{\et}(X, \underline{\Gamma}) \rightarrow \Pic(X)[n] \rightarrow 0,
\]
the image of the Galois covering $f \colon Z \rightarrow X$ in $\Pic(X)[n]$ is the line bundle $e_{\chi} \cdot f_* \OO_{Z}$.

\begin{thm}\label{mainthm}
Suppose that $K$ contains a primitive $p$th root of $1$. Then the homomorphism
\[
	\widehat{H} \rightarrow \Pic(\Sigma^1)[p]^G, \qquad \chi \mapsto \sL_{\chi} = e_{\chi} \cdot f_*\OO_{\Sigma^2},
\]	
is injective.
\end{thm}

\begin{remark}
	When $F$ is unramified the assumption that $K$ contains a primitive $p$th root of $1$ in the statement of Theorem \ref{mainthm} is superfluous. Indeed, $K$ contains $L(\varpi)$, and the Lubin-Tate extensions $\bQ_p(\zeta_p)$ and $\bQ_p((-p)^{1 / (p-1)})$ of $\bQ_p$ are equal.
\end{remark}

\begin{proof}
	Let $\chi \colon H \rightarrow K^\times$ be non-trivial. We want to show that $e_{\chi} \cdot f_*\OO_{\Sigma^2} \in \Pic(\Sigma^1)$ is non-trivial. Because $H$ has exponent $p$ and $\chi$ is non-trivial, $\chi$ induces an isomorphism
	\[
		\chi' \colon H / H_{\chi} \xrightarrow{\sim} \mu_p(K),
	\]
	where $H_{\chi}$ is the kernel of $\chi$. From $H_{\chi}$ we may form the quotient 
	\[
		f' \colon \Sigma^2 / H_{\chi} \rightarrow \Sigma^1.
	\]
	If $U \subset \Sigma^1$ is an admissible open subset, and $V = f^{-1}(U) \subset \Sigma^2$, then above $U$ the quotient $\Sigma^2 / H_{\chi}$ is described by $\Sp(\OO(V)^{H_{\chi}})$. Because $H_{\chi}$ is normal, $f' \colon \Sigma^2 / H_{\chi} \rightarrow \Sigma^1$ is Galois with Galois group $H / H_{\chi}$, which follows from \cite[Thm. 2.2]{CHR} and the fact that each property in the definition of a Galois extension checked affinoid locally.
	
	We first note that we have an equality of $\OO_{\Sigma^1}$-modules,
	\[
		e_{\chi} \cdot f_*\OO_{\Sigma^2} = e_{\chi'} \cdot f'_* \OO_{\Sigma^2 / H_{\chi}}.
	\]
	Indeed, for any admissible open subset $U$ of $\Sigma^1$,
	\[
		(e_{\chi'} \cdot f'_* \OO_{\Sigma^2 / H_{\chi}})(U) = e_{\chi'} \cdot \OO_{\Sigma^2}(f^{-1}(U))^{H_{\chi}},
	\]
	and
	\[
		(e_{\chi} \cdot f_*\OO_{\Sigma^2})(U) = e_{\chi} \cdot \OO_{\Sigma^2}(f^{-1}(U)).
	\]
	Setting $B \coloneqq \OO_{\Sigma^2}(f^{-1}(U))$, we have that
	\begin{align*}
		e_{\chi}\cdot B &= \{b \in B \mid h(b) = \chi(h)b \text{ for all } h \in H\}, \\
		e_{\chi'} \cdot B^{H_{\chi}} &= \{b \in B^{H_{\chi}} \mid h(b) = \chi(h)b \text{ for all }h \in H / H_{\chi}\},
	\end{align*}
	and it is direct to check that these are equal. Therefore we are reduced to showing that $e_{\chi'} \cdot f'_* \OO_{\Sigma^2 / H_{\chi}}$ is non-trivial.

	Now because the action of $G$ on $\Sigma^2$ and $\Sigma^1$ commutes with the action of $H$, $G$ acts on $\Sigma^2 / H_{\chi}$, $f' \colon \Sigma^2 / H_{\chi} \rightarrow \Sigma^1$ is $G$-equivariant, and the $G$-action commutes with the action of $H / H_{\chi}$. Therefore the covering $f' \colon \Sigma^2 / H_{\chi} \rightarrow \Sigma^1$ defines an element of 
	$\HH^1_{\et}(\Sigma^1, \underline{H / H_{\chi}})^G$ \cite[\S 4.1]{JUNEQN}, the middle term of the $G$-invariants of the Kummer exact sequence
	\begin{equation}\label{Ginvkummer}
		0 \rightarrow \left( \OO(\Sigma^1)^\times / \OO(\Sigma^1)^{\times p} \right)^G \rightarrow \HH^1_{\et}(\Sigma^1, \underline{H / H_{\chi}})^G \rightarrow \Pic(\Sigma^1)[p]^G.
	\end{equation}
	Suppose now for a contradiction that the line bundle $e_{\chi'} \cdot f'_* \OO_{\Sigma^2 / H_{\chi}}$ is trivial. Then from the exact sequence (\ref{Ginvkummer}) above, the space $\Sigma^2 / H_{\chi}$ is given as $\kappa(v) = \Sigma^1(v^{1/p})$ for some
	\[
		v \in \left( \OO(\Sigma^1)^\times / \OO(\Sigma^1)^{\times p} \right)^G.
	\]
	By Proposition \ref{globalinvunitssigma}, we actually have $v \in K^\times / K^{\times p}$, and therefore $\Sigma^2 / H_{\chi}$ is not geometrically connected. However, over $\bC_p$ the intermediate extension 
	\[
	\Sigma^2_{\bC_p} \rightarrow (\Sigma^2 / H_{\chi})_{\bC_p}
	\]
	is Galois and hence surjective, and thus $\Sigma^2_{\bC_p}$ is not connected, contradicting Corollary \ref{geomconn}.
\end{proof}

\begin{remark}
If we do not assume that $K$ contains a primitive $p$th root of $1$, then the techniques used in the proof of Theorem \ref{mainthm} still show that $\Pic(\Sigma^1)[p]^G \neq 0$.
\end{remark}

\section{Vector Bundles on the Drinfeld Upper Half Plane}

In this section we provide an elementary proof that all vector bundles on $\Omega^1$ are trivial, which extends and uses the result that all line bundles on $\Omega^1$ are trivial \cite[Thm. A]{JUNCOH}. In the context of Theorem \ref{mainthm}, this says that whilst the line bundles $\sL_{\chi}$ on $\Sigma^1$ are non-trivial whenever $\chi \neq 1$, the pushforward to $\Omega^1$ will be a trivial vector bundle (of constant rank $q+1$). 

Before we state the theorem, we will need the following notions from commutative algebra.

\begin{defn}
Let $R$ be an integral domain. $R$ is called a \emph{Pr\"{u}fer domain} if every finitely generated ideal of $R$ is invertible. $R$ is called a \emph{B\'{e}zout domain} if every finitely generated ideal of $R$ is principal. 
\end{defn}

We provide a proof of the following result, for which we were unable to find a reference.

\begin{lemma}\label{bezoutlemma}
Suppose that $R$ is a B\'{e}zout domain. Then every finitely generated submodule of a finitely generated free module is free.
\end{lemma}

\begin{proof}
	Suppose that $M$ is finitely generated over $R$ and $M \subset R^n$ for some $n \geq 1$. Let $\pi \colon R^n \rightarrow R$ be the projection to the first factor, and let $I \coloneqq \pi(M)$, $K \coloneqq \ker(\pi \colon M \rightarrow R)$. Now $I$ is the homomorphic image of $M$ and thus finitely generated, hence $I$ is principal and thus free, because $R$ is a B\'{e}zout domain. Therefore, the short exact sequence
	\[
		0 \rightarrow K \rightarrow M \rightarrow I \rightarrow 0
	\]
	splits, and $M \cong K \oplus I$. Finally, $K$ is also finitely generated, being a homomorphic image of $M$, and $K \subset R^{n-1}$, so the result follows by induction.
\end{proof}

We remark that this property actually characterises B\'{e}zout domains among integral domains. Indeed, if $I$ is a finitely generated ideal of an integral domain $R$ which satisfies the above property then $I$ is free, but also $I \subset R$, so by passing to the fraction field of $R$, $I$ must have rank $1$, and thus $I$ is principal. This property is analogous to the following property of PID's (which are exactly the Noetherian B\'{e}zout domains): a commutative ring $R$ is a PID if and only if every submodule of a free module is free.

\begin{thm}
Let $\fX$ be a smooth connected one-dimensional quasi-Stein rigid analytic space, with $\Pic(\fX) = 0$. Then any vector bundle on $\fX$ is of the form $\OO_{\fX}^n$ for some $n \geq 0$.
\end{thm}

\begin{proof}
	If $\fX$ is as above, the ring $R \coloneqq \OO_{\fX}(\fX)$ is an integral domain. The global sections functor defines an equivalence of categories between vector bundles on $\fX$, and finitely generated projective modules over $R$ \cite[Prop. 1.1.13]{BSX}. In particular, $\Pic(R) = 0$, and we are reduced to showing that any finitely generated projective module over $R$ is free. The ring $R$ is a Pr\"{u}fer domain \cite[1.1.8]{BSX}, and because $\Pic(R) = 0$, $R$ is furthermore a B\'{e}zout domain. Then we can conclude, as for such rings any finitely generated projective module is free, by Lemma \ref{bezoutlemma}.
\end{proof}

\begin{cor}\label{cor2}
Any vector bundle on $\Omega^1$ is of the form $\OO_{\Omega^1}^n$, for some $n \geq 0$. 
\end{cor}

\bibliography{biblio}{}
\bibliographystyle{plain}
\end{document}