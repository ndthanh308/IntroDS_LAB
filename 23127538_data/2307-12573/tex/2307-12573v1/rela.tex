\section{Related Work}
%Tabletop Role-Playing Games (TRPGs) are a form of interactive, narrative-driven game in which players assume the roles of characters in a fictional setting. This genre of games includes popular titles such as "Dungeons \& Dragons," "Pathfinder," and "Call of Cthulhu."
%In a typical TRPG, one player takes on the role of the game master (GM) or dungeon master (DM), who is responsible for creating the game world, narrating the story, and controlling non-player characters and events. The other players each control a single character and interact with the game world and the narrative presented by the GM.
%The outcomes of character actions in TRPGs are usually determined by a system of rules and dictated by dice rolls, although different games may utilize different rule sets and types of dice. The goal of these games is not necessarily to "win" in the traditional sense, but rather to participate in a collaborative storytelling experience and develop the characters and the story in interesting ways.
%TRPGs can be incredibly diverse, taking place in a variety of settings, from medieval fantasy worlds, to futuristic science-fiction universes, to modern-day detective mysteries, and anything in between. The flexibility and creativity inherent in TRPGs have led to their enduring popularity.

%\textbf{trpg game as an nlp challenge }

Tabletop Role-Playing Games (TRPGs) are immersive games where players assume different character roles in fictional settings, guided by a Game Master (GM) who provides relevant information to progress the game. These games involve diverse and complex grounded natural language interactions among multiple characters with distinct personalities and backgrounds. Due to the diversity and complexity, TRPGs serve as valuable testbeds~\cite{weir2022ontologically,louis2018deep,callison-burch-etal-2022-dungeons} for research in Natural Language Processing (NLP). Several works have explored NLP problems using TRPG game records. For instance, Louis et al.~\cite{louis2018deep} proposed predicting character actions based on previous interactions. Other works~\cite{si-etal-2021-telling,newman-liu-2022-generating} focused on generating flexible dialogue or descriptions in accordance with varying contexts or specific rules in TRPGs.

Furthermore, recent studies have commonly utilized play-by-post data from popular DND forums, providing a substantial corpus for research. This play-by-post format allows players to interact by posting replies, reducing participation barriers and generating a significant number of game rounds on the forum. Chris et al.~\cite{callison-burch-etal-2022-dungeons} have collected extensive corpus from these forums, resulting in the creation of TRPG dialogue datasets. Subsequently, Pei et al.~\cite{gandalf} filtered the dataset and developed a guidance generation task called GANDALF. Given the context from a single round, GANDALF predicts the guidance provided by the DM under the DND rule. Zhu et al.~\cite{zhu2023fireball} further extended the approach by constructing a more comprehensive and larger dataset using the play-by-post format in Discord, a messaging program. This dataset, named FIREBALL, contains additional game details such as dialogues, states, combat procedures, etc. It serves as a versatile testbed for language generation, particularly focusing on generating commands for games, including combat actions, checks, and dice rolls.

%Tabletop Role-Playing Game (TRPG) is a kind of role-playing game in which players need to act different roles of characters in a fictional setting and a Game Master (GM) describe related information for the players to guide the progress of the game. During the game, there will be diverse and complex grounded natural language interactions including multiple characters with different personalities and backgrounds. The diversity and complexity lead TRPG game becomes a valuable testbed for NLP research. Many works are proposed to investigate NLP problem by the records of TRPG games. Louis et al.~\cite{louis2018deep} propose to predict the actions of characters based on the former interactions. Some works~\cite{si-etal-2021-telling,newman-liu-2022-generating} tried to flexibly generate dialogue or descriptions according the different contexts or particular rules in TRPG. 
%Moreover, some recent works usually use play-by-post data from a popular DND forum, which provide larger size of corpus for research. In this playing form, the players interact with each other through posting replies to play the game. This approach reduces the difficulty of participating in the game and generates a significant number of game rounds on the forum. Chris et al.~\cite{callison-burch-etal-2022-dungeons} collect large size of corpus from the forum and produce a TRPG dialogue dataset. Then, Pei et al.~\cite{gandalf} filter the dataset and construct a guidance generation task named GANDALF. By give the context in one round, GANDALF requires to predict the guidance of a DM under the DND rule according. Zhu et al.~\cite{zhu2023fireball} further propose a more comprehensive and larger datasets by play-by-post manner in Discord (a message program) named FIREBALL. FIREBALL contains more details in the games, e.g., dialogues, states, combat procedure, etc. It aims to provide a general testbed for language generation, particularly for the commands generation in games (e.g., combat, actions, checks, dice rolls.). 

In this paper, we address the limitations of previous works in exploring more complex interactions. We introduce Multiple character and novel Object based interaction Estimation (MOE) task and Multiple character and a supporting dataset as valuable resources for interaction understanding for agents. Unlike previous approaches that rely on play-by-post formats, our dataset leverages game logs obtained from real-time interactions, providing a more grounded and complex semantics. MOE requires methods to answer questions about next acting characters and their corresponding actions. This task and dataset open up new possibilities for improving the agents with enhanced factual correctness, naturalness, and groundedness.

%However, previous works lack of the further exploration to the more complex and grounded semantics in the real-time communication. In this paper, we introduce a new task particular for semantic understanding, named Character and Skill check Answering (CSA) and dataset for the task, named Long-context Grounded-language TRPG Logs dataset (LGL). Different from the previous works collected from play-by-post form, we use game logs based on real-time interaction, collected from a popular Chinese TRPG forum, in which the playing logs are typically compiled and updated by GM after the game finish. Since most games are operated based on online voice or face-to-face communication, the game logs contain more instant feedback of players and more grounded semantics. Comparatively, in the play-by-post form, the recent interactions between different players could be several weeks apart. This induce the recent reply of players usually focus on the latest responses. All this factors reveal that our dataset can provide more complex and grounded semantics. Benefiting from LGL, CSA requires methods to answer characters and corresponding skills that need to be checked in the next game log. This is the first task that focus on understanding of semantic in virtual GM generation. We wish the task and dataset can further inspire the exploration in better virtual GM with better factual correctness, naturalness and groundness. 

%Moreover, rather than generate commands like dice roll and combat, we only focus on the skill check in playing record. During games, GM has to judge whether the players can operate their actions and what kinds of check should be operated for corresponding players. This requires GM to communicate with players, estimate players' intentions, and decide according to the current game situation and game rules, which reflect the high ability of the understanding of grounded natural language by real human. Thus, similar to the understanding and decision-making logic of GMs (Game Masters), we have collated long-text contexts of interactions among multiple players, and used the skill check results of real-life GMs as annotation information to construct our new task. In SCG, we requires the proposed methods to predict which characters will act and which skill check should be operated for the corresponding actions. This needs the methods to fully-understand the complex interactions, precisely estimate the intentions of characters and comprehend the game rules. SCG provides a more difficult understanding problem for grounded natural language interaction and we expect it can push the frontier of the research of NLP. 


%FIREBALL present the overall game process, including combat, actions, etc. Ours only on skill check and aims at understanding the context. 
%GANDALF just use only one round of interaction without multiple characters or complex interactions. 
\iffalse

\textbf{Chain-of-Thought (CoT)}
To better utilize the ability of large language models (LLMs), the Chain of thought (CoT) prompting methods have raise more attention recently. CoT methods aims to introduce a series of imtermediate reasoning steps to the original prompting methods, which can boost the understanding performances of LLMs without too much overheads. Kojima et al.~\cite{zero-shot-cot} show an efficient zero-shot-cot for LLMs in which the performances of LLMs boost merely by adding ``Let's think step by step" in prompts. Then, Wang et al.~\cite{cot_self_consistency} further introduce majority voting for the CoT methods and significant reduce the reasoning errors. LtM is another CoT based prompting method, which propose to solve the problem from the least to most. It seperates the final problem to several sub-problem and gradually solve a series of sub-problems. 

In our work, we propose a prompting method based on CoT named Character and motivation seperated prompting method (CMSP). CMSP is designed for SCG and intend to empower the understanding ability of LLMs for solving complex interaction in our task. The proposed method show better performances than the original prompting method and can formulate a strong benchmark for SCG problem. 
 
\textbf{Multi-party dialogue}

Multi-party dialogue focus on the understanding of dialogue. However, our dataset not only contains the complex dialogue, but also complex description from the host. Meanwhile, some playing record are not organized like dialogue. They are writted similar to a report or novel story, which present appearently different problem to dialogue understanding. 

Furthermore, the previous datasets only focus on one kind of rule in TRPG named DND. In ours, we collect more diverse rules in TRPG including DND, COC, PF, SW, etc. The dataset also involved more background story, part settings of the playing models (similar to game script), and some details of the character settings. 

\fi  