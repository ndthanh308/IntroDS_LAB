\documentclass{article}
% if you need to pass options to natbib, use, e.g.:
\PassOptionsToPackage{numbers, compress}{natbib}
% before loading neurips_data_2023

% ready for submission

% to compile a preprint version, add the [preprint] option, e.g.:
\usepackage[preprint]{neurips_data_2023}
% This will indicate that the work is currently under review.

% to compile a camera-ready version, add the [final] option, e.g.:
%     \usepackage[final]{neurips_data_2023}

% to avoid loading the natbib package, add option nonatbib:
%\usepackage[nonatbib]{neurips_data_2023}

% Submissions to the datasets and benchmarks are typically non anonymous,
% but anonymous submissions are allowed. If you feel that you must submit 
% anonymously, you can compile an anonymous version by adding the [anonymous] 
% option, e.g.:
%\usepackage[anonymous]{neurips_data_2023}
% This will hide all author names.

\usepackage[utf8]{inputenc} % allow utf-8 input
%\usepackage[T1]{fontenc}    % use 8-bit T1 fonts
\usepackage{hyperref}       % hyperlinks
\usepackage{url}            % simple URL typesetting
\usepackage{booktabs}       % professional-quality tables
\usepackage{amsfonts}       % blackboard math symbols
\usepackage{nicefrac}       % compact symbols for 1/2, etc.
\usepackage{microtype}      % microtypography
\usepackage{xcolor}         % colors
%\usepackage{ctex}

\usepackage{times}
\usepackage{graphicx}
% for Chinese
\usepackage{CJKutf8}
% for table
\usepackage{tabularx}
\usepackage{float}
\usepackage{subfigure}
\usepackage{caption}
\usepackage{longtable}
\usepackage{supertabular}
\usepackage{arydshln} 
\usepackage{rotating}
\usepackage{enumitem}
\usepackage{amsmath}
%\usepackage{natbib}



\usepackage{epsfig}
\usepackage{amssymb}
\usepackage{algorithm}
\usepackage{algpseudocode}
\usepackage{threeparttable}
\usepackage{booktabs}
\usepackage{mathrsfs}
\usepackage{bbding}

%multi-row
\usepackage{multirow}


\setenumerate[1]{itemsep=0pt,partopsep=0pt,parsep=\parskip,topsep=5pt}
\setitemize[1]{itemsep=0pt,partopsep=0pt,parsep=\parskip,topsep=5pt}
\setdescription{itemsep=0pt,partopsep=0pt,parsep=\parskip,topsep=5pt}


%\title{Boosting Agent Interactions by Introducing World Model: }
%\title{Towards Open WorldAgent Interaction: A Benchmark for LLM to Understand Multi-character and Novel Object Interaction}
%\title{Understanding Agent Interactions by Introducing World Model:  A Benchmark for Multi-character and Novel Object Interaction}
%\title{MOE: Understand Multi-character and Novel Object Interactions for LLMs}
%\title{xxx LLM: A xxx Benchmark for Understanding Multi-character and Novel Object Interactions}

%\title{Towards Multi-Character and Novel Object Interactions: A Benchmark for Virtual Game Master Integration for Agent World Models}
%\title{Towards Multi-Character and Novel Object Interactions: A Benchmark for Virtual Game Master}
\title{Tachikuma: Understading Complex Interactions with Multi-Character and Novel Objects by Large Language Models}

%% Figure removed
%Enhancing Agent Understanding of Complex Interactions: A Benchmark for Intention Estimation

% The \author macro works with any number of authors. There are two commands
% used to separate the names and addresses of multiple authors: \And and \AND.
%
% Using \And between authors leaves it to LaTeX to determine where to break the
% lines. Using \AND forces a line break at that point. So, if LaTeX puts 3 of 4
% authors names on the first line, and the last on the second line, try using
% \AND instead of \And before the third author name.


 
\author{\small Yuanzhi Liang$~\textsuperscript{\rm 1}$, Linchao Zhu$~\textsuperscript{\rm 2}$, Yi Yang$~\textsuperscript{\rm 2}$    \\ 
        \small{$~\textsuperscript{\rm 1}$ University of Technology Sydney}, 
	\small{$~\textsuperscript{\rm 2}$ Zhejiang University}
	\\{\tt\small {yuanzhi.Liang}@student.uts.edu.au} \vspace{-0.15cm}
        \\{\tt\small {zhulinchao7}@gmail.com} \vspace{-0.15cm}
        \\{\tt\small {yangyics}@zju.edu.cn}	
	%	\thanks{Corresponding Authors.}
}
\date{}

\begin{document}

\maketitle

\begin{abstract}
Recent advancements in natural language and Large Language Models (LLMs) have enabled AI agents to simulate human-like interactions within virtual worlds. However, these interactions still face limitations in complexity and flexibility, particularly in scenarios involving multiple characters and novel objects. Pre-defining all interactable objects in the agent's world model presents challenges, and conveying implicit intentions to multiple characters through complex interactions remains difficult. To address these issues, we propose integrating virtual Game Masters (GMs) into the agent's world model, drawing inspiration from Tabletop Role-Playing Games (TRPGs). GMs play a crucial role in overseeing information, estimating players' intentions, providing environment descriptions, and offering feedback, compensating for current world model deficiencies. To facilitate future explorations for complex interactions, we introduce a benchmark named Tachikuma, comprising a Multiple character and novel Object based interaction Estimation (MOE) task and a supporting dataset. MOE challenges models to understand characters' intentions and accurately determine their actions within intricate contexts involving multi-character and novel object interactions. Besides, the dataset captures log data from real-time communications during gameplay, providing diverse, grounded, and complex interactions for further explorations. Finally, we present a simple prompting baseline and evaluate its performance, demonstrating its effectiveness in enhancing interaction understanding. We hope that our dataset and task will inspire further research in complex interactions with natural language, fostering the development of more advanced AI agents.
%Recent advancements in natural language and Large Language Models (LLMs) have enabled AI agents to engage in human-like interactions within virtual world. However, these interactions still suffer from limitations in complexity and flexibility, especially in scenarios involving multiple characters and novel objects. It is challenging to pre-define all interactable objects in the agent world model. Additionally, conveying implicit intentions to multiple characters through complex interactions remains a difficulty. To address these issues, we propose integrating virtual Game Masters (GMs) into the agent's world model to handle complex interactions, drawing inspiration from Tabletop Role-Playing Games (TRPGs). GMs play a crucial role in overseeing information, estimating players' intentions, providing environment descriptions, and offering feedback, compensating for current world model deficiencies. To facilitate this exploration, we introduce a benchmark comprising the Multiple character and novel Object based interaction Estimation (MOE) task and a supporting dataset. MOE challenges models to understand characters' intentions and accurately determine their actions within intricate contexts involving multi-character and novel object interactions. The dataset captures log data from real-time communications during gameplay, providing diverse, grounded and complex interactions for further explorations. Finally, we present a simple prompting baseline and evaluate its performance, demonstrating its effectiveness in enhancing interaction understanding. We hope that our dataset and task will inspire further research in complex interactions with natural language, fostering the development of more advanced AI agents.
\end{abstract}

%Recent research has made strides in developing AI agents capable of human-like interactions using natural language and Large Language Models (LLMs). Agents can live in a virtual world, achieve feedback from the world model, and produce interactions with the environment and other agents. However, these interactions still suffer from limitations in complexity and flexibility, especially when it comes to more intricate interactions close to the real-human that may involve multiple characters and novel objects. Specifically, it is hard to pre-define all possible and interactable objects in the environment. Meanwhile, it is also hard to convey implicit intentions (e.g., persuasion, negotiation) to multiple characters by complex interactions similar to the real-time communications. To address this gap in designation of world models for agents, inspired by rules and settings in TRPG, we proposed to construct virtual GMs into the world model to handle complex interactions. As in the games, GM can oversee all information, estimate intentions of players, provide descriptions for environments, and provide feedback for agents after they operate actions. Integrating a virtual GM into agent world model may reduce limitation in complex interactions of agents. However, due to the absence of a proper benchmark with long and intricate contexts within multi-character and novel object interactions, previous methods are hard to construct a virtual GM that understands intentions and provide correct feedback according the complex interactions. In this paper, we propose a new benchmark consisting of a Multiple character and novel Object based interaction Estimation (MOE) task and a dataset to support the task. MOE challenges models to understand characters' intentions and accurately determine the next character's actions within long and intricate contexts within multi-character and novel object interactions. To support MOE task, we collect a dataset, which contains intricate contexts extracted from game logs obtained during real-time communications. This dataset serves as a valuable resource for MOE task and enables in-depth exploration and simulation of human behavior and interactions. Additionally, we present a simple prompting baseline and evaluate its performance alongside other prompting methods employing various LLMs. Through objective and subjective evaluations, we demonstrate the effectiveness of our baseline and underscore the significance of our dataset and task in advancing interaction understanding.
%We hope that our dataset and task will inspire the research community to deepen exploration of complex interactions with natural language and foster the development of enhanced AI agents.

%we propose a new benchmark consisting of a Multiple character and novel Object based interaction Estimation (MOE) task and a dataset for the task. MOE task challenges models to accurately determine the next character's actions and identify corresponding actions within long and intricate contexts within multi-character and novel object interactions. To support MOE task, we collect a dataset, which contains intricate contexts extracted from game logs obtained during real-time communications. This dataset serves as a valuable resource for MOE task and enables in-depth exploration and simulation of human behavior and interactions. Additionally, we present a simple prompting baseline and evaluate its performance alongside other prompting methods employing various LLMs. Through objective and subjective evaluations, we demonstrate the effectiveness of our baseline and underscore the significance of our dataset and task in advancing interaction understanding.
%We hope that our dataset and task will inspire the research community to deepen exploration of complex interactions with natural language and foster the development of enhanced AI agents.


%In the pursuit of developing AI agents that can engage in human-like interactions, recent research has explored the use of natural language and Large Language Models (LLMs). However, these interactions still face limitations in terms of complexity and flexibility, particularly for more complex interactions in the open world close to the reality, which considers multiple characters and novel objects interactions. However, before constructing better agents that enable to operate complex interactions, there is no proper benchmark for understanding the complex interactions close to the real-human. To encourage further developments, we propose a new benchmark, which comprises a task called Multiple character and novel Object based interaction Estimation (MOE) and a dataset named Multiple character based novel Object dataset (MOD). MOE task challenges understanding ability for complex interactions close to the real-human. It requires the model to accurately determine the next character to act and identify corresponding actions within long and intricate contexts of multi-character and novel object interactions. Moreover, we collect MOD dataset to support MOE task, which provides long and intricate contexts extracted from game logs which are captured from real-time communications. This dataset is a valuable resource for MOE task, facilitating deeper explorations into understanding and simulating human behavior and interactions.  Finally, We also present a simple prompting baseline, and evaluate its performance alongside other prompting methods utilizing different LLMs. Through objective and subjective evaluations, we demonstrate the effectiveness of our baseline and highlight the significance of our dataset and task in improving interaction understanding.

%Toward more capable agents close to the real human, some works let agent to interact by natural language, benefitting from the significant advancements of Large Language Models (LLMs). However, the agent interactions are still possess limited complexity and flexibly. The interactions are with limited length, limited instructions or limited interactable objects and characters in the virtual world. To overcome the limitation and enable AI agent to achieve more powerful language ability to interact, 
%we introduce a new benchmark for understanding interactions, named Complex Contexts based intention Answering (C2A), and a new dataset to support further exploratino in agent interactions, named  Long-context Multi-character Interaction (LMI) dataset. Specifically, in C2A, given lengthy and intricate contexts, the methods must accurately determine the character who will act next and identify the corresponding actions. This task provides explicit requirements for interaction understanding in agent generation, establishing a solid foundation for the development of factually accurate close to real human.
%Furthermore, we collect LMI dataset, which support C2A task. This dataset contains extended and intricate contexts extracted from game logs featuring real-time communication, offering a valuable resource for C2A and enabling further explorations in this domain. We also present a simple prompting baseline based on the dataset and task, and evaluate its performance alongside other prompting methods utilizing different LLMs. Through objective and subjective evaluations, we demonstrate the effectiveness of our prompting baseline. The results also show the significance of our dataset and task, particularly for improving interaction understanding and vivid response generation of agents.
%We hope that our dataset and task will inspire the research community to deepen their understanding of complex interactions with natural language and foster the development of enhanced AI agents.

%Designing an AI agent to fulfill the role of a game master (GM) in Tabletop Role-Playing Games (TRPGs) presents an intriguing and formidable challenge. TRPGs involve players engaging in immersive role-play, interacting with diverse characters, and receiving guidance from the GM to immerse themselves in fantastical worlds. While previous studies have employed Large Language Models (LLMs) to generate responses and guidance for adjacent game turns or simple player descriptions, the lack of data and benchmarks for complex interactions with grounded language limits their effectiveness in handling the intricate and contextually grounded semantics found in real-time communications that closely resemble human interaction.
%In this paper, we introduce a Character and Skill check Answering (CSA) task, which addresses the need to comprehend the complex and grounded semantics presented in game logs. In CSA, given lengthy and intricate contexts, the methods must accurately determine the character who will act next and identify the corresponding skill required by the game rules. This task provides explicit requirements for semantic understanding in agent generation, establishing a solid foundation for the development of factually accurate and immersive virtual GMs.
%Furthermore, we collect the Long-context Grounded-language TRPG Logs dataset (LGL) specifically for CSA task. This dataset contains extended and intricate contexts extracted from game logs featuring real-time communication, offering a valuable resource for CSA and enabling further explorations in this domain. We also present a simple prompting baseline based on the dataset and task, and evaluate its performance alongside other prompting methods utilizing different LLMs. Through objective and subjective evaluations, we demonstrate the effectiveness of our prompting baseline. The results also show the significance of our dataset and task, particularly for semantic understanding of game agent generation. We hope that our dataset and task will inspire the research community to deepen their understanding of complex grounded semantics and foster the development of enhanced AI agents.


%Designing the AI agent for game master (GM) in Tabletop Role-Playing Games (TRPGs) is an interesting but challenging task, due to the unique game mechanism. Within TRPGs, players engage in role-play, interacting with other characters and guided by the GM to immerse themselves in fantasy worlds. The interactions in the games contain complex and grounded semantics. Utilizing Large Language Models (LLMs), previous works generate better responses or guidance for contexts in adjacent game turns or simple descriptions of players. Meanwhile, the explorations are insufficient to handle complex and grounded semantics within real-time communications that close to the real-human. In this paper, we formalize a task for particularly understanding the complex game logs, named Character and Skill check Answering (CSA). Given a long and complex contexts with grounded language, similar to a real-human GM, the methods must correctly answer who will act next and what skill in game rules are required. Rather than directly generating responses, this task formulate an explicit testbed for the understanding of the contexts, which performs a valuable foundation for a factually accurate and vivid visual GM. Moreover, we construct a new dataset specifically for the task, named Long-context Grounded-language TRPG Logs dataset (LGL). This dataset provides extended and intricate contexts, excerped from game logs with real-time communication, which performs a valuable data source for further explorations. Moreover, based on the dataset and task, we further present a simple prompting baseline and evaluate our baseline and other prompting methods with different LLMs in CSA. Experimental results in both objective and subjective evaluations show the effectiveness of our prompting baseline, which also reveal the significance of our dataset and task. We wish our dataset and task can provide further inspiration for the community in understanding complex grounded semantics and generating better AI agents. 

%However, suffering from the forum-based data collections for game logs, adjacent game turns may contain responses separated by a long time and the reply are not as grounded as the real-time communications. This also induce previous works tend to use latest turn or simple descriptions as inputs and generate GM's utterance or intentions without explicit understanding of contexts, which are insufficient to explore complex and more grounded interactions close to the real-human. To encourage further developments, we construct a new dataset specifically for complex and grounded semantics in TRPG game logs, named Long-context Grounded-language TRPG Logs dataset (LGL). This dataset provides extended and intricate contexts, excerped from game logs with real-time communication. Taking advantages of LGL, we formalize a task for particularly understanding the complex game logs, named Character and Skill check Answering (CSA). Given a long and complex contexts with grounded language, similar to a real-human GM, the methods must correctly answer who will act next and what skill in game rules are required. Rather than directly generating responses, this task formulate an explicit testbed for the understanding of the contexts, which performs a valuable foundation for a factually accurate and vivid visual GM. Moreover, based on the dataset and task, we further present a simple prompting baseline to evaluate the performances of a virtual GM that explicitly considered the understanding of complex contexts in games. In experiment, we compare various prompting methods and LLMs. Results in both objective and subjective evaluations show the effectiveness of our prompting baseline, which also reveal the significance of our dataset and task. We wish our dataset and task can provide further inspiration for the community in understanding complex grounded semantics and generating better AI agents. 

%To construct an AI agent as virtual GM to guide players, previous works usually utilizing Large Language Models (LLMs), take latest game turns or simple descriptions for player actions to generate GM's utterance in game logs. In this setting, though proposed methods are effective, the performances of virtual GM are still have large space to improve to close to the real-human responses. 
%Many works are proposed and aim to generate better virtual GM to support players' playing process. However, current works limited on the forum-based data collections and end-to-end utterance generation, which are insufficient to complex and grounded interactions close to the real-time communication in daily life. To overcome the defeats, we introduce a new dataset specifically for complex and grounded semantics in TRPG game logs, named Long-context Grounded-language TRPG Logs dataset (LGL). This dataset provides extended and intricate contexts, excerped from game logs with real-time communication. Moreover, to push frontier of the generation of game agent in TRPG, we further propose a new benchmark, named think before speak (TBS) that provides two-step generation and jointly considers both understanding and generation. In this benchmark, we propose generation of game check and generation of GM utterance, which target to accurately understanding complex interactions and generate vivid utterance, respectively. During evaluation, we can separately evaluate two parts by objective evaluations like precision and recall and subjective evaluations with real-human volunteers. All experimental results indicate our benchmark offer the high-quality responses as a virtual GM for TRPG games. We also wish our dataset and method can provide further inspiration for the community in understanding complex grounded semantics and generating better AI agents. 

%The intricate and diverse grounded language interactions within TRPGs provide a valuable testbed for understanding and stimulating human reactions, e.g., command generation, state narrative. However, current works focus on abbreviated game contents and relies on forum-based communication, thereby limiting the complexity of contexts and the availability of immediate responses within grounded language.
%RTo encourage the development of understanding complex interactions and generating grounded language reactions, we propose the Long-context Grounded-language TRPG Logs dataset (LGL). This dataset provides extended and intricate contexts, excerped from game logs with real-time communication. Taking advantages of LGL, we further introduce a benchmark called skill check generation (SCG), designed to understand contexts and generate guidance akin to real-human GMs. We apply large language models (LLMs) and evaluate various prompting methods in this benchmark. While all methods demonstrate efficacy in understanding complex semantics and generating guidance, there remains a significant gap compared to real-human GMs. In all, by utilizing complex TRPG logs as a testbed for grounded language interactions, this work aims to stimulate further investigations towards a deeper understanding of grounded semantics and generation, facilitating the creation of vivid reactions from virtual humans.

%Though all methods are workable to understand complex semantics and produce guidance, there are still large gaps compared with the real-human DMs. By taking more complex TRPG logs as the testbed for grounded language interactions, this work aim to foster further investigations towards a deeper understanding for grounded semantics and generation for vivid reactions of virtual human.


%TRPG is a popular game form for entertainment. There are usually multiple players and a game master (GM) embark on adventures in a richly detailed fantasy world. The players operate role-play to act, interact with other characters and guided by GM to push on the game process. Due to the diverse and complex interaction within grounded language in the game, the game logs, recording communication details of players, can be a valuable data source for the exploration of stimulating and estimating real human reactions (e.g., command generation, state generation, etc.). However, current works focus on short contents with limited game turns and forum-based communication. These induce the contexts are not complex enough and lack of immediate response within more grounded language. To encourage further development in understanding complex interaction and grounded human communications, we propose Long-context Grounded-language TRPG Logs dataset (LGL), which offers long and complex contexts within more grounded language collected from game logs with real-time communication. Moreover, we construct a benchmark, skill check generation (SCG), which aims to produce the understanding and guidance of real-human GM. By introduce more complex context with grounded language and specifically designed benchmark, we wish to prompt further explorations for enabling the deeper understanding for vivid and grounded interactions.

%Understanding of grounded natural language interaction is a practical and challenging problem. It may involves multiple characters with different personalities, various scenarios, and diverse background information and contexts. Though it is easy for human being to imagine the interaction and behaviours of characters, models, e.g., large language models (LLMs) always struggling in understand the complex semantics, produce incorrect answers, or even generate gibberish. To push the frontier of this problem, we propose a novel task, SCG (Skill Check Generation in Tabletop Role-Playing Game). Specifically, we leverage the playing records of Tabletop Role-Playing Game, which is a role-playing game with multiple players and a host. All players will conduct various role-playing and complete an adventure together. The host will describe background information, behaviours and reactions of non-player characters, and guides the corresponding characters to perform skill check according their intention and interactions. All the information are contained in the playing records. In SCG, we focus on the skill check in playing records, which is provided by the host after multiple rounds of interactions and naturally labels who to act and what the action should be operated. This formulates a testbed for the understanding of grounded natural language. To generate the skill check, the language models are required to fully understand the complex interactions of various characters, abduct the intention of characters according the contexts, and also comprehend the background story in current scenarios. Moreover, we further propose a chain-of-thought (CoT) prompting method named character and motivation separated prompting (CMSP) to introduce LLMs to separately who and how to act in current round. The experimental results show the effectiveness of our CMSP in solving SCG. All dataset and benchmark will be available soon. 


% Figure environment removed

\section{Introduction}
Automatic 3D reconstruction of clothed humans using image inputs has gained increasing significance due to its potential applications in a wide array of AR/VR scenarios. High-fidelity reconstructions typically depend on sophisticated capture systems, which are developed with dense camera arrays~\cite{collet2015high,joo2015panoptic,joo2018total}, programmable light-stages~\cite{Vlasic2009, guo2019relightables}, and depth sensors~\cite{newcombe2011kinectfusion,DoubleFusion,BodyFusion,dou2016fusion4d,newcombe2015dynamicfusion}. However, stringent capture environments equipped with complex hardware pose significant challenges for consumer-level applications.


In this context, considerable research effort has been dedicated to developing methods that allow for more flexible capture configurations, such as utilizing a few RGB inputs. Among these works, learning implicit functions \cite{iccv2020PIFu, saito2020pifuhd, hong2021stereopifu} has proven effective in achieving highly detailed reconstructions by integrating the advancements of deep neural networks. These methods employ large multi-layer perceptrons (MLPs) to predict the occupancy probability or truncated signed distance function (TSDF) value of every queried 3D point based on its associated local feature, which is extracted from images. They can recover a continuous surface at arbitrary resolutions without topology restrictions.


However, in typical MLP-based implicit networks, the occupancy or TSDF value at each location is solved independently with planar image features, rendering them less capable of addressing challenging cases such as occlusions. Consequently, these methods suffer from generalization and robustness issues, particularly when tackling strong occlusions caused by large motion or multiple interacting humans. 
Some follow-up studies  \cite{zheng2021deepmulticap,zheng2021pamir,huang2020arch} utilize an extra geometric model, SMPL~\cite{Loper2015}, to improve robustness by introducing strong shape priors. 
Their success typically relies on the assumption of geometrical similarity \cite{huang2020arch} between the shape prior and target reconstruction, making them intractable for handling complex cases with loose clothes and sensitive to errors in SMPL model fitting.



%\ping{this paragraph sounds like `TSDF is better than MLP/SMPL, and we use TSDF to solve the problem'. But in Sec 3, we are telling a different story, saying `MLP needs a 3D convolutional encoder'. We need to make these two sections consistent.}\sicong{I think in this paragraph we claim that the TSDF}


%We opt for Trucated Signed Distance Funtion (TSDF) volumetric representations as they are naturally suitable for convolution operations, which have shown remarkable performance for learning hierarchical features on 2D visual perception tasks \cite{SunXLW19}. 
%Meanwhile, TSDF also describes the gradual geometry change around shape surface, which is not reflected by occupancy volume. 

We instead revisit the 3D volumetric representation and resort to 3D convolutional neural networks (CNNs) for feature learning, due to their impressive performance in feature learning and the ability to incorporate spatial context. However, volumetric methods and 3D convolution involve discretization, which might raise concerns regarding whether a discretized volume can preserve subtle geometric details as continuous representations learned in implicit functions. We investigate the relationship between volume resolution and quantization error on synthetic data by converting target mesh objects to TSDF volumes, as shown in Figure~\ref{fig:quantization_error}. We observe that the quantization errors are significantly reduced by increasing volume resolution and become nearly negligible when reaching a relatively high resolution (e.g., 512 or higher). In other words, achieving fine-detailed reconstruction is not supposed to be restricted by the use of volume representations as long as a proper volume resolution is utilized. Therefore, we present a method with high-resolution feature volumes, e.g., 256 and 512, while traditional volumetric methods \cite{varol18_bodynet,gilbert2018volumetric} are often limited to much lower resolutions, such as 32 or 128.



On the other hand, an increase in volume resolution may lead to a cubic growth of memory overhead \cite{8100085}. Reducing memory costs while guaranteeing the granularity of volumetric representations is necessary for pursuing high-quality reconstruction. Thus, we adopt a coarse-to-fine approach and cull away irrelevant voxels to build a sparse high-resolution feature volume. At the coarse level, the network computes an initial TSDF by applying a U-Net with sparse 3D CNN \cite{3DSemanticSegmentationWithSubmanifoldSparseConvNet} on the sparse feature volume, which is carved by a visual hull. Through our experiments, it turns out that more than 95\% of the volume grids are discarded by the visual hull culling, making the sparse 3D CNN efficient. At the fine level, the network focuses on a narrow band near the zero-level set of the initial TSDF and discretizes the narrow band with smaller voxels. By employing this narrow-band culling, we further shrink the sampling space, resulting in a relatively small range of grid numbers (usually 300K--500K in our experiments) even with a high volume resolution of 512. The remaining voxels in the narrow band are associated with features that fuse high-frequency information from the computed normal maps upon the low-frequency shape from the coarse level to compute the TSDF at high resolution. The final mesh is then extracted from the TSDF using the Marching-Cube algorithm ~\cite{Lorensen87marchingcubes}.
% Different from the u-net sturcture to preserve global topology context, we then apply a shallow 3dcnn to compute the final TSDF $D_{final}$ which contain more local geometry detail.




% \ping{this paragraph can be expanded. It is an important contribution and often ignored by other works. stress on the novel idea of regressing blending weights instead of colors}

In addition to geometry, high-quality mesh texture is also a crucial factor contributing to visual appearance. Directly computing a color field in 3D space, as in \cite{iccv2020PIFu}, struggles to capture high-frequency texture details, while the neural radiance field (NeRF) \cite{yu2020pixelnerf} or the DoubleField~\cite{shao2022doublefield} require expensive per-instance optimization and are often unstable for sparse input images. In contrast, we adopt an image-based rendering approach to compute a texture atlas map, which is efficient and widely supported in existing computer graphics tools. 
Specifically, we compute a blending weight at each 3D point on the mesh surface to determine its color as a weighted average of the colors at its image projections. The blending weights can be computed at a relatively coarse resolution, e.g., 512 volume resolution in our case, and leave texture details to the high-resolution images, such as 1K or 2K. Unlike previous methods that generate blurry texturing results under sparse input, our method generalizes well on both synthetic and real data with just a few input views. 
Figure~\ref{fig:teaser} shows two examples reconstructed by our method. Despite the challenging garment, pose, and occlusion, our method recovers faithful shape, normal, and texture on the right.

%with a wide variety of poses and clothing styles, and it is also adaptive to handle input image with arbitrary resolutions.
%\sicong{For this concern we claim that when the resolution of dicretized volume meets certain threshold (which is 256 in our experiment), the quantization error can be neglected.} 



In summary, the main contributions of this paper are as follows:
\begin{itemize}
\vspace{-0.1in}
  \item 
  We revisit the 3D volumetric representation and demonstrate that it can support clothed human reconstruction with equal or even better performance compared to implicit representation. 
  \item 
  We develop a memory and computation-efficient method for high-resolution volumetric reconstruction using sophisticated sparse 3D CNN, coarse-to-fine estimation, and voxel culling by visual hull and narrow bands. 
  \item 
  We introduce a novel method to compute a texture atlas map, which captures rich appearance details from high-resolution input images.
  \item 
  We achieve impressive results on standard benchmark datasets Twindom and MultiHuman, significantly reducing the point-2-surface (P2S) precision to approximately 0.2cm from just six input views, with more than $50\%$ error reduction compared to the state-of-the-art methods, including DoubleField~\cite{shao2022doublefield} and PIFuHD~\cite{saito2020pifuhd}.
\end{itemize}

\section{Related Work}
%Tabletop Role-Playing Games (TRPGs) are a form of interactive, narrative-driven game in which players assume the roles of characters in a fictional setting. This genre of games includes popular titles such as "Dungeons \& Dragons," "Pathfinder," and "Call of Cthulhu."
%In a typical TRPG, one player takes on the role of the game master (GM) or dungeon master (DM), who is responsible for creating the game world, narrating the story, and controlling non-player characters and events. The other players each control a single character and interact with the game world and the narrative presented by the GM.
%The outcomes of character actions in TRPGs are usually determined by a system of rules and dictated by dice rolls, although different games may utilize different rule sets and types of dice. The goal of these games is not necessarily to "win" in the traditional sense, but rather to participate in a collaborative storytelling experience and develop the characters and the story in interesting ways.
%TRPGs can be incredibly diverse, taking place in a variety of settings, from medieval fantasy worlds, to futuristic science-fiction universes, to modern-day detective mysteries, and anything in between. The flexibility and creativity inherent in TRPGs have led to their enduring popularity.

%\textbf{trpg game as an nlp challenge }

Tabletop Role-Playing Games (TRPGs) are immersive games where players assume different character roles in fictional settings, guided by a Game Master (GM) who provides relevant information to progress the game. These games involve diverse and complex grounded natural language interactions among multiple characters with distinct personalities and backgrounds. Due to the diversity and complexity, TRPGs serve as valuable testbeds~\cite{weir2022ontologically,louis2018deep,callison-burch-etal-2022-dungeons} for research in Natural Language Processing (NLP). Several works have explored NLP problems using TRPG game records. For instance, Louis et al.~\cite{louis2018deep} proposed predicting character actions based on previous interactions. Other works~\cite{si-etal-2021-telling,newman-liu-2022-generating} focused on generating flexible dialogue or descriptions in accordance with varying contexts or specific rules in TRPGs.

Furthermore, recent studies have commonly utilized play-by-post data from popular DND forums, providing a substantial corpus for research. This play-by-post format allows players to interact by posting replies, reducing participation barriers and generating a significant number of game rounds on the forum. Chris et al.~\cite{callison-burch-etal-2022-dungeons} have collected extensive corpus from these forums, resulting in the creation of TRPG dialogue datasets. Subsequently, Pei et al.~\cite{gandalf} filtered the dataset and developed a guidance generation task called GANDALF. Given the context from a single round, GANDALF predicts the guidance provided by the DM under the DND rule. Zhu et al.~\cite{zhu2023fireball} further extended the approach by constructing a more comprehensive and larger dataset using the play-by-post format in Discord, a messaging program. This dataset, named FIREBALL, contains additional game details such as dialogues, states, combat procedures, etc. It serves as a versatile testbed for language generation, particularly focusing on generating commands for games, including combat actions, checks, and dice rolls.

%Tabletop Role-Playing Game (TRPG) is a kind of role-playing game in which players need to act different roles of characters in a fictional setting and a Game Master (GM) describe related information for the players to guide the progress of the game. During the game, there will be diverse and complex grounded natural language interactions including multiple characters with different personalities and backgrounds. The diversity and complexity lead TRPG game becomes a valuable testbed for NLP research. Many works are proposed to investigate NLP problem by the records of TRPG games. Louis et al.~\cite{louis2018deep} propose to predict the actions of characters based on the former interactions. Some works~\cite{si-etal-2021-telling,newman-liu-2022-generating} tried to flexibly generate dialogue or descriptions according the different contexts or particular rules in TRPG. 
%Moreover, some recent works usually use play-by-post data from a popular DND forum, which provide larger size of corpus for research. In this playing form, the players interact with each other through posting replies to play the game. This approach reduces the difficulty of participating in the game and generates a significant number of game rounds on the forum. Chris et al.~\cite{callison-burch-etal-2022-dungeons} collect large size of corpus from the forum and produce a TRPG dialogue dataset. Then, Pei et al.~\cite{gandalf} filter the dataset and construct a guidance generation task named GANDALF. By give the context in one round, GANDALF requires to predict the guidance of a DM under the DND rule according. Zhu et al.~\cite{zhu2023fireball} further propose a more comprehensive and larger datasets by play-by-post manner in Discord (a message program) named FIREBALL. FIREBALL contains more details in the games, e.g., dialogues, states, combat procedure, etc. It aims to provide a general testbed for language generation, particularly for the commands generation in games (e.g., combat, actions, checks, dice rolls.). 

In this paper, we address the limitations of previous works in exploring more complex interactions. We introduce Multiple character and novel Object based interaction Estimation (MOE) task and Multiple character and a supporting dataset as valuable resources for interaction understanding for agents. Unlike previous approaches that rely on play-by-post formats, our dataset leverages game logs obtained from real-time interactions, providing a more grounded and complex semantics. MOE requires methods to answer questions about next acting characters and their corresponding actions. This task and dataset open up new possibilities for improving the agents with enhanced factual correctness, naturalness, and groundedness.

%However, previous works lack of the further exploration to the more complex and grounded semantics in the real-time communication. In this paper, we introduce a new task particular for semantic understanding, named Character and Skill check Answering (CSA) and dataset for the task, named Long-context Grounded-language TRPG Logs dataset (LGL). Different from the previous works collected from play-by-post form, we use game logs based on real-time interaction, collected from a popular Chinese TRPG forum, in which the playing logs are typically compiled and updated by GM after the game finish. Since most games are operated based on online voice or face-to-face communication, the game logs contain more instant feedback of players and more grounded semantics. Comparatively, in the play-by-post form, the recent interactions between different players could be several weeks apart. This induce the recent reply of players usually focus on the latest responses. All this factors reveal that our dataset can provide more complex and grounded semantics. Benefiting from LGL, CSA requires methods to answer characters and corresponding skills that need to be checked in the next game log. This is the first task that focus on understanding of semantic in virtual GM generation. We wish the task and dataset can further inspire the exploration in better virtual GM with better factual correctness, naturalness and groundness. 

%Moreover, rather than generate commands like dice roll and combat, we only focus on the skill check in playing record. During games, GM has to judge whether the players can operate their actions and what kinds of check should be operated for corresponding players. This requires GM to communicate with players, estimate players' intentions, and decide according to the current game situation and game rules, which reflect the high ability of the understanding of grounded natural language by real human. Thus, similar to the understanding and decision-making logic of GMs (Game Masters), we have collated long-text contexts of interactions among multiple players, and used the skill check results of real-life GMs as annotation information to construct our new task. In SCG, we requires the proposed methods to predict which characters will act and which skill check should be operated for the corresponding actions. This needs the methods to fully-understand the complex interactions, precisely estimate the intentions of characters and comprehend the game rules. SCG provides a more difficult understanding problem for grounded natural language interaction and we expect it can push the frontier of the research of NLP. 


%FIREBALL present the overall game process, including combat, actions, etc. Ours only on skill check and aims at understanding the context. 
%GANDALF just use only one round of interaction without multiple characters or complex interactions. 
\iffalse

\textbf{Chain-of-Thought (CoT)}
To better utilize the ability of large language models (LLMs), the Chain of thought (CoT) prompting methods have raise more attention recently. CoT methods aims to introduce a series of imtermediate reasoning steps to the original prompting methods, which can boost the understanding performances of LLMs without too much overheads. Kojima et al.~\cite{zero-shot-cot} show an efficient zero-shot-cot for LLMs in which the performances of LLMs boost merely by adding ``Let's think step by step" in prompts. Then, Wang et al.~\cite{cot_self_consistency} further introduce majority voting for the CoT methods and significant reduce the reasoning errors. LtM is another CoT based prompting method, which propose to solve the problem from the least to most. It seperates the final problem to several sub-problem and gradually solve a series of sub-problems. 

In our work, we propose a prompting method based on CoT named Character and motivation seperated prompting method (CMSP). CMSP is designed for SCG and intend to empower the understanding ability of LLMs for solving complex interaction in our task. The proposed method show better performances than the original prompting method and can formulate a strong benchmark for SCG problem. 
 
\textbf{Multi-party dialogue}

Multi-party dialogue focus on the understanding of dialogue. However, our dataset not only contains the complex dialogue, but also complex description from the host. Meanwhile, some playing record are not organized like dialogue. They are writted similar to a report or novel story, which present appearently different problem to dialogue understanding. 

Furthermore, the previous datasets only focus on one kind of rule in TRPG named DND. In ours, we collect more diverse rules in TRPG including DND, COC, PF, SW, etc. The dataset also involved more background story, part settings of the playing models (similar to game script), and some details of the character settings. 

\fi  

\section{Dataset Description}
\label{sec:dataset}


In this section, we describe our dataset. 
\dataset contains \num{1209} unique roots.
A root refers to the first element in a package name. 
For example, in the package name "com.example.mypackage", "com" is the root.

We also collected data on the number of fields in the package names of the apps in our dataset. 
A field refers to a dot-separated element in a package name.
For example, in the package name "com.example.mypackage", there are three fields: "com", "example", and "mypackage".
Results are visible in Table~\ref{table:num_of_fields}.
There are 733 package names with only one field, \num{14368} package names with two fields, etc.

\begin{table}
    \centering
    \caption{Number of package names per field in \dataset}
    \begin{adjustbox}{width=.7\columnwidth,center}
        \begin{tabular}{lr|lr}
            \hline
            Fields & Count & Fields & Count \\ \hline
            with 1 field & \num{733} & with 5 fields & \num{100}\\ 
            with 2 fields & \num{14368} & with 6 fields & \num{6}\\ 
            with 3 fields & \num{4231} & with 7 fields & \num{3}\\ 
            with 4 fields & \num{720} & with 8 fields & \num{1}\\
            \hline \hline
            \multicolumn{2}{l|}{Total} & \multicolumn{2}{r}{\libsAfterRefinement}
        \end{tabular}
    \end{adjustbox}
    \label{table:num_of_fields}
\end{table}



There are significantly fewer package names with four or more fields. 
The number of package names with one field is relatively low compared to the others.
This suggests that many package names in the dataset follow a standard naming convention with a domain name followed by one or more subpackages.
The presence of package names with four or more fields may indicate the use of more complex or specialized naming conventions\footnote{Examples of libraries are:  
\href{https://mvnrepository.com/artifact/riddley/riddley}{riddley}, 
\href{https://mvnrepository.com/artifact/jakarta.annotation}{jakarta.annotation}, 
\href{https://mvnrepository.com/artifact/com.vogle.sbpayment}{com.vogle.sbpayment}, 
\href{https://mvnrepository.com/artifact/pl.robakowski.jersey.bootstrap}{pl.robakowski.jersey.bootstrap}, 
\href{https://mvnrepository.com/search?q=de.tudresden.inf.lat.jsexp}{de.tudresden.inf.lat.jsexp}, 
\href{https://mvnrepository.com/artifact/de.hs_rm.cs.vs.tools.vocabularygenerator}{de.hs\_rm.cs.vs.tools.vocabularygenerator}, 
\href{https://mvnrepository.com/artifact/eu.adlogix.com.google.api.ads.dfp}{eu.adlogix.com.google.api.ads.dfp}, 
\href{https://mvnrepository.com/artifact/us.gov.dot.faa.ang.c55/huggs}{us.gov.dot.faa.ang.c55.gradle.huggs}.
}.


Table~\ref{table:top_ten} presents the top 10 most frequent roots and the top 10 most frequent fields found. 
In the first two columns, we can see that the root "com" is by far the most frequent, with more than \num{10000} occurrences. 
The second most frequent root is "net", with \num{1265} occurrences. 
In the second two columns, which represents the most frequent fields, including the roots, we can see that the field "com" is still the most frequent. 
The second two columns do not differ much from the first two columns, except for the two fields "gradle" and "android" that now appear.
This could indicate that Android libraries are prevalent in the dataset.
It is confirmed in the last two columns, which represent the most frequent fields without the roots. 
After "gradle" and "android", the third most frequent field is "sdk", with 138 occurrences.
We see a shift in the most prevalent fields. 
Instead of roots, we now see fields such as "sdk", "maven", "plugin(s)", "api", "tools", and "common".
This may be indicative of the types of libraries.
Overall, the results suggest that most package names are from the "com" domain and that Android libraries are well represented.


\begin{table}
    \centering
    \caption{Top 10 roots and fields present in \dataset}
    \begin{adjustbox}{width=.9\columnwidth,center}
        \begin{tabular}{c|c|c|c|c|c}
            \hline
            \multicolumn{2}{c|}{\textbf{Top 10 roots}} & \multicolumn{2}{c|}{\textbf{Top 10 most used fields}} & \multicolumn{2}{c}{\textbf{Top 10 most used fields w/o roots}}\\ \hline
            \textbf{Root} & \textbf{Count} & \textbf{Field} & \textbf{Count} & \textbf{Field} & \textbf{Count} \\ \hline \hline
            com & \num{10520} & com & \num{10551} & gradle & \num{250} \\ \hline
            net & \num{1265} & net & \num{1273} & android & \num{239} \\ \hline
            de & \num{917} & de & \num{918} & sdk & \num{138} \\ \hline
            cn & \num{663} & cn & \num{663} & plugin & \num{120} \\ \hline
            dev & \num{458} & dev & \num{465} & maven & \num{108} \\ \hline
            me & \num{411} & me & \num{413} & plugins & \num{93} \\ \hline
            eu & \num{223} & gradle & \num{254} & api & \num{60} \\ \hline
            ru & \num{202} & android & \num{240} & oss & \num{57} \\ \hline
            fr & \num{188} & eu & \num{224} & tools & \num{54} \\ \hline
            ch & \num{181} & ru & \num{203} & common & \num{39} \\ \hline
        \end{tabular}
    \end{adjustbox}
    \label{table:top_ten}
\end{table}

\section{Method} \label{method_hybridaugment}
In this section, we formally define the problem, motivate our work and then present our proposed techniques.


\subsection{Preliminaries}
Let $\mathcal{F}(x;W)$ be an image classification CNN trained on the training set $\mathcal{T}_\text{train} = (x_{i}, y_{i})^{N}_{i=1}$  with $N$ samples, where $x$ and $y$ correspond to images and labels. The clean accuracy (CA) of $\mathcal{F}(x;W)$ is formally defined as its accuracy over a clean test set $\mathcal{T}_\text{test} = (x_{j}, y_{j})^{M}_{j=1}$. Assume two operators ${A}(\cdot)$ and ${C}(c, s)$ that adversarially attacks or corrupts a given set of images with the corruption category $c$ and severity $s$, respectively.  Let $A\mathcal{T}_\text{test}$ and $C\mathcal{T}_\text{test}$ be the adversarially attacked and corrupted versions of $\mathcal{T}_\text{test}$, and let $\mathcal{F}(x;W)$ have a robust accuracy (RA) on $A\mathcal{T}_\text{test}$ and a corruption accuracy (CRA) on $C\mathcal{T}_\text{test}$. 
The aim is to fit $\mathcal{F}(x;W)$ such that the model gains robustness (\ie. increased RA and CRA compared its the baseline version), while retaining (or improving) the clean accuracy of its baseline version trained without robustness concerns.


\noindent \textbf{What we know.} Our work builds on the following crucial observations: i) CNNs favour high-frequency content \cite{wang2020high}, ii) adversaries and corruptions often reside in high-frequency \cite{wang2020towards}, iii) images are dominated by low-frequency \cite{Saikia_2021_ICCV} and iv) models relying on low-frequency components are more robust \cite{li2022robust,wang2020towards}. The robustness-accuracy trade-off is visible; low-frequency reliant models are more robust, but tend to miss out on clean accuracy brought by the high-frequency components. 

\subsection{HybridAugment}
We hypothesize that a \textit{sweet spot} in the robustness-accuracy trade-off can be found. Unlike the \textit{hard} approaches that completely rule out the reliance on high-frequency components (i.e. low-pass filters), we propose to \textit{reduce} the reliance on them. To this end, we adopt a data augmentation approach that aims to diversify $\mathcal{T}_\text{train}$ by an operation $\mathcal{HA(\cdot)}$. Keeping the strong relation intact between labels and low-frequency content (i.e. labels come from low-frequency-component image), we propose to swap high and low-frequency components of images in a batch on-the-fly. Unlike \cite{mukai2022improving}, we \textit{do not} restrict the images to belong to the same class; this diversifies the training distribution even further while preserving the image semantics. We call this basic version of our approach \textit{HybridAugment}, which corresponds to: 
%
\begin{equation} \label{hybrid_augment_paired}
    \mathcal{HA_{P}}(x_{i}, x_{j}) = \mathcal{LF}(x_{i}) + \mathcal{HF}(x_{j})
\end{equation}
%
where $x_{i}$ is the input image and $x_{j}$ is a randomly sampled image from the whole training set, which we simply sample from the mini batch at each training iteration in practice. $\mathcal{HF}$ and $\mathcal{LF}$ operators select the high and low-frequency components of an input image, for which we use:
%
\begin{equation} \label{eq:cutoff}
\begin{split}
    \mathcal{LF}(x) = GaussBlur(x) \\
    \mathcal{HF}(x) = x - \mathcal{LF}(x)
    \end{split}
\end{equation}
%
where $GaussBlur$ is used as a low-pass filter. Note that a similar outcome is possible by using Discrete Fourier Transforms (DFT), swapping the frequency bands and then applying Inverse DFT (IDFT). We find the gaussian blur operation to be faster and better in practice. 


Inspired from \cite{chen2021amplitude}, in addition to the image-pair scheme in Eq.~\ref{hybrid_augment_paired}, we propose a single image variant of \textit{HybridAugment}. In the single image variant, instead of combining two images, $x_i$ and $x_{j}$ are obtained by applying randomly sampled augmentations to a single image. The single image variant $\mathcal{HA_{S}}$ can therefore be defined as 
%
\begin{equation} \label{hybrid_augment_single}
    \mathcal{HA_{S}}(x_{i}) = \mathcal{LF}(Aug(x_{i})) + \mathcal{HF}(\hat{Aug}(x_{i}))
\end{equation}
%
where $Aug$ and $\hat{Aug}$ correspond to two sets of randomly sampled augmentation operations. Note that paired and single versions can work in tandem ($\mathcal{HA_{PS}}$), and actually outperform single or paired image versions. 


\subsection{HybridAugment++}


The frequency analysis is a vast literature, however, two core aspects often stand out; frequency-band analysis (i.e. low, high) and the decomposition of signals into amplitude and phase. \textit{HybridAugment} covers the former and shows competitive results in various benchmarks (see Section \ref{sec:exp_hybridaugment}). The latter is investigated in $\mathcal{APR}$ \cite{chen2021amplitude}, where phase is shown to be the more relevant component for correct classification, and training models based on their phase labels and swapping amplitude components of images randomly lead to more robust models. Note that frequency-band and phase/amplitude discussions are arguably orthogonal, since frequency, phase and amplitude provide distinct characterizations of a signal: intuitively speaking, frequency, phase and amplitude can be seen as the separation of visual patterns in terms of scale, location and significance. 


We hypothesize these two approaches can be complementary; a model reliant on low-frequency and spatial information (i.e. phase) can further improve robustness. Inspired by the successes of cascaded augmentation methods \cite{hendrycks2019augmix,wang2021augmax,calian2022defending}, we unify these two core aspects into a single, hierarchical augmentation method. We refer to this method as \textit{HybridAugment++} and define its paired version as:
%
\begin{equation}
  \mathcal{HA_{P}}^{++}(x_{i}, x_{j}, x_{z}) = \mathcal{APR_{P}}(\mathcal{LF}(x_{i}), x_{z}) + \mathcal{HF}(x_{j})
\end{equation}
%
where $x_{i}$, $x_{j}$ and $x_{z}$ are images sampled from the same batch. Here, $\mathcal{APR_{P}}$~\cite{chen2021amplitude} is defined as
\begin{equation}
    \mathcal{APR_{P}}(x_{i}, x_{z}) = \mathcal{IDFT}(A_{x_{z}} \otimes e^{i. P_{x_{i}}}) \\
\end{equation}
%
where $\otimes$ is element-wise multiplication, $A$ is the amplitude and $P$ is the phase component. Similar to $\mathcal{HA}$ and $\mathcal{APR}$, we also define a single-image version of \textit{HybridAugment++} as
%
\begin{equation}
 \mathcal{HA_{S}}^{++}(x_{i}) = \mathcal{APR_{S}}(\mathcal{LF}(Aug(x_{i}))) + \mathcal{HF}(\hat{Aug}(x_{i}))
\end{equation}
%
where $\mathcal{APR_{S}}$~\cite{chen2021amplitude} is defined as
%
\begin{equation}
\mathcal{APR_{S}}(x_{i}) = \mathcal{IDFT}\left(A_{\bar{Aug}(x_{i})} \otimes e^{i. P_{\overline{Aug}\left(x_{i}\right)}}\right)    
\end{equation}
%
where $Aug$, $\hat{Aug}$, $\bar{Aug}$ and $\overline{Aug}$ are different sets of randomly sampled augmentation operations. Note that we essentially propose a framework; one can use different single and paired image augmentations, either individually or together, and can still achieve competitive results (see ablations in Section \ref{sec:exp_hybridaugment}). There are also other alternatives, such as swapping phase/amplitude first and then performing $\mathcal{HA}$, but we observe poor performance in practice; dividing the phase component into frequency-bands is not interpretable as frequencies of the phase component are not well defined. The pseudo-code of our methods can be found in the supplementary material.





\begin{table}[t]
	\centering
	\caption{Preallocation strategy results with $3$ machines per tool group and $10$ operations per lot}
	\label{tab:table}
	\figspace\scriptsize
	%	\resizebox{15.5cm}{!}{
		\begin{tabular}{|l%r
				cl||rr|rr|rr|rr|}
			%			\hline
			%			&                    &                      & %        &
			%			 \multicolumn{8}{c}{\textbf{M = 9}} \\
			\hline
			& \multicolumn{1}{@{\hspace{-3mm}}c@{\hspace{-3mm}}}{\textbf{9 Machines}}                   &                      & % &
			\multicolumn{2}{r|}{\textbf{70 Operations}}                 & \multicolumn{2}{r|}{\textbf{80 Operations}}                 & \multicolumn{2}{r|}{\textbf{90 Operations}}                 & \multicolumn{2}{r|}{\textbf{100 Operations}}                 \\
			& Size % \multicolumn{2}{c}{\textbf{Parameters}}            
			&        &
			Lot                         & Step                        & Lot                         & Step                        & Lot          & Step         & Lot          & Step         \\
			%			& size              % & setup % idx
			%			                  &         & 0                           & 1                           & 0                           & 1                           & 0            & 1            & 0            & 1            \\
			%			&                    & setup                &         &                             &                             &                             &                             &              &              &              &              \\
			\hline\hline
			\multirow{3}{*}{\textbf{Fixed}}    & \multirow{3}{*}{1} & % \multirow{3}{*}{0/1} &
			Makespan    & 483                         & 428                         & 489                         & 440                         & 486          & 531          & 592          & 553         \\
			&                    & %                     &
			Setup/Batch & 6/12                        & 2/12                        & 5/14                        & 0/13                        & 5/14         & 3/12         & 3/12         & 0/16         \\
			&                    & %                     &
			1\ts{st}/2\ts{nd} Stage & 2/1                         & TO/27                          & 6/2                        & TO/13                          & 11/13         & TO           & TO/78           & TO           \\
			\midrule
			\multirow{6}{*}{\textbf{Flexible}} & \multirow{3}{*}{2} & % \multirow{6}{*}{0}   &
			Makespan    & 483                         & 475                         & 592                         & 592                         & 592          & 539          & 745          & 698          \\
			&                    & %                     &
			Setup/Batch & 2/8                        & 0/9                        & 1/8                        & 1/8                        & 1/10         & 0/11          & 0/12          & 0/15          \\
			&                    & %                     &
			1\ts{st}/2\ts{nd} Stage & 5/1                         & TO                          & TO/114                          & TO/1                          & TO/130           & TO           & TO           & TO          \\
			\cline{2-11}
			%			& & & & & & & & & & &   \\
			& \multirow{3}{*}{3} & %                     &
			Makespan    & 559                         & --                          & 815                         & --                          & 1357 & -- & 1486 & -- \\ % \multicolumn{4}{c|}{\multirow{3}{*}{Assignment issue}}     \\
			&                    & %                     &
			Setup/Batch & 0/8                         & --                          & 0/8                        & --                          & 0/10 & -- & 10/18 & -- \\ %\multicolumn{4}{c|}{}                                      \\
			&                    & %                     &
			1\ts{st}/2\ts{nd} Stage & TO                       & --                          & TO/140                          & --                          & TO/79 & -- & TO & -- \\ %\multicolumn{4}{c|}{}                                      \\
			\midrule
			\multirow{6}{*}{\textbf{Setup}}    & \multirow{3}{*}{2} & % \multirow{6}{*}{1}   &
			Makespan    & 483                         & 475                         & 592                         & 592                         & 592          & 536          & 745          & 683          \\
			&                    & %                     &
			Setup/Batch & 2/8                        & 0/9                        & 1/8                        & 1/8                        & 1/10         & 0/12          & 0/13          & 0/16          \\
			&                    & %                     &
			1\ts{st}/2\ts{nd} Stage & 2/1                        & TO                          & TO/21                          & TO/25                          & TO/22           & TO           & TO/76           & TO           \\
			%			& & & & & & & & & & &   \\
			\cline{2-11}
			& \multirow{3}{*}{3} & %                     &
			Makespan    & \textbf{334}                         & --                          & \textbf{345}                         & --                          & \textbf{434}          & --           & \textbf{555}          & --           \\
			&                    & %                     &
			Setup/Batch & 0/8                         & --                          & 0/8                         & --                          & 0/11          & --           & 0/12          & --           \\
			&                    & %                     &
			1\ts{st}/2\ts{nd} Stage & TO/20                       & --                          & TO/123                          & --                          & TO           & --           & TO/73           & --           \\
			\hline
		\end{tabular}
		%	}
\end{table}
%
We constructed a scalable set of benchmark instances, focusing on sub-routes of
$10$ production operations for two product types from the SMT2020 simulation scenario~\cite{kopp2020smt2020}.
The $10$ operations in both sub-routes are processed by machines
belonging to three tool groups and do thus involve re-entrant flow,
as a lot visits the same tool group multiple times.
Moreover, the operations incorporate batching and specific setups, and machines undergo periodic maintenance operations.
In the following, we concentrate on instances with $9$ machines, i.e., $3$ per
tool group, and gradually increasing number of lots.
Further smaller- and larger-scale instances along with our implementation are
available online.\footref{foo:online}

We ran our experiments with \clingodl\ (version 1.4.0) on an Intel® Core™i7-8650U CPU Dell Latitude 5590 machine under Windows 10, imposing two time limits per run:
the first stage for makespan minimization is aborted at $450$ seconds, in which case the best schedule found so far % (if any) 
is taken as upper bound on the makespan for proceeding to minimize setup and batch violations with 
another $150$ seconds time limit.

Table~\ref{tab:table} reports the quality of best schedules obtained within the time limits for both optimization stages, split into `Makespan' and `Setup/Batch'
values, while two runtimes or `TO' for a timeout, respectively, are given in the
`1\ts{st}/2\ts{nd} Stage' rows, only listing a single `TO' entry in case both stages timed out.
The `Size' column provides the value taken for the constant \lstinline{sub_size},
limiting the number of machines in subgroups to which the operations are preallocated.
For the latter, the `Lot' columns include results with value \lstinline{0} for the constant \lstinline{lot_step}, where a common subgroup takes all operations for a lot, or for value \lstinline{1} in the `Step' columns, leading to their distribution among subgroups.

The `Size' value 1 necessarily leads to a fixed machine assignment, for which the
quality indicators clearly show that the `Step' strategy yields better schedules,
although it incurs more timeouts and thus fewer certain optima because operations on different lots increase the flexibility of execution sequences and thus search complexity.
While flexibility within subgroups by setting their `Size' to 2 or 3 in principle allows for improved schedules, we observe a deterioration due to sharply increasing instantiation size and search effort, as already observed in \cite{ali2023flexible}.
The setup strategy to differentiate operations and machines within subgroups,
activated by changing the constant \lstinline{by_setup},
aims to cut down the scheduling complexity in line with the optimization objectives by reducing the need for setup changes.
This leads to significantly improved schedules with `Size' 3, where the
`Lot' and `Step' preallocation strategies are indifferent and redundant results for the latter are omitted, up to a critical size reached with $100$~operations.

With our preliminary approach~\cite{ali2023flexible}, using a more naive and less feature-rich encoding of either fixed or fully flexible machine assignments, the
threshold at which problem size and combinatorics get prohibitive was reached at less than $50$ operations already.
Despite gearing up to double that size, our benchmark instances still represent small excerpts of the large-scale semiconductor fabs with more than $100$ tool groups and from $242$ to $543$ production operations per lot modeled by~\cite{kopp2020smt2020}.
%
The elevated complexity in comparison to basic settings like the traditional FJSP is mainly due to sophisticated setup and maintenance operations, requiring a detailed analysis of execution sequences on machines for SMSP.
We conjecture that similar scalability limits would also be encountered with MIP or CP encodings, yet the first-order modeling language of ASP with difference logic facilitates rapid prototyping and experimentation.
In fact, our performance evaluation aims to explore the feasibility of search and optimization, in order to come up with strategies for breaking down large SMSP instances into more manageable portions, e.g., focusing on some bottleneck tool groups or re-entrant flow of operations.

% This section will show the experimental results performed by applying the machine assignment strategies mentioned before, with several instances ranging from $30$ to $130$ steps and $6$ to $12$ machines. All experiments are run using an Intel\textsuperscript{\textregistered} Core\texttrademark{} i7-8650U CPU Dell Latitude 5590 machine under Windows 10. Our timeout limit is $600$ seconds, splitted to $450$ seconds for the makespan and $150$ seconds for the setup and batching. 

% We considered three tool groups for all generated instances in which batch processing, time/counter-based maintenance, and setup are considered. For generating the instances, we started with a small instance containing $30$ steps and $6$ machines where each tool group has $2$ machines and then we generate the next instance by adding one more lot, which has $10$ steps. We kept the tool group size till the fixed machine assignment strategy could not reach the optimum within the time limit. We created $3$ parameters \textit{size, idx} and \textit{setup} to activate a specific machine assignment strategy. The size determines the size of a sub-group in each tool group. The $idx$ defines the Job/Step-based indexing of all steps in the same tool group where all steps of the same lot will have the same index if the $idx = 0$ and Hence, they are assigned to the same sub-group/machine. If $idx = 1$, then each step in the tool group will have an identical index. The last parameter setup is to activate the setup strategy or not. If the $setup = 1$, then the setup strategy is applied; if $setup = 0$ then it's not applied.

% % To continue tomorrow isA :)
% Table \ref{tab:table01} shows the results of the instances with $2$ machines in each toll group. The first column refers to the strategy applied for the machine assignment. The second and third columns show the parameters for selecting a particular strategy. The assignment is fully flexible if the \textit{size} is greater than or equal to the number of machines in a tool group. Otherwise, the assignment is partially flexible. In the fourth column, we list our optimization criteria and the time limit for the makespan and setup/batching represented by 1st/2nd call. Each following two consecutive columns illustrate the results of an instance when the Job/Step-based indexing is selected. From the \ref{tab:table01}, we observed that the best-obtained results were achieved by the full flexible assignment in the first three instances and for the last instance, the setup strategy was the best. The fixed/setup strategies terminated within the time limit except for only one case.

% \begin{table}[h]
% 	\centering
% 	\caption{Comparison between the allocation strategies with 2 machines per tool group}
% 	\label{tab:table01}
% %	\resizebox{15.5cm}{!}{
% 		\begin{tabular}{|l%r
% 			cl||rr|rr|rr|rr|}
% 			\hline
% %			&                    &                      &         & \multicolumn{8}{c}{\textbf{M = 6}} \\
% %			\hline
% 			& \textbf{M = 6}                   & %                     &
% 			  & \multicolumn{2}{r|}{\textbf{Instance 01}}                 & \multicolumn{2}{r|}{\textbf{Instance 02}}                 & \multicolumn{2}{r|}{\textbf{Instance 03}}                 & \multicolumn{2}{r|}{\textbf{Instance 04}}                 \\
% 			& Size % \multicolumn{2}{c}{\textbf{Parameters}}            
% 			 &			         & Job                         & Step                        & Job                         & Step                        & Job          & Step         & Job          & Step         \\
% 			\hline
% %			& size               & setup %idx
% %			                  &         & 0                           & 1                           & 0                           & 1                           & 0            & 1            & 0            & 1            \\
% %			&                    & setup                &         &                              &                             &                             &                             &              &              &              &              \\
% 			\hline
% 			\multirow{3}{*}{\textbf{Fixed}}    & \multirow{3}{*}{1} & % \multirow{3}{*}{0/1} &
% 			 Makespan    & 409                         & 353                         & 409                         & 409                         & 525          & 424          & 525          & 493          \\
% 			&                    & %                     &
% 			 Setup/Batch & 5/6                         & 4/6                         & 4/8                         & 4/8                         & 4/9          & 1/9          & 3/11          & 2/10          \\
% 			&                    & %                     &
% 			 1\ts{st}/2\ts{nd}-Call & \textless{}1/\textless{}1 & \textless{}1/\textless{}1 & \textless{}1/\textless{}1 & \textless{}1/\textless{}1 & 31/1         & 137/6        & 37/11          & TO/53           \\
% 			\midrule
% 			\multirow{3}{*}{\textbf{Flexible}} & \multirow{3}{*}{2} & % \multirow{3}{*}{0}   &
% 			 Makespan   & \textbf{233}                         & --                          & \textbf{281}                         & --                          & \textbf{365}          & --           & 587          & --           \\
% 			&                    & %                     &
% 			 Setup/Batch & 0/5                         & --                          & 0/6                         & --                          & 0/8          & --           & 3/9          & --           \\
% 			&                    & %                     &
% 			 1\ts{st}/2\ts{nd}-Call & 7/0                         & --                          & TO/6                          & --                          & TO/83           & --           & TO           & --           \\
% 			\midrule
% 			\multirow{3}{*}{\textbf{Setup}}    & \multirow{3}{*}{2} & % \multirow{3}{*}{1}   &
% 			 Makespan  & 277                         & --                          & 321                         & --                          & 381          & --           & \textbf{419}          & --           \\
% 			&                    & %                     &
% 			 Setup/Batch & 0/4                         & --                          & 0/6                         & --                          & 0/8          & --           & 0/9          & --           \\
% 			&                    & %                     &
% 			 1\ts{st}/2\ts{nd}-Call & \textless{}1/\textless{}1 & --                          & 25/1                         & --                          & TO/12        & --           & TO/122           & -- \\
% 			 \hline
% 		\end{tabular}
% %	}
% \end{table}

% Table~\ref{tab:table02} summarizes the results of the subsequent $4$ instances where each tool group has $3$ machines. In this instances group, we can split the machines into sub-group by setting the \textit{size} parameter to $2$; in that case, we have two sub-groups in each tool group. The experiments demonstrated that the fixed strategy has the same or better performance than the flexible. In addition, the flexible strategy could not find a feasible solution for instances $7$ and $8$ when all machines were in the same group. On the other hand, the setup strategy performed better than the other two strategies when all machines were in one group, in addition to reaching the optimal value of the setup for all instances. 

% \begin{table}[h]
% 	\centering
% 	\caption{Comparison between the allocation strategies with 3 machines per tool group}
% 	\label{tab:table02}
% %	\resizebox{15.5cm}{!}{
% 		\begin{tabular}{|l%r
% 			cl||rr|rr|rr|rr|}
% %			\hline
% %			&                    &                      & %        &
% %			 \multicolumn{8}{c}{\textbf{M = 9}} \\
% 			\hline
% 			& \textbf{M = 9}                   &                      & % &
% 			 \multicolumn{2}{r|}{\textbf{Instance 05}}                 & \multicolumn{2}{r|}{\textbf{Instance 06}}                 & \multicolumn{2}{r|}{\textbf{Instance 07}}                 & \multicolumn{2}{r|}{\textbf{Instance 08}}                 \\
% 			& Size % \multicolumn{2}{c}{\textbf{Parameters}}            
% 			&        &
% 			 Job                         & Step                        & Job                         & Step                        & Job          & Step         & Job          & Step         \\
% %			& size              % & setup % idx
% %			                  &         & 0                           & 1                           & 0                           & 1                           & 0            & 1            & 0            & 1            \\
% %			&                    & setup                &         &                             &                             &                             &                             &              &              &              &              \\
% 			\hline\hline
% 			\multirow{3}{*}{\textbf{Fixed}}    & \multirow{3}{*}{1} & % \multirow{3}{*}{0/1} &
% 			 Makespan    & 525                         & 433                         & 525                         & 452                         & 525          & 521          & 643          & \textbf{559}          \\
% 			&                    & %                     &
% 			 Setup/Batch & 6/13                        & 1/13                        & 5/15                        & 0/14                        & 5/16         & 6/16         & 6/12         & 3/12         \\
% 			&                    & %                     &
% 			 1\ts{st}/2\ts{nd}-Call & 30/3                         & TO/153                          & 24/8                        & TO/63                          & 231/81         & TO           & TO           & TO           \\
% 			\midrule
% 			\multirow{6}{*}{\textbf{Flexible}} & \multirow{3}{*}{2} & % \multirow{6}{*}{0}   &
% 			 Makespan    & 525                         & 475                         & 650                         & 650                         & 650          & 595          & 745          & 742          \\
% 			&                    & %                     &
% 			 Setup/Batch & 2/11                        & 0/11                        & 1/12                        & 1/12                        & 6/13         & 4/14          & 3/17          & n/a          \\
% 			&                    & %                     &
% 			 1\ts{st}/2\ts{nd}-Call & 26/7                         & TO                          & TO/12                          & TO                          & TO           & TO           & TO           & TO           \\
% 			\cline{2-11}
% %			& & & & & & & & & & &   \\
% 			& \multirow{3}{*}{3} & %                     &
% 			 Makespan    & 744                         & --                          & 1206                         & --                          & 1698 & -- & n/a & -- \\ % \multicolumn{4}{c|}{\multirow{3}{*}{Assignment issue}}     \\
% 			&                    & %                     &
% 			 Setup/Batch & 2/12                         & --                          & n/a                        & --                          & 8/15 & -- & n/a & -- \\ %\multicolumn{4}{c|}{}                                      \\
% 			&                    & %                     &
% 			 1\ts{st}/2\ts{nd}-Call & TO                       & --                          & TO                          & --                          & TO & -- & TO & -- \\ %\multicolumn{4}{c|}{}                                      \\
% 			\midrule
% 			\multirow{6}{*}{\textbf{Setup}}    & \multirow{3}{*}{2} & % \multirow{6}{*}{1}   &
% 			 Makespan    & 525                         & 475                         & 650                         & 650                         & 643          & 553          & 745          & 642          \\
% 			&                    & %                     &
% 			 Setup/Batch & 2/11                        & 0/11                        & 1/12                        & 1/12                        & 1/14         & 0/13          & 1/14          & 1/16          \\
% 			&                    & %                     &
% 			 1\ts{st}/2\ts{nd}-Call & 44/2                        & TO                          & TO/4                          & TO/2                          & TO           & TO/7           & TO           & TO           \\
% %			& & & & & & & & & & &   \\
% 			\cline{2-11}
% 			& \multirow{3}{*}{3} & %                     &
% 			 Makespan    & \textbf{346}                         & --                          & \textbf{373}                         & --                          & \textbf{429}          & --           & 820          & --           \\
% 			&                    & %                     &
% 			 Setup/Batch & n/a                         & --                          & n/a                         & --                          & n/a          & --           & n/a          & --           \\
% 			&                    & %                     &
% 			 1\ts{st}/2\ts{nd}-Call & TO                       & --                          & TO                          & --                          & TO           & --           & TO           & --           \\
% 			\hline
% 		\end{tabular}
% %	}
% \end{table}

% Table~\ref{tab:table03} considers $4$ machines in each tool group and the flexible strategy obtained the best result for the first instance. However, it had the same feasibility issue when all machines were in the same group. For the rest instances, the setup strategy dominated when the machines were equally distributed into sub-groups. 

% From the conducted experiments, we can conclude that 
% \begin{itemize}
% 	\item The flexible assignment performed well on the small-scale.
% 	\item While increasing the scale, the setup strategy dominates in the most cases
% 	\item Assigning the steps of the same lot independently with the fixed assignment leads to better performance
% 	\item The Setup strategy has a significant impact in minimizing the setup objective through all instances
% 	\item The full flexible assignment has an assignment issue while increasing the number of machines
% \end{itemize}

% \begin{table}[h]
% 	\centering
% 	\caption{Comparison between the allocation strategies with 4 machines per tool group}
% 	\label{tab:table03}
% %	\resizebox{15.5cm}{!}{%
% 		\begin{tabular}{|l%r
% 			cl||rr|rr|rr|rr|}
% 			\hline
% %			&                    &                      &  &  \multicolumn{8}{c}{\textbf{M = 12}} 
% %			\\ \hline
% 			& \textbf{M = 12}                   & %                     & 
% 			 & \multicolumn{2}{r|}{\textbf{Instance 09}}                 & \multicolumn{2}{r|}{\textbf{Instance 10}}                 & \multicolumn{2}{r|}{\textbf{Instance 11}}                 & \multicolumn{2}{r|}{\textbf{Instance 12}}                 \\
% 			& Size % \multicolumn{2}{l}{\textbf{Parameters}}            
% 			 &			 &			 Job                    & Step                   & Job                    & Step                   & Job                    & Step                   & Job                    & Step                   \\
% %			& Size               & setup % idx
% %			                  &  & 0                      & 1                      & 0                      & 1                      & 0                      & 1                      & 0                      & 1                      \\
% %			&                    & setup                &  &  &                        &                        &                        &                        &                        &                        &                                               \\
% 			\hline\hline
% 			\multirow{3}{*}{\textbf{Fixed}}    & \multirow{3}{*}{1} & % \multirow{3}{*}{0/1} &
% 			 Makespan                 & 525                    & 453                    & 525                    & 452                    & 525                    & 493                    & 643                    & 561                    \\
% 			&                    & %                     &
% 			 Setup/Batch              & 7/19                   & 3/20                   & 7/20                  & n/a                   & 6/22                   & 4/20                   & 4/22                   & n/a                   \\
% 			&                    & %                     &
% 			 1\ts{st}/2\ts{nd}-Call              & 124/5                 & TO & 25/17                 & TO & 25/53                 & TO/142 & TO & TO \\
% 			\midrule
% 			\multirow{9}{*}{\textbf{Flexible}} & \multirow{3}{*}{2} & % \multirow{9}{*}{0}   &
% 			 Makespan                 & \textbf{373}                    & 503                    & 491                    & 778                    & 569                    & 569                    & 765                    & 1673                   \\
% 			&                    & %                     &
% 			 Setup/Batch              & n/a                    & 6/17                    & n/a                   & n/a                    & n/a                    & n/a                   & n/a                    & 12/24                  \\
% 			&                    & %                     &
% 			 1\ts{st}/2\ts{nd}-Call              & TO & TO & TO & TO & TO & TO & TO & TO \\
% 			\cline{2-11}
% %			& & & & & & & & & & &   \\
% 			& \multirow{3}{*}{3} & %                     &
% 			 Makespan                 & 709                    & 688                    & 800                    & 907                    & 876                    & 876                    & 905                    & 1643                   \\
% 			&                    & %                     &
% 			 Setup/Batch              & 5/17                    & n/a                   & 3/18                   & 5/19                   & n/a                   & n/a                   & n/a                  & 15/24                    \\
% 			&                    & %                     &
% 			 1st/2nd              & TO & TO & TO & TO & TO & TO & TO & TO \\
% 			\cline{2-11}
% %			& & & & & & & & & & &   \\
% 			& \multirow{3}{*}{4} & %                     &
% 			 Makespan                 & n/a & -- & n/a & -- & n/a & -- & n/a & -- \\ %\multicolumn{8}{c|}{\multirow{3}{*}{Assignment issue}}                                                                                                                                                 \\
% 			&                    & %                     &
% 			 Setup/Batch              & n/a & -- & n/a & -- & n/a & -- & n/a & -- \\ %\multicolumn{8}{c|}{}                                                                                                                                                                                  \\
% 			&                    & %                     &
% 			 1\ts{st}/2\ts{nd}-Call              & TO & -- & TO & -- & TO & -- & TO & -- \\ %\multicolumn{8}{c|}{}                                                                                                                                                                                  \\
% 			\midrule
% 			\multirow{9}{*}{\textbf{Setup}}    & \multirow{3}{*}{2} & % \multirow{9}{*}{1}   &
% 			 Makespan                 & 401                    & 396                    & 419                    & \textbf{416}                    & \textbf{419}                    & \textbf{419}                    & \textbf{457}                    & 471                    \\
% 			&                    & %                     &
% 			 Setup/Batch              & 0/15                   & 0/14                   & 0/16                   & 0/16                   & n/a                   & n/a                   & 0/21                    & n/a                    \\
% 			&                    & %                     &
% 			 1\ts{st}/2\ts{nd}-Call              & TO & TO/92 & TO & TO & TO & TO & TO & TO \\
% 			\cline{2-11}
% %			& & & & & & & & & & &   \\
% 			& \multirow{3}{*}{3} & %                     &
% 			 Makespan                 & 706                    & 642                    & 792                    & 753                    & 942                    & 942                    & 939                    & 894                    \\
% 			&                    & %                     &
% 			 Setup/Batch              & 1/14                    & n/a                    & 2/16                    & n/a                   & n/a                   & n/a                    & n/a                    & 1/22                    \\
% 			&                    & %                     &
% 			 1\ts{st}/2\ts{nd}-Call              & TO & TO & TO & TO & TO & TO & TO & TO \\
% 			\cline{2-11}
% %			& & & & & & & & & & &   \\
% 			& \multirow{3}{*}{4} & %                     &
% 			 Makespan                 & 679                    & -- & 1725                    & -- & n/a                    & -- & n/a                    & -- \\
% 			&                    & %                     &
% 			 Setup/Batch              & n/a                   & -- & n/a                    & -- & n/a                   & -- & n/a                   & -- \\
% 			&                    & %                     &
% 			 1st/2nd              & TO & -- & TO & -- & TO & -- & TO & -- \\
% 			\hline
% 		\end{tabular}%
% %	}
% \end{table}

\section{Conclusion}
This paper proposes a new dataset, task, and benchmark to enhance the understanding ability of AI agents in dealing with complex interactions with multiple characters. The existing works in this field have limitations, particularly their reliance on forum-based data collections and do not consider complex and grounded semantics in the real-time communications. To overcome these limitations, we formalize a new task named Multiple character and Open instances based interaction Estimation (MOE), providing a testbed for the understanding ability of the agents and leading further improvements in agents' factual correctness. We also introduce a dataset to support MOE task, which is derived from real-time game logs in tabletop role-playing games (TRPGs) and provides a richer and more complex context capable of supporting MOE tasks.
Additionally, we introduce a prompting benchmark designed specifically to refine the interaction capabilities of AI agents in TRPGs. This benchmark focuses on understanding complex interactions and generating vibrant game master utterances. The three-stage generation process, which includes game check and GM utterance generation, has been evaluated both objectively and subjectively. The results clearly indicate that this approach significantly enhances the quality of AI responses within the TRPG context. We hope that this work will serve as inspiration for the AI community to further explore and enhance their understanding of complex grounded interactions and advance the interaction ability of AI agents.

%In this paper, we propose a new dataset, task and benchmark to enhance the understanding ability of AI agents for long and grounded semantics in Tabletop Role-Playing Games (TRPGs). Due to the limitations of current works, particularly their reliance on forum-based data collections and end-to-end utterance generation, we introduces the novel Long-context Grounded-language TRPG Logs dataset (LGL). This dataset, derived from real-time game logs, provides a richer, more complex context, which can support CSA. 
%We further introduce a prompting benchmark, designed to refine the generation capabilities of AI agents in TRPGs. It focuses on understanding complex interactions and generating vibrant game master utterances. The three-stage generation process, consisting of game check and GM utterance generation, has been evaluated both objectively and subjectively. The results indicate that this approach can significantly improve the quality of AI responses in a TRPG context. We hope this work will inspire the AI community to further explore and enhance the understanding of complex grounded semantics and the generation of AI agents.

\section{Limitations and Social Impacts}
While the use of an AI agent in a tabletop role-playing game (TRPG) could revolutionize the way these games are played, providing consistent and unbiased decisions, there are potential limitations and social impacts to consider. One key limitation is the AI's ability to simulate human creativity, empathy, and adaptability, which are all fundamental to the role of a game master. For instance, the AI may not fully comprehend nuanced player interactions or adapt the game based on the players' emotional state. Additionally, there could be social implications, such as the potential reduction in human interaction and shared storytelling, which are often crucial elements of TRPGs. For players, part of the joy of a TRPG is the shared human experience, the unpredictable responses, and the subtle non-verbal cues, which an AI might not replicate. The introduction of an AI game master could also result in job loss in professional game-mastering circles. Despite the AI's potential to provide a consistent and more accessible gaming experience, these human and social elements may be irreplaceable in a TRPG context.

%\bibliographystyle{neurips_data_2023}
%\bibliography{reference}
{
		\small
		\bibliographystyle{ieee_fullname}
		\bibliography{reference}
}

\end{document}
