
\begin{CJK*}{UTF8}{gbsn}
\begin{table}[!htbp]
	\begin{center}
	\resizebox{\columnwidth}{!}
    {
\linespread{1.16}
\small
\begin{tabular}{|m{13.8cm}|}
\hline
\textbf{Input Context:} 
比尔:“说道这个大狩猎石......” \\ 
Bill: "Speaking of this Great Hunting Stone..." \\
莉卡·劳拉瓦:“大狩猎石?” \\ 
Lika Lavava: "The Great Hunting Stone?" \\
比尔:比尔一边搬东西,一边开始侃侃而谈,似乎因为不用再考虑寒冷而感到愉快。 \\ 
Bill: While moving things around, Bill starts to chat happily, seemingly pleased that he no longer needs to worry about the cold. \\
DM:比尔了解到,大狩猎石很久以前就孤然耸立在叶格拉玛森林深处,没人知道上面的方尖碑由谁建立。不过当雪凇狮狩猎逐渐风靡时,猎人们在石丘上开凿出一处临时庇护所,可供来往的人休整歇息。 \\ 
DM: Bill has learned that the Great Hunting Stone has stood alone in the depths of the Yeglama Forest for a long time, and no one knows who erected the obelisk on top of it. However, as hunting the snow lion became increasingly popular, hunters carved out a temporary shelter on the stone hill, providing a place for passers-by to rest and recuperate. \\
... (Due to space limitations, several contents are omitted.) \\ 
莉卡·劳拉瓦:“倒是...不用,说实话我不怎么想把名字留在这种地方...”莉卡委婉的拒绝着。 \\ 
Lika Lavava: "Actually... no need, to be honest, I don't really want to leave my name in such a place..." Lika tactfully declines.\\
比尔:比尔·伍德曼。X年X月X日夜地质考研途经此地,谨作纪念。 \\ 
Bill: Bill Woodman. Passed by here on a certain night of a certain month of a certain year, during a geological research journey. Marking the occasion. \\ 
莫瑞斯:莫瑞斯凑过去:『留名吗?倒是不错。借匕首一用。』 \\ 
Maurice: Maurice leans in: "Leave a name? That's not a bad idea. Can I borrow your dagger?" \\ 
埃尔维斯·泽姆: “我就不用了。” \\ 
Elvis Zem: "I'll pass." \\ 
比尔:比尔刻道。听到莉卡那么说有点遗憾,“哎,好吧,”但是听到莫瑞斯这么说又来了兴致,把匕首递给了他。 \\ 
Bill: Bill starts to carve. He feels a bit disappointed when he hears Lika say that, "Ah, well," but he perks up again when Maurice shows interest and hands him the dagger. \\ 
DM:比尔刻完,抬起头欣赏了片刻。在旁边不远处,他留意到几个刻痕较新的名字:吉尔·坦纳,盖德,卡罗,安德尔都。其中安德尔都的名字拼错了。 \\ 
DM: After Bill finishes carving, he lifts his head to admire his work for a moment. Not far away, he notices a few relatively new names: Gil Tanner, Gade, Karo, Anderdu. The name of Anderdu is spelled incorrectly. \\ 
比尔:然后站在一旁准备看看这位潇洒的男人准备刻下怎么样的诗篇。 \\ 
Bill: Then he steps aside, ready to see what kind of verse this dashing man plans to engrave.\\
莫瑞斯:莫瑞斯把自己的名字刻在比尔的旁边,然后传递魔炎,看能不能烫出不同的颜色,让名字显眼一些。 \\ 
Maurice: Maurice carves his name next to Bill's and then channels his magical flame, hoping to create a different color that makes his name stand out more prominently. \\ 
DM:莫瑞斯用魔焰成功让名字呈现出一种漂亮的蓝紫色。他满意地点了点头。 \\ 
DM: Maurice successfully uses the magical flame to make his name appear in a beautiful shade of blue-violet. He nods in satisfaction.\\
比尔:“哦哦!真是狡猾,给我的名字也烫一个吧莫瑞斯先生。”比尔请求道。 \\ 
Bill: "Oh, oh! That's clever. How about giving my name a little touch of the flame too, Mr. Maurice?" Bill requests. \\ 
比尔:“吉尔·坦纳,盖德,卡罗,安德尔都......”比尔复述了一遍名字,“这些都是新的刻痕,他们是我们要找的猎人吗,咕咕?” \\ 
Bill: "Gil Tanner, Gade, Karo, Anderdu..." Bill repeats the names. "These are all new engravings. Are they the hunters we're looking for, Gugu?"\\
DM:“是的。是的。”咕咕扑棱两下翅膀。 \\ 
DM: "Yes, yes!" Gugu flaps its wings excitedly.\\
莫瑞斯:『没问题!』他用匕首一点点蘸上比尔的刻痕,让这一行字在洞窟中显得分外显眼。 \\ 
Maurice: "No problem!" He uses the dagger to carefully dab a bit of the ink from Bill's carving, making the inscription stand out prominently in the cave.\\
莫瑞斯:然后他把匕首换回去,拿出自己的鲁特琴,随意弹奏慵懒的小调。 \\ 
Maurice: Then he puts the dagger back and takes out his lute, casually playing a lazy tune.\\
\textbf{Ground Truth:} 莫瑞斯\ 表演 \\ 
Maurice Performance\\
\textbf{Prediction:} 比尔: 历史, 莉卡·劳拉瓦: 感知, 莫瑞斯: 手上功夫, 埃尔维斯·泽姆: 表演\\
Bill: History, Lika Lavava: Wisdom, Maurice: Sleight of Hand, Elvis Zem: Performance\\ 
\hline
\end{tabular}
}
    \end{center}
\caption{\label{samples}
Samples of text summarization and keyword generation of CSL-T5.
}
\end{table}
\end{CJK*}





\begin{CJK*}{UTF8}{gbsn}
\begin{table}[!htbp]

	\begin{center}
	\resizebox{\columnwidth}{!}
    {

\linespread{1.16}
\small
\begin{tabular}{|m{13.8cm}|}
\hline
\textbf{Input Context:} 
DM:斯劳格静静地看着这一场闹剧。半晌之后,他挠了挠后脑勺,“我是搞不懂你们这是闹哪一出……不过那个小姑娘的尸体怎么办?就埋在我这里,还是带回沃德伦?”\\
DM: Slaug silently watches this spectacle unfold. After a while, he scratches the back of his head. "I don't quite understand what kind of play you're putting on... But what about the girl's body? Should we bury it here or take it back to Walden?"\\
DM:罗彻用颤颤巍巍的手接过纸笔 随即慢慢退到角落里开始苦思冥想。随后是笔用力划过纸张的声音。\\
DM: Roche takes the pen and paper with trembling hands and slowly retreats to a corner, deep in contemplation. Soon, the sound of the pen forcefully moving across the paper can be heard.\\
埃尔维斯·泽姆:“带回去吧,她是维捷丝的牧师,我想交给同行者可能是一个更好的选择”\\
Elvis Zem: "Let's take her back. She was a priestess of Vitis, and I think handing her over to our fellow journeyers might be a better choice."\\
比尔:“莉卡小姐似乎是希望我们将她的遗体带回她的教堂。”\\
Bill: "Miss Lika seems to wish for us to take her remains back to her church."
莫瑞斯:莫瑞斯说:『她的遗言让我们把遗骨送回神秘女士的神殿。我不太懂这些,不知道最近的神殿在哪?』\\
Maurice: Maurice says, "Her last words instructed us to bring her remains to the Temple of the Mysterious Lady. I'm not quite familiar with this, so I don't know where the nearest temple is."\\
DM:“好。”铁壶发出水烧开的尖利声音,斯劳格转身给你们泡茶。“神殿……反正沃德伦没有。要找神殿你们可能要一路回霍洛塔镇上去了。我记得那里有太阳神的神殿。”\\
DM: "Alright." The kettle emits a sharp sound as the water boils, and Slaug turns around to brew tea for you. "Temples... Well, there's none in Walden, that's for sure. If you're looking for a temple, you'll probably have to go all the way back to the town of Holota. I recall there being a temple dedicated to the Sun God there."\\
比尔:“那可能不太符合要求,不过加入能够回到大路上,后面一切就都好说了。”\\
Bill: "That may not quite meet the requirements, but if we can get back on the main road, everything else will fall into place."\\
莫瑞斯:『如果送回给太阳神神殿,莉卡的灵魂怕是会诅咒我哩。不过倒是可以委托那里的牧师施展保存尸体的神术,再做他想。』\\
Maurice: "If we were to return her to the Temple of the Sun God, Lika's soul might curse me. However, we could entrust the priests there to perform preservation rituals on her body, and then decide what to do next."\\
比尔:比尔看着罗彻在糟蹋纸币,不禁问道,“你在做什么,罗?”\\
Bill: Bill looks at Roche, who is scribbling on the paper, and couldn't help but ask, "What are you doing, Roche?"\\
... (Due to space limitations, several contents are omitted.) \\ 
比尔:“啊。那您还有别的什么地方可以容身吗?”\\
Bill: "Ah, do you have any other place where we could take shelter?"\\
比尔:比尔看了看罗彻。\\
Bill: Bill looks at Roche.\\
莫瑞斯:『马厩……或者类似的地方?只能这样了吧。』\\
Maurice: "Stable... or a similar place? It seems we have no other choice but to go with that."\\
DM:“没了。”回答很干脆。“或者睡外面的狗窝。不过老灰风应该不愿意。”\\
DM: "That's it," the response is straightforward. "Or sleep in the doghouse outside. Although Old Grey Wind probably wouldn't be too happy about that."\\
莫瑞斯:莫瑞斯尝试说服老熊:『就把他扔在客厅就可以。我的同伴会施展预警的法术,只要有人入侵就会发现。他被这样捆着有恶意也很难做什么。』\\
Maurice: Maurice tries to persuade the old bear: "Just leave him in the living room. My companions will cast a warding spell, so we'll be alerted if anyone trespasses. It would be difficult for him to do anything malicious while bound like that."
埃尔维斯·泽姆:“让人睡狗窝也有些太过分了吧。”\\
Elvis Zem: "Making someone sleep in a doghouse is a bit too much..."
\textbf{Ground Truth:} 莫瑞斯: 说服\\
Maurice: Persuasion \\
\textbf{Prediction:} 罗彻: 洞悉, 比尔: 感知, 莫瑞斯: 攻击\\
Bill: Wisdom, Maurice: Escape, Elvis Zem: Wisdom \\ 
\hline
\end{tabular}
    }
    \end{center}

\caption{\label{samples}
Samples of text summarization and keyword generation of CSL-T5.
}
\end{table}

\end{CJK*}





\begin{CJK*}{UTF8}{gbsn}
\begin{table}[!t]
    \begin{center}
	%\resizebox{\columnwidth}{!}
        %{
        \linespread{1.16}
        \small
        \begin{tabular}{|m{13.8cm}|}
            \hline
            \textbf{Input Context:} \\
            \text{[Turn1] }比尔:“好!就差一点!”,我离得远远的在马车上喝彩。\\
            Bill: "Great! Just a little more!" I cheered from afar on the carriage. \\
            \text{[Turn2] }GM:棕熊笨拙地伸出双爪,扑向莫瑞斯。虽然动作僵硬迟缓,但力道绝对不容小觑。\\
            GM: The brown bear clumsily stretches out its paws, lunging at Maurice. Although the movement is stiff and slow, the strength behind it is definitely not to be underestimated. \\
            ...\\
            %\text{[Turn3] }DM:棕熊接着张开血盆大口,朝莫瑞斯是肩膀凶猛地撕咬下去。\\
            %DM: The brown bear then opens its massive, blood-stained mouth, ferociously biting down towards Maurice's shoulder. \\
            \text{[Turn4] }埃尔维斯·泽姆:向前一步,然后镰刀上带着电光砍向棕熊。\\
            Elvis Zem: Takes a step forward, then slashes at the brown bear with his sickle, electricity sparking off its blade. \\
            \text{[Turn5] }莫瑞斯:避开熊爪绕后。\\
            Maurice: Dodges the bear's claw and circles to its back. \\
            \text{[Turn6] }GM: 莫瑞斯想好了他如何敏捷地避开熊熊的爪子绕到背后,然后就被熊熊按在了雪地里。\\
            GM: Maurice imagines how he would gracefully dodge the bear's claws and get to its back, only to find himself pinned to the snow by the bear. \\
            ...\\
            %\text{[Turn7] }DM:莫瑞斯根本抵挡不住棕熊强大的力量,被生生拽进熊的怀抱中。但埃尔维斯此时发难,闪耀着雷电的巨镰轰鸣着划过。\\
            %DM: Maurice can't resist the bear's immense strength and is forcibly pulled into the bear's embrace. But at this moment, Elvis strikes. His enormous sickle, sparking with electricity, roars as it slashes through the air. \\
            \text{[Turn8] }比尔:“哦!法术!”比尔眯了眯眼睛,“电击法术,谁是您的师父,泽姆先生?”\\
            Bill: "Oh! A spell!" Bill squints his eyes, "An electric spell, who was your master, Mr. Zem?" \\
            \text{[Turn9] }埃尔维斯·泽姆:“我的老师只是无名之辈而已,我也只是会些小法术。”\\
            Elvis Zem: "My teacher was a nobody, and I only know a few minor spells." \\
            \text{[Turn10] }GM:棕熊被埃尔维斯的这一击重创,发出痛苦的吼叫...\\ %,镰刃划过的地方散发出烧焦的臭味,血喷涌而出又迅速凝固。\\
            GM: The brown bear is severely injured by Elvis's strike, letting out a painful roar... \\ %The area sliced by the sickle gives off a burnt smell, and blood spurts out, rapidly coagulating. \\
            \text{[Turn11] }莫瑞斯: 疯狂想从熊掌中逃脱(使用逃脱进行擒抱对抗)\\
            Maurice: Desperately tries to escape from the bear's grip (using Escape to counter the grapple). \\
            \textbf{Ground Truth:} 莫瑞斯\ 力量 \\
            Maurice Strength \\
            \textbf{Prediction:} 比尔: 感知, 莫瑞斯: 逃脱, 埃尔维斯·泽姆: 感知 \\ 
            Bill: Wisdom, Maurice: Escape, Elvis Zem: Wisdom \\ 
            \hline
        \end{tabular}
        %}
    \end{center}
\caption{Example of MOE. In the given context, a scenario unfolds where three players find themselves facing a formidable brown bear in combat. Each character actively participates in the battle, except for Bill, who observes from the safety of a carriage. During the encounter, Zem casts a spell; however, it is important to note that the skill check for this particular spell has already been performed after Turn 4 and was explained by the DM in Turn 10. Consequently, the only character currently requiring a skill check is Maurice. Despite his intention to escape from the bear, the DND rule does not include a specific "escape" skill. In such a predicament, Maurice must utilize his strength to resist the bear's attempt to grapple him. As a result, the DM advises him to perform a strength check in adherence to the DND rule. Furthermore, the predicted results from GPT-3.5 utilizing template prompts are ``Character name: Bill, Check skill: Wisdom; Character name: Maurice, Check skill: Escape; Character name: Elvis Zem, Check skill: Wisdom''. The results demonstrate a lack of effective context comprehension and highlight the challenges in understanding complex interactions among agents. }
%Example of CSA. In this context, three players encounter a brown bear and need to fight with it. All characters have movements or participate in the fighting. Bill was just watching the battle and didn't get off the carriage. Zem performed the spell. However, the skill check for the spell has already been done after Turn4 and has been described by DM in Turn10. Thus, olny Maurice needs to make a skill check at this time. He wanted to escape from the bear but there is not escape skill in DND. In this condition, Maurice needs to use his strength to counter the grapple of the bear. Thus, DM will guide him to operate a strength check following DND rule. Moreover, we also show a predicted result of GPT-3.5 and the model failed to understand the contexts and predict correct skill checks.}
\label{tab.data_eg1}
\end{table}
\end{CJK*}