\documentclass[twocolumn,floatfix, nofootinbib,prd,preprintnumbers,showpacs,showkeys,superscriptaddress,longbibliography]{revtex4-1}


%\documentclass[prd, twocolumn, nofootinbib, floatfix]{revtex4-1}
%\documentclass[prd, onecolumn, nofootinbib, floatfix]{revtex4} 
%\setlength{\paperheight}{11in}


\usepackage[mathscr]{euscript}
\usepackage{amsmath}
\usepackage{graphicx}
\usepackage{dcolumn}
\usepackage{bm}
\usepackage{epsfig}
\usepackage{amssymb,latexsym,mathrsfs}
\usepackage{graphicx}
\usepackage{color}
%\usepackage{epstopdf}
\usepackage{hyperref}
\usepackage{float}
\usepackage{diagbox}

\usepackage{tikz}
%\usepackage[compat=1.1.0]{tikz-feynman}

\usepackage{amsmath}
\usepackage{physics}
\usepackage{csquotes} % \textquote{ } -- for quotes
\newcommand*\diff{\mathop{}\!\mathrm{d}}
\newcommand*\Diff[1]{\mathop{}\!\mathrm{d^#1}}

\hypersetup{
    colorlinks=true,
    linkcolor=red,
    citecolor=blue,
} 




\usepackage{amsmath}
\usepackage{amssymb}
\usepackage{subfigure}
\usepackage{hyperref}
\usepackage{url}
\usepackage{xcolor}
\usepackage{color}
%The following defines colors for hyperlinks.
\definecolor{amaranth}{rgb}{0.9, 0.17, 0.31}
\definecolor{purple(munsell)}{rgb}{0.62, 0.0, 0.77}
\definecolor{americanrose}{rgb}{1.0, 0.01, 0.24}
\definecolor{palatinateblue}{rgb}{0.15, 0.23, 0.89}
\definecolor{royalblue(web)}{rgb}{0.25, 0.41, 0.88}
\definecolor{hanpurple}{rgb}{0.32, 0.09, 0.98}
\definecolor{beaublue}{rgb}{0.74, 0.83, 0.9}
\definecolor{carminered}{rgb}{1.0, 0.0, 0.22}
\definecolor{brightpink}{rgb}{1.0, 0.0, 0.5}
\definecolor{vividviolet}{rgb}{0.62, 0.0, 1.0}
\hypersetup{ linktoc=all,
    colorlinks, linkcolor={palatinateblue},
    citecolor={brightpink}, urlcolor={amaranth}}

\newcommand{\changeurlcolor}[1]{\hypersetup{urlcolor=#1}}    


% Making life easier
\newcommand{\be}{\begin{equation}}
\newcommand{\ee}{\end{equation}}
\newcommand{\bs}{\begin{split}} 
\newcommand{\bea}{\begin{eqnarray}}
\newcommand{\eea}{\end{eqnarray}}

\newcommand{\eric}[1]{\textcolor{blue}{[{\bf EL}: #1]}} 
\newcommand{\mike}[1]{\textcolor{red}{[{\bf MG}: #1]}} 
\newcommand{\maksat}[1]{\textcolor{blue}{[{\bf MT}: #1]}} 
\newcommand{\ievlev}[1]{\textcolor{blue}{[{\bf EI}: #1]}}
% useful symbols
\newcommand{\om}{\Omega_m}
\newcommand{\Om}{\Omega_m}
%\newcommand{\ode}{\Omega_{de}}
\newcommand{\ode}{\Omega_{\rm de}}
\newcommand{\oode}{{\mathcal O}(\Omega_{\rm de})}
%\newcommand{\op}{\Omega_\phi}
\newcommand{\lcdm}{$\Lambda$CDM} 
\newcommand{\al}{\alpha} 
\newcommand{\kap}{\kappa} 
\newcommand{\kp}{\kappa}
\newcommand{\eps}{\epsilon}
\newcommand{\event}{Even$t^{-1}$} 


%%%%%%%%Paul's commands
\newcommand{\la}{\langle}
\newcommand{\ra}{\rangle}
\newcommand{\w}{\omega}
\newcommand{\ep}{\epsilon}
\newcommand{\Lam}{\Lambda}
\newcommand{\bes}{\begin{subequations}}
\newcommand{\ees}{\end{subequations}}
\newcommand{\eqn}[1]{(\ref{#1})}
\newcommand{\f}[2]{\frac{#1}{#2}}
\renewcommand{\d}[1]{\ensuremath{\operatorname{d}\!{#1}}}
%%%%%%%%%%%%%%%%%%%%%%%%%%%%%

\newcommand{\Abay}[1]{\textcolor{magenta}{[{\bf AZ}: #1]}}
\newcommand{\xiong}[1]{\textcolor{magenta}{[{\bf XC}: #1]}} 
\newcommand{\aizhan}[1]{\textcolor{magenta}{[{\bf AM}: #1]}} 
% useful symbols


\renewcommand{\d}[1]{\ensuremath{\operatorname{d}\!{#1}}}
\def\nn{\nonumber} 

%%%% my personal commands below
\newcommand{\bo}{\raise-1mm\hbox{\Large$\Box$}} 
\newcommand{\ox}[1]{ \noindent\fbox{\parbox{\columnwidth}{#1 }}}
\newcommand{\e}{\epsilon_0}
\newcommand{\pderiv}[2]{\frac{\partial #1}{\partial #2}}
\newcommand{\y}{\gamma}
\newcommand{\bd}{\boldsymbol}

\newcommand{\blue}[1]{\textcolor{blue}{[#1]}} 
\newcommand{\violet}[1]{\textcolor{violet}{[#1]}} 



\def\rcurs{{\mbox{$\resizebox{.09in}{.08in}{% Figure removed}$}}}
\def\brcurs{{\mbox{$\resizebox{.09in}{.08in}{% Figure removed}$}}}

\def\hrcurs{{\mbox{$\hat \brcurs$}}}
\def\nn{\nonumber} 

\begin{document}





%\title{Fermionic electromagnetic radiation}
\title{Non-thermal photons and a Fermi-Dirac spectral distribution}
\author{Evgenii Ievlev}
\email{evgenii.ievlev@nu.edu.kz}
\altaffiliation[On leave of absence from: ]{National Research Center “Kurchatov Institute”, Petersburg Nuclear Physics
Institute, St.\;Petersburg 188300, Russia}
\affiliation{Physics Department \& Energetic Cosmos Laboratory, Nazarbayev University,\\
Astana 010000, Qazaqstan}
\affiliation{Almaty University of Power Engineering and Telecommunications,\\ 
Almaty 050013, Qazaqstan}
%
\author{Michael R.R. Good}
\email{michael.good@nu.edu.kz}
\affiliation{Physics Department \& Energetic Cosmos Laboratory, Nazarbayev University,\\
Astana 010000, Qazaqstan}
\affiliation{Leung Center for Cosmology and Particle Astrophysics,
National Taiwan University,\\ Taipei 10617, Taiwan}
%\author{Pisin Chen}
%\email{pisinchen@phys.ntu.edu.tw}
%\affiliation{Leung Center for Cosmology and Particle Astrophysics,
%National Taiwan University,\\ Taipei 10617, Taiwan}
%\affiliation{Department of Physics and Graduate Institute of Astrophysics, National Taiwan University,\\ Taipei, 10617, Taiwan}
%\affiliation{Kavli Institute for Particle Astrophysics and Cosmology, SLAC National Accelerator Laboratory, Stanford University,\\ Stanford, California 94305, USA}
%Kabanbay Batyr Ave 53, Nur-Sultan 010000, Qazaqstan}
%\author{Paul C.W. Davies}
%\email{paul.davies@asu.edu}
%\affiliation{Department of Physics and Beyond: Center for Fundamental Concepts in Science,
%Arizona State University, Tempe, Arizona 85287, USA}
%\author{Stephen A. Fulling}
%\email{fulling@math.tamu.edu}
%\affiliation{Department of Physics \& Department of Mathematics, Texas A\&M University, College Station, Texas 77843, USA}



\begin{abstract} 
Although non-intuitive, an accelerated electron along a particular trajectory can be shown to emit classical electromagnetic radiation in the form of a Fermi-Dirac spectral distribution when observed in a particular angular regime. We investigate the relationship between the distribution, spectrum, and particle count. The result for the moving point charge is classical, as it accelerates along an exactly known trajectory.  We map to the semi-classical regime of the moving mirror model with a quantized spin-0 field. The scalars also possess a $\beta$ Bogoliubov coefficient distribution with Fermi-Dirac form in the respective frequency regime. 



\end{abstract} 

\keywords{moving mirrors, black hole evaporation, acceleration radiation, Fermi-Dirac statistics}
\pacs{41.60.-m (Radiation by moving charges), 04.70.Dy (Quantum aspects of black holes)}
%\pacs{04.62.+v, 03.67.Hk, 04.70.-s}
\date{\today} 

\maketitle

%\tableofcontents

%%%%%%%%%%%%%%%%%%%%%%%%%%%%%%%%%%%%%%%%%%%%%
%\tableofcontents
 \section{Introduction}
 It is well-known that bosons obey Bose-Einstein statistics, and fermions obey Fermi-Dirac statistics.  
%In the case of an electric charge following the trajectory a particular trajectory, the spin-1 radiation field emitted by the charge obeys Bose-Einstein statistics, but for a scalar charge the emitted radiation will obey Fermi-Dirac statistics \cite{Nikishov:1995qs,Haro:2008zza}.
%(2) When measuring the spectrum of a scalar field by using a DeWitt detector which follows a uniformly accelerated worldline in Minkowski space-time, one can show that, when the dimension of the spacetime is even the Bose-Einstein statistics is obtained; however, when this dimension is odd the reverse change of statistics occurs (see Takagi for further details).
Interestingly, Haro and Elizalde \cite{Haro:2008zza} found a result for the $\beta$-Bogoluybov coefficient of a semitransparent mirror demonstrating  
a flux of
scalar particles obeying a Fermi-Dirac distribution form in the large $\omega'$ limit.  Nicolaevici \cite{Nicolaevici:2009zz} confirmed the Fermi-Dirac form with respect to the energy $\omega$  but made special note that this does not establish the number of particles since it only applies in the large $\omega'$ limit. In a follow-up, Elizalde and Haro \cite{Elizalde:2010zza} recommended further investigation into the Fermi-Dirac form and its relationship to the sign 
in the $\beta$-Bogoliubov coefficient; in particular, the connection with the number of particles emitted per mode.


Here we investigate the situation using an ordinary moving point charge in classical electrodynamics \cite{Jackson:490457}.  We demonstrate the phenomenon without appealing to quantum field theory; i.e., one does not need to use moving mirrors or semi-transparency to understand the situation.  Nevertheless, we find the perfectly reflecting accelerating boundary corresponding to the moving point charge and examine its spectral statistics for clarity.  

The functional mapping between moving mirrors and moving point charges \cite{Ford:1982ct,Unruh:1982ic,Ritus:2003wu,Ritus:2002rq,Ritus:1999eu,Nikishov:1995qs,Zhakenuly:2021pfm,Ritus:2022bph} is leveraged to understand the problem in both contexts.  The situation has different physical meanings when examined for the classical electromagnetic field or quantized scalar field \cite{Ievlev:2023inj,Ievlev:2023bzk}; and thus, different implications for the classical radiation in ordinary 3+1 dimensions and quantum radiation in 1+1 dimensions. 

We use natural units, setting $\hbar = c = k_B = \mu_0 = 1$; the electron's charge is then a dimensionless number $e^2=4 \pi \alpha_{\textrm{fs}} \approx 0.092$. 
%The polar angle $\theta$ runs from $0$ ($z>0$) to $\pi$ ($z<0$).







\section{Fermi-Dirac Trajectory}
\label{sec:FD}


\subsection{Dynamics and total energy}
Let us start with a simple illustration of the situation.  Consider an electron moving in a straight line along the $z$-axis.
We take the trajectory defined implicitly as %\mike{maybe fix dimensions on log argument}  
\be t(z) = \frac{\kappa}{4} z^2 + \frac{2}{\kappa} \ln ( \kappa z ) + z \zeta,\label{eom}\ee
%\mike{positive $z$ and figure} 
where $\kappa > 0$ is  the acceleration scale and $-1 < \zeta < 1$.
The inverse velocity along the trajectory and the maximum velocity are, respectively,
\begin{equation}
    \frac{1}{v} = \frac{\diff t(z)}{\diff z} = \frac{\kappa  z}{2}+\frac{2}{\kappa  z} + \zeta \,, \quad
    v_\text{max} = \frac{1}{2 + \zeta} \,.
\label{velocity}
\end{equation}
%\be \frac{1}{v} = \frac{\diff t(z)}{\diff z} = \frac{\kappa  z}{2}+\frac{2}{\kappa  z} + \zeta.\label{eom-v}\ee
From Eq.~\eqref{velocity}, it is evident that for $\zeta > -1$ this trajectory travels along a time-like, relativistic worldline.
This trajectory is asymptotically static; see Fig. \ref{FDspacetime} for a spacetime diagram and Fig. \ref{FDpenrose} for a Penrose diagram.
% Figure environment removed

% Figure environment removed
The total energy emitted can be calculated with the Larmor formula; this energy is finite for $\zeta > -1$.
For example, when $\zeta = 0$ it takes the analytic form:
%\be \frac{E}{e^2} = \frac{\kappa}{192}-\frac{\kappa}{72\pi}.\ee
\be E = \frac{e^2\kappa}{36}\left(\frac{1}{3\sqrt{3}} - \frac{1}{4\pi}\right).\label{totenergy}\ee
For other values of the parameter $\zeta$, the analytic formula for the total energy exists but is complicated; nevertheless, it is simple to illustrate numerically, see Fig.~\ref{fig:TotalEnergy}.
%
% Figure environment removed
%
One can use the total energy, e.g. Eq.~(\ref{totenergy}), to check the consistency of the spectral results (see the Appendix for detail).  
% One way to derive Eq.~(\ref{totenergy}) without using spectral analysis is to integrate the Larmor power, $P = \alpha^2/6\pi$ over time $t$, where $\alpha(t)$ is the proper acceleration.


\subsection{Spectral distribution of the accelerating electron's radiation}
To find the spectral distribution for Eq.~(\ref{eom}), we use the standard approach in classical electrodynamics \cite{Jackson:490457}. Here the energy $E$ can be found by the spectrum $I(\omega)$, or spectral distribution $\diff{I}/\diff{\Omega}$ 
 \cite{Schwinger1949},
\be E = \int \diff{\omega} I(\omega) = \int \diff{\omega} \int \diff{\Omega} \frac{\diff{I(\omega)}}{\diff{\Omega}}.\ee
For example, by the use of Eq. 13 in \cite{Ievlev:2023inj}, 
\begin{equation}
	\frac{\diff I(\omega)}{\diff \Omega} = \frac{e^2\omega^2}{16\pi^3}\left|\sin\theta \int\displaylimits_{0}^{\infty} \diff z  e^{i \phi(z)}  \right|^2,
\label{I_phi_int}
\end{equation}
where $\phi = \omega (t - z\cos\theta)$, we write
\begin{equation}
	\frac{1}{\sin^2\theta}\frac{\diff I(\omega)}{\diff \Omega} = \frac{e^2\omega^2}{16\pi^3}\left| \int\displaylimits_{0}^{\infty} \diff z  e^{i \phi(z)}  \right|^2.
\label{I_phi_int1}
\end{equation}
%
This integral can be solved exactly as is (see the Appendix), but for simplicity, consider the spectral distribution at a particular angle $\theta_0$ instead.  Specialize to $\theta \to \theta_0$ such that the phase is 
\be \phi(z) = \frac{\kappa \omega}{4}z^2 + \frac{2}{\kappa} \omega \ln z + \omega z (\zeta  - \cos\theta_0).\label{phase}\ee
It is now straightforward to integrate Eq.~(\ref{I_phi_int1}) when $\cos \theta_0 = \zeta$.  One obtains the spectral distribution,
\begin{equation}
	\frac{1}{\sin^2\theta_0}  \left.\frac{\diff I(\omega)}{\diff \Omega}\right|_{\theta_0} = \frac{e^2 }{8\pi^2}\frac{\omega/\kappa}{e^{2\pi \omega/\kappa}+1},
\label{I_phi_int2}
\end{equation}
where the spectral frequency content of the radiation has a Fermi-Dirac form. Before examining this form and its relationship to its particle spectrum $N(\omega)$, let us first look at its quantum dual in the moving mirror model \cite{DeWitt:1975ys,Davies:1976hi,Davies:1977yv}, in the spirit of previous moving mirror studies on the Fermi-Dirac result \cite{Haro:2008zza,Nicolaevici:2009zz,Elizalde:2010zza}.



\subsection{Corresponding Bogolubov Coefficients }

While the above result is classical radiation from the 3+1 dimensional electromagnetic field, we can investigate the quantum radiation from the 1+1 dimensional scalar field of the moving mirror model. 
%The trajectory can be readily put in the product log form,
%\be z(t) = \frac{2}{\kappa} \sqrt{W\left(\frac{1}{4} \kappa ^2 e^{\kappa  t}\right)}.\ee
The mapping recipe \cite{Ievlev:2023inj} between electron and mirror links the spectral distribution on the electron side and the Bogolubov coefficient squared on the mirror side, c.f. \cite{Ievlev:2023bzk}:
\begin{equation}
\begin{aligned}
	&\frac{\diff{I}}{\diff{\Omega}}(\omega,\cos\theta) = \frac{e^2 \omega^2}{4\pi} |\beta_{pq}|^2, \\
	&p + q = \omega \,, \quad p - q = \omega \cos\theta. \\
	%&p = \omega \frac{1 + \cos\theta }{2} \,, \quad 	q = \omega \frac{1 - \cos\theta }{2}
\end{aligned}
\label{recipe_dIdOmega_from_mirror}
\end{equation}
%
Using the full spectral distribution from Eq.~\eqref{exact} one can use this recipe to obtain the corresponding Bogolubov coefficients.
Let us however consider setting $\theta = \theta_0$, $\zeta =\cos\theta_0$.
In this case, the electron's spectral distribution has a simple form Eq.~\eqref{I_phi_int2}.
In terms of the scalar frequencies $p,q$, the corresponding condition reads
\begin{equation}
    \theta = \theta_0 \,, \ \zeta =\cos\theta_0
    \Longleftrightarrow
    p = \omega \frac{1 + \zeta}{2} \,, \ q = \omega \frac{1 - \zeta}{2}
\label{theta_condition_pq_correspondence}
\end{equation}
%
So, the scalar frequencies $p$ and $q$ are not independent, but related to each other through this condition.
The recipe Eq.~\eqref{recipe_dIdOmega_from_mirror} gives particular beta Bogolubov coefficients:
\begin{equation}
    |\beta_{pq}|^2 = \frac{1-\zeta^2}{2\pi (p+q) \kappa}\frac{1}{e^{2\pi  (p+q)/\kappa }+1} \,, \quad
    \frac{p}{q} \equiv \frac{1 + \zeta}{1 - \zeta} \,.
\label{pandq}
\end{equation}
%\begin{equation}
%    |\beta_{pq}|^2 \Bigg|_{ \substack{ p = \omega \frac{1 + \zeta}{2} \\ q = \omega \frac{1 - \zeta}{2} } }
%        %= \frac{4\pi}{e^2 \omega^2}\left.\frac{\diff{I}}{\diff{\Omega}}\right|_{\theta_0} 
%        = \frac{1-\zeta^2 }{2\pi \omega \kappa}\frac{1}{e^{2\pi  \omega/\kappa }+1}
%\end{equation}
%%
%We remind the reader that the scalar frequencies are $(p,q)$ rather\footnote{Recall that $(p,q)$ substitute for the usual notation of $(\omega,\omega')$ so as not to confuse with the photon frequency which already claims the symbol $\omega$.} than the photon frequency $\omega$.    
%This results in
%\be |\beta_{pq}|^2 = \frac{1-\zeta^2}{2\pi (p+q) \kappa}\frac{1}{e^{2\pi  (p+q)/\kappa }+1}, \label{pandq} \ee
%
%For simplicity we can set $\zeta = 0$ (which is the electron's $\cos\theta_0 \to 0$ angular regime), then $p\sim q$:
%\be |\beta_{pq}|^2  = \frac{1 }{4\pi p \kappa}\frac{1}{e^{4\pi  p/\kappa }+1}.\ee
% This demonstrates the Fermi-Dirac form for the $\beta$-Bogolubov coefficients of the quantum scalars. Notice this result is only applicable in the regime $p\sim q$ or $\theta_0 \sim \pi/2$, to the side of the electron's rectilinear motion. 
%demonstrating the Fermi-Dirac form for the $\beta$-Bogolubov coefficients of the quantum scalars.  
%
This result demonstrates the Fermi-Dirac form for the $\beta$-Bogolubov coefficients of the quantum scalars.  

By tuning the value of $\zeta$ one can obtain a trajectory that gives the Fermi-Dirac at any pre-assigned angle (or a bespoke frequency regime).
For instance, 
the `high-frequency' regime \cite{Hawking:1974sw}, $p \sim 0$ ($q\gg p$), corresponds to $\zeta \sim -1$; 
Eq.~(\ref{pandq}) becomes to leading order
\be |\beta_{pq}|^2 = \frac{1+\zeta}{\pi q \kappa}\frac{1}{e^{2\pi  q/\kappa }+1} \,. \label{betaFD}\ee
%corresponding to the regime $\zeta \sim -1$, or $p-q = -w$ which along with $p+q = \omega$ gives $p \sim 0$ ($q\gg p$).  
Using the duality to map back to the electron, the choice $\zeta = \cos\theta_0 = -1$ corresponds to a viewpoint behind the accelerating electron $\theta_0 \sim \pi$. 
% which corresponds to choosing $\zeta = \pm 1$, or $p-q = \pm w$ which along with $p+q = \omega$ gives $q \sim 0$ or $p \sim 0$, respectively.  Using the duality to map back to the electron, the choice $\zeta = \pm 1$ corresponds to a viewpoint in front of the electron $\theta_0 \sim 0$ or behind the electron $\theta_0 \sim \pi$, respectively. 
 %Where the temperature of the scalars are
 %\be T = \frac{\kappa}{2\pi} s.\ee

\subsection{Connection to Particle Count}
The notion of discrete radiation energy $\hbar \omega$ 
%in  classical electrodynamics 
allows an introduction of a 
%continuous 
particle spectrum $N(\omega)$. The connection between the spectral distribution of electromagnetic waves and the particle spectrum is \cite{Jackson:490457}: 
\be N(\omega) = \frac{1}{\omega} I(\omega) = \frac{1}{\omega} \int d\Omega \frac{\diff{I}}{\diff{\Omega}},\label{classicalparticles}\ee
which must be consistent with the total energy emission as computed by the spectral distribution,
\be E = \int \diff{\omega} \; \omega N(\omega) = \int \diff{\omega} \int \diff{\Omega} \frac{\diff{I}}{\diff{\Omega}}.\ee
Therefore, the Fermi-Dirac distribution does not correspond to the particle spectrum $N(\omega)$, or the energy spectrum $I(\omega)$. 
It does not even correspond to the spectral distribution $\diff{I}/{\diff{\Omega}}$ at an arbitrary observation angle $\theta$; 
but only in a specific angular regime $\theta \to \theta_0$: $\left.\diff{I}/{\diff{\Omega}}\right|_{\theta_0}$ using the corresponding trajectory, Eq.~(\ref{eom}).  Nevertheless, the interesting question remains if it is possible to observe such radiation measured in such a specified angular regime $\theta_0$.  Does an observer see Fermionic electromagnetic radiation? Is the spectral content congruent with Fermi-Dirac statistics?

We stress again that the trajectory is easily generalized in a number of different ways to illustrate the robust Fermi-Dirac form of the spectral distribution at particular angles.  For instance, a particular choice of $\zeta$ in Eq.~(\ref{eom}) results in a new trajectory form, capable of a new observation angle $\theta_0$, which gives the Fermi-Dirac result. This means, depending on the particular bespoke trajectory of interest, the relevant zeta-angle $(\zeta,\theta_0)$ could be in any desired direction, such as to the side $ (0,\pi/2)$, in front $(+1,0)$, or behind $(-1,\pi)$ the accelerating electron. 







\section{Discussion}
The particular physics of this result depends in subtle ways on dimension (3+1 vs 1+1), source (electron vs. mirror), and regime (angle vs. frequency). 
Two other notable examples in the literature confirm this, namely, scalar charges and even-odd dimensional dependence. 
Let us consider these examples now.

For an example of a source other than an electric charge or mirror, Nikishov and Ritus found \cite{Nikishov:1995qs} that in the case of a scalar charge, the emitted scalar radiation along a particular trajectory will obey Fermi-Dirac statistics; in contrast to an electric charge following the same particular trajectory whose spin-1 radiation field obeys Bose-Einstein statistics.  

As an example of dimensional dependence, scalar field radiation measured by a uniformly accelerated DeWitt detector obeys Bose-Einstein statistics when the dimension of the spacetime is even, but when the dimension is odd, one obtains Fermi-Dirac statistics \cite{Takagi:1986kn}.

Taken as a whole, it is clear the subtleties involved make it especially important to precisely define the context and the regime of applicability and explicitly examine the form of the computed observables. 

We note that Hawking radiation \cite{Hawking:1974sw} and its Schwarzschild moving mirror analog \cite{Good:2016oey} utilize the high-frequency regime lending support to the notion of thermality and Bose-Einstein distributed scalars at late-times. 
However, in the Schwarzschild case (as opposed to extremal cases \cite{good2020extreme} or asymptotically inertial situations \cite{Good:2019tnf}), there is no finite total energy check corresponding to the Bogoliubov coefficients. Moreover, the total particle count is infinite. This is ultimately due in part to the horizon; and in the analog moving mirror situation, the fact that the proper acceleration is asymptotically infinite. The same goes for the eternal black hole analog of Carlitz-Willey \cite{carlitz1987reflections}. 

In this work, we have shown that the scalars can possess Fermi-Dirac distributed $\beta$-Bogliubov coefficients.  If one considers $\beta$-Bogoliubov coefficients sufficient evidence of a thermal Bose-Einstein distribution for the Schwarzschild or Carlitz-Willey trajectories; then the result Eq.~(\ref{betaFD}) is also sufficient evidence of a thermal Fermi-Dirac distribution for the trajectories Eq.~(\ref{eom}). Moreover, we have an additional check of total finite energy and finite particle emission (see the Appendix for more detail).   

This result demonstrates that, although the high-frequency approximation (or the low-frequency approximation, if one prefers) is frequently used in the literature, one should clearly understand whether this approximation represents the physical system under consideration.
The result Eq.~\eqref{betaFD} does not reveal e.g. the particle spectrum, as the high-frequency region does not dominate the corresponding contribution from the beta Bogolubov coefficients.
In other words, one should be careful when applying the high-frequency (or the low-frequency) approximation; in each case, this approximation should be well-motivated; otherwise, peculiar results may arise.




\section{Conclusion}
We have shown that moving point charge radiation can possess a Fermi-Dirac spectral distribution form.  A particular trajectory and corresponding angular regime demonstrate the result.  By appealing to classical electrodynamics, we have analyzed the physical reason for the resulting unexpected spectral-statistics form.  

The spectral-statistics (as explicitly derived from the spectral distribution in a particular angular regime for the radiation from a moving point charge) do not necessarily characterize the spin-statistics of the electromagnetic field in question.  Instead, they depend crucially on the observation angle and the specific electron trajectory interaction with the radiation field.





\section{Acknowledgements} 
Funding comes in part from the FY2021-SGP-1-STMM Faculty Development Competitive Research Grant No. 021220FD3951 at Nazarbayev University.   

\appendix

\section{Partial contribution from the FD particles}

The partial energy contribution when $\zeta = \cos\theta_0$ can be found from the Fermi-Dirac form Eq.~(\ref{I_phi_int2}) of the electron,
\begin{equation}
    E^{fd}_\textrm{electron} 
        = \int_0^\infty \diff{\omega} \int_0^{2\pi} \diff{\varphi} \left.\frac{\diff I(\omega)}{\diff \Omega}\right|_{\theta_0} 
        = \frac{e^2\kappa (1 - \zeta^2) }{192\pi}
\label{partial_energy_electron}
\end{equation}
%
The corresponding contribution from the mirror can be derived from this result in the following way. 
We insert into Eq.~\eqref{partial_energy_electron} the integral
$1 = \int_{-1}^{1} \diff (\cos\theta) \, \delta( \cos\theta - \zeta ) $ 
and change the variables according to Eq.~\eqref{recipe_dIdOmega_from_mirror}. The Jacobian is 
$2 / (p + q)$, and the resulting contribution is
\begin{equation}
\begin{aligned}
    &E^{fd}_\textrm{mirror}  \\
        &= \int_0^\infty \diff{p}\int_0^\infty \diff{q}\; 
            \delta\left( \frac{p-q}{p+q} - \zeta \right) \,
            (p + q) |\beta_{pq}|^2 \\
        &= \frac{\kappa (1 - \zeta^2) }{192\pi}.
\end{aligned}
\end{equation}
%
This gives the partial contribution to the energy emitted to both sides of the corresponding mirror, counting only the FD particles.


Let us look at how the analogy extends to finite particle count.  
For the moving mirror, the total number of scalars emitted to the right side of the mirror is:
\begin{equation}
\begin{aligned}
    &N_\textrm{mirror} \\
    &= \int_0^\infty \diff{p}\int_0^\infty \diff{q}\; 
        \delta\left( \frac{p-q}{p+q} - \zeta \right) \,
        |\beta_{pq}|^2 \\
    &= \frac{(1 - \zeta^2) \ln 2 }{8\pi^2} \,.
\end{aligned}
\label{bo}
\end{equation}
%
%\be N_\textrm{mirror} = \frac{1}{2}\int_0^\infty \diff{p}\int_0^\infty \diff{q}\; |\beta_{pq}|^2 = \frac{\ln 2 (1 - \zeta^2)}{8\pi^2}.\label{bo}\ee
%
For the case of the accelerating electron, we may integrate over all frequencies $\omega$ on Eq.~(\ref{classicalparticles}) which gives
\begin{equation}
\begin{aligned}
    \int_0^\infty \diff{\omega}\; N(\omega) 
        &= \int_0^\infty \frac{\diff{\omega}}{\omega} \int_0^{2\pi} \diff{\psi} \left.\frac{\diff I(\omega)}{\diff \Omega}\right|_{\theta_0},\\
        &= e^2\frac{(1 - \zeta^2) \ln 2 }{8\pi^2},
\end{aligned}
\end{equation}
in agreement with Eq.~(\ref{bo}). This highlights the dual consistency between the mirror and electron but also demonstrates the advantage of a finite energy and finite particle count with an exact analytic solution, Eq.~(\ref{eom}).  

     
%\newpage
%\begin{widetext}

\section{Exact spectrum}

To demonstrate the difference between a particular choice of observation angle and the spectrum $I(\omega)$ which results from integration over the solid angle, we briefly look at an exact answer for the integral of Eq.~(\ref{I_phi_int1}), setting $\zeta = 0$ and leaving $\theta$ unset.  Integrating Eq.~(\ref{I_phi_int1}) gives
\begin{equation}
\begin{aligned}
%	&\frac{1}{\sin^2\theta}\frac{\diff I(\omega)}{\diff \Omega} =\frac{e^2 \omega  e^{-\frac{\pi  \omega }{\kappa }}}{16 \pi ^3 \kappa} \times \\
	&\frac{\diff I(\omega)}{\diff \Omega} =\frac{e^2  \omega \sin^2\theta }{16 \pi ^3 \kappa}  e^{-\frac{\pi  \omega }{\kappa }} \times \\
	&\times \left| \Gamma \left(\frac{1}{2}-\frac{i \omega }{\kappa }\right)A+2 \cos \theta \sqrt{\frac{i \omega}{\kappa}} \Gamma \left(1-\frac{i \omega }{\kappa }\right)B\right|^2 \,,
\end{aligned}
\label{exact}
\end{equation}
%
where 
\begin{equation}
\begin{aligned}
	A &= \, _1F_1\left(\frac{1}{2}-\frac{i \omega }{\kappa };\frac{1}{2};\frac{i \omega \cos^2\theta }{\kappa }\right) \,, \\
	B &= \, _1F_1\left(1-\frac{i \omega }{\kappa };\frac{3}{2};\frac{i\omega \cos^2\theta  }{\kappa }\right) \,.
\end{aligned}
\end{equation}
%
This spectral distribution, Eq.~(\ref{exact}), gives the total energy, Eq.~(\ref{totenergy}) by numerical integration,
\begin{equation}
\begin{aligned}
	E &= \int_0^{\infty} \diff{\omega} \int_0^{2\pi}\diff{\phi}\int_0^\pi \diff{\theta}\sin\theta \frac{\diff I(\omega)}{\diff \Omega} \\
	&= \frac{e^2\kappa}{36}\left(\frac{1}{3\sqrt{3}} - \frac{1}{4\pi}\right) \,.
\end{aligned}
\end{equation}
%
We cannot integrate Eq.~(\ref{exact}) exactly over the solid angle to obtain an analytic form of $I(\omega)$. However, it is clear that Eq.~(\ref{exact}) will not result in a Fermi-Dirac form for the spectrum $I(\omega)$, even though at $\theta_0 =\pi/2$, Eq.~(\ref{exact}) gives 
\begin{equation}
	\left.\frac{\diff I(\omega)}{\diff \Omega}\right|_{\theta_0} = \frac{e^2 }{8\pi^2}\frac{\omega/\kappa}{e^{2\pi \omega/\kappa}+1},
\end{equation}
which is the result Eq.~(\ref{I_phi_int2}).  For this reason, one cannot say the particle count, $N(\omega)$ rests in a Fermi-Dirac distribution, which is ultimately consistent with the horizonless globally defined motion, Eq.~(\ref{eom}), evolving to an asymptotic stop.

%\end{widetext}
%\twocolumngrid
%%%%%%%%%%%%%%%%%%%%%%%%%%%%%%%%%%%%% References in order
\bibliography{main} 
\end{document}

%%%%%%%%%%%%%%%%%
%%%%%%%%%%%%%%%%%
%%%%%%%%%%%%%%%%%
%%%%%%%%%%%%%%%%% Notes from March 4, 2019 email
%%%%%%%%%%%%%%%%% Large_n_2D_version
\subsection{Fourier Spectra}

We wish to specify a canonical spectrum of radiation for electron emission applications.  In particular, we want to know how the power is distributed over frequency.  To do this, first we introduce the Fourier transform of the proper acceleration of the charged particle through the relativistic Fourier transform pair,
\be \alpha(t) = \frac{1}{\sqrt{2\pi}} \int_{-\infty}^{\infty} \alpha(q) e^{-iq t} \diff q,\ee
\be \alpha(q) = \frac{1}{\sqrt{2\pi}} \int_{-\infty}^{\infty} \alpha(t) e^{iq t} \diff t.\ee
with conjugate variables $t$ and $q$. Then we note, that use of the Parseval-Plancherel theorem, relates the pair via
\be \int_{-\infty}^{\infty} |\alpha(t)|^2 \diff t = \int_{-\infty}^{\infty} |\alpha(q)|^2 \diff q .\ee
Since $\alpha(t)$ is real, we may use the fact that the second integral above is even, and apply the relativistic Larmor power to write
\be E = \int_{-\infty}^{\infty} \frac{\diff E}{\diff t} \diff{t} = \int_0^\infty \frac{\diff E}{\diff q} \diff q,\ee
where
\be E = \int_{-\infty}^{\infty} P \diff t = \frac{2}{6\pi} \int_0^{\infty} |\alpha(q)|^2 \diff q.\ee
Then it suffices to consider the energy per unit bandwidth $q$ as
\be \frac{\diff E}{\diff q} =  \frac{1}{3\pi}|\alpha(q)|^2,\ee
which we call the Fourier spectrum.  This characterizes the energy emitted by the electron during time $t$ as a function of independent variable $q$. 
Interestingly, the Fourier transform of $\alpha(t)$ is analytic. However, the expression is lengthy. We include it here for completeness with $\kappa =1$,
\be \alpha(q) = \sqrt{\frac{8}{\pi }}\frac{i \gamma ^3 s (A -A^*)}{\left(q^2+4\right) \left(q^2+16\right)},\ee
where $ A = B + C$ such that $B = (q-4 i) \left(q^2+4\right) X$, and $C = (q-2 i) \left(q^2+16\right)Y$. Here $X$ and $Y$ are Appell hypergeometric functions, 
\be X \equiv  F_1\left(2-\frac{i q}{2};\frac{3}{2},\frac{3}{2};3-\frac{i q}{2};\frac{s+1}{s-1},\frac{s-1}{s+1}\right),\ee
and
\be Y \equiv  F_1\left(1-\frac{i q}{2};\frac{3}{2},\frac{3}{2};2-\frac{i q}{2};\frac{s+1}{s-1},\frac{s-1}{s+1}\right). \ee
\ievlev{delete?}
% Figure environment removed


\subsection{Notes: peel acceleration}
Here we present a few expressions for peel acceleration, $\bar{\kappa} = 2\pi T$, expressed as local temperature, $T$, which is a good probe for thermal radiation emitted by the moving point charge. In terms of the $p(u)$ trajectory:
\be T = \frac{1}{2\pi} \partial_u \ln p'(u) = \frac{1}{2\pi} \frac{p''(u)}{p'(u)}.\ee
When considering the scalar peel acceleration, one may neglect any sign out in front.  This has the additional benefit of keeping temperature positive definite at will.  In terms of the $f(v)$ trajectory, where $v=t+x$ is coordinate advanced time,
\be T = \frac{1}{2\pi} \frac{1}{f'(v)} \partial_v \ln \frac{1}{f'(v)} = \frac{-1}{2\pi} \frac{f''(v)}{f'(v)^2}.\ee
In terms of velocity $\dot{x} = v$ (no longer coordinate advanced time):
\be T = \frac{1}{2\pi} \frac{1}{1-v} \partial_t \ln \frac{1+v}{1-v} = \frac{1}{2\pi} \frac{1}{v}\frac{1}{1-v} \partial_x \ln \frac{1+v}{1-v},\ee
which spelled out is 
\be T = \frac{\dot{v}}{\pi(1-v)^2(1+v)}.\ee
In terms of celerity, and Lorentz factor,
\be T = \frac{\alpha}{\pi}(\gamma + w).\ee
In terms of the rapidity or proper acceleration:
\be T = \frac{\eta'(u)}{\pi} = \frac{\alpha }{\pi}e^\eta.\ee
It should be written down that the total energy is
\be E = \frac{\pi}{6}\int_{-\infty}^{+\infty} \left(\frac{T}{e^\eta}\right)^2 \diff t.\ee
A couple of helpful relations:
\be e^{\pm 2\eta} = \frac{1\pm v}{1\mp v}   = (\gamma \pm w)^2.\ee
\subsection{Power as measured far away}
We can use 
\be \eta(u) = \frac{1}{2} \ln p'(u).\ee
As well as
\be \alpha^2 = e^{-2 \eta} \left(\frac{\diff{\eta}}{\diff{u}}\right)^2.\ee
and
\be \bar{P} = \frac{\diff{E}}{\diff{u}} = \frac{\alpha^2}{6\pi}\frac{1}{1-\tanh\eta}.\ee


For contrasting dynamics and to understand the ultra-relativistic regime, consider the left-side (blue) trajectory $s=1$ Davies-Fulling equation of motion:
\begin{equation}
z(t)=-\dfrac{1}{\kappa}\ln[\cosh(\kappa t)]\label{leom}.
\end{equation}
The physics is different for this trajectory compared to its sub-light speed counterpart, Eq.~(\ref{eom}). However, there are some similarities. The corresponding accelerations, for $\kappa u > -\ln 2$ and $\kappa v<\ln 2$, are in coordinate time $t$ and proper time $\tau$, are
\begin{equation}
\alpha(t)=-\kappa\cosh(\kappa t), ~~~\quad
\alpha(\tau)=-\kappa\sec(\kappa \tau),
\end{equation}
while the corresponding ray-tracing functions and transcendental inversions give light-cone expressions:
\begin{equation}
p(u)=\dfrac{1}{\kappa}\ln(2-e^{-\kappa u}),~~~\quad
f(v)=-\frac{1}{\kappa}\ln\left(2-e^{\kappa v}\right).
\end{equation}
Asymptotic expansion of the acceleration around $\tau=\frac{\pi}{2\kappa}$ yields,
\begin{equation}
\alpha(\tau)=\frac{1}{\tau-\frac{\pi}{2\kappa}}+O(\tau-\frac{\pi}{2\kappa}),\label{DFacceleration}
\end{equation}
similar to the eternal thermal emission of the Carlitz-Willey trajectory \cite{carlitz1987reflections} proper acceleration in terms of proper time, $\alpha = 1/\tau$ but horizon shifted from $\tau=0$ to $\tau = \pi/2\kappa$. Another insightful way to understand the proper acceleration is with space as the independent variable, $\alpha(z)=\gamma'(z)$, where $\gamma(z)$ is the Lorentz factor in terms of coordinate space $z$, which gives,
\begin{equation}
\alpha(z)=-\kappa e^{-\kappa z}.
\end{equation}
In particular, we point out that the peel acceleration in the  high speed limit, $s\to 1$, using the well-suited light-cone coordinates, has an advanced time horizon in play, $v_H = \frac{\ln 2}{\kappa}$, where asymptotic inertia no longer holds, i.e. $\alpha(t) \to \infty$ as $t\to \infty$ is approached.  Thus, as the trajectory advances to the horizon, $v\to \frac{\ln 2}{\kappa}$, the peel acceleration, $\kappa(v) \to \kappa$ goes to a constant, which signals exact thermal emission in the far future for the left moving (and left-side) trajectory, Eq.~(\ref{leom}).

All this is to say, we expect from the peel acceleration, $\kappa$, that the photons emitted from the electron travelling along the Davies-Fulling trajectory to be Planck-distributed when the electron moves fast and to the left.  More precisely, at late times $t\to +\infty$ when moving to the left when $z<0$ after $t>0$ (upper left quadrant) or vice versa we expect thermality at early times $t\to -\infty$ when the electron moves to the left when $z>0$ before $t<0$ (bottom right quadrant). 