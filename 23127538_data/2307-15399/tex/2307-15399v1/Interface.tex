\documentclass[showpacs,amsmath,superscriptaddress,reprint,aps]{revtex4-2}

\usepackage{graphicx}
\usepackage{hyperref}

\begin{document}

\title
{Room-temperature quantum emission from interface excitons in mixed-dimensional heterostructures}
\author{N.~Fang}
\email[Corresponding author. ]{nan.fang@riken.jp}
\affiliation{Nanoscale Quantum Photonics Laboratory, RIKEN Cluster for Pioneering Research, Saitama 351-0198, Japan}
\author{Y.~R.~Chang}
\affiliation{Nanoscale Quantum Photonics Laboratory, RIKEN Cluster for Pioneering Research, Saitama 351-0198, Japan}
\author{S.~Fujii}
\affiliation{Quantum Optoelectronics Research Team, RIKEN Center for Advanced Photonics, Saitama 351-0198, Japan}
\affiliation{Department of Physics, Keio University, Yokohama 223-8522, Japan}
\author{D.~Yamashita}
\affiliation{Quantum Optoelectronics Research Team, RIKEN Center for Advanced Photonics, Saitama 351-0198, Japan}
\affiliation{Platform Photonics Research Center, National Institute of Advanced Industrial Science and Technology (AIST), Ibaraki 305-8568, Japan}
\author{M.~Maruyama}
\affiliation{Department of Physics, University of Tsukuba, Ibaraki 305-8571, Japan}
\author{Y.~Gao}
\affiliation{Department of Physics, University of Tsukuba, Ibaraki 305-8571, Japan}
\author{C.~F.~Fong}
\affiliation{Nanoscale Quantum Photonics Laboratory, RIKEN Cluster for Pioneering Research, Saitama 351-0198, Japan}
\author{D.~Kozawa}
\affiliation{Nanoscale Quantum Photonics Laboratory, RIKEN Cluster for Pioneering Research, Saitama 351-0198, Japan}
\affiliation{Quantum Optoelectronics Research Team, RIKEN Center for Advanced Photonics, Saitama 351-0198, Japan}
\affiliation{Research Center for Materials, National Institute for Materials Science, Ibaraki 305-0044, Japan}
\author{K.~Otsuka}
\affiliation{Nanoscale Quantum Photonics Laboratory, RIKEN Cluster for Pioneering Research, Saitama 351-0198, Japan}
\affiliation{Department of Mechanical Engineering, The University of Tokyo, Tokyo 113-8656, Japan}
\author{K.~Nagashio}
\affiliation{Department of Materials Engineering, The University of Tokyo, Tokyo 113-8656, Japan}
\author{S.~Okada}
\affiliation{Department of Physics, University of Tsukuba, Ibaraki 305-8571, Japan}
\author{Y.~K.~Kato}
\email[Corresponding author. ]{yuichiro.kato@riken.jp}
\affiliation{Nanoscale Quantum Photonics Laboratory, RIKEN Cluster for Pioneering Research, Saitama 351-0198, Japan}\affiliation{Quantum Optoelectronics Research Team, RIKEN Center for Advanced Photonics, Saitama 351-0198, Japan}


\begin{abstract}
The development of van der Waals heterostructures has introduced unconventional phenomena that emerge at atomically precise interfaces. For example, interlayer excitons in two-dimensional transition metal dichalcogenides show intriguing optical properties at low temperatures. Here we report on room-temperature observation of interface excitons in mixed-dimensional heterostructures consisting of two-dimensional tungsten diselenide and one-dimensional carbon nanotubes. Bright emission peaks originating from the interface are identified, spanning a broad energy range within the telecommunication wavelengths. The effect of band alignment is investigated by systematically varying the nanotube bandgap, and we assign the new peaks to interface excitons as they only appear in type-II heterostructures. Room-temperature localization of low-energy interface excitons is indicated by extended lifetimes as well as small excitation saturation powers, and photon correlation measurements confirm single-photon emission. With mixed-dimensional van der Waals heterostructures where band alignment can be engineered, new opportunities for quantum photonics are envisioned. 
\end{abstract}

\maketitle
\section{Introduction}
The discovery of van der Waals (vdW) materials, including two-dimensional (2D) transition metal dichalcogenides (TMDs) and graphene, has brought about a revolution in the assembly of artificial heterostructures by allowing for the combination of two different materials without the constraints of lattice matching. Such an unprecedented level of flexibility in heterostructure design has led to the emergence of novel properties not seen in individual materials. A prime example is twisted bilayer graphene at magic angles, which exhibits exotic phases such as correlated insulating states~\cite{Cao2:2018} and superconductivity~\cite{Cao:2018}. Another notable development is the stacking of two TMDs, resulting in the observation of unique excitons known as interlayer excitons, characterized by electrons and holes located in separate layers~\cite{Rivera:2015,Xiong:2023,Sun:2022,Ubrig:2020}. The spatially indirect nature of interlayer excitons imparts them with distinct properties, including long exciton lifetimes~\cite{Rivera:2015}, extended diffusion lengths~\cite{Unuchek:2018}, large valley polarization~\cite{Rivera:2016}, and significant modulation by moir\'e potentials~\cite{Jin:2019,Seyler:2019}.

The existing vdW heterostructures comprise of 2D materials with similar lattice structure, excitonic characteristics, and inherently identical dimensions. Development of vdW heterostructures that encompass lower dimensional materials may give rise to unique interface exciton states resulting from the mixed dimensionality. Carbon nanotubes (CNTs), a typical one-dimensional (1D) material, are ideal for such heterostructures as they have all bonds confined to the tube itself~\cite{Ishii:2015,Ishii:2019}. CNTs interact with 2D materials through weak vdW forces, resulting in well-defined, atomically sharp interfaces~\cite{Jariwala:2016,Fang:2020}. The chirality-dependent bandgap of CNTs can be utilized to tune the band alignment~\cite{Fang:2023transfer}, allowing for unambiguous identification of excitonic states at the 1D-2D interface.

Here we report on the observation of multiple new excitonic peaks in the 1D-2D CNT/tungsten diselenide (WSe$_2$) heterostructures at room temperature. These peaks appear exclusively at the interface region with a broad energy range lower than CNT E\textsubscript{11} states, and their dependence on the chirality of CNTs and the layer number of WSe$_2$ is investigated. The emergence of the peaks is found to be highly correlated with the band alignment, and they are interpreted as interface excitons. Prominent linear polarization, low excitation saturation power, and a long lifetime are characteristic of low-energy interface excitons, suggesting strong confinement. Through photon correlation measurements, room-temperature single-photon emission (SPE) has been confirmed. These findings expand the existing concept of spatially indirect excitons based on 2D heterostructures to 1D systems, demonstrating significant potential of the novel interface excitons for nanophotonics and quantum information processing.


\section{Results and discussion}
\paragraph*{Emerging peaks in 1D-2D mixed-dimensional heterostructures.}

% Figure environment removed



The CNT/WSe$_2$ heterostructures under investigation are entirely free-standing to preclude substrate effects~\cite{Sun:2022,Lefebvre:2003}, as depicted in Fig.~\ref{Fig1}a and b. CNTs are initially grown over trenches, followed by a placement of a WSe$_2$ flake upon the tubes using the anthracene-assisted transfer technique~\cite{Otsuka:2021,Fang:2022}. A (9,4) CNT is selected as a representative case, forming type-II band alignment with WSe$_2$~\cite{Fang:2023transfer}. Such alignment should establish new excitonic states between the CNT conduction band minimum and the WSe$_2$ valence band maximum (Fig.~\ref{Fig1}c).

Room-temperature photoluminescence (PL) spectroscopy is employed to investigate the excitonic states present within the heterostructure~\cite{Jiang:2015,Uda:2016,Otsuka:2019}. The PL spectrum of the pristine, suspended (9,4) CNT displays a singular peak at 1.143~eV, corresponding to the E\textsubscript{11} transition (Fig.~\ref{Fig1}d)~\cite{Ishii:2015}. After the transfer of the monolayer WSe$_2$ flake, the CNT E\textsubscript{11} peak is redshifted to 1.102~eV (Fig.~\ref{Fig1}e) as a result of the dielectric screening effect~\cite{Fang:2020}, indicating intimate contact between the two materials. Notably, two excitonic peaks arise with energies lower than E\textsubscript{11}. These peaks cannot be attributed to the suspended monolayer (1L) WSe$_2$ emission as only the A exciton peak at 1.658~eV is expected (see Supplementary Fig.~1), suggesting the existence of new excitonic states in the CNT/WSe$_2$ heterostructure. The newly emerged peaks are initially unstable and exhibit temporal blinking (inset in Fig.~\ref{Fig1}e). Following a spectral development involving fluctuations of the peaks, the unstable excitonic states vanish and stable states remain for which we perform the subsequent measurements (see Supplementary Fig.~2).

Figure~\ref{Fig1}f presents the stable PL spectrum, featuring two prominent peaks at 0.924~eV and 0.821~eV, denoted as IX\textsubscript{1} and IX\textsubscript{2}, respectively. With an energy difference of approximately 0.278~eV from E\textsubscript{11}, IX\textsubscript{2} is at a lower energy than any reported dark or defect states in (9,4) CNTs~\cite{Matsunaga:2010,Kozawa:2022,Yu:2022}. We hypothesize these states to be interface excitons, which could possess substantially lower excitonic energies as determined by the heterostructure band alignment.

The spatial and spectral correlation between E\textsubscript{11} excitons and IXs is investigated by conducting PL imaging measurement and photoluminescence excitation (PLE) spectroscopy. Figure~\ref{Fig1}g is an integrated PL image from E\textsubscript{11}, exhibiting strong signal above the trench due to the quenching from the silicon dioxide (SiO$_2$)/silicon (Si) substrate. Both IX\textsubscript{1} and IX\textsubscript{2} peaks are observed precisely at the position of the E\textsubscript{11} peak, as demonstrated by the IX\textsubscript{1} and IX\textsubscript{2} images displayed in Fig.~\ref{Fig1}h and Supplementary Fig.~3, respectively. In the PLE map (Fig.~\ref{Fig1}i), E\textsubscript{11} shows a strong response to excitation energy corresponding to the CNT E\textsubscript{22} transition. Similar response as E\textsubscript{11} is observed for IX\textsubscript{1} and IX\textsubscript{2} peaks, implying that the carriers forming the IXs are supplied from the CNT. In comparison, we do not observe a clear signature of the WSe$_2$ A exciton peak in the PLE map. Considering IXs only emerge upon transfer of the WSe$_2$ flake, the spatial and spectral overlap with CNT supports the hypothesis that they originate from the interface. 

\paragraph*{Band alignment effect on IX peaks.}

% Figure environment removed


Since interface excitons form between the two materials, manipulating the heterostructure band alignment should affect the IXs~\cite{Ubrig:2020,Tan:2021}. It is possible to vary the CNT bandgap (Fig.~\ref{Fig2}a) by studying different CNT chiralities (Fig.~\ref{Fig2}b), whereas WSe$_2$ bandgap can be altered (Fig.~\ref{Fig2}c) through the layer number (Fig.~\ref{Fig2}d). 

We first investigate the chirality dependence, which significantly modulates the CNT bandgap. The band alignment is systematically tuned by utilizing CNT/WSe$_2$ heterostructures with different CNT chiralities as illustrated in Fig.~\ref{Fig2}b. WSe$_2$ layer numbers $\leq$ 4 are used for heterostructures since IX peaks can be observed as shown in the case for (9,4) CNTs with bilayer (2L) and quadlayer (4L) WSe$_2$. Multiple IX peaks appear in (9,4), (12,1), (8,6), (8,7), and (14,0) CNTs, which have large E\textsubscript{11} energies and therefore large bandgaps. This is consistent with the expectation that a large bandgap is favorable for type-II band alignment as depicted in Fig.~\ref{Fig2}a. The observed IX peaks span a broad energy range within the telecommunication wavelengths, and the highest energy peak in each chirality is located close to E\textsubscript{11}, with a difference of less than ~0.08~eV. Remarkably, a further decrease in the bandgap leads to the disappearance of IXs. The presence of the IX peaks is determined by chirality, consistent with the transition in band alignment from type-II to type-I. We therefore identify IXs as interface excitons.

The IX peaks in the PL spectra from various heterostructures are summarized in Fig.~\ref{Fig2}e by plotting the number of the peaks ($N$) observed during time-trace measurements as a function of E\textsubscript{11}. Two distinct regions can be seen below and above 0.95~eV, corresponding to type-I and type-II alignment, respectively. IX peaks are absent for type-I band alignment, whereas numerous peaks appear for type-II alignment. Such a band alignment transition has also been observed for exciton transfer process in the same system~\cite{Fang:2023transfer}.

The dependence of IXs on the number of WSe$_2$ layers is more subtle, since the number of layers does not significantly modulate the bandgap in comparison to CNT chirality (Fig.~\ref{Fig2}c). For example, (9,4) CNT/WSe$_2$ heterostructures are consistent with complete type-II band alignment irrespective of the WSe$_2$ layer number (Fig.~\ref{Fig1}f and Fig.~\ref{Fig2}b). We therefore study  (10,5) CNT/WSe$_2$ heterostructures located at the band alignment transition~\cite{Fang:2023transfer}, as they should be sensitive to the small changes in WSe$_2$ bandgap. The PL spectrum of the 1L WSe$_2$ heterostructure is presented in Fig.~\ref{Fig2}d, which does not show any observable IX peaks. In contrast, the trilayer (3L) WSe$_2$ heterostructure reveals a discernible IX peak in between the E\textsubscript{11} exciton and the trion (T) peaks. Two IX peaks appear for the 4L WSe$_2$ heterostructure, with increased PL intensity for the higher energy IX peak and an additional lower energy IX peak besides the trion peak. This layer-number dependent behavior of the IX peaks can be explained by the band alignment transition as shown in Fig.~\ref{Fig2}c.

\paragraph*{Optical properties of the interface excitons.}

% Figure environment removed


The interface excitons display several distinct features different from the E\textsubscript{11} excitons. In Fig.~\ref{Fig3}a, we first examine the emission polarization dependence of IX\textsubscript{1} and IX\textsubscript{2} for the (9,4) CNT/1L WSe$_2$ sample used in Fig.~\ref{Fig1}. PL from IXs exhibits near ideal linear polarization of $>$95\%, consistent with confinement in the 1D channel~\cite{Ishii:2015,Bai:2020}. We also note that the polarization angle of IX\textsubscript{1} and IX\textsubscript{2} deviates from that of E\textsubscript{11} polarization by ~13.7$^{\circ}$, potentially suggesting some distortion of the optical dipole moment in the IXs (see Supplementary Fig.~4).

Time-resolved PL is then performed for E\textsubscript{11} and IX\textsubscript{2} excitons, as the lifetime of IXs is expected to be long due to the spatially indirect nature~\cite{Rivera:2015,Unuchek:2018}. PL decay curves corresponding to E\textsubscript{11} and IX\textsubscript{2} are measured as shown in Fig.~\ref{Fig3}b, and the lifetime is extracted by reconvolution fitting using the instrument response function (IRF). Two decay components are obtained for the E\textsubscript{11} PL decay curve, as is the case for suspended CNTs: A main fast component with a lifetime of 59~ps associated with the bright states, and a small slow component with a lifetime of 646~ps associated with the dark states~\cite{Ishii:2019}. In contrast, only one decay component is observed for IX\textsubscript{2} with a long lifetime of 673~ps, consistent with the reduced optical dipole moment. 

The interface excitons exhibit considerably bright emission at low excitation powers as shown in Fig.~\ref{Fig3}c. At a low power of 0.04~$\mu$W, both IX\textsubscript{1} and IX\textsubscript{2} display bright emission with the IX\textsubscript{1} PL even exceeding the E\textsubscript{11} PL. Such high intensity emission from interface excitons is unexpected at room temperature. Generally, indirect excitons exhibit weak PL emission due to diminished dipole coupling, which is the case for 2D-2D heterostructures where interlayer excitons can hardly be observed at room temperature~\cite{Rivera:2015,Xiong:2023,Ubrig:2020}. The evident PL from interface excitons in our system can be ascribed to two primary factors. Firstly, we employ a fully suspended structure, which reduces the substrate-induced screening effect and helps sustain the dipole strength. Secondly, the wavefunction of $\pi$-orbitals in CNTs, which extend significantly out of the tube, could reduce the spatially indirect nature of the interface excitons.

Emission intensity of the interface excitons exhibit intriguing power dependence as shown in Fig.~\ref{Fig3}d. Both IX\textsubscript{1} and IX\textsubscript{2} PL nearly saturate with a low threshold power of approximately 0.6~$\mu$W, while the E\textsubscript{11} PL increases substantially. The saturation observed is much more pronounced compared to interlayer excitons in 2D-2D heterostructures~\cite{Rivera:2015}. It is suspected that the IXs are further confined to a lower dimension, that is, 0D. Interface excitons in CNTs may be more readily localized than interlayer excitons in 2D-2D heterostructures because of the lower dimensionality. In general, localized states show much stronger saturation behavior than free states because of the state-filling effects~\cite{Settele:2021,Shinokita:2021}. In other samples where interface excitons also emerge, we find that the saturation behavior depends on their energies (see Supplementary Fig.~5). The IX peaks with energies substantially lower than the E\textsubscript{11} energy display stronger saturation behaviors. This could be explained by the deeper trap potential in the confinement of the interface excitons. The localization is also supported by the observed blinking noise from the unstable IXs that host only “on” and “off” states (see Supplementary Fig.~2,6). The pure two-level noise is exclusively observed in quasi-0D systems, such as quantum dots and single molecules~\cite{Nirmal:1996,Moerner:1999}, strongly suggesting that the interface exciton state could function as a single-photon emitter.


\paragraph*{Room-temperature single-photon emission from interface excitons.}

% Figure environment removed



To elucidate the quantum nature of interface excitons, we conduct photon correlation measurements on a (9,4) CNT/2L WSe$_2$ sample under continuous-wave laser excitation. Two primary peaks are observed in this sample, originating from the stable low-energy IX and E\textsubscript{11} (see Supplementary Fig.~2,7). The second-order correlation $g^{(2)}(\tau)$ from the IX is shown in Fig.~\ref{Fig4}a, after background correction. A distinct antibunching dip is observed at zero delay with $g^{(2)}(0)=0.467$. Falling below the 0.5 threshold, this unequivocally demonstrates single-photon emission from the interface excitons at room temperature. 

Over a longer timescale, we also observe a bunching peak (Fig.~\ref{Fig4}b). The bunching behavior is often observed in other single-photon sources and is associated with the dynamics between excited states and other dark, charged, or meta-stable states~\cite{Sallen:2009,Tran:2016}. We employ the equation  $g^{(2)}(\tau)=[1-\alpha \exp(-|\tau|/\tau_A ) ]*[1+\beta \exp(-|\tau|/\tau_B )]$  to fit the observed $g^{(2)}(\tau)$ statistics, where factors $\alpha$ and $\beta$ quantify the degree of antibunching and bunching with values of 0.720 and 0.666, respectively~\cite{Gao:2017}. $\tau_A$ and $\tau_B$ indicate the timescales of the antibunching dip and the bunching peak, with values of 0.148~ns and 3.752~ns from the fitting, respectively. The value of $1 - \alpha$ is 0.280, revealing high single-photon purity by considering the effect of bunching behavior. It is noteworthy that most of the low-energy interface excitons display similar peak linewidths and power saturation behavior (see Supplementary Fig.~5), implying that each of them acts as a single-photon emitter. This is further supported by the high reproducibility of the antibunching behavior in other samples (see Supplementary Fig.~8).

For comparison, $g^{(2)}(\tau)$ from E\textsubscript{11} excitons does not exhibit any antibunching or bunching behavior, as shown in Fig.~\ref{Fig4}c. Under pulsed laser excitation, the E\textsubscript{11} excitons are known to go through an efficient exciton-exciton annihilation (EEA) process that could result in SPE~\cite{Ishii:2017}. The exciton density is generally lower with continuous-wave excitation, hindering the SPE through EEA. The confinement effect is crucial for SPE, and the absence of any antibunching behavior therefore indicates the 1D free feature of E\textsubscript{11} excitons.

We now discuss the possible origins of the localized interface exciton states, which would require a potential depth exceeding a few multiples of the thermal energy. The confinement can be provided by defect states, for example in materials such as diamond~\cite{Babinec:2010}, silicon carbide~\cite{Castelletto:2014}, hexagonal boron nitride~\cite{Tran:2016}, and CNTs~\cite{Ma:2015NatNano}. It is unlikely that defects in CNTs play a role, since we use pristine suspended CNTs containing negligible exciton quenching sites~\cite{Ishii:2015,Ishii:2017}. We note that they have been characterized under low-power conditions within dry nitrogen gas environment, precluding the formation of defects in the CNTs~\cite{Ishii:2015}. In comparison, WSe$_2$ flakes inherently encompass a range of defect states, spanning from single vacancies to complex vacancy clusters~\cite{Linhart:2019,Zhang:2017}. Although most defects in WSe$_2$ yield shallow energy traps and exhibit SPE only under cryogenic conditions, these defects may be responsible for the trapping potential for the interface excitons.

As another possible explanation, inhomogeneous strain could also contribute to the localization of interface excitons as in the case of monolayer WSe$_2$~\cite{parto:2021}. Strain might also impact the sample through many aspects in a manner similar to 2D-2D heterostructures, from van der Waals gap fluctuation~\cite{Shin:2016} to lattice reconstruction~\cite{Zhao:2023}. While the exact impact of strain remains unclear, the spatially indirect nature renders interface excitons more susceptible to the aforementioned effects than intralayer excitons.

In conclusion, we observe numerous IX peaks at the CNT/WSe$_2$ interface below the CNT E\textsubscript{11} energy, spanning the telecommunication wavelengths. By systematically varying the chirality of CNTs and the layer number of WSe$_2$, we are able to assign the peaks to interface excitons as they only appear for type-II band alignment. The low saturation power and the long lifetime indicate that low-energy interface excitons are strongly confined, and single-photon emission is confirmed with $g^{(2)}(0)<0.5$. The observation of interface excitons as room-temperature quantum emitters at telecommunication bands opens up new opportunities for applications in quantum technologies and optoelectronics, underscoring the emerging potential of mixed-dimensional heterostructures.

\section*{Methods}
\paragraph*{Air-suspended carbon nanotubes.}
We prepare air-suspended CNTs using trenched SiO$_2$/Si substrates~\cite{Ishii:2015}. First, we pattern alignment markers and trenches with lengths of 900~$\mu$m and widths ranging from 0.5 to 3.0~$\mu$m onto the Si substrates using electron-beam lithography, followed by dry etching. We then thermally oxidize the substrate to form a SiO$_2$ film, with a thickness ranging from 60 to 70 nm. Another electron-beam lithography process is used to define catalyst regions along the edges of the trenches. A 1.5~\AA~thick iron (Fe) film is deposited as a catalyst for CNT growth using an electron beam evaporator. CNTs are synthesized by alcohol chemical vapor deposition at 800$^{\circ}$C for 1 minute. The Fe film thickness is optimized to control the yield for preparing isolated CNTs. We select isolated, fully suspended chirality-identified CNTs with lengths ranging from 0.5 to 2.0~$\mu$m to form the heterostructures with WSe$_2$.

\paragraph*{Anthracene crystal growth.}
For transferring WSe$_2$ flakes onto CNTs, we grow anthracene crystals through an in-air sublimation process~\cite{Otsuka:2021,Fang:2022}. Anthracene powder is heated to 80$^{\circ}$C on a glass slide, while another glass slide is placed 1~mm above the anthracene source. Thin and large-area single crystals are then grown on the glass surface. To promote the growth of large-area single crystals, we pattern the glass slides using ink from commercial markers. The typical growth time for anthracene crystals is 10 hours.

\paragraph*{Transfer of WSe$_2$ by anthracene crystals.}
First, WSe$_2$ (HQ graphene) flakes are prepared on 90-nm-thick SiO$_2$/Si substrates using mechanical exfoliation, and the layer number is determined by optical contrast. An anthracene single crystal is picked up with a glass-supported PDMS sheet to form an anthracene/PDMS stamp. Next, the WSe$_2$ flakes are picked up by pressing the anthracene/PDMS stamp against a substrate with the target WSe$_2$ flakes. The stamp is quickly separated ($>$ 10~mm/s) to ensure that the anthracene crystal remains attached to the PDMS sheet. The stamp is then applied to the receiving substrate with the desired chirality-identified CNT, whose position is determined by a prior measurement. Precise position alignment is accomplished with the aid of markers prepared on the substrate. By slowly peeling off the PDMS ($<$ 0.2~$\mu$m/s), the anthracene crystal with the WSe$_2$ flake is released on the receiving substrate. Sublimation of anthracene in air at 110$^{\circ}$C for 10 minutes removes the anthracene crystal, leaving behind a clean suspended CNT/WSe$_2$ heterostructure. This all-dry process eliminates contamination from solvents, and the solid single-crystal anthracene protects the 2D flakes and the CNT during the transfer, ensuring that the CNT/WSe$_2$ heterostructure experiences minimal strain~\cite{Otsuka:2021,Fang:2022}.


\paragraph*{PL measurements.}
A homebuilt confocal microscopy system is employed to perform PL measurements for interface excitons and E\textsubscript{11} excitons at room temperature in dry nitrogen gas~\cite{Ishii:2015,Fang:2020}. We utilize a wavelength-tunable continuous-wave Ti:sapphire laser for excitation, with its power controlled by neutral density filters. The excitation polarization angle is adjusted to be parallel to the CNT axis and the emission polarization angle dependence is measured using a half-wave plate followed by a polarizer placed in front of a spectrometer. The laser beam is focused on the samples with an objective lens that has a numerical aperture of 0.65 and a working distance of 4.5~mm. The $1/e^2$ spot diameter and the collection spot size defined by a confocal pinhole are approximately 1.2 and 5.4~$\mu$m, respectively. PL is collected through the same objective lens and detected using a liquid-nitrogen-cooled 1024-pixel InGaAs diode array connected to the spectrometer. A 150-lines/mm grating is used to obtain a dispersion of 0.52~nm/pixel at a wavelength of 1340~nm. For photoluminescence measurements of WSe$_2$ A excitons, a 532-nm laser and a charge-coupled device camera are employed.


\paragraph*{Time-resolved and photon correlation measurements.}
Approximately 100 femtosecond pulses at a repetition rate of 76 MHz from a Ti:sapphire laser is utilized for time-resolved measurements. The excitation laser beam is focused onto the sample using an objective lens with a numerical aperture of 0.85 and a working distance of 1.48~mm. The PL from the center of the nanotube within the heterostructure is coupled to a superconducting single-photon detector with an optical fiber, and a time-correlated single-photon counting module is used to collect the data. IRFs dependent on the detection wavelength are acquired by dispersing supercontinuum white light pulses with a spectrometer. Photon correlation measurements are carried out using a Hanbury-Brown-Twiss setup with a 50:50 fiber coupler under excitation with a continuous-wave laser. The experiments are conducted at room temperature.

\begin{acknowledgments}
Parts of this study are supported by JSPS (KAKENHI JP22K14624, JP22K14625, JP21K14484, JP22F22350, JP22K14623, JP22H01893, JP21H05233, JP23H00262, JP20H02558) and MEXT (ARIM JPMXP1222UT1135). Y.R.C. is supported by JSPS (International Research Fellow). N.F. and C.F.F. are supported by RIKEN Special Postdoctoral Researcher Program. We thank the Advanced Manufacturing Support Team at RIKEN for technical assistance.
\end{acknowledgments}

\section*{Author Contributions}
N.F. carried out sample preparation and performed measurements on the samples. Y.R.C. assisted in sample preparation. Y.R.C., C.F.F., and K.N. assisted in optical measurements. D.Y., S.F., and D.K. contributed to the time-resolved PL and photon correlation measurements. M. M., Y.G., and S.O. performed density functional theory calculations. K.O. aided in the development of the anthracene-assisted dry transfer method. Y.K.K. supervised the project. N.F. and Y.K.K. co-wrote the manuscript, with all authors providing input and comments on the manuscript.

%apsrev4-2.bst 2019-01-14 (MD) hand-edited version of apsrev4-1.bst
%Control: key (0)
%Control: author (8) initials jnrlst
%Control: editor formatted (1) identically to author
%Control: production of article title (0) allowed
%Control: page (0) single
%Control: year (1) truncated
%Control: production of eprint (0) enabled
\begin{thebibliography}{42}%
\makeatletter
\providecommand \@ifxundefined [1]{%
 \@ifx{#1\undefined}
}%
\providecommand \@ifnum [1]{%
 \ifnum #1\expandafter \@firstoftwo
 \else \expandafter \@secondoftwo
 \fi
}%
\providecommand \@ifx [1]{%
 \ifx #1\expandafter \@firstoftwo
 \else \expandafter \@secondoftwo
 \fi
}%
\providecommand \natexlab [1]{#1}%
\providecommand \enquote  [1]{``#1''}%
\providecommand \bibnamefont  [1]{#1}%
\providecommand \bibfnamefont [1]{#1}%
\providecommand \citenamefont [1]{#1}%
\providecommand \href@noop [0]{\@secondoftwo}%
\providecommand \href [0]{\begingroup \@sanitize@url \@href}%
\providecommand \@href[1]{\@@startlink{#1}\@@href}%
\providecommand \@@href[1]{\endgroup#1\@@endlink}%
\providecommand \@sanitize@url [0]{\catcode `\\12\catcode `\$12\catcode
  `\&12\catcode `\#12\catcode `\^12\catcode `\_12\catcode `\%12\relax}%
\providecommand \@@startlink[1]{}%
\providecommand \@@endlink[0]{}%
\providecommand \url  [0]{\begingroup\@sanitize@url \@url }%
\providecommand \@url [1]{\endgroup\@href {#1}{\urlprefix }}%
\providecommand \urlprefix  [0]{URL }%
\providecommand \Eprint [0]{\href }%
\providecommand \doibase [0]{https://doi.org/}%
\providecommand \selectlanguage [0]{\@gobble}%
\providecommand \bibinfo  [0]{\@secondoftwo}%
\providecommand \bibfield  [0]{\@secondoftwo}%
\providecommand \translation [1]{[#1]}%
\providecommand \BibitemOpen [0]{}%
\providecommand \bibitemStop [0]{}%
\providecommand \bibitemNoStop [0]{.\EOS\space}%
\providecommand \EOS [0]{\spacefactor3000\relax}%
\providecommand \BibitemShut  [1]{\csname bibitem#1\endcsname}%
\let\auto@bib@innerbib\@empty
%</preamble>
\bibitem [{\citenamefont {Cao}\ \emph {et~al.}(2018{\natexlab{a}})\citenamefont
  {Cao}, \citenamefont {Fatemi}, \citenamefont {Demir}, \citenamefont {Fang},
  \citenamefont {Tomarken}, \citenamefont {Luo}, \citenamefont
  {Sanchez-Yamagishi}, \citenamefont {Watanabe}, \citenamefont {Taniguchi},
  \citenamefont {Kaxiras}, \citenamefont {Ashoori},\ and\ \citenamefont
  {Jarillo{-}Herrero}}]{Cao2:2018}%
  \BibitemOpen
  \bibfield  {author} {\bibinfo {author} {\bibfnamefont {Y.}~\bibnamefont
  {Cao}}, \bibinfo {author} {\bibfnamefont {V.}~\bibnamefont {Fatemi}},
  \bibinfo {author} {\bibfnamefont {A.}~\bibnamefont {Demir}}, \bibinfo
  {author} {\bibfnamefont {S.}~\bibnamefont {Fang}}, \bibinfo {author}
  {\bibfnamefont {S.~L.}\ \bibnamefont {Tomarken}}, \bibinfo {author}
  {\bibfnamefont {J.~Y.}\ \bibnamefont {Luo}}, \bibinfo {author} {\bibfnamefont
  {J.~D.}\ \bibnamefont {Sanchez-Yamagishi}}, \bibinfo {author} {\bibfnamefont
  {K.}~\bibnamefont {Watanabe}}, \bibinfo {author} {\bibfnamefont
  {T.}~\bibnamefont {Taniguchi}}, \bibinfo {author} {\bibfnamefont
  {E.}~\bibnamefont {Kaxiras}}, \bibinfo {author} {\bibfnamefont {R.~C.}\
  \bibnamefont {Ashoori}},\ and\ \bibinfo {author} {\bibfnamefont
  {P.}~\bibnamefont {Jarillo{-}Herrero}},\ }\bibfield  {title} {\bibinfo
  {title} {Correlated insulator behaviour at half-filling in magic-angle
  graphene superlattices},\ }\href {https://doi.org/10.1038/nature26154}
  {\bibfield  {journal} {\bibinfo  {journal} {Nature}\ }\textbf {\bibinfo
  {volume} {556}},\ \bibinfo {pages} {80} (\bibinfo {year}
  {2018}{\natexlab{a}})}\BibitemShut {NoStop}%
\bibitem [{\citenamefont {Cao}\ \emph {et~al.}(2018{\natexlab{b}})\citenamefont
  {Cao}, \citenamefont {Fatemi}, \citenamefont {Fang}, \citenamefont
  {Watanabe}, \citenamefont {Taniguchi}, \citenamefont {Kaxiras},\ and\
  \citenamefont {Jarillo-Herrero}}]{Cao:2018}%
  \BibitemOpen
  \bibfield  {author} {\bibinfo {author} {\bibfnamefont {Y.}~\bibnamefont
  {Cao}}, \bibinfo {author} {\bibfnamefont {V.}~\bibnamefont {Fatemi}},
  \bibinfo {author} {\bibfnamefont {S.}~\bibnamefont {Fang}}, \bibinfo {author}
  {\bibfnamefont {K.}~\bibnamefont {Watanabe}}, \bibinfo {author}
  {\bibfnamefont {T.}~\bibnamefont {Taniguchi}}, \bibinfo {author}
  {\bibfnamefont {E.}~\bibnamefont {Kaxiras}},\ and\ \bibinfo {author}
  {\bibfnamefont {P.}~\bibnamefont {Jarillo-Herrero}},\ }\bibfield  {title}
  {\bibinfo {title} {Unconventional superconductivity in magic-angle graphene
  superlattices},\ }\href {https://doi.org/10.1038/nature26160} {\bibfield
  {journal} {\bibinfo  {journal} {Nature}\ }\textbf {\bibinfo {volume} {556}},\
  \bibinfo {pages} {43} (\bibinfo {year} {2018}{\natexlab{b}})}\BibitemShut
  {NoStop}%
\bibitem [{\citenamefont {Rivera}\ \emph {et~al.}(2015)\citenamefont {Rivera},
  \citenamefont {Schaibley}, \citenamefont {Jones}, \citenamefont {Ross},
  \citenamefont {Wu}, \citenamefont {Aivazian}, \citenamefont {Klement},
  \citenamefont {Seyler}, \citenamefont {Clark}, \citenamefont {Ghimire},
  \citenamefont {Yan}, \citenamefont {Mandrus}, \citenamefont {Yao},\ and\
  \citenamefont {Xu}}]{Rivera:2015}%
  \BibitemOpen
  \bibfield  {author} {\bibinfo {author} {\bibfnamefont {P.}~\bibnamefont
  {Rivera}}, \bibinfo {author} {\bibfnamefont {J.~R.}\ \bibnamefont
  {Schaibley}}, \bibinfo {author} {\bibfnamefont {A.~M.}\ \bibnamefont
  {Jones}}, \bibinfo {author} {\bibfnamefont {J.~S.}\ \bibnamefont {Ross}},
  \bibinfo {author} {\bibfnamefont {S.}~\bibnamefont {Wu}}, \bibinfo {author}
  {\bibfnamefont {G.}~\bibnamefont {Aivazian}}, \bibinfo {author}
  {\bibfnamefont {P.}~\bibnamefont {Klement}}, \bibinfo {author} {\bibfnamefont
  {K.}~\bibnamefont {Seyler}}, \bibinfo {author} {\bibfnamefont
  {G.}~\bibnamefont {Clark}}, \bibinfo {author} {\bibfnamefont {N.~J.}\
  \bibnamefont {Ghimire}}, \bibinfo {author} {\bibfnamefont {J.}~\bibnamefont
  {Yan}}, \bibinfo {author} {\bibfnamefont {D.~G.}\ \bibnamefont {Mandrus}},
  \bibinfo {author} {\bibfnamefont {W.}~\bibnamefont {Yao}},\ and\ \bibinfo
  {author} {\bibfnamefont {X.}~\bibnamefont {Xu}},\ }\bibfield  {title}
  {\bibinfo {title} {Observation of long-lived interlayer excitons in monolayer
  {M}o{S}e$_2$-{WS}e$_2$ heterostructures},\ }\href
  {https://doi.org/10.1038/ncomms7242} {\bibfield  {journal} {\bibinfo
  {journal} {Nat. Commun.}\ }\textbf {\bibinfo {volume} {6}},\ \bibinfo {pages}
  {6242} (\bibinfo {year} {2015})}\BibitemShut {NoStop}%
\bibitem [{\citenamefont {Xiong}\ \emph {et~al.}(2023)\citenamefont {Xiong},
  \citenamefont {Nie}, \citenamefont {Brantly}, \citenamefont {Hays},
  \citenamefont {Sailus}, \citenamefont {Watanabe}, \citenamefont {Taniguchi},
  \citenamefont {Tongay},\ and\ \citenamefont {Jin}}]{Xiong:2023}%
  \BibitemOpen
  \bibfield  {author} {\bibinfo {author} {\bibfnamefont {R.}~\bibnamefont
  {Xiong}}, \bibinfo {author} {\bibfnamefont {J.~H.}\ \bibnamefont {Nie}},
  \bibinfo {author} {\bibfnamefont {S.~L.}\ \bibnamefont {Brantly}}, \bibinfo
  {author} {\bibfnamefont {P.}~\bibnamefont {Hays}}, \bibinfo {author}
  {\bibfnamefont {R.}~\bibnamefont {Sailus}}, \bibinfo {author} {\bibfnamefont
  {K.}~\bibnamefont {Watanabe}}, \bibinfo {author} {\bibfnamefont
  {T.}~\bibnamefont {Taniguchi}}, \bibinfo {author} {\bibfnamefont
  {S.}~\bibnamefont {Tongay}},\ and\ \bibinfo {author} {\bibfnamefont
  {C.}~\bibnamefont {Jin}},\ }\bibfield  {title} {\bibinfo {title} {Correlated
  insulator of excitons in {M}o{S}e$_2$-{WS}e$_2$ moir{\'e} superlattices},\
  }\href {https://doi.org/DOI: 10.1126/science.add5574} {\bibfield  {journal}
  {\bibinfo  {journal} {Science}\ }\textbf {\bibinfo {volume} {380}},\ \bibinfo
  {pages} {860} (\bibinfo {year} {2023})}\BibitemShut {NoStop}%
\bibitem [{\citenamefont {Sun}\ \emph {et~al.}(2022)\citenamefont {Sun},
  \citenamefont {Zhu}, \citenamefont {Qin}, \citenamefont {Liu}, \citenamefont
  {Tang}, \citenamefont {L{\"u}}, \citenamefont {Rahman}, \citenamefont
  {Yildirim},\ and\ \citenamefont {Lu}}]{Sun:2022}%
  \BibitemOpen
  \bibfield  {author} {\bibinfo {author} {\bibfnamefont {X.}~\bibnamefont
  {Sun}}, \bibinfo {author} {\bibfnamefont {Y.}~\bibnamefont {Zhu}}, \bibinfo
  {author} {\bibfnamefont {H.}~\bibnamefont {Qin}}, \bibinfo {author}
  {\bibfnamefont {B.}~\bibnamefont {Liu}}, \bibinfo {author} {\bibfnamefont
  {Y.}~\bibnamefont {Tang}}, \bibinfo {author} {\bibfnamefont {T.}~\bibnamefont
  {L{\"u}}}, \bibinfo {author} {\bibfnamefont {S.}~\bibnamefont {Rahman}},
  \bibinfo {author} {\bibfnamefont {T.}~\bibnamefont {Yildirim}},\ and\
  \bibinfo {author} {\bibfnamefont {Y.}~\bibnamefont {Lu}},\ }\bibfield
  {title} {\bibinfo {title} {Enhanced interactions of interlayer excitons in
  free-standing heterobilayers},\ }\href
  {https://doi.org/10.1038/s41586-022-05193-z} {\bibfield  {journal} {\bibinfo
  {journal} {Nature}\ }\textbf {\bibinfo {volume} {610}},\ \bibinfo {pages}
  {478} (\bibinfo {year} {2022})}\BibitemShut {NoStop}%
\bibitem [{\citenamefont {Ubrig}\ \emph {et~al.}(2020)\citenamefont {Ubrig},
  \citenamefont {Ponomarev}, \citenamefont {Zultak}, \citenamefont
  {Domaretskiy}, \citenamefont {Z{\'o}lyomi}, \citenamefont {Terry},
  \citenamefont {Howarth}, \citenamefont {Guti{\'e}rrez-Lezama}, \citenamefont
  {Zhukov}, \citenamefont {Kudrynskyi}, \citenamefont {Kovalyuk}, \citenamefont
  {Patan{\'e}}, \citenamefont {Taniguchi}, \citenamefont {Watanabe},
  \citenamefont {Gorbachev}, \citenamefont {Fal{'}ko},\ and\ \citenamefont
  {Morpurgo}}]{Ubrig:2020}%
  \BibitemOpen
  \bibfield  {author} {\bibinfo {author} {\bibfnamefont {N.}~\bibnamefont
  {Ubrig}}, \bibinfo {author} {\bibfnamefont {E.}~\bibnamefont {Ponomarev}},
  \bibinfo {author} {\bibfnamefont {J.}~\bibnamefont {Zultak}}, \bibinfo
  {author} {\bibfnamefont {D.}~\bibnamefont {Domaretskiy}}, \bibinfo {author}
  {\bibfnamefont {V.}~\bibnamefont {Z{\'o}lyomi}}, \bibinfo {author}
  {\bibfnamefont {D.}~\bibnamefont {Terry}}, \bibinfo {author} {\bibfnamefont
  {J.}~\bibnamefont {Howarth}}, \bibinfo {author} {\bibfnamefont
  {I.}~\bibnamefont {Guti{\'e}rrez-Lezama}}, \bibinfo {author} {\bibfnamefont
  {A.}~\bibnamefont {Zhukov}}, \bibinfo {author} {\bibfnamefont {Z.~R.}\
  \bibnamefont {Kudrynskyi}}, \bibinfo {author} {\bibfnamefont {Z.~D.}\
  \bibnamefont {Kovalyuk}}, \bibinfo {author} {\bibfnamefont {A.}~\bibnamefont
  {Patan{\'e}}}, \bibinfo {author} {\bibfnamefont {T.}~\bibnamefont
  {Taniguchi}}, \bibinfo {author} {\bibfnamefont {K.}~\bibnamefont {Watanabe}},
  \bibinfo {author} {\bibfnamefont {R.~V.}\ \bibnamefont {Gorbachev}}, \bibinfo
  {author} {\bibfnamefont {V.~I.}\ \bibnamefont {Fal{'}ko}},\ and\ \bibinfo
  {author} {\bibfnamefont {A.~F.}\ \bibnamefont {Morpurgo}},\ }\bibfield
  {title} {\bibinfo {title} {Design of van der {W}aals interfaces for
  broad-spectrum optoelectronics},\ }\href
  {https://doi.org/10.1038/s41563-019-0601-3} {\bibfield  {journal} {\bibinfo
  {journal} {Nat. Mater.}\ }\textbf {\bibinfo {volume} {19}},\ \bibinfo {pages}
  {299} (\bibinfo {year} {2020})}\BibitemShut {NoStop}%
\bibitem [{\citenamefont {Unuchek}\ \emph {et~al.}(2018)\citenamefont
  {Unuchek}, \citenamefont {Ciarrocchi}, \citenamefont {Avsar}, \citenamefont
  {Watanabe}, \citenamefont {Taniguchi},\ and\ \citenamefont
  {Kis}}]{Unuchek:2018}%
  \BibitemOpen
  \bibfield  {author} {\bibinfo {author} {\bibfnamefont {D.}~\bibnamefont
  {Unuchek}}, \bibinfo {author} {\bibfnamefont {A.}~\bibnamefont {Ciarrocchi}},
  \bibinfo {author} {\bibfnamefont {A.}~\bibnamefont {Avsar}}, \bibinfo
  {author} {\bibfnamefont {K.}~\bibnamefont {Watanabe}}, \bibinfo {author}
  {\bibfnamefont {T.}~\bibnamefont {Taniguchi}},\ and\ \bibinfo {author}
  {\bibfnamefont {A.}~\bibnamefont {Kis}},\ }\bibfield  {title} {\bibinfo
  {title} {Room-temperature electrical control of exciton flux in a van der
  {W}aals heterostructure},\ }\href {https://doi.org/10.1038/s41586-018-0357-y}
  {\bibfield  {journal} {\bibinfo  {journal} {Nature}\ }\textbf {\bibinfo
  {volume} {560}},\ \bibinfo {pages} {340} (\bibinfo {year}
  {2018})}\BibitemShut {NoStop}%
\bibitem [{\citenamefont {Rivera}\ \emph {et~al.}(2016)\citenamefont {Rivera},
  \citenamefont {Seyler}, \citenamefont {Yu}, \citenamefont {Schaibley},
  \citenamefont {Yan}, \citenamefont {Mandrus}, \citenamefont {Yao},\ and\
  \citenamefont {Xu}}]{Rivera:2016}%
  \BibitemOpen
  \bibfield  {author} {\bibinfo {author} {\bibfnamefont {P.}~\bibnamefont
  {Rivera}}, \bibinfo {author} {\bibfnamefont {K.~L.}\ \bibnamefont {Seyler}},
  \bibinfo {author} {\bibfnamefont {H.}~\bibnamefont {Yu}}, \bibinfo {author}
  {\bibfnamefont {J.~R.}\ \bibnamefont {Schaibley}}, \bibinfo {author}
  {\bibfnamefont {J.}~\bibnamefont {Yan}}, \bibinfo {author} {\bibfnamefont
  {D.~G.}\ \bibnamefont {Mandrus}}, \bibinfo {author} {\bibfnamefont
  {W.}~\bibnamefont {Yao}},\ and\ \bibinfo {author} {\bibfnamefont
  {X.}~\bibnamefont {Xu}},\ }\bibfield  {title} {\bibinfo {title}
  {Valley-polarized exciton dynamics in a 2{D} semiconductor heterostructure},\
  }\href {https://doi.org/DOI: 10.1126/science.aac7820} {\bibfield  {journal}
  {\bibinfo  {journal} {Science}\ }\textbf {\bibinfo {volume} {351}},\ \bibinfo
  {pages} {688} (\bibinfo {year} {2016})}\BibitemShut {NoStop}%
\bibitem [{\citenamefont {Jin}\ \emph {et~al.}(2019)\citenamefont {Jin},
  \citenamefont {Regan}, \citenamefont {Yan}, \citenamefont {Utama},
  \citenamefont {Wang}, \citenamefont {Zhao}, \citenamefont {Qin},
  \citenamefont {Yang}, \citenamefont {Zheng}, \citenamefont {Shi},
  \citenamefont {Watanabe}, \citenamefont {Taniguchi}, \citenamefont {Tongay},
  \citenamefont {Zettl},\ and\ \citenamefont {Wang}}]{Jin:2019}%
  \BibitemOpen
  \bibfield  {author} {\bibinfo {author} {\bibfnamefont {C.}~\bibnamefont
  {Jin}}, \bibinfo {author} {\bibfnamefont {E.~C.}\ \bibnamefont {Regan}},
  \bibinfo {author} {\bibfnamefont {A.}~\bibnamefont {Yan}}, \bibinfo {author}
  {\bibfnamefont {M.~I.~B.}\ \bibnamefont {Utama}}, \bibinfo {author}
  {\bibfnamefont {D.}~\bibnamefont {Wang}}, \bibinfo {author} {\bibfnamefont
  {S.}~\bibnamefont {Zhao}}, \bibinfo {author} {\bibfnamefont {Y.}~\bibnamefont
  {Qin}}, \bibinfo {author} {\bibfnamefont {S.}~\bibnamefont {Yang}}, \bibinfo
  {author} {\bibfnamefont {Z.}~\bibnamefont {Zheng}}, \bibinfo {author}
  {\bibfnamefont {S.}~\bibnamefont {Shi}}, \bibinfo {author} {\bibfnamefont
  {K.}~\bibnamefont {Watanabe}}, \bibinfo {author} {\bibfnamefont
  {T.}~\bibnamefont {Taniguchi}}, \bibinfo {author} {\bibfnamefont
  {S.}~\bibnamefont {Tongay}}, \bibinfo {author} {\bibfnamefont
  {A.}~\bibnamefont {Zettl}},\ and\ \bibinfo {author} {\bibfnamefont
  {F.}~\bibnamefont {Wang}},\ }\bibfield  {title} {\bibinfo {title}
  {Observation of moir{\'{e}} excitons in {WSe}$_2$/{WS}$_2$ heterostructure
  superlattices},\ }\href {https://doi.org/10.1038/s41586-019-0976-y}
  {\bibfield  {journal} {\bibinfo  {journal} {Nature}\ }\textbf {\bibinfo
  {volume} {567}},\ \bibinfo {pages} {76} (\bibinfo {year} {2019})}\BibitemShut
  {NoStop}%
\bibitem [{\citenamefont {Seyler}\ \emph {et~al.}(2019)\citenamefont {Seyler},
  \citenamefont {Rivera}, \citenamefont {Yu}, \citenamefont {Wilson},
  \citenamefont {Ray}, \citenamefont {Mandrus}, \citenamefont {Yan},
  \citenamefont {Yao},\ and\ \citenamefont {Xu}}]{Seyler:2019}%
  \BibitemOpen
  \bibfield  {author} {\bibinfo {author} {\bibfnamefont {K.~L.}\ \bibnamefont
  {Seyler}}, \bibinfo {author} {\bibfnamefont {P.}~\bibnamefont {Rivera}},
  \bibinfo {author} {\bibfnamefont {H.}~\bibnamefont {Yu}}, \bibinfo {author}
  {\bibfnamefont {N.~P.}\ \bibnamefont {Wilson}}, \bibinfo {author}
  {\bibfnamefont {E.~L.}\ \bibnamefont {Ray}}, \bibinfo {author} {\bibfnamefont
  {D.~G.}\ \bibnamefont {Mandrus}}, \bibinfo {author} {\bibfnamefont
  {J.}~\bibnamefont {Yan}}, \bibinfo {author} {\bibfnamefont {W.}~\bibnamefont
  {Yao}},\ and\ \bibinfo {author} {\bibfnamefont {X.}~\bibnamefont {Xu}},\
  }\bibfield  {title} {\bibinfo {title} {Signatures of moir{\'e}-trapped valley
  excitons in {{MoSe$_2$}}/{{WSe$_2$}} heterobilayers},\ }\href
  {https://doi.org/10.1038/s41586-019-0957-1} {\bibfield  {journal} {\bibinfo
  {journal} {Nature}\ }\textbf {\bibinfo {volume} {567}},\ \bibinfo {pages}
  {66} (\bibinfo {year} {2019})}\BibitemShut {NoStop}%
\bibitem [{\citenamefont {Ishii}\ \emph {et~al.}(2015)\citenamefont {Ishii},
  \citenamefont {Yoshida},\ and\ \citenamefont {Kato}}]{Ishii:2015}%
  \BibitemOpen
  \bibfield  {author} {\bibinfo {author} {\bibfnamefont {A.}~\bibnamefont
  {Ishii}}, \bibinfo {author} {\bibfnamefont {M.}~\bibnamefont {Yoshida}},\
  and\ \bibinfo {author} {\bibfnamefont {Y.~K.}\ \bibnamefont {Kato}},\
  }\bibfield  {title} {\bibinfo {title} {Exciton diffusion, end quenching, and
  exciton-exciton annihilation in individual air-suspended carbon nanotubes},\
  }\href {https://doi.org/10.1103/PhysRevB.91.125427} {\bibfield  {journal}
  {\bibinfo  {journal} {Phys. Rev. B}\ }\textbf {\bibinfo {volume} {91}},\
  \bibinfo {pages} {125427} (\bibinfo {year} {2015})}\BibitemShut {NoStop}%
\bibitem [{\citenamefont {Ishii}\ \emph {et~al.}(2019)\citenamefont {Ishii},
  \citenamefont {Yoshida},\ and\ \citenamefont {Kato}}]{Ishii:2019}%
  \BibitemOpen
  \bibfield  {author} {\bibinfo {author} {\bibfnamefont {A.}~\bibnamefont
  {Ishii}}, \bibinfo {author} {\bibfnamefont {M.}~\bibnamefont {Yoshida}},\
  and\ \bibinfo {author} {\bibfnamefont {Y.~K.}\ \bibnamefont {Kato}},\
  }\bibfield  {title} {\bibinfo {title} {High efficiency dark-to-bright exciton
  conversion in carbon nanotubes},\ }\href
  {https://doi.org/10.1103/physrevx.9.041048} {\bibfield  {journal} {\bibinfo
  {journal} {Phys. Rev. X}\ }\textbf {\bibinfo {volume} {9}},\ \bibinfo {pages}
  {041048} (\bibinfo {year} {2019})}\BibitemShut {NoStop}%
\bibitem [{\citenamefont {Jariwala}\ \emph {et~al.}(2016)\citenamefont
  {Jariwala}, \citenamefont {Marks},\ and\ \citenamefont
  {Hersam}}]{Jariwala:2016}%
  \BibitemOpen
  \bibfield  {author} {\bibinfo {author} {\bibfnamefont {D.}~\bibnamefont
  {Jariwala}}, \bibinfo {author} {\bibfnamefont {T.~J.}\ \bibnamefont
  {Marks}},\ and\ \bibinfo {author} {\bibfnamefont {M.~C.}\ \bibnamefont
  {Hersam}},\ }\bibfield  {title} {\bibinfo {title} {Mixed-dimensional van der
  {W}aals heterostructures},\ }\href {https://doi.org/10.1038/nmat4703}
  {\bibfield  {journal} {\bibinfo  {journal} {Nat. Mater.}\ }\textbf {\bibinfo
  {volume} {16}},\ \bibinfo {pages} {170} (\bibinfo {year} {2016})}\BibitemShut
  {NoStop}%
\bibitem [{\citenamefont {Fang}\ \emph {et~al.}(2020)\citenamefont {Fang},
  \citenamefont {Otsuka}, \citenamefont {Ishii}, \citenamefont {Taniguchi},
  \citenamefont {Watanabe}, \citenamefont {Nagashio},\ and\ \citenamefont
  {Kato}}]{Fang:2020}%
  \BibitemOpen
  \bibfield  {author} {\bibinfo {author} {\bibfnamefont {N.}~\bibnamefont
  {Fang}}, \bibinfo {author} {\bibfnamefont {K.}~\bibnamefont {Otsuka}},
  \bibinfo {author} {\bibfnamefont {A.}~\bibnamefont {Ishii}}, \bibinfo
  {author} {\bibfnamefont {T.}~\bibnamefont {Taniguchi}}, \bibinfo {author}
  {\bibfnamefont {K.}~\bibnamefont {Watanabe}}, \bibinfo {author}
  {\bibfnamefont {K.}~\bibnamefont {Nagashio}},\ and\ \bibinfo {author}
  {\bibfnamefont {Y.~K.}\ \bibnamefont {Kato}},\ }\bibfield  {title} {\bibinfo
  {title} {Hexagonal boron nitride as an ideal substrate for carbon nanotube
  photonics},\ }\href {https://doi.org/10.1021/acsphotonics.0c00406} {\bibfield
   {journal} {\bibinfo  {journal} {ACS Photonics}\ }\textbf {\bibinfo {volume}
  {7}},\ \bibinfo {pages} {1773} (\bibinfo {year} {2020})}\BibitemShut
  {NoStop}%
\bibitem [{\citenamefont {Fang}\ \emph {et~al.}(2023)\citenamefont {Fang},
  \citenamefont {Yamashita}, \citenamefont {Fujii}, \citenamefont {Maruyama},
  \citenamefont {Gao}, \citenamefont {Chang}, \citenamefont {Fong},
  \citenamefont {Otsuka}, \citenamefont {Nagashio}, \citenamefont {Okada},\
  and\ \citenamefont {Kato}}]{Fang:2023transfer}%
  \BibitemOpen
  \bibfield  {author} {\bibinfo {author} {\bibfnamefont {N.}~\bibnamefont
  {Fang}}, \bibinfo {author} {\bibfnamefont {D.}~\bibnamefont {Yamashita}},
  \bibinfo {author} {\bibfnamefont {S.}~\bibnamefont {Fujii}}, \bibinfo
  {author} {\bibfnamefont {M.}~\bibnamefont {Maruyama}}, \bibinfo {author}
  {\bibfnamefont {Y.}~\bibnamefont {Gao}}, \bibinfo {author} {\bibfnamefont
  {Y.-R.}\ \bibnamefont {Chang}}, \bibinfo {author} {\bibfnamefont {C.~F.}\
  \bibnamefont {Fong}}, \bibinfo {author} {\bibfnamefont {K.}~\bibnamefont
  {Otsuka}}, \bibinfo {author} {\bibfnamefont {K.}~\bibnamefont {Nagashio}},
  \bibinfo {author} {\bibfnamefont {S.}~\bibnamefont {Okada}},\ and\ \bibinfo
  {author} {\bibfnamefont {Y.~K.}\ \bibnamefont {Kato}},\ }\bibfield  {title}
  {\bibinfo {title} {Resonant exciton transfer in mixed-dimensional
  heterostructures for overcoming dimensional restrictions in optical
  processes},\ }\href {https://doi.org/10.48550/arXiv.2307.07124} {\bibfield
  {journal} {\bibinfo  {journal} {arXiv:2307.07124}\ } (\bibinfo {year}
  {2023})}\BibitemShut {NoStop}%
\bibitem [{\citenamefont {Lefebvre}\ \emph {et~al.}(2003)\citenamefont
  {Lefebvre}, \citenamefont {Homma},\ and\ \citenamefont
  {Finnie}}]{Lefebvre:2003}%
  \BibitemOpen
  \bibfield  {author} {\bibinfo {author} {\bibfnamefont {J.}~\bibnamefont
  {Lefebvre}}, \bibinfo {author} {\bibfnamefont {Y.}~\bibnamefont {Homma}},\
  and\ \bibinfo {author} {\bibfnamefont {P.}~\bibnamefont {Finnie}},\
  }\bibfield  {title} {\bibinfo {title} {Bright band gap photoluminescence from
  unprocessed single-walled carbon nanotubes},\ }\href
  {https://doi.org/10.1103/PhysRevLett.90.217401} {\bibfield  {journal}
  {\bibinfo  {journal} {Phys. Rev. Lett.}\ }\textbf {\bibinfo {volume} {90}},\
  \bibinfo {pages} {217401} (\bibinfo {year} {2003})}\BibitemShut {NoStop}%
\bibitem [{\citenamefont {Otsuka}\ \emph {et~al.}(2021)\citenamefont {Otsuka},
  \citenamefont {Fang}, \citenamefont {Yamashita}, \citenamefont {Taniguchi},
  \citenamefont {Watanabe},\ and\ \citenamefont {Kato}}]{Otsuka:2021}%
  \BibitemOpen
  \bibfield  {author} {\bibinfo {author} {\bibfnamefont {K.}~\bibnamefont
  {Otsuka}}, \bibinfo {author} {\bibfnamefont {N.}~\bibnamefont {Fang}},
  \bibinfo {author} {\bibfnamefont {D.}~\bibnamefont {Yamashita}}, \bibinfo
  {author} {\bibfnamefont {T.}~\bibnamefont {Taniguchi}}, \bibinfo {author}
  {\bibfnamefont {K.}~\bibnamefont {Watanabe}},\ and\ \bibinfo {author}
  {\bibfnamefont {Y.~K.}\ \bibnamefont {Kato}},\ }\bibfield  {title} {\bibinfo
  {title} {{Deterministic transfer of optical-quality carbon nanotubes for
  atomically defined technology}},\ }\href
  {http://dx.doi.org/10.1038/s41467-021-23413-4} {\bibfield  {journal}
  {\bibinfo  {journal} {Nat. Commun.}\ }\textbf {\bibinfo {volume} {12}},\
  \bibinfo {pages} {3138} (\bibinfo {year} {2021})}\BibitemShut {NoStop}%
\bibitem [{\citenamefont {Fang}\ \emph {et~al.}(2022)\citenamefont {Fang},
  \citenamefont {Yamashita}, \citenamefont {Fujii}, \citenamefont {Otsuka},
  \citenamefont {Taniguchi}, \citenamefont {Watanabe}, \citenamefont
  {Nagashio},\ and\ \citenamefont {Kato}}]{Fang:2022}%
  \BibitemOpen
  \bibfield  {author} {\bibinfo {author} {\bibfnamefont {N.}~\bibnamefont
  {Fang}}, \bibinfo {author} {\bibfnamefont {D.}~\bibnamefont {Yamashita}},
  \bibinfo {author} {\bibfnamefont {S.}~\bibnamefont {Fujii}}, \bibinfo
  {author} {\bibfnamefont {K.}~\bibnamefont {Otsuka}}, \bibinfo {author}
  {\bibfnamefont {T.}~\bibnamefont {Taniguchi}}, \bibinfo {author}
  {\bibfnamefont {K.}~\bibnamefont {Watanabe}}, \bibinfo {author}
  {\bibfnamefont {K.}~\bibnamefont {Nagashio}},\ and\ \bibinfo {author}
  {\bibfnamefont {Y.~K.}\ \bibnamefont {Kato}},\ }\bibfield  {title} {\bibinfo
  {title} {Quantization of mode shifts in nanocavities integrated with
  atomically thin sheets},\ }\href {https://doi.org/10.1002/adom.202200538}
  {\bibfield  {journal} {\bibinfo  {journal} {Adv. Opt. Mater.}\ }\textbf
  {\bibinfo {volume} {10}},\ \bibinfo {pages} {2200538} (\bibinfo {year}
  {2022})}\BibitemShut {NoStop}%
\bibitem [{\citenamefont {Jiang}\ \emph {et~al.}(2015)\citenamefont {Jiang},
  \citenamefont {Kumamoto}, \citenamefont {Ishii}, \citenamefont {Yoshida},
  \citenamefont {Shimada},\ and\ \citenamefont {Kato}}]{Jiang:2015}%
  \BibitemOpen
  \bibfield  {author} {\bibinfo {author} {\bibfnamefont {M.}~\bibnamefont
  {Jiang}}, \bibinfo {author} {\bibfnamefont {Y.}~\bibnamefont {Kumamoto}},
  \bibinfo {author} {\bibfnamefont {A.}~\bibnamefont {Ishii}}, \bibinfo
  {author} {\bibfnamefont {M.}~\bibnamefont {Yoshida}}, \bibinfo {author}
  {\bibfnamefont {T.}~\bibnamefont {Shimada}},\ and\ \bibinfo {author}
  {\bibfnamefont {Y.~K.}\ \bibnamefont {Kato}},\ }\bibfield  {title} {\bibinfo
  {title} {Gate-controlled generation of optical pulse trains using individual
  carbon nanotubes},\ }\href {http://dx.doi.org/10.1038/ncomms7335} {\bibfield
  {journal} {\bibinfo  {journal} {Nat. Commun.}\ }\textbf {\bibinfo {volume}
  {6}},\ \bibinfo {pages} {6335} (\bibinfo {year} {2015})}\BibitemShut
  {NoStop}%
\bibitem [{\citenamefont {Uda}\ \emph {et~al.}(2016)\citenamefont {Uda},
  \citenamefont {Yoshida}, \citenamefont {Ishii},\ and\ \citenamefont
  {Kato}}]{Uda:2016}%
  \BibitemOpen
  \bibfield  {author} {\bibinfo {author} {\bibfnamefont {T.}~\bibnamefont
  {Uda}}, \bibinfo {author} {\bibfnamefont {M.}~\bibnamefont {Yoshida}},
  \bibinfo {author} {\bibfnamefont {A.}~\bibnamefont {Ishii}},\ and\ \bibinfo
  {author} {\bibfnamefont {Y.~K.}\ \bibnamefont {Kato}},\ }\bibfield  {title}
  {\bibinfo {title} {Electric-field induced activation of dark excitonic states
  in carbon nanotubes},\ }\href {https://doi.org/10.1021/acs.nanolett.5b04595}
  {\bibfield  {journal} {\bibinfo  {journal} {Nano Lett.}\ }\textbf {\bibinfo
  {volume} {16}},\ \bibinfo {pages} {2278} (\bibinfo {year}
  {2016})}\BibitemShut {NoStop}%
\bibitem [{\citenamefont {Otsuka}\ \emph {et~al.}(2019)\citenamefont {Otsuka},
  \citenamefont {Ishii},\ and\ \citenamefont {Kato}}]{Otsuka:2019}%
  \BibitemOpen
  \bibfield  {author} {\bibinfo {author} {\bibfnamefont {K.}~\bibnamefont
  {Otsuka}}, \bibinfo {author} {\bibfnamefont {A.}~\bibnamefont {Ishii}},\ and\
  \bibinfo {author} {\bibfnamefont {Y.~K.}\ \bibnamefont {Kato}},\ }\bibfield
  {title} {\bibinfo {title} {Super-resolution fluorescence imaging of carbon
  nanotubes using a nonlinear excitonic process},\ }\href
  {https://doi.org/10.1364/oe.27.017463} {\bibfield  {journal} {\bibinfo
  {journal} {Opt. Express}\ }\textbf {\bibinfo {volume} {27}},\ \bibinfo
  {pages} {17463} (\bibinfo {year} {2019})}\BibitemShut {NoStop}%
\bibitem [{\citenamefont {Matsunaga}\ \emph {et~al.}(2010)\citenamefont
  {Matsunaga}, \citenamefont {Matsuda},\ and\ \citenamefont
  {Kanemitsu}}]{Matsunaga:2010}%
  \BibitemOpen
  \bibfield  {author} {\bibinfo {author} {\bibfnamefont {R.}~\bibnamefont
  {Matsunaga}}, \bibinfo {author} {\bibfnamefont {K.}~\bibnamefont {Matsuda}},\
  and\ \bibinfo {author} {\bibfnamefont {Y.}~\bibnamefont {Kanemitsu}},\
  }\bibfield  {title} {\bibinfo {title} {Origin of low-energy photoluminescence
  peaks in single carbon nanotubes: ${K}$-momentum dark excitons and triplet
  dark excitons},\ }\href {https://doi.org/10.1103/PhysRevB.81.033401}
  {\bibfield  {journal} {\bibinfo  {journal} {Phys. Rev. B}\ }\textbf {\bibinfo
  {volume} {81}},\ \bibinfo {pages} {033401} (\bibinfo {year}
  {2010})}\BibitemShut {NoStop}%
\bibitem [{\citenamefont {Kozawa}\ \emph {et~al.}(2022)\citenamefont {Kozawa},
  \citenamefont {Wu}, \citenamefont {Ishii}, \citenamefont {Fortner},
  \citenamefont {Otsuka}, \citenamefont {Xiang}, \citenamefont {Inoue},
  \citenamefont {Maruyama}, \citenamefont {Wang},\ and\ \citenamefont
  {Kato}}]{Kozawa:2022}%
  \BibitemOpen
  \bibfield  {author} {\bibinfo {author} {\bibfnamefont {D.}~\bibnamefont
  {Kozawa}}, \bibinfo {author} {\bibfnamefont {X.}~\bibnamefont {Wu}}, \bibinfo
  {author} {\bibfnamefont {A.}~\bibnamefont {Ishii}}, \bibinfo {author}
  {\bibfnamefont {J.}~\bibnamefont {Fortner}}, \bibinfo {author} {\bibfnamefont
  {K.}~\bibnamefont {Otsuka}}, \bibinfo {author} {\bibfnamefont
  {R.}~\bibnamefont {Xiang}}, \bibinfo {author} {\bibfnamefont
  {T.}~\bibnamefont {Inoue}}, \bibinfo {author} {\bibfnamefont
  {S.}~\bibnamefont {Maruyama}}, \bibinfo {author} {\bibfnamefont
  {Y.}~\bibnamefont {Wang}},\ and\ \bibinfo {author} {\bibfnamefont {Y.~K.}\
  \bibnamefont {Kato}},\ }\bibfield  {title} {\bibinfo {title} {Formation of
  organic color centers in air-suspended carbon nanotubes using vapor-phase
  reaction},\ }\href {https://doi.org/10.1038/s41467-022-30508-z} {\bibfield
  {journal} {\bibinfo  {journal} {Nat. Commun.}\ }\textbf {\bibinfo {volume}
  {13}},\ \bibinfo {pages} {2814} (\bibinfo {year} {2022})}\BibitemShut
  {NoStop}%
\bibitem [{\citenamefont {Yu}\ \emph {et~al.}(2022)\citenamefont {Yu},
  \citenamefont {Naka}, \citenamefont {Aoki}, \citenamefont {Kato},
  \citenamefont {Yamashita}, \citenamefont {Fujii}, \citenamefont {Kato},
  \citenamefont {Fujigaya},\ and\ \citenamefont {Shiraki}}]{Yu:2022}%
  \BibitemOpen
  \bibfield  {author} {\bibinfo {author} {\bibfnamefont {B.}~\bibnamefont
  {Yu}}, \bibinfo {author} {\bibfnamefont {S.}~\bibnamefont {Naka}}, \bibinfo
  {author} {\bibfnamefont {H.}~\bibnamefont {Aoki}}, \bibinfo {author}
  {\bibfnamefont {K.}~\bibnamefont {Kato}}, \bibinfo {author} {\bibfnamefont
  {D.}~\bibnamefont {Yamashita}}, \bibinfo {author} {\bibfnamefont
  {S.}~\bibnamefont {Fujii}}, \bibinfo {author} {\bibfnamefont {Y.~K.}\
  \bibnamefont {Kato}}, \bibinfo {author} {\bibfnamefont {T.}~\bibnamefont
  {Fujigaya}},\ and\ \bibinfo {author} {\bibfnamefont {T.}~\bibnamefont
  {Shiraki}},\ }\bibfield  {title} {\bibinfo {title} {ortho-substituted
  aryldiazonium design for the defect configuration-controlled photoluminescent
  functionalization of chiral single-walled carbon nanotubes},\ }\href
  {https://doi.org/10.1021/acsnano.2c09897} {\bibfield  {journal} {\bibinfo
  {journal} {ACS Nano}\ }\textbf {\bibinfo {volume} {16}},\ \bibinfo {pages}
  {21452} (\bibinfo {year} {2022})}\BibitemShut {NoStop}%
\bibitem [{\citenamefont {Tan}\ \emph {et~al.}(2021)\citenamefont {Tan},
  \citenamefont {Rasmita}, \citenamefont {Li}, \citenamefont {Liu},
  \citenamefont {Huang}, \citenamefont {Xiong}, \citenamefont {Yang},
  \citenamefont {Novoselov},\ and\ \citenamefont {Gao}}]{Tan:2021}%
  \BibitemOpen
  \bibfield  {author} {\bibinfo {author} {\bibfnamefont {Q.}~\bibnamefont
  {Tan}}, \bibinfo {author} {\bibfnamefont {A.}~\bibnamefont {Rasmita}},
  \bibinfo {author} {\bibfnamefont {S.}~\bibnamefont {Li}}, \bibinfo {author}
  {\bibfnamefont {S.}~\bibnamefont {Liu}}, \bibinfo {author} {\bibfnamefont
  {Z.}~\bibnamefont {Huang}}, \bibinfo {author} {\bibfnamefont
  {Q.}~\bibnamefont {Xiong}}, \bibinfo {author} {\bibfnamefont {S.~A.}\
  \bibnamefont {Yang}}, \bibinfo {author} {\bibfnamefont {K.}~\bibnamefont
  {Novoselov}},\ and\ \bibinfo {author} {\bibfnamefont {W.-b.}\ \bibnamefont
  {Gao}},\ }\bibfield  {title} {\bibinfo {title} {Layer-engineered interlayer
  excitons},\ }\href {https://doi.org/DOI: 10.1126/sciadv.abh0863} {\bibfield
  {journal} {\bibinfo  {journal} {Sci. Adv.}\ }\textbf {\bibinfo {volume}
  {7}},\ \bibinfo {pages} {1} (\bibinfo {year} {2021})}\BibitemShut {NoStop}%
\bibitem [{\citenamefont {Bai}\ \emph {et~al.}(2020)\citenamefont {Bai},
  \citenamefont {Zhou}, \citenamefont {Wang}, \citenamefont {Wu}, \citenamefont
  {McGilly}, \citenamefont {Halbertal}, \citenamefont {Lo}, \citenamefont
  {Liu}, \citenamefont {Ardelean}, \citenamefont {Rivera}, \citenamefont
  {Finney}, \citenamefont {Yang}, \citenamefont {Basov}, \citenamefont {Yao},
  \citenamefont {Xu}, \citenamefont {Hone}, \citenamefont {Pasupathy},\ and\
  \citenamefont {Zhu}}]{Bai:2020}%
  \BibitemOpen
  \bibfield  {author} {\bibinfo {author} {\bibfnamefont {Y.}~\bibnamefont
  {Bai}}, \bibinfo {author} {\bibfnamefont {L.}~\bibnamefont {Zhou}}, \bibinfo
  {author} {\bibfnamefont {J.}~\bibnamefont {Wang}}, \bibinfo {author}
  {\bibfnamefont {W.}~\bibnamefont {Wu}}, \bibinfo {author} {\bibfnamefont
  {L.~J.}\ \bibnamefont {McGilly}}, \bibinfo {author} {\bibfnamefont
  {D.}~\bibnamefont {Halbertal}}, \bibinfo {author} {\bibfnamefont {C.~F.~B.}\
  \bibnamefont {Lo}}, \bibinfo {author} {\bibfnamefont {F.}~\bibnamefont
  {Liu}}, \bibinfo {author} {\bibfnamefont {J.}~\bibnamefont {Ardelean}},
  \bibinfo {author} {\bibfnamefont {P.}~\bibnamefont {Rivera}}, \bibinfo
  {author} {\bibfnamefont {N.~R.}\ \bibnamefont {Finney}}, \bibinfo {author}
  {\bibfnamefont {X.-C.}\ \bibnamefont {Yang}}, \bibinfo {author}
  {\bibfnamefont {D.~N.}\ \bibnamefont {Basov}}, \bibinfo {author}
  {\bibfnamefont {W.}~\bibnamefont {Yao}}, \bibinfo {author} {\bibfnamefont
  {X.}~\bibnamefont {Xu}}, \bibinfo {author} {\bibfnamefont {J.}~\bibnamefont
  {Hone}}, \bibinfo {author} {\bibfnamefont {A.~N.}\ \bibnamefont
  {Pasupathy}},\ and\ \bibinfo {author} {\bibfnamefont {X.}~\bibnamefont
  {Zhu}},\ }\bibfield  {title} {\bibinfo {title} {Excitons in strain-induced
  one-dimensional moir{\'e} potentials at transition metal dichalcogenide
  heterojunctions},\ }\href {https://doi.org/10.1038/s41563-020-0730-8}
  {\bibfield  {journal} {\bibinfo  {journal} {Nat. Mater.}\ }\textbf {\bibinfo
  {volume} {19}},\ \bibinfo {pages} {1068} (\bibinfo {year}
  {2020})}\BibitemShut {NoStop}%
\bibitem [{\citenamefont {Settele}\ \emph {et~al.}(2021)\citenamefont
  {Settele}, \citenamefont {Berger}, \citenamefont {Lindenthal}, \citenamefont
  {Zhao}, \citenamefont {El~Yumin}, \citenamefont {Zorn}, \citenamefont
  {Asyuda}, \citenamefont {Zharnikov}, \citenamefont {H{\"o}gele},\ and\
  \citenamefont {Zaumseil}}]{Settele:2021}%
  \BibitemOpen
  \bibfield  {author} {\bibinfo {author} {\bibfnamefont {S.}~\bibnamefont
  {Settele}}, \bibinfo {author} {\bibfnamefont {F.~J.}\ \bibnamefont {Berger}},
  \bibinfo {author} {\bibfnamefont {S.}~\bibnamefont {Lindenthal}}, \bibinfo
  {author} {\bibfnamefont {S.}~\bibnamefont {Zhao}}, \bibinfo {author}
  {\bibfnamefont {A.~A.}\ \bibnamefont {El~Yumin}}, \bibinfo {author}
  {\bibfnamefont {N.~F.}\ \bibnamefont {Zorn}}, \bibinfo {author}
  {\bibfnamefont {A.}~\bibnamefont {Asyuda}}, \bibinfo {author} {\bibfnamefont
  {M.}~\bibnamefont {Zharnikov}}, \bibinfo {author} {\bibfnamefont
  {A.}~\bibnamefont {H{\"o}gele}},\ and\ \bibinfo {author} {\bibfnamefont
  {J.}~\bibnamefont {Zaumseil}},\ }\bibfield  {title} {\bibinfo {title}
  {Synthetic control over the binding configuration of luminescent
  {sp$^3$}-defects in single-walled carbon nanotubes},\ }\href
  {https://doi.org/10.1038/s41467-021-22307-9} {\bibfield  {journal} {\bibinfo
  {journal} {Nat. Commun.}\ }\textbf {\bibinfo {volume} {12}},\ \bibinfo
  {pages} {2119} (\bibinfo {year} {2021})}\BibitemShut {NoStop}%
\bibitem [{\citenamefont {Shinokita}\ \emph {et~al.}(2021)\citenamefont
  {Shinokita}, \citenamefont {Miyauchi}, \citenamefont {Watanabe},
  \citenamefont {Taniguchi},\ and\ \citenamefont {Matsuda}}]{Shinokita:2021}%
  \BibitemOpen
  \bibfield  {author} {\bibinfo {author} {\bibfnamefont {K.}~\bibnamefont
  {Shinokita}}, \bibinfo {author} {\bibfnamefont {Y.}~\bibnamefont {Miyauchi}},
  \bibinfo {author} {\bibfnamefont {K.}~\bibnamefont {Watanabe}}, \bibinfo
  {author} {\bibfnamefont {T.}~\bibnamefont {Taniguchi}},\ and\ \bibinfo
  {author} {\bibfnamefont {K.}~\bibnamefont {Matsuda}},\ }\bibfield  {title}
  {\bibinfo {title} {Resonant coupling of a moir{\'e} exciton to a phonon in a
  {{WSe$_2$}}/{{MoSe$_2$}} heterobilayer},\ }\href
  {https://doi.org/10.1021/acs.nanolett.1c00733} {\bibfield  {journal}
  {\bibinfo  {journal} {Nano Lett.}\ }\textbf {\bibinfo {volume} {21}},\
  \bibinfo {pages} {5938} (\bibinfo {year} {2021})}\BibitemShut {NoStop}%
\bibitem [{\citenamefont {Nirmal}\ \emph {et~al.}(1996)\citenamefont {Nirmal},
  \citenamefont {Dabbousi}, \citenamefont {Bawendi}, \citenamefont {Macklin},
  \citenamefont {Trautman}, \citenamefont {Harris},\ and\ \citenamefont
  {Brus}}]{Nirmal:1996}%
  \BibitemOpen
  \bibfield  {author} {\bibinfo {author} {\bibfnamefont {M.}~\bibnamefont
  {Nirmal}}, \bibinfo {author} {\bibfnamefont {B.~O.}\ \bibnamefont
  {Dabbousi}}, \bibinfo {author} {\bibfnamefont {M.~G.}\ \bibnamefont
  {Bawendi}}, \bibinfo {author} {\bibfnamefont {J.}~\bibnamefont {Macklin}},
  \bibinfo {author} {\bibfnamefont {J.}~\bibnamefont {Trautman}}, \bibinfo
  {author} {\bibfnamefont {T.}~\bibnamefont {Harris}},\ and\ \bibinfo {author}
  {\bibfnamefont {L.~E.}\ \bibnamefont {Brus}},\ }\bibfield  {title} {\bibinfo
  {title} {Fluorescence intermittency in single cadmium selenide
  nanocrystals},\ }\href {https://doi.org/10.1038/383802a0} {\bibfield
  {journal} {\bibinfo  {journal} {Nature}\ }\textbf {\bibinfo {volume} {383}},\
  \bibinfo {pages} {802} (\bibinfo {year} {1996})}\BibitemShut {NoStop}%
\bibitem [{\citenamefont {Moerner}\ and\ \citenamefont
  {Orrit}(1999)}]{Moerner:1999}%
  \BibitemOpen
  \bibfield  {author} {\bibinfo {author} {\bibfnamefont {W.~E.}\ \bibnamefont
  {Moerner}}\ and\ \bibinfo {author} {\bibfnamefont {M.}~\bibnamefont
  {Orrit}},\ }\bibfield  {title} {\bibinfo {title} {Illuminating single
  molecules in condensed matter},\ }\href {https://doi.org/Illuminating single
  molecules in condensed matter} {\bibfield  {journal} {\bibinfo  {journal}
  {Science}\ }\textbf {\bibinfo {volume} {283}},\ \bibinfo {pages} {1670}
  (\bibinfo {year} {1999})}\BibitemShut {NoStop}%
\bibitem [{\citenamefont {Sallen}\ \emph {et~al.}(2009)\citenamefont {Sallen},
  \citenamefont {Tribu}, \citenamefont {Aichele}, \citenamefont {Andr{\'e}},
  \citenamefont {Besombes}, \citenamefont {Bougerol}, \citenamefont
  {Tatarenko}, \citenamefont {Kheng},\ and\ \citenamefont
  {Poizat}}]{Sallen:2009}%
  \BibitemOpen
  \bibfield  {author} {\bibinfo {author} {\bibfnamefont {G.}~\bibnamefont
  {Sallen}}, \bibinfo {author} {\bibfnamefont {A.}~\bibnamefont {Tribu}},
  \bibinfo {author} {\bibfnamefont {T.}~\bibnamefont {Aichele}}, \bibinfo
  {author} {\bibfnamefont {R.}~\bibnamefont {Andr{\'e}}}, \bibinfo {author}
  {\bibfnamefont {L.}~\bibnamefont {Besombes}}, \bibinfo {author}
  {\bibfnamefont {C.}~\bibnamefont {Bougerol}}, \bibinfo {author}
  {\bibfnamefont {S.}~\bibnamefont {Tatarenko}}, \bibinfo {author}
  {\bibfnamefont {K.}~\bibnamefont {Kheng}},\ and\ \bibinfo {author}
  {\bibfnamefont {J.~P.}\ \bibnamefont {Poizat}},\ }\bibfield  {title}
  {\bibinfo {title} {Exciton dynamics of a single quantum dot embedded in a
  nanowire},\ }\href {https://doi.org/10.1103/PhysRevB.80.085310} {\bibfield
  {journal} {\bibinfo  {journal} {Phys. Rev. B}\ }\textbf {\bibinfo {volume}
  {80}},\ \bibinfo {pages} {085310} (\bibinfo {year} {2009})}\BibitemShut
  {NoStop}%
\bibitem [{\citenamefont {Tran}\ \emph {et~al.}(2016)\citenamefont {Tran},
  \citenamefont {Bray}, \citenamefont {Ford}, \citenamefont {Toth},\ and\
  \citenamefont {Aharonovich}}]{Tran:2016}%
  \BibitemOpen
  \bibfield  {author} {\bibinfo {author} {\bibfnamefont {T.~T.}\ \bibnamefont
  {Tran}}, \bibinfo {author} {\bibfnamefont {K.}~\bibnamefont {Bray}}, \bibinfo
  {author} {\bibfnamefont {M.~J.}\ \bibnamefont {Ford}}, \bibinfo {author}
  {\bibfnamefont {M.}~\bibnamefont {Toth}},\ and\ \bibinfo {author}
  {\bibfnamefont {I.}~\bibnamefont {Aharonovich}},\ }\bibfield  {title}
  {\bibinfo {title} {Quantum emission from hexagonal boron nitride
  monolayers},\ }\href {https://doi.org/10.1038/nnano.2015.242} {\bibfield
  {journal} {\bibinfo  {journal} {Nat. Nanotechnol.}\ }\textbf {\bibinfo
  {volume} {11}},\ \bibinfo {pages} {37} (\bibinfo {year} {2016})}\BibitemShut
  {NoStop}%
\bibitem [{\citenamefont {Gao}\ \emph {et~al.}(2017)\citenamefont {Gao},
  \citenamefont {Solovev}, \citenamefont {Holmes}, \citenamefont {Arita},\ and\
  \citenamefont {Arakawa}}]{Gao:2017}%
  \BibitemOpen
  \bibfield  {author} {\bibinfo {author} {\bibfnamefont {K.}~\bibnamefont
  {Gao}}, \bibinfo {author} {\bibfnamefont {I.}~\bibnamefont {Solovev}},
  \bibinfo {author} {\bibfnamefont {M.}~\bibnamefont {Holmes}}, \bibinfo
  {author} {\bibfnamefont {M.}~\bibnamefont {Arita}},\ and\ \bibinfo {author}
  {\bibfnamefont {Y.}~\bibnamefont {Arakawa}},\ }\bibfield  {title} {\bibinfo
  {title} {Nanosecond-scale spectral diffusion in the single photon emission of
  a gan quantum dot},\ }\href {https://doi.org/10.1063/1.4997117} {\bibfield
  {journal} {\bibinfo  {journal} {AIP Adv.}\ }\textbf {\bibinfo {volume} {7}},\
  \bibinfo {pages} {125216} (\bibinfo {year} {2017})}\BibitemShut {NoStop}%
\bibitem [{\citenamefont {Ishii}\ \emph {et~al.}(2017)\citenamefont {Ishii},
  \citenamefont {Yoshida},\ and\ \citenamefont {Kato}}]{Ishii:2017}%
  \BibitemOpen
  \bibfield  {author} {\bibinfo {author} {\bibfnamefont {A.}~\bibnamefont
  {Ishii}}, \bibinfo {author} {\bibfnamefont {M.}~\bibnamefont {Yoshida}},\
  and\ \bibinfo {author} {\bibfnamefont {Y.~K.}\ \bibnamefont {Kato}},\
  }\bibfield  {title} {\bibinfo {title} {Room-temperature single-photon
  emission from micrometer-long air-suspended carbon nanotubes},\ }\href
  {https://doi.org/10.1103/physrevapplied.8.054039} {\bibfield  {journal}
  {\bibinfo  {journal} {Phys. Rev. Applied}\ }\textbf {\bibinfo {volume} {8}},\
  \bibinfo {pages} {054039} (\bibinfo {year} {2017})}\BibitemShut {NoStop}%
\bibitem [{\citenamefont {Babinec}\ \emph {et~al.}(2010)\citenamefont
  {Babinec}, \citenamefont {Hausmann}, \citenamefont {Khan}, \citenamefont
  {Zhang}, \citenamefont {Maze}, \citenamefont {Hemmer},\ and\ \citenamefont
  {Lon{\v{c}}ar}}]{Babinec:2010}%
  \BibitemOpen
  \bibfield  {author} {\bibinfo {author} {\bibfnamefont {T.~M.}\ \bibnamefont
  {Babinec}}, \bibinfo {author} {\bibfnamefont {B.~J.}\ \bibnamefont
  {Hausmann}}, \bibinfo {author} {\bibfnamefont {M.}~\bibnamefont {Khan}},
  \bibinfo {author} {\bibfnamefont {Y.}~\bibnamefont {Zhang}}, \bibinfo
  {author} {\bibfnamefont {J.~R.}\ \bibnamefont {Maze}}, \bibinfo {author}
  {\bibfnamefont {P.~R.}\ \bibnamefont {Hemmer}},\ and\ \bibinfo {author}
  {\bibfnamefont {M.}~\bibnamefont {Lon{\v{c}}ar}},\ }\bibfield  {title}
  {\bibinfo {title} {A diamond nanowire single-photon source},\ }\href
  {https://doi.org/https://doi.org/10.1038/nnano.2010.6} {\bibfield  {journal}
  {\bibinfo  {journal} {Nat. Nanotechnol.}\ }\textbf {\bibinfo {volume} {5}},\
  \bibinfo {pages} {195} (\bibinfo {year} {2010})}\BibitemShut {NoStop}%
\bibitem [{\citenamefont {Castelletto}\ \emph {et~al.}(2014)\citenamefont
  {Castelletto}, \citenamefont {Johnson}, \citenamefont {Iv{\'a}dy},
  \citenamefont {Stavrias}, \citenamefont {Umeda}, \citenamefont {Gali},\ and\
  \citenamefont {Ohshima}}]{Castelletto:2014}%
  \BibitemOpen
  \bibfield  {author} {\bibinfo {author} {\bibfnamefont {S.}~\bibnamefont
  {Castelletto}}, \bibinfo {author} {\bibfnamefont {B.~C.}\ \bibnamefont
  {Johnson}}, \bibinfo {author} {\bibfnamefont {V.}~\bibnamefont {Iv{\'a}dy}},
  \bibinfo {author} {\bibfnamefont {N.}~\bibnamefont {Stavrias}}, \bibinfo
  {author} {\bibfnamefont {T.}~\bibnamefont {Umeda}}, \bibinfo {author}
  {\bibfnamefont {A.}~\bibnamefont {Gali}},\ and\ \bibinfo {author}
  {\bibfnamefont {T.}~\bibnamefont {Ohshima}},\ }\bibfield  {title} {\bibinfo
  {title} {A silicon carbide room-temperature single-photon source},\ }\href
  {https://doi.org/10.1038/nmat3806} {\bibfield  {journal} {\bibinfo  {journal}
  {Nat. Mater.}\ }\textbf {\bibinfo {volume} {13}},\ \bibinfo {pages} {151}
  (\bibinfo {year} {2014})}\BibitemShut {NoStop}%
\bibitem [{\citenamefont {Ma}\ \emph {et~al.}(2015)\citenamefont {Ma},
  \citenamefont {Hartmann}, \citenamefont {Baldwin}, \citenamefont {Doorn},\
  and\ \citenamefont {Htoon}}]{Ma:2015NatNano}%
  \BibitemOpen
  \bibfield  {author} {\bibinfo {author} {\bibfnamefont {X.}~\bibnamefont
  {Ma}}, \bibinfo {author} {\bibfnamefont {N.~F.}\ \bibnamefont {Hartmann}},
  \bibinfo {author} {\bibfnamefont {J.~K.~S.}\ \bibnamefont {Baldwin}},
  \bibinfo {author} {\bibfnamefont {S.~K.}\ \bibnamefont {Doorn}},\ and\
  \bibinfo {author} {\bibfnamefont {H.}~\bibnamefont {Htoon}},\ }\bibfield
  {title} {\bibinfo {title} {Room-temperature single-photon generation from
  solitary dopants of carbon nanotubes},\ }\href
  {https://doi.org/10.1038/nnano.2015.136} {\bibfield  {journal} {\bibinfo
  {journal} {Nat. Nanotechnol.}\ }\textbf {\bibinfo {volume} {10}},\ \bibinfo
  {pages} {671} (\bibinfo {year} {2015})}\BibitemShut {NoStop}%
\bibitem [{\citenamefont {Linhart}\ \emph {et~al.}(2019)\citenamefont
  {Linhart}, \citenamefont {Paur}, \citenamefont {Smejkal}, \citenamefont
  {Burgd{\"o}rfer}, \citenamefont {Mueller},\ and\ \citenamefont
  {Libisch}}]{Linhart:2019}%
  \BibitemOpen
  \bibfield  {author} {\bibinfo {author} {\bibfnamefont {L.}~\bibnamefont
  {Linhart}}, \bibinfo {author} {\bibfnamefont {M.}~\bibnamefont {Paur}},
  \bibinfo {author} {\bibfnamefont {V.}~\bibnamefont {Smejkal}}, \bibinfo
  {author} {\bibfnamefont {J.}~\bibnamefont {Burgd{\"o}rfer}}, \bibinfo
  {author} {\bibfnamefont {T.}~\bibnamefont {Mueller}},\ and\ \bibinfo {author}
  {\bibfnamefont {F.}~\bibnamefont {Libisch}},\ }\bibfield  {title} {\bibinfo
  {title} {Localized intervalley defect excitons as single-photon emitters in
  {WSe$_2$}},\ }\href {https://doi.org/10.1103/PhysRevLett.123.146401}
  {\bibfield  {journal} {\bibinfo  {journal} {Phys. Rev. Lett.}\ }\textbf
  {\bibinfo {volume} {123}},\ \bibinfo {pages} {146401} (\bibinfo {year}
  {2019})}\BibitemShut {NoStop}%
\bibitem [{\citenamefont {Zhang}\ \emph {et~al.}(2017)\citenamefont {Zhang},
  \citenamefont {Wang}, \citenamefont {Li}, \citenamefont {Huang},
  \citenamefont {Li}, \citenamefont {Ji},\ and\ \citenamefont
  {Wu}}]{Zhang:2017}%
  \BibitemOpen
  \bibfield  {author} {\bibinfo {author} {\bibfnamefont {S.}~\bibnamefont
  {Zhang}}, \bibinfo {author} {\bibfnamefont {C.-G.}\ \bibnamefont {Wang}},
  \bibinfo {author} {\bibfnamefont {M.-Y.}\ \bibnamefont {Li}}, \bibinfo
  {author} {\bibfnamefont {D.}~\bibnamefont {Huang}}, \bibinfo {author}
  {\bibfnamefont {L.-J.}\ \bibnamefont {Li}}, \bibinfo {author} {\bibfnamefont
  {W.}~\bibnamefont {Ji}},\ and\ \bibinfo {author} {\bibfnamefont
  {S.}~\bibnamefont {Wu}},\ }\bibfield  {title} {\bibinfo {title} {Defect
  structure of localized excitons in a {WSe$_2$} monolayer},\ }\href
  {https://doi.org/10.1103/PhysRevLett.119.046101} {\bibfield  {journal}
  {\bibinfo  {journal} {Phys. Rev. Lett.}\ }\textbf {\bibinfo {volume} {119}},\
  \bibinfo {pages} {046101} (\bibinfo {year} {2017})}\BibitemShut {NoStop}%
\bibitem [{\citenamefont {Parto}\ \emph {et~al.}(2021)\citenamefont {Parto},
  \citenamefont {Azzam}, \citenamefont {Banerjee},\ and\ \citenamefont
  {Moody}}]{parto:2021}%
  \BibitemOpen
  \bibfield  {author} {\bibinfo {author} {\bibfnamefont {K.}~\bibnamefont
  {Parto}}, \bibinfo {author} {\bibfnamefont {S.~I.}\ \bibnamefont {Azzam}},
  \bibinfo {author} {\bibfnamefont {K.}~\bibnamefont {Banerjee}},\ and\
  \bibinfo {author} {\bibfnamefont {G.}~\bibnamefont {Moody}},\ }\bibfield
  {title} {\bibinfo {title} {Defect and strain engineering of monolayer
  {WSe$_2$} enables site-controlled single-photon emission up to 150 {K}},\
  }\href {https://doi.org/https://doi.org/10.1038/s41467-021-23709-5}
  {\bibfield  {journal} {\bibinfo  {journal} {Nat. Commun.}\ }\textbf {\bibinfo
  {volume} {12}},\ \bibinfo {pages} {3585} (\bibinfo {year}
  {2021})}\BibitemShut {NoStop}%
\bibitem [{\citenamefont {Shin}\ \emph {et~al.}(2016)\citenamefont {Shin},
  \citenamefont {Han}, \citenamefont {Yun}, \citenamefont {Oh}, \citenamefont
  {Bae}, \citenamefont {Song}, \citenamefont {Park},\ and\ \citenamefont
  {Lee}}]{Shin:2016}%
  \BibitemOpen
  \bibfield  {author} {\bibinfo {author} {\bibfnamefont {B.~G.}\ \bibnamefont
  {Shin}}, \bibinfo {author} {\bibfnamefont {G.~H.}\ \bibnamefont {Han}},
  \bibinfo {author} {\bibfnamefont {S.~J.}\ \bibnamefont {Yun}}, \bibinfo
  {author} {\bibfnamefont {H.~M.}\ \bibnamefont {Oh}}, \bibinfo {author}
  {\bibfnamefont {J.~J.}\ \bibnamefont {Bae}}, \bibinfo {author} {\bibfnamefont
  {Y.~J.}\ \bibnamefont {Song}}, \bibinfo {author} {\bibfnamefont {C.-Y.}\
  \bibnamefont {Park}},\ and\ \bibinfo {author} {\bibfnamefont {Y.~H.}\
  \bibnamefont {Lee}},\ }\bibfield  {title} {\bibinfo {title} {Indirect bandgap
  puddles in monolayer {MoS$_2$} by substrate-induced local strain},\ }\href
  {https://doi.org/10.1002/adma.201602626} {\bibfield  {journal} {\bibinfo
  {journal} {Adv. Mater.}\ }\textbf {\bibinfo {volume} {28}},\ \bibinfo {pages}
  {9378} (\bibinfo {year} {2016})}\BibitemShut {NoStop}%
\bibitem [{\citenamefont {Zhao}\ \emph {et~al.}(2023)\citenamefont {Zhao},
  \citenamefont {Li}, \citenamefont {Huang}, \citenamefont {Rupp},
  \citenamefont {G{\"o}ser}, \citenamefont {Vovk}, \citenamefont {Kruchinin},
  \citenamefont {Watanabe}, \citenamefont {Taniguchi}, \citenamefont {Bilgin},
  \citenamefont {Baimuratov},\ and\ \citenamefont {H{\"o}gele}}]{Zhao:2023}%
  \BibitemOpen
  \bibfield  {author} {\bibinfo {author} {\bibfnamefont {S.}~\bibnamefont
  {Zhao}}, \bibinfo {author} {\bibfnamefont {Z.}~\bibnamefont {Li}}, \bibinfo
  {author} {\bibfnamefont {X.}~\bibnamefont {Huang}}, \bibinfo {author}
  {\bibfnamefont {A.}~\bibnamefont {Rupp}}, \bibinfo {author} {\bibfnamefont
  {J.}~\bibnamefont {G{\"o}ser}}, \bibinfo {author} {\bibfnamefont {I.~A.}\
  \bibnamefont {Vovk}}, \bibinfo {author} {\bibfnamefont {S.~Y.}\ \bibnamefont
  {Kruchinin}}, \bibinfo {author} {\bibfnamefont {K.}~\bibnamefont {Watanabe}},
  \bibinfo {author} {\bibfnamefont {T.}~\bibnamefont {Taniguchi}}, \bibinfo
  {author} {\bibfnamefont {I.}~\bibnamefont {Bilgin}}, \bibinfo {author}
  {\bibfnamefont {A.~S.}\ \bibnamefont {Baimuratov}},\ and\ \bibinfo {author}
  {\bibfnamefont {A.}~\bibnamefont {H{\"o}gele}},\ }\bibfield  {title}
  {\bibinfo {title} {Excitons in mesoscopically reconstructed moir{\'e}
  heterostructures},\ }\href {https://doi.org/10.1038/s41565-023-01356-9}
  {\bibfield  {journal} {\bibinfo  {journal} {Nat. Nanotechnol.}\ }\textbf
  {\bibinfo {volume} {18}},\ \bibinfo {pages} {572} (\bibinfo {year}
  {2023})}\BibitemShut {NoStop}%
\end{thebibliography}%


\end{document}