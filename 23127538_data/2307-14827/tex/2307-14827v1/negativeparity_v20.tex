%%%%%%%%%%%%%%%%%%%%%%%%%%%%%%%%%%%%%%%%%%%%%%%%%%%%%%%%%%%%%%%%%%%%%%%
\documentclass[twocolumn,aps,prc,superscriptaddress]{revtex4-2}
%\documentclass[twocolumn,a4,showpacs,preprintnumbers]{revtex4-2}
%\documentclass[twocolumn,showpacs,preprintnumbers,amsmath,amssymb,prc]{revtex4}
%\documentstyle[aps,twocolumn,psfig]{revtex}
%\documentstyle[prl,aps]{revtex}  %% Journal Style.
%%changing to preprint, three modifications is needed.
%\documentclass[preprint,aps]{revtex4}  %% Draft Style
%\usepackage{epsfig}
\usepackage{color,amsmath,amssymb,epsfig,graphicx}
%\documentclass[aps,twocolumn,showpacs,psfig]
%\tightenlines
%\usepackage{bm}
%\usepackage{dcolumn}% Align table columns on decimal point
\usepackage{bm}% bold math
%\usepackage{amsmath}
\usepackage[colorlinks=true,linkcolor=blue,filecolor=blue,urlcolor=blue,citecolor=blue]{hyperref}
\usepackage{mathptmx, courier, pifont}
\usepackage[scaled=0.92]{helvet}
\usepackage[T1]{fontenc}
\usepackage{textcomp}
%\usepackage[ps2pdf,colorlinks=true]{hyperref}
\usepackage{multirow}
\usepackage{longtable}
\newcommand{\simgeq}{\; \raisebox{-0.4ex}{\tiny$\stackrel
{{\textstyle>}}{\sim}$}\;}
\newcommand{\simleq}{\; \raisebox{-0.4ex}{\tiny$\stackrel
%\renewcommand*\appendixpagename{\Large Appendices}    
{{\textstyle<}}{\sim}$}\;}
%\newcommand{\intercal}{T}
%\newcommand{\text}{\rm}
\def\bra{\,<\!} \def\ket{\!>\,} \def\ack{\,|\,}
%
\begin{document}
%\begin{frontmatter}
%\title{ E2 Transitions and Shape Invariants}
%\title{ Microscopic investigation of $E2$ matrix elements in atomic nuclei }
\title{Triaxial projected shell model approach for negative parity states in even-even nuclei}
\author{ Nazira Nazir}
\affiliation{Department of Physics, University of Kashmir, Srinagar, 190 006, India}
\author{S.~Jehangir}
\affiliation{ Department of  Physics, Islamic University of  Science and Technology, Awantipora, 192 122, India}
\author{ S.P. Rouoof}
\affiliation{ Department of  Physics, Islamic University of  Science and Technology, Awantipora, 192 122, India}
\author{ G.H.~Bhat}
\affiliation{Department of Physics, SP College  Srinagar, Jammu and Kashmir, 190 001, India}
\affiliation{Cluster University Srinagar, Jammu and Kashmir, Srinagar, Goji Bagh, 190 008, India}
\author{ J.A. Sheikh}
\affiliation{Department of Physics, University of Kashmir, Srinagar, 190 006, India}
\author{ N. Rather}
\affiliation{ Department of  Physics, Islamic University of  Science and Technology, Awantipora, 192 122, India}
\author{Manzoor A. Malik}
  \affiliation{Department of Physics, University of Kashmir, Srinagar, 190 006, India}
  \affiliation{ Department of  Physics, Islamic University of  Science and Technology, Awantipora, 192 122, India}

\date{\today}

\begin{abstract}

The triaxial projected shell model (TPSM) approach is generalized to investigate the negative parity band structures in even-even systems. In the earlier version of the TPSM approach, the quasiparticle excitations were restricted to one major oscillator shell and it was possible to study only $+$ive parity states in even-even systems. In the present extension, the excited  quasiparticles are allowed to occupy two major oscillator shells, which makes it possible to generate the $-$ive parity states. As a  major application of this  development, the extended approach is applied to elucidate the $-$ive parity high-spin band structures in $^{102-112}$Ru. It is shown that TPSM approach provides a reasonable description of the observed properties. 

\end{abstract}
\maketitle
%% \begin{keyword}
%%   \sep Chiral symmetry
%%   \sep quasiparticle excitations
%%  \sep triaxial projected shell model
%% \PACS 21.60.Cs, 21.10.Hw, 21.10.Ky, 27.50.+e
%% \end{keyword}
%\end{frontmatter}

\section{Introduction}

To characterize the rich band structures observed in atomic nuclei is one of the
main research themes in nuclear structure physics \cite{BMII,sf01}. Major advancements in the experimental techniques have made it feasible to populate multiple high-spin band structures, and in some nuclei more than fifty band structures have been reported \cite{simpson23,ollier11}. The description of this wealth of nuclear structure information is a major challenge to nuclear structure models \cite{SGH16}. In recent years, tremendous progress has been made in the spherical shell model (SSM) description of the nuclear properties \cite{otuska20,Brown22,Poves12}. It is now possible to apply SSM approach to medium mass nuclei, but studying high-spin band structures in heavy-mass region is still  beyond the scope of this microscopic model. To describe the high-spin band structures, it is imperative to include, atleast, two-major oscillator shells as aligning particles occupy the high-j intruder orbitals. To perform the SSM calculations with two-major oscillator shells
for heavier nuclei is beyond the reach of computational resources presently available \cite{langanke2012}.

On the other hand, although several major oscillator shells are considered in density functional approaches, but most of the calculations are restricted to investigate the ground-state properties only \cite{JAS21}. In order to study the high-spin band structures, the angular-momentum projection is required to be performed from the intrinsic mean-field state \cite{ring80}. However, this approach is plagued with the singularity problem due to the reason that most of the modern energy density functionals are fitted to the experimental data
with fractional density
dependence and employ different forces in particle-hole and particle-particle channels \cite{Bender09,Duguet09,dobacz07}. It is to be added that in some recent works \cite{VRETENAR05,meng,wang2023}, the angular-momentum projection has also been performed in density functional theory (DFT) with projection after variation and the singularity problem does not appear to show up in these studies. However, it has been discussed in Ref.~\cite{JAS21} that projected results will contain spurious components that need to be examined. 

Considering the above problems associated with the SSM and DFT approaches, triaxial projected shell model (TPSM)  approach has become a tool  of choice to investigate the high-spin band structures in well deformed and transitional nuclei \cite{JS99,nazira22,jeh21,ishfaq20,jehangir12}. The advantage of this approach is that computational resources involved are quite modest and it is possible to perform a systematic study of a large set of atomic nuclei. As a matter of fact, several systematic investigations have been performed for chiral, wobbling and $\gamma$-vibrational band structures observed in triaxial nuclei \cite{SGH16,nazira23,nazira22,jeh21,ishfaq20,jehangir12,SGH16,JG12}. The model space in the TPSM approach is spanned by  multi-quasiparticle basis states which allows to investigate high-spin band structures. In the original version of the TPSM approach, the model space was quite limited \cite{JS99} but in recent applications \cite{jehangir12,jeh21,nazira22,nazira23,JS21}, we have generalized the basis space to include higher order quasiparticle states. For instance, for even-even systems \cite{jehangir12}, the TPSM approach has been generalized to include four-neutron and four-proton quasiparticle basis states. This extension allows to investigate the high-spin properties in even-even systems beyond the second band crossing.

Nevertheless, in all the extended versions of the TPSM approach the basis configurations are constructed from one major oscillator shell only, although the vacuum configuration is generated from all the three major shells considered in the model.
%As a matter of fact, this restriction has been forced in all the earlier versions of the projected shell model approach (PSM) \cite{KY95}.
The justification is that the aligning particles occupy high-j intruder subshell and in order to describe band crossing, it is sufficient to consider quasiparticle excitations only from one major shell containing the intruder orbital.

For even-even systems, the restriction of the quasiparticle excitations from one oscillator shell gives rise to only $+$ive parity states and in order to generate the $-$ive parity states, the quasiparticle excitations need to be considered from two oscillator shells having different parities for the single-particle states. The purpose of the present work is to develop the generalized TPSM approach with the quasiparticle excitations from two major oscillator shells. There is a considerable data available for $-$ive parity bands in even-even systems \cite{NPRU,NPRU2,NPRU3,NP4,NP5,NP6,Che04,Jiang03,dejb95}. However, there have been very limited theoretical calculations to investigate these band structures.

In the present work, we shall focus on the application of the new development to neutron-rich nuclei around A $ \sim $ 110. The properties of these nuclei are studied by measuring prompt $\gamma$-rays emitted by secondary fragments produced by spontaneous and induced fission of $^{252}$Cf source \cite{NPRU,ZHU07}. The weak transitions in the excited $-$ive parity bands are identified through triple and higher order coincidence techniques \cite{SNY13} using the state-of-the-art detector arrays. The ground-state $+$ive parity bands in this region are known to have strong prolate shapes \cite{Fotiades1997}, and the $-$ive parity doublet bands identified in some nuclei are proposed to originate from chiral symmetry breaking mechanism \cite{NPRU}. In our earlier publications, we have  studied the $+$ive parity bands in this mass region \cite{JS21,nazira23,Chanli15} using the TPSM approach and in the present work, we shall focus on the $-$ive parity band structures. Some preliminary results of the present approach for the observed $-$ive parity band structures in $^{106,108}$Mo were published with the experimental group \cite{Musangu18}. However, in this work only neutron-excitation were considered. In the present work, both neutron- and proton-excitations are included in the model space.  The remaining manuscript is organized in the following manner. In the next section, we provide a few details of the extended TPSM approach for $-$ive parity bands and some explicit expressions of the matrix elements are included in the appendix. In section III, the results obtained for Ru-isotopes are presented and discussed, and finally the present work is summarized in section IV.

%===============  table 1  ========================
\begin{table}[htp!]
\LTcapwidth=0.4\textwidth
\caption{ Axial and triaxial quadrupole deformation parameters
$\epsilon$ and $\gamma = \textrm{tan}^{-1}{(\epsilon'/\epsilon)}$  employed in the TPSM calculation. Axial 
deformations $\epsilon$ have been considered from \cite{moller08} with some adjustment as discussed in the text. The nonaxial values ($\gamma$) 
are chosen in such a way that observed data is reproduced. }
\resizebox{1\columnwidth}{!}
  {
\begin{tabular}{ccccccc}%{M{2cm}M{2cm}M{2cm}M{2cm}M{2cm}M{2cm}M{2cm}M{2cm}M{2cm}M{2cm}N}
  \hline\hline
  &$^{102}$Ru & $^{104}$Ru  & $^{106}$Ru  & $^{108}$Ru  & $^{110}$Ru  & $^{112}$Ru\\
\hline $\epsilon$ &0.220     & 0.270       & 0.280      &  0.275     &  0.275    &0.270 \\
      $\gamma$    &30$^{\circ}$& 26 $^{\circ}$&  25$^{\circ}$& 26$^{\circ}$& 30$^{\circ}$& 30$^{\circ}$\\\hline\hline
%\end{tabular}\label{tab1}
%\end{table}
\end{tabular}\label{tab1}
}
\end{table}
%\vspace{-0.5cm}
%====================================================


\section{Triaxial Projected Shell Model Approach}
TPSM approach is similar to the SSM technique with the difference that deformed basis are used instead of the spherical one's \cite{JS99,KY95}.
The deformed basis are the optimum basis states to study deformed nuclei, and in the TPSM approach these are generated by solving the three-dimensional Nilsson
mean-field potential \cite{nilsson95}. The pairing interaction is then considered in the Bardeen-Cooper-Scrieffer (BCS)  approximation \cite{ring80}.
The Nilsson + BCS wavefunctions thus constructed form the basis configuration in the TPSM approach. The vacuum state is then constructed by considering valence neutrons and protons to occupy three major oscillator shells \cite{JS99}. However, the quasiparticle excitations are considered from one major oscillator shell only. For instance, to investigate the high-spin band structures in mass $\sim$ 110 region, quasiproton (quasineutron) excitations are considered from the N = 4 (5) shells which contain the $1g_{9/2}(1h_{11/2})$ shell that is responsible for proton (neutron) alignments in this region. This restriction allows to study only $+$ive parity band structures in even-even systems and in order to describe the $-$ive parity band structures, valence pair of particles need to be placed in two different oscillator shells having opposite parities.


In the present work, we have generalized the TPSM approach with valence neutrons and protons occupying  different shells. The extended basis space is composed of:
%===========fig1=====================1===================
% Figure environment removed
%=========================================================
%========================alignment====2=========
% Figure environment removed
\begin{eqnarray}
 % \label{basis}
%\hat P^I_{MK}\ack\Phi\ket;\\\nonumber
%~~\hat P^I_{MK}~a^\dagger_{n} a^\dagger_{n^\prime} \ack\Phi\ket;\\\nonumber
%~~\hat P^I_{MK}~a^\dagger_{n} a^\dagger_{n_1^\prime} a^\dagger_{n_2^\prime} a^\dagger_{n_3^\prime}\ack\Phi\ket;\\\nonumber
%  ~~\hat P^I_{MK}~a^\dagger_{n} a^\dagger_{n^\prime} a^\dagger_{p_1^\prime} a^\dagger_{p_2^\prime} \ack\Phi\ket; \\\nonumber
%~~\hat P^I_{MK}~a^\dagger_{p} a^\dagger_{p^\prime} \ack\Phi\ket;\\\nonumber
%~~\hat P^I_{MK}~a^\dagger_{p} a^\dagger_{p_1^\prime} a^\dagger_{p_2^\prime} a^\dagger_{p_3^\prime}\ack\Phi\ket;\\\nonumber
  %  ~~\hat P^I_{MK}~a^\dagger_{p} a^\dagger_{p^\prime} a^\dagger_{n_1^\prime} a^\dagger_{n_2^\prime} \ack\Phi\ket; \\\nonumber
  \hat P^I_{MK}~a^\dagger_{n_{1}} a^\dagger_{n_{2}^\prime} |\Phi\rangle;\nonumber \\ 
\hat P^I_{MK}~a^\dagger_{n_{1}} a^\dagger_{n_2^\prime} a^\dagger_{n_3^\prime} a^\dagger_{n_4^\prime} |\Phi\rangle;\nonumber\\
\hat P^I_{MK}~a^\dagger_{n_{1}} a^\dagger_{n_2^\prime} a^\dagger_{p_1^\prime} a^\dagger_{p_2^{\prime}} |\Phi\rangle; \nonumber\\
\hat P^I_{MK}~a^\dagger_{p_1} a^\dagger_{p_{2}^\prime} |\Phi\rangle;\nonumber \\
\hat P^I_{MK}~a^\dagger_{p_{1}} a^\dagger_{p_{2}^\prime} a^\dagger_{p_3^\prime} a^\dagger_{p_4^\prime} |\Phi\rangle;\nonumber\\ 
 \hat P^I_{MK}~a^\dagger_{p_1} a^\dagger_{p_2^\prime} a^\dagger_{n_1^\prime} a^\dagger_{n_2^\prime} |\Phi\rangle,\label{basis}
\end{eqnarray}
%===================================================
%===========alignm======================3==================
% Figure environment removed
%++++++++++++++++++++++++++++++++++++++++++++++++++++++++++++++++
where the neutron (proton) major oscillator shells employed are
designated by the quantum numbers $n$ ($p$) with neutrons (protons)
occupying two different oscillator shells, $n (p)$ and $n^\prime (p^\prime)$. We have considered excitations in both proton and neutron sectors, however, the separable interaction employed doesn't mix these excitations and the two can be diagonalized separately. This is shown in the appendix, where matrix elements between some elementary projected states of Eq.~(\ref{basis}) are included.

It has been demonstrated that two-neutron and two-proton excitations are almost at the same energy \cite{NPRU2}, it is therefore necessary to consider both neutron and proton excitations in the TPSM basis. Further, two-neutron and two-proton aligning configurations have been added as $-$ive parity bands in some nuclei have been populated up to quite high-spin and the band crossing phenomena have been observed. $\ack\Phi\rangle$ in Eq.~(\ref{basis}) is the quasiparticle vacuum state which has $+$ive parity, and the three-dimensional angular-momentum projection operator, $\hat P^I_{MK}$, is given by \cite{ring80,HS79,HS80}
\begin{equation}
\hat P^I_{MK} = {2I+1 \over 8\pi^2} \int~d\Omega\,
D^{I}_{MK}(\Omega)\, \hat R(\Omega),
\label{PD}
\end{equation}
with the rotation operator
\begin{eqnarray}
\hat R(\Omega) = e^{-\imath \alpha \hat J_z} e^{-\imath \beta \hat J_y}
e^{-\imath \gamma \hat J_z}~~~.\label{rotop}
\end{eqnarray}
Here, $''\Omega''$ represents a set of Euler angles 
($\alpha, \gamma = [0,2\pi],\, \beta= [0, \pi]$) and the 
$\hat{J}^{,}s$ are angular-momentum operators. 

In the present work we have employed $N=3,4,5$ ($2,3,4$)  for neutrons
(protons). The two valence neutrons (protons)  are occupying N= 4 (3) and 5 (4) shells that give rise to the $-$ive parity states. The deformations used to
generate the Nilsson basis configuration are given in Table \ref{tab1} and have been adopted from earlier works \cite{JS21,Chanli15,moller08}. The Nilsson intrinsic states are then projected onto the states with good angular-momentum through  three-dimensional projection.

%========================alignment====4=========
% Figure environment removed
%===================================================
The projected basis states of Eq.~(\ref{basis})
are then used to diagonalize the shell model Hamiltonian. As in our earlier studies, we have employed the pairing plus quadrupole-quadrupole Hamiltonian \cite{JG12,bh14}
\begin{equation}
\hat H = \hat H_0 - {1 \over 2} \chi \sum_\mu \hat Q^\dagger_\mu
\hat Q^{}_\mu - G_M \hat P^\dagger \hat P - G_Q \sum_\mu \hat
P^\dagger_\mu\hat P^{}_\mu .
\label{hamham}
\end{equation}
The corresponding triaxial Nilsson Hamiltonian is the mean-field of
the above model Hamiltonian and is given by:
\begin{equation}
\hat H_N = \hat H_0 - {2 \over 3}\hbar\omega\left\{\epsilon\hat Q_0
+\epsilon'{{\hat Q_{+2}+\hat Q_{-2}}\over\sqrt{2}}\right\}.
\label{nilsson}
\end{equation}
In the above equation, $\hat H_0$ is the spherical single-particle
Nilsson Hamiltonian \cite{Ni69}. The monopole pairing strength $G_M$ 
of the standard form
%===========alignm======================5==================
% Figure environment removed
%=================================================
\begin{eqnarray}
G_M = {{(G_1 \mp G_2{{N-Z}\over A})}\frac{1}{ A}} (MeV).\label{pairing}
%&&G_M = {G_1 \over A} ~{\rm for~protons.} \label{pairing}
\end{eqnarray}
%+++++++=============figure============4===========
% Figure environment removed
%===================================================
In the present calculation, we considered $G_1=22.68$ and $G_2=16.22$, which approximately reproduce the observed odd-even mass difference in the studied mass region.
%This choice of $G_M$ is appropriate for the single-particle space employed in the model, where three major shells are used for each type of nucleons.
%: i.e. $N=3,4,5$ for neutrons and $=2,3,4$ for protons
The quadrupole pairing strength $G_Q$ is
assumed to be proportional to $G_M$, and the proportionality
constant being fixed as 0.18. These interaction strengths are
consistent with those used earlier for the same mass region
\cite{Chanli15,Chanli16}.

The projected TPSM  wave function is then given by
\begin{equation}
|\sigma, I M \rangle = \sum_{K,\kappa} f^\sigma_\kappa \hat{P}^I_{MK} | \phi_\kappa \rangle .
\label{wave1} 
\end{equation}
Here, the index $\sigma$ labels the states with same angular momentum and $\kappa$
the basis states. In Eq.~(\ref{wave1}), $f^{\sigma}_{\kappa} $ are the weights of the basis state $\phi_\kappa$.

Finally, the minimization of the projected energy with respect to the expansion coefficient,
$f^\sigma_{\kappa}$, leads to the Hill-Wheeler type equation
\begin{equation}
\sum_{\kappa '} (H_{\kappa \kappa'} - E_\sigma N_{\kappa \kappa'} ) f^{\sigma}_{
\kappa'} = 0 ,
\end{equation}
where the normalization is chosen such that
\begin{equation}
\sum_{\kappa \kappa'}f^\sigma_\kappa N_{\kappa \kappa'} f^{\sigma'}_{\kappa'}
= \delta_{\sigma \sigma'}.
\end{equation}
The above equations are then solved to obtain the energies and the wavefunctions \cite{KY95}.

%In the present work, we have also evaluated the transition
%probabilities using the TPSM wavefunctions. The details of the
%transition calculations with explicit expressions are given in the review
%article \cite{SGH16}.


%% Figure environment removed




\section{Results and Discussion}

In comparison to the $+$ive parity bands, there have been only a few theoretical studies to investigate
the $-$ive parity bands in Ru-isotopes and other isotopes in the A $\sim$ 110 mass region. The band heads of the
two-quasiparticle structures have been studied using the D1S Gogny force \cite{NPRU2}, and it has been discussed
that two-proton and two-neutron bands for Ru-isotopes are at a similar excitation energy of about 2 MeV. In the self-consistent
constrained cranking Skyrme calculations with particle number conserving pairing \cite{Dai19}, it
has been shown that calculated moments of inertia
of two-neutron quasiparticle configuration are in better agreement with the experimental data as compared to the two-proton configuration
for $^{108,110,112}$Ru isotopes, and the observed bands have been characterized as neutron excited bands.
%==================108RU========5============
%%%%%%%%%%%%%%%%%%%%%%%%%%%%%%%%%%108ru%%%%%%%
% Figure environment removed

%========================alignment==6===========
% Figure environment removed
%=================alingn protron=============7
% Figure environment removed
%+=========================================8===========
% Figure environment removed
%========================================================
%====================projected energies=======
%===========fig============================9===========

% Figure environment removed
In order to perform the TPSM study of the $-$ive parity bands, the input parameters
required are deformation values and the strengths of the monopole and the
quadrupole pairing interaction terms. It is expected that deformation of the $-$ive parity
two-quasiparticle states will be slightly different from the yrast $+$ive parity band structures. However,
in the absence of any systematic study of the deformation properties of these band structures, we have adopted the axial deformation values of the ground-state bands from theoretical studies using microscopic-macroscopic model
predictions \cite{moller08} with slight adjustments as in our previous studies \cite{SGH16,Musangu18}. The non-axial deformation values have been varied to reproduce the observed properties of these bands. The deformation values adopted
in the present analysis are listed in Table \ref{tab1}.

The pairing parameters are clearly expected to be different from the
ground-state values as these are two-quasiparticle states, and it is known that pairing is reduced for the excited quasiparticle
states. However, it is difficult to study the reduction in the pairing correlations for the quasiparticle states as in the
BCS approximation, the pairing collapses for the blocked  states and it is imperative to perform the particle-number
projected analysis before variation \cite{sheikh02}.  In the present work, we have investigated the sensitivity of the results on the pairing correlations by varying the
monopole pairing strengths.

We have considered $^{112}$Ru as an illustrative example to investigate the deformation and
pairing dependence of the TPSM results. The
pairing strength parameters that best reproduce the experimental data of
$^{112}$Ru are then used to perform the TPSM calculations for other
Ru-isotopes from A=102 to 110. The reason that this system has been chosen is because doublet band structures
have been observed for this
nucleus up to quite high-spin and the system is well deformed \cite{NPRU}. For other isotopes, for instance,
$^{102}$Ru the data is also available up to quite high-spin. However, this system has vibrational character in the low-spin
region \cite{NPRU3} and the application of TPSM approach becomes unreliable as a single deformed mean-field solution
is adopted in this model.

It has been proposed that the doublet band structures observed in $^{112}$Ru originate from the chiral symmetry breaking mechanism \cite{NPRU}
as the difference of the energies between the two bands, $\delta (I) = (E_2(I) -E_1(I))$, is very small. In Fig.~\ref{del112}, this difference is plotted for different values of pair-gaps ($\Delta$) and non-axial deformation. The differences in the excitation energies of both  two-neutron (left panel) and
two-proton (right panel) quasiparticle configurations are plotted. The results are depicted
only for three representative values of the pair-gaps, $\Delta_n=0.6~\textrm{MeV},~ \Delta_p=1.2~\textrm{MeV}$;~ $\Delta_n=0.6~\textrm{MeV},~ \Delta_p=0.6~\textrm{MeV}$ and $\Delta_n=1.2~\textrm{MeV},~\Delta_p=0.6~\textrm{MeV}$. [ The TPSM calculations
have also been performed for other values of the pair-gaps between 0.6 and 1.2, and the results are not very different from the
three cases depicted in Fig.~\ref{del112}]. It is evident from the results that $\gamma = 30^0$ leads to lowest differences in the energies and agrees
with the corresponding experimental numbers. Further, the results with the pairing set of $\Delta_n=\Delta_p=0.6$~MeV appears to be in
better agreement with the data for the neutron excitation case as compared to the proton excitation.

To further examine the optimum pairing set and the non-axial deformation parameter that reproduce the experimental data
more accurately, we have evaluated the aligned angular
momentum values, $i_x$, for the doublet bands and the results are presented in Figs.~\ref{alignneut112}, \ref{alignprot112}, \ref{alignneut112gamma} and \ref{alignprot112gamma}. As compared to the energies, $i_x$
is  sensitive to the single-particle states occupied by the excited particles and should provide a better
estimate of the optimum pairing and deformation set. The calculated $i_x$ for the neutron excited configuration are shown in Fig.~\ref{alignneut112} for three different
pairing sets, but with the same non-axial deformation parameter of $\gamma=30^\circ$.
%in all the cases studied in the present work.
It is evident from the figure that the pairing set of $\Delta_n=\Delta_p=0.6~\textrm{MeV}$ provides a better representation of the experimental values as compared to the other two sets, in  particular, $i_x$ for Bands 1 and 2 is reproduced
remarkably well with this set. Further, the slope of the alignment curve, which is the moment of inertia,
is also in good agreement with the data for this set. The alignment calculated with the proton
excitation is depicted in Fig.~\ref{alignprot112} and it is noted that
none of the parameter set is able to reproduce the experimental values. The calculated $i_x$ values depict backbending phenomena, whereas the
experimental values show a smooth increase with spin for both the bands. In Figs.~\ref{alignneut112gamma} and \ref{alignprot112gamma}, the alignments are displayed
for different values of the non-axial deformation parameters and with the pairing set of
$\Delta_n=0.6~\textrm{MeV},~ \Delta_p=0.6~\textrm{MeV}$. It is evident from these figures that $\gamma=30^\circ$ for the
neutron excitation shows the best agreement with the data.


%The pairing strength parameter set with $\delta_n=\delta_p=0.6$ have been used in the TPSM calculations for all other
%Ru-isotopes studied in the present work. In a more accurate treatment, the pairing strength parameters should be systematically fitted
%to the negative parity bands and we are planning to perform this exercise in future.

The lowest two negative parity bands obtained for $^{112}$Ru after diagonalization of the shell model Hamiltonian,
for both neutron and proton excitations, are compared with the experimental energies in Fig.~\ref{fig:112E}. It has been
already stated that TPSM Hamiltonian, in the absence of the exchange terms, does not mix neutron and proton excitations and the
two basis spaces can be diagonalized separately. The calculated band structures obtained with the neutron excitation are 
lower in energy as compared to the band structures with the proton excitation. The lowest negative band structure (labelled
as B1 and B2 in Fig.~\ref{fig:112E}) is quite reasonably reproduced by the TPSM  calculations with neutron excitation, and the deviation for the
highest spin, I=14, for this band is 0.039 MeV. For the excited band (labelled as B3 and B4), significant deviations are noted for most of the spin states and for the highest spin observed, I=15, the calculated value has a deviation of 0.704 MeV. The alignment plotted in Fig.~\ref{alignneut112}
also depicts significant deviations, although the slope is in agreement. The origin of this deviation is not evident at this stage and
further analysis is needed, for instance, using a self-consistent mean-field approach. In Fig.~\ref{fig:112E}, we have also
provided the energies of the proton excited bands as in future experimental studies more band structures will be populated and some of them may
correspond to the proton excitation.
%===========fig==============================10==========

% Figure environment removed
%======================================================
%===========fig============================11============

% Figure environment removed
%===========fig============================12============

% Figure environment removed
%==========================================
We have performed TPSM study for other Ru-isotopes from A=102 to 110 with the axial and non-axial deformation values listed in Table \ref{tab1}. The pairing strengths are same as those adjusted
to reproduce the properties of $^{112}$Ru. The results of the energy difference between the two doublet band, $\delta (I)$,  are displayed
in Fig.~\ref{fig:del} for both proton and neutron excitations. Neutron excitation energies are slightly lower than the corresponding
proton energies, however, both are compared with the available experimental data as the proton excitation spectra can become
favoured with slight adjustments in the pairing and deformation parameters. It is noticed from the figure that TPSM calculated $\delta(I)$ from neutron excitation energies agrees well with the observed energies of $^{102}$Ru, $^{108}$Ru and $^{110}$Ru.  It is also evident from the figure that $\delta(I)$ for the lowest two proton bands also agrees with the data, except for $^{102}$Ru which shows a staggering pattern. For other isotopes yrare band has not been observed.

The alignments of the isotopes are plotted in Fig.~\ref{fig:k} for neutron excitation spectra and compared with the corresponding experimental numbers, wherever available. It is quite remarkable to note from the figure that $i_x$ values are reproduced  well for all the isotopes. For $^{102}$Ru, upbend is observed for bands B1 and B2 at $\hbar \omega \approx 0.55$ MeV and is well described by the TPSM calculations. For other isotopes, $i_x$ depicts a smooth increase with rotational frequency and is easily understood as unpaired particles align towards the rotational axis. For $^{106}$Ru, TPSM calculated bands B3 and B4 show upbends, but there is no experimental data to confirm this band crossing phenomenon. The alignments for the proton excitation bands are displayed in Fig.~\ref{fig:ak} and it is noted that in most of the cases, the band crossing is expected as either an upbend or a backbend is observed. These alignments considerably differ with the experimental values.

To examine the nature of the backbending phenomenon observed in $^{102}$Ru, the projected energies are displayed in Fig.~\ref{band} before diagonalization of the shell model Hamiltonian. The lowest band is a two-quasiparticle band with neutrons occupying the N= 4 and 5 shells. Four-quasiparticle projected bands with two-protons built on ($1\text{n}_11\text{n}^{\prime}_2$)  configuration is lower in excitation energy as compared to two-neutrons built on this state. This is easily understood as in ($1\text{n}_11\text{n}^{\prime}_2$) one neutron is already blocked from the N= 5 shell. However, it is interesting to note from Fig.~\ref{band} that it is the neutron aligned configuration which becomes favoured and crosses the $-$ive parity lowest configuration. The backbending phenomena
observed for $^{102}$Ru is, therefore, due to the alignment of neutrons.
%===========fig============================13============

% Figure environment removed
%========================================
TPSM calculated energy spectra  are compared with the experimental energies in Figs.~\ref{fig:102E}, \ref{fig:104E}, \ref{fig:106E}, \ref{fig:108E} and \ref{fig:110E} for $^{102}$Ru,$^{104}$Ru, $^{106}$Ru, $^{108}$Ru and $^{110}$Ru, respectively. For $^{102}$Ru, the experimental energies are known up to I=23, and it is observed that TPSM calculations with neutron excitation reproduces the data quite well, the deviations for most of the states is less than 0.2 MeV. The proton excitation bands  at low-spin are almost degenerate with the neutron bands, but at higher spins, they become unfavoured. In the case of $^{104}$Ru and $^{106}$Ru, the data is available up to I=13  and only one band is known. The TPSM results are in reasonable agreement with the data for both the nuclei. For $^{108}$Ru
and $^{110}$Ru, doublet $-$ive parity bands have been observed and it is noted that TPSM calculation are able to reproduce the low-lying states in both the nuclei quite well, however, deviations are noted for the excited states.

\section{Summary and Conclusions}

In the present work, the TPSM approach has been extended to investigate
the negative parity bands in even-even systems. In all the previous
versions of the model, the quasiparticle excitations were considered
from a single oscillator shell and it was possible to investigate only
positive parity bands. In the present study, we have extended the  model space by considering quasiparticle basis from two different
oscillator shells. As a  major application of this extension, the
negative parity bands observed in $^{102-112}$Ru have been investigated as considerable data is available for these isotopes.

Both proton and neutron quasiparticle excitations have been considered in the present analysis and it has been observed that neutron spectra
is slightly lower than the proton one. However, the comparison of alignments clearly delineated the two spectra with neutron alignment observed to be in better agreement with the data, whereas the proton alignment shows large deviations.

For some of the nuclei studied in the present work, the observation of the doublet band structures
have been conjectured to arise from the breaking of the chiral symmetry \cite{NPRU}. In order to investigate the chiral
symmetry origin of these bands in the TPSM framework, electromagnetic transitions need to be evaluated. Presently,
we are in the process of extending the approach to calculate the transition probabilities and the results will be reported in the near future. Futher, in order to improve the predictive power of the present investigation, deformation and pairing parameters need to be
determined for the $-$ive parity bands.
It is quite evident that parameters known for the yrast $+$ive parity configurations cannot be used for the two-quasiparticle $-$ive parity states. It is essential to deduce these parameters from the microscopic models, for instance, the
density functional approaches with  the blocking technique \cite{schunch10}. We are planning to perform this study using the Skyrme density functional approach \cite{HFODD21} and these parameters will then be used to evaluate the transition probabilities and other properties of $-$ive parity band structures to explore  the chiral symmetry origin proposed for some of the studied isotopes.

Moreover, in some of the nuclei studied the octupole deformation is predicted to be an important degree of freedom \cite{dejb95,NPRU3} and TPSM approach needs to be augmented to include the octupole deformation. We are considering to  include the octupole correlations in the TPSM framework. This can be achieved in two phases.
In the first phase, the octupole-octupole interaction will be included in the Hamiltonian with the mean-field having well defined parity. In this way, the octupole correlations will be added as a perturbation correction. In the second phase, the octupole mean-field will be
considered in the Nilsson state with the explicit breaking of the reflection symmetry. This broken symmetry can then be restored using the standard parity projection formalism \cite{EGIDO92,EGIDO91,GARROTE97,garrote98}.

\section*{Acknowledgments}
The authors are grateful to Prof. Stefan Frauendorf for illuminating discussions. 
The authors are also thankful to Science and
Engineering Research Board (SERB), Department of Science and
Technology (Govt. of India) for providing financial assistance under the 
Project No.CRG/2019/004960, and for the INSPIRE fellowship to one of the author (NN). 

%\newpage
\large{
\onecolumngrid
 \appendix
%\counterwithin*{equation}{section}
 \renewcommand{\theequation}{A.\arabic{equation}}
 \setcounter{equation}{0}
\section*{Appendix}
The Hamiltonian in terms of proton and neutron degrees of freedom employed in the TPSM approach is given by
\begin{align}
\hat H &= \hat H_0 - {\chi_{pp} \over 2} \sum_\mu \hat Q^\dagger_\mu (p)\hat Q^{}_\mu(p)-{ \chi_{nn} \over 2 }\sum_\mu\hat Q^\dagger_\mu(n)\hat Q^{}_\mu(n)-\chi_{np}\sum_\mu\left( \hat Q^\dagger_\mu(p)
  \hat Q^{}_\mu(n)+\hat Q^\dagger_\mu(n)\hat Q^{}_\mu(p)\right)\nonumber\\
&- G_{M}^{p} \hat P^\dagger_{0}(p) \hat P^{}_{0}(p)
 - G_{M}^{n} \hat P^\dagger_{0}(n) \hat P^{}_{0}(n) -G_{Q}^p \sum_\mu  \hat P^\dagger_\mu(p)\hat P^{}_\mu(p)
 -G_{Q}^n \sum_\mu \hat P^\dagger_\mu(n)\hat P^{}_\mu(n)~,
\label{hamal}
\end{align}
where the labels ``n (p)''  denote neutron (proton) states. The explicit form of the one-body operators in the above equation are given by\\
\begin{eqnarray}
  \hat Q^\dagger_\mu =\sum_{\alpha \beta}Q_{\mu \alpha \beta} c^{\dagger}_\alpha c_\beta~,\hspace{0.5cm} \hat P^\dagger_0 = \frac{1}{2}\sum_{\alpha} c^{\dagger}_\alpha c^{\dagger}_ {\bar{\alpha}}~,\hspace{0.5cm} \hat P^\dagger_\mu
  = \frac{1}{2}\sum_{\alpha \beta} Q_{\mu \alpha \beta} c^{\dagger}_\alpha c^{\dagger}_{\bar{\beta}}~~.
  \end{eqnarray}
Here the quadrupole matrix elements $Q_{\mu \alpha \alpha^\prime} = \delta_{NN^\prime}\langle Njm|Q_\mu|N^\prime j^\prime m^\prime\rangle$ with  $\alpha =\{Njm\}$, $\bar{\alpha}$ represents the time-reversed state of $\alpha$ and the dimensionless mass quadrupole operator is \cite{KY95}
\begin{equation}
 Q_\mu=\sqrt{4\pi\over 5}~{m \omega r^2\over\hbar}Y_{2\mu}~.
\end{equation}  
In the evaluation of the matrix elements of the Hamiltonian of Eq.~(\ref{hamal}), the exchange terms are disregarded. Using
the Wick's theorem to one of the terms in Eq.~(\ref{hamal}) with $|\Phi\rangle$ as the reference state, we have  
%The matrix elements of the Hamiltonian in Eq.~(\ref{hamal}), which are separable in the absence of exchange terms, can be evaulated using the generalized Wick's theorem \cite{balian}. As all the terms in Eq.~(\ref{hamal}) have the same structure, we first apply the Wick's theorem to a generic term 
\begin{align}
 \hat{O}^\dagger \hat{O}
  ={\langle\Phi| \hat{O}|\Phi\rangle}^{2} + \langle\Phi|  \hat{O}|\Phi\rangle \bigl(:\hat{O}^{\dagger}: + :\hat{O}: \bigl) + :\hat{O}^{\dagger}::\hat{O}:
  =\hat{H}^{(0)}+\hat{H}^{(1)}+\hat{H}^{(2)} \label{HOH1}~~.
\end{align}
Now using the generalized Wick's theorem \cite{balian,HS79,HS80}, we evaluate the matrix elements between the projected quasiparticle states of
Eq.~(\ref{HOH1}). First of all, for the vacuum state, we have
\begin{align}
  \langle\Phi| \hat{H}^{(0)}[\Omega]|\Phi\rangle& ={\langle\Phi| \hat{O}|\Phi\rangle}^2~, \nonumber\\
  \langle\Phi| \hat{H}^{(1)}[\Omega]|\Phi\rangle& = \langle\Phi| \hat{O} |\Phi\rangle \left( \langle\Phi| :\hat{O}^{\dagger}:[\Omega]|\Phi\rangle +\langle\Phi|:\hat{O}:[\Omega]|\Phi\rangle\right)~~,\nonumber\\
    \langle\Phi| \hat{H}^{(2)}[\Omega]|\Phi\rangle&=\langle\Phi| :\hat{O}^{\dagger}:[\Omega]|\Phi\rangle \langle\Phi|:\hat{O}:[\Omega]|\Phi\rangle~~,\label{vacuum}
\end{align}
where the operator $[\Omega]$ defined as $$[\Omega]=\frac{\hat{R}(\Omega)} {\langle \Phi|\hat{R}(\Omega)|\Phi\rangle}~~.$$
The rotation operator $\hat{R}(\Omega)$ is defined in Eq.~(\ref{rotop}) and the operator $\hat{O}^\dagger \hat{O}$ can be of any one of the form  $\hat{O}_n^\dagger \hat{O}_n$,  $\hat{O}_p^\dagger \hat{O}_p$,  $\hat{O}_p^\dagger \hat{O}_n$ or  $\hat{O}_n^\dagger \hat{O}_p$. In the following, we present matrix element of the Hamiltonian of Eq.~(\ref{hamal}) between two- and four-quasiparticle
basis states of Eq.~(\ref{basis}). The expressions between four-quasiparticle states are quite lengthy and are not presented here.

The matrix element of the operator $\hat{H}[\Omega]$ between two-quasiparticle basis states  is given by

%\subsection{ $ \langle \Phi| a_{n_{2}^\prime} a_{n_{1}} \hat{H}[\Omega]  a^\dagger_{n_{3}} a^\dagger_{n_4^\prime}  |\Phi\rangle$ }
\begin{align}
  \langle\Phi| a_{n_{2}^\prime} a_{n_{1}} \hat{H}[\Omega]  a^\dagger_{n_{3}} a^\dagger_{n_4^\prime} |\Phi \rangle=&
  %\langle\Phi| a_{n_{2}^\prime} a_{n_{1}} \bigl(\hat{H}^{(0)}+\hat{H}^{(1)}+\hat{H}^{(2)} \bigl) [\Omega] a^\dagger_{n_{3}} a^\dagger_{n_4^\prime}  |\Phi \rangle \nonumber\\
  \langle\Phi| a_{n_{2}^\prime} a_{n_{1}}\hat{H}^{(0)}[\Omega] a^\dagger_{n_{3}} a^\dagger_{n_4^\prime}  |\Phi \rangle+ \langle\Phi| a_{n_{2}^\prime} a_{n_{1}}\hat{H}^{(1)}[\Omega] a^\dagger_{n_{3}} a^\dagger_{n_4^\prime}  |\Phi \rangle\nonumber\\
 & + \langle\Phi| a_{n_{2}^\prime} a_{n_{1}}\hat{H}^{(2)}[\Omega] a^\dagger_{n_{3}} a^\dagger_{n_4^\prime}  |\Phi \rangle~~,  \label{t0}
\end{align}
where
\begin{align}
 \langle\Phi| a_{n_{2}^\prime} a_{n_{1}} \hat{H}^{(0)}[\Omega]  a^\dagger_{n_{3}} a^\dagger_{n_4^\prime} |\Phi \rangle= {\langle \Phi|\hat{O}|\Phi\rangle}^2\langle\Phi| a_{n_{2}^\prime} [\Omega] a^\dagger_{n_4^\prime} |\Phi \rangle\langle\Phi|  a_{n_{1}}[\Omega]a^\dagger_{n_{3}} |\Phi \rangle~~,\label{t1}
\end{align}
{\allowdisplaybreaks
\begin{align}
  \langle\Phi| &a_{n_{2}^\prime} a_{n_{1}} \hat{H}^{(1)}[\Omega]  a^\dagger_{n_{3}} a^\dagger_{n_4^\prime}|\Phi\rangle\nonumber\\
  &= \langle\Phi| \hat{O}|\Phi \rangle \left[ \langle\Phi| a_{n_{2}^\prime} a_{n_{1}}:\hat{O}^{\dagger}:[\Omega]|\Phi\rangle\langle\Phi|[\Omega] a^\dagger_{n_{3}} a^\dagger_{n_4^\prime} |\Phi \rangle
    +\langle\Phi| a_{n_{2}^\prime} a_{n_{1}}:\hat{O}:[\Omega] |\Phi\rangle
 \langle\Phi| [\Omega] a^\dagger_{n_{3}} a^\dagger_{n_4^\prime}  |\Phi\rangle\right.\nonumber\\
  &\left. \hspace{0.8cm} +\langle\Phi| a_{n_{2}^\prime} a_{n_{1}}[\Omega]|\Phi\rangle\langle\Phi|:\hat{O}^{\dagger}:[\Omega] a^\dagger_{n_{3}} a^\dagger_{n_4^\prime}|\Phi \rangle
    +\langle\Phi| a_{n_{2}^\prime} a_{n_{1}}[\Omega]|\Phi\rangle\langle\Phi|:\hat{O}:[\Omega] a^\dagger_{n_{3}} a^\dagger_{n_4^\prime}|\Phi\rangle\right.\nonumber\\
  &\left. \hspace{0.8cm}  -\langle\Phi|a_{n_{2}^\prime}:\hat{O}^{\dagger}:[\Omega] a^\dagger_{n_{3}}|\Phi\rangle
    \langle\Phi|a_{n_{1}}[\Omega]a^\dagger_{n_4^\prime}|\Phi\rangle
   -\langle \Phi| a_{n_{2}^\prime}:\hat{O}:[\Omega] a^\dagger_{n_{3}}|\Phi\rangle\langle\Phi|a_{n_{1}}[\Omega]a^\dagger_{n_4^\prime}|\Phi\rangle\right.\nonumber\\
  &\left. \hspace{0.8cm} -\langle\Phi|a_{n_{2}^\prime}[\Omega] a^\dagger_{n_{3}}|\Phi\rangle
   \langle\Phi| a_{n_{1}}:\hat{O}^{\dagger}:[\Omega]a^\dagger_{n_4^\prime}|\Phi\rangle
  -\langle\Phi|a_{n_{2}^\prime}[\Omega] a^\dagger_{n_{3}}|\Phi\rangle\langle\Phi| a_{n_{1}}:\hat{O}: [\Omega]a^\dagger_{n_4^\prime}|\Phi\rangle\right.\nonumber\\
   &\left. \hspace{0.8cm}  +\langle\Phi|a_{n_{2}^\prime} :\hat{O}^{\dagger}:[\Omega]a^\dagger_{n_4^\prime}|\Phi\rangle\langle\Phi| a_{n_{1}}[\Omega] a^\dagger_{n_{3}}|\Phi\rangle
     +\langle\Phi| a_{n_{2}^\prime} :\hat{O}:[\Omega]a^\dagger_{n_4^\prime}|\Phi\rangle
     \langle\Phi| a_{n_{1}}[\Omega] a^\dagger_{n_{3}}|\Phi\rangle\right.\nonumber\\
  &\left.\hspace{0.8cm} +\langle\Phi| a_{n_{2}^\prime}[\Omega]a^\dagger_{n_4^\prime}|\Phi\rangle\langle\Phi|a_{n_{1}} :\hat{O}^{\dagger}: [\Omega] a^\dagger_{n_{3}}|\Phi\rangle
  + \langle\Phi|a_{n_{2}^\prime}[\Omega]a^\dagger_{n_4^\prime}|\Phi\rangle
  \langle\Phi| a_{n_{1}} :\hat{O}: [\Omega] a^\dagger_{n_{3}}|\Phi\rangle\right.\nonumber\\
 &\left.\hspace{0.8cm}+\left(\langle \Phi|:\hat{O}^{\dagger}: [\Omega]|\Phi\rangle +\langle \Phi|:\hat{O}: [\Omega]|\Phi\rangle\right) \langle\Phi| a_{n_{2}^\prime} [\Omega] a^\dagger_{n_4^\prime} |\Phi \rangle\langle\Phi|  a_{n_{1}}[\Omega]a^\dagger_{n_{3}} |\Phi \rangle \right]\label{h1}~~,
  \end{align}
%+++++++++++++++++++++++++++++++++++++++++++
Since the operator in Eq.~(\ref{hamal}) is either monopole or quadrupole, they have even-parity and these cannot connect states having different parities. Therefore, the terms of the type $ \langle\Phi|a_{n_2^\prime} a_{n_1}:\hat{O}:[\Omega]|\Phi\rangle$ are zero as  the state $|\Phi\rangle$ has even-parity and the bra-state has odd-parity with the neutrons occupying two different oscillator shells having
opposite parities. Therefore, Eq.~(\ref{h1}) simplifies to
\begin{align}
\langle\Phi| &a_{n_{2}^\prime} a_{n_{1}} \hat{H}^{(1)}[\Omega]  a^\dagger_{n_{3}} a^\dagger_{n_4^\prime}|\Phi\rangle\nonumber\\
  &=\langle\Phi| \hat{O} |\Phi\rangle \left[ \langle\Phi|a_{n_{2}^\prime} :\hat{O}^{\dagger}:[\Omega]a^\dagger_{n_4^\prime}|\Phi\rangle\langle\Phi| a_{n_{1}}[\Omega] a^\dagger_{n_{3}}|\Phi\rangle
    +\langle\Phi|a_{n_{2}^\prime} :\hat{O}:[\Omega]a^\dagger_{n_4^\prime}|\Phi\rangle\langle\Phi| a_{n_{1}}[\Omega] a^\dagger_{n_{3}}|\Phi\rangle\right.\nonumber\\
  &\left.\hspace{0.8cm}   +\langle\Phi|a_{n_{2}^\prime}[\Omega]a^\dagger_{n_4^\prime}|\Phi\rangle
    \langle \Phi|a_{n_{1}} :\hat{O}^{\dagger}: [\Omega] a^\dagger_{n_{3}}|\Phi\rangle
    +\langle\Phi|a_{n_{2}^\prime}[\Omega]a^\dagger_{n_4^\prime}|\Phi\rangle\langle\Phi| a_{n_{1}} :\hat{O}: [\Omega] a^\dagger_{n_{3}}|\Phi\rangle\right.\nonumber\\
    &\left.\hspace{0.8cm}+\left(\langle \Phi|:\hat{O}^{\dagger}: [\Omega]|\Phi\rangle +\langle \Phi|:\hat{O}: [\Omega]|\Phi\rangle\right) \langle\Phi| a_{n_{2}^\prime} [\Omega] a^\dagger_{n_4^\prime} |\Phi \rangle\langle\Phi|  a_{n_{1}}[\Omega]a^\dagger_{n_{3}} |\Phi \rangle
    \right]~, \label{t2}
\end{align}
}
In the first four terms of the above equation, quadrupole-quadrupole and pairing interaction terms among neutrons will contribute. Also, for these terms neutron-proton quadrupole-quadrupole interaction term will contribute. For the last two terms, all the interaction terms of the Hamiltonian  contribute.

The third term of Eq.~(\ref{t0}) is given by
{\allowdisplaybreaks
\begin{align}
  \langle \Phi|& a_{n_{2}^\prime} a_{n_{1}} \hat{H}^{(2)}[\Omega]  a^\dagger_{n_{3}} a^\dagger_{n_4^\prime}|\Phi\rangle\nonumber\\= &\ \langle\Phi| a_{n_{2}^\prime} :\hat{O}^{\dagger}:[\Omega]a^\dagger_{n_4^\prime}|\Phi\rangle\langle\Phi|a_{n_{1}}:\hat{O}:[\Omega] a^\dagger_{n_{3}}|\Phi\rangle
   +\langle\Phi|a_{n_{2}^\prime} :\hat{O}:[\Omega]a^\dagger_{n_4^\prime}|\Phi\rangle
    \langle\Phi|a_{n_{1}}:\hat{O}^{\dagger}:[\Omega] a^\dagger_{n_{3}}|\Phi\rangle\nonumber\\
 & + \langle\Phi|:\hat{O}^{\dagger}:[\Omega]|\Phi\rangle\langle\Phi|a_{n_{2}^\prime}:\hat{O}:[\Omega]a^\dagger_{n_4^\prime}|\Phi\rangle\langle\Phi|a_{n_{1}}[\Omega] a^\dagger_{n_{3}}|\Phi\rangle
 +\langle\Phi|:\hat{O}:[\Omega]|\Phi\rangle
 \langle\Phi| a_{n_{2}^\prime} :\hat{O}^{\dagger}:[\Omega]a^\dagger_{n_4^\prime}|\Phi\rangle\nonumber\\
 & \langle\Phi| a_{n_{1}}[\Omega] a^\dagger_{n_{3}}|\Phi\rangle
  + \langle\Phi|:\hat{O}^{\dagger}:[\Omega]|\Phi\rangle\langle\Phi|a_{n_{1}} :\hat{O}:[\Omega]a^\dagger_{n_3}|\Phi\rangle\langle\Phi| a_{n_{2}^{\prime}}[\Omega] a^\dagger_{n_{4}^{\prime}}|\Phi\rangle
 + \langle\Phi|:\hat{O}:[\Omega]|\Phi\rangle \nonumber\\
 & \langle\Phi| a_{n_{1}} :\hat{O}^{\dagger}:[\Omega]a^\dagger_{n_3}|\Phi\rangle\langle\Phi| a_{n_{2}^{\prime}}[\Omega] a^\dagger_{n_{4}^{\prime}}|\Phi\rangle+\langle \Phi|:\hat{O}^{\dagger}: [\Omega]|\Phi\rangle\langle \Phi|:\hat{O}: [\Omega]|\Phi\rangle  \langle\Phi| a_{n_{2}^\prime} [\Omega] a^\dagger_{n_4^\prime} |\Phi \rangle\nonumber\\
 &\langle\Phi|  a_{n_{1}}[\Omega]a^\dagger_{n_{3}} |\Phi \rangle~~. \label{t3}
\end{align}
}
%The terms like $\langle\Phi| a_{n_{2}^\prime} a_{n_{1}}[\Omega]|\Phi\rangle$, $\langle\Phi| [\Omega] a^\dagger_{n_{3}} a^\dagger_{n_4^\prime}  |\Phi\rangle$,
%$\langle\Phi|a_{n_{2}^\prime}[\Omega] a^\dagger_{n_{3}}|\Phi\rangle$  vanishes because the vacuum state $|\Phi\rangle$ has definite positive parity while  the state $\langle\Phi|a_{n_{2}^\prime} a_{n_{1}}$ has negative parity and $[\Omega]$ is invarient under parity.
%Substituting Eqs. (\ref{t1}), (\ref{t2}) and (\ref{t3}) in Eq. (\ref{t0}), we get the required matrix element.
%===============================================================
Consider now the matrix element between neutron and proton excitations
{\allowdisplaybreaks
\begin{align}
  \langle\Phi|a_{n_{2}^\prime} a_{n_{1}} \hat{H}[\Omega]  a^\dagger_{p_{3}} a^\dagger_{p_4^\prime} |\Phi \rangle=&
  %\langle\Phi| a_{n_{2}^\prime} a_{n_{1}} \bigl(\hat{H}^{(0)}+\hat{H}^{(1)}+\hat{H}^{(2)} \bigl) [\Omega] a^\dagger_{p_{3}} a^\dagger_{p_4^\prime}  |\Phi \rangle \nonumber\\
  \langle\Phi| a_{n_{2}^\prime} a_{n_{1}} \hat{H}^{(0)}[\Omega] a^\dagger_{p_{3}} a^\dagger_{p_4^\prime} |\Phi \rangle+\langle\Phi| a_{n_{2}^\prime} a_{n_{1}} \hat{H}^{(1)}[\Omega] a^\dagger_{p_{3}} a^\dagger_{p_4^\prime} |\Phi \rangle\nonumber\\
  &+\langle\Phi| a_{n_{2}^\prime} a_{n_{1}} \hat{H}^{(2)}[\Omega] a^\dagger_{p_{3}} a^\dagger_{p_4^\prime} |\Phi \rangle~~,\label{c0}
\end{align}
}
where
%==================================================
{\allowdisplaybreaks
\begin{align}
  \langle\Phi|& a_{n_{2}^\prime} a_{n_{1}} \hat{H}^{(0)}[\Omega] a^\dagger_{p_{3}} a^\dagger_{p_4^\prime} |\Phi \rangle={\langle\Phi| \hat{O}|\Phi \rangle}^{2} \langle\Phi| a_{n_{2}^\prime} a_{n_{1}}[\Omega] a^\dagger_{p_{3}} a^\dagger_{p_4^\prime} |\Phi \rangle\nonumber\\
  &={\langle\Phi| \hat{O}|\Phi \rangle}^{2} \left[\langle\Phi| a_{n_{2}^\prime} a_{n_{1}}[\Omega] |\Phi \rangle \langle\Phi|[\Omega] a^\dagger_{p_{3}} a^\dagger_{p_4^\prime} |\Phi \rangle-\langle\Phi| a_{n_{2}^\prime}[\Omega] a^\dagger_{p_{3}} |\Phi \rangle\langle\Phi|a_{n_{1}}[\Omega] a^\dagger_{p_4^\prime} |\Phi \rangle\right.\nonumber\\
    &\left.\hspace{2.3cm}+\langle\Phi| a_{n_{2}^\prime}[\Omega] a^\dagger_{p_4^\prime} |\Phi \rangle\langle\Phi| a_{n_{1}}[\Omega] a^\dagger_{p_{3}} |\Phi \rangle\right]\nonumber\\&=0~~.\label{c1}
\end{align}
}
The terms of the type $\langle\Phi| a_{n_{2}^\prime}[\Omega] a^\dagger_{p_4^\prime} |\Phi \rangle$, $\langle\Phi| a_{n_{1}}[\Omega] a^\dagger_{p_{3}} |\Phi \rangle$ and $\langle\Phi| a_{n_{2}^\prime}[\Omega] a^\dagger_{p_{3}} |\Phi \rangle$ vanish
since $|\Phi\rangle$ is a product of neutron and proton vacuum states, i.e,$$|\Phi\rangle=|\Phi_n\rangle|\Phi_p\rangle~~,$$ and $$\langle\Phi| a_{n_{2}^\prime}[\Omega] a^\dagger_{p_4^\prime} |\Phi \rangle=\langle\Phi_n| a_{n_{2}^\prime}[\Omega]|\Phi_n\rangle\langle\Phi_p|[\Omega] a^\dagger_{p_4^\prime} |\Phi_p \rangle~~,$$ Since  $|\Phi_n\rangle$ and  $|\Phi_p\rangle$ have $+$ive parity, both the overlaps on the right-hand side vanish due to parity symmetry.

Therefore,
{\allowdisplaybreaks
  \begin{align}
    \langle\Phi| a_{n_{2}^\prime}[\Omega] a^\dagger_{p_4^\prime} |\Phi \rangle=0~~.\label{overlap}
  \end{align}
  }
The second term of Eq.~(\ref{c0}) is
%=============================
{\allowdisplaybreaks
  \begin{align}
    \langle\Phi|&a_{n_{2}^\prime} a_{n_{1}} \hat{H}^{(1)}[\Omega] a^\dagger_{p_{3}} a^\dagger_{p_4^\prime} |\Phi \rangle= \langle\Phi| \hat{O}|\Phi \rangle \langle\Phi|a_{n_{2}^\prime} a_{n_{1}} \left(:\hat{O}^\dagger:+:\hat{O}:\right)[\Omega] a^\dagger_{p_{3}} a^\dagger_{p_4^\prime} |\Phi \rangle\nonumber\\
     &= \langle\Phi| \hat{O}|\Phi \rangle \left[ \langle\Phi| a_{n_{2}^\prime} a_{n_{1}}:\hat{O}^{\dagger}:[\Omega]|\Phi\rangle\langle\Phi|[\Omega] a^\dagger_{p_{3}} a^\dagger_{p_4^\prime} |\Phi \rangle
    +\langle\Phi| a_{n_{2}^\prime} a_{n_{1}}:\hat{O}:[\Omega] |\Phi\rangle
 \langle\Phi| [\Omega] a^\dagger_{p_{3}} a^\dagger_{p_4^\prime}  |\Phi\rangle\right.\nonumber\\
  &\left. \hspace{0.8cm} +\langle\Phi| a_{n_{2}^\prime} a_{n_{1}}[\Omega]|\Phi\rangle\langle\Phi|:\hat{O}^{\dagger}:[\Omega] a^\dagger_{p_{3}} a^\dagger_{p_4^\prime}|\Phi \rangle
    +\langle\Phi| a_{n_{2}^\prime} a_{n_{1}}[\Omega]|\Phi\rangle\langle\Phi|:\hat{O}:[\Omega] a^\dagger_{p_{3}} a^\dagger_{p_4^\prime}|\Phi\rangle\right.\nonumber\\
  &\left. \hspace{0.8cm}  -\langle\Phi|a_{n_{2}^\prime}:\hat{O}^{\dagger}:[\Omega] a^\dagger_{p_{3}}|\Phi\rangle
    \langle\Phi|a_{n_{1}}[\Omega]a^\dagger_{p_4^\prime}|\Phi\rangle
   -\langle \Phi| a_{n_{2}^\prime}:\hat{O}:[\Omega] a^\dagger_{p_{3}}|\Phi\rangle\langle\Phi|a_{n_{1}}[\Omega]a^\dagger_{p_4^\prime}|\Phi\rangle\right.\nonumber\\
  &\left. \hspace{0.8cm} -\langle\Phi|a_{n_{2}^\prime}[\Omega] a^\dagger_{p_{3}}|\Phi\rangle
   \langle\Phi| a_{n_{1}}:\hat{O}^{\dagger}:[\Omega]a^\dagger_{p_4^\prime}|\Phi\rangle
  -\langle\Phi|a_{n_{2}^\prime}[\Omega] a^\dagger_{p_{3}}|\Phi\rangle\langle\Phi| a_{n_{1}}:\hat{O}: [\Omega]a^\dagger_{p_4^\prime}|\Phi\rangle\right.\nonumber\\
   &\left. \hspace{0.8cm}  +\langle\Phi|a_{n_{2}^\prime} :\hat{O}^{\dagger}:[\Omega]a^\dagger_{p_4^\prime}|\Phi\rangle\langle\Phi| a_{n_{1}}[\Omega] a^\dagger_{p_{3}}|\Phi\rangle
     +\langle\Phi| a_{n_{2}^\prime} :\hat{O}:[\Omega]a^\dagger_{p_4^\prime}|\Phi\rangle
     \langle\Phi| a_{n_{1}}[\Omega] a^\dagger_{p_{3}}|\Phi\rangle\right.\nonumber\\
  &\left.\hspace{0.8cm} +\langle\Phi| a_{n_{2}^\prime}[\Omega]a^\dagger_{p_4^\prime}|\Phi\rangle\langle\Phi|a_{n_{1}} :\hat{O}^{\dagger}: [\Omega] a^\dagger_{p_{3}}|\Phi\rangle
  + \langle\Phi|a_{n_{2}^\prime}[\Omega]a^\dagger_{p_4^\prime}|\Phi\rangle
  \langle\Phi| a_{n_{1}} :\hat{O}: [\Omega] a^\dagger_{p_{3}}|\Phi\rangle\right.\nonumber\\
 &\left.\hspace{0.8cm}+\left(\langle \Phi|:\hat{O}^{\dagger}: [\Omega]|\Phi\rangle +\langle \Phi|:\hat{O}: [\Omega]|\Phi\rangle\right)\langle\Phi| a_{n_{2}^\prime} a_{n_{1}}[\Omega] a^\dagger_{p_{3}} a^\dagger_{p_4^\prime} |\Phi \rangle\right]\nonumber\\&=0~~.\label{c2}
    \end{align}
}
All the above terms vanish due to Eqs.~(\ref{c1}) and (\ref{overlap}).
%to parity symmetry or the vanishing overlap between neutron and proton states.
The third term of Eq.~(\ref{c0}) is given as
%==================third term==================
{\allowdisplaybreaks
  \begin{align}
    \langle\Phi|&a_{n_{2}^\prime} a_{n_{1}} \hat{H}^{(2)}[\Omega] a^\dagger_{p_{3}} a^\dagger_{p_4^\prime} |\Phi \rangle=\langle\Phi|a_{n_{2}^\prime} a_{n_{1}} \left(:\hat{O}^\dagger::\hat{O}:\right)[\Omega] a^\dagger_{p_{3}} a^\dagger_{p_4^\prime} |\Phi \rangle\nonumber\\
    &= \langle\Phi|a_{n_{2}^\prime} a_{n_{1}} :\hat{O}^{\dagger}: [\Omega]|\Phi \rangle \langle\Phi|:\hat{O}: [\Omega] a^\dagger_{p_{3}} a^\dagger_{p_4^\prime} |\Phi \rangle+\langle\Phi|a_{n_{2}^\prime} a_{n_{1}} :\hat{O}: [\Omega]|\Phi \rangle \langle\Phi|:\hat{O}^\dagger: [\Omega] a^\dagger_{p_{3}} a^\dagger_{p_4^\prime} |\Phi \rangle\nonumber\\
    &- \langle\Phi|a_{n_{2}^\prime} :\hat{O}^{\dagger}: [\Omega] a^\dagger_{p_{3}}|\Phi \rangle \langle\Phi a_{n_{1}} :\hat{O}: [\Omega]a^\dagger_{p_4^\prime} |\Phi \rangle- \langle\Phi|a_{n_{2}^\prime} :\hat{O}: [\Omega] a^\dagger_{p_{3}}|\Phi \rangle \langle\Phi a_{n_{1}} :\hat{O}^\dagger: [\Omega]a^\dagger_{p_4^\prime} |\Phi \rangle\nonumber\\
    &+\langle\Phi|a_{n_{2}^\prime} :\hat{O}^{\dagger}: [\Omega]a^\dagger_{p_4^\prime} |\Phi \rangle \langle\Phi| a_{n_{1}} :\hat{O}: [\Omega] a^\dagger_{p_{3}}|\Phi \rangle+\langle\Phi|a_{n_{2}^\prime} :\hat{O}: [\Omega]a^\dagger_{p_4^\prime} |\Phi \rangle \langle\Phi| a_{n_{1}} :\hat{O}^\dagger: [\Omega] a^\dagger_{p_{3}}|\Phi \rangle\nonumber\\
    &+ \langle\Phi| :\hat{O}^{\dagger}: [\Omega]|\Phi \rangle\left[\langle\Phi|a_{n_{2}^\prime} a_{n_{1}} :\hat{O}: [\Omega]|\Phi \rangle \langle\Phi|[\Omega] a^\dagger_{p_{3}} a^\dagger_{p_4^\prime} |\Phi \rangle+\langle\Phi|a_{n_{2}^\prime} a_{n_{1}} [\Omega]|\Phi \rangle \langle\Phi|:\hat{O}:[\Omega] a^\dagger_{p_{3}} a^\dagger_{p_4^\prime} |\Phi \rangle\right.\nonumber\\
      &\left.- \langle\Phi|a_{n_{2}^\prime} :\hat{O}: [\Omega] a^\dagger_{p_{3}}|\Phi \rangle \langle\Phi a_{n_{1}} [\Omega]a^\dagger_{p_4^\prime} |\Phi \rangle-\langle\Phi|a_{n_{2}^\prime} [\Omega] a^\dagger_{p_{3}}|\Phi \rangle \langle\Phi a_{n_{1}} :\hat{O}: [\Omega]a^\dagger_{p_4^\prime} |\Phi \rangle\right.\nonumber\\
      &\left.+\langle\Phi|a_{n_{2}^\prime} :\hat{O}: [\Omega]a^\dagger_{p_4^\prime} |\Phi \rangle \langle\Phi| a_{n_{1}} [\Omega] a^\dagger_{p_{3}}|\Phi \rangle+\langle\Phi|a_{n_{2}^\prime}[\Omega]a^\dagger_{p_4^\prime} |\Phi \rangle \langle\Phi| a_{n_{1}} :\hat{O}: [\Omega] a^\dagger_{p_{3}}|\Phi \rangle\right]\nonumber\\
    %-------------------------------
      &+ \langle\Phi| :\hat{O}: [\Omega]|\Phi \rangle\left[\langle\Phi|a_{n_{2}^\prime} a_{n_{1}} :\hat{O}^\dagger: [\Omega]|\Phi \rangle \langle\Phi|[\Omega] a^\dagger_{p_{3}} a^\dagger_{p_4^\prime} |\Phi \rangle+\langle\Phi|a_{n_{2}^\prime} a_{n_{1}} [\Omega]|\Phi \rangle \langle\Phi|:\hat{O}^\dagger:[\Omega] a^\dagger_{p_{3}} a^\dagger_{p_4^\prime} |\Phi \rangle\right.\nonumber\\
      &\left.- \langle\Phi|a_{n_{2}^\prime} :\hat{O}^\dagger: [\Omega] a^\dagger_{p_{3}}|\Phi \rangle \langle\Phi a_{n_{1}} [\Omega]a^\dagger_{p_4^\prime} |\Phi \rangle-\langle\Phi|a_{n_{2}^\prime} [\Omega] a^\dagger_{p_{3}}|\Phi \rangle \langle\Phi a_{n_{1}} :\hat{O}^\dagger: [\Omega]a^\dagger_{p_4^\prime} |\Phi \rangle\right.\nonumber\\
      &\left.+\langle\Phi|a_{n_{2}^\prime} :\hat{O}^\dagger: [\Omega]a^\dagger_{p_4^\prime} |\Phi \rangle \langle\Phi| a_{n_{1}} [\Omega] a^\dagger_{p_{3}}|\Phi \rangle+\langle\Phi|a_{n_{2}^\prime}[\Omega]a^\dagger_{p_4^\prime} |\Phi \rangle \langle\Phi| a_{n_{1}} :\hat{O}^\dagger: [\Omega] a^\dagger_{p_{3}}|\Phi \rangle\right]\nonumber\\
    &+\left[\langle\Phi|:\hat{O}^\dagger: [\Omega]|\Phi \rangle\langle\Phi|:\hat{O}: [\Omega]|\Phi \rangle\right]\langle\Phi| a_{n_{2}^\prime} a_{n_{1}}[\Omega] a^\dagger_{p_{3}} a^\dagger_{p_4^\prime} |\Phi \rangle\nonumber\\&=0~~.\label{c3}
    \end{align}
}
Again all the above terms vanish due to Eqs.~(\ref{c1}) and (\ref{overlap}).
%parity symmetry or the overlap between neutron and proton states.


Therefore, we have $$\langle\Phi|a_{n_{2}^\prime} a_{n_{1}} \hat{H}[\Omega] a^\dagger_{p_{3}} a^\dagger_{p_4^\prime} |\Phi \rangle=0~~.$$
%===================================================================
The matrix element of the operator $\hat{H}[\Omega]$ between two-quasiparticle state $a^\dagger_{n_{1}} a^\dagger_{n_{2}^\prime}|\Phi\rangle$ and four-quasiparticle state $a^\dagger_{n_{3}} a^\dagger_{n_4^\prime} a^\dagger_{n_5^\prime} a^\dagger_{n_6^\prime}|\Phi\rangle$ with a coupled neutron pair is given by

%=================<an'2 an1 H an3 an'4 an'5 an'6>==============
%\subsection{$ \langle\Phi| a_{n_{2}^\prime} a_{n_{1}} \hat{H}[\Omega]  a^\dagger_{n_{3}} a^\dagger_{n_4^\prime} a^\dagger_{n_5^\prime} a^\dagger_{n_6^\prime}|\Phi\rangle$ }
\begin{align}
  \langle\Phi|a_{n_{2}^\prime} a_{n_{1}} \hat{H}[\Omega]  a^\dagger_{n_{3}} a^\dagger_{n_4^\prime} a^\dagger_{n_5^\prime} a^\dagger_{n_6^\prime}|\Phi\rangle =&
  %\langle\Phi|  a_{n_2^\prime} a_{n_{1}}\bigl(\hat{H}^{(0)}+ \hat{H}{^{(1)}}+ \hat{H}^{(2)}\bigl) [\Omega]a^\dagger_{n_{3}} a^\dagger_{n_{4}^\prime} a^\dagger_{n_5^\prime} a^\dagger_{n_6^\prime} |\Phi\rangle\nonumber\\
 \langle\Phi|a_{n_2^\prime} a_{n_{1}}\hat{H}^{(0)}[\Omega]a^\dagger_{n_{3}} a^\dagger_{n_{4}^\prime} a^\dagger_{n_5^\prime} a^\dagger_{n_6^\prime} |\Phi\rangle+\langle\Phi|a_{n_2^\prime} a_{n_{1}}\hat{H}^{(1)}[\Omega]a^\dagger_{n_{3}} a^\dagger_{n_{4}^\prime} a^\dagger_{n_5^\prime} a^\dagger_{n_6^\prime} |\Phi\rangle\nonumber\\&+\langle\Phi|a_{n_2^\prime} a_{n_{1}}\hat{H}^{(2)}[\Omega]a^\dagger_{n_{3}} a^\dagger_{n_{4}^\prime} a^\dagger_{n_5^\prime} a^\dagger_{n_6^\prime} |\Phi\rangle~~,\label{eq:th0}
\end{align}
where
%First term
{\allowdisplaybreaks
\begin{align}
  \langle\Phi|& a_{n_{2}^\prime} a_{n_{1}} \hat{H}^{(0)}[\Omega]  a^\dagger_{n_{3}} a^\dagger_{n_4^\prime} a^\dagger_{n_5^\prime} a^\dagger_{n_6^\prime}|\Phi\rangle = {\langle\Phi| \hat{O}|\Phi \rangle}^{2}  \langle\Phi|a_{n_{2}^\prime} a_{n_{1}} [\Omega]  a^\dagger_{n_{3}} a^\dagger_{n_4^\prime} a^\dagger_{n_5^\prime} a^\dagger_{n_6^\prime}|\Phi\rangle\nonumber \\
  &= {\langle\Phi| \hat{O}|\Phi \rangle}^{2}\left[ \langle\Phi|a_{n_{1}}[\Omega] a^\dagger_{n_{3}}|\Phi\rangle\langle\Phi|a_{n_{2}^\prime}[\Omega] a^\dagger_{n_4^\prime}|\Phi\rangle \langle\Phi|[\Omega] a^\dagger_{n_5^\prime} a^\dagger_{n_6^\prime}|\Phi\rangle
    -  \langle\Phi|a_{n_{1}}[\Omega] a^\dagger_{n_{3}}|\Phi\rangle\langle\Phi| a_{n_{2}^\prime}[\Omega] a^\dagger_{n_5^\prime}|\Phi\rangle\right.\nonumber\\
   &\left. \hspace{1cm} \langle\Phi|[\Omega] a^\dagger_{n_4^\prime} a^\dagger_{n_6^\prime}|\Phi\rangle
   +\langle\Phi|a_{n_{1}}[\Omega] a^\dagger_{n_{3}}|\Phi\rangle\langle\Phi| a_{n_{2}^\prime}[\Omega] a^\dagger_{n_6^\prime}|\Phi\rangle \langle\Phi| [\Omega] a^\dagger_{n_4^\prime} a^\dagger_{n_5^\prime}|\Phi\rangle\right],\label{eq:th1}
\end{align}
}
%Second term
{\allowdisplaybreaks
\begin{align}
\langle\Phi| a_{n_{2}^\prime} a_{n_{1}}& \hat{H}^{(1)}[\Omega]a^\dagger_{n_{3}} a^\dagger_{n_4^\prime} a^\dagger_{n_5^\prime} a^\dagger_{n_6^\prime}|\Phi\rangle
  = \langle\Phi| \hat{O} |\Phi\rangle  \langle\Phi| a_{n_{2}^\prime} a_{n_{1}}\bigl(:\hat{O}^{\dagger}:+ :\hat{O}:\bigl)[\Omega] a^\dagger_{n_{3}} a^\dagger_{n_4^\prime} a^\dagger_{n_5^\prime} a^\dagger_{n_6^\prime}|\Phi\rangle\nonumber\\
  &=\langle\Phi| \hat{O}|\Phi \rangle\left[\langle\Phi|a_{n_{1}}[\Omega]a^\dagger_{n_{3}}|\Phi\rangle\left(\langle\Phi|a_{n_{2}^\prime}:\hat{O}^{\dagger}:[\Omega] a^\dagger_{n_4^\prime}|\Phi\rangle\langle\Phi|[\Omega] a^\dagger_{n_5^\prime} a^\dagger_{n_6^\prime}|\Phi\rangle\nonumber\right.\right.\\
      &\left.\left.+\langle a_{n_{2}^\prime}[\Omega] a^\dagger_{n_4^\prime}|\Phi\rangle\langle\Phi|:\hat{O}^{\dagger}:[\Omega] a^\dagger_{n_5^\prime} a^\dagger_{n_6^\prime}|\Phi\rangle-\langle\Phi|a_{n_{2}^\prime}:\hat{O}^{\dagger}:[\Omega] a^\dagger_{n_5^\prime}|\Phi\rangle\langle\Phi|[\Omega] a^\dagger_{n_4^\prime} a^\dagger_{n_6^\prime}|\Phi\rangle\right.\right.\nonumber\\
   & \left.\left.  -\langle\Phi| a_{n_{2}^\prime}[\Omega] a^\dagger_{n_5^\prime}|\Phi\rangle
      \langle \Phi|:\hat{O}^{\dagger}: [\Omega] a^\dagger_{n_4^\prime} a^\dagger_{n_6^\prime}|\Phi\rangle +\langle\Phi| a_{n_{2}^\prime}:\hat{O}^{\dagger}: [\Omega] a^\dagger_{n_6^\prime}|\Phi\rangle\langle\Phi|[\Omega] a^\dagger_{n_4^\prime} a^\dagger_{n_5^\prime}|\Phi\rangle\right. \right.\nonumber\\
      &\left.\left. +\langle\Phi| a_{n_{2}^\prime} [\Omega] a^\dagger_{n_6^\prime}|\Phi\rangle\langle\Phi|:\hat{O}^{\dagger}:[\Omega] a^\dagger_{n_4^\prime} a^\dagger_{n_5^\prime}|\Phi\rangle +\langle\Phi| a_{n_{2}^\prime}:\hat{O}:[\Omega] a^\dagger_{n_4^\prime}|\Phi\rangle\langle\Phi|[\Omega] a^\dagger_{n_5^\prime} a^\dagger_{n_6^\prime}|\Phi\rangle\right.\right.\nonumber\\
      &\left.\left. +\langle\Phi| a_{n_{2}^\prime}[\Omega] a^\dagger_{n_4^\prime}|\Phi\rangle\langle\Phi|:\hat{O}:[\Omega] a^\dagger_{n_5^\prime} a^\dagger_{n_6^\prime}|\Phi\rangle -\langle\Phi| a_{n_{2}^\prime}:\hat{O}:[\Omega] a^\dagger_{n_5^\prime}|\Phi\rangle\langle\Phi|[\Omega] a^\dagger_{n_4^\prime} a^\dagger_{n_6^\prime}|\Phi\rangle\right.\right.\nonumber\\
      &\left.\left. -\langle\Phi| a_{n_{2}^\prime}[\Omega] a^\dagger_{n_5^\prime}|\Phi\rangle \langle\Phi|:\hat{O}: [\Omega] a^\dagger_{n_4^\prime} a^\dagger_{n_6^\prime}|\Phi\rangle +\langle \Phi|a_{n_{2}^\prime}:\hat{O}: [\Omega] a^\dagger_{n_6^\prime}|\Phi\rangle\langle\Phi|[\Omega] a^\dagger_{n_4^\prime} a^\dagger_{n_5^\prime}|\Phi\rangle\right.\right.\nonumber\\
      &\left.\left. +\langle\Phi| a_{n_{2}^\prime} [\Omega] a^\dagger_{n_6^\prime}|\Phi\rangle\langle\Phi| :\hat{O}:[\Omega] a^\dagger_{n_4^\prime} a^\dagger_{n_5^\prime}|\Phi\rangle \right)+\left(\langle\Phi| a_{n_{1}}:\hat{O}^{\dagger}:[\Omega]a^\dagger_{n_{3}}|\Phi\rangle\right.\right.\nonumber\\
      &\left.\left.+\langle\Phi|a_{n_{1}}:\hat{O}:[\Omega]a^\dagger_{n_{3}}|\Phi\rangle\right) \left(\langle\Phi| a_{n_{2}^\prime}[\Omega] a^\dagger_{n_4^\prime}|\Phi\rangle\langle\Phi|[\Omega]a^\dagger_{n_5^\prime} a^\dagger_{n_6^\prime}|\Phi\rangle\right.\right.\nonumber\\
  &\left.\left. -\langle\Phi|a_{n_{2}^\prime}[\Omega] a^\dagger_{n_5^\prime}|\Phi\rangle\langle\Phi|[\Omega] a^\dagger_{n_4^\prime} a^\dagger_{n_6^\prime}|\Phi\rangle+\langle\Phi|a_{n_{2}^\prime}[\Omega] a^\dagger_{n_6^\prime}|\Phi\rangle\langle\Phi|[\Omega] a^\dagger_{n_4^\prime} a^\dagger_{n_5^\prime}|\Phi\rangle\right)\right.\nonumber\\
   &\left. +\left(\langle\Phi|:\hat{O}^\dagger:[\Omega]|\Phi\rangle+\langle\Phi|:\hat{O}:[\Omega]|\Phi\rangle\right)\langle\Phi| a_{n_{2}^\prime} a_{n_{1}}[\Omega]a^\dagger_{n_{3}} a^\dagger_{n_4^\prime} a^\dagger_{n_5^\prime} a^\dagger_{n_6^\prime}|\Phi\rangle\right]
  %====================================================
  ,\label{eq:th2}
\end{align}
}
and
%=============================================================
%Third term
%=============================================================
{\allowdisplaybreaks
  \begin{align}   
 \langle\Phi| a_{n_{2}^\prime} a_{n_{1}}& \hat{H}^{(2)}[\Omega]a^\dagger_{n_{3}} a^\dagger_{n_4^\prime} a^\dagger_{n_5^\prime} a^\dagger_{n_6^\prime}|\Phi\rangle
 = \langle\Phi|a_{n_{2}^\prime} a_{n_{1}}\bigl(:\hat{O}^{\dagger}: :\hat{O}:\bigl)[\Omega] a^\dagger_{n_{3}} a^\dagger_{n_4^\prime} a^\dagger_{n_5^\prime} a^\dagger_{n_6^\prime}|\Phi\rangle\nonumber\\
 &=\langle\Phi| a_{n_{1}}:\hat{O}^{\dagger}:[\Omega]a^\dagger_{n_{3}}|\Phi\rangle\left[\langle\Phi| a_{n_{2}^\prime}:\hat{O}:[\Omega] a^\dagger_{n_4^\prime}|\Phi\rangle\langle\Phi| [\Omega]a^\dagger_{n_5^\prime} a^\dagger_{n_6^\prime}|\Phi\rangle\right.\nonumber\\
   &\left. -\langle\Phi|  a_{n_{2}^\prime}:\hat{O}:[\Omega] a^\dagger_{n_5^\prime}|\Phi\rangle\langle\Phi|[\Omega] a^\dagger_{n_4^\prime}a^\dagger_{n_6^\prime}|\Phi\rangle  +\langle\Phi| a_{n_{2}^\prime}:\hat{O}:[\Omega] a^\dagger_{n_6^\prime}|\Phi\rangle\langle\Phi| [\Omega] a^\dagger_{n_4^\prime} a^\dagger_{n_5^\prime}|\Phi\rangle\right.\nonumber\\
   &\left. +\langle\Phi|:\hat{O}:[\Omega] a^\dagger_{n_4^\prime} a^\dagger_{n_5^\prime}|\Phi\rangle\langle\Phi| a_{n_{2}^\prime}[\Omega] a^\dagger_{n_6^\prime}|\Phi\rangle -\langle\Phi|:\hat{O}:[\Omega] a^\dagger_{n_4^\prime} a^\dagger_{n_6^\prime}|\Phi\rangle \langle\Phi| a_{n_{2}^\prime}[\Omega] a^\dagger_{n_5^\prime}|\Phi\rangle\right.\nonumber\\
   &\left. +\langle\Phi|:\hat{O}:[\Omega] a^\dagger_{n_5^\prime} a^\dagger_{n_6^\prime}|\Phi\rangle\langle\Phi| a_{n_{2}^\prime}[\Omega] a^\dagger_{n_4^\prime}|\Phi\rangle\right]+\langle\Phi| a_{n_{1}}:\hat{O}:[\Omega]a^\dagger_{n_{3}}|\Phi\rangle\nonumber\\
 %%%%%%%%%%%%%%%%%%%%%%%%%%%%%%%%%%%%%%%
 &\hspace{0.5cm}\left[\langle\Phi| a_{n_{2}^\prime}:\hat{O}^\dagger:[\Omega] a^\dagger_{n_4^\prime}|\Phi\rangle\langle\Phi| [\Omega]a^\dagger_{n_5^\prime} a^\dagger_{n_6^\prime}|\Phi\rangle -\langle\Phi|  a_{n_{2}^\prime}:\hat{O}^\dagger:[\Omega] a^\dagger_{n_5^\prime}|\Phi\rangle\langle\Phi|[\Omega] a^\dagger_{n_4^\prime}a^\dagger_{n_6^\prime}|\Phi\rangle\right.\nonumber\\
   &\left.+\langle\Phi| a_{n_{2}^\prime}:\hat{O}^\dagger:[\Omega] a^\dagger_{n_6^\prime}|\Phi\rangle\langle\Phi| [\Omega] a^\dagger_{n_4^\prime} a^\dagger_{n_5^\prime}|\Phi\rangle +\langle\Phi|:\hat{O}^\dagger:[\Omega] a^\dagger_{n_4^\prime} a^\dagger_{n_5^\prime}|\Phi\rangle\langle\Phi| a_{n_{2}^\prime}[\Omega] a^\dagger_{n_6^\prime}|\Phi\rangle\right.\nonumber\\
   &\left.-\langle\Phi|:\hat{O}^\dagger:[\Omega] a^\dagger_{n_4^\prime} a^\dagger_{n_6^\prime}|\Phi\rangle \langle\Phi| a_{n_{2}^\prime}[\Omega] a^\dagger_{n_5^\prime}|\Phi\rangle +\langle\Phi|:\hat{O}^\dagger:[\Omega] a^\dagger_{n_5^\prime} a^\dagger_{n_6^\prime}|\Phi\rangle\langle\Phi| a_{n_{2}^\prime}[\Omega] a^\dagger_{n_4^\prime}|\Phi\rangle\right]\nonumber\\
   %%%%%%%%%%%%%%%%%%%%%%%%%%%%%%%%%%%%%%%%%%%%%%%%%%%%%%%%%%%%%%%%%%% 
 & + \langle\Phi| a_{n_{1}}[\Omega]a^\dagger_{n_{3}}|\Phi\rangle\left[\langle\Phi| a_{n_{2}^\prime}:\hat{O}^{\dagger}:[\Omega] a^\dagger_{n_4^\prime}|\Phi\rangle\langle\Phi|:\hat{O}:[\Omega]a^\dagger_{n_5^\prime} a^\dagger_{n_6^\prime}|\Phi\rangle + \langle\Phi| a_{n_{2}^\prime}:\hat{O}:[\Omega] a^\dagger_{n_4^\prime}|\Phi\rangle\right.\nonumber\\
 &\left.\hspace{0.5cm}\langle\Phi| :\hat{O}^{\dagger}:[\Omega]a^\dagger_{n_5^\prime} a^\dagger_{n_6^\prime}|\Phi\rangle-\langle\Phi| a_{n_{2}^\prime}:\hat{O}^{\dagger}:[\Omega] a^\dagger_{n_5^\prime}|\Phi\rangle\langle\Phi| :\hat{O}:[\Omega]a^\dagger_{n_4^\prime} a^\dagger_{n_6^\prime}|\Phi\rangle\right.\nonumber\\
      &\left.-\langle\Phi| a_{n_{2}^\prime}:\hat{O}:[\Omega] a^\dagger_{n_5^\prime}|\Phi\rangle\langle\Phi| :\hat{O}^{\dagger}:[\Omega]a^\dagger_{n_4^\prime} a^\dagger_{n_6^\prime}|\Phi\rangle +\langle\Phi| a_{n_{2}^\prime}:\hat{O}^{\dagger}:[\Omega] a^\dagger_{n_6^\prime}|\Phi\rangle\langle\Phi| :\hat{O}:[\Omega]a^\dagger_{n_4^\prime} a^\dagger_{n_5^\prime}|\Phi\rangle\right.\nonumber\\
   &\left.+\langle\Phi| a_{n_{2}^\prime}:\hat{O}:[\Omega] a^\dagger_{n_6^\prime}|\Phi\rangle\langle\Phi|:\hat{O}^{\dagger}:[\Omega]a^\dagger_{n_4^\prime} a^\dagger_{n_5^\prime}|\Phi\rangle\right]
    %==============================================
   +\langle\Phi|:\hat{O}^{\dagger}:[\Omega]|\Phi\rangle
   \left[\langle\Phi| a_{n_{1}}:\hat{O}:[\Omega]a^\dagger_{n_{3}}|\Phi\rangle\right.\nonumber\\
     &\left.\hspace{0.5cm}  \langle\Phi| a_{n_{2}^\prime}[\Omega] a^\dagger_{n_4^\prime}|\Phi\rangle\langle\Phi| [\Omega]a^\dagger_{n_5^\prime} a^\dagger_{n_6^\prime}|\Phi\rangle
    + \langle\Phi|a_{n_{1}}[\Omega]a^\dagger_{n_{3}}|\Phi\rangle\langle\Phi| a_{n_{2}^\prime}:\hat{O}: [\Omega] a^\dagger_{n_4^\prime}|\Phi\rangle\langle\Phi|[\Omega]a^\dagger_{n_5^\prime} a^\dagger_{n_6^\prime}|\Phi\rangle\right.\nonumber\\
    &\left.+ \langle\Phi|a_{n_{1}}[\Omega]a^\dagger_{n_{3}}|\Phi\rangle\langle\Phi|a_{n_{2}^\prime} [\Omega] a^\dagger_{n_4^\prime}|\Phi\rangle\langle\Phi|:\hat{O}:[\Omega]a^\dagger_{n_5^\prime} a^\dagger_{n_6^\prime}|\Phi\rangle
    -\langle\Phi|a_{n_{1}}:\hat{O}:[\Omega]a^\dagger_{n_{3}}|\Phi\rangle\right.\nonumber\\
   &\left. \hspace{0.5cm} \langle\Phi| a_{n_{2}^\prime}[\Omega] a^\dagger_{n_5^\prime}|\Phi\rangle\langle\Phi|[\Omega]a^\dagger_{n_4^\prime} a^\dagger_{n_6^\prime}|\Phi\rangle
    -\langle\Phi|a_{n_{1}}[\Omega]a^\dagger_{n_{3}}|\Phi\rangle\langle\Phi| a_{n_{2}^\prime}:\hat{O}:[\Omega] a^\dagger_{n_5^\prime}|\Phi\rangle\langle\Phi|[\Omega]a^\dagger_{n_4^\prime} a^\dagger_{n_6^\prime}|\Phi\rangle\right.\nonumber\\
     &\left.  -\langle\Phi|a_{n_{1}}[\Omega]a^\dagger_{n_{3}}|\Phi\rangle\langle\Phi| a_{n_{2}^\prime}[\Omega] a^\dagger_{n_5^\prime}|\Phi\rangle\langle\Phi|:\hat{O}:[\Omega]a^\dagger_{n_4^\prime} a^\dagger_{n_6^\prime}|\Phi\rangle
   +\langle\Phi|a_{n_{1}}:\hat{O}:[\Omega]a^\dagger_{n_{3}}|\Phi\rangle\right.\nonumber\\
   &\left. \hspace{0.5cm} \langle\Phi|a_{n_{2}^\prime}[\Omega] a^\dagger_{n_6^\prime}|\Phi\rangle\langle\Phi| [\Omega]a^\dagger_{n_4^\prime} a^\dagger_{n_5^\prime}|\Phi\rangle
     +\langle\Phi|a_{n_{1}}[\Omega]a^\dagger_{n_{3}}|\Phi\rangle\langle\Phi|a_{n_{2}^\prime}:\hat{O}:[\Omega] a^\dagger_{n_6^\prime}|\Phi\rangle\langle\Phi|[\Omega]a^\dagger_{n_4^\prime} a^\dagger_{n_5^\prime}|\Phi\rangle\right.\nonumber\\
    &\left. +\langle\Phi|a_{n_{1}}[\Omega]a^\dagger_{n_{3}}|\Phi\rangle\langle\Phi|a_{n_{2}^\prime}[\Omega] a^\dagger_{n_6^\prime}|\Phi\rangle\langle\Phi|:\hat{O}:[\Omega]a^\dagger_{n_4^\prime} a^\dagger_{n_5^\prime}|\Phi\rangle \right] 
    %=====================================================
   +\langle\Phi|:\hat{O}:[\Omega]|\Phi\rangle\nonumber\\
   &\hspace{0.8cm} \left[\langle\Phi|a_{n_{1}}:\hat{O}^{\dagger}:[\Omega]a^\dagger_{n_{3}}|\Phi\rangle\langle\Phi|a_{n_{2}^\prime}[\Omega] a^\dagger_{n_4^\prime}|\Phi\rangle\langle\Phi|[\Omega]a^\dagger_{n_5^\prime} a^\dagger_{n_6^\prime}|\Phi\rangle
     + \langle\Phi|a_{n_{1}}[\Omega]a^\dagger_{n_{3}}|\Phi\rangle\right.\nonumber\\
    &\left.\hspace{0.5cm}  \langle\Phi|a_{n_{2}^\prime}:\hat{O}^{\dagger}:[\Omega] a^\dagger_{n_4^\prime}|\Phi\rangle\langle\Phi|[\Omega]a^\dagger_{n_5^\prime} a^\dagger_{n_6^\prime}|\Phi\rangle
   + \langle\Phi|a_{n_{1}}[\Omega]a^\dagger_{n_{3}}|\Phi\rangle
  \langle\Phi|a_{n_{2}^\prime} [\Omega] a^\dagger_{n_4^\prime}|\Phi\rangle\right.\nonumber\\
    &\left.\hspace{0.5cm}  \langle\Phi|:\hat{O}^{\dagger}:[\Omega]a^\dagger_{n_5^\prime} a^\dagger_{n_6^\prime}|\Phi\rangle
    -\langle\Phi|a_{n_{1}}:\hat{O}^{\dagger}:[\Omega]a^\dagger_{n_{3}}|\Phi\rangle\langle\Phi|a_{n_{2}^\prime}[\Omega] a^\dagger_{n_5^\prime}|\Phi\rangle\langle\Phi|[\Omega]a^\dagger_{n_4^\prime} a^\dagger_{n_6^\prime}|\Phi\rangle\right.\nonumber\\
  &\left. -\langle\Phi|a_{n_{1}}[\Omega]a^\dagger_{n_{3}}|\Phi\rangle
  \langle\Phi|a_{n_{2}^\prime}:\hat{O}^{\dagger}:[\Omega] a^\dagger_{n_5^\prime}|\Phi\rangle\langle\Phi|[\Omega]a^\dagger_{n_4^\prime} a^\dagger_{n_6^\prime}|\Phi\rangle
  -\langle\Phi|a_{n_{1}}[\Omega]a^\dagger_{n_{3}}|\Phi\rangle\right.\nonumber\\
  &\left.\hspace{0.5cm}  \langle\Phi|a_{n_{2}^\prime}[\Omega] a^\dagger_{n_5^\prime}|\Phi\rangle
  \langle\Phi|:\hat{O}^{\dagger}:[\Omega]a^\dagger_{n_4^\prime} a^\dagger_{n_6^\prime}|\Phi\rangle
  +\langle\Phi|a_{n_{1}}:\hat{O}^{\dagger}:[\Omega]a^\dagger_{n_{3}}|\Phi\rangle\langle\Phi|a_{n_{2}^\prime}[\Omega] a^\dagger_{n_6^\prime}|\Phi\rangle\right.\nonumber\\
  & \left.\hspace{0.5cm} \langle\Phi|[\Omega]a^\dagger_{n_4^\prime} a^\dagger_{n_5^\prime}|\Phi\rangle
    +\langle\Phi|a_{n_{1}}[\Omega]a^\dagger_{n_{3}}|\Phi\rangle
  \langle\Phi|a_{n_{2}^\prime}:\hat{O}^{\dagger}:[\Omega] a^\dagger_{n_6^\prime}|\Phi\rangle\langle\Phi|[\Omega]a^\dagger_{n_4^\prime} a^\dagger_{n_5^\prime}|\Phi\rangle\right.\nonumber\\
  & \left.  +\langle\Phi|a_{n_{1}}[\Omega]a^\dagger_{n_{3}}|\Phi\rangle\langle\Phi| a_{n_{2}^\prime}[\Omega] a^\dagger_{n_6^\prime}|\Phi\rangle\langle\Phi|:\hat{O}^{\dagger}:[\Omega]a^\dagger_{n_4^\prime} a^\dagger_{n_5^\prime}|\Phi\rangle +\langle\Phi|:\hat{O}^\dagger[\Omega]:|\Phi\rangle\right.\nonumber\\
  &\left.\hspace{0.5cm} \langle\Phi|:\hat{O}:[\Omega]|\Phi\rangle
 \langle\Phi| a_{n_{2}^\prime} a_{n_{1}} [\Omega]a^\dagger_{n_{3}} a^\dagger_{n_4^\prime} a^\dagger_{n_5^\prime} a^\dagger_{n_6^\prime}|\Phi\rangle \right]~~.\label{eq:th3}
\end{align}
}
%Substituting Eqs. (\ref{eq:th1}), (\ref{eq:th2}) and (\ref{eq:th3}) in Eq. (\ref{eq:th0}), we get the required matrix element.
The matrix element of the operator $\hat{H}[\Omega]$ between two-quasiparticle state $a^\dagger_{n_{1}} a^\dagger_{n_{2}^\prime}|\Phi\rangle$ and four-quasiparticle state $a^\dagger_{n_{3}} a^\dagger_{n_4^\prime} a^\dagger_{p_{5}^{\prime}} a^\dagger_{p_{6}^{\prime}}|\Phi\rangle$  with a coupled proton pair is given by
%====================================third==============
%\subsection{$ \langle\Phi| a_{n_{2}^\prime} a_{n_{1}} \hat{H}[\Omega]  a^\dagger_{n_{3}} a^\dagger_{n_4^\prime} a^\dagger_{p_5} a^\dagger_{p_6}|\Phi\rangle $ }
\begin{align}
 \langle\Phi| a_{n_{2}^\prime} a_{n_{1}} \hat{H}[\Omega]  a^\dagger_{n_{3}} a^\dagger_{n_4^\prime} a^\dagger_{p_{5}^{\prime}} a^\dagger_{p_{6}^{\prime}}|\Phi\rangle 
 %=   \langle\Phi|  a_{n_{2}^\prime} a_{n_{1}}\bigl( \hat{H}^{(0)}+ \hat{H}{^{(1)}}+ \hat{H}^{(2)}\bigl)[\Omega] a^\dagger_{n_{3}} a^\dagger_{n_{4}^\prime} a^\dagger_{p_{5}^{\prime}} a^\dagger_{p_{6}^{\prime}}|\Phi\rangle\nonumber\\
 =& \langle\Phi| a_{n_{2}^\prime} a_{n_{1}} \hat{H}^{(0)}[\Omega]  a^\dagger_{n_{3}} a^\dagger_{n_4^\prime} a^\dagger_{p_{5}^{\prime}} a^\dagger_{p_{6}^{\prime}}|\Phi\rangle+\langle\Phi| a_{n_{2}^\prime} a_{n_{1}} \hat{H}^{(1)}[\Omega]  a^\dagger_{n_{3}} a^\dagger_{n_4^\prime} a^\dagger_{p_{5}^{\prime}} a^\dagger_{p_{6}^{\prime}}|\Phi\rangle\nonumber\\&+\langle\Phi| a_{n_{2}^\prime} a_{n_{1}} \hat{H}^{(2)}[\Omega]  a^\dagger_{n_{3}} a^\dagger_{n_4^\prime} a^\dagger_{p_{5}^{\prime}} a^\dagger_{p_{6}^{\prime}}|\Phi\rangle~~,
  \label{eq:30}
\end{align}
where
  {\allowdisplaybreaks
%First term
\begin{align}
  \langle\Phi|a_{n_{2}^\prime} a_{n_{1}} \hat{H}^{(0)}[\Omega]  a^\dagger_{n_{3}} a^\dagger_{n_4^\prime} a^\dagger_{p_{5}^{\prime}} a^\dagger_{p_{6}^{\prime}}|\Phi\rangle = {\langle\Phi|\hat{O}|\Phi \rangle}^{2} \langle\Phi| a_{n_{1}}[\Omega]  a^\dagger_{n_{3}}|\Phi\rangle\langle\Phi|a_{n_{2}^\prime}[\Omega]a^\dagger_{n_4^\prime}|\Phi\rangle\langle\Phi|[\Omega] a^\dagger_{p_{5}^{\prime}} a^\dagger_{p_{6}^{\prime}} |\Phi\rangle~~,\label{eq:31}
\end{align}
}
%Second term
  {\allowdisplaybreaks
\begin{align}
  \langle\Phi|&a_{n_{2}^\prime} a_{n_{1}} \hat{H}^{(1)}[\Omega]  a^\dagger_{n_{3}} a^\dagger_{n_4^\prime} a^\dagger_{p_{5}^{\prime}} a^\dagger_{p_{6}^{\prime}}|\Phi\rangle\nonumber\\&=\langle\Phi|\hat{O}|\Phi\rangle\left[ \langle\Phi|a_{n_{1}} :\hat{O}^{\dagger}:[\Omega] a^\dagger_{n_{3}}|\Phi\rangle\langle\Phi|a_{n_{2}^\prime}[\Omega]a^\dagger_{n_4^\prime}|\Phi\rangle\langle\Phi|[\Omega] a^\dagger_{p_{5}^{\prime}} a^\dagger_{p_{6}^{\prime}} |\Phi\rangle
   +  \langle\Phi|a_{n_{1}} :\hat{O}:[\Omega] a^\dagger_{n_{3}}|\Phi\rangle\right.\nonumber\\
   &\left.\hspace{0.7cm}\langle\Phi|a_{n_{2}^\prime}[\Omega]a^\dagger_{n_4^\prime}|\Phi\rangle\langle\Phi|[\Omega] a^\dagger_{p_5^\prime} a^\dagger_{p_6^\prime} |\Phi\rangle
    + \langle\Phi|a_{n_{1}}[\Omega] a^\dagger_{n_{3}}|\Phi\rangle\langle\Phi|a_{n_{2}^\prime} :\hat{O}^{\dagger}:[\Omega]a^\dagger_{n_4^\prime}|\Phi\rangle\right.\nonumber\\
   &\left. \hspace{0.7cm} \langle\Phi|[\Omega] a^\dagger_{p_5^\prime} a^\dagger_{p_6^\prime} |\Phi\rangle
   + \langle\Phi|a_{n_{1}}[\Omega] a^\dagger_{n_{3}}|\Phi\rangle\langle\Phi|a_{n_{2}^\prime} :\hat{O}:[\Omega]a^\dagger_{n_4^\prime}|\Phi\rangle\langle\Phi|[\Omega] a^\dagger_{p_5^\prime} a^\dagger_{p_6^\prime} |\Phi\rangle\right.\nonumber\\
   &\left.\hspace{0.7cm} +\langle\Phi|a_{n_{1}}[\Omega] a^\dagger_{n_{3}}|\Phi\rangle\langle\Phi|a_{n_{2}^\prime} [\Omega]a^\dagger_{n_4^\prime}|\Phi\rangle
   \langle\Phi|:\hat{O}^{\dagger}: [\Omega] a^\dagger_{p_5^\prime} a^\dagger_{p_6^\prime}|\Phi\rangle
  +\langle\Phi|a_{n_{1}}[\Omega] a^\dagger_{n_{3}}|\Phi\rangle\right.\nonumber\\
  &\left.\hspace{0.7cm}\langle\Phi| a_{n_{2}^\prime} [\Omega]a^\dagger_{n_4^\prime}|\Phi\rangle\langle\Phi|:\hat{O}: [\Omega] a^\dagger_{p_5^\prime} a^\dagger_{p_6^\prime} |\Phi\rangle+\left(\langle\Phi|:\hat{O}^\dagger:[\Omega]|\Phi\rangle +\langle\Phi|:\hat{O}:[\Omega]|\Phi\rangle \right)\right.\nonumber\\
 &\left.\hspace{0.7cm}\langle\Phi|a_{n_{2}^\prime} a_{n_{1}} [\Omega]  a^\dagger_{n_{3}} a^\dagger_{n_4^\prime} a^\dagger_{p_{5}^{\prime}} a^\dagger_{p_{6}^{\prime}}|\Phi\rangle \right]~~,\label{eq:32}
\end{align}
  }
and
%Third term
    {\allowdisplaybreaks
\begin{align}
  \langle\Phi|&a_{n_{2}^\prime} a_{n_{1}} \hat{H}^{(2)}[\Omega]  a^\dagger_{n_{3}} a^\dagger_{n_4^\prime} a^\dagger_{p_5^\prime} a^\dagger_{p_6^\prime}|\Phi\rangle\nonumber\\&= \langle\Phi| a_{n_{1}} :\hat{O}^{\dagger}:[\Omega] a^\dagger_{n_{3}}|\Phi\rangle\langle\Phi| a_{n_{2}^\prime} :\hat{O}: [\Omega]a^\dagger_{n_4^\prime}|\Phi\rangle\langle\Phi|[\Omega] a^\dagger_{p_5^\prime} a^\dagger_{p_6^\prime}|\Phi\rangle
   + \langle\Phi|a_{n_{1}} :\hat{O}:[\Omega] a^\dagger_{n_{3}}|\Phi\rangle\nonumber\\
   &\hspace{0.6cm}  \langle\Phi|a_{n_{2}^\prime} :\hat{O}^{\dagger}:[\Omega]a^\dagger_{n_4^\prime}|\Phi\rangle\langle\Phi|[\Omega] a^\dagger_{p_5^\prime} a^\dagger_{p_6^\prime}|\Phi\rangle
    +\langle\Phi|a_{n_{1}} :\hat{O}^{\dagger}:[\Omega] a^\dagger_{n_{3}}|\Phi\rangle\langle\Phi|a_{n_{2}^\prime} [\Omega]a^\dagger_{n_4^\prime}|\Phi\rangle\nonumber\\
   & \hspace{0.6cm} \langle\Phi|:\hat{O}:[\Omega] a^\dagger_{p_5^\prime} a^\dagger_{p_6^\prime} |\Phi\rangle
    +\langle\Phi|a_{n_{1}} :\hat{O}:[\Omega] a^\dagger_{n_{3}}|\Phi\rangle\langle\Phi| a_{n_{2}^\prime} [\Omega]a^\dagger_{n_4^\prime}|\Phi\rangle\langle\Phi|:\hat{O}^{\dagger}:[\Omega] a^\dagger_{p_5^\prime} a^\dagger_{p_6^\prime} |\Phi\rangle\nonumber\\
  &\hspace{0.6cm}   +\langle\Phi|a_{n_{1}}[\Omega] a^\dagger_{n_{3}}|\Phi\rangle
    \langle\Phi|a_{n_{2}^\prime} :\hat{O}^{\dagger}:[\Omega]a^\dagger_{n_4^\prime}|\Phi\rangle\langle\Phi|:\hat{O}:[\Omega] a^\dagger_{p_5^\prime} a^\dagger_{p_6^\prime}|\Phi \rangle
    +\langle\Phi| a_{n_{1}}[\Omega] a^\dagger_{n_{3}}|\Phi\rangle\nonumber\\
    &\hspace{0.6cm} \langle\Phi|a_{n_{2}^\prime} :\hat{O}:[\Omega]a^\dagger_{n_4^\prime}|\Phi\rangle\langle\Phi|:\hat{O}^{\dagger}:[\Omega] a^\dagger_{p_5^\prime} a^\dagger_{p_6^\prime}|\Phi\rangle
    +\langle\Phi|:\hat{O}^{\dagger}:[\Omega]|\Phi\rangle\langle\Phi|a_{n_{1}}[\Omega] a^\dagger_{n_{3}}|\Phi\rangle\nonumber\\
  &\hspace{0.6cm}   \langle\Phi|a_{n_{2}^\prime}[\Omega]a^\dagger_{n_4^\prime}|\Phi\rangle\langle\Phi|:\hat{O}:[\Omega] a^\dagger_{p_5^\prime} a^\dagger_{p_6^\prime} |\Phi\rangle
   +\langle\Phi|:\hat{O}:[\Omega]|\Phi\rangle\langle\Phi| a_{n_{1}}[\Omega] a^\dagger_{n_{3}}|\Phi\rangle\nonumber\\
   & \hspace{0.6cm}\langle\Phi|a_{n_{2}^\prime}[\Omega]a^\dagger_{n_4^\prime}|\Phi\rangle\langle\Phi|:\hat{O}^{\dagger}:[\Omega] a^\dagger_{p_5^\prime} a^\dagger_{p_6^\prime}|\Phi\rangle
   +\langle\Phi|:\hat{O}^{\dagger}:[\Omega]|\Phi\rangle\langle\Phi| a_{n_{1}}:\hat{O}:[\Omega] a^\dagger_{n_{3}}|\Phi\rangle\nonumber\\
   &\hspace{0.6cm}  \langle\Phi|a_{n_{2}^\prime}[\Omega]a^\dagger_{n_4^\prime}|\Phi\rangle\langle\Phi|[\Omega] a^\dagger_{p_5^\prime} a^\dagger_{p_6^\prime} |\Phi\rangle
   +\langle\Phi|:\hat{O}:[\Omega]|\Phi\rangle\langle\Phi|a_{n_{1}}:\hat{O}^{\dagger}:[\Omega] a^\dagger_{n_{3}}|\Phi\rangle\nonumber\\
 &\hspace{0.6cm}   \langle\Phi| a_{n_{2}^\prime}[\Omega]a^\dagger_{n_4^\prime}|\Phi\rangle\langle\Phi|[\Omega] a^\dagger_{p_5^\prime} a^\dagger_{p_6^\prime}| \Phi\rangle
    +\langle\Phi|:\hat{O}^{\dagger}:[\Omega]|\Phi\rangle\langle\Phi|a_{n_{1}}[\Omega] a^\dagger_{n_{3}}|\Phi\rangle\nonumber\\
    &\hspace{0.6cm} \langle\Phi| a_{n_{2}^\prime}:\hat{O}:[\Omega]a^\dagger_{n_4^\prime}|\Phi\rangle\langle\Phi|[\Omega] a^\dagger_{p_5^\prime} a^\dagger_{p_6^\prime} |\Phi\rangle
    +\langle\Phi|:\hat{O}:[\Omega]|\Phi\rangle\langle\Phi|a_{n_{1}}[\Omega] a^\dagger_{n_{3}}|\Phi\rangle\nonumber\\
    &\hspace{0.6cm}  \langle\Phi| a_{n_{2}^\prime}:\hat{O}^{\dagger}:[\Omega]a^\dagger_{n_4^\prime}|\Phi\rangle\langle\Phi|[\Omega] a^\dagger_{p_5^\prime} a^\dagger_{p_6^\prime} |\Phi\rangle+\langle\Phi|:\hat{O}^\dagger:[\Omega]|\Phi\rangle\langle\Phi|:\hat{O}:[\Omega]|\Phi\rangle\nonumber\\
 & \hspace{0.6cm}\langle\Phi|a_{n_{2}^\prime} a_{n_{1}}[\Omega]  a^\dagger_{n_{3}} a^\dagger_{n_4^\prime} a^\dagger_{p_5^\prime} a^\dagger_{p_6^\prime}|\Phi\rangle~~. \label{eq:33}
\end{align}
}
%Substituting Eqs. (\ref{eq:31}), (\ref{eq:32}) and (\ref{eq:33}) in Eq. (\ref{eq:30}), we get the required matrix element.
The  matrix element between two-neutron and  two-proton excitation with a coupled neutron pair is given by
{\allowdisplaybreaks
  \begin{align}
    \langle\Phi|a_{n_2^\prime} a_{n_1}\hat{H}[\Omega]& a^\dagger_{p_1} a^\dagger_{p_2^\prime} a^\dagger_{n_3^\prime} a^\dagger_{n_4^\prime}|\Phi\rangle= \langle\Phi|a_{n_2^\prime} a_{n_1}\hat{H}^{(0)}[\Omega] a^\dagger_{p_1} a^\dagger_{p_2^\prime} a^\dagger_{n_3^\prime} a^\dagger_{n_4^\prime}|\Phi\rangle\nonumber\\
    &+\langle\Phi|a_{n_2^\prime} a_{n_1}\hat{H}^{(1)}[\Omega] a^\dagger_{p_1} a^\dagger_{p_2^\prime} a^\dagger_{n_3^\prime} a^\dagger_{n_4^\prime}|\Phi\rangle+\langle\Phi|a_{n_2^\prime} a_{n_1}\hat{H}^{(2)}[\Omega] a^\dagger_{p_1} a^\dagger_{p_2^\prime} a^\dagger_{n_3^\prime} a^\dagger_{n_4^\prime}|\Phi\rangle~~,\label{l0}
  \end{align}
  }
where
{\allowdisplaybreaks
  \begin{align}
    \langle\Phi|a_{n_2^\prime} a_{n_1}&\hat{H}^{(0)}[\Omega] a^\dagger_{p_1} a^\dagger_{p_2^\prime} a^\dagger_{n_3^\prime} a^\dagger_{n_4^\prime}|\Phi\rangle={\langle\Phi|\hat{O}[\Omega]\Phi\rangle}^2\langle\Phi|a_{n_2^\prime} a_{n_1}[\Omega] a^\dagger_{p_1} a^\dagger_{p_2^\prime} a^\dagger_{n_3^\prime} a^\dagger_{n_4^\prime}|\Phi\rangle\nonumber\\
    &={\langle\Phi|\hat{O}[\Omega]\Phi\rangle}^2\left[\langle\Phi|a_{n_2^\prime} a_{n_1}[\Omega] a^\dagger_{p_1} a^\dagger_{p_2^\prime}|\Phi\rangle \langle\Phi|[\Omega]a^\dagger_{n_3^\prime} a^\dagger_{n_4^\prime}|\Phi\rangle\right.\nonumber\\
       &\left.-\langle\Phi|a_{n_2^\prime} a_{n_1}[\Omega] a^\dagger_{p_1} a^\dagger_{n_3^\prime}|\Phi\rangle\langle\Phi|[\Omega] a^\dagger_{p_2^\prime}a^\dagger_{n_4^\prime}|\Phi\rangle+\langle\Phi|a_{n_2^\prime} a_{n_1}[\Omega]a^\dagger_{p_2^\prime}a^\dagger_{n_3^\prime} |\Phi\rangle
    \langle\Phi|[\Omega] a^\dagger_{p_1}a^\dagger_{n_4^\prime}|\Phi\rangle\right.\nonumber\\
    &\left.-\langle\Phi|a_{n_2^\prime}[\Omega] a^\dagger_{p_1}a^\dagger_{p_2^\prime}a^\dagger_{n_3^\prime} |\Phi\rangle\langle\Phi| a_{n_1}[\Omega] a^\dagger_{n_4^\prime}|\Phi\rangle
    +\langle\Phi| a_{n_1}[\Omega]a^\dagger_{p_1}a^\dagger_{p_2^\prime}a^\dagger_{n_3^\prime} |\Phi\rangle\langle\Phi|a_{n_2^\prime}[\Omega]a^\dagger_{n_4^\prime}|\Phi\rangle\right]~~.\label{l1}
 \end{align}
}
First two terms vanish due to  Eq.~(\ref{c1}), third and fourth terms are zero due to  Eq.~(\ref{overlap}).
%third term is zero as $[\Omega]$ does not mix neutron and proton states and fourth term is zero due to parity symmetry.
 Now  evaluating the last term,
{\allowdisplaybreaks
  \begin{align}
    \langle\Phi| a_{n_1}[\Omega] a^\dagger_{p_1}a^\dagger_{p_2^\prime}a^\dagger_{n_3^\prime}|\Phi\rangle=&\langle\Phi| a_{n_1}[\Omega] a^\dagger_{p_1}|\Phi\rangle \langle\Phi|[\Omega]a^\dagger_{p_2^\prime}a^\dagger_{n_3^\prime}|\Phi\rangle-\langle\Phi| a_{n_1}[\Omega]a^\dagger_{p_2^\prime}|\Phi\rangle\langle\Phi|[\Omega]  a^\dagger_{p_1}a^\dagger_{n_3^\prime}|\Phi\rangle\nonumber\\
    &+\langle\Phi| a_{n_1}[\Omega]a^\dagger_{n_3^\prime}|\Phi\rangle\langle\Phi|[\Omega]a^\dagger_{p_1}a^\dagger_{p_2^\prime}|\Phi\rangle\nonumber\\
    &=0~~.\label{pn2}
  \end{align}
}
Therefore, 
{\allowdisplaybreaks
  \begin{align}
    \langle\Phi|a_{n_2^\prime} a_{n_1}\hat{H}^{(0)}[\Omega] a^\dagger_{p_1} a^\dagger_{p_2^\prime} a^\dagger_{n_3^\prime} a^\dagger_{n_4^\prime}|\Phi\rangle=0~~.\label{pn4}
  \end{align}
}
The second term of Eq.~(\ref{l0}) is
%==============H1 term=========================== 
{\allowdisplaybreaks
  \begin{align}
    \langle\Phi|&a_{n_2^\prime} a_{n_1}\hat{H}^{(1)}[\Omega] a^\dagger_{p_1} a^\dagger_{p_2^\prime} a^\dagger_{n_3^\prime} a^\dagger_{n_4^\prime}|\Phi\rangle=\langle\Phi|a_{n_2^\prime} a_{n_1}\hat{H}^{(1)}[\Omega] a^\dagger_{p_1} a^\dagger_{p_2^\prime}|\Phi\rangle\langle\Phi|[\Omega] a^\dagger_{n_3^\prime} a^\dagger_{n_4^\prime}|\Phi\rangle\nonumber\\
    & +\langle\Phi|a_{n_2^\prime} a_{n_1}[\Omega] a^\dagger_{p_1} a^\dagger_{p_2^\prime}|\Phi\rangle\langle\Phi|\hat{H}^{(1)}[\Omega] a^\dagger_{n_3^\prime} a^\dagger_{n_4^\prime}|\Phi\rangle -\langle\Phi|a_{n_2^\prime} a_{n_1}\hat{H}^{(1)}[\Omega] a^\dagger_{p_1}a^\dagger_{n_3^\prime}|\Phi\rangle\langle\Phi|[\Omega]a^\dagger_{p_2^\prime}a^\dagger_{n_4^\prime}|\Phi\rangle\nonumber\\
    & -\langle\Phi|a_{n_2^\prime} a_{n_1}[\Omega] a^\dagger_{p_1}a^\dagger_{n_3^\prime}|\Phi\rangle\langle\Phi|\hat{H}^{(1)}[\Omega]a^\dagger_{p_2^\prime}a^\dagger_{n_4^\prime}|\Phi\rangle
    +\langle\Phi|a_{n_2^\prime} a_{n_1}\hat{H}^{(1)}[\Omega]a^\dagger_{p_2^\prime}a^\dagger_{n_3^\prime}|\Phi\rangle\langle\Phi|[\Omega] a^\dagger_{p_1}a^\dagger_{n_4^\prime}|\Phi\rangle\nonumber\\
    & +\langle\Phi|a_{n_2^\prime} a_{n_1}[\Omega]a^\dagger_{p_2^\prime}a^\dagger_{n_3^\prime}|\Phi\rangle\langle\Phi|\hat{H}^{(1)}[\Omega] a^\dagger_{p_1}a^\dagger_{n_4^\prime}|\Phi\rangle-\langle\Phi|a_{n_2^\prime}\hat{H}^{(1)}[\Omega] a^\dagger_{p_1}a^\dagger_{p_2^\prime}a^\dagger_{n_3^\prime}|\Phi\rangle\langle\Phi| a_{n_1}[\Omega]a^\dagger_{n_4^\prime}|\Phi\rangle\nonumber\\
    &-\langle\Phi|a_{n_2^\prime}[\Omega] a^\dagger_{p_1}a^\dagger_{p_2^\prime}a^\dagger_{n_3^\prime}|\Phi\rangle\langle\Phi| a_{n_1}\hat{H}^{(1)}[\Omega]a^\dagger_{n_4^\prime}|\Phi\rangle
    +\langle\Phi|\hat{H}^{(1)}[\Omega]|\Phi\rangle \langle\Phi|a_{n_2^\prime} a_{n_1}[\Omega] a^\dagger_{p_1} a^\dagger_{p_2^\prime} a^\dagger_{n_3^\prime} a^\dagger_{n_4^\prime}|\Phi\rangle\nonumber\\
    &+\langle\Phi| a_{n_1}[\Omega] a^\dagger_{p_1}a^\dagger_{p_2^\prime}a^\dagger_{n_3^\prime}|\Phi\rangle\langle\Phi|a_{n_2^\prime}\hat{H}^{(1)}[\Omega]a^\dagger_{n_4^\prime}|\Phi\rangle+
    \langle\Phi| a_{n_1}\hat{H}^{(1)}[\Omega] a^\dagger_{p_1}a^\dagger_{p_2^\prime}a^\dagger_{n_3^\prime}|\Phi\rangle\langle\Phi|a_{n_2^\prime}[\Omega]a^\dagger_{n_4^\prime}|\Phi\rangle~~.\label{pn}
\end{align}
}
The first two terms vanish due to Eqs.~(\ref{c2}) and (\ref{c1}), respectively. Third, fourth, fifth, sixth,
seventh and eighth terms vanish due to Eq.~(\ref{overlap}),
%terms vanish due to overlap between neutron and proton states. Seventh and eighth terms are zero due to parity symmetry,
while as  ninth and tenth terms are zero because of  Eqs.~(\ref{pn2}) and (\ref{l1}). Now consider the last term,
{\allowdisplaybreaks
  \begin{align}
    \langle\Phi| a_{n_1}&\hat{H}^{(1)}[\Omega] a^\dagger_{p_1}a^\dagger_{p_2^\prime}a^\dagger_{n_3^\prime}|\Phi\rangle=\langle\Phi| a_{n_1}\hat{H}^{(1)}[\Omega] a^\dagger_{p_1}|\Phi\rangle \langle\Phi|[\Omega] a^\dagger_{p_2^\prime}a^\dagger_{n_3^\prime}|\Phi\rangle\nonumber\\
    &+\langle\Phi| a_{n_1}[\Omega] a^\dagger_{p_1}|\Phi\rangle \langle\Phi|\hat{H}^{(1)}[\Omega] a^\dagger_{p_2^\prime}a^\dagger_{n_3^\prime}|\Phi\rangle-\langle\Phi| a_{n_1}\hat{H}^{(1)}[\Omega] a^\dagger_{p_2^\prime}|\Phi\rangle\langle\Phi|[\Omega] a^\dagger_{p_1}a^\dagger_{n_3^\prime}|\Phi\rangle\nonumber\\
   & -\langle\Phi| a_{n_1}[\Omega] a^\dagger_{p_2^\prime}|\Phi\rangle\langle\Phi|\hat{H}^{(1)}[\Omega] a^\dagger_{p_1}a^\dagger_{n_3^\prime}|\Phi\rangle +\langle\Phi| a_{n_1}\hat{H}^{(1)}[\Omega]a^\dagger_{n_3^\prime}|\Phi\rangle \langle\Phi|[\Omega] a^\dagger_{p_1}a^\dagger_{p_2^\prime}|\Phi\rangle\nonumber\\
   & +\langle\Phi| a_{n_1}[\Omega]a^\dagger_{n_3^\prime}|\Phi\rangle \langle\Phi|\hat{H}^{(1)}[\Omega] a^\dagger_{p_1}a^\dagger_{p_2^\prime}|\Phi\rangle +\langle\Phi|\hat{H}^{(1)}[\Omega]|\Phi\rangle\langle\Phi| a_{n_1}[\Omega] a^\dagger_{p_1}a^\dagger_{p_2^\prime}a^\dagger_{n_3^\prime}|\Phi\rangle\nonumber
  \end{align}
}
Using  Eqs.~(\ref{overlap}) and (\ref{pn2})
%parity symmetry or vanishing overlap between neutron and proton states
in the above equation, we have
{\allowdisplaybreaks
  \begin{align}
    \langle\Phi| a_{n_1}\hat{H}^{(1)}[\Omega] a^\dagger_{p_1}a^\dagger_{p_2^\prime}a^\dagger_{n_3^\prime}|\Phi\rangle=0~~.\label{pn3}
  \end{align}
}
%===============================================================
%===========================================
Similarly, we can prove that
{\allowdisplaybreaks
  \begin{align}
   \langle\Phi|a_{n_2^\prime} a_{n_1}\hat{H}^{(2)}[\Omega] a^\dagger_{p_1} a^\dagger_{p_2^\prime} a^\dagger_{n_3^\prime} a^\dagger_{n_4^\prime}|\Phi\rangle=0~~.\label{lh2}
  \end{align}
}
Therefore, $$ \langle\Phi|a_{n_2^\prime} a_{n_1}\hat{H}[\Omega] a^\dagger_{p_1} a^\dagger_{p_2^\prime} a^\dagger_{n_3^\prime} a^\dagger_{n_4^\prime}|\Phi\rangle=0~~.$$
It can be shown that matrix elements for all other states in Eq.~(\ref{basis}) between neutron and proton excitations vanish.

}
%\newpage
\twocolumngrid

\bibliographystyle{apsrev4-2}
\bibliography{jacwit}






\end{document}
