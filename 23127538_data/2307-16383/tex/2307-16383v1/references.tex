\begin{thebibliography}{100}
\bibitem{Best-COSB-2017} Best, R. B. Computational and theoretical advances in studies of intrinsically disordered proteins. Current Opinion in Structural Biology 42, 147-154 (2017).

\bibitem{IDP_Review1} Ghosh, K., Huihui, J., Phillips, M. \& Haider, A. Rules of Physical Mathematics Govern Intrinsically Disordered Proteins. Annual Review of Biophysics 51, 355-376 (2022).

\bibitem{IDP_Review2} Ehm, T. {\em et al.} Intrinsically disordered proteins at the nano-scale. Nano Futures 5, 022501 (2021).

\bibitem{IDP_Review3} Evans, R., Ramisetty, S., Kulkarni, P. \& Weninger, K. Illuminating Intrinsically Disordered Proteins with Integrative Structural Biology. Biomolecules 13, 124 (2023).

\bibitem{IDP_Review4} Uversky, V. N., Oldfield, C. J. \& Dunker, A. K. Intrinsically Disordered Proteins in Human Diseases: Introducing the D2Concept. Annual Review of Biophysics 37, 215-246 (2008).
  
\bibitem{Uversky2000} Uversky, V. N., Gillespie, J. R. \& Fink, A. L. Why are ``natively unfolded" proteins unstructured under physiologic conditions? Proteins: Structure, Function, and Genetics 41, 415-427 (2000).

\bibitem{DisProt} Sickmeier, M. {\em et al.} DisProt: the Database of Disordered Proteins. Nucleic Acids Research 35, D786-D793 (2007).

\bibitem{Ferrie} Ferrie, J. J., Karr, J. P., Tjian, R. \& Darzacq, X. \textit{``Structure’’}-function relationships in eukaryotic transcription factors: The role of intrinsically disordered regions in gene regulation. Molecular Cell 82, 3970-3984 (2022).
  
\bibitem{Giansanti} Deiana, A., Forcelloni, S., Porrello, A. \& Giansanti, A. Intrinsically disordered proteins and structured proteins with intrinsically disordered regions have different functional roles in the cell. PLOS ONE 14, e0217889 (2019).

\bibitem{Best_Nature2018} Borgia, A. {\em et al.} Extreme disorder in an ultrahigh-affinity protein complex. Nature 555, 61-66 (2018).

\bibitem{Schuler_NatCommn} Andrea Sottini, Alessandro Borgia, Madeleine B. Borgia, Katrine Bugge, Daniel Nettele, 
Aritra Chowdhury, P\'{e}tur O. Heidarsson, Franziska Zosel, Robert B. Best, 
Birthe B. Kragelund,  \& Benjamin Schuler, Nat. Commn. (2020) 11:5736 
  
\bibitem{Fung2018} Fung, H. Y. J., Birol, M. \& Rhoades, E. IDPs in macromolecular complexes: the roles of multivalent interactions in diverse assemblies. Current Opinion in Structural Biology 49, 36-43 (2018).

\bibitem{Uversky2022} Coskuner-Weber, O., Mirzanli, O. \& Uversky, V. N. Intrinsically disordered proteins and proteins with intrinsically disordered regions in neurodegenerative diseases. Biophysical Reviews 14, 679-707 (2022).

\bibitem{Svergun} Bernadó, P. \& Svergun, D. I. Structural analysis of intrinsically disordered proteins by small-angle X-ray scattering. Mol. BioSyst. 8, 151-167 (2012).

\bibitem{Schuler_Review} Schuler, B., Soranno, A., Hofmann, H. \& Nettels, D. Single-Molecule FRET Spectroscopy and the Polymer Physics of Unfolded and Intrinsically Disordered Proteins. Annual Review of Biophysics 45, 207-231 (2016).

\bibitem{Schuler_PNAS2012} Hofmann, H. {\em et al.} Polymer scaling laws of unfolded and intrinsically disordered proteins quantified with single-molecule spectroscopy. Proceedings of the National Academy of Sciences 109, 16155-16160 (2012).

\bibitem{Schuler_JCP2018} Schuler, B. Perspective: Chain dynamics of unfolded and intrinsically disordered proteins from nanosecond fluorescence correlation spectroscopy combined with single-molecule FRET. The Journal of Chemical Physics 149, (2018).
  
\bibitem{Tompa2013} Kosol, S., Contreras-Martos, S., Cedeño, C. \& Tompa, P. Structural Characterization of Intrinsically Disordered Proteins by NMR Spectroscopy. Molecules 18, 10802-10828 (2013).

\bibitem{expt_all} Gomes, G.-N. W. {\em et al.} Conformational Ensembles of an Intrinsically Disordered Protein Consistent with NMR, SAXS, and Single-Molecule FRET. Journal of the American Chemical Society 142, 15697-15710 (2020).

\bibitem{Ausbaugh} Ashbaugh, H. S. \& Hatch, H. W. Natively Unfolded Protein Stability as a Coil-to-Globule Transition in Charge/Hydropathy Space. Journal of the American Chemical Society 130, 9536-9542 (2008).

\bibitem{Mittal2018}Dignon, G. L., Zheng, W., Kim, Y. C., Best, R. B. \& Mittal, J. Sequence determinants of protein phase behavior from a coarse-grained model. PLOS Computational Biology 14, e1005941 (2018).

\bibitem{Larsen2021}Tesei, G., Schulze, T. K., Crehuet, R. \& Lindorff-Larsen, K. Accurate model of liquid-liquid phase behavior of intrinsically disordered proteins from optimization of single-chain properties. Proceedings of the National Academy of Sciences 118, (2021).

\bibitem{Pappu_Package}Lalmansingh, J. M., Keeley, A. T., Ruff, K. M., Pappu, R. V. \& Holehouse, A. S. SOURSOP: A Python package for the analysis of simulations of intrinsically disordered proteins. (2023) doi:10.1101/2023.02.16.528879.

\bibitem{Thirumalai_2019} Baul, U., Chakraborty, D., Mugnai, M. L., Straub, J. E. \& Thirumalai, D. Sequence Effects on Size, Shape, and Structural Heterogeneity in Intrinsically Disordered Proteins. The Journal of Physical Chemistry B 123, 3462-3474 (2019).

\bibitem{COINT} Johnson, C. L. {\em et al.} The Two-State Prehensile Tail of the Antibacterial Toxin Colicin N. Biophysical Journal 113, 1673-1684 (2017).
  
\bibitem{FhuA} Riback, J. A. {\em et al.} Innovative scattering analysis shows that hydrophobic disordered proteins are expanded in water. Science 358, 238–241 (2017).
  
\bibitem{Weninger_SIC1} Evans, R., Ramisetty, S., Kulkarni, P. \& Weninger, K. Illuminating Intrinsically Disordered Proteins with Integrative Structural Biology. Biomolecules 13, 124 (2023).
  
\bibitem{OPN} Platzer, G. {\em et al.} The Metastasis-Associated Extracellular Matrix Protein Osteopontin Forms Transient Structure in Ligand Interaction Sites. Biochemistry 50, 6113-6124 (2011).
  
\bibitem{Gomes_SIC1} Gomes, G.-N. W. {\em et al.} Conformational Ensembles of an Intrinsically Disordered Protein Consistent with NMR, SAXS, and Single-Molecule FRET. Journal of the American Chemical Society 142, 15697-15710 (2020).

\bibitem{histatin5} Jephthah, S., Staby, L., Kragelund, B. B. \& Skep\"{o}, M. Temperature Dependence of Intrinsically Disordered Proteins in Simulations: What are We Missing? Journal of Chemical Theory and Computation 15, 2672-2683 (2019).

\bibitem{Kyte} Kyte, J. \& Doolittle, R. F. A simple method for displaying the hydropathic character of a protein. Journal of Molecular Biology 157, 105-132 (1982).

\bibitem{Habchi}Habchi, J., Tompa, P., Longhi, S. \& Uversky, V. N. Introducing Protein Intrinsic Disorder. Chemical Reviews 114, 6561-6588 (2014).

\bibitem{Mittal2022}Devarajan, D. S. {\em et al.} Effect of Charge Distribution on the Dynamics of Polyampholytic Disordered Proteins. Macromolecules 55, 8987-8997 (2022).

\bibitem{Alberti-LLPS2019}Simon Alberti, Amy Gladfelter, and Tanja Mittag, Cell {\bf 176}, 419 (2019). 

\bibitem{Dorfmann-LLPS2019} Simon Alberti and Dorothee Dormann, Annu. Rev. Genet. 2019. 53:171-94. 

\bibitem{McCarty-LLPS2019} McCarty, J., Delaney, K. T., Danielsen, S. P. O., Fredrickson, G. H. \& Shea, J.-E. Complete Phase Diagram for Liquid-Liquid Phase Separation of Intrinsically Disordered Proteins. The Journal of Physical Chemistry Letters 10, 1644-1652 (2019).

\bibitem{Muthukumar_MM2022} Das, S. \& Muthukumar, M. Microstructural Organization in $\alpha$-Synuclein Solutions. Macromolecules 55, 4228-4236 (2022). 

\bibitem{Aksimentiev_JPCL2020} Chou, H.-Y. \& Aksimentiev, A. Single-Protein Collapse Determines Phase Equilibria of a Biological Condensate. The Journal of Physical Chemistry Letters 11, 4923-4929 (2020).

\bibitem{Akerlof} Akerlof, G. C. \& Oshry, H. I. The Dielectric Constant of Water at High Temperatures and in Equilibrium with its Vapor. Journal of the American Chemical Society 72, 2844-2847 (1950).

\bibitem{Israel}J. N. Israelachvili, {\em Intermolecular and Surface forces}, 3rd edition, Elsevier (2011).

\bibitem{Thirumalai_2023}Mugnai, M. L. {\em et al.} Sizes, conformational fluctuations, and SAXS profiles for Intrinsically Disordered Proteins. (2023) doi:10.1101/2023.04.24.538147.

\bibitem{Pappu2010}Mao, A. H., Crick, S. L., Vitalis, A., Chicoine, C. L. \& Pappu, R. V. Net charge per residue modulates conformational ensembles of intrinsically disordered proteins. Proceedings of the National Academy of Sciences 107, 8183-8188 (2010).

\bibitem{Engelman} Engelman, D. M., Steitz, T. A. \& Goldman, A. IDENTIFYING NONPOLAR TRANSBILAYER HELICES IN AMINO ACID SEQUENCES OF MEMBRANE PROTEINS. Annual Review of Biophysics and Biophysical Chemistry 15, 321-353 (1986).

\bibitem{Hopp} Hopp, T. P. \& Woods, K. R. A computer program for predicting protein antigenic determinants. Molecular Immunology 20, 483-489 (1983).

\bibitem{Eisenberg} Eisenberg, D., Schwarz, E., Komaromy, M. \& Wall, R. Analysis of membrane and surface protein sequences with the hydrophobic moment plot. Journal of Molecular Biology 179, 125-142 (1984).

\bibitem{Cornette} Cornette, J. L. {\em et al.} Hydrophobicity scales and computational techniques for detecting amphipathic structures in proteins. Journal of Molecular Biology 195, 659-685 (1987).

\bibitem{Everaers_PRL} Yamakov, V., Milchev, A., J\"{o}rg Limbach, H., D\"{u}nweg, B. \& Everaers, R. Conformations of Random Polyampholytes. Physical Review Letters 85, 4305-4308 (2000).

\bibitem{Pincus_MM_1980} D.W. Schaefer, J.F. Joanny and P. Pincus, Macromolecules {\bf 13}, 1280 (1980). 

\bibitem{Nakanishi_1987}H. Nakanishi, J. Physique {\bf 48}, 979 (1987).

\bibitem{Rubinstein} M. Rubinstein and R.H. Colby, {\it Polymer Physics} (Oxford Univ. Press, 2003). 

\bibitem{Universal1}
Jacob Bair, Swarnadeep Seth, and Aniket Bhattacharya
Universality in conformations and transverse fluctuations of a semi-flexible polymer in a crowded environment.
J. Chem. Phys. {\bf 158}, 204902 (2023).

\bibitem{Universal2}
Aiqun Huang, Aniket Bhattacharya, and Kurt Binder
Conformations, Transverse Fluctuations and Crossover Dynamics of a Semi-Flexible Chain in Two Dimensions
J. Chem. Phys. {\bf 140}, 214902 (2014).

\bibitem{Pappu2013} Das, R. K. \& Pappu, R. V. Conformations of intrinsically disordered proteins are influenced by linear sequence distributions of oppositely charged residues. Proceedings of the National Academy of Sciences 110, 13392-13397 (2013).

\bibitem{Pappu2014}van der Lee, R. {\em et al.} Classification of Intrinsically Disordered Regions and Proteins. Chemical Reviews 114, 6589-6631 (2014).

\bibitem{SK-Ma} S.~K. Ma, {\em Phase Transition and Critical Phenomena}, Pergamon press.

\bibitem{Landau} L. D. Landau and E. M. Lifshitz, Statistical Physics, Part 1, 3rd ed. (Pergamon Press, 1980).

\bibitem{exponnorm} scipy.stats.exponnorm — SciPy v1.12.0.dev Manual. (n.d.). URL: http://scipy.github.io/devdocs \\
/reference/generated/scipy.stats.exponnorm.html

\bibitem{Ueda2010} Matsunaga, H. \& Ueda, H. Stress-induced non-vesicular release of prothymosin-$\alpha$ initiated by an interaction with S100A13, and its blockade by caspase-3 cleavage. Cell Death \& Differentiation 17, 1760-1772 (2010).

\bibitem{Schuler2010} Müller-Späth, S. {\em et al.} Charge interactions can dominate the dimensions of intrinsically disordered proteins. Proceedings of the National Academy of Sciences 107, 14609-14614 (2010).

\bibitem{Hoffmann} Vancraenenbroeck, R., Harel, Y. S., Zheng, W. \& Hofmann, H. Polymer effects modulate binding affinities in disordered proteins. Proceedings of the National Academy of Sciences 116, 19506-19512 (2019).

\bibitem{Reddy} Maity, H., Baidya, L. \& Reddy, G. Salt-Induced Transitions in the Conformational Ensembles of Intrinsically Disordered Proteins. The Journal of Physical Chemistry B 126, 5959-5971 (2022).

\bibitem{Wohl} Wohl, S., Jakubowski, M. \& Zheng, W. Salt-Dependent Conformational Changes of Intrinsically Disordered Proteins. The Journal of Physical Chemistry Letters 12, 6684-6691 (2021).

\bibitem{Capp2013} Record, M. T., Guinn, E., Pegram, L. \& Capp, M. Introductory Lecture: Interpreting and predicting Hofmeister salt ion and solute effects on biopolymer and model processes using the solute partitioning model. Faraday Discuss. 160, 9-44 (2013).

\bibitem{Pegram2010} Pegram, L. M. {\em et al.} Why Hofmeister effects of many salts favor protein folding but not DNA helix formation. Proceedings of the National Academy of Sciences 107, 7716-7721 (2010).

\bibitem{Origin_life} Pohorille, A., Wilson, M. A. \& Shannon, G. Flexible Proteins at the Origin of Life. Life 7, 23 (2017).


















\end{thebibliography}
