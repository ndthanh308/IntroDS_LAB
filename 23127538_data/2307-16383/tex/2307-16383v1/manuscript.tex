\documentclass[pre,aps,twocolumn,showpacs]{revtex4-2}
\usepackage{graphicx,epstopdf}% Include figure files
\usepackage{dcolumn}% Align table columns on decimal point
\usepackage{bm}% bold math
\usepackage{color}
\usepackage{xcolor}
\usepackage{float}
\usepackage{amssymb,amsmath}
%\usepackage[font=small,skip=0pt]{caption}
%\usepackage[skip=1pt]{caption}
%\captionsetup{justification=raggedleft,singlelinecheck=false}
\setlength{\abovecaptionskip}{0pt plus 3pt minus 2pt}
%\newcommand{\angstrom}{\text{\normalfont\AA}}
\usepackage{tabularx}
\begin{document}
\title{Fine structures of Intrinsically Disordered Proteins}

\author{Swarnadeep Seth}
\author{Brandon Stine}
\author{Aniket Bhattacharya}
\altaffiliation[]
{Author to whom the correspondence should be addressed}
{}
\email{Aniket.Bhattacharya@ucf.edu}

\affiliation{Department of Physics, University of Central Florida, Orlando, Florida 32816-2385, USA}
\date{\today}
\begin{abstract}
We report simulation studies of 33 single intrinsically disordered
proteins (IDPs) using three similar coarse-grained (CG) bead-spring
models where interactions among different amino acids (represented as CG beads) are introduced
through a hydropathy matrix and additional screened Coulomb
interaction to account for charges of the amino acids along the chain
backbone. A comparison of our simulation studies of two CG models [Dignon {\em et al.}, PLOS
Comp. Biology, 14, 2018, Tesei {\em et al.} PNAS, 118, 2021] with two
different hydropathy scales (HPS, M3) for a larger set of amino acids
with the existing experimental data indicates an optimal interaction
parameter $\epsilon \simeq 0.18 - 0.2$ kCal/mole, as reported
previously, however, with a larger variation. We use these 
best-fit parameters to investigate both the universal aspects as well
as the fine structures of the individual IDPs whose
characteristics vary substantially in the charge-hydropathy
space by introducing additional metrics. First, we investigate the polymer specific scaling relations of
the IDPs in comparison to the universal scaling relations
[Bair {\em et al.}, J. Chem. Phys. {\bf 158}, 204902 (2023)] for the
homopolymers and demonstrate that (i) the scaled end-to-end distance
$\langle R_N^2\rangle/(2 L\ell_p)$ and the scaled transverse fluctuation $\tilde{l}^2_{\perp}=\sqrt{\langle{l_{\perp}^2}\rangle}/{L}$ 
are Gaussian. Then we introduce (ii) Wilson charge index
($\mathcal{W}$) that captures the essential features of charge
interactions and distribution in the sequence space, 
(iii) a skewness parameter ($\mathcal{S}$) that captures the finer
shape variation of the gyration radii distribution related to the
charge asymmetry. Finally, our study of the (iv) variation of $\langle R_g\rangle
$ as a function of salt concentration provides another important metric to bring
out finer characteristics of the IDPs which may carry relevant information for the origin of life.
\end{abstract}
%%%%%%%%%%%%%%%%%%%%%%%%%%%%%%%%%%%%%%%%%%
\maketitle
\section{Introduction}
Intrinsically disordered proteins (IDPs) are low complexity proteins characterized by
a low proportion of hydrophobic residues
and a high content of polar and charged amino acids which make them
distinct from those which fold. IDPs lack well defined three-dimensional structures
and do not participate in forming $\alpha$-helices or $\beta$-strands,
and other secondary or tertiary structures~\cite{Best-COSB-2017}-\cite{Uversky2000}.
Since their discovery almost three decades ago the number of IDPs has
been growing at a steady rate~\cite{DisProt}. It is now known that almost 30\% of the
proteins are either IDPs or folded proteins have intrinsically
disordered regions (IDR) which play crucial roles in numerous biological processes, such as regulating
signaling pathways, helping in molecular recognition,
in initiating protein-protein interactions, and serve
as molecular switches~\cite{Ferrie, Giansanti}. 
The conformal flexibility of IDPs help mediate interactions with binding partners to 
form components of macromolecular complexes~\cite{Best_Nature2018,Schuler_NatCommn}. The flexibility and
often faster dynamics allow IDPs to bind to multiple different
proteins~\cite{Fung2018}. The IDP complexes~\cite{Best_Nature2018} has also been realized to play 
a central to the pathology of several degenerative diseases:
$\alpha$-synuclein (Parkinson’s disease), tau (Alzheimer’s disease), and IAPP
(Type II Diabetes)~\cite{Uversky2022}. \par
Evidently the studies of IDPs in the last two decades
has been an active area in various branches of science. Despite
tremendous growth and interest in studying IDPs - the discovery of new
IDPs and their fast dynamics made it difficult to study
experimentally using small angle X-Ray scattering (XAFS)~\cite{Svergun}, single molecule
Fourier resonance energy transfer (smFRET)~\cite{Schuler_PNAS2012, Schuler_Review, Schuler_JCP2018} and solution nuclear magnetic
resonance (sNMR)~\cite{Tompa2013} which have produced conflicting results. The
conformational information, such as end-to-end distance and gyration
radii are also available for a limited number of IDPs.
Thus, an integrative structural biology approach that
combines experimental techniques~\cite{expt_all}, such as NMR spectroscopy
and small-angle X-ray scattering, combined with computational
methods seem to be a practical and feasible approach
to unravel the conformational properties and interactions
of IDPs, shedding light on their structural ensembles.\par
Historically, computer simulation studies of CG models of polymers have played an important role
as a stand-alone discipline between theory and experiments
successfully predicting conformational and dynamic properties of
neutral and charged polymers. Similar studies have been
generalized for the IDPs taking into account different sizes, charges, and hydropathy index of the 20 different CG amino acid
beads~\cite{Ausbaugh,Mittal2018,Larsen2021,Pappu_Package,Thirumalai_2019}. The
first goal of this article is to enlarge the scope of validity of a
subset of these models (HPS and M3) by studying a large number of IDPs
and checking the convergence of the results obtained using other CG models~\cite{Pappu_Package,Thirumalai_2019}, as well as experimentally~\cite{COINT,FhuA,Weninger_SIC1,OPN,Gomes_SIC1,histatin5} as outlined below.
%%%%%%%%%%%%%%%%%%%%%%%%%%%%%%%%%%%%%%%%%%%%%
\section{Coarse grained models of IDPs}
One of the hallmarks of IDPs is their characterization using the
Uversky plot ~\cite{Uversky2000} where it has been shown 
that when the mean net absolute charge $\langle Q \rangle$
of a polypeptide chain at
neutral pH is plotted against the mean side chain hydropathy $\langle
H \rangle $, measured on the Kyte-Doolittle~\cite{Kyte}
hydrophobicity scale, a 
boundary line
\begin{equation}
\langle Q \rangle =
2.785\langle H \rangle - 1.151
\label{QH_plot}
\end{equation}
separates the compact (natively folded or
globular) and expanded (coil-like or pre-molten globular)
conformations~\cite{Uversky2000,Habchi}. Habchi {\em et al.}
~\cite{Habchi} improved
Eqn.~\ref{QH_plot} but the basic observation remains the same.
Evidently, a relatively large charge and a
small hydropathy ensure the extended structure of the IDPs. The simplicity has been appealing to build 
CG models of IDPs based on hydropathy, where the standard bead-spring model of a homopolymer has been generalized to incorporate the relative well depth between any two amino acids by through a hydropathy
matrix.~\cite{Ausbaugh}. 
Mittal and coworkers~\cite{Mittal2018,Mittal2022} have used this HPS model to compare the gyration
radii for several IDPs and found a reasonably good agreement. A
slightly different version has been used by Tesei~{\em et al.}~\cite{Larsen2021}. Unlike the HPS
models~\cite{Ausbaugh,Mittal2018,Larsen2021} where hydropathy is
introduced directly, other implicit solvent CG models have been used to study various properties of IDPs. Pappu and coworkers
developed a software called  ABSINTH (Assembly
of Biomolecules Studied by an Implicit, Novel, and Tunable
Hamiltonian) to study phase transitions in
IDPs~\cite{Pappu_Package}. Thirumalai and coworkers used another CG
model called SOP-IDP (self-organized polymer (SOP) coarse-grained
model for IDPs) with a finer level of granularity where,
except for Glycine and Alanine, the rest of the amino acid residues are
represented using a backbone bead and a side-chain (SC)
bead~\cite{Thirumalai_2019}. All these models are computationally more
efficient compared to the models with explicit solvent molecules and hence can be used to
study macromolecular condensates of IDPs leading to liquid-liquid phase
separation~\cite{Alberti-LLPS2019,Dorfmann-LLPS2019, McCarty-LLPS2019,Aksimentiev_JPCL2020,Muthukumar_MM2022} in membraneless organelles.\par
Studies of IDPs are relatively new and progressively more IDPs are
being cataloged~\cite{DisProt}. Compared to the models for the folded proteins, the
CG models of the IDPs are relatively new. Due to their flexibility and
faster dynamics, the experimental studies of the IDPs are relatively limited and often very difficult to interpret. Thus, studies of several CG models with convergence to the
experimental results is an important aspect of developing a better understanding of the statistical mechanical properties of the IDPs  which share properties of the polyelectrolytes and polyampholytes, but exhibit very different sequence-specific behaviors.
One of the goals of this paper is to critically examine the parameters of two hydropathy models of IDPs as
explained below by enlarging the scope of previous
work~\cite{Mittal2018}-\cite{Larsen2021}.
It is also worth noting,  and as we will provide specific examples in the result section, that IDPs sampled from different regions of the Charge-Hydropthay space may exhibit markedly different
characteristics. We have introduced new physically motivated metrics to analyze these fine structures of the IDPs
%% Figure environment removed

\section{Introduction}
Automatic 3D reconstruction of clothed humans using image inputs has gained increasing significance due to its potential applications in a wide array of AR/VR scenarios. High-fidelity reconstructions typically depend on sophisticated capture systems, which are developed with dense camera arrays~\cite{collet2015high,joo2015panoptic,joo2018total}, programmable light-stages~\cite{Vlasic2009, guo2019relightables}, and depth sensors~\cite{newcombe2011kinectfusion,DoubleFusion,BodyFusion,dou2016fusion4d,newcombe2015dynamicfusion}. However, stringent capture environments equipped with complex hardware pose significant challenges for consumer-level applications.


In this context, considerable research effort has been dedicated to developing methods that allow for more flexible capture configurations, such as utilizing a few RGB inputs. Among these works, learning implicit functions \cite{iccv2020PIFu, saito2020pifuhd, hong2021stereopifu} has proven effective in achieving highly detailed reconstructions by integrating the advancements of deep neural networks. These methods employ large multi-layer perceptrons (MLPs) to predict the occupancy probability or truncated signed distance function (TSDF) value of every queried 3D point based on its associated local feature, which is extracted from images. They can recover a continuous surface at arbitrary resolutions without topology restrictions.


However, in typical MLP-based implicit networks, the occupancy or TSDF value at each location is solved independently with planar image features, rendering them less capable of addressing challenging cases such as occlusions. Consequently, these methods suffer from generalization and robustness issues, particularly when tackling strong occlusions caused by large motion or multiple interacting humans. 
Some follow-up studies  \cite{zheng2021deepmulticap,zheng2021pamir,huang2020arch} utilize an extra geometric model, SMPL~\cite{Loper2015}, to improve robustness by introducing strong shape priors. 
Their success typically relies on the assumption of geometrical similarity \cite{huang2020arch} between the shape prior and target reconstruction, making them intractable for handling complex cases with loose clothes and sensitive to errors in SMPL model fitting.



%\ping{this paragraph sounds like `TSDF is better than MLP/SMPL, and we use TSDF to solve the problem'. But in Sec 3, we are telling a different story, saying `MLP needs a 3D convolutional encoder'. We need to make these two sections consistent.}\sicong{I think in this paragraph we claim that the TSDF}


%We opt for Trucated Signed Distance Funtion (TSDF) volumetric representations as they are naturally suitable for convolution operations, which have shown remarkable performance for learning hierarchical features on 2D visual perception tasks \cite{SunXLW19}. 
%Meanwhile, TSDF also describes the gradual geometry change around shape surface, which is not reflected by occupancy volume. 

We instead revisit the 3D volumetric representation and resort to 3D convolutional neural networks (CNNs) for feature learning, due to their impressive performance in feature learning and the ability to incorporate spatial context. However, volumetric methods and 3D convolution involve discretization, which might raise concerns regarding whether a discretized volume can preserve subtle geometric details as continuous representations learned in implicit functions. We investigate the relationship between volume resolution and quantization error on synthetic data by converting target mesh objects to TSDF volumes, as shown in Figure~\ref{fig:quantization_error}. We observe that the quantization errors are significantly reduced by increasing volume resolution and become nearly negligible when reaching a relatively high resolution (e.g., 512 or higher). In other words, achieving fine-detailed reconstruction is not supposed to be restricted by the use of volume representations as long as a proper volume resolution is utilized. Therefore, we present a method with high-resolution feature volumes, e.g., 256 and 512, while traditional volumetric methods \cite{varol18_bodynet,gilbert2018volumetric} are often limited to much lower resolutions, such as 32 or 128.



On the other hand, an increase in volume resolution may lead to a cubic growth of memory overhead \cite{8100085}. Reducing memory costs while guaranteeing the granularity of volumetric representations is necessary for pursuing high-quality reconstruction. Thus, we adopt a coarse-to-fine approach and cull away irrelevant voxels to build a sparse high-resolution feature volume. At the coarse level, the network computes an initial TSDF by applying a U-Net with sparse 3D CNN \cite{3DSemanticSegmentationWithSubmanifoldSparseConvNet} on the sparse feature volume, which is carved by a visual hull. Through our experiments, it turns out that more than 95\% of the volume grids are discarded by the visual hull culling, making the sparse 3D CNN efficient. At the fine level, the network focuses on a narrow band near the zero-level set of the initial TSDF and discretizes the narrow band with smaller voxels. By employing this narrow-band culling, we further shrink the sampling space, resulting in a relatively small range of grid numbers (usually 300K--500K in our experiments) even with a high volume resolution of 512. The remaining voxels in the narrow band are associated with features that fuse high-frequency information from the computed normal maps upon the low-frequency shape from the coarse level to compute the TSDF at high resolution. The final mesh is then extracted from the TSDF using the Marching-Cube algorithm ~\cite{Lorensen87marchingcubes}.
% Different from the u-net sturcture to preserve global topology context, we then apply a shallow 3dcnn to compute the final TSDF $D_{final}$ which contain more local geometry detail.




% \ping{this paragraph can be expanded. It is an important contribution and often ignored by other works. stress on the novel idea of regressing blending weights instead of colors}

In addition to geometry, high-quality mesh texture is also a crucial factor contributing to visual appearance. Directly computing a color field in 3D space, as in \cite{iccv2020PIFu}, struggles to capture high-frequency texture details, while the neural radiance field (NeRF) \cite{yu2020pixelnerf} or the DoubleField~\cite{shao2022doublefield} require expensive per-instance optimization and are often unstable for sparse input images. In contrast, we adopt an image-based rendering approach to compute a texture atlas map, which is efficient and widely supported in existing computer graphics tools. 
Specifically, we compute a blending weight at each 3D point on the mesh surface to determine its color as a weighted average of the colors at its image projections. The blending weights can be computed at a relatively coarse resolution, e.g., 512 volume resolution in our case, and leave texture details to the high-resolution images, such as 1K or 2K. Unlike previous methods that generate blurry texturing results under sparse input, our method generalizes well on both synthetic and real data with just a few input views. 
Figure~\ref{fig:teaser} shows two examples reconstructed by our method. Despite the challenging garment, pose, and occlusion, our method recovers faithful shape, normal, and texture on the right.

%with a wide variety of poses and clothing styles, and it is also adaptive to handle input image with arbitrary resolutions.
%\sicong{For this concern we claim that when the resolution of dicretized volume meets certain threshold (which is 256 in our experiment), the quantization error can be neglected.} 



In summary, the main contributions of this paper are as follows:
\begin{itemize}
\vspace{-0.1in}
  \item 
  We revisit the 3D volumetric representation and demonstrate that it can support clothed human reconstruction with equal or even better performance compared to implicit representation. 
  \item 
  We develop a memory and computation-efficient method for high-resolution volumetric reconstruction using sophisticated sparse 3D CNN, coarse-to-fine estimation, and voxel culling by visual hull and narrow bands. 
  \item 
  We introduce a novel method to compute a texture atlas map, which captures rich appearance details from high-resolution input images.
  \item 
  We achieve impressive results on standard benchmark datasets Twindom and MultiHuman, significantly reducing the point-2-surface (P2S) precision to approximately 0.2cm from just six input views, with more than $50\%$ error reduction compared to the state-of-the-art methods, including DoubleField~\cite{shao2022doublefield} and PIFuHD~\cite{saito2020pifuhd}.
\end{itemize}
% Figure environment removed
\subsection{HPS Model} 
In this article, we use two versions of the HPS model used previously~\cite{Mittal2018}-\cite{Larsen2021}.
The amino acid residues interact among themselves by a modified Van der Waals interaction potential, first introduced by Ashbaugh and Hatch~\cite{Ausbaugh} given by
\begin{align}
U_{VdW}\left(r_{ij}\right) = \begin{cases}
      U_{LJ}\left(r_{ij}\right) + (1-\lambda_{ij})\epsilon_{ij}, & \text{$r_{j} \leq 2^{\frac{1}{6}} \sigma_{ij}$}\\
      \lambda_{ij} U_{LJ}\left(r_{ij}\right), & \text{otherwise}
    \end{cases}       
\end{align}
where $U_{LJ}$ is the Lennard-Jones (LJ) potential,
\begin{align}
U_{LJ}\left(r_{ij}\right) = 4\epsilon_{ij} \left[\left(\frac{\sigma_{ij}}{r_{ij}}\right)^{12} - \left(\frac{\sigma_{ij}}{r_{ij}}\right)^6\right].
\end{align}
Here, $r_{ij}=\left | \vec{r}_i - \vec{r}_j \right|$ is the distance
between the amino acid beads with indices $i$ and $j$ positioned at $\vec{r}_i$
and $\vec{r}_j$,
$\epsilon_{ij}=\frac{1}{2}\left(\epsilon_i+\epsilon_j\right)$ and $\lambda_{ij}=\frac{1}{2}\left(\lambda_i+\lambda_j\right)$
are the strength of the van der Waal interaction and average hydropathy factor
between any two amino acids with indices $i$ and $j$. 
A harmonic bond potential
\begin{equation}
U_{b}\left(r_{ij}\right) = \frac{k_b}{2}\left(
  \frac{r_{ij}  - r^0_{ij}}{\sigma_{ij}} \right)^2.
\end{equation}
acts between two consecutive amino acid residues $i$ and
$j=i\pm1$. The spring constant $k_b$ = 8033 kJ/(mol$\cdot$nm$^2$) =
1920 kCal/(mol$\cdot$nm$^2$), and the equilibrium bond length is $r_0$
= 0.38 $nm$.\par
A screened-Coulomb (SC) interaction acts between any two charged amino
acids
\begin{equation}
  U_{SC}\left(r_{\alpha\beta}\right) =
  \frac{q_{\alpha}q_{\beta}e^2}{4\pi \epsilon_0 \epsilon_r}
  \left( \frac{e^{-\kappa r_{\alpha \beta}}}{r_{\alpha \beta}}\right) 
\end{equation}
where the indices $\alpha$ and $\beta$ refer to the subset of the indices $i$ and
$j$ for the charged amino acids, $\epsilon_r$ is the dielectric
constant of water, and $\kappa$ is the inverse Debye screening
length~\cite{Israel}.
The inverse Debye length $\kappa^{-1}$ is dependent on the ionic concentration (I) and expressed as
\begin{align}
\kappa^{-1} & = \sqrt{8 \pi l_B I N_A \times 10^{-24}} 
\end{align}
where $N_{A}$ is the Avogadro's number and $l_B$ is the Bjerrum length,
\begin{align}
l_B = \frac{e^2}{4\pi \epsilon_0 \epsilon_r k_BT}.
\end{align}
At higher temperatures, the dielectric constant typically decreases, which affects the strength of the electrostatic interactions. If the dielectric constant does not account for temperature effects, the electrostatic interactions may be overestimated, leading to unrealistic protein conformations or interactions. Hence, we implement the temperature-dependent dielectric constant of water as expressed by the empirical relation~\cite{Akerlof}
\begin{align}
\epsilon_r(T) = \frac{5321}{T} + 233.76 - 0.9297T  \nonumber \\
+ 1.147\times 10^{-3} T^2  - 8.292 \times 10^{-7} T^3. 
\end{align}
\subsection{M3 Model} 
Keeping the energy function the same as above, Tesei {\em et al.} used the Bayesian parameter-learning procedure to further optimize the hydropathy values and showed M3~\cite{Larsen2021} hydropathy scale performs better to produce radius of gyration values closer to the experiments. This M3 hydropathy scales of the amino acids are shown in $7^{th}$ column of Fig.~\ref{Model}(b). We used both the hydropathy scales HPS and M3 to study the properties of IDPs described in the subsequent sections.
\section{Results}
We studied 33 different IDPs with varying number of amino acids (N = 24 - 284) with
high net positive charge to high net negative charge (see the $6^{th}$ column of Table-~\ref{Table}). All these IDPs have been studied earlier by
different CG models.
Within the range of our studied IDPs, An16 is a
polyelectrolyte containing only six positively charged Histidine
residues, Nucleoporin153 which contains only uncharged residues, and the
rest of the 30 IDPs are polyampholytes. The table is sorted according
to their net charge, from highly positive in red to highly negative
shown in dark green and listed in the $6^{th}$ column. The first row of
the table lists K32 that contains 14~$\%$ Lysine (+1) and 4~$\%$
Aspartic acid (-1) which makes it highly positive. On the other hand,
the bottom row is ProTa-N that contains 20~$\%$ Glutamic acid (-1),
17~$\%$  Aspartic acid (-1) and 8~$\%$ Lysine (+1), which makes it
highly negative. We assign unique letter codes for each of the IDPs,
Greek letters $\alpha$-$\rho$ in the ascending order starting from
highly positively charged IDPs and in descending order with the alphabets
starting from the negatively charged IDPs. We present the experimental
conditions such as ionic concentration in mM and temperature at Kelvin
scale in the $4^{th}$ and $5^{th}$ columns respectively. The total number of amino acid residues is denoted by N in the $3^{rd}$ column and the $7^{th}$ column shows the total number of charge residues present in the IDPs. The experimental radius of gyration $R_g^{expt}$ values are listed in $11^{th}$ column. The
{\textcolor{red}{\huge$\circ$}},
{\textcolor{blue}{\large{$\square$}}}, and {\textcolor{green}{\large{$\Diamond$}}} in the 2nd column along with IDPs names denote the refs.~\cite{Thirumalai_2019},\cite{Mittal2018},\cite{Larsen2021} respectively from where these experimental values are obtained. The corresponding simulation $R_g$ values are listed in the $12^{th}$ column for the HPS scale, and the $14^{th}$ column for the M3 scale.
\par
\begin{table*}
\centering
% Figure removed
\caption{\small A comprehensive table shows the comparison between the experimental and the simulation values obtained using two different models. The 1st, 2nd, and 3rd columns denote the letter id, names, and number of amino acids in the IDPs. The symbols {\textcolor{red}{\Large{$\circ$}}}, {\textcolor{blue}{$\square$}},  and {\textcolor{green}{$\Diamond$}} in the 2nd column along with IDPs names are obtained from the list of IDPs considered in the previous studies by Baul~{\em et al.}~\cite{Thirumalai_2019}, Dignon~{\em et al.}~\cite{Mittal2018}, and Tesei~{\em et al.}~\cite{Larsen2021} respectively from where the experimental values and conditions are obtained. The 4th and 5th columns describe the experimental conditions: ionic concentration in mM and temperature in Kelvin scale respectively. The 6th and 7th columns represent the net charge and the number of charged residues. The 8th and 9th columns show the absolute charge per residue and net charge per residue as described in Eq.~\ref{q_definition}. The 10th column represents the fraction charge content~\cite{Pappu2010}. The experimental radius of gyrations of all the IDPs are presented on the 11th column. The 12th and 14th columns represent the simulation radius of gyration results obtained by the HPS hydropathy scale and the M3 scale for $\epsilon=0.18$, $\epsilon=0.3$ respectively where the MSE is obtained to be the least (see Fig.~\ref{Rg}). We show the percentage error in the 13th and 15th columns corresponding to HPS and M3 scale. The last column shows the area under the Wilson charge curve from Fig.~\ref{wilson_chg}.}
\label{Table}
\end{table*}
%%%%%%%%%%%%%%%%%%%%%%%%%%%%%%%%%%%%%%%%%%%
%%%%%%%%%%%%%%%%%%%%%%%%%%%%%%%%%%%%%%%%%%%
% Figure environment removed
%%%%%%%%%%%%%%%%%%%%%%%%%%%%%%%%%%%%%%%%%%%
Historically many hydropathy scales have been introduces to model the
properties of amino acids and provide a quantitative measure of the
hydrophilicity or hydrophobicity of amino acids based on
their propensity to reside in a water-soluble or water-insoluble
environment~\cite{Kyte,Engelman,Hopp, Eisenberg,Cornette}. Each of these scales assigns a numerical value to each amino acid, reflecting its hydrophobic or hydrophilic nature. The scores obtained from hydropathy scales are useful in predicting protein structure and function. Recently, specific hydropathy scales are employed to study the liquid–liquid phase separation (LLPS) behavior of IDPs. Dignon~{\em et al.} proposed HPS~\cite{Mittal2018} hydropathy scale where Proline and Phenylalanine are considered to be the most hydrophobic with $\lambda^{HPS}=1$ and Arginine is the least hydrophobic with $\lambda^{HPS}=0$. All the amino acids' hydropathy are scaled to fit in the range. Later Tesei {\em et al.} used the Bayesian parameter-learning procedure to further optimize the hydropathy values and showed M3~\cite{Larsen2021} hydropathy scale performs better to produce radius of gyration values closer to the experiments.
\par
We use two hydropathy scales HPS~\cite{Mittal2018} and
M3~\cite{Larsen2021} to study 33 IDPs using the coarse grained simulation method described in Sec.~II. In the CG model, only free
parameter to vary is the interaction strength $\epsilon$. We obtained
the simulation radius of gyration values $R_{g}^{HPS}$ and $R_{g}^{M3}$
corresponding to HPS and M3 hydropathy scales for different values of
$\epsilon$. In Fig.~\ref{Rg}, we show the scatter plots of the
experimental vs. simulation radius of gyration data for five different
values of $\epsilon$ ranging from 0.1-0.3 kcal/mol. The deviations of
the simulation $R_g$ values from the experimental values is characterized by mean square error (MSE) defined as 
\begin{align}
MSE = \frac{1}{N} \sum_{i=1}^{N}\left[ R_g^{expt} (i) - R_{g}^{k} (i) \right]^2
\end{align}
where, $N$ is the number of IDPs, and $k$ represents either HPS or M3 depending on the hydropathy scale. From the
scatter plots we find $\epsilon$ = 0.18 has the lowest error
(MSE=0.48) and the best fit for the HPS model. On the other hand for
M3 hydropathy scale, the lowest error (MSE=0.65) is obtained for
$\epsilon$ = 0.3. This results are consistent with the optimum
$\epsilon$ values obtained for HPS scale by Dignon~{\em et
  al.}~\cite{Mittal2018} and for M3 scale by Tesei~{\em et
  al.}~\cite{Larsen2021}. As the HPS hydropathy scale with
$\epsilon$=0.18 yield the best match with the experimental $R_g$ values, we use these parameters to further analyze the properties of
the IDPs and discussed in the subsequent sections.
%%%%%%%%%%%%%%%%%%%%%%%%%%%%%%%%%%%%%%%%%%%%%%%%%%%%%%%%
\subsection{Universal Scaling Properties of the IDPs}
Despite the fact the IDPs are mostly described as polyampholytes (PAs) or polyelectrolytes (PEs)~\cite{Everaers_PRL}, a  fraction of experimental and theoretical studies using the HPS model describe IDPs as Gaussian chains~\cite{Mittal2022}, while  in a recent publication, Thirumalai and coworkers using a two-bead CG model calculated the RMS
$R_g\equiv \sqrt{\langle R_g^2\rangle}$ and
the end-to-end distance $R_N\equiv \sqrt{ \langle R_N^2\rangle }$ and concluded that globally the IDPs 
described not as the Gaussian chains, but rather as fully flexible swollen chains that obey the Flory scaling  $R_g= aN^{0.59}$~\cite{Thirumalai_2023}. We investigate this point further to find out to what extent the properties of the IDPs are universal. 
From theoretical arguments following Schaefer~{\em
  et al.}~\cite{Pincus_MM_1980} and Nakanishi~\cite{Nakanishi_1987} it
is established 
that a proper description of a semi-flexible swollen chain characterized by a contour length $L$ and a persistence length $\ell_p$ in $d$ spatial dimensions is given by 
\begin{equation}
\sqrt {\langle R_N^2 \rangle} \simeq b_l^{\frac{d-2}{d+2}} N^{\frac{3}{d+2}}\ell_p^{\frac{1}{d+2}} = b_l^{\frac{d+1}{d+2}}\left( \frac{L}{b_l}\right)^{\nu}\ell_p^{\frac{1}{d+2}}. 
\label{Rn_EV}
\end{equation}
Here $N$ is the number of monomers of the chain so that
$L = (N-1)b_l \simeq Nb_l$ (for $N \gg 1 $), $b_l$ is the bond length between two
neighboring monomers, and the mean-field Flory exponent $\nu =
3/(d+2)$ in 2D = 0.75 and in 3D = 0.60 ($\approx 0.588$ actual) respectively. 
This EV chain accurate describes the limit $L/\ell_p >> 1$ and supersedes the Worm-like-chain model~\cite{Rubinstein} 
\begin{equation}
\frac{\langle R_N^2\rangle}{L^2} = \frac{2\ell_p}{L}\left(1-\frac{\ell_p}{L}[1-\exp(-L/\ell_p)]\right).
\label{WLC}
\end{equation}
which does not take into account the EV effect and hence saturates to $\langle R_N^2 \rangle = 2 L\ell_p)$ even when $L/\ell_p >> 1$. In a previous publication 
we have shown that scaled end-to-end distance $\langle R_N^2 \rangle/(2 L\ell_p)$ and
 the scaled transverse fluctuation $\sqrt{\langle{l_{\perp}^2}\rangle}/{L}$ as a
 function of $L/\ell_p$ collapse onto the same master curve~\cite{Universal1,Universal2} for all ratios of $L/\ell_p$ spanning rod to Gaussian and the EV limit. We would like to  discuss our findings for the IDPs in the context of these universal scaling plots (Fig.~\ref{Scaling}). \par
 Fig.~\ref{Scaling}(a) summarizes our results for the scaling exponent that shows that $\langle R_g\rangle \approx 0.4L^{0.5}$. For comparison we have also included a dashed line describing $\langle R_g\rangle \approx L^{0.588}$. This result shows that in the HPS models IDPs are described as Gaussian chains. This is also consistent with the slope of the straight line fit (red dashed line in Fig.~\ref{Scaling}(a)) that corresponds to the average bond length $~0.43$. \par
To get a clearer perspective we have calculated the length of the IDPs and compared the scaled $R_g$ for the IDPs with reference to the universal master plot for the homopolymers of different length $L$ and persistence lengths $l_p$ (blue circles) in Fig.~\ref{Scaling}(b). These data points will serve as a guide and help readers to visualize the deviation of the scaling properties of the IDPs from those of the semiflexible swollen homopolymer statistics.  
 All the IDPs fall below the EV region and overall lie along the WLC model (dashed magenta). Our conclusion is further strengthened by Fig.~\ref{Scaling}(c), where we find that the scaled transverse fluctuations are Gaussian.\par
It is worth noting though that on the basis of this plot finer classification of the IDs can be made. We note that a couple of IDPs such as CspTm, ERMTADn and ProTa-C fall {\em below} the WLC line indicating {\em compact globular structures}, while ProTa-N stays above the WLC line indicating an elongated conformation. These findings align with a previous study by Baul {\em et al.}~\cite{Thirumalai_2019} where they found most of the IDPs follow Gaussian statistics. 
This findings confirm the fact that despite varying degree of mass, Gaussian chains in th HPS models.
%%%%%%%%%%%%%%%%%%%%%%%%%%%%%%%%%%%%%%%%%%%
% Figure environment removed
%%%%%%%%%%%%%%%%%%%%%%%%%%%%%%%%%%%%%%%%%%%%
%%!TEX root = ../Schur indices and line operators.tex

\section{\texorpdfstring{Line operator index of $A_1$-theories of class-$\mathcal{S}$}{}\label{section:Wilson-index-A1-theories}}

In this and the following section we discuss the Schur index in the presence of a line operator. For a Lagrangian 4d $\mathcal{N} = 2$ SCFT, the Schur index in the absence of operator insersion can be computed by a multivariate contour integral \cite{Gadde:2011uv,Beem:2013sza}
\begin{align}
  \mathcal{I} = \oint \left[\frac{da}{2\pi i a}\right] \mathcal{Z}(a) \ ,
\end{align}
where the integrand $\mathcal{Z}(a)$ is elliptic with respect to the ``exponent variables'' $\mathfrak{a}_i$ separately, and captures contributions from the vector and hypermultiplets in a gauge theory description.

One can introduce half line operators in the 4d theory that extend from the origin to infinity while preserving certain amount of supercharges \cite{Cordova:2016uwk}. In particular, there are line operators that preserve the supercharges used to construct the Schur index. In the presence of such a BPS half Wilson line operator in the representation $\mathcal{R}$ of the gauge group, the half Wilson line index can be computed simply by\footnote{For simplicity we omit the normalization factor $\mathcal{I}^{-1}$.} \cite{Gang:2012yr,Cordova:2016uwk}
\begin{align}
	\langle W_{\mathcal{R}}\rangle = \oint \left[\frac{da}{2\pi i a}\right]
	\chi_\mathcal{R}(a) \mathcal{Z}(a) \ ,
\end{align}
where $\chi_\mathcal{R}(a)$ denotes the character of representation $\mathcal{R}$ of $G$. The Wilson index counts the local Schur operators (in the free limit) that are gauge-variant and can absorb the charge at the end of the half line. A full Wilson line operator in representation $\mathcal{R}$ can be thought of as a junction at the origin of two half Wilson line operators in complex-conjugating representation $\mathcal{R}, \overline{\mathcal{R}}$, and hence the full Wilson line index can be computed by
\begin{align}
  \langle W_{\mathcal{R}}^\text{full}\rangle = \oint \left[\frac{da}{2\pi i a}\right]
  \chi_\mathcal{R}(a)\chi_{\overline {\mathcal{R}}}(a) \mathcal{Z}(a) \ .
\end{align}
In our notation, we will only add the superscript ``full'' when dealing with a full Wilson line operator.

One can also consider correlators of half Wilson line operators, which take the form
\begin{align}
  \langle W_{\mathcal{R}_1} \cdots W_{\mathcal{R}_n}\rangle
  = \oint \bigg[\frac{da}{2\pi i a}\bigg]
  \bigg[\prod_{i=1}^n\chi_{\mathcal{R}_i}(a)\bigg]
  \mathcal{Z}(a)\ .
\end{align}
One can consider applying the tensor product decomposition $\otimes_{i = 1}^n \mathcal{R}_i = \sum_{j} m_j \mathcal{R}^{(j)}$ and reduce the product of characters on the right to a sum of characters of the irreducible representations $\mathcal{R}^{(j)}$ of the gauge group,
\begin{align}
  \langle W_{\mathcal{R}_1} \cdots W_{\mathcal{R}_n}\rangle = \sum_{j} m_j \langle W_{\mathcal{R}^{(j)}} \rangle \ .
\end{align}
In this sense, half Wilson line index in irreducible representations are the basic building blocks for correlators of half/full Wilson line, which will be our main focus.

In the following we will study line operator index for $A_1$ theories of class-$\mathcal{S}$. We will start with some simple examples where we are able to compute both the Wilson line index and the $S$-dual `t Hooft line index. Eventually we will analyze in detail half Wilson line index for general $A_1$ theories of class-$\mathcal{S}$. When possible, we also comment on the relation between the index and associated chiral algebra characters.


\subsection{\texorpdfstring{$\mathcal{N} = 4 $  $ SU(2)$ theory}{}\label{section:N4SU(2)}}

\subsubsection{Half Wilson line index}

The associated chiral algebra $\mathbb{V}_{\mathcal{N} = 4}$ of the $\mathcal{N} = 4$ theory with an $SU(2)$ gauge group is given by the 2d small $\mathcal{N} = 4$ superconformal algebra. The Schur index, which is identified with the vacuum character of $\mathbb{V}_{\mathcal{N} = 4}$, can be computed by the contour integral
\begin{align}
  \mathcal{I}_{\mathcal{N} = 4} 
  = & \ - \frac{1}{2}\frac{\eta(\tau)^3}{\vartheta_4(\mathfrak{b})}
  \oint_{|a| = 1} \frac{da}{2\pi i a} 
  \frac{
    \vartheta_1(2\mathfrak{a})\vartheta_1(- 2\mathfrak{a})
  }{
    \vartheta_4(2\mathfrak{a} + \mathfrak{b})
    \vartheta_4(-2\mathfrak{a} + \mathfrak{b})
  }
  \coloneqq \oint \frac{da}{2\pi i a} \mathcal{Z}(a)\\
  = & \ \frac{i\vartheta_4(\mathfrak{b})}{\vartheta_1(2 \mathfrak{b})} E_1 \begin{bmatrix}
    -1 \\ b  
  \end{bmatrix} \ . \nonumber
\end{align}
In the following we consider the index in the presence of a half Wilson line operator in the spin-$j$ representation. The index is then given by the integral
\begin{align}
	\langle W_j\rangle =
  \oint_{|a| = 1} \frac{da}{2\pi i a} \left[\sum_{m = - j}^{j} a^{2m}\right]
  \mathcal{Z}(a)\ .
\end{align}
Here the spin-$j$ character is given by $\chi_j(a) = \sum_{m = -j}^j a^{2m}$.

To proceed, we note that there are a collection of poles from the elliptic integrand,
\begin{align}
	\mathfrak{a}_{k\ell}^\pm = \pm \frac{\mathfrak{b}}{2} + \frac{(2k + 1)\tau}{4} + \frac{\ell}{2}, \qquad
  k, \ell = 0, 1 \ .
\end{align}
Due to the presence of $\tau/4$, all these poles are imaginary, with essentially the same residues
\begin{align}
  R^\pm_{k\ell} = \mp \frac{i}{4} \frac{\vartheta_4(\mathfrak{b})}{\vartheta_1(2 \mathfrak{b})} \ .
\end{align}
Applying the integral formula (\ref{integration-formula-monomial}), the index reads
\begin{align}
  \langle W_j\rangle = \mathcal{I}_{\mathcal{N} = 4}\delta_{j \in \mathbb{Z}} - \frac{i}{4} \frac{\vartheta_4(\mathfrak{b})}{\vartheta_1(2 \mathfrak{b})}\sum_{\substack{m = -j \\ m \ne 0}}^{j}\sum_{k, \ell = 0, 1} \frac{(-1)^{2\ell m} (b^m - b^{-m})q^{( - \frac{1}{2} + k )m}}{q^{m} - q^{-m}} \ .
\end{align}
Note that for $j \in \mathbb{Z}$, the character $\chi_j(a)$ contains a constant term $1$, which lead to the original Schur index $\mathcal{I}_{\mathcal{N} = 4}$. In fact, when $j \in \mathbb{Z} + \frac{1}{2}$, the entire expression vanishes identically thanks to the summation over $\ell = 0, 1$. Therefore, we have
\begin{align}
  \langle W_{j \in \mathbb{Z}}\rangle
  = + \mathcal{I}_{\mathcal{N} = 4} - \frac{i}{2} \frac{\vartheta_4(\mathfrak{b})}{\vartheta_1(2 \mathfrak{b})}\sum_{\substack{m = -j \\ m \ne 0}}^{j} \frac{b^m - b^{-m}}{q^{m/2} - q^{-m/2}} \ ,
  \qquad
  \langle W_{j \in \mathbb{Z} + \frac{1}{2}}\rangle = 0 \ .
\end{align}
The first term $\mathcal{I}_{\mathcal{N} = 4} = \operatorname{ch}_0$ is identified with the vacuum character of the associated chiral algebra $\mathbb{V}_{\mathcal{N} = 4}$. The factor $\frac{i\vartheta_4(\mathfrak{b})}{\vartheta_1(2 \mathfrak{b})}$ in the second term is the residue of the integrand $\mathcal{Z}$ which is related to the Schur index of Gukov-Witten type surface defect in the $\mathcal{N} = 4$ theory \cite{Pan:2021ulr}. It can be shown to satisfy $\frac{i\vartheta_4(\mathfrak{b})}{\vartheta_1(2 \mathfrak{b})} = \operatorname{ch}_0 + \operatorname{ch}_M$ where $M$ is another irreducible module $M$ of $\mathbb{V}_{\mathcal{N} = 4}$ \cite{Adamovic:2014lra,Bonetti:2018fqz,Pan:2021ulr}. As module characters of $\mathbb{V}_{\mathcal{N} = 4}$, both $\operatorname{ch}_0$ and $\operatorname{ch}_M$ satisfy the flaovred modular differential equations arising from null states in $\mathbb{V}_{\mathcal{N} = 4}$ \cite{Gaberdiel:2008pr,Gaberdiel:2009vs,Beem:2017ooy,Pan:2021ulr,Zheng:2022zkm}. Therefore, the line index can be written as a combination of the two irreducible characters,
\begin{align}
  \langle W_{j \in \mathbb{Z}}\rangle
  = \bigg(1 - \frac{1}{2}\sum_{\substack{m = -j \\ m \ne 0}}^{+j}\frac{b^m - b^{-m}}{q^{m/2} - q^{- m /2}}\bigg)\operatorname{ch}_0
  - \frac{1}{2}\Big(\sum_{\substack{m = -j \\ m \ne 0}}^{+j}\frac{b^m - b^{-m}}{q^{m/2} - q^{- m /2}}\Big) \operatorname{ch}_M \ .
\end{align}
Note however that the coefficients of the linear combination are rational functions of $b$ and $q$.





\subsubsection{`t Hooft line index}

In the 4d $\mathcal{N} = 4$ SYM (and in general $\mathcal{N} = 2$ superconformal gauge theories), one can define `t Hooft line operators by specifying certain singular profile for the gauge field and scalars in the path integral. By the Dirac quantization condition, the magnetic charge $B$ of a `t Hooft operator is valued in the cocharacter lattice $\Lambda_\text{cochar}$ inside the Cartan $\mathfrak{h}$ of the gauge group $G$. This lattice $\Lambda_\text{cochar}$ corresponds to the weights of the Langland dual group $G^\vee$, and therefore a dominant integral element $B$ corresponds to a $G^\vee$-representation $\mathcal{R}^\vee_B$. The cocharacters as weights in $\mathcal{R}^\vee_B$ are obtained from $B$ by subtracting suitable coroot element $\alpha^\vee$, and weights related by the Weyl group $W$ of the gauge group $G$ are identified. A weight $v$ in $\mathcal{R}^\vee_B$ that is not Weyl-related to $B$ can screen the `t Hooft operator and signals monopole bubbling effect \cite{Lee:1996vz,Gomis:2009ir,Ito:2011ea,Brennan:2018yuj}.

Under S-duality, a full Wilson line in a $\mathcal{N} = 4$ SYM is mapped to a `t Hooft line. If the magnetic charge of a 't Hooft operator corresponds to a minuscule representation of $G^\vee$, then its index is safe from monopole bubbling effect, and the index can be computed by a relatively simple contour integral \cite{Gang:2012yr}. In particular, For the $\mathcal{N} = 4$ $U(2)$ theory, the 't Hooft line with minimal magnetic charge $(1,0)$ corresponds to a minuscule representation, and is dual to the a full Wilson operator in the fundamental representation. The `t Hooft index can be written as a contour integral \cite{Gang:2012yr},
\begin{align}\label{U2-t-hooft}
  \langle H_{(1,0)}^\text{full} \rangle
  = - \oint \frac{da}{2\pi i a} \frac{(a - b)(-1 + a b)}{(\sqrt{q} - a)(-1 + \sqrt{q}a)b}
  \frac{\eta(\tau)^6 \vartheta_4(\mathfrak{a})^2}{
    \vartheta_1(\mathfrak{a} - \mathfrak{b})
    \vartheta_1(\mathfrak{a} + \mathfrak{b})
    \vartheta_4(\mathfrak{b})^2
  } \ .
\end{align}
Note that the parameters and integration variables have been renamed and reorganized compared to the double contour integral in \cite{Gang:2012yr}. In series expansion,
\begin{align}
  \langle H_{(1,0)}^\text{full}\rangle = 1 + 2(b + b^{-1})\sqrt{q}
  + (1 + 3b^2 + 3b^{-2}) q
  + 4(b^3 + b^{-3})q^{3/2} + \cdots \ .
\end{align}
The ratio of $\vartheta$ functions in $\langle H^\text{full}\rangle$ are essentially identical to the original integrand that computes $\mathcal{I}_{\mathcal{N} = 4}$, up to a shift from $\vartheta_{1, 4} \to \vartheta_{4,1}$. It is therefore elliptic in $\mathfrak{a}$, with real poles $\mathfrak{a} = \pm \mathfrak{b}$. The rational factor in the integrand can also be expanded in the $SU(2)$ characters,
\begin{align}
  - \frac{(a-b)(-1 + ab)}{(\sqrt{q} - a)(-1 + \sqrt{q}a)}
  = (1 + b^2) \sum_{n = 0}^{+\infty}q^{\frac{n}{2}}\chi_{j = \frac{n}{2}}(a)
  - b \sum_{n = 0}^{+\infty}q^{n/2}
  \chi_{j = \frac{1}{2}}(a)\chi_{j = \frac{n}{2}}(a) \\
  = (1 + b^2) \sum_{n = 0}^{+\infty}q^{n/2}\chi_{j = \frac{n}{2}}(a)
  - b \sum_{n = 0}^{+\infty}q^{n/2}
  \chi_{j = \frac{n}{2} + \frac{1}{2}}(a)
  - b \sum_{n = 0}^{+\infty}q^{n/2}
  \chi_{j = \frac{n}{2} - \frac{1}{2}}(a) \ . 
\end{align}
Therefore, the integral $\langle H^\text{full}\rangle$ can be computed directly and exactly using (\ref{integration-formula-χf}). In this case, the residues of two real poles $a = b^{\pm}$ are given by
\begin{align}
  R_\pm = \pm \frac{i \eta(\tau)^3}{\vartheta_1(2\mathfrak{b})} \ .
\end{align}
After some algebra, we have
\begin{align}
  \langle H_{(1,0)}^\text{full}\rangle
  = \frac{i \eta(\tau)^3}{\vartheta_1(2\mathfrak{b})}
  (q^{\frac{1}{2}} & \ + q^{-\frac{1}{2}} - b - b^{-1})
  \sum_{n = 0}^{+\infty}
  \sum_{\substack{m = - n/2 \\ m \ne 0}}^{+ n/2}
  q^{\frac{n}{2}} \frac{b^{2m} - b^{-2m}}{1 - q^{-2m}}
  \nonumber\\
  & \ + \frac{2(b + b^{-1} - 2q^{\frac{1}{2}})}{1-q} \frac{i \eta(\tau)^3}{\vartheta_1(2\mathfrak{b})} E_1 \begin{bmatrix}
    -1 \\ b  
  \end{bmatrix} \ ,
\end{align}
where in the second line we applied
\begin{align}
  \oint \frac{da}{2\pi i a}\frac{\eta(\tau)^6 \vartheta_4(\mathfrak{a})^2}{
    \vartheta_1(\mathfrak{a} - \mathfrak{b})
    \vartheta_1(\mathfrak{a} + \mathfrak{b})
    \vartheta_4(\mathfrak{b})^2
  } = \frac{2i \eta(\tau)^3}{\vartheta_1(2 \mathfrak{b})} E_1 \begin{bmatrix}
    -1 \\ b  
  \end{bmatrix} \ .
\end{align}

The dual Wilson operator index can be computed a lot more easily with (\ref{integration-formula-monomial}),
\begin{align}
  \langle W^\text{full}_{j = 1/2}\rangle
  = & \ - \frac{1}{2}\frac{\eta(\tau)^6}{\vartheta_4(\mathfrak{b})^2}
  \oint_{|a| = 1} \frac{da}{2\pi i a}
  (a + \frac{1}{a})^2 \frac{
    \vartheta_1(2\mathfrak{a})\vartheta_1(- 2\mathfrak{a})
  }{
    \vartheta_4(2\mathfrak{a} + \mathfrak{b})
    \vartheta_4(-2\mathfrak{a} + \mathfrak{b})
  } \\
  = & \ \langle W_{j = 1}\rangle_{U(2)} + \mathcal{I}_{\mathcal{N} = 4 \ U(2)}
  = q^{-\frac{1}{2}}
  \frac{i\eta(\tau)^3}{\vartheta_4(\mathfrak{b})}
  \frac{\vartheta_4(\mathfrak{b})}{\vartheta_1(2\mathfrak{b})}\left(
  2E_1\begin{bmatrix}
    -1 \\b  
  \end{bmatrix} -  \frac{b -b^{-1}}{q^{1/2} - q^{-1/2}}
  \right) \ . \nonumber
\end{align}
As required by S-duality, $\langle W^\text{full}_{j = 1/2}\rangle = \langle H_{(1,0)}^\text{full}\rangle$. This equality indeed follows analytically from the identity (\ref{E1-expansions}). Stripping off the $U(1)$ vector multiplet and the free hypermultiplet contribution $\eta(\tau)^3/\vartheta_4(\mathfrak{b})$, both the full Wilson index and the `t Hooft index are linear combinations of two $\mathcal{V}_{\mathcal{N} = 4}$ characters with rational coefficients, so schematically
\begin{align}
  \langle W_{j = 1}^\text{full} \rangle = A \operatorname{ch}_0 + B \operatorname{ch}_M, \qquad
  \langle H_{(1,0)}^\text{full}\rangle = C \operatorname{ch}_0 + D \operatorname{ch}_M \ .
\end{align}
However, the S-duality $\langle W^\text{full}_{j = 1/2}\rangle = \langle H_{(1,0)}^\text{full}\rangle$ is not because $A = C, B = D$; instead, the $S$-duality induces some highly nontrivial mixing between the vacuum and the $M$ module contributions.



Let us also consider `t-Hooft operators with non-minimal charge $B = (2,0)$. In this case, the index receives contribution from monopole bubbling with $v = (1,1)$, and is expected to equal the $U(2)$ Wilson index in the tensor product of fundamental representation. The `t Hooft index reads
\begin{align}
  \langle H^\text{full}_{(2,0)}\rangle
  = q^{-1/2} \oint \frac{da}{2\pi i a} \mathcal{Z}(a) \frac{\eta(\tau)^6}{\vartheta_4(\mathfrak{b})^2} \frac{\vartheta_1(\mathfrak{a})^2}{\vartheta_4(\pm \mathfrak{a} + \mathfrak{b})} \ ,
\end{align}
where
\begin{align}
  \mathcal{Z}(a)
  = \frac{(1 - \frac{\sqrt{q}}{ab}) (1 - \frac{a\sqrt{q}}{b})}{(1 - \frac{1}{a})(1 - a)}& \ \frac{(1 - \frac{b \sqrt{q}}{a})(1 - a b \sqrt{q})}{(1 - \frac{q}{a})(1 - aq)} \nonumber \\
  & \ + \frac{1}{2} \left[\frac{(q - 1)^2 + (b + \frac{1}{b})\sqrt{q}(1 + q) - 2q (a + \frac{1}{a}) }{(1 - \frac{q}{a})(1 - a q)}\right]^2 \ .
\end{align}
Note that
\begin{align}
  \frac{1}{(1 - \frac{q}{a})(1 - aq)} = \sum_{j \in \frac{1}{2}\mathbb{N}}q^{2j} \chi_j(a) \ ,\quad
  (1 - \frac{b^\pm \sqrt{q}}{a})(1 - a b^\pm \sqrt{q}) = (1 + b^{\pm2} q) - b^\pm q^{\frac{1}{2}} \chi_{\frac{1}{2}}(a) \ . \nonumber
\end{align}
Inserting these expansion, we have
\begin{align}
  \mathcal{Z}
  = & \ \frac{1}{(1-z)(1-1/z)} \left[A - B\chi_{1/2}(a) + q \chi_1(a)
  \right]\sum_{j\in \frac{1}{2}\mathbb{N}} q^{2j}\chi_{j}(a) \nonumber \\
  & \ + \Big[
  4q^2(1 + \chi_1 (a)) - C^2 - 2C q \chi_{\frac{1}{2}}(a)
  \Big]\sum_{j, j', j'' \in \frac{1}{2}\mathbb{N}} q^{2(j + j')}N_{j j'}^{j''} \chi_{j''}(a) \\
  \coloneqq & \ \frac{1}{(1-a)(1-1/a)} \sum_{j \in \frac{1}{2}\mathbb{N}} \mathcal{Z}_j \chi_j(a) + \sum_{j \in \frac{1}{2}\mathbb{N}}\mathcal{Z}_j'\chi_j(a),
\end{align}
where
\begin{align}
  A & \ \coloneqq (1 + b^2q)(1 + \frac{q}{b^2}) + q, 
  & B \coloneqq & \ (b + b^{-1})\sqrt{q}(1+q)\\
  C & \ \coloneqq (q-1)^2 + (b + \frac{1}{b}) \sqrt{q}(1 + q), 
  & \chi_{J}(a)\chi_{J'}(a) = & \ \sum_{J''}N_{JJ'}^{J''} \chi_{J''}(a) \ ,
\end{align}
and $\mathcal{Z}_j$, $\mathcal{Z}'_j$ are polynomials of $b, q$ from applying the tensor product rule for the $SU(2)$ characters,
\begin{align}
   \sum_{j \in \frac{1}{2}\mathbb{N}}\mathcal{Z}_j \chi_j(a)  = & \ [A - B \chi_{1/2}(a) + q \chi_1(a)] \sum_{j \in \frac{1}{2}\mathbb{N}}q^{2j} \chi_j(a)\\
   \sum_{j \in \frac{1}{2}\mathbb{N}}\mathcal{Z}'_j \chi_j(a) = & \ \Big[
     4q^2(1 + \chi_1 (a)) - C^2 - 2C q \chi_{\frac{1}{2}}(a)
     \Big]\sum_{j, j', j'' \in \frac{1}{2}\mathbb{N}} q^{2(j + j')}N_{j j'}^{j''} \chi_{j''}(a) \ ,
\end{align}
while their explicit expressions will be left implicit. Plugging this expansion into the integral, we have
\begin{align}
  \langle H^\text{full}_{(2,0)} \rangle
  = & \ \frac{i \eta(\tau)^3}{\vartheta_1(2 \mathfrak{b})}\sum_{j \in \frac{1}{2}\mathbb{N}} \mathcal{Z}_j \left(
  - \lfloor (j + \frac{1}{2})^2 \rfloor 2E_1 \begin{bmatrix}
    -1 \\ b  
  \end{bmatrix}
  + \sum_{m = -j}^{+j}\sum^{+\infty}_{\substack{k = 0 \\ k+2m \ne 0}}
  \frac{k(b^{k + 2m} - b^{-k -2m})}{q^{\frac{k}{2} + m} - q^{- \frac{k}{2} - m}}
  \right) \nonumber \\
  & \ + \frac{i \eta(\tau)^3}{\vartheta_1(2\mathfrak{b})}\sum_{j \in \frac{1}{2} \mathbb{N}} \mathcal{Z}'_j \left(
    \delta_{j \in \mathbb{Z}} 2E_1 \begin{bmatrix}
      -1 \\ b  
    \end{bmatrix}
    - \sum_{\substack{m = -j \\ m \ne 0}}^{+j} \frac{b^{2m} - b^{-2m}}{q^m - q^{-m}}
  \right) \ .
\end{align}
Unfortunately, we are unable to recast the expression to a more elegant form, therefore we do not prove $\langle W_{\mathbf{2} \otimes \mathbf{2}}^\text{full}\rangle_{U(2)} = \langle H_{(2,0)}^\text{full}\rangle$ analytically. Still, once the free contribution $\eta(\tau)^3/\vartheta_4(\mathfrak{b})$ is removed, the index $\langle H^\text{full}_{(2,0)}\rangle$ remains a linear combination of $\mathbb{V}_{\mathcal{N} = 4}$ characters.



\subsection{\texorpdfstring{$SU(2)$ theory with four flavors}{}}

Next we consider the $\mathcal{N} = 2$ $SU(2)$ gauge theory with four fundamental flavors. In terms of the class-$\mathcal{S}$ description, the theory is associated to the four puncture sphere $\Sigma_{0,4}$ and it admits three weak coupling limits corresponding to three different pants-decompositions. For any such limit, we can insert a half or full Wilson line operator of the $SU(2)$ gauge group in the spin-$j$ representation. The half Wilson index can be computed by the following integral,
% Figure environment removed
\begin{align}
  \langle W_j\rangle_{0,4} = - \frac{1}{2} \oint \frac{da}{2\pi i z} \left[\sum_{m = -j}^{j} a^{2m}\right] \frac{da}{2\pi i a}
  \vartheta_1(2\mathfrak{a}) \vartheta_1(-2\mathfrak{a})
  \prod_{j = 1}^{4} \frac{\eta(\tau)^2}{\vartheta_1(\mathfrak{a} + \mathfrak{m}_j)
  \vartheta_1(- \mathfrak{a} + \mathfrak{m}_j)} \ .
\end{align}
The poles of the integrand are all imaginary, given by $\mathfrak{a}_i^\pm = \pm \mathfrak{m}_i + \frac{\tau}{2}$ with residues
\begin{align}
  R_{i, \pm} = \pm \frac{i}{2} \frac{\vartheta_1(2 \mathfrak{m}_i)}{\eta(\tau)}
  \prod_{\ell \ne i} \frac{\eta(\tau)}{\vartheta_1(\mathfrak{m}_i + \mathfrak{m}_\ell) \vartheta_1(\mathfrak{m}_i - \mathfrak{m}_\ell)}
  \coloneqq \pm R_i
\end{align}
Applying the integration formula (\ref{integration-formula-monomial}), we have
\begin{align}\label{Wilson-index-SQCD}
  \langle W_j \rangle_{0,4} =  & \ \mathcal{I}_{0,4}\delta_{j \in \mathbb{Z}} - \sum_{\substack{m = - j \\ m \ne 0}}^{+ j} \sum_{\pm} \sum_{i = 1}^4 R_{i, \pm} \frac{1}{q^{2m} - 1} (b_i^\pm q^{\frac{1}{2}})^{2m} \nonumber\\
  = & \ \mathcal{I}_{0,4}\delta_{j \in \mathbb{Z}}
  - \sum_{i = 1}^{4} \left(\sum_{\substack{m = - j \\ m \ne 0}}^{+j} \frac{M_i^{2m} - M_i^{-2m}}{q^{m} - q^{-m}}\right)R_i \ , 
\end{align}
where $M_i \coloneqq e^{2\pi i \mathfrak{m}_i}$. The theory is of class-$\mathcal{S}$ associated to the four-punctured sphere. The $SU(2)^4$ fugacities $b_i$ are related to the $m_i$ by
\begin{align}
  M_1 = b_1 b_2, \quad
  M_2 = b_1/b_2, \quad
  M_3 = b_3 b_4, \quad
  M_4 = b_3/b_4 \ .
\end{align}

In \cite{Cordova:2016uwk}, several Wilson line index in $SU(2)$ SQCD were computed, and the results can be organized as linear combinations of the infinitely many highest weight characters $\chi_{[m, n, 0,0,0]}$ of $\widehat{\mathfrak{so}}(8)_{-2}$ which were obtained from the Kazhdan-Lusztig formula \cite{Lusztig1979}. Our new computation improves the result and relates all $\langle W_j\rangle_{0,4}$ to just five highest weight characters, with respect to finite weights $\lambda = 0, -2 \omega_1, - \omega_2, - 2 \omega_3, -2 \omega_4$, of the simple vertex operator algebra $\widehat{\mathfrak{so}}(8)_{-2}$ \cite{Arakawa:2015jya,Arakawa:2016hkg}. Indeed, the four residues $R_i$ in the above are related to the Schur index of Gukov-Witten type surface defects, and also to the the module characters \cite{Peelaers,Pan:2021mrw,2023arXiv230409681L,Pan:2023jjw,Arai:2020qaj},
\begin{align}
  \operatorname{ch}_{-2\widehat \omega_1} = & \ \operatorname{ch}_0 - 2R_1\\
  \operatorname{ch}_{-\widehat \omega_2} = & \ -2 \operatorname{ch}_0 + 2R_1 + 2R_2\\
  \operatorname{ch}_{-2\widehat \omega_3} = & \ \operatorname{ch}_0 - R_1 - R_2 - R_3 - R_4\\
  \operatorname{ch}_{-2\widehat \omega_4} = & \ \operatorname{ch}_0 - R_1 - R_2 - R_3 + R_4 \ ,
\end{align}
where $\operatorname{ch}_0$ is the vacuum character of $\widehat{\mathfrak{so}}(8)_{-2}$, identified with the Schur index $\mathcal{I}_{0,4}$. Therefore, one may write the half Wilson line index as a linear combination of the five module characters,
\begin{align}
  \langle W_j\rangle_{0,4}
  = (\delta_{j \in \mathbb{Z}} - \frac{1}{2}\mathcal{M}_{1j} - \frac{1}{2}\mathcal{M}_{2j})\operatorname{ch}_0
  + & \ \frac{1}{2}(\mathcal{M}_{1j} - \mathcal{M}_{2j})\operatorname{ch}_1
  + \frac{1}{2}(\mathcal{M}_{3j} - \mathcal{M}_{2j})\operatorname{ch}_2 \nonumber\\
  & \ + \frac{1}{2}(\mathcal{M}_{3j} + \mathcal{M}_{4j})\operatorname{ch}_3
  + \frac{1}{2}(\mathcal{M}_{3j} - \mathcal{M}_{4j})\operatorname{ch}_4 \ , \nonumber
\end{align}
where we define the rational functions
\begin{align}
  \mathcal{M}_{ij} \coloneqq \sum_{\substack{m = - j\\m \ne 0}}^{+j}\frac{M_i^{2m} - M_i^{-2m}}{q^m - q^{-m}} \ .
\end{align}



With the half-Wilson index, the index of a full Wilson line operator in the fundamental representation is then given by
\begin{align}
  \langle W_{j = \frac{1}{2}}^\text{full}\rangle_{0,4} = \langle W_{j = \frac{1}{2}} W_{j = \frac{1}{2}}\rangle_{0,4}
  = \mathcal{I}_{0,4} + \langle W_{j = 1}\rangle_{0,4} \ .
\end{align}
By S-duality, this Wilson operator is mapped to the `t Hooft operator with a minimal magnetic charge $B = (-1, 1)$ which receives contribution from monopole bubbling \cite{Gang:2012yr}. The `t Hooft index is given by a slightly more involved contour integral,
\begin{align}
  \langle H_{1,-1}\rangle_{0,4}
  = & \ \oint \frac{da}{2\pi i a} \frac{2q^{\frac{5}{12}}\prod_{i =1}^{4}(a - M_i)(-1 + aM_i)}{(-1 + a^2)^2 (a^2 - q)(-1 + a^2 q) \prod_{i = 1}^{4}M_i}
  \left(- \frac{1}{2}\vartheta_1(\pm 2 \mathfrak{a})\right) \prod_{i = 1}^{4}\frac{\eta(\tau)^2}{\vartheta_1(\pm \mathfrak{a} + M_i)} \nonumber \\
  & \ + q^{-\frac{7}{12}} \oint \frac{da}{2\pi i a} Z_\text{mono}\left(- \frac{1}{2}\vartheta_1(\pm 2 \mathfrak{a})\right) \prod_{i = 1}^{4}\frac{\eta(\tau)^2}{\vartheta_4(\pm \mathfrak{a} + M_i)} \ ,
\end{align}
where
\begin{align}
  Z_\text{mono} = \frac{1}{q \prod_{i =1}^{4}M_i}\left[
  - \left( q + \prod_{i = 1}^{4}M_i\right)
  + \sum_{\pm}\frac{\prod_{i = 1}^{4}(q^{\frac{1}{2}}a^\pm - M_i)}{(1 - a^{\pm 2}) (1 - q a^{\pm 2})}
  \right]^2 \ .
\end{align}
We can rewrite
\begin{align}
  & \ \frac{2q^{\frac{5}{12}}\prod_{i =1}^{4}(a - M_i)(-1 + aM_i)}{(-1 + a^2)^2 (a^2 - q)(-1 + a^2 q) \prod_{i = 1}^{4}M_i} \nonumber \\
  = & \ \frac{2q^{\frac{5}{12}}}{(1-a^2)(1-a^{- 2})}
  \left[\sum_{J \in \mathbb{N}}q^{J}\sum_{j = 0}^{J}(-1)^j\chi_{J - j}(a)\right]
  \prod_{i =1}^{4} \Big(  \chi_{1/2}(a) - \chi_{1/2}(M_i)  \Big) \nonumber\\
  \coloneqq & \ \frac{2q^{\frac{5}{12}}}{(1-a^2)(1-a^{- 2})} \sum_{J \in \frac{1}{2}\mathbb{N}} \mathcal{Z}_J\chi_J(a) \ ,  \\
  Z_\text{mono} = & \ \frac{1}{q \prod_{i = 1}^{4}M_i}
  \left[
  \sum_{j \in \frac{1}{2}\mathbb{N}}(-1)^{2j + 1}\chi_j(a) g_j(M) q^{1 + j}
  \right]^2 \coloneqq \sum_{J \in\frac{1}{2} \mathbb{N}} \mathcal{Z}'_J \chi_J(a) \ .
\end{align}
where
\begin{align}
  g_{J \in \mathbb{N}}(M) \coloneqq 1 + \sum_{\substack{i, j = 1 \\ i < j}}^4M_i M_j + \prod_{i = 1}^{4}M_i \ ,\quad
  g_{J \in \mathbb{N} + \frac{1}{2}}(M) \coloneqq \sum_{i = 1}^{4}M_i + \sum_{\substack{i,j,k = 1\\i < j < k}}^{4}M_i M_j M_k \ ,
\end{align}
and $\mathcal{Z}_J$ and $\mathcal{Z}'_J$ are rational function of $q$ and fugacities $M$ which simply follow from expanding tensor product of $SU(2)$ irreps; their explicit form will be left implicit. Therefore, we have the exact formula for the `t-Hooft index,
\begin{align}
  \langle H_{1, -1}\rangle_{0,4}
  = & \ \sum_{J \in \frac{1}{2}\mathbb{N}}\mathcal{Z}_J
  \sum_{m = -J}^{+J}\sum_{\substack{k = 1 \\ 2k + 2m \ne 0}}^{+\infty}\left[\sum_{i, \pm}R_{i} \frac{k(M_i^{2k + 2m} - M_i^{ - 2k - 2m})}{q^{\frac{2k + 2m}{2}} - q^{- \frac{2k + 2m}{2}}}
      + \frac{2m}{2}\delta_{\frac{2m}{2} \in \mathbb{Z}_{< 0}} \mathcal{I}_{0,4}\right] \nonumber \\
  & \ + \sum_{J \in \frac{1}{2}  \mathbb{N}} \mathcal{Z}'_J\sum_{m = -J}^{+J} 
  \sum_{i = 1}^4 R_{i} \frac{M_i^{2m} - M_i^{-2m}}{q^m - q^{-m}} \ .
\end{align}
Unfortunately we are unable to reorganize the expression into a more elegant form. Therefore we do not further compare analytically between this `t-Hooft index with the corresponding Wilson index. Although fairly complicated, the expression $\langle H_{1, -1}\rangle_{0,4}$ remain explicitly a linear combination of $\widehat{\mathfrak{so}}(8)_{-2}$ characters, with rational functions in $b_i, q$ as the coefficients.




\subsection{Genus-one theory with two punctures \label{section:genus-one-two-punctures}}

Let us consider a higher rank theory with $g = 1$ and $n = 2$, which can be obtained by gauging a diagonal $SU(2) \times SU(2)$ subgroup of the flavor symmetry of two copies of trinion theories $\mathcal{T}_{0,3}$. There are essentially two different weak-coupling frames one can consider, and here we focus on the frame illustrated in Figure \ref{fig:genus-one-type-1}. In this frame, the original Schur index is given as a contour integral
\begin{align}
  \mathcal{I}_{1,2} 
  = \oint \prod_{i = 1}^{2}\frac{da_i}{2\pi i a_i}
  \prod_{j = 1}^{2}\prod_{\pm \pm} \frac{\eta(\tau)}{\vartheta_4(\mathfrak{b}_j \pm \mathfrak{a}_1 \pm \mathfrak{a}_2)}
  \prod_{i = 1}^{2}\left(- \frac{1}{2}\vartheta(\pm 2 \mathfrak{a}_i)\right)
  \coloneqq \oint \left[\frac{da}{2\pi i a}\right]\mathcal{Z}_{1,2}(a) \ .
\end{align}
Let us consider a half Wilson line operator associated to one of the $SU(2)$ gauge group, whose index is given by the integral
\begin{align}
  \langle W_j\rangle_{1, 2}^{(1)} = \oint \prod_{i = 1}^{2}\frac{da_i}{2\pi i a_i}
  \left(\sum_{m = - j}^{j}a_1^{2m}\right)
  \prod_{j = 1}^{2}\prod_{\pm \pm} \frac{\eta(\tau)}{\vartheta_4(\mathfrak{b}_j \pm \mathfrak{a}_1 \pm \mathfrak{a}_2)}
  \prod_{i = 1}^{2}\left(- \frac{1}{2}\vartheta(\pm 2 \mathfrak{a}_i)\right)\ .
\end{align}

% Figure environment removed

The integral can be evaluated in two different orders: first $a_1$  or first $a_2$. We choose to integrate over $a_1$ first, where the relevant poles are $ \mathfrak{a}_1 = \alpha \mathfrak{b}_j + \beta \mathfrak{a}_2 + \frac{\tau}{2}$ with residues (where $\alpha, \beta = \pm 1$)
\begin{align}
  R_{i \alpha \beta} = \frac{
    i \eta(\tau)^5 \vartheta_1(2 \beta \mathfrak{a}_2) \vartheta_1( 2 \beta \mathfrak{a}_2 + 2 \alpha \mathfrak{b}_i)
  }
  {
  4\vartheta_1(2 \alpha \mathfrak{b}_i)
  \vartheta_1(\alpha \mathfrak{b}_i - \beta \mathfrak{b}_{3-i})
  \vartheta_1(\alpha \mathfrak{b}_i + \beta \mathfrak{b}_{3-i})
  \vartheta_1(2 \mathfrak{a}_2 + \alpha \beta \mathfrak{b}_i - \mathfrak{b}_{3-i})
  \vartheta_1(2 \mathfrak{a}_2 + \alpha \beta \mathfrak{b}_i + \mathfrak{b}_{3-i}) 
  } \ , \nonumber
\end{align}
The $a_1$ integral leaves integrals of the form
\begin{align}
  \oint \frac{da_2}{2\pi i a_2} f(\mathfrak{a}_2) a_2^{n} \ ,
\end{align}
which can be carried out using formula \eqref{integration-formula-monomial}. Finally, the index in the presence of the Wilson line operator gives
\begin{align}
  \langle W_{j \in \mathbb{Z}}\rangle_{1, 2}
  = \mathcal{I}_{1,2}
     + \frac{\eta(\tau)^2}{2 \prod_{i = 1}^2 \vartheta_1(2\mathfrak{b}_i)}
     \sum_{\substack{m = - j \\ m \ne 0}}^{+j}
     \frac{\prod_{i = 1}^{2}(b_i^m - b_i^{-m})}{(q^{m/2} - q^{-m/2})^2}
     \ ,
  \quad
  \langle W_{j \in \mathbb{Z} + \frac{1}{2}}\rangle = 0 \ .
\end{align}
The result is symmetric in $b_1, b_2$ as expected. Note that the first term is clearly the vacuum character of the associated chiral algebra of $\mathcal{T}[\Sigma_{1,2}]$. The factor $\eta(\tau^2)/\prod_{i = 1}^2 \vartheta_1(2 \mathfrak{b}_i)$ arises as the unique\footnote{One can try different nested residues, but they are either zero or proportional to $\eta(\tau^2)/\prod_{i = 1}^2 \vartheta_1(2 \mathfrak{b}_i)$. } nested residue of $\mathcal{Z}_{1,2}(a)$,
\begin{align}
  \operatorname{Res}_{\mathfrak{a}_2 = - \frac{\mathfrak{b}_1 - \mathfrak{b}_2}{2}}\operatorname{Res}_{\mathfrak{a}_1 = \mathfrak{a}_2 + \mathfrak{b}_1 + \frac{\tau}{2}} \mathcal{Z}_{1,2}(\mathfrak{a}_{1,2}) = \frac{\eta(\tau)^2}{8 \vartheta_1(2 \mathfrak{b}_1)\vartheta_1(2 \mathfrak{b}_2)} \ ,
\end{align}
and is also expected to be a linear combination of non-vacuum module character, since it has been shown to satisfy a set of flavored modular differential equations that should annihilate all module characters \cite{zhu1996modular,Zheng:2022zkm}. For example, at weight-two there are two equations
\begin{align}
  0 = \Bigg[
  D_q^{(1)}
  - \frac{1}{4} \sum_{i = 1,2} D_{b_i}^2
  -\frac{1}{4}& \ \sum_{\alpha_i = \pm} E_1 \begin{bmatrix}
    1 \\ b_1^{\alpha_1}b_2^{\alpha_2}
  \end{bmatrix}
  \sum_{i = 1,2}\alpha_i D_{b_i}
  - \sum_{i = 1,2} E_1 \begin{bmatrix}
    1 \\ b_i^2
  \end{bmatrix}D_{b_i} \\
  & \ + 2 \bigg(
  E_2 + \frac{1}{2} \sum_{\alpha_i = \pm}E_2 \begin{bmatrix}
    1 \\ b_1^{\alpha_1}b_2^{\alpha_2}
  \end{bmatrix}
  + \sum_{i = 1,2} E_2 \begin{bmatrix}
    1 \\ b_i^2
  \end{bmatrix}
  \bigg) \Bigg] \mathcal{I}_{1,2} \ ,
\end{align}
and
\begin{align}
  0 = \left(D_{b_1}^2 + 4 E_1 \begin{bmatrix}
    1 \\ b_1^2 
  \end{bmatrix}
  - 8 E_2 \begin{bmatrix}
    1 \\ b_1^2
  \end{bmatrix}\right) \mathcal{I}_{1,2}
  = \left(D_{b_2}^2 + 4 E_1 \begin{bmatrix}
    1 \\ b_2^2 
  \end{bmatrix}
  - 8 E_2 \begin{bmatrix}
    1 \\ b_2^2
  \end{bmatrix}\right)\mathcal{I}_{1,2} \ .
\end{align}








\subsection{\texorpdfstring{Type-1 half Wilson line index in $\mathcal{T}[\Sigma_{g,n}]$}{}}


% Figure environment removed

Now we are ready to consider more general type $A_1$ class-$\mathcal{S}$ theories $\mathcal{T}[\Sigma_{g,n}]$. Any such theory usually admits several weak-coupling limits as different supersymmetric gauge theories. With respect to each gauge theory description, we can introduce a half Wilson operator associated to one of the $SU(2)$ gauge group. In general one can introduce Wilson line charged under multiple $SU(2)$ gauge groups in the weak-coupling description, however, we leave the study of their index and correlation functions to future work.

Let us build on top of the previous $\langle W_j\rangle_{1,2}$ by extending the corresponding Riemann surface to the left and right, while maintaining the location of the Wilson line operator. We simply refer to such construction of Wilson line operator as type-$1$. The resulting configuration is shown in Figure \ref{Wilson-loop-type-1}, and it is clear from the figure that type-1 Wilson line operator encircles a tube that when cut the Riemann surface $\Sigma_{g, n}$ remain connected. Put differently, the type-1 Wilson operator can be constructed from a single connected Riemann surface $\Sigma_{g, n + 2}$ where one glues two punctures and simultaneous inserts a Wilson operator at the tube. In this subsection we will prove that the index os type-1 Wilson line operator in the spin-$j$ representation is given by
\begin{align}\label{Wilson-index-1-general}
  \langle W_{j \in \mathbb{Z}}\rangle^{(1)}_{g \ge 1, n}
  = & \ \mathcal{I}_{g,n}
  - \frac{1}{2}\left[
    \prod_{i = 1}^{n} \frac{i \eta(\tau)}{\vartheta_1(2 \mathfrak{b}_i)}
  \right]
    \sum_{\substack{m = - j\\m \ne 0}}^{+ j}
    \left[\frac{\eta(\tau)}{q^{m/2} - q^{-m /2}}\right]^{2g - 2}
    \prod_{i = 1}^{n} \frac{b_i^m - b_i^{-m}}{q^{m/2} - q^{- m /2}} \ ,\\
  \langle W_{j \in \mathbb{Z} + \frac{1}{2}}\rangle^{(1)}_{g \ge 1, n} = & \ 0 \ .
\end{align}
Although in any given gauge theory description of $\mathcal{T}[\Sigma_{g,n}]$ there may be different choices of $SU(2)$ gauge groups to support a half Wilson line, the final index is actually independent of the choice, as long as they are all type-1. Also we emphasize that type-1 Wilson line exists only for genus $g \ge 1$.

The factor $\eta(\tau)^{2g - 2}\prod_{i = 1}^n \frac{\eta(\tau)}{\vartheta_1(2 \mathfrak{b}_i)}$ can be shown to be the unique\footnote{Up to some numerical factors.} nested residue of the integrand $\mathcal{Z}_{g,n}$ that computes the original Schur index. The uniqueness is only true for $g \ge 1$, as we have already encountered four different residues $R_i$ in the $\mathcal{T}[\Sigma_{0,4}]$ computation; in this sense, class-$\mathcal{S}$ theories at $g \ge 1$ seem to enjoy some nicer properties than the $g = 0$ counterparts \footnote{See also \cite{Satoshi:2023}, where Landau-Ginzburg description can be found for $g \ge 1$ $\mathcal{N} = (0,2)$ and $(0,4)$ class-$\mathcal{S}$ theories in two dimensions. It might suggest some subtle difference in the representation theory of associated chiral algebras of the $g = 0$ and $g \ge 1$ cases. It will be interesting to clarify this issue in the future.}. Extrapolating from the discussions in \cite{Zheng:2022zkm,Pan:2021ulr}, it is natural to expect that this factor is a solution to the set of flavored modular differential equations that annihilate the Schur index, namely, the vacuum character of the associated chiral algebra $\chi(\mathcal{T}[\Sigma_{g,n}])$ of $\mathcal{T}[\Sigma_{g,n}]$, and therefore a linear combination (with constant coefficients) of non-vacuum module characters. This implies that the Wilson line index is also a linear combination of $\chi(\mathcal{T}[\Sigma_{g,n}])$ characters, with rational coefficients
\begin{align}
  \sum_{\substack{m = - j\\m \ne 0}}^{+ j}
  \left[\frac{1}{q^{m/2} - q^{-m /2}}\right]^{2g - 2}
  \prod_{j = 1}^{n} \frac{b_j^m - b_j^{-m}}{q^{m/2} - q^{- m /2}}\  .
\end{align}
The closed-form expression is essentially a sum of products of contributions from the punctures and a contribution and a ``three point function'' contribution, which closely resembles that of the $q$-deformed Yang-Mills partition function on $\Sigma_{g, n}$. It would be interesting to match our result in detail with results by punctured network \cite{Watanabe:2016bwr,Watanabe:2017bmi}.


The proof of the index formula (\ref{Wilson-index-1-general}) can be done recursively by assuming at $g \ge 1, n\ge 0$ $\langle W_j\rangle^{(1)}_{g,n}$ is given by the anzatz (\ref{Wilson-index-1-general}). We already know that the above anzatz works for the $g = 1, n = 2$ case. We can compute $\langle W_j\rangle_{g, n + 1}^{(1)}$ by gluing $\mathcal{T}[\Sigma_{0,3}]$ to that associated to $\mathcal{T}[\Sigma_{g, n}]$,
\begin{align}
  \langle W_j \rangle_{g, n + 1}^{(1)} = \oint \frac{da}{2\pi i a} \langle W_j\rangle^{(1)}_{g, n}(\mathfrak{b}_1, \ldots, \mathfrak{b}_{n - 2}, \mathfrak{a}) \mathcal{I}_\text{VM}(\mathfrak{a}) \mathcal{I}_{0,3}( - \mathfrak{a}, \mathfrak{b}_{n - 1}, \mathfrak{b}_n) \ .
\end{align}
Let us compute
\begin{align}
  & \ \langle W_{j \in \mathbb{Z}}\rangle_{g, n + 1} \nonumber\\
  = & \ \oint \frac{da}{2\pi i a}\bigg[\mathcal{I}_{g, n}(\mathfrak{b}_1, \ldots, \mathfrak{b}_{n - 1}, \mathfrak{a}) \mathcal{I}_\text{VM}(\mathfrak{a}) \mathcal{I}_{0,3}(-\mathfrak{a}, \mathfrak{b}_n, \mathfrak{b}_{n + 1}) \nonumber \\
  & \ - \frac{i^n \eta(\tau)^n}{2
    \vartheta_1(2 \mathfrak{a})\prod_{j = 1}^{n - 1} \vartheta_1(2 \mathfrak{b}_j)
  }
    \\
  & \ \qquad \times \sum_{\substack{m = - j \\ m \ne 0}}^{+j}
  \left[\frac{\eta(\tau)}{q^{m/2} - q^{-m/2}}\right]^{2g - 2}
  \frac{\prod_{j = 1}^{n - 1}(b_j^m - b_j^{-m})}{(q^{m/2} - q^{-m/2})^n}
  (a^m - a^{-m})\mathcal{I}_\text{VM}(\mathfrak{a}) \mathcal{I}_{0,3}(-\mathfrak{a}, \mathfrak{b}_n, \mathfrak{b}_{n + 1})  \bigg]\ . \nonumber
\end{align}
The first term clearly gives $\mathcal{I}_{g, n + 1}$. The second integral is of the form (up to irrelevant factors pulled out of the integral)
\begin{align}
  \oint \frac{da}{2\pi i a} \frac{a^m - a^{-m}}{\vartheta_1(2\mathfrak{a})}
  \mathcal{I}_\text{VM}(a)\mathcal{I}_{0,3}(-\mathfrak{a}, \mathfrak{b}_n, \mathfrak{b}_{n + 1})\ .
\end{align}
It is easy to check that 
\begin{align}
  \frac{\mathcal{I}_\text{VM}(a)}{\vartheta_1(2\mathfrak{a})}\mathcal{I}_{0,3}(- \mathfrak{a}, \mathfrak{b}_n, \mathfrak{b}_{n + 1})
\end{align}
is elliptic in $\mathfrak{a}$. Therefore \eqref{integration-formula-monomial} implies that
\begin{align}
  & \ \oint \frac{da}{2\pi i a}(a^m - a^{-m})\frac{\mathcal{I}_\text{VM}(a)}{\vartheta_1(2\mathfrak{a})}\mathcal{I}_{0,3}(- \mathfrak{a}, \mathfrak{b}_1, \mathfrak{b}_2)
  =
  \frac{i \eta(\tau)}{\prod_{j = 1}^2\vartheta_1(2 \mathfrak{b}_j)} \frac{\prod_{j = 1}^{2}(b_j^m - b_j^{-m})}{(q^{m/2} - q^{-m/2})} \ .
\end{align}
In other words we have verified $\langle W_j\rangle_{g, n + 1}^{(1)}$ also satisfies (\ref{Wilson-index-1-general}),
\begin{align}
  \langle W_{j \in \mathbb{Z}}\rangle _{g, n + 1}
  = \mathcal{I}_{g, n + 1}
    - \frac{i^{n + 1}\eta(\tau)^{n + 1}}{2\prod_{j = 1}^{n + 1}\vartheta_1(2 \mathfrak{b}_j)} \sum_{\substack{m = -j\\m \ne 0}}^{+j}
    \left[\frac{\eta(\tau)}{q^{m/2} - q^{-m/2}}\right]^{2g - 2}
    \frac{\prod_{j = 1}^{n+1}(b_j^m - b_j^{-m})}{(q^{m/2} - q^{-m/2})^{n + 1}} \ .
\end{align}


In the direction of increasing genus $g$, one can glue pairs of punctures to obtain Wilson line operator index $\langle W_j\rangle_{g + 1, n}^{(1)}$ for theories of higher genus $g + 1$,
\begin{align}
  \langle W_j \rangle^{(1)}_{g + 1, n} = \oint \frac{da}{2\pi i a} \mathcal{I}_\text{VM}(a) \langle W_j \rangle^{(1)}_{g, n + 2}(\mathfrak{b}_1, \ldots, \mathfrak{b}_n, \mathfrak{a}, - \mathfrak{a}) \ .
\end{align}
Assuming the anzatz hods at genus $g$, we have
\begin{align}
  & \ \langle W_j\rangle^{(1)}_{g + 1, n} \nonumber \\
   = & \ \mathcal{I}_{g + 1, n}
   - \oint \frac{da}{2\pi i a} \frac{1}{2} \frac{i^n \eta(\tau)^n}{\prod_{j = 1}^{n}\vartheta_1(2\mathfrak{b}_j)}
   \frac{i^2 \eta(\tau)^2}{\vartheta_1(\pm 2 \mathfrak{a})}\left(-\frac{1}{2} \vartheta_1(\pm 2 \mathfrak{a})\right)\\
   & \ \qquad \qquad\qquad\times \sum_{\substack{m = - j\\ m \ne 0}}^{+ j}
   \left[\frac{\eta(\tau)}{q^{m/2} - q^{-m /2}}\right]^{2g - 2}
   \frac{(a^{m} - a^{-m})(a^{- m} - a^{+m})}{(q^{m/2} - q^{-m/2})^2}
   \prod_{j = 1}^{n}\frac{b_j^{m} - b_j^{-m}}{q^{m/2} - q^{-m/2}} \ .\nonumber
\end{align}
The two $\vartheta_1(\pm 2 \mathfrak{a})$ factors are cancelled, while
\begin{align}
  {(a^{m} - a^{-m})(a^{- m} - a^{+m})} = - a^{2m} - a^{-2m} + 2 \ .
\end{align}
Only the $+2$ survives the $a$-integration since $m \ne 0$. Hence,
\begin{align}
  \langle W_j\rangle^{(1)}_{g + 1, n}
  = \mathcal{I}_{g + 1, n}
  - \frac{1}{2}\prod_{j = 1}^{n}\frac{i \eta(\tau)}{\vartheta_1(2 \mathfrak{b}_j)}
    \sum_{\substack{m = - j\\ m \ne 0}}^{+ j}
     \left[\frac{\eta(\tau)}{q^{m/2} - q^{- m /2}}\right]^{2(g+1) - 2}
     \prod_{j = 1}^{n}\frac{b_j^{m} - b_j^{-m}}{q^{m/2} - q^{-m/2}} \ ,
\end{align}
proving the index formula (\ref{Wilson-index-1-general}).

\vspace{2em}

The Type-1 Wilson index $\langle W_j\rangle_{g \ge 1,n}^{(1)}$ can be computed in a different approach, by gluing two existing punctures and simultaneously insert a half Wilson operator,
\begin{align}
  \langle W_j\rangle_{g \ge 1, n}
  = \oint \frac{da}{2\pi i a} \chi_j(a) \mathcal{I}_{g - 1, n + 2}(\mathfrak{b}_1, \ldots, \mathfrak{b}_n, \mathfrak{a}, - \mathfrak{a}) \mathcal{I}_\text{VM}(\mathfrak{a}) \ .
\end{align}
Recall that for $g \ge 0, n > 0$, the $A_1$ Schur index is given by
\begin{align}
  \mathcal{I}_{g, n} = & \ \frac{i^n}{2} \frac{\eta(\tau)^{n + 2g - 2}}{\prod_{j = 1}^{n}\vartheta_1(2 \mathfrak{b}_j)}
  \sum_{\vec\alpha = \pm}\Big(  \prod_{j = 1}^{n}\alpha_j  \Big)\sum_{k = 1}^{n + 2g - 2}\lambda_k^{(n + 2g - 2)} E_k\left[\begin{matrix}
    (-1)^n \\ \prod_{j = 1}^{n}b_j^{\alpha_j}
  \end{matrix}\right] \ .
\end{align}
After identifying $\mathfrak{b}_{n + 1} = \mathfrak{a}$, $\mathfrak{b}_{n + 2} = - \mathfrak{a}$ and multiplying the vector multiplet contribution $\mathcal{I}_\text{VM}(\mathfrak{a})$, all the $\vartheta_1(2\mathfrak{a})$ factors cancel out, and the integration variable $a$ is only present inside the Eisenstein series. When $j \in \mathbb{Z}$, the constant term in $\chi_j(a)$ leads to a additive term $\mathcal{I}_{g, n}$. For the terms in $\chi_j(a)$ with non-zero $m$, we proceed with the integration,
\begin{align}
  \oint \frac{da}{2\pi i a} a^{2m} \frac{i^{n + 2}}{2} \frac{\eta(\tau)^{2g - 2 + n}}{\prod_{j = 1}^{n}\vartheta_1(2\mathfrak{b}_j)}
  \sum_{\vec \alpha = \pm 1} \left(\prod_{j = 1}^{n + 2}\alpha_j\right)
  \sum_{k = 1}^{2g - 2 + n}
  \lambda_k^{(2g - 2 + n)}
  E_k \begin{bmatrix}
    (-1)^n \\
    \prod_{j = 1}^{n + 2}b_j^{\alpha_j}
  \end{bmatrix}_{\substack{b_{n + 1} = a\\b_{n + 2} = 1/a}} \ .
\end{align}
Only the terms with $\alpha_{n + 1} = - \alpha_{n + 2} \coloneqq \beta$, such that $b_{n + 1}^{\alpha_{n + 1}}b_{n + 2}^{\alpha_{n + 2}} = a^{2\beta}$, survives the integration since $2m \ne 0$. 

Let us look at cases with even $n$, where the integral becomes
\begin{align}
  % & \ \frac{i^{n}}{2} \frac{\eta(\tau)^{2g - 2 + n}}{\prod_{j = 1}^{n}\vartheta_1(2\mathfrak{b}_j)}
  % \oint \frac{da}{2\pi i a} a^{2m}
  % \sum_{\beta = \pm}\sum_{\vec \alpha = \pm 1} \left(
  % \prod_{j = 1}^{n}\alpha_j\right)
  % \sum_{k = 1}^{2g - 2 + n}
  % \lambda_{k}^{(2g - 2 + n)}E_k \begin{bmatrix}
  %   (-1)^n \\
  %   a^{2\beta}\prod_{j = 1}^{n}b_j^{\alpha_j}
  % \end{bmatrix}\\
  = & \ - \frac{i^{n}}{2} \frac{\eta(\tau)^{2g - 2 + n}}{\prod_{j = 1}^{n}\vartheta_1(2\mathfrak{b}_j)}
  \sum_{k = 1}^{2g - 2 + n}
    \lambda_k^{(2g - 2 + n)}
    \frac{q^m}{(k-1)!}
    \frac{\text{Eu}_{k - 1}(q^m)}{(1 - q^m)^k}
  \prod_{j = 1}^{n}(b_j^{m} - b_j^{-m})\ . \nonumber
\end{align}
where we applied integration formula. Note also that $k$ is even in order for the rational numbers $\lambda$ to be non-vanishing, and
\begin{align}
  \sum_{\vec \alpha = \pm}\left(\prod_{j = 1}^{n}\alpha_j\right)
  \left(
  \frac{1}{\prod_{j = 1}^n b_j^{m\alpha_j}}
  + \prod_{j = 1}^{n}b_j^{m \alpha_j}
  \right)
  = 2 \prod_{j = 1}^{n}(b_j^{m} - b_j^{-m}) \ .
\end{align}
Therefore,
\begin{align}
  \langle W_j\rangle_{g, n}^{(1)}
  = & \ \mathcal{I}_{g, n}\delta_{j \in \mathbb{Z}}
  - \sum_{\substack{m = - j\\m \ne 0}}^{+j} \frac{i^{n}}{2} \frac{\eta(\tau)^{2g - 2 + n}}{\prod_{j = 1}^{n}\vartheta_1(2\mathfrak{b}_j)}\prod_{j = 1}^{n}(b_j^{m} - b_j^{-m})
    \sum_{k = 1}^{2g - 2 + n}
      \lambda_k^{(2g - 2 + n)}
      \frac{q^m}{(k-1)!}
      \frac{\text{Eu}_{k - 1}(q^m)}{(1 - q^m)^k} \nonumber \\
  = & \ \mathcal{I}_{g, n}\delta_{j \in \mathbb{Z}}
  - \frac{1}{2}
    \prod_{i = 1}^{n} \frac{i \eta(\tau)^n}{\vartheta_2(\mathfrak{b}_j)}
    \sum_{\substack{m = - j\\m \ne 0}}^{+j} 
    \frac{\eta(\tau)^{2g - 2}}{(q^{m/2} - q^{-m/2})^{2g - 2}}\prod_{j = 1}^{n}\frac{b_j^{m} - b_j^{-m}}{q^{m/2} - q^{-m/2}} \ ,
\end{align}
where in the second equality we apply the identity (for even $n$)
\begin{align}
  \sum_{k = 1}^{2g -2 + n}\lambda_k^{(2g - 2 + n)} \frac{q^m}{(k-1)!} \frac{\text{Eu}_{k - 1}(q^m)}{(1 - q^m)^k}
  = \frac{1}{(q^{m/2} - q^{-m/2})^{2g - 2 + n}} \ .
\end{align}


A similar computation can be carried out with odd $n$. Again, an $m \ne 0$ term integrates to
\begin{align}
  = + \frac{i^n}{2} \frac{\eta(\tau)^{2g - 2 + n}}{\prod_{j = 1}^{n} \vartheta_1(2 \mathfrak{b}_j)}
  \sum_{k = 1}^{2g - 2 + n} 
  \lambda_k^{(2g - 2 + n)} \frac{q^{m/2}}{(k - 1)!}\Phi(q^m, 1 - k, \frac{1}{2})\prod_{j - 1}^{n}(b_j^m - b_j^{-m}) \ ,
\end{align}
where we used for odd $n$,
\begin{align}
  \sum_{\vec \alpha = \pm } \left(\prod_{j = 1}^{n}\alpha_j\right)
  \left(\prod_{j = 1}^{n}b_j^{-m} - \prod_{j = 1}^{n} b_j^m\right)
  = -2 \prod_{j = 1}^{n}(b_j^m - b_j^{-m})\ .
\end{align}
For odd $n$ we continue to have the same formula as the even $n$ case,
\begin{align}
  \langle W_j\rangle^{(1)}_{g,n}
  = & \ \mathcal{I}_{g, n}\delta_{j \in \mathbb{Z}}
  - \frac{1}{2}
    \prod_{i = 1}^{n} \frac{i \eta(\tau)^n}{\vartheta_1(\mathfrak{b}_j)}
    \sum_{\substack{m = - j\\m \ne 0}}^{+j} 
    \frac{\eta(\tau)^{2g - 2}}{(q^{m/2} - q^{-m/2})^{2g - 2}}\prod_{j = 1}^{n}\frac{b_j^{m} - b_j^{-m}}{q^{m/2} - q^{-m/2}} \ ,
\end{align}
thanks to the curious identity for odd $n$,
\begin{align}
  \sum_{k = 1}^{2g - 2 + n}\lambda_k^{(2g - 2 + n)} \frac{q^{m/2}}{(k - 1)!}
  \Phi(q^m, 1 - k, \frac{1}{2}) = - \frac{1}{(q^{m/2} - q^{-m/2})^{2 g - 2 + n}} \ .
\end{align}






\subsection{\texorpdfstring{Type-2 half Wilson line index in $\mathcal{T}[\Sigma_{g,n}]$}{}}

Next we consider another type of half Wilson operator index, which can be built on top of that of the $SU(2)$ SQCD by extending the relevant Riemann surface on either sides (but not further connecting the two sides). Put differently, we consider a half Wilson operator sitting at a tube that separates the Riemann surface into two disconnected pieces $\Sigma_{g_1, n_1}$ and $\Sigma_{g_2, n_2}$. See Figure \ref{fig:type-2-Wilson-line}. Let us denote such a Wilson index by $\langle W_j\rangle^{(2)}_{g_1, n_1; g_2, n_2}$. In this notation, the previous Wilson index $\langle W_j\rangle_{0,4}$ of the $SU(2)$ SQCD can be written as $\langle W\rangle_{0,3;0;3}^{(2)}$.
% Figure environment removed


% Figure environment removed

% Figure environment removed



\subsubsection{Simple type-2 examples}

We begin our analysis by looking at a simple genus-one configuration in Figure \ref{fig:genus-one-type-2}. It can be constructed from the $SU(2)$ SQCD by gauging the diagonal of the $SU(2)_{b_1} \times SU(2)_{b_2}$. The Wilson index can be computed by
\begin{align}
  \langle W_j \rangle^{(2)}_{1, 2}
  = \oint \frac{da}{2\pi i a} \langle W\rangle_{0,3;0;3}^{(2)}\Big|_{\substack{b_1 = a\\b_2 = 1/a}} \left(-\frac{1}{2}\right)\vartheta_1(\pm 2 \mathfrak{a}) \ .
\end{align}
Recall that (\ref{Wilson-index-SQCD})
\begin{align}
  \langle W\rangle_{0,3;0;3}^{(2)}
  = & \ \mathcal{I}_{0,4}\delta_{j \in \mathbb{Z}}
  - \sum_{i = 1}^{4} \left(\sum_{\substack{m = - j \\ m \ne 0}}^{+j} \frac{M_i^{2m} - M_i^{-2m}}{q^{m} - q^{-m}}\right)R_i \ ,
\end{align}
where $M_1 = b_1b_2$, $M_2 = b_1/b_2$, $M_3 = b_3 b_4$ and $M_4 = b_3 / b_4$. Obviously as $b_1 = a, b_1 = 1/a$, the $i = 1$ term does not contribute. Therefore, the Wilson index reads (where we have renamed $b_3, b_4 \to b_1, b_2$),
\begin{align}
  \langle W_j\rangle^{(2)}_{1,1; 0,3}
  = & \ \delta_{j \in \mathbb{Z}}\mathcal{I}_{1,2}
  - \frac{\eta(\tau)^2}{\prod_{i = 1}^2\vartheta_1(2 \mathfrak{b}_i)}
    \sum_{\substack{m = j \\ m\ne 0}}^{+j} (q^m + q^{-m})\prod_{i =1,2}\frac{b_i^{2m} - b_i^{-2m}}{q^m - q^{-m}} \nonumber \\
  & \ - \frac{\eta(\tau)^2}{2 \prod_{i=3,4} \vartheta_1(2 \mathfrak{b}_i)}
  \sum_{\alpha = \pm} \bigg(\alpha
  E_1 \begin{bmatrix}
    1 \\ b_1 b_2^\alpha  
  \end{bmatrix}
  \sum_{\substack{m = -j\\m \ne 0}}^{+j}
  \frac{(b_1b_2^\alpha)^{2m} - (b_1b_2^\alpha)^{- 2m}}{q^m - q^{-m}}
  \bigg) \ .
\end{align}
There are four major terms in this half Wilson index, which are proportional respectively to four linear independent expressions,
\begin{align}
  \mathcal{I}_{1,2}, \qquad
  \frac{\eta(\tau)^2}{\prod_{i = 1}^2 \vartheta_1(2 \mathfrak{b}_i)} , \qquad
  \frac{\eta(\tau)^2}{\prod_{i = 1}^2 \vartheta_1(2 \mathfrak{b}_i)}
    E_1 \begin{bmatrix}
      1 \\ b_1 b_2^\pm
    \end{bmatrix} \ ,
\end{align}
with rational coefficients in $b_i, q$. The first two factors have appeared previously in section \ref{section:genus-one-two-punctures}, both being solutions to the flavored modular differential equations \cite{Zheng:2022zkm}. It turns out that the two new factors containing $E_1$ are also additional solutions to the same set of equations, and therefore the type-2 index $\langle W_j\rangle^{(2)}_{1,1; 0,3}$ is also a linear combinations of $\chi(\mathcal{T}[\Sigma_{1,2}])$ characters with rational coefficients.



Next we consider a Wilson operator as demonstrated in Figure \ref{Wilson-type-2-example-1}. There are different ways to compute the index, and the most straightforward way is through the contour integral
\begin{align}
  \langle W_j\rangle_{1,2; 0,3}^{(2)} = & \ \oint \prod_{i = }^{3}\frac{da_i}{2\pi i a_i}\left[\sum_{m = -j}^{+j}a_3^{2m}\right]
  \prod_{\pm\pm}\frac{\eta(\tau)}{
    \vartheta_4(\mathfrak{b}_1 \pm \mathfrak{a}_1 \pm \mathfrak{a}_2)
  }
  \prod_{\pm\pm}\frac{\eta(\tau)}{
    \vartheta_4(\mathfrak{a}_3 \mp \mathfrak{a}_1 \mp \mathfrak{a}_2)
  }\nonumber\\
  & \ \qquad \times \prod_{\pm \pm}\frac{\eta(\tau)}{\vartheta_4(- \mathfrak{a}_3 \pm \mathfrak{b}_2 \pm \mathfrak{b}_3)}
  \prod_{i = 1}^{3}\left(- \frac{1}{2}\vartheta_1(\pm 2 \mathfrak{a}_i)\right) \ .
\end{align}
We choose to evaluate first the $a_3$-integral, and then $a_1, a_2$-integral. The computation is fairly tedious, and we only show the end result,
\begin{align}
  & \ \langle W_j\rangle_{1,2; 0,3}^{(2)} = \mathcal{I} \delta_{j \in \mathbb{Z}} \nonumber \\
  & \ + \sum_{\alpha, \beta = \pm}\sum_{\substack{m = -j \\ m \ne 0}}^j \frac{i \eta(\tau)^3}{8 \prod_{i = }^{3}\vartheta_1(2\mathfrak{b}_i)} \bigg[
  - \frac{4\alpha \beta b_2^{2m \alpha}b_3^{2m \beta}}{q^m - q^{-m}}
  \sum_{\gamma, \delta = \pm} \delta E_2 \begin{bmatrix}
    1 \\ q^{\frac{\gamma}{2}}  b_1^\delta b_2^\alpha b_3^\beta
  \end{bmatrix}(2\tau) \nonumber \\
  & \ \qquad\qquad\qquad\qquad\qquad + \frac{\alpha \beta b_2^{2m \alpha}b_3^{2m \beta}}{q^m - q^{-m}}
  \sum_{\gamma, \delta = \pm} \delta \gamma E_1 \begin{bmatrix}
    1 \\ q^{\frac{\gamma}{2}} b_1^\delta b_2^\alpha b_3^\beta
  \end{bmatrix}(2\tau)\\
  & \ \qquad\qquad\qquad\qquad\qquad - \frac{2 \alpha \beta b_2^{2m \alpha}b_3^{2m \beta}}{q^m - q^{-m}}
  \sum_{\delta = \pm} \delta E_2 \begin{bmatrix}
    -1 \\ b_1^\delta b_2^\alpha b_3^\beta  
  \end{bmatrix} \nonumber\\
  & \ \qquad\qquad\qquad\qquad\qquad + \frac{1}{q^m - q^{-m}}\frac{1}{1 - q^{-2m\alpha}} \sum_{\kappa, \gamma, \delta = \pm}b_2^{2m\gamma \alpha}b_3^{2m \delta \alpha} \alpha \gamma \delta \kappa E_1 \begin{bmatrix}
    -1 \\ b_1^\kappa b_2^{\gamma \alpha \beta}  b_3^{\delta \alpha \beta}
  \end{bmatrix}
  \bigg] \nonumber \\
  & \ + \sum_{\alpha, \beta = \pm} \sum_m'
    \frac{b_1^{2m \alpha} + b_1^{-2m\alpha}}{(q^m - q^{-m})(q^{m \alpha} - q^{-m \alpha})}
    \frac{\alpha \eta(\tau)^6}{8 \prod_{\pm \pm}\vartheta_4(\mathfrak{b}_1 \pm \mathfrak{b}_2 \pm \mathfrak{b}_3)} \ . \nonumber
\end{align}
Note that the Eisenstein series in the first two lines depend on $2\tau$ instead of just $\tau$, a price to pay for simplifying the result using the following identities,
\begin{align}
  \sum_{\pm}E_k\left[\begin{matrix}
    \phi \\ \pm z
  \end{matrix}\right](\tau) = & \ 2 E_k\left[\begin{matrix}
    \phi \\ z^2
  \end{matrix}\right](2\tau) \ , \nonumber \\
  \sum_{\pm} \pm E_k\left[\begin{matrix}
    \phi \\ \pm z
  \end{matrix}\right](\tau)
  = & \ -2 E_k\left[\begin{matrix}
    \phi \\ z^2
  \end{matrix}\right](2\tau)
   + 2 E_k\left[\begin{matrix}
    \phi \\ z
   \end{matrix}\right](\tau)\ , \nonumber
  \\
  E_k\left[\begin{matrix}
    + 1\\z
  \end{matrix}\right](2\tau)
  + E_k\left[\begin{matrix}
    - 1\\z
  \end{matrix}\right](2\tau) = & \ 
  \frac{2}{2^k}E_k\left[\begin{matrix}
    + 1 \\ z
  \end{matrix}\right] \ ,\\
  \sum_{\pm \pm} E_k\left[\begin{matrix}
    \pm 1 \\ \pm z
  \end{matrix}\right](\tau) = & \ \frac{4}{2^k}E_k\left[
  \begin{matrix}
    + 1 \\ z^2
  \end{matrix}\right](\tau)\ . \nonumber
\end{align}



\subsubsection{General type-2 Wilson index}


From the above two examples, it is somewhat clear that the Wilson index of type-2 are significantly more complex than the type-1 index. Moreover, unlike that in type-1, the Wilson index with spin $j \in \mathbb{Z}+\frac{1}{2}$ is nontrivial. Let us compute the type-2 index from another perspective. We consider gluing two Schur index $\mathcal{I}_{g_i, n_i}$ and insert a Wilson operator at the connecting tube,
\begin{align}
  \langle W\rangle_{g_1, n_1; g_2, n_2}^{(2)}
  = \oint \frac{da}{2\pi i a} \chi_j(a) \mathcal{I}_{g_1, n_1}(\mathfrak{b}_1, \ldots, \mathfrak{b}_{n_1 - 1}, \mathfrak{a}) \mathcal{I}_\text{VM}(\mathfrak{a})
  \mathcal{I}_{g_2, n_2}( - \mathfrak{a}, \tilde{\mathfrak{b}}_1, \ldots, \tilde{\mathfrak{b}}_{n_2 - 1}) \ .
\end{align}
For this we can apply the closed-form expressions for $\mathcal{I}_{g, n}$ \cite{Pan:2021mrw}, and the above becomes
\begin{align}
  - \frac{1}{2}\oint\frac{da}{2\pi i a}
  & \ \chi_j(a)
  \frac{i^{n_1}}{2}
  \frac{\eta(\tau)^{n_1 + 2g_1 - 2}}{\prod_{j = 1}^{n_1 - 1}\vartheta_1(2 \mathfrak{b}_j)}
  \frac{\eta(\tau)^{n_2 + 2g_2 - 2}}{\prod_{j = 1}^{n_2 - 1}\vartheta_1(2 \tilde{\mathfrak{b}}_j)}\\
  & \ \times \sum_{\vec\alpha,\vec \beta} \left(\prod_{j = 1}^{n_1}\alpha_j\right)
  \left(\prod_{j = 1}^{n_2}\beta_j\right)
  \sum_{k = 1}^{n_1 + 2g_1 - 2}\sum_{\ell = 1}^{n_2 + 2g_2 - 2}
  \lambda_k^{(n_1 + 2g_1 - 2)}
  \lambda_\ell^{(n_2 + 2g_2 - 2)}\\
  & \ \qquad\qquad E_k \begin{bmatrix}
    (-1)^{n_1}  \\ a^{\alpha_{n_1}} \prod_{j = 1}^{n_1 - 1}b_j^{\alpha_j}
  \end{bmatrix}
  E_\ell \begin{bmatrix}
      (-1)^{n_2}  \\ a^{ - \beta_{n_2}} \prod_{j = 1}^{n_2 - 1}\tilde b_j^{\beta_j}
    \end{bmatrix} \ .
\end{align}
Note that the vector multiplet factor has cancelled the $\vartheta_1(2 \mathfrak{a})\vartheta_1( - 2 \mathfrak{a})$ in the denominator. Therefore, the integration boils down to computing
\begin{align}
  \oint \frac{da}{2\pi i z}\chi_j(z) E_k \begin{bmatrix}
    \pm 1 \\
    z a
  \end{bmatrix}
  E_\ell \begin{bmatrix}
    \pm 1 \\
    z b
  \end{bmatrix} \ .
\end{align}

% Figure environment removed

For the special case of $n_1 = n_2 = 1$, $g_1 = g_2 = 1$ corresponding to a Wilson line in the genus-two theory as illustrated in Figure \ref{fig:type-2-genus-two}, we can easily compute the type-2 Wilson index by applying the two identities
\begin{align}
  \oint \frac{dz}{2\pi i z} E_1 \begin{bmatrix}
      + 1 \\ z
  \end{bmatrix}^2 = \frac{q^n( (n - 2) - n q^n )}{(1 - q^n)^2}, \quad
  \oint \frac{dz}{2\pi i z} E_1 \begin{bmatrix}
      - 1 \\ z
  \end{bmatrix}^2 = \frac{q^{n/2}( (n - 1) - (n + 1) q^n )}{(1 - q^n)^2} \ . \nonumber
\end{align}
The index then reads,
\begin{align}
  \langle W_j\rangle^{(2)}_{1,1;1,1}
  = & \ \oint \frac{da}{2\pi i a}
  \chi_j(a)
  \frac{i \eta(\tau)}{\vartheta_1(2 \mathfrak{a})}
  \frac{i \eta(\tau)}{\vartheta_1(-2 \mathfrak{a})}
  \left(- \frac{1}{2}\vartheta_1(\pm 2 \mathfrak{a})\right)
  E_1 \begin{bmatrix}
    -1 \\ a  
  \end{bmatrix}
  E_1 \begin{bmatrix}
    -1 \\ a^{-1}
  \end{bmatrix} \nonumber\\
  = & \ \frac{1}{2} \bigg(
     \delta_{j \in \mathbb{Z}}\eta(\tau)^2\left(E_2(\tau) + \frac{1}{12}\right)
    - \eta(\tau)^2\sum_{\substack{m = - j \\ m\ne 0 }}^{+j}
    \frac{ (2m - 1)q^{-m} - (2m + 1)q^{m}}{(q^m - q^{-m})^2}
    \bigg) \ . \nonumber
\end{align}
We note that both $\eta(\tau)^2$ and $\eta(\tau)^2 (E_2 + \frac{1}{12})$ are solutions to the modular differential equation that annihilates the genus two Schur index $\mathcal{I}_{2,0}$ \cite{Beem:2017ooy,Zheng:2022zkm},
\begin{align}
  0 = \Big[D_q^{(6)} - 305 E_4 D_q^{(4)} - 4060E_6 D_q^{(3)}
      + 20275E_4^2 & \ D_q^{(2)} + 2100E_4 E_6 D_q^{(1)} \nonumber \\
      & \ - 68600(E_6^2 - 49125E_4^3) \Big]\mathcal{I}_{2,0} \ ,
\end{align}
and therefore the above Wilson index $\langle W_j\rangle^{(2)}_{1,1;1,1}$ is also expected to be a linear combination of characters of the chiral algebra $\chi(\mathcal{T}[\Sigma_{2,0}])$.

The same structure of linear combination actually holds true for all type-2 index $\langle W_j\rangle^{(2)}_{g_1, 1; g_2, 1}$ illustrated in Figure \ref{fig:genus-g-type-2}. Indeed, the relevant integrals are of the form ($k_i \le 2g_i - 1$)
\begin{align}
  \oint \frac{da}{2\pi i a} E_{k_1} \begin{bmatrix}
      -1 \\ a
  \end{bmatrix}E_{k_2} \begin{bmatrix}
      -1 \\ a
  \end{bmatrix}
  \sim \text{linear combination of } E_{2}, E_4, \cdots, E_{2g - 2} \ ,
\end{align}
where we have used (\ref{integration-formula-zEE-1}), (\ref{integration-formula-zEE-2}). In the end, the Wilson index $\langle W_j\rangle^{(2)}_{g_1, 1; g_2, 1}$ is a linear combination of $\eta(\tau)^{2g - 2}, \eta(\tau)^{2g - 2} E_2(\tau), \cdots, \eta(\tau)^{2g - 2}E_{2g - 2}(\tau)$ with the coefficients being rational functions of $q$. This series of functions are indeed solutions to the modular differential equations annihilating the Schur index $\mathcal{I}_{g, 0}$, as they are simply the Schur index of the vortex surface defects in the 4d theory $\mathcal{T}[\Sigma_{g,0}]$ \cite{Gaiotto:2012xa,Zheng:2022zkm}.



% Figure environment removed


For more general $n_i, g_i$, we need to apply the integration formula \eqref{integration-formula-zEE-1}, \eqref{integration-formula-zEE-2} and their variants. For example, with both $n_1, n_2$ even, we have
\begin{align}
  & \ \langle W_j\rangle_{g_1, n_1; g_2, n_2}^{(2)} \nonumber \\
  = & \ \mathcal{I}_{g_1 + g_2, n_1 + n_2 - 2}\delta_{j \in \mathbb{Z}} \nonumber\\
  & \ + \frac{\eta(\tau)^{2(g_1 + g_2) + (n_1 + n_2 - 2) - 2}}{
    2\prod_{i = 1}^{n_1 + n_2 - 2}\vartheta_1(2 \mathfrak{b}_i)
  }\\
  & \ \qquad \times \sum_{\substack{m = - j \\ m\ne 0}}^j \sum_{\vec \alpha} \left(\prod_{i=1}^{n_1 + n_2 - 2}\alpha_i\right)
  \sum_{\ell = 0}^{\operatorname{max}(n_i + 2g_i - 2)}
    \Lambda_\ell^{(g_1, n_1; g_2, n_2)}(\mathbf{b}^{2m}, q^{2m})
    E_\ell \begin{bmatrix}
    1 \\ \prod_{i}^{n_1 + n_2 -2} b_i^{\alpha_i}
  \end{bmatrix} \ . \nonumber
\end{align}
Here we have merged the two sets of flavor fugacities $(b_1, \ldots, b_{n_1 - 1})$ and $(\tilde b_1, \ldots, \tilde b_{n_2 - 1})$ into a larger set $\mathbf{b} = (b_{i}, \ldots, b_{n_1 + n_2 - 2})$, and the corresponding signs $(\alpha_1, \ldots, \alpha_{n_1 - 1}, \beta_1, \ldots, \beta_{n_2 - 1})$ into $(\alpha_1, \ldots, \alpha_{n_1 + n_2 - 2})$. Finally, the $\Lambda$ are a set of rational functions of $b_i$ and $q$ coming from applying the integration formula (\ref{integration-formula-zEE-3}),
\begin{align}
  \Lambda_\ell^{(g_1, n_1; g_2, n_2)}(\mathbf{b}^{2m}, q^{2m})
  = \sum_{k_i = 0}^{n_i + 2g_i - 2}\frac{1}{\ell!} \frac{(-1)^{k_2 + 1} q^{2m}}{\prod_{i = 1}^{n_2 - 1}\tilde b_i^{2m \beta_i}} \lambda_{k_1}^{(n_1 + 2g_1 - 2)}\lambda_{k_2}^{(n_2 + 2g_2 - 2)} \mathcal{E}_{k_1, k_2; \ell}(\mathbf{b}^{2m \alpha}, q^{2m}) \ . \nonumber
\end{align}
Although it is a finite sum, unlike the beautiful result for the type-1 index formula, we are unable to reorganize the above type-2 result into a more elegant form. It would be interesting to further explore the relation between the type-2 Wilson line index and the characters of the associated chiral algebra $\chi(\mathcal{T}[\Sigma_{g,n}])$, and it is likely that the Wilson line index has access to new characters besides those from the surface defects index \cite{Zheng:2022zkm}.



\section{Line operator index in other gauge theories}


\subsection{\texorpdfstring{$\mathcal{N} = 4$ $SU(3)$ theory}{}}

The flavored $\mathcal{N} = 4$ $SU(N)$ Schur index in the presence of Wilson line operators is studied in \cite{Hatsuda:2023iwi} using the Fermi-gas formalism. In the following we also compute some simple examples using our integration formula. The relevant integral is of the form
\begin{align}
  \langle W_\mathcal{R}\rangle
  = - \frac{1}{N!} \frac{\eta(\tau)^{3N - 3}}{\vartheta_4(\mathfrak{b})^{N - 1}}\oint \prod_{A = 1}^{N - 1}  \frac{da_A}{2\pi i a_A}
  \chi_\mathcal{R}(a)
  \prod_{\substack{A, B = 1 \\ A\ne B}}^N \frac{\vartheta_1(\mathfrak{a}_A - \mathfrak{a}_B)}{\vartheta_4(\mathfrak{b} + \mathfrak{a}_A - \mathfrak{a}_B)} \ .
\end{align}

We will focus on $N = 3$. The $SU(3)$ character $\chi_\mathcal{R}(a)$ is a sum of monomials $a_1^{n_1} a_2^{n_2}$. Note that the ratio of the Jacobi theta functions is symmetric in $a_1 \leftrightarrow a_2$ and $\mathfrak{a}_A \to -\mathfrak{a}_A$, and therefore we can focus on monomials of the form $a_1^{n_1 > 0} a_2^{n_2}$; trivial monomial $a_1^0 a_2^0$ insertion simply integrates to the original $\mathcal{N} = 4$ Schur index. Now we compute
\begin{align}
  - \frac{1}{N!} \frac{\eta(\tau)^{3N - 3}}{\vartheta_4(\mathfrak{b})^{N - 1}}\oint \prod_{A = 1}^{N - 1}  \frac{da_A}{2\pi i a_A}
  a_1^{n_1} a_2^{n_2}
  \prod_{\substack{A, B = 1 \\ A\ne B}}^N \frac{\vartheta_1(\mathfrak{a}_A - \mathfrak{a}_B)}{\vartheta_4(\mathfrak{b} + \mathfrak{a}_A - \mathfrak{a}_B)} \ ,
\end{align}
by first integrating $a_1$ and then $a_2$. The $a_1$ integration is easy, leaving an $a_2$ integration of
\begin{align}
  - a_2^{n_2} \sum_{\pm}R^{(1)}_{1,\pm} \frac{a_2^{n_1}b^{\pm n_1}}{q^{n_1/2} - q^{-n_1/2}}
  & \ - a_2^{n_2}\sum_{\pm} R^{(1)}_{2, \pm}\frac{a_2^{-2n_1}b^{\pm n_1}}{q^{n_1/2} - q^{- n_1/2}} \nonumber \\
  & \ - a_2^{n_2}\sum_{\pm; k,\ell = 0,1}R^{(1)}_{3, \pm, k\ell} \frac{a_2^{- n_1/2 }b^{\pm n_1/2}q^{n_1/4} q^{\frac{k-1}{2} n_1} (-1)^{\ell n_1}}{q^{n_1/2} - q^{- n_1/2}}\ ,
\end{align}
where the poles are all imaginary with residues given by the following table.
\begin{center}
  \renewcommand{\arraystretch}{2}
  \begin{tabular}{c|c}
    $(\mathfrak{a}_1)^{(1)}_{1, \pm}$ & $\mathfrak{a}_2 \pm \mathfrak{b} + \tau/2$\\
    \hline
    $R_{1,\pm}^{(1)}$ & $\displaystyle \frac{i}{6}\eta(\tau)^3
    \frac{
      \vartheta_4(3 \mathfrak{a}_2 \pm \mathfrak{b})
      \vartheta_1(3 \mathfrak{a}_2 \pm 2 \mathfrak{b})
    }{
      \vartheta_1(\pm 2 \mathfrak{b})
      \vartheta_1(3 \mathfrak{a}_2)
      \vartheta_4(3 \mathfrak{a}_2 \pm 3 \mathfrak{b})
    }$\\
    \hline
    $(\mathfrak{a}_1)^{(1)}_{2, \pm}$ & $\mathfrak{a}_1 = - 2 \mathfrak{a}_2 \pm \mathfrak{b} + \tau/2$\\
    \hline
    $R_{2,\pm}^{(1)}$ & $\displaystyle
    \frac{i}{6} \eta(\tau)^3
    \frac{
      \vartheta_4(3 \mathfrak{a}_2 \mp \mathfrak{b})
      \vartheta_1(3 \mathfrak{a}_2 \mp 2 \mathfrak{b})
    }{
      \vartheta_1(\pm 2 \mathfrak{b})
      \vartheta_1(3 \mathfrak{a}_2)
      \vartheta_4(3 \mathfrak{a}_2 \mp 3 \mathfrak{b})}
    = - R_{1, \mp}^{(1)}
    $\\
    \hline
    $(\mathfrak{a}_1)^{(1)}_{3, \pm, k\ell}$ & $\displaystyle - \frac{\mathfrak{a}_2}{2} \pm \frac{\mathfrak{b}}{2} + \frac{\tau}{4} + \frac{k \tau}{2} + \frac{\ell}{2}$\\
    \hline
    $R^{(1)}_{3,\pm, k\ell}$ & $
    \displaystyle
    \frac{i}{12} \frac{\eta(\tau)^3}{\vartheta_1(\pm 2 \mathfrak{b})} \prod_{\gamma = \pm} \frac{\vartheta_1(\frac{3}{2} \gamma \mathfrak{a}_2 \pm \frac{1}{2}\mathfrak{b} + \frac{1}{4}\tau + \frac{k}{2}\tau + \frac{\ell}{2})^2}{
      \vartheta_4(\frac{3}{2} \gamma \mathfrak{a}_2 \pm \frac{3}{2}\mathfrak{b} + \frac{1}{4} \tau + \frac{k}{2}\tau + \frac{\ell}{2})
      \vartheta_4(\frac{3}{2} \gamma \mathfrak{a}_2 \mp \frac{1}{2}\mathfrak{b} + \frac{1}{4} \tau + \frac{k}{2}\tau + \frac{\ell}{2})
    }
    $
  \end{tabular}
\end{center}

% \begin{align}
%   R_{1,\pm}^{(1)} = & \ \frac{i}{6}\eta(\tau)^3
%   \frac{
%     \vartheta_4(3 \mathfrak{a}_2 \pm \mathfrak{b})
%     \vartheta_1(3 \mathfrak{a}_2 \pm 2 \mathfrak{b})
%   }{
%     \vartheta_1(\pm 2 \mathfrak{b})
%     \vartheta_1(3 \mathfrak{a}_2)
%     \vartheta_4(3 \mathfrak{a}_2 \pm 3 \mathfrak{b})
%   } \ , \\
%   R_{2,\pm}^{(1)} = & \ \frac{i}{6} \eta(\tau)^3
%     \frac{
%       \vartheta_4(3 \mathfrak{a}_2 \mp \mathfrak{b})
%       \vartheta_1(3 \mathfrak{a}_2 \mp 2 \mathfrak{b})
%     }{
%       \vartheta_1(\pm 2 \mathfrak{b})
%       \vartheta_1(3 \mathfrak{a}_2)
%       \vartheta_4(3 \mathfrak{a}_2 \mp 3 \mathfrak{b})}
%   = - R^{(1)}_{1, \mp} \ ,\\ 
%   R^{(1)}_{3,\pm, k\ell} = & \ \frac{i}{12} \frac{\eta(\tau)^3}{\vartheta_1(\pm 2 \mathfrak{b})} \prod_{\gamma = \pm} \frac{\vartheta_1(\frac{3}{2} \gamma \mathfrak{a}_2 \pm \frac{1}{2}\mathfrak{b} + \frac{1}{4}\tau + \frac{k}{2}\tau + \frac{\ell}{2})^2}{
%     \vartheta_4(\frac{3}{2} \gamma \mathfrak{a}_2 \pm \frac{3}{2}\mathfrak{b} + \frac{1}{4} \tau + \frac{k}{2}\tau + \frac{\ell}{2})
%     \vartheta_4(\frac{3}{2} \gamma \mathfrak{a}_2 \mp \frac{1}{2}\mathfrak{b} + \frac{1}{4} \tau + \frac{k}{2}\tau + \frac{\ell}{2})
%   } \ .  \nonumber
% \end{align}

It can be shown that,
\begin{align}
  - \oint \frac{da_2}{2\pi i a_2} a_2^n R^{(1)}_{1,\pm} = 0 \ , \qquad
  \text{if } n \not \in 3 \mathbb{Z} \ .
\end{align}
Therefore, we only focus on $n_1 + n_2 = 3p \ge 0$. Note also that $n_2 - 2n_1 = 3(p - n_1)$ in the second sum is also a multiple of $3$. With this assumption,
\begin{align}
  & \ - \sum_{\pm}\frac{b^{\pm n_1}}{q^{n_1/2} - q^{-n_1/2}} \oint \frac{da_2}{2\pi i a_2} a_2^{3p} R^{(1)}_{1,\pm}
  = - \sum_{\pm}\frac{b^{\pm n_1}}{q^{n_1/2} - q^{-n_1/2}} \oint \frac{da_2}{2\pi i a_2} a_2^{p} \left[R^{(1)}_{1,\pm}\right]_{3\mathfrak{a}_2 \to \mathfrak{a}_2} \nonumber \\
  = & \ - \delta_{n_1 + n_2 = 0} \sum_\pm
  \frac{b^{\pm n_1}}{q^{\frac{n_1}{2}} - q^{- \frac{n_1}{2}}}
  \frac{1}{6}\frac{\vartheta_4(\mathfrak{b})}{\vartheta_4(3 \mathfrak{b})}
  \left(
  E_1 \begin{bmatrix}
    -1 \\ b^{\pm 2} q^{\frac{1}{2}}
  \end{bmatrix}
  - E_1 \begin{bmatrix}
    -1 \\ b^{\mp}  
  \end{bmatrix}
  \right) \\
  & - \delta_{n_1 + n_2 \ne 0} \sum_{\pm}
  \frac{b^{\pm n_1}}{q^{\frac{n_1}{2}} - q^{- \frac{n_1}{2}}}
  \frac{1}{6}\frac{\vartheta_4(\mathfrak{b})}{\vartheta_4(3 \mathfrak{b})}
  \left(
  \frac{q^{\frac{1}{2}p} - b^{\mp 3 p}}{q^{p/2} - q^{-p/2}}
  \right) \ . \nonumber
\end{align}
Similarly
\begin{align}
  & \ - \oint \frac{da_2}{2\pi i a_2} a_2^{n_2}\sum_{\pm} R^{(1)}_{2, \pm}\frac{a_2^{-2n_1}b^{\pm n_1}}{q^{n_1/2} - q^{- n_1/2}}  \\
  = & \ - \delta_{n_2 - 2n_1 = 0}
  \frac{1}{6}\frac{\vartheta_4(\mathfrak{b})}{\vartheta_4(3 \mathfrak{b})}
  \sum_{\pm} \frac{b^{\pm n_1}}{q^{n_1/2} - q^{- n_1/2}}
  \left(
    E_1 \begin{bmatrix}
      -1 \\ b^{\mp 2}q^{\frac{1}{2}}
    \end{bmatrix}
    - E_1 \begin{bmatrix}
      -1 \\ b^{\pm}
    \end{bmatrix}
  \right)\\
  & \ + \delta_{n_2 - 2n_1 \ne 0}
  \frac{1}{6}
  \frac{\vartheta_4(\mathfrak{b})}{\vartheta_4(3 \mathfrak{b})}
  \sum_{\pm} \frac{b^{\pm n_1}}{q^{n_1/2} - q^{- n_1/2}}
  \left(
  \frac{q^{\frac{n_2 - 2n_1}{6}} - b^{\pm (n_2 - 2n_1)}}{q^{\frac{n_2 - 2n_1}{6}} - q^{- \frac{n_2 - 2n_1}{6}}}
  \right) \ .
\end{align}
Lastly, one can also check that
\begin{align}
  \oint \frac{da_2}{2\pi i a_2}a_2^n R^{(1)}_{3, \pm, k\ell} = 0, \qquad
  \text{if } n \not \in \frac{3}{2} \mathbb{Z} \ .
\end{align}
Therefore, since $n_1 + n_2 $ is an integer, we may assume $n_2 - n_1/2 = n_1 + n_2 - \frac{3}{2}n_1 = 3p - \frac{3n_1}{2}$ with $p \in \mathbb{Z}$ in order for the integral to be non-zero,
\begin{align}
  & \ - \sum_{\pm, k,\ell} \frac{b^{\pm n_1/2}q^{n_1/4} q^{\frac{k-1}{2} n_1} (-1)^{\ell n_1}}{q^{n_1/2} - q^{- n_1/2}}
  \oint \frac{da_2}{2\pi i a_2} R^{(1)}_{3, \pm, k\ell} a_2^{n_2 - \frac{n_1}{2}} \nonumber \\
  = & \ \delta_{n_2 \ne \frac{1}{2}n_1} \frac{\vartheta_4(\mathfrak{b})}{12\vartheta_4(3 \mathfrak{b})}
  \sum_{\alpha, \gamma = \pm}\sum_{k,\ell = 0,1}
  \gamma
  \frac{
    b^{\frac{\alpha}{2}( (1 + \gamma) n_1 - 2\gamma n_2)}
    q^{- \frac{1}{12}(2k - 1)( (\gamma - 1)n_1 - 2\gamma n_2 )}
  }{
    (q^{\frac{n_1}{2}} - q^{- \frac{n_1}{2}})
    (q^{\frac{1}{6}(2n_2 - 1)}
        - q^{ - \frac{1}{6}(2n_2 - 1)})
  } \\
  & \ - \delta_{n_2 = \frac{1}{2}n_1}
  \frac{\vartheta_4(\mathfrak{b})}{12\vartheta_4(3 \mathfrak{b})}
  \sum_{\alpha, \gamma = \pm}\sum_{k,\ell = 0}^{1}
  \gamma
  \frac{
    b^{\alpha \frac{n_1}{2}}
    q^{\frac{1}{4}n_1(2k - 1)}
    (-1)^{\ell n_1}
  }{
    q^{n_1/2} - q^{- n_1/2}
  }
  E_1 \begin{bmatrix}
    -1\\
    b^{ - \frac{1}{2}\alpha(3\gamma + 1)}  
    q^{- \frac{1}{4}(2k(\gamma - 1) - (\gamma + 1))}
  \end{bmatrix} \ . \nonumber
\end{align}

In the above we have used the poles and residues of the $R$-factors listed in the following table.
{
\renewcommand{\arraystretch}{1.5}
\begin{table}[h!]
\centering
  \begin{tabular}{c|c|c}
    factor & poles & residues\\
    \hline
    $R^{(1)}_{1,\pm}$ & $\mathfrak{a}_2 = 0$ & $ - \frac{i}{6\eta(\tau)} \frac{\vartheta_4( \mathfrak{b})}{\vartheta_4( 3 \mathfrak{b})}$\\
                      & $\mathfrak{a}_2 = \mp 3 \mathfrak{b} + \frac{\tau}{2}$ & $ + \frac{i}{6\eta(\tau)} \frac{\vartheta_4( \mathfrak{b})}{\vartheta_4( 3 \mathfrak{b})}$\\
    \hline
    $R^{(1)}_{2,\pm}$ & $\mathfrak{a}_2 = 0$ & $ + \frac{i}{6\eta(\tau)} \frac{\vartheta_4( \mathfrak{b})}{\vartheta_4( 3 \mathfrak{b})}$\\
                      & $\mathfrak{a}_2 = \pm 3 \mathfrak{b} + \frac{\tau}{2}$ & $ - \frac{i}{6\eta(\tau)} \frac{\vartheta_4( \mathfrak{b})}{\vartheta_4( 3 \mathfrak{b})}$\\
    \hline
    $R^{(1)}_{3,\pm,k\ell}$ & $\mathfrak{a}_2 = \mp \frac{3}{2} \gamma \mathfrak{b} + \frac{\tau}{2} + \frac{1}{4}(2k - 1)\gamma \tau + \frac{\ell}{2}$, $\gamma = \pm 1$ & $ \gamma \frac{\vartheta_4(\mathfrak{b})}{12 \vartheta_4(3 \mathfrak{b})}$
  \end{tabular}
\end{table}
}

Putting all the above together, we have
\begin{align}
  & \ - \frac{1}{N!} \frac{\eta(\tau)^{3N - 3}}{\vartheta_4(\mathfrak{b})^{N - 1}}\oint \prod_{A = 1}^{N - 1}  \frac{da_A}{2\pi i a_A}
  a_1^{n_1} a_2^{n_2}
  \prod_{\substack{A, B = 1 \\ A\ne B}}^N \frac{\vartheta_1(\mathfrak{a}_A - \mathfrak{a}_B)}{\vartheta_4(\mathfrak{b} + \mathfrak{a}_A - \mathfrak{a}_B)}\nonumber\\
  = & \ 0 \qquad \text{ if } n_1 + n_2 \ne 0 \mod 3 ,  \\
  \text{else} = & \ 
  + \delta_{n_1 + n_2 = 0} 
    \frac{1}{6}\frac{\vartheta_4(\mathfrak{b})}{\vartheta_4(3 \mathfrak{b})}
    \sum_\pm
    \frac{b^{\pm n_1}}{q^{\frac{n_1}{2}} - q^{- \frac{n_1}{2}}}
    \left(
    E_1 \begin{bmatrix}
      -1 \\ b^{\pm 2} q^{\frac{1}{2}}
    \end{bmatrix}
    - E_1 \begin{bmatrix}
      -1 \\ b^{\mp}  
    \end{bmatrix}
    \right) \nonumber\\
    & - \delta_{n_1 + n_2 \ne 0} 
    \frac{1}{6}\frac{\vartheta_4(\mathfrak{b})}{\vartheta_4(3 \mathfrak{b})}
    \sum_{\pm}
    \frac{b^{\pm n_1}}{q^{\frac{n_1}{2}} - q^{- \frac{n_1}{2}}}
    \left(
    \frac{q^{\frac{1}{2}p} - b^{\mp 3 p}}{q^{p/2} - q^{-p/2}}
    \right) \nonumber\\
    & \ - \delta_{n_2 = 2n_1}
      \frac{1}{6}\frac{\vartheta_4(\mathfrak{b})}{\vartheta_4(3 \mathfrak{b})}
      \sum_{\pm} \frac{b^{\pm n_1}}{q^{n_1/2} - q^{- n_1/2}}
      \left(
        E_1 \begin{bmatrix}
          -1 \\ b^{\mp 2}q^{\frac{1}{2}}
        \end{bmatrix}
        - E_1 \begin{bmatrix}
          -1 \\ b^{\pm}
        \end{bmatrix}
      \right) \nonumber\\
      & \ + \delta_{n_2 \ne 2n_1}
      \frac{1}{6}
      \frac{\vartheta_4(\mathfrak{b})}{\vartheta_4(3 \mathfrak{b})}
      \sum_{\pm} \frac{b^{\pm n_1}}{q^{n_1/2} - q^{- n_1/2}}
      \left(
      \frac{q^{\frac{n_2 - 2n_1}{6}} - b^{\pm (n_2 - 2n_1)}}{q^{\frac{n_2 - 2n_1}{6}} - q^{- \frac{n_2 - 2n_1}{6}}}
      \right)\nonumber \\
    & \ - \delta_{n_2 = \frac{1}{2}n_1}
    \frac{\vartheta_4(\mathfrak{b})}{12\vartheta_4(3 \mathfrak{b})}
    \sum_{\alpha, \gamma = \pm}\sum_{k,\ell = 0}^{1}
    \gamma
    \frac{
      b^{\alpha \frac{n_1}{2}}
      q^{\frac{1}{4}n_1(2k - 1)}
      (-1)^{\ell n_1}
    }{
      q^{n_1/2} - q^{- n_1/2}
    }
    E_1 \begin{bmatrix}
      -1\\
      b^{ - \frac{1}{2}\alpha(3\gamma + 1)}  
      q^{- \frac{1}{4}(2k(\gamma - 1) - (\gamma + 1))}
    \end{bmatrix} \nonumber\\
    & \ + \delta_{n_2 \ne \frac{1}{2}n_1} \frac{\vartheta_4(\mathfrak{b})}{12\vartheta_4(3 \mathfrak{b})}
      \sum_{\alpha, \gamma = \pm}\sum_{k,\ell = 0,1}
      \gamma
      \frac{
        b^{\frac{\alpha}{2}( (1 + \gamma) n_1 - 2\gamma n_2)}
        q^{- \frac{1}{12}(2k - 1)( (\gamma - 3)n_1 - 2\gamma n_2 )}
      }{
        (q^{\frac{n_1}{2}} - q^{- \frac{n_1}{2}})
        (q^{\frac{1}{6}(2n_2 - n_1)}
            - q^{ - \frac{1}{6}(2n_2 - n_1)})
      } \nonumber \ .
\end{align}
The formula above implies the following symmetry which can be used to simplify computations,
\begin{align}
  \mathcal{I}(n_1, n_2) = & \ \mathcal{I}(n_2, n_1) = \mathcal{I}(- n_1, - n_2), \\
  \mathcal{I}(n_1, n_2) = & \ \mathcal{I}(n_1, n_1 - n_2) = \mathcal{I}(n_2 - n_1, n_2) \ .
\end{align}
With this formula, one can compute any Wilson line index in any $SU(3)$ representation $\mathcal{R}$ in closed-form. For example,
\begin{align}
  \langle W_{[1,1]} \rangle
  = & \ 2\mathcal{I}_{\mathcal{N} = 4 \ SU(3)} + 6 \mathcal{I}_{1,2} \nonumber \\
  = & \ 2 \mathcal{I}_{\mathcal{N}=4 \ SU(3)}\\
  & \ + \frac{\vartheta_4(\mathfrak{b})}{\vartheta_4(3 \mathfrak{b})}
  \Bigg[
    \frac{b\sqrt{q} - (1+b^4)q + b^3 q^{\frac{3}{2}}}{b^2 (1 - q)^2}
    + \frac{(b^2 -1)\sqrt{q}}{b(q-1)}
    \left(
    E_1 \begin{bmatrix}
      -1 \\ b  
    \end{bmatrix}
    + E_1 \begin{bmatrix}
      -1 \\ b^2 q^{\frac{1}{2}}
    \end{bmatrix}
    \right)
  \Bigg] \ . \nonumber
\end{align}

\begin{align}
  \langle W_{[2,2]} \rangle
  = & \ 3\mathcal{I}_{\mathcal{N} = 4 \ SU(3)} + 12 \mathcal{I}_{1,2}
  + 6 \mathcal{I}_{2,4} + 6 \mathcal{I}_{3,0} \nonumber \\
  = & \ 3 \mathcal{I}_{\mathcal{N}=4 \ SU(3)} \nonumber\\
  & \ + \frac{\vartheta_4(\mathfrak{b})}{\vartheta_4(3 \mathfrak{b})}
  \Bigg[
    \frac{\sqrt{q} (b^3 q^{\frac{1}{2}}-1) (-b^5 q-2 b^4 q^{\frac{1}{2}} (q+1)-b^3 (q (q+4)+2))}{b^4 \left(q^2-1\right)^2}
  \Bigg] \\
  & \ + \frac{\vartheta_4(\mathfrak{b})}{\vartheta_4(3 \mathfrak{b})}
  \Bigg[
    \frac{\sqrt{q} (+b^2 (2 q (q+2)+1) q^{\frac{1}{2}}+2 b (q+1) q+q^{3/2})}{b^4 \left(q^2-1\right)^2}
  \Bigg] \nonumber \\
  & \ + \frac{\vartheta_4(\mathfrak{b})}{\vartheta_4(3 \mathfrak{b})}
  \frac{\sqrt{q}(b^2 -1) \Big[(b^2+1) \sqrt{q}+2 b q+2 b \Big]}{b^2(q^2 - 1)}
  \left(
  E_1 \begin{bmatrix}
    -1 \\ b  
  \end{bmatrix}
  + E_1 \begin{bmatrix}
    -1 \\ b^2\sqrt{q}  
  \end{bmatrix}
  \right) \ . \nonumber
\end{align}



\begin{align}
  \langle W_{[3,3]} \rangle
  = & \ 4\mathcal{I}_{\mathcal{N} = 4 \ SU(3)} + 18 \mathcal{I}_{1,2}
  + 12 \mathcal{I}_{2,4} + 12 \mathcal{I}_{3,0} + 12 \mathcal{I}_{4,5}
  + 6 \mathcal{I}_{3,6} \ .
\end{align}














\subsection{\texorpdfstring{$\mathcal{N} = 2$ $SU(3)$ SQCD}{}}

Let us also consider Wilson operator in the $SU(3)$ SQCD. The relevant integral reads
\begin{align}
  \mathcal{I}_{SU(3) \ \text{SQCD}} = & \ - \frac{1}{3!} \eta(\tau)^{16} \oint \prod_{A = 1}^2 \frac{da_A}{2\pi i a_A}
  \chi_\mathcal{R}(a)
  \frac{\prod_{A \ne B} \vartheta_1(\mathfrak{a}_A - \mathfrak{a}_B)}{\prod_{A = 1}^3 \prod_{i = 1}^{6} \vartheta_4(\mathfrak{a}_A - \mathfrak{m}_i)} \\
  \coloneqq & \ \oint \prod_{A = 1}^2 \frac{da_A}{2\pi i a_A}
  \chi_\mathcal{R}(a) \mathcal{Z}(\mathfrak{a}, \mathfrak{m}) \ .
\end{align}


\subsubsection{Fundamental representation}

As the simplest example, we consider the fundamental representation
\begin{align}
  \chi (a) = a_1 + a_2 + \frac{1}{a_1 a_2} \ .
\end{align}
First we note that
\begin{align}
  \oint \prod_{A = 1}^2 \frac{da_A}{2\pi i a_A} a_1 \mathcal{Z}(\mathfrak{a}, \mathfrak{m})
  = \oint \prod_{A = 1}^2 \frac{da_A}{2\pi i a_A} a_2 \mathcal{Z}(\mathfrak{a}, \mathfrak{m}) \ .
\end{align}
Therefore we simply compute the one with $a_1$ insertion. The relevant poles when performing the $a_1$ integration are all imaginary given by
\begin{align}
  \mathfrak{a}_1 = \mathfrak{m}_{j_1} + \frac{\tau}{2}, \qquad
  \mathfrak{a}_1 = - \mathfrak{a}_2 - \mathfrak{m}_{j_1} + \frac{\tau}{2} \ ,
\end{align}
with the respective residues
\begin{align}
  R_{j_1} \coloneqq - \frac{1}{6}
  \frac{
    \eta(\tau)^{13}q^{\frac{1}{8}}
    \prod_{A\ne B} \vartheta_1(\mathfrak{a}_A - \mathfrak{a}_B)|_{\mathfrak{a}_1 = \mathfrak{m}_{j_1} + \frac{\tau}{2}}
  }{
    \prod_i\vartheta_4(\mathfrak{a}_2 - \mathfrak{m}_i)
    \prod_i\vartheta_4(\mathfrak{a}_2 + \mathfrak{m}_{j_1} + \mathfrak{m}_i + \frac{\tau}{2})
    \prod_{i \ne j_1}\vartheta_4(\mathfrak{m}_i - \mathfrak{m}_{j_1} - \frac{\tau}{2})
  } \ ,  \quad -R_{j_1} \ . \nonumber
\end{align}
Therefore, after the $a_1$ integral we are left with
\begin{align}
  \oint \frac{da_2}{2\pi i a_2} \left[- \sum_{j_1 = 1}^{6}R_{j_1} \frac{1}{q^1 - 1} (m_{j_1}q^{\frac{1}{2}})
      + \sum_{j_1 = 1}^{6}R_{j_1} \frac{1}{q^1 - 1} (a_2^{-1} m^{-1}_{j_1}q^{\frac{1}{2}})\right] \ .
\end{align}

Next we perform the $a_2$ integral. Each residue $R_{j_1}$ is an elliptic function with respect to $\mathfrak{a}_2$, with imaginary and real poles
\begin{align}
  \mathfrak{a}_2 = & \ + \mathfrak{m}_{j_2} + \frac{\tau}{2}, & j_2 \ne & \ j_1 \\
  \text{or}, \qquad = & \ - \mathfrak{m}_{j_1} - \mathfrak{m}_{j_2} \ ,  & j_2 \ne & \ j_1 \ .
\end{align}
The corresponding residues are, respectively,
\begin{align}
  R_{j_1 j_2} = \frac{
      \eta(\tau)^{10}
      \vartheta_4(2 \mathfrak{m}_{j_1} + \mathfrak{m}_{j_2})
      \vartheta_4(\mathfrak{m}_{j_1} + 2\mathfrak{m}_{j_2})}{
    6
    \prod_{i\ne j_1, j_2}\vartheta_1(\mathfrak{m}_{j_1} - \mathfrak{m}_i)\vartheta_1(\mathfrak{m}_{j_2} - \mathfrak{m}_i)
    \prod_{i \ne j_1, j_2} \vartheta_4(\mathfrak{m}_{j_1}+ \mathfrak{m}_{j_2} + \mathfrak{m}_i)
  }, \quad
  - R_{j_1 j_2} \ .
\end{align}
We also set $R_{j_1 j_2} = 0$ when $j_1 = j_2$. With this, we have by applying (\ref{integration-formula-f})
\begin{align}
  \oint \frac{da_2}{2\pi i a_2} R_{j_1} = R_{j_1}(\mathfrak{a}_2 = 0)
  + \sum_{j_2 = 1}^{6} R_{j_1 j_2}E_1 \begin{bmatrix}
    -1 \\ m_{j_2}  
  \end{bmatrix}
  + R_{j_1 j_2}E_1 \begin{bmatrix}
    -1 \\ m_{j_1}m_{j_2}q^{-\frac{1}{2}}
  \end{bmatrix}\ ,
\end{align}
where we have picked $\mathfrak{a}_2 = 0$ as the reference point, and
\begin{align}
  \oint \frac{da_2}{2\pi i a_2}a_2^{-1}R_{j_1}
  = & \ + \sum_{j_2 = 1}^6 R_{j_1 j_2} \frac{1}{1 - q}m_{j_1}m_{j_2}
  - \sum_{j_2 = 1}^{6}R_{j_1 j_2} \frac{1}{q^{-1} - 1} (m_{j_2}q^{\frac{1}{2}})^{-1} \\
  = & \ + \sum_{j_2 = 1}^6 R_{j_1 j_2} \frac{m_{j_1}m_{j_2} - m_{j_2}^{-1}q^{\frac{1}{2}}}{1 - q} \ .
\end{align}
Collecting the results, the integral with $a_1$-insertion reads
\begin{align}
  & \ \oint \prod_{A = }^{2} \frac{da_A}{2\pi i a_A} a_1 \mathcal{Z} \\
  = & \ \frac{q^{\frac{1}{2}}}{1 - q} \sum_{j_1 = 1}^{6}m_{j_1}\left(
  R_{j_1}(\mathfrak{a}_2 = 0)
  + \sum_{j_2 = 1}^{6} R_{j_1 j_2}E_1 \begin{bmatrix}
    -1 \\ m_{j_2}  
  \end{bmatrix}
  + R_{j_1 j_2}E_1 \begin{bmatrix}
    -1 \\ m_{j_1}m_{j_2}q^{-\frac{1}{2}}
  \end{bmatrix}
  \right) \\
  & \ - \frac{1}{(1-q)^2} \sum_{j_1 = 1}^{6}
  \sum_{j_2 = 1}^{6}R_{j_1 j_2}(m_{j_2}q^{\frac{1}{2}} - m_{j_1}^{-1}m_{j_2}^{-1}q) \ .
\end{align}

Next we compute the integral
\begin{align}
  \oint \prod_{A = 1}^{2} \frac{da_A}{2\pi i a_A} \frac{1}{a_1 a_2} \mathcal{Z} \ .
\end{align}
Similar to the previous computation, we first integrate $a_1$ with poles $\mathfrak{a}_1 = \mathfrak{m}_{j_1} + \frac{\tau}{2}$ and $\mathfrak{a}_1 = - \mathfrak{a}_2 - \mathfrak{m}_{j_2} + \frac{\tau}{2}$,
\begin{align}
  & \ \oint \frac{da_2}{2\pi i a_2} \frac{1}{a_2} \left[- \sum_{j_1 = 1}^{6}R_{j_1} \frac{1}{q^{ - 1} - 1} (m_{j_1}q^{\frac{1}{2}})^{-1}
      + \sum_{j_1 = 1}^{6}R_{j_1} \frac{1}{q^{-1} - 1} (a_2^{-1} m^{-1}_{j_1}q^{\frac{1}{2}})^{-1}\right]\\
  = & \ \oint \frac{da_2}{2\pi i a_2} \left[- \sum_{j_1 = 1}^{6}R_{j_1} \frac{1}{q^{ - 1} - 1} (a_2^{-1} m_{j_1}^{-1}q^{-\frac{1}{2}})
      + \sum_{j_1 = 1}^{6}R_{j_1} \frac{1}{q^{-1} - 1} m_{j_1}q^{ - \frac{1}{2}}\right] \ .
\end{align}
Carrying out the $a_2$ integration, we have
\begin{align}
  & \ \oint \frac{da_1}{2\pi i a_1}\frac{da_2}{2\pi i a_2} \frac{1}{a_1 a_2} \mathcal{Z}\\
  = & \ + \frac{q^{\frac{1}{2}}}{1 - q} \sum_{j_1 = 1}^{6} m_{j_1} \left(
  R_{j_1}(\mathfrak{a}_2 = 0)
  + \sum_{j_2 = 1}^{6}R_{j_1 j_2} E_1 \begin{bmatrix}
    -1 \\ m_{j_2}  
  \end{bmatrix}
  + R_{j_1 j_2} E_1 \begin{bmatrix}
    -1 \\ m_{j_1} m_{j_2} q^{-1/2}  
  \end{bmatrix}
  \right) \\
  & \ - \frac{1}{(1 - q)^2} \sum_{j_1 = 1}^{6}
  \sum_{j_2 = 1}^{6} R_{j_1 j_2} (m_{j_2}q^{+ \frac{1}{2}} - m_{j_1}^{-1}m_{j_2}^{-1}q) \ .
\end{align}
Actually, this is identical to the previous result,
\begin{align}
  \oint \frac{da_1}{2\pi i a_1}\frac{da_2}{2\pi i a_2} \frac{1}{a_1 a_2} \mathcal{Z}
  = \oint \frac{da_1}{2\pi i a_1}\frac{da_2}{2\pi i a_2} a_1 \mathcal{Z}
  = \oint \frac{da_1}{2\pi i a_1}\frac{da_2}{2\pi i a_2} a_2 \mathcal{Z} \ .
\end{align}



Combining the integration of all three terms in the fundamental characters, we therefore have a fairly simple result,
\begin{align}
  \langle W_{\mathbf{3}} \rangle_{SU(3) \ \text{SQCD}}
  = & \ \frac{3q^{\frac{1}{2}}}{1 - q}\sum_{j_1 = 1}^{6}\left(
  R_{j_10} + \sum_{j_2 = 1}^{6}R_{j_1 j_2} \left(E_1 \begin{bmatrix}
    -1 \\ m_{j_2}  
  \end{bmatrix}
  + E_1 \begin{bmatrix}
    -1 \\ m_{j_1} m_{j_2}q^{-\frac{1}{2}}  
  \end{bmatrix}
  \right)
  \right) \\
  & \ - \frac{3}{(1-q)^2} \sum_{j_1, j_2 = 1}^{6}R_{j_1 j_2} (m_{j_2} q^{\frac{1}{2}} - m_{j_1}^{-1}m_{j_2}^{-1}q) \ .
\end{align}
where we abbreviate
\begin{align}
  R_{j_1 0} \coloneqq R_{j_1}(\mathfrak{a}_2 = 0) \ .
\end{align}


\subsubsection{General representation}

The above computation can be generalized to insertion of all half Wilson operator in any representation of the gauge group $SU(3)$. The basic building block is of course a monomial $a_1^{n_1} a_2^{n_2}$. Let us therefore compute the basic integral
\begin{align}
  \oint \frac{da_1}{2\pi i a_1}\frac{da_2}{2\pi i a_2} a_1^{n_1} a_2^{n_2} \mathcal{Z} \ .
\end{align}
Note that
\begin{align}
  \oint \frac{da_1}{2\pi i a_1}\frac{da_2}{2\pi i a_2}a_2^{n_2}\mathcal{Z}
  = \oint \frac{da_1}{2\pi i a_1}\frac{da_2}{2\pi i a_2}a_1^{n_2}\mathcal{Z} \ .
\end{align}
Therefore, without loss of generality we assume $n_1 \in \mathbb{Z}_{\ne0}$, and we first perform $a_1$ and then $a_2$ integration. The first step picks up the imaginary poles $\mathfrak{a}_1 = \mathfrak{m}_{j_1} + \frac{\tau}{2}$ and $- \mathfrak{a}_2 - \mathfrak{m}_{j_1} + \frac{\tau}{2}$, which produces
\begin{align}
  & \ \oint \frac{da_2}{2\pi i a_2} a_2^{n_2} \left[- \sum_{j_1}^{6}R_{j_1} \frac{1}{q^{n_1} - 1}(m_{j_1} q^{\frac{1}{2}})^{n_1}
  - \sum_{j_1 = 1}^{6}(-R_{j_1}) \frac{1}{q^{n_1} - 1}(a_2^{-1} m_{j_1}^{-1}q^{\frac{1}{2}})^{n_1}
  \right] \\
  = & \ \oint \frac{da_2}{2\pi i a_2} \left[- \sum_{j_1}^{6} a_2^{n_2} R_{j_1} \frac{1}{q^{n_1} - 1}(m_{j_1} q^{\frac{1}{2}})^{n_1}
  - \sum_{j_1 = 1}^{6}(-R_{j_1})a_2^{n_2 - n_1} \frac{1}{q^{n_1} - 1}(m_{j_1}^{-1}q^{\frac{1}{2}})^{n_1}
  \right] \ . \nonumber
\end{align}
Depending on whether $n_2 = 0$ or $n_2 - n_1 = 0$ or a generic $n_2$, the $a_2$-integration of the two terms take different form.

For the first term, if $n_2 = 0$, then the integral picks up contributions from the imaginary poles $\mathfrak{m}_{j_2} + \frac{\tau}{2}$ and the real poles $- \mathfrak{m}_{j_1} - \mathfrak{m}_{j_2}$, which reads
\begin{align}
  & \ \oint \frac{da_2}{2\pi i a_2} \left[- \sum_{j_1}^{6} a_2^{n_2 = 0} R_{j_1} \frac{1}{q^{n_1} - 1}(m_{j_1} q^{\frac{1}{2}})^{n_1}\right] \\
  = & \ - \sum_{j_1 = 1}^{6}\frac{(m_{j_1}q^{1/2})^{n_1}}{q^{n_1} - 1}\left(
  R_{j_10} + R_{j_1 j_2} E_1 \begin{bmatrix}
    -1 \\ m_{j_2}  
  \end{bmatrix}
  + R_{j_1 j_2} E_1 \begin{bmatrix}
    -1 \\ m_{j_1} m_{j_2}q^{-1/2}  
  \end{bmatrix}
  \right) \ .
\end{align}
However, if $n_2 \ne 0$, then
\begin{align}
  & \ \oint \frac{da_2}{2\pi i a_2} \left[- \sum_{j_1}^{6} a_2^{n_2} R_{j_1} \frac{1}{q^{n_1} - 1}(m_{j_1} q^{\frac{1}{2}})^{n_1}\right] \\
  = & \ - \sum_{j_1 = 1}^{6} \frac{(m_{j_1}q^{\frac{1}{2}})^{n_1}}{q^{n_1} - 1}
  \left(
  - \sum_{j_2 = 1}^{6}R_{j_1 j_2} \frac{1}{q^{n_2} - 1}(m_{j_2}q^{\frac{1}{2}})^{n_2}
  - \sum_{j_2 = 1}^{6} (- R_{j_1 j_2}) \frac{1}{1 - q^{-m_2}} (m_{j_1}^{-1} m_{j_2}^{-1})^{n_2}
  \right) \ . \nonumber
\end{align}



For the second term, if $n_2 - n_1 = 0$, then
\begin{align}
  & \ \oint \frac{da_2}{2\pi i a_2}\left[
    \sum_{j_1 = 1}^{6}R_{j_1} a_2^{n_2 - n_1} \frac{1}{q^{n_1} - 1} (m_{j_1}^{-1} q^{\frac{1}{2}})^{n_1}
  \right]\\
  = & \ \sum_{j_1 = 1}^{6} \frac{m_{j_1}^{-n_1}q^{\frac{n_1}{2}}}{q^{n_1} - 1} \left(
  R_{j_10} + R_{j_1 j_2} E_1 \begin{bmatrix}
    -1 \\ m_{j_2}  
  \end{bmatrix}
  + R_{j_1 j_2} E_1 \begin{bmatrix}
    -1 \\ m_{j_1} m_{j_2}q^{-1/2}  
  \end{bmatrix}
  \right) \ .
\end{align}
On the other hand, if $n_2 - n_1 \ne 0$ then 
\begin{align}
  & \ \oint \frac{da_2}{2\pi i a_2}\left[
    \sum_{j_1 = 1}^{6}R_{j_1} a_2^{n_2 - n_1} \frac{1}{q^{n_1} - 1} (m_{j_1}^{-1} q^{\frac{1}{2}})^{n_1}
  \right]\\
  = & \ \sum_{j_1 = 1}^{6} \frac{m_{j_1}^{-n_1} q^{\frac{n_1}{2}}}{q^{n_1} - 1}
  \left(
  - \sum_{j_2 = 1}^{6} R_{j_1j_2} \frac{m_{j_2}^{n_2 - n_1}q^{\frac{1}{2}(n_2 - n_1)}}{q^{n_1 - n_2} - 1}
  + \sum_{j_2 = 1}^{6} R_{j_1 j_2} \frac{(m_{j_1}^{-1} m_{j_2}^{-1})^{n_2 - n_1}}{1 - q^{-(n_2 - n_1)}}
  \right) \ .
\end{align}

Putting all terms together, we have
\begin{align}
  & \ \oint \frac{da_1}{2\pi i a_1}\frac{da_2}{2\pi i a_2}a_1^{n_1 \ne 0} a_2^{n_2} \mathcal{Z} \nonumber\\
  = & \ - \delta_{n_2 = 0} \sum_{j_1 = 1}^{6}\frac{m_{j_1}^{n_1}q^{\frac{1}{2} n_1}}{q^{n_1} - 1}\left(
  R_{j_10} + R_{j_1 j_2} E_1 \begin{bmatrix}
    -1 \\ m_{j_2}  
  \end{bmatrix}
  + R_{j_1 j_2} E_1 \begin{bmatrix}
    -1 \\ m_{j_1} m_{j_2}q^{-1/2}  
  \end{bmatrix}
  \right) \nonumber\\
  & \ + \delta_{n_2 \ne 0}\sum_{j_1, j_2 = 1}^{6}
    R_{j_1j_2}\frac{m_{j_1}^{n_1}q^{\frac{1}{2} n_1}}{q^{n_1} - 1}
    \frac{(m_{j_2}q^{\frac{1}{2}})^{n_2}-(m_{j_1}^{-1}m_{j_2}^{-1})^{n_2}q^{n_2}}{q^{n_2}-1} \nonumber \\
  & \ + \delta_{n_2 = n_1}\sum_{j_1 = 1}^{6} \frac{m_{j_1}^{-n_1}q^{\frac{n_1}{2}}}{q^{n_1} - 1} \left(
  R_{j_10} + R_{j_1 j_2} E_1 \begin{bmatrix}
    -1 \\ m_{j_2}  
  \end{bmatrix}
  + R_{j_1 j_2} E_1 \begin{bmatrix}
    -1 \\ m_{j_1} m_{j_2}q^{-1/2}  
  \end{bmatrix}
  \right) \\
  & \ - \delta_{n_2 \ne n_1}\sum_{j_1, j_2 = 1}^{6}
    R_{j_1j_2}
    \frac{m_{j_1}^{-n_1} q^{\frac{n_1}{2}}}{q^{n_1} - 1}
    \frac{m_{j_2}^{n_2-n_1}q^{\frac{1}{2}(n_2-n_1)}-(m_{j_1}^{-1}m_{j_2}^{-1})^{n_2-n_1}q^{n_2-n_1}}{q^{n_2-n_1}-1} \ . \nonumber
\end{align}

For example, for the Wilson operator in the anti-fundamental representation,
\begin{align}
  \langle W_{\overline {\mathbf{3}}}\rangle_{SU(3) \ \text{SQCD}}
  = & \ 3 \oint \frac{da_1}{2\pi i a_1}\frac{da_2}{2\pi i a_2} a_1^{-1} \mathcal{Z} \nonumber \\
  = & \  - 3 \frac{q^{\frac{1}{2}}}{1 - q} \sum_{j_1 = 1}^{6}m_{j_1}^{-1}\left(
  R_{j_10} + R_{j_1 j_2} E_1 \begin{bmatrix}
    -1 \\ m_{j_2}  
  \end{bmatrix}
  + R_{j_1 j_2} E_1 \begin{bmatrix}
    -1 \\ m_{j_1} m_{j_2}q^{-1/2}  
  \end{bmatrix}
  \right) \nonumber \\
  & \  + 3 \frac{1}{1-q} \sum_{j_1, j_2 = 1}^{6}
    R_{j_1j_2}
    \frac{m_{j_2}^{-1}q^{\frac{1}{2}} - m_{j_1} m_{j_2}q}{q-1}
\end{align}




\subsection{\texorpdfstring{ $\mathcal{N} = 4$ $SO(4)$ SYM}{}}

The Lie algebra $\mathfrak{so}(4)$ is isomorphic to $\mathfrak{su}(2)^2$. The Schur index of a Lagrangian theory is only sensitive to the gauge Lie algebra, and therefore the $\mathcal{N} = 4$ $SO(4)$ and $SU(2)^2$ gauge theory share an identical Schur index,
\begin{align}
  \mathcal{I}_{SU(2)^2} = \mathcal{I}_{SO(4)}
  = \frac{1}{4}\eta(\tau)^{4} \frac{\eta(\tau)^2}{\vartheta_4(\mathfrak{b})^2} & \ \oint \prod_{A = 1}^{2} \frac{da_A}{2\pi i a_A} \prod_{\alpha, \beta = \pm}\prod_{A < B} \frac{\vartheta_1(\alpha \mathfrak{a}_A + \beta \mathfrak{a}_B)}{\vartheta_4 (\alpha \mathfrak{a}_A + \beta\mathfrak{a}_B + \mathfrak{b})} \nonumber \\
  \coloneqq & \ \oint \prod_{A = 1}^{2} \frac{da_A}{2\pi i a_A} \mathcal{Z}(\mathfrak{a}_1, \mathfrak{a}_2) \ .
\end{align}
In the following we will compute a few full Wilson operator index and compare it with the $S$-dual `t Hooft operator index using the formula in \cite{Gang:2012yr}. 


We first analyze the index of a full Wilson operator associated to the vector representation $\mathbf{4}$ and its $S$-dual. The full Wilson index reads
\begin{align}
  \langle W^\text{full}_{\mathbf{4}}\rangle_{SO(4) \ \mathcal{N} = 4} = \oint \prod_{A = 1}^{2} \frac{da_A}{2\pi i a_A} (a_1 + \frac{1}{a_1} + a_2 + \frac{1}{a_2})^2 \mathcal{Z} \ .
\end{align}
By a change of variables $\mathfrak{a}_1' \coloneqq \mathfrak{a}_1 + \mathfrak{a}_2$ and $\mathfrak{a}'_2 \coloneqq \mathfrak{a}_1 - \mathfrak{a}_2$,  the Wilson index can be rewritten as a product
\begin{align}
  \langle W^\text{full}_{\mathbf{4}}\rangle_{SO(4) \ \mathcal{N} = 4}
  = \left[- \frac{1}{2} \oint \frac{da}{2\pi i a} \frac{(a + 1)^2}{a} \frac{\vartheta_1(\pm \mathfrak{a})}{\vartheta_4(\pm \mathfrak{a} + \mathfrak{b})} \frac{\eta(\tau)^3}{\vartheta_4(\mathfrak{b})}\right]^2 \ ,
\end{align}
which is identical to
\begin{align}
  (\langle W_{j = 1/2}^\text{full} \rangle_{SU(2) \ \mathcal{N} = 4})^2 \ .
\end{align}
The vector representation of $SO(4)$ is minuscule, and the S-dual `t Hooft index is safe from monopole bulling, given by
\begin{align}
  \langle H\rangle_{SO(4) \ \mathcal{N} = 4}
  = \oint \prod_{A = 1}^{2} \frac{da_A}{2\pi i a_A}
  \frac{4q^{\frac{1}{4}} (ba_1 - a_2)(-a_1 + ba_2)(b - a_1 a_2)(-1 + b a_1 a_2)}{b^2(\sqrt{q}a_1 - a_2)(\sqrt{q}a_2 - a_1)(\sqrt{q} - a_1 a_2)(-1 + \sqrt{q}a_1 a_2 ) } \mathcal{Z}' \ , \nonumber
\end{align}
where
\begin{align}
  \mathcal{Z}' = \frac{1}{4} \eta(\tau)^4 \frac{\eta(\tau)^2}{\vartheta_4(\mathfrak{b})^2}\prod_{\alpha, \beta = \pm} \frac{\vartheta_4(\alpha \mathfrak{a}_1 + \beta \mathfrak{a}_2)}{\vartheta_1(\alpha \mathfrak{a}_1 + \beta \mathfrak{a}_2 + \mathfrak{b})} \ .
\end{align}
In terms of the $a'$ variables, the above factorizes into
\begin{align}
  \langle H\rangle_{SO(4) \ \mathcal{N} = 4} = \left[\oint \frac{da'}{2\pi i a'_1}\frac{q^{\frac{1}{8}}(b - a'_1)(-1 + b a'_1)}{b(\sqrt{q} - a'_1)(-1 + \sqrt{q}a'_1)}\frac{\eta(\tau)^3}{\vartheta_4(\mathfrak{b})} \frac{\vartheta_4(\pm \mathfrak{a}')}{\vartheta_1(\pm\mathfrak{a}' + \mathfrak{b})}\right]^2 \ .
\end{align}
Up to the square and some simple factors, the result is identical to that of the $U(2)$ minimal `t Hooft operator index (\ref{U2-t-hooft}) in section \ref{section:N4SU(2)}, and naturally
\begin{align}
  \langle H\rangle_{SO(4) \ \mathcal{N} = 4} = \langle W^\text{full}_{\mathbf{4}}\rangle_{SO(4) \ \mathcal{N} = 4} \ .
\end{align}




Next we consider the index of a full Wilson operator in chiral spinor representation $\mathbf{2}$. The corresponding character is
\begin{align}
  \chi_\mathbf{2}(a) = \frac{1}{\sqrt{a_1 a_2}} + \sqrt{a_1 a_2} \ ,
\end{align}
and the relevant index is given by
\begin{align}
  \langle W_\mathbf{2}^\text{f}\rangle_{SO(4) \ \mathcal{N} = 4} = & \ \oint \prod_{A = 1}^{2} \frac{da_A}{2\pi i a_A}
  \chi_{\mathbf{2}}(a)^2 
  \mathcal{Z}(\mathfrak{a}_1, \mathfrak{a}_2) \\
  = & \ \oint \prod_{A = 1}^{2} \frac{da_A}{2\pi i a_A}
  (1 + 1 + a_1 a_2 + \frac{1}{a_1 a_2})
  \mathcal{Z}(\mathfrak{a}_1, \mathfrak{a}_2) \ .
\end{align}
In terms of the $a'$ variable, the above factorizes
\begin{align}
  \langle W_\mathbf{2}^\text{f}\rangle_{SO(4) \ \mathcal{N} = 4}
  = & \ \left[\oint \frac{da'_1}{2\pi i a'_1}(\chi_{j = 0} + \chi_{j = 1})(a'_1) \left(- \frac{1}{2}\right)\frac{\eta(\tau)^3}{\vartheta_4(\mathfrak{b})} \frac{\vartheta_1(\pm\mathfrak{a}'_1)}{\vartheta_1(\pm\mathfrak{a}'_1 + \mathfrak{b})}\right] \mathcal{I}_{SU(2) \ \mathcal{N} = 4} \nonumber \\
  = & \ \left( \mathcal{I}_{\mathcal{N} = 4 \ SU(2)} + \langle W^\text{full}_{j = 1}\rangle_{\mathcal{N} = 4 \ SU(2)} \right) \mathcal{I}_{SU(2) \ \mathcal{N} = 4} \ .
\end{align}

The S-dual `t Hooft line index is given by
\begin{align}
  \langle H\rangle
  = \oint \prod_{A = 1}^{2} \frac{da_A}{2\pi i a_A}
  \frac{2(b - a_1 a_2)(-1 + b a_1a_2)}{bq^{\frac{1}{4}}(\sqrt{q} - a_1 a_2)(-1 + \sqrt{q} a_1 a_2)}
  \mathcal{Z}' \ , \nonumber
\end{align}
where
\begin{equation}
  \mathcal{Z}' = \frac{1}{4}\eta(\tau)^4 \frac{\eta(\tau)^2}{\vartheta_4(\mathfrak{b})^2}
  \frac{\vartheta_4(\pm (\mathfrak{a}_1 + \mathfrak{a}_2))}{\vartheta_1(\pm (\mathfrak{a}_1 + \mathfrak{a}_2) + \mathfrak{b})}
  \frac{\vartheta_1(\pm (\mathfrak{a}_1 - \mathfrak{a}_2))}{\vartheta_4(\pm (\mathfrak{a}_1 - \mathfrak{a}_2) + \mathfrak{b})}  \ .
\end{equation}
In terms of the $a'$ variables,
\begin{align}
  \langle H\rangle
  = \left[- \oint \frac{da'_1}{2\pi i a'_1} \frac{(b - a'_1)(-1 + b a_1' )}{bq^{1/4}(\sqrt{q} - a'_1)(-1 + \sqrt{q}a'_1)} \frac{\vartheta_4(\pm \mathfrak{a}_1)}{\vartheta_1(\pm \mathfrak{a}_1 + \mathfrak{b})} \frac{\eta(\tau)^3}{\vartheta_4(\mathfrak{b})}\right] \mathcal{I}_{\mathcal{N} = 4 \ SU(2)} \ .
\end{align}
The equality from S-duality also follows from the discussion in section \ref{section:N4SU(2)}.




\subsection{\texorpdfstring{$\mathcal{N} = 4$ $SO(5)$ SYM}{}}



Let us now consider $\mathcal{N} = 4$ $SO(5)$ SYM with insertion of a half Wilson operator in the fundamental representation
\begin{align}
  \oint \frac{da_1}{2\pi i a_1}\frac{da_2}{2\pi i a_2}\chi_{\mathbf{5}}(a)
  \mathcal{Z}(\mathfrak{a}_1, \mathfrak{a}_2) \ ,
\end{align}
where
\begin{equation}
\mathcal{Z}(\mathfrak{a}_1, \mathfrak{a}_2) = \frac{1}{8} \frac{\eta(\tau)^6}{\vartheta_4(\mathfrak{b})^2} \frac{
    - \vartheta_1(\mathfrak{a}_1)^2
    \vartheta_1(\mathfrak{a}_2)^2
    \vartheta_1(\mathfrak{a}_1 + \mathfrak{a}_2)^2
    \vartheta_1(\mathfrak{a}_1 - \mathfrak{a}_2)^2
  }{
    \vartheta_4(\mathfrak{a}_1 \pm \mathfrak{b})
    \vartheta_4(\mathfrak{a}_2 \pm \mathfrak{b})
    \vartheta_4(\mathfrak{a}_1 + \mathfrak{a}_2\pm \mathfrak{b})
    \vartheta_4(\mathfrak{a}_1 - \mathfrak{a}_2\pm \mathfrak{b})
  }\ ,
\end{equation}
and
\begin{align}
  \chi_{\mathbf{5}}(a) = a_1 + \frac{1}{a_1} + a_2 + \frac{1}{a_2} + 1 \ .
\end{align}

From the symmetry between $\mathcal{Z}(\mathfrak{a}_1, \mathfrak{a}_2) = \mathcal{Z}(\mathfrak{a}_2, \mathfrak{a}_1)$, we only need to compute
\begin{align}
  \oint \frac{da_1}{2\pi i a_1}\frac{da_2}{2\pi i a_2}a_1^{\pm 1}
  \mathcal{Z}(\mathfrak{a}_1, \mathfrak{a}_2) \ .
\end{align}
Moreover, the symmetry $\mathcal{Z}(\mathfrak{a}_1, \mathfrak{a}_2) = \mathcal{Z}( - \mathfrak{a}_1, \mathfrak{a}_2)$ also implies
\begin{align}
  \oint \frac{da_1}{2\pi i a_1}\frac{da_2}{2\pi i a_2}a_1
  \mathcal{Z}(\mathfrak{a}_1, \mathfrak{a}_2)
  = \oint \frac{da_1}{2\pi i a_1}\frac{da_2}{2\pi i a_2}a_1^{-1}
  \mathcal{Z}(\mathfrak{a}_1, \mathfrak{a}_2) \ .
\end{align}



The $a_1$-integration picks up imaginary poles
\begin{align}
  \mathfrak{a}_1 = \alpha \mathfrak{b} + \frac{\tau}{2}, \qquad
  \mathfrak{a}_1 = \beta \mathfrak{a}_2 + \gamma \mathfrak{b} + \frac{\tau}{2} \ , \qquad \alpha, \beta, \gamma = \pm \ ,
\end{align}
with residues respectively
\begin{align}
  R_\alpha \coloneqq \frac{i}{8}\eta(\tau)^3 \frac{\vartheta_4(\mathfrak{a}_2 + \alpha \mathfrak{b})\vartheta_4(\mathfrak{a}_2 - \alpha \mathfrak{b})}{
        \vartheta_1(2 \alpha \mathfrak{b})
        \vartheta_1(\mathfrak{a}_2 + 2 \alpha \mathfrak{b})
        \vartheta_1(\mathfrak{a}_2 - 2 \alpha \mathfrak{b})} \ ,
\end{align}
and
\begin{align}
  R_{\beta \gamma} \coloneqq\frac{i}{8} \eta(\tau)^3 \frac{
    \vartheta_4(\mathfrak{a}_2 + \beta \gamma \mathfrak{b})
    \vartheta_1(\mathfrak{a}_2)
    \vartheta_4(2 \mathfrak{a}_2 + \beta \gamma \mathfrak{b})^2
  }{
    \vartheta_1(\mathfrak{a}_2 + 2 \beta \gamma \mathfrak{b})
    \vartheta_4(\mathfrak{a}_2 - \beta \gamma \mathfrak{b})
    \vartheta_1(2\mathfrak{a}_2 )
    \vartheta_1(2 \gamma \mathfrak{b}) \vartheta_1(2 \mathfrak{a}_2 + 2 \beta \gamma \mathfrak{b})
  } \ .
\end{align}
The $a_1$-integration leaves us with
\begin{align}
  \oint \frac{da_2}{2\pi i a_2} \left[
  - \sum_{\alpha = \pm} R_\alpha \frac{1}{q^\pm - 1} (b^\alpha q^{\frac{1}{2}})^\pm
  - \sum_{\beta \gamma = \pm} R_{\beta \gamma} \frac{1}{q^\pm - 1} (a_2^\beta b^\gamma q^{\frac{1}{2}})^\pm
  \right] \ .
\end{align}
The residues $R_\alpha$ and $R_{\beta \gamma}$ are all elliptic with respect to $\mathfrak{a}_2$, and therefore the $a_2$-integration of both terms can be carried out. In $R_\alpha$, there are poles and residues
\begin{align}
  \mathfrak{a}_2 = 2 \alpha \delta \mathfrak{b}, \qquad
  \mathop{\operatorname{Res}}_{\mathfrak{a}_2 = 2\alpha \delta \mathfrak{b}}R_\alpha = - \frac{\delta}{8}\frac{\vartheta_4(3 \mathfrak{b}) \vartheta_4(\mathfrak{b})}{\vartheta_1(2 \mathfrak{b}) \vartheta_1(4 \mathfrak{b})} \ .
\end{align}
Hence
\begin{align}
  & \ - \oint \frac{da_2}{2\pi i a_2} \sum_{\alpha = \pm}R_\alpha \frac{1}{q^{\pm} - 1} (b^\alpha q^{\frac{1}{2}})^\pm \\
  = & \ - \sum_{\alpha = \pm} \frac{b^{\pm\alpha} q^{\pm \frac{1}{2}}}{q^{\pm} - 1}
  \left(
  R_\alpha(\mathfrak{a} = 0) + \sum_{\delta = \pm} \frac{- \delta}{8} \frac{\vartheta_4(3 \mathfrak{b}) \vartheta_4(\mathfrak{b})}{\vartheta_1(2 \mathfrak{b}) \vartheta_1(4 \mathfrak{b})} E_1
  \begin{bmatrix}
    -1 \\ b^{2\alpha \delta}q^{\frac{1}{2}}  
  \end{bmatrix}
  \right) \ .
\end{align}
By direct computation, one sees that the above is actually independent of $\pm$ sign in the $a_1^\pm$ insertion, consistent with the symmetry $\mathcal{Z}(\mathfrak{a}_1, \mathfrak{a}_2) = \mathcal{Z}( - \mathfrak{a}_1, \mathfrak{a}_2)$.


The term with $R_{\beta \gamma}$ can be carried using (\ref{integration-formula-monomial}),
\begin{align}
  \oint \frac{da_2}{2\pi i a_2}R_{\beta \gamma} a_2^{\pm \beta}
  = - \sum_{\operatorname{real} \ j} R_{\beta \gamma j} \frac{(a^{(\beta \gamma j)}_{2}q)^{\pm \beta}}{q^{\pm \beta} - 1}
  - \sum_{\operatorname{img} \ j} R_{\beta \gamma j}\frac{(a^{(\beta \gamma j)}_{2})^{\pm \beta}}{q^{\pm \beta} - 1} \ .
\end{align}
Here $a^{(\beta \gamma j)}_{2}$ denotes the simple poles of $R_{\beta \gamma}$ with respect to $a_2$, with the corresponding residue $R_{\beta \gamma j}$. We list the poles and their residues in Table \ref{poles-residues-SO(5)}.
{
\renewcommand{\arraystretch}{1.8}
\begin{table}[h!]
\centering
  \begin{tabular}{c|c|c}
    & poles $a_2^{(\beta \gamma j)}$ & residues $R_{\beta \gamma j}$ \\
    \hline
    Real & $\mathfrak{a}_2 = - 2\beta \gamma \mathfrak{b}$ & $ + \frac{\beta}{8} \frac{\vartheta_4( \mathfrak{b})\vartheta_4(3 \mathfrak{b})}{\vartheta_1(2  \mathfrak{b})\vartheta_1(4  \mathfrak{b})}$\\
    & $\mathfrak{a}_2 = - \beta \gamma \mathfrak{b} + \frac{1}{2}$ & $ + \frac{\beta \vartheta_4(\mathfrak{b})^2 \vartheta_3(0)}{16 \vartheta_1(2 \mathfrak{b})^2 \vartheta_3(2 \mathfrak{b})}$\\
    & $\mathfrak{a}_2 = \frac{1}{2}$ & $ - \frac{\beta \vartheta_4(\mathfrak{b})
    ^2 \vartheta_2(0)}{16 \vartheta_1(2 \mathfrak{b})^2 \vartheta_2(2 \mathfrak{b})}$\\
    & $\mathfrak{a}_2 = - \beta \gamma \mathfrak{b}$ & $ - \frac{\beta \vartheta_4(\mathfrak{b})^2 \vartheta_4(0)}{16 \vartheta_1(2 \mathfrak{b})^2 \vartheta_4(2 \mathfrak{b})}$\\
    \hline
    Imaginary & $\mathfrak{a}_2 = \beta \gamma \mathfrak{b} + \frac{1}{2}$ & $- \frac{\beta}{8} \frac{\vartheta_4( \mathfrak{b})\vartheta_4(3  \mathfrak{b})}{\vartheta_1(2 \mathfrak{b})\vartheta_1(4 \mathfrak{b})}$\\
    & $\mathfrak{a}_2 = \frac{\tau}{2}$ & $\frac{\beta \vartheta_4(\mathfrak{b})^2 \vartheta_4(0)}{16 \vartheta_1(2 \mathfrak{b})^2 \vartheta_4(2 \mathfrak{b})}$\\
    & $\mathfrak{a}_2 = \frac{1}{2} + \frac{\tau}{2}$ & $ - \frac{\beta \vartheta_4(\mathfrak{b})^2 \vartheta_3(0)}{16 \vartheta_1(2 \mathfrak{b})^2 \vartheta_3(2 \mathfrak{b})}$\\
    & $\mathfrak{a}_2 = - \beta \gamma \mathfrak{b} + \frac{1}{2} + \frac{\tau}{2}$ & $ + \frac{\beta \vartheta_4(\mathfrak{b})^2 \vartheta_2(0)}{16 \vartheta_1(2 \mathfrak{b})^2 \vartheta_2(2 \mathfrak{b})}$
  \end{tabular}
  \caption{Poles and residues of the elliptic functions $R_{\beta \gamma}$.\label{poles-residues-SO(5)}}
\end{table}
}

Performing the sum over $\beta, \gamma$,
\begin{align}
  & \ - \oint \frac{da_2}{2\pi i a_2}\sum_{\beta \gamma = \pm 1} R_{\beta \gamma} \frac{1}{q^{\pm} - 1} (a_2^\beta b^\gamma q^{\frac{1}{2}})^\pm \nonumber \\
  = & \ \frac{\sqrt{q}\left( b^2(q+1)-4b\sqrt{q}+q+1 \right)}{2}\frac{\vartheta _4(\mathfrak{b} )^2}{8b(q-1)^2\vartheta _1(2\mathfrak{b} )^2}\frac{\vartheta _2(0)}{\vartheta _2(2\mathfrak{b} )} \nonumber
  \\
  & \ + ( b^2q-b\sqrt{q}(q+1)+q ) \frac{\vartheta _4(\mathfrak{b} )^2}{8b(q-1)^2\vartheta _1(2\mathfrak{b} )^2}\left[ \frac{\vartheta _3(0)}{\vartheta _3(2\mathfrak{b} )}+\frac{\vartheta _4(0)}{\vartheta _4(2\mathfrak{b} )} \right]  \nonumber\\
  & \ +\frac{\sqrt{q} \left(-2 b^4 \sqrt{q}+b^3 (q+1)+b (q+1)-2 \sqrt{q}\right) }{8 b^2 (q-1)^2} \frac{\vartheta_4(3 \mathfrak{b}) \vartheta_4(\mathfrak{b})}{ \vartheta_1(2 \mathfrak{b}) \vartheta_1(4 \mathfrak{b})} \ ,
\end{align}
which is independent of the $\pm$ in $a_2^\pm$, consistent with the symmetry $\mathcal{Z}(\mathfrak{a}_1, \mathfrak{a}_2) = \mathcal{Z}( - \mathfrak{a}_1, \mathfrak{a}_2)$.

To summarize,
\begin{align}
  & \ \oint \frac{da_1}{2\pi i a_1}\frac{da_2}{2\pi i a_2}a_1
  \mathcal{Z}(\mathfrak{a}_1, \mathfrak{a}_2) \\
  = & \ \frac{i(b^2 - 1)\sqrt{q}}{8b(q-1)} \frac{\eta(\tau)^3 \vartheta_4(\mathfrak{b})^2}{\vartheta_1(2 \mathfrak{b})^3}\\
  & \ + \frac{\sqrt{q}(1 - 4b \sqrt{q} + q + b^2(1 + q))}{16 b(q -1)^2}
  \frac{\vartheta_4(\mathfrak{b})^2}{\vartheta_1(2 \mathfrak{b})^2}
  \frac{\vartheta_2(0)}{\vartheta_2(2 \mathfrak{b})}\\
  & \ + \frac{\sqrt{q}(b - \sqrt{q})(b\sqrt{q} - 1)}{8b (q-1)^2} \frac{\vartheta_4(\mathfrak{b})^2}{\vartheta_1(2 \mathfrak{b})^2}
  \left(
  \frac{\vartheta_3(0)}{\vartheta_3(2 \mathfrak{b})}
  + \frac{\vartheta_4(0)}{\vartheta_4(2 \mathfrak{b})}
  \right)\\
  & \ + \frac{\vartheta_4(\mathfrak{b})\vartheta_4(3 \mathfrak{b})}{\vartheta_1(2 \mathfrak{b}) \vartheta_1(4 \mathfrak{b})}
  \left(
  \frac{\sqrt{q}(b - \sqrt{q})(1 - b^3 \sqrt{q})}{4b^2 (q - 1)^2}
  + \frac{(b^2 -1)\sqrt{q}}{4b(q - 1)} E_1 \begin{bmatrix}
    -1 \\ b^2 q^{\frac{1}{2}}  
  \end{bmatrix}
  \right) \ .
\end{align}
Therefore,
\begin{align}
  \langle W_\mathbf{5}\rangle_{\mathcal{N} = 4 \ SO(5)}
  = & \ \mathcal{I}_{\mathcal{N} = 4 \ SO(5)}
  + \frac{i(b^2 - 1)\sqrt{q}}{2b(q-1)} \frac{\eta(\tau)^3 \vartheta_4(\mathfrak{b})^2}{\vartheta_1(2 \mathfrak{b})^3} \nonumber\\
  & \ + \frac{\sqrt{q}(1 - 4b \sqrt{q} + q + b^2(1 + q))}{4 b(q -1)^2}
  \frac{\vartheta_4(\mathfrak{b})^2}{\vartheta_1(2 \mathfrak{b})^2}
  \frac{\vartheta_2(0)}{\vartheta_2(2 \mathfrak{b})} \nonumber\\
  & \ + \frac{\sqrt{q}(b - \sqrt{q})(b\sqrt{q} - 1)}{2b (q-1)^2} \frac{\vartheta_4(\mathfrak{b})^2}{\vartheta_1(2 \mathfrak{b})^2}
  \left(
  \frac{\vartheta_3(0)}{\vartheta_3(2 \mathfrak{b})}
  + \frac{\vartheta_4(0)}{\vartheta_4(2 \mathfrak{b})}
  \right)\\
  & \ + \frac{\vartheta_4(\mathfrak{b})\vartheta_4(3 \mathfrak{b})}{\vartheta_1(2 \mathfrak{b}) \vartheta_1(4 \mathfrak{b})}
  \left(
  \frac{\sqrt{q}(b - \sqrt{q})(1 - b^3 \sqrt{q})}{b^2 (q - 1)^2}
  + \frac{(b^2 -1)\sqrt{q}}{b(q - 1)} E_1 \begin{bmatrix}
    -1 \\ b^2 q^{\frac{1}{2}}  
  \end{bmatrix}
  \right) \ , \nonumber
\end{align}
where the $\mathcal{I}_{\mathcal{N} = 4 \ SO(5)}$ is the original Schur index of the $SO(5)$ $\mathcal{N} = 4$ SYM.



\subsubsection{General representation}

Let us consider the $\mathfrak{so}(5)$ representations whose characters can be written as polynomials of $a_1, a_2$ with integral powers,
\begin{align}
  \chi_\mathcal{R}(a) = \sum_{n_1, n_2} c_{n_1 n_2} a_1^{n_1} a_2^{n_2} \ .
\end{align}
In particular, using the symmetry $a_1 \leftrightarrow a_2$, $a_i \leftrightarrow a_i^{-1}$, we can focus on the integrals of the following form
\begin{align}
  \oint \frac{da_1}{2\pi i a_1} \frac{da_2}{2\pi i a_2}
  a_1^{n_1 > 0}a_2^{n_2 \ge 0} \mathcal{Z} \ .
\end{align}
The $a_1$-integration leaves (recall that the $a_1$-integral picks up $6$ imaginary poles)
\begin{align}\label{a1integral}
  \oint \frac{da_2}{2\pi i a_2} a_2^{n_2} \left[
  - \sum_{\alpha = \pm}R_\alpha \frac{b^{\alpha n_1} q^{\frac{1}{2} n_1}}{q^{n_1} - 1}
  - \sum_{\beta \gamma = \pm}R_{\beta \gamma} \frac{a_2^{n_1\beta} b^{n_1\gamma} q^{\frac{1}{2} n_1}}{q^{n_1} - 1}
  \right] \ .
\end{align}
Depending on whether $n_2 = 0$ or $n_2 \ne 0$ in the first term, and whether $n_2 \pm n_1 = 0$ in the second term, the integral leads to different closed-form result. When $n_2 = 0$, the first term integrates to
\begin{align}\label{Rintegrate}
  = \delta_{n_2 = 0} \frac{1}{4}\frac{\vartheta_4(\mathfrak{b})}{\vartheta_1(2 \mathfrak{b})}
  \left(
  \frac{i \eta(\tau)^3 \vartheta_4(\mathfrak{b})}{2 \vartheta_1(2 \mathfrak{b})^2} + \frac{\vartheta_4(3 \mathfrak{b})}{\vartheta_1(4 \mathfrak{b})}
  E_1 \begin{bmatrix}
    1 \\ b^2  
  \end{bmatrix}
  \right)
  \frac{b^{n_1} - b^{-n_1}}{q^{n_1/2} - q^{- n_1/2}} \ ,
\end{align}
while when $n_2 > 0$, it integrates to
\begin{align}
  & \ - \delta_{n_2 > 0} \sum_{\alpha = \pm} \frac{b^{n_1 \alpha} q^{\frac{1}{2}n_1}}{q^{n_1} - 1}
  \sum_{\delta = \pm 1}
  \left(- \frac{\delta}{8} \frac{\vartheta_4(3 \mathfrak{b}) \vartheta_4( \mathfrak{b})}{\vartheta_1(2 \mathfrak{b})\vartheta_1(4 \mathfrak{b})}\right)
  \frac{(b^{2\alpha \delta} q^{\frac{1}{2}})^{n_2}}{q^{n_2/2} - q^{- n_2/2}} \ \nonumber\\
  = & \ - \delta_{n_2 > 0}\frac{\vartheta_4(\mathfrak{b}) \vartheta_4(3 \mathfrak{b})}{8 \vartheta_1(2 \mathfrak{b}) \vartheta_1(4 \mathfrak{b})} \frac{(b^{n_1} - b^{-n_1})(b^{2n_2} - b^{-2n_2})}{(q^{n_1/2} - q^{- n_1/2})(1 - q^{-n_2})} \ .
\end{align}

In the second term, when $0 < n_1 \ne n_2$, we have $n_2 + n_1 \beta \ne 0$ for either $\beta = \pm 1$. In this case,
\begin{align}\label{RalphabetaIntegrate}
  & \ - \delta_{n_1 \ne n_2} \oint \frac{da_2}{2\pi i a_2}\sum_{\beta \gamma = \pm} \frac{b^{n_1 \gamma} q^{\frac{1}{2}n_1}}{q^{n_1} - 1} R_{\beta \gamma}
  a_2^{n_2 + n_1 \beta} \\
  = & \ + \delta_{n_1 \ne n_2}\sum_{\beta \gamma = \pm}\frac{b^{n_1 \gamma} q^{\frac{1}{2}n_1}}{q^{n_1} - 1}  \sum_{\text{real/img} \ j} R_{\beta \gamma j} \frac{(a_2^{(\beta \gamma j)} q^{\pm \frac{1}{2}})^{n_2 + n_1 \beta}}{q^{\frac{1}{2}(n_2 + n_1 \beta)} - q^{-\frac{1}{2}(n_2 + n_1 \beta)} } \ .
\end{align}
On the other hand, when $n_1 = n_2 > 0$, we have $n_2 + n_1 \beta = 0$ for $\beta = -1$, and $n_2 + n_1 \beta = 2n_1 \ne 0$ for $\beta = 1$. In this situation,
\begin{align}
  & \ - \oint \frac{da_2}{2\pi i a_2}\sum_{\beta \gamma = \pm}R_{\beta \gamma} \frac{a_2^{n_2 + n_1 \beta} b^{n_1 \gamma} q^{\frac{1}{2}n_1}}{q^{n_1} - 1}
  \nonumber\\
  = & \ \delta _{n_1=n_2}\sum_{\gamma =\pm}{\left[ \sum_{\text{real}/\text{img}\ j}{R_{+\gamma j}\frac{( a_{2}^{\left( +\gamma j \right)})^{2n_1} q^{\pm \frac{1}{2}2n_1}}{q^{n_1}-q^{-n_1}}} \right]}\frac{b^{n_1\gamma}q^{\frac{1}{2}n_1}}{q^{n_1}-1}\\
  & \ -\delta _{n_1=n_2}\sum_{\gamma =\pm}
  \frac{b^{n_1\gamma}q^{\frac{1}{2}n_1}}{q^{n_1}-1}
  \left( R_{-\gamma}\left( \mathfrak{a} _2=\mathfrak{a}_2^{(0)} \right) +\sum_{\text{real}/\text{img} \ j}{R_{-\gamma j}}E_1\begin{bmatrix}
    -1\\
    \frac{a_{2}^{\left( -\gamma j \right)}}{a_2^{(0)}}q^{\pm \frac{1}{2}}  
  \end{bmatrix} \right) \nonumber \ ,
\end{align}
where $a_2^{(0)}$ is a generic reference value, for example, $a_2^{(0)} = b^3$. In the above, we have used the poles and residues in Table \ref{poles-residues-SO(5)}. Putting all the contributions together, we deduce that for $n_1 > 0, n_2 \ge 0$,
\begin{align}
  & \ \oint\prod_{i = }^{2}\frac{da_i}{2\pi i a_i}a_1^{n_1}a_2^{n_2} \mathcal{Z}\\
  = & \ \delta_{n_2 = 0}\frac{1}{4}\frac{\vartheta_4(\mathfrak{b})}{\vartheta_1(2 \mathfrak{b})}
  \left(
  \frac{i \eta(\tau)^3 \vartheta_4(\mathfrak{b})}{2 \vartheta_1(2 \mathfrak{b})^2} + \frac{\vartheta_4(3 \mathfrak{b})}{\vartheta_1(4 \mathfrak{b})}
  E_1 \begin{bmatrix}
    1 \\ b^2  
  \end{bmatrix}
  \right)
  \frac{b^{n_1} - b^{-n_1}}{q^{n_1/2} - q^{- n_1/2}} \\
  & \ - \delta_{n_2 > 0}\frac{\vartheta_4(\mathfrak{b}) \vartheta_4(3 \mathfrak{b})}{8 \vartheta_1(2 \mathfrak{b}) \vartheta_1(4 \mathfrak{b})} \frac{(b^{n_1} - b^{-n_1})(b^{2n_2} - b^{-2n_2})}{(q^{n_1/2} - q^{- n_1/2})(1 - q^{-n_2})} \\
  & \ + \delta_{n_1 \ne n_2}\sum_{\beta \gamma = \pm}\frac{b^{n_1 \gamma} q^{\frac{1}{2}n_1}}{q^{n_1} - 1}  \sum_{\text{real/img} \ j} R_{\beta \gamma j} \frac{(a_2^{(\beta \gamma j)} q^{\pm \frac{1}{2}})^{n_2 + n_1 \beta}}{q^{\frac{1}{2}(n_2 + n_1 \beta)} - q^{-\frac{1}{2}(n_2 + n_1 \beta)} }\\
  & \ + \delta _{n_2=n_1}\sum_{\gamma =\pm}{\left[ \sum_{\text{real}/\text{img}\ j}{R_{+\gamma j}\frac{( a_{2}^{\left( +\gamma j \right)})^{2n_1} q^{\pm \frac{1}{2}2n_1}}{q^{n_1}-q^{-n_1}}} \right]}\frac{b^{n_1\gamma}q^{\frac{1}{2}n_1}}{q^{n_1}-1}\\
    & \ -\delta _{n_2=n_1}\sum_{\gamma =\pm}
    \frac{b^{n_1\gamma}q^{\frac{1}{2}n_1}}{q^{n_1}-1}
    \left( R_{-\gamma}\left( \mathfrak{a} _2=\mathfrak{a}_2^{(0)} \right) +\sum_{\text{real}/\text{img} \ j}{R_{-\gamma j}}E_1\begin{bmatrix}
      -1\\
      \frac{a_{2}^{\left( -\gamma j \right)}}{a_2^{(0)}}q^{\pm \frac{1}{2}}  
    \end{bmatrix} \right) \ .
\end{align}
The Wilson index corresponding to the $SO(5)$ representations with Dynkin labels $[n, 0]$ can be computed using the above integration formula by simple substitution, sine the corresponding character can be written as a sum of simple monomials,
\begin{align}
  \chi_{[n,0]}\left(a_1,a_2\right)
  = & \ \sum_{m=0}^n \sum_{j=0}^{m} \sum_{i=0}^{m}
  a_1^{j-i} a_2^{i+j-m} \\
  = & \ \lceil\frac{n+1}{2}\rceil+\sum_{m=0}^n\sum_{\substack{i=0\\i\neq m/2}}^m a_2^{2i-m}+\sum_{m=1}^n \sum_{\substack{i,j = 0 \\ i\ne j}}^n a_2^{i+j-m}a_1^{j-i} \\
  \sim & \ \lceil\frac{n+1}{2}\rceil+\sum_{m=0}^n\sum_{\substack{i=0\\i\neq m/2}}^m a_1^{|2i-m|}+\sum_{m=1}^n \sum_{\substack{i,j = 0 \\ i\ne j}}^n a_2^{|i+j-m|}a_1^{|j-i|} \ .
\end{align}
Here in the last line we have rewritten the expression using the symmetries $a_1 \leftrightarrow a_2$, $a_i \leftrightarrow a_i^{-1}$ of the integral, so that each term can be easily computed with the above integration formula. Although the Wilson line index can be computed straightforwardly simply by substitution, we are unfortunately unable to reorganize the final result in an elegant form, so we will refrain from presenting the final expression of $\langle W_{[n,0]}\rangle_{\mathcal{N} = 4 \ SO(5)}$ here.


% \YP{to be continued}
% \YW{For an $SO(5)$ irreducible representation with Dynkin label $[n_1,n_2]$, the character can be written as
% \begin{align}
% 	\chi_{[n_1,n_2]}(a_1,a_2)=\frac{\left|
% 		\begin{array}{cc}
% 			a_1^{n_1+\frac{n_2}{2}+\frac{3}{2}}-a_1^{-n_1-\frac{n_2}{2}-\frac{3}{2}} & a_2^{n_1+\frac{n_2}{2}+\frac{3}{2}}-a_2^{-n_1-\frac{n_2}{2}-\frac{3}{2}} \\
% 			a_1^{\frac{n_2}{2}+\frac{1}{2}}-a_1^{-\frac{n_2}{2}-\frac{1}{2}} & a_2^{\frac{n_2}{2}+\frac{1}{2}}-a_2^{-\frac{n_2}{2}-\frac{1}{2}} \\
% 		\end{array}
% 		\right|}{\left|
% 		\begin{array}{cc}
% 			a_1^{\frac{3}{2}}-a_1^{-\frac{3}{2}} & a_2^{\frac{3}{2}}-a_2^{-\frac{3}{2}} \\
% 			a_1^{\frac{1}{2}}-a_1^{-\frac{1}{2}} & a_2^{\frac{1}{2}}-a_2^{-\frac{1}{2}} \\
% 		\end{array}
% 		\right|}.
% \end{align}
% We shall concentrate on the case when $n_1=n$, $n_2=0$ first. In this case the character can be recast into:
% \begin{align}\label{character resummation}
% &\chi_{[n,0]}\left(a_1,a_2\right)=\sum_{m=0}^n \sum_{j=0}^{m} \sum_{i=0}^{m}a_2^{i+j-m}a_1^{j-i}\notag\\
% &=\sum_{m=0}^n\sum_{i=0}^m a_2^{2i-m}+\sum_{m=1}^n \sum_{m\geq i>j\geq 0}a_2^{i+j-m}a_1^{j-i}+\sum_{m=1}^n\sum_{m\geq j>i\geq 0}a_2^{i+j-m}a_1^{j-i}\notag\\
% &=\lceil\frac{n+1}{2}\rceil+\sum_{m=0}^n\sum_{\substack{i=0\\i\neq m/2}}^m a_2^{2i-m}+\sum_{m=1}^n \sum_{m\geq i>j\geq 0}a_2^{i+j-m}a_1^{j-i}+\sum_{m=1}^n\sum_{m\geq j>i\geq 0}a_2^{i+j-m}a_1^{j-i}
% \end{align}
% For the second term, we have to deal with the following integral:
% \begin{align}
% \sum_{m=0}^{n}\sum_{\substack{i=0\\ i\neq m/2}}^m \oint_{|a_1|=1}\oint_{|a_2|=1}\frac{da_1}{2\pi i a_1}\frac{da_2}{2\pi i a_2}a_2^{2i-m}\mathcal{Z}(a_1,a_2).
% \end{align}
% Performing the $a_2$ integral by using (\ref{a1integral}) it gives:
% \begin{align}
% \sum_{m=0}^{n}\sum_{\substack{i=0\\ i\neq m/2}}^m \oint_{|a_1|=1}\frac{da_1}{2\pi i a_1}\left(-\sum_{\alpha=\pm}R_{\alpha}\frac{b^{\alpha(2i-m)}q^{(2i-m)/2}}{q^{2i-m}-1}-\sum_{\beta\gamma=\pm}R_{\beta\gamma}\frac{a_1^{\beta(2i-m)}b^{\gamma(2i-m)}q^{(2i-m)/2}}{q^{2i-m}-1}\right)
% \end{align}
% Note that
% \begin{align}
% \sum_{m=0}^{n}\sum_{\substack{i=0\\ i\neq m/2}}^m f(2i-m)=\sum_{i=1}^n\left(f(i)+f(-i)\right)\lceil\frac{n-i+1}{2}\rceil,
% \end{align}
% the result can be recast into:
% \begin{align}
% -\sum_{i=1}^n \sum_{\alpha=\pm}2\lceil\frac{n-i+1}{2}\rceil\frac{b^{\alpha i}}{q^{i/2}-q^{-i/2}}\oint \frac{da_1}{2\pi ia_1}R_\alpha -\sum_{i=1}^n \sum_{\beta,\gamma=\pm}2\lceil\frac{n-i+1}{2}\rceil\frac{b^{\gamma i}}{q^{i/2}-q^{-i/2}}\oint \frac{da_1}{2\pi i a_1}R_{\beta\gamma}a_1^{\beta i}
% \end{align}
% Recall the integral formula (\ref{Rintegrate}) and (\ref{RalphabetaIntegrate}), we can obtain:
% \begin{align}\label{part1SO5SYM}
% &= \sum_{i=1}^n\sum_{\alpha,\delta}\lceil\frac{n-i+1}{2}\rceil\frac{b^{\alpha i}\delta}{4(q^{i/2}-q^{-i/2})}Q E_1\begin{bmatrix}
% -1\\
% b^{2\alpha\delta}q^{1/2}
% \end{bmatrix}\notag\\
% &-\sum_{\beta\gamma}\sum_{i=1}^n \sum_{\text{real/Imag}j}2\lceil\frac{n-i+1}{2}\rceil \frac{\beta b^{\gamma i}}{\left(q^{i/2}-q^{-i/2}\right)^2} R_{\beta\gamma j}\left(a_2^{(\beta\gamma j)}q^{\pm 1/2}\right)^{\beta i},
% \end{align}
% where $Q=\frac{\vartheta_4\left(3\mathfrak{b},q\right)\vartheta_4\left(\mathfrak{b},q\right)}{\vartheta_1\left(2\mathfrak{b},q\right)\vartheta_1\left(4\mathfrak{b},q\right)}$.

% Since the last two terms in (\ref{character resummation}) gives the same result after integration, we only need to compute
% \begin{align}
% \sum_{m=1}^n \sum_{0\leq j<i\leq m}\oint \frac{da_1}{2\pi ia_1}\frac{da_2}{2\pi ia_2}a_2^{i+j-m}a_1^{j-i}\mathcal{Z}(a_1,a_2).
% \end{align}
% Using (\ref{a1integral}), we can perform the $a_1$ integral:
% \begin{align}
% &=-\sum_{m=1}^n \sum_{0\leq j<i\leq m}\sum_{\alpha}\frac{b^{\alpha(j-i)}q^{(j-i)/2}}{q^{j-i}-1}\oint\frac{da_2}{2\pi i a_2}a_2^{i+j-m}R_\alpha\notag\\
% &-\sum_{m=1}^n \sum_{0\leq j<i\leq m} \sum_{\beta\gamma}\frac{b^{\gamma(j-i)}q^{(j-i)/2}}{q^{j-i}-1}\oint\frac{da_2}{2\pi i a_2}a_2^{i+j+\beta(j-i)-m}R_{\beta\gamma}
% \end{align}
% The first term from above equals to:
% \begin{align}\label{part2SO5SYM}
% &=\frac{1}{8}Q\sum_{m=1}^n \sum_{\substack{0\leq j<i\leq m\\i+j=m}}\sum_{\alpha\delta}\frac{\delta b^{\alpha(j-i)}}{q^{(j-i)/2}-q^{-(j-i)/2}}E_1\begin{bmatrix}
% -1\\
% b^{2\alpha\delta}q^{1/2}
% \end{bmatrix}\\
% &+\frac{1}{8}Q\sum_{m=1}^n \sum_{\substack{0\leq j<i\leq m\\i+j\neq m}}\sum_{\alpha\delta}\frac{\delta b^{\alpha(j-i)}}{q^{(j-i)/2}-q^{-(j-i)/2}}\frac{(b^{2\alpha\delta}q^{1/2})^{i+j-m}}{q^{(i+j-m)/2}-q^{-(i+j-m)/2}}
% \end{align}
% The second term gives:
% \begin{align}\label{part3SO5SYM}
% &-\sum_{m=1}^n \sum_{0\leq j<i\leq m} \sum_{\gamma}\frac{b^{\gamma(j-i)}q^{(j-i)/2}}{q^{j-i}-1}\left(\oint\frac{da_2}{2\pi i a_2}a_2^{2j-m}R_{+1\gamma}+\oint\frac{da_2}{2\pi i a_2}a_2^{2i-m}R_{-1\gamma}\right)\\
% &=-\sum_{m=1}^n \sum_{\substack{0\leq j<i\leq m\\j= m/2}}\sum_{\gamma}\frac{b^{\gamma(j-i)}q^{(j-i)/2}}{q^{j-i}-1}\sum_{\substack{\text{Real/Imag}\\J}}R_{+1\gamma J}E_1\begin{bmatrix}
% -1\\
% a_2^{(+1\gamma J)}q^{\pm\frac{1}{2}}
% \end{bmatrix}\\
% &-\sum_{m=1}^n \sum_{\substack{0\leq j<i\leq m\\j\neq m/2}}\sum_{\gamma}\frac{b^{\gamma(j-i)}q^{(j-i)/2}}{q^{j-i}-1}\sum_{\substack{\text{Real/Imag}\\J}}R_{+1\gamma J}\frac{\left(a_2^{(+1\gamma J)}q^{\pm 1/2}\right)^{2j-m}}{q^{(2j-m)/2}-q^{-(2j-m)/2}}\\
% &-\sum_{m=1}^n \sum_{\substack{0\leq j<i\leq m\\i= m/2}}\sum_{\gamma}\frac{b^{\gamma(j-i)}q^{(j-i)/2}}{q^{j-i}-1}\sum_{\substack{\text{Real/Imag}\\J}}R_{-1\gamma J}E_1\begin{bmatrix}
% 	-1\\
% 	a_2^{(-1\gamma J)}q^{\pm\frac{1}{2}}
% \end{bmatrix}\\
% &-\sum_{m=1}^n \sum_{\substack{0\leq j<i\leq m\\i\neq m/2}}\sum_{\gamma}\frac{b^{\gamma(j-i)}q^{(j-i)/2}}{q^{j-i}-1}\sum_{\substack{\text{Real/Imag}\\J}}R_{-1\gamma J}\frac{\left(a_2^{(-1\gamma J)}q^{\pm 1/2}\right)^{2i-m}}{q^{(2i-m)/2}-q^{-(2i-m)/2}}
% \end{align}
% Combine three parts (\ref{part1SO5SYM}), (\ref{part2SO5SYM}), and (\ref{part3SO5SYM}) together with the orginal Schur index, we can get the final result.

% Observing that only when $n_1=n\in \mathbb{N}_{>0}$ and $n_2=2m$, $m\in \mathbb{N}_{>0}$, the insertion of $\chi_{[n_1,n_2]}$ gives non-zero result after integration. The result above can certainly cover this case by replacing $n$ with $n+m$.}
% According to the integral formulas as follows
% \begin{align}
% & \oint_{|a_2|=1}\frac{da_2}{2\pi i a_2}a_2^n \mathcal{Z}(a_1,a_2)=-R(a_1,b,q)\frac{b^{n}-b^{-n}}{q^{n/2}-q^{-n/2}}-\sum_{\alpha,\beta=\pm}R_{\alpha\beta}(a_1,b,q)\frac{a_1^{\alpha n}b^{\beta % n}}{q^{n/2}-q^{-n/2}}\\
% & \oint_{|a_2|=1}\frac{da_2}{2\pi i a_2}R(a_2,b,q)=A\left(E_1\begin{bmatrix}
%	-1\\
%	b
%\end{bmatrix}+E_1\begin{bmatrix}
%	-1\\
%	b^3
% \end{bmatrix}\right)\\
% & \oint_{|a_2|=1}\frac{da_2}{2\pi i a_2}a_2^{n}R_{\alpha\beta}(a_2,b,q)=(-1)^n\alpha B\frac{\left( b^{-n\alpha\beta}q^{n/2}-1\right)}{2\left(q^{n/2}-q^{-n/2}\right)}q^{n/2}-(-1)^n \alpha % C\frac{q^{n/2}-b^{-n\alpha\beta}}{2\left(q^{n/2}-q^{-n/2}\right)}q^{n/2}\notag\\
% &-\alpha A\frac{b^{n\alpha\beta}q^{n/2}-b^{-2n\alpha\beta}}{q^{n/2}-q^{-n/2}}q^{n/2}+\alpha D\frac{q^{n/2}-b^{-n\alpha\beta}}{2\left(q^{n/2}-q^{-n/2}\right)}q^{n/2},
% \end{align}
% We have
% \begin{align}
% &\sum_{m=0}^n\sum_{\substack{i=0\\i\neq m/2}}^m\oint_{|a_1|=1}\frac{da_1}{2\pi ia_1}\oint_{|a_2|=1}\frac{da_2}{2\pi i a_2} a_2^{2i-m}\mathcal{Z}(a_1,a_2)\notag\\
% &=\sum_{m=0}^n\sum_{\substack{i=0\\i\neq m/2}}^m\oint_{|a_1|=1}\frac{da_1}{2\pi % ia_1}\left(-R(a_1,b,q)\frac{b^{2i-m}-b^{m-2i}}{q^{(2i-m)/2}-q^{-(2i-m)/2}}-\sum_{\alpha,\beta=\pm}R_{\alpha\beta}(a_1,b,q)\frac{a_1^{\alpha (2i-m)}b^{\beta % (2i-m)}}{q^{(2i-m)/2}-q^{-(2i-m)/2}}\right)\notag\\
% &=-\sum_{m=0}^n\sum_{\substack{i=0\\i\neq m/2}}^m\frac{b^{2i-m}-b^{m-2i}}{q^{(2i-m)/2}-q^{-(2i-m)/2}}A\left(E_1\begin{bmatrix}
%	-1\\
%	b
% \end{bmatrix}+E_1\begin{bmatrix}
%	-1\\
% 	b^3
% \end{bmatrix}\right)\notag\\
% &-\sum_{m=0}^n\sum_{\substack{i=0\\i\neq m/2}}^m \sum_{\alpha,\beta=\pm}\frac{b^{\beta (2i-m)}}{q^{(2i-m)/2}-q^{-(2i-m)/2}}\left((-1)^m\alpha B\frac{\left( % b^{-(2i-m)\beta}q^{\alpha(2i-m)/2}-1\right)}{2\left(q^{\alpha(2i-m)/2}-q^{-\alpha(2i-m)/2}\right)}q^{\alpha(2i-m)/2}\right.\notag\\
% &-(-1)^m \alpha C\frac{q^{\alpha(2i-m)/2}-b^{-(2i-m)\beta}}{2\left(q^{\alpha(2i-m)/2}-q^{-\alpha(2i-m)/2}\right)}q^{\alpha(2i-m)/2}-\alpha %A\frac{b^{(2i-m)\beta}q^{\alpha(2i-m)/2}-b^{-2(2i-m)\beta}}{q^{\alpha(2i-m)/2}-q^{-\alpha(2i-m)/2}}q^{\alpha(2i-m)/2}\notag\\
% &\left.+\alpha D\frac{q^{\alpha(2i-m)/2}-b^{-(2i-m)\beta}}{2\left(q^{\alpha(2i-m)/2}-q^{-\alpha(2i-m)/2}\right)}q^{\alpha(2i-m)/2}\right)
%\end{align}
%where:
%\begin{align}
%	A=\frac{\vartheta_4\left(3\mathfrak{b},q\right)\vartheta_4\left(\mathfrak{b},q\right)}{\vartheta_1\left(2\mathfrak{b},q\right)\vartheta_1\left(4\mathfrak{b},q\right)}\quad %B=\frac{\vartheta_2\left(0,q\right)\vartheta_4^2\left(\mathfrak{b},q\right)}{\vartheta_1^2\left(2\mathfrak{b},q\right)\vartheta_2\left(2\mathfrak{b},q\right)}\quad %C=\frac{\vartheta_3\left(0,q\right)\vartheta_4^2\left(\mathfrak{b},q\right)}{\vartheta_1^2\left(2\mathfrak{b},q\right)\vartheta_3\left(2\mathfrak{b},q\right)}\quad       %D=\frac{\vartheta_4\left(0,q\right)\vartheta_4^2\left(\mathfrak{b},q\right)}{\vartheta_1^2\left(2\mathfrak{b},q\right)\vartheta_4\left(2\mathfrak{b},q\right)}
%\end{align}
%To simplify a little bit, note that
%\begin{align}
%\sum_{m=0}^n \sum_{\substack{i=0\\ i\neq m/2}}^m f(2i-m)=\sum_{i=1}^{n}\left(f(i)+f(-i)\right)\lceil\frac{-i+n+1}{2}\rceil.
%\end{align}
%In this way,
% \begin{align}
% &\sum_{m=0}^n\sum_{\substack{i=0\\i\neq m/2}}^m\oint_{|a_1|=1}\frac{da_1}{2\pi ia_1}\oint_{|a_2|=1}\frac{da_2}{2\pi i a_2} a_2^{2i-m}\mathcal{Z}(a_1,a_2)\notag\\
% &=-2A\left(E_1\begin{bmatrix}
% -1\\
% b
% \end{bmatrix}+E_1\begin{bmatrix}
% -1\\
% b^3
% \end{bmatrix}\right)\sum_{i=1}^n \lceil\frac{n-i+1}{2}\rceil\frac{b^i-b^{-i}}{q^i-q^{-i}}\notag\\
% &-2\sum_{i=1}^n\lceil\frac{n-i+1}{2}\rceil\left(A\left(\frac{-(q^i+q^{-i})(b^{2i}+b^{-2i})+(b^{-i}+b^{i})(q^{i/2}+q^{-i/2})}{(q^{-i/2}-q^{i/2})^2}\right)\right.\notag\\
% &-(-1)^i B \left(\frac{-2(q^i+q^{-i})+(b^i+b^{-i})(q^{i/2}+q^{-i/2})}{2(-q^{i/2}+q^{-i/2})^2}\right)\notag\\
% &-(-1)^i C\frac{-2(q^{i/2}+q^{-i/2})+(q^i+q^{-i})(b^i+b^{-i}) }{2(-q^{i/2}+q^{-i/2})^2}\notag\\
% &\left. +D\frac{-2(q^{i/2}+q^{-i/2})+(b^i+b^{-i})(q^i+q^{-i})}{2(-q^{i/2}+q^{-i/2})^2}\right)\notag\\
% &=-2A\left(E_1\begin{bmatrix}
%	-1\\
%	b
%\end{bmatrix}+E_1\begin{bmatrix}
%	-1\\
%	b^3
% \end{bmatrix}\right)\sum_{i=1}^n \lceil\frac{n-i+1}{2}\rceil\frac{b^i-b^{-i}}{q^i-q^{-i}}\notag\\
% &+16\sum_{i=1}^n \sum_{\beta,\gamma=\pm}\sum_{\text{Real}j}\lceil\frac{n-i+1}{2}\rceil\beta R_{\beta\gamma j}\left(a_2^{(\beta\gamma j)}q\right)^{\beta i}\notag\\
% &+16\sum_{i=1}^n \sum_{\beta,\gamma=\pm}\sum_{\text{Imag}j}\lceil\frac{n-i+1}{2}\rceil\beta R_{\beta\gamma j}\left(a_2^{(\beta\gamma j)}\right)^{\beta i}.
%\end{align}
%Since the last two terms in (\ref{character resummation}) have the same contribution to the integral, we only need to deal with the integral as follows:
%\begin{align}
%\sum_{m=1}^{n}\sum_{m\geq i> j\geq 0}\oint \frac{da}{2\pi i a}a_2^{i+j-m}a_1^{j-i}\mathcal{Z}(a_1,a_2)
%\end{align}}















\subsection{Wilson index $\cal{W}$  of the IDPs}
Now that we have demonstrated an approximate universal description of the IDPs in terms of the HPS model, we want to demonstrate the sequence specific features those make each IDP distinct and may exhibit very different behaviors from their homopolymer counterparts. Evidently, the charges present along the sequence play a crucial role in shaping the structure and dynamics of the IDPs. Out of  20 amino acids, only five of them carry a charge. Specifically, in the HPS model, ``R" and ``K" each has a $+1$ charge, ``H'' possesses a $+0.5$ charge, and ``D'' and ``E'' have charge of $-1$. However, the charges are randomly distributed along the chain backbone and therefore, IDPs in general can be classified either as PA or PE~\cite{Pappu2013,Pappu2014}. The abundance of highly charged amino acids directly contributes to the formation of IDPs inhibiting the establishment of stable three-dimensional structures. Extensive research has been conducted to study this phenomenon in the existing literature. Nonetheless, our study delves deeper to investigate the positional implications of the amino acid sequence in terms of Wilson index ($\cal{W}$) as described below.
\par
Unlike a homopolymer, an IDP can have varied degree of local stiffness and flexibility resulting the amino acids in different segments interacting with neighboring sequences even if they are far apart in the sequence space.
To capture these potential dynamical interactions, we employ the concept of Wilson Renormalization extensively used to study
the spin systems~\cite{SK-Ma}. This renormalization approach allows us to analyze the sequence of charges and their unfolding interactions, considering interactions up to the next nearest neighbor. Fig.~\ref{Wilson_schematics} illustrates a hypothetical example of a short IDP sequence ``ESRKRT'' of length 6, showcasing the presence of a negative charge at the beginning followed by three positive charges in the middle, and the remaining amino acids being neutral. To initiate the averaging procedure, we select a window of length n. The simplest case $\tilde{\mathcal{W}}(2)$, considers sliding averages of window length n=2 and denotes the next neighbor interactions. The window length for averaging can range from 2 to $N$, where $N$ represents the number of amino acids in the IDP sequence.
%%%%%%%%%%%%%%%%%%%%%%%%%%%%%%%%%%%%%%%%%%%%%
% Figure environment removed
%%%%%%%%%%%%%%%%%%%%%%%%%%%%%%%%%%%%%%%%%%%
For the general case of $\tilde{\mathcal{W}}(n)$, where n consecutive charges are averaged, the process begins by sliding an averaging window from one end of the sequence toward the other. After the first step of averaging, denoted as S1, we obtain a new sequence of length $N-n+1$, and use the new sequence to carry on the averaging procedure as, 
%%%%%%%%%%%%%%%%%%%%%%%%%%%%%%%%%%%%%%%%%%%%%
% Figure environment removed
%%%%%%%%%%%%%%%%%%%%%%%%%%%%%%%%%%%%%%%%%%%
\begin{subequations}
\begin{equation} 
S0:  \left[a_1, a_2, \dots , a_N \right] 
\end{equation} 
\begin{equation} 
S1:  \left[\frac{1}{n}\sum_{i=1}^{n} a_i,\; \frac{1}{n}\sum_{i=2}^{n+1} a_i, \dots, \; \frac{1}{n}\sum_{i=N-n+1}^{N} a_i\right]  
\end{equation} 
\begin{equation} 
SN:  \tilde{\mathcal{W}}(n) = \frac{1}{n}\sum_{i=1}^{n} a_i.
\end{equation} 
\end{subequations} 
We continue this procedure iteratively until we reach the final average value, represented as $\tilde{\mathcal{W}}(n)$. If the length of the charge sequence becomes less than the window length $n$ during the averaging process, we terminate the procedure and calculate a global average to obtain $\tilde{\mathcal{W}}(n)$.
\par
These averaging procedure with varied window size $n \in [2,N]$ can effectively capture the combination of charge interactions at different length scale. One can show that $\tilde{\mathcal{W}}(2)$ consider binomial interactions among the charges and expressed as 
\begin{align}
\tilde{\mathcal{W}}(n=2) = \frac{1}{2^{N-1}} \sum_{m=0}^{N-1} \binom{N-1}m a_m.
\end{align} 
The higher-order window averaging considers interactions of varying magnitudes, which can have an impact on determining dynamic conformations of IDPs. In Fig.~\ref{wilson_chg} we explore $\tilde{\mathcal{W}}(n)$ for 12 IDPs with different total charges from highly positive (a) K32 (Q=22.5), (g) CspTm (Q=-2.0) to highly negatively charged IDP (l) ProTaN (Q=-43.0) as a function of normalized window length n/N. In the case of highly positively and negatively charged IDPs, the Wilson curves consistently remain above or below the zero line respectively. However, for IDPs with lower net charges, we sometimes observe the Wilson curve crossing from negative to positive in the case of (f) R15, (g) CspTm, and from positive to negative in the case of (j) $\alpha$-synuclein, and (K) OPN. The area under the Wilson curve is denoted by $\mathcal{W}=\sum_{n=2}^{N} \tilde{\mathcal{W}}(n)$ and listed in the 16$^{th}$ column of Table.-~\ref{Table}. It is conceivable that when plotted in normalized unit length scale IDPs with similar Wilson charge $\tilde{\mathcal{W}}(n)$ will behave the same way and thus, can be used as their fingerprints.\par
%%%%%%%%%%%%%%%%%%%%%%%%%%%%%%%%%%%%%%%%%%%%
%\input{lp.tex}
\subsection{Charge patches and the local persistence length}
Presence of charge patches introduce varying degree of local stiffness along the chain backbone. During the BD simulation 
use a discrete chain and the persistence length is calculated from~\cite{Rubinstein}
\begin{equation}
  \ell_p/\sigma = -\frac{1}{\ln\left(\cos\theta_i\right)},
\label{lp_sim}
  \end{equation}
  where $\theta_i$ is the angle is the angle between two bond vectors connecting the $i^{th}$ bead to the $(i\pm1)^{th}$ beads~\cite{Universal2}. We have checked that for a homopolymer chain this matches well with the continuum description of persistence length~\cite{Landau}
\begin{equation}
\ell_p/\sigma = \kappa/k_BT \quad {\rm (3D)}.
    \label{lp_3d}
  \end{equation}
IDPs with very similar net charge can have markedly different distribution of charges. An IDP containing correlated charge patches will have increased chain stiffness along that region that will affect its conformations and dynamics.
To demonstrate this, we calculate the local persistence length ($l_p$)
along the chain using Eqn.~\ref{lp_sim} for a few IDPs shown in
Fig.~\ref{lp_IDP}. For example, CspTm has sparsely distributed charged
residues with less net charge compared to ProTa-N containing mostly negatively charged residues in patches, and we observe increase in $l_p$ on those regions. The electrostatic repulsion among the same charge residues make the chain locally stiffer and possibly has a deeper effect in their participation in  biophysical processes.
%%%%%%%%%%%%%%%%%%%%%%%%%%%%%%%%%%%%%%%%%%%
% Figure environment removed
%%%%%%%%%%%%%%%%%%%%%%%%%%%%%%%%%%%%%%%%%%%
%\input{skew.tex}
\subsection{Skewness factor ($\mathcal{S}$) of the radius of gyration}
The variation of the chain persistence length due to different charge species along the chain backbone is manifested in the shapes of the corresponding gyration radii that we measure in terms of a skewness factor $\mathcal{S}$.  The skewness parameter $\mathcal{S} $ is obtained by fitting $P\left(\bar{R}_g\right)$ with the exponentially-modified Gaussian distribution (exponnorm)~\cite{exponnorm} given by 
\begin{equation}
f(x,K) = \frac{1}{2K}\exp\left(\frac{1}{2K^2}-\frac{x}{K} \right)\rm{erfc}\left(-\frac{x-1/k}{\sqrt{2}} \right)
\end{equation}
where $x$ is a real number, $K > 0$ and erfc is the
complementary error function. 
The skewness parameter $\mathcal{S}$ can be obtained as 
\begin{equation}
\mathcal{S} = 2 l^3 \frac{K+2}{(K+3)\sqrt{K}},
\end{equation}
where $l$ and $K$ correspond to the shape parameter, and the scale parameter of the exponnorm distribution. We observe that the shapes of the distribution of the gyration radii vary from being near Gaussian to exponentially modified Gaussian distribution (that exhibits a tail) as shown in Fig.~\ref{skewness} where we 
%%%%%%%%%%%%%%%%%%%%%%%%%%%%%%%%%%%%%%%%%%%
% Figure environment removed
%%%%%%%%%%%%%%%%%%%%%%%%%%%%%%%%%%%%%%%%%%%
% Figure environment removed
%%%%%%%%%%%%%%%%%%%%%%%%%%%%%%%%%%%%%%%%%%% 
plot the distribution $P\left(\bar{R}_g\right) $ of scaled radius of gyration radii $\bar{R}_g= \sqrt{R_g^2}/ \langle \sqrt{R_g^2}\rangle $. By fitting these histograms with exponentially modified Gaussian distributions shown as the colored solid lines in Fig.~\ref{skewness}, we find the skewness $\mathcal{S}$ for the each case. Fig.~\ref{skewness} confirms that most of the IDPs have long exponential tails ($\cal{S}\ge$ 1.0) such as CspTm, FhuA, hCyp, $\alpha$-synuclein, sNase, ACTR, K32, OPN, and SH4UD.
A few IDPs CoINT, ProTa-C, and HST5 have distribution shaped near Gaussian. 
Finally the highly charged IDP ProTa-N is observed to have almost perfect Gaussian distribution with $\cal{S} \le$ 0.5. Moreover, for the highly skewed distributions the peaks shift towards the left that signifies the median is smaller than the mean and there is a propensity of these IDPs to expand occasionally. This skewness parameter can be utilized a classifier to segregate the IDPs into three categories that characterize the propensity of expansion.
\par
A pertinent question is if the skewness is correlated to the charge content of the IDPs. In a previous study, Pappu {\em et al.}~\cite{Pappu2013} demonstrated that radius of gyration depends on the charge asymmetry parameter $f^{*} = \frac{(f^+-f^-)^2}{f^++f^-}$, where $f^+/f^-$ is the net positive/negative charge per residue of an IDP. Here we demonstrate that the charge asymmetry parameter $f^{*}$ correlates inversely with the skewness parameter  $\mathcal{S}$ shown in Fig.~\ref{skewness}(n). A high charge asymmetry uniformly extends the polymer leading to a lower value of $\mathcal{S}$. In particular for ProtaN which has the largest value of $f^*$, the skewness parameter $\mathcal{S} \rightarrow 0$. 
%%%%%%%%%%%%%%%%%%%%%%%%%%%%%%%%%%%%%%%%%%%%
%\input{salt.tex}
\subsection{Chain conformations and Ionic Concentration}
Solvent conditions, such as pH, temperature,
ionic strength, and the presence of specific molecules, can
significantly influence the conformations and hence the behaviors of
the IDPs, particularly in a cellular environment. The robustness of
the IDPs under external conditions can also be associated with the
%%%%%%%%%%%%%%%%%%%%%%%%%%%%%%%%%%%%%%%%%%%
% Figure environment removed
%%%%%%%%%%%%%%%%%%%%%%%%%%%%%%%%%%%%%%%%%%%
origin of life. Previous experimental studies
~\cite{Ueda2010,Schuler2010,Hoffmann} and simulation studies using CG
models~\cite{Reddy,Wohl} have revealed 
conformational changes and salt-induced phase transition and looked at
the liquid-liquid phase transitions in IDPs. IDPs are described either
as PEs or PAs with varying amounts of net charge~\cite{Pappu2010}. Thus, it is
conceivable that screening will affect the conformational aspects in a
significant way. Intuitively one can understand the behavior by using the idea of screening. The IDPs those are PE, an increase in salt
concentration will screen the net charge reducing the electrostatic
repulsion and hence by an large, all the PEs with a net positive or 
negative charge will have reduced gyration radii as a function of
increased screening. The case of PAs is a bit more subtle depending on an IDP's not only the net charge per residue $q_{net}$, but the fraction of the residues that are charged $q_{abs}$
as defined below.
\begin{equation}
  q_{net} = \frac{||Q_{+}|-|Q_{-}||}{N} \quad {\rm and} \quad q_{abs} = \frac{|Q_{+}|+|Q_{-}|}{N}
\label{q_definition}
\end{equation} 
Here, $Q_{+}$, $Q_{-}$, and $N$ represent the total positive and negative charges, and the number of amino acids in the IDP.  For the PA the loss/gain in electrostatic energy and entropy ultimately controls the show.
We have made an extensive study of the dependence of gyration radii of the 33 IDPs (listed in Table-~\ref{Table}) on salt concentration) under physiological conditions ranging from 0-300 mM  shown in Fig.~\ref{i_quads}. The IDPs can be placed on any one of the four quadrants (I, II, III, and iV) of $(q_{abs}, q_{net})$  to study their dependency on salt concentration (Fig.~\ref{i_quads}(a)).
Based on the values of $(q_{abs}, q_{net})$, two decision boundary lines $q_{abs}=0.275$ and $q_{net}=0.13$ place the IDPs into four subclasses.  Fig.~\ref{i_quads}(a) displays scatter plots of the 33 IDPs classified into four quadrants, represented by blue, green, orange, and red symbols corresponding to the I, II, III, and IV quadrants, respectively. For each quadrant,
a plot of the saturation values at each concentration is used to plot $\sqrt{R_g^2/R_g^2(0)}-1$ as a function of ionic concentration, where $R_g(0)$ corresponds to the radius of gyration under ion-free conditions.
Quadrant-I and IV are easy to understand. 
A strong PA, such as ProtaN lies in the quadrant-I as expected. But other PAs (OPN, ProTa-C, HST5) with large $q_{net}$ and $q_{abs}$ belong here. In this case, as the salt concentration increases, charge screening occurs, leading to a decrease in their radii of gyrations. This condition holds when only one type of charged residue is abundant in number.
On the other hand, in quadrant IV, $q_{abs}$ is high, but $q_{net}$ is low, corresponding to a situation where there is a higher number of charged residues, yet they are almost equal in numbers.
As both types of charges are present, at low salt concentration, the attraction between opposite charges reduces their radius of gyration due to electrostatic interactions. However, with increasing salt concentration, the charge screening effect comes into play, and the strength of electrostatic attraction among the oppositely charged residues decreases. Consequently, we observe a swelling of the IDPs, leading to a higher radius of gyration. 13 IDPs belong to this category.
In quadrant II and III, the $q_{abs}$ value is low, indicating a low content of charged residues. In quadrant II, we find that p53 is the only IDP out of the 33 that falls into this category but has a high value of $q_{net}$. This pathological case is characterized by having only 17 negative charge residues (GLU and ASP) and two positively charged residues (ARG and LYS). Due to the charge screening effect mostly on the negatively charged residues, it can be inferred that the radius of gyration will decrease and that is indeed true as observed from the plot.
In quadrant III, the $q_{net}$ is low corresponding to IDPs that have less net charge per residue. We find 15 IDPs belong to this category. Out of which gyration radii of hCyp, FhuA, and K10 increase while K19, K18, k27, K17, K32, and K16 decrease as a function of salt concentration. On the other hand, the remaining 5 IDPs namely Protein-L, SH4UD, An16, Nucleoporin153, and CoINT are robust to the variation of ionic concentration as they are mostly low charge containing IDPs. 
The segregation of IDPs into four subsections unravels insights about their responses to salt concentration and provides a framework to classify other unknown proteins based on how they will behave under a wide range of salt solutions.
\par
We further studied the accompanying variation in the shape of the distribution of the gyration radii by monitoring the skewness factor $\mathcal{S}$ as progressively more screening is introduced for the reason discussed in section D. Some examples are shown Fig.~\ref{i_quads}(c)-(h).  The skewness of the distributions for SIC1 for 10 mM, 100 mM, and 300 mM ionic concentration changes from 1.51, 1.75, and 1.24 respectively, and they span a larger conformational space. On the other hand, for ProTa-N  while the gyration radii decrease at higher salt concentrations implying they become more compact without altering the distribution shape. We also observe that gyration radii for a few IDPs 
(Protein-L, SH4UD, An16, Nucleoporin153, and CoINT) in Fig.~\ref{i_quads}(a) remain unaffected within the low salt limit of our study. With the change in the salt concentration, our simulation studies show that the degree of alteration in the shape of IDPs is different. The shape deformation is drastic in SIC1 compared to ProTa-N.
%%%%%%%%%%%%%%%%%%%%%%%%%%%%%%%%%%%%%%%%%%%%
%\section{Conclusion and Future Work}
In this work, I design corruption-robust algorithms for the Lipschitz contextual search problem. I present the \emph{agnostic checking} technique and demonstrate its effectiveness in designing corruption-robust algorithms. There are several open problems for future research. First, in the algorithm I propose for pricing loss, the schedule for agnostic checks is fixed upfront. Can the learner design an adaptive checking schedule for the pricing loss? Second, this work assumes the learner has knowledge of the Lipschitz constant $L$. Can the learner design efficient no-regret algorithms without knowledge of $L$? 
\subsection{Summary and Outlook}
In conclusion, we used two different CG models (HPS \& M3) to study both universal and fine structures of 33 IDPs and compared our results with available experimental results as well as simulation results for the same IDPs using other CG models. Our systematic  studies of a larger set of IDPs with fairly disparate level of absolute and net charge 
($q_{abs}$, $q_{net}$), and net hydropathy add many interesting characteristics to those studied previously using a similar models. Our larger set of IDPs converge on the interaction parameter $\epsilon=0.18$ with some variability. However, this expanded dataset establishes a more robust and reliable framework for studying IDPs in bulk. A natural question that has been addressed in the community that if sequence specificity makes every IDP distinct from each other, or they share some universal characteristics of homopolymers described by Flory's theory. We have been able to address both the issues. A comparison of the scaled end-to-end distance and transverse fluctuations in reference to our recently established universal results, we observe that IDPs studied here are described better by  Gaussian chains rather self-avoiding chains. The experimental results converge well with the simulation results. A soft interaction potential $\epsilon=0.18$ may be responsible for this. \par
We then study in detail how the absolute and net charge per residue ($q_{abs}$, $q_{net}$) manifest themselves in finer characteristics of the IDPs. We come up with several new metrics those reveal the uniqueness of each IDP, yet leaves room for making further classification of IDPs in different categories. The first one is the Wilson index $\mathcal{W}$ that on a normalized unit length scale demonstrates the uniqueness of each IDP
and hence can be used as their fingerprints. Likewise, we demonstrate how the charge patches control the local stiffness and hence the overall conformations of the IDPs that we further characterize by introducing a skewness index $\mathcal{S}$. We find that $\mathcal{S}$ interpolates from a low value to a larger value ($\mathcal{W} \rightarrow 0$ for a Gaussian distribution and large for a distribution with an exponential tail ) and further relate 
$\mathcal{S}$ to the the charge asymmetry parameter $f^*$ introduced by Das and Pappu~\cite{Pappu2013} to note that they are inversely correlated.\par
An important classification of the IDPs emerges from the study of salt dependence of the conformations of the IDPs. We find that IDPs exhibit very different characteristics and can be broadly placed in four different region in the ($q_{abs}, q_{net}$) space.\par
We conclude with some comments which may promote further studies to perfect the CG models. We and many others used isotropic radius of gyration as the sole physical quantity for comparison as this is the mostly available from the experiments.  The CG model can be refined by introducing other quantities.  For example, Wohl {\em et.al}~\cite{Wohl} studied  salting-out effect on the liquid–liquid phase separation (LLPS) of IDPs by introducing a salt-dependent term into the hydropathy used in the HPS model.
Maity {\em et al.}~\cite{Reddy} introduced Molecular Transfer Model to study the Salt-Induced Transitions. In the low concentrations of salts ($\le$ 1M) IDP conformations are affected by the degree of screening of electrostatic interactions of the charged residues and are independent of the specific salt identity which is likely the regime that we have studied. However, at high concentrations, salts affect IDP conformations through salt-specific Hofmeister effects~\cite{Capp2013,Pegram2010}. Thus, our studies will be useful in refining the existing HPS models for a wider range of parameter space.
\par
We note that some of the IDPs remain unaffected with the variation of salt concentration and thus can be compared with the behavior of other simpler amino acids identified and studied in the context of the origin of life~\cite{Origin_life}. 
We believe studies of IDPs using a variety of CG models open up several exciting avenues for future research, allowing for a deeper understanding of the unique properties and behavior of IDPs.
%\begin{table*}[t]
\caption{Summary of the top-performing teams in each track of the RoboDepth Challenge.}
\centering\scalebox{1}{
\begin{tabular}{c|p{5cm}|p{5cm}}
\toprule
\textbf{Rank} & \textbf{\#1: Robust Self-Supervised MDE} & \textbf{\#2: Robust Supervised MDE}
\\\midrule\midrule
\multirow{13}{*}{\textcolor{robo_blue}{\textbf{1st Place}}} & \textbf{Team Name} & \textbf{Team Name}
\\
& \textcolor{robo_blue}{OpenSpaceAI} & \textcolor{robo_blue}{USTCxNetEaseFuxi}
\\
\cmidrule{2-3}
& \textbf{Team Members} & \textbf{Team Members}
\\
& Ruijie Zhu$^1$, Ziyang Song$^1$, Li Liu$^1$, Tianzhu Zhang$^{1,2}$ & Jun Yu$^1$, Mohan Jing$^1$, Pengwei Li$^1$, Xiaohua Qi$^1$, Cheng Jin$^2$, Yingfeng Chen$^2$, Jie Hou$^2$
\\
\cmidrule{2-3}
& \textbf{Affiliations} & \textbf{Affiliations}
\\
& $^1$University of Science and Technology of China, $^2$Deep Space Exploration Lab & $^1$University of Science and Technology of China, $^2$NetEase Fuxi
% \\
% \cmidrule{2-3}
% & \textbf{Approach} & \textbf{Approach}
% \\
% & IRUDepth with MPViT as depth encoder and PoseNet for camera poses and depth maps with AugMix& <...>
\\\cmidrule{2-3}
& \textbf{Contact} $\textrm{\Letter}$ & \textbf{Contact} $\textrm{\Letter}$
\\
& \texttt{ruijiezhu@mail.ustc.edu.cn} & \texttt{USTC\_IAT\_United@163.com}
\\\midrule\midrule
\multirow{17}{*}{\textcolor{robo_red}{\textbf{2nd Place}}} & \textbf{Team Name} & \textbf{Team Name}
\\
& \textcolor{robo_red}{USTC-IAT-United} & \textcolor{robo_red}{OpenSpaceAI}
\\
\cmidrule{2-3}
& \textbf{Team Members} & \textbf{Team Members}
\\
& Jun Yu$^1$, Xiaohua Qi$^1$, Jie Zhang$^2$, Mohan Jing$^1$, Pengwei Li$^1$, Zhen Kan$^1$, Qiang Ling$^1$, Liang Peng$^3$, Minglei Li$^3$, Di Xu$^3$, Changpeng Yang$^3$ & Li Liu$^1$, Ruijie Zhu$^1$, Ziyang Song$^1$, Tianzhu Zhang$^{1,2}$
\\
\cmidrule{2-3}
& \textbf{Affiliations} & \textbf{Affiliations}
\\
& $^1$University of Science and Technology of China, $^2$Central South University, $^3$Huawei Cloud Computing Technology Co., Ltd & $^1$University of Science and Technology of China, $^2$Deep Space Exploration Lab
\\
\cmidrule{2-3}
& \textbf{Contact} $\textrm{\Letter}$ & \textbf{Contact} $\textrm{\Letter}$
\\
& \texttt{USTC\_IAT\_United@163.com} & \texttt{liu\_li@mail.ustc.edu.cn}
\\\midrule\midrule
\multirow{11}{*}{\textcolor{robo_green}{\textbf{3rd Place}}} & \textbf{Team Name} & \textbf{Team Name}
\\
& \textcolor{robo_green}{YYQ} & \textcolor{robo_green}{GANCV}
\\
\cmidrule{2-3}
& \textbf{Team Members} & \textbf{Team Members}
\\
& Yuanqi Yao$^1$, Gang Wu$^1$, Jian Kuai$^1$, Xianming Liu$^1$, Junjun Jiang$^1$ & Jiamian Huang$^1$, Baojun Li$^1$
\\
\cmidrule{2-3}
& \textbf{Affiliations} & \textbf{Affiliations}
\\
& $^1$Harbin Institute of Technology & $^1$Individual Researcher
\\
\cmidrule{2-3}
& \textbf{Contact} $\textrm{\Letter}$ & \textbf{Contact} $\textrm{\Letter}$
\\
& \texttt{yuanqiyao@stu.hit.edu.cn} & \texttt{huang176368745@gmail.com}
\\\bottomrule
\end{tabular}
}
\label{tab:summary}
\end{table*}
\section{Acknowledgments}
All computations were carried out using STOKES High-Performance Computing Cluster at UCF.
\vfill
\begin{thebibliography}{10}

\bibitem{eilers2021product}Eilers, M., Meier, S. \& Müller, P. Product Programs in the Wild: Retrofitting Program Verifiers to Check Information Flow Security. {\em Computer Aided Verification (CAV)}. (2021)
\bibitem{tiwari2009complete}Tiwari, M., Wassel, H., Mazloom, B., Mysore, S., Chong, F. \& Sherwood, T. Complete information flow tracking from the gates up. {\em Proceedings Of The 14th International Conference On Architectural Support For Programming Languages And Operating Systems}. pp. 109-120 (2009)

\bibitem{tiwari2009execution}Tiwari, M., Li, X., Wassel, H., Chong, F. \& Sherwood, T. Execution leases: A hardware-supported mechanism for enforcing strong non-interference. {\em Proceedings Of The 42nd Annual IEEE/ACM International Symposium On Microarchitecture}. pp. 493-504 (2009)

\bibitem{jin2012proof}Jin, Y. \& Makris, Y. Proof carrying-based information flow tracking for data secrecy protection and hardware trust. {\em 2012 IEEE 30th VLSI Test Symposium (VTS)}. pp. 252-257 (2012)

\bibitem{li2011caisson}Li, X., Tiwari, M., Oberg, J., Kashyap, V., Chong, F., Sherwood, T. \& Hardekopf, B. Caisson: A Hardware Description Language for Secure Information Flow. {\em Proceedings Of The 32Nd ACM SIGPLAN Conference On Programming Language Design And Implementation}. pp. 109-120 (2011), http://doi.acm.org/10.1145/1993498.1993512

\bibitem{li2014sapper}Li, X., Kashyap, V., Oberg, J., Tiwari, M., Rajarathinam, V., Kastner, R., Sherwood, T., Hardekopf, B. \& Chong, F. Sapper: A Language for Hardware-level Security Policy Enforcement. {\em Proceedings Of The 19th International Conference On Architectural Support For Programming Languages And Operating Systems}. pp. 97-112 (2014), http://doi.acm.org/10.1145/2541940.2541947

\bibitem{zhang2015secverilog}Zhang, D., Wang, Y., Suh, G. \& Myers, A. A Hardware Design Language for Timing-Sensitive Information-Flow Security. {\em Proceedings Of The Twentieth International Conference On Architectural Support For Programming Languages And Operating Systems}. pp. 503-516 (2015), http://doi.acm.org/10.1145/2694344.2694372

\bibitem{bidmeshki2015vericoq}Bidmeshki, M. \& Makris, Y. VeriCoq: A Verilog-to-Coq converter for proof-carrying hardware automation. {\em 2015 IEEE International Symposium On Circuits And Systems (ISCAS)}. pp. 29-32 (2015)

\bibitem{hu2016detecting}Hu, W., Mao, B., Oberg, J. \& Kastner, R. Detecting hardware trojans with gate-level information-flow tracking. {\em Computer}. \textbf{49}, 44-52 (2016)

\bibitem{kong2017using}Kong, S., Shen, Y. \& Zhou, H. Using security invariant to verify confidentiality in hardware design. {\em Proceedings Of The On Great Lakes Symposium On VLSI 2017}. pp. 487-490 (2017)

\bibitem{ardeshiricham2017register}Ardeshiricham, A., Hu, W., Marxen, J. \& Kastner, R. Register transfer level information flow tracking for provably secure hardware design. {\em Proceedings Of The Conference On Design, Automation \& Test In Europe (DATE)}. pp. 1695-1700 (2017), http://dl.acm.org/citation.cfm?id=3130379.3130775

\bibitem{ardeshiricham2017clepsydra}Ardeshiricham, A., Hu, W. \& Kastner, R. Clepsydra: Modeling timing flows in hardware designs. {\em 2017 IEEE/ACM International Conference On Computer-Aided Design (ICCAD)}. pp. 147-154 (2017)

\bibitem{deng2017secchisel}Deng, S., Gümüşoğlu, D., Xiong, W., Gener, Y., Demir, O. \& Szefer, J. SecChisel: language and tool for practical and scalable security verification of security-aware hardware architectures. {\em Cryptology EPrint Archive}. (2017)

\bibitem{bidmeshki2017information}Bidmeshki, M., Antonopoulos, A. \& Makris, Y. Information flow tracking in analog/mixed-signal designs through proof-carrying hardware IP. {\em Design, Automation \& Test In Europe Conference \& Exhibition (DATE), 2017}. pp. 1703-1708 (2017)

\bibitem{boraten2018securing}Boraten, T. \& Kodi, A. Securing NoCs against timing attacks with non-interference based adaptive routing. {\em 2018 Twelfth IEEE/ACM International Symposium On Networks-on-Chip (NOCS)}. pp. 1-8 (2018)

\bibitem{pilato2018tainthls}Pilato, C., Wu, K., Garg, S., Karri, R. \& Regazzoni, F. Tainthls: High-level synthesis for dynamic information flow tracking. {\em IEEE Transactions On Computer-Aided Design Of Integrated Circuits And Systems}. \textbf{38}, 798-808 (2018)

\bibitem{zagieboylo2019using}Zagieboylo, D., Suh, G. \& Myers, A. Using information flow to design an ISA that controls timing channels. {\em 2019 IEEE 32nd Computer Security Foundations Symposium (CSF)}. pp. 272-27215 (2019)

\bibitem{pieper2020dynamic}Pieper, P., Herdt, V., Große, D. \& Drechsler, R. Dynamic information flow tracking for embedded binaries using SystemC-based virtual prototypes. {\em 2020 57th ACM/IEEE Design Automation Conference (DAC)}. pp. 1-6 (2020)

\bibitem{restuccia2021aker}Restuccia, F., Meza, A. \& Kastner, R. AKER: A design and verification framework for safe and secure soc access control. {\em IEEE/ACM International Conference On Computer Aided Design (ICCAD)}. (2021), https://par.nsf.gov/servlets/purl/10298115

\bibitem{restuccia2022framework}Restuccia, F., Meza, A., Kastner, R. \& Oberg, J. A Framework for Design, Verification, and Management of SoC Access Control Systems. {\em IEEE Transactions On Computers}. (2022), https://kastner.ucsd.edu/wp-content/uploads/2022/10/admin/tcomputer-aker22.pdf

\bibitem{cherupalli2017software}Cherupalli, H., Duwe, H., Ye, W., Kumar, R. \& Sartori, J. Software-based gate-level information flow security for IoT systems. {\em 50th IEEE/ACM International Symposium On Microarchitecture}. (2017), https://dl.acm.org/doi/pdf/10.1145/3123939.3123955

\bibitem{fadiheh2023exhaustive}Fadiheh, M., Wezel, A., Muller, J., Bormann, J., Ray, S., Fung, J., Mitra, S., Stoffel, D. \& Kunz, W. An Exhaustive Approach to Detecting Transient Execution Side Channels in RTL Designs of Processors. {\em IEEE Transactions On Computers}. \textbf{72}, 222-235 (2023,1)

\bibitem{wu2022exert}Wu, J., Fowze, F. \& Forte, D. EXERT: EXhaustive Integrity Analysis for Information Flow Security. {\em Asian Hardware Oriented Security And Trust Symposium (AsianHOST)}. (2022), https://dforte.ece.ufl.edu/wp-content/uploads/sites/65/2022/09/EXERT%5C_AsianHost.pdf

\bibitem{athalye2022knox}Athalye, A., Kaashoek, M. \& Zeldovich, N. Verifying Hardware Security Modules with Information-Preserving Refinement. {\em OSDI}. (2022)

\bibitem{fowze2022eisec}Fowze, F., Choudhury, M. \& Forte, D. EISec: Exhaustive Information Flow Security of Hardware Intellectual Property Utilizing Symbolic Execution. {\em Asian Hardware Oriented Security And Trust Symposium (AsianHOST)}. (2022)

\bibitem{athalye2019notary}Athalye, A., Belay, A., Kaashoek, M., Morris, R. \& Zeldovich, N. Notary: A Device for Secure Transaction Approval. {\em 27th Symposium On Operating Systems Principles (SOSP)}. (2019), https://doi.org/10.1145/3341301.3359661

\bibitem{meza2023hyperflowgraph}Meza, A. \& Kastner, R. Information Flow Coverage Metrics for Hardware Security Verification.  (2023), arXiv 2304.08263

\bibitem{ryan2023countering}Ryan, K. \& Sturton, C. Countering the Path Explosion Problem in the Symbolic Execution of Hardware Designs.  (2023), arXiv 2304.05445 

\bibitem{dorsey2020intel}Dorsey, V. \& Morhardt, C. Intel Security Development Lifecycle. (Intel,2020)

\bibitem{he2015model}He, S., Roe, N., Wood, E., Nachtigal, N. \& Helms, J. Model of the Product Development Lifecycle. (Sandia National Laboratories,2015)

\bibitem{YangSP2016}Yang, K., Hicks, M., Dong, Q., Austin, T. \& Sylvester, D. A2: Analog Malicious Hardware. {\em 2016 IEEE Symposium On Security And Privacy (SP)}. pp. 18-37 (2016)

\bibitem{or1200}. OpenRISC 1200 Implementation. , https://github.com/openrisc/or1200

\bibitem{msp430}. openMSP430. , https://opencores.org/projects/openmsp430

\bibitem{farzana2019soc}Farzana, N., Rahman, F., Tehranipoor, M. \& Farahmandi, F. SoC Security Verification using Property Checking. {\em 2019 IEEE International Test Conference (ITC)}. pp. 1-10 (2019)

\bibitem{TrustHub2}Farzana, N., Farahmandi, F. \& Tehranipoor, M. SoC Security Properties and Rules. {\em IACR Cryptol. EPrint Arch.}. \textbf{2021} pp. 1014 (2021)

\bibitem{hicks2015specs}Hicks, M., Sturton, C., King, S. \& Smith, J. SPECS: A Lightweight Runtime Mechanism for Protecting Software from Security-Critical Processor Bugs. {\em ASPLOS}. pp. 517-529 (2015)

\bibitem{bilzor2011security}Bilzor, M., Huffmire, T., Irvine, C. \& Levin, T. Security Checkers: Detecting processor malicious inclusions at runtime. {\em HOST}. (2011)

\bibitem{zhang2017scifinder}Zhang, R., Stanley, N., Griggs, C., Chi, A. \& Sturton, C. Identifying Security Critical Properties for the Dynamic Verification of a Processor. {\em ASPLOS}. pp. 541-554 (2017)

\bibitem{zhang2020transys}Zhang, R. \& Sturton, C. Transys: Leveraging Common Security Properties Across Hardware Designs. {\em Proceedings Of The Symposium On Security And Privacy (S\&P)}. (2020)

\bibitem{trippel2020ICAS}Trippel, T., Shin, K., Bush, K. \& Hicks, M. ICAS: an Extensible Framework for Estimating the Susceptibility of IC Layouts to Additive Trojans. {\em 2020 IEEE Symposium On Security And Privacy (SP)}. pp. 1742-1759 (2020)

\bibitem{Deutschbein2022JCEN}Deutschbein, C., Meza, A., Restuccia, F., Kastner, R. \& Sturton, C. Isadora: Automated Information Flow Property Generation for Hardware Security Verification. {\em Journal Of Cryptographic Engineering (JCEN)}. (2022)

\bibitem{zhang2021sidechannel}Zhang, T., Park, J., Tehranipoor, M. \& Farahmandi, F. PSC-TG: RTL Power Side-Channel Leakage Assessment with Test Pattern Generation. {\em 2021 58th ACM/IEEE Design Automation Conference (DAC)}. pp. 709-714 (2021)

\bibitem{torlak2014rosette}Torlak, E. \& Bodik, R. A Lightweight Symbolic Virtual Machine for Solver-Aided Host Languages. {\em Proceedings Of The 35th ACM SIGPLAN Conference On Programming Language Design And Implementation}. pp. 530-541 (2014), https://doi.org/10.1145/2594291.2594340

\bibitem{cha2012mayhem}Cha, S., Avgerinos, T., Rebert, A. \& Brumley, D. Unleashing Mayhem on Binary Code. {\em Proceedings Of The 2012 IEEE Symposium On Security And Privacy}. pp. 380-394 (2012)

\bibitem{bao2021symbolic}Bao, Q., Wang, Z., Li, X., Larus, J. \& Wu, D. Abacus: Precise side-channel analysis. {\em International Conference On Software Engineering (ICSE)}. pp. 797-809 (2021)

\bibitem{wang2017cached}Wang, S., Wang, P., Liu, X., Zhang, D. \& Wu, D. CacheD: Identifying cache-based timing channels in production software. {\em USENIX Security Symposium}. pp. 235-252 (2017)

\bibitem{wang2019identifying}Wang, S., Bao, Y., Liu, X., Wang, P., Zhang, D. \& Wu, D. Identifying Cache-Based Side Channels through Secret-Augmented Abstract Interpretation. {\em 28th USENIX Security Symposium (USENIX Security 19)}. pp. 657-674 (2019,8), https://www.usenix.org/conference/usenixsecurity19/presentation/wang-shuai

\bibitem{brotzman2019casym}Brotzman, R., Liu, S., Zhang, D., Tan, G. \& Kandemir, M. Casym: Cache aware symbolic execution for side channel detection and mitigation. {\em Symposium On Security And Privacy (SP)}. (2019)

\bibitem{guarnier2020spectector}Guarnieri, M., Köpf, B., Morales, J., Reineke, J. \& Sánchez, A. Spectector: Principled Detection of Speculative Information Flows. {\em 2020 IEEE Symposium On Security And Privacy (SP)}. pp. 1-19 (2020)

\bibitem{avgerinos2014automatic}Avgerinos, T., Cha, S., Rebert, A., Schwartz, E., Woo, M. \& Brumley, D. Automatic exploit generation. {\em Communications Of The ACM}. \textbf{57}, 74-84 (2014)

\bibitem{avgerinos2011automatic}Avgerinos, T., Hao, B. \& Brumley, D. Automatic exploit generation. {\em Network And Distributed System Security Symposium (NDSS)}. (2011)

\bibitem{renzelmann2012symdrive}Renzelmann, M., Kadav, A. \& Swift, M. SymDrive: Testing Drivers without Devices. {\em 10th USENIX Symposium On Operating Systems Design And Implementation}. (2012), https://www.usenix.org/conference/osdi12/technical-sessions/presentation/renzelmann

\bibitem{zhang2018end}Zhang, R., Deutschbein, C., Huang, P. \& Sturton, C. End-to-End Automated Exploit Generation for Validating the Security of Processor Designs. {\em Proceedings Of The International Symposium On Microarchitecture (MICRO)}. (2018)

\bibitem{Shen2018SymbolicEB}Shen, L., Mu, D., Cao, G., Qin, M., Blackstone, J. \& Kastner, R. Symbolic execution based test-patterns generation algorithm for hardware Trojan detection. {\em Comput. Secur.}. \textbf{78} pp. 267-280 (2018)

\bibitem{clarkson2010hyperproperties}Clarkson, M. \& Schneider, F. Hyperproperties. {\em J. Comput. Secur.}. \textbf{18}, 1157-1210 (2010,9), http://dl.acm.org/citation.cfm?id=1891823.1891830

\bibitem{Kozyri2022expressing}Kozyri, E., Chong, S. \& Myers, A. Expressing Information Flow Properties. {\em Foundations And Trends® In Privacy And Security}. \textbf{3}, 1-102 (2022), http://dx.doi.org/10.1561/3300000008

\bibitem{meza2022safety}Meza, A., Restuccia, F., Kastner, R. \& Oberg, J. Safety verification of third-party hardware modules via information flow tracking. {\em 1st Real-Time Intelligent Edge Computing Workshop (RAGE)}. (2022), https://kastner.ucsd.edu/wp-content/uploads/2022/08/admin/rage22-safety.pdf

\bibitem{Deutschbein2021Isadora}Deutschbein, C., Meza, A., Restuccia, F., Kastner, R. \& Sturton, C. Isadora: Automated Information Flow Property Generation for Hardware Designs. {\em Proceedings Of The Workshop On Attacks And Solutions In Hardware Security (ASHES)}. (2021)

\bibitem{ferraiuolo2017secverilog}Ferraiuolo, A., Xu, R., Zhang, D., Myers, A. \& Suh, G. Verification of a Practical Hardware Security Architecture Through Static Information Flow Analysis. {\em Proceedings Of The Twenty-Second International Conference On Architectural Support For Programming Languages And Operating Systems}. pp. 555-568 (2017), http://doi.acm.org/10.1145/3037697.3037739

\bibitem{ardeshiricham2019verisketch}Ardeshiricham, A., Takashima, Y., Gao, S. \& Kastner, R. VeriSketch: Synthesizing Secure Hardware Designs with Timing-Sensitive Information Flow Properties. {\em Proceedings Of The 2019 ACM SIGSAC Conference On Computer And Communications Security}. pp. 1623-1638 (2019)


\end{thebibliography}

\end{document}
