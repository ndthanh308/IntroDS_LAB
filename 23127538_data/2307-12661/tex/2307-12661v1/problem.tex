%% =============================================================================
\section{Problem formulation}
\label{s:problem}
%% =============================================================================

Continuing with the notations of \secref{s:intro}, let \(\domain\subset\R[\sysDim]\) be a non-empty open set containing \(0\in\R[\sysDim]\), and let \(\vecfld:\domain\lra\R[\sysDim]\) be a continuous vector field for which \(0\) is an isolated equilibrium point. In this section we establish a systematic, algorithmic, and tractable mechanism to search for candidate Lyapunov functions \(\lyapfn : \domain \lra\lcro{0}{+\infty}\) to assess the properties of Lyapunov stability and asymptotic stability of the origin \(0 \in \R[\sysDim]\) for the dynamical system \eqref{e:nonlinear system}.

As discussed in \secref{s:intro:indirect method}, Lyapunov's theorem asserts that the Lyapunov stability property in \ref{stab:lyapstable} of the equilibrium point \(0\in\R[\sysDim]\) is equivalent to finding a continuously differentiable function \(\domain \ni y \mapsto \lyapfn(y) \in \R\) and a neighborhood \(\nbhd\) of \(0\) satisfying\footnote{Recall that a \emph{neighborhood} for us need not be open, but it must contain a non-empty open subset.}
\begin{equation}
    \tag{\stability}
    \label{e:Lyapunov stability}
    \begin{aligned}
		& \lyapfn(0) = 0,\quad\text{and}\\
		& \begin{dcases}
			\lyapfn(y) > 0 & \text{for all }y\in\nbhd\setmin\set{0},\\
            \inprod{\pdv{\lyapfn}{x}(y)}{\vecfld(x)} \le 0 & \text{for all } y\in\nbhd.
        \end{dcases}
    \end{aligned}
\end{equation}
In a similar vein, the asymptotic stability property of \(0\in\R[\sysDim]\) in \ref{stab:asystable} is equivalent to the existence of positive definite functions \(\lyapfn, \stabilityMargin : \domain \ra \R\) and a neighborhood \(\nbhd\) of \(0\) satisfying
\begin{equation}
    \tag{\asymStability}
	\label{e:asymptotic stability}
    \begin{aligned}
        & \lyapfn(0) = 0 = \stabilityMargin(0), \quad \text{and}\\
		& \begin{dcases}
			\lyapfn(y) > 0 & \text{for all }y\in\nbhd\setmin\set{0},\\
            \stabilityMargin(y) > 0 & \text{for all } y \in \nbhd \setmin \set{0},\\
            % \lowerBound(\norm{y}) \le \stabilityMargin(y) \le \upperBound(\norm{y}) & \text{for all } y \in \nbhd\\
            \inprod{\pdv{\lyapfn}{x}(y)}{\vecfld(y)} + \stabilityMargin(y) \le 0 & \text{for all } y \in \nbhd.
        \end{dcases}
    \end{aligned}
\end{equation}
In what follows, the neighborhood \(\nbhd\) will always be a \emph{compact} subset of \(\domain\).

Recall that a function \(\rho:\lcro{0}{+\infty}\lra\lcro{0}{+\infty}\) is of \embf{class \(\classK\)} if \(\rho\) is continuous, strictly increasing, and \(\rho(0) = 0\).
The requirements of positive definiteness of a function \(\lyapfn\), namely \(\lyapfn(0) = 0\) and \(\lyapfn(y) > 0\) for all \(y \in \domain\setmin \set{0}\), can be equivalently \cite[\S\S41, 42]{ref:Hah-67} captured by means of the condition that there exists a function \(\lowerBound\in\classK\) satisfying \(\lyapfn(y) \ge \lowerBound(\norm{y})\) for all \(y\in\domain\). It is also a standard practice to stipulate the property of decrescence of Lyapunov functions --- that there exists a function \(\upperBound\in\classK\) satisfying \(\lyapfn(y) \le \upperBound(\norm{y})\) for all \(y\in\domain\). For instance, the two properties of positive definiteness and decrescence of a Lyapunov function \(\lyapfn\) are encapsulated by
\begin{equation}
    \label{e:pdf equiv}
    \begin{aligned}
		& \text{there exist class \(\classK\) functions \(\lowerBound, \upperBound\) such that}\\
        & \lowerBound(\norm{y}) \le \lyapfn(y) \le \upperBound(\norm{y}) \quad \text{for all } y \in \domain.
    \end{aligned}
\end{equation}


%% -----------------------------------------------------------------------------
\subsection*{Candidate Lyapunov triplets}
%% -----------------------------------------------------------------------------

Our search for suitable Lyapunov functions for the equilibrium point \(0\in\domain\) of \(\vecfld\) will be restricted to a suitably large space of functions described by means of the following data; we call this data a \embf{candidate Lyapunov triplet}:
% Given an autonomous nonlinear system described by the dynamics in \eqref{e:nonlinear system}, we establish a methodology to search for a candidate Lyapunov function \(\lyapfn\) from a pre specified basis. The following data define the search space for the candidate Lyapunov function:
\begin{enumerate}[label=\textup{(L\arabic*)}, align=left, widest=B, leftmargin=*]
	\item \label{d:nbhd} A compact neighborhood \(\nbhd \subset \domain\) of the origin \(0\in\R[\sysDim]\) that defines the region on which the Lyapunov criteria \eqref{e:Lyapunov function} are verified.
	\item \label{d:pdf} A triple \((\lowerBound, \upperBound, \marginBound)\) of class \(\classK\) functions such that \((\lowerBound, \upperBound)\) characterize positive definiteness and decrescence of the candidate Lyapunov function \(\lyapfn\) as in \eqref{e:pdf equiv}, and \(\marginBound\) characterizes positive definiteness of the stability margin \(\stabilityMargin\) by means of
		\[
			\marginBound(\norm{y}) \le \stabilityMargin(y)\quad\text{for all }y\in\nbhd.
		\]
	\item \label{d:dictionary} A pair \((\basisDict, \marginDict)\) of dictionaries of \emph{linearly independent} and \emph{continuously differentiable} functions \(\basisDict \Let \set{\basisFunc \suchthat i = 1, \ldots, \basisDim}\) and \(\marginDict \Let \set{\marginFunc \suchthat j = 1, \ldots, \marginDim}\) where
		\begin{align*}
			& \basisFunc : \domain \ra \R,\quad \marginFunc: \domain \ra \R, \quad\text{and}\\
			& \basisFunc(0) = 0 = \marginFunc(0)\quad \text{for each }i, j.
		\end{align*}
\end{enumerate}
Our search procedure picks candidate Lyapunov functions \(\lyapfn\) and candidate margin functions \(\stabilityMargin\) from the spans of the two dictionaries:
\[
    \lyapfn \in \linspan \basisDict  \quad \text{and} \quad \stabilityMargin \in \linspan \marginDict.
\]
In summary, a list \(\bigl( \nbhd, (\lowerBound, \upperBound, \marginBound), (\basisDict, \marginDict) \bigr)\) with the properties \ref{d:nbhd}--\ref{d:dictionary} is a candidate Lyapunov triplet.

Fixing a candidate Lyapunov triplet, we proceed to the algorithm for finding a Lyapunov function for the equilibrium point \(0\). This involves checking whether a Lyapunov function \(\lyapfn\in\linspan\basisDict\) and a stability margin \(\stabilityMargin\in\linspan\marginDict\) satisfying the stability requirements \eqref{e:Lyapunov stability} and/or asymptotic stability conditions \eqref{e:asymptotic stability} exist, and the procedure is carried out in \secref{s:results}.% where positive definiteness of \(\lyapfn\) is characterised by \eqref{e:pdf equiv}.


