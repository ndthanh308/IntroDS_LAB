%% ============================================================================
\section{Numerical experiments}
\label{s:numerics}
%% ============================================================================

This section presents a number of numerical experiments to illustrate the technique developed above. A few points concerning these experiments:
\begin{enumerate}[label=\arabic*), align=left, leftmargin=*]
    \item For the purpose of the numerical experiments, we choose the convex quadratic objective function \(\R[\basisDim]\ni z\mapsto \objective(z) \Let \norm{z - \mathbf{1}}^2\), where \(\mathbf{1} \in \R[\basisDim]\) is a vector of all ones
    \item In our simulations, we excluded the function \(\upperBound(\cdot)\) --- serving as an upper bound for the Lyapunov functions to be constructed --- from our consideration; since the number of elements in our dictionaries is finite, the decrescent property of any Lyapunov function follows immediately.
    \item In each experiment, we validated the output of our algorithm by numerically computing the minimum values of the two functions
		\begin{equation*}
			\label{eq:constraint_funcs}
			\begin{aligned}
				y\mapsto \constraintA(y) \Let \lyapfn(y) - \lowerBound(\norm{y}) \quad \text{and}  \quad y\mapsto \constraintB(y) \Let -\stabilityMargin(\ypos) - \inprod{\pdv{\lyapfn}{x}(y)}{\vecfld(y)}
		    \end{aligned}
		\end{equation*}
		over the domain under consideration, and they were confirmed to be non-negative.
	\item \texttt{python 3.10.4} was employed with the support of the libraries \texttt{numpy 1.24.3}, \texttt{scipy 1.8.1}, and \texttt{matplotlib 3.7.1}.
		\begin{itemize}[label=\(\circ\), leftmargin=*]
			\item The `inner' convex optimization problem (in \eqref{e:maxmin}) in each case was solved by employing the \texttt{scipy.optimize.minimize()} function with the \texttt{``SLSQP''} method as an input parameter and \texttt{max iterations = 10000}.
			\item The `outer' (global) maximization problem (in \eqref{e:maxmin}) in each case was solved by employing the \texttt{scipy.optimize.dual\_annealing()} function with \texttt{max iterations = 30}, and a vector (of appropriate dimension) of all entries equal to \(0.5\) as the initial starting point.
			\item In each case, the outcome of the MSA algorithm (which produces a Lyapunov function) was validated against the selected candidate Lyapunov triplet with the aid of the function \texttt{scipy.optimize.differential\_evolution()} together with its default parameters.
		\end{itemize}
\end{enumerate}



%% -------------------------------------------------------------------------
\subsection{A nonlinear planar vector field}
%% -------------------------------------------------------------------------

Consider a nonlinear 2-d system given by
\begin{equation}
    \label{e:random}
    \begin{aligned}
        & \dot{\state_{1}} = 2\state_1\left(1-\frac{\state_1}{2}\right)-\state_1\state_{2}, \quad \\
        & \dot{\state_{2}} = 3\state_2\left(1-\frac{\state_2}{3}\right)-2\state_1\state_{2}.
    \end{aligned}
    \end{equation}
It can be found that \eqref{e:random} has \(3\) equilibrium points, out of which the system is asymptotically stable around the equilibrium points \((2,0)\) and \((0,3)\). A circular disc of radius \(0.2\) centered at the equilibria is chosen as the domain of investigation for the equilibrium points. After shifting the origin to the respective equilibrium points, the following Lyapunov triplets were utilized for both points.

\subsubsection*{Candidate Lyapunov triplet for our experiment}
Our selections were as follows:
\begin{description}
	\item[\ref{d:nbhd}]  \(\nbhd  \Let \set[\big]{ (\state_1, \state_2) \in \R[2] \suchthat \state_1^2 + \state_2^2  \leq 0.04 }\).
	\item[\ref{d:pdf}] \(\lowerBound, \marginBound \in \classK\) were picked as \(\lowerBound(\radius) = \frac{\radius^3}{2}\) and \(\marginBound\left(\radius\right) =  \frac{\radius^4}{2}\) for \(\radius \ge 0\).
	\item[\ref{d:dictionary}] The dictionaries were selected to be:
		\begin{align*}
			\basisDict & \Let \set[\big]{ x_1^{i_1} \cdot x_2^{i_2} \suchthat (i_1, i_2) \in\Nz[2], 2 \leq i_1 + i_2 + \leq 6 },\\ 
			\marginDict & \Let \set[\big]{ x_1^{i_1} \cdot x_2^{i_2} \suchthat (i_1, i_2) \in\Nz[2], i_1 + i_2  = 2, 4, 6 }.
		\end{align*}
\end{description}

The results of our procedure are collected in Table \ref{table_2DField}.
\begin{table}[tbh]
    \begin{center}
    \renewcommand{\arraystretch}{1.5}
    \begin{tabular}{cl}
    \toprule
    Equilibrium Point & Lyapunov Function $\state\mapsto\lyapfn(\state)$ \\ \midrule
    (2,0) & ${\state_1}^2 + \state_1\state_2+{\state_2}^2-4\state_1-2\state_2+4$ \\
    %\cline{1-1}\cline{2-2}
    (0,3) & $1.3{\state_1}^2 + 0.826\state_1\state_2+0.439{\state_2}^2-2.478\state_1-2.635\state_2+3.952$ \\
    \bottomrule
    \end{tabular}
		\caption{\label{table_2DField}
			Lyapunov functions $x\mapsto\lyapfn(x)$ for the two equilibrium points corresponding to \eqref{e:random} in the \emph{original space coordinates}.}
    \end{center}
\end{table}

% % Figure environment removed

% Figure environment removed
% Figure environment removed

% % Figure environment removed

% Figure environment removed

% Figure environment removed


%% -----------------------------------------------------------------------------
\subsection{The van der Pol oscillator}
%% -----------------------------------------------------------------------------

We consider a van der Pol oscillator described by
\begin{equation}
\label{e:vanderpol}
\begin{aligned}
    & \dot\state_{1} = \state_{2}, \quad \\
    & \dot\state_{2} = -\state_1 + \epsilon\state_2(1-\state_1^2),
\end{aligned}
\end{equation}
where \(\eps\in\R[]\) is a parameter. Observe that the stability of the above system depends on \(\epsilon\). To avoid an unstable limit cycle around the equilibrium point (origin), the domain is kept to be a circular disc with a radius of \(0.5\).

\subsubsection*{Candidate Lyapunov triplet for our experiment}
Our selections were as follows:
\begin{description}
	\item[\ref{d:nbhd}] \(\nbhd  \Let \set[\big]{ (\state_1, \state_2) \in \R[2] \suchthat \state_1^2 + \state_2^2  \leq 0.25 }\).
	\item[\ref{d:pdf}] \(\lowerBound, \marginBound \in \classK\) were picked as \(\lowerBound\left(\radius\right) = \frac{\radius^2}{2}\) and \(\marginBound\left(\radius\right) =  \frac{\radius^{10}}{4}\) for \(\radius \ge 0\).
	\item[\ref{d:dictionary}] The dictionaries were selected to be:
		\begin{align*}
			\basisDict & \Let \set[\big]{ x_1^{i_1} \cdot x_2^{i_2} \suchthat (i_1, i_2) \in\Nz[2], 2 \leq i_1 + i_2 + \cdots + i_n \leq 6 },\\ 
			\marginDict & \Let \set[\big]{ x_1^{i_1} \cdot x_2^{i_2} \suchthat (i_1, i_2) \in\Nz[2], i_1 + i_2  = 2, 4, 6 }.
		\end{align*}
\end{description}
The Lyapunov function obtained from our numerical procedure was: 
\[
	\state\mapsto\lyapfn(\state) = 1.106{\state_1}^2 + 0.380\state_1\state_2+ 1.106{\state_2}^2
\]
for all values of \(\eps\) for which the vector fields that are asymptotically stable, i.e., when \(\epsilon <  0\).

% % Figure environment removed
% Figure environment removed
% Figure environment removed

%An important observation in the aforementioned example is the failure of the algorithm to generate a non-trivial solution with a linear selection for the lower bound \(\lowerBound(\cdot)\). This indicates the necessity for the rate of decay of the function x \(\mapsto\) \lyfn \space to surpass linear decay, attributable to the existence of higher-order monomial terms within the vector field itself.

% Figure environment removed



%%--------------------------------------------------------------------------------
\subsection{A whirling pendulum}
%%----------------------------------------------------------------
Consider the whirling pendulum system lifted from \cite[Example 3]{ref:PapPra-02} and described by
\begin{equation}
    \label{e: whirl pendulum}
    \begin{aligned}
        & \dot{\state_{1}} = \state_{2},\\
        & \dot{\state_{2}} = {\dot{\theta}^2}_a\sin(\state_1)\cos(\state_1) - \frac{g}{l_a}\sin(\state_1).
    \end{aligned}
\end{equation}
The linearization of the field \eqref{e: whirl pendulum} around the origin gives us:
\begin{equation}
\label{eq:whirl_params}
\begin{aligned}
       & \dot y = \pmat{0 & 1 \\{\dot{\theta}^2}_a - \frac{g}{l_a} & -0} y.
    \end{aligned}
\end{equation}
The system \eqref{eq:whirl_params} is found to be marginally stable when the condition 
\begin{equation}
    \label{e:whirl pendulum condition}
	{\dot{\theta}^2}_a <\frac{g}{l_a}
\end{equation}
is met and is otherwise unstable. \cite{ref:PapPra-02} provides a Lyapunov function for the case of the above marginally stable whirling pendulum system using \texttt{SOSTOOLS} after performing a transformation on the field. Our method yields no such Lyapunov functions for the system, irrespective of any candidate Lyapunov triplet used in the process. To force asymptotic stability upon the system, we added feedback in the form of a small damping, which changes \eqref{e: whirl pendulum} into
    \begin{equation}
    \label{e: whirl pendulum stable}
    \begin{aligned}
        & \dot{\state_{1}} = \state_{2}, \quad \\
        & \dot{\state_{2}} = {\dot{\theta}^2}_a\sin(\state_1)\cos(\state_1) - \frac{g}{l_a}\sin(\state_1) - k\state_2.
    \end{aligned}  
\end{equation}
The whirling pendulum with feedback is asymptotically stable when \eqref{e:whirl pendulum condition} is satisfied and $k>0$. We carried out our simulations with the numerical values of different parameters presented in Table \ref{table_whirlpendulum}.
\begin{table}[tbh]
		    \begin{center}
		    \renewcommand{\arraystretch}{1.5}
			    \begin{tabular}{cc}
				    \toprule
					Parameter & Numerical Value \\
					\midrule
				    ${\dot{\theta}^2}_a$ & 1 \\
                        $\frac{g}{l_a}$ & 9\\
                        k               & 0.1\\
				    \bottomrule
			    \end{tabular}
			\end{center}
			\caption{\label{table_whirlpendulum} Parameters for the damped whirling pendulum \eqref{e: whirl pendulum stable}. }
\end{table}
We demonstrate the flexibility of our method by picking dictionaries basis for \(V(\cdot)\) consisting of both polynomial and non-polynomial terms.

\subsubsection*{Candidate Lyapunov triplet for our experiment}
Our selections were as follows:
\begin{description}
	\item[\ref{d:nbhd}]  \(\nbhd  \Let \set[\big]{ (\state_1, \state_2) \in \R[2] \suchthat \state_1^2 + \state_2^2  \leq 1 }\).
	\item[\ref{d:pdf}] \(\lowerBound, \marginBound \in \classK\) were picked as \(\lowerBound\left(\radius\right) = \frac{\radius^2}{100}\) and \(\marginBound\left(\radius\right) =  \frac{\radius^4}{200}\) for \(\radius \ge 0\).
	\item[\ref{d:dictionary}] The dictionaries were selected to be:
		\begin{align*}
			\basisDict & \Let \set[\big]{ x_1^{i_1} \cdot x_2^{i_2} \suchthat (i_1, i_2) \in\Nz[2], 2 \leq i_1 + i_2 + \leq 6 } \cup \set[\big]{2 \cos\left(j_{1}x_1\right) \suchthat 0 \leq j_1 \leq 2 } ,\\ 
			\marginDict & \Let \set[\big]{ x_1^{i_1} \cdot x_2^{i_2} \suchthat (i_1, i_2) \in\Nz[2], i_1 + i_2  = 2, 4, 6 }.
		\end{align*}
\end{description}
  
The Lyapunov function for the system \eqref{e: whirl pendulum stable}, with parameters in Table \ref{table_whirlpendulum} and the given Lyapunov triplet, obtained from our numerical procedure was 
\begin{align*}
	\state\mapsto\lyapfn(\state) & = 1.117\state_1^2 - 0.0213\state_1\state_2 +  0.175\state_2^2\\
	& \qquad + 0.042\cos\left(\state_1\right) -0.147\cos\left(2\state_1\right)+0.105.
\end{align*}

% Figure environment removed
% Figure environment removed

% Figure environment removed

%% -----------------------------------------------------------------------------
\subsection{A 5-d hyperchaotic system with linear feedback controller}
%% -----------------------------------------------------------------------------
For our next example, we introduce an illustration of a 5-d autonomous hyperchaotic system, as proposed by \cite{ref:Niu2021}. The system incorporates a linear feedback controller, effectively achieving global asymptotic stability at the origin. The controller can be expressed as follows:
\begin{equation}
    \label{e:5d-system}
    \begin{aligned}
		\dot \state_{1} & = a(\state_{2} - \state_{1}) -k_1\state_1, \quad \\
		\dot \state_{2} & = (c-a)\state_1 + c\state_2 + \state_5 - \state_1\state_3 - k_2\state_2, \quad \\
		\dot \state_{3} & = -b\state_3 + \state_1\state_2 - k_3\state_3, \quad \\
		\dot \state_{4} & = m\state_5 - k_4\state_4, \quad \\
		\dot \state_{5} & = -\state_2 - h\state_4 - k_5\state_5
    \end{aligned}
\end{equation}
where \((k_{1}, k_{2}, k_{3}, k_{4},k_{5}) = (0, 30, 0, 1, 1)\). We keep \(a = 23\), \(b = 3\), \(c = 18\), \(m = 12\), and \(h = 4\) as recommended in \cite{ref:Niu2021} to obtain a global asymptotic stability at origin.

\subsubsection*{Candidate Lyapunov triplet for our experiment}
Our selections were as follows:
\begin{description}
	\item[\ref{d:nbhd}]  \(\nbhd  \Let \set[\big]{ (\state_1, \state_2, \state_3, \state_4, \state_5) \in \R[5] \suchthat \state_1, \state_2, \state_3, \state_4, \state_5 \in \lcrc{-0.5}{0.5} }\).
	\item[\ref{d:pdf}] \(\lowerBound, \marginBound \in \classK\) were picked as \(\lowerBound\left(\radius\right) = \frac{\radius^2}{100}\), and \(\marginBound\left(\radius\right) =  \frac{\radius^4}{20000}\) for \(\radius \ge 0\).
	\item[\ref{d:dictionary}] The dictionaries were selected to be:
		\begin{align*}
			\basisDict & \Let \set[\big]{ x_1^{i_1} \cdot x_2^{i_2} \cdots x_5^{i_5} \suchthat (i_1, i_2, i_3, i_4, i_5) \in\Nz[5], i_1 + i_2 + i_3 + i_4 + i_5 = 2 },\\ 
			\marginDict & \Let \set[\big]{ x_1^{i_1} \cdot x_2^{i_2} \cdots x_5^{i_5} \suchthat (i_1, i_2, i_3, i_4, i_5) \in\Nz[5], i_1 + i_2 + i_3 + i_4 + i_5  = 2, 4, 6 }.
		\end{align*}
\end{description}
Our numerical procedure led to the Lyapunov function%contained in Table \ref{table-5D}.
\begin{align*}
	\state\mapsto \lyapfn(\state) & = 1.229\state_1^2 + 0.982\state_1\state_2 + 0.891\state_1\state_3 + 0.632\state_1\state_4 + 0.236\state_1\state_5 + 0.996\state_2^2\\
	& \qquad + 1.026\state_2\state_3 + 0.663\state_2\state_4 + 0.928\state_2\state_5 + 1.116\state_3^2 + 0.406\state_3\state_4\\
	& \qquad + 1.054\state_3\state_5 + 0.523\state_4^2 + 0.158\state_4\state_5 + 1.595\state_5^2.
\end{align*}

%\begin{table}[tbh]
%	\begin{center}
%	    \begin{tabular}{l}
%		    \toprule
%			 Lyapunov Function $\state\mapsto\lyapfn(\state)$ \\
%			\midrule
%			 $1.229\state_1^2 + 0.982\state_1\state_2 + 0.891\state_1\state_3 + 0.632\state_1\state_4 + 0.236\state_1\state_5 + 0.996\state_2^2 $\\
%			 $+ 1.026\state_2\state_3 + 0.663\state_2\state_4 + 0.928\state_2\state_5 + 1.116\state_3^2 + 0.406\state_3\state_4 + 1.054\state_3\state_5$\\
%			 $+ 0.523\state_4^2 + 0.158\state_4\state_5 + 1.595\state_5^2$ \\
%		    \bottomrule
%	    \end{tabular}
%	\end{center}
%	\caption{\label{table-5D} Lyapunov function $\state\mapsto\lyapfn(\state)$ for the system \eqref{e:5d-system}.}
%\end{table}

% Figure environment removed

% Figure environment removed

%% -----------------------------------------------------------------------------
\subsection{Transient stability of classical power system models}
%% -----------------------------------------------------------------------------

For our next example, we pick a four-dimensional system from \cite{ref:AngMilPap-13} given by the equation (transformed to have the origin as its equilibrium point):
\begin{equation}
  \label{eq:power_model}
  \begin{aligned}
	  \dot \state_{1} & = \state_{2}, \quad \\
	  \dot \state_{2} & = 0.0200\cos(\state_1)\cos(\state_3) - 0.0200\cos(\state_1) - 0.9998\sin(\state_1) - 0.4000\state_2  \\
	  & \qquad + 0.4996\cos(\state_1)\sin(\state_3) - 0.4996\cos(\state_3)\sin(\state_1) + 0.0200\sin(\state_1)\sin(\state_3), \\
	  \dot \state_{3} & = \state_{4}, \\
	  \dot \state_{4} & = 0.4996\cos(\state_3)\sin(\state_1) - 0.0299\cos(\state_3) - 0.4991\sin(\state_3) \\
    & \qquad -0.0200\cos(\state_1)\cos(\state_3) - 
    0.4996\cos(\state_1)\sin(\state_3) - 0.5000\state_4 \\
    & \qquad - 0.0200\sin(\state_1)\sin(\state_3) + 0.0500.
    \end{aligned}
\end{equation}

\subsubsection*{Candidate Lyapunov triplet for our experiment}
    Our selections were as follows:
\begin{description}
	\item[\ref{d:nbhd}]  \(\nbhd  \Let \set[\big]{ (\state_1, \state_2, \state_3, \state_4) \in \R[4] \suchthat \state_1, \state_2, \state_3, \state_4 \in \lcrc{-0.2}{0.2} }\).
	\item[\ref{d:pdf}] \(\lowerBound \in \classK\): \(\lowerBound\left(\radius\right) = \frac{\radius^2}{16}\) and \(\marginBound\left(\radius\right) =  \frac{\radius^2}{200}\) for \(\radius \ge 0\).
	\item[\ref{d:dictionary}] The dictionaries were selected to be:
		\begin{align*}
			\basisDict & \Let \set[\big]{ x_1^{i_1} \cdot x_2^{i_2} \cdot x_3^{i_3} \cdot x_4^{i_4} \suchthat (i_1, i_2, i_3, i_4) \in\Nz[4],  i_1 + i_2 + i_3 + i_4 = 2 },\\ 
			\marginDict & \Let \set[\big]{ x_1^{i_1} \cdot x_2^{i_2} \cdot x_3^{i_3} \cdot x_4^{i_4} \suchthat (i_1, i_2, i_3, i_4) \in\Nz[4], i_1 + i_2 + i_3 + i_4  = 2, 4, 6 }.
		\end{align*}
\end{description} 
This example demonstrates how our algorithm effectively handles higher order systems, enabling the identification of an appropriate Lyapunov function with predefined lower bounds. Our numerical procedure led to the Lyapunov function %contained in Table \ref{table_powersys}.
\begin{align*}
	\state\mapsto\lyapfn(\state) & = 1.339\state_1^2 + 0.591\state_1\state_2 + 0.635\state_1\state_3 + 0.622\state_1\state_4 + 0.989\state_2^2\\
	& \qquad + 0.651\state_2\state_3 + 1.063\state_2\state_4 + 1.064\state_3^2 + 0.970\state_3\state_4 + 1.297\state_4^2.
\end{align*}

%\begin{table}[tbh]
%		    \begin{center}
%		    \renewcommand{\arraystretch}{1.5}
%			    \begin{tabular}{cl}
%				    \toprule
%					 Lyapunov Function $\state\mapsto\lyapfn(\state)$ \\
%					\midrule
%				     $1.339\state_1^2 + 0.591\state_1\state_2 + 0.635\state_1\state_3 + 0.622\state_1\state_4 + 0.989\state_2^2 + 0.651\state_2\state_3 + 1.063\state_2\state_4 + $\\ $  1.064\state_3^2 + 0.970\state_3\state_4 + 1.297\state_4^2 $ \\
%				    \bottomrule
%			    \end{tabular}
%			\end{center}
%			\caption{\label{table_powersys} Lyapunov function $\state\mapsto\lyapfn(\state)$ for the system \eqref{eq:power_model}.}
%\end{table}

% Figure environment removed

% Figure environment removed


%% -----------------------------------------------------------------------------
\subsection{Flexibility of basis}
%% -----------------------------------------------------------------------------

We present a final example illustrating an interesting situation in which the vector field is defined by means of a case statement. This example is lifted from \cite[p. 68]{ref:BhaSze-70}, and the dynamics is given by
\begin{equation}
  \label{eq:bhatia's}
  \begin{aligned}
	  \dot\state_{1} & =
	  \begin{cases}
		  \state_1 & \text{if }\state_{1}^2\state_{2}^2 \geq 1,\\
				2\state_{1}^3\state_{2}^2 - \state_{1} & \text{if } \state_{1}^2\state_{2}^2 < 1,
	\end{cases}\\
	\dot\state_{2} & = -\state_{2}.
\end{aligned}
\end{equation}
Notice that the vector field \eqref{eq:bhatia's} is continuous and the neighborhood \(\nbhd\) selected above contains both the regions differentiated by the case statement.

\subsubsection*{Candidate Lyapunov triplet for our experiment}
    Our selections were as follows:
\begin{description}
	\item[\ref{d:nbhd}]  \(\nbhd  \Let \set[\big]{ (\state_1, \state_2) \in \R[2] \suchthat \state_1, \state_2 \in [-3, 3] }\).
	\item[\ref{d:pdf}] \(\lowerBound, \marginBound \in \classK\) : \(\lowerBound\left(\radius\right) = 0\) and \(\marginBound\left(\radius\right) =  \frac{\radius^2}{32^2\cdot 2}\) for \(\radius \ge 0\).
	\item[\ref{d:dictionary}] The dictionaries were selected to be:
		\begin{align*}
			\basisDict & \Let \set[\big]{ x_1^{i_1} \cdot x_2^{i_2} \suchthat (i_1, i_2) \in\Nz[2], 2 \leq i_1 + i_2  \leq 6 }, \\
				\marginDict & \Let \set[\big]{ x_1^{i_1} \cdot x_2^{i_2}  \suchthat (i_1, i_2) \in\Nz[2], i_1 + i_2 = 2, 4, 6 }.
		\end{align*}
\end{description}
In contrast to the rational polynomial functions mentioned in \cite{ref:BhaSze-70} corresponding to this example, we obtained a Lyapunov function within the class of polynomial functions itself without extending it to include rational functions. The outcome of our numerical experiment was the Lyapunov function
\[
	\state\mapsto\lyapfn(\state) =0.003\state_1^2 + 0.069\state_1\state_2 + 1.063\state_1^2.
\]

% Figure environment removed

% Figure environment removed

% Figure environment removed


%% =============================================================================
\section{A quick excursion into instability via Chetaev's Theorem}
%% =============================================================================

In this brief section we transcend beyond the stability of isolated equilibria into their instability. This gives us one more opportunity to leverage the generality and adaptability of our algorithmic procedure. We consider the problem of determining instability of an equilibrium point by employing Chetaev's theorem \cite[Chapter V, p.\ 188]{ref:Vid-02}. We refer to the example of spinning of a rigid body provided in \cite[pp. 188-190]{ref:Vid-02}, and express the dynamics as:
\begin{equation}
  \label{eq:chetaev's}
  \begin{aligned}
	  \dot\state_{1} & = a\state_{2}\state_{3},\\
	  \dot\state_{2} & = -b\state_{1}\state_{3},\\
	  \dot\state_{3} & = c\state_{1}\state_{2},
    \end{aligned}
\end{equation}
where for the sake of simplicity, we pick \(a = b = c = 1\). We pick an equilibrium point of the above system of the form \( (0, y_0, 0)\) with \(y_0 \ge 0\) to be specific and transform coordinates as shown in \cite[Example 105, p.\ 189]{ref:Vid-02}.

\subsubsection*{Candidate Lyapunov triplet for our experiment}
    Our selections were as follows:
\begin{description}
	\item[\ref{d:nbhd}]  \(\nbhd  \Let \set[\big]{ (\state_1, \state_2, \state_3) \in \R[3] \suchthat \state_1, \state_2, \state_3 \in [0, 1] }\).
	\item[\ref{d:pdf}] \(\lowerBound \in \classK\): \(\lowerBound\left(\radius\right) = 0\) for \(\radius \ge 0\).
	\item[\ref{d:dictionary}] The dictionary for candidate Lyapunov functions was selected to be
		\begin{align*}
			\basisDict & \Let \set[\big]{ x_1^{i_1} \cdot x_2^{i_2} \cdot x_3^{i_3} \suchthat (i_1, i_2, i_3) \in\Nz[3], 2 \leq i_1 + i_2 + i_3 \leq 6 }.
		\end{align*}
\end{description}
No stability margin was incorporated in the numerical experiment, effectively considering it to be zero. Upon executing the dynamics, the resulting Lyapunov-like function is obtained as \(x\mapsto \state_{1}^2 + \state_{2}^2 + \state_{3}^2 + \state_1\state_3\), which fulfills the constraints specified in Chetaev's theorem \cite[Theorem 99, p.\ 188]{ref:Vid-02}. This confirms instability of the system's equilibrium point at the origin \((0, 0, 0)\).

